\documentclass[bachelor_thesis]{subfiles}

\begin{document}
\chapter{Analysis of the bunch characteristics}
The analysis of the bunch and the caused wakefields is split into two parts in this thesis. In \autoref{chap:distro_change} the change of the spatial charge distribution of the driver is discussed as well as its effect on the quality of the produced wakefield, 
quantized by the maximal gainable energy for a potential witness beam.

The second part in \autoref{chap:E_shift} is concentrated on the energy distribution of the driver, especially on changes to the peak energy, as bigger changes would violate the assumption of constant peak energy over the course of the \gls{pwfa}, as made in experiments.

Analysis is for both initially done for a driver with Gaussian charge distribution, a initial kinetic energy of \qty{250}{\MeV} and a divergence of \qty{4.2}{\mrad}. This is then compared to drivers with different charge distributions, initial energy or divergence,
highlighting its effects on the created wakefield. 
 
\section{Movement of the driver electrons} \label{chap:distro_change}
In this section, the movement of the particles in our driver is discussed and the changes in position respective to each other are shown. As such tracking of individual particles is not possible in experiment
this will give further insight into the effects of the \gls{pwfa} on the drive beam. Additionally the created wakefields are analyzed with respect to their formed electric fields and the energy which can be
gained by these fields from a potential witness beam. 

In \autoref{fig:q_series} a time series of 2D histograms, showing the charge density of the driver after entering plasma, can be seen for the $\zeta$-$z$-plane, with $\zeta$ being an axis, which moves with $c$ with the driver. Also given is $y$, the distance to the start of the plasma upramp.
\begin{figure}
	\centering
	%% Creator: Matplotlib, PGF backend
%%
%% To include the figure in your LaTeX document, write
%%   \input{<filename>.pgf}
%%
%% Make sure the required packages are loaded in your preamble
%%   \usepackage{pgf}
%%
%% Also ensure that all the required font packages are loaded; for instance,
%% the lmodern package is sometimes necessary when using math font.
%%   \usepackage{lmodern}
%%
%% Figures using additional raster images can only be included by \input if
%% they are in the same directory as the main LaTeX file. For loading figures
%% from other directories you can use the `import` package
%%   \usepackage{import}
%%
%% and then include the figures with
%%   \import{<path to file>}{<filename>.pgf}
%%
%% Matplotlib used the following preamble
%%
\begingroup%
\makeatletter%
\begin{pgfpicture}%
\pgfpathrectangle{\pgfpointorigin}{\pgfqpoint{10.000000in}{7.000000in}}%
\pgfusepath{use as bounding box, clip}%
\begin{pgfscope}%
\pgfsetbuttcap%
\pgfsetmiterjoin%
\pgfsetlinewidth{0.000000pt}%
\definecolor{currentstroke}{rgb}{1.000000,1.000000,1.000000}%
\pgfsetstrokecolor{currentstroke}%
\pgfsetstrokeopacity{0.000000}%
\pgfsetdash{}{0pt}%
\pgfpathmoveto{\pgfqpoint{0.000000in}{0.000000in}}%
\pgfpathlineto{\pgfqpoint{10.000000in}{0.000000in}}%
\pgfpathlineto{\pgfqpoint{10.000000in}{7.000000in}}%
\pgfpathlineto{\pgfqpoint{0.000000in}{7.000000in}}%
\pgfpathlineto{\pgfqpoint{0.000000in}{0.000000in}}%
\pgfpathclose%
\pgfusepath{}%
\end{pgfscope}%
\begin{pgfscope}%
\pgfsetbuttcap%
\pgfsetmiterjoin%
\definecolor{currentfill}{rgb}{1.000000,1.000000,1.000000}%
\pgfsetfillcolor{currentfill}%
\pgfsetlinewidth{0.000000pt}%
\definecolor{currentstroke}{rgb}{0.000000,0.000000,0.000000}%
\pgfsetstrokecolor{currentstroke}%
\pgfsetstrokeopacity{0.000000}%
\pgfsetdash{}{0pt}%
\pgfpathmoveto{\pgfqpoint{1.250000in}{4.155455in}}%
\pgfpathlineto{\pgfqpoint{3.529412in}{4.155455in}}%
\pgfpathlineto{\pgfqpoint{3.529412in}{6.160000in}}%
\pgfpathlineto{\pgfqpoint{1.250000in}{6.160000in}}%
\pgfpathlineto{\pgfqpoint{1.250000in}{4.155455in}}%
\pgfpathclose%
\pgfusepath{fill}%
\end{pgfscope}%
\begin{pgfscope}%
\pgfpathrectangle{\pgfqpoint{1.250000in}{4.155455in}}{\pgfqpoint{2.279412in}{2.004545in}}%
\pgfusepath{clip}%
\pgfsys@transformcm{2.291667}{0.000000}{0.000000}{2.013889}{1.250000in}{4.155455in}%
\pgftext[left,bottom]{\includegraphics[interpolate=false,width=1.000000in,height=1.000000in]{q_series-img0.png}}%
\end{pgfscope}%
\begin{pgfscope}%
\pgfsetbuttcap%
\pgfsetroundjoin%
\definecolor{currentfill}{rgb}{0.000000,0.000000,0.000000}%
\pgfsetfillcolor{currentfill}%
\pgfsetlinewidth{0.803000pt}%
\definecolor{currentstroke}{rgb}{0.000000,0.000000,0.000000}%
\pgfsetstrokecolor{currentstroke}%
\pgfsetdash{}{0pt}%
\pgfsys@defobject{currentmarker}{\pgfqpoint{0.000000in}{-0.048611in}}{\pgfqpoint{0.000000in}{0.000000in}}{%
\pgfpathmoveto{\pgfqpoint{0.000000in}{0.000000in}}%
\pgfpathlineto{\pgfqpoint{0.000000in}{-0.048611in}}%
\pgfusepath{stroke,fill}%
}%
\begin{pgfscope}%
\pgfsys@transformshift{1.660542in}{4.155455in}%
\pgfsys@useobject{currentmarker}{}%
\end{pgfscope}%
\end{pgfscope}%
\begin{pgfscope}%
\pgfsetbuttcap%
\pgfsetroundjoin%
\definecolor{currentfill}{rgb}{0.000000,0.000000,0.000000}%
\pgfsetfillcolor{currentfill}%
\pgfsetlinewidth{0.803000pt}%
\definecolor{currentstroke}{rgb}{0.000000,0.000000,0.000000}%
\pgfsetstrokecolor{currentstroke}%
\pgfsetdash{}{0pt}%
\pgfsys@defobject{currentmarker}{\pgfqpoint{0.000000in}{-0.048611in}}{\pgfqpoint{0.000000in}{0.000000in}}{%
\pgfpathmoveto{\pgfqpoint{0.000000in}{0.000000in}}%
\pgfpathlineto{\pgfqpoint{0.000000in}{-0.048611in}}%
\pgfusepath{stroke,fill}%
}%
\begin{pgfscope}%
\pgfsys@transformshift{2.139094in}{4.155455in}%
\pgfsys@useobject{currentmarker}{}%
\end{pgfscope}%
\end{pgfscope}%
\begin{pgfscope}%
\pgfsetbuttcap%
\pgfsetroundjoin%
\definecolor{currentfill}{rgb}{0.000000,0.000000,0.000000}%
\pgfsetfillcolor{currentfill}%
\pgfsetlinewidth{0.803000pt}%
\definecolor{currentstroke}{rgb}{0.000000,0.000000,0.000000}%
\pgfsetstrokecolor{currentstroke}%
\pgfsetdash{}{0pt}%
\pgfsys@defobject{currentmarker}{\pgfqpoint{0.000000in}{-0.048611in}}{\pgfqpoint{0.000000in}{0.000000in}}{%
\pgfpathmoveto{\pgfqpoint{0.000000in}{0.000000in}}%
\pgfpathlineto{\pgfqpoint{0.000000in}{-0.048611in}}%
\pgfusepath{stroke,fill}%
}%
\begin{pgfscope}%
\pgfsys@transformshift{2.617647in}{4.155455in}%
\pgfsys@useobject{currentmarker}{}%
\end{pgfscope}%
\end{pgfscope}%
\begin{pgfscope}%
\pgfsetbuttcap%
\pgfsetroundjoin%
\definecolor{currentfill}{rgb}{0.000000,0.000000,0.000000}%
\pgfsetfillcolor{currentfill}%
\pgfsetlinewidth{0.803000pt}%
\definecolor{currentstroke}{rgb}{0.000000,0.000000,0.000000}%
\pgfsetstrokecolor{currentstroke}%
\pgfsetdash{}{0pt}%
\pgfsys@defobject{currentmarker}{\pgfqpoint{0.000000in}{-0.048611in}}{\pgfqpoint{0.000000in}{0.000000in}}{%
\pgfpathmoveto{\pgfqpoint{0.000000in}{0.000000in}}%
\pgfpathlineto{\pgfqpoint{0.000000in}{-0.048611in}}%
\pgfusepath{stroke,fill}%
}%
\begin{pgfscope}%
\pgfsys@transformshift{3.096200in}{4.155455in}%
\pgfsys@useobject{currentmarker}{}%
\end{pgfscope}%
\end{pgfscope}%
\begin{pgfscope}%
\definecolor{textcolor}{rgb}{0.000000,0.000000,0.000000}%
\pgfsetstrokecolor{textcolor}%
\pgfsetfillcolor{textcolor}%
\pgftext[x=2.389706in,y=4.099899in,,top]{\color{textcolor}\sffamily\fontsize{10.000000}{12.000000}\selectfont \(\displaystyle \zeta \, \mathrm{[\mu m]}\)}%
\end{pgfscope}%
\begin{pgfscope}%
\pgfsetbuttcap%
\pgfsetroundjoin%
\definecolor{currentfill}{rgb}{0.000000,0.000000,0.000000}%
\pgfsetfillcolor{currentfill}%
\pgfsetlinewidth{0.803000pt}%
\definecolor{currentstroke}{rgb}{0.000000,0.000000,0.000000}%
\pgfsetstrokecolor{currentstroke}%
\pgfsetdash{}{0pt}%
\pgfsys@defobject{currentmarker}{\pgfqpoint{-0.048611in}{0.000000in}}{\pgfqpoint{-0.000000in}{0.000000in}}{%
\pgfpathmoveto{\pgfqpoint{-0.000000in}{0.000000in}}%
\pgfpathlineto{\pgfqpoint{-0.048611in}{0.000000in}}%
\pgfusepath{stroke,fill}%
}%
\begin{pgfscope}%
\pgfsys@transformshift{1.250000in}{4.163479in}%
\pgfsys@useobject{currentmarker}{}%
\end{pgfscope}%
\end{pgfscope}%
\begin{pgfscope}%
\definecolor{textcolor}{rgb}{0.000000,0.000000,0.000000}%
\pgfsetstrokecolor{textcolor}%
\pgfsetfillcolor{textcolor}%
\pgftext[x=0.905863in, y=4.115254in, left, base]{\color{textcolor}\sffamily\fontsize{10.000000}{12.000000}\selectfont \(\displaystyle {\ensuremath{-}30}\)}%
\end{pgfscope}%
\begin{pgfscope}%
\pgfsetbuttcap%
\pgfsetroundjoin%
\definecolor{currentfill}{rgb}{0.000000,0.000000,0.000000}%
\pgfsetfillcolor{currentfill}%
\pgfsetlinewidth{0.803000pt}%
\definecolor{currentstroke}{rgb}{0.000000,0.000000,0.000000}%
\pgfsetstrokecolor{currentstroke}%
\pgfsetdash{}{0pt}%
\pgfsys@defobject{currentmarker}{\pgfqpoint{-0.048611in}{0.000000in}}{\pgfqpoint{-0.000000in}{0.000000in}}{%
\pgfpathmoveto{\pgfqpoint{-0.000000in}{0.000000in}}%
\pgfpathlineto{\pgfqpoint{-0.048611in}{0.000000in}}%
\pgfusepath{stroke,fill}%
}%
\begin{pgfscope}%
\pgfsys@transformshift{1.250000in}{4.494895in}%
\pgfsys@useobject{currentmarker}{}%
\end{pgfscope}%
\end{pgfscope}%
\begin{pgfscope}%
\definecolor{textcolor}{rgb}{0.000000,0.000000,0.000000}%
\pgfsetstrokecolor{textcolor}%
\pgfsetfillcolor{textcolor}%
\pgftext[x=0.905863in, y=4.446670in, left, base]{\color{textcolor}\sffamily\fontsize{10.000000}{12.000000}\selectfont \(\displaystyle {\ensuremath{-}20}\)}%
\end{pgfscope}%
\begin{pgfscope}%
\pgfsetbuttcap%
\pgfsetroundjoin%
\definecolor{currentfill}{rgb}{0.000000,0.000000,0.000000}%
\pgfsetfillcolor{currentfill}%
\pgfsetlinewidth{0.803000pt}%
\definecolor{currentstroke}{rgb}{0.000000,0.000000,0.000000}%
\pgfsetstrokecolor{currentstroke}%
\pgfsetdash{}{0pt}%
\pgfsys@defobject{currentmarker}{\pgfqpoint{-0.048611in}{0.000000in}}{\pgfqpoint{-0.000000in}{0.000000in}}{%
\pgfpathmoveto{\pgfqpoint{-0.000000in}{0.000000in}}%
\pgfpathlineto{\pgfqpoint{-0.048611in}{0.000000in}}%
\pgfusepath{stroke,fill}%
}%
\begin{pgfscope}%
\pgfsys@transformshift{1.250000in}{4.826311in}%
\pgfsys@useobject{currentmarker}{}%
\end{pgfscope}%
\end{pgfscope}%
\begin{pgfscope}%
\definecolor{textcolor}{rgb}{0.000000,0.000000,0.000000}%
\pgfsetstrokecolor{textcolor}%
\pgfsetfillcolor{textcolor}%
\pgftext[x=0.905863in, y=4.778086in, left, base]{\color{textcolor}\sffamily\fontsize{10.000000}{12.000000}\selectfont \(\displaystyle {\ensuremath{-}10}\)}%
\end{pgfscope}%
\begin{pgfscope}%
\pgfsetbuttcap%
\pgfsetroundjoin%
\definecolor{currentfill}{rgb}{0.000000,0.000000,0.000000}%
\pgfsetfillcolor{currentfill}%
\pgfsetlinewidth{0.803000pt}%
\definecolor{currentstroke}{rgb}{0.000000,0.000000,0.000000}%
\pgfsetstrokecolor{currentstroke}%
\pgfsetdash{}{0pt}%
\pgfsys@defobject{currentmarker}{\pgfqpoint{-0.048611in}{0.000000in}}{\pgfqpoint{-0.000000in}{0.000000in}}{%
\pgfpathmoveto{\pgfqpoint{-0.000000in}{0.000000in}}%
\pgfpathlineto{\pgfqpoint{-0.048611in}{0.000000in}}%
\pgfusepath{stroke,fill}%
}%
\begin{pgfscope}%
\pgfsys@transformshift{1.250000in}{5.157727in}%
\pgfsys@useobject{currentmarker}{}%
\end{pgfscope}%
\end{pgfscope}%
\begin{pgfscope}%
\definecolor{textcolor}{rgb}{0.000000,0.000000,0.000000}%
\pgfsetstrokecolor{textcolor}%
\pgfsetfillcolor{textcolor}%
\pgftext[x=1.083333in, y=5.109502in, left, base]{\color{textcolor}\sffamily\fontsize{10.000000}{12.000000}\selectfont \(\displaystyle {0}\)}%
\end{pgfscope}%
\begin{pgfscope}%
\pgfsetbuttcap%
\pgfsetroundjoin%
\definecolor{currentfill}{rgb}{0.000000,0.000000,0.000000}%
\pgfsetfillcolor{currentfill}%
\pgfsetlinewidth{0.803000pt}%
\definecolor{currentstroke}{rgb}{0.000000,0.000000,0.000000}%
\pgfsetstrokecolor{currentstroke}%
\pgfsetdash{}{0pt}%
\pgfsys@defobject{currentmarker}{\pgfqpoint{-0.048611in}{0.000000in}}{\pgfqpoint{-0.000000in}{0.000000in}}{%
\pgfpathmoveto{\pgfqpoint{-0.000000in}{0.000000in}}%
\pgfpathlineto{\pgfqpoint{-0.048611in}{0.000000in}}%
\pgfusepath{stroke,fill}%
}%
\begin{pgfscope}%
\pgfsys@transformshift{1.250000in}{5.489143in}%
\pgfsys@useobject{currentmarker}{}%
\end{pgfscope}%
\end{pgfscope}%
\begin{pgfscope}%
\definecolor{textcolor}{rgb}{0.000000,0.000000,0.000000}%
\pgfsetstrokecolor{textcolor}%
\pgfsetfillcolor{textcolor}%
\pgftext[x=1.013888in, y=5.440918in, left, base]{\color{textcolor}\sffamily\fontsize{10.000000}{12.000000}\selectfont \(\displaystyle {10}\)}%
\end{pgfscope}%
\begin{pgfscope}%
\pgfsetbuttcap%
\pgfsetroundjoin%
\definecolor{currentfill}{rgb}{0.000000,0.000000,0.000000}%
\pgfsetfillcolor{currentfill}%
\pgfsetlinewidth{0.803000pt}%
\definecolor{currentstroke}{rgb}{0.000000,0.000000,0.000000}%
\pgfsetstrokecolor{currentstroke}%
\pgfsetdash{}{0pt}%
\pgfsys@defobject{currentmarker}{\pgfqpoint{-0.048611in}{0.000000in}}{\pgfqpoint{-0.000000in}{0.000000in}}{%
\pgfpathmoveto{\pgfqpoint{-0.000000in}{0.000000in}}%
\pgfpathlineto{\pgfqpoint{-0.048611in}{0.000000in}}%
\pgfusepath{stroke,fill}%
}%
\begin{pgfscope}%
\pgfsys@transformshift{1.250000in}{5.820559in}%
\pgfsys@useobject{currentmarker}{}%
\end{pgfscope}%
\end{pgfscope}%
\begin{pgfscope}%
\definecolor{textcolor}{rgb}{0.000000,0.000000,0.000000}%
\pgfsetstrokecolor{textcolor}%
\pgfsetfillcolor{textcolor}%
\pgftext[x=1.013888in, y=5.772334in, left, base]{\color{textcolor}\sffamily\fontsize{10.000000}{12.000000}\selectfont \(\displaystyle {20}\)}%
\end{pgfscope}%
\begin{pgfscope}%
\pgfsetbuttcap%
\pgfsetroundjoin%
\definecolor{currentfill}{rgb}{0.000000,0.000000,0.000000}%
\pgfsetfillcolor{currentfill}%
\pgfsetlinewidth{0.803000pt}%
\definecolor{currentstroke}{rgb}{0.000000,0.000000,0.000000}%
\pgfsetstrokecolor{currentstroke}%
\pgfsetdash{}{0pt}%
\pgfsys@defobject{currentmarker}{\pgfqpoint{-0.048611in}{0.000000in}}{\pgfqpoint{-0.000000in}{0.000000in}}{%
\pgfpathmoveto{\pgfqpoint{-0.000000in}{0.000000in}}%
\pgfpathlineto{\pgfqpoint{-0.048611in}{0.000000in}}%
\pgfusepath{stroke,fill}%
}%
\begin{pgfscope}%
\pgfsys@transformshift{1.250000in}{6.151975in}%
\pgfsys@useobject{currentmarker}{}%
\end{pgfscope}%
\end{pgfscope}%
\begin{pgfscope}%
\definecolor{textcolor}{rgb}{0.000000,0.000000,0.000000}%
\pgfsetstrokecolor{textcolor}%
\pgfsetfillcolor{textcolor}%
\pgftext[x=1.013888in, y=6.103750in, left, base]{\color{textcolor}\sffamily\fontsize{10.000000}{12.000000}\selectfont \(\displaystyle {30}\)}%
\end{pgfscope}%
\begin{pgfscope}%
\definecolor{textcolor}{rgb}{0.000000,0.000000,0.000000}%
\pgfsetstrokecolor{textcolor}%
\pgfsetfillcolor{textcolor}%
\pgftext[x=0.850308in,y=5.157727in,,bottom,rotate=90.000000]{\color{textcolor}\sffamily\fontsize{10.000000}{12.000000}\selectfont \(\displaystyle z \, \mathrm{[\mu m]}\)}%
\end{pgfscope}%
\begin{pgfscope}%
\pgfpathrectangle{\pgfqpoint{1.250000in}{4.155455in}}{\pgfqpoint{2.279412in}{2.004545in}}%
\pgfusepath{clip}%
\pgfsetbuttcap%
\pgfsetroundjoin%
\pgfsetlinewidth{0.314766pt}%
\definecolor{currentstroke}{rgb}{0.268510,0.009605,0.335427}%
\pgfsetstrokecolor{currentstroke}%
\pgfsetdash{}{0pt}%
\pgfpathmoveto{\pgfqpoint{3.261668in}{5.067514in}}%
\pgfpathlineto{\pgfqpoint{3.211678in}{5.066024in}}%
\pgfusepath{stroke}%
\end{pgfscope}%
\begin{pgfscope}%
\pgfpathrectangle{\pgfqpoint{1.250000in}{4.155455in}}{\pgfqpoint{2.279412in}{2.004545in}}%
\pgfusepath{clip}%
\pgfsetbuttcap%
\pgfsetroundjoin%
\pgfsetlinewidth{0.320833pt}%
\definecolor{currentstroke}{rgb}{0.269944,0.014625,0.341379}%
\pgfsetstrokecolor{currentstroke}%
\pgfsetdash{}{0pt}%
\pgfpathmoveto{\pgfqpoint{3.211678in}{5.066024in}}%
\pgfpathlineto{\pgfqpoint{3.161670in}{5.065633in}}%
\pgfusepath{stroke}%
\end{pgfscope}%
\begin{pgfscope}%
\pgfpathrectangle{\pgfqpoint{1.250000in}{4.155455in}}{\pgfqpoint{2.279412in}{2.004545in}}%
\pgfusepath{clip}%
\pgfsetbuttcap%
\pgfsetroundjoin%
\pgfsetlinewidth{0.332228pt}%
\definecolor{currentstroke}{rgb}{0.272594,0.025563,0.353093}%
\pgfsetstrokecolor{currentstroke}%
\pgfsetdash{}{0pt}%
\pgfpathmoveto{\pgfqpoint{3.161670in}{5.065633in}}%
\pgfpathlineto{\pgfqpoint{3.111528in}{5.065873in}}%
\pgfusepath{stroke}%
\end{pgfscope}%
\begin{pgfscope}%
\pgfpathrectangle{\pgfqpoint{1.250000in}{4.155455in}}{\pgfqpoint{2.279412in}{2.004545in}}%
\pgfusepath{clip}%
\pgfsetbuttcap%
\pgfsetroundjoin%
\pgfsetlinewidth{0.342263pt}%
\definecolor{currentstroke}{rgb}{0.273809,0.031497,0.358853}%
\pgfsetstrokecolor{currentstroke}%
\pgfsetdash{}{0pt}%
\pgfpathmoveto{\pgfqpoint{3.111528in}{5.065873in}}%
\pgfpathlineto{\pgfqpoint{3.061390in}{5.066243in}}%
\pgfusepath{stroke}%
\end{pgfscope}%
\begin{pgfscope}%
\pgfpathrectangle{\pgfqpoint{1.250000in}{4.155455in}}{\pgfqpoint{2.279412in}{2.004545in}}%
\pgfusepath{clip}%
\pgfsetbuttcap%
\pgfsetroundjoin%
\pgfsetlinewidth{0.351884pt}%
\definecolor{currentstroke}{rgb}{0.276022,0.044167,0.370164}%
\pgfsetstrokecolor{currentstroke}%
\pgfsetdash{}{0pt}%
\pgfpathmoveto{\pgfqpoint{3.061390in}{5.066243in}}%
\pgfpathlineto{\pgfqpoint{3.011249in}{5.066629in}}%
\pgfusepath{stroke}%
\end{pgfscope}%
\begin{pgfscope}%
\pgfpathrectangle{\pgfqpoint{1.250000in}{4.155455in}}{\pgfqpoint{2.279412in}{2.004545in}}%
\pgfusepath{clip}%
\pgfsetbuttcap%
\pgfsetroundjoin%
\pgfsetlinewidth{0.379649pt}%
\definecolor{currentstroke}{rgb}{0.279566,0.067836,0.391917}%
\pgfsetstrokecolor{currentstroke}%
\pgfsetdash{}{0pt}%
\pgfpathmoveto{\pgfqpoint{3.011249in}{5.066629in}}%
\pgfpathlineto{\pgfqpoint{2.961101in}{5.066845in}}%
\pgfusepath{stroke}%
\end{pgfscope}%
\begin{pgfscope}%
\pgfpathrectangle{\pgfqpoint{1.250000in}{4.155455in}}{\pgfqpoint{2.279412in}{2.004545in}}%
\pgfusepath{clip}%
\pgfsetbuttcap%
\pgfsetroundjoin%
\pgfsetlinewidth{0.417723pt}%
\definecolor{currentstroke}{rgb}{0.282327,0.094955,0.417331}%
\pgfsetstrokecolor{currentstroke}%
\pgfsetdash{}{0pt}%
\pgfpathmoveto{\pgfqpoint{2.961101in}{5.066845in}}%
\pgfpathlineto{\pgfqpoint{2.910955in}{5.067481in}}%
\pgfusepath{stroke}%
\end{pgfscope}%
\begin{pgfscope}%
\pgfpathrectangle{\pgfqpoint{1.250000in}{4.155455in}}{\pgfqpoint{2.279412in}{2.004545in}}%
\pgfusepath{clip}%
\pgfsetbuttcap%
\pgfsetroundjoin%
\pgfsetlinewidth{0.450946pt}%
\definecolor{currentstroke}{rgb}{0.283229,0.120777,0.440584}%
\pgfsetstrokecolor{currentstroke}%
\pgfsetdash{}{0pt}%
\pgfpathmoveto{\pgfqpoint{2.910955in}{5.067481in}}%
\pgfpathlineto{\pgfqpoint{2.860811in}{5.068226in}}%
\pgfusepath{stroke}%
\end{pgfscope}%
\begin{pgfscope}%
\pgfpathrectangle{\pgfqpoint{1.250000in}{4.155455in}}{\pgfqpoint{2.279412in}{2.004545in}}%
\pgfusepath{clip}%
\pgfsetbuttcap%
\pgfsetroundjoin%
\pgfsetlinewidth{0.494280pt}%
\definecolor{currentstroke}{rgb}{0.281412,0.155834,0.469201}%
\pgfsetstrokecolor{currentstroke}%
\pgfsetdash{}{0pt}%
\pgfpathmoveto{\pgfqpoint{2.860811in}{5.068226in}}%
\pgfpathlineto{\pgfqpoint{2.810668in}{5.069040in}}%
\pgfusepath{stroke}%
\end{pgfscope}%
\begin{pgfscope}%
\pgfpathrectangle{\pgfqpoint{1.250000in}{4.155455in}}{\pgfqpoint{2.279412in}{2.004545in}}%
\pgfusepath{clip}%
\pgfsetbuttcap%
\pgfsetroundjoin%
\pgfsetlinewidth{0.577859pt}%
\definecolor{currentstroke}{rgb}{0.270595,0.214069,0.507052}%
\pgfsetstrokecolor{currentstroke}%
\pgfsetdash{}{0pt}%
\pgfpathmoveto{\pgfqpoint{2.810668in}{5.069040in}}%
\pgfpathlineto{\pgfqpoint{2.760526in}{5.069867in}}%
\pgfusepath{stroke}%
\end{pgfscope}%
\begin{pgfscope}%
\pgfpathrectangle{\pgfqpoint{1.250000in}{4.155455in}}{\pgfqpoint{2.279412in}{2.004545in}}%
\pgfusepath{clip}%
\pgfsetbuttcap%
\pgfsetroundjoin%
\pgfsetlinewidth{0.650131pt}%
\definecolor{currentstroke}{rgb}{0.255645,0.260703,0.528312}%
\pgfsetstrokecolor{currentstroke}%
\pgfsetdash{}{0pt}%
\pgfpathmoveto{\pgfqpoint{2.760526in}{5.069867in}}%
\pgfpathlineto{\pgfqpoint{2.710387in}{5.070846in}}%
\pgfusepath{stroke}%
\end{pgfscope}%
\begin{pgfscope}%
\pgfpathrectangle{\pgfqpoint{1.250000in}{4.155455in}}{\pgfqpoint{2.279412in}{2.004545in}}%
\pgfusepath{clip}%
\pgfsetbuttcap%
\pgfsetroundjoin%
\pgfsetlinewidth{0.736934pt}%
\definecolor{currentstroke}{rgb}{0.231674,0.318106,0.544834}%
\pgfsetstrokecolor{currentstroke}%
\pgfsetdash{}{0pt}%
\pgfpathmoveto{\pgfqpoint{2.710387in}{5.070846in}}%
\pgfpathlineto{\pgfqpoint{2.660254in}{5.072045in}}%
\pgfusepath{stroke}%
\end{pgfscope}%
\begin{pgfscope}%
\pgfpathrectangle{\pgfqpoint{1.250000in}{4.155455in}}{\pgfqpoint{2.279412in}{2.004545in}}%
\pgfusepath{clip}%
\pgfsetbuttcap%
\pgfsetroundjoin%
\pgfsetlinewidth{0.772768pt}%
\definecolor{currentstroke}{rgb}{0.221989,0.339161,0.548752}%
\pgfsetstrokecolor{currentstroke}%
\pgfsetdash{}{0pt}%
\pgfpathmoveto{\pgfqpoint{2.660254in}{5.072045in}}%
\pgfpathlineto{\pgfqpoint{2.610125in}{5.073372in}}%
\pgfusepath{stroke}%
\end{pgfscope}%
\begin{pgfscope}%
\pgfpathrectangle{\pgfqpoint{1.250000in}{4.155455in}}{\pgfqpoint{2.279412in}{2.004545in}}%
\pgfusepath{clip}%
\pgfsetbuttcap%
\pgfsetroundjoin%
\pgfsetlinewidth{0.844684pt}%
\definecolor{currentstroke}{rgb}{0.201239,0.383670,0.554294}%
\pgfsetstrokecolor{currentstroke}%
\pgfsetdash{}{0pt}%
\pgfpathmoveto{\pgfqpoint{2.610125in}{5.073372in}}%
\pgfpathlineto{\pgfqpoint{2.560002in}{5.074834in}}%
\pgfusepath{stroke}%
\end{pgfscope}%
\begin{pgfscope}%
\pgfpathrectangle{\pgfqpoint{1.250000in}{4.155455in}}{\pgfqpoint{2.279412in}{2.004545in}}%
\pgfusepath{clip}%
\pgfsetbuttcap%
\pgfsetroundjoin%
\pgfsetlinewidth{0.910444pt}%
\definecolor{currentstroke}{rgb}{0.185556,0.418570,0.556753}%
\pgfsetstrokecolor{currentstroke}%
\pgfsetdash{}{0pt}%
\pgfpathmoveto{\pgfqpoint{2.560002in}{5.074834in}}%
\pgfpathlineto{\pgfqpoint{2.509886in}{5.076499in}}%
\pgfusepath{stroke}%
\end{pgfscope}%
\begin{pgfscope}%
\pgfpathrectangle{\pgfqpoint{1.250000in}{4.155455in}}{\pgfqpoint{2.279412in}{2.004545in}}%
\pgfusepath{clip}%
\pgfsetbuttcap%
\pgfsetroundjoin%
\pgfsetlinewidth{0.922568pt}%
\definecolor{currentstroke}{rgb}{0.182256,0.426184,0.557120}%
\pgfsetstrokecolor{currentstroke}%
\pgfsetdash{}{0pt}%
\pgfpathmoveto{\pgfqpoint{2.509886in}{5.076499in}}%
\pgfpathlineto{\pgfqpoint{2.459782in}{5.078402in}}%
\pgfusepath{stroke}%
\end{pgfscope}%
\begin{pgfscope}%
\pgfpathrectangle{\pgfqpoint{1.250000in}{4.155455in}}{\pgfqpoint{2.279412in}{2.004545in}}%
\pgfusepath{clip}%
\pgfsetbuttcap%
\pgfsetroundjoin%
\pgfsetlinewidth{0.937628pt}%
\definecolor{currentstroke}{rgb}{0.179019,0.433756,0.557430}%
\pgfsetstrokecolor{currentstroke}%
\pgfsetdash{}{0pt}%
\pgfpathmoveto{\pgfqpoint{2.459782in}{5.078402in}}%
\pgfpathlineto{\pgfqpoint{2.409700in}{5.080713in}}%
\pgfusepath{stroke}%
\end{pgfscope}%
\begin{pgfscope}%
\pgfpathrectangle{\pgfqpoint{1.250000in}{4.155455in}}{\pgfqpoint{2.279412in}{2.004545in}}%
\pgfusepath{clip}%
\pgfsetbuttcap%
\pgfsetroundjoin%
\pgfsetlinewidth{0.937728pt}%
\definecolor{currentstroke}{rgb}{0.179019,0.433756,0.557430}%
\pgfsetstrokecolor{currentstroke}%
\pgfsetdash{}{0pt}%
\pgfpathmoveto{\pgfqpoint{2.409700in}{5.080713in}}%
\pgfpathlineto{\pgfqpoint{2.359652in}{5.083516in}}%
\pgfusepath{stroke}%
\end{pgfscope}%
\begin{pgfscope}%
\pgfpathrectangle{\pgfqpoint{1.250000in}{4.155455in}}{\pgfqpoint{2.279412in}{2.004545in}}%
\pgfusepath{clip}%
\pgfsetbuttcap%
\pgfsetroundjoin%
\pgfsetlinewidth{0.792661pt}%
\definecolor{currentstroke}{rgb}{0.216210,0.351535,0.550627}%
\pgfsetstrokecolor{currentstroke}%
\pgfsetdash{}{0pt}%
\pgfpathmoveto{\pgfqpoint{2.359652in}{5.083516in}}%
\pgfpathlineto{\pgfqpoint{2.309688in}{5.087205in}}%
\pgfusepath{stroke}%
\end{pgfscope}%
\begin{pgfscope}%
\pgfpathrectangle{\pgfqpoint{1.250000in}{4.155455in}}{\pgfqpoint{2.279412in}{2.004545in}}%
\pgfusepath{clip}%
\pgfsetbuttcap%
\pgfsetroundjoin%
\pgfsetlinewidth{0.761476pt}%
\definecolor{currentstroke}{rgb}{0.223925,0.334994,0.548053}%
\pgfsetstrokecolor{currentstroke}%
\pgfsetdash{}{0pt}%
\pgfpathmoveto{\pgfqpoint{2.309688in}{5.087205in}}%
\pgfpathlineto{\pgfqpoint{2.259835in}{5.091895in}}%
\pgfusepath{stroke}%
\end{pgfscope}%
\begin{pgfscope}%
\pgfpathrectangle{\pgfqpoint{1.250000in}{4.155455in}}{\pgfqpoint{2.279412in}{2.004545in}}%
\pgfusepath{clip}%
\pgfsetbuttcap%
\pgfsetroundjoin%
\pgfsetlinewidth{0.631319pt}%
\definecolor{currentstroke}{rgb}{0.258965,0.251537,0.524736}%
\pgfsetstrokecolor{currentstroke}%
\pgfsetdash{}{0pt}%
\pgfpathmoveto{\pgfqpoint{2.259835in}{5.091895in}}%
\pgfpathlineto{\pgfqpoint{2.210326in}{5.098640in}}%
\pgfusepath{stroke}%
\end{pgfscope}%
\begin{pgfscope}%
\pgfpathrectangle{\pgfqpoint{1.250000in}{4.155455in}}{\pgfqpoint{2.279412in}{2.004545in}}%
\pgfusepath{clip}%
\pgfsetbuttcap%
\pgfsetroundjoin%
\pgfsetlinewidth{0.544955pt}%
\definecolor{currentstroke}{rgb}{0.276194,0.190074,0.493001}%
\pgfsetstrokecolor{currentstroke}%
\pgfsetdash{}{0pt}%
\pgfpathmoveto{\pgfqpoint{2.210326in}{5.098640in}}%
\pgfpathlineto{\pgfqpoint{2.210326in}{5.098640in}}%
\pgfusepath{stroke}%
\end{pgfscope}%
\begin{pgfscope}%
\pgfpathrectangle{\pgfqpoint{1.250000in}{4.155455in}}{\pgfqpoint{2.279412in}{2.004545in}}%
\pgfusepath{clip}%
\pgfsetbuttcap%
\pgfsetroundjoin%
\pgfsetlinewidth{0.544955pt}%
\definecolor{currentstroke}{rgb}{0.276194,0.190074,0.493001}%
\pgfsetstrokecolor{currentstroke}%
\pgfsetdash{}{0pt}%
\pgfpathmoveto{\pgfqpoint{2.210326in}{5.098640in}}%
\pgfpathlineto{\pgfqpoint{2.174776in}{5.107413in}}%
\pgfusepath{stroke}%
\end{pgfscope}%
\begin{pgfscope}%
\pgfpathrectangle{\pgfqpoint{1.250000in}{4.155455in}}{\pgfqpoint{2.279412in}{2.004545in}}%
\pgfusepath{clip}%
\pgfsetbuttcap%
\pgfsetroundjoin%
\pgfsetlinewidth{0.450733pt}%
\definecolor{currentstroke}{rgb}{0.283229,0.120777,0.440584}%
\pgfsetstrokecolor{currentstroke}%
\pgfsetdash{}{0pt}%
\pgfpathmoveto{\pgfqpoint{2.174776in}{5.107413in}}%
\pgfpathlineto{\pgfqpoint{2.174776in}{5.107413in}}%
\pgfusepath{stroke}%
\end{pgfscope}%
\begin{pgfscope}%
\pgfpathrectangle{\pgfqpoint{1.250000in}{4.155455in}}{\pgfqpoint{2.279412in}{2.004545in}}%
\pgfusepath{clip}%
\pgfsetbuttcap%
\pgfsetroundjoin%
\pgfsetlinewidth{0.450733pt}%
\definecolor{currentstroke}{rgb}{0.283229,0.120777,0.440584}%
\pgfsetstrokecolor{currentstroke}%
\pgfsetdash{}{0pt}%
\pgfpathmoveto{\pgfqpoint{2.174776in}{5.107413in}}%
\pgfpathlineto{\pgfqpoint{2.161238in}{5.112875in}}%
\pgfusepath{stroke}%
\end{pgfscope}%
\begin{pgfscope}%
\pgfpathrectangle{\pgfqpoint{1.250000in}{4.155455in}}{\pgfqpoint{2.279412in}{2.004545in}}%
\pgfusepath{clip}%
\pgfsetbuttcap%
\pgfsetroundjoin%
\pgfsetlinewidth{0.422897pt}%
\definecolor{currentstroke}{rgb}{0.282656,0.100196,0.422160}%
\pgfsetstrokecolor{currentstroke}%
\pgfsetdash{}{0pt}%
\pgfpathmoveto{\pgfqpoint{2.161238in}{5.112875in}}%
\pgfpathlineto{\pgfqpoint{2.161238in}{5.112875in}}%
\pgfusepath{stroke}%
\end{pgfscope}%
\begin{pgfscope}%
\pgfpathrectangle{\pgfqpoint{1.250000in}{4.155455in}}{\pgfqpoint{2.279412in}{2.004545in}}%
\pgfusepath{clip}%
\pgfsetbuttcap%
\pgfsetroundjoin%
\pgfsetlinewidth{0.422897pt}%
\definecolor{currentstroke}{rgb}{0.282656,0.100196,0.422160}%
\pgfsetstrokecolor{currentstroke}%
\pgfsetdash{}{0pt}%
\pgfpathmoveto{\pgfqpoint{2.161238in}{5.112875in}}%
\pgfpathlineto{\pgfqpoint{2.154652in}{5.117740in}}%
\pgfusepath{stroke}%
\end{pgfscope}%
\begin{pgfscope}%
\pgfpathrectangle{\pgfqpoint{1.250000in}{4.155455in}}{\pgfqpoint{2.279412in}{2.004545in}}%
\pgfusepath{clip}%
\pgfsetbuttcap%
\pgfsetroundjoin%
\pgfsetlinewidth{0.399226pt}%
\definecolor{currentstroke}{rgb}{0.281446,0.084320,0.407414}%
\pgfsetstrokecolor{currentstroke}%
\pgfsetdash{}{0pt}%
\pgfpathmoveto{\pgfqpoint{2.154652in}{5.117740in}}%
\pgfpathlineto{\pgfqpoint{2.151654in}{5.122910in}}%
\pgfusepath{stroke}%
\end{pgfscope}%
\begin{pgfscope}%
\pgfpathrectangle{\pgfqpoint{1.250000in}{4.155455in}}{\pgfqpoint{2.279412in}{2.004545in}}%
\pgfusepath{clip}%
\pgfsetbuttcap%
\pgfsetroundjoin%
\pgfsetlinewidth{0.389221pt}%
\definecolor{currentstroke}{rgb}{0.280267,0.073417,0.397163}%
\pgfsetstrokecolor{currentstroke}%
\pgfsetdash{}{0pt}%
\pgfpathmoveto{\pgfqpoint{2.151654in}{5.122910in}}%
\pgfpathlineto{\pgfqpoint{2.150420in}{5.127633in}}%
\pgfusepath{stroke}%
\end{pgfscope}%
\begin{pgfscope}%
\pgfpathrectangle{\pgfqpoint{1.250000in}{4.155455in}}{\pgfqpoint{2.279412in}{2.004545in}}%
\pgfusepath{clip}%
\pgfsetbuttcap%
\pgfsetroundjoin%
\pgfsetlinewidth{0.384643pt}%
\definecolor{currentstroke}{rgb}{0.280267,0.073417,0.397163}%
\pgfsetstrokecolor{currentstroke}%
\pgfsetdash{}{0pt}%
\pgfpathmoveto{\pgfqpoint{2.150420in}{5.127633in}}%
\pgfpathlineto{\pgfqpoint{2.150290in}{5.132189in}}%
\pgfusepath{stroke}%
\end{pgfscope}%
\begin{pgfscope}%
\pgfpathrectangle{\pgfqpoint{1.250000in}{4.155455in}}{\pgfqpoint{2.279412in}{2.004545in}}%
\pgfusepath{clip}%
\pgfsetbuttcap%
\pgfsetroundjoin%
\pgfsetlinewidth{0.382712pt}%
\definecolor{currentstroke}{rgb}{0.279566,0.067836,0.391917}%
\pgfsetstrokecolor{currentstroke}%
\pgfsetdash{}{0pt}%
\pgfpathmoveto{\pgfqpoint{2.150290in}{5.132189in}}%
\pgfpathlineto{\pgfqpoint{2.151053in}{5.136250in}}%
\pgfusepath{stroke}%
\end{pgfscope}%
\begin{pgfscope}%
\pgfpathrectangle{\pgfqpoint{1.250000in}{4.155455in}}{\pgfqpoint{2.279412in}{2.004545in}}%
\pgfusepath{clip}%
\pgfsetbuttcap%
\pgfsetroundjoin%
\pgfsetlinewidth{0.382720pt}%
\definecolor{currentstroke}{rgb}{0.279566,0.067836,0.391917}%
\pgfsetstrokecolor{currentstroke}%
\pgfsetdash{}{0pt}%
\pgfpathmoveto{\pgfqpoint{2.151053in}{5.136250in}}%
\pgfpathlineto{\pgfqpoint{2.152204in}{5.139287in}}%
\pgfusepath{stroke}%
\end{pgfscope}%
\begin{pgfscope}%
\pgfpathrectangle{\pgfqpoint{1.250000in}{4.155455in}}{\pgfqpoint{2.279412in}{2.004545in}}%
\pgfusepath{clip}%
\pgfsetbuttcap%
\pgfsetroundjoin%
\pgfsetlinewidth{0.384585pt}%
\definecolor{currentstroke}{rgb}{0.280267,0.073417,0.397163}%
\pgfsetstrokecolor{currentstroke}%
\pgfsetdash{}{0pt}%
\pgfpathmoveto{\pgfqpoint{2.152204in}{5.139287in}}%
\pgfpathlineto{\pgfqpoint{2.152060in}{5.141274in}}%
\pgfusepath{stroke}%
\end{pgfscope}%
\begin{pgfscope}%
\pgfpathrectangle{\pgfqpoint{1.250000in}{4.155455in}}{\pgfqpoint{2.279412in}{2.004545in}}%
\pgfusepath{clip}%
\pgfsetbuttcap%
\pgfsetroundjoin%
\pgfsetlinewidth{0.384857pt}%
\definecolor{currentstroke}{rgb}{0.280267,0.073417,0.397163}%
\pgfsetstrokecolor{currentstroke}%
\pgfsetdash{}{0pt}%
\pgfpathmoveto{\pgfqpoint{2.152060in}{5.141274in}}%
\pgfpathlineto{\pgfqpoint{2.151069in}{5.142644in}}%
\pgfusepath{stroke}%
\end{pgfscope}%
\begin{pgfscope}%
\pgfpathrectangle{\pgfqpoint{1.250000in}{4.155455in}}{\pgfqpoint{2.279412in}{2.004545in}}%
\pgfusepath{clip}%
\pgfsetbuttcap%
\pgfsetroundjoin%
\pgfsetlinewidth{0.384261pt}%
\definecolor{currentstroke}{rgb}{0.280267,0.073417,0.397163}%
\pgfsetstrokecolor{currentstroke}%
\pgfsetdash{}{0pt}%
\pgfpathmoveto{\pgfqpoint{2.151069in}{5.142644in}}%
\pgfpathlineto{\pgfqpoint{2.150506in}{5.143723in}}%
\pgfusepath{stroke}%
\end{pgfscope}%
\begin{pgfscope}%
\pgfpathrectangle{\pgfqpoint{1.250000in}{4.155455in}}{\pgfqpoint{2.279412in}{2.004545in}}%
\pgfusepath{clip}%
\pgfsetbuttcap%
\pgfsetroundjoin%
\pgfsetlinewidth{0.384328pt}%
\definecolor{currentstroke}{rgb}{0.280267,0.073417,0.397163}%
\pgfsetstrokecolor{currentstroke}%
\pgfsetdash{}{0pt}%
\pgfpathmoveto{\pgfqpoint{2.150506in}{5.143723in}}%
\pgfpathlineto{\pgfqpoint{2.150571in}{5.144600in}}%
\pgfusepath{stroke}%
\end{pgfscope}%
\begin{pgfscope}%
\pgfpathrectangle{\pgfqpoint{1.250000in}{4.155455in}}{\pgfqpoint{2.279412in}{2.004545in}}%
\pgfusepath{clip}%
\pgfsetbuttcap%
\pgfsetroundjoin%
\pgfsetlinewidth{0.384856pt}%
\definecolor{currentstroke}{rgb}{0.280267,0.073417,0.397163}%
\pgfsetstrokecolor{currentstroke}%
\pgfsetdash{}{0pt}%
\pgfpathmoveto{\pgfqpoint{2.150571in}{5.144600in}}%
\pgfpathlineto{\pgfqpoint{2.151180in}{5.145247in}}%
\pgfusepath{stroke}%
\end{pgfscope}%
\begin{pgfscope}%
\pgfpathrectangle{\pgfqpoint{1.250000in}{4.155455in}}{\pgfqpoint{2.279412in}{2.004545in}}%
\pgfusepath{clip}%
\pgfsetbuttcap%
\pgfsetroundjoin%
\pgfsetlinewidth{0.385478pt}%
\definecolor{currentstroke}{rgb}{0.280267,0.073417,0.397163}%
\pgfsetstrokecolor{currentstroke}%
\pgfsetdash{}{0pt}%
\pgfpathmoveto{\pgfqpoint{2.151180in}{5.145247in}}%
\pgfpathlineto{\pgfqpoint{2.151759in}{5.145673in}}%
\pgfusepath{stroke}%
\end{pgfscope}%
\begin{pgfscope}%
\pgfpathrectangle{\pgfqpoint{1.250000in}{4.155455in}}{\pgfqpoint{2.279412in}{2.004545in}}%
\pgfusepath{clip}%
\pgfsetbuttcap%
\pgfsetroundjoin%
\pgfsetlinewidth{0.385867pt}%
\definecolor{currentstroke}{rgb}{0.280267,0.073417,0.397163}%
\pgfsetstrokecolor{currentstroke}%
\pgfsetdash{}{0pt}%
\pgfpathmoveto{\pgfqpoint{2.151759in}{5.145673in}}%
\pgfpathlineto{\pgfqpoint{2.151522in}{5.145976in}}%
\pgfusepath{stroke}%
\end{pgfscope}%
\begin{pgfscope}%
\pgfpathrectangle{\pgfqpoint{1.250000in}{4.155455in}}{\pgfqpoint{2.279412in}{2.004545in}}%
\pgfusepath{clip}%
\pgfsetbuttcap%
\pgfsetroundjoin%
\pgfsetlinewidth{0.385891pt}%
\definecolor{currentstroke}{rgb}{0.280267,0.073417,0.397163}%
\pgfsetstrokecolor{currentstroke}%
\pgfsetdash{}{0pt}%
\pgfpathmoveto{\pgfqpoint{2.151522in}{5.145976in}}%
\pgfpathlineto{\pgfqpoint{2.150929in}{5.146205in}}%
\pgfusepath{stroke}%
\end{pgfscope}%
\begin{pgfscope}%
\pgfpathrectangle{\pgfqpoint{1.250000in}{4.155455in}}{\pgfqpoint{2.279412in}{2.004545in}}%
\pgfusepath{clip}%
\pgfsetbuttcap%
\pgfsetroundjoin%
\pgfsetlinewidth{0.385823pt}%
\definecolor{currentstroke}{rgb}{0.280267,0.073417,0.397163}%
\pgfsetstrokecolor{currentstroke}%
\pgfsetdash{}{0pt}%
\pgfpathmoveto{\pgfqpoint{2.150929in}{5.146205in}}%
\pgfpathlineto{\pgfqpoint{2.150636in}{5.146368in}}%
\pgfusepath{stroke}%
\end{pgfscope}%
\begin{pgfscope}%
\pgfpathrectangle{\pgfqpoint{1.250000in}{4.155455in}}{\pgfqpoint{2.279412in}{2.004545in}}%
\pgfusepath{clip}%
\pgfsetbuttcap%
\pgfsetroundjoin%
\pgfsetlinewidth{0.385835pt}%
\definecolor{currentstroke}{rgb}{0.280267,0.073417,0.397163}%
\pgfsetstrokecolor{currentstroke}%
\pgfsetdash{}{0pt}%
\pgfpathmoveto{\pgfqpoint{2.150636in}{5.146368in}}%
\pgfpathlineto{\pgfqpoint{2.150792in}{5.146469in}}%
\pgfusepath{stroke}%
\end{pgfscope}%
\begin{pgfscope}%
\pgfpathrectangle{\pgfqpoint{1.250000in}{4.155455in}}{\pgfqpoint{2.279412in}{2.004545in}}%
\pgfusepath{clip}%
\pgfsetbuttcap%
\pgfsetroundjoin%
\pgfsetlinewidth{0.385921pt}%
\definecolor{currentstroke}{rgb}{0.280267,0.073417,0.397163}%
\pgfsetstrokecolor{currentstroke}%
\pgfsetdash{}{0pt}%
\pgfpathmoveto{\pgfqpoint{2.150792in}{5.146469in}}%
\pgfpathlineto{\pgfqpoint{2.151250in}{5.146522in}}%
\pgfusepath{stroke}%
\end{pgfscope}%
\begin{pgfscope}%
\pgfpathrectangle{\pgfqpoint{1.250000in}{4.155455in}}{\pgfqpoint{2.279412in}{2.004545in}}%
\pgfusepath{clip}%
\pgfsetbuttcap%
\pgfsetroundjoin%
\pgfsetlinewidth{0.386035pt}%
\definecolor{currentstroke}{rgb}{0.280267,0.073417,0.397163}%
\pgfsetstrokecolor{currentstroke}%
\pgfsetdash{}{0pt}%
\pgfpathmoveto{\pgfqpoint{2.151250in}{5.146522in}}%
\pgfpathlineto{\pgfqpoint{2.151512in}{5.146560in}}%
\pgfusepath{stroke}%
\end{pgfscope}%
\begin{pgfscope}%
\pgfpathrectangle{\pgfqpoint{1.250000in}{4.155455in}}{\pgfqpoint{2.279412in}{2.004545in}}%
\pgfusepath{clip}%
\pgfsetbuttcap%
\pgfsetroundjoin%
\pgfsetlinewidth{0.386099pt}%
\definecolor{currentstroke}{rgb}{0.280267,0.073417,0.397163}%
\pgfsetstrokecolor{currentstroke}%
\pgfsetdash{}{0pt}%
\pgfpathmoveto{\pgfqpoint{2.151512in}{5.146560in}}%
\pgfpathlineto{\pgfqpoint{2.151264in}{5.146609in}}%
\pgfusepath{stroke}%
\end{pgfscope}%
\begin{pgfscope}%
\pgfpathrectangle{\pgfqpoint{1.250000in}{4.155455in}}{\pgfqpoint{2.279412in}{2.004545in}}%
\pgfusepath{clip}%
\pgfsetbuttcap%
\pgfsetroundjoin%
\pgfsetlinewidth{0.386074pt}%
\definecolor{currentstroke}{rgb}{0.280267,0.073417,0.397163}%
\pgfsetstrokecolor{currentstroke}%
\pgfsetdash{}{0pt}%
\pgfpathmoveto{\pgfqpoint{2.151264in}{5.146609in}}%
\pgfpathlineto{\pgfqpoint{2.150888in}{5.146655in}}%
\pgfusepath{stroke}%
\end{pgfscope}%
\begin{pgfscope}%
\pgfpathrectangle{\pgfqpoint{1.250000in}{4.155455in}}{\pgfqpoint{2.279412in}{2.004545in}}%
\pgfusepath{clip}%
\pgfsetbuttcap%
\pgfsetroundjoin%
\pgfsetlinewidth{0.386030pt}%
\definecolor{currentstroke}{rgb}{0.280267,0.073417,0.397163}%
\pgfsetstrokecolor{currentstroke}%
\pgfsetdash{}{0pt}%
\pgfpathmoveto{\pgfqpoint{2.150888in}{5.146655in}}%
\pgfpathlineto{\pgfqpoint{2.150759in}{5.146680in}}%
\pgfusepath{stroke}%
\end{pgfscope}%
\begin{pgfscope}%
\pgfpathrectangle{\pgfqpoint{1.250000in}{4.155455in}}{\pgfqpoint{2.279412in}{2.004545in}}%
\pgfusepath{clip}%
\pgfsetbuttcap%
\pgfsetroundjoin%
\pgfsetlinewidth{0.386022pt}%
\definecolor{currentstroke}{rgb}{0.280267,0.073417,0.397163}%
\pgfsetstrokecolor{currentstroke}%
\pgfsetdash{}{0pt}%
\pgfpathmoveto{\pgfqpoint{2.150759in}{5.146680in}}%
\pgfpathlineto{\pgfqpoint{2.150943in}{5.146681in}}%
\pgfusepath{stroke}%
\end{pgfscope}%
\begin{pgfscope}%
\pgfpathrectangle{\pgfqpoint{1.250000in}{4.155455in}}{\pgfqpoint{2.279412in}{2.004545in}}%
\pgfusepath{clip}%
\pgfsetbuttcap%
\pgfsetroundjoin%
\pgfsetlinewidth{0.386052pt}%
\definecolor{currentstroke}{rgb}{0.280267,0.073417,0.397163}%
\pgfsetstrokecolor{currentstroke}%
\pgfsetdash{}{0pt}%
\pgfpathmoveto{\pgfqpoint{2.150943in}{5.146681in}}%
\pgfpathlineto{\pgfqpoint{2.151255in}{5.146672in}}%
\pgfusepath{stroke}%
\end{pgfscope}%
\begin{pgfscope}%
\pgfpathrectangle{\pgfqpoint{1.250000in}{4.155455in}}{\pgfqpoint{2.279412in}{2.004545in}}%
\pgfusepath{clip}%
\pgfsetbuttcap%
\pgfsetroundjoin%
\pgfsetlinewidth{0.386098pt}%
\definecolor{currentstroke}{rgb}{0.280267,0.073417,0.397163}%
\pgfsetstrokecolor{currentstroke}%
\pgfsetdash{}{0pt}%
\pgfpathmoveto{\pgfqpoint{2.151255in}{5.146672in}}%
\pgfpathlineto{\pgfqpoint{2.151342in}{5.146673in}}%
\pgfusepath{stroke}%
\end{pgfscope}%
\begin{pgfscope}%
\pgfpathrectangle{\pgfqpoint{1.250000in}{4.155455in}}{\pgfqpoint{2.279412in}{2.004545in}}%
\pgfusepath{clip}%
\pgfsetbuttcap%
\pgfsetroundjoin%
\pgfsetlinewidth{0.386113pt}%
\definecolor{currentstroke}{rgb}{0.280267,0.073417,0.397163}%
\pgfsetstrokecolor{currentstroke}%
\pgfsetdash{}{0pt}%
\pgfpathmoveto{\pgfqpoint{2.151342in}{5.146673in}}%
\pgfpathlineto{\pgfqpoint{2.151127in}{5.146690in}}%
\pgfusepath{stroke}%
\end{pgfscope}%
\begin{pgfscope}%
\pgfpathrectangle{\pgfqpoint{1.250000in}{4.155455in}}{\pgfqpoint{2.279412in}{2.004545in}}%
\pgfusepath{clip}%
\pgfsetbuttcap%
\pgfsetroundjoin%
\pgfsetlinewidth{0.386085pt}%
\definecolor{currentstroke}{rgb}{0.280267,0.073417,0.397163}%
\pgfsetstrokecolor{currentstroke}%
\pgfsetdash{}{0pt}%
\pgfpathmoveto{\pgfqpoint{2.151127in}{5.146690in}}%
\pgfpathlineto{\pgfqpoint{2.150895in}{5.146706in}}%
\pgfusepath{stroke}%
\end{pgfscope}%
\begin{pgfscope}%
\pgfpathrectangle{\pgfqpoint{1.250000in}{4.155455in}}{\pgfqpoint{2.279412in}{2.004545in}}%
\pgfusepath{clip}%
\pgfsetbuttcap%
\pgfsetroundjoin%
\pgfsetlinewidth{0.386056pt}%
\definecolor{currentstroke}{rgb}{0.280267,0.073417,0.397163}%
\pgfsetstrokecolor{currentstroke}%
\pgfsetdash{}{0pt}%
\pgfpathmoveto{\pgfqpoint{2.150895in}{5.146706in}}%
\pgfpathlineto{\pgfqpoint{2.150862in}{5.146710in}}%
\pgfusepath{stroke}%
\end{pgfscope}%
\begin{pgfscope}%
\pgfpathrectangle{\pgfqpoint{1.250000in}{4.155455in}}{\pgfqpoint{2.279412in}{2.004545in}}%
\pgfusepath{clip}%
\pgfsetbuttcap%
\pgfsetroundjoin%
\pgfsetlinewidth{0.386053pt}%
\definecolor{currentstroke}{rgb}{0.280267,0.073417,0.397163}%
\pgfsetstrokecolor{currentstroke}%
\pgfsetdash{}{0pt}%
\pgfpathmoveto{\pgfqpoint{2.150862in}{5.146710in}}%
\pgfpathlineto{\pgfqpoint{2.151034in}{5.146701in}}%
\pgfusepath{stroke}%
\end{pgfscope}%
\begin{pgfscope}%
\pgfpathrectangle{\pgfqpoint{1.250000in}{4.155455in}}{\pgfqpoint{2.279412in}{2.004545in}}%
\pgfusepath{clip}%
\pgfsetbuttcap%
\pgfsetroundjoin%
\pgfsetlinewidth{0.386075pt}%
\definecolor{currentstroke}{rgb}{0.280267,0.073417,0.397163}%
\pgfsetstrokecolor{currentstroke}%
\pgfsetdash{}{0pt}%
\pgfpathmoveto{\pgfqpoint{2.151034in}{5.146701in}}%
\pgfpathlineto{\pgfqpoint{2.151227in}{5.146691in}}%
\pgfusepath{stroke}%
\end{pgfscope}%
\begin{pgfscope}%
\pgfpathrectangle{\pgfqpoint{1.250000in}{4.155455in}}{\pgfqpoint{2.279412in}{2.004545in}}%
\pgfusepath{clip}%
\pgfsetbuttcap%
\pgfsetroundjoin%
\pgfsetlinewidth{0.386101pt}%
\definecolor{currentstroke}{rgb}{0.280267,0.073417,0.397163}%
\pgfsetstrokecolor{currentstroke}%
\pgfsetdash{}{0pt}%
\pgfpathmoveto{\pgfqpoint{2.151227in}{5.146691in}}%
\pgfpathlineto{\pgfqpoint{2.151226in}{5.146691in}}%
\pgfusepath{stroke}%
\end{pgfscope}%
\begin{pgfscope}%
\pgfpathrectangle{\pgfqpoint{1.250000in}{4.155455in}}{\pgfqpoint{2.279412in}{2.004545in}}%
\pgfusepath{clip}%
\pgfsetbuttcap%
\pgfsetroundjoin%
\pgfsetlinewidth{0.386102pt}%
\definecolor{currentstroke}{rgb}{0.280267,0.073417,0.397163}%
\pgfsetstrokecolor{currentstroke}%
\pgfsetdash{}{0pt}%
\pgfpathmoveto{\pgfqpoint{2.151226in}{5.146691in}}%
\pgfpathlineto{\pgfqpoint{2.151057in}{5.146702in}}%
\pgfusepath{stroke}%
\end{pgfscope}%
\begin{pgfscope}%
\pgfpathrectangle{\pgfqpoint{1.250000in}{4.155455in}}{\pgfqpoint{2.279412in}{2.004545in}}%
\pgfusepath{clip}%
\pgfsetbuttcap%
\pgfsetroundjoin%
\pgfsetlinewidth{0.386080pt}%
\definecolor{currentstroke}{rgb}{0.280267,0.073417,0.397163}%
\pgfsetstrokecolor{currentstroke}%
\pgfsetdash{}{0pt}%
\pgfpathmoveto{\pgfqpoint{2.151057in}{5.146702in}}%
\pgfpathlineto{\pgfqpoint{2.150922in}{5.146710in}}%
\pgfusepath{stroke}%
\end{pgfscope}%
\begin{pgfscope}%
\pgfpathrectangle{\pgfqpoint{1.250000in}{4.155455in}}{\pgfqpoint{2.279412in}{2.004545in}}%
\pgfusepath{clip}%
\pgfsetbuttcap%
\pgfsetroundjoin%
\pgfsetlinewidth{0.386062pt}%
\definecolor{currentstroke}{rgb}{0.280267,0.073417,0.397163}%
\pgfsetstrokecolor{currentstroke}%
\pgfsetdash{}{0pt}%
\pgfpathmoveto{\pgfqpoint{2.150922in}{5.146710in}}%
\pgfpathlineto{\pgfqpoint{2.150942in}{5.146709in}}%
\pgfusepath{stroke}%
\end{pgfscope}%
\begin{pgfscope}%
\pgfpathrectangle{\pgfqpoint{1.250000in}{4.155455in}}{\pgfqpoint{2.279412in}{2.004545in}}%
\pgfusepath{clip}%
\pgfsetbuttcap%
\pgfsetroundjoin%
\pgfsetlinewidth{0.386065pt}%
\definecolor{currentstroke}{rgb}{0.280267,0.073417,0.397163}%
\pgfsetstrokecolor{currentstroke}%
\pgfsetdash{}{0pt}%
\pgfpathmoveto{\pgfqpoint{2.150942in}{5.146709in}}%
\pgfpathlineto{\pgfqpoint{2.151081in}{5.146701in}}%
\pgfusepath{stroke}%
\end{pgfscope}%
\begin{pgfscope}%
\pgfpathrectangle{\pgfqpoint{1.250000in}{4.155455in}}{\pgfqpoint{2.279412in}{2.004545in}}%
\pgfusepath{clip}%
\pgfsetbuttcap%
\pgfsetroundjoin%
\pgfsetlinewidth{0.386083pt}%
\definecolor{currentstroke}{rgb}{0.280267,0.073417,0.397163}%
\pgfsetstrokecolor{currentstroke}%
\pgfsetdash{}{0pt}%
\pgfpathmoveto{\pgfqpoint{2.151081in}{5.146701in}}%
\pgfpathlineto{\pgfqpoint{2.151188in}{5.146695in}}%
\pgfusepath{stroke}%
\end{pgfscope}%
\begin{pgfscope}%
\pgfpathrectangle{\pgfqpoint{1.250000in}{4.155455in}}{\pgfqpoint{2.279412in}{2.004545in}}%
\pgfusepath{clip}%
\pgfsetbuttcap%
\pgfsetroundjoin%
\pgfsetlinewidth{0.386097pt}%
\definecolor{currentstroke}{rgb}{0.280267,0.073417,0.397163}%
\pgfsetstrokecolor{currentstroke}%
\pgfsetdash{}{0pt}%
\pgfpathmoveto{\pgfqpoint{2.151188in}{5.146695in}}%
\pgfpathlineto{\pgfqpoint{2.151149in}{5.146697in}}%
\pgfusepath{stroke}%
\end{pgfscope}%
\begin{pgfscope}%
\pgfpathrectangle{\pgfqpoint{1.250000in}{4.155455in}}{\pgfqpoint{2.279412in}{2.004545in}}%
\pgfusepath{clip}%
\pgfsetbuttcap%
\pgfsetroundjoin%
\pgfsetlinewidth{0.386092pt}%
\definecolor{currentstroke}{rgb}{0.280267,0.073417,0.397163}%
\pgfsetstrokecolor{currentstroke}%
\pgfsetdash{}{0pt}%
\pgfpathmoveto{\pgfqpoint{2.151149in}{5.146697in}}%
\pgfpathlineto{\pgfqpoint{2.151027in}{5.146705in}}%
\pgfusepath{stroke}%
\end{pgfscope}%
\begin{pgfscope}%
\pgfpathrectangle{\pgfqpoint{1.250000in}{4.155455in}}{\pgfqpoint{2.279412in}{2.004545in}}%
\pgfusepath{clip}%
\pgfsetbuttcap%
\pgfsetroundjoin%
\pgfsetlinewidth{0.386076pt}%
\definecolor{currentstroke}{rgb}{0.280267,0.073417,0.397163}%
\pgfsetstrokecolor{currentstroke}%
\pgfsetdash{}{0pt}%
\pgfpathmoveto{\pgfqpoint{2.151027in}{5.146705in}}%
\pgfpathlineto{\pgfqpoint{2.150956in}{5.146709in}}%
\pgfusepath{stroke}%
\end{pgfscope}%
\begin{pgfscope}%
\pgfpathrectangle{\pgfqpoint{1.250000in}{4.155455in}}{\pgfqpoint{2.279412in}{2.004545in}}%
\pgfusepath{clip}%
\pgfsetbuttcap%
\pgfsetroundjoin%
\pgfsetlinewidth{0.386067pt}%
\definecolor{currentstroke}{rgb}{0.280267,0.073417,0.397163}%
\pgfsetstrokecolor{currentstroke}%
\pgfsetdash{}{0pt}%
\pgfpathmoveto{\pgfqpoint{2.150956in}{5.146709in}}%
\pgfpathlineto{\pgfqpoint{2.150998in}{5.146706in}}%
\pgfusepath{stroke}%
\end{pgfscope}%
\begin{pgfscope}%
\pgfpathrectangle{\pgfqpoint{1.250000in}{4.155455in}}{\pgfqpoint{2.279412in}{2.004545in}}%
\pgfusepath{clip}%
\pgfsetbuttcap%
\pgfsetroundjoin%
\pgfsetlinewidth{0.386072pt}%
\definecolor{currentstroke}{rgb}{0.280267,0.073417,0.397163}%
\pgfsetstrokecolor{currentstroke}%
\pgfsetdash{}{0pt}%
\pgfpathmoveto{\pgfqpoint{2.150998in}{5.146706in}}%
\pgfpathlineto{\pgfqpoint{2.151099in}{5.146700in}}%
\pgfusepath{stroke}%
\end{pgfscope}%
\begin{pgfscope}%
\pgfpathrectangle{\pgfqpoint{1.250000in}{4.155455in}}{\pgfqpoint{2.279412in}{2.004545in}}%
\pgfusepath{clip}%
\pgfsetbuttcap%
\pgfsetroundjoin%
\pgfsetlinewidth{0.386085pt}%
\definecolor{currentstroke}{rgb}{0.280267,0.073417,0.397163}%
\pgfsetstrokecolor{currentstroke}%
\pgfsetdash{}{0pt}%
\pgfpathmoveto{\pgfqpoint{2.151099in}{5.146700in}}%
\pgfpathlineto{\pgfqpoint{2.151149in}{5.146697in}}%
\pgfusepath{stroke}%
\end{pgfscope}%
\begin{pgfscope}%
\pgfpathrectangle{\pgfqpoint{1.250000in}{4.155455in}}{\pgfqpoint{2.279412in}{2.004545in}}%
\pgfusepath{clip}%
\pgfsetbuttcap%
\pgfsetroundjoin%
\pgfsetlinewidth{0.386092pt}%
\definecolor{currentstroke}{rgb}{0.280267,0.073417,0.397163}%
\pgfsetstrokecolor{currentstroke}%
\pgfsetdash{}{0pt}%
\pgfpathmoveto{\pgfqpoint{2.151149in}{5.146697in}}%
\pgfpathlineto{\pgfqpoint{2.151101in}{5.146700in}}%
\pgfusepath{stroke}%
\end{pgfscope}%
\begin{pgfscope}%
\pgfpathrectangle{\pgfqpoint{1.250000in}{4.155455in}}{\pgfqpoint{2.279412in}{2.004545in}}%
\pgfusepath{clip}%
\pgfsetbuttcap%
\pgfsetroundjoin%
\pgfsetlinewidth{0.386086pt}%
\definecolor{currentstroke}{rgb}{0.280267,0.073417,0.397163}%
\pgfsetstrokecolor{currentstroke}%
\pgfsetdash{}{0pt}%
\pgfpathmoveto{\pgfqpoint{2.151101in}{5.146700in}}%
\pgfpathlineto{\pgfqpoint{2.151017in}{5.146705in}}%
\pgfusepath{stroke}%
\end{pgfscope}%
\begin{pgfscope}%
\pgfpathrectangle{\pgfqpoint{1.250000in}{4.155455in}}{\pgfqpoint{2.279412in}{2.004545in}}%
\pgfusepath{clip}%
\pgfsetbuttcap%
\pgfsetroundjoin%
\pgfsetlinewidth{0.386075pt}%
\definecolor{currentstroke}{rgb}{0.280267,0.073417,0.397163}%
\pgfsetstrokecolor{currentstroke}%
\pgfsetdash{}{0pt}%
\pgfpathmoveto{\pgfqpoint{2.151017in}{5.146705in}}%
\pgfpathlineto{\pgfqpoint{2.150988in}{5.146707in}}%
\pgfusepath{stroke}%
\end{pgfscope}%
\begin{pgfscope}%
\pgfpathrectangle{\pgfqpoint{1.250000in}{4.155455in}}{\pgfqpoint{2.279412in}{2.004545in}}%
\pgfusepath{clip}%
\pgfsetbuttcap%
\pgfsetroundjoin%
\pgfsetlinewidth{0.386071pt}%
\definecolor{currentstroke}{rgb}{0.280267,0.073417,0.397163}%
\pgfsetstrokecolor{currentstroke}%
\pgfsetdash{}{0pt}%
\pgfpathmoveto{\pgfqpoint{2.150988in}{5.146707in}}%
\pgfpathlineto{\pgfqpoint{2.151034in}{5.146704in}}%
\pgfusepath{stroke}%
\end{pgfscope}%
\begin{pgfscope}%
\pgfpathrectangle{\pgfqpoint{1.250000in}{4.155455in}}{\pgfqpoint{2.279412in}{2.004545in}}%
\pgfusepath{clip}%
\pgfsetbuttcap%
\pgfsetroundjoin%
\pgfsetlinewidth{0.386077pt}%
\definecolor{currentstroke}{rgb}{0.280267,0.073417,0.397163}%
\pgfsetstrokecolor{currentstroke}%
\pgfsetdash{}{0pt}%
\pgfpathmoveto{\pgfqpoint{2.151034in}{5.146704in}}%
\pgfpathlineto{\pgfqpoint{2.151101in}{5.146700in}}%
\pgfusepath{stroke}%
\end{pgfscope}%
\begin{pgfscope}%
\pgfpathrectangle{\pgfqpoint{1.250000in}{4.155455in}}{\pgfqpoint{2.279412in}{2.004545in}}%
\pgfusepath{clip}%
\pgfsetbuttcap%
\pgfsetroundjoin%
\pgfsetlinewidth{0.386086pt}%
\definecolor{currentstroke}{rgb}{0.280267,0.073417,0.397163}%
\pgfsetstrokecolor{currentstroke}%
\pgfsetdash{}{0pt}%
\pgfpathmoveto{\pgfqpoint{2.151101in}{5.146700in}}%
\pgfpathlineto{\pgfqpoint{2.151118in}{5.146699in}}%
\pgfusepath{stroke}%
\end{pgfscope}%
\begin{pgfscope}%
\pgfpathrectangle{\pgfqpoint{1.250000in}{4.155455in}}{\pgfqpoint{2.279412in}{2.004545in}}%
\pgfusepath{clip}%
\pgfsetbuttcap%
\pgfsetroundjoin%
\pgfsetlinewidth{0.386088pt}%
\definecolor{currentstroke}{rgb}{0.280267,0.073417,0.397163}%
\pgfsetstrokecolor{currentstroke}%
\pgfsetdash{}{0pt}%
\pgfpathmoveto{\pgfqpoint{2.151118in}{5.146699in}}%
\pgfpathlineto{\pgfqpoint{2.151072in}{5.146702in}}%
\pgfusepath{stroke}%
\end{pgfscope}%
\begin{pgfscope}%
\pgfpathrectangle{\pgfqpoint{1.250000in}{4.155455in}}{\pgfqpoint{2.279412in}{2.004545in}}%
\pgfusepath{clip}%
\pgfsetbuttcap%
\pgfsetroundjoin%
\pgfsetlinewidth{0.386082pt}%
\definecolor{currentstroke}{rgb}{0.280267,0.073417,0.397163}%
\pgfsetstrokecolor{currentstroke}%
\pgfsetdash{}{0pt}%
\pgfpathmoveto{\pgfqpoint{2.151072in}{5.146702in}}%
\pgfpathlineto{\pgfqpoint{2.151020in}{5.146705in}}%
\pgfusepath{stroke}%
\end{pgfscope}%
\begin{pgfscope}%
\pgfpathrectangle{\pgfqpoint{1.250000in}{4.155455in}}{\pgfqpoint{2.279412in}{2.004545in}}%
\pgfusepath{clip}%
\pgfsetbuttcap%
\pgfsetroundjoin%
\pgfsetlinewidth{0.386075pt}%
\definecolor{currentstroke}{rgb}{0.280267,0.073417,0.397163}%
\pgfsetstrokecolor{currentstroke}%
\pgfsetdash{}{0pt}%
\pgfpathmoveto{\pgfqpoint{2.151020in}{5.146705in}}%
\pgfpathlineto{\pgfqpoint{2.151013in}{5.146706in}}%
\pgfusepath{stroke}%
\end{pgfscope}%
\begin{pgfscope}%
\pgfpathrectangle{\pgfqpoint{1.250000in}{4.155455in}}{\pgfqpoint{2.279412in}{2.004545in}}%
\pgfusepath{clip}%
\pgfsetbuttcap%
\pgfsetroundjoin%
\pgfsetlinewidth{0.386074pt}%
\definecolor{currentstroke}{rgb}{0.280267,0.073417,0.397163}%
\pgfsetstrokecolor{currentstroke}%
\pgfsetdash{}{0pt}%
\pgfpathmoveto{\pgfqpoint{2.151013in}{5.146706in}}%
\pgfpathlineto{\pgfqpoint{2.151054in}{5.146703in}}%
\pgfusepath{stroke}%
\end{pgfscope}%
\begin{pgfscope}%
\pgfpathrectangle{\pgfqpoint{1.250000in}{4.155455in}}{\pgfqpoint{2.279412in}{2.004545in}}%
\pgfusepath{clip}%
\pgfsetbuttcap%
\pgfsetroundjoin%
\pgfsetlinewidth{0.386080pt}%
\definecolor{currentstroke}{rgb}{0.280267,0.073417,0.397163}%
\pgfsetstrokecolor{currentstroke}%
\pgfsetdash{}{0pt}%
\pgfpathmoveto{\pgfqpoint{2.151054in}{5.146703in}}%
\pgfpathlineto{\pgfqpoint{2.151095in}{5.146701in}}%
\pgfusepath{stroke}%
\end{pgfscope}%
\begin{pgfscope}%
\pgfpathrectangle{\pgfqpoint{1.250000in}{4.155455in}}{\pgfqpoint{2.279412in}{2.004545in}}%
\pgfusepath{clip}%
\pgfsetbuttcap%
\pgfsetroundjoin%
\pgfsetlinewidth{0.386085pt}%
\definecolor{currentstroke}{rgb}{0.280267,0.073417,0.397163}%
\pgfsetstrokecolor{currentstroke}%
\pgfsetdash{}{0pt}%
\pgfpathmoveto{\pgfqpoint{2.151095in}{5.146701in}}%
\pgfpathlineto{\pgfqpoint{2.151094in}{5.146701in}}%
\pgfusepath{stroke}%
\end{pgfscope}%
\begin{pgfscope}%
\pgfpathrectangle{\pgfqpoint{1.250000in}{4.155455in}}{\pgfqpoint{2.279412in}{2.004545in}}%
\pgfusepath{clip}%
\pgfsetbuttcap%
\pgfsetroundjoin%
\pgfsetlinewidth{0.386085pt}%
\definecolor{currentstroke}{rgb}{0.280267,0.073417,0.397163}%
\pgfsetstrokecolor{currentstroke}%
\pgfsetdash{}{0pt}%
\pgfpathmoveto{\pgfqpoint{2.151094in}{5.146701in}}%
\pgfpathlineto{\pgfqpoint{2.151057in}{5.146703in}}%
\pgfusepath{stroke}%
\end{pgfscope}%
\begin{pgfscope}%
\pgfpathrectangle{\pgfqpoint{1.250000in}{4.155455in}}{\pgfqpoint{2.279412in}{2.004545in}}%
\pgfusepath{clip}%
\pgfsetbuttcap%
\pgfsetroundjoin%
\pgfsetlinewidth{0.386080pt}%
\definecolor{currentstroke}{rgb}{0.280267,0.073417,0.397163}%
\pgfsetstrokecolor{currentstroke}%
\pgfsetdash{}{0pt}%
\pgfpathmoveto{\pgfqpoint{2.151057in}{5.146703in}}%
\pgfpathlineto{\pgfqpoint{2.151027in}{5.146705in}}%
\pgfusepath{stroke}%
\end{pgfscope}%
\begin{pgfscope}%
\pgfpathrectangle{\pgfqpoint{1.250000in}{4.155455in}}{\pgfqpoint{2.279412in}{2.004545in}}%
\pgfusepath{clip}%
\pgfsetbuttcap%
\pgfsetroundjoin%
\pgfsetlinewidth{0.386076pt}%
\definecolor{currentstroke}{rgb}{0.280267,0.073417,0.397163}%
\pgfsetstrokecolor{currentstroke}%
\pgfsetdash{}{0pt}%
\pgfpathmoveto{\pgfqpoint{2.151027in}{5.146705in}}%
\pgfpathlineto{\pgfqpoint{2.151032in}{5.146704in}}%
\pgfusepath{stroke}%
\end{pgfscope}%
\begin{pgfscope}%
\pgfpathrectangle{\pgfqpoint{1.250000in}{4.155455in}}{\pgfqpoint{2.279412in}{2.004545in}}%
\pgfusepath{clip}%
\pgfsetbuttcap%
\pgfsetroundjoin%
\pgfsetlinewidth{0.386077pt}%
\definecolor{currentstroke}{rgb}{0.280267,0.073417,0.397163}%
\pgfsetstrokecolor{currentstroke}%
\pgfsetdash{}{0pt}%
\pgfpathmoveto{\pgfqpoint{2.151032in}{5.146704in}}%
\pgfpathlineto{\pgfqpoint{2.151064in}{5.146702in}}%
\pgfusepath{stroke}%
\end{pgfscope}%
\begin{pgfscope}%
\pgfpathrectangle{\pgfqpoint{1.250000in}{4.155455in}}{\pgfqpoint{2.279412in}{2.004545in}}%
\pgfusepath{clip}%
\pgfsetbuttcap%
\pgfsetroundjoin%
\pgfsetlinewidth{0.386081pt}%
\definecolor{currentstroke}{rgb}{0.280267,0.073417,0.397163}%
\pgfsetstrokecolor{currentstroke}%
\pgfsetdash{}{0pt}%
\pgfpathmoveto{\pgfqpoint{2.151064in}{5.146702in}}%
\pgfpathlineto{\pgfqpoint{2.151087in}{5.146701in}}%
\pgfusepath{stroke}%
\end{pgfscope}%
\begin{pgfscope}%
\pgfpathrectangle{\pgfqpoint{1.250000in}{4.155455in}}{\pgfqpoint{2.279412in}{2.004545in}}%
\pgfusepath{clip}%
\pgfsetbuttcap%
\pgfsetroundjoin%
\pgfsetlinewidth{0.386084pt}%
\definecolor{currentstroke}{rgb}{0.280267,0.073417,0.397163}%
\pgfsetstrokecolor{currentstroke}%
\pgfsetdash{}{0pt}%
\pgfpathmoveto{\pgfqpoint{2.151087in}{5.146701in}}%
\pgfpathlineto{\pgfqpoint{2.151078in}{5.146702in}}%
\pgfusepath{stroke}%
\end{pgfscope}%
\begin{pgfscope}%
\pgfpathrectangle{\pgfqpoint{1.250000in}{4.155455in}}{\pgfqpoint{2.279412in}{2.004545in}}%
\pgfusepath{clip}%
\pgfsetbuttcap%
\pgfsetroundjoin%
\pgfsetlinewidth{0.386083pt}%
\definecolor{currentstroke}{rgb}{0.280267,0.073417,0.397163}%
\pgfsetstrokecolor{currentstroke}%
\pgfsetdash{}{0pt}%
\pgfpathmoveto{\pgfqpoint{2.151078in}{5.146702in}}%
\pgfpathlineto{\pgfqpoint{2.151051in}{5.146703in}}%
\pgfusepath{stroke}%
\end{pgfscope}%
\begin{pgfscope}%
\pgfpathrectangle{\pgfqpoint{1.250000in}{4.155455in}}{\pgfqpoint{2.279412in}{2.004545in}}%
\pgfusepath{clip}%
\pgfsetbuttcap%
\pgfsetroundjoin%
\pgfsetlinewidth{0.386079pt}%
\definecolor{currentstroke}{rgb}{0.280267,0.073417,0.397163}%
\pgfsetstrokecolor{currentstroke}%
\pgfsetdash{}{0pt}%
\pgfpathmoveto{\pgfqpoint{2.151051in}{5.146703in}}%
\pgfpathlineto{\pgfqpoint{2.151035in}{5.146704in}}%
\pgfusepath{stroke}%
\end{pgfscope}%
\begin{pgfscope}%
\pgfpathrectangle{\pgfqpoint{1.250000in}{4.155455in}}{\pgfqpoint{2.279412in}{2.004545in}}%
\pgfusepath{clip}%
\pgfsetbuttcap%
\pgfsetroundjoin%
\pgfsetlinewidth{0.386077pt}%
\definecolor{currentstroke}{rgb}{0.280267,0.073417,0.397163}%
\pgfsetstrokecolor{currentstroke}%
\pgfsetdash{}{0pt}%
\pgfpathmoveto{\pgfqpoint{2.151035in}{5.146704in}}%
\pgfpathlineto{\pgfqpoint{2.151045in}{5.146704in}}%
\pgfusepath{stroke}%
\end{pgfscope}%
\begin{pgfscope}%
\pgfpathrectangle{\pgfqpoint{1.250000in}{4.155455in}}{\pgfqpoint{2.279412in}{2.004545in}}%
\pgfusepath{clip}%
\pgfsetbuttcap%
\pgfsetroundjoin%
\pgfsetlinewidth{0.386078pt}%
\definecolor{currentstroke}{rgb}{0.280267,0.073417,0.397163}%
\pgfsetstrokecolor{currentstroke}%
\pgfsetdash{}{0pt}%
\pgfpathmoveto{\pgfqpoint{2.151045in}{5.146704in}}%
\pgfpathlineto{\pgfqpoint{2.151068in}{5.146702in}}%
\pgfusepath{stroke}%
\end{pgfscope}%
\begin{pgfscope}%
\pgfpathrectangle{\pgfqpoint{1.250000in}{4.155455in}}{\pgfqpoint{2.279412in}{2.004545in}}%
\pgfusepath{clip}%
\pgfsetbuttcap%
\pgfsetroundjoin%
\pgfsetlinewidth{0.386081pt}%
\definecolor{currentstroke}{rgb}{0.280267,0.073417,0.397163}%
\pgfsetstrokecolor{currentstroke}%
\pgfsetdash{}{0pt}%
\pgfpathmoveto{\pgfqpoint{2.151068in}{5.146702in}}%
\pgfpathlineto{\pgfqpoint{2.151079in}{5.146702in}}%
\pgfusepath{stroke}%
\end{pgfscope}%
\begin{pgfscope}%
\pgfpathrectangle{\pgfqpoint{1.250000in}{4.155455in}}{\pgfqpoint{2.279412in}{2.004545in}}%
\pgfusepath{clip}%
\pgfsetbuttcap%
\pgfsetroundjoin%
\pgfsetlinewidth{0.386083pt}%
\definecolor{currentstroke}{rgb}{0.280267,0.073417,0.397163}%
\pgfsetstrokecolor{currentstroke}%
\pgfsetdash{}{0pt}%
\pgfpathmoveto{\pgfqpoint{2.151079in}{5.146702in}}%
\pgfpathlineto{\pgfqpoint{2.151068in}{5.146702in}}%
\pgfusepath{stroke}%
\end{pgfscope}%
\begin{pgfscope}%
\pgfpathrectangle{\pgfqpoint{1.250000in}{4.155455in}}{\pgfqpoint{2.279412in}{2.004545in}}%
\pgfusepath{clip}%
\pgfsetbuttcap%
\pgfsetroundjoin%
\pgfsetlinewidth{0.386081pt}%
\definecolor{currentstroke}{rgb}{0.280267,0.073417,0.397163}%
\pgfsetstrokecolor{currentstroke}%
\pgfsetdash{}{0pt}%
\pgfpathmoveto{\pgfqpoint{2.151068in}{5.146702in}}%
\pgfpathlineto{\pgfqpoint{2.151049in}{5.146703in}}%
\pgfusepath{stroke}%
\end{pgfscope}%
\begin{pgfscope}%
\pgfpathrectangle{\pgfqpoint{1.250000in}{4.155455in}}{\pgfqpoint{2.279412in}{2.004545in}}%
\pgfusepath{clip}%
\pgfsetbuttcap%
\pgfsetroundjoin%
\pgfsetlinewidth{0.386079pt}%
\definecolor{currentstroke}{rgb}{0.280267,0.073417,0.397163}%
\pgfsetstrokecolor{currentstroke}%
\pgfsetdash{}{0pt}%
\pgfpathmoveto{\pgfqpoint{2.151049in}{5.146703in}}%
\pgfpathlineto{\pgfqpoint{2.151043in}{5.146704in}}%
\pgfusepath{stroke}%
\end{pgfscope}%
\begin{pgfscope}%
\pgfpathrectangle{\pgfqpoint{1.250000in}{4.155455in}}{\pgfqpoint{2.279412in}{2.004545in}}%
\pgfusepath{clip}%
\pgfsetbuttcap%
\pgfsetroundjoin%
\pgfsetlinewidth{0.386078pt}%
\definecolor{currentstroke}{rgb}{0.280267,0.073417,0.397163}%
\pgfsetstrokecolor{currentstroke}%
\pgfsetdash{}{0pt}%
\pgfpathmoveto{\pgfqpoint{2.151043in}{5.146704in}}%
\pgfpathlineto{\pgfqpoint{2.151053in}{5.146703in}}%
\pgfusepath{stroke}%
\end{pgfscope}%
\begin{pgfscope}%
\pgfpathrectangle{\pgfqpoint{1.250000in}{4.155455in}}{\pgfqpoint{2.279412in}{2.004545in}}%
\pgfusepath{clip}%
\pgfsetbuttcap%
\pgfsetroundjoin%
\pgfsetlinewidth{0.386080pt}%
\definecolor{currentstroke}{rgb}{0.280267,0.073417,0.397163}%
\pgfsetstrokecolor{currentstroke}%
\pgfsetdash{}{0pt}%
\pgfpathmoveto{\pgfqpoint{2.151053in}{5.146703in}}%
\pgfpathlineto{\pgfqpoint{2.151068in}{5.146702in}}%
\pgfusepath{stroke}%
\end{pgfscope}%
\begin{pgfscope}%
\pgfpathrectangle{\pgfqpoint{1.250000in}{4.155455in}}{\pgfqpoint{2.279412in}{2.004545in}}%
\pgfusepath{clip}%
\pgfsetbuttcap%
\pgfsetroundjoin%
\pgfsetlinewidth{0.386081pt}%
\definecolor{currentstroke}{rgb}{0.280267,0.073417,0.397163}%
\pgfsetstrokecolor{currentstroke}%
\pgfsetdash{}{0pt}%
\pgfpathmoveto{\pgfqpoint{2.151068in}{5.146702in}}%
\pgfpathlineto{\pgfqpoint{2.151072in}{5.146702in}}%
\pgfusepath{stroke}%
\end{pgfscope}%
\begin{pgfscope}%
\pgfpathrectangle{\pgfqpoint{1.250000in}{4.155455in}}{\pgfqpoint{2.279412in}{2.004545in}}%
\pgfusepath{clip}%
\pgfsetbuttcap%
\pgfsetroundjoin%
\pgfsetlinewidth{0.386082pt}%
\definecolor{currentstroke}{rgb}{0.280267,0.073417,0.397163}%
\pgfsetstrokecolor{currentstroke}%
\pgfsetdash{}{0pt}%
\pgfpathmoveto{\pgfqpoint{2.151072in}{5.146702in}}%
\pgfpathlineto{\pgfqpoint{2.151061in}{5.146703in}}%
\pgfusepath{stroke}%
\end{pgfscope}%
\begin{pgfscope}%
\pgfpathrectangle{\pgfqpoint{1.250000in}{4.155455in}}{\pgfqpoint{2.279412in}{2.004545in}}%
\pgfusepath{clip}%
\pgfsetbuttcap%
\pgfsetroundjoin%
\pgfsetlinewidth{0.386081pt}%
\definecolor{currentstroke}{rgb}{0.280267,0.073417,0.397163}%
\pgfsetstrokecolor{currentstroke}%
\pgfsetdash{}{0pt}%
\pgfpathmoveto{\pgfqpoint{2.151061in}{5.146703in}}%
\pgfpathlineto{\pgfqpoint{2.151050in}{5.146703in}}%
\pgfusepath{stroke}%
\end{pgfscope}%
\begin{pgfscope}%
\pgfpathrectangle{\pgfqpoint{1.250000in}{4.155455in}}{\pgfqpoint{2.279412in}{2.004545in}}%
\pgfusepath{clip}%
\pgfsetbuttcap%
\pgfsetroundjoin%
\pgfsetlinewidth{0.386079pt}%
\definecolor{currentstroke}{rgb}{0.280267,0.073417,0.397163}%
\pgfsetstrokecolor{currentstroke}%
\pgfsetdash{}{0pt}%
\pgfpathmoveto{\pgfqpoint{2.151050in}{5.146703in}}%
\pgfpathlineto{\pgfqpoint{2.151049in}{5.146703in}}%
\pgfusepath{stroke}%
\end{pgfscope}%
\begin{pgfscope}%
\pgfpathrectangle{\pgfqpoint{1.250000in}{4.155455in}}{\pgfqpoint{2.279412in}{2.004545in}}%
\pgfusepath{clip}%
\pgfsetbuttcap%
\pgfsetroundjoin%
\pgfsetlinewidth{0.386079pt}%
\definecolor{currentstroke}{rgb}{0.280267,0.073417,0.397163}%
\pgfsetstrokecolor{currentstroke}%
\pgfsetdash{}{0pt}%
\pgfpathmoveto{\pgfqpoint{2.151049in}{5.146703in}}%
\pgfpathlineto{\pgfqpoint{2.151058in}{5.146703in}}%
\pgfusepath{stroke}%
\end{pgfscope}%
\begin{pgfscope}%
\pgfpathrectangle{\pgfqpoint{1.250000in}{4.155455in}}{\pgfqpoint{2.279412in}{2.004545in}}%
\pgfusepath{clip}%
\pgfsetbuttcap%
\pgfsetroundjoin%
\pgfsetlinewidth{0.386080pt}%
\definecolor{currentstroke}{rgb}{0.280267,0.073417,0.397163}%
\pgfsetstrokecolor{currentstroke}%
\pgfsetdash{}{0pt}%
\pgfpathmoveto{\pgfqpoint{2.151058in}{5.146703in}}%
\pgfpathlineto{\pgfqpoint{2.151067in}{5.146702in}}%
\pgfusepath{stroke}%
\end{pgfscope}%
\begin{pgfscope}%
\pgfpathrectangle{\pgfqpoint{1.250000in}{4.155455in}}{\pgfqpoint{2.279412in}{2.004545in}}%
\pgfusepath{clip}%
\pgfsetbuttcap%
\pgfsetroundjoin%
\pgfsetlinewidth{0.386081pt}%
\definecolor{currentstroke}{rgb}{0.280267,0.073417,0.397163}%
\pgfsetstrokecolor{currentstroke}%
\pgfsetdash{}{0pt}%
\pgfpathmoveto{\pgfqpoint{2.151067in}{5.146702in}}%
\pgfpathlineto{\pgfqpoint{2.151067in}{5.146702in}}%
\pgfusepath{stroke}%
\end{pgfscope}%
\begin{pgfscope}%
\pgfpathrectangle{\pgfqpoint{1.250000in}{4.155455in}}{\pgfqpoint{2.279412in}{2.004545in}}%
\pgfusepath{clip}%
\pgfsetbuttcap%
\pgfsetroundjoin%
\pgfsetlinewidth{0.386081pt}%
\definecolor{currentstroke}{rgb}{0.280267,0.073417,0.397163}%
\pgfsetstrokecolor{currentstroke}%
\pgfsetdash{}{0pt}%
\pgfpathmoveto{\pgfqpoint{2.151067in}{5.146702in}}%
\pgfpathlineto{\pgfqpoint{2.151058in}{5.146703in}}%
\pgfusepath{stroke}%
\end{pgfscope}%
\begin{pgfscope}%
\pgfpathrectangle{\pgfqpoint{1.250000in}{4.155455in}}{\pgfqpoint{2.279412in}{2.004545in}}%
\pgfusepath{clip}%
\pgfsetbuttcap%
\pgfsetroundjoin%
\pgfsetlinewidth{0.386080pt}%
\definecolor{currentstroke}{rgb}{0.280267,0.073417,0.397163}%
\pgfsetstrokecolor{currentstroke}%
\pgfsetdash{}{0pt}%
\pgfpathmoveto{\pgfqpoint{2.151058in}{5.146703in}}%
\pgfpathlineto{\pgfqpoint{2.151051in}{5.146703in}}%
\pgfusepath{stroke}%
\end{pgfscope}%
\begin{pgfscope}%
\pgfpathrectangle{\pgfqpoint{1.250000in}{4.155455in}}{\pgfqpoint{2.279412in}{2.004545in}}%
\pgfusepath{clip}%
\pgfsetbuttcap%
\pgfsetroundjoin%
\pgfsetlinewidth{0.386079pt}%
\definecolor{currentstroke}{rgb}{0.280267,0.073417,0.397163}%
\pgfsetstrokecolor{currentstroke}%
\pgfsetdash{}{0pt}%
\pgfpathmoveto{\pgfqpoint{2.151051in}{5.146703in}}%
\pgfpathlineto{\pgfqpoint{2.151053in}{5.146703in}}%
\pgfusepath{stroke}%
\end{pgfscope}%
\begin{pgfscope}%
\pgfpathrectangle{\pgfqpoint{1.250000in}{4.155455in}}{\pgfqpoint{2.279412in}{2.004545in}}%
\pgfusepath{clip}%
\pgfsetbuttcap%
\pgfsetroundjoin%
\pgfsetlinewidth{0.386079pt}%
\definecolor{currentstroke}{rgb}{0.280267,0.073417,0.397163}%
\pgfsetstrokecolor{currentstroke}%
\pgfsetdash{}{0pt}%
\pgfpathmoveto{\pgfqpoint{2.151053in}{5.146703in}}%
\pgfpathlineto{\pgfqpoint{2.151060in}{5.146703in}}%
\pgfusepath{stroke}%
\end{pgfscope}%
\begin{pgfscope}%
\pgfpathrectangle{\pgfqpoint{1.250000in}{4.155455in}}{\pgfqpoint{2.279412in}{2.004545in}}%
\pgfusepath{clip}%
\pgfsetbuttcap%
\pgfsetroundjoin%
\pgfsetlinewidth{0.386080pt}%
\definecolor{currentstroke}{rgb}{0.280267,0.073417,0.397163}%
\pgfsetstrokecolor{currentstroke}%
\pgfsetdash{}{0pt}%
\pgfpathmoveto{\pgfqpoint{2.151060in}{5.146703in}}%
\pgfpathlineto{\pgfqpoint{2.151065in}{5.146702in}}%
\pgfusepath{stroke}%
\end{pgfscope}%
\begin{pgfscope}%
\pgfpathrectangle{\pgfqpoint{1.250000in}{4.155455in}}{\pgfqpoint{2.279412in}{2.004545in}}%
\pgfusepath{clip}%
\pgfsetbuttcap%
\pgfsetroundjoin%
\pgfsetlinewidth{0.386081pt}%
\definecolor{currentstroke}{rgb}{0.280267,0.073417,0.397163}%
\pgfsetstrokecolor{currentstroke}%
\pgfsetdash{}{0pt}%
\pgfpathmoveto{\pgfqpoint{2.151065in}{5.146702in}}%
\pgfpathlineto{\pgfqpoint{2.151063in}{5.146703in}}%
\pgfusepath{stroke}%
\end{pgfscope}%
\begin{pgfscope}%
\pgfpathrectangle{\pgfqpoint{1.250000in}{4.155455in}}{\pgfqpoint{2.279412in}{2.004545in}}%
\pgfusepath{clip}%
\pgfsetbuttcap%
\pgfsetroundjoin%
\pgfsetlinewidth{0.386081pt}%
\definecolor{currentstroke}{rgb}{0.280267,0.073417,0.397163}%
\pgfsetstrokecolor{currentstroke}%
\pgfsetdash{}{0pt}%
\pgfpathmoveto{\pgfqpoint{2.151063in}{5.146703in}}%
\pgfpathlineto{\pgfqpoint{2.151057in}{5.146703in}}%
\pgfusepath{stroke}%
\end{pgfscope}%
\begin{pgfscope}%
\pgfpathrectangle{\pgfqpoint{1.250000in}{4.155455in}}{\pgfqpoint{2.279412in}{2.004545in}}%
\pgfusepath{clip}%
\pgfsetbuttcap%
\pgfsetroundjoin%
\pgfsetlinewidth{0.386080pt}%
\definecolor{currentstroke}{rgb}{0.280267,0.073417,0.397163}%
\pgfsetstrokecolor{currentstroke}%
\pgfsetdash{}{0pt}%
\pgfpathmoveto{\pgfqpoint{2.151057in}{5.146703in}}%
\pgfpathlineto{\pgfqpoint{2.151053in}{5.146703in}}%
\pgfusepath{stroke}%
\end{pgfscope}%
\begin{pgfscope}%
\pgfpathrectangle{\pgfqpoint{1.250000in}{4.155455in}}{\pgfqpoint{2.279412in}{2.004545in}}%
\pgfusepath{clip}%
\pgfsetbuttcap%
\pgfsetroundjoin%
\pgfsetlinewidth{0.386080pt}%
\definecolor{currentstroke}{rgb}{0.280267,0.073417,0.397163}%
\pgfsetstrokecolor{currentstroke}%
\pgfsetdash{}{0pt}%
\pgfpathmoveto{\pgfqpoint{2.151053in}{5.146703in}}%
\pgfpathlineto{\pgfqpoint{2.151056in}{5.146703in}}%
\pgfusepath{stroke}%
\end{pgfscope}%
\begin{pgfscope}%
\pgfpathrectangle{\pgfqpoint{1.250000in}{4.155455in}}{\pgfqpoint{2.279412in}{2.004545in}}%
\pgfusepath{clip}%
\pgfsetbuttcap%
\pgfsetroundjoin%
\pgfsetlinewidth{0.386080pt}%
\definecolor{currentstroke}{rgb}{0.280267,0.073417,0.397163}%
\pgfsetstrokecolor{currentstroke}%
\pgfsetdash{}{0pt}%
\pgfpathmoveto{\pgfqpoint{2.151056in}{5.146703in}}%
\pgfpathlineto{\pgfqpoint{2.151061in}{5.146703in}}%
\pgfusepath{stroke}%
\end{pgfscope}%
\begin{pgfscope}%
\pgfpathrectangle{\pgfqpoint{1.250000in}{4.155455in}}{\pgfqpoint{2.279412in}{2.004545in}}%
\pgfusepath{clip}%
\pgfsetbuttcap%
\pgfsetroundjoin%
\pgfsetlinewidth{0.386080pt}%
\definecolor{currentstroke}{rgb}{0.280267,0.073417,0.397163}%
\pgfsetstrokecolor{currentstroke}%
\pgfsetdash{}{0pt}%
\pgfpathmoveto{\pgfqpoint{2.151061in}{5.146703in}}%
\pgfpathlineto{\pgfqpoint{2.151063in}{5.146703in}}%
\pgfusepath{stroke}%
\end{pgfscope}%
\begin{pgfscope}%
\pgfpathrectangle{\pgfqpoint{1.250000in}{4.155455in}}{\pgfqpoint{2.279412in}{2.004545in}}%
\pgfusepath{clip}%
\pgfsetbuttcap%
\pgfsetroundjoin%
\pgfsetlinewidth{0.386081pt}%
\definecolor{currentstroke}{rgb}{0.280267,0.073417,0.397163}%
\pgfsetstrokecolor{currentstroke}%
\pgfsetdash{}{0pt}%
\pgfpathmoveto{\pgfqpoint{2.151063in}{5.146703in}}%
\pgfpathlineto{\pgfqpoint{2.151061in}{5.146703in}}%
\pgfusepath{stroke}%
\end{pgfscope}%
\begin{pgfscope}%
\pgfpathrectangle{\pgfqpoint{1.250000in}{4.155455in}}{\pgfqpoint{2.279412in}{2.004545in}}%
\pgfusepath{clip}%
\pgfsetbuttcap%
\pgfsetroundjoin%
\pgfsetlinewidth{0.386080pt}%
\definecolor{currentstroke}{rgb}{0.280267,0.073417,0.397163}%
\pgfsetstrokecolor{currentstroke}%
\pgfsetdash{}{0pt}%
\pgfpathmoveto{\pgfqpoint{2.151061in}{5.146703in}}%
\pgfpathlineto{\pgfqpoint{2.151056in}{5.146703in}}%
\pgfusepath{stroke}%
\end{pgfscope}%
\begin{pgfscope}%
\pgfpathrectangle{\pgfqpoint{1.250000in}{4.155455in}}{\pgfqpoint{2.279412in}{2.004545in}}%
\pgfusepath{clip}%
\pgfsetbuttcap%
\pgfsetroundjoin%
\pgfsetlinewidth{0.386080pt}%
\definecolor{currentstroke}{rgb}{0.280267,0.073417,0.397163}%
\pgfsetstrokecolor{currentstroke}%
\pgfsetdash{}{0pt}%
\pgfpathmoveto{\pgfqpoint{2.151056in}{5.146703in}}%
\pgfpathlineto{\pgfqpoint{2.151055in}{5.146703in}}%
\pgfusepath{stroke}%
\end{pgfscope}%
\begin{pgfscope}%
\pgfpathrectangle{\pgfqpoint{1.250000in}{4.155455in}}{\pgfqpoint{2.279412in}{2.004545in}}%
\pgfusepath{clip}%
\pgfsetbuttcap%
\pgfsetroundjoin%
\pgfsetlinewidth{0.386080pt}%
\definecolor{currentstroke}{rgb}{0.280267,0.073417,0.397163}%
\pgfsetstrokecolor{currentstroke}%
\pgfsetdash{}{0pt}%
\pgfpathmoveto{\pgfqpoint{2.151055in}{5.146703in}}%
\pgfpathlineto{\pgfqpoint{2.151058in}{5.146703in}}%
\pgfusepath{stroke}%
\end{pgfscope}%
\begin{pgfscope}%
\pgfpathrectangle{\pgfqpoint{1.250000in}{4.155455in}}{\pgfqpoint{2.279412in}{2.004545in}}%
\pgfusepath{clip}%
\pgfsetbuttcap%
\pgfsetroundjoin%
\pgfsetlinewidth{0.386080pt}%
\definecolor{currentstroke}{rgb}{0.280267,0.073417,0.397163}%
\pgfsetstrokecolor{currentstroke}%
\pgfsetdash{}{0pt}%
\pgfpathmoveto{\pgfqpoint{2.151058in}{5.146703in}}%
\pgfpathlineto{\pgfqpoint{2.151061in}{5.146703in}}%
\pgfusepath{stroke}%
\end{pgfscope}%
\begin{pgfscope}%
\pgfpathrectangle{\pgfqpoint{1.250000in}{4.155455in}}{\pgfqpoint{2.279412in}{2.004545in}}%
\pgfusepath{clip}%
\pgfsetbuttcap%
\pgfsetroundjoin%
\pgfsetlinewidth{0.386080pt}%
\definecolor{currentstroke}{rgb}{0.280267,0.073417,0.397163}%
\pgfsetstrokecolor{currentstroke}%
\pgfsetdash{}{0pt}%
\pgfpathmoveto{\pgfqpoint{2.151061in}{5.146703in}}%
\pgfpathlineto{\pgfqpoint{2.151062in}{5.146703in}}%
\pgfusepath{stroke}%
\end{pgfscope}%
\begin{pgfscope}%
\pgfpathrectangle{\pgfqpoint{1.250000in}{4.155455in}}{\pgfqpoint{2.279412in}{2.004545in}}%
\pgfusepath{clip}%
\pgfsetbuttcap%
\pgfsetroundjoin%
\pgfsetlinewidth{0.386081pt}%
\definecolor{currentstroke}{rgb}{0.280267,0.073417,0.397163}%
\pgfsetstrokecolor{currentstroke}%
\pgfsetdash{}{0pt}%
\pgfpathmoveto{\pgfqpoint{2.151062in}{5.146703in}}%
\pgfpathlineto{\pgfqpoint{2.151059in}{5.146703in}}%
\pgfusepath{stroke}%
\end{pgfscope}%
\begin{pgfscope}%
\pgfpathrectangle{\pgfqpoint{1.250000in}{4.155455in}}{\pgfqpoint{2.279412in}{2.004545in}}%
\pgfusepath{clip}%
\pgfsetbuttcap%
\pgfsetroundjoin%
\pgfsetlinewidth{0.386080pt}%
\definecolor{currentstroke}{rgb}{0.280267,0.073417,0.397163}%
\pgfsetstrokecolor{currentstroke}%
\pgfsetdash{}{0pt}%
\pgfpathmoveto{\pgfqpoint{2.151059in}{5.146703in}}%
\pgfpathlineto{\pgfqpoint{2.151057in}{5.146703in}}%
\pgfusepath{stroke}%
\end{pgfscope}%
\begin{pgfscope}%
\pgfpathrectangle{\pgfqpoint{1.250000in}{4.155455in}}{\pgfqpoint{2.279412in}{2.004545in}}%
\pgfusepath{clip}%
\pgfsetbuttcap%
\pgfsetroundjoin%
\pgfsetlinewidth{0.386080pt}%
\definecolor{currentstroke}{rgb}{0.280267,0.073417,0.397163}%
\pgfsetstrokecolor{currentstroke}%
\pgfsetdash{}{0pt}%
\pgfpathmoveto{\pgfqpoint{2.151057in}{5.146703in}}%
\pgfpathlineto{\pgfqpoint{2.151057in}{5.146703in}}%
\pgfusepath{stroke}%
\end{pgfscope}%
\begin{pgfscope}%
\pgfpathrectangle{\pgfqpoint{1.250000in}{4.155455in}}{\pgfqpoint{2.279412in}{2.004545in}}%
\pgfusepath{clip}%
\pgfsetbuttcap%
\pgfsetroundjoin%
\pgfsetlinewidth{0.386080pt}%
\definecolor{currentstroke}{rgb}{0.280267,0.073417,0.397163}%
\pgfsetstrokecolor{currentstroke}%
\pgfsetdash{}{0pt}%
\pgfpathmoveto{\pgfqpoint{2.151057in}{5.146703in}}%
\pgfpathlineto{\pgfqpoint{2.151059in}{5.146703in}}%
\pgfusepath{stroke}%
\end{pgfscope}%
\begin{pgfscope}%
\pgfpathrectangle{\pgfqpoint{1.250000in}{4.155455in}}{\pgfqpoint{2.279412in}{2.004545in}}%
\pgfusepath{clip}%
\pgfsetbuttcap%
\pgfsetroundjoin%
\pgfsetlinewidth{0.386080pt}%
\definecolor{currentstroke}{rgb}{0.280267,0.073417,0.397163}%
\pgfsetstrokecolor{currentstroke}%
\pgfsetdash{}{0pt}%
\pgfpathmoveto{\pgfqpoint{2.151059in}{5.146703in}}%
\pgfpathlineto{\pgfqpoint{2.151061in}{5.146703in}}%
\pgfusepath{stroke}%
\end{pgfscope}%
\begin{pgfscope}%
\pgfpathrectangle{\pgfqpoint{1.250000in}{4.155455in}}{\pgfqpoint{2.279412in}{2.004545in}}%
\pgfusepath{clip}%
\pgfsetbuttcap%
\pgfsetroundjoin%
\pgfsetlinewidth{0.386080pt}%
\definecolor{currentstroke}{rgb}{0.280267,0.073417,0.397163}%
\pgfsetstrokecolor{currentstroke}%
\pgfsetdash{}{0pt}%
\pgfpathmoveto{\pgfqpoint{2.151061in}{5.146703in}}%
\pgfpathlineto{\pgfqpoint{2.151061in}{5.146703in}}%
\pgfusepath{stroke}%
\end{pgfscope}%
\begin{pgfscope}%
\pgfpathrectangle{\pgfqpoint{1.250000in}{4.155455in}}{\pgfqpoint{2.279412in}{2.004545in}}%
\pgfusepath{clip}%
\pgfsetbuttcap%
\pgfsetroundjoin%
\pgfsetlinewidth{0.386080pt}%
\definecolor{currentstroke}{rgb}{0.280267,0.073417,0.397163}%
\pgfsetstrokecolor{currentstroke}%
\pgfsetdash{}{0pt}%
\pgfpathmoveto{\pgfqpoint{2.151061in}{5.146703in}}%
\pgfpathlineto{\pgfqpoint{2.151059in}{5.146703in}}%
\pgfusepath{stroke}%
\end{pgfscope}%
\begin{pgfscope}%
\pgfpathrectangle{\pgfqpoint{1.250000in}{4.155455in}}{\pgfqpoint{2.279412in}{2.004545in}}%
\pgfusepath{clip}%
\pgfsetbuttcap%
\pgfsetroundjoin%
\pgfsetlinewidth{0.386080pt}%
\definecolor{currentstroke}{rgb}{0.280267,0.073417,0.397163}%
\pgfsetstrokecolor{currentstroke}%
\pgfsetdash{}{0pt}%
\pgfpathmoveto{\pgfqpoint{2.151059in}{5.146703in}}%
\pgfpathlineto{\pgfqpoint{2.151057in}{5.146703in}}%
\pgfusepath{stroke}%
\end{pgfscope}%
\begin{pgfscope}%
\pgfpathrectangle{\pgfqpoint{1.250000in}{4.155455in}}{\pgfqpoint{2.279412in}{2.004545in}}%
\pgfusepath{clip}%
\pgfsetbuttcap%
\pgfsetroundjoin%
\pgfsetlinewidth{0.386080pt}%
\definecolor{currentstroke}{rgb}{0.280267,0.073417,0.397163}%
\pgfsetstrokecolor{currentstroke}%
\pgfsetdash{}{0pt}%
\pgfpathmoveto{\pgfqpoint{2.151057in}{5.146703in}}%
\pgfpathlineto{\pgfqpoint{2.151058in}{5.146703in}}%
\pgfusepath{stroke}%
\end{pgfscope}%
\begin{pgfscope}%
\pgfpathrectangle{\pgfqpoint{1.250000in}{4.155455in}}{\pgfqpoint{2.279412in}{2.004545in}}%
\pgfusepath{clip}%
\pgfsetbuttcap%
\pgfsetroundjoin%
\pgfsetlinewidth{0.386080pt}%
\definecolor{currentstroke}{rgb}{0.280267,0.073417,0.397163}%
\pgfsetstrokecolor{currentstroke}%
\pgfsetdash{}{0pt}%
\pgfpathmoveto{\pgfqpoint{2.151058in}{5.146703in}}%
\pgfpathlineto{\pgfqpoint{2.151059in}{5.146703in}}%
\pgfusepath{stroke}%
\end{pgfscope}%
\begin{pgfscope}%
\pgfpathrectangle{\pgfqpoint{1.250000in}{4.155455in}}{\pgfqpoint{2.279412in}{2.004545in}}%
\pgfusepath{clip}%
\pgfsetbuttcap%
\pgfsetroundjoin%
\pgfsetlinewidth{0.386080pt}%
\definecolor{currentstroke}{rgb}{0.280267,0.073417,0.397163}%
\pgfsetstrokecolor{currentstroke}%
\pgfsetdash{}{0pt}%
\pgfpathmoveto{\pgfqpoint{2.151059in}{5.146703in}}%
\pgfpathlineto{\pgfqpoint{2.151060in}{5.146703in}}%
\pgfusepath{stroke}%
\end{pgfscope}%
\begin{pgfscope}%
\pgfpathrectangle{\pgfqpoint{1.250000in}{4.155455in}}{\pgfqpoint{2.279412in}{2.004545in}}%
\pgfusepath{clip}%
\pgfsetbuttcap%
\pgfsetroundjoin%
\pgfsetlinewidth{0.386080pt}%
\definecolor{currentstroke}{rgb}{0.280267,0.073417,0.397163}%
\pgfsetstrokecolor{currentstroke}%
\pgfsetdash{}{0pt}%
\pgfpathmoveto{\pgfqpoint{2.151060in}{5.146703in}}%
\pgfpathlineto{\pgfqpoint{2.151060in}{5.146703in}}%
\pgfusepath{stroke}%
\end{pgfscope}%
\begin{pgfscope}%
\pgfpathrectangle{\pgfqpoint{1.250000in}{4.155455in}}{\pgfqpoint{2.279412in}{2.004545in}}%
\pgfusepath{clip}%
\pgfsetbuttcap%
\pgfsetroundjoin%
\pgfsetlinewidth{0.386080pt}%
\definecolor{currentstroke}{rgb}{0.280267,0.073417,0.397163}%
\pgfsetstrokecolor{currentstroke}%
\pgfsetdash{}{0pt}%
\pgfpathmoveto{\pgfqpoint{2.151060in}{5.146703in}}%
\pgfpathlineto{\pgfqpoint{2.151058in}{5.146703in}}%
\pgfusepath{stroke}%
\end{pgfscope}%
\begin{pgfscope}%
\pgfpathrectangle{\pgfqpoint{1.250000in}{4.155455in}}{\pgfqpoint{2.279412in}{2.004545in}}%
\pgfusepath{clip}%
\pgfsetbuttcap%
\pgfsetroundjoin%
\pgfsetlinewidth{0.386080pt}%
\definecolor{currentstroke}{rgb}{0.280267,0.073417,0.397163}%
\pgfsetstrokecolor{currentstroke}%
\pgfsetdash{}{0pt}%
\pgfpathmoveto{\pgfqpoint{2.151058in}{5.146703in}}%
\pgfpathlineto{\pgfqpoint{2.151058in}{5.146703in}}%
\pgfusepath{stroke}%
\end{pgfscope}%
\begin{pgfscope}%
\pgfpathrectangle{\pgfqpoint{1.250000in}{4.155455in}}{\pgfqpoint{2.279412in}{2.004545in}}%
\pgfusepath{clip}%
\pgfsetbuttcap%
\pgfsetroundjoin%
\pgfsetlinewidth{0.386080pt}%
\definecolor{currentstroke}{rgb}{0.280267,0.073417,0.397163}%
\pgfsetstrokecolor{currentstroke}%
\pgfsetdash{}{0pt}%
\pgfpathmoveto{\pgfqpoint{2.151058in}{5.146703in}}%
\pgfpathlineto{\pgfqpoint{2.151058in}{5.146703in}}%
\pgfusepath{stroke}%
\end{pgfscope}%
\begin{pgfscope}%
\pgfpathrectangle{\pgfqpoint{1.250000in}{4.155455in}}{\pgfqpoint{2.279412in}{2.004545in}}%
\pgfusepath{clip}%
\pgfsetbuttcap%
\pgfsetroundjoin%
\pgfsetlinewidth{0.386080pt}%
\definecolor{currentstroke}{rgb}{0.280267,0.073417,0.397163}%
\pgfsetstrokecolor{currentstroke}%
\pgfsetdash{}{0pt}%
\pgfpathmoveto{\pgfqpoint{2.151058in}{5.146703in}}%
\pgfpathlineto{\pgfqpoint{2.151059in}{5.146703in}}%
\pgfusepath{stroke}%
\end{pgfscope}%
\begin{pgfscope}%
\pgfpathrectangle{\pgfqpoint{1.250000in}{4.155455in}}{\pgfqpoint{2.279412in}{2.004545in}}%
\pgfusepath{clip}%
\pgfsetbuttcap%
\pgfsetroundjoin%
\pgfsetlinewidth{0.386080pt}%
\definecolor{currentstroke}{rgb}{0.280267,0.073417,0.397163}%
\pgfsetstrokecolor{currentstroke}%
\pgfsetdash{}{0pt}%
\pgfpathmoveto{\pgfqpoint{2.151059in}{5.146703in}}%
\pgfpathlineto{\pgfqpoint{2.151060in}{5.146703in}}%
\pgfusepath{stroke}%
\end{pgfscope}%
\begin{pgfscope}%
\pgfpathrectangle{\pgfqpoint{1.250000in}{4.155455in}}{\pgfqpoint{2.279412in}{2.004545in}}%
\pgfusepath{clip}%
\pgfsetbuttcap%
\pgfsetroundjoin%
\pgfsetlinewidth{0.386080pt}%
\definecolor{currentstroke}{rgb}{0.280267,0.073417,0.397163}%
\pgfsetstrokecolor{currentstroke}%
\pgfsetdash{}{0pt}%
\pgfpathmoveto{\pgfqpoint{2.151060in}{5.146703in}}%
\pgfpathlineto{\pgfqpoint{2.151059in}{5.146703in}}%
\pgfusepath{stroke}%
\end{pgfscope}%
\begin{pgfscope}%
\pgfpathrectangle{\pgfqpoint{1.250000in}{4.155455in}}{\pgfqpoint{2.279412in}{2.004545in}}%
\pgfusepath{clip}%
\pgfsetbuttcap%
\pgfsetroundjoin%
\pgfsetlinewidth{0.386080pt}%
\definecolor{currentstroke}{rgb}{0.280267,0.073417,0.397163}%
\pgfsetstrokecolor{currentstroke}%
\pgfsetdash{}{0pt}%
\pgfpathmoveto{\pgfqpoint{2.151059in}{5.146703in}}%
\pgfpathlineto{\pgfqpoint{2.151058in}{5.146703in}}%
\pgfusepath{stroke}%
\end{pgfscope}%
\begin{pgfscope}%
\pgfpathrectangle{\pgfqpoint{1.250000in}{4.155455in}}{\pgfqpoint{2.279412in}{2.004545in}}%
\pgfusepath{clip}%
\pgfsetbuttcap%
\pgfsetroundjoin%
\pgfsetlinewidth{0.386080pt}%
\definecolor{currentstroke}{rgb}{0.280267,0.073417,0.397163}%
\pgfsetstrokecolor{currentstroke}%
\pgfsetdash{}{0pt}%
\pgfpathmoveto{\pgfqpoint{2.151058in}{5.146703in}}%
\pgfpathlineto{\pgfqpoint{2.151058in}{5.146703in}}%
\pgfusepath{stroke}%
\end{pgfscope}%
\begin{pgfscope}%
\pgfpathrectangle{\pgfqpoint{1.250000in}{4.155455in}}{\pgfqpoint{2.279412in}{2.004545in}}%
\pgfusepath{clip}%
\pgfsetbuttcap%
\pgfsetroundjoin%
\pgfsetlinewidth{0.386080pt}%
\definecolor{currentstroke}{rgb}{0.280267,0.073417,0.397163}%
\pgfsetstrokecolor{currentstroke}%
\pgfsetdash{}{0pt}%
\pgfpathmoveto{\pgfqpoint{2.151058in}{5.146703in}}%
\pgfpathlineto{\pgfqpoint{2.151059in}{5.146703in}}%
\pgfusepath{stroke}%
\end{pgfscope}%
\begin{pgfscope}%
\pgfpathrectangle{\pgfqpoint{1.250000in}{4.155455in}}{\pgfqpoint{2.279412in}{2.004545in}}%
\pgfusepath{clip}%
\pgfsetbuttcap%
\pgfsetroundjoin%
\pgfsetlinewidth{0.386080pt}%
\definecolor{currentstroke}{rgb}{0.280267,0.073417,0.397163}%
\pgfsetstrokecolor{currentstroke}%
\pgfsetdash{}{0pt}%
\pgfpathmoveto{\pgfqpoint{2.151059in}{5.146703in}}%
\pgfpathlineto{\pgfqpoint{2.151059in}{5.146703in}}%
\pgfusepath{stroke}%
\end{pgfscope}%
\begin{pgfscope}%
\pgfpathrectangle{\pgfqpoint{1.250000in}{4.155455in}}{\pgfqpoint{2.279412in}{2.004545in}}%
\pgfusepath{clip}%
\pgfsetbuttcap%
\pgfsetroundjoin%
\pgfsetlinewidth{0.386080pt}%
\definecolor{currentstroke}{rgb}{0.280267,0.073417,0.397163}%
\pgfsetstrokecolor{currentstroke}%
\pgfsetdash{}{0pt}%
\pgfpathmoveto{\pgfqpoint{2.151059in}{5.146703in}}%
\pgfpathlineto{\pgfqpoint{2.151059in}{5.146703in}}%
\pgfusepath{stroke}%
\end{pgfscope}%
\begin{pgfscope}%
\pgfpathrectangle{\pgfqpoint{1.250000in}{4.155455in}}{\pgfqpoint{2.279412in}{2.004545in}}%
\pgfusepath{clip}%
\pgfsetbuttcap%
\pgfsetroundjoin%
\pgfsetlinewidth{0.386080pt}%
\definecolor{currentstroke}{rgb}{0.280267,0.073417,0.397163}%
\pgfsetstrokecolor{currentstroke}%
\pgfsetdash{}{0pt}%
\pgfpathmoveto{\pgfqpoint{2.151059in}{5.146703in}}%
\pgfpathlineto{\pgfqpoint{2.151059in}{5.146703in}}%
\pgfusepath{stroke}%
\end{pgfscope}%
\begin{pgfscope}%
\pgfpathrectangle{\pgfqpoint{1.250000in}{4.155455in}}{\pgfqpoint{2.279412in}{2.004545in}}%
\pgfusepath{clip}%
\pgfsetbuttcap%
\pgfsetroundjoin%
\pgfsetlinewidth{0.386080pt}%
\definecolor{currentstroke}{rgb}{0.280267,0.073417,0.397163}%
\pgfsetstrokecolor{currentstroke}%
\pgfsetdash{}{0pt}%
\pgfpathmoveto{\pgfqpoint{2.151059in}{5.146703in}}%
\pgfpathlineto{\pgfqpoint{2.151058in}{5.146703in}}%
\pgfusepath{stroke}%
\end{pgfscope}%
\begin{pgfscope}%
\pgfpathrectangle{\pgfqpoint{1.250000in}{4.155455in}}{\pgfqpoint{2.279412in}{2.004545in}}%
\pgfusepath{clip}%
\pgfsetbuttcap%
\pgfsetroundjoin%
\pgfsetlinewidth{0.386080pt}%
\definecolor{currentstroke}{rgb}{0.280267,0.073417,0.397163}%
\pgfsetstrokecolor{currentstroke}%
\pgfsetdash{}{0pt}%
\pgfpathmoveto{\pgfqpoint{2.151058in}{5.146703in}}%
\pgfpathlineto{\pgfqpoint{2.151058in}{5.146703in}}%
\pgfusepath{stroke}%
\end{pgfscope}%
\begin{pgfscope}%
\pgfpathrectangle{\pgfqpoint{1.250000in}{4.155455in}}{\pgfqpoint{2.279412in}{2.004545in}}%
\pgfusepath{clip}%
\pgfsetbuttcap%
\pgfsetroundjoin%
\pgfsetlinewidth{0.386080pt}%
\definecolor{currentstroke}{rgb}{0.280267,0.073417,0.397163}%
\pgfsetstrokecolor{currentstroke}%
\pgfsetdash{}{0pt}%
\pgfpathmoveto{\pgfqpoint{2.151058in}{5.146703in}}%
\pgfpathlineto{\pgfqpoint{2.151059in}{5.146703in}}%
\pgfusepath{stroke}%
\end{pgfscope}%
\begin{pgfscope}%
\pgfpathrectangle{\pgfqpoint{1.250000in}{4.155455in}}{\pgfqpoint{2.279412in}{2.004545in}}%
\pgfusepath{clip}%
\pgfsetbuttcap%
\pgfsetroundjoin%
\pgfsetlinewidth{0.386080pt}%
\definecolor{currentstroke}{rgb}{0.280267,0.073417,0.397163}%
\pgfsetstrokecolor{currentstroke}%
\pgfsetdash{}{0pt}%
\pgfpathmoveto{\pgfqpoint{2.151059in}{5.146703in}}%
\pgfpathlineto{\pgfqpoint{2.151059in}{5.146703in}}%
\pgfusepath{stroke}%
\end{pgfscope}%
\begin{pgfscope}%
\pgfpathrectangle{\pgfqpoint{1.250000in}{4.155455in}}{\pgfqpoint{2.279412in}{2.004545in}}%
\pgfusepath{clip}%
\pgfsetbuttcap%
\pgfsetroundjoin%
\pgfsetlinewidth{0.386080pt}%
\definecolor{currentstroke}{rgb}{0.280267,0.073417,0.397163}%
\pgfsetstrokecolor{currentstroke}%
\pgfsetdash{}{0pt}%
\pgfpathmoveto{\pgfqpoint{2.151059in}{5.146703in}}%
\pgfpathlineto{\pgfqpoint{2.151059in}{5.146703in}}%
\pgfusepath{stroke}%
\end{pgfscope}%
\begin{pgfscope}%
\pgfpathrectangle{\pgfqpoint{1.250000in}{4.155455in}}{\pgfqpoint{2.279412in}{2.004545in}}%
\pgfusepath{clip}%
\pgfsetbuttcap%
\pgfsetroundjoin%
\pgfsetlinewidth{0.386080pt}%
\definecolor{currentstroke}{rgb}{0.280267,0.073417,0.397163}%
\pgfsetstrokecolor{currentstroke}%
\pgfsetdash{}{0pt}%
\pgfpathmoveto{\pgfqpoint{2.151059in}{5.146703in}}%
\pgfpathlineto{\pgfqpoint{2.151059in}{5.146703in}}%
\pgfusepath{stroke}%
\end{pgfscope}%
\begin{pgfscope}%
\pgfpathrectangle{\pgfqpoint{1.250000in}{4.155455in}}{\pgfqpoint{2.279412in}{2.004545in}}%
\pgfusepath{clip}%
\pgfsetbuttcap%
\pgfsetroundjoin%
\pgfsetlinewidth{0.386080pt}%
\definecolor{currentstroke}{rgb}{0.280267,0.073417,0.397163}%
\pgfsetstrokecolor{currentstroke}%
\pgfsetdash{}{0pt}%
\pgfpathmoveto{\pgfqpoint{2.151059in}{5.146703in}}%
\pgfpathlineto{\pgfqpoint{2.151058in}{5.146703in}}%
\pgfusepath{stroke}%
\end{pgfscope}%
\begin{pgfscope}%
\pgfpathrectangle{\pgfqpoint{1.250000in}{4.155455in}}{\pgfqpoint{2.279412in}{2.004545in}}%
\pgfusepath{clip}%
\pgfsetbuttcap%
\pgfsetroundjoin%
\pgfsetlinewidth{0.386080pt}%
\definecolor{currentstroke}{rgb}{0.280267,0.073417,0.397163}%
\pgfsetstrokecolor{currentstroke}%
\pgfsetdash{}{0pt}%
\pgfpathmoveto{\pgfqpoint{2.151058in}{5.146703in}}%
\pgfpathlineto{\pgfqpoint{2.151059in}{5.146703in}}%
\pgfusepath{stroke}%
\end{pgfscope}%
\begin{pgfscope}%
\pgfpathrectangle{\pgfqpoint{1.250000in}{4.155455in}}{\pgfqpoint{2.279412in}{2.004545in}}%
\pgfusepath{clip}%
\pgfsetbuttcap%
\pgfsetroundjoin%
\pgfsetlinewidth{0.386080pt}%
\definecolor{currentstroke}{rgb}{0.280267,0.073417,0.397163}%
\pgfsetstrokecolor{currentstroke}%
\pgfsetdash{}{0pt}%
\pgfpathmoveto{\pgfqpoint{2.151059in}{5.146703in}}%
\pgfpathlineto{\pgfqpoint{2.151059in}{5.146703in}}%
\pgfusepath{stroke}%
\end{pgfscope}%
\begin{pgfscope}%
\pgfpathrectangle{\pgfqpoint{1.250000in}{4.155455in}}{\pgfqpoint{2.279412in}{2.004545in}}%
\pgfusepath{clip}%
\pgfsetbuttcap%
\pgfsetroundjoin%
\pgfsetlinewidth{0.386080pt}%
\definecolor{currentstroke}{rgb}{0.280267,0.073417,0.397163}%
\pgfsetstrokecolor{currentstroke}%
\pgfsetdash{}{0pt}%
\pgfpathmoveto{\pgfqpoint{2.151059in}{5.146703in}}%
\pgfpathlineto{\pgfqpoint{2.151059in}{5.146703in}}%
\pgfusepath{stroke}%
\end{pgfscope}%
\begin{pgfscope}%
\pgfpathrectangle{\pgfqpoint{1.250000in}{4.155455in}}{\pgfqpoint{2.279412in}{2.004545in}}%
\pgfusepath{clip}%
\pgfsetbuttcap%
\pgfsetroundjoin%
\pgfsetlinewidth{0.386080pt}%
\definecolor{currentstroke}{rgb}{0.280267,0.073417,0.397163}%
\pgfsetstrokecolor{currentstroke}%
\pgfsetdash{}{0pt}%
\pgfpathmoveto{\pgfqpoint{2.151059in}{5.146703in}}%
\pgfpathlineto{\pgfqpoint{2.151059in}{5.146703in}}%
\pgfusepath{stroke}%
\end{pgfscope}%
\begin{pgfscope}%
\pgfpathrectangle{\pgfqpoint{1.250000in}{4.155455in}}{\pgfqpoint{2.279412in}{2.004545in}}%
\pgfusepath{clip}%
\pgfsetbuttcap%
\pgfsetroundjoin%
\pgfsetlinewidth{0.386080pt}%
\definecolor{currentstroke}{rgb}{0.280267,0.073417,0.397163}%
\pgfsetstrokecolor{currentstroke}%
\pgfsetdash{}{0pt}%
\pgfpathmoveto{\pgfqpoint{2.151059in}{5.146703in}}%
\pgfpathlineto{\pgfqpoint{2.151059in}{5.146703in}}%
\pgfusepath{stroke}%
\end{pgfscope}%
\begin{pgfscope}%
\pgfpathrectangle{\pgfqpoint{1.250000in}{4.155455in}}{\pgfqpoint{2.279412in}{2.004545in}}%
\pgfusepath{clip}%
\pgfsetbuttcap%
\pgfsetroundjoin%
\pgfsetlinewidth{0.386080pt}%
\definecolor{currentstroke}{rgb}{0.280267,0.073417,0.397163}%
\pgfsetstrokecolor{currentstroke}%
\pgfsetdash{}{0pt}%
\pgfpathmoveto{\pgfqpoint{2.151059in}{5.146703in}}%
\pgfpathlineto{\pgfqpoint{2.151059in}{5.146703in}}%
\pgfusepath{stroke}%
\end{pgfscope}%
\begin{pgfscope}%
\pgfpathrectangle{\pgfqpoint{1.250000in}{4.155455in}}{\pgfqpoint{2.279412in}{2.004545in}}%
\pgfusepath{clip}%
\pgfsetbuttcap%
\pgfsetroundjoin%
\pgfsetlinewidth{0.386080pt}%
\definecolor{currentstroke}{rgb}{0.280267,0.073417,0.397163}%
\pgfsetstrokecolor{currentstroke}%
\pgfsetdash{}{0pt}%
\pgfpathmoveto{\pgfqpoint{2.151059in}{5.146703in}}%
\pgfpathlineto{\pgfqpoint{2.151059in}{5.146703in}}%
\pgfusepath{stroke}%
\end{pgfscope}%
\begin{pgfscope}%
\pgfpathrectangle{\pgfqpoint{1.250000in}{4.155455in}}{\pgfqpoint{2.279412in}{2.004545in}}%
\pgfusepath{clip}%
\pgfsetbuttcap%
\pgfsetroundjoin%
\pgfsetlinewidth{0.386080pt}%
\definecolor{currentstroke}{rgb}{0.280267,0.073417,0.397163}%
\pgfsetstrokecolor{currentstroke}%
\pgfsetdash{}{0pt}%
\pgfpathmoveto{\pgfqpoint{2.151059in}{5.146703in}}%
\pgfpathlineto{\pgfqpoint{2.151059in}{5.146703in}}%
\pgfusepath{stroke}%
\end{pgfscope}%
\begin{pgfscope}%
\pgfpathrectangle{\pgfqpoint{1.250000in}{4.155455in}}{\pgfqpoint{2.279412in}{2.004545in}}%
\pgfusepath{clip}%
\pgfsetbuttcap%
\pgfsetroundjoin%
\pgfsetlinewidth{0.386080pt}%
\definecolor{currentstroke}{rgb}{0.280267,0.073417,0.397163}%
\pgfsetstrokecolor{currentstroke}%
\pgfsetdash{}{0pt}%
\pgfpathmoveto{\pgfqpoint{2.151059in}{5.146703in}}%
\pgfpathlineto{\pgfqpoint{2.151059in}{5.146703in}}%
\pgfusepath{stroke}%
\end{pgfscope}%
\begin{pgfscope}%
\pgfpathrectangle{\pgfqpoint{1.250000in}{4.155455in}}{\pgfqpoint{2.279412in}{2.004545in}}%
\pgfusepath{clip}%
\pgfsetbuttcap%
\pgfsetroundjoin%
\pgfsetlinewidth{0.386080pt}%
\definecolor{currentstroke}{rgb}{0.280267,0.073417,0.397163}%
\pgfsetstrokecolor{currentstroke}%
\pgfsetdash{}{0pt}%
\pgfpathmoveto{\pgfqpoint{2.151059in}{5.146703in}}%
\pgfpathlineto{\pgfqpoint{2.151059in}{5.146703in}}%
\pgfusepath{stroke}%
\end{pgfscope}%
\begin{pgfscope}%
\pgfpathrectangle{\pgfqpoint{1.250000in}{4.155455in}}{\pgfqpoint{2.279412in}{2.004545in}}%
\pgfusepath{clip}%
\pgfsetbuttcap%
\pgfsetroundjoin%
\pgfsetlinewidth{0.386080pt}%
\definecolor{currentstroke}{rgb}{0.280267,0.073417,0.397163}%
\pgfsetstrokecolor{currentstroke}%
\pgfsetdash{}{0pt}%
\pgfpathmoveto{\pgfqpoint{2.151059in}{5.146703in}}%
\pgfpathlineto{\pgfqpoint{2.151059in}{5.146703in}}%
\pgfusepath{stroke}%
\end{pgfscope}%
\begin{pgfscope}%
\pgfpathrectangle{\pgfqpoint{1.250000in}{4.155455in}}{\pgfqpoint{2.279412in}{2.004545in}}%
\pgfusepath{clip}%
\pgfsetbuttcap%
\pgfsetroundjoin%
\pgfsetlinewidth{0.386080pt}%
\definecolor{currentstroke}{rgb}{0.280267,0.073417,0.397163}%
\pgfsetstrokecolor{currentstroke}%
\pgfsetdash{}{0pt}%
\pgfpathmoveto{\pgfqpoint{2.151059in}{5.146703in}}%
\pgfpathlineto{\pgfqpoint{2.151059in}{5.146703in}}%
\pgfusepath{stroke}%
\end{pgfscope}%
\begin{pgfscope}%
\pgfpathrectangle{\pgfqpoint{1.250000in}{4.155455in}}{\pgfqpoint{2.279412in}{2.004545in}}%
\pgfusepath{clip}%
\pgfsetbuttcap%
\pgfsetroundjoin%
\pgfsetlinewidth{0.386080pt}%
\definecolor{currentstroke}{rgb}{0.280267,0.073417,0.397163}%
\pgfsetstrokecolor{currentstroke}%
\pgfsetdash{}{0pt}%
\pgfpathmoveto{\pgfqpoint{2.151059in}{5.146703in}}%
\pgfpathlineto{\pgfqpoint{2.151059in}{5.146703in}}%
\pgfusepath{stroke}%
\end{pgfscope}%
\begin{pgfscope}%
\pgfpathrectangle{\pgfqpoint{1.250000in}{4.155455in}}{\pgfqpoint{2.279412in}{2.004545in}}%
\pgfusepath{clip}%
\pgfsetbuttcap%
\pgfsetroundjoin%
\pgfsetlinewidth{0.386080pt}%
\definecolor{currentstroke}{rgb}{0.280267,0.073417,0.397163}%
\pgfsetstrokecolor{currentstroke}%
\pgfsetdash{}{0pt}%
\pgfpathmoveto{\pgfqpoint{2.151059in}{5.146703in}}%
\pgfpathlineto{\pgfqpoint{2.151059in}{5.146703in}}%
\pgfusepath{stroke}%
\end{pgfscope}%
\begin{pgfscope}%
\pgfpathrectangle{\pgfqpoint{1.250000in}{4.155455in}}{\pgfqpoint{2.279412in}{2.004545in}}%
\pgfusepath{clip}%
\pgfsetbuttcap%
\pgfsetroundjoin%
\pgfsetlinewidth{0.386080pt}%
\definecolor{currentstroke}{rgb}{0.280267,0.073417,0.397163}%
\pgfsetstrokecolor{currentstroke}%
\pgfsetdash{}{0pt}%
\pgfpathmoveto{\pgfqpoint{2.151059in}{5.146703in}}%
\pgfpathlineto{\pgfqpoint{2.151059in}{5.146703in}}%
\pgfusepath{stroke}%
\end{pgfscope}%
\begin{pgfscope}%
\pgfpathrectangle{\pgfqpoint{1.250000in}{4.155455in}}{\pgfqpoint{2.279412in}{2.004545in}}%
\pgfusepath{clip}%
\pgfsetbuttcap%
\pgfsetroundjoin%
\pgfsetlinewidth{0.386080pt}%
\definecolor{currentstroke}{rgb}{0.280267,0.073417,0.397163}%
\pgfsetstrokecolor{currentstroke}%
\pgfsetdash{}{0pt}%
\pgfpathmoveto{\pgfqpoint{2.151059in}{5.146703in}}%
\pgfpathlineto{\pgfqpoint{2.151059in}{5.146703in}}%
\pgfusepath{stroke}%
\end{pgfscope}%
\begin{pgfscope}%
\pgfpathrectangle{\pgfqpoint{1.250000in}{4.155455in}}{\pgfqpoint{2.279412in}{2.004545in}}%
\pgfusepath{clip}%
\pgfsetbuttcap%
\pgfsetroundjoin%
\pgfsetlinewidth{0.386080pt}%
\definecolor{currentstroke}{rgb}{0.280267,0.073417,0.397163}%
\pgfsetstrokecolor{currentstroke}%
\pgfsetdash{}{0pt}%
\pgfpathmoveto{\pgfqpoint{2.151059in}{5.146703in}}%
\pgfpathlineto{\pgfqpoint{2.151059in}{5.146703in}}%
\pgfusepath{stroke}%
\end{pgfscope}%
\begin{pgfscope}%
\pgfpathrectangle{\pgfqpoint{1.250000in}{4.155455in}}{\pgfqpoint{2.279412in}{2.004545in}}%
\pgfusepath{clip}%
\pgfsetbuttcap%
\pgfsetroundjoin%
\pgfsetlinewidth{0.386080pt}%
\definecolor{currentstroke}{rgb}{0.280267,0.073417,0.397163}%
\pgfsetstrokecolor{currentstroke}%
\pgfsetdash{}{0pt}%
\pgfpathmoveto{\pgfqpoint{2.151059in}{5.146703in}}%
\pgfpathlineto{\pgfqpoint{2.151059in}{5.146703in}}%
\pgfusepath{stroke}%
\end{pgfscope}%
\begin{pgfscope}%
\pgfpathrectangle{\pgfqpoint{1.250000in}{4.155455in}}{\pgfqpoint{2.279412in}{2.004545in}}%
\pgfusepath{clip}%
\pgfsetbuttcap%
\pgfsetroundjoin%
\pgfsetlinewidth{0.386080pt}%
\definecolor{currentstroke}{rgb}{0.280267,0.073417,0.397163}%
\pgfsetstrokecolor{currentstroke}%
\pgfsetdash{}{0pt}%
\pgfpathmoveto{\pgfqpoint{2.151059in}{5.146703in}}%
\pgfpathlineto{\pgfqpoint{2.151059in}{5.146703in}}%
\pgfusepath{stroke}%
\end{pgfscope}%
\begin{pgfscope}%
\pgfpathrectangle{\pgfqpoint{1.250000in}{4.155455in}}{\pgfqpoint{2.279412in}{2.004545in}}%
\pgfusepath{clip}%
\pgfsetbuttcap%
\pgfsetroundjoin%
\pgfsetlinewidth{0.386080pt}%
\definecolor{currentstroke}{rgb}{0.280267,0.073417,0.397163}%
\pgfsetstrokecolor{currentstroke}%
\pgfsetdash{}{0pt}%
\pgfpathmoveto{\pgfqpoint{2.151059in}{5.146703in}}%
\pgfpathlineto{\pgfqpoint{2.151059in}{5.146703in}}%
\pgfusepath{stroke}%
\end{pgfscope}%
\begin{pgfscope}%
\pgfpathrectangle{\pgfqpoint{1.250000in}{4.155455in}}{\pgfqpoint{2.279412in}{2.004545in}}%
\pgfusepath{clip}%
\pgfsetbuttcap%
\pgfsetroundjoin%
\pgfsetlinewidth{0.386080pt}%
\definecolor{currentstroke}{rgb}{0.280267,0.073417,0.397163}%
\pgfsetstrokecolor{currentstroke}%
\pgfsetdash{}{0pt}%
\pgfpathmoveto{\pgfqpoint{2.151059in}{5.146703in}}%
\pgfpathlineto{\pgfqpoint{2.151059in}{5.146703in}}%
\pgfusepath{stroke}%
\end{pgfscope}%
\begin{pgfscope}%
\pgfpathrectangle{\pgfqpoint{1.250000in}{4.155455in}}{\pgfqpoint{2.279412in}{2.004545in}}%
\pgfusepath{clip}%
\pgfsetbuttcap%
\pgfsetroundjoin%
\pgfsetlinewidth{0.386080pt}%
\definecolor{currentstroke}{rgb}{0.280267,0.073417,0.397163}%
\pgfsetstrokecolor{currentstroke}%
\pgfsetdash{}{0pt}%
\pgfpathmoveto{\pgfqpoint{2.151059in}{5.146703in}}%
\pgfpathlineto{\pgfqpoint{2.151059in}{5.146703in}}%
\pgfusepath{stroke}%
\end{pgfscope}%
\begin{pgfscope}%
\pgfpathrectangle{\pgfqpoint{1.250000in}{4.155455in}}{\pgfqpoint{2.279412in}{2.004545in}}%
\pgfusepath{clip}%
\pgfsetbuttcap%
\pgfsetroundjoin%
\pgfsetlinewidth{0.386080pt}%
\definecolor{currentstroke}{rgb}{0.280267,0.073417,0.397163}%
\pgfsetstrokecolor{currentstroke}%
\pgfsetdash{}{0pt}%
\pgfpathmoveto{\pgfqpoint{2.151059in}{5.146703in}}%
\pgfpathlineto{\pgfqpoint{2.151059in}{5.146703in}}%
\pgfusepath{stroke}%
\end{pgfscope}%
\begin{pgfscope}%
\pgfpathrectangle{\pgfqpoint{1.250000in}{4.155455in}}{\pgfqpoint{2.279412in}{2.004545in}}%
\pgfusepath{clip}%
\pgfsetbuttcap%
\pgfsetroundjoin%
\pgfsetlinewidth{0.386080pt}%
\definecolor{currentstroke}{rgb}{0.280267,0.073417,0.397163}%
\pgfsetstrokecolor{currentstroke}%
\pgfsetdash{}{0pt}%
\pgfpathmoveto{\pgfqpoint{2.151059in}{5.146703in}}%
\pgfpathlineto{\pgfqpoint{2.151059in}{5.146703in}}%
\pgfusepath{stroke}%
\end{pgfscope}%
\begin{pgfscope}%
\pgfpathrectangle{\pgfqpoint{1.250000in}{4.155455in}}{\pgfqpoint{2.279412in}{2.004545in}}%
\pgfusepath{clip}%
\pgfsetbuttcap%
\pgfsetroundjoin%
\pgfsetlinewidth{0.386080pt}%
\definecolor{currentstroke}{rgb}{0.280267,0.073417,0.397163}%
\pgfsetstrokecolor{currentstroke}%
\pgfsetdash{}{0pt}%
\pgfpathmoveto{\pgfqpoint{2.151059in}{5.146703in}}%
\pgfpathlineto{\pgfqpoint{2.151059in}{5.146703in}}%
\pgfusepath{stroke}%
\end{pgfscope}%
\begin{pgfscope}%
\pgfpathrectangle{\pgfqpoint{1.250000in}{4.155455in}}{\pgfqpoint{2.279412in}{2.004545in}}%
\pgfusepath{clip}%
\pgfsetbuttcap%
\pgfsetroundjoin%
\pgfsetlinewidth{0.386080pt}%
\definecolor{currentstroke}{rgb}{0.280267,0.073417,0.397163}%
\pgfsetstrokecolor{currentstroke}%
\pgfsetdash{}{0pt}%
\pgfpathmoveto{\pgfqpoint{2.151059in}{5.146703in}}%
\pgfpathlineto{\pgfqpoint{2.151059in}{5.146703in}}%
\pgfusepath{stroke}%
\end{pgfscope}%
\begin{pgfscope}%
\pgfpathrectangle{\pgfqpoint{1.250000in}{4.155455in}}{\pgfqpoint{2.279412in}{2.004545in}}%
\pgfusepath{clip}%
\pgfsetbuttcap%
\pgfsetroundjoin%
\pgfsetlinewidth{0.386080pt}%
\definecolor{currentstroke}{rgb}{0.280267,0.073417,0.397163}%
\pgfsetstrokecolor{currentstroke}%
\pgfsetdash{}{0pt}%
\pgfpathmoveto{\pgfqpoint{2.151059in}{5.146703in}}%
\pgfpathlineto{\pgfqpoint{2.151059in}{5.146703in}}%
\pgfusepath{stroke}%
\end{pgfscope}%
\begin{pgfscope}%
\pgfpathrectangle{\pgfqpoint{1.250000in}{4.155455in}}{\pgfqpoint{2.279412in}{2.004545in}}%
\pgfusepath{clip}%
\pgfsetbuttcap%
\pgfsetroundjoin%
\pgfsetlinewidth{0.386080pt}%
\definecolor{currentstroke}{rgb}{0.280267,0.073417,0.397163}%
\pgfsetstrokecolor{currentstroke}%
\pgfsetdash{}{0pt}%
\pgfpathmoveto{\pgfqpoint{2.151059in}{5.146703in}}%
\pgfpathlineto{\pgfqpoint{2.151059in}{5.146703in}}%
\pgfusepath{stroke}%
\end{pgfscope}%
\begin{pgfscope}%
\pgfpathrectangle{\pgfqpoint{1.250000in}{4.155455in}}{\pgfqpoint{2.279412in}{2.004545in}}%
\pgfusepath{clip}%
\pgfsetbuttcap%
\pgfsetroundjoin%
\pgfsetlinewidth{0.386080pt}%
\definecolor{currentstroke}{rgb}{0.280267,0.073417,0.397163}%
\pgfsetstrokecolor{currentstroke}%
\pgfsetdash{}{0pt}%
\pgfpathmoveto{\pgfqpoint{2.151059in}{5.146703in}}%
\pgfpathlineto{\pgfqpoint{2.151059in}{5.146703in}}%
\pgfusepath{stroke}%
\end{pgfscope}%
\begin{pgfscope}%
\pgfpathrectangle{\pgfqpoint{1.250000in}{4.155455in}}{\pgfqpoint{2.279412in}{2.004545in}}%
\pgfusepath{clip}%
\pgfsetbuttcap%
\pgfsetroundjoin%
\pgfsetlinewidth{0.386080pt}%
\definecolor{currentstroke}{rgb}{0.280267,0.073417,0.397163}%
\pgfsetstrokecolor{currentstroke}%
\pgfsetdash{}{0pt}%
\pgfpathmoveto{\pgfqpoint{2.151059in}{5.146703in}}%
\pgfpathlineto{\pgfqpoint{2.151059in}{5.146703in}}%
\pgfusepath{stroke}%
\end{pgfscope}%
\begin{pgfscope}%
\pgfpathrectangle{\pgfqpoint{1.250000in}{4.155455in}}{\pgfqpoint{2.279412in}{2.004545in}}%
\pgfusepath{clip}%
\pgfsetbuttcap%
\pgfsetroundjoin%
\pgfsetlinewidth{0.386080pt}%
\definecolor{currentstroke}{rgb}{0.280267,0.073417,0.397163}%
\pgfsetstrokecolor{currentstroke}%
\pgfsetdash{}{0pt}%
\pgfpathmoveto{\pgfqpoint{2.151059in}{5.146703in}}%
\pgfpathlineto{\pgfqpoint{2.151059in}{5.146703in}}%
\pgfusepath{stroke}%
\end{pgfscope}%
\begin{pgfscope}%
\pgfpathrectangle{\pgfqpoint{1.250000in}{4.155455in}}{\pgfqpoint{2.279412in}{2.004545in}}%
\pgfusepath{clip}%
\pgfsetbuttcap%
\pgfsetroundjoin%
\pgfsetlinewidth{0.386080pt}%
\definecolor{currentstroke}{rgb}{0.280267,0.073417,0.397163}%
\pgfsetstrokecolor{currentstroke}%
\pgfsetdash{}{0pt}%
\pgfpathmoveto{\pgfqpoint{2.151059in}{5.146703in}}%
\pgfpathlineto{\pgfqpoint{2.151059in}{5.146703in}}%
\pgfusepath{stroke}%
\end{pgfscope}%
\begin{pgfscope}%
\pgfpathrectangle{\pgfqpoint{1.250000in}{4.155455in}}{\pgfqpoint{2.279412in}{2.004545in}}%
\pgfusepath{clip}%
\pgfsetbuttcap%
\pgfsetroundjoin%
\pgfsetlinewidth{0.386080pt}%
\definecolor{currentstroke}{rgb}{0.280267,0.073417,0.397163}%
\pgfsetstrokecolor{currentstroke}%
\pgfsetdash{}{0pt}%
\pgfpathmoveto{\pgfqpoint{2.151059in}{5.146703in}}%
\pgfpathlineto{\pgfqpoint{2.151059in}{5.146703in}}%
\pgfusepath{stroke}%
\end{pgfscope}%
\begin{pgfscope}%
\pgfpathrectangle{\pgfqpoint{1.250000in}{4.155455in}}{\pgfqpoint{2.279412in}{2.004545in}}%
\pgfusepath{clip}%
\pgfsetbuttcap%
\pgfsetroundjoin%
\pgfsetlinewidth{0.386080pt}%
\definecolor{currentstroke}{rgb}{0.280267,0.073417,0.397163}%
\pgfsetstrokecolor{currentstroke}%
\pgfsetdash{}{0pt}%
\pgfpathmoveto{\pgfqpoint{2.151059in}{5.146703in}}%
\pgfpathlineto{\pgfqpoint{2.151059in}{5.146703in}}%
\pgfusepath{stroke}%
\end{pgfscope}%
\begin{pgfscope}%
\pgfpathrectangle{\pgfqpoint{1.250000in}{4.155455in}}{\pgfqpoint{2.279412in}{2.004545in}}%
\pgfusepath{clip}%
\pgfsetbuttcap%
\pgfsetroundjoin%
\pgfsetlinewidth{0.386080pt}%
\definecolor{currentstroke}{rgb}{0.280267,0.073417,0.397163}%
\pgfsetstrokecolor{currentstroke}%
\pgfsetdash{}{0pt}%
\pgfpathmoveto{\pgfqpoint{2.151059in}{5.146703in}}%
\pgfpathlineto{\pgfqpoint{2.151059in}{5.146703in}}%
\pgfusepath{stroke}%
\end{pgfscope}%
\begin{pgfscope}%
\pgfpathrectangle{\pgfqpoint{1.250000in}{4.155455in}}{\pgfqpoint{2.279412in}{2.004545in}}%
\pgfusepath{clip}%
\pgfsetbuttcap%
\pgfsetroundjoin%
\pgfsetlinewidth{0.386080pt}%
\definecolor{currentstroke}{rgb}{0.280267,0.073417,0.397163}%
\pgfsetstrokecolor{currentstroke}%
\pgfsetdash{}{0pt}%
\pgfpathmoveto{\pgfqpoint{2.151059in}{5.146703in}}%
\pgfpathlineto{\pgfqpoint{2.151059in}{5.146703in}}%
\pgfusepath{stroke}%
\end{pgfscope}%
\begin{pgfscope}%
\pgfpathrectangle{\pgfqpoint{1.250000in}{4.155455in}}{\pgfqpoint{2.279412in}{2.004545in}}%
\pgfusepath{clip}%
\pgfsetbuttcap%
\pgfsetroundjoin%
\pgfsetlinewidth{0.386080pt}%
\definecolor{currentstroke}{rgb}{0.280267,0.073417,0.397163}%
\pgfsetstrokecolor{currentstroke}%
\pgfsetdash{}{0pt}%
\pgfpathmoveto{\pgfqpoint{2.151059in}{5.146703in}}%
\pgfpathlineto{\pgfqpoint{2.151059in}{5.146703in}}%
\pgfusepath{stroke}%
\end{pgfscope}%
\begin{pgfscope}%
\pgfpathrectangle{\pgfqpoint{1.250000in}{4.155455in}}{\pgfqpoint{2.279412in}{2.004545in}}%
\pgfusepath{clip}%
\pgfsetbuttcap%
\pgfsetroundjoin%
\pgfsetlinewidth{0.386080pt}%
\definecolor{currentstroke}{rgb}{0.280267,0.073417,0.397163}%
\pgfsetstrokecolor{currentstroke}%
\pgfsetdash{}{0pt}%
\pgfpathmoveto{\pgfqpoint{2.151059in}{5.146703in}}%
\pgfpathlineto{\pgfqpoint{2.151059in}{5.146703in}}%
\pgfusepath{stroke}%
\end{pgfscope}%
\begin{pgfscope}%
\pgfpathrectangle{\pgfqpoint{1.250000in}{4.155455in}}{\pgfqpoint{2.279412in}{2.004545in}}%
\pgfusepath{clip}%
\pgfsetbuttcap%
\pgfsetroundjoin%
\pgfsetlinewidth{0.386080pt}%
\definecolor{currentstroke}{rgb}{0.280267,0.073417,0.397163}%
\pgfsetstrokecolor{currentstroke}%
\pgfsetdash{}{0pt}%
\pgfpathmoveto{\pgfqpoint{2.151059in}{5.146703in}}%
\pgfpathlineto{\pgfqpoint{2.151059in}{5.146703in}}%
\pgfusepath{stroke}%
\end{pgfscope}%
\begin{pgfscope}%
\pgfpathrectangle{\pgfqpoint{1.250000in}{4.155455in}}{\pgfqpoint{2.279412in}{2.004545in}}%
\pgfusepath{clip}%
\pgfsetbuttcap%
\pgfsetroundjoin%
\pgfsetlinewidth{0.386080pt}%
\definecolor{currentstroke}{rgb}{0.280267,0.073417,0.397163}%
\pgfsetstrokecolor{currentstroke}%
\pgfsetdash{}{0pt}%
\pgfpathmoveto{\pgfqpoint{2.151059in}{5.146703in}}%
\pgfpathlineto{\pgfqpoint{2.151059in}{5.146703in}}%
\pgfusepath{stroke}%
\end{pgfscope}%
\begin{pgfscope}%
\pgfpathrectangle{\pgfqpoint{1.250000in}{4.155455in}}{\pgfqpoint{2.279412in}{2.004545in}}%
\pgfusepath{clip}%
\pgfsetbuttcap%
\pgfsetroundjoin%
\pgfsetlinewidth{0.386080pt}%
\definecolor{currentstroke}{rgb}{0.280267,0.073417,0.397163}%
\pgfsetstrokecolor{currentstroke}%
\pgfsetdash{}{0pt}%
\pgfpathmoveto{\pgfqpoint{2.151059in}{5.146703in}}%
\pgfpathlineto{\pgfqpoint{2.151059in}{5.146703in}}%
\pgfusepath{stroke}%
\end{pgfscope}%
\begin{pgfscope}%
\pgfpathrectangle{\pgfqpoint{1.250000in}{4.155455in}}{\pgfqpoint{2.279412in}{2.004545in}}%
\pgfusepath{clip}%
\pgfsetbuttcap%
\pgfsetroundjoin%
\pgfsetlinewidth{0.386080pt}%
\definecolor{currentstroke}{rgb}{0.280267,0.073417,0.397163}%
\pgfsetstrokecolor{currentstroke}%
\pgfsetdash{}{0pt}%
\pgfpathmoveto{\pgfqpoint{2.151059in}{5.146703in}}%
\pgfpathlineto{\pgfqpoint{2.151059in}{5.146703in}}%
\pgfusepath{stroke}%
\end{pgfscope}%
\begin{pgfscope}%
\pgfpathrectangle{\pgfqpoint{1.250000in}{4.155455in}}{\pgfqpoint{2.279412in}{2.004545in}}%
\pgfusepath{clip}%
\pgfsetbuttcap%
\pgfsetroundjoin%
\pgfsetlinewidth{0.386080pt}%
\definecolor{currentstroke}{rgb}{0.280267,0.073417,0.397163}%
\pgfsetstrokecolor{currentstroke}%
\pgfsetdash{}{0pt}%
\pgfpathmoveto{\pgfqpoint{2.151059in}{5.146703in}}%
\pgfpathlineto{\pgfqpoint{2.151059in}{5.146703in}}%
\pgfusepath{stroke}%
\end{pgfscope}%
\begin{pgfscope}%
\pgfpathrectangle{\pgfqpoint{1.250000in}{4.155455in}}{\pgfqpoint{2.279412in}{2.004545in}}%
\pgfusepath{clip}%
\pgfsetbuttcap%
\pgfsetroundjoin%
\pgfsetlinewidth{0.386080pt}%
\definecolor{currentstroke}{rgb}{0.280267,0.073417,0.397163}%
\pgfsetstrokecolor{currentstroke}%
\pgfsetdash{}{0pt}%
\pgfpathmoveto{\pgfqpoint{2.151059in}{5.146703in}}%
\pgfpathlineto{\pgfqpoint{2.151059in}{5.146703in}}%
\pgfusepath{stroke}%
\end{pgfscope}%
\begin{pgfscope}%
\pgfpathrectangle{\pgfqpoint{1.250000in}{4.155455in}}{\pgfqpoint{2.279412in}{2.004545in}}%
\pgfusepath{clip}%
\pgfsetbuttcap%
\pgfsetroundjoin%
\pgfsetlinewidth{0.386080pt}%
\definecolor{currentstroke}{rgb}{0.280267,0.073417,0.397163}%
\pgfsetstrokecolor{currentstroke}%
\pgfsetdash{}{0pt}%
\pgfpathmoveto{\pgfqpoint{2.151059in}{5.146703in}}%
\pgfpathlineto{\pgfqpoint{2.151059in}{5.146703in}}%
\pgfusepath{stroke}%
\end{pgfscope}%
\begin{pgfscope}%
\pgfpathrectangle{\pgfqpoint{1.250000in}{4.155455in}}{\pgfqpoint{2.279412in}{2.004545in}}%
\pgfusepath{clip}%
\pgfsetbuttcap%
\pgfsetroundjoin%
\pgfsetlinewidth{0.386080pt}%
\definecolor{currentstroke}{rgb}{0.280267,0.073417,0.397163}%
\pgfsetstrokecolor{currentstroke}%
\pgfsetdash{}{0pt}%
\pgfpathmoveto{\pgfqpoint{2.151059in}{5.146703in}}%
\pgfpathlineto{\pgfqpoint{2.151059in}{5.146703in}}%
\pgfusepath{stroke}%
\end{pgfscope}%
\begin{pgfscope}%
\pgfpathrectangle{\pgfqpoint{1.250000in}{4.155455in}}{\pgfqpoint{2.279412in}{2.004545in}}%
\pgfusepath{clip}%
\pgfsetbuttcap%
\pgfsetroundjoin%
\pgfsetlinewidth{0.386080pt}%
\definecolor{currentstroke}{rgb}{0.280267,0.073417,0.397163}%
\pgfsetstrokecolor{currentstroke}%
\pgfsetdash{}{0pt}%
\pgfpathmoveto{\pgfqpoint{2.151059in}{5.146703in}}%
\pgfpathlineto{\pgfqpoint{2.151059in}{5.146703in}}%
\pgfusepath{stroke}%
\end{pgfscope}%
\begin{pgfscope}%
\pgfpathrectangle{\pgfqpoint{1.250000in}{4.155455in}}{\pgfqpoint{2.279412in}{2.004545in}}%
\pgfusepath{clip}%
\pgfsetbuttcap%
\pgfsetroundjoin%
\pgfsetlinewidth{0.000000pt}%
\definecolor{currentstroke}{rgb}{0.280267,0.073417,0.397163}%
\pgfsetstrokecolor{currentstroke}%
\pgfsetdash{}{0pt}%
\pgfpathmoveto{\pgfqpoint{3.310736in}{5.160505in}}%
\pgfpathlineto{\pgfqpoint{3.261668in}{5.157727in}}%
\pgfusepath{}%
\end{pgfscope}%
\begin{pgfscope}%
\pgfpathrectangle{\pgfqpoint{1.250000in}{4.155455in}}{\pgfqpoint{2.279412in}{2.004545in}}%
\pgfusepath{clip}%
\pgfsetbuttcap%
\pgfsetroundjoin%
\pgfsetlinewidth{0.316510pt}%
\definecolor{currentstroke}{rgb}{0.269944,0.014625,0.341379}%
\pgfsetstrokecolor{currentstroke}%
\pgfsetdash{}{0pt}%
\pgfpathmoveto{\pgfqpoint{3.261668in}{5.157727in}}%
\pgfpathlineto{\pgfqpoint{3.211586in}{5.157822in}}%
\pgfusepath{stroke}%
\end{pgfscope}%
\begin{pgfscope}%
\pgfpathrectangle{\pgfqpoint{1.250000in}{4.155455in}}{\pgfqpoint{2.279412in}{2.004545in}}%
\pgfusepath{clip}%
\pgfsetbuttcap%
\pgfsetroundjoin%
\pgfsetlinewidth{0.324450pt}%
\definecolor{currentstroke}{rgb}{0.271305,0.019942,0.347269}%
\pgfsetstrokecolor{currentstroke}%
\pgfsetdash{}{0pt}%
\pgfpathmoveto{\pgfqpoint{3.211586in}{5.157822in}}%
\pgfpathlineto{\pgfqpoint{3.161448in}{5.156906in}}%
\pgfusepath{stroke}%
\end{pgfscope}%
\begin{pgfscope}%
\pgfpathrectangle{\pgfqpoint{1.250000in}{4.155455in}}{\pgfqpoint{2.279412in}{2.004545in}}%
\pgfusepath{clip}%
\pgfsetbuttcap%
\pgfsetroundjoin%
\pgfsetlinewidth{0.324392pt}%
\definecolor{currentstroke}{rgb}{0.271305,0.019942,0.347269}%
\pgfsetstrokecolor{currentstroke}%
\pgfsetdash{}{0pt}%
\pgfpathmoveto{\pgfqpoint{3.161448in}{5.156906in}}%
\pgfpathlineto{\pgfqpoint{3.111314in}{5.156607in}}%
\pgfusepath{stroke}%
\end{pgfscope}%
\begin{pgfscope}%
\pgfpathrectangle{\pgfqpoint{1.250000in}{4.155455in}}{\pgfqpoint{2.279412in}{2.004545in}}%
\pgfusepath{clip}%
\pgfsetbuttcap%
\pgfsetroundjoin%
\pgfsetlinewidth{0.339024pt}%
\definecolor{currentstroke}{rgb}{0.273809,0.031497,0.358853}%
\pgfsetstrokecolor{currentstroke}%
\pgfsetdash{}{0pt}%
\pgfpathmoveto{\pgfqpoint{3.111314in}{5.156607in}}%
\pgfpathlineto{\pgfqpoint{3.061165in}{5.156906in}}%
\pgfusepath{stroke}%
\end{pgfscope}%
\begin{pgfscope}%
\pgfpathrectangle{\pgfqpoint{1.250000in}{4.155455in}}{\pgfqpoint{2.279412in}{2.004545in}}%
\pgfusepath{clip}%
\pgfsetbuttcap%
\pgfsetroundjoin%
\pgfsetlinewidth{0.355958pt}%
\definecolor{currentstroke}{rgb}{0.276022,0.044167,0.370164}%
\pgfsetstrokecolor{currentstroke}%
\pgfsetdash{}{0pt}%
\pgfpathmoveto{\pgfqpoint{3.061165in}{5.156906in}}%
\pgfpathlineto{\pgfqpoint{3.011016in}{5.156845in}}%
\pgfusepath{stroke}%
\end{pgfscope}%
\begin{pgfscope}%
\pgfpathrectangle{\pgfqpoint{1.250000in}{4.155455in}}{\pgfqpoint{2.279412in}{2.004545in}}%
\pgfusepath{clip}%
\pgfsetbuttcap%
\pgfsetroundjoin%
\pgfsetlinewidth{0.382949pt}%
\definecolor{currentstroke}{rgb}{0.279566,0.067836,0.391917}%
\pgfsetstrokecolor{currentstroke}%
\pgfsetdash{}{0pt}%
\pgfpathmoveto{\pgfqpoint{3.011016in}{5.156845in}}%
\pgfpathlineto{\pgfqpoint{2.960866in}{5.156789in}}%
\pgfusepath{stroke}%
\end{pgfscope}%
\begin{pgfscope}%
\pgfpathrectangle{\pgfqpoint{1.250000in}{4.155455in}}{\pgfqpoint{2.279412in}{2.004545in}}%
\pgfusepath{clip}%
\pgfsetbuttcap%
\pgfsetroundjoin%
\pgfsetlinewidth{0.411358pt}%
\definecolor{currentstroke}{rgb}{0.282327,0.094955,0.417331}%
\pgfsetstrokecolor{currentstroke}%
\pgfsetdash{}{0pt}%
\pgfpathmoveto{\pgfqpoint{2.960866in}{5.156789in}}%
\pgfpathlineto{\pgfqpoint{2.910715in}{5.156918in}}%
\pgfusepath{stroke}%
\end{pgfscope}%
\begin{pgfscope}%
\pgfpathrectangle{\pgfqpoint{1.250000in}{4.155455in}}{\pgfqpoint{2.279412in}{2.004545in}}%
\pgfusepath{clip}%
\pgfsetbuttcap%
\pgfsetroundjoin%
\pgfsetlinewidth{0.458282pt}%
\definecolor{currentstroke}{rgb}{0.283187,0.125848,0.444960}%
\pgfsetstrokecolor{currentstroke}%
\pgfsetdash{}{0pt}%
\pgfpathmoveto{\pgfqpoint{2.910715in}{5.156918in}}%
\pgfpathlineto{\pgfqpoint{2.860564in}{5.157139in}}%
\pgfusepath{stroke}%
\end{pgfscope}%
\begin{pgfscope}%
\pgfpathrectangle{\pgfqpoint{1.250000in}{4.155455in}}{\pgfqpoint{2.279412in}{2.004545in}}%
\pgfusepath{clip}%
\pgfsetbuttcap%
\pgfsetroundjoin%
\pgfsetlinewidth{0.526522pt}%
\definecolor{currentstroke}{rgb}{0.278826,0.175490,0.483397}%
\pgfsetstrokecolor{currentstroke}%
\pgfsetdash{}{0pt}%
\pgfpathmoveto{\pgfqpoint{2.860564in}{5.157139in}}%
\pgfpathlineto{\pgfqpoint{2.810412in}{5.157159in}}%
\pgfusepath{stroke}%
\end{pgfscope}%
\begin{pgfscope}%
\pgfpathrectangle{\pgfqpoint{1.250000in}{4.155455in}}{\pgfqpoint{2.279412in}{2.004545in}}%
\pgfusepath{clip}%
\pgfsetbuttcap%
\pgfsetroundjoin%
\pgfsetlinewidth{0.581152pt}%
\definecolor{currentstroke}{rgb}{0.270595,0.214069,0.507052}%
\pgfsetstrokecolor{currentstroke}%
\pgfsetdash{}{0pt}%
\pgfpathmoveto{\pgfqpoint{2.810412in}{5.157159in}}%
\pgfpathlineto{\pgfqpoint{2.760260in}{5.157111in}}%
\pgfusepath{stroke}%
\end{pgfscope}%
\begin{pgfscope}%
\pgfpathrectangle{\pgfqpoint{1.250000in}{4.155455in}}{\pgfqpoint{2.279412in}{2.004545in}}%
\pgfusepath{clip}%
\pgfsetbuttcap%
\pgfsetroundjoin%
\pgfsetlinewidth{0.657848pt}%
\definecolor{currentstroke}{rgb}{0.253935,0.265254,0.529983}%
\pgfsetstrokecolor{currentstroke}%
\pgfsetdash{}{0pt}%
\pgfpathmoveto{\pgfqpoint{2.760260in}{5.157111in}}%
\pgfpathlineto{\pgfqpoint{2.710108in}{5.157020in}}%
\pgfusepath{stroke}%
\end{pgfscope}%
\begin{pgfscope}%
\pgfpathrectangle{\pgfqpoint{1.250000in}{4.155455in}}{\pgfqpoint{2.279412in}{2.004545in}}%
\pgfusepath{clip}%
\pgfsetbuttcap%
\pgfsetroundjoin%
\pgfsetlinewidth{0.774935pt}%
\definecolor{currentstroke}{rgb}{0.220057,0.343307,0.549413}%
\pgfsetstrokecolor{currentstroke}%
\pgfsetdash{}{0pt}%
\pgfpathmoveto{\pgfqpoint{2.710108in}{5.157020in}}%
\pgfpathlineto{\pgfqpoint{2.659957in}{5.156798in}}%
\pgfusepath{stroke}%
\end{pgfscope}%
\begin{pgfscope}%
\pgfpathrectangle{\pgfqpoint{1.250000in}{4.155455in}}{\pgfqpoint{2.279412in}{2.004545in}}%
\pgfusepath{clip}%
\pgfsetbuttcap%
\pgfsetroundjoin%
\pgfsetlinewidth{0.806102pt}%
\definecolor{currentstroke}{rgb}{0.212395,0.359683,0.551710}%
\pgfsetstrokecolor{currentstroke}%
\pgfsetdash{}{0pt}%
\pgfpathmoveto{\pgfqpoint{2.659957in}{5.156798in}}%
\pgfpathlineto{\pgfqpoint{2.609806in}{5.156687in}}%
\pgfusepath{stroke}%
\end{pgfscope}%
\begin{pgfscope}%
\pgfpathrectangle{\pgfqpoint{1.250000in}{4.155455in}}{\pgfqpoint{2.279412in}{2.004545in}}%
\pgfusepath{clip}%
\pgfsetbuttcap%
\pgfsetroundjoin%
\pgfsetlinewidth{0.917923pt}%
\definecolor{currentstroke}{rgb}{0.183898,0.422383,0.556944}%
\pgfsetstrokecolor{currentstroke}%
\pgfsetdash{}{0pt}%
\pgfpathmoveto{\pgfqpoint{2.609806in}{5.156687in}}%
\pgfpathlineto{\pgfqpoint{2.559654in}{5.156731in}}%
\pgfusepath{stroke}%
\end{pgfscope}%
\begin{pgfscope}%
\pgfpathrectangle{\pgfqpoint{1.250000in}{4.155455in}}{\pgfqpoint{2.279412in}{2.004545in}}%
\pgfusepath{clip}%
\pgfsetbuttcap%
\pgfsetroundjoin%
\pgfsetlinewidth{0.916783pt}%
\definecolor{currentstroke}{rgb}{0.183898,0.422383,0.556944}%
\pgfsetstrokecolor{currentstroke}%
\pgfsetdash{}{0pt}%
\pgfpathmoveto{\pgfqpoint{2.559654in}{5.156731in}}%
\pgfpathlineto{\pgfqpoint{2.509502in}{5.156696in}}%
\pgfusepath{stroke}%
\end{pgfscope}%
\begin{pgfscope}%
\pgfpathrectangle{\pgfqpoint{1.250000in}{4.155455in}}{\pgfqpoint{2.279412in}{2.004545in}}%
\pgfusepath{clip}%
\pgfsetbuttcap%
\pgfsetroundjoin%
\pgfsetlinewidth{0.949927pt}%
\definecolor{currentstroke}{rgb}{0.175841,0.441290,0.557685}%
\pgfsetstrokecolor{currentstroke}%
\pgfsetdash{}{0pt}%
\pgfpathmoveto{\pgfqpoint{2.509502in}{5.156696in}}%
\pgfpathlineto{\pgfqpoint{2.459350in}{5.156643in}}%
\pgfusepath{stroke}%
\end{pgfscope}%
\begin{pgfscope}%
\pgfpathrectangle{\pgfqpoint{1.250000in}{4.155455in}}{\pgfqpoint{2.279412in}{2.004545in}}%
\pgfusepath{clip}%
\pgfsetbuttcap%
\pgfsetroundjoin%
\pgfsetlinewidth{0.956837pt}%
\definecolor{currentstroke}{rgb}{0.174274,0.445044,0.557792}%
\pgfsetstrokecolor{currentstroke}%
\pgfsetdash{}{0pt}%
\pgfpathmoveto{\pgfqpoint{2.459350in}{5.156643in}}%
\pgfpathlineto{\pgfqpoint{2.409199in}{5.156602in}}%
\pgfusepath{stroke}%
\end{pgfscope}%
\begin{pgfscope}%
\pgfpathrectangle{\pgfqpoint{1.250000in}{4.155455in}}{\pgfqpoint{2.279412in}{2.004545in}}%
\pgfusepath{clip}%
\pgfsetbuttcap%
\pgfsetroundjoin%
\pgfsetlinewidth{0.879839pt}%
\definecolor{currentstroke}{rgb}{0.192357,0.403199,0.555836}%
\pgfsetstrokecolor{currentstroke}%
\pgfsetdash{}{0pt}%
\pgfpathmoveto{\pgfqpoint{2.409199in}{5.156602in}}%
\pgfpathlineto{\pgfqpoint{2.359047in}{5.156479in}}%
\pgfusepath{stroke}%
\end{pgfscope}%
\begin{pgfscope}%
\pgfpathrectangle{\pgfqpoint{1.250000in}{4.155455in}}{\pgfqpoint{2.279412in}{2.004545in}}%
\pgfusepath{clip}%
\pgfsetbuttcap%
\pgfsetroundjoin%
\pgfsetlinewidth{0.888691pt}%
\definecolor{currentstroke}{rgb}{0.190631,0.407061,0.556089}%
\pgfsetstrokecolor{currentstroke}%
\pgfsetdash{}{0pt}%
\pgfpathmoveto{\pgfqpoint{2.359047in}{5.156479in}}%
\pgfpathlineto{\pgfqpoint{2.308898in}{5.156155in}}%
\pgfusepath{stroke}%
\end{pgfscope}%
\begin{pgfscope}%
\pgfpathrectangle{\pgfqpoint{1.250000in}{4.155455in}}{\pgfqpoint{2.279412in}{2.004545in}}%
\pgfusepath{clip}%
\pgfsetbuttcap%
\pgfsetroundjoin%
\pgfsetlinewidth{0.751988pt}%
\definecolor{currentstroke}{rgb}{0.227802,0.326594,0.546532}%
\pgfsetstrokecolor{currentstroke}%
\pgfsetdash{}{0pt}%
\pgfpathmoveto{\pgfqpoint{2.308898in}{5.156155in}}%
\pgfpathlineto{\pgfqpoint{2.258751in}{5.155711in}}%
\pgfusepath{stroke}%
\end{pgfscope}%
\begin{pgfscope}%
\pgfpathrectangle{\pgfqpoint{1.250000in}{4.155455in}}{\pgfqpoint{2.279412in}{2.004545in}}%
\pgfusepath{clip}%
\pgfsetbuttcap%
\pgfsetroundjoin%
\pgfsetlinewidth{0.659276pt}%
\definecolor{currentstroke}{rgb}{0.252194,0.269783,0.531579}%
\pgfsetstrokecolor{currentstroke}%
\pgfsetdash{}{0pt}%
\pgfpathmoveto{\pgfqpoint{2.258751in}{5.155711in}}%
\pgfpathlineto{\pgfqpoint{2.208606in}{5.155395in}}%
\pgfusepath{stroke}%
\end{pgfscope}%
\begin{pgfscope}%
\pgfpathrectangle{\pgfqpoint{1.250000in}{4.155455in}}{\pgfqpoint{2.279412in}{2.004545in}}%
\pgfusepath{clip}%
\pgfsetbuttcap%
\pgfsetroundjoin%
\pgfsetlinewidth{0.516727pt}%
\definecolor{currentstroke}{rgb}{0.279574,0.170599,0.479997}%
\pgfsetstrokecolor{currentstroke}%
\pgfsetdash{}{0pt}%
\pgfpathmoveto{\pgfqpoint{2.208606in}{5.155395in}}%
\pgfpathlineto{\pgfqpoint{2.208606in}{5.155395in}}%
\pgfusepath{stroke}%
\end{pgfscope}%
\begin{pgfscope}%
\pgfpathrectangle{\pgfqpoint{1.250000in}{4.155455in}}{\pgfqpoint{2.279412in}{2.004545in}}%
\pgfusepath{clip}%
\pgfsetbuttcap%
\pgfsetroundjoin%
\pgfsetlinewidth{0.516727pt}%
\definecolor{currentstroke}{rgb}{0.279574,0.170599,0.479997}%
\pgfsetstrokecolor{currentstroke}%
\pgfsetdash{}{0pt}%
\pgfpathmoveto{\pgfqpoint{2.208606in}{5.155395in}}%
\pgfpathlineto{\pgfqpoint{2.168719in}{5.153614in}}%
\pgfusepath{stroke}%
\end{pgfscope}%
\begin{pgfscope}%
\pgfpathrectangle{\pgfqpoint{1.250000in}{4.155455in}}{\pgfqpoint{2.279412in}{2.004545in}}%
\pgfusepath{clip}%
\pgfsetbuttcap%
\pgfsetroundjoin%
\pgfsetlinewidth{0.410249pt}%
\definecolor{currentstroke}{rgb}{0.281924,0.089666,0.412415}%
\pgfsetstrokecolor{currentstroke}%
\pgfsetdash{}{0pt}%
\pgfpathmoveto{\pgfqpoint{2.168719in}{5.153614in}}%
\pgfpathlineto{\pgfqpoint{2.168719in}{5.153614in}}%
\pgfusepath{stroke}%
\end{pgfscope}%
\begin{pgfscope}%
\pgfpathrectangle{\pgfqpoint{1.250000in}{4.155455in}}{\pgfqpoint{2.279412in}{2.004545in}}%
\pgfusepath{clip}%
\pgfsetbuttcap%
\pgfsetroundjoin%
\pgfsetlinewidth{0.315957pt}%
\definecolor{currentstroke}{rgb}{0.269944,0.014625,0.341379}%
\pgfsetstrokecolor{currentstroke}%
\pgfsetdash{}{0pt}%
\pgfpathmoveto{\pgfqpoint{3.261668in}{5.202834in}}%
\pgfpathlineto{\pgfqpoint{3.211981in}{5.204770in}}%
\pgfusepath{stroke}%
\end{pgfscope}%
\begin{pgfscope}%
\pgfpathrectangle{\pgfqpoint{1.250000in}{4.155455in}}{\pgfqpoint{2.279412in}{2.004545in}}%
\pgfusepath{clip}%
\pgfsetbuttcap%
\pgfsetroundjoin%
\pgfsetlinewidth{0.321343pt}%
\definecolor{currentstroke}{rgb}{0.269944,0.014625,0.341379}%
\pgfsetstrokecolor{currentstroke}%
\pgfsetdash{}{0pt}%
\pgfpathmoveto{\pgfqpoint{3.211981in}{5.204770in}}%
\pgfpathlineto{\pgfqpoint{3.161848in}{5.205205in}}%
\pgfusepath{stroke}%
\end{pgfscope}%
\begin{pgfscope}%
\pgfpathrectangle{\pgfqpoint{1.250000in}{4.155455in}}{\pgfqpoint{2.279412in}{2.004545in}}%
\pgfusepath{clip}%
\pgfsetbuttcap%
\pgfsetroundjoin%
\pgfsetlinewidth{0.337568pt}%
\definecolor{currentstroke}{rgb}{0.273809,0.031497,0.358853}%
\pgfsetstrokecolor{currentstroke}%
\pgfsetdash{}{0pt}%
\pgfpathmoveto{\pgfqpoint{3.161848in}{5.205205in}}%
\pgfpathlineto{\pgfqpoint{3.111722in}{5.205900in}}%
\pgfusepath{stroke}%
\end{pgfscope}%
\begin{pgfscope}%
\pgfpathrectangle{\pgfqpoint{1.250000in}{4.155455in}}{\pgfqpoint{2.279412in}{2.004545in}}%
\pgfusepath{clip}%
\pgfsetbuttcap%
\pgfsetroundjoin%
\pgfsetlinewidth{0.342734pt}%
\definecolor{currentstroke}{rgb}{0.274952,0.037752,0.364543}%
\pgfsetstrokecolor{currentstroke}%
\pgfsetdash{}{0pt}%
\pgfpathmoveto{\pgfqpoint{3.111722in}{5.205900in}}%
\pgfpathlineto{\pgfqpoint{3.061586in}{5.206522in}}%
\pgfusepath{stroke}%
\end{pgfscope}%
\begin{pgfscope}%
\pgfpathrectangle{\pgfqpoint{1.250000in}{4.155455in}}{\pgfqpoint{2.279412in}{2.004545in}}%
\pgfusepath{clip}%
\pgfsetbuttcap%
\pgfsetroundjoin%
\pgfsetlinewidth{0.351816pt}%
\definecolor{currentstroke}{rgb}{0.276022,0.044167,0.370164}%
\pgfsetstrokecolor{currentstroke}%
\pgfsetdash{}{0pt}%
\pgfpathmoveto{\pgfqpoint{3.061586in}{5.206522in}}%
\pgfpathlineto{\pgfqpoint{3.011436in}{5.206568in}}%
\pgfusepath{stroke}%
\end{pgfscope}%
\begin{pgfscope}%
\pgfpathrectangle{\pgfqpoint{1.250000in}{4.155455in}}{\pgfqpoint{2.279412in}{2.004545in}}%
\pgfusepath{clip}%
\pgfsetbuttcap%
\pgfsetroundjoin%
\pgfsetlinewidth{0.378744pt}%
\definecolor{currentstroke}{rgb}{0.279566,0.067836,0.391917}%
\pgfsetstrokecolor{currentstroke}%
\pgfsetdash{}{0pt}%
\pgfpathmoveto{\pgfqpoint{3.011436in}{5.206568in}}%
\pgfpathlineto{\pgfqpoint{2.961288in}{5.206340in}}%
\pgfusepath{stroke}%
\end{pgfscope}%
\begin{pgfscope}%
\pgfpathrectangle{\pgfqpoint{1.250000in}{4.155455in}}{\pgfqpoint{2.279412in}{2.004545in}}%
\pgfusepath{clip}%
\pgfsetbuttcap%
\pgfsetroundjoin%
\pgfsetlinewidth{0.412531pt}%
\definecolor{currentstroke}{rgb}{0.282327,0.094955,0.417331}%
\pgfsetstrokecolor{currentstroke}%
\pgfsetdash{}{0pt}%
\pgfpathmoveto{\pgfqpoint{2.961288in}{5.206340in}}%
\pgfpathlineto{\pgfqpoint{2.911140in}{5.205876in}}%
\pgfusepath{stroke}%
\end{pgfscope}%
\begin{pgfscope}%
\pgfpathrectangle{\pgfqpoint{1.250000in}{4.155455in}}{\pgfqpoint{2.279412in}{2.004545in}}%
\pgfusepath{clip}%
\pgfsetbuttcap%
\pgfsetroundjoin%
\pgfsetlinewidth{0.452040pt}%
\definecolor{currentstroke}{rgb}{0.283229,0.120777,0.440584}%
\pgfsetstrokecolor{currentstroke}%
\pgfsetdash{}{0pt}%
\pgfpathmoveto{\pgfqpoint{2.911140in}{5.205876in}}%
\pgfpathlineto{\pgfqpoint{2.860991in}{5.205378in}}%
\pgfusepath{stroke}%
\end{pgfscope}%
\begin{pgfscope}%
\pgfpathrectangle{\pgfqpoint{1.250000in}{4.155455in}}{\pgfqpoint{2.279412in}{2.004545in}}%
\pgfusepath{clip}%
\pgfsetbuttcap%
\pgfsetroundjoin%
\pgfsetlinewidth{0.509855pt}%
\definecolor{currentstroke}{rgb}{0.280255,0.165693,0.476498}%
\pgfsetstrokecolor{currentstroke}%
\pgfsetdash{}{0pt}%
\pgfpathmoveto{\pgfqpoint{2.860991in}{5.205378in}}%
\pgfpathlineto{\pgfqpoint{2.810844in}{5.204817in}}%
\pgfusepath{stroke}%
\end{pgfscope}%
\begin{pgfscope}%
\pgfpathrectangle{\pgfqpoint{1.250000in}{4.155455in}}{\pgfqpoint{2.279412in}{2.004545in}}%
\pgfusepath{clip}%
\pgfsetbuttcap%
\pgfsetroundjoin%
\pgfsetlinewidth{0.570096pt}%
\definecolor{currentstroke}{rgb}{0.271828,0.209303,0.504434}%
\pgfsetstrokecolor{currentstroke}%
\pgfsetdash{}{0pt}%
\pgfpathmoveto{\pgfqpoint{2.810844in}{5.204817in}}%
\pgfpathlineto{\pgfqpoint{2.760698in}{5.204200in}}%
\pgfusepath{stroke}%
\end{pgfscope}%
\begin{pgfscope}%
\pgfpathrectangle{\pgfqpoint{1.250000in}{4.155455in}}{\pgfqpoint{2.279412in}{2.004545in}}%
\pgfusepath{clip}%
\pgfsetbuttcap%
\pgfsetroundjoin%
\pgfsetlinewidth{0.675813pt}%
\definecolor{currentstroke}{rgb}{0.248629,0.278775,0.534556}%
\pgfsetstrokecolor{currentstroke}%
\pgfsetdash{}{0pt}%
\pgfpathmoveto{\pgfqpoint{2.760698in}{5.204200in}}%
\pgfpathlineto{\pgfqpoint{2.710552in}{5.203557in}}%
\pgfusepath{stroke}%
\end{pgfscope}%
\begin{pgfscope}%
\pgfpathrectangle{\pgfqpoint{1.250000in}{4.155455in}}{\pgfqpoint{2.279412in}{2.004545in}}%
\pgfusepath{clip}%
\pgfsetbuttcap%
\pgfsetroundjoin%
\pgfsetlinewidth{0.737203pt}%
\definecolor{currentstroke}{rgb}{0.231674,0.318106,0.544834}%
\pgfsetstrokecolor{currentstroke}%
\pgfsetdash{}{0pt}%
\pgfpathmoveto{\pgfqpoint{2.710552in}{5.203557in}}%
\pgfpathlineto{\pgfqpoint{2.660407in}{5.202860in}}%
\pgfusepath{stroke}%
\end{pgfscope}%
\begin{pgfscope}%
\pgfpathrectangle{\pgfqpoint{1.250000in}{4.155455in}}{\pgfqpoint{2.279412in}{2.004545in}}%
\pgfusepath{clip}%
\pgfsetbuttcap%
\pgfsetroundjoin%
\pgfsetlinewidth{0.818811pt}%
\definecolor{currentstroke}{rgb}{0.208623,0.367752,0.552675}%
\pgfsetstrokecolor{currentstroke}%
\pgfsetdash{}{0pt}%
\pgfpathmoveto{\pgfqpoint{2.660407in}{5.202860in}}%
\pgfpathlineto{\pgfqpoint{2.610265in}{5.202001in}}%
\pgfusepath{stroke}%
\end{pgfscope}%
\begin{pgfscope}%
\pgfpathrectangle{\pgfqpoint{1.250000in}{4.155455in}}{\pgfqpoint{2.279412in}{2.004545in}}%
\pgfusepath{clip}%
\pgfsetbuttcap%
\pgfsetroundjoin%
\pgfsetlinewidth{0.863410pt}%
\definecolor{currentstroke}{rgb}{0.197636,0.391528,0.554969}%
\pgfsetstrokecolor{currentstroke}%
\pgfsetdash{}{0pt}%
\pgfpathmoveto{\pgfqpoint{2.610265in}{5.202001in}}%
\pgfpathlineto{\pgfqpoint{2.560127in}{5.200998in}}%
\pgfusepath{stroke}%
\end{pgfscope}%
\begin{pgfscope}%
\pgfpathrectangle{\pgfqpoint{1.250000in}{4.155455in}}{\pgfqpoint{2.279412in}{2.004545in}}%
\pgfusepath{clip}%
\pgfsetbuttcap%
\pgfsetroundjoin%
\pgfsetlinewidth{0.904552pt}%
\definecolor{currentstroke}{rgb}{0.187231,0.414746,0.556547}%
\pgfsetstrokecolor{currentstroke}%
\pgfsetdash{}{0pt}%
\pgfpathmoveto{\pgfqpoint{2.560127in}{5.200998in}}%
\pgfpathlineto{\pgfqpoint{2.509994in}{5.199818in}}%
\pgfusepath{stroke}%
\end{pgfscope}%
\begin{pgfscope}%
\pgfpathrectangle{\pgfqpoint{1.250000in}{4.155455in}}{\pgfqpoint{2.279412in}{2.004545in}}%
\pgfusepath{clip}%
\pgfsetbuttcap%
\pgfsetroundjoin%
\pgfsetlinewidth{0.955884pt}%
\definecolor{currentstroke}{rgb}{0.174274,0.445044,0.557792}%
\pgfsetstrokecolor{currentstroke}%
\pgfsetdash{}{0pt}%
\pgfpathmoveto{\pgfqpoint{2.509994in}{5.199818in}}%
\pgfpathlineto{\pgfqpoint{2.459868in}{5.198403in}}%
\pgfusepath{stroke}%
\end{pgfscope}%
\begin{pgfscope}%
\pgfpathrectangle{\pgfqpoint{1.250000in}{4.155455in}}{\pgfqpoint{2.279412in}{2.004545in}}%
\pgfusepath{clip}%
\pgfsetbuttcap%
\pgfsetroundjoin%
\pgfsetlinewidth{0.933120pt}%
\definecolor{currentstroke}{rgb}{0.179019,0.433756,0.557430}%
\pgfsetstrokecolor{currentstroke}%
\pgfsetdash{}{0pt}%
\pgfpathmoveto{\pgfqpoint{2.459868in}{5.198403in}}%
\pgfpathlineto{\pgfqpoint{2.409752in}{5.196752in}}%
\pgfusepath{stroke}%
\end{pgfscope}%
\begin{pgfscope}%
\pgfpathrectangle{\pgfqpoint{1.250000in}{4.155455in}}{\pgfqpoint{2.279412in}{2.004545in}}%
\pgfusepath{clip}%
\pgfsetbuttcap%
\pgfsetroundjoin%
\pgfsetlinewidth{0.891611pt}%
\definecolor{currentstroke}{rgb}{0.188923,0.410910,0.556326}%
\pgfsetstrokecolor{currentstroke}%
\pgfsetdash{}{0pt}%
\pgfpathmoveto{\pgfqpoint{2.409752in}{5.196752in}}%
\pgfpathlineto{\pgfqpoint{2.359648in}{5.194853in}}%
\pgfusepath{stroke}%
\end{pgfscope}%
\begin{pgfscope}%
\pgfpathrectangle{\pgfqpoint{1.250000in}{4.155455in}}{\pgfqpoint{2.279412in}{2.004545in}}%
\pgfusepath{clip}%
\pgfsetbuttcap%
\pgfsetroundjoin%
\pgfsetlinewidth{0.842831pt}%
\definecolor{currentstroke}{rgb}{0.203063,0.379716,0.553925}%
\pgfsetstrokecolor{currentstroke}%
\pgfsetdash{}{0pt}%
\pgfpathmoveto{\pgfqpoint{2.359648in}{5.194853in}}%
\pgfpathlineto{\pgfqpoint{2.309569in}{5.192578in}}%
\pgfusepath{stroke}%
\end{pgfscope}%
\begin{pgfscope}%
\pgfpathrectangle{\pgfqpoint{1.250000in}{4.155455in}}{\pgfqpoint{2.279412in}{2.004545in}}%
\pgfusepath{clip}%
\pgfsetbuttcap%
\pgfsetroundjoin%
\pgfsetlinewidth{0.778124pt}%
\definecolor{currentstroke}{rgb}{0.220057,0.343307,0.549413}%
\pgfsetstrokecolor{currentstroke}%
\pgfsetdash{}{0pt}%
\pgfpathmoveto{\pgfqpoint{2.309569in}{5.192578in}}%
\pgfpathlineto{\pgfqpoint{2.259559in}{5.189401in}}%
\pgfusepath{stroke}%
\end{pgfscope}%
\begin{pgfscope}%
\pgfpathrectangle{\pgfqpoint{1.250000in}{4.155455in}}{\pgfqpoint{2.279412in}{2.004545in}}%
\pgfusepath{clip}%
\pgfsetbuttcap%
\pgfsetroundjoin%
\pgfsetlinewidth{0.673240pt}%
\definecolor{currentstroke}{rgb}{0.248629,0.278775,0.534556}%
\pgfsetstrokecolor{currentstroke}%
\pgfsetdash{}{0pt}%
\pgfpathmoveto{\pgfqpoint{2.259559in}{5.189401in}}%
\pgfpathlineto{\pgfqpoint{2.209740in}{5.184519in}}%
\pgfusepath{stroke}%
\end{pgfscope}%
\begin{pgfscope}%
\pgfpathrectangle{\pgfqpoint{1.250000in}{4.155455in}}{\pgfqpoint{2.279412in}{2.004545in}}%
\pgfusepath{clip}%
\pgfsetbuttcap%
\pgfsetroundjoin%
\pgfsetlinewidth{0.531114pt}%
\definecolor{currentstroke}{rgb}{0.278012,0.180367,0.486697}%
\pgfsetstrokecolor{currentstroke}%
\pgfsetdash{}{0pt}%
\pgfpathmoveto{\pgfqpoint{2.209740in}{5.184519in}}%
\pgfpathlineto{\pgfqpoint{2.209740in}{5.184519in}}%
\pgfusepath{stroke}%
\end{pgfscope}%
\begin{pgfscope}%
\pgfpathrectangle{\pgfqpoint{1.250000in}{4.155455in}}{\pgfqpoint{2.279412in}{2.004545in}}%
\pgfusepath{clip}%
\pgfsetbuttcap%
\pgfsetroundjoin%
\pgfsetlinewidth{0.316003pt}%
\definecolor{currentstroke}{rgb}{0.269944,0.014625,0.341379}%
\pgfsetstrokecolor{currentstroke}%
\pgfsetdash{}{0pt}%
\pgfpathmoveto{\pgfqpoint{3.308228in}{5.297635in}}%
\pgfpathlineto{\pgfqpoint{3.261668in}{5.293048in}}%
\pgfusepath{stroke}%
\end{pgfscope}%
\begin{pgfscope}%
\pgfpathrectangle{\pgfqpoint{1.250000in}{4.155455in}}{\pgfqpoint{2.279412in}{2.004545in}}%
\pgfusepath{clip}%
\pgfsetbuttcap%
\pgfsetroundjoin%
\pgfsetlinewidth{0.307472pt}%
\definecolor{currentstroke}{rgb}{0.267004,0.004874,0.329415}%
\pgfsetstrokecolor{currentstroke}%
\pgfsetdash{}{0pt}%
\pgfpathmoveto{\pgfqpoint{3.261668in}{5.293048in}}%
\pgfpathlineto{\pgfqpoint{3.212910in}{5.289157in}}%
\pgfusepath{stroke}%
\end{pgfscope}%
\begin{pgfscope}%
\pgfpathrectangle{\pgfqpoint{1.250000in}{4.155455in}}{\pgfqpoint{2.279412in}{2.004545in}}%
\pgfusepath{clip}%
\pgfsetbuttcap%
\pgfsetroundjoin%
\pgfsetlinewidth{0.322418pt}%
\definecolor{currentstroke}{rgb}{0.271305,0.019942,0.347269}%
\pgfsetstrokecolor{currentstroke}%
\pgfsetdash{}{0pt}%
\pgfpathmoveto{\pgfqpoint{3.212910in}{5.289157in}}%
\pgfpathlineto{\pgfqpoint{3.164108in}{5.288957in}}%
\pgfusepath{stroke}%
\end{pgfscope}%
\begin{pgfscope}%
\pgfpathrectangle{\pgfqpoint{1.250000in}{4.155455in}}{\pgfqpoint{2.279412in}{2.004545in}}%
\pgfusepath{clip}%
\pgfsetbuttcap%
\pgfsetroundjoin%
\pgfsetlinewidth{0.322336pt}%
\definecolor{currentstroke}{rgb}{0.271305,0.019942,0.347269}%
\pgfsetstrokecolor{currentstroke}%
\pgfsetdash{}{0pt}%
\pgfpathmoveto{\pgfqpoint{3.164108in}{5.288957in}}%
\pgfpathlineto{\pgfqpoint{3.114000in}{5.288004in}}%
\pgfusepath{stroke}%
\end{pgfscope}%
\begin{pgfscope}%
\pgfpathrectangle{\pgfqpoint{1.250000in}{4.155455in}}{\pgfqpoint{2.279412in}{2.004545in}}%
\pgfusepath{clip}%
\pgfsetbuttcap%
\pgfsetroundjoin%
\pgfsetlinewidth{0.337190pt}%
\definecolor{currentstroke}{rgb}{0.273809,0.031497,0.358853}%
\pgfsetstrokecolor{currentstroke}%
\pgfsetdash{}{0pt}%
\pgfpathmoveto{\pgfqpoint{3.114000in}{5.288004in}}%
\pgfpathlineto{\pgfqpoint{3.063867in}{5.287036in}}%
\pgfusepath{stroke}%
\end{pgfscope}%
\begin{pgfscope}%
\pgfpathrectangle{\pgfqpoint{1.250000in}{4.155455in}}{\pgfqpoint{2.279412in}{2.004545in}}%
\pgfusepath{clip}%
\pgfsetbuttcap%
\pgfsetroundjoin%
\pgfsetlinewidth{0.348377pt}%
\definecolor{currentstroke}{rgb}{0.274952,0.037752,0.364543}%
\pgfsetstrokecolor{currentstroke}%
\pgfsetdash{}{0pt}%
\pgfpathmoveto{\pgfqpoint{3.063867in}{5.287036in}}%
\pgfpathlineto{\pgfqpoint{3.013725in}{5.286268in}}%
\pgfusepath{stroke}%
\end{pgfscope}%
\begin{pgfscope}%
\pgfpathrectangle{\pgfqpoint{1.250000in}{4.155455in}}{\pgfqpoint{2.279412in}{2.004545in}}%
\pgfusepath{clip}%
\pgfsetbuttcap%
\pgfsetroundjoin%
\pgfsetlinewidth{0.371419pt}%
\definecolor{currentstroke}{rgb}{0.278791,0.062145,0.386592}%
\pgfsetstrokecolor{currentstroke}%
\pgfsetdash{}{0pt}%
\pgfpathmoveto{\pgfqpoint{3.013725in}{5.286268in}}%
\pgfpathlineto{\pgfqpoint{2.963581in}{5.285592in}}%
\pgfusepath{stroke}%
\end{pgfscope}%
\begin{pgfscope}%
\pgfpathrectangle{\pgfqpoint{1.250000in}{4.155455in}}{\pgfqpoint{2.279412in}{2.004545in}}%
\pgfusepath{clip}%
\pgfsetbuttcap%
\pgfsetroundjoin%
\pgfsetlinewidth{0.396002pt}%
\definecolor{currentstroke}{rgb}{0.280894,0.078907,0.402329}%
\pgfsetstrokecolor{currentstroke}%
\pgfsetdash{}{0pt}%
\pgfpathmoveto{\pgfqpoint{2.963581in}{5.285592in}}%
\pgfpathlineto{\pgfqpoint{2.913438in}{5.284812in}}%
\pgfusepath{stroke}%
\end{pgfscope}%
\begin{pgfscope}%
\pgfpathrectangle{\pgfqpoint{1.250000in}{4.155455in}}{\pgfqpoint{2.279412in}{2.004545in}}%
\pgfusepath{clip}%
\pgfsetbuttcap%
\pgfsetroundjoin%
\pgfsetlinewidth{0.448721pt}%
\definecolor{currentstroke}{rgb}{0.283229,0.120777,0.440584}%
\pgfsetstrokecolor{currentstroke}%
\pgfsetdash{}{0pt}%
\pgfpathmoveto{\pgfqpoint{2.913438in}{5.284812in}}%
\pgfpathlineto{\pgfqpoint{2.863296in}{5.283955in}}%
\pgfusepath{stroke}%
\end{pgfscope}%
\begin{pgfscope}%
\pgfpathrectangle{\pgfqpoint{1.250000in}{4.155455in}}{\pgfqpoint{2.279412in}{2.004545in}}%
\pgfusepath{clip}%
\pgfsetbuttcap%
\pgfsetroundjoin%
\pgfsetlinewidth{0.496969pt}%
\definecolor{currentstroke}{rgb}{0.281412,0.155834,0.469201}%
\pgfsetstrokecolor{currentstroke}%
\pgfsetdash{}{0pt}%
\pgfpathmoveto{\pgfqpoint{2.863296in}{5.283955in}}%
\pgfpathlineto{\pgfqpoint{2.813157in}{5.282993in}}%
\pgfusepath{stroke}%
\end{pgfscope}%
\begin{pgfscope}%
\pgfpathrectangle{\pgfqpoint{1.250000in}{4.155455in}}{\pgfqpoint{2.279412in}{2.004545in}}%
\pgfusepath{clip}%
\pgfsetbuttcap%
\pgfsetroundjoin%
\pgfsetlinewidth{0.551764pt}%
\definecolor{currentstroke}{rgb}{0.275191,0.194905,0.496005}%
\pgfsetstrokecolor{currentstroke}%
\pgfsetdash{}{0pt}%
\pgfpathmoveto{\pgfqpoint{2.813157in}{5.282993in}}%
\pgfpathlineto{\pgfqpoint{2.763028in}{5.281668in}}%
\pgfusepath{stroke}%
\end{pgfscope}%
\begin{pgfscope}%
\pgfpathrectangle{\pgfqpoint{1.250000in}{4.155455in}}{\pgfqpoint{2.279412in}{2.004545in}}%
\pgfusepath{clip}%
\pgfsetbuttcap%
\pgfsetroundjoin%
\pgfsetlinewidth{0.602633pt}%
\definecolor{currentstroke}{rgb}{0.266580,0.228262,0.514349}%
\pgfsetstrokecolor{currentstroke}%
\pgfsetdash{}{0pt}%
\pgfpathmoveto{\pgfqpoint{2.763028in}{5.281668in}}%
\pgfpathlineto{\pgfqpoint{2.712908in}{5.280097in}}%
\pgfusepath{stroke}%
\end{pgfscope}%
\begin{pgfscope}%
\pgfpathrectangle{\pgfqpoint{1.250000in}{4.155455in}}{\pgfqpoint{2.279412in}{2.004545in}}%
\pgfusepath{clip}%
\pgfsetbuttcap%
\pgfsetroundjoin%
\pgfsetlinewidth{0.691756pt}%
\definecolor{currentstroke}{rgb}{0.244972,0.287675,0.537260}%
\pgfsetstrokecolor{currentstroke}%
\pgfsetdash{}{0pt}%
\pgfpathmoveto{\pgfqpoint{2.712908in}{5.280097in}}%
\pgfpathlineto{\pgfqpoint{2.662797in}{5.278326in}}%
\pgfusepath{stroke}%
\end{pgfscope}%
\begin{pgfscope}%
\pgfpathrectangle{\pgfqpoint{1.250000in}{4.155455in}}{\pgfqpoint{2.279412in}{2.004545in}}%
\pgfusepath{clip}%
\pgfsetbuttcap%
\pgfsetroundjoin%
\pgfsetlinewidth{0.753269pt}%
\definecolor{currentstroke}{rgb}{0.227802,0.326594,0.546532}%
\pgfsetstrokecolor{currentstroke}%
\pgfsetdash{}{0pt}%
\pgfpathmoveto{\pgfqpoint{2.662797in}{5.278326in}}%
\pgfpathlineto{\pgfqpoint{2.612698in}{5.276308in}}%
\pgfusepath{stroke}%
\end{pgfscope}%
\begin{pgfscope}%
\pgfpathrectangle{\pgfqpoint{1.250000in}{4.155455in}}{\pgfqpoint{2.279412in}{2.004545in}}%
\pgfusepath{clip}%
\pgfsetbuttcap%
\pgfsetroundjoin%
\pgfsetlinewidth{0.837951pt}%
\definecolor{currentstroke}{rgb}{0.203063,0.379716,0.553925}%
\pgfsetstrokecolor{currentstroke}%
\pgfsetdash{}{0pt}%
\pgfpathmoveto{\pgfqpoint{2.612698in}{5.276308in}}%
\pgfpathlineto{\pgfqpoint{2.562614in}{5.274027in}}%
\pgfusepath{stroke}%
\end{pgfscope}%
\begin{pgfscope}%
\pgfpathrectangle{\pgfqpoint{1.250000in}{4.155455in}}{\pgfqpoint{2.279412in}{2.004545in}}%
\pgfusepath{clip}%
\pgfsetbuttcap%
\pgfsetroundjoin%
\pgfsetlinewidth{0.873015pt}%
\definecolor{currentstroke}{rgb}{0.194100,0.399323,0.555565}%
\pgfsetstrokecolor{currentstroke}%
\pgfsetdash{}{0pt}%
\pgfpathmoveto{\pgfqpoint{2.562614in}{5.274027in}}%
\pgfpathlineto{\pgfqpoint{2.512549in}{5.271433in}}%
\pgfusepath{stroke}%
\end{pgfscope}%
\begin{pgfscope}%
\pgfpathrectangle{\pgfqpoint{1.250000in}{4.155455in}}{\pgfqpoint{2.279412in}{2.004545in}}%
\pgfusepath{clip}%
\pgfsetbuttcap%
\pgfsetroundjoin%
\pgfsetlinewidth{0.888508pt}%
\definecolor{currentstroke}{rgb}{0.190631,0.407061,0.556089}%
\pgfsetstrokecolor{currentstroke}%
\pgfsetdash{}{0pt}%
\pgfpathmoveto{\pgfqpoint{2.512549in}{5.271433in}}%
\pgfpathlineto{\pgfqpoint{2.462511in}{5.268482in}}%
\pgfusepath{stroke}%
\end{pgfscope}%
\begin{pgfscope}%
\pgfpathrectangle{\pgfqpoint{1.250000in}{4.155455in}}{\pgfqpoint{2.279412in}{2.004545in}}%
\pgfusepath{clip}%
\pgfsetbuttcap%
\pgfsetroundjoin%
\pgfsetlinewidth{0.887245pt}%
\definecolor{currentstroke}{rgb}{0.190631,0.407061,0.556089}%
\pgfsetstrokecolor{currentstroke}%
\pgfsetdash{}{0pt}%
\pgfpathmoveto{\pgfqpoint{2.462511in}{5.268482in}}%
\pgfpathlineto{\pgfqpoint{2.412517in}{5.265022in}}%
\pgfusepath{stroke}%
\end{pgfscope}%
\begin{pgfscope}%
\pgfpathrectangle{\pgfqpoint{1.250000in}{4.155455in}}{\pgfqpoint{2.279412in}{2.004545in}}%
\pgfusepath{clip}%
\pgfsetbuttcap%
\pgfsetroundjoin%
\pgfsetlinewidth{0.843980pt}%
\definecolor{currentstroke}{rgb}{0.201239,0.383670,0.554294}%
\pgfsetstrokecolor{currentstroke}%
\pgfsetdash{}{0pt}%
\pgfpathmoveto{\pgfqpoint{2.412517in}{5.265022in}}%
\pgfpathlineto{\pgfqpoint{2.362584in}{5.260948in}}%
\pgfusepath{stroke}%
\end{pgfscope}%
\begin{pgfscope}%
\pgfpathrectangle{\pgfqpoint{1.250000in}{4.155455in}}{\pgfqpoint{2.279412in}{2.004545in}}%
\pgfusepath{clip}%
\pgfsetbuttcap%
\pgfsetroundjoin%
\pgfsetlinewidth{0.841402pt}%
\definecolor{currentstroke}{rgb}{0.203063,0.379716,0.553925}%
\pgfsetstrokecolor{currentstroke}%
\pgfsetdash{}{0pt}%
\pgfpathmoveto{\pgfqpoint{2.362584in}{5.260948in}}%
\pgfpathlineto{\pgfqpoint{2.312763in}{5.255954in}}%
\pgfusepath{stroke}%
\end{pgfscope}%
\begin{pgfscope}%
\pgfpathrectangle{\pgfqpoint{1.250000in}{4.155455in}}{\pgfqpoint{2.279412in}{2.004545in}}%
\pgfusepath{clip}%
\pgfsetbuttcap%
\pgfsetroundjoin%
\pgfsetlinewidth{0.750001pt}%
\definecolor{currentstroke}{rgb}{0.227802,0.326594,0.546532}%
\pgfsetstrokecolor{currentstroke}%
\pgfsetdash{}{0pt}%
\pgfpathmoveto{\pgfqpoint{2.312763in}{5.255954in}}%
\pgfpathlineto{\pgfqpoint{2.263261in}{5.249049in}}%
\pgfusepath{stroke}%
\end{pgfscope}%
\begin{pgfscope}%
\pgfpathrectangle{\pgfqpoint{1.250000in}{4.155455in}}{\pgfqpoint{2.279412in}{2.004545in}}%
\pgfusepath{clip}%
\pgfsetbuttcap%
\pgfsetroundjoin%
\pgfsetlinewidth{0.636592pt}%
\definecolor{currentstroke}{rgb}{0.258965,0.251537,0.524736}%
\pgfsetstrokecolor{currentstroke}%
\pgfsetdash{}{0pt}%
\pgfpathmoveto{\pgfqpoint{2.263261in}{5.249049in}}%
\pgfpathlineto{\pgfqpoint{2.214557in}{5.238805in}}%
\pgfusepath{stroke}%
\end{pgfscope}%
\begin{pgfscope}%
\pgfpathrectangle{\pgfqpoint{1.250000in}{4.155455in}}{\pgfqpoint{2.279412in}{2.004545in}}%
\pgfusepath{clip}%
\pgfsetbuttcap%
\pgfsetroundjoin%
\pgfsetlinewidth{0.553080pt}%
\definecolor{currentstroke}{rgb}{0.275191,0.194905,0.496005}%
\pgfsetstrokecolor{currentstroke}%
\pgfsetdash{}{0pt}%
\pgfpathmoveto{\pgfqpoint{2.214557in}{5.238805in}}%
\pgfpathlineto{\pgfqpoint{2.214557in}{5.238805in}}%
\pgfusepath{stroke}%
\end{pgfscope}%
\begin{pgfscope}%
\pgfpathrectangle{\pgfqpoint{1.250000in}{4.155455in}}{\pgfqpoint{2.279412in}{2.004545in}}%
\pgfusepath{clip}%
\pgfsetbuttcap%
\pgfsetroundjoin%
\pgfsetlinewidth{0.553080pt}%
\definecolor{currentstroke}{rgb}{0.275191,0.194905,0.496005}%
\pgfsetstrokecolor{currentstroke}%
\pgfsetdash{}{0pt}%
\pgfpathmoveto{\pgfqpoint{2.214557in}{5.238805in}}%
\pgfpathlineto{\pgfqpoint{2.183145in}{5.228287in}}%
\pgfusepath{stroke}%
\end{pgfscope}%
\begin{pgfscope}%
\pgfpathrectangle{\pgfqpoint{1.250000in}{4.155455in}}{\pgfqpoint{2.279412in}{2.004545in}}%
\pgfusepath{clip}%
\pgfsetbuttcap%
\pgfsetroundjoin%
\pgfsetlinewidth{0.504134pt}%
\definecolor{currentstroke}{rgb}{0.280868,0.160771,0.472899}%
\pgfsetstrokecolor{currentstroke}%
\pgfsetdash{}{0pt}%
\pgfpathmoveto{\pgfqpoint{2.183145in}{5.228287in}}%
\pgfpathlineto{\pgfqpoint{2.183145in}{5.228287in}}%
\pgfusepath{stroke}%
\end{pgfscope}%
\begin{pgfscope}%
\pgfpathrectangle{\pgfqpoint{1.250000in}{4.155455in}}{\pgfqpoint{2.279412in}{2.004545in}}%
\pgfusepath{clip}%
\pgfsetbuttcap%
\pgfsetroundjoin%
\pgfsetlinewidth{0.314519pt}%
\definecolor{currentstroke}{rgb}{0.268510,0.009605,0.335427}%
\pgfsetstrokecolor{currentstroke}%
\pgfsetdash{}{0pt}%
\pgfpathmoveto{\pgfqpoint{3.285241in}{5.338870in}}%
\pgfpathlineto{\pgfqpoint{3.261668in}{5.338154in}}%
\pgfusepath{stroke}%
\end{pgfscope}%
\begin{pgfscope}%
\pgfpathrectangle{\pgfqpoint{1.250000in}{4.155455in}}{\pgfqpoint{2.279412in}{2.004545in}}%
\pgfusepath{clip}%
\pgfsetbuttcap%
\pgfsetroundjoin%
\pgfsetlinewidth{0.311532pt}%
\definecolor{currentstroke}{rgb}{0.268510,0.009605,0.335427}%
\pgfsetstrokecolor{currentstroke}%
\pgfsetdash{}{0pt}%
\pgfpathmoveto{\pgfqpoint{3.261668in}{5.338154in}}%
\pgfpathlineto{\pgfqpoint{3.261668in}{5.338154in}}%
\pgfusepath{stroke}%
\end{pgfscope}%
\begin{pgfscope}%
\pgfpathrectangle{\pgfqpoint{1.250000in}{4.155455in}}{\pgfqpoint{2.279412in}{2.004545in}}%
\pgfusepath{clip}%
\pgfsetbuttcap%
\pgfsetroundjoin%
\pgfsetlinewidth{0.311532pt}%
\definecolor{currentstroke}{rgb}{0.268510,0.009605,0.335427}%
\pgfsetstrokecolor{currentstroke}%
\pgfsetdash{}{0pt}%
\pgfpathmoveto{\pgfqpoint{3.261668in}{5.338154in}}%
\pgfpathlineto{\pgfqpoint{3.261668in}{5.338154in}}%
\pgfusepath{stroke}%
\end{pgfscope}%
\begin{pgfscope}%
\pgfpathrectangle{\pgfqpoint{1.250000in}{4.155455in}}{\pgfqpoint{2.279412in}{2.004545in}}%
\pgfusepath{clip}%
\pgfsetbuttcap%
\pgfsetroundjoin%
\pgfsetlinewidth{0.311532pt}%
\definecolor{currentstroke}{rgb}{0.268510,0.009605,0.335427}%
\pgfsetstrokecolor{currentstroke}%
\pgfsetdash{}{0pt}%
\pgfpathmoveto{\pgfqpoint{3.261668in}{5.338154in}}%
\pgfpathlineto{\pgfqpoint{3.226608in}{5.336542in}}%
\pgfusepath{stroke}%
\end{pgfscope}%
\begin{pgfscope}%
\pgfpathrectangle{\pgfqpoint{1.250000in}{4.155455in}}{\pgfqpoint{2.279412in}{2.004545in}}%
\pgfusepath{clip}%
\pgfsetbuttcap%
\pgfsetroundjoin%
\pgfsetlinewidth{0.315923pt}%
\definecolor{currentstroke}{rgb}{0.269944,0.014625,0.341379}%
\pgfsetstrokecolor{currentstroke}%
\pgfsetdash{}{0pt}%
\pgfpathmoveto{\pgfqpoint{3.226608in}{5.336542in}}%
\pgfpathlineto{\pgfqpoint{3.177907in}{5.335774in}}%
\pgfusepath{stroke}%
\end{pgfscope}%
\begin{pgfscope}%
\pgfpathrectangle{\pgfqpoint{1.250000in}{4.155455in}}{\pgfqpoint{2.279412in}{2.004545in}}%
\pgfusepath{clip}%
\pgfsetbuttcap%
\pgfsetroundjoin%
\pgfsetlinewidth{0.329728pt}%
\definecolor{currentstroke}{rgb}{0.272594,0.025563,0.353093}%
\pgfsetstrokecolor{currentstroke}%
\pgfsetdash{}{0pt}%
\pgfpathmoveto{\pgfqpoint{3.177907in}{5.335774in}}%
\pgfpathlineto{\pgfqpoint{3.127854in}{5.333289in}}%
\pgfusepath{stroke}%
\end{pgfscope}%
\begin{pgfscope}%
\pgfpathrectangle{\pgfqpoint{1.250000in}{4.155455in}}{\pgfqpoint{2.279412in}{2.004545in}}%
\pgfusepath{clip}%
\pgfsetbuttcap%
\pgfsetroundjoin%
\pgfsetlinewidth{0.326670pt}%
\definecolor{currentstroke}{rgb}{0.271305,0.019942,0.347269}%
\pgfsetstrokecolor{currentstroke}%
\pgfsetdash{}{0pt}%
\pgfpathmoveto{\pgfqpoint{3.127854in}{5.333289in}}%
\pgfpathlineto{\pgfqpoint{3.077830in}{5.330538in}}%
\pgfusepath{stroke}%
\end{pgfscope}%
\begin{pgfscope}%
\pgfpathrectangle{\pgfqpoint{1.250000in}{4.155455in}}{\pgfqpoint{2.279412in}{2.004545in}}%
\pgfusepath{clip}%
\pgfsetbuttcap%
\pgfsetroundjoin%
\pgfsetlinewidth{0.344946pt}%
\definecolor{currentstroke}{rgb}{0.274952,0.037752,0.364543}%
\pgfsetstrokecolor{currentstroke}%
\pgfsetdash{}{0pt}%
\pgfpathmoveto{\pgfqpoint{3.077830in}{5.330538in}}%
\pgfpathlineto{\pgfqpoint{3.027764in}{5.328542in}}%
\pgfusepath{stroke}%
\end{pgfscope}%
\begin{pgfscope}%
\pgfpathrectangle{\pgfqpoint{1.250000in}{4.155455in}}{\pgfqpoint{2.279412in}{2.004545in}}%
\pgfusepath{clip}%
\pgfsetbuttcap%
\pgfsetroundjoin%
\pgfsetlinewidth{0.366837pt}%
\definecolor{currentstroke}{rgb}{0.277941,0.056324,0.381191}%
\pgfsetstrokecolor{currentstroke}%
\pgfsetdash{}{0pt}%
\pgfpathmoveto{\pgfqpoint{3.027764in}{5.328542in}}%
\pgfpathlineto{\pgfqpoint{2.977632in}{5.327326in}}%
\pgfusepath{stroke}%
\end{pgfscope}%
\begin{pgfscope}%
\pgfpathrectangle{\pgfqpoint{1.250000in}{4.155455in}}{\pgfqpoint{2.279412in}{2.004545in}}%
\pgfusepath{clip}%
\pgfsetbuttcap%
\pgfsetroundjoin%
\pgfsetlinewidth{0.391606pt}%
\definecolor{currentstroke}{rgb}{0.280894,0.078907,0.402329}%
\pgfsetstrokecolor{currentstroke}%
\pgfsetdash{}{0pt}%
\pgfpathmoveto{\pgfqpoint{2.977632in}{5.327326in}}%
\pgfpathlineto{\pgfqpoint{2.927500in}{5.326114in}}%
\pgfusepath{stroke}%
\end{pgfscope}%
\begin{pgfscope}%
\pgfpathrectangle{\pgfqpoint{1.250000in}{4.155455in}}{\pgfqpoint{2.279412in}{2.004545in}}%
\pgfusepath{clip}%
\pgfsetbuttcap%
\pgfsetroundjoin%
\pgfsetlinewidth{0.417674pt}%
\definecolor{currentstroke}{rgb}{0.282327,0.094955,0.417331}%
\pgfsetstrokecolor{currentstroke}%
\pgfsetdash{}{0pt}%
\pgfpathmoveto{\pgfqpoint{2.927500in}{5.326114in}}%
\pgfpathlineto{\pgfqpoint{2.877372in}{5.324779in}}%
\pgfusepath{stroke}%
\end{pgfscope}%
\begin{pgfscope}%
\pgfpathrectangle{\pgfqpoint{1.250000in}{4.155455in}}{\pgfqpoint{2.279412in}{2.004545in}}%
\pgfusepath{clip}%
\pgfsetbuttcap%
\pgfsetroundjoin%
\pgfsetlinewidth{0.467020pt}%
\definecolor{currentstroke}{rgb}{0.282884,0.135920,0.453427}%
\pgfsetstrokecolor{currentstroke}%
\pgfsetdash{}{0pt}%
\pgfpathmoveto{\pgfqpoint{2.877372in}{5.324779in}}%
\pgfpathlineto{\pgfqpoint{2.827254in}{5.323185in}}%
\pgfusepath{stroke}%
\end{pgfscope}%
\begin{pgfscope}%
\pgfpathrectangle{\pgfqpoint{1.250000in}{4.155455in}}{\pgfqpoint{2.279412in}{2.004545in}}%
\pgfusepath{clip}%
\pgfsetbuttcap%
\pgfsetroundjoin%
\pgfsetlinewidth{0.517446pt}%
\definecolor{currentstroke}{rgb}{0.279574,0.170599,0.479997}%
\pgfsetstrokecolor{currentstroke}%
\pgfsetdash{}{0pt}%
\pgfpathmoveto{\pgfqpoint{2.827254in}{5.323185in}}%
\pgfpathlineto{\pgfqpoint{2.777146in}{5.321356in}}%
\pgfusepath{stroke}%
\end{pgfscope}%
\begin{pgfscope}%
\pgfpathrectangle{\pgfqpoint{1.250000in}{4.155455in}}{\pgfqpoint{2.279412in}{2.004545in}}%
\pgfusepath{clip}%
\pgfsetbuttcap%
\pgfsetroundjoin%
\pgfsetlinewidth{0.582352pt}%
\definecolor{currentstroke}{rgb}{0.270595,0.214069,0.507052}%
\pgfsetstrokecolor{currentstroke}%
\pgfsetdash{}{0pt}%
\pgfpathmoveto{\pgfqpoint{2.777146in}{5.321356in}}%
\pgfpathlineto{\pgfqpoint{2.727049in}{5.319311in}}%
\pgfusepath{stroke}%
\end{pgfscope}%
\begin{pgfscope}%
\pgfpathrectangle{\pgfqpoint{1.250000in}{4.155455in}}{\pgfqpoint{2.279412in}{2.004545in}}%
\pgfusepath{clip}%
\pgfsetbuttcap%
\pgfsetroundjoin%
\pgfsetlinewidth{0.642484pt}%
\definecolor{currentstroke}{rgb}{0.257322,0.256130,0.526563}%
\pgfsetstrokecolor{currentstroke}%
\pgfsetdash{}{0pt}%
\pgfpathmoveto{\pgfqpoint{2.727049in}{5.319311in}}%
\pgfpathlineto{\pgfqpoint{2.676970in}{5.316949in}}%
\pgfusepath{stroke}%
\end{pgfscope}%
\begin{pgfscope}%
\pgfpathrectangle{\pgfqpoint{1.250000in}{4.155455in}}{\pgfqpoint{2.279412in}{2.004545in}}%
\pgfusepath{clip}%
\pgfsetbuttcap%
\pgfsetroundjoin%
\pgfsetlinewidth{0.324699pt}%
\definecolor{currentstroke}{rgb}{0.271305,0.019942,0.347269}%
\pgfsetstrokecolor{currentstroke}%
\pgfsetdash{}{0pt}%
\pgfpathmoveto{\pgfqpoint{3.210376in}{5.022407in}}%
\pgfpathlineto{\pgfqpoint{3.160255in}{5.023055in}}%
\pgfusepath{stroke}%
\end{pgfscope}%
\begin{pgfscope}%
\pgfpathrectangle{\pgfqpoint{1.250000in}{4.155455in}}{\pgfqpoint{2.279412in}{2.004545in}}%
\pgfusepath{clip}%
\pgfsetbuttcap%
\pgfsetroundjoin%
\pgfsetlinewidth{0.331751pt}%
\definecolor{currentstroke}{rgb}{0.272594,0.025563,0.353093}%
\pgfsetstrokecolor{currentstroke}%
\pgfsetdash{}{0pt}%
\pgfpathmoveto{\pgfqpoint{3.160255in}{5.023055in}}%
\pgfpathlineto{\pgfqpoint{3.110112in}{5.023650in}}%
\pgfusepath{stroke}%
\end{pgfscope}%
\begin{pgfscope}%
\pgfpathrectangle{\pgfqpoint{1.250000in}{4.155455in}}{\pgfqpoint{2.279412in}{2.004545in}}%
\pgfusepath{clip}%
\pgfsetbuttcap%
\pgfsetroundjoin%
\pgfsetlinewidth{0.337042pt}%
\definecolor{currentstroke}{rgb}{0.273809,0.031497,0.358853}%
\pgfsetstrokecolor{currentstroke}%
\pgfsetdash{}{0pt}%
\pgfpathmoveto{\pgfqpoint{3.110112in}{5.023650in}}%
\pgfpathlineto{\pgfqpoint{3.059972in}{5.024378in}}%
\pgfusepath{stroke}%
\end{pgfscope}%
\begin{pgfscope}%
\pgfpathrectangle{\pgfqpoint{1.250000in}{4.155455in}}{\pgfqpoint{2.279412in}{2.004545in}}%
\pgfusepath{clip}%
\pgfsetbuttcap%
\pgfsetroundjoin%
\pgfsetlinewidth{0.357037pt}%
\definecolor{currentstroke}{rgb}{0.277018,0.050344,0.375715}%
\pgfsetstrokecolor{currentstroke}%
\pgfsetdash{}{0pt}%
\pgfpathmoveto{\pgfqpoint{3.059972in}{5.024378in}}%
\pgfpathlineto{\pgfqpoint{3.009832in}{5.025219in}}%
\pgfusepath{stroke}%
\end{pgfscope}%
\begin{pgfscope}%
\pgfpathrectangle{\pgfqpoint{1.250000in}{4.155455in}}{\pgfqpoint{2.279412in}{2.004545in}}%
\pgfusepath{clip}%
\pgfsetbuttcap%
\pgfsetroundjoin%
\pgfsetlinewidth{0.375027pt}%
\definecolor{currentstroke}{rgb}{0.278791,0.062145,0.386592}%
\pgfsetstrokecolor{currentstroke}%
\pgfsetdash{}{0pt}%
\pgfpathmoveto{\pgfqpoint{3.009832in}{5.025219in}}%
\pgfpathlineto{\pgfqpoint{2.959696in}{5.026301in}}%
\pgfusepath{stroke}%
\end{pgfscope}%
\begin{pgfscope}%
\pgfpathrectangle{\pgfqpoint{1.250000in}{4.155455in}}{\pgfqpoint{2.279412in}{2.004545in}}%
\pgfusepath{clip}%
\pgfsetbuttcap%
\pgfsetroundjoin%
\pgfsetlinewidth{0.407157pt}%
\definecolor{currentstroke}{rgb}{0.281924,0.089666,0.412415}%
\pgfsetstrokecolor{currentstroke}%
\pgfsetdash{}{0pt}%
\pgfpathmoveto{\pgfqpoint{2.959696in}{5.026301in}}%
\pgfpathlineto{\pgfqpoint{2.909561in}{5.027438in}}%
\pgfusepath{stroke}%
\end{pgfscope}%
\begin{pgfscope}%
\pgfpathrectangle{\pgfqpoint{1.250000in}{4.155455in}}{\pgfqpoint{2.279412in}{2.004545in}}%
\pgfusepath{clip}%
\pgfsetbuttcap%
\pgfsetroundjoin%
\pgfsetlinewidth{0.446113pt}%
\definecolor{currentstroke}{rgb}{0.283229,0.120777,0.440584}%
\pgfsetstrokecolor{currentstroke}%
\pgfsetdash{}{0pt}%
\pgfpathmoveto{\pgfqpoint{2.909561in}{5.027438in}}%
\pgfpathlineto{\pgfqpoint{2.859428in}{5.028649in}}%
\pgfusepath{stroke}%
\end{pgfscope}%
\begin{pgfscope}%
\pgfpathrectangle{\pgfqpoint{1.250000in}{4.155455in}}{\pgfqpoint{2.279412in}{2.004545in}}%
\pgfusepath{clip}%
\pgfsetbuttcap%
\pgfsetroundjoin%
\pgfsetlinewidth{0.488952pt}%
\definecolor{currentstroke}{rgb}{0.281887,0.150881,0.465405}%
\pgfsetstrokecolor{currentstroke}%
\pgfsetdash{}{0pt}%
\pgfpathmoveto{\pgfqpoint{2.859428in}{5.028649in}}%
\pgfpathlineto{\pgfqpoint{2.809299in}{5.029953in}}%
\pgfusepath{stroke}%
\end{pgfscope}%
\begin{pgfscope}%
\pgfpathrectangle{\pgfqpoint{1.250000in}{4.155455in}}{\pgfqpoint{2.279412in}{2.004545in}}%
\pgfusepath{clip}%
\pgfsetbuttcap%
\pgfsetroundjoin%
\pgfsetlinewidth{0.558410pt}%
\definecolor{currentstroke}{rgb}{0.274128,0.199721,0.498911}%
\pgfsetstrokecolor{currentstroke}%
\pgfsetdash{}{0pt}%
\pgfpathmoveto{\pgfqpoint{2.809299in}{5.029953in}}%
\pgfpathlineto{\pgfqpoint{2.759172in}{5.031320in}}%
\pgfusepath{stroke}%
\end{pgfscope}%
\begin{pgfscope}%
\pgfpathrectangle{\pgfqpoint{1.250000in}{4.155455in}}{\pgfqpoint{2.279412in}{2.004545in}}%
\pgfusepath{clip}%
\pgfsetbuttcap%
\pgfsetroundjoin%
\pgfsetlinewidth{0.628460pt}%
\definecolor{currentstroke}{rgb}{0.260571,0.246922,0.522828}%
\pgfsetstrokecolor{currentstroke}%
\pgfsetdash{}{0pt}%
\pgfpathmoveto{\pgfqpoint{2.759172in}{5.031320in}}%
\pgfpathlineto{\pgfqpoint{2.709054in}{5.032949in}}%
\pgfusepath{stroke}%
\end{pgfscope}%
\begin{pgfscope}%
\pgfpathrectangle{\pgfqpoint{1.250000in}{4.155455in}}{\pgfqpoint{2.279412in}{2.004545in}}%
\pgfusepath{clip}%
\pgfsetbuttcap%
\pgfsetroundjoin%
\pgfsetlinewidth{0.719122pt}%
\definecolor{currentstroke}{rgb}{0.237441,0.305202,0.541921}%
\pgfsetstrokecolor{currentstroke}%
\pgfsetdash{}{0pt}%
\pgfpathmoveto{\pgfqpoint{2.709054in}{5.032949in}}%
\pgfpathlineto{\pgfqpoint{2.658941in}{5.034681in}}%
\pgfusepath{stroke}%
\end{pgfscope}%
\begin{pgfscope}%
\pgfpathrectangle{\pgfqpoint{1.250000in}{4.155455in}}{\pgfqpoint{2.279412in}{2.004545in}}%
\pgfusepath{clip}%
\pgfsetbuttcap%
\pgfsetroundjoin%
\pgfsetlinewidth{0.747712pt}%
\definecolor{currentstroke}{rgb}{0.227802,0.326594,0.546532}%
\pgfsetstrokecolor{currentstroke}%
\pgfsetdash{}{0pt}%
\pgfpathmoveto{\pgfqpoint{2.658941in}{5.034681in}}%
\pgfpathlineto{\pgfqpoint{2.608837in}{5.036575in}}%
\pgfusepath{stroke}%
\end{pgfscope}%
\begin{pgfscope}%
\pgfpathrectangle{\pgfqpoint{1.250000in}{4.155455in}}{\pgfqpoint{2.279412in}{2.004545in}}%
\pgfusepath{clip}%
\pgfsetbuttcap%
\pgfsetroundjoin%
\pgfsetlinewidth{0.814573pt}%
\definecolor{currentstroke}{rgb}{0.210503,0.363727,0.552206}%
\pgfsetstrokecolor{currentstroke}%
\pgfsetdash{}{0pt}%
\pgfpathmoveto{\pgfqpoint{2.608837in}{5.036575in}}%
\pgfpathlineto{\pgfqpoint{2.558747in}{5.038757in}}%
\pgfusepath{stroke}%
\end{pgfscope}%
\begin{pgfscope}%
\pgfpathrectangle{\pgfqpoint{1.250000in}{4.155455in}}{\pgfqpoint{2.279412in}{2.004545in}}%
\pgfusepath{clip}%
\pgfsetbuttcap%
\pgfsetroundjoin%
\pgfsetlinewidth{0.890641pt}%
\definecolor{currentstroke}{rgb}{0.190631,0.407061,0.556089}%
\pgfsetstrokecolor{currentstroke}%
\pgfsetdash{}{0pt}%
\pgfpathmoveto{\pgfqpoint{2.558747in}{5.038757in}}%
\pgfpathlineto{\pgfqpoint{2.508675in}{5.041241in}}%
\pgfusepath{stroke}%
\end{pgfscope}%
\begin{pgfscope}%
\pgfpathrectangle{\pgfqpoint{1.250000in}{4.155455in}}{\pgfqpoint{2.279412in}{2.004545in}}%
\pgfusepath{clip}%
\pgfsetbuttcap%
\pgfsetroundjoin%
\pgfsetlinewidth{0.902870pt}%
\definecolor{currentstroke}{rgb}{0.187231,0.414746,0.556547}%
\pgfsetstrokecolor{currentstroke}%
\pgfsetdash{}{0pt}%
\pgfpathmoveto{\pgfqpoint{2.508675in}{5.041241in}}%
\pgfpathlineto{\pgfqpoint{2.458628in}{5.044070in}}%
\pgfusepath{stroke}%
\end{pgfscope}%
\begin{pgfscope}%
\pgfpathrectangle{\pgfqpoint{1.250000in}{4.155455in}}{\pgfqpoint{2.279412in}{2.004545in}}%
\pgfusepath{clip}%
\pgfsetbuttcap%
\pgfsetroundjoin%
\pgfsetlinewidth{0.321240pt}%
\definecolor{currentstroke}{rgb}{0.269944,0.014625,0.341379}%
\pgfsetstrokecolor{currentstroke}%
\pgfsetdash{}{0pt}%
\pgfpathmoveto{\pgfqpoint{3.210376in}{5.247941in}}%
\pgfpathlineto{\pgfqpoint{3.160442in}{5.248879in}}%
\pgfusepath{stroke}%
\end{pgfscope}%
\begin{pgfscope}%
\pgfpathrectangle{\pgfqpoint{1.250000in}{4.155455in}}{\pgfqpoint{2.279412in}{2.004545in}}%
\pgfusepath{clip}%
\pgfsetbuttcap%
\pgfsetroundjoin%
\pgfsetlinewidth{0.323699pt}%
\definecolor{currentstroke}{rgb}{0.271305,0.019942,0.347269}%
\pgfsetstrokecolor{currentstroke}%
\pgfsetdash{}{0pt}%
\pgfpathmoveto{\pgfqpoint{3.160442in}{5.248879in}}%
\pgfpathlineto{\pgfqpoint{3.110371in}{5.247779in}}%
\pgfusepath{stroke}%
\end{pgfscope}%
\begin{pgfscope}%
\pgfpathrectangle{\pgfqpoint{1.250000in}{4.155455in}}{\pgfqpoint{2.279412in}{2.004545in}}%
\pgfusepath{clip}%
\pgfsetbuttcap%
\pgfsetroundjoin%
\pgfsetlinewidth{0.342306pt}%
\definecolor{currentstroke}{rgb}{0.274952,0.037752,0.364543}%
\pgfsetstrokecolor{currentstroke}%
\pgfsetdash{}{0pt}%
\pgfpathmoveto{\pgfqpoint{3.110371in}{5.247779in}}%
\pgfpathlineto{\pgfqpoint{3.060231in}{5.247249in}}%
\pgfusepath{stroke}%
\end{pgfscope}%
\begin{pgfscope}%
\pgfpathrectangle{\pgfqpoint{1.250000in}{4.155455in}}{\pgfqpoint{2.279412in}{2.004545in}}%
\pgfusepath{clip}%
\pgfsetbuttcap%
\pgfsetroundjoin%
\pgfsetlinewidth{0.353022pt}%
\definecolor{currentstroke}{rgb}{0.276022,0.044167,0.370164}%
\pgfsetstrokecolor{currentstroke}%
\pgfsetdash{}{0pt}%
\pgfpathmoveto{\pgfqpoint{3.060231in}{5.247249in}}%
\pgfpathlineto{\pgfqpoint{3.010088in}{5.246805in}}%
\pgfusepath{stroke}%
\end{pgfscope}%
\begin{pgfscope}%
\pgfpathrectangle{\pgfqpoint{1.250000in}{4.155455in}}{\pgfqpoint{2.279412in}{2.004545in}}%
\pgfusepath{clip}%
\pgfsetbuttcap%
\pgfsetroundjoin%
\pgfsetlinewidth{0.368456pt}%
\definecolor{currentstroke}{rgb}{0.277941,0.056324,0.381191}%
\pgfsetstrokecolor{currentstroke}%
\pgfsetdash{}{0pt}%
\pgfpathmoveto{\pgfqpoint{3.010088in}{5.246805in}}%
\pgfpathlineto{\pgfqpoint{2.959945in}{5.246061in}}%
\pgfusepath{stroke}%
\end{pgfscope}%
\begin{pgfscope}%
\pgfpathrectangle{\pgfqpoint{1.250000in}{4.155455in}}{\pgfqpoint{2.279412in}{2.004545in}}%
\pgfusepath{clip}%
\pgfsetbuttcap%
\pgfsetroundjoin%
\pgfsetlinewidth{0.408726pt}%
\definecolor{currentstroke}{rgb}{0.281924,0.089666,0.412415}%
\pgfsetstrokecolor{currentstroke}%
\pgfsetdash{}{0pt}%
\pgfpathmoveto{\pgfqpoint{2.959945in}{5.246061in}}%
\pgfpathlineto{\pgfqpoint{2.909806in}{5.245056in}}%
\pgfusepath{stroke}%
\end{pgfscope}%
\begin{pgfscope}%
\pgfpathrectangle{\pgfqpoint{1.250000in}{4.155455in}}{\pgfqpoint{2.279412in}{2.004545in}}%
\pgfusepath{clip}%
\pgfsetbuttcap%
\pgfsetroundjoin%
\pgfsetlinewidth{0.442135pt}%
\definecolor{currentstroke}{rgb}{0.283197,0.115680,0.436115}%
\pgfsetstrokecolor{currentstroke}%
\pgfsetdash{}{0pt}%
\pgfpathmoveto{\pgfqpoint{2.909806in}{5.245056in}}%
\pgfpathlineto{\pgfqpoint{2.859666in}{5.244118in}}%
\pgfusepath{stroke}%
\end{pgfscope}%
\begin{pgfscope}%
\pgfpathrectangle{\pgfqpoint{1.250000in}{4.155455in}}{\pgfqpoint{2.279412in}{2.004545in}}%
\pgfusepath{clip}%
\pgfsetbuttcap%
\pgfsetroundjoin%
\pgfsetlinewidth{0.509446pt}%
\definecolor{currentstroke}{rgb}{0.280255,0.165693,0.476498}%
\pgfsetstrokecolor{currentstroke}%
\pgfsetdash{}{0pt}%
\pgfpathmoveto{\pgfqpoint{2.859666in}{5.244118in}}%
\pgfpathlineto{\pgfqpoint{2.809528in}{5.243075in}}%
\pgfusepath{stroke}%
\end{pgfscope}%
\begin{pgfscope}%
\pgfpathrectangle{\pgfqpoint{1.250000in}{4.155455in}}{\pgfqpoint{2.279412in}{2.004545in}}%
\pgfusepath{clip}%
\pgfsetbuttcap%
\pgfsetroundjoin%
\pgfsetlinewidth{0.561379pt}%
\definecolor{currentstroke}{rgb}{0.274128,0.199721,0.498911}%
\pgfsetstrokecolor{currentstroke}%
\pgfsetdash{}{0pt}%
\pgfpathmoveto{\pgfqpoint{2.809528in}{5.243075in}}%
\pgfpathlineto{\pgfqpoint{2.759393in}{5.241929in}}%
\pgfusepath{stroke}%
\end{pgfscope}%
\begin{pgfscope}%
\pgfpathrectangle{\pgfqpoint{1.250000in}{4.155455in}}{\pgfqpoint{2.279412in}{2.004545in}}%
\pgfusepath{clip}%
\pgfsetbuttcap%
\pgfsetroundjoin%
\pgfsetlinewidth{0.634518pt}%
\definecolor{currentstroke}{rgb}{0.258965,0.251537,0.524736}%
\pgfsetstrokecolor{currentstroke}%
\pgfsetdash{}{0pt}%
\pgfpathmoveto{\pgfqpoint{2.759393in}{5.241929in}}%
\pgfpathlineto{\pgfqpoint{2.709260in}{5.240746in}}%
\pgfusepath{stroke}%
\end{pgfscope}%
\begin{pgfscope}%
\pgfpathrectangle{\pgfqpoint{1.250000in}{4.155455in}}{\pgfqpoint{2.279412in}{2.004545in}}%
\pgfusepath{clip}%
\pgfsetbuttcap%
\pgfsetroundjoin%
\pgfsetlinewidth{0.703703pt}%
\definecolor{currentstroke}{rgb}{0.241237,0.296485,0.539709}%
\pgfsetstrokecolor{currentstroke}%
\pgfsetdash{}{0pt}%
\pgfpathmoveto{\pgfqpoint{2.709260in}{5.240746in}}%
\pgfpathlineto{\pgfqpoint{2.659130in}{5.239424in}}%
\pgfusepath{stroke}%
\end{pgfscope}%
\begin{pgfscope}%
\pgfpathrectangle{\pgfqpoint{1.250000in}{4.155455in}}{\pgfqpoint{2.279412in}{2.004545in}}%
\pgfusepath{clip}%
\pgfsetbuttcap%
\pgfsetroundjoin%
\pgfsetlinewidth{0.812220pt}%
\definecolor{currentstroke}{rgb}{0.210503,0.363727,0.552206}%
\pgfsetstrokecolor{currentstroke}%
\pgfsetdash{}{0pt}%
\pgfpathmoveto{\pgfqpoint{2.659130in}{5.239424in}}%
\pgfpathlineto{\pgfqpoint{2.609010in}{5.237881in}}%
\pgfusepath{stroke}%
\end{pgfscope}%
\begin{pgfscope}%
\pgfpathrectangle{\pgfqpoint{1.250000in}{4.155455in}}{\pgfqpoint{2.279412in}{2.004545in}}%
\pgfusepath{clip}%
\pgfsetbuttcap%
\pgfsetroundjoin%
\pgfsetlinewidth{0.864620pt}%
\definecolor{currentstroke}{rgb}{0.195860,0.395433,0.555276}%
\pgfsetstrokecolor{currentstroke}%
\pgfsetdash{}{0pt}%
\pgfpathmoveto{\pgfqpoint{2.609010in}{5.237881in}}%
\pgfpathlineto{\pgfqpoint{2.558898in}{5.236112in}}%
\pgfusepath{stroke}%
\end{pgfscope}%
\begin{pgfscope}%
\pgfpathrectangle{\pgfqpoint{1.250000in}{4.155455in}}{\pgfqpoint{2.279412in}{2.004545in}}%
\pgfusepath{clip}%
\pgfsetbuttcap%
\pgfsetroundjoin%
\pgfsetlinewidth{0.897347pt}%
\definecolor{currentstroke}{rgb}{0.188923,0.410910,0.556326}%
\pgfsetstrokecolor{currentstroke}%
\pgfsetdash{}{0pt}%
\pgfpathmoveto{\pgfqpoint{2.558898in}{5.236112in}}%
\pgfpathlineto{\pgfqpoint{2.508797in}{5.234134in}}%
\pgfusepath{stroke}%
\end{pgfscope}%
\begin{pgfscope}%
\pgfpathrectangle{\pgfqpoint{1.250000in}{4.155455in}}{\pgfqpoint{2.279412in}{2.004545in}}%
\pgfusepath{clip}%
\pgfsetbuttcap%
\pgfsetroundjoin%
\pgfsetlinewidth{0.320788pt}%
\definecolor{currentstroke}{rgb}{0.269944,0.014625,0.341379}%
\pgfsetstrokecolor{currentstroke}%
\pgfsetdash{}{0pt}%
\pgfpathmoveto{\pgfqpoint{3.210376in}{5.383261in}}%
\pgfpathlineto{\pgfqpoint{3.160274in}{5.382614in}}%
\pgfusepath{stroke}%
\end{pgfscope}%
\begin{pgfscope}%
\pgfpathrectangle{\pgfqpoint{1.250000in}{4.155455in}}{\pgfqpoint{2.279412in}{2.004545in}}%
\pgfusepath{clip}%
\pgfsetbuttcap%
\pgfsetroundjoin%
\pgfsetlinewidth{0.324309pt}%
\definecolor{currentstroke}{rgb}{0.271305,0.019942,0.347269}%
\pgfsetstrokecolor{currentstroke}%
\pgfsetdash{}{0pt}%
\pgfpathmoveto{\pgfqpoint{3.160274in}{5.382614in}}%
\pgfpathlineto{\pgfqpoint{3.110192in}{5.380481in}}%
\pgfusepath{stroke}%
\end{pgfscope}%
\begin{pgfscope}%
\pgfpathrectangle{\pgfqpoint{1.250000in}{4.155455in}}{\pgfqpoint{2.279412in}{2.004545in}}%
\pgfusepath{clip}%
\pgfsetbuttcap%
\pgfsetroundjoin%
\pgfsetlinewidth{0.329792pt}%
\definecolor{currentstroke}{rgb}{0.272594,0.025563,0.353093}%
\pgfsetstrokecolor{currentstroke}%
\pgfsetdash{}{0pt}%
\pgfpathmoveto{\pgfqpoint{3.110192in}{5.380481in}}%
\pgfpathlineto{\pgfqpoint{3.060106in}{5.378599in}}%
\pgfusepath{stroke}%
\end{pgfscope}%
\begin{pgfscope}%
\pgfpathrectangle{\pgfqpoint{1.250000in}{4.155455in}}{\pgfqpoint{2.279412in}{2.004545in}}%
\pgfusepath{clip}%
\pgfsetbuttcap%
\pgfsetroundjoin%
\pgfsetlinewidth{0.341741pt}%
\definecolor{currentstroke}{rgb}{0.273809,0.031497,0.358853}%
\pgfsetstrokecolor{currentstroke}%
\pgfsetdash{}{0pt}%
\pgfpathmoveto{\pgfqpoint{3.060106in}{5.378599in}}%
\pgfpathlineto{\pgfqpoint{3.009991in}{5.377207in}}%
\pgfusepath{stroke}%
\end{pgfscope}%
\begin{pgfscope}%
\pgfpathrectangle{\pgfqpoint{1.250000in}{4.155455in}}{\pgfqpoint{2.279412in}{2.004545in}}%
\pgfusepath{clip}%
\pgfsetbuttcap%
\pgfsetroundjoin%
\pgfsetlinewidth{0.352792pt}%
\definecolor{currentstroke}{rgb}{0.276022,0.044167,0.370164}%
\pgfsetstrokecolor{currentstroke}%
\pgfsetdash{}{0pt}%
\pgfpathmoveto{\pgfqpoint{3.009991in}{5.377207in}}%
\pgfpathlineto{\pgfqpoint{2.959904in}{5.375022in}}%
\pgfusepath{stroke}%
\end{pgfscope}%
\begin{pgfscope}%
\pgfpathrectangle{\pgfqpoint{1.250000in}{4.155455in}}{\pgfqpoint{2.279412in}{2.004545in}}%
\pgfusepath{clip}%
\pgfsetbuttcap%
\pgfsetroundjoin%
\pgfsetlinewidth{0.392128pt}%
\definecolor{currentstroke}{rgb}{0.280894,0.078907,0.402329}%
\pgfsetstrokecolor{currentstroke}%
\pgfsetdash{}{0pt}%
\pgfpathmoveto{\pgfqpoint{2.959904in}{5.375022in}}%
\pgfpathlineto{\pgfqpoint{2.909836in}{5.372480in}}%
\pgfusepath{stroke}%
\end{pgfscope}%
\begin{pgfscope}%
\pgfpathrectangle{\pgfqpoint{1.250000in}{4.155455in}}{\pgfqpoint{2.279412in}{2.004545in}}%
\pgfusepath{clip}%
\pgfsetbuttcap%
\pgfsetroundjoin%
\pgfsetlinewidth{0.427931pt}%
\definecolor{currentstroke}{rgb}{0.282910,0.105393,0.426902}%
\pgfsetstrokecolor{currentstroke}%
\pgfsetdash{}{0pt}%
\pgfpathmoveto{\pgfqpoint{2.909836in}{5.372480in}}%
\pgfpathlineto{\pgfqpoint{2.859762in}{5.370030in}}%
\pgfusepath{stroke}%
\end{pgfscope}%
\begin{pgfscope}%
\pgfpathrectangle{\pgfqpoint{1.250000in}{4.155455in}}{\pgfqpoint{2.279412in}{2.004545in}}%
\pgfusepath{clip}%
\pgfsetbuttcap%
\pgfsetroundjoin%
\pgfsetlinewidth{0.460156pt}%
\definecolor{currentstroke}{rgb}{0.283072,0.130895,0.449241}%
\pgfsetstrokecolor{currentstroke}%
\pgfsetdash{}{0pt}%
\pgfpathmoveto{\pgfqpoint{2.859762in}{5.370030in}}%
\pgfpathlineto{\pgfqpoint{2.809692in}{5.367528in}}%
\pgfusepath{stroke}%
\end{pgfscope}%
\begin{pgfscope}%
\pgfpathrectangle{\pgfqpoint{1.250000in}{4.155455in}}{\pgfqpoint{2.279412in}{2.004545in}}%
\pgfusepath{clip}%
\pgfsetbuttcap%
\pgfsetroundjoin%
\pgfsetlinewidth{0.509038pt}%
\definecolor{currentstroke}{rgb}{0.280255,0.165693,0.476498}%
\pgfsetstrokecolor{currentstroke}%
\pgfsetdash{}{0pt}%
\pgfpathmoveto{\pgfqpoint{2.809692in}{5.367528in}}%
\pgfpathlineto{\pgfqpoint{2.759629in}{5.364913in}}%
\pgfusepath{stroke}%
\end{pgfscope}%
\begin{pgfscope}%
\pgfpathrectangle{\pgfqpoint{1.250000in}{4.155455in}}{\pgfqpoint{2.279412in}{2.004545in}}%
\pgfusepath{clip}%
\pgfsetbuttcap%
\pgfsetroundjoin%
\pgfsetlinewidth{0.548112pt}%
\definecolor{currentstroke}{rgb}{0.276194,0.190074,0.493001}%
\pgfsetstrokecolor{currentstroke}%
\pgfsetdash{}{0pt}%
\pgfpathmoveto{\pgfqpoint{2.759629in}{5.364913in}}%
\pgfpathlineto{\pgfqpoint{2.709579in}{5.362100in}}%
\pgfusepath{stroke}%
\end{pgfscope}%
\begin{pgfscope}%
\pgfpathrectangle{\pgfqpoint{1.250000in}{4.155455in}}{\pgfqpoint{2.279412in}{2.004545in}}%
\pgfusepath{clip}%
\pgfsetbuttcap%
\pgfsetroundjoin%
\pgfsetlinewidth{0.625166pt}%
\definecolor{currentstroke}{rgb}{0.260571,0.246922,0.522828}%
\pgfsetstrokecolor{currentstroke}%
\pgfsetdash{}{0pt}%
\pgfpathmoveto{\pgfqpoint{2.709579in}{5.362100in}}%
\pgfpathlineto{\pgfqpoint{2.659546in}{5.359075in}}%
\pgfusepath{stroke}%
\end{pgfscope}%
\begin{pgfscope}%
\pgfpathrectangle{\pgfqpoint{1.250000in}{4.155455in}}{\pgfqpoint{2.279412in}{2.004545in}}%
\pgfusepath{clip}%
\pgfsetbuttcap%
\pgfsetroundjoin%
\pgfsetlinewidth{0.676486pt}%
\definecolor{currentstroke}{rgb}{0.248629,0.278775,0.534556}%
\pgfsetstrokecolor{currentstroke}%
\pgfsetdash{}{0pt}%
\pgfpathmoveto{\pgfqpoint{2.659546in}{5.359075in}}%
\pgfpathlineto{\pgfqpoint{2.609542in}{5.355694in}}%
\pgfusepath{stroke}%
\end{pgfscope}%
\begin{pgfscope}%
\pgfpathrectangle{\pgfqpoint{1.250000in}{4.155455in}}{\pgfqpoint{2.279412in}{2.004545in}}%
\pgfusepath{clip}%
\pgfsetbuttcap%
\pgfsetroundjoin%
\pgfsetlinewidth{0.742192pt}%
\definecolor{currentstroke}{rgb}{0.229739,0.322361,0.545706}%
\pgfsetstrokecolor{currentstroke}%
\pgfsetdash{}{0pt}%
\pgfpathmoveto{\pgfqpoint{2.609542in}{5.355694in}}%
\pgfpathlineto{\pgfqpoint{2.559586in}{5.351811in}}%
\pgfusepath{stroke}%
\end{pgfscope}%
\begin{pgfscope}%
\pgfpathrectangle{\pgfqpoint{1.250000in}{4.155455in}}{\pgfqpoint{2.279412in}{2.004545in}}%
\pgfusepath{clip}%
\pgfsetbuttcap%
\pgfsetroundjoin%
\pgfsetlinewidth{0.761618pt}%
\definecolor{currentstroke}{rgb}{0.223925,0.334994,0.548053}%
\pgfsetstrokecolor{currentstroke}%
\pgfsetdash{}{0pt}%
\pgfpathmoveto{\pgfqpoint{2.559586in}{5.351811in}}%
\pgfpathlineto{\pgfqpoint{2.509683in}{5.347442in}}%
\pgfusepath{stroke}%
\end{pgfscope}%
\begin{pgfscope}%
\pgfpathrectangle{\pgfqpoint{1.250000in}{4.155455in}}{\pgfqpoint{2.279412in}{2.004545in}}%
\pgfusepath{clip}%
\pgfsetbuttcap%
\pgfsetroundjoin%
\pgfsetlinewidth{0.817042pt}%
\definecolor{currentstroke}{rgb}{0.208623,0.367752,0.552675}%
\pgfsetstrokecolor{currentstroke}%
\pgfsetdash{}{0pt}%
\pgfpathmoveto{\pgfqpoint{2.509683in}{5.347442in}}%
\pgfpathlineto{\pgfqpoint{2.459854in}{5.342463in}}%
\pgfusepath{stroke}%
\end{pgfscope}%
\begin{pgfscope}%
\pgfpathrectangle{\pgfqpoint{1.250000in}{4.155455in}}{\pgfqpoint{2.279412in}{2.004545in}}%
\pgfusepath{clip}%
\pgfsetbuttcap%
\pgfsetroundjoin%
\pgfsetlinewidth{0.801558pt}%
\definecolor{currentstroke}{rgb}{0.214298,0.355619,0.551184}%
\pgfsetstrokecolor{currentstroke}%
\pgfsetdash{}{0pt}%
\pgfpathmoveto{\pgfqpoint{2.459854in}{5.342463in}}%
\pgfpathlineto{\pgfqpoint{2.410158in}{5.336563in}}%
\pgfusepath{stroke}%
\end{pgfscope}%
\begin{pgfscope}%
\pgfpathrectangle{\pgfqpoint{1.250000in}{4.155455in}}{\pgfqpoint{2.279412in}{2.004545in}}%
\pgfusepath{clip}%
\pgfsetbuttcap%
\pgfsetroundjoin%
\pgfsetlinewidth{0.793496pt}%
\definecolor{currentstroke}{rgb}{0.216210,0.351535,0.550627}%
\pgfsetstrokecolor{currentstroke}%
\pgfsetdash{}{0pt}%
\pgfpathmoveto{\pgfqpoint{2.410158in}{5.336563in}}%
\pgfpathlineto{\pgfqpoint{2.360689in}{5.329371in}}%
\pgfusepath{stroke}%
\end{pgfscope}%
\begin{pgfscope}%
\pgfpathrectangle{\pgfqpoint{1.250000in}{4.155455in}}{\pgfqpoint{2.279412in}{2.004545in}}%
\pgfusepath{clip}%
\pgfsetbuttcap%
\pgfsetroundjoin%
\pgfsetlinewidth{0.802958pt}%
\definecolor{currentstroke}{rgb}{0.212395,0.359683,0.551710}%
\pgfsetstrokecolor{currentstroke}%
\pgfsetdash{}{0pt}%
\pgfpathmoveto{\pgfqpoint{2.360689in}{5.329371in}}%
\pgfpathlineto{\pgfqpoint{2.311621in}{5.320348in}}%
\pgfusepath{stroke}%
\end{pgfscope}%
\begin{pgfscope}%
\pgfpathrectangle{\pgfqpoint{1.250000in}{4.155455in}}{\pgfqpoint{2.279412in}{2.004545in}}%
\pgfusepath{clip}%
\pgfsetbuttcap%
\pgfsetroundjoin%
\pgfsetlinewidth{0.728022pt}%
\definecolor{currentstroke}{rgb}{0.233603,0.313828,0.543914}%
\pgfsetstrokecolor{currentstroke}%
\pgfsetdash{}{0pt}%
\pgfpathmoveto{\pgfqpoint{2.311621in}{5.320348in}}%
\pgfpathlineto{\pgfqpoint{2.263424in}{5.308414in}}%
\pgfusepath{stroke}%
\end{pgfscope}%
\begin{pgfscope}%
\pgfpathrectangle{\pgfqpoint{1.250000in}{4.155455in}}{\pgfqpoint{2.279412in}{2.004545in}}%
\pgfusepath{clip}%
\pgfsetbuttcap%
\pgfsetroundjoin%
\pgfsetlinewidth{0.657693pt}%
\definecolor{currentstroke}{rgb}{0.253935,0.265254,0.529983}%
\pgfsetstrokecolor{currentstroke}%
\pgfsetdash{}{0pt}%
\pgfpathmoveto{\pgfqpoint{2.263424in}{5.308414in}}%
\pgfpathlineto{\pgfqpoint{2.217704in}{5.290984in}}%
\pgfusepath{stroke}%
\end{pgfscope}%
\begin{pgfscope}%
\pgfpathrectangle{\pgfqpoint{1.250000in}{4.155455in}}{\pgfqpoint{2.279412in}{2.004545in}}%
\pgfusepath{clip}%
\pgfsetbuttcap%
\pgfsetroundjoin%
\pgfsetlinewidth{0.567064pt}%
\definecolor{currentstroke}{rgb}{0.273006,0.204520,0.501721}%
\pgfsetstrokecolor{currentstroke}%
\pgfsetdash{}{0pt}%
\pgfpathmoveto{\pgfqpoint{2.217704in}{5.290984in}}%
\pgfpathlineto{\pgfqpoint{2.217704in}{5.290984in}}%
\pgfusepath{stroke}%
\end{pgfscope}%
\begin{pgfscope}%
\pgfpathrectangle{\pgfqpoint{1.250000in}{4.155455in}}{\pgfqpoint{2.279412in}{2.004545in}}%
\pgfusepath{clip}%
\pgfsetbuttcap%
\pgfsetroundjoin%
\pgfsetlinewidth{0.567064pt}%
\definecolor{currentstroke}{rgb}{0.273006,0.204520,0.501721}%
\pgfsetstrokecolor{currentstroke}%
\pgfsetdash{}{0pt}%
\pgfpathmoveto{\pgfqpoint{2.217704in}{5.290984in}}%
\pgfpathlineto{\pgfqpoint{2.182527in}{5.271342in}}%
\pgfusepath{stroke}%
\end{pgfscope}%
\begin{pgfscope}%
\pgfpathrectangle{\pgfqpoint{1.250000in}{4.155455in}}{\pgfqpoint{2.279412in}{2.004545in}}%
\pgfusepath{clip}%
\pgfsetbuttcap%
\pgfsetroundjoin%
\pgfsetlinewidth{0.548898pt}%
\definecolor{currentstroke}{rgb}{0.275191,0.194905,0.496005}%
\pgfsetstrokecolor{currentstroke}%
\pgfsetdash{}{0pt}%
\pgfpathmoveto{\pgfqpoint{2.182527in}{5.271342in}}%
\pgfpathlineto{\pgfqpoint{2.182527in}{5.271342in}}%
\pgfusepath{stroke}%
\end{pgfscope}%
\begin{pgfscope}%
\pgfpathrectangle{\pgfqpoint{1.250000in}{4.155455in}}{\pgfqpoint{2.279412in}{2.004545in}}%
\pgfusepath{clip}%
\pgfsetbuttcap%
\pgfsetroundjoin%
\pgfsetlinewidth{0.323032pt}%
\definecolor{currentstroke}{rgb}{0.271305,0.019942,0.347269}%
\pgfsetstrokecolor{currentstroke}%
\pgfsetdash{}{0pt}%
\pgfpathmoveto{\pgfqpoint{3.159084in}{4.751766in}}%
\pgfpathlineto{\pgfqpoint{3.109053in}{4.754152in}}%
\pgfusepath{stroke}%
\end{pgfscope}%
\begin{pgfscope}%
\pgfpathrectangle{\pgfqpoint{1.250000in}{4.155455in}}{\pgfqpoint{2.279412in}{2.004545in}}%
\pgfusepath{clip}%
\pgfsetbuttcap%
\pgfsetroundjoin%
\pgfsetlinewidth{0.326368pt}%
\definecolor{currentstroke}{rgb}{0.271305,0.019942,0.347269}%
\pgfsetstrokecolor{currentstroke}%
\pgfsetdash{}{0pt}%
\pgfpathmoveto{\pgfqpoint{3.109053in}{4.754152in}}%
\pgfpathlineto{\pgfqpoint{3.059015in}{4.756737in}}%
\pgfusepath{stroke}%
\end{pgfscope}%
\begin{pgfscope}%
\pgfpathrectangle{\pgfqpoint{1.250000in}{4.155455in}}{\pgfqpoint{2.279412in}{2.004545in}}%
\pgfusepath{clip}%
\pgfsetbuttcap%
\pgfsetroundjoin%
\pgfsetlinewidth{0.329019pt}%
\definecolor{currentstroke}{rgb}{0.272594,0.025563,0.353093}%
\pgfsetstrokecolor{currentstroke}%
\pgfsetdash{}{0pt}%
\pgfpathmoveto{\pgfqpoint{3.059015in}{4.756737in}}%
\pgfpathlineto{\pgfqpoint{3.008950in}{4.759114in}}%
\pgfusepath{stroke}%
\end{pgfscope}%
\begin{pgfscope}%
\pgfpathrectangle{\pgfqpoint{1.250000in}{4.155455in}}{\pgfqpoint{2.279412in}{2.004545in}}%
\pgfusepath{clip}%
\pgfsetbuttcap%
\pgfsetroundjoin%
\pgfsetlinewidth{0.341983pt}%
\definecolor{currentstroke}{rgb}{0.273809,0.031497,0.358853}%
\pgfsetstrokecolor{currentstroke}%
\pgfsetdash{}{0pt}%
\pgfpathmoveto{\pgfqpoint{3.008950in}{4.759114in}}%
\pgfpathlineto{\pgfqpoint{2.958877in}{4.761503in}}%
\pgfusepath{stroke}%
\end{pgfscope}%
\begin{pgfscope}%
\pgfpathrectangle{\pgfqpoint{1.250000in}{4.155455in}}{\pgfqpoint{2.279412in}{2.004545in}}%
\pgfusepath{clip}%
\pgfsetbuttcap%
\pgfsetroundjoin%
\pgfsetlinewidth{0.345691pt}%
\definecolor{currentstroke}{rgb}{0.274952,0.037752,0.364543}%
\pgfsetstrokecolor{currentstroke}%
\pgfsetdash{}{0pt}%
\pgfpathmoveto{\pgfqpoint{2.958877in}{4.761503in}}%
\pgfpathlineto{\pgfqpoint{2.908803in}{4.763888in}}%
\pgfusepath{stroke}%
\end{pgfscope}%
\begin{pgfscope}%
\pgfpathrectangle{\pgfqpoint{1.250000in}{4.155455in}}{\pgfqpoint{2.279412in}{2.004545in}}%
\pgfusepath{clip}%
\pgfsetbuttcap%
\pgfsetroundjoin%
\pgfsetlinewidth{0.360770pt}%
\definecolor{currentstroke}{rgb}{0.277018,0.050344,0.375715}%
\pgfsetstrokecolor{currentstroke}%
\pgfsetdash{}{0pt}%
\pgfpathmoveto{\pgfqpoint{2.908803in}{4.763888in}}%
\pgfpathlineto{\pgfqpoint{2.858755in}{4.766603in}}%
\pgfusepath{stroke}%
\end{pgfscope}%
\begin{pgfscope}%
\pgfpathrectangle{\pgfqpoint{1.250000in}{4.155455in}}{\pgfqpoint{2.279412in}{2.004545in}}%
\pgfusepath{clip}%
\pgfsetbuttcap%
\pgfsetroundjoin%
\pgfsetlinewidth{0.381301pt}%
\definecolor{currentstroke}{rgb}{0.279566,0.067836,0.391917}%
\pgfsetstrokecolor{currentstroke}%
\pgfsetdash{}{0pt}%
\pgfpathmoveto{\pgfqpoint{2.858755in}{4.766603in}}%
\pgfpathlineto{\pgfqpoint{2.808796in}{4.770419in}}%
\pgfusepath{stroke}%
\end{pgfscope}%
\begin{pgfscope}%
\pgfpathrectangle{\pgfqpoint{1.250000in}{4.155455in}}{\pgfqpoint{2.279412in}{2.004545in}}%
\pgfusepath{clip}%
\pgfsetbuttcap%
\pgfsetroundjoin%
\pgfsetlinewidth{0.414541pt}%
\definecolor{currentstroke}{rgb}{0.282327,0.094955,0.417331}%
\pgfsetstrokecolor{currentstroke}%
\pgfsetdash{}{0pt}%
\pgfpathmoveto{\pgfqpoint{2.808796in}{4.770419in}}%
\pgfpathlineto{\pgfqpoint{2.758868in}{4.774578in}}%
\pgfusepath{stroke}%
\end{pgfscope}%
\begin{pgfscope}%
\pgfpathrectangle{\pgfqpoint{1.250000in}{4.155455in}}{\pgfqpoint{2.279412in}{2.004545in}}%
\pgfusepath{clip}%
\pgfsetbuttcap%
\pgfsetroundjoin%
\pgfsetlinewidth{0.440418pt}%
\definecolor{currentstroke}{rgb}{0.283197,0.115680,0.436115}%
\pgfsetstrokecolor{currentstroke}%
\pgfsetdash{}{0pt}%
\pgfpathmoveto{\pgfqpoint{2.758868in}{4.774578in}}%
\pgfpathlineto{\pgfqpoint{2.709017in}{4.779348in}}%
\pgfusepath{stroke}%
\end{pgfscope}%
\begin{pgfscope}%
\pgfpathrectangle{\pgfqpoint{1.250000in}{4.155455in}}{\pgfqpoint{2.279412in}{2.004545in}}%
\pgfusepath{clip}%
\pgfsetbuttcap%
\pgfsetroundjoin%
\pgfsetlinewidth{0.485036pt}%
\definecolor{currentstroke}{rgb}{0.282290,0.145912,0.461510}%
\pgfsetstrokecolor{currentstroke}%
\pgfsetdash{}{0pt}%
\pgfpathmoveto{\pgfqpoint{2.709017in}{4.779348in}}%
\pgfpathlineto{\pgfqpoint{2.659252in}{4.784808in}}%
\pgfusepath{stroke}%
\end{pgfscope}%
\begin{pgfscope}%
\pgfpathrectangle{\pgfqpoint{1.250000in}{4.155455in}}{\pgfqpoint{2.279412in}{2.004545in}}%
\pgfusepath{clip}%
\pgfsetbuttcap%
\pgfsetroundjoin%
\pgfsetlinewidth{0.499330pt}%
\definecolor{currentstroke}{rgb}{0.281412,0.155834,0.469201}%
\pgfsetstrokecolor{currentstroke}%
\pgfsetdash{}{0pt}%
\pgfpathmoveto{\pgfqpoint{2.659252in}{4.784808in}}%
\pgfpathlineto{\pgfqpoint{2.609538in}{4.790614in}}%
\pgfusepath{stroke}%
\end{pgfscope}%
\begin{pgfscope}%
\pgfpathrectangle{\pgfqpoint{1.250000in}{4.155455in}}{\pgfqpoint{2.279412in}{2.004545in}}%
\pgfusepath{clip}%
\pgfsetbuttcap%
\pgfsetroundjoin%
\pgfsetlinewidth{0.549193pt}%
\definecolor{currentstroke}{rgb}{0.275191,0.194905,0.496005}%
\pgfsetstrokecolor{currentstroke}%
\pgfsetdash{}{0pt}%
\pgfpathmoveto{\pgfqpoint{2.609538in}{4.790614in}}%
\pgfpathlineto{\pgfqpoint{2.559934in}{4.797101in}}%
\pgfusepath{stroke}%
\end{pgfscope}%
\begin{pgfscope}%
\pgfpathrectangle{\pgfqpoint{1.250000in}{4.155455in}}{\pgfqpoint{2.279412in}{2.004545in}}%
\pgfusepath{clip}%
\pgfsetbuttcap%
\pgfsetroundjoin%
\pgfsetlinewidth{0.593245pt}%
\definecolor{currentstroke}{rgb}{0.267968,0.223549,0.512008}%
\pgfsetstrokecolor{currentstroke}%
\pgfsetdash{}{0pt}%
\pgfpathmoveto{\pgfqpoint{2.559934in}{4.797101in}}%
\pgfpathlineto{\pgfqpoint{2.510514in}{4.804568in}}%
\pgfusepath{stroke}%
\end{pgfscope}%
\begin{pgfscope}%
\pgfpathrectangle{\pgfqpoint{1.250000in}{4.155455in}}{\pgfqpoint{2.279412in}{2.004545in}}%
\pgfusepath{clip}%
\pgfsetbuttcap%
\pgfsetroundjoin%
\pgfsetlinewidth{0.579972pt}%
\definecolor{currentstroke}{rgb}{0.270595,0.214069,0.507052}%
\pgfsetstrokecolor{currentstroke}%
\pgfsetdash{}{0pt}%
\pgfpathmoveto{\pgfqpoint{2.510514in}{4.804568in}}%
\pgfpathlineto{\pgfqpoint{2.461457in}{4.813669in}}%
\pgfusepath{stroke}%
\end{pgfscope}%
\begin{pgfscope}%
\pgfpathrectangle{\pgfqpoint{1.250000in}{4.155455in}}{\pgfqpoint{2.279412in}{2.004545in}}%
\pgfusepath{clip}%
\pgfsetbuttcap%
\pgfsetroundjoin%
\pgfsetlinewidth{0.609228pt}%
\definecolor{currentstroke}{rgb}{0.265145,0.232956,0.516599}%
\pgfsetstrokecolor{currentstroke}%
\pgfsetdash{}{0pt}%
\pgfpathmoveto{\pgfqpoint{2.461457in}{4.813669in}}%
\pgfpathlineto{\pgfqpoint{2.412881in}{4.824586in}}%
\pgfusepath{stroke}%
\end{pgfscope}%
\begin{pgfscope}%
\pgfpathrectangle{\pgfqpoint{1.250000in}{4.155455in}}{\pgfqpoint{2.279412in}{2.004545in}}%
\pgfusepath{clip}%
\pgfsetbuttcap%
\pgfsetroundjoin%
\pgfsetlinewidth{0.631326pt}%
\definecolor{currentstroke}{rgb}{0.258965,0.251537,0.524736}%
\pgfsetstrokecolor{currentstroke}%
\pgfsetdash{}{0pt}%
\pgfpathmoveto{\pgfqpoint{2.412881in}{4.824586in}}%
\pgfpathlineto{\pgfqpoint{2.364760in}{4.836953in}}%
\pgfusepath{stroke}%
\end{pgfscope}%
\begin{pgfscope}%
\pgfpathrectangle{\pgfqpoint{1.250000in}{4.155455in}}{\pgfqpoint{2.279412in}{2.004545in}}%
\pgfusepath{clip}%
\pgfsetbuttcap%
\pgfsetroundjoin%
\pgfsetlinewidth{0.629490pt}%
\definecolor{currentstroke}{rgb}{0.260571,0.246922,0.522828}%
\pgfsetstrokecolor{currentstroke}%
\pgfsetdash{}{0pt}%
\pgfpathmoveto{\pgfqpoint{2.364760in}{4.836953in}}%
\pgfpathlineto{\pgfqpoint{2.318023in}{4.852622in}}%
\pgfusepath{stroke}%
\end{pgfscope}%
\begin{pgfscope}%
\pgfpathrectangle{\pgfqpoint{1.250000in}{4.155455in}}{\pgfqpoint{2.279412in}{2.004545in}}%
\pgfusepath{clip}%
\pgfsetbuttcap%
\pgfsetroundjoin%
\pgfsetlinewidth{0.609203pt}%
\definecolor{currentstroke}{rgb}{0.265145,0.232956,0.516599}%
\pgfsetstrokecolor{currentstroke}%
\pgfsetdash{}{0pt}%
\pgfpathmoveto{\pgfqpoint{2.318023in}{4.852622in}}%
\pgfpathlineto{\pgfqpoint{2.273949in}{4.873316in}}%
\pgfusepath{stroke}%
\end{pgfscope}%
\begin{pgfscope}%
\pgfpathrectangle{\pgfqpoint{1.250000in}{4.155455in}}{\pgfqpoint{2.279412in}{2.004545in}}%
\pgfusepath{clip}%
\pgfsetbuttcap%
\pgfsetroundjoin%
\pgfsetlinewidth{0.627607pt}%
\definecolor{currentstroke}{rgb}{0.260571,0.246922,0.522828}%
\pgfsetstrokecolor{currentstroke}%
\pgfsetdash{}{0pt}%
\pgfpathmoveto{\pgfqpoint{2.273949in}{4.873316in}}%
\pgfpathlineto{\pgfqpoint{2.234794in}{4.900176in}}%
\pgfusepath{stroke}%
\end{pgfscope}%
\begin{pgfscope}%
\pgfpathrectangle{\pgfqpoint{1.250000in}{4.155455in}}{\pgfqpoint{2.279412in}{2.004545in}}%
\pgfusepath{clip}%
\pgfsetbuttcap%
\pgfsetroundjoin%
\pgfsetlinewidth{0.594539pt}%
\definecolor{currentstroke}{rgb}{0.267968,0.223549,0.512008}%
\pgfsetstrokecolor{currentstroke}%
\pgfsetdash{}{0pt}%
\pgfpathmoveto{\pgfqpoint{2.234794in}{4.900176in}}%
\pgfpathlineto{\pgfqpoint{2.205311in}{4.928557in}}%
\pgfusepath{stroke}%
\end{pgfscope}%
\begin{pgfscope}%
\pgfpathrectangle{\pgfqpoint{1.250000in}{4.155455in}}{\pgfqpoint{2.279412in}{2.004545in}}%
\pgfusepath{clip}%
\pgfsetbuttcap%
\pgfsetroundjoin%
\pgfsetlinewidth{0.624445pt}%
\definecolor{currentstroke}{rgb}{0.260571,0.246922,0.522828}%
\pgfsetstrokecolor{currentstroke}%
\pgfsetdash{}{0pt}%
\pgfpathmoveto{\pgfqpoint{2.205311in}{4.928557in}}%
\pgfpathlineto{\pgfqpoint{2.205311in}{4.928557in}}%
\pgfusepath{stroke}%
\end{pgfscope}%
\begin{pgfscope}%
\pgfpathrectangle{\pgfqpoint{1.250000in}{4.155455in}}{\pgfqpoint{2.279412in}{2.004545in}}%
\pgfusepath{clip}%
\pgfsetbuttcap%
\pgfsetroundjoin%
\pgfsetlinewidth{0.624445pt}%
\definecolor{currentstroke}{rgb}{0.260571,0.246922,0.522828}%
\pgfsetstrokecolor{currentstroke}%
\pgfsetdash{}{0pt}%
\pgfpathmoveto{\pgfqpoint{2.205311in}{4.928557in}}%
\pgfpathlineto{\pgfqpoint{2.188867in}{4.954081in}}%
\pgfusepath{stroke}%
\end{pgfscope}%
\begin{pgfscope}%
\pgfpathrectangle{\pgfqpoint{1.250000in}{4.155455in}}{\pgfqpoint{2.279412in}{2.004545in}}%
\pgfusepath{clip}%
\pgfsetbuttcap%
\pgfsetroundjoin%
\pgfsetlinewidth{0.588791pt}%
\definecolor{currentstroke}{rgb}{0.269308,0.218818,0.509577}%
\pgfsetstrokecolor{currentstroke}%
\pgfsetdash{}{0pt}%
\pgfpathmoveto{\pgfqpoint{2.188867in}{4.954081in}}%
\pgfpathlineto{\pgfqpoint{2.176716in}{4.978638in}}%
\pgfusepath{stroke}%
\end{pgfscope}%
\begin{pgfscope}%
\pgfpathrectangle{\pgfqpoint{1.250000in}{4.155455in}}{\pgfqpoint{2.279412in}{2.004545in}}%
\pgfusepath{clip}%
\pgfsetbuttcap%
\pgfsetroundjoin%
\pgfsetlinewidth{0.605817pt}%
\definecolor{currentstroke}{rgb}{0.265145,0.232956,0.516599}%
\pgfsetstrokecolor{currentstroke}%
\pgfsetdash{}{0pt}%
\pgfpathmoveto{\pgfqpoint{2.176716in}{4.978638in}}%
\pgfpathlineto{\pgfqpoint{2.176716in}{4.978638in}}%
\pgfusepath{stroke}%
\end{pgfscope}%
\begin{pgfscope}%
\pgfpathrectangle{\pgfqpoint{1.250000in}{4.155455in}}{\pgfqpoint{2.279412in}{2.004545in}}%
\pgfusepath{clip}%
\pgfsetbuttcap%
\pgfsetroundjoin%
\pgfsetlinewidth{0.605817pt}%
\definecolor{currentstroke}{rgb}{0.265145,0.232956,0.516599}%
\pgfsetstrokecolor{currentstroke}%
\pgfsetdash{}{0pt}%
\pgfpathmoveto{\pgfqpoint{2.176716in}{4.978638in}}%
\pgfpathlineto{\pgfqpoint{2.170915in}{5.002117in}}%
\pgfusepath{stroke}%
\end{pgfscope}%
\begin{pgfscope}%
\pgfpathrectangle{\pgfqpoint{1.250000in}{4.155455in}}{\pgfqpoint{2.279412in}{2.004545in}}%
\pgfusepath{clip}%
\pgfsetbuttcap%
\pgfsetroundjoin%
\pgfsetlinewidth{0.538949pt}%
\definecolor{currentstroke}{rgb}{0.277134,0.185228,0.489898}%
\pgfsetstrokecolor{currentstroke}%
\pgfsetdash{}{0pt}%
\pgfpathmoveto{\pgfqpoint{2.170915in}{5.002117in}}%
\pgfpathlineto{\pgfqpoint{2.166014in}{5.024319in}}%
\pgfusepath{stroke}%
\end{pgfscope}%
\begin{pgfscope}%
\pgfpathrectangle{\pgfqpoint{1.250000in}{4.155455in}}{\pgfqpoint{2.279412in}{2.004545in}}%
\pgfusepath{clip}%
\pgfsetbuttcap%
\pgfsetroundjoin%
\pgfsetlinewidth{0.488867pt}%
\definecolor{currentstroke}{rgb}{0.281887,0.150881,0.465405}%
\pgfsetstrokecolor{currentstroke}%
\pgfsetdash{}{0pt}%
\pgfpathmoveto{\pgfqpoint{2.166014in}{5.024319in}}%
\pgfpathlineto{\pgfqpoint{2.154033in}{5.052150in}}%
\pgfusepath{stroke}%
\end{pgfscope}%
\begin{pgfscope}%
\pgfpathrectangle{\pgfqpoint{1.250000in}{4.155455in}}{\pgfqpoint{2.279412in}{2.004545in}}%
\pgfusepath{clip}%
\pgfsetbuttcap%
\pgfsetroundjoin%
\pgfsetlinewidth{0.540836pt}%
\definecolor{currentstroke}{rgb}{0.277134,0.185228,0.489898}%
\pgfsetstrokecolor{currentstroke}%
\pgfsetdash{}{0pt}%
\pgfpathmoveto{\pgfqpoint{2.154033in}{5.052150in}}%
\pgfpathlineto{\pgfqpoint{2.154033in}{5.052150in}}%
\pgfusepath{stroke}%
\end{pgfscope}%
\begin{pgfscope}%
\pgfpathrectangle{\pgfqpoint{1.250000in}{4.155455in}}{\pgfqpoint{2.279412in}{2.004545in}}%
\pgfusepath{clip}%
\pgfsetbuttcap%
\pgfsetroundjoin%
\pgfsetlinewidth{0.540836pt}%
\definecolor{currentstroke}{rgb}{0.277134,0.185228,0.489898}%
\pgfsetstrokecolor{currentstroke}%
\pgfsetdash{}{0pt}%
\pgfpathmoveto{\pgfqpoint{2.154033in}{5.052150in}}%
\pgfpathlineto{\pgfqpoint{2.154033in}{5.052150in}}%
\pgfusepath{stroke}%
\end{pgfscope}%
\begin{pgfscope}%
\pgfpathrectangle{\pgfqpoint{1.250000in}{4.155455in}}{\pgfqpoint{2.279412in}{2.004545in}}%
\pgfusepath{clip}%
\pgfsetbuttcap%
\pgfsetroundjoin%
\pgfsetlinewidth{0.540836pt}%
\definecolor{currentstroke}{rgb}{0.277134,0.185228,0.489898}%
\pgfsetstrokecolor{currentstroke}%
\pgfsetdash{}{0pt}%
\pgfpathmoveto{\pgfqpoint{2.154033in}{5.052150in}}%
\pgfpathlineto{\pgfqpoint{2.150856in}{5.069324in}}%
\pgfusepath{stroke}%
\end{pgfscope}%
\begin{pgfscope}%
\pgfpathrectangle{\pgfqpoint{1.250000in}{4.155455in}}{\pgfqpoint{2.279412in}{2.004545in}}%
\pgfusepath{clip}%
\pgfsetbuttcap%
\pgfsetroundjoin%
\pgfsetlinewidth{0.503411pt}%
\definecolor{currentstroke}{rgb}{0.280868,0.160771,0.472899}%
\pgfsetstrokecolor{currentstroke}%
\pgfsetdash{}{0pt}%
\pgfpathmoveto{\pgfqpoint{2.150856in}{5.069324in}}%
\pgfpathlineto{\pgfqpoint{2.147243in}{5.086513in}}%
\pgfusepath{stroke}%
\end{pgfscope}%
\begin{pgfscope}%
\pgfpathrectangle{\pgfqpoint{1.250000in}{4.155455in}}{\pgfqpoint{2.279412in}{2.004545in}}%
\pgfusepath{clip}%
\pgfsetbuttcap%
\pgfsetroundjoin%
\pgfsetlinewidth{0.485904pt}%
\definecolor{currentstroke}{rgb}{0.282290,0.145912,0.461510}%
\pgfsetstrokecolor{currentstroke}%
\pgfsetdash{}{0pt}%
\pgfpathmoveto{\pgfqpoint{2.147243in}{5.086513in}}%
\pgfpathlineto{\pgfqpoint{2.147243in}{5.086513in}}%
\pgfusepath{stroke}%
\end{pgfscope}%
\begin{pgfscope}%
\pgfpathrectangle{\pgfqpoint{1.250000in}{4.155455in}}{\pgfqpoint{2.279412in}{2.004545in}}%
\pgfusepath{clip}%
\pgfsetbuttcap%
\pgfsetroundjoin%
\pgfsetlinewidth{0.320553pt}%
\definecolor{currentstroke}{rgb}{0.269944,0.014625,0.341379}%
\pgfsetstrokecolor{currentstroke}%
\pgfsetdash{}{0pt}%
\pgfpathmoveto{\pgfqpoint{3.204942in}{4.839585in}}%
\pgfpathlineto{\pgfqpoint{3.159084in}{4.841980in}}%
\pgfusepath{stroke}%
\end{pgfscope}%
\begin{pgfscope}%
\pgfpathrectangle{\pgfqpoint{1.250000in}{4.155455in}}{\pgfqpoint{2.279412in}{2.004545in}}%
\pgfusepath{clip}%
\pgfsetbuttcap%
\pgfsetroundjoin%
\pgfsetlinewidth{0.325248pt}%
\definecolor{currentstroke}{rgb}{0.271305,0.019942,0.347269}%
\pgfsetstrokecolor{currentstroke}%
\pgfsetdash{}{0pt}%
\pgfpathmoveto{\pgfqpoint{3.159084in}{4.841980in}}%
\pgfpathlineto{\pgfqpoint{3.108985in}{4.843554in}}%
\pgfusepath{stroke}%
\end{pgfscope}%
\begin{pgfscope}%
\pgfpathrectangle{\pgfqpoint{1.250000in}{4.155455in}}{\pgfqpoint{2.279412in}{2.004545in}}%
\pgfusepath{clip}%
\pgfsetbuttcap%
\pgfsetroundjoin%
\pgfsetlinewidth{0.327628pt}%
\definecolor{currentstroke}{rgb}{0.271305,0.019942,0.347269}%
\pgfsetstrokecolor{currentstroke}%
\pgfsetdash{}{0pt}%
\pgfpathmoveto{\pgfqpoint{3.108985in}{4.843554in}}%
\pgfpathlineto{\pgfqpoint{3.058865in}{4.844858in}}%
\pgfusepath{stroke}%
\end{pgfscope}%
\begin{pgfscope}%
\pgfpathrectangle{\pgfqpoint{1.250000in}{4.155455in}}{\pgfqpoint{2.279412in}{2.004545in}}%
\pgfusepath{clip}%
\pgfsetbuttcap%
\pgfsetroundjoin%
\pgfsetlinewidth{0.338137pt}%
\definecolor{currentstroke}{rgb}{0.273809,0.031497,0.358853}%
\pgfsetstrokecolor{currentstroke}%
\pgfsetdash{}{0pt}%
\pgfpathmoveto{\pgfqpoint{3.058865in}{4.844858in}}%
\pgfpathlineto{\pgfqpoint{3.008771in}{4.846819in}}%
\pgfusepath{stroke}%
\end{pgfscope}%
\begin{pgfscope}%
\pgfpathrectangle{\pgfqpoint{1.250000in}{4.155455in}}{\pgfqpoint{2.279412in}{2.004545in}}%
\pgfusepath{clip}%
\pgfsetbuttcap%
\pgfsetroundjoin%
\pgfsetlinewidth{0.348726pt}%
\definecolor{currentstroke}{rgb}{0.274952,0.037752,0.364543}%
\pgfsetstrokecolor{currentstroke}%
\pgfsetdash{}{0pt}%
\pgfpathmoveto{\pgfqpoint{3.008771in}{4.846819in}}%
\pgfpathlineto{\pgfqpoint{2.958693in}{4.849081in}}%
\pgfusepath{stroke}%
\end{pgfscope}%
\begin{pgfscope}%
\pgfpathrectangle{\pgfqpoint{1.250000in}{4.155455in}}{\pgfqpoint{2.279412in}{2.004545in}}%
\pgfusepath{clip}%
\pgfsetbuttcap%
\pgfsetroundjoin%
\pgfsetlinewidth{0.382530pt}%
\definecolor{currentstroke}{rgb}{0.279566,0.067836,0.391917}%
\pgfsetstrokecolor{currentstroke}%
\pgfsetdash{}{0pt}%
\pgfpathmoveto{\pgfqpoint{2.958693in}{4.849081in}}%
\pgfpathlineto{\pgfqpoint{2.908663in}{4.852109in}}%
\pgfusepath{stroke}%
\end{pgfscope}%
\begin{pgfscope}%
\pgfpathrectangle{\pgfqpoint{1.250000in}{4.155455in}}{\pgfqpoint{2.279412in}{2.004545in}}%
\pgfusepath{clip}%
\pgfsetbuttcap%
\pgfsetroundjoin%
\pgfsetlinewidth{0.397028pt}%
\definecolor{currentstroke}{rgb}{0.280894,0.078907,0.402329}%
\pgfsetstrokecolor{currentstroke}%
\pgfsetdash{}{0pt}%
\pgfpathmoveto{\pgfqpoint{2.908663in}{4.852109in}}%
\pgfpathlineto{\pgfqpoint{2.858667in}{4.855576in}}%
\pgfusepath{stroke}%
\end{pgfscope}%
\begin{pgfscope}%
\pgfpathrectangle{\pgfqpoint{1.250000in}{4.155455in}}{\pgfqpoint{2.279412in}{2.004545in}}%
\pgfusepath{clip}%
\pgfsetbuttcap%
\pgfsetroundjoin%
\pgfsetlinewidth{0.431225pt}%
\definecolor{currentstroke}{rgb}{0.282910,0.105393,0.426902}%
\pgfsetstrokecolor{currentstroke}%
\pgfsetdash{}{0pt}%
\pgfpathmoveto{\pgfqpoint{2.858667in}{4.855576in}}%
\pgfpathlineto{\pgfqpoint{2.808679in}{4.859135in}}%
\pgfusepath{stroke}%
\end{pgfscope}%
\begin{pgfscope}%
\pgfpathrectangle{\pgfqpoint{1.250000in}{4.155455in}}{\pgfqpoint{2.279412in}{2.004545in}}%
\pgfusepath{clip}%
\pgfsetbuttcap%
\pgfsetroundjoin%
\pgfsetlinewidth{0.453274pt}%
\definecolor{currentstroke}{rgb}{0.283187,0.125848,0.444960}%
\pgfsetstrokecolor{currentstroke}%
\pgfsetdash{}{0pt}%
\pgfpathmoveto{\pgfqpoint{2.808679in}{4.859135in}}%
\pgfpathlineto{\pgfqpoint{2.758691in}{4.862684in}}%
\pgfusepath{stroke}%
\end{pgfscope}%
\begin{pgfscope}%
\pgfpathrectangle{\pgfqpoint{1.250000in}{4.155455in}}{\pgfqpoint{2.279412in}{2.004545in}}%
\pgfusepath{clip}%
\pgfsetbuttcap%
\pgfsetroundjoin%
\pgfsetlinewidth{0.505809pt}%
\definecolor{currentstroke}{rgb}{0.280868,0.160771,0.472899}%
\pgfsetstrokecolor{currentstroke}%
\pgfsetdash{}{0pt}%
\pgfpathmoveto{\pgfqpoint{2.758691in}{4.862684in}}%
\pgfpathlineto{\pgfqpoint{2.708720in}{4.866409in}}%
\pgfusepath{stroke}%
\end{pgfscope}%
\begin{pgfscope}%
\pgfpathrectangle{\pgfqpoint{1.250000in}{4.155455in}}{\pgfqpoint{2.279412in}{2.004545in}}%
\pgfusepath{clip}%
\pgfsetbuttcap%
\pgfsetroundjoin%
\pgfsetlinewidth{0.548635pt}%
\definecolor{currentstroke}{rgb}{0.275191,0.194905,0.496005}%
\pgfsetstrokecolor{currentstroke}%
\pgfsetdash{}{0pt}%
\pgfpathmoveto{\pgfqpoint{2.708720in}{4.866409in}}%
\pgfpathlineto{\pgfqpoint{2.658790in}{4.870540in}}%
\pgfusepath{stroke}%
\end{pgfscope}%
\begin{pgfscope}%
\pgfpathrectangle{\pgfqpoint{1.250000in}{4.155455in}}{\pgfqpoint{2.279412in}{2.004545in}}%
\pgfusepath{clip}%
\pgfsetbuttcap%
\pgfsetroundjoin%
\pgfsetlinewidth{0.592659pt}%
\definecolor{currentstroke}{rgb}{0.267968,0.223549,0.512008}%
\pgfsetstrokecolor{currentstroke}%
\pgfsetdash{}{0pt}%
\pgfpathmoveto{\pgfqpoint{2.658790in}{4.870540in}}%
\pgfpathlineto{\pgfqpoint{2.608905in}{4.875072in}}%
\pgfusepath{stroke}%
\end{pgfscope}%
\begin{pgfscope}%
\pgfpathrectangle{\pgfqpoint{1.250000in}{4.155455in}}{\pgfqpoint{2.279412in}{2.004545in}}%
\pgfusepath{clip}%
\pgfsetbuttcap%
\pgfsetroundjoin%
\pgfsetlinewidth{0.635047pt}%
\definecolor{currentstroke}{rgb}{0.258965,0.251537,0.524736}%
\pgfsetstrokecolor{currentstroke}%
\pgfsetdash{}{0pt}%
\pgfpathmoveto{\pgfqpoint{2.608905in}{4.875072in}}%
\pgfpathlineto{\pgfqpoint{2.559072in}{4.880024in}}%
\pgfusepath{stroke}%
\end{pgfscope}%
\begin{pgfscope}%
\pgfpathrectangle{\pgfqpoint{1.250000in}{4.155455in}}{\pgfqpoint{2.279412in}{2.004545in}}%
\pgfusepath{clip}%
\pgfsetbuttcap%
\pgfsetroundjoin%
\pgfsetlinewidth{0.674378pt}%
\definecolor{currentstroke}{rgb}{0.248629,0.278775,0.534556}%
\pgfsetstrokecolor{currentstroke}%
\pgfsetdash{}{0pt}%
\pgfpathmoveto{\pgfqpoint{2.559072in}{4.880024in}}%
\pgfpathlineto{\pgfqpoint{2.509324in}{4.885592in}}%
\pgfusepath{stroke}%
\end{pgfscope}%
\begin{pgfscope}%
\pgfpathrectangle{\pgfqpoint{1.250000in}{4.155455in}}{\pgfqpoint{2.279412in}{2.004545in}}%
\pgfusepath{clip}%
\pgfsetbuttcap%
\pgfsetroundjoin%
\pgfsetlinewidth{0.686240pt}%
\definecolor{currentstroke}{rgb}{0.244972,0.287675,0.537260}%
\pgfsetstrokecolor{currentstroke}%
\pgfsetdash{}{0pt}%
\pgfpathmoveto{\pgfqpoint{2.509324in}{4.885592in}}%
\pgfpathlineto{\pgfqpoint{2.459764in}{4.892303in}}%
\pgfusepath{stroke}%
\end{pgfscope}%
\begin{pgfscope}%
\pgfpathrectangle{\pgfqpoint{1.250000in}{4.155455in}}{\pgfqpoint{2.279412in}{2.004545in}}%
\pgfusepath{clip}%
\pgfsetbuttcap%
\pgfsetroundjoin%
\pgfsetlinewidth{0.712501pt}%
\definecolor{currentstroke}{rgb}{0.239346,0.300855,0.540844}%
\pgfsetstrokecolor{currentstroke}%
\pgfsetdash{}{0pt}%
\pgfpathmoveto{\pgfqpoint{2.459764in}{4.892303in}}%
\pgfpathlineto{\pgfqpoint{2.410427in}{4.900192in}}%
\pgfusepath{stroke}%
\end{pgfscope}%
\begin{pgfscope}%
\pgfpathrectangle{\pgfqpoint{1.250000in}{4.155455in}}{\pgfqpoint{2.279412in}{2.004545in}}%
\pgfusepath{clip}%
\pgfsetbuttcap%
\pgfsetroundjoin%
\pgfsetlinewidth{0.757959pt}%
\definecolor{currentstroke}{rgb}{0.225863,0.330805,0.547314}%
\pgfsetstrokecolor{currentstroke}%
\pgfsetdash{}{0pt}%
\pgfpathmoveto{\pgfqpoint{2.410427in}{4.900192in}}%
\pgfpathlineto{\pgfqpoint{2.361495in}{4.909765in}}%
\pgfusepath{stroke}%
\end{pgfscope}%
\begin{pgfscope}%
\pgfpathrectangle{\pgfqpoint{1.250000in}{4.155455in}}{\pgfqpoint{2.279412in}{2.004545in}}%
\pgfusepath{clip}%
\pgfsetbuttcap%
\pgfsetroundjoin%
\pgfsetlinewidth{0.708797pt}%
\definecolor{currentstroke}{rgb}{0.239346,0.300855,0.540844}%
\pgfsetstrokecolor{currentstroke}%
\pgfsetdash{}{0pt}%
\pgfpathmoveto{\pgfqpoint{2.361495in}{4.909765in}}%
\pgfpathlineto{\pgfqpoint{2.313425in}{4.922162in}}%
\pgfusepath{stroke}%
\end{pgfscope}%
\begin{pgfscope}%
\pgfpathrectangle{\pgfqpoint{1.250000in}{4.155455in}}{\pgfqpoint{2.279412in}{2.004545in}}%
\pgfusepath{clip}%
\pgfsetbuttcap%
\pgfsetroundjoin%
\pgfsetlinewidth{0.699175pt}%
\definecolor{currentstroke}{rgb}{0.241237,0.296485,0.539709}%
\pgfsetstrokecolor{currentstroke}%
\pgfsetdash{}{0pt}%
\pgfpathmoveto{\pgfqpoint{2.313425in}{4.922162in}}%
\pgfpathlineto{\pgfqpoint{2.266920in}{4.938403in}}%
\pgfusepath{stroke}%
\end{pgfscope}%
\begin{pgfscope}%
\pgfpathrectangle{\pgfqpoint{1.250000in}{4.155455in}}{\pgfqpoint{2.279412in}{2.004545in}}%
\pgfusepath{clip}%
\pgfsetbuttcap%
\pgfsetroundjoin%
\pgfsetlinewidth{0.646964pt}%
\definecolor{currentstroke}{rgb}{0.255645,0.260703,0.528312}%
\pgfsetstrokecolor{currentstroke}%
\pgfsetdash{}{0pt}%
\pgfpathmoveto{\pgfqpoint{2.266920in}{4.938403in}}%
\pgfpathlineto{\pgfqpoint{2.224271in}{4.960585in}}%
\pgfusepath{stroke}%
\end{pgfscope}%
\begin{pgfscope}%
\pgfpathrectangle{\pgfqpoint{1.250000in}{4.155455in}}{\pgfqpoint{2.279412in}{2.004545in}}%
\pgfusepath{clip}%
\pgfsetbuttcap%
\pgfsetroundjoin%
\pgfsetlinewidth{0.601611pt}%
\definecolor{currentstroke}{rgb}{0.266580,0.228262,0.514349}%
\pgfsetstrokecolor{currentstroke}%
\pgfsetdash{}{0pt}%
\pgfpathmoveto{\pgfqpoint{2.224271in}{4.960585in}}%
\pgfpathlineto{\pgfqpoint{2.224271in}{4.960585in}}%
\pgfusepath{stroke}%
\end{pgfscope}%
\begin{pgfscope}%
\pgfpathrectangle{\pgfqpoint{1.250000in}{4.155455in}}{\pgfqpoint{2.279412in}{2.004545in}}%
\pgfusepath{clip}%
\pgfsetbuttcap%
\pgfsetroundjoin%
\pgfsetlinewidth{0.330032pt}%
\definecolor{currentstroke}{rgb}{0.272594,0.025563,0.353093}%
\pgfsetstrokecolor{currentstroke}%
\pgfsetdash{}{0pt}%
\pgfpathmoveto{\pgfqpoint{3.159084in}{4.887087in}}%
\pgfpathlineto{\pgfqpoint{3.108958in}{4.888298in}}%
\pgfusepath{stroke}%
\end{pgfscope}%
\begin{pgfscope}%
\pgfpathrectangle{\pgfqpoint{1.250000in}{4.155455in}}{\pgfqpoint{2.279412in}{2.004545in}}%
\pgfusepath{clip}%
\pgfsetbuttcap%
\pgfsetroundjoin%
\pgfsetlinewidth{0.331108pt}%
\definecolor{currentstroke}{rgb}{0.272594,0.025563,0.353093}%
\pgfsetstrokecolor{currentstroke}%
\pgfsetdash{}{0pt}%
\pgfpathmoveto{\pgfqpoint{3.108958in}{4.888298in}}%
\pgfpathlineto{\pgfqpoint{3.058843in}{4.889946in}}%
\pgfusepath{stroke}%
\end{pgfscope}%
\begin{pgfscope}%
\pgfpathrectangle{\pgfqpoint{1.250000in}{4.155455in}}{\pgfqpoint{2.279412in}{2.004545in}}%
\pgfusepath{clip}%
\pgfsetbuttcap%
\pgfsetroundjoin%
\pgfsetlinewidth{0.343678pt}%
\definecolor{currentstroke}{rgb}{0.274952,0.037752,0.364543}%
\pgfsetstrokecolor{currentstroke}%
\pgfsetdash{}{0pt}%
\pgfpathmoveto{\pgfqpoint{3.058843in}{4.889946in}}%
\pgfpathlineto{\pgfqpoint{3.008757in}{4.892149in}}%
\pgfusepath{stroke}%
\end{pgfscope}%
\begin{pgfscope}%
\pgfpathrectangle{\pgfqpoint{1.250000in}{4.155455in}}{\pgfqpoint{2.279412in}{2.004545in}}%
\pgfusepath{clip}%
\pgfsetbuttcap%
\pgfsetroundjoin%
\pgfsetlinewidth{0.359608pt}%
\definecolor{currentstroke}{rgb}{0.277018,0.050344,0.375715}%
\pgfsetstrokecolor{currentstroke}%
\pgfsetdash{}{0pt}%
\pgfpathmoveto{\pgfqpoint{3.008757in}{4.892149in}}%
\pgfpathlineto{\pgfqpoint{2.958677in}{4.894420in}}%
\pgfusepath{stroke}%
\end{pgfscope}%
\begin{pgfscope}%
\pgfpathrectangle{\pgfqpoint{1.250000in}{4.155455in}}{\pgfqpoint{2.279412in}{2.004545in}}%
\pgfusepath{clip}%
\pgfsetbuttcap%
\pgfsetroundjoin%
\pgfsetlinewidth{0.390832pt}%
\definecolor{currentstroke}{rgb}{0.280894,0.078907,0.402329}%
\pgfsetstrokecolor{currentstroke}%
\pgfsetdash{}{0pt}%
\pgfpathmoveto{\pgfqpoint{2.958677in}{4.894420in}}%
\pgfpathlineto{\pgfqpoint{2.908595in}{4.896662in}}%
\pgfusepath{stroke}%
\end{pgfscope}%
\begin{pgfscope}%
\pgfpathrectangle{\pgfqpoint{1.250000in}{4.155455in}}{\pgfqpoint{2.279412in}{2.004545in}}%
\pgfusepath{clip}%
\pgfsetbuttcap%
\pgfsetroundjoin%
\pgfsetlinewidth{0.402445pt}%
\definecolor{currentstroke}{rgb}{0.281446,0.084320,0.407414}%
\pgfsetstrokecolor{currentstroke}%
\pgfsetdash{}{0pt}%
\pgfpathmoveto{\pgfqpoint{2.908595in}{4.896662in}}%
\pgfpathlineto{\pgfqpoint{2.858520in}{4.899098in}}%
\pgfusepath{stroke}%
\end{pgfscope}%
\begin{pgfscope}%
\pgfpathrectangle{\pgfqpoint{1.250000in}{4.155455in}}{\pgfqpoint{2.279412in}{2.004545in}}%
\pgfusepath{clip}%
\pgfsetbuttcap%
\pgfsetroundjoin%
\pgfsetlinewidth{0.433310pt}%
\definecolor{currentstroke}{rgb}{0.283091,0.110553,0.431554}%
\pgfsetstrokecolor{currentstroke}%
\pgfsetdash{}{0pt}%
\pgfpathmoveto{\pgfqpoint{2.858520in}{4.899098in}}%
\pgfpathlineto{\pgfqpoint{2.808441in}{4.901477in}}%
\pgfusepath{stroke}%
\end{pgfscope}%
\begin{pgfscope}%
\pgfpathrectangle{\pgfqpoint{1.250000in}{4.155455in}}{\pgfqpoint{2.279412in}{2.004545in}}%
\pgfusepath{clip}%
\pgfsetbuttcap%
\pgfsetroundjoin%
\pgfsetlinewidth{0.497023pt}%
\definecolor{currentstroke}{rgb}{0.281412,0.155834,0.469201}%
\pgfsetstrokecolor{currentstroke}%
\pgfsetdash{}{0pt}%
\pgfpathmoveto{\pgfqpoint{2.808441in}{4.901477in}}%
\pgfpathlineto{\pgfqpoint{2.758377in}{4.904084in}}%
\pgfusepath{stroke}%
\end{pgfscope}%
\begin{pgfscope}%
\pgfpathrectangle{\pgfqpoint{1.250000in}{4.155455in}}{\pgfqpoint{2.279412in}{2.004545in}}%
\pgfusepath{clip}%
\pgfsetbuttcap%
\pgfsetroundjoin%
\pgfsetlinewidth{0.314823pt}%
\definecolor{currentstroke}{rgb}{0.268510,0.009605,0.335427}%
\pgfsetstrokecolor{currentstroke}%
\pgfsetdash{}{0pt}%
\pgfpathmoveto{\pgfqpoint{3.159084in}{4.932193in}}%
\pgfpathlineto{\pgfqpoint{3.109409in}{4.935552in}}%
\pgfusepath{stroke}%
\end{pgfscope}%
\begin{pgfscope}%
\pgfpathrectangle{\pgfqpoint{1.250000in}{4.155455in}}{\pgfqpoint{2.279412in}{2.004545in}}%
\pgfusepath{clip}%
\pgfsetbuttcap%
\pgfsetroundjoin%
\pgfsetlinewidth{0.337719pt}%
\definecolor{currentstroke}{rgb}{0.273809,0.031497,0.358853}%
\pgfsetstrokecolor{currentstroke}%
\pgfsetdash{}{0pt}%
\pgfpathmoveto{\pgfqpoint{3.109409in}{4.935552in}}%
\pgfpathlineto{\pgfqpoint{3.059267in}{4.935832in}}%
\pgfusepath{stroke}%
\end{pgfscope}%
\begin{pgfscope}%
\pgfpathrectangle{\pgfqpoint{1.250000in}{4.155455in}}{\pgfqpoint{2.279412in}{2.004545in}}%
\pgfusepath{clip}%
\pgfsetbuttcap%
\pgfsetroundjoin%
\pgfsetlinewidth{0.343544pt}%
\definecolor{currentstroke}{rgb}{0.274952,0.037752,0.364543}%
\pgfsetstrokecolor{currentstroke}%
\pgfsetdash{}{0pt}%
\pgfpathmoveto{\pgfqpoint{3.059267in}{4.935832in}}%
\pgfpathlineto{\pgfqpoint{3.009164in}{4.937105in}}%
\pgfusepath{stroke}%
\end{pgfscope}%
\begin{pgfscope}%
\pgfpathrectangle{\pgfqpoint{1.250000in}{4.155455in}}{\pgfqpoint{2.279412in}{2.004545in}}%
\pgfusepath{clip}%
\pgfsetbuttcap%
\pgfsetroundjoin%
\pgfsetlinewidth{0.361672pt}%
\definecolor{currentstroke}{rgb}{0.277018,0.050344,0.375715}%
\pgfsetstrokecolor{currentstroke}%
\pgfsetdash{}{0pt}%
\pgfpathmoveto{\pgfqpoint{3.009164in}{4.937105in}}%
\pgfpathlineto{\pgfqpoint{2.959079in}{4.939328in}}%
\pgfusepath{stroke}%
\end{pgfscope}%
\begin{pgfscope}%
\pgfpathrectangle{\pgfqpoint{1.250000in}{4.155455in}}{\pgfqpoint{2.279412in}{2.004545in}}%
\pgfusepath{clip}%
\pgfsetbuttcap%
\pgfsetroundjoin%
\pgfsetlinewidth{0.389859pt}%
\definecolor{currentstroke}{rgb}{0.280267,0.073417,0.397163}%
\pgfsetstrokecolor{currentstroke}%
\pgfsetdash{}{0pt}%
\pgfpathmoveto{\pgfqpoint{2.959079in}{4.939328in}}%
\pgfpathlineto{\pgfqpoint{2.908994in}{4.941551in}}%
\pgfusepath{stroke}%
\end{pgfscope}%
\begin{pgfscope}%
\pgfpathrectangle{\pgfqpoint{1.250000in}{4.155455in}}{\pgfqpoint{2.279412in}{2.004545in}}%
\pgfusepath{clip}%
\pgfsetbuttcap%
\pgfsetroundjoin%
\pgfsetlinewidth{0.426814pt}%
\definecolor{currentstroke}{rgb}{0.282910,0.105393,0.426902}%
\pgfsetstrokecolor{currentstroke}%
\pgfsetdash{}{0pt}%
\pgfpathmoveto{\pgfqpoint{2.908994in}{4.941551in}}%
\pgfpathlineto{\pgfqpoint{2.858912in}{4.943851in}}%
\pgfusepath{stroke}%
\end{pgfscope}%
\begin{pgfscope}%
\pgfpathrectangle{\pgfqpoint{1.250000in}{4.155455in}}{\pgfqpoint{2.279412in}{2.004545in}}%
\pgfusepath{clip}%
\pgfsetbuttcap%
\pgfsetroundjoin%
\pgfsetlinewidth{0.464310pt}%
\definecolor{currentstroke}{rgb}{0.283072,0.130895,0.449241}%
\pgfsetstrokecolor{currentstroke}%
\pgfsetdash{}{0pt}%
\pgfpathmoveto{\pgfqpoint{2.858912in}{4.943851in}}%
\pgfpathlineto{\pgfqpoint{2.808821in}{4.946028in}}%
\pgfusepath{stroke}%
\end{pgfscope}%
\begin{pgfscope}%
\pgfpathrectangle{\pgfqpoint{1.250000in}{4.155455in}}{\pgfqpoint{2.279412in}{2.004545in}}%
\pgfusepath{clip}%
\pgfsetbuttcap%
\pgfsetroundjoin%
\pgfsetlinewidth{0.517837pt}%
\definecolor{currentstroke}{rgb}{0.279574,0.170599,0.479997}%
\pgfsetstrokecolor{currentstroke}%
\pgfsetdash{}{0pt}%
\pgfpathmoveto{\pgfqpoint{2.808821in}{4.946028in}}%
\pgfpathlineto{\pgfqpoint{2.758742in}{4.948408in}}%
\pgfusepath{stroke}%
\end{pgfscope}%
\begin{pgfscope}%
\pgfpathrectangle{\pgfqpoint{1.250000in}{4.155455in}}{\pgfqpoint{2.279412in}{2.004545in}}%
\pgfusepath{clip}%
\pgfsetbuttcap%
\pgfsetroundjoin%
\pgfsetlinewidth{0.566584pt}%
\definecolor{currentstroke}{rgb}{0.273006,0.204520,0.501721}%
\pgfsetstrokecolor{currentstroke}%
\pgfsetdash{}{0pt}%
\pgfpathmoveto{\pgfqpoint{2.758742in}{4.948408in}}%
\pgfpathlineto{\pgfqpoint{2.708678in}{4.951013in}}%
\pgfusepath{stroke}%
\end{pgfscope}%
\begin{pgfscope}%
\pgfpathrectangle{\pgfqpoint{1.250000in}{4.155455in}}{\pgfqpoint{2.279412in}{2.004545in}}%
\pgfusepath{clip}%
\pgfsetbuttcap%
\pgfsetroundjoin%
\pgfsetlinewidth{0.629383pt}%
\definecolor{currentstroke}{rgb}{0.260571,0.246922,0.522828}%
\pgfsetstrokecolor{currentstroke}%
\pgfsetdash{}{0pt}%
\pgfpathmoveto{\pgfqpoint{2.708678in}{4.951013in}}%
\pgfpathlineto{\pgfqpoint{2.658630in}{4.953836in}}%
\pgfusepath{stroke}%
\end{pgfscope}%
\begin{pgfscope}%
\pgfpathrectangle{\pgfqpoint{1.250000in}{4.155455in}}{\pgfqpoint{2.279412in}{2.004545in}}%
\pgfusepath{clip}%
\pgfsetbuttcap%
\pgfsetroundjoin%
\pgfsetlinewidth{0.675546pt}%
\definecolor{currentstroke}{rgb}{0.248629,0.278775,0.534556}%
\pgfsetstrokecolor{currentstroke}%
\pgfsetdash{}{0pt}%
\pgfpathmoveto{\pgfqpoint{2.658630in}{4.953836in}}%
\pgfpathlineto{\pgfqpoint{2.608607in}{4.956972in}}%
\pgfusepath{stroke}%
\end{pgfscope}%
\begin{pgfscope}%
\pgfpathrectangle{\pgfqpoint{1.250000in}{4.155455in}}{\pgfqpoint{2.279412in}{2.004545in}}%
\pgfusepath{clip}%
\pgfsetbuttcap%
\pgfsetroundjoin%
\pgfsetlinewidth{0.758248pt}%
\definecolor{currentstroke}{rgb}{0.225863,0.330805,0.547314}%
\pgfsetstrokecolor{currentstroke}%
\pgfsetdash{}{0pt}%
\pgfpathmoveto{\pgfqpoint{2.608607in}{4.956972in}}%
\pgfpathlineto{\pgfqpoint{2.558629in}{4.960628in}}%
\pgfusepath{stroke}%
\end{pgfscope}%
\begin{pgfscope}%
\pgfpathrectangle{\pgfqpoint{1.250000in}{4.155455in}}{\pgfqpoint{2.279412in}{2.004545in}}%
\pgfusepath{clip}%
\pgfsetbuttcap%
\pgfsetroundjoin%
\pgfsetlinewidth{0.791234pt}%
\definecolor{currentstroke}{rgb}{0.216210,0.351535,0.550627}%
\pgfsetstrokecolor{currentstroke}%
\pgfsetdash{}{0pt}%
\pgfpathmoveto{\pgfqpoint{2.558629in}{4.960628in}}%
\pgfpathlineto{\pgfqpoint{2.508703in}{4.964804in}}%
\pgfusepath{stroke}%
\end{pgfscope}%
\begin{pgfscope}%
\pgfpathrectangle{\pgfqpoint{1.250000in}{4.155455in}}{\pgfqpoint{2.279412in}{2.004545in}}%
\pgfusepath{clip}%
\pgfsetbuttcap%
\pgfsetroundjoin%
\pgfsetlinewidth{0.816887pt}%
\definecolor{currentstroke}{rgb}{0.208623,0.367752,0.552675}%
\pgfsetstrokecolor{currentstroke}%
\pgfsetdash{}{0pt}%
\pgfpathmoveto{\pgfqpoint{2.508703in}{4.964804in}}%
\pgfpathlineto{\pgfqpoint{2.458868in}{4.969725in}}%
\pgfusepath{stroke}%
\end{pgfscope}%
\begin{pgfscope}%
\pgfpathrectangle{\pgfqpoint{1.250000in}{4.155455in}}{\pgfqpoint{2.279412in}{2.004545in}}%
\pgfusepath{clip}%
\pgfsetbuttcap%
\pgfsetroundjoin%
\pgfsetlinewidth{0.818276pt}%
\definecolor{currentstroke}{rgb}{0.208623,0.367752,0.552675}%
\pgfsetstrokecolor{currentstroke}%
\pgfsetdash{}{0pt}%
\pgfpathmoveto{\pgfqpoint{2.458868in}{4.969725in}}%
\pgfpathlineto{\pgfqpoint{2.409204in}{4.975824in}}%
\pgfusepath{stroke}%
\end{pgfscope}%
\begin{pgfscope}%
\pgfpathrectangle{\pgfqpoint{1.250000in}{4.155455in}}{\pgfqpoint{2.279412in}{2.004545in}}%
\pgfusepath{clip}%
\pgfsetbuttcap%
\pgfsetroundjoin%
\pgfsetlinewidth{0.819577pt}%
\definecolor{currentstroke}{rgb}{0.208623,0.367752,0.552675}%
\pgfsetstrokecolor{currentstroke}%
\pgfsetdash{}{0pt}%
\pgfpathmoveto{\pgfqpoint{2.409204in}{4.975824in}}%
\pgfpathlineto{\pgfqpoint{2.359822in}{4.983452in}}%
\pgfusepath{stroke}%
\end{pgfscope}%
\begin{pgfscope}%
\pgfpathrectangle{\pgfqpoint{1.250000in}{4.155455in}}{\pgfqpoint{2.279412in}{2.004545in}}%
\pgfusepath{clip}%
\pgfsetbuttcap%
\pgfsetroundjoin%
\pgfsetlinewidth{0.793900pt}%
\definecolor{currentstroke}{rgb}{0.216210,0.351535,0.550627}%
\pgfsetstrokecolor{currentstroke}%
\pgfsetdash{}{0pt}%
\pgfpathmoveto{\pgfqpoint{2.359822in}{4.983452in}}%
\pgfpathlineto{\pgfqpoint{2.310767in}{4.992566in}}%
\pgfusepath{stroke}%
\end{pgfscope}%
\begin{pgfscope}%
\pgfpathrectangle{\pgfqpoint{1.250000in}{4.155455in}}{\pgfqpoint{2.279412in}{2.004545in}}%
\pgfusepath{clip}%
\pgfsetbuttcap%
\pgfsetroundjoin%
\pgfsetlinewidth{0.722427pt}%
\definecolor{currentstroke}{rgb}{0.235526,0.309527,0.542944}%
\pgfsetstrokecolor{currentstroke}%
\pgfsetdash{}{0pt}%
\pgfpathmoveto{\pgfqpoint{2.310767in}{4.992566in}}%
\pgfpathlineto{\pgfqpoint{2.262434in}{5.004133in}}%
\pgfusepath{stroke}%
\end{pgfscope}%
\begin{pgfscope}%
\pgfpathrectangle{\pgfqpoint{1.250000in}{4.155455in}}{\pgfqpoint{2.279412in}{2.004545in}}%
\pgfusepath{clip}%
\pgfsetbuttcap%
\pgfsetroundjoin%
\pgfsetlinewidth{0.694980pt}%
\definecolor{currentstroke}{rgb}{0.243113,0.292092,0.538516}%
\pgfsetstrokecolor{currentstroke}%
\pgfsetdash{}{0pt}%
\pgfpathmoveto{\pgfqpoint{2.262434in}{5.004133in}}%
\pgfpathlineto{\pgfqpoint{2.216337in}{5.020835in}}%
\pgfusepath{stroke}%
\end{pgfscope}%
\begin{pgfscope}%
\pgfpathrectangle{\pgfqpoint{1.250000in}{4.155455in}}{\pgfqpoint{2.279412in}{2.004545in}}%
\pgfusepath{clip}%
\pgfsetbuttcap%
\pgfsetroundjoin%
\pgfsetlinewidth{0.559614pt}%
\definecolor{currentstroke}{rgb}{0.274128,0.199721,0.498911}%
\pgfsetstrokecolor{currentstroke}%
\pgfsetdash{}{0pt}%
\pgfpathmoveto{\pgfqpoint{2.216337in}{5.020835in}}%
\pgfpathlineto{\pgfqpoint{2.216337in}{5.020835in}}%
\pgfusepath{stroke}%
\end{pgfscope}%
\begin{pgfscope}%
\pgfpathrectangle{\pgfqpoint{1.250000in}{4.155455in}}{\pgfqpoint{2.279412in}{2.004545in}}%
\pgfusepath{clip}%
\pgfsetbuttcap%
\pgfsetroundjoin%
\pgfsetlinewidth{0.333280pt}%
\definecolor{currentstroke}{rgb}{0.272594,0.025563,0.353093}%
\pgfsetstrokecolor{currentstroke}%
\pgfsetdash{}{0pt}%
\pgfpathmoveto{\pgfqpoint{3.159084in}{4.977300in}}%
\pgfpathlineto{\pgfqpoint{3.108936in}{4.977675in}}%
\pgfusepath{stroke}%
\end{pgfscope}%
\begin{pgfscope}%
\pgfpathrectangle{\pgfqpoint{1.250000in}{4.155455in}}{\pgfqpoint{2.279412in}{2.004545in}}%
\pgfusepath{clip}%
\pgfsetbuttcap%
\pgfsetroundjoin%
\pgfsetlinewidth{0.334996pt}%
\definecolor{currentstroke}{rgb}{0.272594,0.025563,0.353093}%
\pgfsetstrokecolor{currentstroke}%
\pgfsetdash{}{0pt}%
\pgfpathmoveto{\pgfqpoint{3.108936in}{4.977675in}}%
\pgfpathlineto{\pgfqpoint{3.058813in}{4.978908in}}%
\pgfusepath{stroke}%
\end{pgfscope}%
\begin{pgfscope}%
\pgfpathrectangle{\pgfqpoint{1.250000in}{4.155455in}}{\pgfqpoint{2.279412in}{2.004545in}}%
\pgfusepath{clip}%
\pgfsetbuttcap%
\pgfsetroundjoin%
\pgfsetlinewidth{0.342670pt}%
\definecolor{currentstroke}{rgb}{0.274952,0.037752,0.364543}%
\pgfsetstrokecolor{currentstroke}%
\pgfsetdash{}{0pt}%
\pgfpathmoveto{\pgfqpoint{3.058813in}{4.978908in}}%
\pgfpathlineto{\pgfqpoint{3.008718in}{4.980798in}}%
\pgfusepath{stroke}%
\end{pgfscope}%
\begin{pgfscope}%
\pgfpathrectangle{\pgfqpoint{1.250000in}{4.155455in}}{\pgfqpoint{2.279412in}{2.004545in}}%
\pgfusepath{clip}%
\pgfsetbuttcap%
\pgfsetroundjoin%
\pgfsetlinewidth{0.376266pt}%
\definecolor{currentstroke}{rgb}{0.278791,0.062145,0.386592}%
\pgfsetstrokecolor{currentstroke}%
\pgfsetdash{}{0pt}%
\pgfpathmoveto{\pgfqpoint{3.008718in}{4.980798in}}%
\pgfpathlineto{\pgfqpoint{2.958625in}{4.982764in}}%
\pgfusepath{stroke}%
\end{pgfscope}%
\begin{pgfscope}%
\pgfpathrectangle{\pgfqpoint{1.250000in}{4.155455in}}{\pgfqpoint{2.279412in}{2.004545in}}%
\pgfusepath{clip}%
\pgfsetbuttcap%
\pgfsetroundjoin%
\pgfsetlinewidth{0.406572pt}%
\definecolor{currentstroke}{rgb}{0.281924,0.089666,0.412415}%
\pgfsetstrokecolor{currentstroke}%
\pgfsetdash{}{0pt}%
\pgfpathmoveto{\pgfqpoint{2.958625in}{4.982764in}}%
\pgfpathlineto{\pgfqpoint{2.908533in}{4.984916in}}%
\pgfusepath{stroke}%
\end{pgfscope}%
\begin{pgfscope}%
\pgfpathrectangle{\pgfqpoint{1.250000in}{4.155455in}}{\pgfqpoint{2.279412in}{2.004545in}}%
\pgfusepath{clip}%
\pgfsetbuttcap%
\pgfsetroundjoin%
\pgfsetlinewidth{0.434830pt}%
\definecolor{currentstroke}{rgb}{0.283091,0.110553,0.431554}%
\pgfsetstrokecolor{currentstroke}%
\pgfsetdash{}{0pt}%
\pgfpathmoveto{\pgfqpoint{2.908533in}{4.984916in}}%
\pgfpathlineto{\pgfqpoint{2.858441in}{4.987066in}}%
\pgfusepath{stroke}%
\end{pgfscope}%
\begin{pgfscope}%
\pgfpathrectangle{\pgfqpoint{1.250000in}{4.155455in}}{\pgfqpoint{2.279412in}{2.004545in}}%
\pgfusepath{clip}%
\pgfsetbuttcap%
\pgfsetroundjoin%
\pgfsetlinewidth{0.484595pt}%
\definecolor{currentstroke}{rgb}{0.282290,0.145912,0.461510}%
\pgfsetstrokecolor{currentstroke}%
\pgfsetdash{}{0pt}%
\pgfpathmoveto{\pgfqpoint{2.858441in}{4.987066in}}%
\pgfpathlineto{\pgfqpoint{2.808346in}{4.989150in}}%
\pgfusepath{stroke}%
\end{pgfscope}%
\begin{pgfscope}%
\pgfpathrectangle{\pgfqpoint{1.250000in}{4.155455in}}{\pgfqpoint{2.279412in}{2.004545in}}%
\pgfusepath{clip}%
\pgfsetbuttcap%
\pgfsetroundjoin%
\pgfsetlinewidth{0.531508pt}%
\definecolor{currentstroke}{rgb}{0.278012,0.180367,0.486697}%
\pgfsetstrokecolor{currentstroke}%
\pgfsetdash{}{0pt}%
\pgfpathmoveto{\pgfqpoint{2.808346in}{4.989150in}}%
\pgfpathlineto{\pgfqpoint{2.758257in}{4.991351in}}%
\pgfusepath{stroke}%
\end{pgfscope}%
\begin{pgfscope}%
\pgfpathrectangle{\pgfqpoint{1.250000in}{4.155455in}}{\pgfqpoint{2.279412in}{2.004545in}}%
\pgfusepath{clip}%
\pgfsetbuttcap%
\pgfsetroundjoin%
\pgfsetlinewidth{0.607451pt}%
\definecolor{currentstroke}{rgb}{0.265145,0.232956,0.516599}%
\pgfsetstrokecolor{currentstroke}%
\pgfsetdash{}{0pt}%
\pgfpathmoveto{\pgfqpoint{2.758257in}{4.991351in}}%
\pgfpathlineto{\pgfqpoint{2.708177in}{4.993698in}}%
\pgfusepath{stroke}%
\end{pgfscope}%
\begin{pgfscope}%
\pgfpathrectangle{\pgfqpoint{1.250000in}{4.155455in}}{\pgfqpoint{2.279412in}{2.004545in}}%
\pgfusepath{clip}%
\pgfsetbuttcap%
\pgfsetroundjoin%
\pgfsetlinewidth{0.682555pt}%
\definecolor{currentstroke}{rgb}{0.246811,0.283237,0.535941}%
\pgfsetstrokecolor{currentstroke}%
\pgfsetdash{}{0pt}%
\pgfpathmoveto{\pgfqpoint{2.708177in}{4.993698in}}%
\pgfpathlineto{\pgfqpoint{2.658111in}{4.996286in}}%
\pgfusepath{stroke}%
\end{pgfscope}%
\begin{pgfscope}%
\pgfpathrectangle{\pgfqpoint{1.250000in}{4.155455in}}{\pgfqpoint{2.279412in}{2.004545in}}%
\pgfusepath{clip}%
\pgfsetbuttcap%
\pgfsetroundjoin%
\pgfsetlinewidth{0.333297pt}%
\definecolor{currentstroke}{rgb}{0.272594,0.025563,0.353093}%
\pgfsetstrokecolor{currentstroke}%
\pgfsetdash{}{0pt}%
\pgfpathmoveto{\pgfqpoint{3.159084in}{5.112620in}}%
\pgfpathlineto{\pgfqpoint{3.108941in}{5.112868in}}%
\pgfusepath{stroke}%
\end{pgfscope}%
\begin{pgfscope}%
\pgfpathrectangle{\pgfqpoint{1.250000in}{4.155455in}}{\pgfqpoint{2.279412in}{2.004545in}}%
\pgfusepath{clip}%
\pgfsetbuttcap%
\pgfsetroundjoin%
\pgfsetlinewidth{0.334709pt}%
\definecolor{currentstroke}{rgb}{0.272594,0.025563,0.353093}%
\pgfsetstrokecolor{currentstroke}%
\pgfsetdash{}{0pt}%
\pgfpathmoveto{\pgfqpoint{3.108941in}{5.112868in}}%
\pgfpathlineto{\pgfqpoint{3.058795in}{5.113316in}}%
\pgfusepath{stroke}%
\end{pgfscope}%
\begin{pgfscope}%
\pgfpathrectangle{\pgfqpoint{1.250000in}{4.155455in}}{\pgfqpoint{2.279412in}{2.004545in}}%
\pgfusepath{clip}%
\pgfsetbuttcap%
\pgfsetroundjoin%
\pgfsetlinewidth{0.352597pt}%
\definecolor{currentstroke}{rgb}{0.276022,0.044167,0.370164}%
\pgfsetstrokecolor{currentstroke}%
\pgfsetdash{}{0pt}%
\pgfpathmoveto{\pgfqpoint{3.058795in}{5.113316in}}%
\pgfpathlineto{\pgfqpoint{3.008644in}{5.113531in}}%
\pgfusepath{stroke}%
\end{pgfscope}%
\begin{pgfscope}%
\pgfpathrectangle{\pgfqpoint{1.250000in}{4.155455in}}{\pgfqpoint{2.279412in}{2.004545in}}%
\pgfusepath{clip}%
\pgfsetbuttcap%
\pgfsetroundjoin%
\pgfsetlinewidth{0.375216pt}%
\definecolor{currentstroke}{rgb}{0.278791,0.062145,0.386592}%
\pgfsetstrokecolor{currentstroke}%
\pgfsetdash{}{0pt}%
\pgfpathmoveto{\pgfqpoint{3.008644in}{5.113531in}}%
\pgfpathlineto{\pgfqpoint{2.958492in}{5.113639in}}%
\pgfusepath{stroke}%
\end{pgfscope}%
\begin{pgfscope}%
\pgfpathrectangle{\pgfqpoint{1.250000in}{4.155455in}}{\pgfqpoint{2.279412in}{2.004545in}}%
\pgfusepath{clip}%
\pgfsetbuttcap%
\pgfsetroundjoin%
\pgfsetlinewidth{0.422357pt}%
\definecolor{currentstroke}{rgb}{0.282656,0.100196,0.422160}%
\pgfsetstrokecolor{currentstroke}%
\pgfsetdash{}{0pt}%
\pgfpathmoveto{\pgfqpoint{2.958492in}{5.113639in}}%
\pgfpathlineto{\pgfqpoint{2.908341in}{5.113871in}}%
\pgfusepath{stroke}%
\end{pgfscope}%
\begin{pgfscope}%
\pgfpathrectangle{\pgfqpoint{1.250000in}{4.155455in}}{\pgfqpoint{2.279412in}{2.004545in}}%
\pgfusepath{clip}%
\pgfsetbuttcap%
\pgfsetroundjoin%
\pgfsetlinewidth{0.462827pt}%
\definecolor{currentstroke}{rgb}{0.283072,0.130895,0.449241}%
\pgfsetstrokecolor{currentstroke}%
\pgfsetdash{}{0pt}%
\pgfpathmoveto{\pgfqpoint{2.908341in}{5.113871in}}%
\pgfpathlineto{\pgfqpoint{2.858191in}{5.114157in}}%
\pgfusepath{stroke}%
\end{pgfscope}%
\begin{pgfscope}%
\pgfpathrectangle{\pgfqpoint{1.250000in}{4.155455in}}{\pgfqpoint{2.279412in}{2.004545in}}%
\pgfusepath{clip}%
\pgfsetbuttcap%
\pgfsetroundjoin%
\pgfsetlinewidth{0.519326pt}%
\definecolor{currentstroke}{rgb}{0.279574,0.170599,0.479997}%
\pgfsetstrokecolor{currentstroke}%
\pgfsetdash{}{0pt}%
\pgfpathmoveto{\pgfqpoint{2.858191in}{5.114157in}}%
\pgfpathlineto{\pgfqpoint{2.808041in}{5.114504in}}%
\pgfusepath{stroke}%
\end{pgfscope}%
\begin{pgfscope}%
\pgfpathrectangle{\pgfqpoint{1.250000in}{4.155455in}}{\pgfqpoint{2.279412in}{2.004545in}}%
\pgfusepath{clip}%
\pgfsetbuttcap%
\pgfsetroundjoin%
\pgfsetlinewidth{0.577849pt}%
\definecolor{currentstroke}{rgb}{0.270595,0.214069,0.507052}%
\pgfsetstrokecolor{currentstroke}%
\pgfsetdash{}{0pt}%
\pgfpathmoveto{\pgfqpoint{2.808041in}{5.114504in}}%
\pgfpathlineto{\pgfqpoint{2.757892in}{5.114974in}}%
\pgfusepath{stroke}%
\end{pgfscope}%
\begin{pgfscope}%
\pgfpathrectangle{\pgfqpoint{1.250000in}{4.155455in}}{\pgfqpoint{2.279412in}{2.004545in}}%
\pgfusepath{clip}%
\pgfsetbuttcap%
\pgfsetroundjoin%
\pgfsetlinewidth{0.662941pt}%
\definecolor{currentstroke}{rgb}{0.252194,0.269783,0.531579}%
\pgfsetstrokecolor{currentstroke}%
\pgfsetdash{}{0pt}%
\pgfpathmoveto{\pgfqpoint{2.757892in}{5.114974in}}%
\pgfpathlineto{\pgfqpoint{2.707742in}{5.115371in}}%
\pgfusepath{stroke}%
\end{pgfscope}%
\begin{pgfscope}%
\pgfpathrectangle{\pgfqpoint{1.250000in}{4.155455in}}{\pgfqpoint{2.279412in}{2.004545in}}%
\pgfusepath{clip}%
\pgfsetbuttcap%
\pgfsetroundjoin%
\pgfsetlinewidth{0.751400pt}%
\definecolor{currentstroke}{rgb}{0.227802,0.326594,0.546532}%
\pgfsetstrokecolor{currentstroke}%
\pgfsetdash{}{0pt}%
\pgfpathmoveto{\pgfqpoint{2.707742in}{5.115371in}}%
\pgfpathlineto{\pgfqpoint{2.657593in}{5.115848in}}%
\pgfusepath{stroke}%
\end{pgfscope}%
\begin{pgfscope}%
\pgfpathrectangle{\pgfqpoint{1.250000in}{4.155455in}}{\pgfqpoint{2.279412in}{2.004545in}}%
\pgfusepath{clip}%
\pgfsetbuttcap%
\pgfsetroundjoin%
\pgfsetlinewidth{0.809179pt}%
\definecolor{currentstroke}{rgb}{0.210503,0.363727,0.552206}%
\pgfsetstrokecolor{currentstroke}%
\pgfsetdash{}{0pt}%
\pgfpathmoveto{\pgfqpoint{2.657593in}{5.115848in}}%
\pgfpathlineto{\pgfqpoint{2.607446in}{5.116436in}}%
\pgfusepath{stroke}%
\end{pgfscope}%
\begin{pgfscope}%
\pgfpathrectangle{\pgfqpoint{1.250000in}{4.155455in}}{\pgfqpoint{2.279412in}{2.004545in}}%
\pgfusepath{clip}%
\pgfsetbuttcap%
\pgfsetroundjoin%
\pgfsetlinewidth{0.866505pt}%
\definecolor{currentstroke}{rgb}{0.195860,0.395433,0.555276}%
\pgfsetstrokecolor{currentstroke}%
\pgfsetdash{}{0pt}%
\pgfpathmoveto{\pgfqpoint{2.607446in}{5.116436in}}%
\pgfpathlineto{\pgfqpoint{2.557299in}{5.117046in}}%
\pgfusepath{stroke}%
\end{pgfscope}%
\begin{pgfscope}%
\pgfpathrectangle{\pgfqpoint{1.250000in}{4.155455in}}{\pgfqpoint{2.279412in}{2.004545in}}%
\pgfusepath{clip}%
\pgfsetbuttcap%
\pgfsetroundjoin%
\pgfsetlinewidth{0.920814pt}%
\definecolor{currentstroke}{rgb}{0.182256,0.426184,0.557120}%
\pgfsetstrokecolor{currentstroke}%
\pgfsetdash{}{0pt}%
\pgfpathmoveto{\pgfqpoint{2.557299in}{5.117046in}}%
\pgfpathlineto{\pgfqpoint{2.507154in}{5.117804in}}%
\pgfusepath{stroke}%
\end{pgfscope}%
\begin{pgfscope}%
\pgfpathrectangle{\pgfqpoint{1.250000in}{4.155455in}}{\pgfqpoint{2.279412in}{2.004545in}}%
\pgfusepath{clip}%
\pgfsetbuttcap%
\pgfsetroundjoin%
\pgfsetlinewidth{0.943248pt}%
\definecolor{currentstroke}{rgb}{0.177423,0.437527,0.557565}%
\pgfsetstrokecolor{currentstroke}%
\pgfsetdash{}{0pt}%
\pgfpathmoveto{\pgfqpoint{2.507154in}{5.117804in}}%
\pgfpathlineto{\pgfqpoint{2.457011in}{5.118623in}}%
\pgfusepath{stroke}%
\end{pgfscope}%
\begin{pgfscope}%
\pgfpathrectangle{\pgfqpoint{1.250000in}{4.155455in}}{\pgfqpoint{2.279412in}{2.004545in}}%
\pgfusepath{clip}%
\pgfsetbuttcap%
\pgfsetroundjoin%
\pgfsetlinewidth{0.885662pt}%
\definecolor{currentstroke}{rgb}{0.190631,0.407061,0.556089}%
\pgfsetstrokecolor{currentstroke}%
\pgfsetdash{}{0pt}%
\pgfpathmoveto{\pgfqpoint{2.457011in}{5.118623in}}%
\pgfpathlineto{\pgfqpoint{2.406872in}{5.119607in}}%
\pgfusepath{stroke}%
\end{pgfscope}%
\begin{pgfscope}%
\pgfpathrectangle{\pgfqpoint{1.250000in}{4.155455in}}{\pgfqpoint{2.279412in}{2.004545in}}%
\pgfusepath{clip}%
\pgfsetbuttcap%
\pgfsetroundjoin%
\pgfsetlinewidth{0.903143pt}%
\definecolor{currentstroke}{rgb}{0.187231,0.414746,0.556547}%
\pgfsetstrokecolor{currentstroke}%
\pgfsetdash{}{0pt}%
\pgfpathmoveto{\pgfqpoint{2.406872in}{5.119607in}}%
\pgfpathlineto{\pgfqpoint{2.356747in}{5.120968in}}%
\pgfusepath{stroke}%
\end{pgfscope}%
\begin{pgfscope}%
\pgfpathrectangle{\pgfqpoint{1.250000in}{4.155455in}}{\pgfqpoint{2.279412in}{2.004545in}}%
\pgfusepath{clip}%
\pgfsetbuttcap%
\pgfsetroundjoin%
\pgfsetlinewidth{0.804110pt}%
\definecolor{currentstroke}{rgb}{0.212395,0.359683,0.551710}%
\pgfsetstrokecolor{currentstroke}%
\pgfsetdash{}{0pt}%
\pgfpathmoveto{\pgfqpoint{2.356747in}{5.120968in}}%
\pgfpathlineto{\pgfqpoint{2.306629in}{5.122523in}}%
\pgfusepath{stroke}%
\end{pgfscope}%
\begin{pgfscope}%
\pgfpathrectangle{\pgfqpoint{1.250000in}{4.155455in}}{\pgfqpoint{2.279412in}{2.004545in}}%
\pgfusepath{clip}%
\pgfsetbuttcap%
\pgfsetroundjoin%
\pgfsetlinewidth{0.314223pt}%
\definecolor{currentstroke}{rgb}{0.268510,0.009605,0.335427}%
\pgfsetstrokecolor{currentstroke}%
\pgfsetdash{}{0pt}%
\pgfpathmoveto{\pgfqpoint{3.159084in}{5.428368in}}%
\pgfpathlineto{\pgfqpoint{3.109127in}{5.425100in}}%
\pgfusepath{stroke}%
\end{pgfscope}%
\begin{pgfscope}%
\pgfpathrectangle{\pgfqpoint{1.250000in}{4.155455in}}{\pgfqpoint{2.279412in}{2.004545in}}%
\pgfusepath{clip}%
\pgfsetbuttcap%
\pgfsetroundjoin%
\pgfsetlinewidth{0.331358pt}%
\definecolor{currentstroke}{rgb}{0.272594,0.025563,0.353093}%
\pgfsetstrokecolor{currentstroke}%
\pgfsetdash{}{0pt}%
\pgfpathmoveto{\pgfqpoint{3.109127in}{5.425100in}}%
\pgfpathlineto{\pgfqpoint{3.059124in}{5.421936in}}%
\pgfusepath{stroke}%
\end{pgfscope}%
\begin{pgfscope}%
\pgfpathrectangle{\pgfqpoint{1.250000in}{4.155455in}}{\pgfqpoint{2.279412in}{2.004545in}}%
\pgfusepath{clip}%
\pgfsetbuttcap%
\pgfsetroundjoin%
\pgfsetlinewidth{0.333971pt}%
\definecolor{currentstroke}{rgb}{0.272594,0.025563,0.353093}%
\pgfsetstrokecolor{currentstroke}%
\pgfsetdash{}{0pt}%
\pgfpathmoveto{\pgfqpoint{3.059124in}{5.421936in}}%
\pgfpathlineto{\pgfqpoint{3.009112in}{5.418861in}}%
\pgfusepath{stroke}%
\end{pgfscope}%
\begin{pgfscope}%
\pgfpathrectangle{\pgfqpoint{1.250000in}{4.155455in}}{\pgfqpoint{2.279412in}{2.004545in}}%
\pgfusepath{clip}%
\pgfsetbuttcap%
\pgfsetroundjoin%
\pgfsetlinewidth{0.365128pt}%
\definecolor{currentstroke}{rgb}{0.277941,0.056324,0.381191}%
\pgfsetstrokecolor{currentstroke}%
\pgfsetdash{}{0pt}%
\pgfpathmoveto{\pgfqpoint{3.009112in}{5.418861in}}%
\pgfpathlineto{\pgfqpoint{2.959024in}{5.416697in}}%
\pgfusepath{stroke}%
\end{pgfscope}%
\begin{pgfscope}%
\pgfpathrectangle{\pgfqpoint{1.250000in}{4.155455in}}{\pgfqpoint{2.279412in}{2.004545in}}%
\pgfusepath{clip}%
\pgfsetbuttcap%
\pgfsetroundjoin%
\pgfsetlinewidth{0.382935pt}%
\definecolor{currentstroke}{rgb}{0.279566,0.067836,0.391917}%
\pgfsetstrokecolor{currentstroke}%
\pgfsetdash{}{0pt}%
\pgfpathmoveto{\pgfqpoint{2.959024in}{5.416697in}}%
\pgfpathlineto{\pgfqpoint{2.908928in}{5.414642in}}%
\pgfusepath{stroke}%
\end{pgfscope}%
\begin{pgfscope}%
\pgfpathrectangle{\pgfqpoint{1.250000in}{4.155455in}}{\pgfqpoint{2.279412in}{2.004545in}}%
\pgfusepath{clip}%
\pgfsetbuttcap%
\pgfsetroundjoin%
\pgfsetlinewidth{0.428965pt}%
\definecolor{currentstroke}{rgb}{0.282910,0.105393,0.426902}%
\pgfsetstrokecolor{currentstroke}%
\pgfsetdash{}{0pt}%
\pgfpathmoveto{\pgfqpoint{2.908928in}{5.414642in}}%
\pgfpathlineto{\pgfqpoint{2.858875in}{5.411975in}}%
\pgfusepath{stroke}%
\end{pgfscope}%
\begin{pgfscope}%
\pgfpathrectangle{\pgfqpoint{1.250000in}{4.155455in}}{\pgfqpoint{2.279412in}{2.004545in}}%
\pgfusepath{clip}%
\pgfsetbuttcap%
\pgfsetroundjoin%
\pgfsetlinewidth{0.437536pt}%
\definecolor{currentstroke}{rgb}{0.283091,0.110553,0.431554}%
\pgfsetstrokecolor{currentstroke}%
\pgfsetdash{}{0pt}%
\pgfpathmoveto{\pgfqpoint{2.858875in}{5.411975in}}%
\pgfpathlineto{\pgfqpoint{2.808866in}{5.408656in}}%
\pgfusepath{stroke}%
\end{pgfscope}%
\begin{pgfscope}%
\pgfpathrectangle{\pgfqpoint{1.250000in}{4.155455in}}{\pgfqpoint{2.279412in}{2.004545in}}%
\pgfusepath{clip}%
\pgfsetbuttcap%
\pgfsetroundjoin%
\pgfsetlinewidth{0.310922pt}%
\definecolor{currentstroke}{rgb}{0.268510,0.009605,0.335427}%
\pgfsetstrokecolor{currentstroke}%
\pgfsetdash{}{0pt}%
\pgfpathmoveto{\pgfqpoint{3.159084in}{5.473475in}}%
\pgfpathlineto{\pgfqpoint{3.109012in}{5.472769in}}%
\pgfusepath{stroke}%
\end{pgfscope}%
\begin{pgfscope}%
\pgfpathrectangle{\pgfqpoint{1.250000in}{4.155455in}}{\pgfqpoint{2.279412in}{2.004545in}}%
\pgfusepath{clip}%
\pgfsetbuttcap%
\pgfsetroundjoin%
\pgfsetlinewidth{0.330315pt}%
\definecolor{currentstroke}{rgb}{0.272594,0.025563,0.353093}%
\pgfsetstrokecolor{currentstroke}%
\pgfsetdash{}{0pt}%
\pgfpathmoveto{\pgfqpoint{3.109012in}{5.472769in}}%
\pgfpathlineto{\pgfqpoint{3.058969in}{5.469975in}}%
\pgfusepath{stroke}%
\end{pgfscope}%
\begin{pgfscope}%
\pgfpathrectangle{\pgfqpoint{1.250000in}{4.155455in}}{\pgfqpoint{2.279412in}{2.004545in}}%
\pgfusepath{clip}%
\pgfsetbuttcap%
\pgfsetroundjoin%
\pgfsetlinewidth{0.334415pt}%
\definecolor{currentstroke}{rgb}{0.272594,0.025563,0.353093}%
\pgfsetstrokecolor{currentstroke}%
\pgfsetdash{}{0pt}%
\pgfpathmoveto{\pgfqpoint{3.058969in}{5.469975in}}%
\pgfpathlineto{\pgfqpoint{3.008937in}{5.466992in}}%
\pgfusepath{stroke}%
\end{pgfscope}%
\begin{pgfscope}%
\pgfpathrectangle{\pgfqpoint{1.250000in}{4.155455in}}{\pgfqpoint{2.279412in}{2.004545in}}%
\pgfusepath{clip}%
\pgfsetbuttcap%
\pgfsetroundjoin%
\pgfsetlinewidth{0.349184pt}%
\definecolor{currentstroke}{rgb}{0.276022,0.044167,0.370164}%
\pgfsetstrokecolor{currentstroke}%
\pgfsetdash{}{0pt}%
\pgfpathmoveto{\pgfqpoint{3.008937in}{5.466992in}}%
\pgfpathlineto{\pgfqpoint{2.958904in}{5.463984in}}%
\pgfusepath{stroke}%
\end{pgfscope}%
\begin{pgfscope}%
\pgfpathrectangle{\pgfqpoint{1.250000in}{4.155455in}}{\pgfqpoint{2.279412in}{2.004545in}}%
\pgfusepath{clip}%
\pgfsetbuttcap%
\pgfsetroundjoin%
\pgfsetlinewidth{0.362529pt}%
\definecolor{currentstroke}{rgb}{0.277018,0.050344,0.375715}%
\pgfsetstrokecolor{currentstroke}%
\pgfsetdash{}{0pt}%
\pgfpathmoveto{\pgfqpoint{2.958904in}{5.463984in}}%
\pgfpathlineto{\pgfqpoint{2.908846in}{5.461314in}}%
\pgfusepath{stroke}%
\end{pgfscope}%
\begin{pgfscope}%
\pgfpathrectangle{\pgfqpoint{1.250000in}{4.155455in}}{\pgfqpoint{2.279412in}{2.004545in}}%
\pgfusepath{clip}%
\pgfsetbuttcap%
\pgfsetroundjoin%
\pgfsetlinewidth{0.394136pt}%
\definecolor{currentstroke}{rgb}{0.280894,0.078907,0.402329}%
\pgfsetstrokecolor{currentstroke}%
\pgfsetdash{}{0pt}%
\pgfpathmoveto{\pgfqpoint{2.908846in}{5.461314in}}%
\pgfpathlineto{\pgfqpoint{2.858825in}{5.458205in}}%
\pgfusepath{stroke}%
\end{pgfscope}%
\begin{pgfscope}%
\pgfpathrectangle{\pgfqpoint{1.250000in}{4.155455in}}{\pgfqpoint{2.279412in}{2.004545in}}%
\pgfusepath{clip}%
\pgfsetbuttcap%
\pgfsetroundjoin%
\pgfsetlinewidth{0.418513pt}%
\definecolor{currentstroke}{rgb}{0.282656,0.100196,0.422160}%
\pgfsetstrokecolor{currentstroke}%
\pgfsetdash{}{0pt}%
\pgfpathmoveto{\pgfqpoint{2.858825in}{5.458205in}}%
\pgfpathlineto{\pgfqpoint{2.808863in}{5.454376in}}%
\pgfusepath{stroke}%
\end{pgfscope}%
\begin{pgfscope}%
\pgfpathrectangle{\pgfqpoint{1.250000in}{4.155455in}}{\pgfqpoint{2.279412in}{2.004545in}}%
\pgfusepath{clip}%
\pgfsetbuttcap%
\pgfsetroundjoin%
\pgfsetlinewidth{0.455375pt}%
\definecolor{currentstroke}{rgb}{0.283187,0.125848,0.444960}%
\pgfsetstrokecolor{currentstroke}%
\pgfsetdash{}{0pt}%
\pgfpathmoveto{\pgfqpoint{2.808863in}{5.454376in}}%
\pgfpathlineto{\pgfqpoint{2.758911in}{5.450443in}}%
\pgfusepath{stroke}%
\end{pgfscope}%
\begin{pgfscope}%
\pgfpathrectangle{\pgfqpoint{1.250000in}{4.155455in}}{\pgfqpoint{2.279412in}{2.004545in}}%
\pgfusepath{clip}%
\pgfsetbuttcap%
\pgfsetroundjoin%
\pgfsetlinewidth{0.497820pt}%
\definecolor{currentstroke}{rgb}{0.281412,0.155834,0.469201}%
\pgfsetstrokecolor{currentstroke}%
\pgfsetdash{}{0pt}%
\pgfpathmoveto{\pgfqpoint{2.758911in}{5.450443in}}%
\pgfpathlineto{\pgfqpoint{2.708979in}{5.446325in}}%
\pgfusepath{stroke}%
\end{pgfscope}%
\begin{pgfscope}%
\pgfpathrectangle{\pgfqpoint{1.250000in}{4.155455in}}{\pgfqpoint{2.279412in}{2.004545in}}%
\pgfusepath{clip}%
\pgfsetbuttcap%
\pgfsetroundjoin%
\pgfsetlinewidth{0.551962pt}%
\definecolor{currentstroke}{rgb}{0.275191,0.194905,0.496005}%
\pgfsetstrokecolor{currentstroke}%
\pgfsetdash{}{0pt}%
\pgfpathmoveto{\pgfqpoint{2.708979in}{5.446325in}}%
\pgfpathlineto{\pgfqpoint{2.659078in}{5.441920in}}%
\pgfusepath{stroke}%
\end{pgfscope}%
\begin{pgfscope}%
\pgfpathrectangle{\pgfqpoint{1.250000in}{4.155455in}}{\pgfqpoint{2.279412in}{2.004545in}}%
\pgfusepath{clip}%
\pgfsetbuttcap%
\pgfsetroundjoin%
\pgfsetlinewidth{0.599244pt}%
\definecolor{currentstroke}{rgb}{0.266580,0.228262,0.514349}%
\pgfsetstrokecolor{currentstroke}%
\pgfsetdash{}{0pt}%
\pgfpathmoveto{\pgfqpoint{2.659078in}{5.441920in}}%
\pgfpathlineto{\pgfqpoint{2.609230in}{5.437083in}}%
\pgfusepath{stroke}%
\end{pgfscope}%
\begin{pgfscope}%
\pgfpathrectangle{\pgfqpoint{1.250000in}{4.155455in}}{\pgfqpoint{2.279412in}{2.004545in}}%
\pgfusepath{clip}%
\pgfsetbuttcap%
\pgfsetroundjoin%
\pgfsetlinewidth{0.639855pt}%
\definecolor{currentstroke}{rgb}{0.257322,0.256130,0.526563}%
\pgfsetstrokecolor{currentstroke}%
\pgfsetdash{}{0pt}%
\pgfpathmoveto{\pgfqpoint{2.609230in}{5.437083in}}%
\pgfpathlineto{\pgfqpoint{2.559459in}{5.431670in}}%
\pgfusepath{stroke}%
\end{pgfscope}%
\begin{pgfscope}%
\pgfpathrectangle{\pgfqpoint{1.250000in}{4.155455in}}{\pgfqpoint{2.279412in}{2.004545in}}%
\pgfusepath{clip}%
\pgfsetbuttcap%
\pgfsetroundjoin%
\pgfsetlinewidth{0.685214pt}%
\definecolor{currentstroke}{rgb}{0.246811,0.283237,0.535941}%
\pgfsetstrokecolor{currentstroke}%
\pgfsetdash{}{0pt}%
\pgfpathmoveto{\pgfqpoint{2.559459in}{5.431670in}}%
\pgfpathlineto{\pgfqpoint{2.509792in}{5.425575in}}%
\pgfusepath{stroke}%
\end{pgfscope}%
\begin{pgfscope}%
\pgfpathrectangle{\pgfqpoint{1.250000in}{4.155455in}}{\pgfqpoint{2.279412in}{2.004545in}}%
\pgfusepath{clip}%
\pgfsetbuttcap%
\pgfsetroundjoin%
\pgfsetlinewidth{0.693941pt}%
\definecolor{currentstroke}{rgb}{0.243113,0.292092,0.538516}%
\pgfsetstrokecolor{currentstroke}%
\pgfsetdash{}{0pt}%
\pgfpathmoveto{\pgfqpoint{2.509792in}{5.425575in}}%
\pgfpathlineto{\pgfqpoint{2.460294in}{5.418500in}}%
\pgfusepath{stroke}%
\end{pgfscope}%
\begin{pgfscope}%
\pgfpathrectangle{\pgfqpoint{1.250000in}{4.155455in}}{\pgfqpoint{2.279412in}{2.004545in}}%
\pgfusepath{clip}%
\pgfsetbuttcap%
\pgfsetroundjoin%
\pgfsetlinewidth{0.719532pt}%
\definecolor{currentstroke}{rgb}{0.237441,0.305202,0.541921}%
\pgfsetstrokecolor{currentstroke}%
\pgfsetdash{}{0pt}%
\pgfpathmoveto{\pgfqpoint{2.460294in}{5.418500in}}%
\pgfpathlineto{\pgfqpoint{2.411052in}{5.410175in}}%
\pgfusepath{stroke}%
\end{pgfscope}%
\begin{pgfscope}%
\pgfpathrectangle{\pgfqpoint{1.250000in}{4.155455in}}{\pgfqpoint{2.279412in}{2.004545in}}%
\pgfusepath{clip}%
\pgfsetbuttcap%
\pgfsetroundjoin%
\pgfsetlinewidth{0.734508pt}%
\definecolor{currentstroke}{rgb}{0.231674,0.318106,0.544834}%
\pgfsetstrokecolor{currentstroke}%
\pgfsetdash{}{0pt}%
\pgfpathmoveto{\pgfqpoint{2.411052in}{5.410175in}}%
\pgfpathlineto{\pgfqpoint{2.362206in}{5.400246in}}%
\pgfusepath{stroke}%
\end{pgfscope}%
\begin{pgfscope}%
\pgfpathrectangle{\pgfqpoint{1.250000in}{4.155455in}}{\pgfqpoint{2.279412in}{2.004545in}}%
\pgfusepath{clip}%
\pgfsetbuttcap%
\pgfsetroundjoin%
\pgfsetlinewidth{0.678714pt}%
\definecolor{currentstroke}{rgb}{0.246811,0.283237,0.535941}%
\pgfsetstrokecolor{currentstroke}%
\pgfsetdash{}{0pt}%
\pgfpathmoveto{\pgfqpoint{2.362206in}{5.400246in}}%
\pgfpathlineto{\pgfqpoint{2.314104in}{5.387934in}}%
\pgfusepath{stroke}%
\end{pgfscope}%
\begin{pgfscope}%
\pgfpathrectangle{\pgfqpoint{1.250000in}{4.155455in}}{\pgfqpoint{2.279412in}{2.004545in}}%
\pgfusepath{clip}%
\pgfsetbuttcap%
\pgfsetroundjoin%
\pgfsetlinewidth{0.681150pt}%
\definecolor{currentstroke}{rgb}{0.246811,0.283237,0.535941}%
\pgfsetstrokecolor{currentstroke}%
\pgfsetdash{}{0pt}%
\pgfpathmoveto{\pgfqpoint{2.314104in}{5.387934in}}%
\pgfpathlineto{\pgfqpoint{2.267452in}{5.372003in}}%
\pgfusepath{stroke}%
\end{pgfscope}%
\begin{pgfscope}%
\pgfpathrectangle{\pgfqpoint{1.250000in}{4.155455in}}{\pgfqpoint{2.279412in}{2.004545in}}%
\pgfusepath{clip}%
\pgfsetbuttcap%
\pgfsetroundjoin%
\pgfsetlinewidth{0.633347pt}%
\definecolor{currentstroke}{rgb}{0.258965,0.251537,0.524736}%
\pgfsetstrokecolor{currentstroke}%
\pgfsetdash{}{0pt}%
\pgfpathmoveto{\pgfqpoint{2.267452in}{5.372003in}}%
\pgfpathlineto{\pgfqpoint{2.224502in}{5.350044in}}%
\pgfusepath{stroke}%
\end{pgfscope}%
\begin{pgfscope}%
\pgfpathrectangle{\pgfqpoint{1.250000in}{4.155455in}}{\pgfqpoint{2.279412in}{2.004545in}}%
\pgfusepath{clip}%
\pgfsetbuttcap%
\pgfsetroundjoin%
\pgfsetlinewidth{0.596158pt}%
\definecolor{currentstroke}{rgb}{0.267968,0.223549,0.512008}%
\pgfsetstrokecolor{currentstroke}%
\pgfsetdash{}{0pt}%
\pgfpathmoveto{\pgfqpoint{2.224502in}{5.350044in}}%
\pgfpathlineto{\pgfqpoint{2.224502in}{5.350044in}}%
\pgfusepath{stroke}%
\end{pgfscope}%
\begin{pgfscope}%
\pgfpathrectangle{\pgfqpoint{1.250000in}{4.155455in}}{\pgfqpoint{2.279412in}{2.004545in}}%
\pgfusepath{clip}%
\pgfsetbuttcap%
\pgfsetroundjoin%
\pgfsetlinewidth{0.596158pt}%
\definecolor{currentstroke}{rgb}{0.267968,0.223549,0.512008}%
\pgfsetstrokecolor{currentstroke}%
\pgfsetdash{}{0pt}%
\pgfpathmoveto{\pgfqpoint{2.224502in}{5.350044in}}%
\pgfpathlineto{\pgfqpoint{2.195245in}{5.328436in}}%
\pgfusepath{stroke}%
\end{pgfscope}%
\begin{pgfscope}%
\pgfpathrectangle{\pgfqpoint{1.250000in}{4.155455in}}{\pgfqpoint{2.279412in}{2.004545in}}%
\pgfusepath{clip}%
\pgfsetbuttcap%
\pgfsetroundjoin%
\pgfsetlinewidth{0.576076pt}%
\definecolor{currentstroke}{rgb}{0.270595,0.214069,0.507052}%
\pgfsetstrokecolor{currentstroke}%
\pgfsetdash{}{0pt}%
\pgfpathmoveto{\pgfqpoint{2.195245in}{5.328436in}}%
\pgfpathlineto{\pgfqpoint{2.195245in}{5.328436in}}%
\pgfusepath{stroke}%
\end{pgfscope}%
\begin{pgfscope}%
\pgfpathrectangle{\pgfqpoint{1.250000in}{4.155455in}}{\pgfqpoint{2.279412in}{2.004545in}}%
\pgfusepath{clip}%
\pgfsetbuttcap%
\pgfsetroundjoin%
\pgfsetlinewidth{0.576076pt}%
\definecolor{currentstroke}{rgb}{0.270595,0.214069,0.507052}%
\pgfsetstrokecolor{currentstroke}%
\pgfsetdash{}{0pt}%
\pgfpathmoveto{\pgfqpoint{2.195245in}{5.328436in}}%
\pgfpathlineto{\pgfqpoint{2.195245in}{5.328436in}}%
\pgfusepath{stroke}%
\end{pgfscope}%
\begin{pgfscope}%
\pgfpathrectangle{\pgfqpoint{1.250000in}{4.155455in}}{\pgfqpoint{2.279412in}{2.004545in}}%
\pgfusepath{clip}%
\pgfsetbuttcap%
\pgfsetroundjoin%
\pgfsetlinewidth{0.332944pt}%
\definecolor{currentstroke}{rgb}{0.272594,0.025563,0.353093}%
\pgfsetstrokecolor{currentstroke}%
\pgfsetdash{}{0pt}%
\pgfpathmoveto{\pgfqpoint{3.107792in}{4.796873in}}%
\pgfpathlineto{\pgfqpoint{3.057765in}{4.798331in}}%
\pgfusepath{stroke}%
\end{pgfscope}%
\begin{pgfscope}%
\pgfpathrectangle{\pgfqpoint{1.250000in}{4.155455in}}{\pgfqpoint{2.279412in}{2.004545in}}%
\pgfusepath{clip}%
\pgfsetbuttcap%
\pgfsetroundjoin%
\pgfsetlinewidth{0.317212pt}%
\definecolor{currentstroke}{rgb}{0.269944,0.014625,0.341379}%
\pgfsetstrokecolor{currentstroke}%
\pgfsetdash{}{0pt}%
\pgfpathmoveto{\pgfqpoint{3.057765in}{4.798331in}}%
\pgfpathlineto{\pgfqpoint{3.007696in}{4.800357in}}%
\pgfusepath{stroke}%
\end{pgfscope}%
\begin{pgfscope}%
\pgfpathrectangle{\pgfqpoint{1.250000in}{4.155455in}}{\pgfqpoint{2.279412in}{2.004545in}}%
\pgfusepath{clip}%
\pgfsetbuttcap%
\pgfsetroundjoin%
\pgfsetlinewidth{0.344765pt}%
\definecolor{currentstroke}{rgb}{0.274952,0.037752,0.364543}%
\pgfsetstrokecolor{currentstroke}%
\pgfsetdash{}{0pt}%
\pgfpathmoveto{\pgfqpoint{3.007696in}{4.800357in}}%
\pgfpathlineto{\pgfqpoint{2.957630in}{4.802755in}}%
\pgfusepath{stroke}%
\end{pgfscope}%
\begin{pgfscope}%
\pgfpathrectangle{\pgfqpoint{1.250000in}{4.155455in}}{\pgfqpoint{2.279412in}{2.004545in}}%
\pgfusepath{clip}%
\pgfsetbuttcap%
\pgfsetroundjoin%
\pgfsetlinewidth{0.356068pt}%
\definecolor{currentstroke}{rgb}{0.277018,0.050344,0.375715}%
\pgfsetstrokecolor{currentstroke}%
\pgfsetdash{}{0pt}%
\pgfpathmoveto{\pgfqpoint{2.957630in}{4.802755in}}%
\pgfpathlineto{\pgfqpoint{2.907578in}{4.805430in}}%
\pgfusepath{stroke}%
\end{pgfscope}%
\begin{pgfscope}%
\pgfpathrectangle{\pgfqpoint{1.250000in}{4.155455in}}{\pgfqpoint{2.279412in}{2.004545in}}%
\pgfusepath{clip}%
\pgfsetbuttcap%
\pgfsetroundjoin%
\pgfsetlinewidth{0.380700pt}%
\definecolor{currentstroke}{rgb}{0.279566,0.067836,0.391917}%
\pgfsetstrokecolor{currentstroke}%
\pgfsetdash{}{0pt}%
\pgfpathmoveto{\pgfqpoint{2.907578in}{4.805430in}}%
\pgfpathlineto{\pgfqpoint{2.857578in}{4.808858in}}%
\pgfusepath{stroke}%
\end{pgfscope}%
\begin{pgfscope}%
\pgfpathrectangle{\pgfqpoint{1.250000in}{4.155455in}}{\pgfqpoint{2.279412in}{2.004545in}}%
\pgfusepath{clip}%
\pgfsetbuttcap%
\pgfsetroundjoin%
\pgfsetlinewidth{0.409336pt}%
\definecolor{currentstroke}{rgb}{0.281924,0.089666,0.412415}%
\pgfsetstrokecolor{currentstroke}%
\pgfsetdash{}{0pt}%
\pgfpathmoveto{\pgfqpoint{2.857578in}{4.808858in}}%
\pgfpathlineto{\pgfqpoint{2.807603in}{4.812545in}}%
\pgfusepath{stroke}%
\end{pgfscope}%
\begin{pgfscope}%
\pgfpathrectangle{\pgfqpoint{1.250000in}{4.155455in}}{\pgfqpoint{2.279412in}{2.004545in}}%
\pgfusepath{clip}%
\pgfsetbuttcap%
\pgfsetroundjoin%
\pgfsetlinewidth{0.327702pt}%
\definecolor{currentstroke}{rgb}{0.271305,0.019942,0.347269}%
\pgfsetstrokecolor{currentstroke}%
\pgfsetdash{}{0pt}%
\pgfpathmoveto{\pgfqpoint{2.846180in}{4.552263in}}%
\pgfpathlineto{\pgfqpoint{2.796698in}{4.559078in}}%
\pgfusepath{stroke}%
\end{pgfscope}%
\begin{pgfscope}%
\pgfpathrectangle{\pgfqpoint{1.250000in}{4.155455in}}{\pgfqpoint{2.279412in}{2.004545in}}%
\pgfusepath{clip}%
\pgfsetbuttcap%
\pgfsetroundjoin%
\pgfsetlinewidth{0.330157pt}%
\definecolor{currentstroke}{rgb}{0.272594,0.025563,0.353093}%
\pgfsetstrokecolor{currentstroke}%
\pgfsetdash{}{0pt}%
\pgfpathmoveto{\pgfqpoint{2.796698in}{4.559078in}}%
\pgfpathlineto{\pgfqpoint{2.747218in}{4.565881in}}%
\pgfusepath{stroke}%
\end{pgfscope}%
\begin{pgfscope}%
\pgfpathrectangle{\pgfqpoint{1.250000in}{4.155455in}}{\pgfqpoint{2.279412in}{2.004545in}}%
\pgfusepath{clip}%
\pgfsetbuttcap%
\pgfsetroundjoin%
\pgfsetlinewidth{0.348333pt}%
\definecolor{currentstroke}{rgb}{0.274952,0.037752,0.364543}%
\pgfsetstrokecolor{currentstroke}%
\pgfsetdash{}{0pt}%
\pgfpathmoveto{\pgfqpoint{2.747218in}{4.565881in}}%
\pgfpathlineto{\pgfqpoint{2.697457in}{4.571339in}}%
\pgfusepath{stroke}%
\end{pgfscope}%
\begin{pgfscope}%
\pgfpathrectangle{\pgfqpoint{1.250000in}{4.155455in}}{\pgfqpoint{2.279412in}{2.004545in}}%
\pgfusepath{clip}%
\pgfsetbuttcap%
\pgfsetroundjoin%
\pgfsetlinewidth{0.355939pt}%
\definecolor{currentstroke}{rgb}{0.276022,0.044167,0.370164}%
\pgfsetstrokecolor{currentstroke}%
\pgfsetdash{}{0pt}%
\pgfpathmoveto{\pgfqpoint{2.697457in}{4.571339in}}%
\pgfpathlineto{\pgfqpoint{2.647872in}{4.577888in}}%
\pgfusepath{stroke}%
\end{pgfscope}%
\begin{pgfscope}%
\pgfpathrectangle{\pgfqpoint{1.250000in}{4.155455in}}{\pgfqpoint{2.279412in}{2.004545in}}%
\pgfusepath{clip}%
\pgfsetbuttcap%
\pgfsetroundjoin%
\pgfsetlinewidth{0.372349pt}%
\definecolor{currentstroke}{rgb}{0.278791,0.062145,0.386592}%
\pgfsetstrokecolor{currentstroke}%
\pgfsetdash{}{0pt}%
\pgfpathmoveto{\pgfqpoint{2.647872in}{4.577888in}}%
\pgfpathlineto{\pgfqpoint{2.598440in}{4.585269in}}%
\pgfusepath{stroke}%
\end{pgfscope}%
\begin{pgfscope}%
\pgfpathrectangle{\pgfqpoint{1.250000in}{4.155455in}}{\pgfqpoint{2.279412in}{2.004545in}}%
\pgfusepath{clip}%
\pgfsetbuttcap%
\pgfsetroundjoin%
\pgfsetlinewidth{0.382761pt}%
\definecolor{currentstroke}{rgb}{0.279566,0.067836,0.391917}%
\pgfsetstrokecolor{currentstroke}%
\pgfsetdash{}{0pt}%
\pgfpathmoveto{\pgfqpoint{2.598440in}{4.585269in}}%
\pgfpathlineto{\pgfqpoint{2.549185in}{4.593496in}}%
\pgfusepath{stroke}%
\end{pgfscope}%
\begin{pgfscope}%
\pgfpathrectangle{\pgfqpoint{1.250000in}{4.155455in}}{\pgfqpoint{2.279412in}{2.004545in}}%
\pgfusepath{clip}%
\pgfsetbuttcap%
\pgfsetroundjoin%
\pgfsetlinewidth{0.406391pt}%
\definecolor{currentstroke}{rgb}{0.281924,0.089666,0.412415}%
\pgfsetstrokecolor{currentstroke}%
\pgfsetdash{}{0pt}%
\pgfpathmoveto{\pgfqpoint{2.549185in}{4.593496in}}%
\pgfpathlineto{\pgfqpoint{2.499995in}{4.601947in}}%
\pgfusepath{stroke}%
\end{pgfscope}%
\begin{pgfscope}%
\pgfpathrectangle{\pgfqpoint{1.250000in}{4.155455in}}{\pgfqpoint{2.279412in}{2.004545in}}%
\pgfusepath{clip}%
\pgfsetbuttcap%
\pgfsetroundjoin%
\pgfsetlinewidth{0.304530pt}%
\definecolor{currentstroke}{rgb}{0.267004,0.004874,0.329415}%
\pgfsetstrokecolor{currentstroke}%
\pgfsetdash{}{0pt}%
\pgfpathmoveto{\pgfqpoint{3.056501in}{4.661553in}}%
\pgfpathlineto{\pgfqpoint{3.011719in}{4.666772in}}%
\pgfusepath{stroke}%
\end{pgfscope}%
\begin{pgfscope}%
\pgfpathrectangle{\pgfqpoint{1.250000in}{4.155455in}}{\pgfqpoint{2.279412in}{2.004545in}}%
\pgfusepath{clip}%
\pgfsetbuttcap%
\pgfsetroundjoin%
\pgfsetlinewidth{0.323576pt}%
\definecolor{currentstroke}{rgb}{0.271305,0.019942,0.347269}%
\pgfsetstrokecolor{currentstroke}%
\pgfsetdash{}{0pt}%
\pgfpathmoveto{\pgfqpoint{3.011719in}{4.666772in}}%
\pgfpathlineto{\pgfqpoint{2.962316in}{4.672072in}}%
\pgfusepath{stroke}%
\end{pgfscope}%
\begin{pgfscope}%
\pgfpathrectangle{\pgfqpoint{1.250000in}{4.155455in}}{\pgfqpoint{2.279412in}{2.004545in}}%
\pgfusepath{clip}%
\pgfsetbuttcap%
\pgfsetroundjoin%
\pgfsetlinewidth{0.337483pt}%
\definecolor{currentstroke}{rgb}{0.273809,0.031497,0.358853}%
\pgfsetstrokecolor{currentstroke}%
\pgfsetdash{}{0pt}%
\pgfpathmoveto{\pgfqpoint{2.962316in}{4.672072in}}%
\pgfpathlineto{\pgfqpoint{2.912615in}{4.677244in}}%
\pgfusepath{stroke}%
\end{pgfscope}%
\begin{pgfscope}%
\pgfpathrectangle{\pgfqpoint{1.250000in}{4.155455in}}{\pgfqpoint{2.279412in}{2.004545in}}%
\pgfusepath{clip}%
\pgfsetbuttcap%
\pgfsetroundjoin%
\pgfsetlinewidth{0.345754pt}%
\definecolor{currentstroke}{rgb}{0.274952,0.037752,0.364543}%
\pgfsetstrokecolor{currentstroke}%
\pgfsetdash{}{0pt}%
\pgfpathmoveto{\pgfqpoint{2.912615in}{4.677244in}}%
\pgfpathlineto{\pgfqpoint{2.862856in}{4.682631in}}%
\pgfusepath{stroke}%
\end{pgfscope}%
\begin{pgfscope}%
\pgfpathrectangle{\pgfqpoint{1.250000in}{4.155455in}}{\pgfqpoint{2.279412in}{2.004545in}}%
\pgfusepath{clip}%
\pgfsetbuttcap%
\pgfsetroundjoin%
\pgfsetlinewidth{0.340239pt}%
\definecolor{currentstroke}{rgb}{0.273809,0.031497,0.358853}%
\pgfsetstrokecolor{currentstroke}%
\pgfsetdash{}{0pt}%
\pgfpathmoveto{\pgfqpoint{2.862856in}{4.682631in}}%
\pgfpathlineto{\pgfqpoint{2.813143in}{4.688412in}}%
\pgfusepath{stroke}%
\end{pgfscope}%
\begin{pgfscope}%
\pgfpathrectangle{\pgfqpoint{1.250000in}{4.155455in}}{\pgfqpoint{2.279412in}{2.004545in}}%
\pgfusepath{clip}%
\pgfsetbuttcap%
\pgfsetroundjoin%
\pgfsetlinewidth{0.377104pt}%
\definecolor{currentstroke}{rgb}{0.279566,0.067836,0.391917}%
\pgfsetstrokecolor{currentstroke}%
\pgfsetdash{}{0pt}%
\pgfpathmoveto{\pgfqpoint{2.813143in}{4.688412in}}%
\pgfpathlineto{\pgfqpoint{2.763384in}{4.693882in}}%
\pgfusepath{stroke}%
\end{pgfscope}%
\begin{pgfscope}%
\pgfpathrectangle{\pgfqpoint{1.250000in}{4.155455in}}{\pgfqpoint{2.279412in}{2.004545in}}%
\pgfusepath{clip}%
\pgfsetbuttcap%
\pgfsetroundjoin%
\pgfsetlinewidth{0.389555pt}%
\definecolor{currentstroke}{rgb}{0.280267,0.073417,0.397163}%
\pgfsetstrokecolor{currentstroke}%
\pgfsetdash{}{0pt}%
\pgfpathmoveto{\pgfqpoint{2.763384in}{4.693882in}}%
\pgfpathlineto{\pgfqpoint{2.713761in}{4.700197in}}%
\pgfusepath{stroke}%
\end{pgfscope}%
\begin{pgfscope}%
\pgfpathrectangle{\pgfqpoint{1.250000in}{4.155455in}}{\pgfqpoint{2.279412in}{2.004545in}}%
\pgfusepath{clip}%
\pgfsetbuttcap%
\pgfsetroundjoin%
\pgfsetlinewidth{0.420901pt}%
\definecolor{currentstroke}{rgb}{0.282656,0.100196,0.422160}%
\pgfsetstrokecolor{currentstroke}%
\pgfsetdash{}{0pt}%
\pgfpathmoveto{\pgfqpoint{2.713761in}{4.700197in}}%
\pgfpathlineto{\pgfqpoint{2.664266in}{4.707301in}}%
\pgfusepath{stroke}%
\end{pgfscope}%
\begin{pgfscope}%
\pgfpathrectangle{\pgfqpoint{1.250000in}{4.155455in}}{\pgfqpoint{2.279412in}{2.004545in}}%
\pgfusepath{clip}%
\pgfsetbuttcap%
\pgfsetroundjoin%
\pgfsetlinewidth{0.441219pt}%
\definecolor{currentstroke}{rgb}{0.283197,0.115680,0.436115}%
\pgfsetstrokecolor{currentstroke}%
\pgfsetdash{}{0pt}%
\pgfpathmoveto{\pgfqpoint{2.664266in}{4.707301in}}%
\pgfpathlineto{\pgfqpoint{2.614795in}{4.714535in}}%
\pgfusepath{stroke}%
\end{pgfscope}%
\begin{pgfscope}%
\pgfpathrectangle{\pgfqpoint{1.250000in}{4.155455in}}{\pgfqpoint{2.279412in}{2.004545in}}%
\pgfusepath{clip}%
\pgfsetbuttcap%
\pgfsetroundjoin%
\pgfsetlinewidth{0.472823pt}%
\definecolor{currentstroke}{rgb}{0.282623,0.140926,0.457517}%
\pgfsetstrokecolor{currentstroke}%
\pgfsetdash{}{0pt}%
\pgfpathmoveto{\pgfqpoint{2.614795in}{4.714535in}}%
\pgfpathlineto{\pgfqpoint{2.565468in}{4.722467in}}%
\pgfusepath{stroke}%
\end{pgfscope}%
\begin{pgfscope}%
\pgfpathrectangle{\pgfqpoint{1.250000in}{4.155455in}}{\pgfqpoint{2.279412in}{2.004545in}}%
\pgfusepath{clip}%
\pgfsetbuttcap%
\pgfsetroundjoin%
\pgfsetlinewidth{0.499202pt}%
\definecolor{currentstroke}{rgb}{0.281412,0.155834,0.469201}%
\pgfsetstrokecolor{currentstroke}%
\pgfsetdash{}{0pt}%
\pgfpathmoveto{\pgfqpoint{2.565468in}{4.722467in}}%
\pgfpathlineto{\pgfqpoint{2.516377in}{4.731453in}}%
\pgfusepath{stroke}%
\end{pgfscope}%
\begin{pgfscope}%
\pgfpathrectangle{\pgfqpoint{1.250000in}{4.155455in}}{\pgfqpoint{2.279412in}{2.004545in}}%
\pgfusepath{clip}%
\pgfsetbuttcap%
\pgfsetroundjoin%
\pgfsetlinewidth{0.527259pt}%
\definecolor{currentstroke}{rgb}{0.278826,0.175490,0.483397}%
\pgfsetstrokecolor{currentstroke}%
\pgfsetdash{}{0pt}%
\pgfpathmoveto{\pgfqpoint{2.516377in}{4.731453in}}%
\pgfpathlineto{\pgfqpoint{2.467713in}{4.742054in}}%
\pgfusepath{stroke}%
\end{pgfscope}%
\begin{pgfscope}%
\pgfpathrectangle{\pgfqpoint{1.250000in}{4.155455in}}{\pgfqpoint{2.279412in}{2.004545in}}%
\pgfusepath{clip}%
\pgfsetbuttcap%
\pgfsetroundjoin%
\pgfsetlinewidth{0.495905pt}%
\definecolor{currentstroke}{rgb}{0.281412,0.155834,0.469201}%
\pgfsetstrokecolor{currentstroke}%
\pgfsetdash{}{0pt}%
\pgfpathmoveto{\pgfqpoint{2.467713in}{4.742054in}}%
\pgfpathlineto{\pgfqpoint{2.419548in}{4.754310in}}%
\pgfusepath{stroke}%
\end{pgfscope}%
\begin{pgfscope}%
\pgfpathrectangle{\pgfqpoint{1.250000in}{4.155455in}}{\pgfqpoint{2.279412in}{2.004545in}}%
\pgfusepath{clip}%
\pgfsetbuttcap%
\pgfsetroundjoin%
\pgfsetlinewidth{0.545531pt}%
\definecolor{currentstroke}{rgb}{0.276194,0.190074,0.493001}%
\pgfsetstrokecolor{currentstroke}%
\pgfsetdash{}{0pt}%
\pgfpathmoveto{\pgfqpoint{2.419548in}{4.754310in}}%
\pgfpathlineto{\pgfqpoint{2.372274in}{4.768898in}}%
\pgfusepath{stroke}%
\end{pgfscope}%
\begin{pgfscope}%
\pgfpathrectangle{\pgfqpoint{1.250000in}{4.155455in}}{\pgfqpoint{2.279412in}{2.004545in}}%
\pgfusepath{clip}%
\pgfsetbuttcap%
\pgfsetroundjoin%
\pgfsetlinewidth{0.523651pt}%
\definecolor{currentstroke}{rgb}{0.278826,0.175490,0.483397}%
\pgfsetstrokecolor{currentstroke}%
\pgfsetdash{}{0pt}%
\pgfpathmoveto{\pgfqpoint{2.372274in}{4.768898in}}%
\pgfpathlineto{\pgfqpoint{2.327256in}{4.788029in}}%
\pgfusepath{stroke}%
\end{pgfscope}%
\begin{pgfscope}%
\pgfpathrectangle{\pgfqpoint{1.250000in}{4.155455in}}{\pgfqpoint{2.279412in}{2.004545in}}%
\pgfusepath{clip}%
\pgfsetbuttcap%
\pgfsetroundjoin%
\pgfsetlinewidth{0.542862pt}%
\definecolor{currentstroke}{rgb}{0.276194,0.190074,0.493001}%
\pgfsetstrokecolor{currentstroke}%
\pgfsetdash{}{0pt}%
\pgfpathmoveto{\pgfqpoint{2.327256in}{4.788029in}}%
\pgfpathlineto{\pgfqpoint{2.285490in}{4.812073in}}%
\pgfusepath{stroke}%
\end{pgfscope}%
\begin{pgfscope}%
\pgfpathrectangle{\pgfqpoint{1.250000in}{4.155455in}}{\pgfqpoint{2.279412in}{2.004545in}}%
\pgfusepath{clip}%
\pgfsetbuttcap%
\pgfsetroundjoin%
\pgfsetlinewidth{0.540651pt}%
\definecolor{currentstroke}{rgb}{0.277134,0.185228,0.489898}%
\pgfsetstrokecolor{currentstroke}%
\pgfsetdash{}{0pt}%
\pgfpathmoveto{\pgfqpoint{2.285490in}{4.812073in}}%
\pgfpathlineto{\pgfqpoint{2.247408in}{4.840351in}}%
\pgfusepath{stroke}%
\end{pgfscope}%
\begin{pgfscope}%
\pgfpathrectangle{\pgfqpoint{1.250000in}{4.155455in}}{\pgfqpoint{2.279412in}{2.004545in}}%
\pgfusepath{clip}%
\pgfsetbuttcap%
\pgfsetroundjoin%
\pgfsetlinewidth{0.630094pt}%
\definecolor{currentstroke}{rgb}{0.260571,0.246922,0.522828}%
\pgfsetstrokecolor{currentstroke}%
\pgfsetdash{}{0pt}%
\pgfpathmoveto{\pgfqpoint{2.247408in}{4.840351in}}%
\pgfpathlineto{\pgfqpoint{2.247408in}{4.840351in}}%
\pgfusepath{stroke}%
\end{pgfscope}%
\begin{pgfscope}%
\pgfpathrectangle{\pgfqpoint{1.250000in}{4.155455in}}{\pgfqpoint{2.279412in}{2.004545in}}%
\pgfusepath{clip}%
\pgfsetbuttcap%
\pgfsetroundjoin%
\pgfsetlinewidth{0.331395pt}%
\definecolor{currentstroke}{rgb}{0.272594,0.025563,0.353093}%
\pgfsetstrokecolor{currentstroke}%
\pgfsetdash{}{0pt}%
\pgfpathmoveto{\pgfqpoint{3.056501in}{5.518582in}}%
\pgfpathlineto{\pgfqpoint{3.006459in}{5.515801in}}%
\pgfusepath{stroke}%
\end{pgfscope}%
\begin{pgfscope}%
\pgfpathrectangle{\pgfqpoint{1.250000in}{4.155455in}}{\pgfqpoint{2.279412in}{2.004545in}}%
\pgfusepath{clip}%
\pgfsetbuttcap%
\pgfsetroundjoin%
\pgfsetlinewidth{0.340862pt}%
\definecolor{currentstroke}{rgb}{0.273809,0.031497,0.358853}%
\pgfsetstrokecolor{currentstroke}%
\pgfsetdash{}{0pt}%
\pgfpathmoveto{\pgfqpoint{3.006459in}{5.515801in}}%
\pgfpathlineto{\pgfqpoint{2.956423in}{5.512840in}}%
\pgfusepath{stroke}%
\end{pgfscope}%
\begin{pgfscope}%
\pgfpathrectangle{\pgfqpoint{1.250000in}{4.155455in}}{\pgfqpoint{2.279412in}{2.004545in}}%
\pgfusepath{clip}%
\pgfsetbuttcap%
\pgfsetroundjoin%
\pgfsetlinewidth{0.359491pt}%
\definecolor{currentstroke}{rgb}{0.277018,0.050344,0.375715}%
\pgfsetstrokecolor{currentstroke}%
\pgfsetdash{}{0pt}%
\pgfpathmoveto{\pgfqpoint{2.956423in}{5.512840in}}%
\pgfpathlineto{\pgfqpoint{2.906426in}{5.509426in}}%
\pgfusepath{stroke}%
\end{pgfscope}%
\begin{pgfscope}%
\pgfpathrectangle{\pgfqpoint{1.250000in}{4.155455in}}{\pgfqpoint{2.279412in}{2.004545in}}%
\pgfusepath{clip}%
\pgfsetbuttcap%
\pgfsetroundjoin%
\pgfsetlinewidth{0.367463pt}%
\definecolor{currentstroke}{rgb}{0.277941,0.056324,0.381191}%
\pgfsetstrokecolor{currentstroke}%
\pgfsetdash{}{0pt}%
\pgfpathmoveto{\pgfqpoint{2.906426in}{5.509426in}}%
\pgfpathlineto{\pgfqpoint{2.856487in}{5.505398in}}%
\pgfusepath{stroke}%
\end{pgfscope}%
\begin{pgfscope}%
\pgfpathrectangle{\pgfqpoint{1.250000in}{4.155455in}}{\pgfqpoint{2.279412in}{2.004545in}}%
\pgfusepath{clip}%
\pgfsetbuttcap%
\pgfsetroundjoin%
\pgfsetlinewidth{0.400774pt}%
\definecolor{currentstroke}{rgb}{0.281446,0.084320,0.407414}%
\pgfsetstrokecolor{currentstroke}%
\pgfsetdash{}{0pt}%
\pgfpathmoveto{\pgfqpoint{2.856487in}{5.505398in}}%
\pgfpathlineto{\pgfqpoint{2.806562in}{5.501227in}}%
\pgfusepath{stroke}%
\end{pgfscope}%
\begin{pgfscope}%
\pgfpathrectangle{\pgfqpoint{1.250000in}{4.155455in}}{\pgfqpoint{2.279412in}{2.004545in}}%
\pgfusepath{clip}%
\pgfsetbuttcap%
\pgfsetroundjoin%
\pgfsetlinewidth{0.427710pt}%
\definecolor{currentstroke}{rgb}{0.282910,0.105393,0.426902}%
\pgfsetstrokecolor{currentstroke}%
\pgfsetdash{}{0pt}%
\pgfpathmoveto{\pgfqpoint{2.806562in}{5.501227in}}%
\pgfpathlineto{\pgfqpoint{2.756636in}{5.497053in}}%
\pgfusepath{stroke}%
\end{pgfscope}%
\begin{pgfscope}%
\pgfpathrectangle{\pgfqpoint{1.250000in}{4.155455in}}{\pgfqpoint{2.279412in}{2.004545in}}%
\pgfusepath{clip}%
\pgfsetbuttcap%
\pgfsetroundjoin%
\pgfsetlinewidth{0.462036pt}%
\definecolor{currentstroke}{rgb}{0.283072,0.130895,0.449241}%
\pgfsetstrokecolor{currentstroke}%
\pgfsetdash{}{0pt}%
\pgfpathmoveto{\pgfqpoint{2.756636in}{5.497053in}}%
\pgfpathlineto{\pgfqpoint{2.706750in}{5.492539in}}%
\pgfusepath{stroke}%
\end{pgfscope}%
\begin{pgfscope}%
\pgfpathrectangle{\pgfqpoint{1.250000in}{4.155455in}}{\pgfqpoint{2.279412in}{2.004545in}}%
\pgfusepath{clip}%
\pgfsetbuttcap%
\pgfsetroundjoin%
\pgfsetlinewidth{0.507614pt}%
\definecolor{currentstroke}{rgb}{0.280255,0.165693,0.476498}%
\pgfsetstrokecolor{currentstroke}%
\pgfsetdash{}{0pt}%
\pgfpathmoveto{\pgfqpoint{2.706750in}{5.492539in}}%
\pgfpathlineto{\pgfqpoint{2.656909in}{5.487641in}}%
\pgfusepath{stroke}%
\end{pgfscope}%
\begin{pgfscope}%
\pgfpathrectangle{\pgfqpoint{1.250000in}{4.155455in}}{\pgfqpoint{2.279412in}{2.004545in}}%
\pgfusepath{clip}%
\pgfsetbuttcap%
\pgfsetroundjoin%
\pgfsetlinewidth{0.560449pt}%
\definecolor{currentstroke}{rgb}{0.274128,0.199721,0.498911}%
\pgfsetstrokecolor{currentstroke}%
\pgfsetdash{}{0pt}%
\pgfpathmoveto{\pgfqpoint{2.656909in}{5.487641in}}%
\pgfpathlineto{\pgfqpoint{2.607144in}{5.482209in}}%
\pgfusepath{stroke}%
\end{pgfscope}%
\begin{pgfscope}%
\pgfpathrectangle{\pgfqpoint{1.250000in}{4.155455in}}{\pgfqpoint{2.279412in}{2.004545in}}%
\pgfusepath{clip}%
\pgfsetbuttcap%
\pgfsetroundjoin%
\pgfsetlinewidth{0.577294pt}%
\definecolor{currentstroke}{rgb}{0.270595,0.214069,0.507052}%
\pgfsetstrokecolor{currentstroke}%
\pgfsetdash{}{0pt}%
\pgfpathmoveto{\pgfqpoint{2.607144in}{5.482209in}}%
\pgfpathlineto{\pgfqpoint{2.557478in}{5.476083in}}%
\pgfusepath{stroke}%
\end{pgfscope}%
\begin{pgfscope}%
\pgfpathrectangle{\pgfqpoint{1.250000in}{4.155455in}}{\pgfqpoint{2.279412in}{2.004545in}}%
\pgfusepath{clip}%
\pgfsetbuttcap%
\pgfsetroundjoin%
\pgfsetlinewidth{0.610231pt}%
\definecolor{currentstroke}{rgb}{0.263663,0.237631,0.518762}%
\pgfsetstrokecolor{currentstroke}%
\pgfsetdash{}{0pt}%
\pgfpathmoveto{\pgfqpoint{2.557478in}{5.476083in}}%
\pgfpathlineto{\pgfqpoint{2.507896in}{5.469463in}}%
\pgfusepath{stroke}%
\end{pgfscope}%
\begin{pgfscope}%
\pgfpathrectangle{\pgfqpoint{1.250000in}{4.155455in}}{\pgfqpoint{2.279412in}{2.004545in}}%
\pgfusepath{clip}%
\pgfsetbuttcap%
\pgfsetroundjoin%
\pgfsetlinewidth{0.627557pt}%
\definecolor{currentstroke}{rgb}{0.260571,0.246922,0.522828}%
\pgfsetstrokecolor{currentstroke}%
\pgfsetdash{}{0pt}%
\pgfpathmoveto{\pgfqpoint{2.507896in}{5.469463in}}%
\pgfpathlineto{\pgfqpoint{2.458536in}{5.461705in}}%
\pgfusepath{stroke}%
\end{pgfscope}%
\begin{pgfscope}%
\pgfpathrectangle{\pgfqpoint{1.250000in}{4.155455in}}{\pgfqpoint{2.279412in}{2.004545in}}%
\pgfusepath{clip}%
\pgfsetbuttcap%
\pgfsetroundjoin%
\pgfsetlinewidth{0.652811pt}%
\definecolor{currentstroke}{rgb}{0.253935,0.265254,0.529983}%
\pgfsetstrokecolor{currentstroke}%
\pgfsetdash{}{0pt}%
\pgfpathmoveto{\pgfqpoint{2.458536in}{5.461705in}}%
\pgfpathlineto{\pgfqpoint{2.409513in}{5.452441in}}%
\pgfusepath{stroke}%
\end{pgfscope}%
\begin{pgfscope}%
\pgfpathrectangle{\pgfqpoint{1.250000in}{4.155455in}}{\pgfqpoint{2.279412in}{2.004545in}}%
\pgfusepath{clip}%
\pgfsetbuttcap%
\pgfsetroundjoin%
\pgfsetlinewidth{0.680768pt}%
\definecolor{currentstroke}{rgb}{0.246811,0.283237,0.535941}%
\pgfsetstrokecolor{currentstroke}%
\pgfsetdash{}{0pt}%
\pgfpathmoveto{\pgfqpoint{2.409513in}{5.452441in}}%
\pgfpathlineto{\pgfqpoint{2.361116in}{5.441025in}}%
\pgfusepath{stroke}%
\end{pgfscope}%
\begin{pgfscope}%
\pgfpathrectangle{\pgfqpoint{1.250000in}{4.155455in}}{\pgfqpoint{2.279412in}{2.004545in}}%
\pgfusepath{clip}%
\pgfsetbuttcap%
\pgfsetroundjoin%
\pgfsetlinewidth{0.662322pt}%
\definecolor{currentstroke}{rgb}{0.252194,0.269783,0.531579}%
\pgfsetstrokecolor{currentstroke}%
\pgfsetdash{}{0pt}%
\pgfpathmoveto{\pgfqpoint{2.361116in}{5.441025in}}%
\pgfpathlineto{\pgfqpoint{2.313852in}{5.426495in}}%
\pgfusepath{stroke}%
\end{pgfscope}%
\begin{pgfscope}%
\pgfpathrectangle{\pgfqpoint{1.250000in}{4.155455in}}{\pgfqpoint{2.279412in}{2.004545in}}%
\pgfusepath{clip}%
\pgfsetbuttcap%
\pgfsetroundjoin%
\pgfsetlinewidth{0.624273pt}%
\definecolor{currentstroke}{rgb}{0.260571,0.246922,0.522828}%
\pgfsetstrokecolor{currentstroke}%
\pgfsetdash{}{0pt}%
\pgfpathmoveto{\pgfqpoint{2.313852in}{5.426495in}}%
\pgfpathlineto{\pgfqpoint{2.268685in}{5.407714in}}%
\pgfusepath{stroke}%
\end{pgfscope}%
\begin{pgfscope}%
\pgfpathrectangle{\pgfqpoint{1.250000in}{4.155455in}}{\pgfqpoint{2.279412in}{2.004545in}}%
\pgfusepath{clip}%
\pgfsetbuttcap%
\pgfsetroundjoin%
\pgfsetlinewidth{0.623648pt}%
\definecolor{currentstroke}{rgb}{0.260571,0.246922,0.522828}%
\pgfsetstrokecolor{currentstroke}%
\pgfsetdash{}{0pt}%
\pgfpathmoveto{\pgfqpoint{2.268685in}{5.407714in}}%
\pgfpathlineto{\pgfqpoint{2.227273in}{5.383420in}}%
\pgfusepath{stroke}%
\end{pgfscope}%
\begin{pgfscope}%
\pgfpathrectangle{\pgfqpoint{1.250000in}{4.155455in}}{\pgfqpoint{2.279412in}{2.004545in}}%
\pgfusepath{clip}%
\pgfsetbuttcap%
\pgfsetroundjoin%
\pgfsetlinewidth{0.578625pt}%
\definecolor{currentstroke}{rgb}{0.270595,0.214069,0.507052}%
\pgfsetstrokecolor{currentstroke}%
\pgfsetdash{}{0pt}%
\pgfpathmoveto{\pgfqpoint{2.227273in}{5.383420in}}%
\pgfpathlineto{\pgfqpoint{2.227273in}{5.383420in}}%
\pgfusepath{stroke}%
\end{pgfscope}%
\begin{pgfscope}%
\pgfpathrectangle{\pgfqpoint{1.250000in}{4.155455in}}{\pgfqpoint{2.279412in}{2.004545in}}%
\pgfusepath{clip}%
\pgfsetbuttcap%
\pgfsetroundjoin%
\pgfsetlinewidth{0.328277pt}%
\definecolor{currentstroke}{rgb}{0.271305,0.019942,0.347269}%
\pgfsetstrokecolor{currentstroke}%
\pgfsetdash{}{0pt}%
\pgfpathmoveto{\pgfqpoint{3.056501in}{5.563688in}}%
\pgfpathlineto{\pgfqpoint{3.006485in}{5.561576in}}%
\pgfusepath{stroke}%
\end{pgfscope}%
\begin{pgfscope}%
\pgfpathrectangle{\pgfqpoint{1.250000in}{4.155455in}}{\pgfqpoint{2.279412in}{2.004545in}}%
\pgfusepath{clip}%
\pgfsetbuttcap%
\pgfsetroundjoin%
\pgfsetlinewidth{0.331322pt}%
\definecolor{currentstroke}{rgb}{0.272594,0.025563,0.353093}%
\pgfsetstrokecolor{currentstroke}%
\pgfsetdash{}{0pt}%
\pgfpathmoveto{\pgfqpoint{3.006485in}{5.561576in}}%
\pgfpathlineto{\pgfqpoint{2.956444in}{5.559019in}}%
\pgfusepath{stroke}%
\end{pgfscope}%
\begin{pgfscope}%
\pgfpathrectangle{\pgfqpoint{1.250000in}{4.155455in}}{\pgfqpoint{2.279412in}{2.004545in}}%
\pgfusepath{clip}%
\pgfsetbuttcap%
\pgfsetroundjoin%
\pgfsetlinewidth{0.356250pt}%
\definecolor{currentstroke}{rgb}{0.277018,0.050344,0.375715}%
\pgfsetstrokecolor{currentstroke}%
\pgfsetdash{}{0pt}%
\pgfpathmoveto{\pgfqpoint{2.956444in}{5.559019in}}%
\pgfpathlineto{\pgfqpoint{2.906447in}{5.555655in}}%
\pgfusepath{stroke}%
\end{pgfscope}%
\begin{pgfscope}%
\pgfpathrectangle{\pgfqpoint{1.250000in}{4.155455in}}{\pgfqpoint{2.279412in}{2.004545in}}%
\pgfusepath{clip}%
\pgfsetbuttcap%
\pgfsetroundjoin%
\pgfsetlinewidth{0.364894pt}%
\definecolor{currentstroke}{rgb}{0.277941,0.056324,0.381191}%
\pgfsetstrokecolor{currentstroke}%
\pgfsetdash{}{0pt}%
\pgfpathmoveto{\pgfqpoint{2.906447in}{5.555655in}}%
\pgfpathlineto{\pgfqpoint{2.856453in}{5.552170in}}%
\pgfusepath{stroke}%
\end{pgfscope}%
\begin{pgfscope}%
\pgfpathrectangle{\pgfqpoint{1.250000in}{4.155455in}}{\pgfqpoint{2.279412in}{2.004545in}}%
\pgfusepath{clip}%
\pgfsetbuttcap%
\pgfsetroundjoin%
\pgfsetlinewidth{0.386079pt}%
\definecolor{currentstroke}{rgb}{0.280267,0.073417,0.397163}%
\pgfsetstrokecolor{currentstroke}%
\pgfsetdash{}{0pt}%
\pgfpathmoveto{\pgfqpoint{2.856453in}{5.552170in}}%
\pgfpathlineto{\pgfqpoint{2.806533in}{5.547979in}}%
\pgfusepath{stroke}%
\end{pgfscope}%
\begin{pgfscope}%
\pgfpathrectangle{\pgfqpoint{1.250000in}{4.155455in}}{\pgfqpoint{2.279412in}{2.004545in}}%
\pgfusepath{clip}%
\pgfsetbuttcap%
\pgfsetroundjoin%
\pgfsetlinewidth{0.404844pt}%
\definecolor{currentstroke}{rgb}{0.281924,0.089666,0.412415}%
\pgfsetstrokecolor{currentstroke}%
\pgfsetdash{}{0pt}%
\pgfpathmoveto{\pgfqpoint{2.806533in}{5.547979in}}%
\pgfpathlineto{\pgfqpoint{2.756687in}{5.543183in}}%
\pgfusepath{stroke}%
\end{pgfscope}%
\begin{pgfscope}%
\pgfpathrectangle{\pgfqpoint{1.250000in}{4.155455in}}{\pgfqpoint{2.279412in}{2.004545in}}%
\pgfusepath{clip}%
\pgfsetbuttcap%
\pgfsetroundjoin%
\pgfsetlinewidth{0.443362pt}%
\definecolor{currentstroke}{rgb}{0.283197,0.115680,0.436115}%
\pgfsetstrokecolor{currentstroke}%
\pgfsetdash{}{0pt}%
\pgfpathmoveto{\pgfqpoint{2.756687in}{5.543183in}}%
\pgfpathlineto{\pgfqpoint{2.706867in}{5.538171in}}%
\pgfusepath{stroke}%
\end{pgfscope}%
\begin{pgfscope}%
\pgfpathrectangle{\pgfqpoint{1.250000in}{4.155455in}}{\pgfqpoint{2.279412in}{2.004545in}}%
\pgfusepath{clip}%
\pgfsetbuttcap%
\pgfsetroundjoin%
\pgfsetlinewidth{0.468666pt}%
\definecolor{currentstroke}{rgb}{0.282884,0.135920,0.453427}%
\pgfsetstrokecolor{currentstroke}%
\pgfsetdash{}{0pt}%
\pgfpathmoveto{\pgfqpoint{2.706867in}{5.538171in}}%
\pgfpathlineto{\pgfqpoint{2.657052in}{5.533072in}}%
\pgfusepath{stroke}%
\end{pgfscope}%
\begin{pgfscope}%
\pgfpathrectangle{\pgfqpoint{1.250000in}{4.155455in}}{\pgfqpoint{2.279412in}{2.004545in}}%
\pgfusepath{clip}%
\pgfsetbuttcap%
\pgfsetroundjoin%
\pgfsetlinewidth{0.498708pt}%
\definecolor{currentstroke}{rgb}{0.281412,0.155834,0.469201}%
\pgfsetstrokecolor{currentstroke}%
\pgfsetdash{}{0pt}%
\pgfpathmoveto{\pgfqpoint{2.657052in}{5.533072in}}%
\pgfpathlineto{\pgfqpoint{2.607302in}{5.527533in}}%
\pgfusepath{stroke}%
\end{pgfscope}%
\begin{pgfscope}%
\pgfpathrectangle{\pgfqpoint{1.250000in}{4.155455in}}{\pgfqpoint{2.279412in}{2.004545in}}%
\pgfusepath{clip}%
\pgfsetbuttcap%
\pgfsetroundjoin%
\pgfsetlinewidth{0.534302pt}%
\definecolor{currentstroke}{rgb}{0.278012,0.180367,0.486697}%
\pgfsetstrokecolor{currentstroke}%
\pgfsetdash{}{0pt}%
\pgfpathmoveto{\pgfqpoint{2.607302in}{5.527533in}}%
\pgfpathlineto{\pgfqpoint{2.557675in}{5.521216in}}%
\pgfusepath{stroke}%
\end{pgfscope}%
\begin{pgfscope}%
\pgfpathrectangle{\pgfqpoint{1.250000in}{4.155455in}}{\pgfqpoint{2.279412in}{2.004545in}}%
\pgfusepath{clip}%
\pgfsetbuttcap%
\pgfsetroundjoin%
\pgfsetlinewidth{0.550647pt}%
\definecolor{currentstroke}{rgb}{0.275191,0.194905,0.496005}%
\pgfsetstrokecolor{currentstroke}%
\pgfsetdash{}{0pt}%
\pgfpathmoveto{\pgfqpoint{2.557675in}{5.521216in}}%
\pgfpathlineto{\pgfqpoint{2.508201in}{5.514045in}}%
\pgfusepath{stroke}%
\end{pgfscope}%
\begin{pgfscope}%
\pgfpathrectangle{\pgfqpoint{1.250000in}{4.155455in}}{\pgfqpoint{2.279412in}{2.004545in}}%
\pgfusepath{clip}%
\pgfsetbuttcap%
\pgfsetroundjoin%
\pgfsetlinewidth{0.571081pt}%
\definecolor{currentstroke}{rgb}{0.271828,0.209303,0.504434}%
\pgfsetstrokecolor{currentstroke}%
\pgfsetdash{}{0pt}%
\pgfpathmoveto{\pgfqpoint{2.508201in}{5.514045in}}%
\pgfpathlineto{\pgfqpoint{2.459130in}{5.505026in}}%
\pgfusepath{stroke}%
\end{pgfscope}%
\begin{pgfscope}%
\pgfpathrectangle{\pgfqpoint{1.250000in}{4.155455in}}{\pgfqpoint{2.279412in}{2.004545in}}%
\pgfusepath{clip}%
\pgfsetbuttcap%
\pgfsetroundjoin%
\pgfsetlinewidth{0.342589pt}%
\definecolor{currentstroke}{rgb}{0.274952,0.037752,0.364543}%
\pgfsetstrokecolor{currentstroke}%
\pgfsetdash{}{0pt}%
\pgfpathmoveto{\pgfqpoint{2.800041in}{5.744115in}}%
\pgfpathlineto{\pgfqpoint{2.750177in}{5.739815in}}%
\pgfusepath{stroke}%
\end{pgfscope}%
\begin{pgfscope}%
\pgfpathrectangle{\pgfqpoint{1.250000in}{4.155455in}}{\pgfqpoint{2.279412in}{2.004545in}}%
\pgfusepath{clip}%
\pgfsetbuttcap%
\pgfsetroundjoin%
\pgfsetlinewidth{0.360991pt}%
\definecolor{currentstroke}{rgb}{0.277018,0.050344,0.375715}%
\pgfsetstrokecolor{currentstroke}%
\pgfsetdash{}{0pt}%
\pgfpathmoveto{\pgfqpoint{2.750177in}{5.739815in}}%
\pgfpathlineto{\pgfqpoint{2.701161in}{5.731552in}}%
\pgfusepath{stroke}%
\end{pgfscope}%
\begin{pgfscope}%
\pgfpathrectangle{\pgfqpoint{1.250000in}{4.155455in}}{\pgfqpoint{2.279412in}{2.004545in}}%
\pgfusepath{clip}%
\pgfsetbuttcap%
\pgfsetroundjoin%
\pgfsetlinewidth{0.349347pt}%
\definecolor{currentstroke}{rgb}{0.276022,0.044167,0.370164}%
\pgfsetstrokecolor{currentstroke}%
\pgfsetdash{}{0pt}%
\pgfpathmoveto{\pgfqpoint{2.701161in}{5.731552in}}%
\pgfpathlineto{\pgfqpoint{2.652307in}{5.722121in}}%
\pgfusepath{stroke}%
\end{pgfscope}%
\begin{pgfscope}%
\pgfpathrectangle{\pgfqpoint{1.250000in}{4.155455in}}{\pgfqpoint{2.279412in}{2.004545in}}%
\pgfusepath{clip}%
\pgfsetbuttcap%
\pgfsetroundjoin%
\pgfsetlinewidth{0.378044pt}%
\definecolor{currentstroke}{rgb}{0.279566,0.067836,0.391917}%
\pgfsetstrokecolor{currentstroke}%
\pgfsetdash{}{0pt}%
\pgfpathmoveto{\pgfqpoint{2.652307in}{5.722121in}}%
\pgfpathlineto{\pgfqpoint{2.603064in}{5.713942in}}%
\pgfusepath{stroke}%
\end{pgfscope}%
\begin{pgfscope}%
\pgfpathrectangle{\pgfqpoint{1.250000in}{4.155455in}}{\pgfqpoint{2.279412in}{2.004545in}}%
\pgfusepath{clip}%
\pgfsetbuttcap%
\pgfsetroundjoin%
\pgfsetlinewidth{0.384910pt}%
\definecolor{currentstroke}{rgb}{0.280267,0.073417,0.397163}%
\pgfsetstrokecolor{currentstroke}%
\pgfsetdash{}{0pt}%
\pgfpathmoveto{\pgfqpoint{2.603064in}{5.713942in}}%
\pgfpathlineto{\pgfqpoint{2.553997in}{5.704841in}}%
\pgfusepath{stroke}%
\end{pgfscope}%
\begin{pgfscope}%
\pgfpathrectangle{\pgfqpoint{1.250000in}{4.155455in}}{\pgfqpoint{2.279412in}{2.004545in}}%
\pgfusepath{clip}%
\pgfsetbuttcap%
\pgfsetroundjoin%
\pgfsetlinewidth{0.396258pt}%
\definecolor{currentstroke}{rgb}{0.280894,0.078907,0.402329}%
\pgfsetstrokecolor{currentstroke}%
\pgfsetdash{}{0pt}%
\pgfpathmoveto{\pgfqpoint{2.553997in}{5.704841in}}%
\pgfpathlineto{\pgfqpoint{2.505199in}{5.694770in}}%
\pgfusepath{stroke}%
\end{pgfscope}%
\begin{pgfscope}%
\pgfpathrectangle{\pgfqpoint{1.250000in}{4.155455in}}{\pgfqpoint{2.279412in}{2.004545in}}%
\pgfusepath{clip}%
\pgfsetbuttcap%
\pgfsetroundjoin%
\pgfsetlinewidth{0.404512pt}%
\definecolor{currentstroke}{rgb}{0.281924,0.089666,0.412415}%
\pgfsetstrokecolor{currentstroke}%
\pgfsetdash{}{0pt}%
\pgfpathmoveto{\pgfqpoint{2.505199in}{5.694770in}}%
\pgfpathlineto{\pgfqpoint{2.456904in}{5.682898in}}%
\pgfusepath{stroke}%
\end{pgfscope}%
\begin{pgfscope}%
\pgfpathrectangle{\pgfqpoint{1.250000in}{4.155455in}}{\pgfqpoint{2.279412in}{2.004545in}}%
\pgfusepath{clip}%
\pgfsetbuttcap%
\pgfsetroundjoin%
\pgfsetlinewidth{0.409619pt}%
\definecolor{currentstroke}{rgb}{0.281924,0.089666,0.412415}%
\pgfsetstrokecolor{currentstroke}%
\pgfsetdash{}{0pt}%
\pgfpathmoveto{\pgfqpoint{2.456904in}{5.682898in}}%
\pgfpathlineto{\pgfqpoint{2.409339in}{5.669036in}}%
\pgfusepath{stroke}%
\end{pgfscope}%
\begin{pgfscope}%
\pgfpathrectangle{\pgfqpoint{1.250000in}{4.155455in}}{\pgfqpoint{2.279412in}{2.004545in}}%
\pgfusepath{clip}%
\pgfsetbuttcap%
\pgfsetroundjoin%
\pgfsetlinewidth{0.417164pt}%
\definecolor{currentstroke}{rgb}{0.282327,0.094955,0.417331}%
\pgfsetstrokecolor{currentstroke}%
\pgfsetdash{}{0pt}%
\pgfpathmoveto{\pgfqpoint{2.409339in}{5.669036in}}%
\pgfpathlineto{\pgfqpoint{2.362685in}{5.652917in}}%
\pgfusepath{stroke}%
\end{pgfscope}%
\begin{pgfscope}%
\pgfpathrectangle{\pgfqpoint{1.250000in}{4.155455in}}{\pgfqpoint{2.279412in}{2.004545in}}%
\pgfusepath{clip}%
\pgfsetbuttcap%
\pgfsetroundjoin%
\pgfsetlinewidth{0.446092pt}%
\definecolor{currentstroke}{rgb}{0.283229,0.120777,0.440584}%
\pgfsetstrokecolor{currentstroke}%
\pgfsetdash{}{0pt}%
\pgfpathmoveto{\pgfqpoint{2.362685in}{5.652917in}}%
\pgfpathlineto{\pgfqpoint{2.318392in}{5.632632in}}%
\pgfusepath{stroke}%
\end{pgfscope}%
\begin{pgfscope}%
\pgfpathrectangle{\pgfqpoint{1.250000in}{4.155455in}}{\pgfqpoint{2.279412in}{2.004545in}}%
\pgfusepath{clip}%
\pgfsetbuttcap%
\pgfsetroundjoin%
\pgfsetlinewidth{0.463586pt}%
\definecolor{currentstroke}{rgb}{0.283072,0.130895,0.449241}%
\pgfsetstrokecolor{currentstroke}%
\pgfsetdash{}{0pt}%
\pgfpathmoveto{\pgfqpoint{2.318392in}{5.632632in}}%
\pgfpathlineto{\pgfqpoint{2.279342in}{5.605522in}}%
\pgfusepath{stroke}%
\end{pgfscope}%
\begin{pgfscope}%
\pgfpathrectangle{\pgfqpoint{1.250000in}{4.155455in}}{\pgfqpoint{2.279412in}{2.004545in}}%
\pgfusepath{clip}%
\pgfsetbuttcap%
\pgfsetroundjoin%
\pgfsetlinewidth{0.462631pt}%
\definecolor{currentstroke}{rgb}{0.283072,0.130895,0.449241}%
\pgfsetstrokecolor{currentstroke}%
\pgfsetdash{}{0pt}%
\pgfpathmoveto{\pgfqpoint{2.279342in}{5.605522in}}%
\pgfpathlineto{\pgfqpoint{2.279342in}{5.605522in}}%
\pgfusepath{stroke}%
\end{pgfscope}%
\begin{pgfscope}%
\pgfpathrectangle{\pgfqpoint{1.250000in}{4.155455in}}{\pgfqpoint{2.279412in}{2.004545in}}%
\pgfusepath{clip}%
\pgfsetbuttcap%
\pgfsetroundjoin%
\pgfsetlinewidth{0.462631pt}%
\definecolor{currentstroke}{rgb}{0.283072,0.130895,0.449241}%
\pgfsetstrokecolor{currentstroke}%
\pgfsetdash{}{0pt}%
\pgfpathmoveto{\pgfqpoint{2.279342in}{5.605522in}}%
\pgfpathlineto{\pgfqpoint{2.259276in}{5.580812in}}%
\pgfusepath{stroke}%
\end{pgfscope}%
\begin{pgfscope}%
\pgfpathrectangle{\pgfqpoint{1.250000in}{4.155455in}}{\pgfqpoint{2.279412in}{2.004545in}}%
\pgfusepath{clip}%
\pgfsetbuttcap%
\pgfsetroundjoin%
\pgfsetlinewidth{0.498823pt}%
\definecolor{currentstroke}{rgb}{0.281412,0.155834,0.469201}%
\pgfsetstrokecolor{currentstroke}%
\pgfsetdash{}{0pt}%
\pgfpathmoveto{\pgfqpoint{2.259276in}{5.580812in}}%
\pgfpathlineto{\pgfqpoint{2.244517in}{5.557811in}}%
\pgfusepath{stroke}%
\end{pgfscope}%
\begin{pgfscope}%
\pgfpathrectangle{\pgfqpoint{1.250000in}{4.155455in}}{\pgfqpoint{2.279412in}{2.004545in}}%
\pgfusepath{clip}%
\pgfsetbuttcap%
\pgfsetroundjoin%
\pgfsetlinewidth{0.490348pt}%
\definecolor{currentstroke}{rgb}{0.281887,0.150881,0.465405}%
\pgfsetstrokecolor{currentstroke}%
\pgfsetdash{}{0pt}%
\pgfpathmoveto{\pgfqpoint{2.244517in}{5.557811in}}%
\pgfpathlineto{\pgfqpoint{2.225922in}{5.521858in}}%
\pgfusepath{stroke}%
\end{pgfscope}%
\begin{pgfscope}%
\pgfpathrectangle{\pgfqpoint{1.250000in}{4.155455in}}{\pgfqpoint{2.279412in}{2.004545in}}%
\pgfusepath{clip}%
\pgfsetbuttcap%
\pgfsetroundjoin%
\pgfsetlinewidth{0.570913pt}%
\definecolor{currentstroke}{rgb}{0.271828,0.209303,0.504434}%
\pgfsetstrokecolor{currentstroke}%
\pgfsetdash{}{0pt}%
\pgfpathmoveto{\pgfqpoint{2.225922in}{5.521858in}}%
\pgfpathlineto{\pgfqpoint{2.207541in}{5.481471in}}%
\pgfusepath{stroke}%
\end{pgfscope}%
\begin{pgfscope}%
\pgfpathrectangle{\pgfqpoint{1.250000in}{4.155455in}}{\pgfqpoint{2.279412in}{2.004545in}}%
\pgfusepath{clip}%
\pgfsetbuttcap%
\pgfsetroundjoin%
\pgfsetlinewidth{0.599937pt}%
\definecolor{currentstroke}{rgb}{0.266580,0.228262,0.514349}%
\pgfsetstrokecolor{currentstroke}%
\pgfsetdash{}{0pt}%
\pgfpathmoveto{\pgfqpoint{2.207541in}{5.481471in}}%
\pgfpathlineto{\pgfqpoint{2.194423in}{5.439652in}}%
\pgfusepath{stroke}%
\end{pgfscope}%
\begin{pgfscope}%
\pgfpathrectangle{\pgfqpoint{1.250000in}{4.155455in}}{\pgfqpoint{2.279412in}{2.004545in}}%
\pgfusepath{clip}%
\pgfsetbuttcap%
\pgfsetroundjoin%
\pgfsetlinewidth{0.577379pt}%
\definecolor{currentstroke}{rgb}{0.270595,0.214069,0.507052}%
\pgfsetstrokecolor{currentstroke}%
\pgfsetdash{}{0pt}%
\pgfpathmoveto{\pgfqpoint{2.194423in}{5.439652in}}%
\pgfpathlineto{\pgfqpoint{2.184065in}{5.398957in}}%
\pgfusepath{stroke}%
\end{pgfscope}%
\begin{pgfscope}%
\pgfpathrectangle{\pgfqpoint{1.250000in}{4.155455in}}{\pgfqpoint{2.279412in}{2.004545in}}%
\pgfusepath{clip}%
\pgfsetbuttcap%
\pgfsetroundjoin%
\pgfsetlinewidth{0.373801pt}%
\definecolor{currentstroke}{rgb}{0.278791,0.062145,0.386592}%
\pgfsetstrokecolor{currentstroke}%
\pgfsetdash{}{0pt}%
\pgfpathmoveto{\pgfqpoint{2.436644in}{4.602607in}}%
\pgfpathlineto{\pgfqpoint{2.389706in}{4.616446in}}%
\pgfusepath{stroke}%
\end{pgfscope}%
\begin{pgfscope}%
\pgfpathrectangle{\pgfqpoint{1.250000in}{4.155455in}}{\pgfqpoint{2.279412in}{2.004545in}}%
\pgfusepath{clip}%
\pgfsetbuttcap%
\pgfsetroundjoin%
\pgfsetlinewidth{0.416665pt}%
\definecolor{currentstroke}{rgb}{0.282327,0.094955,0.417331}%
\pgfsetstrokecolor{currentstroke}%
\pgfsetdash{}{0pt}%
\pgfpathmoveto{\pgfqpoint{2.389706in}{4.616446in}}%
\pgfpathlineto{\pgfqpoint{2.345831in}{4.637314in}}%
\pgfusepath{stroke}%
\end{pgfscope}%
\begin{pgfscope}%
\pgfpathrectangle{\pgfqpoint{1.250000in}{4.155455in}}{\pgfqpoint{2.279412in}{2.004545in}}%
\pgfusepath{clip}%
\pgfsetbuttcap%
\pgfsetroundjoin%
\pgfsetlinewidth{0.412347pt}%
\definecolor{currentstroke}{rgb}{0.282327,0.094955,0.417331}%
\pgfsetstrokecolor{currentstroke}%
\pgfsetdash{}{0pt}%
\pgfpathmoveto{\pgfqpoint{2.345831in}{4.637314in}}%
\pgfpathlineto{\pgfqpoint{2.345831in}{4.637314in}}%
\pgfusepath{stroke}%
\end{pgfscope}%
\begin{pgfscope}%
\pgfpathrectangle{\pgfqpoint{1.250000in}{4.155455in}}{\pgfqpoint{2.279412in}{2.004545in}}%
\pgfusepath{clip}%
\pgfsetbuttcap%
\pgfsetroundjoin%
\pgfsetlinewidth{0.412347pt}%
\definecolor{currentstroke}{rgb}{0.282327,0.094955,0.417331}%
\pgfsetstrokecolor{currentstroke}%
\pgfsetdash{}{0pt}%
\pgfpathmoveto{\pgfqpoint{2.345831in}{4.637314in}}%
\pgfpathlineto{\pgfqpoint{2.313435in}{4.659322in}}%
\pgfusepath{stroke}%
\end{pgfscope}%
\begin{pgfscope}%
\pgfpathrectangle{\pgfqpoint{1.250000in}{4.155455in}}{\pgfqpoint{2.279412in}{2.004545in}}%
\pgfusepath{clip}%
\pgfsetbuttcap%
\pgfsetroundjoin%
\pgfsetlinewidth{0.432400pt}%
\definecolor{currentstroke}{rgb}{0.283091,0.110553,0.431554}%
\pgfsetstrokecolor{currentstroke}%
\pgfsetdash{}{0pt}%
\pgfpathmoveto{\pgfqpoint{2.313435in}{4.659322in}}%
\pgfpathlineto{\pgfqpoint{2.283923in}{4.684851in}}%
\pgfusepath{stroke}%
\end{pgfscope}%
\begin{pgfscope}%
\pgfpathrectangle{\pgfqpoint{1.250000in}{4.155455in}}{\pgfqpoint{2.279412in}{2.004545in}}%
\pgfusepath{clip}%
\pgfsetbuttcap%
\pgfsetroundjoin%
\pgfsetlinewidth{0.453675pt}%
\definecolor{currentstroke}{rgb}{0.283187,0.125848,0.444960}%
\pgfsetstrokecolor{currentstroke}%
\pgfsetdash{}{0pt}%
\pgfpathmoveto{\pgfqpoint{2.283923in}{4.684851in}}%
\pgfpathlineto{\pgfqpoint{2.283923in}{4.684851in}}%
\pgfusepath{stroke}%
\end{pgfscope}%
\begin{pgfscope}%
\pgfpathrectangle{\pgfqpoint{1.250000in}{4.155455in}}{\pgfqpoint{2.279412in}{2.004545in}}%
\pgfusepath{clip}%
\pgfsetbuttcap%
\pgfsetroundjoin%
\pgfsetlinewidth{0.453675pt}%
\definecolor{currentstroke}{rgb}{0.283187,0.125848,0.444960}%
\pgfsetstrokecolor{currentstroke}%
\pgfsetdash{}{0pt}%
\pgfpathmoveto{\pgfqpoint{2.283923in}{4.684851in}}%
\pgfpathlineto{\pgfqpoint{2.260525in}{4.708508in}}%
\pgfusepath{stroke}%
\end{pgfscope}%
\begin{pgfscope}%
\pgfpathrectangle{\pgfqpoint{1.250000in}{4.155455in}}{\pgfqpoint{2.279412in}{2.004545in}}%
\pgfusepath{clip}%
\pgfsetbuttcap%
\pgfsetroundjoin%
\pgfsetlinewidth{0.476667pt}%
\definecolor{currentstroke}{rgb}{0.282623,0.140926,0.457517}%
\pgfsetstrokecolor{currentstroke}%
\pgfsetdash{}{0pt}%
\pgfpathmoveto{\pgfqpoint{2.260525in}{4.708508in}}%
\pgfpathlineto{\pgfqpoint{2.260525in}{4.708508in}}%
\pgfusepath{stroke}%
\end{pgfscope}%
\begin{pgfscope}%
\pgfpathrectangle{\pgfqpoint{1.250000in}{4.155455in}}{\pgfqpoint{2.279412in}{2.004545in}}%
\pgfusepath{clip}%
\pgfsetbuttcap%
\pgfsetroundjoin%
\pgfsetlinewidth{0.476667pt}%
\definecolor{currentstroke}{rgb}{0.282623,0.140926,0.457517}%
\pgfsetstrokecolor{currentstroke}%
\pgfsetdash{}{0pt}%
\pgfpathmoveto{\pgfqpoint{2.260525in}{4.708508in}}%
\pgfpathlineto{\pgfqpoint{2.244865in}{4.734108in}}%
\pgfusepath{stroke}%
\end{pgfscope}%
\begin{pgfscope}%
\pgfpathrectangle{\pgfqpoint{1.250000in}{4.155455in}}{\pgfqpoint{2.279412in}{2.004545in}}%
\pgfusepath{clip}%
\pgfsetbuttcap%
\pgfsetroundjoin%
\pgfsetlinewidth{0.474515pt}%
\definecolor{currentstroke}{rgb}{0.282623,0.140926,0.457517}%
\pgfsetstrokecolor{currentstroke}%
\pgfsetdash{}{0pt}%
\pgfpathmoveto{\pgfqpoint{2.244865in}{4.734108in}}%
\pgfpathlineto{\pgfqpoint{2.231112in}{4.760486in}}%
\pgfusepath{stroke}%
\end{pgfscope}%
\begin{pgfscope}%
\pgfpathrectangle{\pgfqpoint{1.250000in}{4.155455in}}{\pgfqpoint{2.279412in}{2.004545in}}%
\pgfusepath{clip}%
\pgfsetbuttcap%
\pgfsetroundjoin%
\pgfsetlinewidth{0.497693pt}%
\definecolor{currentstroke}{rgb}{0.281412,0.155834,0.469201}%
\pgfsetstrokecolor{currentstroke}%
\pgfsetdash{}{0pt}%
\pgfpathmoveto{\pgfqpoint{2.231112in}{4.760486in}}%
\pgfpathlineto{\pgfqpoint{2.231112in}{4.760486in}}%
\pgfusepath{stroke}%
\end{pgfscope}%
\begin{pgfscope}%
\pgfpathrectangle{\pgfqpoint{1.250000in}{4.155455in}}{\pgfqpoint{2.279412in}{2.004545in}}%
\pgfusepath{clip}%
\pgfsetbuttcap%
\pgfsetroundjoin%
\pgfsetlinewidth{0.497693pt}%
\definecolor{currentstroke}{rgb}{0.281412,0.155834,0.469201}%
\pgfsetstrokecolor{currentstroke}%
\pgfsetdash{}{0pt}%
\pgfpathmoveto{\pgfqpoint{2.231112in}{4.760486in}}%
\pgfpathlineto{\pgfqpoint{2.217456in}{4.790397in}}%
\pgfusepath{stroke}%
\end{pgfscope}%
\begin{pgfscope}%
\pgfpathrectangle{\pgfqpoint{1.250000in}{4.155455in}}{\pgfqpoint{2.279412in}{2.004545in}}%
\pgfusepath{clip}%
\pgfsetbuttcap%
\pgfsetroundjoin%
\pgfsetlinewidth{0.505059pt}%
\definecolor{currentstroke}{rgb}{0.280868,0.160771,0.472899}%
\pgfsetstrokecolor{currentstroke}%
\pgfsetdash{}{0pt}%
\pgfpathmoveto{\pgfqpoint{2.217456in}{4.790397in}}%
\pgfpathlineto{\pgfqpoint{2.208180in}{4.823645in}}%
\pgfusepath{stroke}%
\end{pgfscope}%
\begin{pgfscope}%
\pgfpathrectangle{\pgfqpoint{1.250000in}{4.155455in}}{\pgfqpoint{2.279412in}{2.004545in}}%
\pgfusepath{clip}%
\pgfsetbuttcap%
\pgfsetroundjoin%
\pgfsetlinewidth{0.565136pt}%
\definecolor{currentstroke}{rgb}{0.273006,0.204520,0.501721}%
\pgfsetstrokecolor{currentstroke}%
\pgfsetdash{}{0pt}%
\pgfpathmoveto{\pgfqpoint{2.208180in}{4.823645in}}%
\pgfpathlineto{\pgfqpoint{2.195073in}{4.865293in}}%
\pgfusepath{stroke}%
\end{pgfscope}%
\begin{pgfscope}%
\pgfpathrectangle{\pgfqpoint{1.250000in}{4.155455in}}{\pgfqpoint{2.279412in}{2.004545in}}%
\pgfusepath{clip}%
\pgfsetbuttcap%
\pgfsetroundjoin%
\pgfsetlinewidth{0.572638pt}%
\definecolor{currentstroke}{rgb}{0.271828,0.209303,0.504434}%
\pgfsetstrokecolor{currentstroke}%
\pgfsetdash{}{0pt}%
\pgfpathmoveto{\pgfqpoint{2.195073in}{4.865293in}}%
\pgfpathlineto{\pgfqpoint{2.182757in}{4.906526in}}%
\pgfusepath{stroke}%
\end{pgfscope}%
\begin{pgfscope}%
\pgfpathrectangle{\pgfqpoint{1.250000in}{4.155455in}}{\pgfqpoint{2.279412in}{2.004545in}}%
\pgfusepath{clip}%
\pgfsetbuttcap%
\pgfsetroundjoin%
\pgfsetlinewidth{0.353689pt}%
\definecolor{currentstroke}{rgb}{0.276022,0.044167,0.370164}%
\pgfsetstrokecolor{currentstroke}%
\pgfsetdash{}{0pt}%
\pgfpathmoveto{\pgfqpoint{2.745269in}{4.602894in}}%
\pgfpathlineto{\pgfqpoint{2.695719in}{4.609688in}}%
\pgfusepath{stroke}%
\end{pgfscope}%
\begin{pgfscope}%
\pgfpathrectangle{\pgfqpoint{1.250000in}{4.155455in}}{\pgfqpoint{2.279412in}{2.004545in}}%
\pgfusepath{clip}%
\pgfsetbuttcap%
\pgfsetroundjoin%
\pgfsetlinewidth{0.379545pt}%
\definecolor{currentstroke}{rgb}{0.279566,0.067836,0.391917}%
\pgfsetstrokecolor{currentstroke}%
\pgfsetdash{}{0pt}%
\pgfpathmoveto{\pgfqpoint{2.695719in}{4.609688in}}%
\pgfpathlineto{\pgfqpoint{2.646165in}{4.616446in}}%
\pgfusepath{stroke}%
\end{pgfscope}%
\begin{pgfscope}%
\pgfpathrectangle{\pgfqpoint{1.250000in}{4.155455in}}{\pgfqpoint{2.279412in}{2.004545in}}%
\pgfusepath{clip}%
\pgfsetbuttcap%
\pgfsetroundjoin%
\pgfsetlinewidth{0.377201pt}%
\definecolor{currentstroke}{rgb}{0.279566,0.067836,0.391917}%
\pgfsetstrokecolor{currentstroke}%
\pgfsetdash{}{0pt}%
\pgfpathmoveto{\pgfqpoint{2.646165in}{4.616446in}}%
\pgfpathlineto{\pgfqpoint{2.596803in}{4.624168in}}%
\pgfusepath{stroke}%
\end{pgfscope}%
\begin{pgfscope}%
\pgfpathrectangle{\pgfqpoint{1.250000in}{4.155455in}}{\pgfqpoint{2.279412in}{2.004545in}}%
\pgfusepath{clip}%
\pgfsetbuttcap%
\pgfsetroundjoin%
\pgfsetlinewidth{0.389478pt}%
\definecolor{currentstroke}{rgb}{0.280267,0.073417,0.397163}%
\pgfsetstrokecolor{currentstroke}%
\pgfsetdash{}{0pt}%
\pgfpathmoveto{\pgfqpoint{2.596803in}{4.624168in}}%
\pgfpathlineto{\pgfqpoint{2.547703in}{4.633127in}}%
\pgfusepath{stroke}%
\end{pgfscope}%
\begin{pgfscope}%
\pgfpathrectangle{\pgfqpoint{1.250000in}{4.155455in}}{\pgfqpoint{2.279412in}{2.004545in}}%
\pgfusepath{clip}%
\pgfsetbuttcap%
\pgfsetroundjoin%
\pgfsetlinewidth{0.412063pt}%
\definecolor{currentstroke}{rgb}{0.282327,0.094955,0.417331}%
\pgfsetstrokecolor{currentstroke}%
\pgfsetdash{}{0pt}%
\pgfpathmoveto{\pgfqpoint{2.547703in}{4.633127in}}%
\pgfpathlineto{\pgfqpoint{2.499072in}{4.643819in}}%
\pgfusepath{stroke}%
\end{pgfscope}%
\begin{pgfscope}%
\pgfpathrectangle{\pgfqpoint{1.250000in}{4.155455in}}{\pgfqpoint{2.279412in}{2.004545in}}%
\pgfusepath{clip}%
\pgfsetbuttcap%
\pgfsetroundjoin%
\pgfsetlinewidth{0.451495pt}%
\definecolor{currentstroke}{rgb}{0.283229,0.120777,0.440584}%
\pgfsetstrokecolor{currentstroke}%
\pgfsetdash{}{0pt}%
\pgfpathmoveto{\pgfqpoint{2.499072in}{4.643819in}}%
\pgfpathlineto{\pgfqpoint{2.451154in}{4.656741in}}%
\pgfusepath{stroke}%
\end{pgfscope}%
\begin{pgfscope}%
\pgfpathrectangle{\pgfqpoint{1.250000in}{4.155455in}}{\pgfqpoint{2.279412in}{2.004545in}}%
\pgfusepath{clip}%
\pgfsetbuttcap%
\pgfsetroundjoin%
\pgfsetlinewidth{0.441727pt}%
\definecolor{currentstroke}{rgb}{0.283197,0.115680,0.436115}%
\pgfsetstrokecolor{currentstroke}%
\pgfsetdash{}{0pt}%
\pgfpathmoveto{\pgfqpoint{2.451154in}{4.656741in}}%
\pgfpathlineto{\pgfqpoint{2.404092in}{4.671843in}}%
\pgfusepath{stroke}%
\end{pgfscope}%
\begin{pgfscope}%
\pgfpathrectangle{\pgfqpoint{1.250000in}{4.155455in}}{\pgfqpoint{2.279412in}{2.004545in}}%
\pgfusepath{clip}%
\pgfsetbuttcap%
\pgfsetroundjoin%
\pgfsetlinewidth{0.483024pt}%
\definecolor{currentstroke}{rgb}{0.282290,0.145912,0.461510}%
\pgfsetstrokecolor{currentstroke}%
\pgfsetdash{}{0pt}%
\pgfpathmoveto{\pgfqpoint{2.404092in}{4.671843in}}%
\pgfpathlineto{\pgfqpoint{2.358540in}{4.690111in}}%
\pgfusepath{stroke}%
\end{pgfscope}%
\begin{pgfscope}%
\pgfpathrectangle{\pgfqpoint{1.250000in}{4.155455in}}{\pgfqpoint{2.279412in}{2.004545in}}%
\pgfusepath{clip}%
\pgfsetbuttcap%
\pgfsetroundjoin%
\pgfsetlinewidth{0.458802pt}%
\definecolor{currentstroke}{rgb}{0.283187,0.125848,0.444960}%
\pgfsetstrokecolor{currentstroke}%
\pgfsetdash{}{0pt}%
\pgfpathmoveto{\pgfqpoint{2.358540in}{4.690111in}}%
\pgfpathlineto{\pgfqpoint{2.316524in}{4.713742in}}%
\pgfusepath{stroke}%
\end{pgfscope}%
\begin{pgfscope}%
\pgfpathrectangle{\pgfqpoint{1.250000in}{4.155455in}}{\pgfqpoint{2.279412in}{2.004545in}}%
\pgfusepath{clip}%
\pgfsetbuttcap%
\pgfsetroundjoin%
\pgfsetlinewidth{0.500943pt}%
\definecolor{currentstroke}{rgb}{0.280868,0.160771,0.472899}%
\pgfsetstrokecolor{currentstroke}%
\pgfsetdash{}{0pt}%
\pgfpathmoveto{\pgfqpoint{2.316524in}{4.713742in}}%
\pgfpathlineto{\pgfqpoint{2.280249in}{4.743645in}}%
\pgfusepath{stroke}%
\end{pgfscope}%
\begin{pgfscope}%
\pgfpathrectangle{\pgfqpoint{1.250000in}{4.155455in}}{\pgfqpoint{2.279412in}{2.004545in}}%
\pgfusepath{clip}%
\pgfsetbuttcap%
\pgfsetroundjoin%
\pgfsetlinewidth{0.320464pt}%
\definecolor{currentstroke}{rgb}{0.269944,0.014625,0.341379}%
\pgfsetstrokecolor{currentstroke}%
\pgfsetdash{}{0pt}%
\pgfpathmoveto{\pgfqpoint{3.005209in}{5.699009in}}%
\pgfpathlineto{\pgfqpoint{2.955322in}{5.695074in}}%
\pgfusepath{stroke}%
\end{pgfscope}%
\begin{pgfscope}%
\pgfpathrectangle{\pgfqpoint{1.250000in}{4.155455in}}{\pgfqpoint{2.279412in}{2.004545in}}%
\pgfusepath{clip}%
\pgfsetbuttcap%
\pgfsetroundjoin%
\pgfsetlinewidth{0.328374pt}%
\definecolor{currentstroke}{rgb}{0.271305,0.019942,0.347269}%
\pgfsetstrokecolor{currentstroke}%
\pgfsetdash{}{0pt}%
\pgfpathmoveto{\pgfqpoint{2.955322in}{5.695074in}}%
\pgfpathlineto{\pgfqpoint{2.905631in}{5.689246in}}%
\pgfusepath{stroke}%
\end{pgfscope}%
\begin{pgfscope}%
\pgfpathrectangle{\pgfqpoint{1.250000in}{4.155455in}}{\pgfqpoint{2.279412in}{2.004545in}}%
\pgfusepath{clip}%
\pgfsetbuttcap%
\pgfsetroundjoin%
\pgfsetlinewidth{0.339895pt}%
\definecolor{currentstroke}{rgb}{0.273809,0.031497,0.358853}%
\pgfsetstrokecolor{currentstroke}%
\pgfsetdash{}{0pt}%
\pgfpathmoveto{\pgfqpoint{2.905631in}{5.689246in}}%
\pgfpathlineto{\pgfqpoint{2.855855in}{5.684012in}}%
\pgfusepath{stroke}%
\end{pgfscope}%
\begin{pgfscope}%
\pgfpathrectangle{\pgfqpoint{1.250000in}{4.155455in}}{\pgfqpoint{2.279412in}{2.004545in}}%
\pgfusepath{clip}%
\pgfsetbuttcap%
\pgfsetroundjoin%
\pgfsetlinewidth{0.341424pt}%
\definecolor{currentstroke}{rgb}{0.273809,0.031497,0.358853}%
\pgfsetstrokecolor{currentstroke}%
\pgfsetdash{}{0pt}%
\pgfpathmoveto{\pgfqpoint{2.855855in}{5.684012in}}%
\pgfpathlineto{\pgfqpoint{2.806086in}{5.678681in}}%
\pgfusepath{stroke}%
\end{pgfscope}%
\begin{pgfscope}%
\pgfpathrectangle{\pgfqpoint{1.250000in}{4.155455in}}{\pgfqpoint{2.279412in}{2.004545in}}%
\pgfusepath{clip}%
\pgfsetbuttcap%
\pgfsetroundjoin%
\pgfsetlinewidth{0.350428pt}%
\definecolor{currentstroke}{rgb}{0.276022,0.044167,0.370164}%
\pgfsetstrokecolor{currentstroke}%
\pgfsetdash{}{0pt}%
\pgfpathmoveto{\pgfqpoint{2.806086in}{5.678681in}}%
\pgfpathlineto{\pgfqpoint{2.756388in}{5.672818in}}%
\pgfusepath{stroke}%
\end{pgfscope}%
\begin{pgfscope}%
\pgfpathrectangle{\pgfqpoint{1.250000in}{4.155455in}}{\pgfqpoint{2.279412in}{2.004545in}}%
\pgfusepath{clip}%
\pgfsetbuttcap%
\pgfsetroundjoin%
\pgfsetlinewidth{0.359144pt}%
\definecolor{currentstroke}{rgb}{0.277018,0.050344,0.375715}%
\pgfsetstrokecolor{currentstroke}%
\pgfsetdash{}{0pt}%
\pgfpathmoveto{\pgfqpoint{2.756388in}{5.672818in}}%
\pgfpathlineto{\pgfqpoint{2.706676in}{5.667035in}}%
\pgfusepath{stroke}%
\end{pgfscope}%
\begin{pgfscope}%
\pgfpathrectangle{\pgfqpoint{1.250000in}{4.155455in}}{\pgfqpoint{2.279412in}{2.004545in}}%
\pgfusepath{clip}%
\pgfsetbuttcap%
\pgfsetroundjoin%
\pgfsetlinewidth{0.385223pt}%
\definecolor{currentstroke}{rgb}{0.280267,0.073417,0.397163}%
\pgfsetstrokecolor{currentstroke}%
\pgfsetdash{}{0pt}%
\pgfpathmoveto{\pgfqpoint{2.706676in}{5.667035in}}%
\pgfpathlineto{\pgfqpoint{2.657004in}{5.661005in}}%
\pgfusepath{stroke}%
\end{pgfscope}%
\begin{pgfscope}%
\pgfpathrectangle{\pgfqpoint{1.250000in}{4.155455in}}{\pgfqpoint{2.279412in}{2.004545in}}%
\pgfusepath{clip}%
\pgfsetbuttcap%
\pgfsetroundjoin%
\pgfsetlinewidth{0.405847pt}%
\definecolor{currentstroke}{rgb}{0.281924,0.089666,0.412415}%
\pgfsetstrokecolor{currentstroke}%
\pgfsetdash{}{0pt}%
\pgfpathmoveto{\pgfqpoint{2.657004in}{5.661005in}}%
\pgfpathlineto{\pgfqpoint{2.607466in}{5.654204in}}%
\pgfusepath{stroke}%
\end{pgfscope}%
\begin{pgfscope}%
\pgfpathrectangle{\pgfqpoint{1.250000in}{4.155455in}}{\pgfqpoint{2.279412in}{2.004545in}}%
\pgfusepath{clip}%
\pgfsetbuttcap%
\pgfsetroundjoin%
\pgfsetlinewidth{0.385369pt}%
\definecolor{currentstroke}{rgb}{0.280267,0.073417,0.397163}%
\pgfsetstrokecolor{currentstroke}%
\pgfsetdash{}{0pt}%
\pgfpathmoveto{\pgfqpoint{2.607466in}{5.654204in}}%
\pgfpathlineto{\pgfqpoint{2.558063in}{5.646706in}}%
\pgfusepath{stroke}%
\end{pgfscope}%
\begin{pgfscope}%
\pgfpathrectangle{\pgfqpoint{1.250000in}{4.155455in}}{\pgfqpoint{2.279412in}{2.004545in}}%
\pgfusepath{clip}%
\pgfsetbuttcap%
\pgfsetroundjoin%
\pgfsetlinewidth{0.437468pt}%
\definecolor{currentstroke}{rgb}{0.283091,0.110553,0.431554}%
\pgfsetstrokecolor{currentstroke}%
\pgfsetdash{}{0pt}%
\pgfpathmoveto{\pgfqpoint{2.558063in}{5.646706in}}%
\pgfpathlineto{\pgfqpoint{2.508963in}{5.637821in}}%
\pgfusepath{stroke}%
\end{pgfscope}%
\begin{pgfscope}%
\pgfpathrectangle{\pgfqpoint{1.250000in}{4.155455in}}{\pgfqpoint{2.279412in}{2.004545in}}%
\pgfusepath{clip}%
\pgfsetbuttcap%
\pgfsetroundjoin%
\pgfsetlinewidth{0.448635pt}%
\definecolor{currentstroke}{rgb}{0.283229,0.120777,0.440584}%
\pgfsetstrokecolor{currentstroke}%
\pgfsetdash{}{0pt}%
\pgfpathmoveto{\pgfqpoint{2.508963in}{5.637821in}}%
\pgfpathlineto{\pgfqpoint{2.460357in}{5.626994in}}%
\pgfusepath{stroke}%
\end{pgfscope}%
\begin{pgfscope}%
\pgfpathrectangle{\pgfqpoint{1.250000in}{4.155455in}}{\pgfqpoint{2.279412in}{2.004545in}}%
\pgfusepath{clip}%
\pgfsetbuttcap%
\pgfsetroundjoin%
\pgfsetlinewidth{0.462999pt}%
\definecolor{currentstroke}{rgb}{0.283072,0.130895,0.449241}%
\pgfsetstrokecolor{currentstroke}%
\pgfsetdash{}{0pt}%
\pgfpathmoveto{\pgfqpoint{2.460357in}{5.626994in}}%
\pgfpathlineto{\pgfqpoint{2.412450in}{5.614019in}}%
\pgfusepath{stroke}%
\end{pgfscope}%
\begin{pgfscope}%
\pgfpathrectangle{\pgfqpoint{1.250000in}{4.155455in}}{\pgfqpoint{2.279412in}{2.004545in}}%
\pgfusepath{clip}%
\pgfsetbuttcap%
\pgfsetroundjoin%
\pgfsetlinewidth{0.516544pt}%
\definecolor{currentstroke}{rgb}{0.279574,0.170599,0.479997}%
\pgfsetstrokecolor{currentstroke}%
\pgfsetdash{}{0pt}%
\pgfpathmoveto{\pgfqpoint{2.412450in}{5.614019in}}%
\pgfpathlineto{\pgfqpoint{2.365726in}{5.598119in}}%
\pgfusepath{stroke}%
\end{pgfscope}%
\begin{pgfscope}%
\pgfpathrectangle{\pgfqpoint{1.250000in}{4.155455in}}{\pgfqpoint{2.279412in}{2.004545in}}%
\pgfusepath{clip}%
\pgfsetbuttcap%
\pgfsetroundjoin%
\pgfsetlinewidth{0.475636pt}%
\definecolor{currentstroke}{rgb}{0.282623,0.140926,0.457517}%
\pgfsetstrokecolor{currentstroke}%
\pgfsetdash{}{0pt}%
\pgfpathmoveto{\pgfqpoint{2.365726in}{5.598119in}}%
\pgfpathlineto{\pgfqpoint{2.322034in}{5.576948in}}%
\pgfusepath{stroke}%
\end{pgfscope}%
\begin{pgfscope}%
\pgfpathrectangle{\pgfqpoint{1.250000in}{4.155455in}}{\pgfqpoint{2.279412in}{2.004545in}}%
\pgfusepath{clip}%
\pgfsetbuttcap%
\pgfsetroundjoin%
\pgfsetlinewidth{0.483086pt}%
\definecolor{currentstroke}{rgb}{0.282290,0.145912,0.461510}%
\pgfsetstrokecolor{currentstroke}%
\pgfsetdash{}{0pt}%
\pgfpathmoveto{\pgfqpoint{2.322034in}{5.576948in}}%
\pgfpathlineto{\pgfqpoint{2.282535in}{5.550328in}}%
\pgfusepath{stroke}%
\end{pgfscope}%
\begin{pgfscope}%
\pgfpathrectangle{\pgfqpoint{1.250000in}{4.155455in}}{\pgfqpoint{2.279412in}{2.004545in}}%
\pgfusepath{clip}%
\pgfsetbuttcap%
\pgfsetroundjoin%
\pgfsetlinewidth{0.533677pt}%
\definecolor{currentstroke}{rgb}{0.278012,0.180367,0.486697}%
\pgfsetstrokecolor{currentstroke}%
\pgfsetdash{}{0pt}%
\pgfpathmoveto{\pgfqpoint{2.282535in}{5.550328in}}%
\pgfpathlineto{\pgfqpoint{2.282535in}{5.550328in}}%
\pgfusepath{stroke}%
\end{pgfscope}%
\begin{pgfscope}%
\pgfpathrectangle{\pgfqpoint{1.250000in}{4.155455in}}{\pgfqpoint{2.279412in}{2.004545in}}%
\pgfusepath{clip}%
\pgfsetbuttcap%
\pgfsetroundjoin%
\pgfsetlinewidth{0.334901pt}%
\definecolor{currentstroke}{rgb}{0.272594,0.025563,0.353093}%
\pgfsetstrokecolor{currentstroke}%
\pgfsetdash{}{0pt}%
\pgfpathmoveto{\pgfqpoint{2.892336in}{4.622722in}}%
\pgfpathlineto{\pgfqpoint{2.842614in}{4.628110in}}%
\pgfusepath{stroke}%
\end{pgfscope}%
\begin{pgfscope}%
\pgfpathrectangle{\pgfqpoint{1.250000in}{4.155455in}}{\pgfqpoint{2.279412in}{2.004545in}}%
\pgfusepath{clip}%
\pgfsetbuttcap%
\pgfsetroundjoin%
\pgfsetlinewidth{0.331048pt}%
\definecolor{currentstroke}{rgb}{0.272594,0.025563,0.353093}%
\pgfsetstrokecolor{currentstroke}%
\pgfsetdash{}{0pt}%
\pgfpathmoveto{\pgfqpoint{2.842614in}{4.628110in}}%
\pgfpathlineto{\pgfqpoint{2.792961in}{4.634243in}}%
\pgfusepath{stroke}%
\end{pgfscope}%
\begin{pgfscope}%
\pgfpathrectangle{\pgfqpoint{1.250000in}{4.155455in}}{\pgfqpoint{2.279412in}{2.004545in}}%
\pgfusepath{clip}%
\pgfsetbuttcap%
\pgfsetroundjoin%
\pgfsetlinewidth{0.354169pt}%
\definecolor{currentstroke}{rgb}{0.276022,0.044167,0.370164}%
\pgfsetstrokecolor{currentstroke}%
\pgfsetdash{}{0pt}%
\pgfpathmoveto{\pgfqpoint{2.792961in}{4.634243in}}%
\pgfpathlineto{\pgfqpoint{2.743216in}{4.639833in}}%
\pgfusepath{stroke}%
\end{pgfscope}%
\begin{pgfscope}%
\pgfpathrectangle{\pgfqpoint{1.250000in}{4.155455in}}{\pgfqpoint{2.279412in}{2.004545in}}%
\pgfusepath{clip}%
\pgfsetbuttcap%
\pgfsetroundjoin%
\pgfsetlinewidth{0.370426pt}%
\definecolor{currentstroke}{rgb}{0.278791,0.062145,0.386592}%
\pgfsetstrokecolor{currentstroke}%
\pgfsetdash{}{0pt}%
\pgfpathmoveto{\pgfqpoint{2.743216in}{4.639833in}}%
\pgfpathlineto{\pgfqpoint{2.693584in}{4.646117in}}%
\pgfusepath{stroke}%
\end{pgfscope}%
\begin{pgfscope}%
\pgfpathrectangle{\pgfqpoint{1.250000in}{4.155455in}}{\pgfqpoint{2.279412in}{2.004545in}}%
\pgfusepath{clip}%
\pgfsetbuttcap%
\pgfsetroundjoin%
\pgfsetlinewidth{0.382164pt}%
\definecolor{currentstroke}{rgb}{0.279566,0.067836,0.391917}%
\pgfsetstrokecolor{currentstroke}%
\pgfsetdash{}{0pt}%
\pgfpathmoveto{\pgfqpoint{2.693584in}{4.646117in}}%
\pgfpathlineto{\pgfqpoint{2.644074in}{4.653124in}}%
\pgfusepath{stroke}%
\end{pgfscope}%
\begin{pgfscope}%
\pgfpathrectangle{\pgfqpoint{1.250000in}{4.155455in}}{\pgfqpoint{2.279412in}{2.004545in}}%
\pgfusepath{clip}%
\pgfsetbuttcap%
\pgfsetroundjoin%
\pgfsetlinewidth{0.398587pt}%
\definecolor{currentstroke}{rgb}{0.281446,0.084320,0.407414}%
\pgfsetstrokecolor{currentstroke}%
\pgfsetdash{}{0pt}%
\pgfpathmoveto{\pgfqpoint{2.644074in}{4.653124in}}%
\pgfpathlineto{\pgfqpoint{2.594873in}{4.661553in}}%
\pgfusepath{stroke}%
\end{pgfscope}%
\begin{pgfscope}%
\pgfpathrectangle{\pgfqpoint{1.250000in}{4.155455in}}{\pgfqpoint{2.279412in}{2.004545in}}%
\pgfusepath{clip}%
\pgfsetbuttcap%
\pgfsetroundjoin%
\pgfsetlinewidth{0.433347pt}%
\definecolor{currentstroke}{rgb}{0.283091,0.110553,0.431554}%
\pgfsetstrokecolor{currentstroke}%
\pgfsetdash{}{0pt}%
\pgfpathmoveto{\pgfqpoint{2.594873in}{4.661553in}}%
\pgfpathlineto{\pgfqpoint{2.545805in}{4.670634in}}%
\pgfusepath{stroke}%
\end{pgfscope}%
\begin{pgfscope}%
\pgfpathrectangle{\pgfqpoint{1.250000in}{4.155455in}}{\pgfqpoint{2.279412in}{2.004545in}}%
\pgfusepath{clip}%
\pgfsetbuttcap%
\pgfsetroundjoin%
\pgfsetlinewidth{0.321757pt}%
\definecolor{currentstroke}{rgb}{0.271305,0.019942,0.347269}%
\pgfsetstrokecolor{currentstroke}%
\pgfsetdash{}{0pt}%
\pgfpathmoveto{\pgfqpoint{3.053047in}{5.619320in}}%
\pgfpathlineto{\pgfqpoint{3.003340in}{5.615595in}}%
\pgfusepath{stroke}%
\end{pgfscope}%
\begin{pgfscope}%
\pgfpathrectangle{\pgfqpoint{1.250000in}{4.155455in}}{\pgfqpoint{2.279412in}{2.004545in}}%
\pgfusepath{clip}%
\pgfsetbuttcap%
\pgfsetroundjoin%
\pgfsetlinewidth{0.326451pt}%
\definecolor{currentstroke}{rgb}{0.271305,0.019942,0.347269}%
\pgfsetstrokecolor{currentstroke}%
\pgfsetdash{}{0pt}%
\pgfpathmoveto{\pgfqpoint{3.003340in}{5.615595in}}%
\pgfpathlineto{\pgfqpoint{2.953917in}{5.608795in}}%
\pgfusepath{stroke}%
\end{pgfscope}%
\begin{pgfscope}%
\pgfpathrectangle{\pgfqpoint{1.250000in}{4.155455in}}{\pgfqpoint{2.279412in}{2.004545in}}%
\pgfusepath{clip}%
\pgfsetbuttcap%
\pgfsetroundjoin%
\pgfsetlinewidth{0.334746pt}%
\definecolor{currentstroke}{rgb}{0.272594,0.025563,0.353093}%
\pgfsetstrokecolor{currentstroke}%
\pgfsetdash{}{0pt}%
\pgfpathmoveto{\pgfqpoint{2.953917in}{5.608795in}}%
\pgfpathlineto{\pgfqpoint{2.904150in}{5.603887in}}%
\pgfusepath{stroke}%
\end{pgfscope}%
\begin{pgfscope}%
\pgfpathrectangle{\pgfqpoint{1.250000in}{4.155455in}}{\pgfqpoint{2.279412in}{2.004545in}}%
\pgfusepath{clip}%
\pgfsetbuttcap%
\pgfsetroundjoin%
\pgfsetlinewidth{0.351177pt}%
\definecolor{currentstroke}{rgb}{0.276022,0.044167,0.370164}%
\pgfsetstrokecolor{currentstroke}%
\pgfsetdash{}{0pt}%
\pgfpathmoveto{\pgfqpoint{2.904150in}{5.603887in}}%
\pgfpathlineto{\pgfqpoint{2.854178in}{5.600308in}}%
\pgfusepath{stroke}%
\end{pgfscope}%
\begin{pgfscope}%
\pgfpathrectangle{\pgfqpoint{1.250000in}{4.155455in}}{\pgfqpoint{2.279412in}{2.004545in}}%
\pgfusepath{clip}%
\pgfsetbuttcap%
\pgfsetroundjoin%
\pgfsetlinewidth{0.363681pt}%
\definecolor{currentstroke}{rgb}{0.277941,0.056324,0.381191}%
\pgfsetstrokecolor{currentstroke}%
\pgfsetdash{}{0pt}%
\pgfpathmoveto{\pgfqpoint{2.854178in}{5.600308in}}%
\pgfpathlineto{\pgfqpoint{2.804328in}{5.595601in}}%
\pgfusepath{stroke}%
\end{pgfscope}%
\begin{pgfscope}%
\pgfpathrectangle{\pgfqpoint{1.250000in}{4.155455in}}{\pgfqpoint{2.279412in}{2.004545in}}%
\pgfusepath{clip}%
\pgfsetbuttcap%
\pgfsetroundjoin%
\pgfsetlinewidth{0.384830pt}%
\definecolor{currentstroke}{rgb}{0.280267,0.073417,0.397163}%
\pgfsetstrokecolor{currentstroke}%
\pgfsetdash{}{0pt}%
\pgfpathmoveto{\pgfqpoint{2.804328in}{5.595601in}}%
\pgfpathlineto{\pgfqpoint{2.754530in}{5.590399in}}%
\pgfusepath{stroke}%
\end{pgfscope}%
\begin{pgfscope}%
\pgfpathrectangle{\pgfqpoint{1.250000in}{4.155455in}}{\pgfqpoint{2.279412in}{2.004545in}}%
\pgfusepath{clip}%
\pgfsetbuttcap%
\pgfsetroundjoin%
\pgfsetlinewidth{0.415982pt}%
\definecolor{currentstroke}{rgb}{0.282327,0.094955,0.417331}%
\pgfsetstrokecolor{currentstroke}%
\pgfsetdash{}{0pt}%
\pgfpathmoveto{\pgfqpoint{2.754530in}{5.590399in}}%
\pgfpathlineto{\pgfqpoint{2.704728in}{5.585227in}}%
\pgfusepath{stroke}%
\end{pgfscope}%
\begin{pgfscope}%
\pgfpathrectangle{\pgfqpoint{1.250000in}{4.155455in}}{\pgfqpoint{2.279412in}{2.004545in}}%
\pgfusepath{clip}%
\pgfsetbuttcap%
\pgfsetroundjoin%
\pgfsetlinewidth{0.432081pt}%
\definecolor{currentstroke}{rgb}{0.283091,0.110553,0.431554}%
\pgfsetstrokecolor{currentstroke}%
\pgfsetdash{}{0pt}%
\pgfpathmoveto{\pgfqpoint{2.704728in}{5.585227in}}%
\pgfpathlineto{\pgfqpoint{2.655001in}{5.579509in}}%
\pgfusepath{stroke}%
\end{pgfscope}%
\begin{pgfscope}%
\pgfpathrectangle{\pgfqpoint{1.250000in}{4.155455in}}{\pgfqpoint{2.279412in}{2.004545in}}%
\pgfusepath{clip}%
\pgfsetbuttcap%
\pgfsetroundjoin%
\pgfsetlinewidth{0.474489pt}%
\definecolor{currentstroke}{rgb}{0.282623,0.140926,0.457517}%
\pgfsetstrokecolor{currentstroke}%
\pgfsetdash{}{0pt}%
\pgfpathmoveto{\pgfqpoint{2.655001in}{5.579509in}}%
\pgfpathlineto{\pgfqpoint{2.605354in}{5.573289in}}%
\pgfusepath{stroke}%
\end{pgfscope}%
\begin{pgfscope}%
\pgfpathrectangle{\pgfqpoint{1.250000in}{4.155455in}}{\pgfqpoint{2.279412in}{2.004545in}}%
\pgfusepath{clip}%
\pgfsetbuttcap%
\pgfsetroundjoin%
\pgfsetlinewidth{0.476344pt}%
\definecolor{currentstroke}{rgb}{0.282623,0.140926,0.457517}%
\pgfsetstrokecolor{currentstroke}%
\pgfsetdash{}{0pt}%
\pgfpathmoveto{\pgfqpoint{2.605354in}{5.573289in}}%
\pgfpathlineto{\pgfqpoint{2.555881in}{5.566084in}}%
\pgfusepath{stroke}%
\end{pgfscope}%
\begin{pgfscope}%
\pgfpathrectangle{\pgfqpoint{1.250000in}{4.155455in}}{\pgfqpoint{2.279412in}{2.004545in}}%
\pgfusepath{clip}%
\pgfsetbuttcap%
\pgfsetroundjoin%
\pgfsetlinewidth{0.517767pt}%
\definecolor{currentstroke}{rgb}{0.279574,0.170599,0.479997}%
\pgfsetstrokecolor{currentstroke}%
\pgfsetdash{}{0pt}%
\pgfpathmoveto{\pgfqpoint{2.555881in}{5.566084in}}%
\pgfpathlineto{\pgfqpoint{2.506604in}{5.557897in}}%
\pgfusepath{stroke}%
\end{pgfscope}%
\begin{pgfscope}%
\pgfpathrectangle{\pgfqpoint{1.250000in}{4.155455in}}{\pgfqpoint{2.279412in}{2.004545in}}%
\pgfusepath{clip}%
\pgfsetbuttcap%
\pgfsetroundjoin%
\pgfsetlinewidth{0.535662pt}%
\definecolor{currentstroke}{rgb}{0.277134,0.185228,0.489898}%
\pgfsetstrokecolor{currentstroke}%
\pgfsetdash{}{0pt}%
\pgfpathmoveto{\pgfqpoint{2.506604in}{5.557897in}}%
\pgfpathlineto{\pgfqpoint{2.457589in}{5.548592in}}%
\pgfusepath{stroke}%
\end{pgfscope}%
\begin{pgfscope}%
\pgfpathrectangle{\pgfqpoint{1.250000in}{4.155455in}}{\pgfqpoint{2.279412in}{2.004545in}}%
\pgfusepath{clip}%
\pgfsetbuttcap%
\pgfsetroundjoin%
\pgfsetlinewidth{0.531960pt}%
\definecolor{currentstroke}{rgb}{0.278012,0.180367,0.486697}%
\pgfsetstrokecolor{currentstroke}%
\pgfsetdash{}{0pt}%
\pgfpathmoveto{\pgfqpoint{2.457589in}{5.548592in}}%
\pgfpathlineto{\pgfqpoint{2.409139in}{5.537283in}}%
\pgfusepath{stroke}%
\end{pgfscope}%
\begin{pgfscope}%
\pgfpathrectangle{\pgfqpoint{1.250000in}{4.155455in}}{\pgfqpoint{2.279412in}{2.004545in}}%
\pgfusepath{clip}%
\pgfsetbuttcap%
\pgfsetroundjoin%
\pgfsetlinewidth{0.514896pt}%
\definecolor{currentstroke}{rgb}{0.279574,0.170599,0.479997}%
\pgfsetstrokecolor{currentstroke}%
\pgfsetdash{}{0pt}%
\pgfpathmoveto{\pgfqpoint{2.409139in}{5.537283in}}%
\pgfpathlineto{\pgfqpoint{2.361680in}{5.523154in}}%
\pgfusepath{stroke}%
\end{pgfscope}%
\begin{pgfscope}%
\pgfpathrectangle{\pgfqpoint{1.250000in}{4.155455in}}{\pgfqpoint{2.279412in}{2.004545in}}%
\pgfusepath{clip}%
\pgfsetbuttcap%
\pgfsetroundjoin%
\pgfsetlinewidth{0.550825pt}%
\definecolor{currentstroke}{rgb}{0.275191,0.194905,0.496005}%
\pgfsetstrokecolor{currentstroke}%
\pgfsetdash{}{0pt}%
\pgfpathmoveto{\pgfqpoint{2.361680in}{5.523154in}}%
\pgfpathlineto{\pgfqpoint{2.316023in}{5.505113in}}%
\pgfusepath{stroke}%
\end{pgfscope}%
\begin{pgfscope}%
\pgfpathrectangle{\pgfqpoint{1.250000in}{4.155455in}}{\pgfqpoint{2.279412in}{2.004545in}}%
\pgfusepath{clip}%
\pgfsetbuttcap%
\pgfsetroundjoin%
\pgfsetlinewidth{0.567040pt}%
\definecolor{currentstroke}{rgb}{0.273006,0.204520,0.501721}%
\pgfsetstrokecolor{currentstroke}%
\pgfsetdash{}{0pt}%
\pgfpathmoveto{\pgfqpoint{2.316023in}{5.505113in}}%
\pgfpathlineto{\pgfqpoint{2.273794in}{5.481625in}}%
\pgfusepath{stroke}%
\end{pgfscope}%
\begin{pgfscope}%
\pgfpathrectangle{\pgfqpoint{1.250000in}{4.155455in}}{\pgfqpoint{2.279412in}{2.004545in}}%
\pgfusepath{clip}%
\pgfsetbuttcap%
\pgfsetroundjoin%
\pgfsetlinewidth{0.573473pt}%
\definecolor{currentstroke}{rgb}{0.271828,0.209303,0.504434}%
\pgfsetstrokecolor{currentstroke}%
\pgfsetdash{}{0pt}%
\pgfpathmoveto{\pgfqpoint{2.273794in}{5.481625in}}%
\pgfpathlineto{\pgfqpoint{2.236712in}{5.452216in}}%
\pgfusepath{stroke}%
\end{pgfscope}%
\begin{pgfscope}%
\pgfpathrectangle{\pgfqpoint{1.250000in}{4.155455in}}{\pgfqpoint{2.279412in}{2.004545in}}%
\pgfusepath{clip}%
\pgfsetbuttcap%
\pgfsetroundjoin%
\pgfsetlinewidth{0.529007pt}%
\definecolor{currentstroke}{rgb}{0.278012,0.180367,0.486697}%
\pgfsetstrokecolor{currentstroke}%
\pgfsetdash{}{0pt}%
\pgfpathmoveto{\pgfqpoint{2.236712in}{5.452216in}}%
\pgfpathlineto{\pgfqpoint{2.236712in}{5.452216in}}%
\pgfusepath{stroke}%
\end{pgfscope}%
\begin{pgfscope}%
\pgfpathrectangle{\pgfqpoint{1.250000in}{4.155455in}}{\pgfqpoint{2.279412in}{2.004545in}}%
\pgfusepath{clip}%
\pgfsetbuttcap%
\pgfsetroundjoin%
\pgfsetlinewidth{0.529007pt}%
\definecolor{currentstroke}{rgb}{0.278012,0.180367,0.486697}%
\pgfsetstrokecolor{currentstroke}%
\pgfsetdash{}{0pt}%
\pgfpathmoveto{\pgfqpoint{2.236712in}{5.452216in}}%
\pgfpathlineto{\pgfqpoint{2.213781in}{5.426123in}}%
\pgfusepath{stroke}%
\end{pgfscope}%
\begin{pgfscope}%
\pgfpathrectangle{\pgfqpoint{1.250000in}{4.155455in}}{\pgfqpoint{2.279412in}{2.004545in}}%
\pgfusepath{clip}%
\pgfsetbuttcap%
\pgfsetroundjoin%
\pgfsetlinewidth{0.307390pt}%
\definecolor{currentstroke}{rgb}{0.267004,0.004874,0.329415}%
\pgfsetstrokecolor{currentstroke}%
\pgfsetdash{}{0pt}%
\pgfpathmoveto{\pgfqpoint{3.101036in}{5.666502in}}%
\pgfpathlineto{\pgfqpoint{3.053041in}{5.663899in}}%
\pgfusepath{stroke}%
\end{pgfscope}%
\begin{pgfscope}%
\pgfpathrectangle{\pgfqpoint{1.250000in}{4.155455in}}{\pgfqpoint{2.279412in}{2.004545in}}%
\pgfusepath{clip}%
\pgfsetbuttcap%
\pgfsetroundjoin%
\pgfsetlinewidth{0.319411pt}%
\definecolor{currentstroke}{rgb}{0.269944,0.014625,0.341379}%
\pgfsetstrokecolor{currentstroke}%
\pgfsetdash{}{0pt}%
\pgfpathmoveto{\pgfqpoint{3.053041in}{5.663899in}}%
\pgfpathlineto{\pgfqpoint{3.003482in}{5.660110in}}%
\pgfusepath{stroke}%
\end{pgfscope}%
\begin{pgfscope}%
\pgfpathrectangle{\pgfqpoint{1.250000in}{4.155455in}}{\pgfqpoint{2.279412in}{2.004545in}}%
\pgfusepath{clip}%
\pgfsetbuttcap%
\pgfsetroundjoin%
\pgfsetlinewidth{0.323742pt}%
\definecolor{currentstroke}{rgb}{0.271305,0.019942,0.347269}%
\pgfsetstrokecolor{currentstroke}%
\pgfsetdash{}{0pt}%
\pgfpathmoveto{\pgfqpoint{3.003482in}{5.660110in}}%
\pgfpathlineto{\pgfqpoint{2.953917in}{5.653902in}}%
\pgfusepath{stroke}%
\end{pgfscope}%
\begin{pgfscope}%
\pgfpathrectangle{\pgfqpoint{1.250000in}{4.155455in}}{\pgfqpoint{2.279412in}{2.004545in}}%
\pgfusepath{clip}%
\pgfsetbuttcap%
\pgfsetroundjoin%
\pgfsetlinewidth{0.327907pt}%
\definecolor{currentstroke}{rgb}{0.271305,0.019942,0.347269}%
\pgfsetstrokecolor{currentstroke}%
\pgfsetdash{}{0pt}%
\pgfpathmoveto{\pgfqpoint{2.953917in}{5.653902in}}%
\pgfpathlineto{\pgfqpoint{2.904337in}{5.647766in}}%
\pgfusepath{stroke}%
\end{pgfscope}%
\begin{pgfscope}%
\pgfpathrectangle{\pgfqpoint{1.250000in}{4.155455in}}{\pgfqpoint{2.279412in}{2.004545in}}%
\pgfusepath{clip}%
\pgfsetbuttcap%
\pgfsetroundjoin%
\pgfsetlinewidth{0.342704pt}%
\definecolor{currentstroke}{rgb}{0.274952,0.037752,0.364543}%
\pgfsetstrokecolor{currentstroke}%
\pgfsetdash{}{0pt}%
\pgfpathmoveto{\pgfqpoint{2.904337in}{5.647766in}}%
\pgfpathlineto{\pgfqpoint{2.854371in}{5.644010in}}%
\pgfusepath{stroke}%
\end{pgfscope}%
\begin{pgfscope}%
\pgfpathrectangle{\pgfqpoint{1.250000in}{4.155455in}}{\pgfqpoint{2.279412in}{2.004545in}}%
\pgfusepath{clip}%
\pgfsetbuttcap%
\pgfsetroundjoin%
\pgfsetlinewidth{0.353509pt}%
\definecolor{currentstroke}{rgb}{0.276022,0.044167,0.370164}%
\pgfsetstrokecolor{currentstroke}%
\pgfsetdash{}{0pt}%
\pgfpathmoveto{\pgfqpoint{2.854371in}{5.644010in}}%
\pgfpathlineto{\pgfqpoint{2.804491in}{5.639493in}}%
\pgfusepath{stroke}%
\end{pgfscope}%
\begin{pgfscope}%
\pgfpathrectangle{\pgfqpoint{1.250000in}{4.155455in}}{\pgfqpoint{2.279412in}{2.004545in}}%
\pgfusepath{clip}%
\pgfsetbuttcap%
\pgfsetroundjoin%
\pgfsetlinewidth{0.540455pt}%
\definecolor{currentstroke}{rgb}{0.277134,0.185228,0.489898}%
\pgfsetstrokecolor{currentstroke}%
\pgfsetdash{}{0pt}%
\pgfpathmoveto{\pgfqpoint{2.692807in}{4.825180in}}%
\pgfpathlineto{\pgfqpoint{2.642997in}{4.830313in}}%
\pgfusepath{stroke}%
\end{pgfscope}%
\begin{pgfscope}%
\pgfpathrectangle{\pgfqpoint{1.250000in}{4.155455in}}{\pgfqpoint{2.279412in}{2.004545in}}%
\pgfusepath{clip}%
\pgfsetbuttcap%
\pgfsetroundjoin%
\pgfsetlinewidth{0.569045pt}%
\definecolor{currentstroke}{rgb}{0.271828,0.209303,0.504434}%
\pgfsetstrokecolor{currentstroke}%
\pgfsetdash{}{0pt}%
\pgfpathmoveto{\pgfqpoint{2.642997in}{4.830313in}}%
\pgfpathlineto{\pgfqpoint{2.593241in}{4.835830in}}%
\pgfusepath{stroke}%
\end{pgfscope}%
\begin{pgfscope}%
\pgfpathrectangle{\pgfqpoint{1.250000in}{4.155455in}}{\pgfqpoint{2.279412in}{2.004545in}}%
\pgfusepath{clip}%
\pgfsetbuttcap%
\pgfsetroundjoin%
\pgfsetlinewidth{0.590181pt}%
\definecolor{currentstroke}{rgb}{0.267968,0.223549,0.512008}%
\pgfsetstrokecolor{currentstroke}%
\pgfsetdash{}{0pt}%
\pgfpathmoveto{\pgfqpoint{2.593241in}{4.835830in}}%
\pgfpathlineto{\pgfqpoint{2.543582in}{4.841980in}}%
\pgfusepath{stroke}%
\end{pgfscope}%
\begin{pgfscope}%
\pgfpathrectangle{\pgfqpoint{1.250000in}{4.155455in}}{\pgfqpoint{2.279412in}{2.004545in}}%
\pgfusepath{clip}%
\pgfsetbuttcap%
\pgfsetroundjoin%
\pgfsetlinewidth{0.644432pt}%
\definecolor{currentstroke}{rgb}{0.255645,0.260703,0.528312}%
\pgfsetstrokecolor{currentstroke}%
\pgfsetdash{}{0pt}%
\pgfpathmoveto{\pgfqpoint{2.543582in}{4.841980in}}%
\pgfpathlineto{\pgfqpoint{2.494079in}{4.849007in}}%
\pgfusepath{stroke}%
\end{pgfscope}%
\begin{pgfscope}%
\pgfpathrectangle{\pgfqpoint{1.250000in}{4.155455in}}{\pgfqpoint{2.279412in}{2.004545in}}%
\pgfusepath{clip}%
\pgfsetbuttcap%
\pgfsetroundjoin%
\pgfsetlinewidth{0.672187pt}%
\definecolor{currentstroke}{rgb}{0.248629,0.278775,0.534556}%
\pgfsetstrokecolor{currentstroke}%
\pgfsetdash{}{0pt}%
\pgfpathmoveto{\pgfqpoint{2.494079in}{4.849007in}}%
\pgfpathlineto{\pgfqpoint{2.444853in}{4.857400in}}%
\pgfusepath{stroke}%
\end{pgfscope}%
\begin{pgfscope}%
\pgfpathrectangle{\pgfqpoint{1.250000in}{4.155455in}}{\pgfqpoint{2.279412in}{2.004545in}}%
\pgfusepath{clip}%
\pgfsetbuttcap%
\pgfsetroundjoin%
\pgfsetlinewidth{0.736253pt}%
\definecolor{currentstroke}{rgb}{0.231674,0.318106,0.544834}%
\pgfsetstrokecolor{currentstroke}%
\pgfsetdash{}{0pt}%
\pgfpathmoveto{\pgfqpoint{2.030663in}{5.428368in}}%
\pgfpathlineto{\pgfqpoint{2.074598in}{5.407662in}}%
\pgfusepath{stroke}%
\end{pgfscope}%
\begin{pgfscope}%
\pgfpathrectangle{\pgfqpoint{1.250000in}{4.155455in}}{\pgfqpoint{2.279412in}{2.004545in}}%
\pgfusepath{clip}%
\pgfsetbuttcap%
\pgfsetroundjoin%
\pgfsetlinewidth{0.718333pt}%
\definecolor{currentstroke}{rgb}{0.237441,0.305202,0.541921}%
\pgfsetstrokecolor{currentstroke}%
\pgfsetdash{}{0pt}%
\pgfpathmoveto{\pgfqpoint{2.074598in}{5.407662in}}%
\pgfpathlineto{\pgfqpoint{2.074598in}{5.407662in}}%
\pgfusepath{stroke}%
\end{pgfscope}%
\begin{pgfscope}%
\pgfpathrectangle{\pgfqpoint{1.250000in}{4.155455in}}{\pgfqpoint{2.279412in}{2.004545in}}%
\pgfusepath{clip}%
\pgfsetbuttcap%
\pgfsetroundjoin%
\pgfsetlinewidth{0.718333pt}%
\definecolor{currentstroke}{rgb}{0.237441,0.305202,0.541921}%
\pgfsetstrokecolor{currentstroke}%
\pgfsetdash{}{0pt}%
\pgfpathmoveto{\pgfqpoint{2.074598in}{5.407662in}}%
\pgfpathlineto{\pgfqpoint{2.107334in}{5.385856in}}%
\pgfusepath{stroke}%
\end{pgfscope}%
\begin{pgfscope}%
\pgfpathrectangle{\pgfqpoint{1.250000in}{4.155455in}}{\pgfqpoint{2.279412in}{2.004545in}}%
\pgfusepath{clip}%
\pgfsetbuttcap%
\pgfsetroundjoin%
\pgfsetlinewidth{0.623968pt}%
\definecolor{currentstroke}{rgb}{0.260571,0.246922,0.522828}%
\pgfsetstrokecolor{currentstroke}%
\pgfsetdash{}{0pt}%
\pgfpathmoveto{\pgfqpoint{2.107334in}{5.385856in}}%
\pgfpathlineto{\pgfqpoint{2.107334in}{5.385856in}}%
\pgfusepath{stroke}%
\end{pgfscope}%
\begin{pgfscope}%
\pgfpathrectangle{\pgfqpoint{1.250000in}{4.155455in}}{\pgfqpoint{2.279412in}{2.004545in}}%
\pgfusepath{clip}%
\pgfsetbuttcap%
\pgfsetroundjoin%
\pgfsetlinewidth{0.623968pt}%
\definecolor{currentstroke}{rgb}{0.260571,0.246922,0.522828}%
\pgfsetstrokecolor{currentstroke}%
\pgfsetdash{}{0pt}%
\pgfpathmoveto{\pgfqpoint{2.107334in}{5.385856in}}%
\pgfpathlineto{\pgfqpoint{2.125461in}{5.367792in}}%
\pgfusepath{stroke}%
\end{pgfscope}%
\begin{pgfscope}%
\pgfpathrectangle{\pgfqpoint{1.250000in}{4.155455in}}{\pgfqpoint{2.279412in}{2.004545in}}%
\pgfusepath{clip}%
\pgfsetbuttcap%
\pgfsetroundjoin%
\pgfsetlinewidth{0.595593pt}%
\definecolor{currentstroke}{rgb}{0.267968,0.223549,0.512008}%
\pgfsetstrokecolor{currentstroke}%
\pgfsetdash{}{0pt}%
\pgfpathmoveto{\pgfqpoint{2.125461in}{5.367792in}}%
\pgfpathlineto{\pgfqpoint{2.140041in}{5.343796in}}%
\pgfusepath{stroke}%
\end{pgfscope}%
\begin{pgfscope}%
\pgfpathrectangle{\pgfqpoint{1.250000in}{4.155455in}}{\pgfqpoint{2.279412in}{2.004545in}}%
\pgfusepath{clip}%
\pgfsetbuttcap%
\pgfsetroundjoin%
\pgfsetlinewidth{0.640444pt}%
\definecolor{currentstroke}{rgb}{0.257322,0.256130,0.526563}%
\pgfsetstrokecolor{currentstroke}%
\pgfsetdash{}{0pt}%
\pgfpathmoveto{\pgfqpoint{2.140041in}{5.343796in}}%
\pgfpathlineto{\pgfqpoint{2.147872in}{5.316665in}}%
\pgfusepath{stroke}%
\end{pgfscope}%
\begin{pgfscope}%
\pgfpathrectangle{\pgfqpoint{1.250000in}{4.155455in}}{\pgfqpoint{2.279412in}{2.004545in}}%
\pgfusepath{clip}%
\pgfsetbuttcap%
\pgfsetroundjoin%
\pgfsetlinewidth{0.666141pt}%
\definecolor{currentstroke}{rgb}{0.250425,0.274290,0.533103}%
\pgfsetstrokecolor{currentstroke}%
\pgfsetdash{}{0pt}%
\pgfpathmoveto{\pgfqpoint{2.147872in}{5.316665in}}%
\pgfpathlineto{\pgfqpoint{2.148339in}{5.291142in}}%
\pgfusepath{stroke}%
\end{pgfscope}%
\begin{pgfscope}%
\pgfpathrectangle{\pgfqpoint{1.250000in}{4.155455in}}{\pgfqpoint{2.279412in}{2.004545in}}%
\pgfusepath{clip}%
\pgfsetbuttcap%
\pgfsetroundjoin%
\pgfsetlinewidth{0.568752pt}%
\definecolor{currentstroke}{rgb}{0.271828,0.209303,0.504434}%
\pgfsetstrokecolor{currentstroke}%
\pgfsetdash{}{0pt}%
\pgfpathmoveto{\pgfqpoint{2.148339in}{5.291142in}}%
\pgfpathlineto{\pgfqpoint{2.144148in}{5.251281in}}%
\pgfusepath{stroke}%
\end{pgfscope}%
\begin{pgfscope}%
\pgfpathrectangle{\pgfqpoint{1.250000in}{4.155455in}}{\pgfqpoint{2.279412in}{2.004545in}}%
\pgfusepath{clip}%
\pgfsetbuttcap%
\pgfsetroundjoin%
\pgfsetlinewidth{0.501284pt}%
\definecolor{currentstroke}{rgb}{0.280868,0.160771,0.472899}%
\pgfsetstrokecolor{currentstroke}%
\pgfsetdash{}{0pt}%
\pgfpathmoveto{\pgfqpoint{2.144148in}{5.251281in}}%
\pgfpathlineto{\pgfqpoint{2.141746in}{5.215466in}}%
\pgfusepath{stroke}%
\end{pgfscope}%
\begin{pgfscope}%
\pgfpathrectangle{\pgfqpoint{1.250000in}{4.155455in}}{\pgfqpoint{2.279412in}{2.004545in}}%
\pgfusepath{clip}%
\pgfsetbuttcap%
\pgfsetroundjoin%
\pgfsetlinewidth{0.467723pt}%
\definecolor{currentstroke}{rgb}{0.282884,0.135920,0.453427}%
\pgfsetstrokecolor{currentstroke}%
\pgfsetdash{}{0pt}%
\pgfpathmoveto{\pgfqpoint{2.141746in}{5.215466in}}%
\pgfpathlineto{\pgfqpoint{2.141746in}{5.215466in}}%
\pgfusepath{stroke}%
\end{pgfscope}%
\begin{pgfscope}%
\pgfpathrectangle{\pgfqpoint{1.250000in}{4.155455in}}{\pgfqpoint{2.279412in}{2.004545in}}%
\pgfusepath{clip}%
\pgfsetbuttcap%
\pgfsetroundjoin%
\pgfsetlinewidth{0.467723pt}%
\definecolor{currentstroke}{rgb}{0.282884,0.135920,0.453427}%
\pgfsetstrokecolor{currentstroke}%
\pgfsetdash{}{0pt}%
\pgfpathmoveto{\pgfqpoint{2.141746in}{5.215466in}}%
\pgfpathlineto{\pgfqpoint{2.142444in}{5.195952in}}%
\pgfusepath{stroke}%
\end{pgfscope}%
\begin{pgfscope}%
\pgfpathrectangle{\pgfqpoint{1.250000in}{4.155455in}}{\pgfqpoint{2.279412in}{2.004545in}}%
\pgfusepath{clip}%
\pgfsetbuttcap%
\pgfsetroundjoin%
\pgfsetlinewidth{0.389647pt}%
\definecolor{currentstroke}{rgb}{0.280267,0.073417,0.397163}%
\pgfsetstrokecolor{currentstroke}%
\pgfsetdash{}{0pt}%
\pgfpathmoveto{\pgfqpoint{2.142444in}{5.195952in}}%
\pgfpathlineto{\pgfqpoint{2.142444in}{5.195952in}}%
\pgfusepath{stroke}%
\end{pgfscope}%
\begin{pgfscope}%
\pgfpathrectangle{\pgfqpoint{1.250000in}{4.155455in}}{\pgfqpoint{2.279412in}{2.004545in}}%
\pgfusepath{clip}%
\pgfsetbuttcap%
\pgfsetroundjoin%
\pgfsetlinewidth{0.389647pt}%
\definecolor{currentstroke}{rgb}{0.280267,0.073417,0.397163}%
\pgfsetstrokecolor{currentstroke}%
\pgfsetdash{}{0pt}%
\pgfpathmoveto{\pgfqpoint{2.142444in}{5.195952in}}%
\pgfpathlineto{\pgfqpoint{2.143126in}{5.188047in}}%
\pgfusepath{stroke}%
\end{pgfscope}%
\begin{pgfscope}%
\pgfpathrectangle{\pgfqpoint{1.250000in}{4.155455in}}{\pgfqpoint{2.279412in}{2.004545in}}%
\pgfusepath{clip}%
\pgfsetbuttcap%
\pgfsetroundjoin%
\pgfsetlinewidth{0.868268pt}%
\definecolor{currentstroke}{rgb}{0.195860,0.395433,0.555276}%
\pgfsetstrokecolor{currentstroke}%
\pgfsetdash{}{0pt}%
\pgfpathmoveto{\pgfqpoint{1.983758in}{4.916790in}}%
\pgfpathlineto{\pgfqpoint{2.030663in}{4.932193in}}%
\pgfusepath{stroke}%
\end{pgfscope}%
\begin{pgfscope}%
\pgfpathrectangle{\pgfqpoint{1.250000in}{4.155455in}}{\pgfqpoint{2.279412in}{2.004545in}}%
\pgfusepath{clip}%
\pgfsetbuttcap%
\pgfsetroundjoin%
\pgfsetlinewidth{0.684650pt}%
\definecolor{currentstroke}{rgb}{0.246811,0.283237,0.535941}%
\pgfsetstrokecolor{currentstroke}%
\pgfsetdash{}{0pt}%
\pgfpathmoveto{\pgfqpoint{2.030663in}{4.932193in}}%
\pgfpathlineto{\pgfqpoint{2.075039in}{4.951878in}}%
\pgfusepath{stroke}%
\end{pgfscope}%
\begin{pgfscope}%
\pgfpathrectangle{\pgfqpoint{1.250000in}{4.155455in}}{\pgfqpoint{2.279412in}{2.004545in}}%
\pgfusepath{clip}%
\pgfsetbuttcap%
\pgfsetroundjoin%
\pgfsetlinewidth{0.734637pt}%
\definecolor{currentstroke}{rgb}{0.231674,0.318106,0.544834}%
\pgfsetstrokecolor{currentstroke}%
\pgfsetdash{}{0pt}%
\pgfpathmoveto{\pgfqpoint{2.075039in}{4.951878in}}%
\pgfpathlineto{\pgfqpoint{2.075039in}{4.951878in}}%
\pgfusepath{stroke}%
\end{pgfscope}%
\begin{pgfscope}%
\pgfpathrectangle{\pgfqpoint{1.250000in}{4.155455in}}{\pgfqpoint{2.279412in}{2.004545in}}%
\pgfusepath{clip}%
\pgfsetbuttcap%
\pgfsetroundjoin%
\pgfsetlinewidth{0.734637pt}%
\definecolor{currentstroke}{rgb}{0.231674,0.318106,0.544834}%
\pgfsetstrokecolor{currentstroke}%
\pgfsetdash{}{0pt}%
\pgfpathmoveto{\pgfqpoint{2.075039in}{4.951878in}}%
\pgfpathlineto{\pgfqpoint{2.102782in}{4.968454in}}%
\pgfusepath{stroke}%
\end{pgfscope}%
\begin{pgfscope}%
\pgfpathrectangle{\pgfqpoint{1.250000in}{4.155455in}}{\pgfqpoint{2.279412in}{2.004545in}}%
\pgfusepath{clip}%
\pgfsetbuttcap%
\pgfsetroundjoin%
\pgfsetlinewidth{0.589586pt}%
\definecolor{currentstroke}{rgb}{0.267968,0.223549,0.512008}%
\pgfsetstrokecolor{currentstroke}%
\pgfsetdash{}{0pt}%
\pgfpathmoveto{\pgfqpoint{2.102782in}{4.968454in}}%
\pgfpathlineto{\pgfqpoint{2.102782in}{4.968454in}}%
\pgfusepath{stroke}%
\end{pgfscope}%
\begin{pgfscope}%
\pgfpathrectangle{\pgfqpoint{1.250000in}{4.155455in}}{\pgfqpoint{2.279412in}{2.004545in}}%
\pgfusepath{clip}%
\pgfsetbuttcap%
\pgfsetroundjoin%
\pgfsetlinewidth{0.589586pt}%
\definecolor{currentstroke}{rgb}{0.267968,0.223549,0.512008}%
\pgfsetstrokecolor{currentstroke}%
\pgfsetdash{}{0pt}%
\pgfpathmoveto{\pgfqpoint{2.102782in}{4.968454in}}%
\pgfpathlineto{\pgfqpoint{2.102782in}{4.968454in}}%
\pgfusepath{stroke}%
\end{pgfscope}%
\begin{pgfscope}%
\pgfpathrectangle{\pgfqpoint{1.250000in}{4.155455in}}{\pgfqpoint{2.279412in}{2.004545in}}%
\pgfusepath{clip}%
\pgfsetbuttcap%
\pgfsetroundjoin%
\pgfsetlinewidth{0.589586pt}%
\definecolor{currentstroke}{rgb}{0.267968,0.223549,0.512008}%
\pgfsetstrokecolor{currentstroke}%
\pgfsetdash{}{0pt}%
\pgfpathmoveto{\pgfqpoint{2.102782in}{4.968454in}}%
\pgfpathlineto{\pgfqpoint{2.118714in}{4.985469in}}%
\pgfusepath{stroke}%
\end{pgfscope}%
\begin{pgfscope}%
\pgfpathrectangle{\pgfqpoint{1.250000in}{4.155455in}}{\pgfqpoint{2.279412in}{2.004545in}}%
\pgfusepath{clip}%
\pgfsetbuttcap%
\pgfsetroundjoin%
\pgfsetlinewidth{0.684853pt}%
\definecolor{currentstroke}{rgb}{0.246811,0.283237,0.535941}%
\pgfsetstrokecolor{currentstroke}%
\pgfsetdash{}{0pt}%
\pgfpathmoveto{\pgfqpoint{2.118714in}{4.985469in}}%
\pgfpathlineto{\pgfqpoint{2.130413in}{5.002863in}}%
\pgfusepath{stroke}%
\end{pgfscope}%
\begin{pgfscope}%
\pgfpathrectangle{\pgfqpoint{1.250000in}{4.155455in}}{\pgfqpoint{2.279412in}{2.004545in}}%
\pgfusepath{clip}%
\pgfsetbuttcap%
\pgfsetroundjoin%
\pgfsetlinewidth{0.602730pt}%
\definecolor{currentstroke}{rgb}{0.266580,0.228262,0.514349}%
\pgfsetstrokecolor{currentstroke}%
\pgfsetdash{}{0pt}%
\pgfpathmoveto{\pgfqpoint{2.130413in}{5.002863in}}%
\pgfpathlineto{\pgfqpoint{2.130413in}{5.002863in}}%
\pgfusepath{stroke}%
\end{pgfscope}%
\begin{pgfscope}%
\pgfpathrectangle{\pgfqpoint{1.250000in}{4.155455in}}{\pgfqpoint{2.279412in}{2.004545in}}%
\pgfusepath{clip}%
\pgfsetbuttcap%
\pgfsetroundjoin%
\pgfsetlinewidth{0.602730pt}%
\definecolor{currentstroke}{rgb}{0.266580,0.228262,0.514349}%
\pgfsetstrokecolor{currentstroke}%
\pgfsetdash{}{0pt}%
\pgfpathmoveto{\pgfqpoint{2.130413in}{5.002863in}}%
\pgfpathlineto{\pgfqpoint{2.140975in}{5.021723in}}%
\pgfusepath{stroke}%
\end{pgfscope}%
\begin{pgfscope}%
\pgfpathrectangle{\pgfqpoint{1.250000in}{4.155455in}}{\pgfqpoint{2.279412in}{2.004545in}}%
\pgfusepath{clip}%
\pgfsetbuttcap%
\pgfsetroundjoin%
\pgfsetlinewidth{0.593750pt}%
\definecolor{currentstroke}{rgb}{0.267968,0.223549,0.512008}%
\pgfsetstrokecolor{currentstroke}%
\pgfsetdash{}{0pt}%
\pgfpathmoveto{\pgfqpoint{2.140975in}{5.021723in}}%
\pgfpathlineto{\pgfqpoint{2.140975in}{5.021723in}}%
\pgfusepath{stroke}%
\end{pgfscope}%
\begin{pgfscope}%
\pgfpathrectangle{\pgfqpoint{1.250000in}{4.155455in}}{\pgfqpoint{2.279412in}{2.004545in}}%
\pgfusepath{clip}%
\pgfsetbuttcap%
\pgfsetroundjoin%
\pgfsetlinewidth{0.593750pt}%
\definecolor{currentstroke}{rgb}{0.267968,0.223549,0.512008}%
\pgfsetstrokecolor{currentstroke}%
\pgfsetdash{}{0pt}%
\pgfpathmoveto{\pgfqpoint{2.140975in}{5.021723in}}%
\pgfpathlineto{\pgfqpoint{2.144215in}{5.040934in}}%
\pgfusepath{stroke}%
\end{pgfscope}%
\begin{pgfscope}%
\pgfpathrectangle{\pgfqpoint{1.250000in}{4.155455in}}{\pgfqpoint{2.279412in}{2.004545in}}%
\pgfusepath{clip}%
\pgfsetbuttcap%
\pgfsetroundjoin%
\pgfsetlinewidth{0.684284pt}%
\definecolor{currentstroke}{rgb}{0.246811,0.283237,0.535941}%
\pgfsetstrokecolor{currentstroke}%
\pgfsetdash{}{0pt}%
\pgfpathmoveto{\pgfqpoint{2.644824in}{5.387186in}}%
\pgfpathlineto{\pgfqpoint{2.594873in}{5.383261in}}%
\pgfusepath{stroke}%
\end{pgfscope}%
\begin{pgfscope}%
\pgfpathrectangle{\pgfqpoint{1.250000in}{4.155455in}}{\pgfqpoint{2.279412in}{2.004545in}}%
\pgfusepath{clip}%
\pgfsetbuttcap%
\pgfsetroundjoin%
\pgfsetlinewidth{0.706442pt}%
\definecolor{currentstroke}{rgb}{0.239346,0.300855,0.540844}%
\pgfsetstrokecolor{currentstroke}%
\pgfsetdash{}{0pt}%
\pgfpathmoveto{\pgfqpoint{2.594873in}{5.383261in}}%
\pgfpathlineto{\pgfqpoint{2.544982in}{5.378797in}}%
\pgfusepath{stroke}%
\end{pgfscope}%
\begin{pgfscope}%
\pgfpathrectangle{\pgfqpoint{1.250000in}{4.155455in}}{\pgfqpoint{2.279412in}{2.004545in}}%
\pgfusepath{clip}%
\pgfsetbuttcap%
\pgfsetroundjoin%
\pgfsetlinewidth{0.746721pt}%
\definecolor{currentstroke}{rgb}{0.229739,0.322361,0.545706}%
\pgfsetstrokecolor{currentstroke}%
\pgfsetdash{}{0pt}%
\pgfpathmoveto{\pgfqpoint{2.544982in}{5.378797in}}%
\pgfpathlineto{\pgfqpoint{2.495150in}{5.373830in}}%
\pgfusepath{stroke}%
\end{pgfscope}%
\begin{pgfscope}%
\pgfpathrectangle{\pgfqpoint{1.250000in}{4.155455in}}{\pgfqpoint{2.279412in}{2.004545in}}%
\pgfusepath{clip}%
\pgfsetbuttcap%
\pgfsetroundjoin%
\pgfsetlinewidth{0.765026pt}%
\definecolor{currentstroke}{rgb}{0.223925,0.334994,0.548053}%
\pgfsetstrokecolor{currentstroke}%
\pgfsetdash{}{0pt}%
\pgfpathmoveto{\pgfqpoint{2.495150in}{5.373830in}}%
\pgfpathlineto{\pgfqpoint{2.445413in}{5.368199in}}%
\pgfusepath{stroke}%
\end{pgfscope}%
\begin{pgfscope}%
\pgfpathrectangle{\pgfqpoint{1.250000in}{4.155455in}}{\pgfqpoint{2.279412in}{2.004545in}}%
\pgfusepath{clip}%
\pgfsetbuttcap%
\pgfsetroundjoin%
\pgfsetlinewidth{0.773978pt}%
\definecolor{currentstroke}{rgb}{0.221989,0.339161,0.548752}%
\pgfsetstrokecolor{currentstroke}%
\pgfsetdash{}{0pt}%
\pgfpathmoveto{\pgfqpoint{2.445413in}{5.368199in}}%
\pgfpathlineto{\pgfqpoint{2.395882in}{5.361333in}}%
\pgfusepath{stroke}%
\end{pgfscope}%
\begin{pgfscope}%
\pgfpathrectangle{\pgfqpoint{1.250000in}{4.155455in}}{\pgfqpoint{2.279412in}{2.004545in}}%
\pgfusepath{clip}%
\pgfsetroundcap%
\pgfsetroundjoin%
\pgfsetlinewidth{0.772768pt}%
\definecolor{currentstroke}{rgb}{0.221989,0.339161,0.548752}%
\pgfsetstrokecolor{currentstroke}%
\pgfsetdash{}{0pt}%
\pgfpathmoveto{\pgfqpoint{2.660254in}{5.072045in}}%
\pgfpathquadraticcurveto{\pgfqpoint{2.647722in}{5.072377in}}{\pgfqpoint{2.647141in}{5.072392in}}%
\pgfusepath{stroke}%
\end{pgfscope}%
\begin{pgfscope}%
\pgfpathrectangle{\pgfqpoint{1.250000in}{4.155455in}}{\pgfqpoint{2.279412in}{2.004545in}}%
\pgfusepath{clip}%
\pgfsetroundcap%
\pgfsetroundjoin%
\definecolor{currentfill}{rgb}{0.221989,0.339161,0.548752}%
\pgfsetfillcolor{currentfill}%
\pgfsetlinewidth{0.772768pt}%
\definecolor{currentstroke}{rgb}{0.221989,0.339161,0.548752}%
\pgfsetstrokecolor{currentstroke}%
\pgfsetdash{}{0pt}%
\pgfpathmoveto{\pgfqpoint{2.701942in}{5.043154in}}%
\pgfpathlineto{\pgfqpoint{2.647141in}{5.072392in}}%
\pgfpathlineto{\pgfqpoint{2.703412in}{5.098690in}}%
\pgfpathlineto{\pgfqpoint{2.701942in}{5.043154in}}%
\pgfpathlineto{\pgfqpoint{2.701942in}{5.043154in}}%
\pgfpathclose%
\pgfusepath{stroke,fill}%
\end{pgfscope}%
\begin{pgfscope}%
\pgfpathrectangle{\pgfqpoint{1.250000in}{4.155455in}}{\pgfqpoint{2.279412in}{2.004545in}}%
\pgfusepath{clip}%
\pgfsetroundcap%
\pgfsetroundjoin%
\pgfsetlinewidth{0.657848pt}%
\definecolor{currentstroke}{rgb}{0.253935,0.265254,0.529983}%
\pgfsetstrokecolor{currentstroke}%
\pgfsetdash{}{0pt}%
\pgfpathmoveto{\pgfqpoint{2.760260in}{5.157111in}}%
\pgfpathquadraticcurveto{\pgfqpoint{2.747722in}{5.157088in}}{\pgfqpoint{2.745361in}{5.157084in}}%
\pgfusepath{stroke}%
\end{pgfscope}%
\begin{pgfscope}%
\pgfpathrectangle{\pgfqpoint{1.250000in}{4.155455in}}{\pgfqpoint{2.279412in}{2.004545in}}%
\pgfusepath{clip}%
\pgfsetroundcap%
\pgfsetroundjoin%
\definecolor{currentfill}{rgb}{0.253935,0.265254,0.529983}%
\pgfsetfillcolor{currentfill}%
\pgfsetlinewidth{0.657848pt}%
\definecolor{currentstroke}{rgb}{0.253935,0.265254,0.529983}%
\pgfsetstrokecolor{currentstroke}%
\pgfsetdash{}{0pt}%
\pgfpathmoveto{\pgfqpoint{2.800967in}{5.129407in}}%
\pgfpathlineto{\pgfqpoint{2.745361in}{5.157084in}}%
\pgfpathlineto{\pgfqpoint{2.800866in}{5.184963in}}%
\pgfpathlineto{\pgfqpoint{2.800967in}{5.129407in}}%
\pgfpathlineto{\pgfqpoint{2.800967in}{5.129407in}}%
\pgfpathclose%
\pgfusepath{stroke,fill}%
\end{pgfscope}%
\begin{pgfscope}%
\pgfpathrectangle{\pgfqpoint{1.250000in}{4.155455in}}{\pgfqpoint{2.279412in}{2.004545in}}%
\pgfusepath{clip}%
\pgfsetroundcap%
\pgfsetroundjoin%
\pgfsetlinewidth{0.675813pt}%
\definecolor{currentstroke}{rgb}{0.248629,0.278775,0.534556}%
\pgfsetstrokecolor{currentstroke}%
\pgfsetdash{}{0pt}%
\pgfpathmoveto{\pgfqpoint{2.760698in}{5.204200in}}%
\pgfpathquadraticcurveto{\pgfqpoint{2.748161in}{5.204039in}}{\pgfqpoint{2.746079in}{5.204012in}}%
\pgfusepath{stroke}%
\end{pgfscope}%
\begin{pgfscope}%
\pgfpathrectangle{\pgfqpoint{1.250000in}{4.155455in}}{\pgfqpoint{2.279412in}{2.004545in}}%
\pgfusepath{clip}%
\pgfsetroundcap%
\pgfsetroundjoin%
\definecolor{currentfill}{rgb}{0.248629,0.278775,0.534556}%
\pgfsetfillcolor{currentfill}%
\pgfsetlinewidth{0.675813pt}%
\definecolor{currentstroke}{rgb}{0.248629,0.278775,0.534556}%
\pgfsetstrokecolor{currentstroke}%
\pgfsetdash{}{0pt}%
\pgfpathmoveto{\pgfqpoint{2.801986in}{5.176949in}}%
\pgfpathlineto{\pgfqpoint{2.746079in}{5.204012in}}%
\pgfpathlineto{\pgfqpoint{2.801274in}{5.232500in}}%
\pgfpathlineto{\pgfqpoint{2.801986in}{5.176949in}}%
\pgfpathlineto{\pgfqpoint{2.801986in}{5.176949in}}%
\pgfpathclose%
\pgfusepath{stroke,fill}%
\end{pgfscope}%
\begin{pgfscope}%
\pgfpathrectangle{\pgfqpoint{1.250000in}{4.155455in}}{\pgfqpoint{2.279412in}{2.004545in}}%
\pgfusepath{clip}%
\pgfsetroundcap%
\pgfsetroundjoin%
\pgfsetlinewidth{0.602633pt}%
\definecolor{currentstroke}{rgb}{0.266580,0.228262,0.514349}%
\pgfsetstrokecolor{currentstroke}%
\pgfsetdash{}{0pt}%
\pgfpathmoveto{\pgfqpoint{2.763028in}{5.281668in}}%
\pgfpathquadraticcurveto{\pgfqpoint{2.750498in}{5.281275in}}{\pgfqpoint{2.747287in}{5.281174in}}%
\pgfusepath{stroke}%
\end{pgfscope}%
\begin{pgfscope}%
\pgfpathrectangle{\pgfqpoint{1.250000in}{4.155455in}}{\pgfqpoint{2.279412in}{2.004545in}}%
\pgfusepath{clip}%
\pgfsetroundcap%
\pgfsetroundjoin%
\definecolor{currentfill}{rgb}{0.266580,0.228262,0.514349}%
\pgfsetfillcolor{currentfill}%
\pgfsetlinewidth{0.602633pt}%
\definecolor{currentstroke}{rgb}{0.266580,0.228262,0.514349}%
\pgfsetstrokecolor{currentstroke}%
\pgfsetdash{}{0pt}%
\pgfpathmoveto{\pgfqpoint{2.803685in}{5.255151in}}%
\pgfpathlineto{\pgfqpoint{2.747287in}{5.281174in}}%
\pgfpathlineto{\pgfqpoint{2.801944in}{5.310680in}}%
\pgfpathlineto{\pgfqpoint{2.803685in}{5.255151in}}%
\pgfpathlineto{\pgfqpoint{2.803685in}{5.255151in}}%
\pgfpathclose%
\pgfusepath{stroke,fill}%
\end{pgfscope}%
\begin{pgfscope}%
\pgfpathrectangle{\pgfqpoint{1.250000in}{4.155455in}}{\pgfqpoint{2.279412in}{2.004545in}}%
\pgfusepath{clip}%
\pgfsetroundcap%
\pgfsetroundjoin%
\pgfsetlinewidth{0.366837pt}%
\definecolor{currentstroke}{rgb}{0.277941,0.056324,0.381191}%
\pgfsetstrokecolor{currentstroke}%
\pgfsetdash{}{0pt}%
\pgfpathmoveto{\pgfqpoint{3.027764in}{5.328542in}}%
\pgfpathquadraticcurveto{\pgfqpoint{3.015231in}{5.328238in}}{\pgfqpoint{3.008371in}{5.328071in}}%
\pgfusepath{stroke}%
\end{pgfscope}%
\begin{pgfscope}%
\pgfpathrectangle{\pgfqpoint{1.250000in}{4.155455in}}{\pgfqpoint{2.279412in}{2.004545in}}%
\pgfusepath{clip}%
\pgfsetroundcap%
\pgfsetroundjoin%
\definecolor{currentfill}{rgb}{0.277941,0.056324,0.381191}%
\pgfsetfillcolor{currentfill}%
\pgfsetlinewidth{0.366837pt}%
\definecolor{currentstroke}{rgb}{0.277941,0.056324,0.381191}%
\pgfsetstrokecolor{currentstroke}%
\pgfsetdash{}{0pt}%
\pgfpathmoveto{\pgfqpoint{3.064584in}{5.301648in}}%
\pgfpathlineto{\pgfqpoint{3.008371in}{5.328071in}}%
\pgfpathlineto{\pgfqpoint{3.063237in}{5.357188in}}%
\pgfpathlineto{\pgfqpoint{3.064584in}{5.301648in}}%
\pgfpathlineto{\pgfqpoint{3.064584in}{5.301648in}}%
\pgfpathclose%
\pgfusepath{stroke,fill}%
\end{pgfscope}%
\begin{pgfscope}%
\pgfpathrectangle{\pgfqpoint{1.250000in}{4.155455in}}{\pgfqpoint{2.279412in}{2.004545in}}%
\pgfusepath{clip}%
\pgfsetroundcap%
\pgfsetroundjoin%
\pgfsetlinewidth{0.488952pt}%
\definecolor{currentstroke}{rgb}{0.281887,0.150881,0.465405}%
\pgfsetstrokecolor{currentstroke}%
\pgfsetdash{}{0pt}%
\pgfpathmoveto{\pgfqpoint{2.859428in}{5.028649in}}%
\pgfpathquadraticcurveto{\pgfqpoint{2.846896in}{5.028975in}}{\pgfqpoint{2.841925in}{5.029105in}}%
\pgfusepath{stroke}%
\end{pgfscope}%
\begin{pgfscope}%
\pgfpathrectangle{\pgfqpoint{1.250000in}{4.155455in}}{\pgfqpoint{2.279412in}{2.004545in}}%
\pgfusepath{clip}%
\pgfsetroundcap%
\pgfsetroundjoin%
\definecolor{currentfill}{rgb}{0.281887,0.150881,0.465405}%
\pgfsetfillcolor{currentfill}%
\pgfsetlinewidth{0.488952pt}%
\definecolor{currentstroke}{rgb}{0.281887,0.150881,0.465405}%
\pgfsetstrokecolor{currentstroke}%
\pgfsetdash{}{0pt}%
\pgfpathmoveto{\pgfqpoint{2.896740in}{4.999892in}}%
\pgfpathlineto{\pgfqpoint{2.841925in}{5.029105in}}%
\pgfpathlineto{\pgfqpoint{2.898184in}{5.055429in}}%
\pgfpathlineto{\pgfqpoint{2.896740in}{4.999892in}}%
\pgfpathlineto{\pgfqpoint{2.896740in}{4.999892in}}%
\pgfpathclose%
\pgfusepath{stroke,fill}%
\end{pgfscope}%
\begin{pgfscope}%
\pgfpathrectangle{\pgfqpoint{1.250000in}{4.155455in}}{\pgfqpoint{2.279412in}{2.004545in}}%
\pgfusepath{clip}%
\pgfsetroundcap%
\pgfsetroundjoin%
\pgfsetlinewidth{0.509446pt}%
\definecolor{currentstroke}{rgb}{0.280255,0.165693,0.476498}%
\pgfsetstrokecolor{currentstroke}%
\pgfsetdash{}{0pt}%
\pgfpathmoveto{\pgfqpoint{2.859666in}{5.244118in}}%
\pgfpathquadraticcurveto{\pgfqpoint{2.847131in}{5.243857in}}{\pgfqpoint{2.842477in}{5.243760in}}%
\pgfusepath{stroke}%
\end{pgfscope}%
\begin{pgfscope}%
\pgfpathrectangle{\pgfqpoint{1.250000in}{4.155455in}}{\pgfqpoint{2.279412in}{2.004545in}}%
\pgfusepath{clip}%
\pgfsetroundcap%
\pgfsetroundjoin%
\definecolor{currentfill}{rgb}{0.280255,0.165693,0.476498}%
\pgfsetfillcolor{currentfill}%
\pgfsetlinewidth{0.509446pt}%
\definecolor{currentstroke}{rgb}{0.280255,0.165693,0.476498}%
\pgfsetstrokecolor{currentstroke}%
\pgfsetdash{}{0pt}%
\pgfpathmoveto{\pgfqpoint{2.898597in}{5.217143in}}%
\pgfpathlineto{\pgfqpoint{2.842477in}{5.243760in}}%
\pgfpathlineto{\pgfqpoint{2.897443in}{5.272687in}}%
\pgfpathlineto{\pgfqpoint{2.898597in}{5.217143in}}%
\pgfpathlineto{\pgfqpoint{2.898597in}{5.217143in}}%
\pgfpathclose%
\pgfusepath{stroke,fill}%
\end{pgfscope}%
\begin{pgfscope}%
\pgfpathrectangle{\pgfqpoint{1.250000in}{4.155455in}}{\pgfqpoint{2.279412in}{2.004545in}}%
\pgfusepath{clip}%
\pgfsetroundcap%
\pgfsetroundjoin%
\pgfsetlinewidth{0.676486pt}%
\definecolor{currentstroke}{rgb}{0.248629,0.278775,0.534556}%
\pgfsetstrokecolor{currentstroke}%
\pgfsetdash{}{0pt}%
\pgfpathmoveto{\pgfqpoint{2.659546in}{5.359075in}}%
\pgfpathquadraticcurveto{\pgfqpoint{2.647045in}{5.358230in}}{\pgfqpoint{2.644986in}{5.358090in}}%
\pgfusepath{stroke}%
\end{pgfscope}%
\begin{pgfscope}%
\pgfpathrectangle{\pgfqpoint{1.250000in}{4.155455in}}{\pgfqpoint{2.279412in}{2.004545in}}%
\pgfusepath{clip}%
\pgfsetroundcap%
\pgfsetroundjoin%
\definecolor{currentfill}{rgb}{0.248629,0.278775,0.534556}%
\pgfsetfillcolor{currentfill}%
\pgfsetlinewidth{0.676486pt}%
\definecolor{currentstroke}{rgb}{0.248629,0.278775,0.534556}%
\pgfsetstrokecolor{currentstroke}%
\pgfsetdash{}{0pt}%
\pgfpathmoveto{\pgfqpoint{2.702288in}{5.334123in}}%
\pgfpathlineto{\pgfqpoint{2.644986in}{5.358090in}}%
\pgfpathlineto{\pgfqpoint{2.698541in}{5.389552in}}%
\pgfpathlineto{\pgfqpoint{2.702288in}{5.334123in}}%
\pgfpathlineto{\pgfqpoint{2.702288in}{5.334123in}}%
\pgfpathclose%
\pgfusepath{stroke,fill}%
\end{pgfscope}%
\begin{pgfscope}%
\pgfpathrectangle{\pgfqpoint{1.250000in}{4.155455in}}{\pgfqpoint{2.279412in}{2.004545in}}%
\pgfusepath{clip}%
\pgfsetroundcap%
\pgfsetroundjoin%
\pgfsetlinewidth{0.631326pt}%
\definecolor{currentstroke}{rgb}{0.258965,0.251537,0.524736}%
\pgfsetstrokecolor{currentstroke}%
\pgfsetdash{}{0pt}%
\pgfpathmoveto{\pgfqpoint{2.412881in}{4.824586in}}%
\pgfpathquadraticcurveto{\pgfqpoint{2.400851in}{4.827678in}}{\pgfqpoint{2.398280in}{4.828339in}}%
\pgfusepath{stroke}%
\end{pgfscope}%
\begin{pgfscope}%
\pgfpathrectangle{\pgfqpoint{1.250000in}{4.155455in}}{\pgfqpoint{2.279412in}{2.004545in}}%
\pgfusepath{clip}%
\pgfsetroundcap%
\pgfsetroundjoin%
\definecolor{currentfill}{rgb}{0.258965,0.251537,0.524736}%
\pgfsetfillcolor{currentfill}%
\pgfsetlinewidth{0.631326pt}%
\definecolor{currentstroke}{rgb}{0.258965,0.251537,0.524736}%
\pgfsetstrokecolor{currentstroke}%
\pgfsetdash{}{0pt}%
\pgfpathmoveto{\pgfqpoint{2.445173in}{4.787607in}}%
\pgfpathlineto{\pgfqpoint{2.398280in}{4.828339in}}%
\pgfpathlineto{\pgfqpoint{2.459001in}{4.841415in}}%
\pgfpathlineto{\pgfqpoint{2.445173in}{4.787607in}}%
\pgfpathlineto{\pgfqpoint{2.445173in}{4.787607in}}%
\pgfpathclose%
\pgfusepath{stroke,fill}%
\end{pgfscope}%
\begin{pgfscope}%
\pgfpathrectangle{\pgfqpoint{1.250000in}{4.155455in}}{\pgfqpoint{2.279412in}{2.004545in}}%
\pgfusepath{clip}%
\pgfsetroundcap%
\pgfsetroundjoin%
\pgfsetlinewidth{0.548635pt}%
\definecolor{currentstroke}{rgb}{0.275191,0.194905,0.496005}%
\pgfsetstrokecolor{currentstroke}%
\pgfsetdash{}{0pt}%
\pgfpathmoveto{\pgfqpoint{2.708720in}{4.866409in}}%
\pgfpathquadraticcurveto{\pgfqpoint{2.696237in}{4.867442in}}{\pgfqpoint{2.692213in}{4.867775in}}%
\pgfusepath{stroke}%
\end{pgfscope}%
\begin{pgfscope}%
\pgfpathrectangle{\pgfqpoint{1.250000in}{4.155455in}}{\pgfqpoint{2.279412in}{2.004545in}}%
\pgfusepath{clip}%
\pgfsetroundcap%
\pgfsetroundjoin%
\definecolor{currentfill}{rgb}{0.275191,0.194905,0.496005}%
\pgfsetfillcolor{currentfill}%
\pgfsetlinewidth{0.548635pt}%
\definecolor{currentstroke}{rgb}{0.275191,0.194905,0.496005}%
\pgfsetstrokecolor{currentstroke}%
\pgfsetdash{}{0pt}%
\pgfpathmoveto{\pgfqpoint{2.745289in}{4.835510in}}%
\pgfpathlineto{\pgfqpoint{2.692213in}{4.867775in}}%
\pgfpathlineto{\pgfqpoint{2.749870in}{4.890876in}}%
\pgfpathlineto{\pgfqpoint{2.745289in}{4.835510in}}%
\pgfpathlineto{\pgfqpoint{2.745289in}{4.835510in}}%
\pgfpathclose%
\pgfusepath{stroke,fill}%
\end{pgfscope}%
\begin{pgfscope}%
\pgfpathrectangle{\pgfqpoint{1.250000in}{4.155455in}}{\pgfqpoint{2.279412in}{2.004545in}}%
\pgfusepath{clip}%
\pgfsetroundcap%
\pgfsetroundjoin%
\pgfsetlinewidth{0.390832pt}%
\definecolor{currentstroke}{rgb}{0.280894,0.078907,0.402329}%
\pgfsetstrokecolor{currentstroke}%
\pgfsetdash{}{0pt}%
\pgfpathmoveto{\pgfqpoint{2.958677in}{4.894420in}}%
\pgfpathquadraticcurveto{\pgfqpoint{2.946156in}{4.894980in}}{\pgfqpoint{2.939676in}{4.895270in}}%
\pgfusepath{stroke}%
\end{pgfscope}%
\begin{pgfscope}%
\pgfpathrectangle{\pgfqpoint{1.250000in}{4.155455in}}{\pgfqpoint{2.279412in}{2.004545in}}%
\pgfusepath{clip}%
\pgfsetroundcap%
\pgfsetroundjoin%
\definecolor{currentfill}{rgb}{0.280894,0.078907,0.402329}%
\pgfsetfillcolor{currentfill}%
\pgfsetlinewidth{0.390832pt}%
\definecolor{currentstroke}{rgb}{0.280894,0.078907,0.402329}%
\pgfsetstrokecolor{currentstroke}%
\pgfsetdash{}{0pt}%
\pgfpathmoveto{\pgfqpoint{2.993933in}{4.865035in}}%
\pgfpathlineto{\pgfqpoint{2.939676in}{4.895270in}}%
\pgfpathlineto{\pgfqpoint{2.996418in}{4.920535in}}%
\pgfpathlineto{\pgfqpoint{2.993933in}{4.865035in}}%
\pgfpathlineto{\pgfqpoint{2.993933in}{4.865035in}}%
\pgfpathclose%
\pgfusepath{stroke,fill}%
\end{pgfscope}%
\begin{pgfscope}%
\pgfpathrectangle{\pgfqpoint{1.250000in}{4.155455in}}{\pgfqpoint{2.279412in}{2.004545in}}%
\pgfusepath{clip}%
\pgfsetroundcap%
\pgfsetroundjoin%
\pgfsetlinewidth{0.629383pt}%
\definecolor{currentstroke}{rgb}{0.260571,0.246922,0.522828}%
\pgfsetstrokecolor{currentstroke}%
\pgfsetdash{}{0pt}%
\pgfpathmoveto{\pgfqpoint{2.708678in}{4.951013in}}%
\pgfpathquadraticcurveto{\pgfqpoint{2.696166in}{4.951719in}}{\pgfqpoint{2.693375in}{4.951876in}}%
\pgfusepath{stroke}%
\end{pgfscope}%
\begin{pgfscope}%
\pgfpathrectangle{\pgfqpoint{1.250000in}{4.155455in}}{\pgfqpoint{2.279412in}{2.004545in}}%
\pgfusepath{clip}%
\pgfsetroundcap%
\pgfsetroundjoin%
\definecolor{currentfill}{rgb}{0.260571,0.246922,0.522828}%
\pgfsetfillcolor{currentfill}%
\pgfsetlinewidth{0.629383pt}%
\definecolor{currentstroke}{rgb}{0.260571,0.246922,0.522828}%
\pgfsetstrokecolor{currentstroke}%
\pgfsetdash{}{0pt}%
\pgfpathmoveto{\pgfqpoint{2.747278in}{4.921014in}}%
\pgfpathlineto{\pgfqpoint{2.693375in}{4.951876in}}%
\pgfpathlineto{\pgfqpoint{2.750407in}{4.976481in}}%
\pgfpathlineto{\pgfqpoint{2.747278in}{4.921014in}}%
\pgfpathlineto{\pgfqpoint{2.747278in}{4.921014in}}%
\pgfpathclose%
\pgfusepath{stroke,fill}%
\end{pgfscope}%
\begin{pgfscope}%
\pgfpathrectangle{\pgfqpoint{1.250000in}{4.155455in}}{\pgfqpoint{2.279412in}{2.004545in}}%
\pgfusepath{clip}%
\pgfsetroundcap%
\pgfsetroundjoin%
\pgfsetlinewidth{0.434830pt}%
\definecolor{currentstroke}{rgb}{0.283091,0.110553,0.431554}%
\pgfsetstrokecolor{currentstroke}%
\pgfsetdash{}{0pt}%
\pgfpathmoveto{\pgfqpoint{2.908533in}{4.984916in}}%
\pgfpathquadraticcurveto{\pgfqpoint{2.896010in}{4.985454in}}{\pgfqpoint{2.890208in}{4.985703in}}%
\pgfusepath{stroke}%
\end{pgfscope}%
\begin{pgfscope}%
\pgfpathrectangle{\pgfqpoint{1.250000in}{4.155455in}}{\pgfqpoint{2.279412in}{2.004545in}}%
\pgfusepath{clip}%
\pgfsetroundcap%
\pgfsetroundjoin%
\definecolor{currentfill}{rgb}{0.283091,0.110553,0.431554}%
\pgfsetfillcolor{currentfill}%
\pgfsetlinewidth{0.434830pt}%
\definecolor{currentstroke}{rgb}{0.283091,0.110553,0.431554}%
\pgfsetstrokecolor{currentstroke}%
\pgfsetdash{}{0pt}%
\pgfpathmoveto{\pgfqpoint{2.944521in}{4.955569in}}%
\pgfpathlineto{\pgfqpoint{2.890208in}{4.985703in}}%
\pgfpathlineto{\pgfqpoint{2.946903in}{5.011073in}}%
\pgfpathlineto{\pgfqpoint{2.944521in}{4.955569in}}%
\pgfpathlineto{\pgfqpoint{2.944521in}{4.955569in}}%
\pgfpathclose%
\pgfusepath{stroke,fill}%
\end{pgfscope}%
\begin{pgfscope}%
\pgfpathrectangle{\pgfqpoint{1.250000in}{4.155455in}}{\pgfqpoint{2.279412in}{2.004545in}}%
\pgfusepath{clip}%
\pgfsetroundcap%
\pgfsetroundjoin%
\pgfsetlinewidth{0.662941pt}%
\definecolor{currentstroke}{rgb}{0.252194,0.269783,0.531579}%
\pgfsetstrokecolor{currentstroke}%
\pgfsetdash{}{0pt}%
\pgfpathmoveto{\pgfqpoint{2.757892in}{5.114974in}}%
\pgfpathquadraticcurveto{\pgfqpoint{2.745354in}{5.115073in}}{\pgfqpoint{2.743072in}{5.115091in}}%
\pgfusepath{stroke}%
\end{pgfscope}%
\begin{pgfscope}%
\pgfpathrectangle{\pgfqpoint{1.250000in}{4.155455in}}{\pgfqpoint{2.279412in}{2.004545in}}%
\pgfusepath{clip}%
\pgfsetroundcap%
\pgfsetroundjoin%
\definecolor{currentfill}{rgb}{0.252194,0.269783,0.531579}%
\pgfsetfillcolor{currentfill}%
\pgfsetlinewidth{0.662941pt}%
\definecolor{currentstroke}{rgb}{0.252194,0.269783,0.531579}%
\pgfsetstrokecolor{currentstroke}%
\pgfsetdash{}{0pt}%
\pgfpathmoveto{\pgfqpoint{2.798406in}{5.086874in}}%
\pgfpathlineto{\pgfqpoint{2.743072in}{5.115091in}}%
\pgfpathlineto{\pgfqpoint{2.798846in}{5.142427in}}%
\pgfpathlineto{\pgfqpoint{2.798406in}{5.086874in}}%
\pgfpathlineto{\pgfqpoint{2.798406in}{5.086874in}}%
\pgfpathclose%
\pgfusepath{stroke,fill}%
\end{pgfscope}%
\begin{pgfscope}%
\pgfpathrectangle{\pgfqpoint{1.250000in}{4.155455in}}{\pgfqpoint{2.279412in}{2.004545in}}%
\pgfusepath{clip}%
\pgfsetroundcap%
\pgfsetroundjoin%
\pgfsetlinewidth{0.365128pt}%
\definecolor{currentstroke}{rgb}{0.277941,0.056324,0.381191}%
\pgfsetstrokecolor{currentstroke}%
\pgfsetdash{}{0pt}%
\pgfpathmoveto{\pgfqpoint{3.009112in}{5.418861in}}%
\pgfpathquadraticcurveto{\pgfqpoint{2.996590in}{5.418320in}}{\pgfqpoint{2.989711in}{5.418023in}}%
\pgfusepath{stroke}%
\end{pgfscope}%
\begin{pgfscope}%
\pgfpathrectangle{\pgfqpoint{1.250000in}{4.155455in}}{\pgfqpoint{2.279412in}{2.004545in}}%
\pgfusepath{clip}%
\pgfsetroundcap%
\pgfsetroundjoin%
\definecolor{currentfill}{rgb}{0.277941,0.056324,0.381191}%
\pgfsetfillcolor{currentfill}%
\pgfsetlinewidth{0.365128pt}%
\definecolor{currentstroke}{rgb}{0.277941,0.056324,0.381191}%
\pgfsetstrokecolor{currentstroke}%
\pgfsetdash{}{0pt}%
\pgfpathmoveto{\pgfqpoint{3.046414in}{5.392669in}}%
\pgfpathlineto{\pgfqpoint{2.989711in}{5.418023in}}%
\pgfpathlineto{\pgfqpoint{3.044016in}{5.448173in}}%
\pgfpathlineto{\pgfqpoint{3.046414in}{5.392669in}}%
\pgfpathlineto{\pgfqpoint{3.046414in}{5.392669in}}%
\pgfpathclose%
\pgfusepath{stroke,fill}%
\end{pgfscope}%
\begin{pgfscope}%
\pgfpathrectangle{\pgfqpoint{1.250000in}{4.155455in}}{\pgfqpoint{2.279412in}{2.004545in}}%
\pgfusepath{clip}%
\pgfsetroundcap%
\pgfsetroundjoin%
\pgfsetlinewidth{0.599244pt}%
\definecolor{currentstroke}{rgb}{0.266580,0.228262,0.514349}%
\pgfsetstrokecolor{currentstroke}%
\pgfsetdash{}{0pt}%
\pgfpathmoveto{\pgfqpoint{2.659078in}{5.441920in}}%
\pgfpathquadraticcurveto{\pgfqpoint{2.646616in}{5.440711in}}{\pgfqpoint{2.643381in}{5.440397in}}%
\pgfusepath{stroke}%
\end{pgfscope}%
\begin{pgfscope}%
\pgfpathrectangle{\pgfqpoint{1.250000in}{4.155455in}}{\pgfqpoint{2.279412in}{2.004545in}}%
\pgfusepath{clip}%
\pgfsetroundcap%
\pgfsetroundjoin%
\definecolor{currentfill}{rgb}{0.266580,0.228262,0.514349}%
\pgfsetfillcolor{currentfill}%
\pgfsetlinewidth{0.599244pt}%
\definecolor{currentstroke}{rgb}{0.266580,0.228262,0.514349}%
\pgfsetstrokecolor{currentstroke}%
\pgfsetdash{}{0pt}%
\pgfpathmoveto{\pgfqpoint{2.701360in}{5.418115in}}%
\pgfpathlineto{\pgfqpoint{2.643381in}{5.440397in}}%
\pgfpathlineto{\pgfqpoint{2.695994in}{5.473411in}}%
\pgfpathlineto{\pgfqpoint{2.701360in}{5.418115in}}%
\pgfpathlineto{\pgfqpoint{2.701360in}{5.418115in}}%
\pgfpathclose%
\pgfusepath{stroke,fill}%
\end{pgfscope}%
\begin{pgfscope}%
\pgfpathrectangle{\pgfqpoint{1.250000in}{4.155455in}}{\pgfqpoint{2.279412in}{2.004545in}}%
\pgfusepath{clip}%
\pgfsetroundcap%
\pgfsetroundjoin%
\pgfsetlinewidth{0.356068pt}%
\definecolor{currentstroke}{rgb}{0.277018,0.050344,0.375715}%
\pgfsetstrokecolor{currentstroke}%
\pgfsetdash{}{0pt}%
\pgfpathmoveto{\pgfqpoint{2.957630in}{4.802755in}}%
\pgfpathquadraticcurveto{\pgfqpoint{2.945117in}{4.803423in}}{\pgfqpoint{2.938104in}{4.803798in}}%
\pgfusepath{stroke}%
\end{pgfscope}%
\begin{pgfscope}%
\pgfpathrectangle{\pgfqpoint{1.250000in}{4.155455in}}{\pgfqpoint{2.279412in}{2.004545in}}%
\pgfusepath{clip}%
\pgfsetroundcap%
\pgfsetroundjoin%
\definecolor{currentfill}{rgb}{0.277018,0.050344,0.375715}%
\pgfsetfillcolor{currentfill}%
\pgfsetlinewidth{0.356068pt}%
\definecolor{currentstroke}{rgb}{0.277018,0.050344,0.375715}%
\pgfsetstrokecolor{currentstroke}%
\pgfsetdash{}{0pt}%
\pgfpathmoveto{\pgfqpoint{2.992098in}{4.773095in}}%
\pgfpathlineto{\pgfqpoint{2.938104in}{4.803798in}}%
\pgfpathlineto{\pgfqpoint{2.995063in}{4.828571in}}%
\pgfpathlineto{\pgfqpoint{2.992098in}{4.773095in}}%
\pgfpathlineto{\pgfqpoint{2.992098in}{4.773095in}}%
\pgfpathclose%
\pgfusepath{stroke,fill}%
\end{pgfscope}%
\begin{pgfscope}%
\pgfpathrectangle{\pgfqpoint{1.250000in}{4.155455in}}{\pgfqpoint{2.279412in}{2.004545in}}%
\pgfusepath{clip}%
\pgfsetroundcap%
\pgfsetroundjoin%
\pgfsetlinewidth{0.355939pt}%
\definecolor{currentstroke}{rgb}{0.276022,0.044167,0.370164}%
\pgfsetstrokecolor{currentstroke}%
\pgfsetdash{}{0pt}%
\pgfpathmoveto{\pgfqpoint{2.697457in}{4.571339in}}%
\pgfpathquadraticcurveto{\pgfqpoint{2.685061in}{4.572976in}}{\pgfqpoint{2.678124in}{4.573892in}}%
\pgfusepath{stroke}%
\end{pgfscope}%
\begin{pgfscope}%
\pgfpathrectangle{\pgfqpoint{1.250000in}{4.155455in}}{\pgfqpoint{2.279412in}{2.004545in}}%
\pgfusepath{clip}%
\pgfsetroundcap%
\pgfsetroundjoin%
\definecolor{currentfill}{rgb}{0.276022,0.044167,0.370164}%
\pgfsetfillcolor{currentfill}%
\pgfsetlinewidth{0.355939pt}%
\definecolor{currentstroke}{rgb}{0.276022,0.044167,0.370164}%
\pgfsetstrokecolor{currentstroke}%
\pgfsetdash{}{0pt}%
\pgfpathmoveto{\pgfqpoint{2.729564in}{4.539080in}}%
\pgfpathlineto{\pgfqpoint{2.678124in}{4.573892in}}%
\pgfpathlineto{\pgfqpoint{2.736838in}{4.594157in}}%
\pgfpathlineto{\pgfqpoint{2.729564in}{4.539080in}}%
\pgfpathlineto{\pgfqpoint{2.729564in}{4.539080in}}%
\pgfpathclose%
\pgfusepath{stroke,fill}%
\end{pgfscope}%
\begin{pgfscope}%
\pgfpathrectangle{\pgfqpoint{1.250000in}{4.155455in}}{\pgfqpoint{2.279412in}{2.004545in}}%
\pgfusepath{clip}%
\pgfsetroundcap%
\pgfsetroundjoin%
\pgfsetlinewidth{0.472823pt}%
\definecolor{currentstroke}{rgb}{0.282623,0.140926,0.457517}%
\pgfsetstrokecolor{currentstroke}%
\pgfsetdash{}{0pt}%
\pgfpathmoveto{\pgfqpoint{2.614795in}{4.714535in}}%
\pgfpathquadraticcurveto{\pgfqpoint{2.602464in}{4.716518in}}{\pgfqpoint{2.597354in}{4.717340in}}%
\pgfusepath{stroke}%
\end{pgfscope}%
\begin{pgfscope}%
\pgfpathrectangle{\pgfqpoint{1.250000in}{4.155455in}}{\pgfqpoint{2.279412in}{2.004545in}}%
\pgfusepath{clip}%
\pgfsetroundcap%
\pgfsetroundjoin%
\definecolor{currentfill}{rgb}{0.282623,0.140926,0.457517}%
\pgfsetfillcolor{currentfill}%
\pgfsetlinewidth{0.472823pt}%
\definecolor{currentstroke}{rgb}{0.282623,0.140926,0.457517}%
\pgfsetstrokecolor{currentstroke}%
\pgfsetdash{}{0pt}%
\pgfpathmoveto{\pgfqpoint{2.647795in}{4.681094in}}%
\pgfpathlineto{\pgfqpoint{2.597354in}{4.717340in}}%
\pgfpathlineto{\pgfqpoint{2.656615in}{4.735945in}}%
\pgfpathlineto{\pgfqpoint{2.647795in}{4.681094in}}%
\pgfpathlineto{\pgfqpoint{2.647795in}{4.681094in}}%
\pgfpathclose%
\pgfusepath{stroke,fill}%
\end{pgfscope}%
\begin{pgfscope}%
\pgfpathrectangle{\pgfqpoint{1.250000in}{4.155455in}}{\pgfqpoint{2.279412in}{2.004545in}}%
\pgfusepath{clip}%
\pgfsetroundcap%
\pgfsetroundjoin%
\pgfsetlinewidth{0.577294pt}%
\definecolor{currentstroke}{rgb}{0.270595,0.214069,0.507052}%
\pgfsetstrokecolor{currentstroke}%
\pgfsetdash{}{0pt}%
\pgfpathmoveto{\pgfqpoint{2.607144in}{5.482209in}}%
\pgfpathquadraticcurveto{\pgfqpoint{2.594727in}{5.480678in}}{\pgfqpoint{2.591175in}{5.480240in}}%
\pgfusepath{stroke}%
\end{pgfscope}%
\begin{pgfscope}%
\pgfpathrectangle{\pgfqpoint{1.250000in}{4.155455in}}{\pgfqpoint{2.279412in}{2.004545in}}%
\pgfusepath{clip}%
\pgfsetroundcap%
\pgfsetroundjoin%
\definecolor{currentfill}{rgb}{0.270595,0.214069,0.507052}%
\pgfsetfillcolor{currentfill}%
\pgfsetlinewidth{0.577294pt}%
\definecolor{currentstroke}{rgb}{0.270595,0.214069,0.507052}%
\pgfsetstrokecolor{currentstroke}%
\pgfsetdash{}{0pt}%
\pgfpathmoveto{\pgfqpoint{2.649713in}{5.459472in}}%
\pgfpathlineto{\pgfqpoint{2.591175in}{5.480240in}}%
\pgfpathlineto{\pgfqpoint{2.642912in}{5.514610in}}%
\pgfpathlineto{\pgfqpoint{2.649713in}{5.459472in}}%
\pgfpathlineto{\pgfqpoint{2.649713in}{5.459472in}}%
\pgfpathclose%
\pgfusepath{stroke,fill}%
\end{pgfscope}%
\begin{pgfscope}%
\pgfpathrectangle{\pgfqpoint{1.250000in}{4.155455in}}{\pgfqpoint{2.279412in}{2.004545in}}%
\pgfusepath{clip}%
\pgfsetroundcap%
\pgfsetroundjoin%
\pgfsetlinewidth{0.443362pt}%
\definecolor{currentstroke}{rgb}{0.283197,0.115680,0.436115}%
\pgfsetstrokecolor{currentstroke}%
\pgfsetdash{}{0pt}%
\pgfpathmoveto{\pgfqpoint{2.756687in}{5.543183in}}%
\pgfpathquadraticcurveto{\pgfqpoint{2.744232in}{5.541930in}}{\pgfqpoint{2.738601in}{5.541363in}}%
\pgfusepath{stroke}%
\end{pgfscope}%
\begin{pgfscope}%
\pgfpathrectangle{\pgfqpoint{1.250000in}{4.155455in}}{\pgfqpoint{2.279412in}{2.004545in}}%
\pgfusepath{clip}%
\pgfsetroundcap%
\pgfsetroundjoin%
\definecolor{currentfill}{rgb}{0.283197,0.115680,0.436115}%
\pgfsetfillcolor{currentfill}%
\pgfsetlinewidth{0.443362pt}%
\definecolor{currentstroke}{rgb}{0.283197,0.115680,0.436115}%
\pgfsetstrokecolor{currentstroke}%
\pgfsetdash{}{0pt}%
\pgfpathmoveto{\pgfqpoint{2.796658in}{5.519286in}}%
\pgfpathlineto{\pgfqpoint{2.738601in}{5.541363in}}%
\pgfpathlineto{\pgfqpoint{2.791097in}{5.574562in}}%
\pgfpathlineto{\pgfqpoint{2.796658in}{5.519286in}}%
\pgfpathlineto{\pgfqpoint{2.796658in}{5.519286in}}%
\pgfpathclose%
\pgfusepath{stroke,fill}%
\end{pgfscope}%
\begin{pgfscope}%
\pgfpathrectangle{\pgfqpoint{1.250000in}{4.155455in}}{\pgfqpoint{2.279412in}{2.004545in}}%
\pgfusepath{clip}%
\pgfsetroundcap%
\pgfsetroundjoin%
\pgfsetlinewidth{0.463586pt}%
\definecolor{currentstroke}{rgb}{0.283072,0.130895,0.449241}%
\pgfsetstrokecolor{currentstroke}%
\pgfsetdash{}{0pt}%
\pgfpathmoveto{\pgfqpoint{2.318392in}{5.632632in}}%
\pgfpathquadraticcurveto{\pgfqpoint{2.308630in}{5.625855in}}{\pgfqpoint{2.304759in}{5.623167in}}%
\pgfusepath{stroke}%
\end{pgfscope}%
\begin{pgfscope}%
\pgfpathrectangle{\pgfqpoint{1.250000in}{4.155455in}}{\pgfqpoint{2.279412in}{2.004545in}}%
\pgfusepath{clip}%
\pgfsetroundcap%
\pgfsetroundjoin%
\definecolor{currentfill}{rgb}{0.283072,0.130895,0.449241}%
\pgfsetfillcolor{currentfill}%
\pgfsetlinewidth{0.463586pt}%
\definecolor{currentstroke}{rgb}{0.283072,0.130895,0.449241}%
\pgfsetstrokecolor{currentstroke}%
\pgfsetdash{}{0pt}%
\pgfpathmoveto{\pgfqpoint{2.366236in}{5.632032in}}%
\pgfpathlineto{\pgfqpoint{2.304759in}{5.623167in}}%
\pgfpathlineto{\pgfqpoint{2.334553in}{5.677668in}}%
\pgfpathlineto{\pgfqpoint{2.366236in}{5.632032in}}%
\pgfpathlineto{\pgfqpoint{2.366236in}{5.632032in}}%
\pgfpathclose%
\pgfusepath{stroke,fill}%
\end{pgfscope}%
\begin{pgfscope}%
\pgfpathrectangle{\pgfqpoint{1.250000in}{4.155455in}}{\pgfqpoint{2.279412in}{2.004545in}}%
\pgfusepath{clip}%
\pgfsetroundcap%
\pgfsetroundjoin%
\pgfsetlinewidth{0.474515pt}%
\definecolor{currentstroke}{rgb}{0.282623,0.140926,0.457517}%
\pgfsetstrokecolor{currentstroke}%
\pgfsetdash{}{0pt}%
\pgfpathmoveto{\pgfqpoint{2.244865in}{4.734108in}}%
\pgfpathquadraticcurveto{\pgfqpoint{2.241427in}{4.740703in}}{\pgfqpoint{2.241383in}{4.740788in}}%
\pgfusepath{stroke}%
\end{pgfscope}%
\begin{pgfscope}%
\pgfpathrectangle{\pgfqpoint{1.250000in}{4.155455in}}{\pgfqpoint{2.279412in}{2.004545in}}%
\pgfusepath{clip}%
\pgfsetroundcap%
\pgfsetroundjoin%
\definecolor{currentfill}{rgb}{0.282623,0.140926,0.457517}%
\pgfsetfillcolor{currentfill}%
\pgfsetlinewidth{0.474515pt}%
\definecolor{currentstroke}{rgb}{0.282623,0.140926,0.457517}%
\pgfsetstrokecolor{currentstroke}%
\pgfsetdash{}{0pt}%
\pgfpathmoveto{\pgfqpoint{2.242438in}{4.678684in}}%
\pgfpathlineto{\pgfqpoint{2.241383in}{4.740788in}}%
\pgfpathlineto{\pgfqpoint{2.291699in}{4.704370in}}%
\pgfpathlineto{\pgfqpoint{2.242438in}{4.678684in}}%
\pgfpathlineto{\pgfqpoint{2.242438in}{4.678684in}}%
\pgfpathclose%
\pgfusepath{stroke,fill}%
\end{pgfscope}%
\begin{pgfscope}%
\pgfpathrectangle{\pgfqpoint{1.250000in}{4.155455in}}{\pgfqpoint{2.279412in}{2.004545in}}%
\pgfusepath{clip}%
\pgfsetroundcap%
\pgfsetroundjoin%
\pgfsetlinewidth{0.451495pt}%
\definecolor{currentstroke}{rgb}{0.283229,0.120777,0.440584}%
\pgfsetstrokecolor{currentstroke}%
\pgfsetdash{}{0pt}%
\pgfpathmoveto{\pgfqpoint{2.499072in}{4.643819in}}%
\pgfpathquadraticcurveto{\pgfqpoint{2.487092in}{4.647049in}}{\pgfqpoint{2.481857in}{4.648461in}}%
\pgfusepath{stroke}%
\end{pgfscope}%
\begin{pgfscope}%
\pgfpathrectangle{\pgfqpoint{1.250000in}{4.155455in}}{\pgfqpoint{2.279412in}{2.004545in}}%
\pgfusepath{clip}%
\pgfsetroundcap%
\pgfsetroundjoin%
\definecolor{currentfill}{rgb}{0.283229,0.120777,0.440584}%
\pgfsetfillcolor{currentfill}%
\pgfsetlinewidth{0.451495pt}%
\definecolor{currentstroke}{rgb}{0.283229,0.120777,0.440584}%
\pgfsetstrokecolor{currentstroke}%
\pgfsetdash{}{0pt}%
\pgfpathmoveto{\pgfqpoint{2.528264in}{4.607176in}}%
\pgfpathlineto{\pgfqpoint{2.481857in}{4.648461in}}%
\pgfpathlineto{\pgfqpoint{2.542729in}{4.660816in}}%
\pgfpathlineto{\pgfqpoint{2.528264in}{4.607176in}}%
\pgfpathlineto{\pgfqpoint{2.528264in}{4.607176in}}%
\pgfpathclose%
\pgfusepath{stroke,fill}%
\end{pgfscope}%
\begin{pgfscope}%
\pgfpathrectangle{\pgfqpoint{1.250000in}{4.155455in}}{\pgfqpoint{2.279412in}{2.004545in}}%
\pgfusepath{clip}%
\pgfsetroundcap%
\pgfsetroundjoin%
\pgfsetlinewidth{0.385369pt}%
\definecolor{currentstroke}{rgb}{0.280267,0.073417,0.397163}%
\pgfsetstrokecolor{currentstroke}%
\pgfsetdash{}{0pt}%
\pgfpathmoveto{\pgfqpoint{2.607466in}{5.654204in}}%
\pgfpathquadraticcurveto{\pgfqpoint{2.595115in}{5.652330in}}{\pgfqpoint{2.588659in}{5.651350in}}%
\pgfusepath{stroke}%
\end{pgfscope}%
\begin{pgfscope}%
\pgfpathrectangle{\pgfqpoint{1.250000in}{4.155455in}}{\pgfqpoint{2.279412in}{2.004545in}}%
\pgfusepath{clip}%
\pgfsetroundcap%
\pgfsetroundjoin%
\definecolor{currentfill}{rgb}{0.280267,0.073417,0.397163}%
\pgfsetfillcolor{currentfill}%
\pgfsetlinewidth{0.385369pt}%
\definecolor{currentstroke}{rgb}{0.280267,0.073417,0.397163}%
\pgfsetstrokecolor{currentstroke}%
\pgfsetdash{}{0pt}%
\pgfpathmoveto{\pgfqpoint{2.647753in}{5.632223in}}%
\pgfpathlineto{\pgfqpoint{2.588659in}{5.651350in}}%
\pgfpathlineto{\pgfqpoint{2.639417in}{5.687149in}}%
\pgfpathlineto{\pgfqpoint{2.647753in}{5.632223in}}%
\pgfpathlineto{\pgfqpoint{2.647753in}{5.632223in}}%
\pgfpathclose%
\pgfusepath{stroke,fill}%
\end{pgfscope}%
\begin{pgfscope}%
\pgfpathrectangle{\pgfqpoint{1.250000in}{4.155455in}}{\pgfqpoint{2.279412in}{2.004545in}}%
\pgfusepath{clip}%
\pgfsetroundcap%
\pgfsetroundjoin%
\pgfsetlinewidth{0.370426pt}%
\definecolor{currentstroke}{rgb}{0.278791,0.062145,0.386592}%
\pgfsetstrokecolor{currentstroke}%
\pgfsetdash{}{0pt}%
\pgfpathmoveto{\pgfqpoint{2.743216in}{4.639833in}}%
\pgfpathquadraticcurveto{\pgfqpoint{2.730808in}{4.641404in}}{\pgfqpoint{2.724085in}{4.642255in}}%
\pgfusepath{stroke}%
\end{pgfscope}%
\begin{pgfscope}%
\pgfpathrectangle{\pgfqpoint{1.250000in}{4.155455in}}{\pgfqpoint{2.279412in}{2.004545in}}%
\pgfusepath{clip}%
\pgfsetroundcap%
\pgfsetroundjoin%
\definecolor{currentfill}{rgb}{0.278791,0.062145,0.386592}%
\pgfsetfillcolor{currentfill}%
\pgfsetlinewidth{0.370426pt}%
\definecolor{currentstroke}{rgb}{0.278791,0.062145,0.386592}%
\pgfsetstrokecolor{currentstroke}%
\pgfsetdash{}{0pt}%
\pgfpathmoveto{\pgfqpoint{2.775711in}{4.607718in}}%
\pgfpathlineto{\pgfqpoint{2.724085in}{4.642255in}}%
\pgfpathlineto{\pgfqpoint{2.782690in}{4.662834in}}%
\pgfpathlineto{\pgfqpoint{2.775711in}{4.607718in}}%
\pgfpathlineto{\pgfqpoint{2.775711in}{4.607718in}}%
\pgfpathclose%
\pgfusepath{stroke,fill}%
\end{pgfscope}%
\begin{pgfscope}%
\pgfpathrectangle{\pgfqpoint{1.250000in}{4.155455in}}{\pgfqpoint{2.279412in}{2.004545in}}%
\pgfusepath{clip}%
\pgfsetroundcap%
\pgfsetroundjoin%
\pgfsetlinewidth{0.517767pt}%
\definecolor{currentstroke}{rgb}{0.279574,0.170599,0.479997}%
\pgfsetstrokecolor{currentstroke}%
\pgfsetdash{}{0pt}%
\pgfpathmoveto{\pgfqpoint{2.555881in}{5.566084in}}%
\pgfpathquadraticcurveto{\pgfqpoint{2.543562in}{5.564037in}}{\pgfqpoint{2.539144in}{5.563303in}}%
\pgfusepath{stroke}%
\end{pgfscope}%
\begin{pgfscope}%
\pgfpathrectangle{\pgfqpoint{1.250000in}{4.155455in}}{\pgfqpoint{2.279412in}{2.004545in}}%
\pgfusepath{clip}%
\pgfsetroundcap%
\pgfsetroundjoin%
\definecolor{currentfill}{rgb}{0.279574,0.170599,0.479997}%
\pgfsetfillcolor{currentfill}%
\pgfsetlinewidth{0.517767pt}%
\definecolor{currentstroke}{rgb}{0.279574,0.170599,0.479997}%
\pgfsetstrokecolor{currentstroke}%
\pgfsetdash{}{0pt}%
\pgfpathmoveto{\pgfqpoint{2.598501in}{5.545006in}}%
\pgfpathlineto{\pgfqpoint{2.539144in}{5.563303in}}%
\pgfpathlineto{\pgfqpoint{2.589396in}{5.599810in}}%
\pgfpathlineto{\pgfqpoint{2.598501in}{5.545006in}}%
\pgfpathlineto{\pgfqpoint{2.598501in}{5.545006in}}%
\pgfpathclose%
\pgfusepath{stroke,fill}%
\end{pgfscope}%
\begin{pgfscope}%
\pgfpathrectangle{\pgfqpoint{1.250000in}{4.155455in}}{\pgfqpoint{2.279412in}{2.004545in}}%
\pgfusepath{clip}%
\pgfsetroundcap%
\pgfsetroundjoin%
\pgfsetlinewidth{0.327907pt}%
\definecolor{currentstroke}{rgb}{0.271305,0.019942,0.347269}%
\pgfsetstrokecolor{currentstroke}%
\pgfsetdash{}{0pt}%
\pgfpathmoveto{\pgfqpoint{2.953917in}{5.653902in}}%
\pgfpathquadraticcurveto{\pgfqpoint{2.941522in}{5.652368in}}{\pgfqpoint{2.934161in}{5.651457in}}%
\pgfusepath{stroke}%
\end{pgfscope}%
\begin{pgfscope}%
\pgfpathrectangle{\pgfqpoint{1.250000in}{4.155455in}}{\pgfqpoint{2.279412in}{2.004545in}}%
\pgfusepath{clip}%
\pgfsetroundcap%
\pgfsetroundjoin%
\definecolor{currentfill}{rgb}{0.271305,0.019942,0.347269}%
\pgfsetfillcolor{currentfill}%
\pgfsetlinewidth{0.327907pt}%
\definecolor{currentstroke}{rgb}{0.271305,0.019942,0.347269}%
\pgfsetstrokecolor{currentstroke}%
\pgfsetdash{}{0pt}%
\pgfpathmoveto{\pgfqpoint{2.992708in}{5.630713in}}%
\pgfpathlineto{\pgfqpoint{2.934161in}{5.651457in}}%
\pgfpathlineto{\pgfqpoint{2.985885in}{5.685848in}}%
\pgfpathlineto{\pgfqpoint{2.992708in}{5.630713in}}%
\pgfpathlineto{\pgfqpoint{2.992708in}{5.630713in}}%
\pgfpathclose%
\pgfusepath{stroke,fill}%
\end{pgfscope}%
\begin{pgfscope}%
\pgfpathrectangle{\pgfqpoint{1.250000in}{4.155455in}}{\pgfqpoint{2.279412in}{2.004545in}}%
\pgfusepath{clip}%
\pgfsetroundcap%
\pgfsetroundjoin%
\pgfsetlinewidth{0.590181pt}%
\definecolor{currentstroke}{rgb}{0.267968,0.223549,0.512008}%
\pgfsetstrokecolor{currentstroke}%
\pgfsetdash{}{0pt}%
\pgfpathmoveto{\pgfqpoint{2.593241in}{4.835830in}}%
\pgfpathquadraticcurveto{\pgfqpoint{2.580826in}{4.837367in}}{\pgfqpoint{2.577473in}{4.837783in}}%
\pgfusepath{stroke}%
\end{pgfscope}%
\begin{pgfscope}%
\pgfpathrectangle{\pgfqpoint{1.250000in}{4.155455in}}{\pgfqpoint{2.279412in}{2.004545in}}%
\pgfusepath{clip}%
\pgfsetroundcap%
\pgfsetroundjoin%
\definecolor{currentfill}{rgb}{0.267968,0.223549,0.512008}%
\pgfsetfillcolor{currentfill}%
\pgfsetlinewidth{0.590181pt}%
\definecolor{currentstroke}{rgb}{0.267968,0.223549,0.512008}%
\pgfsetstrokecolor{currentstroke}%
\pgfsetdash{}{0pt}%
\pgfpathmoveto{\pgfqpoint{2.629193in}{4.803387in}}%
\pgfpathlineto{\pgfqpoint{2.577473in}{4.837783in}}%
\pgfpathlineto{\pgfqpoint{2.636021in}{4.858522in}}%
\pgfpathlineto{\pgfqpoint{2.629193in}{4.803387in}}%
\pgfpathlineto{\pgfqpoint{2.629193in}{4.803387in}}%
\pgfpathclose%
\pgfusepath{stroke,fill}%
\end{pgfscope}%
\begin{pgfscope}%
\pgfpathrectangle{\pgfqpoint{1.250000in}{4.155455in}}{\pgfqpoint{2.279412in}{2.004545in}}%
\pgfusepath{clip}%
\pgfsetroundcap%
\pgfsetroundjoin%
\pgfsetlinewidth{0.666141pt}%
\definecolor{currentstroke}{rgb}{0.250425,0.274290,0.533103}%
\pgfsetstrokecolor{currentstroke}%
\pgfsetdash{}{0pt}%
\pgfpathmoveto{\pgfqpoint{2.147872in}{5.316665in}}%
\pgfpathquadraticcurveto{\pgfqpoint{2.147989in}{5.310284in}}{\pgfqpoint{2.147917in}{5.314207in}}%
\pgfusepath{stroke}%
\end{pgfscope}%
\begin{pgfscope}%
\pgfpathrectangle{\pgfqpoint{1.250000in}{4.155455in}}{\pgfqpoint{2.279412in}{2.004545in}}%
\pgfusepath{clip}%
\pgfsetroundcap%
\pgfsetroundjoin%
\definecolor{currentfill}{rgb}{0.250425,0.274290,0.533103}%
\pgfsetfillcolor{currentfill}%
\pgfsetlinewidth{0.666141pt}%
\definecolor{currentstroke}{rgb}{0.250425,0.274290,0.533103}%
\pgfsetstrokecolor{currentstroke}%
\pgfsetdash{}{0pt}%
\pgfpathmoveto{\pgfqpoint{2.174673in}{5.370262in}}%
\pgfpathlineto{\pgfqpoint{2.147917in}{5.314207in}}%
\pgfpathlineto{\pgfqpoint{2.119127in}{5.369245in}}%
\pgfpathlineto{\pgfqpoint{2.174673in}{5.370262in}}%
\pgfpathlineto{\pgfqpoint{2.174673in}{5.370262in}}%
\pgfpathclose%
\pgfusepath{stroke,fill}%
\end{pgfscope}%
\begin{pgfscope}%
\pgfpathrectangle{\pgfqpoint{1.250000in}{4.155455in}}{\pgfqpoint{2.279412in}{2.004545in}}%
\pgfusepath{clip}%
\pgfsetroundcap%
\pgfsetroundjoin%
\pgfsetlinewidth{0.589586pt}%
\definecolor{currentstroke}{rgb}{0.267968,0.223549,0.512008}%
\pgfsetstrokecolor{currentstroke}%
\pgfsetdash{}{0pt}%
\pgfpathmoveto{\pgfqpoint{2.102782in}{4.968454in}}%
\pgfpathquadraticcurveto{\pgfqpoint{2.106765in}{4.972708in}}{\pgfqpoint{2.104514in}{4.970303in}}%
\pgfusepath{stroke}%
\end{pgfscope}%
\begin{pgfscope}%
\pgfpathrectangle{\pgfqpoint{1.250000in}{4.155455in}}{\pgfqpoint{2.279412in}{2.004545in}}%
\pgfusepath{clip}%
\pgfsetroundcap%
\pgfsetroundjoin%
\definecolor{currentfill}{rgb}{0.267968,0.223549,0.512008}%
\pgfsetfillcolor{currentfill}%
\pgfsetlinewidth{0.589586pt}%
\definecolor{currentstroke}{rgb}{0.267968,0.223549,0.512008}%
\pgfsetstrokecolor{currentstroke}%
\pgfsetdash{}{0pt}%
\pgfpathmoveto{\pgfqpoint{2.046266in}{4.948735in}}%
\pgfpathlineto{\pgfqpoint{2.104514in}{4.970303in}}%
\pgfpathlineto{\pgfqpoint{2.086820in}{4.910764in}}%
\pgfpathlineto{\pgfqpoint{2.046266in}{4.948735in}}%
\pgfpathlineto{\pgfqpoint{2.046266in}{4.948735in}}%
\pgfpathclose%
\pgfusepath{stroke,fill}%
\end{pgfscope}%
\begin{pgfscope}%
\pgfpathrectangle{\pgfqpoint{1.250000in}{4.155455in}}{\pgfqpoint{2.279412in}{2.004545in}}%
\pgfusepath{clip}%
\pgfsetroundcap%
\pgfsetroundjoin%
\pgfsetlinewidth{0.746721pt}%
\definecolor{currentstroke}{rgb}{0.229739,0.322361,0.545706}%
\pgfsetstrokecolor{currentstroke}%
\pgfsetdash{}{0pt}%
\pgfpathmoveto{\pgfqpoint{2.544982in}{5.378797in}}%
\pgfpathquadraticcurveto{\pgfqpoint{2.532524in}{5.377555in}}{\pgfqpoint{2.531561in}{5.377459in}}%
\pgfusepath{stroke}%
\end{pgfscope}%
\begin{pgfscope}%
\pgfpathrectangle{\pgfqpoint{1.250000in}{4.155455in}}{\pgfqpoint{2.279412in}{2.004545in}}%
\pgfusepath{clip}%
\pgfsetroundcap%
\pgfsetroundjoin%
\definecolor{currentfill}{rgb}{0.229739,0.322361,0.545706}%
\pgfsetfillcolor{currentfill}%
\pgfsetlinewidth{0.746721pt}%
\definecolor{currentstroke}{rgb}{0.229739,0.322361,0.545706}%
\pgfsetstrokecolor{currentstroke}%
\pgfsetdash{}{0pt}%
\pgfpathmoveto{\pgfqpoint{2.589598in}{5.355329in}}%
\pgfpathlineto{\pgfqpoint{2.531561in}{5.377459in}}%
\pgfpathlineto{\pgfqpoint{2.584087in}{5.410610in}}%
\pgfpathlineto{\pgfqpoint{2.589598in}{5.355329in}}%
\pgfpathlineto{\pgfqpoint{2.589598in}{5.355329in}}%
\pgfpathclose%
\pgfusepath{stroke,fill}%
\end{pgfscope}%
\begin{pgfscope}%
\pgfpathrectangle{\pgfqpoint{1.250000in}{4.155455in}}{\pgfqpoint{2.279412in}{2.004545in}}%
\pgfusepath{clip}%
\pgfsetbuttcap%
\pgfsetroundjoin%
\pgfsetlinewidth{1.505625pt}%
\definecolor{currentstroke}{rgb}{0.000000,0.000000,0.000000}%
\pgfsetstrokecolor{currentstroke}%
\pgfsetdash{}{0pt}%
\pgfpathmoveto{\pgfqpoint{2.043384in}{4.494895in}}%
\pgfpathlineto{\pgfqpoint{2.043384in}{5.820559in}}%
\pgfusepath{stroke}%
\end{pgfscope}%
\begin{pgfscope}%
\pgfpathrectangle{\pgfqpoint{1.250000in}{4.155455in}}{\pgfqpoint{2.279412in}{2.004545in}}%
\pgfusepath{clip}%
\pgfsetbuttcap%
\pgfsetroundjoin%
\pgfsetlinewidth{1.505625pt}%
\definecolor{currentstroke}{rgb}{0.000000,0.000000,0.000000}%
\pgfsetstrokecolor{currentstroke}%
\pgfsetdash{}{0pt}%
\pgfpathmoveto{\pgfqpoint{3.191910in}{4.494895in}}%
\pgfpathlineto{\pgfqpoint{3.191910in}{5.820559in}}%
\pgfusepath{stroke}%
\end{pgfscope}%
\begin{pgfscope}%
\pgfsetrectcap%
\pgfsetmiterjoin%
\pgfsetlinewidth{0.803000pt}%
\definecolor{currentstroke}{rgb}{0.000000,0.000000,0.000000}%
\pgfsetstrokecolor{currentstroke}%
\pgfsetdash{}{0pt}%
\pgfpathmoveto{\pgfqpoint{1.250000in}{4.155455in}}%
\pgfpathlineto{\pgfqpoint{1.250000in}{6.160000in}}%
\pgfusepath{stroke}%
\end{pgfscope}%
\begin{pgfscope}%
\pgfsetrectcap%
\pgfsetmiterjoin%
\pgfsetlinewidth{0.803000pt}%
\definecolor{currentstroke}{rgb}{0.000000,0.000000,0.000000}%
\pgfsetstrokecolor{currentstroke}%
\pgfsetdash{}{0pt}%
\pgfpathmoveto{\pgfqpoint{3.529412in}{4.155455in}}%
\pgfpathlineto{\pgfqpoint{3.529412in}{6.160000in}}%
\pgfusepath{stroke}%
\end{pgfscope}%
\begin{pgfscope}%
\pgfsetrectcap%
\pgfsetmiterjoin%
\pgfsetlinewidth{0.803000pt}%
\definecolor{currentstroke}{rgb}{0.000000,0.000000,0.000000}%
\pgfsetstrokecolor{currentstroke}%
\pgfsetdash{}{0pt}%
\pgfpathmoveto{\pgfqpoint{1.250000in}{4.155455in}}%
\pgfpathlineto{\pgfqpoint{3.529412in}{4.155455in}}%
\pgfusepath{stroke}%
\end{pgfscope}%
\begin{pgfscope}%
\pgfsetrectcap%
\pgfsetmiterjoin%
\pgfsetlinewidth{0.803000pt}%
\definecolor{currentstroke}{rgb}{0.000000,0.000000,0.000000}%
\pgfsetstrokecolor{currentstroke}%
\pgfsetdash{}{0pt}%
\pgfpathmoveto{\pgfqpoint{1.250000in}{6.160000in}}%
\pgfpathlineto{\pgfqpoint{3.529412in}{6.160000in}}%
\pgfusepath{stroke}%
\end{pgfscope}%
\begin{pgfscope}%
\definecolor{textcolor}{rgb}{0.000000,0.000000,0.000000}%
\pgfsetstrokecolor{textcolor}%
\pgfsetfillcolor{textcolor}%
\pgftext[x=2.389706in,y=6.243333in,,base]{\color{textcolor}\sffamily\fontsize{12.000000}{14.400000}\selectfont a)}%
\end{pgfscope}%
\begin{pgfscope}%
\pgfsetbuttcap%
\pgfsetmiterjoin%
\definecolor{currentfill}{rgb}{1.000000,1.000000,1.000000}%
\pgfsetfillcolor{currentfill}%
\pgfsetlinewidth{0.000000pt}%
\definecolor{currentstroke}{rgb}{0.000000,0.000000,0.000000}%
\pgfsetstrokecolor{currentstroke}%
\pgfsetstrokeopacity{0.000000}%
\pgfsetdash{}{0pt}%
\pgfpathmoveto{\pgfqpoint{3.985294in}{4.155455in}}%
\pgfpathlineto{\pgfqpoint{6.264706in}{4.155455in}}%
\pgfpathlineto{\pgfqpoint{6.264706in}{6.160000in}}%
\pgfpathlineto{\pgfqpoint{3.985294in}{6.160000in}}%
\pgfpathlineto{\pgfqpoint{3.985294in}{4.155455in}}%
\pgfpathclose%
\pgfusepath{fill}%
\end{pgfscope}%
\begin{pgfscope}%
\pgfpathrectangle{\pgfqpoint{3.985294in}{4.155455in}}{\pgfqpoint{2.279412in}{2.004545in}}%
\pgfusepath{clip}%
\pgfsys@transformcm{2.291667}{0.000000}{0.000000}{2.013889}{3.985294in}{4.155455in}%
\pgftext[left,bottom]{\includegraphics[interpolate=false,width=1.000000in,height=1.000000in]{q_series-img1.png}}%
\end{pgfscope}%
\begin{pgfscope}%
\pgfsetbuttcap%
\pgfsetroundjoin%
\definecolor{currentfill}{rgb}{0.000000,0.000000,0.000000}%
\pgfsetfillcolor{currentfill}%
\pgfsetlinewidth{0.803000pt}%
\definecolor{currentstroke}{rgb}{0.000000,0.000000,0.000000}%
\pgfsetstrokecolor{currentstroke}%
\pgfsetdash{}{0pt}%
\pgfsys@defobject{currentmarker}{\pgfqpoint{0.000000in}{-0.048611in}}{\pgfqpoint{0.000000in}{0.000000in}}{%
\pgfpathmoveto{\pgfqpoint{0.000000in}{0.000000in}}%
\pgfpathlineto{\pgfqpoint{0.000000in}{-0.048611in}}%
\pgfusepath{stroke,fill}%
}%
\begin{pgfscope}%
\pgfsys@transformshift{4.395836in}{4.155455in}%
\pgfsys@useobject{currentmarker}{}%
\end{pgfscope}%
\end{pgfscope}%
\begin{pgfscope}%
\pgfsetbuttcap%
\pgfsetroundjoin%
\definecolor{currentfill}{rgb}{0.000000,0.000000,0.000000}%
\pgfsetfillcolor{currentfill}%
\pgfsetlinewidth{0.803000pt}%
\definecolor{currentstroke}{rgb}{0.000000,0.000000,0.000000}%
\pgfsetstrokecolor{currentstroke}%
\pgfsetdash{}{0pt}%
\pgfsys@defobject{currentmarker}{\pgfqpoint{0.000000in}{-0.048611in}}{\pgfqpoint{0.000000in}{0.000000in}}{%
\pgfpathmoveto{\pgfqpoint{0.000000in}{0.000000in}}%
\pgfpathlineto{\pgfqpoint{0.000000in}{-0.048611in}}%
\pgfusepath{stroke,fill}%
}%
\begin{pgfscope}%
\pgfsys@transformshift{4.874388in}{4.155455in}%
\pgfsys@useobject{currentmarker}{}%
\end{pgfscope}%
\end{pgfscope}%
\begin{pgfscope}%
\pgfsetbuttcap%
\pgfsetroundjoin%
\definecolor{currentfill}{rgb}{0.000000,0.000000,0.000000}%
\pgfsetfillcolor{currentfill}%
\pgfsetlinewidth{0.803000pt}%
\definecolor{currentstroke}{rgb}{0.000000,0.000000,0.000000}%
\pgfsetstrokecolor{currentstroke}%
\pgfsetdash{}{0pt}%
\pgfsys@defobject{currentmarker}{\pgfqpoint{0.000000in}{-0.048611in}}{\pgfqpoint{0.000000in}{0.000000in}}{%
\pgfpathmoveto{\pgfqpoint{0.000000in}{0.000000in}}%
\pgfpathlineto{\pgfqpoint{0.000000in}{-0.048611in}}%
\pgfusepath{stroke,fill}%
}%
\begin{pgfscope}%
\pgfsys@transformshift{5.352941in}{4.155455in}%
\pgfsys@useobject{currentmarker}{}%
\end{pgfscope}%
\end{pgfscope}%
\begin{pgfscope}%
\pgfsetbuttcap%
\pgfsetroundjoin%
\definecolor{currentfill}{rgb}{0.000000,0.000000,0.000000}%
\pgfsetfillcolor{currentfill}%
\pgfsetlinewidth{0.803000pt}%
\definecolor{currentstroke}{rgb}{0.000000,0.000000,0.000000}%
\pgfsetstrokecolor{currentstroke}%
\pgfsetdash{}{0pt}%
\pgfsys@defobject{currentmarker}{\pgfqpoint{0.000000in}{-0.048611in}}{\pgfqpoint{0.000000in}{0.000000in}}{%
\pgfpathmoveto{\pgfqpoint{0.000000in}{0.000000in}}%
\pgfpathlineto{\pgfqpoint{0.000000in}{-0.048611in}}%
\pgfusepath{stroke,fill}%
}%
\begin{pgfscope}%
\pgfsys@transformshift{5.831494in}{4.155455in}%
\pgfsys@useobject{currentmarker}{}%
\end{pgfscope}%
\end{pgfscope}%
\begin{pgfscope}%
\definecolor{textcolor}{rgb}{0.000000,0.000000,0.000000}%
\pgfsetstrokecolor{textcolor}%
\pgfsetfillcolor{textcolor}%
\pgftext[x=5.125000in,y=4.099899in,,top]{\color{textcolor}\sffamily\fontsize{10.000000}{12.000000}\selectfont \(\displaystyle \zeta \, \mathrm{[\mu m]}\)}%
\end{pgfscope}%
\begin{pgfscope}%
\pgfsetbuttcap%
\pgfsetroundjoin%
\definecolor{currentfill}{rgb}{0.000000,0.000000,0.000000}%
\pgfsetfillcolor{currentfill}%
\pgfsetlinewidth{0.803000pt}%
\definecolor{currentstroke}{rgb}{0.000000,0.000000,0.000000}%
\pgfsetstrokecolor{currentstroke}%
\pgfsetdash{}{0pt}%
\pgfsys@defobject{currentmarker}{\pgfqpoint{-0.048611in}{0.000000in}}{\pgfqpoint{-0.000000in}{0.000000in}}{%
\pgfpathmoveto{\pgfqpoint{-0.000000in}{0.000000in}}%
\pgfpathlineto{\pgfqpoint{-0.048611in}{0.000000in}}%
\pgfusepath{stroke,fill}%
}%
\begin{pgfscope}%
\pgfsys@transformshift{3.985294in}{4.163479in}%
\pgfsys@useobject{currentmarker}{}%
\end{pgfscope}%
\end{pgfscope}%
\begin{pgfscope}%
\pgfsetbuttcap%
\pgfsetroundjoin%
\definecolor{currentfill}{rgb}{0.000000,0.000000,0.000000}%
\pgfsetfillcolor{currentfill}%
\pgfsetlinewidth{0.803000pt}%
\definecolor{currentstroke}{rgb}{0.000000,0.000000,0.000000}%
\pgfsetstrokecolor{currentstroke}%
\pgfsetdash{}{0pt}%
\pgfsys@defobject{currentmarker}{\pgfqpoint{-0.048611in}{0.000000in}}{\pgfqpoint{-0.000000in}{0.000000in}}{%
\pgfpathmoveto{\pgfqpoint{-0.000000in}{0.000000in}}%
\pgfpathlineto{\pgfqpoint{-0.048611in}{0.000000in}}%
\pgfusepath{stroke,fill}%
}%
\begin{pgfscope}%
\pgfsys@transformshift{3.985294in}{4.494895in}%
\pgfsys@useobject{currentmarker}{}%
\end{pgfscope}%
\end{pgfscope}%
\begin{pgfscope}%
\pgfsetbuttcap%
\pgfsetroundjoin%
\definecolor{currentfill}{rgb}{0.000000,0.000000,0.000000}%
\pgfsetfillcolor{currentfill}%
\pgfsetlinewidth{0.803000pt}%
\definecolor{currentstroke}{rgb}{0.000000,0.000000,0.000000}%
\pgfsetstrokecolor{currentstroke}%
\pgfsetdash{}{0pt}%
\pgfsys@defobject{currentmarker}{\pgfqpoint{-0.048611in}{0.000000in}}{\pgfqpoint{-0.000000in}{0.000000in}}{%
\pgfpathmoveto{\pgfqpoint{-0.000000in}{0.000000in}}%
\pgfpathlineto{\pgfqpoint{-0.048611in}{0.000000in}}%
\pgfusepath{stroke,fill}%
}%
\begin{pgfscope}%
\pgfsys@transformshift{3.985294in}{4.826311in}%
\pgfsys@useobject{currentmarker}{}%
\end{pgfscope}%
\end{pgfscope}%
\begin{pgfscope}%
\pgfsetbuttcap%
\pgfsetroundjoin%
\definecolor{currentfill}{rgb}{0.000000,0.000000,0.000000}%
\pgfsetfillcolor{currentfill}%
\pgfsetlinewidth{0.803000pt}%
\definecolor{currentstroke}{rgb}{0.000000,0.000000,0.000000}%
\pgfsetstrokecolor{currentstroke}%
\pgfsetdash{}{0pt}%
\pgfsys@defobject{currentmarker}{\pgfqpoint{-0.048611in}{0.000000in}}{\pgfqpoint{-0.000000in}{0.000000in}}{%
\pgfpathmoveto{\pgfqpoint{-0.000000in}{0.000000in}}%
\pgfpathlineto{\pgfqpoint{-0.048611in}{0.000000in}}%
\pgfusepath{stroke,fill}%
}%
\begin{pgfscope}%
\pgfsys@transformshift{3.985294in}{5.157727in}%
\pgfsys@useobject{currentmarker}{}%
\end{pgfscope}%
\end{pgfscope}%
\begin{pgfscope}%
\pgfsetbuttcap%
\pgfsetroundjoin%
\definecolor{currentfill}{rgb}{0.000000,0.000000,0.000000}%
\pgfsetfillcolor{currentfill}%
\pgfsetlinewidth{0.803000pt}%
\definecolor{currentstroke}{rgb}{0.000000,0.000000,0.000000}%
\pgfsetstrokecolor{currentstroke}%
\pgfsetdash{}{0pt}%
\pgfsys@defobject{currentmarker}{\pgfqpoint{-0.048611in}{0.000000in}}{\pgfqpoint{-0.000000in}{0.000000in}}{%
\pgfpathmoveto{\pgfqpoint{-0.000000in}{0.000000in}}%
\pgfpathlineto{\pgfqpoint{-0.048611in}{0.000000in}}%
\pgfusepath{stroke,fill}%
}%
\begin{pgfscope}%
\pgfsys@transformshift{3.985294in}{5.489143in}%
\pgfsys@useobject{currentmarker}{}%
\end{pgfscope}%
\end{pgfscope}%
\begin{pgfscope}%
\pgfsetbuttcap%
\pgfsetroundjoin%
\definecolor{currentfill}{rgb}{0.000000,0.000000,0.000000}%
\pgfsetfillcolor{currentfill}%
\pgfsetlinewidth{0.803000pt}%
\definecolor{currentstroke}{rgb}{0.000000,0.000000,0.000000}%
\pgfsetstrokecolor{currentstroke}%
\pgfsetdash{}{0pt}%
\pgfsys@defobject{currentmarker}{\pgfqpoint{-0.048611in}{0.000000in}}{\pgfqpoint{-0.000000in}{0.000000in}}{%
\pgfpathmoveto{\pgfqpoint{-0.000000in}{0.000000in}}%
\pgfpathlineto{\pgfqpoint{-0.048611in}{0.000000in}}%
\pgfusepath{stroke,fill}%
}%
\begin{pgfscope}%
\pgfsys@transformshift{3.985294in}{5.820559in}%
\pgfsys@useobject{currentmarker}{}%
\end{pgfscope}%
\end{pgfscope}%
\begin{pgfscope}%
\pgfsetbuttcap%
\pgfsetroundjoin%
\definecolor{currentfill}{rgb}{0.000000,0.000000,0.000000}%
\pgfsetfillcolor{currentfill}%
\pgfsetlinewidth{0.803000pt}%
\definecolor{currentstroke}{rgb}{0.000000,0.000000,0.000000}%
\pgfsetstrokecolor{currentstroke}%
\pgfsetdash{}{0pt}%
\pgfsys@defobject{currentmarker}{\pgfqpoint{-0.048611in}{0.000000in}}{\pgfqpoint{-0.000000in}{0.000000in}}{%
\pgfpathmoveto{\pgfqpoint{-0.000000in}{0.000000in}}%
\pgfpathlineto{\pgfqpoint{-0.048611in}{0.000000in}}%
\pgfusepath{stroke,fill}%
}%
\begin{pgfscope}%
\pgfsys@transformshift{3.985294in}{6.151975in}%
\pgfsys@useobject{currentmarker}{}%
\end{pgfscope}%
\end{pgfscope}%
\begin{pgfscope}%
\definecolor{textcolor}{rgb}{0.000000,0.000000,0.000000}%
\pgfsetstrokecolor{textcolor}%
\pgfsetfillcolor{textcolor}%
\pgftext[x=3.929739in,y=5.157727in,,bottom,rotate=90.000000]{\color{textcolor}\sffamily\fontsize{10.000000}{12.000000}\selectfont \(\displaystyle z \, \mathrm{[\mu m]}\)}%
\end{pgfscope}%
\begin{pgfscope}%
\pgfpathrectangle{\pgfqpoint{3.985294in}{4.155455in}}{\pgfqpoint{2.279412in}{2.004545in}}%
\pgfusepath{clip}%
\pgfsetbuttcap%
\pgfsetroundjoin%
\pgfsetlinewidth{0.000000pt}%
\definecolor{currentstroke}{rgb}{0.000000,0.000000,0.000000}%
\pgfsetstrokecolor{currentstroke}%
\pgfsetdash{}{0pt}%
\pgfpathmoveto{\pgfqpoint{6.028836in}{5.061185in}}%
\pgfpathlineto{\pgfqpoint{6.027320in}{5.060322in}}%
\pgfusepath{}%
\end{pgfscope}%
\begin{pgfscope}%
\pgfpathrectangle{\pgfqpoint{3.985294in}{4.155455in}}{\pgfqpoint{2.279412in}{2.004545in}}%
\pgfusepath{clip}%
\pgfsetbuttcap%
\pgfsetroundjoin%
\pgfsetlinewidth{0.305143pt}%
\definecolor{currentstroke}{rgb}{0.267004,0.004874,0.329415}%
\pgfsetstrokecolor{currentstroke}%
\pgfsetdash{}{0pt}%
\pgfpathmoveto{\pgfqpoint{6.027320in}{5.060322in}}%
\pgfpathlineto{\pgfqpoint{6.026713in}{5.059604in}}%
\pgfusepath{stroke}%
\end{pgfscope}%
\begin{pgfscope}%
\pgfpathrectangle{\pgfqpoint{3.985294in}{4.155455in}}{\pgfqpoint{2.279412in}{2.004545in}}%
\pgfusepath{clip}%
\pgfsetbuttcap%
\pgfsetroundjoin%
\pgfsetlinewidth{0.304854pt}%
\definecolor{currentstroke}{rgb}{0.267004,0.004874,0.329415}%
\pgfsetstrokecolor{currentstroke}%
\pgfsetdash{}{0pt}%
\pgfpathmoveto{\pgfqpoint{6.026713in}{5.059604in}}%
\pgfpathlineto{\pgfqpoint{6.026378in}{5.058832in}}%
\pgfusepath{stroke}%
\end{pgfscope}%
\begin{pgfscope}%
\pgfpathrectangle{\pgfqpoint{3.985294in}{4.155455in}}{\pgfqpoint{2.279412in}{2.004545in}}%
\pgfusepath{clip}%
\pgfsetbuttcap%
\pgfsetroundjoin%
\pgfsetlinewidth{0.304440pt}%
\definecolor{currentstroke}{rgb}{0.267004,0.004874,0.329415}%
\pgfsetstrokecolor{currentstroke}%
\pgfsetdash{}{0pt}%
\pgfpathmoveto{\pgfqpoint{6.026378in}{5.058832in}}%
\pgfpathlineto{\pgfqpoint{6.026006in}{5.057898in}}%
\pgfusepath{stroke}%
\end{pgfscope}%
\begin{pgfscope}%
\pgfpathrectangle{\pgfqpoint{3.985294in}{4.155455in}}{\pgfqpoint{2.279412in}{2.004545in}}%
\pgfusepath{clip}%
\pgfsetbuttcap%
\pgfsetroundjoin%
\pgfsetlinewidth{0.304459pt}%
\definecolor{currentstroke}{rgb}{0.267004,0.004874,0.329415}%
\pgfsetstrokecolor{currentstroke}%
\pgfsetdash{}{0pt}%
\pgfpathmoveto{\pgfqpoint{6.026006in}{5.057898in}}%
\pgfpathlineto{\pgfqpoint{6.025764in}{5.056788in}}%
\pgfusepath{stroke}%
\end{pgfscope}%
\begin{pgfscope}%
\pgfpathrectangle{\pgfqpoint{3.985294in}{4.155455in}}{\pgfqpoint{2.279412in}{2.004545in}}%
\pgfusepath{clip}%
\pgfsetbuttcap%
\pgfsetroundjoin%
\pgfsetlinewidth{0.304748pt}%
\definecolor{currentstroke}{rgb}{0.267004,0.004874,0.329415}%
\pgfsetstrokecolor{currentstroke}%
\pgfsetdash{}{0pt}%
\pgfpathmoveto{\pgfqpoint{6.025764in}{5.056788in}}%
\pgfpathlineto{\pgfqpoint{6.025517in}{5.055557in}}%
\pgfusepath{stroke}%
\end{pgfscope}%
\begin{pgfscope}%
\pgfpathrectangle{\pgfqpoint{3.985294in}{4.155455in}}{\pgfqpoint{2.279412in}{2.004545in}}%
\pgfusepath{clip}%
\pgfsetbuttcap%
\pgfsetroundjoin%
\pgfsetlinewidth{0.305057pt}%
\definecolor{currentstroke}{rgb}{0.267004,0.004874,0.329415}%
\pgfsetstrokecolor{currentstroke}%
\pgfsetdash{}{0pt}%
\pgfpathmoveto{\pgfqpoint{6.025517in}{5.055557in}}%
\pgfpathlineto{\pgfqpoint{6.025390in}{5.054233in}}%
\pgfusepath{stroke}%
\end{pgfscope}%
\begin{pgfscope}%
\pgfpathrectangle{\pgfqpoint{3.985294in}{4.155455in}}{\pgfqpoint{2.279412in}{2.004545in}}%
\pgfusepath{clip}%
\pgfsetbuttcap%
\pgfsetroundjoin%
\pgfsetlinewidth{0.305329pt}%
\definecolor{currentstroke}{rgb}{0.267004,0.004874,0.329415}%
\pgfsetstrokecolor{currentstroke}%
\pgfsetdash{}{0pt}%
\pgfpathmoveto{\pgfqpoint{6.025390in}{5.054233in}}%
\pgfpathlineto{\pgfqpoint{6.025365in}{5.052802in}}%
\pgfusepath{stroke}%
\end{pgfscope}%
\begin{pgfscope}%
\pgfpathrectangle{\pgfqpoint{3.985294in}{4.155455in}}{\pgfqpoint{2.279412in}{2.004545in}}%
\pgfusepath{clip}%
\pgfsetbuttcap%
\pgfsetroundjoin%
\pgfsetlinewidth{0.305578pt}%
\definecolor{currentstroke}{rgb}{0.267004,0.004874,0.329415}%
\pgfsetstrokecolor{currentstroke}%
\pgfsetdash{}{0pt}%
\pgfpathmoveto{\pgfqpoint{6.025365in}{5.052802in}}%
\pgfpathlineto{\pgfqpoint{6.025318in}{5.051218in}}%
\pgfusepath{stroke}%
\end{pgfscope}%
\begin{pgfscope}%
\pgfpathrectangle{\pgfqpoint{3.985294in}{4.155455in}}{\pgfqpoint{2.279412in}{2.004545in}}%
\pgfusepath{clip}%
\pgfsetbuttcap%
\pgfsetroundjoin%
\pgfsetlinewidth{0.305860pt}%
\definecolor{currentstroke}{rgb}{0.267004,0.004874,0.329415}%
\pgfsetstrokecolor{currentstroke}%
\pgfsetdash{}{0pt}%
\pgfpathmoveto{\pgfqpoint{6.025318in}{5.051218in}}%
\pgfpathlineto{\pgfqpoint{6.025129in}{5.049445in}}%
\pgfusepath{stroke}%
\end{pgfscope}%
\begin{pgfscope}%
\pgfpathrectangle{\pgfqpoint{3.985294in}{4.155455in}}{\pgfqpoint{2.279412in}{2.004545in}}%
\pgfusepath{clip}%
\pgfsetbuttcap%
\pgfsetroundjoin%
\pgfsetlinewidth{0.306224pt}%
\definecolor{currentstroke}{rgb}{0.267004,0.004874,0.329415}%
\pgfsetstrokecolor{currentstroke}%
\pgfsetdash{}{0pt}%
\pgfpathmoveto{\pgfqpoint{6.025129in}{5.049445in}}%
\pgfpathlineto{\pgfqpoint{6.024823in}{5.047492in}}%
\pgfusepath{stroke}%
\end{pgfscope}%
\begin{pgfscope}%
\pgfpathrectangle{\pgfqpoint{3.985294in}{4.155455in}}{\pgfqpoint{2.279412in}{2.004545in}}%
\pgfusepath{clip}%
\pgfsetbuttcap%
\pgfsetroundjoin%
\pgfsetlinewidth{0.306657pt}%
\definecolor{currentstroke}{rgb}{0.267004,0.004874,0.329415}%
\pgfsetstrokecolor{currentstroke}%
\pgfsetdash{}{0pt}%
\pgfpathmoveto{\pgfqpoint{6.024823in}{5.047492in}}%
\pgfpathlineto{\pgfqpoint{6.024823in}{5.047492in}}%
\pgfusepath{stroke}%
\end{pgfscope}%
\begin{pgfscope}%
\pgfpathrectangle{\pgfqpoint{3.985294in}{4.155455in}}{\pgfqpoint{2.279412in}{2.004545in}}%
\pgfusepath{clip}%
\pgfsetbuttcap%
\pgfsetroundjoin%
\pgfsetlinewidth{0.306657pt}%
\definecolor{currentstroke}{rgb}{0.267004,0.004874,0.329415}%
\pgfsetstrokecolor{currentstroke}%
\pgfsetdash{}{0pt}%
\pgfpathmoveto{\pgfqpoint{6.024823in}{5.047492in}}%
\pgfpathlineto{\pgfqpoint{6.023615in}{5.044909in}}%
\pgfusepath{stroke}%
\end{pgfscope}%
\begin{pgfscope}%
\pgfpathrectangle{\pgfqpoint{3.985294in}{4.155455in}}{\pgfqpoint{2.279412in}{2.004545in}}%
\pgfusepath{clip}%
\pgfsetbuttcap%
\pgfsetroundjoin%
\pgfsetlinewidth{0.307496pt}%
\definecolor{currentstroke}{rgb}{0.267004,0.004874,0.329415}%
\pgfsetstrokecolor{currentstroke}%
\pgfsetdash{}{0pt}%
\pgfpathmoveto{\pgfqpoint{6.023615in}{5.044909in}}%
\pgfpathlineto{\pgfqpoint{6.023526in}{5.042806in}}%
\pgfusepath{stroke}%
\end{pgfscope}%
\begin{pgfscope}%
\pgfpathrectangle{\pgfqpoint{3.985294in}{4.155455in}}{\pgfqpoint{2.279412in}{2.004545in}}%
\pgfusepath{clip}%
\pgfsetbuttcap%
\pgfsetroundjoin%
\pgfsetlinewidth{0.307855pt}%
\definecolor{currentstroke}{rgb}{0.267004,0.004874,0.329415}%
\pgfsetstrokecolor{currentstroke}%
\pgfsetdash{}{0pt}%
\pgfpathmoveto{\pgfqpoint{6.023526in}{5.042806in}}%
\pgfpathlineto{\pgfqpoint{6.024314in}{5.041070in}}%
\pgfusepath{stroke}%
\end{pgfscope}%
\begin{pgfscope}%
\pgfpathrectangle{\pgfqpoint{3.985294in}{4.155455in}}{\pgfqpoint{2.279412in}{2.004545in}}%
\pgfusepath{clip}%
\pgfsetbuttcap%
\pgfsetroundjoin%
\pgfsetlinewidth{0.307872pt}%
\definecolor{currentstroke}{rgb}{0.267004,0.004874,0.329415}%
\pgfsetstrokecolor{currentstroke}%
\pgfsetdash{}{0pt}%
\pgfpathmoveto{\pgfqpoint{6.024314in}{5.041070in}}%
\pgfpathlineto{\pgfqpoint{6.025719in}{5.039112in}}%
\pgfusepath{stroke}%
\end{pgfscope}%
\begin{pgfscope}%
\pgfpathrectangle{\pgfqpoint{3.985294in}{4.155455in}}{\pgfqpoint{2.279412in}{2.004545in}}%
\pgfusepath{clip}%
\pgfsetbuttcap%
\pgfsetroundjoin%
\pgfsetlinewidth{0.307750pt}%
\definecolor{currentstroke}{rgb}{0.267004,0.004874,0.329415}%
\pgfsetstrokecolor{currentstroke}%
\pgfsetdash{}{0pt}%
\pgfpathmoveto{\pgfqpoint{6.025719in}{5.039112in}}%
\pgfpathlineto{\pgfqpoint{6.025719in}{5.039112in}}%
\pgfusepath{stroke}%
\end{pgfscope}%
\begin{pgfscope}%
\pgfpathrectangle{\pgfqpoint{3.985294in}{4.155455in}}{\pgfqpoint{2.279412in}{2.004545in}}%
\pgfusepath{clip}%
\pgfsetbuttcap%
\pgfsetroundjoin%
\pgfsetlinewidth{0.307750pt}%
\definecolor{currentstroke}{rgb}{0.267004,0.004874,0.329415}%
\pgfsetstrokecolor{currentstroke}%
\pgfsetdash{}{0pt}%
\pgfpathmoveto{\pgfqpoint{6.025719in}{5.039112in}}%
\pgfpathlineto{\pgfqpoint{6.028397in}{5.036421in}}%
\pgfusepath{stroke}%
\end{pgfscope}%
\begin{pgfscope}%
\pgfpathrectangle{\pgfqpoint{3.985294in}{4.155455in}}{\pgfqpoint{2.279412in}{2.004545in}}%
\pgfusepath{clip}%
\pgfsetbuttcap%
\pgfsetroundjoin%
\pgfsetlinewidth{0.307912pt}%
\definecolor{currentstroke}{rgb}{0.267004,0.004874,0.329415}%
\pgfsetstrokecolor{currentstroke}%
\pgfsetdash{}{0pt}%
\pgfpathmoveto{\pgfqpoint{6.028397in}{5.036421in}}%
\pgfpathlineto{\pgfqpoint{6.028397in}{5.036421in}}%
\pgfusepath{stroke}%
\end{pgfscope}%
\begin{pgfscope}%
\pgfpathrectangle{\pgfqpoint{3.985294in}{4.155455in}}{\pgfqpoint{2.279412in}{2.004545in}}%
\pgfusepath{clip}%
\pgfsetbuttcap%
\pgfsetroundjoin%
\pgfsetlinewidth{0.307912pt}%
\definecolor{currentstroke}{rgb}{0.267004,0.004874,0.329415}%
\pgfsetstrokecolor{currentstroke}%
\pgfsetdash{}{0pt}%
\pgfpathmoveto{\pgfqpoint{6.028397in}{5.036421in}}%
\pgfpathlineto{\pgfqpoint{6.035927in}{5.032747in}}%
\pgfusepath{stroke}%
\end{pgfscope}%
\begin{pgfscope}%
\pgfpathrectangle{\pgfqpoint{3.985294in}{4.155455in}}{\pgfqpoint{2.279412in}{2.004545in}}%
\pgfusepath{clip}%
\pgfsetbuttcap%
\pgfsetroundjoin%
\pgfsetlinewidth{0.307884pt}%
\definecolor{currentstroke}{rgb}{0.267004,0.004874,0.329415}%
\pgfsetstrokecolor{currentstroke}%
\pgfsetdash{}{0pt}%
\pgfpathmoveto{\pgfqpoint{6.035927in}{5.032747in}}%
\pgfpathlineto{\pgfqpoint{6.035927in}{5.032747in}}%
\pgfusepath{stroke}%
\end{pgfscope}%
\begin{pgfscope}%
\pgfpathrectangle{\pgfqpoint{3.985294in}{4.155455in}}{\pgfqpoint{2.279412in}{2.004545in}}%
\pgfusepath{clip}%
\pgfsetbuttcap%
\pgfsetroundjoin%
\pgfsetlinewidth{0.307884pt}%
\definecolor{currentstroke}{rgb}{0.267004,0.004874,0.329415}%
\pgfsetstrokecolor{currentstroke}%
\pgfsetdash{}{0pt}%
\pgfpathmoveto{\pgfqpoint{6.035927in}{5.032747in}}%
\pgfpathlineto{\pgfqpoint{6.040657in}{5.029197in}}%
\pgfusepath{stroke}%
\end{pgfscope}%
\begin{pgfscope}%
\pgfpathrectangle{\pgfqpoint{3.985294in}{4.155455in}}{\pgfqpoint{2.279412in}{2.004545in}}%
\pgfusepath{clip}%
\pgfsetbuttcap%
\pgfsetroundjoin%
\pgfsetlinewidth{0.307931pt}%
\definecolor{currentstroke}{rgb}{0.267004,0.004874,0.329415}%
\pgfsetstrokecolor{currentstroke}%
\pgfsetdash{}{0pt}%
\pgfpathmoveto{\pgfqpoint{6.040657in}{5.029197in}}%
\pgfpathlineto{\pgfqpoint{6.042409in}{5.026700in}}%
\pgfusepath{stroke}%
\end{pgfscope}%
\begin{pgfscope}%
\pgfpathrectangle{\pgfqpoint{3.985294in}{4.155455in}}{\pgfqpoint{2.279412in}{2.004545in}}%
\pgfusepath{clip}%
\pgfsetbuttcap%
\pgfsetroundjoin%
\pgfsetlinewidth{0.308150pt}%
\definecolor{currentstroke}{rgb}{0.268510,0.009605,0.335427}%
\pgfsetstrokecolor{currentstroke}%
\pgfsetdash{}{0pt}%
\pgfpathmoveto{\pgfqpoint{6.042409in}{5.026700in}}%
\pgfpathlineto{\pgfqpoint{6.043603in}{5.025079in}}%
\pgfusepath{stroke}%
\end{pgfscope}%
\begin{pgfscope}%
\pgfpathrectangle{\pgfqpoint{3.985294in}{4.155455in}}{\pgfqpoint{2.279412in}{2.004545in}}%
\pgfusepath{clip}%
\pgfsetbuttcap%
\pgfsetroundjoin%
\pgfsetlinewidth{0.308238pt}%
\definecolor{currentstroke}{rgb}{0.268510,0.009605,0.335427}%
\pgfsetstrokecolor{currentstroke}%
\pgfsetdash{}{0pt}%
\pgfpathmoveto{\pgfqpoint{6.043603in}{5.025079in}}%
\pgfpathlineto{\pgfqpoint{6.045274in}{5.023924in}}%
\pgfusepath{stroke}%
\end{pgfscope}%
\begin{pgfscope}%
\pgfpathrectangle{\pgfqpoint{3.985294in}{4.155455in}}{\pgfqpoint{2.279412in}{2.004545in}}%
\pgfusepath{clip}%
\pgfsetbuttcap%
\pgfsetroundjoin%
\pgfsetlinewidth{0.308074pt}%
\definecolor{currentstroke}{rgb}{0.268510,0.009605,0.335427}%
\pgfsetstrokecolor{currentstroke}%
\pgfsetdash{}{0pt}%
\pgfpathmoveto{\pgfqpoint{6.045274in}{5.023924in}}%
\pgfpathlineto{\pgfqpoint{6.048254in}{5.022407in}}%
\pgfusepath{stroke}%
\end{pgfscope}%
\begin{pgfscope}%
\pgfpathrectangle{\pgfqpoint{3.985294in}{4.155455in}}{\pgfqpoint{2.279412in}{2.004545in}}%
\pgfusepath{clip}%
\pgfsetbuttcap%
\pgfsetroundjoin%
\pgfsetlinewidth{0.307598pt}%
\definecolor{currentstroke}{rgb}{0.267004,0.004874,0.329415}%
\pgfsetstrokecolor{currentstroke}%
\pgfsetdash{}{0pt}%
\pgfpathmoveto{\pgfqpoint{6.048254in}{5.022407in}}%
\pgfpathlineto{\pgfqpoint{6.048254in}{5.022407in}}%
\pgfusepath{stroke}%
\end{pgfscope}%
\begin{pgfscope}%
\pgfpathrectangle{\pgfqpoint{3.985294in}{4.155455in}}{\pgfqpoint{2.279412in}{2.004545in}}%
\pgfusepath{clip}%
\pgfsetbuttcap%
\pgfsetroundjoin%
\pgfsetlinewidth{0.307598pt}%
\definecolor{currentstroke}{rgb}{0.267004,0.004874,0.329415}%
\pgfsetstrokecolor{currentstroke}%
\pgfsetdash{}{0pt}%
\pgfpathmoveto{\pgfqpoint{6.048254in}{5.022407in}}%
\pgfpathlineto{\pgfqpoint{6.048254in}{5.022407in}}%
\pgfusepath{stroke}%
\end{pgfscope}%
\begin{pgfscope}%
\pgfpathrectangle{\pgfqpoint{3.985294in}{4.155455in}}{\pgfqpoint{2.279412in}{2.004545in}}%
\pgfusepath{clip}%
\pgfsetbuttcap%
\pgfsetroundjoin%
\pgfsetlinewidth{0.308931pt}%
\definecolor{currentstroke}{rgb}{0.268510,0.009605,0.335427}%
\pgfsetstrokecolor{currentstroke}%
\pgfsetdash{}{0pt}%
\pgfpathmoveto{\pgfqpoint{6.014399in}{5.022290in}}%
\pgfpathlineto{\pgfqpoint{5.996962in}{5.022407in}}%
\pgfusepath{stroke}%
\end{pgfscope}%
\begin{pgfscope}%
\pgfpathrectangle{\pgfqpoint{3.985294in}{4.155455in}}{\pgfqpoint{2.279412in}{2.004545in}}%
\pgfusepath{clip}%
\pgfsetbuttcap%
\pgfsetroundjoin%
\pgfsetlinewidth{0.310171pt}%
\definecolor{currentstroke}{rgb}{0.268510,0.009605,0.335427}%
\pgfsetstrokecolor{currentstroke}%
\pgfsetdash{}{0pt}%
\pgfpathmoveto{\pgfqpoint{5.996962in}{5.022407in}}%
\pgfpathlineto{\pgfqpoint{5.996962in}{5.022407in}}%
\pgfusepath{stroke}%
\end{pgfscope}%
\begin{pgfscope}%
\pgfpathrectangle{\pgfqpoint{3.985294in}{4.155455in}}{\pgfqpoint{2.279412in}{2.004545in}}%
\pgfusepath{clip}%
\pgfsetbuttcap%
\pgfsetroundjoin%
\pgfsetlinewidth{0.310171pt}%
\definecolor{currentstroke}{rgb}{0.268510,0.009605,0.335427}%
\pgfsetstrokecolor{currentstroke}%
\pgfsetdash{}{0pt}%
\pgfpathmoveto{\pgfqpoint{5.996962in}{5.022407in}}%
\pgfpathlineto{\pgfqpoint{5.946864in}{5.022777in}}%
\pgfusepath{stroke}%
\end{pgfscope}%
\begin{pgfscope}%
\pgfpathrectangle{\pgfqpoint{3.985294in}{4.155455in}}{\pgfqpoint{2.279412in}{2.004545in}}%
\pgfusepath{clip}%
\pgfsetbuttcap%
\pgfsetroundjoin%
\pgfsetlinewidth{0.320468pt}%
\definecolor{currentstroke}{rgb}{0.269944,0.014625,0.341379}%
\pgfsetstrokecolor{currentstroke}%
\pgfsetdash{}{0pt}%
\pgfpathmoveto{\pgfqpoint{5.946864in}{5.022777in}}%
\pgfpathlineto{\pgfqpoint{5.896745in}{5.023449in}}%
\pgfusepath{stroke}%
\end{pgfscope}%
\begin{pgfscope}%
\pgfpathrectangle{\pgfqpoint{3.985294in}{4.155455in}}{\pgfqpoint{2.279412in}{2.004545in}}%
\pgfusepath{clip}%
\pgfsetbuttcap%
\pgfsetroundjoin%
\pgfsetlinewidth{0.322000pt}%
\definecolor{currentstroke}{rgb}{0.271305,0.019942,0.347269}%
\pgfsetstrokecolor{currentstroke}%
\pgfsetdash{}{0pt}%
\pgfpathmoveto{\pgfqpoint{5.896745in}{5.023449in}}%
\pgfpathlineto{\pgfqpoint{5.846630in}{5.024493in}}%
\pgfusepath{stroke}%
\end{pgfscope}%
\begin{pgfscope}%
\pgfpathrectangle{\pgfqpoint{3.985294in}{4.155455in}}{\pgfqpoint{2.279412in}{2.004545in}}%
\pgfusepath{clip}%
\pgfsetbuttcap%
\pgfsetroundjoin%
\pgfsetlinewidth{0.342656pt}%
\definecolor{currentstroke}{rgb}{0.274952,0.037752,0.364543}%
\pgfsetstrokecolor{currentstroke}%
\pgfsetdash{}{0pt}%
\pgfpathmoveto{\pgfqpoint{5.846630in}{5.024493in}}%
\pgfpathlineto{\pgfqpoint{5.796518in}{5.026111in}}%
\pgfusepath{stroke}%
\end{pgfscope}%
\begin{pgfscope}%
\pgfpathrectangle{\pgfqpoint{3.985294in}{4.155455in}}{\pgfqpoint{2.279412in}{2.004545in}}%
\pgfusepath{clip}%
\pgfsetbuttcap%
\pgfsetroundjoin%
\pgfsetlinewidth{0.350010pt}%
\definecolor{currentstroke}{rgb}{0.276022,0.044167,0.370164}%
\pgfsetstrokecolor{currentstroke}%
\pgfsetdash{}{0pt}%
\pgfpathmoveto{\pgfqpoint{5.796518in}{5.026111in}}%
\pgfpathlineto{\pgfqpoint{5.746420in}{5.027875in}}%
\pgfusepath{stroke}%
\end{pgfscope}%
\begin{pgfscope}%
\pgfpathrectangle{\pgfqpoint{3.985294in}{4.155455in}}{\pgfqpoint{2.279412in}{2.004545in}}%
\pgfusepath{clip}%
\pgfsetbuttcap%
\pgfsetroundjoin%
\pgfsetlinewidth{0.371405pt}%
\definecolor{currentstroke}{rgb}{0.278791,0.062145,0.386592}%
\pgfsetstrokecolor{currentstroke}%
\pgfsetdash{}{0pt}%
\pgfpathmoveto{\pgfqpoint{5.746420in}{5.027875in}}%
\pgfpathlineto{\pgfqpoint{5.696308in}{5.029401in}}%
\pgfusepath{stroke}%
\end{pgfscope}%
\begin{pgfscope}%
\pgfpathrectangle{\pgfqpoint{3.985294in}{4.155455in}}{\pgfqpoint{2.279412in}{2.004545in}}%
\pgfusepath{clip}%
\pgfsetbuttcap%
\pgfsetroundjoin%
\pgfsetlinewidth{0.395365pt}%
\definecolor{currentstroke}{rgb}{0.280894,0.078907,0.402329}%
\pgfsetstrokecolor{currentstroke}%
\pgfsetdash{}{0pt}%
\pgfpathmoveto{\pgfqpoint{5.696308in}{5.029401in}}%
\pgfpathlineto{\pgfqpoint{5.646176in}{5.030608in}}%
\pgfusepath{stroke}%
\end{pgfscope}%
\begin{pgfscope}%
\pgfpathrectangle{\pgfqpoint{3.985294in}{4.155455in}}{\pgfqpoint{2.279412in}{2.004545in}}%
\pgfusepath{clip}%
\pgfsetbuttcap%
\pgfsetroundjoin%
\pgfsetlinewidth{0.443340pt}%
\definecolor{currentstroke}{rgb}{0.283197,0.115680,0.436115}%
\pgfsetstrokecolor{currentstroke}%
\pgfsetdash{}{0pt}%
\pgfpathmoveto{\pgfqpoint{5.646176in}{5.030608in}}%
\pgfpathlineto{\pgfqpoint{5.596043in}{5.031841in}}%
\pgfusepath{stroke}%
\end{pgfscope}%
\begin{pgfscope}%
\pgfpathrectangle{\pgfqpoint{3.985294in}{4.155455in}}{\pgfqpoint{2.279412in}{2.004545in}}%
\pgfusepath{clip}%
\pgfsetbuttcap%
\pgfsetroundjoin%
\pgfsetlinewidth{0.495012pt}%
\definecolor{currentstroke}{rgb}{0.281412,0.155834,0.469201}%
\pgfsetstrokecolor{currentstroke}%
\pgfsetdash{}{0pt}%
\pgfpathmoveto{\pgfqpoint{5.596043in}{5.031841in}}%
\pgfpathlineto{\pgfqpoint{5.545915in}{5.033165in}}%
\pgfusepath{stroke}%
\end{pgfscope}%
\begin{pgfscope}%
\pgfpathrectangle{\pgfqpoint{3.985294in}{4.155455in}}{\pgfqpoint{2.279412in}{2.004545in}}%
\pgfusepath{clip}%
\pgfsetbuttcap%
\pgfsetroundjoin%
\pgfsetlinewidth{0.545425pt}%
\definecolor{currentstroke}{rgb}{0.276194,0.190074,0.493001}%
\pgfsetstrokecolor{currentstroke}%
\pgfsetdash{}{0pt}%
\pgfpathmoveto{\pgfqpoint{5.545915in}{5.033165in}}%
\pgfpathlineto{\pgfqpoint{5.495793in}{5.034657in}}%
\pgfusepath{stroke}%
\end{pgfscope}%
\begin{pgfscope}%
\pgfpathrectangle{\pgfqpoint{3.985294in}{4.155455in}}{\pgfqpoint{2.279412in}{2.004545in}}%
\pgfusepath{clip}%
\pgfsetbuttcap%
\pgfsetroundjoin%
\pgfsetlinewidth{0.631651pt}%
\definecolor{currentstroke}{rgb}{0.258965,0.251537,0.524736}%
\pgfsetstrokecolor{currentstroke}%
\pgfsetdash{}{0pt}%
\pgfpathmoveto{\pgfqpoint{5.495793in}{5.034657in}}%
\pgfpathlineto{\pgfqpoint{5.445683in}{5.036451in}}%
\pgfusepath{stroke}%
\end{pgfscope}%
\begin{pgfscope}%
\pgfpathrectangle{\pgfqpoint{3.985294in}{4.155455in}}{\pgfqpoint{2.279412in}{2.004545in}}%
\pgfusepath{clip}%
\pgfsetbuttcap%
\pgfsetroundjoin%
\pgfsetlinewidth{0.710791pt}%
\definecolor{currentstroke}{rgb}{0.239346,0.300855,0.540844}%
\pgfsetstrokecolor{currentstroke}%
\pgfsetdash{}{0pt}%
\pgfpathmoveto{\pgfqpoint{5.445683in}{5.036451in}}%
\pgfpathlineto{\pgfqpoint{5.395588in}{5.038548in}}%
\pgfusepath{stroke}%
\end{pgfscope}%
\begin{pgfscope}%
\pgfpathrectangle{\pgfqpoint{3.985294in}{4.155455in}}{\pgfqpoint{2.279412in}{2.004545in}}%
\pgfusepath{clip}%
\pgfsetbuttcap%
\pgfsetroundjoin%
\pgfsetlinewidth{0.796738pt}%
\definecolor{currentstroke}{rgb}{0.214298,0.355619,0.551184}%
\pgfsetstrokecolor{currentstroke}%
\pgfsetdash{}{0pt}%
\pgfpathmoveto{\pgfqpoint{5.395588in}{5.038548in}}%
\pgfpathlineto{\pgfqpoint{5.345506in}{5.040865in}}%
\pgfusepath{stroke}%
\end{pgfscope}%
\begin{pgfscope}%
\pgfpathrectangle{\pgfqpoint{3.985294in}{4.155455in}}{\pgfqpoint{2.279412in}{2.004545in}}%
\pgfusepath{clip}%
\pgfsetbuttcap%
\pgfsetroundjoin%
\pgfsetlinewidth{0.911975pt}%
\definecolor{currentstroke}{rgb}{0.183898,0.422383,0.556944}%
\pgfsetstrokecolor{currentstroke}%
\pgfsetdash{}{0pt}%
\pgfpathmoveto{\pgfqpoint{5.345506in}{5.040865in}}%
\pgfpathlineto{\pgfqpoint{5.295448in}{5.043548in}}%
\pgfusepath{stroke}%
\end{pgfscope}%
\begin{pgfscope}%
\pgfpathrectangle{\pgfqpoint{3.985294in}{4.155455in}}{\pgfqpoint{2.279412in}{2.004545in}}%
\pgfusepath{clip}%
\pgfsetbuttcap%
\pgfsetroundjoin%
\pgfsetlinewidth{1.023449pt}%
\definecolor{currentstroke}{rgb}{0.159194,0.482237,0.558073}%
\pgfsetstrokecolor{currentstroke}%
\pgfsetdash{}{0pt}%
\pgfpathmoveto{\pgfqpoint{5.295448in}{5.043548in}}%
\pgfpathlineto{\pgfqpoint{5.245429in}{5.046726in}}%
\pgfusepath{stroke}%
\end{pgfscope}%
\begin{pgfscope}%
\pgfpathrectangle{\pgfqpoint{3.985294in}{4.155455in}}{\pgfqpoint{2.279412in}{2.004545in}}%
\pgfusepath{clip}%
\pgfsetbuttcap%
\pgfsetroundjoin%
\pgfsetlinewidth{1.070889pt}%
\definecolor{currentstroke}{rgb}{0.149039,0.508051,0.557250}%
\pgfsetstrokecolor{currentstroke}%
\pgfsetdash{}{0pt}%
\pgfpathmoveto{\pgfqpoint{5.245429in}{5.046726in}}%
\pgfpathlineto{\pgfqpoint{5.195455in}{5.050389in}}%
\pgfusepath{stroke}%
\end{pgfscope}%
\begin{pgfscope}%
\pgfpathrectangle{\pgfqpoint{3.985294in}{4.155455in}}{\pgfqpoint{2.279412in}{2.004545in}}%
\pgfusepath{clip}%
\pgfsetbuttcap%
\pgfsetroundjoin%
\pgfsetlinewidth{1.146455pt}%
\definecolor{currentstroke}{rgb}{0.133743,0.548535,0.553541}%
\pgfsetstrokecolor{currentstroke}%
\pgfsetdash{}{0pt}%
\pgfpathmoveto{\pgfqpoint{5.195455in}{5.050389in}}%
\pgfpathlineto{\pgfqpoint{5.145544in}{5.054646in}}%
\pgfusepath{stroke}%
\end{pgfscope}%
\begin{pgfscope}%
\pgfpathrectangle{\pgfqpoint{3.985294in}{4.155455in}}{\pgfqpoint{2.279412in}{2.004545in}}%
\pgfusepath{clip}%
\pgfsetbuttcap%
\pgfsetroundjoin%
\pgfsetlinewidth{1.222406pt}%
\definecolor{currentstroke}{rgb}{0.121831,0.589055,0.545623}%
\pgfsetstrokecolor{currentstroke}%
\pgfsetdash{}{0pt}%
\pgfpathmoveto{\pgfqpoint{5.145544in}{5.054646in}}%
\pgfpathlineto{\pgfqpoint{5.095722in}{5.059660in}}%
\pgfusepath{stroke}%
\end{pgfscope}%
\begin{pgfscope}%
\pgfpathrectangle{\pgfqpoint{3.985294in}{4.155455in}}{\pgfqpoint{2.279412in}{2.004545in}}%
\pgfusepath{clip}%
\pgfsetbuttcap%
\pgfsetroundjoin%
\pgfsetlinewidth{1.176525pt}%
\definecolor{currentstroke}{rgb}{0.128729,0.563265,0.551229}%
\pgfsetstrokecolor{currentstroke}%
\pgfsetdash{}{0pt}%
\pgfpathmoveto{\pgfqpoint{5.095722in}{5.059660in}}%
\pgfpathlineto{\pgfqpoint{5.046040in}{5.065624in}}%
\pgfusepath{stroke}%
\end{pgfscope}%
\begin{pgfscope}%
\pgfpathrectangle{\pgfqpoint{3.985294in}{4.155455in}}{\pgfqpoint{2.279412in}{2.004545in}}%
\pgfusepath{clip}%
\pgfsetbuttcap%
\pgfsetroundjoin%
\pgfsetlinewidth{1.284555pt}%
\definecolor{currentstroke}{rgb}{0.120081,0.622161,0.534946}%
\pgfsetstrokecolor{currentstroke}%
\pgfsetdash{}{0pt}%
\pgfpathmoveto{\pgfqpoint{5.046040in}{5.065624in}}%
\pgfpathlineto{\pgfqpoint{4.996559in}{5.072723in}}%
\pgfusepath{stroke}%
\end{pgfscope}%
\begin{pgfscope}%
\pgfpathrectangle{\pgfqpoint{3.985294in}{4.155455in}}{\pgfqpoint{2.279412in}{2.004545in}}%
\pgfusepath{clip}%
\pgfsetbuttcap%
\pgfsetroundjoin%
\pgfsetlinewidth{1.255930pt}%
\definecolor{currentstroke}{rgb}{0.119512,0.607464,0.540218}%
\pgfsetstrokecolor{currentstroke}%
\pgfsetdash{}{0pt}%
\pgfpathmoveto{\pgfqpoint{4.996559in}{5.072723in}}%
\pgfpathlineto{\pgfqpoint{4.947372in}{5.081230in}}%
\pgfusepath{stroke}%
\end{pgfscope}%
\begin{pgfscope}%
\pgfpathrectangle{\pgfqpoint{3.985294in}{4.155455in}}{\pgfqpoint{2.279412in}{2.004545in}}%
\pgfusepath{clip}%
\pgfsetbuttcap%
\pgfsetroundjoin%
\pgfsetlinewidth{1.313789pt}%
\definecolor{currentstroke}{rgb}{0.123444,0.636809,0.528763}%
\pgfsetstrokecolor{currentstroke}%
\pgfsetdash{}{0pt}%
\pgfpathmoveto{\pgfqpoint{4.947372in}{5.081230in}}%
\pgfpathlineto{\pgfqpoint{4.898560in}{5.091211in}}%
\pgfusepath{stroke}%
\end{pgfscope}%
\begin{pgfscope}%
\pgfpathrectangle{\pgfqpoint{3.985294in}{4.155455in}}{\pgfqpoint{2.279412in}{2.004545in}}%
\pgfusepath{clip}%
\pgfsetbuttcap%
\pgfsetroundjoin%
\pgfsetlinewidth{1.353322pt}%
\definecolor{currentstroke}{rgb}{0.134692,0.658636,0.517649}%
\pgfsetstrokecolor{currentstroke}%
\pgfsetdash{}{0pt}%
\pgfpathmoveto{\pgfqpoint{4.898560in}{5.091211in}}%
\pgfpathlineto{\pgfqpoint{4.850240in}{5.102772in}}%
\pgfusepath{stroke}%
\end{pgfscope}%
\begin{pgfscope}%
\pgfpathrectangle{\pgfqpoint{3.985294in}{4.155455in}}{\pgfqpoint{2.279412in}{2.004545in}}%
\pgfusepath{clip}%
\pgfsetbuttcap%
\pgfsetroundjoin%
\pgfsetlinewidth{1.177334pt}%
\definecolor{currentstroke}{rgb}{0.128729,0.563265,0.551229}%
\pgfsetstrokecolor{currentstroke}%
\pgfsetdash{}{0pt}%
\pgfpathmoveto{\pgfqpoint{4.850240in}{5.102772in}}%
\pgfpathlineto{\pgfqpoint{4.802113in}{5.114742in}}%
\pgfusepath{stroke}%
\end{pgfscope}%
\begin{pgfscope}%
\pgfpathrectangle{\pgfqpoint{3.985294in}{4.155455in}}{\pgfqpoint{2.279412in}{2.004545in}}%
\pgfusepath{clip}%
\pgfsetbuttcap%
\pgfsetroundjoin%
\pgfsetlinewidth{1.053311pt}%
\definecolor{currentstroke}{rgb}{0.153364,0.497000,0.557724}%
\pgfsetstrokecolor{currentstroke}%
\pgfsetdash{}{0pt}%
\pgfpathmoveto{\pgfqpoint{4.802113in}{5.114742in}}%
\pgfpathlineto{\pgfqpoint{4.754363in}{5.127107in}}%
\pgfusepath{stroke}%
\end{pgfscope}%
\begin{pgfscope}%
\pgfpathrectangle{\pgfqpoint{3.985294in}{4.155455in}}{\pgfqpoint{2.279412in}{2.004545in}}%
\pgfusepath{clip}%
\pgfsetbuttcap%
\pgfsetroundjoin%
\pgfsetlinewidth{0.892051pt}%
\definecolor{currentstroke}{rgb}{0.188923,0.410910,0.556326}%
\pgfsetstrokecolor{currentstroke}%
\pgfsetdash{}{0pt}%
\pgfpathmoveto{\pgfqpoint{4.754363in}{5.127107in}}%
\pgfpathlineto{\pgfqpoint{4.707121in}{5.139953in}}%
\pgfusepath{stroke}%
\end{pgfscope}%
\begin{pgfscope}%
\pgfpathrectangle{\pgfqpoint{3.985294in}{4.155455in}}{\pgfqpoint{2.279412in}{2.004545in}}%
\pgfusepath{clip}%
\pgfsetbuttcap%
\pgfsetroundjoin%
\pgfsetlinewidth{0.593420pt}%
\definecolor{currentstroke}{rgb}{0.267968,0.223549,0.512008}%
\pgfsetstrokecolor{currentstroke}%
\pgfsetdash{}{0pt}%
\pgfpathmoveto{\pgfqpoint{4.707121in}{5.139953in}}%
\pgfpathlineto{\pgfqpoint{4.662349in}{5.150514in}}%
\pgfusepath{stroke}%
\end{pgfscope}%
\begin{pgfscope}%
\pgfpathrectangle{\pgfqpoint{3.985294in}{4.155455in}}{\pgfqpoint{2.279412in}{2.004545in}}%
\pgfusepath{clip}%
\pgfsetbuttcap%
\pgfsetroundjoin%
\pgfsetlinewidth{0.570283pt}%
\definecolor{currentstroke}{rgb}{0.271828,0.209303,0.504434}%
\pgfsetstrokecolor{currentstroke}%
\pgfsetdash{}{0pt}%
\pgfpathmoveto{\pgfqpoint{4.662349in}{5.150514in}}%
\pgfpathlineto{\pgfqpoint{4.662349in}{5.150514in}}%
\pgfusepath{stroke}%
\end{pgfscope}%
\begin{pgfscope}%
\pgfpathrectangle{\pgfqpoint{3.985294in}{4.155455in}}{\pgfqpoint{2.279412in}{2.004545in}}%
\pgfusepath{clip}%
\pgfsetbuttcap%
\pgfsetroundjoin%
\pgfsetlinewidth{0.570283pt}%
\definecolor{currentstroke}{rgb}{0.271828,0.209303,0.504434}%
\pgfsetstrokecolor{currentstroke}%
\pgfsetdash{}{0pt}%
\pgfpathmoveto{\pgfqpoint{4.662349in}{5.150514in}}%
\pgfpathlineto{\pgfqpoint{4.654512in}{5.151881in}}%
\pgfusepath{stroke}%
\end{pgfscope}%
\begin{pgfscope}%
\pgfpathrectangle{\pgfqpoint{3.985294in}{4.155455in}}{\pgfqpoint{2.279412in}{2.004545in}}%
\pgfusepath{clip}%
\pgfsetbuttcap%
\pgfsetroundjoin%
\pgfsetlinewidth{0.541384pt}%
\definecolor{currentstroke}{rgb}{0.276194,0.190074,0.493001}%
\pgfsetstrokecolor{currentstroke}%
\pgfsetdash{}{0pt}%
\pgfpathmoveto{\pgfqpoint{4.654512in}{5.151881in}}%
\pgfpathlineto{\pgfqpoint{4.654512in}{5.151881in}}%
\pgfusepath{stroke}%
\end{pgfscope}%
\begin{pgfscope}%
\pgfpathrectangle{\pgfqpoint{3.985294in}{4.155455in}}{\pgfqpoint{2.279412in}{2.004545in}}%
\pgfusepath{clip}%
\pgfsetbuttcap%
\pgfsetroundjoin%
\pgfsetlinewidth{0.541384pt}%
\definecolor{currentstroke}{rgb}{0.276194,0.190074,0.493001}%
\pgfsetstrokecolor{currentstroke}%
\pgfsetdash{}{0pt}%
\pgfpathmoveto{\pgfqpoint{4.654512in}{5.151881in}}%
\pgfpathlineto{\pgfqpoint{4.651144in}{5.152659in}}%
\pgfusepath{stroke}%
\end{pgfscope}%
\begin{pgfscope}%
\pgfpathrectangle{\pgfqpoint{3.985294in}{4.155455in}}{\pgfqpoint{2.279412in}{2.004545in}}%
\pgfusepath{clip}%
\pgfsetbuttcap%
\pgfsetroundjoin%
\pgfsetlinewidth{0.524874pt}%
\definecolor{currentstroke}{rgb}{0.278826,0.175490,0.483397}%
\pgfsetstrokecolor{currentstroke}%
\pgfsetdash{}{0pt}%
\pgfpathmoveto{\pgfqpoint{4.651144in}{5.152659in}}%
\pgfpathlineto{\pgfqpoint{4.649490in}{5.152968in}}%
\pgfusepath{stroke}%
\end{pgfscope}%
\begin{pgfscope}%
\pgfpathrectangle{\pgfqpoint{3.985294in}{4.155455in}}{\pgfqpoint{2.279412in}{2.004545in}}%
\pgfusepath{clip}%
\pgfsetbuttcap%
\pgfsetroundjoin%
\pgfsetlinewidth{0.526226pt}%
\definecolor{currentstroke}{rgb}{0.278826,0.175490,0.483397}%
\pgfsetstrokecolor{currentstroke}%
\pgfsetdash{}{0pt}%
\pgfpathmoveto{\pgfqpoint{4.649490in}{5.152968in}}%
\pgfpathlineto{\pgfqpoint{4.648531in}{5.153173in}}%
\pgfusepath{stroke}%
\end{pgfscope}%
\begin{pgfscope}%
\pgfpathrectangle{\pgfqpoint{3.985294in}{4.155455in}}{\pgfqpoint{2.279412in}{2.004545in}}%
\pgfusepath{clip}%
\pgfsetbuttcap%
\pgfsetroundjoin%
\pgfsetlinewidth{0.527379pt}%
\definecolor{currentstroke}{rgb}{0.278826,0.175490,0.483397}%
\pgfsetstrokecolor{currentstroke}%
\pgfsetdash{}{0pt}%
\pgfpathmoveto{\pgfqpoint{4.648531in}{5.153173in}}%
\pgfpathlineto{\pgfqpoint{4.648352in}{5.153177in}}%
\pgfusepath{stroke}%
\end{pgfscope}%
\begin{pgfscope}%
\pgfpathrectangle{\pgfqpoint{3.985294in}{4.155455in}}{\pgfqpoint{2.279412in}{2.004545in}}%
\pgfusepath{clip}%
\pgfsetbuttcap%
\pgfsetroundjoin%
\pgfsetlinewidth{0.527956pt}%
\definecolor{currentstroke}{rgb}{0.278012,0.180367,0.486697}%
\pgfsetstrokecolor{currentstroke}%
\pgfsetdash{}{0pt}%
\pgfpathmoveto{\pgfqpoint{4.648352in}{5.153177in}}%
\pgfpathlineto{\pgfqpoint{4.649119in}{5.152914in}}%
\pgfusepath{stroke}%
\end{pgfscope}%
\begin{pgfscope}%
\pgfpathrectangle{\pgfqpoint{3.985294in}{4.155455in}}{\pgfqpoint{2.279412in}{2.004545in}}%
\pgfusepath{clip}%
\pgfsetbuttcap%
\pgfsetroundjoin%
\pgfsetlinewidth{0.527998pt}%
\definecolor{currentstroke}{rgb}{0.278012,0.180367,0.486697}%
\pgfsetstrokecolor{currentstroke}%
\pgfsetdash{}{0pt}%
\pgfpathmoveto{\pgfqpoint{4.649119in}{5.152914in}}%
\pgfpathlineto{\pgfqpoint{4.649119in}{5.152914in}}%
\pgfusepath{stroke}%
\end{pgfscope}%
\begin{pgfscope}%
\pgfpathrectangle{\pgfqpoint{3.985294in}{4.155455in}}{\pgfqpoint{2.279412in}{2.004545in}}%
\pgfusepath{clip}%
\pgfsetbuttcap%
\pgfsetroundjoin%
\pgfsetlinewidth{0.527998pt}%
\definecolor{currentstroke}{rgb}{0.278012,0.180367,0.486697}%
\pgfsetstrokecolor{currentstroke}%
\pgfsetdash{}{0pt}%
\pgfpathmoveto{\pgfqpoint{4.649119in}{5.152914in}}%
\pgfpathlineto{\pgfqpoint{4.649443in}{5.152794in}}%
\pgfusepath{stroke}%
\end{pgfscope}%
\begin{pgfscope}%
\pgfpathrectangle{\pgfqpoint{3.985294in}{4.155455in}}{\pgfqpoint{2.279412in}{2.004545in}}%
\pgfusepath{clip}%
\pgfsetbuttcap%
\pgfsetroundjoin%
\pgfsetlinewidth{0.528210pt}%
\definecolor{currentstroke}{rgb}{0.278012,0.180367,0.486697}%
\pgfsetstrokecolor{currentstroke}%
\pgfsetdash{}{0pt}%
\pgfpathmoveto{\pgfqpoint{4.649443in}{5.152794in}}%
\pgfpathlineto{\pgfqpoint{4.649619in}{5.152726in}}%
\pgfusepath{stroke}%
\end{pgfscope}%
\begin{pgfscope}%
\pgfpathrectangle{\pgfqpoint{3.985294in}{4.155455in}}{\pgfqpoint{2.279412in}{2.004545in}}%
\pgfusepath{clip}%
\pgfsetbuttcap%
\pgfsetroundjoin%
\pgfsetlinewidth{0.528375pt}%
\definecolor{currentstroke}{rgb}{0.278012,0.180367,0.486697}%
\pgfsetstrokecolor{currentstroke}%
\pgfsetdash{}{0pt}%
\pgfpathmoveto{\pgfqpoint{4.649619in}{5.152726in}}%
\pgfpathlineto{\pgfqpoint{4.649421in}{5.152782in}}%
\pgfusepath{stroke}%
\end{pgfscope}%
\begin{pgfscope}%
\pgfpathrectangle{\pgfqpoint{3.985294in}{4.155455in}}{\pgfqpoint{2.279412in}{2.004545in}}%
\pgfusepath{clip}%
\pgfsetbuttcap%
\pgfsetroundjoin%
\pgfsetlinewidth{0.528402pt}%
\definecolor{currentstroke}{rgb}{0.278012,0.180367,0.486697}%
\pgfsetstrokecolor{currentstroke}%
\pgfsetdash{}{0pt}%
\pgfpathmoveto{\pgfqpoint{4.649421in}{5.152782in}}%
\pgfpathlineto{\pgfqpoint{4.649089in}{5.152886in}}%
\pgfusepath{stroke}%
\end{pgfscope}%
\begin{pgfscope}%
\pgfpathrectangle{\pgfqpoint{3.985294in}{4.155455in}}{\pgfqpoint{2.279412in}{2.004545in}}%
\pgfusepath{clip}%
\pgfsetbuttcap%
\pgfsetroundjoin%
\pgfsetlinewidth{0.528385pt}%
\definecolor{currentstroke}{rgb}{0.278012,0.180367,0.486697}%
\pgfsetstrokecolor{currentstroke}%
\pgfsetdash{}{0pt}%
\pgfpathmoveto{\pgfqpoint{4.649089in}{5.152886in}}%
\pgfpathlineto{\pgfqpoint{4.648935in}{5.152936in}}%
\pgfusepath{stroke}%
\end{pgfscope}%
\begin{pgfscope}%
\pgfpathrectangle{\pgfqpoint{3.985294in}{4.155455in}}{\pgfqpoint{2.279412in}{2.004545in}}%
\pgfusepath{clip}%
\pgfsetbuttcap%
\pgfsetroundjoin%
\pgfsetlinewidth{0.528376pt}%
\definecolor{currentstroke}{rgb}{0.278012,0.180367,0.486697}%
\pgfsetstrokecolor{currentstroke}%
\pgfsetdash{}{0pt}%
\pgfpathmoveto{\pgfqpoint{4.648935in}{5.152936in}}%
\pgfpathlineto{\pgfqpoint{4.649088in}{5.152889in}}%
\pgfusepath{stroke}%
\end{pgfscope}%
\begin{pgfscope}%
\pgfpathrectangle{\pgfqpoint{3.985294in}{4.155455in}}{\pgfqpoint{2.279412in}{2.004545in}}%
\pgfusepath{clip}%
\pgfsetbuttcap%
\pgfsetroundjoin%
\pgfsetlinewidth{0.528354pt}%
\definecolor{currentstroke}{rgb}{0.278012,0.180367,0.486697}%
\pgfsetstrokecolor{currentstroke}%
\pgfsetdash{}{0pt}%
\pgfpathmoveto{\pgfqpoint{4.649088in}{5.152889in}}%
\pgfpathlineto{\pgfqpoint{4.649459in}{5.152771in}}%
\pgfusepath{stroke}%
\end{pgfscope}%
\begin{pgfscope}%
\pgfpathrectangle{\pgfqpoint{3.985294in}{4.155455in}}{\pgfqpoint{2.279412in}{2.004545in}}%
\pgfusepath{clip}%
\pgfsetbuttcap%
\pgfsetroundjoin%
\pgfsetlinewidth{0.528394pt}%
\definecolor{currentstroke}{rgb}{0.278012,0.180367,0.486697}%
\pgfsetstrokecolor{currentstroke}%
\pgfsetdash{}{0pt}%
\pgfpathmoveto{\pgfqpoint{4.649459in}{5.152771in}}%
\pgfpathlineto{\pgfqpoint{4.649662in}{5.152703in}}%
\pgfusepath{stroke}%
\end{pgfscope}%
\begin{pgfscope}%
\pgfpathrectangle{\pgfqpoint{3.985294in}{4.155455in}}{\pgfqpoint{2.279412in}{2.004545in}}%
\pgfusepath{clip}%
\pgfsetbuttcap%
\pgfsetroundjoin%
\pgfsetlinewidth{0.528481pt}%
\definecolor{currentstroke}{rgb}{0.278012,0.180367,0.486697}%
\pgfsetstrokecolor{currentstroke}%
\pgfsetdash{}{0pt}%
\pgfpathmoveto{\pgfqpoint{4.649662in}{5.152703in}}%
\pgfpathlineto{\pgfqpoint{4.649441in}{5.152771in}}%
\pgfusepath{stroke}%
\end{pgfscope}%
\begin{pgfscope}%
\pgfpathrectangle{\pgfqpoint{3.985294in}{4.155455in}}{\pgfqpoint{2.279412in}{2.004545in}}%
\pgfusepath{clip}%
\pgfsetbuttcap%
\pgfsetroundjoin%
\pgfsetlinewidth{0.528455pt}%
\definecolor{currentstroke}{rgb}{0.278012,0.180367,0.486697}%
\pgfsetstrokecolor{currentstroke}%
\pgfsetdash{}{0pt}%
\pgfpathmoveto{\pgfqpoint{4.649441in}{5.152771in}}%
\pgfpathlineto{\pgfqpoint{4.649073in}{5.152888in}}%
\pgfusepath{stroke}%
\end{pgfscope}%
\begin{pgfscope}%
\pgfpathrectangle{\pgfqpoint{3.985294in}{4.155455in}}{\pgfqpoint{2.279412in}{2.004545in}}%
\pgfusepath{clip}%
\pgfsetbuttcap%
\pgfsetroundjoin%
\pgfsetlinewidth{0.528410pt}%
\definecolor{currentstroke}{rgb}{0.278012,0.180367,0.486697}%
\pgfsetstrokecolor{currentstroke}%
\pgfsetdash{}{0pt}%
\pgfpathmoveto{\pgfqpoint{4.649073in}{5.152888in}}%
\pgfpathlineto{\pgfqpoint{4.648902in}{5.152945in}}%
\pgfusepath{stroke}%
\end{pgfscope}%
\begin{pgfscope}%
\pgfpathrectangle{\pgfqpoint{3.985294in}{4.155455in}}{\pgfqpoint{2.279412in}{2.004545in}}%
\pgfusepath{clip}%
\pgfsetbuttcap%
\pgfsetroundjoin%
\pgfsetlinewidth{0.528390pt}%
\definecolor{currentstroke}{rgb}{0.278012,0.180367,0.486697}%
\pgfsetstrokecolor{currentstroke}%
\pgfsetdash{}{0pt}%
\pgfpathmoveto{\pgfqpoint{4.648902in}{5.152945in}}%
\pgfpathlineto{\pgfqpoint{4.649069in}{5.152895in}}%
\pgfusepath{stroke}%
\end{pgfscope}%
\begin{pgfscope}%
\pgfpathrectangle{\pgfqpoint{3.985294in}{4.155455in}}{\pgfqpoint{2.279412in}{2.004545in}}%
\pgfusepath{clip}%
\pgfsetbuttcap%
\pgfsetroundjoin%
\pgfsetlinewidth{0.528358pt}%
\definecolor{currentstroke}{rgb}{0.278012,0.180367,0.486697}%
\pgfsetstrokecolor{currentstroke}%
\pgfsetdash{}{0pt}%
\pgfpathmoveto{\pgfqpoint{4.649069in}{5.152895in}}%
\pgfpathlineto{\pgfqpoint{4.649482in}{5.152764in}}%
\pgfusepath{stroke}%
\end{pgfscope}%
\begin{pgfscope}%
\pgfpathrectangle{\pgfqpoint{3.985294in}{4.155455in}}{\pgfqpoint{2.279412in}{2.004545in}}%
\pgfusepath{clip}%
\pgfsetbuttcap%
\pgfsetroundjoin%
\pgfsetlinewidth{0.528399pt}%
\definecolor{currentstroke}{rgb}{0.278012,0.180367,0.486697}%
\pgfsetstrokecolor{currentstroke}%
\pgfsetdash{}{0pt}%
\pgfpathmoveto{\pgfqpoint{4.649482in}{5.152764in}}%
\pgfpathlineto{\pgfqpoint{4.649712in}{5.152687in}}%
\pgfusepath{stroke}%
\end{pgfscope}%
\begin{pgfscope}%
\pgfpathrectangle{\pgfqpoint{3.985294in}{4.155455in}}{\pgfqpoint{2.279412in}{2.004545in}}%
\pgfusepath{clip}%
\pgfsetbuttcap%
\pgfsetroundjoin%
\pgfsetlinewidth{0.528501pt}%
\definecolor{currentstroke}{rgb}{0.278012,0.180367,0.486697}%
\pgfsetstrokecolor{currentstroke}%
\pgfsetdash{}{0pt}%
\pgfpathmoveto{\pgfqpoint{4.649712in}{5.152687in}}%
\pgfpathlineto{\pgfqpoint{4.649464in}{5.152763in}}%
\pgfusepath{stroke}%
\end{pgfscope}%
\begin{pgfscope}%
\pgfpathrectangle{\pgfqpoint{3.985294in}{4.155455in}}{\pgfqpoint{2.279412in}{2.004545in}}%
\pgfusepath{clip}%
\pgfsetbuttcap%
\pgfsetroundjoin%
\pgfsetlinewidth{0.528466pt}%
\definecolor{currentstroke}{rgb}{0.278012,0.180367,0.486697}%
\pgfsetstrokecolor{currentstroke}%
\pgfsetdash{}{0pt}%
\pgfpathmoveto{\pgfqpoint{4.649464in}{5.152763in}}%
\pgfpathlineto{\pgfqpoint{4.649057in}{5.152893in}}%
\pgfusepath{stroke}%
\end{pgfscope}%
\begin{pgfscope}%
\pgfpathrectangle{\pgfqpoint{3.985294in}{4.155455in}}{\pgfqpoint{2.279412in}{2.004545in}}%
\pgfusepath{clip}%
\pgfsetbuttcap%
\pgfsetroundjoin%
\pgfsetlinewidth{0.528415pt}%
\definecolor{currentstroke}{rgb}{0.278012,0.180367,0.486697}%
\pgfsetstrokecolor{currentstroke}%
\pgfsetdash{}{0pt}%
\pgfpathmoveto{\pgfqpoint{4.649057in}{5.152893in}}%
\pgfpathlineto{\pgfqpoint{4.648869in}{5.152955in}}%
\pgfusepath{stroke}%
\end{pgfscope}%
\begin{pgfscope}%
\pgfpathrectangle{\pgfqpoint{3.985294in}{4.155455in}}{\pgfqpoint{2.279412in}{2.004545in}}%
\pgfusepath{clip}%
\pgfsetbuttcap%
\pgfsetroundjoin%
\pgfsetlinewidth{0.528395pt}%
\definecolor{currentstroke}{rgb}{0.278012,0.180367,0.486697}%
\pgfsetstrokecolor{currentstroke}%
\pgfsetdash{}{0pt}%
\pgfpathmoveto{\pgfqpoint{4.648869in}{5.152955in}}%
\pgfpathlineto{\pgfqpoint{4.649048in}{5.152901in}}%
\pgfusepath{stroke}%
\end{pgfscope}%
\begin{pgfscope}%
\pgfpathrectangle{\pgfqpoint{3.985294in}{4.155455in}}{\pgfqpoint{2.279412in}{2.004545in}}%
\pgfusepath{clip}%
\pgfsetbuttcap%
\pgfsetroundjoin%
\pgfsetlinewidth{0.528357pt}%
\definecolor{currentstroke}{rgb}{0.278012,0.180367,0.486697}%
\pgfsetstrokecolor{currentstroke}%
\pgfsetdash{}{0pt}%
\pgfpathmoveto{\pgfqpoint{4.649048in}{5.152901in}}%
\pgfpathlineto{\pgfqpoint{4.649507in}{5.152756in}}%
\pgfusepath{stroke}%
\end{pgfscope}%
\begin{pgfscope}%
\pgfpathrectangle{\pgfqpoint{3.985294in}{4.155455in}}{\pgfqpoint{2.279412in}{2.004545in}}%
\pgfusepath{clip}%
\pgfsetbuttcap%
\pgfsetroundjoin%
\pgfsetlinewidth{0.528402pt}%
\definecolor{currentstroke}{rgb}{0.278012,0.180367,0.486697}%
\pgfsetstrokecolor{currentstroke}%
\pgfsetdash{}{0pt}%
\pgfpathmoveto{\pgfqpoint{4.649507in}{5.152756in}}%
\pgfpathlineto{\pgfqpoint{4.649767in}{5.152669in}}%
\pgfusepath{stroke}%
\end{pgfscope}%
\begin{pgfscope}%
\pgfpathrectangle{\pgfqpoint{3.985294in}{4.155455in}}{\pgfqpoint{2.279412in}{2.004545in}}%
\pgfusepath{clip}%
\pgfsetbuttcap%
\pgfsetroundjoin%
\pgfsetlinewidth{0.528523pt}%
\definecolor{currentstroke}{rgb}{0.278012,0.180367,0.486697}%
\pgfsetstrokecolor{currentstroke}%
\pgfsetdash{}{0pt}%
\pgfpathmoveto{\pgfqpoint{4.649767in}{5.152669in}}%
\pgfpathlineto{\pgfqpoint{4.649489in}{5.152754in}}%
\pgfusepath{stroke}%
\end{pgfscope}%
\begin{pgfscope}%
\pgfpathrectangle{\pgfqpoint{3.985294in}{4.155455in}}{\pgfqpoint{2.279412in}{2.004545in}}%
\pgfusepath{clip}%
\pgfsetbuttcap%
\pgfsetroundjoin%
\pgfsetlinewidth{0.528479pt}%
\definecolor{currentstroke}{rgb}{0.278012,0.180367,0.486697}%
\pgfsetstrokecolor{currentstroke}%
\pgfsetdash{}{0pt}%
\pgfpathmoveto{\pgfqpoint{4.649489in}{5.152754in}}%
\pgfpathlineto{\pgfqpoint{4.649041in}{5.152897in}}%
\pgfusepath{stroke}%
\end{pgfscope}%
\begin{pgfscope}%
\pgfpathrectangle{\pgfqpoint{3.985294in}{4.155455in}}{\pgfqpoint{2.279412in}{2.004545in}}%
\pgfusepath{clip}%
\pgfsetbuttcap%
\pgfsetroundjoin%
\pgfsetlinewidth{0.528420pt}%
\definecolor{currentstroke}{rgb}{0.278012,0.180367,0.486697}%
\pgfsetstrokecolor{currentstroke}%
\pgfsetdash{}{0pt}%
\pgfpathmoveto{\pgfqpoint{4.649041in}{5.152897in}}%
\pgfpathlineto{\pgfqpoint{4.648835in}{5.152966in}}%
\pgfusepath{stroke}%
\end{pgfscope}%
\begin{pgfscope}%
\pgfpathrectangle{\pgfqpoint{3.985294in}{4.155455in}}{\pgfqpoint{2.279412in}{2.004545in}}%
\pgfusepath{clip}%
\pgfsetbuttcap%
\pgfsetroundjoin%
\pgfsetlinewidth{0.528401pt}%
\definecolor{currentstroke}{rgb}{0.278012,0.180367,0.486697}%
\pgfsetstrokecolor{currentstroke}%
\pgfsetdash{}{0pt}%
\pgfpathmoveto{\pgfqpoint{4.648835in}{5.152966in}}%
\pgfpathlineto{\pgfqpoint{4.649026in}{5.152909in}}%
\pgfusepath{stroke}%
\end{pgfscope}%
\begin{pgfscope}%
\pgfpathrectangle{\pgfqpoint{3.985294in}{4.155455in}}{\pgfqpoint{2.279412in}{2.004545in}}%
\pgfusepath{clip}%
\pgfsetbuttcap%
\pgfsetroundjoin%
\pgfsetlinewidth{0.528356pt}%
\definecolor{currentstroke}{rgb}{0.278012,0.180367,0.486697}%
\pgfsetstrokecolor{currentstroke}%
\pgfsetdash{}{0pt}%
\pgfpathmoveto{\pgfqpoint{4.649026in}{5.152909in}}%
\pgfpathlineto{\pgfqpoint{4.649532in}{5.152748in}}%
\pgfusepath{stroke}%
\end{pgfscope}%
\begin{pgfscope}%
\pgfpathrectangle{\pgfqpoint{3.985294in}{4.155455in}}{\pgfqpoint{2.279412in}{2.004545in}}%
\pgfusepath{clip}%
\pgfsetbuttcap%
\pgfsetroundjoin%
\pgfsetlinewidth{0.528406pt}%
\definecolor{currentstroke}{rgb}{0.278012,0.180367,0.486697}%
\pgfsetstrokecolor{currentstroke}%
\pgfsetdash{}{0pt}%
\pgfpathmoveto{\pgfqpoint{4.649532in}{5.152748in}}%
\pgfpathlineto{\pgfqpoint{4.649826in}{5.152650in}}%
\pgfusepath{stroke}%
\end{pgfscope}%
\begin{pgfscope}%
\pgfpathrectangle{\pgfqpoint{3.985294in}{4.155455in}}{\pgfqpoint{2.279412in}{2.004545in}}%
\pgfusepath{clip}%
\pgfsetbuttcap%
\pgfsetroundjoin%
\pgfsetlinewidth{0.528549pt}%
\definecolor{currentstroke}{rgb}{0.278012,0.180367,0.486697}%
\pgfsetstrokecolor{currentstroke}%
\pgfsetdash{}{0pt}%
\pgfpathmoveto{\pgfqpoint{4.649826in}{5.152650in}}%
\pgfpathlineto{\pgfqpoint{4.649518in}{5.152744in}}%
\pgfusepath{stroke}%
\end{pgfscope}%
\begin{pgfscope}%
\pgfpathrectangle{\pgfqpoint{3.985294in}{4.155455in}}{\pgfqpoint{2.279412in}{2.004545in}}%
\pgfusepath{clip}%
\pgfsetbuttcap%
\pgfsetroundjoin%
\pgfsetlinewidth{0.528493pt}%
\definecolor{currentstroke}{rgb}{0.278012,0.180367,0.486697}%
\pgfsetstrokecolor{currentstroke}%
\pgfsetdash{}{0pt}%
\pgfpathmoveto{\pgfqpoint{4.649518in}{5.152744in}}%
\pgfpathlineto{\pgfqpoint{4.649026in}{5.152902in}}%
\pgfusepath{stroke}%
\end{pgfscope}%
\begin{pgfscope}%
\pgfpathrectangle{\pgfqpoint{3.985294in}{4.155455in}}{\pgfqpoint{2.279412in}{2.004545in}}%
\pgfusepath{clip}%
\pgfsetbuttcap%
\pgfsetroundjoin%
\pgfsetlinewidth{0.528426pt}%
\definecolor{currentstroke}{rgb}{0.278012,0.180367,0.486697}%
\pgfsetstrokecolor{currentstroke}%
\pgfsetdash{}{0pt}%
\pgfpathmoveto{\pgfqpoint{4.649026in}{5.152902in}}%
\pgfpathlineto{\pgfqpoint{4.648800in}{5.152977in}}%
\pgfusepath{stroke}%
\end{pgfscope}%
\begin{pgfscope}%
\pgfpathrectangle{\pgfqpoint{3.985294in}{4.155455in}}{\pgfqpoint{2.279412in}{2.004545in}}%
\pgfusepath{clip}%
\pgfsetbuttcap%
\pgfsetroundjoin%
\pgfsetlinewidth{0.528408pt}%
\definecolor{currentstroke}{rgb}{0.278012,0.180367,0.486697}%
\pgfsetstrokecolor{currentstroke}%
\pgfsetdash{}{0pt}%
\pgfpathmoveto{\pgfqpoint{4.648800in}{5.152977in}}%
\pgfpathlineto{\pgfqpoint{4.649002in}{5.152916in}}%
\pgfusepath{stroke}%
\end{pgfscope}%
\begin{pgfscope}%
\pgfpathrectangle{\pgfqpoint{3.985294in}{4.155455in}}{\pgfqpoint{2.279412in}{2.004545in}}%
\pgfusepath{clip}%
\pgfsetbuttcap%
\pgfsetroundjoin%
\pgfsetlinewidth{0.528357pt}%
\definecolor{currentstroke}{rgb}{0.278012,0.180367,0.486697}%
\pgfsetstrokecolor{currentstroke}%
\pgfsetdash{}{0pt}%
\pgfpathmoveto{\pgfqpoint{4.649002in}{5.152916in}}%
\pgfpathlineto{\pgfqpoint{4.649558in}{5.152740in}}%
\pgfusepath{stroke}%
\end{pgfscope}%
\begin{pgfscope}%
\pgfpathrectangle{\pgfqpoint{3.985294in}{4.155455in}}{\pgfqpoint{2.279412in}{2.004545in}}%
\pgfusepath{clip}%
\pgfsetbuttcap%
\pgfsetroundjoin%
\pgfsetlinewidth{0.528410pt}%
\definecolor{currentstroke}{rgb}{0.278012,0.180367,0.486697}%
\pgfsetstrokecolor{currentstroke}%
\pgfsetdash{}{0pt}%
\pgfpathmoveto{\pgfqpoint{4.649558in}{5.152740in}}%
\pgfpathlineto{\pgfqpoint{4.649889in}{5.152629in}}%
\pgfusepath{stroke}%
\end{pgfscope}%
\begin{pgfscope}%
\pgfpathrectangle{\pgfqpoint{3.985294in}{4.155455in}}{\pgfqpoint{2.279412in}{2.004545in}}%
\pgfusepath{clip}%
\pgfsetbuttcap%
\pgfsetroundjoin%
\pgfsetlinewidth{0.528579pt}%
\definecolor{currentstroke}{rgb}{0.278012,0.180367,0.486697}%
\pgfsetstrokecolor{currentstroke}%
\pgfsetdash{}{0pt}%
\pgfpathmoveto{\pgfqpoint{4.649889in}{5.152629in}}%
\pgfpathlineto{\pgfqpoint{4.649549in}{5.152734in}}%
\pgfusepath{stroke}%
\end{pgfscope}%
\begin{pgfscope}%
\pgfpathrectangle{\pgfqpoint{3.985294in}{4.155455in}}{\pgfqpoint{2.279412in}{2.004545in}}%
\pgfusepath{clip}%
\pgfsetbuttcap%
\pgfsetroundjoin%
\pgfsetlinewidth{0.528510pt}%
\definecolor{currentstroke}{rgb}{0.278012,0.180367,0.486697}%
\pgfsetstrokecolor{currentstroke}%
\pgfsetdash{}{0pt}%
\pgfpathmoveto{\pgfqpoint{4.649549in}{5.152734in}}%
\pgfpathlineto{\pgfqpoint{4.649012in}{5.152905in}}%
\pgfusepath{stroke}%
\end{pgfscope}%
\begin{pgfscope}%
\pgfpathrectangle{\pgfqpoint{3.985294in}{4.155455in}}{\pgfqpoint{2.279412in}{2.004545in}}%
\pgfusepath{clip}%
\pgfsetbuttcap%
\pgfsetroundjoin%
\pgfsetlinewidth{0.528433pt}%
\definecolor{currentstroke}{rgb}{0.278012,0.180367,0.486697}%
\pgfsetstrokecolor{currentstroke}%
\pgfsetdash{}{0pt}%
\pgfpathmoveto{\pgfqpoint{4.649012in}{5.152905in}}%
\pgfpathlineto{\pgfqpoint{4.648765in}{5.152988in}}%
\pgfusepath{stroke}%
\end{pgfscope}%
\begin{pgfscope}%
\pgfpathrectangle{\pgfqpoint{3.985294in}{4.155455in}}{\pgfqpoint{2.279412in}{2.004545in}}%
\pgfusepath{clip}%
\pgfsetbuttcap%
\pgfsetroundjoin%
\pgfsetlinewidth{0.528416pt}%
\definecolor{currentstroke}{rgb}{0.278012,0.180367,0.486697}%
\pgfsetstrokecolor{currentstroke}%
\pgfsetdash{}{0pt}%
\pgfpathmoveto{\pgfqpoint{4.648765in}{5.152988in}}%
\pgfpathlineto{\pgfqpoint{4.648977in}{5.152924in}}%
\pgfusepath{stroke}%
\end{pgfscope}%
\begin{pgfscope}%
\pgfpathrectangle{\pgfqpoint{3.985294in}{4.155455in}}{\pgfqpoint{2.279412in}{2.004545in}}%
\pgfusepath{clip}%
\pgfsetbuttcap%
\pgfsetroundjoin%
\pgfsetlinewidth{0.528357pt}%
\definecolor{currentstroke}{rgb}{0.278012,0.180367,0.486697}%
\pgfsetstrokecolor{currentstroke}%
\pgfsetdash{}{0pt}%
\pgfpathmoveto{\pgfqpoint{4.648977in}{5.152924in}}%
\pgfpathlineto{\pgfqpoint{4.649583in}{5.152733in}}%
\pgfusepath{stroke}%
\end{pgfscope}%
\begin{pgfscope}%
\pgfpathrectangle{\pgfqpoint{3.985294in}{4.155455in}}{\pgfqpoint{2.279412in}{2.004545in}}%
\pgfusepath{clip}%
\pgfsetbuttcap%
\pgfsetroundjoin%
\pgfsetlinewidth{0.528414pt}%
\definecolor{currentstroke}{rgb}{0.278012,0.180367,0.486697}%
\pgfsetstrokecolor{currentstroke}%
\pgfsetdash{}{0pt}%
\pgfpathmoveto{\pgfqpoint{4.649583in}{5.152733in}}%
\pgfpathlineto{\pgfqpoint{4.649583in}{5.152733in}}%
\pgfusepath{stroke}%
\end{pgfscope}%
\begin{pgfscope}%
\pgfpathrectangle{\pgfqpoint{3.985294in}{4.155455in}}{\pgfqpoint{2.279412in}{2.004545in}}%
\pgfusepath{clip}%
\pgfsetbuttcap%
\pgfsetroundjoin%
\pgfsetlinewidth{0.528414pt}%
\definecolor{currentstroke}{rgb}{0.278012,0.180367,0.486697}%
\pgfsetstrokecolor{currentstroke}%
\pgfsetdash{}{0pt}%
\pgfpathmoveto{\pgfqpoint{4.649583in}{5.152733in}}%
\pgfpathlineto{\pgfqpoint{4.649300in}{5.152819in}}%
\pgfusepath{stroke}%
\end{pgfscope}%
\begin{pgfscope}%
\pgfpathrectangle{\pgfqpoint{3.985294in}{4.155455in}}{\pgfqpoint{2.279412in}{2.004545in}}%
\pgfusepath{clip}%
\pgfsetbuttcap%
\pgfsetroundjoin%
\pgfsetlinewidth{0.528402pt}%
\definecolor{currentstroke}{rgb}{0.278012,0.180367,0.486697}%
\pgfsetstrokecolor{currentstroke}%
\pgfsetdash{}{0pt}%
\pgfpathmoveto{\pgfqpoint{4.649300in}{5.152819in}}%
\pgfpathlineto{\pgfqpoint{4.649006in}{5.152912in}}%
\pgfusepath{stroke}%
\end{pgfscope}%
\begin{pgfscope}%
\pgfpathrectangle{\pgfqpoint{3.985294in}{4.155455in}}{\pgfqpoint{2.279412in}{2.004545in}}%
\pgfusepath{clip}%
\pgfsetbuttcap%
\pgfsetroundjoin%
\pgfsetlinewidth{0.528387pt}%
\definecolor{currentstroke}{rgb}{0.278012,0.180367,0.486697}%
\pgfsetstrokecolor{currentstroke}%
\pgfsetdash{}{0pt}%
\pgfpathmoveto{\pgfqpoint{4.649006in}{5.152912in}}%
\pgfpathlineto{\pgfqpoint{4.648955in}{5.152930in}}%
\pgfusepath{stroke}%
\end{pgfscope}%
\begin{pgfscope}%
\pgfpathrectangle{\pgfqpoint{3.985294in}{4.155455in}}{\pgfqpoint{2.279412in}{2.004545in}}%
\pgfusepath{clip}%
\pgfsetbuttcap%
\pgfsetroundjoin%
\pgfsetlinewidth{0.528370pt}%
\definecolor{currentstroke}{rgb}{0.278012,0.180367,0.486697}%
\pgfsetstrokecolor{currentstroke}%
\pgfsetdash{}{0pt}%
\pgfpathmoveto{\pgfqpoint{4.648955in}{5.152930in}}%
\pgfpathlineto{\pgfqpoint{4.649208in}{5.152851in}}%
\pgfusepath{stroke}%
\end{pgfscope}%
\begin{pgfscope}%
\pgfpathrectangle{\pgfqpoint{3.985294in}{4.155455in}}{\pgfqpoint{2.279412in}{2.004545in}}%
\pgfusepath{clip}%
\pgfsetbuttcap%
\pgfsetroundjoin%
\pgfsetlinewidth{0.528356pt}%
\definecolor{currentstroke}{rgb}{0.278012,0.180367,0.486697}%
\pgfsetstrokecolor{currentstroke}%
\pgfsetdash{}{0pt}%
\pgfpathmoveto{\pgfqpoint{4.649208in}{5.152851in}}%
\pgfpathlineto{\pgfqpoint{4.649569in}{5.152735in}}%
\pgfusepath{stroke}%
\end{pgfscope}%
\begin{pgfscope}%
\pgfpathrectangle{\pgfqpoint{3.985294in}{4.155455in}}{\pgfqpoint{2.279412in}{2.004545in}}%
\pgfusepath{clip}%
\pgfsetbuttcap%
\pgfsetroundjoin%
\pgfsetlinewidth{0.528429pt}%
\definecolor{currentstroke}{rgb}{0.278012,0.180367,0.486697}%
\pgfsetstrokecolor{currentstroke}%
\pgfsetdash{}{0pt}%
\pgfpathmoveto{\pgfqpoint{4.649569in}{5.152735in}}%
\pgfpathlineto{\pgfqpoint{4.649622in}{5.152715in}}%
\pgfusepath{stroke}%
\end{pgfscope}%
\begin{pgfscope}%
\pgfpathrectangle{\pgfqpoint{3.985294in}{4.155455in}}{\pgfqpoint{2.279412in}{2.004545in}}%
\pgfusepath{clip}%
\pgfsetbuttcap%
\pgfsetroundjoin%
\pgfsetlinewidth{0.528483pt}%
\definecolor{currentstroke}{rgb}{0.278012,0.180367,0.486697}%
\pgfsetstrokecolor{currentstroke}%
\pgfsetdash{}{0pt}%
\pgfpathmoveto{\pgfqpoint{4.649622in}{5.152715in}}%
\pgfpathlineto{\pgfqpoint{4.649306in}{5.152814in}}%
\pgfusepath{stroke}%
\end{pgfscope}%
\begin{pgfscope}%
\pgfpathrectangle{\pgfqpoint{3.985294in}{4.155455in}}{\pgfqpoint{2.279412in}{2.004545in}}%
\pgfusepath{clip}%
\pgfsetbuttcap%
\pgfsetroundjoin%
\pgfsetlinewidth{0.528434pt}%
\definecolor{currentstroke}{rgb}{0.278012,0.180367,0.486697}%
\pgfsetstrokecolor{currentstroke}%
\pgfsetdash{}{0pt}%
\pgfpathmoveto{\pgfqpoint{4.649306in}{5.152814in}}%
\pgfpathlineto{\pgfqpoint{4.648982in}{5.152918in}}%
\pgfusepath{stroke}%
\end{pgfscope}%
\begin{pgfscope}%
\pgfpathrectangle{\pgfqpoint{3.985294in}{4.155455in}}{\pgfqpoint{2.279412in}{2.004545in}}%
\pgfusepath{clip}%
\pgfsetbuttcap%
\pgfsetroundjoin%
\pgfsetlinewidth{0.528403pt}%
\definecolor{currentstroke}{rgb}{0.278012,0.180367,0.486697}%
\pgfsetstrokecolor{currentstroke}%
\pgfsetdash{}{0pt}%
\pgfpathmoveto{\pgfqpoint{4.648982in}{5.152918in}}%
\pgfpathlineto{\pgfqpoint{4.648924in}{5.152939in}}%
\pgfusepath{stroke}%
\end{pgfscope}%
\begin{pgfscope}%
\pgfpathrectangle{\pgfqpoint{3.985294in}{4.155455in}}{\pgfqpoint{2.279412in}{2.004545in}}%
\pgfusepath{clip}%
\pgfsetbuttcap%
\pgfsetroundjoin%
\pgfsetlinewidth{0.528378pt}%
\definecolor{currentstroke}{rgb}{0.278012,0.180367,0.486697}%
\pgfsetstrokecolor{currentstroke}%
\pgfsetdash{}{0pt}%
\pgfpathmoveto{\pgfqpoint{4.648924in}{5.152939in}}%
\pgfpathlineto{\pgfqpoint{4.649201in}{5.152853in}}%
\pgfusepath{stroke}%
\end{pgfscope}%
\begin{pgfscope}%
\pgfpathrectangle{\pgfqpoint{3.985294in}{4.155455in}}{\pgfqpoint{2.279412in}{2.004545in}}%
\pgfusepath{clip}%
\pgfsetbuttcap%
\pgfsetroundjoin%
\pgfsetlinewidth{0.528357pt}%
\definecolor{currentstroke}{rgb}{0.278012,0.180367,0.486697}%
\pgfsetstrokecolor{currentstroke}%
\pgfsetdash{}{0pt}%
\pgfpathmoveto{\pgfqpoint{4.649201in}{5.152853in}}%
\pgfpathlineto{\pgfqpoint{4.649607in}{5.152723in}}%
\pgfusepath{stroke}%
\end{pgfscope}%
\begin{pgfscope}%
\pgfpathrectangle{\pgfqpoint{3.985294in}{4.155455in}}{\pgfqpoint{2.279412in}{2.004545in}}%
\pgfusepath{clip}%
\pgfsetbuttcap%
\pgfsetroundjoin%
\pgfsetlinewidth{0.528440pt}%
\definecolor{currentstroke}{rgb}{0.278012,0.180367,0.486697}%
\pgfsetstrokecolor{currentstroke}%
\pgfsetdash{}{0pt}%
\pgfpathmoveto{\pgfqpoint{4.649607in}{5.152723in}}%
\pgfpathlineto{\pgfqpoint{4.649667in}{5.152700in}}%
\pgfusepath{stroke}%
\end{pgfscope}%
\begin{pgfscope}%
\pgfpathrectangle{\pgfqpoint{3.985294in}{4.155455in}}{\pgfqpoint{2.279412in}{2.004545in}}%
\pgfusepath{clip}%
\pgfsetbuttcap%
\pgfsetroundjoin%
\pgfsetlinewidth{0.528502pt}%
\definecolor{currentstroke}{rgb}{0.278012,0.180367,0.486697}%
\pgfsetstrokecolor{currentstroke}%
\pgfsetdash{}{0pt}%
\pgfpathmoveto{\pgfqpoint{4.649667in}{5.152700in}}%
\pgfpathlineto{\pgfqpoint{4.649314in}{5.152811in}}%
\pgfusepath{stroke}%
\end{pgfscope}%
\begin{pgfscope}%
\pgfpathrectangle{\pgfqpoint{3.985294in}{4.155455in}}{\pgfqpoint{2.279412in}{2.004545in}}%
\pgfusepath{clip}%
\pgfsetbuttcap%
\pgfsetroundjoin%
\pgfsetlinewidth{0.528441pt}%
\definecolor{currentstroke}{rgb}{0.278012,0.180367,0.486697}%
\pgfsetstrokecolor{currentstroke}%
\pgfsetdash{}{0pt}%
\pgfpathmoveto{\pgfqpoint{4.649314in}{5.152811in}}%
\pgfpathlineto{\pgfqpoint{4.648957in}{5.152926in}}%
\pgfusepath{stroke}%
\end{pgfscope}%
\begin{pgfscope}%
\pgfpathrectangle{\pgfqpoint{3.985294in}{4.155455in}}{\pgfqpoint{2.279412in}{2.004545in}}%
\pgfusepath{clip}%
\pgfsetbuttcap%
\pgfsetroundjoin%
\pgfsetlinewidth{0.528408pt}%
\definecolor{currentstroke}{rgb}{0.278012,0.180367,0.486697}%
\pgfsetstrokecolor{currentstroke}%
\pgfsetdash{}{0pt}%
\pgfpathmoveto{\pgfqpoint{4.648957in}{5.152926in}}%
\pgfpathlineto{\pgfqpoint{4.648892in}{5.152949in}}%
\pgfusepath{stroke}%
\end{pgfscope}%
\begin{pgfscope}%
\pgfpathrectangle{\pgfqpoint{3.985294in}{4.155455in}}{\pgfqpoint{2.279412in}{2.004545in}}%
\pgfusepath{clip}%
\pgfsetbuttcap%
\pgfsetroundjoin%
\pgfsetlinewidth{0.528382pt}%
\definecolor{currentstroke}{rgb}{0.278012,0.180367,0.486697}%
\pgfsetstrokecolor{currentstroke}%
\pgfsetdash{}{0pt}%
\pgfpathmoveto{\pgfqpoint{4.648892in}{5.152949in}}%
\pgfpathlineto{\pgfqpoint{4.649194in}{5.152856in}}%
\pgfusepath{stroke}%
\end{pgfscope}%
\begin{pgfscope}%
\pgfpathrectangle{\pgfqpoint{3.985294in}{4.155455in}}{\pgfqpoint{2.279412in}{2.004545in}}%
\pgfusepath{clip}%
\pgfsetbuttcap%
\pgfsetroundjoin%
\pgfsetlinewidth{0.528355pt}%
\definecolor{currentstroke}{rgb}{0.278012,0.180367,0.486697}%
\pgfsetstrokecolor{currentstroke}%
\pgfsetdash{}{0pt}%
\pgfpathmoveto{\pgfqpoint{4.649194in}{5.152856in}}%
\pgfpathlineto{\pgfqpoint{4.649648in}{5.152710in}}%
\pgfusepath{stroke}%
\end{pgfscope}%
\begin{pgfscope}%
\pgfpathrectangle{\pgfqpoint{3.985294in}{4.155455in}}{\pgfqpoint{2.279412in}{2.004545in}}%
\pgfusepath{clip}%
\pgfsetbuttcap%
\pgfsetroundjoin%
\pgfsetlinewidth{0.528450pt}%
\definecolor{currentstroke}{rgb}{0.278012,0.180367,0.486697}%
\pgfsetstrokecolor{currentstroke}%
\pgfsetdash{}{0pt}%
\pgfpathmoveto{\pgfqpoint{4.649648in}{5.152710in}}%
\pgfpathlineto{\pgfqpoint{4.649717in}{5.152683in}}%
\pgfusepath{stroke}%
\end{pgfscope}%
\begin{pgfscope}%
\pgfpathrectangle{\pgfqpoint{3.985294in}{4.155455in}}{\pgfqpoint{2.279412in}{2.004545in}}%
\pgfusepath{clip}%
\pgfsetbuttcap%
\pgfsetroundjoin%
\pgfsetlinewidth{0.528524pt}%
\definecolor{currentstroke}{rgb}{0.278012,0.180367,0.486697}%
\pgfsetstrokecolor{currentstroke}%
\pgfsetdash{}{0pt}%
\pgfpathmoveto{\pgfqpoint{4.649717in}{5.152683in}}%
\pgfpathlineto{\pgfqpoint{4.649323in}{5.152807in}}%
\pgfusepath{stroke}%
\end{pgfscope}%
\begin{pgfscope}%
\pgfpathrectangle{\pgfqpoint{3.985294in}{4.155455in}}{\pgfqpoint{2.279412in}{2.004545in}}%
\pgfusepath{clip}%
\pgfsetbuttcap%
\pgfsetroundjoin%
\pgfsetlinewidth{0.528449pt}%
\definecolor{currentstroke}{rgb}{0.278012,0.180367,0.486697}%
\pgfsetstrokecolor{currentstroke}%
\pgfsetdash{}{0pt}%
\pgfpathmoveto{\pgfqpoint{4.649323in}{5.152807in}}%
\pgfpathlineto{\pgfqpoint{4.648932in}{5.152933in}}%
\pgfusepath{stroke}%
\end{pgfscope}%
\begin{pgfscope}%
\pgfpathrectangle{\pgfqpoint{3.985294in}{4.155455in}}{\pgfqpoint{2.279412in}{2.004545in}}%
\pgfusepath{clip}%
\pgfsetbuttcap%
\pgfsetroundjoin%
\pgfsetlinewidth{0.528414pt}%
\definecolor{currentstroke}{rgb}{0.278012,0.180367,0.486697}%
\pgfsetstrokecolor{currentstroke}%
\pgfsetdash{}{0pt}%
\pgfpathmoveto{\pgfqpoint{4.648932in}{5.152933in}}%
\pgfpathlineto{\pgfqpoint{4.648858in}{5.152960in}}%
\pgfusepath{stroke}%
\end{pgfscope}%
\begin{pgfscope}%
\pgfpathrectangle{\pgfqpoint{3.985294in}{4.155455in}}{\pgfqpoint{2.279412in}{2.004545in}}%
\pgfusepath{clip}%
\pgfsetbuttcap%
\pgfsetroundjoin%
\pgfsetlinewidth{0.528386pt}%
\definecolor{currentstroke}{rgb}{0.278012,0.180367,0.486697}%
\pgfsetstrokecolor{currentstroke}%
\pgfsetdash{}{0pt}%
\pgfpathmoveto{\pgfqpoint{4.648858in}{5.152960in}}%
\pgfpathlineto{\pgfqpoint{4.649185in}{5.152859in}}%
\pgfusepath{stroke}%
\end{pgfscope}%
\begin{pgfscope}%
\pgfpathrectangle{\pgfqpoint{3.985294in}{4.155455in}}{\pgfqpoint{2.279412in}{2.004545in}}%
\pgfusepath{clip}%
\pgfsetbuttcap%
\pgfsetroundjoin%
\pgfsetlinewidth{0.528353pt}%
\definecolor{currentstroke}{rgb}{0.278012,0.180367,0.486697}%
\pgfsetstrokecolor{currentstroke}%
\pgfsetdash{}{0pt}%
\pgfpathmoveto{\pgfqpoint{4.649185in}{5.152859in}}%
\pgfpathlineto{\pgfqpoint{4.649692in}{5.152696in}}%
\pgfusepath{stroke}%
\end{pgfscope}%
\begin{pgfscope}%
\pgfpathrectangle{\pgfqpoint{3.985294in}{4.155455in}}{\pgfqpoint{2.279412in}{2.004545in}}%
\pgfusepath{clip}%
\pgfsetbuttcap%
\pgfsetroundjoin%
\pgfsetlinewidth{0.528463pt}%
\definecolor{currentstroke}{rgb}{0.278012,0.180367,0.486697}%
\pgfsetstrokecolor{currentstroke}%
\pgfsetdash{}{0pt}%
\pgfpathmoveto{\pgfqpoint{4.649692in}{5.152696in}}%
\pgfpathlineto{\pgfqpoint{4.649771in}{5.152666in}}%
\pgfusepath{stroke}%
\end{pgfscope}%
\begin{pgfscope}%
\pgfpathrectangle{\pgfqpoint{3.985294in}{4.155455in}}{\pgfqpoint{2.279412in}{2.004545in}}%
\pgfusepath{clip}%
\pgfsetbuttcap%
\pgfsetroundjoin%
\pgfsetlinewidth{0.528548pt}%
\definecolor{currentstroke}{rgb}{0.278012,0.180367,0.486697}%
\pgfsetstrokecolor{currentstroke}%
\pgfsetdash{}{0pt}%
\pgfpathmoveto{\pgfqpoint{4.649771in}{5.152666in}}%
\pgfpathlineto{\pgfqpoint{4.649334in}{5.152803in}}%
\pgfusepath{stroke}%
\end{pgfscope}%
\begin{pgfscope}%
\pgfpathrectangle{\pgfqpoint{3.985294in}{4.155455in}}{\pgfqpoint{2.279412in}{2.004545in}}%
\pgfusepath{clip}%
\pgfsetbuttcap%
\pgfsetroundjoin%
\pgfsetlinewidth{0.528458pt}%
\definecolor{currentstroke}{rgb}{0.278012,0.180367,0.486697}%
\pgfsetstrokecolor{currentstroke}%
\pgfsetdash{}{0pt}%
\pgfpathmoveto{\pgfqpoint{4.649334in}{5.152803in}}%
\pgfpathlineto{\pgfqpoint{4.648906in}{5.152941in}}%
\pgfusepath{stroke}%
\end{pgfscope}%
\begin{pgfscope}%
\pgfpathrectangle{\pgfqpoint{3.985294in}{4.155455in}}{\pgfqpoint{2.279412in}{2.004545in}}%
\pgfusepath{clip}%
\pgfsetbuttcap%
\pgfsetroundjoin%
\pgfsetlinewidth{0.528420pt}%
\definecolor{currentstroke}{rgb}{0.278012,0.180367,0.486697}%
\pgfsetstrokecolor{currentstroke}%
\pgfsetdash{}{0pt}%
\pgfpathmoveto{\pgfqpoint{4.648906in}{5.152941in}}%
\pgfpathlineto{\pgfqpoint{4.648823in}{5.152971in}}%
\pgfusepath{stroke}%
\end{pgfscope}%
\begin{pgfscope}%
\pgfpathrectangle{\pgfqpoint{3.985294in}{4.155455in}}{\pgfqpoint{2.279412in}{2.004545in}}%
\pgfusepath{clip}%
\pgfsetbuttcap%
\pgfsetroundjoin%
\pgfsetlinewidth{0.528392pt}%
\definecolor{currentstroke}{rgb}{0.278012,0.180367,0.486697}%
\pgfsetstrokecolor{currentstroke}%
\pgfsetdash{}{0pt}%
\pgfpathmoveto{\pgfqpoint{4.648823in}{5.152971in}}%
\pgfpathlineto{\pgfqpoint{4.649175in}{5.152862in}}%
\pgfusepath{stroke}%
\end{pgfscope}%
\begin{pgfscope}%
\pgfpathrectangle{\pgfqpoint{3.985294in}{4.155455in}}{\pgfqpoint{2.279412in}{2.004545in}}%
\pgfusepath{clip}%
\pgfsetbuttcap%
\pgfsetroundjoin%
\pgfsetlinewidth{0.528351pt}%
\definecolor{currentstroke}{rgb}{0.278012,0.180367,0.486697}%
\pgfsetstrokecolor{currentstroke}%
\pgfsetdash{}{0pt}%
\pgfpathmoveto{\pgfqpoint{4.649175in}{5.152862in}}%
\pgfpathlineto{\pgfqpoint{4.649737in}{5.152682in}}%
\pgfusepath{stroke}%
\end{pgfscope}%
\begin{pgfscope}%
\pgfpathrectangle{\pgfqpoint{3.985294in}{4.155455in}}{\pgfqpoint{2.279412in}{2.004545in}}%
\pgfusepath{clip}%
\pgfsetbuttcap%
\pgfsetroundjoin%
\pgfsetlinewidth{0.528477pt}%
\definecolor{currentstroke}{rgb}{0.278012,0.180367,0.486697}%
\pgfsetstrokecolor{currentstroke}%
\pgfsetdash{}{0pt}%
\pgfpathmoveto{\pgfqpoint{4.649737in}{5.152682in}}%
\pgfpathlineto{\pgfqpoint{4.649828in}{5.152647in}}%
\pgfusepath{stroke}%
\end{pgfscope}%
\begin{pgfscope}%
\pgfpathrectangle{\pgfqpoint{3.985294in}{4.155455in}}{\pgfqpoint{2.279412in}{2.004545in}}%
\pgfusepath{clip}%
\pgfsetbuttcap%
\pgfsetroundjoin%
\pgfsetlinewidth{0.528577pt}%
\definecolor{currentstroke}{rgb}{0.278012,0.180367,0.486697}%
\pgfsetstrokecolor{currentstroke}%
\pgfsetdash{}{0pt}%
\pgfpathmoveto{\pgfqpoint{4.649828in}{5.152647in}}%
\pgfpathlineto{\pgfqpoint{4.649347in}{5.152798in}}%
\pgfusepath{stroke}%
\end{pgfscope}%
\begin{pgfscope}%
\pgfpathrectangle{\pgfqpoint{3.985294in}{4.155455in}}{\pgfqpoint{2.279412in}{2.004545in}}%
\pgfusepath{clip}%
\pgfsetbuttcap%
\pgfsetroundjoin%
\pgfsetlinewidth{0.528468pt}%
\definecolor{currentstroke}{rgb}{0.278012,0.180367,0.486697}%
\pgfsetstrokecolor{currentstroke}%
\pgfsetdash{}{0pt}%
\pgfpathmoveto{\pgfqpoint{4.649347in}{5.152798in}}%
\pgfpathlineto{\pgfqpoint{4.648882in}{5.152948in}}%
\pgfusepath{stroke}%
\end{pgfscope}%
\begin{pgfscope}%
\pgfpathrectangle{\pgfqpoint{3.985294in}{4.155455in}}{\pgfqpoint{2.279412in}{2.004545in}}%
\pgfusepath{clip}%
\pgfsetbuttcap%
\pgfsetroundjoin%
\pgfsetlinewidth{0.528427pt}%
\definecolor{currentstroke}{rgb}{0.278012,0.180367,0.486697}%
\pgfsetstrokecolor{currentstroke}%
\pgfsetdash{}{0pt}%
\pgfpathmoveto{\pgfqpoint{4.648882in}{5.152948in}}%
\pgfpathlineto{\pgfqpoint{4.648788in}{5.152982in}}%
\pgfusepath{stroke}%
\end{pgfscope}%
\begin{pgfscope}%
\pgfpathrectangle{\pgfqpoint{3.985294in}{4.155455in}}{\pgfqpoint{2.279412in}{2.004545in}}%
\pgfusepath{clip}%
\pgfsetbuttcap%
\pgfsetroundjoin%
\pgfsetlinewidth{0.528398pt}%
\definecolor{currentstroke}{rgb}{0.278012,0.180367,0.486697}%
\pgfsetstrokecolor{currentstroke}%
\pgfsetdash{}{0pt}%
\pgfpathmoveto{\pgfqpoint{4.648788in}{5.152982in}}%
\pgfpathlineto{\pgfqpoint{4.649163in}{5.152866in}}%
\pgfusepath{stroke}%
\end{pgfscope}%
\begin{pgfscope}%
\pgfpathrectangle{\pgfqpoint{3.985294in}{4.155455in}}{\pgfqpoint{2.279412in}{2.004545in}}%
\pgfusepath{clip}%
\pgfsetbuttcap%
\pgfsetroundjoin%
\pgfsetlinewidth{0.528349pt}%
\definecolor{currentstroke}{rgb}{0.278012,0.180367,0.486697}%
\pgfsetstrokecolor{currentstroke}%
\pgfsetdash{}{0pt}%
\pgfpathmoveto{\pgfqpoint{4.649163in}{5.152866in}}%
\pgfpathlineto{\pgfqpoint{4.649783in}{5.152667in}}%
\pgfusepath{stroke}%
\end{pgfscope}%
\begin{pgfscope}%
\pgfpathrectangle{\pgfqpoint{3.985294in}{4.155455in}}{\pgfqpoint{2.279412in}{2.004545in}}%
\pgfusepath{clip}%
\pgfsetbuttcap%
\pgfsetroundjoin%
\pgfsetlinewidth{0.528492pt}%
\definecolor{currentstroke}{rgb}{0.278012,0.180367,0.486697}%
\pgfsetstrokecolor{currentstroke}%
\pgfsetdash{}{0pt}%
\pgfpathmoveto{\pgfqpoint{4.649783in}{5.152667in}}%
\pgfpathlineto{\pgfqpoint{4.649889in}{5.152627in}}%
\pgfusepath{stroke}%
\end{pgfscope}%
\begin{pgfscope}%
\pgfpathrectangle{\pgfqpoint{3.985294in}{4.155455in}}{\pgfqpoint{2.279412in}{2.004545in}}%
\pgfusepath{clip}%
\pgfsetbuttcap%
\pgfsetroundjoin%
\pgfsetlinewidth{0.528610pt}%
\definecolor{currentstroke}{rgb}{0.278012,0.180367,0.486697}%
\pgfsetstrokecolor{currentstroke}%
\pgfsetdash{}{0pt}%
\pgfpathmoveto{\pgfqpoint{4.649889in}{5.152627in}}%
\pgfpathlineto{\pgfqpoint{4.649363in}{5.152792in}}%
\pgfusepath{stroke}%
\end{pgfscope}%
\begin{pgfscope}%
\pgfpathrectangle{\pgfqpoint{3.985294in}{4.155455in}}{\pgfqpoint{2.279412in}{2.004545in}}%
\pgfusepath{clip}%
\pgfsetbuttcap%
\pgfsetroundjoin%
\pgfsetlinewidth{0.528479pt}%
\definecolor{currentstroke}{rgb}{0.278012,0.180367,0.486697}%
\pgfsetstrokecolor{currentstroke}%
\pgfsetdash{}{0pt}%
\pgfpathmoveto{\pgfqpoint{4.649363in}{5.152792in}}%
\pgfpathlineto{\pgfqpoint{4.648859in}{5.152955in}}%
\pgfusepath{stroke}%
\end{pgfscope}%
\begin{pgfscope}%
\pgfpathrectangle{\pgfqpoint{3.985294in}{4.155455in}}{\pgfqpoint{2.279412in}{2.004545in}}%
\pgfusepath{clip}%
\pgfsetbuttcap%
\pgfsetroundjoin%
\pgfsetlinewidth{0.528435pt}%
\definecolor{currentstroke}{rgb}{0.278012,0.180367,0.486697}%
\pgfsetstrokecolor{currentstroke}%
\pgfsetdash{}{0pt}%
\pgfpathmoveto{\pgfqpoint{4.648859in}{5.152955in}}%
\pgfpathlineto{\pgfqpoint{4.648753in}{5.152993in}}%
\pgfusepath{stroke}%
\end{pgfscope}%
\begin{pgfscope}%
\pgfpathrectangle{\pgfqpoint{3.985294in}{4.155455in}}{\pgfqpoint{2.279412in}{2.004545in}}%
\pgfusepath{clip}%
\pgfsetbuttcap%
\pgfsetroundjoin%
\pgfsetlinewidth{0.528406pt}%
\definecolor{currentstroke}{rgb}{0.278012,0.180367,0.486697}%
\pgfsetstrokecolor{currentstroke}%
\pgfsetdash{}{0pt}%
\pgfpathmoveto{\pgfqpoint{4.648753in}{5.152993in}}%
\pgfpathlineto{\pgfqpoint{4.649149in}{5.152871in}}%
\pgfusepath{stroke}%
\end{pgfscope}%
\begin{pgfscope}%
\pgfpathrectangle{\pgfqpoint{3.985294in}{4.155455in}}{\pgfqpoint{2.279412in}{2.004545in}}%
\pgfusepath{clip}%
\pgfsetbuttcap%
\pgfsetroundjoin%
\pgfsetlinewidth{0.528347pt}%
\definecolor{currentstroke}{rgb}{0.278012,0.180367,0.486697}%
\pgfsetstrokecolor{currentstroke}%
\pgfsetdash{}{0pt}%
\pgfpathmoveto{\pgfqpoint{4.649149in}{5.152871in}}%
\pgfpathlineto{\pgfqpoint{4.649828in}{5.152653in}}%
\pgfusepath{stroke}%
\end{pgfscope}%
\begin{pgfscope}%
\pgfpathrectangle{\pgfqpoint{3.985294in}{4.155455in}}{\pgfqpoint{2.279412in}{2.004545in}}%
\pgfusepath{clip}%
\pgfsetbuttcap%
\pgfsetroundjoin%
\pgfsetlinewidth{0.528508pt}%
\definecolor{currentstroke}{rgb}{0.278012,0.180367,0.486697}%
\pgfsetstrokecolor{currentstroke}%
\pgfsetdash{}{0pt}%
\pgfpathmoveto{\pgfqpoint{4.649828in}{5.152653in}}%
\pgfpathlineto{\pgfqpoint{4.649828in}{5.152653in}}%
\pgfusepath{stroke}%
\end{pgfscope}%
\begin{pgfscope}%
\pgfpathrectangle{\pgfqpoint{3.985294in}{4.155455in}}{\pgfqpoint{2.279412in}{2.004545in}}%
\pgfusepath{clip}%
\pgfsetbuttcap%
\pgfsetroundjoin%
\pgfsetlinewidth{0.528508pt}%
\definecolor{currentstroke}{rgb}{0.278012,0.180367,0.486697}%
\pgfsetstrokecolor{currentstroke}%
\pgfsetdash{}{0pt}%
\pgfpathmoveto{\pgfqpoint{4.649828in}{5.152653in}}%
\pgfpathlineto{\pgfqpoint{4.649293in}{5.152818in}}%
\pgfusepath{stroke}%
\end{pgfscope}%
\begin{pgfscope}%
\pgfpathrectangle{\pgfqpoint{3.985294in}{4.155455in}}{\pgfqpoint{2.279412in}{2.004545in}}%
\pgfusepath{clip}%
\pgfsetbuttcap%
\pgfsetroundjoin%
\pgfsetlinewidth{0.528426pt}%
\definecolor{currentstroke}{rgb}{0.278012,0.180367,0.486697}%
\pgfsetstrokecolor{currentstroke}%
\pgfsetdash{}{0pt}%
\pgfpathmoveto{\pgfqpoint{4.649293in}{5.152818in}}%
\pgfpathlineto{\pgfqpoint{4.648837in}{5.152964in}}%
\pgfusepath{stroke}%
\end{pgfscope}%
\begin{pgfscope}%
\pgfpathrectangle{\pgfqpoint{3.985294in}{4.155455in}}{\pgfqpoint{2.279412in}{2.004545in}}%
\pgfusepath{clip}%
\pgfsetbuttcap%
\pgfsetroundjoin%
\pgfsetlinewidth{0.528412pt}%
\definecolor{currentstroke}{rgb}{0.278012,0.180367,0.486697}%
\pgfsetstrokecolor{currentstroke}%
\pgfsetdash{}{0pt}%
\pgfpathmoveto{\pgfqpoint{4.648837in}{5.152964in}}%
\pgfpathlineto{\pgfqpoint{4.648787in}{5.152984in}}%
\pgfusepath{stroke}%
\end{pgfscope}%
\begin{pgfscope}%
\pgfpathrectangle{\pgfqpoint{3.985294in}{4.155455in}}{\pgfqpoint{2.279412in}{2.004545in}}%
\pgfusepath{clip}%
\pgfsetbuttcap%
\pgfsetroundjoin%
\pgfsetlinewidth{0.528386pt}%
\definecolor{currentstroke}{rgb}{0.278012,0.180367,0.486697}%
\pgfsetstrokecolor{currentstroke}%
\pgfsetdash{}{0pt}%
\pgfpathmoveto{\pgfqpoint{4.648787in}{5.152984in}}%
\pgfpathlineto{\pgfqpoint{4.649223in}{5.152848in}}%
\pgfusepath{stroke}%
\end{pgfscope}%
\begin{pgfscope}%
\pgfpathrectangle{\pgfqpoint{3.985294in}{4.155455in}}{\pgfqpoint{2.279412in}{2.004545in}}%
\pgfusepath{clip}%
\pgfsetbuttcap%
\pgfsetroundjoin%
\pgfsetlinewidth{0.528346pt}%
\definecolor{currentstroke}{rgb}{0.278012,0.180367,0.486697}%
\pgfsetstrokecolor{currentstroke}%
\pgfsetdash{}{0pt}%
\pgfpathmoveto{\pgfqpoint{4.649223in}{5.152848in}}%
\pgfpathlineto{\pgfqpoint{4.649861in}{5.152642in}}%
\pgfusepath{stroke}%
\end{pgfscope}%
\begin{pgfscope}%
\pgfpathrectangle{\pgfqpoint{3.985294in}{4.155455in}}{\pgfqpoint{2.279412in}{2.004545in}}%
\pgfusepath{clip}%
\pgfsetbuttcap%
\pgfsetroundjoin%
\pgfsetlinewidth{0.528526pt}%
\definecolor{currentstroke}{rgb}{0.278012,0.180367,0.486697}%
\pgfsetstrokecolor{currentstroke}%
\pgfsetdash{}{0pt}%
\pgfpathmoveto{\pgfqpoint{4.649861in}{5.152642in}}%
\pgfpathlineto{\pgfqpoint{4.649861in}{5.152642in}}%
\pgfusepath{stroke}%
\end{pgfscope}%
\begin{pgfscope}%
\pgfpathrectangle{\pgfqpoint{3.985294in}{4.155455in}}{\pgfqpoint{2.279412in}{2.004545in}}%
\pgfusepath{clip}%
\pgfsetbuttcap%
\pgfsetroundjoin%
\pgfsetlinewidth{0.528526pt}%
\definecolor{currentstroke}{rgb}{0.278012,0.180367,0.486697}%
\pgfsetstrokecolor{currentstroke}%
\pgfsetdash{}{0pt}%
\pgfpathmoveto{\pgfqpoint{4.649861in}{5.152642in}}%
\pgfpathlineto{\pgfqpoint{4.649287in}{5.152820in}}%
\pgfusepath{stroke}%
\end{pgfscope}%
\begin{pgfscope}%
\pgfpathrectangle{\pgfqpoint{3.985294in}{4.155455in}}{\pgfqpoint{2.279412in}{2.004545in}}%
\pgfusepath{clip}%
\pgfsetbuttcap%
\pgfsetroundjoin%
\pgfsetlinewidth{0.528430pt}%
\definecolor{currentstroke}{rgb}{0.278012,0.180367,0.486697}%
\pgfsetstrokecolor{currentstroke}%
\pgfsetdash{}{0pt}%
\pgfpathmoveto{\pgfqpoint{4.649287in}{5.152820in}}%
\pgfpathlineto{\pgfqpoint{4.648812in}{5.152972in}}%
\pgfusepath{stroke}%
\end{pgfscope}%
\begin{pgfscope}%
\pgfpathrectangle{\pgfqpoint{3.985294in}{4.155455in}}{\pgfqpoint{2.279412in}{2.004545in}}%
\pgfusepath{clip}%
\pgfsetbuttcap%
\pgfsetroundjoin%
\pgfsetlinewidth{0.528417pt}%
\definecolor{currentstroke}{rgb}{0.278012,0.180367,0.486697}%
\pgfsetstrokecolor{currentstroke}%
\pgfsetdash{}{0pt}%
\pgfpathmoveto{\pgfqpoint{4.648812in}{5.152972in}}%
\pgfpathlineto{\pgfqpoint{4.648766in}{5.152990in}}%
\pgfusepath{stroke}%
\end{pgfscope}%
\begin{pgfscope}%
\pgfpathrectangle{\pgfqpoint{3.985294in}{4.155455in}}{\pgfqpoint{2.279412in}{2.004545in}}%
\pgfusepath{clip}%
\pgfsetbuttcap%
\pgfsetroundjoin%
\pgfsetlinewidth{0.528389pt}%
\definecolor{currentstroke}{rgb}{0.278012,0.180367,0.486697}%
\pgfsetstrokecolor{currentstroke}%
\pgfsetdash{}{0pt}%
\pgfpathmoveto{\pgfqpoint{4.648766in}{5.152990in}}%
\pgfpathlineto{\pgfqpoint{4.649232in}{5.152845in}}%
\pgfusepath{stroke}%
\end{pgfscope}%
\begin{pgfscope}%
\pgfpathrectangle{\pgfqpoint{3.985294in}{4.155455in}}{\pgfqpoint{2.279412in}{2.004545in}}%
\pgfusepath{clip}%
\pgfsetbuttcap%
\pgfsetroundjoin%
\pgfsetlinewidth{0.528345pt}%
\definecolor{currentstroke}{rgb}{0.278012,0.180367,0.486697}%
\pgfsetstrokecolor{currentstroke}%
\pgfsetdash{}{0pt}%
\pgfpathmoveto{\pgfqpoint{4.649232in}{5.152845in}}%
\pgfpathlineto{\pgfqpoint{4.649910in}{5.152626in}}%
\pgfusepath{stroke}%
\end{pgfscope}%
\begin{pgfscope}%
\pgfpathrectangle{\pgfqpoint{3.985294in}{4.155455in}}{\pgfqpoint{2.279412in}{2.004545in}}%
\pgfusepath{clip}%
\pgfsetbuttcap%
\pgfsetroundjoin%
\pgfsetlinewidth{0.528548pt}%
\definecolor{currentstroke}{rgb}{0.278012,0.180367,0.486697}%
\pgfsetstrokecolor{currentstroke}%
\pgfsetdash{}{0pt}%
\pgfpathmoveto{\pgfqpoint{4.649910in}{5.152626in}}%
\pgfpathlineto{\pgfqpoint{4.649910in}{5.152626in}}%
\pgfusepath{stroke}%
\end{pgfscope}%
\begin{pgfscope}%
\pgfpathrectangle{\pgfqpoint{3.985294in}{4.155455in}}{\pgfqpoint{2.279412in}{2.004545in}}%
\pgfusepath{clip}%
\pgfsetbuttcap%
\pgfsetroundjoin%
\pgfsetlinewidth{0.528548pt}%
\definecolor{currentstroke}{rgb}{0.278012,0.180367,0.486697}%
\pgfsetstrokecolor{currentstroke}%
\pgfsetdash{}{0pt}%
\pgfpathmoveto{\pgfqpoint{4.649910in}{5.152626in}}%
\pgfpathlineto{\pgfqpoint{4.649291in}{5.152818in}}%
\pgfusepath{stroke}%
\end{pgfscope}%
\begin{pgfscope}%
\pgfpathrectangle{\pgfqpoint{3.985294in}{4.155455in}}{\pgfqpoint{2.279412in}{2.004545in}}%
\pgfusepath{clip}%
\pgfsetbuttcap%
\pgfsetroundjoin%
\pgfsetlinewidth{0.528435pt}%
\definecolor{currentstroke}{rgb}{0.278012,0.180367,0.486697}%
\pgfsetstrokecolor{currentstroke}%
\pgfsetdash{}{0pt}%
\pgfpathmoveto{\pgfqpoint{4.649291in}{5.152818in}}%
\pgfpathlineto{\pgfqpoint{4.648789in}{5.152979in}}%
\pgfusepath{stroke}%
\end{pgfscope}%
\begin{pgfscope}%
\pgfpathrectangle{\pgfqpoint{3.985294in}{4.155455in}}{\pgfqpoint{2.279412in}{2.004545in}}%
\pgfusepath{clip}%
\pgfsetbuttcap%
\pgfsetroundjoin%
\pgfsetlinewidth{0.528423pt}%
\definecolor{currentstroke}{rgb}{0.278012,0.180367,0.486697}%
\pgfsetstrokecolor{currentstroke}%
\pgfsetdash{}{0pt}%
\pgfpathmoveto{\pgfqpoint{4.648789in}{5.152979in}}%
\pgfpathlineto{\pgfqpoint{4.648738in}{5.152999in}}%
\pgfusepath{stroke}%
\end{pgfscope}%
\begin{pgfscope}%
\pgfpathrectangle{\pgfqpoint{3.985294in}{4.155455in}}{\pgfqpoint{2.279412in}{2.004545in}}%
\pgfusepath{clip}%
\pgfsetbuttcap%
\pgfsetroundjoin%
\pgfsetlinewidth{0.528395pt}%
\definecolor{currentstroke}{rgb}{0.278012,0.180367,0.486697}%
\pgfsetstrokecolor{currentstroke}%
\pgfsetdash{}{0pt}%
\pgfpathmoveto{\pgfqpoint{4.648738in}{5.152999in}}%
\pgfpathlineto{\pgfqpoint{4.649230in}{5.152846in}}%
\pgfusepath{stroke}%
\end{pgfscope}%
\begin{pgfscope}%
\pgfpathrectangle{\pgfqpoint{3.985294in}{4.155455in}}{\pgfqpoint{2.279412in}{2.004545in}}%
\pgfusepath{clip}%
\pgfsetbuttcap%
\pgfsetroundjoin%
\pgfsetlinewidth{0.528343pt}%
\definecolor{currentstroke}{rgb}{0.278012,0.180367,0.486697}%
\pgfsetstrokecolor{currentstroke}%
\pgfsetdash{}{0pt}%
\pgfpathmoveto{\pgfqpoint{4.649230in}{5.152846in}}%
\pgfpathlineto{\pgfqpoint{4.649961in}{5.152610in}}%
\pgfusepath{stroke}%
\end{pgfscope}%
\begin{pgfscope}%
\pgfpathrectangle{\pgfqpoint{3.985294in}{4.155455in}}{\pgfqpoint{2.279412in}{2.004545in}}%
\pgfusepath{clip}%
\pgfsetbuttcap%
\pgfsetroundjoin%
\pgfsetlinewidth{0.528570pt}%
\definecolor{currentstroke}{rgb}{0.278012,0.180367,0.486697}%
\pgfsetstrokecolor{currentstroke}%
\pgfsetdash{}{0pt}%
\pgfpathmoveto{\pgfqpoint{4.649961in}{5.152610in}}%
\pgfpathlineto{\pgfqpoint{4.649961in}{5.152610in}}%
\pgfusepath{stroke}%
\end{pgfscope}%
\begin{pgfscope}%
\pgfpathrectangle{\pgfqpoint{3.985294in}{4.155455in}}{\pgfqpoint{2.279412in}{2.004545in}}%
\pgfusepath{clip}%
\pgfsetbuttcap%
\pgfsetroundjoin%
\pgfsetlinewidth{0.528570pt}%
\definecolor{currentstroke}{rgb}{0.278012,0.180367,0.486697}%
\pgfsetstrokecolor{currentstroke}%
\pgfsetdash{}{0pt}%
\pgfpathmoveto{\pgfqpoint{4.649961in}{5.152610in}}%
\pgfpathlineto{\pgfqpoint{4.649297in}{5.152816in}}%
\pgfusepath{stroke}%
\end{pgfscope}%
\begin{pgfscope}%
\pgfpathrectangle{\pgfqpoint{3.985294in}{4.155455in}}{\pgfqpoint{2.279412in}{2.004545in}}%
\pgfusepath{clip}%
\pgfsetbuttcap%
\pgfsetroundjoin%
\pgfsetlinewidth{0.528440pt}%
\definecolor{currentstroke}{rgb}{0.278012,0.180367,0.486697}%
\pgfsetstrokecolor{currentstroke}%
\pgfsetdash{}{0pt}%
\pgfpathmoveto{\pgfqpoint{4.649297in}{5.152816in}}%
\pgfpathlineto{\pgfqpoint{4.648767in}{5.152986in}}%
\pgfusepath{stroke}%
\end{pgfscope}%
\begin{pgfscope}%
\pgfpathrectangle{\pgfqpoint{3.985294in}{4.155455in}}{\pgfqpoint{2.279412in}{2.004545in}}%
\pgfusepath{clip}%
\pgfsetbuttcap%
\pgfsetroundjoin%
\pgfsetlinewidth{0.528429pt}%
\definecolor{currentstroke}{rgb}{0.278012,0.180367,0.486697}%
\pgfsetstrokecolor{currentstroke}%
\pgfsetdash{}{0pt}%
\pgfpathmoveto{\pgfqpoint{4.648767in}{5.152986in}}%
\pgfpathlineto{\pgfqpoint{4.648711in}{5.153007in}}%
\pgfusepath{stroke}%
\end{pgfscope}%
\begin{pgfscope}%
\pgfpathrectangle{\pgfqpoint{3.985294in}{4.155455in}}{\pgfqpoint{2.279412in}{2.004545in}}%
\pgfusepath{clip}%
\pgfsetbuttcap%
\pgfsetroundjoin%
\pgfsetlinewidth{0.528400pt}%
\definecolor{currentstroke}{rgb}{0.278012,0.180367,0.486697}%
\pgfsetstrokecolor{currentstroke}%
\pgfsetdash{}{0pt}%
\pgfpathmoveto{\pgfqpoint{4.648711in}{5.153007in}}%
\pgfpathlineto{\pgfqpoint{4.649224in}{5.152848in}}%
\pgfusepath{stroke}%
\end{pgfscope}%
\begin{pgfscope}%
\pgfpathrectangle{\pgfqpoint{3.985294in}{4.155455in}}{\pgfqpoint{2.279412in}{2.004545in}}%
\pgfusepath{clip}%
\pgfsetbuttcap%
\pgfsetroundjoin%
\pgfsetlinewidth{0.528342pt}%
\definecolor{currentstroke}{rgb}{0.278012,0.180367,0.486697}%
\pgfsetstrokecolor{currentstroke}%
\pgfsetdash{}{0pt}%
\pgfpathmoveto{\pgfqpoint{4.649224in}{5.152848in}}%
\pgfpathlineto{\pgfqpoint{4.650008in}{5.152595in}}%
\pgfusepath{stroke}%
\end{pgfscope}%
\begin{pgfscope}%
\pgfpathrectangle{\pgfqpoint{3.985294in}{4.155455in}}{\pgfqpoint{2.279412in}{2.004545in}}%
\pgfusepath{clip}%
\pgfsetbuttcap%
\pgfsetroundjoin%
\pgfsetlinewidth{0.528593pt}%
\definecolor{currentstroke}{rgb}{0.278012,0.180367,0.486697}%
\pgfsetstrokecolor{currentstroke}%
\pgfsetdash{}{0pt}%
\pgfpathmoveto{\pgfqpoint{4.650008in}{5.152595in}}%
\pgfpathlineto{\pgfqpoint{4.650008in}{5.152595in}}%
\pgfusepath{stroke}%
\end{pgfscope}%
\begin{pgfscope}%
\pgfpathrectangle{\pgfqpoint{3.985294in}{4.155455in}}{\pgfqpoint{2.279412in}{2.004545in}}%
\pgfusepath{clip}%
\pgfsetbuttcap%
\pgfsetroundjoin%
\pgfsetlinewidth{0.528593pt}%
\definecolor{currentstroke}{rgb}{0.278012,0.180367,0.486697}%
\pgfsetstrokecolor{currentstroke}%
\pgfsetdash{}{0pt}%
\pgfpathmoveto{\pgfqpoint{4.650008in}{5.152595in}}%
\pgfpathlineto{\pgfqpoint{4.649304in}{5.152813in}}%
\pgfusepath{stroke}%
\end{pgfscope}%
\begin{pgfscope}%
\pgfpathrectangle{\pgfqpoint{3.985294in}{4.155455in}}{\pgfqpoint{2.279412in}{2.004545in}}%
\pgfusepath{clip}%
\pgfsetbuttcap%
\pgfsetroundjoin%
\pgfsetlinewidth{0.528445pt}%
\definecolor{currentstroke}{rgb}{0.278012,0.180367,0.486697}%
\pgfsetstrokecolor{currentstroke}%
\pgfsetdash{}{0pt}%
\pgfpathmoveto{\pgfqpoint{4.649304in}{5.152813in}}%
\pgfpathlineto{\pgfqpoint{4.648748in}{5.152991in}}%
\pgfusepath{stroke}%
\end{pgfscope}%
\begin{pgfscope}%
\pgfpathrectangle{\pgfqpoint{3.985294in}{4.155455in}}{\pgfqpoint{2.279412in}{2.004545in}}%
\pgfusepath{clip}%
\pgfsetbuttcap%
\pgfsetroundjoin%
\pgfsetlinewidth{0.528435pt}%
\definecolor{currentstroke}{rgb}{0.278012,0.180367,0.486697}%
\pgfsetstrokecolor{currentstroke}%
\pgfsetdash{}{0pt}%
\pgfpathmoveto{\pgfqpoint{4.648748in}{5.152991in}}%
\pgfpathlineto{\pgfqpoint{4.648686in}{5.153015in}}%
\pgfusepath{stroke}%
\end{pgfscope}%
\begin{pgfscope}%
\pgfpathrectangle{\pgfqpoint{3.985294in}{4.155455in}}{\pgfqpoint{2.279412in}{2.004545in}}%
\pgfusepath{clip}%
\pgfsetbuttcap%
\pgfsetroundjoin%
\pgfsetlinewidth{0.528406pt}%
\definecolor{currentstroke}{rgb}{0.278012,0.180367,0.486697}%
\pgfsetstrokecolor{currentstroke}%
\pgfsetdash{}{0pt}%
\pgfpathmoveto{\pgfqpoint{4.648686in}{5.153015in}}%
\pgfpathlineto{\pgfqpoint{4.649219in}{5.152850in}}%
\pgfusepath{stroke}%
\end{pgfscope}%
\begin{pgfscope}%
\pgfpathrectangle{\pgfqpoint{3.985294in}{4.155455in}}{\pgfqpoint{2.279412in}{2.004545in}}%
\pgfusepath{clip}%
\pgfsetbuttcap%
\pgfsetroundjoin%
\pgfsetlinewidth{0.528340pt}%
\definecolor{currentstroke}{rgb}{0.278012,0.180367,0.486697}%
\pgfsetstrokecolor{currentstroke}%
\pgfsetdash{}{0pt}%
\pgfpathmoveto{\pgfqpoint{4.649219in}{5.152850in}}%
\pgfpathlineto{\pgfqpoint{4.650051in}{5.152582in}}%
\pgfusepath{stroke}%
\end{pgfscope}%
\begin{pgfscope}%
\pgfpathrectangle{\pgfqpoint{3.985294in}{4.155455in}}{\pgfqpoint{2.279412in}{2.004545in}}%
\pgfusepath{clip}%
\pgfsetbuttcap%
\pgfsetroundjoin%
\pgfsetlinewidth{0.528614pt}%
\definecolor{currentstroke}{rgb}{0.278012,0.180367,0.486697}%
\pgfsetstrokecolor{currentstroke}%
\pgfsetdash{}{0pt}%
\pgfpathmoveto{\pgfqpoint{4.650051in}{5.152582in}}%
\pgfpathlineto{\pgfqpoint{4.650051in}{5.152582in}}%
\pgfusepath{stroke}%
\end{pgfscope}%
\begin{pgfscope}%
\pgfpathrectangle{\pgfqpoint{3.985294in}{4.155455in}}{\pgfqpoint{2.279412in}{2.004545in}}%
\pgfusepath{clip}%
\pgfsetbuttcap%
\pgfsetroundjoin%
\pgfsetlinewidth{0.528614pt}%
\definecolor{currentstroke}{rgb}{0.278012,0.180367,0.486697}%
\pgfsetstrokecolor{currentstroke}%
\pgfsetdash{}{0pt}%
\pgfpathmoveto{\pgfqpoint{4.650051in}{5.152582in}}%
\pgfpathlineto{\pgfqpoint{4.649311in}{5.152811in}}%
\pgfusepath{stroke}%
\end{pgfscope}%
\begin{pgfscope}%
\pgfpathrectangle{\pgfqpoint{3.985294in}{4.155455in}}{\pgfqpoint{2.279412in}{2.004545in}}%
\pgfusepath{clip}%
\pgfsetbuttcap%
\pgfsetroundjoin%
\pgfsetlinewidth{0.528449pt}%
\definecolor{currentstroke}{rgb}{0.278012,0.180367,0.486697}%
\pgfsetstrokecolor{currentstroke}%
\pgfsetdash{}{0pt}%
\pgfpathmoveto{\pgfqpoint{4.649311in}{5.152811in}}%
\pgfpathlineto{\pgfqpoint{4.648732in}{5.152996in}}%
\pgfusepath{stroke}%
\end{pgfscope}%
\begin{pgfscope}%
\pgfpathrectangle{\pgfqpoint{3.985294in}{4.155455in}}{\pgfqpoint{2.279412in}{2.004545in}}%
\pgfusepath{clip}%
\pgfsetbuttcap%
\pgfsetroundjoin%
\pgfsetlinewidth{0.528440pt}%
\definecolor{currentstroke}{rgb}{0.278012,0.180367,0.486697}%
\pgfsetstrokecolor{currentstroke}%
\pgfsetdash{}{0pt}%
\pgfpathmoveto{\pgfqpoint{4.648732in}{5.152996in}}%
\pgfpathlineto{\pgfqpoint{4.648664in}{5.153022in}}%
\pgfusepath{stroke}%
\end{pgfscope}%
\begin{pgfscope}%
\pgfpathrectangle{\pgfqpoint{3.985294in}{4.155455in}}{\pgfqpoint{2.279412in}{2.004545in}}%
\pgfusepath{clip}%
\pgfsetbuttcap%
\pgfsetroundjoin%
\pgfsetlinewidth{0.528411pt}%
\definecolor{currentstroke}{rgb}{0.278012,0.180367,0.486697}%
\pgfsetstrokecolor{currentstroke}%
\pgfsetdash{}{0pt}%
\pgfpathmoveto{\pgfqpoint{4.648664in}{5.153022in}}%
\pgfpathlineto{\pgfqpoint{4.649213in}{5.152851in}}%
\pgfusepath{stroke}%
\end{pgfscope}%
\begin{pgfscope}%
\pgfpathrectangle{\pgfqpoint{3.985294in}{4.155455in}}{\pgfqpoint{2.279412in}{2.004545in}}%
\pgfusepath{clip}%
\pgfsetbuttcap%
\pgfsetroundjoin%
\pgfsetlinewidth{0.528339pt}%
\definecolor{currentstroke}{rgb}{0.278012,0.180367,0.486697}%
\pgfsetstrokecolor{currentstroke}%
\pgfsetdash{}{0pt}%
\pgfpathmoveto{\pgfqpoint{4.649213in}{5.152851in}}%
\pgfpathlineto{\pgfqpoint{4.650089in}{5.152569in}}%
\pgfusepath{stroke}%
\end{pgfscope}%
\begin{pgfscope}%
\pgfpathrectangle{\pgfqpoint{3.985294in}{4.155455in}}{\pgfqpoint{2.279412in}{2.004545in}}%
\pgfusepath{clip}%
\pgfsetbuttcap%
\pgfsetroundjoin%
\pgfsetlinewidth{0.528633pt}%
\definecolor{currentstroke}{rgb}{0.278012,0.180367,0.486697}%
\pgfsetstrokecolor{currentstroke}%
\pgfsetdash{}{0pt}%
\pgfpathmoveto{\pgfqpoint{4.650089in}{5.152569in}}%
\pgfpathlineto{\pgfqpoint{4.650089in}{5.152569in}}%
\pgfusepath{stroke}%
\end{pgfscope}%
\begin{pgfscope}%
\pgfpathrectangle{\pgfqpoint{3.985294in}{4.155455in}}{\pgfqpoint{2.279412in}{2.004545in}}%
\pgfusepath{clip}%
\pgfsetbuttcap%
\pgfsetroundjoin%
\pgfsetlinewidth{0.528633pt}%
\definecolor{currentstroke}{rgb}{0.278012,0.180367,0.486697}%
\pgfsetstrokecolor{currentstroke}%
\pgfsetdash{}{0pt}%
\pgfpathmoveto{\pgfqpoint{4.650089in}{5.152569in}}%
\pgfpathlineto{\pgfqpoint{4.649317in}{5.152808in}}%
\pgfusepath{stroke}%
\end{pgfscope}%
\begin{pgfscope}%
\pgfpathrectangle{\pgfqpoint{3.985294in}{4.155455in}}{\pgfqpoint{2.279412in}{2.004545in}}%
\pgfusepath{clip}%
\pgfsetbuttcap%
\pgfsetroundjoin%
\pgfsetlinewidth{0.528453pt}%
\definecolor{currentstroke}{rgb}{0.278012,0.180367,0.486697}%
\pgfsetstrokecolor{currentstroke}%
\pgfsetdash{}{0pt}%
\pgfpathmoveto{\pgfqpoint{4.649317in}{5.152808in}}%
\pgfpathlineto{\pgfqpoint{4.648719in}{5.153000in}}%
\pgfusepath{stroke}%
\end{pgfscope}%
\begin{pgfscope}%
\pgfpathrectangle{\pgfqpoint{3.985294in}{4.155455in}}{\pgfqpoint{2.279412in}{2.004545in}}%
\pgfusepath{clip}%
\pgfsetbuttcap%
\pgfsetroundjoin%
\pgfsetlinewidth{0.528445pt}%
\definecolor{currentstroke}{rgb}{0.278012,0.180367,0.486697}%
\pgfsetstrokecolor{currentstroke}%
\pgfsetdash{}{0pt}%
\pgfpathmoveto{\pgfqpoint{4.648719in}{5.153000in}}%
\pgfpathlineto{\pgfqpoint{4.648646in}{5.153028in}}%
\pgfusepath{stroke}%
\end{pgfscope}%
\begin{pgfscope}%
\pgfpathrectangle{\pgfqpoint{3.985294in}{4.155455in}}{\pgfqpoint{2.279412in}{2.004545in}}%
\pgfusepath{clip}%
\pgfsetbuttcap%
\pgfsetroundjoin%
\pgfsetlinewidth{0.528416pt}%
\definecolor{currentstroke}{rgb}{0.278012,0.180367,0.486697}%
\pgfsetstrokecolor{currentstroke}%
\pgfsetdash{}{0pt}%
\pgfpathmoveto{\pgfqpoint{4.648646in}{5.153028in}}%
\pgfpathlineto{\pgfqpoint{4.649207in}{5.152853in}}%
\pgfusepath{stroke}%
\end{pgfscope}%
\begin{pgfscope}%
\pgfpathrectangle{\pgfqpoint{3.985294in}{4.155455in}}{\pgfqpoint{2.279412in}{2.004545in}}%
\pgfusepath{clip}%
\pgfsetbuttcap%
\pgfsetroundjoin%
\pgfsetlinewidth{0.528338pt}%
\definecolor{currentstroke}{rgb}{0.278012,0.180367,0.486697}%
\pgfsetstrokecolor{currentstroke}%
\pgfsetdash{}{0pt}%
\pgfpathmoveto{\pgfqpoint{4.649207in}{5.152853in}}%
\pgfpathlineto{\pgfqpoint{4.649207in}{5.152853in}}%
\pgfusepath{stroke}%
\end{pgfscope}%
\begin{pgfscope}%
\pgfpathrectangle{\pgfqpoint{3.985294in}{4.155455in}}{\pgfqpoint{2.279412in}{2.004545in}}%
\pgfusepath{clip}%
\pgfsetbuttcap%
\pgfsetroundjoin%
\pgfsetlinewidth{0.528338pt}%
\definecolor{currentstroke}{rgb}{0.278012,0.180367,0.486697}%
\pgfsetstrokecolor{currentstroke}%
\pgfsetdash{}{0pt}%
\pgfpathmoveto{\pgfqpoint{4.649207in}{5.152853in}}%
\pgfpathlineto{\pgfqpoint{4.649352in}{5.152805in}}%
\pgfusepath{stroke}%
\end{pgfscope}%
\begin{pgfscope}%
\pgfpathrectangle{\pgfqpoint{3.985294in}{4.155455in}}{\pgfqpoint{2.279412in}{2.004545in}}%
\pgfusepath{clip}%
\pgfsetbuttcap%
\pgfsetroundjoin%
\pgfsetlinewidth{0.528373pt}%
\definecolor{currentstroke}{rgb}{0.278012,0.180367,0.486697}%
\pgfsetstrokecolor{currentstroke}%
\pgfsetdash{}{0pt}%
\pgfpathmoveto{\pgfqpoint{4.649352in}{5.152805in}}%
\pgfpathlineto{\pgfqpoint{4.649404in}{5.152787in}}%
\pgfusepath{stroke}%
\end{pgfscope}%
\begin{pgfscope}%
\pgfpathrectangle{\pgfqpoint{3.985294in}{4.155455in}}{\pgfqpoint{2.279412in}{2.004545in}}%
\pgfusepath{clip}%
\pgfsetbuttcap%
\pgfsetroundjoin%
\pgfsetlinewidth{0.528404pt}%
\definecolor{currentstroke}{rgb}{0.278012,0.180367,0.486697}%
\pgfsetstrokecolor{currentstroke}%
\pgfsetdash{}{0pt}%
\pgfpathmoveto{\pgfqpoint{4.649404in}{5.152787in}}%
\pgfpathlineto{\pgfqpoint{4.649303in}{5.152818in}}%
\pgfusepath{stroke}%
\end{pgfscope}%
\begin{pgfscope}%
\pgfpathrectangle{\pgfqpoint{3.985294in}{4.155455in}}{\pgfqpoint{2.279412in}{2.004545in}}%
\pgfusepath{clip}%
\pgfsetbuttcap%
\pgfsetroundjoin%
\pgfsetlinewidth{0.528402pt}%
\definecolor{currentstroke}{rgb}{0.278012,0.180367,0.486697}%
\pgfsetstrokecolor{currentstroke}%
\pgfsetdash{}{0pt}%
\pgfpathmoveto{\pgfqpoint{4.649303in}{5.152818in}}%
\pgfpathlineto{\pgfqpoint{4.649155in}{5.152865in}}%
\pgfusepath{stroke}%
\end{pgfscope}%
\begin{pgfscope}%
\pgfpathrectangle{\pgfqpoint{3.985294in}{4.155455in}}{\pgfqpoint{2.279412in}{2.004545in}}%
\pgfusepath{clip}%
\pgfsetbuttcap%
\pgfsetroundjoin%
\pgfsetlinewidth{0.528388pt}%
\definecolor{currentstroke}{rgb}{0.278012,0.180367,0.486697}%
\pgfsetstrokecolor{currentstroke}%
\pgfsetdash{}{0pt}%
\pgfpathmoveto{\pgfqpoint{4.649155in}{5.152865in}}%
\pgfpathlineto{\pgfqpoint{4.649104in}{5.152882in}}%
\pgfusepath{stroke}%
\end{pgfscope}%
\begin{pgfscope}%
\pgfpathrectangle{\pgfqpoint{3.985294in}{4.155455in}}{\pgfqpoint{2.279412in}{2.004545in}}%
\pgfusepath{clip}%
\pgfsetbuttcap%
\pgfsetroundjoin%
\pgfsetlinewidth{0.528375pt}%
\definecolor{currentstroke}{rgb}{0.278012,0.180367,0.486697}%
\pgfsetstrokecolor{currentstroke}%
\pgfsetdash{}{0pt}%
\pgfpathmoveto{\pgfqpoint{4.649104in}{5.152882in}}%
\pgfpathlineto{\pgfqpoint{4.649200in}{5.152852in}}%
\pgfusepath{stroke}%
\end{pgfscope}%
\begin{pgfscope}%
\pgfpathrectangle{\pgfqpoint{3.985294in}{4.155455in}}{\pgfqpoint{2.279412in}{2.004545in}}%
\pgfusepath{clip}%
\pgfsetbuttcap%
\pgfsetroundjoin%
\pgfsetlinewidth{0.528371pt}%
\definecolor{currentstroke}{rgb}{0.278012,0.180367,0.486697}%
\pgfsetstrokecolor{currentstroke}%
\pgfsetdash{}{0pt}%
\pgfpathmoveto{\pgfqpoint{4.649200in}{5.152852in}}%
\pgfpathlineto{\pgfqpoint{4.649364in}{5.152800in}}%
\pgfusepath{stroke}%
\end{pgfscope}%
\begin{pgfscope}%
\pgfpathrectangle{\pgfqpoint{3.985294in}{4.155455in}}{\pgfqpoint{2.279412in}{2.004545in}}%
\pgfusepath{clip}%
\pgfsetbuttcap%
\pgfsetroundjoin%
\pgfsetlinewidth{0.528392pt}%
\definecolor{currentstroke}{rgb}{0.278012,0.180367,0.486697}%
\pgfsetstrokecolor{currentstroke}%
\pgfsetdash{}{0pt}%
\pgfpathmoveto{\pgfqpoint{4.649364in}{5.152800in}}%
\pgfpathlineto{\pgfqpoint{4.649424in}{5.152779in}}%
\pgfusepath{stroke}%
\end{pgfscope}%
\begin{pgfscope}%
\pgfpathrectangle{\pgfqpoint{3.985294in}{4.155455in}}{\pgfqpoint{2.279412in}{2.004545in}}%
\pgfusepath{clip}%
\pgfsetbuttcap%
\pgfsetroundjoin%
\pgfsetlinewidth{0.528417pt}%
\definecolor{currentstroke}{rgb}{0.278012,0.180367,0.486697}%
\pgfsetstrokecolor{currentstroke}%
\pgfsetdash{}{0pt}%
\pgfpathmoveto{\pgfqpoint{4.649424in}{5.152779in}}%
\pgfpathlineto{\pgfqpoint{4.649309in}{5.152815in}}%
\pgfusepath{stroke}%
\end{pgfscope}%
\begin{pgfscope}%
\pgfpathrectangle{\pgfqpoint{3.985294in}{4.155455in}}{\pgfqpoint{2.279412in}{2.004545in}}%
\pgfusepath{clip}%
\pgfsetbuttcap%
\pgfsetroundjoin%
\pgfsetlinewidth{0.528409pt}%
\definecolor{currentstroke}{rgb}{0.278012,0.180367,0.486697}%
\pgfsetstrokecolor{currentstroke}%
\pgfsetdash{}{0pt}%
\pgfpathmoveto{\pgfqpoint{4.649309in}{5.152815in}}%
\pgfpathlineto{\pgfqpoint{4.649144in}{5.152868in}}%
\pgfusepath{stroke}%
\end{pgfscope}%
\begin{pgfscope}%
\pgfpathrectangle{\pgfqpoint{3.985294in}{4.155455in}}{\pgfqpoint{2.279412in}{2.004545in}}%
\pgfusepath{clip}%
\pgfsetbuttcap%
\pgfsetroundjoin%
\pgfsetlinewidth{0.528391pt}%
\definecolor{currentstroke}{rgb}{0.278012,0.180367,0.486697}%
\pgfsetstrokecolor{currentstroke}%
\pgfsetdash{}{0pt}%
\pgfpathmoveto{\pgfqpoint{4.649144in}{5.152868in}}%
\pgfpathlineto{\pgfqpoint{4.649086in}{5.152888in}}%
\pgfusepath{stroke}%
\end{pgfscope}%
\begin{pgfscope}%
\pgfpathrectangle{\pgfqpoint{3.985294in}{4.155455in}}{\pgfqpoint{2.279412in}{2.004545in}}%
\pgfusepath{clip}%
\pgfsetbuttcap%
\pgfsetroundjoin%
\pgfsetlinewidth{0.528376pt}%
\definecolor{currentstroke}{rgb}{0.278012,0.180367,0.486697}%
\pgfsetstrokecolor{currentstroke}%
\pgfsetdash{}{0pt}%
\pgfpathmoveto{\pgfqpoint{4.649086in}{5.152888in}}%
\pgfpathlineto{\pgfqpoint{4.649194in}{5.152854in}}%
\pgfusepath{stroke}%
\end{pgfscope}%
\begin{pgfscope}%
\pgfpathrectangle{\pgfqpoint{3.985294in}{4.155455in}}{\pgfqpoint{2.279412in}{2.004545in}}%
\pgfusepath{clip}%
\pgfsetbuttcap%
\pgfsetroundjoin%
\pgfsetlinewidth{0.528369pt}%
\definecolor{currentstroke}{rgb}{0.278012,0.180367,0.486697}%
\pgfsetstrokecolor{currentstroke}%
\pgfsetdash{}{0pt}%
\pgfpathmoveto{\pgfqpoint{4.649194in}{5.152854in}}%
\pgfpathlineto{\pgfqpoint{4.649379in}{5.152795in}}%
\pgfusepath{stroke}%
\end{pgfscope}%
\begin{pgfscope}%
\pgfpathrectangle{\pgfqpoint{3.985294in}{4.155455in}}{\pgfqpoint{2.279412in}{2.004545in}}%
\pgfusepath{clip}%
\pgfsetbuttcap%
\pgfsetroundjoin%
\pgfsetlinewidth{0.528393pt}%
\definecolor{currentstroke}{rgb}{0.278012,0.180367,0.486697}%
\pgfsetstrokecolor{currentstroke}%
\pgfsetdash{}{0pt}%
\pgfpathmoveto{\pgfqpoint{4.649379in}{5.152795in}}%
\pgfpathlineto{\pgfqpoint{4.649446in}{5.152772in}}%
\pgfusepath{stroke}%
\end{pgfscope}%
\begin{pgfscope}%
\pgfpathrectangle{\pgfqpoint{3.985294in}{4.155455in}}{\pgfqpoint{2.279412in}{2.004545in}}%
\pgfusepath{clip}%
\pgfsetbuttcap%
\pgfsetroundjoin%
\pgfsetlinewidth{0.528422pt}%
\definecolor{currentstroke}{rgb}{0.278012,0.180367,0.486697}%
\pgfsetstrokecolor{currentstroke}%
\pgfsetdash{}{0pt}%
\pgfpathmoveto{\pgfqpoint{4.649446in}{5.152772in}}%
\pgfpathlineto{\pgfqpoint{4.649316in}{5.152812in}}%
\pgfusepath{stroke}%
\end{pgfscope}%
\begin{pgfscope}%
\pgfpathrectangle{\pgfqpoint{3.985294in}{4.155455in}}{\pgfqpoint{2.279412in}{2.004545in}}%
\pgfusepath{clip}%
\pgfsetbuttcap%
\pgfsetroundjoin%
\pgfsetlinewidth{0.528413pt}%
\definecolor{currentstroke}{rgb}{0.278012,0.180367,0.486697}%
\pgfsetstrokecolor{currentstroke}%
\pgfsetdash{}{0pt}%
\pgfpathmoveto{\pgfqpoint{4.649316in}{5.152812in}}%
\pgfpathlineto{\pgfqpoint{4.649131in}{5.152872in}}%
\pgfusepath{stroke}%
\end{pgfscope}%
\begin{pgfscope}%
\pgfpathrectangle{\pgfqpoint{3.985294in}{4.155455in}}{\pgfqpoint{2.279412in}{2.004545in}}%
\pgfusepath{clip}%
\pgfsetbuttcap%
\pgfsetroundjoin%
\pgfsetlinewidth{0.528392pt}%
\definecolor{currentstroke}{rgb}{0.278012,0.180367,0.486697}%
\pgfsetstrokecolor{currentstroke}%
\pgfsetdash{}{0pt}%
\pgfpathmoveto{\pgfqpoint{4.649131in}{5.152872in}}%
\pgfpathlineto{\pgfqpoint{4.649066in}{5.152894in}}%
\pgfusepath{stroke}%
\end{pgfscope}%
\begin{pgfscope}%
\pgfpathrectangle{\pgfqpoint{3.985294in}{4.155455in}}{\pgfqpoint{2.279412in}{2.004545in}}%
\pgfusepath{clip}%
\pgfsetbuttcap%
\pgfsetroundjoin%
\pgfsetlinewidth{0.528376pt}%
\definecolor{currentstroke}{rgb}{0.278012,0.180367,0.486697}%
\pgfsetstrokecolor{currentstroke}%
\pgfsetdash{}{0pt}%
\pgfpathmoveto{\pgfqpoint{4.649066in}{5.152894in}}%
\pgfpathlineto{\pgfqpoint{4.649187in}{5.152857in}}%
\pgfusepath{stroke}%
\end{pgfscope}%
\begin{pgfscope}%
\pgfpathrectangle{\pgfqpoint{3.985294in}{4.155455in}}{\pgfqpoint{2.279412in}{2.004545in}}%
\pgfusepath{clip}%
\pgfsetbuttcap%
\pgfsetroundjoin%
\pgfsetlinewidth{0.528368pt}%
\definecolor{currentstroke}{rgb}{0.278012,0.180367,0.486697}%
\pgfsetstrokecolor{currentstroke}%
\pgfsetdash{}{0pt}%
\pgfpathmoveto{\pgfqpoint{4.649187in}{5.152857in}}%
\pgfpathlineto{\pgfqpoint{4.649396in}{5.152790in}}%
\pgfusepath{stroke}%
\end{pgfscope}%
\begin{pgfscope}%
\pgfpathrectangle{\pgfqpoint{3.985294in}{4.155455in}}{\pgfqpoint{2.279412in}{2.004545in}}%
\pgfusepath{clip}%
\pgfsetbuttcap%
\pgfsetroundjoin%
\pgfsetlinewidth{0.528395pt}%
\definecolor{currentstroke}{rgb}{0.278012,0.180367,0.486697}%
\pgfsetstrokecolor{currentstroke}%
\pgfsetdash{}{0pt}%
\pgfpathmoveto{\pgfqpoint{4.649396in}{5.152790in}}%
\pgfpathlineto{\pgfqpoint{4.649472in}{5.152764in}}%
\pgfusepath{stroke}%
\end{pgfscope}%
\begin{pgfscope}%
\pgfpathrectangle{\pgfqpoint{3.985294in}{4.155455in}}{\pgfqpoint{2.279412in}{2.004545in}}%
\pgfusepath{clip}%
\pgfsetbuttcap%
\pgfsetroundjoin%
\pgfsetlinewidth{0.528428pt}%
\definecolor{currentstroke}{rgb}{0.278012,0.180367,0.486697}%
\pgfsetstrokecolor{currentstroke}%
\pgfsetdash{}{0pt}%
\pgfpathmoveto{\pgfqpoint{4.649472in}{5.152764in}}%
\pgfpathlineto{\pgfqpoint{4.649324in}{5.152810in}}%
\pgfusepath{stroke}%
\end{pgfscope}%
\begin{pgfscope}%
\pgfpathrectangle{\pgfqpoint{3.985294in}{4.155455in}}{\pgfqpoint{2.279412in}{2.004545in}}%
\pgfusepath{clip}%
\pgfsetbuttcap%
\pgfsetroundjoin%
\pgfsetlinewidth{0.528416pt}%
\definecolor{currentstroke}{rgb}{0.278012,0.180367,0.486697}%
\pgfsetstrokecolor{currentstroke}%
\pgfsetdash{}{0pt}%
\pgfpathmoveto{\pgfqpoint{4.649324in}{5.152810in}}%
\pgfpathlineto{\pgfqpoint{4.649117in}{5.152876in}}%
\pgfusepath{stroke}%
\end{pgfscope}%
\begin{pgfscope}%
\pgfpathrectangle{\pgfqpoint{3.985294in}{4.155455in}}{\pgfqpoint{2.279412in}{2.004545in}}%
\pgfusepath{clip}%
\pgfsetbuttcap%
\pgfsetroundjoin%
\pgfsetlinewidth{0.528394pt}%
\definecolor{currentstroke}{rgb}{0.278012,0.180367,0.486697}%
\pgfsetstrokecolor{currentstroke}%
\pgfsetdash{}{0pt}%
\pgfpathmoveto{\pgfqpoint{4.649117in}{5.152876in}}%
\pgfpathlineto{\pgfqpoint{4.649045in}{5.152901in}}%
\pgfusepath{stroke}%
\end{pgfscope}%
\begin{pgfscope}%
\pgfpathrectangle{\pgfqpoint{3.985294in}{4.155455in}}{\pgfqpoint{2.279412in}{2.004545in}}%
\pgfusepath{clip}%
\pgfsetbuttcap%
\pgfsetroundjoin%
\pgfsetlinewidth{0.528376pt}%
\definecolor{currentstroke}{rgb}{0.278012,0.180367,0.486697}%
\pgfsetstrokecolor{currentstroke}%
\pgfsetdash{}{0pt}%
\pgfpathmoveto{\pgfqpoint{4.649045in}{5.152901in}}%
\pgfpathlineto{\pgfqpoint{4.649179in}{5.152859in}}%
\pgfusepath{stroke}%
\end{pgfscope}%
\begin{pgfscope}%
\pgfpathrectangle{\pgfqpoint{3.985294in}{4.155455in}}{\pgfqpoint{2.279412in}{2.004545in}}%
\pgfusepath{clip}%
\pgfsetbuttcap%
\pgfsetroundjoin%
\pgfsetlinewidth{0.528366pt}%
\definecolor{currentstroke}{rgb}{0.278012,0.180367,0.486697}%
\pgfsetstrokecolor{currentstroke}%
\pgfsetdash{}{0pt}%
\pgfpathmoveto{\pgfqpoint{4.649179in}{5.152859in}}%
\pgfpathlineto{\pgfqpoint{4.649414in}{5.152784in}}%
\pgfusepath{stroke}%
\end{pgfscope}%
\begin{pgfscope}%
\pgfpathrectangle{\pgfqpoint{3.985294in}{4.155455in}}{\pgfqpoint{2.279412in}{2.004545in}}%
\pgfusepath{clip}%
\pgfsetbuttcap%
\pgfsetroundjoin%
\pgfsetlinewidth{0.528397pt}%
\definecolor{currentstroke}{rgb}{0.278012,0.180367,0.486697}%
\pgfsetstrokecolor{currentstroke}%
\pgfsetdash{}{0pt}%
\pgfpathmoveto{\pgfqpoint{4.649414in}{5.152784in}}%
\pgfpathlineto{\pgfqpoint{4.649500in}{5.152755in}}%
\pgfusepath{stroke}%
\end{pgfscope}%
\begin{pgfscope}%
\pgfpathrectangle{\pgfqpoint{3.985294in}{4.155455in}}{\pgfqpoint{2.279412in}{2.004545in}}%
\pgfusepath{clip}%
\pgfsetbuttcap%
\pgfsetroundjoin%
\pgfsetlinewidth{0.528436pt}%
\definecolor{currentstroke}{rgb}{0.278012,0.180367,0.486697}%
\pgfsetstrokecolor{currentstroke}%
\pgfsetdash{}{0pt}%
\pgfpathmoveto{\pgfqpoint{4.649500in}{5.152755in}}%
\pgfpathlineto{\pgfqpoint{4.649334in}{5.152806in}}%
\pgfusepath{stroke}%
\end{pgfscope}%
\begin{pgfscope}%
\pgfpathrectangle{\pgfqpoint{3.985294in}{4.155455in}}{\pgfqpoint{2.279412in}{2.004545in}}%
\pgfusepath{clip}%
\pgfsetbuttcap%
\pgfsetroundjoin%
\pgfsetlinewidth{0.528421pt}%
\definecolor{currentstroke}{rgb}{0.278012,0.180367,0.486697}%
\pgfsetstrokecolor{currentstroke}%
\pgfsetdash{}{0pt}%
\pgfpathmoveto{\pgfqpoint{4.649334in}{5.152806in}}%
\pgfpathlineto{\pgfqpoint{4.649101in}{5.152881in}}%
\pgfusepath{stroke}%
\end{pgfscope}%
\begin{pgfscope}%
\pgfpathrectangle{\pgfqpoint{3.985294in}{4.155455in}}{\pgfqpoint{2.279412in}{2.004545in}}%
\pgfusepath{clip}%
\pgfsetbuttcap%
\pgfsetroundjoin%
\pgfsetlinewidth{0.528395pt}%
\definecolor{currentstroke}{rgb}{0.278012,0.180367,0.486697}%
\pgfsetstrokecolor{currentstroke}%
\pgfsetdash{}{0pt}%
\pgfpathmoveto{\pgfqpoint{4.649101in}{5.152881in}}%
\pgfpathlineto{\pgfqpoint{4.649021in}{5.152908in}}%
\pgfusepath{stroke}%
\end{pgfscope}%
\begin{pgfscope}%
\pgfpathrectangle{\pgfqpoint{3.985294in}{4.155455in}}{\pgfqpoint{2.279412in}{2.004545in}}%
\pgfusepath{clip}%
\pgfsetbuttcap%
\pgfsetroundjoin%
\pgfsetlinewidth{0.528377pt}%
\definecolor{currentstroke}{rgb}{0.278012,0.180367,0.486697}%
\pgfsetstrokecolor{currentstroke}%
\pgfsetdash{}{0pt}%
\pgfpathmoveto{\pgfqpoint{4.649021in}{5.152908in}}%
\pgfpathlineto{\pgfqpoint{4.649170in}{5.152862in}}%
\pgfusepath{stroke}%
\end{pgfscope}%
\begin{pgfscope}%
\pgfpathrectangle{\pgfqpoint{3.985294in}{4.155455in}}{\pgfqpoint{2.279412in}{2.004545in}}%
\pgfusepath{clip}%
\pgfsetbuttcap%
\pgfsetroundjoin%
\pgfsetlinewidth{0.528364pt}%
\definecolor{currentstroke}{rgb}{0.278012,0.180367,0.486697}%
\pgfsetstrokecolor{currentstroke}%
\pgfsetdash{}{0pt}%
\pgfpathmoveto{\pgfqpoint{4.649170in}{5.152862in}}%
\pgfpathlineto{\pgfqpoint{4.649435in}{5.152778in}}%
\pgfusepath{stroke}%
\end{pgfscope}%
\begin{pgfscope}%
\pgfpathrectangle{\pgfqpoint{3.985294in}{4.155455in}}{\pgfqpoint{2.279412in}{2.004545in}}%
\pgfusepath{clip}%
\pgfsetbuttcap%
\pgfsetroundjoin%
\pgfsetlinewidth{0.528399pt}%
\definecolor{currentstroke}{rgb}{0.278012,0.180367,0.486697}%
\pgfsetstrokecolor{currentstroke}%
\pgfsetdash{}{0pt}%
\pgfpathmoveto{\pgfqpoint{4.649435in}{5.152778in}}%
\pgfpathlineto{\pgfqpoint{4.649533in}{5.152744in}}%
\pgfusepath{stroke}%
\end{pgfscope}%
\begin{pgfscope}%
\pgfpathrectangle{\pgfqpoint{3.985294in}{4.155455in}}{\pgfqpoint{2.279412in}{2.004545in}}%
\pgfusepath{clip}%
\pgfsetbuttcap%
\pgfsetroundjoin%
\pgfsetlinewidth{0.528445pt}%
\definecolor{currentstroke}{rgb}{0.278012,0.180367,0.486697}%
\pgfsetstrokecolor{currentstroke}%
\pgfsetdash{}{0pt}%
\pgfpathmoveto{\pgfqpoint{4.649533in}{5.152744in}}%
\pgfpathlineto{\pgfqpoint{4.649344in}{5.152803in}}%
\pgfusepath{stroke}%
\end{pgfscope}%
\begin{pgfscope}%
\pgfpathrectangle{\pgfqpoint{3.985294in}{4.155455in}}{\pgfqpoint{2.279412in}{2.004545in}}%
\pgfusepath{clip}%
\pgfsetbuttcap%
\pgfsetroundjoin%
\pgfsetlinewidth{0.528426pt}%
\definecolor{currentstroke}{rgb}{0.278012,0.180367,0.486697}%
\pgfsetstrokecolor{currentstroke}%
\pgfsetdash{}{0pt}%
\pgfpathmoveto{\pgfqpoint{4.649344in}{5.152803in}}%
\pgfpathlineto{\pgfqpoint{4.649085in}{5.152886in}}%
\pgfusepath{stroke}%
\end{pgfscope}%
\begin{pgfscope}%
\pgfpathrectangle{\pgfqpoint{3.985294in}{4.155455in}}{\pgfqpoint{2.279412in}{2.004545in}}%
\pgfusepath{clip}%
\pgfsetbuttcap%
\pgfsetroundjoin%
\pgfsetlinewidth{0.528397pt}%
\definecolor{currentstroke}{rgb}{0.278012,0.180367,0.486697}%
\pgfsetstrokecolor{currentstroke}%
\pgfsetdash{}{0pt}%
\pgfpathmoveto{\pgfqpoint{4.649085in}{5.152886in}}%
\pgfpathlineto{\pgfqpoint{4.648995in}{5.152916in}}%
\pgfusepath{stroke}%
\end{pgfscope}%
\begin{pgfscope}%
\pgfpathrectangle{\pgfqpoint{3.985294in}{4.155455in}}{\pgfqpoint{2.279412in}{2.004545in}}%
\pgfusepath{clip}%
\pgfsetbuttcap%
\pgfsetroundjoin%
\pgfsetlinewidth{0.528378pt}%
\definecolor{currentstroke}{rgb}{0.278012,0.180367,0.486697}%
\pgfsetstrokecolor{currentstroke}%
\pgfsetdash{}{0pt}%
\pgfpathmoveto{\pgfqpoint{4.648995in}{5.152916in}}%
\pgfpathlineto{\pgfqpoint{4.649159in}{5.152866in}}%
\pgfusepath{stroke}%
\end{pgfscope}%
\begin{pgfscope}%
\pgfpathrectangle{\pgfqpoint{3.985294in}{4.155455in}}{\pgfqpoint{2.279412in}{2.004545in}}%
\pgfusepath{clip}%
\pgfsetbuttcap%
\pgfsetroundjoin%
\pgfsetlinewidth{0.528362pt}%
\definecolor{currentstroke}{rgb}{0.278012,0.180367,0.486697}%
\pgfsetstrokecolor{currentstroke}%
\pgfsetdash{}{0pt}%
\pgfpathmoveto{\pgfqpoint{4.649159in}{5.152866in}}%
\pgfpathlineto{\pgfqpoint{4.649458in}{5.152771in}}%
\pgfusepath{stroke}%
\end{pgfscope}%
\begin{pgfscope}%
\pgfpathrectangle{\pgfqpoint{3.985294in}{4.155455in}}{\pgfqpoint{2.279412in}{2.004545in}}%
\pgfusepath{clip}%
\pgfsetbuttcap%
\pgfsetroundjoin%
\pgfsetlinewidth{0.528402pt}%
\definecolor{currentstroke}{rgb}{0.278012,0.180367,0.486697}%
\pgfsetstrokecolor{currentstroke}%
\pgfsetdash{}{0pt}%
\pgfpathmoveto{\pgfqpoint{4.649458in}{5.152771in}}%
\pgfpathlineto{\pgfqpoint{4.649569in}{5.152733in}}%
\pgfusepath{stroke}%
\end{pgfscope}%
\begin{pgfscope}%
\pgfpathrectangle{\pgfqpoint{3.985294in}{4.155455in}}{\pgfqpoint{2.279412in}{2.004545in}}%
\pgfusepath{clip}%
\pgfsetbuttcap%
\pgfsetroundjoin%
\pgfsetlinewidth{0.528457pt}%
\definecolor{currentstroke}{rgb}{0.278012,0.180367,0.486697}%
\pgfsetstrokecolor{currentstroke}%
\pgfsetdash{}{0pt}%
\pgfpathmoveto{\pgfqpoint{4.649569in}{5.152733in}}%
\pgfpathlineto{\pgfqpoint{4.649356in}{5.152799in}}%
\pgfusepath{stroke}%
\end{pgfscope}%
\begin{pgfscope}%
\pgfpathrectangle{\pgfqpoint{3.985294in}{4.155455in}}{\pgfqpoint{2.279412in}{2.004545in}}%
\pgfusepath{clip}%
\pgfsetbuttcap%
\pgfsetroundjoin%
\pgfsetlinewidth{0.528432pt}%
\definecolor{currentstroke}{rgb}{0.278012,0.180367,0.486697}%
\pgfsetstrokecolor{currentstroke}%
\pgfsetdash{}{0pt}%
\pgfpathmoveto{\pgfqpoint{4.649356in}{5.152799in}}%
\pgfpathlineto{\pgfqpoint{4.649067in}{5.152891in}}%
\pgfusepath{stroke}%
\end{pgfscope}%
\begin{pgfscope}%
\pgfpathrectangle{\pgfqpoint{3.985294in}{4.155455in}}{\pgfqpoint{2.279412in}{2.004545in}}%
\pgfusepath{clip}%
\pgfsetbuttcap%
\pgfsetroundjoin%
\pgfsetlinewidth{0.528400pt}%
\definecolor{currentstroke}{rgb}{0.278012,0.180367,0.486697}%
\pgfsetstrokecolor{currentstroke}%
\pgfsetdash{}{0pt}%
\pgfpathmoveto{\pgfqpoint{4.649067in}{5.152891in}}%
\pgfpathlineto{\pgfqpoint{4.648967in}{5.152925in}}%
\pgfusepath{stroke}%
\end{pgfscope}%
\begin{pgfscope}%
\pgfpathrectangle{\pgfqpoint{3.985294in}{4.155455in}}{\pgfqpoint{2.279412in}{2.004545in}}%
\pgfusepath{clip}%
\pgfsetbuttcap%
\pgfsetroundjoin%
\pgfsetlinewidth{0.528380pt}%
\definecolor{currentstroke}{rgb}{0.278012,0.180367,0.486697}%
\pgfsetstrokecolor{currentstroke}%
\pgfsetdash{}{0pt}%
\pgfpathmoveto{\pgfqpoint{4.648967in}{5.152925in}}%
\pgfpathlineto{\pgfqpoint{4.649148in}{5.152870in}}%
\pgfusepath{stroke}%
\end{pgfscope}%
\begin{pgfscope}%
\pgfpathrectangle{\pgfqpoint{3.985294in}{4.155455in}}{\pgfqpoint{2.279412in}{2.004545in}}%
\pgfusepath{clip}%
\pgfsetbuttcap%
\pgfsetroundjoin%
\pgfsetlinewidth{0.528360pt}%
\definecolor{currentstroke}{rgb}{0.278012,0.180367,0.486697}%
\pgfsetstrokecolor{currentstroke}%
\pgfsetdash{}{0pt}%
\pgfpathmoveto{\pgfqpoint{4.649148in}{5.152870in}}%
\pgfpathlineto{\pgfqpoint{4.649483in}{5.152763in}}%
\pgfusepath{stroke}%
\end{pgfscope}%
\begin{pgfscope}%
\pgfpathrectangle{\pgfqpoint{3.985294in}{4.155455in}}{\pgfqpoint{2.279412in}{2.004545in}}%
\pgfusepath{clip}%
\pgfsetbuttcap%
\pgfsetroundjoin%
\pgfsetlinewidth{0.528406pt}%
\definecolor{currentstroke}{rgb}{0.278012,0.180367,0.486697}%
\pgfsetstrokecolor{currentstroke}%
\pgfsetdash{}{0pt}%
\pgfpathmoveto{\pgfqpoint{4.649483in}{5.152763in}}%
\pgfpathlineto{\pgfqpoint{4.649609in}{5.152720in}}%
\pgfusepath{stroke}%
\end{pgfscope}%
\begin{pgfscope}%
\pgfpathrectangle{\pgfqpoint{3.985294in}{4.155455in}}{\pgfqpoint{2.279412in}{2.004545in}}%
\pgfusepath{clip}%
\pgfsetbuttcap%
\pgfsetroundjoin%
\pgfsetlinewidth{0.528470pt}%
\definecolor{currentstroke}{rgb}{0.278012,0.180367,0.486697}%
\pgfsetstrokecolor{currentstroke}%
\pgfsetdash{}{0pt}%
\pgfpathmoveto{\pgfqpoint{4.649609in}{5.152720in}}%
\pgfpathlineto{\pgfqpoint{4.649369in}{5.152794in}}%
\pgfusepath{stroke}%
\end{pgfscope}%
\begin{pgfscope}%
\pgfpathrectangle{\pgfqpoint{3.985294in}{4.155455in}}{\pgfqpoint{2.279412in}{2.004545in}}%
\pgfusepath{clip}%
\pgfsetbuttcap%
\pgfsetroundjoin%
\pgfsetlinewidth{0.528438pt}%
\definecolor{currentstroke}{rgb}{0.278012,0.180367,0.486697}%
\pgfsetstrokecolor{currentstroke}%
\pgfsetdash{}{0pt}%
\pgfpathmoveto{\pgfqpoint{4.649369in}{5.152794in}}%
\pgfpathlineto{\pgfqpoint{4.649049in}{5.152897in}}%
\pgfusepath{stroke}%
\end{pgfscope}%
\begin{pgfscope}%
\pgfpathrectangle{\pgfqpoint{3.985294in}{4.155455in}}{\pgfqpoint{2.279412in}{2.004545in}}%
\pgfusepath{clip}%
\pgfsetbuttcap%
\pgfsetroundjoin%
\pgfsetlinewidth{0.528403pt}%
\definecolor{currentstroke}{rgb}{0.278012,0.180367,0.486697}%
\pgfsetstrokecolor{currentstroke}%
\pgfsetdash{}{0pt}%
\pgfpathmoveto{\pgfqpoint{4.649049in}{5.152897in}}%
\pgfpathlineto{\pgfqpoint{4.648937in}{5.152935in}}%
\pgfusepath{stroke}%
\end{pgfscope}%
\begin{pgfscope}%
\pgfpathrectangle{\pgfqpoint{3.985294in}{4.155455in}}{\pgfqpoint{2.279412in}{2.004545in}}%
\pgfusepath{clip}%
\pgfsetbuttcap%
\pgfsetroundjoin%
\pgfsetlinewidth{0.528382pt}%
\definecolor{currentstroke}{rgb}{0.278012,0.180367,0.486697}%
\pgfsetstrokecolor{currentstroke}%
\pgfsetdash{}{0pt}%
\pgfpathmoveto{\pgfqpoint{4.648937in}{5.152935in}}%
\pgfpathlineto{\pgfqpoint{4.649136in}{5.152874in}}%
\pgfusepath{stroke}%
\end{pgfscope}%
\begin{pgfscope}%
\pgfpathrectangle{\pgfqpoint{3.985294in}{4.155455in}}{\pgfqpoint{2.279412in}{2.004545in}}%
\pgfusepath{clip}%
\pgfsetbuttcap%
\pgfsetroundjoin%
\pgfsetlinewidth{0.528358pt}%
\definecolor{currentstroke}{rgb}{0.278012,0.180367,0.486697}%
\pgfsetstrokecolor{currentstroke}%
\pgfsetdash{}{0pt}%
\pgfpathmoveto{\pgfqpoint{4.649136in}{5.152874in}}%
\pgfpathlineto{\pgfqpoint{4.649511in}{5.152754in}}%
\pgfusepath{stroke}%
\end{pgfscope}%
\begin{pgfscope}%
\pgfpathrectangle{\pgfqpoint{3.985294in}{4.155455in}}{\pgfqpoint{2.279412in}{2.004545in}}%
\pgfusepath{clip}%
\pgfsetbuttcap%
\pgfsetroundjoin%
\pgfsetlinewidth{0.528411pt}%
\definecolor{currentstroke}{rgb}{0.278012,0.180367,0.486697}%
\pgfsetstrokecolor{currentstroke}%
\pgfsetdash{}{0pt}%
\pgfpathmoveto{\pgfqpoint{4.649511in}{5.152754in}}%
\pgfpathlineto{\pgfqpoint{4.649654in}{5.152705in}}%
\pgfusepath{stroke}%
\end{pgfscope}%
\begin{pgfscope}%
\pgfpathrectangle{\pgfqpoint{3.985294in}{4.155455in}}{\pgfqpoint{2.279412in}{2.004545in}}%
\pgfusepath{clip}%
\pgfsetbuttcap%
\pgfsetroundjoin%
\pgfsetlinewidth{0.528486pt}%
\definecolor{currentstroke}{rgb}{0.278012,0.180367,0.486697}%
\pgfsetstrokecolor{currentstroke}%
\pgfsetdash{}{0pt}%
\pgfpathmoveto{\pgfqpoint{4.649654in}{5.152705in}}%
\pgfpathlineto{\pgfqpoint{4.649384in}{5.152789in}}%
\pgfusepath{stroke}%
\end{pgfscope}%
\begin{pgfscope}%
\pgfpathrectangle{\pgfqpoint{3.985294in}{4.155455in}}{\pgfqpoint{2.279412in}{2.004545in}}%
\pgfusepath{clip}%
\pgfsetbuttcap%
\pgfsetroundjoin%
\pgfsetlinewidth{0.528446pt}%
\definecolor{currentstroke}{rgb}{0.278012,0.180367,0.486697}%
\pgfsetstrokecolor{currentstroke}%
\pgfsetdash{}{0pt}%
\pgfpathmoveto{\pgfqpoint{4.649384in}{5.152789in}}%
\pgfpathlineto{\pgfqpoint{4.649029in}{5.152903in}}%
\pgfusepath{stroke}%
\end{pgfscope}%
\begin{pgfscope}%
\pgfpathrectangle{\pgfqpoint{3.985294in}{4.155455in}}{\pgfqpoint{2.279412in}{2.004545in}}%
\pgfusepath{clip}%
\pgfsetbuttcap%
\pgfsetroundjoin%
\pgfsetlinewidth{0.528407pt}%
\definecolor{currentstroke}{rgb}{0.278012,0.180367,0.486697}%
\pgfsetstrokecolor{currentstroke}%
\pgfsetdash{}{0pt}%
\pgfpathmoveto{\pgfqpoint{4.649029in}{5.152903in}}%
\pgfpathlineto{\pgfqpoint{4.648905in}{5.152945in}}%
\pgfusepath{stroke}%
\end{pgfscope}%
\begin{pgfscope}%
\pgfpathrectangle{\pgfqpoint{3.985294in}{4.155455in}}{\pgfqpoint{2.279412in}{2.004545in}}%
\pgfusepath{clip}%
\pgfsetbuttcap%
\pgfsetroundjoin%
\pgfsetlinewidth{0.528386pt}%
\definecolor{currentstroke}{rgb}{0.278012,0.180367,0.486697}%
\pgfsetstrokecolor{currentstroke}%
\pgfsetdash{}{0pt}%
\pgfpathmoveto{\pgfqpoint{4.648905in}{5.152945in}}%
\pgfpathlineto{\pgfqpoint{4.649122in}{5.152878in}}%
\pgfusepath{stroke}%
\end{pgfscope}%
\begin{pgfscope}%
\pgfpathrectangle{\pgfqpoint{3.985294in}{4.155455in}}{\pgfqpoint{2.279412in}{2.004545in}}%
\pgfusepath{clip}%
\pgfsetbuttcap%
\pgfsetroundjoin%
\pgfsetlinewidth{0.528356pt}%
\definecolor{currentstroke}{rgb}{0.278012,0.180367,0.486697}%
\pgfsetstrokecolor{currentstroke}%
\pgfsetdash{}{0pt}%
\pgfpathmoveto{\pgfqpoint{4.649122in}{5.152878in}}%
\pgfpathlineto{\pgfqpoint{4.649541in}{5.152745in}}%
\pgfusepath{stroke}%
\end{pgfscope}%
\begin{pgfscope}%
\pgfpathrectangle{\pgfqpoint{3.985294in}{4.155455in}}{\pgfqpoint{2.279412in}{2.004545in}}%
\pgfusepath{clip}%
\pgfsetbuttcap%
\pgfsetroundjoin%
\pgfsetlinewidth{0.528416pt}%
\definecolor{currentstroke}{rgb}{0.278012,0.180367,0.486697}%
\pgfsetstrokecolor{currentstroke}%
\pgfsetdash{}{0pt}%
\pgfpathmoveto{\pgfqpoint{4.649541in}{5.152745in}}%
\pgfpathlineto{\pgfqpoint{4.649703in}{5.152689in}}%
\pgfusepath{stroke}%
\end{pgfscope}%
\begin{pgfscope}%
\pgfpathrectangle{\pgfqpoint{3.985294in}{4.155455in}}{\pgfqpoint{2.279412in}{2.004545in}}%
\pgfusepath{clip}%
\pgfsetbuttcap%
\pgfsetroundjoin%
\pgfsetlinewidth{0.528505pt}%
\definecolor{currentstroke}{rgb}{0.278012,0.180367,0.486697}%
\pgfsetstrokecolor{currentstroke}%
\pgfsetdash{}{0pt}%
\pgfpathmoveto{\pgfqpoint{4.649703in}{5.152689in}}%
\pgfpathlineto{\pgfqpoint{4.649401in}{5.152783in}}%
\pgfusepath{stroke}%
\end{pgfscope}%
\begin{pgfscope}%
\pgfpathrectangle{\pgfqpoint{3.985294in}{4.155455in}}{\pgfqpoint{2.279412in}{2.004545in}}%
\pgfusepath{clip}%
\pgfsetbuttcap%
\pgfsetroundjoin%
\pgfsetlinewidth{0.528455pt}%
\definecolor{currentstroke}{rgb}{0.278012,0.180367,0.486697}%
\pgfsetstrokecolor{currentstroke}%
\pgfsetdash{}{0pt}%
\pgfpathmoveto{\pgfqpoint{4.649401in}{5.152783in}}%
\pgfpathlineto{\pgfqpoint{4.649009in}{5.152909in}}%
\pgfusepath{stroke}%
\end{pgfscope}%
\begin{pgfscope}%
\pgfpathrectangle{\pgfqpoint{3.985294in}{4.155455in}}{\pgfqpoint{2.279412in}{2.004545in}}%
\pgfusepath{clip}%
\pgfsetbuttcap%
\pgfsetroundjoin%
\pgfsetlinewidth{0.528412pt}%
\definecolor{currentstroke}{rgb}{0.278012,0.180367,0.486697}%
\pgfsetstrokecolor{currentstroke}%
\pgfsetdash{}{0pt}%
\pgfpathmoveto{\pgfqpoint{4.649009in}{5.152909in}}%
\pgfpathlineto{\pgfqpoint{4.648872in}{5.152955in}}%
\pgfusepath{stroke}%
\end{pgfscope}%
\begin{pgfscope}%
\pgfpathrectangle{\pgfqpoint{3.985294in}{4.155455in}}{\pgfqpoint{2.279412in}{2.004545in}}%
\pgfusepath{clip}%
\pgfsetbuttcap%
\pgfsetroundjoin%
\pgfsetlinewidth{0.528390pt}%
\definecolor{currentstroke}{rgb}{0.278012,0.180367,0.486697}%
\pgfsetstrokecolor{currentstroke}%
\pgfsetdash{}{0pt}%
\pgfpathmoveto{\pgfqpoint{4.648872in}{5.152955in}}%
\pgfpathlineto{\pgfqpoint{4.649106in}{5.152883in}}%
\pgfusepath{stroke}%
\end{pgfscope}%
\begin{pgfscope}%
\pgfpathrectangle{\pgfqpoint{3.985294in}{4.155455in}}{\pgfqpoint{2.279412in}{2.004545in}}%
\pgfusepath{clip}%
\pgfsetbuttcap%
\pgfsetroundjoin%
\pgfsetlinewidth{0.528354pt}%
\definecolor{currentstroke}{rgb}{0.278012,0.180367,0.486697}%
\pgfsetstrokecolor{currentstroke}%
\pgfsetdash{}{0pt}%
\pgfpathmoveto{\pgfqpoint{4.649106in}{5.152883in}}%
\pgfpathlineto{\pgfqpoint{4.649574in}{5.152734in}}%
\pgfusepath{stroke}%
\end{pgfscope}%
\begin{pgfscope}%
\pgfpathrectangle{\pgfqpoint{3.985294in}{4.155455in}}{\pgfqpoint{2.279412in}{2.004545in}}%
\pgfusepath{clip}%
\pgfsetbuttcap%
\pgfsetroundjoin%
\pgfsetlinewidth{0.528422pt}%
\definecolor{currentstroke}{rgb}{0.278012,0.180367,0.486697}%
\pgfsetstrokecolor{currentstroke}%
\pgfsetdash{}{0pt}%
\pgfpathmoveto{\pgfqpoint{4.649574in}{5.152734in}}%
\pgfpathlineto{\pgfqpoint{4.649757in}{5.152672in}}%
\pgfusepath{stroke}%
\end{pgfscope}%
\begin{pgfscope}%
\pgfpathrectangle{\pgfqpoint{3.985294in}{4.155455in}}{\pgfqpoint{2.279412in}{2.004545in}}%
\pgfusepath{clip}%
\pgfsetbuttcap%
\pgfsetroundjoin%
\pgfsetlinewidth{0.528528pt}%
\definecolor{currentstroke}{rgb}{0.278012,0.180367,0.486697}%
\pgfsetstrokecolor{currentstroke}%
\pgfsetdash{}{0pt}%
\pgfpathmoveto{\pgfqpoint{4.649757in}{5.152672in}}%
\pgfpathlineto{\pgfqpoint{4.649420in}{5.152776in}}%
\pgfusepath{stroke}%
\end{pgfscope}%
\begin{pgfscope}%
\pgfpathrectangle{\pgfqpoint{3.985294in}{4.155455in}}{\pgfqpoint{2.279412in}{2.004545in}}%
\pgfusepath{clip}%
\pgfsetbuttcap%
\pgfsetroundjoin%
\pgfsetlinewidth{0.528466pt}%
\definecolor{currentstroke}{rgb}{0.278012,0.180367,0.486697}%
\pgfsetstrokecolor{currentstroke}%
\pgfsetdash{}{0pt}%
\pgfpathmoveto{\pgfqpoint{4.649420in}{5.152776in}}%
\pgfpathlineto{\pgfqpoint{4.648989in}{5.152915in}}%
\pgfusepath{stroke}%
\end{pgfscope}%
\begin{pgfscope}%
\pgfpathrectangle{\pgfqpoint{3.985294in}{4.155455in}}{\pgfqpoint{2.279412in}{2.004545in}}%
\pgfusepath{clip}%
\pgfsetbuttcap%
\pgfsetroundjoin%
\pgfsetlinewidth{0.528417pt}%
\definecolor{currentstroke}{rgb}{0.278012,0.180367,0.486697}%
\pgfsetstrokecolor{currentstroke}%
\pgfsetdash{}{0pt}%
\pgfpathmoveto{\pgfqpoint{4.648989in}{5.152915in}}%
\pgfpathlineto{\pgfqpoint{4.648837in}{5.152966in}}%
\pgfusepath{stroke}%
\end{pgfscope}%
\begin{pgfscope}%
\pgfpathrectangle{\pgfqpoint{3.985294in}{4.155455in}}{\pgfqpoint{2.279412in}{2.004545in}}%
\pgfusepath{clip}%
\pgfsetbuttcap%
\pgfsetroundjoin%
\pgfsetlinewidth{0.528395pt}%
\definecolor{currentstroke}{rgb}{0.278012,0.180367,0.486697}%
\pgfsetstrokecolor{currentstroke}%
\pgfsetdash{}{0pt}%
\pgfpathmoveto{\pgfqpoint{4.648837in}{5.152966in}}%
\pgfpathlineto{\pgfqpoint{4.649089in}{5.152889in}}%
\pgfusepath{stroke}%
\end{pgfscope}%
\begin{pgfscope}%
\pgfpathrectangle{\pgfqpoint{3.985294in}{4.155455in}}{\pgfqpoint{2.279412in}{2.004545in}}%
\pgfusepath{clip}%
\pgfsetbuttcap%
\pgfsetroundjoin%
\pgfsetlinewidth{0.528353pt}%
\definecolor{currentstroke}{rgb}{0.278012,0.180367,0.486697}%
\pgfsetstrokecolor{currentstroke}%
\pgfsetdash{}{0pt}%
\pgfpathmoveto{\pgfqpoint{4.649089in}{5.152889in}}%
\pgfpathlineto{\pgfqpoint{4.649607in}{5.152724in}}%
\pgfusepath{stroke}%
\end{pgfscope}%
\begin{pgfscope}%
\pgfpathrectangle{\pgfqpoint{3.985294in}{4.155455in}}{\pgfqpoint{2.279412in}{2.004545in}}%
\pgfusepath{clip}%
\pgfsetbuttcap%
\pgfsetroundjoin%
\pgfsetlinewidth{0.528430pt}%
\definecolor{currentstroke}{rgb}{0.278012,0.180367,0.486697}%
\pgfsetstrokecolor{currentstroke}%
\pgfsetdash{}{0pt}%
\pgfpathmoveto{\pgfqpoint{4.649607in}{5.152724in}}%
\pgfpathlineto{\pgfqpoint{4.649815in}{5.152653in}}%
\pgfusepath{stroke}%
\end{pgfscope}%
\begin{pgfscope}%
\pgfpathrectangle{\pgfqpoint{3.985294in}{4.155455in}}{\pgfqpoint{2.279412in}{2.004545in}}%
\pgfusepath{clip}%
\pgfsetbuttcap%
\pgfsetroundjoin%
\pgfsetlinewidth{0.528554pt}%
\definecolor{currentstroke}{rgb}{0.278012,0.180367,0.486697}%
\pgfsetstrokecolor{currentstroke}%
\pgfsetdash{}{0pt}%
\pgfpathmoveto{\pgfqpoint{4.649815in}{5.152653in}}%
\pgfpathlineto{\pgfqpoint{4.649441in}{5.152769in}}%
\pgfusepath{stroke}%
\end{pgfscope}%
\begin{pgfscope}%
\pgfpathrectangle{\pgfqpoint{3.985294in}{4.155455in}}{\pgfqpoint{2.279412in}{2.004545in}}%
\pgfusepath{clip}%
\pgfsetbuttcap%
\pgfsetroundjoin%
\pgfsetlinewidth{0.528478pt}%
\definecolor{currentstroke}{rgb}{0.278012,0.180367,0.486697}%
\pgfsetstrokecolor{currentstroke}%
\pgfsetdash{}{0pt}%
\pgfpathmoveto{\pgfqpoint{4.649441in}{5.152769in}}%
\pgfpathlineto{\pgfqpoint{4.648969in}{5.152920in}}%
\pgfusepath{stroke}%
\end{pgfscope}%
\begin{pgfscope}%
\pgfpathrectangle{\pgfqpoint{3.985294in}{4.155455in}}{\pgfqpoint{2.279412in}{2.004545in}}%
\pgfusepath{clip}%
\pgfsetbuttcap%
\pgfsetroundjoin%
\pgfsetlinewidth{0.528423pt}%
\definecolor{currentstroke}{rgb}{0.278012,0.180367,0.486697}%
\pgfsetstrokecolor{currentstroke}%
\pgfsetdash{}{0pt}%
\pgfpathmoveto{\pgfqpoint{4.648969in}{5.152920in}}%
\pgfpathlineto{\pgfqpoint{4.648802in}{5.152977in}}%
\pgfusepath{stroke}%
\end{pgfscope}%
\begin{pgfscope}%
\pgfpathrectangle{\pgfqpoint{3.985294in}{4.155455in}}{\pgfqpoint{2.279412in}{2.004545in}}%
\pgfusepath{clip}%
\pgfsetbuttcap%
\pgfsetroundjoin%
\pgfsetlinewidth{0.528402pt}%
\definecolor{currentstroke}{rgb}{0.278012,0.180367,0.486697}%
\pgfsetstrokecolor{currentstroke}%
\pgfsetdash{}{0pt}%
\pgfpathmoveto{\pgfqpoint{4.648802in}{5.152977in}}%
\pgfpathlineto{\pgfqpoint{4.649070in}{5.152895in}}%
\pgfusepath{stroke}%
\end{pgfscope}%
\begin{pgfscope}%
\pgfpathrectangle{\pgfqpoint{3.985294in}{4.155455in}}{\pgfqpoint{2.279412in}{2.004545in}}%
\pgfusepath{clip}%
\pgfsetbuttcap%
\pgfsetroundjoin%
\pgfsetlinewidth{0.528352pt}%
\definecolor{currentstroke}{rgb}{0.278012,0.180367,0.486697}%
\pgfsetstrokecolor{currentstroke}%
\pgfsetdash{}{0pt}%
\pgfpathmoveto{\pgfqpoint{4.649070in}{5.152895in}}%
\pgfpathlineto{\pgfqpoint{4.649642in}{5.152713in}}%
\pgfusepath{stroke}%
\end{pgfscope}%
\begin{pgfscope}%
\pgfpathrectangle{\pgfqpoint{3.985294in}{4.155455in}}{\pgfqpoint{2.279412in}{2.004545in}}%
\pgfusepath{clip}%
\pgfsetbuttcap%
\pgfsetroundjoin%
\pgfsetlinewidth{0.528438pt}%
\definecolor{currentstroke}{rgb}{0.278012,0.180367,0.486697}%
\pgfsetstrokecolor{currentstroke}%
\pgfsetdash{}{0pt}%
\pgfpathmoveto{\pgfqpoint{4.649642in}{5.152713in}}%
\pgfpathlineto{\pgfqpoint{4.649877in}{5.152632in}}%
\pgfusepath{stroke}%
\end{pgfscope}%
\begin{pgfscope}%
\pgfpathrectangle{\pgfqpoint{3.985294in}{4.155455in}}{\pgfqpoint{2.279412in}{2.004545in}}%
\pgfusepath{clip}%
\pgfsetbuttcap%
\pgfsetroundjoin%
\pgfsetlinewidth{0.528585pt}%
\definecolor{currentstroke}{rgb}{0.278012,0.180367,0.486697}%
\pgfsetstrokecolor{currentstroke}%
\pgfsetdash{}{0pt}%
\pgfpathmoveto{\pgfqpoint{4.649877in}{5.152632in}}%
\pgfpathlineto{\pgfqpoint{4.649464in}{5.152760in}}%
\pgfusepath{stroke}%
\end{pgfscope}%
\begin{pgfscope}%
\pgfpathrectangle{\pgfqpoint{3.985294in}{4.155455in}}{\pgfqpoint{2.279412in}{2.004545in}}%
\pgfusepath{clip}%
\pgfsetbuttcap%
\pgfsetroundjoin%
\pgfsetlinewidth{0.528491pt}%
\definecolor{currentstroke}{rgb}{0.278012,0.180367,0.486697}%
\pgfsetstrokecolor{currentstroke}%
\pgfsetdash{}{0pt}%
\pgfpathmoveto{\pgfqpoint{4.649464in}{5.152760in}}%
\pgfpathlineto{\pgfqpoint{4.648950in}{5.152926in}}%
\pgfusepath{stroke}%
\end{pgfscope}%
\begin{pgfscope}%
\pgfpathrectangle{\pgfqpoint{3.985294in}{4.155455in}}{\pgfqpoint{2.279412in}{2.004545in}}%
\pgfusepath{clip}%
\pgfsetbuttcap%
\pgfsetroundjoin%
\pgfsetlinewidth{0.528430pt}%
\definecolor{currentstroke}{rgb}{0.278012,0.180367,0.486697}%
\pgfsetstrokecolor{currentstroke}%
\pgfsetdash{}{0pt}%
\pgfpathmoveto{\pgfqpoint{4.648950in}{5.152926in}}%
\pgfpathlineto{\pgfqpoint{4.648766in}{5.152988in}}%
\pgfusepath{stroke}%
\end{pgfscope}%
\begin{pgfscope}%
\pgfpathrectangle{\pgfqpoint{3.985294in}{4.155455in}}{\pgfqpoint{2.279412in}{2.004545in}}%
\pgfusepath{clip}%
\pgfsetbuttcap%
\pgfsetroundjoin%
\pgfsetlinewidth{0.528409pt}%
\definecolor{currentstroke}{rgb}{0.278012,0.180367,0.486697}%
\pgfsetstrokecolor{currentstroke}%
\pgfsetdash{}{0pt}%
\pgfpathmoveto{\pgfqpoint{4.648766in}{5.152988in}}%
\pgfpathlineto{\pgfqpoint{4.649050in}{5.152902in}}%
\pgfusepath{stroke}%
\end{pgfscope}%
\begin{pgfscope}%
\pgfpathrectangle{\pgfqpoint{3.985294in}{4.155455in}}{\pgfqpoint{2.279412in}{2.004545in}}%
\pgfusepath{clip}%
\pgfsetbuttcap%
\pgfsetroundjoin%
\pgfsetlinewidth{0.528351pt}%
\definecolor{currentstroke}{rgb}{0.278012,0.180367,0.486697}%
\pgfsetstrokecolor{currentstroke}%
\pgfsetdash{}{0pt}%
\pgfpathmoveto{\pgfqpoint{4.649050in}{5.152902in}}%
\pgfpathlineto{\pgfqpoint{4.649676in}{5.152702in}}%
\pgfusepath{stroke}%
\end{pgfscope}%
\begin{pgfscope}%
\pgfpathrectangle{\pgfqpoint{3.985294in}{4.155455in}}{\pgfqpoint{2.279412in}{2.004545in}}%
\pgfusepath{clip}%
\pgfsetbuttcap%
\pgfsetroundjoin%
\pgfsetlinewidth{0.528446pt}%
\definecolor{currentstroke}{rgb}{0.278012,0.180367,0.486697}%
\pgfsetstrokecolor{currentstroke}%
\pgfsetdash{}{0pt}%
\pgfpathmoveto{\pgfqpoint{4.649676in}{5.152702in}}%
\pgfpathlineto{\pgfqpoint{4.649676in}{5.152702in}}%
\pgfusepath{stroke}%
\end{pgfscope}%
\begin{pgfscope}%
\pgfpathrectangle{\pgfqpoint{3.985294in}{4.155455in}}{\pgfqpoint{2.279412in}{2.004545in}}%
\pgfusepath{clip}%
\pgfsetbuttcap%
\pgfsetroundjoin%
\pgfsetlinewidth{0.528446pt}%
\definecolor{currentstroke}{rgb}{0.278012,0.180367,0.486697}%
\pgfsetstrokecolor{currentstroke}%
\pgfsetdash{}{0pt}%
\pgfpathmoveto{\pgfqpoint{4.649676in}{5.152702in}}%
\pgfpathlineto{\pgfqpoint{4.649294in}{5.152819in}}%
\pgfusepath{stroke}%
\end{pgfscope}%
\begin{pgfscope}%
\pgfpathrectangle{\pgfqpoint{3.985294in}{4.155455in}}{\pgfqpoint{2.279412in}{2.004545in}}%
\pgfusepath{clip}%
\pgfsetbuttcap%
\pgfsetroundjoin%
\pgfsetlinewidth{0.528411pt}%
\definecolor{currentstroke}{rgb}{0.278012,0.180367,0.486697}%
\pgfsetstrokecolor{currentstroke}%
\pgfsetdash{}{0pt}%
\pgfpathmoveto{\pgfqpoint{4.649294in}{5.152819in}}%
\pgfpathlineto{\pgfqpoint{4.648935in}{5.152934in}}%
\pgfusepath{stroke}%
\end{pgfscope}%
\begin{pgfscope}%
\pgfpathrectangle{\pgfqpoint{3.985294in}{4.155455in}}{\pgfqpoint{2.279412in}{2.004545in}}%
\pgfusepath{clip}%
\pgfsetbuttcap%
\pgfsetroundjoin%
\pgfsetlinewidth{0.528395pt}%
\definecolor{currentstroke}{rgb}{0.278012,0.180367,0.486697}%
\pgfsetstrokecolor{currentstroke}%
\pgfsetdash{}{0pt}%
\pgfpathmoveto{\pgfqpoint{4.648935in}{5.152934in}}%
\pgfpathlineto{\pgfqpoint{4.648887in}{5.152952in}}%
\pgfusepath{stroke}%
\end{pgfscope}%
\begin{pgfscope}%
\pgfpathrectangle{\pgfqpoint{3.985294in}{4.155455in}}{\pgfqpoint{2.279412in}{2.004545in}}%
\pgfusepath{clip}%
\pgfsetbuttcap%
\pgfsetroundjoin%
\pgfsetlinewidth{0.528374pt}%
\definecolor{currentstroke}{rgb}{0.278012,0.180367,0.486697}%
\pgfsetstrokecolor{currentstroke}%
\pgfsetdash{}{0pt}%
\pgfpathmoveto{\pgfqpoint{4.648887in}{5.152952in}}%
\pgfpathlineto{\pgfqpoint{4.649216in}{5.152849in}}%
\pgfusepath{stroke}%
\end{pgfscope}%
\begin{pgfscope}%
\pgfpathrectangle{\pgfqpoint{3.985294in}{4.155455in}}{\pgfqpoint{2.279412in}{2.004545in}}%
\pgfusepath{clip}%
\pgfsetbuttcap%
\pgfsetroundjoin%
\pgfsetlinewidth{0.314895pt}%
\definecolor{currentstroke}{rgb}{0.269944,0.014625,0.341379}%
\pgfsetstrokecolor{currentstroke}%
\pgfsetdash{}{0pt}%
\pgfpathmoveto{\pgfqpoint{5.996962in}{5.067514in}}%
\pgfpathlineto{\pgfqpoint{5.946864in}{5.067801in}}%
\pgfusepath{stroke}%
\end{pgfscope}%
\begin{pgfscope}%
\pgfpathrectangle{\pgfqpoint{3.985294in}{4.155455in}}{\pgfqpoint{2.279412in}{2.004545in}}%
\pgfusepath{clip}%
\pgfsetbuttcap%
\pgfsetroundjoin%
\pgfsetlinewidth{0.316764pt}%
\definecolor{currentstroke}{rgb}{0.269944,0.014625,0.341379}%
\pgfsetstrokecolor{currentstroke}%
\pgfsetdash{}{0pt}%
\pgfpathmoveto{\pgfqpoint{5.946864in}{5.067801in}}%
\pgfpathlineto{\pgfqpoint{5.896737in}{5.067112in}}%
\pgfusepath{stroke}%
\end{pgfscope}%
\begin{pgfscope}%
\pgfpathrectangle{\pgfqpoint{3.985294in}{4.155455in}}{\pgfqpoint{2.279412in}{2.004545in}}%
\pgfusepath{clip}%
\pgfsetbuttcap%
\pgfsetroundjoin%
\pgfsetlinewidth{0.322364pt}%
\definecolor{currentstroke}{rgb}{0.271305,0.019942,0.347269}%
\pgfsetstrokecolor{currentstroke}%
\pgfsetdash{}{0pt}%
\pgfpathmoveto{\pgfqpoint{5.896737in}{5.067112in}}%
\pgfpathlineto{\pgfqpoint{5.846612in}{5.067206in}}%
\pgfusepath{stroke}%
\end{pgfscope}%
\begin{pgfscope}%
\pgfpathrectangle{\pgfqpoint{3.985294in}{4.155455in}}{\pgfqpoint{2.279412in}{2.004545in}}%
\pgfusepath{clip}%
\pgfsetbuttcap%
\pgfsetroundjoin%
\pgfsetlinewidth{0.334407pt}%
\definecolor{currentstroke}{rgb}{0.272594,0.025563,0.353093}%
\pgfsetstrokecolor{currentstroke}%
\pgfsetdash{}{0pt}%
\pgfpathmoveto{\pgfqpoint{5.846612in}{5.067206in}}%
\pgfpathlineto{\pgfqpoint{5.796483in}{5.067697in}}%
\pgfusepath{stroke}%
\end{pgfscope}%
\begin{pgfscope}%
\pgfpathrectangle{\pgfqpoint{3.985294in}{4.155455in}}{\pgfqpoint{2.279412in}{2.004545in}}%
\pgfusepath{clip}%
\pgfsetbuttcap%
\pgfsetroundjoin%
\pgfsetlinewidth{0.355299pt}%
\definecolor{currentstroke}{rgb}{0.276022,0.044167,0.370164}%
\pgfsetstrokecolor{currentstroke}%
\pgfsetdash{}{0pt}%
\pgfpathmoveto{\pgfqpoint{5.796483in}{5.067697in}}%
\pgfpathlineto{\pgfqpoint{5.746348in}{5.068307in}}%
\pgfusepath{stroke}%
\end{pgfscope}%
\begin{pgfscope}%
\pgfpathrectangle{\pgfqpoint{3.985294in}{4.155455in}}{\pgfqpoint{2.279412in}{2.004545in}}%
\pgfusepath{clip}%
\pgfsetbuttcap%
\pgfsetroundjoin%
\pgfsetlinewidth{0.375569pt}%
\definecolor{currentstroke}{rgb}{0.278791,0.062145,0.386592}%
\pgfsetstrokecolor{currentstroke}%
\pgfsetdash{}{0pt}%
\pgfpathmoveto{\pgfqpoint{5.746348in}{5.068307in}}%
\pgfpathlineto{\pgfqpoint{5.696206in}{5.069109in}}%
\pgfusepath{stroke}%
\end{pgfscope}%
\begin{pgfscope}%
\pgfpathrectangle{\pgfqpoint{3.985294in}{4.155455in}}{\pgfqpoint{2.279412in}{2.004545in}}%
\pgfusepath{clip}%
\pgfsetbuttcap%
\pgfsetroundjoin%
\pgfsetlinewidth{0.413006pt}%
\definecolor{currentstroke}{rgb}{0.282327,0.094955,0.417331}%
\pgfsetstrokecolor{currentstroke}%
\pgfsetdash{}{0pt}%
\pgfpathmoveto{\pgfqpoint{5.696206in}{5.069109in}}%
\pgfpathlineto{\pgfqpoint{5.646067in}{5.069907in}}%
\pgfusepath{stroke}%
\end{pgfscope}%
\begin{pgfscope}%
\pgfpathrectangle{\pgfqpoint{3.985294in}{4.155455in}}{\pgfqpoint{2.279412in}{2.004545in}}%
\pgfusepath{clip}%
\pgfsetbuttcap%
\pgfsetroundjoin%
\pgfsetlinewidth{0.449938pt}%
\definecolor{currentstroke}{rgb}{0.283229,0.120777,0.440584}%
\pgfsetstrokecolor{currentstroke}%
\pgfsetdash{}{0pt}%
\pgfpathmoveto{\pgfqpoint{5.646067in}{5.069907in}}%
\pgfpathlineto{\pgfqpoint{5.595929in}{5.070821in}}%
\pgfusepath{stroke}%
\end{pgfscope}%
\begin{pgfscope}%
\pgfpathrectangle{\pgfqpoint{3.985294in}{4.155455in}}{\pgfqpoint{2.279412in}{2.004545in}}%
\pgfusepath{clip}%
\pgfsetbuttcap%
\pgfsetroundjoin%
\pgfsetlinewidth{0.491048pt}%
\definecolor{currentstroke}{rgb}{0.281887,0.150881,0.465405}%
\pgfsetstrokecolor{currentstroke}%
\pgfsetdash{}{0pt}%
\pgfpathmoveto{\pgfqpoint{5.595929in}{5.070821in}}%
\pgfpathlineto{\pgfqpoint{5.545789in}{5.071658in}}%
\pgfusepath{stroke}%
\end{pgfscope}%
\begin{pgfscope}%
\pgfpathrectangle{\pgfqpoint{3.985294in}{4.155455in}}{\pgfqpoint{2.279412in}{2.004545in}}%
\pgfusepath{clip}%
\pgfsetbuttcap%
\pgfsetroundjoin%
\pgfsetlinewidth{0.579984pt}%
\definecolor{currentstroke}{rgb}{0.270595,0.214069,0.507052}%
\pgfsetstrokecolor{currentstroke}%
\pgfsetdash{}{0pt}%
\pgfpathmoveto{\pgfqpoint{5.545789in}{5.071658in}}%
\pgfpathlineto{\pgfqpoint{5.495648in}{5.072538in}}%
\pgfusepath{stroke}%
\end{pgfscope}%
\begin{pgfscope}%
\pgfpathrectangle{\pgfqpoint{3.985294in}{4.155455in}}{\pgfqpoint{2.279412in}{2.004545in}}%
\pgfusepath{clip}%
\pgfsetbuttcap%
\pgfsetroundjoin%
\pgfsetlinewidth{0.644897pt}%
\definecolor{currentstroke}{rgb}{0.255645,0.260703,0.528312}%
\pgfsetstrokecolor{currentstroke}%
\pgfsetdash{}{0pt}%
\pgfpathmoveto{\pgfqpoint{5.495648in}{5.072538in}}%
\pgfpathlineto{\pgfqpoint{5.445514in}{5.073693in}}%
\pgfusepath{stroke}%
\end{pgfscope}%
\begin{pgfscope}%
\pgfpathrectangle{\pgfqpoint{3.985294in}{4.155455in}}{\pgfqpoint{2.279412in}{2.004545in}}%
\pgfusepath{clip}%
\pgfsetbuttcap%
\pgfsetroundjoin%
\pgfsetlinewidth{0.761355pt}%
\definecolor{currentstroke}{rgb}{0.223925,0.334994,0.548053}%
\pgfsetstrokecolor{currentstroke}%
\pgfsetdash{}{0pt}%
\pgfpathmoveto{\pgfqpoint{5.445514in}{5.073693in}}%
\pgfpathlineto{\pgfqpoint{5.395383in}{5.074974in}}%
\pgfusepath{stroke}%
\end{pgfscope}%
\begin{pgfscope}%
\pgfpathrectangle{\pgfqpoint{3.985294in}{4.155455in}}{\pgfqpoint{2.279412in}{2.004545in}}%
\pgfusepath{clip}%
\pgfsetbuttcap%
\pgfsetroundjoin%
\pgfsetlinewidth{0.858954pt}%
\definecolor{currentstroke}{rgb}{0.197636,0.391528,0.554969}%
\pgfsetstrokecolor{currentstroke}%
\pgfsetdash{}{0pt}%
\pgfpathmoveto{\pgfqpoint{5.395383in}{5.074974in}}%
\pgfpathlineto{\pgfqpoint{5.345258in}{5.076394in}}%
\pgfusepath{stroke}%
\end{pgfscope}%
\begin{pgfscope}%
\pgfpathrectangle{\pgfqpoint{3.985294in}{4.155455in}}{\pgfqpoint{2.279412in}{2.004545in}}%
\pgfusepath{clip}%
\pgfsetbuttcap%
\pgfsetroundjoin%
\pgfsetlinewidth{0.965048pt}%
\definecolor{currentstroke}{rgb}{0.172719,0.448791,0.557885}%
\pgfsetstrokecolor{currentstroke}%
\pgfsetdash{}{0pt}%
\pgfpathmoveto{\pgfqpoint{5.345258in}{5.076394in}}%
\pgfpathlineto{\pgfqpoint{5.295143in}{5.078077in}}%
\pgfusepath{stroke}%
\end{pgfscope}%
\begin{pgfscope}%
\pgfpathrectangle{\pgfqpoint{3.985294in}{4.155455in}}{\pgfqpoint{2.279412in}{2.004545in}}%
\pgfusepath{clip}%
\pgfsetbuttcap%
\pgfsetroundjoin%
\pgfsetlinewidth{0.304887pt}%
\definecolor{currentstroke}{rgb}{0.267004,0.004874,0.329415}%
\pgfsetstrokecolor{currentstroke}%
\pgfsetdash{}{0pt}%
\pgfpathmoveto{\pgfqpoint{5.996962in}{5.157727in}}%
\pgfpathlineto{\pgfqpoint{5.950134in}{5.151860in}}%
\pgfusepath{stroke}%
\end{pgfscope}%
\begin{pgfscope}%
\pgfpathrectangle{\pgfqpoint{3.985294in}{4.155455in}}{\pgfqpoint{2.279412in}{2.004545in}}%
\pgfusepath{clip}%
\pgfsetbuttcap%
\pgfsetroundjoin%
\pgfsetlinewidth{0.315480pt}%
\definecolor{currentstroke}{rgb}{0.269944,0.014625,0.341379}%
\pgfsetstrokecolor{currentstroke}%
\pgfsetdash{}{0pt}%
\pgfpathmoveto{\pgfqpoint{5.950134in}{5.151860in}}%
\pgfpathlineto{\pgfqpoint{5.905772in}{5.152066in}}%
\pgfusepath{stroke}%
\end{pgfscope}%
\begin{pgfscope}%
\pgfpathrectangle{\pgfqpoint{3.985294in}{4.155455in}}{\pgfqpoint{2.279412in}{2.004545in}}%
\pgfusepath{clip}%
\pgfsetbuttcap%
\pgfsetroundjoin%
\pgfsetlinewidth{0.327626pt}%
\definecolor{currentstroke}{rgb}{0.271305,0.019942,0.347269}%
\pgfsetstrokecolor{currentstroke}%
\pgfsetdash{}{0pt}%
\pgfpathmoveto{\pgfqpoint{5.905772in}{5.152066in}}%
\pgfpathlineto{\pgfqpoint{5.855629in}{5.152522in}}%
\pgfusepath{stroke}%
\end{pgfscope}%
\begin{pgfscope}%
\pgfpathrectangle{\pgfqpoint{3.985294in}{4.155455in}}{\pgfqpoint{2.279412in}{2.004545in}}%
\pgfusepath{clip}%
\pgfsetbuttcap%
\pgfsetroundjoin%
\pgfsetlinewidth{0.335010pt}%
\definecolor{currentstroke}{rgb}{0.272594,0.025563,0.353093}%
\pgfsetstrokecolor{currentstroke}%
\pgfsetdash{}{0pt}%
\pgfpathmoveto{\pgfqpoint{5.855629in}{5.152522in}}%
\pgfpathlineto{\pgfqpoint{5.805487in}{5.153165in}}%
\pgfusepath{stroke}%
\end{pgfscope}%
\begin{pgfscope}%
\pgfpathrectangle{\pgfqpoint{3.985294in}{4.155455in}}{\pgfqpoint{2.279412in}{2.004545in}}%
\pgfusepath{clip}%
\pgfsetbuttcap%
\pgfsetroundjoin%
\pgfsetlinewidth{0.357534pt}%
\definecolor{currentstroke}{rgb}{0.277018,0.050344,0.375715}%
\pgfsetstrokecolor{currentstroke}%
\pgfsetdash{}{0pt}%
\pgfpathmoveto{\pgfqpoint{5.805487in}{5.153165in}}%
\pgfpathlineto{\pgfqpoint{5.755341in}{5.153310in}}%
\pgfusepath{stroke}%
\end{pgfscope}%
\begin{pgfscope}%
\pgfpathrectangle{\pgfqpoint{3.985294in}{4.155455in}}{\pgfqpoint{2.279412in}{2.004545in}}%
\pgfusepath{clip}%
\pgfsetbuttcap%
\pgfsetroundjoin%
\pgfsetlinewidth{0.377219pt}%
\definecolor{currentstroke}{rgb}{0.279566,0.067836,0.391917}%
\pgfsetstrokecolor{currentstroke}%
\pgfsetdash{}{0pt}%
\pgfpathmoveto{\pgfqpoint{5.755341in}{5.153310in}}%
\pgfpathlineto{\pgfqpoint{5.705192in}{5.153196in}}%
\pgfusepath{stroke}%
\end{pgfscope}%
\begin{pgfscope}%
\pgfpathrectangle{\pgfqpoint{3.985294in}{4.155455in}}{\pgfqpoint{2.279412in}{2.004545in}}%
\pgfusepath{clip}%
\pgfsetbuttcap%
\pgfsetroundjoin%
\pgfsetlinewidth{0.408067pt}%
\definecolor{currentstroke}{rgb}{0.281924,0.089666,0.412415}%
\pgfsetstrokecolor{currentstroke}%
\pgfsetdash{}{0pt}%
\pgfpathmoveto{\pgfqpoint{5.705192in}{5.153196in}}%
\pgfpathlineto{\pgfqpoint{5.655040in}{5.153213in}}%
\pgfusepath{stroke}%
\end{pgfscope}%
\begin{pgfscope}%
\pgfpathrectangle{\pgfqpoint{3.985294in}{4.155455in}}{\pgfqpoint{2.279412in}{2.004545in}}%
\pgfusepath{clip}%
\pgfsetbuttcap%
\pgfsetroundjoin%
\pgfsetlinewidth{0.439536pt}%
\definecolor{currentstroke}{rgb}{0.283197,0.115680,0.436115}%
\pgfsetstrokecolor{currentstroke}%
\pgfsetdash{}{0pt}%
\pgfpathmoveto{\pgfqpoint{5.655040in}{5.153213in}}%
\pgfpathlineto{\pgfqpoint{5.604890in}{5.153191in}}%
\pgfusepath{stroke}%
\end{pgfscope}%
\begin{pgfscope}%
\pgfpathrectangle{\pgfqpoint{3.985294in}{4.155455in}}{\pgfqpoint{2.279412in}{2.004545in}}%
\pgfusepath{clip}%
\pgfsetbuttcap%
\pgfsetroundjoin%
\pgfsetlinewidth{0.504137pt}%
\definecolor{currentstroke}{rgb}{0.280868,0.160771,0.472899}%
\pgfsetstrokecolor{currentstroke}%
\pgfsetdash{}{0pt}%
\pgfpathmoveto{\pgfqpoint{5.604890in}{5.153191in}}%
\pgfpathlineto{\pgfqpoint{5.554738in}{5.153201in}}%
\pgfusepath{stroke}%
\end{pgfscope}%
\begin{pgfscope}%
\pgfpathrectangle{\pgfqpoint{3.985294in}{4.155455in}}{\pgfqpoint{2.279412in}{2.004545in}}%
\pgfusepath{clip}%
\pgfsetbuttcap%
\pgfsetroundjoin%
\pgfsetlinewidth{0.559889pt}%
\definecolor{currentstroke}{rgb}{0.274128,0.199721,0.498911}%
\pgfsetstrokecolor{currentstroke}%
\pgfsetdash{}{0pt}%
\pgfpathmoveto{\pgfqpoint{5.554738in}{5.153201in}}%
\pgfpathlineto{\pgfqpoint{5.504587in}{5.153137in}}%
\pgfusepath{stroke}%
\end{pgfscope}%
\begin{pgfscope}%
\pgfpathrectangle{\pgfqpoint{3.985294in}{4.155455in}}{\pgfqpoint{2.279412in}{2.004545in}}%
\pgfusepath{clip}%
\pgfsetbuttcap%
\pgfsetroundjoin%
\pgfsetlinewidth{0.677445pt}%
\definecolor{currentstroke}{rgb}{0.248629,0.278775,0.534556}%
\pgfsetstrokecolor{currentstroke}%
\pgfsetdash{}{0pt}%
\pgfpathmoveto{\pgfqpoint{5.504587in}{5.153137in}}%
\pgfpathlineto{\pgfqpoint{5.454436in}{5.152934in}}%
\pgfusepath{stroke}%
\end{pgfscope}%
\begin{pgfscope}%
\pgfpathrectangle{\pgfqpoint{3.985294in}{4.155455in}}{\pgfqpoint{2.279412in}{2.004545in}}%
\pgfusepath{clip}%
\pgfsetbuttcap%
\pgfsetroundjoin%
\pgfsetlinewidth{0.772743pt}%
\definecolor{currentstroke}{rgb}{0.221989,0.339161,0.548752}%
\pgfsetstrokecolor{currentstroke}%
\pgfsetdash{}{0pt}%
\pgfpathmoveto{\pgfqpoint{5.454436in}{5.152934in}}%
\pgfpathlineto{\pgfqpoint{5.404285in}{5.152784in}}%
\pgfusepath{stroke}%
\end{pgfscope}%
\begin{pgfscope}%
\pgfpathrectangle{\pgfqpoint{3.985294in}{4.155455in}}{\pgfqpoint{2.279412in}{2.004545in}}%
\pgfusepath{clip}%
\pgfsetbuttcap%
\pgfsetroundjoin%
\pgfsetlinewidth{0.883034pt}%
\definecolor{currentstroke}{rgb}{0.192357,0.403199,0.555836}%
\pgfsetstrokecolor{currentstroke}%
\pgfsetdash{}{0pt}%
\pgfpathmoveto{\pgfqpoint{5.404285in}{5.152784in}}%
\pgfpathlineto{\pgfqpoint{5.354134in}{5.152612in}}%
\pgfusepath{stroke}%
\end{pgfscope}%
\begin{pgfscope}%
\pgfpathrectangle{\pgfqpoint{3.985294in}{4.155455in}}{\pgfqpoint{2.279412in}{2.004545in}}%
\pgfusepath{clip}%
\pgfsetbuttcap%
\pgfsetroundjoin%
\pgfsetlinewidth{1.011035pt}%
\definecolor{currentstroke}{rgb}{0.162142,0.474838,0.558140}%
\pgfsetstrokecolor{currentstroke}%
\pgfsetdash{}{0pt}%
\pgfpathmoveto{\pgfqpoint{5.354134in}{5.152612in}}%
\pgfpathlineto{\pgfqpoint{5.303985in}{5.152341in}}%
\pgfusepath{stroke}%
\end{pgfscope}%
\begin{pgfscope}%
\pgfpathrectangle{\pgfqpoint{3.985294in}{4.155455in}}{\pgfqpoint{2.279412in}{2.004545in}}%
\pgfusepath{clip}%
\pgfsetbuttcap%
\pgfsetroundjoin%
\pgfsetlinewidth{1.164974pt}%
\definecolor{currentstroke}{rgb}{0.131172,0.555899,0.552459}%
\pgfsetstrokecolor{currentstroke}%
\pgfsetdash{}{0pt}%
\pgfpathmoveto{\pgfqpoint{5.303985in}{5.152341in}}%
\pgfpathlineto{\pgfqpoint{5.253835in}{5.152119in}}%
\pgfusepath{stroke}%
\end{pgfscope}%
\begin{pgfscope}%
\pgfpathrectangle{\pgfqpoint{3.985294in}{4.155455in}}{\pgfqpoint{2.279412in}{2.004545in}}%
\pgfusepath{clip}%
\pgfsetbuttcap%
\pgfsetroundjoin%
\pgfsetlinewidth{1.274030pt}%
\definecolor{currentstroke}{rgb}{0.119483,0.614817,0.537692}%
\pgfsetstrokecolor{currentstroke}%
\pgfsetdash{}{0pt}%
\pgfpathmoveto{\pgfqpoint{5.253835in}{5.152119in}}%
\pgfpathlineto{\pgfqpoint{5.203685in}{5.151998in}}%
\pgfusepath{stroke}%
\end{pgfscope}%
\begin{pgfscope}%
\pgfpathrectangle{\pgfqpoint{3.985294in}{4.155455in}}{\pgfqpoint{2.279412in}{2.004545in}}%
\pgfusepath{clip}%
\pgfsetbuttcap%
\pgfsetroundjoin%
\pgfsetlinewidth{1.412911pt}%
\definecolor{currentstroke}{rgb}{0.166383,0.690856,0.496502}%
\pgfsetstrokecolor{currentstroke}%
\pgfsetdash{}{0pt}%
\pgfpathmoveto{\pgfqpoint{5.203685in}{5.151998in}}%
\pgfpathlineto{\pgfqpoint{5.153536in}{5.151907in}}%
\pgfusepath{stroke}%
\end{pgfscope}%
\begin{pgfscope}%
\pgfpathrectangle{\pgfqpoint{3.985294in}{4.155455in}}{\pgfqpoint{2.279412in}{2.004545in}}%
\pgfusepath{clip}%
\pgfsetbuttcap%
\pgfsetroundjoin%
\pgfsetlinewidth{1.534934pt}%
\definecolor{currentstroke}{rgb}{0.266941,0.748751,0.440573}%
\pgfsetstrokecolor{currentstroke}%
\pgfsetdash{}{0pt}%
\pgfpathmoveto{\pgfqpoint{5.153536in}{5.151907in}}%
\pgfpathlineto{\pgfqpoint{5.103388in}{5.151745in}}%
\pgfusepath{stroke}%
\end{pgfscope}%
\begin{pgfscope}%
\pgfpathrectangle{\pgfqpoint{3.985294in}{4.155455in}}{\pgfqpoint{2.279412in}{2.004545in}}%
\pgfusepath{clip}%
\pgfsetbuttcap%
\pgfsetroundjoin%
\pgfsetlinewidth{1.664811pt}%
\definecolor{currentstroke}{rgb}{0.421908,0.805774,0.351910}%
\pgfsetstrokecolor{currentstroke}%
\pgfsetdash{}{0pt}%
\pgfpathmoveto{\pgfqpoint{5.103388in}{5.151745in}}%
\pgfpathlineto{\pgfqpoint{5.053243in}{5.151649in}}%
\pgfusepath{stroke}%
\end{pgfscope}%
\begin{pgfscope}%
\pgfpathrectangle{\pgfqpoint{3.985294in}{4.155455in}}{\pgfqpoint{2.279412in}{2.004545in}}%
\pgfusepath{clip}%
\pgfsetbuttcap%
\pgfsetroundjoin%
\pgfsetlinewidth{1.611233pt}%
\definecolor{currentstroke}{rgb}{0.352360,0.783011,0.392636}%
\pgfsetstrokecolor{currentstroke}%
\pgfsetdash{}{0pt}%
\pgfpathmoveto{\pgfqpoint{5.053243in}{5.151649in}}%
\pgfpathlineto{\pgfqpoint{5.003102in}{5.151604in}}%
\pgfusepath{stroke}%
\end{pgfscope}%
\begin{pgfscope}%
\pgfpathrectangle{\pgfqpoint{3.985294in}{4.155455in}}{\pgfqpoint{2.279412in}{2.004545in}}%
\pgfusepath{clip}%
\pgfsetbuttcap%
\pgfsetroundjoin%
\pgfsetlinewidth{1.620421pt}%
\definecolor{currentstroke}{rgb}{0.369214,0.788888,0.382914}%
\pgfsetstrokecolor{currentstroke}%
\pgfsetdash{}{0pt}%
\pgfpathmoveto{\pgfqpoint{5.003102in}{5.151604in}}%
\pgfpathlineto{\pgfqpoint{4.952968in}{5.151561in}}%
\pgfusepath{stroke}%
\end{pgfscope}%
\begin{pgfscope}%
\pgfpathrectangle{\pgfqpoint{3.985294in}{4.155455in}}{\pgfqpoint{2.279412in}{2.004545in}}%
\pgfusepath{clip}%
\pgfsetbuttcap%
\pgfsetroundjoin%
\pgfsetlinewidth{1.498576pt}%
\definecolor{currentstroke}{rgb}{0.232815,0.732247,0.459277}%
\pgfsetstrokecolor{currentstroke}%
\pgfsetdash{}{0pt}%
\pgfpathmoveto{\pgfqpoint{4.952968in}{5.151561in}}%
\pgfpathlineto{\pgfqpoint{4.902841in}{5.151673in}}%
\pgfusepath{stroke}%
\end{pgfscope}%
\begin{pgfscope}%
\pgfpathrectangle{\pgfqpoint{3.985294in}{4.155455in}}{\pgfqpoint{2.279412in}{2.004545in}}%
\pgfusepath{clip}%
\pgfsetbuttcap%
\pgfsetroundjoin%
\pgfsetlinewidth{1.368795pt}%
\definecolor{currentstroke}{rgb}{0.140210,0.665859,0.513427}%
\pgfsetstrokecolor{currentstroke}%
\pgfsetdash{}{0pt}%
\pgfpathmoveto{\pgfqpoint{4.902841in}{5.151673in}}%
\pgfpathlineto{\pgfqpoint{4.852732in}{5.151755in}}%
\pgfusepath{stroke}%
\end{pgfscope}%
\begin{pgfscope}%
\pgfpathrectangle{\pgfqpoint{3.985294in}{4.155455in}}{\pgfqpoint{2.279412in}{2.004545in}}%
\pgfusepath{clip}%
\pgfsetbuttcap%
\pgfsetroundjoin%
\pgfsetlinewidth{1.160367pt}%
\definecolor{currentstroke}{rgb}{0.131172,0.555899,0.552459}%
\pgfsetstrokecolor{currentstroke}%
\pgfsetdash{}{0pt}%
\pgfpathmoveto{\pgfqpoint{4.852732in}{5.151755in}}%
\pgfpathlineto{\pgfqpoint{4.802667in}{5.151964in}}%
\pgfusepath{stroke}%
\end{pgfscope}%
\begin{pgfscope}%
\pgfpathrectangle{\pgfqpoint{3.985294in}{4.155455in}}{\pgfqpoint{2.279412in}{2.004545in}}%
\pgfusepath{clip}%
\pgfsetbuttcap%
\pgfsetroundjoin%
\pgfsetlinewidth{1.000225pt}%
\definecolor{currentstroke}{rgb}{0.165117,0.467423,0.558141}%
\pgfsetstrokecolor{currentstroke}%
\pgfsetdash{}{0pt}%
\pgfpathmoveto{\pgfqpoint{4.802667in}{5.151964in}}%
\pgfpathlineto{\pgfqpoint{4.752712in}{5.151988in}}%
\pgfusepath{stroke}%
\end{pgfscope}%
\begin{pgfscope}%
\pgfpathrectangle{\pgfqpoint{3.985294in}{4.155455in}}{\pgfqpoint{2.279412in}{2.004545in}}%
\pgfusepath{clip}%
\pgfsetbuttcap%
\pgfsetroundjoin%
\pgfsetlinewidth{0.318163pt}%
\definecolor{currentstroke}{rgb}{0.269944,0.014625,0.341379}%
\pgfsetstrokecolor{currentstroke}%
\pgfsetdash{}{0pt}%
\pgfpathmoveto{\pgfqpoint{5.996962in}{5.202834in}}%
\pgfpathlineto{\pgfqpoint{5.946877in}{5.203440in}}%
\pgfusepath{stroke}%
\end{pgfscope}%
\begin{pgfscope}%
\pgfpathrectangle{\pgfqpoint{3.985294in}{4.155455in}}{\pgfqpoint{2.279412in}{2.004545in}}%
\pgfusepath{clip}%
\pgfsetbuttcap%
\pgfsetroundjoin%
\pgfsetlinewidth{0.321276pt}%
\definecolor{currentstroke}{rgb}{0.269944,0.014625,0.341379}%
\pgfsetstrokecolor{currentstroke}%
\pgfsetdash{}{0pt}%
\pgfpathmoveto{\pgfqpoint{5.946877in}{5.203440in}}%
\pgfpathlineto{\pgfqpoint{5.896774in}{5.203984in}}%
\pgfusepath{stroke}%
\end{pgfscope}%
\begin{pgfscope}%
\pgfpathrectangle{\pgfqpoint{3.985294in}{4.155455in}}{\pgfqpoint{2.279412in}{2.004545in}}%
\pgfusepath{clip}%
\pgfsetbuttcap%
\pgfsetroundjoin%
\pgfsetlinewidth{0.340688pt}%
\definecolor{currentstroke}{rgb}{0.273809,0.031497,0.358853}%
\pgfsetstrokecolor{currentstroke}%
\pgfsetdash{}{0pt}%
\pgfpathmoveto{\pgfqpoint{5.896774in}{5.203984in}}%
\pgfpathlineto{\pgfqpoint{5.846634in}{5.204539in}}%
\pgfusepath{stroke}%
\end{pgfscope}%
\begin{pgfscope}%
\pgfpathrectangle{\pgfqpoint{3.985294in}{4.155455in}}{\pgfqpoint{2.279412in}{2.004545in}}%
\pgfusepath{clip}%
\pgfsetbuttcap%
\pgfsetroundjoin%
\pgfsetlinewidth{0.329763pt}%
\definecolor{currentstroke}{rgb}{0.272594,0.025563,0.353093}%
\pgfsetstrokecolor{currentstroke}%
\pgfsetdash{}{0pt}%
\pgfpathmoveto{\pgfqpoint{5.846634in}{5.204539in}}%
\pgfpathlineto{\pgfqpoint{5.796496in}{5.205090in}}%
\pgfusepath{stroke}%
\end{pgfscope}%
\begin{pgfscope}%
\pgfpathrectangle{\pgfqpoint{3.985294in}{4.155455in}}{\pgfqpoint{2.279412in}{2.004545in}}%
\pgfusepath{clip}%
\pgfsetbuttcap%
\pgfsetroundjoin%
\pgfsetlinewidth{0.347732pt}%
\definecolor{currentstroke}{rgb}{0.274952,0.037752,0.364543}%
\pgfsetstrokecolor{currentstroke}%
\pgfsetdash{}{0pt}%
\pgfpathmoveto{\pgfqpoint{5.796496in}{5.205090in}}%
\pgfpathlineto{\pgfqpoint{5.746350in}{5.205503in}}%
\pgfusepath{stroke}%
\end{pgfscope}%
\begin{pgfscope}%
\pgfpathrectangle{\pgfqpoint{3.985294in}{4.155455in}}{\pgfqpoint{2.279412in}{2.004545in}}%
\pgfusepath{clip}%
\pgfsetbuttcap%
\pgfsetroundjoin%
\pgfsetlinewidth{0.370771pt}%
\definecolor{currentstroke}{rgb}{0.278791,0.062145,0.386592}%
\pgfsetstrokecolor{currentstroke}%
\pgfsetdash{}{0pt}%
\pgfpathmoveto{\pgfqpoint{5.746350in}{5.205503in}}%
\pgfpathlineto{\pgfqpoint{5.696200in}{5.205455in}}%
\pgfusepath{stroke}%
\end{pgfscope}%
\begin{pgfscope}%
\pgfpathrectangle{\pgfqpoint{3.985294in}{4.155455in}}{\pgfqpoint{2.279412in}{2.004545in}}%
\pgfusepath{clip}%
\pgfsetbuttcap%
\pgfsetroundjoin%
\pgfsetlinewidth{0.398948pt}%
\definecolor{currentstroke}{rgb}{0.281446,0.084320,0.407414}%
\pgfsetstrokecolor{currentstroke}%
\pgfsetdash{}{0pt}%
\pgfpathmoveto{\pgfqpoint{5.696200in}{5.205455in}}%
\pgfpathlineto{\pgfqpoint{5.646050in}{5.205200in}}%
\pgfusepath{stroke}%
\end{pgfscope}%
\begin{pgfscope}%
\pgfpathrectangle{\pgfqpoint{3.985294in}{4.155455in}}{\pgfqpoint{2.279412in}{2.004545in}}%
\pgfusepath{clip}%
\pgfsetbuttcap%
\pgfsetroundjoin%
\pgfsetlinewidth{0.462930pt}%
\definecolor{currentstroke}{rgb}{0.283072,0.130895,0.449241}%
\pgfsetstrokecolor{currentstroke}%
\pgfsetdash{}{0pt}%
\pgfpathmoveto{\pgfqpoint{5.646050in}{5.205200in}}%
\pgfpathlineto{\pgfqpoint{5.595902in}{5.204843in}}%
\pgfusepath{stroke}%
\end{pgfscope}%
\begin{pgfscope}%
\pgfpathrectangle{\pgfqpoint{3.985294in}{4.155455in}}{\pgfqpoint{2.279412in}{2.004545in}}%
\pgfusepath{clip}%
\pgfsetbuttcap%
\pgfsetroundjoin%
\pgfsetlinewidth{0.499202pt}%
\definecolor{currentstroke}{rgb}{0.281412,0.155834,0.469201}%
\pgfsetstrokecolor{currentstroke}%
\pgfsetdash{}{0pt}%
\pgfpathmoveto{\pgfqpoint{5.595902in}{5.204843in}}%
\pgfpathlineto{\pgfqpoint{5.545755in}{5.204349in}}%
\pgfusepath{stroke}%
\end{pgfscope}%
\begin{pgfscope}%
\pgfpathrectangle{\pgfqpoint{3.985294in}{4.155455in}}{\pgfqpoint{2.279412in}{2.004545in}}%
\pgfusepath{clip}%
\pgfsetbuttcap%
\pgfsetroundjoin%
\pgfsetlinewidth{0.577181pt}%
\definecolor{currentstroke}{rgb}{0.270595,0.214069,0.507052}%
\pgfsetstrokecolor{currentstroke}%
\pgfsetdash{}{0pt}%
\pgfpathmoveto{\pgfqpoint{5.545755in}{5.204349in}}%
\pgfpathlineto{\pgfqpoint{5.495607in}{5.203794in}}%
\pgfusepath{stroke}%
\end{pgfscope}%
\begin{pgfscope}%
\pgfpathrectangle{\pgfqpoint{3.985294in}{4.155455in}}{\pgfqpoint{2.279412in}{2.004545in}}%
\pgfusepath{clip}%
\pgfsetbuttcap%
\pgfsetroundjoin%
\pgfsetlinewidth{0.672362pt}%
\definecolor{currentstroke}{rgb}{0.248629,0.278775,0.534556}%
\pgfsetstrokecolor{currentstroke}%
\pgfsetdash{}{0pt}%
\pgfpathmoveto{\pgfqpoint{5.495607in}{5.203794in}}%
\pgfpathlineto{\pgfqpoint{5.445462in}{5.203067in}}%
\pgfusepath{stroke}%
\end{pgfscope}%
\begin{pgfscope}%
\pgfpathrectangle{\pgfqpoint{3.985294in}{4.155455in}}{\pgfqpoint{2.279412in}{2.004545in}}%
\pgfusepath{clip}%
\pgfsetbuttcap%
\pgfsetroundjoin%
\pgfsetlinewidth{0.774705pt}%
\definecolor{currentstroke}{rgb}{0.220057,0.343307,0.549413}%
\pgfsetstrokecolor{currentstroke}%
\pgfsetdash{}{0pt}%
\pgfpathmoveto{\pgfqpoint{5.445462in}{5.203067in}}%
\pgfpathlineto{\pgfqpoint{5.395321in}{5.202145in}}%
\pgfusepath{stroke}%
\end{pgfscope}%
\begin{pgfscope}%
\pgfpathrectangle{\pgfqpoint{3.985294in}{4.155455in}}{\pgfqpoint{2.279412in}{2.004545in}}%
\pgfusepath{clip}%
\pgfsetbuttcap%
\pgfsetroundjoin%
\pgfsetlinewidth{0.880753pt}%
\definecolor{currentstroke}{rgb}{0.192357,0.403199,0.555836}%
\pgfsetstrokecolor{currentstroke}%
\pgfsetdash{}{0pt}%
\pgfpathmoveto{\pgfqpoint{5.395321in}{5.202145in}}%
\pgfpathlineto{\pgfqpoint{5.345184in}{5.201098in}}%
\pgfusepath{stroke}%
\end{pgfscope}%
\begin{pgfscope}%
\pgfpathrectangle{\pgfqpoint{3.985294in}{4.155455in}}{\pgfqpoint{2.279412in}{2.004545in}}%
\pgfusepath{clip}%
\pgfsetbuttcap%
\pgfsetroundjoin%
\pgfsetlinewidth{0.989033pt}%
\definecolor{currentstroke}{rgb}{0.166617,0.463708,0.558119}%
\pgfsetstrokecolor{currentstroke}%
\pgfsetdash{}{0pt}%
\pgfpathmoveto{\pgfqpoint{5.345184in}{5.201098in}}%
\pgfpathlineto{\pgfqpoint{5.295052in}{5.199854in}}%
\pgfusepath{stroke}%
\end{pgfscope}%
\begin{pgfscope}%
\pgfpathrectangle{\pgfqpoint{3.985294in}{4.155455in}}{\pgfqpoint{2.279412in}{2.004545in}}%
\pgfusepath{clip}%
\pgfsetbuttcap%
\pgfsetroundjoin%
\pgfsetlinewidth{1.140974pt}%
\definecolor{currentstroke}{rgb}{0.135066,0.544853,0.554029}%
\pgfsetstrokecolor{currentstroke}%
\pgfsetdash{}{0pt}%
\pgfpathmoveto{\pgfqpoint{5.295052in}{5.199854in}}%
\pgfpathlineto{\pgfqpoint{5.244931in}{5.198317in}}%
\pgfusepath{stroke}%
\end{pgfscope}%
\begin{pgfscope}%
\pgfpathrectangle{\pgfqpoint{3.985294in}{4.155455in}}{\pgfqpoint{2.279412in}{2.004545in}}%
\pgfusepath{clip}%
\pgfsetbuttcap%
\pgfsetroundjoin%
\pgfsetlinewidth{1.207102pt}%
\definecolor{currentstroke}{rgb}{0.123463,0.581687,0.547445}%
\pgfsetstrokecolor{currentstroke}%
\pgfsetdash{}{0pt}%
\pgfpathmoveto{\pgfqpoint{5.244931in}{5.198317in}}%
\pgfpathlineto{\pgfqpoint{5.194818in}{5.196573in}}%
\pgfusepath{stroke}%
\end{pgfscope}%
\begin{pgfscope}%
\pgfpathrectangle{\pgfqpoint{3.985294in}{4.155455in}}{\pgfqpoint{2.279412in}{2.004545in}}%
\pgfusepath{clip}%
\pgfsetbuttcap%
\pgfsetroundjoin%
\pgfsetlinewidth{1.314564pt}%
\definecolor{currentstroke}{rgb}{0.123444,0.636809,0.528763}%
\pgfsetstrokecolor{currentstroke}%
\pgfsetdash{}{0pt}%
\pgfpathmoveto{\pgfqpoint{5.194818in}{5.196573in}}%
\pgfpathlineto{\pgfqpoint{5.144727in}{5.194441in}}%
\pgfusepath{stroke}%
\end{pgfscope}%
\begin{pgfscope}%
\pgfpathrectangle{\pgfqpoint{3.985294in}{4.155455in}}{\pgfqpoint{2.279412in}{2.004545in}}%
\pgfusepath{clip}%
\pgfsetbuttcap%
\pgfsetroundjoin%
\pgfsetlinewidth{1.436925pt}%
\definecolor{currentstroke}{rgb}{0.180653,0.701402,0.488189}%
\pgfsetstrokecolor{currentstroke}%
\pgfsetdash{}{0pt}%
\pgfpathmoveto{\pgfqpoint{5.144727in}{5.194441in}}%
\pgfpathlineto{\pgfqpoint{5.094663in}{5.191867in}}%
\pgfusepath{stroke}%
\end{pgfscope}%
\begin{pgfscope}%
\pgfpathrectangle{\pgfqpoint{3.985294in}{4.155455in}}{\pgfqpoint{2.279412in}{2.004545in}}%
\pgfusepath{clip}%
\pgfsetbuttcap%
\pgfsetroundjoin%
\pgfsetlinewidth{1.522158pt}%
\definecolor{currentstroke}{rgb}{0.252899,0.742211,0.448284}%
\pgfsetstrokecolor{currentstroke}%
\pgfsetdash{}{0pt}%
\pgfpathmoveto{\pgfqpoint{5.094663in}{5.191867in}}%
\pgfpathlineto{\pgfqpoint{5.044631in}{5.188882in}}%
\pgfusepath{stroke}%
\end{pgfscope}%
\begin{pgfscope}%
\pgfpathrectangle{\pgfqpoint{3.985294in}{4.155455in}}{\pgfqpoint{2.279412in}{2.004545in}}%
\pgfusepath{clip}%
\pgfsetbuttcap%
\pgfsetroundjoin%
\pgfsetlinewidth{1.498547pt}%
\definecolor{currentstroke}{rgb}{0.232815,0.732247,0.459277}%
\pgfsetstrokecolor{currentstroke}%
\pgfsetdash{}{0pt}%
\pgfpathmoveto{\pgfqpoint{5.044631in}{5.188882in}}%
\pgfpathlineto{\pgfqpoint{4.994647in}{5.185370in}}%
\pgfusepath{stroke}%
\end{pgfscope}%
\begin{pgfscope}%
\pgfpathrectangle{\pgfqpoint{3.985294in}{4.155455in}}{\pgfqpoint{2.279412in}{2.004545in}}%
\pgfusepath{clip}%
\pgfsetbuttcap%
\pgfsetroundjoin%
\pgfsetlinewidth{1.504095pt}%
\definecolor{currentstroke}{rgb}{0.239374,0.735588,0.455688}%
\pgfsetstrokecolor{currentstroke}%
\pgfsetdash{}{0pt}%
\pgfpathmoveto{\pgfqpoint{4.994647in}{5.185370in}}%
\pgfpathlineto{\pgfqpoint{4.944728in}{5.181224in}}%
\pgfusepath{stroke}%
\end{pgfscope}%
\begin{pgfscope}%
\pgfpathrectangle{\pgfqpoint{3.985294in}{4.155455in}}{\pgfqpoint{2.279412in}{2.004545in}}%
\pgfusepath{clip}%
\pgfsetbuttcap%
\pgfsetroundjoin%
\pgfsetlinewidth{0.313669pt}%
\definecolor{currentstroke}{rgb}{0.268510,0.009605,0.335427}%
\pgfsetstrokecolor{currentstroke}%
\pgfsetdash{}{0pt}%
\pgfpathmoveto{\pgfqpoint{5.996962in}{5.293048in}}%
\pgfpathlineto{\pgfqpoint{5.947063in}{5.291326in}}%
\pgfusepath{stroke}%
\end{pgfscope}%
\begin{pgfscope}%
\pgfpathrectangle{\pgfqpoint{3.985294in}{4.155455in}}{\pgfqpoint{2.279412in}{2.004545in}}%
\pgfusepath{clip}%
\pgfsetbuttcap%
\pgfsetroundjoin%
\pgfsetlinewidth{0.318996pt}%
\definecolor{currentstroke}{rgb}{0.269944,0.014625,0.341379}%
\pgfsetstrokecolor{currentstroke}%
\pgfsetdash{}{0pt}%
\pgfpathmoveto{\pgfqpoint{5.947063in}{5.291326in}}%
\pgfpathlineto{\pgfqpoint{5.896935in}{5.290978in}}%
\pgfusepath{stroke}%
\end{pgfscope}%
\begin{pgfscope}%
\pgfpathrectangle{\pgfqpoint{3.985294in}{4.155455in}}{\pgfqpoint{2.279412in}{2.004545in}}%
\pgfusepath{clip}%
\pgfsetbuttcap%
\pgfsetroundjoin%
\pgfsetlinewidth{0.333422pt}%
\definecolor{currentstroke}{rgb}{0.272594,0.025563,0.353093}%
\pgfsetstrokecolor{currentstroke}%
\pgfsetdash{}{0pt}%
\pgfpathmoveto{\pgfqpoint{5.896935in}{5.290978in}}%
\pgfpathlineto{\pgfqpoint{5.846801in}{5.290198in}}%
\pgfusepath{stroke}%
\end{pgfscope}%
\begin{pgfscope}%
\pgfpathrectangle{\pgfqpoint{3.985294in}{4.155455in}}{\pgfqpoint{2.279412in}{2.004545in}}%
\pgfusepath{clip}%
\pgfsetbuttcap%
\pgfsetroundjoin%
\pgfsetlinewidth{0.334839pt}%
\definecolor{currentstroke}{rgb}{0.272594,0.025563,0.353093}%
\pgfsetstrokecolor{currentstroke}%
\pgfsetdash{}{0pt}%
\pgfpathmoveto{\pgfqpoint{5.846801in}{5.290198in}}%
\pgfpathlineto{\pgfqpoint{5.796652in}{5.290024in}}%
\pgfusepath{stroke}%
\end{pgfscope}%
\begin{pgfscope}%
\pgfpathrectangle{\pgfqpoint{3.985294in}{4.155455in}}{\pgfqpoint{2.279412in}{2.004545in}}%
\pgfusepath{clip}%
\pgfsetbuttcap%
\pgfsetroundjoin%
\pgfsetlinewidth{0.346950pt}%
\definecolor{currentstroke}{rgb}{0.274952,0.037752,0.364543}%
\pgfsetstrokecolor{currentstroke}%
\pgfsetdash{}{0pt}%
\pgfpathmoveto{\pgfqpoint{5.796652in}{5.290024in}}%
\pgfpathlineto{\pgfqpoint{5.746508in}{5.289578in}}%
\pgfusepath{stroke}%
\end{pgfscope}%
\begin{pgfscope}%
\pgfpathrectangle{\pgfqpoint{3.985294in}{4.155455in}}{\pgfqpoint{2.279412in}{2.004545in}}%
\pgfusepath{clip}%
\pgfsetbuttcap%
\pgfsetroundjoin%
\pgfsetlinewidth{0.366202pt}%
\definecolor{currentstroke}{rgb}{0.277941,0.056324,0.381191}%
\pgfsetstrokecolor{currentstroke}%
\pgfsetdash{}{0pt}%
\pgfpathmoveto{\pgfqpoint{5.746508in}{5.289578in}}%
\pgfpathlineto{\pgfqpoint{5.696368in}{5.288795in}}%
\pgfusepath{stroke}%
\end{pgfscope}%
\begin{pgfscope}%
\pgfpathrectangle{\pgfqpoint{3.985294in}{4.155455in}}{\pgfqpoint{2.279412in}{2.004545in}}%
\pgfusepath{clip}%
\pgfsetbuttcap%
\pgfsetroundjoin%
\pgfsetlinewidth{0.400406pt}%
\definecolor{currentstroke}{rgb}{0.281446,0.084320,0.407414}%
\pgfsetstrokecolor{currentstroke}%
\pgfsetdash{}{0pt}%
\pgfpathmoveto{\pgfqpoint{5.696368in}{5.288795in}}%
\pgfpathlineto{\pgfqpoint{5.646233in}{5.287811in}}%
\pgfusepath{stroke}%
\end{pgfscope}%
\begin{pgfscope}%
\pgfpathrectangle{\pgfqpoint{3.985294in}{4.155455in}}{\pgfqpoint{2.279412in}{2.004545in}}%
\pgfusepath{clip}%
\pgfsetbuttcap%
\pgfsetroundjoin%
\pgfsetlinewidth{0.432981pt}%
\definecolor{currentstroke}{rgb}{0.283091,0.110553,0.431554}%
\pgfsetstrokecolor{currentstroke}%
\pgfsetdash{}{0pt}%
\pgfpathmoveto{\pgfqpoint{5.646233in}{5.287811in}}%
\pgfpathlineto{\pgfqpoint{5.596100in}{5.286682in}}%
\pgfusepath{stroke}%
\end{pgfscope}%
\begin{pgfscope}%
\pgfpathrectangle{\pgfqpoint{3.985294in}{4.155455in}}{\pgfqpoint{2.279412in}{2.004545in}}%
\pgfusepath{clip}%
\pgfsetbuttcap%
\pgfsetroundjoin%
\pgfsetlinewidth{0.489770pt}%
\definecolor{currentstroke}{rgb}{0.281887,0.150881,0.465405}%
\pgfsetstrokecolor{currentstroke}%
\pgfsetdash{}{0pt}%
\pgfpathmoveto{\pgfqpoint{5.596100in}{5.286682in}}%
\pgfpathlineto{\pgfqpoint{5.545970in}{5.285408in}}%
\pgfusepath{stroke}%
\end{pgfscope}%
\begin{pgfscope}%
\pgfpathrectangle{\pgfqpoint{3.985294in}{4.155455in}}{\pgfqpoint{2.279412in}{2.004545in}}%
\pgfusepath{clip}%
\pgfsetbuttcap%
\pgfsetroundjoin%
\pgfsetlinewidth{0.538401pt}%
\definecolor{currentstroke}{rgb}{0.277134,0.185228,0.489898}%
\pgfsetstrokecolor{currentstroke}%
\pgfsetdash{}{0pt}%
\pgfpathmoveto{\pgfqpoint{5.545970in}{5.285408in}}%
\pgfpathlineto{\pgfqpoint{5.495853in}{5.283786in}}%
\pgfusepath{stroke}%
\end{pgfscope}%
\begin{pgfscope}%
\pgfpathrectangle{\pgfqpoint{3.985294in}{4.155455in}}{\pgfqpoint{2.279412in}{2.004545in}}%
\pgfusepath{clip}%
\pgfsetbuttcap%
\pgfsetroundjoin%
\pgfsetlinewidth{0.625824pt}%
\definecolor{currentstroke}{rgb}{0.260571,0.246922,0.522828}%
\pgfsetstrokecolor{currentstroke}%
\pgfsetdash{}{0pt}%
\pgfpathmoveto{\pgfqpoint{5.495853in}{5.283786in}}%
\pgfpathlineto{\pgfqpoint{5.445742in}{5.282034in}}%
\pgfusepath{stroke}%
\end{pgfscope}%
\begin{pgfscope}%
\pgfpathrectangle{\pgfqpoint{3.985294in}{4.155455in}}{\pgfqpoint{2.279412in}{2.004545in}}%
\pgfusepath{clip}%
\pgfsetbuttcap%
\pgfsetroundjoin%
\pgfsetlinewidth{0.702021pt}%
\definecolor{currentstroke}{rgb}{0.241237,0.296485,0.539709}%
\pgfsetstrokecolor{currentstroke}%
\pgfsetdash{}{0pt}%
\pgfpathmoveto{\pgfqpoint{5.445742in}{5.282034in}}%
\pgfpathlineto{\pgfqpoint{5.395650in}{5.279927in}}%
\pgfusepath{stroke}%
\end{pgfscope}%
\begin{pgfscope}%
\pgfpathrectangle{\pgfqpoint{3.985294in}{4.155455in}}{\pgfqpoint{2.279412in}{2.004545in}}%
\pgfusepath{clip}%
\pgfsetbuttcap%
\pgfsetroundjoin%
\pgfsetlinewidth{0.787501pt}%
\definecolor{currentstroke}{rgb}{0.218130,0.347432,0.550038}%
\pgfsetstrokecolor{currentstroke}%
\pgfsetdash{}{0pt}%
\pgfpathmoveto{\pgfqpoint{5.395650in}{5.279927in}}%
\pgfpathlineto{\pgfqpoint{5.345578in}{5.277440in}}%
\pgfusepath{stroke}%
\end{pgfscope}%
\begin{pgfscope}%
\pgfpathrectangle{\pgfqpoint{3.985294in}{4.155455in}}{\pgfqpoint{2.279412in}{2.004545in}}%
\pgfusepath{clip}%
\pgfsetbuttcap%
\pgfsetroundjoin%
\pgfsetlinewidth{0.881833pt}%
\definecolor{currentstroke}{rgb}{0.192357,0.403199,0.555836}%
\pgfsetstrokecolor{currentstroke}%
\pgfsetdash{}{0pt}%
\pgfpathmoveto{\pgfqpoint{5.345578in}{5.277440in}}%
\pgfpathlineto{\pgfqpoint{5.295516in}{5.274801in}}%
\pgfusepath{stroke}%
\end{pgfscope}%
\begin{pgfscope}%
\pgfpathrectangle{\pgfqpoint{3.985294in}{4.155455in}}{\pgfqpoint{2.279412in}{2.004545in}}%
\pgfusepath{clip}%
\pgfsetbuttcap%
\pgfsetroundjoin%
\pgfsetlinewidth{0.952325pt}%
\definecolor{currentstroke}{rgb}{0.175841,0.441290,0.557685}%
\pgfsetstrokecolor{currentstroke}%
\pgfsetdash{}{0pt}%
\pgfpathmoveto{\pgfqpoint{5.295516in}{5.274801in}}%
\pgfpathlineto{\pgfqpoint{5.245483in}{5.271790in}}%
\pgfusepath{stroke}%
\end{pgfscope}%
\begin{pgfscope}%
\pgfpathrectangle{\pgfqpoint{3.985294in}{4.155455in}}{\pgfqpoint{2.279412in}{2.004545in}}%
\pgfusepath{clip}%
\pgfsetbuttcap%
\pgfsetroundjoin%
\pgfsetlinewidth{1.066179pt}%
\definecolor{currentstroke}{rgb}{0.150476,0.504369,0.557430}%
\pgfsetstrokecolor{currentstroke}%
\pgfsetdash{}{0pt}%
\pgfpathmoveto{\pgfqpoint{5.245483in}{5.271790in}}%
\pgfpathlineto{\pgfqpoint{5.195519in}{5.268027in}}%
\pgfusepath{stroke}%
\end{pgfscope}%
\begin{pgfscope}%
\pgfpathrectangle{\pgfqpoint{3.985294in}{4.155455in}}{\pgfqpoint{2.279412in}{2.004545in}}%
\pgfusepath{clip}%
\pgfsetbuttcap%
\pgfsetroundjoin%
\pgfsetlinewidth{1.099460pt}%
\definecolor{currentstroke}{rgb}{0.143343,0.522773,0.556295}%
\pgfsetstrokecolor{currentstroke}%
\pgfsetdash{}{0pt}%
\pgfpathmoveto{\pgfqpoint{5.195519in}{5.268027in}}%
\pgfpathlineto{\pgfqpoint{5.145632in}{5.263536in}}%
\pgfusepath{stroke}%
\end{pgfscope}%
\begin{pgfscope}%
\pgfpathrectangle{\pgfqpoint{3.985294in}{4.155455in}}{\pgfqpoint{2.279412in}{2.004545in}}%
\pgfusepath{clip}%
\pgfsetbuttcap%
\pgfsetroundjoin%
\pgfsetlinewidth{1.140245pt}%
\definecolor{currentstroke}{rgb}{0.135066,0.544853,0.554029}%
\pgfsetstrokecolor{currentstroke}%
\pgfsetdash{}{0pt}%
\pgfpathmoveto{\pgfqpoint{5.145632in}{5.263536in}}%
\pgfpathlineto{\pgfqpoint{5.095843in}{5.258283in}}%
\pgfusepath{stroke}%
\end{pgfscope}%
\begin{pgfscope}%
\pgfpathrectangle{\pgfqpoint{3.985294in}{4.155455in}}{\pgfqpoint{2.279412in}{2.004545in}}%
\pgfusepath{clip}%
\pgfsetbuttcap%
\pgfsetroundjoin%
\pgfsetlinewidth{1.262622pt}%
\definecolor{currentstroke}{rgb}{0.119423,0.611141,0.538982}%
\pgfsetstrokecolor{currentstroke}%
\pgfsetdash{}{0pt}%
\pgfpathmoveto{\pgfqpoint{5.095843in}{5.258283in}}%
\pgfpathlineto{\pgfqpoint{5.046237in}{5.251866in}}%
\pgfusepath{stroke}%
\end{pgfscope}%
\begin{pgfscope}%
\pgfpathrectangle{\pgfqpoint{3.985294in}{4.155455in}}{\pgfqpoint{2.279412in}{2.004545in}}%
\pgfusepath{clip}%
\pgfsetbuttcap%
\pgfsetroundjoin%
\pgfsetlinewidth{1.204529pt}%
\definecolor{currentstroke}{rgb}{0.124395,0.578002,0.548287}%
\pgfsetstrokecolor{currentstroke}%
\pgfsetdash{}{0pt}%
\pgfpathmoveto{\pgfqpoint{5.046237in}{5.251866in}}%
\pgfpathlineto{\pgfqpoint{4.996939in}{5.243883in}}%
\pgfusepath{stroke}%
\end{pgfscope}%
\begin{pgfscope}%
\pgfpathrectangle{\pgfqpoint{3.985294in}{4.155455in}}{\pgfqpoint{2.279412in}{2.004545in}}%
\pgfusepath{clip}%
\pgfsetbuttcap%
\pgfsetroundjoin%
\pgfsetlinewidth{1.175857pt}%
\definecolor{currentstroke}{rgb}{0.128729,0.563265,0.551229}%
\pgfsetstrokecolor{currentstroke}%
\pgfsetdash{}{0pt}%
\pgfpathmoveto{\pgfqpoint{4.996939in}{5.243883in}}%
\pgfpathlineto{\pgfqpoint{4.948039in}{5.234176in}}%
\pgfusepath{stroke}%
\end{pgfscope}%
\begin{pgfscope}%
\pgfpathrectangle{\pgfqpoint{3.985294in}{4.155455in}}{\pgfqpoint{2.279412in}{2.004545in}}%
\pgfusepath{clip}%
\pgfsetbuttcap%
\pgfsetroundjoin%
\pgfsetlinewidth{0.308226pt}%
\definecolor{currentstroke}{rgb}{0.268510,0.009605,0.335427}%
\pgfsetstrokecolor{currentstroke}%
\pgfsetdash{}{0pt}%
\pgfpathmoveto{\pgfqpoint{5.996962in}{5.338154in}}%
\pgfpathlineto{\pgfqpoint{5.947059in}{5.336327in}}%
\pgfusepath{stroke}%
\end{pgfscope}%
\begin{pgfscope}%
\pgfpathrectangle{\pgfqpoint{3.985294in}{4.155455in}}{\pgfqpoint{2.279412in}{2.004545in}}%
\pgfusepath{clip}%
\pgfsetbuttcap%
\pgfsetroundjoin%
\pgfsetlinewidth{0.314576pt}%
\definecolor{currentstroke}{rgb}{0.268510,0.009605,0.335427}%
\pgfsetstrokecolor{currentstroke}%
\pgfsetdash{}{0pt}%
\pgfpathmoveto{\pgfqpoint{5.947059in}{5.336327in}}%
\pgfpathlineto{\pgfqpoint{5.897016in}{5.334800in}}%
\pgfusepath{stroke}%
\end{pgfscope}%
\begin{pgfscope}%
\pgfpathrectangle{\pgfqpoint{3.985294in}{4.155455in}}{\pgfqpoint{2.279412in}{2.004545in}}%
\pgfusepath{clip}%
\pgfsetbuttcap%
\pgfsetroundjoin%
\pgfsetlinewidth{0.321728pt}%
\definecolor{currentstroke}{rgb}{0.271305,0.019942,0.347269}%
\pgfsetstrokecolor{currentstroke}%
\pgfsetdash{}{0pt}%
\pgfpathmoveto{\pgfqpoint{5.897016in}{5.334800in}}%
\pgfpathlineto{\pgfqpoint{5.846929in}{5.332851in}}%
\pgfusepath{stroke}%
\end{pgfscope}%
\begin{pgfscope}%
\pgfpathrectangle{\pgfqpoint{3.985294in}{4.155455in}}{\pgfqpoint{2.279412in}{2.004545in}}%
\pgfusepath{clip}%
\pgfsetbuttcap%
\pgfsetroundjoin%
\pgfsetlinewidth{0.342718pt}%
\definecolor{currentstroke}{rgb}{0.274952,0.037752,0.364543}%
\pgfsetstrokecolor{currentstroke}%
\pgfsetdash{}{0pt}%
\pgfpathmoveto{\pgfqpoint{5.846929in}{5.332851in}}%
\pgfpathlineto{\pgfqpoint{5.796816in}{5.331206in}}%
\pgfusepath{stroke}%
\end{pgfscope}%
\begin{pgfscope}%
\pgfpathrectangle{\pgfqpoint{3.985294in}{4.155455in}}{\pgfqpoint{2.279412in}{2.004545in}}%
\pgfusepath{clip}%
\pgfsetbuttcap%
\pgfsetroundjoin%
\pgfsetlinewidth{0.352134pt}%
\definecolor{currentstroke}{rgb}{0.276022,0.044167,0.370164}%
\pgfsetstrokecolor{currentstroke}%
\pgfsetdash{}{0pt}%
\pgfpathmoveto{\pgfqpoint{5.796816in}{5.331206in}}%
\pgfpathlineto{\pgfqpoint{5.746687in}{5.330236in}}%
\pgfusepath{stroke}%
\end{pgfscope}%
\begin{pgfscope}%
\pgfpathrectangle{\pgfqpoint{3.985294in}{4.155455in}}{\pgfqpoint{2.279412in}{2.004545in}}%
\pgfusepath{clip}%
\pgfsetbuttcap%
\pgfsetroundjoin%
\pgfsetlinewidth{0.367105pt}%
\definecolor{currentstroke}{rgb}{0.277941,0.056324,0.381191}%
\pgfsetstrokecolor{currentstroke}%
\pgfsetdash{}{0pt}%
\pgfpathmoveto{\pgfqpoint{5.746687in}{5.330236in}}%
\pgfpathlineto{\pgfqpoint{5.696559in}{5.329208in}}%
\pgfusepath{stroke}%
\end{pgfscope}%
\begin{pgfscope}%
\pgfpathrectangle{\pgfqpoint{3.985294in}{4.155455in}}{\pgfqpoint{2.279412in}{2.004545in}}%
\pgfusepath{clip}%
\pgfsetbuttcap%
\pgfsetroundjoin%
\pgfsetlinewidth{0.391386pt}%
\definecolor{currentstroke}{rgb}{0.280894,0.078907,0.402329}%
\pgfsetstrokecolor{currentstroke}%
\pgfsetdash{}{0pt}%
\pgfpathmoveto{\pgfqpoint{5.696559in}{5.329208in}}%
\pgfpathlineto{\pgfqpoint{5.646434in}{5.327852in}}%
\pgfusepath{stroke}%
\end{pgfscope}%
\begin{pgfscope}%
\pgfpathrectangle{\pgfqpoint{3.985294in}{4.155455in}}{\pgfqpoint{2.279412in}{2.004545in}}%
\pgfusepath{clip}%
\pgfsetbuttcap%
\pgfsetroundjoin%
\pgfsetlinewidth{0.425059pt}%
\definecolor{currentstroke}{rgb}{0.282910,0.105393,0.426902}%
\pgfsetstrokecolor{currentstroke}%
\pgfsetdash{}{0pt}%
\pgfpathmoveto{\pgfqpoint{5.646434in}{5.327852in}}%
\pgfpathlineto{\pgfqpoint{5.596308in}{5.326497in}}%
\pgfusepath{stroke}%
\end{pgfscope}%
\begin{pgfscope}%
\pgfpathrectangle{\pgfqpoint{3.985294in}{4.155455in}}{\pgfqpoint{2.279412in}{2.004545in}}%
\pgfusepath{clip}%
\pgfsetbuttcap%
\pgfsetroundjoin%
\pgfsetlinewidth{0.470696pt}%
\definecolor{currentstroke}{rgb}{0.282884,0.135920,0.453427}%
\pgfsetstrokecolor{currentstroke}%
\pgfsetdash{}{0pt}%
\pgfpathmoveto{\pgfqpoint{5.596308in}{5.326497in}}%
\pgfpathlineto{\pgfqpoint{5.546202in}{5.324628in}}%
\pgfusepath{stroke}%
\end{pgfscope}%
\begin{pgfscope}%
\pgfpathrectangle{\pgfqpoint{3.985294in}{4.155455in}}{\pgfqpoint{2.279412in}{2.004545in}}%
\pgfusepath{clip}%
\pgfsetbuttcap%
\pgfsetroundjoin%
\pgfsetlinewidth{0.523522pt}%
\definecolor{currentstroke}{rgb}{0.278826,0.175490,0.483397}%
\pgfsetstrokecolor{currentstroke}%
\pgfsetdash{}{0pt}%
\pgfpathmoveto{\pgfqpoint{5.546202in}{5.324628in}}%
\pgfpathlineto{\pgfqpoint{5.496105in}{5.322573in}}%
\pgfusepath{stroke}%
\end{pgfscope}%
\begin{pgfscope}%
\pgfpathrectangle{\pgfqpoint{3.985294in}{4.155455in}}{\pgfqpoint{2.279412in}{2.004545in}}%
\pgfusepath{clip}%
\pgfsetbuttcap%
\pgfsetroundjoin%
\pgfsetlinewidth{0.569014pt}%
\definecolor{currentstroke}{rgb}{0.271828,0.209303,0.504434}%
\pgfsetstrokecolor{currentstroke}%
\pgfsetdash{}{0pt}%
\pgfpathmoveto{\pgfqpoint{5.496105in}{5.322573in}}%
\pgfpathlineto{\pgfqpoint{5.446023in}{5.320267in}}%
\pgfusepath{stroke}%
\end{pgfscope}%
\begin{pgfscope}%
\pgfpathrectangle{\pgfqpoint{3.985294in}{4.155455in}}{\pgfqpoint{2.279412in}{2.004545in}}%
\pgfusepath{clip}%
\pgfsetbuttcap%
\pgfsetroundjoin%
\pgfsetlinewidth{0.654261pt}%
\definecolor{currentstroke}{rgb}{0.253935,0.265254,0.529983}%
\pgfsetstrokecolor{currentstroke}%
\pgfsetdash{}{0pt}%
\pgfpathmoveto{\pgfqpoint{5.446023in}{5.320267in}}%
\pgfpathlineto{\pgfqpoint{5.395962in}{5.317607in}}%
\pgfusepath{stroke}%
\end{pgfscope}%
\begin{pgfscope}%
\pgfpathrectangle{\pgfqpoint{3.985294in}{4.155455in}}{\pgfqpoint{2.279412in}{2.004545in}}%
\pgfusepath{clip}%
\pgfsetbuttcap%
\pgfsetroundjoin%
\pgfsetlinewidth{0.315650pt}%
\definecolor{currentstroke}{rgb}{0.269944,0.014625,0.341379}%
\pgfsetstrokecolor{currentstroke}%
\pgfsetdash{}{0pt}%
\pgfpathmoveto{\pgfqpoint{5.945670in}{4.841980in}}%
\pgfpathlineto{\pgfqpoint{5.900564in}{4.842537in}}%
\pgfusepath{stroke}%
\end{pgfscope}%
\begin{pgfscope}%
\pgfpathrectangle{\pgfqpoint{3.985294in}{4.155455in}}{\pgfqpoint{2.279412in}{2.004545in}}%
\pgfusepath{clip}%
\pgfsetbuttcap%
\pgfsetroundjoin%
\pgfsetlinewidth{0.323755pt}%
\definecolor{currentstroke}{rgb}{0.271305,0.019942,0.347269}%
\pgfsetstrokecolor{currentstroke}%
\pgfsetdash{}{0pt}%
\pgfpathmoveto{\pgfqpoint{5.900564in}{4.842537in}}%
\pgfpathlineto{\pgfqpoint{5.851156in}{4.843061in}}%
\pgfusepath{stroke}%
\end{pgfscope}%
\begin{pgfscope}%
\pgfpathrectangle{\pgfqpoint{3.985294in}{4.155455in}}{\pgfqpoint{2.279412in}{2.004545in}}%
\pgfusepath{clip}%
\pgfsetbuttcap%
\pgfsetroundjoin%
\pgfsetlinewidth{0.330659pt}%
\definecolor{currentstroke}{rgb}{0.272594,0.025563,0.353093}%
\pgfsetstrokecolor{currentstroke}%
\pgfsetdash{}{0pt}%
\pgfpathmoveto{\pgfqpoint{5.851156in}{4.843061in}}%
\pgfpathlineto{\pgfqpoint{5.801100in}{4.844655in}}%
\pgfusepath{stroke}%
\end{pgfscope}%
\begin{pgfscope}%
\pgfpathrectangle{\pgfqpoint{3.985294in}{4.155455in}}{\pgfqpoint{2.279412in}{2.004545in}}%
\pgfusepath{clip}%
\pgfsetbuttcap%
\pgfsetroundjoin%
\pgfsetlinewidth{0.331070pt}%
\definecolor{currentstroke}{rgb}{0.272594,0.025563,0.353093}%
\pgfsetstrokecolor{currentstroke}%
\pgfsetdash{}{0pt}%
\pgfpathmoveto{\pgfqpoint{5.801100in}{4.844655in}}%
\pgfpathlineto{\pgfqpoint{5.751054in}{4.846864in}}%
\pgfusepath{stroke}%
\end{pgfscope}%
\begin{pgfscope}%
\pgfpathrectangle{\pgfqpoint{3.985294in}{4.155455in}}{\pgfqpoint{2.279412in}{2.004545in}}%
\pgfusepath{clip}%
\pgfsetbuttcap%
\pgfsetroundjoin%
\pgfsetlinewidth{0.352729pt}%
\definecolor{currentstroke}{rgb}{0.276022,0.044167,0.370164}%
\pgfsetstrokecolor{currentstroke}%
\pgfsetdash{}{0pt}%
\pgfpathmoveto{\pgfqpoint{5.751054in}{4.846864in}}%
\pgfpathlineto{\pgfqpoint{5.700952in}{4.848414in}}%
\pgfusepath{stroke}%
\end{pgfscope}%
\begin{pgfscope}%
\pgfpathrectangle{\pgfqpoint{3.985294in}{4.155455in}}{\pgfqpoint{2.279412in}{2.004545in}}%
\pgfusepath{clip}%
\pgfsetbuttcap%
\pgfsetroundjoin%
\pgfsetlinewidth{0.363616pt}%
\definecolor{currentstroke}{rgb}{0.277941,0.056324,0.381191}%
\pgfsetstrokecolor{currentstroke}%
\pgfsetdash{}{0pt}%
\pgfpathmoveto{\pgfqpoint{5.700952in}{4.848414in}}%
\pgfpathlineto{\pgfqpoint{5.650868in}{4.850520in}}%
\pgfusepath{stroke}%
\end{pgfscope}%
\begin{pgfscope}%
\pgfpathrectangle{\pgfqpoint{3.985294in}{4.155455in}}{\pgfqpoint{2.279412in}{2.004545in}}%
\pgfusepath{clip}%
\pgfsetbuttcap%
\pgfsetroundjoin%
\pgfsetlinewidth{0.387641pt}%
\definecolor{currentstroke}{rgb}{0.280267,0.073417,0.397163}%
\pgfsetstrokecolor{currentstroke}%
\pgfsetdash{}{0pt}%
\pgfpathmoveto{\pgfqpoint{5.650868in}{4.850520in}}%
\pgfpathlineto{\pgfqpoint{5.600798in}{4.852988in}}%
\pgfusepath{stroke}%
\end{pgfscope}%
\begin{pgfscope}%
\pgfpathrectangle{\pgfqpoint{3.985294in}{4.155455in}}{\pgfqpoint{2.279412in}{2.004545in}}%
\pgfusepath{clip}%
\pgfsetbuttcap%
\pgfsetroundjoin%
\pgfsetlinewidth{0.407921pt}%
\definecolor{currentstroke}{rgb}{0.281924,0.089666,0.412415}%
\pgfsetstrokecolor{currentstroke}%
\pgfsetdash{}{0pt}%
\pgfpathmoveto{\pgfqpoint{5.600798in}{4.852988in}}%
\pgfpathlineto{\pgfqpoint{5.550741in}{4.855683in}}%
\pgfusepath{stroke}%
\end{pgfscope}%
\begin{pgfscope}%
\pgfpathrectangle{\pgfqpoint{3.985294in}{4.155455in}}{\pgfqpoint{2.279412in}{2.004545in}}%
\pgfusepath{clip}%
\pgfsetbuttcap%
\pgfsetroundjoin%
\pgfsetlinewidth{0.472312pt}%
\definecolor{currentstroke}{rgb}{0.282884,0.135920,0.453427}%
\pgfsetstrokecolor{currentstroke}%
\pgfsetdash{}{0pt}%
\pgfpathmoveto{\pgfqpoint{5.550741in}{4.855683in}}%
\pgfpathlineto{\pgfqpoint{5.500712in}{4.858768in}}%
\pgfusepath{stroke}%
\end{pgfscope}%
\begin{pgfscope}%
\pgfpathrectangle{\pgfqpoint{3.985294in}{4.155455in}}{\pgfqpoint{2.279412in}{2.004545in}}%
\pgfusepath{clip}%
\pgfsetbuttcap%
\pgfsetroundjoin%
\pgfsetlinewidth{0.486071pt}%
\definecolor{currentstroke}{rgb}{0.282290,0.145912,0.461510}%
\pgfsetstrokecolor{currentstroke}%
\pgfsetdash{}{0pt}%
\pgfpathmoveto{\pgfqpoint{5.500712in}{4.858768in}}%
\pgfpathlineto{\pgfqpoint{5.450733in}{4.862406in}}%
\pgfusepath{stroke}%
\end{pgfscope}%
\begin{pgfscope}%
\pgfpathrectangle{\pgfqpoint{3.985294in}{4.155455in}}{\pgfqpoint{2.279412in}{2.004545in}}%
\pgfusepath{clip}%
\pgfsetbuttcap%
\pgfsetroundjoin%
\pgfsetlinewidth{0.529770pt}%
\definecolor{currentstroke}{rgb}{0.278012,0.180367,0.486697}%
\pgfsetstrokecolor{currentstroke}%
\pgfsetdash{}{0pt}%
\pgfpathmoveto{\pgfqpoint{5.450733in}{4.862406in}}%
\pgfpathlineto{\pgfqpoint{5.400810in}{4.866609in}}%
\pgfusepath{stroke}%
\end{pgfscope}%
\begin{pgfscope}%
\pgfpathrectangle{\pgfqpoint{3.985294in}{4.155455in}}{\pgfqpoint{2.279412in}{2.004545in}}%
\pgfusepath{clip}%
\pgfsetbuttcap%
\pgfsetroundjoin%
\pgfsetlinewidth{0.588247pt}%
\definecolor{currentstroke}{rgb}{0.269308,0.218818,0.509577}%
\pgfsetstrokecolor{currentstroke}%
\pgfsetdash{}{0pt}%
\pgfpathmoveto{\pgfqpoint{5.400810in}{4.866609in}}%
\pgfpathlineto{\pgfqpoint{5.350969in}{4.871477in}}%
\pgfusepath{stroke}%
\end{pgfscope}%
\begin{pgfscope}%
\pgfpathrectangle{\pgfqpoint{3.985294in}{4.155455in}}{\pgfqpoint{2.279412in}{2.004545in}}%
\pgfusepath{clip}%
\pgfsetbuttcap%
\pgfsetroundjoin%
\pgfsetlinewidth{0.646512pt}%
\definecolor{currentstroke}{rgb}{0.255645,0.260703,0.528312}%
\pgfsetstrokecolor{currentstroke}%
\pgfsetdash{}{0pt}%
\pgfpathmoveto{\pgfqpoint{5.350969in}{4.871477in}}%
\pgfpathlineto{\pgfqpoint{5.301232in}{4.877118in}}%
\pgfusepath{stroke}%
\end{pgfscope}%
\begin{pgfscope}%
\pgfpathrectangle{\pgfqpoint{3.985294in}{4.155455in}}{\pgfqpoint{2.279412in}{2.004545in}}%
\pgfusepath{clip}%
\pgfsetbuttcap%
\pgfsetroundjoin%
\pgfsetlinewidth{0.682462pt}%
\definecolor{currentstroke}{rgb}{0.246811,0.283237,0.535941}%
\pgfsetstrokecolor{currentstroke}%
\pgfsetdash{}{0pt}%
\pgfpathmoveto{\pgfqpoint{5.301232in}{4.877118in}}%
\pgfpathlineto{\pgfqpoint{5.251608in}{4.883480in}}%
\pgfusepath{stroke}%
\end{pgfscope}%
\begin{pgfscope}%
\pgfpathrectangle{\pgfqpoint{3.985294in}{4.155455in}}{\pgfqpoint{2.279412in}{2.004545in}}%
\pgfusepath{clip}%
\pgfsetbuttcap%
\pgfsetroundjoin%
\pgfsetlinewidth{0.697437pt}%
\definecolor{currentstroke}{rgb}{0.243113,0.292092,0.538516}%
\pgfsetstrokecolor{currentstroke}%
\pgfsetdash{}{0pt}%
\pgfpathmoveto{\pgfqpoint{5.251608in}{4.883480in}}%
\pgfpathlineto{\pgfqpoint{5.202203in}{4.891004in}}%
\pgfusepath{stroke}%
\end{pgfscope}%
\begin{pgfscope}%
\pgfpathrectangle{\pgfqpoint{3.985294in}{4.155455in}}{\pgfqpoint{2.279412in}{2.004545in}}%
\pgfusepath{clip}%
\pgfsetbuttcap%
\pgfsetroundjoin%
\pgfsetlinewidth{0.709698pt}%
\definecolor{currentstroke}{rgb}{0.239346,0.300855,0.540844}%
\pgfsetstrokecolor{currentstroke}%
\pgfsetdash{}{0pt}%
\pgfpathmoveto{\pgfqpoint{5.202203in}{4.891004in}}%
\pgfpathlineto{\pgfqpoint{5.153080in}{4.899853in}}%
\pgfusepath{stroke}%
\end{pgfscope}%
\begin{pgfscope}%
\pgfpathrectangle{\pgfqpoint{3.985294in}{4.155455in}}{\pgfqpoint{2.279412in}{2.004545in}}%
\pgfusepath{clip}%
\pgfsetbuttcap%
\pgfsetroundjoin%
\pgfsetlinewidth{0.756834pt}%
\definecolor{currentstroke}{rgb}{0.225863,0.330805,0.547314}%
\pgfsetstrokecolor{currentstroke}%
\pgfsetdash{}{0pt}%
\pgfpathmoveto{\pgfqpoint{5.153080in}{4.899853in}}%
\pgfpathlineto{\pgfqpoint{5.104394in}{4.910371in}}%
\pgfusepath{stroke}%
\end{pgfscope}%
\begin{pgfscope}%
\pgfpathrectangle{\pgfqpoint{3.985294in}{4.155455in}}{\pgfqpoint{2.279412in}{2.004545in}}%
\pgfusepath{clip}%
\pgfsetbuttcap%
\pgfsetroundjoin%
\pgfsetlinewidth{0.796932pt}%
\definecolor{currentstroke}{rgb}{0.214298,0.355619,0.551184}%
\pgfsetstrokecolor{currentstroke}%
\pgfsetdash{}{0pt}%
\pgfpathmoveto{\pgfqpoint{5.104394in}{4.910371in}}%
\pgfpathlineto{\pgfqpoint{5.056816in}{4.924121in}}%
\pgfusepath{stroke}%
\end{pgfscope}%
\begin{pgfscope}%
\pgfpathrectangle{\pgfqpoint{3.985294in}{4.155455in}}{\pgfqpoint{2.279412in}{2.004545in}}%
\pgfusepath{clip}%
\pgfsetbuttcap%
\pgfsetroundjoin%
\pgfsetlinewidth{0.706691pt}%
\definecolor{currentstroke}{rgb}{0.239346,0.300855,0.540844}%
\pgfsetstrokecolor{currentstroke}%
\pgfsetdash{}{0pt}%
\pgfpathmoveto{\pgfqpoint{5.056816in}{4.924121in}}%
\pgfpathlineto{\pgfqpoint{5.010893in}{4.941687in}}%
\pgfusepath{stroke}%
\end{pgfscope}%
\begin{pgfscope}%
\pgfpathrectangle{\pgfqpoint{3.985294in}{4.155455in}}{\pgfqpoint{2.279412in}{2.004545in}}%
\pgfusepath{clip}%
\pgfsetbuttcap%
\pgfsetroundjoin%
\pgfsetlinewidth{0.810312pt}%
\definecolor{currentstroke}{rgb}{0.210503,0.363727,0.552206}%
\pgfsetstrokecolor{currentstroke}%
\pgfsetdash{}{0pt}%
\pgfpathmoveto{\pgfqpoint{5.010893in}{4.941687in}}%
\pgfpathlineto{\pgfqpoint{4.966653in}{4.962383in}}%
\pgfusepath{stroke}%
\end{pgfscope}%
\begin{pgfscope}%
\pgfpathrectangle{\pgfqpoint{3.985294in}{4.155455in}}{\pgfqpoint{2.279412in}{2.004545in}}%
\pgfusepath{clip}%
\pgfsetbuttcap%
\pgfsetroundjoin%
\pgfsetlinewidth{0.822161pt}%
\definecolor{currentstroke}{rgb}{0.208623,0.367752,0.552675}%
\pgfsetstrokecolor{currentstroke}%
\pgfsetdash{}{0pt}%
\pgfpathmoveto{\pgfqpoint{4.966653in}{4.962383in}}%
\pgfpathlineto{\pgfqpoint{4.924848in}{4.986556in}}%
\pgfusepath{stroke}%
\end{pgfscope}%
\begin{pgfscope}%
\pgfpathrectangle{\pgfqpoint{3.985294in}{4.155455in}}{\pgfqpoint{2.279412in}{2.004545in}}%
\pgfusepath{clip}%
\pgfsetbuttcap%
\pgfsetroundjoin%
\pgfsetlinewidth{1.006728pt}%
\definecolor{currentstroke}{rgb}{0.163625,0.471133,0.558148}%
\pgfsetstrokecolor{currentstroke}%
\pgfsetdash{}{0pt}%
\pgfpathmoveto{\pgfqpoint{4.924848in}{4.986556in}}%
\pgfpathlineto{\pgfqpoint{4.889016in}{5.016109in}}%
\pgfusepath{stroke}%
\end{pgfscope}%
\begin{pgfscope}%
\pgfpathrectangle{\pgfqpoint{3.985294in}{4.155455in}}{\pgfqpoint{2.279412in}{2.004545in}}%
\pgfusepath{clip}%
\pgfsetbuttcap%
\pgfsetroundjoin%
\pgfsetlinewidth{0.797530pt}%
\definecolor{currentstroke}{rgb}{0.214298,0.355619,0.551184}%
\pgfsetstrokecolor{currentstroke}%
\pgfsetdash{}{0pt}%
\pgfpathmoveto{\pgfqpoint{4.889016in}{5.016109in}}%
\pgfpathlineto{\pgfqpoint{4.857852in}{5.042943in}}%
\pgfusepath{stroke}%
\end{pgfscope}%
\begin{pgfscope}%
\pgfpathrectangle{\pgfqpoint{3.985294in}{4.155455in}}{\pgfqpoint{2.279412in}{2.004545in}}%
\pgfusepath{clip}%
\pgfsetbuttcap%
\pgfsetroundjoin%
\pgfsetlinewidth{0.865021pt}%
\definecolor{currentstroke}{rgb}{0.195860,0.395433,0.555276}%
\pgfsetstrokecolor{currentstroke}%
\pgfsetdash{}{0pt}%
\pgfpathmoveto{\pgfqpoint{4.857852in}{5.042943in}}%
\pgfpathlineto{\pgfqpoint{4.823491in}{5.070571in}}%
\pgfusepath{stroke}%
\end{pgfscope}%
\begin{pgfscope}%
\pgfpathrectangle{\pgfqpoint{3.985294in}{4.155455in}}{\pgfqpoint{2.279412in}{2.004545in}}%
\pgfusepath{clip}%
\pgfsetbuttcap%
\pgfsetroundjoin%
\pgfsetlinewidth{0.326963pt}%
\definecolor{currentstroke}{rgb}{0.271305,0.019942,0.347269}%
\pgfsetstrokecolor{currentstroke}%
\pgfsetdash{}{0pt}%
\pgfpathmoveto{\pgfqpoint{5.894378in}{4.977300in}}%
\pgfpathlineto{\pgfqpoint{5.844249in}{4.978562in}}%
\pgfusepath{stroke}%
\end{pgfscope}%
\begin{pgfscope}%
\pgfpathrectangle{\pgfqpoint{3.985294in}{4.155455in}}{\pgfqpoint{2.279412in}{2.004545in}}%
\pgfusepath{clip}%
\pgfsetbuttcap%
\pgfsetroundjoin%
\pgfsetlinewidth{0.344188pt}%
\definecolor{currentstroke}{rgb}{0.274952,0.037752,0.364543}%
\pgfsetstrokecolor{currentstroke}%
\pgfsetdash{}{0pt}%
\pgfpathmoveto{\pgfqpoint{5.844249in}{4.978562in}}%
\pgfpathlineto{\pgfqpoint{5.794126in}{4.979990in}}%
\pgfusepath{stroke}%
\end{pgfscope}%
\begin{pgfscope}%
\pgfpathrectangle{\pgfqpoint{3.985294in}{4.155455in}}{\pgfqpoint{2.279412in}{2.004545in}}%
\pgfusepath{clip}%
\pgfsetbuttcap%
\pgfsetroundjoin%
\pgfsetlinewidth{0.350286pt}%
\definecolor{currentstroke}{rgb}{0.276022,0.044167,0.370164}%
\pgfsetstrokecolor{currentstroke}%
\pgfsetdash{}{0pt}%
\pgfpathmoveto{\pgfqpoint{5.794126in}{4.979990in}}%
\pgfpathlineto{\pgfqpoint{5.744012in}{4.981603in}}%
\pgfusepath{stroke}%
\end{pgfscope}%
\begin{pgfscope}%
\pgfpathrectangle{\pgfqpoint{3.985294in}{4.155455in}}{\pgfqpoint{2.279412in}{2.004545in}}%
\pgfusepath{clip}%
\pgfsetbuttcap%
\pgfsetroundjoin%
\pgfsetlinewidth{0.356182pt}%
\definecolor{currentstroke}{rgb}{0.277018,0.050344,0.375715}%
\pgfsetstrokecolor{currentstroke}%
\pgfsetdash{}{0pt}%
\pgfpathmoveto{\pgfqpoint{5.744012in}{4.981603in}}%
\pgfpathlineto{\pgfqpoint{5.693898in}{4.983278in}}%
\pgfusepath{stroke}%
\end{pgfscope}%
\begin{pgfscope}%
\pgfpathrectangle{\pgfqpoint{3.985294in}{4.155455in}}{\pgfqpoint{2.279412in}{2.004545in}}%
\pgfusepath{clip}%
\pgfsetbuttcap%
\pgfsetroundjoin%
\pgfsetlinewidth{0.395104pt}%
\definecolor{currentstroke}{rgb}{0.280894,0.078907,0.402329}%
\pgfsetstrokecolor{currentstroke}%
\pgfsetdash{}{0pt}%
\pgfpathmoveto{\pgfqpoint{5.693898in}{4.983278in}}%
\pgfpathlineto{\pgfqpoint{5.643781in}{4.984906in}}%
\pgfusepath{stroke}%
\end{pgfscope}%
\begin{pgfscope}%
\pgfpathrectangle{\pgfqpoint{3.985294in}{4.155455in}}{\pgfqpoint{2.279412in}{2.004545in}}%
\pgfusepath{clip}%
\pgfsetbuttcap%
\pgfsetroundjoin%
\pgfsetlinewidth{0.428818pt}%
\definecolor{currentstroke}{rgb}{0.282910,0.105393,0.426902}%
\pgfsetstrokecolor{currentstroke}%
\pgfsetdash{}{0pt}%
\pgfpathmoveto{\pgfqpoint{5.643781in}{4.984906in}}%
\pgfpathlineto{\pgfqpoint{5.593672in}{4.986732in}}%
\pgfusepath{stroke}%
\end{pgfscope}%
\begin{pgfscope}%
\pgfpathrectangle{\pgfqpoint{3.985294in}{4.155455in}}{\pgfqpoint{2.279412in}{2.004545in}}%
\pgfusepath{clip}%
\pgfsetbuttcap%
\pgfsetroundjoin%
\pgfsetlinewidth{0.467141pt}%
\definecolor{currentstroke}{rgb}{0.282884,0.135920,0.453427}%
\pgfsetstrokecolor{currentstroke}%
\pgfsetdash{}{0pt}%
\pgfpathmoveto{\pgfqpoint{5.593672in}{4.986732in}}%
\pgfpathlineto{\pgfqpoint{5.543574in}{4.988760in}}%
\pgfusepath{stroke}%
\end{pgfscope}%
\begin{pgfscope}%
\pgfpathrectangle{\pgfqpoint{3.985294in}{4.155455in}}{\pgfqpoint{2.279412in}{2.004545in}}%
\pgfusepath{clip}%
\pgfsetbuttcap%
\pgfsetroundjoin%
\pgfsetlinewidth{0.535697pt}%
\definecolor{currentstroke}{rgb}{0.277134,0.185228,0.489898}%
\pgfsetstrokecolor{currentstroke}%
\pgfsetdash{}{0pt}%
\pgfpathmoveto{\pgfqpoint{5.543574in}{4.988760in}}%
\pgfpathlineto{\pgfqpoint{5.493481in}{4.990898in}}%
\pgfusepath{stroke}%
\end{pgfscope}%
\begin{pgfscope}%
\pgfpathrectangle{\pgfqpoint{3.985294in}{4.155455in}}{\pgfqpoint{2.279412in}{2.004545in}}%
\pgfusepath{clip}%
\pgfsetbuttcap%
\pgfsetroundjoin%
\pgfsetlinewidth{0.582445pt}%
\definecolor{currentstroke}{rgb}{0.270595,0.214069,0.507052}%
\pgfsetstrokecolor{currentstroke}%
\pgfsetdash{}{0pt}%
\pgfpathmoveto{\pgfqpoint{5.493481in}{4.990898in}}%
\pgfpathlineto{\pgfqpoint{5.443396in}{4.993167in}}%
\pgfusepath{stroke}%
\end{pgfscope}%
\begin{pgfscope}%
\pgfpathrectangle{\pgfqpoint{3.985294in}{4.155455in}}{\pgfqpoint{2.279412in}{2.004545in}}%
\pgfusepath{clip}%
\pgfsetbuttcap%
\pgfsetroundjoin%
\pgfsetlinewidth{0.692424pt}%
\definecolor{currentstroke}{rgb}{0.243113,0.292092,0.538516}%
\pgfsetstrokecolor{currentstroke}%
\pgfsetdash{}{0pt}%
\pgfpathmoveto{\pgfqpoint{5.443396in}{4.993167in}}%
\pgfpathlineto{\pgfqpoint{5.393327in}{4.995690in}}%
\pgfusepath{stroke}%
\end{pgfscope}%
\begin{pgfscope}%
\pgfpathrectangle{\pgfqpoint{3.985294in}{4.155455in}}{\pgfqpoint{2.279412in}{2.004545in}}%
\pgfusepath{clip}%
\pgfsetbuttcap%
\pgfsetroundjoin%
\pgfsetlinewidth{0.714902pt}%
\definecolor{currentstroke}{rgb}{0.237441,0.305202,0.541921}%
\pgfsetstrokecolor{currentstroke}%
\pgfsetdash{}{0pt}%
\pgfpathmoveto{\pgfqpoint{5.393327in}{4.995690in}}%
\pgfpathlineto{\pgfqpoint{5.343284in}{4.998582in}}%
\pgfusepath{stroke}%
\end{pgfscope}%
\begin{pgfscope}%
\pgfpathrectangle{\pgfqpoint{3.985294in}{4.155455in}}{\pgfqpoint{2.279412in}{2.004545in}}%
\pgfusepath{clip}%
\pgfsetbuttcap%
\pgfsetroundjoin%
\pgfsetlinewidth{0.332470pt}%
\definecolor{currentstroke}{rgb}{0.272594,0.025563,0.353093}%
\pgfsetstrokecolor{currentstroke}%
\pgfsetdash{}{0pt}%
\pgfpathmoveto{\pgfqpoint{5.894378in}{5.112620in}}%
\pgfpathlineto{\pgfqpoint{5.844240in}{5.111982in}}%
\pgfusepath{stroke}%
\end{pgfscope}%
\begin{pgfscope}%
\pgfpathrectangle{\pgfqpoint{3.985294in}{4.155455in}}{\pgfqpoint{2.279412in}{2.004545in}}%
\pgfusepath{clip}%
\pgfsetbuttcap%
\pgfsetroundjoin%
\pgfsetlinewidth{0.338307pt}%
\definecolor{currentstroke}{rgb}{0.273809,0.031497,0.358853}%
\pgfsetstrokecolor{currentstroke}%
\pgfsetdash{}{0pt}%
\pgfpathmoveto{\pgfqpoint{5.844240in}{5.111982in}}%
\pgfpathlineto{\pgfqpoint{5.794097in}{5.111911in}}%
\pgfusepath{stroke}%
\end{pgfscope}%
\begin{pgfscope}%
\pgfpathrectangle{\pgfqpoint{3.985294in}{4.155455in}}{\pgfqpoint{2.279412in}{2.004545in}}%
\pgfusepath{clip}%
\pgfsetbuttcap%
\pgfsetroundjoin%
\pgfsetlinewidth{0.350460pt}%
\definecolor{currentstroke}{rgb}{0.276022,0.044167,0.370164}%
\pgfsetstrokecolor{currentstroke}%
\pgfsetdash{}{0pt}%
\pgfpathmoveto{\pgfqpoint{5.794097in}{5.111911in}}%
\pgfpathlineto{\pgfqpoint{5.743949in}{5.112173in}}%
\pgfusepath{stroke}%
\end{pgfscope}%
\begin{pgfscope}%
\pgfpathrectangle{\pgfqpoint{3.985294in}{4.155455in}}{\pgfqpoint{2.279412in}{2.004545in}}%
\pgfusepath{clip}%
\pgfsetbuttcap%
\pgfsetroundjoin%
\pgfsetlinewidth{0.378910pt}%
\definecolor{currentstroke}{rgb}{0.279566,0.067836,0.391917}%
\pgfsetstrokecolor{currentstroke}%
\pgfsetdash{}{0pt}%
\pgfpathmoveto{\pgfqpoint{5.743949in}{5.112173in}}%
\pgfpathlineto{\pgfqpoint{5.693802in}{5.112621in}}%
\pgfusepath{stroke}%
\end{pgfscope}%
\begin{pgfscope}%
\pgfpathrectangle{\pgfqpoint{3.985294in}{4.155455in}}{\pgfqpoint{2.279412in}{2.004545in}}%
\pgfusepath{clip}%
\pgfsetbuttcap%
\pgfsetroundjoin%
\pgfsetlinewidth{0.416722pt}%
\definecolor{currentstroke}{rgb}{0.282327,0.094955,0.417331}%
\pgfsetstrokecolor{currentstroke}%
\pgfsetdash{}{0pt}%
\pgfpathmoveto{\pgfqpoint{5.693802in}{5.112621in}}%
\pgfpathlineto{\pgfqpoint{5.643655in}{5.113155in}}%
\pgfusepath{stroke}%
\end{pgfscope}%
\begin{pgfscope}%
\pgfpathrectangle{\pgfqpoint{3.985294in}{4.155455in}}{\pgfqpoint{2.279412in}{2.004545in}}%
\pgfusepath{clip}%
\pgfsetbuttcap%
\pgfsetroundjoin%
\pgfsetlinewidth{0.470418pt}%
\definecolor{currentstroke}{rgb}{0.282884,0.135920,0.453427}%
\pgfsetstrokecolor{currentstroke}%
\pgfsetdash{}{0pt}%
\pgfpathmoveto{\pgfqpoint{5.643655in}{5.113155in}}%
\pgfpathlineto{\pgfqpoint{5.593506in}{5.113530in}}%
\pgfusepath{stroke}%
\end{pgfscope}%
\begin{pgfscope}%
\pgfpathrectangle{\pgfqpoint{3.985294in}{4.155455in}}{\pgfqpoint{2.279412in}{2.004545in}}%
\pgfusepath{clip}%
\pgfsetbuttcap%
\pgfsetroundjoin%
\pgfsetlinewidth{0.524568pt}%
\definecolor{currentstroke}{rgb}{0.278826,0.175490,0.483397}%
\pgfsetstrokecolor{currentstroke}%
\pgfsetdash{}{0pt}%
\pgfpathmoveto{\pgfqpoint{5.593506in}{5.113530in}}%
\pgfpathlineto{\pgfqpoint{5.543357in}{5.113974in}}%
\pgfusepath{stroke}%
\end{pgfscope}%
\begin{pgfscope}%
\pgfpathrectangle{\pgfqpoint{3.985294in}{4.155455in}}{\pgfqpoint{2.279412in}{2.004545in}}%
\pgfusepath{clip}%
\pgfsetbuttcap%
\pgfsetroundjoin%
\pgfsetlinewidth{0.568974pt}%
\definecolor{currentstroke}{rgb}{0.271828,0.209303,0.504434}%
\pgfsetstrokecolor{currentstroke}%
\pgfsetdash{}{0pt}%
\pgfpathmoveto{\pgfqpoint{5.543357in}{5.113974in}}%
\pgfpathlineto{\pgfqpoint{5.493207in}{5.114362in}}%
\pgfusepath{stroke}%
\end{pgfscope}%
\begin{pgfscope}%
\pgfpathrectangle{\pgfqpoint{3.985294in}{4.155455in}}{\pgfqpoint{2.279412in}{2.004545in}}%
\pgfusepath{clip}%
\pgfsetbuttcap%
\pgfsetroundjoin%
\pgfsetlinewidth{0.699173pt}%
\definecolor{currentstroke}{rgb}{0.241237,0.296485,0.539709}%
\pgfsetstrokecolor{currentstroke}%
\pgfsetdash{}{0pt}%
\pgfpathmoveto{\pgfqpoint{5.493207in}{5.114362in}}%
\pgfpathlineto{\pgfqpoint{5.443058in}{5.114826in}}%
\pgfusepath{stroke}%
\end{pgfscope}%
\begin{pgfscope}%
\pgfpathrectangle{\pgfqpoint{3.985294in}{4.155455in}}{\pgfqpoint{2.279412in}{2.004545in}}%
\pgfusepath{clip}%
\pgfsetbuttcap%
\pgfsetroundjoin%
\pgfsetlinewidth{0.793142pt}%
\definecolor{currentstroke}{rgb}{0.216210,0.351535,0.550627}%
\pgfsetstrokecolor{currentstroke}%
\pgfsetdash{}{0pt}%
\pgfpathmoveto{\pgfqpoint{5.443058in}{5.114826in}}%
\pgfpathlineto{\pgfqpoint{5.392911in}{5.115455in}}%
\pgfusepath{stroke}%
\end{pgfscope}%
\begin{pgfscope}%
\pgfpathrectangle{\pgfqpoint{3.985294in}{4.155455in}}{\pgfqpoint{2.279412in}{2.004545in}}%
\pgfusepath{clip}%
\pgfsetbuttcap%
\pgfsetroundjoin%
\pgfsetlinewidth{0.905987pt}%
\definecolor{currentstroke}{rgb}{0.185556,0.418570,0.556753}%
\pgfsetstrokecolor{currentstroke}%
\pgfsetdash{}{0pt}%
\pgfpathmoveto{\pgfqpoint{5.392911in}{5.115455in}}%
\pgfpathlineto{\pgfqpoint{5.342769in}{5.116294in}}%
\pgfusepath{stroke}%
\end{pgfscope}%
\begin{pgfscope}%
\pgfpathrectangle{\pgfqpoint{3.985294in}{4.155455in}}{\pgfqpoint{2.279412in}{2.004545in}}%
\pgfusepath{clip}%
\pgfsetbuttcap%
\pgfsetroundjoin%
\pgfsetlinewidth{1.007675pt}%
\definecolor{currentstroke}{rgb}{0.163625,0.471133,0.558148}%
\pgfsetstrokecolor{currentstroke}%
\pgfsetdash{}{0pt}%
\pgfpathmoveto{\pgfqpoint{5.342769in}{5.116294in}}%
\pgfpathlineto{\pgfqpoint{5.292628in}{5.117232in}}%
\pgfusepath{stroke}%
\end{pgfscope}%
\begin{pgfscope}%
\pgfpathrectangle{\pgfqpoint{3.985294in}{4.155455in}}{\pgfqpoint{2.279412in}{2.004545in}}%
\pgfusepath{clip}%
\pgfsetbuttcap%
\pgfsetroundjoin%
\pgfsetlinewidth{1.136319pt}%
\definecolor{currentstroke}{rgb}{0.136408,0.541173,0.554483}%
\pgfsetstrokecolor{currentstroke}%
\pgfsetdash{}{0pt}%
\pgfpathmoveto{\pgfqpoint{5.292628in}{5.117232in}}%
\pgfpathlineto{\pgfqpoint{5.242491in}{5.118286in}}%
\pgfusepath{stroke}%
\end{pgfscope}%
\begin{pgfscope}%
\pgfpathrectangle{\pgfqpoint{3.985294in}{4.155455in}}{\pgfqpoint{2.279412in}{2.004545in}}%
\pgfusepath{clip}%
\pgfsetbuttcap%
\pgfsetroundjoin%
\pgfsetlinewidth{1.316148pt}%
\definecolor{currentstroke}{rgb}{0.123444,0.636809,0.528763}%
\pgfsetstrokecolor{currentstroke}%
\pgfsetdash{}{0pt}%
\pgfpathmoveto{\pgfqpoint{5.242491in}{5.118286in}}%
\pgfpathlineto{\pgfqpoint{5.192359in}{5.119521in}}%
\pgfusepath{stroke}%
\end{pgfscope}%
\begin{pgfscope}%
\pgfpathrectangle{\pgfqpoint{3.985294in}{4.155455in}}{\pgfqpoint{2.279412in}{2.004545in}}%
\pgfusepath{clip}%
\pgfsetbuttcap%
\pgfsetroundjoin%
\pgfsetlinewidth{1.409475pt}%
\definecolor{currentstroke}{rgb}{0.162016,0.687316,0.499129}%
\pgfsetstrokecolor{currentstroke}%
\pgfsetdash{}{0pt}%
\pgfpathmoveto{\pgfqpoint{5.192359in}{5.119521in}}%
\pgfpathlineto{\pgfqpoint{5.142235in}{5.120974in}}%
\pgfusepath{stroke}%
\end{pgfscope}%
\begin{pgfscope}%
\pgfpathrectangle{\pgfqpoint{3.985294in}{4.155455in}}{\pgfqpoint{2.279412in}{2.004545in}}%
\pgfusepath{clip}%
\pgfsetbuttcap%
\pgfsetroundjoin%
\pgfsetlinewidth{1.533328pt}%
\definecolor{currentstroke}{rgb}{0.266941,0.748751,0.440573}%
\pgfsetstrokecolor{currentstroke}%
\pgfsetdash{}{0pt}%
\pgfpathmoveto{\pgfqpoint{5.142235in}{5.120974in}}%
\pgfpathlineto{\pgfqpoint{5.092129in}{5.122775in}}%
\pgfusepath{stroke}%
\end{pgfscope}%
\begin{pgfscope}%
\pgfpathrectangle{\pgfqpoint{3.985294in}{4.155455in}}{\pgfqpoint{2.279412in}{2.004545in}}%
\pgfusepath{clip}%
\pgfsetbuttcap%
\pgfsetroundjoin%
\pgfsetlinewidth{1.578885pt}%
\definecolor{currentstroke}{rgb}{0.319809,0.770914,0.411152}%
\pgfsetstrokecolor{currentstroke}%
\pgfsetdash{}{0pt}%
\pgfpathmoveto{\pgfqpoint{5.092129in}{5.122775in}}%
\pgfpathlineto{\pgfqpoint{5.042041in}{5.124892in}}%
\pgfusepath{stroke}%
\end{pgfscope}%
\begin{pgfscope}%
\pgfpathrectangle{\pgfqpoint{3.985294in}{4.155455in}}{\pgfqpoint{2.279412in}{2.004545in}}%
\pgfusepath{clip}%
\pgfsetbuttcap%
\pgfsetroundjoin%
\pgfsetlinewidth{1.577983pt}%
\definecolor{currentstroke}{rgb}{0.319809,0.770914,0.411152}%
\pgfsetstrokecolor{currentstroke}%
\pgfsetdash{}{0pt}%
\pgfpathmoveto{\pgfqpoint{5.042041in}{5.124892in}}%
\pgfpathlineto{\pgfqpoint{4.991988in}{5.127473in}}%
\pgfusepath{stroke}%
\end{pgfscope}%
\begin{pgfscope}%
\pgfpathrectangle{\pgfqpoint{3.985294in}{4.155455in}}{\pgfqpoint{2.279412in}{2.004545in}}%
\pgfusepath{clip}%
\pgfsetbuttcap%
\pgfsetroundjoin%
\pgfsetlinewidth{0.328728pt}%
\definecolor{currentstroke}{rgb}{0.272594,0.025563,0.353093}%
\pgfsetstrokecolor{currentstroke}%
\pgfsetdash{}{0pt}%
\pgfpathmoveto{\pgfqpoint{5.894378in}{5.247941in}}%
\pgfpathlineto{\pgfqpoint{5.844276in}{5.248447in}}%
\pgfusepath{stroke}%
\end{pgfscope}%
\begin{pgfscope}%
\pgfpathrectangle{\pgfqpoint{3.985294in}{4.155455in}}{\pgfqpoint{2.279412in}{2.004545in}}%
\pgfusepath{clip}%
\pgfsetbuttcap%
\pgfsetroundjoin%
\pgfsetlinewidth{0.337263pt}%
\definecolor{currentstroke}{rgb}{0.273809,0.031497,0.358853}%
\pgfsetstrokecolor{currentstroke}%
\pgfsetdash{}{0pt}%
\pgfpathmoveto{\pgfqpoint{5.844276in}{5.248447in}}%
\pgfpathlineto{\pgfqpoint{5.794144in}{5.247800in}}%
\pgfusepath{stroke}%
\end{pgfscope}%
\begin{pgfscope}%
\pgfpathrectangle{\pgfqpoint{3.985294in}{4.155455in}}{\pgfqpoint{2.279412in}{2.004545in}}%
\pgfusepath{clip}%
\pgfsetbuttcap%
\pgfsetroundjoin%
\pgfsetlinewidth{0.353037pt}%
\definecolor{currentstroke}{rgb}{0.276022,0.044167,0.370164}%
\pgfsetstrokecolor{currentstroke}%
\pgfsetdash{}{0pt}%
\pgfpathmoveto{\pgfqpoint{5.794144in}{5.247800in}}%
\pgfpathlineto{\pgfqpoint{5.744000in}{5.247461in}}%
\pgfusepath{stroke}%
\end{pgfscope}%
\begin{pgfscope}%
\pgfpathrectangle{\pgfqpoint{3.985294in}{4.155455in}}{\pgfqpoint{2.279412in}{2.004545in}}%
\pgfusepath{clip}%
\pgfsetbuttcap%
\pgfsetroundjoin%
\pgfsetlinewidth{0.373645pt}%
\definecolor{currentstroke}{rgb}{0.278791,0.062145,0.386592}%
\pgfsetstrokecolor{currentstroke}%
\pgfsetdash{}{0pt}%
\pgfpathmoveto{\pgfqpoint{5.744000in}{5.247461in}}%
\pgfpathlineto{\pgfqpoint{5.693861in}{5.246668in}}%
\pgfusepath{stroke}%
\end{pgfscope}%
\begin{pgfscope}%
\pgfpathrectangle{\pgfqpoint{3.985294in}{4.155455in}}{\pgfqpoint{2.279412in}{2.004545in}}%
\pgfusepath{clip}%
\pgfsetbuttcap%
\pgfsetroundjoin%
\pgfsetlinewidth{0.393793pt}%
\definecolor{currentstroke}{rgb}{0.280894,0.078907,0.402329}%
\pgfsetstrokecolor{currentstroke}%
\pgfsetdash{}{0pt}%
\pgfpathmoveto{\pgfqpoint{5.693861in}{5.246668in}}%
\pgfpathlineto{\pgfqpoint{5.643725in}{5.245642in}}%
\pgfusepath{stroke}%
\end{pgfscope}%
\begin{pgfscope}%
\pgfpathrectangle{\pgfqpoint{3.985294in}{4.155455in}}{\pgfqpoint{2.279412in}{2.004545in}}%
\pgfusepath{clip}%
\pgfsetbuttcap%
\pgfsetroundjoin%
\pgfsetlinewidth{0.443937pt}%
\definecolor{currentstroke}{rgb}{0.283197,0.115680,0.436115}%
\pgfsetstrokecolor{currentstroke}%
\pgfsetdash{}{0pt}%
\pgfpathmoveto{\pgfqpoint{5.643725in}{5.245642in}}%
\pgfpathlineto{\pgfqpoint{5.593583in}{5.244794in}}%
\pgfusepath{stroke}%
\end{pgfscope}%
\begin{pgfscope}%
\pgfpathrectangle{\pgfqpoint{3.985294in}{4.155455in}}{\pgfqpoint{2.279412in}{2.004545in}}%
\pgfusepath{clip}%
\pgfsetbuttcap%
\pgfsetroundjoin%
\pgfsetlinewidth{0.510461pt}%
\definecolor{currentstroke}{rgb}{0.280255,0.165693,0.476498}%
\pgfsetstrokecolor{currentstroke}%
\pgfsetdash{}{0pt}%
\pgfpathmoveto{\pgfqpoint{5.593583in}{5.244794in}}%
\pgfpathlineto{\pgfqpoint{5.543446in}{5.243777in}}%
\pgfusepath{stroke}%
\end{pgfscope}%
\begin{pgfscope}%
\pgfpathrectangle{\pgfqpoint{3.985294in}{4.155455in}}{\pgfqpoint{2.279412in}{2.004545in}}%
\pgfusepath{clip}%
\pgfsetbuttcap%
\pgfsetroundjoin%
\pgfsetlinewidth{0.566474pt}%
\definecolor{currentstroke}{rgb}{0.273006,0.204520,0.501721}%
\pgfsetstrokecolor{currentstroke}%
\pgfsetdash{}{0pt}%
\pgfpathmoveto{\pgfqpoint{5.543446in}{5.243777in}}%
\pgfpathlineto{\pgfqpoint{5.493315in}{5.242496in}}%
\pgfusepath{stroke}%
\end{pgfscope}%
\begin{pgfscope}%
\pgfpathrectangle{\pgfqpoint{3.985294in}{4.155455in}}{\pgfqpoint{2.279412in}{2.004545in}}%
\pgfusepath{clip}%
\pgfsetbuttcap%
\pgfsetroundjoin%
\pgfsetlinewidth{0.658898pt}%
\definecolor{currentstroke}{rgb}{0.252194,0.269783,0.531579}%
\pgfsetstrokecolor{currentstroke}%
\pgfsetdash{}{0pt}%
\pgfpathmoveto{\pgfqpoint{5.493315in}{5.242496in}}%
\pgfpathlineto{\pgfqpoint{5.443185in}{5.241190in}}%
\pgfusepath{stroke}%
\end{pgfscope}%
\begin{pgfscope}%
\pgfpathrectangle{\pgfqpoint{3.985294in}{4.155455in}}{\pgfqpoint{2.279412in}{2.004545in}}%
\pgfusepath{clip}%
\pgfsetbuttcap%
\pgfsetroundjoin%
\pgfsetlinewidth{0.730542pt}%
\definecolor{currentstroke}{rgb}{0.233603,0.313828,0.543914}%
\pgfsetstrokecolor{currentstroke}%
\pgfsetdash{}{0pt}%
\pgfpathmoveto{\pgfqpoint{5.443185in}{5.241190in}}%
\pgfpathlineto{\pgfqpoint{5.393064in}{5.239665in}}%
\pgfusepath{stroke}%
\end{pgfscope}%
\begin{pgfscope}%
\pgfpathrectangle{\pgfqpoint{3.985294in}{4.155455in}}{\pgfqpoint{2.279412in}{2.004545in}}%
\pgfusepath{clip}%
\pgfsetbuttcap%
\pgfsetroundjoin%
\pgfsetlinewidth{0.816320pt}%
\definecolor{currentstroke}{rgb}{0.208623,0.367752,0.552675}%
\pgfsetstrokecolor{currentstroke}%
\pgfsetdash{}{0pt}%
\pgfpathmoveto{\pgfqpoint{5.393064in}{5.239665in}}%
\pgfpathlineto{\pgfqpoint{5.342954in}{5.237858in}}%
\pgfusepath{stroke}%
\end{pgfscope}%
\begin{pgfscope}%
\pgfpathrectangle{\pgfqpoint{3.985294in}{4.155455in}}{\pgfqpoint{2.279412in}{2.004545in}}%
\pgfusepath{clip}%
\pgfsetbuttcap%
\pgfsetroundjoin%
\pgfsetlinewidth{0.946282pt}%
\definecolor{currentstroke}{rgb}{0.175841,0.441290,0.557685}%
\pgfsetstrokecolor{currentstroke}%
\pgfsetdash{}{0pt}%
\pgfpathmoveto{\pgfqpoint{5.342954in}{5.237858in}}%
\pgfpathlineto{\pgfqpoint{5.292854in}{5.235870in}}%
\pgfusepath{stroke}%
\end{pgfscope}%
\begin{pgfscope}%
\pgfpathrectangle{\pgfqpoint{3.985294in}{4.155455in}}{\pgfqpoint{2.279412in}{2.004545in}}%
\pgfusepath{clip}%
\pgfsetbuttcap%
\pgfsetroundjoin%
\pgfsetlinewidth{1.036774pt}%
\definecolor{currentstroke}{rgb}{0.156270,0.489624,0.557936}%
\pgfsetstrokecolor{currentstroke}%
\pgfsetdash{}{0pt}%
\pgfpathmoveto{\pgfqpoint{5.292854in}{5.235870in}}%
\pgfpathlineto{\pgfqpoint{5.242775in}{5.233503in}}%
\pgfusepath{stroke}%
\end{pgfscope}%
\begin{pgfscope}%
\pgfpathrectangle{\pgfqpoint{3.985294in}{4.155455in}}{\pgfqpoint{2.279412in}{2.004545in}}%
\pgfusepath{clip}%
\pgfsetbuttcap%
\pgfsetroundjoin%
\pgfsetlinewidth{0.320856pt}%
\definecolor{currentstroke}{rgb}{0.269944,0.014625,0.341379}%
\pgfsetstrokecolor{currentstroke}%
\pgfsetdash{}{0pt}%
\pgfpathmoveto{\pgfqpoint{5.894378in}{5.383261in}}%
\pgfpathlineto{\pgfqpoint{5.844277in}{5.381382in}}%
\pgfusepath{stroke}%
\end{pgfscope}%
\begin{pgfscope}%
\pgfpathrectangle{\pgfqpoint{3.985294in}{4.155455in}}{\pgfqpoint{2.279412in}{2.004545in}}%
\pgfusepath{clip}%
\pgfsetbuttcap%
\pgfsetroundjoin%
\pgfsetlinewidth{0.335084pt}%
\definecolor{currentstroke}{rgb}{0.272594,0.025563,0.353093}%
\pgfsetstrokecolor{currentstroke}%
\pgfsetdash{}{0pt}%
\pgfpathmoveto{\pgfqpoint{5.844277in}{5.381382in}}%
\pgfpathlineto{\pgfqpoint{5.794169in}{5.379629in}}%
\pgfusepath{stroke}%
\end{pgfscope}%
\begin{pgfscope}%
\pgfpathrectangle{\pgfqpoint{3.985294in}{4.155455in}}{\pgfqpoint{2.279412in}{2.004545in}}%
\pgfusepath{clip}%
\pgfsetbuttcap%
\pgfsetroundjoin%
\pgfsetlinewidth{0.343990pt}%
\definecolor{currentstroke}{rgb}{0.274952,0.037752,0.364543}%
\pgfsetstrokecolor{currentstroke}%
\pgfsetdash{}{0pt}%
\pgfpathmoveto{\pgfqpoint{5.794169in}{5.379629in}}%
\pgfpathlineto{\pgfqpoint{5.744046in}{5.378237in}}%
\pgfusepath{stroke}%
\end{pgfscope}%
\begin{pgfscope}%
\pgfpathrectangle{\pgfqpoint{3.985294in}{4.155455in}}{\pgfqpoint{2.279412in}{2.004545in}}%
\pgfusepath{clip}%
\pgfsetbuttcap%
\pgfsetroundjoin%
\pgfsetlinewidth{0.355653pt}%
\definecolor{currentstroke}{rgb}{0.276022,0.044167,0.370164}%
\pgfsetstrokecolor{currentstroke}%
\pgfsetdash{}{0pt}%
\pgfpathmoveto{\pgfqpoint{5.744046in}{5.378237in}}%
\pgfpathlineto{\pgfqpoint{5.693935in}{5.376527in}}%
\pgfusepath{stroke}%
\end{pgfscope}%
\begin{pgfscope}%
\pgfpathrectangle{\pgfqpoint{3.985294in}{4.155455in}}{\pgfqpoint{2.279412in}{2.004545in}}%
\pgfusepath{clip}%
\pgfsetbuttcap%
\pgfsetroundjoin%
\pgfsetlinewidth{0.386694pt}%
\definecolor{currentstroke}{rgb}{0.280267,0.073417,0.397163}%
\pgfsetstrokecolor{currentstroke}%
\pgfsetdash{}{0pt}%
\pgfpathmoveto{\pgfqpoint{5.693935in}{5.376527in}}%
\pgfpathlineto{\pgfqpoint{5.643844in}{5.374361in}}%
\pgfusepath{stroke}%
\end{pgfscope}%
\begin{pgfscope}%
\pgfpathrectangle{\pgfqpoint{3.985294in}{4.155455in}}{\pgfqpoint{2.279412in}{2.004545in}}%
\pgfusepath{clip}%
\pgfsetbuttcap%
\pgfsetroundjoin%
\pgfsetlinewidth{0.421717pt}%
\definecolor{currentstroke}{rgb}{0.282656,0.100196,0.422160}%
\pgfsetstrokecolor{currentstroke}%
\pgfsetdash{}{0pt}%
\pgfpathmoveto{\pgfqpoint{5.643844in}{5.374361in}}%
\pgfpathlineto{\pgfqpoint{5.593759in}{5.372077in}}%
\pgfusepath{stroke}%
\end{pgfscope}%
\begin{pgfscope}%
\pgfpathrectangle{\pgfqpoint{3.985294in}{4.155455in}}{\pgfqpoint{2.279412in}{2.004545in}}%
\pgfusepath{clip}%
\pgfsetbuttcap%
\pgfsetroundjoin%
\pgfsetlinewidth{0.449624pt}%
\definecolor{currentstroke}{rgb}{0.283229,0.120777,0.440584}%
\pgfsetstrokecolor{currentstroke}%
\pgfsetdash{}{0pt}%
\pgfpathmoveto{\pgfqpoint{5.593759in}{5.372077in}}%
\pgfpathlineto{\pgfqpoint{5.543680in}{5.369705in}}%
\pgfusepath{stroke}%
\end{pgfscope}%
\begin{pgfscope}%
\pgfpathrectangle{\pgfqpoint{3.985294in}{4.155455in}}{\pgfqpoint{2.279412in}{2.004545in}}%
\pgfusepath{clip}%
\pgfsetbuttcap%
\pgfsetroundjoin%
\pgfsetlinewidth{0.495056pt}%
\definecolor{currentstroke}{rgb}{0.281412,0.155834,0.469201}%
\pgfsetstrokecolor{currentstroke}%
\pgfsetdash{}{0pt}%
\pgfpathmoveto{\pgfqpoint{5.543680in}{5.369705in}}%
\pgfpathlineto{\pgfqpoint{5.493605in}{5.367279in}}%
\pgfusepath{stroke}%
\end{pgfscope}%
\begin{pgfscope}%
\pgfpathrectangle{\pgfqpoint{3.985294in}{4.155455in}}{\pgfqpoint{2.279412in}{2.004545in}}%
\pgfusepath{clip}%
\pgfsetbuttcap%
\pgfsetroundjoin%
\pgfsetlinewidth{0.549845pt}%
\definecolor{currentstroke}{rgb}{0.275191,0.194905,0.496005}%
\pgfsetstrokecolor{currentstroke}%
\pgfsetdash{}{0pt}%
\pgfpathmoveto{\pgfqpoint{5.493605in}{5.367279in}}%
\pgfpathlineto{\pgfqpoint{5.443547in}{5.364613in}}%
\pgfusepath{stroke}%
\end{pgfscope}%
\begin{pgfscope}%
\pgfpathrectangle{\pgfqpoint{3.985294in}{4.155455in}}{\pgfqpoint{2.279412in}{2.004545in}}%
\pgfusepath{clip}%
\pgfsetbuttcap%
\pgfsetroundjoin%
\pgfsetlinewidth{0.619339pt}%
\definecolor{currentstroke}{rgb}{0.262138,0.242286,0.520837}%
\pgfsetstrokecolor{currentstroke}%
\pgfsetdash{}{0pt}%
\pgfpathmoveto{\pgfqpoint{5.443547in}{5.364613in}}%
\pgfpathlineto{\pgfqpoint{5.393534in}{5.361342in}}%
\pgfusepath{stroke}%
\end{pgfscope}%
\begin{pgfscope}%
\pgfpathrectangle{\pgfqpoint{3.985294in}{4.155455in}}{\pgfqpoint{2.279412in}{2.004545in}}%
\pgfusepath{clip}%
\pgfsetbuttcap%
\pgfsetroundjoin%
\pgfsetlinewidth{0.650271pt}%
\definecolor{currentstroke}{rgb}{0.255645,0.260703,0.528312}%
\pgfsetstrokecolor{currentstroke}%
\pgfsetdash{}{0pt}%
\pgfpathmoveto{\pgfqpoint{5.393534in}{5.361342in}}%
\pgfpathlineto{\pgfqpoint{5.343573in}{5.357511in}}%
\pgfusepath{stroke}%
\end{pgfscope}%
\begin{pgfscope}%
\pgfpathrectangle{\pgfqpoint{3.985294in}{4.155455in}}{\pgfqpoint{2.279412in}{2.004545in}}%
\pgfusepath{clip}%
\pgfsetbuttcap%
\pgfsetroundjoin%
\pgfsetlinewidth{0.739091pt}%
\definecolor{currentstroke}{rgb}{0.231674,0.318106,0.544834}%
\pgfsetstrokecolor{currentstroke}%
\pgfsetdash{}{0pt}%
\pgfpathmoveto{\pgfqpoint{5.343573in}{5.357511in}}%
\pgfpathlineto{\pgfqpoint{5.293690in}{5.352985in}}%
\pgfusepath{stroke}%
\end{pgfscope}%
\begin{pgfscope}%
\pgfpathrectangle{\pgfqpoint{3.985294in}{4.155455in}}{\pgfqpoint{2.279412in}{2.004545in}}%
\pgfusepath{clip}%
\pgfsetbuttcap%
\pgfsetroundjoin%
\pgfsetlinewidth{0.790335pt}%
\definecolor{currentstroke}{rgb}{0.216210,0.351535,0.550627}%
\pgfsetstrokecolor{currentstroke}%
\pgfsetdash{}{0pt}%
\pgfpathmoveto{\pgfqpoint{5.293690in}{5.352985in}}%
\pgfpathlineto{\pgfqpoint{5.243914in}{5.347625in}}%
\pgfusepath{stroke}%
\end{pgfscope}%
\begin{pgfscope}%
\pgfpathrectangle{\pgfqpoint{3.985294in}{4.155455in}}{\pgfqpoint{2.279412in}{2.004545in}}%
\pgfusepath{clip}%
\pgfsetbuttcap%
\pgfsetroundjoin%
\pgfsetlinewidth{0.805512pt}%
\definecolor{currentstroke}{rgb}{0.212395,0.359683,0.551710}%
\pgfsetstrokecolor{currentstroke}%
\pgfsetdash{}{0pt}%
\pgfpathmoveto{\pgfqpoint{5.243914in}{5.347625in}}%
\pgfpathlineto{\pgfqpoint{5.194291in}{5.341270in}}%
\pgfusepath{stroke}%
\end{pgfscope}%
\begin{pgfscope}%
\pgfpathrectangle{\pgfqpoint{3.985294in}{4.155455in}}{\pgfqpoint{2.279412in}{2.004545in}}%
\pgfusepath{clip}%
\pgfsetbuttcap%
\pgfsetroundjoin%
\pgfsetlinewidth{0.907815pt}%
\definecolor{currentstroke}{rgb}{0.185556,0.418570,0.556753}%
\pgfsetstrokecolor{currentstroke}%
\pgfsetdash{}{0pt}%
\pgfpathmoveto{\pgfqpoint{5.194291in}{5.341270in}}%
\pgfpathlineto{\pgfqpoint{5.144897in}{5.333669in}}%
\pgfusepath{stroke}%
\end{pgfscope}%
\begin{pgfscope}%
\pgfpathrectangle{\pgfqpoint{3.985294in}{4.155455in}}{\pgfqpoint{2.279412in}{2.004545in}}%
\pgfusepath{clip}%
\pgfsetbuttcap%
\pgfsetroundjoin%
\pgfsetlinewidth{0.927536pt}%
\definecolor{currentstroke}{rgb}{0.180629,0.429975,0.557282}%
\pgfsetstrokecolor{currentstroke}%
\pgfsetdash{}{0pt}%
\pgfpathmoveto{\pgfqpoint{5.144897in}{5.333669in}}%
\pgfpathlineto{\pgfqpoint{5.095807in}{5.324691in}}%
\pgfusepath{stroke}%
\end{pgfscope}%
\begin{pgfscope}%
\pgfpathrectangle{\pgfqpoint{3.985294in}{4.155455in}}{\pgfqpoint{2.279412in}{2.004545in}}%
\pgfusepath{clip}%
\pgfsetbuttcap%
\pgfsetroundjoin%
\pgfsetlinewidth{0.958271pt}%
\definecolor{currentstroke}{rgb}{0.174274,0.445044,0.557792}%
\pgfsetstrokecolor{currentstroke}%
\pgfsetdash{}{0pt}%
\pgfpathmoveto{\pgfqpoint{5.095807in}{5.324691in}}%
\pgfpathlineto{\pgfqpoint{5.047141in}{5.314106in}}%
\pgfusepath{stroke}%
\end{pgfscope}%
\begin{pgfscope}%
\pgfpathrectangle{\pgfqpoint{3.985294in}{4.155455in}}{\pgfqpoint{2.279412in}{2.004545in}}%
\pgfusepath{clip}%
\pgfsetbuttcap%
\pgfsetroundjoin%
\pgfsetlinewidth{1.012551pt}%
\definecolor{currentstroke}{rgb}{0.162142,0.474838,0.558140}%
\pgfsetstrokecolor{currentstroke}%
\pgfsetdash{}{0pt}%
\pgfpathmoveto{\pgfqpoint{5.047141in}{5.314106in}}%
\pgfpathlineto{\pgfqpoint{4.999217in}{5.301277in}}%
\pgfusepath{stroke}%
\end{pgfscope}%
\begin{pgfscope}%
\pgfpathrectangle{\pgfqpoint{3.985294in}{4.155455in}}{\pgfqpoint{2.279412in}{2.004545in}}%
\pgfusepath{clip}%
\pgfsetbuttcap%
\pgfsetroundjoin%
\pgfsetlinewidth{0.954570pt}%
\definecolor{currentstroke}{rgb}{0.174274,0.445044,0.557792}%
\pgfsetstrokecolor{currentstroke}%
\pgfsetdash{}{0pt}%
\pgfpathmoveto{\pgfqpoint{4.999217in}{5.301277in}}%
\pgfpathlineto{\pgfqpoint{4.952521in}{5.285497in}}%
\pgfusepath{stroke}%
\end{pgfscope}%
\begin{pgfscope}%
\pgfpathrectangle{\pgfqpoint{3.985294in}{4.155455in}}{\pgfqpoint{2.279412in}{2.004545in}}%
\pgfusepath{clip}%
\pgfsetbuttcap%
\pgfsetroundjoin%
\pgfsetlinewidth{0.898657pt}%
\definecolor{currentstroke}{rgb}{0.187231,0.414746,0.556547}%
\pgfsetstrokecolor{currentstroke}%
\pgfsetdash{}{0pt}%
\pgfpathmoveto{\pgfqpoint{4.952521in}{5.285497in}}%
\pgfpathlineto{\pgfqpoint{4.907308in}{5.266756in}}%
\pgfusepath{stroke}%
\end{pgfscope}%
\begin{pgfscope}%
\pgfpathrectangle{\pgfqpoint{3.985294in}{4.155455in}}{\pgfqpoint{2.279412in}{2.004545in}}%
\pgfusepath{clip}%
\pgfsetbuttcap%
\pgfsetroundjoin%
\pgfsetlinewidth{0.947468pt}%
\definecolor{currentstroke}{rgb}{0.175841,0.441290,0.557685}%
\pgfsetstrokecolor{currentstroke}%
\pgfsetdash{}{0pt}%
\pgfpathmoveto{\pgfqpoint{4.907308in}{5.266756in}}%
\pgfpathlineto{\pgfqpoint{4.862550in}{5.247237in}}%
\pgfusepath{stroke}%
\end{pgfscope}%
\begin{pgfscope}%
\pgfpathrectangle{\pgfqpoint{3.985294in}{4.155455in}}{\pgfqpoint{2.279412in}{2.004545in}}%
\pgfusepath{clip}%
\pgfsetbuttcap%
\pgfsetroundjoin%
\pgfsetlinewidth{1.110340pt}%
\definecolor{currentstroke}{rgb}{0.141935,0.526453,0.555991}%
\pgfsetstrokecolor{currentstroke}%
\pgfsetdash{}{0pt}%
\pgfpathmoveto{\pgfqpoint{4.862550in}{5.247237in}}%
\pgfpathlineto{\pgfqpoint{4.819097in}{5.225878in}}%
\pgfusepath{stroke}%
\end{pgfscope}%
\begin{pgfscope}%
\pgfpathrectangle{\pgfqpoint{3.985294in}{4.155455in}}{\pgfqpoint{2.279412in}{2.004545in}}%
\pgfusepath{clip}%
\pgfsetbuttcap%
\pgfsetroundjoin%
\pgfsetlinewidth{1.095859pt}%
\definecolor{currentstroke}{rgb}{0.144759,0.519093,0.556572}%
\pgfsetstrokecolor{currentstroke}%
\pgfsetdash{}{0pt}%
\pgfpathmoveto{\pgfqpoint{4.819097in}{5.225878in}}%
\pgfpathlineto{\pgfqpoint{4.776411in}{5.203717in}}%
\pgfusepath{stroke}%
\end{pgfscope}%
\begin{pgfscope}%
\pgfpathrectangle{\pgfqpoint{3.985294in}{4.155455in}}{\pgfqpoint{2.279412in}{2.004545in}}%
\pgfusepath{clip}%
\pgfsetbuttcap%
\pgfsetroundjoin%
\pgfsetlinewidth{0.937661pt}%
\definecolor{currentstroke}{rgb}{0.179019,0.433756,0.557430}%
\pgfsetstrokecolor{currentstroke}%
\pgfsetdash{}{0pt}%
\pgfpathmoveto{\pgfqpoint{4.776411in}{5.203717in}}%
\pgfpathlineto{\pgfqpoint{4.733122in}{5.183231in}}%
\pgfusepath{stroke}%
\end{pgfscope}%
\begin{pgfscope}%
\pgfpathrectangle{\pgfqpoint{3.985294in}{4.155455in}}{\pgfqpoint{2.279412in}{2.004545in}}%
\pgfusepath{clip}%
\pgfsetbuttcap%
\pgfsetroundjoin%
\pgfsetlinewidth{0.322833pt}%
\definecolor{currentstroke}{rgb}{0.271305,0.019942,0.347269}%
\pgfsetstrokecolor{currentstroke}%
\pgfsetdash{}{0pt}%
\pgfpathmoveto{\pgfqpoint{5.894378in}{5.428368in}}%
\pgfpathlineto{\pgfqpoint{5.844264in}{5.427046in}}%
\pgfusepath{stroke}%
\end{pgfscope}%
\begin{pgfscope}%
\pgfpathrectangle{\pgfqpoint{3.985294in}{4.155455in}}{\pgfqpoint{2.279412in}{2.004545in}}%
\pgfusepath{clip}%
\pgfsetbuttcap%
\pgfsetroundjoin%
\pgfsetlinewidth{0.327222pt}%
\definecolor{currentstroke}{rgb}{0.271305,0.019942,0.347269}%
\pgfsetstrokecolor{currentstroke}%
\pgfsetdash{}{0pt}%
\pgfpathmoveto{\pgfqpoint{5.844264in}{5.427046in}}%
\pgfpathlineto{\pgfqpoint{5.794127in}{5.426351in}}%
\pgfusepath{stroke}%
\end{pgfscope}%
\begin{pgfscope}%
\pgfpathrectangle{\pgfqpoint{3.985294in}{4.155455in}}{\pgfqpoint{2.279412in}{2.004545in}}%
\pgfusepath{clip}%
\pgfsetbuttcap%
\pgfsetroundjoin%
\pgfsetlinewidth{0.337054pt}%
\definecolor{currentstroke}{rgb}{0.273809,0.031497,0.358853}%
\pgfsetstrokecolor{currentstroke}%
\pgfsetdash{}{0pt}%
\pgfpathmoveto{\pgfqpoint{5.794127in}{5.426351in}}%
\pgfpathlineto{\pgfqpoint{5.744035in}{5.424633in}}%
\pgfusepath{stroke}%
\end{pgfscope}%
\begin{pgfscope}%
\pgfpathrectangle{\pgfqpoint{3.985294in}{4.155455in}}{\pgfqpoint{2.279412in}{2.004545in}}%
\pgfusepath{clip}%
\pgfsetbuttcap%
\pgfsetroundjoin%
\pgfsetlinewidth{0.357269pt}%
\definecolor{currentstroke}{rgb}{0.277018,0.050344,0.375715}%
\pgfsetstrokecolor{currentstroke}%
\pgfsetdash{}{0pt}%
\pgfpathmoveto{\pgfqpoint{5.744035in}{5.424633in}}%
\pgfpathlineto{\pgfqpoint{5.693957in}{5.422301in}}%
\pgfusepath{stroke}%
\end{pgfscope}%
\begin{pgfscope}%
\pgfpathrectangle{\pgfqpoint{3.985294in}{4.155455in}}{\pgfqpoint{2.279412in}{2.004545in}}%
\pgfusepath{clip}%
\pgfsetbuttcap%
\pgfsetroundjoin%
\pgfsetlinewidth{0.375300pt}%
\definecolor{currentstroke}{rgb}{0.278791,0.062145,0.386592}%
\pgfsetstrokecolor{currentstroke}%
\pgfsetdash{}{0pt}%
\pgfpathmoveto{\pgfqpoint{5.693957in}{5.422301in}}%
\pgfpathlineto{\pgfqpoint{5.643865in}{5.420263in}}%
\pgfusepath{stroke}%
\end{pgfscope}%
\begin{pgfscope}%
\pgfpathrectangle{\pgfqpoint{3.985294in}{4.155455in}}{\pgfqpoint{2.279412in}{2.004545in}}%
\pgfusepath{clip}%
\pgfsetbuttcap%
\pgfsetroundjoin%
\pgfsetlinewidth{0.410446pt}%
\definecolor{currentstroke}{rgb}{0.281924,0.089666,0.412415}%
\pgfsetstrokecolor{currentstroke}%
\pgfsetdash{}{0pt}%
\pgfpathmoveto{\pgfqpoint{5.643865in}{5.420263in}}%
\pgfpathlineto{\pgfqpoint{5.593780in}{5.418067in}}%
\pgfusepath{stroke}%
\end{pgfscope}%
\begin{pgfscope}%
\pgfpathrectangle{\pgfqpoint{3.985294in}{4.155455in}}{\pgfqpoint{2.279412in}{2.004545in}}%
\pgfusepath{clip}%
\pgfsetbuttcap%
\pgfsetroundjoin%
\pgfsetlinewidth{0.442758pt}%
\definecolor{currentstroke}{rgb}{0.283197,0.115680,0.436115}%
\pgfsetstrokecolor{currentstroke}%
\pgfsetdash{}{0pt}%
\pgfpathmoveto{\pgfqpoint{5.593780in}{5.418067in}}%
\pgfpathlineto{\pgfqpoint{5.543709in}{5.415578in}}%
\pgfusepath{stroke}%
\end{pgfscope}%
\begin{pgfscope}%
\pgfpathrectangle{\pgfqpoint{3.985294in}{4.155455in}}{\pgfqpoint{2.279412in}{2.004545in}}%
\pgfusepath{clip}%
\pgfsetbuttcap%
\pgfsetroundjoin%
\pgfsetlinewidth{0.465127pt}%
\definecolor{currentstroke}{rgb}{0.283072,0.130895,0.449241}%
\pgfsetstrokecolor{currentstroke}%
\pgfsetdash{}{0pt}%
\pgfpathmoveto{\pgfqpoint{5.543709in}{5.415578in}}%
\pgfpathlineto{\pgfqpoint{5.493655in}{5.412834in}}%
\pgfusepath{stroke}%
\end{pgfscope}%
\begin{pgfscope}%
\pgfpathrectangle{\pgfqpoint{3.985294in}{4.155455in}}{\pgfqpoint{2.279412in}{2.004545in}}%
\pgfusepath{clip}%
\pgfsetbuttcap%
\pgfsetroundjoin%
\pgfsetlinewidth{0.524295pt}%
\definecolor{currentstroke}{rgb}{0.278826,0.175490,0.483397}%
\pgfsetstrokecolor{currentstroke}%
\pgfsetdash{}{0pt}%
\pgfpathmoveto{\pgfqpoint{5.493655in}{5.412834in}}%
\pgfpathlineto{\pgfqpoint{5.443634in}{5.409682in}}%
\pgfusepath{stroke}%
\end{pgfscope}%
\begin{pgfscope}%
\pgfpathrectangle{\pgfqpoint{3.985294in}{4.155455in}}{\pgfqpoint{2.279412in}{2.004545in}}%
\pgfusepath{clip}%
\pgfsetbuttcap%
\pgfsetroundjoin%
\pgfsetlinewidth{0.561883pt}%
\definecolor{currentstroke}{rgb}{0.273006,0.204520,0.501721}%
\pgfsetstrokecolor{currentstroke}%
\pgfsetdash{}{0pt}%
\pgfpathmoveto{\pgfqpoint{5.443634in}{5.409682in}}%
\pgfpathlineto{\pgfqpoint{5.393662in}{5.405968in}}%
\pgfusepath{stroke}%
\end{pgfscope}%
\begin{pgfscope}%
\pgfpathrectangle{\pgfqpoint{3.985294in}{4.155455in}}{\pgfqpoint{2.279412in}{2.004545in}}%
\pgfusepath{clip}%
\pgfsetbuttcap%
\pgfsetroundjoin%
\pgfsetlinewidth{0.614183pt}%
\definecolor{currentstroke}{rgb}{0.263663,0.237631,0.518762}%
\pgfsetstrokecolor{currentstroke}%
\pgfsetdash{}{0pt}%
\pgfpathmoveto{\pgfqpoint{5.393662in}{5.405968in}}%
\pgfpathlineto{\pgfqpoint{5.343752in}{5.401662in}}%
\pgfusepath{stroke}%
\end{pgfscope}%
\begin{pgfscope}%
\pgfpathrectangle{\pgfqpoint{3.985294in}{4.155455in}}{\pgfqpoint{2.279412in}{2.004545in}}%
\pgfusepath{clip}%
\pgfsetbuttcap%
\pgfsetroundjoin%
\pgfsetlinewidth{0.650809pt}%
\definecolor{currentstroke}{rgb}{0.255645,0.260703,0.528312}%
\pgfsetstrokecolor{currentstroke}%
\pgfsetdash{}{0pt}%
\pgfpathmoveto{\pgfqpoint{5.343752in}{5.401662in}}%
\pgfpathlineto{\pgfqpoint{5.293914in}{5.396751in}}%
\pgfusepath{stroke}%
\end{pgfscope}%
\begin{pgfscope}%
\pgfpathrectangle{\pgfqpoint{3.985294in}{4.155455in}}{\pgfqpoint{2.279412in}{2.004545in}}%
\pgfusepath{clip}%
\pgfsetbuttcap%
\pgfsetroundjoin%
\pgfsetlinewidth{0.733456pt}%
\definecolor{currentstroke}{rgb}{0.231674,0.318106,0.544834}%
\pgfsetstrokecolor{currentstroke}%
\pgfsetdash{}{0pt}%
\pgfpathmoveto{\pgfqpoint{5.293914in}{5.396751in}}%
\pgfpathlineto{\pgfqpoint{5.244199in}{5.390973in}}%
\pgfusepath{stroke}%
\end{pgfscope}%
\begin{pgfscope}%
\pgfpathrectangle{\pgfqpoint{3.985294in}{4.155455in}}{\pgfqpoint{2.279412in}{2.004545in}}%
\pgfusepath{clip}%
\pgfsetbuttcap%
\pgfsetroundjoin%
\pgfsetlinewidth{0.725150pt}%
\definecolor{currentstroke}{rgb}{0.235526,0.309527,0.542944}%
\pgfsetstrokecolor{currentstroke}%
\pgfsetdash{}{0pt}%
\pgfpathmoveto{\pgfqpoint{5.244199in}{5.390973in}}%
\pgfpathlineto{\pgfqpoint{5.194743in}{5.383722in}}%
\pgfusepath{stroke}%
\end{pgfscope}%
\begin{pgfscope}%
\pgfpathrectangle{\pgfqpoint{3.985294in}{4.155455in}}{\pgfqpoint{2.279412in}{2.004545in}}%
\pgfusepath{clip}%
\pgfsetbuttcap%
\pgfsetroundjoin%
\pgfsetlinewidth{0.798808pt}%
\definecolor{currentstroke}{rgb}{0.214298,0.355619,0.551184}%
\pgfsetstrokecolor{currentstroke}%
\pgfsetdash{}{0pt}%
\pgfpathmoveto{\pgfqpoint{5.194743in}{5.383722in}}%
\pgfpathlineto{\pgfqpoint{5.145612in}{5.374906in}}%
\pgfusepath{stroke}%
\end{pgfscope}%
\begin{pgfscope}%
\pgfpathrectangle{\pgfqpoint{3.985294in}{4.155455in}}{\pgfqpoint{2.279412in}{2.004545in}}%
\pgfusepath{clip}%
\pgfsetbuttcap%
\pgfsetroundjoin%
\pgfsetlinewidth{0.794592pt}%
\definecolor{currentstroke}{rgb}{0.216210,0.351535,0.550627}%
\pgfsetstrokecolor{currentstroke}%
\pgfsetdash{}{0pt}%
\pgfpathmoveto{\pgfqpoint{5.145612in}{5.374906in}}%
\pgfpathlineto{\pgfqpoint{5.096961in}{5.364284in}}%
\pgfusepath{stroke}%
\end{pgfscope}%
\begin{pgfscope}%
\pgfpathrectangle{\pgfqpoint{3.985294in}{4.155455in}}{\pgfqpoint{2.279412in}{2.004545in}}%
\pgfusepath{clip}%
\pgfsetbuttcap%
\pgfsetroundjoin%
\pgfsetlinewidth{0.327013pt}%
\definecolor{currentstroke}{rgb}{0.271305,0.019942,0.347269}%
\pgfsetstrokecolor{currentstroke}%
\pgfsetdash{}{0pt}%
\pgfpathmoveto{\pgfqpoint{5.843087in}{4.796873in}}%
\pgfpathlineto{\pgfqpoint{5.792964in}{4.797567in}}%
\pgfusepath{stroke}%
\end{pgfscope}%
\begin{pgfscope}%
\pgfpathrectangle{\pgfqpoint{3.985294in}{4.155455in}}{\pgfqpoint{2.279412in}{2.004545in}}%
\pgfusepath{clip}%
\pgfsetbuttcap%
\pgfsetroundjoin%
\pgfsetlinewidth{0.317025pt}%
\definecolor{currentstroke}{rgb}{0.269944,0.014625,0.341379}%
\pgfsetstrokecolor{currentstroke}%
\pgfsetdash{}{0pt}%
\pgfpathmoveto{\pgfqpoint{5.792964in}{4.797567in}}%
\pgfpathlineto{\pgfqpoint{5.742866in}{4.798860in}}%
\pgfusepath{stroke}%
\end{pgfscope}%
\begin{pgfscope}%
\pgfpathrectangle{\pgfqpoint{3.985294in}{4.155455in}}{\pgfqpoint{2.279412in}{2.004545in}}%
\pgfusepath{clip}%
\pgfsetbuttcap%
\pgfsetroundjoin%
\pgfsetlinewidth{0.338551pt}%
\definecolor{currentstroke}{rgb}{0.273809,0.031497,0.358853}%
\pgfsetstrokecolor{currentstroke}%
\pgfsetdash{}{0pt}%
\pgfpathmoveto{\pgfqpoint{5.742866in}{4.798860in}}%
\pgfpathlineto{\pgfqpoint{5.692764in}{4.800288in}}%
\pgfusepath{stroke}%
\end{pgfscope}%
\begin{pgfscope}%
\pgfpathrectangle{\pgfqpoint{3.985294in}{4.155455in}}{\pgfqpoint{2.279412in}{2.004545in}}%
\pgfusepath{clip}%
\pgfsetbuttcap%
\pgfsetroundjoin%
\pgfsetlinewidth{0.340832pt}%
\definecolor{currentstroke}{rgb}{0.273809,0.031497,0.358853}%
\pgfsetstrokecolor{currentstroke}%
\pgfsetdash{}{0pt}%
\pgfpathmoveto{\pgfqpoint{5.692764in}{4.800288in}}%
\pgfpathlineto{\pgfqpoint{5.642681in}{4.802392in}}%
\pgfusepath{stroke}%
\end{pgfscope}%
\begin{pgfscope}%
\pgfpathrectangle{\pgfqpoint{3.985294in}{4.155455in}}{\pgfqpoint{2.279412in}{2.004545in}}%
\pgfusepath{clip}%
\pgfsetbuttcap%
\pgfsetroundjoin%
\pgfsetlinewidth{0.392745pt}%
\definecolor{currentstroke}{rgb}{0.280894,0.078907,0.402329}%
\pgfsetstrokecolor{currentstroke}%
\pgfsetdash{}{0pt}%
\pgfpathmoveto{\pgfqpoint{5.642681in}{4.802392in}}%
\pgfpathlineto{\pgfqpoint{5.592636in}{4.805262in}}%
\pgfusepath{stroke}%
\end{pgfscope}%
\begin{pgfscope}%
\pgfpathrectangle{\pgfqpoint{3.985294in}{4.155455in}}{\pgfqpoint{2.279412in}{2.004545in}}%
\pgfusepath{clip}%
\pgfsetbuttcap%
\pgfsetroundjoin%
\pgfsetlinewidth{0.402144pt}%
\definecolor{currentstroke}{rgb}{0.281446,0.084320,0.407414}%
\pgfsetstrokecolor{currentstroke}%
\pgfsetdash{}{0pt}%
\pgfpathmoveto{\pgfqpoint{5.592636in}{4.805262in}}%
\pgfpathlineto{\pgfqpoint{5.542632in}{4.808584in}}%
\pgfusepath{stroke}%
\end{pgfscope}%
\begin{pgfscope}%
\pgfpathrectangle{\pgfqpoint{3.985294in}{4.155455in}}{\pgfqpoint{2.279412in}{2.004545in}}%
\pgfusepath{clip}%
\pgfsetbuttcap%
\pgfsetroundjoin%
\pgfsetlinewidth{0.428696pt}%
\definecolor{currentstroke}{rgb}{0.282910,0.105393,0.426902}%
\pgfsetstrokecolor{currentstroke}%
\pgfsetdash{}{0pt}%
\pgfpathmoveto{\pgfqpoint{5.542632in}{4.808584in}}%
\pgfpathlineto{\pgfqpoint{5.492677in}{4.812467in}}%
\pgfusepath{stroke}%
\end{pgfscope}%
\begin{pgfscope}%
\pgfpathrectangle{\pgfqpoint{3.985294in}{4.155455in}}{\pgfqpoint{2.279412in}{2.004545in}}%
\pgfusepath{clip}%
\pgfsetbuttcap%
\pgfsetroundjoin%
\pgfsetlinewidth{0.476352pt}%
\definecolor{currentstroke}{rgb}{0.282623,0.140926,0.457517}%
\pgfsetstrokecolor{currentstroke}%
\pgfsetdash{}{0pt}%
\pgfpathmoveto{\pgfqpoint{5.492677in}{4.812467in}}%
\pgfpathlineto{\pgfqpoint{5.442796in}{4.817000in}}%
\pgfusepath{stroke}%
\end{pgfscope}%
\begin{pgfscope}%
\pgfpathrectangle{\pgfqpoint{3.985294in}{4.155455in}}{\pgfqpoint{2.279412in}{2.004545in}}%
\pgfusepath{clip}%
\pgfsetbuttcap%
\pgfsetroundjoin%
\pgfsetlinewidth{0.496276pt}%
\definecolor{currentstroke}{rgb}{0.281412,0.155834,0.469201}%
\pgfsetstrokecolor{currentstroke}%
\pgfsetdash{}{0pt}%
\pgfpathmoveto{\pgfqpoint{5.442796in}{4.817000in}}%
\pgfpathlineto{\pgfqpoint{5.392978in}{4.822082in}}%
\pgfusepath{stroke}%
\end{pgfscope}%
\begin{pgfscope}%
\pgfpathrectangle{\pgfqpoint{3.985294in}{4.155455in}}{\pgfqpoint{2.279412in}{2.004545in}}%
\pgfusepath{clip}%
\pgfsetbuttcap%
\pgfsetroundjoin%
\pgfsetlinewidth{0.514545pt}%
\definecolor{currentstroke}{rgb}{0.279574,0.170599,0.479997}%
\pgfsetstrokecolor{currentstroke}%
\pgfsetdash{}{0pt}%
\pgfpathmoveto{\pgfqpoint{5.392978in}{4.822082in}}%
\pgfpathlineto{\pgfqpoint{5.343241in}{4.827711in}}%
\pgfusepath{stroke}%
\end{pgfscope}%
\begin{pgfscope}%
\pgfpathrectangle{\pgfqpoint{3.985294in}{4.155455in}}{\pgfqpoint{2.279412in}{2.004545in}}%
\pgfusepath{clip}%
\pgfsetbuttcap%
\pgfsetroundjoin%
\pgfsetlinewidth{0.565794pt}%
\definecolor{currentstroke}{rgb}{0.273006,0.204520,0.501721}%
\pgfsetstrokecolor{currentstroke}%
\pgfsetdash{}{0pt}%
\pgfpathmoveto{\pgfqpoint{5.343241in}{4.827711in}}%
\pgfpathlineto{\pgfqpoint{5.293640in}{4.834218in}}%
\pgfusepath{stroke}%
\end{pgfscope}%
\begin{pgfscope}%
\pgfpathrectangle{\pgfqpoint{3.985294in}{4.155455in}}{\pgfqpoint{2.279412in}{2.004545in}}%
\pgfusepath{clip}%
\pgfsetbuttcap%
\pgfsetroundjoin%
\pgfsetlinewidth{0.610820pt}%
\definecolor{currentstroke}{rgb}{0.263663,0.237631,0.518762}%
\pgfsetstrokecolor{currentstroke}%
\pgfsetdash{}{0pt}%
\pgfpathmoveto{\pgfqpoint{5.293640in}{4.834218in}}%
\pgfpathlineto{\pgfqpoint{5.244299in}{4.842028in}}%
\pgfusepath{stroke}%
\end{pgfscope}%
\begin{pgfscope}%
\pgfpathrectangle{\pgfqpoint{3.985294in}{4.155455in}}{\pgfqpoint{2.279412in}{2.004545in}}%
\pgfusepath{clip}%
\pgfsetbuttcap%
\pgfsetroundjoin%
\pgfsetlinewidth{0.633633pt}%
\definecolor{currentstroke}{rgb}{0.258965,0.251537,0.524736}%
\pgfsetstrokecolor{currentstroke}%
\pgfsetdash{}{0pt}%
\pgfpathmoveto{\pgfqpoint{5.244299in}{4.842028in}}%
\pgfpathlineto{\pgfqpoint{5.195322in}{4.851474in}}%
\pgfusepath{stroke}%
\end{pgfscope}%
\begin{pgfscope}%
\pgfpathrectangle{\pgfqpoint{3.985294in}{4.155455in}}{\pgfqpoint{2.279412in}{2.004545in}}%
\pgfusepath{clip}%
\pgfsetbuttcap%
\pgfsetroundjoin%
\pgfsetlinewidth{0.610492pt}%
\definecolor{currentstroke}{rgb}{0.263663,0.237631,0.518762}%
\pgfsetstrokecolor{currentstroke}%
\pgfsetdash{}{0pt}%
\pgfpathmoveto{\pgfqpoint{5.195322in}{4.851474in}}%
\pgfpathlineto{\pgfqpoint{5.146825in}{4.862647in}}%
\pgfusepath{stroke}%
\end{pgfscope}%
\begin{pgfscope}%
\pgfpathrectangle{\pgfqpoint{3.985294in}{4.155455in}}{\pgfqpoint{2.279412in}{2.004545in}}%
\pgfusepath{clip}%
\pgfsetbuttcap%
\pgfsetroundjoin%
\pgfsetlinewidth{0.618173pt}%
\definecolor{currentstroke}{rgb}{0.262138,0.242286,0.520837}%
\pgfsetstrokecolor{currentstroke}%
\pgfsetdash{}{0pt}%
\pgfpathmoveto{\pgfqpoint{5.146825in}{4.862647in}}%
\pgfpathlineto{\pgfqpoint{5.099193in}{4.876341in}}%
\pgfusepath{stroke}%
\end{pgfscope}%
\begin{pgfscope}%
\pgfpathrectangle{\pgfqpoint{3.985294in}{4.155455in}}{\pgfqpoint{2.279412in}{2.004545in}}%
\pgfusepath{clip}%
\pgfsetbuttcap%
\pgfsetroundjoin%
\pgfsetlinewidth{0.645396pt}%
\definecolor{currentstroke}{rgb}{0.255645,0.260703,0.528312}%
\pgfsetstrokecolor{currentstroke}%
\pgfsetdash{}{0pt}%
\pgfpathmoveto{\pgfqpoint{5.099193in}{4.876341in}}%
\pgfpathlineto{\pgfqpoint{5.052898in}{4.893161in}}%
\pgfusepath{stroke}%
\end{pgfscope}%
\begin{pgfscope}%
\pgfpathrectangle{\pgfqpoint{3.985294in}{4.155455in}}{\pgfqpoint{2.279412in}{2.004545in}}%
\pgfusepath{clip}%
\pgfsetbuttcap%
\pgfsetroundjoin%
\pgfsetlinewidth{0.333335pt}%
\definecolor{currentstroke}{rgb}{0.272594,0.025563,0.353093}%
\pgfsetstrokecolor{currentstroke}%
\pgfsetdash{}{0pt}%
\pgfpathmoveto{\pgfqpoint{5.843087in}{4.887087in}}%
\pgfpathlineto{\pgfqpoint{5.792980in}{4.888670in}}%
\pgfusepath{stroke}%
\end{pgfscope}%
\begin{pgfscope}%
\pgfpathrectangle{\pgfqpoint{3.985294in}{4.155455in}}{\pgfqpoint{2.279412in}{2.004545in}}%
\pgfusepath{clip}%
\pgfsetbuttcap%
\pgfsetroundjoin%
\pgfsetlinewidth{0.341085pt}%
\definecolor{currentstroke}{rgb}{0.273809,0.031497,0.358853}%
\pgfsetstrokecolor{currentstroke}%
\pgfsetdash{}{0pt}%
\pgfpathmoveto{\pgfqpoint{5.792980in}{4.888670in}}%
\pgfpathlineto{\pgfqpoint{5.742867in}{4.889924in}}%
\pgfusepath{stroke}%
\end{pgfscope}%
\begin{pgfscope}%
\pgfpathrectangle{\pgfqpoint{3.985294in}{4.155455in}}{\pgfqpoint{2.279412in}{2.004545in}}%
\pgfusepath{clip}%
\pgfsetbuttcap%
\pgfsetroundjoin%
\pgfsetlinewidth{0.350158pt}%
\definecolor{currentstroke}{rgb}{0.276022,0.044167,0.370164}%
\pgfsetstrokecolor{currentstroke}%
\pgfsetdash{}{0pt}%
\pgfpathmoveto{\pgfqpoint{5.742867in}{4.889924in}}%
\pgfpathlineto{\pgfqpoint{5.692796in}{4.892109in}}%
\pgfusepath{stroke}%
\end{pgfscope}%
\begin{pgfscope}%
\pgfpathrectangle{\pgfqpoint{3.985294in}{4.155455in}}{\pgfqpoint{2.279412in}{2.004545in}}%
\pgfusepath{clip}%
\pgfsetbuttcap%
\pgfsetroundjoin%
\pgfsetlinewidth{0.381470pt}%
\definecolor{currentstroke}{rgb}{0.279566,0.067836,0.391917}%
\pgfsetstrokecolor{currentstroke}%
\pgfsetdash{}{0pt}%
\pgfpathmoveto{\pgfqpoint{5.692796in}{4.892109in}}%
\pgfpathlineto{\pgfqpoint{5.642739in}{4.894810in}}%
\pgfusepath{stroke}%
\end{pgfscope}%
\begin{pgfscope}%
\pgfpathrectangle{\pgfqpoint{3.985294in}{4.155455in}}{\pgfqpoint{2.279412in}{2.004545in}}%
\pgfusepath{clip}%
\pgfsetbuttcap%
\pgfsetroundjoin%
\pgfsetlinewidth{0.396765pt}%
\definecolor{currentstroke}{rgb}{0.280894,0.078907,0.402329}%
\pgfsetstrokecolor{currentstroke}%
\pgfsetdash{}{0pt}%
\pgfpathmoveto{\pgfqpoint{5.642739in}{4.894810in}}%
\pgfpathlineto{\pgfqpoint{5.592659in}{4.897142in}}%
\pgfusepath{stroke}%
\end{pgfscope}%
\begin{pgfscope}%
\pgfpathrectangle{\pgfqpoint{3.985294in}{4.155455in}}{\pgfqpoint{2.279412in}{2.004545in}}%
\pgfusepath{clip}%
\pgfsetbuttcap%
\pgfsetroundjoin%
\pgfsetlinewidth{0.439251pt}%
\definecolor{currentstroke}{rgb}{0.283197,0.115680,0.436115}%
\pgfsetstrokecolor{currentstroke}%
\pgfsetdash{}{0pt}%
\pgfpathmoveto{\pgfqpoint{5.592659in}{4.897142in}}%
\pgfpathlineto{\pgfqpoint{5.542585in}{4.899564in}}%
\pgfusepath{stroke}%
\end{pgfscope}%
\begin{pgfscope}%
\pgfpathrectangle{\pgfqpoint{3.985294in}{4.155455in}}{\pgfqpoint{2.279412in}{2.004545in}}%
\pgfusepath{clip}%
\pgfsetbuttcap%
\pgfsetroundjoin%
\pgfsetlinewidth{0.481617pt}%
\definecolor{currentstroke}{rgb}{0.282290,0.145912,0.461510}%
\pgfsetstrokecolor{currentstroke}%
\pgfsetdash{}{0pt}%
\pgfpathmoveto{\pgfqpoint{5.542585in}{4.899564in}}%
\pgfpathlineto{\pgfqpoint{5.492536in}{4.902390in}}%
\pgfusepath{stroke}%
\end{pgfscope}%
\begin{pgfscope}%
\pgfpathrectangle{\pgfqpoint{3.985294in}{4.155455in}}{\pgfqpoint{2.279412in}{2.004545in}}%
\pgfusepath{clip}%
\pgfsetbuttcap%
\pgfsetroundjoin%
\pgfsetlinewidth{0.515974pt}%
\definecolor{currentstroke}{rgb}{0.279574,0.170599,0.479997}%
\pgfsetstrokecolor{currentstroke}%
\pgfsetdash{}{0pt}%
\pgfpathmoveto{\pgfqpoint{5.492536in}{4.902390in}}%
\pgfpathlineto{\pgfqpoint{5.442513in}{4.905539in}}%
\pgfusepath{stroke}%
\end{pgfscope}%
\begin{pgfscope}%
\pgfpathrectangle{\pgfqpoint{3.985294in}{4.155455in}}{\pgfqpoint{2.279412in}{2.004545in}}%
\pgfusepath{clip}%
\pgfsetbuttcap%
\pgfsetroundjoin%
\pgfsetlinewidth{0.334960pt}%
\definecolor{currentstroke}{rgb}{0.272594,0.025563,0.353093}%
\pgfsetstrokecolor{currentstroke}%
\pgfsetdash{}{0pt}%
\pgfpathmoveto{\pgfqpoint{5.843087in}{4.932193in}}%
\pgfpathlineto{\pgfqpoint{5.793015in}{4.934485in}}%
\pgfusepath{stroke}%
\end{pgfscope}%
\begin{pgfscope}%
\pgfpathrectangle{\pgfqpoint{3.985294in}{4.155455in}}{\pgfqpoint{2.279412in}{2.004545in}}%
\pgfusepath{clip}%
\pgfsetbuttcap%
\pgfsetroundjoin%
\pgfsetlinewidth{0.345322pt}%
\definecolor{currentstroke}{rgb}{0.274952,0.037752,0.364543}%
\pgfsetstrokecolor{currentstroke}%
\pgfsetdash{}{0pt}%
\pgfpathmoveto{\pgfqpoint{5.793015in}{4.934485in}}%
\pgfpathlineto{\pgfqpoint{5.742906in}{4.936260in}}%
\pgfusepath{stroke}%
\end{pgfscope}%
\begin{pgfscope}%
\pgfpathrectangle{\pgfqpoint{3.985294in}{4.155455in}}{\pgfqpoint{2.279412in}{2.004545in}}%
\pgfusepath{clip}%
\pgfsetbuttcap%
\pgfsetroundjoin%
\pgfsetlinewidth{0.356467pt}%
\definecolor{currentstroke}{rgb}{0.277018,0.050344,0.375715}%
\pgfsetstrokecolor{currentstroke}%
\pgfsetdash{}{0pt}%
\pgfpathmoveto{\pgfqpoint{5.742906in}{4.936260in}}%
\pgfpathlineto{\pgfqpoint{5.692795in}{4.938010in}}%
\pgfusepath{stroke}%
\end{pgfscope}%
\begin{pgfscope}%
\pgfpathrectangle{\pgfqpoint{3.985294in}{4.155455in}}{\pgfqpoint{2.279412in}{2.004545in}}%
\pgfusepath{clip}%
\pgfsetbuttcap%
\pgfsetroundjoin%
\pgfsetlinewidth{0.384589pt}%
\definecolor{currentstroke}{rgb}{0.280267,0.073417,0.397163}%
\pgfsetstrokecolor{currentstroke}%
\pgfsetdash{}{0pt}%
\pgfpathmoveto{\pgfqpoint{5.692795in}{4.938010in}}%
\pgfpathlineto{\pgfqpoint{5.642686in}{4.939834in}}%
\pgfusepath{stroke}%
\end{pgfscope}%
\begin{pgfscope}%
\pgfpathrectangle{\pgfqpoint{3.985294in}{4.155455in}}{\pgfqpoint{2.279412in}{2.004545in}}%
\pgfusepath{clip}%
\pgfsetbuttcap%
\pgfsetroundjoin%
\pgfsetlinewidth{0.414712pt}%
\definecolor{currentstroke}{rgb}{0.282327,0.094955,0.417331}%
\pgfsetstrokecolor{currentstroke}%
\pgfsetdash{}{0pt}%
\pgfpathmoveto{\pgfqpoint{5.642686in}{4.939834in}}%
\pgfpathlineto{\pgfqpoint{5.592598in}{4.942044in}}%
\pgfusepath{stroke}%
\end{pgfscope}%
\begin{pgfscope}%
\pgfpathrectangle{\pgfqpoint{3.985294in}{4.155455in}}{\pgfqpoint{2.279412in}{2.004545in}}%
\pgfusepath{clip}%
\pgfsetbuttcap%
\pgfsetroundjoin%
\pgfsetlinewidth{0.442910pt}%
\definecolor{currentstroke}{rgb}{0.283197,0.115680,0.436115}%
\pgfsetstrokecolor{currentstroke}%
\pgfsetdash{}{0pt}%
\pgfpathmoveto{\pgfqpoint{5.592598in}{4.942044in}}%
\pgfpathlineto{\pgfqpoint{5.542511in}{4.944264in}}%
\pgfusepath{stroke}%
\end{pgfscope}%
\begin{pgfscope}%
\pgfpathrectangle{\pgfqpoint{3.985294in}{4.155455in}}{\pgfqpoint{2.279412in}{2.004545in}}%
\pgfusepath{clip}%
\pgfsetbuttcap%
\pgfsetroundjoin%
\pgfsetlinewidth{0.500851pt}%
\definecolor{currentstroke}{rgb}{0.280868,0.160771,0.472899}%
\pgfsetstrokecolor{currentstroke}%
\pgfsetdash{}{0pt}%
\pgfpathmoveto{\pgfqpoint{5.542511in}{4.944264in}}%
\pgfpathlineto{\pgfqpoint{5.492426in}{4.946526in}}%
\pgfusepath{stroke}%
\end{pgfscope}%
\begin{pgfscope}%
\pgfpathrectangle{\pgfqpoint{3.985294in}{4.155455in}}{\pgfqpoint{2.279412in}{2.004545in}}%
\pgfusepath{clip}%
\pgfsetbuttcap%
\pgfsetroundjoin%
\pgfsetlinewidth{0.565303pt}%
\definecolor{currentstroke}{rgb}{0.273006,0.204520,0.501721}%
\pgfsetstrokecolor{currentstroke}%
\pgfsetdash{}{0pt}%
\pgfpathmoveto{\pgfqpoint{5.492426in}{4.946526in}}%
\pgfpathlineto{\pgfqpoint{5.442367in}{4.949198in}}%
\pgfusepath{stroke}%
\end{pgfscope}%
\begin{pgfscope}%
\pgfpathrectangle{\pgfqpoint{3.985294in}{4.155455in}}{\pgfqpoint{2.279412in}{2.004545in}}%
\pgfusepath{clip}%
\pgfsetbuttcap%
\pgfsetroundjoin%
\pgfsetlinewidth{0.606262pt}%
\definecolor{currentstroke}{rgb}{0.265145,0.232956,0.516599}%
\pgfsetstrokecolor{currentstroke}%
\pgfsetdash{}{0pt}%
\pgfpathmoveto{\pgfqpoint{5.442367in}{4.949198in}}%
\pgfpathlineto{\pgfqpoint{5.392343in}{4.952342in}}%
\pgfusepath{stroke}%
\end{pgfscope}%
\begin{pgfscope}%
\pgfpathrectangle{\pgfqpoint{3.985294in}{4.155455in}}{\pgfqpoint{2.279412in}{2.004545in}}%
\pgfusepath{clip}%
\pgfsetbuttcap%
\pgfsetroundjoin%
\pgfsetlinewidth{0.321368pt}%
\definecolor{currentstroke}{rgb}{0.269944,0.014625,0.341379}%
\pgfsetstrokecolor{currentstroke}%
\pgfsetdash{}{0pt}%
\pgfpathmoveto{\pgfqpoint{5.843087in}{5.473475in}}%
\pgfpathlineto{\pgfqpoint{5.793017in}{5.471338in}}%
\pgfusepath{stroke}%
\end{pgfscope}%
\begin{pgfscope}%
\pgfpathrectangle{\pgfqpoint{3.985294in}{4.155455in}}{\pgfqpoint{2.279412in}{2.004545in}}%
\pgfusepath{clip}%
\pgfsetbuttcap%
\pgfsetroundjoin%
\pgfsetlinewidth{0.332259pt}%
\definecolor{currentstroke}{rgb}{0.272594,0.025563,0.353093}%
\pgfsetstrokecolor{currentstroke}%
\pgfsetdash{}{0pt}%
\pgfpathmoveto{\pgfqpoint{5.793017in}{5.471338in}}%
\pgfpathlineto{\pgfqpoint{5.742960in}{5.469072in}}%
\pgfusepath{stroke}%
\end{pgfscope}%
\begin{pgfscope}%
\pgfpathrectangle{\pgfqpoint{3.985294in}{4.155455in}}{\pgfqpoint{2.279412in}{2.004545in}}%
\pgfusepath{clip}%
\pgfsetbuttcap%
\pgfsetroundjoin%
\pgfsetlinewidth{0.343228pt}%
\definecolor{currentstroke}{rgb}{0.274952,0.037752,0.364543}%
\pgfsetstrokecolor{currentstroke}%
\pgfsetdash{}{0pt}%
\pgfpathmoveto{\pgfqpoint{5.742960in}{5.469072in}}%
\pgfpathlineto{\pgfqpoint{5.692929in}{5.466155in}}%
\pgfusepath{stroke}%
\end{pgfscope}%
\begin{pgfscope}%
\pgfpathrectangle{\pgfqpoint{3.985294in}{4.155455in}}{\pgfqpoint{2.279412in}{2.004545in}}%
\pgfusepath{clip}%
\pgfsetbuttcap%
\pgfsetroundjoin%
\pgfsetlinewidth{0.365977pt}%
\definecolor{currentstroke}{rgb}{0.277941,0.056324,0.381191}%
\pgfsetstrokecolor{currentstroke}%
\pgfsetdash{}{0pt}%
\pgfpathmoveto{\pgfqpoint{5.692929in}{5.466155in}}%
\pgfpathlineto{\pgfqpoint{5.642894in}{5.463158in}}%
\pgfusepath{stroke}%
\end{pgfscope}%
\begin{pgfscope}%
\pgfpathrectangle{\pgfqpoint{3.985294in}{4.155455in}}{\pgfqpoint{2.279412in}{2.004545in}}%
\pgfusepath{clip}%
\pgfsetbuttcap%
\pgfsetroundjoin%
\pgfsetlinewidth{0.384460pt}%
\definecolor{currentstroke}{rgb}{0.280267,0.073417,0.397163}%
\pgfsetstrokecolor{currentstroke}%
\pgfsetdash{}{0pt}%
\pgfpathmoveto{\pgfqpoint{5.642894in}{5.463158in}}%
\pgfpathlineto{\pgfqpoint{5.592875in}{5.459987in}}%
\pgfusepath{stroke}%
\end{pgfscope}%
\begin{pgfscope}%
\pgfpathrectangle{\pgfqpoint{3.985294in}{4.155455in}}{\pgfqpoint{2.279412in}{2.004545in}}%
\pgfusepath{clip}%
\pgfsetbuttcap%
\pgfsetroundjoin%
\pgfsetlinewidth{0.410041pt}%
\definecolor{currentstroke}{rgb}{0.281924,0.089666,0.412415}%
\pgfsetstrokecolor{currentstroke}%
\pgfsetdash{}{0pt}%
\pgfpathmoveto{\pgfqpoint{5.592875in}{5.459987in}}%
\pgfpathlineto{\pgfqpoint{5.542906in}{5.456231in}}%
\pgfusepath{stroke}%
\end{pgfscope}%
\begin{pgfscope}%
\pgfpathrectangle{\pgfqpoint{3.985294in}{4.155455in}}{\pgfqpoint{2.279412in}{2.004545in}}%
\pgfusepath{clip}%
\pgfsetbuttcap%
\pgfsetroundjoin%
\pgfsetlinewidth{0.461001pt}%
\definecolor{currentstroke}{rgb}{0.283072,0.130895,0.449241}%
\pgfsetstrokecolor{currentstroke}%
\pgfsetdash{}{0pt}%
\pgfpathmoveto{\pgfqpoint{5.542906in}{5.456231in}}%
\pgfpathlineto{\pgfqpoint{5.492966in}{5.452189in}}%
\pgfusepath{stroke}%
\end{pgfscope}%
\begin{pgfscope}%
\pgfpathrectangle{\pgfqpoint{3.985294in}{4.155455in}}{\pgfqpoint{2.279412in}{2.004545in}}%
\pgfusepath{clip}%
\pgfsetbuttcap%
\pgfsetroundjoin%
\pgfsetlinewidth{0.340407pt}%
\definecolor{currentstroke}{rgb}{0.273809,0.031497,0.358853}%
\pgfsetstrokecolor{currentstroke}%
\pgfsetdash{}{0pt}%
\pgfpathmoveto{\pgfqpoint{5.843087in}{5.518582in}}%
\pgfpathlineto{\pgfqpoint{5.792972in}{5.517161in}}%
\pgfusepath{stroke}%
\end{pgfscope}%
\begin{pgfscope}%
\pgfpathrectangle{\pgfqpoint{3.985294in}{4.155455in}}{\pgfqpoint{2.279412in}{2.004545in}}%
\pgfusepath{clip}%
\pgfsetbuttcap%
\pgfsetroundjoin%
\pgfsetlinewidth{0.328814pt}%
\definecolor{currentstroke}{rgb}{0.272594,0.025563,0.353093}%
\pgfsetstrokecolor{currentstroke}%
\pgfsetdash{}{0pt}%
\pgfpathmoveto{\pgfqpoint{5.792972in}{5.517161in}}%
\pgfpathlineto{\pgfqpoint{5.742903in}{5.514705in}}%
\pgfusepath{stroke}%
\end{pgfscope}%
\begin{pgfscope}%
\pgfpathrectangle{\pgfqpoint{3.985294in}{4.155455in}}{\pgfqpoint{2.279412in}{2.004545in}}%
\pgfusepath{clip}%
\pgfsetbuttcap%
\pgfsetroundjoin%
\pgfsetlinewidth{0.339568pt}%
\definecolor{currentstroke}{rgb}{0.273809,0.031497,0.358853}%
\pgfsetstrokecolor{currentstroke}%
\pgfsetdash{}{0pt}%
\pgfpathmoveto{\pgfqpoint{5.742903in}{5.514705in}}%
\pgfpathlineto{\pgfqpoint{5.692847in}{5.512076in}}%
\pgfusepath{stroke}%
\end{pgfscope}%
\begin{pgfscope}%
\pgfpathrectangle{\pgfqpoint{3.985294in}{4.155455in}}{\pgfqpoint{2.279412in}{2.004545in}}%
\pgfusepath{clip}%
\pgfsetbuttcap%
\pgfsetroundjoin%
\pgfsetlinewidth{0.359015pt}%
\definecolor{currentstroke}{rgb}{0.277018,0.050344,0.375715}%
\pgfsetstrokecolor{currentstroke}%
\pgfsetdash{}{0pt}%
\pgfpathmoveto{\pgfqpoint{5.692847in}{5.512076in}}%
\pgfpathlineto{\pgfqpoint{5.642842in}{5.508870in}}%
\pgfusepath{stroke}%
\end{pgfscope}%
\begin{pgfscope}%
\pgfpathrectangle{\pgfqpoint{3.985294in}{4.155455in}}{\pgfqpoint{2.279412in}{2.004545in}}%
\pgfusepath{clip}%
\pgfsetbuttcap%
\pgfsetroundjoin%
\pgfsetlinewidth{0.358898pt}%
\definecolor{currentstroke}{rgb}{0.277018,0.050344,0.375715}%
\pgfsetstrokecolor{currentstroke}%
\pgfsetdash{}{0pt}%
\pgfpathmoveto{\pgfqpoint{5.642842in}{5.508870in}}%
\pgfpathlineto{\pgfqpoint{5.592874in}{5.505133in}}%
\pgfusepath{stroke}%
\end{pgfscope}%
\begin{pgfscope}%
\pgfpathrectangle{\pgfqpoint{3.985294in}{4.155455in}}{\pgfqpoint{2.279412in}{2.004545in}}%
\pgfusepath{clip}%
\pgfsetbuttcap%
\pgfsetroundjoin%
\pgfsetlinewidth{0.394781pt}%
\definecolor{currentstroke}{rgb}{0.280894,0.078907,0.402329}%
\pgfsetstrokecolor{currentstroke}%
\pgfsetdash{}{0pt}%
\pgfpathmoveto{\pgfqpoint{5.592874in}{5.505133in}}%
\pgfpathlineto{\pgfqpoint{5.542924in}{5.501208in}}%
\pgfusepath{stroke}%
\end{pgfscope}%
\begin{pgfscope}%
\pgfpathrectangle{\pgfqpoint{3.985294in}{4.155455in}}{\pgfqpoint{2.279412in}{2.004545in}}%
\pgfusepath{clip}%
\pgfsetbuttcap%
\pgfsetroundjoin%
\pgfsetlinewidth{0.425580pt}%
\definecolor{currentstroke}{rgb}{0.282910,0.105393,0.426902}%
\pgfsetstrokecolor{currentstroke}%
\pgfsetdash{}{0pt}%
\pgfpathmoveto{\pgfqpoint{5.542924in}{5.501208in}}%
\pgfpathlineto{\pgfqpoint{5.493023in}{5.496808in}}%
\pgfusepath{stroke}%
\end{pgfscope}%
\begin{pgfscope}%
\pgfpathrectangle{\pgfqpoint{3.985294in}{4.155455in}}{\pgfqpoint{2.279412in}{2.004545in}}%
\pgfusepath{clip}%
\pgfsetbuttcap%
\pgfsetroundjoin%
\pgfsetlinewidth{0.459905pt}%
\definecolor{currentstroke}{rgb}{0.283072,0.130895,0.449241}%
\pgfsetstrokecolor{currentstroke}%
\pgfsetdash{}{0pt}%
\pgfpathmoveto{\pgfqpoint{5.493023in}{5.496808in}}%
\pgfpathlineto{\pgfqpoint{5.443190in}{5.491872in}}%
\pgfusepath{stroke}%
\end{pgfscope}%
\begin{pgfscope}%
\pgfpathrectangle{\pgfqpoint{3.985294in}{4.155455in}}{\pgfqpoint{2.279412in}{2.004545in}}%
\pgfusepath{clip}%
\pgfsetbuttcap%
\pgfsetroundjoin%
\pgfsetlinewidth{0.503417pt}%
\definecolor{currentstroke}{rgb}{0.280868,0.160771,0.472899}%
\pgfsetstrokecolor{currentstroke}%
\pgfsetdash{}{0pt}%
\pgfpathmoveto{\pgfqpoint{5.443190in}{5.491872in}}%
\pgfpathlineto{\pgfqpoint{5.393409in}{5.486521in}}%
\pgfusepath{stroke}%
\end{pgfscope}%
\begin{pgfscope}%
\pgfpathrectangle{\pgfqpoint{3.985294in}{4.155455in}}{\pgfqpoint{2.279412in}{2.004545in}}%
\pgfusepath{clip}%
\pgfsetbuttcap%
\pgfsetroundjoin%
\pgfsetlinewidth{0.515285pt}%
\definecolor{currentstroke}{rgb}{0.279574,0.170599,0.479997}%
\pgfsetstrokecolor{currentstroke}%
\pgfsetdash{}{0pt}%
\pgfpathmoveto{\pgfqpoint{5.393409in}{5.486521in}}%
\pgfpathlineto{\pgfqpoint{5.343725in}{5.480537in}}%
\pgfusepath{stroke}%
\end{pgfscope}%
\begin{pgfscope}%
\pgfpathrectangle{\pgfqpoint{3.985294in}{4.155455in}}{\pgfqpoint{2.279412in}{2.004545in}}%
\pgfusepath{clip}%
\pgfsetbuttcap%
\pgfsetroundjoin%
\pgfsetlinewidth{0.551844pt}%
\definecolor{currentstroke}{rgb}{0.275191,0.194905,0.496005}%
\pgfsetstrokecolor{currentstroke}%
\pgfsetdash{}{0pt}%
\pgfpathmoveto{\pgfqpoint{5.343725in}{5.480537in}}%
\pgfpathlineto{\pgfqpoint{5.294212in}{5.473538in}}%
\pgfusepath{stroke}%
\end{pgfscope}%
\begin{pgfscope}%
\pgfpathrectangle{\pgfqpoint{3.985294in}{4.155455in}}{\pgfqpoint{2.279412in}{2.004545in}}%
\pgfusepath{clip}%
\pgfsetbuttcap%
\pgfsetroundjoin%
\pgfsetlinewidth{0.594172pt}%
\definecolor{currentstroke}{rgb}{0.267968,0.223549,0.512008}%
\pgfsetstrokecolor{currentstroke}%
\pgfsetdash{}{0pt}%
\pgfpathmoveto{\pgfqpoint{5.294212in}{5.473538in}}%
\pgfpathlineto{\pgfqpoint{5.244882in}{5.465614in}}%
\pgfusepath{stroke}%
\end{pgfscope}%
\begin{pgfscope}%
\pgfpathrectangle{\pgfqpoint{3.985294in}{4.155455in}}{\pgfqpoint{2.279412in}{2.004545in}}%
\pgfusepath{clip}%
\pgfsetbuttcap%
\pgfsetroundjoin%
\pgfsetlinewidth{0.606916pt}%
\definecolor{currentstroke}{rgb}{0.265145,0.232956,0.516599}%
\pgfsetstrokecolor{currentstroke}%
\pgfsetdash{}{0pt}%
\pgfpathmoveto{\pgfqpoint{5.244882in}{5.465614in}}%
\pgfpathlineto{\pgfqpoint{5.195813in}{5.456527in}}%
\pgfusepath{stroke}%
\end{pgfscope}%
\begin{pgfscope}%
\pgfpathrectangle{\pgfqpoint{3.985294in}{4.155455in}}{\pgfqpoint{2.279412in}{2.004545in}}%
\pgfusepath{clip}%
\pgfsetbuttcap%
\pgfsetroundjoin%
\pgfsetlinewidth{0.644168pt}%
\definecolor{currentstroke}{rgb}{0.257322,0.256130,0.526563}%
\pgfsetstrokecolor{currentstroke}%
\pgfsetdash{}{0pt}%
\pgfpathmoveto{\pgfqpoint{5.195813in}{5.456527in}}%
\pgfpathlineto{\pgfqpoint{5.147201in}{5.445737in}}%
\pgfusepath{stroke}%
\end{pgfscope}%
\begin{pgfscope}%
\pgfpathrectangle{\pgfqpoint{3.985294in}{4.155455in}}{\pgfqpoint{2.279412in}{2.004545in}}%
\pgfusepath{clip}%
\pgfsetbuttcap%
\pgfsetroundjoin%
\pgfsetlinewidth{0.673727pt}%
\definecolor{currentstroke}{rgb}{0.248629,0.278775,0.534556}%
\pgfsetstrokecolor{currentstroke}%
\pgfsetdash{}{0pt}%
\pgfpathmoveto{\pgfqpoint{5.147201in}{5.445737in}}%
\pgfpathlineto{\pgfqpoint{5.099632in}{5.431938in}}%
\pgfusepath{stroke}%
\end{pgfscope}%
\begin{pgfscope}%
\pgfpathrectangle{\pgfqpoint{3.985294in}{4.155455in}}{\pgfqpoint{2.279412in}{2.004545in}}%
\pgfusepath{clip}%
\pgfsetbuttcap%
\pgfsetroundjoin%
\pgfsetlinewidth{0.692063pt}%
\definecolor{currentstroke}{rgb}{0.244972,0.287675,0.537260}%
\pgfsetstrokecolor{currentstroke}%
\pgfsetdash{}{0pt}%
\pgfpathmoveto{\pgfqpoint{5.099632in}{5.431938in}}%
\pgfpathlineto{\pgfqpoint{5.053415in}{5.414917in}}%
\pgfusepath{stroke}%
\end{pgfscope}%
\begin{pgfscope}%
\pgfpathrectangle{\pgfqpoint{3.985294in}{4.155455in}}{\pgfqpoint{2.279412in}{2.004545in}}%
\pgfusepath{clip}%
\pgfsetbuttcap%
\pgfsetroundjoin%
\pgfsetlinewidth{0.627671pt}%
\definecolor{currentstroke}{rgb}{0.260571,0.246922,0.522828}%
\pgfsetstrokecolor{currentstroke}%
\pgfsetdash{}{0pt}%
\pgfpathmoveto{\pgfqpoint{5.053415in}{5.414917in}}%
\pgfpathlineto{\pgfqpoint{5.008633in}{5.395142in}}%
\pgfusepath{stroke}%
\end{pgfscope}%
\begin{pgfscope}%
\pgfpathrectangle{\pgfqpoint{3.985294in}{4.155455in}}{\pgfqpoint{2.279412in}{2.004545in}}%
\pgfusepath{clip}%
\pgfsetbuttcap%
\pgfsetroundjoin%
\pgfsetlinewidth{0.732619pt}%
\definecolor{currentstroke}{rgb}{0.233603,0.313828,0.543914}%
\pgfsetstrokecolor{currentstroke}%
\pgfsetdash{}{0pt}%
\pgfpathmoveto{\pgfqpoint{5.008633in}{5.395142in}}%
\pgfpathlineto{\pgfqpoint{4.966364in}{5.371670in}}%
\pgfusepath{stroke}%
\end{pgfscope}%
\begin{pgfscope}%
\pgfpathrectangle{\pgfqpoint{3.985294in}{4.155455in}}{\pgfqpoint{2.279412in}{2.004545in}}%
\pgfusepath{clip}%
\pgfsetbuttcap%
\pgfsetroundjoin%
\pgfsetlinewidth{0.812944pt}%
\definecolor{currentstroke}{rgb}{0.210503,0.363727,0.552206}%
\pgfsetstrokecolor{currentstroke}%
\pgfsetdash{}{0pt}%
\pgfpathmoveto{\pgfqpoint{4.966364in}{5.371670in}}%
\pgfpathlineto{\pgfqpoint{4.928032in}{5.343467in}}%
\pgfusepath{stroke}%
\end{pgfscope}%
\begin{pgfscope}%
\pgfpathrectangle{\pgfqpoint{3.985294in}{4.155455in}}{\pgfqpoint{2.279412in}{2.004545in}}%
\pgfusepath{clip}%
\pgfsetbuttcap%
\pgfsetroundjoin%
\pgfsetlinewidth{0.807264pt}%
\definecolor{currentstroke}{rgb}{0.212395,0.359683,0.551710}%
\pgfsetstrokecolor{currentstroke}%
\pgfsetdash{}{0pt}%
\pgfpathmoveto{\pgfqpoint{4.928032in}{5.343467in}}%
\pgfpathlineto{\pgfqpoint{4.894383in}{5.311051in}}%
\pgfusepath{stroke}%
\end{pgfscope}%
\begin{pgfscope}%
\pgfpathrectangle{\pgfqpoint{3.985294in}{4.155455in}}{\pgfqpoint{2.279412in}{2.004545in}}%
\pgfusepath{clip}%
\pgfsetbuttcap%
\pgfsetroundjoin%
\pgfsetlinewidth{0.861486pt}%
\definecolor{currentstroke}{rgb}{0.197636,0.391528,0.554969}%
\pgfsetstrokecolor{currentstroke}%
\pgfsetdash{}{0pt}%
\pgfpathmoveto{\pgfqpoint{4.894383in}{5.311051in}}%
\pgfpathlineto{\pgfqpoint{4.859210in}{5.280089in}}%
\pgfusepath{stroke}%
\end{pgfscope}%
\begin{pgfscope}%
\pgfpathrectangle{\pgfqpoint{3.985294in}{4.155455in}}{\pgfqpoint{2.279412in}{2.004545in}}%
\pgfusepath{clip}%
\pgfsetbuttcap%
\pgfsetroundjoin%
\pgfsetlinewidth{0.327599pt}%
\definecolor{currentstroke}{rgb}{0.271305,0.019942,0.347269}%
\pgfsetstrokecolor{currentstroke}%
\pgfsetdash{}{0pt}%
\pgfpathmoveto{\pgfqpoint{5.843087in}{5.563688in}}%
\pgfpathlineto{\pgfqpoint{5.792964in}{5.562417in}}%
\pgfusepath{stroke}%
\end{pgfscope}%
\begin{pgfscope}%
\pgfpathrectangle{\pgfqpoint{3.985294in}{4.155455in}}{\pgfqpoint{2.279412in}{2.004545in}}%
\pgfusepath{clip}%
\pgfsetbuttcap%
\pgfsetroundjoin%
\pgfsetlinewidth{0.327808pt}%
\definecolor{currentstroke}{rgb}{0.271305,0.019942,0.347269}%
\pgfsetstrokecolor{currentstroke}%
\pgfsetdash{}{0pt}%
\pgfpathmoveto{\pgfqpoint{5.792964in}{5.562417in}}%
\pgfpathlineto{\pgfqpoint{5.742868in}{5.560699in}}%
\pgfusepath{stroke}%
\end{pgfscope}%
\begin{pgfscope}%
\pgfpathrectangle{\pgfqpoint{3.985294in}{4.155455in}}{\pgfqpoint{2.279412in}{2.004545in}}%
\pgfusepath{clip}%
\pgfsetbuttcap%
\pgfsetroundjoin%
\pgfsetlinewidth{0.331063pt}%
\definecolor{currentstroke}{rgb}{0.272594,0.025563,0.353093}%
\pgfsetstrokecolor{currentstroke}%
\pgfsetdash{}{0pt}%
\pgfpathmoveto{\pgfqpoint{5.742868in}{5.560699in}}%
\pgfpathlineto{\pgfqpoint{5.692819in}{5.557949in}}%
\pgfusepath{stroke}%
\end{pgfscope}%
\begin{pgfscope}%
\pgfpathrectangle{\pgfqpoint{3.985294in}{4.155455in}}{\pgfqpoint{2.279412in}{2.004545in}}%
\pgfusepath{clip}%
\pgfsetbuttcap%
\pgfsetroundjoin%
\pgfsetlinewidth{0.347110pt}%
\definecolor{currentstroke}{rgb}{0.274952,0.037752,0.364543}%
\pgfsetstrokecolor{currentstroke}%
\pgfsetdash{}{0pt}%
\pgfpathmoveto{\pgfqpoint{5.692819in}{5.557949in}}%
\pgfpathlineto{\pgfqpoint{5.642790in}{5.554890in}}%
\pgfusepath{stroke}%
\end{pgfscope}%
\begin{pgfscope}%
\pgfpathrectangle{\pgfqpoint{3.985294in}{4.155455in}}{\pgfqpoint{2.279412in}{2.004545in}}%
\pgfusepath{clip}%
\pgfsetbuttcap%
\pgfsetroundjoin%
\pgfsetlinewidth{0.367111pt}%
\definecolor{currentstroke}{rgb}{0.277941,0.056324,0.381191}%
\pgfsetstrokecolor{currentstroke}%
\pgfsetdash{}{0pt}%
\pgfpathmoveto{\pgfqpoint{5.642790in}{5.554890in}}%
\pgfpathlineto{\pgfqpoint{5.592816in}{5.551221in}}%
\pgfusepath{stroke}%
\end{pgfscope}%
\begin{pgfscope}%
\pgfpathrectangle{\pgfqpoint{3.985294in}{4.155455in}}{\pgfqpoint{2.279412in}{2.004545in}}%
\pgfusepath{clip}%
\pgfsetbuttcap%
\pgfsetroundjoin%
\pgfsetlinewidth{0.378307pt}%
\definecolor{currentstroke}{rgb}{0.279566,0.067836,0.391917}%
\pgfsetstrokecolor{currentstroke}%
\pgfsetdash{}{0pt}%
\pgfpathmoveto{\pgfqpoint{5.592816in}{5.551221in}}%
\pgfpathlineto{\pgfqpoint{5.542915in}{5.546855in}}%
\pgfusepath{stroke}%
\end{pgfscope}%
\begin{pgfscope}%
\pgfpathrectangle{\pgfqpoint{3.985294in}{4.155455in}}{\pgfqpoint{2.279412in}{2.004545in}}%
\pgfusepath{clip}%
\pgfsetbuttcap%
\pgfsetroundjoin%
\pgfsetlinewidth{0.401847pt}%
\definecolor{currentstroke}{rgb}{0.281446,0.084320,0.407414}%
\pgfsetstrokecolor{currentstroke}%
\pgfsetdash{}{0pt}%
\pgfpathmoveto{\pgfqpoint{5.542915in}{5.546855in}}%
\pgfpathlineto{\pgfqpoint{5.493072in}{5.542034in}}%
\pgfusepath{stroke}%
\end{pgfscope}%
\begin{pgfscope}%
\pgfpathrectangle{\pgfqpoint{3.985294in}{4.155455in}}{\pgfqpoint{2.279412in}{2.004545in}}%
\pgfusepath{clip}%
\pgfsetbuttcap%
\pgfsetroundjoin%
\pgfsetlinewidth{0.433171pt}%
\definecolor{currentstroke}{rgb}{0.283091,0.110553,0.431554}%
\pgfsetstrokecolor{currentstroke}%
\pgfsetdash{}{0pt}%
\pgfpathmoveto{\pgfqpoint{5.493072in}{5.542034in}}%
\pgfpathlineto{\pgfqpoint{5.443260in}{5.536938in}}%
\pgfusepath{stroke}%
\end{pgfscope}%
\begin{pgfscope}%
\pgfpathrectangle{\pgfqpoint{3.985294in}{4.155455in}}{\pgfqpoint{2.279412in}{2.004545in}}%
\pgfusepath{clip}%
\pgfsetbuttcap%
\pgfsetroundjoin%
\pgfsetlinewidth{0.451262pt}%
\definecolor{currentstroke}{rgb}{0.283229,0.120777,0.440584}%
\pgfsetstrokecolor{currentstroke}%
\pgfsetdash{}{0pt}%
\pgfpathmoveto{\pgfqpoint{5.443260in}{5.536938in}}%
\pgfpathlineto{\pgfqpoint{5.393499in}{5.531458in}}%
\pgfusepath{stroke}%
\end{pgfscope}%
\begin{pgfscope}%
\pgfpathrectangle{\pgfqpoint{3.985294in}{4.155455in}}{\pgfqpoint{2.279412in}{2.004545in}}%
\pgfusepath{clip}%
\pgfsetbuttcap%
\pgfsetroundjoin%
\pgfsetlinewidth{0.471024pt}%
\definecolor{currentstroke}{rgb}{0.282884,0.135920,0.453427}%
\pgfsetstrokecolor{currentstroke}%
\pgfsetdash{}{0pt}%
\pgfpathmoveto{\pgfqpoint{5.393499in}{5.531458in}}%
\pgfpathlineto{\pgfqpoint{5.343926in}{5.524839in}}%
\pgfusepath{stroke}%
\end{pgfscope}%
\begin{pgfscope}%
\pgfpathrectangle{\pgfqpoint{3.985294in}{4.155455in}}{\pgfqpoint{2.279412in}{2.004545in}}%
\pgfusepath{clip}%
\pgfsetbuttcap%
\pgfsetroundjoin%
\pgfsetlinewidth{0.514398pt}%
\definecolor{currentstroke}{rgb}{0.279574,0.170599,0.479997}%
\pgfsetstrokecolor{currentstroke}%
\pgfsetdash{}{0pt}%
\pgfpathmoveto{\pgfqpoint{5.343926in}{5.524839in}}%
\pgfpathlineto{\pgfqpoint{5.294525in}{5.517266in}}%
\pgfusepath{stroke}%
\end{pgfscope}%
\begin{pgfscope}%
\pgfpathrectangle{\pgfqpoint{3.985294in}{4.155455in}}{\pgfqpoint{2.279412in}{2.004545in}}%
\pgfusepath{clip}%
\pgfsetbuttcap%
\pgfsetroundjoin%
\pgfsetlinewidth{0.518872pt}%
\definecolor{currentstroke}{rgb}{0.279574,0.170599,0.479997}%
\pgfsetstrokecolor{currentstroke}%
\pgfsetdash{}{0pt}%
\pgfpathmoveto{\pgfqpoint{5.294525in}{5.517266in}}%
\pgfpathlineto{\pgfqpoint{5.245255in}{5.509057in}}%
\pgfusepath{stroke}%
\end{pgfscope}%
\begin{pgfscope}%
\pgfpathrectangle{\pgfqpoint{3.985294in}{4.155455in}}{\pgfqpoint{2.279412in}{2.004545in}}%
\pgfusepath{clip}%
\pgfsetbuttcap%
\pgfsetroundjoin%
\pgfsetlinewidth{0.542862pt}%
\definecolor{currentstroke}{rgb}{0.276194,0.190074,0.493001}%
\pgfsetstrokecolor{currentstroke}%
\pgfsetdash{}{0pt}%
\pgfpathmoveto{\pgfqpoint{5.245255in}{5.509057in}}%
\pgfpathlineto{\pgfqpoint{5.196426in}{5.499101in}}%
\pgfusepath{stroke}%
\end{pgfscope}%
\begin{pgfscope}%
\pgfpathrectangle{\pgfqpoint{3.985294in}{4.155455in}}{\pgfqpoint{2.279412in}{2.004545in}}%
\pgfusepath{clip}%
\pgfsetbuttcap%
\pgfsetroundjoin%
\pgfsetlinewidth{0.534298pt}%
\definecolor{currentstroke}{rgb}{0.278012,0.180367,0.486697}%
\pgfsetstrokecolor{currentstroke}%
\pgfsetdash{}{0pt}%
\pgfpathmoveto{\pgfqpoint{5.196426in}{5.499101in}}%
\pgfpathlineto{\pgfqpoint{5.148208in}{5.487007in}}%
\pgfusepath{stroke}%
\end{pgfscope}%
\begin{pgfscope}%
\pgfpathrectangle{\pgfqpoint{3.985294in}{4.155455in}}{\pgfqpoint{2.279412in}{2.004545in}}%
\pgfusepath{clip}%
\pgfsetbuttcap%
\pgfsetroundjoin%
\pgfsetlinewidth{0.520454pt}%
\definecolor{currentstroke}{rgb}{0.279574,0.170599,0.479997}%
\pgfsetstrokecolor{currentstroke}%
\pgfsetdash{}{0pt}%
\pgfpathmoveto{\pgfqpoint{5.148208in}{5.487007in}}%
\pgfpathlineto{\pgfqpoint{5.100938in}{5.472412in}}%
\pgfusepath{stroke}%
\end{pgfscope}%
\begin{pgfscope}%
\pgfpathrectangle{\pgfqpoint{3.985294in}{4.155455in}}{\pgfqpoint{2.279412in}{2.004545in}}%
\pgfusepath{clip}%
\pgfsetbuttcap%
\pgfsetroundjoin%
\pgfsetlinewidth{0.587172pt}%
\definecolor{currentstroke}{rgb}{0.269308,0.218818,0.509577}%
\pgfsetstrokecolor{currentstroke}%
\pgfsetdash{}{0pt}%
\pgfpathmoveto{\pgfqpoint{5.100938in}{5.472412in}}%
\pgfpathlineto{\pgfqpoint{5.056133in}{5.452968in}}%
\pgfusepath{stroke}%
\end{pgfscope}%
\begin{pgfscope}%
\pgfpathrectangle{\pgfqpoint{3.985294in}{4.155455in}}{\pgfqpoint{2.279412in}{2.004545in}}%
\pgfusepath{clip}%
\pgfsetbuttcap%
\pgfsetroundjoin%
\pgfsetlinewidth{0.576283pt}%
\definecolor{currentstroke}{rgb}{0.270595,0.214069,0.507052}%
\pgfsetstrokecolor{currentstroke}%
\pgfsetdash{}{0pt}%
\pgfpathmoveto{\pgfqpoint{5.056133in}{5.452968in}}%
\pgfpathlineto{\pgfqpoint{5.013715in}{5.429606in}}%
\pgfusepath{stroke}%
\end{pgfscope}%
\begin{pgfscope}%
\pgfpathrectangle{\pgfqpoint{3.985294in}{4.155455in}}{\pgfqpoint{2.279412in}{2.004545in}}%
\pgfusepath{clip}%
\pgfsetbuttcap%
\pgfsetroundjoin%
\pgfsetlinewidth{0.318516pt}%
\definecolor{currentstroke}{rgb}{0.269944,0.014625,0.341379}%
\pgfsetstrokecolor{currentstroke}%
\pgfsetdash{}{0pt}%
\pgfpathmoveto{\pgfqpoint{5.791795in}{4.706659in}}%
\pgfpathlineto{\pgfqpoint{5.741712in}{4.708663in}}%
\pgfusepath{stroke}%
\end{pgfscope}%
\begin{pgfscope}%
\pgfpathrectangle{\pgfqpoint{3.985294in}{4.155455in}}{\pgfqpoint{2.279412in}{2.004545in}}%
\pgfusepath{clip}%
\pgfsetbuttcap%
\pgfsetroundjoin%
\pgfsetlinewidth{0.330957pt}%
\definecolor{currentstroke}{rgb}{0.272594,0.025563,0.353093}%
\pgfsetstrokecolor{currentstroke}%
\pgfsetdash{}{0pt}%
\pgfpathmoveto{\pgfqpoint{5.741712in}{4.708663in}}%
\pgfpathlineto{\pgfqpoint{5.691646in}{4.711018in}}%
\pgfusepath{stroke}%
\end{pgfscope}%
\begin{pgfscope}%
\pgfpathrectangle{\pgfqpoint{3.985294in}{4.155455in}}{\pgfqpoint{2.279412in}{2.004545in}}%
\pgfusepath{clip}%
\pgfsetbuttcap%
\pgfsetroundjoin%
\pgfsetlinewidth{0.346540pt}%
\definecolor{currentstroke}{rgb}{0.274952,0.037752,0.364543}%
\pgfsetstrokecolor{currentstroke}%
\pgfsetdash{}{0pt}%
\pgfpathmoveto{\pgfqpoint{5.691646in}{4.711018in}}%
\pgfpathlineto{\pgfqpoint{5.641630in}{4.714070in}}%
\pgfusepath{stroke}%
\end{pgfscope}%
\begin{pgfscope}%
\pgfpathrectangle{\pgfqpoint{3.985294in}{4.155455in}}{\pgfqpoint{2.279412in}{2.004545in}}%
\pgfusepath{clip}%
\pgfsetbuttcap%
\pgfsetroundjoin%
\pgfsetlinewidth{0.346036pt}%
\definecolor{currentstroke}{rgb}{0.274952,0.037752,0.364543}%
\pgfsetstrokecolor{currentstroke}%
\pgfsetdash{}{0pt}%
\pgfpathmoveto{\pgfqpoint{5.641630in}{4.714070in}}%
\pgfpathlineto{\pgfqpoint{5.591710in}{4.718169in}}%
\pgfusepath{stroke}%
\end{pgfscope}%
\begin{pgfscope}%
\pgfpathrectangle{\pgfqpoint{3.985294in}{4.155455in}}{\pgfqpoint{2.279412in}{2.004545in}}%
\pgfusepath{clip}%
\pgfsetbuttcap%
\pgfsetroundjoin%
\pgfsetlinewidth{0.373803pt}%
\definecolor{currentstroke}{rgb}{0.278791,0.062145,0.386592}%
\pgfsetstrokecolor{currentstroke}%
\pgfsetdash{}{0pt}%
\pgfpathmoveto{\pgfqpoint{5.591710in}{4.718169in}}%
\pgfpathlineto{\pgfqpoint{5.541900in}{4.723222in}}%
\pgfusepath{stroke}%
\end{pgfscope}%
\begin{pgfscope}%
\pgfpathrectangle{\pgfqpoint{3.985294in}{4.155455in}}{\pgfqpoint{2.279412in}{2.004545in}}%
\pgfusepath{clip}%
\pgfsetbuttcap%
\pgfsetroundjoin%
\pgfsetlinewidth{0.389306pt}%
\definecolor{currentstroke}{rgb}{0.280267,0.073417,0.397163}%
\pgfsetstrokecolor{currentstroke}%
\pgfsetdash{}{0pt}%
\pgfpathmoveto{\pgfqpoint{5.541900in}{4.723222in}}%
\pgfpathlineto{\pgfqpoint{5.492128in}{4.728608in}}%
\pgfusepath{stroke}%
\end{pgfscope}%
\begin{pgfscope}%
\pgfpathrectangle{\pgfqpoint{3.985294in}{4.155455in}}{\pgfqpoint{2.279412in}{2.004545in}}%
\pgfusepath{clip}%
\pgfsetbuttcap%
\pgfsetroundjoin%
\pgfsetlinewidth{0.398488pt}%
\definecolor{currentstroke}{rgb}{0.281446,0.084320,0.407414}%
\pgfsetstrokecolor{currentstroke}%
\pgfsetdash{}{0pt}%
\pgfpathmoveto{\pgfqpoint{5.492128in}{4.728608in}}%
\pgfpathlineto{\pgfqpoint{5.442354in}{4.733993in}}%
\pgfusepath{stroke}%
\end{pgfscope}%
\begin{pgfscope}%
\pgfpathrectangle{\pgfqpoint{3.985294in}{4.155455in}}{\pgfqpoint{2.279412in}{2.004545in}}%
\pgfusepath{clip}%
\pgfsetbuttcap%
\pgfsetroundjoin%
\pgfsetlinewidth{0.432059pt}%
\definecolor{currentstroke}{rgb}{0.283091,0.110553,0.431554}%
\pgfsetstrokecolor{currentstroke}%
\pgfsetdash{}{0pt}%
\pgfpathmoveto{\pgfqpoint{5.442354in}{4.733993in}}%
\pgfpathlineto{\pgfqpoint{5.392675in}{4.740007in}}%
\pgfusepath{stroke}%
\end{pgfscope}%
\begin{pgfscope}%
\pgfpathrectangle{\pgfqpoint{3.985294in}{4.155455in}}{\pgfqpoint{2.279412in}{2.004545in}}%
\pgfusepath{clip}%
\pgfsetbuttcap%
\pgfsetroundjoin%
\pgfsetlinewidth{0.447987pt}%
\definecolor{currentstroke}{rgb}{0.283229,0.120777,0.440584}%
\pgfsetstrokecolor{currentstroke}%
\pgfsetdash{}{0pt}%
\pgfpathmoveto{\pgfqpoint{5.392675in}{4.740007in}}%
\pgfpathlineto{\pgfqpoint{5.343175in}{4.747049in}}%
\pgfusepath{stroke}%
\end{pgfscope}%
\begin{pgfscope}%
\pgfpathrectangle{\pgfqpoint{3.985294in}{4.155455in}}{\pgfqpoint{2.279412in}{2.004545in}}%
\pgfusepath{clip}%
\pgfsetbuttcap%
\pgfsetroundjoin%
\pgfsetlinewidth{0.483902pt}%
\definecolor{currentstroke}{rgb}{0.282290,0.145912,0.461510}%
\pgfsetstrokecolor{currentstroke}%
\pgfsetdash{}{0pt}%
\pgfpathmoveto{\pgfqpoint{5.343175in}{4.747049in}}%
\pgfpathlineto{\pgfqpoint{5.293862in}{4.755023in}}%
\pgfusepath{stroke}%
\end{pgfscope}%
\begin{pgfscope}%
\pgfpathrectangle{\pgfqpoint{3.985294in}{4.155455in}}{\pgfqpoint{2.279412in}{2.004545in}}%
\pgfusepath{clip}%
\pgfsetbuttcap%
\pgfsetroundjoin%
\pgfsetlinewidth{0.504077pt}%
\definecolor{currentstroke}{rgb}{0.280868,0.160771,0.472899}%
\pgfsetstrokecolor{currentstroke}%
\pgfsetdash{}{0pt}%
\pgfpathmoveto{\pgfqpoint{5.293862in}{4.755023in}}%
\pgfpathlineto{\pgfqpoint{5.244836in}{4.764224in}}%
\pgfusepath{stroke}%
\end{pgfscope}%
\begin{pgfscope}%
\pgfpathrectangle{\pgfqpoint{3.985294in}{4.155455in}}{\pgfqpoint{2.279412in}{2.004545in}}%
\pgfusepath{clip}%
\pgfsetbuttcap%
\pgfsetroundjoin%
\pgfsetlinewidth{0.473964pt}%
\definecolor{currentstroke}{rgb}{0.282623,0.140926,0.457517}%
\pgfsetstrokecolor{currentstroke}%
\pgfsetdash{}{0pt}%
\pgfpathmoveto{\pgfqpoint{5.244836in}{4.764224in}}%
\pgfpathlineto{\pgfqpoint{5.196550in}{4.775949in}}%
\pgfusepath{stroke}%
\end{pgfscope}%
\begin{pgfscope}%
\pgfpathrectangle{\pgfqpoint{3.985294in}{4.155455in}}{\pgfqpoint{2.279412in}{2.004545in}}%
\pgfusepath{clip}%
\pgfsetbuttcap%
\pgfsetroundjoin%
\pgfsetlinewidth{0.472725pt}%
\definecolor{currentstroke}{rgb}{0.282623,0.140926,0.457517}%
\pgfsetstrokecolor{currentstroke}%
\pgfsetdash{}{0pt}%
\pgfpathmoveto{\pgfqpoint{5.196550in}{4.775949in}}%
\pgfpathlineto{\pgfqpoint{5.149179in}{4.790300in}}%
\pgfusepath{stroke}%
\end{pgfscope}%
\begin{pgfscope}%
\pgfpathrectangle{\pgfqpoint{3.985294in}{4.155455in}}{\pgfqpoint{2.279412in}{2.004545in}}%
\pgfusepath{clip}%
\pgfsetbuttcap%
\pgfsetroundjoin%
\pgfsetlinewidth{0.515725pt}%
\definecolor{currentstroke}{rgb}{0.279574,0.170599,0.479997}%
\pgfsetstrokecolor{currentstroke}%
\pgfsetdash{}{0pt}%
\pgfpathmoveto{\pgfqpoint{5.149179in}{4.790300in}}%
\pgfpathlineto{\pgfqpoint{5.102859in}{4.807071in}}%
\pgfusepath{stroke}%
\end{pgfscope}%
\begin{pgfscope}%
\pgfpathrectangle{\pgfqpoint{3.985294in}{4.155455in}}{\pgfqpoint{2.279412in}{2.004545in}}%
\pgfusepath{clip}%
\pgfsetbuttcap%
\pgfsetroundjoin%
\pgfsetlinewidth{0.577690pt}%
\definecolor{currentstroke}{rgb}{0.270595,0.214069,0.507052}%
\pgfsetstrokecolor{currentstroke}%
\pgfsetdash{}{0pt}%
\pgfpathmoveto{\pgfqpoint{5.102859in}{4.807071in}}%
\pgfpathlineto{\pgfqpoint{5.058867in}{4.827942in}}%
\pgfusepath{stroke}%
\end{pgfscope}%
\begin{pgfscope}%
\pgfpathrectangle{\pgfqpoint{3.985294in}{4.155455in}}{\pgfqpoint{2.279412in}{2.004545in}}%
\pgfusepath{clip}%
\pgfsetbuttcap%
\pgfsetroundjoin%
\pgfsetlinewidth{0.510573pt}%
\definecolor{currentstroke}{rgb}{0.280255,0.165693,0.476498}%
\pgfsetstrokecolor{currentstroke}%
\pgfsetdash{}{0pt}%
\pgfpathmoveto{\pgfqpoint{5.058867in}{4.827942in}}%
\pgfpathlineto{\pgfqpoint{5.017821in}{4.853094in}}%
\pgfusepath{stroke}%
\end{pgfscope}%
\begin{pgfscope}%
\pgfpathrectangle{\pgfqpoint{3.985294in}{4.155455in}}{\pgfqpoint{2.279412in}{2.004545in}}%
\pgfusepath{clip}%
\pgfsetbuttcap%
\pgfsetroundjoin%
\pgfsetlinewidth{0.608166pt}%
\definecolor{currentstroke}{rgb}{0.265145,0.232956,0.516599}%
\pgfsetstrokecolor{currentstroke}%
\pgfsetdash{}{0pt}%
\pgfpathmoveto{\pgfqpoint{5.017821in}{4.853094in}}%
\pgfpathlineto{\pgfqpoint{4.980290in}{4.882046in}}%
\pgfusepath{stroke}%
\end{pgfscope}%
\begin{pgfscope}%
\pgfpathrectangle{\pgfqpoint{3.985294in}{4.155455in}}{\pgfqpoint{2.279412in}{2.004545in}}%
\pgfusepath{clip}%
\pgfsetbuttcap%
\pgfsetroundjoin%
\pgfsetlinewidth{0.699106pt}%
\definecolor{currentstroke}{rgb}{0.241237,0.296485,0.539709}%
\pgfsetstrokecolor{currentstroke}%
\pgfsetdash{}{0pt}%
\pgfpathmoveto{\pgfqpoint{4.980290in}{4.882046in}}%
\pgfpathlineto{\pgfqpoint{4.945610in}{4.913673in}}%
\pgfusepath{stroke}%
\end{pgfscope}%
\begin{pgfscope}%
\pgfpathrectangle{\pgfqpoint{3.985294in}{4.155455in}}{\pgfqpoint{2.279412in}{2.004545in}}%
\pgfusepath{clip}%
\pgfsetbuttcap%
\pgfsetroundjoin%
\pgfsetlinewidth{0.766848pt}%
\definecolor{currentstroke}{rgb}{0.223925,0.334994,0.548053}%
\pgfsetstrokecolor{currentstroke}%
\pgfsetdash{}{0pt}%
\pgfpathmoveto{\pgfqpoint{4.945610in}{4.913673in}}%
\pgfpathlineto{\pgfqpoint{4.911852in}{4.945548in}}%
\pgfusepath{stroke}%
\end{pgfscope}%
\begin{pgfscope}%
\pgfpathrectangle{\pgfqpoint{3.985294in}{4.155455in}}{\pgfqpoint{2.279412in}{2.004545in}}%
\pgfusepath{clip}%
\pgfsetbuttcap%
\pgfsetroundjoin%
\pgfsetlinewidth{0.940412pt}%
\definecolor{currentstroke}{rgb}{0.177423,0.437527,0.557565}%
\pgfsetstrokecolor{currentstroke}%
\pgfsetdash{}{0pt}%
\pgfpathmoveto{\pgfqpoint{4.911852in}{4.945548in}}%
\pgfpathlineto{\pgfqpoint{4.911852in}{4.945548in}}%
\pgfusepath{stroke}%
\end{pgfscope}%
\begin{pgfscope}%
\pgfpathrectangle{\pgfqpoint{3.985294in}{4.155455in}}{\pgfqpoint{2.279412in}{2.004545in}}%
\pgfusepath{clip}%
\pgfsetbuttcap%
\pgfsetroundjoin%
\pgfsetlinewidth{0.940412pt}%
\definecolor{currentstroke}{rgb}{0.177423,0.437527,0.557565}%
\pgfsetstrokecolor{currentstroke}%
\pgfsetdash{}{0pt}%
\pgfpathmoveto{\pgfqpoint{4.911852in}{4.945548in}}%
\pgfpathlineto{\pgfqpoint{4.911852in}{4.945548in}}%
\pgfusepath{stroke}%
\end{pgfscope}%
\begin{pgfscope}%
\pgfpathrectangle{\pgfqpoint{3.985294in}{4.155455in}}{\pgfqpoint{2.279412in}{2.004545in}}%
\pgfusepath{clip}%
\pgfsetbuttcap%
\pgfsetroundjoin%
\pgfsetlinewidth{0.940412pt}%
\definecolor{currentstroke}{rgb}{0.177423,0.437527,0.557565}%
\pgfsetstrokecolor{currentstroke}%
\pgfsetdash{}{0pt}%
\pgfpathmoveto{\pgfqpoint{4.911852in}{4.945548in}}%
\pgfpathlineto{\pgfqpoint{4.911852in}{4.945548in}}%
\pgfusepath{stroke}%
\end{pgfscope}%
\begin{pgfscope}%
\pgfpathrectangle{\pgfqpoint{3.985294in}{4.155455in}}{\pgfqpoint{2.279412in}{2.004545in}}%
\pgfusepath{clip}%
\pgfsetbuttcap%
\pgfsetroundjoin%
\pgfsetlinewidth{0.310931pt}%
\definecolor{currentstroke}{rgb}{0.268510,0.009605,0.335427}%
\pgfsetstrokecolor{currentstroke}%
\pgfsetdash{}{0pt}%
\pgfpathmoveto{\pgfqpoint{5.876105in}{5.706395in}}%
\pgfpathlineto{\pgfqpoint{5.833987in}{5.703725in}}%
\pgfusepath{stroke}%
\end{pgfscope}%
\begin{pgfscope}%
\pgfpathrectangle{\pgfqpoint{3.985294in}{4.155455in}}{\pgfqpoint{2.279412in}{2.004545in}}%
\pgfusepath{clip}%
\pgfsetbuttcap%
\pgfsetroundjoin%
\pgfsetlinewidth{0.317385pt}%
\definecolor{currentstroke}{rgb}{0.269944,0.014625,0.341379}%
\pgfsetstrokecolor{currentstroke}%
\pgfsetdash{}{0pt}%
\pgfpathmoveto{\pgfqpoint{5.833987in}{5.703725in}}%
\pgfpathlineto{\pgfqpoint{5.791795in}{5.699009in}}%
\pgfusepath{stroke}%
\end{pgfscope}%
\begin{pgfscope}%
\pgfpathrectangle{\pgfqpoint{3.985294in}{4.155455in}}{\pgfqpoint{2.279412in}{2.004545in}}%
\pgfusepath{clip}%
\pgfsetbuttcap%
\pgfsetroundjoin%
\pgfsetlinewidth{0.317927pt}%
\definecolor{currentstroke}{rgb}{0.269944,0.014625,0.341379}%
\pgfsetstrokecolor{currentstroke}%
\pgfsetdash{}{0pt}%
\pgfpathmoveto{\pgfqpoint{5.791795in}{5.699009in}}%
\pgfpathlineto{\pgfqpoint{5.743137in}{5.696620in}}%
\pgfusepath{stroke}%
\end{pgfscope}%
\begin{pgfscope}%
\pgfpathrectangle{\pgfqpoint{3.985294in}{4.155455in}}{\pgfqpoint{2.279412in}{2.004545in}}%
\pgfusepath{clip}%
\pgfsetbuttcap%
\pgfsetroundjoin%
\pgfsetlinewidth{0.324859pt}%
\definecolor{currentstroke}{rgb}{0.271305,0.019942,0.347269}%
\pgfsetstrokecolor{currentstroke}%
\pgfsetdash{}{0pt}%
\pgfpathmoveto{\pgfqpoint{5.743137in}{5.696620in}}%
\pgfpathlineto{\pgfqpoint{5.693890in}{5.690483in}}%
\pgfusepath{stroke}%
\end{pgfscope}%
\begin{pgfscope}%
\pgfpathrectangle{\pgfqpoint{3.985294in}{4.155455in}}{\pgfqpoint{2.279412in}{2.004545in}}%
\pgfusepath{clip}%
\pgfsetbuttcap%
\pgfsetroundjoin%
\pgfsetlinewidth{0.321607pt}%
\definecolor{currentstroke}{rgb}{0.269944,0.014625,0.341379}%
\pgfsetstrokecolor{currentstroke}%
\pgfsetdash{}{0pt}%
\pgfpathmoveto{\pgfqpoint{5.693890in}{5.690483in}}%
\pgfpathlineto{\pgfqpoint{5.644501in}{5.684682in}}%
\pgfusepath{stroke}%
\end{pgfscope}%
\begin{pgfscope}%
\pgfpathrectangle{\pgfqpoint{3.985294in}{4.155455in}}{\pgfqpoint{2.279412in}{2.004545in}}%
\pgfusepath{clip}%
\pgfsetbuttcap%
\pgfsetroundjoin%
\pgfsetlinewidth{0.333702pt}%
\definecolor{currentstroke}{rgb}{0.272594,0.025563,0.353093}%
\pgfsetstrokecolor{currentstroke}%
\pgfsetdash{}{0pt}%
\pgfpathmoveto{\pgfqpoint{5.644501in}{5.684682in}}%
\pgfpathlineto{\pgfqpoint{5.594501in}{5.681329in}}%
\pgfusepath{stroke}%
\end{pgfscope}%
\begin{pgfscope}%
\pgfpathrectangle{\pgfqpoint{3.985294in}{4.155455in}}{\pgfqpoint{2.279412in}{2.004545in}}%
\pgfusepath{clip}%
\pgfsetbuttcap%
\pgfsetroundjoin%
\pgfsetlinewidth{0.335115pt}%
\definecolor{currentstroke}{rgb}{0.272594,0.025563,0.353093}%
\pgfsetstrokecolor{currentstroke}%
\pgfsetdash{}{0pt}%
\pgfpathmoveto{\pgfqpoint{5.594501in}{5.681329in}}%
\pgfpathlineto{\pgfqpoint{5.544537in}{5.677704in}}%
\pgfusepath{stroke}%
\end{pgfscope}%
\begin{pgfscope}%
\pgfpathrectangle{\pgfqpoint{3.985294in}{4.155455in}}{\pgfqpoint{2.279412in}{2.004545in}}%
\pgfusepath{clip}%
\pgfsetbuttcap%
\pgfsetroundjoin%
\pgfsetlinewidth{0.348179pt}%
\definecolor{currentstroke}{rgb}{0.274952,0.037752,0.364543}%
\pgfsetstrokecolor{currentstroke}%
\pgfsetdash{}{0pt}%
\pgfpathmoveto{\pgfqpoint{5.544537in}{5.677704in}}%
\pgfpathlineto{\pgfqpoint{5.494595in}{5.673832in}}%
\pgfusepath{stroke}%
\end{pgfscope}%
\begin{pgfscope}%
\pgfpathrectangle{\pgfqpoint{3.985294in}{4.155455in}}{\pgfqpoint{2.279412in}{2.004545in}}%
\pgfusepath{clip}%
\pgfsetbuttcap%
\pgfsetroundjoin%
\pgfsetlinewidth{0.375515pt}%
\definecolor{currentstroke}{rgb}{0.278791,0.062145,0.386592}%
\pgfsetstrokecolor{currentstroke}%
\pgfsetdash{}{0pt}%
\pgfpathmoveto{\pgfqpoint{5.494595in}{5.673832in}}%
\pgfpathlineto{\pgfqpoint{5.444771in}{5.668909in}}%
\pgfusepath{stroke}%
\end{pgfscope}%
\begin{pgfscope}%
\pgfpathrectangle{\pgfqpoint{3.985294in}{4.155455in}}{\pgfqpoint{2.279412in}{2.004545in}}%
\pgfusepath{clip}%
\pgfsetbuttcap%
\pgfsetroundjoin%
\pgfsetlinewidth{0.380611pt}%
\definecolor{currentstroke}{rgb}{0.279566,0.067836,0.391917}%
\pgfsetstrokecolor{currentstroke}%
\pgfsetdash{}{0pt}%
\pgfpathmoveto{\pgfqpoint{5.444771in}{5.668909in}}%
\pgfpathlineto{\pgfqpoint{5.395189in}{5.662336in}}%
\pgfusepath{stroke}%
\end{pgfscope}%
\begin{pgfscope}%
\pgfpathrectangle{\pgfqpoint{3.985294in}{4.155455in}}{\pgfqpoint{2.279412in}{2.004545in}}%
\pgfusepath{clip}%
\pgfsetbuttcap%
\pgfsetroundjoin%
\pgfsetlinewidth{0.394710pt}%
\definecolor{currentstroke}{rgb}{0.280894,0.078907,0.402329}%
\pgfsetstrokecolor{currentstroke}%
\pgfsetdash{}{0pt}%
\pgfpathmoveto{\pgfqpoint{5.395189in}{5.662336in}}%
\pgfpathlineto{\pgfqpoint{5.345972in}{5.653971in}}%
\pgfusepath{stroke}%
\end{pgfscope}%
\begin{pgfscope}%
\pgfpathrectangle{\pgfqpoint{3.985294in}{4.155455in}}{\pgfqpoint{2.279412in}{2.004545in}}%
\pgfusepath{clip}%
\pgfsetbuttcap%
\pgfsetroundjoin%
\pgfsetlinewidth{0.400815pt}%
\definecolor{currentstroke}{rgb}{0.281446,0.084320,0.407414}%
\pgfsetstrokecolor{currentstroke}%
\pgfsetdash{}{0pt}%
\pgfpathmoveto{\pgfqpoint{5.345972in}{5.653971in}}%
\pgfpathlineto{\pgfqpoint{5.296982in}{5.644604in}}%
\pgfusepath{stroke}%
\end{pgfscope}%
\begin{pgfscope}%
\pgfpathrectangle{\pgfqpoint{3.985294in}{4.155455in}}{\pgfqpoint{2.279412in}{2.004545in}}%
\pgfusepath{clip}%
\pgfsetbuttcap%
\pgfsetroundjoin%
\pgfsetlinewidth{0.417331pt}%
\definecolor{currentstroke}{rgb}{0.282327,0.094955,0.417331}%
\pgfsetstrokecolor{currentstroke}%
\pgfsetdash{}{0pt}%
\pgfpathmoveto{\pgfqpoint{5.296982in}{5.644604in}}%
\pgfpathlineto{\pgfqpoint{5.248191in}{5.634491in}}%
\pgfusepath{stroke}%
\end{pgfscope}%
\begin{pgfscope}%
\pgfpathrectangle{\pgfqpoint{3.985294in}{4.155455in}}{\pgfqpoint{2.279412in}{2.004545in}}%
\pgfusepath{clip}%
\pgfsetbuttcap%
\pgfsetroundjoin%
\pgfsetlinewidth{0.437539pt}%
\definecolor{currentstroke}{rgb}{0.283091,0.110553,0.431554}%
\pgfsetstrokecolor{currentstroke}%
\pgfsetdash{}{0pt}%
\pgfpathmoveto{\pgfqpoint{5.248191in}{5.634491in}}%
\pgfpathlineto{\pgfqpoint{5.200048in}{5.622260in}}%
\pgfusepath{stroke}%
\end{pgfscope}%
\begin{pgfscope}%
\pgfpathrectangle{\pgfqpoint{3.985294in}{4.155455in}}{\pgfqpoint{2.279412in}{2.004545in}}%
\pgfusepath{clip}%
\pgfsetbuttcap%
\pgfsetroundjoin%
\pgfsetlinewidth{0.419820pt}%
\definecolor{currentstroke}{rgb}{0.282656,0.100196,0.422160}%
\pgfsetstrokecolor{currentstroke}%
\pgfsetdash{}{0pt}%
\pgfpathmoveto{\pgfqpoint{5.200048in}{5.622260in}}%
\pgfpathlineto{\pgfqpoint{5.154122in}{5.605350in}}%
\pgfusepath{stroke}%
\end{pgfscope}%
\begin{pgfscope}%
\pgfpathrectangle{\pgfqpoint{3.985294in}{4.155455in}}{\pgfqpoint{2.279412in}{2.004545in}}%
\pgfusepath{clip}%
\pgfsetbuttcap%
\pgfsetroundjoin%
\pgfsetlinewidth{0.427521pt}%
\definecolor{currentstroke}{rgb}{0.282910,0.105393,0.426902}%
\pgfsetstrokecolor{currentstroke}%
\pgfsetdash{}{0pt}%
\pgfpathmoveto{\pgfqpoint{5.154122in}{5.605350in}}%
\pgfpathlineto{\pgfqpoint{5.109312in}{5.585966in}}%
\pgfusepath{stroke}%
\end{pgfscope}%
\begin{pgfscope}%
\pgfpathrectangle{\pgfqpoint{3.985294in}{4.155455in}}{\pgfqpoint{2.279412in}{2.004545in}}%
\pgfusepath{clip}%
\pgfsetbuttcap%
\pgfsetroundjoin%
\pgfsetlinewidth{0.493509pt}%
\definecolor{currentstroke}{rgb}{0.281412,0.155834,0.469201}%
\pgfsetstrokecolor{currentstroke}%
\pgfsetdash{}{0pt}%
\pgfpathmoveto{\pgfqpoint{5.109312in}{5.585966in}}%
\pgfpathlineto{\pgfqpoint{5.066402in}{5.563609in}}%
\pgfusepath{stroke}%
\end{pgfscope}%
\begin{pgfscope}%
\pgfpathrectangle{\pgfqpoint{3.985294in}{4.155455in}}{\pgfqpoint{2.279412in}{2.004545in}}%
\pgfusepath{clip}%
\pgfsetbuttcap%
\pgfsetroundjoin%
\pgfsetlinewidth{0.464177pt}%
\definecolor{currentstroke}{rgb}{0.283072,0.130895,0.449241}%
\pgfsetstrokecolor{currentstroke}%
\pgfsetdash{}{0pt}%
\pgfpathmoveto{\pgfqpoint{5.066402in}{5.563609in}}%
\pgfpathlineto{\pgfqpoint{5.027627in}{5.536171in}}%
\pgfusepath{stroke}%
\end{pgfscope}%
\begin{pgfscope}%
\pgfpathrectangle{\pgfqpoint{3.985294in}{4.155455in}}{\pgfqpoint{2.279412in}{2.004545in}}%
\pgfusepath{clip}%
\pgfsetbuttcap%
\pgfsetroundjoin%
\pgfsetlinewidth{0.532574pt}%
\definecolor{currentstroke}{rgb}{0.278012,0.180367,0.486697}%
\pgfsetstrokecolor{currentstroke}%
\pgfsetdash{}{0pt}%
\pgfpathmoveto{\pgfqpoint{5.027627in}{5.536171in}}%
\pgfpathlineto{\pgfqpoint{5.027627in}{5.536171in}}%
\pgfusepath{stroke}%
\end{pgfscope}%
\begin{pgfscope}%
\pgfpathrectangle{\pgfqpoint{3.985294in}{4.155455in}}{\pgfqpoint{2.279412in}{2.004545in}}%
\pgfusepath{clip}%
\pgfsetbuttcap%
\pgfsetroundjoin%
\pgfsetlinewidth{0.532574pt}%
\definecolor{currentstroke}{rgb}{0.278012,0.180367,0.486697}%
\pgfsetstrokecolor{currentstroke}%
\pgfsetdash{}{0pt}%
\pgfpathmoveto{\pgfqpoint{5.027627in}{5.536171in}}%
\pgfpathlineto{\pgfqpoint{5.003372in}{5.510144in}}%
\pgfusepath{stroke}%
\end{pgfscope}%
\begin{pgfscope}%
\pgfpathrectangle{\pgfqpoint{3.985294in}{4.155455in}}{\pgfqpoint{2.279412in}{2.004545in}}%
\pgfusepath{clip}%
\pgfsetbuttcap%
\pgfsetroundjoin%
\pgfsetlinewidth{0.572507pt}%
\definecolor{currentstroke}{rgb}{0.271828,0.209303,0.504434}%
\pgfsetstrokecolor{currentstroke}%
\pgfsetdash{}{0pt}%
\pgfpathmoveto{\pgfqpoint{5.003372in}{5.510144in}}%
\pgfpathlineto{\pgfqpoint{4.979499in}{5.481632in}}%
\pgfusepath{stroke}%
\end{pgfscope}%
\begin{pgfscope}%
\pgfpathrectangle{\pgfqpoint{3.985294in}{4.155455in}}{\pgfqpoint{2.279412in}{2.004545in}}%
\pgfusepath{clip}%
\pgfsetbuttcap%
\pgfsetroundjoin%
\pgfsetlinewidth{0.547992pt}%
\definecolor{currentstroke}{rgb}{0.276194,0.190074,0.493001}%
\pgfsetstrokecolor{currentstroke}%
\pgfsetdash{}{0pt}%
\pgfpathmoveto{\pgfqpoint{4.979499in}{5.481632in}}%
\pgfpathlineto{\pgfqpoint{4.953959in}{5.445063in}}%
\pgfusepath{stroke}%
\end{pgfscope}%
\begin{pgfscope}%
\pgfpathrectangle{\pgfqpoint{3.985294in}{4.155455in}}{\pgfqpoint{2.279412in}{2.004545in}}%
\pgfusepath{clip}%
\pgfsetbuttcap%
\pgfsetroundjoin%
\pgfsetlinewidth{0.638273pt}%
\definecolor{currentstroke}{rgb}{0.257322,0.256130,0.526563}%
\pgfsetstrokecolor{currentstroke}%
\pgfsetdash{}{0pt}%
\pgfpathmoveto{\pgfqpoint{4.953959in}{5.445063in}}%
\pgfpathlineto{\pgfqpoint{4.930649in}{5.406866in}}%
\pgfusepath{stroke}%
\end{pgfscope}%
\begin{pgfscope}%
\pgfpathrectangle{\pgfqpoint{3.985294in}{4.155455in}}{\pgfqpoint{2.279412in}{2.004545in}}%
\pgfusepath{clip}%
\pgfsetbuttcap%
\pgfsetroundjoin%
\pgfsetlinewidth{0.706192pt}%
\definecolor{currentstroke}{rgb}{0.239346,0.300855,0.540844}%
\pgfsetstrokecolor{currentstroke}%
\pgfsetdash{}{0pt}%
\pgfpathmoveto{\pgfqpoint{4.930649in}{5.406866in}}%
\pgfpathlineto{\pgfqpoint{4.905638in}{5.369056in}}%
\pgfusepath{stroke}%
\end{pgfscope}%
\begin{pgfscope}%
\pgfpathrectangle{\pgfqpoint{3.985294in}{4.155455in}}{\pgfqpoint{2.279412in}{2.004545in}}%
\pgfusepath{clip}%
\pgfsetbuttcap%
\pgfsetroundjoin%
\pgfsetlinewidth{0.808125pt}%
\definecolor{currentstroke}{rgb}{0.212395,0.359683,0.551710}%
\pgfsetstrokecolor{currentstroke}%
\pgfsetdash{}{0pt}%
\pgfpathmoveto{\pgfqpoint{4.905638in}{5.369056in}}%
\pgfpathlineto{\pgfqpoint{4.878884in}{5.331946in}}%
\pgfusepath{stroke}%
\end{pgfscope}%
\begin{pgfscope}%
\pgfpathrectangle{\pgfqpoint{3.985294in}{4.155455in}}{\pgfqpoint{2.279412in}{2.004545in}}%
\pgfusepath{clip}%
\pgfsetbuttcap%
\pgfsetroundjoin%
\pgfsetlinewidth{0.350920pt}%
\definecolor{currentstroke}{rgb}{0.276022,0.044167,0.370164}%
\pgfsetstrokecolor{currentstroke}%
\pgfsetdash{}{0pt}%
\pgfpathmoveto{\pgfqpoint{5.484043in}{5.744115in}}%
\pgfpathlineto{\pgfqpoint{5.434820in}{5.738140in}}%
\pgfusepath{stroke}%
\end{pgfscope}%
\begin{pgfscope}%
\pgfpathrectangle{\pgfqpoint{3.985294in}{4.155455in}}{\pgfqpoint{2.279412in}{2.004545in}}%
\pgfusepath{clip}%
\pgfsetbuttcap%
\pgfsetroundjoin%
\pgfsetlinewidth{0.344779pt}%
\definecolor{currentstroke}{rgb}{0.274952,0.037752,0.364543}%
\pgfsetstrokecolor{currentstroke}%
\pgfsetdash{}{0pt}%
\pgfpathmoveto{\pgfqpoint{5.434820in}{5.738140in}}%
\pgfpathlineto{\pgfqpoint{5.385706in}{5.732053in}}%
\pgfusepath{stroke}%
\end{pgfscope}%
\begin{pgfscope}%
\pgfpathrectangle{\pgfqpoint{3.985294in}{4.155455in}}{\pgfqpoint{2.279412in}{2.004545in}}%
\pgfusepath{clip}%
\pgfsetbuttcap%
\pgfsetroundjoin%
\pgfsetlinewidth{0.357717pt}%
\definecolor{currentstroke}{rgb}{0.277018,0.050344,0.375715}%
\pgfsetstrokecolor{currentstroke}%
\pgfsetdash{}{0pt}%
\pgfpathmoveto{\pgfqpoint{5.385706in}{5.732053in}}%
\pgfpathlineto{\pgfqpoint{5.336571in}{5.723646in}}%
\pgfusepath{stroke}%
\end{pgfscope}%
\begin{pgfscope}%
\pgfpathrectangle{\pgfqpoint{3.985294in}{4.155455in}}{\pgfqpoint{2.279412in}{2.004545in}}%
\pgfusepath{clip}%
\pgfsetbuttcap%
\pgfsetroundjoin%
\pgfsetlinewidth{0.377136pt}%
\definecolor{currentstroke}{rgb}{0.279566,0.067836,0.391917}%
\pgfsetstrokecolor{currentstroke}%
\pgfsetdash{}{0pt}%
\pgfpathmoveto{\pgfqpoint{5.336571in}{5.723646in}}%
\pgfpathlineto{\pgfqpoint{5.288162in}{5.712319in}}%
\pgfusepath{stroke}%
\end{pgfscope}%
\begin{pgfscope}%
\pgfpathrectangle{\pgfqpoint{3.985294in}{4.155455in}}{\pgfqpoint{2.279412in}{2.004545in}}%
\pgfusepath{clip}%
\pgfsetbuttcap%
\pgfsetroundjoin%
\pgfsetlinewidth{0.363276pt}%
\definecolor{currentstroke}{rgb}{0.277941,0.056324,0.381191}%
\pgfsetstrokecolor{currentstroke}%
\pgfsetdash{}{0pt}%
\pgfpathmoveto{\pgfqpoint{5.288162in}{5.712319in}}%
\pgfpathlineto{\pgfqpoint{5.240023in}{5.700168in}}%
\pgfusepath{stroke}%
\end{pgfscope}%
\begin{pgfscope}%
\pgfpathrectangle{\pgfqpoint{3.985294in}{4.155455in}}{\pgfqpoint{2.279412in}{2.004545in}}%
\pgfusepath{clip}%
\pgfsetbuttcap%
\pgfsetroundjoin%
\pgfsetlinewidth{0.400106pt}%
\definecolor{currentstroke}{rgb}{0.281446,0.084320,0.407414}%
\pgfsetstrokecolor{currentstroke}%
\pgfsetdash{}{0pt}%
\pgfpathmoveto{\pgfqpoint{5.240023in}{5.700168in}}%
\pgfpathlineto{\pgfqpoint{5.192316in}{5.686806in}}%
\pgfusepath{stroke}%
\end{pgfscope}%
\begin{pgfscope}%
\pgfpathrectangle{\pgfqpoint{3.985294in}{4.155455in}}{\pgfqpoint{2.279412in}{2.004545in}}%
\pgfusepath{clip}%
\pgfsetbuttcap%
\pgfsetroundjoin%
\pgfsetlinewidth{0.368507pt}%
\definecolor{currentstroke}{rgb}{0.277941,0.056324,0.381191}%
\pgfsetstrokecolor{currentstroke}%
\pgfsetdash{}{0pt}%
\pgfpathmoveto{\pgfqpoint{5.275078in}{4.602878in}}%
\pgfpathlineto{\pgfqpoint{5.227584in}{4.616446in}}%
\pgfusepath{stroke}%
\end{pgfscope}%
\begin{pgfscope}%
\pgfpathrectangle{\pgfqpoint{3.985294in}{4.155455in}}{\pgfqpoint{2.279412in}{2.004545in}}%
\pgfusepath{clip}%
\pgfsetbuttcap%
\pgfsetroundjoin%
\pgfsetlinewidth{0.382702pt}%
\definecolor{currentstroke}{rgb}{0.279566,0.067836,0.391917}%
\pgfsetstrokecolor{currentstroke}%
\pgfsetdash{}{0pt}%
\pgfpathmoveto{\pgfqpoint{5.227584in}{4.616446in}}%
\pgfpathlineto{\pgfqpoint{5.181825in}{4.634231in}}%
\pgfusepath{stroke}%
\end{pgfscope}%
\begin{pgfscope}%
\pgfpathrectangle{\pgfqpoint{3.985294in}{4.155455in}}{\pgfqpoint{2.279412in}{2.004545in}}%
\pgfusepath{clip}%
\pgfsetbuttcap%
\pgfsetroundjoin%
\pgfsetlinewidth{0.395621pt}%
\definecolor{currentstroke}{rgb}{0.280894,0.078907,0.402329}%
\pgfsetstrokecolor{currentstroke}%
\pgfsetdash{}{0pt}%
\pgfpathmoveto{\pgfqpoint{5.181825in}{4.634231in}}%
\pgfpathlineto{\pgfqpoint{5.137310in}{4.654331in}}%
\pgfusepath{stroke}%
\end{pgfscope}%
\begin{pgfscope}%
\pgfpathrectangle{\pgfqpoint{3.985294in}{4.155455in}}{\pgfqpoint{2.279412in}{2.004545in}}%
\pgfusepath{clip}%
\pgfsetbuttcap%
\pgfsetroundjoin%
\pgfsetlinewidth{0.426845pt}%
\definecolor{currentstroke}{rgb}{0.282910,0.105393,0.426902}%
\pgfsetstrokecolor{currentstroke}%
\pgfsetdash{}{0pt}%
\pgfpathmoveto{\pgfqpoint{5.137310in}{4.654331in}}%
\pgfpathlineto{\pgfqpoint{5.095748in}{4.678639in}}%
\pgfusepath{stroke}%
\end{pgfscope}%
\begin{pgfscope}%
\pgfpathrectangle{\pgfqpoint{3.985294in}{4.155455in}}{\pgfqpoint{2.279412in}{2.004545in}}%
\pgfusepath{clip}%
\pgfsetbuttcap%
\pgfsetroundjoin%
\pgfsetlinewidth{0.404975pt}%
\definecolor{currentstroke}{rgb}{0.281924,0.089666,0.412415}%
\pgfsetstrokecolor{currentstroke}%
\pgfsetdash{}{0pt}%
\pgfpathmoveto{\pgfqpoint{5.095748in}{4.678639in}}%
\pgfpathlineto{\pgfqpoint{5.056782in}{4.706345in}}%
\pgfusepath{stroke}%
\end{pgfscope}%
\begin{pgfscope}%
\pgfpathrectangle{\pgfqpoint{3.985294in}{4.155455in}}{\pgfqpoint{2.279412in}{2.004545in}}%
\pgfusepath{clip}%
\pgfsetbuttcap%
\pgfsetroundjoin%
\pgfsetlinewidth{0.360400pt}%
\definecolor{currentstroke}{rgb}{0.277018,0.050344,0.375715}%
\pgfsetstrokecolor{currentstroke}%
\pgfsetdash{}{0pt}%
\pgfpathmoveto{\pgfqpoint{5.577718in}{4.582166in}}%
\pgfpathlineto{\pgfqpoint{5.527952in}{4.587528in}}%
\pgfusepath{stroke}%
\end{pgfscope}%
\begin{pgfscope}%
\pgfpathrectangle{\pgfqpoint{3.985294in}{4.155455in}}{\pgfqpoint{2.279412in}{2.004545in}}%
\pgfusepath{clip}%
\pgfsetbuttcap%
\pgfsetroundjoin%
\pgfsetlinewidth{0.347088pt}%
\definecolor{currentstroke}{rgb}{0.274952,0.037752,0.364543}%
\pgfsetstrokecolor{currentstroke}%
\pgfsetdash{}{0pt}%
\pgfpathmoveto{\pgfqpoint{5.527952in}{4.587528in}}%
\pgfpathlineto{\pgfqpoint{5.478280in}{4.593489in}}%
\pgfusepath{stroke}%
\end{pgfscope}%
\begin{pgfscope}%
\pgfpathrectangle{\pgfqpoint{3.985294in}{4.155455in}}{\pgfqpoint{2.279412in}{2.004545in}}%
\pgfusepath{clip}%
\pgfsetbuttcap%
\pgfsetroundjoin%
\pgfsetlinewidth{0.355842pt}%
\definecolor{currentstroke}{rgb}{0.276022,0.044167,0.370164}%
\pgfsetstrokecolor{currentstroke}%
\pgfsetdash{}{0pt}%
\pgfpathmoveto{\pgfqpoint{5.478280in}{4.593489in}}%
\pgfpathlineto{\pgfqpoint{5.428738in}{4.600267in}}%
\pgfusepath{stroke}%
\end{pgfscope}%
\begin{pgfscope}%
\pgfpathrectangle{\pgfqpoint{3.985294in}{4.155455in}}{\pgfqpoint{2.279412in}{2.004545in}}%
\pgfusepath{clip}%
\pgfsetbuttcap%
\pgfsetroundjoin%
\pgfsetlinewidth{0.373865pt}%
\definecolor{currentstroke}{rgb}{0.278791,0.062145,0.386592}%
\pgfsetstrokecolor{currentstroke}%
\pgfsetdash{}{0pt}%
\pgfpathmoveto{\pgfqpoint{5.428738in}{4.600267in}}%
\pgfpathlineto{\pgfqpoint{5.379431in}{4.608252in}}%
\pgfusepath{stroke}%
\end{pgfscope}%
\begin{pgfscope}%
\pgfpathrectangle{\pgfqpoint{3.985294in}{4.155455in}}{\pgfqpoint{2.279412in}{2.004545in}}%
\pgfusepath{clip}%
\pgfsetbuttcap%
\pgfsetroundjoin%
\pgfsetlinewidth{0.378832pt}%
\definecolor{currentstroke}{rgb}{0.279566,0.067836,0.391917}%
\pgfsetstrokecolor{currentstroke}%
\pgfsetdash{}{0pt}%
\pgfpathmoveto{\pgfqpoint{5.379431in}{4.608252in}}%
\pgfpathlineto{\pgfqpoint{5.330168in}{4.616446in}}%
\pgfusepath{stroke}%
\end{pgfscope}%
\begin{pgfscope}%
\pgfpathrectangle{\pgfqpoint{3.985294in}{4.155455in}}{\pgfqpoint{2.279412in}{2.004545in}}%
\pgfusepath{clip}%
\pgfsetbuttcap%
\pgfsetroundjoin%
\pgfsetlinewidth{0.340923pt}%
\definecolor{currentstroke}{rgb}{0.273809,0.031497,0.358853}%
\pgfsetstrokecolor{currentstroke}%
\pgfsetdash{}{0pt}%
\pgfpathmoveto{\pgfqpoint{5.740503in}{4.751766in}}%
\pgfpathlineto{\pgfqpoint{5.690481in}{4.752424in}}%
\pgfusepath{stroke}%
\end{pgfscope}%
\begin{pgfscope}%
\pgfpathrectangle{\pgfqpoint{3.985294in}{4.155455in}}{\pgfqpoint{2.279412in}{2.004545in}}%
\pgfusepath{clip}%
\pgfsetbuttcap%
\pgfsetroundjoin%
\pgfsetlinewidth{0.340914pt}%
\definecolor{currentstroke}{rgb}{0.273809,0.031497,0.358853}%
\pgfsetstrokecolor{currentstroke}%
\pgfsetdash{}{0pt}%
\pgfpathmoveto{\pgfqpoint{5.690481in}{4.752424in}}%
\pgfpathlineto{\pgfqpoint{5.640508in}{4.753907in}}%
\pgfusepath{stroke}%
\end{pgfscope}%
\begin{pgfscope}%
\pgfpathrectangle{\pgfqpoint{3.985294in}{4.155455in}}{\pgfqpoint{2.279412in}{2.004545in}}%
\pgfusepath{clip}%
\pgfsetbuttcap%
\pgfsetroundjoin%
\pgfsetlinewidth{0.361167pt}%
\definecolor{currentstroke}{rgb}{0.277018,0.050344,0.375715}%
\pgfsetstrokecolor{currentstroke}%
\pgfsetdash{}{0pt}%
\pgfpathmoveto{\pgfqpoint{5.640508in}{4.753907in}}%
\pgfpathlineto{\pgfqpoint{5.590537in}{4.757584in}}%
\pgfusepath{stroke}%
\end{pgfscope}%
\begin{pgfscope}%
\pgfpathrectangle{\pgfqpoint{3.985294in}{4.155455in}}{\pgfqpoint{2.279412in}{2.004545in}}%
\pgfusepath{clip}%
\pgfsetbuttcap%
\pgfsetroundjoin%
\pgfsetlinewidth{0.382251pt}%
\definecolor{currentstroke}{rgb}{0.279566,0.067836,0.391917}%
\pgfsetstrokecolor{currentstroke}%
\pgfsetdash{}{0pt}%
\pgfpathmoveto{\pgfqpoint{5.590537in}{4.757584in}}%
\pgfpathlineto{\pgfqpoint{5.540596in}{4.761562in}}%
\pgfusepath{stroke}%
\end{pgfscope}%
\begin{pgfscope}%
\pgfpathrectangle{\pgfqpoint{3.985294in}{4.155455in}}{\pgfqpoint{2.279412in}{2.004545in}}%
\pgfusepath{clip}%
\pgfsetbuttcap%
\pgfsetroundjoin%
\pgfsetlinewidth{0.404529pt}%
\definecolor{currentstroke}{rgb}{0.281924,0.089666,0.412415}%
\pgfsetstrokecolor{currentstroke}%
\pgfsetdash{}{0pt}%
\pgfpathmoveto{\pgfqpoint{5.540596in}{4.761562in}}%
\pgfpathlineto{\pgfqpoint{5.490713in}{4.766109in}}%
\pgfusepath{stroke}%
\end{pgfscope}%
\begin{pgfscope}%
\pgfpathrectangle{\pgfqpoint{3.985294in}{4.155455in}}{\pgfqpoint{2.279412in}{2.004545in}}%
\pgfusepath{clip}%
\pgfsetbuttcap%
\pgfsetroundjoin%
\pgfsetlinewidth{0.335516pt}%
\definecolor{currentstroke}{rgb}{0.273809,0.031497,0.358853}%
\pgfsetstrokecolor{currentstroke}%
\pgfsetdash{}{0pt}%
\pgfpathmoveto{\pgfqpoint{5.627138in}{4.621141in}}%
\pgfpathlineto{\pgfqpoint{5.577149in}{4.624685in}}%
\pgfusepath{stroke}%
\end{pgfscope}%
\begin{pgfscope}%
\pgfpathrectangle{\pgfqpoint{3.985294in}{4.155455in}}{\pgfqpoint{2.279412in}{2.004545in}}%
\pgfusepath{clip}%
\pgfsetbuttcap%
\pgfsetroundjoin%
\pgfsetlinewidth{0.342960pt}%
\definecolor{currentstroke}{rgb}{0.274952,0.037752,0.364543}%
\pgfsetstrokecolor{currentstroke}%
\pgfsetdash{}{0pt}%
\pgfpathmoveto{\pgfqpoint{5.577149in}{4.624685in}}%
\pgfpathlineto{\pgfqpoint{5.527400in}{4.629921in}}%
\pgfusepath{stroke}%
\end{pgfscope}%
\begin{pgfscope}%
\pgfpathrectangle{\pgfqpoint{3.985294in}{4.155455in}}{\pgfqpoint{2.279412in}{2.004545in}}%
\pgfusepath{clip}%
\pgfsetbuttcap%
\pgfsetroundjoin%
\pgfsetlinewidth{0.353718pt}%
\definecolor{currentstroke}{rgb}{0.276022,0.044167,0.370164}%
\pgfsetstrokecolor{currentstroke}%
\pgfsetdash{}{0pt}%
\pgfpathmoveto{\pgfqpoint{5.527400in}{4.629921in}}%
\pgfpathlineto{\pgfqpoint{5.477923in}{4.636958in}}%
\pgfusepath{stroke}%
\end{pgfscope}%
\begin{pgfscope}%
\pgfpathrectangle{\pgfqpoint{3.985294in}{4.155455in}}{\pgfqpoint{2.279412in}{2.004545in}}%
\pgfusepath{clip}%
\pgfsetbuttcap%
\pgfsetroundjoin%
\pgfsetlinewidth{0.365177pt}%
\definecolor{currentstroke}{rgb}{0.277941,0.056324,0.381191}%
\pgfsetstrokecolor{currentstroke}%
\pgfsetdash{}{0pt}%
\pgfpathmoveto{\pgfqpoint{5.477923in}{4.636958in}}%
\pgfpathlineto{\pgfqpoint{5.428506in}{4.644391in}}%
\pgfusepath{stroke}%
\end{pgfscope}%
\begin{pgfscope}%
\pgfpathrectangle{\pgfqpoint{3.985294in}{4.155455in}}{\pgfqpoint{2.279412in}{2.004545in}}%
\pgfusepath{clip}%
\pgfsetbuttcap%
\pgfsetroundjoin%
\pgfsetlinewidth{0.377199pt}%
\definecolor{currentstroke}{rgb}{0.279566,0.067836,0.391917}%
\pgfsetstrokecolor{currentstroke}%
\pgfsetdash{}{0pt}%
\pgfpathmoveto{\pgfqpoint{5.428506in}{4.644391in}}%
\pgfpathlineto{\pgfqpoint{5.379156in}{4.652216in}}%
\pgfusepath{stroke}%
\end{pgfscope}%
\begin{pgfscope}%
\pgfpathrectangle{\pgfqpoint{3.985294in}{4.155455in}}{\pgfqpoint{2.279412in}{2.004545in}}%
\pgfusepath{clip}%
\pgfsetbuttcap%
\pgfsetroundjoin%
\pgfsetlinewidth{0.384966pt}%
\definecolor{currentstroke}{rgb}{0.280267,0.073417,0.397163}%
\pgfsetstrokecolor{currentstroke}%
\pgfsetdash{}{0pt}%
\pgfpathmoveto{\pgfqpoint{5.379156in}{4.652216in}}%
\pgfpathlineto{\pgfqpoint{5.330168in}{4.661553in}}%
\pgfusepath{stroke}%
\end{pgfscope}%
\begin{pgfscope}%
\pgfpathrectangle{\pgfqpoint{3.985294in}{4.155455in}}{\pgfqpoint{2.279412in}{2.004545in}}%
\pgfusepath{clip}%
\pgfsetbuttcap%
\pgfsetroundjoin%
\pgfsetlinewidth{0.403744pt}%
\definecolor{currentstroke}{rgb}{0.281446,0.084320,0.407414}%
\pgfsetstrokecolor{currentstroke}%
\pgfsetdash{}{0pt}%
\pgfpathmoveto{\pgfqpoint{5.330168in}{4.661553in}}%
\pgfpathlineto{\pgfqpoint{5.281496in}{4.672166in}}%
\pgfusepath{stroke}%
\end{pgfscope}%
\begin{pgfscope}%
\pgfpathrectangle{\pgfqpoint{3.985294in}{4.155455in}}{\pgfqpoint{2.279412in}{2.004545in}}%
\pgfusepath{clip}%
\pgfsetbuttcap%
\pgfsetroundjoin%
\pgfsetlinewidth{0.427791pt}%
\definecolor{currentstroke}{rgb}{0.282910,0.105393,0.426902}%
\pgfsetstrokecolor{currentstroke}%
\pgfsetdash{}{0pt}%
\pgfpathmoveto{\pgfqpoint{5.281496in}{4.672166in}}%
\pgfpathlineto{\pgfqpoint{5.233210in}{4.683988in}}%
\pgfusepath{stroke}%
\end{pgfscope}%
\begin{pgfscope}%
\pgfpathrectangle{\pgfqpoint{3.985294in}{4.155455in}}{\pgfqpoint{2.279412in}{2.004545in}}%
\pgfusepath{clip}%
\pgfsetbuttcap%
\pgfsetroundjoin%
\pgfsetlinewidth{0.322204pt}%
\definecolor{currentstroke}{rgb}{0.271305,0.019942,0.347269}%
\pgfsetstrokecolor{currentstroke}%
\pgfsetdash{}{0pt}%
\pgfpathmoveto{\pgfqpoint{5.823455in}{4.635851in}}%
\pgfpathlineto{\pgfqpoint{5.773360in}{4.637322in}}%
\pgfusepath{stroke}%
\end{pgfscope}%
\begin{pgfscope}%
\pgfpathrectangle{\pgfqpoint{3.985294in}{4.155455in}}{\pgfqpoint{2.279412in}{2.004545in}}%
\pgfusepath{clip}%
\pgfsetbuttcap%
\pgfsetroundjoin%
\pgfsetlinewidth{0.326500pt}%
\definecolor{currentstroke}{rgb}{0.271305,0.019942,0.347269}%
\pgfsetstrokecolor{currentstroke}%
\pgfsetdash{}{0pt}%
\pgfpathmoveto{\pgfqpoint{5.773360in}{4.637322in}}%
\pgfpathlineto{\pgfqpoint{5.723824in}{4.640806in}}%
\pgfusepath{stroke}%
\end{pgfscope}%
\begin{pgfscope}%
\pgfpathrectangle{\pgfqpoint{3.985294in}{4.155455in}}{\pgfqpoint{2.279412in}{2.004545in}}%
\pgfusepath{clip}%
\pgfsetbuttcap%
\pgfsetroundjoin%
\pgfsetlinewidth{0.316606pt}%
\definecolor{currentstroke}{rgb}{0.269944,0.014625,0.341379}%
\pgfsetstrokecolor{currentstroke}%
\pgfsetdash{}{0pt}%
\pgfpathmoveto{\pgfqpoint{5.723824in}{4.640806in}}%
\pgfpathlineto{\pgfqpoint{5.683584in}{4.644145in}}%
\pgfusepath{stroke}%
\end{pgfscope}%
\begin{pgfscope}%
\pgfpathrectangle{\pgfqpoint{3.985294in}{4.155455in}}{\pgfqpoint{2.279412in}{2.004545in}}%
\pgfusepath{clip}%
\pgfsetbuttcap%
\pgfsetroundjoin%
\pgfsetlinewidth{0.331704pt}%
\definecolor{currentstroke}{rgb}{0.272594,0.025563,0.353093}%
\pgfsetstrokecolor{currentstroke}%
\pgfsetdash{}{0pt}%
\pgfpathmoveto{\pgfqpoint{5.683584in}{4.644145in}}%
\pgfpathlineto{\pgfqpoint{5.683584in}{4.644145in}}%
\pgfusepath{stroke}%
\end{pgfscope}%
\begin{pgfscope}%
\pgfpathrectangle{\pgfqpoint{3.985294in}{4.155455in}}{\pgfqpoint{2.279412in}{2.004545in}}%
\pgfusepath{clip}%
\pgfsetbuttcap%
\pgfsetroundjoin%
\pgfsetlinewidth{0.331704pt}%
\definecolor{currentstroke}{rgb}{0.272594,0.025563,0.353093}%
\pgfsetstrokecolor{currentstroke}%
\pgfsetdash{}{0pt}%
\pgfpathmoveto{\pgfqpoint{5.683584in}{4.644145in}}%
\pgfpathlineto{\pgfqpoint{5.633625in}{4.647604in}}%
\pgfusepath{stroke}%
\end{pgfscope}%
\begin{pgfscope}%
\pgfpathrectangle{\pgfqpoint{3.985294in}{4.155455in}}{\pgfqpoint{2.279412in}{2.004545in}}%
\pgfusepath{clip}%
\pgfsetbuttcap%
\pgfsetroundjoin%
\pgfsetlinewidth{0.353609pt}%
\definecolor{currentstroke}{rgb}{0.276022,0.044167,0.370164}%
\pgfsetstrokecolor{currentstroke}%
\pgfsetdash{}{0pt}%
\pgfpathmoveto{\pgfqpoint{5.633625in}{4.647604in}}%
\pgfpathlineto{\pgfqpoint{5.583688in}{4.651542in}}%
\pgfusepath{stroke}%
\end{pgfscope}%
\begin{pgfscope}%
\pgfpathrectangle{\pgfqpoint{3.985294in}{4.155455in}}{\pgfqpoint{2.279412in}{2.004545in}}%
\pgfusepath{clip}%
\pgfsetbuttcap%
\pgfsetroundjoin%
\pgfsetlinewidth{0.355482pt}%
\definecolor{currentstroke}{rgb}{0.276022,0.044167,0.370164}%
\pgfsetstrokecolor{currentstroke}%
\pgfsetdash{}{0pt}%
\pgfpathmoveto{\pgfqpoint{5.583688in}{4.651542in}}%
\pgfpathlineto{\pgfqpoint{5.533837in}{4.656327in}}%
\pgfusepath{stroke}%
\end{pgfscope}%
\begin{pgfscope}%
\pgfpathrectangle{\pgfqpoint{3.985294in}{4.155455in}}{\pgfqpoint{2.279412in}{2.004545in}}%
\pgfusepath{clip}%
\pgfsetbuttcap%
\pgfsetroundjoin%
\pgfsetlinewidth{0.368525pt}%
\definecolor{currentstroke}{rgb}{0.277941,0.056324,0.381191}%
\pgfsetstrokecolor{currentstroke}%
\pgfsetdash{}{0pt}%
\pgfpathmoveto{\pgfqpoint{5.533837in}{4.656327in}}%
\pgfpathlineto{\pgfqpoint{5.484043in}{4.661553in}}%
\pgfusepath{stroke}%
\end{pgfscope}%
\begin{pgfscope}%
\pgfpathrectangle{\pgfqpoint{3.985294in}{4.155455in}}{\pgfqpoint{2.279412in}{2.004545in}}%
\pgfusepath{clip}%
\pgfsetbuttcap%
\pgfsetroundjoin%
\pgfsetlinewidth{0.440873pt}%
\definecolor{currentstroke}{rgb}{0.283197,0.115680,0.436115}%
\pgfsetstrokecolor{currentstroke}%
\pgfsetdash{}{0pt}%
\pgfpathmoveto{\pgfqpoint{5.176292in}{5.653902in}}%
\pgfpathlineto{\pgfqpoint{5.130437in}{5.636131in}}%
\pgfusepath{stroke}%
\end{pgfscope}%
\begin{pgfscope}%
\pgfpathrectangle{\pgfqpoint{3.985294in}{4.155455in}}{\pgfqpoint{2.279412in}{2.004545in}}%
\pgfusepath{clip}%
\pgfsetbuttcap%
\pgfsetroundjoin%
\pgfsetlinewidth{0.433881pt}%
\definecolor{currentstroke}{rgb}{0.283091,0.110553,0.431554}%
\pgfsetstrokecolor{currentstroke}%
\pgfsetdash{}{0pt}%
\pgfpathmoveto{\pgfqpoint{5.130437in}{5.636131in}}%
\pgfpathlineto{\pgfqpoint{5.086501in}{5.615132in}}%
\pgfusepath{stroke}%
\end{pgfscope}%
\begin{pgfscope}%
\pgfpathrectangle{\pgfqpoint{3.985294in}{4.155455in}}{\pgfqpoint{2.279412in}{2.004545in}}%
\pgfusepath{clip}%
\pgfsetbuttcap%
\pgfsetroundjoin%
\pgfsetlinewidth{0.397800pt}%
\definecolor{currentstroke}{rgb}{0.281446,0.084320,0.407414}%
\pgfsetstrokecolor{currentstroke}%
\pgfsetdash{}{0pt}%
\pgfpathmoveto{\pgfqpoint{5.086501in}{5.615132in}}%
\pgfpathlineto{\pgfqpoint{5.046347in}{5.589189in}}%
\pgfusepath{stroke}%
\end{pgfscope}%
\begin{pgfscope}%
\pgfpathrectangle{\pgfqpoint{3.985294in}{4.155455in}}{\pgfqpoint{2.279412in}{2.004545in}}%
\pgfusepath{clip}%
\pgfsetbuttcap%
\pgfsetroundjoin%
\pgfsetlinewidth{0.478409pt}%
\definecolor{currentstroke}{rgb}{0.282623,0.140926,0.457517}%
\pgfsetstrokecolor{currentstroke}%
\pgfsetdash{}{0pt}%
\pgfpathmoveto{\pgfqpoint{5.046347in}{5.589189in}}%
\pgfpathlineto{\pgfqpoint{5.010100in}{5.559938in}}%
\pgfusepath{stroke}%
\end{pgfscope}%
\begin{pgfscope}%
\pgfpathrectangle{\pgfqpoint{3.985294in}{4.155455in}}{\pgfqpoint{2.279412in}{2.004545in}}%
\pgfusepath{clip}%
\pgfsetbuttcap%
\pgfsetroundjoin%
\pgfsetlinewidth{0.446839pt}%
\definecolor{currentstroke}{rgb}{0.283229,0.120777,0.440584}%
\pgfsetstrokecolor{currentstroke}%
\pgfsetdash{}{0pt}%
\pgfpathmoveto{\pgfqpoint{5.010100in}{5.559938in}}%
\pgfpathlineto{\pgfqpoint{5.010100in}{5.559938in}}%
\pgfusepath{stroke}%
\end{pgfscope}%
\begin{pgfscope}%
\pgfpathrectangle{\pgfqpoint{3.985294in}{4.155455in}}{\pgfqpoint{2.279412in}{2.004545in}}%
\pgfusepath{clip}%
\pgfsetbuttcap%
\pgfsetroundjoin%
\pgfsetlinewidth{0.446839pt}%
\definecolor{currentstroke}{rgb}{0.283229,0.120777,0.440584}%
\pgfsetstrokecolor{currentstroke}%
\pgfsetdash{}{0pt}%
\pgfpathmoveto{\pgfqpoint{5.010100in}{5.559938in}}%
\pgfpathlineto{\pgfqpoint{4.994188in}{5.540465in}}%
\pgfusepath{stroke}%
\end{pgfscope}%
\begin{pgfscope}%
\pgfpathrectangle{\pgfqpoint{3.985294in}{4.155455in}}{\pgfqpoint{2.279412in}{2.004545in}}%
\pgfusepath{clip}%
\pgfsetbuttcap%
\pgfsetroundjoin%
\pgfsetlinewidth{0.526062pt}%
\definecolor{currentstroke}{rgb}{0.278826,0.175490,0.483397}%
\pgfsetstrokecolor{currentstroke}%
\pgfsetdash{}{0pt}%
\pgfpathmoveto{\pgfqpoint{4.994188in}{5.540465in}}%
\pgfpathlineto{\pgfqpoint{4.983554in}{5.519677in}}%
\pgfusepath{stroke}%
\end{pgfscope}%
\begin{pgfscope}%
\pgfpathrectangle{\pgfqpoint{3.985294in}{4.155455in}}{\pgfqpoint{2.279412in}{2.004545in}}%
\pgfusepath{clip}%
\pgfsetbuttcap%
\pgfsetroundjoin%
\pgfsetlinewidth{0.407421pt}%
\definecolor{currentstroke}{rgb}{0.281924,0.089666,0.412415}%
\pgfsetstrokecolor{currentstroke}%
\pgfsetdash{}{0pt}%
\pgfpathmoveto{\pgfqpoint{5.324671in}{4.684625in}}%
\pgfpathlineto{\pgfqpoint{5.276010in}{4.695264in}}%
\pgfusepath{stroke}%
\end{pgfscope}%
\begin{pgfscope}%
\pgfpathrectangle{\pgfqpoint{3.985294in}{4.155455in}}{\pgfqpoint{2.279412in}{2.004545in}}%
\pgfusepath{clip}%
\pgfsetbuttcap%
\pgfsetroundjoin%
\pgfsetlinewidth{0.438419pt}%
\definecolor{currentstroke}{rgb}{0.283197,0.115680,0.436115}%
\pgfsetstrokecolor{currentstroke}%
\pgfsetdash{}{0pt}%
\pgfpathmoveto{\pgfqpoint{5.276010in}{4.695264in}}%
\pgfpathlineto{\pgfqpoint{5.227584in}{4.706659in}}%
\pgfusepath{stroke}%
\end{pgfscope}%
\begin{pgfscope}%
\pgfpathrectangle{\pgfqpoint{3.985294in}{4.155455in}}{\pgfqpoint{2.279412in}{2.004545in}}%
\pgfusepath{clip}%
\pgfsetbuttcap%
\pgfsetroundjoin%
\pgfsetlinewidth{0.457470pt}%
\definecolor{currentstroke}{rgb}{0.283187,0.125848,0.444960}%
\pgfsetstrokecolor{currentstroke}%
\pgfsetdash{}{0pt}%
\pgfpathmoveto{\pgfqpoint{5.227584in}{4.706659in}}%
\pgfpathlineto{\pgfqpoint{5.179875in}{4.720170in}}%
\pgfusepath{stroke}%
\end{pgfscope}%
\begin{pgfscope}%
\pgfpathrectangle{\pgfqpoint{3.985294in}{4.155455in}}{\pgfqpoint{2.279412in}{2.004545in}}%
\pgfusepath{clip}%
\pgfsetbuttcap%
\pgfsetroundjoin%
\pgfsetlinewidth{0.473288pt}%
\definecolor{currentstroke}{rgb}{0.282623,0.140926,0.457517}%
\pgfsetstrokecolor{currentstroke}%
\pgfsetdash{}{0pt}%
\pgfpathmoveto{\pgfqpoint{5.179875in}{4.720170in}}%
\pgfpathlineto{\pgfqpoint{5.133239in}{4.736253in}}%
\pgfusepath{stroke}%
\end{pgfscope}%
\begin{pgfscope}%
\pgfpathrectangle{\pgfqpoint{3.985294in}{4.155455in}}{\pgfqpoint{2.279412in}{2.004545in}}%
\pgfusepath{clip}%
\pgfsetbuttcap%
\pgfsetroundjoin%
\pgfsetlinewidth{0.463304pt}%
\definecolor{currentstroke}{rgb}{0.283072,0.130895,0.449241}%
\pgfsetstrokecolor{currentstroke}%
\pgfsetdash{}{0pt}%
\pgfpathmoveto{\pgfqpoint{5.133239in}{4.736253in}}%
\pgfpathlineto{\pgfqpoint{5.088608in}{4.756085in}}%
\pgfusepath{stroke}%
\end{pgfscope}%
\begin{pgfscope}%
\pgfpathrectangle{\pgfqpoint{3.985294in}{4.155455in}}{\pgfqpoint{2.279412in}{2.004545in}}%
\pgfusepath{clip}%
\pgfsetbuttcap%
\pgfsetroundjoin%
\pgfsetlinewidth{0.496788pt}%
\definecolor{currentstroke}{rgb}{0.281412,0.155834,0.469201}%
\pgfsetstrokecolor{currentstroke}%
\pgfsetdash{}{0pt}%
\pgfpathmoveto{\pgfqpoint{5.088608in}{4.756085in}}%
\pgfpathlineto{\pgfqpoint{5.048002in}{4.781332in}}%
\pgfusepath{stroke}%
\end{pgfscope}%
\begin{pgfscope}%
\pgfpathrectangle{\pgfqpoint{3.985294in}{4.155455in}}{\pgfqpoint{2.279412in}{2.004545in}}%
\pgfusepath{clip}%
\pgfsetbuttcap%
\pgfsetroundjoin%
\pgfsetlinewidth{0.518354pt}%
\definecolor{currentstroke}{rgb}{0.279574,0.170599,0.479997}%
\pgfsetstrokecolor{currentstroke}%
\pgfsetdash{}{0pt}%
\pgfpathmoveto{\pgfqpoint{5.048002in}{4.781332in}}%
\pgfpathlineto{\pgfqpoint{5.011877in}{4.811372in}}%
\pgfusepath{stroke}%
\end{pgfscope}%
\begin{pgfscope}%
\pgfpathrectangle{\pgfqpoint{3.985294in}{4.155455in}}{\pgfqpoint{2.279412in}{2.004545in}}%
\pgfusepath{clip}%
\pgfsetbuttcap%
\pgfsetroundjoin%
\pgfsetlinewidth{0.601501pt}%
\definecolor{currentstroke}{rgb}{0.266580,0.228262,0.514349}%
\pgfsetstrokecolor{currentstroke}%
\pgfsetdash{}{0pt}%
\pgfpathmoveto{\pgfqpoint{5.011877in}{4.811372in}}%
\pgfpathlineto{\pgfqpoint{4.982813in}{4.845481in}}%
\pgfusepath{stroke}%
\end{pgfscope}%
\begin{pgfscope}%
\pgfpathrectangle{\pgfqpoint{3.985294in}{4.155455in}}{\pgfqpoint{2.279412in}{2.004545in}}%
\pgfusepath{clip}%
\pgfsetbuttcap%
\pgfsetroundjoin%
\pgfsetlinewidth{0.350167pt}%
\definecolor{currentstroke}{rgb}{0.276022,0.044167,0.370164}%
\pgfsetstrokecolor{currentstroke}%
\pgfsetdash{}{0pt}%
\pgfpathmoveto{\pgfqpoint{5.637919in}{5.608795in}}%
\pgfpathlineto{\pgfqpoint{5.587947in}{5.605142in}}%
\pgfusepath{stroke}%
\end{pgfscope}%
\begin{pgfscope}%
\pgfpathrectangle{\pgfqpoint{3.985294in}{4.155455in}}{\pgfqpoint{2.279412in}{2.004545in}}%
\pgfusepath{clip}%
\pgfsetbuttcap%
\pgfsetroundjoin%
\pgfsetlinewidth{0.361011pt}%
\definecolor{currentstroke}{rgb}{0.277018,0.050344,0.375715}%
\pgfsetstrokecolor{currentstroke}%
\pgfsetdash{}{0pt}%
\pgfpathmoveto{\pgfqpoint{5.587947in}{5.605142in}}%
\pgfpathlineto{\pgfqpoint{5.538055in}{5.600720in}}%
\pgfusepath{stroke}%
\end{pgfscope}%
\begin{pgfscope}%
\pgfpathrectangle{\pgfqpoint{3.985294in}{4.155455in}}{\pgfqpoint{2.279412in}{2.004545in}}%
\pgfusepath{clip}%
\pgfsetbuttcap%
\pgfsetroundjoin%
\pgfsetlinewidth{0.377966pt}%
\definecolor{currentstroke}{rgb}{0.279566,0.067836,0.391917}%
\pgfsetstrokecolor{currentstroke}%
\pgfsetdash{}{0pt}%
\pgfpathmoveto{\pgfqpoint{5.538055in}{5.600720in}}%
\pgfpathlineto{\pgfqpoint{5.488199in}{5.595955in}}%
\pgfusepath{stroke}%
\end{pgfscope}%
\begin{pgfscope}%
\pgfpathrectangle{\pgfqpoint{3.985294in}{4.155455in}}{\pgfqpoint{2.279412in}{2.004545in}}%
\pgfusepath{clip}%
\pgfsetbuttcap%
\pgfsetroundjoin%
\pgfsetlinewidth{0.403108pt}%
\definecolor{currentstroke}{rgb}{0.281446,0.084320,0.407414}%
\pgfsetstrokecolor{currentstroke}%
\pgfsetdash{}{0pt}%
\pgfpathmoveto{\pgfqpoint{5.488199in}{5.595955in}}%
\pgfpathlineto{\pgfqpoint{5.438423in}{5.590624in}}%
\pgfusepath{stroke}%
\end{pgfscope}%
\begin{pgfscope}%
\pgfpathrectangle{\pgfqpoint{3.985294in}{4.155455in}}{\pgfqpoint{2.279412in}{2.004545in}}%
\pgfusepath{clip}%
\pgfsetbuttcap%
\pgfsetroundjoin%
\pgfsetlinewidth{0.426416pt}%
\definecolor{currentstroke}{rgb}{0.282910,0.105393,0.426902}%
\pgfsetstrokecolor{currentstroke}%
\pgfsetdash{}{0pt}%
\pgfpathmoveto{\pgfqpoint{5.438423in}{5.590624in}}%
\pgfpathlineto{\pgfqpoint{5.388787in}{5.584325in}}%
\pgfusepath{stroke}%
\end{pgfscope}%
\begin{pgfscope}%
\pgfpathrectangle{\pgfqpoint{3.985294in}{4.155455in}}{\pgfqpoint{2.279412in}{2.004545in}}%
\pgfusepath{clip}%
\pgfsetbuttcap%
\pgfsetroundjoin%
\pgfsetlinewidth{0.439890pt}%
\definecolor{currentstroke}{rgb}{0.283197,0.115680,0.436115}%
\pgfsetstrokecolor{currentstroke}%
\pgfsetdash{}{0pt}%
\pgfpathmoveto{\pgfqpoint{5.388787in}{5.584325in}}%
\pgfpathlineto{\pgfqpoint{5.339370in}{5.576853in}}%
\pgfusepath{stroke}%
\end{pgfscope}%
\begin{pgfscope}%
\pgfpathrectangle{\pgfqpoint{3.985294in}{4.155455in}}{\pgfqpoint{2.279412in}{2.004545in}}%
\pgfusepath{clip}%
\pgfsetbuttcap%
\pgfsetroundjoin%
\pgfsetlinewidth{0.448162pt}%
\definecolor{currentstroke}{rgb}{0.283229,0.120777,0.440584}%
\pgfsetstrokecolor{currentstroke}%
\pgfsetdash{}{0pt}%
\pgfpathmoveto{\pgfqpoint{5.339370in}{5.576853in}}%
\pgfpathlineto{\pgfqpoint{5.290172in}{5.568307in}}%
\pgfusepath{stroke}%
\end{pgfscope}%
\begin{pgfscope}%
\pgfpathrectangle{\pgfqpoint{3.985294in}{4.155455in}}{\pgfqpoint{2.279412in}{2.004545in}}%
\pgfusepath{clip}%
\pgfsetbuttcap%
\pgfsetroundjoin%
\pgfsetlinewidth{0.465788pt}%
\definecolor{currentstroke}{rgb}{0.283072,0.130895,0.449241}%
\pgfsetstrokecolor{currentstroke}%
\pgfsetdash{}{0pt}%
\pgfpathmoveto{\pgfqpoint{5.290172in}{5.568307in}}%
\pgfpathlineto{\pgfqpoint{5.241316in}{5.558423in}}%
\pgfusepath{stroke}%
\end{pgfscope}%
\begin{pgfscope}%
\pgfpathrectangle{\pgfqpoint{3.985294in}{4.155455in}}{\pgfqpoint{2.279412in}{2.004545in}}%
\pgfusepath{clip}%
\pgfsetbuttcap%
\pgfsetroundjoin%
\pgfsetlinewidth{0.503949pt}%
\definecolor{currentstroke}{rgb}{0.280868,0.160771,0.472899}%
\pgfsetstrokecolor{currentstroke}%
\pgfsetdash{}{0pt}%
\pgfpathmoveto{\pgfqpoint{5.241316in}{5.558423in}}%
\pgfpathlineto{\pgfqpoint{5.193001in}{5.546628in}}%
\pgfusepath{stroke}%
\end{pgfscope}%
\begin{pgfscope}%
\pgfpathrectangle{\pgfqpoint{3.985294in}{4.155455in}}{\pgfqpoint{2.279412in}{2.004545in}}%
\pgfusepath{clip}%
\pgfsetbuttcap%
\pgfsetroundjoin%
\pgfsetlinewidth{0.521465pt}%
\definecolor{currentstroke}{rgb}{0.278826,0.175490,0.483397}%
\pgfsetstrokecolor{currentstroke}%
\pgfsetdash{}{0pt}%
\pgfpathmoveto{\pgfqpoint{5.193001in}{5.546628in}}%
\pgfpathlineto{\pgfqpoint{5.145507in}{5.532549in}}%
\pgfusepath{stroke}%
\end{pgfscope}%
\begin{pgfscope}%
\pgfpathrectangle{\pgfqpoint{3.985294in}{4.155455in}}{\pgfqpoint{2.279412in}{2.004545in}}%
\pgfusepath{clip}%
\pgfsetbuttcap%
\pgfsetroundjoin%
\pgfsetlinewidth{0.546342pt}%
\definecolor{currentstroke}{rgb}{0.276194,0.190074,0.493001}%
\pgfsetstrokecolor{currentstroke}%
\pgfsetdash{}{0pt}%
\pgfpathmoveto{\pgfqpoint{5.145507in}{5.532549in}}%
\pgfpathlineto{\pgfqpoint{5.099283in}{5.515536in}}%
\pgfusepath{stroke}%
\end{pgfscope}%
\begin{pgfscope}%
\pgfpathrectangle{\pgfqpoint{3.985294in}{4.155455in}}{\pgfqpoint{2.279412in}{2.004545in}}%
\pgfusepath{clip}%
\pgfsetbuttcap%
\pgfsetroundjoin%
\pgfsetlinewidth{0.000000pt}%
\definecolor{currentstroke}{rgb}{0.276194,0.190074,0.493001}%
\pgfsetstrokecolor{currentstroke}%
\pgfsetdash{}{0pt}%
\pgfpathmoveto{\pgfqpoint{4.828028in}{4.817346in}}%
\pgfpathlineto{\pgfqpoint{4.868541in}{4.841980in}}%
\pgfusepath{}%
\end{pgfscope}%
\begin{pgfscope}%
\pgfpathrectangle{\pgfqpoint{3.985294in}{4.155455in}}{\pgfqpoint{2.279412in}{2.004545in}}%
\pgfusepath{clip}%
\pgfsetbuttcap%
\pgfsetroundjoin%
\pgfsetlinewidth{0.634477pt}%
\definecolor{currentstroke}{rgb}{0.258965,0.251537,0.524736}%
\pgfsetstrokecolor{currentstroke}%
\pgfsetdash{}{0pt}%
\pgfpathmoveto{\pgfqpoint{4.868541in}{4.841980in}}%
\pgfpathlineto{\pgfqpoint{4.868541in}{4.841980in}}%
\pgfusepath{stroke}%
\end{pgfscope}%
\begin{pgfscope}%
\pgfpathrectangle{\pgfqpoint{3.985294in}{4.155455in}}{\pgfqpoint{2.279412in}{2.004545in}}%
\pgfusepath{clip}%
\pgfsetbuttcap%
\pgfsetroundjoin%
\pgfsetlinewidth{0.634477pt}%
\definecolor{currentstroke}{rgb}{0.258965,0.251537,0.524736}%
\pgfsetstrokecolor{currentstroke}%
\pgfsetdash{}{0pt}%
\pgfpathmoveto{\pgfqpoint{4.868541in}{4.841980in}}%
\pgfpathlineto{\pgfqpoint{4.883376in}{4.858290in}}%
\pgfusepath{stroke}%
\end{pgfscope}%
\begin{pgfscope}%
\pgfpathrectangle{\pgfqpoint{3.985294in}{4.155455in}}{\pgfqpoint{2.279412in}{2.004545in}}%
\pgfusepath{clip}%
\pgfsetbuttcap%
\pgfsetroundjoin%
\pgfsetlinewidth{0.574842pt}%
\definecolor{currentstroke}{rgb}{0.271828,0.209303,0.504434}%
\pgfsetstrokecolor{currentstroke}%
\pgfsetdash{}{0pt}%
\pgfpathmoveto{\pgfqpoint{4.883376in}{4.858290in}}%
\pgfpathlineto{\pgfqpoint{4.883376in}{4.858290in}}%
\pgfusepath{stroke}%
\end{pgfscope}%
\begin{pgfscope}%
\pgfpathrectangle{\pgfqpoint{3.985294in}{4.155455in}}{\pgfqpoint{2.279412in}{2.004545in}}%
\pgfusepath{clip}%
\pgfsetbuttcap%
\pgfsetroundjoin%
\pgfsetlinewidth{0.574842pt}%
\definecolor{currentstroke}{rgb}{0.271828,0.209303,0.504434}%
\pgfsetstrokecolor{currentstroke}%
\pgfsetdash{}{0pt}%
\pgfpathmoveto{\pgfqpoint{4.883376in}{4.858290in}}%
\pgfpathlineto{\pgfqpoint{4.888982in}{4.874060in}}%
\pgfusepath{stroke}%
\end{pgfscope}%
\begin{pgfscope}%
\pgfpathrectangle{\pgfqpoint{3.985294in}{4.155455in}}{\pgfqpoint{2.279412in}{2.004545in}}%
\pgfusepath{clip}%
\pgfsetbuttcap%
\pgfsetroundjoin%
\pgfsetlinewidth{0.583899pt}%
\definecolor{currentstroke}{rgb}{0.269308,0.218818,0.509577}%
\pgfsetstrokecolor{currentstroke}%
\pgfsetdash{}{0pt}%
\pgfpathmoveto{\pgfqpoint{4.888982in}{4.874060in}}%
\pgfpathlineto{\pgfqpoint{4.889760in}{4.889934in}}%
\pgfusepath{stroke}%
\end{pgfscope}%
\begin{pgfscope}%
\pgfpathrectangle{\pgfqpoint{3.985294in}{4.155455in}}{\pgfqpoint{2.279412in}{2.004545in}}%
\pgfusepath{clip}%
\pgfsetbuttcap%
\pgfsetroundjoin%
\pgfsetlinewidth{0.643711pt}%
\definecolor{currentstroke}{rgb}{0.257322,0.256130,0.526563}%
\pgfsetstrokecolor{currentstroke}%
\pgfsetdash{}{0pt}%
\pgfpathmoveto{\pgfqpoint{4.889760in}{4.889934in}}%
\pgfpathlineto{\pgfqpoint{4.890756in}{4.917333in}}%
\pgfusepath{stroke}%
\end{pgfscope}%
\begin{pgfscope}%
\pgfpathrectangle{\pgfqpoint{3.985294in}{4.155455in}}{\pgfqpoint{2.279412in}{2.004545in}}%
\pgfusepath{clip}%
\pgfsetbuttcap%
\pgfsetroundjoin%
\pgfsetlinewidth{0.640524pt}%
\definecolor{currentstroke}{rgb}{0.257322,0.256130,0.526563}%
\pgfsetstrokecolor{currentstroke}%
\pgfsetdash{}{0pt}%
\pgfpathmoveto{\pgfqpoint{4.890756in}{4.917333in}}%
\pgfpathlineto{\pgfqpoint{4.889561in}{4.960071in}}%
\pgfusepath{stroke}%
\end{pgfscope}%
\begin{pgfscope}%
\pgfpathrectangle{\pgfqpoint{3.985294in}{4.155455in}}{\pgfqpoint{2.279412in}{2.004545in}}%
\pgfusepath{clip}%
\pgfsetbuttcap%
\pgfsetroundjoin%
\pgfsetlinewidth{0.694629pt}%
\definecolor{currentstroke}{rgb}{0.243113,0.292092,0.538516}%
\pgfsetstrokecolor{currentstroke}%
\pgfsetdash{}{0pt}%
\pgfpathmoveto{\pgfqpoint{4.889561in}{4.960071in}}%
\pgfpathlineto{\pgfqpoint{4.889561in}{4.960071in}}%
\pgfusepath{stroke}%
\end{pgfscope}%
\begin{pgfscope}%
\pgfpathrectangle{\pgfqpoint{3.985294in}{4.155455in}}{\pgfqpoint{2.279412in}{2.004545in}}%
\pgfusepath{clip}%
\pgfsetbuttcap%
\pgfsetroundjoin%
\pgfsetlinewidth{0.694629pt}%
\definecolor{currentstroke}{rgb}{0.243113,0.292092,0.538516}%
\pgfsetstrokecolor{currentstroke}%
\pgfsetdash{}{0pt}%
\pgfpathmoveto{\pgfqpoint{4.889561in}{4.960071in}}%
\pgfpathlineto{\pgfqpoint{4.883523in}{4.980422in}}%
\pgfusepath{stroke}%
\end{pgfscope}%
\begin{pgfscope}%
\pgfpathrectangle{\pgfqpoint{3.985294in}{4.155455in}}{\pgfqpoint{2.279412in}{2.004545in}}%
\pgfusepath{clip}%
\pgfsetbuttcap%
\pgfsetroundjoin%
\pgfsetlinewidth{0.939730pt}%
\definecolor{currentstroke}{rgb}{0.177423,0.437527,0.557565}%
\pgfsetstrokecolor{currentstroke}%
\pgfsetdash{}{0pt}%
\pgfpathmoveto{\pgfqpoint{4.883523in}{4.980422in}}%
\pgfpathlineto{\pgfqpoint{4.872063in}{4.999380in}}%
\pgfusepath{stroke}%
\end{pgfscope}%
\begin{pgfscope}%
\pgfpathrectangle{\pgfqpoint{3.985294in}{4.155455in}}{\pgfqpoint{2.279412in}{2.004545in}}%
\pgfusepath{clip}%
\pgfsetbuttcap%
\pgfsetroundjoin%
\pgfsetlinewidth{0.703759pt}%
\definecolor{currentstroke}{rgb}{0.241237,0.296485,0.539709}%
\pgfsetstrokecolor{currentstroke}%
\pgfsetdash{}{0pt}%
\pgfpathmoveto{\pgfqpoint{5.330168in}{4.932193in}}%
\pgfpathlineto{\pgfqpoint{5.280313in}{4.936955in}}%
\pgfusepath{stroke}%
\end{pgfscope}%
\begin{pgfscope}%
\pgfpathrectangle{\pgfqpoint{3.985294in}{4.155455in}}{\pgfqpoint{2.279412in}{2.004545in}}%
\pgfusepath{clip}%
\pgfsetbuttcap%
\pgfsetroundjoin%
\pgfsetlinewidth{0.792613pt}%
\definecolor{currentstroke}{rgb}{0.216210,0.351535,0.550627}%
\pgfsetstrokecolor{currentstroke}%
\pgfsetdash{}{0pt}%
\pgfpathmoveto{\pgfqpoint{5.280313in}{4.936955in}}%
\pgfpathlineto{\pgfqpoint{5.230594in}{4.942713in}}%
\pgfusepath{stroke}%
\end{pgfscope}%
\begin{pgfscope}%
\pgfpathrectangle{\pgfqpoint{3.985294in}{4.155455in}}{\pgfqpoint{2.279412in}{2.004545in}}%
\pgfusepath{clip}%
\pgfsetbuttcap%
\pgfsetroundjoin%
\pgfsetlinewidth{0.805336pt}%
\definecolor{currentstroke}{rgb}{0.212395,0.359683,0.551710}%
\pgfsetstrokecolor{currentstroke}%
\pgfsetdash{}{0pt}%
\pgfpathmoveto{\pgfqpoint{5.230594in}{4.942713in}}%
\pgfpathlineto{\pgfqpoint{5.181062in}{4.949577in}}%
\pgfusepath{stroke}%
\end{pgfscope}%
\begin{pgfscope}%
\pgfpathrectangle{\pgfqpoint{3.985294in}{4.155455in}}{\pgfqpoint{2.279412in}{2.004545in}}%
\pgfusepath{clip}%
\pgfsetbuttcap%
\pgfsetroundjoin%
\pgfsetlinewidth{0.858018pt}%
\definecolor{currentstroke}{rgb}{0.197636,0.391528,0.554969}%
\pgfsetstrokecolor{currentstroke}%
\pgfsetdash{}{0pt}%
\pgfpathmoveto{\pgfqpoint{5.181062in}{4.949577in}}%
\pgfpathlineto{\pgfqpoint{5.131779in}{4.957704in}}%
\pgfusepath{stroke}%
\end{pgfscope}%
\begin{pgfscope}%
\pgfpathrectangle{\pgfqpoint{3.985294in}{4.155455in}}{\pgfqpoint{2.279412in}{2.004545in}}%
\pgfusepath{clip}%
\pgfsetbuttcap%
\pgfsetroundjoin%
\pgfsetlinewidth{0.833414pt}%
\definecolor{currentstroke}{rgb}{0.204903,0.375746,0.553533}%
\pgfsetstrokecolor{currentstroke}%
\pgfsetdash{}{0pt}%
\pgfpathmoveto{\pgfqpoint{5.131779in}{4.957704in}}%
\pgfpathlineto{\pgfqpoint{5.082923in}{4.967582in}}%
\pgfusepath{stroke}%
\end{pgfscope}%
\begin{pgfscope}%
\pgfpathrectangle{\pgfqpoint{3.985294in}{4.155455in}}{\pgfqpoint{2.279412in}{2.004545in}}%
\pgfusepath{clip}%
\pgfsetbuttcap%
\pgfsetroundjoin%
\pgfsetlinewidth{0.947737pt}%
\definecolor{currentstroke}{rgb}{0.175841,0.441290,0.557685}%
\pgfsetstrokecolor{currentstroke}%
\pgfsetdash{}{0pt}%
\pgfpathmoveto{\pgfqpoint{5.082923in}{4.967582in}}%
\pgfpathlineto{\pgfqpoint{5.034783in}{4.979860in}}%
\pgfusepath{stroke}%
\end{pgfscope}%
\begin{pgfscope}%
\pgfpathrectangle{\pgfqpoint{3.985294in}{4.155455in}}{\pgfqpoint{2.279412in}{2.004545in}}%
\pgfusepath{clip}%
\pgfsetbuttcap%
\pgfsetroundjoin%
\pgfsetlinewidth{0.864446pt}%
\definecolor{currentstroke}{rgb}{0.195860,0.395433,0.555276}%
\pgfsetstrokecolor{currentstroke}%
\pgfsetdash{}{0pt}%
\pgfpathmoveto{\pgfqpoint{5.255262in}{4.980680in}}%
\pgfpathlineto{\pgfqpoint{5.205503in}{4.986163in}}%
\pgfusepath{stroke}%
\end{pgfscope}%
\begin{pgfscope}%
\pgfpathrectangle{\pgfqpoint{3.985294in}{4.155455in}}{\pgfqpoint{2.279412in}{2.004545in}}%
\pgfusepath{clip}%
\pgfsetbuttcap%
\pgfsetroundjoin%
\pgfsetlinewidth{0.955237pt}%
\definecolor{currentstroke}{rgb}{0.174274,0.445044,0.557792}%
\pgfsetstrokecolor{currentstroke}%
\pgfsetdash{}{0pt}%
\pgfpathmoveto{\pgfqpoint{5.205503in}{4.986163in}}%
\pgfpathlineto{\pgfqpoint{5.155888in}{4.992569in}}%
\pgfusepath{stroke}%
\end{pgfscope}%
\begin{pgfscope}%
\pgfpathrectangle{\pgfqpoint{3.985294in}{4.155455in}}{\pgfqpoint{2.279412in}{2.004545in}}%
\pgfusepath{clip}%
\pgfsetbuttcap%
\pgfsetroundjoin%
\pgfsetlinewidth{1.029814pt}%
\definecolor{currentstroke}{rgb}{0.157729,0.485932,0.558013}%
\pgfsetstrokecolor{currentstroke}%
\pgfsetdash{}{0pt}%
\pgfpathmoveto{\pgfqpoint{5.155888in}{4.992569in}}%
\pgfpathlineto{\pgfqpoint{5.106534in}{5.000342in}}%
\pgfusepath{stroke}%
\end{pgfscope}%
\begin{pgfscope}%
\pgfpathrectangle{\pgfqpoint{3.985294in}{4.155455in}}{\pgfqpoint{2.279412in}{2.004545in}}%
\pgfusepath{clip}%
\pgfsetbuttcap%
\pgfsetroundjoin%
\pgfsetlinewidth{1.012791pt}%
\definecolor{currentstroke}{rgb}{0.162142,0.474838,0.558140}%
\pgfsetstrokecolor{currentstroke}%
\pgfsetdash{}{0pt}%
\pgfpathmoveto{\pgfqpoint{5.106534in}{5.000342in}}%
\pgfpathlineto{\pgfqpoint{5.057597in}{5.009929in}}%
\pgfusepath{stroke}%
\end{pgfscope}%
\begin{pgfscope}%
\pgfpathrectangle{\pgfqpoint{3.985294in}{4.155455in}}{\pgfqpoint{2.279412in}{2.004545in}}%
\pgfusepath{clip}%
\pgfsetbuttcap%
\pgfsetroundjoin%
\pgfsetlinewidth{1.026264pt}%
\definecolor{currentstroke}{rgb}{0.159194,0.482237,0.558073}%
\pgfsetstrokecolor{currentstroke}%
\pgfsetdash{}{0pt}%
\pgfpathmoveto{\pgfqpoint{5.057597in}{5.009929in}}%
\pgfpathlineto{\pgfqpoint{5.009155in}{5.021284in}}%
\pgfusepath{stroke}%
\end{pgfscope}%
\begin{pgfscope}%
\pgfpathrectangle{\pgfqpoint{3.985294in}{4.155455in}}{\pgfqpoint{2.279412in}{2.004545in}}%
\pgfusepath{clip}%
\pgfsetbuttcap%
\pgfsetroundjoin%
\pgfsetlinewidth{1.169510pt}%
\definecolor{currentstroke}{rgb}{0.129933,0.559582,0.551864}%
\pgfsetstrokecolor{currentstroke}%
\pgfsetdash{}{0pt}%
\pgfpathmoveto{\pgfqpoint{5.009155in}{5.021284in}}%
\pgfpathlineto{\pgfqpoint{4.961410in}{5.034702in}}%
\pgfusepath{stroke}%
\end{pgfscope}%
\begin{pgfscope}%
\pgfpathrectangle{\pgfqpoint{3.985294in}{4.155455in}}{\pgfqpoint{2.279412in}{2.004545in}}%
\pgfusepath{clip}%
\pgfsetbuttcap%
\pgfsetroundjoin%
\pgfsetlinewidth{1.095416pt}%
\definecolor{currentstroke}{rgb}{0.144759,0.519093,0.556572}%
\pgfsetstrokecolor{currentstroke}%
\pgfsetdash{}{0pt}%
\pgfpathmoveto{\pgfqpoint{4.961410in}{5.034702in}}%
\pgfpathlineto{\pgfqpoint{4.914523in}{5.050196in}}%
\pgfusepath{stroke}%
\end{pgfscope}%
\begin{pgfscope}%
\pgfpathrectangle{\pgfqpoint{3.985294in}{4.155455in}}{\pgfqpoint{2.279412in}{2.004545in}}%
\pgfusepath{clip}%
\pgfsetbuttcap%
\pgfsetroundjoin%
\pgfsetlinewidth{1.115895pt}%
\definecolor{currentstroke}{rgb}{0.140536,0.530132,0.555659}%
\pgfsetstrokecolor{currentstroke}%
\pgfsetdash{}{0pt}%
\pgfpathmoveto{\pgfqpoint{4.914523in}{5.050196in}}%
\pgfpathlineto{\pgfqpoint{4.868541in}{5.067514in}}%
\pgfusepath{stroke}%
\end{pgfscope}%
\begin{pgfscope}%
\pgfpathrectangle{\pgfqpoint{3.985294in}{4.155455in}}{\pgfqpoint{2.279412in}{2.004545in}}%
\pgfusepath{clip}%
\pgfsetroundcap%
\pgfsetroundjoin%
\pgfsetlinewidth{0.307872pt}%
\definecolor{currentstroke}{rgb}{0.267004,0.004874,0.329415}%
\pgfsetstrokecolor{currentstroke}%
\pgfsetdash{}{0pt}%
\pgfpathmoveto{\pgfqpoint{6.024314in}{5.041070in}}%
\pgfpathquadraticcurveto{\pgfqpoint{6.024665in}{5.040580in}}{\pgfqpoint{6.022239in}{5.043960in}}%
\pgfusepath{stroke}%
\end{pgfscope}%
\begin{pgfscope}%
\pgfpathrectangle{\pgfqpoint{3.985294in}{4.155455in}}{\pgfqpoint{2.279412in}{2.004545in}}%
\pgfusepath{clip}%
\pgfsetroundcap%
\pgfsetroundjoin%
\definecolor{currentfill}{rgb}{0.267004,0.004874,0.329415}%
\pgfsetfillcolor{currentfill}%
\pgfsetlinewidth{0.307872pt}%
\definecolor{currentstroke}{rgb}{0.267004,0.004874,0.329415}%
\pgfsetstrokecolor{currentstroke}%
\pgfsetdash{}{0pt}%
\pgfpathmoveto{\pgfqpoint{6.012401in}{5.105289in}}%
\pgfpathlineto{\pgfqpoint{6.022239in}{5.043960in}}%
\pgfpathlineto{\pgfqpoint{5.967273in}{5.072887in}}%
\pgfpathlineto{\pgfqpoint{6.012401in}{5.105289in}}%
\pgfpathlineto{\pgfqpoint{6.012401in}{5.105289in}}%
\pgfpathclose%
\pgfusepath{stroke,fill}%
\end{pgfscope}%
\begin{pgfscope}%
\pgfpathrectangle{\pgfqpoint{3.985294in}{4.155455in}}{\pgfqpoint{2.279412in}{2.004545in}}%
\pgfusepath{clip}%
\pgfsetroundcap%
\pgfsetroundjoin%
\pgfsetlinewidth{1.023449pt}%
\definecolor{currentstroke}{rgb}{0.159194,0.482237,0.558073}%
\pgfsetstrokecolor{currentstroke}%
\pgfsetdash{}{0pt}%
\pgfpathmoveto{\pgfqpoint{5.295448in}{5.043548in}}%
\pgfpathquadraticcurveto{\pgfqpoint{5.282943in}{5.044343in}}{\pgfqpoint{5.286240in}{5.044133in}}%
\pgfusepath{stroke}%
\end{pgfscope}%
\begin{pgfscope}%
\pgfpathrectangle{\pgfqpoint{3.985294in}{4.155455in}}{\pgfqpoint{2.279412in}{2.004545in}}%
\pgfusepath{clip}%
\pgfsetroundcap%
\pgfsetroundjoin%
\definecolor{currentfill}{rgb}{0.159194,0.482237,0.558073}%
\pgfsetfillcolor{currentfill}%
\pgfsetlinewidth{1.023449pt}%
\definecolor{currentstroke}{rgb}{0.159194,0.482237,0.558073}%
\pgfsetstrokecolor{currentstroke}%
\pgfsetdash{}{0pt}%
\pgfpathmoveto{\pgfqpoint{5.339922in}{5.012889in}}%
\pgfpathlineto{\pgfqpoint{5.286240in}{5.044133in}}%
\pgfpathlineto{\pgfqpoint{5.343445in}{5.068333in}}%
\pgfpathlineto{\pgfqpoint{5.339922in}{5.012889in}}%
\pgfpathlineto{\pgfqpoint{5.339922in}{5.012889in}}%
\pgfpathclose%
\pgfusepath{stroke,fill}%
\end{pgfscope}%
\begin{pgfscope}%
\pgfpathrectangle{\pgfqpoint{3.985294in}{4.155455in}}{\pgfqpoint{2.279412in}{2.004545in}}%
\pgfusepath{clip}%
\pgfsetroundcap%
\pgfsetroundjoin%
\pgfsetlinewidth{0.449938pt}%
\definecolor{currentstroke}{rgb}{0.283229,0.120777,0.440584}%
\pgfsetstrokecolor{currentstroke}%
\pgfsetdash{}{0pt}%
\pgfpathmoveto{\pgfqpoint{5.646067in}{5.069907in}}%
\pgfpathquadraticcurveto{\pgfqpoint{5.633532in}{5.070135in}}{\pgfqpoint{5.627958in}{5.070237in}}%
\pgfusepath{stroke}%
\end{pgfscope}%
\begin{pgfscope}%
\pgfpathrectangle{\pgfqpoint{3.985294in}{4.155455in}}{\pgfqpoint{2.279412in}{2.004545in}}%
\pgfusepath{clip}%
\pgfsetroundcap%
\pgfsetroundjoin%
\definecolor{currentfill}{rgb}{0.283229,0.120777,0.440584}%
\pgfsetfillcolor{currentfill}%
\pgfsetlinewidth{0.449938pt}%
\definecolor{currentstroke}{rgb}{0.283229,0.120777,0.440584}%
\pgfsetstrokecolor{currentstroke}%
\pgfsetdash{}{0pt}%
\pgfpathmoveto{\pgfqpoint{5.682998in}{5.041451in}}%
\pgfpathlineto{\pgfqpoint{5.627958in}{5.070237in}}%
\pgfpathlineto{\pgfqpoint{5.684010in}{5.096997in}}%
\pgfpathlineto{\pgfqpoint{5.682998in}{5.041451in}}%
\pgfpathlineto{\pgfqpoint{5.682998in}{5.041451in}}%
\pgfpathclose%
\pgfusepath{stroke,fill}%
\end{pgfscope}%
\begin{pgfscope}%
\pgfpathrectangle{\pgfqpoint{3.985294in}{4.155455in}}{\pgfqpoint{2.279412in}{2.004545in}}%
\pgfusepath{clip}%
\pgfsetroundcap%
\pgfsetroundjoin%
\pgfsetlinewidth{0.883034pt}%
\definecolor{currentstroke}{rgb}{0.192357,0.403199,0.555836}%
\pgfsetstrokecolor{currentstroke}%
\pgfsetdash{}{0pt}%
\pgfpathmoveto{\pgfqpoint{5.404285in}{5.152784in}}%
\pgfpathquadraticcurveto{\pgfqpoint{5.391747in}{5.152741in}}{\pgfqpoint{5.392870in}{5.152745in}}%
\pgfusepath{stroke}%
\end{pgfscope}%
\begin{pgfscope}%
\pgfpathrectangle{\pgfqpoint{3.985294in}{4.155455in}}{\pgfqpoint{2.279412in}{2.004545in}}%
\pgfusepath{clip}%
\pgfsetroundcap%
\pgfsetroundjoin%
\definecolor{currentfill}{rgb}{0.192357,0.403199,0.555836}%
\pgfsetfillcolor{currentfill}%
\pgfsetlinewidth{0.883034pt}%
\definecolor{currentstroke}{rgb}{0.192357,0.403199,0.555836}%
\pgfsetstrokecolor{currentstroke}%
\pgfsetdash{}{0pt}%
\pgfpathmoveto{\pgfqpoint{5.448521in}{5.125158in}}%
\pgfpathlineto{\pgfqpoint{5.392870in}{5.152745in}}%
\pgfpathlineto{\pgfqpoint{5.448330in}{5.180713in}}%
\pgfpathlineto{\pgfqpoint{5.448521in}{5.125158in}}%
\pgfpathlineto{\pgfqpoint{5.448521in}{5.125158in}}%
\pgfpathclose%
\pgfusepath{stroke,fill}%
\end{pgfscope}%
\begin{pgfscope}%
\pgfpathrectangle{\pgfqpoint{3.985294in}{4.155455in}}{\pgfqpoint{2.279412in}{2.004545in}}%
\pgfusepath{clip}%
\pgfsetroundcap%
\pgfsetroundjoin%
\pgfsetlinewidth{0.672362pt}%
\definecolor{currentstroke}{rgb}{0.248629,0.278775,0.534556}%
\pgfsetstrokecolor{currentstroke}%
\pgfsetdash{}{0pt}%
\pgfpathmoveto{\pgfqpoint{5.495607in}{5.203794in}}%
\pgfpathquadraticcurveto{\pgfqpoint{5.483071in}{5.203613in}}{\pgfqpoint{5.480935in}{5.203582in}}%
\pgfusepath{stroke}%
\end{pgfscope}%
\begin{pgfscope}%
\pgfpathrectangle{\pgfqpoint{3.985294in}{4.155455in}}{\pgfqpoint{2.279412in}{2.004545in}}%
\pgfusepath{clip}%
\pgfsetroundcap%
\pgfsetroundjoin%
\definecolor{currentfill}{rgb}{0.248629,0.278775,0.534556}%
\pgfsetfillcolor{currentfill}%
\pgfsetlinewidth{0.672362pt}%
\definecolor{currentstroke}{rgb}{0.248629,0.278775,0.534556}%
\pgfsetstrokecolor{currentstroke}%
\pgfsetdash{}{0pt}%
\pgfpathmoveto{\pgfqpoint{5.536888in}{5.176612in}}%
\pgfpathlineto{\pgfqpoint{5.480935in}{5.203582in}}%
\pgfpathlineto{\pgfqpoint{5.536082in}{5.232162in}}%
\pgfpathlineto{\pgfqpoint{5.536888in}{5.176612in}}%
\pgfpathlineto{\pgfqpoint{5.536888in}{5.176612in}}%
\pgfpathclose%
\pgfusepath{stroke,fill}%
\end{pgfscope}%
\begin{pgfscope}%
\pgfpathrectangle{\pgfqpoint{3.985294in}{4.155455in}}{\pgfqpoint{2.279412in}{2.004545in}}%
\pgfusepath{clip}%
\pgfsetroundcap%
\pgfsetroundjoin%
\pgfsetlinewidth{0.625824pt}%
\definecolor{currentstroke}{rgb}{0.260571,0.246922,0.522828}%
\pgfsetstrokecolor{currentstroke}%
\pgfsetdash{}{0pt}%
\pgfpathmoveto{\pgfqpoint{5.495853in}{5.283786in}}%
\pgfpathquadraticcurveto{\pgfqpoint{5.483326in}{5.283348in}}{\pgfqpoint{5.480474in}{5.283248in}}%
\pgfusepath{stroke}%
\end{pgfscope}%
\begin{pgfscope}%
\pgfpathrectangle{\pgfqpoint{3.985294in}{4.155455in}}{\pgfqpoint{2.279412in}{2.004545in}}%
\pgfusepath{clip}%
\pgfsetroundcap%
\pgfsetroundjoin%
\definecolor{currentfill}{rgb}{0.260571,0.246922,0.522828}%
\pgfsetfillcolor{currentfill}%
\pgfsetlinewidth{0.625824pt}%
\definecolor{currentstroke}{rgb}{0.260571,0.246922,0.522828}%
\pgfsetstrokecolor{currentstroke}%
\pgfsetdash{}{0pt}%
\pgfpathmoveto{\pgfqpoint{5.536966in}{5.257428in}}%
\pgfpathlineto{\pgfqpoint{5.480474in}{5.283248in}}%
\pgfpathlineto{\pgfqpoint{5.535025in}{5.312950in}}%
\pgfpathlineto{\pgfqpoint{5.536966in}{5.257428in}}%
\pgfpathlineto{\pgfqpoint{5.536966in}{5.257428in}}%
\pgfpathclose%
\pgfusepath{stroke,fill}%
\end{pgfscope}%
\begin{pgfscope}%
\pgfpathrectangle{\pgfqpoint{3.985294in}{4.155455in}}{\pgfqpoint{2.279412in}{2.004545in}}%
\pgfusepath{clip}%
\pgfsetroundcap%
\pgfsetroundjoin%
\pgfsetlinewidth{0.391386pt}%
\definecolor{currentstroke}{rgb}{0.280894,0.078907,0.402329}%
\pgfsetstrokecolor{currentstroke}%
\pgfsetdash{}{0pt}%
\pgfpathmoveto{\pgfqpoint{5.696559in}{5.329208in}}%
\pgfpathquadraticcurveto{\pgfqpoint{5.684027in}{5.328869in}}{\pgfqpoint{5.677549in}{5.328693in}}%
\pgfusepath{stroke}%
\end{pgfscope}%
\begin{pgfscope}%
\pgfpathrectangle{\pgfqpoint{3.985294in}{4.155455in}}{\pgfqpoint{2.279412in}{2.004545in}}%
\pgfusepath{clip}%
\pgfsetroundcap%
\pgfsetroundjoin%
\definecolor{currentfill}{rgb}{0.280894,0.078907,0.402329}%
\pgfsetfillcolor{currentfill}%
\pgfsetlinewidth{0.391386pt}%
\definecolor{currentstroke}{rgb}{0.280894,0.078907,0.402329}%
\pgfsetstrokecolor{currentstroke}%
\pgfsetdash{}{0pt}%
\pgfpathmoveto{\pgfqpoint{5.733835in}{5.302428in}}%
\pgfpathlineto{\pgfqpoint{5.677549in}{5.328693in}}%
\pgfpathlineto{\pgfqpoint{5.732333in}{5.357963in}}%
\pgfpathlineto{\pgfqpoint{5.733835in}{5.302428in}}%
\pgfpathlineto{\pgfqpoint{5.733835in}{5.302428in}}%
\pgfpathclose%
\pgfusepath{stroke,fill}%
\end{pgfscope}%
\begin{pgfscope}%
\pgfpathrectangle{\pgfqpoint{3.985294in}{4.155455in}}{\pgfqpoint{2.279412in}{2.004545in}}%
\pgfusepath{clip}%
\pgfsetroundcap%
\pgfsetroundjoin%
\pgfsetlinewidth{0.697437pt}%
\definecolor{currentstroke}{rgb}{0.243113,0.292092,0.538516}%
\pgfsetstrokecolor{currentstroke}%
\pgfsetdash{}{0pt}%
\pgfpathmoveto{\pgfqpoint{5.251608in}{4.883480in}}%
\pgfpathquadraticcurveto{\pgfqpoint{5.239257in}{4.885361in}}{\pgfqpoint{5.237572in}{4.885618in}}%
\pgfusepath{stroke}%
\end{pgfscope}%
\begin{pgfscope}%
\pgfpathrectangle{\pgfqpoint{3.985294in}{4.155455in}}{\pgfqpoint{2.279412in}{2.004545in}}%
\pgfusepath{clip}%
\pgfsetroundcap%
\pgfsetroundjoin%
\definecolor{currentfill}{rgb}{0.243113,0.292092,0.538516}%
\pgfsetfillcolor{currentfill}%
\pgfsetlinewidth{0.697437pt}%
\definecolor{currentstroke}{rgb}{0.243113,0.292092,0.538516}%
\pgfsetstrokecolor{currentstroke}%
\pgfsetdash{}{0pt}%
\pgfpathmoveto{\pgfqpoint{5.288312in}{4.849792in}}%
\pgfpathlineto{\pgfqpoint{5.237572in}{4.885618in}}%
\pgfpathlineto{\pgfqpoint{5.296676in}{4.904715in}}%
\pgfpathlineto{\pgfqpoint{5.288312in}{4.849792in}}%
\pgfpathlineto{\pgfqpoint{5.288312in}{4.849792in}}%
\pgfpathclose%
\pgfusepath{stroke,fill}%
\end{pgfscope}%
\begin{pgfscope}%
\pgfpathrectangle{\pgfqpoint{3.985294in}{4.155455in}}{\pgfqpoint{2.279412in}{2.004545in}}%
\pgfusepath{clip}%
\pgfsetroundcap%
\pgfsetroundjoin%
\pgfsetlinewidth{0.428818pt}%
\definecolor{currentstroke}{rgb}{0.282910,0.105393,0.426902}%
\pgfsetstrokecolor{currentstroke}%
\pgfsetdash{}{0pt}%
\pgfpathmoveto{\pgfqpoint{5.643781in}{4.984906in}}%
\pgfpathquadraticcurveto{\pgfqpoint{5.631254in}{4.985362in}}{\pgfqpoint{5.625356in}{4.985577in}}%
\pgfusepath{stroke}%
\end{pgfscope}%
\begin{pgfscope}%
\pgfpathrectangle{\pgfqpoint{3.985294in}{4.155455in}}{\pgfqpoint{2.279412in}{2.004545in}}%
\pgfusepath{clip}%
\pgfsetroundcap%
\pgfsetroundjoin%
\definecolor{currentfill}{rgb}{0.282910,0.105393,0.426902}%
\pgfsetfillcolor{currentfill}%
\pgfsetlinewidth{0.428818pt}%
\definecolor{currentstroke}{rgb}{0.282910,0.105393,0.426902}%
\pgfsetstrokecolor{currentstroke}%
\pgfsetdash{}{0pt}%
\pgfpathmoveto{\pgfqpoint{5.679863in}{4.955795in}}%
\pgfpathlineto{\pgfqpoint{5.625356in}{4.985577in}}%
\pgfpathlineto{\pgfqpoint{5.681886in}{5.011313in}}%
\pgfpathlineto{\pgfqpoint{5.679863in}{4.955795in}}%
\pgfpathlineto{\pgfqpoint{5.679863in}{4.955795in}}%
\pgfpathclose%
\pgfusepath{stroke,fill}%
\end{pgfscope}%
\begin{pgfscope}%
\pgfpathrectangle{\pgfqpoint{3.985294in}{4.155455in}}{\pgfqpoint{2.279412in}{2.004545in}}%
\pgfusepath{clip}%
\pgfsetroundcap%
\pgfsetroundjoin%
\pgfsetlinewidth{0.793142pt}%
\definecolor{currentstroke}{rgb}{0.216210,0.351535,0.550627}%
\pgfsetstrokecolor{currentstroke}%
\pgfsetdash{}{0pt}%
\pgfpathmoveto{\pgfqpoint{5.443058in}{5.114826in}}%
\pgfpathquadraticcurveto{\pgfqpoint{5.430521in}{5.114984in}}{\pgfqpoint{5.430254in}{5.114987in}}%
\pgfusepath{stroke}%
\end{pgfscope}%
\begin{pgfscope}%
\pgfpathrectangle{\pgfqpoint{3.985294in}{4.155455in}}{\pgfqpoint{2.279412in}{2.004545in}}%
\pgfusepath{clip}%
\pgfsetroundcap%
\pgfsetroundjoin%
\definecolor{currentfill}{rgb}{0.216210,0.351535,0.550627}%
\pgfsetfillcolor{currentfill}%
\pgfsetlinewidth{0.793142pt}%
\definecolor{currentstroke}{rgb}{0.216210,0.351535,0.550627}%
\pgfsetstrokecolor{currentstroke}%
\pgfsetdash{}{0pt}%
\pgfpathmoveto{\pgfqpoint{5.485457in}{5.086514in}}%
\pgfpathlineto{\pgfqpoint{5.430254in}{5.114987in}}%
\pgfpathlineto{\pgfqpoint{5.486153in}{5.142066in}}%
\pgfpathlineto{\pgfqpoint{5.485457in}{5.086514in}}%
\pgfpathlineto{\pgfqpoint{5.485457in}{5.086514in}}%
\pgfpathclose%
\pgfusepath{stroke,fill}%
\end{pgfscope}%
\begin{pgfscope}%
\pgfpathrectangle{\pgfqpoint{3.985294in}{4.155455in}}{\pgfqpoint{2.279412in}{2.004545in}}%
\pgfusepath{clip}%
\pgfsetroundcap%
\pgfsetroundjoin%
\pgfsetlinewidth{0.510461pt}%
\definecolor{currentstroke}{rgb}{0.280255,0.165693,0.476498}%
\pgfsetstrokecolor{currentstroke}%
\pgfsetdash{}{0pt}%
\pgfpathmoveto{\pgfqpoint{5.593583in}{5.244794in}}%
\pgfpathquadraticcurveto{\pgfqpoint{5.581049in}{5.244540in}}{\pgfqpoint{5.576410in}{5.244446in}}%
\pgfusepath{stroke}%
\end{pgfscope}%
\begin{pgfscope}%
\pgfpathrectangle{\pgfqpoint{3.985294in}{4.155455in}}{\pgfqpoint{2.279412in}{2.004545in}}%
\pgfusepath{clip}%
\pgfsetroundcap%
\pgfsetroundjoin%
\definecolor{currentfill}{rgb}{0.280255,0.165693,0.476498}%
\pgfsetfillcolor{currentfill}%
\pgfsetlinewidth{0.510461pt}%
\definecolor{currentstroke}{rgb}{0.280255,0.165693,0.476498}%
\pgfsetstrokecolor{currentstroke}%
\pgfsetdash{}{0pt}%
\pgfpathmoveto{\pgfqpoint{5.632517in}{5.217800in}}%
\pgfpathlineto{\pgfqpoint{5.576410in}{5.244446in}}%
\pgfpathlineto{\pgfqpoint{5.631391in}{5.273344in}}%
\pgfpathlineto{\pgfqpoint{5.632517in}{5.217800in}}%
\pgfpathlineto{\pgfqpoint{5.632517in}{5.217800in}}%
\pgfpathclose%
\pgfusepath{stroke,fill}%
\end{pgfscope}%
\begin{pgfscope}%
\pgfpathrectangle{\pgfqpoint{3.985294in}{4.155455in}}{\pgfqpoint{2.279412in}{2.004545in}}%
\pgfusepath{clip}%
\pgfsetroundcap%
\pgfsetroundjoin%
\pgfsetlinewidth{0.805512pt}%
\definecolor{currentstroke}{rgb}{0.212395,0.359683,0.551710}%
\pgfsetstrokecolor{currentstroke}%
\pgfsetdash{}{0pt}%
\pgfpathmoveto{\pgfqpoint{5.243914in}{5.347625in}}%
\pgfpathquadraticcurveto{\pgfqpoint{5.231508in}{5.346037in}}{\pgfqpoint{5.231463in}{5.346031in}}%
\pgfusepath{stroke}%
\end{pgfscope}%
\begin{pgfscope}%
\pgfpathrectangle{\pgfqpoint{3.985294in}{4.155455in}}{\pgfqpoint{2.279412in}{2.004545in}}%
\pgfusepath{clip}%
\pgfsetroundcap%
\pgfsetroundjoin%
\definecolor{currentfill}{rgb}{0.212395,0.359683,0.551710}%
\pgfsetfillcolor{currentfill}%
\pgfsetlinewidth{0.805512pt}%
\definecolor{currentstroke}{rgb}{0.212395,0.359683,0.551710}%
\pgfsetstrokecolor{currentstroke}%
\pgfsetdash{}{0pt}%
\pgfpathmoveto{\pgfqpoint{5.290097in}{5.325536in}}%
\pgfpathlineto{\pgfqpoint{5.231463in}{5.346031in}}%
\pgfpathlineto{\pgfqpoint{5.283039in}{5.380642in}}%
\pgfpathlineto{\pgfqpoint{5.290097in}{5.325536in}}%
\pgfpathlineto{\pgfqpoint{5.290097in}{5.325536in}}%
\pgfpathclose%
\pgfusepath{stroke,fill}%
\end{pgfscope}%
\begin{pgfscope}%
\pgfpathrectangle{\pgfqpoint{3.985294in}{4.155455in}}{\pgfqpoint{2.279412in}{2.004545in}}%
\pgfusepath{clip}%
\pgfsetroundcap%
\pgfsetroundjoin%
\pgfsetlinewidth{0.524295pt}%
\definecolor{currentstroke}{rgb}{0.278826,0.175490,0.483397}%
\pgfsetstrokecolor{currentstroke}%
\pgfsetdash{}{0pt}%
\pgfpathmoveto{\pgfqpoint{5.493655in}{5.412834in}}%
\pgfpathquadraticcurveto{\pgfqpoint{5.481149in}{5.412046in}}{\pgfqpoint{5.476739in}{5.411768in}}%
\pgfusepath{stroke}%
\end{pgfscope}%
\begin{pgfscope}%
\pgfpathrectangle{\pgfqpoint{3.985294in}{4.155455in}}{\pgfqpoint{2.279412in}{2.004545in}}%
\pgfusepath{clip}%
\pgfsetroundcap%
\pgfsetroundjoin%
\definecolor{currentfill}{rgb}{0.278826,0.175490,0.483397}%
\pgfsetfillcolor{currentfill}%
\pgfsetlinewidth{0.524295pt}%
\definecolor{currentstroke}{rgb}{0.278826,0.175490,0.483397}%
\pgfsetstrokecolor{currentstroke}%
\pgfsetdash{}{0pt}%
\pgfpathmoveto{\pgfqpoint{5.533932in}{5.387539in}}%
\pgfpathlineto{\pgfqpoint{5.476739in}{5.411768in}}%
\pgfpathlineto{\pgfqpoint{5.530438in}{5.442985in}}%
\pgfpathlineto{\pgfqpoint{5.533932in}{5.387539in}}%
\pgfpathlineto{\pgfqpoint{5.533932in}{5.387539in}}%
\pgfpathclose%
\pgfusepath{stroke,fill}%
\end{pgfscope}%
\begin{pgfscope}%
\pgfpathrectangle{\pgfqpoint{3.985294in}{4.155455in}}{\pgfqpoint{2.279412in}{2.004545in}}%
\pgfusepath{clip}%
\pgfsetroundcap%
\pgfsetroundjoin%
\pgfsetlinewidth{0.496276pt}%
\definecolor{currentstroke}{rgb}{0.281412,0.155834,0.469201}%
\pgfsetstrokecolor{currentstroke}%
\pgfsetdash{}{0pt}%
\pgfpathmoveto{\pgfqpoint{5.442796in}{4.817000in}}%
\pgfpathquadraticcurveto{\pgfqpoint{5.430341in}{4.818271in}}{\pgfqpoint{5.425525in}{4.818762in}}%
\pgfusepath{stroke}%
\end{pgfscope}%
\begin{pgfscope}%
\pgfpathrectangle{\pgfqpoint{3.985294in}{4.155455in}}{\pgfqpoint{2.279412in}{2.004545in}}%
\pgfusepath{clip}%
\pgfsetroundcap%
\pgfsetroundjoin%
\definecolor{currentfill}{rgb}{0.281412,0.155834,0.469201}%
\pgfsetfillcolor{currentfill}%
\pgfsetlinewidth{0.496276pt}%
\definecolor{currentstroke}{rgb}{0.281412,0.155834,0.469201}%
\pgfsetstrokecolor{currentstroke}%
\pgfsetdash{}{0pt}%
\pgfpathmoveto{\pgfqpoint{5.477975in}{4.785490in}}%
\pgfpathlineto{\pgfqpoint{5.425525in}{4.818762in}}%
\pgfpathlineto{\pgfqpoint{5.483612in}{4.840759in}}%
\pgfpathlineto{\pgfqpoint{5.477975in}{4.785490in}}%
\pgfpathlineto{\pgfqpoint{5.477975in}{4.785490in}}%
\pgfpathclose%
\pgfusepath{stroke,fill}%
\end{pgfscope}%
\begin{pgfscope}%
\pgfpathrectangle{\pgfqpoint{3.985294in}{4.155455in}}{\pgfqpoint{2.279412in}{2.004545in}}%
\pgfusepath{clip}%
\pgfsetroundcap%
\pgfsetroundjoin%
\pgfsetlinewidth{0.396765pt}%
\definecolor{currentstroke}{rgb}{0.280894,0.078907,0.402329}%
\pgfsetstrokecolor{currentstroke}%
\pgfsetdash{}{0pt}%
\pgfpathmoveto{\pgfqpoint{5.642739in}{4.894810in}}%
\pgfpathquadraticcurveto{\pgfqpoint{5.630219in}{4.895393in}}{\pgfqpoint{5.623831in}{4.895691in}}%
\pgfusepath{stroke}%
\end{pgfscope}%
\begin{pgfscope}%
\pgfpathrectangle{\pgfqpoint{3.985294in}{4.155455in}}{\pgfqpoint{2.279412in}{2.004545in}}%
\pgfusepath{clip}%
\pgfsetroundcap%
\pgfsetroundjoin%
\definecolor{currentfill}{rgb}{0.280894,0.078907,0.402329}%
\pgfsetfillcolor{currentfill}%
\pgfsetlinewidth{0.396765pt}%
\definecolor{currentstroke}{rgb}{0.280894,0.078907,0.402329}%
\pgfsetstrokecolor{currentstroke}%
\pgfsetdash{}{0pt}%
\pgfpathmoveto{\pgfqpoint{5.678034in}{4.865360in}}%
\pgfpathlineto{\pgfqpoint{5.623831in}{4.895691in}}%
\pgfpathlineto{\pgfqpoint{5.680618in}{4.920855in}}%
\pgfpathlineto{\pgfqpoint{5.678034in}{4.865360in}}%
\pgfpathlineto{\pgfqpoint{5.678034in}{4.865360in}}%
\pgfpathclose%
\pgfusepath{stroke,fill}%
\end{pgfscope}%
\begin{pgfscope}%
\pgfpathrectangle{\pgfqpoint{3.985294in}{4.155455in}}{\pgfqpoint{2.279412in}{2.004545in}}%
\pgfusepath{clip}%
\pgfsetroundcap%
\pgfsetroundjoin%
\pgfsetlinewidth{0.414712pt}%
\definecolor{currentstroke}{rgb}{0.282327,0.094955,0.417331}%
\pgfsetstrokecolor{currentstroke}%
\pgfsetdash{}{0pt}%
\pgfpathmoveto{\pgfqpoint{5.642686in}{4.939834in}}%
\pgfpathquadraticcurveto{\pgfqpoint{5.630164in}{4.940387in}}{\pgfqpoint{5.624051in}{4.940656in}}%
\pgfusepath{stroke}%
\end{pgfscope}%
\begin{pgfscope}%
\pgfpathrectangle{\pgfqpoint{3.985294in}{4.155455in}}{\pgfqpoint{2.279412in}{2.004545in}}%
\pgfusepath{clip}%
\pgfsetroundcap%
\pgfsetroundjoin%
\definecolor{currentfill}{rgb}{0.282327,0.094955,0.417331}%
\pgfsetfillcolor{currentfill}%
\pgfsetlinewidth{0.414712pt}%
\definecolor{currentstroke}{rgb}{0.282327,0.094955,0.417331}%
\pgfsetstrokecolor{currentstroke}%
\pgfsetdash{}{0pt}%
\pgfpathmoveto{\pgfqpoint{5.678328in}{4.910456in}}%
\pgfpathlineto{\pgfqpoint{5.624051in}{4.940656in}}%
\pgfpathlineto{\pgfqpoint{5.680778in}{4.965958in}}%
\pgfpathlineto{\pgfqpoint{5.678328in}{4.910456in}}%
\pgfpathlineto{\pgfqpoint{5.678328in}{4.910456in}}%
\pgfpathclose%
\pgfusepath{stroke,fill}%
\end{pgfscope}%
\begin{pgfscope}%
\pgfpathrectangle{\pgfqpoint{3.985294in}{4.155455in}}{\pgfqpoint{2.279412in}{2.004545in}}%
\pgfusepath{clip}%
\pgfsetroundcap%
\pgfsetroundjoin%
\pgfsetlinewidth{0.365977pt}%
\definecolor{currentstroke}{rgb}{0.277941,0.056324,0.381191}%
\pgfsetstrokecolor{currentstroke}%
\pgfsetdash{}{0pt}%
\pgfpathmoveto{\pgfqpoint{5.692929in}{5.466155in}}%
\pgfpathquadraticcurveto{\pgfqpoint{5.680420in}{5.465406in}}{\pgfqpoint{5.673563in}{5.464995in}}%
\pgfusepath{stroke}%
\end{pgfscope}%
\begin{pgfscope}%
\pgfpathrectangle{\pgfqpoint{3.985294in}{4.155455in}}{\pgfqpoint{2.279412in}{2.004545in}}%
\pgfusepath{clip}%
\pgfsetroundcap%
\pgfsetroundjoin%
\definecolor{currentfill}{rgb}{0.277941,0.056324,0.381191}%
\pgfsetfillcolor{currentfill}%
\pgfsetlinewidth{0.365977pt}%
\definecolor{currentstroke}{rgb}{0.277941,0.056324,0.381191}%
\pgfsetstrokecolor{currentstroke}%
\pgfsetdash{}{0pt}%
\pgfpathmoveto{\pgfqpoint{5.730680in}{5.440589in}}%
\pgfpathlineto{\pgfqpoint{5.673563in}{5.464995in}}%
\pgfpathlineto{\pgfqpoint{5.727358in}{5.496045in}}%
\pgfpathlineto{\pgfqpoint{5.730680in}{5.440589in}}%
\pgfpathlineto{\pgfqpoint{5.730680in}{5.440589in}}%
\pgfpathclose%
\pgfusepath{stroke,fill}%
\end{pgfscope}%
\begin{pgfscope}%
\pgfpathrectangle{\pgfqpoint{3.985294in}{4.155455in}}{\pgfqpoint{2.279412in}{2.004545in}}%
\pgfusepath{clip}%
\pgfsetroundcap%
\pgfsetroundjoin%
\pgfsetlinewidth{0.606916pt}%
\definecolor{currentstroke}{rgb}{0.265145,0.232956,0.516599}%
\pgfsetstrokecolor{currentstroke}%
\pgfsetdash{}{0pt}%
\pgfpathmoveto{\pgfqpoint{5.244882in}{5.465614in}}%
\pgfpathquadraticcurveto{\pgfqpoint{5.232614in}{5.463342in}}{\pgfqpoint{5.229579in}{5.462780in}}%
\pgfusepath{stroke}%
\end{pgfscope}%
\begin{pgfscope}%
\pgfpathrectangle{\pgfqpoint{3.985294in}{4.155455in}}{\pgfqpoint{2.279412in}{2.004545in}}%
\pgfusepath{clip}%
\pgfsetroundcap%
\pgfsetroundjoin%
\definecolor{currentfill}{rgb}{0.265145,0.232956,0.516599}%
\pgfsetfillcolor{currentfill}%
\pgfsetlinewidth{0.606916pt}%
\definecolor{currentstroke}{rgb}{0.265145,0.232956,0.516599}%
\pgfsetstrokecolor{currentstroke}%
\pgfsetdash{}{0pt}%
\pgfpathmoveto{\pgfqpoint{5.289264in}{5.445582in}}%
\pgfpathlineto{\pgfqpoint{5.229579in}{5.462780in}}%
\pgfpathlineto{\pgfqpoint{5.279148in}{5.500209in}}%
\pgfpathlineto{\pgfqpoint{5.289264in}{5.445582in}}%
\pgfpathlineto{\pgfqpoint{5.289264in}{5.445582in}}%
\pgfpathclose%
\pgfusepath{stroke,fill}%
\end{pgfscope}%
\begin{pgfscope}%
\pgfpathrectangle{\pgfqpoint{3.985294in}{4.155455in}}{\pgfqpoint{2.279412in}{2.004545in}}%
\pgfusepath{clip}%
\pgfsetroundcap%
\pgfsetroundjoin%
\pgfsetlinewidth{0.471024pt}%
\definecolor{currentstroke}{rgb}{0.282884,0.135920,0.453427}%
\pgfsetstrokecolor{currentstroke}%
\pgfsetdash{}{0pt}%
\pgfpathmoveto{\pgfqpoint{5.393499in}{5.531458in}}%
\pgfpathquadraticcurveto{\pgfqpoint{5.381106in}{5.529803in}}{\pgfqpoint{5.375935in}{5.529113in}}%
\pgfusepath{stroke}%
\end{pgfscope}%
\begin{pgfscope}%
\pgfpathrectangle{\pgfqpoint{3.985294in}{4.155455in}}{\pgfqpoint{2.279412in}{2.004545in}}%
\pgfusepath{clip}%
\pgfsetroundcap%
\pgfsetroundjoin%
\definecolor{currentfill}{rgb}{0.282884,0.135920,0.453427}%
\pgfsetfillcolor{currentfill}%
\pgfsetlinewidth{0.471024pt}%
\definecolor{currentstroke}{rgb}{0.282884,0.135920,0.453427}%
\pgfsetstrokecolor{currentstroke}%
\pgfsetdash{}{0pt}%
\pgfpathmoveto{\pgfqpoint{5.434678in}{5.508932in}}%
\pgfpathlineto{\pgfqpoint{5.375935in}{5.529113in}}%
\pgfpathlineto{\pgfqpoint{5.427326in}{5.563999in}}%
\pgfpathlineto{\pgfqpoint{5.434678in}{5.508932in}}%
\pgfpathlineto{\pgfqpoint{5.434678in}{5.508932in}}%
\pgfpathclose%
\pgfusepath{stroke,fill}%
\end{pgfscope}%
\begin{pgfscope}%
\pgfpathrectangle{\pgfqpoint{3.985294in}{4.155455in}}{\pgfqpoint{2.279412in}{2.004545in}}%
\pgfusepath{clip}%
\pgfsetroundcap%
\pgfsetroundjoin%
\pgfsetlinewidth{0.473964pt}%
\definecolor{currentstroke}{rgb}{0.282623,0.140926,0.457517}%
\pgfsetstrokecolor{currentstroke}%
\pgfsetdash{}{0pt}%
\pgfpathmoveto{\pgfqpoint{5.244836in}{4.764224in}}%
\pgfpathquadraticcurveto{\pgfqpoint{5.232764in}{4.767155in}}{\pgfqpoint{5.227818in}{4.768356in}}%
\pgfusepath{stroke}%
\end{pgfscope}%
\begin{pgfscope}%
\pgfpathrectangle{\pgfqpoint{3.985294in}{4.155455in}}{\pgfqpoint{2.279412in}{2.004545in}}%
\pgfusepath{clip}%
\pgfsetroundcap%
\pgfsetroundjoin%
\definecolor{currentfill}{rgb}{0.282623,0.140926,0.457517}%
\pgfsetfillcolor{currentfill}%
\pgfsetlinewidth{0.473964pt}%
\definecolor{currentstroke}{rgb}{0.282623,0.140926,0.457517}%
\pgfsetstrokecolor{currentstroke}%
\pgfsetdash{}{0pt}%
\pgfpathmoveto{\pgfqpoint{5.275250in}{4.728253in}}%
\pgfpathlineto{\pgfqpoint{5.227818in}{4.768356in}}%
\pgfpathlineto{\pgfqpoint{5.288360in}{4.782240in}}%
\pgfpathlineto{\pgfqpoint{5.275250in}{4.728253in}}%
\pgfpathlineto{\pgfqpoint{5.275250in}{4.728253in}}%
\pgfpathclose%
\pgfusepath{stroke,fill}%
\end{pgfscope}%
\begin{pgfscope}%
\pgfpathrectangle{\pgfqpoint{3.985294in}{4.155455in}}{\pgfqpoint{2.279412in}{2.004545in}}%
\pgfusepath{clip}%
\pgfsetroundcap%
\pgfsetroundjoin%
\pgfsetlinewidth{0.427521pt}%
\definecolor{currentstroke}{rgb}{0.282910,0.105393,0.426902}%
\pgfsetstrokecolor{currentstroke}%
\pgfsetdash{}{0pt}%
\pgfpathmoveto{\pgfqpoint{5.154122in}{5.605350in}}%
\pgfpathquadraticcurveto{\pgfqpoint{5.142920in}{5.600504in}}{\pgfqpoint{5.137787in}{5.598284in}}%
\pgfusepath{stroke}%
\end{pgfscope}%
\begin{pgfscope}%
\pgfpathrectangle{\pgfqpoint{3.985294in}{4.155455in}}{\pgfqpoint{2.279412in}{2.004545in}}%
\pgfusepath{clip}%
\pgfsetroundcap%
\pgfsetroundjoin%
\definecolor{currentfill}{rgb}{0.282910,0.105393,0.426902}%
\pgfsetfillcolor{currentfill}%
\pgfsetlinewidth{0.427521pt}%
\definecolor{currentstroke}{rgb}{0.282910,0.105393,0.426902}%
\pgfsetstrokecolor{currentstroke}%
\pgfsetdash{}{0pt}%
\pgfpathmoveto{\pgfqpoint{5.199805in}{5.594846in}}%
\pgfpathlineto{\pgfqpoint{5.137787in}{5.598284in}}%
\pgfpathlineto{\pgfqpoint{5.177749in}{5.645835in}}%
\pgfpathlineto{\pgfqpoint{5.199805in}{5.594846in}}%
\pgfpathlineto{\pgfqpoint{5.199805in}{5.594846in}}%
\pgfpathclose%
\pgfusepath{stroke,fill}%
\end{pgfscope}%
\begin{pgfscope}%
\pgfpathrectangle{\pgfqpoint{3.985294in}{4.155455in}}{\pgfqpoint{2.279412in}{2.004545in}}%
\pgfusepath{clip}%
\pgfsetroundcap%
\pgfsetroundjoin%
\pgfsetlinewidth{0.377136pt}%
\definecolor{currentstroke}{rgb}{0.279566,0.067836,0.391917}%
\pgfsetstrokecolor{currentstroke}%
\pgfsetdash{}{0pt}%
\pgfpathmoveto{\pgfqpoint{5.336571in}{5.723646in}}%
\pgfpathquadraticcurveto{\pgfqpoint{5.324469in}{5.720815in}}{\pgfqpoint{5.318048in}{5.719312in}}%
\pgfusepath{stroke}%
\end{pgfscope}%
\begin{pgfscope}%
\pgfpathrectangle{\pgfqpoint{3.985294in}{4.155455in}}{\pgfqpoint{2.279412in}{2.004545in}}%
\pgfusepath{clip}%
\pgfsetroundcap%
\pgfsetroundjoin%
\definecolor{currentfill}{rgb}{0.279566,0.067836,0.391917}%
\pgfsetfillcolor{currentfill}%
\pgfsetlinewidth{0.377136pt}%
\definecolor{currentstroke}{rgb}{0.279566,0.067836,0.391917}%
\pgfsetstrokecolor{currentstroke}%
\pgfsetdash{}{0pt}%
\pgfpathmoveto{\pgfqpoint{5.378471in}{5.704922in}}%
\pgfpathlineto{\pgfqpoint{5.318048in}{5.719312in}}%
\pgfpathlineto{\pgfqpoint{5.365814in}{5.759016in}}%
\pgfpathlineto{\pgfqpoint{5.378471in}{5.704922in}}%
\pgfpathlineto{\pgfqpoint{5.378471in}{5.704922in}}%
\pgfpathclose%
\pgfusepath{stroke,fill}%
\end{pgfscope}%
\begin{pgfscope}%
\pgfpathrectangle{\pgfqpoint{3.985294in}{4.155455in}}{\pgfqpoint{2.279412in}{2.004545in}}%
\pgfusepath{clip}%
\pgfsetroundcap%
\pgfsetroundjoin%
\pgfsetlinewidth{0.395621pt}%
\definecolor{currentstroke}{rgb}{0.280894,0.078907,0.402329}%
\pgfsetstrokecolor{currentstroke}%
\pgfsetdash{}{0pt}%
\pgfpathmoveto{\pgfqpoint{5.181825in}{4.634231in}}%
\pgfpathquadraticcurveto{\pgfqpoint{5.170697in}{4.639256in}}{\pgfqpoint{5.165146in}{4.641762in}}%
\pgfusepath{stroke}%
\end{pgfscope}%
\begin{pgfscope}%
\pgfpathrectangle{\pgfqpoint{3.985294in}{4.155455in}}{\pgfqpoint{2.279412in}{2.004545in}}%
\pgfusepath{clip}%
\pgfsetroundcap%
\pgfsetroundjoin%
\definecolor{currentfill}{rgb}{0.280894,0.078907,0.402329}%
\pgfsetfillcolor{currentfill}%
\pgfsetlinewidth{0.395621pt}%
\definecolor{currentstroke}{rgb}{0.280894,0.078907,0.402329}%
\pgfsetstrokecolor{currentstroke}%
\pgfsetdash{}{0pt}%
\pgfpathmoveto{\pgfqpoint{5.204348in}{4.593583in}}%
\pgfpathlineto{\pgfqpoint{5.165146in}{4.641762in}}%
\pgfpathlineto{\pgfqpoint{5.227210in}{4.644216in}}%
\pgfpathlineto{\pgfqpoint{5.204348in}{4.593583in}}%
\pgfpathlineto{\pgfqpoint{5.204348in}{4.593583in}}%
\pgfpathclose%
\pgfusepath{stroke,fill}%
\end{pgfscope}%
\begin{pgfscope}%
\pgfpathrectangle{\pgfqpoint{3.985294in}{4.155455in}}{\pgfqpoint{2.279412in}{2.004545in}}%
\pgfusepath{clip}%
\pgfsetroundcap%
\pgfsetroundjoin%
\pgfsetlinewidth{0.355842pt}%
\definecolor{currentstroke}{rgb}{0.276022,0.044167,0.370164}%
\pgfsetstrokecolor{currentstroke}%
\pgfsetdash{}{0pt}%
\pgfpathmoveto{\pgfqpoint{5.478280in}{4.593489in}}%
\pgfpathquadraticcurveto{\pgfqpoint{5.465894in}{4.595183in}}{\pgfqpoint{5.458963in}{4.596132in}}%
\pgfusepath{stroke}%
\end{pgfscope}%
\begin{pgfscope}%
\pgfpathrectangle{\pgfqpoint{3.985294in}{4.155455in}}{\pgfqpoint{2.279412in}{2.004545in}}%
\pgfusepath{clip}%
\pgfsetroundcap%
\pgfsetroundjoin%
\definecolor{currentfill}{rgb}{0.276022,0.044167,0.370164}%
\pgfsetfillcolor{currentfill}%
\pgfsetlinewidth{0.355842pt}%
\definecolor{currentstroke}{rgb}{0.276022,0.044167,0.370164}%
\pgfsetstrokecolor{currentstroke}%
\pgfsetdash{}{0pt}%
\pgfpathmoveto{\pgfqpoint{5.510240in}{4.561079in}}%
\pgfpathlineto{\pgfqpoint{5.458963in}{4.596132in}}%
\pgfpathlineto{\pgfqpoint{5.517771in}{4.616122in}}%
\pgfpathlineto{\pgfqpoint{5.510240in}{4.561079in}}%
\pgfpathlineto{\pgfqpoint{5.510240in}{4.561079in}}%
\pgfpathclose%
\pgfusepath{stroke,fill}%
\end{pgfscope}%
\begin{pgfscope}%
\pgfpathrectangle{\pgfqpoint{3.985294in}{4.155455in}}{\pgfqpoint{2.279412in}{2.004545in}}%
\pgfusepath{clip}%
\pgfsetroundcap%
\pgfsetroundjoin%
\pgfsetlinewidth{0.361167pt}%
\definecolor{currentstroke}{rgb}{0.277018,0.050344,0.375715}%
\pgfsetstrokecolor{currentstroke}%
\pgfsetdash{}{0pt}%
\pgfpathmoveto{\pgfqpoint{5.640508in}{4.753907in}}%
\pgfpathquadraticcurveto{\pgfqpoint{5.628015in}{4.754827in}}{\pgfqpoint{5.621095in}{4.755336in}}%
\pgfusepath{stroke}%
\end{pgfscope}%
\begin{pgfscope}%
\pgfpathrectangle{\pgfqpoint{3.985294in}{4.155455in}}{\pgfqpoint{2.279412in}{2.004545in}}%
\pgfusepath{clip}%
\pgfsetroundcap%
\pgfsetroundjoin%
\definecolor{currentfill}{rgb}{0.277018,0.050344,0.375715}%
\pgfsetfillcolor{currentfill}%
\pgfsetlinewidth{0.361167pt}%
\definecolor{currentstroke}{rgb}{0.277018,0.050344,0.375715}%
\pgfsetstrokecolor{currentstroke}%
\pgfsetdash{}{0pt}%
\pgfpathmoveto{\pgfqpoint{5.674462in}{4.723556in}}%
\pgfpathlineto{\pgfqpoint{5.621095in}{4.755336in}}%
\pgfpathlineto{\pgfqpoint{5.678539in}{4.778962in}}%
\pgfpathlineto{\pgfqpoint{5.674462in}{4.723556in}}%
\pgfpathlineto{\pgfqpoint{5.674462in}{4.723556in}}%
\pgfpathclose%
\pgfusepath{stroke,fill}%
\end{pgfscope}%
\begin{pgfscope}%
\pgfpathrectangle{\pgfqpoint{3.985294in}{4.155455in}}{\pgfqpoint{2.279412in}{2.004545in}}%
\pgfusepath{clip}%
\pgfsetroundcap%
\pgfsetroundjoin%
\pgfsetlinewidth{0.377199pt}%
\definecolor{currentstroke}{rgb}{0.279566,0.067836,0.391917}%
\pgfsetstrokecolor{currentstroke}%
\pgfsetdash{}{0pt}%
\pgfpathmoveto{\pgfqpoint{5.428506in}{4.644391in}}%
\pgfpathquadraticcurveto{\pgfqpoint{5.416168in}{4.646347in}}{\pgfqpoint{5.409594in}{4.647390in}}%
\pgfusepath{stroke}%
\end{pgfscope}%
\begin{pgfscope}%
\pgfpathrectangle{\pgfqpoint{3.985294in}{4.155455in}}{\pgfqpoint{2.279412in}{2.004545in}}%
\pgfusepath{clip}%
\pgfsetroundcap%
\pgfsetroundjoin%
\definecolor{currentfill}{rgb}{0.279566,0.067836,0.391917}%
\pgfsetfillcolor{currentfill}%
\pgfsetlinewidth{0.377199pt}%
\definecolor{currentstroke}{rgb}{0.279566,0.067836,0.391917}%
\pgfsetstrokecolor{currentstroke}%
\pgfsetdash{}{0pt}%
\pgfpathmoveto{\pgfqpoint{5.460114in}{4.611255in}}%
\pgfpathlineto{\pgfqpoint{5.409594in}{4.647390in}}%
\pgfpathlineto{\pgfqpoint{5.468814in}{4.666125in}}%
\pgfpathlineto{\pgfqpoint{5.460114in}{4.611255in}}%
\pgfpathlineto{\pgfqpoint{5.460114in}{4.611255in}}%
\pgfpathclose%
\pgfusepath{stroke,fill}%
\end{pgfscope}%
\begin{pgfscope}%
\pgfpathrectangle{\pgfqpoint{3.985294in}{4.155455in}}{\pgfqpoint{2.279412in}{2.004545in}}%
\pgfusepath{clip}%
\pgfsetroundcap%
\pgfsetroundjoin%
\pgfsetlinewidth{0.331704pt}%
\definecolor{currentstroke}{rgb}{0.272594,0.025563,0.353093}%
\pgfsetstrokecolor{currentstroke}%
\pgfsetdash{}{0pt}%
\pgfpathmoveto{\pgfqpoint{5.683584in}{4.644145in}}%
\pgfpathquadraticcurveto{\pgfqpoint{5.671094in}{4.645010in}}{\pgfqpoint{5.663724in}{4.645520in}}%
\pgfusepath{stroke}%
\end{pgfscope}%
\begin{pgfscope}%
\pgfpathrectangle{\pgfqpoint{3.985294in}{4.155455in}}{\pgfqpoint{2.279412in}{2.004545in}}%
\pgfusepath{clip}%
\pgfsetroundcap%
\pgfsetroundjoin%
\definecolor{currentfill}{rgb}{0.272594,0.025563,0.353093}%
\pgfsetfillcolor{currentfill}%
\pgfsetlinewidth{0.331704pt}%
\definecolor{currentstroke}{rgb}{0.272594,0.025563,0.353093}%
\pgfsetstrokecolor{currentstroke}%
\pgfsetdash{}{0pt}%
\pgfpathmoveto{\pgfqpoint{5.717228in}{4.613971in}}%
\pgfpathlineto{\pgfqpoint{5.663724in}{4.645520in}}%
\pgfpathlineto{\pgfqpoint{5.721066in}{4.669394in}}%
\pgfpathlineto{\pgfqpoint{5.717228in}{4.613971in}}%
\pgfpathlineto{\pgfqpoint{5.717228in}{4.613971in}}%
\pgfpathclose%
\pgfusepath{stroke,fill}%
\end{pgfscope}%
\begin{pgfscope}%
\pgfpathrectangle{\pgfqpoint{3.985294in}{4.155455in}}{\pgfqpoint{2.279412in}{2.004545in}}%
\pgfusepath{clip}%
\pgfsetroundcap%
\pgfsetroundjoin%
\pgfsetlinewidth{0.397800pt}%
\definecolor{currentstroke}{rgb}{0.281446,0.084320,0.407414}%
\pgfsetstrokecolor{currentstroke}%
\pgfsetdash{}{0pt}%
\pgfpathmoveto{\pgfqpoint{5.086501in}{5.615132in}}%
\pgfpathquadraticcurveto{\pgfqpoint{5.076463in}{5.608647in}}{\pgfqpoint{5.071593in}{5.605500in}}%
\pgfusepath{stroke}%
\end{pgfscope}%
\begin{pgfscope}%
\pgfpathrectangle{\pgfqpoint{3.985294in}{4.155455in}}{\pgfqpoint{2.279412in}{2.004545in}}%
\pgfusepath{clip}%
\pgfsetroundcap%
\pgfsetroundjoin%
\definecolor{currentfill}{rgb}{0.281446,0.084320,0.407414}%
\pgfsetfillcolor{currentfill}%
\pgfsetlinewidth{0.397800pt}%
\definecolor{currentstroke}{rgb}{0.281446,0.084320,0.407414}%
\pgfsetstrokecolor{currentstroke}%
\pgfsetdash{}{0pt}%
\pgfpathmoveto{\pgfqpoint{5.133331in}{5.612318in}}%
\pgfpathlineto{\pgfqpoint{5.071593in}{5.605500in}}%
\pgfpathlineto{\pgfqpoint{5.103182in}{5.658981in}}%
\pgfpathlineto{\pgfqpoint{5.133331in}{5.612318in}}%
\pgfpathlineto{\pgfqpoint{5.133331in}{5.612318in}}%
\pgfpathclose%
\pgfusepath{stroke,fill}%
\end{pgfscope}%
\begin{pgfscope}%
\pgfpathrectangle{\pgfqpoint{3.985294in}{4.155455in}}{\pgfqpoint{2.279412in}{2.004545in}}%
\pgfusepath{clip}%
\pgfsetroundcap%
\pgfsetroundjoin%
\pgfsetlinewidth{0.463304pt}%
\definecolor{currentstroke}{rgb}{0.283072,0.130895,0.449241}%
\pgfsetstrokecolor{currentstroke}%
\pgfsetdash{}{0pt}%
\pgfpathmoveto{\pgfqpoint{5.133239in}{4.736253in}}%
\pgfpathquadraticcurveto{\pgfqpoint{5.122081in}{4.741211in}}{\pgfqpoint{5.117473in}{4.743258in}}%
\pgfusepath{stroke}%
\end{pgfscope}%
\begin{pgfscope}%
\pgfpathrectangle{\pgfqpoint{3.985294in}{4.155455in}}{\pgfqpoint{2.279412in}{2.004545in}}%
\pgfusepath{clip}%
\pgfsetroundcap%
\pgfsetroundjoin%
\definecolor{currentfill}{rgb}{0.283072,0.130895,0.449241}%
\pgfsetfillcolor{currentfill}%
\pgfsetlinewidth{0.463304pt}%
\definecolor{currentstroke}{rgb}{0.283072,0.130895,0.449241}%
\pgfsetstrokecolor{currentstroke}%
\pgfsetdash{}{0pt}%
\pgfpathmoveto{\pgfqpoint{5.156963in}{4.695315in}}%
\pgfpathlineto{\pgfqpoint{5.117473in}{4.743258in}}%
\pgfpathlineto{\pgfqpoint{5.179522in}{4.746084in}}%
\pgfpathlineto{\pgfqpoint{5.156963in}{4.695315in}}%
\pgfpathlineto{\pgfqpoint{5.156963in}{4.695315in}}%
\pgfpathclose%
\pgfusepath{stroke,fill}%
\end{pgfscope}%
\begin{pgfscope}%
\pgfpathrectangle{\pgfqpoint{3.985294in}{4.155455in}}{\pgfqpoint{2.279412in}{2.004545in}}%
\pgfusepath{clip}%
\pgfsetroundcap%
\pgfsetroundjoin%
\pgfsetlinewidth{0.439890pt}%
\definecolor{currentstroke}{rgb}{0.283197,0.115680,0.436115}%
\pgfsetstrokecolor{currentstroke}%
\pgfsetdash{}{0pt}%
\pgfpathmoveto{\pgfqpoint{5.388787in}{5.584325in}}%
\pgfpathquadraticcurveto{\pgfqpoint{5.376433in}{5.582457in}}{\pgfqpoint{5.370807in}{5.581606in}}%
\pgfusepath{stroke}%
\end{pgfscope}%
\begin{pgfscope}%
\pgfpathrectangle{\pgfqpoint{3.985294in}{4.155455in}}{\pgfqpoint{2.279412in}{2.004545in}}%
\pgfusepath{clip}%
\pgfsetroundcap%
\pgfsetroundjoin%
\definecolor{currentfill}{rgb}{0.283197,0.115680,0.436115}%
\pgfsetfillcolor{currentfill}%
\pgfsetlinewidth{0.439890pt}%
\definecolor{currentstroke}{rgb}{0.283197,0.115680,0.436115}%
\pgfsetstrokecolor{currentstroke}%
\pgfsetdash{}{0pt}%
\pgfpathmoveto{\pgfqpoint{5.429891in}{5.562447in}}%
\pgfpathlineto{\pgfqpoint{5.370807in}{5.581606in}}%
\pgfpathlineto{\pgfqpoint{5.421585in}{5.617378in}}%
\pgfpathlineto{\pgfqpoint{5.429891in}{5.562447in}}%
\pgfpathlineto{\pgfqpoint{5.429891in}{5.562447in}}%
\pgfpathclose%
\pgfusepath{stroke,fill}%
\end{pgfscope}%
\begin{pgfscope}%
\pgfpathrectangle{\pgfqpoint{3.985294in}{4.155455in}}{\pgfqpoint{2.279412in}{2.004545in}}%
\pgfusepath{clip}%
\pgfsetroundcap%
\pgfsetroundjoin%
\pgfsetlinewidth{0.643711pt}%
\definecolor{currentstroke}{rgb}{0.257322,0.256130,0.526563}%
\pgfsetstrokecolor{currentstroke}%
\pgfsetdash{}{0pt}%
\pgfpathmoveto{\pgfqpoint{4.889760in}{4.889934in}}%
\pgfpathquadraticcurveto{\pgfqpoint{4.890009in}{4.896783in}}{\pgfqpoint{4.889896in}{4.893681in}}%
\pgfusepath{stroke}%
\end{pgfscope}%
\begin{pgfscope}%
\pgfpathrectangle{\pgfqpoint{3.985294in}{4.155455in}}{\pgfqpoint{2.279412in}{2.004545in}}%
\pgfusepath{clip}%
\pgfsetroundcap%
\pgfsetroundjoin%
\definecolor{currentfill}{rgb}{0.257322,0.256130,0.526563}%
\pgfsetfillcolor{currentfill}%
\pgfsetlinewidth{0.643711pt}%
\definecolor{currentstroke}{rgb}{0.257322,0.256130,0.526563}%
\pgfsetstrokecolor{currentstroke}%
\pgfsetdash{}{0pt}%
\pgfpathmoveto{\pgfqpoint{4.860119in}{4.839171in}}%
\pgfpathlineto{\pgfqpoint{4.889896in}{4.893681in}}%
\pgfpathlineto{\pgfqpoint{4.915638in}{4.837154in}}%
\pgfpathlineto{\pgfqpoint{4.860119in}{4.839171in}}%
\pgfpathlineto{\pgfqpoint{4.860119in}{4.839171in}}%
\pgfpathclose%
\pgfusepath{stroke,fill}%
\end{pgfscope}%
\begin{pgfscope}%
\pgfpathrectangle{\pgfqpoint{3.985294in}{4.155455in}}{\pgfqpoint{2.279412in}{2.004545in}}%
\pgfusepath{clip}%
\pgfsetroundcap%
\pgfsetroundjoin%
\pgfsetlinewidth{0.858018pt}%
\definecolor{currentstroke}{rgb}{0.197636,0.391528,0.554969}%
\pgfsetstrokecolor{currentstroke}%
\pgfsetdash{}{0pt}%
\pgfpathmoveto{\pgfqpoint{5.181062in}{4.949577in}}%
\pgfpathquadraticcurveto{\pgfqpoint{5.168742in}{4.951609in}}{\pgfqpoint{5.169517in}{4.951481in}}%
\pgfusepath{stroke}%
\end{pgfscope}%
\begin{pgfscope}%
\pgfpathrectangle{\pgfqpoint{3.985294in}{4.155455in}}{\pgfqpoint{2.279412in}{2.004545in}}%
\pgfusepath{clip}%
\pgfsetroundcap%
\pgfsetroundjoin%
\definecolor{currentfill}{rgb}{0.197636,0.391528,0.554969}%
\pgfsetfillcolor{currentfill}%
\pgfsetlinewidth{0.858018pt}%
\definecolor{currentstroke}{rgb}{0.197636,0.391528,0.554969}%
\pgfsetstrokecolor{currentstroke}%
\pgfsetdash{}{0pt}%
\pgfpathmoveto{\pgfqpoint{5.219814in}{4.915035in}}%
\pgfpathlineto{\pgfqpoint{5.169517in}{4.951481in}}%
\pgfpathlineto{\pgfqpoint{5.228852in}{4.969850in}}%
\pgfpathlineto{\pgfqpoint{5.219814in}{4.915035in}}%
\pgfpathlineto{\pgfqpoint{5.219814in}{4.915035in}}%
\pgfpathclose%
\pgfusepath{stroke,fill}%
\end{pgfscope}%
\begin{pgfscope}%
\pgfpathrectangle{\pgfqpoint{3.985294in}{4.155455in}}{\pgfqpoint{2.279412in}{2.004545in}}%
\pgfusepath{clip}%
\pgfsetroundcap%
\pgfsetroundjoin%
\pgfsetlinewidth{1.026264pt}%
\definecolor{currentstroke}{rgb}{0.159194,0.482237,0.558073}%
\pgfsetstrokecolor{currentstroke}%
\pgfsetdash{}{0pt}%
\pgfpathmoveto{\pgfqpoint{5.057597in}{5.009929in}}%
\pgfpathquadraticcurveto{\pgfqpoint{5.045486in}{5.012768in}}{\pgfqpoint{5.048833in}{5.011983in}}%
\pgfusepath{stroke}%
\end{pgfscope}%
\begin{pgfscope}%
\pgfpathrectangle{\pgfqpoint{3.985294in}{4.155455in}}{\pgfqpoint{2.279412in}{2.004545in}}%
\pgfusepath{clip}%
\pgfsetroundcap%
\pgfsetroundjoin%
\definecolor{currentfill}{rgb}{0.159194,0.482237,0.558073}%
\pgfsetfillcolor{currentfill}%
\pgfsetlinewidth{1.026264pt}%
\definecolor{currentstroke}{rgb}{0.159194,0.482237,0.558073}%
\pgfsetstrokecolor{currentstroke}%
\pgfsetdash{}{0pt}%
\pgfpathmoveto{\pgfqpoint{5.096584in}{4.972260in}}%
\pgfpathlineto{\pgfqpoint{5.048833in}{5.011983in}}%
\pgfpathlineto{\pgfqpoint{5.109262in}{5.026350in}}%
\pgfpathlineto{\pgfqpoint{5.096584in}{4.972260in}}%
\pgfpathlineto{\pgfqpoint{5.096584in}{4.972260in}}%
\pgfpathclose%
\pgfusepath{stroke,fill}%
\end{pgfscope}%
\begin{pgfscope}%
\pgfpathrectangle{\pgfqpoint{3.985294in}{4.155455in}}{\pgfqpoint{2.279412in}{2.004545in}}%
\pgfusepath{clip}%
\pgfsetbuttcap%
\pgfsetroundjoin%
\pgfsetlinewidth{1.505625pt}%
\definecolor{currentstroke}{rgb}{0.000000,0.000000,0.000000}%
\pgfsetstrokecolor{currentstroke}%
\pgfsetdash{}{0pt}%
\pgfpathmoveto{\pgfqpoint{4.778678in}{4.494895in}}%
\pgfpathlineto{\pgfqpoint{4.778678in}{5.820559in}}%
\pgfusepath{stroke}%
\end{pgfscope}%
\begin{pgfscope}%
\pgfpathrectangle{\pgfqpoint{3.985294in}{4.155455in}}{\pgfqpoint{2.279412in}{2.004545in}}%
\pgfusepath{clip}%
\pgfsetbuttcap%
\pgfsetroundjoin%
\pgfsetlinewidth{1.505625pt}%
\definecolor{currentstroke}{rgb}{0.000000,0.000000,0.000000}%
\pgfsetstrokecolor{currentstroke}%
\pgfsetdash{}{0pt}%
\pgfpathmoveto{\pgfqpoint{5.927204in}{4.494895in}}%
\pgfpathlineto{\pgfqpoint{5.927204in}{5.820559in}}%
\pgfusepath{stroke}%
\end{pgfscope}%
\begin{pgfscope}%
\pgfsetrectcap%
\pgfsetmiterjoin%
\pgfsetlinewidth{0.803000pt}%
\definecolor{currentstroke}{rgb}{0.000000,0.000000,0.000000}%
\pgfsetstrokecolor{currentstroke}%
\pgfsetdash{}{0pt}%
\pgfpathmoveto{\pgfqpoint{3.985294in}{4.155455in}}%
\pgfpathlineto{\pgfqpoint{3.985294in}{6.160000in}}%
\pgfusepath{stroke}%
\end{pgfscope}%
\begin{pgfscope}%
\pgfsetrectcap%
\pgfsetmiterjoin%
\pgfsetlinewidth{0.803000pt}%
\definecolor{currentstroke}{rgb}{0.000000,0.000000,0.000000}%
\pgfsetstrokecolor{currentstroke}%
\pgfsetdash{}{0pt}%
\pgfpathmoveto{\pgfqpoint{6.264706in}{4.155455in}}%
\pgfpathlineto{\pgfqpoint{6.264706in}{6.160000in}}%
\pgfusepath{stroke}%
\end{pgfscope}%
\begin{pgfscope}%
\pgfsetrectcap%
\pgfsetmiterjoin%
\pgfsetlinewidth{0.803000pt}%
\definecolor{currentstroke}{rgb}{0.000000,0.000000,0.000000}%
\pgfsetstrokecolor{currentstroke}%
\pgfsetdash{}{0pt}%
\pgfpathmoveto{\pgfqpoint{3.985294in}{4.155455in}}%
\pgfpathlineto{\pgfqpoint{6.264706in}{4.155455in}}%
\pgfusepath{stroke}%
\end{pgfscope}%
\begin{pgfscope}%
\pgfsetrectcap%
\pgfsetmiterjoin%
\pgfsetlinewidth{0.803000pt}%
\definecolor{currentstroke}{rgb}{0.000000,0.000000,0.000000}%
\pgfsetstrokecolor{currentstroke}%
\pgfsetdash{}{0pt}%
\pgfpathmoveto{\pgfqpoint{3.985294in}{6.160000in}}%
\pgfpathlineto{\pgfqpoint{6.264706in}{6.160000in}}%
\pgfusepath{stroke}%
\end{pgfscope}%
\begin{pgfscope}%
\definecolor{textcolor}{rgb}{0.000000,0.000000,0.000000}%
\pgfsetstrokecolor{textcolor}%
\pgfsetfillcolor{textcolor}%
\pgftext[x=5.125000in,y=6.243333in,,base]{\color{textcolor}\sffamily\fontsize{12.000000}{14.400000}\selectfont b)}%
\end{pgfscope}%
\begin{pgfscope}%
\pgfsetbuttcap%
\pgfsetmiterjoin%
\definecolor{currentfill}{rgb}{1.000000,1.000000,1.000000}%
\pgfsetfillcolor{currentfill}%
\pgfsetlinewidth{0.000000pt}%
\definecolor{currentstroke}{rgb}{0.000000,0.000000,0.000000}%
\pgfsetstrokecolor{currentstroke}%
\pgfsetstrokeopacity{0.000000}%
\pgfsetdash{}{0pt}%
\pgfpathmoveto{\pgfqpoint{6.720588in}{4.155455in}}%
\pgfpathlineto{\pgfqpoint{9.000000in}{4.155455in}}%
\pgfpathlineto{\pgfqpoint{9.000000in}{6.160000in}}%
\pgfpathlineto{\pgfqpoint{6.720588in}{6.160000in}}%
\pgfpathlineto{\pgfqpoint{6.720588in}{4.155455in}}%
\pgfpathclose%
\pgfusepath{fill}%
\end{pgfscope}%
\begin{pgfscope}%
\pgfpathrectangle{\pgfqpoint{6.720588in}{4.155455in}}{\pgfqpoint{2.279412in}{2.004545in}}%
\pgfusepath{clip}%
\pgfsys@transformcm{2.291667}{0.000000}{0.000000}{2.013889}{6.720588in}{4.155455in}%
\pgftext[left,bottom]{\includegraphics[interpolate=false,width=1.000000in,height=1.000000in]{q_series-img2.png}}%
\end{pgfscope}%
\begin{pgfscope}%
\pgfsetbuttcap%
\pgfsetroundjoin%
\definecolor{currentfill}{rgb}{0.000000,0.000000,0.000000}%
\pgfsetfillcolor{currentfill}%
\pgfsetlinewidth{0.803000pt}%
\definecolor{currentstroke}{rgb}{0.000000,0.000000,0.000000}%
\pgfsetstrokecolor{currentstroke}%
\pgfsetdash{}{0pt}%
\pgfsys@defobject{currentmarker}{\pgfqpoint{0.000000in}{-0.048611in}}{\pgfqpoint{0.000000in}{0.000000in}}{%
\pgfpathmoveto{\pgfqpoint{0.000000in}{0.000000in}}%
\pgfpathlineto{\pgfqpoint{0.000000in}{-0.048611in}}%
\pgfusepath{stroke,fill}%
}%
\begin{pgfscope}%
\pgfsys@transformshift{7.131130in}{4.155455in}%
\pgfsys@useobject{currentmarker}{}%
\end{pgfscope}%
\end{pgfscope}%
\begin{pgfscope}%
\pgfsetbuttcap%
\pgfsetroundjoin%
\definecolor{currentfill}{rgb}{0.000000,0.000000,0.000000}%
\pgfsetfillcolor{currentfill}%
\pgfsetlinewidth{0.803000pt}%
\definecolor{currentstroke}{rgb}{0.000000,0.000000,0.000000}%
\pgfsetstrokecolor{currentstroke}%
\pgfsetdash{}{0pt}%
\pgfsys@defobject{currentmarker}{\pgfqpoint{0.000000in}{-0.048611in}}{\pgfqpoint{0.000000in}{0.000000in}}{%
\pgfpathmoveto{\pgfqpoint{0.000000in}{0.000000in}}%
\pgfpathlineto{\pgfqpoint{0.000000in}{-0.048611in}}%
\pgfusepath{stroke,fill}%
}%
\begin{pgfscope}%
\pgfsys@transformshift{7.609683in}{4.155455in}%
\pgfsys@useobject{currentmarker}{}%
\end{pgfscope}%
\end{pgfscope}%
\begin{pgfscope}%
\pgfsetbuttcap%
\pgfsetroundjoin%
\definecolor{currentfill}{rgb}{0.000000,0.000000,0.000000}%
\pgfsetfillcolor{currentfill}%
\pgfsetlinewidth{0.803000pt}%
\definecolor{currentstroke}{rgb}{0.000000,0.000000,0.000000}%
\pgfsetstrokecolor{currentstroke}%
\pgfsetdash{}{0pt}%
\pgfsys@defobject{currentmarker}{\pgfqpoint{0.000000in}{-0.048611in}}{\pgfqpoint{0.000000in}{0.000000in}}{%
\pgfpathmoveto{\pgfqpoint{0.000000in}{0.000000in}}%
\pgfpathlineto{\pgfqpoint{0.000000in}{-0.048611in}}%
\pgfusepath{stroke,fill}%
}%
\begin{pgfscope}%
\pgfsys@transformshift{8.088235in}{4.155455in}%
\pgfsys@useobject{currentmarker}{}%
\end{pgfscope}%
\end{pgfscope}%
\begin{pgfscope}%
\pgfsetbuttcap%
\pgfsetroundjoin%
\definecolor{currentfill}{rgb}{0.000000,0.000000,0.000000}%
\pgfsetfillcolor{currentfill}%
\pgfsetlinewidth{0.803000pt}%
\definecolor{currentstroke}{rgb}{0.000000,0.000000,0.000000}%
\pgfsetstrokecolor{currentstroke}%
\pgfsetdash{}{0pt}%
\pgfsys@defobject{currentmarker}{\pgfqpoint{0.000000in}{-0.048611in}}{\pgfqpoint{0.000000in}{0.000000in}}{%
\pgfpathmoveto{\pgfqpoint{0.000000in}{0.000000in}}%
\pgfpathlineto{\pgfqpoint{0.000000in}{-0.048611in}}%
\pgfusepath{stroke,fill}%
}%
\begin{pgfscope}%
\pgfsys@transformshift{8.566788in}{4.155455in}%
\pgfsys@useobject{currentmarker}{}%
\end{pgfscope}%
\end{pgfscope}%
\begin{pgfscope}%
\definecolor{textcolor}{rgb}{0.000000,0.000000,0.000000}%
\pgfsetstrokecolor{textcolor}%
\pgfsetfillcolor{textcolor}%
\pgftext[x=7.860294in,y=4.099899in,,top]{\color{textcolor}\sffamily\fontsize{10.000000}{12.000000}\selectfont \(\displaystyle \zeta \, \mathrm{[\mu m]}\)}%
\end{pgfscope}%
\begin{pgfscope}%
\pgfsetbuttcap%
\pgfsetroundjoin%
\definecolor{currentfill}{rgb}{0.000000,0.000000,0.000000}%
\pgfsetfillcolor{currentfill}%
\pgfsetlinewidth{0.803000pt}%
\definecolor{currentstroke}{rgb}{0.000000,0.000000,0.000000}%
\pgfsetstrokecolor{currentstroke}%
\pgfsetdash{}{0pt}%
\pgfsys@defobject{currentmarker}{\pgfqpoint{-0.048611in}{0.000000in}}{\pgfqpoint{-0.000000in}{0.000000in}}{%
\pgfpathmoveto{\pgfqpoint{-0.000000in}{0.000000in}}%
\pgfpathlineto{\pgfqpoint{-0.048611in}{0.000000in}}%
\pgfusepath{stroke,fill}%
}%
\begin{pgfscope}%
\pgfsys@transformshift{6.720588in}{4.163479in}%
\pgfsys@useobject{currentmarker}{}%
\end{pgfscope}%
\end{pgfscope}%
\begin{pgfscope}%
\pgfsetbuttcap%
\pgfsetroundjoin%
\definecolor{currentfill}{rgb}{0.000000,0.000000,0.000000}%
\pgfsetfillcolor{currentfill}%
\pgfsetlinewidth{0.803000pt}%
\definecolor{currentstroke}{rgb}{0.000000,0.000000,0.000000}%
\pgfsetstrokecolor{currentstroke}%
\pgfsetdash{}{0pt}%
\pgfsys@defobject{currentmarker}{\pgfqpoint{-0.048611in}{0.000000in}}{\pgfqpoint{-0.000000in}{0.000000in}}{%
\pgfpathmoveto{\pgfqpoint{-0.000000in}{0.000000in}}%
\pgfpathlineto{\pgfqpoint{-0.048611in}{0.000000in}}%
\pgfusepath{stroke,fill}%
}%
\begin{pgfscope}%
\pgfsys@transformshift{6.720588in}{4.494895in}%
\pgfsys@useobject{currentmarker}{}%
\end{pgfscope}%
\end{pgfscope}%
\begin{pgfscope}%
\pgfsetbuttcap%
\pgfsetroundjoin%
\definecolor{currentfill}{rgb}{0.000000,0.000000,0.000000}%
\pgfsetfillcolor{currentfill}%
\pgfsetlinewidth{0.803000pt}%
\definecolor{currentstroke}{rgb}{0.000000,0.000000,0.000000}%
\pgfsetstrokecolor{currentstroke}%
\pgfsetdash{}{0pt}%
\pgfsys@defobject{currentmarker}{\pgfqpoint{-0.048611in}{0.000000in}}{\pgfqpoint{-0.000000in}{0.000000in}}{%
\pgfpathmoveto{\pgfqpoint{-0.000000in}{0.000000in}}%
\pgfpathlineto{\pgfqpoint{-0.048611in}{0.000000in}}%
\pgfusepath{stroke,fill}%
}%
\begin{pgfscope}%
\pgfsys@transformshift{6.720588in}{4.826311in}%
\pgfsys@useobject{currentmarker}{}%
\end{pgfscope}%
\end{pgfscope}%
\begin{pgfscope}%
\pgfsetbuttcap%
\pgfsetroundjoin%
\definecolor{currentfill}{rgb}{0.000000,0.000000,0.000000}%
\pgfsetfillcolor{currentfill}%
\pgfsetlinewidth{0.803000pt}%
\definecolor{currentstroke}{rgb}{0.000000,0.000000,0.000000}%
\pgfsetstrokecolor{currentstroke}%
\pgfsetdash{}{0pt}%
\pgfsys@defobject{currentmarker}{\pgfqpoint{-0.048611in}{0.000000in}}{\pgfqpoint{-0.000000in}{0.000000in}}{%
\pgfpathmoveto{\pgfqpoint{-0.000000in}{0.000000in}}%
\pgfpathlineto{\pgfqpoint{-0.048611in}{0.000000in}}%
\pgfusepath{stroke,fill}%
}%
\begin{pgfscope}%
\pgfsys@transformshift{6.720588in}{5.157727in}%
\pgfsys@useobject{currentmarker}{}%
\end{pgfscope}%
\end{pgfscope}%
\begin{pgfscope}%
\pgfsetbuttcap%
\pgfsetroundjoin%
\definecolor{currentfill}{rgb}{0.000000,0.000000,0.000000}%
\pgfsetfillcolor{currentfill}%
\pgfsetlinewidth{0.803000pt}%
\definecolor{currentstroke}{rgb}{0.000000,0.000000,0.000000}%
\pgfsetstrokecolor{currentstroke}%
\pgfsetdash{}{0pt}%
\pgfsys@defobject{currentmarker}{\pgfqpoint{-0.048611in}{0.000000in}}{\pgfqpoint{-0.000000in}{0.000000in}}{%
\pgfpathmoveto{\pgfqpoint{-0.000000in}{0.000000in}}%
\pgfpathlineto{\pgfqpoint{-0.048611in}{0.000000in}}%
\pgfusepath{stroke,fill}%
}%
\begin{pgfscope}%
\pgfsys@transformshift{6.720588in}{5.489143in}%
\pgfsys@useobject{currentmarker}{}%
\end{pgfscope}%
\end{pgfscope}%
\begin{pgfscope}%
\pgfsetbuttcap%
\pgfsetroundjoin%
\definecolor{currentfill}{rgb}{0.000000,0.000000,0.000000}%
\pgfsetfillcolor{currentfill}%
\pgfsetlinewidth{0.803000pt}%
\definecolor{currentstroke}{rgb}{0.000000,0.000000,0.000000}%
\pgfsetstrokecolor{currentstroke}%
\pgfsetdash{}{0pt}%
\pgfsys@defobject{currentmarker}{\pgfqpoint{-0.048611in}{0.000000in}}{\pgfqpoint{-0.000000in}{0.000000in}}{%
\pgfpathmoveto{\pgfqpoint{-0.000000in}{0.000000in}}%
\pgfpathlineto{\pgfqpoint{-0.048611in}{0.000000in}}%
\pgfusepath{stroke,fill}%
}%
\begin{pgfscope}%
\pgfsys@transformshift{6.720588in}{5.820559in}%
\pgfsys@useobject{currentmarker}{}%
\end{pgfscope}%
\end{pgfscope}%
\begin{pgfscope}%
\pgfsetbuttcap%
\pgfsetroundjoin%
\definecolor{currentfill}{rgb}{0.000000,0.000000,0.000000}%
\pgfsetfillcolor{currentfill}%
\pgfsetlinewidth{0.803000pt}%
\definecolor{currentstroke}{rgb}{0.000000,0.000000,0.000000}%
\pgfsetstrokecolor{currentstroke}%
\pgfsetdash{}{0pt}%
\pgfsys@defobject{currentmarker}{\pgfqpoint{-0.048611in}{0.000000in}}{\pgfqpoint{-0.000000in}{0.000000in}}{%
\pgfpathmoveto{\pgfqpoint{-0.000000in}{0.000000in}}%
\pgfpathlineto{\pgfqpoint{-0.048611in}{0.000000in}}%
\pgfusepath{stroke,fill}%
}%
\begin{pgfscope}%
\pgfsys@transformshift{6.720588in}{6.151975in}%
\pgfsys@useobject{currentmarker}{}%
\end{pgfscope}%
\end{pgfscope}%
\begin{pgfscope}%
\definecolor{textcolor}{rgb}{0.000000,0.000000,0.000000}%
\pgfsetstrokecolor{textcolor}%
\pgfsetfillcolor{textcolor}%
\pgftext[x=6.665033in,y=5.157727in,,bottom,rotate=90.000000]{\color{textcolor}\sffamily\fontsize{10.000000}{12.000000}\selectfont \(\displaystyle z \, \mathrm{[\mu m]}\)}%
\end{pgfscope}%
\begin{pgfscope}%
\pgfpathrectangle{\pgfqpoint{6.720588in}{4.155455in}}{\pgfqpoint{2.279412in}{2.004545in}}%
\pgfusepath{clip}%
\pgfsetbuttcap%
\pgfsetroundjoin%
\pgfsetlinewidth{0.320723pt}%
\definecolor{currentstroke}{rgb}{0.269944,0.014625,0.341379}%
\pgfsetstrokecolor{currentstroke}%
\pgfsetdash{}{0pt}%
\pgfpathmoveto{\pgfqpoint{8.732256in}{4.977300in}}%
\pgfpathlineto{\pgfqpoint{8.683755in}{4.977786in}}%
\pgfusepath{stroke}%
\end{pgfscope}%
\begin{pgfscope}%
\pgfpathrectangle{\pgfqpoint{6.720588in}{4.155455in}}{\pgfqpoint{2.279412in}{2.004545in}}%
\pgfusepath{clip}%
\pgfsetbuttcap%
\pgfsetroundjoin%
\pgfsetlinewidth{0.321799pt}%
\definecolor{currentstroke}{rgb}{0.271305,0.019942,0.347269}%
\pgfsetstrokecolor{currentstroke}%
\pgfsetdash{}{0pt}%
\pgfpathmoveto{\pgfqpoint{8.683755in}{4.977786in}}%
\pgfpathlineto{\pgfqpoint{8.633756in}{4.979149in}}%
\pgfusepath{stroke}%
\end{pgfscope}%
\begin{pgfscope}%
\pgfpathrectangle{\pgfqpoint{6.720588in}{4.155455in}}{\pgfqpoint{2.279412in}{2.004545in}}%
\pgfusepath{clip}%
\pgfsetbuttcap%
\pgfsetroundjoin%
\pgfsetlinewidth{0.323030pt}%
\definecolor{currentstroke}{rgb}{0.271305,0.019942,0.347269}%
\pgfsetstrokecolor{currentstroke}%
\pgfsetdash{}{0pt}%
\pgfpathmoveto{\pgfqpoint{8.633756in}{4.979149in}}%
\pgfpathlineto{\pgfqpoint{8.583829in}{4.981575in}}%
\pgfusepath{stroke}%
\end{pgfscope}%
\begin{pgfscope}%
\pgfpathrectangle{\pgfqpoint{6.720588in}{4.155455in}}{\pgfqpoint{2.279412in}{2.004545in}}%
\pgfusepath{clip}%
\pgfsetbuttcap%
\pgfsetroundjoin%
\pgfsetlinewidth{0.340082pt}%
\definecolor{currentstroke}{rgb}{0.273809,0.031497,0.358853}%
\pgfsetstrokecolor{currentstroke}%
\pgfsetdash{}{0pt}%
\pgfpathmoveto{\pgfqpoint{8.583829in}{4.981575in}}%
\pgfpathlineto{\pgfqpoint{8.533778in}{4.984069in}}%
\pgfusepath{stroke}%
\end{pgfscope}%
\begin{pgfscope}%
\pgfpathrectangle{\pgfqpoint{6.720588in}{4.155455in}}{\pgfqpoint{2.279412in}{2.004545in}}%
\pgfusepath{clip}%
\pgfsetbuttcap%
\pgfsetroundjoin%
\pgfsetlinewidth{0.342932pt}%
\definecolor{currentstroke}{rgb}{0.274952,0.037752,0.364543}%
\pgfsetstrokecolor{currentstroke}%
\pgfsetdash{}{0pt}%
\pgfpathmoveto{\pgfqpoint{8.533778in}{4.984069in}}%
\pgfpathlineto{\pgfqpoint{8.483672in}{4.985818in}}%
\pgfusepath{stroke}%
\end{pgfscope}%
\begin{pgfscope}%
\pgfpathrectangle{\pgfqpoint{6.720588in}{4.155455in}}{\pgfqpoint{2.279412in}{2.004545in}}%
\pgfusepath{clip}%
\pgfsetbuttcap%
\pgfsetroundjoin%
\pgfsetlinewidth{0.364065pt}%
\definecolor{currentstroke}{rgb}{0.277941,0.056324,0.381191}%
\pgfsetstrokecolor{currentstroke}%
\pgfsetdash{}{0pt}%
\pgfpathmoveto{\pgfqpoint{8.483672in}{4.985818in}}%
\pgfpathlineto{\pgfqpoint{8.433564in}{4.987502in}}%
\pgfusepath{stroke}%
\end{pgfscope}%
\begin{pgfscope}%
\pgfpathrectangle{\pgfqpoint{6.720588in}{4.155455in}}{\pgfqpoint{2.279412in}{2.004545in}}%
\pgfusepath{clip}%
\pgfsetbuttcap%
\pgfsetroundjoin%
\pgfsetlinewidth{0.385518pt}%
\definecolor{currentstroke}{rgb}{0.280267,0.073417,0.397163}%
\pgfsetstrokecolor{currentstroke}%
\pgfsetdash{}{0pt}%
\pgfpathmoveto{\pgfqpoint{8.433564in}{4.987502in}}%
\pgfpathlineto{\pgfqpoint{8.383452in}{4.989203in}}%
\pgfusepath{stroke}%
\end{pgfscope}%
\begin{pgfscope}%
\pgfpathrectangle{\pgfqpoint{6.720588in}{4.155455in}}{\pgfqpoint{2.279412in}{2.004545in}}%
\pgfusepath{clip}%
\pgfsetbuttcap%
\pgfsetroundjoin%
\pgfsetlinewidth{0.432187pt}%
\definecolor{currentstroke}{rgb}{0.283091,0.110553,0.431554}%
\pgfsetstrokecolor{currentstroke}%
\pgfsetdash{}{0pt}%
\pgfpathmoveto{\pgfqpoint{8.383452in}{4.989203in}}%
\pgfpathlineto{\pgfqpoint{8.333344in}{4.990992in}}%
\pgfusepath{stroke}%
\end{pgfscope}%
\begin{pgfscope}%
\pgfpathrectangle{\pgfqpoint{6.720588in}{4.155455in}}{\pgfqpoint{2.279412in}{2.004545in}}%
\pgfusepath{clip}%
\pgfsetbuttcap%
\pgfsetroundjoin%
\pgfsetlinewidth{0.472810pt}%
\definecolor{currentstroke}{rgb}{0.282623,0.140926,0.457517}%
\pgfsetstrokecolor{currentstroke}%
\pgfsetdash{}{0pt}%
\pgfpathmoveto{\pgfqpoint{8.333344in}{4.990992in}}%
\pgfpathlineto{\pgfqpoint{8.283238in}{4.992872in}}%
\pgfusepath{stroke}%
\end{pgfscope}%
\begin{pgfscope}%
\pgfpathrectangle{\pgfqpoint{6.720588in}{4.155455in}}{\pgfqpoint{2.279412in}{2.004545in}}%
\pgfusepath{clip}%
\pgfsetbuttcap%
\pgfsetroundjoin%
\pgfsetlinewidth{0.549242pt}%
\definecolor{currentstroke}{rgb}{0.275191,0.194905,0.496005}%
\pgfsetstrokecolor{currentstroke}%
\pgfsetdash{}{0pt}%
\pgfpathmoveto{\pgfqpoint{8.283238in}{4.992872in}}%
\pgfpathlineto{\pgfqpoint{8.233144in}{4.994989in}}%
\pgfusepath{stroke}%
\end{pgfscope}%
\begin{pgfscope}%
\pgfpathrectangle{\pgfqpoint{6.720588in}{4.155455in}}{\pgfqpoint{2.279412in}{2.004545in}}%
\pgfusepath{clip}%
\pgfsetbuttcap%
\pgfsetroundjoin%
\pgfsetlinewidth{0.593635pt}%
\definecolor{currentstroke}{rgb}{0.267968,0.223549,0.512008}%
\pgfsetstrokecolor{currentstroke}%
\pgfsetdash{}{0pt}%
\pgfpathmoveto{\pgfqpoint{8.233144in}{4.994989in}}%
\pgfpathlineto{\pgfqpoint{8.183069in}{4.997416in}}%
\pgfusepath{stroke}%
\end{pgfscope}%
\begin{pgfscope}%
\pgfpathrectangle{\pgfqpoint{6.720588in}{4.155455in}}{\pgfqpoint{2.279412in}{2.004545in}}%
\pgfusepath{clip}%
\pgfsetbuttcap%
\pgfsetroundjoin%
\pgfsetlinewidth{0.699427pt}%
\definecolor{currentstroke}{rgb}{0.241237,0.296485,0.539709}%
\pgfsetstrokecolor{currentstroke}%
\pgfsetdash{}{0pt}%
\pgfpathmoveto{\pgfqpoint{8.183069in}{4.997416in}}%
\pgfpathlineto{\pgfqpoint{8.133027in}{5.000317in}}%
\pgfusepath{stroke}%
\end{pgfscope}%
\begin{pgfscope}%
\pgfpathrectangle{\pgfqpoint{6.720588in}{4.155455in}}{\pgfqpoint{2.279412in}{2.004545in}}%
\pgfusepath{clip}%
\pgfsetbuttcap%
\pgfsetroundjoin%
\pgfsetlinewidth{0.807666pt}%
\definecolor{currentstroke}{rgb}{0.212395,0.359683,0.551710}%
\pgfsetstrokecolor{currentstroke}%
\pgfsetdash{}{0pt}%
\pgfpathmoveto{\pgfqpoint{8.133027in}{5.000317in}}%
\pgfpathlineto{\pgfqpoint{8.083044in}{5.003908in}}%
\pgfusepath{stroke}%
\end{pgfscope}%
\begin{pgfscope}%
\pgfpathrectangle{\pgfqpoint{6.720588in}{4.155455in}}{\pgfqpoint{2.279412in}{2.004545in}}%
\pgfusepath{clip}%
\pgfsetbuttcap%
\pgfsetroundjoin%
\pgfsetlinewidth{0.944019pt}%
\definecolor{currentstroke}{rgb}{0.177423,0.437527,0.557565}%
\pgfsetstrokecolor{currentstroke}%
\pgfsetdash{}{0pt}%
\pgfpathmoveto{\pgfqpoint{8.083044in}{5.003908in}}%
\pgfpathlineto{\pgfqpoint{8.033161in}{5.008423in}}%
\pgfusepath{stroke}%
\end{pgfscope}%
\begin{pgfscope}%
\pgfpathrectangle{\pgfqpoint{6.720588in}{4.155455in}}{\pgfqpoint{2.279412in}{2.004545in}}%
\pgfusepath{clip}%
\pgfsetbuttcap%
\pgfsetroundjoin%
\pgfsetlinewidth{1.130353pt}%
\definecolor{currentstroke}{rgb}{0.137770,0.537492,0.554906}%
\pgfsetstrokecolor{currentstroke}%
\pgfsetdash{}{0pt}%
\pgfpathmoveto{\pgfqpoint{8.033161in}{5.008423in}}%
\pgfpathlineto{\pgfqpoint{7.983451in}{5.014191in}}%
\pgfusepath{stroke}%
\end{pgfscope}%
\begin{pgfscope}%
\pgfpathrectangle{\pgfqpoint{6.720588in}{4.155455in}}{\pgfqpoint{2.279412in}{2.004545in}}%
\pgfusepath{clip}%
\pgfsetbuttcap%
\pgfsetroundjoin%
\pgfsetlinewidth{1.575947pt}%
\definecolor{currentstroke}{rgb}{0.311925,0.767822,0.415586}%
\pgfsetstrokecolor{currentstroke}%
\pgfsetdash{}{0pt}%
\pgfpathmoveto{\pgfqpoint{7.983451in}{5.014191in}}%
\pgfpathlineto{\pgfqpoint{7.933954in}{5.021264in}}%
\pgfusepath{stroke}%
\end{pgfscope}%
\begin{pgfscope}%
\pgfpathrectangle{\pgfqpoint{6.720588in}{4.155455in}}{\pgfqpoint{2.279412in}{2.004545in}}%
\pgfusepath{clip}%
\pgfsetbuttcap%
\pgfsetroundjoin%
\pgfsetlinewidth{1.777832pt}%
\definecolor{currentstroke}{rgb}{0.585678,0.846661,0.249897}%
\pgfsetstrokecolor{currentstroke}%
\pgfsetdash{}{0pt}%
\pgfpathmoveto{\pgfqpoint{7.933954in}{5.021264in}}%
\pgfpathlineto{\pgfqpoint{7.884639in}{5.029253in}}%
\pgfusepath{stroke}%
\end{pgfscope}%
\begin{pgfscope}%
\pgfpathrectangle{\pgfqpoint{6.720588in}{4.155455in}}{\pgfqpoint{2.279412in}{2.004545in}}%
\pgfusepath{clip}%
\pgfsetbuttcap%
\pgfsetroundjoin%
\pgfsetlinewidth{1.951114pt}%
\definecolor{currentstroke}{rgb}{0.845561,0.887322,0.099702}%
\pgfsetstrokecolor{currentstroke}%
\pgfsetdash{}{0pt}%
\pgfpathmoveto{\pgfqpoint{7.884639in}{5.029253in}}%
\pgfpathlineto{\pgfqpoint{7.835474in}{5.037929in}}%
\pgfusepath{stroke}%
\end{pgfscope}%
\begin{pgfscope}%
\pgfpathrectangle{\pgfqpoint{6.720588in}{4.155455in}}{\pgfqpoint{2.279412in}{2.004545in}}%
\pgfusepath{clip}%
\pgfsetbuttcap%
\pgfsetroundjoin%
\pgfsetlinewidth{2.010161pt}%
\definecolor{currentstroke}{rgb}{0.935904,0.898570,0.108131}%
\pgfsetstrokecolor{currentstroke}%
\pgfsetdash{}{0pt}%
\pgfpathmoveto{\pgfqpoint{7.835474in}{5.037929in}}%
\pgfpathlineto{\pgfqpoint{7.786508in}{5.047411in}}%
\pgfusepath{stroke}%
\end{pgfscope}%
\begin{pgfscope}%
\pgfpathrectangle{\pgfqpoint{6.720588in}{4.155455in}}{\pgfqpoint{2.279412in}{2.004545in}}%
\pgfusepath{clip}%
\pgfsetbuttcap%
\pgfsetroundjoin%
\pgfsetlinewidth{2.020870pt}%
\definecolor{currentstroke}{rgb}{0.945636,0.899815,0.112838}%
\pgfsetstrokecolor{currentstroke}%
\pgfsetdash{}{0pt}%
\pgfpathmoveto{\pgfqpoint{7.786508in}{5.047411in}}%
\pgfpathlineto{\pgfqpoint{7.737731in}{5.057626in}}%
\pgfusepath{stroke}%
\end{pgfscope}%
\begin{pgfscope}%
\pgfpathrectangle{\pgfqpoint{6.720588in}{4.155455in}}{\pgfqpoint{2.279412in}{2.004545in}}%
\pgfusepath{clip}%
\pgfsetbuttcap%
\pgfsetroundjoin%
\pgfsetlinewidth{2.085110pt}%
\definecolor{currentstroke}{rgb}{0.993248,0.906157,0.143936}%
\pgfsetstrokecolor{currentstroke}%
\pgfsetdash{}{0pt}%
\pgfpathmoveto{\pgfqpoint{7.737731in}{5.057626in}}%
\pgfpathlineto{\pgfqpoint{7.689066in}{5.068226in}}%
\pgfusepath{stroke}%
\end{pgfscope}%
\begin{pgfscope}%
\pgfpathrectangle{\pgfqpoint{6.720588in}{4.155455in}}{\pgfqpoint{2.279412in}{2.004545in}}%
\pgfusepath{clip}%
\pgfsetbuttcap%
\pgfsetroundjoin%
\pgfsetlinewidth{1.988105pt}%
\definecolor{currentstroke}{rgb}{0.896320,0.893616,0.096335}%
\pgfsetstrokecolor{currentstroke}%
\pgfsetdash{}{0pt}%
\pgfpathmoveto{\pgfqpoint{7.689066in}{5.068226in}}%
\pgfpathlineto{\pgfqpoint{7.640447in}{5.078985in}}%
\pgfusepath{stroke}%
\end{pgfscope}%
\begin{pgfscope}%
\pgfpathrectangle{\pgfqpoint{6.720588in}{4.155455in}}{\pgfqpoint{2.279412in}{2.004545in}}%
\pgfusepath{clip}%
\pgfsetbuttcap%
\pgfsetroundjoin%
\pgfsetlinewidth{2.129298pt}%
\definecolor{currentstroke}{rgb}{0.993248,0.906157,0.143936}%
\pgfsetstrokecolor{currentstroke}%
\pgfsetdash{}{0pt}%
\pgfpathmoveto{\pgfqpoint{7.640447in}{5.078985in}}%
\pgfpathlineto{\pgfqpoint{7.592004in}{5.090335in}}%
\pgfusepath{stroke}%
\end{pgfscope}%
\begin{pgfscope}%
\pgfpathrectangle{\pgfqpoint{6.720588in}{4.155455in}}{\pgfqpoint{2.279412in}{2.004545in}}%
\pgfusepath{clip}%
\pgfsetbuttcap%
\pgfsetroundjoin%
\pgfsetlinewidth{1.807633pt}%
\definecolor{currentstroke}{rgb}{0.626579,0.854645,0.223353}%
\pgfsetstrokecolor{currentstroke}%
\pgfsetdash{}{0pt}%
\pgfpathmoveto{\pgfqpoint{7.592004in}{5.090335in}}%
\pgfpathlineto{\pgfqpoint{7.543549in}{5.101600in}}%
\pgfusepath{stroke}%
\end{pgfscope}%
\begin{pgfscope}%
\pgfpathrectangle{\pgfqpoint{6.720588in}{4.155455in}}{\pgfqpoint{2.279412in}{2.004545in}}%
\pgfusepath{clip}%
\pgfsetbuttcap%
\pgfsetroundjoin%
\pgfsetlinewidth{1.833775pt}%
\definecolor{currentstroke}{rgb}{0.668054,0.861999,0.196293}%
\pgfsetstrokecolor{currentstroke}%
\pgfsetdash{}{0pt}%
\pgfpathmoveto{\pgfqpoint{7.543549in}{5.101600in}}%
\pgfpathlineto{\pgfqpoint{7.495109in}{5.112770in}}%
\pgfusepath{stroke}%
\end{pgfscope}%
\begin{pgfscope}%
\pgfpathrectangle{\pgfqpoint{6.720588in}{4.155455in}}{\pgfqpoint{2.279412in}{2.004545in}}%
\pgfusepath{clip}%
\pgfsetbuttcap%
\pgfsetroundjoin%
\pgfsetlinewidth{1.524042pt}%
\definecolor{currentstroke}{rgb}{0.259857,0.745492,0.444467}%
\pgfsetstrokecolor{currentstroke}%
\pgfsetdash{}{0pt}%
\pgfpathmoveto{\pgfqpoint{7.495109in}{5.112770in}}%
\pgfpathlineto{\pgfqpoint{7.446802in}{5.124220in}}%
\pgfusepath{stroke}%
\end{pgfscope}%
\begin{pgfscope}%
\pgfpathrectangle{\pgfqpoint{6.720588in}{4.155455in}}{\pgfqpoint{2.279412in}{2.004545in}}%
\pgfusepath{clip}%
\pgfsetbuttcap%
\pgfsetroundjoin%
\pgfsetlinewidth{1.380752pt}%
\definecolor{currentstroke}{rgb}{0.146616,0.673050,0.508936}%
\pgfsetstrokecolor{currentstroke}%
\pgfsetdash{}{0pt}%
\pgfpathmoveto{\pgfqpoint{7.446802in}{5.124220in}}%
\pgfpathlineto{\pgfqpoint{7.398396in}{5.134983in}}%
\pgfusepath{stroke}%
\end{pgfscope}%
\begin{pgfscope}%
\pgfpathrectangle{\pgfqpoint{6.720588in}{4.155455in}}{\pgfqpoint{2.279412in}{2.004545in}}%
\pgfusepath{clip}%
\pgfsetbuttcap%
\pgfsetroundjoin%
\pgfsetlinewidth{1.186528pt}%
\definecolor{currentstroke}{rgb}{0.126453,0.570633,0.549841}%
\pgfsetstrokecolor{currentstroke}%
\pgfsetdash{}{0pt}%
\pgfpathmoveto{\pgfqpoint{7.398396in}{5.134983in}}%
\pgfpathlineto{\pgfqpoint{7.349742in}{5.142316in}}%
\pgfusepath{stroke}%
\end{pgfscope}%
\begin{pgfscope}%
\pgfpathrectangle{\pgfqpoint{6.720588in}{4.155455in}}{\pgfqpoint{2.279412in}{2.004545in}}%
\pgfusepath{clip}%
\pgfsetbuttcap%
\pgfsetroundjoin%
\pgfsetlinewidth{0.311984pt}%
\definecolor{currentstroke}{rgb}{0.268510,0.009605,0.335427}%
\pgfsetstrokecolor{currentstroke}%
\pgfsetdash{}{0pt}%
\pgfpathmoveto{\pgfqpoint{8.732256in}{5.247941in}}%
\pgfpathlineto{\pgfqpoint{8.682175in}{5.247574in}}%
\pgfusepath{stroke}%
\end{pgfscope}%
\begin{pgfscope}%
\pgfpathrectangle{\pgfqpoint{6.720588in}{4.155455in}}{\pgfqpoint{2.279412in}{2.004545in}}%
\pgfusepath{clip}%
\pgfsetbuttcap%
\pgfsetroundjoin%
\pgfsetlinewidth{0.318798pt}%
\definecolor{currentstroke}{rgb}{0.269944,0.014625,0.341379}%
\pgfsetstrokecolor{currentstroke}%
\pgfsetdash{}{0pt}%
\pgfpathmoveto{\pgfqpoint{8.682175in}{5.247574in}}%
\pgfpathlineto{\pgfqpoint{8.632052in}{5.248322in}}%
\pgfusepath{stroke}%
\end{pgfscope}%
\begin{pgfscope}%
\pgfpathrectangle{\pgfqpoint{6.720588in}{4.155455in}}{\pgfqpoint{2.279412in}{2.004545in}}%
\pgfusepath{clip}%
\pgfsetbuttcap%
\pgfsetroundjoin%
\pgfsetlinewidth{0.330351pt}%
\definecolor{currentstroke}{rgb}{0.272594,0.025563,0.353093}%
\pgfsetstrokecolor{currentstroke}%
\pgfsetdash{}{0pt}%
\pgfpathmoveto{\pgfqpoint{8.632052in}{5.248322in}}%
\pgfpathlineto{\pgfqpoint{8.581920in}{5.248613in}}%
\pgfusepath{stroke}%
\end{pgfscope}%
\begin{pgfscope}%
\pgfpathrectangle{\pgfqpoint{6.720588in}{4.155455in}}{\pgfqpoint{2.279412in}{2.004545in}}%
\pgfusepath{clip}%
\pgfsetbuttcap%
\pgfsetroundjoin%
\pgfsetlinewidth{0.331845pt}%
\definecolor{currentstroke}{rgb}{0.272594,0.025563,0.353093}%
\pgfsetstrokecolor{currentstroke}%
\pgfsetdash{}{0pt}%
\pgfpathmoveto{\pgfqpoint{8.581920in}{5.248613in}}%
\pgfpathlineto{\pgfqpoint{8.531785in}{5.247899in}}%
\pgfusepath{stroke}%
\end{pgfscope}%
\begin{pgfscope}%
\pgfpathrectangle{\pgfqpoint{6.720588in}{4.155455in}}{\pgfqpoint{2.279412in}{2.004545in}}%
\pgfusepath{clip}%
\pgfsetbuttcap%
\pgfsetroundjoin%
\pgfsetlinewidth{0.347175pt}%
\definecolor{currentstroke}{rgb}{0.274952,0.037752,0.364543}%
\pgfsetstrokecolor{currentstroke}%
\pgfsetdash{}{0pt}%
\pgfpathmoveto{\pgfqpoint{8.531785in}{5.247899in}}%
\pgfpathlineto{\pgfqpoint{8.481642in}{5.247230in}}%
\pgfusepath{stroke}%
\end{pgfscope}%
\begin{pgfscope}%
\pgfpathrectangle{\pgfqpoint{6.720588in}{4.155455in}}{\pgfqpoint{2.279412in}{2.004545in}}%
\pgfusepath{clip}%
\pgfsetbuttcap%
\pgfsetroundjoin%
\pgfsetlinewidth{0.372784pt}%
\definecolor{currentstroke}{rgb}{0.278791,0.062145,0.386592}%
\pgfsetstrokecolor{currentstroke}%
\pgfsetdash{}{0pt}%
\pgfpathmoveto{\pgfqpoint{8.481642in}{5.247230in}}%
\pgfpathlineto{\pgfqpoint{8.431497in}{5.246739in}}%
\pgfusepath{stroke}%
\end{pgfscope}%
\begin{pgfscope}%
\pgfpathrectangle{\pgfqpoint{6.720588in}{4.155455in}}{\pgfqpoint{2.279412in}{2.004545in}}%
\pgfusepath{clip}%
\pgfsetbuttcap%
\pgfsetroundjoin%
\pgfsetlinewidth{0.392605pt}%
\definecolor{currentstroke}{rgb}{0.280894,0.078907,0.402329}%
\pgfsetstrokecolor{currentstroke}%
\pgfsetdash{}{0pt}%
\pgfpathmoveto{\pgfqpoint{8.431497in}{5.246739in}}%
\pgfpathlineto{\pgfqpoint{8.381354in}{5.246000in}}%
\pgfusepath{stroke}%
\end{pgfscope}%
\begin{pgfscope}%
\pgfpathrectangle{\pgfqpoint{6.720588in}{4.155455in}}{\pgfqpoint{2.279412in}{2.004545in}}%
\pgfusepath{clip}%
\pgfsetbuttcap%
\pgfsetroundjoin%
\pgfsetlinewidth{0.441200pt}%
\definecolor{currentstroke}{rgb}{0.283197,0.115680,0.436115}%
\pgfsetstrokecolor{currentstroke}%
\pgfsetdash{}{0pt}%
\pgfpathmoveto{\pgfqpoint{8.381354in}{5.246000in}}%
\pgfpathlineto{\pgfqpoint{8.331213in}{5.245086in}}%
\pgfusepath{stroke}%
\end{pgfscope}%
\begin{pgfscope}%
\pgfpathrectangle{\pgfqpoint{6.720588in}{4.155455in}}{\pgfqpoint{2.279412in}{2.004545in}}%
\pgfusepath{clip}%
\pgfsetbuttcap%
\pgfsetroundjoin%
\pgfsetlinewidth{0.494482pt}%
\definecolor{currentstroke}{rgb}{0.281412,0.155834,0.469201}%
\pgfsetstrokecolor{currentstroke}%
\pgfsetdash{}{0pt}%
\pgfpathmoveto{\pgfqpoint{8.331213in}{5.245086in}}%
\pgfpathlineto{\pgfqpoint{8.281075in}{5.244071in}}%
\pgfusepath{stroke}%
\end{pgfscope}%
\begin{pgfscope}%
\pgfpathrectangle{\pgfqpoint{6.720588in}{4.155455in}}{\pgfqpoint{2.279412in}{2.004545in}}%
\pgfusepath{clip}%
\pgfsetbuttcap%
\pgfsetroundjoin%
\pgfsetlinewidth{0.575201pt}%
\definecolor{currentstroke}{rgb}{0.271828,0.209303,0.504434}%
\pgfsetstrokecolor{currentstroke}%
\pgfsetdash{}{0pt}%
\pgfpathmoveto{\pgfqpoint{8.281075in}{5.244071in}}%
\pgfpathlineto{\pgfqpoint{8.230941in}{5.242908in}}%
\pgfusepath{stroke}%
\end{pgfscope}%
\begin{pgfscope}%
\pgfpathrectangle{\pgfqpoint{6.720588in}{4.155455in}}{\pgfqpoint{2.279412in}{2.004545in}}%
\pgfusepath{clip}%
\pgfsetbuttcap%
\pgfsetroundjoin%
\pgfsetlinewidth{0.692210pt}%
\definecolor{currentstroke}{rgb}{0.244972,0.287675,0.537260}%
\pgfsetstrokecolor{currentstroke}%
\pgfsetdash{}{0pt}%
\pgfpathmoveto{\pgfqpoint{8.230941in}{5.242908in}}%
\pgfpathlineto{\pgfqpoint{8.180817in}{5.241480in}}%
\pgfusepath{stroke}%
\end{pgfscope}%
\begin{pgfscope}%
\pgfpathrectangle{\pgfqpoint{6.720588in}{4.155455in}}{\pgfqpoint{2.279412in}{2.004545in}}%
\pgfusepath{clip}%
\pgfsetbuttcap%
\pgfsetroundjoin%
\pgfsetlinewidth{0.852124pt}%
\definecolor{currentstroke}{rgb}{0.199430,0.387607,0.554642}%
\pgfsetstrokecolor{currentstroke}%
\pgfsetdash{}{0pt}%
\pgfpathmoveto{\pgfqpoint{8.180817in}{5.241480in}}%
\pgfpathlineto{\pgfqpoint{8.130706in}{5.239711in}}%
\pgfusepath{stroke}%
\end{pgfscope}%
\begin{pgfscope}%
\pgfpathrectangle{\pgfqpoint{6.720588in}{4.155455in}}{\pgfqpoint{2.279412in}{2.004545in}}%
\pgfusepath{clip}%
\pgfsetbuttcap%
\pgfsetroundjoin%
\pgfsetlinewidth{1.030497pt}%
\definecolor{currentstroke}{rgb}{0.157729,0.485932,0.558013}%
\pgfsetstrokecolor{currentstroke}%
\pgfsetdash{}{0pt}%
\pgfpathmoveto{\pgfqpoint{8.130706in}{5.239711in}}%
\pgfpathlineto{\pgfqpoint{8.080619in}{5.237486in}}%
\pgfusepath{stroke}%
\end{pgfscope}%
\begin{pgfscope}%
\pgfpathrectangle{\pgfqpoint{6.720588in}{4.155455in}}{\pgfqpoint{2.279412in}{2.004545in}}%
\pgfusepath{clip}%
\pgfsetbuttcap%
\pgfsetroundjoin%
\pgfsetlinewidth{1.304331pt}%
\definecolor{currentstroke}{rgb}{0.122312,0.633153,0.530398}%
\pgfsetstrokecolor{currentstroke}%
\pgfsetdash{}{0pt}%
\pgfpathmoveto{\pgfqpoint{8.080619in}{5.237486in}}%
\pgfpathlineto{\pgfqpoint{8.030578in}{5.234581in}}%
\pgfusepath{stroke}%
\end{pgfscope}%
\begin{pgfscope}%
\pgfpathrectangle{\pgfqpoint{6.720588in}{4.155455in}}{\pgfqpoint{2.279412in}{2.004545in}}%
\pgfusepath{clip}%
\pgfsetbuttcap%
\pgfsetroundjoin%
\pgfsetlinewidth{1.688733pt}%
\definecolor{currentstroke}{rgb}{0.458674,0.816363,0.329727}%
\pgfsetstrokecolor{currentstroke}%
\pgfsetdash{}{0pt}%
\pgfpathmoveto{\pgfqpoint{8.030578in}{5.234581in}}%
\pgfpathlineto{\pgfqpoint{7.980595in}{5.230986in}}%
\pgfusepath{stroke}%
\end{pgfscope}%
\begin{pgfscope}%
\pgfpathrectangle{\pgfqpoint{6.720588in}{4.155455in}}{\pgfqpoint{2.279412in}{2.004545in}}%
\pgfusepath{clip}%
\pgfsetbuttcap%
\pgfsetroundjoin%
\pgfsetlinewidth{1.923966pt}%
\definecolor{currentstroke}{rgb}{0.804182,0.882046,0.114965}%
\pgfsetstrokecolor{currentstroke}%
\pgfsetdash{}{0pt}%
\pgfpathmoveto{\pgfqpoint{7.980595in}{5.230986in}}%
\pgfpathlineto{\pgfqpoint{7.930691in}{5.226649in}}%
\pgfusepath{stroke}%
\end{pgfscope}%
\begin{pgfscope}%
\pgfpathrectangle{\pgfqpoint{6.720588in}{4.155455in}}{\pgfqpoint{2.279412in}{2.004545in}}%
\pgfusepath{clip}%
\pgfsetbuttcap%
\pgfsetroundjoin%
\pgfsetlinewidth{2.095647pt}%
\definecolor{currentstroke}{rgb}{0.993248,0.906157,0.143936}%
\pgfsetstrokecolor{currentstroke}%
\pgfsetdash{}{0pt}%
\pgfpathmoveto{\pgfqpoint{7.930691in}{5.226649in}}%
\pgfpathlineto{\pgfqpoint{7.880858in}{5.221703in}}%
\pgfusepath{stroke}%
\end{pgfscope}%
\begin{pgfscope}%
\pgfpathrectangle{\pgfqpoint{6.720588in}{4.155455in}}{\pgfqpoint{2.279412in}{2.004545in}}%
\pgfusepath{clip}%
\pgfsetbuttcap%
\pgfsetroundjoin%
\pgfsetlinewidth{2.157884pt}%
\definecolor{currentstroke}{rgb}{0.993248,0.906157,0.143936}%
\pgfsetstrokecolor{currentstroke}%
\pgfsetdash{}{0pt}%
\pgfpathmoveto{\pgfqpoint{7.880858in}{5.221703in}}%
\pgfpathlineto{\pgfqpoint{7.831095in}{5.216249in}}%
\pgfusepath{stroke}%
\end{pgfscope}%
\begin{pgfscope}%
\pgfpathrectangle{\pgfqpoint{6.720588in}{4.155455in}}{\pgfqpoint{2.279412in}{2.004545in}}%
\pgfusepath{clip}%
\pgfsetbuttcap%
\pgfsetroundjoin%
\pgfsetlinewidth{2.262567pt}%
\definecolor{currentstroke}{rgb}{0.993248,0.906157,0.143936}%
\pgfsetstrokecolor{currentstroke}%
\pgfsetdash{}{0pt}%
\pgfpathmoveto{\pgfqpoint{7.831095in}{5.216249in}}%
\pgfpathlineto{\pgfqpoint{7.781390in}{5.210411in}}%
\pgfusepath{stroke}%
\end{pgfscope}%
\begin{pgfscope}%
\pgfpathrectangle{\pgfqpoint{6.720588in}{4.155455in}}{\pgfqpoint{2.279412in}{2.004545in}}%
\pgfusepath{clip}%
\pgfsetbuttcap%
\pgfsetroundjoin%
\pgfsetlinewidth{2.302523pt}%
\definecolor{currentstroke}{rgb}{0.993248,0.906157,0.143936}%
\pgfsetstrokecolor{currentstroke}%
\pgfsetdash{}{0pt}%
\pgfpathmoveto{\pgfqpoint{7.781390in}{5.210411in}}%
\pgfpathlineto{\pgfqpoint{7.731758in}{5.204165in}}%
\pgfusepath{stroke}%
\end{pgfscope}%
\begin{pgfscope}%
\pgfpathrectangle{\pgfqpoint{6.720588in}{4.155455in}}{\pgfqpoint{2.279412in}{2.004545in}}%
\pgfusepath{clip}%
\pgfsetbuttcap%
\pgfsetroundjoin%
\pgfsetlinewidth{2.356476pt}%
\definecolor{currentstroke}{rgb}{0.993248,0.906157,0.143936}%
\pgfsetstrokecolor{currentstroke}%
\pgfsetdash{}{0pt}%
\pgfpathmoveto{\pgfqpoint{7.731758in}{5.204165in}}%
\pgfpathlineto{\pgfqpoint{7.682164in}{5.197665in}}%
\pgfusepath{stroke}%
\end{pgfscope}%
\begin{pgfscope}%
\pgfpathrectangle{\pgfqpoint{6.720588in}{4.155455in}}{\pgfqpoint{2.279412in}{2.004545in}}%
\pgfusepath{clip}%
\pgfsetbuttcap%
\pgfsetroundjoin%
\pgfsetlinewidth{2.122661pt}%
\definecolor{currentstroke}{rgb}{0.993248,0.906157,0.143936}%
\pgfsetstrokecolor{currentstroke}%
\pgfsetdash{}{0pt}%
\pgfpathmoveto{\pgfqpoint{7.682164in}{5.197665in}}%
\pgfpathlineto{\pgfqpoint{7.632621in}{5.190940in}}%
\pgfusepath{stroke}%
\end{pgfscope}%
\begin{pgfscope}%
\pgfpathrectangle{\pgfqpoint{6.720588in}{4.155455in}}{\pgfqpoint{2.279412in}{2.004545in}}%
\pgfusepath{clip}%
\pgfsetbuttcap%
\pgfsetroundjoin%
\pgfsetlinewidth{1.972738pt}%
\definecolor{currentstroke}{rgb}{0.876168,0.891125,0.095250}%
\pgfsetstrokecolor{currentstroke}%
\pgfsetdash{}{0pt}%
\pgfpathmoveto{\pgfqpoint{7.632621in}{5.190940in}}%
\pgfpathlineto{\pgfqpoint{7.583126in}{5.184128in}}%
\pgfusepath{stroke}%
\end{pgfscope}%
\begin{pgfscope}%
\pgfpathrectangle{\pgfqpoint{6.720588in}{4.155455in}}{\pgfqpoint{2.279412in}{2.004545in}}%
\pgfusepath{clip}%
\pgfsetbuttcap%
\pgfsetroundjoin%
\pgfsetlinewidth{1.876535pt}%
\definecolor{currentstroke}{rgb}{0.730889,0.871916,0.156029}%
\pgfsetstrokecolor{currentstroke}%
\pgfsetdash{}{0pt}%
\pgfpathmoveto{\pgfqpoint{7.583126in}{5.184128in}}%
\pgfpathlineto{\pgfqpoint{7.533642in}{5.177170in}}%
\pgfusepath{stroke}%
\end{pgfscope}%
\begin{pgfscope}%
\pgfpathrectangle{\pgfqpoint{6.720588in}{4.155455in}}{\pgfqpoint{2.279412in}{2.004545in}}%
\pgfusepath{clip}%
\pgfsetbuttcap%
\pgfsetroundjoin%
\pgfsetlinewidth{1.569561pt}%
\definecolor{currentstroke}{rgb}{0.304148,0.764704,0.419943}%
\pgfsetstrokecolor{currentstroke}%
\pgfsetdash{}{0pt}%
\pgfpathmoveto{\pgfqpoint{7.533642in}{5.177170in}}%
\pgfpathlineto{\pgfqpoint{7.484109in}{5.170852in}}%
\pgfusepath{stroke}%
\end{pgfscope}%
\begin{pgfscope}%
\pgfpathrectangle{\pgfqpoint{6.720588in}{4.155455in}}{\pgfqpoint{2.279412in}{2.004545in}}%
\pgfusepath{clip}%
\pgfsetbuttcap%
\pgfsetroundjoin%
\pgfsetlinewidth{1.359678pt}%
\definecolor{currentstroke}{rgb}{0.137339,0.662252,0.515571}%
\pgfsetstrokecolor{currentstroke}%
\pgfsetdash{}{0pt}%
\pgfpathmoveto{\pgfqpoint{7.484109in}{5.170852in}}%
\pgfpathlineto{\pgfqpoint{7.434741in}{5.164547in}}%
\pgfusepath{stroke}%
\end{pgfscope}%
\begin{pgfscope}%
\pgfpathrectangle{\pgfqpoint{6.720588in}{4.155455in}}{\pgfqpoint{2.279412in}{2.004545in}}%
\pgfusepath{clip}%
\pgfsetbuttcap%
\pgfsetroundjoin%
\pgfsetlinewidth{1.228794pt}%
\definecolor{currentstroke}{rgb}{0.121148,0.592739,0.544641}%
\pgfsetstrokecolor{currentstroke}%
\pgfsetdash{}{0pt}%
\pgfpathmoveto{\pgfqpoint{7.434741in}{5.164547in}}%
\pgfpathlineto{\pgfqpoint{7.385082in}{5.160671in}}%
\pgfusepath{stroke}%
\end{pgfscope}%
\begin{pgfscope}%
\pgfpathrectangle{\pgfqpoint{6.720588in}{4.155455in}}{\pgfqpoint{2.279412in}{2.004545in}}%
\pgfusepath{clip}%
\pgfsetbuttcap%
\pgfsetroundjoin%
\pgfsetlinewidth{0.309954pt}%
\definecolor{currentstroke}{rgb}{0.268510,0.009605,0.335427}%
\pgfsetstrokecolor{currentstroke}%
\pgfsetdash{}{0pt}%
\pgfpathmoveto{\pgfqpoint{8.732256in}{5.293048in}}%
\pgfpathlineto{\pgfqpoint{8.682228in}{5.294577in}}%
\pgfusepath{stroke}%
\end{pgfscope}%
\begin{pgfscope}%
\pgfpathrectangle{\pgfqpoint{6.720588in}{4.155455in}}{\pgfqpoint{2.279412in}{2.004545in}}%
\pgfusepath{clip}%
\pgfsetbuttcap%
\pgfsetroundjoin%
\pgfsetlinewidth{0.326466pt}%
\definecolor{currentstroke}{rgb}{0.271305,0.019942,0.347269}%
\pgfsetstrokecolor{currentstroke}%
\pgfsetdash{}{0pt}%
\pgfpathmoveto{\pgfqpoint{8.682228in}{5.294577in}}%
\pgfpathlineto{\pgfqpoint{8.632094in}{5.294340in}}%
\pgfusepath{stroke}%
\end{pgfscope}%
\begin{pgfscope}%
\pgfpathrectangle{\pgfqpoint{6.720588in}{4.155455in}}{\pgfqpoint{2.279412in}{2.004545in}}%
\pgfusepath{clip}%
\pgfsetbuttcap%
\pgfsetroundjoin%
\pgfsetlinewidth{0.328203pt}%
\definecolor{currentstroke}{rgb}{0.271305,0.019942,0.347269}%
\pgfsetstrokecolor{currentstroke}%
\pgfsetdash{}{0pt}%
\pgfpathmoveto{\pgfqpoint{8.632094in}{5.294340in}}%
\pgfpathlineto{\pgfqpoint{8.581960in}{5.293449in}}%
\pgfusepath{stroke}%
\end{pgfscope}%
\begin{pgfscope}%
\pgfpathrectangle{\pgfqpoint{6.720588in}{4.155455in}}{\pgfqpoint{2.279412in}{2.004545in}}%
\pgfusepath{clip}%
\pgfsetbuttcap%
\pgfsetroundjoin%
\pgfsetlinewidth{0.332265pt}%
\definecolor{currentstroke}{rgb}{0.272594,0.025563,0.353093}%
\pgfsetstrokecolor{currentstroke}%
\pgfsetdash{}{0pt}%
\pgfpathmoveto{\pgfqpoint{8.581960in}{5.293449in}}%
\pgfpathlineto{\pgfqpoint{8.531820in}{5.292720in}}%
\pgfusepath{stroke}%
\end{pgfscope}%
\begin{pgfscope}%
\pgfpathrectangle{\pgfqpoint{6.720588in}{4.155455in}}{\pgfqpoint{2.279412in}{2.004545in}}%
\pgfusepath{clip}%
\pgfsetbuttcap%
\pgfsetroundjoin%
\pgfsetlinewidth{0.348154pt}%
\definecolor{currentstroke}{rgb}{0.274952,0.037752,0.364543}%
\pgfsetstrokecolor{currentstroke}%
\pgfsetdash{}{0pt}%
\pgfpathmoveto{\pgfqpoint{8.531820in}{5.292720in}}%
\pgfpathlineto{\pgfqpoint{8.481676in}{5.292055in}}%
\pgfusepath{stroke}%
\end{pgfscope}%
\begin{pgfscope}%
\pgfpathrectangle{\pgfqpoint{6.720588in}{4.155455in}}{\pgfqpoint{2.279412in}{2.004545in}}%
\pgfusepath{clip}%
\pgfsetbuttcap%
\pgfsetroundjoin%
\pgfsetlinewidth{0.366666pt}%
\definecolor{currentstroke}{rgb}{0.277941,0.056324,0.381191}%
\pgfsetstrokecolor{currentstroke}%
\pgfsetdash{}{0pt}%
\pgfpathmoveto{\pgfqpoint{8.481676in}{5.292055in}}%
\pgfpathlineto{\pgfqpoint{8.431538in}{5.291017in}}%
\pgfusepath{stroke}%
\end{pgfscope}%
\begin{pgfscope}%
\pgfpathrectangle{\pgfqpoint{6.720588in}{4.155455in}}{\pgfqpoint{2.279412in}{2.004545in}}%
\pgfusepath{clip}%
\pgfsetbuttcap%
\pgfsetroundjoin%
\pgfsetlinewidth{0.395579pt}%
\definecolor{currentstroke}{rgb}{0.280894,0.078907,0.402329}%
\pgfsetstrokecolor{currentstroke}%
\pgfsetdash{}{0pt}%
\pgfpathmoveto{\pgfqpoint{8.431538in}{5.291017in}}%
\pgfpathlineto{\pgfqpoint{8.381407in}{5.289785in}}%
\pgfusepath{stroke}%
\end{pgfscope}%
\begin{pgfscope}%
\pgfpathrectangle{\pgfqpoint{6.720588in}{4.155455in}}{\pgfqpoint{2.279412in}{2.004545in}}%
\pgfusepath{clip}%
\pgfsetbuttcap%
\pgfsetroundjoin%
\pgfsetlinewidth{0.423890pt}%
\definecolor{currentstroke}{rgb}{0.282656,0.100196,0.422160}%
\pgfsetstrokecolor{currentstroke}%
\pgfsetdash{}{0pt}%
\pgfpathmoveto{\pgfqpoint{8.381407in}{5.289785in}}%
\pgfpathlineto{\pgfqpoint{8.331288in}{5.288252in}}%
\pgfusepath{stroke}%
\end{pgfscope}%
\begin{pgfscope}%
\pgfpathrectangle{\pgfqpoint{6.720588in}{4.155455in}}{\pgfqpoint{2.279412in}{2.004545in}}%
\pgfusepath{clip}%
\pgfsetbuttcap%
\pgfsetroundjoin%
\pgfsetlinewidth{0.474252pt}%
\definecolor{currentstroke}{rgb}{0.282623,0.140926,0.457517}%
\pgfsetstrokecolor{currentstroke}%
\pgfsetdash{}{0pt}%
\pgfpathmoveto{\pgfqpoint{8.331288in}{5.288252in}}%
\pgfpathlineto{\pgfqpoint{8.281172in}{5.286610in}}%
\pgfusepath{stroke}%
\end{pgfscope}%
\begin{pgfscope}%
\pgfpathrectangle{\pgfqpoint{6.720588in}{4.155455in}}{\pgfqpoint{2.279412in}{2.004545in}}%
\pgfusepath{clip}%
\pgfsetbuttcap%
\pgfsetroundjoin%
\pgfsetlinewidth{0.547323pt}%
\definecolor{currentstroke}{rgb}{0.276194,0.190074,0.493001}%
\pgfsetstrokecolor{currentstroke}%
\pgfsetdash{}{0pt}%
\pgfpathmoveto{\pgfqpoint{8.281172in}{5.286610in}}%
\pgfpathlineto{\pgfqpoint{8.231059in}{5.284897in}}%
\pgfusepath{stroke}%
\end{pgfscope}%
\begin{pgfscope}%
\pgfpathrectangle{\pgfqpoint{6.720588in}{4.155455in}}{\pgfqpoint{2.279412in}{2.004545in}}%
\pgfusepath{clip}%
\pgfsetbuttcap%
\pgfsetroundjoin%
\pgfsetlinewidth{0.643836pt}%
\definecolor{currentstroke}{rgb}{0.257322,0.256130,0.526563}%
\pgfsetstrokecolor{currentstroke}%
\pgfsetdash{}{0pt}%
\pgfpathmoveto{\pgfqpoint{8.231059in}{5.284897in}}%
\pgfpathlineto{\pgfqpoint{8.180964in}{5.282827in}}%
\pgfusepath{stroke}%
\end{pgfscope}%
\begin{pgfscope}%
\pgfpathrectangle{\pgfqpoint{6.720588in}{4.155455in}}{\pgfqpoint{2.279412in}{2.004545in}}%
\pgfusepath{clip}%
\pgfsetbuttcap%
\pgfsetroundjoin%
\pgfsetlinewidth{0.731130pt}%
\definecolor{currentstroke}{rgb}{0.233603,0.313828,0.543914}%
\pgfsetstrokecolor{currentstroke}%
\pgfsetdash{}{0pt}%
\pgfpathmoveto{\pgfqpoint{8.180964in}{5.282827in}}%
\pgfpathlineto{\pgfqpoint{8.130894in}{5.280313in}}%
\pgfusepath{stroke}%
\end{pgfscope}%
\begin{pgfscope}%
\pgfpathrectangle{\pgfqpoint{6.720588in}{4.155455in}}{\pgfqpoint{2.279412in}{2.004545in}}%
\pgfusepath{clip}%
\pgfsetbuttcap%
\pgfsetroundjoin%
\pgfsetlinewidth{0.880983pt}%
\definecolor{currentstroke}{rgb}{0.192357,0.403199,0.555836}%
\pgfsetstrokecolor{currentstroke}%
\pgfsetdash{}{0pt}%
\pgfpathmoveto{\pgfqpoint{8.130894in}{5.280313in}}%
\pgfpathlineto{\pgfqpoint{8.080873in}{5.277169in}}%
\pgfusepath{stroke}%
\end{pgfscope}%
\begin{pgfscope}%
\pgfpathrectangle{\pgfqpoint{6.720588in}{4.155455in}}{\pgfqpoint{2.279412in}{2.004545in}}%
\pgfusepath{clip}%
\pgfsetbuttcap%
\pgfsetroundjoin%
\pgfsetlinewidth{1.094085pt}%
\definecolor{currentstroke}{rgb}{0.144759,0.519093,0.556572}%
\pgfsetstrokecolor{currentstroke}%
\pgfsetdash{}{0pt}%
\pgfpathmoveto{\pgfqpoint{8.080873in}{5.277169in}}%
\pgfpathlineto{\pgfqpoint{8.030940in}{5.273097in}}%
\pgfusepath{stroke}%
\end{pgfscope}%
\begin{pgfscope}%
\pgfpathrectangle{\pgfqpoint{6.720588in}{4.155455in}}{\pgfqpoint{2.279412in}{2.004545in}}%
\pgfusepath{clip}%
\pgfsetbuttcap%
\pgfsetroundjoin%
\pgfsetlinewidth{0.313200pt}%
\definecolor{currentstroke}{rgb}{0.268510,0.009605,0.335427}%
\pgfsetstrokecolor{currentstroke}%
\pgfsetdash{}{0pt}%
\pgfpathmoveto{\pgfqpoint{8.680964in}{4.932193in}}%
\pgfpathlineto{\pgfqpoint{8.632352in}{4.932963in}}%
\pgfusepath{stroke}%
\end{pgfscope}%
\begin{pgfscope}%
\pgfpathrectangle{\pgfqpoint{6.720588in}{4.155455in}}{\pgfqpoint{2.279412in}{2.004545in}}%
\pgfusepath{clip}%
\pgfsetbuttcap%
\pgfsetroundjoin%
\pgfsetlinewidth{0.318671pt}%
\definecolor{currentstroke}{rgb}{0.269944,0.014625,0.341379}%
\pgfsetstrokecolor{currentstroke}%
\pgfsetdash{}{0pt}%
\pgfpathmoveto{\pgfqpoint{8.632352in}{4.932963in}}%
\pgfpathlineto{\pgfqpoint{8.582213in}{4.933271in}}%
\pgfusepath{stroke}%
\end{pgfscope}%
\begin{pgfscope}%
\pgfpathrectangle{\pgfqpoint{6.720588in}{4.155455in}}{\pgfqpoint{2.279412in}{2.004545in}}%
\pgfusepath{clip}%
\pgfsetbuttcap%
\pgfsetroundjoin%
\pgfsetlinewidth{0.324895pt}%
\definecolor{currentstroke}{rgb}{0.271305,0.019942,0.347269}%
\pgfsetstrokecolor{currentstroke}%
\pgfsetdash{}{0pt}%
\pgfpathmoveto{\pgfqpoint{8.582213in}{4.933271in}}%
\pgfpathlineto{\pgfqpoint{8.532091in}{4.934246in}}%
\pgfusepath{stroke}%
\end{pgfscope}%
\begin{pgfscope}%
\pgfpathrectangle{\pgfqpoint{6.720588in}{4.155455in}}{\pgfqpoint{2.279412in}{2.004545in}}%
\pgfusepath{clip}%
\pgfsetbuttcap%
\pgfsetroundjoin%
\pgfsetlinewidth{0.339833pt}%
\definecolor{currentstroke}{rgb}{0.273809,0.031497,0.358853}%
\pgfsetstrokecolor{currentstroke}%
\pgfsetdash{}{0pt}%
\pgfpathmoveto{\pgfqpoint{8.532091in}{4.934246in}}%
\pgfpathlineto{\pgfqpoint{8.482023in}{4.936669in}}%
\pgfusepath{stroke}%
\end{pgfscope}%
\begin{pgfscope}%
\pgfpathrectangle{\pgfqpoint{6.720588in}{4.155455in}}{\pgfqpoint{2.279412in}{2.004545in}}%
\pgfusepath{clip}%
\pgfsetbuttcap%
\pgfsetroundjoin%
\pgfsetlinewidth{0.351874pt}%
\definecolor{currentstroke}{rgb}{0.276022,0.044167,0.370164}%
\pgfsetstrokecolor{currentstroke}%
\pgfsetdash{}{0pt}%
\pgfpathmoveto{\pgfqpoint{8.482023in}{4.936669in}}%
\pgfpathlineto{\pgfqpoint{8.431944in}{4.938904in}}%
\pgfusepath{stroke}%
\end{pgfscope}%
\begin{pgfscope}%
\pgfpathrectangle{\pgfqpoint{6.720588in}{4.155455in}}{\pgfqpoint{2.279412in}{2.004545in}}%
\pgfusepath{clip}%
\pgfsetbuttcap%
\pgfsetroundjoin%
\pgfsetlinewidth{0.374677pt}%
\definecolor{currentstroke}{rgb}{0.278791,0.062145,0.386592}%
\pgfsetstrokecolor{currentstroke}%
\pgfsetdash{}{0pt}%
\pgfpathmoveto{\pgfqpoint{8.431944in}{4.938904in}}%
\pgfpathlineto{\pgfqpoint{8.381844in}{4.940863in}}%
\pgfusepath{stroke}%
\end{pgfscope}%
\begin{pgfscope}%
\pgfpathrectangle{\pgfqpoint{6.720588in}{4.155455in}}{\pgfqpoint{2.279412in}{2.004545in}}%
\pgfusepath{clip}%
\pgfsetbuttcap%
\pgfsetroundjoin%
\pgfsetlinewidth{0.405932pt}%
\definecolor{currentstroke}{rgb}{0.281924,0.089666,0.412415}%
\pgfsetstrokecolor{currentstroke}%
\pgfsetdash{}{0pt}%
\pgfpathmoveto{\pgfqpoint{8.381844in}{4.940863in}}%
\pgfpathlineto{\pgfqpoint{8.331761in}{4.943184in}}%
\pgfusepath{stroke}%
\end{pgfscope}%
\begin{pgfscope}%
\pgfpathrectangle{\pgfqpoint{6.720588in}{4.155455in}}{\pgfqpoint{2.279412in}{2.004545in}}%
\pgfusepath{clip}%
\pgfsetbuttcap%
\pgfsetroundjoin%
\pgfsetlinewidth{0.456233pt}%
\definecolor{currentstroke}{rgb}{0.283187,0.125848,0.444960}%
\pgfsetstrokecolor{currentstroke}%
\pgfsetdash{}{0pt}%
\pgfpathmoveto{\pgfqpoint{8.331761in}{4.943184in}}%
\pgfpathlineto{\pgfqpoint{8.281679in}{4.945505in}}%
\pgfusepath{stroke}%
\end{pgfscope}%
\begin{pgfscope}%
\pgfpathrectangle{\pgfqpoint{6.720588in}{4.155455in}}{\pgfqpoint{2.279412in}{2.004545in}}%
\pgfusepath{clip}%
\pgfsetbuttcap%
\pgfsetroundjoin%
\pgfsetlinewidth{0.505737pt}%
\definecolor{currentstroke}{rgb}{0.280868,0.160771,0.472899}%
\pgfsetstrokecolor{currentstroke}%
\pgfsetdash{}{0pt}%
\pgfpathmoveto{\pgfqpoint{8.281679in}{4.945505in}}%
\pgfpathlineto{\pgfqpoint{8.231614in}{4.948075in}}%
\pgfusepath{stroke}%
\end{pgfscope}%
\begin{pgfscope}%
\pgfpathrectangle{\pgfqpoint{6.720588in}{4.155455in}}{\pgfqpoint{2.279412in}{2.004545in}}%
\pgfusepath{clip}%
\pgfsetbuttcap%
\pgfsetroundjoin%
\pgfsetlinewidth{0.557660pt}%
\definecolor{currentstroke}{rgb}{0.274128,0.199721,0.498911}%
\pgfsetstrokecolor{currentstroke}%
\pgfsetdash{}{0pt}%
\pgfpathmoveto{\pgfqpoint{8.231614in}{4.948075in}}%
\pgfpathlineto{\pgfqpoint{8.181593in}{4.951228in}}%
\pgfusepath{stroke}%
\end{pgfscope}%
\begin{pgfscope}%
\pgfpathrectangle{\pgfqpoint{6.720588in}{4.155455in}}{\pgfqpoint{2.279412in}{2.004545in}}%
\pgfusepath{clip}%
\pgfsetbuttcap%
\pgfsetroundjoin%
\pgfsetlinewidth{0.638000pt}%
\definecolor{currentstroke}{rgb}{0.257322,0.256130,0.526563}%
\pgfsetstrokecolor{currentstroke}%
\pgfsetdash{}{0pt}%
\pgfpathmoveto{\pgfqpoint{8.181593in}{4.951228in}}%
\pgfpathlineto{\pgfqpoint{8.131631in}{4.955041in}}%
\pgfusepath{stroke}%
\end{pgfscope}%
\begin{pgfscope}%
\pgfpathrectangle{\pgfqpoint{6.720588in}{4.155455in}}{\pgfqpoint{2.279412in}{2.004545in}}%
\pgfusepath{clip}%
\pgfsetbuttcap%
\pgfsetroundjoin%
\pgfsetlinewidth{0.736629pt}%
\definecolor{currentstroke}{rgb}{0.231674,0.318106,0.544834}%
\pgfsetstrokecolor{currentstroke}%
\pgfsetdash{}{0pt}%
\pgfpathmoveto{\pgfqpoint{8.131631in}{4.955041in}}%
\pgfpathlineto{\pgfqpoint{8.081760in}{4.959679in}}%
\pgfusepath{stroke}%
\end{pgfscope}%
\begin{pgfscope}%
\pgfpathrectangle{\pgfqpoint{6.720588in}{4.155455in}}{\pgfqpoint{2.279412in}{2.004545in}}%
\pgfusepath{clip}%
\pgfsetbuttcap%
\pgfsetroundjoin%
\pgfsetlinewidth{0.815715pt}%
\definecolor{currentstroke}{rgb}{0.210503,0.363727,0.552206}%
\pgfsetstrokecolor{currentstroke}%
\pgfsetdash{}{0pt}%
\pgfpathmoveto{\pgfqpoint{8.081760in}{4.959679in}}%
\pgfpathlineto{\pgfqpoint{8.032016in}{4.965259in}}%
\pgfusepath{stroke}%
\end{pgfscope}%
\begin{pgfscope}%
\pgfpathrectangle{\pgfqpoint{6.720588in}{4.155455in}}{\pgfqpoint{2.279412in}{2.004545in}}%
\pgfusepath{clip}%
\pgfsetbuttcap%
\pgfsetroundjoin%
\pgfsetlinewidth{0.808555pt}%
\definecolor{currentstroke}{rgb}{0.212395,0.359683,0.551710}%
\pgfsetstrokecolor{currentstroke}%
\pgfsetdash{}{0pt}%
\pgfpathmoveto{\pgfqpoint{8.032016in}{4.965259in}}%
\pgfpathlineto{\pgfqpoint{7.982467in}{4.972025in}}%
\pgfusepath{stroke}%
\end{pgfscope}%
\begin{pgfscope}%
\pgfpathrectangle{\pgfqpoint{6.720588in}{4.155455in}}{\pgfqpoint{2.279412in}{2.004545in}}%
\pgfusepath{clip}%
\pgfsetbuttcap%
\pgfsetroundjoin%
\pgfsetlinewidth{0.984415pt}%
\definecolor{currentstroke}{rgb}{0.168126,0.459988,0.558082}%
\pgfsetstrokecolor{currentstroke}%
\pgfsetdash{}{0pt}%
\pgfpathmoveto{\pgfqpoint{7.982467in}{4.972025in}}%
\pgfpathlineto{\pgfqpoint{7.933214in}{4.980285in}}%
\pgfusepath{stroke}%
\end{pgfscope}%
\begin{pgfscope}%
\pgfpathrectangle{\pgfqpoint{6.720588in}{4.155455in}}{\pgfqpoint{2.279412in}{2.004545in}}%
\pgfusepath{clip}%
\pgfsetbuttcap%
\pgfsetroundjoin%
\pgfsetlinewidth{1.297118pt}%
\definecolor{currentstroke}{rgb}{0.121380,0.629492,0.531973}%
\pgfsetstrokecolor{currentstroke}%
\pgfsetdash{}{0pt}%
\pgfpathmoveto{\pgfqpoint{7.933214in}{4.980285in}}%
\pgfpathlineto{\pgfqpoint{7.884336in}{4.990121in}}%
\pgfusepath{stroke}%
\end{pgfscope}%
\begin{pgfscope}%
\pgfpathrectangle{\pgfqpoint{6.720588in}{4.155455in}}{\pgfqpoint{2.279412in}{2.004545in}}%
\pgfusepath{clip}%
\pgfsetbuttcap%
\pgfsetroundjoin%
\pgfsetlinewidth{0.320983pt}%
\definecolor{currentstroke}{rgb}{0.269944,0.014625,0.341379}%
\pgfsetstrokecolor{currentstroke}%
\pgfsetdash{}{0pt}%
\pgfpathmoveto{\pgfqpoint{8.680964in}{5.022407in}}%
\pgfpathlineto{\pgfqpoint{8.630902in}{5.024210in}}%
\pgfusepath{stroke}%
\end{pgfscope}%
\begin{pgfscope}%
\pgfpathrectangle{\pgfqpoint{6.720588in}{4.155455in}}{\pgfqpoint{2.279412in}{2.004545in}}%
\pgfusepath{clip}%
\pgfsetbuttcap%
\pgfsetroundjoin%
\pgfsetlinewidth{0.332038pt}%
\definecolor{currentstroke}{rgb}{0.272594,0.025563,0.353093}%
\pgfsetstrokecolor{currentstroke}%
\pgfsetdash{}{0pt}%
\pgfpathmoveto{\pgfqpoint{8.630902in}{5.024210in}}%
\pgfpathlineto{\pgfqpoint{8.580764in}{5.024635in}}%
\pgfusepath{stroke}%
\end{pgfscope}%
\begin{pgfscope}%
\pgfpathrectangle{\pgfqpoint{6.720588in}{4.155455in}}{\pgfqpoint{2.279412in}{2.004545in}}%
\pgfusepath{clip}%
\pgfsetbuttcap%
\pgfsetroundjoin%
\pgfsetlinewidth{0.337136pt}%
\definecolor{currentstroke}{rgb}{0.273809,0.031497,0.358853}%
\pgfsetstrokecolor{currentstroke}%
\pgfsetdash{}{0pt}%
\pgfpathmoveto{\pgfqpoint{8.580764in}{5.024635in}}%
\pgfpathlineto{\pgfqpoint{8.530618in}{5.024760in}}%
\pgfusepath{stroke}%
\end{pgfscope}%
\begin{pgfscope}%
\pgfpathrectangle{\pgfqpoint{6.720588in}{4.155455in}}{\pgfqpoint{2.279412in}{2.004545in}}%
\pgfusepath{clip}%
\pgfsetbuttcap%
\pgfsetroundjoin%
\pgfsetlinewidth{0.347779pt}%
\definecolor{currentstroke}{rgb}{0.274952,0.037752,0.364543}%
\pgfsetstrokecolor{currentstroke}%
\pgfsetdash{}{0pt}%
\pgfpathmoveto{\pgfqpoint{8.530618in}{5.024760in}}%
\pgfpathlineto{\pgfqpoint{8.480474in}{5.025293in}}%
\pgfusepath{stroke}%
\end{pgfscope}%
\begin{pgfscope}%
\pgfpathrectangle{\pgfqpoint{6.720588in}{4.155455in}}{\pgfqpoint{2.279412in}{2.004545in}}%
\pgfusepath{clip}%
\pgfsetbuttcap%
\pgfsetroundjoin%
\pgfsetlinewidth{0.367421pt}%
\definecolor{currentstroke}{rgb}{0.277941,0.056324,0.381191}%
\pgfsetstrokecolor{currentstroke}%
\pgfsetdash{}{0pt}%
\pgfpathmoveto{\pgfqpoint{8.480474in}{5.025293in}}%
\pgfpathlineto{\pgfqpoint{8.430331in}{5.025995in}}%
\pgfusepath{stroke}%
\end{pgfscope}%
\begin{pgfscope}%
\pgfpathrectangle{\pgfqpoint{6.720588in}{4.155455in}}{\pgfqpoint{2.279412in}{2.004545in}}%
\pgfusepath{clip}%
\pgfsetbuttcap%
\pgfsetroundjoin%
\pgfsetlinewidth{0.402634pt}%
\definecolor{currentstroke}{rgb}{0.281446,0.084320,0.407414}%
\pgfsetstrokecolor{currentstroke}%
\pgfsetdash{}{0pt}%
\pgfpathmoveto{\pgfqpoint{8.430331in}{5.025995in}}%
\pgfpathlineto{\pgfqpoint{8.380192in}{5.026890in}}%
\pgfusepath{stroke}%
\end{pgfscope}%
\begin{pgfscope}%
\pgfpathrectangle{\pgfqpoint{6.720588in}{4.155455in}}{\pgfqpoint{2.279412in}{2.004545in}}%
\pgfusepath{clip}%
\pgfsetbuttcap%
\pgfsetroundjoin%
\pgfsetlinewidth{0.428825pt}%
\definecolor{currentstroke}{rgb}{0.282910,0.105393,0.426902}%
\pgfsetstrokecolor{currentstroke}%
\pgfsetdash{}{0pt}%
\pgfpathmoveto{\pgfqpoint{8.380192in}{5.026890in}}%
\pgfpathlineto{\pgfqpoint{8.330054in}{5.027931in}}%
\pgfusepath{stroke}%
\end{pgfscope}%
\begin{pgfscope}%
\pgfpathrectangle{\pgfqpoint{6.720588in}{4.155455in}}{\pgfqpoint{2.279412in}{2.004545in}}%
\pgfusepath{clip}%
\pgfsetbuttcap%
\pgfsetroundjoin%
\pgfsetlinewidth{0.485426pt}%
\definecolor{currentstroke}{rgb}{0.282290,0.145912,0.461510}%
\pgfsetstrokecolor{currentstroke}%
\pgfsetdash{}{0pt}%
\pgfpathmoveto{\pgfqpoint{8.330054in}{5.027931in}}%
\pgfpathlineto{\pgfqpoint{8.279919in}{5.029103in}}%
\pgfusepath{stroke}%
\end{pgfscope}%
\begin{pgfscope}%
\pgfpathrectangle{\pgfqpoint{6.720588in}{4.155455in}}{\pgfqpoint{2.279412in}{2.004545in}}%
\pgfusepath{clip}%
\pgfsetbuttcap%
\pgfsetroundjoin%
\pgfsetlinewidth{0.559520pt}%
\definecolor{currentstroke}{rgb}{0.274128,0.199721,0.498911}%
\pgfsetstrokecolor{currentstroke}%
\pgfsetdash{}{0pt}%
\pgfpathmoveto{\pgfqpoint{8.279919in}{5.029103in}}%
\pgfpathlineto{\pgfqpoint{8.229792in}{5.030476in}}%
\pgfusepath{stroke}%
\end{pgfscope}%
\begin{pgfscope}%
\pgfpathrectangle{\pgfqpoint{6.720588in}{4.155455in}}{\pgfqpoint{2.279412in}{2.004545in}}%
\pgfusepath{clip}%
\pgfsetbuttcap%
\pgfsetroundjoin%
\pgfsetlinewidth{0.683910pt}%
\definecolor{currentstroke}{rgb}{0.246811,0.283237,0.535941}%
\pgfsetstrokecolor{currentstroke}%
\pgfsetdash{}{0pt}%
\pgfpathmoveto{\pgfqpoint{8.229792in}{5.030476in}}%
\pgfpathlineto{\pgfqpoint{8.179679in}{5.032175in}}%
\pgfusepath{stroke}%
\end{pgfscope}%
\begin{pgfscope}%
\pgfpathrectangle{\pgfqpoint{6.720588in}{4.155455in}}{\pgfqpoint{2.279412in}{2.004545in}}%
\pgfusepath{clip}%
\pgfsetbuttcap%
\pgfsetroundjoin%
\pgfsetlinewidth{0.310521pt}%
\definecolor{currentstroke}{rgb}{0.268510,0.009605,0.335427}%
\pgfsetstrokecolor{currentstroke}%
\pgfsetdash{}{0pt}%
\pgfpathmoveto{\pgfqpoint{8.680964in}{5.067514in}}%
\pgfpathlineto{\pgfqpoint{8.631933in}{5.071808in}}%
\pgfusepath{stroke}%
\end{pgfscope}%
\begin{pgfscope}%
\pgfpathrectangle{\pgfqpoint{6.720588in}{4.155455in}}{\pgfqpoint{2.279412in}{2.004545in}}%
\pgfusepath{clip}%
\pgfsetbuttcap%
\pgfsetroundjoin%
\pgfsetlinewidth{0.321239pt}%
\definecolor{currentstroke}{rgb}{0.269944,0.014625,0.341379}%
\pgfsetstrokecolor{currentstroke}%
\pgfsetdash{}{0pt}%
\pgfpathmoveto{\pgfqpoint{8.631933in}{5.071808in}}%
\pgfpathlineto{\pgfqpoint{8.581804in}{5.072721in}}%
\pgfusepath{stroke}%
\end{pgfscope}%
\begin{pgfscope}%
\pgfpathrectangle{\pgfqpoint{6.720588in}{4.155455in}}{\pgfqpoint{2.279412in}{2.004545in}}%
\pgfusepath{clip}%
\pgfsetbuttcap%
\pgfsetroundjoin%
\pgfsetlinewidth{0.326684pt}%
\definecolor{currentstroke}{rgb}{0.271305,0.019942,0.347269}%
\pgfsetstrokecolor{currentstroke}%
\pgfsetdash{}{0pt}%
\pgfpathmoveto{\pgfqpoint{8.581804in}{5.072721in}}%
\pgfpathlineto{\pgfqpoint{8.531679in}{5.073306in}}%
\pgfusepath{stroke}%
\end{pgfscope}%
\begin{pgfscope}%
\pgfpathrectangle{\pgfqpoint{6.720588in}{4.155455in}}{\pgfqpoint{2.279412in}{2.004545in}}%
\pgfusepath{clip}%
\pgfsetbuttcap%
\pgfsetroundjoin%
\pgfsetlinewidth{0.344202pt}%
\definecolor{currentstroke}{rgb}{0.274952,0.037752,0.364543}%
\pgfsetstrokecolor{currentstroke}%
\pgfsetdash{}{0pt}%
\pgfpathmoveto{\pgfqpoint{8.531679in}{5.073306in}}%
\pgfpathlineto{\pgfqpoint{8.481534in}{5.073347in}}%
\pgfusepath{stroke}%
\end{pgfscope}%
\begin{pgfscope}%
\pgfpathrectangle{\pgfqpoint{6.720588in}{4.155455in}}{\pgfqpoint{2.279412in}{2.004545in}}%
\pgfusepath{clip}%
\pgfsetbuttcap%
\pgfsetroundjoin%
\pgfsetlinewidth{0.371211pt}%
\definecolor{currentstroke}{rgb}{0.278791,0.062145,0.386592}%
\pgfsetstrokecolor{currentstroke}%
\pgfsetdash{}{0pt}%
\pgfpathmoveto{\pgfqpoint{8.481534in}{5.073347in}}%
\pgfpathlineto{\pgfqpoint{8.431386in}{5.073631in}}%
\pgfusepath{stroke}%
\end{pgfscope}%
\begin{pgfscope}%
\pgfpathrectangle{\pgfqpoint{6.720588in}{4.155455in}}{\pgfqpoint{2.279412in}{2.004545in}}%
\pgfusepath{clip}%
\pgfsetbuttcap%
\pgfsetroundjoin%
\pgfsetlinewidth{0.400179pt}%
\definecolor{currentstroke}{rgb}{0.281446,0.084320,0.407414}%
\pgfsetstrokecolor{currentstroke}%
\pgfsetdash{}{0pt}%
\pgfpathmoveto{\pgfqpoint{8.431386in}{5.073631in}}%
\pgfpathlineto{\pgfqpoint{8.381238in}{5.074009in}}%
\pgfusepath{stroke}%
\end{pgfscope}%
\begin{pgfscope}%
\pgfpathrectangle{\pgfqpoint{6.720588in}{4.155455in}}{\pgfqpoint{2.279412in}{2.004545in}}%
\pgfusepath{clip}%
\pgfsetbuttcap%
\pgfsetroundjoin%
\pgfsetlinewidth{0.435678pt}%
\definecolor{currentstroke}{rgb}{0.283091,0.110553,0.431554}%
\pgfsetstrokecolor{currentstroke}%
\pgfsetdash{}{0pt}%
\pgfpathmoveto{\pgfqpoint{8.381238in}{5.074009in}}%
\pgfpathlineto{\pgfqpoint{8.331092in}{5.074689in}}%
\pgfusepath{stroke}%
\end{pgfscope}%
\begin{pgfscope}%
\pgfpathrectangle{\pgfqpoint{6.720588in}{4.155455in}}{\pgfqpoint{2.279412in}{2.004545in}}%
\pgfusepath{clip}%
\pgfsetbuttcap%
\pgfsetroundjoin%
\pgfsetlinewidth{0.507141pt}%
\definecolor{currentstroke}{rgb}{0.280255,0.165693,0.476498}%
\pgfsetstrokecolor{currentstroke}%
\pgfsetdash{}{0pt}%
\pgfpathmoveto{\pgfqpoint{8.331092in}{5.074689in}}%
\pgfpathlineto{\pgfqpoint{8.280946in}{5.075337in}}%
\pgfusepath{stroke}%
\end{pgfscope}%
\begin{pgfscope}%
\pgfpathrectangle{\pgfqpoint{6.720588in}{4.155455in}}{\pgfqpoint{2.279412in}{2.004545in}}%
\pgfusepath{clip}%
\pgfsetbuttcap%
\pgfsetroundjoin%
\pgfsetlinewidth{0.582539pt}%
\definecolor{currentstroke}{rgb}{0.269308,0.218818,0.509577}%
\pgfsetstrokecolor{currentstroke}%
\pgfsetdash{}{0pt}%
\pgfpathmoveto{\pgfqpoint{8.280946in}{5.075337in}}%
\pgfpathlineto{\pgfqpoint{8.230801in}{5.076039in}}%
\pgfusepath{stroke}%
\end{pgfscope}%
\begin{pgfscope}%
\pgfpathrectangle{\pgfqpoint{6.720588in}{4.155455in}}{\pgfqpoint{2.279412in}{2.004545in}}%
\pgfusepath{clip}%
\pgfsetbuttcap%
\pgfsetroundjoin%
\pgfsetlinewidth{0.720570pt}%
\definecolor{currentstroke}{rgb}{0.235526,0.309527,0.542944}%
\pgfsetstrokecolor{currentstroke}%
\pgfsetdash{}{0pt}%
\pgfpathmoveto{\pgfqpoint{8.230801in}{5.076039in}}%
\pgfpathlineto{\pgfqpoint{8.180660in}{5.076971in}}%
\pgfusepath{stroke}%
\end{pgfscope}%
\begin{pgfscope}%
\pgfpathrectangle{\pgfqpoint{6.720588in}{4.155455in}}{\pgfqpoint{2.279412in}{2.004545in}}%
\pgfusepath{clip}%
\pgfsetbuttcap%
\pgfsetroundjoin%
\pgfsetlinewidth{0.896089pt}%
\definecolor{currentstroke}{rgb}{0.188923,0.410910,0.556326}%
\pgfsetstrokecolor{currentstroke}%
\pgfsetdash{}{0pt}%
\pgfpathmoveto{\pgfqpoint{8.180660in}{5.076971in}}%
\pgfpathlineto{\pgfqpoint{8.130532in}{5.078301in}}%
\pgfusepath{stroke}%
\end{pgfscope}%
\begin{pgfscope}%
\pgfpathrectangle{\pgfqpoint{6.720588in}{4.155455in}}{\pgfqpoint{2.279412in}{2.004545in}}%
\pgfusepath{clip}%
\pgfsetbuttcap%
\pgfsetroundjoin%
\pgfsetlinewidth{1.169811pt}%
\definecolor{currentstroke}{rgb}{0.129933,0.559582,0.551864}%
\pgfsetstrokecolor{currentstroke}%
\pgfsetdash{}{0pt}%
\pgfpathmoveto{\pgfqpoint{8.130532in}{5.078301in}}%
\pgfpathlineto{\pgfqpoint{8.080427in}{5.080174in}}%
\pgfusepath{stroke}%
\end{pgfscope}%
\begin{pgfscope}%
\pgfpathrectangle{\pgfqpoint{6.720588in}{4.155455in}}{\pgfqpoint{2.279412in}{2.004545in}}%
\pgfusepath{clip}%
\pgfsetbuttcap%
\pgfsetroundjoin%
\pgfsetlinewidth{1.490603pt}%
\definecolor{currentstroke}{rgb}{0.226397,0.728888,0.462789}%
\pgfsetstrokecolor{currentstroke}%
\pgfsetdash{}{0pt}%
\pgfpathmoveto{\pgfqpoint{8.080427in}{5.080174in}}%
\pgfpathlineto{\pgfqpoint{8.030350in}{5.082567in}}%
\pgfusepath{stroke}%
\end{pgfscope}%
\begin{pgfscope}%
\pgfpathrectangle{\pgfqpoint{6.720588in}{4.155455in}}{\pgfqpoint{2.279412in}{2.004545in}}%
\pgfusepath{clip}%
\pgfsetbuttcap%
\pgfsetroundjoin%
\pgfsetlinewidth{1.820453pt}%
\definecolor{currentstroke}{rgb}{0.647257,0.858400,0.209861}%
\pgfsetstrokecolor{currentstroke}%
\pgfsetdash{}{0pt}%
\pgfpathmoveto{\pgfqpoint{8.030350in}{5.082567in}}%
\pgfpathlineto{\pgfqpoint{7.980316in}{5.085526in}}%
\pgfusepath{stroke}%
\end{pgfscope}%
\begin{pgfscope}%
\pgfpathrectangle{\pgfqpoint{6.720588in}{4.155455in}}{\pgfqpoint{2.279412in}{2.004545in}}%
\pgfusepath{clip}%
\pgfsetbuttcap%
\pgfsetroundjoin%
\pgfsetlinewidth{2.094366pt}%
\definecolor{currentstroke}{rgb}{0.993248,0.906157,0.143936}%
\pgfsetstrokecolor{currentstroke}%
\pgfsetdash{}{0pt}%
\pgfpathmoveto{\pgfqpoint{7.980316in}{5.085526in}}%
\pgfpathlineto{\pgfqpoint{7.930325in}{5.088999in}}%
\pgfusepath{stroke}%
\end{pgfscope}%
\begin{pgfscope}%
\pgfpathrectangle{\pgfqpoint{6.720588in}{4.155455in}}{\pgfqpoint{2.279412in}{2.004545in}}%
\pgfusepath{clip}%
\pgfsetbuttcap%
\pgfsetroundjoin%
\pgfsetlinewidth{2.189055pt}%
\definecolor{currentstroke}{rgb}{0.993248,0.906157,0.143936}%
\pgfsetstrokecolor{currentstroke}%
\pgfsetdash{}{0pt}%
\pgfpathmoveto{\pgfqpoint{7.930325in}{5.088999in}}%
\pgfpathlineto{\pgfqpoint{7.880369in}{5.092830in}}%
\pgfusepath{stroke}%
\end{pgfscope}%
\begin{pgfscope}%
\pgfpathrectangle{\pgfqpoint{6.720588in}{4.155455in}}{\pgfqpoint{2.279412in}{2.004545in}}%
\pgfusepath{clip}%
\pgfsetbuttcap%
\pgfsetroundjoin%
\pgfsetlinewidth{2.306179pt}%
\definecolor{currentstroke}{rgb}{0.993248,0.906157,0.143936}%
\pgfsetstrokecolor{currentstroke}%
\pgfsetdash{}{0pt}%
\pgfpathmoveto{\pgfqpoint{7.880369in}{5.092830in}}%
\pgfpathlineto{\pgfqpoint{7.830451in}{5.097033in}}%
\pgfusepath{stroke}%
\end{pgfscope}%
\begin{pgfscope}%
\pgfpathrectangle{\pgfqpoint{6.720588in}{4.155455in}}{\pgfqpoint{2.279412in}{2.004545in}}%
\pgfusepath{clip}%
\pgfsetbuttcap%
\pgfsetroundjoin%
\pgfsetlinewidth{2.369909pt}%
\definecolor{currentstroke}{rgb}{0.993248,0.906157,0.143936}%
\pgfsetstrokecolor{currentstroke}%
\pgfsetdash{}{0pt}%
\pgfpathmoveto{\pgfqpoint{7.830451in}{5.097033in}}%
\pgfpathlineto{\pgfqpoint{7.780569in}{5.101575in}}%
\pgfusepath{stroke}%
\end{pgfscope}%
\begin{pgfscope}%
\pgfpathrectangle{\pgfqpoint{6.720588in}{4.155455in}}{\pgfqpoint{2.279412in}{2.004545in}}%
\pgfusepath{clip}%
\pgfsetbuttcap%
\pgfsetroundjoin%
\pgfsetlinewidth{2.389047pt}%
\definecolor{currentstroke}{rgb}{0.993248,0.906157,0.143936}%
\pgfsetstrokecolor{currentstroke}%
\pgfsetdash{}{0pt}%
\pgfpathmoveto{\pgfqpoint{7.780569in}{5.101575in}}%
\pgfpathlineto{\pgfqpoint{7.730734in}{5.106441in}}%
\pgfusepath{stroke}%
\end{pgfscope}%
\begin{pgfscope}%
\pgfpathrectangle{\pgfqpoint{6.720588in}{4.155455in}}{\pgfqpoint{2.279412in}{2.004545in}}%
\pgfusepath{clip}%
\pgfsetbuttcap%
\pgfsetroundjoin%
\pgfsetlinewidth{2.338154pt}%
\definecolor{currentstroke}{rgb}{0.993248,0.906157,0.143936}%
\pgfsetstrokecolor{currentstroke}%
\pgfsetdash{}{0pt}%
\pgfpathmoveto{\pgfqpoint{7.730734in}{5.106441in}}%
\pgfpathlineto{\pgfqpoint{7.680948in}{5.111620in}}%
\pgfusepath{stroke}%
\end{pgfscope}%
\begin{pgfscope}%
\pgfpathrectangle{\pgfqpoint{6.720588in}{4.155455in}}{\pgfqpoint{2.279412in}{2.004545in}}%
\pgfusepath{clip}%
\pgfsetbuttcap%
\pgfsetroundjoin%
\pgfsetlinewidth{2.188187pt}%
\definecolor{currentstroke}{rgb}{0.993248,0.906157,0.143936}%
\pgfsetstrokecolor{currentstroke}%
\pgfsetdash{}{0pt}%
\pgfpathmoveto{\pgfqpoint{7.680948in}{5.111620in}}%
\pgfpathlineto{\pgfqpoint{7.631173in}{5.116897in}}%
\pgfusepath{stroke}%
\end{pgfscope}%
\begin{pgfscope}%
\pgfpathrectangle{\pgfqpoint{6.720588in}{4.155455in}}{\pgfqpoint{2.279412in}{2.004545in}}%
\pgfusepath{clip}%
\pgfsetbuttcap%
\pgfsetroundjoin%
\pgfsetlinewidth{0.319456pt}%
\definecolor{currentstroke}{rgb}{0.269944,0.014625,0.341379}%
\pgfsetstrokecolor{currentstroke}%
\pgfsetdash{}{0pt}%
\pgfpathmoveto{\pgfqpoint{8.680964in}{5.157727in}}%
\pgfpathlineto{\pgfqpoint{8.630832in}{5.158008in}}%
\pgfusepath{stroke}%
\end{pgfscope}%
\begin{pgfscope}%
\pgfpathrectangle{\pgfqpoint{6.720588in}{4.155455in}}{\pgfqpoint{2.279412in}{2.004545in}}%
\pgfusepath{clip}%
\pgfsetbuttcap%
\pgfsetroundjoin%
\pgfsetlinewidth{0.333371pt}%
\definecolor{currentstroke}{rgb}{0.272594,0.025563,0.353093}%
\pgfsetstrokecolor{currentstroke}%
\pgfsetdash{}{0pt}%
\pgfpathmoveto{\pgfqpoint{8.630832in}{5.158008in}}%
\pgfpathlineto{\pgfqpoint{8.580692in}{5.157241in}}%
\pgfusepath{stroke}%
\end{pgfscope}%
\begin{pgfscope}%
\pgfpathrectangle{\pgfqpoint{6.720588in}{4.155455in}}{\pgfqpoint{2.279412in}{2.004545in}}%
\pgfusepath{clip}%
\pgfsetbuttcap%
\pgfsetroundjoin%
\pgfsetlinewidth{0.332842pt}%
\definecolor{currentstroke}{rgb}{0.272594,0.025563,0.353093}%
\pgfsetstrokecolor{currentstroke}%
\pgfsetdash{}{0pt}%
\pgfpathmoveto{\pgfqpoint{8.580692in}{5.157241in}}%
\pgfpathlineto{\pgfqpoint{8.530551in}{5.157124in}}%
\pgfusepath{stroke}%
\end{pgfscope}%
\begin{pgfscope}%
\pgfpathrectangle{\pgfqpoint{6.720588in}{4.155455in}}{\pgfqpoint{2.279412in}{2.004545in}}%
\pgfusepath{clip}%
\pgfsetbuttcap%
\pgfsetroundjoin%
\pgfsetlinewidth{0.344839pt}%
\definecolor{currentstroke}{rgb}{0.274952,0.037752,0.364543}%
\pgfsetstrokecolor{currentstroke}%
\pgfsetdash{}{0pt}%
\pgfpathmoveto{\pgfqpoint{8.530551in}{5.157124in}}%
\pgfpathlineto{\pgfqpoint{8.480403in}{5.157550in}}%
\pgfusepath{stroke}%
\end{pgfscope}%
\begin{pgfscope}%
\pgfpathrectangle{\pgfqpoint{6.720588in}{4.155455in}}{\pgfqpoint{2.279412in}{2.004545in}}%
\pgfusepath{clip}%
\pgfsetbuttcap%
\pgfsetroundjoin%
\pgfsetlinewidth{0.368345pt}%
\definecolor{currentstroke}{rgb}{0.277941,0.056324,0.381191}%
\pgfsetstrokecolor{currentstroke}%
\pgfsetdash{}{0pt}%
\pgfpathmoveto{\pgfqpoint{8.480403in}{5.157550in}}%
\pgfpathlineto{\pgfqpoint{8.430253in}{5.157764in}}%
\pgfusepath{stroke}%
\end{pgfscope}%
\begin{pgfscope}%
\pgfpathrectangle{\pgfqpoint{6.720588in}{4.155455in}}{\pgfqpoint{2.279412in}{2.004545in}}%
\pgfusepath{clip}%
\pgfsetbuttcap%
\pgfsetroundjoin%
\pgfsetlinewidth{0.394473pt}%
\definecolor{currentstroke}{rgb}{0.280894,0.078907,0.402329}%
\pgfsetstrokecolor{currentstroke}%
\pgfsetdash{}{0pt}%
\pgfpathmoveto{\pgfqpoint{8.430253in}{5.157764in}}%
\pgfpathlineto{\pgfqpoint{8.380108in}{5.158379in}}%
\pgfusepath{stroke}%
\end{pgfscope}%
\begin{pgfscope}%
\pgfpathrectangle{\pgfqpoint{6.720588in}{4.155455in}}{\pgfqpoint{2.279412in}{2.004545in}}%
\pgfusepath{clip}%
\pgfsetbuttcap%
\pgfsetroundjoin%
\pgfsetlinewidth{0.445773pt}%
\definecolor{currentstroke}{rgb}{0.283229,0.120777,0.440584}%
\pgfsetstrokecolor{currentstroke}%
\pgfsetdash{}{0pt}%
\pgfpathmoveto{\pgfqpoint{8.380108in}{5.158379in}}%
\pgfpathlineto{\pgfqpoint{8.329965in}{5.159121in}}%
\pgfusepath{stroke}%
\end{pgfscope}%
\begin{pgfscope}%
\pgfpathrectangle{\pgfqpoint{6.720588in}{4.155455in}}{\pgfqpoint{2.279412in}{2.004545in}}%
\pgfusepath{clip}%
\pgfsetbuttcap%
\pgfsetroundjoin%
\pgfsetlinewidth{0.506978pt}%
\definecolor{currentstroke}{rgb}{0.280255,0.165693,0.476498}%
\pgfsetstrokecolor{currentstroke}%
\pgfsetdash{}{0pt}%
\pgfpathmoveto{\pgfqpoint{8.329965in}{5.159121in}}%
\pgfpathlineto{\pgfqpoint{8.279815in}{5.159381in}}%
\pgfusepath{stroke}%
\end{pgfscope}%
\begin{pgfscope}%
\pgfpathrectangle{\pgfqpoint{6.720588in}{4.155455in}}{\pgfqpoint{2.279412in}{2.004545in}}%
\pgfusepath{clip}%
\pgfsetbuttcap%
\pgfsetroundjoin%
\pgfsetlinewidth{0.629521pt}%
\definecolor{currentstroke}{rgb}{0.260571,0.246922,0.522828}%
\pgfsetstrokecolor{currentstroke}%
\pgfsetdash{}{0pt}%
\pgfpathmoveto{\pgfqpoint{8.279815in}{5.159381in}}%
\pgfpathlineto{\pgfqpoint{8.229663in}{5.159357in}}%
\pgfusepath{stroke}%
\end{pgfscope}%
\begin{pgfscope}%
\pgfpathrectangle{\pgfqpoint{6.720588in}{4.155455in}}{\pgfqpoint{2.279412in}{2.004545in}}%
\pgfusepath{clip}%
\pgfsetbuttcap%
\pgfsetroundjoin%
\pgfsetlinewidth{0.746714pt}%
\definecolor{currentstroke}{rgb}{0.229739,0.322361,0.545706}%
\pgfsetstrokecolor{currentstroke}%
\pgfsetdash{}{0pt}%
\pgfpathmoveto{\pgfqpoint{8.229663in}{5.159357in}}%
\pgfpathlineto{\pgfqpoint{8.179511in}{5.159322in}}%
\pgfusepath{stroke}%
\end{pgfscope}%
\begin{pgfscope}%
\pgfpathrectangle{\pgfqpoint{6.720588in}{4.155455in}}{\pgfqpoint{2.279412in}{2.004545in}}%
\pgfusepath{clip}%
\pgfsetbuttcap%
\pgfsetroundjoin%
\pgfsetlinewidth{0.963546pt}%
\definecolor{currentstroke}{rgb}{0.172719,0.448791,0.557885}%
\pgfsetstrokecolor{currentstroke}%
\pgfsetdash{}{0pt}%
\pgfpathmoveto{\pgfqpoint{8.179511in}{5.159322in}}%
\pgfpathlineto{\pgfqpoint{8.129360in}{5.159248in}}%
\pgfusepath{stroke}%
\end{pgfscope}%
\begin{pgfscope}%
\pgfpathrectangle{\pgfqpoint{6.720588in}{4.155455in}}{\pgfqpoint{2.279412in}{2.004545in}}%
\pgfusepath{clip}%
\pgfsetbuttcap%
\pgfsetroundjoin%
\pgfsetlinewidth{1.321034pt}%
\definecolor{currentstroke}{rgb}{0.124780,0.640461,0.527068}%
\pgfsetstrokecolor{currentstroke}%
\pgfsetdash{}{0pt}%
\pgfpathmoveto{\pgfqpoint{8.129360in}{5.159248in}}%
\pgfpathlineto{\pgfqpoint{8.079208in}{5.159036in}}%
\pgfusepath{stroke}%
\end{pgfscope}%
\begin{pgfscope}%
\pgfpathrectangle{\pgfqpoint{6.720588in}{4.155455in}}{\pgfqpoint{2.279412in}{2.004545in}}%
\pgfusepath{clip}%
\pgfsetbuttcap%
\pgfsetroundjoin%
\pgfsetlinewidth{1.637098pt}%
\definecolor{currentstroke}{rgb}{0.386433,0.794644,0.372886}%
\pgfsetstrokecolor{currentstroke}%
\pgfsetdash{}{0pt}%
\pgfpathmoveto{\pgfqpoint{8.079208in}{5.159036in}}%
\pgfpathlineto{\pgfqpoint{8.029058in}{5.158682in}}%
\pgfusepath{stroke}%
\end{pgfscope}%
\begin{pgfscope}%
\pgfpathrectangle{\pgfqpoint{6.720588in}{4.155455in}}{\pgfqpoint{2.279412in}{2.004545in}}%
\pgfusepath{clip}%
\pgfsetbuttcap%
\pgfsetroundjoin%
\pgfsetlinewidth{2.064650pt}%
\definecolor{currentstroke}{rgb}{0.993248,0.906157,0.143936}%
\pgfsetstrokecolor{currentstroke}%
\pgfsetdash{}{0pt}%
\pgfpathmoveto{\pgfqpoint{8.029058in}{5.158682in}}%
\pgfpathlineto{\pgfqpoint{7.978909in}{5.158286in}}%
\pgfusepath{stroke}%
\end{pgfscope}%
\begin{pgfscope}%
\pgfpathrectangle{\pgfqpoint{6.720588in}{4.155455in}}{\pgfqpoint{2.279412in}{2.004545in}}%
\pgfusepath{clip}%
\pgfsetbuttcap%
\pgfsetroundjoin%
\pgfsetlinewidth{2.228268pt}%
\definecolor{currentstroke}{rgb}{0.993248,0.906157,0.143936}%
\pgfsetstrokecolor{currentstroke}%
\pgfsetdash{}{0pt}%
\pgfpathmoveto{\pgfqpoint{7.978909in}{5.158286in}}%
\pgfpathlineto{\pgfqpoint{7.928759in}{5.157924in}}%
\pgfusepath{stroke}%
\end{pgfscope}%
\begin{pgfscope}%
\pgfpathrectangle{\pgfqpoint{6.720588in}{4.155455in}}{\pgfqpoint{2.279412in}{2.004545in}}%
\pgfusepath{clip}%
\pgfsetbuttcap%
\pgfsetroundjoin%
\pgfsetlinewidth{2.349178pt}%
\definecolor{currentstroke}{rgb}{0.993248,0.906157,0.143936}%
\pgfsetstrokecolor{currentstroke}%
\pgfsetdash{}{0pt}%
\pgfpathmoveto{\pgfqpoint{7.928759in}{5.157924in}}%
\pgfpathlineto{\pgfqpoint{7.878609in}{5.157500in}}%
\pgfusepath{stroke}%
\end{pgfscope}%
\begin{pgfscope}%
\pgfpathrectangle{\pgfqpoint{6.720588in}{4.155455in}}{\pgfqpoint{2.279412in}{2.004545in}}%
\pgfusepath{clip}%
\pgfsetbuttcap%
\pgfsetroundjoin%
\pgfsetlinewidth{2.449764pt}%
\definecolor{currentstroke}{rgb}{0.993248,0.906157,0.143936}%
\pgfsetstrokecolor{currentstroke}%
\pgfsetdash{}{0pt}%
\pgfpathmoveto{\pgfqpoint{7.878609in}{5.157500in}}%
\pgfpathlineto{\pgfqpoint{7.828461in}{5.157030in}}%
\pgfusepath{stroke}%
\end{pgfscope}%
\begin{pgfscope}%
\pgfpathrectangle{\pgfqpoint{6.720588in}{4.155455in}}{\pgfqpoint{2.279412in}{2.004545in}}%
\pgfusepath{clip}%
\pgfsetbuttcap%
\pgfsetroundjoin%
\pgfsetlinewidth{2.547319pt}%
\definecolor{currentstroke}{rgb}{0.993248,0.906157,0.143936}%
\pgfsetstrokecolor{currentstroke}%
\pgfsetdash{}{0pt}%
\pgfpathmoveto{\pgfqpoint{7.828461in}{5.157030in}}%
\pgfpathlineto{\pgfqpoint{7.778314in}{5.156540in}}%
\pgfusepath{stroke}%
\end{pgfscope}%
\begin{pgfscope}%
\pgfpathrectangle{\pgfqpoint{6.720588in}{4.155455in}}{\pgfqpoint{2.279412in}{2.004545in}}%
\pgfusepath{clip}%
\pgfsetbuttcap%
\pgfsetroundjoin%
\pgfsetlinewidth{2.487285pt}%
\definecolor{currentstroke}{rgb}{0.993248,0.906157,0.143936}%
\pgfsetstrokecolor{currentstroke}%
\pgfsetdash{}{0pt}%
\pgfpathmoveto{\pgfqpoint{7.778314in}{5.156540in}}%
\pgfpathlineto{\pgfqpoint{7.728170in}{5.156009in}}%
\pgfusepath{stroke}%
\end{pgfscope}%
\begin{pgfscope}%
\pgfpathrectangle{\pgfqpoint{6.720588in}{4.155455in}}{\pgfqpoint{2.279412in}{2.004545in}}%
\pgfusepath{clip}%
\pgfsetbuttcap%
\pgfsetroundjoin%
\pgfsetlinewidth{2.340463pt}%
\definecolor{currentstroke}{rgb}{0.993248,0.906157,0.143936}%
\pgfsetstrokecolor{currentstroke}%
\pgfsetdash{}{0pt}%
\pgfpathmoveto{\pgfqpoint{7.728170in}{5.156009in}}%
\pgfpathlineto{\pgfqpoint{7.678027in}{5.155581in}}%
\pgfusepath{stroke}%
\end{pgfscope}%
\begin{pgfscope}%
\pgfpathrectangle{\pgfqpoint{6.720588in}{4.155455in}}{\pgfqpoint{2.279412in}{2.004545in}}%
\pgfusepath{clip}%
\pgfsetbuttcap%
\pgfsetroundjoin%
\pgfsetlinewidth{2.179625pt}%
\definecolor{currentstroke}{rgb}{0.993248,0.906157,0.143936}%
\pgfsetstrokecolor{currentstroke}%
\pgfsetdash{}{0pt}%
\pgfpathmoveto{\pgfqpoint{7.678027in}{5.155581in}}%
\pgfpathlineto{\pgfqpoint{7.627892in}{5.155089in}}%
\pgfusepath{stroke}%
\end{pgfscope}%
\begin{pgfscope}%
\pgfpathrectangle{\pgfqpoint{6.720588in}{4.155455in}}{\pgfqpoint{2.279412in}{2.004545in}}%
\pgfusepath{clip}%
\pgfsetbuttcap%
\pgfsetroundjoin%
\pgfsetlinewidth{0.323545pt}%
\definecolor{currentstroke}{rgb}{0.271305,0.019942,0.347269}%
\pgfsetstrokecolor{currentstroke}%
\pgfsetdash{}{0pt}%
\pgfpathmoveto{\pgfqpoint{8.680964in}{5.202834in}}%
\pgfpathlineto{\pgfqpoint{8.630887in}{5.201602in}}%
\pgfusepath{stroke}%
\end{pgfscope}%
\begin{pgfscope}%
\pgfpathrectangle{\pgfqpoint{6.720588in}{4.155455in}}{\pgfqpoint{2.279412in}{2.004545in}}%
\pgfusepath{clip}%
\pgfsetbuttcap%
\pgfsetroundjoin%
\pgfsetlinewidth{0.329608pt}%
\definecolor{currentstroke}{rgb}{0.272594,0.025563,0.353093}%
\pgfsetstrokecolor{currentstroke}%
\pgfsetdash{}{0pt}%
\pgfpathmoveto{\pgfqpoint{8.630887in}{5.201602in}}%
\pgfpathlineto{\pgfqpoint{8.580745in}{5.201453in}}%
\pgfusepath{stroke}%
\end{pgfscope}%
\begin{pgfscope}%
\pgfpathrectangle{\pgfqpoint{6.720588in}{4.155455in}}{\pgfqpoint{2.279412in}{2.004545in}}%
\pgfusepath{clip}%
\pgfsetbuttcap%
\pgfsetroundjoin%
\pgfsetlinewidth{0.350223pt}%
\definecolor{currentstroke}{rgb}{0.276022,0.044167,0.370164}%
\pgfsetstrokecolor{currentstroke}%
\pgfsetdash{}{0pt}%
\pgfpathmoveto{\pgfqpoint{8.580745in}{5.201453in}}%
\pgfpathlineto{\pgfqpoint{8.530604in}{5.201549in}}%
\pgfusepath{stroke}%
\end{pgfscope}%
\begin{pgfscope}%
\pgfpathrectangle{\pgfqpoint{6.720588in}{4.155455in}}{\pgfqpoint{2.279412in}{2.004545in}}%
\pgfusepath{clip}%
\pgfsetbuttcap%
\pgfsetroundjoin%
\pgfsetlinewidth{0.345792pt}%
\definecolor{currentstroke}{rgb}{0.274952,0.037752,0.364543}%
\pgfsetstrokecolor{currentstroke}%
\pgfsetdash{}{0pt}%
\pgfpathmoveto{\pgfqpoint{8.530604in}{5.201549in}}%
\pgfpathlineto{\pgfqpoint{8.480462in}{5.201579in}}%
\pgfusepath{stroke}%
\end{pgfscope}%
\begin{pgfscope}%
\pgfpathrectangle{\pgfqpoint{6.720588in}{4.155455in}}{\pgfqpoint{2.279412in}{2.004545in}}%
\pgfusepath{clip}%
\pgfsetbuttcap%
\pgfsetroundjoin%
\pgfsetlinewidth{0.373848pt}%
\definecolor{currentstroke}{rgb}{0.278791,0.062145,0.386592}%
\pgfsetstrokecolor{currentstroke}%
\pgfsetdash{}{0pt}%
\pgfpathmoveto{\pgfqpoint{8.480462in}{5.201579in}}%
\pgfpathlineto{\pgfqpoint{8.430313in}{5.201551in}}%
\pgfusepath{stroke}%
\end{pgfscope}%
\begin{pgfscope}%
\pgfpathrectangle{\pgfqpoint{6.720588in}{4.155455in}}{\pgfqpoint{2.279412in}{2.004545in}}%
\pgfusepath{clip}%
\pgfsetbuttcap%
\pgfsetroundjoin%
\pgfsetlinewidth{0.398029pt}%
\definecolor{currentstroke}{rgb}{0.281446,0.084320,0.407414}%
\pgfsetstrokecolor{currentstroke}%
\pgfsetdash{}{0pt}%
\pgfpathmoveto{\pgfqpoint{8.430313in}{5.201551in}}%
\pgfpathlineto{\pgfqpoint{8.380162in}{5.201396in}}%
\pgfusepath{stroke}%
\end{pgfscope}%
\begin{pgfscope}%
\pgfpathrectangle{\pgfqpoint{6.720588in}{4.155455in}}{\pgfqpoint{2.279412in}{2.004545in}}%
\pgfusepath{clip}%
\pgfsetbuttcap%
\pgfsetroundjoin%
\pgfsetlinewidth{0.449112pt}%
\definecolor{currentstroke}{rgb}{0.283229,0.120777,0.440584}%
\pgfsetstrokecolor{currentstroke}%
\pgfsetdash{}{0pt}%
\pgfpathmoveto{\pgfqpoint{8.380162in}{5.201396in}}%
\pgfpathlineto{\pgfqpoint{8.330012in}{5.201088in}}%
\pgfusepath{stroke}%
\end{pgfscope}%
\begin{pgfscope}%
\pgfpathrectangle{\pgfqpoint{6.720588in}{4.155455in}}{\pgfqpoint{2.279412in}{2.004545in}}%
\pgfusepath{clip}%
\pgfsetbuttcap%
\pgfsetroundjoin%
\pgfsetlinewidth{0.509566pt}%
\definecolor{currentstroke}{rgb}{0.280255,0.165693,0.476498}%
\pgfsetstrokecolor{currentstroke}%
\pgfsetdash{}{0pt}%
\pgfpathmoveto{\pgfqpoint{8.330012in}{5.201088in}}%
\pgfpathlineto{\pgfqpoint{8.279863in}{5.200632in}}%
\pgfusepath{stroke}%
\end{pgfscope}%
\begin{pgfscope}%
\pgfpathrectangle{\pgfqpoint{6.720588in}{4.155455in}}{\pgfqpoint{2.279412in}{2.004545in}}%
\pgfusepath{clip}%
\pgfsetbuttcap%
\pgfsetroundjoin%
\pgfsetlinewidth{0.591869pt}%
\definecolor{currentstroke}{rgb}{0.267968,0.223549,0.512008}%
\pgfsetstrokecolor{currentstroke}%
\pgfsetdash{}{0pt}%
\pgfpathmoveto{\pgfqpoint{8.279863in}{5.200632in}}%
\pgfpathlineto{\pgfqpoint{8.229715in}{5.200099in}}%
\pgfusepath{stroke}%
\end{pgfscope}%
\begin{pgfscope}%
\pgfpathrectangle{\pgfqpoint{6.720588in}{4.155455in}}{\pgfqpoint{2.279412in}{2.004545in}}%
\pgfusepath{clip}%
\pgfsetbuttcap%
\pgfsetroundjoin%
\pgfsetlinewidth{0.740481pt}%
\definecolor{currentstroke}{rgb}{0.229739,0.322361,0.545706}%
\pgfsetstrokecolor{currentstroke}%
\pgfsetdash{}{0pt}%
\pgfpathmoveto{\pgfqpoint{8.229715in}{5.200099in}}%
\pgfpathlineto{\pgfqpoint{8.179570in}{5.199352in}}%
\pgfusepath{stroke}%
\end{pgfscope}%
\begin{pgfscope}%
\pgfpathrectangle{\pgfqpoint{6.720588in}{4.155455in}}{\pgfqpoint{2.279412in}{2.004545in}}%
\pgfusepath{clip}%
\pgfsetbuttcap%
\pgfsetroundjoin%
\pgfsetlinewidth{0.937432pt}%
\definecolor{currentstroke}{rgb}{0.179019,0.433756,0.557430}%
\pgfsetstrokecolor{currentstroke}%
\pgfsetdash{}{0pt}%
\pgfpathmoveto{\pgfqpoint{8.179570in}{5.199352in}}%
\pgfpathlineto{\pgfqpoint{8.129434in}{5.198279in}}%
\pgfusepath{stroke}%
\end{pgfscope}%
\begin{pgfscope}%
\pgfpathrectangle{\pgfqpoint{6.720588in}{4.155455in}}{\pgfqpoint{2.279412in}{2.004545in}}%
\pgfusepath{clip}%
\pgfsetbuttcap%
\pgfsetroundjoin%
\pgfsetlinewidth{1.206388pt}%
\definecolor{currentstroke}{rgb}{0.124395,0.578002,0.548287}%
\pgfsetstrokecolor{currentstroke}%
\pgfsetdash{}{0pt}%
\pgfpathmoveto{\pgfqpoint{8.129434in}{5.198279in}}%
\pgfpathlineto{\pgfqpoint{8.079308in}{5.196866in}}%
\pgfusepath{stroke}%
\end{pgfscope}%
\begin{pgfscope}%
\pgfpathrectangle{\pgfqpoint{6.720588in}{4.155455in}}{\pgfqpoint{2.279412in}{2.004545in}}%
\pgfusepath{clip}%
\pgfsetbuttcap%
\pgfsetroundjoin%
\pgfsetlinewidth{1.572263pt}%
\definecolor{currentstroke}{rgb}{0.311925,0.767822,0.415586}%
\pgfsetstrokecolor{currentstroke}%
\pgfsetdash{}{0pt}%
\pgfpathmoveto{\pgfqpoint{8.079308in}{5.196866in}}%
\pgfpathlineto{\pgfqpoint{8.029199in}{5.195058in}}%
\pgfusepath{stroke}%
\end{pgfscope}%
\begin{pgfscope}%
\pgfpathrectangle{\pgfqpoint{6.720588in}{4.155455in}}{\pgfqpoint{2.279412in}{2.004545in}}%
\pgfusepath{clip}%
\pgfsetbuttcap%
\pgfsetroundjoin%
\pgfsetlinewidth{1.896808pt}%
\definecolor{currentstroke}{rgb}{0.762373,0.876424,0.137064}%
\pgfsetstrokecolor{currentstroke}%
\pgfsetdash{}{0pt}%
\pgfpathmoveto{\pgfqpoint{8.029199in}{5.195058in}}%
\pgfpathlineto{\pgfqpoint{7.979112in}{5.192862in}}%
\pgfusepath{stroke}%
\end{pgfscope}%
\begin{pgfscope}%
\pgfpathrectangle{\pgfqpoint{6.720588in}{4.155455in}}{\pgfqpoint{2.279412in}{2.004545in}}%
\pgfusepath{clip}%
\pgfsetbuttcap%
\pgfsetroundjoin%
\pgfsetlinewidth{2.136414pt}%
\definecolor{currentstroke}{rgb}{0.993248,0.906157,0.143936}%
\pgfsetstrokecolor{currentstroke}%
\pgfsetdash{}{0pt}%
\pgfpathmoveto{\pgfqpoint{7.979112in}{5.192862in}}%
\pgfpathlineto{\pgfqpoint{7.929047in}{5.190319in}}%
\pgfusepath{stroke}%
\end{pgfscope}%
\begin{pgfscope}%
\pgfpathrectangle{\pgfqpoint{6.720588in}{4.155455in}}{\pgfqpoint{2.279412in}{2.004545in}}%
\pgfusepath{clip}%
\pgfsetbuttcap%
\pgfsetroundjoin%
\pgfsetlinewidth{0.316801pt}%
\definecolor{currentstroke}{rgb}{0.269944,0.014625,0.341379}%
\pgfsetstrokecolor{currentstroke}%
\pgfsetdash{}{0pt}%
\pgfpathmoveto{\pgfqpoint{8.680964in}{5.338154in}}%
\pgfpathlineto{\pgfqpoint{8.631013in}{5.339550in}}%
\pgfusepath{stroke}%
\end{pgfscope}%
\begin{pgfscope}%
\pgfpathrectangle{\pgfqpoint{6.720588in}{4.155455in}}{\pgfqpoint{2.279412in}{2.004545in}}%
\pgfusepath{clip}%
\pgfsetbuttcap%
\pgfsetroundjoin%
\pgfsetlinewidth{0.328058pt}%
\definecolor{currentstroke}{rgb}{0.271305,0.019942,0.347269}%
\pgfsetstrokecolor{currentstroke}%
\pgfsetdash{}{0pt}%
\pgfpathmoveto{\pgfqpoint{8.631013in}{5.339550in}}%
\pgfpathlineto{\pgfqpoint{8.580876in}{5.338803in}}%
\pgfusepath{stroke}%
\end{pgfscope}%
\begin{pgfscope}%
\pgfpathrectangle{\pgfqpoint{6.720588in}{4.155455in}}{\pgfqpoint{2.279412in}{2.004545in}}%
\pgfusepath{clip}%
\pgfsetbuttcap%
\pgfsetroundjoin%
\pgfsetlinewidth{0.329268pt}%
\definecolor{currentstroke}{rgb}{0.272594,0.025563,0.353093}%
\pgfsetstrokecolor{currentstroke}%
\pgfsetdash{}{0pt}%
\pgfpathmoveto{\pgfqpoint{8.580876in}{5.338803in}}%
\pgfpathlineto{\pgfqpoint{8.530755in}{5.337396in}}%
\pgfusepath{stroke}%
\end{pgfscope}%
\begin{pgfscope}%
\pgfpathrectangle{\pgfqpoint{6.720588in}{4.155455in}}{\pgfqpoint{2.279412in}{2.004545in}}%
\pgfusepath{clip}%
\pgfsetbuttcap%
\pgfsetroundjoin%
\pgfsetlinewidth{0.337879pt}%
\definecolor{currentstroke}{rgb}{0.273809,0.031497,0.358853}%
\pgfsetstrokecolor{currentstroke}%
\pgfsetdash{}{0pt}%
\pgfpathmoveto{\pgfqpoint{8.530755in}{5.337396in}}%
\pgfpathlineto{\pgfqpoint{8.480628in}{5.336102in}}%
\pgfusepath{stroke}%
\end{pgfscope}%
\begin{pgfscope}%
\pgfpathrectangle{\pgfqpoint{6.720588in}{4.155455in}}{\pgfqpoint{2.279412in}{2.004545in}}%
\pgfusepath{clip}%
\pgfsetbuttcap%
\pgfsetroundjoin%
\pgfsetlinewidth{0.353350pt}%
\definecolor{currentstroke}{rgb}{0.276022,0.044167,0.370164}%
\pgfsetstrokecolor{currentstroke}%
\pgfsetdash{}{0pt}%
\pgfpathmoveto{\pgfqpoint{8.480628in}{5.336102in}}%
\pgfpathlineto{\pgfqpoint{8.430486in}{5.335278in}}%
\pgfusepath{stroke}%
\end{pgfscope}%
\begin{pgfscope}%
\pgfpathrectangle{\pgfqpoint{6.720588in}{4.155455in}}{\pgfqpoint{2.279412in}{2.004545in}}%
\pgfusepath{clip}%
\pgfsetbuttcap%
\pgfsetroundjoin%
\pgfsetlinewidth{0.384662pt}%
\definecolor{currentstroke}{rgb}{0.280267,0.073417,0.397163}%
\pgfsetstrokecolor{currentstroke}%
\pgfsetdash{}{0pt}%
\pgfpathmoveto{\pgfqpoint{8.430486in}{5.335278in}}%
\pgfpathlineto{\pgfqpoint{8.380352in}{5.334174in}}%
\pgfusepath{stroke}%
\end{pgfscope}%
\begin{pgfscope}%
\pgfpathrectangle{\pgfqpoint{6.720588in}{4.155455in}}{\pgfqpoint{2.279412in}{2.004545in}}%
\pgfusepath{clip}%
\pgfsetbuttcap%
\pgfsetroundjoin%
\pgfsetlinewidth{0.422772pt}%
\definecolor{currentstroke}{rgb}{0.282656,0.100196,0.422160}%
\pgfsetstrokecolor{currentstroke}%
\pgfsetdash{}{0pt}%
\pgfpathmoveto{\pgfqpoint{8.380352in}{5.334174in}}%
\pgfpathlineto{\pgfqpoint{8.330226in}{5.332789in}}%
\pgfusepath{stroke}%
\end{pgfscope}%
\begin{pgfscope}%
\pgfpathrectangle{\pgfqpoint{6.720588in}{4.155455in}}{\pgfqpoint{2.279412in}{2.004545in}}%
\pgfusepath{clip}%
\pgfsetbuttcap%
\pgfsetroundjoin%
\pgfsetlinewidth{0.466906pt}%
\definecolor{currentstroke}{rgb}{0.282884,0.135920,0.453427}%
\pgfsetstrokecolor{currentstroke}%
\pgfsetdash{}{0pt}%
\pgfpathmoveto{\pgfqpoint{8.330226in}{5.332789in}}%
\pgfpathlineto{\pgfqpoint{8.280103in}{5.331295in}}%
\pgfusepath{stroke}%
\end{pgfscope}%
\begin{pgfscope}%
\pgfpathrectangle{\pgfqpoint{6.720588in}{4.155455in}}{\pgfqpoint{2.279412in}{2.004545in}}%
\pgfusepath{clip}%
\pgfsetbuttcap%
\pgfsetroundjoin%
\pgfsetlinewidth{0.523430pt}%
\definecolor{currentstroke}{rgb}{0.278826,0.175490,0.483397}%
\pgfsetstrokecolor{currentstroke}%
\pgfsetdash{}{0pt}%
\pgfpathmoveto{\pgfqpoint{8.280103in}{5.331295in}}%
\pgfpathlineto{\pgfqpoint{8.229998in}{5.329411in}}%
\pgfusepath{stroke}%
\end{pgfscope}%
\begin{pgfscope}%
\pgfpathrectangle{\pgfqpoint{6.720588in}{4.155455in}}{\pgfqpoint{2.279412in}{2.004545in}}%
\pgfusepath{clip}%
\pgfsetbuttcap%
\pgfsetroundjoin%
\pgfsetlinewidth{0.588780pt}%
\definecolor{currentstroke}{rgb}{0.269308,0.218818,0.509577}%
\pgfsetstrokecolor{currentstroke}%
\pgfsetdash{}{0pt}%
\pgfpathmoveto{\pgfqpoint{8.229998in}{5.329411in}}%
\pgfpathlineto{\pgfqpoint{8.179921in}{5.327023in}}%
\pgfusepath{stroke}%
\end{pgfscope}%
\begin{pgfscope}%
\pgfpathrectangle{\pgfqpoint{6.720588in}{4.155455in}}{\pgfqpoint{2.279412in}{2.004545in}}%
\pgfusepath{clip}%
\pgfsetbuttcap%
\pgfsetroundjoin%
\pgfsetlinewidth{0.695193pt}%
\definecolor{currentstroke}{rgb}{0.243113,0.292092,0.538516}%
\pgfsetstrokecolor{currentstroke}%
\pgfsetdash{}{0pt}%
\pgfpathmoveto{\pgfqpoint{8.179921in}{5.327023in}}%
\pgfpathlineto{\pgfqpoint{8.129891in}{5.323988in}}%
\pgfusepath{stroke}%
\end{pgfscope}%
\begin{pgfscope}%
\pgfpathrectangle{\pgfqpoint{6.720588in}{4.155455in}}{\pgfqpoint{2.279412in}{2.004545in}}%
\pgfusepath{clip}%
\pgfsetbuttcap%
\pgfsetroundjoin%
\pgfsetlinewidth{0.726318pt}%
\definecolor{currentstroke}{rgb}{0.235526,0.309527,0.542944}%
\pgfsetstrokecolor{currentstroke}%
\pgfsetdash{}{0pt}%
\pgfpathmoveto{\pgfqpoint{8.129891in}{5.323988in}}%
\pgfpathlineto{\pgfqpoint{8.079947in}{5.320020in}}%
\pgfusepath{stroke}%
\end{pgfscope}%
\begin{pgfscope}%
\pgfpathrectangle{\pgfqpoint{6.720588in}{4.155455in}}{\pgfqpoint{2.279412in}{2.004545in}}%
\pgfusepath{clip}%
\pgfsetbuttcap%
\pgfsetroundjoin%
\pgfsetlinewidth{0.307436pt}%
\definecolor{currentstroke}{rgb}{0.267004,0.004874,0.329415}%
\pgfsetstrokecolor{currentstroke}%
\pgfsetdash{}{0pt}%
\pgfpathmoveto{\pgfqpoint{8.629673in}{4.796873in}}%
\pgfpathlineto{\pgfqpoint{8.581614in}{4.797026in}}%
\pgfusepath{stroke}%
\end{pgfscope}%
\begin{pgfscope}%
\pgfpathrectangle{\pgfqpoint{6.720588in}{4.155455in}}{\pgfqpoint{2.279412in}{2.004545in}}%
\pgfusepath{clip}%
\pgfsetbuttcap%
\pgfsetroundjoin%
\pgfsetlinewidth{0.318523pt}%
\definecolor{currentstroke}{rgb}{0.269944,0.014625,0.341379}%
\pgfsetstrokecolor{currentstroke}%
\pgfsetdash{}{0pt}%
\pgfpathmoveto{\pgfqpoint{8.581614in}{4.797026in}}%
\pgfpathlineto{\pgfqpoint{8.538883in}{4.799762in}}%
\pgfusepath{stroke}%
\end{pgfscope}%
\begin{pgfscope}%
\pgfpathrectangle{\pgfqpoint{6.720588in}{4.155455in}}{\pgfqpoint{2.279412in}{2.004545in}}%
\pgfusepath{clip}%
\pgfsetbuttcap%
\pgfsetroundjoin%
\pgfsetlinewidth{0.331069pt}%
\definecolor{currentstroke}{rgb}{0.272594,0.025563,0.353093}%
\pgfsetstrokecolor{currentstroke}%
\pgfsetdash{}{0pt}%
\pgfpathmoveto{\pgfqpoint{8.538883in}{4.799762in}}%
\pgfpathlineto{\pgfqpoint{8.488807in}{4.801509in}}%
\pgfusepath{stroke}%
\end{pgfscope}%
\begin{pgfscope}%
\pgfpathrectangle{\pgfqpoint{6.720588in}{4.155455in}}{\pgfqpoint{2.279412in}{2.004545in}}%
\pgfusepath{clip}%
\pgfsetbuttcap%
\pgfsetroundjoin%
\pgfsetlinewidth{0.330209pt}%
\definecolor{currentstroke}{rgb}{0.272594,0.025563,0.353093}%
\pgfsetstrokecolor{currentstroke}%
\pgfsetdash{}{0pt}%
\pgfpathmoveto{\pgfqpoint{8.488807in}{4.801509in}}%
\pgfpathlineto{\pgfqpoint{8.438795in}{4.804558in}}%
\pgfusepath{stroke}%
\end{pgfscope}%
\begin{pgfscope}%
\pgfpathrectangle{\pgfqpoint{6.720588in}{4.155455in}}{\pgfqpoint{2.279412in}{2.004545in}}%
\pgfusepath{clip}%
\pgfsetbuttcap%
\pgfsetroundjoin%
\pgfsetlinewidth{0.342854pt}%
\definecolor{currentstroke}{rgb}{0.274952,0.037752,0.364543}%
\pgfsetstrokecolor{currentstroke}%
\pgfsetdash{}{0pt}%
\pgfpathmoveto{\pgfqpoint{8.438795in}{4.804558in}}%
\pgfpathlineto{\pgfqpoint{8.388741in}{4.807034in}}%
\pgfusepath{stroke}%
\end{pgfscope}%
\begin{pgfscope}%
\pgfpathrectangle{\pgfqpoint{6.720588in}{4.155455in}}{\pgfqpoint{2.279412in}{2.004545in}}%
\pgfusepath{clip}%
\pgfsetbuttcap%
\pgfsetroundjoin%
\pgfsetlinewidth{0.367776pt}%
\definecolor{currentstroke}{rgb}{0.277941,0.056324,0.381191}%
\pgfsetstrokecolor{currentstroke}%
\pgfsetdash{}{0pt}%
\pgfpathmoveto{\pgfqpoint{8.388741in}{4.807034in}}%
\pgfpathlineto{\pgfqpoint{8.338664in}{4.809404in}}%
\pgfusepath{stroke}%
\end{pgfscope}%
\begin{pgfscope}%
\pgfpathrectangle{\pgfqpoint{6.720588in}{4.155455in}}{\pgfqpoint{2.279412in}{2.004545in}}%
\pgfusepath{clip}%
\pgfsetbuttcap%
\pgfsetroundjoin%
\pgfsetlinewidth{0.389477pt}%
\definecolor{currentstroke}{rgb}{0.280267,0.073417,0.397163}%
\pgfsetstrokecolor{currentstroke}%
\pgfsetdash{}{0pt}%
\pgfpathmoveto{\pgfqpoint{8.338664in}{4.809404in}}%
\pgfpathlineto{\pgfqpoint{8.288613in}{4.812180in}}%
\pgfusepath{stroke}%
\end{pgfscope}%
\begin{pgfscope}%
\pgfpathrectangle{\pgfqpoint{6.720588in}{4.155455in}}{\pgfqpoint{2.279412in}{2.004545in}}%
\pgfusepath{clip}%
\pgfsetbuttcap%
\pgfsetroundjoin%
\pgfsetlinewidth{0.427987pt}%
\definecolor{currentstroke}{rgb}{0.282910,0.105393,0.426902}%
\pgfsetstrokecolor{currentstroke}%
\pgfsetdash{}{0pt}%
\pgfpathmoveto{\pgfqpoint{8.288613in}{4.812180in}}%
\pgfpathlineto{\pgfqpoint{8.238606in}{4.815505in}}%
\pgfusepath{stroke}%
\end{pgfscope}%
\begin{pgfscope}%
\pgfpathrectangle{\pgfqpoint{6.720588in}{4.155455in}}{\pgfqpoint{2.279412in}{2.004545in}}%
\pgfusepath{clip}%
\pgfsetbuttcap%
\pgfsetroundjoin%
\pgfsetlinewidth{0.454375pt}%
\definecolor{currentstroke}{rgb}{0.283187,0.125848,0.444960}%
\pgfsetstrokecolor{currentstroke}%
\pgfsetdash{}{0pt}%
\pgfpathmoveto{\pgfqpoint{8.238606in}{4.815505in}}%
\pgfpathlineto{\pgfqpoint{8.188659in}{4.819429in}}%
\pgfusepath{stroke}%
\end{pgfscope}%
\begin{pgfscope}%
\pgfpathrectangle{\pgfqpoint{6.720588in}{4.155455in}}{\pgfqpoint{2.279412in}{2.004545in}}%
\pgfusepath{clip}%
\pgfsetbuttcap%
\pgfsetroundjoin%
\pgfsetlinewidth{0.463344pt}%
\definecolor{currentstroke}{rgb}{0.283072,0.130895,0.449241}%
\pgfsetstrokecolor{currentstroke}%
\pgfsetdash{}{0pt}%
\pgfpathmoveto{\pgfqpoint{8.188659in}{4.819429in}}%
\pgfpathlineto{\pgfqpoint{8.138786in}{4.824012in}}%
\pgfusepath{stroke}%
\end{pgfscope}%
\begin{pgfscope}%
\pgfpathrectangle{\pgfqpoint{6.720588in}{4.155455in}}{\pgfqpoint{2.279412in}{2.004545in}}%
\pgfusepath{clip}%
\pgfsetbuttcap%
\pgfsetroundjoin%
\pgfsetlinewidth{0.506488pt}%
\definecolor{currentstroke}{rgb}{0.280868,0.160771,0.472899}%
\pgfsetstrokecolor{currentstroke}%
\pgfsetdash{}{0pt}%
\pgfpathmoveto{\pgfqpoint{8.138786in}{4.824012in}}%
\pgfpathlineto{\pgfqpoint{8.089100in}{4.829892in}}%
\pgfusepath{stroke}%
\end{pgfscope}%
\begin{pgfscope}%
\pgfpathrectangle{\pgfqpoint{6.720588in}{4.155455in}}{\pgfqpoint{2.279412in}{2.004545in}}%
\pgfusepath{clip}%
\pgfsetbuttcap%
\pgfsetroundjoin%
\pgfsetlinewidth{0.530222pt}%
\definecolor{currentstroke}{rgb}{0.278012,0.180367,0.486697}%
\pgfsetstrokecolor{currentstroke}%
\pgfsetdash{}{0pt}%
\pgfpathmoveto{\pgfqpoint{8.089100in}{4.829892in}}%
\pgfpathlineto{\pgfqpoint{8.039616in}{4.836992in}}%
\pgfusepath{stroke}%
\end{pgfscope}%
\begin{pgfscope}%
\pgfpathrectangle{\pgfqpoint{6.720588in}{4.155455in}}{\pgfqpoint{2.279412in}{2.004545in}}%
\pgfusepath{clip}%
\pgfsetbuttcap%
\pgfsetroundjoin%
\pgfsetlinewidth{0.490843pt}%
\definecolor{currentstroke}{rgb}{0.281887,0.150881,0.465405}%
\pgfsetstrokecolor{currentstroke}%
\pgfsetdash{}{0pt}%
\pgfpathmoveto{\pgfqpoint{8.039616in}{4.836992in}}%
\pgfpathlineto{\pgfqpoint{7.990513in}{4.845819in}}%
\pgfusepath{stroke}%
\end{pgfscope}%
\begin{pgfscope}%
\pgfpathrectangle{\pgfqpoint{6.720588in}{4.155455in}}{\pgfqpoint{2.279412in}{2.004545in}}%
\pgfusepath{clip}%
\pgfsetbuttcap%
\pgfsetroundjoin%
\pgfsetlinewidth{0.548378pt}%
\definecolor{currentstroke}{rgb}{0.275191,0.194905,0.496005}%
\pgfsetstrokecolor{currentstroke}%
\pgfsetdash{}{0pt}%
\pgfpathmoveto{\pgfqpoint{7.990513in}{4.845819in}}%
\pgfpathlineto{\pgfqpoint{7.942008in}{4.856965in}}%
\pgfusepath{stroke}%
\end{pgfscope}%
\begin{pgfscope}%
\pgfpathrectangle{\pgfqpoint{6.720588in}{4.155455in}}{\pgfqpoint{2.279412in}{2.004545in}}%
\pgfusepath{clip}%
\pgfsetbuttcap%
\pgfsetroundjoin%
\pgfsetlinewidth{0.522149pt}%
\definecolor{currentstroke}{rgb}{0.278826,0.175490,0.483397}%
\pgfsetstrokecolor{currentstroke}%
\pgfsetdash{}{0pt}%
\pgfpathmoveto{\pgfqpoint{7.942008in}{4.856965in}}%
\pgfpathlineto{\pgfqpoint{7.894123in}{4.870018in}}%
\pgfusepath{stroke}%
\end{pgfscope}%
\begin{pgfscope}%
\pgfpathrectangle{\pgfqpoint{6.720588in}{4.155455in}}{\pgfqpoint{2.279412in}{2.004545in}}%
\pgfusepath{clip}%
\pgfsetbuttcap%
\pgfsetroundjoin%
\pgfsetlinewidth{0.659858pt}%
\definecolor{currentstroke}{rgb}{0.252194,0.269783,0.531579}%
\pgfsetstrokecolor{currentstroke}%
\pgfsetdash{}{0pt}%
\pgfpathmoveto{\pgfqpoint{7.894123in}{4.870018in}}%
\pgfpathlineto{\pgfqpoint{7.847710in}{4.886518in}}%
\pgfusepath{stroke}%
\end{pgfscope}%
\begin{pgfscope}%
\pgfpathrectangle{\pgfqpoint{6.720588in}{4.155455in}}{\pgfqpoint{2.279412in}{2.004545in}}%
\pgfusepath{clip}%
\pgfsetbuttcap%
\pgfsetroundjoin%
\pgfsetlinewidth{0.674110pt}%
\definecolor{currentstroke}{rgb}{0.248629,0.278775,0.534556}%
\pgfsetstrokecolor{currentstroke}%
\pgfsetdash{}{0pt}%
\pgfpathmoveto{\pgfqpoint{7.847710in}{4.886518in}}%
\pgfpathlineto{\pgfqpoint{7.803358in}{4.906650in}}%
\pgfusepath{stroke}%
\end{pgfscope}%
\begin{pgfscope}%
\pgfpathrectangle{\pgfqpoint{6.720588in}{4.155455in}}{\pgfqpoint{2.279412in}{2.004545in}}%
\pgfusepath{clip}%
\pgfsetbuttcap%
\pgfsetroundjoin%
\pgfsetlinewidth{0.545389pt}%
\definecolor{currentstroke}{rgb}{0.276194,0.190074,0.493001}%
\pgfsetstrokecolor{currentstroke}%
\pgfsetdash{}{0pt}%
\pgfpathmoveto{\pgfqpoint{7.803358in}{4.906650in}}%
\pgfpathlineto{\pgfqpoint{7.759809in}{4.928147in}}%
\pgfusepath{stroke}%
\end{pgfscope}%
\begin{pgfscope}%
\pgfpathrectangle{\pgfqpoint{6.720588in}{4.155455in}}{\pgfqpoint{2.279412in}{2.004545in}}%
\pgfusepath{clip}%
\pgfsetbuttcap%
\pgfsetroundjoin%
\pgfsetlinewidth{1.295505pt}%
\definecolor{currentstroke}{rgb}{0.120638,0.625828,0.533488}%
\pgfsetstrokecolor{currentstroke}%
\pgfsetdash{}{0pt}%
\pgfpathmoveto{\pgfqpoint{7.759809in}{4.928147in}}%
\pgfpathlineto{\pgfqpoint{7.716871in}{4.950881in}}%
\pgfusepath{stroke}%
\end{pgfscope}%
\begin{pgfscope}%
\pgfpathrectangle{\pgfqpoint{6.720588in}{4.155455in}}{\pgfqpoint{2.279412in}{2.004545in}}%
\pgfusepath{clip}%
\pgfsetbuttcap%
\pgfsetroundjoin%
\pgfsetlinewidth{1.482175pt}%
\definecolor{currentstroke}{rgb}{0.220124,0.725509,0.466226}%
\pgfsetstrokecolor{currentstroke}%
\pgfsetdash{}{0pt}%
\pgfpathmoveto{\pgfqpoint{7.716871in}{4.950881in}}%
\pgfpathlineto{\pgfqpoint{7.675235in}{4.975388in}}%
\pgfusepath{stroke}%
\end{pgfscope}%
\begin{pgfscope}%
\pgfpathrectangle{\pgfqpoint{6.720588in}{4.155455in}}{\pgfqpoint{2.279412in}{2.004545in}}%
\pgfusepath{clip}%
\pgfsetbuttcap%
\pgfsetroundjoin%
\pgfsetlinewidth{1.400540pt}%
\definecolor{currentstroke}{rgb}{0.157851,0.683765,0.501686}%
\pgfsetstrokecolor{currentstroke}%
\pgfsetdash{}{0pt}%
\pgfpathmoveto{\pgfqpoint{7.675235in}{4.975388in}}%
\pgfpathlineto{\pgfqpoint{7.634535in}{5.001114in}}%
\pgfusepath{stroke}%
\end{pgfscope}%
\begin{pgfscope}%
\pgfpathrectangle{\pgfqpoint{6.720588in}{4.155455in}}{\pgfqpoint{2.279412in}{2.004545in}}%
\pgfusepath{clip}%
\pgfsetbuttcap%
\pgfsetroundjoin%
\pgfsetlinewidth{1.534453pt}%
\definecolor{currentstroke}{rgb}{0.266941,0.748751,0.440573}%
\pgfsetstrokecolor{currentstroke}%
\pgfsetdash{}{0pt}%
\pgfpathmoveto{\pgfqpoint{7.634535in}{5.001114in}}%
\pgfpathlineto{\pgfqpoint{7.592831in}{5.025458in}}%
\pgfusepath{stroke}%
\end{pgfscope}%
\begin{pgfscope}%
\pgfpathrectangle{\pgfqpoint{6.720588in}{4.155455in}}{\pgfqpoint{2.279412in}{2.004545in}}%
\pgfusepath{clip}%
\pgfsetbuttcap%
\pgfsetroundjoin%
\pgfsetlinewidth{0.325777pt}%
\definecolor{currentstroke}{rgb}{0.271305,0.019942,0.347269}%
\pgfsetstrokecolor{currentstroke}%
\pgfsetdash{}{0pt}%
\pgfpathmoveto{\pgfqpoint{8.629673in}{5.112620in}}%
\pgfpathlineto{\pgfqpoint{8.579804in}{5.115288in}}%
\pgfusepath{stroke}%
\end{pgfscope}%
\begin{pgfscope}%
\pgfpathrectangle{\pgfqpoint{6.720588in}{4.155455in}}{\pgfqpoint{2.279412in}{2.004545in}}%
\pgfusepath{clip}%
\pgfsetbuttcap%
\pgfsetroundjoin%
\pgfsetlinewidth{0.315741pt}%
\definecolor{currentstroke}{rgb}{0.269944,0.014625,0.341379}%
\pgfsetstrokecolor{currentstroke}%
\pgfsetdash{}{0pt}%
\pgfpathmoveto{\pgfqpoint{8.579804in}{5.115288in}}%
\pgfpathlineto{\pgfqpoint{8.529878in}{5.117183in}}%
\pgfusepath{stroke}%
\end{pgfscope}%
\begin{pgfscope}%
\pgfpathrectangle{\pgfqpoint{6.720588in}{4.155455in}}{\pgfqpoint{2.279412in}{2.004545in}}%
\pgfusepath{clip}%
\pgfsetbuttcap%
\pgfsetroundjoin%
\pgfsetlinewidth{0.348984pt}%
\definecolor{currentstroke}{rgb}{0.274952,0.037752,0.364543}%
\pgfsetstrokecolor{currentstroke}%
\pgfsetdash{}{0pt}%
\pgfpathmoveto{\pgfqpoint{8.529878in}{5.117183in}}%
\pgfpathlineto{\pgfqpoint{8.479729in}{5.116951in}}%
\pgfusepath{stroke}%
\end{pgfscope}%
\begin{pgfscope}%
\pgfpathrectangle{\pgfqpoint{6.720588in}{4.155455in}}{\pgfqpoint{2.279412in}{2.004545in}}%
\pgfusepath{clip}%
\pgfsetbuttcap%
\pgfsetroundjoin%
\pgfsetlinewidth{0.373856pt}%
\definecolor{currentstroke}{rgb}{0.278791,0.062145,0.386592}%
\pgfsetstrokecolor{currentstroke}%
\pgfsetdash{}{0pt}%
\pgfpathmoveto{\pgfqpoint{8.479729in}{5.116951in}}%
\pgfpathlineto{\pgfqpoint{8.429581in}{5.116829in}}%
\pgfusepath{stroke}%
\end{pgfscope}%
\begin{pgfscope}%
\pgfpathrectangle{\pgfqpoint{6.720588in}{4.155455in}}{\pgfqpoint{2.279412in}{2.004545in}}%
\pgfusepath{clip}%
\pgfsetbuttcap%
\pgfsetroundjoin%
\pgfsetlinewidth{0.398590pt}%
\definecolor{currentstroke}{rgb}{0.281446,0.084320,0.407414}%
\pgfsetstrokecolor{currentstroke}%
\pgfsetdash{}{0pt}%
\pgfpathmoveto{\pgfqpoint{8.429581in}{5.116829in}}%
\pgfpathlineto{\pgfqpoint{8.379432in}{5.117217in}}%
\pgfusepath{stroke}%
\end{pgfscope}%
\begin{pgfscope}%
\pgfpathrectangle{\pgfqpoint{6.720588in}{4.155455in}}{\pgfqpoint{2.279412in}{2.004545in}}%
\pgfusepath{clip}%
\pgfsetbuttcap%
\pgfsetroundjoin%
\pgfsetlinewidth{0.454951pt}%
\definecolor{currentstroke}{rgb}{0.283187,0.125848,0.444960}%
\pgfsetstrokecolor{currentstroke}%
\pgfsetdash{}{0pt}%
\pgfpathmoveto{\pgfqpoint{8.379432in}{5.117217in}}%
\pgfpathlineto{\pgfqpoint{8.329282in}{5.117490in}}%
\pgfusepath{stroke}%
\end{pgfscope}%
\begin{pgfscope}%
\pgfpathrectangle{\pgfqpoint{6.720588in}{4.155455in}}{\pgfqpoint{2.279412in}{2.004545in}}%
\pgfusepath{clip}%
\pgfsetbuttcap%
\pgfsetroundjoin%
\pgfsetlinewidth{0.510605pt}%
\definecolor{currentstroke}{rgb}{0.280255,0.165693,0.476498}%
\pgfsetstrokecolor{currentstroke}%
\pgfsetdash{}{0pt}%
\pgfpathmoveto{\pgfqpoint{8.329282in}{5.117490in}}%
\pgfpathlineto{\pgfqpoint{8.279130in}{5.117648in}}%
\pgfusepath{stroke}%
\end{pgfscope}%
\begin{pgfscope}%
\pgfpathrectangle{\pgfqpoint{6.720588in}{4.155455in}}{\pgfqpoint{2.279412in}{2.004545in}}%
\pgfusepath{clip}%
\pgfsetbuttcap%
\pgfsetroundjoin%
\pgfsetlinewidth{0.593142pt}%
\definecolor{currentstroke}{rgb}{0.267968,0.223549,0.512008}%
\pgfsetstrokecolor{currentstroke}%
\pgfsetdash{}{0pt}%
\pgfpathmoveto{\pgfqpoint{8.279130in}{5.117648in}}%
\pgfpathlineto{\pgfqpoint{8.228979in}{5.117877in}}%
\pgfusepath{stroke}%
\end{pgfscope}%
\begin{pgfscope}%
\pgfpathrectangle{\pgfqpoint{6.720588in}{4.155455in}}{\pgfqpoint{2.279412in}{2.004545in}}%
\pgfusepath{clip}%
\pgfsetbuttcap%
\pgfsetroundjoin%
\pgfsetlinewidth{0.761016pt}%
\definecolor{currentstroke}{rgb}{0.223925,0.334994,0.548053}%
\pgfsetstrokecolor{currentstroke}%
\pgfsetdash{}{0pt}%
\pgfpathmoveto{\pgfqpoint{8.228979in}{5.117877in}}%
\pgfpathlineto{\pgfqpoint{8.178828in}{5.118156in}}%
\pgfusepath{stroke}%
\end{pgfscope}%
\begin{pgfscope}%
\pgfpathrectangle{\pgfqpoint{6.720588in}{4.155455in}}{\pgfqpoint{2.279412in}{2.004545in}}%
\pgfusepath{clip}%
\pgfsetbuttcap%
\pgfsetroundjoin%
\pgfsetlinewidth{0.961852pt}%
\definecolor{currentstroke}{rgb}{0.172719,0.448791,0.557885}%
\pgfsetstrokecolor{currentstroke}%
\pgfsetdash{}{0pt}%
\pgfpathmoveto{\pgfqpoint{8.178828in}{5.118156in}}%
\pgfpathlineto{\pgfqpoint{8.128680in}{5.118673in}}%
\pgfusepath{stroke}%
\end{pgfscope}%
\begin{pgfscope}%
\pgfpathrectangle{\pgfqpoint{6.720588in}{4.155455in}}{\pgfqpoint{2.279412in}{2.004545in}}%
\pgfusepath{clip}%
\pgfsetbuttcap%
\pgfsetroundjoin%
\pgfsetlinewidth{1.262428pt}%
\definecolor{currentstroke}{rgb}{0.119423,0.611141,0.538982}%
\pgfsetstrokecolor{currentstroke}%
\pgfsetdash{}{0pt}%
\pgfpathmoveto{\pgfqpoint{8.128680in}{5.118673in}}%
\pgfpathlineto{\pgfqpoint{8.078536in}{5.119447in}}%
\pgfusepath{stroke}%
\end{pgfscope}%
\begin{pgfscope}%
\pgfpathrectangle{\pgfqpoint{6.720588in}{4.155455in}}{\pgfqpoint{2.279412in}{2.004545in}}%
\pgfusepath{clip}%
\pgfsetbuttcap%
\pgfsetroundjoin%
\pgfsetlinewidth{1.632054pt}%
\definecolor{currentstroke}{rgb}{0.377779,0.791781,0.377939}%
\pgfsetstrokecolor{currentstroke}%
\pgfsetdash{}{0pt}%
\pgfpathmoveto{\pgfqpoint{8.078536in}{5.119447in}}%
\pgfpathlineto{\pgfqpoint{8.028398in}{5.120474in}}%
\pgfusepath{stroke}%
\end{pgfscope}%
\begin{pgfscope}%
\pgfpathrectangle{\pgfqpoint{6.720588in}{4.155455in}}{\pgfqpoint{2.279412in}{2.004545in}}%
\pgfusepath{clip}%
\pgfsetbuttcap%
\pgfsetroundjoin%
\pgfsetlinewidth{1.981989pt}%
\definecolor{currentstroke}{rgb}{0.886271,0.892374,0.095374}%
\pgfsetstrokecolor{currentstroke}%
\pgfsetdash{}{0pt}%
\pgfpathmoveto{\pgfqpoint{8.028398in}{5.120474in}}%
\pgfpathlineto{\pgfqpoint{7.978270in}{5.121794in}}%
\pgfusepath{stroke}%
\end{pgfscope}%
\begin{pgfscope}%
\pgfpathrectangle{\pgfqpoint{6.720588in}{4.155455in}}{\pgfqpoint{2.279412in}{2.004545in}}%
\pgfusepath{clip}%
\pgfsetbuttcap%
\pgfsetroundjoin%
\pgfsetlinewidth{2.180273pt}%
\definecolor{currentstroke}{rgb}{0.993248,0.906157,0.143936}%
\pgfsetstrokecolor{currentstroke}%
\pgfsetdash{}{0pt}%
\pgfpathmoveto{\pgfqpoint{7.978270in}{5.121794in}}%
\pgfpathlineto{\pgfqpoint{7.928154in}{5.123381in}}%
\pgfusepath{stroke}%
\end{pgfscope}%
\begin{pgfscope}%
\pgfpathrectangle{\pgfqpoint{6.720588in}{4.155455in}}{\pgfqpoint{2.279412in}{2.004545in}}%
\pgfusepath{clip}%
\pgfsetbuttcap%
\pgfsetroundjoin%
\pgfsetlinewidth{0.314399pt}%
\definecolor{currentstroke}{rgb}{0.268510,0.009605,0.335427}%
\pgfsetstrokecolor{currentstroke}%
\pgfsetdash{}{0pt}%
\pgfpathmoveto{\pgfqpoint{8.629673in}{5.428368in}}%
\pgfpathlineto{\pgfqpoint{8.579644in}{5.426530in}}%
\pgfusepath{stroke}%
\end{pgfscope}%
\begin{pgfscope}%
\pgfpathrectangle{\pgfqpoint{6.720588in}{4.155455in}}{\pgfqpoint{2.279412in}{2.004545in}}%
\pgfusepath{clip}%
\pgfsetbuttcap%
\pgfsetroundjoin%
\pgfsetlinewidth{0.324575pt}%
\definecolor{currentstroke}{rgb}{0.271305,0.019942,0.347269}%
\pgfsetstrokecolor{currentstroke}%
\pgfsetdash{}{0pt}%
\pgfpathmoveto{\pgfqpoint{8.579644in}{5.426530in}}%
\pgfpathlineto{\pgfqpoint{8.529557in}{5.424533in}}%
\pgfusepath{stroke}%
\end{pgfscope}%
\begin{pgfscope}%
\pgfpathrectangle{\pgfqpoint{6.720588in}{4.155455in}}{\pgfqpoint{2.279412in}{2.004545in}}%
\pgfusepath{clip}%
\pgfsetbuttcap%
\pgfsetroundjoin%
\pgfsetlinewidth{0.332560pt}%
\definecolor{currentstroke}{rgb}{0.272594,0.025563,0.353093}%
\pgfsetstrokecolor{currentstroke}%
\pgfsetdash{}{0pt}%
\pgfpathmoveto{\pgfqpoint{8.529557in}{5.424533in}}%
\pgfpathlineto{\pgfqpoint{8.479434in}{5.423425in}}%
\pgfusepath{stroke}%
\end{pgfscope}%
\begin{pgfscope}%
\pgfpathrectangle{\pgfqpoint{6.720588in}{4.155455in}}{\pgfqpoint{2.279412in}{2.004545in}}%
\pgfusepath{clip}%
\pgfsetbuttcap%
\pgfsetroundjoin%
\pgfsetlinewidth{0.350807pt}%
\definecolor{currentstroke}{rgb}{0.276022,0.044167,0.370164}%
\pgfsetstrokecolor{currentstroke}%
\pgfsetdash{}{0pt}%
\pgfpathmoveto{\pgfqpoint{8.479434in}{5.423425in}}%
\pgfpathlineto{\pgfqpoint{8.429302in}{5.422439in}}%
\pgfusepath{stroke}%
\end{pgfscope}%
\begin{pgfscope}%
\pgfpathrectangle{\pgfqpoint{6.720588in}{4.155455in}}{\pgfqpoint{2.279412in}{2.004545in}}%
\pgfusepath{clip}%
\pgfsetbuttcap%
\pgfsetroundjoin%
\pgfsetlinewidth{0.371469pt}%
\definecolor{currentstroke}{rgb}{0.278791,0.062145,0.386592}%
\pgfsetstrokecolor{currentstroke}%
\pgfsetdash{}{0pt}%
\pgfpathmoveto{\pgfqpoint{8.429302in}{5.422439in}}%
\pgfpathlineto{\pgfqpoint{8.379203in}{5.420614in}}%
\pgfusepath{stroke}%
\end{pgfscope}%
\begin{pgfscope}%
\pgfpathrectangle{\pgfqpoint{6.720588in}{4.155455in}}{\pgfqpoint{2.279412in}{2.004545in}}%
\pgfusepath{clip}%
\pgfsetbuttcap%
\pgfsetroundjoin%
\pgfsetlinewidth{0.401446pt}%
\definecolor{currentstroke}{rgb}{0.281446,0.084320,0.407414}%
\pgfsetstrokecolor{currentstroke}%
\pgfsetdash{}{0pt}%
\pgfpathmoveto{\pgfqpoint{8.379203in}{5.420614in}}%
\pgfpathlineto{\pgfqpoint{8.329134in}{5.418105in}}%
\pgfusepath{stroke}%
\end{pgfscope}%
\begin{pgfscope}%
\pgfpathrectangle{\pgfqpoint{6.720588in}{4.155455in}}{\pgfqpoint{2.279412in}{2.004545in}}%
\pgfusepath{clip}%
\pgfsetbuttcap%
\pgfsetroundjoin%
\pgfsetlinewidth{0.417860pt}%
\definecolor{currentstroke}{rgb}{0.282656,0.100196,0.422160}%
\pgfsetstrokecolor{currentstroke}%
\pgfsetdash{}{0pt}%
\pgfpathmoveto{\pgfqpoint{8.329134in}{5.418105in}}%
\pgfpathlineto{\pgfqpoint{8.279073in}{5.415467in}}%
\pgfusepath{stroke}%
\end{pgfscope}%
\begin{pgfscope}%
\pgfpathrectangle{\pgfqpoint{6.720588in}{4.155455in}}{\pgfqpoint{2.279412in}{2.004545in}}%
\pgfusepath{clip}%
\pgfsetbuttcap%
\pgfsetroundjoin%
\pgfsetlinewidth{0.471943pt}%
\definecolor{currentstroke}{rgb}{0.282884,0.135920,0.453427}%
\pgfsetstrokecolor{currentstroke}%
\pgfsetdash{}{0pt}%
\pgfpathmoveto{\pgfqpoint{8.279073in}{5.415467in}}%
\pgfpathlineto{\pgfqpoint{8.229032in}{5.412537in}}%
\pgfusepath{stroke}%
\end{pgfscope}%
\begin{pgfscope}%
\pgfpathrectangle{\pgfqpoint{6.720588in}{4.155455in}}{\pgfqpoint{2.279412in}{2.004545in}}%
\pgfusepath{clip}%
\pgfsetbuttcap%
\pgfsetroundjoin%
\pgfsetlinewidth{0.491934pt}%
\definecolor{currentstroke}{rgb}{0.281887,0.150881,0.465405}%
\pgfsetstrokecolor{currentstroke}%
\pgfsetdash{}{0pt}%
\pgfpathmoveto{\pgfqpoint{8.229032in}{5.412537in}}%
\pgfpathlineto{\pgfqpoint{8.179045in}{5.408994in}}%
\pgfusepath{stroke}%
\end{pgfscope}%
\begin{pgfscope}%
\pgfpathrectangle{\pgfqpoint{6.720588in}{4.155455in}}{\pgfqpoint{2.279412in}{2.004545in}}%
\pgfusepath{clip}%
\pgfsetbuttcap%
\pgfsetroundjoin%
\pgfsetlinewidth{0.556807pt}%
\definecolor{currentstroke}{rgb}{0.274128,0.199721,0.498911}%
\pgfsetstrokecolor{currentstroke}%
\pgfsetdash{}{0pt}%
\pgfpathmoveto{\pgfqpoint{8.179045in}{5.408994in}}%
\pgfpathlineto{\pgfqpoint{8.129130in}{5.404731in}}%
\pgfusepath{stroke}%
\end{pgfscope}%
\begin{pgfscope}%
\pgfpathrectangle{\pgfqpoint{6.720588in}{4.155455in}}{\pgfqpoint{2.279412in}{2.004545in}}%
\pgfusepath{clip}%
\pgfsetbuttcap%
\pgfsetroundjoin%
\pgfsetlinewidth{0.629860pt}%
\definecolor{currentstroke}{rgb}{0.260571,0.246922,0.522828}%
\pgfsetstrokecolor{currentstroke}%
\pgfsetdash{}{0pt}%
\pgfpathmoveto{\pgfqpoint{8.129130in}{5.404731in}}%
\pgfpathlineto{\pgfqpoint{8.079325in}{5.399607in}}%
\pgfusepath{stroke}%
\end{pgfscope}%
\begin{pgfscope}%
\pgfpathrectangle{\pgfqpoint{6.720588in}{4.155455in}}{\pgfqpoint{2.279412in}{2.004545in}}%
\pgfusepath{clip}%
\pgfsetbuttcap%
\pgfsetroundjoin%
\pgfsetlinewidth{0.607965pt}%
\definecolor{currentstroke}{rgb}{0.265145,0.232956,0.516599}%
\pgfsetstrokecolor{currentstroke}%
\pgfsetdash{}{0pt}%
\pgfpathmoveto{\pgfqpoint{8.079325in}{5.399607in}}%
\pgfpathlineto{\pgfqpoint{8.029723in}{5.393171in}}%
\pgfusepath{stroke}%
\end{pgfscope}%
\begin{pgfscope}%
\pgfpathrectangle{\pgfqpoint{6.720588in}{4.155455in}}{\pgfqpoint{2.279412in}{2.004545in}}%
\pgfusepath{clip}%
\pgfsetbuttcap%
\pgfsetroundjoin%
\pgfsetlinewidth{0.638447pt}%
\definecolor{currentstroke}{rgb}{0.257322,0.256130,0.526563}%
\pgfsetstrokecolor{currentstroke}%
\pgfsetdash{}{0pt}%
\pgfpathmoveto{\pgfqpoint{8.029723in}{5.393171in}}%
\pgfpathlineto{\pgfqpoint{7.980388in}{5.385319in}}%
\pgfusepath{stroke}%
\end{pgfscope}%
\begin{pgfscope}%
\pgfpathrectangle{\pgfqpoint{6.720588in}{4.155455in}}{\pgfqpoint{2.279412in}{2.004545in}}%
\pgfusepath{clip}%
\pgfsetbuttcap%
\pgfsetroundjoin%
\pgfsetlinewidth{0.632013pt}%
\definecolor{currentstroke}{rgb}{0.258965,0.251537,0.524736}%
\pgfsetstrokecolor{currentstroke}%
\pgfsetdash{}{0pt}%
\pgfpathmoveto{\pgfqpoint{7.980388in}{5.385319in}}%
\pgfpathlineto{\pgfqpoint{7.931481in}{5.375649in}}%
\pgfusepath{stroke}%
\end{pgfscope}%
\begin{pgfscope}%
\pgfpathrectangle{\pgfqpoint{6.720588in}{4.155455in}}{\pgfqpoint{2.279412in}{2.004545in}}%
\pgfusepath{clip}%
\pgfsetbuttcap%
\pgfsetroundjoin%
\pgfsetlinewidth{0.804295pt}%
\definecolor{currentstroke}{rgb}{0.212395,0.359683,0.551710}%
\pgfsetstrokecolor{currentstroke}%
\pgfsetdash{}{0pt}%
\pgfpathmoveto{\pgfqpoint{7.931481in}{5.375649in}}%
\pgfpathlineto{\pgfqpoint{7.883234in}{5.363676in}}%
\pgfusepath{stroke}%
\end{pgfscope}%
\begin{pgfscope}%
\pgfpathrectangle{\pgfqpoint{6.720588in}{4.155455in}}{\pgfqpoint{2.279412in}{2.004545in}}%
\pgfusepath{clip}%
\pgfsetbuttcap%
\pgfsetroundjoin%
\pgfsetlinewidth{0.801582pt}%
\definecolor{currentstroke}{rgb}{0.214298,0.355619,0.551184}%
\pgfsetstrokecolor{currentstroke}%
\pgfsetdash{}{0pt}%
\pgfpathmoveto{\pgfqpoint{7.883234in}{5.363676in}}%
\pgfpathlineto{\pgfqpoint{7.835717in}{5.349616in}}%
\pgfusepath{stroke}%
\end{pgfscope}%
\begin{pgfscope}%
\pgfpathrectangle{\pgfqpoint{6.720588in}{4.155455in}}{\pgfqpoint{2.279412in}{2.004545in}}%
\pgfusepath{clip}%
\pgfsetbuttcap%
\pgfsetroundjoin%
\pgfsetlinewidth{1.141484pt}%
\definecolor{currentstroke}{rgb}{0.135066,0.544853,0.554029}%
\pgfsetstrokecolor{currentstroke}%
\pgfsetdash{}{0pt}%
\pgfpathmoveto{\pgfqpoint{7.835717in}{5.349616in}}%
\pgfpathlineto{\pgfqpoint{7.788940in}{5.333766in}}%
\pgfusepath{stroke}%
\end{pgfscope}%
\begin{pgfscope}%
\pgfpathrectangle{\pgfqpoint{6.720588in}{4.155455in}}{\pgfqpoint{2.279412in}{2.004545in}}%
\pgfusepath{clip}%
\pgfsetbuttcap%
\pgfsetroundjoin%
\pgfsetlinewidth{1.533022pt}%
\definecolor{currentstroke}{rgb}{0.266941,0.748751,0.440573}%
\pgfsetstrokecolor{currentstroke}%
\pgfsetdash{}{0pt}%
\pgfpathmoveto{\pgfqpoint{7.788940in}{5.333766in}}%
\pgfpathlineto{\pgfqpoint{7.742972in}{5.316172in}}%
\pgfusepath{stroke}%
\end{pgfscope}%
\begin{pgfscope}%
\pgfpathrectangle{\pgfqpoint{6.720588in}{4.155455in}}{\pgfqpoint{2.279412in}{2.004545in}}%
\pgfusepath{clip}%
\pgfsetbuttcap%
\pgfsetroundjoin%
\pgfsetlinewidth{1.706212pt}%
\definecolor{currentstroke}{rgb}{0.477504,0.821444,0.318195}%
\pgfsetstrokecolor{currentstroke}%
\pgfsetdash{}{0pt}%
\pgfpathmoveto{\pgfqpoint{7.742972in}{5.316172in}}%
\pgfpathlineto{\pgfqpoint{7.697652in}{5.297320in}}%
\pgfusepath{stroke}%
\end{pgfscope}%
\begin{pgfscope}%
\pgfpathrectangle{\pgfqpoint{6.720588in}{4.155455in}}{\pgfqpoint{2.279412in}{2.004545in}}%
\pgfusepath{clip}%
\pgfsetbuttcap%
\pgfsetroundjoin%
\pgfsetlinewidth{1.826763pt}%
\definecolor{currentstroke}{rgb}{0.657642,0.860219,0.203082}%
\pgfsetstrokecolor{currentstroke}%
\pgfsetdash{}{0pt}%
\pgfpathmoveto{\pgfqpoint{7.697652in}{5.297320in}}%
\pgfpathlineto{\pgfqpoint{7.652546in}{5.278095in}}%
\pgfusepath{stroke}%
\end{pgfscope}%
\begin{pgfscope}%
\pgfpathrectangle{\pgfqpoint{6.720588in}{4.155455in}}{\pgfqpoint{2.279412in}{2.004545in}}%
\pgfusepath{clip}%
\pgfsetbuttcap%
\pgfsetroundjoin%
\pgfsetlinewidth{2.084222pt}%
\definecolor{currentstroke}{rgb}{0.993248,0.906157,0.143936}%
\pgfsetstrokecolor{currentstroke}%
\pgfsetdash{}{0pt}%
\pgfpathmoveto{\pgfqpoint{7.652546in}{5.278095in}}%
\pgfpathlineto{\pgfqpoint{7.607320in}{5.259232in}}%
\pgfusepath{stroke}%
\end{pgfscope}%
\begin{pgfscope}%
\pgfpathrectangle{\pgfqpoint{6.720588in}{4.155455in}}{\pgfqpoint{2.279412in}{2.004545in}}%
\pgfusepath{clip}%
\pgfsetbuttcap%
\pgfsetroundjoin%
\pgfsetlinewidth{1.798254pt}%
\definecolor{currentstroke}{rgb}{0.616293,0.852709,0.230052}%
\pgfsetstrokecolor{currentstroke}%
\pgfsetdash{}{0pt}%
\pgfpathmoveto{\pgfqpoint{7.607320in}{5.259232in}}%
\pgfpathlineto{\pgfqpoint{7.562156in}{5.240253in}}%
\pgfusepath{stroke}%
\end{pgfscope}%
\begin{pgfscope}%
\pgfpathrectangle{\pgfqpoint{6.720588in}{4.155455in}}{\pgfqpoint{2.279412in}{2.004545in}}%
\pgfusepath{clip}%
\pgfsetbuttcap%
\pgfsetroundjoin%
\pgfsetlinewidth{2.003580pt}%
\definecolor{currentstroke}{rgb}{0.926106,0.897330,0.104071}%
\pgfsetstrokecolor{currentstroke}%
\pgfsetdash{}{0pt}%
\pgfpathmoveto{\pgfqpoint{7.562156in}{5.240253in}}%
\pgfpathlineto{\pgfqpoint{7.517083in}{5.221011in}}%
\pgfusepath{stroke}%
\end{pgfscope}%
\begin{pgfscope}%
\pgfpathrectangle{\pgfqpoint{6.720588in}{4.155455in}}{\pgfqpoint{2.279412in}{2.004545in}}%
\pgfusepath{clip}%
\pgfsetbuttcap%
\pgfsetroundjoin%
\pgfsetlinewidth{1.788611pt}%
\definecolor{currentstroke}{rgb}{0.595839,0.848717,0.243329}%
\pgfsetstrokecolor{currentstroke}%
\pgfsetdash{}{0pt}%
\pgfpathmoveto{\pgfqpoint{7.517083in}{5.221011in}}%
\pgfpathlineto{\pgfqpoint{7.471804in}{5.202205in}}%
\pgfusepath{stroke}%
\end{pgfscope}%
\begin{pgfscope}%
\pgfpathrectangle{\pgfqpoint{6.720588in}{4.155455in}}{\pgfqpoint{2.279412in}{2.004545in}}%
\pgfusepath{clip}%
\pgfsetbuttcap%
\pgfsetroundjoin%
\pgfsetlinewidth{1.398741pt}%
\definecolor{currentstroke}{rgb}{0.153894,0.680203,0.504172}%
\pgfsetstrokecolor{currentstroke}%
\pgfsetdash{}{0pt}%
\pgfpathmoveto{\pgfqpoint{7.471804in}{5.202205in}}%
\pgfpathlineto{\pgfqpoint{7.426131in}{5.184281in}}%
\pgfusepath{stroke}%
\end{pgfscope}%
\begin{pgfscope}%
\pgfpathrectangle{\pgfqpoint{6.720588in}{4.155455in}}{\pgfqpoint{2.279412in}{2.004545in}}%
\pgfusepath{clip}%
\pgfsetbuttcap%
\pgfsetroundjoin%
\pgfsetlinewidth{0.313881pt}%
\definecolor{currentstroke}{rgb}{0.268510,0.009605,0.335427}%
\pgfsetstrokecolor{currentstroke}%
\pgfsetdash{}{0pt}%
\pgfpathmoveto{\pgfqpoint{8.578381in}{4.706659in}}%
\pgfpathlineto{\pgfqpoint{8.528292in}{4.707183in}}%
\pgfusepath{stroke}%
\end{pgfscope}%
\begin{pgfscope}%
\pgfpathrectangle{\pgfqpoint{6.720588in}{4.155455in}}{\pgfqpoint{2.279412in}{2.004545in}}%
\pgfusepath{clip}%
\pgfsetbuttcap%
\pgfsetroundjoin%
\pgfsetlinewidth{0.328420pt}%
\definecolor{currentstroke}{rgb}{0.271305,0.019942,0.347269}%
\pgfsetstrokecolor{currentstroke}%
\pgfsetdash{}{0pt}%
\pgfpathmoveto{\pgfqpoint{8.528292in}{4.707183in}}%
\pgfpathlineto{\pgfqpoint{8.478285in}{4.709304in}}%
\pgfusepath{stroke}%
\end{pgfscope}%
\begin{pgfscope}%
\pgfpathrectangle{\pgfqpoint{6.720588in}{4.155455in}}{\pgfqpoint{2.279412in}{2.004545in}}%
\pgfusepath{clip}%
\pgfsetbuttcap%
\pgfsetroundjoin%
\pgfsetlinewidth{0.330226pt}%
\definecolor{currentstroke}{rgb}{0.272594,0.025563,0.353093}%
\pgfsetstrokecolor{currentstroke}%
\pgfsetdash{}{0pt}%
\pgfpathmoveto{\pgfqpoint{8.478285in}{4.709304in}}%
\pgfpathlineto{\pgfqpoint{8.428313in}{4.712304in}}%
\pgfusepath{stroke}%
\end{pgfscope}%
\begin{pgfscope}%
\pgfpathrectangle{\pgfqpoint{6.720588in}{4.155455in}}{\pgfqpoint{2.279412in}{2.004545in}}%
\pgfusepath{clip}%
\pgfsetbuttcap%
\pgfsetroundjoin%
\pgfsetlinewidth{0.330844pt}%
\definecolor{currentstroke}{rgb}{0.272594,0.025563,0.353093}%
\pgfsetstrokecolor{currentstroke}%
\pgfsetdash{}{0pt}%
\pgfpathmoveto{\pgfqpoint{8.428313in}{4.712304in}}%
\pgfpathlineto{\pgfqpoint{8.378246in}{4.714550in}}%
\pgfusepath{stroke}%
\end{pgfscope}%
\begin{pgfscope}%
\pgfpathrectangle{\pgfqpoint{6.720588in}{4.155455in}}{\pgfqpoint{2.279412in}{2.004545in}}%
\pgfusepath{clip}%
\pgfsetbuttcap%
\pgfsetroundjoin%
\pgfsetlinewidth{0.339517pt}%
\definecolor{currentstroke}{rgb}{0.273809,0.031497,0.358853}%
\pgfsetstrokecolor{currentstroke}%
\pgfsetdash{}{0pt}%
\pgfpathmoveto{\pgfqpoint{8.378246in}{4.714550in}}%
\pgfpathlineto{\pgfqpoint{8.328220in}{4.717428in}}%
\pgfusepath{stroke}%
\end{pgfscope}%
\begin{pgfscope}%
\pgfpathrectangle{\pgfqpoint{6.720588in}{4.155455in}}{\pgfqpoint{2.279412in}{2.004545in}}%
\pgfusepath{clip}%
\pgfsetbuttcap%
\pgfsetroundjoin%
\pgfsetlinewidth{0.357629pt}%
\definecolor{currentstroke}{rgb}{0.277018,0.050344,0.375715}%
\pgfsetstrokecolor{currentstroke}%
\pgfsetdash{}{0pt}%
\pgfpathmoveto{\pgfqpoint{8.328220in}{4.717428in}}%
\pgfpathlineto{\pgfqpoint{8.278282in}{4.721469in}}%
\pgfusepath{stroke}%
\end{pgfscope}%
\begin{pgfscope}%
\pgfpathrectangle{\pgfqpoint{6.720588in}{4.155455in}}{\pgfqpoint{2.279412in}{2.004545in}}%
\pgfusepath{clip}%
\pgfsetbuttcap%
\pgfsetroundjoin%
\pgfsetlinewidth{0.386404pt}%
\definecolor{currentstroke}{rgb}{0.280267,0.073417,0.397163}%
\pgfsetstrokecolor{currentstroke}%
\pgfsetdash{}{0pt}%
\pgfpathmoveto{\pgfqpoint{8.278282in}{4.721469in}}%
\pgfpathlineto{\pgfqpoint{8.228454in}{4.726374in}}%
\pgfusepath{stroke}%
\end{pgfscope}%
\begin{pgfscope}%
\pgfpathrectangle{\pgfqpoint{6.720588in}{4.155455in}}{\pgfqpoint{2.279412in}{2.004545in}}%
\pgfusepath{clip}%
\pgfsetbuttcap%
\pgfsetroundjoin%
\pgfsetlinewidth{0.387552pt}%
\definecolor{currentstroke}{rgb}{0.280267,0.073417,0.397163}%
\pgfsetstrokecolor{currentstroke}%
\pgfsetdash{}{0pt}%
\pgfpathmoveto{\pgfqpoint{8.228454in}{4.726374in}}%
\pgfpathlineto{\pgfqpoint{8.178750in}{4.732204in}}%
\pgfusepath{stroke}%
\end{pgfscope}%
\begin{pgfscope}%
\pgfpathrectangle{\pgfqpoint{6.720588in}{4.155455in}}{\pgfqpoint{2.279412in}{2.004545in}}%
\pgfusepath{clip}%
\pgfsetbuttcap%
\pgfsetroundjoin%
\pgfsetlinewidth{0.420780pt}%
\definecolor{currentstroke}{rgb}{0.282656,0.100196,0.422160}%
\pgfsetstrokecolor{currentstroke}%
\pgfsetdash{}{0pt}%
\pgfpathmoveto{\pgfqpoint{8.178750in}{4.732204in}}%
\pgfpathlineto{\pgfqpoint{8.129239in}{4.739147in}}%
\pgfusepath{stroke}%
\end{pgfscope}%
\begin{pgfscope}%
\pgfpathrectangle{\pgfqpoint{6.720588in}{4.155455in}}{\pgfqpoint{2.279412in}{2.004545in}}%
\pgfusepath{clip}%
\pgfsetbuttcap%
\pgfsetroundjoin%
\pgfsetlinewidth{0.398536pt}%
\definecolor{currentstroke}{rgb}{0.281446,0.084320,0.407414}%
\pgfsetstrokecolor{currentstroke}%
\pgfsetdash{}{0pt}%
\pgfpathmoveto{\pgfqpoint{8.129239in}{4.739147in}}%
\pgfpathlineto{\pgfqpoint{8.080060in}{4.747747in}}%
\pgfusepath{stroke}%
\end{pgfscope}%
\begin{pgfscope}%
\pgfpathrectangle{\pgfqpoint{6.720588in}{4.155455in}}{\pgfqpoint{2.279412in}{2.004545in}}%
\pgfusepath{clip}%
\pgfsetbuttcap%
\pgfsetroundjoin%
\pgfsetlinewidth{0.424005pt}%
\definecolor{currentstroke}{rgb}{0.282656,0.100196,0.422160}%
\pgfsetstrokecolor{currentstroke}%
\pgfsetdash{}{0pt}%
\pgfpathmoveto{\pgfqpoint{8.080060in}{4.747747in}}%
\pgfpathlineto{\pgfqpoint{8.031026in}{4.756937in}}%
\pgfusepath{stroke}%
\end{pgfscope}%
\begin{pgfscope}%
\pgfpathrectangle{\pgfqpoint{6.720588in}{4.155455in}}{\pgfqpoint{2.279412in}{2.004545in}}%
\pgfusepath{clip}%
\pgfsetbuttcap%
\pgfsetroundjoin%
\pgfsetlinewidth{0.419875pt}%
\definecolor{currentstroke}{rgb}{0.282656,0.100196,0.422160}%
\pgfsetstrokecolor{currentstroke}%
\pgfsetdash{}{0pt}%
\pgfpathmoveto{\pgfqpoint{8.031026in}{4.756937in}}%
\pgfpathlineto{\pgfqpoint{7.982656in}{4.768150in}}%
\pgfusepath{stroke}%
\end{pgfscope}%
\begin{pgfscope}%
\pgfpathrectangle{\pgfqpoint{6.720588in}{4.155455in}}{\pgfqpoint{2.279412in}{2.004545in}}%
\pgfusepath{clip}%
\pgfsetbuttcap%
\pgfsetroundjoin%
\pgfsetlinewidth{0.427464pt}%
\definecolor{currentstroke}{rgb}{0.282910,0.105393,0.426902}%
\pgfsetstrokecolor{currentstroke}%
\pgfsetdash{}{0pt}%
\pgfpathmoveto{\pgfqpoint{7.982656in}{4.768150in}}%
\pgfpathlineto{\pgfqpoint{7.937210in}{4.785738in}}%
\pgfusepath{stroke}%
\end{pgfscope}%
\begin{pgfscope}%
\pgfpathrectangle{\pgfqpoint{6.720588in}{4.155455in}}{\pgfqpoint{2.279412in}{2.004545in}}%
\pgfusepath{clip}%
\pgfsetbuttcap%
\pgfsetroundjoin%
\pgfsetlinewidth{0.462706pt}%
\definecolor{currentstroke}{rgb}{0.283072,0.130895,0.449241}%
\pgfsetstrokecolor{currentstroke}%
\pgfsetdash{}{0pt}%
\pgfpathmoveto{\pgfqpoint{7.937210in}{4.785738in}}%
\pgfpathlineto{\pgfqpoint{7.896410in}{4.802988in}}%
\pgfusepath{stroke}%
\end{pgfscope}%
\begin{pgfscope}%
\pgfpathrectangle{\pgfqpoint{6.720588in}{4.155455in}}{\pgfqpoint{2.279412in}{2.004545in}}%
\pgfusepath{clip}%
\pgfsetbuttcap%
\pgfsetroundjoin%
\pgfsetlinewidth{0.462831pt}%
\definecolor{currentstroke}{rgb}{0.283072,0.130895,0.449241}%
\pgfsetstrokecolor{currentstroke}%
\pgfsetdash{}{0pt}%
\pgfpathmoveto{\pgfqpoint{7.896410in}{4.802988in}}%
\pgfpathlineto{\pgfqpoint{7.852430in}{4.823883in}}%
\pgfusepath{stroke}%
\end{pgfscope}%
\begin{pgfscope}%
\pgfpathrectangle{\pgfqpoint{6.720588in}{4.155455in}}{\pgfqpoint{2.279412in}{2.004545in}}%
\pgfusepath{clip}%
\pgfsetbuttcap%
\pgfsetroundjoin%
\pgfsetlinewidth{0.570023pt}%
\definecolor{currentstroke}{rgb}{0.271828,0.209303,0.504434}%
\pgfsetstrokecolor{currentstroke}%
\pgfsetdash{}{0pt}%
\pgfpathmoveto{\pgfqpoint{7.852430in}{4.823883in}}%
\pgfpathlineto{\pgfqpoint{7.810245in}{4.847554in}}%
\pgfusepath{stroke}%
\end{pgfscope}%
\begin{pgfscope}%
\pgfpathrectangle{\pgfqpoint{6.720588in}{4.155455in}}{\pgfqpoint{2.279412in}{2.004545in}}%
\pgfusepath{clip}%
\pgfsetbuttcap%
\pgfsetroundjoin%
\pgfsetlinewidth{0.325698pt}%
\definecolor{currentstroke}{rgb}{0.271305,0.019942,0.347269}%
\pgfsetstrokecolor{currentstroke}%
\pgfsetdash{}{0pt}%
\pgfpathmoveto{\pgfqpoint{8.578381in}{4.887087in}}%
\pgfpathlineto{\pgfqpoint{8.528267in}{4.888554in}}%
\pgfusepath{stroke}%
\end{pgfscope}%
\begin{pgfscope}%
\pgfpathrectangle{\pgfqpoint{6.720588in}{4.155455in}}{\pgfqpoint{2.279412in}{2.004545in}}%
\pgfusepath{clip}%
\pgfsetbuttcap%
\pgfsetroundjoin%
\pgfsetlinewidth{0.340609pt}%
\definecolor{currentstroke}{rgb}{0.273809,0.031497,0.358853}%
\pgfsetstrokecolor{currentstroke}%
\pgfsetdash{}{0pt}%
\pgfpathmoveto{\pgfqpoint{8.528267in}{4.888554in}}%
\pgfpathlineto{\pgfqpoint{8.478158in}{4.890248in}}%
\pgfusepath{stroke}%
\end{pgfscope}%
\begin{pgfscope}%
\pgfpathrectangle{\pgfqpoint{6.720588in}{4.155455in}}{\pgfqpoint{2.279412in}{2.004545in}}%
\pgfusepath{clip}%
\pgfsetbuttcap%
\pgfsetroundjoin%
\pgfsetlinewidth{0.347737pt}%
\definecolor{currentstroke}{rgb}{0.274952,0.037752,0.364543}%
\pgfsetstrokecolor{currentstroke}%
\pgfsetdash{}{0pt}%
\pgfpathmoveto{\pgfqpoint{8.478158in}{4.890248in}}%
\pgfpathlineto{\pgfqpoint{8.428062in}{4.892189in}}%
\pgfusepath{stroke}%
\end{pgfscope}%
\begin{pgfscope}%
\pgfpathrectangle{\pgfqpoint{6.720588in}{4.155455in}}{\pgfqpoint{2.279412in}{2.004545in}}%
\pgfusepath{clip}%
\pgfsetbuttcap%
\pgfsetroundjoin%
\pgfsetlinewidth{0.375640pt}%
\definecolor{currentstroke}{rgb}{0.278791,0.062145,0.386592}%
\pgfsetstrokecolor{currentstroke}%
\pgfsetdash{}{0pt}%
\pgfpathmoveto{\pgfqpoint{8.428062in}{4.892189in}}%
\pgfpathlineto{\pgfqpoint{8.377986in}{4.894601in}}%
\pgfusepath{stroke}%
\end{pgfscope}%
\begin{pgfscope}%
\pgfpathrectangle{\pgfqpoint{6.720588in}{4.155455in}}{\pgfqpoint{2.279412in}{2.004545in}}%
\pgfusepath{clip}%
\pgfsetbuttcap%
\pgfsetroundjoin%
\pgfsetlinewidth{0.395463pt}%
\definecolor{currentstroke}{rgb}{0.280894,0.078907,0.402329}%
\pgfsetstrokecolor{currentstroke}%
\pgfsetdash{}{0pt}%
\pgfpathmoveto{\pgfqpoint{8.377986in}{4.894601in}}%
\pgfpathlineto{\pgfqpoint{8.327903in}{4.896868in}}%
\pgfusepath{stroke}%
\end{pgfscope}%
\begin{pgfscope}%
\pgfpathrectangle{\pgfqpoint{6.720588in}{4.155455in}}{\pgfqpoint{2.279412in}{2.004545in}}%
\pgfusepath{clip}%
\pgfsetbuttcap%
\pgfsetroundjoin%
\pgfsetlinewidth{0.421448pt}%
\definecolor{currentstroke}{rgb}{0.282656,0.100196,0.422160}%
\pgfsetstrokecolor{currentstroke}%
\pgfsetdash{}{0pt}%
\pgfpathmoveto{\pgfqpoint{8.327903in}{4.896868in}}%
\pgfpathlineto{\pgfqpoint{8.277834in}{4.899348in}}%
\pgfusepath{stroke}%
\end{pgfscope}%
\begin{pgfscope}%
\pgfpathrectangle{\pgfqpoint{6.720588in}{4.155455in}}{\pgfqpoint{2.279412in}{2.004545in}}%
\pgfusepath{clip}%
\pgfsetbuttcap%
\pgfsetroundjoin%
\pgfsetlinewidth{0.466138pt}%
\definecolor{currentstroke}{rgb}{0.282884,0.135920,0.453427}%
\pgfsetstrokecolor{currentstroke}%
\pgfsetdash{}{0pt}%
\pgfpathmoveto{\pgfqpoint{8.277834in}{4.899348in}}%
\pgfpathlineto{\pgfqpoint{8.227798in}{4.902329in}}%
\pgfusepath{stroke}%
\end{pgfscope}%
\begin{pgfscope}%
\pgfpathrectangle{\pgfqpoint{6.720588in}{4.155455in}}{\pgfqpoint{2.279412in}{2.004545in}}%
\pgfusepath{clip}%
\pgfsetbuttcap%
\pgfsetroundjoin%
\pgfsetlinewidth{0.482657pt}%
\definecolor{currentstroke}{rgb}{0.282290,0.145912,0.461510}%
\pgfsetstrokecolor{currentstroke}%
\pgfsetdash{}{0pt}%
\pgfpathmoveto{\pgfqpoint{8.227798in}{4.902329in}}%
\pgfpathlineto{\pgfqpoint{8.177801in}{4.905778in}}%
\pgfusepath{stroke}%
\end{pgfscope}%
\begin{pgfscope}%
\pgfpathrectangle{\pgfqpoint{6.720588in}{4.155455in}}{\pgfqpoint{2.279412in}{2.004545in}}%
\pgfusepath{clip}%
\pgfsetbuttcap%
\pgfsetroundjoin%
\pgfsetlinewidth{0.554707pt}%
\definecolor{currentstroke}{rgb}{0.275191,0.194905,0.496005}%
\pgfsetstrokecolor{currentstroke}%
\pgfsetdash{}{0pt}%
\pgfpathmoveto{\pgfqpoint{8.177801in}{4.905778in}}%
\pgfpathlineto{\pgfqpoint{8.127879in}{4.909968in}}%
\pgfusepath{stroke}%
\end{pgfscope}%
\begin{pgfscope}%
\pgfpathrectangle{\pgfqpoint{6.720588in}{4.155455in}}{\pgfqpoint{2.279412in}{2.004545in}}%
\pgfusepath{clip}%
\pgfsetbuttcap%
\pgfsetroundjoin%
\pgfsetlinewidth{0.568344pt}%
\definecolor{currentstroke}{rgb}{0.273006,0.204520,0.501721}%
\pgfsetstrokecolor{currentstroke}%
\pgfsetdash{}{0pt}%
\pgfpathmoveto{\pgfqpoint{8.127879in}{4.909968in}}%
\pgfpathlineto{\pgfqpoint{8.078088in}{4.915206in}}%
\pgfusepath{stroke}%
\end{pgfscope}%
\begin{pgfscope}%
\pgfpathrectangle{\pgfqpoint{6.720588in}{4.155455in}}{\pgfqpoint{2.279412in}{2.004545in}}%
\pgfusepath{clip}%
\pgfsetbuttcap%
\pgfsetroundjoin%
\pgfsetlinewidth{0.601530pt}%
\definecolor{currentstroke}{rgb}{0.266580,0.228262,0.514349}%
\pgfsetstrokecolor{currentstroke}%
\pgfsetdash{}{0pt}%
\pgfpathmoveto{\pgfqpoint{8.078088in}{4.915206in}}%
\pgfpathlineto{\pgfqpoint{8.028509in}{4.921779in}}%
\pgfusepath{stroke}%
\end{pgfscope}%
\begin{pgfscope}%
\pgfpathrectangle{\pgfqpoint{6.720588in}{4.155455in}}{\pgfqpoint{2.279412in}{2.004545in}}%
\pgfusepath{clip}%
\pgfsetbuttcap%
\pgfsetroundjoin%
\pgfsetlinewidth{0.664419pt}%
\definecolor{currentstroke}{rgb}{0.252194,0.269783,0.531579}%
\pgfsetstrokecolor{currentstroke}%
\pgfsetdash{}{0pt}%
\pgfpathmoveto{\pgfqpoint{8.028509in}{4.921779in}}%
\pgfpathlineto{\pgfqpoint{7.979172in}{4.929664in}}%
\pgfusepath{stroke}%
\end{pgfscope}%
\begin{pgfscope}%
\pgfpathrectangle{\pgfqpoint{6.720588in}{4.155455in}}{\pgfqpoint{2.279412in}{2.004545in}}%
\pgfusepath{clip}%
\pgfsetbuttcap%
\pgfsetroundjoin%
\pgfsetlinewidth{0.683618pt}%
\definecolor{currentstroke}{rgb}{0.246811,0.283237,0.535941}%
\pgfsetstrokecolor{currentstroke}%
\pgfsetdash{}{0pt}%
\pgfpathmoveto{\pgfqpoint{7.979172in}{4.929664in}}%
\pgfpathlineto{\pgfqpoint{7.930198in}{4.939105in}}%
\pgfusepath{stroke}%
\end{pgfscope}%
\begin{pgfscope}%
\pgfpathrectangle{\pgfqpoint{6.720588in}{4.155455in}}{\pgfqpoint{2.279412in}{2.004545in}}%
\pgfusepath{clip}%
\pgfsetbuttcap%
\pgfsetroundjoin%
\pgfsetlinewidth{0.573759pt}%
\definecolor{currentstroke}{rgb}{0.271828,0.209303,0.504434}%
\pgfsetstrokecolor{currentstroke}%
\pgfsetdash{}{0pt}%
\pgfpathmoveto{\pgfqpoint{7.930198in}{4.939105in}}%
\pgfpathlineto{\pgfqpoint{7.881790in}{4.950539in}}%
\pgfusepath{stroke}%
\end{pgfscope}%
\begin{pgfscope}%
\pgfpathrectangle{\pgfqpoint{6.720588in}{4.155455in}}{\pgfqpoint{2.279412in}{2.004545in}}%
\pgfusepath{clip}%
\pgfsetbuttcap%
\pgfsetroundjoin%
\pgfsetlinewidth{0.924094pt}%
\definecolor{currentstroke}{rgb}{0.182256,0.426184,0.557120}%
\pgfsetstrokecolor{currentstroke}%
\pgfsetdash{}{0pt}%
\pgfpathmoveto{\pgfqpoint{7.881790in}{4.950539in}}%
\pgfpathlineto{\pgfqpoint{7.834081in}{4.964043in}}%
\pgfusepath{stroke}%
\end{pgfscope}%
\begin{pgfscope}%
\pgfpathrectangle{\pgfqpoint{6.720588in}{4.155455in}}{\pgfqpoint{2.279412in}{2.004545in}}%
\pgfusepath{clip}%
\pgfsetbuttcap%
\pgfsetroundjoin%
\pgfsetlinewidth{1.578202pt}%
\definecolor{currentstroke}{rgb}{0.319809,0.770914,0.411152}%
\pgfsetstrokecolor{currentstroke}%
\pgfsetdash{}{0pt}%
\pgfpathmoveto{\pgfqpoint{7.834081in}{4.964043in}}%
\pgfpathlineto{\pgfqpoint{7.787098in}{4.979427in}}%
\pgfusepath{stroke}%
\end{pgfscope}%
\begin{pgfscope}%
\pgfpathrectangle{\pgfqpoint{6.720588in}{4.155455in}}{\pgfqpoint{2.279412in}{2.004545in}}%
\pgfusepath{clip}%
\pgfsetbuttcap%
\pgfsetroundjoin%
\pgfsetlinewidth{1.708293pt}%
\definecolor{currentstroke}{rgb}{0.487026,0.823929,0.312321}%
\pgfsetstrokecolor{currentstroke}%
\pgfsetdash{}{0pt}%
\pgfpathmoveto{\pgfqpoint{7.787098in}{4.979427in}}%
\pgfpathlineto{\pgfqpoint{7.740686in}{4.996107in}}%
\pgfusepath{stroke}%
\end{pgfscope}%
\begin{pgfscope}%
\pgfpathrectangle{\pgfqpoint{6.720588in}{4.155455in}}{\pgfqpoint{2.279412in}{2.004545in}}%
\pgfusepath{clip}%
\pgfsetbuttcap%
\pgfsetroundjoin%
\pgfsetlinewidth{1.831409pt}%
\definecolor{currentstroke}{rgb}{0.668054,0.861999,0.196293}%
\pgfsetstrokecolor{currentstroke}%
\pgfsetdash{}{0pt}%
\pgfpathmoveto{\pgfqpoint{7.740686in}{4.996107in}}%
\pgfpathlineto{\pgfqpoint{7.694654in}{5.013558in}}%
\pgfusepath{stroke}%
\end{pgfscope}%
\begin{pgfscope}%
\pgfpathrectangle{\pgfqpoint{6.720588in}{4.155455in}}{\pgfqpoint{2.279412in}{2.004545in}}%
\pgfusepath{clip}%
\pgfsetbuttcap%
\pgfsetroundjoin%
\pgfsetlinewidth{0.321897pt}%
\definecolor{currentstroke}{rgb}{0.271305,0.019942,0.347269}%
\pgfsetstrokecolor{currentstroke}%
\pgfsetdash{}{0pt}%
\pgfpathmoveto{\pgfqpoint{8.578381in}{5.383261in}}%
\pgfpathlineto{\pgfqpoint{8.528275in}{5.381966in}}%
\pgfusepath{stroke}%
\end{pgfscope}%
\begin{pgfscope}%
\pgfpathrectangle{\pgfqpoint{6.720588in}{4.155455in}}{\pgfqpoint{2.279412in}{2.004545in}}%
\pgfusepath{clip}%
\pgfsetbuttcap%
\pgfsetroundjoin%
\pgfsetlinewidth{0.336339pt}%
\definecolor{currentstroke}{rgb}{0.273809,0.031497,0.358853}%
\pgfsetstrokecolor{currentstroke}%
\pgfsetdash{}{0pt}%
\pgfpathmoveto{\pgfqpoint{8.528275in}{5.381966in}}%
\pgfpathlineto{\pgfqpoint{8.478157in}{5.380388in}}%
\pgfusepath{stroke}%
\end{pgfscope}%
\begin{pgfscope}%
\pgfpathrectangle{\pgfqpoint{6.720588in}{4.155455in}}{\pgfqpoint{2.279412in}{2.004545in}}%
\pgfusepath{clip}%
\pgfsetbuttcap%
\pgfsetroundjoin%
\pgfsetlinewidth{0.363674pt}%
\definecolor{currentstroke}{rgb}{0.277941,0.056324,0.381191}%
\pgfsetstrokecolor{currentstroke}%
\pgfsetdash{}{0pt}%
\pgfpathmoveto{\pgfqpoint{8.478157in}{5.380388in}}%
\pgfpathlineto{\pgfqpoint{8.428042in}{5.378736in}}%
\pgfusepath{stroke}%
\end{pgfscope}%
\begin{pgfscope}%
\pgfpathrectangle{\pgfqpoint{6.720588in}{4.155455in}}{\pgfqpoint{2.279412in}{2.004545in}}%
\pgfusepath{clip}%
\pgfsetbuttcap%
\pgfsetroundjoin%
\pgfsetlinewidth{0.397265pt}%
\definecolor{currentstroke}{rgb}{0.281446,0.084320,0.407414}%
\pgfsetstrokecolor{currentstroke}%
\pgfsetdash{}{0pt}%
\pgfpathmoveto{\pgfqpoint{8.428042in}{5.378736in}}%
\pgfpathlineto{\pgfqpoint{8.377935in}{5.376878in}}%
\pgfusepath{stroke}%
\end{pgfscope}%
\begin{pgfscope}%
\pgfpathrectangle{\pgfqpoint{6.720588in}{4.155455in}}{\pgfqpoint{2.279412in}{2.004545in}}%
\pgfusepath{clip}%
\pgfsetbuttcap%
\pgfsetroundjoin%
\pgfsetlinewidth{0.410694pt}%
\definecolor{currentstroke}{rgb}{0.281924,0.089666,0.412415}%
\pgfsetstrokecolor{currentstroke}%
\pgfsetdash{}{0pt}%
\pgfpathmoveto{\pgfqpoint{8.377935in}{5.376878in}}%
\pgfpathlineto{\pgfqpoint{8.327840in}{5.374789in}}%
\pgfusepath{stroke}%
\end{pgfscope}%
\begin{pgfscope}%
\pgfpathrectangle{\pgfqpoint{6.720588in}{4.155455in}}{\pgfqpoint{2.279412in}{2.004545in}}%
\pgfusepath{clip}%
\pgfsetbuttcap%
\pgfsetroundjoin%
\pgfsetlinewidth{0.432487pt}%
\definecolor{currentstroke}{rgb}{0.283091,0.110553,0.431554}%
\pgfsetstrokecolor{currentstroke}%
\pgfsetdash{}{0pt}%
\pgfpathmoveto{\pgfqpoint{8.327840in}{5.374789in}}%
\pgfpathlineto{\pgfqpoint{8.277765in}{5.372384in}}%
\pgfusepath{stroke}%
\end{pgfscope}%
\begin{pgfscope}%
\pgfpathrectangle{\pgfqpoint{6.720588in}{4.155455in}}{\pgfqpoint{2.279412in}{2.004545in}}%
\pgfusepath{clip}%
\pgfsetbuttcap%
\pgfsetroundjoin%
\pgfsetlinewidth{0.496796pt}%
\definecolor{currentstroke}{rgb}{0.281412,0.155834,0.469201}%
\pgfsetstrokecolor{currentstroke}%
\pgfsetdash{}{0pt}%
\pgfpathmoveto{\pgfqpoint{8.277765in}{5.372384in}}%
\pgfpathlineto{\pgfqpoint{8.227702in}{5.369784in}}%
\pgfusepath{stroke}%
\end{pgfscope}%
\begin{pgfscope}%
\pgfpathrectangle{\pgfqpoint{6.720588in}{4.155455in}}{\pgfqpoint{2.279412in}{2.004545in}}%
\pgfusepath{clip}%
\pgfsetbuttcap%
\pgfsetroundjoin%
\pgfsetlinewidth{0.551588pt}%
\definecolor{currentstroke}{rgb}{0.275191,0.194905,0.496005}%
\pgfsetstrokecolor{currentstroke}%
\pgfsetdash{}{0pt}%
\pgfpathmoveto{\pgfqpoint{8.227702in}{5.369784in}}%
\pgfpathlineto{\pgfqpoint{8.177663in}{5.366853in}}%
\pgfusepath{stroke}%
\end{pgfscope}%
\begin{pgfscope}%
\pgfpathrectangle{\pgfqpoint{6.720588in}{4.155455in}}{\pgfqpoint{2.279412in}{2.004545in}}%
\pgfusepath{clip}%
\pgfsetbuttcap%
\pgfsetroundjoin%
\pgfsetlinewidth{0.327884pt}%
\definecolor{currentstroke}{rgb}{0.271305,0.019942,0.347269}%
\pgfsetstrokecolor{currentstroke}%
\pgfsetdash{}{0pt}%
\pgfpathmoveto{\pgfqpoint{8.578381in}{5.473475in}}%
\pgfpathlineto{\pgfqpoint{8.528269in}{5.472299in}}%
\pgfusepath{stroke}%
\end{pgfscope}%
\begin{pgfscope}%
\pgfpathrectangle{\pgfqpoint{6.720588in}{4.155455in}}{\pgfqpoint{2.279412in}{2.004545in}}%
\pgfusepath{clip}%
\pgfsetbuttcap%
\pgfsetroundjoin%
\pgfsetlinewidth{0.331990pt}%
\definecolor{currentstroke}{rgb}{0.272594,0.025563,0.353093}%
\pgfsetstrokecolor{currentstroke}%
\pgfsetdash{}{0pt}%
\pgfpathmoveto{\pgfqpoint{8.528269in}{5.472299in}}%
\pgfpathlineto{\pgfqpoint{8.478186in}{5.470139in}}%
\pgfusepath{stroke}%
\end{pgfscope}%
\begin{pgfscope}%
\pgfpathrectangle{\pgfqpoint{6.720588in}{4.155455in}}{\pgfqpoint{2.279412in}{2.004545in}}%
\pgfusepath{clip}%
\pgfsetbuttcap%
\pgfsetroundjoin%
\pgfsetlinewidth{0.341269pt}%
\definecolor{currentstroke}{rgb}{0.273809,0.031497,0.358853}%
\pgfsetstrokecolor{currentstroke}%
\pgfsetdash{}{0pt}%
\pgfpathmoveto{\pgfqpoint{8.478186in}{5.470139in}}%
\pgfpathlineto{\pgfqpoint{8.428111in}{5.467783in}}%
\pgfusepath{stroke}%
\end{pgfscope}%
\begin{pgfscope}%
\pgfpathrectangle{\pgfqpoint{6.720588in}{4.155455in}}{\pgfqpoint{2.279412in}{2.004545in}}%
\pgfusepath{clip}%
\pgfsetbuttcap%
\pgfsetroundjoin%
\pgfsetlinewidth{0.363470pt}%
\definecolor{currentstroke}{rgb}{0.277941,0.056324,0.381191}%
\pgfsetstrokecolor{currentstroke}%
\pgfsetdash{}{0pt}%
\pgfpathmoveto{\pgfqpoint{8.428111in}{5.467783in}}%
\pgfpathlineto{\pgfqpoint{8.378052in}{5.465208in}}%
\pgfusepath{stroke}%
\end{pgfscope}%
\begin{pgfscope}%
\pgfpathrectangle{\pgfqpoint{6.720588in}{4.155455in}}{\pgfqpoint{2.279412in}{2.004545in}}%
\pgfusepath{clip}%
\pgfsetbuttcap%
\pgfsetroundjoin%
\pgfsetlinewidth{0.386992pt}%
\definecolor{currentstroke}{rgb}{0.280267,0.073417,0.397163}%
\pgfsetstrokecolor{currentstroke}%
\pgfsetdash{}{0pt}%
\pgfpathmoveto{\pgfqpoint{8.378052in}{5.465208in}}%
\pgfpathlineto{\pgfqpoint{8.328036in}{5.462017in}}%
\pgfusepath{stroke}%
\end{pgfscope}%
\begin{pgfscope}%
\pgfpathrectangle{\pgfqpoint{6.720588in}{4.155455in}}{\pgfqpoint{2.279412in}{2.004545in}}%
\pgfusepath{clip}%
\pgfsetbuttcap%
\pgfsetroundjoin%
\pgfsetlinewidth{0.409994pt}%
\definecolor{currentstroke}{rgb}{0.281924,0.089666,0.412415}%
\pgfsetstrokecolor{currentstroke}%
\pgfsetdash{}{0pt}%
\pgfpathmoveto{\pgfqpoint{8.328036in}{5.462017in}}%
\pgfpathlineto{\pgfqpoint{8.278033in}{5.458630in}}%
\pgfusepath{stroke}%
\end{pgfscope}%
\begin{pgfscope}%
\pgfpathrectangle{\pgfqpoint{6.720588in}{4.155455in}}{\pgfqpoint{2.279412in}{2.004545in}}%
\pgfusepath{clip}%
\pgfsetbuttcap%
\pgfsetroundjoin%
\pgfsetlinewidth{0.449286pt}%
\definecolor{currentstroke}{rgb}{0.283229,0.120777,0.440584}%
\pgfsetstrokecolor{currentstroke}%
\pgfsetdash{}{0pt}%
\pgfpathmoveto{\pgfqpoint{8.278033in}{5.458630in}}%
\pgfpathlineto{\pgfqpoint{8.228082in}{5.454719in}}%
\pgfusepath{stroke}%
\end{pgfscope}%
\begin{pgfscope}%
\pgfpathrectangle{\pgfqpoint{6.720588in}{4.155455in}}{\pgfqpoint{2.279412in}{2.004545in}}%
\pgfusepath{clip}%
\pgfsetbuttcap%
\pgfsetroundjoin%
\pgfsetlinewidth{0.322942pt}%
\definecolor{currentstroke}{rgb}{0.271305,0.019942,0.347269}%
\pgfsetstrokecolor{currentstroke}%
\pgfsetdash{}{0pt}%
\pgfpathmoveto{\pgfqpoint{8.578381in}{5.518582in}}%
\pgfpathlineto{\pgfqpoint{8.528352in}{5.517716in}}%
\pgfusepath{stroke}%
\end{pgfscope}%
\begin{pgfscope}%
\pgfpathrectangle{\pgfqpoint{6.720588in}{4.155455in}}{\pgfqpoint{2.279412in}{2.004545in}}%
\pgfusepath{clip}%
\pgfsetbuttcap%
\pgfsetroundjoin%
\pgfsetlinewidth{0.327334pt}%
\definecolor{currentstroke}{rgb}{0.271305,0.019942,0.347269}%
\pgfsetstrokecolor{currentstroke}%
\pgfsetdash{}{0pt}%
\pgfpathmoveto{\pgfqpoint{8.528352in}{5.517716in}}%
\pgfpathlineto{\pgfqpoint{8.478342in}{5.515207in}}%
\pgfusepath{stroke}%
\end{pgfscope}%
\begin{pgfscope}%
\pgfpathrectangle{\pgfqpoint{6.720588in}{4.155455in}}{\pgfqpoint{2.279412in}{2.004545in}}%
\pgfusepath{clip}%
\pgfsetbuttcap%
\pgfsetroundjoin%
\pgfsetlinewidth{0.338451pt}%
\definecolor{currentstroke}{rgb}{0.273809,0.031497,0.358853}%
\pgfsetstrokecolor{currentstroke}%
\pgfsetdash{}{0pt}%
\pgfpathmoveto{\pgfqpoint{8.478342in}{5.515207in}}%
\pgfpathlineto{\pgfqpoint{8.428342in}{5.512248in}}%
\pgfusepath{stroke}%
\end{pgfscope}%
\begin{pgfscope}%
\pgfpathrectangle{\pgfqpoint{6.720588in}{4.155455in}}{\pgfqpoint{2.279412in}{2.004545in}}%
\pgfusepath{clip}%
\pgfsetbuttcap%
\pgfsetroundjoin%
\pgfsetlinewidth{0.355846pt}%
\definecolor{currentstroke}{rgb}{0.276022,0.044167,0.370164}%
\pgfsetstrokecolor{currentstroke}%
\pgfsetdash{}{0pt}%
\pgfpathmoveto{\pgfqpoint{8.428342in}{5.512248in}}%
\pgfpathlineto{\pgfqpoint{8.378282in}{5.509777in}}%
\pgfusepath{stroke}%
\end{pgfscope}%
\begin{pgfscope}%
\pgfpathrectangle{\pgfqpoint{6.720588in}{4.155455in}}{\pgfqpoint{2.279412in}{2.004545in}}%
\pgfusepath{clip}%
\pgfsetbuttcap%
\pgfsetroundjoin%
\pgfsetlinewidth{0.375126pt}%
\definecolor{currentstroke}{rgb}{0.278791,0.062145,0.386592}%
\pgfsetstrokecolor{currentstroke}%
\pgfsetdash{}{0pt}%
\pgfpathmoveto{\pgfqpoint{8.378282in}{5.509777in}}%
\pgfpathlineto{\pgfqpoint{8.328250in}{5.507153in}}%
\pgfusepath{stroke}%
\end{pgfscope}%
\begin{pgfscope}%
\pgfpathrectangle{\pgfqpoint{6.720588in}{4.155455in}}{\pgfqpoint{2.279412in}{2.004545in}}%
\pgfusepath{clip}%
\pgfsetbuttcap%
\pgfsetroundjoin%
\pgfsetlinewidth{0.372567pt}%
\definecolor{currentstroke}{rgb}{0.278791,0.062145,0.386592}%
\pgfsetstrokecolor{currentstroke}%
\pgfsetdash{}{0pt}%
\pgfpathmoveto{\pgfqpoint{8.328250in}{5.507153in}}%
\pgfpathlineto{\pgfqpoint{8.278317in}{5.503255in}}%
\pgfusepath{stroke}%
\end{pgfscope}%
\begin{pgfscope}%
\pgfpathrectangle{\pgfqpoint{6.720588in}{4.155455in}}{\pgfqpoint{2.279412in}{2.004545in}}%
\pgfusepath{clip}%
\pgfsetbuttcap%
\pgfsetroundjoin%
\pgfsetlinewidth{0.415036pt}%
\definecolor{currentstroke}{rgb}{0.282327,0.094955,0.417331}%
\pgfsetstrokecolor{currentstroke}%
\pgfsetdash{}{0pt}%
\pgfpathmoveto{\pgfqpoint{8.278317in}{5.503255in}}%
\pgfpathlineto{\pgfqpoint{8.228443in}{5.498660in}}%
\pgfusepath{stroke}%
\end{pgfscope}%
\begin{pgfscope}%
\pgfpathrectangle{\pgfqpoint{6.720588in}{4.155455in}}{\pgfqpoint{2.279412in}{2.004545in}}%
\pgfusepath{clip}%
\pgfsetbuttcap%
\pgfsetroundjoin%
\pgfsetlinewidth{0.435126pt}%
\definecolor{currentstroke}{rgb}{0.283091,0.110553,0.431554}%
\pgfsetstrokecolor{currentstroke}%
\pgfsetdash{}{0pt}%
\pgfpathmoveto{\pgfqpoint{8.228443in}{5.498660in}}%
\pgfpathlineto{\pgfqpoint{8.178647in}{5.493425in}}%
\pgfusepath{stroke}%
\end{pgfscope}%
\begin{pgfscope}%
\pgfpathrectangle{\pgfqpoint{6.720588in}{4.155455in}}{\pgfqpoint{2.279412in}{2.004545in}}%
\pgfusepath{clip}%
\pgfsetbuttcap%
\pgfsetroundjoin%
\pgfsetlinewidth{0.467618pt}%
\definecolor{currentstroke}{rgb}{0.282884,0.135920,0.453427}%
\pgfsetstrokecolor{currentstroke}%
\pgfsetdash{}{0pt}%
\pgfpathmoveto{\pgfqpoint{8.178647in}{5.493425in}}%
\pgfpathlineto{\pgfqpoint{8.128945in}{5.487547in}}%
\pgfusepath{stroke}%
\end{pgfscope}%
\begin{pgfscope}%
\pgfpathrectangle{\pgfqpoint{6.720588in}{4.155455in}}{\pgfqpoint{2.279412in}{2.004545in}}%
\pgfusepath{clip}%
\pgfsetbuttcap%
\pgfsetroundjoin%
\pgfsetlinewidth{0.487938pt}%
\definecolor{currentstroke}{rgb}{0.281887,0.150881,0.465405}%
\pgfsetstrokecolor{currentstroke}%
\pgfsetdash{}{0pt}%
\pgfpathmoveto{\pgfqpoint{8.128945in}{5.487547in}}%
\pgfpathlineto{\pgfqpoint{8.079458in}{5.480461in}}%
\pgfusepath{stroke}%
\end{pgfscope}%
\begin{pgfscope}%
\pgfpathrectangle{\pgfqpoint{6.720588in}{4.155455in}}{\pgfqpoint{2.279412in}{2.004545in}}%
\pgfusepath{clip}%
\pgfsetbuttcap%
\pgfsetroundjoin%
\pgfsetlinewidth{0.502495pt}%
\definecolor{currentstroke}{rgb}{0.280868,0.160771,0.472899}%
\pgfsetstrokecolor{currentstroke}%
\pgfsetdash{}{0pt}%
\pgfpathmoveto{\pgfqpoint{8.079458in}{5.480461in}}%
\pgfpathlineto{\pgfqpoint{8.030336in}{5.471658in}}%
\pgfusepath{stroke}%
\end{pgfscope}%
\begin{pgfscope}%
\pgfpathrectangle{\pgfqpoint{6.720588in}{4.155455in}}{\pgfqpoint{2.279412in}{2.004545in}}%
\pgfusepath{clip}%
\pgfsetbuttcap%
\pgfsetroundjoin%
\pgfsetlinewidth{0.496115pt}%
\definecolor{currentstroke}{rgb}{0.281412,0.155834,0.469201}%
\pgfsetstrokecolor{currentstroke}%
\pgfsetdash{}{0pt}%
\pgfpathmoveto{\pgfqpoint{8.030336in}{5.471658in}}%
\pgfpathlineto{\pgfqpoint{7.981634in}{5.461175in}}%
\pgfusepath{stroke}%
\end{pgfscope}%
\begin{pgfscope}%
\pgfpathrectangle{\pgfqpoint{6.720588in}{4.155455in}}{\pgfqpoint{2.279412in}{2.004545in}}%
\pgfusepath{clip}%
\pgfsetbuttcap%
\pgfsetroundjoin%
\pgfsetlinewidth{0.484528pt}%
\definecolor{currentstroke}{rgb}{0.282290,0.145912,0.461510}%
\pgfsetstrokecolor{currentstroke}%
\pgfsetdash{}{0pt}%
\pgfpathmoveto{\pgfqpoint{7.981634in}{5.461175in}}%
\pgfpathlineto{\pgfqpoint{7.933594in}{5.448619in}}%
\pgfusepath{stroke}%
\end{pgfscope}%
\begin{pgfscope}%
\pgfpathrectangle{\pgfqpoint{6.720588in}{4.155455in}}{\pgfqpoint{2.279412in}{2.004545in}}%
\pgfusepath{clip}%
\pgfsetbuttcap%
\pgfsetroundjoin%
\pgfsetlinewidth{0.587401pt}%
\definecolor{currentstroke}{rgb}{0.269308,0.218818,0.509577}%
\pgfsetstrokecolor{currentstroke}%
\pgfsetdash{}{0pt}%
\pgfpathmoveto{\pgfqpoint{7.933594in}{5.448619in}}%
\pgfpathlineto{\pgfqpoint{7.886585in}{5.433364in}}%
\pgfusepath{stroke}%
\end{pgfscope}%
\begin{pgfscope}%
\pgfpathrectangle{\pgfqpoint{6.720588in}{4.155455in}}{\pgfqpoint{2.279412in}{2.004545in}}%
\pgfusepath{clip}%
\pgfsetbuttcap%
\pgfsetroundjoin%
\pgfsetlinewidth{0.320129pt}%
\definecolor{currentstroke}{rgb}{0.269944,0.014625,0.341379}%
\pgfsetstrokecolor{currentstroke}%
\pgfsetdash{}{0pt}%
\pgfpathmoveto{\pgfqpoint{8.578381in}{5.563688in}}%
\pgfpathlineto{\pgfqpoint{8.528299in}{5.561761in}}%
\pgfusepath{stroke}%
\end{pgfscope}%
\begin{pgfscope}%
\pgfpathrectangle{\pgfqpoint{6.720588in}{4.155455in}}{\pgfqpoint{2.279412in}{2.004545in}}%
\pgfusepath{clip}%
\pgfsetbuttcap%
\pgfsetroundjoin%
\pgfsetlinewidth{0.327463pt}%
\definecolor{currentstroke}{rgb}{0.271305,0.019942,0.347269}%
\pgfsetstrokecolor{currentstroke}%
\pgfsetdash{}{0pt}%
\pgfpathmoveto{\pgfqpoint{8.528299in}{5.561761in}}%
\pgfpathlineto{\pgfqpoint{8.478210in}{5.559983in}}%
\pgfusepath{stroke}%
\end{pgfscope}%
\begin{pgfscope}%
\pgfpathrectangle{\pgfqpoint{6.720588in}{4.155455in}}{\pgfqpoint{2.279412in}{2.004545in}}%
\pgfusepath{clip}%
\pgfsetbuttcap%
\pgfsetroundjoin%
\pgfsetlinewidth{0.330622pt}%
\definecolor{currentstroke}{rgb}{0.272594,0.025563,0.353093}%
\pgfsetstrokecolor{currentstroke}%
\pgfsetdash{}{0pt}%
\pgfpathmoveto{\pgfqpoint{8.478210in}{5.559983in}}%
\pgfpathlineto{\pgfqpoint{8.428122in}{5.557990in}}%
\pgfusepath{stroke}%
\end{pgfscope}%
\begin{pgfscope}%
\pgfpathrectangle{\pgfqpoint{6.720588in}{4.155455in}}{\pgfqpoint{2.279412in}{2.004545in}}%
\pgfusepath{clip}%
\pgfsetbuttcap%
\pgfsetroundjoin%
\pgfsetlinewidth{0.353175pt}%
\definecolor{currentstroke}{rgb}{0.276022,0.044167,0.370164}%
\pgfsetstrokecolor{currentstroke}%
\pgfsetdash{}{0pt}%
\pgfpathmoveto{\pgfqpoint{8.428122in}{5.557990in}}%
\pgfpathlineto{\pgfqpoint{8.378066in}{5.555367in}}%
\pgfusepath{stroke}%
\end{pgfscope}%
\begin{pgfscope}%
\pgfpathrectangle{\pgfqpoint{6.720588in}{4.155455in}}{\pgfqpoint{2.279412in}{2.004545in}}%
\pgfusepath{clip}%
\pgfsetbuttcap%
\pgfsetroundjoin%
\pgfsetlinewidth{0.357655pt}%
\definecolor{currentstroke}{rgb}{0.277018,0.050344,0.375715}%
\pgfsetstrokecolor{currentstroke}%
\pgfsetdash{}{0pt}%
\pgfpathmoveto{\pgfqpoint{8.378066in}{5.555367in}}%
\pgfpathlineto{\pgfqpoint{8.328086in}{5.551848in}}%
\pgfusepath{stroke}%
\end{pgfscope}%
\begin{pgfscope}%
\pgfpathrectangle{\pgfqpoint{6.720588in}{4.155455in}}{\pgfqpoint{2.279412in}{2.004545in}}%
\pgfusepath{clip}%
\pgfsetbuttcap%
\pgfsetroundjoin%
\pgfsetlinewidth{0.372149pt}%
\definecolor{currentstroke}{rgb}{0.278791,0.062145,0.386592}%
\pgfsetstrokecolor{currentstroke}%
\pgfsetdash{}{0pt}%
\pgfpathmoveto{\pgfqpoint{8.328086in}{5.551848in}}%
\pgfpathlineto{\pgfqpoint{8.278158in}{5.547741in}}%
\pgfusepath{stroke}%
\end{pgfscope}%
\begin{pgfscope}%
\pgfpathrectangle{\pgfqpoint{6.720588in}{4.155455in}}{\pgfqpoint{2.279412in}{2.004545in}}%
\pgfusepath{clip}%
\pgfsetbuttcap%
\pgfsetroundjoin%
\pgfsetlinewidth{0.386692pt}%
\definecolor{currentstroke}{rgb}{0.280267,0.073417,0.397163}%
\pgfsetstrokecolor{currentstroke}%
\pgfsetdash{}{0pt}%
\pgfpathmoveto{\pgfqpoint{8.278158in}{5.547741in}}%
\pgfpathlineto{\pgfqpoint{8.228258in}{5.543359in}}%
\pgfusepath{stroke}%
\end{pgfscope}%
\begin{pgfscope}%
\pgfpathrectangle{\pgfqpoint{6.720588in}{4.155455in}}{\pgfqpoint{2.279412in}{2.004545in}}%
\pgfusepath{clip}%
\pgfsetbuttcap%
\pgfsetroundjoin%
\pgfsetlinewidth{0.408485pt}%
\definecolor{currentstroke}{rgb}{0.281924,0.089666,0.412415}%
\pgfsetstrokecolor{currentstroke}%
\pgfsetdash{}{0pt}%
\pgfpathmoveto{\pgfqpoint{8.228258in}{5.543359in}}%
\pgfpathlineto{\pgfqpoint{8.178420in}{5.538446in}}%
\pgfusepath{stroke}%
\end{pgfscope}%
\begin{pgfscope}%
\pgfpathrectangle{\pgfqpoint{6.720588in}{4.155455in}}{\pgfqpoint{2.279412in}{2.004545in}}%
\pgfusepath{clip}%
\pgfsetbuttcap%
\pgfsetroundjoin%
\pgfsetlinewidth{0.418058pt}%
\definecolor{currentstroke}{rgb}{0.282656,0.100196,0.422160}%
\pgfsetstrokecolor{currentstroke}%
\pgfsetdash{}{0pt}%
\pgfpathmoveto{\pgfqpoint{8.178420in}{5.538446in}}%
\pgfpathlineto{\pgfqpoint{8.128671in}{5.532895in}}%
\pgfusepath{stroke}%
\end{pgfscope}%
\begin{pgfscope}%
\pgfpathrectangle{\pgfqpoint{6.720588in}{4.155455in}}{\pgfqpoint{2.279412in}{2.004545in}}%
\pgfusepath{clip}%
\pgfsetbuttcap%
\pgfsetroundjoin%
\pgfsetlinewidth{0.410583pt}%
\definecolor{currentstroke}{rgb}{0.281924,0.089666,0.412415}%
\pgfsetstrokecolor{currentstroke}%
\pgfsetdash{}{0pt}%
\pgfpathmoveto{\pgfqpoint{8.128671in}{5.532895in}}%
\pgfpathlineto{\pgfqpoint{8.079231in}{5.525630in}}%
\pgfusepath{stroke}%
\end{pgfscope}%
\begin{pgfscope}%
\pgfpathrectangle{\pgfqpoint{6.720588in}{4.155455in}}{\pgfqpoint{2.279412in}{2.004545in}}%
\pgfusepath{clip}%
\pgfsetbuttcap%
\pgfsetroundjoin%
\pgfsetlinewidth{0.466259pt}%
\definecolor{currentstroke}{rgb}{0.282884,0.135920,0.453427}%
\pgfsetstrokecolor{currentstroke}%
\pgfsetdash{}{0pt}%
\pgfpathmoveto{\pgfqpoint{8.079231in}{5.525630in}}%
\pgfpathlineto{\pgfqpoint{8.030229in}{5.516339in}}%
\pgfusepath{stroke}%
\end{pgfscope}%
\begin{pgfscope}%
\pgfpathrectangle{\pgfqpoint{6.720588in}{4.155455in}}{\pgfqpoint{2.279412in}{2.004545in}}%
\pgfusepath{clip}%
\pgfsetbuttcap%
\pgfsetroundjoin%
\pgfsetlinewidth{0.471948pt}%
\definecolor{currentstroke}{rgb}{0.282884,0.135920,0.453427}%
\pgfsetstrokecolor{currentstroke}%
\pgfsetdash{}{0pt}%
\pgfpathmoveto{\pgfqpoint{8.030229in}{5.516339in}}%
\pgfpathlineto{\pgfqpoint{7.981779in}{5.505038in}}%
\pgfusepath{stroke}%
\end{pgfscope}%
\begin{pgfscope}%
\pgfpathrectangle{\pgfqpoint{6.720588in}{4.155455in}}{\pgfqpoint{2.279412in}{2.004545in}}%
\pgfusepath{clip}%
\pgfsetbuttcap%
\pgfsetroundjoin%
\pgfsetlinewidth{0.438123pt}%
\definecolor{currentstroke}{rgb}{0.283091,0.110553,0.431554}%
\pgfsetstrokecolor{currentstroke}%
\pgfsetdash{}{0pt}%
\pgfpathmoveto{\pgfqpoint{7.981779in}{5.505038in}}%
\pgfpathlineto{\pgfqpoint{7.934667in}{5.490240in}}%
\pgfusepath{stroke}%
\end{pgfscope}%
\begin{pgfscope}%
\pgfpathrectangle{\pgfqpoint{6.720588in}{4.155455in}}{\pgfqpoint{2.279412in}{2.004545in}}%
\pgfusepath{clip}%
\pgfsetbuttcap%
\pgfsetroundjoin%
\pgfsetlinewidth{0.339579pt}%
\definecolor{currentstroke}{rgb}{0.273809,0.031497,0.358853}%
\pgfsetstrokecolor{currentstroke}%
\pgfsetdash{}{0pt}%
\pgfpathmoveto{\pgfqpoint{8.314272in}{5.814716in}}%
\pgfpathlineto{\pgfqpoint{8.265133in}{5.809807in}}%
\pgfusepath{stroke}%
\end{pgfscope}%
\begin{pgfscope}%
\pgfpathrectangle{\pgfqpoint{6.720588in}{4.155455in}}{\pgfqpoint{2.279412in}{2.004545in}}%
\pgfusepath{clip}%
\pgfsetbuttcap%
\pgfsetroundjoin%
\pgfsetlinewidth{0.329552pt}%
\definecolor{currentstroke}{rgb}{0.272594,0.025563,0.353093}%
\pgfsetstrokecolor{currentstroke}%
\pgfsetdash{}{0pt}%
\pgfpathmoveto{\pgfqpoint{8.265133in}{5.809807in}}%
\pgfpathlineto{\pgfqpoint{8.215742in}{5.802216in}}%
\pgfusepath{stroke}%
\end{pgfscope}%
\begin{pgfscope}%
\pgfpathrectangle{\pgfqpoint{6.720588in}{4.155455in}}{\pgfqpoint{2.279412in}{2.004545in}}%
\pgfusepath{clip}%
\pgfsetbuttcap%
\pgfsetroundjoin%
\pgfsetlinewidth{0.331855pt}%
\definecolor{currentstroke}{rgb}{0.272594,0.025563,0.353093}%
\pgfsetstrokecolor{currentstroke}%
\pgfsetdash{}{0pt}%
\pgfpathmoveto{\pgfqpoint{8.215742in}{5.802216in}}%
\pgfpathlineto{\pgfqpoint{8.166113in}{5.795989in}}%
\pgfusepath{stroke}%
\end{pgfscope}%
\begin{pgfscope}%
\pgfpathrectangle{\pgfqpoint{6.720588in}{4.155455in}}{\pgfqpoint{2.279412in}{2.004545in}}%
\pgfusepath{clip}%
\pgfsetbuttcap%
\pgfsetroundjoin%
\pgfsetlinewidth{0.343132pt}%
\definecolor{currentstroke}{rgb}{0.274952,0.037752,0.364543}%
\pgfsetstrokecolor{currentstroke}%
\pgfsetdash{}{0pt}%
\pgfpathmoveto{\pgfqpoint{8.166113in}{5.795989in}}%
\pgfpathlineto{\pgfqpoint{8.116754in}{5.789222in}}%
\pgfusepath{stroke}%
\end{pgfscope}%
\begin{pgfscope}%
\pgfpathrectangle{\pgfqpoint{6.720588in}{4.155455in}}{\pgfqpoint{2.279412in}{2.004545in}}%
\pgfusepath{clip}%
\pgfsetbuttcap%
\pgfsetroundjoin%
\pgfsetlinewidth{0.350865pt}%
\definecolor{currentstroke}{rgb}{0.276022,0.044167,0.370164}%
\pgfsetstrokecolor{currentstroke}%
\pgfsetdash{}{0pt}%
\pgfpathmoveto{\pgfqpoint{8.116754in}{5.789222in}}%
\pgfpathlineto{\pgfqpoint{8.067918in}{5.780186in}}%
\pgfusepath{stroke}%
\end{pgfscope}%
\begin{pgfscope}%
\pgfpathrectangle{\pgfqpoint{6.720588in}{4.155455in}}{\pgfqpoint{2.279412in}{2.004545in}}%
\pgfusepath{clip}%
\pgfsetbuttcap%
\pgfsetroundjoin%
\pgfsetlinewidth{0.346609pt}%
\definecolor{currentstroke}{rgb}{0.274952,0.037752,0.364543}%
\pgfsetstrokecolor{currentstroke}%
\pgfsetdash{}{0pt}%
\pgfpathmoveto{\pgfqpoint{8.154043in}{4.544970in}}%
\pgfpathlineto{\pgfqpoint{8.110719in}{4.549714in}}%
\pgfusepath{stroke}%
\end{pgfscope}%
\begin{pgfscope}%
\pgfpathrectangle{\pgfqpoint{6.720588in}{4.155455in}}{\pgfqpoint{2.279412in}{2.004545in}}%
\pgfusepath{clip}%
\pgfsetbuttcap%
\pgfsetroundjoin%
\pgfsetlinewidth{0.343683pt}%
\definecolor{currentstroke}{rgb}{0.274952,0.037752,0.364543}%
\pgfsetstrokecolor{currentstroke}%
\pgfsetdash{}{0pt}%
\pgfpathmoveto{\pgfqpoint{8.110719in}{4.549714in}}%
\pgfpathlineto{\pgfqpoint{8.062403in}{4.559914in}}%
\pgfusepath{stroke}%
\end{pgfscope}%
\begin{pgfscope}%
\pgfpathrectangle{\pgfqpoint{6.720588in}{4.155455in}}{\pgfqpoint{2.279412in}{2.004545in}}%
\pgfusepath{clip}%
\pgfsetbuttcap%
\pgfsetroundjoin%
\pgfsetlinewidth{0.327050pt}%
\definecolor{currentstroke}{rgb}{0.271305,0.019942,0.347269}%
\pgfsetstrokecolor{currentstroke}%
\pgfsetdash{}{0pt}%
\pgfpathmoveto{\pgfqpoint{8.062403in}{4.559914in}}%
\pgfpathlineto{\pgfqpoint{8.014170in}{4.571339in}}%
\pgfusepath{stroke}%
\end{pgfscope}%
\begin{pgfscope}%
\pgfpathrectangle{\pgfqpoint{6.720588in}{4.155455in}}{\pgfqpoint{2.279412in}{2.004545in}}%
\pgfusepath{clip}%
\pgfsetbuttcap%
\pgfsetroundjoin%
\pgfsetlinewidth{0.336916pt}%
\definecolor{currentstroke}{rgb}{0.273809,0.031497,0.358853}%
\pgfsetstrokecolor{currentstroke}%
\pgfsetdash{}{0pt}%
\pgfpathmoveto{\pgfqpoint{8.014170in}{4.571339in}}%
\pgfpathlineto{\pgfqpoint{7.967225in}{4.584854in}}%
\pgfusepath{stroke}%
\end{pgfscope}%
\begin{pgfscope}%
\pgfpathrectangle{\pgfqpoint{6.720588in}{4.155455in}}{\pgfqpoint{2.279412in}{2.004545in}}%
\pgfusepath{clip}%
\pgfsetbuttcap%
\pgfsetroundjoin%
\pgfsetlinewidth{0.344672pt}%
\definecolor{currentstroke}{rgb}{0.274952,0.037752,0.364543}%
\pgfsetstrokecolor{currentstroke}%
\pgfsetdash{}{0pt}%
\pgfpathmoveto{\pgfqpoint{7.967225in}{4.584854in}}%
\pgfpathlineto{\pgfqpoint{7.926492in}{4.598605in}}%
\pgfusepath{stroke}%
\end{pgfscope}%
\begin{pgfscope}%
\pgfpathrectangle{\pgfqpoint{6.720588in}{4.155455in}}{\pgfqpoint{2.279412in}{2.004545in}}%
\pgfusepath{clip}%
\pgfsetbuttcap%
\pgfsetroundjoin%
\pgfsetlinewidth{0.336748pt}%
\definecolor{currentstroke}{rgb}{0.273809,0.031497,0.358853}%
\pgfsetstrokecolor{currentstroke}%
\pgfsetdash{}{0pt}%
\pgfpathmoveto{\pgfqpoint{8.527089in}{4.751766in}}%
\pgfpathlineto{\pgfqpoint{8.477338in}{4.755562in}}%
\pgfusepath{stroke}%
\end{pgfscope}%
\begin{pgfscope}%
\pgfpathrectangle{\pgfqpoint{6.720588in}{4.155455in}}{\pgfqpoint{2.279412in}{2.004545in}}%
\pgfusepath{clip}%
\pgfsetbuttcap%
\pgfsetroundjoin%
\pgfsetlinewidth{0.334940pt}%
\definecolor{currentstroke}{rgb}{0.272594,0.025563,0.353093}%
\pgfsetstrokecolor{currentstroke}%
\pgfsetdash{}{0pt}%
\pgfpathmoveto{\pgfqpoint{8.477338in}{4.755562in}}%
\pgfpathlineto{\pgfqpoint{8.427634in}{4.759518in}}%
\pgfusepath{stroke}%
\end{pgfscope}%
\begin{pgfscope}%
\pgfpathrectangle{\pgfqpoint{6.720588in}{4.155455in}}{\pgfqpoint{2.279412in}{2.004545in}}%
\pgfusepath{clip}%
\pgfsetbuttcap%
\pgfsetroundjoin%
\pgfsetlinewidth{0.333227pt}%
\definecolor{currentstroke}{rgb}{0.272594,0.025563,0.353093}%
\pgfsetstrokecolor{currentstroke}%
\pgfsetdash{}{0pt}%
\pgfpathmoveto{\pgfqpoint{8.427634in}{4.759518in}}%
\pgfpathlineto{\pgfqpoint{8.377545in}{4.761529in}}%
\pgfusepath{stroke}%
\end{pgfscope}%
\begin{pgfscope}%
\pgfpathrectangle{\pgfqpoint{6.720588in}{4.155455in}}{\pgfqpoint{2.279412in}{2.004545in}}%
\pgfusepath{clip}%
\pgfsetbuttcap%
\pgfsetroundjoin%
\pgfsetlinewidth{0.359474pt}%
\definecolor{currentstroke}{rgb}{0.277018,0.050344,0.375715}%
\pgfsetstrokecolor{currentstroke}%
\pgfsetdash{}{0pt}%
\pgfpathmoveto{\pgfqpoint{8.377545in}{4.761529in}}%
\pgfpathlineto{\pgfqpoint{8.327497in}{4.764337in}}%
\pgfusepath{stroke}%
\end{pgfscope}%
\begin{pgfscope}%
\pgfpathrectangle{\pgfqpoint{6.720588in}{4.155455in}}{\pgfqpoint{2.279412in}{2.004545in}}%
\pgfusepath{clip}%
\pgfsetbuttcap%
\pgfsetroundjoin%
\pgfsetlinewidth{0.376508pt}%
\definecolor{currentstroke}{rgb}{0.278791,0.062145,0.386592}%
\pgfsetstrokecolor{currentstroke}%
\pgfsetdash{}{0pt}%
\pgfpathmoveto{\pgfqpoint{8.327497in}{4.764337in}}%
\pgfpathlineto{\pgfqpoint{8.277498in}{4.767636in}}%
\pgfusepath{stroke}%
\end{pgfscope}%
\begin{pgfscope}%
\pgfpathrectangle{\pgfqpoint{6.720588in}{4.155455in}}{\pgfqpoint{2.279412in}{2.004545in}}%
\pgfusepath{clip}%
\pgfsetbuttcap%
\pgfsetroundjoin%
\pgfsetlinewidth{0.389342pt}%
\definecolor{currentstroke}{rgb}{0.280267,0.073417,0.397163}%
\pgfsetstrokecolor{currentstroke}%
\pgfsetdash{}{0pt}%
\pgfpathmoveto{\pgfqpoint{8.277498in}{4.767636in}}%
\pgfpathlineto{\pgfqpoint{8.227586in}{4.771808in}}%
\pgfusepath{stroke}%
\end{pgfscope}%
\begin{pgfscope}%
\pgfpathrectangle{\pgfqpoint{6.720588in}{4.155455in}}{\pgfqpoint{2.279412in}{2.004545in}}%
\pgfusepath{clip}%
\pgfsetbuttcap%
\pgfsetroundjoin%
\pgfsetlinewidth{0.420497pt}%
\definecolor{currentstroke}{rgb}{0.282656,0.100196,0.422160}%
\pgfsetstrokecolor{currentstroke}%
\pgfsetdash{}{0pt}%
\pgfpathmoveto{\pgfqpoint{8.227586in}{4.771808in}}%
\pgfpathlineto{\pgfqpoint{8.177794in}{4.777055in}}%
\pgfusepath{stroke}%
\end{pgfscope}%
\begin{pgfscope}%
\pgfpathrectangle{\pgfqpoint{6.720588in}{4.155455in}}{\pgfqpoint{2.279412in}{2.004545in}}%
\pgfusepath{clip}%
\pgfsetbuttcap%
\pgfsetroundjoin%
\pgfsetlinewidth{0.432331pt}%
\definecolor{currentstroke}{rgb}{0.283091,0.110553,0.431554}%
\pgfsetstrokecolor{currentstroke}%
\pgfsetdash{}{0pt}%
\pgfpathmoveto{\pgfqpoint{8.177794in}{4.777055in}}%
\pgfpathlineto{\pgfqpoint{8.128153in}{4.783310in}}%
\pgfusepath{stroke}%
\end{pgfscope}%
\begin{pgfscope}%
\pgfpathrectangle{\pgfqpoint{6.720588in}{4.155455in}}{\pgfqpoint{2.279412in}{2.004545in}}%
\pgfusepath{clip}%
\pgfsetbuttcap%
\pgfsetroundjoin%
\pgfsetlinewidth{0.438527pt}%
\definecolor{currentstroke}{rgb}{0.283197,0.115680,0.436115}%
\pgfsetstrokecolor{currentstroke}%
\pgfsetdash{}{0pt}%
\pgfpathmoveto{\pgfqpoint{8.128153in}{4.783310in}}%
\pgfpathlineto{\pgfqpoint{8.078743in}{4.790783in}}%
\pgfusepath{stroke}%
\end{pgfscope}%
\begin{pgfscope}%
\pgfpathrectangle{\pgfqpoint{6.720588in}{4.155455in}}{\pgfqpoint{2.279412in}{2.004545in}}%
\pgfusepath{clip}%
\pgfsetbuttcap%
\pgfsetroundjoin%
\pgfsetlinewidth{0.480344pt}%
\definecolor{currentstroke}{rgb}{0.282290,0.145912,0.461510}%
\pgfsetstrokecolor{currentstroke}%
\pgfsetdash{}{0pt}%
\pgfpathmoveto{\pgfqpoint{8.078743in}{4.790783in}}%
\pgfpathlineto{\pgfqpoint{8.029598in}{4.799480in}}%
\pgfusepath{stroke}%
\end{pgfscope}%
\begin{pgfscope}%
\pgfpathrectangle{\pgfqpoint{6.720588in}{4.155455in}}{\pgfqpoint{2.279412in}{2.004545in}}%
\pgfusepath{clip}%
\pgfsetbuttcap%
\pgfsetroundjoin%
\pgfsetlinewidth{0.490314pt}%
\definecolor{currentstroke}{rgb}{0.281887,0.150881,0.465405}%
\pgfsetstrokecolor{currentstroke}%
\pgfsetdash{}{0pt}%
\pgfpathmoveto{\pgfqpoint{8.029598in}{4.799480in}}%
\pgfpathlineto{\pgfqpoint{7.981076in}{4.810391in}}%
\pgfusepath{stroke}%
\end{pgfscope}%
\begin{pgfscope}%
\pgfpathrectangle{\pgfqpoint{6.720588in}{4.155455in}}{\pgfqpoint{2.279412in}{2.004545in}}%
\pgfusepath{clip}%
\pgfsetbuttcap%
\pgfsetroundjoin%
\pgfsetlinewidth{0.486882pt}%
\definecolor{currentstroke}{rgb}{0.281887,0.150881,0.465405}%
\pgfsetstrokecolor{currentstroke}%
\pgfsetdash{}{0pt}%
\pgfpathmoveto{\pgfqpoint{7.981076in}{4.810391in}}%
\pgfpathlineto{\pgfqpoint{7.933529in}{4.824180in}}%
\pgfusepath{stroke}%
\end{pgfscope}%
\begin{pgfscope}%
\pgfpathrectangle{\pgfqpoint{6.720588in}{4.155455in}}{\pgfqpoint{2.279412in}{2.004545in}}%
\pgfusepath{clip}%
\pgfsetbuttcap%
\pgfsetroundjoin%
\pgfsetlinewidth{0.321546pt}%
\definecolor{currentstroke}{rgb}{0.269944,0.014625,0.341379}%
\pgfsetstrokecolor{currentstroke}%
\pgfsetdash{}{0pt}%
\pgfpathmoveto{\pgfqpoint{8.527089in}{5.608795in}}%
\pgfpathlineto{\pgfqpoint{8.476992in}{5.607548in}}%
\pgfusepath{stroke}%
\end{pgfscope}%
\begin{pgfscope}%
\pgfpathrectangle{\pgfqpoint{6.720588in}{4.155455in}}{\pgfqpoint{2.279412in}{2.004545in}}%
\pgfusepath{clip}%
\pgfsetbuttcap%
\pgfsetroundjoin%
\pgfsetlinewidth{0.331632pt}%
\definecolor{currentstroke}{rgb}{0.272594,0.025563,0.353093}%
\pgfsetstrokecolor{currentstroke}%
\pgfsetdash{}{0pt}%
\pgfpathmoveto{\pgfqpoint{8.476992in}{5.607548in}}%
\pgfpathlineto{\pgfqpoint{8.426927in}{5.605307in}}%
\pgfusepath{stroke}%
\end{pgfscope}%
\begin{pgfscope}%
\pgfpathrectangle{\pgfqpoint{6.720588in}{4.155455in}}{\pgfqpoint{2.279412in}{2.004545in}}%
\pgfusepath{clip}%
\pgfsetbuttcap%
\pgfsetroundjoin%
\pgfsetlinewidth{0.341576pt}%
\definecolor{currentstroke}{rgb}{0.273809,0.031497,0.358853}%
\pgfsetstrokecolor{currentstroke}%
\pgfsetdash{}{0pt}%
\pgfpathmoveto{\pgfqpoint{8.426927in}{5.605307in}}%
\pgfpathlineto{\pgfqpoint{8.376931in}{5.601931in}}%
\pgfusepath{stroke}%
\end{pgfscope}%
\begin{pgfscope}%
\pgfpathrectangle{\pgfqpoint{6.720588in}{4.155455in}}{\pgfqpoint{2.279412in}{2.004545in}}%
\pgfusepath{clip}%
\pgfsetbuttcap%
\pgfsetroundjoin%
\pgfsetlinewidth{0.352230pt}%
\definecolor{currentstroke}{rgb}{0.276022,0.044167,0.370164}%
\pgfsetstrokecolor{currentstroke}%
\pgfsetdash{}{0pt}%
\pgfpathmoveto{\pgfqpoint{8.376931in}{5.601931in}}%
\pgfpathlineto{\pgfqpoint{8.326950in}{5.598327in}}%
\pgfusepath{stroke}%
\end{pgfscope}%
\begin{pgfscope}%
\pgfpathrectangle{\pgfqpoint{6.720588in}{4.155455in}}{\pgfqpoint{2.279412in}{2.004545in}}%
\pgfusepath{clip}%
\pgfsetbuttcap%
\pgfsetroundjoin%
\pgfsetlinewidth{0.362761pt}%
\definecolor{currentstroke}{rgb}{0.277018,0.050344,0.375715}%
\pgfsetstrokecolor{currentstroke}%
\pgfsetdash{}{0pt}%
\pgfpathmoveto{\pgfqpoint{8.326950in}{5.598327in}}%
\pgfpathlineto{\pgfqpoint{8.276984in}{5.594602in}}%
\pgfusepath{stroke}%
\end{pgfscope}%
\begin{pgfscope}%
\pgfpathrectangle{\pgfqpoint{6.720588in}{4.155455in}}{\pgfqpoint{2.279412in}{2.004545in}}%
\pgfusepath{clip}%
\pgfsetbuttcap%
\pgfsetroundjoin%
\pgfsetlinewidth{0.367713pt}%
\definecolor{currentstroke}{rgb}{0.277941,0.056324,0.381191}%
\pgfsetstrokecolor{currentstroke}%
\pgfsetdash{}{0pt}%
\pgfpathmoveto{\pgfqpoint{8.276984in}{5.594602in}}%
\pgfpathlineto{\pgfqpoint{8.227093in}{5.590197in}}%
\pgfusepath{stroke}%
\end{pgfscope}%
\begin{pgfscope}%
\pgfpathrectangle{\pgfqpoint{6.720588in}{4.155455in}}{\pgfqpoint{2.279412in}{2.004545in}}%
\pgfusepath{clip}%
\pgfsetbuttcap%
\pgfsetroundjoin%
\pgfsetlinewidth{0.384605pt}%
\definecolor{currentstroke}{rgb}{0.280267,0.073417,0.397163}%
\pgfsetstrokecolor{currentstroke}%
\pgfsetdash{}{0pt}%
\pgfpathmoveto{\pgfqpoint{8.227093in}{5.590197in}}%
\pgfpathlineto{\pgfqpoint{8.177360in}{5.584580in}}%
\pgfusepath{stroke}%
\end{pgfscope}%
\begin{pgfscope}%
\pgfpathrectangle{\pgfqpoint{6.720588in}{4.155455in}}{\pgfqpoint{2.279412in}{2.004545in}}%
\pgfusepath{clip}%
\pgfsetbuttcap%
\pgfsetroundjoin%
\pgfsetlinewidth{0.392481pt}%
\definecolor{currentstroke}{rgb}{0.280894,0.078907,0.402329}%
\pgfsetstrokecolor{currentstroke}%
\pgfsetdash{}{0pt}%
\pgfpathmoveto{\pgfqpoint{8.177360in}{5.584580in}}%
\pgfpathlineto{\pgfqpoint{8.127882in}{5.577449in}}%
\pgfusepath{stroke}%
\end{pgfscope}%
\begin{pgfscope}%
\pgfpathrectangle{\pgfqpoint{6.720588in}{4.155455in}}{\pgfqpoint{2.279412in}{2.004545in}}%
\pgfusepath{clip}%
\pgfsetbuttcap%
\pgfsetroundjoin%
\pgfsetlinewidth{0.429400pt}%
\definecolor{currentstroke}{rgb}{0.282910,0.105393,0.426902}%
\pgfsetstrokecolor{currentstroke}%
\pgfsetdash{}{0pt}%
\pgfpathmoveto{\pgfqpoint{8.127882in}{5.577449in}}%
\pgfpathlineto{\pgfqpoint{8.078632in}{5.569178in}}%
\pgfusepath{stroke}%
\end{pgfscope}%
\begin{pgfscope}%
\pgfpathrectangle{\pgfqpoint{6.720588in}{4.155455in}}{\pgfqpoint{2.279412in}{2.004545in}}%
\pgfusepath{clip}%
\pgfsetbuttcap%
\pgfsetroundjoin%
\pgfsetlinewidth{0.425429pt}%
\definecolor{currentstroke}{rgb}{0.282910,0.105393,0.426902}%
\pgfsetstrokecolor{currentstroke}%
\pgfsetdash{}{0pt}%
\pgfpathmoveto{\pgfqpoint{8.078632in}{5.569178in}}%
\pgfpathlineto{\pgfqpoint{8.029840in}{5.559172in}}%
\pgfusepath{stroke}%
\end{pgfscope}%
\begin{pgfscope}%
\pgfpathrectangle{\pgfqpoint{6.720588in}{4.155455in}}{\pgfqpoint{2.279412in}{2.004545in}}%
\pgfusepath{clip}%
\pgfsetbuttcap%
\pgfsetroundjoin%
\pgfsetlinewidth{0.384798pt}%
\definecolor{currentstroke}{rgb}{0.280267,0.073417,0.397163}%
\pgfsetstrokecolor{currentstroke}%
\pgfsetdash{}{0pt}%
\pgfpathmoveto{\pgfqpoint{8.029840in}{5.559172in}}%
\pgfpathlineto{\pgfqpoint{7.981765in}{5.546786in}}%
\pgfusepath{stroke}%
\end{pgfscope}%
\begin{pgfscope}%
\pgfpathrectangle{\pgfqpoint{6.720588in}{4.155455in}}{\pgfqpoint{2.279412in}{2.004545in}}%
\pgfusepath{clip}%
\pgfsetbuttcap%
\pgfsetroundjoin%
\pgfsetlinewidth{0.429082pt}%
\definecolor{currentstroke}{rgb}{0.282910,0.105393,0.426902}%
\pgfsetstrokecolor{currentstroke}%
\pgfsetdash{}{0pt}%
\pgfpathmoveto{\pgfqpoint{7.981765in}{5.546786in}}%
\pgfpathlineto{\pgfqpoint{7.934846in}{5.531543in}}%
\pgfusepath{stroke}%
\end{pgfscope}%
\begin{pgfscope}%
\pgfpathrectangle{\pgfqpoint{6.720588in}{4.155455in}}{\pgfqpoint{2.279412in}{2.004545in}}%
\pgfusepath{clip}%
\pgfsetbuttcap%
\pgfsetroundjoin%
\pgfsetlinewidth{0.333171pt}%
\definecolor{currentstroke}{rgb}{0.272594,0.025563,0.353093}%
\pgfsetstrokecolor{currentstroke}%
\pgfsetdash{}{0pt}%
\pgfpathmoveto{\pgfqpoint{8.475797in}{4.841980in}}%
\pgfpathlineto{\pgfqpoint{8.425694in}{4.843666in}}%
\pgfusepath{stroke}%
\end{pgfscope}%
\begin{pgfscope}%
\pgfpathrectangle{\pgfqpoint{6.720588in}{4.155455in}}{\pgfqpoint{2.279412in}{2.004545in}}%
\pgfusepath{clip}%
\pgfsetbuttcap%
\pgfsetroundjoin%
\pgfsetlinewidth{0.364284pt}%
\definecolor{currentstroke}{rgb}{0.277941,0.056324,0.381191}%
\pgfsetstrokecolor{currentstroke}%
\pgfsetdash{}{0pt}%
\pgfpathmoveto{\pgfqpoint{8.425694in}{4.843666in}}%
\pgfpathlineto{\pgfqpoint{8.375580in}{4.845333in}}%
\pgfusepath{stroke}%
\end{pgfscope}%
\begin{pgfscope}%
\pgfpathrectangle{\pgfqpoint{6.720588in}{4.155455in}}{\pgfqpoint{2.279412in}{2.004545in}}%
\pgfusepath{clip}%
\pgfsetbuttcap%
\pgfsetroundjoin%
\pgfsetlinewidth{0.372541pt}%
\definecolor{currentstroke}{rgb}{0.278791,0.062145,0.386592}%
\pgfsetstrokecolor{currentstroke}%
\pgfsetdash{}{0pt}%
\pgfpathmoveto{\pgfqpoint{8.375580in}{4.845333in}}%
\pgfpathlineto{\pgfqpoint{8.325489in}{4.847392in}}%
\pgfusepath{stroke}%
\end{pgfscope}%
\begin{pgfscope}%
\pgfpathrectangle{\pgfqpoint{6.720588in}{4.155455in}}{\pgfqpoint{2.279412in}{2.004545in}}%
\pgfusepath{clip}%
\pgfsetbuttcap%
\pgfsetroundjoin%
\pgfsetlinewidth{0.418000pt}%
\definecolor{currentstroke}{rgb}{0.282656,0.100196,0.422160}%
\pgfsetstrokecolor{currentstroke}%
\pgfsetdash{}{0pt}%
\pgfpathmoveto{\pgfqpoint{8.325489in}{4.847392in}}%
\pgfpathlineto{\pgfqpoint{8.275432in}{4.850063in}}%
\pgfusepath{stroke}%
\end{pgfscope}%
\begin{pgfscope}%
\pgfpathrectangle{\pgfqpoint{6.720588in}{4.155455in}}{\pgfqpoint{2.279412in}{2.004545in}}%
\pgfusepath{clip}%
\pgfsetbuttcap%
\pgfsetroundjoin%
\pgfsetlinewidth{0.448029pt}%
\definecolor{currentstroke}{rgb}{0.283229,0.120777,0.440584}%
\pgfsetstrokecolor{currentstroke}%
\pgfsetdash{}{0pt}%
\pgfpathmoveto{\pgfqpoint{8.275432in}{4.850063in}}%
\pgfpathlineto{\pgfqpoint{8.225417in}{4.853282in}}%
\pgfusepath{stroke}%
\end{pgfscope}%
\begin{pgfscope}%
\pgfpathrectangle{\pgfqpoint{6.720588in}{4.155455in}}{\pgfqpoint{2.279412in}{2.004545in}}%
\pgfusepath{clip}%
\pgfsetbuttcap%
\pgfsetroundjoin%
\pgfsetlinewidth{0.330828pt}%
\definecolor{currentstroke}{rgb}{0.272594,0.025563,0.353093}%
\pgfsetstrokecolor{currentstroke}%
\pgfsetdash{}{0pt}%
\pgfpathmoveto{\pgfqpoint{8.373213in}{5.699009in}}%
\pgfpathlineto{\pgfqpoint{8.323504in}{5.693438in}}%
\pgfusepath{stroke}%
\end{pgfscope}%
\begin{pgfscope}%
\pgfpathrectangle{\pgfqpoint{6.720588in}{4.155455in}}{\pgfqpoint{2.279412in}{2.004545in}}%
\pgfusepath{clip}%
\pgfsetbuttcap%
\pgfsetroundjoin%
\pgfsetlinewidth{0.335356pt}%
\definecolor{currentstroke}{rgb}{0.272594,0.025563,0.353093}%
\pgfsetstrokecolor{currentstroke}%
\pgfsetdash{}{0pt}%
\pgfpathmoveto{\pgfqpoint{8.323504in}{5.693438in}}%
\pgfpathlineto{\pgfqpoint{8.273916in}{5.687051in}}%
\pgfusepath{stroke}%
\end{pgfscope}%
\begin{pgfscope}%
\pgfpathrectangle{\pgfqpoint{6.720588in}{4.155455in}}{\pgfqpoint{2.279412in}{2.004545in}}%
\pgfusepath{clip}%
\pgfsetbuttcap%
\pgfsetroundjoin%
\pgfsetlinewidth{0.355540pt}%
\definecolor{currentstroke}{rgb}{0.276022,0.044167,0.370164}%
\pgfsetstrokecolor{currentstroke}%
\pgfsetdash{}{0pt}%
\pgfpathmoveto{\pgfqpoint{8.273916in}{5.687051in}}%
\pgfpathlineto{\pgfqpoint{8.224285in}{5.681005in}}%
\pgfusepath{stroke}%
\end{pgfscope}%
\begin{pgfscope}%
\pgfpathrectangle{\pgfqpoint{6.720588in}{4.155455in}}{\pgfqpoint{2.279412in}{2.004545in}}%
\pgfusepath{clip}%
\pgfsetbuttcap%
\pgfsetroundjoin%
\pgfsetlinewidth{0.366859pt}%
\definecolor{currentstroke}{rgb}{0.277941,0.056324,0.381191}%
\pgfsetstrokecolor{currentstroke}%
\pgfsetdash{}{0pt}%
\pgfpathmoveto{\pgfqpoint{8.224285in}{5.681005in}}%
\pgfpathlineto{\pgfqpoint{8.175044in}{5.673007in}}%
\pgfusepath{stroke}%
\end{pgfscope}%
\begin{pgfscope}%
\pgfpathrectangle{\pgfqpoint{6.720588in}{4.155455in}}{\pgfqpoint{2.279412in}{2.004545in}}%
\pgfusepath{clip}%
\pgfsetbuttcap%
\pgfsetroundjoin%
\pgfsetlinewidth{0.327758pt}%
\definecolor{currentstroke}{rgb}{0.271305,0.019942,0.347269}%
\pgfsetstrokecolor{currentstroke}%
\pgfsetdash{}{0pt}%
\pgfpathmoveto{\pgfqpoint{8.175044in}{5.673007in}}%
\pgfpathlineto{\pgfqpoint{8.126011in}{5.663885in}}%
\pgfusepath{stroke}%
\end{pgfscope}%
\begin{pgfscope}%
\pgfpathrectangle{\pgfqpoint{6.720588in}{4.155455in}}{\pgfqpoint{2.279412in}{2.004545in}}%
\pgfusepath{clip}%
\pgfsetbuttcap%
\pgfsetroundjoin%
\pgfsetlinewidth{0.383011pt}%
\definecolor{currentstroke}{rgb}{0.279566,0.067836,0.391917}%
\pgfsetstrokecolor{currentstroke}%
\pgfsetdash{}{0pt}%
\pgfpathmoveto{\pgfqpoint{8.126011in}{5.663885in}}%
\pgfpathlineto{\pgfqpoint{8.077124in}{5.654252in}}%
\pgfusepath{stroke}%
\end{pgfscope}%
\begin{pgfscope}%
\pgfpathrectangle{\pgfqpoint{6.720588in}{4.155455in}}{\pgfqpoint{2.279412in}{2.004545in}}%
\pgfusepath{clip}%
\pgfsetbuttcap%
\pgfsetroundjoin%
\pgfsetlinewidth{0.356543pt}%
\definecolor{currentstroke}{rgb}{0.277018,0.050344,0.375715}%
\pgfsetstrokecolor{currentstroke}%
\pgfsetdash{}{0pt}%
\pgfpathmoveto{\pgfqpoint{8.077124in}{5.654252in}}%
\pgfpathlineto{\pgfqpoint{8.028816in}{5.642491in}}%
\pgfusepath{stroke}%
\end{pgfscope}%
\begin{pgfscope}%
\pgfpathrectangle{\pgfqpoint{6.720588in}{4.155455in}}{\pgfqpoint{2.279412in}{2.004545in}}%
\pgfusepath{clip}%
\pgfsetbuttcap%
\pgfsetroundjoin%
\pgfsetlinewidth{0.367428pt}%
\definecolor{currentstroke}{rgb}{0.277941,0.056324,0.381191}%
\pgfsetstrokecolor{currentstroke}%
\pgfsetdash{}{0pt}%
\pgfpathmoveto{\pgfqpoint{8.028816in}{5.642491in}}%
\pgfpathlineto{\pgfqpoint{7.981100in}{5.629004in}}%
\pgfusepath{stroke}%
\end{pgfscope}%
\begin{pgfscope}%
\pgfpathrectangle{\pgfqpoint{6.720588in}{4.155455in}}{\pgfqpoint{2.279412in}{2.004545in}}%
\pgfusepath{clip}%
\pgfsetbuttcap%
\pgfsetroundjoin%
\pgfsetlinewidth{0.370196pt}%
\definecolor{currentstroke}{rgb}{0.278791,0.062145,0.386592}%
\pgfsetstrokecolor{currentstroke}%
\pgfsetdash{}{0pt}%
\pgfpathmoveto{\pgfqpoint{7.981100in}{5.629004in}}%
\pgfpathlineto{\pgfqpoint{7.934997in}{5.611923in}}%
\pgfusepath{stroke}%
\end{pgfscope}%
\begin{pgfscope}%
\pgfpathrectangle{\pgfqpoint{6.720588in}{4.155455in}}{\pgfqpoint{2.279412in}{2.004545in}}%
\pgfusepath{clip}%
\pgfsetbuttcap%
\pgfsetroundjoin%
\pgfsetlinewidth{0.408245pt}%
\definecolor{currentstroke}{rgb}{0.281924,0.089666,0.412415}%
\pgfsetstrokecolor{currentstroke}%
\pgfsetdash{}{0pt}%
\pgfpathmoveto{\pgfqpoint{7.934997in}{5.611923in}}%
\pgfpathlineto{\pgfqpoint{7.890220in}{5.593222in}}%
\pgfusepath{stroke}%
\end{pgfscope}%
\begin{pgfscope}%
\pgfpathrectangle{\pgfqpoint{6.720588in}{4.155455in}}{\pgfqpoint{2.279412in}{2.004545in}}%
\pgfusepath{clip}%
\pgfsetbuttcap%
\pgfsetroundjoin%
\pgfsetlinewidth{0.369317pt}%
\definecolor{currentstroke}{rgb}{0.277941,0.056324,0.381191}%
\pgfsetstrokecolor{currentstroke}%
\pgfsetdash{}{0pt}%
\pgfpathmoveto{\pgfqpoint{8.165934in}{5.707330in}}%
\pgfpathlineto{\pgfqpoint{8.116754in}{5.699009in}}%
\pgfusepath{stroke}%
\end{pgfscope}%
\begin{pgfscope}%
\pgfpathrectangle{\pgfqpoint{6.720588in}{4.155455in}}{\pgfqpoint{2.279412in}{2.004545in}}%
\pgfusepath{clip}%
\pgfsetbuttcap%
\pgfsetroundjoin%
\pgfsetlinewidth{0.364308pt}%
\definecolor{currentstroke}{rgb}{0.277941,0.056324,0.381191}%
\pgfsetstrokecolor{currentstroke}%
\pgfsetdash{}{0pt}%
\pgfpathmoveto{\pgfqpoint{8.116754in}{5.699009in}}%
\pgfpathlineto{\pgfqpoint{8.068034in}{5.688682in}}%
\pgfusepath{stroke}%
\end{pgfscope}%
\begin{pgfscope}%
\pgfpathrectangle{\pgfqpoint{6.720588in}{4.155455in}}{\pgfqpoint{2.279412in}{2.004545in}}%
\pgfusepath{clip}%
\pgfsetbuttcap%
\pgfsetroundjoin%
\pgfsetlinewidth{0.357904pt}%
\definecolor{currentstroke}{rgb}{0.277018,0.050344,0.375715}%
\pgfsetstrokecolor{currentstroke}%
\pgfsetdash{}{0pt}%
\pgfpathmoveto{\pgfqpoint{8.068034in}{5.688682in}}%
\pgfpathlineto{\pgfqpoint{8.019416in}{5.678091in}}%
\pgfusepath{stroke}%
\end{pgfscope}%
\begin{pgfscope}%
\pgfpathrectangle{\pgfqpoint{6.720588in}{4.155455in}}{\pgfqpoint{2.279412in}{2.004545in}}%
\pgfusepath{clip}%
\pgfsetbuttcap%
\pgfsetroundjoin%
\pgfsetlinewidth{0.353716pt}%
\definecolor{currentstroke}{rgb}{0.276022,0.044167,0.370164}%
\pgfsetstrokecolor{currentstroke}%
\pgfsetdash{}{0pt}%
\pgfpathmoveto{\pgfqpoint{8.019416in}{5.678091in}}%
\pgfpathlineto{\pgfqpoint{7.973044in}{5.662418in}}%
\pgfusepath{stroke}%
\end{pgfscope}%
\begin{pgfscope}%
\pgfpathrectangle{\pgfqpoint{6.720588in}{4.155455in}}{\pgfqpoint{2.279412in}{2.004545in}}%
\pgfusepath{clip}%
\pgfsetbuttcap%
\pgfsetroundjoin%
\pgfsetlinewidth{0.397387pt}%
\definecolor{currentstroke}{rgb}{0.281446,0.084320,0.407414}%
\pgfsetstrokecolor{currentstroke}%
\pgfsetdash{}{0pt}%
\pgfpathmoveto{\pgfqpoint{7.973044in}{5.662418in}}%
\pgfpathlineto{\pgfqpoint{7.933223in}{5.646745in}}%
\pgfusepath{stroke}%
\end{pgfscope}%
\begin{pgfscope}%
\pgfpathrectangle{\pgfqpoint{6.720588in}{4.155455in}}{\pgfqpoint{2.279412in}{2.004545in}}%
\pgfusepath{clip}%
\pgfsetbuttcap%
\pgfsetroundjoin%
\pgfsetlinewidth{0.324255pt}%
\definecolor{currentstroke}{rgb}{0.271305,0.019942,0.347269}%
\pgfsetstrokecolor{currentstroke}%
\pgfsetdash{}{0pt}%
\pgfpathmoveto{\pgfqpoint{8.465158in}{4.627407in}}%
\pgfpathlineto{\pgfqpoint{8.415215in}{4.631005in}}%
\pgfusepath{stroke}%
\end{pgfscope}%
\begin{pgfscope}%
\pgfpathrectangle{\pgfqpoint{6.720588in}{4.155455in}}{\pgfqpoint{2.279412in}{2.004545in}}%
\pgfusepath{clip}%
\pgfsetbuttcap%
\pgfsetroundjoin%
\pgfsetlinewidth{0.327498pt}%
\definecolor{currentstroke}{rgb}{0.271305,0.019942,0.347269}%
\pgfsetstrokecolor{currentstroke}%
\pgfsetdash{}{0pt}%
\pgfpathmoveto{\pgfqpoint{8.415215in}{4.631005in}}%
\pgfpathlineto{\pgfqpoint{8.365221in}{4.633720in}}%
\pgfusepath{stroke}%
\end{pgfscope}%
\begin{pgfscope}%
\pgfpathrectangle{\pgfqpoint{6.720588in}{4.155455in}}{\pgfqpoint{2.279412in}{2.004545in}}%
\pgfusepath{clip}%
\pgfsetbuttcap%
\pgfsetroundjoin%
\pgfsetlinewidth{0.338093pt}%
\definecolor{currentstroke}{rgb}{0.273809,0.031497,0.358853}%
\pgfsetstrokecolor{currentstroke}%
\pgfsetdash{}{0pt}%
\pgfpathmoveto{\pgfqpoint{8.365221in}{4.633720in}}%
\pgfpathlineto{\pgfqpoint{8.315211in}{4.636565in}}%
\pgfusepath{stroke}%
\end{pgfscope}%
\begin{pgfscope}%
\pgfpathrectangle{\pgfqpoint{6.720588in}{4.155455in}}{\pgfqpoint{2.279412in}{2.004545in}}%
\pgfusepath{clip}%
\pgfsetbuttcap%
\pgfsetroundjoin%
\pgfsetlinewidth{0.349303pt}%
\definecolor{currentstroke}{rgb}{0.276022,0.044167,0.370164}%
\pgfsetstrokecolor{currentstroke}%
\pgfsetdash{}{0pt}%
\pgfpathmoveto{\pgfqpoint{8.315211in}{4.636565in}}%
\pgfpathlineto{\pgfqpoint{8.265436in}{4.641832in}}%
\pgfusepath{stroke}%
\end{pgfscope}%
\begin{pgfscope}%
\pgfpathrectangle{\pgfqpoint{6.720588in}{4.155455in}}{\pgfqpoint{2.279412in}{2.004545in}}%
\pgfusepath{clip}%
\pgfsetbuttcap%
\pgfsetroundjoin%
\pgfsetlinewidth{0.374347pt}%
\definecolor{currentstroke}{rgb}{0.278791,0.062145,0.386592}%
\pgfsetstrokecolor{currentstroke}%
\pgfsetdash{}{0pt}%
\pgfpathmoveto{\pgfqpoint{8.265436in}{4.641832in}}%
\pgfpathlineto{\pgfqpoint{8.215755in}{4.647793in}}%
\pgfusepath{stroke}%
\end{pgfscope}%
\begin{pgfscope}%
\pgfpathrectangle{\pgfqpoint{6.720588in}{4.155455in}}{\pgfqpoint{2.279412in}{2.004545in}}%
\pgfusepath{clip}%
\pgfsetbuttcap%
\pgfsetroundjoin%
\pgfsetlinewidth{0.367686pt}%
\definecolor{currentstroke}{rgb}{0.277941,0.056324,0.381191}%
\pgfsetstrokecolor{currentstroke}%
\pgfsetdash{}{0pt}%
\pgfpathmoveto{\pgfqpoint{8.215755in}{4.647793in}}%
\pgfpathlineto{\pgfqpoint{8.166125in}{4.654027in}}%
\pgfusepath{stroke}%
\end{pgfscope}%
\begin{pgfscope}%
\pgfpathrectangle{\pgfqpoint{6.720588in}{4.155455in}}{\pgfqpoint{2.279412in}{2.004545in}}%
\pgfusepath{clip}%
\pgfsetbuttcap%
\pgfsetroundjoin%
\pgfsetlinewidth{0.382533pt}%
\definecolor{currentstroke}{rgb}{0.279566,0.067836,0.391917}%
\pgfsetstrokecolor{currentstroke}%
\pgfsetdash{}{0pt}%
\pgfpathmoveto{\pgfqpoint{8.166125in}{4.654027in}}%
\pgfpathlineto{\pgfqpoint{8.116754in}{4.661553in}}%
\pgfusepath{stroke}%
\end{pgfscope}%
\begin{pgfscope}%
\pgfpathrectangle{\pgfqpoint{6.720588in}{4.155455in}}{\pgfqpoint{2.279412in}{2.004545in}}%
\pgfusepath{clip}%
\pgfsetbuttcap%
\pgfsetroundjoin%
\pgfsetlinewidth{0.387584pt}%
\definecolor{currentstroke}{rgb}{0.280267,0.073417,0.397163}%
\pgfsetstrokecolor{currentstroke}%
\pgfsetdash{}{0pt}%
\pgfpathmoveto{\pgfqpoint{8.116754in}{4.661553in}}%
\pgfpathlineto{\pgfqpoint{8.067633in}{4.670266in}}%
\pgfusepath{stroke}%
\end{pgfscope}%
\begin{pgfscope}%
\pgfpathrectangle{\pgfqpoint{6.720588in}{4.155455in}}{\pgfqpoint{2.279412in}{2.004545in}}%
\pgfusepath{clip}%
\pgfsetbuttcap%
\pgfsetroundjoin%
\pgfsetlinewidth{0.328017pt}%
\definecolor{currentstroke}{rgb}{0.271305,0.019942,0.347269}%
\pgfsetstrokecolor{currentstroke}%
\pgfsetdash{}{0pt}%
\pgfpathmoveto{\pgfqpoint{8.468707in}{5.674783in}}%
\pgfpathlineto{\pgfqpoint{8.418762in}{5.671547in}}%
\pgfusepath{stroke}%
\end{pgfscope}%
\begin{pgfscope}%
\pgfpathrectangle{\pgfqpoint{6.720588in}{4.155455in}}{\pgfqpoint{2.279412in}{2.004545in}}%
\pgfusepath{clip}%
\pgfsetbuttcap%
\pgfsetroundjoin%
\pgfsetlinewidth{0.331448pt}%
\definecolor{currentstroke}{rgb}{0.272594,0.025563,0.353093}%
\pgfsetstrokecolor{currentstroke}%
\pgfsetdash{}{0pt}%
\pgfpathmoveto{\pgfqpoint{8.418762in}{5.671547in}}%
\pgfpathlineto{\pgfqpoint{8.368871in}{5.667434in}}%
\pgfusepath{stroke}%
\end{pgfscope}%
\begin{pgfscope}%
\pgfpathrectangle{\pgfqpoint{6.720588in}{4.155455in}}{\pgfqpoint{2.279412in}{2.004545in}}%
\pgfusepath{clip}%
\pgfsetbuttcap%
\pgfsetroundjoin%
\pgfsetlinewidth{0.334284pt}%
\definecolor{currentstroke}{rgb}{0.272594,0.025563,0.353093}%
\pgfsetstrokecolor{currentstroke}%
\pgfsetdash{}{0pt}%
\pgfpathmoveto{\pgfqpoint{8.368871in}{5.667434in}}%
\pgfpathlineto{\pgfqpoint{8.319051in}{5.662934in}}%
\pgfusepath{stroke}%
\end{pgfscope}%
\begin{pgfscope}%
\pgfpathrectangle{\pgfqpoint{6.720588in}{4.155455in}}{\pgfqpoint{2.279412in}{2.004545in}}%
\pgfusepath{clip}%
\pgfsetbuttcap%
\pgfsetroundjoin%
\pgfsetlinewidth{0.332550pt}%
\definecolor{currentstroke}{rgb}{0.272594,0.025563,0.353093}%
\pgfsetstrokecolor{currentstroke}%
\pgfsetdash{}{0pt}%
\pgfpathmoveto{\pgfqpoint{8.319051in}{5.662934in}}%
\pgfpathlineto{\pgfqpoint{8.269289in}{5.657763in}}%
\pgfusepath{stroke}%
\end{pgfscope}%
\begin{pgfscope}%
\pgfpathrectangle{\pgfqpoint{6.720588in}{4.155455in}}{\pgfqpoint{2.279412in}{2.004545in}}%
\pgfusepath{clip}%
\pgfsetbuttcap%
\pgfsetroundjoin%
\pgfsetlinewidth{0.348352pt}%
\definecolor{currentstroke}{rgb}{0.274952,0.037752,0.364543}%
\pgfsetstrokecolor{currentstroke}%
\pgfsetdash{}{0pt}%
\pgfpathmoveto{\pgfqpoint{8.269289in}{5.657763in}}%
\pgfpathlineto{\pgfqpoint{8.219337in}{5.653902in}}%
\pgfusepath{stroke}%
\end{pgfscope}%
\begin{pgfscope}%
\pgfpathrectangle{\pgfqpoint{6.720588in}{4.155455in}}{\pgfqpoint{2.279412in}{2.004545in}}%
\pgfusepath{clip}%
\pgfsetbuttcap%
\pgfsetroundjoin%
\pgfsetlinewidth{0.394580pt}%
\definecolor{currentstroke}{rgb}{0.280894,0.078907,0.402329}%
\pgfsetstrokecolor{currentstroke}%
\pgfsetdash{}{0pt}%
\pgfpathmoveto{\pgfqpoint{8.010169in}{4.692222in}}%
\pgfpathlineto{\pgfqpoint{7.962878in}{4.706659in}}%
\pgfusepath{stroke}%
\end{pgfscope}%
\begin{pgfscope}%
\pgfpathrectangle{\pgfqpoint{6.720588in}{4.155455in}}{\pgfqpoint{2.279412in}{2.004545in}}%
\pgfusepath{clip}%
\pgfsetbuttcap%
\pgfsetroundjoin%
\pgfsetlinewidth{0.398277pt}%
\definecolor{currentstroke}{rgb}{0.281446,0.084320,0.407414}%
\pgfsetstrokecolor{currentstroke}%
\pgfsetdash{}{0pt}%
\pgfpathmoveto{\pgfqpoint{7.962878in}{4.706659in}}%
\pgfpathlineto{\pgfqpoint{7.917023in}{4.724172in}}%
\pgfusepath{stroke}%
\end{pgfscope}%
\begin{pgfscope}%
\pgfpathrectangle{\pgfqpoint{6.720588in}{4.155455in}}{\pgfqpoint{2.279412in}{2.004545in}}%
\pgfusepath{clip}%
\pgfsetbuttcap%
\pgfsetroundjoin%
\pgfsetlinewidth{0.390755pt}%
\definecolor{currentstroke}{rgb}{0.280894,0.078907,0.402329}%
\pgfsetstrokecolor{currentstroke}%
\pgfsetdash{}{0pt}%
\pgfpathmoveto{\pgfqpoint{7.917023in}{4.724172in}}%
\pgfpathlineto{\pgfqpoint{7.872799in}{4.744530in}}%
\pgfusepath{stroke}%
\end{pgfscope}%
\begin{pgfscope}%
\pgfpathrectangle{\pgfqpoint{6.720588in}{4.155455in}}{\pgfqpoint{2.279412in}{2.004545in}}%
\pgfusepath{clip}%
\pgfsetbuttcap%
\pgfsetroundjoin%
\pgfsetlinewidth{0.412902pt}%
\definecolor{currentstroke}{rgb}{0.282327,0.094955,0.417331}%
\pgfsetstrokecolor{currentstroke}%
\pgfsetdash{}{0pt}%
\pgfpathmoveto{\pgfqpoint{7.872799in}{4.744530in}}%
\pgfpathlineto{\pgfqpoint{7.831576in}{4.768900in}}%
\pgfusepath{stroke}%
\end{pgfscope}%
\begin{pgfscope}%
\pgfpathrectangle{\pgfqpoint{6.720588in}{4.155455in}}{\pgfqpoint{2.279412in}{2.004545in}}%
\pgfusepath{clip}%
\pgfsetbuttcap%
\pgfsetroundjoin%
\pgfsetlinewidth{0.408013pt}%
\definecolor{currentstroke}{rgb}{0.281924,0.089666,0.412415}%
\pgfsetstrokecolor{currentstroke}%
\pgfsetdash{}{0pt}%
\pgfpathmoveto{\pgfqpoint{7.831576in}{4.768900in}}%
\pgfpathlineto{\pgfqpoint{7.831576in}{4.768900in}}%
\pgfusepath{stroke}%
\end{pgfscope}%
\begin{pgfscope}%
\pgfpathrectangle{\pgfqpoint{6.720588in}{4.155455in}}{\pgfqpoint{2.279412in}{2.004545in}}%
\pgfusepath{clip}%
\pgfsetbuttcap%
\pgfsetroundjoin%
\pgfsetlinewidth{0.408013pt}%
\definecolor{currentstroke}{rgb}{0.281924,0.089666,0.412415}%
\pgfsetstrokecolor{currentstroke}%
\pgfsetdash{}{0pt}%
\pgfpathmoveto{\pgfqpoint{7.831576in}{4.768900in}}%
\pgfpathlineto{\pgfqpoint{7.808428in}{4.789937in}}%
\pgfusepath{stroke}%
\end{pgfscope}%
\begin{pgfscope}%
\pgfpathrectangle{\pgfqpoint{6.720588in}{4.155455in}}{\pgfqpoint{2.279412in}{2.004545in}}%
\pgfusepath{clip}%
\pgfsetbuttcap%
\pgfsetroundjoin%
\pgfsetlinewidth{0.513738pt}%
\definecolor{currentstroke}{rgb}{0.280255,0.165693,0.476498}%
\pgfsetstrokecolor{currentstroke}%
\pgfsetdash{}{0pt}%
\pgfpathmoveto{\pgfqpoint{7.808428in}{4.789937in}}%
\pgfpathlineto{\pgfqpoint{7.787348in}{4.812027in}}%
\pgfusepath{stroke}%
\end{pgfscope}%
\begin{pgfscope}%
\pgfpathrectangle{\pgfqpoint{6.720588in}{4.155455in}}{\pgfqpoint{2.279412in}{2.004545in}}%
\pgfusepath{clip}%
\pgfsetbuttcap%
\pgfsetroundjoin%
\pgfsetlinewidth{0.618147pt}%
\definecolor{currentstroke}{rgb}{0.262138,0.242286,0.520837}%
\pgfsetstrokecolor{currentstroke}%
\pgfsetdash{}{0pt}%
\pgfpathmoveto{\pgfqpoint{7.757710in}{5.428368in}}%
\pgfpathlineto{\pgfqpoint{7.723022in}{5.397320in}}%
\pgfusepath{stroke}%
\end{pgfscope}%
\begin{pgfscope}%
\pgfpathrectangle{\pgfqpoint{6.720588in}{4.155455in}}{\pgfqpoint{2.279412in}{2.004545in}}%
\pgfusepath{clip}%
\pgfsetbuttcap%
\pgfsetroundjoin%
\pgfsetlinewidth{1.227616pt}%
\definecolor{currentstroke}{rgb}{0.121148,0.592739,0.544641}%
\pgfsetstrokecolor{currentstroke}%
\pgfsetdash{}{0pt}%
\pgfpathmoveto{\pgfqpoint{7.723022in}{5.397320in}}%
\pgfpathlineto{\pgfqpoint{7.684888in}{5.368714in}}%
\pgfusepath{stroke}%
\end{pgfscope}%
\begin{pgfscope}%
\pgfpathrectangle{\pgfqpoint{6.720588in}{4.155455in}}{\pgfqpoint{2.279412in}{2.004545in}}%
\pgfusepath{clip}%
\pgfsetbuttcap%
\pgfsetroundjoin%
\pgfsetlinewidth{1.326505pt}%
\definecolor{currentstroke}{rgb}{0.126326,0.644107,0.525311}%
\pgfsetstrokecolor{currentstroke}%
\pgfsetdash{}{0pt}%
\pgfpathmoveto{\pgfqpoint{7.684888in}{5.368714in}}%
\pgfpathlineto{\pgfqpoint{7.647286in}{5.339613in}}%
\pgfusepath{stroke}%
\end{pgfscope}%
\begin{pgfscope}%
\pgfpathrectangle{\pgfqpoint{6.720588in}{4.155455in}}{\pgfqpoint{2.279412in}{2.004545in}}%
\pgfusepath{clip}%
\pgfsetbuttcap%
\pgfsetroundjoin%
\pgfsetlinewidth{1.520294pt}%
\definecolor{currentstroke}{rgb}{0.252899,0.742211,0.448284}%
\pgfsetstrokecolor{currentstroke}%
\pgfsetdash{}{0pt}%
\pgfpathmoveto{\pgfqpoint{7.647286in}{5.339613in}}%
\pgfpathlineto{\pgfqpoint{7.609442in}{5.310796in}}%
\pgfusepath{stroke}%
\end{pgfscope}%
\begin{pgfscope}%
\pgfpathrectangle{\pgfqpoint{6.720588in}{4.155455in}}{\pgfqpoint{2.279412in}{2.004545in}}%
\pgfusepath{clip}%
\pgfsetbuttcap%
\pgfsetroundjoin%
\pgfsetlinewidth{1.998693pt}%
\definecolor{currentstroke}{rgb}{0.916242,0.896091,0.100717}%
\pgfsetstrokecolor{currentstroke}%
\pgfsetdash{}{0pt}%
\pgfpathmoveto{\pgfqpoint{7.609442in}{5.310796in}}%
\pgfpathlineto{\pgfqpoint{7.571175in}{5.282398in}}%
\pgfusepath{stroke}%
\end{pgfscope}%
\begin{pgfscope}%
\pgfpathrectangle{\pgfqpoint{6.720588in}{4.155455in}}{\pgfqpoint{2.279412in}{2.004545in}}%
\pgfusepath{clip}%
\pgfsetbuttcap%
\pgfsetroundjoin%
\pgfsetlinewidth{0.900607pt}%
\definecolor{currentstroke}{rgb}{0.187231,0.414746,0.556547}%
\pgfsetstrokecolor{currentstroke}%
\pgfsetdash{}{0pt}%
\pgfpathmoveto{\pgfqpoint{8.014170in}{5.338154in}}%
\pgfpathlineto{\pgfqpoint{7.964734in}{5.330802in}}%
\pgfusepath{stroke}%
\end{pgfscope}%
\begin{pgfscope}%
\pgfpathrectangle{\pgfqpoint{6.720588in}{4.155455in}}{\pgfqpoint{2.279412in}{2.004545in}}%
\pgfusepath{clip}%
\pgfsetbuttcap%
\pgfsetroundjoin%
\pgfsetlinewidth{1.005876pt}%
\definecolor{currentstroke}{rgb}{0.163625,0.471133,0.558148}%
\pgfsetstrokecolor{currentstroke}%
\pgfsetdash{}{0pt}%
\pgfpathmoveto{\pgfqpoint{7.964734in}{5.330802in}}%
\pgfpathlineto{\pgfqpoint{7.915681in}{5.321677in}}%
\pgfusepath{stroke}%
\end{pgfscope}%
\begin{pgfscope}%
\pgfpathrectangle{\pgfqpoint{6.720588in}{4.155455in}}{\pgfqpoint{2.279412in}{2.004545in}}%
\pgfusepath{clip}%
\pgfsetbuttcap%
\pgfsetroundjoin%
\pgfsetlinewidth{1.298551pt}%
\definecolor{currentstroke}{rgb}{0.121380,0.629492,0.531973}%
\pgfsetstrokecolor{currentstroke}%
\pgfsetdash{}{0pt}%
\pgfpathmoveto{\pgfqpoint{7.915681in}{5.321677in}}%
\pgfpathlineto{\pgfqpoint{7.867032in}{5.311007in}}%
\pgfusepath{stroke}%
\end{pgfscope}%
\begin{pgfscope}%
\pgfpathrectangle{\pgfqpoint{6.720588in}{4.155455in}}{\pgfqpoint{2.279412in}{2.004545in}}%
\pgfusepath{clip}%
\pgfsetbuttcap%
\pgfsetroundjoin%
\pgfsetlinewidth{1.677197pt}%
\definecolor{currentstroke}{rgb}{0.440137,0.811138,0.340967}%
\pgfsetstrokecolor{currentstroke}%
\pgfsetdash{}{0pt}%
\pgfpathmoveto{\pgfqpoint{7.867032in}{5.311007in}}%
\pgfpathlineto{\pgfqpoint{7.818767in}{5.299059in}}%
\pgfusepath{stroke}%
\end{pgfscope}%
\begin{pgfscope}%
\pgfpathrectangle{\pgfqpoint{6.720588in}{4.155455in}}{\pgfqpoint{2.279412in}{2.004545in}}%
\pgfusepath{clip}%
\pgfsetbuttcap%
\pgfsetroundjoin%
\pgfsetlinewidth{1.799357pt}%
\definecolor{currentstroke}{rgb}{0.616293,0.852709,0.230052}%
\pgfsetstrokecolor{currentstroke}%
\pgfsetdash{}{0pt}%
\pgfpathmoveto{\pgfqpoint{7.818767in}{5.299059in}}%
\pgfpathlineto{\pgfqpoint{7.770856in}{5.286053in}}%
\pgfusepath{stroke}%
\end{pgfscope}%
\begin{pgfscope}%
\pgfpathrectangle{\pgfqpoint{6.720588in}{4.155455in}}{\pgfqpoint{2.279412in}{2.004545in}}%
\pgfusepath{clip}%
\pgfsetbuttcap%
\pgfsetroundjoin%
\pgfsetlinewidth{1.644951pt}%
\definecolor{currentstroke}{rgb}{0.395174,0.797475,0.367757}%
\pgfsetstrokecolor{currentstroke}%
\pgfsetdash{}{0pt}%
\pgfpathmoveto{\pgfqpoint{7.954177in}{5.283216in}}%
\pgfpathlineto{\pgfqpoint{7.904763in}{5.275697in}}%
\pgfusepath{stroke}%
\end{pgfscope}%
\begin{pgfscope}%
\pgfpathrectangle{\pgfqpoint{6.720588in}{4.155455in}}{\pgfqpoint{2.279412in}{2.004545in}}%
\pgfusepath{clip}%
\pgfsetbuttcap%
\pgfsetroundjoin%
\pgfsetlinewidth{1.866655pt}%
\definecolor{currentstroke}{rgb}{0.720391,0.870350,0.162603}%
\pgfsetstrokecolor{currentstroke}%
\pgfsetdash{}{0pt}%
\pgfpathmoveto{\pgfqpoint{7.904763in}{5.275697in}}%
\pgfpathlineto{\pgfqpoint{7.855531in}{5.267323in}}%
\pgfusepath{stroke}%
\end{pgfscope}%
\begin{pgfscope}%
\pgfpathrectangle{\pgfqpoint{6.720588in}{4.155455in}}{\pgfqpoint{2.279412in}{2.004545in}}%
\pgfusepath{clip}%
\pgfsetbuttcap%
\pgfsetroundjoin%
\pgfsetlinewidth{1.830566pt}%
\definecolor{currentstroke}{rgb}{0.657642,0.860219,0.203082}%
\pgfsetstrokecolor{currentstroke}%
\pgfsetdash{}{0pt}%
\pgfpathmoveto{\pgfqpoint{7.855531in}{5.267323in}}%
\pgfpathlineto{\pgfqpoint{7.806511in}{5.258045in}}%
\pgfusepath{stroke}%
\end{pgfscope}%
\begin{pgfscope}%
\pgfpathrectangle{\pgfqpoint{6.720588in}{4.155455in}}{\pgfqpoint{2.279412in}{2.004545in}}%
\pgfusepath{clip}%
\pgfsetbuttcap%
\pgfsetroundjoin%
\pgfsetlinewidth{1.993114pt}%
\definecolor{currentstroke}{rgb}{0.906311,0.894855,0.098125}%
\pgfsetstrokecolor{currentstroke}%
\pgfsetdash{}{0pt}%
\pgfpathmoveto{\pgfqpoint{7.806511in}{5.258045in}}%
\pgfpathlineto{\pgfqpoint{7.757710in}{5.247941in}}%
\pgfusepath{stroke}%
\end{pgfscope}%
\begin{pgfscope}%
\pgfpathrectangle{\pgfqpoint{6.720588in}{4.155455in}}{\pgfqpoint{2.279412in}{2.004545in}}%
\pgfusepath{clip}%
\pgfsetbuttcap%
\pgfsetroundjoin%
\pgfsetlinewidth{1.880991pt}%
\definecolor{currentstroke}{rgb}{0.741388,0.873449,0.149561}%
\pgfsetstrokecolor{currentstroke}%
\pgfsetdash{}{0pt}%
\pgfpathmoveto{\pgfqpoint{7.757710in}{5.247941in}}%
\pgfpathlineto{\pgfqpoint{7.709063in}{5.237275in}}%
\pgfusepath{stroke}%
\end{pgfscope}%
\begin{pgfscope}%
\pgfpathrectangle{\pgfqpoint{6.720588in}{4.155455in}}{\pgfqpoint{2.279412in}{2.004545in}}%
\pgfusepath{clip}%
\pgfsetbuttcap%
\pgfsetroundjoin%
\pgfsetlinewidth{2.080433pt}%
\definecolor{currentstroke}{rgb}{0.993248,0.906157,0.143936}%
\pgfsetstrokecolor{currentstroke}%
\pgfsetdash{}{0pt}%
\pgfpathmoveto{\pgfqpoint{7.709063in}{5.237275in}}%
\pgfpathlineto{\pgfqpoint{7.660537in}{5.226195in}}%
\pgfusepath{stroke}%
\end{pgfscope}%
\begin{pgfscope}%
\pgfpathrectangle{\pgfqpoint{6.720588in}{4.155455in}}{\pgfqpoint{2.279412in}{2.004545in}}%
\pgfusepath{clip}%
\pgfsetroundcap%
\pgfsetroundjoin%
\pgfsetlinewidth{1.130353pt}%
\definecolor{currentstroke}{rgb}{0.137770,0.537492,0.554906}%
\pgfsetstrokecolor{currentstroke}%
\pgfsetdash{}{0pt}%
\pgfpathmoveto{\pgfqpoint{8.033161in}{5.008423in}}%
\pgfpathquadraticcurveto{\pgfqpoint{8.020734in}{5.009865in}}{\pgfqpoint{8.025676in}{5.009292in}}%
\pgfusepath{stroke}%
\end{pgfscope}%
\begin{pgfscope}%
\pgfpathrectangle{\pgfqpoint{6.720588in}{4.155455in}}{\pgfqpoint{2.279412in}{2.004545in}}%
\pgfusepath{clip}%
\pgfsetroundcap%
\pgfsetroundjoin%
\definecolor{currentfill}{rgb}{0.137770,0.537492,0.554906}%
\pgfsetfillcolor{currentfill}%
\pgfsetlinewidth{1.130353pt}%
\definecolor{currentstroke}{rgb}{0.137770,0.537492,0.554906}%
\pgfsetstrokecolor{currentstroke}%
\pgfsetdash{}{0pt}%
\pgfpathmoveto{\pgfqpoint{8.077660in}{4.975296in}}%
\pgfpathlineto{\pgfqpoint{8.025676in}{5.009292in}}%
\pgfpathlineto{\pgfqpoint{8.084063in}{5.030482in}}%
\pgfpathlineto{\pgfqpoint{8.077660in}{4.975296in}}%
\pgfpathlineto{\pgfqpoint{8.077660in}{4.975296in}}%
\pgfpathclose%
\pgfusepath{stroke,fill}%
\end{pgfscope}%
\begin{pgfscope}%
\pgfpathrectangle{\pgfqpoint{6.720588in}{4.155455in}}{\pgfqpoint{2.279412in}{2.004545in}}%
\pgfusepath{clip}%
\pgfsetroundcap%
\pgfsetroundjoin%
\pgfsetlinewidth{1.304331pt}%
\definecolor{currentstroke}{rgb}{0.122312,0.633153,0.530398}%
\pgfsetstrokecolor{currentstroke}%
\pgfsetdash{}{0pt}%
\pgfpathmoveto{\pgfqpoint{8.080619in}{5.237486in}}%
\pgfpathquadraticcurveto{\pgfqpoint{8.068109in}{5.236760in}}{\pgfqpoint{8.075743in}{5.237203in}}%
\pgfusepath{stroke}%
\end{pgfscope}%
\begin{pgfscope}%
\pgfpathrectangle{\pgfqpoint{6.720588in}{4.155455in}}{\pgfqpoint{2.279412in}{2.004545in}}%
\pgfusepath{clip}%
\pgfsetroundcap%
\pgfsetroundjoin%
\definecolor{currentfill}{rgb}{0.122312,0.633153,0.530398}%
\pgfsetfillcolor{currentfill}%
\pgfsetlinewidth{1.304331pt}%
\definecolor{currentstroke}{rgb}{0.122312,0.633153,0.530398}%
\pgfsetstrokecolor{currentstroke}%
\pgfsetdash{}{0pt}%
\pgfpathmoveto{\pgfqpoint{8.132815in}{5.212691in}}%
\pgfpathlineto{\pgfqpoint{8.075743in}{5.237203in}}%
\pgfpathlineto{\pgfqpoint{8.129596in}{5.268153in}}%
\pgfpathlineto{\pgfqpoint{8.132815in}{5.212691in}}%
\pgfpathlineto{\pgfqpoint{8.132815in}{5.212691in}}%
\pgfpathclose%
\pgfusepath{stroke,fill}%
\end{pgfscope}%
\begin{pgfscope}%
\pgfpathrectangle{\pgfqpoint{6.720588in}{4.155455in}}{\pgfqpoint{2.279412in}{2.004545in}}%
\pgfusepath{clip}%
\pgfsetroundcap%
\pgfsetroundjoin%
\pgfsetlinewidth{0.423890pt}%
\definecolor{currentstroke}{rgb}{0.282656,0.100196,0.422160}%
\pgfsetstrokecolor{currentstroke}%
\pgfsetdash{}{0pt}%
\pgfpathmoveto{\pgfqpoint{8.381407in}{5.289785in}}%
\pgfpathquadraticcurveto{\pgfqpoint{8.368877in}{5.289402in}}{\pgfqpoint{8.362902in}{5.289219in}}%
\pgfusepath{stroke}%
\end{pgfscope}%
\begin{pgfscope}%
\pgfpathrectangle{\pgfqpoint{6.720588in}{4.155455in}}{\pgfqpoint{2.279412in}{2.004545in}}%
\pgfusepath{clip}%
\pgfsetroundcap%
\pgfsetroundjoin%
\definecolor{currentfill}{rgb}{0.282656,0.100196,0.422160}%
\pgfsetfillcolor{currentfill}%
\pgfsetlinewidth{0.423890pt}%
\definecolor{currentstroke}{rgb}{0.282656,0.100196,0.422160}%
\pgfsetstrokecolor{currentstroke}%
\pgfsetdash{}{0pt}%
\pgfpathmoveto{\pgfqpoint{8.419281in}{5.263153in}}%
\pgfpathlineto{\pgfqpoint{8.362902in}{5.289219in}}%
\pgfpathlineto{\pgfqpoint{8.417582in}{5.318683in}}%
\pgfpathlineto{\pgfqpoint{8.419281in}{5.263153in}}%
\pgfpathlineto{\pgfqpoint{8.419281in}{5.263153in}}%
\pgfpathclose%
\pgfusepath{stroke,fill}%
\end{pgfscope}%
\begin{pgfscope}%
\pgfpathrectangle{\pgfqpoint{6.720588in}{4.155455in}}{\pgfqpoint{2.279412in}{2.004545in}}%
\pgfusepath{clip}%
\pgfsetroundcap%
\pgfsetroundjoin%
\pgfsetlinewidth{0.505737pt}%
\definecolor{currentstroke}{rgb}{0.280868,0.160771,0.472899}%
\pgfsetstrokecolor{currentstroke}%
\pgfsetdash{}{0pt}%
\pgfpathmoveto{\pgfqpoint{8.281679in}{4.945505in}}%
\pgfpathquadraticcurveto{\pgfqpoint{8.269163in}{4.946148in}}{\pgfqpoint{8.264460in}{4.946389in}}%
\pgfusepath{stroke}%
\end{pgfscope}%
\begin{pgfscope}%
\pgfpathrectangle{\pgfqpoint{6.720588in}{4.155455in}}{\pgfqpoint{2.279412in}{2.004545in}}%
\pgfusepath{clip}%
\pgfsetroundcap%
\pgfsetroundjoin%
\definecolor{currentfill}{rgb}{0.280868,0.160771,0.472899}%
\pgfsetfillcolor{currentfill}%
\pgfsetlinewidth{0.505737pt}%
\definecolor{currentstroke}{rgb}{0.280868,0.160771,0.472899}%
\pgfsetstrokecolor{currentstroke}%
\pgfsetdash{}{0pt}%
\pgfpathmoveto{\pgfqpoint{8.318518in}{4.915799in}}%
\pgfpathlineto{\pgfqpoint{8.264460in}{4.946389in}}%
\pgfpathlineto{\pgfqpoint{8.321367in}{4.971282in}}%
\pgfpathlineto{\pgfqpoint{8.318518in}{4.915799in}}%
\pgfpathlineto{\pgfqpoint{8.318518in}{4.915799in}}%
\pgfpathclose%
\pgfusepath{stroke,fill}%
\end{pgfscope}%
\begin{pgfscope}%
\pgfpathrectangle{\pgfqpoint{6.720588in}{4.155455in}}{\pgfqpoint{2.279412in}{2.004545in}}%
\pgfusepath{clip}%
\pgfsetroundcap%
\pgfsetroundjoin%
\pgfsetlinewidth{0.402634pt}%
\definecolor{currentstroke}{rgb}{0.281446,0.084320,0.407414}%
\pgfsetstrokecolor{currentstroke}%
\pgfsetdash{}{0pt}%
\pgfpathmoveto{\pgfqpoint{8.430331in}{5.025995in}}%
\pgfpathquadraticcurveto{\pgfqpoint{8.417796in}{5.026219in}}{\pgfqpoint{8.411489in}{5.026331in}}%
\pgfusepath{stroke}%
\end{pgfscope}%
\begin{pgfscope}%
\pgfpathrectangle{\pgfqpoint{6.720588in}{4.155455in}}{\pgfqpoint{2.279412in}{2.004545in}}%
\pgfusepath{clip}%
\pgfsetroundcap%
\pgfsetroundjoin%
\definecolor{currentfill}{rgb}{0.281446,0.084320,0.407414}%
\pgfsetfillcolor{currentfill}%
\pgfsetlinewidth{0.402634pt}%
\definecolor{currentstroke}{rgb}{0.281446,0.084320,0.407414}%
\pgfsetstrokecolor{currentstroke}%
\pgfsetdash{}{0pt}%
\pgfpathmoveto{\pgfqpoint{8.466540in}{4.997566in}}%
\pgfpathlineto{\pgfqpoint{8.411489in}{5.026331in}}%
\pgfpathlineto{\pgfqpoint{8.467532in}{5.053113in}}%
\pgfpathlineto{\pgfqpoint{8.466540in}{4.997566in}}%
\pgfpathlineto{\pgfqpoint{8.466540in}{4.997566in}}%
\pgfpathclose%
\pgfusepath{stroke,fill}%
\end{pgfscope}%
\begin{pgfscope}%
\pgfpathrectangle{\pgfqpoint{6.720588in}{4.155455in}}{\pgfqpoint{2.279412in}{2.004545in}}%
\pgfusepath{clip}%
\pgfsetroundcap%
\pgfsetroundjoin%
\pgfsetlinewidth{0.896089pt}%
\definecolor{currentstroke}{rgb}{0.188923,0.410910,0.556326}%
\pgfsetstrokecolor{currentstroke}%
\pgfsetdash{}{0pt}%
\pgfpathmoveto{\pgfqpoint{8.180660in}{5.076971in}}%
\pgfpathquadraticcurveto{\pgfqpoint{8.168128in}{5.077304in}}{\pgfqpoint{8.169454in}{5.077269in}}%
\pgfusepath{stroke}%
\end{pgfscope}%
\begin{pgfscope}%
\pgfpathrectangle{\pgfqpoint{6.720588in}{4.155455in}}{\pgfqpoint{2.279412in}{2.004545in}}%
\pgfusepath{clip}%
\pgfsetroundcap%
\pgfsetroundjoin%
\definecolor{currentfill}{rgb}{0.188923,0.410910,0.556326}%
\pgfsetfillcolor{currentfill}%
\pgfsetlinewidth{0.896089pt}%
\definecolor{currentstroke}{rgb}{0.188923,0.410910,0.556326}%
\pgfsetstrokecolor{currentstroke}%
\pgfsetdash{}{0pt}%
\pgfpathmoveto{\pgfqpoint{8.224254in}{5.048028in}}%
\pgfpathlineto{\pgfqpoint{8.169454in}{5.077269in}}%
\pgfpathlineto{\pgfqpoint{8.225726in}{5.103564in}}%
\pgfpathlineto{\pgfqpoint{8.224254in}{5.048028in}}%
\pgfpathlineto{\pgfqpoint{8.224254in}{5.048028in}}%
\pgfpathclose%
\pgfusepath{stroke,fill}%
\end{pgfscope}%
\begin{pgfscope}%
\pgfpathrectangle{\pgfqpoint{6.720588in}{4.155455in}}{\pgfqpoint{2.279412in}{2.004545in}}%
\pgfusepath{clip}%
\pgfsetroundcap%
\pgfsetroundjoin%
\pgfsetlinewidth{0.963546pt}%
\definecolor{currentstroke}{rgb}{0.172719,0.448791,0.557885}%
\pgfsetstrokecolor{currentstroke}%
\pgfsetdash{}{0pt}%
\pgfpathmoveto{\pgfqpoint{8.179511in}{5.159322in}}%
\pgfpathquadraticcurveto{\pgfqpoint{8.166973in}{5.159303in}}{\pgfqpoint{8.169342in}{5.159307in}}%
\pgfusepath{stroke}%
\end{pgfscope}%
\begin{pgfscope}%
\pgfpathrectangle{\pgfqpoint{6.720588in}{4.155455in}}{\pgfqpoint{2.279412in}{2.004545in}}%
\pgfusepath{clip}%
\pgfsetroundcap%
\pgfsetroundjoin%
\definecolor{currentfill}{rgb}{0.172719,0.448791,0.557885}%
\pgfsetfillcolor{currentfill}%
\pgfsetlinewidth{0.963546pt}%
\definecolor{currentstroke}{rgb}{0.172719,0.448791,0.557885}%
\pgfsetstrokecolor{currentstroke}%
\pgfsetdash{}{0pt}%
\pgfpathmoveto{\pgfqpoint{8.224938in}{5.131611in}}%
\pgfpathlineto{\pgfqpoint{8.169342in}{5.159307in}}%
\pgfpathlineto{\pgfqpoint{8.224856in}{5.187167in}}%
\pgfpathlineto{\pgfqpoint{8.224938in}{5.131611in}}%
\pgfpathlineto{\pgfqpoint{8.224938in}{5.131611in}}%
\pgfpathclose%
\pgfusepath{stroke,fill}%
\end{pgfscope}%
\begin{pgfscope}%
\pgfpathrectangle{\pgfqpoint{6.720588in}{4.155455in}}{\pgfqpoint{2.279412in}{2.004545in}}%
\pgfusepath{clip}%
\pgfsetroundcap%
\pgfsetroundjoin%
\pgfsetlinewidth{0.509566pt}%
\definecolor{currentstroke}{rgb}{0.280255,0.165693,0.476498}%
\pgfsetstrokecolor{currentstroke}%
\pgfsetdash{}{0pt}%
\pgfpathmoveto{\pgfqpoint{8.330012in}{5.201088in}}%
\pgfpathquadraticcurveto{\pgfqpoint{8.317475in}{5.200974in}}{\pgfqpoint{8.312820in}{5.200932in}}%
\pgfusepath{stroke}%
\end{pgfscope}%
\begin{pgfscope}%
\pgfpathrectangle{\pgfqpoint{6.720588in}{4.155455in}}{\pgfqpoint{2.279412in}{2.004545in}}%
\pgfusepath{clip}%
\pgfsetroundcap%
\pgfsetroundjoin%
\definecolor{currentfill}{rgb}{0.280255,0.165693,0.476498}%
\pgfsetfillcolor{currentfill}%
\pgfsetlinewidth{0.509566pt}%
\definecolor{currentstroke}{rgb}{0.280255,0.165693,0.476498}%
\pgfsetstrokecolor{currentstroke}%
\pgfsetdash{}{0pt}%
\pgfpathmoveto{\pgfqpoint{8.368627in}{5.173661in}}%
\pgfpathlineto{\pgfqpoint{8.312820in}{5.200932in}}%
\pgfpathlineto{\pgfqpoint{8.368121in}{5.229215in}}%
\pgfpathlineto{\pgfqpoint{8.368627in}{5.173661in}}%
\pgfpathlineto{\pgfqpoint{8.368627in}{5.173661in}}%
\pgfpathclose%
\pgfusepath{stroke,fill}%
\end{pgfscope}%
\begin{pgfscope}%
\pgfpathrectangle{\pgfqpoint{6.720588in}{4.155455in}}{\pgfqpoint{2.279412in}{2.004545in}}%
\pgfusepath{clip}%
\pgfsetroundcap%
\pgfsetroundjoin%
\pgfsetlinewidth{0.422772pt}%
\definecolor{currentstroke}{rgb}{0.282656,0.100196,0.422160}%
\pgfsetstrokecolor{currentstroke}%
\pgfsetdash{}{0pt}%
\pgfpathmoveto{\pgfqpoint{8.380352in}{5.334174in}}%
\pgfpathquadraticcurveto{\pgfqpoint{8.367821in}{5.333828in}}{\pgfqpoint{8.361827in}{5.333662in}}%
\pgfusepath{stroke}%
\end{pgfscope}%
\begin{pgfscope}%
\pgfpathrectangle{\pgfqpoint{6.720588in}{4.155455in}}{\pgfqpoint{2.279412in}{2.004545in}}%
\pgfusepath{clip}%
\pgfsetroundcap%
\pgfsetroundjoin%
\definecolor{currentfill}{rgb}{0.282656,0.100196,0.422160}%
\pgfsetfillcolor{currentfill}%
\pgfsetlinewidth{0.422772pt}%
\definecolor{currentstroke}{rgb}{0.282656,0.100196,0.422160}%
\pgfsetstrokecolor{currentstroke}%
\pgfsetdash{}{0pt}%
\pgfpathmoveto{\pgfqpoint{8.418128in}{5.307429in}}%
\pgfpathlineto{\pgfqpoint{8.361827in}{5.333662in}}%
\pgfpathlineto{\pgfqpoint{8.416594in}{5.362963in}}%
\pgfpathlineto{\pgfqpoint{8.418128in}{5.307429in}}%
\pgfpathlineto{\pgfqpoint{8.418128in}{5.307429in}}%
\pgfpathclose%
\pgfusepath{stroke,fill}%
\end{pgfscope}%
\begin{pgfscope}%
\pgfpathrectangle{\pgfqpoint{6.720588in}{4.155455in}}{\pgfqpoint{2.279412in}{2.004545in}}%
\pgfusepath{clip}%
\pgfsetroundcap%
\pgfsetroundjoin%
\pgfsetlinewidth{0.548378pt}%
\definecolor{currentstroke}{rgb}{0.275191,0.194905,0.496005}%
\pgfsetstrokecolor{currentstroke}%
\pgfsetdash{}{0pt}%
\pgfpathmoveto{\pgfqpoint{7.990513in}{4.845819in}}%
\pgfpathquadraticcurveto{\pgfqpoint{7.978387in}{4.848606in}}{\pgfqpoint{7.974528in}{4.849492in}}%
\pgfusepath{stroke}%
\end{pgfscope}%
\begin{pgfscope}%
\pgfpathrectangle{\pgfqpoint{6.720588in}{4.155455in}}{\pgfqpoint{2.279412in}{2.004545in}}%
\pgfusepath{clip}%
\pgfsetroundcap%
\pgfsetroundjoin%
\definecolor{currentfill}{rgb}{0.275191,0.194905,0.496005}%
\pgfsetfillcolor{currentfill}%
\pgfsetlinewidth{0.548378pt}%
\definecolor{currentstroke}{rgb}{0.275191,0.194905,0.496005}%
\pgfsetstrokecolor{currentstroke}%
\pgfsetdash{}{0pt}%
\pgfpathmoveto{\pgfqpoint{8.022452in}{4.809979in}}%
\pgfpathlineto{\pgfqpoint{7.974528in}{4.849492in}}%
\pgfpathlineto{\pgfqpoint{8.034894in}{4.864123in}}%
\pgfpathlineto{\pgfqpoint{8.022452in}{4.809979in}}%
\pgfpathlineto{\pgfqpoint{8.022452in}{4.809979in}}%
\pgfpathclose%
\pgfusepath{stroke,fill}%
\end{pgfscope}%
\begin{pgfscope}%
\pgfpathrectangle{\pgfqpoint{6.720588in}{4.155455in}}{\pgfqpoint{2.279412in}{2.004545in}}%
\pgfusepath{clip}%
\pgfsetroundcap%
\pgfsetroundjoin%
\pgfsetlinewidth{0.593142pt}%
\definecolor{currentstroke}{rgb}{0.267968,0.223549,0.512008}%
\pgfsetstrokecolor{currentstroke}%
\pgfsetdash{}{0pt}%
\pgfpathmoveto{\pgfqpoint{8.279130in}{5.117648in}}%
\pgfpathquadraticcurveto{\pgfqpoint{8.266592in}{5.117705in}}{\pgfqpoint{8.263231in}{5.117720in}}%
\pgfusepath{stroke}%
\end{pgfscope}%
\begin{pgfscope}%
\pgfpathrectangle{\pgfqpoint{6.720588in}{4.155455in}}{\pgfqpoint{2.279412in}{2.004545in}}%
\pgfusepath{clip}%
\pgfsetroundcap%
\pgfsetroundjoin%
\definecolor{currentfill}{rgb}{0.267968,0.223549,0.512008}%
\pgfsetfillcolor{currentfill}%
\pgfsetlinewidth{0.593142pt}%
\definecolor{currentstroke}{rgb}{0.267968,0.223549,0.512008}%
\pgfsetstrokecolor{currentstroke}%
\pgfsetdash{}{0pt}%
\pgfpathmoveto{\pgfqpoint{8.318658in}{5.089688in}}%
\pgfpathlineto{\pgfqpoint{8.263231in}{5.117720in}}%
\pgfpathlineto{\pgfqpoint{8.318913in}{5.145243in}}%
\pgfpathlineto{\pgfqpoint{8.318658in}{5.089688in}}%
\pgfpathlineto{\pgfqpoint{8.318658in}{5.089688in}}%
\pgfpathclose%
\pgfusepath{stroke,fill}%
\end{pgfscope}%
\begin{pgfscope}%
\pgfpathrectangle{\pgfqpoint{6.720588in}{4.155455in}}{\pgfqpoint{2.279412in}{2.004545in}}%
\pgfusepath{clip}%
\pgfsetroundcap%
\pgfsetroundjoin%
\pgfsetlinewidth{0.804295pt}%
\definecolor{currentstroke}{rgb}{0.212395,0.359683,0.551710}%
\pgfsetstrokecolor{currentstroke}%
\pgfsetdash{}{0pt}%
\pgfpathmoveto{\pgfqpoint{7.931481in}{5.375649in}}%
\pgfpathquadraticcurveto{\pgfqpoint{7.919419in}{5.372656in}}{\pgfqpoint{7.919434in}{5.372660in}}%
\pgfusepath{stroke}%
\end{pgfscope}%
\begin{pgfscope}%
\pgfpathrectangle{\pgfqpoint{6.720588in}{4.155455in}}{\pgfqpoint{2.279412in}{2.004545in}}%
\pgfusepath{clip}%
\pgfsetroundcap%
\pgfsetroundjoin%
\definecolor{currentfill}{rgb}{0.212395,0.359683,0.551710}%
\pgfsetfillcolor{currentfill}%
\pgfsetlinewidth{0.804295pt}%
\definecolor{currentstroke}{rgb}{0.212395,0.359683,0.551710}%
\pgfsetstrokecolor{currentstroke}%
\pgfsetdash{}{0pt}%
\pgfpathmoveto{\pgfqpoint{7.980044in}{5.359081in}}%
\pgfpathlineto{\pgfqpoint{7.919434in}{5.372660in}}%
\pgfpathlineto{\pgfqpoint{7.966663in}{5.413001in}}%
\pgfpathlineto{\pgfqpoint{7.980044in}{5.359081in}}%
\pgfpathlineto{\pgfqpoint{7.980044in}{5.359081in}}%
\pgfpathclose%
\pgfusepath{stroke,fill}%
\end{pgfscope}%
\begin{pgfscope}%
\pgfpathrectangle{\pgfqpoint{6.720588in}{4.155455in}}{\pgfqpoint{2.279412in}{2.004545in}}%
\pgfusepath{clip}%
\pgfsetroundcap%
\pgfsetroundjoin%
\pgfsetlinewidth{0.420780pt}%
\definecolor{currentstroke}{rgb}{0.282656,0.100196,0.422160}%
\pgfsetstrokecolor{currentstroke}%
\pgfsetdash{}{0pt}%
\pgfpathmoveto{\pgfqpoint{8.178750in}{4.732204in}}%
\pgfpathquadraticcurveto{\pgfqpoint{8.166372in}{4.733939in}}{\pgfqpoint{8.160441in}{4.734771in}}%
\pgfusepath{stroke}%
\end{pgfscope}%
\begin{pgfscope}%
\pgfpathrectangle{\pgfqpoint{6.720588in}{4.155455in}}{\pgfqpoint{2.279412in}{2.004545in}}%
\pgfusepath{clip}%
\pgfsetroundcap%
\pgfsetroundjoin%
\definecolor{currentfill}{rgb}{0.282656,0.100196,0.422160}%
\pgfsetfillcolor{currentfill}%
\pgfsetlinewidth{0.420780pt}%
\definecolor{currentstroke}{rgb}{0.282656,0.100196,0.422160}%
\pgfsetstrokecolor{currentstroke}%
\pgfsetdash{}{0pt}%
\pgfpathmoveto{\pgfqpoint{8.211600in}{4.699547in}}%
\pgfpathlineto{\pgfqpoint{8.160441in}{4.734771in}}%
\pgfpathlineto{\pgfqpoint{8.219316in}{4.754565in}}%
\pgfpathlineto{\pgfqpoint{8.211600in}{4.699547in}}%
\pgfpathlineto{\pgfqpoint{8.211600in}{4.699547in}}%
\pgfpathclose%
\pgfusepath{stroke,fill}%
\end{pgfscope}%
\begin{pgfscope}%
\pgfpathrectangle{\pgfqpoint{6.720588in}{4.155455in}}{\pgfqpoint{2.279412in}{2.004545in}}%
\pgfusepath{clip}%
\pgfsetroundcap%
\pgfsetroundjoin%
\pgfsetlinewidth{0.568344pt}%
\definecolor{currentstroke}{rgb}{0.273006,0.204520,0.501721}%
\pgfsetstrokecolor{currentstroke}%
\pgfsetdash{}{0pt}%
\pgfpathmoveto{\pgfqpoint{8.127879in}{4.909968in}}%
\pgfpathquadraticcurveto{\pgfqpoint{8.115431in}{4.911278in}}{\pgfqpoint{8.111728in}{4.911667in}}%
\pgfusepath{stroke}%
\end{pgfscope}%
\begin{pgfscope}%
\pgfpathrectangle{\pgfqpoint{6.720588in}{4.155455in}}{\pgfqpoint{2.279412in}{2.004545in}}%
\pgfusepath{clip}%
\pgfsetroundcap%
\pgfsetroundjoin%
\definecolor{currentfill}{rgb}{0.273006,0.204520,0.501721}%
\pgfsetfillcolor{currentfill}%
\pgfsetlinewidth{0.568344pt}%
\definecolor{currentstroke}{rgb}{0.273006,0.204520,0.501721}%
\pgfsetstrokecolor{currentstroke}%
\pgfsetdash{}{0pt}%
\pgfpathmoveto{\pgfqpoint{8.164072in}{4.878229in}}%
\pgfpathlineto{\pgfqpoint{8.111728in}{4.911667in}}%
\pgfpathlineto{\pgfqpoint{8.169885in}{4.933480in}}%
\pgfpathlineto{\pgfqpoint{8.164072in}{4.878229in}}%
\pgfpathlineto{\pgfqpoint{8.164072in}{4.878229in}}%
\pgfpathclose%
\pgfusepath{stroke,fill}%
\end{pgfscope}%
\begin{pgfscope}%
\pgfpathrectangle{\pgfqpoint{6.720588in}{4.155455in}}{\pgfqpoint{2.279412in}{2.004545in}}%
\pgfusepath{clip}%
\pgfsetroundcap%
\pgfsetroundjoin%
\pgfsetlinewidth{0.410694pt}%
\definecolor{currentstroke}{rgb}{0.281924,0.089666,0.412415}%
\pgfsetstrokecolor{currentstroke}%
\pgfsetdash{}{0pt}%
\pgfpathmoveto{\pgfqpoint{8.377935in}{5.376878in}}%
\pgfpathquadraticcurveto{\pgfqpoint{8.365411in}{5.376356in}}{\pgfqpoint{8.359235in}{5.376098in}}%
\pgfusepath{stroke}%
\end{pgfscope}%
\begin{pgfscope}%
\pgfpathrectangle{\pgfqpoint{6.720588in}{4.155455in}}{\pgfqpoint{2.279412in}{2.004545in}}%
\pgfusepath{clip}%
\pgfsetroundcap%
\pgfsetroundjoin%
\definecolor{currentfill}{rgb}{0.281924,0.089666,0.412415}%
\pgfsetfillcolor{currentfill}%
\pgfsetlinewidth{0.410694pt}%
\definecolor{currentstroke}{rgb}{0.281924,0.089666,0.412415}%
\pgfsetstrokecolor{currentstroke}%
\pgfsetdash{}{0pt}%
\pgfpathmoveto{\pgfqpoint{8.415900in}{5.350659in}}%
\pgfpathlineto{\pgfqpoint{8.359235in}{5.376098in}}%
\pgfpathlineto{\pgfqpoint{8.413586in}{5.406166in}}%
\pgfpathlineto{\pgfqpoint{8.415900in}{5.350659in}}%
\pgfpathlineto{\pgfqpoint{8.415900in}{5.350659in}}%
\pgfpathclose%
\pgfusepath{stroke,fill}%
\end{pgfscope}%
\begin{pgfscope}%
\pgfpathrectangle{\pgfqpoint{6.720588in}{4.155455in}}{\pgfqpoint{2.279412in}{2.004545in}}%
\pgfusepath{clip}%
\pgfsetroundcap%
\pgfsetroundjoin%
\pgfsetlinewidth{0.363470pt}%
\definecolor{currentstroke}{rgb}{0.277941,0.056324,0.381191}%
\pgfsetstrokecolor{currentstroke}%
\pgfsetdash{}{0pt}%
\pgfpathmoveto{\pgfqpoint{8.428111in}{5.467783in}}%
\pgfpathquadraticcurveto{\pgfqpoint{8.415597in}{5.467139in}}{\pgfqpoint{8.408697in}{5.466784in}}%
\pgfusepath{stroke}%
\end{pgfscope}%
\begin{pgfscope}%
\pgfpathrectangle{\pgfqpoint{6.720588in}{4.155455in}}{\pgfqpoint{2.279412in}{2.004545in}}%
\pgfusepath{clip}%
\pgfsetroundcap%
\pgfsetroundjoin%
\definecolor{currentfill}{rgb}{0.277941,0.056324,0.381191}%
\pgfsetfillcolor{currentfill}%
\pgfsetlinewidth{0.363470pt}%
\definecolor{currentstroke}{rgb}{0.277941,0.056324,0.381191}%
\pgfsetstrokecolor{currentstroke}%
\pgfsetdash{}{0pt}%
\pgfpathmoveto{\pgfqpoint{8.465606in}{5.441896in}}%
\pgfpathlineto{\pgfqpoint{8.408697in}{5.466784in}}%
\pgfpathlineto{\pgfqpoint{8.462753in}{5.497379in}}%
\pgfpathlineto{\pgfqpoint{8.465606in}{5.441896in}}%
\pgfpathlineto{\pgfqpoint{8.465606in}{5.441896in}}%
\pgfpathclose%
\pgfusepath{stroke,fill}%
\end{pgfscope}%
\begin{pgfscope}%
\pgfpathrectangle{\pgfqpoint{6.720588in}{4.155455in}}{\pgfqpoint{2.279412in}{2.004545in}}%
\pgfusepath{clip}%
\pgfsetroundcap%
\pgfsetroundjoin%
\pgfsetlinewidth{0.435126pt}%
\definecolor{currentstroke}{rgb}{0.283091,0.110553,0.431554}%
\pgfsetstrokecolor{currentstroke}%
\pgfsetdash{}{0pt}%
\pgfpathmoveto{\pgfqpoint{8.228443in}{5.498660in}}%
\pgfpathquadraticcurveto{\pgfqpoint{8.215994in}{5.497351in}}{\pgfqpoint{8.210240in}{5.496746in}}%
\pgfusepath{stroke}%
\end{pgfscope}%
\begin{pgfscope}%
\pgfpathrectangle{\pgfqpoint{6.720588in}{4.155455in}}{\pgfqpoint{2.279412in}{2.004545in}}%
\pgfusepath{clip}%
\pgfsetroundcap%
\pgfsetroundjoin%
\definecolor{currentfill}{rgb}{0.283091,0.110553,0.431554}%
\pgfsetfillcolor{currentfill}%
\pgfsetlinewidth{0.435126pt}%
\definecolor{currentstroke}{rgb}{0.283091,0.110553,0.431554}%
\pgfsetstrokecolor{currentstroke}%
\pgfsetdash{}{0pt}%
\pgfpathmoveto{\pgfqpoint{8.268395in}{5.474929in}}%
\pgfpathlineto{\pgfqpoint{8.210240in}{5.496746in}}%
\pgfpathlineto{\pgfqpoint{8.262587in}{5.530180in}}%
\pgfpathlineto{\pgfqpoint{8.268395in}{5.474929in}}%
\pgfpathlineto{\pgfqpoint{8.268395in}{5.474929in}}%
\pgfpathclose%
\pgfusepath{stroke,fill}%
\end{pgfscope}%
\begin{pgfscope}%
\pgfpathrectangle{\pgfqpoint{6.720588in}{4.155455in}}{\pgfqpoint{2.279412in}{2.004545in}}%
\pgfusepath{clip}%
\pgfsetroundcap%
\pgfsetroundjoin%
\pgfsetlinewidth{0.386692pt}%
\definecolor{currentstroke}{rgb}{0.280267,0.073417,0.397163}%
\pgfsetstrokecolor{currentstroke}%
\pgfsetdash{}{0pt}%
\pgfpathmoveto{\pgfqpoint{8.278158in}{5.547741in}}%
\pgfpathquadraticcurveto{\pgfqpoint{8.265683in}{5.546645in}}{\pgfqpoint{8.259167in}{5.546073in}}%
\pgfusepath{stroke}%
\end{pgfscope}%
\begin{pgfscope}%
\pgfpathrectangle{\pgfqpoint{6.720588in}{4.155455in}}{\pgfqpoint{2.279412in}{2.004545in}}%
\pgfusepath{clip}%
\pgfsetroundcap%
\pgfsetroundjoin%
\definecolor{currentfill}{rgb}{0.280267,0.073417,0.397163}%
\pgfsetfillcolor{currentfill}%
\pgfsetlinewidth{0.386692pt}%
\definecolor{currentstroke}{rgb}{0.280267,0.073417,0.397163}%
\pgfsetstrokecolor{currentstroke}%
\pgfsetdash{}{0pt}%
\pgfpathmoveto{\pgfqpoint{8.316940in}{5.523262in}}%
\pgfpathlineto{\pgfqpoint{8.259167in}{5.546073in}}%
\pgfpathlineto{\pgfqpoint{8.312080in}{5.578604in}}%
\pgfpathlineto{\pgfqpoint{8.316940in}{5.523262in}}%
\pgfpathlineto{\pgfqpoint{8.316940in}{5.523262in}}%
\pgfpathclose%
\pgfusepath{stroke,fill}%
\end{pgfscope}%
\begin{pgfscope}%
\pgfpathrectangle{\pgfqpoint{6.720588in}{4.155455in}}{\pgfqpoint{2.279412in}{2.004545in}}%
\pgfusepath{clip}%
\pgfsetroundcap%
\pgfsetroundjoin%
\pgfsetlinewidth{0.331855pt}%
\definecolor{currentstroke}{rgb}{0.272594,0.025563,0.353093}%
\pgfsetstrokecolor{currentstroke}%
\pgfsetdash{}{0pt}%
\pgfpathmoveto{\pgfqpoint{8.215742in}{5.802216in}}%
\pgfpathquadraticcurveto{\pgfqpoint{8.203335in}{5.800659in}}{\pgfqpoint{8.196021in}{5.799742in}}%
\pgfusepath{stroke}%
\end{pgfscope}%
\begin{pgfscope}%
\pgfpathrectangle{\pgfqpoint{6.720588in}{4.155455in}}{\pgfqpoint{2.279412in}{2.004545in}}%
\pgfusepath{clip}%
\pgfsetroundcap%
\pgfsetroundjoin%
\definecolor{currentfill}{rgb}{0.272594,0.025563,0.353093}%
\pgfsetfillcolor{currentfill}%
\pgfsetlinewidth{0.331855pt}%
\definecolor{currentstroke}{rgb}{0.272594,0.025563,0.353093}%
\pgfsetstrokecolor{currentstroke}%
\pgfsetdash{}{0pt}%
\pgfpathmoveto{\pgfqpoint{8.254603in}{5.779096in}}%
\pgfpathlineto{\pgfqpoint{8.196021in}{5.799742in}}%
\pgfpathlineto{\pgfqpoint{8.247687in}{5.834219in}}%
\pgfpathlineto{\pgfqpoint{8.254603in}{5.779096in}}%
\pgfpathlineto{\pgfqpoint{8.254603in}{5.779096in}}%
\pgfpathclose%
\pgfusepath{stroke,fill}%
\end{pgfscope}%
\begin{pgfscope}%
\pgfpathrectangle{\pgfqpoint{6.720588in}{4.155455in}}{\pgfqpoint{2.279412in}{2.004545in}}%
\pgfusepath{clip}%
\pgfsetroundcap%
\pgfsetroundjoin%
\pgfsetlinewidth{0.327050pt}%
\definecolor{currentstroke}{rgb}{0.271305,0.019942,0.347269}%
\pgfsetstrokecolor{currentstroke}%
\pgfsetdash{}{0pt}%
\pgfpathmoveto{\pgfqpoint{8.062403in}{4.559914in}}%
\pgfpathquadraticcurveto{\pgfqpoint{8.050345in}{4.562771in}}{\pgfqpoint{8.043210in}{4.564461in}}%
\pgfusepath{stroke}%
\end{pgfscope}%
\begin{pgfscope}%
\pgfpathrectangle{\pgfqpoint{6.720588in}{4.155455in}}{\pgfqpoint{2.279412in}{2.004545in}}%
\pgfusepath{clip}%
\pgfsetroundcap%
\pgfsetroundjoin%
\definecolor{currentfill}{rgb}{0.271305,0.019942,0.347269}%
\pgfsetfillcolor{currentfill}%
\pgfsetlinewidth{0.327050pt}%
\definecolor{currentstroke}{rgb}{0.271305,0.019942,0.347269}%
\pgfsetstrokecolor{currentstroke}%
\pgfsetdash{}{0pt}%
\pgfpathmoveto{\pgfqpoint{8.090867in}{4.524626in}}%
\pgfpathlineto{\pgfqpoint{8.043210in}{4.564461in}}%
\pgfpathlineto{\pgfqpoint{8.103672in}{4.578686in}}%
\pgfpathlineto{\pgfqpoint{8.090867in}{4.524626in}}%
\pgfpathlineto{\pgfqpoint{8.090867in}{4.524626in}}%
\pgfpathclose%
\pgfusepath{stroke,fill}%
\end{pgfscope}%
\begin{pgfscope}%
\pgfpathrectangle{\pgfqpoint{6.720588in}{4.155455in}}{\pgfqpoint{2.279412in}{2.004545in}}%
\pgfusepath{clip}%
\pgfsetroundcap%
\pgfsetroundjoin%
\pgfsetlinewidth{0.420497pt}%
\definecolor{currentstroke}{rgb}{0.282656,0.100196,0.422160}%
\pgfsetstrokecolor{currentstroke}%
\pgfsetdash{}{0pt}%
\pgfpathmoveto{\pgfqpoint{8.227586in}{4.771808in}}%
\pgfpathquadraticcurveto{\pgfqpoint{8.215138in}{4.773120in}}{\pgfqpoint{8.209159in}{4.773750in}}%
\pgfusepath{stroke}%
\end{pgfscope}%
\begin{pgfscope}%
\pgfpathrectangle{\pgfqpoint{6.720588in}{4.155455in}}{\pgfqpoint{2.279412in}{2.004545in}}%
\pgfusepath{clip}%
\pgfsetroundcap%
\pgfsetroundjoin%
\definecolor{currentfill}{rgb}{0.282656,0.100196,0.422160}%
\pgfsetfillcolor{currentfill}%
\pgfsetlinewidth{0.420497pt}%
\definecolor{currentstroke}{rgb}{0.282656,0.100196,0.422160}%
\pgfsetstrokecolor{currentstroke}%
\pgfsetdash{}{0pt}%
\pgfpathmoveto{\pgfqpoint{8.261498in}{4.740303in}}%
\pgfpathlineto{\pgfqpoint{8.209159in}{4.773750in}}%
\pgfpathlineto{\pgfqpoint{8.267320in}{4.795553in}}%
\pgfpathlineto{\pgfqpoint{8.261498in}{4.740303in}}%
\pgfpathlineto{\pgfqpoint{8.261498in}{4.740303in}}%
\pgfpathclose%
\pgfusepath{stroke,fill}%
\end{pgfscope}%
\begin{pgfscope}%
\pgfpathrectangle{\pgfqpoint{6.720588in}{4.155455in}}{\pgfqpoint{2.279412in}{2.004545in}}%
\pgfusepath{clip}%
\pgfsetroundcap%
\pgfsetroundjoin%
\pgfsetlinewidth{0.384605pt}%
\definecolor{currentstroke}{rgb}{0.280267,0.073417,0.397163}%
\pgfsetstrokecolor{currentstroke}%
\pgfsetdash{}{0pt}%
\pgfpathmoveto{\pgfqpoint{8.227093in}{5.590197in}}%
\pgfpathquadraticcurveto{\pgfqpoint{8.214660in}{5.588793in}}{\pgfqpoint{8.208139in}{5.588056in}}%
\pgfusepath{stroke}%
\end{pgfscope}%
\begin{pgfscope}%
\pgfpathrectangle{\pgfqpoint{6.720588in}{4.155455in}}{\pgfqpoint{2.279412in}{2.004545in}}%
\pgfusepath{clip}%
\pgfsetroundcap%
\pgfsetroundjoin%
\definecolor{currentfill}{rgb}{0.280267,0.073417,0.397163}%
\pgfsetfillcolor{currentfill}%
\pgfsetlinewidth{0.384605pt}%
\definecolor{currentstroke}{rgb}{0.280267,0.073417,0.397163}%
\pgfsetstrokecolor{currentstroke}%
\pgfsetdash{}{0pt}%
\pgfpathmoveto{\pgfqpoint{8.266461in}{5.566688in}}%
\pgfpathlineto{\pgfqpoint{8.208139in}{5.588056in}}%
\pgfpathlineto{\pgfqpoint{8.260226in}{5.621893in}}%
\pgfpathlineto{\pgfqpoint{8.266461in}{5.566688in}}%
\pgfpathlineto{\pgfqpoint{8.266461in}{5.566688in}}%
\pgfpathclose%
\pgfusepath{stroke,fill}%
\end{pgfscope}%
\begin{pgfscope}%
\pgfpathrectangle{\pgfqpoint{6.720588in}{4.155455in}}{\pgfqpoint{2.279412in}{2.004545in}}%
\pgfusepath{clip}%
\pgfsetroundcap%
\pgfsetroundjoin%
\pgfsetlinewidth{0.372541pt}%
\definecolor{currentstroke}{rgb}{0.278791,0.062145,0.386592}%
\pgfsetstrokecolor{currentstroke}%
\pgfsetdash{}{0pt}%
\pgfpathmoveto{\pgfqpoint{8.375580in}{4.845333in}}%
\pgfpathquadraticcurveto{\pgfqpoint{8.363057in}{4.845848in}}{\pgfqpoint{8.356293in}{4.846126in}}%
\pgfusepath{stroke}%
\end{pgfscope}%
\begin{pgfscope}%
\pgfpathrectangle{\pgfqpoint{6.720588in}{4.155455in}}{\pgfqpoint{2.279412in}{2.004545in}}%
\pgfusepath{clip}%
\pgfsetroundcap%
\pgfsetroundjoin%
\definecolor{currentfill}{rgb}{0.278791,0.062145,0.386592}%
\pgfsetfillcolor{currentfill}%
\pgfsetlinewidth{0.372541pt}%
\definecolor{currentstroke}{rgb}{0.278791,0.062145,0.386592}%
\pgfsetstrokecolor{currentstroke}%
\pgfsetdash{}{0pt}%
\pgfpathmoveto{\pgfqpoint{8.410660in}{4.816089in}}%
\pgfpathlineto{\pgfqpoint{8.356293in}{4.846126in}}%
\pgfpathlineto{\pgfqpoint{8.412943in}{4.871598in}}%
\pgfpathlineto{\pgfqpoint{8.410660in}{4.816089in}}%
\pgfpathlineto{\pgfqpoint{8.410660in}{4.816089in}}%
\pgfpathclose%
\pgfusepath{stroke,fill}%
\end{pgfscope}%
\begin{pgfscope}%
\pgfpathrectangle{\pgfqpoint{6.720588in}{4.155455in}}{\pgfqpoint{2.279412in}{2.004545in}}%
\pgfusepath{clip}%
\pgfsetroundcap%
\pgfsetroundjoin%
\pgfsetlinewidth{0.383011pt}%
\definecolor{currentstroke}{rgb}{0.279566,0.067836,0.391917}%
\pgfsetstrokecolor{currentstroke}%
\pgfsetdash{}{0pt}%
\pgfpathmoveto{\pgfqpoint{8.126011in}{5.663885in}}%
\pgfpathquadraticcurveto{\pgfqpoint{8.113789in}{5.661477in}}{\pgfqpoint{8.107381in}{5.660214in}}%
\pgfusepath{stroke}%
\end{pgfscope}%
\begin{pgfscope}%
\pgfpathrectangle{\pgfqpoint{6.720588in}{4.155455in}}{\pgfqpoint{2.279412in}{2.004545in}}%
\pgfusepath{clip}%
\pgfsetroundcap%
\pgfsetroundjoin%
\definecolor{currentfill}{rgb}{0.279566,0.067836,0.391917}%
\pgfsetfillcolor{currentfill}%
\pgfsetlinewidth{0.383011pt}%
\definecolor{currentstroke}{rgb}{0.279566,0.067836,0.391917}%
\pgfsetstrokecolor{currentstroke}%
\pgfsetdash{}{0pt}%
\pgfpathmoveto{\pgfqpoint{8.167259in}{5.643701in}}%
\pgfpathlineto{\pgfqpoint{8.107381in}{5.660214in}}%
\pgfpathlineto{\pgfqpoint{8.156518in}{5.698209in}}%
\pgfpathlineto{\pgfqpoint{8.167259in}{5.643701in}}%
\pgfpathlineto{\pgfqpoint{8.167259in}{5.643701in}}%
\pgfpathclose%
\pgfusepath{stroke,fill}%
\end{pgfscope}%
\begin{pgfscope}%
\pgfpathrectangle{\pgfqpoint{6.720588in}{4.155455in}}{\pgfqpoint{2.279412in}{2.004545in}}%
\pgfusepath{clip}%
\pgfsetroundcap%
\pgfsetroundjoin%
\pgfsetlinewidth{0.357904pt}%
\definecolor{currentstroke}{rgb}{0.277018,0.050344,0.375715}%
\pgfsetstrokecolor{currentstroke}%
\pgfsetdash{}{0pt}%
\pgfpathmoveto{\pgfqpoint{8.068034in}{5.688682in}}%
\pgfpathquadraticcurveto{\pgfqpoint{8.055880in}{5.686034in}}{\pgfqpoint{8.049135in}{5.684565in}}%
\pgfusepath{stroke}%
\end{pgfscope}%
\begin{pgfscope}%
\pgfpathrectangle{\pgfqpoint{6.720588in}{4.155455in}}{\pgfqpoint{2.279412in}{2.004545in}}%
\pgfusepath{clip}%
\pgfsetroundcap%
\pgfsetroundjoin%
\definecolor{currentfill}{rgb}{0.277018,0.050344,0.375715}%
\pgfsetfillcolor{currentfill}%
\pgfsetlinewidth{0.357904pt}%
\definecolor{currentstroke}{rgb}{0.277018,0.050344,0.375715}%
\pgfsetstrokecolor{currentstroke}%
\pgfsetdash{}{0pt}%
\pgfpathmoveto{\pgfqpoint{8.109330in}{5.669248in}}%
\pgfpathlineto{\pgfqpoint{8.049135in}{5.684565in}}%
\pgfpathlineto{\pgfqpoint{8.097505in}{5.723530in}}%
\pgfpathlineto{\pgfqpoint{8.109330in}{5.669248in}}%
\pgfpathlineto{\pgfqpoint{8.109330in}{5.669248in}}%
\pgfpathclose%
\pgfusepath{stroke,fill}%
\end{pgfscope}%
\begin{pgfscope}%
\pgfpathrectangle{\pgfqpoint{6.720588in}{4.155455in}}{\pgfqpoint{2.279412in}{2.004545in}}%
\pgfusepath{clip}%
\pgfsetroundcap%
\pgfsetroundjoin%
\pgfsetlinewidth{0.374347pt}%
\definecolor{currentstroke}{rgb}{0.278791,0.062145,0.386592}%
\pgfsetstrokecolor{currentstroke}%
\pgfsetdash{}{0pt}%
\pgfpathmoveto{\pgfqpoint{8.265436in}{4.641832in}}%
\pgfpathquadraticcurveto{\pgfqpoint{8.253016in}{4.643322in}}{\pgfqpoint{8.246346in}{4.644123in}}%
\pgfusepath{stroke}%
\end{pgfscope}%
\begin{pgfscope}%
\pgfpathrectangle{\pgfqpoint{6.720588in}{4.155455in}}{\pgfqpoint{2.279412in}{2.004545in}}%
\pgfusepath{clip}%
\pgfsetroundcap%
\pgfsetroundjoin%
\definecolor{currentfill}{rgb}{0.278791,0.062145,0.386592}%
\pgfsetfillcolor{currentfill}%
\pgfsetlinewidth{0.374347pt}%
\definecolor{currentstroke}{rgb}{0.278791,0.062145,0.386592}%
\pgfsetstrokecolor{currentstroke}%
\pgfsetdash{}{0pt}%
\pgfpathmoveto{\pgfqpoint{8.298196in}{4.609924in}}%
\pgfpathlineto{\pgfqpoint{8.246346in}{4.644123in}}%
\pgfpathlineto{\pgfqpoint{8.304815in}{4.665084in}}%
\pgfpathlineto{\pgfqpoint{8.298196in}{4.609924in}}%
\pgfpathlineto{\pgfqpoint{8.298196in}{4.609924in}}%
\pgfpathclose%
\pgfusepath{stroke,fill}%
\end{pgfscope}%
\begin{pgfscope}%
\pgfpathrectangle{\pgfqpoint{6.720588in}{4.155455in}}{\pgfqpoint{2.279412in}{2.004545in}}%
\pgfusepath{clip}%
\pgfsetroundcap%
\pgfsetroundjoin%
\pgfsetlinewidth{0.334284pt}%
\definecolor{currentstroke}{rgb}{0.272594,0.025563,0.353093}%
\pgfsetstrokecolor{currentstroke}%
\pgfsetdash{}{0pt}%
\pgfpathmoveto{\pgfqpoint{8.368871in}{5.667434in}}%
\pgfpathquadraticcurveto{\pgfqpoint{8.356416in}{5.666309in}}{\pgfqpoint{8.349111in}{5.665649in}}%
\pgfusepath{stroke}%
\end{pgfscope}%
\begin{pgfscope}%
\pgfpathrectangle{\pgfqpoint{6.720588in}{4.155455in}}{\pgfqpoint{2.279412in}{2.004545in}}%
\pgfusepath{clip}%
\pgfsetroundcap%
\pgfsetroundjoin%
\definecolor{currentfill}{rgb}{0.272594,0.025563,0.353093}%
\pgfsetfillcolor{currentfill}%
\pgfsetlinewidth{0.334284pt}%
\definecolor{currentstroke}{rgb}{0.272594,0.025563,0.353093}%
\pgfsetstrokecolor{currentstroke}%
\pgfsetdash{}{0pt}%
\pgfpathmoveto{\pgfqpoint{8.406940in}{5.642982in}}%
\pgfpathlineto{\pgfqpoint{8.349111in}{5.665649in}}%
\pgfpathlineto{\pgfqpoint{8.401943in}{5.698312in}}%
\pgfpathlineto{\pgfqpoint{8.406940in}{5.642982in}}%
\pgfpathlineto{\pgfqpoint{8.406940in}{5.642982in}}%
\pgfpathclose%
\pgfusepath{stroke,fill}%
\end{pgfscope}%
\begin{pgfscope}%
\pgfpathrectangle{\pgfqpoint{6.720588in}{4.155455in}}{\pgfqpoint{2.279412in}{2.004545in}}%
\pgfusepath{clip}%
\pgfsetroundcap%
\pgfsetroundjoin%
\pgfsetlinewidth{0.412902pt}%
\definecolor{currentstroke}{rgb}{0.282327,0.094955,0.417331}%
\pgfsetstrokecolor{currentstroke}%
\pgfsetdash{}{0pt}%
\pgfpathmoveto{\pgfqpoint{7.872799in}{4.744530in}}%
\pgfpathquadraticcurveto{\pgfqpoint{7.862493in}{4.750622in}}{\pgfqpoint{7.857686in}{4.753464in}}%
\pgfusepath{stroke}%
\end{pgfscope}%
\begin{pgfscope}%
\pgfpathrectangle{\pgfqpoint{6.720588in}{4.155455in}}{\pgfqpoint{2.279412in}{2.004545in}}%
\pgfusepath{clip}%
\pgfsetroundcap%
\pgfsetroundjoin%
\definecolor{currentfill}{rgb}{0.282327,0.094955,0.417331}%
\pgfsetfillcolor{currentfill}%
\pgfsetlinewidth{0.412902pt}%
\definecolor{currentstroke}{rgb}{0.282327,0.094955,0.417331}%
\pgfsetstrokecolor{currentstroke}%
\pgfsetdash{}{0pt}%
\pgfpathmoveto{\pgfqpoint{7.891373in}{4.701280in}}%
\pgfpathlineto{\pgfqpoint{7.857686in}{4.753464in}}%
\pgfpathlineto{\pgfqpoint{7.919646in}{4.749104in}}%
\pgfpathlineto{\pgfqpoint{7.891373in}{4.701280in}}%
\pgfpathlineto{\pgfqpoint{7.891373in}{4.701280in}}%
\pgfpathclose%
\pgfusepath{stroke,fill}%
\end{pgfscope}%
\begin{pgfscope}%
\pgfpathrectangle{\pgfqpoint{6.720588in}{4.155455in}}{\pgfqpoint{2.279412in}{2.004545in}}%
\pgfusepath{clip}%
\pgfsetroundcap%
\pgfsetroundjoin%
\pgfsetlinewidth{1.326505pt}%
\definecolor{currentstroke}{rgb}{0.126326,0.644107,0.525311}%
\pgfsetstrokecolor{currentstroke}%
\pgfsetdash{}{0pt}%
\pgfpathmoveto{\pgfqpoint{7.684888in}{5.368714in}}%
\pgfpathquadraticcurveto{\pgfqpoint{7.675488in}{5.361439in}}{\pgfqpoint{7.682316in}{5.366723in}}%
\pgfusepath{stroke}%
\end{pgfscope}%
\begin{pgfscope}%
\pgfpathrectangle{\pgfqpoint{6.720588in}{4.155455in}}{\pgfqpoint{2.279412in}{2.004545in}}%
\pgfusepath{clip}%
\pgfsetroundcap%
\pgfsetroundjoin%
\definecolor{currentfill}{rgb}{0.126326,0.644107,0.525311}%
\pgfsetfillcolor{currentfill}%
\pgfsetlinewidth{1.326505pt}%
\definecolor{currentstroke}{rgb}{0.126326,0.644107,0.525311}%
\pgfsetstrokecolor{currentstroke}%
\pgfsetdash{}{0pt}%
\pgfpathmoveto{\pgfqpoint{7.743252in}{5.378758in}}%
\pgfpathlineto{\pgfqpoint{7.682316in}{5.366723in}}%
\pgfpathlineto{\pgfqpoint{7.709249in}{5.422693in}}%
\pgfpathlineto{\pgfqpoint{7.743252in}{5.378758in}}%
\pgfpathlineto{\pgfqpoint{7.743252in}{5.378758in}}%
\pgfpathclose%
\pgfusepath{stroke,fill}%
\end{pgfscope}%
\begin{pgfscope}%
\pgfpathrectangle{\pgfqpoint{6.720588in}{4.155455in}}{\pgfqpoint{2.279412in}{2.004545in}}%
\pgfusepath{clip}%
\pgfsetroundcap%
\pgfsetroundjoin%
\pgfsetlinewidth{1.298551pt}%
\definecolor{currentstroke}{rgb}{0.121380,0.629492,0.531973}%
\pgfsetstrokecolor{currentstroke}%
\pgfsetdash{}{0pt}%
\pgfpathmoveto{\pgfqpoint{7.915681in}{5.321677in}}%
\pgfpathquadraticcurveto{\pgfqpoint{7.903519in}{5.319009in}}{\pgfqpoint{7.910979in}{5.320646in}}%
\pgfusepath{stroke}%
\end{pgfscope}%
\begin{pgfscope}%
\pgfpathrectangle{\pgfqpoint{6.720588in}{4.155455in}}{\pgfqpoint{2.279412in}{2.004545in}}%
\pgfusepath{clip}%
\pgfsetroundcap%
\pgfsetroundjoin%
\definecolor{currentfill}{rgb}{0.121380,0.629492,0.531973}%
\pgfsetfillcolor{currentfill}%
\pgfsetlinewidth{1.298551pt}%
\definecolor{currentstroke}{rgb}{0.121380,0.629492,0.531973}%
\pgfsetstrokecolor{currentstroke}%
\pgfsetdash{}{0pt}%
\pgfpathmoveto{\pgfqpoint{7.971196in}{5.305414in}}%
\pgfpathlineto{\pgfqpoint{7.910979in}{5.320646in}}%
\pgfpathlineto{\pgfqpoint{7.959294in}{5.359680in}}%
\pgfpathlineto{\pgfqpoint{7.971196in}{5.305414in}}%
\pgfpathlineto{\pgfqpoint{7.971196in}{5.305414in}}%
\pgfpathclose%
\pgfusepath{stroke,fill}%
\end{pgfscope}%
\begin{pgfscope}%
\pgfpathrectangle{\pgfqpoint{6.720588in}{4.155455in}}{\pgfqpoint{2.279412in}{2.004545in}}%
\pgfusepath{clip}%
\pgfsetroundcap%
\pgfsetroundjoin%
\pgfsetlinewidth{1.993114pt}%
\definecolor{currentstroke}{rgb}{0.906311,0.894855,0.098125}%
\pgfsetstrokecolor{currentstroke}%
\pgfsetdash{}{0pt}%
\pgfpathmoveto{\pgfqpoint{7.806511in}{5.258045in}}%
\pgfpathquadraticcurveto{\pgfqpoint{7.794311in}{5.255519in}}{\pgfqpoint{7.812304in}{5.259245in}}%
\pgfusepath{stroke}%
\end{pgfscope}%
\begin{pgfscope}%
\pgfpathrectangle{\pgfqpoint{6.720588in}{4.155455in}}{\pgfqpoint{2.279412in}{2.004545in}}%
\pgfusepath{clip}%
\pgfsetroundcap%
\pgfsetroundjoin%
\definecolor{currentfill}{rgb}{0.906311,0.894855,0.098125}%
\pgfsetfillcolor{currentfill}%
\pgfsetlinewidth{1.993114pt}%
\definecolor{currentstroke}{rgb}{0.906311,0.894855,0.098125}%
\pgfsetstrokecolor{currentstroke}%
\pgfsetdash{}{0pt}%
\pgfpathmoveto{\pgfqpoint{7.872338in}{5.243308in}}%
\pgfpathlineto{\pgfqpoint{7.812304in}{5.259245in}}%
\pgfpathlineto{\pgfqpoint{7.861074in}{5.297710in}}%
\pgfpathlineto{\pgfqpoint{7.872338in}{5.243308in}}%
\pgfpathlineto{\pgfqpoint{7.872338in}{5.243308in}}%
\pgfpathclose%
\pgfusepath{stroke,fill}%
\end{pgfscope}%
\begin{pgfscope}%
\pgfpathrectangle{\pgfqpoint{6.720588in}{4.155455in}}{\pgfqpoint{2.279412in}{2.004545in}}%
\pgfusepath{clip}%
\pgfsetbuttcap%
\pgfsetroundjoin%
\pgfsetlinewidth{1.505625pt}%
\definecolor{currentstroke}{rgb}{0.000000,0.000000,0.000000}%
\pgfsetstrokecolor{currentstroke}%
\pgfsetdash{}{0pt}%
\pgfpathmoveto{\pgfqpoint{7.513972in}{4.494895in}}%
\pgfpathlineto{\pgfqpoint{7.513972in}{5.820559in}}%
\pgfusepath{stroke}%
\end{pgfscope}%
\begin{pgfscope}%
\pgfpathrectangle{\pgfqpoint{6.720588in}{4.155455in}}{\pgfqpoint{2.279412in}{2.004545in}}%
\pgfusepath{clip}%
\pgfsetbuttcap%
\pgfsetroundjoin%
\pgfsetlinewidth{1.505625pt}%
\definecolor{currentstroke}{rgb}{0.000000,0.000000,0.000000}%
\pgfsetstrokecolor{currentstroke}%
\pgfsetdash{}{0pt}%
\pgfpathmoveto{\pgfqpoint{8.662499in}{4.494895in}}%
\pgfpathlineto{\pgfqpoint{8.662499in}{5.820559in}}%
\pgfusepath{stroke}%
\end{pgfscope}%
\begin{pgfscope}%
\pgfsetrectcap%
\pgfsetmiterjoin%
\pgfsetlinewidth{0.803000pt}%
\definecolor{currentstroke}{rgb}{0.000000,0.000000,0.000000}%
\pgfsetstrokecolor{currentstroke}%
\pgfsetdash{}{0pt}%
\pgfpathmoveto{\pgfqpoint{6.720588in}{4.155455in}}%
\pgfpathlineto{\pgfqpoint{6.720588in}{6.160000in}}%
\pgfusepath{stroke}%
\end{pgfscope}%
\begin{pgfscope}%
\pgfsetrectcap%
\pgfsetmiterjoin%
\pgfsetlinewidth{0.803000pt}%
\definecolor{currentstroke}{rgb}{0.000000,0.000000,0.000000}%
\pgfsetstrokecolor{currentstroke}%
\pgfsetdash{}{0pt}%
\pgfpathmoveto{\pgfqpoint{9.000000in}{4.155455in}}%
\pgfpathlineto{\pgfqpoint{9.000000in}{6.160000in}}%
\pgfusepath{stroke}%
\end{pgfscope}%
\begin{pgfscope}%
\pgfsetrectcap%
\pgfsetmiterjoin%
\pgfsetlinewidth{0.803000pt}%
\definecolor{currentstroke}{rgb}{0.000000,0.000000,0.000000}%
\pgfsetstrokecolor{currentstroke}%
\pgfsetdash{}{0pt}%
\pgfpathmoveto{\pgfqpoint{6.720588in}{4.155455in}}%
\pgfpathlineto{\pgfqpoint{9.000000in}{4.155455in}}%
\pgfusepath{stroke}%
\end{pgfscope}%
\begin{pgfscope}%
\pgfsetrectcap%
\pgfsetmiterjoin%
\pgfsetlinewidth{0.803000pt}%
\definecolor{currentstroke}{rgb}{0.000000,0.000000,0.000000}%
\pgfsetstrokecolor{currentstroke}%
\pgfsetdash{}{0pt}%
\pgfpathmoveto{\pgfqpoint{6.720588in}{6.160000in}}%
\pgfpathlineto{\pgfqpoint{9.000000in}{6.160000in}}%
\pgfusepath{stroke}%
\end{pgfscope}%
\begin{pgfscope}%
\definecolor{textcolor}{rgb}{0.000000,0.000000,0.000000}%
\pgfsetstrokecolor{textcolor}%
\pgfsetfillcolor{textcolor}%
\pgftext[x=7.860294in,y=6.243333in,,base]{\color{textcolor}\sffamily\fontsize{12.000000}{14.400000}\selectfont c)}%
\end{pgfscope}%
\begin{pgfscope}%
\pgfsetbuttcap%
\pgfsetmiterjoin%
\definecolor{currentfill}{rgb}{1.000000,1.000000,1.000000}%
\pgfsetfillcolor{currentfill}%
\pgfsetlinewidth{0.000000pt}%
\definecolor{currentstroke}{rgb}{0.000000,0.000000,0.000000}%
\pgfsetstrokecolor{currentstroke}%
\pgfsetstrokeopacity{0.000000}%
\pgfsetdash{}{0pt}%
\pgfpathmoveto{\pgfqpoint{1.250000in}{1.750000in}}%
\pgfpathlineto{\pgfqpoint{3.529412in}{1.750000in}}%
\pgfpathlineto{\pgfqpoint{3.529412in}{3.754545in}}%
\pgfpathlineto{\pgfqpoint{1.250000in}{3.754545in}}%
\pgfpathlineto{\pgfqpoint{1.250000in}{1.750000in}}%
\pgfpathclose%
\pgfusepath{fill}%
\end{pgfscope}%
\begin{pgfscope}%
\pgfpathrectangle{\pgfqpoint{1.250000in}{1.750000in}}{\pgfqpoint{2.279412in}{2.004545in}}%
\pgfusepath{clip}%
\pgfsys@transformcm{2.291667}{0.000000}{0.000000}{2.013889}{1.250000in}{1.750000in}%
\pgftext[left,bottom]{\includegraphics[interpolate=false,width=1.000000in,height=1.000000in]{q_series-img3.png}}%
\end{pgfscope}%
\begin{pgfscope}%
\pgfsetbuttcap%
\pgfsetroundjoin%
\definecolor{currentfill}{rgb}{0.000000,0.000000,0.000000}%
\pgfsetfillcolor{currentfill}%
\pgfsetlinewidth{0.803000pt}%
\definecolor{currentstroke}{rgb}{0.000000,0.000000,0.000000}%
\pgfsetstrokecolor{currentstroke}%
\pgfsetdash{}{0pt}%
\pgfsys@defobject{currentmarker}{\pgfqpoint{0.000000in}{-0.048611in}}{\pgfqpoint{0.000000in}{0.000000in}}{%
\pgfpathmoveto{\pgfqpoint{0.000000in}{0.000000in}}%
\pgfpathlineto{\pgfqpoint{0.000000in}{-0.048611in}}%
\pgfusepath{stroke,fill}%
}%
\begin{pgfscope}%
\pgfsys@transformshift{1.660542in}{1.750000in}%
\pgfsys@useobject{currentmarker}{}%
\end{pgfscope}%
\end{pgfscope}%
\begin{pgfscope}%
\definecolor{textcolor}{rgb}{0.000000,0.000000,0.000000}%
\pgfsetstrokecolor{textcolor}%
\pgfsetfillcolor{textcolor}%
\pgftext[x=1.660542in,y=1.652778in,,top]{\color{textcolor}\sffamily\fontsize{10.000000}{12.000000}\selectfont \(\displaystyle {\ensuremath{-}10}\)}%
\end{pgfscope}%
\begin{pgfscope}%
\pgfsetbuttcap%
\pgfsetroundjoin%
\definecolor{currentfill}{rgb}{0.000000,0.000000,0.000000}%
\pgfsetfillcolor{currentfill}%
\pgfsetlinewidth{0.803000pt}%
\definecolor{currentstroke}{rgb}{0.000000,0.000000,0.000000}%
\pgfsetstrokecolor{currentstroke}%
\pgfsetdash{}{0pt}%
\pgfsys@defobject{currentmarker}{\pgfqpoint{0.000000in}{-0.048611in}}{\pgfqpoint{0.000000in}{0.000000in}}{%
\pgfpathmoveto{\pgfqpoint{0.000000in}{0.000000in}}%
\pgfpathlineto{\pgfqpoint{0.000000in}{-0.048611in}}%
\pgfusepath{stroke,fill}%
}%
\begin{pgfscope}%
\pgfsys@transformshift{2.139094in}{1.750000in}%
\pgfsys@useobject{currentmarker}{}%
\end{pgfscope}%
\end{pgfscope}%
\begin{pgfscope}%
\definecolor{textcolor}{rgb}{0.000000,0.000000,0.000000}%
\pgfsetstrokecolor{textcolor}%
\pgfsetfillcolor{textcolor}%
\pgftext[x=2.139094in,y=1.652778in,,top]{\color{textcolor}\sffamily\fontsize{10.000000}{12.000000}\selectfont \(\displaystyle {\ensuremath{-}5}\)}%
\end{pgfscope}%
\begin{pgfscope}%
\pgfsetbuttcap%
\pgfsetroundjoin%
\definecolor{currentfill}{rgb}{0.000000,0.000000,0.000000}%
\pgfsetfillcolor{currentfill}%
\pgfsetlinewidth{0.803000pt}%
\definecolor{currentstroke}{rgb}{0.000000,0.000000,0.000000}%
\pgfsetstrokecolor{currentstroke}%
\pgfsetdash{}{0pt}%
\pgfsys@defobject{currentmarker}{\pgfqpoint{0.000000in}{-0.048611in}}{\pgfqpoint{0.000000in}{0.000000in}}{%
\pgfpathmoveto{\pgfqpoint{0.000000in}{0.000000in}}%
\pgfpathlineto{\pgfqpoint{0.000000in}{-0.048611in}}%
\pgfusepath{stroke,fill}%
}%
\begin{pgfscope}%
\pgfsys@transformshift{2.617647in}{1.750000in}%
\pgfsys@useobject{currentmarker}{}%
\end{pgfscope}%
\end{pgfscope}%
\begin{pgfscope}%
\definecolor{textcolor}{rgb}{0.000000,0.000000,0.000000}%
\pgfsetstrokecolor{textcolor}%
\pgfsetfillcolor{textcolor}%
\pgftext[x=2.617647in,y=1.652778in,,top]{\color{textcolor}\sffamily\fontsize{10.000000}{12.000000}\selectfont \(\displaystyle {0}\)}%
\end{pgfscope}%
\begin{pgfscope}%
\pgfsetbuttcap%
\pgfsetroundjoin%
\definecolor{currentfill}{rgb}{0.000000,0.000000,0.000000}%
\pgfsetfillcolor{currentfill}%
\pgfsetlinewidth{0.803000pt}%
\definecolor{currentstroke}{rgb}{0.000000,0.000000,0.000000}%
\pgfsetstrokecolor{currentstroke}%
\pgfsetdash{}{0pt}%
\pgfsys@defobject{currentmarker}{\pgfqpoint{0.000000in}{-0.048611in}}{\pgfqpoint{0.000000in}{0.000000in}}{%
\pgfpathmoveto{\pgfqpoint{0.000000in}{0.000000in}}%
\pgfpathlineto{\pgfqpoint{0.000000in}{-0.048611in}}%
\pgfusepath{stroke,fill}%
}%
\begin{pgfscope}%
\pgfsys@transformshift{3.096200in}{1.750000in}%
\pgfsys@useobject{currentmarker}{}%
\end{pgfscope}%
\end{pgfscope}%
\begin{pgfscope}%
\definecolor{textcolor}{rgb}{0.000000,0.000000,0.000000}%
\pgfsetstrokecolor{textcolor}%
\pgfsetfillcolor{textcolor}%
\pgftext[x=3.096200in,y=1.652778in,,top]{\color{textcolor}\sffamily\fontsize{10.000000}{12.000000}\selectfont \(\displaystyle {5}\)}%
\end{pgfscope}%
\begin{pgfscope}%
\definecolor{textcolor}{rgb}{0.000000,0.000000,0.000000}%
\pgfsetstrokecolor{textcolor}%
\pgfsetfillcolor{textcolor}%
\pgftext[x=2.389706in,y=1.473766in,,top]{\color{textcolor}\sffamily\fontsize{10.000000}{12.000000}\selectfont \(\displaystyle \zeta \, \mathrm{[\mu m]}\)}%
\end{pgfscope}%
\begin{pgfscope}%
\pgfsetbuttcap%
\pgfsetroundjoin%
\definecolor{currentfill}{rgb}{0.000000,0.000000,0.000000}%
\pgfsetfillcolor{currentfill}%
\pgfsetlinewidth{0.803000pt}%
\definecolor{currentstroke}{rgb}{0.000000,0.000000,0.000000}%
\pgfsetstrokecolor{currentstroke}%
\pgfsetdash{}{0pt}%
\pgfsys@defobject{currentmarker}{\pgfqpoint{-0.048611in}{0.000000in}}{\pgfqpoint{-0.000000in}{0.000000in}}{%
\pgfpathmoveto{\pgfqpoint{-0.000000in}{0.000000in}}%
\pgfpathlineto{\pgfqpoint{-0.048611in}{0.000000in}}%
\pgfusepath{stroke,fill}%
}%
\begin{pgfscope}%
\pgfsys@transformshift{1.250000in}{1.758025in}%
\pgfsys@useobject{currentmarker}{}%
\end{pgfscope}%
\end{pgfscope}%
\begin{pgfscope}%
\definecolor{textcolor}{rgb}{0.000000,0.000000,0.000000}%
\pgfsetstrokecolor{textcolor}%
\pgfsetfillcolor{textcolor}%
\pgftext[x=0.905863in, y=1.709799in, left, base]{\color{textcolor}\sffamily\fontsize{10.000000}{12.000000}\selectfont \(\displaystyle {\ensuremath{-}30}\)}%
\end{pgfscope}%
\begin{pgfscope}%
\pgfsetbuttcap%
\pgfsetroundjoin%
\definecolor{currentfill}{rgb}{0.000000,0.000000,0.000000}%
\pgfsetfillcolor{currentfill}%
\pgfsetlinewidth{0.803000pt}%
\definecolor{currentstroke}{rgb}{0.000000,0.000000,0.000000}%
\pgfsetstrokecolor{currentstroke}%
\pgfsetdash{}{0pt}%
\pgfsys@defobject{currentmarker}{\pgfqpoint{-0.048611in}{0.000000in}}{\pgfqpoint{-0.000000in}{0.000000in}}{%
\pgfpathmoveto{\pgfqpoint{-0.000000in}{0.000000in}}%
\pgfpathlineto{\pgfqpoint{-0.048611in}{0.000000in}}%
\pgfusepath{stroke,fill}%
}%
\begin{pgfscope}%
\pgfsys@transformshift{1.250000in}{2.089441in}%
\pgfsys@useobject{currentmarker}{}%
\end{pgfscope}%
\end{pgfscope}%
\begin{pgfscope}%
\definecolor{textcolor}{rgb}{0.000000,0.000000,0.000000}%
\pgfsetstrokecolor{textcolor}%
\pgfsetfillcolor{textcolor}%
\pgftext[x=0.905863in, y=2.041215in, left, base]{\color{textcolor}\sffamily\fontsize{10.000000}{12.000000}\selectfont \(\displaystyle {\ensuremath{-}20}\)}%
\end{pgfscope}%
\begin{pgfscope}%
\pgfsetbuttcap%
\pgfsetroundjoin%
\definecolor{currentfill}{rgb}{0.000000,0.000000,0.000000}%
\pgfsetfillcolor{currentfill}%
\pgfsetlinewidth{0.803000pt}%
\definecolor{currentstroke}{rgb}{0.000000,0.000000,0.000000}%
\pgfsetstrokecolor{currentstroke}%
\pgfsetdash{}{0pt}%
\pgfsys@defobject{currentmarker}{\pgfqpoint{-0.048611in}{0.000000in}}{\pgfqpoint{-0.000000in}{0.000000in}}{%
\pgfpathmoveto{\pgfqpoint{-0.000000in}{0.000000in}}%
\pgfpathlineto{\pgfqpoint{-0.048611in}{0.000000in}}%
\pgfusepath{stroke,fill}%
}%
\begin{pgfscope}%
\pgfsys@transformshift{1.250000in}{2.420857in}%
\pgfsys@useobject{currentmarker}{}%
\end{pgfscope}%
\end{pgfscope}%
\begin{pgfscope}%
\definecolor{textcolor}{rgb}{0.000000,0.000000,0.000000}%
\pgfsetstrokecolor{textcolor}%
\pgfsetfillcolor{textcolor}%
\pgftext[x=0.905863in, y=2.372631in, left, base]{\color{textcolor}\sffamily\fontsize{10.000000}{12.000000}\selectfont \(\displaystyle {\ensuremath{-}10}\)}%
\end{pgfscope}%
\begin{pgfscope}%
\pgfsetbuttcap%
\pgfsetroundjoin%
\definecolor{currentfill}{rgb}{0.000000,0.000000,0.000000}%
\pgfsetfillcolor{currentfill}%
\pgfsetlinewidth{0.803000pt}%
\definecolor{currentstroke}{rgb}{0.000000,0.000000,0.000000}%
\pgfsetstrokecolor{currentstroke}%
\pgfsetdash{}{0pt}%
\pgfsys@defobject{currentmarker}{\pgfqpoint{-0.048611in}{0.000000in}}{\pgfqpoint{-0.000000in}{0.000000in}}{%
\pgfpathmoveto{\pgfqpoint{-0.000000in}{0.000000in}}%
\pgfpathlineto{\pgfqpoint{-0.048611in}{0.000000in}}%
\pgfusepath{stroke,fill}%
}%
\begin{pgfscope}%
\pgfsys@transformshift{1.250000in}{2.752273in}%
\pgfsys@useobject{currentmarker}{}%
\end{pgfscope}%
\end{pgfscope}%
\begin{pgfscope}%
\definecolor{textcolor}{rgb}{0.000000,0.000000,0.000000}%
\pgfsetstrokecolor{textcolor}%
\pgfsetfillcolor{textcolor}%
\pgftext[x=1.083333in, y=2.704047in, left, base]{\color{textcolor}\sffamily\fontsize{10.000000}{12.000000}\selectfont \(\displaystyle {0}\)}%
\end{pgfscope}%
\begin{pgfscope}%
\pgfsetbuttcap%
\pgfsetroundjoin%
\definecolor{currentfill}{rgb}{0.000000,0.000000,0.000000}%
\pgfsetfillcolor{currentfill}%
\pgfsetlinewidth{0.803000pt}%
\definecolor{currentstroke}{rgb}{0.000000,0.000000,0.000000}%
\pgfsetstrokecolor{currentstroke}%
\pgfsetdash{}{0pt}%
\pgfsys@defobject{currentmarker}{\pgfqpoint{-0.048611in}{0.000000in}}{\pgfqpoint{-0.000000in}{0.000000in}}{%
\pgfpathmoveto{\pgfqpoint{-0.000000in}{0.000000in}}%
\pgfpathlineto{\pgfqpoint{-0.048611in}{0.000000in}}%
\pgfusepath{stroke,fill}%
}%
\begin{pgfscope}%
\pgfsys@transformshift{1.250000in}{3.083689in}%
\pgfsys@useobject{currentmarker}{}%
\end{pgfscope}%
\end{pgfscope}%
\begin{pgfscope}%
\definecolor{textcolor}{rgb}{0.000000,0.000000,0.000000}%
\pgfsetstrokecolor{textcolor}%
\pgfsetfillcolor{textcolor}%
\pgftext[x=1.013888in, y=3.035463in, left, base]{\color{textcolor}\sffamily\fontsize{10.000000}{12.000000}\selectfont \(\displaystyle {10}\)}%
\end{pgfscope}%
\begin{pgfscope}%
\pgfsetbuttcap%
\pgfsetroundjoin%
\definecolor{currentfill}{rgb}{0.000000,0.000000,0.000000}%
\pgfsetfillcolor{currentfill}%
\pgfsetlinewidth{0.803000pt}%
\definecolor{currentstroke}{rgb}{0.000000,0.000000,0.000000}%
\pgfsetstrokecolor{currentstroke}%
\pgfsetdash{}{0pt}%
\pgfsys@defobject{currentmarker}{\pgfqpoint{-0.048611in}{0.000000in}}{\pgfqpoint{-0.000000in}{0.000000in}}{%
\pgfpathmoveto{\pgfqpoint{-0.000000in}{0.000000in}}%
\pgfpathlineto{\pgfqpoint{-0.048611in}{0.000000in}}%
\pgfusepath{stroke,fill}%
}%
\begin{pgfscope}%
\pgfsys@transformshift{1.250000in}{3.415105in}%
\pgfsys@useobject{currentmarker}{}%
\end{pgfscope}%
\end{pgfscope}%
\begin{pgfscope}%
\definecolor{textcolor}{rgb}{0.000000,0.000000,0.000000}%
\pgfsetstrokecolor{textcolor}%
\pgfsetfillcolor{textcolor}%
\pgftext[x=1.013888in, y=3.366879in, left, base]{\color{textcolor}\sffamily\fontsize{10.000000}{12.000000}\selectfont \(\displaystyle {20}\)}%
\end{pgfscope}%
\begin{pgfscope}%
\pgfsetbuttcap%
\pgfsetroundjoin%
\definecolor{currentfill}{rgb}{0.000000,0.000000,0.000000}%
\pgfsetfillcolor{currentfill}%
\pgfsetlinewidth{0.803000pt}%
\definecolor{currentstroke}{rgb}{0.000000,0.000000,0.000000}%
\pgfsetstrokecolor{currentstroke}%
\pgfsetdash{}{0pt}%
\pgfsys@defobject{currentmarker}{\pgfqpoint{-0.048611in}{0.000000in}}{\pgfqpoint{-0.000000in}{0.000000in}}{%
\pgfpathmoveto{\pgfqpoint{-0.000000in}{0.000000in}}%
\pgfpathlineto{\pgfqpoint{-0.048611in}{0.000000in}}%
\pgfusepath{stroke,fill}%
}%
\begin{pgfscope}%
\pgfsys@transformshift{1.250000in}{3.746521in}%
\pgfsys@useobject{currentmarker}{}%
\end{pgfscope}%
\end{pgfscope}%
\begin{pgfscope}%
\definecolor{textcolor}{rgb}{0.000000,0.000000,0.000000}%
\pgfsetstrokecolor{textcolor}%
\pgfsetfillcolor{textcolor}%
\pgftext[x=1.013888in, y=3.698295in, left, base]{\color{textcolor}\sffamily\fontsize{10.000000}{12.000000}\selectfont \(\displaystyle {30}\)}%
\end{pgfscope}%
\begin{pgfscope}%
\definecolor{textcolor}{rgb}{0.000000,0.000000,0.000000}%
\pgfsetstrokecolor{textcolor}%
\pgfsetfillcolor{textcolor}%
\pgftext[x=0.850308in,y=2.752273in,,bottom,rotate=90.000000]{\color{textcolor}\sffamily\fontsize{10.000000}{12.000000}\selectfont \(\displaystyle z \, \mathrm{[\mu m]}\)}%
\end{pgfscope}%
\begin{pgfscope}%
\pgfpathrectangle{\pgfqpoint{1.250000in}{1.750000in}}{\pgfqpoint{2.279412in}{2.004545in}}%
\pgfusepath{clip}%
\pgfsetbuttcap%
\pgfsetroundjoin%
\pgfsetlinewidth{0.312301pt}%
\definecolor{currentstroke}{rgb}{0.268510,0.009605,0.335427}%
\pgfsetstrokecolor{currentstroke}%
\pgfsetdash{}{0pt}%
\pgfpathmoveto{\pgfqpoint{3.261668in}{2.842486in}}%
\pgfpathlineto{\pgfqpoint{3.261668in}{2.842486in}}%
\pgfusepath{stroke}%
\end{pgfscope}%
\begin{pgfscope}%
\pgfpathrectangle{\pgfqpoint{1.250000in}{1.750000in}}{\pgfqpoint{2.279412in}{2.004545in}}%
\pgfusepath{clip}%
\pgfsetbuttcap%
\pgfsetroundjoin%
\pgfsetlinewidth{0.312301pt}%
\definecolor{currentstroke}{rgb}{0.268510,0.009605,0.335427}%
\pgfsetstrokecolor{currentstroke}%
\pgfsetdash{}{0pt}%
\pgfpathmoveto{\pgfqpoint{3.261668in}{2.842486in}}%
\pgfpathlineto{\pgfqpoint{3.261668in}{2.842486in}}%
\pgfusepath{stroke}%
\end{pgfscope}%
\begin{pgfscope}%
\pgfpathrectangle{\pgfqpoint{1.250000in}{1.750000in}}{\pgfqpoint{2.279412in}{2.004545in}}%
\pgfusepath{clip}%
\pgfsetbuttcap%
\pgfsetroundjoin%
\pgfsetlinewidth{0.312301pt}%
\definecolor{currentstroke}{rgb}{0.268510,0.009605,0.335427}%
\pgfsetstrokecolor{currentstroke}%
\pgfsetdash{}{0pt}%
\pgfpathmoveto{\pgfqpoint{3.261668in}{2.842486in}}%
\pgfpathlineto{\pgfqpoint{3.246937in}{2.841358in}}%
\pgfusepath{stroke}%
\end{pgfscope}%
\begin{pgfscope}%
\pgfpathrectangle{\pgfqpoint{1.250000in}{1.750000in}}{\pgfqpoint{2.279412in}{2.004545in}}%
\pgfusepath{clip}%
\pgfsetbuttcap%
\pgfsetroundjoin%
\pgfsetlinewidth{0.320187pt}%
\definecolor{currentstroke}{rgb}{0.269944,0.014625,0.341379}%
\pgfsetstrokecolor{currentstroke}%
\pgfsetdash{}{0pt}%
\pgfpathmoveto{\pgfqpoint{3.246937in}{2.841358in}}%
\pgfpathlineto{\pgfqpoint{3.224977in}{2.840191in}}%
\pgfusepath{stroke}%
\end{pgfscope}%
\begin{pgfscope}%
\pgfpathrectangle{\pgfqpoint{1.250000in}{1.750000in}}{\pgfqpoint{2.279412in}{2.004545in}}%
\pgfusepath{clip}%
\pgfsetbuttcap%
\pgfsetroundjoin%
\pgfsetlinewidth{0.316338pt}%
\definecolor{currentstroke}{rgb}{0.269944,0.014625,0.341379}%
\pgfsetstrokecolor{currentstroke}%
\pgfsetdash{}{0pt}%
\pgfpathmoveto{\pgfqpoint{3.224977in}{2.840191in}}%
\pgfpathlineto{\pgfqpoint{3.174943in}{2.837755in}}%
\pgfusepath{stroke}%
\end{pgfscope}%
\begin{pgfscope}%
\pgfpathrectangle{\pgfqpoint{1.250000in}{1.750000in}}{\pgfqpoint{2.279412in}{2.004545in}}%
\pgfusepath{clip}%
\pgfsetbuttcap%
\pgfsetroundjoin%
\pgfsetlinewidth{0.321250pt}%
\definecolor{currentstroke}{rgb}{0.269944,0.014625,0.341379}%
\pgfsetstrokecolor{currentstroke}%
\pgfsetdash{}{0pt}%
\pgfpathmoveto{\pgfqpoint{3.174943in}{2.837755in}}%
\pgfpathlineto{\pgfqpoint{3.124818in}{2.836800in}}%
\pgfusepath{stroke}%
\end{pgfscope}%
\begin{pgfscope}%
\pgfpathrectangle{\pgfqpoint{1.250000in}{1.750000in}}{\pgfqpoint{2.279412in}{2.004545in}}%
\pgfusepath{clip}%
\pgfsetbuttcap%
\pgfsetroundjoin%
\pgfsetlinewidth{0.332243pt}%
\definecolor{currentstroke}{rgb}{0.272594,0.025563,0.353093}%
\pgfsetstrokecolor{currentstroke}%
\pgfsetdash{}{0pt}%
\pgfpathmoveto{\pgfqpoint{3.124818in}{2.836800in}}%
\pgfpathlineto{\pgfqpoint{3.074686in}{2.836038in}}%
\pgfusepath{stroke}%
\end{pgfscope}%
\begin{pgfscope}%
\pgfpathrectangle{\pgfqpoint{1.250000in}{1.750000in}}{\pgfqpoint{2.279412in}{2.004545in}}%
\pgfusepath{clip}%
\pgfsetbuttcap%
\pgfsetroundjoin%
\pgfsetlinewidth{0.335165pt}%
\definecolor{currentstroke}{rgb}{0.272594,0.025563,0.353093}%
\pgfsetstrokecolor{currentstroke}%
\pgfsetdash{}{0pt}%
\pgfpathmoveto{\pgfqpoint{3.074686in}{2.836038in}}%
\pgfpathlineto{\pgfqpoint{3.024546in}{2.835376in}}%
\pgfusepath{stroke}%
\end{pgfscope}%
\begin{pgfscope}%
\pgfpathrectangle{\pgfqpoint{1.250000in}{1.750000in}}{\pgfqpoint{2.279412in}{2.004545in}}%
\pgfusepath{clip}%
\pgfsetbuttcap%
\pgfsetroundjoin%
\pgfsetlinewidth{0.357648pt}%
\definecolor{currentstroke}{rgb}{0.277018,0.050344,0.375715}%
\pgfsetstrokecolor{currentstroke}%
\pgfsetdash{}{0pt}%
\pgfpathmoveto{\pgfqpoint{3.024546in}{2.835376in}}%
\pgfpathlineto{\pgfqpoint{2.974410in}{2.834395in}}%
\pgfusepath{stroke}%
\end{pgfscope}%
\begin{pgfscope}%
\pgfpathrectangle{\pgfqpoint{1.250000in}{1.750000in}}{\pgfqpoint{2.279412in}{2.004545in}}%
\pgfusepath{clip}%
\pgfsetbuttcap%
\pgfsetroundjoin%
\pgfsetlinewidth{0.376431pt}%
\definecolor{currentstroke}{rgb}{0.278791,0.062145,0.386592}%
\pgfsetstrokecolor{currentstroke}%
\pgfsetdash{}{0pt}%
\pgfpathmoveto{\pgfqpoint{2.974410in}{2.834395in}}%
\pgfpathlineto{\pgfqpoint{2.924269in}{2.833590in}}%
\pgfusepath{stroke}%
\end{pgfscope}%
\begin{pgfscope}%
\pgfpathrectangle{\pgfqpoint{1.250000in}{1.750000in}}{\pgfqpoint{2.279412in}{2.004545in}}%
\pgfusepath{clip}%
\pgfsetbuttcap%
\pgfsetroundjoin%
\pgfsetlinewidth{0.418115pt}%
\definecolor{currentstroke}{rgb}{0.282656,0.100196,0.422160}%
\pgfsetstrokecolor{currentstroke}%
\pgfsetdash{}{0pt}%
\pgfpathmoveto{\pgfqpoint{2.924269in}{2.833590in}}%
\pgfpathlineto{\pgfqpoint{2.874127in}{2.832762in}}%
\pgfusepath{stroke}%
\end{pgfscope}%
\begin{pgfscope}%
\pgfpathrectangle{\pgfqpoint{1.250000in}{1.750000in}}{\pgfqpoint{2.279412in}{2.004545in}}%
\pgfusepath{clip}%
\pgfsetbuttcap%
\pgfsetroundjoin%
\pgfsetlinewidth{0.482349pt}%
\definecolor{currentstroke}{rgb}{0.282290,0.145912,0.461510}%
\pgfsetstrokecolor{currentstroke}%
\pgfsetdash{}{0pt}%
\pgfpathmoveto{\pgfqpoint{2.874127in}{2.832762in}}%
\pgfpathlineto{\pgfqpoint{2.823988in}{2.831767in}}%
\pgfusepath{stroke}%
\end{pgfscope}%
\begin{pgfscope}%
\pgfpathrectangle{\pgfqpoint{1.250000in}{1.750000in}}{\pgfqpoint{2.279412in}{2.004545in}}%
\pgfusepath{clip}%
\pgfsetbuttcap%
\pgfsetroundjoin%
\pgfsetlinewidth{0.558793pt}%
\definecolor{currentstroke}{rgb}{0.274128,0.199721,0.498911}%
\pgfsetstrokecolor{currentstroke}%
\pgfsetdash{}{0pt}%
\pgfpathmoveto{\pgfqpoint{2.823988in}{2.831767in}}%
\pgfpathlineto{\pgfqpoint{2.773850in}{2.830732in}}%
\pgfusepath{stroke}%
\end{pgfscope}%
\begin{pgfscope}%
\pgfpathrectangle{\pgfqpoint{1.250000in}{1.750000in}}{\pgfqpoint{2.279412in}{2.004545in}}%
\pgfusepath{clip}%
\pgfsetbuttcap%
\pgfsetroundjoin%
\pgfsetlinewidth{0.710282pt}%
\definecolor{currentstroke}{rgb}{0.239346,0.300855,0.540844}%
\pgfsetstrokecolor{currentstroke}%
\pgfsetdash{}{0pt}%
\pgfpathmoveto{\pgfqpoint{2.773850in}{2.830732in}}%
\pgfpathlineto{\pgfqpoint{2.723717in}{2.829542in}}%
\pgfusepath{stroke}%
\end{pgfscope}%
\begin{pgfscope}%
\pgfpathrectangle{\pgfqpoint{1.250000in}{1.750000in}}{\pgfqpoint{2.279412in}{2.004545in}}%
\pgfusepath{clip}%
\pgfsetbuttcap%
\pgfsetroundjoin%
\pgfsetlinewidth{0.899260pt}%
\definecolor{currentstroke}{rgb}{0.187231,0.414746,0.556547}%
\pgfsetstrokecolor{currentstroke}%
\pgfsetdash{}{0pt}%
\pgfpathmoveto{\pgfqpoint{2.723717in}{2.829542in}}%
\pgfpathlineto{\pgfqpoint{2.673602in}{2.827900in}}%
\pgfusepath{stroke}%
\end{pgfscope}%
\begin{pgfscope}%
\pgfpathrectangle{\pgfqpoint{1.250000in}{1.750000in}}{\pgfqpoint{2.279412in}{2.004545in}}%
\pgfusepath{clip}%
\pgfsetbuttcap%
\pgfsetroundjoin%
\pgfsetlinewidth{1.204249pt}%
\definecolor{currentstroke}{rgb}{0.124395,0.578002,0.548287}%
\pgfsetstrokecolor{currentstroke}%
\pgfsetdash{}{0pt}%
\pgfpathmoveto{\pgfqpoint{2.673602in}{2.827900in}}%
\pgfpathlineto{\pgfqpoint{2.623520in}{2.825612in}}%
\pgfusepath{stroke}%
\end{pgfscope}%
\begin{pgfscope}%
\pgfpathrectangle{\pgfqpoint{1.250000in}{1.750000in}}{\pgfqpoint{2.279412in}{2.004545in}}%
\pgfusepath{clip}%
\pgfsetbuttcap%
\pgfsetroundjoin%
\pgfsetlinewidth{1.410375pt}%
\definecolor{currentstroke}{rgb}{0.162016,0.687316,0.499129}%
\pgfsetstrokecolor{currentstroke}%
\pgfsetdash{}{0pt}%
\pgfpathmoveto{\pgfqpoint{2.623520in}{2.825612in}}%
\pgfpathlineto{\pgfqpoint{2.573469in}{2.822846in}}%
\pgfusepath{stroke}%
\end{pgfscope}%
\begin{pgfscope}%
\pgfpathrectangle{\pgfqpoint{1.250000in}{1.750000in}}{\pgfqpoint{2.279412in}{2.004545in}}%
\pgfusepath{clip}%
\pgfsetbuttcap%
\pgfsetroundjoin%
\pgfsetlinewidth{1.600860pt}%
\definecolor{currentstroke}{rgb}{0.344074,0.780029,0.397381}%
\pgfsetstrokecolor{currentstroke}%
\pgfsetdash{}{0pt}%
\pgfpathmoveto{\pgfqpoint{2.573469in}{2.822846in}}%
\pgfpathlineto{\pgfqpoint{2.523460in}{2.819576in}}%
\pgfusepath{stroke}%
\end{pgfscope}%
\begin{pgfscope}%
\pgfpathrectangle{\pgfqpoint{1.250000in}{1.750000in}}{\pgfqpoint{2.279412in}{2.004545in}}%
\pgfusepath{clip}%
\pgfsetbuttcap%
\pgfsetroundjoin%
\pgfsetlinewidth{1.713784pt}%
\definecolor{currentstroke}{rgb}{0.487026,0.823929,0.312321}%
\pgfsetstrokecolor{currentstroke}%
\pgfsetdash{}{0pt}%
\pgfpathmoveto{\pgfqpoint{2.523460in}{2.819576in}}%
\pgfpathlineto{\pgfqpoint{2.473500in}{2.815755in}}%
\pgfusepath{stroke}%
\end{pgfscope}%
\begin{pgfscope}%
\pgfpathrectangle{\pgfqpoint{1.250000in}{1.750000in}}{\pgfqpoint{2.279412in}{2.004545in}}%
\pgfusepath{clip}%
\pgfsetbuttcap%
\pgfsetroundjoin%
\pgfsetlinewidth{1.979392pt}%
\definecolor{currentstroke}{rgb}{0.886271,0.892374,0.095374}%
\pgfsetstrokecolor{currentstroke}%
\pgfsetdash{}{0pt}%
\pgfpathmoveto{\pgfqpoint{2.473500in}{2.815755in}}%
\pgfpathlineto{\pgfqpoint{2.423587in}{2.811482in}}%
\pgfusepath{stroke}%
\end{pgfscope}%
\begin{pgfscope}%
\pgfpathrectangle{\pgfqpoint{1.250000in}{1.750000in}}{\pgfqpoint{2.279412in}{2.004545in}}%
\pgfusepath{clip}%
\pgfsetbuttcap%
\pgfsetroundjoin%
\pgfsetlinewidth{2.151564pt}%
\definecolor{currentstroke}{rgb}{0.993248,0.906157,0.143936}%
\pgfsetstrokecolor{currentstroke}%
\pgfsetdash{}{0pt}%
\pgfpathmoveto{\pgfqpoint{2.423587in}{2.811482in}}%
\pgfpathlineto{\pgfqpoint{2.373719in}{2.806821in}}%
\pgfusepath{stroke}%
\end{pgfscope}%
\begin{pgfscope}%
\pgfpathrectangle{\pgfqpoint{1.250000in}{1.750000in}}{\pgfqpoint{2.279412in}{2.004545in}}%
\pgfusepath{clip}%
\pgfsetbuttcap%
\pgfsetroundjoin%
\pgfsetlinewidth{2.076310pt}%
\definecolor{currentstroke}{rgb}{0.993248,0.906157,0.143936}%
\pgfsetstrokecolor{currentstroke}%
\pgfsetdash{}{0pt}%
\pgfpathmoveto{\pgfqpoint{2.373719in}{2.806821in}}%
\pgfpathlineto{\pgfqpoint{2.323909in}{2.801751in}}%
\pgfusepath{stroke}%
\end{pgfscope}%
\begin{pgfscope}%
\pgfpathrectangle{\pgfqpoint{1.250000in}{1.750000in}}{\pgfqpoint{2.279412in}{2.004545in}}%
\pgfusepath{clip}%
\pgfsetbuttcap%
\pgfsetroundjoin%
\pgfsetlinewidth{2.153260pt}%
\definecolor{currentstroke}{rgb}{0.993248,0.906157,0.143936}%
\pgfsetstrokecolor{currentstroke}%
\pgfsetdash{}{0pt}%
\pgfpathmoveto{\pgfqpoint{2.323909in}{2.801751in}}%
\pgfpathlineto{\pgfqpoint{2.274154in}{2.796284in}}%
\pgfusepath{stroke}%
\end{pgfscope}%
\begin{pgfscope}%
\pgfpathrectangle{\pgfqpoint{1.250000in}{1.750000in}}{\pgfqpoint{2.279412in}{2.004545in}}%
\pgfusepath{clip}%
\pgfsetbuttcap%
\pgfsetroundjoin%
\pgfsetlinewidth{2.182352pt}%
\definecolor{currentstroke}{rgb}{0.993248,0.906157,0.143936}%
\pgfsetstrokecolor{currentstroke}%
\pgfsetdash{}{0pt}%
\pgfpathmoveto{\pgfqpoint{2.274154in}{2.796284in}}%
\pgfpathlineto{\pgfqpoint{2.224461in}{2.790396in}}%
\pgfusepath{stroke}%
\end{pgfscope}%
\begin{pgfscope}%
\pgfpathrectangle{\pgfqpoint{1.250000in}{1.750000in}}{\pgfqpoint{2.279412in}{2.004545in}}%
\pgfusepath{clip}%
\pgfsetbuttcap%
\pgfsetroundjoin%
\pgfsetlinewidth{1.991862pt}%
\definecolor{currentstroke}{rgb}{0.906311,0.894855,0.098125}%
\pgfsetstrokecolor{currentstroke}%
\pgfsetdash{}{0pt}%
\pgfpathmoveto{\pgfqpoint{2.224461in}{2.790396in}}%
\pgfpathlineto{\pgfqpoint{2.174822in}{2.784232in}}%
\pgfusepath{stroke}%
\end{pgfscope}%
\begin{pgfscope}%
\pgfpathrectangle{\pgfqpoint{1.250000in}{1.750000in}}{\pgfqpoint{2.279412in}{2.004545in}}%
\pgfusepath{clip}%
\pgfsetbuttcap%
\pgfsetroundjoin%
\pgfsetlinewidth{1.925530pt}%
\definecolor{currentstroke}{rgb}{0.804182,0.882046,0.114965}%
\pgfsetstrokecolor{currentstroke}%
\pgfsetdash{}{0pt}%
\pgfpathmoveto{\pgfqpoint{2.174822in}{2.784232in}}%
\pgfpathlineto{\pgfqpoint{2.125238in}{2.777843in}}%
\pgfusepath{stroke}%
\end{pgfscope}%
\begin{pgfscope}%
\pgfpathrectangle{\pgfqpoint{1.250000in}{1.750000in}}{\pgfqpoint{2.279412in}{2.004545in}}%
\pgfusepath{clip}%
\pgfsetbuttcap%
\pgfsetroundjoin%
\pgfsetlinewidth{1.698256pt}%
\definecolor{currentstroke}{rgb}{0.468053,0.818921,0.323998}%
\pgfsetstrokecolor{currentstroke}%
\pgfsetdash{}{0pt}%
\pgfpathmoveto{\pgfqpoint{2.125238in}{2.777843in}}%
\pgfpathlineto{\pgfqpoint{2.075697in}{2.771142in}}%
\pgfusepath{stroke}%
\end{pgfscope}%
\begin{pgfscope}%
\pgfpathrectangle{\pgfqpoint{1.250000in}{1.750000in}}{\pgfqpoint{2.279412in}{2.004545in}}%
\pgfusepath{clip}%
\pgfsetbuttcap%
\pgfsetroundjoin%
\pgfsetlinewidth{1.499752pt}%
\definecolor{currentstroke}{rgb}{0.232815,0.732247,0.459277}%
\pgfsetstrokecolor{currentstroke}%
\pgfsetdash{}{0pt}%
\pgfpathmoveto{\pgfqpoint{2.075697in}{2.771142in}}%
\pgfpathlineto{\pgfqpoint{2.026230in}{2.764312in}}%
\pgfusepath{stroke}%
\end{pgfscope}%
\begin{pgfscope}%
\pgfpathrectangle{\pgfqpoint{1.250000in}{1.750000in}}{\pgfqpoint{2.279412in}{2.004545in}}%
\pgfusepath{clip}%
\pgfsetbuttcap%
\pgfsetroundjoin%
\pgfsetlinewidth{1.135869pt}%
\definecolor{currentstroke}{rgb}{0.136408,0.541173,0.554483}%
\pgfsetstrokecolor{currentstroke}%
\pgfsetdash{}{0pt}%
\pgfpathmoveto{\pgfqpoint{2.026230in}{2.764312in}}%
\pgfpathlineto{\pgfqpoint{1.976903in}{2.757656in}}%
\pgfusepath{stroke}%
\end{pgfscope}%
\begin{pgfscope}%
\pgfpathrectangle{\pgfqpoint{1.250000in}{1.750000in}}{\pgfqpoint{2.279412in}{2.004545in}}%
\pgfusepath{clip}%
\pgfsetbuttcap%
\pgfsetroundjoin%
\pgfsetlinewidth{1.240380pt}%
\definecolor{currentstroke}{rgb}{0.120565,0.596422,0.543611}%
\pgfsetstrokecolor{currentstroke}%
\pgfsetdash{}{0pt}%
\pgfpathmoveto{\pgfqpoint{1.976903in}{2.757656in}}%
\pgfpathlineto{\pgfqpoint{1.927546in}{2.751125in}}%
\pgfusepath{stroke}%
\end{pgfscope}%
\begin{pgfscope}%
\pgfpathrectangle{\pgfqpoint{1.250000in}{1.750000in}}{\pgfqpoint{2.279412in}{2.004545in}}%
\pgfusepath{clip}%
\pgfsetbuttcap%
\pgfsetroundjoin%
\pgfsetlinewidth{0.315376pt}%
\definecolor{currentstroke}{rgb}{0.269944,0.014625,0.341379}%
\pgfsetstrokecolor{currentstroke}%
\pgfsetdash{}{0pt}%
\pgfpathmoveto{\pgfqpoint{3.261668in}{2.887593in}}%
\pgfpathlineto{\pgfqpoint{3.261668in}{2.887593in}}%
\pgfusepath{stroke}%
\end{pgfscope}%
\begin{pgfscope}%
\pgfpathrectangle{\pgfqpoint{1.250000in}{1.750000in}}{\pgfqpoint{2.279412in}{2.004545in}}%
\pgfusepath{clip}%
\pgfsetbuttcap%
\pgfsetroundjoin%
\pgfsetlinewidth{0.315376pt}%
\definecolor{currentstroke}{rgb}{0.269944,0.014625,0.341379}%
\pgfsetstrokecolor{currentstroke}%
\pgfsetdash{}{0pt}%
\pgfpathmoveto{\pgfqpoint{3.261668in}{2.887593in}}%
\pgfpathlineto{\pgfqpoint{3.261668in}{2.887593in}}%
\pgfusepath{stroke}%
\end{pgfscope}%
\begin{pgfscope}%
\pgfpathrectangle{\pgfqpoint{1.250000in}{1.750000in}}{\pgfqpoint{2.279412in}{2.004545in}}%
\pgfusepath{clip}%
\pgfsetbuttcap%
\pgfsetroundjoin%
\pgfsetlinewidth{0.315376pt}%
\definecolor{currentstroke}{rgb}{0.269944,0.014625,0.341379}%
\pgfsetstrokecolor{currentstroke}%
\pgfsetdash{}{0pt}%
\pgfpathmoveto{\pgfqpoint{3.261668in}{2.887593in}}%
\pgfpathlineto{\pgfqpoint{3.247445in}{2.887001in}}%
\pgfusepath{stroke}%
\end{pgfscope}%
\begin{pgfscope}%
\pgfpathrectangle{\pgfqpoint{1.250000in}{1.750000in}}{\pgfqpoint{2.279412in}{2.004545in}}%
\pgfusepath{clip}%
\pgfsetbuttcap%
\pgfsetroundjoin%
\pgfsetlinewidth{0.324049pt}%
\definecolor{currentstroke}{rgb}{0.271305,0.019942,0.347269}%
\pgfsetstrokecolor{currentstroke}%
\pgfsetdash{}{0pt}%
\pgfpathmoveto{\pgfqpoint{3.247445in}{2.887001in}}%
\pgfpathlineto{\pgfqpoint{3.225216in}{2.886755in}}%
\pgfusepath{stroke}%
\end{pgfscope}%
\begin{pgfscope}%
\pgfpathrectangle{\pgfqpoint{1.250000in}{1.750000in}}{\pgfqpoint{2.279412in}{2.004545in}}%
\pgfusepath{clip}%
\pgfsetbuttcap%
\pgfsetroundjoin%
\pgfsetlinewidth{0.316195pt}%
\definecolor{currentstroke}{rgb}{0.269944,0.014625,0.341379}%
\pgfsetstrokecolor{currentstroke}%
\pgfsetdash{}{0pt}%
\pgfpathmoveto{\pgfqpoint{3.225216in}{2.886755in}}%
\pgfpathlineto{\pgfqpoint{3.178477in}{2.887528in}}%
\pgfusepath{stroke}%
\end{pgfscope}%
\begin{pgfscope}%
\pgfpathrectangle{\pgfqpoint{1.250000in}{1.750000in}}{\pgfqpoint{2.279412in}{2.004545in}}%
\pgfusepath{clip}%
\pgfsetbuttcap%
\pgfsetroundjoin%
\pgfsetlinewidth{0.305500pt}%
\definecolor{currentstroke}{rgb}{0.267004,0.004874,0.329415}%
\pgfsetstrokecolor{currentstroke}%
\pgfsetdash{}{0pt}%
\pgfpathmoveto{\pgfqpoint{3.178477in}{2.887528in}}%
\pgfpathlineto{\pgfqpoint{3.133098in}{2.886789in}}%
\pgfusepath{stroke}%
\end{pgfscope}%
\begin{pgfscope}%
\pgfpathrectangle{\pgfqpoint{1.250000in}{1.750000in}}{\pgfqpoint{2.279412in}{2.004545in}}%
\pgfusepath{clip}%
\pgfsetbuttcap%
\pgfsetroundjoin%
\pgfsetlinewidth{0.317350pt}%
\definecolor{currentstroke}{rgb}{0.269944,0.014625,0.341379}%
\pgfsetstrokecolor{currentstroke}%
\pgfsetdash{}{0pt}%
\pgfpathmoveto{\pgfqpoint{3.133098in}{2.886789in}}%
\pgfpathlineto{\pgfqpoint{3.083327in}{2.885373in}}%
\pgfusepath{stroke}%
\end{pgfscope}%
\begin{pgfscope}%
\pgfpathrectangle{\pgfqpoint{1.250000in}{1.750000in}}{\pgfqpoint{2.279412in}{2.004545in}}%
\pgfusepath{clip}%
\pgfsetbuttcap%
\pgfsetroundjoin%
\pgfsetlinewidth{0.336222pt}%
\definecolor{currentstroke}{rgb}{0.273809,0.031497,0.358853}%
\pgfsetstrokecolor{currentstroke}%
\pgfsetdash{}{0pt}%
\pgfpathmoveto{\pgfqpoint{3.083327in}{2.885373in}}%
\pgfpathlineto{\pgfqpoint{3.033188in}{2.884740in}}%
\pgfusepath{stroke}%
\end{pgfscope}%
\begin{pgfscope}%
\pgfpathrectangle{\pgfqpoint{1.250000in}{1.750000in}}{\pgfqpoint{2.279412in}{2.004545in}}%
\pgfusepath{clip}%
\pgfsetbuttcap%
\pgfsetroundjoin%
\pgfsetlinewidth{0.349370pt}%
\definecolor{currentstroke}{rgb}{0.276022,0.044167,0.370164}%
\pgfsetstrokecolor{currentstroke}%
\pgfsetdash{}{0pt}%
\pgfpathmoveto{\pgfqpoint{3.033188in}{2.884740in}}%
\pgfpathlineto{\pgfqpoint{2.983056in}{2.883681in}}%
\pgfusepath{stroke}%
\end{pgfscope}%
\begin{pgfscope}%
\pgfpathrectangle{\pgfqpoint{1.250000in}{1.750000in}}{\pgfqpoint{2.279412in}{2.004545in}}%
\pgfusepath{clip}%
\pgfsetbuttcap%
\pgfsetroundjoin%
\pgfsetlinewidth{0.367987pt}%
\definecolor{currentstroke}{rgb}{0.277941,0.056324,0.381191}%
\pgfsetstrokecolor{currentstroke}%
\pgfsetdash{}{0pt}%
\pgfpathmoveto{\pgfqpoint{2.983056in}{2.883681in}}%
\pgfpathlineto{\pgfqpoint{2.932925in}{2.882526in}}%
\pgfusepath{stroke}%
\end{pgfscope}%
\begin{pgfscope}%
\pgfpathrectangle{\pgfqpoint{1.250000in}{1.750000in}}{\pgfqpoint{2.279412in}{2.004545in}}%
\pgfusepath{clip}%
\pgfsetbuttcap%
\pgfsetroundjoin%
\pgfsetlinewidth{0.394733pt}%
\definecolor{currentstroke}{rgb}{0.280894,0.078907,0.402329}%
\pgfsetstrokecolor{currentstroke}%
\pgfsetdash{}{0pt}%
\pgfpathmoveto{\pgfqpoint{2.932925in}{2.882526in}}%
\pgfpathlineto{\pgfqpoint{2.882786in}{2.881606in}}%
\pgfusepath{stroke}%
\end{pgfscope}%
\begin{pgfscope}%
\pgfpathrectangle{\pgfqpoint{1.250000in}{1.750000in}}{\pgfqpoint{2.279412in}{2.004545in}}%
\pgfusepath{clip}%
\pgfsetbuttcap%
\pgfsetroundjoin%
\pgfsetlinewidth{0.453713pt}%
\definecolor{currentstroke}{rgb}{0.283187,0.125848,0.444960}%
\pgfsetstrokecolor{currentstroke}%
\pgfsetdash{}{0pt}%
\pgfpathmoveto{\pgfqpoint{2.882786in}{2.881606in}}%
\pgfpathlineto{\pgfqpoint{2.832655in}{2.880387in}}%
\pgfusepath{stroke}%
\end{pgfscope}%
\begin{pgfscope}%
\pgfpathrectangle{\pgfqpoint{1.250000in}{1.750000in}}{\pgfqpoint{2.279412in}{2.004545in}}%
\pgfusepath{clip}%
\pgfsetbuttcap%
\pgfsetroundjoin%
\pgfsetlinewidth{0.510378pt}%
\definecolor{currentstroke}{rgb}{0.280255,0.165693,0.476498}%
\pgfsetstrokecolor{currentstroke}%
\pgfsetdash{}{0pt}%
\pgfpathmoveto{\pgfqpoint{2.832655in}{2.880387in}}%
\pgfpathlineto{\pgfqpoint{2.782535in}{2.878836in}}%
\pgfusepath{stroke}%
\end{pgfscope}%
\begin{pgfscope}%
\pgfpathrectangle{\pgfqpoint{1.250000in}{1.750000in}}{\pgfqpoint{2.279412in}{2.004545in}}%
\pgfusepath{clip}%
\pgfsetbuttcap%
\pgfsetroundjoin%
\pgfsetlinewidth{0.612056pt}%
\definecolor{currentstroke}{rgb}{0.263663,0.237631,0.518762}%
\pgfsetstrokecolor{currentstroke}%
\pgfsetdash{}{0pt}%
\pgfpathmoveto{\pgfqpoint{2.782535in}{2.878836in}}%
\pgfpathlineto{\pgfqpoint{2.732430in}{2.876947in}}%
\pgfusepath{stroke}%
\end{pgfscope}%
\begin{pgfscope}%
\pgfpathrectangle{\pgfqpoint{1.250000in}{1.750000in}}{\pgfqpoint{2.279412in}{2.004545in}}%
\pgfusepath{clip}%
\pgfsetbuttcap%
\pgfsetroundjoin%
\pgfsetlinewidth{0.777335pt}%
\definecolor{currentstroke}{rgb}{0.220057,0.343307,0.549413}%
\pgfsetstrokecolor{currentstroke}%
\pgfsetdash{}{0pt}%
\pgfpathmoveto{\pgfqpoint{2.732430in}{2.876947in}}%
\pgfpathlineto{\pgfqpoint{2.682360in}{2.874459in}}%
\pgfusepath{stroke}%
\end{pgfscope}%
\begin{pgfscope}%
\pgfpathrectangle{\pgfqpoint{1.250000in}{1.750000in}}{\pgfqpoint{2.279412in}{2.004545in}}%
\pgfusepath{clip}%
\pgfsetbuttcap%
\pgfsetroundjoin%
\pgfsetlinewidth{0.948281pt}%
\definecolor{currentstroke}{rgb}{0.175841,0.441290,0.557685}%
\pgfsetstrokecolor{currentstroke}%
\pgfsetdash{}{0pt}%
\pgfpathmoveto{\pgfqpoint{2.682360in}{2.874459in}}%
\pgfpathlineto{\pgfqpoint{2.632341in}{2.871266in}}%
\pgfusepath{stroke}%
\end{pgfscope}%
\begin{pgfscope}%
\pgfpathrectangle{\pgfqpoint{1.250000in}{1.750000in}}{\pgfqpoint{2.279412in}{2.004545in}}%
\pgfusepath{clip}%
\pgfsetbuttcap%
\pgfsetroundjoin%
\pgfsetlinewidth{1.131860pt}%
\definecolor{currentstroke}{rgb}{0.136408,0.541173,0.554483}%
\pgfsetstrokecolor{currentstroke}%
\pgfsetdash{}{0pt}%
\pgfpathmoveto{\pgfqpoint{2.632341in}{2.871266in}}%
\pgfpathlineto{\pgfqpoint{2.582394in}{2.867313in}}%
\pgfusepath{stroke}%
\end{pgfscope}%
\begin{pgfscope}%
\pgfpathrectangle{\pgfqpoint{1.250000in}{1.750000in}}{\pgfqpoint{2.279412in}{2.004545in}}%
\pgfusepath{clip}%
\pgfsetbuttcap%
\pgfsetroundjoin%
\pgfsetlinewidth{1.373917pt}%
\definecolor{currentstroke}{rgb}{0.143303,0.669459,0.511215}%
\pgfsetstrokecolor{currentstroke}%
\pgfsetdash{}{0pt}%
\pgfpathmoveto{\pgfqpoint{2.582394in}{2.867313in}}%
\pgfpathlineto{\pgfqpoint{2.532537in}{2.862580in}}%
\pgfusepath{stroke}%
\end{pgfscope}%
\begin{pgfscope}%
\pgfpathrectangle{\pgfqpoint{1.250000in}{1.750000in}}{\pgfqpoint{2.279412in}{2.004545in}}%
\pgfusepath{clip}%
\pgfsetbuttcap%
\pgfsetroundjoin%
\pgfsetlinewidth{1.450399pt}%
\definecolor{currentstroke}{rgb}{0.191090,0.708366,0.482284}%
\pgfsetstrokecolor{currentstroke}%
\pgfsetdash{}{0pt}%
\pgfpathmoveto{\pgfqpoint{2.532537in}{2.862580in}}%
\pgfpathlineto{\pgfqpoint{2.482781in}{2.857084in}}%
\pgfusepath{stroke}%
\end{pgfscope}%
\begin{pgfscope}%
\pgfpathrectangle{\pgfqpoint{1.250000in}{1.750000in}}{\pgfqpoint{2.279412in}{2.004545in}}%
\pgfusepath{clip}%
\pgfsetbuttcap%
\pgfsetroundjoin%
\pgfsetlinewidth{1.544243pt}%
\definecolor{currentstroke}{rgb}{0.281477,0.755203,0.432552}%
\pgfsetstrokecolor{currentstroke}%
\pgfsetdash{}{0pt}%
\pgfpathmoveto{\pgfqpoint{2.482781in}{2.857084in}}%
\pgfpathlineto{\pgfqpoint{2.433158in}{2.850723in}}%
\pgfusepath{stroke}%
\end{pgfscope}%
\begin{pgfscope}%
\pgfpathrectangle{\pgfqpoint{1.250000in}{1.750000in}}{\pgfqpoint{2.279412in}{2.004545in}}%
\pgfusepath{clip}%
\pgfsetbuttcap%
\pgfsetroundjoin%
\pgfsetlinewidth{1.871190pt}%
\definecolor{currentstroke}{rgb}{0.720391,0.870350,0.162603}%
\pgfsetstrokecolor{currentstroke}%
\pgfsetdash{}{0pt}%
\pgfpathmoveto{\pgfqpoint{2.433158in}{2.850723in}}%
\pgfpathlineto{\pgfqpoint{2.383682in}{2.843542in}}%
\pgfusepath{stroke}%
\end{pgfscope}%
\begin{pgfscope}%
\pgfpathrectangle{\pgfqpoint{1.250000in}{1.750000in}}{\pgfqpoint{2.279412in}{2.004545in}}%
\pgfusepath{clip}%
\pgfsetbuttcap%
\pgfsetroundjoin%
\pgfsetlinewidth{1.942274pt}%
\definecolor{currentstroke}{rgb}{0.835270,0.886029,0.102646}%
\pgfsetstrokecolor{currentstroke}%
\pgfsetdash{}{0pt}%
\pgfpathmoveto{\pgfqpoint{2.383682in}{2.843542in}}%
\pgfpathlineto{\pgfqpoint{2.334364in}{2.835579in}}%
\pgfusepath{stroke}%
\end{pgfscope}%
\begin{pgfscope}%
\pgfpathrectangle{\pgfqpoint{1.250000in}{1.750000in}}{\pgfqpoint{2.279412in}{2.004545in}}%
\pgfusepath{clip}%
\pgfsetbuttcap%
\pgfsetroundjoin%
\pgfsetlinewidth{1.903922pt}%
\definecolor{currentstroke}{rgb}{0.772852,0.877868,0.131109}%
\pgfsetstrokecolor{currentstroke}%
\pgfsetdash{}{0pt}%
\pgfpathmoveto{\pgfqpoint{2.334364in}{2.835579in}}%
\pgfpathlineto{\pgfqpoint{2.285157in}{2.827103in}}%
\pgfusepath{stroke}%
\end{pgfscope}%
\begin{pgfscope}%
\pgfpathrectangle{\pgfqpoint{1.250000in}{1.750000in}}{\pgfqpoint{2.279412in}{2.004545in}}%
\pgfusepath{clip}%
\pgfsetbuttcap%
\pgfsetroundjoin%
\pgfsetlinewidth{0.323460pt}%
\definecolor{currentstroke}{rgb}{0.271305,0.019942,0.347269}%
\pgfsetstrokecolor{currentstroke}%
\pgfsetdash{}{0pt}%
\pgfpathmoveto{\pgfqpoint{3.210376in}{2.481632in}}%
\pgfpathlineto{\pgfqpoint{3.160243in}{2.482569in}}%
\pgfusepath{stroke}%
\end{pgfscope}%
\begin{pgfscope}%
\pgfpathrectangle{\pgfqpoint{1.250000in}{1.750000in}}{\pgfqpoint{2.279412in}{2.004545in}}%
\pgfusepath{clip}%
\pgfsetbuttcap%
\pgfsetroundjoin%
\pgfsetlinewidth{0.323893pt}%
\definecolor{currentstroke}{rgb}{0.271305,0.019942,0.347269}%
\pgfsetstrokecolor{currentstroke}%
\pgfsetdash{}{0pt}%
\pgfpathmoveto{\pgfqpoint{3.160243in}{2.482569in}}%
\pgfpathlineto{\pgfqpoint{3.110116in}{2.482864in}}%
\pgfusepath{stroke}%
\end{pgfscope}%
\begin{pgfscope}%
\pgfpathrectangle{\pgfqpoint{1.250000in}{1.750000in}}{\pgfqpoint{2.279412in}{2.004545in}}%
\pgfusepath{clip}%
\pgfsetbuttcap%
\pgfsetroundjoin%
\pgfsetlinewidth{0.316869pt}%
\definecolor{currentstroke}{rgb}{0.269944,0.014625,0.341379}%
\pgfsetstrokecolor{currentstroke}%
\pgfsetdash{}{0pt}%
\pgfpathmoveto{\pgfqpoint{3.110116in}{2.482864in}}%
\pgfpathlineto{\pgfqpoint{3.060068in}{2.484435in}}%
\pgfusepath{stroke}%
\end{pgfscope}%
\begin{pgfscope}%
\pgfpathrectangle{\pgfqpoint{1.250000in}{1.750000in}}{\pgfqpoint{2.279412in}{2.004545in}}%
\pgfusepath{clip}%
\pgfsetbuttcap%
\pgfsetroundjoin%
\pgfsetlinewidth{0.327763pt}%
\definecolor{currentstroke}{rgb}{0.271305,0.019942,0.347269}%
\pgfsetstrokecolor{currentstroke}%
\pgfsetdash{}{0pt}%
\pgfpathmoveto{\pgfqpoint{3.060068in}{2.484435in}}%
\pgfpathlineto{\pgfqpoint{3.010019in}{2.486817in}}%
\pgfusepath{stroke}%
\end{pgfscope}%
\begin{pgfscope}%
\pgfpathrectangle{\pgfqpoint{1.250000in}{1.750000in}}{\pgfqpoint{2.279412in}{2.004545in}}%
\pgfusepath{clip}%
\pgfsetbuttcap%
\pgfsetroundjoin%
\pgfsetlinewidth{0.343346pt}%
\definecolor{currentstroke}{rgb}{0.274952,0.037752,0.364543}%
\pgfsetstrokecolor{currentstroke}%
\pgfsetdash{}{0pt}%
\pgfpathmoveto{\pgfqpoint{3.010019in}{2.486817in}}%
\pgfpathlineto{\pgfqpoint{2.959917in}{2.488788in}}%
\pgfusepath{stroke}%
\end{pgfscope}%
\begin{pgfscope}%
\pgfpathrectangle{\pgfqpoint{1.250000in}{1.750000in}}{\pgfqpoint{2.279412in}{2.004545in}}%
\pgfusepath{clip}%
\pgfsetbuttcap%
\pgfsetroundjoin%
\pgfsetlinewidth{0.366440pt}%
\definecolor{currentstroke}{rgb}{0.277941,0.056324,0.381191}%
\pgfsetstrokecolor{currentstroke}%
\pgfsetdash{}{0pt}%
\pgfpathmoveto{\pgfqpoint{2.959917in}{2.488788in}}%
\pgfpathlineto{\pgfqpoint{2.909846in}{2.491249in}}%
\pgfusepath{stroke}%
\end{pgfscope}%
\begin{pgfscope}%
\pgfpathrectangle{\pgfqpoint{1.250000in}{1.750000in}}{\pgfqpoint{2.279412in}{2.004545in}}%
\pgfusepath{clip}%
\pgfsetbuttcap%
\pgfsetroundjoin%
\pgfsetlinewidth{0.403134pt}%
\definecolor{currentstroke}{rgb}{0.281446,0.084320,0.407414}%
\pgfsetstrokecolor{currentstroke}%
\pgfsetdash{}{0pt}%
\pgfpathmoveto{\pgfqpoint{2.909846in}{2.491249in}}%
\pgfpathlineto{\pgfqpoint{2.859777in}{2.493754in}}%
\pgfusepath{stroke}%
\end{pgfscope}%
\begin{pgfscope}%
\pgfpathrectangle{\pgfqpoint{1.250000in}{1.750000in}}{\pgfqpoint{2.279412in}{2.004545in}}%
\pgfusepath{clip}%
\pgfsetbuttcap%
\pgfsetroundjoin%
\pgfsetlinewidth{0.411007pt}%
\definecolor{currentstroke}{rgb}{0.282327,0.094955,0.417331}%
\pgfsetstrokecolor{currentstroke}%
\pgfsetdash{}{0pt}%
\pgfpathmoveto{\pgfqpoint{2.859777in}{2.493754in}}%
\pgfpathlineto{\pgfqpoint{2.809704in}{2.496188in}}%
\pgfusepath{stroke}%
\end{pgfscope}%
\begin{pgfscope}%
\pgfpathrectangle{\pgfqpoint{1.250000in}{1.750000in}}{\pgfqpoint{2.279412in}{2.004545in}}%
\pgfusepath{clip}%
\pgfsetbuttcap%
\pgfsetroundjoin%
\pgfsetlinewidth{0.469112pt}%
\definecolor{currentstroke}{rgb}{0.282884,0.135920,0.453427}%
\pgfsetstrokecolor{currentstroke}%
\pgfsetdash{}{0pt}%
\pgfpathmoveto{\pgfqpoint{2.809704in}{2.496188in}}%
\pgfpathlineto{\pgfqpoint{2.759671in}{2.499205in}}%
\pgfusepath{stroke}%
\end{pgfscope}%
\begin{pgfscope}%
\pgfpathrectangle{\pgfqpoint{1.250000in}{1.750000in}}{\pgfqpoint{2.279412in}{2.004545in}}%
\pgfusepath{clip}%
\pgfsetbuttcap%
\pgfsetroundjoin%
\pgfsetlinewidth{0.502734pt}%
\definecolor{currentstroke}{rgb}{0.280868,0.160771,0.472899}%
\pgfsetstrokecolor{currentstroke}%
\pgfsetdash{}{0pt}%
\pgfpathmoveto{\pgfqpoint{2.759671in}{2.499205in}}%
\pgfpathlineto{\pgfqpoint{2.709703in}{2.502948in}}%
\pgfusepath{stroke}%
\end{pgfscope}%
\begin{pgfscope}%
\pgfpathrectangle{\pgfqpoint{1.250000in}{1.750000in}}{\pgfqpoint{2.279412in}{2.004545in}}%
\pgfusepath{clip}%
\pgfsetbuttcap%
\pgfsetroundjoin%
\pgfsetlinewidth{0.583507pt}%
\definecolor{currentstroke}{rgb}{0.269308,0.218818,0.509577}%
\pgfsetstrokecolor{currentstroke}%
\pgfsetdash{}{0pt}%
\pgfpathmoveto{\pgfqpoint{2.709703in}{2.502948in}}%
\pgfpathlineto{\pgfqpoint{2.659836in}{2.507615in}}%
\pgfusepath{stroke}%
\end{pgfscope}%
\begin{pgfscope}%
\pgfpathrectangle{\pgfqpoint{1.250000in}{1.750000in}}{\pgfqpoint{2.279412in}{2.004545in}}%
\pgfusepath{clip}%
\pgfsetbuttcap%
\pgfsetroundjoin%
\pgfsetlinewidth{0.606711pt}%
\definecolor{currentstroke}{rgb}{0.265145,0.232956,0.516599}%
\pgfsetstrokecolor{currentstroke}%
\pgfsetdash{}{0pt}%
\pgfpathmoveto{\pgfqpoint{2.659836in}{2.507615in}}%
\pgfpathlineto{\pgfqpoint{2.610154in}{2.513581in}}%
\pgfusepath{stroke}%
\end{pgfscope}%
\begin{pgfscope}%
\pgfpathrectangle{\pgfqpoint{1.250000in}{1.750000in}}{\pgfqpoint{2.279412in}{2.004545in}}%
\pgfusepath{clip}%
\pgfsetbuttcap%
\pgfsetroundjoin%
\pgfsetlinewidth{0.721397pt}%
\definecolor{currentstroke}{rgb}{0.235526,0.309527,0.542944}%
\pgfsetstrokecolor{currentstroke}%
\pgfsetdash{}{0pt}%
\pgfpathmoveto{\pgfqpoint{2.610154in}{2.513581in}}%
\pgfpathlineto{\pgfqpoint{2.560697in}{2.520874in}}%
\pgfusepath{stroke}%
\end{pgfscope}%
\begin{pgfscope}%
\pgfpathrectangle{\pgfqpoint{1.250000in}{1.750000in}}{\pgfqpoint{2.279412in}{2.004545in}}%
\pgfusepath{clip}%
\pgfsetbuttcap%
\pgfsetroundjoin%
\pgfsetlinewidth{0.886117pt}%
\definecolor{currentstroke}{rgb}{0.190631,0.407061,0.556089}%
\pgfsetstrokecolor{currentstroke}%
\pgfsetdash{}{0pt}%
\pgfpathmoveto{\pgfqpoint{2.560697in}{2.520874in}}%
\pgfpathlineto{\pgfqpoint{2.511536in}{2.529545in}}%
\pgfusepath{stroke}%
\end{pgfscope}%
\begin{pgfscope}%
\pgfpathrectangle{\pgfqpoint{1.250000in}{1.750000in}}{\pgfqpoint{2.279412in}{2.004545in}}%
\pgfusepath{clip}%
\pgfsetbuttcap%
\pgfsetroundjoin%
\pgfsetlinewidth{1.084088pt}%
\definecolor{currentstroke}{rgb}{0.146180,0.515413,0.556823}%
\pgfsetstrokecolor{currentstroke}%
\pgfsetdash{}{0pt}%
\pgfpathmoveto{\pgfqpoint{2.511536in}{2.529545in}}%
\pgfpathlineto{\pgfqpoint{2.462759in}{2.539768in}}%
\pgfusepath{stroke}%
\end{pgfscope}%
\begin{pgfscope}%
\pgfpathrectangle{\pgfqpoint{1.250000in}{1.750000in}}{\pgfqpoint{2.279412in}{2.004545in}}%
\pgfusepath{clip}%
\pgfsetbuttcap%
\pgfsetroundjoin%
\pgfsetlinewidth{1.080757pt}%
\definecolor{currentstroke}{rgb}{0.147607,0.511733,0.557049}%
\pgfsetstrokecolor{currentstroke}%
\pgfsetdash{}{0pt}%
\pgfpathmoveto{\pgfqpoint{2.462759in}{2.539768in}}%
\pgfpathlineto{\pgfqpoint{2.414446in}{2.551558in}}%
\pgfusepath{stroke}%
\end{pgfscope}%
\begin{pgfscope}%
\pgfpathrectangle{\pgfqpoint{1.250000in}{1.750000in}}{\pgfqpoint{2.279412in}{2.004545in}}%
\pgfusepath{clip}%
\pgfsetbuttcap%
\pgfsetroundjoin%
\pgfsetlinewidth{1.096219pt}%
\definecolor{currentstroke}{rgb}{0.144759,0.519093,0.556572}%
\pgfsetstrokecolor{currentstroke}%
\pgfsetdash{}{0pt}%
\pgfpathmoveto{\pgfqpoint{2.414446in}{2.551558in}}%
\pgfpathlineto{\pgfqpoint{2.366556in}{2.564631in}}%
\pgfusepath{stroke}%
\end{pgfscope}%
\begin{pgfscope}%
\pgfpathrectangle{\pgfqpoint{1.250000in}{1.750000in}}{\pgfqpoint{2.279412in}{2.004545in}}%
\pgfusepath{clip}%
\pgfsetbuttcap%
\pgfsetroundjoin%
\pgfsetlinewidth{1.313659pt}%
\definecolor{currentstroke}{rgb}{0.123444,0.636809,0.528763}%
\pgfsetstrokecolor{currentstroke}%
\pgfsetdash{}{0pt}%
\pgfpathmoveto{\pgfqpoint{2.366556in}{2.564631in}}%
\pgfpathlineto{\pgfqpoint{2.319108in}{2.578863in}}%
\pgfusepath{stroke}%
\end{pgfscope}%
\begin{pgfscope}%
\pgfpathrectangle{\pgfqpoint{1.250000in}{1.750000in}}{\pgfqpoint{2.279412in}{2.004545in}}%
\pgfusepath{clip}%
\pgfsetbuttcap%
\pgfsetroundjoin%
\pgfsetlinewidth{1.596481pt}%
\definecolor{currentstroke}{rgb}{0.335885,0.777018,0.402049}%
\pgfsetstrokecolor{currentstroke}%
\pgfsetdash{}{0pt}%
\pgfpathmoveto{\pgfqpoint{2.319108in}{2.578863in}}%
\pgfpathlineto{\pgfqpoint{2.272104in}{2.594209in}}%
\pgfusepath{stroke}%
\end{pgfscope}%
\begin{pgfscope}%
\pgfpathrectangle{\pgfqpoint{1.250000in}{1.750000in}}{\pgfqpoint{2.279412in}{2.004545in}}%
\pgfusepath{clip}%
\pgfsetbuttcap%
\pgfsetroundjoin%
\pgfsetlinewidth{1.810939pt}%
\definecolor{currentstroke}{rgb}{0.636902,0.856542,0.216620}%
\pgfsetstrokecolor{currentstroke}%
\pgfsetdash{}{0pt}%
\pgfpathmoveto{\pgfqpoint{2.272104in}{2.594209in}}%
\pgfpathlineto{\pgfqpoint{2.225444in}{2.610351in}}%
\pgfusepath{stroke}%
\end{pgfscope}%
\begin{pgfscope}%
\pgfpathrectangle{\pgfqpoint{1.250000in}{1.750000in}}{\pgfqpoint{2.279412in}{2.004545in}}%
\pgfusepath{clip}%
\pgfsetbuttcap%
\pgfsetroundjoin%
\pgfsetlinewidth{1.764458pt}%
\definecolor{currentstroke}{rgb}{0.565498,0.842430,0.262877}%
\pgfsetstrokecolor{currentstroke}%
\pgfsetdash{}{0pt}%
\pgfpathmoveto{\pgfqpoint{2.225444in}{2.610351in}}%
\pgfpathlineto{\pgfqpoint{2.179008in}{2.626947in}}%
\pgfusepath{stroke}%
\end{pgfscope}%
\begin{pgfscope}%
\pgfpathrectangle{\pgfqpoint{1.250000in}{1.750000in}}{\pgfqpoint{2.279412in}{2.004545in}}%
\pgfusepath{clip}%
\pgfsetbuttcap%
\pgfsetroundjoin%
\pgfsetlinewidth{2.092217pt}%
\definecolor{currentstroke}{rgb}{0.993248,0.906157,0.143936}%
\pgfsetstrokecolor{currentstroke}%
\pgfsetdash{}{0pt}%
\pgfpathmoveto{\pgfqpoint{2.179008in}{2.626947in}}%
\pgfpathlineto{\pgfqpoint{2.132793in}{2.643976in}}%
\pgfusepath{stroke}%
\end{pgfscope}%
\begin{pgfscope}%
\pgfpathrectangle{\pgfqpoint{1.250000in}{1.750000in}}{\pgfqpoint{2.279412in}{2.004545in}}%
\pgfusepath{clip}%
\pgfsetbuttcap%
\pgfsetroundjoin%
\pgfsetlinewidth{1.856792pt}%
\definecolor{currentstroke}{rgb}{0.699415,0.867117,0.175971}%
\pgfsetstrokecolor{currentstroke}%
\pgfsetdash{}{0pt}%
\pgfpathmoveto{\pgfqpoint{2.132793in}{2.643976in}}%
\pgfpathlineto{\pgfqpoint{2.086711in}{2.661278in}}%
\pgfusepath{stroke}%
\end{pgfscope}%
\begin{pgfscope}%
\pgfpathrectangle{\pgfqpoint{1.250000in}{1.750000in}}{\pgfqpoint{2.279412in}{2.004545in}}%
\pgfusepath{clip}%
\pgfsetbuttcap%
\pgfsetroundjoin%
\pgfsetlinewidth{1.838868pt}%
\definecolor{currentstroke}{rgb}{0.678489,0.863742,0.189503}%
\pgfsetstrokecolor{currentstroke}%
\pgfsetdash{}{0pt}%
\pgfpathmoveto{\pgfqpoint{2.086711in}{2.661278in}}%
\pgfpathlineto{\pgfqpoint{2.040886in}{2.679023in}}%
\pgfusepath{stroke}%
\end{pgfscope}%
\begin{pgfscope}%
\pgfpathrectangle{\pgfqpoint{1.250000in}{1.750000in}}{\pgfqpoint{2.279412in}{2.004545in}}%
\pgfusepath{clip}%
\pgfsetbuttcap%
\pgfsetroundjoin%
\pgfsetlinewidth{1.654215pt}%
\definecolor{currentstroke}{rgb}{0.412913,0.803041,0.357269}%
\pgfsetstrokecolor{currentstroke}%
\pgfsetdash{}{0pt}%
\pgfpathmoveto{\pgfqpoint{2.040886in}{2.679023in}}%
\pgfpathlineto{\pgfqpoint{1.994925in}{2.696370in}}%
\pgfusepath{stroke}%
\end{pgfscope}%
\begin{pgfscope}%
\pgfpathrectangle{\pgfqpoint{1.250000in}{1.750000in}}{\pgfqpoint{2.279412in}{2.004545in}}%
\pgfusepath{clip}%
\pgfsetbuttcap%
\pgfsetroundjoin%
\pgfsetlinewidth{0.313659pt}%
\definecolor{currentstroke}{rgb}{0.268510,0.009605,0.335427}%
\pgfsetstrokecolor{currentstroke}%
\pgfsetdash{}{0pt}%
\pgfpathmoveto{\pgfqpoint{3.210376in}{2.526739in}}%
\pgfpathlineto{\pgfqpoint{3.160384in}{2.527775in}}%
\pgfusepath{stroke}%
\end{pgfscope}%
\begin{pgfscope}%
\pgfpathrectangle{\pgfqpoint{1.250000in}{1.750000in}}{\pgfqpoint{2.279412in}{2.004545in}}%
\pgfusepath{clip}%
\pgfsetbuttcap%
\pgfsetroundjoin%
\pgfsetlinewidth{0.332586pt}%
\definecolor{currentstroke}{rgb}{0.272594,0.025563,0.353093}%
\pgfsetstrokecolor{currentstroke}%
\pgfsetdash{}{0pt}%
\pgfpathmoveto{\pgfqpoint{3.160384in}{2.527775in}}%
\pgfpathlineto{\pgfqpoint{3.110272in}{2.529191in}}%
\pgfusepath{stroke}%
\end{pgfscope}%
\begin{pgfscope}%
\pgfpathrectangle{\pgfqpoint{1.250000in}{1.750000in}}{\pgfqpoint{2.279412in}{2.004545in}}%
\pgfusepath{clip}%
\pgfsetbuttcap%
\pgfsetroundjoin%
\pgfsetlinewidth{0.329216pt}%
\definecolor{currentstroke}{rgb}{0.272594,0.025563,0.353093}%
\pgfsetstrokecolor{currentstroke}%
\pgfsetdash{}{0pt}%
\pgfpathmoveto{\pgfqpoint{3.110272in}{2.529191in}}%
\pgfpathlineto{\pgfqpoint{3.060148in}{2.529986in}}%
\pgfusepath{stroke}%
\end{pgfscope}%
\begin{pgfscope}%
\pgfpathrectangle{\pgfqpoint{1.250000in}{1.750000in}}{\pgfqpoint{2.279412in}{2.004545in}}%
\pgfusepath{clip}%
\pgfsetbuttcap%
\pgfsetroundjoin%
\pgfsetlinewidth{0.331598pt}%
\definecolor{currentstroke}{rgb}{0.272594,0.025563,0.353093}%
\pgfsetstrokecolor{currentstroke}%
\pgfsetdash{}{0pt}%
\pgfpathmoveto{\pgfqpoint{3.060148in}{2.529986in}}%
\pgfpathlineto{\pgfqpoint{3.010037in}{2.531289in}}%
\pgfusepath{stroke}%
\end{pgfscope}%
\begin{pgfscope}%
\pgfpathrectangle{\pgfqpoint{1.250000in}{1.750000in}}{\pgfqpoint{2.279412in}{2.004545in}}%
\pgfusepath{clip}%
\pgfsetbuttcap%
\pgfsetroundjoin%
\pgfsetlinewidth{0.358347pt}%
\definecolor{currentstroke}{rgb}{0.277018,0.050344,0.375715}%
\pgfsetstrokecolor{currentstroke}%
\pgfsetdash{}{0pt}%
\pgfpathmoveto{\pgfqpoint{3.010037in}{2.531289in}}%
\pgfpathlineto{\pgfqpoint{2.959943in}{2.533348in}}%
\pgfusepath{stroke}%
\end{pgfscope}%
\begin{pgfscope}%
\pgfpathrectangle{\pgfqpoint{1.250000in}{1.750000in}}{\pgfqpoint{2.279412in}{2.004545in}}%
\pgfusepath{clip}%
\pgfsetbuttcap%
\pgfsetroundjoin%
\pgfsetlinewidth{0.369778pt}%
\definecolor{currentstroke}{rgb}{0.278791,0.062145,0.386592}%
\pgfsetstrokecolor{currentstroke}%
\pgfsetdash{}{0pt}%
\pgfpathmoveto{\pgfqpoint{2.959943in}{2.533348in}}%
\pgfpathlineto{\pgfqpoint{2.909831in}{2.535048in}}%
\pgfusepath{stroke}%
\end{pgfscope}%
\begin{pgfscope}%
\pgfpathrectangle{\pgfqpoint{1.250000in}{1.750000in}}{\pgfqpoint{2.279412in}{2.004545in}}%
\pgfusepath{clip}%
\pgfsetbuttcap%
\pgfsetroundjoin%
\pgfsetlinewidth{0.399920pt}%
\definecolor{currentstroke}{rgb}{0.281446,0.084320,0.407414}%
\pgfsetstrokecolor{currentstroke}%
\pgfsetdash{}{0pt}%
\pgfpathmoveto{\pgfqpoint{2.909831in}{2.535048in}}%
\pgfpathlineto{\pgfqpoint{2.859707in}{2.536486in}}%
\pgfusepath{stroke}%
\end{pgfscope}%
\begin{pgfscope}%
\pgfpathrectangle{\pgfqpoint{1.250000in}{1.750000in}}{\pgfqpoint{2.279412in}{2.004545in}}%
\pgfusepath{clip}%
\pgfsetbuttcap%
\pgfsetroundjoin%
\pgfsetlinewidth{0.441067pt}%
\definecolor{currentstroke}{rgb}{0.283197,0.115680,0.436115}%
\pgfsetstrokecolor{currentstroke}%
\pgfsetdash{}{0pt}%
\pgfpathmoveto{\pgfqpoint{2.859707in}{2.536486in}}%
\pgfpathlineto{\pgfqpoint{2.809616in}{2.538570in}}%
\pgfusepath{stroke}%
\end{pgfscope}%
\begin{pgfscope}%
\pgfpathrectangle{\pgfqpoint{1.250000in}{1.750000in}}{\pgfqpoint{2.279412in}{2.004545in}}%
\pgfusepath{clip}%
\pgfsetbuttcap%
\pgfsetroundjoin%
\pgfsetlinewidth{0.514578pt}%
\definecolor{currentstroke}{rgb}{0.279574,0.170599,0.479997}%
\pgfsetstrokecolor{currentstroke}%
\pgfsetdash{}{0pt}%
\pgfpathmoveto{\pgfqpoint{2.809616in}{2.538570in}}%
\pgfpathlineto{\pgfqpoint{2.759558in}{2.541227in}}%
\pgfusepath{stroke}%
\end{pgfscope}%
\begin{pgfscope}%
\pgfpathrectangle{\pgfqpoint{1.250000in}{1.750000in}}{\pgfqpoint{2.279412in}{2.004545in}}%
\pgfusepath{clip}%
\pgfsetbuttcap%
\pgfsetroundjoin%
\pgfsetlinewidth{0.543635pt}%
\definecolor{currentstroke}{rgb}{0.276194,0.190074,0.493001}%
\pgfsetstrokecolor{currentstroke}%
\pgfsetdash{}{0pt}%
\pgfpathmoveto{\pgfqpoint{2.759558in}{2.541227in}}%
\pgfpathlineto{\pgfqpoint{2.709550in}{2.544522in}}%
\pgfusepath{stroke}%
\end{pgfscope}%
\begin{pgfscope}%
\pgfpathrectangle{\pgfqpoint{1.250000in}{1.750000in}}{\pgfqpoint{2.279412in}{2.004545in}}%
\pgfusepath{clip}%
\pgfsetbuttcap%
\pgfsetroundjoin%
\pgfsetlinewidth{0.311252pt}%
\definecolor{currentstroke}{rgb}{0.268510,0.009605,0.335427}%
\pgfsetstrokecolor{currentstroke}%
\pgfsetdash{}{0pt}%
\pgfpathmoveto{\pgfqpoint{3.210376in}{2.616952in}}%
\pgfpathlineto{\pgfqpoint{3.160280in}{2.617333in}}%
\pgfusepath{stroke}%
\end{pgfscope}%
\begin{pgfscope}%
\pgfpathrectangle{\pgfqpoint{1.250000in}{1.750000in}}{\pgfqpoint{2.279412in}{2.004545in}}%
\pgfusepath{clip}%
\pgfsetbuttcap%
\pgfsetroundjoin%
\pgfsetlinewidth{0.326309pt}%
\definecolor{currentstroke}{rgb}{0.271305,0.019942,0.347269}%
\pgfsetstrokecolor{currentstroke}%
\pgfsetdash{}{0pt}%
\pgfpathmoveto{\pgfqpoint{3.160280in}{2.617333in}}%
\pgfpathlineto{\pgfqpoint{3.110186in}{2.618890in}}%
\pgfusepath{stroke}%
\end{pgfscope}%
\begin{pgfscope}%
\pgfpathrectangle{\pgfqpoint{1.250000in}{1.750000in}}{\pgfqpoint{2.279412in}{2.004545in}}%
\pgfusepath{clip}%
\pgfsetbuttcap%
\pgfsetroundjoin%
\pgfsetlinewidth{0.330197pt}%
\definecolor{currentstroke}{rgb}{0.272594,0.025563,0.353093}%
\pgfsetstrokecolor{currentstroke}%
\pgfsetdash{}{0pt}%
\pgfpathmoveto{\pgfqpoint{3.110186in}{2.618890in}}%
\pgfpathlineto{\pgfqpoint{3.060060in}{2.619629in}}%
\pgfusepath{stroke}%
\end{pgfscope}%
\begin{pgfscope}%
\pgfpathrectangle{\pgfqpoint{1.250000in}{1.750000in}}{\pgfqpoint{2.279412in}{2.004545in}}%
\pgfusepath{clip}%
\pgfsetbuttcap%
\pgfsetroundjoin%
\pgfsetlinewidth{0.343220pt}%
\definecolor{currentstroke}{rgb}{0.274952,0.037752,0.364543}%
\pgfsetstrokecolor{currentstroke}%
\pgfsetdash{}{0pt}%
\pgfpathmoveto{\pgfqpoint{3.060060in}{2.619629in}}%
\pgfpathlineto{\pgfqpoint{3.009924in}{2.620429in}}%
\pgfusepath{stroke}%
\end{pgfscope}%
\begin{pgfscope}%
\pgfpathrectangle{\pgfqpoint{1.250000in}{1.750000in}}{\pgfqpoint{2.279412in}{2.004545in}}%
\pgfusepath{clip}%
\pgfsetbuttcap%
\pgfsetroundjoin%
\pgfsetlinewidth{0.365675pt}%
\definecolor{currentstroke}{rgb}{0.277941,0.056324,0.381191}%
\pgfsetstrokecolor{currentstroke}%
\pgfsetdash{}{0pt}%
\pgfpathmoveto{\pgfqpoint{3.009924in}{2.620429in}}%
\pgfpathlineto{\pgfqpoint{2.959788in}{2.621336in}}%
\pgfusepath{stroke}%
\end{pgfscope}%
\begin{pgfscope}%
\pgfpathrectangle{\pgfqpoint{1.250000in}{1.750000in}}{\pgfqpoint{2.279412in}{2.004545in}}%
\pgfusepath{clip}%
\pgfsetbuttcap%
\pgfsetroundjoin%
\pgfsetlinewidth{0.386221pt}%
\definecolor{currentstroke}{rgb}{0.280267,0.073417,0.397163}%
\pgfsetstrokecolor{currentstroke}%
\pgfsetdash{}{0pt}%
\pgfpathmoveto{\pgfqpoint{2.959788in}{2.621336in}}%
\pgfpathlineto{\pgfqpoint{2.909646in}{2.622129in}}%
\pgfusepath{stroke}%
\end{pgfscope}%
\begin{pgfscope}%
\pgfpathrectangle{\pgfqpoint{1.250000in}{1.750000in}}{\pgfqpoint{2.279412in}{2.004545in}}%
\pgfusepath{clip}%
\pgfsetbuttcap%
\pgfsetroundjoin%
\pgfsetlinewidth{0.421665pt}%
\definecolor{currentstroke}{rgb}{0.282656,0.100196,0.422160}%
\pgfsetstrokecolor{currentstroke}%
\pgfsetdash{}{0pt}%
\pgfpathmoveto{\pgfqpoint{2.909646in}{2.622129in}}%
\pgfpathlineto{\pgfqpoint{2.859507in}{2.623084in}}%
\pgfusepath{stroke}%
\end{pgfscope}%
\begin{pgfscope}%
\pgfpathrectangle{\pgfqpoint{1.250000in}{1.750000in}}{\pgfqpoint{2.279412in}{2.004545in}}%
\pgfusepath{clip}%
\pgfsetbuttcap%
\pgfsetroundjoin%
\pgfsetlinewidth{0.491170pt}%
\definecolor{currentstroke}{rgb}{0.281887,0.150881,0.465405}%
\pgfsetstrokecolor{currentstroke}%
\pgfsetdash{}{0pt}%
\pgfpathmoveto{\pgfqpoint{2.859507in}{2.623084in}}%
\pgfpathlineto{\pgfqpoint{2.809370in}{2.624132in}}%
\pgfusepath{stroke}%
\end{pgfscope}%
\begin{pgfscope}%
\pgfpathrectangle{\pgfqpoint{1.250000in}{1.750000in}}{\pgfqpoint{2.279412in}{2.004545in}}%
\pgfusepath{clip}%
\pgfsetbuttcap%
\pgfsetroundjoin%
\pgfsetlinewidth{0.585347pt}%
\definecolor{currentstroke}{rgb}{0.269308,0.218818,0.509577}%
\pgfsetstrokecolor{currentstroke}%
\pgfsetdash{}{0pt}%
\pgfpathmoveto{\pgfqpoint{2.809370in}{2.624132in}}%
\pgfpathlineto{\pgfqpoint{2.759243in}{2.625496in}}%
\pgfusepath{stroke}%
\end{pgfscope}%
\begin{pgfscope}%
\pgfpathrectangle{\pgfqpoint{1.250000in}{1.750000in}}{\pgfqpoint{2.279412in}{2.004545in}}%
\pgfusepath{clip}%
\pgfsetbuttcap%
\pgfsetroundjoin%
\pgfsetlinewidth{0.724120pt}%
\definecolor{currentstroke}{rgb}{0.235526,0.309527,0.542944}%
\pgfsetstrokecolor{currentstroke}%
\pgfsetdash{}{0pt}%
\pgfpathmoveto{\pgfqpoint{2.759243in}{2.625496in}}%
\pgfpathlineto{\pgfqpoint{2.709135in}{2.627294in}}%
\pgfusepath{stroke}%
\end{pgfscope}%
\begin{pgfscope}%
\pgfpathrectangle{\pgfqpoint{1.250000in}{1.750000in}}{\pgfqpoint{2.279412in}{2.004545in}}%
\pgfusepath{clip}%
\pgfsetbuttcap%
\pgfsetroundjoin%
\pgfsetlinewidth{0.918855pt}%
\definecolor{currentstroke}{rgb}{0.182256,0.426184,0.557120}%
\pgfsetstrokecolor{currentstroke}%
\pgfsetdash{}{0pt}%
\pgfpathmoveto{\pgfqpoint{2.709135in}{2.627294in}}%
\pgfpathlineto{\pgfqpoint{2.659059in}{2.629694in}}%
\pgfusepath{stroke}%
\end{pgfscope}%
\begin{pgfscope}%
\pgfpathrectangle{\pgfqpoint{1.250000in}{1.750000in}}{\pgfqpoint{2.279412in}{2.004545in}}%
\pgfusepath{clip}%
\pgfsetbuttcap%
\pgfsetroundjoin%
\pgfsetlinewidth{1.120220pt}%
\definecolor{currentstroke}{rgb}{0.139147,0.533812,0.555298}%
\pgfsetstrokecolor{currentstroke}%
\pgfsetdash{}{0pt}%
\pgfpathmoveto{\pgfqpoint{2.659059in}{2.629694in}}%
\pgfpathlineto{\pgfqpoint{2.609034in}{2.632793in}}%
\pgfusepath{stroke}%
\end{pgfscope}%
\begin{pgfscope}%
\pgfpathrectangle{\pgfqpoint{1.250000in}{1.750000in}}{\pgfqpoint{2.279412in}{2.004545in}}%
\pgfusepath{clip}%
\pgfsetbuttcap%
\pgfsetroundjoin%
\pgfsetlinewidth{1.297197pt}%
\definecolor{currentstroke}{rgb}{0.121380,0.629492,0.531973}%
\pgfsetstrokecolor{currentstroke}%
\pgfsetdash{}{0pt}%
\pgfpathmoveto{\pgfqpoint{2.609034in}{2.632793in}}%
\pgfpathlineto{\pgfqpoint{2.559069in}{2.636573in}}%
\pgfusepath{stroke}%
\end{pgfscope}%
\begin{pgfscope}%
\pgfpathrectangle{\pgfqpoint{1.250000in}{1.750000in}}{\pgfqpoint{2.279412in}{2.004545in}}%
\pgfusepath{clip}%
\pgfsetbuttcap%
\pgfsetroundjoin%
\pgfsetlinewidth{1.514642pt}%
\definecolor{currentstroke}{rgb}{0.246070,0.738910,0.452024}%
\pgfsetstrokecolor{currentstroke}%
\pgfsetdash{}{0pt}%
\pgfpathmoveto{\pgfqpoint{2.559069in}{2.636573in}}%
\pgfpathlineto{\pgfqpoint{2.509170in}{2.640966in}}%
\pgfusepath{stroke}%
\end{pgfscope}%
\begin{pgfscope}%
\pgfpathrectangle{\pgfqpoint{1.250000in}{1.750000in}}{\pgfqpoint{2.279412in}{2.004545in}}%
\pgfusepath{clip}%
\pgfsetbuttcap%
\pgfsetroundjoin%
\pgfsetlinewidth{1.634007pt}%
\definecolor{currentstroke}{rgb}{0.386433,0.794644,0.372886}%
\pgfsetstrokecolor{currentstroke}%
\pgfsetdash{}{0pt}%
\pgfpathmoveto{\pgfqpoint{2.509170in}{2.640966in}}%
\pgfpathlineto{\pgfqpoint{2.459354in}{2.646020in}}%
\pgfusepath{stroke}%
\end{pgfscope}%
\begin{pgfscope}%
\pgfpathrectangle{\pgfqpoint{1.250000in}{1.750000in}}{\pgfqpoint{2.279412in}{2.004545in}}%
\pgfusepath{clip}%
\pgfsetbuttcap%
\pgfsetroundjoin%
\pgfsetlinewidth{1.875591pt}%
\definecolor{currentstroke}{rgb}{0.730889,0.871916,0.156029}%
\pgfsetstrokecolor{currentstroke}%
\pgfsetdash{}{0pt}%
\pgfpathmoveto{\pgfqpoint{2.459354in}{2.646020in}}%
\pgfpathlineto{\pgfqpoint{2.409635in}{2.651764in}}%
\pgfusepath{stroke}%
\end{pgfscope}%
\begin{pgfscope}%
\pgfpathrectangle{\pgfqpoint{1.250000in}{1.750000in}}{\pgfqpoint{2.279412in}{2.004545in}}%
\pgfusepath{clip}%
\pgfsetbuttcap%
\pgfsetroundjoin%
\pgfsetlinewidth{2.090758pt}%
\definecolor{currentstroke}{rgb}{0.993248,0.906157,0.143936}%
\pgfsetstrokecolor{currentstroke}%
\pgfsetdash{}{0pt}%
\pgfpathmoveto{\pgfqpoint{2.409635in}{2.651764in}}%
\pgfpathlineto{\pgfqpoint{2.360009in}{2.658104in}}%
\pgfusepath{stroke}%
\end{pgfscope}%
\begin{pgfscope}%
\pgfpathrectangle{\pgfqpoint{1.250000in}{1.750000in}}{\pgfqpoint{2.279412in}{2.004545in}}%
\pgfusepath{clip}%
\pgfsetbuttcap%
\pgfsetroundjoin%
\pgfsetlinewidth{2.079396pt}%
\definecolor{currentstroke}{rgb}{0.993248,0.906157,0.143936}%
\pgfsetstrokecolor{currentstroke}%
\pgfsetdash{}{0pt}%
\pgfpathmoveto{\pgfqpoint{2.360009in}{2.658104in}}%
\pgfpathlineto{\pgfqpoint{2.310474in}{2.664953in}}%
\pgfusepath{stroke}%
\end{pgfscope}%
\begin{pgfscope}%
\pgfpathrectangle{\pgfqpoint{1.250000in}{1.750000in}}{\pgfqpoint{2.279412in}{2.004545in}}%
\pgfusepath{clip}%
\pgfsetbuttcap%
\pgfsetroundjoin%
\pgfsetlinewidth{2.165650pt}%
\definecolor{currentstroke}{rgb}{0.993248,0.906157,0.143936}%
\pgfsetstrokecolor{currentstroke}%
\pgfsetdash{}{0pt}%
\pgfpathmoveto{\pgfqpoint{2.310474in}{2.664953in}}%
\pgfpathlineto{\pgfqpoint{2.261065in}{2.672481in}}%
\pgfusepath{stroke}%
\end{pgfscope}%
\begin{pgfscope}%
\pgfpathrectangle{\pgfqpoint{1.250000in}{1.750000in}}{\pgfqpoint{2.279412in}{2.004545in}}%
\pgfusepath{clip}%
\pgfsetbuttcap%
\pgfsetroundjoin%
\pgfsetlinewidth{2.104936pt}%
\definecolor{currentstroke}{rgb}{0.993248,0.906157,0.143936}%
\pgfsetstrokecolor{currentstroke}%
\pgfsetdash{}{0pt}%
\pgfpathmoveto{\pgfqpoint{2.261065in}{2.672481in}}%
\pgfpathlineto{\pgfqpoint{2.211803in}{2.680701in}}%
\pgfusepath{stroke}%
\end{pgfscope}%
\begin{pgfscope}%
\pgfpathrectangle{\pgfqpoint{1.250000in}{1.750000in}}{\pgfqpoint{2.279412in}{2.004545in}}%
\pgfusepath{clip}%
\pgfsetbuttcap%
\pgfsetroundjoin%
\pgfsetlinewidth{2.026138pt}%
\definecolor{currentstroke}{rgb}{0.955300,0.901065,0.118128}%
\pgfsetstrokecolor{currentstroke}%
\pgfsetdash{}{0pt}%
\pgfpathmoveto{\pgfqpoint{2.211803in}{2.680701in}}%
\pgfpathlineto{\pgfqpoint{2.162596in}{2.689110in}}%
\pgfusepath{stroke}%
\end{pgfscope}%
\begin{pgfscope}%
\pgfpathrectangle{\pgfqpoint{1.250000in}{1.750000in}}{\pgfqpoint{2.279412in}{2.004545in}}%
\pgfusepath{clip}%
\pgfsetbuttcap%
\pgfsetroundjoin%
\pgfsetlinewidth{1.998195pt}%
\definecolor{currentstroke}{rgb}{0.916242,0.896091,0.100717}%
\pgfsetstrokecolor{currentstroke}%
\pgfsetdash{}{0pt}%
\pgfpathmoveto{\pgfqpoint{2.162596in}{2.689110in}}%
\pgfpathlineto{\pgfqpoint{2.113504in}{2.697959in}}%
\pgfusepath{stroke}%
\end{pgfscope}%
\begin{pgfscope}%
\pgfpathrectangle{\pgfqpoint{1.250000in}{1.750000in}}{\pgfqpoint{2.279412in}{2.004545in}}%
\pgfusepath{clip}%
\pgfsetbuttcap%
\pgfsetroundjoin%
\pgfsetlinewidth{1.688900pt}%
\definecolor{currentstroke}{rgb}{0.458674,0.816363,0.329727}%
\pgfsetstrokecolor{currentstroke}%
\pgfsetdash{}{0pt}%
\pgfpathmoveto{\pgfqpoint{2.113504in}{2.697959in}}%
\pgfpathlineto{\pgfqpoint{2.064567in}{2.707360in}}%
\pgfusepath{stroke}%
\end{pgfscope}%
\begin{pgfscope}%
\pgfpathrectangle{\pgfqpoint{1.250000in}{1.750000in}}{\pgfqpoint{2.279412in}{2.004545in}}%
\pgfusepath{clip}%
\pgfsetbuttcap%
\pgfsetroundjoin%
\pgfsetlinewidth{1.431389pt}%
\definecolor{currentstroke}{rgb}{0.175707,0.697900,0.491033}%
\pgfsetstrokecolor{currentstroke}%
\pgfsetdash{}{0pt}%
\pgfpathmoveto{\pgfqpoint{2.064567in}{2.707360in}}%
\pgfpathlineto{\pgfqpoint{2.015589in}{2.716452in}}%
\pgfusepath{stroke}%
\end{pgfscope}%
\begin{pgfscope}%
\pgfpathrectangle{\pgfqpoint{1.250000in}{1.750000in}}{\pgfqpoint{2.279412in}{2.004545in}}%
\pgfusepath{clip}%
\pgfsetbuttcap%
\pgfsetroundjoin%
\pgfsetlinewidth{0.318818pt}%
\definecolor{currentstroke}{rgb}{0.269944,0.014625,0.341379}%
\pgfsetstrokecolor{currentstroke}%
\pgfsetdash{}{0pt}%
\pgfpathmoveto{\pgfqpoint{3.159084in}{2.346312in}}%
\pgfpathlineto{\pgfqpoint{3.109474in}{2.348675in}}%
\pgfusepath{stroke}%
\end{pgfscope}%
\begin{pgfscope}%
\pgfpathrectangle{\pgfqpoint{1.250000in}{1.750000in}}{\pgfqpoint{2.279412in}{2.004545in}}%
\pgfusepath{clip}%
\pgfsetbuttcap%
\pgfsetroundjoin%
\pgfsetlinewidth{0.313336pt}%
\definecolor{currentstroke}{rgb}{0.268510,0.009605,0.335427}%
\pgfsetstrokecolor{currentstroke}%
\pgfsetdash{}{0pt}%
\pgfpathmoveto{\pgfqpoint{3.109474in}{2.348675in}}%
\pgfpathlineto{\pgfqpoint{3.059836in}{2.351203in}}%
\pgfusepath{stroke}%
\end{pgfscope}%
\begin{pgfscope}%
\pgfpathrectangle{\pgfqpoint{1.250000in}{1.750000in}}{\pgfqpoint{2.279412in}{2.004545in}}%
\pgfusepath{clip}%
\pgfsetbuttcap%
\pgfsetroundjoin%
\pgfsetlinewidth{0.334472pt}%
\definecolor{currentstroke}{rgb}{0.272594,0.025563,0.353093}%
\pgfsetstrokecolor{currentstroke}%
\pgfsetdash{}{0pt}%
\pgfpathmoveto{\pgfqpoint{3.059836in}{2.351203in}}%
\pgfpathlineto{\pgfqpoint{3.009768in}{2.353682in}}%
\pgfusepath{stroke}%
\end{pgfscope}%
\begin{pgfscope}%
\pgfpathrectangle{\pgfqpoint{1.250000in}{1.750000in}}{\pgfqpoint{2.279412in}{2.004545in}}%
\pgfusepath{clip}%
\pgfsetbuttcap%
\pgfsetroundjoin%
\pgfsetlinewidth{0.334261pt}%
\definecolor{currentstroke}{rgb}{0.272594,0.025563,0.353093}%
\pgfsetstrokecolor{currentstroke}%
\pgfsetdash{}{0pt}%
\pgfpathmoveto{\pgfqpoint{3.009768in}{2.353682in}}%
\pgfpathlineto{\pgfqpoint{2.959679in}{2.355688in}}%
\pgfusepath{stroke}%
\end{pgfscope}%
\begin{pgfscope}%
\pgfpathrectangle{\pgfqpoint{1.250000in}{1.750000in}}{\pgfqpoint{2.279412in}{2.004545in}}%
\pgfusepath{clip}%
\pgfsetbuttcap%
\pgfsetroundjoin%
\pgfsetlinewidth{0.348659pt}%
\definecolor{currentstroke}{rgb}{0.274952,0.037752,0.364543}%
\pgfsetstrokecolor{currentstroke}%
\pgfsetdash{}{0pt}%
\pgfpathmoveto{\pgfqpoint{2.959679in}{2.355688in}}%
\pgfpathlineto{\pgfqpoint{2.909640in}{2.358323in}}%
\pgfusepath{stroke}%
\end{pgfscope}%
\begin{pgfscope}%
\pgfpathrectangle{\pgfqpoint{1.250000in}{1.750000in}}{\pgfqpoint{2.279412in}{2.004545in}}%
\pgfusepath{clip}%
\pgfsetbuttcap%
\pgfsetroundjoin%
\pgfsetlinewidth{0.359401pt}%
\definecolor{currentstroke}{rgb}{0.277018,0.050344,0.375715}%
\pgfsetstrokecolor{currentstroke}%
\pgfsetdash{}{0pt}%
\pgfpathmoveto{\pgfqpoint{2.909640in}{2.358323in}}%
\pgfpathlineto{\pgfqpoint{2.859621in}{2.361454in}}%
\pgfusepath{stroke}%
\end{pgfscope}%
\begin{pgfscope}%
\pgfpathrectangle{\pgfqpoint{1.250000in}{1.750000in}}{\pgfqpoint{2.279412in}{2.004545in}}%
\pgfusepath{clip}%
\pgfsetbuttcap%
\pgfsetroundjoin%
\pgfsetlinewidth{0.379432pt}%
\definecolor{currentstroke}{rgb}{0.279566,0.067836,0.391917}%
\pgfsetstrokecolor{currentstroke}%
\pgfsetdash{}{0pt}%
\pgfpathmoveto{\pgfqpoint{2.859621in}{2.361454in}}%
\pgfpathlineto{\pgfqpoint{2.809556in}{2.363881in}}%
\pgfusepath{stroke}%
\end{pgfscope}%
\begin{pgfscope}%
\pgfpathrectangle{\pgfqpoint{1.250000in}{1.750000in}}{\pgfqpoint{2.279412in}{2.004545in}}%
\pgfusepath{clip}%
\pgfsetbuttcap%
\pgfsetroundjoin%
\pgfsetlinewidth{0.386348pt}%
\definecolor{currentstroke}{rgb}{0.280267,0.073417,0.397163}%
\pgfsetstrokecolor{currentstroke}%
\pgfsetdash{}{0pt}%
\pgfpathmoveto{\pgfqpoint{2.809556in}{2.363881in}}%
\pgfpathlineto{\pgfqpoint{2.759658in}{2.367810in}}%
\pgfusepath{stroke}%
\end{pgfscope}%
\begin{pgfscope}%
\pgfpathrectangle{\pgfqpoint{1.250000in}{1.750000in}}{\pgfqpoint{2.279412in}{2.004545in}}%
\pgfusepath{clip}%
\pgfsetbuttcap%
\pgfsetroundjoin%
\pgfsetlinewidth{0.417744pt}%
\definecolor{currentstroke}{rgb}{0.282327,0.094955,0.417331}%
\pgfsetstrokecolor{currentstroke}%
\pgfsetdash{}{0pt}%
\pgfpathmoveto{\pgfqpoint{2.759658in}{2.367810in}}%
\pgfpathlineto{\pgfqpoint{2.709938in}{2.373515in}}%
\pgfusepath{stroke}%
\end{pgfscope}%
\begin{pgfscope}%
\pgfpathrectangle{\pgfqpoint{1.250000in}{1.750000in}}{\pgfqpoint{2.279412in}{2.004545in}}%
\pgfusepath{clip}%
\pgfsetbuttcap%
\pgfsetroundjoin%
\pgfsetlinewidth{0.433725pt}%
\definecolor{currentstroke}{rgb}{0.283091,0.110553,0.431554}%
\pgfsetstrokecolor{currentstroke}%
\pgfsetdash{}{0pt}%
\pgfpathmoveto{\pgfqpoint{2.709938in}{2.373515in}}%
\pgfpathlineto{\pgfqpoint{2.660232in}{2.379375in}}%
\pgfusepath{stroke}%
\end{pgfscope}%
\begin{pgfscope}%
\pgfpathrectangle{\pgfqpoint{1.250000in}{1.750000in}}{\pgfqpoint{2.279412in}{2.004545in}}%
\pgfusepath{clip}%
\pgfsetbuttcap%
\pgfsetroundjoin%
\pgfsetlinewidth{0.410104pt}%
\definecolor{currentstroke}{rgb}{0.281924,0.089666,0.412415}%
\pgfsetstrokecolor{currentstroke}%
\pgfsetdash{}{0pt}%
\pgfpathmoveto{\pgfqpoint{2.660232in}{2.379375in}}%
\pgfpathlineto{\pgfqpoint{2.610756in}{2.386402in}}%
\pgfusepath{stroke}%
\end{pgfscope}%
\begin{pgfscope}%
\pgfpathrectangle{\pgfqpoint{1.250000in}{1.750000in}}{\pgfqpoint{2.279412in}{2.004545in}}%
\pgfusepath{clip}%
\pgfsetbuttcap%
\pgfsetroundjoin%
\pgfsetlinewidth{0.450017pt}%
\definecolor{currentstroke}{rgb}{0.283229,0.120777,0.440584}%
\pgfsetstrokecolor{currentstroke}%
\pgfsetdash{}{0pt}%
\pgfpathmoveto{\pgfqpoint{2.610756in}{2.386402in}}%
\pgfpathlineto{\pgfqpoint{2.561705in}{2.395503in}}%
\pgfusepath{stroke}%
\end{pgfscope}%
\begin{pgfscope}%
\pgfpathrectangle{\pgfqpoint{1.250000in}{1.750000in}}{\pgfqpoint{2.279412in}{2.004545in}}%
\pgfusepath{clip}%
\pgfsetbuttcap%
\pgfsetroundjoin%
\pgfsetlinewidth{0.516129pt}%
\definecolor{currentstroke}{rgb}{0.279574,0.170599,0.479997}%
\pgfsetstrokecolor{currentstroke}%
\pgfsetdash{}{0pt}%
\pgfpathmoveto{\pgfqpoint{2.561705in}{2.395503in}}%
\pgfpathlineto{\pgfqpoint{2.513326in}{2.406989in}}%
\pgfusepath{stroke}%
\end{pgfscope}%
\begin{pgfscope}%
\pgfpathrectangle{\pgfqpoint{1.250000in}{1.750000in}}{\pgfqpoint{2.279412in}{2.004545in}}%
\pgfusepath{clip}%
\pgfsetbuttcap%
\pgfsetroundjoin%
\pgfsetlinewidth{0.422664pt}%
\definecolor{currentstroke}{rgb}{0.282656,0.100196,0.422160}%
\pgfsetstrokecolor{currentstroke}%
\pgfsetdash{}{0pt}%
\pgfpathmoveto{\pgfqpoint{2.513326in}{2.406989in}}%
\pgfpathlineto{\pgfqpoint{2.466004in}{2.421370in}}%
\pgfusepath{stroke}%
\end{pgfscope}%
\begin{pgfscope}%
\pgfpathrectangle{\pgfqpoint{1.250000in}{1.750000in}}{\pgfqpoint{2.279412in}{2.004545in}}%
\pgfusepath{clip}%
\pgfsetbuttcap%
\pgfsetroundjoin%
\pgfsetlinewidth{0.465663pt}%
\definecolor{currentstroke}{rgb}{0.283072,0.130895,0.449241}%
\pgfsetstrokecolor{currentstroke}%
\pgfsetdash{}{0pt}%
\pgfpathmoveto{\pgfqpoint{2.466004in}{2.421370in}}%
\pgfpathlineto{\pgfqpoint{2.419637in}{2.437967in}}%
\pgfusepath{stroke}%
\end{pgfscope}%
\begin{pgfscope}%
\pgfpathrectangle{\pgfqpoint{1.250000in}{1.750000in}}{\pgfqpoint{2.279412in}{2.004545in}}%
\pgfusepath{clip}%
\pgfsetbuttcap%
\pgfsetroundjoin%
\pgfsetlinewidth{0.632789pt}%
\definecolor{currentstroke}{rgb}{0.258965,0.251537,0.524736}%
\pgfsetstrokecolor{currentstroke}%
\pgfsetdash{}{0pt}%
\pgfpathmoveto{\pgfqpoint{2.419637in}{2.437967in}}%
\pgfpathlineto{\pgfqpoint{2.374541in}{2.457154in}}%
\pgfusepath{stroke}%
\end{pgfscope}%
\begin{pgfscope}%
\pgfpathrectangle{\pgfqpoint{1.250000in}{1.750000in}}{\pgfqpoint{2.279412in}{2.004545in}}%
\pgfusepath{clip}%
\pgfsetbuttcap%
\pgfsetroundjoin%
\pgfsetlinewidth{0.757511pt}%
\definecolor{currentstroke}{rgb}{0.225863,0.330805,0.547314}%
\pgfsetstrokecolor{currentstroke}%
\pgfsetdash{}{0pt}%
\pgfpathmoveto{\pgfqpoint{2.374541in}{2.457154in}}%
\pgfpathlineto{\pgfqpoint{2.330736in}{2.478541in}}%
\pgfusepath{stroke}%
\end{pgfscope}%
\begin{pgfscope}%
\pgfpathrectangle{\pgfqpoint{1.250000in}{1.750000in}}{\pgfqpoint{2.279412in}{2.004545in}}%
\pgfusepath{clip}%
\pgfsetbuttcap%
\pgfsetroundjoin%
\pgfsetlinewidth{0.767693pt}%
\definecolor{currentstroke}{rgb}{0.223925,0.334994,0.548053}%
\pgfsetstrokecolor{currentstroke}%
\pgfsetdash{}{0pt}%
\pgfpathmoveto{\pgfqpoint{2.330736in}{2.478541in}}%
\pgfpathlineto{\pgfqpoint{2.288388in}{2.502064in}}%
\pgfusepath{stroke}%
\end{pgfscope}%
\begin{pgfscope}%
\pgfpathrectangle{\pgfqpoint{1.250000in}{1.750000in}}{\pgfqpoint{2.279412in}{2.004545in}}%
\pgfusepath{clip}%
\pgfsetbuttcap%
\pgfsetroundjoin%
\pgfsetlinewidth{0.921152pt}%
\definecolor{currentstroke}{rgb}{0.182256,0.426184,0.557120}%
\pgfsetstrokecolor{currentstroke}%
\pgfsetdash{}{0pt}%
\pgfpathmoveto{\pgfqpoint{2.288388in}{2.502064in}}%
\pgfpathlineto{\pgfqpoint{2.247150in}{2.527103in}}%
\pgfusepath{stroke}%
\end{pgfscope}%
\begin{pgfscope}%
\pgfpathrectangle{\pgfqpoint{1.250000in}{1.750000in}}{\pgfqpoint{2.279412in}{2.004545in}}%
\pgfusepath{clip}%
\pgfsetbuttcap%
\pgfsetroundjoin%
\pgfsetlinewidth{1.216397pt}%
\definecolor{currentstroke}{rgb}{0.122606,0.585371,0.546557}%
\pgfsetstrokecolor{currentstroke}%
\pgfsetdash{}{0pt}%
\pgfpathmoveto{\pgfqpoint{2.247150in}{2.527103in}}%
\pgfpathlineto{\pgfqpoint{2.206228in}{2.552564in}}%
\pgfusepath{stroke}%
\end{pgfscope}%
\begin{pgfscope}%
\pgfpathrectangle{\pgfqpoint{1.250000in}{1.750000in}}{\pgfqpoint{2.279412in}{2.004545in}}%
\pgfusepath{clip}%
\pgfsetbuttcap%
\pgfsetroundjoin%
\pgfsetlinewidth{1.541118pt}%
\definecolor{currentstroke}{rgb}{0.274149,0.751988,0.436601}%
\pgfsetstrokecolor{currentstroke}%
\pgfsetdash{}{0pt}%
\pgfpathmoveto{\pgfqpoint{2.206228in}{2.552564in}}%
\pgfpathlineto{\pgfqpoint{2.165448in}{2.578220in}}%
\pgfusepath{stroke}%
\end{pgfscope}%
\begin{pgfscope}%
\pgfpathrectangle{\pgfqpoint{1.250000in}{1.750000in}}{\pgfqpoint{2.279412in}{2.004545in}}%
\pgfusepath{clip}%
\pgfsetbuttcap%
\pgfsetroundjoin%
\pgfsetlinewidth{1.956508pt}%
\definecolor{currentstroke}{rgb}{0.855810,0.888601,0.097452}%
\pgfsetstrokecolor{currentstroke}%
\pgfsetdash{}{0pt}%
\pgfpathmoveto{\pgfqpoint{2.165448in}{2.578220in}}%
\pgfpathlineto{\pgfqpoint{2.124609in}{2.603755in}}%
\pgfusepath{stroke}%
\end{pgfscope}%
\begin{pgfscope}%
\pgfpathrectangle{\pgfqpoint{1.250000in}{1.750000in}}{\pgfqpoint{2.279412in}{2.004545in}}%
\pgfusepath{clip}%
\pgfsetbuttcap%
\pgfsetroundjoin%
\pgfsetlinewidth{1.972809pt}%
\definecolor{currentstroke}{rgb}{0.876168,0.891125,0.095250}%
\pgfsetstrokecolor{currentstroke}%
\pgfsetdash{}{0pt}%
\pgfpathmoveto{\pgfqpoint{2.124609in}{2.603755in}}%
\pgfpathlineto{\pgfqpoint{2.082861in}{2.628024in}}%
\pgfusepath{stroke}%
\end{pgfscope}%
\begin{pgfscope}%
\pgfpathrectangle{\pgfqpoint{1.250000in}{1.750000in}}{\pgfqpoint{2.279412in}{2.004545in}}%
\pgfusepath{clip}%
\pgfsetbuttcap%
\pgfsetroundjoin%
\pgfsetlinewidth{0.330086pt}%
\definecolor{currentstroke}{rgb}{0.272594,0.025563,0.353093}%
\pgfsetstrokecolor{currentstroke}%
\pgfsetdash{}{0pt}%
\pgfpathmoveto{\pgfqpoint{3.159084in}{2.436525in}}%
\pgfpathlineto{\pgfqpoint{3.108948in}{2.436291in}}%
\pgfusepath{stroke}%
\end{pgfscope}%
\begin{pgfscope}%
\pgfpathrectangle{\pgfqpoint{1.250000in}{1.750000in}}{\pgfqpoint{2.279412in}{2.004545in}}%
\pgfusepath{clip}%
\pgfsetbuttcap%
\pgfsetroundjoin%
\pgfsetlinewidth{0.325685pt}%
\definecolor{currentstroke}{rgb}{0.271305,0.019942,0.347269}%
\pgfsetstrokecolor{currentstroke}%
\pgfsetdash{}{0pt}%
\pgfpathmoveto{\pgfqpoint{3.108948in}{2.436291in}}%
\pgfpathlineto{\pgfqpoint{3.058833in}{2.437424in}}%
\pgfusepath{stroke}%
\end{pgfscope}%
\begin{pgfscope}%
\pgfpathrectangle{\pgfqpoint{1.250000in}{1.750000in}}{\pgfqpoint{2.279412in}{2.004545in}}%
\pgfusepath{clip}%
\pgfsetbuttcap%
\pgfsetroundjoin%
\pgfsetlinewidth{0.325605pt}%
\definecolor{currentstroke}{rgb}{0.271305,0.019942,0.347269}%
\pgfsetstrokecolor{currentstroke}%
\pgfsetdash{}{0pt}%
\pgfpathmoveto{\pgfqpoint{3.058833in}{2.437424in}}%
\pgfpathlineto{\pgfqpoint{3.008724in}{2.439116in}}%
\pgfusepath{stroke}%
\end{pgfscope}%
\begin{pgfscope}%
\pgfpathrectangle{\pgfqpoint{1.250000in}{1.750000in}}{\pgfqpoint{2.279412in}{2.004545in}}%
\pgfusepath{clip}%
\pgfsetbuttcap%
\pgfsetroundjoin%
\pgfsetlinewidth{0.338586pt}%
\definecolor{currentstroke}{rgb}{0.273809,0.031497,0.358853}%
\pgfsetstrokecolor{currentstroke}%
\pgfsetdash{}{0pt}%
\pgfpathmoveto{\pgfqpoint{3.008724in}{2.439116in}}%
\pgfpathlineto{\pgfqpoint{2.958616in}{2.440787in}}%
\pgfusepath{stroke}%
\end{pgfscope}%
\begin{pgfscope}%
\pgfpathrectangle{\pgfqpoint{1.250000in}{1.750000in}}{\pgfqpoint{2.279412in}{2.004545in}}%
\pgfusepath{clip}%
\pgfsetbuttcap%
\pgfsetroundjoin%
\pgfsetlinewidth{0.365243pt}%
\definecolor{currentstroke}{rgb}{0.277941,0.056324,0.381191}%
\pgfsetstrokecolor{currentstroke}%
\pgfsetdash{}{0pt}%
\pgfpathmoveto{\pgfqpoint{2.958616in}{2.440787in}}%
\pgfpathlineto{\pgfqpoint{2.908546in}{2.443214in}}%
\pgfusepath{stroke}%
\end{pgfscope}%
\begin{pgfscope}%
\pgfpathrectangle{\pgfqpoint{1.250000in}{1.750000in}}{\pgfqpoint{2.279412in}{2.004545in}}%
\pgfusepath{clip}%
\pgfsetbuttcap%
\pgfsetroundjoin%
\pgfsetlinewidth{0.373135pt}%
\definecolor{currentstroke}{rgb}{0.278791,0.062145,0.386592}%
\pgfsetstrokecolor{currentstroke}%
\pgfsetdash{}{0pt}%
\pgfpathmoveto{\pgfqpoint{2.908546in}{2.443214in}}%
\pgfpathlineto{\pgfqpoint{2.858484in}{2.445826in}}%
\pgfusepath{stroke}%
\end{pgfscope}%
\begin{pgfscope}%
\pgfpathrectangle{\pgfqpoint{1.250000in}{1.750000in}}{\pgfqpoint{2.279412in}{2.004545in}}%
\pgfusepath{clip}%
\pgfsetbuttcap%
\pgfsetroundjoin%
\pgfsetlinewidth{0.399679pt}%
\definecolor{currentstroke}{rgb}{0.281446,0.084320,0.407414}%
\pgfsetstrokecolor{currentstroke}%
\pgfsetdash{}{0pt}%
\pgfpathmoveto{\pgfqpoint{2.858484in}{2.445826in}}%
\pgfpathlineto{\pgfqpoint{2.808419in}{2.448416in}}%
\pgfusepath{stroke}%
\end{pgfscope}%
\begin{pgfscope}%
\pgfpathrectangle{\pgfqpoint{1.250000in}{1.750000in}}{\pgfqpoint{2.279412in}{2.004545in}}%
\pgfusepath{clip}%
\pgfsetbuttcap%
\pgfsetroundjoin%
\pgfsetlinewidth{0.427753pt}%
\definecolor{currentstroke}{rgb}{0.282910,0.105393,0.426902}%
\pgfsetstrokecolor{currentstroke}%
\pgfsetdash{}{0pt}%
\pgfpathmoveto{\pgfqpoint{2.808419in}{2.448416in}}%
\pgfpathlineto{\pgfqpoint{2.758425in}{2.451841in}}%
\pgfusepath{stroke}%
\end{pgfscope}%
\begin{pgfscope}%
\pgfpathrectangle{\pgfqpoint{1.250000in}{1.750000in}}{\pgfqpoint{2.279412in}{2.004545in}}%
\pgfusepath{clip}%
\pgfsetbuttcap%
\pgfsetroundjoin%
\pgfsetlinewidth{0.464249pt}%
\definecolor{currentstroke}{rgb}{0.283072,0.130895,0.449241}%
\pgfsetstrokecolor{currentstroke}%
\pgfsetdash{}{0pt}%
\pgfpathmoveto{\pgfqpoint{2.758425in}{2.451841in}}%
\pgfpathlineto{\pgfqpoint{2.708517in}{2.456146in}}%
\pgfusepath{stroke}%
\end{pgfscope}%
\begin{pgfscope}%
\pgfpathrectangle{\pgfqpoint{1.250000in}{1.750000in}}{\pgfqpoint{2.279412in}{2.004545in}}%
\pgfusepath{clip}%
\pgfsetbuttcap%
\pgfsetroundjoin%
\pgfsetlinewidth{0.495057pt}%
\definecolor{currentstroke}{rgb}{0.281412,0.155834,0.469201}%
\pgfsetstrokecolor{currentstroke}%
\pgfsetdash{}{0pt}%
\pgfpathmoveto{\pgfqpoint{2.708517in}{2.456146in}}%
\pgfpathlineto{\pgfqpoint{2.658724in}{2.461378in}}%
\pgfusepath{stroke}%
\end{pgfscope}%
\begin{pgfscope}%
\pgfpathrectangle{\pgfqpoint{1.250000in}{1.750000in}}{\pgfqpoint{2.279412in}{2.004545in}}%
\pgfusepath{clip}%
\pgfsetbuttcap%
\pgfsetroundjoin%
\pgfsetlinewidth{0.503802pt}%
\definecolor{currentstroke}{rgb}{0.280868,0.160771,0.472899}%
\pgfsetstrokecolor{currentstroke}%
\pgfsetdash{}{0pt}%
\pgfpathmoveto{\pgfqpoint{2.658724in}{2.461378in}}%
\pgfpathlineto{\pgfqpoint{2.609159in}{2.468049in}}%
\pgfusepath{stroke}%
\end{pgfscope}%
\begin{pgfscope}%
\pgfpathrectangle{\pgfqpoint{1.250000in}{1.750000in}}{\pgfqpoint{2.279412in}{2.004545in}}%
\pgfusepath{clip}%
\pgfsetbuttcap%
\pgfsetroundjoin%
\pgfsetlinewidth{0.519015pt}%
\definecolor{currentstroke}{rgb}{0.279574,0.170599,0.479997}%
\pgfsetstrokecolor{currentstroke}%
\pgfsetdash{}{0pt}%
\pgfpathmoveto{\pgfqpoint{2.609159in}{2.468049in}}%
\pgfpathlineto{\pgfqpoint{2.559825in}{2.475962in}}%
\pgfusepath{stroke}%
\end{pgfscope}%
\begin{pgfscope}%
\pgfpathrectangle{\pgfqpoint{1.250000in}{1.750000in}}{\pgfqpoint{2.279412in}{2.004545in}}%
\pgfusepath{clip}%
\pgfsetbuttcap%
\pgfsetroundjoin%
\pgfsetlinewidth{0.501999pt}%
\definecolor{currentstroke}{rgb}{0.280868,0.160771,0.472899}%
\pgfsetstrokecolor{currentstroke}%
\pgfsetdash{}{0pt}%
\pgfpathmoveto{\pgfqpoint{2.559825in}{2.475962in}}%
\pgfpathlineto{\pgfqpoint{2.510903in}{2.485571in}}%
\pgfusepath{stroke}%
\end{pgfscope}%
\begin{pgfscope}%
\pgfpathrectangle{\pgfqpoint{1.250000in}{1.750000in}}{\pgfqpoint{2.279412in}{2.004545in}}%
\pgfusepath{clip}%
\pgfsetbuttcap%
\pgfsetroundjoin%
\pgfsetlinewidth{0.771507pt}%
\definecolor{currentstroke}{rgb}{0.221989,0.339161,0.548752}%
\pgfsetstrokecolor{currentstroke}%
\pgfsetdash{}{0pt}%
\pgfpathmoveto{\pgfqpoint{2.510903in}{2.485571in}}%
\pgfpathlineto{\pgfqpoint{2.462584in}{2.497336in}}%
\pgfusepath{stroke}%
\end{pgfscope}%
\begin{pgfscope}%
\pgfpathrectangle{\pgfqpoint{1.250000in}{1.750000in}}{\pgfqpoint{2.279412in}{2.004545in}}%
\pgfusepath{clip}%
\pgfsetbuttcap%
\pgfsetroundjoin%
\pgfsetlinewidth{0.953561pt}%
\definecolor{currentstroke}{rgb}{0.174274,0.445044,0.557792}%
\pgfsetstrokecolor{currentstroke}%
\pgfsetdash{}{0pt}%
\pgfpathmoveto{\pgfqpoint{2.462584in}{2.497336in}}%
\pgfpathlineto{\pgfqpoint{2.414883in}{2.510902in}}%
\pgfusepath{stroke}%
\end{pgfscope}%
\begin{pgfscope}%
\pgfpathrectangle{\pgfqpoint{1.250000in}{1.750000in}}{\pgfqpoint{2.279412in}{2.004545in}}%
\pgfusepath{clip}%
\pgfsetbuttcap%
\pgfsetroundjoin%
\pgfsetlinewidth{0.957089pt}%
\definecolor{currentstroke}{rgb}{0.174274,0.445044,0.557792}%
\pgfsetstrokecolor{currentstroke}%
\pgfsetdash{}{0pt}%
\pgfpathmoveto{\pgfqpoint{2.414883in}{2.510902in}}%
\pgfpathlineto{\pgfqpoint{2.367817in}{2.526102in}}%
\pgfusepath{stroke}%
\end{pgfscope}%
\begin{pgfscope}%
\pgfpathrectangle{\pgfqpoint{1.250000in}{1.750000in}}{\pgfqpoint{2.279412in}{2.004545in}}%
\pgfusepath{clip}%
\pgfsetbuttcap%
\pgfsetroundjoin%
\pgfsetlinewidth{0.919488pt}%
\definecolor{currentstroke}{rgb}{0.182256,0.426184,0.557120}%
\pgfsetstrokecolor{currentstroke}%
\pgfsetdash{}{0pt}%
\pgfpathmoveto{\pgfqpoint{2.367817in}{2.526102in}}%
\pgfpathlineto{\pgfqpoint{2.321501in}{2.542969in}}%
\pgfusepath{stroke}%
\end{pgfscope}%
\begin{pgfscope}%
\pgfpathrectangle{\pgfqpoint{1.250000in}{1.750000in}}{\pgfqpoint{2.279412in}{2.004545in}}%
\pgfusepath{clip}%
\pgfsetbuttcap%
\pgfsetroundjoin%
\pgfsetlinewidth{0.325625pt}%
\definecolor{currentstroke}{rgb}{0.271305,0.019942,0.347269}%
\pgfsetstrokecolor{currentstroke}%
\pgfsetdash{}{0pt}%
\pgfpathmoveto{\pgfqpoint{3.159084in}{2.571846in}}%
\pgfpathlineto{\pgfqpoint{3.108937in}{2.572434in}}%
\pgfusepath{stroke}%
\end{pgfscope}%
\begin{pgfscope}%
\pgfpathrectangle{\pgfqpoint{1.250000in}{1.750000in}}{\pgfqpoint{2.279412in}{2.004545in}}%
\pgfusepath{clip}%
\pgfsetbuttcap%
\pgfsetroundjoin%
\pgfsetlinewidth{0.336896pt}%
\definecolor{currentstroke}{rgb}{0.273809,0.031497,0.358853}%
\pgfsetstrokecolor{currentstroke}%
\pgfsetdash{}{0pt}%
\pgfpathmoveto{\pgfqpoint{3.108937in}{2.572434in}}%
\pgfpathlineto{\pgfqpoint{3.058797in}{2.573323in}}%
\pgfusepath{stroke}%
\end{pgfscope}%
\begin{pgfscope}%
\pgfpathrectangle{\pgfqpoint{1.250000in}{1.750000in}}{\pgfqpoint{2.279412in}{2.004545in}}%
\pgfusepath{clip}%
\pgfsetbuttcap%
\pgfsetroundjoin%
\pgfsetlinewidth{0.339672pt}%
\definecolor{currentstroke}{rgb}{0.273809,0.031497,0.358853}%
\pgfsetstrokecolor{currentstroke}%
\pgfsetdash{}{0pt}%
\pgfpathmoveto{\pgfqpoint{3.058797in}{2.573323in}}%
\pgfpathlineto{\pgfqpoint{3.008673in}{2.574647in}}%
\pgfusepath{stroke}%
\end{pgfscope}%
\begin{pgfscope}%
\pgfpathrectangle{\pgfqpoint{1.250000in}{1.750000in}}{\pgfqpoint{2.279412in}{2.004545in}}%
\pgfusepath{clip}%
\pgfsetbuttcap%
\pgfsetroundjoin%
\pgfsetlinewidth{0.353253pt}%
\definecolor{currentstroke}{rgb}{0.276022,0.044167,0.370164}%
\pgfsetstrokecolor{currentstroke}%
\pgfsetdash{}{0pt}%
\pgfpathmoveto{\pgfqpoint{3.008673in}{2.574647in}}%
\pgfpathlineto{\pgfqpoint{2.958552in}{2.576077in}}%
\pgfusepath{stroke}%
\end{pgfscope}%
\begin{pgfscope}%
\pgfpathrectangle{\pgfqpoint{1.250000in}{1.750000in}}{\pgfqpoint{2.279412in}{2.004545in}}%
\pgfusepath{clip}%
\pgfsetbuttcap%
\pgfsetroundjoin%
\pgfsetlinewidth{0.384848pt}%
\definecolor{currentstroke}{rgb}{0.280267,0.073417,0.397163}%
\pgfsetstrokecolor{currentstroke}%
\pgfsetdash{}{0pt}%
\pgfpathmoveto{\pgfqpoint{2.958552in}{2.576077in}}%
\pgfpathlineto{\pgfqpoint{2.908432in}{2.577626in}}%
\pgfusepath{stroke}%
\end{pgfscope}%
\begin{pgfscope}%
\pgfpathrectangle{\pgfqpoint{1.250000in}{1.750000in}}{\pgfqpoint{2.279412in}{2.004545in}}%
\pgfusepath{clip}%
\pgfsetbuttcap%
\pgfsetroundjoin%
\pgfsetlinewidth{0.407229pt}%
\definecolor{currentstroke}{rgb}{0.281924,0.089666,0.412415}%
\pgfsetstrokecolor{currentstroke}%
\pgfsetdash{}{0pt}%
\pgfpathmoveto{\pgfqpoint{2.908432in}{2.577626in}}%
\pgfpathlineto{\pgfqpoint{2.858312in}{2.579166in}}%
\pgfusepath{stroke}%
\end{pgfscope}%
\begin{pgfscope}%
\pgfpathrectangle{\pgfqpoint{1.250000in}{1.750000in}}{\pgfqpoint{2.279412in}{2.004545in}}%
\pgfusepath{clip}%
\pgfsetbuttcap%
\pgfsetroundjoin%
\pgfsetlinewidth{0.475888pt}%
\definecolor{currentstroke}{rgb}{0.282623,0.140926,0.457517}%
\pgfsetstrokecolor{currentstroke}%
\pgfsetdash{}{0pt}%
\pgfpathmoveto{\pgfqpoint{2.858312in}{2.579166in}}%
\pgfpathlineto{\pgfqpoint{2.808195in}{2.580766in}}%
\pgfusepath{stroke}%
\end{pgfscope}%
\begin{pgfscope}%
\pgfpathrectangle{\pgfqpoint{1.250000in}{1.750000in}}{\pgfqpoint{2.279412in}{2.004545in}}%
\pgfusepath{clip}%
\pgfsetbuttcap%
\pgfsetroundjoin%
\pgfsetlinewidth{0.551251pt}%
\definecolor{currentstroke}{rgb}{0.275191,0.194905,0.496005}%
\pgfsetstrokecolor{currentstroke}%
\pgfsetdash{}{0pt}%
\pgfpathmoveto{\pgfqpoint{2.808195in}{2.580766in}}%
\pgfpathlineto{\pgfqpoint{2.758106in}{2.582949in}}%
\pgfusepath{stroke}%
\end{pgfscope}%
\begin{pgfscope}%
\pgfpathrectangle{\pgfqpoint{1.250000in}{1.750000in}}{\pgfqpoint{2.279412in}{2.004545in}}%
\pgfusepath{clip}%
\pgfsetbuttcap%
\pgfsetroundjoin%
\pgfsetlinewidth{0.637944pt}%
\definecolor{currentstroke}{rgb}{0.257322,0.256130,0.526563}%
\pgfsetstrokecolor{currentstroke}%
\pgfsetdash{}{0pt}%
\pgfpathmoveto{\pgfqpoint{2.758106in}{2.582949in}}%
\pgfpathlineto{\pgfqpoint{2.708048in}{2.585634in}}%
\pgfusepath{stroke}%
\end{pgfscope}%
\begin{pgfscope}%
\pgfpathrectangle{\pgfqpoint{1.250000in}{1.750000in}}{\pgfqpoint{2.279412in}{2.004545in}}%
\pgfusepath{clip}%
\pgfsetbuttcap%
\pgfsetroundjoin%
\pgfsetlinewidth{0.765092pt}%
\definecolor{currentstroke}{rgb}{0.223925,0.334994,0.548053}%
\pgfsetstrokecolor{currentstroke}%
\pgfsetdash{}{0pt}%
\pgfpathmoveto{\pgfqpoint{2.708048in}{2.585634in}}%
\pgfpathlineto{\pgfqpoint{2.658038in}{2.588926in}}%
\pgfusepath{stroke}%
\end{pgfscope}%
\begin{pgfscope}%
\pgfpathrectangle{\pgfqpoint{1.250000in}{1.750000in}}{\pgfqpoint{2.279412in}{2.004545in}}%
\pgfusepath{clip}%
\pgfsetbuttcap%
\pgfsetroundjoin%
\pgfsetlinewidth{0.927791pt}%
\definecolor{currentstroke}{rgb}{0.180629,0.429975,0.557282}%
\pgfsetstrokecolor{currentstroke}%
\pgfsetdash{}{0pt}%
\pgfpathmoveto{\pgfqpoint{2.658038in}{2.588926in}}%
\pgfpathlineto{\pgfqpoint{2.608115in}{2.593104in}}%
\pgfusepath{stroke}%
\end{pgfscope}%
\begin{pgfscope}%
\pgfpathrectangle{\pgfqpoint{1.250000in}{1.750000in}}{\pgfqpoint{2.279412in}{2.004545in}}%
\pgfusepath{clip}%
\pgfsetbuttcap%
\pgfsetroundjoin%
\pgfsetlinewidth{0.322942pt}%
\definecolor{currentstroke}{rgb}{0.271305,0.019942,0.347269}%
\pgfsetstrokecolor{currentstroke}%
\pgfsetdash{}{0pt}%
\pgfpathmoveto{\pgfqpoint{3.159084in}{2.662059in}}%
\pgfpathlineto{\pgfqpoint{3.108952in}{2.662249in}}%
\pgfusepath{stroke}%
\end{pgfscope}%
\begin{pgfscope}%
\pgfpathrectangle{\pgfqpoint{1.250000in}{1.750000in}}{\pgfqpoint{2.279412in}{2.004545in}}%
\pgfusepath{clip}%
\pgfsetbuttcap%
\pgfsetroundjoin%
\pgfsetlinewidth{0.332303pt}%
\definecolor{currentstroke}{rgb}{0.272594,0.025563,0.353093}%
\pgfsetstrokecolor{currentstroke}%
\pgfsetdash{}{0pt}%
\pgfpathmoveto{\pgfqpoint{3.108952in}{2.662249in}}%
\pgfpathlineto{\pgfqpoint{3.058816in}{2.662221in}}%
\pgfusepath{stroke}%
\end{pgfscope}%
\begin{pgfscope}%
\pgfpathrectangle{\pgfqpoint{1.250000in}{1.750000in}}{\pgfqpoint{2.279412in}{2.004545in}}%
\pgfusepath{clip}%
\pgfsetbuttcap%
\pgfsetroundjoin%
\pgfsetlinewidth{0.339987pt}%
\definecolor{currentstroke}{rgb}{0.273809,0.031497,0.358853}%
\pgfsetstrokecolor{currentstroke}%
\pgfsetdash{}{0pt}%
\pgfpathmoveto{\pgfqpoint{3.058816in}{2.662221in}}%
\pgfpathlineto{\pgfqpoint{3.008668in}{2.662235in}}%
\pgfusepath{stroke}%
\end{pgfscope}%
\begin{pgfscope}%
\pgfpathrectangle{\pgfqpoint{1.250000in}{1.750000in}}{\pgfqpoint{2.279412in}{2.004545in}}%
\pgfusepath{clip}%
\pgfsetbuttcap%
\pgfsetroundjoin%
\pgfsetlinewidth{0.362895pt}%
\definecolor{currentstroke}{rgb}{0.277941,0.056324,0.381191}%
\pgfsetstrokecolor{currentstroke}%
\pgfsetdash{}{0pt}%
\pgfpathmoveto{\pgfqpoint{3.008668in}{2.662235in}}%
\pgfpathlineto{\pgfqpoint{2.958521in}{2.662051in}}%
\pgfusepath{stroke}%
\end{pgfscope}%
\begin{pgfscope}%
\pgfpathrectangle{\pgfqpoint{1.250000in}{1.750000in}}{\pgfqpoint{2.279412in}{2.004545in}}%
\pgfusepath{clip}%
\pgfsetbuttcap%
\pgfsetroundjoin%
\pgfsetlinewidth{0.393917pt}%
\definecolor{currentstroke}{rgb}{0.280894,0.078907,0.402329}%
\pgfsetstrokecolor{currentstroke}%
\pgfsetdash{}{0pt}%
\pgfpathmoveto{\pgfqpoint{2.958521in}{2.662051in}}%
\pgfpathlineto{\pgfqpoint{2.908373in}{2.662076in}}%
\pgfusepath{stroke}%
\end{pgfscope}%
\begin{pgfscope}%
\pgfpathrectangle{\pgfqpoint{1.250000in}{1.750000in}}{\pgfqpoint{2.279412in}{2.004545in}}%
\pgfusepath{clip}%
\pgfsetbuttcap%
\pgfsetroundjoin%
\pgfsetlinewidth{0.442860pt}%
\definecolor{currentstroke}{rgb}{0.283197,0.115680,0.436115}%
\pgfsetstrokecolor{currentstroke}%
\pgfsetdash{}{0pt}%
\pgfpathmoveto{\pgfqpoint{2.908373in}{2.662076in}}%
\pgfpathlineto{\pgfqpoint{2.858222in}{2.662221in}}%
\pgfusepath{stroke}%
\end{pgfscope}%
\begin{pgfscope}%
\pgfpathrectangle{\pgfqpoint{1.250000in}{1.750000in}}{\pgfqpoint{2.279412in}{2.004545in}}%
\pgfusepath{clip}%
\pgfsetbuttcap%
\pgfsetroundjoin%
\pgfsetlinewidth{0.509195pt}%
\definecolor{currentstroke}{rgb}{0.280255,0.165693,0.476498}%
\pgfsetstrokecolor{currentstroke}%
\pgfsetdash{}{0pt}%
\pgfpathmoveto{\pgfqpoint{2.858222in}{2.662221in}}%
\pgfpathlineto{\pgfqpoint{2.808073in}{2.662596in}}%
\pgfusepath{stroke}%
\end{pgfscope}%
\begin{pgfscope}%
\pgfpathrectangle{\pgfqpoint{1.250000in}{1.750000in}}{\pgfqpoint{2.279412in}{2.004545in}}%
\pgfusepath{clip}%
\pgfsetbuttcap%
\pgfsetroundjoin%
\pgfsetlinewidth{0.605382pt}%
\definecolor{currentstroke}{rgb}{0.265145,0.232956,0.516599}%
\pgfsetstrokecolor{currentstroke}%
\pgfsetdash{}{0pt}%
\pgfpathmoveto{\pgfqpoint{2.808073in}{2.662596in}}%
\pgfpathlineto{\pgfqpoint{2.757926in}{2.663192in}}%
\pgfusepath{stroke}%
\end{pgfscope}%
\begin{pgfscope}%
\pgfpathrectangle{\pgfqpoint{1.250000in}{1.750000in}}{\pgfqpoint{2.279412in}{2.004545in}}%
\pgfusepath{clip}%
\pgfsetbuttcap%
\pgfsetroundjoin%
\pgfsetlinewidth{0.765677pt}%
\definecolor{currentstroke}{rgb}{0.223925,0.334994,0.548053}%
\pgfsetstrokecolor{currentstroke}%
\pgfsetdash{}{0pt}%
\pgfpathmoveto{\pgfqpoint{2.757926in}{2.663192in}}%
\pgfpathlineto{\pgfqpoint{2.707787in}{2.664144in}}%
\pgfusepath{stroke}%
\end{pgfscope}%
\begin{pgfscope}%
\pgfpathrectangle{\pgfqpoint{1.250000in}{1.750000in}}{\pgfqpoint{2.279412in}{2.004545in}}%
\pgfusepath{clip}%
\pgfsetbuttcap%
\pgfsetroundjoin%
\pgfsetlinewidth{0.997099pt}%
\definecolor{currentstroke}{rgb}{0.165117,0.467423,0.558141}%
\pgfsetstrokecolor{currentstroke}%
\pgfsetdash{}{0pt}%
\pgfpathmoveto{\pgfqpoint{2.707787in}{2.664144in}}%
\pgfpathlineto{\pgfqpoint{2.657665in}{2.665638in}}%
\pgfusepath{stroke}%
\end{pgfscope}%
\begin{pgfscope}%
\pgfpathrectangle{\pgfqpoint{1.250000in}{1.750000in}}{\pgfqpoint{2.279412in}{2.004545in}}%
\pgfusepath{clip}%
\pgfsetbuttcap%
\pgfsetroundjoin%
\pgfsetlinewidth{1.252314pt}%
\definecolor{currentstroke}{rgb}{0.119738,0.603785,0.541400}%
\pgfsetstrokecolor{currentstroke}%
\pgfsetdash{}{0pt}%
\pgfpathmoveto{\pgfqpoint{2.657665in}{2.665638in}}%
\pgfpathlineto{\pgfqpoint{2.607566in}{2.667638in}}%
\pgfusepath{stroke}%
\end{pgfscope}%
\begin{pgfscope}%
\pgfpathrectangle{\pgfqpoint{1.250000in}{1.750000in}}{\pgfqpoint{2.279412in}{2.004545in}}%
\pgfusepath{clip}%
\pgfsetbuttcap%
\pgfsetroundjoin%
\pgfsetlinewidth{1.540257pt}%
\definecolor{currentstroke}{rgb}{0.274149,0.751988,0.436601}%
\pgfsetstrokecolor{currentstroke}%
\pgfsetdash{}{0pt}%
\pgfpathmoveto{\pgfqpoint{2.607566in}{2.667638in}}%
\pgfpathlineto{\pgfqpoint{2.557496in}{2.670123in}}%
\pgfusepath{stroke}%
\end{pgfscope}%
\begin{pgfscope}%
\pgfpathrectangle{\pgfqpoint{1.250000in}{1.750000in}}{\pgfqpoint{2.279412in}{2.004545in}}%
\pgfusepath{clip}%
\pgfsetbuttcap%
\pgfsetroundjoin%
\pgfsetlinewidth{0.337443pt}%
\definecolor{currentstroke}{rgb}{0.273809,0.031497,0.358853}%
\pgfsetstrokecolor{currentstroke}%
\pgfsetdash{}{0pt}%
\pgfpathmoveto{\pgfqpoint{3.159084in}{2.707166in}}%
\pgfpathlineto{\pgfqpoint{3.108952in}{2.708195in}}%
\pgfusepath{stroke}%
\end{pgfscope}%
\begin{pgfscope}%
\pgfpathrectangle{\pgfqpoint{1.250000in}{1.750000in}}{\pgfqpoint{2.279412in}{2.004545in}}%
\pgfusepath{clip}%
\pgfsetbuttcap%
\pgfsetroundjoin%
\pgfsetlinewidth{0.326715pt}%
\definecolor{currentstroke}{rgb}{0.271305,0.019942,0.347269}%
\pgfsetstrokecolor{currentstroke}%
\pgfsetdash{}{0pt}%
\pgfpathmoveto{\pgfqpoint{3.108952in}{2.708195in}}%
\pgfpathlineto{\pgfqpoint{3.058815in}{2.708701in}}%
\pgfusepath{stroke}%
\end{pgfscope}%
\begin{pgfscope}%
\pgfpathrectangle{\pgfqpoint{1.250000in}{1.750000in}}{\pgfqpoint{2.279412in}{2.004545in}}%
\pgfusepath{clip}%
\pgfsetbuttcap%
\pgfsetroundjoin%
\pgfsetlinewidth{0.340646pt}%
\definecolor{currentstroke}{rgb}{0.273809,0.031497,0.358853}%
\pgfsetstrokecolor{currentstroke}%
\pgfsetdash{}{0pt}%
\pgfpathmoveto{\pgfqpoint{3.058815in}{2.708701in}}%
\pgfpathlineto{\pgfqpoint{3.008671in}{2.708844in}}%
\pgfusepath{stroke}%
\end{pgfscope}%
\begin{pgfscope}%
\pgfpathrectangle{\pgfqpoint{1.250000in}{1.750000in}}{\pgfqpoint{2.279412in}{2.004545in}}%
\pgfusepath{clip}%
\pgfsetbuttcap%
\pgfsetroundjoin%
\pgfsetlinewidth{0.360706pt}%
\definecolor{currentstroke}{rgb}{0.277018,0.050344,0.375715}%
\pgfsetstrokecolor{currentstroke}%
\pgfsetdash{}{0pt}%
\pgfpathmoveto{\pgfqpoint{3.008671in}{2.708844in}}%
\pgfpathlineto{\pgfqpoint{2.958529in}{2.709189in}}%
\pgfusepath{stroke}%
\end{pgfscope}%
\begin{pgfscope}%
\pgfpathrectangle{\pgfqpoint{1.250000in}{1.750000in}}{\pgfqpoint{2.279412in}{2.004545in}}%
\pgfusepath{clip}%
\pgfsetbuttcap%
\pgfsetroundjoin%
\pgfsetlinewidth{0.388634pt}%
\definecolor{currentstroke}{rgb}{0.280267,0.073417,0.397163}%
\pgfsetstrokecolor{currentstroke}%
\pgfsetdash{}{0pt}%
\pgfpathmoveto{\pgfqpoint{2.958529in}{2.709189in}}%
\pgfpathlineto{\pgfqpoint{2.908383in}{2.709640in}}%
\pgfusepath{stroke}%
\end{pgfscope}%
\begin{pgfscope}%
\pgfpathrectangle{\pgfqpoint{1.250000in}{1.750000in}}{\pgfqpoint{2.279412in}{2.004545in}}%
\pgfusepath{clip}%
\pgfsetbuttcap%
\pgfsetroundjoin%
\pgfsetlinewidth{0.447725pt}%
\definecolor{currentstroke}{rgb}{0.283229,0.120777,0.440584}%
\pgfsetstrokecolor{currentstroke}%
\pgfsetdash{}{0pt}%
\pgfpathmoveto{\pgfqpoint{2.908383in}{2.709640in}}%
\pgfpathlineto{\pgfqpoint{2.858235in}{2.710177in}}%
\pgfusepath{stroke}%
\end{pgfscope}%
\begin{pgfscope}%
\pgfpathrectangle{\pgfqpoint{1.250000in}{1.750000in}}{\pgfqpoint{2.279412in}{2.004545in}}%
\pgfusepath{clip}%
\pgfsetbuttcap%
\pgfsetroundjoin%
\pgfsetlinewidth{0.490309pt}%
\definecolor{currentstroke}{rgb}{0.281887,0.150881,0.465405}%
\pgfsetstrokecolor{currentstroke}%
\pgfsetdash{}{0pt}%
\pgfpathmoveto{\pgfqpoint{2.858235in}{2.710177in}}%
\pgfpathlineto{\pgfqpoint{2.808086in}{2.710591in}}%
\pgfusepath{stroke}%
\end{pgfscope}%
\begin{pgfscope}%
\pgfpathrectangle{\pgfqpoint{1.250000in}{1.750000in}}{\pgfqpoint{2.279412in}{2.004545in}}%
\pgfusepath{clip}%
\pgfsetbuttcap%
\pgfsetroundjoin%
\pgfsetlinewidth{0.621973pt}%
\definecolor{currentstroke}{rgb}{0.262138,0.242286,0.520837}%
\pgfsetstrokecolor{currentstroke}%
\pgfsetdash{}{0pt}%
\pgfpathmoveto{\pgfqpoint{2.808086in}{2.710591in}}%
\pgfpathlineto{\pgfqpoint{2.757938in}{2.711154in}}%
\pgfusepath{stroke}%
\end{pgfscope}%
\begin{pgfscope}%
\pgfpathrectangle{\pgfqpoint{1.250000in}{1.750000in}}{\pgfqpoint{2.279412in}{2.004545in}}%
\pgfusepath{clip}%
\pgfsetbuttcap%
\pgfsetroundjoin%
\pgfsetlinewidth{0.826513pt}%
\definecolor{currentstroke}{rgb}{0.206756,0.371758,0.553117}%
\pgfsetstrokecolor{currentstroke}%
\pgfsetdash{}{0pt}%
\pgfpathmoveto{\pgfqpoint{2.757938in}{2.711154in}}%
\pgfpathlineto{\pgfqpoint{2.707792in}{2.711862in}}%
\pgfusepath{stroke}%
\end{pgfscope}%
\begin{pgfscope}%
\pgfpathrectangle{\pgfqpoint{1.250000in}{1.750000in}}{\pgfqpoint{2.279412in}{2.004545in}}%
\pgfusepath{clip}%
\pgfsetbuttcap%
\pgfsetroundjoin%
\pgfsetlinewidth{1.077231pt}%
\definecolor{currentstroke}{rgb}{0.147607,0.511733,0.557049}%
\pgfsetstrokecolor{currentstroke}%
\pgfsetdash{}{0pt}%
\pgfpathmoveto{\pgfqpoint{2.707792in}{2.711862in}}%
\pgfpathlineto{\pgfqpoint{2.657649in}{2.712696in}}%
\pgfusepath{stroke}%
\end{pgfscope}%
\begin{pgfscope}%
\pgfpathrectangle{\pgfqpoint{1.250000in}{1.750000in}}{\pgfqpoint{2.279412in}{2.004545in}}%
\pgfusepath{clip}%
\pgfsetbuttcap%
\pgfsetroundjoin%
\pgfsetlinewidth{1.409190pt}%
\definecolor{currentstroke}{rgb}{0.162016,0.687316,0.499129}%
\pgfsetstrokecolor{currentstroke}%
\pgfsetdash{}{0pt}%
\pgfpathmoveto{\pgfqpoint{2.657649in}{2.712696in}}%
\pgfpathlineto{\pgfqpoint{2.607511in}{2.713708in}}%
\pgfusepath{stroke}%
\end{pgfscope}%
\begin{pgfscope}%
\pgfpathrectangle{\pgfqpoint{1.250000in}{1.750000in}}{\pgfqpoint{2.279412in}{2.004545in}}%
\pgfusepath{clip}%
\pgfsetbuttcap%
\pgfsetroundjoin%
\pgfsetlinewidth{1.618036pt}%
\definecolor{currentstroke}{rgb}{0.360741,0.785964,0.387814}%
\pgfsetstrokecolor{currentstroke}%
\pgfsetdash{}{0pt}%
\pgfpathmoveto{\pgfqpoint{2.607511in}{2.713708in}}%
\pgfpathlineto{\pgfqpoint{2.557378in}{2.714899in}}%
\pgfusepath{stroke}%
\end{pgfscope}%
\begin{pgfscope}%
\pgfpathrectangle{\pgfqpoint{1.250000in}{1.750000in}}{\pgfqpoint{2.279412in}{2.004545in}}%
\pgfusepath{clip}%
\pgfsetbuttcap%
\pgfsetroundjoin%
\pgfsetlinewidth{1.846552pt}%
\definecolor{currentstroke}{rgb}{0.688944,0.865448,0.182725}%
\pgfsetstrokecolor{currentstroke}%
\pgfsetdash{}{0pt}%
\pgfpathmoveto{\pgfqpoint{2.557378in}{2.714899in}}%
\pgfpathlineto{\pgfqpoint{2.507250in}{2.716229in}}%
\pgfusepath{stroke}%
\end{pgfscope}%
\begin{pgfscope}%
\pgfpathrectangle{\pgfqpoint{1.250000in}{1.750000in}}{\pgfqpoint{2.279412in}{2.004545in}}%
\pgfusepath{clip}%
\pgfsetbuttcap%
\pgfsetroundjoin%
\pgfsetlinewidth{2.123266pt}%
\definecolor{currentstroke}{rgb}{0.993248,0.906157,0.143936}%
\pgfsetstrokecolor{currentstroke}%
\pgfsetdash{}{0pt}%
\pgfpathmoveto{\pgfqpoint{2.507250in}{2.716229in}}%
\pgfpathlineto{\pgfqpoint{2.457130in}{2.717730in}}%
\pgfusepath{stroke}%
\end{pgfscope}%
\begin{pgfscope}%
\pgfpathrectangle{\pgfqpoint{1.250000in}{1.750000in}}{\pgfqpoint{2.279412in}{2.004545in}}%
\pgfusepath{clip}%
\pgfsetbuttcap%
\pgfsetroundjoin%
\pgfsetlinewidth{2.302834pt}%
\definecolor{currentstroke}{rgb}{0.993248,0.906157,0.143936}%
\pgfsetstrokecolor{currentstroke}%
\pgfsetdash{}{0pt}%
\pgfpathmoveto{\pgfqpoint{2.457130in}{2.717730in}}%
\pgfpathlineto{\pgfqpoint{2.407018in}{2.719399in}}%
\pgfusepath{stroke}%
\end{pgfscope}%
\begin{pgfscope}%
\pgfpathrectangle{\pgfqpoint{1.250000in}{1.750000in}}{\pgfqpoint{2.279412in}{2.004545in}}%
\pgfusepath{clip}%
\pgfsetbuttcap%
\pgfsetroundjoin%
\pgfsetlinewidth{2.370850pt}%
\definecolor{currentstroke}{rgb}{0.993248,0.906157,0.143936}%
\pgfsetstrokecolor{currentstroke}%
\pgfsetdash{}{0pt}%
\pgfpathmoveto{\pgfqpoint{2.407018in}{2.719399in}}%
\pgfpathlineto{\pgfqpoint{2.356917in}{2.721244in}}%
\pgfusepath{stroke}%
\end{pgfscope}%
\begin{pgfscope}%
\pgfpathrectangle{\pgfqpoint{1.250000in}{1.750000in}}{\pgfqpoint{2.279412in}{2.004545in}}%
\pgfusepath{clip}%
\pgfsetbuttcap%
\pgfsetroundjoin%
\pgfsetlinewidth{2.339015pt}%
\definecolor{currentstroke}{rgb}{0.993248,0.906157,0.143936}%
\pgfsetstrokecolor{currentstroke}%
\pgfsetdash{}{0pt}%
\pgfpathmoveto{\pgfqpoint{2.356917in}{2.721244in}}%
\pgfpathlineto{\pgfqpoint{2.306825in}{2.723205in}}%
\pgfusepath{stroke}%
\end{pgfscope}%
\begin{pgfscope}%
\pgfpathrectangle{\pgfqpoint{1.250000in}{1.750000in}}{\pgfqpoint{2.279412in}{2.004545in}}%
\pgfusepath{clip}%
\pgfsetbuttcap%
\pgfsetroundjoin%
\pgfsetlinewidth{2.277781pt}%
\definecolor{currentstroke}{rgb}{0.993248,0.906157,0.143936}%
\pgfsetstrokecolor{currentstroke}%
\pgfsetdash{}{0pt}%
\pgfpathmoveto{\pgfqpoint{2.306825in}{2.723205in}}%
\pgfpathlineto{\pgfqpoint{2.256743in}{2.725312in}}%
\pgfusepath{stroke}%
\end{pgfscope}%
\begin{pgfscope}%
\pgfpathrectangle{\pgfqpoint{1.250000in}{1.750000in}}{\pgfqpoint{2.279412in}{2.004545in}}%
\pgfusepath{clip}%
\pgfsetbuttcap%
\pgfsetroundjoin%
\pgfsetlinewidth{0.327881pt}%
\definecolor{currentstroke}{rgb}{0.271305,0.019942,0.347269}%
\pgfsetstrokecolor{currentstroke}%
\pgfsetdash{}{0pt}%
\pgfpathmoveto{\pgfqpoint{3.159084in}{2.797380in}}%
\pgfpathlineto{\pgfqpoint{3.108942in}{2.796819in}}%
\pgfusepath{stroke}%
\end{pgfscope}%
\begin{pgfscope}%
\pgfpathrectangle{\pgfqpoint{1.250000in}{1.750000in}}{\pgfqpoint{2.279412in}{2.004545in}}%
\pgfusepath{clip}%
\pgfsetbuttcap%
\pgfsetroundjoin%
\pgfsetlinewidth{0.332204pt}%
\definecolor{currentstroke}{rgb}{0.272594,0.025563,0.353093}%
\pgfsetstrokecolor{currentstroke}%
\pgfsetdash{}{0pt}%
\pgfpathmoveto{\pgfqpoint{3.108942in}{2.796819in}}%
\pgfpathlineto{\pgfqpoint{3.058795in}{2.796324in}}%
\pgfusepath{stroke}%
\end{pgfscope}%
\begin{pgfscope}%
\pgfpathrectangle{\pgfqpoint{1.250000in}{1.750000in}}{\pgfqpoint{2.279412in}{2.004545in}}%
\pgfusepath{clip}%
\pgfsetbuttcap%
\pgfsetroundjoin%
\pgfsetlinewidth{0.338061pt}%
\definecolor{currentstroke}{rgb}{0.273809,0.031497,0.358853}%
\pgfsetstrokecolor{currentstroke}%
\pgfsetdash{}{0pt}%
\pgfpathmoveto{\pgfqpoint{3.058795in}{2.796324in}}%
\pgfpathlineto{\pgfqpoint{3.008645in}{2.796064in}}%
\pgfusepath{stroke}%
\end{pgfscope}%
\begin{pgfscope}%
\pgfpathrectangle{\pgfqpoint{1.250000in}{1.750000in}}{\pgfqpoint{2.279412in}{2.004545in}}%
\pgfusepath{clip}%
\pgfsetbuttcap%
\pgfsetroundjoin%
\pgfsetlinewidth{0.362589pt}%
\definecolor{currentstroke}{rgb}{0.277018,0.050344,0.375715}%
\pgfsetstrokecolor{currentstroke}%
\pgfsetdash{}{0pt}%
\pgfpathmoveto{\pgfqpoint{3.008645in}{2.796064in}}%
\pgfpathlineto{\pgfqpoint{2.958494in}{2.796051in}}%
\pgfusepath{stroke}%
\end{pgfscope}%
\begin{pgfscope}%
\pgfpathrectangle{\pgfqpoint{1.250000in}{1.750000in}}{\pgfqpoint{2.279412in}{2.004545in}}%
\pgfusepath{clip}%
\pgfsetbuttcap%
\pgfsetroundjoin%
\pgfsetlinewidth{0.395319pt}%
\definecolor{currentstroke}{rgb}{0.280894,0.078907,0.402329}%
\pgfsetstrokecolor{currentstroke}%
\pgfsetdash{}{0pt}%
\pgfpathmoveto{\pgfqpoint{2.958494in}{2.796051in}}%
\pgfpathlineto{\pgfqpoint{2.908345in}{2.795939in}}%
\pgfusepath{stroke}%
\end{pgfscope}%
\begin{pgfscope}%
\pgfpathrectangle{\pgfqpoint{1.250000in}{1.750000in}}{\pgfqpoint{2.279412in}{2.004545in}}%
\pgfusepath{clip}%
\pgfsetbuttcap%
\pgfsetroundjoin%
\pgfsetlinewidth{0.440638pt}%
\definecolor{currentstroke}{rgb}{0.283197,0.115680,0.436115}%
\pgfsetstrokecolor{currentstroke}%
\pgfsetdash{}{0pt}%
\pgfpathmoveto{\pgfqpoint{2.908345in}{2.795939in}}%
\pgfpathlineto{\pgfqpoint{2.858196in}{2.795562in}}%
\pgfusepath{stroke}%
\end{pgfscope}%
\begin{pgfscope}%
\pgfpathrectangle{\pgfqpoint{1.250000in}{1.750000in}}{\pgfqpoint{2.279412in}{2.004545in}}%
\pgfusepath{clip}%
\pgfsetbuttcap%
\pgfsetroundjoin%
\pgfsetlinewidth{0.510114pt}%
\definecolor{currentstroke}{rgb}{0.280255,0.165693,0.476498}%
\pgfsetstrokecolor{currentstroke}%
\pgfsetdash{}{0pt}%
\pgfpathmoveto{\pgfqpoint{2.858196in}{2.795562in}}%
\pgfpathlineto{\pgfqpoint{2.808046in}{2.795197in}}%
\pgfusepath{stroke}%
\end{pgfscope}%
\begin{pgfscope}%
\pgfpathrectangle{\pgfqpoint{1.250000in}{1.750000in}}{\pgfqpoint{2.279412in}{2.004545in}}%
\pgfusepath{clip}%
\pgfsetbuttcap%
\pgfsetroundjoin%
\pgfsetlinewidth{0.619922pt}%
\definecolor{currentstroke}{rgb}{0.262138,0.242286,0.520837}%
\pgfsetstrokecolor{currentstroke}%
\pgfsetdash{}{0pt}%
\pgfpathmoveto{\pgfqpoint{2.808046in}{2.795197in}}%
\pgfpathlineto{\pgfqpoint{2.757897in}{2.794731in}}%
\pgfusepath{stroke}%
\end{pgfscope}%
\begin{pgfscope}%
\pgfpathrectangle{\pgfqpoint{1.250000in}{1.750000in}}{\pgfqpoint{2.279412in}{2.004545in}}%
\pgfusepath{clip}%
\pgfsetbuttcap%
\pgfsetroundjoin%
\pgfsetlinewidth{0.797722pt}%
\definecolor{currentstroke}{rgb}{0.214298,0.355619,0.551184}%
\pgfsetstrokecolor{currentstroke}%
\pgfsetdash{}{0pt}%
\pgfpathmoveto{\pgfqpoint{2.757897in}{2.794731in}}%
\pgfpathlineto{\pgfqpoint{2.707750in}{2.794110in}}%
\pgfusepath{stroke}%
\end{pgfscope}%
\begin{pgfscope}%
\pgfpathrectangle{\pgfqpoint{1.250000in}{1.750000in}}{\pgfqpoint{2.279412in}{2.004545in}}%
\pgfusepath{clip}%
\pgfsetbuttcap%
\pgfsetroundjoin%
\pgfsetlinewidth{1.071473pt}%
\definecolor{currentstroke}{rgb}{0.149039,0.508051,0.557250}%
\pgfsetstrokecolor{currentstroke}%
\pgfsetdash{}{0pt}%
\pgfpathmoveto{\pgfqpoint{2.707750in}{2.794110in}}%
\pgfpathlineto{\pgfqpoint{2.657611in}{2.793139in}}%
\pgfusepath{stroke}%
\end{pgfscope}%
\begin{pgfscope}%
\pgfpathrectangle{\pgfqpoint{1.250000in}{1.750000in}}{\pgfqpoint{2.279412in}{2.004545in}}%
\pgfusepath{clip}%
\pgfsetbuttcap%
\pgfsetroundjoin%
\pgfsetlinewidth{1.333162pt}%
\definecolor{currentstroke}{rgb}{0.128087,0.647749,0.523491}%
\pgfsetstrokecolor{currentstroke}%
\pgfsetdash{}{0pt}%
\pgfpathmoveto{\pgfqpoint{2.657611in}{2.793139in}}%
\pgfpathlineto{\pgfqpoint{2.607482in}{2.791806in}}%
\pgfusepath{stroke}%
\end{pgfscope}%
\begin{pgfscope}%
\pgfpathrectangle{\pgfqpoint{1.250000in}{1.750000in}}{\pgfqpoint{2.279412in}{2.004545in}}%
\pgfusepath{clip}%
\pgfsetbuttcap%
\pgfsetroundjoin%
\pgfsetlinewidth{0.314298pt}%
\definecolor{currentstroke}{rgb}{0.268510,0.009605,0.335427}%
\pgfsetstrokecolor{currentstroke}%
\pgfsetdash{}{0pt}%
\pgfpathmoveto{\pgfqpoint{3.159084in}{2.932700in}}%
\pgfpathlineto{\pgfqpoint{3.109043in}{2.934071in}}%
\pgfusepath{stroke}%
\end{pgfscope}%
\begin{pgfscope}%
\pgfpathrectangle{\pgfqpoint{1.250000in}{1.750000in}}{\pgfqpoint{2.279412in}{2.004545in}}%
\pgfusepath{clip}%
\pgfsetbuttcap%
\pgfsetroundjoin%
\pgfsetlinewidth{0.328112pt}%
\definecolor{currentstroke}{rgb}{0.271305,0.019942,0.347269}%
\pgfsetstrokecolor{currentstroke}%
\pgfsetdash{}{0pt}%
\pgfpathmoveto{\pgfqpoint{3.109043in}{2.934071in}}%
\pgfpathlineto{\pgfqpoint{3.058925in}{2.933217in}}%
\pgfusepath{stroke}%
\end{pgfscope}%
\begin{pgfscope}%
\pgfpathrectangle{\pgfqpoint{1.250000in}{1.750000in}}{\pgfqpoint{2.279412in}{2.004545in}}%
\pgfusepath{clip}%
\pgfsetbuttcap%
\pgfsetroundjoin%
\pgfsetlinewidth{0.345706pt}%
\definecolor{currentstroke}{rgb}{0.274952,0.037752,0.364543}%
\pgfsetstrokecolor{currentstroke}%
\pgfsetdash{}{0pt}%
\pgfpathmoveto{\pgfqpoint{3.058925in}{2.933217in}}%
\pgfpathlineto{\pgfqpoint{3.008826in}{2.931363in}}%
\pgfusepath{stroke}%
\end{pgfscope}%
\begin{pgfscope}%
\pgfpathrectangle{\pgfqpoint{1.250000in}{1.750000in}}{\pgfqpoint{2.279412in}{2.004545in}}%
\pgfusepath{clip}%
\pgfsetbuttcap%
\pgfsetroundjoin%
\pgfsetlinewidth{0.359223pt}%
\definecolor{currentstroke}{rgb}{0.277018,0.050344,0.375715}%
\pgfsetstrokecolor{currentstroke}%
\pgfsetdash{}{0pt}%
\pgfpathmoveto{\pgfqpoint{3.008826in}{2.931363in}}%
\pgfpathlineto{\pgfqpoint{2.958711in}{2.929818in}}%
\pgfusepath{stroke}%
\end{pgfscope}%
\begin{pgfscope}%
\pgfpathrectangle{\pgfqpoint{1.250000in}{1.750000in}}{\pgfqpoint{2.279412in}{2.004545in}}%
\pgfusepath{clip}%
\pgfsetbuttcap%
\pgfsetroundjoin%
\pgfsetlinewidth{0.383343pt}%
\definecolor{currentstroke}{rgb}{0.279566,0.067836,0.391917}%
\pgfsetstrokecolor{currentstroke}%
\pgfsetdash{}{0pt}%
\pgfpathmoveto{\pgfqpoint{2.958711in}{2.929818in}}%
\pgfpathlineto{\pgfqpoint{2.908573in}{2.928796in}}%
\pgfusepath{stroke}%
\end{pgfscope}%
\begin{pgfscope}%
\pgfpathrectangle{\pgfqpoint{1.250000in}{1.750000in}}{\pgfqpoint{2.279412in}{2.004545in}}%
\pgfusepath{clip}%
\pgfsetbuttcap%
\pgfsetroundjoin%
\pgfsetlinewidth{0.415120pt}%
\definecolor{currentstroke}{rgb}{0.282327,0.094955,0.417331}%
\pgfsetstrokecolor{currentstroke}%
\pgfsetdash{}{0pt}%
\pgfpathmoveto{\pgfqpoint{2.908573in}{2.928796in}}%
\pgfpathlineto{\pgfqpoint{2.858447in}{2.927405in}}%
\pgfusepath{stroke}%
\end{pgfscope}%
\begin{pgfscope}%
\pgfpathrectangle{\pgfqpoint{1.250000in}{1.750000in}}{\pgfqpoint{2.279412in}{2.004545in}}%
\pgfusepath{clip}%
\pgfsetbuttcap%
\pgfsetroundjoin%
\pgfsetlinewidth{0.446007pt}%
\definecolor{currentstroke}{rgb}{0.283229,0.120777,0.440584}%
\pgfsetstrokecolor{currentstroke}%
\pgfsetdash{}{0pt}%
\pgfpathmoveto{\pgfqpoint{2.858447in}{2.927405in}}%
\pgfpathlineto{\pgfqpoint{2.808334in}{2.925671in}}%
\pgfusepath{stroke}%
\end{pgfscope}%
\begin{pgfscope}%
\pgfpathrectangle{\pgfqpoint{1.250000in}{1.750000in}}{\pgfqpoint{2.279412in}{2.004545in}}%
\pgfusepath{clip}%
\pgfsetbuttcap%
\pgfsetroundjoin%
\pgfsetlinewidth{0.514840pt}%
\definecolor{currentstroke}{rgb}{0.279574,0.170599,0.479997}%
\pgfsetstrokecolor{currentstroke}%
\pgfsetdash{}{0pt}%
\pgfpathmoveto{\pgfqpoint{2.808334in}{2.925671in}}%
\pgfpathlineto{\pgfqpoint{2.758232in}{2.923727in}}%
\pgfusepath{stroke}%
\end{pgfscope}%
\begin{pgfscope}%
\pgfpathrectangle{\pgfqpoint{1.250000in}{1.750000in}}{\pgfqpoint{2.279412in}{2.004545in}}%
\pgfusepath{clip}%
\pgfsetbuttcap%
\pgfsetroundjoin%
\pgfsetlinewidth{0.594575pt}%
\definecolor{currentstroke}{rgb}{0.267968,0.223549,0.512008}%
\pgfsetstrokecolor{currentstroke}%
\pgfsetdash{}{0pt}%
\pgfpathmoveto{\pgfqpoint{2.758232in}{2.923727in}}%
\pgfpathlineto{\pgfqpoint{2.708161in}{2.921253in}}%
\pgfusepath{stroke}%
\end{pgfscope}%
\begin{pgfscope}%
\pgfpathrectangle{\pgfqpoint{1.250000in}{1.750000in}}{\pgfqpoint{2.279412in}{2.004545in}}%
\pgfusepath{clip}%
\pgfsetbuttcap%
\pgfsetroundjoin%
\pgfsetlinewidth{0.708632pt}%
\definecolor{currentstroke}{rgb}{0.239346,0.300855,0.540844}%
\pgfsetstrokecolor{currentstroke}%
\pgfsetdash{}{0pt}%
\pgfpathmoveto{\pgfqpoint{2.708161in}{2.921253in}}%
\pgfpathlineto{\pgfqpoint{2.658161in}{2.917868in}}%
\pgfusepath{stroke}%
\end{pgfscope}%
\begin{pgfscope}%
\pgfpathrectangle{\pgfqpoint{1.250000in}{1.750000in}}{\pgfqpoint{2.279412in}{2.004545in}}%
\pgfusepath{clip}%
\pgfsetbuttcap%
\pgfsetroundjoin%
\pgfsetlinewidth{0.870849pt}%
\definecolor{currentstroke}{rgb}{0.194100,0.399323,0.555565}%
\pgfsetstrokecolor{currentstroke}%
\pgfsetdash{}{0pt}%
\pgfpathmoveto{\pgfqpoint{2.658161in}{2.917868in}}%
\pgfpathlineto{\pgfqpoint{2.608282in}{2.913319in}}%
\pgfusepath{stroke}%
\end{pgfscope}%
\begin{pgfscope}%
\pgfpathrectangle{\pgfqpoint{1.250000in}{1.750000in}}{\pgfqpoint{2.279412in}{2.004545in}}%
\pgfusepath{clip}%
\pgfsetbuttcap%
\pgfsetroundjoin%
\pgfsetlinewidth{1.045579pt}%
\definecolor{currentstroke}{rgb}{0.154815,0.493313,0.557840}%
\pgfsetstrokecolor{currentstroke}%
\pgfsetdash{}{0pt}%
\pgfpathmoveto{\pgfqpoint{2.608282in}{2.913319in}}%
\pgfpathlineto{\pgfqpoint{2.558537in}{2.907733in}}%
\pgfusepath{stroke}%
\end{pgfscope}%
\begin{pgfscope}%
\pgfpathrectangle{\pgfqpoint{1.250000in}{1.750000in}}{\pgfqpoint{2.279412in}{2.004545in}}%
\pgfusepath{clip}%
\pgfsetbuttcap%
\pgfsetroundjoin%
\pgfsetlinewidth{1.252847pt}%
\definecolor{currentstroke}{rgb}{0.119738,0.603785,0.541400}%
\pgfsetstrokecolor{currentstroke}%
\pgfsetdash{}{0pt}%
\pgfpathmoveto{\pgfqpoint{2.558537in}{2.907733in}}%
\pgfpathlineto{\pgfqpoint{2.508944in}{2.901200in}}%
\pgfusepath{stroke}%
\end{pgfscope}%
\begin{pgfscope}%
\pgfpathrectangle{\pgfqpoint{1.250000in}{1.750000in}}{\pgfqpoint{2.279412in}{2.004545in}}%
\pgfusepath{clip}%
\pgfsetbuttcap%
\pgfsetroundjoin%
\pgfsetlinewidth{1.265483pt}%
\definecolor{currentstroke}{rgb}{0.119423,0.611141,0.538982}%
\pgfsetstrokecolor{currentstroke}%
\pgfsetdash{}{0pt}%
\pgfpathmoveto{\pgfqpoint{2.508944in}{2.901200in}}%
\pgfpathlineto{\pgfqpoint{2.459570in}{2.893503in}}%
\pgfusepath{stroke}%
\end{pgfscope}%
\begin{pgfscope}%
\pgfpathrectangle{\pgfqpoint{1.250000in}{1.750000in}}{\pgfqpoint{2.279412in}{2.004545in}}%
\pgfusepath{clip}%
\pgfsetbuttcap%
\pgfsetroundjoin%
\pgfsetlinewidth{1.254705pt}%
\definecolor{currentstroke}{rgb}{0.119738,0.603785,0.541400}%
\pgfsetstrokecolor{currentstroke}%
\pgfsetdash{}{0pt}%
\pgfpathmoveto{\pgfqpoint{2.459570in}{2.893503in}}%
\pgfpathlineto{\pgfqpoint{2.410435in}{2.884690in}}%
\pgfusepath{stroke}%
\end{pgfscope}%
\begin{pgfscope}%
\pgfpathrectangle{\pgfqpoint{1.250000in}{1.750000in}}{\pgfqpoint{2.279412in}{2.004545in}}%
\pgfusepath{clip}%
\pgfsetbuttcap%
\pgfsetroundjoin%
\pgfsetlinewidth{1.650091pt}%
\definecolor{currentstroke}{rgb}{0.404001,0.800275,0.362552}%
\pgfsetstrokecolor{currentstroke}%
\pgfsetdash{}{0pt}%
\pgfpathmoveto{\pgfqpoint{2.410435in}{2.884690in}}%
\pgfpathlineto{\pgfqpoint{2.361531in}{2.874940in}}%
\pgfusepath{stroke}%
\end{pgfscope}%
\begin{pgfscope}%
\pgfpathrectangle{\pgfqpoint{1.250000in}{1.750000in}}{\pgfqpoint{2.279412in}{2.004545in}}%
\pgfusepath{clip}%
\pgfsetbuttcap%
\pgfsetroundjoin%
\pgfsetlinewidth{0.312280pt}%
\definecolor{currentstroke}{rgb}{0.268510,0.009605,0.335427}%
\pgfsetstrokecolor{currentstroke}%
\pgfsetdash{}{0pt}%
\pgfpathmoveto{\pgfqpoint{3.159084in}{2.977807in}}%
\pgfpathlineto{\pgfqpoint{3.108995in}{2.976406in}}%
\pgfusepath{stroke}%
\end{pgfscope}%
\begin{pgfscope}%
\pgfpathrectangle{\pgfqpoint{1.250000in}{1.750000in}}{\pgfqpoint{2.279412in}{2.004545in}}%
\pgfusepath{clip}%
\pgfsetbuttcap%
\pgfsetroundjoin%
\pgfsetlinewidth{0.328858pt}%
\definecolor{currentstroke}{rgb}{0.272594,0.025563,0.353093}%
\pgfsetstrokecolor{currentstroke}%
\pgfsetdash{}{0pt}%
\pgfpathmoveto{\pgfqpoint{3.108995in}{2.976406in}}%
\pgfpathlineto{\pgfqpoint{3.058857in}{2.975895in}}%
\pgfusepath{stroke}%
\end{pgfscope}%
\begin{pgfscope}%
\pgfpathrectangle{\pgfqpoint{1.250000in}{1.750000in}}{\pgfqpoint{2.279412in}{2.004545in}}%
\pgfusepath{clip}%
\pgfsetbuttcap%
\pgfsetroundjoin%
\pgfsetlinewidth{0.335713pt}%
\definecolor{currentstroke}{rgb}{0.273809,0.031497,0.358853}%
\pgfsetstrokecolor{currentstroke}%
\pgfsetdash{}{0pt}%
\pgfpathmoveto{\pgfqpoint{3.058857in}{2.975895in}}%
\pgfpathlineto{\pgfqpoint{3.008735in}{2.974478in}}%
\pgfusepath{stroke}%
\end{pgfscope}%
\begin{pgfscope}%
\pgfpathrectangle{\pgfqpoint{1.250000in}{1.750000in}}{\pgfqpoint{2.279412in}{2.004545in}}%
\pgfusepath{clip}%
\pgfsetbuttcap%
\pgfsetroundjoin%
\pgfsetlinewidth{0.354652pt}%
\definecolor{currentstroke}{rgb}{0.276022,0.044167,0.370164}%
\pgfsetstrokecolor{currentstroke}%
\pgfsetdash{}{0pt}%
\pgfpathmoveto{\pgfqpoint{3.008735in}{2.974478in}}%
\pgfpathlineto{\pgfqpoint{2.958606in}{2.973282in}}%
\pgfusepath{stroke}%
\end{pgfscope}%
\begin{pgfscope}%
\pgfpathrectangle{\pgfqpoint{1.250000in}{1.750000in}}{\pgfqpoint{2.279412in}{2.004545in}}%
\pgfusepath{clip}%
\pgfsetbuttcap%
\pgfsetroundjoin%
\pgfsetlinewidth{0.370246pt}%
\definecolor{currentstroke}{rgb}{0.278791,0.062145,0.386592}%
\pgfsetstrokecolor{currentstroke}%
\pgfsetdash{}{0pt}%
\pgfpathmoveto{\pgfqpoint{2.958606in}{2.973282in}}%
\pgfpathlineto{\pgfqpoint{2.908474in}{2.972112in}}%
\pgfusepath{stroke}%
\end{pgfscope}%
\begin{pgfscope}%
\pgfpathrectangle{\pgfqpoint{1.250000in}{1.750000in}}{\pgfqpoint{2.279412in}{2.004545in}}%
\pgfusepath{clip}%
\pgfsetbuttcap%
\pgfsetroundjoin%
\pgfsetlinewidth{0.403051pt}%
\definecolor{currentstroke}{rgb}{0.281446,0.084320,0.407414}%
\pgfsetstrokecolor{currentstroke}%
\pgfsetdash{}{0pt}%
\pgfpathmoveto{\pgfqpoint{2.908474in}{2.972112in}}%
\pgfpathlineto{\pgfqpoint{2.858362in}{2.970377in}}%
\pgfusepath{stroke}%
\end{pgfscope}%
\begin{pgfscope}%
\pgfpathrectangle{\pgfqpoint{1.250000in}{1.750000in}}{\pgfqpoint{2.279412in}{2.004545in}}%
\pgfusepath{clip}%
\pgfsetbuttcap%
\pgfsetroundjoin%
\pgfsetlinewidth{0.434540pt}%
\definecolor{currentstroke}{rgb}{0.283091,0.110553,0.431554}%
\pgfsetstrokecolor{currentstroke}%
\pgfsetdash{}{0pt}%
\pgfpathmoveto{\pgfqpoint{2.858362in}{2.970377in}}%
\pgfpathlineto{\pgfqpoint{2.808263in}{2.968361in}}%
\pgfusepath{stroke}%
\end{pgfscope}%
\begin{pgfscope}%
\pgfpathrectangle{\pgfqpoint{1.250000in}{1.750000in}}{\pgfqpoint{2.279412in}{2.004545in}}%
\pgfusepath{clip}%
\pgfsetbuttcap%
\pgfsetroundjoin%
\pgfsetlinewidth{0.482235pt}%
\definecolor{currentstroke}{rgb}{0.282290,0.145912,0.461510}%
\pgfsetstrokecolor{currentstroke}%
\pgfsetdash{}{0pt}%
\pgfpathmoveto{\pgfqpoint{2.808263in}{2.968361in}}%
\pgfpathlineto{\pgfqpoint{2.758192in}{2.965884in}}%
\pgfusepath{stroke}%
\end{pgfscope}%
\begin{pgfscope}%
\pgfpathrectangle{\pgfqpoint{1.250000in}{1.750000in}}{\pgfqpoint{2.279412in}{2.004545in}}%
\pgfusepath{clip}%
\pgfsetbuttcap%
\pgfsetroundjoin%
\pgfsetlinewidth{0.554103pt}%
\definecolor{currentstroke}{rgb}{0.275191,0.194905,0.496005}%
\pgfsetstrokecolor{currentstroke}%
\pgfsetdash{}{0pt}%
\pgfpathmoveto{\pgfqpoint{2.758192in}{2.965884in}}%
\pgfpathlineto{\pgfqpoint{2.708160in}{2.962856in}}%
\pgfusepath{stroke}%
\end{pgfscope}%
\begin{pgfscope}%
\pgfpathrectangle{\pgfqpoint{1.250000in}{1.750000in}}{\pgfqpoint{2.279412in}{2.004545in}}%
\pgfusepath{clip}%
\pgfsetbuttcap%
\pgfsetroundjoin%
\pgfsetlinewidth{0.619956pt}%
\definecolor{currentstroke}{rgb}{0.262138,0.242286,0.520837}%
\pgfsetstrokecolor{currentstroke}%
\pgfsetdash{}{0pt}%
\pgfpathmoveto{\pgfqpoint{2.708160in}{2.962856in}}%
\pgfpathlineto{\pgfqpoint{2.658204in}{2.959012in}}%
\pgfusepath{stroke}%
\end{pgfscope}%
\begin{pgfscope}%
\pgfpathrectangle{\pgfqpoint{1.250000in}{1.750000in}}{\pgfqpoint{2.279412in}{2.004545in}}%
\pgfusepath{clip}%
\pgfsetbuttcap%
\pgfsetroundjoin%
\pgfsetlinewidth{0.321538pt}%
\definecolor{currentstroke}{rgb}{0.269944,0.014625,0.341379}%
\pgfsetstrokecolor{currentstroke}%
\pgfsetdash{}{0pt}%
\pgfpathmoveto{\pgfqpoint{3.107792in}{2.301205in}}%
\pgfpathlineto{\pgfqpoint{3.057727in}{2.303051in}}%
\pgfusepath{stroke}%
\end{pgfscope}%
\begin{pgfscope}%
\pgfpathrectangle{\pgfqpoint{1.250000in}{1.750000in}}{\pgfqpoint{2.279412in}{2.004545in}}%
\pgfusepath{clip}%
\pgfsetbuttcap%
\pgfsetroundjoin%
\pgfsetlinewidth{0.324639pt}%
\definecolor{currentstroke}{rgb}{0.271305,0.019942,0.347269}%
\pgfsetstrokecolor{currentstroke}%
\pgfsetdash{}{0pt}%
\pgfpathmoveto{\pgfqpoint{3.057727in}{2.303051in}}%
\pgfpathlineto{\pgfqpoint{3.007662in}{2.304856in}}%
\pgfusepath{stroke}%
\end{pgfscope}%
\begin{pgfscope}%
\pgfpathrectangle{\pgfqpoint{1.250000in}{1.750000in}}{\pgfqpoint{2.279412in}{2.004545in}}%
\pgfusepath{clip}%
\pgfsetbuttcap%
\pgfsetroundjoin%
\pgfsetlinewidth{0.326631pt}%
\definecolor{currentstroke}{rgb}{0.271305,0.019942,0.347269}%
\pgfsetstrokecolor{currentstroke}%
\pgfsetdash{}{0pt}%
\pgfpathmoveto{\pgfqpoint{3.007662in}{2.304856in}}%
\pgfpathlineto{\pgfqpoint{2.957554in}{2.306464in}}%
\pgfusepath{stroke}%
\end{pgfscope}%
\begin{pgfscope}%
\pgfpathrectangle{\pgfqpoint{1.250000in}{1.750000in}}{\pgfqpoint{2.279412in}{2.004545in}}%
\pgfusepath{clip}%
\pgfsetbuttcap%
\pgfsetroundjoin%
\pgfsetlinewidth{0.326567pt}%
\definecolor{currentstroke}{rgb}{0.271305,0.019942,0.347269}%
\pgfsetstrokecolor{currentstroke}%
\pgfsetdash{}{0pt}%
\pgfpathmoveto{\pgfqpoint{2.957554in}{2.306464in}}%
\pgfpathlineto{\pgfqpoint{2.907480in}{2.308767in}}%
\pgfusepath{stroke}%
\end{pgfscope}%
\begin{pgfscope}%
\pgfpathrectangle{\pgfqpoint{1.250000in}{1.750000in}}{\pgfqpoint{2.279412in}{2.004545in}}%
\pgfusepath{clip}%
\pgfsetbuttcap%
\pgfsetroundjoin%
\pgfsetlinewidth{0.345446pt}%
\definecolor{currentstroke}{rgb}{0.274952,0.037752,0.364543}%
\pgfsetstrokecolor{currentstroke}%
\pgfsetdash{}{0pt}%
\pgfpathmoveto{\pgfqpoint{2.907480in}{2.308767in}}%
\pgfpathlineto{\pgfqpoint{2.857488in}{2.312160in}}%
\pgfusepath{stroke}%
\end{pgfscope}%
\begin{pgfscope}%
\pgfpathrectangle{\pgfqpoint{1.250000in}{1.750000in}}{\pgfqpoint{2.279412in}{2.004545in}}%
\pgfusepath{clip}%
\pgfsetbuttcap%
\pgfsetroundjoin%
\pgfsetlinewidth{0.318152pt}%
\definecolor{currentstroke}{rgb}{0.269944,0.014625,0.341379}%
\pgfsetstrokecolor{currentstroke}%
\pgfsetdash{}{0pt}%
\pgfpathmoveto{\pgfqpoint{3.107792in}{2.391418in}}%
\pgfpathlineto{\pgfqpoint{3.057738in}{2.393953in}}%
\pgfusepath{stroke}%
\end{pgfscope}%
\begin{pgfscope}%
\pgfpathrectangle{\pgfqpoint{1.250000in}{1.750000in}}{\pgfqpoint{2.279412in}{2.004545in}}%
\pgfusepath{clip}%
\pgfsetbuttcap%
\pgfsetroundjoin%
\pgfsetlinewidth{0.324016pt}%
\definecolor{currentstroke}{rgb}{0.271305,0.019942,0.347269}%
\pgfsetstrokecolor{currentstroke}%
\pgfsetdash{}{0pt}%
\pgfpathmoveto{\pgfqpoint{3.057738in}{2.393953in}}%
\pgfpathlineto{\pgfqpoint{3.007646in}{2.395855in}}%
\pgfusepath{stroke}%
\end{pgfscope}%
\begin{pgfscope}%
\pgfpathrectangle{\pgfqpoint{1.250000in}{1.750000in}}{\pgfqpoint{2.279412in}{2.004545in}}%
\pgfusepath{clip}%
\pgfsetbuttcap%
\pgfsetroundjoin%
\pgfsetlinewidth{0.341073pt}%
\definecolor{currentstroke}{rgb}{0.273809,0.031497,0.358853}%
\pgfsetstrokecolor{currentstroke}%
\pgfsetdash{}{0pt}%
\pgfpathmoveto{\pgfqpoint{3.007646in}{2.395855in}}%
\pgfpathlineto{\pgfqpoint{2.957524in}{2.396522in}}%
\pgfusepath{stroke}%
\end{pgfscope}%
\begin{pgfscope}%
\pgfpathrectangle{\pgfqpoint{1.250000in}{1.750000in}}{\pgfqpoint{2.279412in}{2.004545in}}%
\pgfusepath{clip}%
\pgfsetbuttcap%
\pgfsetroundjoin%
\pgfsetlinewidth{0.350454pt}%
\definecolor{currentstroke}{rgb}{0.276022,0.044167,0.370164}%
\pgfsetstrokecolor{currentstroke}%
\pgfsetdash{}{0pt}%
\pgfpathmoveto{\pgfqpoint{2.957524in}{2.396522in}}%
\pgfpathlineto{\pgfqpoint{2.907400in}{2.397277in}}%
\pgfusepath{stroke}%
\end{pgfscope}%
\begin{pgfscope}%
\pgfpathrectangle{\pgfqpoint{1.250000in}{1.750000in}}{\pgfqpoint{2.279412in}{2.004545in}}%
\pgfusepath{clip}%
\pgfsetbuttcap%
\pgfsetroundjoin%
\pgfsetlinewidth{0.368328pt}%
\definecolor{currentstroke}{rgb}{0.277941,0.056324,0.381191}%
\pgfsetstrokecolor{currentstroke}%
\pgfsetdash{}{0pt}%
\pgfpathmoveto{\pgfqpoint{2.907400in}{2.397277in}}%
\pgfpathlineto{\pgfqpoint{2.857300in}{2.399232in}}%
\pgfusepath{stroke}%
\end{pgfscope}%
\begin{pgfscope}%
\pgfpathrectangle{\pgfqpoint{1.250000in}{1.750000in}}{\pgfqpoint{2.279412in}{2.004545in}}%
\pgfusepath{clip}%
\pgfsetbuttcap%
\pgfsetroundjoin%
\pgfsetlinewidth{0.379754pt}%
\definecolor{currentstroke}{rgb}{0.279566,0.067836,0.391917}%
\pgfsetstrokecolor{currentstroke}%
\pgfsetdash{}{0pt}%
\pgfpathmoveto{\pgfqpoint{2.857300in}{2.399232in}}%
\pgfpathlineto{\pgfqpoint{2.807238in}{2.401843in}}%
\pgfusepath{stroke}%
\end{pgfscope}%
\begin{pgfscope}%
\pgfpathrectangle{\pgfqpoint{1.250000in}{1.750000in}}{\pgfqpoint{2.279412in}{2.004545in}}%
\pgfusepath{clip}%
\pgfsetbuttcap%
\pgfsetroundjoin%
\pgfsetlinewidth{0.417507pt}%
\definecolor{currentstroke}{rgb}{0.282327,0.094955,0.417331}%
\pgfsetstrokecolor{currentstroke}%
\pgfsetdash{}{0pt}%
\pgfpathmoveto{\pgfqpoint{2.807238in}{2.401843in}}%
\pgfpathlineto{\pgfqpoint{2.757267in}{2.405427in}}%
\pgfusepath{stroke}%
\end{pgfscope}%
\begin{pgfscope}%
\pgfpathrectangle{\pgfqpoint{1.250000in}{1.750000in}}{\pgfqpoint{2.279412in}{2.004545in}}%
\pgfusepath{clip}%
\pgfsetbuttcap%
\pgfsetroundjoin%
\pgfsetlinewidth{0.332518pt}%
\definecolor{currentstroke}{rgb}{0.272594,0.025563,0.353093}%
\pgfsetstrokecolor{currentstroke}%
\pgfsetdash{}{0pt}%
\pgfpathmoveto{\pgfqpoint{3.107792in}{2.752273in}}%
\pgfpathlineto{\pgfqpoint{3.057656in}{2.751713in}}%
\pgfusepath{stroke}%
\end{pgfscope}%
\begin{pgfscope}%
\pgfpathrectangle{\pgfqpoint{1.250000in}{1.750000in}}{\pgfqpoint{2.279412in}{2.004545in}}%
\pgfusepath{clip}%
\pgfsetbuttcap%
\pgfsetroundjoin%
\pgfsetlinewidth{0.344625pt}%
\definecolor{currentstroke}{rgb}{0.274952,0.037752,0.364543}%
\pgfsetstrokecolor{currentstroke}%
\pgfsetdash{}{0pt}%
\pgfpathmoveto{\pgfqpoint{3.057656in}{2.751713in}}%
\pgfpathlineto{\pgfqpoint{3.007512in}{2.751494in}}%
\pgfusepath{stroke}%
\end{pgfscope}%
\begin{pgfscope}%
\pgfpathrectangle{\pgfqpoint{1.250000in}{1.750000in}}{\pgfqpoint{2.279412in}{2.004545in}}%
\pgfusepath{clip}%
\pgfsetbuttcap%
\pgfsetroundjoin%
\pgfsetlinewidth{0.366257pt}%
\definecolor{currentstroke}{rgb}{0.277941,0.056324,0.381191}%
\pgfsetstrokecolor{currentstroke}%
\pgfsetdash{}{0pt}%
\pgfpathmoveto{\pgfqpoint{3.007512in}{2.751494in}}%
\pgfpathlineto{\pgfqpoint{2.957364in}{2.751594in}}%
\pgfusepath{stroke}%
\end{pgfscope}%
\begin{pgfscope}%
\pgfpathrectangle{\pgfqpoint{1.250000in}{1.750000in}}{\pgfqpoint{2.279412in}{2.004545in}}%
\pgfusepath{clip}%
\pgfsetbuttcap%
\pgfsetroundjoin%
\pgfsetlinewidth{0.393770pt}%
\definecolor{currentstroke}{rgb}{0.280894,0.078907,0.402329}%
\pgfsetstrokecolor{currentstroke}%
\pgfsetdash{}{0pt}%
\pgfpathmoveto{\pgfqpoint{2.957364in}{2.751594in}}%
\pgfpathlineto{\pgfqpoint{2.907214in}{2.751722in}}%
\pgfusepath{stroke}%
\end{pgfscope}%
\begin{pgfscope}%
\pgfpathrectangle{\pgfqpoint{1.250000in}{1.750000in}}{\pgfqpoint{2.279412in}{2.004545in}}%
\pgfusepath{clip}%
\pgfsetbuttcap%
\pgfsetroundjoin%
\pgfsetlinewidth{0.454441pt}%
\definecolor{currentstroke}{rgb}{0.283187,0.125848,0.444960}%
\pgfsetstrokecolor{currentstroke}%
\pgfsetdash{}{0pt}%
\pgfpathmoveto{\pgfqpoint{2.907214in}{2.751722in}}%
\pgfpathlineto{\pgfqpoint{2.857062in}{2.751583in}}%
\pgfusepath{stroke}%
\end{pgfscope}%
\begin{pgfscope}%
\pgfpathrectangle{\pgfqpoint{1.250000in}{1.750000in}}{\pgfqpoint{2.279412in}{2.004545in}}%
\pgfusepath{clip}%
\pgfsetbuttcap%
\pgfsetroundjoin%
\pgfsetlinewidth{0.506277pt}%
\definecolor{currentstroke}{rgb}{0.280868,0.160771,0.472899}%
\pgfsetstrokecolor{currentstroke}%
\pgfsetdash{}{0pt}%
\pgfpathmoveto{\pgfqpoint{2.857062in}{2.751583in}}%
\pgfpathlineto{\pgfqpoint{2.806910in}{2.751517in}}%
\pgfusepath{stroke}%
\end{pgfscope}%
\begin{pgfscope}%
\pgfpathrectangle{\pgfqpoint{1.250000in}{1.750000in}}{\pgfqpoint{2.279412in}{2.004545in}}%
\pgfusepath{clip}%
\pgfsetbuttcap%
\pgfsetroundjoin%
\pgfsetlinewidth{0.638080pt}%
\definecolor{currentstroke}{rgb}{0.257322,0.256130,0.526563}%
\pgfsetstrokecolor{currentstroke}%
\pgfsetdash{}{0pt}%
\pgfpathmoveto{\pgfqpoint{2.806910in}{2.751517in}}%
\pgfpathlineto{\pgfqpoint{2.756759in}{2.751528in}}%
\pgfusepath{stroke}%
\end{pgfscope}%
\begin{pgfscope}%
\pgfpathrectangle{\pgfqpoint{1.250000in}{1.750000in}}{\pgfqpoint{2.279412in}{2.004545in}}%
\pgfusepath{clip}%
\pgfsetbuttcap%
\pgfsetroundjoin%
\pgfsetlinewidth{0.847703pt}%
\definecolor{currentstroke}{rgb}{0.201239,0.383670,0.554294}%
\pgfsetstrokecolor{currentstroke}%
\pgfsetdash{}{0pt}%
\pgfpathmoveto{\pgfqpoint{2.756759in}{2.751528in}}%
\pgfpathlineto{\pgfqpoint{2.706607in}{2.751420in}}%
\pgfusepath{stroke}%
\end{pgfscope}%
\begin{pgfscope}%
\pgfpathrectangle{\pgfqpoint{1.250000in}{1.750000in}}{\pgfqpoint{2.279412in}{2.004545in}}%
\pgfusepath{clip}%
\pgfsetbuttcap%
\pgfsetroundjoin%
\pgfsetlinewidth{1.123203pt}%
\definecolor{currentstroke}{rgb}{0.139147,0.533812,0.555298}%
\pgfsetstrokecolor{currentstroke}%
\pgfsetdash{}{0pt}%
\pgfpathmoveto{\pgfqpoint{2.706607in}{2.751420in}}%
\pgfpathlineto{\pgfqpoint{2.656456in}{2.751222in}}%
\pgfusepath{stroke}%
\end{pgfscope}%
\begin{pgfscope}%
\pgfpathrectangle{\pgfqpoint{1.250000in}{1.750000in}}{\pgfqpoint{2.279412in}{2.004545in}}%
\pgfusepath{clip}%
\pgfsetbuttcap%
\pgfsetroundjoin%
\pgfsetlinewidth{1.421565pt}%
\definecolor{currentstroke}{rgb}{0.170948,0.694384,0.493803}%
\pgfsetstrokecolor{currentstroke}%
\pgfsetdash{}{0pt}%
\pgfpathmoveto{\pgfqpoint{2.656456in}{2.751222in}}%
\pgfpathlineto{\pgfqpoint{2.606304in}{2.751076in}}%
\pgfusepath{stroke}%
\end{pgfscope}%
\begin{pgfscope}%
\pgfpathrectangle{\pgfqpoint{1.250000in}{1.750000in}}{\pgfqpoint{2.279412in}{2.004545in}}%
\pgfusepath{clip}%
\pgfsetbuttcap%
\pgfsetroundjoin%
\pgfsetlinewidth{1.656000pt}%
\definecolor{currentstroke}{rgb}{0.412913,0.803041,0.357269}%
\pgfsetstrokecolor{currentstroke}%
\pgfsetdash{}{0pt}%
\pgfpathmoveto{\pgfqpoint{2.606304in}{2.751076in}}%
\pgfpathlineto{\pgfqpoint{2.556153in}{2.750867in}}%
\pgfusepath{stroke}%
\end{pgfscope}%
\begin{pgfscope}%
\pgfpathrectangle{\pgfqpoint{1.250000in}{1.750000in}}{\pgfqpoint{2.279412in}{2.004545in}}%
\pgfusepath{clip}%
\pgfsetbuttcap%
\pgfsetroundjoin%
\pgfsetlinewidth{1.934087pt}%
\definecolor{currentstroke}{rgb}{0.814576,0.883393,0.110347}%
\pgfsetstrokecolor{currentstroke}%
\pgfsetdash{}{0pt}%
\pgfpathmoveto{\pgfqpoint{2.556153in}{2.750867in}}%
\pgfpathlineto{\pgfqpoint{2.506003in}{2.750626in}}%
\pgfusepath{stroke}%
\end{pgfscope}%
\begin{pgfscope}%
\pgfpathrectangle{\pgfqpoint{1.250000in}{1.750000in}}{\pgfqpoint{2.279412in}{2.004545in}}%
\pgfusepath{clip}%
\pgfsetbuttcap%
\pgfsetroundjoin%
\pgfsetlinewidth{2.211943pt}%
\definecolor{currentstroke}{rgb}{0.993248,0.906157,0.143936}%
\pgfsetstrokecolor{currentstroke}%
\pgfsetdash{}{0pt}%
\pgfpathmoveto{\pgfqpoint{2.506003in}{2.750626in}}%
\pgfpathlineto{\pgfqpoint{2.455853in}{2.750321in}}%
\pgfusepath{stroke}%
\end{pgfscope}%
\begin{pgfscope}%
\pgfpathrectangle{\pgfqpoint{1.250000in}{1.750000in}}{\pgfqpoint{2.279412in}{2.004545in}}%
\pgfusepath{clip}%
\pgfsetbuttcap%
\pgfsetroundjoin%
\pgfsetlinewidth{2.260436pt}%
\definecolor{currentstroke}{rgb}{0.993248,0.906157,0.143936}%
\pgfsetstrokecolor{currentstroke}%
\pgfsetdash{}{0pt}%
\pgfpathmoveto{\pgfqpoint{2.455853in}{2.750321in}}%
\pgfpathlineto{\pgfqpoint{2.405705in}{2.749947in}}%
\pgfusepath{stroke}%
\end{pgfscope}%
\begin{pgfscope}%
\pgfpathrectangle{\pgfqpoint{1.250000in}{1.750000in}}{\pgfqpoint{2.279412in}{2.004545in}}%
\pgfusepath{clip}%
\pgfsetbuttcap%
\pgfsetroundjoin%
\pgfsetlinewidth{2.356766pt}%
\definecolor{currentstroke}{rgb}{0.993248,0.906157,0.143936}%
\pgfsetstrokecolor{currentstroke}%
\pgfsetdash{}{0pt}%
\pgfpathmoveto{\pgfqpoint{2.405705in}{2.749947in}}%
\pgfpathlineto{\pgfqpoint{2.355559in}{2.749532in}}%
\pgfusepath{stroke}%
\end{pgfscope}%
\begin{pgfscope}%
\pgfpathrectangle{\pgfqpoint{1.250000in}{1.750000in}}{\pgfqpoint{2.279412in}{2.004545in}}%
\pgfusepath{clip}%
\pgfsetbuttcap%
\pgfsetroundjoin%
\pgfsetlinewidth{2.338725pt}%
\definecolor{currentstroke}{rgb}{0.993248,0.906157,0.143936}%
\pgfsetstrokecolor{currentstroke}%
\pgfsetdash{}{0pt}%
\pgfpathmoveto{\pgfqpoint{2.355559in}{2.749532in}}%
\pgfpathlineto{\pgfqpoint{2.305415in}{2.749084in}}%
\pgfusepath{stroke}%
\end{pgfscope}%
\begin{pgfscope}%
\pgfpathrectangle{\pgfqpoint{1.250000in}{1.750000in}}{\pgfqpoint{2.279412in}{2.004545in}}%
\pgfusepath{clip}%
\pgfsetbuttcap%
\pgfsetroundjoin%
\pgfsetlinewidth{2.358918pt}%
\definecolor{currentstroke}{rgb}{0.993248,0.906157,0.143936}%
\pgfsetstrokecolor{currentstroke}%
\pgfsetdash{}{0pt}%
\pgfpathmoveto{\pgfqpoint{2.305415in}{2.749084in}}%
\pgfpathlineto{\pgfqpoint{2.255274in}{2.748631in}}%
\pgfusepath{stroke}%
\end{pgfscope}%
\begin{pgfscope}%
\pgfpathrectangle{\pgfqpoint{1.250000in}{1.750000in}}{\pgfqpoint{2.279412in}{2.004545in}}%
\pgfusepath{clip}%
\pgfsetbuttcap%
\pgfsetroundjoin%
\pgfsetlinewidth{2.199408pt}%
\definecolor{currentstroke}{rgb}{0.993248,0.906157,0.143936}%
\pgfsetstrokecolor{currentstroke}%
\pgfsetdash{}{0pt}%
\pgfpathmoveto{\pgfqpoint{2.255274in}{2.748631in}}%
\pgfpathlineto{\pgfqpoint{2.205140in}{2.748070in}}%
\pgfusepath{stroke}%
\end{pgfscope}%
\begin{pgfscope}%
\pgfpathrectangle{\pgfqpoint{1.250000in}{1.750000in}}{\pgfqpoint{2.279412in}{2.004545in}}%
\pgfusepath{clip}%
\pgfsetbuttcap%
\pgfsetroundjoin%
\pgfsetlinewidth{2.036367pt}%
\definecolor{currentstroke}{rgb}{0.964894,0.902323,0.123941}%
\pgfsetstrokecolor{currentstroke}%
\pgfsetdash{}{0pt}%
\pgfpathmoveto{\pgfqpoint{2.205140in}{2.748070in}}%
\pgfpathlineto{\pgfqpoint{2.155015in}{2.747475in}}%
\pgfusepath{stroke}%
\end{pgfscope}%
\begin{pgfscope}%
\pgfpathrectangle{\pgfqpoint{1.250000in}{1.750000in}}{\pgfqpoint{2.279412in}{2.004545in}}%
\pgfusepath{clip}%
\pgfsetbuttcap%
\pgfsetroundjoin%
\pgfsetlinewidth{0.326157pt}%
\definecolor{currentstroke}{rgb}{0.271305,0.019942,0.347269}%
\pgfsetstrokecolor{currentstroke}%
\pgfsetdash{}{0pt}%
\pgfpathmoveto{\pgfqpoint{3.107792in}{3.022913in}}%
\pgfpathlineto{\pgfqpoint{3.057656in}{3.022321in}}%
\pgfusepath{stroke}%
\end{pgfscope}%
\begin{pgfscope}%
\pgfpathrectangle{\pgfqpoint{1.250000in}{1.750000in}}{\pgfqpoint{2.279412in}{2.004545in}}%
\pgfusepath{clip}%
\pgfsetbuttcap%
\pgfsetroundjoin%
\pgfsetlinewidth{0.329213pt}%
\definecolor{currentstroke}{rgb}{0.272594,0.025563,0.353093}%
\pgfsetstrokecolor{currentstroke}%
\pgfsetdash{}{0pt}%
\pgfpathmoveto{\pgfqpoint{3.057656in}{3.022321in}}%
\pgfpathlineto{\pgfqpoint{3.007539in}{3.020927in}}%
\pgfusepath{stroke}%
\end{pgfscope}%
\begin{pgfscope}%
\pgfpathrectangle{\pgfqpoint{1.250000in}{1.750000in}}{\pgfqpoint{2.279412in}{2.004545in}}%
\pgfusepath{clip}%
\pgfsetbuttcap%
\pgfsetroundjoin%
\pgfsetlinewidth{0.345420pt}%
\definecolor{currentstroke}{rgb}{0.274952,0.037752,0.364543}%
\pgfsetstrokecolor{currentstroke}%
\pgfsetdash{}{0pt}%
\pgfpathmoveto{\pgfqpoint{3.007539in}{3.020927in}}%
\pgfpathlineto{\pgfqpoint{2.957419in}{3.019532in}}%
\pgfusepath{stroke}%
\end{pgfscope}%
\begin{pgfscope}%
\pgfpathrectangle{\pgfqpoint{1.250000in}{1.750000in}}{\pgfqpoint{2.279412in}{2.004545in}}%
\pgfusepath{clip}%
\pgfsetbuttcap%
\pgfsetroundjoin%
\pgfsetlinewidth{0.373715pt}%
\definecolor{currentstroke}{rgb}{0.278791,0.062145,0.386592}%
\pgfsetstrokecolor{currentstroke}%
\pgfsetdash{}{0pt}%
\pgfpathmoveto{\pgfqpoint{2.957419in}{3.019532in}}%
\pgfpathlineto{\pgfqpoint{2.907317in}{3.017698in}}%
\pgfusepath{stroke}%
\end{pgfscope}%
\begin{pgfscope}%
\pgfpathrectangle{\pgfqpoint{1.250000in}{1.750000in}}{\pgfqpoint{2.279412in}{2.004545in}}%
\pgfusepath{clip}%
\pgfsetbuttcap%
\pgfsetroundjoin%
\pgfsetlinewidth{0.403580pt}%
\definecolor{currentstroke}{rgb}{0.281446,0.084320,0.407414}%
\pgfsetstrokecolor{currentstroke}%
\pgfsetdash{}{0pt}%
\pgfpathmoveto{\pgfqpoint{2.907317in}{3.017698in}}%
\pgfpathlineto{\pgfqpoint{2.857219in}{3.015864in}}%
\pgfusepath{stroke}%
\end{pgfscope}%
\begin{pgfscope}%
\pgfpathrectangle{\pgfqpoint{1.250000in}{1.750000in}}{\pgfqpoint{2.279412in}{2.004545in}}%
\pgfusepath{clip}%
\pgfsetbuttcap%
\pgfsetroundjoin%
\pgfsetlinewidth{0.419582pt}%
\definecolor{currentstroke}{rgb}{0.282656,0.100196,0.422160}%
\pgfsetstrokecolor{currentstroke}%
\pgfsetdash{}{0pt}%
\pgfpathmoveto{\pgfqpoint{2.857219in}{3.015864in}}%
\pgfpathlineto{\pgfqpoint{2.807128in}{3.013911in}}%
\pgfusepath{stroke}%
\end{pgfscope}%
\begin{pgfscope}%
\pgfpathrectangle{\pgfqpoint{1.250000in}{1.750000in}}{\pgfqpoint{2.279412in}{2.004545in}}%
\pgfusepath{clip}%
\pgfsetbuttcap%
\pgfsetroundjoin%
\pgfsetlinewidth{0.458439pt}%
\definecolor{currentstroke}{rgb}{0.283187,0.125848,0.444960}%
\pgfsetstrokecolor{currentstroke}%
\pgfsetdash{}{0pt}%
\pgfpathmoveto{\pgfqpoint{2.807128in}{3.013911in}}%
\pgfpathlineto{\pgfqpoint{2.757083in}{3.011046in}}%
\pgfusepath{stroke}%
\end{pgfscope}%
\begin{pgfscope}%
\pgfpathrectangle{\pgfqpoint{1.250000in}{1.750000in}}{\pgfqpoint{2.279412in}{2.004545in}}%
\pgfusepath{clip}%
\pgfsetbuttcap%
\pgfsetroundjoin%
\pgfsetlinewidth{0.483881pt}%
\definecolor{currentstroke}{rgb}{0.282290,0.145912,0.461510}%
\pgfsetstrokecolor{currentstroke}%
\pgfsetdash{}{0pt}%
\pgfpathmoveto{\pgfqpoint{2.757083in}{3.011046in}}%
\pgfpathlineto{\pgfqpoint{2.707057in}{3.007929in}}%
\pgfusepath{stroke}%
\end{pgfscope}%
\begin{pgfscope}%
\pgfpathrectangle{\pgfqpoint{1.250000in}{1.750000in}}{\pgfqpoint{2.279412in}{2.004545in}}%
\pgfusepath{clip}%
\pgfsetbuttcap%
\pgfsetroundjoin%
\pgfsetlinewidth{0.448980pt}%
\definecolor{currentstroke}{rgb}{0.283229,0.120777,0.440584}%
\pgfsetstrokecolor{currentstroke}%
\pgfsetdash{}{0pt}%
\pgfpathmoveto{\pgfqpoint{2.707057in}{3.007929in}}%
\pgfpathlineto{\pgfqpoint{2.657175in}{3.003546in}}%
\pgfusepath{stroke}%
\end{pgfscope}%
\begin{pgfscope}%
\pgfpathrectangle{\pgfqpoint{1.250000in}{1.750000in}}{\pgfqpoint{2.279412in}{2.004545in}}%
\pgfusepath{clip}%
\pgfsetbuttcap%
\pgfsetroundjoin%
\pgfsetlinewidth{0.611878pt}%
\definecolor{currentstroke}{rgb}{0.263663,0.237631,0.518762}%
\pgfsetstrokecolor{currentstroke}%
\pgfsetdash{}{0pt}%
\pgfpathmoveto{\pgfqpoint{2.657175in}{3.003546in}}%
\pgfpathlineto{\pgfqpoint{2.607516in}{2.997433in}}%
\pgfusepath{stroke}%
\end{pgfscope}%
\begin{pgfscope}%
\pgfpathrectangle{\pgfqpoint{1.250000in}{1.750000in}}{\pgfqpoint{2.279412in}{2.004545in}}%
\pgfusepath{clip}%
\pgfsetbuttcap%
\pgfsetroundjoin%
\pgfsetlinewidth{0.603921pt}%
\definecolor{currentstroke}{rgb}{0.265145,0.232956,0.516599}%
\pgfsetstrokecolor{currentstroke}%
\pgfsetdash{}{0pt}%
\pgfpathmoveto{\pgfqpoint{2.607516in}{2.997433in}}%
\pgfpathlineto{\pgfqpoint{2.558107in}{2.989957in}}%
\pgfusepath{stroke}%
\end{pgfscope}%
\begin{pgfscope}%
\pgfpathrectangle{\pgfqpoint{1.250000in}{1.750000in}}{\pgfqpoint{2.279412in}{2.004545in}}%
\pgfusepath{clip}%
\pgfsetbuttcap%
\pgfsetroundjoin%
\pgfsetlinewidth{0.732528pt}%
\definecolor{currentstroke}{rgb}{0.233603,0.313828,0.543914}%
\pgfsetstrokecolor{currentstroke}%
\pgfsetdash{}{0pt}%
\pgfpathmoveto{\pgfqpoint{2.558107in}{2.989957in}}%
\pgfpathlineto{\pgfqpoint{2.509052in}{2.980836in}}%
\pgfusepath{stroke}%
\end{pgfscope}%
\begin{pgfscope}%
\pgfpathrectangle{\pgfqpoint{1.250000in}{1.750000in}}{\pgfqpoint{2.279412in}{2.004545in}}%
\pgfusepath{clip}%
\pgfsetbuttcap%
\pgfsetroundjoin%
\pgfsetlinewidth{0.863142pt}%
\definecolor{currentstroke}{rgb}{0.197636,0.391528,0.554969}%
\pgfsetstrokecolor{currentstroke}%
\pgfsetdash{}{0pt}%
\pgfpathmoveto{\pgfqpoint{2.509052in}{2.980836in}}%
\pgfpathlineto{\pgfqpoint{2.460439in}{2.970049in}}%
\pgfusepath{stroke}%
\end{pgfscope}%
\begin{pgfscope}%
\pgfpathrectangle{\pgfqpoint{1.250000in}{1.750000in}}{\pgfqpoint{2.279412in}{2.004545in}}%
\pgfusepath{clip}%
\pgfsetbuttcap%
\pgfsetroundjoin%
\pgfsetlinewidth{1.065430pt}%
\definecolor{currentstroke}{rgb}{0.150476,0.504369,0.557430}%
\pgfsetstrokecolor{currentstroke}%
\pgfsetdash{}{0pt}%
\pgfpathmoveto{\pgfqpoint{2.460439in}{2.970049in}}%
\pgfpathlineto{\pgfqpoint{2.412295in}{2.957723in}}%
\pgfusepath{stroke}%
\end{pgfscope}%
\begin{pgfscope}%
\pgfpathrectangle{\pgfqpoint{1.250000in}{1.750000in}}{\pgfqpoint{2.279412in}{2.004545in}}%
\pgfusepath{clip}%
\pgfsetbuttcap%
\pgfsetroundjoin%
\pgfsetlinewidth{1.034784pt}%
\definecolor{currentstroke}{rgb}{0.157729,0.485932,0.558013}%
\pgfsetstrokecolor{currentstroke}%
\pgfsetdash{}{0pt}%
\pgfpathmoveto{\pgfqpoint{2.412295in}{2.957723in}}%
\pgfpathlineto{\pgfqpoint{2.364840in}{2.943531in}}%
\pgfusepath{stroke}%
\end{pgfscope}%
\begin{pgfscope}%
\pgfpathrectangle{\pgfqpoint{1.250000in}{1.750000in}}{\pgfqpoint{2.279412in}{2.004545in}}%
\pgfusepath{clip}%
\pgfsetbuttcap%
\pgfsetroundjoin%
\pgfsetlinewidth{1.005665pt}%
\definecolor{currentstroke}{rgb}{0.163625,0.471133,0.558148}%
\pgfsetstrokecolor{currentstroke}%
\pgfsetdash{}{0pt}%
\pgfpathmoveto{\pgfqpoint{2.364840in}{2.943531in}}%
\pgfpathlineto{\pgfqpoint{2.318085in}{2.927615in}}%
\pgfusepath{stroke}%
\end{pgfscope}%
\begin{pgfscope}%
\pgfpathrectangle{\pgfqpoint{1.250000in}{1.750000in}}{\pgfqpoint{2.279412in}{2.004545in}}%
\pgfusepath{clip}%
\pgfsetbuttcap%
\pgfsetroundjoin%
\pgfsetlinewidth{1.259150pt}%
\definecolor{currentstroke}{rgb}{0.119512,0.607464,0.540218}%
\pgfsetstrokecolor{currentstroke}%
\pgfsetdash{}{0pt}%
\pgfpathmoveto{\pgfqpoint{2.318085in}{2.927615in}}%
\pgfpathlineto{\pgfqpoint{2.271853in}{2.910551in}}%
\pgfusepath{stroke}%
\end{pgfscope}%
\begin{pgfscope}%
\pgfpathrectangle{\pgfqpoint{1.250000in}{1.750000in}}{\pgfqpoint{2.279412in}{2.004545in}}%
\pgfusepath{clip}%
\pgfsetbuttcap%
\pgfsetroundjoin%
\pgfsetlinewidth{1.728984pt}%
\definecolor{currentstroke}{rgb}{0.515992,0.831158,0.294279}%
\pgfsetstrokecolor{currentstroke}%
\pgfsetdash{}{0pt}%
\pgfpathmoveto{\pgfqpoint{2.271853in}{2.910551in}}%
\pgfpathlineto{\pgfqpoint{2.226497in}{2.891818in}}%
\pgfusepath{stroke}%
\end{pgfscope}%
\begin{pgfscope}%
\pgfpathrectangle{\pgfqpoint{1.250000in}{1.750000in}}{\pgfqpoint{2.279412in}{2.004545in}}%
\pgfusepath{clip}%
\pgfsetbuttcap%
\pgfsetroundjoin%
\pgfsetlinewidth{1.394549pt}%
\definecolor{currentstroke}{rgb}{0.153894,0.680203,0.504172}%
\pgfsetstrokecolor{currentstroke}%
\pgfsetdash{}{0pt}%
\pgfpathmoveto{\pgfqpoint{2.226497in}{2.891818in}}%
\pgfpathlineto{\pgfqpoint{2.182002in}{2.871519in}}%
\pgfusepath{stroke}%
\end{pgfscope}%
\begin{pgfscope}%
\pgfpathrectangle{\pgfqpoint{1.250000in}{1.750000in}}{\pgfqpoint{2.279412in}{2.004545in}}%
\pgfusepath{clip}%
\pgfsetbuttcap%
\pgfsetroundjoin%
\pgfsetlinewidth{1.578883pt}%
\definecolor{currentstroke}{rgb}{0.319809,0.770914,0.411152}%
\pgfsetstrokecolor{currentstroke}%
\pgfsetdash{}{0pt}%
\pgfpathmoveto{\pgfqpoint{2.182002in}{2.871519in}}%
\pgfpathlineto{\pgfqpoint{2.137210in}{2.851728in}}%
\pgfusepath{stroke}%
\end{pgfscope}%
\begin{pgfscope}%
\pgfpathrectangle{\pgfqpoint{1.250000in}{1.750000in}}{\pgfqpoint{2.279412in}{2.004545in}}%
\pgfusepath{clip}%
\pgfsetbuttcap%
\pgfsetroundjoin%
\pgfsetlinewidth{2.161473pt}%
\definecolor{currentstroke}{rgb}{0.993248,0.906157,0.143936}%
\pgfsetstrokecolor{currentstroke}%
\pgfsetdash{}{0pt}%
\pgfpathmoveto{\pgfqpoint{2.137210in}{2.851728in}}%
\pgfpathlineto{\pgfqpoint{2.092129in}{2.832442in}}%
\pgfusepath{stroke}%
\end{pgfscope}%
\begin{pgfscope}%
\pgfpathrectangle{\pgfqpoint{1.250000in}{1.750000in}}{\pgfqpoint{2.279412in}{2.004545in}}%
\pgfusepath{clip}%
\pgfsetbuttcap%
\pgfsetroundjoin%
\pgfsetlinewidth{1.767649pt}%
\definecolor{currentstroke}{rgb}{0.565498,0.842430,0.262877}%
\pgfsetstrokecolor{currentstroke}%
\pgfsetdash{}{0pt}%
\pgfpathmoveto{\pgfqpoint{2.092129in}{2.832442in}}%
\pgfpathlineto{\pgfqpoint{2.046758in}{2.813728in}}%
\pgfusepath{stroke}%
\end{pgfscope}%
\begin{pgfscope}%
\pgfpathrectangle{\pgfqpoint{1.250000in}{1.750000in}}{\pgfqpoint{2.279412in}{2.004545in}}%
\pgfusepath{clip}%
\pgfsetbuttcap%
\pgfsetroundjoin%
\pgfsetlinewidth{1.792309pt}%
\definecolor{currentstroke}{rgb}{0.606045,0.850733,0.236712}%
\pgfsetstrokecolor{currentstroke}%
\pgfsetdash{}{0pt}%
\pgfpathmoveto{\pgfqpoint{2.046758in}{2.813728in}}%
\pgfpathlineto{\pgfqpoint{2.001348in}{2.795165in}}%
\pgfusepath{stroke}%
\end{pgfscope}%
\begin{pgfscope}%
\pgfpathrectangle{\pgfqpoint{1.250000in}{1.750000in}}{\pgfqpoint{2.279412in}{2.004545in}}%
\pgfusepath{clip}%
\pgfsetbuttcap%
\pgfsetroundjoin%
\pgfsetlinewidth{1.554934pt}%
\definecolor{currentstroke}{rgb}{0.288921,0.758394,0.428426}%
\pgfsetstrokecolor{currentstroke}%
\pgfsetdash{}{0pt}%
\pgfpathmoveto{\pgfqpoint{2.001348in}{2.795165in}}%
\pgfpathlineto{\pgfqpoint{1.955622in}{2.777631in}}%
\pgfusepath{stroke}%
\end{pgfscope}%
\begin{pgfscope}%
\pgfpathrectangle{\pgfqpoint{1.250000in}{1.750000in}}{\pgfqpoint{2.279412in}{2.004545in}}%
\pgfusepath{clip}%
\pgfsetbuttcap%
\pgfsetroundjoin%
\pgfsetlinewidth{0.315374pt}%
\definecolor{currentstroke}{rgb}{0.269944,0.014625,0.341379}%
\pgfsetstrokecolor{currentstroke}%
\pgfsetdash{}{0pt}%
\pgfpathmoveto{\pgfqpoint{3.107792in}{3.068020in}}%
\pgfpathlineto{\pgfqpoint{3.057947in}{3.065440in}}%
\pgfusepath{stroke}%
\end{pgfscope}%
\begin{pgfscope}%
\pgfpathrectangle{\pgfqpoint{1.250000in}{1.750000in}}{\pgfqpoint{2.279412in}{2.004545in}}%
\pgfusepath{clip}%
\pgfsetbuttcap%
\pgfsetroundjoin%
\pgfsetlinewidth{0.321889pt}%
\definecolor{currentstroke}{rgb}{0.271305,0.019942,0.347269}%
\pgfsetstrokecolor{currentstroke}%
\pgfsetdash{}{0pt}%
\pgfpathmoveto{\pgfqpoint{3.057947in}{3.065440in}}%
\pgfpathlineto{\pgfqpoint{3.007826in}{3.064308in}}%
\pgfusepath{stroke}%
\end{pgfscope}%
\begin{pgfscope}%
\pgfpathrectangle{\pgfqpoint{1.250000in}{1.750000in}}{\pgfqpoint{2.279412in}{2.004545in}}%
\pgfusepath{clip}%
\pgfsetbuttcap%
\pgfsetroundjoin%
\pgfsetlinewidth{0.338438pt}%
\definecolor{currentstroke}{rgb}{0.273809,0.031497,0.358853}%
\pgfsetstrokecolor{currentstroke}%
\pgfsetdash{}{0pt}%
\pgfpathmoveto{\pgfqpoint{3.007826in}{3.064308in}}%
\pgfpathlineto{\pgfqpoint{2.957748in}{3.062384in}}%
\pgfusepath{stroke}%
\end{pgfscope}%
\begin{pgfscope}%
\pgfpathrectangle{\pgfqpoint{1.250000in}{1.750000in}}{\pgfqpoint{2.279412in}{2.004545in}}%
\pgfusepath{clip}%
\pgfsetbuttcap%
\pgfsetroundjoin%
\pgfsetlinewidth{0.352825pt}%
\definecolor{currentstroke}{rgb}{0.276022,0.044167,0.370164}%
\pgfsetstrokecolor{currentstroke}%
\pgfsetdash{}{0pt}%
\pgfpathmoveto{\pgfqpoint{2.957748in}{3.062384in}}%
\pgfpathlineto{\pgfqpoint{2.907706in}{3.059675in}}%
\pgfusepath{stroke}%
\end{pgfscope}%
\begin{pgfscope}%
\pgfpathrectangle{\pgfqpoint{1.250000in}{1.750000in}}{\pgfqpoint{2.279412in}{2.004545in}}%
\pgfusepath{clip}%
\pgfsetbuttcap%
\pgfsetroundjoin%
\pgfsetlinewidth{0.385099pt}%
\definecolor{currentstroke}{rgb}{0.280267,0.073417,0.397163}%
\pgfsetstrokecolor{currentstroke}%
\pgfsetdash{}{0pt}%
\pgfpathmoveto{\pgfqpoint{2.907706in}{3.059675in}}%
\pgfpathlineto{\pgfqpoint{2.857656in}{3.056910in}}%
\pgfusepath{stroke}%
\end{pgfscope}%
\begin{pgfscope}%
\pgfpathrectangle{\pgfqpoint{1.250000in}{1.750000in}}{\pgfqpoint{2.279412in}{2.004545in}}%
\pgfusepath{clip}%
\pgfsetbuttcap%
\pgfsetroundjoin%
\pgfsetlinewidth{0.398495pt}%
\definecolor{currentstroke}{rgb}{0.281446,0.084320,0.407414}%
\pgfsetstrokecolor{currentstroke}%
\pgfsetdash{}{0pt}%
\pgfpathmoveto{\pgfqpoint{2.857656in}{3.056910in}}%
\pgfpathlineto{\pgfqpoint{2.807624in}{3.053936in}}%
\pgfusepath{stroke}%
\end{pgfscope}%
\begin{pgfscope}%
\pgfpathrectangle{\pgfqpoint{1.250000in}{1.750000in}}{\pgfqpoint{2.279412in}{2.004545in}}%
\pgfusepath{clip}%
\pgfsetbuttcap%
\pgfsetroundjoin%
\pgfsetlinewidth{0.436622pt}%
\definecolor{currentstroke}{rgb}{0.283091,0.110553,0.431554}%
\pgfsetstrokecolor{currentstroke}%
\pgfsetdash{}{0pt}%
\pgfpathmoveto{\pgfqpoint{2.807624in}{3.053936in}}%
\pgfpathlineto{\pgfqpoint{2.757663in}{3.050122in}}%
\pgfusepath{stroke}%
\end{pgfscope}%
\begin{pgfscope}%
\pgfpathrectangle{\pgfqpoint{1.250000in}{1.750000in}}{\pgfqpoint{2.279412in}{2.004545in}}%
\pgfusepath{clip}%
\pgfsetbuttcap%
\pgfsetroundjoin%
\pgfsetlinewidth{0.461249pt}%
\definecolor{currentstroke}{rgb}{0.283072,0.130895,0.449241}%
\pgfsetstrokecolor{currentstroke}%
\pgfsetdash{}{0pt}%
\pgfpathmoveto{\pgfqpoint{2.757663in}{3.050122in}}%
\pgfpathlineto{\pgfqpoint{2.707786in}{3.045536in}}%
\pgfusepath{stroke}%
\end{pgfscope}%
\begin{pgfscope}%
\pgfpathrectangle{\pgfqpoint{1.250000in}{1.750000in}}{\pgfqpoint{2.279412in}{2.004545in}}%
\pgfusepath{clip}%
\pgfsetbuttcap%
\pgfsetroundjoin%
\pgfsetlinewidth{0.318419pt}%
\definecolor{currentstroke}{rgb}{0.269944,0.014625,0.341379}%
\pgfsetstrokecolor{currentstroke}%
\pgfsetdash{}{0pt}%
\pgfpathmoveto{\pgfqpoint{3.107792in}{3.113127in}}%
\pgfpathlineto{\pgfqpoint{3.057674in}{3.111987in}}%
\pgfusepath{stroke}%
\end{pgfscope}%
\begin{pgfscope}%
\pgfpathrectangle{\pgfqpoint{1.250000in}{1.750000in}}{\pgfqpoint{2.279412in}{2.004545in}}%
\pgfusepath{clip}%
\pgfsetbuttcap%
\pgfsetroundjoin%
\pgfsetlinewidth{0.332076pt}%
\definecolor{currentstroke}{rgb}{0.272594,0.025563,0.353093}%
\pgfsetstrokecolor{currentstroke}%
\pgfsetdash{}{0pt}%
\pgfpathmoveto{\pgfqpoint{3.057674in}{3.111987in}}%
\pgfpathlineto{\pgfqpoint{3.007587in}{3.110278in}}%
\pgfusepath{stroke}%
\end{pgfscope}%
\begin{pgfscope}%
\pgfpathrectangle{\pgfqpoint{1.250000in}{1.750000in}}{\pgfqpoint{2.279412in}{2.004545in}}%
\pgfusepath{clip}%
\pgfsetbuttcap%
\pgfsetroundjoin%
\pgfsetlinewidth{0.342549pt}%
\definecolor{currentstroke}{rgb}{0.274952,0.037752,0.364543}%
\pgfsetstrokecolor{currentstroke}%
\pgfsetdash{}{0pt}%
\pgfpathmoveto{\pgfqpoint{3.007587in}{3.110278in}}%
\pgfpathlineto{\pgfqpoint{2.957493in}{3.108668in}}%
\pgfusepath{stroke}%
\end{pgfscope}%
\begin{pgfscope}%
\pgfpathrectangle{\pgfqpoint{1.250000in}{1.750000in}}{\pgfqpoint{2.279412in}{2.004545in}}%
\pgfusepath{clip}%
\pgfsetbuttcap%
\pgfsetroundjoin%
\pgfsetlinewidth{0.352555pt}%
\definecolor{currentstroke}{rgb}{0.276022,0.044167,0.370164}%
\pgfsetstrokecolor{currentstroke}%
\pgfsetdash{}{0pt}%
\pgfpathmoveto{\pgfqpoint{2.957493in}{3.108668in}}%
\pgfpathlineto{\pgfqpoint{2.907452in}{3.106495in}}%
\pgfusepath{stroke}%
\end{pgfscope}%
\begin{pgfscope}%
\pgfpathrectangle{\pgfqpoint{1.250000in}{1.750000in}}{\pgfqpoint{2.279412in}{2.004545in}}%
\pgfusepath{clip}%
\pgfsetbuttcap%
\pgfsetroundjoin%
\pgfsetlinewidth{0.372187pt}%
\definecolor{currentstroke}{rgb}{0.278791,0.062145,0.386592}%
\pgfsetstrokecolor{currentstroke}%
\pgfsetdash{}{0pt}%
\pgfpathmoveto{\pgfqpoint{2.907452in}{3.106495in}}%
\pgfpathlineto{\pgfqpoint{2.857463in}{3.103023in}}%
\pgfusepath{stroke}%
\end{pgfscope}%
\begin{pgfscope}%
\pgfpathrectangle{\pgfqpoint{1.250000in}{1.750000in}}{\pgfqpoint{2.279412in}{2.004545in}}%
\pgfusepath{clip}%
\pgfsetbuttcap%
\pgfsetroundjoin%
\pgfsetlinewidth{0.375675pt}%
\definecolor{currentstroke}{rgb}{0.278791,0.062145,0.386592}%
\pgfsetstrokecolor{currentstroke}%
\pgfsetdash{}{0pt}%
\pgfpathmoveto{\pgfqpoint{2.857463in}{3.103023in}}%
\pgfpathlineto{\pgfqpoint{2.807473in}{3.099514in}}%
\pgfusepath{stroke}%
\end{pgfscope}%
\begin{pgfscope}%
\pgfpathrectangle{\pgfqpoint{1.250000in}{1.750000in}}{\pgfqpoint{2.279412in}{2.004545in}}%
\pgfusepath{clip}%
\pgfsetbuttcap%
\pgfsetroundjoin%
\pgfsetlinewidth{0.402319pt}%
\definecolor{currentstroke}{rgb}{0.281446,0.084320,0.407414}%
\pgfsetstrokecolor{currentstroke}%
\pgfsetdash{}{0pt}%
\pgfpathmoveto{\pgfqpoint{2.807473in}{3.099514in}}%
\pgfpathlineto{\pgfqpoint{2.757605in}{3.094907in}}%
\pgfusepath{stroke}%
\end{pgfscope}%
\begin{pgfscope}%
\pgfpathrectangle{\pgfqpoint{1.250000in}{1.750000in}}{\pgfqpoint{2.279412in}{2.004545in}}%
\pgfusepath{clip}%
\pgfsetbuttcap%
\pgfsetroundjoin%
\pgfsetlinewidth{0.325537pt}%
\definecolor{currentstroke}{rgb}{0.271305,0.019942,0.347269}%
\pgfsetstrokecolor{currentstroke}%
\pgfsetdash{}{0pt}%
\pgfpathmoveto{\pgfqpoint{3.107792in}{3.158234in}}%
\pgfpathlineto{\pgfqpoint{3.057664in}{3.158475in}}%
\pgfusepath{stroke}%
\end{pgfscope}%
\begin{pgfscope}%
\pgfpathrectangle{\pgfqpoint{1.250000in}{1.750000in}}{\pgfqpoint{2.279412in}{2.004545in}}%
\pgfusepath{clip}%
\pgfsetbuttcap%
\pgfsetroundjoin%
\pgfsetlinewidth{0.328219pt}%
\definecolor{currentstroke}{rgb}{0.271305,0.019942,0.347269}%
\pgfsetstrokecolor{currentstroke}%
\pgfsetdash{}{0pt}%
\pgfpathmoveto{\pgfqpoint{3.057664in}{3.158475in}}%
\pgfpathlineto{\pgfqpoint{3.007597in}{3.157482in}}%
\pgfusepath{stroke}%
\end{pgfscope}%
\begin{pgfscope}%
\pgfpathrectangle{\pgfqpoint{1.250000in}{1.750000in}}{\pgfqpoint{2.279412in}{2.004545in}}%
\pgfusepath{clip}%
\pgfsetbuttcap%
\pgfsetroundjoin%
\pgfsetlinewidth{0.331031pt}%
\definecolor{currentstroke}{rgb}{0.272594,0.025563,0.353093}%
\pgfsetstrokecolor{currentstroke}%
\pgfsetdash{}{0pt}%
\pgfpathmoveto{\pgfqpoint{3.007597in}{3.157482in}}%
\pgfpathlineto{\pgfqpoint{2.957517in}{3.155783in}}%
\pgfusepath{stroke}%
\end{pgfscope}%
\begin{pgfscope}%
\pgfpathrectangle{\pgfqpoint{1.250000in}{1.750000in}}{\pgfqpoint{2.279412in}{2.004545in}}%
\pgfusepath{clip}%
\pgfsetbuttcap%
\pgfsetroundjoin%
\pgfsetlinewidth{0.354432pt}%
\definecolor{currentstroke}{rgb}{0.276022,0.044167,0.370164}%
\pgfsetstrokecolor{currentstroke}%
\pgfsetdash{}{0pt}%
\pgfpathmoveto{\pgfqpoint{2.957517in}{3.155783in}}%
\pgfpathlineto{\pgfqpoint{2.907412in}{3.154198in}}%
\pgfusepath{stroke}%
\end{pgfscope}%
\begin{pgfscope}%
\pgfpathrectangle{\pgfqpoint{1.250000in}{1.750000in}}{\pgfqpoint{2.279412in}{2.004545in}}%
\pgfusepath{clip}%
\pgfsetbuttcap%
\pgfsetroundjoin%
\pgfsetlinewidth{0.354569pt}%
\definecolor{currentstroke}{rgb}{0.276022,0.044167,0.370164}%
\pgfsetstrokecolor{currentstroke}%
\pgfsetdash{}{0pt}%
\pgfpathmoveto{\pgfqpoint{2.907412in}{3.154198in}}%
\pgfpathlineto{\pgfqpoint{2.857355in}{3.151555in}}%
\pgfusepath{stroke}%
\end{pgfscope}%
\begin{pgfscope}%
\pgfpathrectangle{\pgfqpoint{1.250000in}{1.750000in}}{\pgfqpoint{2.279412in}{2.004545in}}%
\pgfusepath{clip}%
\pgfsetbuttcap%
\pgfsetroundjoin%
\pgfsetlinewidth{0.368546pt}%
\definecolor{currentstroke}{rgb}{0.277941,0.056324,0.381191}%
\pgfsetstrokecolor{currentstroke}%
\pgfsetdash{}{0pt}%
\pgfpathmoveto{\pgfqpoint{2.857355in}{3.151555in}}%
\pgfpathlineto{\pgfqpoint{2.807382in}{3.147956in}}%
\pgfusepath{stroke}%
\end{pgfscope}%
\begin{pgfscope}%
\pgfpathrectangle{\pgfqpoint{1.250000in}{1.750000in}}{\pgfqpoint{2.279412in}{2.004545in}}%
\pgfusepath{clip}%
\pgfsetbuttcap%
\pgfsetroundjoin%
\pgfsetlinewidth{0.384009pt}%
\definecolor{currentstroke}{rgb}{0.280267,0.073417,0.397163}%
\pgfsetstrokecolor{currentstroke}%
\pgfsetdash{}{0pt}%
\pgfpathmoveto{\pgfqpoint{2.807382in}{3.147956in}}%
\pgfpathlineto{\pgfqpoint{2.757543in}{3.143099in}}%
\pgfusepath{stroke}%
\end{pgfscope}%
\begin{pgfscope}%
\pgfpathrectangle{\pgfqpoint{1.250000in}{1.750000in}}{\pgfqpoint{2.279412in}{2.004545in}}%
\pgfusepath{clip}%
\pgfsetbuttcap%
\pgfsetroundjoin%
\pgfsetlinewidth{0.392659pt}%
\definecolor{currentstroke}{rgb}{0.280894,0.078907,0.402329}%
\pgfsetstrokecolor{currentstroke}%
\pgfsetdash{}{0pt}%
\pgfpathmoveto{\pgfqpoint{2.757543in}{3.143099in}}%
\pgfpathlineto{\pgfqpoint{2.707829in}{3.137341in}}%
\pgfusepath{stroke}%
\end{pgfscope}%
\begin{pgfscope}%
\pgfpathrectangle{\pgfqpoint{1.250000in}{1.750000in}}{\pgfqpoint{2.279412in}{2.004545in}}%
\pgfusepath{clip}%
\pgfsetbuttcap%
\pgfsetroundjoin%
\pgfsetlinewidth{0.391705pt}%
\definecolor{currentstroke}{rgb}{0.280894,0.078907,0.402329}%
\pgfsetstrokecolor{currentstroke}%
\pgfsetdash{}{0pt}%
\pgfpathmoveto{\pgfqpoint{2.707829in}{3.137341in}}%
\pgfpathlineto{\pgfqpoint{2.658359in}{3.130227in}}%
\pgfusepath{stroke}%
\end{pgfscope}%
\begin{pgfscope}%
\pgfpathrectangle{\pgfqpoint{1.250000in}{1.750000in}}{\pgfqpoint{2.279412in}{2.004545in}}%
\pgfusepath{clip}%
\pgfsetbuttcap%
\pgfsetroundjoin%
\pgfsetlinewidth{0.407281pt}%
\definecolor{currentstroke}{rgb}{0.281924,0.089666,0.412415}%
\pgfsetstrokecolor{currentstroke}%
\pgfsetdash{}{0pt}%
\pgfpathmoveto{\pgfqpoint{2.658359in}{3.130227in}}%
\pgfpathlineto{\pgfqpoint{2.609151in}{3.121760in}}%
\pgfusepath{stroke}%
\end{pgfscope}%
\begin{pgfscope}%
\pgfpathrectangle{\pgfqpoint{1.250000in}{1.750000in}}{\pgfqpoint{2.279412in}{2.004545in}}%
\pgfusepath{clip}%
\pgfsetbuttcap%
\pgfsetroundjoin%
\pgfsetlinewidth{0.433833pt}%
\definecolor{currentstroke}{rgb}{0.283091,0.110553,0.431554}%
\pgfsetstrokecolor{currentstroke}%
\pgfsetdash{}{0pt}%
\pgfpathmoveto{\pgfqpoint{2.609151in}{3.121760in}}%
\pgfpathlineto{\pgfqpoint{2.560246in}{3.112090in}}%
\pgfusepath{stroke}%
\end{pgfscope}%
\begin{pgfscope}%
\pgfpathrectangle{\pgfqpoint{1.250000in}{1.750000in}}{\pgfqpoint{2.279412in}{2.004545in}}%
\pgfusepath{clip}%
\pgfsetbuttcap%
\pgfsetroundjoin%
\pgfsetlinewidth{0.434481pt}%
\definecolor{currentstroke}{rgb}{0.283091,0.110553,0.431554}%
\pgfsetstrokecolor{currentstroke}%
\pgfsetdash{}{0pt}%
\pgfpathmoveto{\pgfqpoint{2.560246in}{3.112090in}}%
\pgfpathlineto{\pgfqpoint{2.512089in}{3.099967in}}%
\pgfusepath{stroke}%
\end{pgfscope}%
\begin{pgfscope}%
\pgfpathrectangle{\pgfqpoint{1.250000in}{1.750000in}}{\pgfqpoint{2.279412in}{2.004545in}}%
\pgfusepath{clip}%
\pgfsetbuttcap%
\pgfsetroundjoin%
\pgfsetlinewidth{0.432016pt}%
\definecolor{currentstroke}{rgb}{0.283091,0.110553,0.431554}%
\pgfsetstrokecolor{currentstroke}%
\pgfsetdash{}{0pt}%
\pgfpathmoveto{\pgfqpoint{2.512089in}{3.099967in}}%
\pgfpathlineto{\pgfqpoint{2.465219in}{3.084443in}}%
\pgfusepath{stroke}%
\end{pgfscope}%
\begin{pgfscope}%
\pgfpathrectangle{\pgfqpoint{1.250000in}{1.750000in}}{\pgfqpoint{2.279412in}{2.004545in}}%
\pgfusepath{clip}%
\pgfsetbuttcap%
\pgfsetroundjoin%
\pgfsetlinewidth{0.525397pt}%
\definecolor{currentstroke}{rgb}{0.278826,0.175490,0.483397}%
\pgfsetstrokecolor{currentstroke}%
\pgfsetdash{}{0pt}%
\pgfpathmoveto{\pgfqpoint{2.465219in}{3.084443in}}%
\pgfpathlineto{\pgfqpoint{2.419809in}{3.065944in}}%
\pgfusepath{stroke}%
\end{pgfscope}%
\begin{pgfscope}%
\pgfpathrectangle{\pgfqpoint{1.250000in}{1.750000in}}{\pgfqpoint{2.279412in}{2.004545in}}%
\pgfusepath{clip}%
\pgfsetbuttcap%
\pgfsetroundjoin%
\pgfsetlinewidth{0.519969pt}%
\definecolor{currentstroke}{rgb}{0.279574,0.170599,0.479997}%
\pgfsetstrokecolor{currentstroke}%
\pgfsetdash{}{0pt}%
\pgfpathmoveto{\pgfqpoint{2.419809in}{3.065944in}}%
\pgfpathlineto{\pgfqpoint{2.375612in}{3.045293in}}%
\pgfusepath{stroke}%
\end{pgfscope}%
\begin{pgfscope}%
\pgfpathrectangle{\pgfqpoint{1.250000in}{1.750000in}}{\pgfqpoint{2.279412in}{2.004545in}}%
\pgfusepath{clip}%
\pgfsetbuttcap%
\pgfsetroundjoin%
\pgfsetlinewidth{0.765358pt}%
\definecolor{currentstroke}{rgb}{0.223925,0.334994,0.548053}%
\pgfsetstrokecolor{currentstroke}%
\pgfsetdash{}{0pt}%
\pgfpathmoveto{\pgfqpoint{2.375612in}{3.045293in}}%
\pgfpathlineto{\pgfqpoint{2.332569in}{3.022719in}}%
\pgfusepath{stroke}%
\end{pgfscope}%
\begin{pgfscope}%
\pgfpathrectangle{\pgfqpoint{1.250000in}{1.750000in}}{\pgfqpoint{2.279412in}{2.004545in}}%
\pgfusepath{clip}%
\pgfsetbuttcap%
\pgfsetroundjoin%
\pgfsetlinewidth{0.922381pt}%
\definecolor{currentstroke}{rgb}{0.182256,0.426184,0.557120}%
\pgfsetstrokecolor{currentstroke}%
\pgfsetdash{}{0pt}%
\pgfpathmoveto{\pgfqpoint{2.332569in}{3.022719in}}%
\pgfpathlineto{\pgfqpoint{2.291074in}{2.998071in}}%
\pgfusepath{stroke}%
\end{pgfscope}%
\begin{pgfscope}%
\pgfpathrectangle{\pgfqpoint{1.250000in}{1.750000in}}{\pgfqpoint{2.279412in}{2.004545in}}%
\pgfusepath{clip}%
\pgfsetbuttcap%
\pgfsetroundjoin%
\pgfsetlinewidth{0.876047pt}%
\definecolor{currentstroke}{rgb}{0.194100,0.399323,0.555565}%
\pgfsetstrokecolor{currentstroke}%
\pgfsetdash{}{0pt}%
\pgfpathmoveto{\pgfqpoint{2.291074in}{2.998071in}}%
\pgfpathlineto{\pgfqpoint{2.251651in}{2.970915in}}%
\pgfusepath{stroke}%
\end{pgfscope}%
\begin{pgfscope}%
\pgfpathrectangle{\pgfqpoint{1.250000in}{1.750000in}}{\pgfqpoint{2.279412in}{2.004545in}}%
\pgfusepath{clip}%
\pgfsetbuttcap%
\pgfsetroundjoin%
\pgfsetlinewidth{0.831986pt}%
\definecolor{currentstroke}{rgb}{0.204903,0.375746,0.553533}%
\pgfsetstrokecolor{currentstroke}%
\pgfsetdash{}{0pt}%
\pgfpathmoveto{\pgfqpoint{2.251651in}{2.970915in}}%
\pgfpathlineto{\pgfqpoint{2.212149in}{2.943829in}}%
\pgfusepath{stroke}%
\end{pgfscope}%
\begin{pgfscope}%
\pgfpathrectangle{\pgfqpoint{1.250000in}{1.750000in}}{\pgfqpoint{2.279412in}{2.004545in}}%
\pgfusepath{clip}%
\pgfsetbuttcap%
\pgfsetroundjoin%
\pgfsetlinewidth{1.384768pt}%
\definecolor{currentstroke}{rgb}{0.146616,0.673050,0.508936}%
\pgfsetstrokecolor{currentstroke}%
\pgfsetdash{}{0pt}%
\pgfpathmoveto{\pgfqpoint{2.212149in}{2.943829in}}%
\pgfpathlineto{\pgfqpoint{2.172652in}{2.916741in}}%
\pgfusepath{stroke}%
\end{pgfscope}%
\begin{pgfscope}%
\pgfpathrectangle{\pgfqpoint{1.250000in}{1.750000in}}{\pgfqpoint{2.279412in}{2.004545in}}%
\pgfusepath{clip}%
\pgfsetbuttcap%
\pgfsetroundjoin%
\pgfsetlinewidth{1.497277pt}%
\definecolor{currentstroke}{rgb}{0.232815,0.732247,0.459277}%
\pgfsetstrokecolor{currentstroke}%
\pgfsetdash{}{0pt}%
\pgfpathmoveto{\pgfqpoint{2.172652in}{2.916741in}}%
\pgfpathlineto{\pgfqpoint{2.133353in}{2.889407in}}%
\pgfusepath{stroke}%
\end{pgfscope}%
\begin{pgfscope}%
\pgfpathrectangle{\pgfqpoint{1.250000in}{1.750000in}}{\pgfqpoint{2.279412in}{2.004545in}}%
\pgfusepath{clip}%
\pgfsetbuttcap%
\pgfsetroundjoin%
\pgfsetlinewidth{0.317110pt}%
\definecolor{currentstroke}{rgb}{0.269944,0.014625,0.341379}%
\pgfsetstrokecolor{currentstroke}%
\pgfsetdash{}{0pt}%
\pgfpathmoveto{\pgfqpoint{3.056501in}{2.256098in}}%
\pgfpathlineto{\pgfqpoint{3.006590in}{2.257762in}}%
\pgfusepath{stroke}%
\end{pgfscope}%
\begin{pgfscope}%
\pgfpathrectangle{\pgfqpoint{1.250000in}{1.750000in}}{\pgfqpoint{2.279412in}{2.004545in}}%
\pgfusepath{clip}%
\pgfsetbuttcap%
\pgfsetroundjoin%
\pgfsetlinewidth{0.316278pt}%
\definecolor{currentstroke}{rgb}{0.269944,0.014625,0.341379}%
\pgfsetstrokecolor{currentstroke}%
\pgfsetdash{}{0pt}%
\pgfpathmoveto{\pgfqpoint{3.006590in}{2.257762in}}%
\pgfpathlineto{\pgfqpoint{2.956602in}{2.260122in}}%
\pgfusepath{stroke}%
\end{pgfscope}%
\begin{pgfscope}%
\pgfpathrectangle{\pgfqpoint{1.250000in}{1.750000in}}{\pgfqpoint{2.279412in}{2.004545in}}%
\pgfusepath{clip}%
\pgfsetbuttcap%
\pgfsetroundjoin%
\pgfsetlinewidth{0.327154pt}%
\definecolor{currentstroke}{rgb}{0.271305,0.019942,0.347269}%
\pgfsetstrokecolor{currentstroke}%
\pgfsetdash{}{0pt}%
\pgfpathmoveto{\pgfqpoint{2.956602in}{2.260122in}}%
\pgfpathlineto{\pgfqpoint{2.906551in}{2.262696in}}%
\pgfusepath{stroke}%
\end{pgfscope}%
\begin{pgfscope}%
\pgfpathrectangle{\pgfqpoint{1.250000in}{1.750000in}}{\pgfqpoint{2.279412in}{2.004545in}}%
\pgfusepath{clip}%
\pgfsetbuttcap%
\pgfsetroundjoin%
\pgfsetlinewidth{0.338910pt}%
\definecolor{currentstroke}{rgb}{0.273809,0.031497,0.358853}%
\pgfsetstrokecolor{currentstroke}%
\pgfsetdash{}{0pt}%
\pgfpathmoveto{\pgfqpoint{2.906551in}{2.262696in}}%
\pgfpathlineto{\pgfqpoint{2.856667in}{2.266965in}}%
\pgfusepath{stroke}%
\end{pgfscope}%
\begin{pgfscope}%
\pgfpathrectangle{\pgfqpoint{1.250000in}{1.750000in}}{\pgfqpoint{2.279412in}{2.004545in}}%
\pgfusepath{clip}%
\pgfsetbuttcap%
\pgfsetroundjoin%
\pgfsetlinewidth{0.349828pt}%
\definecolor{currentstroke}{rgb}{0.276022,0.044167,0.370164}%
\pgfsetstrokecolor{currentstroke}%
\pgfsetdash{}{0pt}%
\pgfpathmoveto{\pgfqpoint{2.856667in}{2.266965in}}%
\pgfpathlineto{\pgfqpoint{2.806856in}{2.271929in}}%
\pgfusepath{stroke}%
\end{pgfscope}%
\begin{pgfscope}%
\pgfpathrectangle{\pgfqpoint{1.250000in}{1.750000in}}{\pgfqpoint{2.279412in}{2.004545in}}%
\pgfusepath{clip}%
\pgfsetbuttcap%
\pgfsetroundjoin%
\pgfsetlinewidth{0.360658pt}%
\definecolor{currentstroke}{rgb}{0.277018,0.050344,0.375715}%
\pgfsetstrokecolor{currentstroke}%
\pgfsetdash{}{0pt}%
\pgfpathmoveto{\pgfqpoint{2.806856in}{2.271929in}}%
\pgfpathlineto{\pgfqpoint{2.757006in}{2.276692in}}%
\pgfusepath{stroke}%
\end{pgfscope}%
\begin{pgfscope}%
\pgfpathrectangle{\pgfqpoint{1.250000in}{1.750000in}}{\pgfqpoint{2.279412in}{2.004545in}}%
\pgfusepath{clip}%
\pgfsetbuttcap%
\pgfsetroundjoin%
\pgfsetlinewidth{0.374608pt}%
\definecolor{currentstroke}{rgb}{0.278791,0.062145,0.386592}%
\pgfsetstrokecolor{currentstroke}%
\pgfsetdash{}{0pt}%
\pgfpathmoveto{\pgfqpoint{2.757006in}{2.276692in}}%
\pgfpathlineto{\pgfqpoint{2.707210in}{2.281872in}}%
\pgfusepath{stroke}%
\end{pgfscope}%
\begin{pgfscope}%
\pgfpathrectangle{\pgfqpoint{1.250000in}{1.750000in}}{\pgfqpoint{2.279412in}{2.004545in}}%
\pgfusepath{clip}%
\pgfsetbuttcap%
\pgfsetroundjoin%
\pgfsetlinewidth{0.378945pt}%
\definecolor{currentstroke}{rgb}{0.279566,0.067836,0.391917}%
\pgfsetstrokecolor{currentstroke}%
\pgfsetdash{}{0pt}%
\pgfpathmoveto{\pgfqpoint{2.707210in}{2.281872in}}%
\pgfpathlineto{\pgfqpoint{2.657858in}{2.289306in}}%
\pgfusepath{stroke}%
\end{pgfscope}%
\begin{pgfscope}%
\pgfpathrectangle{\pgfqpoint{1.250000in}{1.750000in}}{\pgfqpoint{2.279412in}{2.004545in}}%
\pgfusepath{clip}%
\pgfsetbuttcap%
\pgfsetroundjoin%
\pgfsetlinewidth{0.386684pt}%
\definecolor{currentstroke}{rgb}{0.280267,0.073417,0.397163}%
\pgfsetstrokecolor{currentstroke}%
\pgfsetdash{}{0pt}%
\pgfpathmoveto{\pgfqpoint{2.657858in}{2.289306in}}%
\pgfpathlineto{\pgfqpoint{2.608891in}{2.298690in}}%
\pgfusepath{stroke}%
\end{pgfscope}%
\begin{pgfscope}%
\pgfpathrectangle{\pgfqpoint{1.250000in}{1.750000in}}{\pgfqpoint{2.279412in}{2.004545in}}%
\pgfusepath{clip}%
\pgfsetbuttcap%
\pgfsetroundjoin%
\pgfsetlinewidth{0.323382pt}%
\definecolor{currentstroke}{rgb}{0.271305,0.019942,0.347269}%
\pgfsetstrokecolor{currentstroke}%
\pgfsetdash{}{0pt}%
\pgfpathmoveto{\pgfqpoint{3.056501in}{3.203341in}}%
\pgfpathlineto{\pgfqpoint{3.006514in}{3.200802in}}%
\pgfusepath{stroke}%
\end{pgfscope}%
\begin{pgfscope}%
\pgfpathrectangle{\pgfqpoint{1.250000in}{1.750000in}}{\pgfqpoint{2.279412in}{2.004545in}}%
\pgfusepath{clip}%
\pgfsetbuttcap%
\pgfsetroundjoin%
\pgfsetlinewidth{0.326274pt}%
\definecolor{currentstroke}{rgb}{0.271305,0.019942,0.347269}%
\pgfsetstrokecolor{currentstroke}%
\pgfsetdash{}{0pt}%
\pgfpathmoveto{\pgfqpoint{3.006514in}{3.200802in}}%
\pgfpathlineto{\pgfqpoint{2.956422in}{3.198997in}}%
\pgfusepath{stroke}%
\end{pgfscope}%
\begin{pgfscope}%
\pgfpathrectangle{\pgfqpoint{1.250000in}{1.750000in}}{\pgfqpoint{2.279412in}{2.004545in}}%
\pgfusepath{clip}%
\pgfsetbuttcap%
\pgfsetroundjoin%
\pgfsetlinewidth{0.338646pt}%
\definecolor{currentstroke}{rgb}{0.273809,0.031497,0.358853}%
\pgfsetstrokecolor{currentstroke}%
\pgfsetdash{}{0pt}%
\pgfpathmoveto{\pgfqpoint{2.956422in}{3.198997in}}%
\pgfpathlineto{\pgfqpoint{2.906342in}{3.196825in}}%
\pgfusepath{stroke}%
\end{pgfscope}%
\begin{pgfscope}%
\pgfpathrectangle{\pgfqpoint{1.250000in}{1.750000in}}{\pgfqpoint{2.279412in}{2.004545in}}%
\pgfusepath{clip}%
\pgfsetbuttcap%
\pgfsetroundjoin%
\pgfsetlinewidth{0.344738pt}%
\definecolor{currentstroke}{rgb}{0.274952,0.037752,0.364543}%
\pgfsetstrokecolor{currentstroke}%
\pgfsetdash{}{0pt}%
\pgfpathmoveto{\pgfqpoint{2.906342in}{3.196825in}}%
\pgfpathlineto{\pgfqpoint{2.856337in}{3.193632in}}%
\pgfusepath{stroke}%
\end{pgfscope}%
\begin{pgfscope}%
\pgfpathrectangle{\pgfqpoint{1.250000in}{1.750000in}}{\pgfqpoint{2.279412in}{2.004545in}}%
\pgfusepath{clip}%
\pgfsetbuttcap%
\pgfsetroundjoin%
\pgfsetlinewidth{0.362418pt}%
\definecolor{currentstroke}{rgb}{0.277018,0.050344,0.375715}%
\pgfsetstrokecolor{currentstroke}%
\pgfsetdash{}{0pt}%
\pgfpathmoveto{\pgfqpoint{2.856337in}{3.193632in}}%
\pgfpathlineto{\pgfqpoint{2.806525in}{3.188681in}}%
\pgfusepath{stroke}%
\end{pgfscope}%
\begin{pgfscope}%
\pgfpathrectangle{\pgfqpoint{1.250000in}{1.750000in}}{\pgfqpoint{2.279412in}{2.004545in}}%
\pgfusepath{clip}%
\pgfsetbuttcap%
\pgfsetroundjoin%
\pgfsetlinewidth{0.339778pt}%
\definecolor{currentstroke}{rgb}{0.273809,0.031497,0.358853}%
\pgfsetstrokecolor{currentstroke}%
\pgfsetdash{}{0pt}%
\pgfpathmoveto{\pgfqpoint{2.806525in}{3.188681in}}%
\pgfpathlineto{\pgfqpoint{2.756774in}{3.183266in}}%
\pgfusepath{stroke}%
\end{pgfscope}%
\begin{pgfscope}%
\pgfpathrectangle{\pgfqpoint{1.250000in}{1.750000in}}{\pgfqpoint{2.279412in}{2.004545in}}%
\pgfusepath{clip}%
\pgfsetbuttcap%
\pgfsetroundjoin%
\pgfsetlinewidth{0.318889pt}%
\definecolor{currentstroke}{rgb}{0.269944,0.014625,0.341379}%
\pgfsetstrokecolor{currentstroke}%
\pgfsetdash{}{0pt}%
\pgfpathmoveto{\pgfqpoint{3.056501in}{3.248447in}}%
\pgfpathlineto{\pgfqpoint{3.006689in}{3.244878in}}%
\pgfusepath{stroke}%
\end{pgfscope}%
\begin{pgfscope}%
\pgfpathrectangle{\pgfqpoint{1.250000in}{1.750000in}}{\pgfqpoint{2.279412in}{2.004545in}}%
\pgfusepath{clip}%
\pgfsetbuttcap%
\pgfsetroundjoin%
\pgfsetlinewidth{0.325662pt}%
\definecolor{currentstroke}{rgb}{0.271305,0.019942,0.347269}%
\pgfsetstrokecolor{currentstroke}%
\pgfsetdash{}{0pt}%
\pgfpathmoveto{\pgfqpoint{3.006689in}{3.244878in}}%
\pgfpathlineto{\pgfqpoint{2.956917in}{3.240122in}}%
\pgfusepath{stroke}%
\end{pgfscope}%
\begin{pgfscope}%
\pgfpathrectangle{\pgfqpoint{1.250000in}{1.750000in}}{\pgfqpoint{2.279412in}{2.004545in}}%
\pgfusepath{clip}%
\pgfsetbuttcap%
\pgfsetroundjoin%
\pgfsetlinewidth{0.340760pt}%
\definecolor{currentstroke}{rgb}{0.273809,0.031497,0.358853}%
\pgfsetstrokecolor{currentstroke}%
\pgfsetdash{}{0pt}%
\pgfpathmoveto{\pgfqpoint{2.956917in}{3.240122in}}%
\pgfpathlineto{\pgfqpoint{2.906978in}{3.236492in}}%
\pgfusepath{stroke}%
\end{pgfscope}%
\begin{pgfscope}%
\pgfpathrectangle{\pgfqpoint{1.250000in}{1.750000in}}{\pgfqpoint{2.279412in}{2.004545in}}%
\pgfusepath{clip}%
\pgfsetbuttcap%
\pgfsetroundjoin%
\pgfsetlinewidth{0.334152pt}%
\definecolor{currentstroke}{rgb}{0.272594,0.025563,0.353093}%
\pgfsetstrokecolor{currentstroke}%
\pgfsetdash{}{0pt}%
\pgfpathmoveto{\pgfqpoint{2.906978in}{3.236492in}}%
\pgfpathlineto{\pgfqpoint{2.857119in}{3.232025in}}%
\pgfusepath{stroke}%
\end{pgfscope}%
\begin{pgfscope}%
\pgfpathrectangle{\pgfqpoint{1.250000in}{1.750000in}}{\pgfqpoint{2.279412in}{2.004545in}}%
\pgfusepath{clip}%
\pgfsetbuttcap%
\pgfsetroundjoin%
\pgfsetlinewidth{0.345433pt}%
\definecolor{currentstroke}{rgb}{0.274952,0.037752,0.364543}%
\pgfsetstrokecolor{currentstroke}%
\pgfsetdash{}{0pt}%
\pgfpathmoveto{\pgfqpoint{2.857119in}{3.232025in}}%
\pgfpathlineto{\pgfqpoint{2.807304in}{3.227004in}}%
\pgfusepath{stroke}%
\end{pgfscope}%
\begin{pgfscope}%
\pgfpathrectangle{\pgfqpoint{1.250000in}{1.750000in}}{\pgfqpoint{2.279412in}{2.004545in}}%
\pgfusepath{clip}%
\pgfsetbuttcap%
\pgfsetroundjoin%
\pgfsetlinewidth{0.318203pt}%
\definecolor{currentstroke}{rgb}{0.269944,0.014625,0.341379}%
\pgfsetstrokecolor{currentstroke}%
\pgfsetdash{}{0pt}%
\pgfpathmoveto{\pgfqpoint{3.056501in}{3.293554in}}%
\pgfpathlineto{\pgfqpoint{3.006868in}{3.291340in}}%
\pgfusepath{stroke}%
\end{pgfscope}%
\begin{pgfscope}%
\pgfpathrectangle{\pgfqpoint{1.250000in}{1.750000in}}{\pgfqpoint{2.279412in}{2.004545in}}%
\pgfusepath{clip}%
\pgfsetbuttcap%
\pgfsetroundjoin%
\pgfsetlinewidth{0.323240pt}%
\definecolor{currentstroke}{rgb}{0.271305,0.019942,0.347269}%
\pgfsetstrokecolor{currentstroke}%
\pgfsetdash{}{0pt}%
\pgfpathmoveto{\pgfqpoint{3.006868in}{3.291340in}}%
\pgfpathlineto{\pgfqpoint{2.957191in}{3.287323in}}%
\pgfusepath{stroke}%
\end{pgfscope}%
\begin{pgfscope}%
\pgfpathrectangle{\pgfqpoint{1.250000in}{1.750000in}}{\pgfqpoint{2.279412in}{2.004545in}}%
\pgfusepath{clip}%
\pgfsetbuttcap%
\pgfsetroundjoin%
\pgfsetlinewidth{0.316501pt}%
\definecolor{currentstroke}{rgb}{0.269944,0.014625,0.341379}%
\pgfsetstrokecolor{currentstroke}%
\pgfsetdash{}{0pt}%
\pgfpathmoveto{\pgfqpoint{2.957191in}{3.287323in}}%
\pgfpathlineto{\pgfqpoint{2.907659in}{3.281367in}}%
\pgfusepath{stroke}%
\end{pgfscope}%
\begin{pgfscope}%
\pgfpathrectangle{\pgfqpoint{1.250000in}{1.750000in}}{\pgfqpoint{2.279412in}{2.004545in}}%
\pgfusepath{clip}%
\pgfsetbuttcap%
\pgfsetroundjoin%
\pgfsetlinewidth{0.326744pt}%
\definecolor{currentstroke}{rgb}{0.271305,0.019942,0.347269}%
\pgfsetstrokecolor{currentstroke}%
\pgfsetdash{}{0pt}%
\pgfpathmoveto{\pgfqpoint{2.907659in}{3.281367in}}%
\pgfpathlineto{\pgfqpoint{2.857815in}{3.276595in}}%
\pgfusepath{stroke}%
\end{pgfscope}%
\begin{pgfscope}%
\pgfpathrectangle{\pgfqpoint{1.250000in}{1.750000in}}{\pgfqpoint{2.279412in}{2.004545in}}%
\pgfusepath{clip}%
\pgfsetbuttcap%
\pgfsetroundjoin%
\pgfsetlinewidth{0.340024pt}%
\definecolor{currentstroke}{rgb}{0.273809,0.031497,0.358853}%
\pgfsetstrokecolor{currentstroke}%
\pgfsetdash{}{0pt}%
\pgfpathmoveto{\pgfqpoint{2.857815in}{3.276595in}}%
\pgfpathlineto{\pgfqpoint{2.807952in}{3.272043in}}%
\pgfusepath{stroke}%
\end{pgfscope}%
\begin{pgfscope}%
\pgfpathrectangle{\pgfqpoint{1.250000in}{1.750000in}}{\pgfqpoint{2.279412in}{2.004545in}}%
\pgfusepath{clip}%
\pgfsetbuttcap%
\pgfsetroundjoin%
\pgfsetlinewidth{0.343765pt}%
\definecolor{currentstroke}{rgb}{0.274952,0.037752,0.364543}%
\pgfsetstrokecolor{currentstroke}%
\pgfsetdash{}{0pt}%
\pgfpathmoveto{\pgfqpoint{2.807952in}{3.272043in}}%
\pgfpathlineto{\pgfqpoint{2.758087in}{3.267667in}}%
\pgfusepath{stroke}%
\end{pgfscope}%
\begin{pgfscope}%
\pgfpathrectangle{\pgfqpoint{1.250000in}{1.750000in}}{\pgfqpoint{2.279412in}{2.004545in}}%
\pgfusepath{clip}%
\pgfsetbuttcap%
\pgfsetroundjoin%
\pgfsetlinewidth{0.322446pt}%
\definecolor{currentstroke}{rgb}{0.271305,0.019942,0.347269}%
\pgfsetstrokecolor{currentstroke}%
\pgfsetdash{}{0pt}%
\pgfpathmoveto{\pgfqpoint{2.945372in}{2.181084in}}%
\pgfpathlineto{\pgfqpoint{2.895548in}{2.185015in}}%
\pgfusepath{stroke}%
\end{pgfscope}%
\begin{pgfscope}%
\pgfpathrectangle{\pgfqpoint{1.250000in}{1.750000in}}{\pgfqpoint{2.279412in}{2.004545in}}%
\pgfusepath{clip}%
\pgfsetbuttcap%
\pgfsetroundjoin%
\pgfsetlinewidth{0.328434pt}%
\definecolor{currentstroke}{rgb}{0.271305,0.019942,0.347269}%
\pgfsetstrokecolor{currentstroke}%
\pgfsetdash{}{0pt}%
\pgfpathmoveto{\pgfqpoint{2.895548in}{2.185015in}}%
\pgfpathlineto{\pgfqpoint{2.845793in}{2.189731in}}%
\pgfusepath{stroke}%
\end{pgfscope}%
\begin{pgfscope}%
\pgfpathrectangle{\pgfqpoint{1.250000in}{1.750000in}}{\pgfqpoint{2.279412in}{2.004545in}}%
\pgfusepath{clip}%
\pgfsetbuttcap%
\pgfsetroundjoin%
\pgfsetlinewidth{0.336762pt}%
\definecolor{currentstroke}{rgb}{0.273809,0.031497,0.358853}%
\pgfsetstrokecolor{currentstroke}%
\pgfsetdash{}{0pt}%
\pgfpathmoveto{\pgfqpoint{2.845793in}{2.189731in}}%
\pgfpathlineto{\pgfqpoint{2.796203in}{2.195951in}}%
\pgfusepath{stroke}%
\end{pgfscope}%
\begin{pgfscope}%
\pgfpathrectangle{\pgfqpoint{1.250000in}{1.750000in}}{\pgfqpoint{2.279412in}{2.004545in}}%
\pgfusepath{clip}%
\pgfsetbuttcap%
\pgfsetroundjoin%
\pgfsetlinewidth{0.356941pt}%
\definecolor{currentstroke}{rgb}{0.277018,0.050344,0.375715}%
\pgfsetstrokecolor{currentstroke}%
\pgfsetdash{}{0pt}%
\pgfpathmoveto{\pgfqpoint{2.796203in}{2.195951in}}%
\pgfpathlineto{\pgfqpoint{2.746858in}{2.203620in}}%
\pgfusepath{stroke}%
\end{pgfscope}%
\begin{pgfscope}%
\pgfpathrectangle{\pgfqpoint{1.250000in}{1.750000in}}{\pgfqpoint{2.279412in}{2.004545in}}%
\pgfusepath{clip}%
\pgfsetbuttcap%
\pgfsetroundjoin%
\pgfsetlinewidth{0.356794pt}%
\definecolor{currentstroke}{rgb}{0.277018,0.050344,0.375715}%
\pgfsetstrokecolor{currentstroke}%
\pgfsetdash{}{0pt}%
\pgfpathmoveto{\pgfqpoint{2.746858in}{2.203620in}}%
\pgfpathlineto{\pgfqpoint{2.697457in}{2.210991in}}%
\pgfusepath{stroke}%
\end{pgfscope}%
\begin{pgfscope}%
\pgfpathrectangle{\pgfqpoint{1.250000in}{1.750000in}}{\pgfqpoint{2.279412in}{2.004545in}}%
\pgfusepath{clip}%
\pgfsetbuttcap%
\pgfsetroundjoin%
\pgfsetlinewidth{0.351657pt}%
\definecolor{currentstroke}{rgb}{0.276022,0.044167,0.370164}%
\pgfsetstrokecolor{currentstroke}%
\pgfsetdash{}{0pt}%
\pgfpathmoveto{\pgfqpoint{2.697457in}{2.210991in}}%
\pgfpathlineto{\pgfqpoint{2.647973in}{2.218055in}}%
\pgfusepath{stroke}%
\end{pgfscope}%
\begin{pgfscope}%
\pgfpathrectangle{\pgfqpoint{1.250000in}{1.750000in}}{\pgfqpoint{2.279412in}{2.004545in}}%
\pgfusepath{clip}%
\pgfsetbuttcap%
\pgfsetroundjoin%
\pgfsetlinewidth{0.360877pt}%
\definecolor{currentstroke}{rgb}{0.277018,0.050344,0.375715}%
\pgfsetstrokecolor{currentstroke}%
\pgfsetdash{}{0pt}%
\pgfpathmoveto{\pgfqpoint{2.693104in}{2.238360in}}%
\pgfpathlineto{\pgfqpoint{2.643916in}{2.246917in}}%
\pgfusepath{stroke}%
\end{pgfscope}%
\begin{pgfscope}%
\pgfpathrectangle{\pgfqpoint{1.250000in}{1.750000in}}{\pgfqpoint{2.279412in}{2.004545in}}%
\pgfusepath{clip}%
\pgfsetbuttcap%
\pgfsetroundjoin%
\pgfsetlinewidth{0.331253pt}%
\definecolor{currentstroke}{rgb}{0.272594,0.025563,0.353093}%
\pgfsetstrokecolor{currentstroke}%
\pgfsetdash{}{0pt}%
\pgfpathmoveto{\pgfqpoint{2.643916in}{2.246917in}}%
\pgfpathlineto{\pgfqpoint{2.594873in}{2.256098in}}%
\pgfusepath{stroke}%
\end{pgfscope}%
\begin{pgfscope}%
\pgfpathrectangle{\pgfqpoint{1.250000in}{1.750000in}}{\pgfqpoint{2.279412in}{2.004545in}}%
\pgfusepath{clip}%
\pgfsetbuttcap%
\pgfsetroundjoin%
\pgfsetlinewidth{0.363590pt}%
\definecolor{currentstroke}{rgb}{0.277941,0.056324,0.381191}%
\pgfsetstrokecolor{currentstroke}%
\pgfsetdash{}{0pt}%
\pgfpathmoveto{\pgfqpoint{2.594873in}{2.256098in}}%
\pgfpathlineto{\pgfqpoint{2.546385in}{2.267223in}}%
\pgfusepath{stroke}%
\end{pgfscope}%
\begin{pgfscope}%
\pgfpathrectangle{\pgfqpoint{1.250000in}{1.750000in}}{\pgfqpoint{2.279412in}{2.004545in}}%
\pgfusepath{clip}%
\pgfsetbuttcap%
\pgfsetroundjoin%
\pgfsetlinewidth{0.391029pt}%
\definecolor{currentstroke}{rgb}{0.280894,0.078907,0.402329}%
\pgfsetstrokecolor{currentstroke}%
\pgfsetdash{}{0pt}%
\pgfpathmoveto{\pgfqpoint{2.546385in}{2.267223in}}%
\pgfpathlineto{\pgfqpoint{2.499251in}{2.281927in}}%
\pgfusepath{stroke}%
\end{pgfscope}%
\begin{pgfscope}%
\pgfpathrectangle{\pgfqpoint{1.250000in}{1.750000in}}{\pgfqpoint{2.279412in}{2.004545in}}%
\pgfusepath{clip}%
\pgfsetbuttcap%
\pgfsetroundjoin%
\pgfsetlinewidth{0.398366pt}%
\definecolor{currentstroke}{rgb}{0.281446,0.084320,0.407414}%
\pgfsetstrokecolor{currentstroke}%
\pgfsetdash{}{0pt}%
\pgfpathmoveto{\pgfqpoint{2.499251in}{2.281927in}}%
\pgfpathlineto{\pgfqpoint{2.453557in}{2.299759in}}%
\pgfusepath{stroke}%
\end{pgfscope}%
\begin{pgfscope}%
\pgfpathrectangle{\pgfqpoint{1.250000in}{1.750000in}}{\pgfqpoint{2.279412in}{2.004545in}}%
\pgfusepath{clip}%
\pgfsetbuttcap%
\pgfsetroundjoin%
\pgfsetlinewidth{0.361789pt}%
\definecolor{currentstroke}{rgb}{0.277018,0.050344,0.375715}%
\pgfsetstrokecolor{currentstroke}%
\pgfsetdash{}{0pt}%
\pgfpathmoveto{\pgfqpoint{2.690339in}{3.231318in}}%
\pgfpathlineto{\pgfqpoint{2.640874in}{3.224323in}}%
\pgfusepath{stroke}%
\end{pgfscope}%
\begin{pgfscope}%
\pgfpathrectangle{\pgfqpoint{1.250000in}{1.750000in}}{\pgfqpoint{2.279412in}{2.004545in}}%
\pgfusepath{clip}%
\pgfsetbuttcap%
\pgfsetroundjoin%
\pgfsetlinewidth{0.370121pt}%
\definecolor{currentstroke}{rgb}{0.278791,0.062145,0.386592}%
\pgfsetstrokecolor{currentstroke}%
\pgfsetdash{}{0pt}%
\pgfpathmoveto{\pgfqpoint{2.640874in}{3.224323in}}%
\pgfpathlineto{\pgfqpoint{2.591978in}{3.214766in}}%
\pgfusepath{stroke}%
\end{pgfscope}%
\begin{pgfscope}%
\pgfpathrectangle{\pgfqpoint{1.250000in}{1.750000in}}{\pgfqpoint{2.279412in}{2.004545in}}%
\pgfusepath{clip}%
\pgfsetbuttcap%
\pgfsetroundjoin%
\pgfsetlinewidth{0.370805pt}%
\definecolor{currentstroke}{rgb}{0.278791,0.062145,0.386592}%
\pgfsetstrokecolor{currentstroke}%
\pgfsetdash{}{0pt}%
\pgfpathmoveto{\pgfqpoint{2.591978in}{3.214766in}}%
\pgfpathlineto{\pgfqpoint{2.543582in}{3.203341in}}%
\pgfusepath{stroke}%
\end{pgfscope}%
\begin{pgfscope}%
\pgfpathrectangle{\pgfqpoint{1.250000in}{1.750000in}}{\pgfqpoint{2.279412in}{2.004545in}}%
\pgfusepath{clip}%
\pgfsetbuttcap%
\pgfsetroundjoin%
\pgfsetlinewidth{0.394037pt}%
\definecolor{currentstroke}{rgb}{0.280894,0.078907,0.402329}%
\pgfsetstrokecolor{currentstroke}%
\pgfsetdash{}{0pt}%
\pgfpathmoveto{\pgfqpoint{2.543582in}{3.203341in}}%
\pgfpathlineto{\pgfqpoint{2.495827in}{3.190304in}}%
\pgfusepath{stroke}%
\end{pgfscope}%
\begin{pgfscope}%
\pgfpathrectangle{\pgfqpoint{1.250000in}{1.750000in}}{\pgfqpoint{2.279412in}{2.004545in}}%
\pgfusepath{clip}%
\pgfsetbuttcap%
\pgfsetroundjoin%
\pgfsetlinewidth{0.347209pt}%
\definecolor{currentstroke}{rgb}{0.274952,0.037752,0.364543}%
\pgfsetstrokecolor{currentstroke}%
\pgfsetdash{}{0pt}%
\pgfpathmoveto{\pgfqpoint{2.495827in}{3.190304in}}%
\pgfpathlineto{\pgfqpoint{2.450745in}{3.171902in}}%
\pgfusepath{stroke}%
\end{pgfscope}%
\begin{pgfscope}%
\pgfpathrectangle{\pgfqpoint{1.250000in}{1.750000in}}{\pgfqpoint{2.279412in}{2.004545in}}%
\pgfusepath{clip}%
\pgfsetbuttcap%
\pgfsetroundjoin%
\pgfsetlinewidth{0.728689pt}%
\definecolor{currentstroke}{rgb}{0.233603,0.313828,0.543914}%
\pgfsetstrokecolor{currentstroke}%
\pgfsetdash{}{0pt}%
\pgfpathmoveto{\pgfqpoint{2.287122in}{2.436525in}}%
\pgfpathlineto{\pgfqpoint{2.251431in}{2.467453in}}%
\pgfusepath{stroke}%
\end{pgfscope}%
\begin{pgfscope}%
\pgfpathrectangle{\pgfqpoint{1.250000in}{1.750000in}}{\pgfqpoint{2.279412in}{2.004545in}}%
\pgfusepath{clip}%
\pgfsetbuttcap%
\pgfsetroundjoin%
\pgfsetlinewidth{0.965641pt}%
\definecolor{currentstroke}{rgb}{0.172719,0.448791,0.557885}%
\pgfsetstrokecolor{currentstroke}%
\pgfsetdash{}{0pt}%
\pgfpathmoveto{\pgfqpoint{2.251431in}{2.467453in}}%
\pgfpathlineto{\pgfqpoint{2.221269in}{2.501016in}}%
\pgfusepath{stroke}%
\end{pgfscope}%
\begin{pgfscope}%
\pgfpathrectangle{\pgfqpoint{1.250000in}{1.750000in}}{\pgfqpoint{2.279412in}{2.004545in}}%
\pgfusepath{clip}%
\pgfsetbuttcap%
\pgfsetroundjoin%
\pgfsetlinewidth{0.698171pt}%
\definecolor{currentstroke}{rgb}{0.243113,0.292092,0.538516}%
\pgfsetstrokecolor{currentstroke}%
\pgfsetdash{}{0pt}%
\pgfpathmoveto{\pgfqpoint{2.221269in}{2.501016in}}%
\pgfpathlineto{\pgfqpoint{2.189060in}{2.530376in}}%
\pgfusepath{stroke}%
\end{pgfscope}%
\begin{pgfscope}%
\pgfpathrectangle{\pgfqpoint{1.250000in}{1.750000in}}{\pgfqpoint{2.279412in}{2.004545in}}%
\pgfusepath{clip}%
\pgfsetbuttcap%
\pgfsetroundjoin%
\pgfsetlinewidth{1.236329pt}%
\definecolor{currentstroke}{rgb}{0.120565,0.596422,0.543611}%
\pgfsetstrokecolor{currentstroke}%
\pgfsetdash{}{0pt}%
\pgfpathmoveto{\pgfqpoint{2.189060in}{2.530376in}}%
\pgfpathlineto{\pgfqpoint{2.152327in}{2.560229in}}%
\pgfusepath{stroke}%
\end{pgfscope}%
\begin{pgfscope}%
\pgfpathrectangle{\pgfqpoint{1.250000in}{1.750000in}}{\pgfqpoint{2.279412in}{2.004545in}}%
\pgfusepath{clip}%
\pgfsetbuttcap%
\pgfsetroundjoin%
\pgfsetlinewidth{1.787770pt}%
\definecolor{currentstroke}{rgb}{0.595839,0.848717,0.243329}%
\pgfsetstrokecolor{currentstroke}%
\pgfsetdash{}{0pt}%
\pgfpathmoveto{\pgfqpoint{2.152327in}{2.560229in}}%
\pgfpathlineto{\pgfqpoint{2.114821in}{2.589345in}}%
\pgfusepath{stroke}%
\end{pgfscope}%
\begin{pgfscope}%
\pgfpathrectangle{\pgfqpoint{1.250000in}{1.750000in}}{\pgfqpoint{2.279412in}{2.004545in}}%
\pgfusepath{clip}%
\pgfsetbuttcap%
\pgfsetroundjoin%
\pgfsetlinewidth{0.434665pt}%
\definecolor{currentstroke}{rgb}{0.283091,0.110553,0.431554}%
\pgfsetstrokecolor{currentstroke}%
\pgfsetdash{}{0pt}%
\pgfpathmoveto{\pgfqpoint{2.691051in}{3.093301in}}%
\pgfpathlineto{\pgfqpoint{2.641407in}{3.087063in}}%
\pgfusepath{stroke}%
\end{pgfscope}%
\begin{pgfscope}%
\pgfpathrectangle{\pgfqpoint{1.250000in}{1.750000in}}{\pgfqpoint{2.279412in}{2.004545in}}%
\pgfusepath{clip}%
\pgfsetbuttcap%
\pgfsetroundjoin%
\pgfsetlinewidth{0.451864pt}%
\definecolor{currentstroke}{rgb}{0.283229,0.120777,0.440584}%
\pgfsetstrokecolor{currentstroke}%
\pgfsetdash{}{0pt}%
\pgfpathmoveto{\pgfqpoint{2.641407in}{3.087063in}}%
\pgfpathlineto{\pgfqpoint{2.592182in}{3.078845in}}%
\pgfusepath{stroke}%
\end{pgfscope}%
\begin{pgfscope}%
\pgfpathrectangle{\pgfqpoint{1.250000in}{1.750000in}}{\pgfqpoint{2.279412in}{2.004545in}}%
\pgfusepath{clip}%
\pgfsetbuttcap%
\pgfsetroundjoin%
\pgfsetlinewidth{0.502956pt}%
\definecolor{currentstroke}{rgb}{0.280868,0.160771,0.472899}%
\pgfsetstrokecolor{currentstroke}%
\pgfsetdash{}{0pt}%
\pgfpathmoveto{\pgfqpoint{2.592182in}{3.078845in}}%
\pgfpathlineto{\pgfqpoint{2.543582in}{3.068020in}}%
\pgfusepath{stroke}%
\end{pgfscope}%
\begin{pgfscope}%
\pgfpathrectangle{\pgfqpoint{1.250000in}{1.750000in}}{\pgfqpoint{2.279412in}{2.004545in}}%
\pgfusepath{clip}%
\pgfsetbuttcap%
\pgfsetroundjoin%
\pgfsetlinewidth{0.497151pt}%
\definecolor{currentstroke}{rgb}{0.281412,0.155834,0.469201}%
\pgfsetstrokecolor{currentstroke}%
\pgfsetdash{}{0pt}%
\pgfpathmoveto{\pgfqpoint{2.543582in}{3.068020in}}%
\pgfpathlineto{\pgfqpoint{2.495987in}{3.054566in}}%
\pgfusepath{stroke}%
\end{pgfscope}%
\begin{pgfscope}%
\pgfpathrectangle{\pgfqpoint{1.250000in}{1.750000in}}{\pgfqpoint{2.279412in}{2.004545in}}%
\pgfusepath{clip}%
\pgfsetbuttcap%
\pgfsetroundjoin%
\pgfsetlinewidth{0.432602pt}%
\definecolor{currentstroke}{rgb}{0.283091,0.110553,0.431554}%
\pgfsetstrokecolor{currentstroke}%
\pgfsetdash{}{0pt}%
\pgfpathmoveto{\pgfqpoint{2.495987in}{3.054566in}}%
\pgfpathlineto{\pgfqpoint{2.449233in}{3.038872in}}%
\pgfusepath{stroke}%
\end{pgfscope}%
\begin{pgfscope}%
\pgfpathrectangle{\pgfqpoint{1.250000in}{1.750000in}}{\pgfqpoint{2.279412in}{2.004545in}}%
\pgfusepath{clip}%
\pgfsetbuttcap%
\pgfsetroundjoin%
\pgfsetlinewidth{0.791914pt}%
\definecolor{currentstroke}{rgb}{0.216210,0.351535,0.550627}%
\pgfsetstrokecolor{currentstroke}%
\pgfsetdash{}{0pt}%
\pgfpathmoveto{\pgfqpoint{2.133246in}{3.068020in}}%
\pgfpathlineto{\pgfqpoint{2.133246in}{3.068020in}}%
\pgfusepath{stroke}%
\end{pgfscope}%
\begin{pgfscope}%
\pgfpathrectangle{\pgfqpoint{1.250000in}{1.750000in}}{\pgfqpoint{2.279412in}{2.004545in}}%
\pgfusepath{clip}%
\pgfsetbuttcap%
\pgfsetroundjoin%
\pgfsetlinewidth{0.791914pt}%
\definecolor{currentstroke}{rgb}{0.216210,0.351535,0.550627}%
\pgfsetstrokecolor{currentstroke}%
\pgfsetdash{}{0pt}%
\pgfpathmoveto{\pgfqpoint{2.133246in}{3.068020in}}%
\pgfpathlineto{\pgfqpoint{2.135228in}{3.036278in}}%
\pgfusepath{stroke}%
\end{pgfscope}%
\begin{pgfscope}%
\pgfpathrectangle{\pgfqpoint{1.250000in}{1.750000in}}{\pgfqpoint{2.279412in}{2.004545in}}%
\pgfusepath{clip}%
\pgfsetbuttcap%
\pgfsetroundjoin%
\pgfsetlinewidth{1.017424pt}%
\definecolor{currentstroke}{rgb}{0.160665,0.478540,0.558115}%
\pgfsetstrokecolor{currentstroke}%
\pgfsetdash{}{0pt}%
\pgfpathmoveto{\pgfqpoint{2.135228in}{3.036278in}}%
\pgfpathlineto{\pgfqpoint{2.132559in}{3.007114in}}%
\pgfusepath{stroke}%
\end{pgfscope}%
\begin{pgfscope}%
\pgfpathrectangle{\pgfqpoint{1.250000in}{1.750000in}}{\pgfqpoint{2.279412in}{2.004545in}}%
\pgfusepath{clip}%
\pgfsetbuttcap%
\pgfsetroundjoin%
\pgfsetlinewidth{0.966536pt}%
\definecolor{currentstroke}{rgb}{0.172719,0.448791,0.557885}%
\pgfsetstrokecolor{currentstroke}%
\pgfsetdash{}{0pt}%
\pgfpathmoveto{\pgfqpoint{2.132559in}{3.007114in}}%
\pgfpathlineto{\pgfqpoint{2.132559in}{3.007114in}}%
\pgfusepath{stroke}%
\end{pgfscope}%
\begin{pgfscope}%
\pgfpathrectangle{\pgfqpoint{1.250000in}{1.750000in}}{\pgfqpoint{2.279412in}{2.004545in}}%
\pgfusepath{clip}%
\pgfsetbuttcap%
\pgfsetroundjoin%
\pgfsetlinewidth{0.966536pt}%
\definecolor{currentstroke}{rgb}{0.172719,0.448791,0.557885}%
\pgfsetstrokecolor{currentstroke}%
\pgfsetdash{}{0pt}%
\pgfpathmoveto{\pgfqpoint{2.132559in}{3.007114in}}%
\pgfpathlineto{\pgfqpoint{2.128517in}{2.977625in}}%
\pgfusepath{stroke}%
\end{pgfscope}%
\begin{pgfscope}%
\pgfpathrectangle{\pgfqpoint{1.250000in}{1.750000in}}{\pgfqpoint{2.279412in}{2.004545in}}%
\pgfusepath{clip}%
\pgfsetbuttcap%
\pgfsetroundjoin%
\pgfsetlinewidth{0.846787pt}%
\definecolor{currentstroke}{rgb}{0.201239,0.383670,0.554294}%
\pgfsetstrokecolor{currentstroke}%
\pgfsetdash{}{0pt}%
\pgfpathmoveto{\pgfqpoint{2.128517in}{2.977625in}}%
\pgfpathlineto{\pgfqpoint{2.128517in}{2.977625in}}%
\pgfusepath{stroke}%
\end{pgfscope}%
\begin{pgfscope}%
\pgfpathrectangle{\pgfqpoint{1.250000in}{1.750000in}}{\pgfqpoint{2.279412in}{2.004545in}}%
\pgfusepath{clip}%
\pgfsetbuttcap%
\pgfsetroundjoin%
\pgfsetlinewidth{0.846787pt}%
\definecolor{currentstroke}{rgb}{0.201239,0.383670,0.554294}%
\pgfsetstrokecolor{currentstroke}%
\pgfsetdash{}{0pt}%
\pgfpathmoveto{\pgfqpoint{2.128517in}{2.977625in}}%
\pgfpathlineto{\pgfqpoint{2.118483in}{2.955174in}}%
\pgfusepath{stroke}%
\end{pgfscope}%
\begin{pgfscope}%
\pgfpathrectangle{\pgfqpoint{1.250000in}{1.750000in}}{\pgfqpoint{2.279412in}{2.004545in}}%
\pgfusepath{clip}%
\pgfsetbuttcap%
\pgfsetroundjoin%
\pgfsetlinewidth{1.381056pt}%
\definecolor{currentstroke}{rgb}{0.146616,0.673050,0.508936}%
\pgfsetstrokecolor{currentstroke}%
\pgfsetdash{}{0pt}%
\pgfpathmoveto{\pgfqpoint{2.118483in}{2.955174in}}%
\pgfpathlineto{\pgfqpoint{2.103924in}{2.934830in}}%
\pgfusepath{stroke}%
\end{pgfscope}%
\begin{pgfscope}%
\pgfpathrectangle{\pgfqpoint{1.250000in}{1.750000in}}{\pgfqpoint{2.279412in}{2.004545in}}%
\pgfusepath{clip}%
\pgfsetbuttcap%
\pgfsetroundjoin%
\pgfsetlinewidth{1.512066pt}%
\definecolor{currentstroke}{rgb}{0.246070,0.738910,0.452024}%
\pgfsetstrokecolor{currentstroke}%
\pgfsetdash{}{0pt}%
\pgfpathmoveto{\pgfqpoint{2.103924in}{2.934830in}}%
\pgfpathlineto{\pgfqpoint{2.074391in}{2.899508in}}%
\pgfusepath{stroke}%
\end{pgfscope}%
\begin{pgfscope}%
\pgfpathrectangle{\pgfqpoint{1.250000in}{1.750000in}}{\pgfqpoint{2.279412in}{2.004545in}}%
\pgfusepath{clip}%
\pgfsetbuttcap%
\pgfsetroundjoin%
\pgfsetlinewidth{2.045854pt}%
\definecolor{currentstroke}{rgb}{0.983868,0.904867,0.136897}%
\pgfsetstrokecolor{currentstroke}%
\pgfsetdash{}{0pt}%
\pgfpathmoveto{\pgfqpoint{2.074391in}{2.899508in}}%
\pgfpathlineto{\pgfqpoint{2.045979in}{2.864745in}}%
\pgfusepath{stroke}%
\end{pgfscope}%
\begin{pgfscope}%
\pgfpathrectangle{\pgfqpoint{1.250000in}{1.750000in}}{\pgfqpoint{2.279412in}{2.004545in}}%
\pgfusepath{clip}%
\pgfsetbuttcap%
\pgfsetroundjoin%
\pgfsetlinewidth{1.355683pt}%
\definecolor{currentstroke}{rgb}{0.134692,0.658636,0.517649}%
\pgfsetstrokecolor{currentstroke}%
\pgfsetdash{}{0pt}%
\pgfpathmoveto{\pgfqpoint{2.045979in}{2.864745in}}%
\pgfpathlineto{\pgfqpoint{2.045979in}{2.864745in}}%
\pgfusepath{stroke}%
\end{pgfscope}%
\begin{pgfscope}%
\pgfpathrectangle{\pgfqpoint{1.250000in}{1.750000in}}{\pgfqpoint{2.279412in}{2.004545in}}%
\pgfusepath{clip}%
\pgfsetbuttcap%
\pgfsetroundjoin%
\pgfsetlinewidth{1.355683pt}%
\definecolor{currentstroke}{rgb}{0.134692,0.658636,0.517649}%
\pgfsetstrokecolor{currentstroke}%
\pgfsetdash{}{0pt}%
\pgfpathmoveto{\pgfqpoint{2.045979in}{2.864745in}}%
\pgfpathlineto{\pgfqpoint{2.045979in}{2.864745in}}%
\pgfusepath{stroke}%
\end{pgfscope}%
\begin{pgfscope}%
\pgfpathrectangle{\pgfqpoint{1.250000in}{1.750000in}}{\pgfqpoint{2.279412in}{2.004545in}}%
\pgfusepath{clip}%
\pgfsetbuttcap%
\pgfsetroundjoin%
\pgfsetlinewidth{1.355683pt}%
\definecolor{currentstroke}{rgb}{0.134692,0.658636,0.517649}%
\pgfsetstrokecolor{currentstroke}%
\pgfsetdash{}{0pt}%
\pgfpathmoveto{\pgfqpoint{2.045979in}{2.864745in}}%
\pgfpathlineto{\pgfqpoint{2.025647in}{2.846319in}}%
\pgfusepath{stroke}%
\end{pgfscope}%
\begin{pgfscope}%
\pgfpathrectangle{\pgfqpoint{1.250000in}{1.750000in}}{\pgfqpoint{2.279412in}{2.004545in}}%
\pgfusepath{clip}%
\pgfsetbuttcap%
\pgfsetroundjoin%
\pgfsetlinewidth{2.117182pt}%
\definecolor{currentstroke}{rgb}{0.993248,0.906157,0.143936}%
\pgfsetstrokecolor{currentstroke}%
\pgfsetdash{}{0pt}%
\pgfpathmoveto{\pgfqpoint{2.025647in}{2.846319in}}%
\pgfpathlineto{\pgfqpoint{2.003578in}{2.830534in}}%
\pgfusepath{stroke}%
\end{pgfscope}%
\begin{pgfscope}%
\pgfpathrectangle{\pgfqpoint{1.250000in}{1.750000in}}{\pgfqpoint{2.279412in}{2.004545in}}%
\pgfusepath{clip}%
\pgfsetbuttcap%
\pgfsetroundjoin%
\pgfsetlinewidth{1.222151pt}%
\definecolor{currentstroke}{rgb}{0.121831,0.589055,0.545623}%
\pgfsetstrokecolor{currentstroke}%
\pgfsetdash{}{0pt}%
\pgfpathmoveto{\pgfqpoint{2.543582in}{2.571846in}}%
\pgfpathlineto{\pgfqpoint{2.494176in}{2.579390in}}%
\pgfusepath{stroke}%
\end{pgfscope}%
\begin{pgfscope}%
\pgfpathrectangle{\pgfqpoint{1.250000in}{1.750000in}}{\pgfqpoint{2.279412in}{2.004545in}}%
\pgfusepath{clip}%
\pgfsetbuttcap%
\pgfsetroundjoin%
\pgfsetlinewidth{1.292544pt}%
\definecolor{currentstroke}{rgb}{0.120638,0.625828,0.533488}%
\pgfsetstrokecolor{currentstroke}%
\pgfsetdash{}{0pt}%
\pgfpathmoveto{\pgfqpoint{2.494176in}{2.579390in}}%
\pgfpathlineto{\pgfqpoint{2.444979in}{2.587928in}}%
\pgfusepath{stroke}%
\end{pgfscope}%
\begin{pgfscope}%
\pgfpathrectangle{\pgfqpoint{1.250000in}{1.750000in}}{\pgfqpoint{2.279412in}{2.004545in}}%
\pgfusepath{clip}%
\pgfsetbuttcap%
\pgfsetroundjoin%
\pgfsetlinewidth{1.298395pt}%
\definecolor{currentstroke}{rgb}{0.121380,0.629492,0.531973}%
\pgfsetstrokecolor{currentstroke}%
\pgfsetdash{}{0pt}%
\pgfpathmoveto{\pgfqpoint{2.444979in}{2.587928in}}%
\pgfpathlineto{\pgfqpoint{2.396011in}{2.597418in}}%
\pgfusepath{stroke}%
\end{pgfscope}%
\begin{pgfscope}%
\pgfpathrectangle{\pgfqpoint{1.250000in}{1.750000in}}{\pgfqpoint{2.279412in}{2.004545in}}%
\pgfusepath{clip}%
\pgfsetbuttcap%
\pgfsetroundjoin%
\pgfsetlinewidth{1.573272pt}%
\definecolor{currentstroke}{rgb}{0.311925,0.767822,0.415586}%
\pgfsetstrokecolor{currentstroke}%
\pgfsetdash{}{0pt}%
\pgfpathmoveto{\pgfqpoint{2.396011in}{2.597418in}}%
\pgfpathlineto{\pgfqpoint{2.347306in}{2.607900in}}%
\pgfusepath{stroke}%
\end{pgfscope}%
\begin{pgfscope}%
\pgfpathrectangle{\pgfqpoint{1.250000in}{1.750000in}}{\pgfqpoint{2.279412in}{2.004545in}}%
\pgfusepath{clip}%
\pgfsetbuttcap%
\pgfsetroundjoin%
\pgfsetlinewidth{1.856345pt}%
\definecolor{currentstroke}{rgb}{0.699415,0.867117,0.175971}%
\pgfsetstrokecolor{currentstroke}%
\pgfsetdash{}{0pt}%
\pgfpathmoveto{\pgfqpoint{2.347306in}{2.607900in}}%
\pgfpathlineto{\pgfqpoint{2.298819in}{2.619137in}}%
\pgfusepath{stroke}%
\end{pgfscope}%
\begin{pgfscope}%
\pgfpathrectangle{\pgfqpoint{1.250000in}{1.750000in}}{\pgfqpoint{2.279412in}{2.004545in}}%
\pgfusepath{clip}%
\pgfsetroundcap%
\pgfsetroundjoin%
\pgfsetlinewidth{1.410375pt}%
\definecolor{currentstroke}{rgb}{0.162016,0.687316,0.499129}%
\pgfsetstrokecolor{currentstroke}%
\pgfsetdash{}{0pt}%
\pgfpathmoveto{\pgfqpoint{2.623520in}{2.825612in}}%
\pgfpathquadraticcurveto{\pgfqpoint{2.611007in}{2.824920in}}{\pgfqpoint{2.620280in}{2.825433in}}%
\pgfusepath{stroke}%
\end{pgfscope}%
\begin{pgfscope}%
\pgfpathrectangle{\pgfqpoint{1.250000in}{1.750000in}}{\pgfqpoint{2.279412in}{2.004545in}}%
\pgfusepath{clip}%
\pgfsetroundcap%
\pgfsetroundjoin%
\definecolor{currentfill}{rgb}{0.162016,0.687316,0.499129}%
\pgfsetfillcolor{currentfill}%
\pgfsetlinewidth{1.410375pt}%
\definecolor{currentstroke}{rgb}{0.162016,0.687316,0.499129}%
\pgfsetstrokecolor{currentstroke}%
\pgfsetdash{}{0pt}%
\pgfpathmoveto{\pgfqpoint{2.677284in}{2.800762in}}%
\pgfpathlineto{\pgfqpoint{2.620280in}{2.825433in}}%
\pgfpathlineto{\pgfqpoint{2.674219in}{2.856233in}}%
\pgfpathlineto{\pgfqpoint{2.677284in}{2.800762in}}%
\pgfpathlineto{\pgfqpoint{2.677284in}{2.800762in}}%
\pgfpathclose%
\pgfusepath{stroke,fill}%
\end{pgfscope}%
\begin{pgfscope}%
\pgfpathrectangle{\pgfqpoint{1.250000in}{1.750000in}}{\pgfqpoint{2.279412in}{2.004545in}}%
\pgfusepath{clip}%
\pgfsetroundcap%
\pgfsetroundjoin%
\pgfsetlinewidth{0.612056pt}%
\definecolor{currentstroke}{rgb}{0.263663,0.237631,0.518762}%
\pgfsetstrokecolor{currentstroke}%
\pgfsetdash{}{0pt}%
\pgfpathmoveto{\pgfqpoint{2.782535in}{2.878836in}}%
\pgfpathquadraticcurveto{\pgfqpoint{2.770008in}{2.878364in}}{\pgfqpoint{2.766944in}{2.878249in}}%
\pgfusepath{stroke}%
\end{pgfscope}%
\begin{pgfscope}%
\pgfpathrectangle{\pgfqpoint{1.250000in}{1.750000in}}{\pgfqpoint{2.279412in}{2.004545in}}%
\pgfusepath{clip}%
\pgfsetroundcap%
\pgfsetroundjoin%
\definecolor{currentfill}{rgb}{0.263663,0.237631,0.518762}%
\pgfsetfillcolor{currentfill}%
\pgfsetlinewidth{0.612056pt}%
\definecolor{currentstroke}{rgb}{0.263663,0.237631,0.518762}%
\pgfsetstrokecolor{currentstroke}%
\pgfsetdash{}{0pt}%
\pgfpathmoveto{\pgfqpoint{2.823507in}{2.852584in}}%
\pgfpathlineto{\pgfqpoint{2.766944in}{2.878249in}}%
\pgfpathlineto{\pgfqpoint{2.821414in}{2.908100in}}%
\pgfpathlineto{\pgfqpoint{2.823507in}{2.852584in}}%
\pgfpathlineto{\pgfqpoint{2.823507in}{2.852584in}}%
\pgfpathclose%
\pgfusepath{stroke,fill}%
\end{pgfscope}%
\begin{pgfscope}%
\pgfpathrectangle{\pgfqpoint{1.250000in}{1.750000in}}{\pgfqpoint{2.279412in}{2.004545in}}%
\pgfusepath{clip}%
\pgfsetroundcap%
\pgfsetroundjoin%
\pgfsetlinewidth{0.886117pt}%
\definecolor{currentstroke}{rgb}{0.190631,0.407061,0.556089}%
\pgfsetstrokecolor{currentstroke}%
\pgfsetdash{}{0pt}%
\pgfpathmoveto{\pgfqpoint{2.560697in}{2.520874in}}%
\pgfpathquadraticcurveto{\pgfqpoint{2.548407in}{2.523042in}}{\pgfqpoint{2.549617in}{2.522829in}}%
\pgfusepath{stroke}%
\end{pgfscope}%
\begin{pgfscope}%
\pgfpathrectangle{\pgfqpoint{1.250000in}{1.750000in}}{\pgfqpoint{2.279412in}{2.004545in}}%
\pgfusepath{clip}%
\pgfsetroundcap%
\pgfsetroundjoin%
\definecolor{currentfill}{rgb}{0.190631,0.407061,0.556089}%
\pgfsetfillcolor{currentfill}%
\pgfsetlinewidth{0.886117pt}%
\definecolor{currentstroke}{rgb}{0.190631,0.407061,0.556089}%
\pgfsetstrokecolor{currentstroke}%
\pgfsetdash{}{0pt}%
\pgfpathmoveto{\pgfqpoint{2.599503in}{2.485823in}}%
\pgfpathlineto{\pgfqpoint{2.549617in}{2.522829in}}%
\pgfpathlineto{\pgfqpoint{2.609152in}{2.540534in}}%
\pgfpathlineto{\pgfqpoint{2.599503in}{2.485823in}}%
\pgfpathlineto{\pgfqpoint{2.599503in}{2.485823in}}%
\pgfpathclose%
\pgfusepath{stroke,fill}%
\end{pgfscope}%
\begin{pgfscope}%
\pgfpathrectangle{\pgfqpoint{1.250000in}{1.750000in}}{\pgfqpoint{2.279412in}{2.004545in}}%
\pgfusepath{clip}%
\pgfsetroundcap%
\pgfsetroundjoin%
\pgfsetlinewidth{0.369778pt}%
\definecolor{currentstroke}{rgb}{0.278791,0.062145,0.386592}%
\pgfsetstrokecolor{currentstroke}%
\pgfsetdash{}{0pt}%
\pgfpathmoveto{\pgfqpoint{2.959943in}{2.533348in}}%
\pgfpathquadraticcurveto{\pgfqpoint{2.947415in}{2.533773in}}{\pgfqpoint{2.940604in}{2.534004in}}%
\pgfusepath{stroke}%
\end{pgfscope}%
\begin{pgfscope}%
\pgfpathrectangle{\pgfqpoint{1.250000in}{1.750000in}}{\pgfqpoint{2.279412in}{2.004545in}}%
\pgfusepath{clip}%
\pgfsetroundcap%
\pgfsetroundjoin%
\definecolor{currentfill}{rgb}{0.278791,0.062145,0.386592}%
\pgfsetfillcolor{currentfill}%
\pgfsetlinewidth{0.369778pt}%
\definecolor{currentstroke}{rgb}{0.278791,0.062145,0.386592}%
\pgfsetstrokecolor{currentstroke}%
\pgfsetdash{}{0pt}%
\pgfpathmoveto{\pgfqpoint{2.995186in}{2.504358in}}%
\pgfpathlineto{\pgfqpoint{2.940604in}{2.534004in}}%
\pgfpathlineto{\pgfqpoint{2.997070in}{2.559882in}}%
\pgfpathlineto{\pgfqpoint{2.995186in}{2.504358in}}%
\pgfpathlineto{\pgfqpoint{2.995186in}{2.504358in}}%
\pgfpathclose%
\pgfusepath{stroke,fill}%
\end{pgfscope}%
\begin{pgfscope}%
\pgfpathrectangle{\pgfqpoint{1.250000in}{1.750000in}}{\pgfqpoint{2.279412in}{2.004545in}}%
\pgfusepath{clip}%
\pgfsetroundcap%
\pgfsetroundjoin%
\pgfsetlinewidth{1.297197pt}%
\definecolor{currentstroke}{rgb}{0.121380,0.629492,0.531973}%
\pgfsetstrokecolor{currentstroke}%
\pgfsetdash{}{0pt}%
\pgfpathmoveto{\pgfqpoint{2.609034in}{2.632793in}}%
\pgfpathquadraticcurveto{\pgfqpoint{2.596542in}{2.633738in}}{\pgfqpoint{2.604062in}{2.633169in}}%
\pgfusepath{stroke}%
\end{pgfscope}%
\begin{pgfscope}%
\pgfpathrectangle{\pgfqpoint{1.250000in}{1.750000in}}{\pgfqpoint{2.279412in}{2.004545in}}%
\pgfusepath{clip}%
\pgfsetroundcap%
\pgfsetroundjoin%
\definecolor{currentfill}{rgb}{0.121380,0.629492,0.531973}%
\pgfsetfillcolor{currentfill}%
\pgfsetlinewidth{1.297197pt}%
\definecolor{currentstroke}{rgb}{0.121380,0.629492,0.531973}%
\pgfsetstrokecolor{currentstroke}%
\pgfsetdash{}{0pt}%
\pgfpathmoveto{\pgfqpoint{2.657363in}{2.601280in}}%
\pgfpathlineto{\pgfqpoint{2.604062in}{2.633169in}}%
\pgfpathlineto{\pgfqpoint{2.661555in}{2.656677in}}%
\pgfpathlineto{\pgfqpoint{2.657363in}{2.601280in}}%
\pgfpathlineto{\pgfqpoint{2.657363in}{2.601280in}}%
\pgfpathclose%
\pgfusepath{stroke,fill}%
\end{pgfscope}%
\begin{pgfscope}%
\pgfpathrectangle{\pgfqpoint{1.250000in}{1.750000in}}{\pgfqpoint{2.279412in}{2.004545in}}%
\pgfusepath{clip}%
\pgfsetroundcap%
\pgfsetroundjoin%
\pgfsetlinewidth{0.422664pt}%
\definecolor{currentstroke}{rgb}{0.282656,0.100196,0.422160}%
\pgfsetstrokecolor{currentstroke}%
\pgfsetdash{}{0pt}%
\pgfpathmoveto{\pgfqpoint{2.513326in}{2.406989in}}%
\pgfpathquadraticcurveto{\pgfqpoint{2.501496in}{2.410584in}}{\pgfqpoint{2.495922in}{2.412278in}}%
\pgfusepath{stroke}%
\end{pgfscope}%
\begin{pgfscope}%
\pgfpathrectangle{\pgfqpoint{1.250000in}{1.750000in}}{\pgfqpoint{2.279412in}{2.004545in}}%
\pgfusepath{clip}%
\pgfsetroundcap%
\pgfsetroundjoin%
\definecolor{currentfill}{rgb}{0.282656,0.100196,0.422160}%
\pgfsetfillcolor{currentfill}%
\pgfsetlinewidth{0.422664pt}%
\definecolor{currentstroke}{rgb}{0.282656,0.100196,0.422160}%
\pgfsetstrokecolor{currentstroke}%
\pgfsetdash{}{0pt}%
\pgfpathmoveto{\pgfqpoint{2.541000in}{2.369547in}}%
\pgfpathlineto{\pgfqpoint{2.495922in}{2.412278in}}%
\pgfpathlineto{\pgfqpoint{2.557154in}{2.422702in}}%
\pgfpathlineto{\pgfqpoint{2.541000in}{2.369547in}}%
\pgfpathlineto{\pgfqpoint{2.541000in}{2.369547in}}%
\pgfpathclose%
\pgfusepath{stroke,fill}%
\end{pgfscope}%
\begin{pgfscope}%
\pgfpathrectangle{\pgfqpoint{1.250000in}{1.750000in}}{\pgfqpoint{2.279412in}{2.004545in}}%
\pgfusepath{clip}%
\pgfsetroundcap%
\pgfsetroundjoin%
\pgfsetlinewidth{0.495057pt}%
\definecolor{currentstroke}{rgb}{0.281412,0.155834,0.469201}%
\pgfsetstrokecolor{currentstroke}%
\pgfsetdash{}{0pt}%
\pgfpathmoveto{\pgfqpoint{2.708517in}{2.456146in}}%
\pgfpathquadraticcurveto{\pgfqpoint{2.696068in}{2.457454in}}{\pgfqpoint{2.691237in}{2.457962in}}%
\pgfusepath{stroke}%
\end{pgfscope}%
\begin{pgfscope}%
\pgfpathrectangle{\pgfqpoint{1.250000in}{1.750000in}}{\pgfqpoint{2.279412in}{2.004545in}}%
\pgfusepath{clip}%
\pgfsetroundcap%
\pgfsetroundjoin%
\definecolor{currentfill}{rgb}{0.281412,0.155834,0.469201}%
\pgfsetfillcolor{currentfill}%
\pgfsetlinewidth{0.495057pt}%
\definecolor{currentstroke}{rgb}{0.281412,0.155834,0.469201}%
\pgfsetstrokecolor{currentstroke}%
\pgfsetdash{}{0pt}%
\pgfpathmoveto{\pgfqpoint{2.743585in}{2.424530in}}%
\pgfpathlineto{\pgfqpoint{2.691237in}{2.457962in}}%
\pgfpathlineto{\pgfqpoint{2.749391in}{2.479781in}}%
\pgfpathlineto{\pgfqpoint{2.743585in}{2.424530in}}%
\pgfpathlineto{\pgfqpoint{2.743585in}{2.424530in}}%
\pgfpathclose%
\pgfusepath{stroke,fill}%
\end{pgfscope}%
\begin{pgfscope}%
\pgfpathrectangle{\pgfqpoint{1.250000in}{1.750000in}}{\pgfqpoint{2.279412in}{2.004545in}}%
\pgfusepath{clip}%
\pgfsetroundcap%
\pgfsetroundjoin%
\pgfsetlinewidth{0.407229pt}%
\definecolor{currentstroke}{rgb}{0.281924,0.089666,0.412415}%
\pgfsetstrokecolor{currentstroke}%
\pgfsetdash{}{0pt}%
\pgfpathmoveto{\pgfqpoint{2.908432in}{2.577626in}}%
\pgfpathquadraticcurveto{\pgfqpoint{2.895902in}{2.578011in}}{\pgfqpoint{2.889669in}{2.578203in}}%
\pgfusepath{stroke}%
\end{pgfscope}%
\begin{pgfscope}%
\pgfpathrectangle{\pgfqpoint{1.250000in}{1.750000in}}{\pgfqpoint{2.279412in}{2.004545in}}%
\pgfusepath{clip}%
\pgfsetroundcap%
\pgfsetroundjoin%
\definecolor{currentfill}{rgb}{0.281924,0.089666,0.412415}%
\pgfsetfillcolor{currentfill}%
\pgfsetlinewidth{0.407229pt}%
\definecolor{currentstroke}{rgb}{0.281924,0.089666,0.412415}%
\pgfsetstrokecolor{currentstroke}%
\pgfsetdash{}{0pt}%
\pgfpathmoveto{\pgfqpoint{2.944346in}{2.548732in}}%
\pgfpathlineto{\pgfqpoint{2.889669in}{2.578203in}}%
\pgfpathlineto{\pgfqpoint{2.946052in}{2.604262in}}%
\pgfpathlineto{\pgfqpoint{2.944346in}{2.548732in}}%
\pgfpathlineto{\pgfqpoint{2.944346in}{2.548732in}}%
\pgfpathclose%
\pgfusepath{stroke,fill}%
\end{pgfscope}%
\begin{pgfscope}%
\pgfpathrectangle{\pgfqpoint{1.250000in}{1.750000in}}{\pgfqpoint{2.279412in}{2.004545in}}%
\pgfusepath{clip}%
\pgfsetroundcap%
\pgfsetroundjoin%
\pgfsetlinewidth{0.509195pt}%
\definecolor{currentstroke}{rgb}{0.280255,0.165693,0.476498}%
\pgfsetstrokecolor{currentstroke}%
\pgfsetdash{}{0pt}%
\pgfpathmoveto{\pgfqpoint{2.858222in}{2.662221in}}%
\pgfpathquadraticcurveto{\pgfqpoint{2.845685in}{2.662315in}}{\pgfqpoint{2.841025in}{2.662349in}}%
\pgfusepath{stroke}%
\end{pgfscope}%
\begin{pgfscope}%
\pgfpathrectangle{\pgfqpoint{1.250000in}{1.750000in}}{\pgfqpoint{2.279412in}{2.004545in}}%
\pgfusepath{clip}%
\pgfsetroundcap%
\pgfsetroundjoin%
\definecolor{currentfill}{rgb}{0.280255,0.165693,0.476498}%
\pgfsetfillcolor{currentfill}%
\pgfsetlinewidth{0.509195pt}%
\definecolor{currentstroke}{rgb}{0.280255,0.165693,0.476498}%
\pgfsetstrokecolor{currentstroke}%
\pgfsetdash{}{0pt}%
\pgfpathmoveto{\pgfqpoint{2.896370in}{2.634156in}}%
\pgfpathlineto{\pgfqpoint{2.841025in}{2.662349in}}%
\pgfpathlineto{\pgfqpoint{2.896787in}{2.689710in}}%
\pgfpathlineto{\pgfqpoint{2.896370in}{2.634156in}}%
\pgfpathlineto{\pgfqpoint{2.896370in}{2.634156in}}%
\pgfpathclose%
\pgfusepath{stroke,fill}%
\end{pgfscope}%
\begin{pgfscope}%
\pgfpathrectangle{\pgfqpoint{1.250000in}{1.750000in}}{\pgfqpoint{2.279412in}{2.004545in}}%
\pgfusepath{clip}%
\pgfsetroundcap%
\pgfsetroundjoin%
\pgfsetlinewidth{1.077231pt}%
\definecolor{currentstroke}{rgb}{0.147607,0.511733,0.557049}%
\pgfsetstrokecolor{currentstroke}%
\pgfsetdash{}{0pt}%
\pgfpathmoveto{\pgfqpoint{2.707792in}{2.711862in}}%
\pgfpathquadraticcurveto{\pgfqpoint{2.695256in}{2.712070in}}{\pgfqpoint{2.699383in}{2.712002in}}%
\pgfusepath{stroke}%
\end{pgfscope}%
\begin{pgfscope}%
\pgfpathrectangle{\pgfqpoint{1.250000in}{1.750000in}}{\pgfqpoint{2.279412in}{2.004545in}}%
\pgfusepath{clip}%
\pgfsetroundcap%
\pgfsetroundjoin%
\definecolor{currentfill}{rgb}{0.147607,0.511733,0.557049}%
\pgfsetfillcolor{currentfill}%
\pgfsetlinewidth{1.077231pt}%
\definecolor{currentstroke}{rgb}{0.147607,0.511733,0.557049}%
\pgfsetstrokecolor{currentstroke}%
\pgfsetdash{}{0pt}%
\pgfpathmoveto{\pgfqpoint{2.754469in}{2.683304in}}%
\pgfpathlineto{\pgfqpoint{2.699383in}{2.712002in}}%
\pgfpathlineto{\pgfqpoint{2.755393in}{2.738852in}}%
\pgfpathlineto{\pgfqpoint{2.754469in}{2.683304in}}%
\pgfpathlineto{\pgfqpoint{2.754469in}{2.683304in}}%
\pgfpathclose%
\pgfusepath{stroke,fill}%
\end{pgfscope}%
\begin{pgfscope}%
\pgfpathrectangle{\pgfqpoint{1.250000in}{1.750000in}}{\pgfqpoint{2.279412in}{2.004545in}}%
\pgfusepath{clip}%
\pgfsetroundcap%
\pgfsetroundjoin%
\pgfsetlinewidth{0.440638pt}%
\definecolor{currentstroke}{rgb}{0.283197,0.115680,0.436115}%
\pgfsetstrokecolor{currentstroke}%
\pgfsetdash{}{0pt}%
\pgfpathmoveto{\pgfqpoint{2.908345in}{2.795939in}}%
\pgfpathquadraticcurveto{\pgfqpoint{2.895808in}{2.795844in}}{\pgfqpoint{2.890087in}{2.795801in}}%
\pgfusepath{stroke}%
\end{pgfscope}%
\begin{pgfscope}%
\pgfpathrectangle{\pgfqpoint{1.250000in}{1.750000in}}{\pgfqpoint{2.279412in}{2.004545in}}%
\pgfusepath{clip}%
\pgfsetroundcap%
\pgfsetroundjoin%
\definecolor{currentfill}{rgb}{0.283197,0.115680,0.436115}%
\pgfsetfillcolor{currentfill}%
\pgfsetlinewidth{0.440638pt}%
\definecolor{currentstroke}{rgb}{0.283197,0.115680,0.436115}%
\pgfsetstrokecolor{currentstroke}%
\pgfsetdash{}{0pt}%
\pgfpathmoveto{\pgfqpoint{2.945850in}{2.768442in}}%
\pgfpathlineto{\pgfqpoint{2.890087in}{2.795801in}}%
\pgfpathlineto{\pgfqpoint{2.945432in}{2.823996in}}%
\pgfpathlineto{\pgfqpoint{2.945850in}{2.768442in}}%
\pgfpathlineto{\pgfqpoint{2.945850in}{2.768442in}}%
\pgfpathclose%
\pgfusepath{stroke,fill}%
\end{pgfscope}%
\begin{pgfscope}%
\pgfpathrectangle{\pgfqpoint{1.250000in}{1.750000in}}{\pgfqpoint{2.279412in}{2.004545in}}%
\pgfusepath{clip}%
\pgfsetroundcap%
\pgfsetroundjoin%
\pgfsetlinewidth{0.594575pt}%
\definecolor{currentstroke}{rgb}{0.267968,0.223549,0.512008}%
\pgfsetstrokecolor{currentstroke}%
\pgfsetdash{}{0pt}%
\pgfpathmoveto{\pgfqpoint{2.758232in}{2.923727in}}%
\pgfpathquadraticcurveto{\pgfqpoint{2.745714in}{2.923109in}}{\pgfqpoint{2.742384in}{2.922944in}}%
\pgfusepath{stroke}%
\end{pgfscope}%
\begin{pgfscope}%
\pgfpathrectangle{\pgfqpoint{1.250000in}{1.750000in}}{\pgfqpoint{2.279412in}{2.004545in}}%
\pgfusepath{clip}%
\pgfsetroundcap%
\pgfsetroundjoin%
\definecolor{currentfill}{rgb}{0.267968,0.223549,0.512008}%
\pgfsetfillcolor{currentfill}%
\pgfsetlinewidth{0.594575pt}%
\definecolor{currentstroke}{rgb}{0.267968,0.223549,0.512008}%
\pgfsetstrokecolor{currentstroke}%
\pgfsetdash{}{0pt}%
\pgfpathmoveto{\pgfqpoint{2.799242in}{2.897942in}}%
\pgfpathlineto{\pgfqpoint{2.742384in}{2.922944in}}%
\pgfpathlineto{\pgfqpoint{2.796501in}{2.953430in}}%
\pgfpathlineto{\pgfqpoint{2.799242in}{2.897942in}}%
\pgfpathlineto{\pgfqpoint{2.799242in}{2.897942in}}%
\pgfpathclose%
\pgfusepath{stroke,fill}%
\end{pgfscope}%
\begin{pgfscope}%
\pgfpathrectangle{\pgfqpoint{1.250000in}{1.750000in}}{\pgfqpoint{2.279412in}{2.004545in}}%
\pgfusepath{clip}%
\pgfsetroundcap%
\pgfsetroundjoin%
\pgfsetlinewidth{0.403051pt}%
\definecolor{currentstroke}{rgb}{0.281446,0.084320,0.407414}%
\pgfsetstrokecolor{currentstroke}%
\pgfsetdash{}{0pt}%
\pgfpathmoveto{\pgfqpoint{2.908474in}{2.972112in}}%
\pgfpathquadraticcurveto{\pgfqpoint{2.895946in}{2.971678in}}{\pgfqpoint{2.889650in}{2.971460in}}%
\pgfusepath{stroke}%
\end{pgfscope}%
\begin{pgfscope}%
\pgfpathrectangle{\pgfqpoint{1.250000in}{1.750000in}}{\pgfqpoint{2.279412in}{2.004545in}}%
\pgfusepath{clip}%
\pgfsetroundcap%
\pgfsetroundjoin%
\definecolor{currentfill}{rgb}{0.281446,0.084320,0.407414}%
\pgfsetfillcolor{currentfill}%
\pgfsetlinewidth{0.403051pt}%
\definecolor{currentstroke}{rgb}{0.281446,0.084320,0.407414}%
\pgfsetstrokecolor{currentstroke}%
\pgfsetdash{}{0pt}%
\pgfpathmoveto{\pgfqpoint{2.946133in}{2.945621in}}%
\pgfpathlineto{\pgfqpoint{2.889650in}{2.971460in}}%
\pgfpathlineto{\pgfqpoint{2.944211in}{3.001143in}}%
\pgfpathlineto{\pgfqpoint{2.946133in}{2.945621in}}%
\pgfpathlineto{\pgfqpoint{2.946133in}{2.945621in}}%
\pgfpathclose%
\pgfusepath{stroke,fill}%
\end{pgfscope}%
\begin{pgfscope}%
\pgfpathrectangle{\pgfqpoint{1.250000in}{1.750000in}}{\pgfqpoint{2.279412in}{2.004545in}}%
\pgfusepath{clip}%
\pgfsetroundcap%
\pgfsetroundjoin%
\pgfsetlinewidth{0.326631pt}%
\definecolor{currentstroke}{rgb}{0.271305,0.019942,0.347269}%
\pgfsetstrokecolor{currentstroke}%
\pgfsetdash{}{0pt}%
\pgfpathmoveto{\pgfqpoint{3.007662in}{2.304856in}}%
\pgfpathquadraticcurveto{\pgfqpoint{2.995135in}{2.305258in}}{\pgfqpoint{2.987659in}{2.305498in}}%
\pgfusepath{stroke}%
\end{pgfscope}%
\begin{pgfscope}%
\pgfpathrectangle{\pgfqpoint{1.250000in}{1.750000in}}{\pgfqpoint{2.279412in}{2.004545in}}%
\pgfusepath{clip}%
\pgfsetroundcap%
\pgfsetroundjoin%
\definecolor{currentfill}{rgb}{0.271305,0.019942,0.347269}%
\pgfsetfillcolor{currentfill}%
\pgfsetlinewidth{0.326631pt}%
\definecolor{currentstroke}{rgb}{0.271305,0.019942,0.347269}%
\pgfsetstrokecolor{currentstroke}%
\pgfsetdash{}{0pt}%
\pgfpathmoveto{\pgfqpoint{3.042295in}{2.275953in}}%
\pgfpathlineto{\pgfqpoint{2.987659in}{2.305498in}}%
\pgfpathlineto{\pgfqpoint{3.044077in}{2.331480in}}%
\pgfpathlineto{\pgfqpoint{3.042295in}{2.275953in}}%
\pgfpathlineto{\pgfqpoint{3.042295in}{2.275953in}}%
\pgfpathclose%
\pgfusepath{stroke,fill}%
\end{pgfscope}%
\begin{pgfscope}%
\pgfpathrectangle{\pgfqpoint{1.250000in}{1.750000in}}{\pgfqpoint{2.279412in}{2.004545in}}%
\pgfusepath{clip}%
\pgfsetroundcap%
\pgfsetroundjoin%
\pgfsetlinewidth{0.350454pt}%
\definecolor{currentstroke}{rgb}{0.276022,0.044167,0.370164}%
\pgfsetstrokecolor{currentstroke}%
\pgfsetdash{}{0pt}%
\pgfpathmoveto{\pgfqpoint{2.957524in}{2.396522in}}%
\pgfpathquadraticcurveto{\pgfqpoint{2.944993in}{2.396710in}}{\pgfqpoint{2.937883in}{2.396818in}}%
\pgfusepath{stroke}%
\end{pgfscope}%
\begin{pgfscope}%
\pgfpathrectangle{\pgfqpoint{1.250000in}{1.750000in}}{\pgfqpoint{2.279412in}{2.004545in}}%
\pgfusepath{clip}%
\pgfsetroundcap%
\pgfsetroundjoin%
\definecolor{currentfill}{rgb}{0.276022,0.044167,0.370164}%
\pgfsetfillcolor{currentfill}%
\pgfsetlinewidth{0.350454pt}%
\definecolor{currentstroke}{rgb}{0.276022,0.044167,0.370164}%
\pgfsetstrokecolor{currentstroke}%
\pgfsetdash{}{0pt}%
\pgfpathmoveto{\pgfqpoint{2.993014in}{2.368206in}}%
\pgfpathlineto{\pgfqpoint{2.937883in}{2.396818in}}%
\pgfpathlineto{\pgfqpoint{2.993851in}{2.423755in}}%
\pgfpathlineto{\pgfqpoint{2.993014in}{2.368206in}}%
\pgfpathlineto{\pgfqpoint{2.993014in}{2.368206in}}%
\pgfpathclose%
\pgfusepath{stroke,fill}%
\end{pgfscope}%
\begin{pgfscope}%
\pgfpathrectangle{\pgfqpoint{1.250000in}{1.750000in}}{\pgfqpoint{2.279412in}{2.004545in}}%
\pgfusepath{clip}%
\pgfsetroundcap%
\pgfsetroundjoin%
\pgfsetlinewidth{1.421565pt}%
\definecolor{currentstroke}{rgb}{0.170948,0.694384,0.493803}%
\pgfsetstrokecolor{currentstroke}%
\pgfsetdash{}{0pt}%
\pgfpathmoveto{\pgfqpoint{2.656456in}{2.751222in}}%
\pgfpathquadraticcurveto{\pgfqpoint{2.643918in}{2.751185in}}{\pgfqpoint{2.653372in}{2.751213in}}%
\pgfusepath{stroke}%
\end{pgfscope}%
\begin{pgfscope}%
\pgfpathrectangle{\pgfqpoint{1.250000in}{1.750000in}}{\pgfqpoint{2.279412in}{2.004545in}}%
\pgfusepath{clip}%
\pgfsetroundcap%
\pgfsetroundjoin%
\definecolor{currentfill}{rgb}{0.170948,0.694384,0.493803}%
\pgfsetfillcolor{currentfill}%
\pgfsetlinewidth{1.421565pt}%
\definecolor{currentstroke}{rgb}{0.170948,0.694384,0.493803}%
\pgfsetstrokecolor{currentstroke}%
\pgfsetdash{}{0pt}%
\pgfpathmoveto{\pgfqpoint{2.709008in}{2.723597in}}%
\pgfpathlineto{\pgfqpoint{2.653372in}{2.751213in}}%
\pgfpathlineto{\pgfqpoint{2.708847in}{2.779152in}}%
\pgfpathlineto{\pgfqpoint{2.709008in}{2.723597in}}%
\pgfpathlineto{\pgfqpoint{2.709008in}{2.723597in}}%
\pgfpathclose%
\pgfusepath{stroke,fill}%
\end{pgfscope}%
\begin{pgfscope}%
\pgfpathrectangle{\pgfqpoint{1.250000in}{1.750000in}}{\pgfqpoint{2.279412in}{2.004545in}}%
\pgfusepath{clip}%
\pgfsetroundcap%
\pgfsetroundjoin%
\pgfsetlinewidth{1.065430pt}%
\definecolor{currentstroke}{rgb}{0.150476,0.504369,0.557430}%
\pgfsetstrokecolor{currentstroke}%
\pgfsetdash{}{0pt}%
\pgfpathmoveto{\pgfqpoint{2.460439in}{2.970049in}}%
\pgfpathquadraticcurveto{\pgfqpoint{2.448403in}{2.966967in}}{\pgfqpoint{2.452334in}{2.967974in}}%
\pgfusepath{stroke}%
\end{pgfscope}%
\begin{pgfscope}%
\pgfpathrectangle{\pgfqpoint{1.250000in}{1.750000in}}{\pgfqpoint{2.279412in}{2.004545in}}%
\pgfusepath{clip}%
\pgfsetroundcap%
\pgfsetroundjoin%
\definecolor{currentfill}{rgb}{0.150476,0.504369,0.557430}%
\pgfsetfillcolor{currentfill}%
\pgfsetlinewidth{1.065430pt}%
\definecolor{currentstroke}{rgb}{0.150476,0.504369,0.557430}%
\pgfsetstrokecolor{currentstroke}%
\pgfsetdash{}{0pt}%
\pgfpathmoveto{\pgfqpoint{2.513043in}{2.954843in}}%
\pgfpathlineto{\pgfqpoint{2.452334in}{2.967974in}}%
\pgfpathlineto{\pgfqpoint{2.499264in}{3.008663in}}%
\pgfpathlineto{\pgfqpoint{2.513043in}{2.954843in}}%
\pgfpathlineto{\pgfqpoint{2.513043in}{2.954843in}}%
\pgfpathclose%
\pgfusepath{stroke,fill}%
\end{pgfscope}%
\begin{pgfscope}%
\pgfpathrectangle{\pgfqpoint{1.250000in}{1.750000in}}{\pgfqpoint{2.279412in}{2.004545in}}%
\pgfusepath{clip}%
\pgfsetroundcap%
\pgfsetroundjoin%
\pgfsetlinewidth{0.385099pt}%
\definecolor{currentstroke}{rgb}{0.280267,0.073417,0.397163}%
\pgfsetstrokecolor{currentstroke}%
\pgfsetdash{}{0pt}%
\pgfpathmoveto{\pgfqpoint{2.907706in}{3.059675in}}%
\pgfpathquadraticcurveto{\pgfqpoint{2.895193in}{3.058984in}}{\pgfqpoint{2.888629in}{3.058621in}}%
\pgfusepath{stroke}%
\end{pgfscope}%
\begin{pgfscope}%
\pgfpathrectangle{\pgfqpoint{1.250000in}{1.750000in}}{\pgfqpoint{2.279412in}{2.004545in}}%
\pgfusepath{clip}%
\pgfsetroundcap%
\pgfsetroundjoin%
\definecolor{currentfill}{rgb}{0.280267,0.073417,0.397163}%
\pgfsetfillcolor{currentfill}%
\pgfsetlinewidth{0.385099pt}%
\definecolor{currentstroke}{rgb}{0.280267,0.073417,0.397163}%
\pgfsetstrokecolor{currentstroke}%
\pgfsetdash{}{0pt}%
\pgfpathmoveto{\pgfqpoint{2.945632in}{3.033950in}}%
\pgfpathlineto{\pgfqpoint{2.888629in}{3.058621in}}%
\pgfpathlineto{\pgfqpoint{2.942568in}{3.089421in}}%
\pgfpathlineto{\pgfqpoint{2.945632in}{3.033950in}}%
\pgfpathlineto{\pgfqpoint{2.945632in}{3.033950in}}%
\pgfpathclose%
\pgfusepath{stroke,fill}%
\end{pgfscope}%
\begin{pgfscope}%
\pgfpathrectangle{\pgfqpoint{1.250000in}{1.750000in}}{\pgfqpoint{2.279412in}{2.004545in}}%
\pgfusepath{clip}%
\pgfsetroundcap%
\pgfsetroundjoin%
\pgfsetlinewidth{0.352555pt}%
\definecolor{currentstroke}{rgb}{0.276022,0.044167,0.370164}%
\pgfsetstrokecolor{currentstroke}%
\pgfsetdash{}{0pt}%
\pgfpathmoveto{\pgfqpoint{2.957493in}{3.108668in}}%
\pgfpathquadraticcurveto{\pgfqpoint{2.944983in}{3.108125in}}{\pgfqpoint{2.937921in}{3.107818in}}%
\pgfusepath{stroke}%
\end{pgfscope}%
\begin{pgfscope}%
\pgfpathrectangle{\pgfqpoint{1.250000in}{1.750000in}}{\pgfqpoint{2.279412in}{2.004545in}}%
\pgfusepath{clip}%
\pgfsetroundcap%
\pgfsetroundjoin%
\definecolor{currentfill}{rgb}{0.276022,0.044167,0.370164}%
\pgfsetfillcolor{currentfill}%
\pgfsetlinewidth{0.352555pt}%
\definecolor{currentstroke}{rgb}{0.276022,0.044167,0.370164}%
\pgfsetstrokecolor{currentstroke}%
\pgfsetdash{}{0pt}%
\pgfpathmoveto{\pgfqpoint{2.994630in}{3.082477in}}%
\pgfpathlineto{\pgfqpoint{2.937921in}{3.107818in}}%
\pgfpathlineto{\pgfqpoint{2.992220in}{3.137980in}}%
\pgfpathlineto{\pgfqpoint{2.994630in}{3.082477in}}%
\pgfpathlineto{\pgfqpoint{2.994630in}{3.082477in}}%
\pgfpathclose%
\pgfusepath{stroke,fill}%
\end{pgfscope}%
\begin{pgfscope}%
\pgfpathrectangle{\pgfqpoint{1.250000in}{1.750000in}}{\pgfqpoint{2.279412in}{2.004545in}}%
\pgfusepath{clip}%
\pgfsetroundcap%
\pgfsetroundjoin%
\pgfsetlinewidth{0.432016pt}%
\definecolor{currentstroke}{rgb}{0.283091,0.110553,0.431554}%
\pgfsetstrokecolor{currentstroke}%
\pgfsetdash{}{0pt}%
\pgfpathmoveto{\pgfqpoint{2.512089in}{3.099967in}}%
\pgfpathquadraticcurveto{\pgfqpoint{2.500372in}{3.096086in}}{\pgfqpoint{2.494999in}{3.094306in}}%
\pgfusepath{stroke}%
\end{pgfscope}%
\begin{pgfscope}%
\pgfpathrectangle{\pgfqpoint{1.250000in}{1.750000in}}{\pgfqpoint{2.279412in}{2.004545in}}%
\pgfusepath{clip}%
\pgfsetroundcap%
\pgfsetroundjoin%
\definecolor{currentfill}{rgb}{0.283091,0.110553,0.431554}%
\pgfsetfillcolor{currentfill}%
\pgfsetlinewidth{0.432016pt}%
\definecolor{currentstroke}{rgb}{0.283091,0.110553,0.431554}%
\pgfsetstrokecolor{currentstroke}%
\pgfsetdash{}{0pt}%
\pgfpathmoveto{\pgfqpoint{2.556471in}{3.085405in}}%
\pgfpathlineto{\pgfqpoint{2.494999in}{3.094306in}}%
\pgfpathlineto{\pgfqpoint{2.539003in}{3.138143in}}%
\pgfpathlineto{\pgfqpoint{2.556471in}{3.085405in}}%
\pgfpathlineto{\pgfqpoint{2.556471in}{3.085405in}}%
\pgfpathclose%
\pgfusepath{stroke,fill}%
\end{pgfscope}%
\begin{pgfscope}%
\pgfpathrectangle{\pgfqpoint{1.250000in}{1.750000in}}{\pgfqpoint{2.279412in}{2.004545in}}%
\pgfusepath{clip}%
\pgfsetroundcap%
\pgfsetroundjoin%
\pgfsetlinewidth{0.349828pt}%
\definecolor{currentstroke}{rgb}{0.276022,0.044167,0.370164}%
\pgfsetstrokecolor{currentstroke}%
\pgfsetdash{}{0pt}%
\pgfpathmoveto{\pgfqpoint{2.856667in}{2.266965in}}%
\pgfpathquadraticcurveto{\pgfqpoint{2.844214in}{2.268206in}}{\pgfqpoint{2.837147in}{2.268910in}}%
\pgfusepath{stroke}%
\end{pgfscope}%
\begin{pgfscope}%
\pgfpathrectangle{\pgfqpoint{1.250000in}{1.750000in}}{\pgfqpoint{2.279412in}{2.004545in}}%
\pgfusepath{clip}%
\pgfsetroundcap%
\pgfsetroundjoin%
\definecolor{currentfill}{rgb}{0.276022,0.044167,0.370164}%
\pgfsetfillcolor{currentfill}%
\pgfsetlinewidth{0.349828pt}%
\definecolor{currentstroke}{rgb}{0.276022,0.044167,0.370164}%
\pgfsetstrokecolor{currentstroke}%
\pgfsetdash{}{0pt}%
\pgfpathmoveto{\pgfqpoint{2.889673in}{2.235759in}}%
\pgfpathlineto{\pgfqpoint{2.837147in}{2.268910in}}%
\pgfpathlineto{\pgfqpoint{2.895184in}{2.291041in}}%
\pgfpathlineto{\pgfqpoint{2.889673in}{2.235759in}}%
\pgfpathlineto{\pgfqpoint{2.889673in}{2.235759in}}%
\pgfpathclose%
\pgfusepath{stroke,fill}%
\end{pgfscope}%
\begin{pgfscope}%
\pgfpathrectangle{\pgfqpoint{1.250000in}{1.750000in}}{\pgfqpoint{2.279412in}{2.004545in}}%
\pgfusepath{clip}%
\pgfsetroundcap%
\pgfsetroundjoin%
\pgfsetlinewidth{0.344738pt}%
\definecolor{currentstroke}{rgb}{0.274952,0.037752,0.364543}%
\pgfsetstrokecolor{currentstroke}%
\pgfsetdash{}{0pt}%
\pgfpathmoveto{\pgfqpoint{2.906342in}{3.196825in}}%
\pgfpathquadraticcurveto{\pgfqpoint{2.893841in}{3.196027in}}{\pgfqpoint{2.886662in}{3.195568in}}%
\pgfusepath{stroke}%
\end{pgfscope}%
\begin{pgfscope}%
\pgfpathrectangle{\pgfqpoint{1.250000in}{1.750000in}}{\pgfqpoint{2.279412in}{2.004545in}}%
\pgfusepath{clip}%
\pgfsetroundcap%
\pgfsetroundjoin%
\definecolor{currentfill}{rgb}{0.274952,0.037752,0.364543}%
\pgfsetfillcolor{currentfill}%
\pgfsetlinewidth{0.344738pt}%
\definecolor{currentstroke}{rgb}{0.274952,0.037752,0.364543}%
\pgfsetstrokecolor{currentstroke}%
\pgfsetdash{}{0pt}%
\pgfpathmoveto{\pgfqpoint{2.943875in}{3.171387in}}%
\pgfpathlineto{\pgfqpoint{2.886662in}{3.195568in}}%
\pgfpathlineto{\pgfqpoint{2.940335in}{3.226830in}}%
\pgfpathlineto{\pgfqpoint{2.943875in}{3.171387in}}%
\pgfpathlineto{\pgfqpoint{2.943875in}{3.171387in}}%
\pgfpathclose%
\pgfusepath{stroke,fill}%
\end{pgfscope}%
\begin{pgfscope}%
\pgfpathrectangle{\pgfqpoint{1.250000in}{1.750000in}}{\pgfqpoint{2.279412in}{2.004545in}}%
\pgfusepath{clip}%
\pgfsetroundcap%
\pgfsetroundjoin%
\pgfsetlinewidth{0.340760pt}%
\definecolor{currentstroke}{rgb}{0.273809,0.031497,0.358853}%
\pgfsetstrokecolor{currentstroke}%
\pgfsetdash{}{0pt}%
\pgfpathmoveto{\pgfqpoint{2.956917in}{3.240122in}}%
\pgfpathquadraticcurveto{\pgfqpoint{2.944432in}{3.239215in}}{\pgfqpoint{2.937205in}{3.238689in}}%
\pgfusepath{stroke}%
\end{pgfscope}%
\begin{pgfscope}%
\pgfpathrectangle{\pgfqpoint{1.250000in}{1.750000in}}{\pgfqpoint{2.279412in}{2.004545in}}%
\pgfusepath{clip}%
\pgfsetroundcap%
\pgfsetroundjoin%
\definecolor{currentfill}{rgb}{0.273809,0.031497,0.358853}%
\pgfsetfillcolor{currentfill}%
\pgfsetlinewidth{0.340760pt}%
\definecolor{currentstroke}{rgb}{0.273809,0.031497,0.358853}%
\pgfsetstrokecolor{currentstroke}%
\pgfsetdash{}{0pt}%
\pgfpathmoveto{\pgfqpoint{2.994629in}{3.215012in}}%
\pgfpathlineto{\pgfqpoint{2.937205in}{3.238689in}}%
\pgfpathlineto{\pgfqpoint{2.990601in}{3.270422in}}%
\pgfpathlineto{\pgfqpoint{2.994629in}{3.215012in}}%
\pgfpathlineto{\pgfqpoint{2.994629in}{3.215012in}}%
\pgfpathclose%
\pgfusepath{stroke,fill}%
\end{pgfscope}%
\begin{pgfscope}%
\pgfpathrectangle{\pgfqpoint{1.250000in}{1.750000in}}{\pgfqpoint{2.279412in}{2.004545in}}%
\pgfusepath{clip}%
\pgfsetroundcap%
\pgfsetroundjoin%
\pgfsetlinewidth{0.326744pt}%
\definecolor{currentstroke}{rgb}{0.271305,0.019942,0.347269}%
\pgfsetstrokecolor{currentstroke}%
\pgfsetdash{}{0pt}%
\pgfpathmoveto{\pgfqpoint{2.907659in}{3.281367in}}%
\pgfpathquadraticcurveto{\pgfqpoint{2.895198in}{3.280174in}}{\pgfqpoint{2.887769in}{3.279463in}}%
\pgfusepath{stroke}%
\end{pgfscope}%
\begin{pgfscope}%
\pgfpathrectangle{\pgfqpoint{1.250000in}{1.750000in}}{\pgfqpoint{2.279412in}{2.004545in}}%
\pgfusepath{clip}%
\pgfsetroundcap%
\pgfsetroundjoin%
\definecolor{currentfill}{rgb}{0.271305,0.019942,0.347269}%
\pgfsetfillcolor{currentfill}%
\pgfsetlinewidth{0.326744pt}%
\definecolor{currentstroke}{rgb}{0.271305,0.019942,0.347269}%
\pgfsetstrokecolor{currentstroke}%
\pgfsetdash{}{0pt}%
\pgfpathmoveto{\pgfqpoint{2.945719in}{3.257106in}}%
\pgfpathlineto{\pgfqpoint{2.887769in}{3.279463in}}%
\pgfpathlineto{\pgfqpoint{2.940424in}{3.312409in}}%
\pgfpathlineto{\pgfqpoint{2.945719in}{3.257106in}}%
\pgfpathlineto{\pgfqpoint{2.945719in}{3.257106in}}%
\pgfpathclose%
\pgfusepath{stroke,fill}%
\end{pgfscope}%
\begin{pgfscope}%
\pgfpathrectangle{\pgfqpoint{1.250000in}{1.750000in}}{\pgfqpoint{2.279412in}{2.004545in}}%
\pgfusepath{clip}%
\pgfsetroundcap%
\pgfsetroundjoin%
\pgfsetlinewidth{0.356941pt}%
\definecolor{currentstroke}{rgb}{0.277018,0.050344,0.375715}%
\pgfsetstrokecolor{currentstroke}%
\pgfsetdash{}{0pt}%
\pgfpathmoveto{\pgfqpoint{2.796203in}{2.195951in}}%
\pgfpathquadraticcurveto{\pgfqpoint{2.783867in}{2.197868in}}{\pgfqpoint{2.776987in}{2.198937in}}%
\pgfusepath{stroke}%
\end{pgfscope}%
\begin{pgfscope}%
\pgfpathrectangle{\pgfqpoint{1.250000in}{1.750000in}}{\pgfqpoint{2.279412in}{2.004545in}}%
\pgfusepath{clip}%
\pgfsetroundcap%
\pgfsetroundjoin%
\definecolor{currentfill}{rgb}{0.277018,0.050344,0.375715}%
\pgfsetfillcolor{currentfill}%
\pgfsetlinewidth{0.356941pt}%
\definecolor{currentstroke}{rgb}{0.277018,0.050344,0.375715}%
\pgfsetstrokecolor{currentstroke}%
\pgfsetdash{}{0pt}%
\pgfpathmoveto{\pgfqpoint{2.827618in}{2.162957in}}%
\pgfpathlineto{\pgfqpoint{2.776987in}{2.198937in}}%
\pgfpathlineto{\pgfqpoint{2.836150in}{2.217854in}}%
\pgfpathlineto{\pgfqpoint{2.827618in}{2.162957in}}%
\pgfpathlineto{\pgfqpoint{2.827618in}{2.162957in}}%
\pgfpathclose%
\pgfusepath{stroke,fill}%
\end{pgfscope}%
\begin{pgfscope}%
\pgfpathrectangle{\pgfqpoint{1.250000in}{1.750000in}}{\pgfqpoint{2.279412in}{2.004545in}}%
\pgfusepath{clip}%
\pgfsetroundcap%
\pgfsetroundjoin%
\pgfsetlinewidth{0.363590pt}%
\definecolor{currentstroke}{rgb}{0.277941,0.056324,0.381191}%
\pgfsetstrokecolor{currentstroke}%
\pgfsetdash{}{0pt}%
\pgfpathmoveto{\pgfqpoint{2.594873in}{2.256098in}}%
\pgfpathquadraticcurveto{\pgfqpoint{2.582751in}{2.258879in}}{\pgfqpoint{2.576111in}{2.260403in}}%
\pgfusepath{stroke}%
\end{pgfscope}%
\begin{pgfscope}%
\pgfpathrectangle{\pgfqpoint{1.250000in}{1.750000in}}{\pgfqpoint{2.279412in}{2.004545in}}%
\pgfusepath{clip}%
\pgfsetroundcap%
\pgfsetroundjoin%
\definecolor{currentfill}{rgb}{0.277941,0.056324,0.381191}%
\pgfsetfillcolor{currentfill}%
\pgfsetlinewidth{0.363590pt}%
\definecolor{currentstroke}{rgb}{0.277941,0.056324,0.381191}%
\pgfsetstrokecolor{currentstroke}%
\pgfsetdash{}{0pt}%
\pgfpathmoveto{\pgfqpoint{2.624049in}{2.220905in}}%
\pgfpathlineto{\pgfqpoint{2.576111in}{2.260403in}}%
\pgfpathlineto{\pgfqpoint{2.636472in}{2.275054in}}%
\pgfpathlineto{\pgfqpoint{2.624049in}{2.220905in}}%
\pgfpathlineto{\pgfqpoint{2.624049in}{2.220905in}}%
\pgfpathclose%
\pgfusepath{stroke,fill}%
\end{pgfscope}%
\begin{pgfscope}%
\pgfpathrectangle{\pgfqpoint{1.250000in}{1.750000in}}{\pgfqpoint{2.279412in}{2.004545in}}%
\pgfusepath{clip}%
\pgfsetroundcap%
\pgfsetroundjoin%
\pgfsetlinewidth{0.370805pt}%
\definecolor{currentstroke}{rgb}{0.278791,0.062145,0.386592}%
\pgfsetstrokecolor{currentstroke}%
\pgfsetdash{}{0pt}%
\pgfpathmoveto{\pgfqpoint{2.591978in}{3.214766in}}%
\pgfpathquadraticcurveto{\pgfqpoint{2.579879in}{3.211910in}}{\pgfqpoint{2.573363in}{3.210371in}}%
\pgfusepath{stroke}%
\end{pgfscope}%
\begin{pgfscope}%
\pgfpathrectangle{\pgfqpoint{1.250000in}{1.750000in}}{\pgfqpoint{2.279412in}{2.004545in}}%
\pgfusepath{clip}%
\pgfsetroundcap%
\pgfsetroundjoin%
\definecolor{currentfill}{rgb}{0.278791,0.062145,0.386592}%
\pgfsetfillcolor{currentfill}%
\pgfsetlinewidth{0.370805pt}%
\definecolor{currentstroke}{rgb}{0.278791,0.062145,0.386592}%
\pgfsetstrokecolor{currentstroke}%
\pgfsetdash{}{0pt}%
\pgfpathmoveto{\pgfqpoint{2.633814in}{3.196102in}}%
\pgfpathlineto{\pgfqpoint{2.573363in}{3.210371in}}%
\pgfpathlineto{\pgfqpoint{2.621049in}{3.250171in}}%
\pgfpathlineto{\pgfqpoint{2.633814in}{3.196102in}}%
\pgfpathlineto{\pgfqpoint{2.633814in}{3.196102in}}%
\pgfpathclose%
\pgfusepath{stroke,fill}%
\end{pgfscope}%
\begin{pgfscope}%
\pgfpathrectangle{\pgfqpoint{1.250000in}{1.750000in}}{\pgfqpoint{2.279412in}{2.004545in}}%
\pgfusepath{clip}%
\pgfsetroundcap%
\pgfsetroundjoin%
\pgfsetlinewidth{0.698171pt}%
\definecolor{currentstroke}{rgb}{0.243113,0.292092,0.538516}%
\pgfsetstrokecolor{currentstroke}%
\pgfsetdash{}{0pt}%
\pgfpathmoveto{\pgfqpoint{2.221269in}{2.501016in}}%
\pgfpathquadraticcurveto{\pgfqpoint{2.213217in}{2.508356in}}{\pgfqpoint{2.213147in}{2.508420in}}%
\pgfusepath{stroke}%
\end{pgfscope}%
\begin{pgfscope}%
\pgfpathrectangle{\pgfqpoint{1.250000in}{1.750000in}}{\pgfqpoint{2.279412in}{2.004545in}}%
\pgfusepath{clip}%
\pgfsetroundcap%
\pgfsetroundjoin%
\definecolor{currentfill}{rgb}{0.243113,0.292092,0.538516}%
\pgfsetfillcolor{currentfill}%
\pgfsetlinewidth{0.698171pt}%
\definecolor{currentstroke}{rgb}{0.243113,0.292092,0.538516}%
\pgfsetstrokecolor{currentstroke}%
\pgfsetdash{}{0pt}%
\pgfpathmoveto{\pgfqpoint{2.235492in}{2.450465in}}%
\pgfpathlineto{\pgfqpoint{2.213147in}{2.508420in}}%
\pgfpathlineto{\pgfqpoint{2.272917in}{2.491523in}}%
\pgfpathlineto{\pgfqpoint{2.235492in}{2.450465in}}%
\pgfpathlineto{\pgfqpoint{2.235492in}{2.450465in}}%
\pgfpathclose%
\pgfusepath{stroke,fill}%
\end{pgfscope}%
\begin{pgfscope}%
\pgfpathrectangle{\pgfqpoint{1.250000in}{1.750000in}}{\pgfqpoint{2.279412in}{2.004545in}}%
\pgfusepath{clip}%
\pgfsetroundcap%
\pgfsetroundjoin%
\pgfsetlinewidth{0.502956pt}%
\definecolor{currentstroke}{rgb}{0.280868,0.160771,0.472899}%
\pgfsetstrokecolor{currentstroke}%
\pgfsetdash{}{0pt}%
\pgfpathmoveto{\pgfqpoint{2.592182in}{3.078845in}}%
\pgfpathquadraticcurveto{\pgfqpoint{2.580032in}{3.076139in}}{\pgfqpoint{2.575477in}{3.075124in}}%
\pgfusepath{stroke}%
\end{pgfscope}%
\begin{pgfscope}%
\pgfpathrectangle{\pgfqpoint{1.250000in}{1.750000in}}{\pgfqpoint{2.279412in}{2.004545in}}%
\pgfusepath{clip}%
\pgfsetroundcap%
\pgfsetroundjoin%
\definecolor{currentfill}{rgb}{0.280868,0.160771,0.472899}%
\pgfsetfillcolor{currentfill}%
\pgfsetlinewidth{0.502956pt}%
\definecolor{currentstroke}{rgb}{0.280868,0.160771,0.472899}%
\pgfsetstrokecolor{currentstroke}%
\pgfsetdash{}{0pt}%
\pgfpathmoveto{\pgfqpoint{2.635742in}{3.060089in}}%
\pgfpathlineto{\pgfqpoint{2.575477in}{3.075124in}}%
\pgfpathlineto{\pgfqpoint{2.623664in}{3.114316in}}%
\pgfpathlineto{\pgfqpoint{2.635742in}{3.060089in}}%
\pgfpathlineto{\pgfqpoint{2.635742in}{3.060089in}}%
\pgfpathclose%
\pgfusepath{stroke,fill}%
\end{pgfscope}%
\begin{pgfscope}%
\pgfpathrectangle{\pgfqpoint{1.250000in}{1.750000in}}{\pgfqpoint{2.279412in}{2.004545in}}%
\pgfusepath{clip}%
\pgfsetroundcap%
\pgfsetroundjoin%
\pgfsetlinewidth{1.381056pt}%
\definecolor{currentstroke}{rgb}{0.146616,0.673050,0.508936}%
\pgfsetstrokecolor{currentstroke}%
\pgfsetdash{}{0pt}%
\pgfpathmoveto{\pgfqpoint{2.118483in}{2.955174in}}%
\pgfpathquadraticcurveto{\pgfqpoint{2.114843in}{2.950088in}}{\pgfqpoint{2.123638in}{2.962376in}}%
\pgfusepath{stroke}%
\end{pgfscope}%
\begin{pgfscope}%
\pgfpathrectangle{\pgfqpoint{1.250000in}{1.750000in}}{\pgfqpoint{2.279412in}{2.004545in}}%
\pgfusepath{clip}%
\pgfsetroundcap%
\pgfsetroundjoin%
\definecolor{currentfill}{rgb}{0.146616,0.673050,0.508936}%
\pgfsetfillcolor{currentfill}%
\pgfsetlinewidth{1.381056pt}%
\definecolor{currentstroke}{rgb}{0.146616,0.673050,0.508936}%
\pgfsetstrokecolor{currentstroke}%
\pgfsetdash{}{0pt}%
\pgfpathmoveto{\pgfqpoint{2.178559in}{2.991388in}}%
\pgfpathlineto{\pgfqpoint{2.123638in}{2.962376in}}%
\pgfpathlineto{\pgfqpoint{2.133381in}{3.023720in}}%
\pgfpathlineto{\pgfqpoint{2.178559in}{2.991388in}}%
\pgfpathlineto{\pgfqpoint{2.178559in}{2.991388in}}%
\pgfpathclose%
\pgfusepath{stroke,fill}%
\end{pgfscope}%
\begin{pgfscope}%
\pgfpathrectangle{\pgfqpoint{1.250000in}{1.750000in}}{\pgfqpoint{2.279412in}{2.004545in}}%
\pgfusepath{clip}%
\pgfsetroundcap%
\pgfsetroundjoin%
\pgfsetlinewidth{1.298395pt}%
\definecolor{currentstroke}{rgb}{0.121380,0.629492,0.531973}%
\pgfsetstrokecolor{currentstroke}%
\pgfsetdash{}{0pt}%
\pgfpathmoveto{\pgfqpoint{2.444979in}{2.587928in}}%
\pgfpathquadraticcurveto{\pgfqpoint{2.432737in}{2.590301in}}{\pgfqpoint{2.440214in}{2.588852in}}%
\pgfusepath{stroke}%
\end{pgfscope}%
\begin{pgfscope}%
\pgfpathrectangle{\pgfqpoint{1.250000in}{1.750000in}}{\pgfqpoint{2.279412in}{2.004545in}}%
\pgfusepath{clip}%
\pgfsetroundcap%
\pgfsetroundjoin%
\definecolor{currentfill}{rgb}{0.121380,0.629492,0.531973}%
\pgfsetfillcolor{currentfill}%
\pgfsetlinewidth{1.298395pt}%
\definecolor{currentstroke}{rgb}{0.121380,0.629492,0.531973}%
\pgfsetstrokecolor{currentstroke}%
\pgfsetdash{}{0pt}%
\pgfpathmoveto{\pgfqpoint{2.489470in}{2.551011in}}%
\pgfpathlineto{\pgfqpoint{2.440214in}{2.588852in}}%
\pgfpathlineto{\pgfqpoint{2.500040in}{2.605552in}}%
\pgfpathlineto{\pgfqpoint{2.489470in}{2.551011in}}%
\pgfpathlineto{\pgfqpoint{2.489470in}{2.551011in}}%
\pgfpathclose%
\pgfusepath{stroke,fill}%
\end{pgfscope}%
\begin{pgfscope}%
\pgfpathrectangle{\pgfqpoint{1.250000in}{1.750000in}}{\pgfqpoint{2.279412in}{2.004545in}}%
\pgfusepath{clip}%
\pgfsetbuttcap%
\pgfsetroundjoin%
\pgfsetlinewidth{1.505625pt}%
\definecolor{currentstroke}{rgb}{0.000000,0.000000,0.000000}%
\pgfsetstrokecolor{currentstroke}%
\pgfsetdash{}{0pt}%
\pgfpathmoveto{\pgfqpoint{2.043384in}{2.089441in}}%
\pgfpathlineto{\pgfqpoint{2.043384in}{3.415105in}}%
\pgfusepath{stroke}%
\end{pgfscope}%
\begin{pgfscope}%
\pgfpathrectangle{\pgfqpoint{1.250000in}{1.750000in}}{\pgfqpoint{2.279412in}{2.004545in}}%
\pgfusepath{clip}%
\pgfsetbuttcap%
\pgfsetroundjoin%
\pgfsetlinewidth{1.505625pt}%
\definecolor{currentstroke}{rgb}{0.000000,0.000000,0.000000}%
\pgfsetstrokecolor{currentstroke}%
\pgfsetdash{}{0pt}%
\pgfpathmoveto{\pgfqpoint{3.191910in}{2.089441in}}%
\pgfpathlineto{\pgfqpoint{3.191910in}{3.415105in}}%
\pgfusepath{stroke}%
\end{pgfscope}%
\begin{pgfscope}%
\pgfsetrectcap%
\pgfsetmiterjoin%
\pgfsetlinewidth{0.803000pt}%
\definecolor{currentstroke}{rgb}{0.000000,0.000000,0.000000}%
\pgfsetstrokecolor{currentstroke}%
\pgfsetdash{}{0pt}%
\pgfpathmoveto{\pgfqpoint{1.250000in}{1.750000in}}%
\pgfpathlineto{\pgfqpoint{1.250000in}{3.754545in}}%
\pgfusepath{stroke}%
\end{pgfscope}%
\begin{pgfscope}%
\pgfsetrectcap%
\pgfsetmiterjoin%
\pgfsetlinewidth{0.803000pt}%
\definecolor{currentstroke}{rgb}{0.000000,0.000000,0.000000}%
\pgfsetstrokecolor{currentstroke}%
\pgfsetdash{}{0pt}%
\pgfpathmoveto{\pgfqpoint{3.529412in}{1.750000in}}%
\pgfpathlineto{\pgfqpoint{3.529412in}{3.754545in}}%
\pgfusepath{stroke}%
\end{pgfscope}%
\begin{pgfscope}%
\pgfsetrectcap%
\pgfsetmiterjoin%
\pgfsetlinewidth{0.803000pt}%
\definecolor{currentstroke}{rgb}{0.000000,0.000000,0.000000}%
\pgfsetstrokecolor{currentstroke}%
\pgfsetdash{}{0pt}%
\pgfpathmoveto{\pgfqpoint{1.250000in}{1.750000in}}%
\pgfpathlineto{\pgfqpoint{3.529412in}{1.750000in}}%
\pgfusepath{stroke}%
\end{pgfscope}%
\begin{pgfscope}%
\pgfsetrectcap%
\pgfsetmiterjoin%
\pgfsetlinewidth{0.803000pt}%
\definecolor{currentstroke}{rgb}{0.000000,0.000000,0.000000}%
\pgfsetstrokecolor{currentstroke}%
\pgfsetdash{}{0pt}%
\pgfpathmoveto{\pgfqpoint{1.250000in}{3.754545in}}%
\pgfpathlineto{\pgfqpoint{3.529412in}{3.754545in}}%
\pgfusepath{stroke}%
\end{pgfscope}%
\begin{pgfscope}%
\definecolor{textcolor}{rgb}{0.000000,0.000000,0.000000}%
\pgfsetstrokecolor{textcolor}%
\pgfsetfillcolor{textcolor}%
\pgftext[x=2.389706in,y=3.837879in,,base]{\color{textcolor}\sffamily\fontsize{12.000000}{14.400000}\selectfont d)}%
\end{pgfscope}%
\begin{pgfscope}%
\pgfsetbuttcap%
\pgfsetmiterjoin%
\definecolor{currentfill}{rgb}{1.000000,1.000000,1.000000}%
\pgfsetfillcolor{currentfill}%
\pgfsetlinewidth{0.000000pt}%
\definecolor{currentstroke}{rgb}{0.000000,0.000000,0.000000}%
\pgfsetstrokecolor{currentstroke}%
\pgfsetstrokeopacity{0.000000}%
\pgfsetdash{}{0pt}%
\pgfpathmoveto{\pgfqpoint{3.985294in}{1.750000in}}%
\pgfpathlineto{\pgfqpoint{6.264706in}{1.750000in}}%
\pgfpathlineto{\pgfqpoint{6.264706in}{3.754545in}}%
\pgfpathlineto{\pgfqpoint{3.985294in}{3.754545in}}%
\pgfpathlineto{\pgfqpoint{3.985294in}{1.750000in}}%
\pgfpathclose%
\pgfusepath{fill}%
\end{pgfscope}%
\begin{pgfscope}%
\pgfpathrectangle{\pgfqpoint{3.985294in}{1.750000in}}{\pgfqpoint{2.279412in}{2.004545in}}%
\pgfusepath{clip}%
\pgfsys@transformcm{2.291667}{0.000000}{0.000000}{2.013889}{3.985294in}{1.750000in}%
\pgftext[left,bottom]{\includegraphics[interpolate=false,width=1.000000in,height=1.000000in]{q_series-img4.png}}%
\end{pgfscope}%
\begin{pgfscope}%
\pgfsetbuttcap%
\pgfsetroundjoin%
\definecolor{currentfill}{rgb}{0.000000,0.000000,0.000000}%
\pgfsetfillcolor{currentfill}%
\pgfsetlinewidth{0.803000pt}%
\definecolor{currentstroke}{rgb}{0.000000,0.000000,0.000000}%
\pgfsetstrokecolor{currentstroke}%
\pgfsetdash{}{0pt}%
\pgfsys@defobject{currentmarker}{\pgfqpoint{0.000000in}{-0.048611in}}{\pgfqpoint{0.000000in}{0.000000in}}{%
\pgfpathmoveto{\pgfqpoint{0.000000in}{0.000000in}}%
\pgfpathlineto{\pgfqpoint{0.000000in}{-0.048611in}}%
\pgfusepath{stroke,fill}%
}%
\begin{pgfscope}%
\pgfsys@transformshift{4.395836in}{1.750000in}%
\pgfsys@useobject{currentmarker}{}%
\end{pgfscope}%
\end{pgfscope}%
\begin{pgfscope}%
\definecolor{textcolor}{rgb}{0.000000,0.000000,0.000000}%
\pgfsetstrokecolor{textcolor}%
\pgfsetfillcolor{textcolor}%
\pgftext[x=4.395836in,y=1.652778in,,top]{\color{textcolor}\sffamily\fontsize{10.000000}{12.000000}\selectfont \(\displaystyle {\ensuremath{-}10}\)}%
\end{pgfscope}%
\begin{pgfscope}%
\pgfsetbuttcap%
\pgfsetroundjoin%
\definecolor{currentfill}{rgb}{0.000000,0.000000,0.000000}%
\pgfsetfillcolor{currentfill}%
\pgfsetlinewidth{0.803000pt}%
\definecolor{currentstroke}{rgb}{0.000000,0.000000,0.000000}%
\pgfsetstrokecolor{currentstroke}%
\pgfsetdash{}{0pt}%
\pgfsys@defobject{currentmarker}{\pgfqpoint{0.000000in}{-0.048611in}}{\pgfqpoint{0.000000in}{0.000000in}}{%
\pgfpathmoveto{\pgfqpoint{0.000000in}{0.000000in}}%
\pgfpathlineto{\pgfqpoint{0.000000in}{-0.048611in}}%
\pgfusepath{stroke,fill}%
}%
\begin{pgfscope}%
\pgfsys@transformshift{4.874388in}{1.750000in}%
\pgfsys@useobject{currentmarker}{}%
\end{pgfscope}%
\end{pgfscope}%
\begin{pgfscope}%
\definecolor{textcolor}{rgb}{0.000000,0.000000,0.000000}%
\pgfsetstrokecolor{textcolor}%
\pgfsetfillcolor{textcolor}%
\pgftext[x=4.874388in,y=1.652778in,,top]{\color{textcolor}\sffamily\fontsize{10.000000}{12.000000}\selectfont \(\displaystyle {\ensuremath{-}5}\)}%
\end{pgfscope}%
\begin{pgfscope}%
\pgfsetbuttcap%
\pgfsetroundjoin%
\definecolor{currentfill}{rgb}{0.000000,0.000000,0.000000}%
\pgfsetfillcolor{currentfill}%
\pgfsetlinewidth{0.803000pt}%
\definecolor{currentstroke}{rgb}{0.000000,0.000000,0.000000}%
\pgfsetstrokecolor{currentstroke}%
\pgfsetdash{}{0pt}%
\pgfsys@defobject{currentmarker}{\pgfqpoint{0.000000in}{-0.048611in}}{\pgfqpoint{0.000000in}{0.000000in}}{%
\pgfpathmoveto{\pgfqpoint{0.000000in}{0.000000in}}%
\pgfpathlineto{\pgfqpoint{0.000000in}{-0.048611in}}%
\pgfusepath{stroke,fill}%
}%
\begin{pgfscope}%
\pgfsys@transformshift{5.352941in}{1.750000in}%
\pgfsys@useobject{currentmarker}{}%
\end{pgfscope}%
\end{pgfscope}%
\begin{pgfscope}%
\definecolor{textcolor}{rgb}{0.000000,0.000000,0.000000}%
\pgfsetstrokecolor{textcolor}%
\pgfsetfillcolor{textcolor}%
\pgftext[x=5.352941in,y=1.652778in,,top]{\color{textcolor}\sffamily\fontsize{10.000000}{12.000000}\selectfont \(\displaystyle {0}\)}%
\end{pgfscope}%
\begin{pgfscope}%
\pgfsetbuttcap%
\pgfsetroundjoin%
\definecolor{currentfill}{rgb}{0.000000,0.000000,0.000000}%
\pgfsetfillcolor{currentfill}%
\pgfsetlinewidth{0.803000pt}%
\definecolor{currentstroke}{rgb}{0.000000,0.000000,0.000000}%
\pgfsetstrokecolor{currentstroke}%
\pgfsetdash{}{0pt}%
\pgfsys@defobject{currentmarker}{\pgfqpoint{0.000000in}{-0.048611in}}{\pgfqpoint{0.000000in}{0.000000in}}{%
\pgfpathmoveto{\pgfqpoint{0.000000in}{0.000000in}}%
\pgfpathlineto{\pgfqpoint{0.000000in}{-0.048611in}}%
\pgfusepath{stroke,fill}%
}%
\begin{pgfscope}%
\pgfsys@transformshift{5.831494in}{1.750000in}%
\pgfsys@useobject{currentmarker}{}%
\end{pgfscope}%
\end{pgfscope}%
\begin{pgfscope}%
\definecolor{textcolor}{rgb}{0.000000,0.000000,0.000000}%
\pgfsetstrokecolor{textcolor}%
\pgfsetfillcolor{textcolor}%
\pgftext[x=5.831494in,y=1.652778in,,top]{\color{textcolor}\sffamily\fontsize{10.000000}{12.000000}\selectfont \(\displaystyle {5}\)}%
\end{pgfscope}%
\begin{pgfscope}%
\definecolor{textcolor}{rgb}{0.000000,0.000000,0.000000}%
\pgfsetstrokecolor{textcolor}%
\pgfsetfillcolor{textcolor}%
\pgftext[x=5.125000in,y=1.473766in,,top]{\color{textcolor}\sffamily\fontsize{10.000000}{12.000000}\selectfont \(\displaystyle \zeta \, \mathrm{[\mu m]}\)}%
\end{pgfscope}%
\begin{pgfscope}%
\pgfsetbuttcap%
\pgfsetroundjoin%
\definecolor{currentfill}{rgb}{0.000000,0.000000,0.000000}%
\pgfsetfillcolor{currentfill}%
\pgfsetlinewidth{0.803000pt}%
\definecolor{currentstroke}{rgb}{0.000000,0.000000,0.000000}%
\pgfsetstrokecolor{currentstroke}%
\pgfsetdash{}{0pt}%
\pgfsys@defobject{currentmarker}{\pgfqpoint{-0.048611in}{0.000000in}}{\pgfqpoint{-0.000000in}{0.000000in}}{%
\pgfpathmoveto{\pgfqpoint{-0.000000in}{0.000000in}}%
\pgfpathlineto{\pgfqpoint{-0.048611in}{0.000000in}}%
\pgfusepath{stroke,fill}%
}%
\begin{pgfscope}%
\pgfsys@transformshift{3.985294in}{1.758025in}%
\pgfsys@useobject{currentmarker}{}%
\end{pgfscope}%
\end{pgfscope}%
\begin{pgfscope}%
\pgfsetbuttcap%
\pgfsetroundjoin%
\definecolor{currentfill}{rgb}{0.000000,0.000000,0.000000}%
\pgfsetfillcolor{currentfill}%
\pgfsetlinewidth{0.803000pt}%
\definecolor{currentstroke}{rgb}{0.000000,0.000000,0.000000}%
\pgfsetstrokecolor{currentstroke}%
\pgfsetdash{}{0pt}%
\pgfsys@defobject{currentmarker}{\pgfqpoint{-0.048611in}{0.000000in}}{\pgfqpoint{-0.000000in}{0.000000in}}{%
\pgfpathmoveto{\pgfqpoint{-0.000000in}{0.000000in}}%
\pgfpathlineto{\pgfqpoint{-0.048611in}{0.000000in}}%
\pgfusepath{stroke,fill}%
}%
\begin{pgfscope}%
\pgfsys@transformshift{3.985294in}{2.089441in}%
\pgfsys@useobject{currentmarker}{}%
\end{pgfscope}%
\end{pgfscope}%
\begin{pgfscope}%
\pgfsetbuttcap%
\pgfsetroundjoin%
\definecolor{currentfill}{rgb}{0.000000,0.000000,0.000000}%
\pgfsetfillcolor{currentfill}%
\pgfsetlinewidth{0.803000pt}%
\definecolor{currentstroke}{rgb}{0.000000,0.000000,0.000000}%
\pgfsetstrokecolor{currentstroke}%
\pgfsetdash{}{0pt}%
\pgfsys@defobject{currentmarker}{\pgfqpoint{-0.048611in}{0.000000in}}{\pgfqpoint{-0.000000in}{0.000000in}}{%
\pgfpathmoveto{\pgfqpoint{-0.000000in}{0.000000in}}%
\pgfpathlineto{\pgfqpoint{-0.048611in}{0.000000in}}%
\pgfusepath{stroke,fill}%
}%
\begin{pgfscope}%
\pgfsys@transformshift{3.985294in}{2.420857in}%
\pgfsys@useobject{currentmarker}{}%
\end{pgfscope}%
\end{pgfscope}%
\begin{pgfscope}%
\pgfsetbuttcap%
\pgfsetroundjoin%
\definecolor{currentfill}{rgb}{0.000000,0.000000,0.000000}%
\pgfsetfillcolor{currentfill}%
\pgfsetlinewidth{0.803000pt}%
\definecolor{currentstroke}{rgb}{0.000000,0.000000,0.000000}%
\pgfsetstrokecolor{currentstroke}%
\pgfsetdash{}{0pt}%
\pgfsys@defobject{currentmarker}{\pgfqpoint{-0.048611in}{0.000000in}}{\pgfqpoint{-0.000000in}{0.000000in}}{%
\pgfpathmoveto{\pgfqpoint{-0.000000in}{0.000000in}}%
\pgfpathlineto{\pgfqpoint{-0.048611in}{0.000000in}}%
\pgfusepath{stroke,fill}%
}%
\begin{pgfscope}%
\pgfsys@transformshift{3.985294in}{2.752273in}%
\pgfsys@useobject{currentmarker}{}%
\end{pgfscope}%
\end{pgfscope}%
\begin{pgfscope}%
\pgfsetbuttcap%
\pgfsetroundjoin%
\definecolor{currentfill}{rgb}{0.000000,0.000000,0.000000}%
\pgfsetfillcolor{currentfill}%
\pgfsetlinewidth{0.803000pt}%
\definecolor{currentstroke}{rgb}{0.000000,0.000000,0.000000}%
\pgfsetstrokecolor{currentstroke}%
\pgfsetdash{}{0pt}%
\pgfsys@defobject{currentmarker}{\pgfqpoint{-0.048611in}{0.000000in}}{\pgfqpoint{-0.000000in}{0.000000in}}{%
\pgfpathmoveto{\pgfqpoint{-0.000000in}{0.000000in}}%
\pgfpathlineto{\pgfqpoint{-0.048611in}{0.000000in}}%
\pgfusepath{stroke,fill}%
}%
\begin{pgfscope}%
\pgfsys@transformshift{3.985294in}{3.083689in}%
\pgfsys@useobject{currentmarker}{}%
\end{pgfscope}%
\end{pgfscope}%
\begin{pgfscope}%
\pgfsetbuttcap%
\pgfsetroundjoin%
\definecolor{currentfill}{rgb}{0.000000,0.000000,0.000000}%
\pgfsetfillcolor{currentfill}%
\pgfsetlinewidth{0.803000pt}%
\definecolor{currentstroke}{rgb}{0.000000,0.000000,0.000000}%
\pgfsetstrokecolor{currentstroke}%
\pgfsetdash{}{0pt}%
\pgfsys@defobject{currentmarker}{\pgfqpoint{-0.048611in}{0.000000in}}{\pgfqpoint{-0.000000in}{0.000000in}}{%
\pgfpathmoveto{\pgfqpoint{-0.000000in}{0.000000in}}%
\pgfpathlineto{\pgfqpoint{-0.048611in}{0.000000in}}%
\pgfusepath{stroke,fill}%
}%
\begin{pgfscope}%
\pgfsys@transformshift{3.985294in}{3.415105in}%
\pgfsys@useobject{currentmarker}{}%
\end{pgfscope}%
\end{pgfscope}%
\begin{pgfscope}%
\pgfsetbuttcap%
\pgfsetroundjoin%
\definecolor{currentfill}{rgb}{0.000000,0.000000,0.000000}%
\pgfsetfillcolor{currentfill}%
\pgfsetlinewidth{0.803000pt}%
\definecolor{currentstroke}{rgb}{0.000000,0.000000,0.000000}%
\pgfsetstrokecolor{currentstroke}%
\pgfsetdash{}{0pt}%
\pgfsys@defobject{currentmarker}{\pgfqpoint{-0.048611in}{0.000000in}}{\pgfqpoint{-0.000000in}{0.000000in}}{%
\pgfpathmoveto{\pgfqpoint{-0.000000in}{0.000000in}}%
\pgfpathlineto{\pgfqpoint{-0.048611in}{0.000000in}}%
\pgfusepath{stroke,fill}%
}%
\begin{pgfscope}%
\pgfsys@transformshift{3.985294in}{3.746521in}%
\pgfsys@useobject{currentmarker}{}%
\end{pgfscope}%
\end{pgfscope}%
\begin{pgfscope}%
\definecolor{textcolor}{rgb}{0.000000,0.000000,0.000000}%
\pgfsetstrokecolor{textcolor}%
\pgfsetfillcolor{textcolor}%
\pgftext[x=3.929739in,y=2.752273in,,bottom,rotate=90.000000]{\color{textcolor}\sffamily\fontsize{10.000000}{12.000000}\selectfont \(\displaystyle z \, \mathrm{[\mu m]}\)}%
\end{pgfscope}%
\begin{pgfscope}%
\pgfpathrectangle{\pgfqpoint{3.985294in}{1.750000in}}{\pgfqpoint{2.279412in}{2.004545in}}%
\pgfusepath{clip}%
\pgfsetbuttcap%
\pgfsetroundjoin%
\pgfsetlinewidth{0.312828pt}%
\definecolor{currentstroke}{rgb}{0.268510,0.009605,0.335427}%
\pgfsetstrokecolor{currentstroke}%
\pgfsetdash{}{0pt}%
\pgfpathmoveto{\pgfqpoint{5.945670in}{2.797380in}}%
\pgfpathlineto{\pgfqpoint{5.895578in}{2.797547in}}%
\pgfusepath{stroke}%
\end{pgfscope}%
\begin{pgfscope}%
\pgfpathrectangle{\pgfqpoint{3.985294in}{1.750000in}}{\pgfqpoint{2.279412in}{2.004545in}}%
\pgfusepath{clip}%
\pgfsetbuttcap%
\pgfsetroundjoin%
\pgfsetlinewidth{0.310269pt}%
\definecolor{currentstroke}{rgb}{0.268510,0.009605,0.335427}%
\pgfsetstrokecolor{currentstroke}%
\pgfsetdash{}{0pt}%
\pgfpathmoveto{\pgfqpoint{5.895578in}{2.797547in}}%
\pgfpathlineto{\pgfqpoint{5.845497in}{2.797272in}}%
\pgfusepath{stroke}%
\end{pgfscope}%
\begin{pgfscope}%
\pgfpathrectangle{\pgfqpoint{3.985294in}{1.750000in}}{\pgfqpoint{2.279412in}{2.004545in}}%
\pgfusepath{clip}%
\pgfsetbuttcap%
\pgfsetroundjoin%
\pgfsetlinewidth{0.319025pt}%
\definecolor{currentstroke}{rgb}{0.269944,0.014625,0.341379}%
\pgfsetstrokecolor{currentstroke}%
\pgfsetdash{}{0pt}%
\pgfpathmoveto{\pgfqpoint{5.845497in}{2.797272in}}%
\pgfpathlineto{\pgfqpoint{5.795424in}{2.796711in}}%
\pgfusepath{stroke}%
\end{pgfscope}%
\begin{pgfscope}%
\pgfpathrectangle{\pgfqpoint{3.985294in}{1.750000in}}{\pgfqpoint{2.279412in}{2.004545in}}%
\pgfusepath{clip}%
\pgfsetbuttcap%
\pgfsetroundjoin%
\pgfsetlinewidth{0.321601pt}%
\definecolor{currentstroke}{rgb}{0.269944,0.014625,0.341379}%
\pgfsetstrokecolor{currentstroke}%
\pgfsetdash{}{0pt}%
\pgfpathmoveto{\pgfqpoint{5.795424in}{2.796711in}}%
\pgfpathlineto{\pgfqpoint{5.745304in}{2.796056in}}%
\pgfusepath{stroke}%
\end{pgfscope}%
\begin{pgfscope}%
\pgfpathrectangle{\pgfqpoint{3.985294in}{1.750000in}}{\pgfqpoint{2.279412in}{2.004545in}}%
\pgfusepath{clip}%
\pgfsetbuttcap%
\pgfsetroundjoin%
\pgfsetlinewidth{0.333302pt}%
\definecolor{currentstroke}{rgb}{0.272594,0.025563,0.353093}%
\pgfsetstrokecolor{currentstroke}%
\pgfsetdash{}{0pt}%
\pgfpathmoveto{\pgfqpoint{5.745304in}{2.796056in}}%
\pgfpathlineto{\pgfqpoint{5.695157in}{2.795917in}}%
\pgfusepath{stroke}%
\end{pgfscope}%
\begin{pgfscope}%
\pgfpathrectangle{\pgfqpoint{3.985294in}{1.750000in}}{\pgfqpoint{2.279412in}{2.004545in}}%
\pgfusepath{clip}%
\pgfsetbuttcap%
\pgfsetroundjoin%
\pgfsetlinewidth{0.342905pt}%
\definecolor{currentstroke}{rgb}{0.274952,0.037752,0.364543}%
\pgfsetstrokecolor{currentstroke}%
\pgfsetdash{}{0pt}%
\pgfpathmoveto{\pgfqpoint{5.695157in}{2.795917in}}%
\pgfpathlineto{\pgfqpoint{5.645006in}{2.795948in}}%
\pgfusepath{stroke}%
\end{pgfscope}%
\begin{pgfscope}%
\pgfpathrectangle{\pgfqpoint{3.985294in}{1.750000in}}{\pgfqpoint{2.279412in}{2.004545in}}%
\pgfusepath{clip}%
\pgfsetbuttcap%
\pgfsetroundjoin%
\pgfsetlinewidth{0.377669pt}%
\definecolor{currentstroke}{rgb}{0.279566,0.067836,0.391917}%
\pgfsetstrokecolor{currentstroke}%
\pgfsetdash{}{0pt}%
\pgfpathmoveto{\pgfqpoint{5.645006in}{2.795948in}}%
\pgfpathlineto{\pgfqpoint{5.594855in}{2.795778in}}%
\pgfusepath{stroke}%
\end{pgfscope}%
\begin{pgfscope}%
\pgfpathrectangle{\pgfqpoint{3.985294in}{1.750000in}}{\pgfqpoint{2.279412in}{2.004545in}}%
\pgfusepath{clip}%
\pgfsetbuttcap%
\pgfsetroundjoin%
\pgfsetlinewidth{0.430685pt}%
\definecolor{currentstroke}{rgb}{0.282910,0.105393,0.426902}%
\pgfsetstrokecolor{currentstroke}%
\pgfsetdash{}{0pt}%
\pgfpathmoveto{\pgfqpoint{5.594855in}{2.795778in}}%
\pgfpathlineto{\pgfqpoint{5.544705in}{2.795438in}}%
\pgfusepath{stroke}%
\end{pgfscope}%
\begin{pgfscope}%
\pgfpathrectangle{\pgfqpoint{3.985294in}{1.750000in}}{\pgfqpoint{2.279412in}{2.004545in}}%
\pgfusepath{clip}%
\pgfsetbuttcap%
\pgfsetroundjoin%
\pgfsetlinewidth{0.476832pt}%
\definecolor{currentstroke}{rgb}{0.282623,0.140926,0.457517}%
\pgfsetstrokecolor{currentstroke}%
\pgfsetdash{}{0pt}%
\pgfpathmoveto{\pgfqpoint{5.544705in}{2.795438in}}%
\pgfpathlineto{\pgfqpoint{5.494554in}{2.795154in}}%
\pgfusepath{stroke}%
\end{pgfscope}%
\begin{pgfscope}%
\pgfpathrectangle{\pgfqpoint{3.985294in}{1.750000in}}{\pgfqpoint{2.279412in}{2.004545in}}%
\pgfusepath{clip}%
\pgfsetbuttcap%
\pgfsetroundjoin%
\pgfsetlinewidth{0.538580pt}%
\definecolor{currentstroke}{rgb}{0.277134,0.185228,0.489898}%
\pgfsetstrokecolor{currentstroke}%
\pgfsetdash{}{0pt}%
\pgfpathmoveto{\pgfqpoint{5.494554in}{2.795154in}}%
\pgfpathlineto{\pgfqpoint{5.444404in}{2.794861in}}%
\pgfusepath{stroke}%
\end{pgfscope}%
\begin{pgfscope}%
\pgfpathrectangle{\pgfqpoint{3.985294in}{1.750000in}}{\pgfqpoint{2.279412in}{2.004545in}}%
\pgfusepath{clip}%
\pgfsetbuttcap%
\pgfsetroundjoin%
\pgfsetlinewidth{0.629436pt}%
\definecolor{currentstroke}{rgb}{0.260571,0.246922,0.522828}%
\pgfsetstrokecolor{currentstroke}%
\pgfsetdash{}{0pt}%
\pgfpathmoveto{\pgfqpoint{5.444404in}{2.794861in}}%
\pgfpathlineto{\pgfqpoint{5.394253in}{2.794519in}}%
\pgfusepath{stroke}%
\end{pgfscope}%
\begin{pgfscope}%
\pgfpathrectangle{\pgfqpoint{3.985294in}{1.750000in}}{\pgfqpoint{2.279412in}{2.004545in}}%
\pgfusepath{clip}%
\pgfsetbuttcap%
\pgfsetroundjoin%
\pgfsetlinewidth{0.778921pt}%
\definecolor{currentstroke}{rgb}{0.220057,0.343307,0.549413}%
\pgfsetstrokecolor{currentstroke}%
\pgfsetdash{}{0pt}%
\pgfpathmoveto{\pgfqpoint{5.394253in}{2.794519in}}%
\pgfpathlineto{\pgfqpoint{5.344106in}{2.793973in}}%
\pgfusepath{stroke}%
\end{pgfscope}%
\begin{pgfscope}%
\pgfpathrectangle{\pgfqpoint{3.985294in}{1.750000in}}{\pgfqpoint{2.279412in}{2.004545in}}%
\pgfusepath{clip}%
\pgfsetbuttcap%
\pgfsetroundjoin%
\pgfsetlinewidth{0.872830pt}%
\definecolor{currentstroke}{rgb}{0.194100,0.399323,0.555565}%
\pgfsetstrokecolor{currentstroke}%
\pgfsetdash{}{0pt}%
\pgfpathmoveto{\pgfqpoint{5.344106in}{2.793973in}}%
\pgfpathlineto{\pgfqpoint{5.293965in}{2.793064in}}%
\pgfusepath{stroke}%
\end{pgfscope}%
\begin{pgfscope}%
\pgfpathrectangle{\pgfqpoint{3.985294in}{1.750000in}}{\pgfqpoint{2.279412in}{2.004545in}}%
\pgfusepath{clip}%
\pgfsetbuttcap%
\pgfsetroundjoin%
\pgfsetlinewidth{0.935027pt}%
\definecolor{currentstroke}{rgb}{0.179019,0.433756,0.557430}%
\pgfsetstrokecolor{currentstroke}%
\pgfsetdash{}{0pt}%
\pgfpathmoveto{\pgfqpoint{5.293965in}{2.793064in}}%
\pgfpathlineto{\pgfqpoint{5.243833in}{2.791850in}}%
\pgfusepath{stroke}%
\end{pgfscope}%
\begin{pgfscope}%
\pgfpathrectangle{\pgfqpoint{3.985294in}{1.750000in}}{\pgfqpoint{2.279412in}{2.004545in}}%
\pgfusepath{clip}%
\pgfsetbuttcap%
\pgfsetroundjoin%
\pgfsetlinewidth{0.997916pt}%
\definecolor{currentstroke}{rgb}{0.165117,0.467423,0.558141}%
\pgfsetstrokecolor{currentstroke}%
\pgfsetdash{}{0pt}%
\pgfpathmoveto{\pgfqpoint{5.243833in}{2.791850in}}%
\pgfpathlineto{\pgfqpoint{5.193711in}{2.790332in}}%
\pgfusepath{stroke}%
\end{pgfscope}%
\begin{pgfscope}%
\pgfpathrectangle{\pgfqpoint{3.985294in}{1.750000in}}{\pgfqpoint{2.279412in}{2.004545in}}%
\pgfusepath{clip}%
\pgfsetbuttcap%
\pgfsetroundjoin%
\pgfsetlinewidth{0.989865pt}%
\definecolor{currentstroke}{rgb}{0.166617,0.463708,0.558119}%
\pgfsetstrokecolor{currentstroke}%
\pgfsetdash{}{0pt}%
\pgfpathmoveto{\pgfqpoint{5.193711in}{2.790332in}}%
\pgfpathlineto{\pgfqpoint{5.143613in}{2.788340in}}%
\pgfusepath{stroke}%
\end{pgfscope}%
\begin{pgfscope}%
\pgfpathrectangle{\pgfqpoint{3.985294in}{1.750000in}}{\pgfqpoint{2.279412in}{2.004545in}}%
\pgfusepath{clip}%
\pgfsetbuttcap%
\pgfsetroundjoin%
\pgfsetlinewidth{0.958060pt}%
\definecolor{currentstroke}{rgb}{0.174274,0.445044,0.557792}%
\pgfsetstrokecolor{currentstroke}%
\pgfsetdash{}{0pt}%
\pgfpathmoveto{\pgfqpoint{5.143613in}{2.788340in}}%
\pgfpathlineto{\pgfqpoint{5.093546in}{2.785823in}}%
\pgfusepath{stroke}%
\end{pgfscope}%
\begin{pgfscope}%
\pgfpathrectangle{\pgfqpoint{3.985294in}{1.750000in}}{\pgfqpoint{2.279412in}{2.004545in}}%
\pgfusepath{clip}%
\pgfsetbuttcap%
\pgfsetroundjoin%
\pgfsetlinewidth{1.028129pt}%
\definecolor{currentstroke}{rgb}{0.159194,0.482237,0.558073}%
\pgfsetstrokecolor{currentstroke}%
\pgfsetdash{}{0pt}%
\pgfpathmoveto{\pgfqpoint{5.093546in}{2.785823in}}%
\pgfpathlineto{\pgfqpoint{5.043511in}{2.782872in}}%
\pgfusepath{stroke}%
\end{pgfscope}%
\begin{pgfscope}%
\pgfpathrectangle{\pgfqpoint{3.985294in}{1.750000in}}{\pgfqpoint{2.279412in}{2.004545in}}%
\pgfusepath{clip}%
\pgfsetbuttcap%
\pgfsetroundjoin%
\pgfsetlinewidth{0.949756pt}%
\definecolor{currentstroke}{rgb}{0.175841,0.441290,0.557685}%
\pgfsetstrokecolor{currentstroke}%
\pgfsetdash{}{0pt}%
\pgfpathmoveto{\pgfqpoint{5.043511in}{2.782872in}}%
\pgfpathlineto{\pgfqpoint{4.993511in}{2.779513in}}%
\pgfusepath{stroke}%
\end{pgfscope}%
\begin{pgfscope}%
\pgfpathrectangle{\pgfqpoint{3.985294in}{1.750000in}}{\pgfqpoint{2.279412in}{2.004545in}}%
\pgfusepath{clip}%
\pgfsetbuttcap%
\pgfsetroundjoin%
\pgfsetlinewidth{0.963076pt}%
\definecolor{currentstroke}{rgb}{0.172719,0.448791,0.557885}%
\pgfsetstrokecolor{currentstroke}%
\pgfsetdash{}{0pt}%
\pgfpathmoveto{\pgfqpoint{4.993511in}{2.779513in}}%
\pgfpathlineto{\pgfqpoint{4.943565in}{2.775603in}}%
\pgfusepath{stroke}%
\end{pgfscope}%
\begin{pgfscope}%
\pgfpathrectangle{\pgfqpoint{3.985294in}{1.750000in}}{\pgfqpoint{2.279412in}{2.004545in}}%
\pgfusepath{clip}%
\pgfsetbuttcap%
\pgfsetroundjoin%
\pgfsetlinewidth{0.926902pt}%
\definecolor{currentstroke}{rgb}{0.180629,0.429975,0.557282}%
\pgfsetstrokecolor{currentstroke}%
\pgfsetdash{}{0pt}%
\pgfpathmoveto{\pgfqpoint{4.943565in}{2.775603in}}%
\pgfpathlineto{\pgfqpoint{4.893659in}{2.771283in}}%
\pgfusepath{stroke}%
\end{pgfscope}%
\begin{pgfscope}%
\pgfpathrectangle{\pgfqpoint{3.985294in}{1.750000in}}{\pgfqpoint{2.279412in}{2.004545in}}%
\pgfusepath{clip}%
\pgfsetbuttcap%
\pgfsetroundjoin%
\pgfsetlinewidth{0.781780pt}%
\definecolor{currentstroke}{rgb}{0.218130,0.347432,0.550038}%
\pgfsetstrokecolor{currentstroke}%
\pgfsetdash{}{0pt}%
\pgfpathmoveto{\pgfqpoint{4.893659in}{2.771283in}}%
\pgfpathlineto{\pgfqpoint{4.843832in}{2.766430in}}%
\pgfusepath{stroke}%
\end{pgfscope}%
\begin{pgfscope}%
\pgfpathrectangle{\pgfqpoint{3.985294in}{1.750000in}}{\pgfqpoint{2.279412in}{2.004545in}}%
\pgfusepath{clip}%
\pgfsetbuttcap%
\pgfsetroundjoin%
\pgfsetlinewidth{0.770538pt}%
\definecolor{currentstroke}{rgb}{0.221989,0.339161,0.548752}%
\pgfsetstrokecolor{currentstroke}%
\pgfsetdash{}{0pt}%
\pgfpathmoveto{\pgfqpoint{4.843832in}{2.766430in}}%
\pgfpathlineto{\pgfqpoint{4.794211in}{2.760317in}}%
\pgfusepath{stroke}%
\end{pgfscope}%
\begin{pgfscope}%
\pgfpathrectangle{\pgfqpoint{3.985294in}{1.750000in}}{\pgfqpoint{2.279412in}{2.004545in}}%
\pgfusepath{clip}%
\pgfsetbuttcap%
\pgfsetroundjoin%
\pgfsetlinewidth{0.675290pt}%
\definecolor{currentstroke}{rgb}{0.248629,0.278775,0.534556}%
\pgfsetstrokecolor{currentstroke}%
\pgfsetdash{}{0pt}%
\pgfpathmoveto{\pgfqpoint{4.794211in}{2.760317in}}%
\pgfpathlineto{\pgfqpoint{4.744990in}{2.752312in}}%
\pgfusepath{stroke}%
\end{pgfscope}%
\begin{pgfscope}%
\pgfpathrectangle{\pgfqpoint{3.985294in}{1.750000in}}{\pgfqpoint{2.279412in}{2.004545in}}%
\pgfusepath{clip}%
\pgfsetbuttcap%
\pgfsetroundjoin%
\pgfsetlinewidth{0.593919pt}%
\definecolor{currentstroke}{rgb}{0.267968,0.223549,0.512008}%
\pgfsetstrokecolor{currentstroke}%
\pgfsetdash{}{0pt}%
\pgfpathmoveto{\pgfqpoint{4.744990in}{2.752312in}}%
\pgfpathlineto{\pgfqpoint{4.697476in}{2.740838in}}%
\pgfusepath{stroke}%
\end{pgfscope}%
\begin{pgfscope}%
\pgfpathrectangle{\pgfqpoint{3.985294in}{1.750000in}}{\pgfqpoint{2.279412in}{2.004545in}}%
\pgfusepath{clip}%
\pgfsetbuttcap%
\pgfsetroundjoin%
\pgfsetlinewidth{0.469707pt}%
\definecolor{currentstroke}{rgb}{0.282884,0.135920,0.453427}%
\pgfsetstrokecolor{currentstroke}%
\pgfsetdash{}{0pt}%
\pgfpathmoveto{\pgfqpoint{4.697476in}{2.740838in}}%
\pgfpathlineto{\pgfqpoint{4.697476in}{2.740838in}}%
\pgfusepath{stroke}%
\end{pgfscope}%
\begin{pgfscope}%
\pgfpathrectangle{\pgfqpoint{3.985294in}{1.750000in}}{\pgfqpoint{2.279412in}{2.004545in}}%
\pgfusepath{clip}%
\pgfsetbuttcap%
\pgfsetroundjoin%
\pgfsetlinewidth{0.469707pt}%
\definecolor{currentstroke}{rgb}{0.282884,0.135920,0.453427}%
\pgfsetstrokecolor{currentstroke}%
\pgfsetdash{}{0pt}%
\pgfpathmoveto{\pgfqpoint{4.697476in}{2.740838in}}%
\pgfpathlineto{\pgfqpoint{4.688732in}{2.736616in}}%
\pgfusepath{stroke}%
\end{pgfscope}%
\begin{pgfscope}%
\pgfpathrectangle{\pgfqpoint{3.985294in}{1.750000in}}{\pgfqpoint{2.279412in}{2.004545in}}%
\pgfusepath{clip}%
\pgfsetbuttcap%
\pgfsetroundjoin%
\pgfsetlinewidth{0.440900pt}%
\definecolor{currentstroke}{rgb}{0.283197,0.115680,0.436115}%
\pgfsetstrokecolor{currentstroke}%
\pgfsetdash{}{0pt}%
\pgfpathmoveto{\pgfqpoint{4.688732in}{2.736616in}}%
\pgfpathlineto{\pgfqpoint{4.685177in}{2.733211in}}%
\pgfusepath{stroke}%
\end{pgfscope}%
\begin{pgfscope}%
\pgfpathrectangle{\pgfqpoint{3.985294in}{1.750000in}}{\pgfqpoint{2.279412in}{2.004545in}}%
\pgfusepath{clip}%
\pgfsetbuttcap%
\pgfsetroundjoin%
\pgfsetlinewidth{0.420652pt}%
\definecolor{currentstroke}{rgb}{0.282656,0.100196,0.422160}%
\pgfsetstrokecolor{currentstroke}%
\pgfsetdash{}{0pt}%
\pgfpathmoveto{\pgfqpoint{4.685177in}{2.733211in}}%
\pgfpathlineto{\pgfqpoint{4.684051in}{2.730690in}}%
\pgfusepath{stroke}%
\end{pgfscope}%
\begin{pgfscope}%
\pgfpathrectangle{\pgfqpoint{3.985294in}{1.750000in}}{\pgfqpoint{2.279412in}{2.004545in}}%
\pgfusepath{clip}%
\pgfsetbuttcap%
\pgfsetroundjoin%
\pgfsetlinewidth{0.414309pt}%
\definecolor{currentstroke}{rgb}{0.282327,0.094955,0.417331}%
\pgfsetstrokecolor{currentstroke}%
\pgfsetdash{}{0pt}%
\pgfpathmoveto{\pgfqpoint{4.684051in}{2.730690in}}%
\pgfpathlineto{\pgfqpoint{4.684108in}{2.728896in}}%
\pgfusepath{stroke}%
\end{pgfscope}%
\begin{pgfscope}%
\pgfpathrectangle{\pgfqpoint{3.985294in}{1.750000in}}{\pgfqpoint{2.279412in}{2.004545in}}%
\pgfusepath{clip}%
\pgfsetbuttcap%
\pgfsetroundjoin%
\pgfsetlinewidth{0.414242pt}%
\definecolor{currentstroke}{rgb}{0.282327,0.094955,0.417331}%
\pgfsetstrokecolor{currentstroke}%
\pgfsetdash{}{0pt}%
\pgfpathmoveto{\pgfqpoint{4.684108in}{2.728896in}}%
\pgfpathlineto{\pgfqpoint{4.684800in}{2.727678in}}%
\pgfusepath{stroke}%
\end{pgfscope}%
\begin{pgfscope}%
\pgfpathrectangle{\pgfqpoint{3.985294in}{1.750000in}}{\pgfqpoint{2.279412in}{2.004545in}}%
\pgfusepath{clip}%
\pgfsetbuttcap%
\pgfsetroundjoin%
\pgfsetlinewidth{0.415561pt}%
\definecolor{currentstroke}{rgb}{0.282327,0.094955,0.417331}%
\pgfsetstrokecolor{currentstroke}%
\pgfsetdash{}{0pt}%
\pgfpathmoveto{\pgfqpoint{4.684800in}{2.727678in}}%
\pgfpathlineto{\pgfqpoint{4.686282in}{2.726872in}}%
\pgfusepath{stroke}%
\end{pgfscope}%
\begin{pgfscope}%
\pgfpathrectangle{\pgfqpoint{3.985294in}{1.750000in}}{\pgfqpoint{2.279412in}{2.004545in}}%
\pgfusepath{clip}%
\pgfsetbuttcap%
\pgfsetroundjoin%
\pgfsetlinewidth{0.418112pt}%
\definecolor{currentstroke}{rgb}{0.282656,0.100196,0.422160}%
\pgfsetstrokecolor{currentstroke}%
\pgfsetdash{}{0pt}%
\pgfpathmoveto{\pgfqpoint{4.686282in}{2.726872in}}%
\pgfpathlineto{\pgfqpoint{4.686282in}{2.726872in}}%
\pgfusepath{stroke}%
\end{pgfscope}%
\begin{pgfscope}%
\pgfpathrectangle{\pgfqpoint{3.985294in}{1.750000in}}{\pgfqpoint{2.279412in}{2.004545in}}%
\pgfusepath{clip}%
\pgfsetbuttcap%
\pgfsetroundjoin%
\pgfsetlinewidth{0.418112pt}%
\definecolor{currentstroke}{rgb}{0.282656,0.100196,0.422160}%
\pgfsetstrokecolor{currentstroke}%
\pgfsetdash{}{0pt}%
\pgfpathmoveto{\pgfqpoint{4.686282in}{2.726872in}}%
\pgfpathlineto{\pgfqpoint{4.686372in}{2.726564in}}%
\pgfusepath{stroke}%
\end{pgfscope}%
\begin{pgfscope}%
\pgfpathrectangle{\pgfqpoint{3.985294in}{1.750000in}}{\pgfqpoint{2.279412in}{2.004545in}}%
\pgfusepath{clip}%
\pgfsetbuttcap%
\pgfsetroundjoin%
\pgfsetlinewidth{0.418037pt}%
\definecolor{currentstroke}{rgb}{0.282656,0.100196,0.422160}%
\pgfsetstrokecolor{currentstroke}%
\pgfsetdash{}{0pt}%
\pgfpathmoveto{\pgfqpoint{4.686372in}{2.726564in}}%
\pgfpathlineto{\pgfqpoint{4.686369in}{2.726420in}}%
\pgfusepath{stroke}%
\end{pgfscope}%
\begin{pgfscope}%
\pgfpathrectangle{\pgfqpoint{3.985294in}{1.750000in}}{\pgfqpoint{2.279412in}{2.004545in}}%
\pgfusepath{clip}%
\pgfsetbuttcap%
\pgfsetroundjoin%
\pgfsetlinewidth{0.417916pt}%
\definecolor{currentstroke}{rgb}{0.282656,0.100196,0.422160}%
\pgfsetstrokecolor{currentstroke}%
\pgfsetdash{}{0pt}%
\pgfpathmoveto{\pgfqpoint{4.686369in}{2.726420in}}%
\pgfpathlineto{\pgfqpoint{4.686403in}{2.726341in}}%
\pgfusepath{stroke}%
\end{pgfscope}%
\begin{pgfscope}%
\pgfpathrectangle{\pgfqpoint{3.985294in}{1.750000in}}{\pgfqpoint{2.279412in}{2.004545in}}%
\pgfusepath{clip}%
\pgfsetbuttcap%
\pgfsetroundjoin%
\pgfsetlinewidth{0.417910pt}%
\definecolor{currentstroke}{rgb}{0.282656,0.100196,0.422160}%
\pgfsetstrokecolor{currentstroke}%
\pgfsetdash{}{0pt}%
\pgfpathmoveto{\pgfqpoint{4.686403in}{2.726341in}}%
\pgfpathlineto{\pgfqpoint{4.686501in}{2.726287in}}%
\pgfusepath{stroke}%
\end{pgfscope}%
\begin{pgfscope}%
\pgfpathrectangle{\pgfqpoint{3.985294in}{1.750000in}}{\pgfqpoint{2.279412in}{2.004545in}}%
\pgfusepath{clip}%
\pgfsetbuttcap%
\pgfsetroundjoin%
\pgfsetlinewidth{0.418036pt}%
\definecolor{currentstroke}{rgb}{0.282656,0.100196,0.422160}%
\pgfsetstrokecolor{currentstroke}%
\pgfsetdash{}{0pt}%
\pgfpathmoveto{\pgfqpoint{4.686501in}{2.726287in}}%
\pgfpathlineto{\pgfqpoint{4.686600in}{2.726250in}}%
\pgfusepath{stroke}%
\end{pgfscope}%
\begin{pgfscope}%
\pgfpathrectangle{\pgfqpoint{3.985294in}{1.750000in}}{\pgfqpoint{2.279412in}{2.004545in}}%
\pgfusepath{clip}%
\pgfsetbuttcap%
\pgfsetroundjoin%
\pgfsetlinewidth{0.418176pt}%
\definecolor{currentstroke}{rgb}{0.282656,0.100196,0.422160}%
\pgfsetstrokecolor{currentstroke}%
\pgfsetdash{}{0pt}%
\pgfpathmoveto{\pgfqpoint{4.686600in}{2.726250in}}%
\pgfpathlineto{\pgfqpoint{4.686612in}{2.726239in}}%
\pgfusepath{stroke}%
\end{pgfscope}%
\begin{pgfscope}%
\pgfpathrectangle{\pgfqpoint{3.985294in}{1.750000in}}{\pgfqpoint{2.279412in}{2.004545in}}%
\pgfusepath{clip}%
\pgfsetbuttcap%
\pgfsetroundjoin%
\pgfsetlinewidth{0.418187pt}%
\definecolor{currentstroke}{rgb}{0.282656,0.100196,0.422160}%
\pgfsetstrokecolor{currentstroke}%
\pgfsetdash{}{0pt}%
\pgfpathmoveto{\pgfqpoint{4.686612in}{2.726239in}}%
\pgfpathlineto{\pgfqpoint{4.686534in}{2.726249in}}%
\pgfusepath{stroke}%
\end{pgfscope}%
\begin{pgfscope}%
\pgfpathrectangle{\pgfqpoint{3.985294in}{1.750000in}}{\pgfqpoint{2.279412in}{2.004545in}}%
\pgfusepath{clip}%
\pgfsetbuttcap%
\pgfsetroundjoin%
\pgfsetlinewidth{0.418061pt}%
\definecolor{currentstroke}{rgb}{0.282656,0.100196,0.422160}%
\pgfsetstrokecolor{currentstroke}%
\pgfsetdash{}{0pt}%
\pgfpathmoveto{\pgfqpoint{4.686534in}{2.726249in}}%
\pgfpathlineto{\pgfqpoint{4.686451in}{2.726262in}}%
\pgfusepath{stroke}%
\end{pgfscope}%
\begin{pgfscope}%
\pgfpathrectangle{\pgfqpoint{3.985294in}{1.750000in}}{\pgfqpoint{2.279412in}{2.004545in}}%
\pgfusepath{clip}%
\pgfsetbuttcap%
\pgfsetroundjoin%
\pgfsetlinewidth{0.417930pt}%
\definecolor{currentstroke}{rgb}{0.282656,0.100196,0.422160}%
\pgfsetstrokecolor{currentstroke}%
\pgfsetdash{}{0pt}%
\pgfpathmoveto{\pgfqpoint{4.686451in}{2.726262in}}%
\pgfpathlineto{\pgfqpoint{4.686446in}{2.726262in}}%
\pgfusepath{stroke}%
\end{pgfscope}%
\begin{pgfscope}%
\pgfpathrectangle{\pgfqpoint{3.985294in}{1.750000in}}{\pgfqpoint{2.279412in}{2.004545in}}%
\pgfusepath{clip}%
\pgfsetbuttcap%
\pgfsetroundjoin%
\pgfsetlinewidth{0.417920pt}%
\definecolor{currentstroke}{rgb}{0.282656,0.100196,0.422160}%
\pgfsetstrokecolor{currentstroke}%
\pgfsetdash{}{0pt}%
\pgfpathmoveto{\pgfqpoint{4.686446in}{2.726262in}}%
\pgfpathlineto{\pgfqpoint{4.686523in}{2.726247in}}%
\pgfusepath{stroke}%
\end{pgfscope}%
\begin{pgfscope}%
\pgfpathrectangle{\pgfqpoint{3.985294in}{1.750000in}}{\pgfqpoint{2.279412in}{2.004545in}}%
\pgfusepath{clip}%
\pgfsetbuttcap%
\pgfsetroundjoin%
\pgfsetlinewidth{0.418041pt}%
\definecolor{currentstroke}{rgb}{0.282656,0.100196,0.422160}%
\pgfsetstrokecolor{currentstroke}%
\pgfsetdash{}{0pt}%
\pgfpathmoveto{\pgfqpoint{4.686523in}{2.726247in}}%
\pgfpathlineto{\pgfqpoint{4.686610in}{2.726231in}}%
\pgfusepath{stroke}%
\end{pgfscope}%
\begin{pgfscope}%
\pgfpathrectangle{\pgfqpoint{3.985294in}{1.750000in}}{\pgfqpoint{2.279412in}{2.004545in}}%
\pgfusepath{clip}%
\pgfsetbuttcap%
\pgfsetroundjoin%
\pgfsetlinewidth{0.418176pt}%
\definecolor{currentstroke}{rgb}{0.282656,0.100196,0.422160}%
\pgfsetstrokecolor{currentstroke}%
\pgfsetdash{}{0pt}%
\pgfpathmoveto{\pgfqpoint{4.686610in}{2.726231in}}%
\pgfpathlineto{\pgfqpoint{4.686615in}{2.726229in}}%
\pgfusepath{stroke}%
\end{pgfscope}%
\begin{pgfscope}%
\pgfpathrectangle{\pgfqpoint{3.985294in}{1.750000in}}{\pgfqpoint{2.279412in}{2.004545in}}%
\pgfusepath{clip}%
\pgfsetbuttcap%
\pgfsetroundjoin%
\pgfsetlinewidth{0.418184pt}%
\definecolor{currentstroke}{rgb}{0.282656,0.100196,0.422160}%
\pgfsetstrokecolor{currentstroke}%
\pgfsetdash{}{0pt}%
\pgfpathmoveto{\pgfqpoint{4.686615in}{2.726229in}}%
\pgfpathlineto{\pgfqpoint{4.686535in}{2.726244in}}%
\pgfusepath{stroke}%
\end{pgfscope}%
\begin{pgfscope}%
\pgfpathrectangle{\pgfqpoint{3.985294in}{1.750000in}}{\pgfqpoint{2.279412in}{2.004545in}}%
\pgfusepath{clip}%
\pgfsetbuttcap%
\pgfsetroundjoin%
\pgfsetlinewidth{0.418058pt}%
\definecolor{currentstroke}{rgb}{0.282656,0.100196,0.422160}%
\pgfsetstrokecolor{currentstroke}%
\pgfsetdash{}{0pt}%
\pgfpathmoveto{\pgfqpoint{4.686535in}{2.726244in}}%
\pgfpathlineto{\pgfqpoint{4.686453in}{2.726259in}}%
\pgfusepath{stroke}%
\end{pgfscope}%
\begin{pgfscope}%
\pgfpathrectangle{\pgfqpoint{3.985294in}{1.750000in}}{\pgfqpoint{2.279412in}{2.004545in}}%
\pgfusepath{clip}%
\pgfsetbuttcap%
\pgfsetroundjoin%
\pgfsetlinewidth{0.417930pt}%
\definecolor{currentstroke}{rgb}{0.282656,0.100196,0.422160}%
\pgfsetstrokecolor{currentstroke}%
\pgfsetdash{}{0pt}%
\pgfpathmoveto{\pgfqpoint{4.686453in}{2.726259in}}%
\pgfpathlineto{\pgfqpoint{4.686448in}{2.726260in}}%
\pgfusepath{stroke}%
\end{pgfscope}%
\begin{pgfscope}%
\pgfpathrectangle{\pgfqpoint{3.985294in}{1.750000in}}{\pgfqpoint{2.279412in}{2.004545in}}%
\pgfusepath{clip}%
\pgfsetbuttcap%
\pgfsetroundjoin%
\pgfsetlinewidth{0.417923pt}%
\definecolor{currentstroke}{rgb}{0.282656,0.100196,0.422160}%
\pgfsetstrokecolor{currentstroke}%
\pgfsetdash{}{0pt}%
\pgfpathmoveto{\pgfqpoint{4.686448in}{2.726260in}}%
\pgfpathlineto{\pgfqpoint{4.686526in}{2.726246in}}%
\pgfusepath{stroke}%
\end{pgfscope}%
\begin{pgfscope}%
\pgfpathrectangle{\pgfqpoint{3.985294in}{1.750000in}}{\pgfqpoint{2.279412in}{2.004545in}}%
\pgfusepath{clip}%
\pgfsetbuttcap%
\pgfsetroundjoin%
\pgfsetlinewidth{0.418045pt}%
\definecolor{currentstroke}{rgb}{0.282656,0.100196,0.422160}%
\pgfsetstrokecolor{currentstroke}%
\pgfsetdash{}{0pt}%
\pgfpathmoveto{\pgfqpoint{4.686526in}{2.726246in}}%
\pgfpathlineto{\pgfqpoint{4.686610in}{2.726231in}}%
\pgfusepath{stroke}%
\end{pgfscope}%
\begin{pgfscope}%
\pgfpathrectangle{\pgfqpoint{3.985294in}{1.750000in}}{\pgfqpoint{2.279412in}{2.004545in}}%
\pgfusepath{clip}%
\pgfsetbuttcap%
\pgfsetroundjoin%
\pgfsetlinewidth{0.418176pt}%
\definecolor{currentstroke}{rgb}{0.282656,0.100196,0.422160}%
\pgfsetstrokecolor{currentstroke}%
\pgfsetdash{}{0pt}%
\pgfpathmoveto{\pgfqpoint{4.686610in}{2.726231in}}%
\pgfpathlineto{\pgfqpoint{4.686613in}{2.726230in}}%
\pgfusepath{stroke}%
\end{pgfscope}%
\begin{pgfscope}%
\pgfpathrectangle{\pgfqpoint{3.985294in}{1.750000in}}{\pgfqpoint{2.279412in}{2.004545in}}%
\pgfusepath{clip}%
\pgfsetbuttcap%
\pgfsetroundjoin%
\pgfsetlinewidth{0.418181pt}%
\definecolor{currentstroke}{rgb}{0.282656,0.100196,0.422160}%
\pgfsetstrokecolor{currentstroke}%
\pgfsetdash{}{0pt}%
\pgfpathmoveto{\pgfqpoint{4.686613in}{2.726230in}}%
\pgfpathlineto{\pgfqpoint{4.686533in}{2.726244in}}%
\pgfusepath{stroke}%
\end{pgfscope}%
\begin{pgfscope}%
\pgfpathrectangle{\pgfqpoint{3.985294in}{1.750000in}}{\pgfqpoint{2.279412in}{2.004545in}}%
\pgfusepath{clip}%
\pgfsetbuttcap%
\pgfsetroundjoin%
\pgfsetlinewidth{0.418055pt}%
\definecolor{currentstroke}{rgb}{0.282656,0.100196,0.422160}%
\pgfsetstrokecolor{currentstroke}%
\pgfsetdash{}{0pt}%
\pgfpathmoveto{\pgfqpoint{4.686533in}{2.726244in}}%
\pgfpathlineto{\pgfqpoint{4.686453in}{2.726259in}}%
\pgfusepath{stroke}%
\end{pgfscope}%
\begin{pgfscope}%
\pgfpathrectangle{\pgfqpoint{3.985294in}{1.750000in}}{\pgfqpoint{2.279412in}{2.004545in}}%
\pgfusepath{clip}%
\pgfsetbuttcap%
\pgfsetroundjoin%
\pgfsetlinewidth{0.417930pt}%
\definecolor{currentstroke}{rgb}{0.282656,0.100196,0.422160}%
\pgfsetstrokecolor{currentstroke}%
\pgfsetdash{}{0pt}%
\pgfpathmoveto{\pgfqpoint{4.686453in}{2.726259in}}%
\pgfpathlineto{\pgfqpoint{4.686450in}{2.726260in}}%
\pgfusepath{stroke}%
\end{pgfscope}%
\begin{pgfscope}%
\pgfpathrectangle{\pgfqpoint{3.985294in}{1.750000in}}{\pgfqpoint{2.279412in}{2.004545in}}%
\pgfusepath{clip}%
\pgfsetbuttcap%
\pgfsetroundjoin%
\pgfsetlinewidth{0.417926pt}%
\definecolor{currentstroke}{rgb}{0.282656,0.100196,0.422160}%
\pgfsetstrokecolor{currentstroke}%
\pgfsetdash{}{0pt}%
\pgfpathmoveto{\pgfqpoint{4.686450in}{2.726260in}}%
\pgfpathlineto{\pgfqpoint{4.686528in}{2.726246in}}%
\pgfusepath{stroke}%
\end{pgfscope}%
\begin{pgfscope}%
\pgfpathrectangle{\pgfqpoint{3.985294in}{1.750000in}}{\pgfqpoint{2.279412in}{2.004545in}}%
\pgfusepath{clip}%
\pgfsetbuttcap%
\pgfsetroundjoin%
\pgfsetlinewidth{0.418047pt}%
\definecolor{currentstroke}{rgb}{0.282656,0.100196,0.422160}%
\pgfsetstrokecolor{currentstroke}%
\pgfsetdash{}{0pt}%
\pgfpathmoveto{\pgfqpoint{4.686528in}{2.726246in}}%
\pgfpathlineto{\pgfqpoint{4.686610in}{2.726231in}}%
\pgfusepath{stroke}%
\end{pgfscope}%
\begin{pgfscope}%
\pgfpathrectangle{\pgfqpoint{3.985294in}{1.750000in}}{\pgfqpoint{2.279412in}{2.004545in}}%
\pgfusepath{clip}%
\pgfsetbuttcap%
\pgfsetroundjoin%
\pgfsetlinewidth{0.418176pt}%
\definecolor{currentstroke}{rgb}{0.282656,0.100196,0.422160}%
\pgfsetstrokecolor{currentstroke}%
\pgfsetdash{}{0pt}%
\pgfpathmoveto{\pgfqpoint{4.686610in}{2.726231in}}%
\pgfpathlineto{\pgfqpoint{4.686611in}{2.726230in}}%
\pgfusepath{stroke}%
\end{pgfscope}%
\begin{pgfscope}%
\pgfpathrectangle{\pgfqpoint{3.985294in}{1.750000in}}{\pgfqpoint{2.279412in}{2.004545in}}%
\pgfusepath{clip}%
\pgfsetbuttcap%
\pgfsetroundjoin%
\pgfsetlinewidth{0.418177pt}%
\definecolor{currentstroke}{rgb}{0.282656,0.100196,0.422160}%
\pgfsetstrokecolor{currentstroke}%
\pgfsetdash{}{0pt}%
\pgfpathmoveto{\pgfqpoint{4.686611in}{2.726230in}}%
\pgfpathlineto{\pgfqpoint{4.686531in}{2.726245in}}%
\pgfusepath{stroke}%
\end{pgfscope}%
\begin{pgfscope}%
\pgfpathrectangle{\pgfqpoint{3.985294in}{1.750000in}}{\pgfqpoint{2.279412in}{2.004545in}}%
\pgfusepath{clip}%
\pgfsetbuttcap%
\pgfsetroundjoin%
\pgfsetlinewidth{0.418052pt}%
\definecolor{currentstroke}{rgb}{0.282656,0.100196,0.422160}%
\pgfsetstrokecolor{currentstroke}%
\pgfsetdash{}{0pt}%
\pgfpathmoveto{\pgfqpoint{4.686531in}{2.726245in}}%
\pgfpathlineto{\pgfqpoint{4.686453in}{2.726259in}}%
\pgfusepath{stroke}%
\end{pgfscope}%
\begin{pgfscope}%
\pgfpathrectangle{\pgfqpoint{3.985294in}{1.750000in}}{\pgfqpoint{2.279412in}{2.004545in}}%
\pgfusepath{clip}%
\pgfsetbuttcap%
\pgfsetroundjoin%
\pgfsetlinewidth{0.417930pt}%
\definecolor{currentstroke}{rgb}{0.282656,0.100196,0.422160}%
\pgfsetstrokecolor{currentstroke}%
\pgfsetdash{}{0pt}%
\pgfpathmoveto{\pgfqpoint{4.686453in}{2.726259in}}%
\pgfpathlineto{\pgfqpoint{4.686452in}{2.726260in}}%
\pgfusepath{stroke}%
\end{pgfscope}%
\begin{pgfscope}%
\pgfpathrectangle{\pgfqpoint{3.985294in}{1.750000in}}{\pgfqpoint{2.279412in}{2.004545in}}%
\pgfusepath{clip}%
\pgfsetbuttcap%
\pgfsetroundjoin%
\pgfsetlinewidth{0.417929pt}%
\definecolor{currentstroke}{rgb}{0.282656,0.100196,0.422160}%
\pgfsetstrokecolor{currentstroke}%
\pgfsetdash{}{0pt}%
\pgfpathmoveto{\pgfqpoint{4.686452in}{2.726260in}}%
\pgfpathlineto{\pgfqpoint{4.686529in}{2.726246in}}%
\pgfusepath{stroke}%
\end{pgfscope}%
\begin{pgfscope}%
\pgfpathrectangle{\pgfqpoint{3.985294in}{1.750000in}}{\pgfqpoint{2.279412in}{2.004545in}}%
\pgfusepath{clip}%
\pgfsetbuttcap%
\pgfsetroundjoin%
\pgfsetlinewidth{0.418050pt}%
\definecolor{currentstroke}{rgb}{0.282656,0.100196,0.422160}%
\pgfsetstrokecolor{currentstroke}%
\pgfsetdash{}{0pt}%
\pgfpathmoveto{\pgfqpoint{4.686529in}{2.726246in}}%
\pgfpathlineto{\pgfqpoint{4.686610in}{2.726231in}}%
\pgfusepath{stroke}%
\end{pgfscope}%
\begin{pgfscope}%
\pgfpathrectangle{\pgfqpoint{3.985294in}{1.750000in}}{\pgfqpoint{2.279412in}{2.004545in}}%
\pgfusepath{clip}%
\pgfsetbuttcap%
\pgfsetroundjoin%
\pgfsetlinewidth{0.418175pt}%
\definecolor{currentstroke}{rgb}{0.282656,0.100196,0.422160}%
\pgfsetstrokecolor{currentstroke}%
\pgfsetdash{}{0pt}%
\pgfpathmoveto{\pgfqpoint{4.686610in}{2.726231in}}%
\pgfpathlineto{\pgfqpoint{4.686609in}{2.726230in}}%
\pgfusepath{stroke}%
\end{pgfscope}%
\begin{pgfscope}%
\pgfpathrectangle{\pgfqpoint{3.985294in}{1.750000in}}{\pgfqpoint{2.279412in}{2.004545in}}%
\pgfusepath{clip}%
\pgfsetbuttcap%
\pgfsetroundjoin%
\pgfsetlinewidth{0.418174pt}%
\definecolor{currentstroke}{rgb}{0.282656,0.100196,0.422160}%
\pgfsetstrokecolor{currentstroke}%
\pgfsetdash{}{0pt}%
\pgfpathmoveto{\pgfqpoint{4.686609in}{2.726230in}}%
\pgfpathlineto{\pgfqpoint{4.686529in}{2.726245in}}%
\pgfusepath{stroke}%
\end{pgfscope}%
\begin{pgfscope}%
\pgfpathrectangle{\pgfqpoint{3.985294in}{1.750000in}}{\pgfqpoint{2.279412in}{2.004545in}}%
\pgfusepath{clip}%
\pgfsetbuttcap%
\pgfsetroundjoin%
\pgfsetlinewidth{0.418049pt}%
\definecolor{currentstroke}{rgb}{0.282656,0.100196,0.422160}%
\pgfsetstrokecolor{currentstroke}%
\pgfsetdash{}{0pt}%
\pgfpathmoveto{\pgfqpoint{4.686529in}{2.726245in}}%
\pgfpathlineto{\pgfqpoint{4.686453in}{2.726259in}}%
\pgfusepath{stroke}%
\end{pgfscope}%
\begin{pgfscope}%
\pgfpathrectangle{\pgfqpoint{3.985294in}{1.750000in}}{\pgfqpoint{2.279412in}{2.004545in}}%
\pgfusepath{clip}%
\pgfsetbuttcap%
\pgfsetroundjoin%
\pgfsetlinewidth{0.417930pt}%
\definecolor{currentstroke}{rgb}{0.282656,0.100196,0.422160}%
\pgfsetstrokecolor{currentstroke}%
\pgfsetdash{}{0pt}%
\pgfpathmoveto{\pgfqpoint{4.686453in}{2.726259in}}%
\pgfpathlineto{\pgfqpoint{4.686454in}{2.726259in}}%
\pgfusepath{stroke}%
\end{pgfscope}%
\begin{pgfscope}%
\pgfpathrectangle{\pgfqpoint{3.985294in}{1.750000in}}{\pgfqpoint{2.279412in}{2.004545in}}%
\pgfusepath{clip}%
\pgfsetbuttcap%
\pgfsetroundjoin%
\pgfsetlinewidth{0.417932pt}%
\definecolor{currentstroke}{rgb}{0.282656,0.100196,0.422160}%
\pgfsetstrokecolor{currentstroke}%
\pgfsetdash{}{0pt}%
\pgfpathmoveto{\pgfqpoint{4.686454in}{2.726259in}}%
\pgfpathlineto{\pgfqpoint{4.686531in}{2.726245in}}%
\pgfusepath{stroke}%
\end{pgfscope}%
\begin{pgfscope}%
\pgfpathrectangle{\pgfqpoint{3.985294in}{1.750000in}}{\pgfqpoint{2.279412in}{2.004545in}}%
\pgfusepath{clip}%
\pgfsetbuttcap%
\pgfsetroundjoin%
\pgfsetlinewidth{0.418053pt}%
\definecolor{currentstroke}{rgb}{0.282656,0.100196,0.422160}%
\pgfsetstrokecolor{currentstroke}%
\pgfsetdash{}{0pt}%
\pgfpathmoveto{\pgfqpoint{4.686531in}{2.726245in}}%
\pgfpathlineto{\pgfqpoint{4.686610in}{2.726231in}}%
\pgfusepath{stroke}%
\end{pgfscope}%
\begin{pgfscope}%
\pgfpathrectangle{\pgfqpoint{3.985294in}{1.750000in}}{\pgfqpoint{2.279412in}{2.004545in}}%
\pgfusepath{clip}%
\pgfsetbuttcap%
\pgfsetroundjoin%
\pgfsetlinewidth{0.418175pt}%
\definecolor{currentstroke}{rgb}{0.282656,0.100196,0.422160}%
\pgfsetstrokecolor{currentstroke}%
\pgfsetdash{}{0pt}%
\pgfpathmoveto{\pgfqpoint{4.686610in}{2.726231in}}%
\pgfpathlineto{\pgfqpoint{4.686607in}{2.726231in}}%
\pgfusepath{stroke}%
\end{pgfscope}%
\begin{pgfscope}%
\pgfpathrectangle{\pgfqpoint{3.985294in}{1.750000in}}{\pgfqpoint{2.279412in}{2.004545in}}%
\pgfusepath{clip}%
\pgfsetbuttcap%
\pgfsetroundjoin%
\pgfsetlinewidth{0.418171pt}%
\definecolor{currentstroke}{rgb}{0.282656,0.100196,0.422160}%
\pgfsetstrokecolor{currentstroke}%
\pgfsetdash{}{0pt}%
\pgfpathmoveto{\pgfqpoint{4.686607in}{2.726231in}}%
\pgfpathlineto{\pgfqpoint{4.686527in}{2.726245in}}%
\pgfusepath{stroke}%
\end{pgfscope}%
\begin{pgfscope}%
\pgfpathrectangle{\pgfqpoint{3.985294in}{1.750000in}}{\pgfqpoint{2.279412in}{2.004545in}}%
\pgfusepath{clip}%
\pgfsetbuttcap%
\pgfsetroundjoin%
\pgfsetlinewidth{0.418047pt}%
\definecolor{currentstroke}{rgb}{0.282656,0.100196,0.422160}%
\pgfsetstrokecolor{currentstroke}%
\pgfsetdash{}{0pt}%
\pgfpathmoveto{\pgfqpoint{4.686527in}{2.726245in}}%
\pgfpathlineto{\pgfqpoint{4.686453in}{2.726259in}}%
\pgfusepath{stroke}%
\end{pgfscope}%
\begin{pgfscope}%
\pgfpathrectangle{\pgfqpoint{3.985294in}{1.750000in}}{\pgfqpoint{2.279412in}{2.004545in}}%
\pgfusepath{clip}%
\pgfsetbuttcap%
\pgfsetroundjoin%
\pgfsetlinewidth{0.417930pt}%
\definecolor{currentstroke}{rgb}{0.282656,0.100196,0.422160}%
\pgfsetstrokecolor{currentstroke}%
\pgfsetdash{}{0pt}%
\pgfpathmoveto{\pgfqpoint{4.686453in}{2.726259in}}%
\pgfpathlineto{\pgfqpoint{4.686456in}{2.726259in}}%
\pgfusepath{stroke}%
\end{pgfscope}%
\begin{pgfscope}%
\pgfpathrectangle{\pgfqpoint{3.985294in}{1.750000in}}{\pgfqpoint{2.279412in}{2.004545in}}%
\pgfusepath{clip}%
\pgfsetbuttcap%
\pgfsetroundjoin%
\pgfsetlinewidth{0.417935pt}%
\definecolor{currentstroke}{rgb}{0.282656,0.100196,0.422160}%
\pgfsetstrokecolor{currentstroke}%
\pgfsetdash{}{0pt}%
\pgfpathmoveto{\pgfqpoint{4.686456in}{2.726259in}}%
\pgfpathlineto{\pgfqpoint{4.686533in}{2.726245in}}%
\pgfusepath{stroke}%
\end{pgfscope}%
\begin{pgfscope}%
\pgfpathrectangle{\pgfqpoint{3.985294in}{1.750000in}}{\pgfqpoint{2.279412in}{2.004545in}}%
\pgfusepath{clip}%
\pgfsetbuttcap%
\pgfsetroundjoin%
\pgfsetlinewidth{0.418056pt}%
\definecolor{currentstroke}{rgb}{0.282656,0.100196,0.422160}%
\pgfsetstrokecolor{currentstroke}%
\pgfsetdash{}{0pt}%
\pgfpathmoveto{\pgfqpoint{4.686533in}{2.726245in}}%
\pgfpathlineto{\pgfqpoint{4.686609in}{2.726231in}}%
\pgfusepath{stroke}%
\end{pgfscope}%
\begin{pgfscope}%
\pgfpathrectangle{\pgfqpoint{3.985294in}{1.750000in}}{\pgfqpoint{2.279412in}{2.004545in}}%
\pgfusepath{clip}%
\pgfsetbuttcap%
\pgfsetroundjoin%
\pgfsetlinewidth{0.418175pt}%
\definecolor{currentstroke}{rgb}{0.282656,0.100196,0.422160}%
\pgfsetstrokecolor{currentstroke}%
\pgfsetdash{}{0pt}%
\pgfpathmoveto{\pgfqpoint{4.686609in}{2.726231in}}%
\pgfpathlineto{\pgfqpoint{4.686605in}{2.726231in}}%
\pgfusepath{stroke}%
\end{pgfscope}%
\begin{pgfscope}%
\pgfpathrectangle{\pgfqpoint{3.985294in}{1.750000in}}{\pgfqpoint{2.279412in}{2.004545in}}%
\pgfusepath{clip}%
\pgfsetbuttcap%
\pgfsetroundjoin%
\pgfsetlinewidth{0.418168pt}%
\definecolor{currentstroke}{rgb}{0.282656,0.100196,0.422160}%
\pgfsetstrokecolor{currentstroke}%
\pgfsetdash{}{0pt}%
\pgfpathmoveto{\pgfqpoint{4.686605in}{2.726231in}}%
\pgfpathlineto{\pgfqpoint{4.686526in}{2.726246in}}%
\pgfusepath{stroke}%
\end{pgfscope}%
\begin{pgfscope}%
\pgfpathrectangle{\pgfqpoint{3.985294in}{1.750000in}}{\pgfqpoint{2.279412in}{2.004545in}}%
\pgfusepath{clip}%
\pgfsetbuttcap%
\pgfsetroundjoin%
\pgfsetlinewidth{0.418044pt}%
\definecolor{currentstroke}{rgb}{0.282656,0.100196,0.422160}%
\pgfsetstrokecolor{currentstroke}%
\pgfsetdash{}{0pt}%
\pgfpathmoveto{\pgfqpoint{4.686526in}{2.726246in}}%
\pgfpathlineto{\pgfqpoint{4.686453in}{2.726259in}}%
\pgfusepath{stroke}%
\end{pgfscope}%
\begin{pgfscope}%
\pgfpathrectangle{\pgfqpoint{3.985294in}{1.750000in}}{\pgfqpoint{2.279412in}{2.004545in}}%
\pgfusepath{clip}%
\pgfsetbuttcap%
\pgfsetroundjoin%
\pgfsetlinewidth{0.417931pt}%
\definecolor{currentstroke}{rgb}{0.282656,0.100196,0.422160}%
\pgfsetstrokecolor{currentstroke}%
\pgfsetdash{}{0pt}%
\pgfpathmoveto{\pgfqpoint{4.686453in}{2.726259in}}%
\pgfpathlineto{\pgfqpoint{4.686458in}{2.726259in}}%
\pgfusepath{stroke}%
\end{pgfscope}%
\begin{pgfscope}%
\pgfpathrectangle{\pgfqpoint{3.985294in}{1.750000in}}{\pgfqpoint{2.279412in}{2.004545in}}%
\pgfusepath{clip}%
\pgfsetbuttcap%
\pgfsetroundjoin%
\pgfsetlinewidth{0.417938pt}%
\definecolor{currentstroke}{rgb}{0.282656,0.100196,0.422160}%
\pgfsetstrokecolor{currentstroke}%
\pgfsetdash{}{0pt}%
\pgfpathmoveto{\pgfqpoint{4.686458in}{2.726259in}}%
\pgfpathlineto{\pgfqpoint{4.686535in}{2.726245in}}%
\pgfusepath{stroke}%
\end{pgfscope}%
\begin{pgfscope}%
\pgfpathrectangle{\pgfqpoint{3.985294in}{1.750000in}}{\pgfqpoint{2.279412in}{2.004545in}}%
\pgfusepath{clip}%
\pgfsetbuttcap%
\pgfsetroundjoin%
\pgfsetlinewidth{0.418059pt}%
\definecolor{currentstroke}{rgb}{0.282656,0.100196,0.422160}%
\pgfsetstrokecolor{currentstroke}%
\pgfsetdash{}{0pt}%
\pgfpathmoveto{\pgfqpoint{4.686535in}{2.726245in}}%
\pgfpathlineto{\pgfqpoint{4.686609in}{2.726231in}}%
\pgfusepath{stroke}%
\end{pgfscope}%
\begin{pgfscope}%
\pgfpathrectangle{\pgfqpoint{3.985294in}{1.750000in}}{\pgfqpoint{2.279412in}{2.004545in}}%
\pgfusepath{clip}%
\pgfsetbuttcap%
\pgfsetroundjoin%
\pgfsetlinewidth{0.418174pt}%
\definecolor{currentstroke}{rgb}{0.282656,0.100196,0.422160}%
\pgfsetstrokecolor{currentstroke}%
\pgfsetdash{}{0pt}%
\pgfpathmoveto{\pgfqpoint{4.686609in}{2.726231in}}%
\pgfpathlineto{\pgfqpoint{4.686603in}{2.726232in}}%
\pgfusepath{stroke}%
\end{pgfscope}%
\begin{pgfscope}%
\pgfpathrectangle{\pgfqpoint{3.985294in}{1.750000in}}{\pgfqpoint{2.279412in}{2.004545in}}%
\pgfusepath{clip}%
\pgfsetbuttcap%
\pgfsetroundjoin%
\pgfsetlinewidth{0.418165pt}%
\definecolor{currentstroke}{rgb}{0.282656,0.100196,0.422160}%
\pgfsetstrokecolor{currentstroke}%
\pgfsetdash{}{0pt}%
\pgfpathmoveto{\pgfqpoint{4.686603in}{2.726232in}}%
\pgfpathlineto{\pgfqpoint{4.686524in}{2.726246in}}%
\pgfusepath{stroke}%
\end{pgfscope}%
\begin{pgfscope}%
\pgfpathrectangle{\pgfqpoint{3.985294in}{1.750000in}}{\pgfqpoint{2.279412in}{2.004545in}}%
\pgfusepath{clip}%
\pgfsetbuttcap%
\pgfsetroundjoin%
\pgfsetlinewidth{0.418041pt}%
\definecolor{currentstroke}{rgb}{0.282656,0.100196,0.422160}%
\pgfsetstrokecolor{currentstroke}%
\pgfsetdash{}{0pt}%
\pgfpathmoveto{\pgfqpoint{4.686524in}{2.726246in}}%
\pgfpathlineto{\pgfqpoint{4.686453in}{2.726259in}}%
\pgfusepath{stroke}%
\end{pgfscope}%
\begin{pgfscope}%
\pgfpathrectangle{\pgfqpoint{3.985294in}{1.750000in}}{\pgfqpoint{2.279412in}{2.004545in}}%
\pgfusepath{clip}%
\pgfsetbuttcap%
\pgfsetroundjoin%
\pgfsetlinewidth{0.417931pt}%
\definecolor{currentstroke}{rgb}{0.282656,0.100196,0.422160}%
\pgfsetstrokecolor{currentstroke}%
\pgfsetdash{}{0pt}%
\pgfpathmoveto{\pgfqpoint{4.686453in}{2.726259in}}%
\pgfpathlineto{\pgfqpoint{4.686459in}{2.726258in}}%
\pgfusepath{stroke}%
\end{pgfscope}%
\begin{pgfscope}%
\pgfpathrectangle{\pgfqpoint{3.985294in}{1.750000in}}{\pgfqpoint{2.279412in}{2.004545in}}%
\pgfusepath{clip}%
\pgfsetbuttcap%
\pgfsetroundjoin%
\pgfsetlinewidth{0.417941pt}%
\definecolor{currentstroke}{rgb}{0.282656,0.100196,0.422160}%
\pgfsetstrokecolor{currentstroke}%
\pgfsetdash{}{0pt}%
\pgfpathmoveto{\pgfqpoint{4.686459in}{2.726258in}}%
\pgfpathlineto{\pgfqpoint{4.686536in}{2.726244in}}%
\pgfusepath{stroke}%
\end{pgfscope}%
\begin{pgfscope}%
\pgfpathrectangle{\pgfqpoint{3.985294in}{1.750000in}}{\pgfqpoint{2.279412in}{2.004545in}}%
\pgfusepath{clip}%
\pgfsetbuttcap%
\pgfsetroundjoin%
\pgfsetlinewidth{0.418061pt}%
\definecolor{currentstroke}{rgb}{0.282656,0.100196,0.422160}%
\pgfsetstrokecolor{currentstroke}%
\pgfsetdash{}{0pt}%
\pgfpathmoveto{\pgfqpoint{4.686536in}{2.726244in}}%
\pgfpathlineto{\pgfqpoint{4.686609in}{2.726231in}}%
\pgfusepath{stroke}%
\end{pgfscope}%
\begin{pgfscope}%
\pgfpathrectangle{\pgfqpoint{3.985294in}{1.750000in}}{\pgfqpoint{2.279412in}{2.004545in}}%
\pgfusepath{clip}%
\pgfsetbuttcap%
\pgfsetroundjoin%
\pgfsetlinewidth{0.418174pt}%
\definecolor{currentstroke}{rgb}{0.282656,0.100196,0.422160}%
\pgfsetstrokecolor{currentstroke}%
\pgfsetdash{}{0pt}%
\pgfpathmoveto{\pgfqpoint{4.686609in}{2.726231in}}%
\pgfpathlineto{\pgfqpoint{4.686601in}{2.726232in}}%
\pgfusepath{stroke}%
\end{pgfscope}%
\begin{pgfscope}%
\pgfpathrectangle{\pgfqpoint{3.985294in}{1.750000in}}{\pgfqpoint{2.279412in}{2.004545in}}%
\pgfusepath{clip}%
\pgfsetbuttcap%
\pgfsetroundjoin%
\pgfsetlinewidth{0.418162pt}%
\definecolor{currentstroke}{rgb}{0.282656,0.100196,0.422160}%
\pgfsetstrokecolor{currentstroke}%
\pgfsetdash{}{0pt}%
\pgfpathmoveto{\pgfqpoint{4.686601in}{2.726232in}}%
\pgfpathlineto{\pgfqpoint{4.686522in}{2.726246in}}%
\pgfusepath{stroke}%
\end{pgfscope}%
\begin{pgfscope}%
\pgfpathrectangle{\pgfqpoint{3.985294in}{1.750000in}}{\pgfqpoint{2.279412in}{2.004545in}}%
\pgfusepath{clip}%
\pgfsetbuttcap%
\pgfsetroundjoin%
\pgfsetlinewidth{0.418039pt}%
\definecolor{currentstroke}{rgb}{0.282656,0.100196,0.422160}%
\pgfsetstrokecolor{currentstroke}%
\pgfsetdash{}{0pt}%
\pgfpathmoveto{\pgfqpoint{4.686522in}{2.726246in}}%
\pgfpathlineto{\pgfqpoint{4.686454in}{2.726259in}}%
\pgfusepath{stroke}%
\end{pgfscope}%
\begin{pgfscope}%
\pgfpathrectangle{\pgfqpoint{3.985294in}{1.750000in}}{\pgfqpoint{2.279412in}{2.004545in}}%
\pgfusepath{clip}%
\pgfsetbuttcap%
\pgfsetroundjoin%
\pgfsetlinewidth{0.417931pt}%
\definecolor{currentstroke}{rgb}{0.282656,0.100196,0.422160}%
\pgfsetstrokecolor{currentstroke}%
\pgfsetdash{}{0pt}%
\pgfpathmoveto{\pgfqpoint{4.686454in}{2.726259in}}%
\pgfpathlineto{\pgfqpoint{4.686461in}{2.726258in}}%
\pgfusepath{stroke}%
\end{pgfscope}%
\begin{pgfscope}%
\pgfpathrectangle{\pgfqpoint{3.985294in}{1.750000in}}{\pgfqpoint{2.279412in}{2.004545in}}%
\pgfusepath{clip}%
\pgfsetbuttcap%
\pgfsetroundjoin%
\pgfsetlinewidth{0.417944pt}%
\definecolor{currentstroke}{rgb}{0.282656,0.100196,0.422160}%
\pgfsetstrokecolor{currentstroke}%
\pgfsetdash{}{0pt}%
\pgfpathmoveto{\pgfqpoint{4.686461in}{2.726258in}}%
\pgfpathlineto{\pgfqpoint{4.686538in}{2.726244in}}%
\pgfusepath{stroke}%
\end{pgfscope}%
\begin{pgfscope}%
\pgfpathrectangle{\pgfqpoint{3.985294in}{1.750000in}}{\pgfqpoint{2.279412in}{2.004545in}}%
\pgfusepath{clip}%
\pgfsetbuttcap%
\pgfsetroundjoin%
\pgfsetlinewidth{0.418064pt}%
\definecolor{currentstroke}{rgb}{0.282656,0.100196,0.422160}%
\pgfsetstrokecolor{currentstroke}%
\pgfsetdash{}{0pt}%
\pgfpathmoveto{\pgfqpoint{4.686538in}{2.726244in}}%
\pgfpathlineto{\pgfqpoint{4.686608in}{2.726231in}}%
\pgfusepath{stroke}%
\end{pgfscope}%
\begin{pgfscope}%
\pgfpathrectangle{\pgfqpoint{3.985294in}{1.750000in}}{\pgfqpoint{2.279412in}{2.004545in}}%
\pgfusepath{clip}%
\pgfsetbuttcap%
\pgfsetroundjoin%
\pgfsetlinewidth{0.418173pt}%
\definecolor{currentstroke}{rgb}{0.282656,0.100196,0.422160}%
\pgfsetstrokecolor{currentstroke}%
\pgfsetdash{}{0pt}%
\pgfpathmoveto{\pgfqpoint{4.686608in}{2.726231in}}%
\pgfpathlineto{\pgfqpoint{4.686599in}{2.726232in}}%
\pgfusepath{stroke}%
\end{pgfscope}%
\begin{pgfscope}%
\pgfpathrectangle{\pgfqpoint{3.985294in}{1.750000in}}{\pgfqpoint{2.279412in}{2.004545in}}%
\pgfusepath{clip}%
\pgfsetbuttcap%
\pgfsetroundjoin%
\pgfsetlinewidth{0.418159pt}%
\definecolor{currentstroke}{rgb}{0.282656,0.100196,0.422160}%
\pgfsetstrokecolor{currentstroke}%
\pgfsetdash{}{0pt}%
\pgfpathmoveto{\pgfqpoint{4.686599in}{2.726232in}}%
\pgfpathlineto{\pgfqpoint{4.686521in}{2.726247in}}%
\pgfusepath{stroke}%
\end{pgfscope}%
\begin{pgfscope}%
\pgfpathrectangle{\pgfqpoint{3.985294in}{1.750000in}}{\pgfqpoint{2.279412in}{2.004545in}}%
\pgfusepath{clip}%
\pgfsetbuttcap%
\pgfsetroundjoin%
\pgfsetlinewidth{0.418036pt}%
\definecolor{currentstroke}{rgb}{0.282656,0.100196,0.422160}%
\pgfsetstrokecolor{currentstroke}%
\pgfsetdash{}{0pt}%
\pgfpathmoveto{\pgfqpoint{4.686521in}{2.726247in}}%
\pgfpathlineto{\pgfqpoint{4.686454in}{2.726259in}}%
\pgfusepath{stroke}%
\end{pgfscope}%
\begin{pgfscope}%
\pgfpathrectangle{\pgfqpoint{3.985294in}{1.750000in}}{\pgfqpoint{2.279412in}{2.004545in}}%
\pgfusepath{clip}%
\pgfsetbuttcap%
\pgfsetroundjoin%
\pgfsetlinewidth{0.417932pt}%
\definecolor{currentstroke}{rgb}{0.282656,0.100196,0.422160}%
\pgfsetstrokecolor{currentstroke}%
\pgfsetdash{}{0pt}%
\pgfpathmoveto{\pgfqpoint{4.686454in}{2.726259in}}%
\pgfpathlineto{\pgfqpoint{4.686463in}{2.726258in}}%
\pgfusepath{stroke}%
\end{pgfscope}%
\begin{pgfscope}%
\pgfpathrectangle{\pgfqpoint{3.985294in}{1.750000in}}{\pgfqpoint{2.279412in}{2.004545in}}%
\pgfusepath{clip}%
\pgfsetbuttcap%
\pgfsetroundjoin%
\pgfsetlinewidth{0.417946pt}%
\definecolor{currentstroke}{rgb}{0.282656,0.100196,0.422160}%
\pgfsetstrokecolor{currentstroke}%
\pgfsetdash{}{0pt}%
\pgfpathmoveto{\pgfqpoint{4.686463in}{2.726258in}}%
\pgfpathlineto{\pgfqpoint{4.686540in}{2.726244in}}%
\pgfusepath{stroke}%
\end{pgfscope}%
\begin{pgfscope}%
\pgfpathrectangle{\pgfqpoint{3.985294in}{1.750000in}}{\pgfqpoint{2.279412in}{2.004545in}}%
\pgfusepath{clip}%
\pgfsetbuttcap%
\pgfsetroundjoin%
\pgfsetlinewidth{0.418066pt}%
\definecolor{currentstroke}{rgb}{0.282656,0.100196,0.422160}%
\pgfsetstrokecolor{currentstroke}%
\pgfsetdash{}{0pt}%
\pgfpathmoveto{\pgfqpoint{4.686540in}{2.726244in}}%
\pgfpathlineto{\pgfqpoint{4.686608in}{2.726231in}}%
\pgfusepath{stroke}%
\end{pgfscope}%
\begin{pgfscope}%
\pgfpathrectangle{\pgfqpoint{3.985294in}{1.750000in}}{\pgfqpoint{2.279412in}{2.004545in}}%
\pgfusepath{clip}%
\pgfsetbuttcap%
\pgfsetroundjoin%
\pgfsetlinewidth{0.418173pt}%
\definecolor{currentstroke}{rgb}{0.282656,0.100196,0.422160}%
\pgfsetstrokecolor{currentstroke}%
\pgfsetdash{}{0pt}%
\pgfpathmoveto{\pgfqpoint{4.686608in}{2.726231in}}%
\pgfpathlineto{\pgfqpoint{4.686597in}{2.726233in}}%
\pgfusepath{stroke}%
\end{pgfscope}%
\begin{pgfscope}%
\pgfpathrectangle{\pgfqpoint{3.985294in}{1.750000in}}{\pgfqpoint{2.279412in}{2.004545in}}%
\pgfusepath{clip}%
\pgfsetbuttcap%
\pgfsetroundjoin%
\pgfsetlinewidth{0.418155pt}%
\definecolor{currentstroke}{rgb}{0.282656,0.100196,0.422160}%
\pgfsetstrokecolor{currentstroke}%
\pgfsetdash{}{0pt}%
\pgfpathmoveto{\pgfqpoint{4.686597in}{2.726233in}}%
\pgfpathlineto{\pgfqpoint{4.686519in}{2.726247in}}%
\pgfusepath{stroke}%
\end{pgfscope}%
\begin{pgfscope}%
\pgfpathrectangle{\pgfqpoint{3.985294in}{1.750000in}}{\pgfqpoint{2.279412in}{2.004545in}}%
\pgfusepath{clip}%
\pgfsetbuttcap%
\pgfsetroundjoin%
\pgfsetlinewidth{0.418034pt}%
\definecolor{currentstroke}{rgb}{0.282656,0.100196,0.422160}%
\pgfsetstrokecolor{currentstroke}%
\pgfsetdash{}{0pt}%
\pgfpathmoveto{\pgfqpoint{4.686519in}{2.726247in}}%
\pgfpathlineto{\pgfqpoint{4.686454in}{2.726259in}}%
\pgfusepath{stroke}%
\end{pgfscope}%
\begin{pgfscope}%
\pgfpathrectangle{\pgfqpoint{3.985294in}{1.750000in}}{\pgfqpoint{2.279412in}{2.004545in}}%
\pgfusepath{clip}%
\pgfsetbuttcap%
\pgfsetroundjoin%
\pgfsetlinewidth{0.417932pt}%
\definecolor{currentstroke}{rgb}{0.282656,0.100196,0.422160}%
\pgfsetstrokecolor{currentstroke}%
\pgfsetdash{}{0pt}%
\pgfpathmoveto{\pgfqpoint{4.686454in}{2.726259in}}%
\pgfpathlineto{\pgfqpoint{4.686465in}{2.726257in}}%
\pgfusepath{stroke}%
\end{pgfscope}%
\begin{pgfscope}%
\pgfpathrectangle{\pgfqpoint{3.985294in}{1.750000in}}{\pgfqpoint{2.279412in}{2.004545in}}%
\pgfusepath{clip}%
\pgfsetbuttcap%
\pgfsetroundjoin%
\pgfsetlinewidth{0.417949pt}%
\definecolor{currentstroke}{rgb}{0.282656,0.100196,0.422160}%
\pgfsetstrokecolor{currentstroke}%
\pgfsetdash{}{0pt}%
\pgfpathmoveto{\pgfqpoint{4.686465in}{2.726257in}}%
\pgfpathlineto{\pgfqpoint{4.686541in}{2.726243in}}%
\pgfusepath{stroke}%
\end{pgfscope}%
\begin{pgfscope}%
\pgfpathrectangle{\pgfqpoint{3.985294in}{1.750000in}}{\pgfqpoint{2.279412in}{2.004545in}}%
\pgfusepath{clip}%
\pgfsetbuttcap%
\pgfsetroundjoin%
\pgfsetlinewidth{0.418069pt}%
\definecolor{currentstroke}{rgb}{0.282656,0.100196,0.422160}%
\pgfsetstrokecolor{currentstroke}%
\pgfsetdash{}{0pt}%
\pgfpathmoveto{\pgfqpoint{4.686541in}{2.726243in}}%
\pgfpathlineto{\pgfqpoint{4.686608in}{2.726231in}}%
\pgfusepath{stroke}%
\end{pgfscope}%
\begin{pgfscope}%
\pgfpathrectangle{\pgfqpoint{3.985294in}{1.750000in}}{\pgfqpoint{2.279412in}{2.004545in}}%
\pgfusepath{clip}%
\pgfsetbuttcap%
\pgfsetroundjoin%
\pgfsetlinewidth{0.418172pt}%
\definecolor{currentstroke}{rgb}{0.282656,0.100196,0.422160}%
\pgfsetstrokecolor{currentstroke}%
\pgfsetdash{}{0pt}%
\pgfpathmoveto{\pgfqpoint{4.686608in}{2.726231in}}%
\pgfpathlineto{\pgfqpoint{4.686595in}{2.726233in}}%
\pgfusepath{stroke}%
\end{pgfscope}%
\begin{pgfscope}%
\pgfpathrectangle{\pgfqpoint{3.985294in}{1.750000in}}{\pgfqpoint{2.279412in}{2.004545in}}%
\pgfusepath{clip}%
\pgfsetbuttcap%
\pgfsetroundjoin%
\pgfsetlinewidth{0.418152pt}%
\definecolor{currentstroke}{rgb}{0.282656,0.100196,0.422160}%
\pgfsetstrokecolor{currentstroke}%
\pgfsetdash{}{0pt}%
\pgfpathmoveto{\pgfqpoint{4.686595in}{2.726233in}}%
\pgfpathlineto{\pgfqpoint{4.686518in}{2.726247in}}%
\pgfusepath{stroke}%
\end{pgfscope}%
\begin{pgfscope}%
\pgfpathrectangle{\pgfqpoint{3.985294in}{1.750000in}}{\pgfqpoint{2.279412in}{2.004545in}}%
\pgfusepath{clip}%
\pgfsetbuttcap%
\pgfsetroundjoin%
\pgfsetlinewidth{0.418031pt}%
\definecolor{currentstroke}{rgb}{0.282656,0.100196,0.422160}%
\pgfsetstrokecolor{currentstroke}%
\pgfsetdash{}{0pt}%
\pgfpathmoveto{\pgfqpoint{4.686518in}{2.726247in}}%
\pgfpathlineto{\pgfqpoint{4.686455in}{2.726259in}}%
\pgfusepath{stroke}%
\end{pgfscope}%
\begin{pgfscope}%
\pgfpathrectangle{\pgfqpoint{3.985294in}{1.750000in}}{\pgfqpoint{2.279412in}{2.004545in}}%
\pgfusepath{clip}%
\pgfsetbuttcap%
\pgfsetroundjoin%
\pgfsetlinewidth{0.417933pt}%
\definecolor{currentstroke}{rgb}{0.282656,0.100196,0.422160}%
\pgfsetstrokecolor{currentstroke}%
\pgfsetdash{}{0pt}%
\pgfpathmoveto{\pgfqpoint{4.686455in}{2.726259in}}%
\pgfpathlineto{\pgfqpoint{4.686467in}{2.726257in}}%
\pgfusepath{stroke}%
\end{pgfscope}%
\begin{pgfscope}%
\pgfpathrectangle{\pgfqpoint{3.985294in}{1.750000in}}{\pgfqpoint{2.279412in}{2.004545in}}%
\pgfusepath{clip}%
\pgfsetbuttcap%
\pgfsetroundjoin%
\pgfsetlinewidth{0.417952pt}%
\definecolor{currentstroke}{rgb}{0.282656,0.100196,0.422160}%
\pgfsetstrokecolor{currentstroke}%
\pgfsetdash{}{0pt}%
\pgfpathmoveto{\pgfqpoint{4.686467in}{2.726257in}}%
\pgfpathlineto{\pgfqpoint{4.686543in}{2.726243in}}%
\pgfusepath{stroke}%
\end{pgfscope}%
\begin{pgfscope}%
\pgfpathrectangle{\pgfqpoint{3.985294in}{1.750000in}}{\pgfqpoint{2.279412in}{2.004545in}}%
\pgfusepath{clip}%
\pgfsetbuttcap%
\pgfsetroundjoin%
\pgfsetlinewidth{0.418071pt}%
\definecolor{currentstroke}{rgb}{0.282656,0.100196,0.422160}%
\pgfsetstrokecolor{currentstroke}%
\pgfsetdash{}{0pt}%
\pgfpathmoveto{\pgfqpoint{4.686543in}{2.726243in}}%
\pgfpathlineto{\pgfqpoint{4.686607in}{2.726231in}}%
\pgfusepath{stroke}%
\end{pgfscope}%
\begin{pgfscope}%
\pgfpathrectangle{\pgfqpoint{3.985294in}{1.750000in}}{\pgfqpoint{2.279412in}{2.004545in}}%
\pgfusepath{clip}%
\pgfsetbuttcap%
\pgfsetroundjoin%
\pgfsetlinewidth{0.418171pt}%
\definecolor{currentstroke}{rgb}{0.282656,0.100196,0.422160}%
\pgfsetstrokecolor{currentstroke}%
\pgfsetdash{}{0pt}%
\pgfpathmoveto{\pgfqpoint{4.686607in}{2.726231in}}%
\pgfpathlineto{\pgfqpoint{4.686593in}{2.726233in}}%
\pgfusepath{stroke}%
\end{pgfscope}%
\begin{pgfscope}%
\pgfpathrectangle{\pgfqpoint{3.985294in}{1.750000in}}{\pgfqpoint{2.279412in}{2.004545in}}%
\pgfusepath{clip}%
\pgfsetbuttcap%
\pgfsetroundjoin%
\pgfsetlinewidth{0.418149pt}%
\definecolor{currentstroke}{rgb}{0.282656,0.100196,0.422160}%
\pgfsetstrokecolor{currentstroke}%
\pgfsetdash{}{0pt}%
\pgfpathmoveto{\pgfqpoint{4.686593in}{2.726233in}}%
\pgfpathlineto{\pgfqpoint{4.686516in}{2.726247in}}%
\pgfusepath{stroke}%
\end{pgfscope}%
\begin{pgfscope}%
\pgfpathrectangle{\pgfqpoint{3.985294in}{1.750000in}}{\pgfqpoint{2.279412in}{2.004545in}}%
\pgfusepath{clip}%
\pgfsetbuttcap%
\pgfsetroundjoin%
\pgfsetlinewidth{0.418029pt}%
\definecolor{currentstroke}{rgb}{0.282656,0.100196,0.422160}%
\pgfsetstrokecolor{currentstroke}%
\pgfsetdash{}{0pt}%
\pgfpathmoveto{\pgfqpoint{4.686516in}{2.726247in}}%
\pgfpathlineto{\pgfqpoint{4.686455in}{2.726259in}}%
\pgfusepath{stroke}%
\end{pgfscope}%
\begin{pgfscope}%
\pgfpathrectangle{\pgfqpoint{3.985294in}{1.750000in}}{\pgfqpoint{2.279412in}{2.004545in}}%
\pgfusepath{clip}%
\pgfsetbuttcap%
\pgfsetroundjoin%
\pgfsetlinewidth{0.417934pt}%
\definecolor{currentstroke}{rgb}{0.282656,0.100196,0.422160}%
\pgfsetstrokecolor{currentstroke}%
\pgfsetdash{}{0pt}%
\pgfpathmoveto{\pgfqpoint{4.686455in}{2.726259in}}%
\pgfpathlineto{\pgfqpoint{4.686469in}{2.726257in}}%
\pgfusepath{stroke}%
\end{pgfscope}%
\begin{pgfscope}%
\pgfpathrectangle{\pgfqpoint{3.985294in}{1.750000in}}{\pgfqpoint{2.279412in}{2.004545in}}%
\pgfusepath{clip}%
\pgfsetbuttcap%
\pgfsetroundjoin%
\pgfsetlinewidth{0.417955pt}%
\definecolor{currentstroke}{rgb}{0.282656,0.100196,0.422160}%
\pgfsetstrokecolor{currentstroke}%
\pgfsetdash{}{0pt}%
\pgfpathmoveto{\pgfqpoint{4.686469in}{2.726257in}}%
\pgfpathlineto{\pgfqpoint{4.686544in}{2.726243in}}%
\pgfusepath{stroke}%
\end{pgfscope}%
\begin{pgfscope}%
\pgfpathrectangle{\pgfqpoint{3.985294in}{1.750000in}}{\pgfqpoint{2.279412in}{2.004545in}}%
\pgfusepath{clip}%
\pgfsetbuttcap%
\pgfsetroundjoin%
\pgfsetlinewidth{0.418073pt}%
\definecolor{currentstroke}{rgb}{0.282656,0.100196,0.422160}%
\pgfsetstrokecolor{currentstroke}%
\pgfsetdash{}{0pt}%
\pgfpathmoveto{\pgfqpoint{4.686544in}{2.726243in}}%
\pgfpathlineto{\pgfqpoint{4.686607in}{2.726231in}}%
\pgfusepath{stroke}%
\end{pgfscope}%
\begin{pgfscope}%
\pgfpathrectangle{\pgfqpoint{3.985294in}{1.750000in}}{\pgfqpoint{2.279412in}{2.004545in}}%
\pgfusepath{clip}%
\pgfsetbuttcap%
\pgfsetroundjoin%
\pgfsetlinewidth{0.418171pt}%
\definecolor{currentstroke}{rgb}{0.282656,0.100196,0.422160}%
\pgfsetstrokecolor{currentstroke}%
\pgfsetdash{}{0pt}%
\pgfpathmoveto{\pgfqpoint{4.686607in}{2.726231in}}%
\pgfpathlineto{\pgfqpoint{4.686591in}{2.726234in}}%
\pgfusepath{stroke}%
\end{pgfscope}%
\begin{pgfscope}%
\pgfpathrectangle{\pgfqpoint{3.985294in}{1.750000in}}{\pgfqpoint{2.279412in}{2.004545in}}%
\pgfusepath{clip}%
\pgfsetbuttcap%
\pgfsetroundjoin%
\pgfsetlinewidth{0.418146pt}%
\definecolor{currentstroke}{rgb}{0.282656,0.100196,0.422160}%
\pgfsetstrokecolor{currentstroke}%
\pgfsetdash{}{0pt}%
\pgfpathmoveto{\pgfqpoint{4.686591in}{2.726234in}}%
\pgfpathlineto{\pgfqpoint{4.686515in}{2.726248in}}%
\pgfusepath{stroke}%
\end{pgfscope}%
\begin{pgfscope}%
\pgfpathrectangle{\pgfqpoint{3.985294in}{1.750000in}}{\pgfqpoint{2.279412in}{2.004545in}}%
\pgfusepath{clip}%
\pgfsetbuttcap%
\pgfsetroundjoin%
\pgfsetlinewidth{0.418027pt}%
\definecolor{currentstroke}{rgb}{0.282656,0.100196,0.422160}%
\pgfsetstrokecolor{currentstroke}%
\pgfsetdash{}{0pt}%
\pgfpathmoveto{\pgfqpoint{4.686515in}{2.726248in}}%
\pgfpathlineto{\pgfqpoint{4.686456in}{2.726259in}}%
\pgfusepath{stroke}%
\end{pgfscope}%
\begin{pgfscope}%
\pgfpathrectangle{\pgfqpoint{3.985294in}{1.750000in}}{\pgfqpoint{2.279412in}{2.004545in}}%
\pgfusepath{clip}%
\pgfsetbuttcap%
\pgfsetroundjoin%
\pgfsetlinewidth{0.417934pt}%
\definecolor{currentstroke}{rgb}{0.282656,0.100196,0.422160}%
\pgfsetstrokecolor{currentstroke}%
\pgfsetdash{}{0pt}%
\pgfpathmoveto{\pgfqpoint{4.686456in}{2.726259in}}%
\pgfpathlineto{\pgfqpoint{4.686470in}{2.726256in}}%
\pgfusepath{stroke}%
\end{pgfscope}%
\begin{pgfscope}%
\pgfpathrectangle{\pgfqpoint{3.985294in}{1.750000in}}{\pgfqpoint{2.279412in}{2.004545in}}%
\pgfusepath{clip}%
\pgfsetbuttcap%
\pgfsetroundjoin%
\pgfsetlinewidth{0.417958pt}%
\definecolor{currentstroke}{rgb}{0.282656,0.100196,0.422160}%
\pgfsetstrokecolor{currentstroke}%
\pgfsetdash{}{0pt}%
\pgfpathmoveto{\pgfqpoint{4.686470in}{2.726256in}}%
\pgfpathlineto{\pgfqpoint{4.686545in}{2.726243in}}%
\pgfusepath{stroke}%
\end{pgfscope}%
\begin{pgfscope}%
\pgfpathrectangle{\pgfqpoint{3.985294in}{1.750000in}}{\pgfqpoint{2.279412in}{2.004545in}}%
\pgfusepath{clip}%
\pgfsetbuttcap%
\pgfsetroundjoin%
\pgfsetlinewidth{0.418075pt}%
\definecolor{currentstroke}{rgb}{0.282656,0.100196,0.422160}%
\pgfsetstrokecolor{currentstroke}%
\pgfsetdash{}{0pt}%
\pgfpathmoveto{\pgfqpoint{4.686545in}{2.726243in}}%
\pgfpathlineto{\pgfqpoint{4.686606in}{2.726231in}}%
\pgfusepath{stroke}%
\end{pgfscope}%
\begin{pgfscope}%
\pgfpathrectangle{\pgfqpoint{3.985294in}{1.750000in}}{\pgfqpoint{2.279412in}{2.004545in}}%
\pgfusepath{clip}%
\pgfsetbuttcap%
\pgfsetroundjoin%
\pgfsetlinewidth{0.418170pt}%
\definecolor{currentstroke}{rgb}{0.282656,0.100196,0.422160}%
\pgfsetstrokecolor{currentstroke}%
\pgfsetdash{}{0pt}%
\pgfpathmoveto{\pgfqpoint{4.686606in}{2.726231in}}%
\pgfpathlineto{\pgfqpoint{4.686589in}{2.726234in}}%
\pgfusepath{stroke}%
\end{pgfscope}%
\begin{pgfscope}%
\pgfpathrectangle{\pgfqpoint{3.985294in}{1.750000in}}{\pgfqpoint{2.279412in}{2.004545in}}%
\pgfusepath{clip}%
\pgfsetbuttcap%
\pgfsetroundjoin%
\pgfsetlinewidth{0.418143pt}%
\definecolor{currentstroke}{rgb}{0.282656,0.100196,0.422160}%
\pgfsetstrokecolor{currentstroke}%
\pgfsetdash{}{0pt}%
\pgfpathmoveto{\pgfqpoint{4.686589in}{2.726234in}}%
\pgfpathlineto{\pgfqpoint{4.686513in}{2.726248in}}%
\pgfusepath{stroke}%
\end{pgfscope}%
\begin{pgfscope}%
\pgfpathrectangle{\pgfqpoint{3.985294in}{1.750000in}}{\pgfqpoint{2.279412in}{2.004545in}}%
\pgfusepath{clip}%
\pgfsetbuttcap%
\pgfsetroundjoin%
\pgfsetlinewidth{0.418025pt}%
\definecolor{currentstroke}{rgb}{0.282656,0.100196,0.422160}%
\pgfsetstrokecolor{currentstroke}%
\pgfsetdash{}{0pt}%
\pgfpathmoveto{\pgfqpoint{4.686513in}{2.726248in}}%
\pgfpathlineto{\pgfqpoint{4.686456in}{2.726259in}}%
\pgfusepath{stroke}%
\end{pgfscope}%
\begin{pgfscope}%
\pgfpathrectangle{\pgfqpoint{3.985294in}{1.750000in}}{\pgfqpoint{2.279412in}{2.004545in}}%
\pgfusepath{clip}%
\pgfsetbuttcap%
\pgfsetroundjoin%
\pgfsetlinewidth{0.417935pt}%
\definecolor{currentstroke}{rgb}{0.282656,0.100196,0.422160}%
\pgfsetstrokecolor{currentstroke}%
\pgfsetdash{}{0pt}%
\pgfpathmoveto{\pgfqpoint{4.686456in}{2.726259in}}%
\pgfpathlineto{\pgfqpoint{4.686472in}{2.726256in}}%
\pgfusepath{stroke}%
\end{pgfscope}%
\begin{pgfscope}%
\pgfpathrectangle{\pgfqpoint{3.985294in}{1.750000in}}{\pgfqpoint{2.279412in}{2.004545in}}%
\pgfusepath{clip}%
\pgfsetbuttcap%
\pgfsetroundjoin%
\pgfsetlinewidth{0.417961pt}%
\definecolor{currentstroke}{rgb}{0.282656,0.100196,0.422160}%
\pgfsetstrokecolor{currentstroke}%
\pgfsetdash{}{0pt}%
\pgfpathmoveto{\pgfqpoint{4.686472in}{2.726256in}}%
\pgfpathlineto{\pgfqpoint{4.686547in}{2.726242in}}%
\pgfusepath{stroke}%
\end{pgfscope}%
\begin{pgfscope}%
\pgfpathrectangle{\pgfqpoint{3.985294in}{1.750000in}}{\pgfqpoint{2.279412in}{2.004545in}}%
\pgfusepath{clip}%
\pgfsetbuttcap%
\pgfsetroundjoin%
\pgfsetlinewidth{0.418077pt}%
\definecolor{currentstroke}{rgb}{0.282656,0.100196,0.422160}%
\pgfsetstrokecolor{currentstroke}%
\pgfsetdash{}{0pt}%
\pgfpathmoveto{\pgfqpoint{4.686547in}{2.726242in}}%
\pgfpathlineto{\pgfqpoint{4.686606in}{2.726231in}}%
\pgfusepath{stroke}%
\end{pgfscope}%
\begin{pgfscope}%
\pgfpathrectangle{\pgfqpoint{3.985294in}{1.750000in}}{\pgfqpoint{2.279412in}{2.004545in}}%
\pgfusepath{clip}%
\pgfsetbuttcap%
\pgfsetroundjoin%
\pgfsetlinewidth{0.418169pt}%
\definecolor{currentstroke}{rgb}{0.282656,0.100196,0.422160}%
\pgfsetstrokecolor{currentstroke}%
\pgfsetdash{}{0pt}%
\pgfpathmoveto{\pgfqpoint{4.686606in}{2.726231in}}%
\pgfpathlineto{\pgfqpoint{4.686587in}{2.726234in}}%
\pgfusepath{stroke}%
\end{pgfscope}%
\begin{pgfscope}%
\pgfpathrectangle{\pgfqpoint{3.985294in}{1.750000in}}{\pgfqpoint{2.279412in}{2.004545in}}%
\pgfusepath{clip}%
\pgfsetbuttcap%
\pgfsetroundjoin%
\pgfsetlinewidth{0.418140pt}%
\definecolor{currentstroke}{rgb}{0.282656,0.100196,0.422160}%
\pgfsetstrokecolor{currentstroke}%
\pgfsetdash{}{0pt}%
\pgfpathmoveto{\pgfqpoint{4.686587in}{2.726234in}}%
\pgfpathlineto{\pgfqpoint{4.686512in}{2.726248in}}%
\pgfusepath{stroke}%
\end{pgfscope}%
\begin{pgfscope}%
\pgfpathrectangle{\pgfqpoint{3.985294in}{1.750000in}}{\pgfqpoint{2.279412in}{2.004545in}}%
\pgfusepath{clip}%
\pgfsetbuttcap%
\pgfsetroundjoin%
\pgfsetlinewidth{0.418023pt}%
\definecolor{currentstroke}{rgb}{0.282656,0.100196,0.422160}%
\pgfsetstrokecolor{currentstroke}%
\pgfsetdash{}{0pt}%
\pgfpathmoveto{\pgfqpoint{4.686512in}{2.726248in}}%
\pgfpathlineto{\pgfqpoint{4.686457in}{2.726259in}}%
\pgfusepath{stroke}%
\end{pgfscope}%
\begin{pgfscope}%
\pgfpathrectangle{\pgfqpoint{3.985294in}{1.750000in}}{\pgfqpoint{2.279412in}{2.004545in}}%
\pgfusepath{clip}%
\pgfsetbuttcap%
\pgfsetroundjoin%
\pgfsetlinewidth{0.417936pt}%
\definecolor{currentstroke}{rgb}{0.282656,0.100196,0.422160}%
\pgfsetstrokecolor{currentstroke}%
\pgfsetdash{}{0pt}%
\pgfpathmoveto{\pgfqpoint{4.686457in}{2.726259in}}%
\pgfpathlineto{\pgfqpoint{4.686474in}{2.726256in}}%
\pgfusepath{stroke}%
\end{pgfscope}%
\begin{pgfscope}%
\pgfpathrectangle{\pgfqpoint{3.985294in}{1.750000in}}{\pgfqpoint{2.279412in}{2.004545in}}%
\pgfusepath{clip}%
\pgfsetbuttcap%
\pgfsetroundjoin%
\pgfsetlinewidth{0.417963pt}%
\definecolor{currentstroke}{rgb}{0.282656,0.100196,0.422160}%
\pgfsetstrokecolor{currentstroke}%
\pgfsetdash{}{0pt}%
\pgfpathmoveto{\pgfqpoint{4.686474in}{2.726256in}}%
\pgfpathlineto{\pgfqpoint{4.686548in}{2.726242in}}%
\pgfusepath{stroke}%
\end{pgfscope}%
\begin{pgfscope}%
\pgfpathrectangle{\pgfqpoint{3.985294in}{1.750000in}}{\pgfqpoint{2.279412in}{2.004545in}}%
\pgfusepath{clip}%
\pgfsetbuttcap%
\pgfsetroundjoin%
\pgfsetlinewidth{0.418079pt}%
\definecolor{currentstroke}{rgb}{0.282656,0.100196,0.422160}%
\pgfsetstrokecolor{currentstroke}%
\pgfsetdash{}{0pt}%
\pgfpathmoveto{\pgfqpoint{4.686548in}{2.726242in}}%
\pgfpathlineto{\pgfqpoint{4.686605in}{2.726231in}}%
\pgfusepath{stroke}%
\end{pgfscope}%
\begin{pgfscope}%
\pgfpathrectangle{\pgfqpoint{3.985294in}{1.750000in}}{\pgfqpoint{2.279412in}{2.004545in}}%
\pgfusepath{clip}%
\pgfsetbuttcap%
\pgfsetroundjoin%
\pgfsetlinewidth{0.418168pt}%
\definecolor{currentstroke}{rgb}{0.282656,0.100196,0.422160}%
\pgfsetstrokecolor{currentstroke}%
\pgfsetdash{}{0pt}%
\pgfpathmoveto{\pgfqpoint{4.686605in}{2.726231in}}%
\pgfpathlineto{\pgfqpoint{4.686586in}{2.726235in}}%
\pgfusepath{stroke}%
\end{pgfscope}%
\begin{pgfscope}%
\pgfpathrectangle{\pgfqpoint{3.985294in}{1.750000in}}{\pgfqpoint{2.279412in}{2.004545in}}%
\pgfusepath{clip}%
\pgfsetbuttcap%
\pgfsetroundjoin%
\pgfsetlinewidth{0.418138pt}%
\definecolor{currentstroke}{rgb}{0.282656,0.100196,0.422160}%
\pgfsetstrokecolor{currentstroke}%
\pgfsetdash{}{0pt}%
\pgfpathmoveto{\pgfqpoint{4.686586in}{2.726235in}}%
\pgfpathlineto{\pgfqpoint{4.686511in}{2.726249in}}%
\pgfusepath{stroke}%
\end{pgfscope}%
\begin{pgfscope}%
\pgfpathrectangle{\pgfqpoint{3.985294in}{1.750000in}}{\pgfqpoint{2.279412in}{2.004545in}}%
\pgfusepath{clip}%
\pgfsetbuttcap%
\pgfsetroundjoin%
\pgfsetlinewidth{0.418021pt}%
\definecolor{currentstroke}{rgb}{0.282656,0.100196,0.422160}%
\pgfsetstrokecolor{currentstroke}%
\pgfsetdash{}{0pt}%
\pgfpathmoveto{\pgfqpoint{4.686511in}{2.726249in}}%
\pgfpathlineto{\pgfqpoint{4.686457in}{2.726259in}}%
\pgfusepath{stroke}%
\end{pgfscope}%
\begin{pgfscope}%
\pgfpathrectangle{\pgfqpoint{3.985294in}{1.750000in}}{\pgfqpoint{2.279412in}{2.004545in}}%
\pgfusepath{clip}%
\pgfsetbuttcap%
\pgfsetroundjoin%
\pgfsetlinewidth{0.417937pt}%
\definecolor{currentstroke}{rgb}{0.282656,0.100196,0.422160}%
\pgfsetstrokecolor{currentstroke}%
\pgfsetdash{}{0pt}%
\pgfpathmoveto{\pgfqpoint{4.686457in}{2.726259in}}%
\pgfpathlineto{\pgfqpoint{4.686476in}{2.726255in}}%
\pgfusepath{stroke}%
\end{pgfscope}%
\begin{pgfscope}%
\pgfpathrectangle{\pgfqpoint{3.985294in}{1.750000in}}{\pgfqpoint{2.279412in}{2.004545in}}%
\pgfusepath{clip}%
\pgfsetbuttcap%
\pgfsetroundjoin%
\pgfsetlinewidth{0.417966pt}%
\definecolor{currentstroke}{rgb}{0.282656,0.100196,0.422160}%
\pgfsetstrokecolor{currentstroke}%
\pgfsetdash{}{0pt}%
\pgfpathmoveto{\pgfqpoint{4.686476in}{2.726255in}}%
\pgfpathlineto{\pgfqpoint{4.686549in}{2.726242in}}%
\pgfusepath{stroke}%
\end{pgfscope}%
\begin{pgfscope}%
\pgfpathrectangle{\pgfqpoint{3.985294in}{1.750000in}}{\pgfqpoint{2.279412in}{2.004545in}}%
\pgfusepath{clip}%
\pgfsetbuttcap%
\pgfsetroundjoin%
\pgfsetlinewidth{0.418081pt}%
\definecolor{currentstroke}{rgb}{0.282656,0.100196,0.422160}%
\pgfsetstrokecolor{currentstroke}%
\pgfsetdash{}{0pt}%
\pgfpathmoveto{\pgfqpoint{4.686549in}{2.726242in}}%
\pgfpathlineto{\pgfqpoint{4.686605in}{2.726231in}}%
\pgfusepath{stroke}%
\end{pgfscope}%
\begin{pgfscope}%
\pgfpathrectangle{\pgfqpoint{3.985294in}{1.750000in}}{\pgfqpoint{2.279412in}{2.004545in}}%
\pgfusepath{clip}%
\pgfsetbuttcap%
\pgfsetroundjoin%
\pgfsetlinewidth{0.418167pt}%
\definecolor{currentstroke}{rgb}{0.282656,0.100196,0.422160}%
\pgfsetstrokecolor{currentstroke}%
\pgfsetdash{}{0pt}%
\pgfpathmoveto{\pgfqpoint{4.686605in}{2.726231in}}%
\pgfpathlineto{\pgfqpoint{4.686584in}{2.726235in}}%
\pgfusepath{stroke}%
\end{pgfscope}%
\begin{pgfscope}%
\pgfpathrectangle{\pgfqpoint{3.985294in}{1.750000in}}{\pgfqpoint{2.279412in}{2.004545in}}%
\pgfusepath{clip}%
\pgfsetbuttcap%
\pgfsetroundjoin%
\pgfsetlinewidth{0.418135pt}%
\definecolor{currentstroke}{rgb}{0.282656,0.100196,0.422160}%
\pgfsetstrokecolor{currentstroke}%
\pgfsetdash{}{0pt}%
\pgfpathmoveto{\pgfqpoint{4.686584in}{2.726235in}}%
\pgfpathlineto{\pgfqpoint{4.686510in}{2.726249in}}%
\pgfusepath{stroke}%
\end{pgfscope}%
\begin{pgfscope}%
\pgfpathrectangle{\pgfqpoint{3.985294in}{1.750000in}}{\pgfqpoint{2.279412in}{2.004545in}}%
\pgfusepath{clip}%
\pgfsetbuttcap%
\pgfsetroundjoin%
\pgfsetlinewidth{0.418019pt}%
\definecolor{currentstroke}{rgb}{0.282656,0.100196,0.422160}%
\pgfsetstrokecolor{currentstroke}%
\pgfsetdash{}{0pt}%
\pgfpathmoveto{\pgfqpoint{4.686510in}{2.726249in}}%
\pgfpathlineto{\pgfqpoint{4.686458in}{2.726259in}}%
\pgfusepath{stroke}%
\end{pgfscope}%
\begin{pgfscope}%
\pgfpathrectangle{\pgfqpoint{3.985294in}{1.750000in}}{\pgfqpoint{2.279412in}{2.004545in}}%
\pgfusepath{clip}%
\pgfsetbuttcap%
\pgfsetroundjoin%
\pgfsetlinewidth{0.417938pt}%
\definecolor{currentstroke}{rgb}{0.282656,0.100196,0.422160}%
\pgfsetstrokecolor{currentstroke}%
\pgfsetdash{}{0pt}%
\pgfpathmoveto{\pgfqpoint{4.686458in}{2.726259in}}%
\pgfpathlineto{\pgfqpoint{4.686478in}{2.726255in}}%
\pgfusepath{stroke}%
\end{pgfscope}%
\begin{pgfscope}%
\pgfpathrectangle{\pgfqpoint{3.985294in}{1.750000in}}{\pgfqpoint{2.279412in}{2.004545in}}%
\pgfusepath{clip}%
\pgfsetbuttcap%
\pgfsetroundjoin%
\pgfsetlinewidth{0.417969pt}%
\definecolor{currentstroke}{rgb}{0.282656,0.100196,0.422160}%
\pgfsetstrokecolor{currentstroke}%
\pgfsetdash{}{0pt}%
\pgfpathmoveto{\pgfqpoint{4.686478in}{2.726255in}}%
\pgfpathlineto{\pgfqpoint{4.686551in}{2.726242in}}%
\pgfusepath{stroke}%
\end{pgfscope}%
\begin{pgfscope}%
\pgfpathrectangle{\pgfqpoint{3.985294in}{1.750000in}}{\pgfqpoint{2.279412in}{2.004545in}}%
\pgfusepath{clip}%
\pgfsetbuttcap%
\pgfsetroundjoin%
\pgfsetlinewidth{0.418083pt}%
\definecolor{currentstroke}{rgb}{0.282656,0.100196,0.422160}%
\pgfsetstrokecolor{currentstroke}%
\pgfsetdash{}{0pt}%
\pgfpathmoveto{\pgfqpoint{4.686551in}{2.726242in}}%
\pgfpathlineto{\pgfqpoint{4.686604in}{2.726232in}}%
\pgfusepath{stroke}%
\end{pgfscope}%
\begin{pgfscope}%
\pgfpathrectangle{\pgfqpoint{3.985294in}{1.750000in}}{\pgfqpoint{2.279412in}{2.004545in}}%
\pgfusepath{clip}%
\pgfsetbuttcap%
\pgfsetroundjoin%
\pgfsetlinewidth{0.418166pt}%
\definecolor{currentstroke}{rgb}{0.282656,0.100196,0.422160}%
\pgfsetstrokecolor{currentstroke}%
\pgfsetdash{}{0pt}%
\pgfpathmoveto{\pgfqpoint{4.686604in}{2.726232in}}%
\pgfpathlineto{\pgfqpoint{4.686582in}{2.726235in}}%
\pgfusepath{stroke}%
\end{pgfscope}%
\begin{pgfscope}%
\pgfpathrectangle{\pgfqpoint{3.985294in}{1.750000in}}{\pgfqpoint{2.279412in}{2.004545in}}%
\pgfusepath{clip}%
\pgfsetbuttcap%
\pgfsetroundjoin%
\pgfsetlinewidth{0.418132pt}%
\definecolor{currentstroke}{rgb}{0.282656,0.100196,0.422160}%
\pgfsetstrokecolor{currentstroke}%
\pgfsetdash{}{0pt}%
\pgfpathmoveto{\pgfqpoint{4.686582in}{2.726235in}}%
\pgfpathlineto{\pgfqpoint{4.686508in}{2.726249in}}%
\pgfusepath{stroke}%
\end{pgfscope}%
\begin{pgfscope}%
\pgfpathrectangle{\pgfqpoint{3.985294in}{1.750000in}}{\pgfqpoint{2.279412in}{2.004545in}}%
\pgfusepath{clip}%
\pgfsetbuttcap%
\pgfsetroundjoin%
\pgfsetlinewidth{0.418017pt}%
\definecolor{currentstroke}{rgb}{0.282656,0.100196,0.422160}%
\pgfsetstrokecolor{currentstroke}%
\pgfsetdash{}{0pt}%
\pgfpathmoveto{\pgfqpoint{4.686508in}{2.726249in}}%
\pgfpathlineto{\pgfqpoint{4.686458in}{2.726258in}}%
\pgfusepath{stroke}%
\end{pgfscope}%
\begin{pgfscope}%
\pgfpathrectangle{\pgfqpoint{3.985294in}{1.750000in}}{\pgfqpoint{2.279412in}{2.004545in}}%
\pgfusepath{clip}%
\pgfsetbuttcap%
\pgfsetroundjoin%
\pgfsetlinewidth{0.417938pt}%
\definecolor{currentstroke}{rgb}{0.282656,0.100196,0.422160}%
\pgfsetstrokecolor{currentstroke}%
\pgfsetdash{}{0pt}%
\pgfpathmoveto{\pgfqpoint{4.686458in}{2.726258in}}%
\pgfpathlineto{\pgfqpoint{4.686479in}{2.726255in}}%
\pgfusepath{stroke}%
\end{pgfscope}%
\begin{pgfscope}%
\pgfpathrectangle{\pgfqpoint{3.985294in}{1.750000in}}{\pgfqpoint{2.279412in}{2.004545in}}%
\pgfusepath{clip}%
\pgfsetbuttcap%
\pgfsetroundjoin%
\pgfsetlinewidth{0.417972pt}%
\definecolor{currentstroke}{rgb}{0.282656,0.100196,0.422160}%
\pgfsetstrokecolor{currentstroke}%
\pgfsetdash{}{0pt}%
\pgfpathmoveto{\pgfqpoint{4.686479in}{2.726255in}}%
\pgfpathlineto{\pgfqpoint{4.686552in}{2.726241in}}%
\pgfusepath{stroke}%
\end{pgfscope}%
\begin{pgfscope}%
\pgfpathrectangle{\pgfqpoint{3.985294in}{1.750000in}}{\pgfqpoint{2.279412in}{2.004545in}}%
\pgfusepath{clip}%
\pgfsetbuttcap%
\pgfsetroundjoin%
\pgfsetlinewidth{0.418085pt}%
\definecolor{currentstroke}{rgb}{0.282656,0.100196,0.422160}%
\pgfsetstrokecolor{currentstroke}%
\pgfsetdash{}{0pt}%
\pgfpathmoveto{\pgfqpoint{4.686552in}{2.726241in}}%
\pgfpathlineto{\pgfqpoint{4.686603in}{2.726232in}}%
\pgfusepath{stroke}%
\end{pgfscope}%
\begin{pgfscope}%
\pgfpathrectangle{\pgfqpoint{3.985294in}{1.750000in}}{\pgfqpoint{2.279412in}{2.004545in}}%
\pgfusepath{clip}%
\pgfsetbuttcap%
\pgfsetroundjoin%
\pgfsetlinewidth{0.418165pt}%
\definecolor{currentstroke}{rgb}{0.282656,0.100196,0.422160}%
\pgfsetstrokecolor{currentstroke}%
\pgfsetdash{}{0pt}%
\pgfpathmoveto{\pgfqpoint{4.686603in}{2.726232in}}%
\pgfpathlineto{\pgfqpoint{4.686580in}{2.726236in}}%
\pgfusepath{stroke}%
\end{pgfscope}%
\begin{pgfscope}%
\pgfpathrectangle{\pgfqpoint{3.985294in}{1.750000in}}{\pgfqpoint{2.279412in}{2.004545in}}%
\pgfusepath{clip}%
\pgfsetbuttcap%
\pgfsetroundjoin%
\pgfsetlinewidth{0.311077pt}%
\definecolor{currentstroke}{rgb}{0.268510,0.009605,0.335427}%
\pgfsetstrokecolor{currentstroke}%
\pgfsetdash{}{0pt}%
\pgfpathmoveto{\pgfqpoint{5.945670in}{2.842486in}}%
\pgfpathlineto{\pgfqpoint{5.927266in}{2.843711in}}%
\pgfusepath{stroke}%
\end{pgfscope}%
\begin{pgfscope}%
\pgfpathrectangle{\pgfqpoint{3.985294in}{1.750000in}}{\pgfqpoint{2.279412in}{2.004545in}}%
\pgfusepath{clip}%
\pgfsetbuttcap%
\pgfsetroundjoin%
\pgfsetlinewidth{0.317529pt}%
\definecolor{currentstroke}{rgb}{0.269944,0.014625,0.341379}%
\pgfsetstrokecolor{currentstroke}%
\pgfsetdash{}{0pt}%
\pgfpathmoveto{\pgfqpoint{5.927266in}{2.843711in}}%
\pgfpathlineto{\pgfqpoint{5.927266in}{2.843711in}}%
\pgfusepath{stroke}%
\end{pgfscope}%
\begin{pgfscope}%
\pgfpathrectangle{\pgfqpoint{3.985294in}{1.750000in}}{\pgfqpoint{2.279412in}{2.004545in}}%
\pgfusepath{clip}%
\pgfsetbuttcap%
\pgfsetroundjoin%
\pgfsetlinewidth{0.317529pt}%
\definecolor{currentstroke}{rgb}{0.269944,0.014625,0.341379}%
\pgfsetstrokecolor{currentstroke}%
\pgfsetdash{}{0pt}%
\pgfpathmoveto{\pgfqpoint{5.927266in}{2.843711in}}%
\pgfpathlineto{\pgfqpoint{5.927266in}{2.843711in}}%
\pgfusepath{stroke}%
\end{pgfscope}%
\begin{pgfscope}%
\pgfpathrectangle{\pgfqpoint{3.985294in}{1.750000in}}{\pgfqpoint{2.279412in}{2.004545in}}%
\pgfusepath{clip}%
\pgfsetbuttcap%
\pgfsetroundjoin%
\pgfsetlinewidth{0.317529pt}%
\definecolor{currentstroke}{rgb}{0.269944,0.014625,0.341379}%
\pgfsetstrokecolor{currentstroke}%
\pgfsetdash{}{0pt}%
\pgfpathmoveto{\pgfqpoint{5.927266in}{2.843711in}}%
\pgfpathlineto{\pgfqpoint{5.903828in}{2.843715in}}%
\pgfusepath{stroke}%
\end{pgfscope}%
\begin{pgfscope}%
\pgfpathrectangle{\pgfqpoint{3.985294in}{1.750000in}}{\pgfqpoint{2.279412in}{2.004545in}}%
\pgfusepath{clip}%
\pgfsetbuttcap%
\pgfsetroundjoin%
\pgfsetlinewidth{0.320596pt}%
\definecolor{currentstroke}{rgb}{0.269944,0.014625,0.341379}%
\pgfsetstrokecolor{currentstroke}%
\pgfsetdash{}{0pt}%
\pgfpathmoveto{\pgfqpoint{5.903828in}{2.843715in}}%
\pgfpathlineto{\pgfqpoint{5.877253in}{2.843105in}}%
\pgfusepath{stroke}%
\end{pgfscope}%
\begin{pgfscope}%
\pgfpathrectangle{\pgfqpoint{3.985294in}{1.750000in}}{\pgfqpoint{2.279412in}{2.004545in}}%
\pgfusepath{clip}%
\pgfsetbuttcap%
\pgfsetroundjoin%
\pgfsetlinewidth{0.308808pt}%
\definecolor{currentstroke}{rgb}{0.268510,0.009605,0.335427}%
\pgfsetstrokecolor{currentstroke}%
\pgfsetdash{}{0pt}%
\pgfpathmoveto{\pgfqpoint{5.877253in}{2.843105in}}%
\pgfpathlineto{\pgfqpoint{5.827603in}{2.843653in}}%
\pgfusepath{stroke}%
\end{pgfscope}%
\begin{pgfscope}%
\pgfpathrectangle{\pgfqpoint{3.985294in}{1.750000in}}{\pgfqpoint{2.279412in}{2.004545in}}%
\pgfusepath{clip}%
\pgfsetbuttcap%
\pgfsetroundjoin%
\pgfsetlinewidth{0.328558pt}%
\definecolor{currentstroke}{rgb}{0.271305,0.019942,0.347269}%
\pgfsetstrokecolor{currentstroke}%
\pgfsetdash{}{0pt}%
\pgfpathmoveto{\pgfqpoint{5.827603in}{2.843653in}}%
\pgfpathlineto{\pgfqpoint{5.777590in}{2.844917in}}%
\pgfusepath{stroke}%
\end{pgfscope}%
\begin{pgfscope}%
\pgfpathrectangle{\pgfqpoint{3.985294in}{1.750000in}}{\pgfqpoint{2.279412in}{2.004545in}}%
\pgfusepath{clip}%
\pgfsetbuttcap%
\pgfsetroundjoin%
\pgfsetlinewidth{0.324377pt}%
\definecolor{currentstroke}{rgb}{0.271305,0.019942,0.347269}%
\pgfsetstrokecolor{currentstroke}%
\pgfsetdash{}{0pt}%
\pgfpathmoveto{\pgfqpoint{5.777590in}{2.844917in}}%
\pgfpathlineto{\pgfqpoint{5.727457in}{2.844873in}}%
\pgfusepath{stroke}%
\end{pgfscope}%
\begin{pgfscope}%
\pgfpathrectangle{\pgfqpoint{3.985294in}{1.750000in}}{\pgfqpoint{2.279412in}{2.004545in}}%
\pgfusepath{clip}%
\pgfsetbuttcap%
\pgfsetroundjoin%
\pgfsetlinewidth{0.335559pt}%
\definecolor{currentstroke}{rgb}{0.273809,0.031497,0.358853}%
\pgfsetstrokecolor{currentstroke}%
\pgfsetdash{}{0pt}%
\pgfpathmoveto{\pgfqpoint{5.727457in}{2.844873in}}%
\pgfpathlineto{\pgfqpoint{5.677314in}{2.844194in}}%
\pgfusepath{stroke}%
\end{pgfscope}%
\begin{pgfscope}%
\pgfpathrectangle{\pgfqpoint{3.985294in}{1.750000in}}{\pgfqpoint{2.279412in}{2.004545in}}%
\pgfusepath{clip}%
\pgfsetbuttcap%
\pgfsetroundjoin%
\pgfsetlinewidth{0.357765pt}%
\definecolor{currentstroke}{rgb}{0.277018,0.050344,0.375715}%
\pgfsetstrokecolor{currentstroke}%
\pgfsetdash{}{0pt}%
\pgfpathmoveto{\pgfqpoint{5.677314in}{2.844194in}}%
\pgfpathlineto{\pgfqpoint{5.627172in}{2.843452in}}%
\pgfusepath{stroke}%
\end{pgfscope}%
\begin{pgfscope}%
\pgfpathrectangle{\pgfqpoint{3.985294in}{1.750000in}}{\pgfqpoint{2.279412in}{2.004545in}}%
\pgfusepath{clip}%
\pgfsetbuttcap%
\pgfsetroundjoin%
\pgfsetlinewidth{0.391448pt}%
\definecolor{currentstroke}{rgb}{0.280894,0.078907,0.402329}%
\pgfsetstrokecolor{currentstroke}%
\pgfsetdash{}{0pt}%
\pgfpathmoveto{\pgfqpoint{5.627172in}{2.843452in}}%
\pgfpathlineto{\pgfqpoint{5.577024in}{2.842909in}}%
\pgfusepath{stroke}%
\end{pgfscope}%
\begin{pgfscope}%
\pgfpathrectangle{\pgfqpoint{3.985294in}{1.750000in}}{\pgfqpoint{2.279412in}{2.004545in}}%
\pgfusepath{clip}%
\pgfsetbuttcap%
\pgfsetroundjoin%
\pgfsetlinewidth{0.443446pt}%
\definecolor{currentstroke}{rgb}{0.283197,0.115680,0.436115}%
\pgfsetstrokecolor{currentstroke}%
\pgfsetdash{}{0pt}%
\pgfpathmoveto{\pgfqpoint{5.577024in}{2.842909in}}%
\pgfpathlineto{\pgfqpoint{5.526876in}{2.842356in}}%
\pgfusepath{stroke}%
\end{pgfscope}%
\begin{pgfscope}%
\pgfpathrectangle{\pgfqpoint{3.985294in}{1.750000in}}{\pgfqpoint{2.279412in}{2.004545in}}%
\pgfusepath{clip}%
\pgfsetbuttcap%
\pgfsetroundjoin%
\pgfsetlinewidth{0.498873pt}%
\definecolor{currentstroke}{rgb}{0.281412,0.155834,0.469201}%
\pgfsetstrokecolor{currentstroke}%
\pgfsetdash{}{0pt}%
\pgfpathmoveto{\pgfqpoint{5.526876in}{2.842356in}}%
\pgfpathlineto{\pgfqpoint{5.476730in}{2.841696in}}%
\pgfusepath{stroke}%
\end{pgfscope}%
\begin{pgfscope}%
\pgfpathrectangle{\pgfqpoint{3.985294in}{1.750000in}}{\pgfqpoint{2.279412in}{2.004545in}}%
\pgfusepath{clip}%
\pgfsetbuttcap%
\pgfsetroundjoin%
\pgfsetlinewidth{0.557983pt}%
\definecolor{currentstroke}{rgb}{0.274128,0.199721,0.498911}%
\pgfsetstrokecolor{currentstroke}%
\pgfsetdash{}{0pt}%
\pgfpathmoveto{\pgfqpoint{5.476730in}{2.841696in}}%
\pgfpathlineto{\pgfqpoint{5.426588in}{2.840855in}}%
\pgfusepath{stroke}%
\end{pgfscope}%
\begin{pgfscope}%
\pgfpathrectangle{\pgfqpoint{3.985294in}{1.750000in}}{\pgfqpoint{2.279412in}{2.004545in}}%
\pgfusepath{clip}%
\pgfsetbuttcap%
\pgfsetroundjoin%
\pgfsetlinewidth{0.669225pt}%
\definecolor{currentstroke}{rgb}{0.250425,0.274290,0.533103}%
\pgfsetstrokecolor{currentstroke}%
\pgfsetdash{}{0pt}%
\pgfpathmoveto{\pgfqpoint{5.426588in}{2.840855in}}%
\pgfpathlineto{\pgfqpoint{5.376452in}{2.839743in}}%
\pgfusepath{stroke}%
\end{pgfscope}%
\begin{pgfscope}%
\pgfpathrectangle{\pgfqpoint{3.985294in}{1.750000in}}{\pgfqpoint{2.279412in}{2.004545in}}%
\pgfusepath{clip}%
\pgfsetbuttcap%
\pgfsetroundjoin%
\pgfsetlinewidth{0.746227pt}%
\definecolor{currentstroke}{rgb}{0.229739,0.322361,0.545706}%
\pgfsetstrokecolor{currentstroke}%
\pgfsetdash{}{0pt}%
\pgfpathmoveto{\pgfqpoint{5.376452in}{2.839743in}}%
\pgfpathlineto{\pgfqpoint{5.326327in}{2.838329in}}%
\pgfusepath{stroke}%
\end{pgfscope}%
\begin{pgfscope}%
\pgfpathrectangle{\pgfqpoint{3.985294in}{1.750000in}}{\pgfqpoint{2.279412in}{2.004545in}}%
\pgfusepath{clip}%
\pgfsetbuttcap%
\pgfsetroundjoin%
\pgfsetlinewidth{0.815967pt}%
\definecolor{currentstroke}{rgb}{0.208623,0.367752,0.552675}%
\pgfsetstrokecolor{currentstroke}%
\pgfsetdash{}{0pt}%
\pgfpathmoveto{\pgfqpoint{5.326327in}{2.838329in}}%
\pgfpathlineto{\pgfqpoint{5.276221in}{2.836475in}}%
\pgfusepath{stroke}%
\end{pgfscope}%
\begin{pgfscope}%
\pgfpathrectangle{\pgfqpoint{3.985294in}{1.750000in}}{\pgfqpoint{2.279412in}{2.004545in}}%
\pgfusepath{clip}%
\pgfsetbuttcap%
\pgfsetroundjoin%
\pgfsetlinewidth{0.913673pt}%
\definecolor{currentstroke}{rgb}{0.183898,0.422383,0.556944}%
\pgfsetstrokecolor{currentstroke}%
\pgfsetdash{}{0pt}%
\pgfpathmoveto{\pgfqpoint{5.276221in}{2.836475in}}%
\pgfpathlineto{\pgfqpoint{5.226147in}{2.834027in}}%
\pgfusepath{stroke}%
\end{pgfscope}%
\begin{pgfscope}%
\pgfpathrectangle{\pgfqpoint{3.985294in}{1.750000in}}{\pgfqpoint{2.279412in}{2.004545in}}%
\pgfusepath{clip}%
\pgfsetbuttcap%
\pgfsetroundjoin%
\pgfsetlinewidth{0.963045pt}%
\definecolor{currentstroke}{rgb}{0.172719,0.448791,0.557885}%
\pgfsetstrokecolor{currentstroke}%
\pgfsetdash{}{0pt}%
\pgfpathmoveto{\pgfqpoint{5.226147in}{2.834027in}}%
\pgfpathlineto{\pgfqpoint{5.176106in}{2.831110in}}%
\pgfusepath{stroke}%
\end{pgfscope}%
\begin{pgfscope}%
\pgfpathrectangle{\pgfqpoint{3.985294in}{1.750000in}}{\pgfqpoint{2.279412in}{2.004545in}}%
\pgfusepath{clip}%
\pgfsetbuttcap%
\pgfsetroundjoin%
\pgfsetlinewidth{0.909707pt}%
\definecolor{currentstroke}{rgb}{0.185556,0.418570,0.556753}%
\pgfsetstrokecolor{currentstroke}%
\pgfsetdash{}{0pt}%
\pgfpathmoveto{\pgfqpoint{5.176106in}{2.831110in}}%
\pgfpathlineto{\pgfqpoint{5.126123in}{2.827532in}}%
\pgfusepath{stroke}%
\end{pgfscope}%
\begin{pgfscope}%
\pgfpathrectangle{\pgfqpoint{3.985294in}{1.750000in}}{\pgfqpoint{2.279412in}{2.004545in}}%
\pgfusepath{clip}%
\pgfsetbuttcap%
\pgfsetroundjoin%
\pgfsetlinewidth{0.929159pt}%
\definecolor{currentstroke}{rgb}{0.180629,0.429975,0.557282}%
\pgfsetstrokecolor{currentstroke}%
\pgfsetdash{}{0pt}%
\pgfpathmoveto{\pgfqpoint{5.126123in}{2.827532in}}%
\pgfpathlineto{\pgfqpoint{5.076225in}{2.823140in}}%
\pgfusepath{stroke}%
\end{pgfscope}%
\begin{pgfscope}%
\pgfpathrectangle{\pgfqpoint{3.985294in}{1.750000in}}{\pgfqpoint{2.279412in}{2.004545in}}%
\pgfusepath{clip}%
\pgfsetbuttcap%
\pgfsetroundjoin%
\pgfsetlinewidth{0.305703pt}%
\definecolor{currentstroke}{rgb}{0.267004,0.004874,0.329415}%
\pgfsetstrokecolor{currentstroke}%
\pgfsetdash{}{0pt}%
\pgfpathmoveto{\pgfqpoint{5.945670in}{2.932700in}}%
\pgfpathlineto{\pgfqpoint{5.899362in}{2.935319in}}%
\pgfusepath{stroke}%
\end{pgfscope}%
\begin{pgfscope}%
\pgfpathrectangle{\pgfqpoint{3.985294in}{1.750000in}}{\pgfqpoint{2.279412in}{2.004545in}}%
\pgfusepath{clip}%
\pgfsetbuttcap%
\pgfsetroundjoin%
\pgfsetlinewidth{0.318105pt}%
\definecolor{currentstroke}{rgb}{0.269944,0.014625,0.341379}%
\pgfsetstrokecolor{currentstroke}%
\pgfsetdash{}{0pt}%
\pgfpathmoveto{\pgfqpoint{5.899362in}{2.935319in}}%
\pgfpathlineto{\pgfqpoint{5.851513in}{2.935880in}}%
\pgfusepath{stroke}%
\end{pgfscope}%
\begin{pgfscope}%
\pgfpathrectangle{\pgfqpoint{3.985294in}{1.750000in}}{\pgfqpoint{2.279412in}{2.004545in}}%
\pgfusepath{clip}%
\pgfsetbuttcap%
\pgfsetroundjoin%
\pgfsetlinewidth{0.323357pt}%
\definecolor{currentstroke}{rgb}{0.271305,0.019942,0.347269}%
\pgfsetstrokecolor{currentstroke}%
\pgfsetdash{}{0pt}%
\pgfpathmoveto{\pgfqpoint{5.851513in}{2.935880in}}%
\pgfpathlineto{\pgfqpoint{5.801379in}{2.935711in}}%
\pgfusepath{stroke}%
\end{pgfscope}%
\begin{pgfscope}%
\pgfpathrectangle{\pgfqpoint{3.985294in}{1.750000in}}{\pgfqpoint{2.279412in}{2.004545in}}%
\pgfusepath{clip}%
\pgfsetbuttcap%
\pgfsetroundjoin%
\pgfsetlinewidth{0.313582pt}%
\definecolor{currentstroke}{rgb}{0.268510,0.009605,0.335427}%
\pgfsetstrokecolor{currentstroke}%
\pgfsetdash{}{0pt}%
\pgfpathmoveto{\pgfqpoint{5.801379in}{2.935711in}}%
\pgfpathlineto{\pgfqpoint{5.751240in}{2.935646in}}%
\pgfusepath{stroke}%
\end{pgfscope}%
\begin{pgfscope}%
\pgfpathrectangle{\pgfqpoint{3.985294in}{1.750000in}}{\pgfqpoint{2.279412in}{2.004545in}}%
\pgfusepath{clip}%
\pgfsetbuttcap%
\pgfsetroundjoin%
\pgfsetlinewidth{0.334608pt}%
\definecolor{currentstroke}{rgb}{0.272594,0.025563,0.353093}%
\pgfsetstrokecolor{currentstroke}%
\pgfsetdash{}{0pt}%
\pgfpathmoveto{\pgfqpoint{5.751240in}{2.935646in}}%
\pgfpathlineto{\pgfqpoint{5.701098in}{2.934937in}}%
\pgfusepath{stroke}%
\end{pgfscope}%
\begin{pgfscope}%
\pgfpathrectangle{\pgfqpoint{3.985294in}{1.750000in}}{\pgfqpoint{2.279412in}{2.004545in}}%
\pgfusepath{clip}%
\pgfsetbuttcap%
\pgfsetroundjoin%
\pgfsetlinewidth{0.343178pt}%
\definecolor{currentstroke}{rgb}{0.274952,0.037752,0.364543}%
\pgfsetstrokecolor{currentstroke}%
\pgfsetdash{}{0pt}%
\pgfpathmoveto{\pgfqpoint{5.701098in}{2.934937in}}%
\pgfpathlineto{\pgfqpoint{5.650957in}{2.934088in}}%
\pgfusepath{stroke}%
\end{pgfscope}%
\begin{pgfscope}%
\pgfpathrectangle{\pgfqpoint{3.985294in}{1.750000in}}{\pgfqpoint{2.279412in}{2.004545in}}%
\pgfusepath{clip}%
\pgfsetbuttcap%
\pgfsetroundjoin%
\pgfsetlinewidth{0.360611pt}%
\definecolor{currentstroke}{rgb}{0.277018,0.050344,0.375715}%
\pgfsetstrokecolor{currentstroke}%
\pgfsetdash{}{0pt}%
\pgfpathmoveto{\pgfqpoint{5.650957in}{2.934088in}}%
\pgfpathlineto{\pgfqpoint{5.600811in}{2.933501in}}%
\pgfusepath{stroke}%
\end{pgfscope}%
\begin{pgfscope}%
\pgfpathrectangle{\pgfqpoint{3.985294in}{1.750000in}}{\pgfqpoint{2.279412in}{2.004545in}}%
\pgfusepath{clip}%
\pgfsetbuttcap%
\pgfsetroundjoin%
\pgfsetlinewidth{0.402059pt}%
\definecolor{currentstroke}{rgb}{0.281446,0.084320,0.407414}%
\pgfsetstrokecolor{currentstroke}%
\pgfsetdash{}{0pt}%
\pgfpathmoveto{\pgfqpoint{5.600811in}{2.933501in}}%
\pgfpathlineto{\pgfqpoint{5.550665in}{2.932875in}}%
\pgfusepath{stroke}%
\end{pgfscope}%
\begin{pgfscope}%
\pgfpathrectangle{\pgfqpoint{3.985294in}{1.750000in}}{\pgfqpoint{2.279412in}{2.004545in}}%
\pgfusepath{clip}%
\pgfsetbuttcap%
\pgfsetroundjoin%
\pgfsetlinewidth{0.453073pt}%
\definecolor{currentstroke}{rgb}{0.283187,0.125848,0.444960}%
\pgfsetstrokecolor{currentstroke}%
\pgfsetdash{}{0pt}%
\pgfpathmoveto{\pgfqpoint{5.550665in}{2.932875in}}%
\pgfpathlineto{\pgfqpoint{5.500526in}{2.931908in}}%
\pgfusepath{stroke}%
\end{pgfscope}%
\begin{pgfscope}%
\pgfpathrectangle{\pgfqpoint{3.985294in}{1.750000in}}{\pgfqpoint{2.279412in}{2.004545in}}%
\pgfusepath{clip}%
\pgfsetbuttcap%
\pgfsetroundjoin%
\pgfsetlinewidth{0.496072pt}%
\definecolor{currentstroke}{rgb}{0.281412,0.155834,0.469201}%
\pgfsetstrokecolor{currentstroke}%
\pgfsetdash{}{0pt}%
\pgfpathmoveto{\pgfqpoint{5.500526in}{2.931908in}}%
\pgfpathlineto{\pgfqpoint{5.450392in}{2.930753in}}%
\pgfusepath{stroke}%
\end{pgfscope}%
\begin{pgfscope}%
\pgfpathrectangle{\pgfqpoint{3.985294in}{1.750000in}}{\pgfqpoint{2.279412in}{2.004545in}}%
\pgfusepath{clip}%
\pgfsetbuttcap%
\pgfsetroundjoin%
\pgfsetlinewidth{0.560874pt}%
\definecolor{currentstroke}{rgb}{0.274128,0.199721,0.498911}%
\pgfsetstrokecolor{currentstroke}%
\pgfsetdash{}{0pt}%
\pgfpathmoveto{\pgfqpoint{5.450392in}{2.930753in}}%
\pgfpathlineto{\pgfqpoint{5.400267in}{2.929319in}}%
\pgfusepath{stroke}%
\end{pgfscope}%
\begin{pgfscope}%
\pgfpathrectangle{\pgfqpoint{3.985294in}{1.750000in}}{\pgfqpoint{2.279412in}{2.004545in}}%
\pgfusepath{clip}%
\pgfsetbuttcap%
\pgfsetroundjoin%
\pgfsetlinewidth{0.620817pt}%
\definecolor{currentstroke}{rgb}{0.262138,0.242286,0.520837}%
\pgfsetstrokecolor{currentstroke}%
\pgfsetdash{}{0pt}%
\pgfpathmoveto{\pgfqpoint{5.400267in}{2.929319in}}%
\pgfpathlineto{\pgfqpoint{5.350173in}{2.927264in}}%
\pgfusepath{stroke}%
\end{pgfscope}%
\begin{pgfscope}%
\pgfpathrectangle{\pgfqpoint{3.985294in}{1.750000in}}{\pgfqpoint{2.279412in}{2.004545in}}%
\pgfusepath{clip}%
\pgfsetbuttcap%
\pgfsetroundjoin%
\pgfsetlinewidth{0.649719pt}%
\definecolor{currentstroke}{rgb}{0.255645,0.260703,0.528312}%
\pgfsetstrokecolor{currentstroke}%
\pgfsetdash{}{0pt}%
\pgfpathmoveto{\pgfqpoint{5.350173in}{2.927264in}}%
\pgfpathlineto{\pgfqpoint{5.300121in}{2.924508in}}%
\pgfusepath{stroke}%
\end{pgfscope}%
\begin{pgfscope}%
\pgfpathrectangle{\pgfqpoint{3.985294in}{1.750000in}}{\pgfqpoint{2.279412in}{2.004545in}}%
\pgfusepath{clip}%
\pgfsetbuttcap%
\pgfsetroundjoin%
\pgfsetlinewidth{0.746195pt}%
\definecolor{currentstroke}{rgb}{0.229739,0.322361,0.545706}%
\pgfsetstrokecolor{currentstroke}%
\pgfsetdash{}{0pt}%
\pgfpathmoveto{\pgfqpoint{5.300121in}{2.924508in}}%
\pgfpathlineto{\pgfqpoint{5.250138in}{2.920950in}}%
\pgfusepath{stroke}%
\end{pgfscope}%
\begin{pgfscope}%
\pgfpathrectangle{\pgfqpoint{3.985294in}{1.750000in}}{\pgfqpoint{2.279412in}{2.004545in}}%
\pgfusepath{clip}%
\pgfsetbuttcap%
\pgfsetroundjoin%
\pgfsetlinewidth{0.774500pt}%
\definecolor{currentstroke}{rgb}{0.221989,0.339161,0.548752}%
\pgfsetstrokecolor{currentstroke}%
\pgfsetdash{}{0pt}%
\pgfpathmoveto{\pgfqpoint{5.250138in}{2.920950in}}%
\pgfpathlineto{\pgfqpoint{5.200269in}{2.916324in}}%
\pgfusepath{stroke}%
\end{pgfscope}%
\begin{pgfscope}%
\pgfpathrectangle{\pgfqpoint{3.985294in}{1.750000in}}{\pgfqpoint{2.279412in}{2.004545in}}%
\pgfusepath{clip}%
\pgfsetbuttcap%
\pgfsetroundjoin%
\pgfsetlinewidth{0.780271pt}%
\definecolor{currentstroke}{rgb}{0.220057,0.343307,0.549413}%
\pgfsetstrokecolor{currentstroke}%
\pgfsetdash{}{0pt}%
\pgfpathmoveto{\pgfqpoint{5.200269in}{2.916324in}}%
\pgfpathlineto{\pgfqpoint{5.150579in}{2.910413in}}%
\pgfusepath{stroke}%
\end{pgfscope}%
\begin{pgfscope}%
\pgfpathrectangle{\pgfqpoint{3.985294in}{1.750000in}}{\pgfqpoint{2.279412in}{2.004545in}}%
\pgfusepath{clip}%
\pgfsetbuttcap%
\pgfsetroundjoin%
\pgfsetlinewidth{0.795525pt}%
\definecolor{currentstroke}{rgb}{0.214298,0.355619,0.551184}%
\pgfsetstrokecolor{currentstroke}%
\pgfsetdash{}{0pt}%
\pgfpathmoveto{\pgfqpoint{5.150579in}{2.910413in}}%
\pgfpathlineto{\pgfqpoint{5.101171in}{2.902916in}}%
\pgfusepath{stroke}%
\end{pgfscope}%
\begin{pgfscope}%
\pgfpathrectangle{\pgfqpoint{3.985294in}{1.750000in}}{\pgfqpoint{2.279412in}{2.004545in}}%
\pgfusepath{clip}%
\pgfsetbuttcap%
\pgfsetroundjoin%
\pgfsetlinewidth{0.812628pt}%
\definecolor{currentstroke}{rgb}{0.210503,0.363727,0.552206}%
\pgfsetstrokecolor{currentstroke}%
\pgfsetdash{}{0pt}%
\pgfpathmoveto{\pgfqpoint{5.101171in}{2.902916in}}%
\pgfpathlineto{\pgfqpoint{5.052093in}{2.893890in}}%
\pgfusepath{stroke}%
\end{pgfscope}%
\begin{pgfscope}%
\pgfpathrectangle{\pgfqpoint{3.985294in}{1.750000in}}{\pgfqpoint{2.279412in}{2.004545in}}%
\pgfusepath{clip}%
\pgfsetbuttcap%
\pgfsetroundjoin%
\pgfsetlinewidth{0.835888pt}%
\definecolor{currentstroke}{rgb}{0.204903,0.375746,0.553533}%
\pgfsetstrokecolor{currentstroke}%
\pgfsetdash{}{0pt}%
\pgfpathmoveto{\pgfqpoint{5.052093in}{2.893890in}}%
\pgfpathlineto{\pgfqpoint{5.003400in}{2.883385in}}%
\pgfusepath{stroke}%
\end{pgfscope}%
\begin{pgfscope}%
\pgfpathrectangle{\pgfqpoint{3.985294in}{1.750000in}}{\pgfqpoint{2.279412in}{2.004545in}}%
\pgfusepath{clip}%
\pgfsetbuttcap%
\pgfsetroundjoin%
\pgfsetlinewidth{0.816415pt}%
\definecolor{currentstroke}{rgb}{0.208623,0.367752,0.552675}%
\pgfsetstrokecolor{currentstroke}%
\pgfsetdash{}{0pt}%
\pgfpathmoveto{\pgfqpoint{5.003400in}{2.883385in}}%
\pgfpathlineto{\pgfqpoint{4.955208in}{2.871241in}}%
\pgfusepath{stroke}%
\end{pgfscope}%
\begin{pgfscope}%
\pgfpathrectangle{\pgfqpoint{3.985294in}{1.750000in}}{\pgfqpoint{2.279412in}{2.004545in}}%
\pgfusepath{clip}%
\pgfsetbuttcap%
\pgfsetroundjoin%
\pgfsetlinewidth{0.824940pt}%
\definecolor{currentstroke}{rgb}{0.206756,0.371758,0.553117}%
\pgfsetstrokecolor{currentstroke}%
\pgfsetdash{}{0pt}%
\pgfpathmoveto{\pgfqpoint{4.955208in}{2.871241in}}%
\pgfpathlineto{\pgfqpoint{4.907641in}{2.857360in}}%
\pgfusepath{stroke}%
\end{pgfscope}%
\begin{pgfscope}%
\pgfpathrectangle{\pgfqpoint{3.985294in}{1.750000in}}{\pgfqpoint{2.279412in}{2.004545in}}%
\pgfusepath{clip}%
\pgfsetbuttcap%
\pgfsetroundjoin%
\pgfsetlinewidth{0.806710pt}%
\definecolor{currentstroke}{rgb}{0.212395,0.359683,0.551710}%
\pgfsetstrokecolor{currentstroke}%
\pgfsetdash{}{0pt}%
\pgfpathmoveto{\pgfqpoint{4.907641in}{2.857360in}}%
\pgfpathlineto{\pgfqpoint{4.861187in}{2.841034in}}%
\pgfusepath{stroke}%
\end{pgfscope}%
\begin{pgfscope}%
\pgfpathrectangle{\pgfqpoint{3.985294in}{1.750000in}}{\pgfqpoint{2.279412in}{2.004545in}}%
\pgfusepath{clip}%
\pgfsetbuttcap%
\pgfsetroundjoin%
\pgfsetlinewidth{0.759001pt}%
\definecolor{currentstroke}{rgb}{0.225863,0.330805,0.547314}%
\pgfsetstrokecolor{currentstroke}%
\pgfsetdash{}{0pt}%
\pgfpathmoveto{\pgfqpoint{4.861187in}{2.841034in}}%
\pgfpathlineto{\pgfqpoint{4.816088in}{2.822174in}}%
\pgfusepath{stroke}%
\end{pgfscope}%
\begin{pgfscope}%
\pgfpathrectangle{\pgfqpoint{3.985294in}{1.750000in}}{\pgfqpoint{2.279412in}{2.004545in}}%
\pgfusepath{clip}%
\pgfsetbuttcap%
\pgfsetroundjoin%
\pgfsetlinewidth{0.760565pt}%
\definecolor{currentstroke}{rgb}{0.225863,0.330805,0.547314}%
\pgfsetstrokecolor{currentstroke}%
\pgfsetdash{}{0pt}%
\pgfpathmoveto{\pgfqpoint{4.816088in}{2.822174in}}%
\pgfpathlineto{\pgfqpoint{4.772151in}{2.801353in}}%
\pgfusepath{stroke}%
\end{pgfscope}%
\begin{pgfscope}%
\pgfpathrectangle{\pgfqpoint{3.985294in}{1.750000in}}{\pgfqpoint{2.279412in}{2.004545in}}%
\pgfusepath{clip}%
\pgfsetbuttcap%
\pgfsetroundjoin%
\pgfsetlinewidth{0.666970pt}%
\definecolor{currentstroke}{rgb}{0.250425,0.274290,0.533103}%
\pgfsetstrokecolor{currentstroke}%
\pgfsetdash{}{0pt}%
\pgfpathmoveto{\pgfqpoint{4.772151in}{2.801353in}}%
\pgfpathlineto{\pgfqpoint{4.772151in}{2.801353in}}%
\pgfusepath{stroke}%
\end{pgfscope}%
\begin{pgfscope}%
\pgfpathrectangle{\pgfqpoint{3.985294in}{1.750000in}}{\pgfqpoint{2.279412in}{2.004545in}}%
\pgfusepath{clip}%
\pgfsetbuttcap%
\pgfsetroundjoin%
\pgfsetlinewidth{0.666970pt}%
\definecolor{currentstroke}{rgb}{0.250425,0.274290,0.533103}%
\pgfsetstrokecolor{currentstroke}%
\pgfsetdash{}{0pt}%
\pgfpathmoveto{\pgfqpoint{4.772151in}{2.801353in}}%
\pgfpathlineto{\pgfqpoint{4.738735in}{2.780393in}}%
\pgfusepath{stroke}%
\end{pgfscope}%
\begin{pgfscope}%
\pgfpathrectangle{\pgfqpoint{3.985294in}{1.750000in}}{\pgfqpoint{2.279412in}{2.004545in}}%
\pgfusepath{clip}%
\pgfsetbuttcap%
\pgfsetroundjoin%
\pgfsetlinewidth{0.314052pt}%
\definecolor{currentstroke}{rgb}{0.268510,0.009605,0.335427}%
\pgfsetstrokecolor{currentstroke}%
\pgfsetdash{}{0pt}%
\pgfpathmoveto{\pgfqpoint{5.421579in}{2.039108in}}%
\pgfpathlineto{\pgfqpoint{5.372191in}{2.044932in}}%
\pgfusepath{stroke}%
\end{pgfscope}%
\begin{pgfscope}%
\pgfpathrectangle{\pgfqpoint{3.985294in}{1.750000in}}{\pgfqpoint{2.279412in}{2.004545in}}%
\pgfusepath{clip}%
\pgfsetbuttcap%
\pgfsetroundjoin%
\pgfsetlinewidth{0.338796pt}%
\definecolor{currentstroke}{rgb}{0.273809,0.031497,0.358853}%
\pgfsetstrokecolor{currentstroke}%
\pgfsetdash{}{0pt}%
\pgfpathmoveto{\pgfqpoint{5.372191in}{2.044932in}}%
\pgfpathlineto{\pgfqpoint{5.322332in}{2.048726in}}%
\pgfusepath{stroke}%
\end{pgfscope}%
\begin{pgfscope}%
\pgfpathrectangle{\pgfqpoint{3.985294in}{1.750000in}}{\pgfqpoint{2.279412in}{2.004545in}}%
\pgfusepath{clip}%
\pgfsetbuttcap%
\pgfsetroundjoin%
\pgfsetlinewidth{0.331335pt}%
\definecolor{currentstroke}{rgb}{0.272594,0.025563,0.353093}%
\pgfsetstrokecolor{currentstroke}%
\pgfsetdash{}{0pt}%
\pgfpathmoveto{\pgfqpoint{5.322332in}{2.048726in}}%
\pgfpathlineto{\pgfqpoint{5.272660in}{2.053255in}}%
\pgfusepath{stroke}%
\end{pgfscope}%
\begin{pgfscope}%
\pgfpathrectangle{\pgfqpoint{3.985294in}{1.750000in}}{\pgfqpoint{2.279412in}{2.004545in}}%
\pgfusepath{clip}%
\pgfsetbuttcap%
\pgfsetroundjoin%
\pgfsetlinewidth{0.342152pt}%
\definecolor{currentstroke}{rgb}{0.273809,0.031497,0.358853}%
\pgfsetstrokecolor{currentstroke}%
\pgfsetdash{}{0pt}%
\pgfpathmoveto{\pgfqpoint{5.272660in}{2.053255in}}%
\pgfpathlineto{\pgfqpoint{5.224136in}{2.063320in}}%
\pgfusepath{stroke}%
\end{pgfscope}%
\begin{pgfscope}%
\pgfpathrectangle{\pgfqpoint{3.985294in}{1.750000in}}{\pgfqpoint{2.279412in}{2.004545in}}%
\pgfusepath{clip}%
\pgfsetbuttcap%
\pgfsetroundjoin%
\pgfsetlinewidth{0.330342pt}%
\definecolor{currentstroke}{rgb}{0.272594,0.025563,0.353093}%
\pgfsetstrokecolor{currentstroke}%
\pgfsetdash{}{0pt}%
\pgfpathmoveto{\pgfqpoint{5.224136in}{2.063320in}}%
\pgfpathlineto{\pgfqpoint{5.176292in}{2.075671in}}%
\pgfusepath{stroke}%
\end{pgfscope}%
\begin{pgfscope}%
\pgfpathrectangle{\pgfqpoint{3.985294in}{1.750000in}}{\pgfqpoint{2.279412in}{2.004545in}}%
\pgfusepath{clip}%
\pgfsetbuttcap%
\pgfsetroundjoin%
\pgfsetlinewidth{0.324495pt}%
\definecolor{currentstroke}{rgb}{0.271305,0.019942,0.347269}%
\pgfsetstrokecolor{currentstroke}%
\pgfsetdash{}{0pt}%
\pgfpathmoveto{\pgfqpoint{5.894378in}{2.301205in}}%
\pgfpathlineto{\pgfqpoint{5.844239in}{2.301848in}}%
\pgfusepath{stroke}%
\end{pgfscope}%
\begin{pgfscope}%
\pgfpathrectangle{\pgfqpoint{3.985294in}{1.750000in}}{\pgfqpoint{2.279412in}{2.004545in}}%
\pgfusepath{clip}%
\pgfsetbuttcap%
\pgfsetroundjoin%
\pgfsetlinewidth{0.315274pt}%
\definecolor{currentstroke}{rgb}{0.269944,0.014625,0.341379}%
\pgfsetstrokecolor{currentstroke}%
\pgfsetdash{}{0pt}%
\pgfpathmoveto{\pgfqpoint{5.844239in}{2.301848in}}%
\pgfpathlineto{\pgfqpoint{5.794175in}{2.301503in}}%
\pgfusepath{stroke}%
\end{pgfscope}%
\begin{pgfscope}%
\pgfpathrectangle{\pgfqpoint{3.985294in}{1.750000in}}{\pgfqpoint{2.279412in}{2.004545in}}%
\pgfusepath{clip}%
\pgfsetbuttcap%
\pgfsetroundjoin%
\pgfsetlinewidth{0.316420pt}%
\definecolor{currentstroke}{rgb}{0.269944,0.014625,0.341379}%
\pgfsetstrokecolor{currentstroke}%
\pgfsetdash{}{0pt}%
\pgfpathmoveto{\pgfqpoint{5.794175in}{2.301503in}}%
\pgfpathlineto{\pgfqpoint{5.744114in}{2.300200in}}%
\pgfusepath{stroke}%
\end{pgfscope}%
\begin{pgfscope}%
\pgfpathrectangle{\pgfqpoint{3.985294in}{1.750000in}}{\pgfqpoint{2.279412in}{2.004545in}}%
\pgfusepath{clip}%
\pgfsetbuttcap%
\pgfsetroundjoin%
\pgfsetlinewidth{0.323885pt}%
\definecolor{currentstroke}{rgb}{0.271305,0.019942,0.347269}%
\pgfsetstrokecolor{currentstroke}%
\pgfsetdash{}{0pt}%
\pgfpathmoveto{\pgfqpoint{5.744114in}{2.300200in}}%
\pgfpathlineto{\pgfqpoint{5.694002in}{2.300757in}}%
\pgfusepath{stroke}%
\end{pgfscope}%
\begin{pgfscope}%
\pgfpathrectangle{\pgfqpoint{3.985294in}{1.750000in}}{\pgfqpoint{2.279412in}{2.004545in}}%
\pgfusepath{clip}%
\pgfsetbuttcap%
\pgfsetroundjoin%
\pgfsetlinewidth{0.329138pt}%
\definecolor{currentstroke}{rgb}{0.272594,0.025563,0.353093}%
\pgfsetstrokecolor{currentstroke}%
\pgfsetdash{}{0pt}%
\pgfpathmoveto{\pgfqpoint{5.694002in}{2.300757in}}%
\pgfpathlineto{\pgfqpoint{5.643889in}{2.302388in}}%
\pgfusepath{stroke}%
\end{pgfscope}%
\begin{pgfscope}%
\pgfpathrectangle{\pgfqpoint{3.985294in}{1.750000in}}{\pgfqpoint{2.279412in}{2.004545in}}%
\pgfusepath{clip}%
\pgfsetbuttcap%
\pgfsetroundjoin%
\pgfsetlinewidth{0.340950pt}%
\definecolor{currentstroke}{rgb}{0.273809,0.031497,0.358853}%
\pgfsetstrokecolor{currentstroke}%
\pgfsetdash{}{0pt}%
\pgfpathmoveto{\pgfqpoint{5.643889in}{2.302388in}}%
\pgfpathlineto{\pgfqpoint{5.593762in}{2.303726in}}%
\pgfusepath{stroke}%
\end{pgfscope}%
\begin{pgfscope}%
\pgfpathrectangle{\pgfqpoint{3.985294in}{1.750000in}}{\pgfqpoint{2.279412in}{2.004545in}}%
\pgfusepath{clip}%
\pgfsetbuttcap%
\pgfsetroundjoin%
\pgfsetlinewidth{0.352242pt}%
\definecolor{currentstroke}{rgb}{0.276022,0.044167,0.370164}%
\pgfsetstrokecolor{currentstroke}%
\pgfsetdash{}{0pt}%
\pgfpathmoveto{\pgfqpoint{5.593762in}{2.303726in}}%
\pgfpathlineto{\pgfqpoint{5.543659in}{2.305508in}}%
\pgfusepath{stroke}%
\end{pgfscope}%
\begin{pgfscope}%
\pgfpathrectangle{\pgfqpoint{3.985294in}{1.750000in}}{\pgfqpoint{2.279412in}{2.004545in}}%
\pgfusepath{clip}%
\pgfsetbuttcap%
\pgfsetroundjoin%
\pgfsetlinewidth{0.363049pt}%
\definecolor{currentstroke}{rgb}{0.277941,0.056324,0.381191}%
\pgfsetstrokecolor{currentstroke}%
\pgfsetdash{}{0pt}%
\pgfpathmoveto{\pgfqpoint{5.543659in}{2.305508in}}%
\pgfpathlineto{\pgfqpoint{5.493589in}{2.307943in}}%
\pgfusepath{stroke}%
\end{pgfscope}%
\begin{pgfscope}%
\pgfpathrectangle{\pgfqpoint{3.985294in}{1.750000in}}{\pgfqpoint{2.279412in}{2.004545in}}%
\pgfusepath{clip}%
\pgfsetbuttcap%
\pgfsetroundjoin%
\pgfsetlinewidth{0.387669pt}%
\definecolor{currentstroke}{rgb}{0.280267,0.073417,0.397163}%
\pgfsetstrokecolor{currentstroke}%
\pgfsetdash{}{0pt}%
\pgfpathmoveto{\pgfqpoint{5.493589in}{2.307943in}}%
\pgfpathlineto{\pgfqpoint{5.443575in}{2.311089in}}%
\pgfusepath{stroke}%
\end{pgfscope}%
\begin{pgfscope}%
\pgfpathrectangle{\pgfqpoint{3.985294in}{1.750000in}}{\pgfqpoint{2.279412in}{2.004545in}}%
\pgfusepath{clip}%
\pgfsetbuttcap%
\pgfsetroundjoin%
\pgfsetlinewidth{0.422992pt}%
\definecolor{currentstroke}{rgb}{0.282656,0.100196,0.422160}%
\pgfsetstrokecolor{currentstroke}%
\pgfsetdash{}{0pt}%
\pgfpathmoveto{\pgfqpoint{5.443575in}{2.311089in}}%
\pgfpathlineto{\pgfqpoint{5.393643in}{2.315192in}}%
\pgfusepath{stroke}%
\end{pgfscope}%
\begin{pgfscope}%
\pgfpathrectangle{\pgfqpoint{3.985294in}{1.750000in}}{\pgfqpoint{2.279412in}{2.004545in}}%
\pgfusepath{clip}%
\pgfsetbuttcap%
\pgfsetroundjoin%
\pgfsetlinewidth{0.408562pt}%
\definecolor{currentstroke}{rgb}{0.281924,0.089666,0.412415}%
\pgfsetstrokecolor{currentstroke}%
\pgfsetdash{}{0pt}%
\pgfpathmoveto{\pgfqpoint{5.393643in}{2.315192in}}%
\pgfpathlineto{\pgfqpoint{5.343833in}{2.320280in}}%
\pgfusepath{stroke}%
\end{pgfscope}%
\begin{pgfscope}%
\pgfpathrectangle{\pgfqpoint{3.985294in}{1.750000in}}{\pgfqpoint{2.279412in}{2.004545in}}%
\pgfusepath{clip}%
\pgfsetbuttcap%
\pgfsetroundjoin%
\pgfsetlinewidth{0.408493pt}%
\definecolor{currentstroke}{rgb}{0.281924,0.089666,0.412415}%
\pgfsetstrokecolor{currentstroke}%
\pgfsetdash{}{0pt}%
\pgfpathmoveto{\pgfqpoint{5.343833in}{2.320280in}}%
\pgfpathlineto{\pgfqpoint{5.294292in}{2.327040in}}%
\pgfusepath{stroke}%
\end{pgfscope}%
\begin{pgfscope}%
\pgfpathrectangle{\pgfqpoint{3.985294in}{1.750000in}}{\pgfqpoint{2.279412in}{2.004545in}}%
\pgfusepath{clip}%
\pgfsetbuttcap%
\pgfsetroundjoin%
\pgfsetlinewidth{0.424931pt}%
\definecolor{currentstroke}{rgb}{0.282910,0.105393,0.426902}%
\pgfsetstrokecolor{currentstroke}%
\pgfsetdash{}{0pt}%
\pgfpathmoveto{\pgfqpoint{5.294292in}{2.327040in}}%
\pgfpathlineto{\pgfqpoint{5.245188in}{2.335907in}}%
\pgfusepath{stroke}%
\end{pgfscope}%
\begin{pgfscope}%
\pgfpathrectangle{\pgfqpoint{3.985294in}{1.750000in}}{\pgfqpoint{2.279412in}{2.004545in}}%
\pgfusepath{clip}%
\pgfsetbuttcap%
\pgfsetroundjoin%
\pgfsetlinewidth{0.446102pt}%
\definecolor{currentstroke}{rgb}{0.283229,0.120777,0.440584}%
\pgfsetstrokecolor{currentstroke}%
\pgfsetdash{}{0pt}%
\pgfpathmoveto{\pgfqpoint{5.245188in}{2.335907in}}%
\pgfpathlineto{\pgfqpoint{5.196679in}{2.347014in}}%
\pgfusepath{stroke}%
\end{pgfscope}%
\begin{pgfscope}%
\pgfpathrectangle{\pgfqpoint{3.985294in}{1.750000in}}{\pgfqpoint{2.279412in}{2.004545in}}%
\pgfusepath{clip}%
\pgfsetbuttcap%
\pgfsetroundjoin%
\pgfsetlinewidth{0.417045pt}%
\definecolor{currentstroke}{rgb}{0.282327,0.094955,0.417331}%
\pgfsetstrokecolor{currentstroke}%
\pgfsetdash{}{0pt}%
\pgfpathmoveto{\pgfqpoint{5.196679in}{2.347014in}}%
\pgfpathlineto{\pgfqpoint{5.148593in}{2.359507in}}%
\pgfusepath{stroke}%
\end{pgfscope}%
\begin{pgfscope}%
\pgfpathrectangle{\pgfqpoint{3.985294in}{1.750000in}}{\pgfqpoint{2.279412in}{2.004545in}}%
\pgfusepath{clip}%
\pgfsetbuttcap%
\pgfsetroundjoin%
\pgfsetlinewidth{0.447524pt}%
\definecolor{currentstroke}{rgb}{0.283229,0.120777,0.440584}%
\pgfsetstrokecolor{currentstroke}%
\pgfsetdash{}{0pt}%
\pgfpathmoveto{\pgfqpoint{5.148593in}{2.359507in}}%
\pgfpathlineto{\pgfqpoint{5.101132in}{2.373642in}}%
\pgfusepath{stroke}%
\end{pgfscope}%
\begin{pgfscope}%
\pgfpathrectangle{\pgfqpoint{3.985294in}{1.750000in}}{\pgfqpoint{2.279412in}{2.004545in}}%
\pgfusepath{clip}%
\pgfsetbuttcap%
\pgfsetroundjoin%
\pgfsetlinewidth{0.479036pt}%
\definecolor{currentstroke}{rgb}{0.282623,0.140926,0.457517}%
\pgfsetstrokecolor{currentstroke}%
\pgfsetdash{}{0pt}%
\pgfpathmoveto{\pgfqpoint{5.101132in}{2.373642in}}%
\pgfpathlineto{\pgfqpoint{5.056514in}{2.393137in}}%
\pgfusepath{stroke}%
\end{pgfscope}%
\begin{pgfscope}%
\pgfpathrectangle{\pgfqpoint{3.985294in}{1.750000in}}{\pgfqpoint{2.279412in}{2.004545in}}%
\pgfusepath{clip}%
\pgfsetbuttcap%
\pgfsetroundjoin%
\pgfsetlinewidth{0.550522pt}%
\definecolor{currentstroke}{rgb}{0.275191,0.194905,0.496005}%
\pgfsetstrokecolor{currentstroke}%
\pgfsetdash{}{0pt}%
\pgfpathmoveto{\pgfqpoint{5.056514in}{2.393137in}}%
\pgfpathlineto{\pgfqpoint{5.018519in}{2.417581in}}%
\pgfusepath{stroke}%
\end{pgfscope}%
\begin{pgfscope}%
\pgfpathrectangle{\pgfqpoint{3.985294in}{1.750000in}}{\pgfqpoint{2.279412in}{2.004545in}}%
\pgfusepath{clip}%
\pgfsetbuttcap%
\pgfsetroundjoin%
\pgfsetlinewidth{0.513729pt}%
\definecolor{currentstroke}{rgb}{0.280255,0.165693,0.476498}%
\pgfsetstrokecolor{currentstroke}%
\pgfsetdash{}{0pt}%
\pgfpathmoveto{\pgfqpoint{5.018519in}{2.417581in}}%
\pgfpathlineto{\pgfqpoint{4.979486in}{2.445020in}}%
\pgfusepath{stroke}%
\end{pgfscope}%
\begin{pgfscope}%
\pgfpathrectangle{\pgfqpoint{3.985294in}{1.750000in}}{\pgfqpoint{2.279412in}{2.004545in}}%
\pgfusepath{clip}%
\pgfsetbuttcap%
\pgfsetroundjoin%
\pgfsetlinewidth{0.588135pt}%
\definecolor{currentstroke}{rgb}{0.269308,0.218818,0.509577}%
\pgfsetstrokecolor{currentstroke}%
\pgfsetdash{}{0pt}%
\pgfpathmoveto{\pgfqpoint{4.979486in}{2.445020in}}%
\pgfpathlineto{\pgfqpoint{4.942550in}{2.474667in}}%
\pgfusepath{stroke}%
\end{pgfscope}%
\begin{pgfscope}%
\pgfpathrectangle{\pgfqpoint{3.985294in}{1.750000in}}{\pgfqpoint{2.279412in}{2.004545in}}%
\pgfusepath{clip}%
\pgfsetbuttcap%
\pgfsetroundjoin%
\pgfsetlinewidth{0.652013pt}%
\definecolor{currentstroke}{rgb}{0.253935,0.265254,0.529983}%
\pgfsetstrokecolor{currentstroke}%
\pgfsetdash{}{0pt}%
\pgfpathmoveto{\pgfqpoint{4.942550in}{2.474667in}}%
\pgfpathlineto{\pgfqpoint{4.906305in}{2.505114in}}%
\pgfusepath{stroke}%
\end{pgfscope}%
\begin{pgfscope}%
\pgfpathrectangle{\pgfqpoint{3.985294in}{1.750000in}}{\pgfqpoint{2.279412in}{2.004545in}}%
\pgfusepath{clip}%
\pgfsetbuttcap%
\pgfsetroundjoin%
\pgfsetlinewidth{0.747438pt}%
\definecolor{currentstroke}{rgb}{0.227802,0.326594,0.546532}%
\pgfsetstrokecolor{currentstroke}%
\pgfsetdash{}{0pt}%
\pgfpathmoveto{\pgfqpoint{4.906305in}{2.505114in}}%
\pgfpathlineto{\pgfqpoint{4.869869in}{2.535346in}}%
\pgfusepath{stroke}%
\end{pgfscope}%
\begin{pgfscope}%
\pgfpathrectangle{\pgfqpoint{3.985294in}{1.750000in}}{\pgfqpoint{2.279412in}{2.004545in}}%
\pgfusepath{clip}%
\pgfsetbuttcap%
\pgfsetroundjoin%
\pgfsetlinewidth{0.849848pt}%
\definecolor{currentstroke}{rgb}{0.201239,0.383670,0.554294}%
\pgfsetstrokecolor{currentstroke}%
\pgfsetdash{}{0pt}%
\pgfpathmoveto{\pgfqpoint{4.869869in}{2.535346in}}%
\pgfpathlineto{\pgfqpoint{4.834601in}{2.566596in}}%
\pgfusepath{stroke}%
\end{pgfscope}%
\begin{pgfscope}%
\pgfpathrectangle{\pgfqpoint{3.985294in}{1.750000in}}{\pgfqpoint{2.279412in}{2.004545in}}%
\pgfusepath{clip}%
\pgfsetbuttcap%
\pgfsetroundjoin%
\pgfsetlinewidth{0.784452pt}%
\definecolor{currentstroke}{rgb}{0.218130,0.347432,0.550038}%
\pgfsetstrokecolor{currentstroke}%
\pgfsetdash{}{0pt}%
\pgfpathmoveto{\pgfqpoint{4.834601in}{2.566596in}}%
\pgfpathlineto{\pgfqpoint{4.801066in}{2.598739in}}%
\pgfusepath{stroke}%
\end{pgfscope}%
\begin{pgfscope}%
\pgfpathrectangle{\pgfqpoint{3.985294in}{1.750000in}}{\pgfqpoint{2.279412in}{2.004545in}}%
\pgfusepath{clip}%
\pgfsetbuttcap%
\pgfsetroundjoin%
\pgfsetlinewidth{0.730644pt}%
\definecolor{currentstroke}{rgb}{0.233603,0.313828,0.543914}%
\pgfsetstrokecolor{currentstroke}%
\pgfsetdash{}{0pt}%
\pgfpathmoveto{\pgfqpoint{4.801066in}{2.598739in}}%
\pgfpathlineto{\pgfqpoint{4.766098in}{2.629610in}}%
\pgfusepath{stroke}%
\end{pgfscope}%
\begin{pgfscope}%
\pgfpathrectangle{\pgfqpoint{3.985294in}{1.750000in}}{\pgfqpoint{2.279412in}{2.004545in}}%
\pgfusepath{clip}%
\pgfsetbuttcap%
\pgfsetroundjoin%
\pgfsetlinewidth{0.740024pt}%
\definecolor{currentstroke}{rgb}{0.231674,0.318106,0.544834}%
\pgfsetstrokecolor{currentstroke}%
\pgfsetdash{}{0pt}%
\pgfpathmoveto{\pgfqpoint{4.766098in}{2.629610in}}%
\pgfpathlineto{\pgfqpoint{4.736219in}{2.662479in}}%
\pgfusepath{stroke}%
\end{pgfscope}%
\begin{pgfscope}%
\pgfpathrectangle{\pgfqpoint{3.985294in}{1.750000in}}{\pgfqpoint{2.279412in}{2.004545in}}%
\pgfusepath{clip}%
\pgfsetbuttcap%
\pgfsetroundjoin%
\pgfsetlinewidth{0.677603pt}%
\definecolor{currentstroke}{rgb}{0.248629,0.278775,0.534556}%
\pgfsetstrokecolor{currentstroke}%
\pgfsetdash{}{0pt}%
\pgfpathmoveto{\pgfqpoint{4.736219in}{2.662479in}}%
\pgfpathlineto{\pgfqpoint{4.707857in}{2.689638in}}%
\pgfusepath{stroke}%
\end{pgfscope}%
\begin{pgfscope}%
\pgfpathrectangle{\pgfqpoint{3.985294in}{1.750000in}}{\pgfqpoint{2.279412in}{2.004545in}}%
\pgfusepath{clip}%
\pgfsetbuttcap%
\pgfsetroundjoin%
\pgfsetlinewidth{0.532116pt}%
\definecolor{currentstroke}{rgb}{0.278012,0.180367,0.486697}%
\pgfsetstrokecolor{currentstroke}%
\pgfsetdash{}{0pt}%
\pgfpathmoveto{\pgfqpoint{4.707857in}{2.689638in}}%
\pgfpathlineto{\pgfqpoint{4.707857in}{2.689638in}}%
\pgfusepath{stroke}%
\end{pgfscope}%
\begin{pgfscope}%
\pgfpathrectangle{\pgfqpoint{3.985294in}{1.750000in}}{\pgfqpoint{2.279412in}{2.004545in}}%
\pgfusepath{clip}%
\pgfsetbuttcap%
\pgfsetroundjoin%
\pgfsetlinewidth{0.532116pt}%
\definecolor{currentstroke}{rgb}{0.278012,0.180367,0.486697}%
\pgfsetstrokecolor{currentstroke}%
\pgfsetdash{}{0pt}%
\pgfpathmoveto{\pgfqpoint{4.707857in}{2.689638in}}%
\pgfpathlineto{\pgfqpoint{4.696138in}{2.700163in}}%
\pgfusepath{stroke}%
\end{pgfscope}%
\begin{pgfscope}%
\pgfpathrectangle{\pgfqpoint{3.985294in}{1.750000in}}{\pgfqpoint{2.279412in}{2.004545in}}%
\pgfusepath{clip}%
\pgfsetbuttcap%
\pgfsetroundjoin%
\pgfsetlinewidth{0.499655pt}%
\definecolor{currentstroke}{rgb}{0.281412,0.155834,0.469201}%
\pgfsetstrokecolor{currentstroke}%
\pgfsetdash{}{0pt}%
\pgfpathmoveto{\pgfqpoint{4.696138in}{2.700163in}}%
\pgfpathlineto{\pgfqpoint{4.696138in}{2.700163in}}%
\pgfusepath{stroke}%
\end{pgfscope}%
\begin{pgfscope}%
\pgfpathrectangle{\pgfqpoint{3.985294in}{1.750000in}}{\pgfqpoint{2.279412in}{2.004545in}}%
\pgfusepath{clip}%
\pgfsetbuttcap%
\pgfsetroundjoin%
\pgfsetlinewidth{0.499655pt}%
\definecolor{currentstroke}{rgb}{0.281412,0.155834,0.469201}%
\pgfsetstrokecolor{currentstroke}%
\pgfsetdash{}{0pt}%
\pgfpathmoveto{\pgfqpoint{4.696138in}{2.700163in}}%
\pgfpathlineto{\pgfqpoint{4.696138in}{2.700163in}}%
\pgfusepath{stroke}%
\end{pgfscope}%
\begin{pgfscope}%
\pgfpathrectangle{\pgfqpoint{3.985294in}{1.750000in}}{\pgfqpoint{2.279412in}{2.004545in}}%
\pgfusepath{clip}%
\pgfsetbuttcap%
\pgfsetroundjoin%
\pgfsetlinewidth{0.499655pt}%
\definecolor{currentstroke}{rgb}{0.281412,0.155834,0.469201}%
\pgfsetstrokecolor{currentstroke}%
\pgfsetdash{}{0pt}%
\pgfpathmoveto{\pgfqpoint{4.696138in}{2.700163in}}%
\pgfpathlineto{\pgfqpoint{4.691957in}{2.707254in}}%
\pgfusepath{stroke}%
\end{pgfscope}%
\begin{pgfscope}%
\pgfpathrectangle{\pgfqpoint{3.985294in}{1.750000in}}{\pgfqpoint{2.279412in}{2.004545in}}%
\pgfusepath{clip}%
\pgfsetbuttcap%
\pgfsetroundjoin%
\pgfsetlinewidth{0.439413pt}%
\definecolor{currentstroke}{rgb}{0.283197,0.115680,0.436115}%
\pgfsetstrokecolor{currentstroke}%
\pgfsetdash{}{0pt}%
\pgfpathmoveto{\pgfqpoint{4.691957in}{2.707254in}}%
\pgfpathlineto{\pgfqpoint{4.690574in}{2.713247in}}%
\pgfusepath{stroke}%
\end{pgfscope}%
\begin{pgfscope}%
\pgfpathrectangle{\pgfqpoint{3.985294in}{1.750000in}}{\pgfqpoint{2.279412in}{2.004545in}}%
\pgfusepath{clip}%
\pgfsetbuttcap%
\pgfsetroundjoin%
\pgfsetlinewidth{0.328396pt}%
\definecolor{currentstroke}{rgb}{0.271305,0.019942,0.347269}%
\pgfsetstrokecolor{currentstroke}%
\pgfsetdash{}{0pt}%
\pgfpathmoveto{\pgfqpoint{5.894378in}{2.346312in}}%
\pgfpathlineto{\pgfqpoint{5.844332in}{2.346694in}}%
\pgfusepath{stroke}%
\end{pgfscope}%
\begin{pgfscope}%
\pgfpathrectangle{\pgfqpoint{3.985294in}{1.750000in}}{\pgfqpoint{2.279412in}{2.004545in}}%
\pgfusepath{clip}%
\pgfsetbuttcap%
\pgfsetroundjoin%
\pgfsetlinewidth{0.314543pt}%
\definecolor{currentstroke}{rgb}{0.268510,0.009605,0.335427}%
\pgfsetstrokecolor{currentstroke}%
\pgfsetdash{}{0pt}%
\pgfpathmoveto{\pgfqpoint{5.844332in}{2.346694in}}%
\pgfpathlineto{\pgfqpoint{5.794292in}{2.347268in}}%
\pgfusepath{stroke}%
\end{pgfscope}%
\begin{pgfscope}%
\pgfpathrectangle{\pgfqpoint{3.985294in}{1.750000in}}{\pgfqpoint{2.279412in}{2.004545in}}%
\pgfusepath{clip}%
\pgfsetbuttcap%
\pgfsetroundjoin%
\pgfsetlinewidth{0.320653pt}%
\definecolor{currentstroke}{rgb}{0.269944,0.014625,0.341379}%
\pgfsetstrokecolor{currentstroke}%
\pgfsetdash{}{0pt}%
\pgfpathmoveto{\pgfqpoint{5.794292in}{2.347268in}}%
\pgfpathlineto{\pgfqpoint{5.744217in}{2.348905in}}%
\pgfusepath{stroke}%
\end{pgfscope}%
\begin{pgfscope}%
\pgfpathrectangle{\pgfqpoint{3.985294in}{1.750000in}}{\pgfqpoint{2.279412in}{2.004545in}}%
\pgfusepath{clip}%
\pgfsetbuttcap%
\pgfsetroundjoin%
\pgfsetlinewidth{0.325720pt}%
\definecolor{currentstroke}{rgb}{0.271305,0.019942,0.347269}%
\pgfsetstrokecolor{currentstroke}%
\pgfsetdash{}{0pt}%
\pgfpathmoveto{\pgfqpoint{5.744217in}{2.348905in}}%
\pgfpathlineto{\pgfqpoint{5.694140in}{2.350180in}}%
\pgfusepath{stroke}%
\end{pgfscope}%
\begin{pgfscope}%
\pgfpathrectangle{\pgfqpoint{3.985294in}{1.750000in}}{\pgfqpoint{2.279412in}{2.004545in}}%
\pgfusepath{clip}%
\pgfsetbuttcap%
\pgfsetroundjoin%
\pgfsetlinewidth{0.331984pt}%
\definecolor{currentstroke}{rgb}{0.272594,0.025563,0.353093}%
\pgfsetstrokecolor{currentstroke}%
\pgfsetdash{}{0pt}%
\pgfpathmoveto{\pgfqpoint{5.694140in}{2.350180in}}%
\pgfpathlineto{\pgfqpoint{5.644003in}{2.350384in}}%
\pgfusepath{stroke}%
\end{pgfscope}%
\begin{pgfscope}%
\pgfpathrectangle{\pgfqpoint{3.985294in}{1.750000in}}{\pgfqpoint{2.279412in}{2.004545in}}%
\pgfusepath{clip}%
\pgfsetbuttcap%
\pgfsetroundjoin%
\pgfsetlinewidth{0.347773pt}%
\definecolor{currentstroke}{rgb}{0.274952,0.037752,0.364543}%
\pgfsetstrokecolor{currentstroke}%
\pgfsetdash{}{0pt}%
\pgfpathmoveto{\pgfqpoint{5.644003in}{2.350384in}}%
\pgfpathlineto{\pgfqpoint{5.593880in}{2.351748in}}%
\pgfusepath{stroke}%
\end{pgfscope}%
\begin{pgfscope}%
\pgfpathrectangle{\pgfqpoint{3.985294in}{1.750000in}}{\pgfqpoint{2.279412in}{2.004545in}}%
\pgfusepath{clip}%
\pgfsetbuttcap%
\pgfsetroundjoin%
\pgfsetlinewidth{0.366458pt}%
\definecolor{currentstroke}{rgb}{0.277941,0.056324,0.381191}%
\pgfsetstrokecolor{currentstroke}%
\pgfsetdash{}{0pt}%
\pgfpathmoveto{\pgfqpoint{5.593880in}{2.351748in}}%
\pgfpathlineto{\pgfqpoint{5.543776in}{2.353611in}}%
\pgfusepath{stroke}%
\end{pgfscope}%
\begin{pgfscope}%
\pgfpathrectangle{\pgfqpoint{3.985294in}{1.750000in}}{\pgfqpoint{2.279412in}{2.004545in}}%
\pgfusepath{clip}%
\pgfsetbuttcap%
\pgfsetroundjoin%
\pgfsetlinewidth{0.380754pt}%
\definecolor{currentstroke}{rgb}{0.279566,0.067836,0.391917}%
\pgfsetstrokecolor{currentstroke}%
\pgfsetdash{}{0pt}%
\pgfpathmoveto{\pgfqpoint{5.543776in}{2.353611in}}%
\pgfpathlineto{\pgfqpoint{5.493702in}{2.355988in}}%
\pgfusepath{stroke}%
\end{pgfscope}%
\begin{pgfscope}%
\pgfpathrectangle{\pgfqpoint{3.985294in}{1.750000in}}{\pgfqpoint{2.279412in}{2.004545in}}%
\pgfusepath{clip}%
\pgfsetbuttcap%
\pgfsetroundjoin%
\pgfsetlinewidth{0.398451pt}%
\definecolor{currentstroke}{rgb}{0.281446,0.084320,0.407414}%
\pgfsetstrokecolor{currentstroke}%
\pgfsetdash{}{0pt}%
\pgfpathmoveto{\pgfqpoint{5.493702in}{2.355988in}}%
\pgfpathlineto{\pgfqpoint{5.443671in}{2.359040in}}%
\pgfusepath{stroke}%
\end{pgfscope}%
\begin{pgfscope}%
\pgfpathrectangle{\pgfqpoint{3.985294in}{1.750000in}}{\pgfqpoint{2.279412in}{2.004545in}}%
\pgfusepath{clip}%
\pgfsetbuttcap%
\pgfsetroundjoin%
\pgfsetlinewidth{0.420717pt}%
\definecolor{currentstroke}{rgb}{0.282656,0.100196,0.422160}%
\pgfsetstrokecolor{currentstroke}%
\pgfsetdash{}{0pt}%
\pgfpathmoveto{\pgfqpoint{5.443671in}{2.359040in}}%
\pgfpathlineto{\pgfqpoint{5.393725in}{2.362966in}}%
\pgfusepath{stroke}%
\end{pgfscope}%
\begin{pgfscope}%
\pgfpathrectangle{\pgfqpoint{3.985294in}{1.750000in}}{\pgfqpoint{2.279412in}{2.004545in}}%
\pgfusepath{clip}%
\pgfsetbuttcap%
\pgfsetroundjoin%
\pgfsetlinewidth{0.438557pt}%
\definecolor{currentstroke}{rgb}{0.283197,0.115680,0.436115}%
\pgfsetstrokecolor{currentstroke}%
\pgfsetdash{}{0pt}%
\pgfpathmoveto{\pgfqpoint{5.393725in}{2.362966in}}%
\pgfpathlineto{\pgfqpoint{5.343884in}{2.367855in}}%
\pgfusepath{stroke}%
\end{pgfscope}%
\begin{pgfscope}%
\pgfpathrectangle{\pgfqpoint{3.985294in}{1.750000in}}{\pgfqpoint{2.279412in}{2.004545in}}%
\pgfusepath{clip}%
\pgfsetbuttcap%
\pgfsetroundjoin%
\pgfsetlinewidth{0.454391pt}%
\definecolor{currentstroke}{rgb}{0.283187,0.125848,0.444960}%
\pgfsetstrokecolor{currentstroke}%
\pgfsetdash{}{0pt}%
\pgfpathmoveto{\pgfqpoint{5.343884in}{2.367855in}}%
\pgfpathlineto{\pgfqpoint{5.294206in}{2.373777in}}%
\pgfusepath{stroke}%
\end{pgfscope}%
\begin{pgfscope}%
\pgfpathrectangle{\pgfqpoint{3.985294in}{1.750000in}}{\pgfqpoint{2.279412in}{2.004545in}}%
\pgfusepath{clip}%
\pgfsetbuttcap%
\pgfsetroundjoin%
\pgfsetlinewidth{0.470802pt}%
\definecolor{currentstroke}{rgb}{0.282884,0.135920,0.453427}%
\pgfsetstrokecolor{currentstroke}%
\pgfsetdash{}{0pt}%
\pgfpathmoveto{\pgfqpoint{5.294206in}{2.373777in}}%
\pgfpathlineto{\pgfqpoint{5.244873in}{2.381576in}}%
\pgfusepath{stroke}%
\end{pgfscope}%
\begin{pgfscope}%
\pgfpathrectangle{\pgfqpoint{3.985294in}{1.750000in}}{\pgfqpoint{2.279412in}{2.004545in}}%
\pgfusepath{clip}%
\pgfsetbuttcap%
\pgfsetroundjoin%
\pgfsetlinewidth{0.487341pt}%
\definecolor{currentstroke}{rgb}{0.281887,0.150881,0.465405}%
\pgfsetstrokecolor{currentstroke}%
\pgfsetdash{}{0pt}%
\pgfpathmoveto{\pgfqpoint{5.244873in}{2.381576in}}%
\pgfpathlineto{\pgfqpoint{5.195883in}{2.390968in}}%
\pgfusepath{stroke}%
\end{pgfscope}%
\begin{pgfscope}%
\pgfpathrectangle{\pgfqpoint{3.985294in}{1.750000in}}{\pgfqpoint{2.279412in}{2.004545in}}%
\pgfusepath{clip}%
\pgfsetbuttcap%
\pgfsetroundjoin%
\pgfsetlinewidth{0.319573pt}%
\definecolor{currentstroke}{rgb}{0.269944,0.014625,0.341379}%
\pgfsetstrokecolor{currentstroke}%
\pgfsetdash{}{0pt}%
\pgfpathmoveto{\pgfqpoint{5.894378in}{2.391418in}}%
\pgfpathlineto{\pgfqpoint{5.844619in}{2.396179in}}%
\pgfusepath{stroke}%
\end{pgfscope}%
\begin{pgfscope}%
\pgfpathrectangle{\pgfqpoint{3.985294in}{1.750000in}}{\pgfqpoint{2.279412in}{2.004545in}}%
\pgfusepath{clip}%
\pgfsetbuttcap%
\pgfsetroundjoin%
\pgfsetlinewidth{0.311921pt}%
\definecolor{currentstroke}{rgb}{0.268510,0.009605,0.335427}%
\pgfsetstrokecolor{currentstroke}%
\pgfsetdash{}{0pt}%
\pgfpathmoveto{\pgfqpoint{5.844619in}{2.396179in}}%
\pgfpathlineto{\pgfqpoint{5.794793in}{2.398574in}}%
\pgfusepath{stroke}%
\end{pgfscope}%
\begin{pgfscope}%
\pgfpathrectangle{\pgfqpoint{3.985294in}{1.750000in}}{\pgfqpoint{2.279412in}{2.004545in}}%
\pgfusepath{clip}%
\pgfsetbuttcap%
\pgfsetroundjoin%
\pgfsetlinewidth{0.332302pt}%
\definecolor{currentstroke}{rgb}{0.272594,0.025563,0.353093}%
\pgfsetstrokecolor{currentstroke}%
\pgfsetdash{}{0pt}%
\pgfpathmoveto{\pgfqpoint{5.794793in}{2.398574in}}%
\pgfpathlineto{\pgfqpoint{5.744663in}{2.397946in}}%
\pgfusepath{stroke}%
\end{pgfscope}%
\begin{pgfscope}%
\pgfpathrectangle{\pgfqpoint{3.985294in}{1.750000in}}{\pgfqpoint{2.279412in}{2.004545in}}%
\pgfusepath{clip}%
\pgfsetbuttcap%
\pgfsetroundjoin%
\pgfsetlinewidth{0.325429pt}%
\definecolor{currentstroke}{rgb}{0.271305,0.019942,0.347269}%
\pgfsetstrokecolor{currentstroke}%
\pgfsetdash{}{0pt}%
\pgfpathmoveto{\pgfqpoint{5.744663in}{2.397946in}}%
\pgfpathlineto{\pgfqpoint{5.694525in}{2.398525in}}%
\pgfusepath{stroke}%
\end{pgfscope}%
\begin{pgfscope}%
\pgfpathrectangle{\pgfqpoint{3.985294in}{1.750000in}}{\pgfqpoint{2.279412in}{2.004545in}}%
\pgfusepath{clip}%
\pgfsetbuttcap%
\pgfsetroundjoin%
\pgfsetlinewidth{0.333796pt}%
\definecolor{currentstroke}{rgb}{0.272594,0.025563,0.353093}%
\pgfsetstrokecolor{currentstroke}%
\pgfsetdash{}{0pt}%
\pgfpathmoveto{\pgfqpoint{5.694525in}{2.398525in}}%
\pgfpathlineto{\pgfqpoint{5.644388in}{2.399487in}}%
\pgfusepath{stroke}%
\end{pgfscope}%
\begin{pgfscope}%
\pgfpathrectangle{\pgfqpoint{3.985294in}{1.750000in}}{\pgfqpoint{2.279412in}{2.004545in}}%
\pgfusepath{clip}%
\pgfsetbuttcap%
\pgfsetroundjoin%
\pgfsetlinewidth{0.347659pt}%
\definecolor{currentstroke}{rgb}{0.274952,0.037752,0.364543}%
\pgfsetstrokecolor{currentstroke}%
\pgfsetdash{}{0pt}%
\pgfpathmoveto{\pgfqpoint{5.644388in}{2.399487in}}%
\pgfpathlineto{\pgfqpoint{5.594264in}{2.400743in}}%
\pgfusepath{stroke}%
\end{pgfscope}%
\begin{pgfscope}%
\pgfpathrectangle{\pgfqpoint{3.985294in}{1.750000in}}{\pgfqpoint{2.279412in}{2.004545in}}%
\pgfusepath{clip}%
\pgfsetbuttcap%
\pgfsetroundjoin%
\pgfsetlinewidth{0.371600pt}%
\definecolor{currentstroke}{rgb}{0.278791,0.062145,0.386592}%
\pgfsetstrokecolor{currentstroke}%
\pgfsetdash{}{0pt}%
\pgfpathmoveto{\pgfqpoint{5.594264in}{2.400743in}}%
\pgfpathlineto{\pgfqpoint{5.544157in}{2.402494in}}%
\pgfusepath{stroke}%
\end{pgfscope}%
\begin{pgfscope}%
\pgfpathrectangle{\pgfqpoint{3.985294in}{1.750000in}}{\pgfqpoint{2.279412in}{2.004545in}}%
\pgfusepath{clip}%
\pgfsetbuttcap%
\pgfsetroundjoin%
\pgfsetlinewidth{0.408713pt}%
\definecolor{currentstroke}{rgb}{0.281924,0.089666,0.412415}%
\pgfsetstrokecolor{currentstroke}%
\pgfsetdash{}{0pt}%
\pgfpathmoveto{\pgfqpoint{5.544157in}{2.402494in}}%
\pgfpathlineto{\pgfqpoint{5.494073in}{2.404708in}}%
\pgfusepath{stroke}%
\end{pgfscope}%
\begin{pgfscope}%
\pgfpathrectangle{\pgfqpoint{3.985294in}{1.750000in}}{\pgfqpoint{2.279412in}{2.004545in}}%
\pgfusepath{clip}%
\pgfsetbuttcap%
\pgfsetroundjoin%
\pgfsetlinewidth{0.439982pt}%
\definecolor{currentstroke}{rgb}{0.283197,0.115680,0.436115}%
\pgfsetstrokecolor{currentstroke}%
\pgfsetdash{}{0pt}%
\pgfpathmoveto{\pgfqpoint{5.494073in}{2.404708in}}%
\pgfpathlineto{\pgfqpoint{5.444023in}{2.407507in}}%
\pgfusepath{stroke}%
\end{pgfscope}%
\begin{pgfscope}%
\pgfpathrectangle{\pgfqpoint{3.985294in}{1.750000in}}{\pgfqpoint{2.279412in}{2.004545in}}%
\pgfusepath{clip}%
\pgfsetbuttcap%
\pgfsetroundjoin%
\pgfsetlinewidth{0.482866pt}%
\definecolor{currentstroke}{rgb}{0.282290,0.145912,0.461510}%
\pgfsetstrokecolor{currentstroke}%
\pgfsetdash{}{0pt}%
\pgfpathmoveto{\pgfqpoint{5.444023in}{2.407507in}}%
\pgfpathlineto{\pgfqpoint{5.394017in}{2.410836in}}%
\pgfusepath{stroke}%
\end{pgfscope}%
\begin{pgfscope}%
\pgfpathrectangle{\pgfqpoint{3.985294in}{1.750000in}}{\pgfqpoint{2.279412in}{2.004545in}}%
\pgfusepath{clip}%
\pgfsetbuttcap%
\pgfsetroundjoin%
\pgfsetlinewidth{0.488200pt}%
\definecolor{currentstroke}{rgb}{0.281887,0.150881,0.465405}%
\pgfsetstrokecolor{currentstroke}%
\pgfsetdash{}{0pt}%
\pgfpathmoveto{\pgfqpoint{5.394017in}{2.410836in}}%
\pgfpathlineto{\pgfqpoint{5.344141in}{2.415355in}}%
\pgfusepath{stroke}%
\end{pgfscope}%
\begin{pgfscope}%
\pgfpathrectangle{\pgfqpoint{3.985294in}{1.750000in}}{\pgfqpoint{2.279412in}{2.004545in}}%
\pgfusepath{clip}%
\pgfsetbuttcap%
\pgfsetroundjoin%
\pgfsetlinewidth{0.533314pt}%
\definecolor{currentstroke}{rgb}{0.278012,0.180367,0.486697}%
\pgfsetstrokecolor{currentstroke}%
\pgfsetdash{}{0pt}%
\pgfpathmoveto{\pgfqpoint{5.344141in}{2.415355in}}%
\pgfpathlineto{\pgfqpoint{5.294420in}{2.421083in}}%
\pgfusepath{stroke}%
\end{pgfscope}%
\begin{pgfscope}%
\pgfpathrectangle{\pgfqpoint{3.985294in}{1.750000in}}{\pgfqpoint{2.279412in}{2.004545in}}%
\pgfusepath{clip}%
\pgfsetbuttcap%
\pgfsetroundjoin%
\pgfsetlinewidth{0.539498pt}%
\definecolor{currentstroke}{rgb}{0.277134,0.185228,0.489898}%
\pgfsetstrokecolor{currentstroke}%
\pgfsetdash{}{0pt}%
\pgfpathmoveto{\pgfqpoint{5.294420in}{2.421083in}}%
\pgfpathlineto{\pgfqpoint{5.244956in}{2.428264in}}%
\pgfusepath{stroke}%
\end{pgfscope}%
\begin{pgfscope}%
\pgfpathrectangle{\pgfqpoint{3.985294in}{1.750000in}}{\pgfqpoint{2.279412in}{2.004545in}}%
\pgfusepath{clip}%
\pgfsetbuttcap%
\pgfsetroundjoin%
\pgfsetlinewidth{0.565446pt}%
\definecolor{currentstroke}{rgb}{0.273006,0.204520,0.501721}%
\pgfsetstrokecolor{currentstroke}%
\pgfsetdash{}{0pt}%
\pgfpathmoveto{\pgfqpoint{5.244956in}{2.428264in}}%
\pgfpathlineto{\pgfqpoint{5.195904in}{2.437394in}}%
\pgfusepath{stroke}%
\end{pgfscope}%
\begin{pgfscope}%
\pgfpathrectangle{\pgfqpoint{3.985294in}{1.750000in}}{\pgfqpoint{2.279412in}{2.004545in}}%
\pgfusepath{clip}%
\pgfsetbuttcap%
\pgfsetroundjoin%
\pgfsetlinewidth{0.602908pt}%
\definecolor{currentstroke}{rgb}{0.266580,0.228262,0.514349}%
\pgfsetstrokecolor{currentstroke}%
\pgfsetdash{}{0pt}%
\pgfpathmoveto{\pgfqpoint{5.195904in}{2.437394in}}%
\pgfpathlineto{\pgfqpoint{5.147361in}{2.448423in}}%
\pgfusepath{stroke}%
\end{pgfscope}%
\begin{pgfscope}%
\pgfpathrectangle{\pgfqpoint{3.985294in}{1.750000in}}{\pgfqpoint{2.279412in}{2.004545in}}%
\pgfusepath{clip}%
\pgfsetbuttcap%
\pgfsetroundjoin%
\pgfsetlinewidth{0.596666pt}%
\definecolor{currentstroke}{rgb}{0.266580,0.228262,0.514349}%
\pgfsetstrokecolor{currentstroke}%
\pgfsetdash{}{0pt}%
\pgfpathmoveto{\pgfqpoint{5.147361in}{2.448423in}}%
\pgfpathlineto{\pgfqpoint{5.099502in}{2.461538in}}%
\pgfusepath{stroke}%
\end{pgfscope}%
\begin{pgfscope}%
\pgfpathrectangle{\pgfqpoint{3.985294in}{1.750000in}}{\pgfqpoint{2.279412in}{2.004545in}}%
\pgfusepath{clip}%
\pgfsetbuttcap%
\pgfsetroundjoin%
\pgfsetlinewidth{0.625277pt}%
\definecolor{currentstroke}{rgb}{0.260571,0.246922,0.522828}%
\pgfsetstrokecolor{currentstroke}%
\pgfsetdash{}{0pt}%
\pgfpathmoveto{\pgfqpoint{5.099502in}{2.461538in}}%
\pgfpathlineto{\pgfqpoint{5.052537in}{2.476932in}}%
\pgfusepath{stroke}%
\end{pgfscope}%
\begin{pgfscope}%
\pgfpathrectangle{\pgfqpoint{3.985294in}{1.750000in}}{\pgfqpoint{2.279412in}{2.004545in}}%
\pgfusepath{clip}%
\pgfsetbuttcap%
\pgfsetroundjoin%
\pgfsetlinewidth{0.646195pt}%
\definecolor{currentstroke}{rgb}{0.255645,0.260703,0.528312}%
\pgfsetstrokecolor{currentstroke}%
\pgfsetdash{}{0pt}%
\pgfpathmoveto{\pgfqpoint{5.052537in}{2.476932in}}%
\pgfpathlineto{\pgfqpoint{5.007165in}{2.495574in}}%
\pgfusepath{stroke}%
\end{pgfscope}%
\begin{pgfscope}%
\pgfpathrectangle{\pgfqpoint{3.985294in}{1.750000in}}{\pgfqpoint{2.279412in}{2.004545in}}%
\pgfusepath{clip}%
\pgfsetbuttcap%
\pgfsetroundjoin%
\pgfsetlinewidth{0.664952pt}%
\definecolor{currentstroke}{rgb}{0.250425,0.274290,0.533103}%
\pgfsetstrokecolor{currentstroke}%
\pgfsetdash{}{0pt}%
\pgfpathmoveto{\pgfqpoint{5.007165in}{2.495574in}}%
\pgfpathlineto{\pgfqpoint{4.963175in}{2.516687in}}%
\pgfusepath{stroke}%
\end{pgfscope}%
\begin{pgfscope}%
\pgfpathrectangle{\pgfqpoint{3.985294in}{1.750000in}}{\pgfqpoint{2.279412in}{2.004545in}}%
\pgfusepath{clip}%
\pgfsetbuttcap%
\pgfsetroundjoin%
\pgfsetlinewidth{0.308846pt}%
\definecolor{currentstroke}{rgb}{0.268510,0.009605,0.335427}%
\pgfsetstrokecolor{currentstroke}%
\pgfsetdash{}{0pt}%
\pgfpathmoveto{\pgfqpoint{5.894378in}{2.571846in}}%
\pgfpathlineto{\pgfqpoint{5.845686in}{2.574329in}}%
\pgfusepath{stroke}%
\end{pgfscope}%
\begin{pgfscope}%
\pgfpathrectangle{\pgfqpoint{3.985294in}{1.750000in}}{\pgfqpoint{2.279412in}{2.004545in}}%
\pgfusepath{clip}%
\pgfsetbuttcap%
\pgfsetroundjoin%
\pgfsetlinewidth{0.315924pt}%
\definecolor{currentstroke}{rgb}{0.269944,0.014625,0.341379}%
\pgfsetstrokecolor{currentstroke}%
\pgfsetdash{}{0pt}%
\pgfpathmoveto{\pgfqpoint{5.845686in}{2.574329in}}%
\pgfpathlineto{\pgfqpoint{5.795950in}{2.577618in}}%
\pgfusepath{stroke}%
\end{pgfscope}%
\begin{pgfscope}%
\pgfpathrectangle{\pgfqpoint{3.985294in}{1.750000in}}{\pgfqpoint{2.279412in}{2.004545in}}%
\pgfusepath{clip}%
\pgfsetbuttcap%
\pgfsetroundjoin%
\pgfsetlinewidth{0.323387pt}%
\definecolor{currentstroke}{rgb}{0.271305,0.019942,0.347269}%
\pgfsetstrokecolor{currentstroke}%
\pgfsetdash{}{0pt}%
\pgfpathmoveto{\pgfqpoint{5.795950in}{2.577618in}}%
\pgfpathlineto{\pgfqpoint{5.745840in}{2.577183in}}%
\pgfusepath{stroke}%
\end{pgfscope}%
\begin{pgfscope}%
\pgfpathrectangle{\pgfqpoint{3.985294in}{1.750000in}}{\pgfqpoint{2.279412in}{2.004545in}}%
\pgfusepath{clip}%
\pgfsetbuttcap%
\pgfsetroundjoin%
\pgfsetlinewidth{0.325670pt}%
\definecolor{currentstroke}{rgb}{0.271305,0.019942,0.347269}%
\pgfsetstrokecolor{currentstroke}%
\pgfsetdash{}{0pt}%
\pgfpathmoveto{\pgfqpoint{5.745840in}{2.577183in}}%
\pgfpathlineto{\pgfqpoint{5.695739in}{2.577382in}}%
\pgfusepath{stroke}%
\end{pgfscope}%
\begin{pgfscope}%
\pgfpathrectangle{\pgfqpoint{3.985294in}{1.750000in}}{\pgfqpoint{2.279412in}{2.004545in}}%
\pgfusepath{clip}%
\pgfsetbuttcap%
\pgfsetroundjoin%
\pgfsetlinewidth{0.346897pt}%
\definecolor{currentstroke}{rgb}{0.274952,0.037752,0.364543}%
\pgfsetstrokecolor{currentstroke}%
\pgfsetdash{}{0pt}%
\pgfpathmoveto{\pgfqpoint{5.695739in}{2.577382in}}%
\pgfpathlineto{\pgfqpoint{5.645601in}{2.578221in}}%
\pgfusepath{stroke}%
\end{pgfscope}%
\begin{pgfscope}%
\pgfpathrectangle{\pgfqpoint{3.985294in}{1.750000in}}{\pgfqpoint{2.279412in}{2.004545in}}%
\pgfusepath{clip}%
\pgfsetbuttcap%
\pgfsetroundjoin%
\pgfsetlinewidth{0.380817pt}%
\definecolor{currentstroke}{rgb}{0.279566,0.067836,0.391917}%
\pgfsetstrokecolor{currentstroke}%
\pgfsetdash{}{0pt}%
\pgfpathmoveto{\pgfqpoint{5.645601in}{2.578221in}}%
\pgfpathlineto{\pgfqpoint{5.595458in}{2.578988in}}%
\pgfusepath{stroke}%
\end{pgfscope}%
\begin{pgfscope}%
\pgfpathrectangle{\pgfqpoint{3.985294in}{1.750000in}}{\pgfqpoint{2.279412in}{2.004545in}}%
\pgfusepath{clip}%
\pgfsetbuttcap%
\pgfsetroundjoin%
\pgfsetlinewidth{0.410853pt}%
\definecolor{currentstroke}{rgb}{0.281924,0.089666,0.412415}%
\pgfsetstrokecolor{currentstroke}%
\pgfsetdash{}{0pt}%
\pgfpathmoveto{\pgfqpoint{5.595458in}{2.578988in}}%
\pgfpathlineto{\pgfqpoint{5.545322in}{2.580084in}}%
\pgfusepath{stroke}%
\end{pgfscope}%
\begin{pgfscope}%
\pgfpathrectangle{\pgfqpoint{3.985294in}{1.750000in}}{\pgfqpoint{2.279412in}{2.004545in}}%
\pgfusepath{clip}%
\pgfsetbuttcap%
\pgfsetroundjoin%
\pgfsetlinewidth{0.470508pt}%
\definecolor{currentstroke}{rgb}{0.282884,0.135920,0.453427}%
\pgfsetstrokecolor{currentstroke}%
\pgfsetdash{}{0pt}%
\pgfpathmoveto{\pgfqpoint{5.545322in}{2.580084in}}%
\pgfpathlineto{\pgfqpoint{5.495188in}{2.581256in}}%
\pgfusepath{stroke}%
\end{pgfscope}%
\begin{pgfscope}%
\pgfpathrectangle{\pgfqpoint{3.985294in}{1.750000in}}{\pgfqpoint{2.279412in}{2.004545in}}%
\pgfusepath{clip}%
\pgfsetbuttcap%
\pgfsetroundjoin%
\pgfsetlinewidth{0.531884pt}%
\definecolor{currentstroke}{rgb}{0.278012,0.180367,0.486697}%
\pgfsetstrokecolor{currentstroke}%
\pgfsetdash{}{0pt}%
\pgfpathmoveto{\pgfqpoint{5.495188in}{2.581256in}}%
\pgfpathlineto{\pgfqpoint{5.445061in}{2.582649in}}%
\pgfusepath{stroke}%
\end{pgfscope}%
\begin{pgfscope}%
\pgfpathrectangle{\pgfqpoint{3.985294in}{1.750000in}}{\pgfqpoint{2.279412in}{2.004545in}}%
\pgfusepath{clip}%
\pgfsetbuttcap%
\pgfsetroundjoin%
\pgfsetlinewidth{0.628070pt}%
\definecolor{currentstroke}{rgb}{0.260571,0.246922,0.522828}%
\pgfsetstrokecolor{currentstroke}%
\pgfsetdash{}{0pt}%
\pgfpathmoveto{\pgfqpoint{5.445061in}{2.582649in}}%
\pgfpathlineto{\pgfqpoint{5.394947in}{2.584338in}}%
\pgfusepath{stroke}%
\end{pgfscope}%
\begin{pgfscope}%
\pgfpathrectangle{\pgfqpoint{3.985294in}{1.750000in}}{\pgfqpoint{2.279412in}{2.004545in}}%
\pgfusepath{clip}%
\pgfsetbuttcap%
\pgfsetroundjoin%
\pgfsetlinewidth{0.706990pt}%
\definecolor{currentstroke}{rgb}{0.239346,0.300855,0.540844}%
\pgfsetstrokecolor{currentstroke}%
\pgfsetdash{}{0pt}%
\pgfpathmoveto{\pgfqpoint{5.394947in}{2.584338in}}%
\pgfpathlineto{\pgfqpoint{5.344854in}{2.586444in}}%
\pgfusepath{stroke}%
\end{pgfscope}%
\begin{pgfscope}%
\pgfpathrectangle{\pgfqpoint{3.985294in}{1.750000in}}{\pgfqpoint{2.279412in}{2.004545in}}%
\pgfusepath{clip}%
\pgfsetbuttcap%
\pgfsetroundjoin%
\pgfsetlinewidth{0.788378pt}%
\definecolor{currentstroke}{rgb}{0.216210,0.351535,0.550627}%
\pgfsetstrokecolor{currentstroke}%
\pgfsetdash{}{0pt}%
\pgfpathmoveto{\pgfqpoint{5.344854in}{2.586444in}}%
\pgfpathlineto{\pgfqpoint{5.294802in}{2.589195in}}%
\pgfusepath{stroke}%
\end{pgfscope}%
\begin{pgfscope}%
\pgfpathrectangle{\pgfqpoint{3.985294in}{1.750000in}}{\pgfqpoint{2.279412in}{2.004545in}}%
\pgfusepath{clip}%
\pgfsetbuttcap%
\pgfsetroundjoin%
\pgfsetlinewidth{0.840385pt}%
\definecolor{currentstroke}{rgb}{0.203063,0.379716,0.553925}%
\pgfsetstrokecolor{currentstroke}%
\pgfsetdash{}{0pt}%
\pgfpathmoveto{\pgfqpoint{5.294802in}{2.589195in}}%
\pgfpathlineto{\pgfqpoint{5.244798in}{2.592562in}}%
\pgfusepath{stroke}%
\end{pgfscope}%
\begin{pgfscope}%
\pgfpathrectangle{\pgfqpoint{3.985294in}{1.750000in}}{\pgfqpoint{2.279412in}{2.004545in}}%
\pgfusepath{clip}%
\pgfsetbuttcap%
\pgfsetroundjoin%
\pgfsetlinewidth{0.876851pt}%
\definecolor{currentstroke}{rgb}{0.194100,0.399323,0.555565}%
\pgfsetstrokecolor{currentstroke}%
\pgfsetdash{}{0pt}%
\pgfpathmoveto{\pgfqpoint{5.244798in}{2.592562in}}%
\pgfpathlineto{\pgfqpoint{5.194869in}{2.596686in}}%
\pgfusepath{stroke}%
\end{pgfscope}%
\begin{pgfscope}%
\pgfpathrectangle{\pgfqpoint{3.985294in}{1.750000in}}{\pgfqpoint{2.279412in}{2.004545in}}%
\pgfusepath{clip}%
\pgfsetbuttcap%
\pgfsetroundjoin%
\pgfsetlinewidth{0.961658pt}%
\definecolor{currentstroke}{rgb}{0.172719,0.448791,0.557885}%
\pgfsetstrokecolor{currentstroke}%
\pgfsetdash{}{0pt}%
\pgfpathmoveto{\pgfqpoint{5.194869in}{2.596686in}}%
\pgfpathlineto{\pgfqpoint{5.145044in}{2.601671in}}%
\pgfusepath{stroke}%
\end{pgfscope}%
\begin{pgfscope}%
\pgfpathrectangle{\pgfqpoint{3.985294in}{1.750000in}}{\pgfqpoint{2.279412in}{2.004545in}}%
\pgfusepath{clip}%
\pgfsetbuttcap%
\pgfsetroundjoin%
\pgfsetlinewidth{0.953343pt}%
\definecolor{currentstroke}{rgb}{0.174274,0.445044,0.557792}%
\pgfsetstrokecolor{currentstroke}%
\pgfsetdash{}{0pt}%
\pgfpathmoveto{\pgfqpoint{5.145044in}{2.601671in}}%
\pgfpathlineto{\pgfqpoint{5.095344in}{2.607520in}}%
\pgfusepath{stroke}%
\end{pgfscope}%
\begin{pgfscope}%
\pgfpathrectangle{\pgfqpoint{3.985294in}{1.750000in}}{\pgfqpoint{2.279412in}{2.004545in}}%
\pgfusepath{clip}%
\pgfsetbuttcap%
\pgfsetroundjoin%
\pgfsetlinewidth{0.919277pt}%
\definecolor{currentstroke}{rgb}{0.182256,0.426184,0.557120}%
\pgfsetstrokecolor{currentstroke}%
\pgfsetdash{}{0pt}%
\pgfpathmoveto{\pgfqpoint{5.095344in}{2.607520in}}%
\pgfpathlineto{\pgfqpoint{5.045797in}{2.614313in}}%
\pgfusepath{stroke}%
\end{pgfscope}%
\begin{pgfscope}%
\pgfpathrectangle{\pgfqpoint{3.985294in}{1.750000in}}{\pgfqpoint{2.279412in}{2.004545in}}%
\pgfusepath{clip}%
\pgfsetbuttcap%
\pgfsetroundjoin%
\pgfsetlinewidth{0.891168pt}%
\definecolor{currentstroke}{rgb}{0.190631,0.407061,0.556089}%
\pgfsetstrokecolor{currentstroke}%
\pgfsetdash{}{0pt}%
\pgfpathmoveto{\pgfqpoint{5.045797in}{2.614313in}}%
\pgfpathlineto{\pgfqpoint{4.996436in}{2.622068in}}%
\pgfusepath{stroke}%
\end{pgfscope}%
\begin{pgfscope}%
\pgfpathrectangle{\pgfqpoint{3.985294in}{1.750000in}}{\pgfqpoint{2.279412in}{2.004545in}}%
\pgfusepath{clip}%
\pgfsetbuttcap%
\pgfsetroundjoin%
\pgfsetlinewidth{0.880827pt}%
\definecolor{currentstroke}{rgb}{0.192357,0.403199,0.555836}%
\pgfsetstrokecolor{currentstroke}%
\pgfsetdash{}{0pt}%
\pgfpathmoveto{\pgfqpoint{4.996436in}{2.622068in}}%
\pgfpathlineto{\pgfqpoint{4.947335in}{2.630991in}}%
\pgfusepath{stroke}%
\end{pgfscope}%
\begin{pgfscope}%
\pgfpathrectangle{\pgfqpoint{3.985294in}{1.750000in}}{\pgfqpoint{2.279412in}{2.004545in}}%
\pgfusepath{clip}%
\pgfsetbuttcap%
\pgfsetroundjoin%
\pgfsetlinewidth{0.874269pt}%
\definecolor{currentstroke}{rgb}{0.194100,0.399323,0.555565}%
\pgfsetstrokecolor{currentstroke}%
\pgfsetdash{}{0pt}%
\pgfpathmoveto{\pgfqpoint{4.947335in}{2.630991in}}%
\pgfpathlineto{\pgfqpoint{4.898503in}{2.640992in}}%
\pgfusepath{stroke}%
\end{pgfscope}%
\begin{pgfscope}%
\pgfpathrectangle{\pgfqpoint{3.985294in}{1.750000in}}{\pgfqpoint{2.279412in}{2.004545in}}%
\pgfusepath{clip}%
\pgfsetbuttcap%
\pgfsetroundjoin%
\pgfsetlinewidth{0.868195pt}%
\definecolor{currentstroke}{rgb}{0.195860,0.395433,0.555276}%
\pgfsetstrokecolor{currentstroke}%
\pgfsetdash{}{0pt}%
\pgfpathmoveto{\pgfqpoint{4.898503in}{2.640992in}}%
\pgfpathlineto{\pgfqpoint{4.849965in}{2.652011in}}%
\pgfusepath{stroke}%
\end{pgfscope}%
\begin{pgfscope}%
\pgfpathrectangle{\pgfqpoint{3.985294in}{1.750000in}}{\pgfqpoint{2.279412in}{2.004545in}}%
\pgfusepath{clip}%
\pgfsetbuttcap%
\pgfsetroundjoin%
\pgfsetlinewidth{0.799530pt}%
\definecolor{currentstroke}{rgb}{0.214298,0.355619,0.551184}%
\pgfsetstrokecolor{currentstroke}%
\pgfsetdash{}{0pt}%
\pgfpathmoveto{\pgfqpoint{4.849965in}{2.652011in}}%
\pgfpathlineto{\pgfqpoint{4.802025in}{2.664745in}}%
\pgfusepath{stroke}%
\end{pgfscope}%
\begin{pgfscope}%
\pgfpathrectangle{\pgfqpoint{3.985294in}{1.750000in}}{\pgfqpoint{2.279412in}{2.004545in}}%
\pgfusepath{clip}%
\pgfsetbuttcap%
\pgfsetroundjoin%
\pgfsetlinewidth{0.681505pt}%
\definecolor{currentstroke}{rgb}{0.246811,0.283237,0.535941}%
\pgfsetstrokecolor{currentstroke}%
\pgfsetdash{}{0pt}%
\pgfpathmoveto{\pgfqpoint{4.802025in}{2.664745in}}%
\pgfpathlineto{\pgfqpoint{4.754940in}{2.679019in}}%
\pgfusepath{stroke}%
\end{pgfscope}%
\begin{pgfscope}%
\pgfpathrectangle{\pgfqpoint{3.985294in}{1.750000in}}{\pgfqpoint{2.279412in}{2.004545in}}%
\pgfusepath{clip}%
\pgfsetbuttcap%
\pgfsetroundjoin%
\pgfsetlinewidth{0.683668pt}%
\definecolor{currentstroke}{rgb}{0.246811,0.283237,0.535941}%
\pgfsetstrokecolor{currentstroke}%
\pgfsetdash{}{0pt}%
\pgfpathmoveto{\pgfqpoint{4.754940in}{2.679019in}}%
\pgfpathlineto{\pgfqpoint{4.754940in}{2.679019in}}%
\pgfusepath{stroke}%
\end{pgfscope}%
\begin{pgfscope}%
\pgfpathrectangle{\pgfqpoint{3.985294in}{1.750000in}}{\pgfqpoint{2.279412in}{2.004545in}}%
\pgfusepath{clip}%
\pgfsetbuttcap%
\pgfsetroundjoin%
\pgfsetlinewidth{0.320760pt}%
\definecolor{currentstroke}{rgb}{0.269944,0.014625,0.341379}%
\pgfsetstrokecolor{currentstroke}%
\pgfsetdash{}{0pt}%
\pgfpathmoveto{\pgfqpoint{5.894378in}{2.616952in}}%
\pgfpathlineto{\pgfqpoint{5.844272in}{2.615712in}}%
\pgfusepath{stroke}%
\end{pgfscope}%
\begin{pgfscope}%
\pgfpathrectangle{\pgfqpoint{3.985294in}{1.750000in}}{\pgfqpoint{2.279412in}{2.004545in}}%
\pgfusepath{clip}%
\pgfsetbuttcap%
\pgfsetroundjoin%
\pgfsetlinewidth{0.317877pt}%
\definecolor{currentstroke}{rgb}{0.269944,0.014625,0.341379}%
\pgfsetstrokecolor{currentstroke}%
\pgfsetdash{}{0pt}%
\pgfpathmoveto{\pgfqpoint{5.844272in}{2.615712in}}%
\pgfpathlineto{\pgfqpoint{5.794138in}{2.615386in}}%
\pgfusepath{stroke}%
\end{pgfscope}%
\begin{pgfscope}%
\pgfpathrectangle{\pgfqpoint{3.985294in}{1.750000in}}{\pgfqpoint{2.279412in}{2.004545in}}%
\pgfusepath{clip}%
\pgfsetbuttcap%
\pgfsetroundjoin%
\pgfsetlinewidth{0.330738pt}%
\definecolor{currentstroke}{rgb}{0.272594,0.025563,0.353093}%
\pgfsetstrokecolor{currentstroke}%
\pgfsetdash{}{0pt}%
\pgfpathmoveto{\pgfqpoint{5.794138in}{2.615386in}}%
\pgfpathlineto{\pgfqpoint{5.743997in}{2.615483in}}%
\pgfusepath{stroke}%
\end{pgfscope}%
\begin{pgfscope}%
\pgfpathrectangle{\pgfqpoint{3.985294in}{1.750000in}}{\pgfqpoint{2.279412in}{2.004545in}}%
\pgfusepath{clip}%
\pgfsetbuttcap%
\pgfsetroundjoin%
\pgfsetlinewidth{0.332748pt}%
\definecolor{currentstroke}{rgb}{0.272594,0.025563,0.353093}%
\pgfsetstrokecolor{currentstroke}%
\pgfsetdash{}{0pt}%
\pgfpathmoveto{\pgfqpoint{5.743997in}{2.615483in}}%
\pgfpathlineto{\pgfqpoint{5.693855in}{2.616023in}}%
\pgfusepath{stroke}%
\end{pgfscope}%
\begin{pgfscope}%
\pgfpathrectangle{\pgfqpoint{3.985294in}{1.750000in}}{\pgfqpoint{2.279412in}{2.004545in}}%
\pgfusepath{clip}%
\pgfsetbuttcap%
\pgfsetroundjoin%
\pgfsetlinewidth{0.348199pt}%
\definecolor{currentstroke}{rgb}{0.274952,0.037752,0.364543}%
\pgfsetstrokecolor{currentstroke}%
\pgfsetdash{}{0pt}%
\pgfpathmoveto{\pgfqpoint{5.693855in}{2.616023in}}%
\pgfpathlineto{\pgfqpoint{5.643716in}{2.616966in}}%
\pgfusepath{stroke}%
\end{pgfscope}%
\begin{pgfscope}%
\pgfpathrectangle{\pgfqpoint{3.985294in}{1.750000in}}{\pgfqpoint{2.279412in}{2.004545in}}%
\pgfusepath{clip}%
\pgfsetbuttcap%
\pgfsetroundjoin%
\pgfsetlinewidth{0.379817pt}%
\definecolor{currentstroke}{rgb}{0.279566,0.067836,0.391917}%
\pgfsetstrokecolor{currentstroke}%
\pgfsetdash{}{0pt}%
\pgfpathmoveto{\pgfqpoint{5.643716in}{2.616966in}}%
\pgfpathlineto{\pgfqpoint{5.593571in}{2.617655in}}%
\pgfusepath{stroke}%
\end{pgfscope}%
\begin{pgfscope}%
\pgfpathrectangle{\pgfqpoint{3.985294in}{1.750000in}}{\pgfqpoint{2.279412in}{2.004545in}}%
\pgfusepath{clip}%
\pgfsetbuttcap%
\pgfsetroundjoin%
\pgfsetlinewidth{0.424067pt}%
\definecolor{currentstroke}{rgb}{0.282656,0.100196,0.422160}%
\pgfsetstrokecolor{currentstroke}%
\pgfsetdash{}{0pt}%
\pgfpathmoveto{\pgfqpoint{5.593571in}{2.617655in}}%
\pgfpathlineto{\pgfqpoint{5.543426in}{2.618294in}}%
\pgfusepath{stroke}%
\end{pgfscope}%
\begin{pgfscope}%
\pgfpathrectangle{\pgfqpoint{3.985294in}{1.750000in}}{\pgfqpoint{2.279412in}{2.004545in}}%
\pgfusepath{clip}%
\pgfsetbuttcap%
\pgfsetroundjoin%
\pgfsetlinewidth{0.488655pt}%
\definecolor{currentstroke}{rgb}{0.281887,0.150881,0.465405}%
\pgfsetstrokecolor{currentstroke}%
\pgfsetdash{}{0pt}%
\pgfpathmoveto{\pgfqpoint{5.543426in}{2.618294in}}%
\pgfpathlineto{\pgfqpoint{5.493283in}{2.619078in}}%
\pgfusepath{stroke}%
\end{pgfscope}%
\begin{pgfscope}%
\pgfpathrectangle{\pgfqpoint{3.985294in}{1.750000in}}{\pgfqpoint{2.279412in}{2.004545in}}%
\pgfusepath{clip}%
\pgfsetbuttcap%
\pgfsetroundjoin%
\pgfsetlinewidth{0.558487pt}%
\definecolor{currentstroke}{rgb}{0.274128,0.199721,0.498911}%
\pgfsetstrokecolor{currentstroke}%
\pgfsetdash{}{0pt}%
\pgfpathmoveto{\pgfqpoint{5.493283in}{2.619078in}}%
\pgfpathlineto{\pgfqpoint{5.443146in}{2.620082in}}%
\pgfusepath{stroke}%
\end{pgfscope}%
\begin{pgfscope}%
\pgfpathrectangle{\pgfqpoint{3.985294in}{1.750000in}}{\pgfqpoint{2.279412in}{2.004545in}}%
\pgfusepath{clip}%
\pgfsetbuttcap%
\pgfsetroundjoin%
\pgfsetlinewidth{0.645230pt}%
\definecolor{currentstroke}{rgb}{0.255645,0.260703,0.528312}%
\pgfsetstrokecolor{currentstroke}%
\pgfsetdash{}{0pt}%
\pgfpathmoveto{\pgfqpoint{5.443146in}{2.620082in}}%
\pgfpathlineto{\pgfqpoint{5.393014in}{2.621298in}}%
\pgfusepath{stroke}%
\end{pgfscope}%
\begin{pgfscope}%
\pgfpathrectangle{\pgfqpoint{3.985294in}{1.750000in}}{\pgfqpoint{2.279412in}{2.004545in}}%
\pgfusepath{clip}%
\pgfsetbuttcap%
\pgfsetroundjoin%
\pgfsetlinewidth{0.734053pt}%
\definecolor{currentstroke}{rgb}{0.231674,0.318106,0.544834}%
\pgfsetstrokecolor{currentstroke}%
\pgfsetdash{}{0pt}%
\pgfpathmoveto{\pgfqpoint{5.393014in}{2.621298in}}%
\pgfpathlineto{\pgfqpoint{5.342893in}{2.622819in}}%
\pgfusepath{stroke}%
\end{pgfscope}%
\begin{pgfscope}%
\pgfpathrectangle{\pgfqpoint{3.985294in}{1.750000in}}{\pgfqpoint{2.279412in}{2.004545in}}%
\pgfusepath{clip}%
\pgfsetbuttcap%
\pgfsetroundjoin%
\pgfsetlinewidth{0.821126pt}%
\definecolor{currentstroke}{rgb}{0.208623,0.367752,0.552675}%
\pgfsetstrokecolor{currentstroke}%
\pgfsetdash{}{0pt}%
\pgfpathmoveto{\pgfqpoint{5.342893in}{2.622819in}}%
\pgfpathlineto{\pgfqpoint{5.292789in}{2.624720in}}%
\pgfusepath{stroke}%
\end{pgfscope}%
\begin{pgfscope}%
\pgfpathrectangle{\pgfqpoint{3.985294in}{1.750000in}}{\pgfqpoint{2.279412in}{2.004545in}}%
\pgfusepath{clip}%
\pgfsetbuttcap%
\pgfsetroundjoin%
\pgfsetlinewidth{0.894935pt}%
\definecolor{currentstroke}{rgb}{0.188923,0.410910,0.556326}%
\pgfsetstrokecolor{currentstroke}%
\pgfsetdash{}{0pt}%
\pgfpathmoveto{\pgfqpoint{5.292789in}{2.624720in}}%
\pgfpathlineto{\pgfqpoint{5.242708in}{2.627029in}}%
\pgfusepath{stroke}%
\end{pgfscope}%
\begin{pgfscope}%
\pgfpathrectangle{\pgfqpoint{3.985294in}{1.750000in}}{\pgfqpoint{2.279412in}{2.004545in}}%
\pgfusepath{clip}%
\pgfsetbuttcap%
\pgfsetroundjoin%
\pgfsetlinewidth{0.317180pt}%
\definecolor{currentstroke}{rgb}{0.269944,0.014625,0.341379}%
\pgfsetstrokecolor{currentstroke}%
\pgfsetdash{}{0pt}%
\pgfpathmoveto{\pgfqpoint{5.894378in}{2.707166in}}%
\pgfpathlineto{\pgfqpoint{5.845085in}{2.706562in}}%
\pgfusepath{stroke}%
\end{pgfscope}%
\begin{pgfscope}%
\pgfpathrectangle{\pgfqpoint{3.985294in}{1.750000in}}{\pgfqpoint{2.279412in}{2.004545in}}%
\pgfusepath{clip}%
\pgfsetbuttcap%
\pgfsetroundjoin%
\pgfsetlinewidth{0.318567pt}%
\definecolor{currentstroke}{rgb}{0.269944,0.014625,0.341379}%
\pgfsetstrokecolor{currentstroke}%
\pgfsetdash{}{0pt}%
\pgfpathmoveto{\pgfqpoint{5.845085in}{2.706562in}}%
\pgfpathlineto{\pgfqpoint{5.796116in}{2.706901in}}%
\pgfusepath{stroke}%
\end{pgfscope}%
\begin{pgfscope}%
\pgfpathrectangle{\pgfqpoint{3.985294in}{1.750000in}}{\pgfqpoint{2.279412in}{2.004545in}}%
\pgfusepath{clip}%
\pgfsetbuttcap%
\pgfsetroundjoin%
\pgfsetlinewidth{0.326923pt}%
\definecolor{currentstroke}{rgb}{0.271305,0.019942,0.347269}%
\pgfsetstrokecolor{currentstroke}%
\pgfsetdash{}{0pt}%
\pgfpathmoveto{\pgfqpoint{5.796116in}{2.706901in}}%
\pgfpathlineto{\pgfqpoint{5.746047in}{2.707629in}}%
\pgfusepath{stroke}%
\end{pgfscope}%
\begin{pgfscope}%
\pgfpathrectangle{\pgfqpoint{3.985294in}{1.750000in}}{\pgfqpoint{2.279412in}{2.004545in}}%
\pgfusepath{clip}%
\pgfsetbuttcap%
\pgfsetroundjoin%
\pgfsetlinewidth{0.331982pt}%
\definecolor{currentstroke}{rgb}{0.272594,0.025563,0.353093}%
\pgfsetstrokecolor{currentstroke}%
\pgfsetdash{}{0pt}%
\pgfpathmoveto{\pgfqpoint{5.746047in}{2.707629in}}%
\pgfpathlineto{\pgfqpoint{5.695916in}{2.707102in}}%
\pgfusepath{stroke}%
\end{pgfscope}%
\begin{pgfscope}%
\pgfpathrectangle{\pgfqpoint{3.985294in}{1.750000in}}{\pgfqpoint{2.279412in}{2.004545in}}%
\pgfusepath{clip}%
\pgfsetbuttcap%
\pgfsetroundjoin%
\pgfsetlinewidth{0.349282pt}%
\definecolor{currentstroke}{rgb}{0.276022,0.044167,0.370164}%
\pgfsetstrokecolor{currentstroke}%
\pgfsetdash{}{0pt}%
\pgfpathmoveto{\pgfqpoint{5.695916in}{2.707102in}}%
\pgfpathlineto{\pgfqpoint{5.645767in}{2.706903in}}%
\pgfusepath{stroke}%
\end{pgfscope}%
\begin{pgfscope}%
\pgfpathrectangle{\pgfqpoint{3.985294in}{1.750000in}}{\pgfqpoint{2.279412in}{2.004545in}}%
\pgfusepath{clip}%
\pgfsetbuttcap%
\pgfsetroundjoin%
\pgfsetlinewidth{0.390397pt}%
\definecolor{currentstroke}{rgb}{0.280894,0.078907,0.402329}%
\pgfsetstrokecolor{currentstroke}%
\pgfsetdash{}{0pt}%
\pgfpathmoveto{\pgfqpoint{5.645767in}{2.706903in}}%
\pgfpathlineto{\pgfqpoint{5.595617in}{2.706637in}}%
\pgfusepath{stroke}%
\end{pgfscope}%
\begin{pgfscope}%
\pgfpathrectangle{\pgfqpoint{3.985294in}{1.750000in}}{\pgfqpoint{2.279412in}{2.004545in}}%
\pgfusepath{clip}%
\pgfsetbuttcap%
\pgfsetroundjoin%
\pgfsetlinewidth{0.419161pt}%
\definecolor{currentstroke}{rgb}{0.282656,0.100196,0.422160}%
\pgfsetstrokecolor{currentstroke}%
\pgfsetdash{}{0pt}%
\pgfpathmoveto{\pgfqpoint{5.595617in}{2.706637in}}%
\pgfpathlineto{\pgfqpoint{5.545466in}{2.706603in}}%
\pgfusepath{stroke}%
\end{pgfscope}%
\begin{pgfscope}%
\pgfpathrectangle{\pgfqpoint{3.985294in}{1.750000in}}{\pgfqpoint{2.279412in}{2.004545in}}%
\pgfusepath{clip}%
\pgfsetbuttcap%
\pgfsetroundjoin%
\pgfsetlinewidth{0.482730pt}%
\definecolor{currentstroke}{rgb}{0.282290,0.145912,0.461510}%
\pgfsetstrokecolor{currentstroke}%
\pgfsetdash{}{0pt}%
\pgfpathmoveto{\pgfqpoint{5.545466in}{2.706603in}}%
\pgfpathlineto{\pgfqpoint{5.495315in}{2.706703in}}%
\pgfusepath{stroke}%
\end{pgfscope}%
\begin{pgfscope}%
\pgfpathrectangle{\pgfqpoint{3.985294in}{1.750000in}}{\pgfqpoint{2.279412in}{2.004545in}}%
\pgfusepath{clip}%
\pgfsetbuttcap%
\pgfsetroundjoin%
\pgfsetlinewidth{0.552856pt}%
\definecolor{currentstroke}{rgb}{0.275191,0.194905,0.496005}%
\pgfsetstrokecolor{currentstroke}%
\pgfsetdash{}{0pt}%
\pgfpathmoveto{\pgfqpoint{5.495315in}{2.706703in}}%
\pgfpathlineto{\pgfqpoint{5.445163in}{2.706701in}}%
\pgfusepath{stroke}%
\end{pgfscope}%
\begin{pgfscope}%
\pgfpathrectangle{\pgfqpoint{3.985294in}{1.750000in}}{\pgfqpoint{2.279412in}{2.004545in}}%
\pgfusepath{clip}%
\pgfsetbuttcap%
\pgfsetroundjoin%
\pgfsetlinewidth{0.660096pt}%
\definecolor{currentstroke}{rgb}{0.252194,0.269783,0.531579}%
\pgfsetstrokecolor{currentstroke}%
\pgfsetdash{}{0pt}%
\pgfpathmoveto{\pgfqpoint{5.445163in}{2.706701in}}%
\pgfpathlineto{\pgfqpoint{5.395011in}{2.706649in}}%
\pgfusepath{stroke}%
\end{pgfscope}%
\begin{pgfscope}%
\pgfpathrectangle{\pgfqpoint{3.985294in}{1.750000in}}{\pgfqpoint{2.279412in}{2.004545in}}%
\pgfusepath{clip}%
\pgfsetbuttcap%
\pgfsetroundjoin%
\pgfsetlinewidth{0.781581pt}%
\definecolor{currentstroke}{rgb}{0.218130,0.347432,0.550038}%
\pgfsetstrokecolor{currentstroke}%
\pgfsetdash{}{0pt}%
\pgfpathmoveto{\pgfqpoint{5.395011in}{2.706649in}}%
\pgfpathlineto{\pgfqpoint{5.344859in}{2.706719in}}%
\pgfusepath{stroke}%
\end{pgfscope}%
\begin{pgfscope}%
\pgfpathrectangle{\pgfqpoint{3.985294in}{1.750000in}}{\pgfqpoint{2.279412in}{2.004545in}}%
\pgfusepath{clip}%
\pgfsetbuttcap%
\pgfsetroundjoin%
\pgfsetlinewidth{0.885654pt}%
\definecolor{currentstroke}{rgb}{0.190631,0.407061,0.556089}%
\pgfsetstrokecolor{currentstroke}%
\pgfsetdash{}{0pt}%
\pgfpathmoveto{\pgfqpoint{5.344859in}{2.706719in}}%
\pgfpathlineto{\pgfqpoint{5.294708in}{2.706910in}}%
\pgfusepath{stroke}%
\end{pgfscope}%
\begin{pgfscope}%
\pgfpathrectangle{\pgfqpoint{3.985294in}{1.750000in}}{\pgfqpoint{2.279412in}{2.004545in}}%
\pgfusepath{clip}%
\pgfsetbuttcap%
\pgfsetroundjoin%
\pgfsetlinewidth{0.993764pt}%
\definecolor{currentstroke}{rgb}{0.166617,0.463708,0.558119}%
\pgfsetstrokecolor{currentstroke}%
\pgfsetdash{}{0pt}%
\pgfpathmoveto{\pgfqpoint{5.294708in}{2.706910in}}%
\pgfpathlineto{\pgfqpoint{5.244558in}{2.707202in}}%
\pgfusepath{stroke}%
\end{pgfscope}%
\begin{pgfscope}%
\pgfpathrectangle{\pgfqpoint{3.985294in}{1.750000in}}{\pgfqpoint{2.279412in}{2.004545in}}%
\pgfusepath{clip}%
\pgfsetbuttcap%
\pgfsetroundjoin%
\pgfsetlinewidth{1.061991pt}%
\definecolor{currentstroke}{rgb}{0.151918,0.500685,0.557587}%
\pgfsetstrokecolor{currentstroke}%
\pgfsetdash{}{0pt}%
\pgfpathmoveto{\pgfqpoint{5.244558in}{2.707202in}}%
\pgfpathlineto{\pgfqpoint{5.194410in}{2.707696in}}%
\pgfusepath{stroke}%
\end{pgfscope}%
\begin{pgfscope}%
\pgfpathrectangle{\pgfqpoint{3.985294in}{1.750000in}}{\pgfqpoint{2.279412in}{2.004545in}}%
\pgfusepath{clip}%
\pgfsetbuttcap%
\pgfsetroundjoin%
\pgfsetlinewidth{1.057819pt}%
\definecolor{currentstroke}{rgb}{0.151918,0.500685,0.557587}%
\pgfsetstrokecolor{currentstroke}%
\pgfsetdash{}{0pt}%
\pgfpathmoveto{\pgfqpoint{5.194410in}{2.707696in}}%
\pgfpathlineto{\pgfqpoint{5.144264in}{2.708331in}}%
\pgfusepath{stroke}%
\end{pgfscope}%
\begin{pgfscope}%
\pgfpathrectangle{\pgfqpoint{3.985294in}{1.750000in}}{\pgfqpoint{2.279412in}{2.004545in}}%
\pgfusepath{clip}%
\pgfsetbuttcap%
\pgfsetroundjoin%
\pgfsetlinewidth{1.079151pt}%
\definecolor{currentstroke}{rgb}{0.147607,0.511733,0.557049}%
\pgfsetstrokecolor{currentstroke}%
\pgfsetdash{}{0pt}%
\pgfpathmoveto{\pgfqpoint{5.144264in}{2.708331in}}%
\pgfpathlineto{\pgfqpoint{5.094121in}{2.709040in}}%
\pgfusepath{stroke}%
\end{pgfscope}%
\begin{pgfscope}%
\pgfpathrectangle{\pgfqpoint{3.985294in}{1.750000in}}{\pgfqpoint{2.279412in}{2.004545in}}%
\pgfusepath{clip}%
\pgfsetbuttcap%
\pgfsetroundjoin%
\pgfsetlinewidth{0.999253pt}%
\definecolor{currentstroke}{rgb}{0.165117,0.467423,0.558141}%
\pgfsetstrokecolor{currentstroke}%
\pgfsetdash{}{0pt}%
\pgfpathmoveto{\pgfqpoint{5.094121in}{2.709040in}}%
\pgfpathlineto{\pgfqpoint{5.043986in}{2.710107in}}%
\pgfusepath{stroke}%
\end{pgfscope}%
\begin{pgfscope}%
\pgfpathrectangle{\pgfqpoint{3.985294in}{1.750000in}}{\pgfqpoint{2.279412in}{2.004545in}}%
\pgfusepath{clip}%
\pgfsetbuttcap%
\pgfsetroundjoin%
\pgfsetlinewidth{0.990668pt}%
\definecolor{currentstroke}{rgb}{0.166617,0.463708,0.558119}%
\pgfsetstrokecolor{currentstroke}%
\pgfsetdash{}{0pt}%
\pgfpathmoveto{\pgfqpoint{5.043986in}{2.710107in}}%
\pgfpathlineto{\pgfqpoint{4.993857in}{2.711391in}}%
\pgfusepath{stroke}%
\end{pgfscope}%
\begin{pgfscope}%
\pgfpathrectangle{\pgfqpoint{3.985294in}{1.750000in}}{\pgfqpoint{2.279412in}{2.004545in}}%
\pgfusepath{clip}%
\pgfsetbuttcap%
\pgfsetroundjoin%
\pgfsetlinewidth{0.942005pt}%
\definecolor{currentstroke}{rgb}{0.177423,0.437527,0.557565}%
\pgfsetstrokecolor{currentstroke}%
\pgfsetdash{}{0pt}%
\pgfpathmoveto{\pgfqpoint{4.993857in}{2.711391in}}%
\pgfpathlineto{\pgfqpoint{4.943737in}{2.712913in}}%
\pgfusepath{stroke}%
\end{pgfscope}%
\begin{pgfscope}%
\pgfpathrectangle{\pgfqpoint{3.985294in}{1.750000in}}{\pgfqpoint{2.279412in}{2.004545in}}%
\pgfusepath{clip}%
\pgfsetbuttcap%
\pgfsetroundjoin%
\pgfsetlinewidth{0.900751pt}%
\definecolor{currentstroke}{rgb}{0.187231,0.414746,0.556547}%
\pgfsetstrokecolor{currentstroke}%
\pgfsetdash{}{0pt}%
\pgfpathmoveto{\pgfqpoint{4.943737in}{2.712913in}}%
\pgfpathlineto{\pgfqpoint{4.893626in}{2.714539in}}%
\pgfusepath{stroke}%
\end{pgfscope}%
\begin{pgfscope}%
\pgfpathrectangle{\pgfqpoint{3.985294in}{1.750000in}}{\pgfqpoint{2.279412in}{2.004545in}}%
\pgfusepath{clip}%
\pgfsetbuttcap%
\pgfsetroundjoin%
\pgfsetlinewidth{0.823160pt}%
\definecolor{currentstroke}{rgb}{0.206756,0.371758,0.553117}%
\pgfsetstrokecolor{currentstroke}%
\pgfsetdash{}{0pt}%
\pgfpathmoveto{\pgfqpoint{4.893626in}{2.714539in}}%
\pgfpathlineto{\pgfqpoint{4.843538in}{2.716367in}}%
\pgfusepath{stroke}%
\end{pgfscope}%
\begin{pgfscope}%
\pgfpathrectangle{\pgfqpoint{3.985294in}{1.750000in}}{\pgfqpoint{2.279412in}{2.004545in}}%
\pgfusepath{clip}%
\pgfsetbuttcap%
\pgfsetroundjoin%
\pgfsetlinewidth{0.767705pt}%
\definecolor{currentstroke}{rgb}{0.223925,0.334994,0.548053}%
\pgfsetstrokecolor{currentstroke}%
\pgfsetdash{}{0pt}%
\pgfpathmoveto{\pgfqpoint{4.843538in}{2.716367in}}%
\pgfpathlineto{\pgfqpoint{4.793474in}{2.718505in}}%
\pgfusepath{stroke}%
\end{pgfscope}%
\begin{pgfscope}%
\pgfpathrectangle{\pgfqpoint{3.985294in}{1.750000in}}{\pgfqpoint{2.279412in}{2.004545in}}%
\pgfusepath{clip}%
\pgfsetbuttcap%
\pgfsetroundjoin%
\pgfsetlinewidth{0.631578pt}%
\definecolor{currentstroke}{rgb}{0.258965,0.251537,0.524736}%
\pgfsetstrokecolor{currentstroke}%
\pgfsetdash{}{0pt}%
\pgfpathmoveto{\pgfqpoint{4.793474in}{2.718505in}}%
\pgfpathlineto{\pgfqpoint{4.743651in}{2.720639in}}%
\pgfusepath{stroke}%
\end{pgfscope}%
\begin{pgfscope}%
\pgfpathrectangle{\pgfqpoint{3.985294in}{1.750000in}}{\pgfqpoint{2.279412in}{2.004545in}}%
\pgfusepath{clip}%
\pgfsetbuttcap%
\pgfsetroundjoin%
\pgfsetlinewidth{0.504082pt}%
\definecolor{currentstroke}{rgb}{0.280868,0.160771,0.472899}%
\pgfsetstrokecolor{currentstroke}%
\pgfsetdash{}{0pt}%
\pgfpathmoveto{\pgfqpoint{4.743651in}{2.720639in}}%
\pgfpathlineto{\pgfqpoint{4.743651in}{2.720639in}}%
\pgfusepath{stroke}%
\end{pgfscope}%
\begin{pgfscope}%
\pgfpathrectangle{\pgfqpoint{3.985294in}{1.750000in}}{\pgfqpoint{2.279412in}{2.004545in}}%
\pgfusepath{clip}%
\pgfsetbuttcap%
\pgfsetroundjoin%
\pgfsetlinewidth{0.313958pt}%
\definecolor{currentstroke}{rgb}{0.268510,0.009605,0.335427}%
\pgfsetstrokecolor{currentstroke}%
\pgfsetdash{}{0pt}%
\pgfpathmoveto{\pgfqpoint{5.915884in}{2.751962in}}%
\pgfpathlineto{\pgfqpoint{5.894378in}{2.752273in}}%
\pgfusepath{stroke}%
\end{pgfscope}%
\begin{pgfscope}%
\pgfpathrectangle{\pgfqpoint{3.985294in}{1.750000in}}{\pgfqpoint{2.279412in}{2.004545in}}%
\pgfusepath{clip}%
\pgfsetbuttcap%
\pgfsetroundjoin%
\pgfsetlinewidth{0.331175pt}%
\definecolor{currentstroke}{rgb}{0.272594,0.025563,0.353093}%
\pgfsetstrokecolor{currentstroke}%
\pgfsetdash{}{0pt}%
\pgfpathmoveto{\pgfqpoint{5.894378in}{2.752273in}}%
\pgfpathlineto{\pgfqpoint{5.894378in}{2.752273in}}%
\pgfusepath{stroke}%
\end{pgfscope}%
\begin{pgfscope}%
\pgfpathrectangle{\pgfqpoint{3.985294in}{1.750000in}}{\pgfqpoint{2.279412in}{2.004545in}}%
\pgfusepath{clip}%
\pgfsetbuttcap%
\pgfsetroundjoin%
\pgfsetlinewidth{0.331175pt}%
\definecolor{currentstroke}{rgb}{0.272594,0.025563,0.353093}%
\pgfsetstrokecolor{currentstroke}%
\pgfsetdash{}{0pt}%
\pgfpathmoveto{\pgfqpoint{5.894378in}{2.752273in}}%
\pgfpathlineto{\pgfqpoint{5.894378in}{2.752273in}}%
\pgfusepath{stroke}%
\end{pgfscope}%
\begin{pgfscope}%
\pgfpathrectangle{\pgfqpoint{3.985294in}{1.750000in}}{\pgfqpoint{2.279412in}{2.004545in}}%
\pgfusepath{clip}%
\pgfsetbuttcap%
\pgfsetroundjoin%
\pgfsetlinewidth{0.331175pt}%
\definecolor{currentstroke}{rgb}{0.272594,0.025563,0.353093}%
\pgfsetstrokecolor{currentstroke}%
\pgfsetdash{}{0pt}%
\pgfpathmoveto{\pgfqpoint{5.894378in}{2.752273in}}%
\pgfpathlineto{\pgfqpoint{5.844229in}{2.752366in}}%
\pgfusepath{stroke}%
\end{pgfscope}%
\begin{pgfscope}%
\pgfpathrectangle{\pgfqpoint{3.985294in}{1.750000in}}{\pgfqpoint{2.279412in}{2.004545in}}%
\pgfusepath{clip}%
\pgfsetbuttcap%
\pgfsetroundjoin%
\pgfsetlinewidth{0.323499pt}%
\definecolor{currentstroke}{rgb}{0.271305,0.019942,0.347269}%
\pgfsetstrokecolor{currentstroke}%
\pgfsetdash{}{0pt}%
\pgfpathmoveto{\pgfqpoint{5.844229in}{2.752366in}}%
\pgfpathlineto{\pgfqpoint{5.794083in}{2.752625in}}%
\pgfusepath{stroke}%
\end{pgfscope}%
\begin{pgfscope}%
\pgfpathrectangle{\pgfqpoint{3.985294in}{1.750000in}}{\pgfqpoint{2.279412in}{2.004545in}}%
\pgfusepath{clip}%
\pgfsetbuttcap%
\pgfsetroundjoin%
\pgfsetlinewidth{0.323585pt}%
\definecolor{currentstroke}{rgb}{0.271305,0.019942,0.347269}%
\pgfsetstrokecolor{currentstroke}%
\pgfsetdash{}{0pt}%
\pgfpathmoveto{\pgfqpoint{5.794083in}{2.752625in}}%
\pgfpathlineto{\pgfqpoint{5.743939in}{2.752661in}}%
\pgfusepath{stroke}%
\end{pgfscope}%
\begin{pgfscope}%
\pgfpathrectangle{\pgfqpoint{3.985294in}{1.750000in}}{\pgfqpoint{2.279412in}{2.004545in}}%
\pgfusepath{clip}%
\pgfsetbuttcap%
\pgfsetroundjoin%
\pgfsetlinewidth{0.332505pt}%
\definecolor{currentstroke}{rgb}{0.272594,0.025563,0.353093}%
\pgfsetstrokecolor{currentstroke}%
\pgfsetdash{}{0pt}%
\pgfpathmoveto{\pgfqpoint{5.743939in}{2.752661in}}%
\pgfpathlineto{\pgfqpoint{5.693791in}{2.752463in}}%
\pgfusepath{stroke}%
\end{pgfscope}%
\begin{pgfscope}%
\pgfpathrectangle{\pgfqpoint{3.985294in}{1.750000in}}{\pgfqpoint{2.279412in}{2.004545in}}%
\pgfusepath{clip}%
\pgfsetbuttcap%
\pgfsetroundjoin%
\pgfsetlinewidth{0.355124pt}%
\definecolor{currentstroke}{rgb}{0.276022,0.044167,0.370164}%
\pgfsetstrokecolor{currentstroke}%
\pgfsetdash{}{0pt}%
\pgfpathmoveto{\pgfqpoint{5.693791in}{2.752463in}}%
\pgfpathlineto{\pgfqpoint{5.643646in}{2.751957in}}%
\pgfusepath{stroke}%
\end{pgfscope}%
\begin{pgfscope}%
\pgfpathrectangle{\pgfqpoint{3.985294in}{1.750000in}}{\pgfqpoint{2.279412in}{2.004545in}}%
\pgfusepath{clip}%
\pgfsetbuttcap%
\pgfsetroundjoin%
\pgfsetlinewidth{0.383702pt}%
\definecolor{currentstroke}{rgb}{0.280267,0.073417,0.397163}%
\pgfsetstrokecolor{currentstroke}%
\pgfsetdash{}{0pt}%
\pgfpathmoveto{\pgfqpoint{5.643646in}{2.751957in}}%
\pgfpathlineto{\pgfqpoint{5.593502in}{2.751406in}}%
\pgfusepath{stroke}%
\end{pgfscope}%
\begin{pgfscope}%
\pgfpathrectangle{\pgfqpoint{3.985294in}{1.750000in}}{\pgfqpoint{2.279412in}{2.004545in}}%
\pgfusepath{clip}%
\pgfsetbuttcap%
\pgfsetroundjoin%
\pgfsetlinewidth{0.409730pt}%
\definecolor{currentstroke}{rgb}{0.281924,0.089666,0.412415}%
\pgfsetstrokecolor{currentstroke}%
\pgfsetdash{}{0pt}%
\pgfpathmoveto{\pgfqpoint{5.593502in}{2.751406in}}%
\pgfpathlineto{\pgfqpoint{5.543351in}{2.751288in}}%
\pgfusepath{stroke}%
\end{pgfscope}%
\begin{pgfscope}%
\pgfpathrectangle{\pgfqpoint{3.985294in}{1.750000in}}{\pgfqpoint{2.279412in}{2.004545in}}%
\pgfusepath{clip}%
\pgfsetbuttcap%
\pgfsetroundjoin%
\pgfsetlinewidth{0.474802pt}%
\definecolor{currentstroke}{rgb}{0.282623,0.140926,0.457517}%
\pgfsetstrokecolor{currentstroke}%
\pgfsetdash{}{0pt}%
\pgfpathmoveto{\pgfqpoint{5.543351in}{2.751288in}}%
\pgfpathlineto{\pgfqpoint{5.493201in}{2.750932in}}%
\pgfusepath{stroke}%
\end{pgfscope}%
\begin{pgfscope}%
\pgfpathrectangle{\pgfqpoint{3.985294in}{1.750000in}}{\pgfqpoint{2.279412in}{2.004545in}}%
\pgfusepath{clip}%
\pgfsetbuttcap%
\pgfsetroundjoin%
\pgfsetlinewidth{0.565857pt}%
\definecolor{currentstroke}{rgb}{0.273006,0.204520,0.501721}%
\pgfsetstrokecolor{currentstroke}%
\pgfsetdash{}{0pt}%
\pgfpathmoveto{\pgfqpoint{5.493201in}{2.750932in}}%
\pgfpathlineto{\pgfqpoint{5.443051in}{2.750661in}}%
\pgfusepath{stroke}%
\end{pgfscope}%
\begin{pgfscope}%
\pgfpathrectangle{\pgfqpoint{3.985294in}{1.750000in}}{\pgfqpoint{2.279412in}{2.004545in}}%
\pgfusepath{clip}%
\pgfsetbuttcap%
\pgfsetroundjoin%
\pgfsetlinewidth{0.658390pt}%
\definecolor{currentstroke}{rgb}{0.252194,0.269783,0.531579}%
\pgfsetstrokecolor{currentstroke}%
\pgfsetdash{}{0pt}%
\pgfpathmoveto{\pgfqpoint{5.443051in}{2.750661in}}%
\pgfpathlineto{\pgfqpoint{5.392899in}{2.750480in}}%
\pgfusepath{stroke}%
\end{pgfscope}%
\begin{pgfscope}%
\pgfpathrectangle{\pgfqpoint{3.985294in}{1.750000in}}{\pgfqpoint{2.279412in}{2.004545in}}%
\pgfusepath{clip}%
\pgfsetbuttcap%
\pgfsetroundjoin%
\pgfsetlinewidth{0.784436pt}%
\definecolor{currentstroke}{rgb}{0.218130,0.347432,0.550038}%
\pgfsetstrokecolor{currentstroke}%
\pgfsetdash{}{0pt}%
\pgfpathmoveto{\pgfqpoint{5.392899in}{2.750480in}}%
\pgfpathlineto{\pgfqpoint{5.342749in}{2.750144in}}%
\pgfusepath{stroke}%
\end{pgfscope}%
\begin{pgfscope}%
\pgfpathrectangle{\pgfqpoint{3.985294in}{1.750000in}}{\pgfqpoint{2.279412in}{2.004545in}}%
\pgfusepath{clip}%
\pgfsetbuttcap%
\pgfsetroundjoin%
\pgfsetlinewidth{0.899794pt}%
\definecolor{currentstroke}{rgb}{0.187231,0.414746,0.556547}%
\pgfsetstrokecolor{currentstroke}%
\pgfsetdash{}{0pt}%
\pgfpathmoveto{\pgfqpoint{5.342749in}{2.750144in}}%
\pgfpathlineto{\pgfqpoint{5.292599in}{2.749771in}}%
\pgfusepath{stroke}%
\end{pgfscope}%
\begin{pgfscope}%
\pgfpathrectangle{\pgfqpoint{3.985294in}{1.750000in}}{\pgfqpoint{2.279412in}{2.004545in}}%
\pgfusepath{clip}%
\pgfsetbuttcap%
\pgfsetroundjoin%
\pgfsetlinewidth{0.977620pt}%
\definecolor{currentstroke}{rgb}{0.169646,0.456262,0.558030}%
\pgfsetstrokecolor{currentstroke}%
\pgfsetdash{}{0pt}%
\pgfpathmoveto{\pgfqpoint{5.292599in}{2.749771in}}%
\pgfpathlineto{\pgfqpoint{5.242451in}{2.749289in}}%
\pgfusepath{stroke}%
\end{pgfscope}%
\begin{pgfscope}%
\pgfpathrectangle{\pgfqpoint{3.985294in}{1.750000in}}{\pgfqpoint{2.279412in}{2.004545in}}%
\pgfusepath{clip}%
\pgfsetbuttcap%
\pgfsetroundjoin%
\pgfsetlinewidth{1.061920pt}%
\definecolor{currentstroke}{rgb}{0.151918,0.500685,0.557587}%
\pgfsetstrokecolor{currentstroke}%
\pgfsetdash{}{0pt}%
\pgfpathmoveto{\pgfqpoint{5.242451in}{2.749289in}}%
\pgfpathlineto{\pgfqpoint{5.192304in}{2.748703in}}%
\pgfusepath{stroke}%
\end{pgfscope}%
\begin{pgfscope}%
\pgfpathrectangle{\pgfqpoint{3.985294in}{1.750000in}}{\pgfqpoint{2.279412in}{2.004545in}}%
\pgfusepath{clip}%
\pgfsetbuttcap%
\pgfsetroundjoin%
\pgfsetlinewidth{1.051635pt}%
\definecolor{currentstroke}{rgb}{0.153364,0.497000,0.557724}%
\pgfsetstrokecolor{currentstroke}%
\pgfsetdash{}{0pt}%
\pgfpathmoveto{\pgfqpoint{5.192304in}{2.748703in}}%
\pgfpathlineto{\pgfqpoint{5.142159in}{2.748006in}}%
\pgfusepath{stroke}%
\end{pgfscope}%
\begin{pgfscope}%
\pgfpathrectangle{\pgfqpoint{3.985294in}{1.750000in}}{\pgfqpoint{2.279412in}{2.004545in}}%
\pgfusepath{clip}%
\pgfsetbuttcap%
\pgfsetroundjoin%
\pgfsetlinewidth{1.051172pt}%
\definecolor{currentstroke}{rgb}{0.153364,0.497000,0.557724}%
\pgfsetstrokecolor{currentstroke}%
\pgfsetdash{}{0pt}%
\pgfpathmoveto{\pgfqpoint{5.142159in}{2.748006in}}%
\pgfpathlineto{\pgfqpoint{5.092018in}{2.747178in}}%
\pgfusepath{stroke}%
\end{pgfscope}%
\begin{pgfscope}%
\pgfpathrectangle{\pgfqpoint{3.985294in}{1.750000in}}{\pgfqpoint{2.279412in}{2.004545in}}%
\pgfusepath{clip}%
\pgfsetbuttcap%
\pgfsetroundjoin%
\pgfsetlinewidth{1.007214pt}%
\definecolor{currentstroke}{rgb}{0.163625,0.471133,0.558148}%
\pgfsetstrokecolor{currentstroke}%
\pgfsetdash{}{0pt}%
\pgfpathmoveto{\pgfqpoint{5.092018in}{2.747178in}}%
\pgfpathlineto{\pgfqpoint{5.041885in}{2.746152in}}%
\pgfusepath{stroke}%
\end{pgfscope}%
\begin{pgfscope}%
\pgfpathrectangle{\pgfqpoint{3.985294in}{1.750000in}}{\pgfqpoint{2.279412in}{2.004545in}}%
\pgfusepath{clip}%
\pgfsetbuttcap%
\pgfsetroundjoin%
\pgfsetlinewidth{0.999881pt}%
\definecolor{currentstroke}{rgb}{0.165117,0.467423,0.558141}%
\pgfsetstrokecolor{currentstroke}%
\pgfsetdash{}{0pt}%
\pgfpathmoveto{\pgfqpoint{5.041885in}{2.746152in}}%
\pgfpathlineto{\pgfqpoint{4.991758in}{2.744939in}}%
\pgfusepath{stroke}%
\end{pgfscope}%
\begin{pgfscope}%
\pgfpathrectangle{\pgfqpoint{3.985294in}{1.750000in}}{\pgfqpoint{2.279412in}{2.004545in}}%
\pgfusepath{clip}%
\pgfsetbuttcap%
\pgfsetroundjoin%
\pgfsetlinewidth{0.911292pt}%
\definecolor{currentstroke}{rgb}{0.185556,0.418570,0.556753}%
\pgfsetstrokecolor{currentstroke}%
\pgfsetdash{}{0pt}%
\pgfpathmoveto{\pgfqpoint{4.991758in}{2.744939in}}%
\pgfpathlineto{\pgfqpoint{4.941636in}{2.743542in}}%
\pgfusepath{stroke}%
\end{pgfscope}%
\begin{pgfscope}%
\pgfpathrectangle{\pgfqpoint{3.985294in}{1.750000in}}{\pgfqpoint{2.279412in}{2.004545in}}%
\pgfusepath{clip}%
\pgfsetbuttcap%
\pgfsetroundjoin%
\pgfsetlinewidth{0.326151pt}%
\definecolor{currentstroke}{rgb}{0.271305,0.019942,0.347269}%
\pgfsetstrokecolor{currentstroke}%
\pgfsetdash{}{0pt}%
\pgfpathmoveto{\pgfqpoint{5.894378in}{3.068020in}}%
\pgfpathlineto{\pgfqpoint{5.844241in}{3.067552in}}%
\pgfusepath{stroke}%
\end{pgfscope}%
\begin{pgfscope}%
\pgfpathrectangle{\pgfqpoint{3.985294in}{1.750000in}}{\pgfqpoint{2.279412in}{2.004545in}}%
\pgfusepath{clip}%
\pgfsetbuttcap%
\pgfsetroundjoin%
\pgfsetlinewidth{0.322636pt}%
\definecolor{currentstroke}{rgb}{0.271305,0.019942,0.347269}%
\pgfsetstrokecolor{currentstroke}%
\pgfsetdash{}{0pt}%
\pgfpathmoveto{\pgfqpoint{5.844241in}{3.067552in}}%
\pgfpathlineto{\pgfqpoint{5.794207in}{3.068197in}}%
\pgfusepath{stroke}%
\end{pgfscope}%
\begin{pgfscope}%
\pgfpathrectangle{\pgfqpoint{3.985294in}{1.750000in}}{\pgfqpoint{2.279412in}{2.004545in}}%
\pgfusepath{clip}%
\pgfsetbuttcap%
\pgfsetroundjoin%
\pgfsetlinewidth{0.316358pt}%
\definecolor{currentstroke}{rgb}{0.269944,0.014625,0.341379}%
\pgfsetstrokecolor{currentstroke}%
\pgfsetdash{}{0pt}%
\pgfpathmoveto{\pgfqpoint{5.794207in}{3.068197in}}%
\pgfpathlineto{\pgfqpoint{5.744188in}{3.068442in}}%
\pgfusepath{stroke}%
\end{pgfscope}%
\begin{pgfscope}%
\pgfpathrectangle{\pgfqpoint{3.985294in}{1.750000in}}{\pgfqpoint{2.279412in}{2.004545in}}%
\pgfusepath{clip}%
\pgfsetbuttcap%
\pgfsetroundjoin%
\pgfsetlinewidth{0.327152pt}%
\definecolor{currentstroke}{rgb}{0.271305,0.019942,0.347269}%
\pgfsetstrokecolor{currentstroke}%
\pgfsetdash{}{0pt}%
\pgfpathmoveto{\pgfqpoint{5.744188in}{3.068442in}}%
\pgfpathlineto{\pgfqpoint{5.694066in}{3.067739in}}%
\pgfusepath{stroke}%
\end{pgfscope}%
\begin{pgfscope}%
\pgfpathrectangle{\pgfqpoint{3.985294in}{1.750000in}}{\pgfqpoint{2.279412in}{2.004545in}}%
\pgfusepath{clip}%
\pgfsetbuttcap%
\pgfsetroundjoin%
\pgfsetlinewidth{0.335220pt}%
\definecolor{currentstroke}{rgb}{0.272594,0.025563,0.353093}%
\pgfsetstrokecolor{currentstroke}%
\pgfsetdash{}{0pt}%
\pgfpathmoveto{\pgfqpoint{5.694066in}{3.067739in}}%
\pgfpathlineto{\pgfqpoint{5.643936in}{3.066584in}}%
\pgfusepath{stroke}%
\end{pgfscope}%
\begin{pgfscope}%
\pgfpathrectangle{\pgfqpoint{3.985294in}{1.750000in}}{\pgfqpoint{2.279412in}{2.004545in}}%
\pgfusepath{clip}%
\pgfsetbuttcap%
\pgfsetroundjoin%
\pgfsetlinewidth{0.355082pt}%
\definecolor{currentstroke}{rgb}{0.276022,0.044167,0.370164}%
\pgfsetstrokecolor{currentstroke}%
\pgfsetdash{}{0pt}%
\pgfpathmoveto{\pgfqpoint{5.643936in}{3.066584in}}%
\pgfpathlineto{\pgfqpoint{5.593818in}{3.065039in}}%
\pgfusepath{stroke}%
\end{pgfscope}%
\begin{pgfscope}%
\pgfpathrectangle{\pgfqpoint{3.985294in}{1.750000in}}{\pgfqpoint{2.279412in}{2.004545in}}%
\pgfusepath{clip}%
\pgfsetbuttcap%
\pgfsetroundjoin%
\pgfsetlinewidth{0.376924pt}%
\definecolor{currentstroke}{rgb}{0.279566,0.067836,0.391917}%
\pgfsetstrokecolor{currentstroke}%
\pgfsetdash{}{0pt}%
\pgfpathmoveto{\pgfqpoint{5.593818in}{3.065039in}}%
\pgfpathlineto{\pgfqpoint{5.543708in}{3.063330in}}%
\pgfusepath{stroke}%
\end{pgfscope}%
\begin{pgfscope}%
\pgfpathrectangle{\pgfqpoint{3.985294in}{1.750000in}}{\pgfqpoint{2.279412in}{2.004545in}}%
\pgfusepath{clip}%
\pgfsetbuttcap%
\pgfsetroundjoin%
\pgfsetlinewidth{0.401396pt}%
\definecolor{currentstroke}{rgb}{0.281446,0.084320,0.407414}%
\pgfsetstrokecolor{currentstroke}%
\pgfsetdash{}{0pt}%
\pgfpathmoveto{\pgfqpoint{5.543708in}{3.063330in}}%
\pgfpathlineto{\pgfqpoint{5.493631in}{3.060957in}}%
\pgfusepath{stroke}%
\end{pgfscope}%
\begin{pgfscope}%
\pgfpathrectangle{\pgfqpoint{3.985294in}{1.750000in}}{\pgfqpoint{2.279412in}{2.004545in}}%
\pgfusepath{clip}%
\pgfsetbuttcap%
\pgfsetroundjoin%
\pgfsetlinewidth{0.445045pt}%
\definecolor{currentstroke}{rgb}{0.283197,0.115680,0.436115}%
\pgfsetstrokecolor{currentstroke}%
\pgfsetdash{}{0pt}%
\pgfpathmoveto{\pgfqpoint{5.493631in}{3.060957in}}%
\pgfpathlineto{\pgfqpoint{5.443607in}{3.057846in}}%
\pgfusepath{stroke}%
\end{pgfscope}%
\begin{pgfscope}%
\pgfpathrectangle{\pgfqpoint{3.985294in}{1.750000in}}{\pgfqpoint{2.279412in}{2.004545in}}%
\pgfusepath{clip}%
\pgfsetbuttcap%
\pgfsetroundjoin%
\pgfsetlinewidth{0.441956pt}%
\definecolor{currentstroke}{rgb}{0.283197,0.115680,0.436115}%
\pgfsetstrokecolor{currentstroke}%
\pgfsetdash{}{0pt}%
\pgfpathmoveto{\pgfqpoint{5.443607in}{3.057846in}}%
\pgfpathlineto{\pgfqpoint{5.393664in}{3.053847in}}%
\pgfusepath{stroke}%
\end{pgfscope}%
\begin{pgfscope}%
\pgfpathrectangle{\pgfqpoint{3.985294in}{1.750000in}}{\pgfqpoint{2.279412in}{2.004545in}}%
\pgfusepath{clip}%
\pgfsetbuttcap%
\pgfsetroundjoin%
\pgfsetlinewidth{0.484216pt}%
\definecolor{currentstroke}{rgb}{0.282290,0.145912,0.461510}%
\pgfsetstrokecolor{currentstroke}%
\pgfsetdash{}{0pt}%
\pgfpathmoveto{\pgfqpoint{5.393664in}{3.053847in}}%
\pgfpathlineto{\pgfqpoint{5.343847in}{3.048804in}}%
\pgfusepath{stroke}%
\end{pgfscope}%
\begin{pgfscope}%
\pgfpathrectangle{\pgfqpoint{3.985294in}{1.750000in}}{\pgfqpoint{2.279412in}{2.004545in}}%
\pgfusepath{clip}%
\pgfsetbuttcap%
\pgfsetroundjoin%
\pgfsetlinewidth{0.554254pt}%
\definecolor{currentstroke}{rgb}{0.275191,0.194905,0.496005}%
\pgfsetstrokecolor{currentstroke}%
\pgfsetdash{}{0pt}%
\pgfpathmoveto{\pgfqpoint{5.343847in}{3.048804in}}%
\pgfpathlineto{\pgfqpoint{5.294152in}{3.042877in}}%
\pgfusepath{stroke}%
\end{pgfscope}%
\begin{pgfscope}%
\pgfpathrectangle{\pgfqpoint{3.985294in}{1.750000in}}{\pgfqpoint{2.279412in}{2.004545in}}%
\pgfusepath{clip}%
\pgfsetbuttcap%
\pgfsetroundjoin%
\pgfsetlinewidth{0.539557pt}%
\definecolor{currentstroke}{rgb}{0.277134,0.185228,0.489898}%
\pgfsetstrokecolor{currentstroke}%
\pgfsetdash{}{0pt}%
\pgfpathmoveto{\pgfqpoint{5.294152in}{3.042877in}}%
\pgfpathlineto{\pgfqpoint{5.244647in}{3.035880in}}%
\pgfusepath{stroke}%
\end{pgfscope}%
\begin{pgfscope}%
\pgfpathrectangle{\pgfqpoint{3.985294in}{1.750000in}}{\pgfqpoint{2.279412in}{2.004545in}}%
\pgfusepath{clip}%
\pgfsetbuttcap%
\pgfsetroundjoin%
\pgfsetlinewidth{0.580949pt}%
\definecolor{currentstroke}{rgb}{0.270595,0.214069,0.507052}%
\pgfsetstrokecolor{currentstroke}%
\pgfsetdash{}{0pt}%
\pgfpathmoveto{\pgfqpoint{5.244647in}{3.035880in}}%
\pgfpathlineto{\pgfqpoint{5.195451in}{3.027362in}}%
\pgfusepath{stroke}%
\end{pgfscope}%
\begin{pgfscope}%
\pgfpathrectangle{\pgfqpoint{3.985294in}{1.750000in}}{\pgfqpoint{2.279412in}{2.004545in}}%
\pgfusepath{clip}%
\pgfsetbuttcap%
\pgfsetroundjoin%
\pgfsetlinewidth{0.571151pt}%
\definecolor{currentstroke}{rgb}{0.271828,0.209303,0.504434}%
\pgfsetstrokecolor{currentstroke}%
\pgfsetdash{}{0pt}%
\pgfpathmoveto{\pgfqpoint{5.195451in}{3.027362in}}%
\pgfpathlineto{\pgfqpoint{5.146825in}{3.016708in}}%
\pgfusepath{stroke}%
\end{pgfscope}%
\begin{pgfscope}%
\pgfpathrectangle{\pgfqpoint{3.985294in}{1.750000in}}{\pgfqpoint{2.279412in}{2.004545in}}%
\pgfusepath{clip}%
\pgfsetbuttcap%
\pgfsetroundjoin%
\pgfsetlinewidth{0.614198pt}%
\definecolor{currentstroke}{rgb}{0.263663,0.237631,0.518762}%
\pgfsetstrokecolor{currentstroke}%
\pgfsetdash{}{0pt}%
\pgfpathmoveto{\pgfqpoint{5.146825in}{3.016708in}}%
\pgfpathlineto{\pgfqpoint{5.099034in}{3.003465in}}%
\pgfusepath{stroke}%
\end{pgfscope}%
\begin{pgfscope}%
\pgfpathrectangle{\pgfqpoint{3.985294in}{1.750000in}}{\pgfqpoint{2.279412in}{2.004545in}}%
\pgfusepath{clip}%
\pgfsetbuttcap%
\pgfsetroundjoin%
\pgfsetlinewidth{0.633116pt}%
\definecolor{currentstroke}{rgb}{0.258965,0.251537,0.524736}%
\pgfsetstrokecolor{currentstroke}%
\pgfsetdash{}{0pt}%
\pgfpathmoveto{\pgfqpoint{5.099034in}{3.003465in}}%
\pgfpathlineto{\pgfqpoint{5.052448in}{2.987281in}}%
\pgfusepath{stroke}%
\end{pgfscope}%
\begin{pgfscope}%
\pgfpathrectangle{\pgfqpoint{3.985294in}{1.750000in}}{\pgfqpoint{2.279412in}{2.004545in}}%
\pgfusepath{clip}%
\pgfsetbuttcap%
\pgfsetroundjoin%
\pgfsetlinewidth{0.633118pt}%
\definecolor{currentstroke}{rgb}{0.258965,0.251537,0.524736}%
\pgfsetstrokecolor{currentstroke}%
\pgfsetdash{}{0pt}%
\pgfpathmoveto{\pgfqpoint{5.052448in}{2.987281in}}%
\pgfpathlineto{\pgfqpoint{5.007278in}{2.968224in}}%
\pgfusepath{stroke}%
\end{pgfscope}%
\begin{pgfscope}%
\pgfpathrectangle{\pgfqpoint{3.985294in}{1.750000in}}{\pgfqpoint{2.279412in}{2.004545in}}%
\pgfusepath{clip}%
\pgfsetbuttcap%
\pgfsetroundjoin%
\pgfsetlinewidth{0.610704pt}%
\definecolor{currentstroke}{rgb}{0.263663,0.237631,0.518762}%
\pgfsetstrokecolor{currentstroke}%
\pgfsetdash{}{0pt}%
\pgfpathmoveto{\pgfqpoint{5.007278in}{2.968224in}}%
\pgfpathlineto{\pgfqpoint{4.963518in}{2.946755in}}%
\pgfusepath{stroke}%
\end{pgfscope}%
\begin{pgfscope}%
\pgfpathrectangle{\pgfqpoint{3.985294in}{1.750000in}}{\pgfqpoint{2.279412in}{2.004545in}}%
\pgfusepath{clip}%
\pgfsetbuttcap%
\pgfsetroundjoin%
\pgfsetlinewidth{0.685661pt}%
\definecolor{currentstroke}{rgb}{0.244972,0.287675,0.537260}%
\pgfsetstrokecolor{currentstroke}%
\pgfsetdash{}{0pt}%
\pgfpathmoveto{\pgfqpoint{4.963518in}{2.946755in}}%
\pgfpathlineto{\pgfqpoint{4.920975in}{2.923517in}}%
\pgfusepath{stroke}%
\end{pgfscope}%
\begin{pgfscope}%
\pgfpathrectangle{\pgfqpoint{3.985294in}{1.750000in}}{\pgfqpoint{2.279412in}{2.004545in}}%
\pgfusepath{clip}%
\pgfsetbuttcap%
\pgfsetroundjoin%
\pgfsetlinewidth{0.751205pt}%
\definecolor{currentstroke}{rgb}{0.227802,0.326594,0.546532}%
\pgfsetstrokecolor{currentstroke}%
\pgfsetdash{}{0pt}%
\pgfpathmoveto{\pgfqpoint{4.920975in}{2.923517in}}%
\pgfpathlineto{\pgfqpoint{4.879834in}{2.898449in}}%
\pgfusepath{stroke}%
\end{pgfscope}%
\begin{pgfscope}%
\pgfpathrectangle{\pgfqpoint{3.985294in}{1.750000in}}{\pgfqpoint{2.279412in}{2.004545in}}%
\pgfusepath{clip}%
\pgfsetbuttcap%
\pgfsetroundjoin%
\pgfsetlinewidth{0.720321pt}%
\definecolor{currentstroke}{rgb}{0.235526,0.309527,0.542944}%
\pgfsetstrokecolor{currentstroke}%
\pgfsetdash{}{0pt}%
\pgfpathmoveto{\pgfqpoint{4.879834in}{2.898449in}}%
\pgfpathlineto{\pgfqpoint{4.839615in}{2.872278in}}%
\pgfusepath{stroke}%
\end{pgfscope}%
\begin{pgfscope}%
\pgfpathrectangle{\pgfqpoint{3.985294in}{1.750000in}}{\pgfqpoint{2.279412in}{2.004545in}}%
\pgfusepath{clip}%
\pgfsetbuttcap%
\pgfsetroundjoin%
\pgfsetlinewidth{0.316890pt}%
\definecolor{currentstroke}{rgb}{0.269944,0.014625,0.341379}%
\pgfsetstrokecolor{currentstroke}%
\pgfsetdash{}{0pt}%
\pgfpathmoveto{\pgfqpoint{5.894378in}{3.113127in}}%
\pgfpathlineto{\pgfqpoint{5.844319in}{3.114894in}}%
\pgfusepath{stroke}%
\end{pgfscope}%
\begin{pgfscope}%
\pgfpathrectangle{\pgfqpoint{3.985294in}{1.750000in}}{\pgfqpoint{2.279412in}{2.004545in}}%
\pgfusepath{clip}%
\pgfsetbuttcap%
\pgfsetroundjoin%
\pgfsetlinewidth{0.322072pt}%
\definecolor{currentstroke}{rgb}{0.271305,0.019942,0.347269}%
\pgfsetstrokecolor{currentstroke}%
\pgfsetdash{}{0pt}%
\pgfpathmoveto{\pgfqpoint{5.844319in}{3.114894in}}%
\pgfpathlineto{\pgfqpoint{5.794307in}{3.114974in}}%
\pgfusepath{stroke}%
\end{pgfscope}%
\begin{pgfscope}%
\pgfpathrectangle{\pgfqpoint{3.985294in}{1.750000in}}{\pgfqpoint{2.279412in}{2.004545in}}%
\pgfusepath{clip}%
\pgfsetbuttcap%
\pgfsetroundjoin%
\pgfsetlinewidth{0.316224pt}%
\definecolor{currentstroke}{rgb}{0.269944,0.014625,0.341379}%
\pgfsetstrokecolor{currentstroke}%
\pgfsetdash{}{0pt}%
\pgfpathmoveto{\pgfqpoint{5.794307in}{3.114974in}}%
\pgfpathlineto{\pgfqpoint{5.744291in}{3.114390in}}%
\pgfusepath{stroke}%
\end{pgfscope}%
\begin{pgfscope}%
\pgfpathrectangle{\pgfqpoint{3.985294in}{1.750000in}}{\pgfqpoint{2.279412in}{2.004545in}}%
\pgfusepath{clip}%
\pgfsetbuttcap%
\pgfsetroundjoin%
\pgfsetlinewidth{0.324814pt}%
\definecolor{currentstroke}{rgb}{0.271305,0.019942,0.347269}%
\pgfsetstrokecolor{currentstroke}%
\pgfsetdash{}{0pt}%
\pgfpathmoveto{\pgfqpoint{5.744291in}{3.114390in}}%
\pgfpathlineto{\pgfqpoint{5.694156in}{3.113703in}}%
\pgfusepath{stroke}%
\end{pgfscope}%
\begin{pgfscope}%
\pgfpathrectangle{\pgfqpoint{3.985294in}{1.750000in}}{\pgfqpoint{2.279412in}{2.004545in}}%
\pgfusepath{clip}%
\pgfsetbuttcap%
\pgfsetroundjoin%
\pgfsetlinewidth{0.339049pt}%
\definecolor{currentstroke}{rgb}{0.273809,0.031497,0.358853}%
\pgfsetstrokecolor{currentstroke}%
\pgfsetdash{}{0pt}%
\pgfpathmoveto{\pgfqpoint{5.694156in}{3.113703in}}%
\pgfpathlineto{\pgfqpoint{5.644026in}{3.112670in}}%
\pgfusepath{stroke}%
\end{pgfscope}%
\begin{pgfscope}%
\pgfpathrectangle{\pgfqpoint{3.985294in}{1.750000in}}{\pgfqpoint{2.279412in}{2.004545in}}%
\pgfusepath{clip}%
\pgfsetbuttcap%
\pgfsetroundjoin%
\pgfsetlinewidth{0.343565pt}%
\definecolor{currentstroke}{rgb}{0.274952,0.037752,0.364543}%
\pgfsetstrokecolor{currentstroke}%
\pgfsetdash{}{0pt}%
\pgfpathmoveto{\pgfqpoint{5.644026in}{3.112670in}}%
\pgfpathlineto{\pgfqpoint{5.593936in}{3.110546in}}%
\pgfusepath{stroke}%
\end{pgfscope}%
\begin{pgfscope}%
\pgfpathrectangle{\pgfqpoint{3.985294in}{1.750000in}}{\pgfqpoint{2.279412in}{2.004545in}}%
\pgfusepath{clip}%
\pgfsetbuttcap%
\pgfsetroundjoin%
\pgfsetlinewidth{0.360487pt}%
\definecolor{currentstroke}{rgb}{0.277018,0.050344,0.375715}%
\pgfsetstrokecolor{currentstroke}%
\pgfsetdash{}{0pt}%
\pgfpathmoveto{\pgfqpoint{5.593936in}{3.110546in}}%
\pgfpathlineto{\pgfqpoint{5.543859in}{3.108163in}}%
\pgfusepath{stroke}%
\end{pgfscope}%
\begin{pgfscope}%
\pgfpathrectangle{\pgfqpoint{3.985294in}{1.750000in}}{\pgfqpoint{2.279412in}{2.004545in}}%
\pgfusepath{clip}%
\pgfsetbuttcap%
\pgfsetroundjoin%
\pgfsetlinewidth{0.391256pt}%
\definecolor{currentstroke}{rgb}{0.280894,0.078907,0.402329}%
\pgfsetstrokecolor{currentstroke}%
\pgfsetdash{}{0pt}%
\pgfpathmoveto{\pgfqpoint{5.543859in}{3.108163in}}%
\pgfpathlineto{\pgfqpoint{5.493798in}{3.105581in}}%
\pgfusepath{stroke}%
\end{pgfscope}%
\begin{pgfscope}%
\pgfpathrectangle{\pgfqpoint{3.985294in}{1.750000in}}{\pgfqpoint{2.279412in}{2.004545in}}%
\pgfusepath{clip}%
\pgfsetbuttcap%
\pgfsetroundjoin%
\pgfsetlinewidth{0.426189pt}%
\definecolor{currentstroke}{rgb}{0.282910,0.105393,0.426902}%
\pgfsetstrokecolor{currentstroke}%
\pgfsetdash{}{0pt}%
\pgfpathmoveto{\pgfqpoint{5.493798in}{3.105581in}}%
\pgfpathlineto{\pgfqpoint{5.443817in}{3.102029in}}%
\pgfusepath{stroke}%
\end{pgfscope}%
\begin{pgfscope}%
\pgfpathrectangle{\pgfqpoint{3.985294in}{1.750000in}}{\pgfqpoint{2.279412in}{2.004545in}}%
\pgfusepath{clip}%
\pgfsetbuttcap%
\pgfsetroundjoin%
\pgfsetlinewidth{0.441710pt}%
\definecolor{currentstroke}{rgb}{0.283197,0.115680,0.436115}%
\pgfsetstrokecolor{currentstroke}%
\pgfsetdash{}{0pt}%
\pgfpathmoveto{\pgfqpoint{5.443817in}{3.102029in}}%
\pgfpathlineto{\pgfqpoint{5.393903in}{3.097786in}}%
\pgfusepath{stroke}%
\end{pgfscope}%
\begin{pgfscope}%
\pgfpathrectangle{\pgfqpoint{3.985294in}{1.750000in}}{\pgfqpoint{2.279412in}{2.004545in}}%
\pgfusepath{clip}%
\pgfsetbuttcap%
\pgfsetroundjoin%
\pgfsetlinewidth{0.469980pt}%
\definecolor{currentstroke}{rgb}{0.282884,0.135920,0.453427}%
\pgfsetstrokecolor{currentstroke}%
\pgfsetdash{}{0pt}%
\pgfpathmoveto{\pgfqpoint{5.393903in}{3.097786in}}%
\pgfpathlineto{\pgfqpoint{5.344082in}{3.092778in}}%
\pgfusepath{stroke}%
\end{pgfscope}%
\begin{pgfscope}%
\pgfpathrectangle{\pgfqpoint{3.985294in}{1.750000in}}{\pgfqpoint{2.279412in}{2.004545in}}%
\pgfusepath{clip}%
\pgfsetbuttcap%
\pgfsetroundjoin%
\pgfsetlinewidth{0.471543pt}%
\definecolor{currentstroke}{rgb}{0.282884,0.135920,0.453427}%
\pgfsetstrokecolor{currentstroke}%
\pgfsetdash{}{0pt}%
\pgfpathmoveto{\pgfqpoint{5.344082in}{3.092778in}}%
\pgfpathlineto{\pgfqpoint{5.294427in}{3.086643in}}%
\pgfusepath{stroke}%
\end{pgfscope}%
\begin{pgfscope}%
\pgfpathrectangle{\pgfqpoint{3.985294in}{1.750000in}}{\pgfqpoint{2.279412in}{2.004545in}}%
\pgfusepath{clip}%
\pgfsetbuttcap%
\pgfsetroundjoin%
\pgfsetlinewidth{0.487467pt}%
\definecolor{currentstroke}{rgb}{0.281887,0.150881,0.465405}%
\pgfsetstrokecolor{currentstroke}%
\pgfsetdash{}{0pt}%
\pgfpathmoveto{\pgfqpoint{5.294427in}{3.086643in}}%
\pgfpathlineto{\pgfqpoint{5.245030in}{3.079086in}}%
\pgfusepath{stroke}%
\end{pgfscope}%
\begin{pgfscope}%
\pgfpathrectangle{\pgfqpoint{3.985294in}{1.750000in}}{\pgfqpoint{2.279412in}{2.004545in}}%
\pgfusepath{clip}%
\pgfsetbuttcap%
\pgfsetroundjoin%
\pgfsetlinewidth{0.510452pt}%
\definecolor{currentstroke}{rgb}{0.280255,0.165693,0.476498}%
\pgfsetstrokecolor{currentstroke}%
\pgfsetdash{}{0pt}%
\pgfpathmoveto{\pgfqpoint{5.245030in}{3.079086in}}%
\pgfpathlineto{\pgfqpoint{5.196199in}{3.069160in}}%
\pgfusepath{stroke}%
\end{pgfscope}%
\begin{pgfscope}%
\pgfpathrectangle{\pgfqpoint{3.985294in}{1.750000in}}{\pgfqpoint{2.279412in}{2.004545in}}%
\pgfusepath{clip}%
\pgfsetbuttcap%
\pgfsetroundjoin%
\pgfsetlinewidth{0.470521pt}%
\definecolor{currentstroke}{rgb}{0.282884,0.135920,0.453427}%
\pgfsetstrokecolor{currentstroke}%
\pgfsetdash{}{0pt}%
\pgfpathmoveto{\pgfqpoint{5.196199in}{3.069160in}}%
\pgfpathlineto{\pgfqpoint{5.148440in}{3.055916in}}%
\pgfusepath{stroke}%
\end{pgfscope}%
\begin{pgfscope}%
\pgfpathrectangle{\pgfqpoint{3.985294in}{1.750000in}}{\pgfqpoint{2.279412in}{2.004545in}}%
\pgfusepath{clip}%
\pgfsetbuttcap%
\pgfsetroundjoin%
\pgfsetlinewidth{0.312622pt}%
\definecolor{currentstroke}{rgb}{0.268510,0.009605,0.335427}%
\pgfsetstrokecolor{currentstroke}%
\pgfsetdash{}{0pt}%
\pgfpathmoveto{\pgfqpoint{5.843087in}{3.428874in}}%
\pgfpathlineto{\pgfqpoint{5.792977in}{3.429329in}}%
\pgfusepath{stroke}%
\end{pgfscope}%
\begin{pgfscope}%
\pgfpathrectangle{\pgfqpoint{3.985294in}{1.750000in}}{\pgfqpoint{2.279412in}{2.004545in}}%
\pgfusepath{clip}%
\pgfsetbuttcap%
\pgfsetroundjoin%
\pgfsetlinewidth{0.319677pt}%
\definecolor{currentstroke}{rgb}{0.269944,0.014625,0.341379}%
\pgfsetstrokecolor{currentstroke}%
\pgfsetdash{}{0pt}%
\pgfpathmoveto{\pgfqpoint{5.792977in}{3.429329in}}%
\pgfpathlineto{\pgfqpoint{5.743331in}{3.425848in}}%
\pgfusepath{stroke}%
\end{pgfscope}%
\begin{pgfscope}%
\pgfpathrectangle{\pgfqpoint{3.985294in}{1.750000in}}{\pgfqpoint{2.279412in}{2.004545in}}%
\pgfusepath{clip}%
\pgfsetbuttcap%
\pgfsetroundjoin%
\pgfsetlinewidth{0.314650pt}%
\definecolor{currentstroke}{rgb}{0.268510,0.009605,0.335427}%
\pgfsetstrokecolor{currentstroke}%
\pgfsetdash{}{0pt}%
\pgfpathmoveto{\pgfqpoint{5.743331in}{3.425848in}}%
\pgfpathlineto{\pgfqpoint{5.693810in}{3.420543in}}%
\pgfusepath{stroke}%
\end{pgfscope}%
\begin{pgfscope}%
\pgfpathrectangle{\pgfqpoint{3.985294in}{1.750000in}}{\pgfqpoint{2.279412in}{2.004545in}}%
\pgfusepath{clip}%
\pgfsetbuttcap%
\pgfsetroundjoin%
\pgfsetlinewidth{0.327560pt}%
\definecolor{currentstroke}{rgb}{0.271305,0.019942,0.347269}%
\pgfsetstrokecolor{currentstroke}%
\pgfsetdash{}{0pt}%
\pgfpathmoveto{\pgfqpoint{5.693810in}{3.420543in}}%
\pgfpathlineto{\pgfqpoint{5.643748in}{3.420629in}}%
\pgfusepath{stroke}%
\end{pgfscope}%
\begin{pgfscope}%
\pgfpathrectangle{\pgfqpoint{3.985294in}{1.750000in}}{\pgfqpoint{2.279412in}{2.004545in}}%
\pgfusepath{clip}%
\pgfsetbuttcap%
\pgfsetroundjoin%
\pgfsetlinewidth{0.319603pt}%
\definecolor{currentstroke}{rgb}{0.269944,0.014625,0.341379}%
\pgfsetstrokecolor{currentstroke}%
\pgfsetdash{}{0pt}%
\pgfpathmoveto{\pgfqpoint{5.643748in}{3.420629in}}%
\pgfpathlineto{\pgfqpoint{5.593833in}{3.418793in}}%
\pgfusepath{stroke}%
\end{pgfscope}%
\begin{pgfscope}%
\pgfpathrectangle{\pgfqpoint{3.985294in}{1.750000in}}{\pgfqpoint{2.279412in}{2.004545in}}%
\pgfusepath{clip}%
\pgfsetbuttcap%
\pgfsetroundjoin%
\pgfsetlinewidth{0.312445pt}%
\definecolor{currentstroke}{rgb}{0.268510,0.009605,0.335427}%
\pgfsetstrokecolor{currentstroke}%
\pgfsetdash{}{0pt}%
\pgfpathmoveto{\pgfqpoint{5.889227in}{2.436758in}}%
\pgfpathlineto{\pgfqpoint{5.843087in}{2.436525in}}%
\pgfusepath{stroke}%
\end{pgfscope}%
\begin{pgfscope}%
\pgfpathrectangle{\pgfqpoint{3.985294in}{1.750000in}}{\pgfqpoint{2.279412in}{2.004545in}}%
\pgfusepath{clip}%
\pgfsetbuttcap%
\pgfsetroundjoin%
\pgfsetlinewidth{0.317222pt}%
\definecolor{currentstroke}{rgb}{0.269944,0.014625,0.341379}%
\pgfsetstrokecolor{currentstroke}%
\pgfsetdash{}{0pt}%
\pgfpathmoveto{\pgfqpoint{5.843087in}{2.436525in}}%
\pgfpathlineto{\pgfqpoint{5.796945in}{2.435863in}}%
\pgfusepath{stroke}%
\end{pgfscope}%
\begin{pgfscope}%
\pgfpathrectangle{\pgfqpoint{3.985294in}{1.750000in}}{\pgfqpoint{2.279412in}{2.004545in}}%
\pgfusepath{clip}%
\pgfsetbuttcap%
\pgfsetroundjoin%
\pgfsetlinewidth{0.323305pt}%
\definecolor{currentstroke}{rgb}{0.271305,0.019942,0.347269}%
\pgfsetstrokecolor{currentstroke}%
\pgfsetdash{}{0pt}%
\pgfpathmoveto{\pgfqpoint{5.796945in}{2.435863in}}%
\pgfpathlineto{\pgfqpoint{5.746817in}{2.435508in}}%
\pgfusepath{stroke}%
\end{pgfscope}%
\begin{pgfscope}%
\pgfpathrectangle{\pgfqpoint{3.985294in}{1.750000in}}{\pgfqpoint{2.279412in}{2.004545in}}%
\pgfusepath{clip}%
\pgfsetbuttcap%
\pgfsetroundjoin%
\pgfsetlinewidth{0.333541pt}%
\definecolor{currentstroke}{rgb}{0.272594,0.025563,0.353093}%
\pgfsetstrokecolor{currentstroke}%
\pgfsetdash{}{0pt}%
\pgfpathmoveto{\pgfqpoint{5.746817in}{2.435508in}}%
\pgfpathlineto{\pgfqpoint{5.696686in}{2.435286in}}%
\pgfusepath{stroke}%
\end{pgfscope}%
\begin{pgfscope}%
\pgfpathrectangle{\pgfqpoint{3.985294in}{1.750000in}}{\pgfqpoint{2.279412in}{2.004545in}}%
\pgfusepath{clip}%
\pgfsetbuttcap%
\pgfsetroundjoin%
\pgfsetlinewidth{0.342604pt}%
\definecolor{currentstroke}{rgb}{0.274952,0.037752,0.364543}%
\pgfsetstrokecolor{currentstroke}%
\pgfsetdash{}{0pt}%
\pgfpathmoveto{\pgfqpoint{5.696686in}{2.435286in}}%
\pgfpathlineto{\pgfqpoint{5.646539in}{2.434918in}}%
\pgfusepath{stroke}%
\end{pgfscope}%
\begin{pgfscope}%
\pgfpathrectangle{\pgfqpoint{3.985294in}{1.750000in}}{\pgfqpoint{2.279412in}{2.004545in}}%
\pgfusepath{clip}%
\pgfsetbuttcap%
\pgfsetroundjoin%
\pgfsetlinewidth{0.356956pt}%
\definecolor{currentstroke}{rgb}{0.277018,0.050344,0.375715}%
\pgfsetstrokecolor{currentstroke}%
\pgfsetdash{}{0pt}%
\pgfpathmoveto{\pgfqpoint{5.646539in}{2.434918in}}%
\pgfpathlineto{\pgfqpoint{5.596408in}{2.435539in}}%
\pgfusepath{stroke}%
\end{pgfscope}%
\begin{pgfscope}%
\pgfpathrectangle{\pgfqpoint{3.985294in}{1.750000in}}{\pgfqpoint{2.279412in}{2.004545in}}%
\pgfusepath{clip}%
\pgfsetbuttcap%
\pgfsetroundjoin%
\pgfsetlinewidth{0.377201pt}%
\definecolor{currentstroke}{rgb}{0.279566,0.067836,0.391917}%
\pgfsetstrokecolor{currentstroke}%
\pgfsetdash{}{0pt}%
\pgfpathmoveto{\pgfqpoint{5.596408in}{2.435539in}}%
\pgfpathlineto{\pgfqpoint{5.546297in}{2.437281in}}%
\pgfusepath{stroke}%
\end{pgfscope}%
\begin{pgfscope}%
\pgfpathrectangle{\pgfqpoint{3.985294in}{1.750000in}}{\pgfqpoint{2.279412in}{2.004545in}}%
\pgfusepath{clip}%
\pgfsetbuttcap%
\pgfsetroundjoin%
\pgfsetlinewidth{0.416547pt}%
\definecolor{currentstroke}{rgb}{0.282327,0.094955,0.417331}%
\pgfsetstrokecolor{currentstroke}%
\pgfsetdash{}{0pt}%
\pgfpathmoveto{\pgfqpoint{5.546297in}{2.437281in}}%
\pgfpathlineto{\pgfqpoint{5.496191in}{2.439142in}}%
\pgfusepath{stroke}%
\end{pgfscope}%
\begin{pgfscope}%
\pgfpathrectangle{\pgfqpoint{3.985294in}{1.750000in}}{\pgfqpoint{2.279412in}{2.004545in}}%
\pgfusepath{clip}%
\pgfsetbuttcap%
\pgfsetroundjoin%
\pgfsetlinewidth{0.459295pt}%
\definecolor{currentstroke}{rgb}{0.283072,0.130895,0.449241}%
\pgfsetstrokecolor{currentstroke}%
\pgfsetdash{}{0pt}%
\pgfpathmoveto{\pgfqpoint{5.496191in}{2.439142in}}%
\pgfpathlineto{\pgfqpoint{5.446139in}{2.441807in}}%
\pgfusepath{stroke}%
\end{pgfscope}%
\begin{pgfscope}%
\pgfpathrectangle{\pgfqpoint{3.985294in}{1.750000in}}{\pgfqpoint{2.279412in}{2.004545in}}%
\pgfusepath{clip}%
\pgfsetbuttcap%
\pgfsetroundjoin%
\pgfsetlinewidth{0.478824pt}%
\definecolor{currentstroke}{rgb}{0.282623,0.140926,0.457517}%
\pgfsetstrokecolor{currentstroke}%
\pgfsetdash{}{0pt}%
\pgfpathmoveto{\pgfqpoint{5.446139in}{2.441807in}}%
\pgfpathlineto{\pgfqpoint{5.396177in}{2.445610in}}%
\pgfusepath{stroke}%
\end{pgfscope}%
\begin{pgfscope}%
\pgfpathrectangle{\pgfqpoint{3.985294in}{1.750000in}}{\pgfqpoint{2.279412in}{2.004545in}}%
\pgfusepath{clip}%
\pgfsetbuttcap%
\pgfsetroundjoin%
\pgfsetlinewidth{0.316556pt}%
\definecolor{currentstroke}{rgb}{0.269944,0.014625,0.341379}%
\pgfsetstrokecolor{currentstroke}%
\pgfsetdash{}{0pt}%
\pgfpathmoveto{\pgfqpoint{5.843087in}{2.481632in}}%
\pgfpathlineto{\pgfqpoint{5.792949in}{2.481802in}}%
\pgfusepath{stroke}%
\end{pgfscope}%
\begin{pgfscope}%
\pgfpathrectangle{\pgfqpoint{3.985294in}{1.750000in}}{\pgfqpoint{2.279412in}{2.004545in}}%
\pgfusepath{clip}%
\pgfsetbuttcap%
\pgfsetroundjoin%
\pgfsetlinewidth{0.323306pt}%
\definecolor{currentstroke}{rgb}{0.271305,0.019942,0.347269}%
\pgfsetstrokecolor{currentstroke}%
\pgfsetdash{}{0pt}%
\pgfpathmoveto{\pgfqpoint{5.792949in}{2.481802in}}%
\pgfpathlineto{\pgfqpoint{5.742816in}{2.481600in}}%
\pgfusepath{stroke}%
\end{pgfscope}%
\begin{pgfscope}%
\pgfpathrectangle{\pgfqpoint{3.985294in}{1.750000in}}{\pgfqpoint{2.279412in}{2.004545in}}%
\pgfusepath{clip}%
\pgfsetbuttcap%
\pgfsetroundjoin%
\pgfsetlinewidth{0.328559pt}%
\definecolor{currentstroke}{rgb}{0.271305,0.019942,0.347269}%
\pgfsetstrokecolor{currentstroke}%
\pgfsetdash{}{0pt}%
\pgfpathmoveto{\pgfqpoint{5.742816in}{2.481600in}}%
\pgfpathlineto{\pgfqpoint{5.692676in}{2.481978in}}%
\pgfusepath{stroke}%
\end{pgfscope}%
\begin{pgfscope}%
\pgfpathrectangle{\pgfqpoint{3.985294in}{1.750000in}}{\pgfqpoint{2.279412in}{2.004545in}}%
\pgfusepath{clip}%
\pgfsetbuttcap%
\pgfsetroundjoin%
\pgfsetlinewidth{0.344703pt}%
\definecolor{currentstroke}{rgb}{0.274952,0.037752,0.364543}%
\pgfsetstrokecolor{currentstroke}%
\pgfsetdash{}{0pt}%
\pgfpathmoveto{\pgfqpoint{5.692676in}{2.481978in}}%
\pgfpathlineto{\pgfqpoint{5.642543in}{2.483033in}}%
\pgfusepath{stroke}%
\end{pgfscope}%
\begin{pgfscope}%
\pgfpathrectangle{\pgfqpoint{3.985294in}{1.750000in}}{\pgfqpoint{2.279412in}{2.004545in}}%
\pgfusepath{clip}%
\pgfsetbuttcap%
\pgfsetroundjoin%
\pgfsetlinewidth{0.367215pt}%
\definecolor{currentstroke}{rgb}{0.277941,0.056324,0.381191}%
\pgfsetstrokecolor{currentstroke}%
\pgfsetdash{}{0pt}%
\pgfpathmoveto{\pgfqpoint{5.642543in}{2.483033in}}%
\pgfpathlineto{\pgfqpoint{5.592418in}{2.484407in}}%
\pgfusepath{stroke}%
\end{pgfscope}%
\begin{pgfscope}%
\pgfpathrectangle{\pgfqpoint{3.985294in}{1.750000in}}{\pgfqpoint{2.279412in}{2.004545in}}%
\pgfusepath{clip}%
\pgfsetbuttcap%
\pgfsetroundjoin%
\pgfsetlinewidth{0.394714pt}%
\definecolor{currentstroke}{rgb}{0.280894,0.078907,0.402329}%
\pgfsetstrokecolor{currentstroke}%
\pgfsetdash{}{0pt}%
\pgfpathmoveto{\pgfqpoint{5.592418in}{2.484407in}}%
\pgfpathlineto{\pgfqpoint{5.542298in}{2.485923in}}%
\pgfusepath{stroke}%
\end{pgfscope}%
\begin{pgfscope}%
\pgfpathrectangle{\pgfqpoint{3.985294in}{1.750000in}}{\pgfqpoint{2.279412in}{2.004545in}}%
\pgfusepath{clip}%
\pgfsetbuttcap%
\pgfsetroundjoin%
\pgfsetlinewidth{0.432137pt}%
\definecolor{currentstroke}{rgb}{0.283091,0.110553,0.431554}%
\pgfsetstrokecolor{currentstroke}%
\pgfsetdash{}{0pt}%
\pgfpathmoveto{\pgfqpoint{5.542298in}{2.485923in}}%
\pgfpathlineto{\pgfqpoint{5.492190in}{2.487759in}}%
\pgfusepath{stroke}%
\end{pgfscope}%
\begin{pgfscope}%
\pgfpathrectangle{\pgfqpoint{3.985294in}{1.750000in}}{\pgfqpoint{2.279412in}{2.004545in}}%
\pgfusepath{clip}%
\pgfsetbuttcap%
\pgfsetroundjoin%
\pgfsetlinewidth{0.482819pt}%
\definecolor{currentstroke}{rgb}{0.282290,0.145912,0.461510}%
\pgfsetstrokecolor{currentstroke}%
\pgfsetdash{}{0pt}%
\pgfpathmoveto{\pgfqpoint{5.492190in}{2.487759in}}%
\pgfpathlineto{\pgfqpoint{5.442108in}{2.490062in}}%
\pgfusepath{stroke}%
\end{pgfscope}%
\begin{pgfscope}%
\pgfpathrectangle{\pgfqpoint{3.985294in}{1.750000in}}{\pgfqpoint{2.279412in}{2.004545in}}%
\pgfusepath{clip}%
\pgfsetbuttcap%
\pgfsetroundjoin%
\pgfsetlinewidth{0.514619pt}%
\definecolor{currentstroke}{rgb}{0.279574,0.170599,0.479997}%
\pgfsetstrokecolor{currentstroke}%
\pgfsetdash{}{0pt}%
\pgfpathmoveto{\pgfqpoint{5.442108in}{2.490062in}}%
\pgfpathlineto{\pgfqpoint{5.392066in}{2.492967in}}%
\pgfusepath{stroke}%
\end{pgfscope}%
\begin{pgfscope}%
\pgfpathrectangle{\pgfqpoint{3.985294in}{1.750000in}}{\pgfqpoint{2.279412in}{2.004545in}}%
\pgfusepath{clip}%
\pgfsetbuttcap%
\pgfsetroundjoin%
\pgfsetlinewidth{0.600859pt}%
\definecolor{currentstroke}{rgb}{0.266580,0.228262,0.514349}%
\pgfsetstrokecolor{currentstroke}%
\pgfsetdash{}{0pt}%
\pgfpathmoveto{\pgfqpoint{5.392066in}{2.492967in}}%
\pgfpathlineto{\pgfqpoint{5.342087in}{2.496580in}}%
\pgfusepath{stroke}%
\end{pgfscope}%
\begin{pgfscope}%
\pgfpathrectangle{\pgfqpoint{3.985294in}{1.750000in}}{\pgfqpoint{2.279412in}{2.004545in}}%
\pgfusepath{clip}%
\pgfsetbuttcap%
\pgfsetroundjoin%
\pgfsetlinewidth{0.642925pt}%
\definecolor{currentstroke}{rgb}{0.257322,0.256130,0.526563}%
\pgfsetstrokecolor{currentstroke}%
\pgfsetdash{}{0pt}%
\pgfpathmoveto{\pgfqpoint{5.342087in}{2.496580in}}%
\pgfpathlineto{\pgfqpoint{5.292206in}{2.501121in}}%
\pgfusepath{stroke}%
\end{pgfscope}%
\begin{pgfscope}%
\pgfpathrectangle{\pgfqpoint{3.985294in}{1.750000in}}{\pgfqpoint{2.279412in}{2.004545in}}%
\pgfusepath{clip}%
\pgfsetbuttcap%
\pgfsetroundjoin%
\pgfsetlinewidth{0.673228pt}%
\definecolor{currentstroke}{rgb}{0.248629,0.278775,0.534556}%
\pgfsetstrokecolor{currentstroke}%
\pgfsetdash{}{0pt}%
\pgfpathmoveto{\pgfqpoint{5.292206in}{2.501121in}}%
\pgfpathlineto{\pgfqpoint{5.242438in}{2.506526in}}%
\pgfusepath{stroke}%
\end{pgfscope}%
\begin{pgfscope}%
\pgfpathrectangle{\pgfqpoint{3.985294in}{1.750000in}}{\pgfqpoint{2.279412in}{2.004545in}}%
\pgfusepath{clip}%
\pgfsetbuttcap%
\pgfsetroundjoin%
\pgfsetlinewidth{0.757968pt}%
\definecolor{currentstroke}{rgb}{0.225863,0.330805,0.547314}%
\pgfsetstrokecolor{currentstroke}%
\pgfsetdash{}{0pt}%
\pgfpathmoveto{\pgfqpoint{5.242438in}{2.506526in}}%
\pgfpathlineto{\pgfqpoint{5.192867in}{2.513159in}}%
\pgfusepath{stroke}%
\end{pgfscope}%
\begin{pgfscope}%
\pgfpathrectangle{\pgfqpoint{3.985294in}{1.750000in}}{\pgfqpoint{2.279412in}{2.004545in}}%
\pgfusepath{clip}%
\pgfsetbuttcap%
\pgfsetroundjoin%
\pgfsetlinewidth{0.725278pt}%
\definecolor{currentstroke}{rgb}{0.235526,0.309527,0.542944}%
\pgfsetstrokecolor{currentstroke}%
\pgfsetdash{}{0pt}%
\pgfpathmoveto{\pgfqpoint{5.192867in}{2.513159in}}%
\pgfpathlineto{\pgfqpoint{5.143577in}{2.521264in}}%
\pgfusepath{stroke}%
\end{pgfscope}%
\begin{pgfscope}%
\pgfpathrectangle{\pgfqpoint{3.985294in}{1.750000in}}{\pgfqpoint{2.279412in}{2.004545in}}%
\pgfusepath{clip}%
\pgfsetbuttcap%
\pgfsetroundjoin%
\pgfsetlinewidth{0.731164pt}%
\definecolor{currentstroke}{rgb}{0.233603,0.313828,0.543914}%
\pgfsetstrokecolor{currentstroke}%
\pgfsetdash{}{0pt}%
\pgfpathmoveto{\pgfqpoint{5.143577in}{2.521264in}}%
\pgfpathlineto{\pgfqpoint{5.094629in}{2.530800in}}%
\pgfusepath{stroke}%
\end{pgfscope}%
\begin{pgfscope}%
\pgfpathrectangle{\pgfqpoint{3.985294in}{1.750000in}}{\pgfqpoint{2.279412in}{2.004545in}}%
\pgfusepath{clip}%
\pgfsetbuttcap%
\pgfsetroundjoin%
\pgfsetlinewidth{0.762922pt}%
\definecolor{currentstroke}{rgb}{0.223925,0.334994,0.548053}%
\pgfsetstrokecolor{currentstroke}%
\pgfsetdash{}{0pt}%
\pgfpathmoveto{\pgfqpoint{5.094629in}{2.530800in}}%
\pgfpathlineto{\pgfqpoint{5.046184in}{2.542122in}}%
\pgfusepath{stroke}%
\end{pgfscope}%
\begin{pgfscope}%
\pgfpathrectangle{\pgfqpoint{3.985294in}{1.750000in}}{\pgfqpoint{2.279412in}{2.004545in}}%
\pgfusepath{clip}%
\pgfsetbuttcap%
\pgfsetroundjoin%
\pgfsetlinewidth{0.747276pt}%
\definecolor{currentstroke}{rgb}{0.227802,0.326594,0.546532}%
\pgfsetstrokecolor{currentstroke}%
\pgfsetdash{}{0pt}%
\pgfpathmoveto{\pgfqpoint{5.046184in}{2.542122in}}%
\pgfpathlineto{\pgfqpoint{4.998523in}{2.555758in}}%
\pgfusepath{stroke}%
\end{pgfscope}%
\begin{pgfscope}%
\pgfpathrectangle{\pgfqpoint{3.985294in}{1.750000in}}{\pgfqpoint{2.279412in}{2.004545in}}%
\pgfusepath{clip}%
\pgfsetbuttcap%
\pgfsetroundjoin%
\pgfsetlinewidth{0.813264pt}%
\definecolor{currentstroke}{rgb}{0.210503,0.363727,0.552206}%
\pgfsetstrokecolor{currentstroke}%
\pgfsetdash{}{0pt}%
\pgfpathmoveto{\pgfqpoint{4.998523in}{2.555758in}}%
\pgfpathlineto{\pgfqpoint{4.951559in}{2.571176in}}%
\pgfusepath{stroke}%
\end{pgfscope}%
\begin{pgfscope}%
\pgfpathrectangle{\pgfqpoint{3.985294in}{1.750000in}}{\pgfqpoint{2.279412in}{2.004545in}}%
\pgfusepath{clip}%
\pgfsetbuttcap%
\pgfsetroundjoin%
\pgfsetlinewidth{0.838826pt}%
\definecolor{currentstroke}{rgb}{0.203063,0.379716,0.553925}%
\pgfsetstrokecolor{currentstroke}%
\pgfsetdash{}{0pt}%
\pgfpathmoveto{\pgfqpoint{4.951559in}{2.571176in}}%
\pgfpathlineto{\pgfqpoint{4.905382in}{2.588282in}}%
\pgfusepath{stroke}%
\end{pgfscope}%
\begin{pgfscope}%
\pgfpathrectangle{\pgfqpoint{3.985294in}{1.750000in}}{\pgfqpoint{2.279412in}{2.004545in}}%
\pgfusepath{clip}%
\pgfsetbuttcap%
\pgfsetroundjoin%
\pgfsetlinewidth{0.811673pt}%
\definecolor{currentstroke}{rgb}{0.210503,0.363727,0.552206}%
\pgfsetstrokecolor{currentstroke}%
\pgfsetdash{}{0pt}%
\pgfpathmoveto{\pgfqpoint{4.905382in}{2.588282in}}%
\pgfpathlineto{\pgfqpoint{4.859564in}{2.606068in}}%
\pgfusepath{stroke}%
\end{pgfscope}%
\begin{pgfscope}%
\pgfpathrectangle{\pgfqpoint{3.985294in}{1.750000in}}{\pgfqpoint{2.279412in}{2.004545in}}%
\pgfusepath{clip}%
\pgfsetbuttcap%
\pgfsetroundjoin%
\pgfsetlinewidth{0.321468pt}%
\definecolor{currentstroke}{rgb}{0.269944,0.014625,0.341379}%
\pgfsetstrokecolor{currentstroke}%
\pgfsetdash{}{0pt}%
\pgfpathmoveto{\pgfqpoint{5.843087in}{2.662059in}}%
\pgfpathlineto{\pgfqpoint{5.792964in}{2.662193in}}%
\pgfusepath{stroke}%
\end{pgfscope}%
\begin{pgfscope}%
\pgfpathrectangle{\pgfqpoint{3.985294in}{1.750000in}}{\pgfqpoint{2.279412in}{2.004545in}}%
\pgfusepath{clip}%
\pgfsetbuttcap%
\pgfsetroundjoin%
\pgfsetlinewidth{0.325606pt}%
\definecolor{currentstroke}{rgb}{0.271305,0.019942,0.347269}%
\pgfsetstrokecolor{currentstroke}%
\pgfsetdash{}{0pt}%
\pgfpathmoveto{\pgfqpoint{5.792964in}{2.662193in}}%
\pgfpathlineto{\pgfqpoint{5.742843in}{2.662295in}}%
\pgfusepath{stroke}%
\end{pgfscope}%
\begin{pgfscope}%
\pgfpathrectangle{\pgfqpoint{3.985294in}{1.750000in}}{\pgfqpoint{2.279412in}{2.004545in}}%
\pgfusepath{clip}%
\pgfsetbuttcap%
\pgfsetroundjoin%
\pgfsetlinewidth{0.332664pt}%
\definecolor{currentstroke}{rgb}{0.272594,0.025563,0.353093}%
\pgfsetstrokecolor{currentstroke}%
\pgfsetdash{}{0pt}%
\pgfpathmoveto{\pgfqpoint{5.742843in}{2.662295in}}%
\pgfpathlineto{\pgfqpoint{5.692712in}{2.662791in}}%
\pgfusepath{stroke}%
\end{pgfscope}%
\begin{pgfscope}%
\pgfpathrectangle{\pgfqpoint{3.985294in}{1.750000in}}{\pgfqpoint{2.279412in}{2.004545in}}%
\pgfusepath{clip}%
\pgfsetbuttcap%
\pgfsetroundjoin%
\pgfsetlinewidth{0.350174pt}%
\definecolor{currentstroke}{rgb}{0.276022,0.044167,0.370164}%
\pgfsetstrokecolor{currentstroke}%
\pgfsetdash{}{0pt}%
\pgfpathmoveto{\pgfqpoint{5.692712in}{2.662791in}}%
\pgfpathlineto{\pgfqpoint{5.642568in}{2.662754in}}%
\pgfusepath{stroke}%
\end{pgfscope}%
\begin{pgfscope}%
\pgfpathrectangle{\pgfqpoint{3.985294in}{1.750000in}}{\pgfqpoint{2.279412in}{2.004545in}}%
\pgfusepath{clip}%
\pgfsetbuttcap%
\pgfsetroundjoin%
\pgfsetlinewidth{0.377938pt}%
\definecolor{currentstroke}{rgb}{0.279566,0.067836,0.391917}%
\pgfsetstrokecolor{currentstroke}%
\pgfsetdash{}{0pt}%
\pgfpathmoveto{\pgfqpoint{5.642568in}{2.662754in}}%
\pgfpathlineto{\pgfqpoint{5.592418in}{2.662874in}}%
\pgfusepath{stroke}%
\end{pgfscope}%
\begin{pgfscope}%
\pgfpathrectangle{\pgfqpoint{3.985294in}{1.750000in}}{\pgfqpoint{2.279412in}{2.004545in}}%
\pgfusepath{clip}%
\pgfsetbuttcap%
\pgfsetroundjoin%
\pgfsetlinewidth{0.419775pt}%
\definecolor{currentstroke}{rgb}{0.282656,0.100196,0.422160}%
\pgfsetstrokecolor{currentstroke}%
\pgfsetdash{}{0pt}%
\pgfpathmoveto{\pgfqpoint{5.592418in}{2.662874in}}%
\pgfpathlineto{\pgfqpoint{5.542267in}{2.663088in}}%
\pgfusepath{stroke}%
\end{pgfscope}%
\begin{pgfscope}%
\pgfpathrectangle{\pgfqpoint{3.985294in}{1.750000in}}{\pgfqpoint{2.279412in}{2.004545in}}%
\pgfusepath{clip}%
\pgfsetbuttcap%
\pgfsetroundjoin%
\pgfsetlinewidth{0.481217pt}%
\definecolor{currentstroke}{rgb}{0.282290,0.145912,0.461510}%
\pgfsetstrokecolor{currentstroke}%
\pgfsetdash{}{0pt}%
\pgfpathmoveto{\pgfqpoint{5.542267in}{2.663088in}}%
\pgfpathlineto{\pgfqpoint{5.492116in}{2.663380in}}%
\pgfusepath{stroke}%
\end{pgfscope}%
\begin{pgfscope}%
\pgfpathrectangle{\pgfqpoint{3.985294in}{1.750000in}}{\pgfqpoint{2.279412in}{2.004545in}}%
\pgfusepath{clip}%
\pgfsetbuttcap%
\pgfsetroundjoin%
\pgfsetlinewidth{0.564277pt}%
\definecolor{currentstroke}{rgb}{0.273006,0.204520,0.501721}%
\pgfsetstrokecolor{currentstroke}%
\pgfsetdash{}{0pt}%
\pgfpathmoveto{\pgfqpoint{5.492116in}{2.663380in}}%
\pgfpathlineto{\pgfqpoint{5.441966in}{2.663778in}}%
\pgfusepath{stroke}%
\end{pgfscope}%
\begin{pgfscope}%
\pgfpathrectangle{\pgfqpoint{3.985294in}{1.750000in}}{\pgfqpoint{2.279412in}{2.004545in}}%
\pgfusepath{clip}%
\pgfsetbuttcap%
\pgfsetroundjoin%
\pgfsetlinewidth{0.675207pt}%
\definecolor{currentstroke}{rgb}{0.248629,0.278775,0.534556}%
\pgfsetstrokecolor{currentstroke}%
\pgfsetdash{}{0pt}%
\pgfpathmoveto{\pgfqpoint{5.441966in}{2.663778in}}%
\pgfpathlineto{\pgfqpoint{5.391817in}{2.664248in}}%
\pgfusepath{stroke}%
\end{pgfscope}%
\begin{pgfscope}%
\pgfpathrectangle{\pgfqpoint{3.985294in}{1.750000in}}{\pgfqpoint{2.279412in}{2.004545in}}%
\pgfusepath{clip}%
\pgfsetbuttcap%
\pgfsetroundjoin%
\pgfsetlinewidth{0.788553pt}%
\definecolor{currentstroke}{rgb}{0.216210,0.351535,0.550627}%
\pgfsetstrokecolor{currentstroke}%
\pgfsetdash{}{0pt}%
\pgfpathmoveto{\pgfqpoint{5.391817in}{2.664248in}}%
\pgfpathlineto{\pgfqpoint{5.341671in}{2.664898in}}%
\pgfusepath{stroke}%
\end{pgfscope}%
\begin{pgfscope}%
\pgfpathrectangle{\pgfqpoint{3.985294in}{1.750000in}}{\pgfqpoint{2.279412in}{2.004545in}}%
\pgfusepath{clip}%
\pgfsetbuttcap%
\pgfsetroundjoin%
\pgfsetlinewidth{0.864729pt}%
\definecolor{currentstroke}{rgb}{0.195860,0.395433,0.555276}%
\pgfsetstrokecolor{currentstroke}%
\pgfsetdash{}{0pt}%
\pgfpathmoveto{\pgfqpoint{5.341671in}{2.664898in}}%
\pgfpathlineto{\pgfqpoint{5.291530in}{2.665785in}}%
\pgfusepath{stroke}%
\end{pgfscope}%
\begin{pgfscope}%
\pgfpathrectangle{\pgfqpoint{3.985294in}{1.750000in}}{\pgfqpoint{2.279412in}{2.004545in}}%
\pgfusepath{clip}%
\pgfsetbuttcap%
\pgfsetroundjoin%
\pgfsetlinewidth{0.951399pt}%
\definecolor{currentstroke}{rgb}{0.175841,0.441290,0.557685}%
\pgfsetstrokecolor{currentstroke}%
\pgfsetdash{}{0pt}%
\pgfpathmoveto{\pgfqpoint{5.291530in}{2.665785in}}%
\pgfpathlineto{\pgfqpoint{5.241396in}{2.666958in}}%
\pgfusepath{stroke}%
\end{pgfscope}%
\begin{pgfscope}%
\pgfpathrectangle{\pgfqpoint{3.985294in}{1.750000in}}{\pgfqpoint{2.279412in}{2.004545in}}%
\pgfusepath{clip}%
\pgfsetbuttcap%
\pgfsetroundjoin%
\pgfsetlinewidth{0.987429pt}%
\definecolor{currentstroke}{rgb}{0.166617,0.463708,0.558119}%
\pgfsetstrokecolor{currentstroke}%
\pgfsetdash{}{0pt}%
\pgfpathmoveto{\pgfqpoint{5.241396in}{2.666958in}}%
\pgfpathlineto{\pgfqpoint{5.191272in}{2.668378in}}%
\pgfusepath{stroke}%
\end{pgfscope}%
\begin{pgfscope}%
\pgfpathrectangle{\pgfqpoint{3.985294in}{1.750000in}}{\pgfqpoint{2.279412in}{2.004545in}}%
\pgfusepath{clip}%
\pgfsetbuttcap%
\pgfsetroundjoin%
\pgfsetlinewidth{0.998662pt}%
\definecolor{currentstroke}{rgb}{0.165117,0.467423,0.558141}%
\pgfsetstrokecolor{currentstroke}%
\pgfsetdash{}{0pt}%
\pgfpathmoveto{\pgfqpoint{5.191272in}{2.668378in}}%
\pgfpathlineto{\pgfqpoint{5.141166in}{2.670214in}}%
\pgfusepath{stroke}%
\end{pgfscope}%
\begin{pgfscope}%
\pgfpathrectangle{\pgfqpoint{3.985294in}{1.750000in}}{\pgfqpoint{2.279412in}{2.004545in}}%
\pgfusepath{clip}%
\pgfsetbuttcap%
\pgfsetroundjoin%
\pgfsetlinewidth{0.991063pt}%
\definecolor{currentstroke}{rgb}{0.166617,0.463708,0.558119}%
\pgfsetstrokecolor{currentstroke}%
\pgfsetdash{}{0pt}%
\pgfpathmoveto{\pgfqpoint{5.141166in}{2.670214in}}%
\pgfpathlineto{\pgfqpoint{5.091086in}{2.672553in}}%
\pgfusepath{stroke}%
\end{pgfscope}%
\begin{pgfscope}%
\pgfpathrectangle{\pgfqpoint{3.985294in}{1.750000in}}{\pgfqpoint{2.279412in}{2.004545in}}%
\pgfusepath{clip}%
\pgfsetbuttcap%
\pgfsetroundjoin%
\pgfsetlinewidth{1.021856pt}%
\definecolor{currentstroke}{rgb}{0.159194,0.482237,0.558073}%
\pgfsetstrokecolor{currentstroke}%
\pgfsetdash{}{0pt}%
\pgfpathmoveto{\pgfqpoint{5.091086in}{2.672553in}}%
\pgfpathlineto{\pgfqpoint{5.041031in}{2.675275in}}%
\pgfusepath{stroke}%
\end{pgfscope}%
\begin{pgfscope}%
\pgfpathrectangle{\pgfqpoint{3.985294in}{1.750000in}}{\pgfqpoint{2.279412in}{2.004545in}}%
\pgfusepath{clip}%
\pgfsetbuttcap%
\pgfsetroundjoin%
\pgfsetlinewidth{0.964873pt}%
\definecolor{currentstroke}{rgb}{0.172719,0.448791,0.557885}%
\pgfsetstrokecolor{currentstroke}%
\pgfsetdash{}{0pt}%
\pgfpathmoveto{\pgfqpoint{5.041031in}{2.675275in}}%
\pgfpathlineto{\pgfqpoint{4.991021in}{2.678524in}}%
\pgfusepath{stroke}%
\end{pgfscope}%
\begin{pgfscope}%
\pgfpathrectangle{\pgfqpoint{3.985294in}{1.750000in}}{\pgfqpoint{2.279412in}{2.004545in}}%
\pgfusepath{clip}%
\pgfsetbuttcap%
\pgfsetroundjoin%
\pgfsetlinewidth{0.323155pt}%
\definecolor{currentstroke}{rgb}{0.271305,0.019942,0.347269}%
\pgfsetstrokecolor{currentstroke}%
\pgfsetdash{}{0pt}%
\pgfpathmoveto{\pgfqpoint{5.843087in}{2.887593in}}%
\pgfpathlineto{\pgfqpoint{5.792951in}{2.886911in}}%
\pgfusepath{stroke}%
\end{pgfscope}%
\begin{pgfscope}%
\pgfpathrectangle{\pgfqpoint{3.985294in}{1.750000in}}{\pgfqpoint{2.279412in}{2.004545in}}%
\pgfusepath{clip}%
\pgfsetbuttcap%
\pgfsetroundjoin%
\pgfsetlinewidth{0.320576pt}%
\definecolor{currentstroke}{rgb}{0.269944,0.014625,0.341379}%
\pgfsetstrokecolor{currentstroke}%
\pgfsetdash{}{0pt}%
\pgfpathmoveto{\pgfqpoint{5.792951in}{2.886911in}}%
\pgfpathlineto{\pgfqpoint{5.742809in}{2.886637in}}%
\pgfusepath{stroke}%
\end{pgfscope}%
\begin{pgfscope}%
\pgfpathrectangle{\pgfqpoint{3.985294in}{1.750000in}}{\pgfqpoint{2.279412in}{2.004545in}}%
\pgfusepath{clip}%
\pgfsetbuttcap%
\pgfsetroundjoin%
\pgfsetlinewidth{0.339115pt}%
\definecolor{currentstroke}{rgb}{0.273809,0.031497,0.358853}%
\pgfsetstrokecolor{currentstroke}%
\pgfsetdash{}{0pt}%
\pgfpathmoveto{\pgfqpoint{5.742809in}{2.886637in}}%
\pgfpathlineto{\pgfqpoint{5.692660in}{2.886926in}}%
\pgfusepath{stroke}%
\end{pgfscope}%
\begin{pgfscope}%
\pgfpathrectangle{\pgfqpoint{3.985294in}{1.750000in}}{\pgfqpoint{2.279412in}{2.004545in}}%
\pgfusepath{clip}%
\pgfsetbuttcap%
\pgfsetroundjoin%
\pgfsetlinewidth{0.344296pt}%
\definecolor{currentstroke}{rgb}{0.274952,0.037752,0.364543}%
\pgfsetstrokecolor{currentstroke}%
\pgfsetdash{}{0pt}%
\pgfpathmoveto{\pgfqpoint{5.692660in}{2.886926in}}%
\pgfpathlineto{\pgfqpoint{5.642521in}{2.887680in}}%
\pgfusepath{stroke}%
\end{pgfscope}%
\begin{pgfscope}%
\pgfpathrectangle{\pgfqpoint{3.985294in}{1.750000in}}{\pgfqpoint{2.279412in}{2.004545in}}%
\pgfusepath{clip}%
\pgfsetbuttcap%
\pgfsetroundjoin%
\pgfsetlinewidth{0.375296pt}%
\definecolor{currentstroke}{rgb}{0.278791,0.062145,0.386592}%
\pgfsetstrokecolor{currentstroke}%
\pgfsetdash{}{0pt}%
\pgfpathmoveto{\pgfqpoint{5.642521in}{2.887680in}}%
\pgfpathlineto{\pgfqpoint{5.592381in}{2.887911in}}%
\pgfusepath{stroke}%
\end{pgfscope}%
\begin{pgfscope}%
\pgfpathrectangle{\pgfqpoint{3.985294in}{1.750000in}}{\pgfqpoint{2.279412in}{2.004545in}}%
\pgfusepath{clip}%
\pgfsetbuttcap%
\pgfsetroundjoin%
\pgfsetlinewidth{0.420584pt}%
\definecolor{currentstroke}{rgb}{0.282656,0.100196,0.422160}%
\pgfsetstrokecolor{currentstroke}%
\pgfsetdash{}{0pt}%
\pgfpathmoveto{\pgfqpoint{5.592381in}{2.887911in}}%
\pgfpathlineto{\pgfqpoint{5.542235in}{2.887284in}}%
\pgfusepath{stroke}%
\end{pgfscope}%
\begin{pgfscope}%
\pgfpathrectangle{\pgfqpoint{3.985294in}{1.750000in}}{\pgfqpoint{2.279412in}{2.004545in}}%
\pgfusepath{clip}%
\pgfsetbuttcap%
\pgfsetroundjoin%
\pgfsetlinewidth{0.464085pt}%
\definecolor{currentstroke}{rgb}{0.283072,0.130895,0.449241}%
\pgfsetstrokecolor{currentstroke}%
\pgfsetdash{}{0pt}%
\pgfpathmoveto{\pgfqpoint{5.542235in}{2.887284in}}%
\pgfpathlineto{\pgfqpoint{5.492093in}{2.886445in}}%
\pgfusepath{stroke}%
\end{pgfscope}%
\begin{pgfscope}%
\pgfpathrectangle{\pgfqpoint{3.985294in}{1.750000in}}{\pgfqpoint{2.279412in}{2.004545in}}%
\pgfusepath{clip}%
\pgfsetbuttcap%
\pgfsetroundjoin%
\pgfsetlinewidth{0.539487pt}%
\definecolor{currentstroke}{rgb}{0.277134,0.185228,0.489898}%
\pgfsetstrokecolor{currentstroke}%
\pgfsetdash{}{0pt}%
\pgfpathmoveto{\pgfqpoint{5.492093in}{2.886445in}}%
\pgfpathlineto{\pgfqpoint{5.441954in}{2.885438in}}%
\pgfusepath{stroke}%
\end{pgfscope}%
\begin{pgfscope}%
\pgfpathrectangle{\pgfqpoint{3.985294in}{1.750000in}}{\pgfqpoint{2.279412in}{2.004545in}}%
\pgfusepath{clip}%
\pgfsetbuttcap%
\pgfsetroundjoin%
\pgfsetlinewidth{0.622699pt}%
\definecolor{currentstroke}{rgb}{0.262138,0.242286,0.520837}%
\pgfsetstrokecolor{currentstroke}%
\pgfsetdash{}{0pt}%
\pgfpathmoveto{\pgfqpoint{5.441954in}{2.885438in}}%
\pgfpathlineto{\pgfqpoint{5.391823in}{2.884175in}}%
\pgfusepath{stroke}%
\end{pgfscope}%
\begin{pgfscope}%
\pgfpathrectangle{\pgfqpoint{3.985294in}{1.750000in}}{\pgfqpoint{2.279412in}{2.004545in}}%
\pgfusepath{clip}%
\pgfsetbuttcap%
\pgfsetroundjoin%
\pgfsetlinewidth{0.676509pt}%
\definecolor{currentstroke}{rgb}{0.248629,0.278775,0.534556}%
\pgfsetstrokecolor{currentstroke}%
\pgfsetdash{}{0pt}%
\pgfpathmoveto{\pgfqpoint{5.391823in}{2.884175in}}%
\pgfpathlineto{\pgfqpoint{5.341712in}{2.882430in}}%
\pgfusepath{stroke}%
\end{pgfscope}%
\begin{pgfscope}%
\pgfpathrectangle{\pgfqpoint{3.985294in}{1.750000in}}{\pgfqpoint{2.279412in}{2.004545in}}%
\pgfusepath{clip}%
\pgfsetbuttcap%
\pgfsetroundjoin%
\pgfsetlinewidth{0.745548pt}%
\definecolor{currentstroke}{rgb}{0.229739,0.322361,0.545706}%
\pgfsetstrokecolor{currentstroke}%
\pgfsetdash{}{0pt}%
\pgfpathmoveto{\pgfqpoint{5.341712in}{2.882430in}}%
\pgfpathlineto{\pgfqpoint{5.291634in}{2.880080in}}%
\pgfusepath{stroke}%
\end{pgfscope}%
\begin{pgfscope}%
\pgfpathrectangle{\pgfqpoint{3.985294in}{1.750000in}}{\pgfqpoint{2.279412in}{2.004545in}}%
\pgfusepath{clip}%
\pgfsetbuttcap%
\pgfsetroundjoin%
\pgfsetlinewidth{0.809140pt}%
\definecolor{currentstroke}{rgb}{0.210503,0.363727,0.552206}%
\pgfsetstrokecolor{currentstroke}%
\pgfsetdash{}{0pt}%
\pgfpathmoveto{\pgfqpoint{5.291634in}{2.880080in}}%
\pgfpathlineto{\pgfqpoint{5.241603in}{2.877057in}}%
\pgfusepath{stroke}%
\end{pgfscope}%
\begin{pgfscope}%
\pgfpathrectangle{\pgfqpoint{3.985294in}{1.750000in}}{\pgfqpoint{2.279412in}{2.004545in}}%
\pgfusepath{clip}%
\pgfsetbuttcap%
\pgfsetroundjoin%
\pgfsetlinewidth{0.872398pt}%
\definecolor{currentstroke}{rgb}{0.194100,0.399323,0.555565}%
\pgfsetstrokecolor{currentstroke}%
\pgfsetdash{}{0pt}%
\pgfpathmoveto{\pgfqpoint{5.241603in}{2.877057in}}%
\pgfpathlineto{\pgfqpoint{5.191622in}{2.873430in}}%
\pgfusepath{stroke}%
\end{pgfscope}%
\begin{pgfscope}%
\pgfpathrectangle{\pgfqpoint{3.985294in}{1.750000in}}{\pgfqpoint{2.279412in}{2.004545in}}%
\pgfusepath{clip}%
\pgfsetbuttcap%
\pgfsetroundjoin%
\pgfsetlinewidth{0.313308pt}%
\definecolor{currentstroke}{rgb}{0.268510,0.009605,0.335427}%
\pgfsetstrokecolor{currentstroke}%
\pgfsetdash{}{0pt}%
\pgfpathmoveto{\pgfqpoint{5.843087in}{2.977807in}}%
\pgfpathlineto{\pgfqpoint{5.792979in}{2.977111in}}%
\pgfusepath{stroke}%
\end{pgfscope}%
\begin{pgfscope}%
\pgfpathrectangle{\pgfqpoint{3.985294in}{1.750000in}}{\pgfqpoint{2.279412in}{2.004545in}}%
\pgfusepath{clip}%
\pgfsetbuttcap%
\pgfsetroundjoin%
\pgfsetlinewidth{0.320968pt}%
\definecolor{currentstroke}{rgb}{0.269944,0.014625,0.341379}%
\pgfsetstrokecolor{currentstroke}%
\pgfsetdash{}{0pt}%
\pgfpathmoveto{\pgfqpoint{5.792979in}{2.977111in}}%
\pgfpathlineto{\pgfqpoint{5.742844in}{2.976303in}}%
\pgfusepath{stroke}%
\end{pgfscope}%
\begin{pgfscope}%
\pgfpathrectangle{\pgfqpoint{3.985294in}{1.750000in}}{\pgfqpoint{2.279412in}{2.004545in}}%
\pgfusepath{clip}%
\pgfsetbuttcap%
\pgfsetroundjoin%
\pgfsetlinewidth{0.329863pt}%
\definecolor{currentstroke}{rgb}{0.272594,0.025563,0.353093}%
\pgfsetstrokecolor{currentstroke}%
\pgfsetdash{}{0pt}%
\pgfpathmoveto{\pgfqpoint{5.742844in}{2.976303in}}%
\pgfpathlineto{\pgfqpoint{5.692701in}{2.975776in}}%
\pgfusepath{stroke}%
\end{pgfscope}%
\begin{pgfscope}%
\pgfpathrectangle{\pgfqpoint{3.985294in}{1.750000in}}{\pgfqpoint{2.279412in}{2.004545in}}%
\pgfusepath{clip}%
\pgfsetbuttcap%
\pgfsetroundjoin%
\pgfsetlinewidth{0.347330pt}%
\definecolor{currentstroke}{rgb}{0.274952,0.037752,0.364543}%
\pgfsetstrokecolor{currentstroke}%
\pgfsetdash{}{0pt}%
\pgfpathmoveto{\pgfqpoint{5.692701in}{2.975776in}}%
\pgfpathlineto{\pgfqpoint{5.642559in}{2.975047in}}%
\pgfusepath{stroke}%
\end{pgfscope}%
\begin{pgfscope}%
\pgfpathrectangle{\pgfqpoint{3.985294in}{1.750000in}}{\pgfqpoint{2.279412in}{2.004545in}}%
\pgfusepath{clip}%
\pgfsetbuttcap%
\pgfsetroundjoin%
\pgfsetlinewidth{0.368063pt}%
\definecolor{currentstroke}{rgb}{0.277941,0.056324,0.381191}%
\pgfsetstrokecolor{currentstroke}%
\pgfsetdash{}{0pt}%
\pgfpathmoveto{\pgfqpoint{5.642559in}{2.975047in}}%
\pgfpathlineto{\pgfqpoint{5.592421in}{2.974043in}}%
\pgfusepath{stroke}%
\end{pgfscope}%
\begin{pgfscope}%
\pgfpathrectangle{\pgfqpoint{3.985294in}{1.750000in}}{\pgfqpoint{2.279412in}{2.004545in}}%
\pgfusepath{clip}%
\pgfsetbuttcap%
\pgfsetroundjoin%
\pgfsetlinewidth{0.391561pt}%
\definecolor{currentstroke}{rgb}{0.280894,0.078907,0.402329}%
\pgfsetstrokecolor{currentstroke}%
\pgfsetdash{}{0pt}%
\pgfpathmoveto{\pgfqpoint{5.592421in}{2.974043in}}%
\pgfpathlineto{\pgfqpoint{5.542287in}{2.972874in}}%
\pgfusepath{stroke}%
\end{pgfscope}%
\begin{pgfscope}%
\pgfpathrectangle{\pgfqpoint{3.985294in}{1.750000in}}{\pgfqpoint{2.279412in}{2.004545in}}%
\pgfusepath{clip}%
\pgfsetbuttcap%
\pgfsetroundjoin%
\pgfsetlinewidth{0.446461pt}%
\definecolor{currentstroke}{rgb}{0.283229,0.120777,0.440584}%
\pgfsetstrokecolor{currentstroke}%
\pgfsetdash{}{0pt}%
\pgfpathmoveto{\pgfqpoint{5.542287in}{2.972874in}}%
\pgfpathlineto{\pgfqpoint{5.492167in}{2.971323in}}%
\pgfusepath{stroke}%
\end{pgfscope}%
\begin{pgfscope}%
\pgfpathrectangle{\pgfqpoint{3.985294in}{1.750000in}}{\pgfqpoint{2.279412in}{2.004545in}}%
\pgfusepath{clip}%
\pgfsetbuttcap%
\pgfsetroundjoin%
\pgfsetlinewidth{0.493124pt}%
\definecolor{currentstroke}{rgb}{0.281887,0.150881,0.465405}%
\pgfsetstrokecolor{currentstroke}%
\pgfsetdash{}{0pt}%
\pgfpathmoveto{\pgfqpoint{5.492167in}{2.971323in}}%
\pgfpathlineto{\pgfqpoint{5.442065in}{2.969359in}}%
\pgfusepath{stroke}%
\end{pgfscope}%
\begin{pgfscope}%
\pgfpathrectangle{\pgfqpoint{3.985294in}{1.750000in}}{\pgfqpoint{2.279412in}{2.004545in}}%
\pgfusepath{clip}%
\pgfsetbuttcap%
\pgfsetroundjoin%
\pgfsetlinewidth{0.536074pt}%
\definecolor{currentstroke}{rgb}{0.277134,0.185228,0.489898}%
\pgfsetstrokecolor{currentstroke}%
\pgfsetdash{}{0pt}%
\pgfpathmoveto{\pgfqpoint{5.442065in}{2.969359in}}%
\pgfpathlineto{\pgfqpoint{5.392002in}{2.966786in}}%
\pgfusepath{stroke}%
\end{pgfscope}%
\begin{pgfscope}%
\pgfpathrectangle{\pgfqpoint{3.985294in}{1.750000in}}{\pgfqpoint{2.279412in}{2.004545in}}%
\pgfusepath{clip}%
\pgfsetbuttcap%
\pgfsetroundjoin%
\pgfsetlinewidth{0.603283pt}%
\definecolor{currentstroke}{rgb}{0.265145,0.232956,0.516599}%
\pgfsetstrokecolor{currentstroke}%
\pgfsetdash{}{0pt}%
\pgfpathmoveto{\pgfqpoint{5.392002in}{2.966786in}}%
\pgfpathlineto{\pgfqpoint{5.341992in}{2.963502in}}%
\pgfusepath{stroke}%
\end{pgfscope}%
\begin{pgfscope}%
\pgfpathrectangle{\pgfqpoint{3.985294in}{1.750000in}}{\pgfqpoint{2.279412in}{2.004545in}}%
\pgfusepath{clip}%
\pgfsetbuttcap%
\pgfsetroundjoin%
\pgfsetlinewidth{0.630663pt}%
\definecolor{currentstroke}{rgb}{0.258965,0.251537,0.524736}%
\pgfsetstrokecolor{currentstroke}%
\pgfsetdash{}{0pt}%
\pgfpathmoveto{\pgfqpoint{5.341992in}{2.963502in}}%
\pgfpathlineto{\pgfqpoint{5.292048in}{2.959532in}}%
\pgfusepath{stroke}%
\end{pgfscope}%
\begin{pgfscope}%
\pgfpathrectangle{\pgfqpoint{3.985294in}{1.750000in}}{\pgfqpoint{2.279412in}{2.004545in}}%
\pgfusepath{clip}%
\pgfsetbuttcap%
\pgfsetroundjoin%
\pgfsetlinewidth{0.320971pt}%
\definecolor{currentstroke}{rgb}{0.269944,0.014625,0.341379}%
\pgfsetstrokecolor{currentstroke}%
\pgfsetdash{}{0pt}%
\pgfpathmoveto{\pgfqpoint{5.843087in}{3.022913in}}%
\pgfpathlineto{\pgfqpoint{5.793026in}{3.021658in}}%
\pgfusepath{stroke}%
\end{pgfscope}%
\begin{pgfscope}%
\pgfpathrectangle{\pgfqpoint{3.985294in}{1.750000in}}{\pgfqpoint{2.279412in}{2.004545in}}%
\pgfusepath{clip}%
\pgfsetbuttcap%
\pgfsetroundjoin%
\pgfsetlinewidth{0.318784pt}%
\definecolor{currentstroke}{rgb}{0.269944,0.014625,0.341379}%
\pgfsetstrokecolor{currentstroke}%
\pgfsetdash{}{0pt}%
\pgfpathmoveto{\pgfqpoint{5.793026in}{3.021658in}}%
\pgfpathlineto{\pgfqpoint{5.742923in}{3.020707in}}%
\pgfusepath{stroke}%
\end{pgfscope}%
\begin{pgfscope}%
\pgfpathrectangle{\pgfqpoint{3.985294in}{1.750000in}}{\pgfqpoint{2.279412in}{2.004545in}}%
\pgfusepath{clip}%
\pgfsetbuttcap%
\pgfsetroundjoin%
\pgfsetlinewidth{0.331222pt}%
\definecolor{currentstroke}{rgb}{0.272594,0.025563,0.353093}%
\pgfsetstrokecolor{currentstroke}%
\pgfsetdash{}{0pt}%
\pgfpathmoveto{\pgfqpoint{5.742923in}{3.020707in}}%
\pgfpathlineto{\pgfqpoint{5.692793in}{3.019718in}}%
\pgfusepath{stroke}%
\end{pgfscope}%
\begin{pgfscope}%
\pgfpathrectangle{\pgfqpoint{3.985294in}{1.750000in}}{\pgfqpoint{2.279412in}{2.004545in}}%
\pgfusepath{clip}%
\pgfsetbuttcap%
\pgfsetroundjoin%
\pgfsetlinewidth{0.344419pt}%
\definecolor{currentstroke}{rgb}{0.274952,0.037752,0.364543}%
\pgfsetstrokecolor{currentstroke}%
\pgfsetdash{}{0pt}%
\pgfpathmoveto{\pgfqpoint{5.692793in}{3.019718in}}%
\pgfpathlineto{\pgfqpoint{5.642671in}{3.018237in}}%
\pgfusepath{stroke}%
\end{pgfscope}%
\begin{pgfscope}%
\pgfpathrectangle{\pgfqpoint{3.985294in}{1.750000in}}{\pgfqpoint{2.279412in}{2.004545in}}%
\pgfusepath{clip}%
\pgfsetbuttcap%
\pgfsetroundjoin%
\pgfsetlinewidth{0.369752pt}%
\definecolor{currentstroke}{rgb}{0.278791,0.062145,0.386592}%
\pgfsetstrokecolor{currentstroke}%
\pgfsetdash{}{0pt}%
\pgfpathmoveto{\pgfqpoint{5.642671in}{3.018237in}}%
\pgfpathlineto{\pgfqpoint{5.592552in}{3.016668in}}%
\pgfusepath{stroke}%
\end{pgfscope}%
\begin{pgfscope}%
\pgfpathrectangle{\pgfqpoint{3.985294in}{1.750000in}}{\pgfqpoint{2.279412in}{2.004545in}}%
\pgfusepath{clip}%
\pgfsetbuttcap%
\pgfsetroundjoin%
\pgfsetlinewidth{0.387876pt}%
\definecolor{currentstroke}{rgb}{0.280267,0.073417,0.397163}%
\pgfsetstrokecolor{currentstroke}%
\pgfsetdash{}{0pt}%
\pgfpathmoveto{\pgfqpoint{5.592552in}{3.016668in}}%
\pgfpathlineto{\pgfqpoint{5.542438in}{3.014999in}}%
\pgfusepath{stroke}%
\end{pgfscope}%
\begin{pgfscope}%
\pgfpathrectangle{\pgfqpoint{3.985294in}{1.750000in}}{\pgfqpoint{2.279412in}{2.004545in}}%
\pgfusepath{clip}%
\pgfsetbuttcap%
\pgfsetroundjoin%
\pgfsetlinewidth{0.428392pt}%
\definecolor{currentstroke}{rgb}{0.282910,0.105393,0.426902}%
\pgfsetstrokecolor{currentstroke}%
\pgfsetdash{}{0pt}%
\pgfpathmoveto{\pgfqpoint{5.542438in}{3.014999in}}%
\pgfpathlineto{\pgfqpoint{5.492351in}{3.012800in}}%
\pgfusepath{stroke}%
\end{pgfscope}%
\begin{pgfscope}%
\pgfpathrectangle{\pgfqpoint{3.985294in}{1.750000in}}{\pgfqpoint{2.279412in}{2.004545in}}%
\pgfusepath{clip}%
\pgfsetbuttcap%
\pgfsetroundjoin%
\pgfsetlinewidth{0.457362pt}%
\definecolor{currentstroke}{rgb}{0.283187,0.125848,0.444960}%
\pgfsetstrokecolor{currentstroke}%
\pgfsetdash{}{0pt}%
\pgfpathmoveto{\pgfqpoint{5.492351in}{3.012800in}}%
\pgfpathlineto{\pgfqpoint{5.442297in}{3.010048in}}%
\pgfusepath{stroke}%
\end{pgfscope}%
\begin{pgfscope}%
\pgfpathrectangle{\pgfqpoint{3.985294in}{1.750000in}}{\pgfqpoint{2.279412in}{2.004545in}}%
\pgfusepath{clip}%
\pgfsetbuttcap%
\pgfsetroundjoin%
\pgfsetlinewidth{0.495927pt}%
\definecolor{currentstroke}{rgb}{0.281412,0.155834,0.469201}%
\pgfsetstrokecolor{currentstroke}%
\pgfsetdash{}{0pt}%
\pgfpathmoveto{\pgfqpoint{5.442297in}{3.010048in}}%
\pgfpathlineto{\pgfqpoint{5.392271in}{3.006940in}}%
\pgfusepath{stroke}%
\end{pgfscope}%
\begin{pgfscope}%
\pgfpathrectangle{\pgfqpoint{3.985294in}{1.750000in}}{\pgfqpoint{2.279412in}{2.004545in}}%
\pgfusepath{clip}%
\pgfsetbuttcap%
\pgfsetroundjoin%
\pgfsetlinewidth{0.557416pt}%
\definecolor{currentstroke}{rgb}{0.274128,0.199721,0.498911}%
\pgfsetstrokecolor{currentstroke}%
\pgfsetdash{}{0pt}%
\pgfpathmoveto{\pgfqpoint{5.392271in}{3.006940in}}%
\pgfpathlineto{\pgfqpoint{5.342311in}{3.003139in}}%
\pgfusepath{stroke}%
\end{pgfscope}%
\begin{pgfscope}%
\pgfpathrectangle{\pgfqpoint{3.985294in}{1.750000in}}{\pgfqpoint{2.279412in}{2.004545in}}%
\pgfusepath{clip}%
\pgfsetbuttcap%
\pgfsetroundjoin%
\pgfsetlinewidth{0.308259pt}%
\definecolor{currentstroke}{rgb}{0.268510,0.009605,0.335427}%
\pgfsetstrokecolor{currentstroke}%
\pgfsetdash{}{0pt}%
\pgfpathmoveto{\pgfqpoint{5.843087in}{3.203341in}}%
\pgfpathlineto{\pgfqpoint{5.843087in}{3.203341in}}%
\pgfusepath{stroke}%
\end{pgfscope}%
\begin{pgfscope}%
\pgfpathrectangle{\pgfqpoint{3.985294in}{1.750000in}}{\pgfqpoint{2.279412in}{2.004545in}}%
\pgfusepath{clip}%
\pgfsetbuttcap%
\pgfsetroundjoin%
\pgfsetlinewidth{0.308259pt}%
\definecolor{currentstroke}{rgb}{0.268510,0.009605,0.335427}%
\pgfsetstrokecolor{currentstroke}%
\pgfsetdash{}{0pt}%
\pgfpathmoveto{\pgfqpoint{5.843087in}{3.203341in}}%
\pgfpathlineto{\pgfqpoint{5.830324in}{3.199484in}}%
\pgfusepath{stroke}%
\end{pgfscope}%
\begin{pgfscope}%
\pgfpathrectangle{\pgfqpoint{3.985294in}{1.750000in}}{\pgfqpoint{2.279412in}{2.004545in}}%
\pgfusepath{clip}%
\pgfsetbuttcap%
\pgfsetroundjoin%
\pgfsetlinewidth{0.309473pt}%
\definecolor{currentstroke}{rgb}{0.268510,0.009605,0.335427}%
\pgfsetstrokecolor{currentstroke}%
\pgfsetdash{}{0pt}%
\pgfpathmoveto{\pgfqpoint{5.830324in}{3.199484in}}%
\pgfpathlineto{\pgfqpoint{5.830324in}{3.199484in}}%
\pgfusepath{stroke}%
\end{pgfscope}%
\begin{pgfscope}%
\pgfpathrectangle{\pgfqpoint{3.985294in}{1.750000in}}{\pgfqpoint{2.279412in}{2.004545in}}%
\pgfusepath{clip}%
\pgfsetbuttcap%
\pgfsetroundjoin%
\pgfsetlinewidth{0.309473pt}%
\definecolor{currentstroke}{rgb}{0.268510,0.009605,0.335427}%
\pgfsetstrokecolor{currentstroke}%
\pgfsetdash{}{0pt}%
\pgfpathmoveto{\pgfqpoint{5.830324in}{3.199484in}}%
\pgfpathlineto{\pgfqpoint{5.811095in}{3.198203in}}%
\pgfusepath{stroke}%
\end{pgfscope}%
\begin{pgfscope}%
\pgfpathrectangle{\pgfqpoint{3.985294in}{1.750000in}}{\pgfqpoint{2.279412in}{2.004545in}}%
\pgfusepath{clip}%
\pgfsetbuttcap%
\pgfsetroundjoin%
\pgfsetlinewidth{0.310875pt}%
\definecolor{currentstroke}{rgb}{0.268510,0.009605,0.335427}%
\pgfsetstrokecolor{currentstroke}%
\pgfsetdash{}{0pt}%
\pgfpathmoveto{\pgfqpoint{5.811095in}{3.198203in}}%
\pgfpathlineto{\pgfqpoint{5.773060in}{3.198553in}}%
\pgfusepath{stroke}%
\end{pgfscope}%
\begin{pgfscope}%
\pgfpathrectangle{\pgfqpoint{3.985294in}{1.750000in}}{\pgfqpoint{2.279412in}{2.004545in}}%
\pgfusepath{clip}%
\pgfsetbuttcap%
\pgfsetroundjoin%
\pgfsetlinewidth{0.315560pt}%
\definecolor{currentstroke}{rgb}{0.269944,0.014625,0.341379}%
\pgfsetstrokecolor{currentstroke}%
\pgfsetdash{}{0pt}%
\pgfpathmoveto{\pgfqpoint{5.773060in}{3.198553in}}%
\pgfpathlineto{\pgfqpoint{5.731039in}{3.198329in}}%
\pgfusepath{stroke}%
\end{pgfscope}%
\begin{pgfscope}%
\pgfpathrectangle{\pgfqpoint{3.985294in}{1.750000in}}{\pgfqpoint{2.279412in}{2.004545in}}%
\pgfusepath{clip}%
\pgfsetbuttcap%
\pgfsetroundjoin%
\pgfsetlinewidth{0.318445pt}%
\definecolor{currentstroke}{rgb}{0.269944,0.014625,0.341379}%
\pgfsetstrokecolor{currentstroke}%
\pgfsetdash{}{0pt}%
\pgfpathmoveto{\pgfqpoint{5.731039in}{3.198329in}}%
\pgfpathlineto{\pgfqpoint{5.680931in}{3.196977in}}%
\pgfusepath{stroke}%
\end{pgfscope}%
\begin{pgfscope}%
\pgfpathrectangle{\pgfqpoint{3.985294in}{1.750000in}}{\pgfqpoint{2.279412in}{2.004545in}}%
\pgfusepath{clip}%
\pgfsetbuttcap%
\pgfsetroundjoin%
\pgfsetlinewidth{0.326132pt}%
\definecolor{currentstroke}{rgb}{0.271305,0.019942,0.347269}%
\pgfsetstrokecolor{currentstroke}%
\pgfsetdash{}{0pt}%
\pgfpathmoveto{\pgfqpoint{5.680931in}{3.196977in}}%
\pgfpathlineto{\pgfqpoint{5.630828in}{3.195400in}}%
\pgfusepath{stroke}%
\end{pgfscope}%
\begin{pgfscope}%
\pgfpathrectangle{\pgfqpoint{3.985294in}{1.750000in}}{\pgfqpoint{2.279412in}{2.004545in}}%
\pgfusepath{clip}%
\pgfsetbuttcap%
\pgfsetroundjoin%
\pgfsetlinewidth{0.345276pt}%
\definecolor{currentstroke}{rgb}{0.274952,0.037752,0.364543}%
\pgfsetstrokecolor{currentstroke}%
\pgfsetdash{}{0pt}%
\pgfpathmoveto{\pgfqpoint{5.630828in}{3.195400in}}%
\pgfpathlineto{\pgfqpoint{5.580748in}{3.193216in}}%
\pgfusepath{stroke}%
\end{pgfscope}%
\begin{pgfscope}%
\pgfpathrectangle{\pgfqpoint{3.985294in}{1.750000in}}{\pgfqpoint{2.279412in}{2.004545in}}%
\pgfusepath{clip}%
\pgfsetbuttcap%
\pgfsetroundjoin%
\pgfsetlinewidth{0.361099pt}%
\definecolor{currentstroke}{rgb}{0.277018,0.050344,0.375715}%
\pgfsetstrokecolor{currentstroke}%
\pgfsetdash{}{0pt}%
\pgfpathmoveto{\pgfqpoint{5.580748in}{3.193216in}}%
\pgfpathlineto{\pgfqpoint{5.530685in}{3.190671in}}%
\pgfusepath{stroke}%
\end{pgfscope}%
\begin{pgfscope}%
\pgfpathrectangle{\pgfqpoint{3.985294in}{1.750000in}}{\pgfqpoint{2.279412in}{2.004545in}}%
\pgfusepath{clip}%
\pgfsetbuttcap%
\pgfsetroundjoin%
\pgfsetlinewidth{0.357739pt}%
\definecolor{currentstroke}{rgb}{0.277018,0.050344,0.375715}%
\pgfsetstrokecolor{currentstroke}%
\pgfsetdash{}{0pt}%
\pgfpathmoveto{\pgfqpoint{5.530685in}{3.190671in}}%
\pgfpathlineto{\pgfqpoint{5.480617in}{3.188137in}}%
\pgfusepath{stroke}%
\end{pgfscope}%
\begin{pgfscope}%
\pgfpathrectangle{\pgfqpoint{3.985294in}{1.750000in}}{\pgfqpoint{2.279412in}{2.004545in}}%
\pgfusepath{clip}%
\pgfsetbuttcap%
\pgfsetroundjoin%
\pgfsetlinewidth{0.377746pt}%
\definecolor{currentstroke}{rgb}{0.279566,0.067836,0.391917}%
\pgfsetstrokecolor{currentstroke}%
\pgfsetdash{}{0pt}%
\pgfpathmoveto{\pgfqpoint{5.480617in}{3.188137in}}%
\pgfpathlineto{\pgfqpoint{5.430621in}{3.184830in}}%
\pgfusepath{stroke}%
\end{pgfscope}%
\begin{pgfscope}%
\pgfpathrectangle{\pgfqpoint{3.985294in}{1.750000in}}{\pgfqpoint{2.279412in}{2.004545in}}%
\pgfusepath{clip}%
\pgfsetbuttcap%
\pgfsetroundjoin%
\pgfsetlinewidth{0.406160pt}%
\definecolor{currentstroke}{rgb}{0.281924,0.089666,0.412415}%
\pgfsetstrokecolor{currentstroke}%
\pgfsetdash{}{0pt}%
\pgfpathmoveto{\pgfqpoint{5.430621in}{3.184830in}}%
\pgfpathlineto{\pgfqpoint{5.380779in}{3.180010in}}%
\pgfusepath{stroke}%
\end{pgfscope}%
\begin{pgfscope}%
\pgfpathrectangle{\pgfqpoint{3.985294in}{1.750000in}}{\pgfqpoint{2.279412in}{2.004545in}}%
\pgfusepath{clip}%
\pgfsetbuttcap%
\pgfsetroundjoin%
\pgfsetlinewidth{0.396644pt}%
\definecolor{currentstroke}{rgb}{0.280894,0.078907,0.402329}%
\pgfsetstrokecolor{currentstroke}%
\pgfsetdash{}{0pt}%
\pgfpathmoveto{\pgfqpoint{5.380779in}{3.180010in}}%
\pgfpathlineto{\pgfqpoint{5.331160in}{3.173700in}}%
\pgfusepath{stroke}%
\end{pgfscope}%
\begin{pgfscope}%
\pgfpathrectangle{\pgfqpoint{3.985294in}{1.750000in}}{\pgfqpoint{2.279412in}{2.004545in}}%
\pgfusepath{clip}%
\pgfsetbuttcap%
\pgfsetroundjoin%
\pgfsetlinewidth{0.387251pt}%
\definecolor{currentstroke}{rgb}{0.280267,0.073417,0.397163}%
\pgfsetstrokecolor{currentstroke}%
\pgfsetdash{}{0pt}%
\pgfpathmoveto{\pgfqpoint{5.331160in}{3.173700in}}%
\pgfpathlineto{\pgfqpoint{5.281796in}{3.165978in}}%
\pgfusepath{stroke}%
\end{pgfscope}%
\begin{pgfscope}%
\pgfpathrectangle{\pgfqpoint{3.985294in}{1.750000in}}{\pgfqpoint{2.279412in}{2.004545in}}%
\pgfusepath{clip}%
\pgfsetbuttcap%
\pgfsetroundjoin%
\pgfsetlinewidth{0.416125pt}%
\definecolor{currentstroke}{rgb}{0.282327,0.094955,0.417331}%
\pgfsetstrokecolor{currentstroke}%
\pgfsetdash{}{0pt}%
\pgfpathmoveto{\pgfqpoint{5.281796in}{3.165978in}}%
\pgfpathlineto{\pgfqpoint{5.232766in}{3.156861in}}%
\pgfusepath{stroke}%
\end{pgfscope}%
\begin{pgfscope}%
\pgfpathrectangle{\pgfqpoint{3.985294in}{1.750000in}}{\pgfqpoint{2.279412in}{2.004545in}}%
\pgfusepath{clip}%
\pgfsetbuttcap%
\pgfsetroundjoin%
\pgfsetlinewidth{0.394633pt}%
\definecolor{currentstroke}{rgb}{0.280894,0.078907,0.402329}%
\pgfsetstrokecolor{currentstroke}%
\pgfsetdash{}{0pt}%
\pgfpathmoveto{\pgfqpoint{5.232766in}{3.156861in}}%
\pgfpathlineto{\pgfqpoint{5.184457in}{3.145253in}}%
\pgfusepath{stroke}%
\end{pgfscope}%
\begin{pgfscope}%
\pgfpathrectangle{\pgfqpoint{3.985294in}{1.750000in}}{\pgfqpoint{2.279412in}{2.004545in}}%
\pgfusepath{clip}%
\pgfsetbuttcap%
\pgfsetroundjoin%
\pgfsetlinewidth{0.399984pt}%
\definecolor{currentstroke}{rgb}{0.281446,0.084320,0.407414}%
\pgfsetstrokecolor{currentstroke}%
\pgfsetdash{}{0pt}%
\pgfpathmoveto{\pgfqpoint{5.184457in}{3.145253in}}%
\pgfpathlineto{\pgfqpoint{5.137574in}{3.129812in}}%
\pgfusepath{stroke}%
\end{pgfscope}%
\begin{pgfscope}%
\pgfpathrectangle{\pgfqpoint{3.985294in}{1.750000in}}{\pgfqpoint{2.279412in}{2.004545in}}%
\pgfusepath{clip}%
\pgfsetbuttcap%
\pgfsetroundjoin%
\pgfsetlinewidth{0.459515pt}%
\definecolor{currentstroke}{rgb}{0.283072,0.130895,0.449241}%
\pgfsetstrokecolor{currentstroke}%
\pgfsetdash{}{0pt}%
\pgfpathmoveto{\pgfqpoint{5.137574in}{3.129812in}}%
\pgfpathlineto{\pgfqpoint{5.092923in}{3.110099in}}%
\pgfusepath{stroke}%
\end{pgfscope}%
\begin{pgfscope}%
\pgfpathrectangle{\pgfqpoint{3.985294in}{1.750000in}}{\pgfqpoint{2.279412in}{2.004545in}}%
\pgfusepath{clip}%
\pgfsetbuttcap%
\pgfsetroundjoin%
\pgfsetlinewidth{0.458497pt}%
\definecolor{currentstroke}{rgb}{0.283187,0.125848,0.444960}%
\pgfsetstrokecolor{currentstroke}%
\pgfsetdash{}{0pt}%
\pgfpathmoveto{\pgfqpoint{5.092923in}{3.110099in}}%
\pgfpathlineto{\pgfqpoint{5.051892in}{3.085358in}}%
\pgfusepath{stroke}%
\end{pgfscope}%
\begin{pgfscope}%
\pgfpathrectangle{\pgfqpoint{3.985294in}{1.750000in}}{\pgfqpoint{2.279412in}{2.004545in}}%
\pgfusepath{clip}%
\pgfsetbuttcap%
\pgfsetroundjoin%
\pgfsetlinewidth{0.455475pt}%
\definecolor{currentstroke}{rgb}{0.283187,0.125848,0.444960}%
\pgfsetstrokecolor{currentstroke}%
\pgfsetdash{}{0pt}%
\pgfpathmoveto{\pgfqpoint{5.051892in}{3.085358in}}%
\pgfpathlineto{\pgfqpoint{5.014172in}{3.057967in}}%
\pgfusepath{stroke}%
\end{pgfscope}%
\begin{pgfscope}%
\pgfpathrectangle{\pgfqpoint{3.985294in}{1.750000in}}{\pgfqpoint{2.279412in}{2.004545in}}%
\pgfusepath{clip}%
\pgfsetbuttcap%
\pgfsetroundjoin%
\pgfsetlinewidth{0.533510pt}%
\definecolor{currentstroke}{rgb}{0.278012,0.180367,0.486697}%
\pgfsetstrokecolor{currentstroke}%
\pgfsetdash{}{0pt}%
\pgfpathmoveto{\pgfqpoint{5.014172in}{3.057967in}}%
\pgfpathlineto{\pgfqpoint{4.978555in}{3.027336in}}%
\pgfusepath{stroke}%
\end{pgfscope}%
\begin{pgfscope}%
\pgfpathrectangle{\pgfqpoint{3.985294in}{1.750000in}}{\pgfqpoint{2.279412in}{2.004545in}}%
\pgfusepath{clip}%
\pgfsetbuttcap%
\pgfsetroundjoin%
\pgfsetlinewidth{0.550351pt}%
\definecolor{currentstroke}{rgb}{0.275191,0.194905,0.496005}%
\pgfsetstrokecolor{currentstroke}%
\pgfsetdash{}{0pt}%
\pgfpathmoveto{\pgfqpoint{4.978555in}{3.027336in}}%
\pgfpathlineto{\pgfqpoint{4.945961in}{2.994082in}}%
\pgfusepath{stroke}%
\end{pgfscope}%
\begin{pgfscope}%
\pgfpathrectangle{\pgfqpoint{3.985294in}{1.750000in}}{\pgfqpoint{2.279412in}{2.004545in}}%
\pgfusepath{clip}%
\pgfsetbuttcap%
\pgfsetroundjoin%
\pgfsetlinewidth{0.575940pt}%
\definecolor{currentstroke}{rgb}{0.270595,0.214069,0.507052}%
\pgfsetstrokecolor{currentstroke}%
\pgfsetdash{}{0pt}%
\pgfpathmoveto{\pgfqpoint{4.945961in}{2.994082in}}%
\pgfpathlineto{\pgfqpoint{4.912919in}{2.961216in}}%
\pgfusepath{stroke}%
\end{pgfscope}%
\begin{pgfscope}%
\pgfpathrectangle{\pgfqpoint{3.985294in}{1.750000in}}{\pgfqpoint{2.279412in}{2.004545in}}%
\pgfusepath{clip}%
\pgfsetbuttcap%
\pgfsetroundjoin%
\pgfsetlinewidth{0.639095pt}%
\definecolor{currentstroke}{rgb}{0.257322,0.256130,0.526563}%
\pgfsetstrokecolor{currentstroke}%
\pgfsetdash{}{0pt}%
\pgfpathmoveto{\pgfqpoint{4.912919in}{2.961216in}}%
\pgfpathlineto{\pgfqpoint{4.878968in}{2.929124in}}%
\pgfusepath{stroke}%
\end{pgfscope}%
\begin{pgfscope}%
\pgfpathrectangle{\pgfqpoint{3.985294in}{1.750000in}}{\pgfqpoint{2.279412in}{2.004545in}}%
\pgfusepath{clip}%
\pgfsetbuttcap%
\pgfsetroundjoin%
\pgfsetlinewidth{0.324853pt}%
\definecolor{currentstroke}{rgb}{0.271305,0.019942,0.347269}%
\pgfsetstrokecolor{currentstroke}%
\pgfsetdash{}{0pt}%
\pgfpathmoveto{\pgfqpoint{5.843087in}{3.293554in}}%
\pgfpathlineto{\pgfqpoint{5.793295in}{3.293758in}}%
\pgfusepath{stroke}%
\end{pgfscope}%
\begin{pgfscope}%
\pgfpathrectangle{\pgfqpoint{3.985294in}{1.750000in}}{\pgfqpoint{2.279412in}{2.004545in}}%
\pgfusepath{clip}%
\pgfsetbuttcap%
\pgfsetroundjoin%
\pgfsetlinewidth{0.319188pt}%
\definecolor{currentstroke}{rgb}{0.269944,0.014625,0.341379}%
\pgfsetstrokecolor{currentstroke}%
\pgfsetdash{}{0pt}%
\pgfpathmoveto{\pgfqpoint{5.793295in}{3.293758in}}%
\pgfpathlineto{\pgfqpoint{5.743779in}{3.294449in}}%
\pgfusepath{stroke}%
\end{pgfscope}%
\begin{pgfscope}%
\pgfpathrectangle{\pgfqpoint{3.985294in}{1.750000in}}{\pgfqpoint{2.279412in}{2.004545in}}%
\pgfusepath{clip}%
\pgfsetbuttcap%
\pgfsetroundjoin%
\pgfsetlinewidth{0.318684pt}%
\definecolor{currentstroke}{rgb}{0.269944,0.014625,0.341379}%
\pgfsetstrokecolor{currentstroke}%
\pgfsetdash{}{0pt}%
\pgfpathmoveto{\pgfqpoint{5.743779in}{3.294449in}}%
\pgfpathlineto{\pgfqpoint{5.693982in}{3.294061in}}%
\pgfusepath{stroke}%
\end{pgfscope}%
\begin{pgfscope}%
\pgfpathrectangle{\pgfqpoint{3.985294in}{1.750000in}}{\pgfqpoint{2.279412in}{2.004545in}}%
\pgfusepath{clip}%
\pgfsetbuttcap%
\pgfsetroundjoin%
\pgfsetlinewidth{0.325898pt}%
\definecolor{currentstroke}{rgb}{0.271305,0.019942,0.347269}%
\pgfsetstrokecolor{currentstroke}%
\pgfsetdash{}{0pt}%
\pgfpathmoveto{\pgfqpoint{5.693982in}{3.294061in}}%
\pgfpathlineto{\pgfqpoint{5.643866in}{3.292589in}}%
\pgfusepath{stroke}%
\end{pgfscope}%
\begin{pgfscope}%
\pgfpathrectangle{\pgfqpoint{3.985294in}{1.750000in}}{\pgfqpoint{2.279412in}{2.004545in}}%
\pgfusepath{clip}%
\pgfsetbuttcap%
\pgfsetroundjoin%
\pgfsetlinewidth{0.334003pt}%
\definecolor{currentstroke}{rgb}{0.272594,0.025563,0.353093}%
\pgfsetstrokecolor{currentstroke}%
\pgfsetdash{}{0pt}%
\pgfpathmoveto{\pgfqpoint{5.643866in}{3.292589in}}%
\pgfpathlineto{\pgfqpoint{5.593736in}{3.291486in}}%
\pgfusepath{stroke}%
\end{pgfscope}%
\begin{pgfscope}%
\pgfpathrectangle{\pgfqpoint{3.985294in}{1.750000in}}{\pgfqpoint{2.279412in}{2.004545in}}%
\pgfusepath{clip}%
\pgfsetbuttcap%
\pgfsetroundjoin%
\pgfsetlinewidth{0.344863pt}%
\definecolor{currentstroke}{rgb}{0.274952,0.037752,0.364543}%
\pgfsetstrokecolor{currentstroke}%
\pgfsetdash{}{0pt}%
\pgfpathmoveto{\pgfqpoint{5.593736in}{3.291486in}}%
\pgfpathlineto{\pgfqpoint{5.543642in}{3.289555in}}%
\pgfusepath{stroke}%
\end{pgfscope}%
\begin{pgfscope}%
\pgfpathrectangle{\pgfqpoint{3.985294in}{1.750000in}}{\pgfqpoint{2.279412in}{2.004545in}}%
\pgfusepath{clip}%
\pgfsetbuttcap%
\pgfsetroundjoin%
\pgfsetlinewidth{0.335178pt}%
\definecolor{currentstroke}{rgb}{0.272594,0.025563,0.353093}%
\pgfsetstrokecolor{currentstroke}%
\pgfsetdash{}{0pt}%
\pgfpathmoveto{\pgfqpoint{5.432751in}{3.383768in}}%
\pgfpathlineto{\pgfqpoint{5.383582in}{3.375413in}}%
\pgfusepath{stroke}%
\end{pgfscope}%
\begin{pgfscope}%
\pgfpathrectangle{\pgfqpoint{3.985294in}{1.750000in}}{\pgfqpoint{2.279412in}{2.004545in}}%
\pgfusepath{clip}%
\pgfsetbuttcap%
\pgfsetroundjoin%
\pgfsetlinewidth{0.332966pt}%
\definecolor{currentstroke}{rgb}{0.272594,0.025563,0.353093}%
\pgfsetstrokecolor{currentstroke}%
\pgfsetdash{}{0pt}%
\pgfpathmoveto{\pgfqpoint{5.383582in}{3.375413in}}%
\pgfpathlineto{\pgfqpoint{5.334787in}{3.365383in}}%
\pgfusepath{stroke}%
\end{pgfscope}%
\begin{pgfscope}%
\pgfpathrectangle{\pgfqpoint{3.985294in}{1.750000in}}{\pgfqpoint{2.279412in}{2.004545in}}%
\pgfusepath{clip}%
\pgfsetbuttcap%
\pgfsetroundjoin%
\pgfsetlinewidth{0.342813pt}%
\definecolor{currentstroke}{rgb}{0.274952,0.037752,0.364543}%
\pgfsetstrokecolor{currentstroke}%
\pgfsetdash{}{0pt}%
\pgfpathmoveto{\pgfqpoint{5.334787in}{3.365383in}}%
\pgfpathlineto{\pgfqpoint{5.286092in}{3.354954in}}%
\pgfusepath{stroke}%
\end{pgfscope}%
\begin{pgfscope}%
\pgfpathrectangle{\pgfqpoint{3.985294in}{1.750000in}}{\pgfqpoint{2.279412in}{2.004545in}}%
\pgfusepath{clip}%
\pgfsetbuttcap%
\pgfsetroundjoin%
\pgfsetlinewidth{0.338855pt}%
\definecolor{currentstroke}{rgb}{0.273809,0.031497,0.358853}%
\pgfsetstrokecolor{currentstroke}%
\pgfsetdash{}{0pt}%
\pgfpathmoveto{\pgfqpoint{5.286092in}{3.354954in}}%
\pgfpathlineto{\pgfqpoint{5.237418in}{3.344479in}}%
\pgfusepath{stroke}%
\end{pgfscope}%
\begin{pgfscope}%
\pgfpathrectangle{\pgfqpoint{3.985294in}{1.750000in}}{\pgfqpoint{2.279412in}{2.004545in}}%
\pgfusepath{clip}%
\pgfsetbuttcap%
\pgfsetroundjoin%
\pgfsetlinewidth{0.337151pt}%
\definecolor{currentstroke}{rgb}{0.273809,0.031497,0.358853}%
\pgfsetstrokecolor{currentstroke}%
\pgfsetdash{}{0pt}%
\pgfpathmoveto{\pgfqpoint{5.237418in}{3.344479in}}%
\pgfpathlineto{\pgfqpoint{5.237418in}{3.344479in}}%
\pgfusepath{stroke}%
\end{pgfscope}%
\begin{pgfscope}%
\pgfpathrectangle{\pgfqpoint{3.985294in}{1.750000in}}{\pgfqpoint{2.279412in}{2.004545in}}%
\pgfusepath{clip}%
\pgfsetbuttcap%
\pgfsetroundjoin%
\pgfsetlinewidth{0.337151pt}%
\definecolor{currentstroke}{rgb}{0.273809,0.031497,0.358853}%
\pgfsetstrokecolor{currentstroke}%
\pgfsetdash{}{0pt}%
\pgfpathmoveto{\pgfqpoint{5.237418in}{3.344479in}}%
\pgfpathlineto{\pgfqpoint{5.198586in}{3.331051in}}%
\pgfusepath{stroke}%
\end{pgfscope}%
\begin{pgfscope}%
\pgfpathrectangle{\pgfqpoint{3.985294in}{1.750000in}}{\pgfqpoint{2.279412in}{2.004545in}}%
\pgfusepath{clip}%
\pgfsetbuttcap%
\pgfsetroundjoin%
\pgfsetlinewidth{0.339159pt}%
\definecolor{currentstroke}{rgb}{0.273809,0.031497,0.358853}%
\pgfsetstrokecolor{currentstroke}%
\pgfsetdash{}{0pt}%
\pgfpathmoveto{\pgfqpoint{5.198586in}{3.331051in}}%
\pgfpathlineto{\pgfqpoint{5.162225in}{3.318050in}}%
\pgfusepath{stroke}%
\end{pgfscope}%
\begin{pgfscope}%
\pgfpathrectangle{\pgfqpoint{3.985294in}{1.750000in}}{\pgfqpoint{2.279412in}{2.004545in}}%
\pgfusepath{clip}%
\pgfsetbuttcap%
\pgfsetroundjoin%
\pgfsetlinewidth{0.353578pt}%
\definecolor{currentstroke}{rgb}{0.276022,0.044167,0.370164}%
\pgfsetstrokecolor{currentstroke}%
\pgfsetdash{}{0pt}%
\pgfpathmoveto{\pgfqpoint{5.162225in}{3.318050in}}%
\pgfpathlineto{\pgfqpoint{5.162225in}{3.318050in}}%
\pgfusepath{stroke}%
\end{pgfscope}%
\begin{pgfscope}%
\pgfpathrectangle{\pgfqpoint{3.985294in}{1.750000in}}{\pgfqpoint{2.279412in}{2.004545in}}%
\pgfusepath{clip}%
\pgfsetbuttcap%
\pgfsetroundjoin%
\pgfsetlinewidth{0.353578pt}%
\definecolor{currentstroke}{rgb}{0.276022,0.044167,0.370164}%
\pgfsetstrokecolor{currentstroke}%
\pgfsetdash{}{0pt}%
\pgfpathmoveto{\pgfqpoint{5.162225in}{3.318050in}}%
\pgfpathlineto{\pgfqpoint{5.133090in}{3.304178in}}%
\pgfusepath{stroke}%
\end{pgfscope}%
\begin{pgfscope}%
\pgfpathrectangle{\pgfqpoint{3.985294in}{1.750000in}}{\pgfqpoint{2.279412in}{2.004545in}}%
\pgfusepath{clip}%
\pgfsetbuttcap%
\pgfsetroundjoin%
\pgfsetlinewidth{0.343928pt}%
\definecolor{currentstroke}{rgb}{0.274952,0.037752,0.364543}%
\pgfsetstrokecolor{currentstroke}%
\pgfsetdash{}{0pt}%
\pgfpathmoveto{\pgfqpoint{5.133090in}{3.304178in}}%
\pgfpathlineto{\pgfqpoint{5.111502in}{3.286419in}}%
\pgfusepath{stroke}%
\end{pgfscope}%
\begin{pgfscope}%
\pgfpathrectangle{\pgfqpoint{3.985294in}{1.750000in}}{\pgfqpoint{2.279412in}{2.004545in}}%
\pgfusepath{clip}%
\pgfsetbuttcap%
\pgfsetroundjoin%
\pgfsetlinewidth{0.350505pt}%
\definecolor{currentstroke}{rgb}{0.276022,0.044167,0.370164}%
\pgfsetstrokecolor{currentstroke}%
\pgfsetdash{}{0pt}%
\pgfpathmoveto{\pgfqpoint{5.111502in}{3.286419in}}%
\pgfpathlineto{\pgfqpoint{5.093469in}{3.267547in}}%
\pgfusepath{stroke}%
\end{pgfscope}%
\begin{pgfscope}%
\pgfpathrectangle{\pgfqpoint{3.985294in}{1.750000in}}{\pgfqpoint{2.279412in}{2.004545in}}%
\pgfusepath{clip}%
\pgfsetbuttcap%
\pgfsetroundjoin%
\pgfsetlinewidth{0.343954pt}%
\definecolor{currentstroke}{rgb}{0.274952,0.037752,0.364543}%
\pgfsetstrokecolor{currentstroke}%
\pgfsetdash{}{0pt}%
\pgfpathmoveto{\pgfqpoint{5.073708in}{2.165885in}}%
\pgfpathlineto{\pgfqpoint{5.054097in}{2.202004in}}%
\pgfusepath{stroke}%
\end{pgfscope}%
\begin{pgfscope}%
\pgfpathrectangle{\pgfqpoint{3.985294in}{1.750000in}}{\pgfqpoint{2.279412in}{2.004545in}}%
\pgfusepath{clip}%
\pgfsetbuttcap%
\pgfsetroundjoin%
\pgfsetlinewidth{0.351156pt}%
\definecolor{currentstroke}{rgb}{0.276022,0.044167,0.370164}%
\pgfsetstrokecolor{currentstroke}%
\pgfsetdash{}{0pt}%
\pgfpathmoveto{\pgfqpoint{5.054097in}{2.202004in}}%
\pgfpathlineto{\pgfqpoint{5.054097in}{2.202004in}}%
\pgfusepath{stroke}%
\end{pgfscope}%
\begin{pgfscope}%
\pgfpathrectangle{\pgfqpoint{3.985294in}{1.750000in}}{\pgfqpoint{2.279412in}{2.004545in}}%
\pgfusepath{clip}%
\pgfsetbuttcap%
\pgfsetroundjoin%
\pgfsetlinewidth{0.351156pt}%
\definecolor{currentstroke}{rgb}{0.276022,0.044167,0.370164}%
\pgfsetstrokecolor{currentstroke}%
\pgfsetdash{}{0pt}%
\pgfpathmoveto{\pgfqpoint{5.054097in}{2.202004in}}%
\pgfpathlineto{\pgfqpoint{5.039648in}{2.233470in}}%
\pgfusepath{stroke}%
\end{pgfscope}%
\begin{pgfscope}%
\pgfpathrectangle{\pgfqpoint{3.985294in}{1.750000in}}{\pgfqpoint{2.279412in}{2.004545in}}%
\pgfusepath{clip}%
\pgfsetbuttcap%
\pgfsetroundjoin%
\pgfsetlinewidth{0.379428pt}%
\definecolor{currentstroke}{rgb}{0.279566,0.067836,0.391917}%
\pgfsetstrokecolor{currentstroke}%
\pgfsetdash{}{0pt}%
\pgfpathmoveto{\pgfqpoint{5.039648in}{2.233470in}}%
\pgfpathlineto{\pgfqpoint{5.030303in}{2.266779in}}%
\pgfusepath{stroke}%
\end{pgfscope}%
\begin{pgfscope}%
\pgfpathrectangle{\pgfqpoint{3.985294in}{1.750000in}}{\pgfqpoint{2.279412in}{2.004545in}}%
\pgfusepath{clip}%
\pgfsetbuttcap%
\pgfsetroundjoin%
\pgfsetlinewidth{0.378993pt}%
\definecolor{currentstroke}{rgb}{0.279566,0.067836,0.391917}%
\pgfsetstrokecolor{currentstroke}%
\pgfsetdash{}{0pt}%
\pgfpathmoveto{\pgfqpoint{5.030303in}{2.266779in}}%
\pgfpathlineto{\pgfqpoint{5.030303in}{2.266779in}}%
\pgfusepath{stroke}%
\end{pgfscope}%
\begin{pgfscope}%
\pgfpathrectangle{\pgfqpoint{3.985294in}{1.750000in}}{\pgfqpoint{2.279412in}{2.004545in}}%
\pgfusepath{clip}%
\pgfsetbuttcap%
\pgfsetroundjoin%
\pgfsetlinewidth{0.378993pt}%
\definecolor{currentstroke}{rgb}{0.279566,0.067836,0.391917}%
\pgfsetstrokecolor{currentstroke}%
\pgfsetdash{}{0pt}%
\pgfpathmoveto{\pgfqpoint{5.030303in}{2.266779in}}%
\pgfpathlineto{\pgfqpoint{5.018946in}{2.293498in}}%
\pgfusepath{stroke}%
\end{pgfscope}%
\begin{pgfscope}%
\pgfpathrectangle{\pgfqpoint{3.985294in}{1.750000in}}{\pgfqpoint{2.279412in}{2.004545in}}%
\pgfusepath{clip}%
\pgfsetbuttcap%
\pgfsetroundjoin%
\pgfsetlinewidth{0.381900pt}%
\definecolor{currentstroke}{rgb}{0.279566,0.067836,0.391917}%
\pgfsetstrokecolor{currentstroke}%
\pgfsetdash{}{0pt}%
\pgfpathmoveto{\pgfqpoint{5.018946in}{2.293498in}}%
\pgfpathlineto{\pgfqpoint{5.006401in}{2.318016in}}%
\pgfusepath{stroke}%
\end{pgfscope}%
\begin{pgfscope}%
\pgfpathrectangle{\pgfqpoint{3.985294in}{1.750000in}}{\pgfqpoint{2.279412in}{2.004545in}}%
\pgfusepath{clip}%
\pgfsetbuttcap%
\pgfsetroundjoin%
\pgfsetlinewidth{0.352447pt}%
\definecolor{currentstroke}{rgb}{0.276022,0.044167,0.370164}%
\pgfsetstrokecolor{currentstroke}%
\pgfsetdash{}{0pt}%
\pgfpathmoveto{\pgfqpoint{5.006401in}{2.318016in}}%
\pgfpathlineto{\pgfqpoint{4.985837in}{2.357411in}}%
\pgfusepath{stroke}%
\end{pgfscope}%
\begin{pgfscope}%
\pgfpathrectangle{\pgfqpoint{3.985294in}{1.750000in}}{\pgfqpoint{2.279412in}{2.004545in}}%
\pgfusepath{clip}%
\pgfsetbuttcap%
\pgfsetroundjoin%
\pgfsetlinewidth{0.433526pt}%
\definecolor{currentstroke}{rgb}{0.283091,0.110553,0.431554}%
\pgfsetstrokecolor{currentstroke}%
\pgfsetdash{}{0pt}%
\pgfpathmoveto{\pgfqpoint{4.985837in}{2.357411in}}%
\pgfpathlineto{\pgfqpoint{4.960382in}{2.394914in}}%
\pgfusepath{stroke}%
\end{pgfscope}%
\begin{pgfscope}%
\pgfpathrectangle{\pgfqpoint{3.985294in}{1.750000in}}{\pgfqpoint{2.279412in}{2.004545in}}%
\pgfusepath{clip}%
\pgfsetbuttcap%
\pgfsetroundjoin%
\pgfsetlinewidth{0.549425pt}%
\definecolor{currentstroke}{rgb}{0.275191,0.194905,0.496005}%
\pgfsetstrokecolor{currentstroke}%
\pgfsetdash{}{0pt}%
\pgfpathmoveto{\pgfqpoint{4.960382in}{2.394914in}}%
\pgfpathlineto{\pgfqpoint{4.937827in}{2.428264in}}%
\pgfusepath{stroke}%
\end{pgfscope}%
\begin{pgfscope}%
\pgfpathrectangle{\pgfqpoint{3.985294in}{1.750000in}}{\pgfqpoint{2.279412in}{2.004545in}}%
\pgfusepath{clip}%
\pgfsetbuttcap%
\pgfsetroundjoin%
\pgfsetlinewidth{0.495031pt}%
\definecolor{currentstroke}{rgb}{0.281412,0.155834,0.469201}%
\pgfsetstrokecolor{currentstroke}%
\pgfsetdash{}{0pt}%
\pgfpathmoveto{\pgfqpoint{4.937827in}{2.428264in}}%
\pgfpathlineto{\pgfqpoint{4.937827in}{2.428264in}}%
\pgfusepath{stroke}%
\end{pgfscope}%
\begin{pgfscope}%
\pgfpathrectangle{\pgfqpoint{3.985294in}{1.750000in}}{\pgfqpoint{2.279412in}{2.004545in}}%
\pgfusepath{clip}%
\pgfsetbuttcap%
\pgfsetroundjoin%
\pgfsetlinewidth{0.495031pt}%
\definecolor{currentstroke}{rgb}{0.281412,0.155834,0.469201}%
\pgfsetstrokecolor{currentstroke}%
\pgfsetdash{}{0pt}%
\pgfpathmoveto{\pgfqpoint{4.937827in}{2.428264in}}%
\pgfpathlineto{\pgfqpoint{4.916933in}{2.457927in}}%
\pgfusepath{stroke}%
\end{pgfscope}%
\begin{pgfscope}%
\pgfpathrectangle{\pgfqpoint{3.985294in}{1.750000in}}{\pgfqpoint{2.279412in}{2.004545in}}%
\pgfusepath{clip}%
\pgfsetbuttcap%
\pgfsetroundjoin%
\pgfsetlinewidth{0.350239pt}%
\definecolor{currentstroke}{rgb}{0.276022,0.044167,0.370164}%
\pgfsetstrokecolor{currentstroke}%
\pgfsetdash{}{0pt}%
\pgfpathmoveto{\pgfqpoint{5.330168in}{2.165885in}}%
\pgfpathlineto{\pgfqpoint{5.281276in}{2.175659in}}%
\pgfusepath{stroke}%
\end{pgfscope}%
\begin{pgfscope}%
\pgfpathrectangle{\pgfqpoint{3.985294in}{1.750000in}}{\pgfqpoint{2.279412in}{2.004545in}}%
\pgfusepath{clip}%
\pgfsetbuttcap%
\pgfsetroundjoin%
\pgfsetlinewidth{0.360264pt}%
\definecolor{currentstroke}{rgb}{0.277018,0.050344,0.375715}%
\pgfsetstrokecolor{currentstroke}%
\pgfsetdash{}{0pt}%
\pgfpathmoveto{\pgfqpoint{5.281276in}{2.175659in}}%
\pgfpathlineto{\pgfqpoint{5.233176in}{2.186844in}}%
\pgfusepath{stroke}%
\end{pgfscope}%
\begin{pgfscope}%
\pgfpathrectangle{\pgfqpoint{3.985294in}{1.750000in}}{\pgfqpoint{2.279412in}{2.004545in}}%
\pgfusepath{clip}%
\pgfsetbuttcap%
\pgfsetroundjoin%
\pgfsetlinewidth{0.355372pt}%
\definecolor{currentstroke}{rgb}{0.276022,0.044167,0.370164}%
\pgfsetstrokecolor{currentstroke}%
\pgfsetdash{}{0pt}%
\pgfpathmoveto{\pgfqpoint{5.233176in}{2.186844in}}%
\pgfpathlineto{\pgfqpoint{5.187049in}{2.202726in}}%
\pgfusepath{stroke}%
\end{pgfscope}%
\begin{pgfscope}%
\pgfpathrectangle{\pgfqpoint{3.985294in}{1.750000in}}{\pgfqpoint{2.279412in}{2.004545in}}%
\pgfusepath{clip}%
\pgfsetbuttcap%
\pgfsetroundjoin%
\pgfsetlinewidth{0.343553pt}%
\definecolor{currentstroke}{rgb}{0.274952,0.037752,0.364543}%
\pgfsetstrokecolor{currentstroke}%
\pgfsetdash{}{0pt}%
\pgfpathmoveto{\pgfqpoint{5.187049in}{2.202726in}}%
\pgfpathlineto{\pgfqpoint{5.143920in}{2.224020in}}%
\pgfusepath{stroke}%
\end{pgfscope}%
\begin{pgfscope}%
\pgfpathrectangle{\pgfqpoint{3.985294in}{1.750000in}}{\pgfqpoint{2.279412in}{2.004545in}}%
\pgfusepath{clip}%
\pgfsetbuttcap%
\pgfsetroundjoin%
\pgfsetlinewidth{0.360667pt}%
\definecolor{currentstroke}{rgb}{0.277018,0.050344,0.375715}%
\pgfsetstrokecolor{currentstroke}%
\pgfsetdash{}{0pt}%
\pgfpathmoveto{\pgfqpoint{5.143920in}{2.224020in}}%
\pgfpathlineto{\pgfqpoint{5.101867in}{2.247452in}}%
\pgfusepath{stroke}%
\end{pgfscope}%
\begin{pgfscope}%
\pgfpathrectangle{\pgfqpoint{3.985294in}{1.750000in}}{\pgfqpoint{2.279412in}{2.004545in}}%
\pgfusepath{clip}%
\pgfsetbuttcap%
\pgfsetroundjoin%
\pgfsetlinewidth{0.359189pt}%
\definecolor{currentstroke}{rgb}{0.277018,0.050344,0.375715}%
\pgfsetstrokecolor{currentstroke}%
\pgfsetdash{}{0pt}%
\pgfpathmoveto{\pgfqpoint{5.101867in}{2.247452in}}%
\pgfpathlineto{\pgfqpoint{5.058263in}{2.267747in}}%
\pgfusepath{stroke}%
\end{pgfscope}%
\begin{pgfscope}%
\pgfpathrectangle{\pgfqpoint{3.985294in}{1.750000in}}{\pgfqpoint{2.279412in}{2.004545in}}%
\pgfusepath{clip}%
\pgfsetbuttcap%
\pgfsetroundjoin%
\pgfsetlinewidth{0.330966pt}%
\definecolor{currentstroke}{rgb}{0.272594,0.025563,0.353093}%
\pgfsetstrokecolor{currentstroke}%
\pgfsetdash{}{0pt}%
\pgfpathmoveto{\pgfqpoint{5.058263in}{2.267747in}}%
\pgfpathlineto{\pgfqpoint{5.058263in}{2.267747in}}%
\pgfusepath{stroke}%
\end{pgfscope}%
\begin{pgfscope}%
\pgfpathrectangle{\pgfqpoint{3.985294in}{1.750000in}}{\pgfqpoint{2.279412in}{2.004545in}}%
\pgfusepath{clip}%
\pgfsetbuttcap%
\pgfsetroundjoin%
\pgfsetlinewidth{0.330966pt}%
\definecolor{currentstroke}{rgb}{0.272594,0.025563,0.353093}%
\pgfsetstrokecolor{currentstroke}%
\pgfsetdash{}{0pt}%
\pgfpathmoveto{\pgfqpoint{5.058263in}{2.267747in}}%
\pgfpathlineto{\pgfqpoint{5.058263in}{2.267747in}}%
\pgfusepath{stroke}%
\end{pgfscope}%
\begin{pgfscope}%
\pgfpathrectangle{\pgfqpoint{3.985294in}{1.750000in}}{\pgfqpoint{2.279412in}{2.004545in}}%
\pgfusepath{clip}%
\pgfsetbuttcap%
\pgfsetroundjoin%
\pgfsetlinewidth{0.310811pt}%
\definecolor{currentstroke}{rgb}{0.268510,0.009605,0.335427}%
\pgfsetstrokecolor{currentstroke}%
\pgfsetdash{}{0pt}%
\pgfpathmoveto{\pgfqpoint{5.863514in}{2.253412in}}%
\pgfpathlineto{\pgfqpoint{5.826864in}{2.254532in}}%
\pgfusepath{stroke}%
\end{pgfscope}%
\begin{pgfscope}%
\pgfpathrectangle{\pgfqpoint{3.985294in}{1.750000in}}{\pgfqpoint{2.279412in}{2.004545in}}%
\pgfusepath{clip}%
\pgfsetbuttcap%
\pgfsetroundjoin%
\pgfsetlinewidth{0.317216pt}%
\definecolor{currentstroke}{rgb}{0.269944,0.014625,0.341379}%
\pgfsetstrokecolor{currentstroke}%
\pgfsetdash{}{0pt}%
\pgfpathmoveto{\pgfqpoint{5.826864in}{2.254532in}}%
\pgfpathlineto{\pgfqpoint{5.791795in}{2.256098in}}%
\pgfusepath{stroke}%
\end{pgfscope}%
\begin{pgfscope}%
\pgfpathrectangle{\pgfqpoint{3.985294in}{1.750000in}}{\pgfqpoint{2.279412in}{2.004545in}}%
\pgfusepath{clip}%
\pgfsetbuttcap%
\pgfsetroundjoin%
\pgfsetlinewidth{0.315873pt}%
\definecolor{currentstroke}{rgb}{0.269944,0.014625,0.341379}%
\pgfsetstrokecolor{currentstroke}%
\pgfsetdash{}{0pt}%
\pgfpathmoveto{\pgfqpoint{5.791795in}{2.256098in}}%
\pgfpathlineto{\pgfqpoint{5.791795in}{2.256098in}}%
\pgfusepath{stroke}%
\end{pgfscope}%
\begin{pgfscope}%
\pgfpathrectangle{\pgfqpoint{3.985294in}{1.750000in}}{\pgfqpoint{2.279412in}{2.004545in}}%
\pgfusepath{clip}%
\pgfsetbuttcap%
\pgfsetroundjoin%
\pgfsetlinewidth{0.315873pt}%
\definecolor{currentstroke}{rgb}{0.269944,0.014625,0.341379}%
\pgfsetstrokecolor{currentstroke}%
\pgfsetdash{}{0pt}%
\pgfpathmoveto{\pgfqpoint{5.791795in}{2.256098in}}%
\pgfpathlineto{\pgfqpoint{5.791795in}{2.256098in}}%
\pgfusepath{stroke}%
\end{pgfscope}%
\begin{pgfscope}%
\pgfpathrectangle{\pgfqpoint{3.985294in}{1.750000in}}{\pgfqpoint{2.279412in}{2.004545in}}%
\pgfusepath{clip}%
\pgfsetbuttcap%
\pgfsetroundjoin%
\pgfsetlinewidth{0.315873pt}%
\definecolor{currentstroke}{rgb}{0.269944,0.014625,0.341379}%
\pgfsetstrokecolor{currentstroke}%
\pgfsetdash{}{0pt}%
\pgfpathmoveto{\pgfqpoint{5.791795in}{2.256098in}}%
\pgfpathlineto{\pgfqpoint{5.758706in}{2.256774in}}%
\pgfusepath{stroke}%
\end{pgfscope}%
\begin{pgfscope}%
\pgfpathrectangle{\pgfqpoint{3.985294in}{1.750000in}}{\pgfqpoint{2.279412in}{2.004545in}}%
\pgfusepath{clip}%
\pgfsetbuttcap%
\pgfsetroundjoin%
\pgfsetlinewidth{0.327232pt}%
\definecolor{currentstroke}{rgb}{0.271305,0.019942,0.347269}%
\pgfsetstrokecolor{currentstroke}%
\pgfsetdash{}{0pt}%
\pgfpathmoveto{\pgfqpoint{5.758706in}{2.256774in}}%
\pgfpathlineto{\pgfqpoint{5.713424in}{2.258256in}}%
\pgfusepath{stroke}%
\end{pgfscope}%
\begin{pgfscope}%
\pgfpathrectangle{\pgfqpoint{3.985294in}{1.750000in}}{\pgfqpoint{2.279412in}{2.004545in}}%
\pgfusepath{clip}%
\pgfsetbuttcap%
\pgfsetroundjoin%
\pgfsetlinewidth{0.327319pt}%
\definecolor{currentstroke}{rgb}{0.271305,0.019942,0.347269}%
\pgfsetstrokecolor{currentstroke}%
\pgfsetdash{}{0pt}%
\pgfpathmoveto{\pgfqpoint{5.713424in}{2.258256in}}%
\pgfpathlineto{\pgfqpoint{5.663321in}{2.259869in}}%
\pgfusepath{stroke}%
\end{pgfscope}%
\begin{pgfscope}%
\pgfpathrectangle{\pgfqpoint{3.985294in}{1.750000in}}{\pgfqpoint{2.279412in}{2.004545in}}%
\pgfusepath{clip}%
\pgfsetbuttcap%
\pgfsetroundjoin%
\pgfsetlinewidth{0.324662pt}%
\definecolor{currentstroke}{rgb}{0.271305,0.019942,0.347269}%
\pgfsetstrokecolor{currentstroke}%
\pgfsetdash{}{0pt}%
\pgfpathmoveto{\pgfqpoint{5.663321in}{2.259869in}}%
\pgfpathlineto{\pgfqpoint{5.613236in}{2.261881in}}%
\pgfusepath{stroke}%
\end{pgfscope}%
\begin{pgfscope}%
\pgfpathrectangle{\pgfqpoint{3.985294in}{1.750000in}}{\pgfqpoint{2.279412in}{2.004545in}}%
\pgfusepath{clip}%
\pgfsetbuttcap%
\pgfsetroundjoin%
\pgfsetlinewidth{0.347178pt}%
\definecolor{currentstroke}{rgb}{0.274952,0.037752,0.364543}%
\pgfsetstrokecolor{currentstroke}%
\pgfsetdash{}{0pt}%
\pgfpathmoveto{\pgfqpoint{5.613236in}{2.261881in}}%
\pgfpathlineto{\pgfqpoint{5.563154in}{2.263901in}}%
\pgfusepath{stroke}%
\end{pgfscope}%
\begin{pgfscope}%
\pgfpathrectangle{\pgfqpoint{3.985294in}{1.750000in}}{\pgfqpoint{2.279412in}{2.004545in}}%
\pgfusepath{clip}%
\pgfsetbuttcap%
\pgfsetroundjoin%
\pgfsetlinewidth{0.355644pt}%
\definecolor{currentstroke}{rgb}{0.276022,0.044167,0.370164}%
\pgfsetstrokecolor{currentstroke}%
\pgfsetdash{}{0pt}%
\pgfpathmoveto{\pgfqpoint{5.563154in}{2.263901in}}%
\pgfpathlineto{\pgfqpoint{5.513099in}{2.266480in}}%
\pgfusepath{stroke}%
\end{pgfscope}%
\begin{pgfscope}%
\pgfpathrectangle{\pgfqpoint{3.985294in}{1.750000in}}{\pgfqpoint{2.279412in}{2.004545in}}%
\pgfusepath{clip}%
\pgfsetbuttcap%
\pgfsetroundjoin%
\pgfsetlinewidth{0.371428pt}%
\definecolor{currentstroke}{rgb}{0.278791,0.062145,0.386592}%
\pgfsetstrokecolor{currentstroke}%
\pgfsetdash{}{0pt}%
\pgfpathmoveto{\pgfqpoint{5.513099in}{2.266480in}}%
\pgfpathlineto{\pgfqpoint{5.463145in}{2.270277in}}%
\pgfusepath{stroke}%
\end{pgfscope}%
\begin{pgfscope}%
\pgfpathrectangle{\pgfqpoint{3.985294in}{1.750000in}}{\pgfqpoint{2.279412in}{2.004545in}}%
\pgfusepath{clip}%
\pgfsetbuttcap%
\pgfsetroundjoin%
\pgfsetlinewidth{0.382822pt}%
\definecolor{currentstroke}{rgb}{0.279566,0.067836,0.391917}%
\pgfsetstrokecolor{currentstroke}%
\pgfsetdash{}{0pt}%
\pgfpathmoveto{\pgfqpoint{5.463145in}{2.270277in}}%
\pgfpathlineto{\pgfqpoint{5.413268in}{2.274861in}}%
\pgfusepath{stroke}%
\end{pgfscope}%
\begin{pgfscope}%
\pgfpathrectangle{\pgfqpoint{3.985294in}{1.750000in}}{\pgfqpoint{2.279412in}{2.004545in}}%
\pgfusepath{clip}%
\pgfsetbuttcap%
\pgfsetroundjoin%
\pgfsetlinewidth{0.332036pt}%
\definecolor{currentstroke}{rgb}{0.272594,0.025563,0.353093}%
\pgfsetstrokecolor{currentstroke}%
\pgfsetdash{}{0pt}%
\pgfpathmoveto{\pgfqpoint{5.791795in}{2.526739in}}%
\pgfpathlineto{\pgfqpoint{5.741650in}{2.527059in}}%
\pgfusepath{stroke}%
\end{pgfscope}%
\begin{pgfscope}%
\pgfpathrectangle{\pgfqpoint{3.985294in}{1.750000in}}{\pgfqpoint{2.279412in}{2.004545in}}%
\pgfusepath{clip}%
\pgfsetbuttcap%
\pgfsetroundjoin%
\pgfsetlinewidth{0.330438pt}%
\definecolor{currentstroke}{rgb}{0.272594,0.025563,0.353093}%
\pgfsetstrokecolor{currentstroke}%
\pgfsetdash{}{0pt}%
\pgfpathmoveto{\pgfqpoint{5.741650in}{2.527059in}}%
\pgfpathlineto{\pgfqpoint{5.691503in}{2.527526in}}%
\pgfusepath{stroke}%
\end{pgfscope}%
\begin{pgfscope}%
\pgfpathrectangle{\pgfqpoint{3.985294in}{1.750000in}}{\pgfqpoint{2.279412in}{2.004545in}}%
\pgfusepath{clip}%
\pgfsetbuttcap%
\pgfsetroundjoin%
\pgfsetlinewidth{0.338217pt}%
\definecolor{currentstroke}{rgb}{0.273809,0.031497,0.358853}%
\pgfsetstrokecolor{currentstroke}%
\pgfsetdash{}{0pt}%
\pgfpathmoveto{\pgfqpoint{5.691503in}{2.527526in}}%
\pgfpathlineto{\pgfqpoint{5.641362in}{2.528308in}}%
\pgfusepath{stroke}%
\end{pgfscope}%
\begin{pgfscope}%
\pgfpathrectangle{\pgfqpoint{3.985294in}{1.750000in}}{\pgfqpoint{2.279412in}{2.004545in}}%
\pgfusepath{clip}%
\pgfsetbuttcap%
\pgfsetroundjoin%
\pgfsetlinewidth{0.365912pt}%
\definecolor{currentstroke}{rgb}{0.277941,0.056324,0.381191}%
\pgfsetstrokecolor{currentstroke}%
\pgfsetdash{}{0pt}%
\pgfpathmoveto{\pgfqpoint{5.641362in}{2.528308in}}%
\pgfpathlineto{\pgfqpoint{5.591229in}{2.529502in}}%
\pgfusepath{stroke}%
\end{pgfscope}%
\begin{pgfscope}%
\pgfpathrectangle{\pgfqpoint{3.985294in}{1.750000in}}{\pgfqpoint{2.279412in}{2.004545in}}%
\pgfusepath{clip}%
\pgfsetbuttcap%
\pgfsetroundjoin%
\pgfsetlinewidth{0.410704pt}%
\definecolor{currentstroke}{rgb}{0.281924,0.089666,0.412415}%
\pgfsetstrokecolor{currentstroke}%
\pgfsetdash{}{0pt}%
\pgfpathmoveto{\pgfqpoint{5.591229in}{2.529502in}}%
\pgfpathlineto{\pgfqpoint{5.541099in}{2.530796in}}%
\pgfusepath{stroke}%
\end{pgfscope}%
\begin{pgfscope}%
\pgfpathrectangle{\pgfqpoint{3.985294in}{1.750000in}}{\pgfqpoint{2.279412in}{2.004545in}}%
\pgfusepath{clip}%
\pgfsetbuttcap%
\pgfsetroundjoin%
\pgfsetlinewidth{0.460182pt}%
\definecolor{currentstroke}{rgb}{0.283072,0.130895,0.449241}%
\pgfsetstrokecolor{currentstroke}%
\pgfsetdash{}{0pt}%
\pgfpathmoveto{\pgfqpoint{5.541099in}{2.530796in}}%
\pgfpathlineto{\pgfqpoint{5.490971in}{2.532174in}}%
\pgfusepath{stroke}%
\end{pgfscope}%
\begin{pgfscope}%
\pgfpathrectangle{\pgfqpoint{3.985294in}{1.750000in}}{\pgfqpoint{2.279412in}{2.004545in}}%
\pgfusepath{clip}%
\pgfsetbuttcap%
\pgfsetroundjoin%
\pgfsetlinewidth{0.507923pt}%
\definecolor{currentstroke}{rgb}{0.280255,0.165693,0.476498}%
\pgfsetstrokecolor{currentstroke}%
\pgfsetdash{}{0pt}%
\pgfpathmoveto{\pgfqpoint{5.490971in}{2.532174in}}%
\pgfpathlineto{\pgfqpoint{5.440861in}{2.533937in}}%
\pgfusepath{stroke}%
\end{pgfscope}%
\begin{pgfscope}%
\pgfpathrectangle{\pgfqpoint{3.985294in}{1.750000in}}{\pgfqpoint{2.279412in}{2.004545in}}%
\pgfusepath{clip}%
\pgfsetbuttcap%
\pgfsetroundjoin%
\pgfsetlinewidth{0.563479pt}%
\definecolor{currentstroke}{rgb}{0.273006,0.204520,0.501721}%
\pgfsetstrokecolor{currentstroke}%
\pgfsetdash{}{0pt}%
\pgfpathmoveto{\pgfqpoint{5.440861in}{2.533937in}}%
\pgfpathlineto{\pgfqpoint{5.390789in}{2.536399in}}%
\pgfusepath{stroke}%
\end{pgfscope}%
\begin{pgfscope}%
\pgfpathrectangle{\pgfqpoint{3.985294in}{1.750000in}}{\pgfqpoint{2.279412in}{2.004545in}}%
\pgfusepath{clip}%
\pgfsetbuttcap%
\pgfsetroundjoin%
\pgfsetlinewidth{0.635505pt}%
\definecolor{currentstroke}{rgb}{0.258965,0.251537,0.524736}%
\pgfsetstrokecolor{currentstroke}%
\pgfsetdash{}{0pt}%
\pgfpathmoveto{\pgfqpoint{5.390789in}{2.536399in}}%
\pgfpathlineto{\pgfqpoint{5.340760in}{2.539476in}}%
\pgfusepath{stroke}%
\end{pgfscope}%
\begin{pgfscope}%
\pgfpathrectangle{\pgfqpoint{3.985294in}{1.750000in}}{\pgfqpoint{2.279412in}{2.004545in}}%
\pgfusepath{clip}%
\pgfsetbuttcap%
\pgfsetroundjoin%
\pgfsetlinewidth{0.684846pt}%
\definecolor{currentstroke}{rgb}{0.246811,0.283237,0.535941}%
\pgfsetstrokecolor{currentstroke}%
\pgfsetdash{}{0pt}%
\pgfpathmoveto{\pgfqpoint{5.340760in}{2.539476in}}%
\pgfpathlineto{\pgfqpoint{5.290807in}{2.543353in}}%
\pgfusepath{stroke}%
\end{pgfscope}%
\begin{pgfscope}%
\pgfpathrectangle{\pgfqpoint{3.985294in}{1.750000in}}{\pgfqpoint{2.279412in}{2.004545in}}%
\pgfusepath{clip}%
\pgfsetbuttcap%
\pgfsetroundjoin%
\pgfsetlinewidth{0.330662pt}%
\definecolor{currentstroke}{rgb}{0.272594,0.025563,0.353093}%
\pgfsetstrokecolor{currentstroke}%
\pgfsetdash{}{0pt}%
\pgfpathmoveto{\pgfqpoint{5.535335in}{3.338661in}}%
\pgfpathlineto{\pgfqpoint{5.485381in}{3.335670in}}%
\pgfusepath{stroke}%
\end{pgfscope}%
\begin{pgfscope}%
\pgfpathrectangle{\pgfqpoint{3.985294in}{1.750000in}}{\pgfqpoint{2.279412in}{2.004545in}}%
\pgfusepath{clip}%
\pgfsetbuttcap%
\pgfsetroundjoin%
\pgfsetlinewidth{0.346881pt}%
\definecolor{currentstroke}{rgb}{0.274952,0.037752,0.364543}%
\pgfsetstrokecolor{currentstroke}%
\pgfsetdash{}{0pt}%
\pgfpathmoveto{\pgfqpoint{5.485381in}{3.335670in}}%
\pgfpathlineto{\pgfqpoint{5.435679in}{3.330333in}}%
\pgfusepath{stroke}%
\end{pgfscope}%
\begin{pgfscope}%
\pgfpathrectangle{\pgfqpoint{3.985294in}{1.750000in}}{\pgfqpoint{2.279412in}{2.004545in}}%
\pgfusepath{clip}%
\pgfsetbuttcap%
\pgfsetroundjoin%
\pgfsetlinewidth{0.366078pt}%
\definecolor{currentstroke}{rgb}{0.277941,0.056324,0.381191}%
\pgfsetstrokecolor{currentstroke}%
\pgfsetdash{}{0pt}%
\pgfpathmoveto{\pgfqpoint{5.435679in}{3.330333in}}%
\pgfpathlineto{\pgfqpoint{5.386172in}{3.323584in}}%
\pgfusepath{stroke}%
\end{pgfscope}%
\begin{pgfscope}%
\pgfpathrectangle{\pgfqpoint{3.985294in}{1.750000in}}{\pgfqpoint{2.279412in}{2.004545in}}%
\pgfusepath{clip}%
\pgfsetbuttcap%
\pgfsetroundjoin%
\pgfsetlinewidth{0.343407pt}%
\definecolor{currentstroke}{rgb}{0.274952,0.037752,0.364543}%
\pgfsetstrokecolor{currentstroke}%
\pgfsetdash{}{0pt}%
\pgfpathmoveto{\pgfqpoint{5.386172in}{3.323584in}}%
\pgfpathlineto{\pgfqpoint{5.336580in}{3.317370in}}%
\pgfusepath{stroke}%
\end{pgfscope}%
\begin{pgfscope}%
\pgfpathrectangle{\pgfqpoint{3.985294in}{1.750000in}}{\pgfqpoint{2.279412in}{2.004545in}}%
\pgfusepath{clip}%
\pgfsetbuttcap%
\pgfsetroundjoin%
\pgfsetlinewidth{0.352975pt}%
\definecolor{currentstroke}{rgb}{0.276022,0.044167,0.370164}%
\pgfsetstrokecolor{currentstroke}%
\pgfsetdash{}{0pt}%
\pgfpathmoveto{\pgfqpoint{5.336580in}{3.317370in}}%
\pgfpathlineto{\pgfqpoint{5.287339in}{3.309388in}}%
\pgfusepath{stroke}%
\end{pgfscope}%
\begin{pgfscope}%
\pgfpathrectangle{\pgfqpoint{3.985294in}{1.750000in}}{\pgfqpoint{2.279412in}{2.004545in}}%
\pgfusepath{clip}%
\pgfsetbuttcap%
\pgfsetroundjoin%
\pgfsetlinewidth{0.347975pt}%
\definecolor{currentstroke}{rgb}{0.274952,0.037752,0.364543}%
\pgfsetstrokecolor{currentstroke}%
\pgfsetdash{}{0pt}%
\pgfpathmoveto{\pgfqpoint{5.287339in}{3.309388in}}%
\pgfpathlineto{\pgfqpoint{5.238308in}{3.301242in}}%
\pgfusepath{stroke}%
\end{pgfscope}%
\begin{pgfscope}%
\pgfpathrectangle{\pgfqpoint{3.985294in}{1.750000in}}{\pgfqpoint{2.279412in}{2.004545in}}%
\pgfusepath{clip}%
\pgfsetbuttcap%
\pgfsetroundjoin%
\pgfsetlinewidth{0.348587pt}%
\definecolor{currentstroke}{rgb}{0.274952,0.037752,0.364543}%
\pgfsetstrokecolor{currentstroke}%
\pgfsetdash{}{0pt}%
\pgfpathmoveto{\pgfqpoint{5.238308in}{3.301242in}}%
\pgfpathlineto{\pgfqpoint{5.189509in}{3.292665in}}%
\pgfusepath{stroke}%
\end{pgfscope}%
\begin{pgfscope}%
\pgfpathrectangle{\pgfqpoint{3.985294in}{1.750000in}}{\pgfqpoint{2.279412in}{2.004545in}}%
\pgfusepath{clip}%
\pgfsetbuttcap%
\pgfsetroundjoin%
\pgfsetlinewidth{0.388454pt}%
\definecolor{currentstroke}{rgb}{0.280267,0.073417,0.397163}%
\pgfsetstrokecolor{currentstroke}%
\pgfsetdash{}{0pt}%
\pgfpathmoveto{\pgfqpoint{5.428948in}{2.196133in}}%
\pgfpathlineto{\pgfqpoint{5.379412in}{2.202713in}}%
\pgfusepath{stroke}%
\end{pgfscope}%
\begin{pgfscope}%
\pgfpathrectangle{\pgfqpoint{3.985294in}{1.750000in}}{\pgfqpoint{2.279412in}{2.004545in}}%
\pgfusepath{clip}%
\pgfsetbuttcap%
\pgfsetroundjoin%
\pgfsetlinewidth{0.372057pt}%
\definecolor{currentstroke}{rgb}{0.278791,0.062145,0.386592}%
\pgfsetstrokecolor{currentstroke}%
\pgfsetdash{}{0pt}%
\pgfpathmoveto{\pgfqpoint{5.379412in}{2.202713in}}%
\pgfpathlineto{\pgfqpoint{5.330168in}{2.210991in}}%
\pgfusepath{stroke}%
\end{pgfscope}%
\begin{pgfscope}%
\pgfpathrectangle{\pgfqpoint{3.985294in}{1.750000in}}{\pgfqpoint{2.279412in}{2.004545in}}%
\pgfusepath{clip}%
\pgfsetbuttcap%
\pgfsetroundjoin%
\pgfsetlinewidth{0.367601pt}%
\definecolor{currentstroke}{rgb}{0.277941,0.056324,0.381191}%
\pgfsetstrokecolor{currentstroke}%
\pgfsetdash{}{0pt}%
\pgfpathmoveto{\pgfqpoint{5.330168in}{2.210991in}}%
\pgfpathlineto{\pgfqpoint{5.281062in}{2.219870in}}%
\pgfusepath{stroke}%
\end{pgfscope}%
\begin{pgfscope}%
\pgfpathrectangle{\pgfqpoint{3.985294in}{1.750000in}}{\pgfqpoint{2.279412in}{2.004545in}}%
\pgfusepath{clip}%
\pgfsetbuttcap%
\pgfsetroundjoin%
\pgfsetlinewidth{0.360678pt}%
\definecolor{currentstroke}{rgb}{0.277018,0.050344,0.375715}%
\pgfsetstrokecolor{currentstroke}%
\pgfsetdash{}{0pt}%
\pgfpathmoveto{\pgfqpoint{5.281062in}{2.219870in}}%
\pgfpathlineto{\pgfqpoint{5.231950in}{2.228606in}}%
\pgfusepath{stroke}%
\end{pgfscope}%
\begin{pgfscope}%
\pgfpathrectangle{\pgfqpoint{3.985294in}{1.750000in}}{\pgfqpoint{2.279412in}{2.004545in}}%
\pgfusepath{clip}%
\pgfsetbuttcap%
\pgfsetroundjoin%
\pgfsetlinewidth{0.339275pt}%
\definecolor{currentstroke}{rgb}{0.273809,0.031497,0.358853}%
\pgfsetstrokecolor{currentstroke}%
\pgfsetdash{}{0pt}%
\pgfpathmoveto{\pgfqpoint{5.231950in}{2.228606in}}%
\pgfpathlineto{\pgfqpoint{5.183345in}{2.238918in}}%
\pgfusepath{stroke}%
\end{pgfscope}%
\begin{pgfscope}%
\pgfpathrectangle{\pgfqpoint{3.985294in}{1.750000in}}{\pgfqpoint{2.279412in}{2.004545in}}%
\pgfusepath{clip}%
\pgfsetbuttcap%
\pgfsetroundjoin%
\pgfsetlinewidth{0.312863pt}%
\definecolor{currentstroke}{rgb}{0.268510,0.009605,0.335427}%
\pgfsetstrokecolor{currentstroke}%
\pgfsetdash{}{0pt}%
\pgfpathmoveto{\pgfqpoint{5.740503in}{3.158234in}}%
\pgfpathlineto{\pgfqpoint{5.690402in}{3.159422in}}%
\pgfusepath{stroke}%
\end{pgfscope}%
\begin{pgfscope}%
\pgfpathrectangle{\pgfqpoint{3.985294in}{1.750000in}}{\pgfqpoint{2.279412in}{2.004545in}}%
\pgfusepath{clip}%
\pgfsetbuttcap%
\pgfsetroundjoin%
\pgfsetlinewidth{0.333853pt}%
\definecolor{currentstroke}{rgb}{0.272594,0.025563,0.353093}%
\pgfsetstrokecolor{currentstroke}%
\pgfsetdash{}{0pt}%
\pgfpathmoveto{\pgfqpoint{5.690402in}{3.159422in}}%
\pgfpathlineto{\pgfqpoint{5.640270in}{3.158658in}}%
\pgfusepath{stroke}%
\end{pgfscope}%
\begin{pgfscope}%
\pgfpathrectangle{\pgfqpoint{3.985294in}{1.750000in}}{\pgfqpoint{2.279412in}{2.004545in}}%
\pgfusepath{clip}%
\pgfsetbuttcap%
\pgfsetroundjoin%
\pgfsetlinewidth{0.342365pt}%
\definecolor{currentstroke}{rgb}{0.274952,0.037752,0.364543}%
\pgfsetstrokecolor{currentstroke}%
\pgfsetdash{}{0pt}%
\pgfpathmoveto{\pgfqpoint{5.640270in}{3.158658in}}%
\pgfpathlineto{\pgfqpoint{5.590171in}{3.156867in}}%
\pgfusepath{stroke}%
\end{pgfscope}%
\begin{pgfscope}%
\pgfpathrectangle{\pgfqpoint{3.985294in}{1.750000in}}{\pgfqpoint{2.279412in}{2.004545in}}%
\pgfusepath{clip}%
\pgfsetbuttcap%
\pgfsetroundjoin%
\pgfsetlinewidth{0.355737pt}%
\definecolor{currentstroke}{rgb}{0.276022,0.044167,0.370164}%
\pgfsetstrokecolor{currentstroke}%
\pgfsetdash{}{0pt}%
\pgfpathmoveto{\pgfqpoint{5.590171in}{3.156867in}}%
\pgfpathlineto{\pgfqpoint{5.540085in}{3.154774in}}%
\pgfusepath{stroke}%
\end{pgfscope}%
\begin{pgfscope}%
\pgfpathrectangle{\pgfqpoint{3.985294in}{1.750000in}}{\pgfqpoint{2.279412in}{2.004545in}}%
\pgfusepath{clip}%
\pgfsetbuttcap%
\pgfsetroundjoin%
\pgfsetlinewidth{0.374007pt}%
\definecolor{currentstroke}{rgb}{0.278791,0.062145,0.386592}%
\pgfsetstrokecolor{currentstroke}%
\pgfsetdash{}{0pt}%
\pgfpathmoveto{\pgfqpoint{5.540085in}{3.154774in}}%
\pgfpathlineto{\pgfqpoint{5.490033in}{3.152052in}}%
\pgfusepath{stroke}%
\end{pgfscope}%
\begin{pgfscope}%
\pgfpathrectangle{\pgfqpoint{3.985294in}{1.750000in}}{\pgfqpoint{2.279412in}{2.004545in}}%
\pgfusepath{clip}%
\pgfsetbuttcap%
\pgfsetroundjoin%
\pgfsetlinewidth{0.388676pt}%
\definecolor{currentstroke}{rgb}{0.280267,0.073417,0.397163}%
\pgfsetstrokecolor{currentstroke}%
\pgfsetdash{}{0pt}%
\pgfpathmoveto{\pgfqpoint{5.490033in}{3.152052in}}%
\pgfpathlineto{\pgfqpoint{5.440082in}{3.148185in}}%
\pgfusepath{stroke}%
\end{pgfscope}%
\begin{pgfscope}%
\pgfpathrectangle{\pgfqpoint{3.985294in}{1.750000in}}{\pgfqpoint{2.279412in}{2.004545in}}%
\pgfusepath{clip}%
\pgfsetbuttcap%
\pgfsetroundjoin%
\pgfsetlinewidth{0.317890pt}%
\definecolor{currentstroke}{rgb}{0.269944,0.014625,0.341379}%
\pgfsetstrokecolor{currentstroke}%
\pgfsetdash{}{0pt}%
\pgfpathmoveto{\pgfqpoint{5.740503in}{3.248447in}}%
\pgfpathlineto{\pgfqpoint{5.690431in}{3.247947in}}%
\pgfusepath{stroke}%
\end{pgfscope}%
\begin{pgfscope}%
\pgfpathrectangle{\pgfqpoint{3.985294in}{1.750000in}}{\pgfqpoint{2.279412in}{2.004545in}}%
\pgfusepath{clip}%
\pgfsetbuttcap%
\pgfsetroundjoin%
\pgfsetlinewidth{0.323565pt}%
\definecolor{currentstroke}{rgb}{0.271305,0.019942,0.347269}%
\pgfsetstrokecolor{currentstroke}%
\pgfsetdash{}{0pt}%
\pgfpathmoveto{\pgfqpoint{5.690431in}{3.247947in}}%
\pgfpathlineto{\pgfqpoint{5.640310in}{3.248135in}}%
\pgfusepath{stroke}%
\end{pgfscope}%
\begin{pgfscope}%
\pgfpathrectangle{\pgfqpoint{3.985294in}{1.750000in}}{\pgfqpoint{2.279412in}{2.004545in}}%
\pgfusepath{clip}%
\pgfsetbuttcap%
\pgfsetroundjoin%
\pgfsetlinewidth{0.333218pt}%
\definecolor{currentstroke}{rgb}{0.272594,0.025563,0.353093}%
\pgfsetstrokecolor{currentstroke}%
\pgfsetdash{}{0pt}%
\pgfpathmoveto{\pgfqpoint{5.640310in}{3.248135in}}%
\pgfpathlineto{\pgfqpoint{5.590296in}{3.246041in}}%
\pgfusepath{stroke}%
\end{pgfscope}%
\begin{pgfscope}%
\pgfpathrectangle{\pgfqpoint{3.985294in}{1.750000in}}{\pgfqpoint{2.279412in}{2.004545in}}%
\pgfusepath{clip}%
\pgfsetbuttcap%
\pgfsetroundjoin%
\pgfsetlinewidth{0.340591pt}%
\definecolor{currentstroke}{rgb}{0.273809,0.031497,0.358853}%
\pgfsetstrokecolor{currentstroke}%
\pgfsetdash{}{0pt}%
\pgfpathmoveto{\pgfqpoint{5.590296in}{3.246041in}}%
\pgfpathlineto{\pgfqpoint{5.540221in}{3.244121in}}%
\pgfusepath{stroke}%
\end{pgfscope}%
\begin{pgfscope}%
\pgfpathrectangle{\pgfqpoint{3.985294in}{1.750000in}}{\pgfqpoint{2.279412in}{2.004545in}}%
\pgfusepath{clip}%
\pgfsetbuttcap%
\pgfsetroundjoin%
\pgfsetlinewidth{0.357469pt}%
\definecolor{currentstroke}{rgb}{0.277018,0.050344,0.375715}%
\pgfsetstrokecolor{currentstroke}%
\pgfsetdash{}{0pt}%
\pgfpathmoveto{\pgfqpoint{5.540221in}{3.244121in}}%
\pgfpathlineto{\pgfqpoint{5.490159in}{3.241943in}}%
\pgfusepath{stroke}%
\end{pgfscope}%
\begin{pgfscope}%
\pgfpathrectangle{\pgfqpoint{3.985294in}{1.750000in}}{\pgfqpoint{2.279412in}{2.004545in}}%
\pgfusepath{clip}%
\pgfsetbuttcap%
\pgfsetroundjoin%
\pgfsetlinewidth{0.362457pt}%
\definecolor{currentstroke}{rgb}{0.277018,0.050344,0.375715}%
\pgfsetstrokecolor{currentstroke}%
\pgfsetdash{}{0pt}%
\pgfpathmoveto{\pgfqpoint{5.490159in}{3.241943in}}%
\pgfpathlineto{\pgfqpoint{5.440193in}{3.238188in}}%
\pgfusepath{stroke}%
\end{pgfscope}%
\begin{pgfscope}%
\pgfpathrectangle{\pgfqpoint{3.985294in}{1.750000in}}{\pgfqpoint{2.279412in}{2.004545in}}%
\pgfusepath{clip}%
\pgfsetbuttcap%
\pgfsetroundjoin%
\pgfsetlinewidth{0.376976pt}%
\definecolor{currentstroke}{rgb}{0.279566,0.067836,0.391917}%
\pgfsetstrokecolor{currentstroke}%
\pgfsetdash{}{0pt}%
\pgfpathmoveto{\pgfqpoint{5.440193in}{3.238188in}}%
\pgfpathlineto{\pgfqpoint{5.390342in}{3.233429in}}%
\pgfusepath{stroke}%
\end{pgfscope}%
\begin{pgfscope}%
\pgfpathrectangle{\pgfqpoint{3.985294in}{1.750000in}}{\pgfqpoint{2.279412in}{2.004545in}}%
\pgfusepath{clip}%
\pgfsetbuttcap%
\pgfsetroundjoin%
\pgfsetlinewidth{0.388201pt}%
\definecolor{currentstroke}{rgb}{0.280267,0.073417,0.397163}%
\pgfsetstrokecolor{currentstroke}%
\pgfsetdash{}{0pt}%
\pgfpathmoveto{\pgfqpoint{5.390342in}{3.233429in}}%
\pgfpathlineto{\pgfqpoint{5.340654in}{3.227538in}}%
\pgfusepath{stroke}%
\end{pgfscope}%
\begin{pgfscope}%
\pgfpathrectangle{\pgfqpoint{3.985294in}{1.750000in}}{\pgfqpoint{2.279412in}{2.004545in}}%
\pgfusepath{clip}%
\pgfsetbuttcap%
\pgfsetroundjoin%
\pgfsetlinewidth{0.371796pt}%
\definecolor{currentstroke}{rgb}{0.278791,0.062145,0.386592}%
\pgfsetstrokecolor{currentstroke}%
\pgfsetdash{}{0pt}%
\pgfpathmoveto{\pgfqpoint{5.340654in}{3.227538in}}%
\pgfpathlineto{\pgfqpoint{5.291286in}{3.219966in}}%
\pgfusepath{stroke}%
\end{pgfscope}%
\begin{pgfscope}%
\pgfpathrectangle{\pgfqpoint{3.985294in}{1.750000in}}{\pgfqpoint{2.279412in}{2.004545in}}%
\pgfusepath{clip}%
\pgfsetbuttcap%
\pgfsetroundjoin%
\pgfsetlinewidth{0.388474pt}%
\definecolor{currentstroke}{rgb}{0.280267,0.073417,0.397163}%
\pgfsetstrokecolor{currentstroke}%
\pgfsetdash{}{0pt}%
\pgfpathmoveto{\pgfqpoint{5.291286in}{3.219966in}}%
\pgfpathlineto{\pgfqpoint{5.242332in}{3.210545in}}%
\pgfusepath{stroke}%
\end{pgfscope}%
\begin{pgfscope}%
\pgfpathrectangle{\pgfqpoint{3.985294in}{1.750000in}}{\pgfqpoint{2.279412in}{2.004545in}}%
\pgfusepath{clip}%
\pgfsetbuttcap%
\pgfsetroundjoin%
\pgfsetlinewidth{0.378960pt}%
\definecolor{currentstroke}{rgb}{0.279566,0.067836,0.391917}%
\pgfsetstrokecolor{currentstroke}%
\pgfsetdash{}{0pt}%
\pgfpathmoveto{\pgfqpoint{5.242332in}{3.210545in}}%
\pgfpathlineto{\pgfqpoint{5.194120in}{3.198644in}}%
\pgfusepath{stroke}%
\end{pgfscope}%
\begin{pgfscope}%
\pgfpathrectangle{\pgfqpoint{3.985294in}{1.750000in}}{\pgfqpoint{2.279412in}{2.004545in}}%
\pgfusepath{clip}%
\pgfsetbuttcap%
\pgfsetroundjoin%
\pgfsetlinewidth{0.388065pt}%
\definecolor{currentstroke}{rgb}{0.280267,0.073417,0.397163}%
\pgfsetstrokecolor{currentstroke}%
\pgfsetdash{}{0pt}%
\pgfpathmoveto{\pgfqpoint{5.194120in}{3.198644in}}%
\pgfpathlineto{\pgfqpoint{5.147100in}{3.183401in}}%
\pgfusepath{stroke}%
\end{pgfscope}%
\begin{pgfscope}%
\pgfpathrectangle{\pgfqpoint{3.985294in}{1.750000in}}{\pgfqpoint{2.279412in}{2.004545in}}%
\pgfusepath{clip}%
\pgfsetbuttcap%
\pgfsetroundjoin%
\pgfsetlinewidth{0.391463pt}%
\definecolor{currentstroke}{rgb}{0.280894,0.078907,0.402329}%
\pgfsetstrokecolor{currentstroke}%
\pgfsetdash{}{0pt}%
\pgfpathmoveto{\pgfqpoint{5.147100in}{3.183401in}}%
\pgfpathlineto{\pgfqpoint{5.101784in}{3.164690in}}%
\pgfusepath{stroke}%
\end{pgfscope}%
\begin{pgfscope}%
\pgfpathrectangle{\pgfqpoint{3.985294in}{1.750000in}}{\pgfqpoint{2.279412in}{2.004545in}}%
\pgfusepath{clip}%
\pgfsetbuttcap%
\pgfsetroundjoin%
\pgfsetlinewidth{0.417246pt}%
\definecolor{currentstroke}{rgb}{0.282327,0.094955,0.417331}%
\pgfsetstrokecolor{currentstroke}%
\pgfsetdash{}{0pt}%
\pgfpathmoveto{\pgfqpoint{5.101784in}{3.164690in}}%
\pgfpathlineto{\pgfqpoint{5.061341in}{3.139506in}}%
\pgfusepath{stroke}%
\end{pgfscope}%
\begin{pgfscope}%
\pgfpathrectangle{\pgfqpoint{3.985294in}{1.750000in}}{\pgfqpoint{2.279412in}{2.004545in}}%
\pgfusepath{clip}%
\pgfsetbuttcap%
\pgfsetroundjoin%
\pgfsetlinewidth{0.438138pt}%
\definecolor{currentstroke}{rgb}{0.283091,0.110553,0.431554}%
\pgfsetstrokecolor{currentstroke}%
\pgfsetdash{}{0pt}%
\pgfpathmoveto{\pgfqpoint{5.061341in}{3.139506in}}%
\pgfpathlineto{\pgfqpoint{5.027967in}{3.112795in}}%
\pgfusepath{stroke}%
\end{pgfscope}%
\begin{pgfscope}%
\pgfpathrectangle{\pgfqpoint{3.985294in}{1.750000in}}{\pgfqpoint{2.279412in}{2.004545in}}%
\pgfusepath{clip}%
\pgfsetbuttcap%
\pgfsetroundjoin%
\pgfsetlinewidth{0.477260pt}%
\definecolor{currentstroke}{rgb}{0.282623,0.140926,0.457517}%
\pgfsetstrokecolor{currentstroke}%
\pgfsetdash{}{0pt}%
\pgfpathmoveto{\pgfqpoint{5.027967in}{3.112795in}}%
\pgfpathlineto{\pgfqpoint{4.995511in}{3.079532in}}%
\pgfusepath{stroke}%
\end{pgfscope}%
\begin{pgfscope}%
\pgfpathrectangle{\pgfqpoint{3.985294in}{1.750000in}}{\pgfqpoint{2.279412in}{2.004545in}}%
\pgfusepath{clip}%
\pgfsetbuttcap%
\pgfsetroundjoin%
\pgfsetlinewidth{0.342298pt}%
\definecolor{currentstroke}{rgb}{0.274952,0.037752,0.364543}%
\pgfsetstrokecolor{currentstroke}%
\pgfsetdash{}{0pt}%
\pgfpathmoveto{\pgfqpoint{5.484043in}{3.293554in}}%
\pgfpathlineto{\pgfqpoint{5.434025in}{3.290362in}}%
\pgfusepath{stroke}%
\end{pgfscope}%
\begin{pgfscope}%
\pgfpathrectangle{\pgfqpoint{3.985294in}{1.750000in}}{\pgfqpoint{2.279412in}{2.004545in}}%
\pgfusepath{clip}%
\pgfsetbuttcap%
\pgfsetroundjoin%
\pgfsetlinewidth{0.362360pt}%
\definecolor{currentstroke}{rgb}{0.277018,0.050344,0.375715}%
\pgfsetstrokecolor{currentstroke}%
\pgfsetdash{}{0pt}%
\pgfpathmoveto{\pgfqpoint{5.434025in}{3.290362in}}%
\pgfpathlineto{\pgfqpoint{5.384162in}{3.285799in}}%
\pgfusepath{stroke}%
\end{pgfscope}%
\begin{pgfscope}%
\pgfpathrectangle{\pgfqpoint{3.985294in}{1.750000in}}{\pgfqpoint{2.279412in}{2.004545in}}%
\pgfusepath{clip}%
\pgfsetbuttcap%
\pgfsetroundjoin%
\pgfsetlinewidth{0.376321pt}%
\definecolor{currentstroke}{rgb}{0.278791,0.062145,0.386592}%
\pgfsetstrokecolor{currentstroke}%
\pgfsetdash{}{0pt}%
\pgfpathmoveto{\pgfqpoint{5.384162in}{3.285799in}}%
\pgfpathlineto{\pgfqpoint{5.334603in}{3.279401in}}%
\pgfusepath{stroke}%
\end{pgfscope}%
\begin{pgfscope}%
\pgfpathrectangle{\pgfqpoint{3.985294in}{1.750000in}}{\pgfqpoint{2.279412in}{2.004545in}}%
\pgfusepath{clip}%
\pgfsetbuttcap%
\pgfsetroundjoin%
\pgfsetlinewidth{0.368544pt}%
\definecolor{currentstroke}{rgb}{0.277941,0.056324,0.381191}%
\pgfsetstrokecolor{currentstroke}%
\pgfsetdash{}{0pt}%
\pgfpathmoveto{\pgfqpoint{5.334603in}{3.279401in}}%
\pgfpathlineto{\pgfqpoint{5.286361in}{3.269194in}}%
\pgfusepath{stroke}%
\end{pgfscope}%
\begin{pgfscope}%
\pgfpathrectangle{\pgfqpoint{3.985294in}{1.750000in}}{\pgfqpoint{2.279412in}{2.004545in}}%
\pgfusepath{clip}%
\pgfsetbuttcap%
\pgfsetroundjoin%
\pgfsetlinewidth{0.324107pt}%
\definecolor{currentstroke}{rgb}{0.271305,0.019942,0.347269}%
\pgfsetstrokecolor{currentstroke}%
\pgfsetdash{}{0pt}%
\pgfpathmoveto{\pgfqpoint{5.286361in}{3.269194in}}%
\pgfpathlineto{\pgfqpoint{5.239398in}{3.255012in}}%
\pgfusepath{stroke}%
\end{pgfscope}%
\begin{pgfscope}%
\pgfpathrectangle{\pgfqpoint{3.985294in}{1.750000in}}{\pgfqpoint{2.279412in}{2.004545in}}%
\pgfusepath{clip}%
\pgfsetbuttcap%
\pgfsetroundjoin%
\pgfsetlinewidth{0.337729pt}%
\definecolor{currentstroke}{rgb}{0.273809,0.031497,0.358853}%
\pgfsetstrokecolor{currentstroke}%
\pgfsetdash{}{0pt}%
\pgfpathmoveto{\pgfqpoint{5.239398in}{3.255012in}}%
\pgfpathlineto{\pgfqpoint{5.192174in}{3.240570in}}%
\pgfusepath{stroke}%
\end{pgfscope}%
\begin{pgfscope}%
\pgfpathrectangle{\pgfqpoint{3.985294in}{1.750000in}}{\pgfqpoint{2.279412in}{2.004545in}}%
\pgfusepath{clip}%
\pgfsetbuttcap%
\pgfsetroundjoin%
\pgfsetlinewidth{0.367150pt}%
\definecolor{currentstroke}{rgb}{0.277941,0.056324,0.381191}%
\pgfsetstrokecolor{currentstroke}%
\pgfsetdash{}{0pt}%
\pgfpathmoveto{\pgfqpoint{5.192174in}{3.240570in}}%
\pgfpathlineto{\pgfqpoint{5.144346in}{3.228396in}}%
\pgfusepath{stroke}%
\end{pgfscope}%
\begin{pgfscope}%
\pgfpathrectangle{\pgfqpoint{3.985294in}{1.750000in}}{\pgfqpoint{2.279412in}{2.004545in}}%
\pgfusepath{clip}%
\pgfsetbuttcap%
\pgfsetroundjoin%
\pgfsetlinewidth{0.366580pt}%
\definecolor{currentstroke}{rgb}{0.277941,0.056324,0.381191}%
\pgfsetstrokecolor{currentstroke}%
\pgfsetdash{}{0pt}%
\pgfpathmoveto{\pgfqpoint{5.144346in}{3.228396in}}%
\pgfpathlineto{\pgfqpoint{5.144346in}{3.228396in}}%
\pgfusepath{stroke}%
\end{pgfscope}%
\begin{pgfscope}%
\pgfpathrectangle{\pgfqpoint{3.985294in}{1.750000in}}{\pgfqpoint{2.279412in}{2.004545in}}%
\pgfusepath{clip}%
\pgfsetbuttcap%
\pgfsetroundjoin%
\pgfsetlinewidth{0.394626pt}%
\definecolor{currentstroke}{rgb}{0.280894,0.078907,0.402329}%
\pgfsetstrokecolor{currentstroke}%
\pgfsetdash{}{0pt}%
\pgfpathmoveto{\pgfqpoint{5.370113in}{2.258977in}}%
\pgfpathlineto{\pgfqpoint{5.320682in}{2.266343in}}%
\pgfusepath{stroke}%
\end{pgfscope}%
\begin{pgfscope}%
\pgfpathrectangle{\pgfqpoint{3.985294in}{1.750000in}}{\pgfqpoint{2.279412in}{2.004545in}}%
\pgfusepath{clip}%
\pgfsetbuttcap%
\pgfsetroundjoin%
\pgfsetlinewidth{0.374979pt}%
\definecolor{currentstroke}{rgb}{0.278791,0.062145,0.386592}%
\pgfsetstrokecolor{currentstroke}%
\pgfsetdash{}{0pt}%
\pgfpathmoveto{\pgfqpoint{5.320682in}{2.266343in}}%
\pgfpathlineto{\pgfqpoint{5.271643in}{2.275381in}}%
\pgfusepath{stroke}%
\end{pgfscope}%
\begin{pgfscope}%
\pgfpathrectangle{\pgfqpoint{3.985294in}{1.750000in}}{\pgfqpoint{2.279412in}{2.004545in}}%
\pgfusepath{clip}%
\pgfsetbuttcap%
\pgfsetroundjoin%
\pgfsetlinewidth{0.405384pt}%
\definecolor{currentstroke}{rgb}{0.281924,0.089666,0.412415}%
\pgfsetstrokecolor{currentstroke}%
\pgfsetdash{}{0pt}%
\pgfpathmoveto{\pgfqpoint{5.271643in}{2.275381in}}%
\pgfpathlineto{\pgfqpoint{5.223015in}{2.286008in}}%
\pgfusepath{stroke}%
\end{pgfscope}%
\begin{pgfscope}%
\pgfpathrectangle{\pgfqpoint{3.985294in}{1.750000in}}{\pgfqpoint{2.279412in}{2.004545in}}%
\pgfusepath{clip}%
\pgfsetbuttcap%
\pgfsetroundjoin%
\pgfsetlinewidth{0.395971pt}%
\definecolor{currentstroke}{rgb}{0.280894,0.078907,0.402329}%
\pgfsetstrokecolor{currentstroke}%
\pgfsetdash{}{0pt}%
\pgfpathmoveto{\pgfqpoint{5.223015in}{2.286008in}}%
\pgfpathlineto{\pgfqpoint{5.176292in}{2.301205in}}%
\pgfusepath{stroke}%
\end{pgfscope}%
\begin{pgfscope}%
\pgfpathrectangle{\pgfqpoint{3.985294in}{1.750000in}}{\pgfqpoint{2.279412in}{2.004545in}}%
\pgfusepath{clip}%
\pgfsetbuttcap%
\pgfsetroundjoin%
\pgfsetlinewidth{0.392427pt}%
\definecolor{currentstroke}{rgb}{0.280894,0.078907,0.402329}%
\pgfsetstrokecolor{currentstroke}%
\pgfsetdash{}{0pt}%
\pgfpathmoveto{\pgfqpoint{5.176292in}{2.301205in}}%
\pgfpathlineto{\pgfqpoint{5.130488in}{2.318857in}}%
\pgfusepath{stroke}%
\end{pgfscope}%
\begin{pgfscope}%
\pgfpathrectangle{\pgfqpoint{3.985294in}{1.750000in}}{\pgfqpoint{2.279412in}{2.004545in}}%
\pgfusepath{clip}%
\pgfsetbuttcap%
\pgfsetroundjoin%
\pgfsetlinewidth{0.421594pt}%
\definecolor{currentstroke}{rgb}{0.282656,0.100196,0.422160}%
\pgfsetstrokecolor{currentstroke}%
\pgfsetdash{}{0pt}%
\pgfpathmoveto{\pgfqpoint{5.130488in}{2.318857in}}%
\pgfpathlineto{\pgfqpoint{5.085967in}{2.338724in}}%
\pgfusepath{stroke}%
\end{pgfscope}%
\begin{pgfscope}%
\pgfpathrectangle{\pgfqpoint{3.985294in}{1.750000in}}{\pgfqpoint{2.279412in}{2.004545in}}%
\pgfusepath{clip}%
\pgfsetbuttcap%
\pgfsetroundjoin%
\pgfsetlinewidth{0.423166pt}%
\definecolor{currentstroke}{rgb}{0.282656,0.100196,0.422160}%
\pgfsetstrokecolor{currentstroke}%
\pgfsetdash{}{0pt}%
\pgfpathmoveto{\pgfqpoint{5.085967in}{2.338724in}}%
\pgfpathlineto{\pgfqpoint{5.043709in}{2.362327in}}%
\pgfusepath{stroke}%
\end{pgfscope}%
\begin{pgfscope}%
\pgfpathrectangle{\pgfqpoint{3.985294in}{1.750000in}}{\pgfqpoint{2.279412in}{2.004545in}}%
\pgfusepath{clip}%
\pgfsetbuttcap%
\pgfsetroundjoin%
\pgfsetlinewidth{0.496440pt}%
\definecolor{currentstroke}{rgb}{0.281412,0.155834,0.469201}%
\pgfsetstrokecolor{currentstroke}%
\pgfsetdash{}{0pt}%
\pgfpathmoveto{\pgfqpoint{5.043709in}{2.362327in}}%
\pgfpathlineto{\pgfqpoint{5.005819in}{2.390153in}}%
\pgfusepath{stroke}%
\end{pgfscope}%
\begin{pgfscope}%
\pgfpathrectangle{\pgfqpoint{3.985294in}{1.750000in}}{\pgfqpoint{2.279412in}{2.004545in}}%
\pgfusepath{clip}%
\pgfsetbuttcap%
\pgfsetroundjoin%
\pgfsetlinewidth{0.930796pt}%
\definecolor{currentstroke}{rgb}{0.180629,0.429975,0.557282}%
\pgfsetstrokecolor{currentstroke}%
\pgfsetdash{}{0pt}%
\pgfpathmoveto{\pgfqpoint{4.612081in}{2.977807in}}%
\pgfpathlineto{\pgfqpoint{4.657697in}{2.959846in}}%
\pgfusepath{stroke}%
\end{pgfscope}%
\begin{pgfscope}%
\pgfpathrectangle{\pgfqpoint{3.985294in}{1.750000in}}{\pgfqpoint{2.279412in}{2.004545in}}%
\pgfusepath{clip}%
\pgfsetbuttcap%
\pgfsetroundjoin%
\pgfsetlinewidth{0.805218pt}%
\definecolor{currentstroke}{rgb}{0.212395,0.359683,0.551710}%
\pgfsetstrokecolor{currentstroke}%
\pgfsetdash{}{0pt}%
\pgfpathmoveto{\pgfqpoint{4.657697in}{2.959846in}}%
\pgfpathlineto{\pgfqpoint{4.657697in}{2.959846in}}%
\pgfusepath{stroke}%
\end{pgfscope}%
\begin{pgfscope}%
\pgfpathrectangle{\pgfqpoint{3.985294in}{1.750000in}}{\pgfqpoint{2.279412in}{2.004545in}}%
\pgfusepath{clip}%
\pgfsetbuttcap%
\pgfsetroundjoin%
\pgfsetlinewidth{0.805218pt}%
\definecolor{currentstroke}{rgb}{0.212395,0.359683,0.551710}%
\pgfsetstrokecolor{currentstroke}%
\pgfsetdash{}{0pt}%
\pgfpathmoveto{\pgfqpoint{4.657697in}{2.959846in}}%
\pgfpathlineto{\pgfqpoint{4.686683in}{2.942966in}}%
\pgfusepath{stroke}%
\end{pgfscope}%
\begin{pgfscope}%
\pgfpathrectangle{\pgfqpoint{3.985294in}{1.750000in}}{\pgfqpoint{2.279412in}{2.004545in}}%
\pgfusepath{clip}%
\pgfsetbuttcap%
\pgfsetroundjoin%
\pgfsetlinewidth{0.899117pt}%
\definecolor{currentstroke}{rgb}{0.187231,0.414746,0.556547}%
\pgfsetstrokecolor{currentstroke}%
\pgfsetdash{}{0pt}%
\pgfpathmoveto{\pgfqpoint{4.686683in}{2.942966in}}%
\pgfpathlineto{\pgfqpoint{4.710448in}{2.922228in}}%
\pgfusepath{stroke}%
\end{pgfscope}%
\begin{pgfscope}%
\pgfpathrectangle{\pgfqpoint{3.985294in}{1.750000in}}{\pgfqpoint{2.279412in}{2.004545in}}%
\pgfusepath{clip}%
\pgfsetbuttcap%
\pgfsetroundjoin%
\pgfsetlinewidth{0.858781pt}%
\definecolor{currentstroke}{rgb}{0.197636,0.391528,0.554969}%
\pgfsetstrokecolor{currentstroke}%
\pgfsetdash{}{0pt}%
\pgfpathmoveto{\pgfqpoint{4.710448in}{2.922228in}}%
\pgfpathlineto{\pgfqpoint{4.739210in}{2.890053in}}%
\pgfusepath{stroke}%
\end{pgfscope}%
\begin{pgfscope}%
\pgfpathrectangle{\pgfqpoint{3.985294in}{1.750000in}}{\pgfqpoint{2.279412in}{2.004545in}}%
\pgfusepath{clip}%
\pgfsetbuttcap%
\pgfsetroundjoin%
\pgfsetlinewidth{0.704103pt}%
\definecolor{currentstroke}{rgb}{0.241237,0.296485,0.539709}%
\pgfsetstrokecolor{currentstroke}%
\pgfsetdash{}{0pt}%
\pgfpathmoveto{\pgfqpoint{4.739210in}{2.890053in}}%
\pgfpathlineto{\pgfqpoint{4.739210in}{2.890053in}}%
\pgfusepath{stroke}%
\end{pgfscope}%
\begin{pgfscope}%
\pgfpathrectangle{\pgfqpoint{3.985294in}{1.750000in}}{\pgfqpoint{2.279412in}{2.004545in}}%
\pgfusepath{clip}%
\pgfsetbuttcap%
\pgfsetroundjoin%
\pgfsetlinewidth{0.704103pt}%
\definecolor{currentstroke}{rgb}{0.241237,0.296485,0.539709}%
\pgfsetstrokecolor{currentstroke}%
\pgfsetdash{}{0pt}%
\pgfpathmoveto{\pgfqpoint{4.739210in}{2.890053in}}%
\pgfpathlineto{\pgfqpoint{4.746044in}{2.873189in}}%
\pgfusepath{stroke}%
\end{pgfscope}%
\begin{pgfscope}%
\pgfpathrectangle{\pgfqpoint{3.985294in}{1.750000in}}{\pgfqpoint{2.279412in}{2.004545in}}%
\pgfusepath{clip}%
\pgfsetbuttcap%
\pgfsetroundjoin%
\pgfsetlinewidth{0.770836pt}%
\definecolor{currentstroke}{rgb}{0.221989,0.339161,0.548752}%
\pgfsetstrokecolor{currentstroke}%
\pgfsetdash{}{0pt}%
\pgfpathmoveto{\pgfqpoint{4.746044in}{2.873189in}}%
\pgfpathlineto{\pgfqpoint{4.745555in}{2.856810in}}%
\pgfusepath{stroke}%
\end{pgfscope}%
\begin{pgfscope}%
\pgfpathrectangle{\pgfqpoint{3.985294in}{1.750000in}}{\pgfqpoint{2.279412in}{2.004545in}}%
\pgfusepath{clip}%
\pgfsetbuttcap%
\pgfsetroundjoin%
\pgfsetlinewidth{0.781214pt}%
\definecolor{currentstroke}{rgb}{0.220057,0.343307,0.549413}%
\pgfsetstrokecolor{currentstroke}%
\pgfsetdash{}{0pt}%
\pgfpathmoveto{\pgfqpoint{4.745555in}{2.856810in}}%
\pgfpathlineto{\pgfqpoint{4.739626in}{2.837916in}}%
\pgfusepath{stroke}%
\end{pgfscope}%
\begin{pgfscope}%
\pgfpathrectangle{\pgfqpoint{3.985294in}{1.750000in}}{\pgfqpoint{2.279412in}{2.004545in}}%
\pgfusepath{clip}%
\pgfsetroundcap%
\pgfsetroundjoin%
\pgfsetlinewidth{0.935027pt}%
\definecolor{currentstroke}{rgb}{0.179019,0.433756,0.557430}%
\pgfsetstrokecolor{currentstroke}%
\pgfsetdash{}{0pt}%
\pgfpathmoveto{\pgfqpoint{5.293965in}{2.793064in}}%
\pgfpathquadraticcurveto{\pgfqpoint{5.281432in}{2.792761in}}{\pgfqpoint{5.283360in}{2.792808in}}%
\pgfusepath{stroke}%
\end{pgfscope}%
\begin{pgfscope}%
\pgfpathrectangle{\pgfqpoint{3.985294in}{1.750000in}}{\pgfqpoint{2.279412in}{2.004545in}}%
\pgfusepath{clip}%
\pgfsetroundcap%
\pgfsetroundjoin%
\definecolor{currentfill}{rgb}{0.179019,0.433756,0.557430}%
\pgfsetfillcolor{currentfill}%
\pgfsetlinewidth{0.935027pt}%
\definecolor{currentstroke}{rgb}{0.179019,0.433756,0.557430}%
\pgfsetstrokecolor{currentstroke}%
\pgfsetdash{}{0pt}%
\pgfpathmoveto{\pgfqpoint{5.339572in}{2.766383in}}%
\pgfpathlineto{\pgfqpoint{5.283360in}{2.792808in}}%
\pgfpathlineto{\pgfqpoint{5.338226in}{2.821922in}}%
\pgfpathlineto{\pgfqpoint{5.339572in}{2.766383in}}%
\pgfpathlineto{\pgfqpoint{5.339572in}{2.766383in}}%
\pgfpathclose%
\pgfusepath{stroke,fill}%
\end{pgfscope}%
\begin{pgfscope}%
\pgfpathrectangle{\pgfqpoint{3.985294in}{1.750000in}}{\pgfqpoint{2.279412in}{2.004545in}}%
\pgfusepath{clip}%
\pgfsetroundcap%
\pgfsetroundjoin%
\pgfsetlinewidth{0.498873pt}%
\definecolor{currentstroke}{rgb}{0.281412,0.155834,0.469201}%
\pgfsetstrokecolor{currentstroke}%
\pgfsetdash{}{0pt}%
\pgfpathmoveto{\pgfqpoint{5.526876in}{2.842356in}}%
\pgfpathquadraticcurveto{\pgfqpoint{5.514340in}{2.842191in}}{\pgfqpoint{5.509520in}{2.842128in}}%
\pgfusepath{stroke}%
\end{pgfscope}%
\begin{pgfscope}%
\pgfpathrectangle{\pgfqpoint{3.985294in}{1.750000in}}{\pgfqpoint{2.279412in}{2.004545in}}%
\pgfusepath{clip}%
\pgfsetroundcap%
\pgfsetroundjoin%
\definecolor{currentfill}{rgb}{0.281412,0.155834,0.469201}%
\pgfsetfillcolor{currentfill}%
\pgfsetlinewidth{0.498873pt}%
\definecolor{currentstroke}{rgb}{0.281412,0.155834,0.469201}%
\pgfsetstrokecolor{currentstroke}%
\pgfsetdash{}{0pt}%
\pgfpathmoveto{\pgfqpoint{5.565437in}{2.815083in}}%
\pgfpathlineto{\pgfqpoint{5.509520in}{2.842128in}}%
\pgfpathlineto{\pgfqpoint{5.564706in}{2.870634in}}%
\pgfpathlineto{\pgfqpoint{5.565437in}{2.815083in}}%
\pgfpathlineto{\pgfqpoint{5.565437in}{2.815083in}}%
\pgfpathclose%
\pgfusepath{stroke,fill}%
\end{pgfscope}%
\begin{pgfscope}%
\pgfpathrectangle{\pgfqpoint{3.985294in}{1.750000in}}{\pgfqpoint{2.279412in}{2.004545in}}%
\pgfusepath{clip}%
\pgfsetroundcap%
\pgfsetroundjoin%
\pgfsetlinewidth{0.746195pt}%
\definecolor{currentstroke}{rgb}{0.229739,0.322361,0.545706}%
\pgfsetstrokecolor{currentstroke}%
\pgfsetdash{}{0pt}%
\pgfpathmoveto{\pgfqpoint{5.300121in}{2.924508in}}%
\pgfpathquadraticcurveto{\pgfqpoint{5.287625in}{2.923619in}}{\pgfqpoint{5.286644in}{2.923549in}}%
\pgfusepath{stroke}%
\end{pgfscope}%
\begin{pgfscope}%
\pgfpathrectangle{\pgfqpoint{3.985294in}{1.750000in}}{\pgfqpoint{2.279412in}{2.004545in}}%
\pgfusepath{clip}%
\pgfsetroundcap%
\pgfsetroundjoin%
\definecolor{currentfill}{rgb}{0.229739,0.322361,0.545706}%
\pgfsetfillcolor{currentfill}%
\pgfsetlinewidth{0.746195pt}%
\definecolor{currentstroke}{rgb}{0.229739,0.322361,0.545706}%
\pgfsetstrokecolor{currentstroke}%
\pgfsetdash{}{0pt}%
\pgfpathmoveto{\pgfqpoint{5.344032in}{2.899787in}}%
\pgfpathlineto{\pgfqpoint{5.286644in}{2.923549in}}%
\pgfpathlineto{\pgfqpoint{5.340087in}{2.955202in}}%
\pgfpathlineto{\pgfqpoint{5.344032in}{2.899787in}}%
\pgfpathlineto{\pgfqpoint{5.344032in}{2.899787in}}%
\pgfpathclose%
\pgfusepath{stroke,fill}%
\end{pgfscope}%
\begin{pgfscope}%
\pgfpathrectangle{\pgfqpoint{3.985294in}{1.750000in}}{\pgfqpoint{2.279412in}{2.004545in}}%
\pgfusepath{clip}%
\pgfsetroundcap%
\pgfsetroundjoin%
\pgfsetlinewidth{0.331335pt}%
\definecolor{currentstroke}{rgb}{0.272594,0.025563,0.353093}%
\pgfsetstrokecolor{currentstroke}%
\pgfsetdash{}{0pt}%
\pgfpathmoveto{\pgfqpoint{5.322332in}{2.048726in}}%
\pgfpathquadraticcurveto{\pgfqpoint{5.309914in}{2.049858in}}{\pgfqpoint{5.302601in}{2.050525in}}%
\pgfusepath{stroke}%
\end{pgfscope}%
\begin{pgfscope}%
\pgfpathrectangle{\pgfqpoint{3.985294in}{1.750000in}}{\pgfqpoint{2.279412in}{2.004545in}}%
\pgfusepath{clip}%
\pgfsetroundcap%
\pgfsetroundjoin%
\definecolor{currentfill}{rgb}{0.272594,0.025563,0.353093}%
\pgfsetfillcolor{currentfill}%
\pgfsetlinewidth{0.331335pt}%
\definecolor{currentstroke}{rgb}{0.272594,0.025563,0.353093}%
\pgfsetstrokecolor{currentstroke}%
\pgfsetdash{}{0pt}%
\pgfpathmoveto{\pgfqpoint{5.355405in}{2.017818in}}%
\pgfpathlineto{\pgfqpoint{5.302601in}{2.050525in}}%
\pgfpathlineto{\pgfqpoint{5.360449in}{2.073144in}}%
\pgfpathlineto{\pgfqpoint{5.355405in}{2.017818in}}%
\pgfpathlineto{\pgfqpoint{5.355405in}{2.017818in}}%
\pgfpathclose%
\pgfusepath{stroke,fill}%
\end{pgfscope}%
\begin{pgfscope}%
\pgfpathrectangle{\pgfqpoint{3.985294in}{1.750000in}}{\pgfqpoint{2.279412in}{2.004545in}}%
\pgfusepath{clip}%
\pgfsetroundcap%
\pgfsetroundjoin%
\pgfsetlinewidth{0.550522pt}%
\definecolor{currentstroke}{rgb}{0.275191,0.194905,0.496005}%
\pgfsetstrokecolor{currentstroke}%
\pgfsetdash{}{0pt}%
\pgfpathmoveto{\pgfqpoint{5.056514in}{2.393137in}}%
\pgfpathquadraticcurveto{\pgfqpoint{5.047015in}{2.399248in}}{\pgfqpoint{5.044679in}{2.400751in}}%
\pgfusepath{stroke}%
\end{pgfscope}%
\begin{pgfscope}%
\pgfpathrectangle{\pgfqpoint{3.985294in}{1.750000in}}{\pgfqpoint{2.279412in}{2.004545in}}%
\pgfusepath{clip}%
\pgfsetroundcap%
\pgfsetroundjoin%
\definecolor{currentfill}{rgb}{0.275191,0.194905,0.496005}%
\pgfsetfillcolor{currentfill}%
\pgfsetlinewidth{0.550522pt}%
\definecolor{currentstroke}{rgb}{0.275191,0.194905,0.496005}%
\pgfsetstrokecolor{currentstroke}%
\pgfsetdash{}{0pt}%
\pgfpathmoveto{\pgfqpoint{5.076372in}{2.347332in}}%
\pgfpathlineto{\pgfqpoint{5.044679in}{2.400751in}}%
\pgfpathlineto{\pgfqpoint{5.106430in}{2.394055in}}%
\pgfpathlineto{\pgfqpoint{5.076372in}{2.347332in}}%
\pgfpathlineto{\pgfqpoint{5.076372in}{2.347332in}}%
\pgfpathclose%
\pgfusepath{stroke,fill}%
\end{pgfscope}%
\begin{pgfscope}%
\pgfpathrectangle{\pgfqpoint{3.985294in}{1.750000in}}{\pgfqpoint{2.279412in}{2.004545in}}%
\pgfusepath{clip}%
\pgfsetroundcap%
\pgfsetroundjoin%
\pgfsetlinewidth{0.380754pt}%
\definecolor{currentstroke}{rgb}{0.279566,0.067836,0.391917}%
\pgfsetstrokecolor{currentstroke}%
\pgfsetdash{}{0pt}%
\pgfpathmoveto{\pgfqpoint{5.543776in}{2.353611in}}%
\pgfpathquadraticcurveto{\pgfqpoint{5.531257in}{2.354205in}}{\pgfqpoint{5.524623in}{2.354520in}}%
\pgfusepath{stroke}%
\end{pgfscope}%
\begin{pgfscope}%
\pgfpathrectangle{\pgfqpoint{3.985294in}{1.750000in}}{\pgfqpoint{2.279412in}{2.004545in}}%
\pgfusepath{clip}%
\pgfsetroundcap%
\pgfsetroundjoin%
\definecolor{currentfill}{rgb}{0.279566,0.067836,0.391917}%
\pgfsetfillcolor{currentfill}%
\pgfsetlinewidth{0.380754pt}%
\definecolor{currentstroke}{rgb}{0.279566,0.067836,0.391917}%
\pgfsetstrokecolor{currentstroke}%
\pgfsetdash{}{0pt}%
\pgfpathmoveto{\pgfqpoint{5.578799in}{2.324140in}}%
\pgfpathlineto{\pgfqpoint{5.524623in}{2.354520in}}%
\pgfpathlineto{\pgfqpoint{5.581433in}{2.379633in}}%
\pgfpathlineto{\pgfqpoint{5.578799in}{2.324140in}}%
\pgfpathlineto{\pgfqpoint{5.578799in}{2.324140in}}%
\pgfpathclose%
\pgfusepath{stroke,fill}%
\end{pgfscope}%
\begin{pgfscope}%
\pgfpathrectangle{\pgfqpoint{3.985294in}{1.750000in}}{\pgfqpoint{2.279412in}{2.004545in}}%
\pgfusepath{clip}%
\pgfsetroundcap%
\pgfsetroundjoin%
\pgfsetlinewidth{0.488200pt}%
\definecolor{currentstroke}{rgb}{0.281887,0.150881,0.465405}%
\pgfsetstrokecolor{currentstroke}%
\pgfsetdash{}{0pt}%
\pgfpathmoveto{\pgfqpoint{5.394017in}{2.410836in}}%
\pgfpathquadraticcurveto{\pgfqpoint{5.381548in}{2.411966in}}{\pgfqpoint{5.376601in}{2.412414in}}%
\pgfusepath{stroke}%
\end{pgfscope}%
\begin{pgfscope}%
\pgfpathrectangle{\pgfqpoint{3.985294in}{1.750000in}}{\pgfqpoint{2.279412in}{2.004545in}}%
\pgfusepath{clip}%
\pgfsetroundcap%
\pgfsetroundjoin%
\definecolor{currentfill}{rgb}{0.281887,0.150881,0.465405}%
\pgfsetfillcolor{currentfill}%
\pgfsetlinewidth{0.488200pt}%
\definecolor{currentstroke}{rgb}{0.281887,0.150881,0.465405}%
\pgfsetstrokecolor{currentstroke}%
\pgfsetdash{}{0pt}%
\pgfpathmoveto{\pgfqpoint{5.429424in}{2.379737in}}%
\pgfpathlineto{\pgfqpoint{5.376601in}{2.412414in}}%
\pgfpathlineto{\pgfqpoint{5.434436in}{2.435066in}}%
\pgfpathlineto{\pgfqpoint{5.429424in}{2.379737in}}%
\pgfpathlineto{\pgfqpoint{5.429424in}{2.379737in}}%
\pgfpathclose%
\pgfusepath{stroke,fill}%
\end{pgfscope}%
\begin{pgfscope}%
\pgfpathrectangle{\pgfqpoint{3.985294in}{1.750000in}}{\pgfqpoint{2.279412in}{2.004545in}}%
\pgfusepath{clip}%
\pgfsetroundcap%
\pgfsetroundjoin%
\pgfsetlinewidth{0.840385pt}%
\definecolor{currentstroke}{rgb}{0.203063,0.379716,0.553925}%
\pgfsetstrokecolor{currentstroke}%
\pgfsetdash{}{0pt}%
\pgfpathmoveto{\pgfqpoint{5.294802in}{2.589195in}}%
\pgfpathquadraticcurveto{\pgfqpoint{5.282301in}{2.590037in}}{\pgfqpoint{5.282771in}{2.590005in}}%
\pgfusepath{stroke}%
\end{pgfscope}%
\begin{pgfscope}%
\pgfpathrectangle{\pgfqpoint{3.985294in}{1.750000in}}{\pgfqpoint{2.279412in}{2.004545in}}%
\pgfusepath{clip}%
\pgfsetroundcap%
\pgfsetroundjoin%
\definecolor{currentfill}{rgb}{0.203063,0.379716,0.553925}%
\pgfsetfillcolor{currentfill}%
\pgfsetlinewidth{0.840385pt}%
\definecolor{currentstroke}{rgb}{0.203063,0.379716,0.553925}%
\pgfsetstrokecolor{currentstroke}%
\pgfsetdash{}{0pt}%
\pgfpathmoveto{\pgfqpoint{5.336335in}{2.558558in}}%
\pgfpathlineto{\pgfqpoint{5.282771in}{2.590005in}}%
\pgfpathlineto{\pgfqpoint{5.340068in}{2.613988in}}%
\pgfpathlineto{\pgfqpoint{5.336335in}{2.558558in}}%
\pgfpathlineto{\pgfqpoint{5.336335in}{2.558558in}}%
\pgfpathclose%
\pgfusepath{stroke,fill}%
\end{pgfscope}%
\begin{pgfscope}%
\pgfpathrectangle{\pgfqpoint{3.985294in}{1.750000in}}{\pgfqpoint{2.279412in}{2.004545in}}%
\pgfusepath{clip}%
\pgfsetroundcap%
\pgfsetroundjoin%
\pgfsetlinewidth{0.424067pt}%
\definecolor{currentstroke}{rgb}{0.282656,0.100196,0.422160}%
\pgfsetstrokecolor{currentstroke}%
\pgfsetdash{}{0pt}%
\pgfpathmoveto{\pgfqpoint{5.593571in}{2.617655in}}%
\pgfpathquadraticcurveto{\pgfqpoint{5.581035in}{2.617814in}}{\pgfqpoint{5.575059in}{2.617891in}}%
\pgfusepath{stroke}%
\end{pgfscope}%
\begin{pgfscope}%
\pgfpathrectangle{\pgfqpoint{3.985294in}{1.750000in}}{\pgfqpoint{2.279412in}{2.004545in}}%
\pgfusepath{clip}%
\pgfsetroundcap%
\pgfsetroundjoin%
\definecolor{currentfill}{rgb}{0.282656,0.100196,0.422160}%
\pgfsetfillcolor{currentfill}%
\pgfsetlinewidth{0.424067pt}%
\definecolor{currentstroke}{rgb}{0.282656,0.100196,0.422160}%
\pgfsetstrokecolor{currentstroke}%
\pgfsetdash{}{0pt}%
\pgfpathmoveto{\pgfqpoint{5.630256in}{2.589407in}}%
\pgfpathlineto{\pgfqpoint{5.575059in}{2.617891in}}%
\pgfpathlineto{\pgfqpoint{5.630964in}{2.644958in}}%
\pgfpathlineto{\pgfqpoint{5.630256in}{2.589407in}}%
\pgfpathlineto{\pgfqpoint{5.630256in}{2.589407in}}%
\pgfpathclose%
\pgfusepath{stroke,fill}%
\end{pgfscope}%
\begin{pgfscope}%
\pgfpathrectangle{\pgfqpoint{3.985294in}{1.750000in}}{\pgfqpoint{2.279412in}{2.004545in}}%
\pgfusepath{clip}%
\pgfsetroundcap%
\pgfsetroundjoin%
\pgfsetlinewidth{0.885654pt}%
\definecolor{currentstroke}{rgb}{0.190631,0.407061,0.556089}%
\pgfsetstrokecolor{currentstroke}%
\pgfsetdash{}{0pt}%
\pgfpathmoveto{\pgfqpoint{5.344859in}{2.706719in}}%
\pgfpathquadraticcurveto{\pgfqpoint{5.332321in}{2.706767in}}{\pgfqpoint{5.333485in}{2.706762in}}%
\pgfusepath{stroke}%
\end{pgfscope}%
\begin{pgfscope}%
\pgfpathrectangle{\pgfqpoint{3.985294in}{1.750000in}}{\pgfqpoint{2.279412in}{2.004545in}}%
\pgfusepath{clip}%
\pgfsetroundcap%
\pgfsetroundjoin%
\definecolor{currentfill}{rgb}{0.190631,0.407061,0.556089}%
\pgfsetfillcolor{currentfill}%
\pgfsetlinewidth{0.885654pt}%
\definecolor{currentstroke}{rgb}{0.190631,0.407061,0.556089}%
\pgfsetstrokecolor{currentstroke}%
\pgfsetdash{}{0pt}%
\pgfpathmoveto{\pgfqpoint{5.388935in}{2.678774in}}%
\pgfpathlineto{\pgfqpoint{5.333485in}{2.706762in}}%
\pgfpathlineto{\pgfqpoint{5.389145in}{2.734329in}}%
\pgfpathlineto{\pgfqpoint{5.388935in}{2.678774in}}%
\pgfpathlineto{\pgfqpoint{5.388935in}{2.678774in}}%
\pgfpathclose%
\pgfusepath{stroke,fill}%
\end{pgfscope}%
\begin{pgfscope}%
\pgfpathrectangle{\pgfqpoint{3.985294in}{1.750000in}}{\pgfqpoint{2.279412in}{2.004545in}}%
\pgfusepath{clip}%
\pgfsetroundcap%
\pgfsetroundjoin%
\pgfsetlinewidth{0.658390pt}%
\definecolor{currentstroke}{rgb}{0.252194,0.269783,0.531579}%
\pgfsetstrokecolor{currentstroke}%
\pgfsetdash{}{0pt}%
\pgfpathmoveto{\pgfqpoint{5.443051in}{2.750661in}}%
\pgfpathquadraticcurveto{\pgfqpoint{5.430513in}{2.750616in}}{\pgfqpoint{5.428160in}{2.750607in}}%
\pgfusepath{stroke}%
\end{pgfscope}%
\begin{pgfscope}%
\pgfpathrectangle{\pgfqpoint{3.985294in}{1.750000in}}{\pgfqpoint{2.279412in}{2.004545in}}%
\pgfusepath{clip}%
\pgfsetroundcap%
\pgfsetroundjoin%
\definecolor{currentfill}{rgb}{0.252194,0.269783,0.531579}%
\pgfsetfillcolor{currentfill}%
\pgfsetlinewidth{0.658390pt}%
\definecolor{currentstroke}{rgb}{0.252194,0.269783,0.531579}%
\pgfsetstrokecolor{currentstroke}%
\pgfsetdash{}{0pt}%
\pgfpathmoveto{\pgfqpoint{5.483816in}{2.723030in}}%
\pgfpathlineto{\pgfqpoint{5.428160in}{2.750607in}}%
\pgfpathlineto{\pgfqpoint{5.483615in}{2.778585in}}%
\pgfpathlineto{\pgfqpoint{5.483816in}{2.723030in}}%
\pgfpathlineto{\pgfqpoint{5.483816in}{2.723030in}}%
\pgfpathclose%
\pgfusepath{stroke,fill}%
\end{pgfscope}%
\begin{pgfscope}%
\pgfpathrectangle{\pgfqpoint{3.985294in}{1.750000in}}{\pgfqpoint{2.279412in}{2.004545in}}%
\pgfusepath{clip}%
\pgfsetroundcap%
\pgfsetroundjoin%
\pgfsetlinewidth{0.539557pt}%
\definecolor{currentstroke}{rgb}{0.277134,0.185228,0.489898}%
\pgfsetstrokecolor{currentstroke}%
\pgfsetdash{}{0pt}%
\pgfpathmoveto{\pgfqpoint{5.294152in}{3.042877in}}%
\pgfpathquadraticcurveto{\pgfqpoint{5.281776in}{3.041127in}}{\pgfqpoint{5.277664in}{3.040546in}}%
\pgfusepath{stroke}%
\end{pgfscope}%
\begin{pgfscope}%
\pgfpathrectangle{\pgfqpoint{3.985294in}{1.750000in}}{\pgfqpoint{2.279412in}{2.004545in}}%
\pgfusepath{clip}%
\pgfsetroundcap%
\pgfsetroundjoin%
\definecolor{currentfill}{rgb}{0.277134,0.185228,0.489898}%
\pgfsetfillcolor{currentfill}%
\pgfsetlinewidth{0.539557pt}%
\definecolor{currentstroke}{rgb}{0.277134,0.185228,0.489898}%
\pgfsetstrokecolor{currentstroke}%
\pgfsetdash{}{0pt}%
\pgfpathmoveto{\pgfqpoint{5.336560in}{3.020816in}}%
\pgfpathlineto{\pgfqpoint{5.277664in}{3.040546in}}%
\pgfpathlineto{\pgfqpoint{5.328786in}{3.075825in}}%
\pgfpathlineto{\pgfqpoint{5.336560in}{3.020816in}}%
\pgfpathlineto{\pgfqpoint{5.336560in}{3.020816in}}%
\pgfpathclose%
\pgfusepath{stroke,fill}%
\end{pgfscope}%
\begin{pgfscope}%
\pgfpathrectangle{\pgfqpoint{3.985294in}{1.750000in}}{\pgfqpoint{2.279412in}{2.004545in}}%
\pgfusepath{clip}%
\pgfsetroundcap%
\pgfsetroundjoin%
\pgfsetlinewidth{0.391256pt}%
\definecolor{currentstroke}{rgb}{0.280894,0.078907,0.402329}%
\pgfsetstrokecolor{currentstroke}%
\pgfsetdash{}{0pt}%
\pgfpathmoveto{\pgfqpoint{5.543859in}{3.108163in}}%
\pgfpathquadraticcurveto{\pgfqpoint{5.531343in}{3.107517in}}{\pgfqpoint{5.524873in}{3.107184in}}%
\pgfusepath{stroke}%
\end{pgfscope}%
\begin{pgfscope}%
\pgfpathrectangle{\pgfqpoint{3.985294in}{1.750000in}}{\pgfqpoint{2.279412in}{2.004545in}}%
\pgfusepath{clip}%
\pgfsetroundcap%
\pgfsetroundjoin%
\definecolor{currentfill}{rgb}{0.280894,0.078907,0.402329}%
\pgfsetfillcolor{currentfill}%
\pgfsetlinewidth{0.391256pt}%
\definecolor{currentstroke}{rgb}{0.280894,0.078907,0.402329}%
\pgfsetstrokecolor{currentstroke}%
\pgfsetdash{}{0pt}%
\pgfpathmoveto{\pgfqpoint{5.581785in}{3.082304in}}%
\pgfpathlineto{\pgfqpoint{5.524873in}{3.107184in}}%
\pgfpathlineto{\pgfqpoint{5.578924in}{3.137786in}}%
\pgfpathlineto{\pgfqpoint{5.581785in}{3.082304in}}%
\pgfpathlineto{\pgfqpoint{5.581785in}{3.082304in}}%
\pgfpathclose%
\pgfusepath{stroke,fill}%
\end{pgfscope}%
\begin{pgfscope}%
\pgfpathrectangle{\pgfqpoint{3.985294in}{1.750000in}}{\pgfqpoint{2.279412in}{2.004545in}}%
\pgfusepath{clip}%
\pgfsetroundcap%
\pgfsetroundjoin%
\pgfsetlinewidth{0.314650pt}%
\definecolor{currentstroke}{rgb}{0.268510,0.009605,0.335427}%
\pgfsetstrokecolor{currentstroke}%
\pgfsetdash{}{0pt}%
\pgfpathmoveto{\pgfqpoint{5.743331in}{3.425848in}}%
\pgfpathquadraticcurveto{\pgfqpoint{5.730951in}{3.424521in}}{\pgfqpoint{5.723411in}{3.423714in}}%
\pgfusepath{stroke}%
\end{pgfscope}%
\begin{pgfscope}%
\pgfpathrectangle{\pgfqpoint{3.985294in}{1.750000in}}{\pgfqpoint{2.279412in}{2.004545in}}%
\pgfusepath{clip}%
\pgfsetroundcap%
\pgfsetroundjoin%
\definecolor{currentfill}{rgb}{0.268510,0.009605,0.335427}%
\pgfsetfillcolor{currentfill}%
\pgfsetlinewidth{0.314650pt}%
\definecolor{currentstroke}{rgb}{0.268510,0.009605,0.335427}%
\pgfsetstrokecolor{currentstroke}%
\pgfsetdash{}{0pt}%
\pgfpathmoveto{\pgfqpoint{5.781609in}{3.402011in}}%
\pgfpathlineto{\pgfqpoint{5.723411in}{3.423714in}}%
\pgfpathlineto{\pgfqpoint{5.775692in}{3.457250in}}%
\pgfpathlineto{\pgfqpoint{5.781609in}{3.402011in}}%
\pgfpathlineto{\pgfqpoint{5.781609in}{3.402011in}}%
\pgfpathclose%
\pgfusepath{stroke,fill}%
\end{pgfscope}%
\begin{pgfscope}%
\pgfpathrectangle{\pgfqpoint{3.985294in}{1.750000in}}{\pgfqpoint{2.279412in}{2.004545in}}%
\pgfusepath{clip}%
\pgfsetroundcap%
\pgfsetroundjoin%
\pgfsetlinewidth{0.356956pt}%
\definecolor{currentstroke}{rgb}{0.277018,0.050344,0.375715}%
\pgfsetstrokecolor{currentstroke}%
\pgfsetdash{}{0pt}%
\pgfpathmoveto{\pgfqpoint{5.646539in}{2.434918in}}%
\pgfpathquadraticcurveto{\pgfqpoint{5.634006in}{2.435073in}}{\pgfqpoint{5.626995in}{2.435160in}}%
\pgfusepath{stroke}%
\end{pgfscope}%
\begin{pgfscope}%
\pgfpathrectangle{\pgfqpoint{3.985294in}{1.750000in}}{\pgfqpoint{2.279412in}{2.004545in}}%
\pgfusepath{clip}%
\pgfsetroundcap%
\pgfsetroundjoin%
\definecolor{currentfill}{rgb}{0.277018,0.050344,0.375715}%
\pgfsetfillcolor{currentfill}%
\pgfsetlinewidth{0.356956pt}%
\definecolor{currentstroke}{rgb}{0.277018,0.050344,0.375715}%
\pgfsetstrokecolor{currentstroke}%
\pgfsetdash{}{0pt}%
\pgfpathmoveto{\pgfqpoint{5.682202in}{2.406696in}}%
\pgfpathlineto{\pgfqpoint{5.626995in}{2.435160in}}%
\pgfpathlineto{\pgfqpoint{5.682891in}{2.462247in}}%
\pgfpathlineto{\pgfqpoint{5.682202in}{2.406696in}}%
\pgfpathlineto{\pgfqpoint{5.682202in}{2.406696in}}%
\pgfpathclose%
\pgfusepath{stroke,fill}%
\end{pgfscope}%
\begin{pgfscope}%
\pgfpathrectangle{\pgfqpoint{3.985294in}{1.750000in}}{\pgfqpoint{2.279412in}{2.004545in}}%
\pgfusepath{clip}%
\pgfsetroundcap%
\pgfsetroundjoin%
\pgfsetlinewidth{0.642925pt}%
\definecolor{currentstroke}{rgb}{0.257322,0.256130,0.526563}%
\pgfsetstrokecolor{currentstroke}%
\pgfsetdash{}{0pt}%
\pgfpathmoveto{\pgfqpoint{5.342087in}{2.496580in}}%
\pgfpathquadraticcurveto{\pgfqpoint{5.329617in}{2.497715in}}{\pgfqpoint{5.327052in}{2.497949in}}%
\pgfusepath{stroke}%
\end{pgfscope}%
\begin{pgfscope}%
\pgfpathrectangle{\pgfqpoint{3.985294in}{1.750000in}}{\pgfqpoint{2.279412in}{2.004545in}}%
\pgfusepath{clip}%
\pgfsetroundcap%
\pgfsetroundjoin%
\definecolor{currentfill}{rgb}{0.257322,0.256130,0.526563}%
\pgfsetfillcolor{currentfill}%
\pgfsetlinewidth{0.642925pt}%
\definecolor{currentstroke}{rgb}{0.257322,0.256130,0.526563}%
\pgfsetstrokecolor{currentstroke}%
\pgfsetdash{}{0pt}%
\pgfpathmoveto{\pgfqpoint{5.379860in}{2.465248in}}%
\pgfpathlineto{\pgfqpoint{5.327052in}{2.497949in}}%
\pgfpathlineto{\pgfqpoint{5.384897in}{2.520575in}}%
\pgfpathlineto{\pgfqpoint{5.379860in}{2.465248in}}%
\pgfpathlineto{\pgfqpoint{5.379860in}{2.465248in}}%
\pgfpathclose%
\pgfusepath{stroke,fill}%
\end{pgfscope}%
\begin{pgfscope}%
\pgfpathrectangle{\pgfqpoint{3.985294in}{1.750000in}}{\pgfqpoint{2.279412in}{2.004545in}}%
\pgfusepath{clip}%
\pgfsetroundcap%
\pgfsetroundjoin%
\pgfsetlinewidth{0.675207pt}%
\definecolor{currentstroke}{rgb}{0.248629,0.278775,0.534556}%
\pgfsetstrokecolor{currentstroke}%
\pgfsetdash{}{0pt}%
\pgfpathmoveto{\pgfqpoint{5.441966in}{2.663778in}}%
\pgfpathquadraticcurveto{\pgfqpoint{5.429429in}{2.663895in}}{\pgfqpoint{5.427337in}{2.663915in}}%
\pgfusepath{stroke}%
\end{pgfscope}%
\begin{pgfscope}%
\pgfpathrectangle{\pgfqpoint{3.985294in}{1.750000in}}{\pgfqpoint{2.279412in}{2.004545in}}%
\pgfusepath{clip}%
\pgfsetroundcap%
\pgfsetroundjoin%
\definecolor{currentfill}{rgb}{0.248629,0.278775,0.534556}%
\pgfsetfillcolor{currentfill}%
\pgfsetlinewidth{0.675207pt}%
\definecolor{currentstroke}{rgb}{0.248629,0.278775,0.534556}%
\pgfsetstrokecolor{currentstroke}%
\pgfsetdash{}{0pt}%
\pgfpathmoveto{\pgfqpoint{5.482630in}{2.635618in}}%
\pgfpathlineto{\pgfqpoint{5.427337in}{2.663915in}}%
\pgfpathlineto{\pgfqpoint{5.483150in}{2.691171in}}%
\pgfpathlineto{\pgfqpoint{5.482630in}{2.635618in}}%
\pgfpathlineto{\pgfqpoint{5.482630in}{2.635618in}}%
\pgfpathclose%
\pgfusepath{stroke,fill}%
\end{pgfscope}%
\begin{pgfscope}%
\pgfpathrectangle{\pgfqpoint{3.985294in}{1.750000in}}{\pgfqpoint{2.279412in}{2.004545in}}%
\pgfusepath{clip}%
\pgfsetroundcap%
\pgfsetroundjoin%
\pgfsetlinewidth{0.464085pt}%
\definecolor{currentstroke}{rgb}{0.283072,0.130895,0.449241}%
\pgfsetstrokecolor{currentstroke}%
\pgfsetdash{}{0pt}%
\pgfpathmoveto{\pgfqpoint{5.542235in}{2.887284in}}%
\pgfpathquadraticcurveto{\pgfqpoint{5.529700in}{2.887074in}}{\pgfqpoint{5.524343in}{2.886985in}}%
\pgfusepath{stroke}%
\end{pgfscope}%
\begin{pgfscope}%
\pgfpathrectangle{\pgfqpoint{3.985294in}{1.750000in}}{\pgfqpoint{2.279412in}{2.004545in}}%
\pgfusepath{clip}%
\pgfsetroundcap%
\pgfsetroundjoin%
\definecolor{currentfill}{rgb}{0.283072,0.130895,0.449241}%
\pgfsetfillcolor{currentfill}%
\pgfsetlinewidth{0.464085pt}%
\definecolor{currentstroke}{rgb}{0.283072,0.130895,0.449241}%
\pgfsetstrokecolor{currentstroke}%
\pgfsetdash{}{0pt}%
\pgfpathmoveto{\pgfqpoint{5.580355in}{2.860140in}}%
\pgfpathlineto{\pgfqpoint{5.524343in}{2.886985in}}%
\pgfpathlineto{\pgfqpoint{5.579426in}{2.915688in}}%
\pgfpathlineto{\pgfqpoint{5.580355in}{2.860140in}}%
\pgfpathlineto{\pgfqpoint{5.580355in}{2.860140in}}%
\pgfpathclose%
\pgfusepath{stroke,fill}%
\end{pgfscope}%
\begin{pgfscope}%
\pgfpathrectangle{\pgfqpoint{3.985294in}{1.750000in}}{\pgfqpoint{2.279412in}{2.004545in}}%
\pgfusepath{clip}%
\pgfsetroundcap%
\pgfsetroundjoin%
\pgfsetlinewidth{0.391561pt}%
\definecolor{currentstroke}{rgb}{0.280894,0.078907,0.402329}%
\pgfsetstrokecolor{currentstroke}%
\pgfsetdash{}{0pt}%
\pgfpathmoveto{\pgfqpoint{5.592421in}{2.974043in}}%
\pgfpathquadraticcurveto{\pgfqpoint{5.579887in}{2.973751in}}{\pgfqpoint{5.573410in}{2.973600in}}%
\pgfusepath{stroke}%
\end{pgfscope}%
\begin{pgfscope}%
\pgfpathrectangle{\pgfqpoint{3.985294in}{1.750000in}}{\pgfqpoint{2.279412in}{2.004545in}}%
\pgfusepath{clip}%
\pgfsetroundcap%
\pgfsetroundjoin%
\definecolor{currentfill}{rgb}{0.280894,0.078907,0.402329}%
\pgfsetfillcolor{currentfill}%
\pgfsetlinewidth{0.391561pt}%
\definecolor{currentstroke}{rgb}{0.280894,0.078907,0.402329}%
\pgfsetstrokecolor{currentstroke}%
\pgfsetdash{}{0pt}%
\pgfpathmoveto{\pgfqpoint{5.629598in}{2.947125in}}%
\pgfpathlineto{\pgfqpoint{5.573410in}{2.973600in}}%
\pgfpathlineto{\pgfqpoint{5.628302in}{3.002666in}}%
\pgfpathlineto{\pgfqpoint{5.629598in}{2.947125in}}%
\pgfpathlineto{\pgfqpoint{5.629598in}{2.947125in}}%
\pgfpathclose%
\pgfusepath{stroke,fill}%
\end{pgfscope}%
\begin{pgfscope}%
\pgfpathrectangle{\pgfqpoint{3.985294in}{1.750000in}}{\pgfqpoint{2.279412in}{2.004545in}}%
\pgfusepath{clip}%
\pgfsetroundcap%
\pgfsetroundjoin%
\pgfsetlinewidth{0.387876pt}%
\definecolor{currentstroke}{rgb}{0.280267,0.073417,0.397163}%
\pgfsetstrokecolor{currentstroke}%
\pgfsetdash{}{0pt}%
\pgfpathmoveto{\pgfqpoint{5.592552in}{3.016668in}}%
\pgfpathquadraticcurveto{\pgfqpoint{5.580024in}{3.016251in}}{\pgfqpoint{5.573492in}{3.016033in}}%
\pgfusepath{stroke}%
\end{pgfscope}%
\begin{pgfscope}%
\pgfpathrectangle{\pgfqpoint{3.985294in}{1.750000in}}{\pgfqpoint{2.279412in}{2.004545in}}%
\pgfusepath{clip}%
\pgfsetroundcap%
\pgfsetroundjoin%
\definecolor{currentfill}{rgb}{0.280267,0.073417,0.397163}%
\pgfsetfillcolor{currentfill}%
\pgfsetlinewidth{0.387876pt}%
\definecolor{currentstroke}{rgb}{0.280267,0.073417,0.397163}%
\pgfsetstrokecolor{currentstroke}%
\pgfsetdash{}{0pt}%
\pgfpathmoveto{\pgfqpoint{5.629942in}{2.990120in}}%
\pgfpathlineto{\pgfqpoint{5.573492in}{3.016033in}}%
\pgfpathlineto{\pgfqpoint{5.628092in}{3.045645in}}%
\pgfpathlineto{\pgfqpoint{5.629942in}{2.990120in}}%
\pgfpathlineto{\pgfqpoint{5.629942in}{2.990120in}}%
\pgfpathclose%
\pgfusepath{stroke,fill}%
\end{pgfscope}%
\begin{pgfscope}%
\pgfpathrectangle{\pgfqpoint{3.985294in}{1.750000in}}{\pgfqpoint{2.279412in}{2.004545in}}%
\pgfusepath{clip}%
\pgfsetroundcap%
\pgfsetroundjoin%
\pgfsetlinewidth{0.399984pt}%
\definecolor{currentstroke}{rgb}{0.281446,0.084320,0.407414}%
\pgfsetstrokecolor{currentstroke}%
\pgfsetdash{}{0pt}%
\pgfpathmoveto{\pgfqpoint{5.184457in}{3.145253in}}%
\pgfpathquadraticcurveto{\pgfqpoint{5.172736in}{3.141393in}}{\pgfqpoint{5.166893in}{3.139468in}}%
\pgfusepath{stroke}%
\end{pgfscope}%
\begin{pgfscope}%
\pgfpathrectangle{\pgfqpoint{3.985294in}{1.750000in}}{\pgfqpoint{2.279412in}{2.004545in}}%
\pgfusepath{clip}%
\pgfsetroundcap%
\pgfsetroundjoin%
\definecolor{currentfill}{rgb}{0.281446,0.084320,0.407414}%
\pgfsetfillcolor{currentfill}%
\pgfsetlinewidth{0.399984pt}%
\definecolor{currentstroke}{rgb}{0.281446,0.084320,0.407414}%
\pgfsetstrokecolor{currentstroke}%
\pgfsetdash{}{0pt}%
\pgfpathmoveto{\pgfqpoint{5.228350in}{3.130464in}}%
\pgfpathlineto{\pgfqpoint{5.166893in}{3.139468in}}%
\pgfpathlineto{\pgfqpoint{5.210970in}{3.183231in}}%
\pgfpathlineto{\pgfqpoint{5.228350in}{3.130464in}}%
\pgfpathlineto{\pgfqpoint{5.228350in}{3.130464in}}%
\pgfpathclose%
\pgfusepath{stroke,fill}%
\end{pgfscope}%
\begin{pgfscope}%
\pgfpathrectangle{\pgfqpoint{3.985294in}{1.750000in}}{\pgfqpoint{2.279412in}{2.004545in}}%
\pgfusepath{clip}%
\pgfsetroundcap%
\pgfsetroundjoin%
\pgfsetlinewidth{0.325898pt}%
\definecolor{currentstroke}{rgb}{0.271305,0.019942,0.347269}%
\pgfsetstrokecolor{currentstroke}%
\pgfsetdash{}{0pt}%
\pgfpathmoveto{\pgfqpoint{5.693982in}{3.294061in}}%
\pgfpathquadraticcurveto{\pgfqpoint{5.681453in}{3.293693in}}{\pgfqpoint{5.673964in}{3.293473in}}%
\pgfusepath{stroke}%
\end{pgfscope}%
\begin{pgfscope}%
\pgfpathrectangle{\pgfqpoint{3.985294in}{1.750000in}}{\pgfqpoint{2.279412in}{2.004545in}}%
\pgfusepath{clip}%
\pgfsetroundcap%
\pgfsetroundjoin%
\definecolor{currentfill}{rgb}{0.271305,0.019942,0.347269}%
\pgfsetfillcolor{currentfill}%
\pgfsetlinewidth{0.325898pt}%
\definecolor{currentstroke}{rgb}{0.271305,0.019942,0.347269}%
\pgfsetstrokecolor{currentstroke}%
\pgfsetdash{}{0pt}%
\pgfpathmoveto{\pgfqpoint{5.730311in}{3.267339in}}%
\pgfpathlineto{\pgfqpoint{5.673964in}{3.293473in}}%
\pgfpathlineto{\pgfqpoint{5.728679in}{3.322870in}}%
\pgfpathlineto{\pgfqpoint{5.730311in}{3.267339in}}%
\pgfpathlineto{\pgfqpoint{5.730311in}{3.267339in}}%
\pgfpathclose%
\pgfusepath{stroke,fill}%
\end{pgfscope}%
\begin{pgfscope}%
\pgfpathrectangle{\pgfqpoint{3.985294in}{1.750000in}}{\pgfqpoint{2.279412in}{2.004545in}}%
\pgfusepath{clip}%
\pgfsetroundcap%
\pgfsetroundjoin%
\pgfsetlinewidth{0.337151pt}%
\definecolor{currentstroke}{rgb}{0.273809,0.031497,0.358853}%
\pgfsetstrokecolor{currentstroke}%
\pgfsetdash{}{0pt}%
\pgfpathmoveto{\pgfqpoint{5.237418in}{3.344479in}}%
\pgfpathquadraticcurveto{\pgfqpoint{5.227710in}{3.341122in}}{\pgfqpoint{5.222931in}{3.339469in}}%
\pgfusepath{stroke}%
\end{pgfscope}%
\begin{pgfscope}%
\pgfpathrectangle{\pgfqpoint{3.985294in}{1.750000in}}{\pgfqpoint{2.279412in}{2.004545in}}%
\pgfusepath{clip}%
\pgfsetroundcap%
\pgfsetroundjoin%
\definecolor{currentfill}{rgb}{0.273809,0.031497,0.358853}%
\pgfsetfillcolor{currentfill}%
\pgfsetlinewidth{0.337151pt}%
\definecolor{currentstroke}{rgb}{0.273809,0.031497,0.358853}%
\pgfsetstrokecolor{currentstroke}%
\pgfsetdash{}{0pt}%
\pgfpathmoveto{\pgfqpoint{5.284514in}{3.331373in}}%
\pgfpathlineto{\pgfqpoint{5.222931in}{3.339469in}}%
\pgfpathlineto{\pgfqpoint{5.266358in}{3.383878in}}%
\pgfpathlineto{\pgfqpoint{5.284514in}{3.331373in}}%
\pgfpathlineto{\pgfqpoint{5.284514in}{3.331373in}}%
\pgfpathclose%
\pgfusepath{stroke,fill}%
\end{pgfscope}%
\begin{pgfscope}%
\pgfpathrectangle{\pgfqpoint{3.985294in}{1.750000in}}{\pgfqpoint{2.279412in}{2.004545in}}%
\pgfusepath{clip}%
\pgfsetroundcap%
\pgfsetroundjoin%
\pgfsetlinewidth{0.381900pt}%
\definecolor{currentstroke}{rgb}{0.279566,0.067836,0.391917}%
\pgfsetstrokecolor{currentstroke}%
\pgfsetdash{}{0pt}%
\pgfpathmoveto{\pgfqpoint{5.018946in}{2.293498in}}%
\pgfpathquadraticcurveto{\pgfqpoint{5.015810in}{2.299628in}}{\pgfqpoint{5.015365in}{2.300497in}}%
\pgfusepath{stroke}%
\end{pgfscope}%
\begin{pgfscope}%
\pgfpathrectangle{\pgfqpoint{3.985294in}{1.750000in}}{\pgfqpoint{2.279412in}{2.004545in}}%
\pgfusepath{clip}%
\pgfsetroundcap%
\pgfsetroundjoin%
\definecolor{currentfill}{rgb}{0.279566,0.067836,0.391917}%
\pgfsetfillcolor{currentfill}%
\pgfsetlinewidth{0.381900pt}%
\definecolor{currentstroke}{rgb}{0.279566,0.067836,0.391917}%
\pgfsetstrokecolor{currentstroke}%
\pgfsetdash{}{0pt}%
\pgfpathmoveto{\pgfqpoint{5.015944in}{2.238387in}}%
\pgfpathlineto{\pgfqpoint{5.015365in}{2.300497in}}%
\pgfpathlineto{\pgfqpoint{5.065400in}{2.263694in}}%
\pgfpathlineto{\pgfqpoint{5.015944in}{2.238387in}}%
\pgfpathlineto{\pgfqpoint{5.015944in}{2.238387in}}%
\pgfpathclose%
\pgfusepath{stroke,fill}%
\end{pgfscope}%
\begin{pgfscope}%
\pgfpathrectangle{\pgfqpoint{3.985294in}{1.750000in}}{\pgfqpoint{2.279412in}{2.004545in}}%
\pgfusepath{clip}%
\pgfsetroundcap%
\pgfsetroundjoin%
\pgfsetlinewidth{0.343553pt}%
\definecolor{currentstroke}{rgb}{0.274952,0.037752,0.364543}%
\pgfsetstrokecolor{currentstroke}%
\pgfsetdash{}{0pt}%
\pgfpathmoveto{\pgfqpoint{5.187049in}{2.202726in}}%
\pgfpathquadraticcurveto{\pgfqpoint{5.176267in}{2.208050in}}{\pgfqpoint{5.170250in}{2.211020in}}%
\pgfusepath{stroke}%
\end{pgfscope}%
\begin{pgfscope}%
\pgfpathrectangle{\pgfqpoint{3.985294in}{1.750000in}}{\pgfqpoint{2.279412in}{2.004545in}}%
\pgfusepath{clip}%
\pgfsetroundcap%
\pgfsetroundjoin%
\definecolor{currentfill}{rgb}{0.274952,0.037752,0.364543}%
\pgfsetfillcolor{currentfill}%
\pgfsetlinewidth{0.343553pt}%
\definecolor{currentstroke}{rgb}{0.274952,0.037752,0.364543}%
\pgfsetstrokecolor{currentstroke}%
\pgfsetdash{}{0pt}%
\pgfpathmoveto{\pgfqpoint{5.207767in}{2.161518in}}%
\pgfpathlineto{\pgfqpoint{5.170250in}{2.211020in}}%
\pgfpathlineto{\pgfqpoint{5.232362in}{2.211333in}}%
\pgfpathlineto{\pgfqpoint{5.207767in}{2.161518in}}%
\pgfpathlineto{\pgfqpoint{5.207767in}{2.161518in}}%
\pgfpathclose%
\pgfusepath{stroke,fill}%
\end{pgfscope}%
\begin{pgfscope}%
\pgfpathrectangle{\pgfqpoint{3.985294in}{1.750000in}}{\pgfqpoint{2.279412in}{2.004545in}}%
\pgfusepath{clip}%
\pgfsetroundcap%
\pgfsetroundjoin%
\pgfsetlinewidth{0.324662pt}%
\definecolor{currentstroke}{rgb}{0.271305,0.019942,0.347269}%
\pgfsetstrokecolor{currentstroke}%
\pgfsetdash{}{0pt}%
\pgfpathmoveto{\pgfqpoint{5.663321in}{2.259869in}}%
\pgfpathquadraticcurveto{\pgfqpoint{5.650800in}{2.260372in}}{\pgfqpoint{5.643297in}{2.260674in}}%
\pgfusepath{stroke}%
\end{pgfscope}%
\begin{pgfscope}%
\pgfpathrectangle{\pgfqpoint{3.985294in}{1.750000in}}{\pgfqpoint{2.279412in}{2.004545in}}%
\pgfusepath{clip}%
\pgfsetroundcap%
\pgfsetroundjoin%
\definecolor{currentfill}{rgb}{0.271305,0.019942,0.347269}%
\pgfsetfillcolor{currentfill}%
\pgfsetlinewidth{0.324662pt}%
\definecolor{currentstroke}{rgb}{0.271305,0.019942,0.347269}%
\pgfsetstrokecolor{currentstroke}%
\pgfsetdash{}{0pt}%
\pgfpathmoveto{\pgfqpoint{5.697693in}{2.230689in}}%
\pgfpathlineto{\pgfqpoint{5.643297in}{2.260674in}}%
\pgfpathlineto{\pgfqpoint{5.699923in}{2.286200in}}%
\pgfpathlineto{\pgfqpoint{5.697693in}{2.230689in}}%
\pgfpathlineto{\pgfqpoint{5.697693in}{2.230689in}}%
\pgfpathclose%
\pgfusepath{stroke,fill}%
\end{pgfscope}%
\begin{pgfscope}%
\pgfpathrectangle{\pgfqpoint{3.985294in}{1.750000in}}{\pgfqpoint{2.279412in}{2.004545in}}%
\pgfusepath{clip}%
\pgfsetroundcap%
\pgfsetroundjoin%
\pgfsetlinewidth{0.460182pt}%
\definecolor{currentstroke}{rgb}{0.283072,0.130895,0.449241}%
\pgfsetstrokecolor{currentstroke}%
\pgfsetdash{}{0pt}%
\pgfpathmoveto{\pgfqpoint{5.541099in}{2.530796in}}%
\pgfpathquadraticcurveto{\pgfqpoint{5.528567in}{2.531141in}}{\pgfqpoint{5.523152in}{2.531290in}}%
\pgfusepath{stroke}%
\end{pgfscope}%
\begin{pgfscope}%
\pgfpathrectangle{\pgfqpoint{3.985294in}{1.750000in}}{\pgfqpoint{2.279412in}{2.004545in}}%
\pgfusepath{clip}%
\pgfsetroundcap%
\pgfsetroundjoin%
\definecolor{currentfill}{rgb}{0.283072,0.130895,0.449241}%
\pgfsetfillcolor{currentfill}%
\pgfsetlinewidth{0.460182pt}%
\definecolor{currentstroke}{rgb}{0.283072,0.130895,0.449241}%
\pgfsetstrokecolor{currentstroke}%
\pgfsetdash{}{0pt}%
\pgfpathmoveto{\pgfqpoint{5.577923in}{2.501996in}}%
\pgfpathlineto{\pgfqpoint{5.523152in}{2.531290in}}%
\pgfpathlineto{\pgfqpoint{5.579449in}{2.557531in}}%
\pgfpathlineto{\pgfqpoint{5.577923in}{2.501996in}}%
\pgfpathlineto{\pgfqpoint{5.577923in}{2.501996in}}%
\pgfpathclose%
\pgfusepath{stroke,fill}%
\end{pgfscope}%
\begin{pgfscope}%
\pgfpathrectangle{\pgfqpoint{3.985294in}{1.750000in}}{\pgfqpoint{2.279412in}{2.004545in}}%
\pgfusepath{clip}%
\pgfsetroundcap%
\pgfsetroundjoin%
\pgfsetlinewidth{0.343407pt}%
\definecolor{currentstroke}{rgb}{0.274952,0.037752,0.364543}%
\pgfsetstrokecolor{currentstroke}%
\pgfsetdash{}{0pt}%
\pgfpathmoveto{\pgfqpoint{5.386172in}{3.323584in}}%
\pgfpathquadraticcurveto{\pgfqpoint{5.373774in}{3.322031in}}{\pgfqpoint{5.366648in}{3.321138in}}%
\pgfusepath{stroke}%
\end{pgfscope}%
\begin{pgfscope}%
\pgfpathrectangle{\pgfqpoint{3.985294in}{1.750000in}}{\pgfqpoint{2.279412in}{2.004545in}}%
\pgfusepath{clip}%
\pgfsetroundcap%
\pgfsetroundjoin%
\definecolor{currentfill}{rgb}{0.274952,0.037752,0.364543}%
\pgfsetfillcolor{currentfill}%
\pgfsetlinewidth{0.343407pt}%
\definecolor{currentstroke}{rgb}{0.274952,0.037752,0.364543}%
\pgfsetstrokecolor{currentstroke}%
\pgfsetdash{}{0pt}%
\pgfpathmoveto{\pgfqpoint{5.425226in}{3.300482in}}%
\pgfpathlineto{\pgfqpoint{5.366648in}{3.321138in}}%
\pgfpathlineto{\pgfqpoint{5.418319in}{3.355607in}}%
\pgfpathlineto{\pgfqpoint{5.425226in}{3.300482in}}%
\pgfpathlineto{\pgfqpoint{5.425226in}{3.300482in}}%
\pgfpathclose%
\pgfusepath{stroke,fill}%
\end{pgfscope}%
\begin{pgfscope}%
\pgfpathrectangle{\pgfqpoint{3.985294in}{1.750000in}}{\pgfqpoint{2.279412in}{2.004545in}}%
\pgfusepath{clip}%
\pgfsetroundcap%
\pgfsetroundjoin%
\pgfsetlinewidth{0.367601pt}%
\definecolor{currentstroke}{rgb}{0.277941,0.056324,0.381191}%
\pgfsetstrokecolor{currentstroke}%
\pgfsetdash{}{0pt}%
\pgfpathmoveto{\pgfqpoint{5.330168in}{2.210991in}}%
\pgfpathquadraticcurveto{\pgfqpoint{5.317891in}{2.213211in}}{\pgfqpoint{5.311211in}{2.214419in}}%
\pgfusepath{stroke}%
\end{pgfscope}%
\begin{pgfscope}%
\pgfpathrectangle{\pgfqpoint{3.985294in}{1.750000in}}{\pgfqpoint{2.279412in}{2.004545in}}%
\pgfusepath{clip}%
\pgfsetroundcap%
\pgfsetroundjoin%
\definecolor{currentfill}{rgb}{0.277941,0.056324,0.381191}%
\pgfsetfillcolor{currentfill}%
\pgfsetlinewidth{0.367601pt}%
\definecolor{currentstroke}{rgb}{0.277941,0.056324,0.381191}%
\pgfsetstrokecolor{currentstroke}%
\pgfsetdash{}{0pt}%
\pgfpathmoveto{\pgfqpoint{5.360938in}{2.177200in}}%
\pgfpathlineto{\pgfqpoint{5.311211in}{2.214419in}}%
\pgfpathlineto{\pgfqpoint{5.370822in}{2.231869in}}%
\pgfpathlineto{\pgfqpoint{5.360938in}{2.177200in}}%
\pgfpathlineto{\pgfqpoint{5.360938in}{2.177200in}}%
\pgfpathclose%
\pgfusepath{stroke,fill}%
\end{pgfscope}%
\begin{pgfscope}%
\pgfpathrectangle{\pgfqpoint{3.985294in}{1.750000in}}{\pgfqpoint{2.279412in}{2.004545in}}%
\pgfusepath{clip}%
\pgfsetroundcap%
\pgfsetroundjoin%
\pgfsetlinewidth{0.355737pt}%
\definecolor{currentstroke}{rgb}{0.276022,0.044167,0.370164}%
\pgfsetstrokecolor{currentstroke}%
\pgfsetdash{}{0pt}%
\pgfpathmoveto{\pgfqpoint{5.590171in}{3.156867in}}%
\pgfpathquadraticcurveto{\pgfqpoint{5.577649in}{3.156343in}}{\pgfqpoint{5.570626in}{3.156050in}}%
\pgfusepath{stroke}%
\end{pgfscope}%
\begin{pgfscope}%
\pgfpathrectangle{\pgfqpoint{3.985294in}{1.750000in}}{\pgfqpoint{2.279412in}{2.004545in}}%
\pgfusepath{clip}%
\pgfsetroundcap%
\pgfsetroundjoin%
\definecolor{currentfill}{rgb}{0.276022,0.044167,0.370164}%
\pgfsetfillcolor{currentfill}%
\pgfsetlinewidth{0.355737pt}%
\definecolor{currentstroke}{rgb}{0.276022,0.044167,0.370164}%
\pgfsetstrokecolor{currentstroke}%
\pgfsetdash{}{0pt}%
\pgfpathmoveto{\pgfqpoint{5.627293in}{3.130616in}}%
\pgfpathlineto{\pgfqpoint{5.570626in}{3.156050in}}%
\pgfpathlineto{\pgfqpoint{5.624974in}{3.186123in}}%
\pgfpathlineto{\pgfqpoint{5.627293in}{3.130616in}}%
\pgfpathlineto{\pgfqpoint{5.627293in}{3.130616in}}%
\pgfpathclose%
\pgfusepath{stroke,fill}%
\end{pgfscope}%
\begin{pgfscope}%
\pgfpathrectangle{\pgfqpoint{3.985294in}{1.750000in}}{\pgfqpoint{2.279412in}{2.004545in}}%
\pgfusepath{clip}%
\pgfsetroundcap%
\pgfsetroundjoin%
\pgfsetlinewidth{0.388474pt}%
\definecolor{currentstroke}{rgb}{0.280267,0.073417,0.397163}%
\pgfsetstrokecolor{currentstroke}%
\pgfsetdash{}{0pt}%
\pgfpathmoveto{\pgfqpoint{5.291286in}{3.219966in}}%
\pgfpathquadraticcurveto{\pgfqpoint{5.279047in}{3.217611in}}{\pgfqpoint{5.272710in}{3.216391in}}%
\pgfusepath{stroke}%
\end{pgfscope}%
\begin{pgfscope}%
\pgfpathrectangle{\pgfqpoint{3.985294in}{1.750000in}}{\pgfqpoint{2.279412in}{2.004545in}}%
\pgfusepath{clip}%
\pgfsetroundcap%
\pgfsetroundjoin%
\definecolor{currentfill}{rgb}{0.280267,0.073417,0.397163}%
\pgfsetfillcolor{currentfill}%
\pgfsetlinewidth{0.388474pt}%
\definecolor{currentstroke}{rgb}{0.280267,0.073417,0.397163}%
\pgfsetstrokecolor{currentstroke}%
\pgfsetdash{}{0pt}%
\pgfpathmoveto{\pgfqpoint{5.332514in}{3.199613in}}%
\pgfpathlineto{\pgfqpoint{5.272710in}{3.216391in}}%
\pgfpathlineto{\pgfqpoint{5.322015in}{3.254168in}}%
\pgfpathlineto{\pgfqpoint{5.332514in}{3.199613in}}%
\pgfpathlineto{\pgfqpoint{5.332514in}{3.199613in}}%
\pgfpathclose%
\pgfusepath{stroke,fill}%
\end{pgfscope}%
\begin{pgfscope}%
\pgfpathrectangle{\pgfqpoint{3.985294in}{1.750000in}}{\pgfqpoint{2.279412in}{2.004545in}}%
\pgfusepath{clip}%
\pgfsetroundcap%
\pgfsetroundjoin%
\pgfsetlinewidth{0.368544pt}%
\definecolor{currentstroke}{rgb}{0.277941,0.056324,0.381191}%
\pgfsetstrokecolor{currentstroke}%
\pgfsetdash{}{0pt}%
\pgfpathmoveto{\pgfqpoint{5.334603in}{3.279401in}}%
\pgfpathquadraticcurveto{\pgfqpoint{5.322543in}{3.276850in}}{\pgfqpoint{5.316060in}{3.275478in}}%
\pgfusepath{stroke}%
\end{pgfscope}%
\begin{pgfscope}%
\pgfpathrectangle{\pgfqpoint{3.985294in}{1.750000in}}{\pgfqpoint{2.279412in}{2.004545in}}%
\pgfusepath{clip}%
\pgfsetroundcap%
\pgfsetroundjoin%
\definecolor{currentfill}{rgb}{0.277941,0.056324,0.381191}%
\pgfsetfillcolor{currentfill}%
\pgfsetlinewidth{0.368544pt}%
\definecolor{currentstroke}{rgb}{0.277941,0.056324,0.381191}%
\pgfsetstrokecolor{currentstroke}%
\pgfsetdash{}{0pt}%
\pgfpathmoveto{\pgfqpoint{5.376162in}{3.259802in}}%
\pgfpathlineto{\pgfqpoint{5.316060in}{3.275478in}}%
\pgfpathlineto{\pgfqpoint{5.364662in}{3.314154in}}%
\pgfpathlineto{\pgfqpoint{5.376162in}{3.259802in}}%
\pgfpathlineto{\pgfqpoint{5.376162in}{3.259802in}}%
\pgfpathclose%
\pgfusepath{stroke,fill}%
\end{pgfscope}%
\begin{pgfscope}%
\pgfpathrectangle{\pgfqpoint{3.985294in}{1.750000in}}{\pgfqpoint{2.279412in}{2.004545in}}%
\pgfusepath{clip}%
\pgfsetroundcap%
\pgfsetroundjoin%
\pgfsetlinewidth{0.392427pt}%
\definecolor{currentstroke}{rgb}{0.280894,0.078907,0.402329}%
\pgfsetstrokecolor{currentstroke}%
\pgfsetdash{}{0pt}%
\pgfpathmoveto{\pgfqpoint{5.176292in}{2.301205in}}%
\pgfpathquadraticcurveto{\pgfqpoint{5.164841in}{2.305618in}}{\pgfqpoint{5.159055in}{2.307848in}}%
\pgfusepath{stroke}%
\end{pgfscope}%
\begin{pgfscope}%
\pgfpathrectangle{\pgfqpoint{3.985294in}{1.750000in}}{\pgfqpoint{2.279412in}{2.004545in}}%
\pgfusepath{clip}%
\pgfsetroundcap%
\pgfsetroundjoin%
\definecolor{currentfill}{rgb}{0.280894,0.078907,0.402329}%
\pgfsetfillcolor{currentfill}%
\pgfsetlinewidth{0.392427pt}%
\definecolor{currentstroke}{rgb}{0.280894,0.078907,0.402329}%
\pgfsetstrokecolor{currentstroke}%
\pgfsetdash{}{0pt}%
\pgfpathmoveto{\pgfqpoint{5.200905in}{2.261950in}}%
\pgfpathlineto{\pgfqpoint{5.159055in}{2.307848in}}%
\pgfpathlineto{\pgfqpoint{5.220883in}{2.313789in}}%
\pgfpathlineto{\pgfqpoint{5.200905in}{2.261950in}}%
\pgfpathlineto{\pgfqpoint{5.200905in}{2.261950in}}%
\pgfpathclose%
\pgfusepath{stroke,fill}%
\end{pgfscope}%
\begin{pgfscope}%
\pgfpathrectangle{\pgfqpoint{3.985294in}{1.750000in}}{\pgfqpoint{2.279412in}{2.004545in}}%
\pgfusepath{clip}%
\pgfsetroundcap%
\pgfsetroundjoin%
\pgfsetlinewidth{0.858781pt}%
\definecolor{currentstroke}{rgb}{0.197636,0.391528,0.554969}%
\pgfsetstrokecolor{currentstroke}%
\pgfsetdash{}{0pt}%
\pgfpathmoveto{\pgfqpoint{4.710448in}{2.922228in}}%
\pgfpathquadraticcurveto{\pgfqpoint{4.717638in}{2.914184in}}{\pgfqpoint{4.715974in}{2.916045in}}%
\pgfusepath{stroke}%
\end{pgfscope}%
\begin{pgfscope}%
\pgfpathrectangle{\pgfqpoint{3.985294in}{1.750000in}}{\pgfqpoint{2.279412in}{2.004545in}}%
\pgfusepath{clip}%
\pgfsetroundcap%
\pgfsetroundjoin%
\definecolor{currentfill}{rgb}{0.197636,0.391528,0.554969}%
\pgfsetfillcolor{currentfill}%
\pgfsetlinewidth{0.858781pt}%
\definecolor{currentstroke}{rgb}{0.197636,0.391528,0.554969}%
\pgfsetstrokecolor{currentstroke}%
\pgfsetdash{}{0pt}%
\pgfpathmoveto{\pgfqpoint{4.699658in}{2.975977in}}%
\pgfpathlineto{\pgfqpoint{4.715974in}{2.916045in}}%
\pgfpathlineto{\pgfqpoint{4.658239in}{2.938951in}}%
\pgfpathlineto{\pgfqpoint{4.699658in}{2.975977in}}%
\pgfpathlineto{\pgfqpoint{4.699658in}{2.975977in}}%
\pgfpathclose%
\pgfusepath{stroke,fill}%
\end{pgfscope}%
\begin{pgfscope}%
\pgfpathrectangle{\pgfqpoint{3.985294in}{1.750000in}}{\pgfqpoint{2.279412in}{2.004545in}}%
\pgfusepath{clip}%
\pgfsetbuttcap%
\pgfsetroundjoin%
\pgfsetlinewidth{1.505625pt}%
\definecolor{currentstroke}{rgb}{0.000000,0.000000,0.000000}%
\pgfsetstrokecolor{currentstroke}%
\pgfsetdash{}{0pt}%
\pgfpathmoveto{\pgfqpoint{4.778678in}{2.089441in}}%
\pgfpathlineto{\pgfqpoint{4.778678in}{3.415105in}}%
\pgfusepath{stroke}%
\end{pgfscope}%
\begin{pgfscope}%
\pgfpathrectangle{\pgfqpoint{3.985294in}{1.750000in}}{\pgfqpoint{2.279412in}{2.004545in}}%
\pgfusepath{clip}%
\pgfsetbuttcap%
\pgfsetroundjoin%
\pgfsetlinewidth{1.505625pt}%
\definecolor{currentstroke}{rgb}{0.000000,0.000000,0.000000}%
\pgfsetstrokecolor{currentstroke}%
\pgfsetdash{}{0pt}%
\pgfpathmoveto{\pgfqpoint{5.927204in}{2.089441in}}%
\pgfpathlineto{\pgfqpoint{5.927204in}{3.415105in}}%
\pgfusepath{stroke}%
\end{pgfscope}%
\begin{pgfscope}%
\pgfsetrectcap%
\pgfsetmiterjoin%
\pgfsetlinewidth{0.803000pt}%
\definecolor{currentstroke}{rgb}{0.000000,0.000000,0.000000}%
\pgfsetstrokecolor{currentstroke}%
\pgfsetdash{}{0pt}%
\pgfpathmoveto{\pgfqpoint{3.985294in}{1.750000in}}%
\pgfpathlineto{\pgfqpoint{3.985294in}{3.754545in}}%
\pgfusepath{stroke}%
\end{pgfscope}%
\begin{pgfscope}%
\pgfsetrectcap%
\pgfsetmiterjoin%
\pgfsetlinewidth{0.803000pt}%
\definecolor{currentstroke}{rgb}{0.000000,0.000000,0.000000}%
\pgfsetstrokecolor{currentstroke}%
\pgfsetdash{}{0pt}%
\pgfpathmoveto{\pgfqpoint{6.264706in}{1.750000in}}%
\pgfpathlineto{\pgfqpoint{6.264706in}{3.754545in}}%
\pgfusepath{stroke}%
\end{pgfscope}%
\begin{pgfscope}%
\pgfsetrectcap%
\pgfsetmiterjoin%
\pgfsetlinewidth{0.803000pt}%
\definecolor{currentstroke}{rgb}{0.000000,0.000000,0.000000}%
\pgfsetstrokecolor{currentstroke}%
\pgfsetdash{}{0pt}%
\pgfpathmoveto{\pgfqpoint{3.985294in}{1.750000in}}%
\pgfpathlineto{\pgfqpoint{6.264706in}{1.750000in}}%
\pgfusepath{stroke}%
\end{pgfscope}%
\begin{pgfscope}%
\pgfsetrectcap%
\pgfsetmiterjoin%
\pgfsetlinewidth{0.803000pt}%
\definecolor{currentstroke}{rgb}{0.000000,0.000000,0.000000}%
\pgfsetstrokecolor{currentstroke}%
\pgfsetdash{}{0pt}%
\pgfpathmoveto{\pgfqpoint{3.985294in}{3.754545in}}%
\pgfpathlineto{\pgfqpoint{6.264706in}{3.754545in}}%
\pgfusepath{stroke}%
\end{pgfscope}%
\begin{pgfscope}%
\definecolor{textcolor}{rgb}{0.000000,0.000000,0.000000}%
\pgfsetstrokecolor{textcolor}%
\pgfsetfillcolor{textcolor}%
\pgftext[x=5.125000in,y=3.837879in,,base]{\color{textcolor}\sffamily\fontsize{12.000000}{14.400000}\selectfont e)}%
\end{pgfscope}%
\begin{pgfscope}%
\pgfsetbuttcap%
\pgfsetmiterjoin%
\definecolor{currentfill}{rgb}{1.000000,1.000000,1.000000}%
\pgfsetfillcolor{currentfill}%
\pgfsetlinewidth{0.000000pt}%
\definecolor{currentstroke}{rgb}{0.000000,0.000000,0.000000}%
\pgfsetstrokecolor{currentstroke}%
\pgfsetstrokeopacity{0.000000}%
\pgfsetdash{}{0pt}%
\pgfpathmoveto{\pgfqpoint{6.720588in}{1.750000in}}%
\pgfpathlineto{\pgfqpoint{9.000000in}{1.750000in}}%
\pgfpathlineto{\pgfqpoint{9.000000in}{3.754545in}}%
\pgfpathlineto{\pgfqpoint{6.720588in}{3.754545in}}%
\pgfpathlineto{\pgfqpoint{6.720588in}{1.750000in}}%
\pgfpathclose%
\pgfusepath{fill}%
\end{pgfscope}%
\begin{pgfscope}%
\pgfpathrectangle{\pgfqpoint{6.720588in}{1.750000in}}{\pgfqpoint{2.279412in}{2.004545in}}%
\pgfusepath{clip}%
\pgfsys@transformcm{2.291667}{0.000000}{0.000000}{2.013889}{6.720588in}{1.750000in}%
\pgftext[left,bottom]{\includegraphics[interpolate=false,width=1.000000in,height=1.000000in]{q_series-img5.png}}%
\end{pgfscope}%
\begin{pgfscope}%
\pgfsetbuttcap%
\pgfsetroundjoin%
\definecolor{currentfill}{rgb}{0.000000,0.000000,0.000000}%
\pgfsetfillcolor{currentfill}%
\pgfsetlinewidth{0.803000pt}%
\definecolor{currentstroke}{rgb}{0.000000,0.000000,0.000000}%
\pgfsetstrokecolor{currentstroke}%
\pgfsetdash{}{0pt}%
\pgfsys@defobject{currentmarker}{\pgfqpoint{0.000000in}{-0.048611in}}{\pgfqpoint{0.000000in}{0.000000in}}{%
\pgfpathmoveto{\pgfqpoint{0.000000in}{0.000000in}}%
\pgfpathlineto{\pgfqpoint{0.000000in}{-0.048611in}}%
\pgfusepath{stroke,fill}%
}%
\begin{pgfscope}%
\pgfsys@transformshift{7.131130in}{1.750000in}%
\pgfsys@useobject{currentmarker}{}%
\end{pgfscope}%
\end{pgfscope}%
\begin{pgfscope}%
\definecolor{textcolor}{rgb}{0.000000,0.000000,0.000000}%
\pgfsetstrokecolor{textcolor}%
\pgfsetfillcolor{textcolor}%
\pgftext[x=7.131130in,y=1.652778in,,top]{\color{textcolor}\sffamily\fontsize{10.000000}{12.000000}\selectfont \(\displaystyle {\ensuremath{-}10}\)}%
\end{pgfscope}%
\begin{pgfscope}%
\pgfsetbuttcap%
\pgfsetroundjoin%
\definecolor{currentfill}{rgb}{0.000000,0.000000,0.000000}%
\pgfsetfillcolor{currentfill}%
\pgfsetlinewidth{0.803000pt}%
\definecolor{currentstroke}{rgb}{0.000000,0.000000,0.000000}%
\pgfsetstrokecolor{currentstroke}%
\pgfsetdash{}{0pt}%
\pgfsys@defobject{currentmarker}{\pgfqpoint{0.000000in}{-0.048611in}}{\pgfqpoint{0.000000in}{0.000000in}}{%
\pgfpathmoveto{\pgfqpoint{0.000000in}{0.000000in}}%
\pgfpathlineto{\pgfqpoint{0.000000in}{-0.048611in}}%
\pgfusepath{stroke,fill}%
}%
\begin{pgfscope}%
\pgfsys@transformshift{7.609683in}{1.750000in}%
\pgfsys@useobject{currentmarker}{}%
\end{pgfscope}%
\end{pgfscope}%
\begin{pgfscope}%
\definecolor{textcolor}{rgb}{0.000000,0.000000,0.000000}%
\pgfsetstrokecolor{textcolor}%
\pgfsetfillcolor{textcolor}%
\pgftext[x=7.609683in,y=1.652778in,,top]{\color{textcolor}\sffamily\fontsize{10.000000}{12.000000}\selectfont \(\displaystyle {\ensuremath{-}5}\)}%
\end{pgfscope}%
\begin{pgfscope}%
\pgfsetbuttcap%
\pgfsetroundjoin%
\definecolor{currentfill}{rgb}{0.000000,0.000000,0.000000}%
\pgfsetfillcolor{currentfill}%
\pgfsetlinewidth{0.803000pt}%
\definecolor{currentstroke}{rgb}{0.000000,0.000000,0.000000}%
\pgfsetstrokecolor{currentstroke}%
\pgfsetdash{}{0pt}%
\pgfsys@defobject{currentmarker}{\pgfqpoint{0.000000in}{-0.048611in}}{\pgfqpoint{0.000000in}{0.000000in}}{%
\pgfpathmoveto{\pgfqpoint{0.000000in}{0.000000in}}%
\pgfpathlineto{\pgfqpoint{0.000000in}{-0.048611in}}%
\pgfusepath{stroke,fill}%
}%
\begin{pgfscope}%
\pgfsys@transformshift{8.088235in}{1.750000in}%
\pgfsys@useobject{currentmarker}{}%
\end{pgfscope}%
\end{pgfscope}%
\begin{pgfscope}%
\definecolor{textcolor}{rgb}{0.000000,0.000000,0.000000}%
\pgfsetstrokecolor{textcolor}%
\pgfsetfillcolor{textcolor}%
\pgftext[x=8.088235in,y=1.652778in,,top]{\color{textcolor}\sffamily\fontsize{10.000000}{12.000000}\selectfont \(\displaystyle {0}\)}%
\end{pgfscope}%
\begin{pgfscope}%
\pgfsetbuttcap%
\pgfsetroundjoin%
\definecolor{currentfill}{rgb}{0.000000,0.000000,0.000000}%
\pgfsetfillcolor{currentfill}%
\pgfsetlinewidth{0.803000pt}%
\definecolor{currentstroke}{rgb}{0.000000,0.000000,0.000000}%
\pgfsetstrokecolor{currentstroke}%
\pgfsetdash{}{0pt}%
\pgfsys@defobject{currentmarker}{\pgfqpoint{0.000000in}{-0.048611in}}{\pgfqpoint{0.000000in}{0.000000in}}{%
\pgfpathmoveto{\pgfqpoint{0.000000in}{0.000000in}}%
\pgfpathlineto{\pgfqpoint{0.000000in}{-0.048611in}}%
\pgfusepath{stroke,fill}%
}%
\begin{pgfscope}%
\pgfsys@transformshift{8.566788in}{1.750000in}%
\pgfsys@useobject{currentmarker}{}%
\end{pgfscope}%
\end{pgfscope}%
\begin{pgfscope}%
\definecolor{textcolor}{rgb}{0.000000,0.000000,0.000000}%
\pgfsetstrokecolor{textcolor}%
\pgfsetfillcolor{textcolor}%
\pgftext[x=8.566788in,y=1.652778in,,top]{\color{textcolor}\sffamily\fontsize{10.000000}{12.000000}\selectfont \(\displaystyle {5}\)}%
\end{pgfscope}%
\begin{pgfscope}%
\definecolor{textcolor}{rgb}{0.000000,0.000000,0.000000}%
\pgfsetstrokecolor{textcolor}%
\pgfsetfillcolor{textcolor}%
\pgftext[x=7.860294in,y=1.473766in,,top]{\color{textcolor}\sffamily\fontsize{10.000000}{12.000000}\selectfont \(\displaystyle \zeta \, \mathrm{[\mu m]}\)}%
\end{pgfscope}%
\begin{pgfscope}%
\pgfsetbuttcap%
\pgfsetroundjoin%
\definecolor{currentfill}{rgb}{0.000000,0.000000,0.000000}%
\pgfsetfillcolor{currentfill}%
\pgfsetlinewidth{0.803000pt}%
\definecolor{currentstroke}{rgb}{0.000000,0.000000,0.000000}%
\pgfsetstrokecolor{currentstroke}%
\pgfsetdash{}{0pt}%
\pgfsys@defobject{currentmarker}{\pgfqpoint{-0.048611in}{0.000000in}}{\pgfqpoint{-0.000000in}{0.000000in}}{%
\pgfpathmoveto{\pgfqpoint{-0.000000in}{0.000000in}}%
\pgfpathlineto{\pgfqpoint{-0.048611in}{0.000000in}}%
\pgfusepath{stroke,fill}%
}%
\begin{pgfscope}%
\pgfsys@transformshift{6.720588in}{1.758025in}%
\pgfsys@useobject{currentmarker}{}%
\end{pgfscope}%
\end{pgfscope}%
\begin{pgfscope}%
\pgfsetbuttcap%
\pgfsetroundjoin%
\definecolor{currentfill}{rgb}{0.000000,0.000000,0.000000}%
\pgfsetfillcolor{currentfill}%
\pgfsetlinewidth{0.803000pt}%
\definecolor{currentstroke}{rgb}{0.000000,0.000000,0.000000}%
\pgfsetstrokecolor{currentstroke}%
\pgfsetdash{}{0pt}%
\pgfsys@defobject{currentmarker}{\pgfqpoint{-0.048611in}{0.000000in}}{\pgfqpoint{-0.000000in}{0.000000in}}{%
\pgfpathmoveto{\pgfqpoint{-0.000000in}{0.000000in}}%
\pgfpathlineto{\pgfqpoint{-0.048611in}{0.000000in}}%
\pgfusepath{stroke,fill}%
}%
\begin{pgfscope}%
\pgfsys@transformshift{6.720588in}{2.089441in}%
\pgfsys@useobject{currentmarker}{}%
\end{pgfscope}%
\end{pgfscope}%
\begin{pgfscope}%
\pgfsetbuttcap%
\pgfsetroundjoin%
\definecolor{currentfill}{rgb}{0.000000,0.000000,0.000000}%
\pgfsetfillcolor{currentfill}%
\pgfsetlinewidth{0.803000pt}%
\definecolor{currentstroke}{rgb}{0.000000,0.000000,0.000000}%
\pgfsetstrokecolor{currentstroke}%
\pgfsetdash{}{0pt}%
\pgfsys@defobject{currentmarker}{\pgfqpoint{-0.048611in}{0.000000in}}{\pgfqpoint{-0.000000in}{0.000000in}}{%
\pgfpathmoveto{\pgfqpoint{-0.000000in}{0.000000in}}%
\pgfpathlineto{\pgfqpoint{-0.048611in}{0.000000in}}%
\pgfusepath{stroke,fill}%
}%
\begin{pgfscope}%
\pgfsys@transformshift{6.720588in}{2.420857in}%
\pgfsys@useobject{currentmarker}{}%
\end{pgfscope}%
\end{pgfscope}%
\begin{pgfscope}%
\pgfsetbuttcap%
\pgfsetroundjoin%
\definecolor{currentfill}{rgb}{0.000000,0.000000,0.000000}%
\pgfsetfillcolor{currentfill}%
\pgfsetlinewidth{0.803000pt}%
\definecolor{currentstroke}{rgb}{0.000000,0.000000,0.000000}%
\pgfsetstrokecolor{currentstroke}%
\pgfsetdash{}{0pt}%
\pgfsys@defobject{currentmarker}{\pgfqpoint{-0.048611in}{0.000000in}}{\pgfqpoint{-0.000000in}{0.000000in}}{%
\pgfpathmoveto{\pgfqpoint{-0.000000in}{0.000000in}}%
\pgfpathlineto{\pgfqpoint{-0.048611in}{0.000000in}}%
\pgfusepath{stroke,fill}%
}%
\begin{pgfscope}%
\pgfsys@transformshift{6.720588in}{2.752273in}%
\pgfsys@useobject{currentmarker}{}%
\end{pgfscope}%
\end{pgfscope}%
\begin{pgfscope}%
\pgfsetbuttcap%
\pgfsetroundjoin%
\definecolor{currentfill}{rgb}{0.000000,0.000000,0.000000}%
\pgfsetfillcolor{currentfill}%
\pgfsetlinewidth{0.803000pt}%
\definecolor{currentstroke}{rgb}{0.000000,0.000000,0.000000}%
\pgfsetstrokecolor{currentstroke}%
\pgfsetdash{}{0pt}%
\pgfsys@defobject{currentmarker}{\pgfqpoint{-0.048611in}{0.000000in}}{\pgfqpoint{-0.000000in}{0.000000in}}{%
\pgfpathmoveto{\pgfqpoint{-0.000000in}{0.000000in}}%
\pgfpathlineto{\pgfqpoint{-0.048611in}{0.000000in}}%
\pgfusepath{stroke,fill}%
}%
\begin{pgfscope}%
\pgfsys@transformshift{6.720588in}{3.083689in}%
\pgfsys@useobject{currentmarker}{}%
\end{pgfscope}%
\end{pgfscope}%
\begin{pgfscope}%
\pgfsetbuttcap%
\pgfsetroundjoin%
\definecolor{currentfill}{rgb}{0.000000,0.000000,0.000000}%
\pgfsetfillcolor{currentfill}%
\pgfsetlinewidth{0.803000pt}%
\definecolor{currentstroke}{rgb}{0.000000,0.000000,0.000000}%
\pgfsetstrokecolor{currentstroke}%
\pgfsetdash{}{0pt}%
\pgfsys@defobject{currentmarker}{\pgfqpoint{-0.048611in}{0.000000in}}{\pgfqpoint{-0.000000in}{0.000000in}}{%
\pgfpathmoveto{\pgfqpoint{-0.000000in}{0.000000in}}%
\pgfpathlineto{\pgfqpoint{-0.048611in}{0.000000in}}%
\pgfusepath{stroke,fill}%
}%
\begin{pgfscope}%
\pgfsys@transformshift{6.720588in}{3.415105in}%
\pgfsys@useobject{currentmarker}{}%
\end{pgfscope}%
\end{pgfscope}%
\begin{pgfscope}%
\pgfsetbuttcap%
\pgfsetroundjoin%
\definecolor{currentfill}{rgb}{0.000000,0.000000,0.000000}%
\pgfsetfillcolor{currentfill}%
\pgfsetlinewidth{0.803000pt}%
\definecolor{currentstroke}{rgb}{0.000000,0.000000,0.000000}%
\pgfsetstrokecolor{currentstroke}%
\pgfsetdash{}{0pt}%
\pgfsys@defobject{currentmarker}{\pgfqpoint{-0.048611in}{0.000000in}}{\pgfqpoint{-0.000000in}{0.000000in}}{%
\pgfpathmoveto{\pgfqpoint{-0.000000in}{0.000000in}}%
\pgfpathlineto{\pgfqpoint{-0.048611in}{0.000000in}}%
\pgfusepath{stroke,fill}%
}%
\begin{pgfscope}%
\pgfsys@transformshift{6.720588in}{3.746521in}%
\pgfsys@useobject{currentmarker}{}%
\end{pgfscope}%
\end{pgfscope}%
\begin{pgfscope}%
\definecolor{textcolor}{rgb}{0.000000,0.000000,0.000000}%
\pgfsetstrokecolor{textcolor}%
\pgfsetfillcolor{textcolor}%
\pgftext[x=6.665033in,y=2.752273in,,bottom,rotate=90.000000]{\color{textcolor}\sffamily\fontsize{10.000000}{12.000000}\selectfont \(\displaystyle z \, \mathrm{[\mu m]}\)}%
\end{pgfscope}%
\begin{pgfscope}%
\pgfpathrectangle{\pgfqpoint{6.720588in}{1.750000in}}{\pgfqpoint{2.279412in}{2.004545in}}%
\pgfusepath{clip}%
\pgfsetbuttcap%
\pgfsetroundjoin%
\pgfsetlinewidth{0.307977pt}%
\definecolor{currentstroke}{rgb}{0.267004,0.004874,0.329415}%
\pgfsetstrokecolor{currentstroke}%
\pgfsetdash{}{0pt}%
\pgfpathmoveto{\pgfqpoint{8.424505in}{1.850137in}}%
\pgfpathlineto{\pgfqpoint{8.424505in}{1.850137in}}%
\pgfusepath{stroke}%
\end{pgfscope}%
\begin{pgfscope}%
\pgfpathrectangle{\pgfqpoint{6.720588in}{1.750000in}}{\pgfqpoint{2.279412in}{2.004545in}}%
\pgfusepath{clip}%
\pgfsetbuttcap%
\pgfsetroundjoin%
\pgfsetlinewidth{0.307977pt}%
\definecolor{currentstroke}{rgb}{0.267004,0.004874,0.329415}%
\pgfsetstrokecolor{currentstroke}%
\pgfsetdash{}{0pt}%
\pgfpathmoveto{\pgfqpoint{8.424505in}{1.850137in}}%
\pgfpathlineto{\pgfqpoint{8.424505in}{1.850137in}}%
\pgfusepath{stroke}%
\end{pgfscope}%
\begin{pgfscope}%
\pgfpathrectangle{\pgfqpoint{6.720588in}{1.750000in}}{\pgfqpoint{2.279412in}{2.004545in}}%
\pgfusepath{clip}%
\pgfsetbuttcap%
\pgfsetroundjoin%
\pgfsetlinewidth{0.307977pt}%
\definecolor{currentstroke}{rgb}{0.267004,0.004874,0.329415}%
\pgfsetstrokecolor{currentstroke}%
\pgfsetdash{}{0pt}%
\pgfpathmoveto{\pgfqpoint{8.424505in}{1.850137in}}%
\pgfpathlineto{\pgfqpoint{8.427741in}{1.850465in}}%
\pgfusepath{stroke}%
\end{pgfscope}%
\begin{pgfscope}%
\pgfpathrectangle{\pgfqpoint{6.720588in}{1.750000in}}{\pgfqpoint{2.279412in}{2.004545in}}%
\pgfusepath{clip}%
\pgfsetbuttcap%
\pgfsetroundjoin%
\pgfsetlinewidth{0.308712pt}%
\definecolor{currentstroke}{rgb}{0.268510,0.009605,0.335427}%
\pgfsetstrokecolor{currentstroke}%
\pgfsetdash{}{0pt}%
\pgfpathmoveto{\pgfqpoint{8.427741in}{1.850465in}}%
\pgfpathlineto{\pgfqpoint{8.429201in}{1.850717in}}%
\pgfusepath{stroke}%
\end{pgfscope}%
\begin{pgfscope}%
\pgfpathrectangle{\pgfqpoint{6.720588in}{1.750000in}}{\pgfqpoint{2.279412in}{2.004545in}}%
\pgfusepath{clip}%
\pgfsetbuttcap%
\pgfsetroundjoin%
\pgfsetlinewidth{0.309076pt}%
\definecolor{currentstroke}{rgb}{0.268510,0.009605,0.335427}%
\pgfsetstrokecolor{currentstroke}%
\pgfsetdash{}{0pt}%
\pgfpathmoveto{\pgfqpoint{8.429201in}{1.850717in}}%
\pgfpathlineto{\pgfqpoint{8.429567in}{1.851087in}}%
\pgfusepath{stroke}%
\end{pgfscope}%
\begin{pgfscope}%
\pgfpathrectangle{\pgfqpoint{6.720588in}{1.750000in}}{\pgfqpoint{2.279412in}{2.004545in}}%
\pgfusepath{clip}%
\pgfsetbuttcap%
\pgfsetroundjoin%
\pgfsetlinewidth{0.309205pt}%
\definecolor{currentstroke}{rgb}{0.268510,0.009605,0.335427}%
\pgfsetstrokecolor{currentstroke}%
\pgfsetdash{}{0pt}%
\pgfpathmoveto{\pgfqpoint{8.429567in}{1.851087in}}%
\pgfpathlineto{\pgfqpoint{8.428880in}{1.851565in}}%
\pgfusepath{stroke}%
\end{pgfscope}%
\begin{pgfscope}%
\pgfpathrectangle{\pgfqpoint{6.720588in}{1.750000in}}{\pgfqpoint{2.279412in}{2.004545in}}%
\pgfusepath{clip}%
\pgfsetbuttcap%
\pgfsetroundjoin%
\pgfsetlinewidth{0.309075pt}%
\definecolor{currentstroke}{rgb}{0.268510,0.009605,0.335427}%
\pgfsetstrokecolor{currentstroke}%
\pgfsetdash{}{0pt}%
\pgfpathmoveto{\pgfqpoint{8.428880in}{1.851565in}}%
\pgfpathlineto{\pgfqpoint{8.428880in}{1.851565in}}%
\pgfusepath{stroke}%
\end{pgfscope}%
\begin{pgfscope}%
\pgfpathrectangle{\pgfqpoint{6.720588in}{1.750000in}}{\pgfqpoint{2.279412in}{2.004545in}}%
\pgfusepath{clip}%
\pgfsetbuttcap%
\pgfsetroundjoin%
\pgfsetlinewidth{0.309075pt}%
\definecolor{currentstroke}{rgb}{0.268510,0.009605,0.335427}%
\pgfsetstrokecolor{currentstroke}%
\pgfsetdash{}{0pt}%
\pgfpathmoveto{\pgfqpoint{8.428880in}{1.851565in}}%
\pgfpathlineto{\pgfqpoint{8.427815in}{1.851957in}}%
\pgfusepath{stroke}%
\end{pgfscope}%
\begin{pgfscope}%
\pgfpathrectangle{\pgfqpoint{6.720588in}{1.750000in}}{\pgfqpoint{2.279412in}{2.004545in}}%
\pgfusepath{clip}%
\pgfsetbuttcap%
\pgfsetroundjoin%
\pgfsetlinewidth{0.308815pt}%
\definecolor{currentstroke}{rgb}{0.268510,0.009605,0.335427}%
\pgfsetstrokecolor{currentstroke}%
\pgfsetdash{}{0pt}%
\pgfpathmoveto{\pgfqpoint{8.427815in}{1.851957in}}%
\pgfpathlineto{\pgfqpoint{8.427815in}{1.851957in}}%
\pgfusepath{stroke}%
\end{pgfscope}%
\begin{pgfscope}%
\pgfpathrectangle{\pgfqpoint{6.720588in}{1.750000in}}{\pgfqpoint{2.279412in}{2.004545in}}%
\pgfusepath{clip}%
\pgfsetbuttcap%
\pgfsetroundjoin%
\pgfsetlinewidth{0.308815pt}%
\definecolor{currentstroke}{rgb}{0.268510,0.009605,0.335427}%
\pgfsetstrokecolor{currentstroke}%
\pgfsetdash{}{0pt}%
\pgfpathmoveto{\pgfqpoint{8.427815in}{1.851957in}}%
\pgfpathlineto{\pgfqpoint{8.427934in}{1.852157in}}%
\pgfusepath{stroke}%
\end{pgfscope}%
\begin{pgfscope}%
\pgfpathrectangle{\pgfqpoint{6.720588in}{1.750000in}}{\pgfqpoint{2.279412in}{2.004545in}}%
\pgfusepath{clip}%
\pgfsetbuttcap%
\pgfsetroundjoin%
\pgfsetlinewidth{0.308860pt}%
\definecolor{currentstroke}{rgb}{0.268510,0.009605,0.335427}%
\pgfsetstrokecolor{currentstroke}%
\pgfsetdash{}{0pt}%
\pgfpathmoveto{\pgfqpoint{8.427934in}{1.852157in}}%
\pgfpathlineto{\pgfqpoint{8.428148in}{1.852306in}}%
\pgfusepath{stroke}%
\end{pgfscope}%
\begin{pgfscope}%
\pgfpathrectangle{\pgfqpoint{6.720588in}{1.750000in}}{\pgfqpoint{2.279412in}{2.004545in}}%
\pgfusepath{clip}%
\pgfsetbuttcap%
\pgfsetroundjoin%
\pgfsetlinewidth{0.308931pt}%
\definecolor{currentstroke}{rgb}{0.268510,0.009605,0.335427}%
\pgfsetstrokecolor{currentstroke}%
\pgfsetdash{}{0pt}%
\pgfpathmoveto{\pgfqpoint{8.428148in}{1.852306in}}%
\pgfpathlineto{\pgfqpoint{8.428206in}{1.852432in}}%
\pgfusepath{stroke}%
\end{pgfscope}%
\begin{pgfscope}%
\pgfpathrectangle{\pgfqpoint{6.720588in}{1.750000in}}{\pgfqpoint{2.279412in}{2.004545in}}%
\pgfusepath{clip}%
\pgfsetbuttcap%
\pgfsetroundjoin%
\pgfsetlinewidth{0.308957pt}%
\definecolor{currentstroke}{rgb}{0.268510,0.009605,0.335427}%
\pgfsetstrokecolor{currentstroke}%
\pgfsetdash{}{0pt}%
\pgfpathmoveto{\pgfqpoint{8.428206in}{1.852432in}}%
\pgfpathlineto{\pgfqpoint{8.428009in}{1.852550in}}%
\pgfusepath{stroke}%
\end{pgfscope}%
\begin{pgfscope}%
\pgfpathrectangle{\pgfqpoint{6.720588in}{1.750000in}}{\pgfqpoint{2.279412in}{2.004545in}}%
\pgfusepath{clip}%
\pgfsetbuttcap%
\pgfsetroundjoin%
\pgfsetlinewidth{0.308907pt}%
\definecolor{currentstroke}{rgb}{0.268510,0.009605,0.335427}%
\pgfsetstrokecolor{currentstroke}%
\pgfsetdash{}{0pt}%
\pgfpathmoveto{\pgfqpoint{8.428009in}{1.852550in}}%
\pgfpathlineto{\pgfqpoint{8.427671in}{1.852651in}}%
\pgfusepath{stroke}%
\end{pgfscope}%
\begin{pgfscope}%
\pgfpathrectangle{\pgfqpoint{6.720588in}{1.750000in}}{\pgfqpoint{2.279412in}{2.004545in}}%
\pgfusepath{clip}%
\pgfsetbuttcap%
\pgfsetroundjoin%
\pgfsetlinewidth{0.308812pt}%
\definecolor{currentstroke}{rgb}{0.268510,0.009605,0.335427}%
\pgfsetstrokecolor{currentstroke}%
\pgfsetdash{}{0pt}%
\pgfpathmoveto{\pgfqpoint{8.427671in}{1.852651in}}%
\pgfpathlineto{\pgfqpoint{8.427529in}{1.852721in}}%
\pgfusepath{stroke}%
\end{pgfscope}%
\begin{pgfscope}%
\pgfpathrectangle{\pgfqpoint{6.720588in}{1.750000in}}{\pgfqpoint{2.279412in}{2.004545in}}%
\pgfusepath{clip}%
\pgfsetbuttcap%
\pgfsetroundjoin%
\pgfsetlinewidth{0.308773pt}%
\definecolor{currentstroke}{rgb}{0.268510,0.009605,0.335427}%
\pgfsetstrokecolor{currentstroke}%
\pgfsetdash{}{0pt}%
\pgfpathmoveto{\pgfqpoint{8.427529in}{1.852721in}}%
\pgfpathlineto{\pgfqpoint{8.427746in}{1.852756in}}%
\pgfusepath{stroke}%
\end{pgfscope}%
\begin{pgfscope}%
\pgfpathrectangle{\pgfqpoint{6.720588in}{1.750000in}}{\pgfqpoint{2.279412in}{2.004545in}}%
\pgfusepath{clip}%
\pgfsetbuttcap%
\pgfsetroundjoin%
\pgfsetlinewidth{0.308840pt}%
\definecolor{currentstroke}{rgb}{0.268510,0.009605,0.335427}%
\pgfsetstrokecolor{currentstroke}%
\pgfsetdash{}{0pt}%
\pgfpathmoveto{\pgfqpoint{8.427746in}{1.852756in}}%
\pgfpathlineto{\pgfqpoint{8.428060in}{1.852776in}}%
\pgfusepath{stroke}%
\end{pgfscope}%
\begin{pgfscope}%
\pgfpathrectangle{\pgfqpoint{6.720588in}{1.750000in}}{\pgfqpoint{2.279412in}{2.004545in}}%
\pgfusepath{clip}%
\pgfsetbuttcap%
\pgfsetroundjoin%
\pgfsetlinewidth{0.308936pt}%
\definecolor{currentstroke}{rgb}{0.268510,0.009605,0.335427}%
\pgfsetstrokecolor{currentstroke}%
\pgfsetdash{}{0pt}%
\pgfpathmoveto{\pgfqpoint{8.428060in}{1.852776in}}%
\pgfpathlineto{\pgfqpoint{8.428169in}{1.852802in}}%
\pgfusepath{stroke}%
\end{pgfscope}%
\begin{pgfscope}%
\pgfpathrectangle{\pgfqpoint{6.720588in}{1.750000in}}{\pgfqpoint{2.279412in}{2.004545in}}%
\pgfusepath{clip}%
\pgfsetbuttcap%
\pgfsetroundjoin%
\pgfsetlinewidth{0.308971pt}%
\definecolor{currentstroke}{rgb}{0.268510,0.009605,0.335427}%
\pgfsetstrokecolor{currentstroke}%
\pgfsetdash{}{0pt}%
\pgfpathmoveto{\pgfqpoint{8.428169in}{1.852802in}}%
\pgfpathlineto{\pgfqpoint{8.427956in}{1.852838in}}%
\pgfusepath{stroke}%
\end{pgfscope}%
\begin{pgfscope}%
\pgfpathrectangle{\pgfqpoint{6.720588in}{1.750000in}}{\pgfqpoint{2.279412in}{2.004545in}}%
\pgfusepath{clip}%
\pgfsetbuttcap%
\pgfsetroundjoin%
\pgfsetlinewidth{0.308909pt}%
\definecolor{currentstroke}{rgb}{0.268510,0.009605,0.335427}%
\pgfsetstrokecolor{currentstroke}%
\pgfsetdash{}{0pt}%
\pgfpathmoveto{\pgfqpoint{8.427956in}{1.852838in}}%
\pgfpathlineto{\pgfqpoint{8.427547in}{1.852879in}}%
\pgfusepath{stroke}%
\end{pgfscope}%
\begin{pgfscope}%
\pgfpathrectangle{\pgfqpoint{6.720588in}{1.750000in}}{\pgfqpoint{2.279412in}{2.004545in}}%
\pgfusepath{clip}%
\pgfsetbuttcap%
\pgfsetroundjoin%
\pgfsetlinewidth{0.308786pt}%
\definecolor{currentstroke}{rgb}{0.268510,0.009605,0.335427}%
\pgfsetstrokecolor{currentstroke}%
\pgfsetdash{}{0pt}%
\pgfpathmoveto{\pgfqpoint{8.427547in}{1.852879in}}%
\pgfpathlineto{\pgfqpoint{8.427381in}{1.852900in}}%
\pgfusepath{stroke}%
\end{pgfscope}%
\begin{pgfscope}%
\pgfpathrectangle{\pgfqpoint{6.720588in}{1.750000in}}{\pgfqpoint{2.279412in}{2.004545in}}%
\pgfusepath{clip}%
\pgfsetbuttcap%
\pgfsetroundjoin%
\pgfsetlinewidth{0.308736pt}%
\definecolor{currentstroke}{rgb}{0.268510,0.009605,0.335427}%
\pgfsetstrokecolor{currentstroke}%
\pgfsetdash{}{0pt}%
\pgfpathmoveto{\pgfqpoint{8.427381in}{1.852900in}}%
\pgfpathlineto{\pgfqpoint{8.427680in}{1.852894in}}%
\pgfusepath{stroke}%
\end{pgfscope}%
\begin{pgfscope}%
\pgfpathrectangle{\pgfqpoint{6.720588in}{1.750000in}}{\pgfqpoint{2.279412in}{2.004545in}}%
\pgfusepath{clip}%
\pgfsetbuttcap%
\pgfsetroundjoin%
\pgfsetlinewidth{0.308827pt}%
\definecolor{currentstroke}{rgb}{0.268510,0.009605,0.335427}%
\pgfsetstrokecolor{currentstroke}%
\pgfsetdash{}{0pt}%
\pgfpathmoveto{\pgfqpoint{8.427680in}{1.852894in}}%
\pgfpathlineto{\pgfqpoint{8.428082in}{1.852882in}}%
\pgfusepath{stroke}%
\end{pgfscope}%
\begin{pgfscope}%
\pgfpathrectangle{\pgfqpoint{6.720588in}{1.750000in}}{\pgfqpoint{2.279412in}{2.004545in}}%
\pgfusepath{clip}%
\pgfsetbuttcap%
\pgfsetroundjoin%
\pgfsetlinewidth{0.308950pt}%
\definecolor{currentstroke}{rgb}{0.268510,0.009605,0.335427}%
\pgfsetstrokecolor{currentstroke}%
\pgfsetdash{}{0pt}%
\pgfpathmoveto{\pgfqpoint{8.428082in}{1.852882in}}%
\pgfpathlineto{\pgfqpoint{8.428225in}{1.852884in}}%
\pgfusepath{stroke}%
\end{pgfscope}%
\begin{pgfscope}%
\pgfpathrectangle{\pgfqpoint{6.720588in}{1.750000in}}{\pgfqpoint{2.279412in}{2.004545in}}%
\pgfusepath{clip}%
\pgfsetbuttcap%
\pgfsetroundjoin%
\pgfsetlinewidth{0.308994pt}%
\definecolor{currentstroke}{rgb}{0.268510,0.009605,0.335427}%
\pgfsetstrokecolor{currentstroke}%
\pgfsetdash{}{0pt}%
\pgfpathmoveto{\pgfqpoint{8.428225in}{1.852884in}}%
\pgfpathlineto{\pgfqpoint{8.427969in}{1.852904in}}%
\pgfusepath{stroke}%
\end{pgfscope}%
\begin{pgfscope}%
\pgfpathrectangle{\pgfqpoint{6.720588in}{1.750000in}}{\pgfqpoint{2.279412in}{2.004545in}}%
\pgfusepath{clip}%
\pgfsetbuttcap%
\pgfsetroundjoin%
\pgfsetlinewidth{0.308917pt}%
\definecolor{currentstroke}{rgb}{0.268510,0.009605,0.335427}%
\pgfsetstrokecolor{currentstroke}%
\pgfsetdash{}{0pt}%
\pgfpathmoveto{\pgfqpoint{8.427969in}{1.852904in}}%
\pgfpathlineto{\pgfqpoint{8.427447in}{1.852935in}}%
\pgfusepath{stroke}%
\end{pgfscope}%
\begin{pgfscope}%
\pgfpathrectangle{\pgfqpoint{6.720588in}{1.750000in}}{\pgfqpoint{2.279412in}{2.004545in}}%
\pgfusepath{clip}%
\pgfsetbuttcap%
\pgfsetroundjoin%
\pgfsetlinewidth{0.308758pt}%
\definecolor{currentstroke}{rgb}{0.268510,0.009605,0.335427}%
\pgfsetstrokecolor{currentstroke}%
\pgfsetdash{}{0pt}%
\pgfpathmoveto{\pgfqpoint{8.427447in}{1.852935in}}%
\pgfpathlineto{\pgfqpoint{8.427236in}{1.852946in}}%
\pgfusepath{stroke}%
\end{pgfscope}%
\begin{pgfscope}%
\pgfpathrectangle{\pgfqpoint{6.720588in}{1.750000in}}{\pgfqpoint{2.279412in}{2.004545in}}%
\pgfusepath{clip}%
\pgfsetbuttcap%
\pgfsetroundjoin%
\pgfsetlinewidth{0.308693pt}%
\definecolor{currentstroke}{rgb}{0.268510,0.009605,0.335427}%
\pgfsetstrokecolor{currentstroke}%
\pgfsetdash{}{0pt}%
\pgfpathmoveto{\pgfqpoint{8.427236in}{1.852946in}}%
\pgfpathlineto{\pgfqpoint{8.427632in}{1.852927in}}%
\pgfusepath{stroke}%
\end{pgfscope}%
\begin{pgfscope}%
\pgfpathrectangle{\pgfqpoint{6.720588in}{1.750000in}}{\pgfqpoint{2.279412in}{2.004545in}}%
\pgfusepath{clip}%
\pgfsetbuttcap%
\pgfsetroundjoin%
\pgfsetlinewidth{0.308815pt}%
\definecolor{currentstroke}{rgb}{0.268510,0.009605,0.335427}%
\pgfsetstrokecolor{currentstroke}%
\pgfsetdash{}{0pt}%
\pgfpathmoveto{\pgfqpoint{8.427632in}{1.852927in}}%
\pgfpathlineto{\pgfqpoint{8.428136in}{1.852904in}}%
\pgfusepath{stroke}%
\end{pgfscope}%
\begin{pgfscope}%
\pgfpathrectangle{\pgfqpoint{6.720588in}{1.750000in}}{\pgfqpoint{2.279412in}{2.004545in}}%
\pgfusepath{clip}%
\pgfsetbuttcap%
\pgfsetroundjoin%
\pgfsetlinewidth{0.308968pt}%
\definecolor{currentstroke}{rgb}{0.268510,0.009605,0.335427}%
\pgfsetstrokecolor{currentstroke}%
\pgfsetdash{}{0pt}%
\pgfpathmoveto{\pgfqpoint{8.428136in}{1.852904in}}%
\pgfpathlineto{\pgfqpoint{8.428311in}{1.852899in}}%
\pgfusepath{stroke}%
\end{pgfscope}%
\begin{pgfscope}%
\pgfpathrectangle{\pgfqpoint{6.720588in}{1.750000in}}{\pgfqpoint{2.279412in}{2.004545in}}%
\pgfusepath{clip}%
\pgfsetbuttcap%
\pgfsetroundjoin%
\pgfsetlinewidth{0.309021pt}%
\definecolor{currentstroke}{rgb}{0.268510,0.009605,0.335427}%
\pgfsetstrokecolor{currentstroke}%
\pgfsetdash{}{0pt}%
\pgfpathmoveto{\pgfqpoint{8.428311in}{1.852899in}}%
\pgfpathlineto{\pgfqpoint{8.428002in}{1.852918in}}%
\pgfusepath{stroke}%
\end{pgfscope}%
\begin{pgfscope}%
\pgfpathrectangle{\pgfqpoint{6.720588in}{1.750000in}}{\pgfqpoint{2.279412in}{2.004545in}}%
\pgfusepath{clip}%
\pgfsetbuttcap%
\pgfsetroundjoin%
\pgfsetlinewidth{0.308928pt}%
\definecolor{currentstroke}{rgb}{0.268510,0.009605,0.335427}%
\pgfsetstrokecolor{currentstroke}%
\pgfsetdash{}{0pt}%
\pgfpathmoveto{\pgfqpoint{8.428002in}{1.852918in}}%
\pgfpathlineto{\pgfqpoint{8.427333in}{1.852953in}}%
\pgfusepath{stroke}%
\end{pgfscope}%
\begin{pgfscope}%
\pgfpathrectangle{\pgfqpoint{6.720588in}{1.750000in}}{\pgfqpoint{2.279412in}{2.004545in}}%
\pgfusepath{clip}%
\pgfsetbuttcap%
\pgfsetroundjoin%
\pgfsetlinewidth{0.308724pt}%
\definecolor{currentstroke}{rgb}{0.268510,0.009605,0.335427}%
\pgfsetstrokecolor{currentstroke}%
\pgfsetdash{}{0pt}%
\pgfpathmoveto{\pgfqpoint{8.427333in}{1.852953in}}%
\pgfpathlineto{\pgfqpoint{8.427333in}{1.852953in}}%
\pgfusepath{stroke}%
\end{pgfscope}%
\begin{pgfscope}%
\pgfpathrectangle{\pgfqpoint{6.720588in}{1.750000in}}{\pgfqpoint{2.279412in}{2.004545in}}%
\pgfusepath{clip}%
\pgfsetbuttcap%
\pgfsetroundjoin%
\pgfsetlinewidth{0.308724pt}%
\definecolor{currentstroke}{rgb}{0.268510,0.009605,0.335427}%
\pgfsetstrokecolor{currentstroke}%
\pgfsetdash{}{0pt}%
\pgfpathmoveto{\pgfqpoint{8.427333in}{1.852953in}}%
\pgfpathlineto{\pgfqpoint{8.427779in}{1.852929in}}%
\pgfusepath{stroke}%
\end{pgfscope}%
\begin{pgfscope}%
\pgfpathrectangle{\pgfqpoint{6.720588in}{1.750000in}}{\pgfqpoint{2.279412in}{2.004545in}}%
\pgfusepath{clip}%
\pgfsetbuttcap%
\pgfsetroundjoin%
\pgfsetlinewidth{0.308860pt}%
\definecolor{currentstroke}{rgb}{0.268510,0.009605,0.335427}%
\pgfsetstrokecolor{currentstroke}%
\pgfsetdash{}{0pt}%
\pgfpathmoveto{\pgfqpoint{8.427779in}{1.852929in}}%
\pgfpathlineto{\pgfqpoint{8.428196in}{1.852909in}}%
\pgfusepath{stroke}%
\end{pgfscope}%
\begin{pgfscope}%
\pgfpathrectangle{\pgfqpoint{6.720588in}{1.750000in}}{\pgfqpoint{2.279412in}{2.004545in}}%
\pgfusepath{clip}%
\pgfsetbuttcap%
\pgfsetroundjoin%
\pgfsetlinewidth{0.308987pt}%
\definecolor{currentstroke}{rgb}{0.268510,0.009605,0.335427}%
\pgfsetstrokecolor{currentstroke}%
\pgfsetdash{}{0pt}%
\pgfpathmoveto{\pgfqpoint{8.428196in}{1.852909in}}%
\pgfpathlineto{\pgfqpoint{8.428248in}{1.852909in}}%
\pgfusepath{stroke}%
\end{pgfscope}%
\begin{pgfscope}%
\pgfpathrectangle{\pgfqpoint{6.720588in}{1.750000in}}{\pgfqpoint{2.279412in}{2.004545in}}%
\pgfusepath{clip}%
\pgfsetbuttcap%
\pgfsetroundjoin%
\pgfsetlinewidth{0.309003pt}%
\definecolor{currentstroke}{rgb}{0.268510,0.009605,0.335427}%
\pgfsetstrokecolor{currentstroke}%
\pgfsetdash{}{0pt}%
\pgfpathmoveto{\pgfqpoint{8.428248in}{1.852909in}}%
\pgfpathlineto{\pgfqpoint{8.427840in}{1.852931in}}%
\pgfusepath{stroke}%
\end{pgfscope}%
\begin{pgfscope}%
\pgfpathrectangle{\pgfqpoint{6.720588in}{1.750000in}}{\pgfqpoint{2.279412in}{2.004545in}}%
\pgfusepath{clip}%
\pgfsetbuttcap%
\pgfsetroundjoin%
\pgfsetlinewidth{0.308879pt}%
\definecolor{currentstroke}{rgb}{0.268510,0.009605,0.335427}%
\pgfsetstrokecolor{currentstroke}%
\pgfsetdash{}{0pt}%
\pgfpathmoveto{\pgfqpoint{8.427840in}{1.852931in}}%
\pgfpathlineto{\pgfqpoint{8.427234in}{1.852960in}}%
\pgfusepath{stroke}%
\end{pgfscope}%
\begin{pgfscope}%
\pgfpathrectangle{\pgfqpoint{6.720588in}{1.750000in}}{\pgfqpoint{2.279412in}{2.004545in}}%
\pgfusepath{clip}%
\pgfsetbuttcap%
\pgfsetroundjoin%
\pgfsetlinewidth{0.308693pt}%
\definecolor{currentstroke}{rgb}{0.268510,0.009605,0.335427}%
\pgfsetstrokecolor{currentstroke}%
\pgfsetdash{}{0pt}%
\pgfpathmoveto{\pgfqpoint{8.427234in}{1.852960in}}%
\pgfpathlineto{\pgfqpoint{8.427193in}{1.852958in}}%
\pgfusepath{stroke}%
\end{pgfscope}%
\begin{pgfscope}%
\pgfpathrectangle{\pgfqpoint{6.720588in}{1.750000in}}{\pgfqpoint{2.279412in}{2.004545in}}%
\pgfusepath{clip}%
\pgfsetbuttcap%
\pgfsetroundjoin%
\pgfsetlinewidth{0.308681pt}%
\definecolor{currentstroke}{rgb}{0.268510,0.009605,0.335427}%
\pgfsetstrokecolor{currentstroke}%
\pgfsetdash{}{0pt}%
\pgfpathmoveto{\pgfqpoint{8.427193in}{1.852958in}}%
\pgfpathlineto{\pgfqpoint{8.427769in}{1.852928in}}%
\pgfusepath{stroke}%
\end{pgfscope}%
\begin{pgfscope}%
\pgfpathrectangle{\pgfqpoint{6.720588in}{1.750000in}}{\pgfqpoint{2.279412in}{2.004545in}}%
\pgfusepath{clip}%
\pgfsetbuttcap%
\pgfsetroundjoin%
\pgfsetlinewidth{0.308857pt}%
\definecolor{currentstroke}{rgb}{0.268510,0.009605,0.335427}%
\pgfsetstrokecolor{currentstroke}%
\pgfsetdash{}{0pt}%
\pgfpathmoveto{\pgfqpoint{8.427769in}{1.852928in}}%
\pgfpathlineto{\pgfqpoint{8.428277in}{1.852904in}}%
\pgfusepath{stroke}%
\end{pgfscope}%
\begin{pgfscope}%
\pgfpathrectangle{\pgfqpoint{6.720588in}{1.750000in}}{\pgfqpoint{2.279412in}{2.004545in}}%
\pgfusepath{clip}%
\pgfsetbuttcap%
\pgfsetroundjoin%
\pgfsetlinewidth{0.309011pt}%
\definecolor{currentstroke}{rgb}{0.268510,0.009605,0.335427}%
\pgfsetstrokecolor{currentstroke}%
\pgfsetdash{}{0pt}%
\pgfpathmoveto{\pgfqpoint{8.428277in}{1.852904in}}%
\pgfpathlineto{\pgfqpoint{8.428342in}{1.852904in}}%
\pgfusepath{stroke}%
\end{pgfscope}%
\begin{pgfscope}%
\pgfpathrectangle{\pgfqpoint{6.720588in}{1.750000in}}{\pgfqpoint{2.279412in}{2.004545in}}%
\pgfusepath{clip}%
\pgfsetbuttcap%
\pgfsetroundjoin%
\pgfsetlinewidth{0.309031pt}%
\definecolor{currentstroke}{rgb}{0.268510,0.009605,0.335427}%
\pgfsetstrokecolor{currentstroke}%
\pgfsetdash{}{0pt}%
\pgfpathmoveto{\pgfqpoint{8.428342in}{1.852904in}}%
\pgfpathlineto{\pgfqpoint{8.427847in}{1.852931in}}%
\pgfusepath{stroke}%
\end{pgfscope}%
\begin{pgfscope}%
\pgfpathrectangle{\pgfqpoint{6.720588in}{1.750000in}}{\pgfqpoint{2.279412in}{2.004545in}}%
\pgfusepath{clip}%
\pgfsetbuttcap%
\pgfsetroundjoin%
\pgfsetlinewidth{0.308881pt}%
\definecolor{currentstroke}{rgb}{0.268510,0.009605,0.335427}%
\pgfsetstrokecolor{currentstroke}%
\pgfsetdash{}{0pt}%
\pgfpathmoveto{\pgfqpoint{8.427847in}{1.852931in}}%
\pgfpathlineto{\pgfqpoint{8.427064in}{1.852968in}}%
\pgfusepath{stroke}%
\end{pgfscope}%
\begin{pgfscope}%
\pgfpathrectangle{\pgfqpoint{6.720588in}{1.750000in}}{\pgfqpoint{2.279412in}{2.004545in}}%
\pgfusepath{clip}%
\pgfsetbuttcap%
\pgfsetroundjoin%
\pgfsetlinewidth{0.308641pt}%
\definecolor{currentstroke}{rgb}{0.268510,0.009605,0.335427}%
\pgfsetstrokecolor{currentstroke}%
\pgfsetdash{}{0pt}%
\pgfpathmoveto{\pgfqpoint{8.427064in}{1.852968in}}%
\pgfpathlineto{\pgfqpoint{8.427064in}{1.852968in}}%
\pgfusepath{stroke}%
\end{pgfscope}%
\begin{pgfscope}%
\pgfpathrectangle{\pgfqpoint{6.720588in}{1.750000in}}{\pgfqpoint{2.279412in}{2.004545in}}%
\pgfusepath{clip}%
\pgfsetbuttcap%
\pgfsetroundjoin%
\pgfsetlinewidth{0.308641pt}%
\definecolor{currentstroke}{rgb}{0.268510,0.009605,0.335427}%
\pgfsetstrokecolor{currentstroke}%
\pgfsetdash{}{0pt}%
\pgfpathmoveto{\pgfqpoint{8.427064in}{1.852968in}}%
\pgfpathlineto{\pgfqpoint{8.427782in}{1.852930in}}%
\pgfusepath{stroke}%
\end{pgfscope}%
\begin{pgfscope}%
\pgfpathrectangle{\pgfqpoint{6.720588in}{1.750000in}}{\pgfqpoint{2.279412in}{2.004545in}}%
\pgfusepath{clip}%
\pgfsetbuttcap%
\pgfsetroundjoin%
\pgfsetlinewidth{0.308861pt}%
\definecolor{currentstroke}{rgb}{0.268510,0.009605,0.335427}%
\pgfsetstrokecolor{currentstroke}%
\pgfsetdash{}{0pt}%
\pgfpathmoveto{\pgfqpoint{8.427782in}{1.852930in}}%
\pgfpathlineto{\pgfqpoint{8.428360in}{1.852902in}}%
\pgfusepath{stroke}%
\end{pgfscope}%
\begin{pgfscope}%
\pgfpathrectangle{\pgfqpoint{6.720588in}{1.750000in}}{\pgfqpoint{2.279412in}{2.004545in}}%
\pgfusepath{clip}%
\pgfsetbuttcap%
\pgfsetroundjoin%
\pgfsetlinewidth{0.309037pt}%
\definecolor{currentstroke}{rgb}{0.268510,0.009605,0.335427}%
\pgfsetstrokecolor{currentstroke}%
\pgfsetdash{}{0pt}%
\pgfpathmoveto{\pgfqpoint{8.428360in}{1.852902in}}%
\pgfpathlineto{\pgfqpoint{8.428418in}{1.852902in}}%
\pgfusepath{stroke}%
\end{pgfscope}%
\begin{pgfscope}%
\pgfpathrectangle{\pgfqpoint{6.720588in}{1.750000in}}{\pgfqpoint{2.279412in}{2.004545in}}%
\pgfusepath{clip}%
\pgfsetbuttcap%
\pgfsetroundjoin%
\pgfsetlinewidth{0.309055pt}%
\definecolor{currentstroke}{rgb}{0.268510,0.009605,0.335427}%
\pgfsetstrokecolor{currentstroke}%
\pgfsetdash{}{0pt}%
\pgfpathmoveto{\pgfqpoint{8.428418in}{1.852902in}}%
\pgfpathlineto{\pgfqpoint{8.427824in}{1.852934in}}%
\pgfusepath{stroke}%
\end{pgfscope}%
\begin{pgfscope}%
\pgfpathrectangle{\pgfqpoint{6.720588in}{1.750000in}}{\pgfqpoint{2.279412in}{2.004545in}}%
\pgfusepath{clip}%
\pgfsetbuttcap%
\pgfsetroundjoin%
\pgfsetlinewidth{0.308874pt}%
\definecolor{currentstroke}{rgb}{0.268510,0.009605,0.335427}%
\pgfsetstrokecolor{currentstroke}%
\pgfsetdash{}{0pt}%
\pgfpathmoveto{\pgfqpoint{8.427824in}{1.852934in}}%
\pgfpathlineto{\pgfqpoint{8.427824in}{1.852934in}}%
\pgfusepath{stroke}%
\end{pgfscope}%
\begin{pgfscope}%
\pgfpathrectangle{\pgfqpoint{6.720588in}{1.750000in}}{\pgfqpoint{2.279412in}{2.004545in}}%
\pgfusepath{clip}%
\pgfsetbuttcap%
\pgfsetroundjoin%
\pgfsetlinewidth{0.308874pt}%
\definecolor{currentstroke}{rgb}{0.268510,0.009605,0.335427}%
\pgfsetstrokecolor{currentstroke}%
\pgfsetdash{}{0pt}%
\pgfpathmoveto{\pgfqpoint{8.427824in}{1.852934in}}%
\pgfpathlineto{\pgfqpoint{8.427705in}{1.852939in}}%
\pgfusepath{stroke}%
\end{pgfscope}%
\begin{pgfscope}%
\pgfpathrectangle{\pgfqpoint{6.720588in}{1.750000in}}{\pgfqpoint{2.279412in}{2.004545in}}%
\pgfusepath{clip}%
\pgfsetbuttcap%
\pgfsetroundjoin%
\pgfsetlinewidth{0.308838pt}%
\definecolor{currentstroke}{rgb}{0.268510,0.009605,0.335427}%
\pgfsetstrokecolor{currentstroke}%
\pgfsetdash{}{0pt}%
\pgfpathmoveto{\pgfqpoint{8.427705in}{1.852939in}}%
\pgfpathlineto{\pgfqpoint{8.427688in}{1.852939in}}%
\pgfusepath{stroke}%
\end{pgfscope}%
\begin{pgfscope}%
\pgfpathrectangle{\pgfqpoint{6.720588in}{1.750000in}}{\pgfqpoint{2.279412in}{2.004545in}}%
\pgfusepath{clip}%
\pgfsetbuttcap%
\pgfsetroundjoin%
\pgfsetlinewidth{0.308832pt}%
\definecolor{currentstroke}{rgb}{0.268510,0.009605,0.335427}%
\pgfsetstrokecolor{currentstroke}%
\pgfsetdash{}{0pt}%
\pgfpathmoveto{\pgfqpoint{8.427688in}{1.852939in}}%
\pgfpathlineto{\pgfqpoint{8.427799in}{1.852932in}}%
\pgfusepath{stroke}%
\end{pgfscope}%
\begin{pgfscope}%
\pgfpathrectangle{\pgfqpoint{6.720588in}{1.750000in}}{\pgfqpoint{2.279412in}{2.004545in}}%
\pgfusepath{clip}%
\pgfsetbuttcap%
\pgfsetroundjoin%
\pgfsetlinewidth{0.308866pt}%
\definecolor{currentstroke}{rgb}{0.268510,0.009605,0.335427}%
\pgfsetstrokecolor{currentstroke}%
\pgfsetdash{}{0pt}%
\pgfpathmoveto{\pgfqpoint{8.427799in}{1.852932in}}%
\pgfpathlineto{\pgfqpoint{8.427927in}{1.852926in}}%
\pgfusepath{stroke}%
\end{pgfscope}%
\begin{pgfscope}%
\pgfpathrectangle{\pgfqpoint{6.720588in}{1.750000in}}{\pgfqpoint{2.279412in}{2.004545in}}%
\pgfusepath{clip}%
\pgfsetbuttcap%
\pgfsetroundjoin%
\pgfsetlinewidth{0.308905pt}%
\definecolor{currentstroke}{rgb}{0.268510,0.009605,0.335427}%
\pgfsetstrokecolor{currentstroke}%
\pgfsetdash{}{0pt}%
\pgfpathmoveto{\pgfqpoint{8.427927in}{1.852926in}}%
\pgfpathlineto{\pgfqpoint{8.427946in}{1.852926in}}%
\pgfusepath{stroke}%
\end{pgfscope}%
\begin{pgfscope}%
\pgfpathrectangle{\pgfqpoint{6.720588in}{1.750000in}}{\pgfqpoint{2.279412in}{2.004545in}}%
\pgfusepath{clip}%
\pgfsetbuttcap%
\pgfsetroundjoin%
\pgfsetlinewidth{0.308911pt}%
\definecolor{currentstroke}{rgb}{0.268510,0.009605,0.335427}%
\pgfsetstrokecolor{currentstroke}%
\pgfsetdash{}{0pt}%
\pgfpathmoveto{\pgfqpoint{8.427946in}{1.852926in}}%
\pgfpathlineto{\pgfqpoint{8.427826in}{1.852932in}}%
\pgfusepath{stroke}%
\end{pgfscope}%
\begin{pgfscope}%
\pgfpathrectangle{\pgfqpoint{6.720588in}{1.750000in}}{\pgfqpoint{2.279412in}{2.004545in}}%
\pgfusepath{clip}%
\pgfsetbuttcap%
\pgfsetroundjoin%
\pgfsetlinewidth{0.308875pt}%
\definecolor{currentstroke}{rgb}{0.268510,0.009605,0.335427}%
\pgfsetstrokecolor{currentstroke}%
\pgfsetdash{}{0pt}%
\pgfpathmoveto{\pgfqpoint{8.427826in}{1.852932in}}%
\pgfpathlineto{\pgfqpoint{8.427672in}{1.852939in}}%
\pgfusepath{stroke}%
\end{pgfscope}%
\begin{pgfscope}%
\pgfpathrectangle{\pgfqpoint{6.720588in}{1.750000in}}{\pgfqpoint{2.279412in}{2.004545in}}%
\pgfusepath{clip}%
\pgfsetbuttcap%
\pgfsetroundjoin%
\pgfsetlinewidth{0.308827pt}%
\definecolor{currentstroke}{rgb}{0.268510,0.009605,0.335427}%
\pgfsetstrokecolor{currentstroke}%
\pgfsetdash{}{0pt}%
\pgfpathmoveto{\pgfqpoint{8.427672in}{1.852939in}}%
\pgfpathlineto{\pgfqpoint{8.427651in}{1.852939in}}%
\pgfusepath{stroke}%
\end{pgfscope}%
\begin{pgfscope}%
\pgfpathrectangle{\pgfqpoint{6.720588in}{1.750000in}}{\pgfqpoint{2.279412in}{2.004545in}}%
\pgfusepath{clip}%
\pgfsetbuttcap%
\pgfsetroundjoin%
\pgfsetlinewidth{0.308821pt}%
\definecolor{currentstroke}{rgb}{0.268510,0.009605,0.335427}%
\pgfsetstrokecolor{currentstroke}%
\pgfsetdash{}{0pt}%
\pgfpathmoveto{\pgfqpoint{8.427651in}{1.852939in}}%
\pgfpathlineto{\pgfqpoint{8.427797in}{1.852931in}}%
\pgfusepath{stroke}%
\end{pgfscope}%
\begin{pgfscope}%
\pgfpathrectangle{\pgfqpoint{6.720588in}{1.750000in}}{\pgfqpoint{2.279412in}{2.004545in}}%
\pgfusepath{clip}%
\pgfsetbuttcap%
\pgfsetroundjoin%
\pgfsetlinewidth{0.308865pt}%
\definecolor{currentstroke}{rgb}{0.268510,0.009605,0.335427}%
\pgfsetstrokecolor{currentstroke}%
\pgfsetdash{}{0pt}%
\pgfpathmoveto{\pgfqpoint{8.427797in}{1.852931in}}%
\pgfpathlineto{\pgfqpoint{8.427959in}{1.852924in}}%
\pgfusepath{stroke}%
\end{pgfscope}%
\begin{pgfscope}%
\pgfpathrectangle{\pgfqpoint{6.720588in}{1.750000in}}{\pgfqpoint{2.279412in}{2.004545in}}%
\pgfusepath{clip}%
\pgfsetbuttcap%
\pgfsetroundjoin%
\pgfsetlinewidth{0.308915pt}%
\definecolor{currentstroke}{rgb}{0.268510,0.009605,0.335427}%
\pgfsetstrokecolor{currentstroke}%
\pgfsetdash{}{0pt}%
\pgfpathmoveto{\pgfqpoint{8.427959in}{1.852924in}}%
\pgfpathlineto{\pgfqpoint{8.427982in}{1.852923in}}%
\pgfusepath{stroke}%
\end{pgfscope}%
\begin{pgfscope}%
\pgfpathrectangle{\pgfqpoint{6.720588in}{1.750000in}}{\pgfqpoint{2.279412in}{2.004545in}}%
\pgfusepath{clip}%
\pgfsetbuttcap%
\pgfsetroundjoin%
\pgfsetlinewidth{0.308922pt}%
\definecolor{currentstroke}{rgb}{0.268510,0.009605,0.335427}%
\pgfsetstrokecolor{currentstroke}%
\pgfsetdash{}{0pt}%
\pgfpathmoveto{\pgfqpoint{8.427982in}{1.852923in}}%
\pgfpathlineto{\pgfqpoint{8.427829in}{1.852932in}}%
\pgfusepath{stroke}%
\end{pgfscope}%
\begin{pgfscope}%
\pgfpathrectangle{\pgfqpoint{6.720588in}{1.750000in}}{\pgfqpoint{2.279412in}{2.004545in}}%
\pgfusepath{clip}%
\pgfsetbuttcap%
\pgfsetroundjoin%
\pgfsetlinewidth{0.308875pt}%
\definecolor{currentstroke}{rgb}{0.268510,0.009605,0.335427}%
\pgfsetstrokecolor{currentstroke}%
\pgfsetdash{}{0pt}%
\pgfpathmoveto{\pgfqpoint{8.427829in}{1.852932in}}%
\pgfpathlineto{\pgfqpoint{8.427628in}{1.852941in}}%
\pgfusepath{stroke}%
\end{pgfscope}%
\begin{pgfscope}%
\pgfpathrectangle{\pgfqpoint{6.720588in}{1.750000in}}{\pgfqpoint{2.279412in}{2.004545in}}%
\pgfusepath{clip}%
\pgfsetbuttcap%
\pgfsetroundjoin%
\pgfsetlinewidth{0.308814pt}%
\definecolor{currentstroke}{rgb}{0.268510,0.009605,0.335427}%
\pgfsetstrokecolor{currentstroke}%
\pgfsetdash{}{0pt}%
\pgfpathmoveto{\pgfqpoint{8.427628in}{1.852941in}}%
\pgfpathlineto{\pgfqpoint{8.427603in}{1.852941in}}%
\pgfusepath{stroke}%
\end{pgfscope}%
\begin{pgfscope}%
\pgfpathrectangle{\pgfqpoint{6.720588in}{1.750000in}}{\pgfqpoint{2.279412in}{2.004545in}}%
\pgfusepath{clip}%
\pgfsetbuttcap%
\pgfsetroundjoin%
\pgfsetlinewidth{0.308806pt}%
\definecolor{currentstroke}{rgb}{0.268510,0.009605,0.335427}%
\pgfsetstrokecolor{currentstroke}%
\pgfsetdash{}{0pt}%
\pgfpathmoveto{\pgfqpoint{8.427603in}{1.852941in}}%
\pgfpathlineto{\pgfqpoint{8.427793in}{1.852931in}}%
\pgfusepath{stroke}%
\end{pgfscope}%
\begin{pgfscope}%
\pgfpathrectangle{\pgfqpoint{6.720588in}{1.750000in}}{\pgfqpoint{2.279412in}{2.004545in}}%
\pgfusepath{clip}%
\pgfsetbuttcap%
\pgfsetroundjoin%
\pgfsetlinewidth{0.308864pt}%
\definecolor{currentstroke}{rgb}{0.268510,0.009605,0.335427}%
\pgfsetstrokecolor{currentstroke}%
\pgfsetdash{}{0pt}%
\pgfpathmoveto{\pgfqpoint{8.427793in}{1.852931in}}%
\pgfpathlineto{\pgfqpoint{8.427999in}{1.852921in}}%
\pgfusepath{stroke}%
\end{pgfscope}%
\begin{pgfscope}%
\pgfpathrectangle{\pgfqpoint{6.720588in}{1.750000in}}{\pgfqpoint{2.279412in}{2.004545in}}%
\pgfusepath{clip}%
\pgfsetbuttcap%
\pgfsetroundjoin%
\pgfsetlinewidth{0.308927pt}%
\definecolor{currentstroke}{rgb}{0.268510,0.009605,0.335427}%
\pgfsetstrokecolor{currentstroke}%
\pgfsetdash{}{0pt}%
\pgfpathmoveto{\pgfqpoint{8.427999in}{1.852921in}}%
\pgfpathlineto{\pgfqpoint{8.428028in}{1.852921in}}%
\pgfusepath{stroke}%
\end{pgfscope}%
\begin{pgfscope}%
\pgfpathrectangle{\pgfqpoint{6.720588in}{1.750000in}}{\pgfqpoint{2.279412in}{2.004545in}}%
\pgfusepath{clip}%
\pgfsetbuttcap%
\pgfsetroundjoin%
\pgfsetlinewidth{0.308936pt}%
\definecolor{currentstroke}{rgb}{0.268510,0.009605,0.335427}%
\pgfsetstrokecolor{currentstroke}%
\pgfsetdash{}{0pt}%
\pgfpathmoveto{\pgfqpoint{8.428028in}{1.852921in}}%
\pgfpathlineto{\pgfqpoint{8.427832in}{1.852931in}}%
\pgfusepath{stroke}%
\end{pgfscope}%
\begin{pgfscope}%
\pgfpathrectangle{\pgfqpoint{6.720588in}{1.750000in}}{\pgfqpoint{2.279412in}{2.004545in}}%
\pgfusepath{clip}%
\pgfsetbuttcap%
\pgfsetroundjoin%
\pgfsetlinewidth{0.308876pt}%
\definecolor{currentstroke}{rgb}{0.268510,0.009605,0.335427}%
\pgfsetstrokecolor{currentstroke}%
\pgfsetdash{}{0pt}%
\pgfpathmoveto{\pgfqpoint{8.427832in}{1.852931in}}%
\pgfpathlineto{\pgfqpoint{8.427570in}{1.852944in}}%
\pgfusepath{stroke}%
\end{pgfscope}%
\begin{pgfscope}%
\pgfpathrectangle{\pgfqpoint{6.720588in}{1.750000in}}{\pgfqpoint{2.279412in}{2.004545in}}%
\pgfusepath{clip}%
\pgfsetbuttcap%
\pgfsetroundjoin%
\pgfsetlinewidth{0.308796pt}%
\definecolor{currentstroke}{rgb}{0.268510,0.009605,0.335427}%
\pgfsetstrokecolor{currentstroke}%
\pgfsetdash{}{0pt}%
\pgfpathmoveto{\pgfqpoint{8.427570in}{1.852944in}}%
\pgfpathlineto{\pgfqpoint{8.427540in}{1.852944in}}%
\pgfusepath{stroke}%
\end{pgfscope}%
\begin{pgfscope}%
\pgfpathrectangle{\pgfqpoint{6.720588in}{1.750000in}}{\pgfqpoint{2.279412in}{2.004545in}}%
\pgfusepath{clip}%
\pgfsetbuttcap%
\pgfsetroundjoin%
\pgfsetlinewidth{0.308787pt}%
\definecolor{currentstroke}{rgb}{0.268510,0.009605,0.335427}%
\pgfsetstrokecolor{currentstroke}%
\pgfsetdash{}{0pt}%
\pgfpathmoveto{\pgfqpoint{8.427540in}{1.852944in}}%
\pgfpathlineto{\pgfqpoint{8.427789in}{1.852931in}}%
\pgfusepath{stroke}%
\end{pgfscope}%
\begin{pgfscope}%
\pgfpathrectangle{\pgfqpoint{6.720588in}{1.750000in}}{\pgfqpoint{2.279412in}{2.004545in}}%
\pgfusepath{clip}%
\pgfsetbuttcap%
\pgfsetroundjoin%
\pgfsetlinewidth{0.308863pt}%
\definecolor{currentstroke}{rgb}{0.268510,0.009605,0.335427}%
\pgfsetstrokecolor{currentstroke}%
\pgfsetdash{}{0pt}%
\pgfpathmoveto{\pgfqpoint{8.427789in}{1.852931in}}%
\pgfpathlineto{\pgfqpoint{8.428049in}{1.852918in}}%
\pgfusepath{stroke}%
\end{pgfscope}%
\begin{pgfscope}%
\pgfpathrectangle{\pgfqpoint{6.720588in}{1.750000in}}{\pgfqpoint{2.279412in}{2.004545in}}%
\pgfusepath{clip}%
\pgfsetbuttcap%
\pgfsetroundjoin%
\pgfsetlinewidth{0.308942pt}%
\definecolor{currentstroke}{rgb}{0.268510,0.009605,0.335427}%
\pgfsetstrokecolor{currentstroke}%
\pgfsetdash{}{0pt}%
\pgfpathmoveto{\pgfqpoint{8.428049in}{1.852918in}}%
\pgfpathlineto{\pgfqpoint{8.428085in}{1.852918in}}%
\pgfusepath{stroke}%
\end{pgfscope}%
\begin{pgfscope}%
\pgfpathrectangle{\pgfqpoint{6.720588in}{1.750000in}}{\pgfqpoint{2.279412in}{2.004545in}}%
\pgfusepath{clip}%
\pgfsetbuttcap%
\pgfsetroundjoin%
\pgfsetlinewidth{0.308953pt}%
\definecolor{currentstroke}{rgb}{0.268510,0.009605,0.335427}%
\pgfsetstrokecolor{currentstroke}%
\pgfsetdash{}{0pt}%
\pgfpathmoveto{\pgfqpoint{8.428085in}{1.852918in}}%
\pgfpathlineto{\pgfqpoint{8.427835in}{1.852931in}}%
\pgfusepath{stroke}%
\end{pgfscope}%
\begin{pgfscope}%
\pgfpathrectangle{\pgfqpoint{6.720588in}{1.750000in}}{\pgfqpoint{2.279412in}{2.004545in}}%
\pgfusepath{clip}%
\pgfsetbuttcap%
\pgfsetroundjoin%
\pgfsetlinewidth{0.308877pt}%
\definecolor{currentstroke}{rgb}{0.268510,0.009605,0.335427}%
\pgfsetstrokecolor{currentstroke}%
\pgfsetdash{}{0pt}%
\pgfpathmoveto{\pgfqpoint{8.427835in}{1.852931in}}%
\pgfpathlineto{\pgfqpoint{8.427493in}{1.852948in}}%
\pgfusepath{stroke}%
\end{pgfscope}%
\begin{pgfscope}%
\pgfpathrectangle{\pgfqpoint{6.720588in}{1.750000in}}{\pgfqpoint{2.279412in}{2.004545in}}%
\pgfusepath{clip}%
\pgfsetbuttcap%
\pgfsetroundjoin%
\pgfsetlinewidth{0.308773pt}%
\definecolor{currentstroke}{rgb}{0.268510,0.009605,0.335427}%
\pgfsetstrokecolor{currentstroke}%
\pgfsetdash{}{0pt}%
\pgfpathmoveto{\pgfqpoint{8.427493in}{1.852948in}}%
\pgfpathlineto{\pgfqpoint{8.427457in}{1.852948in}}%
\pgfusepath{stroke}%
\end{pgfscope}%
\begin{pgfscope}%
\pgfpathrectangle{\pgfqpoint{6.720588in}{1.750000in}}{\pgfqpoint{2.279412in}{2.004545in}}%
\pgfusepath{clip}%
\pgfsetbuttcap%
\pgfsetroundjoin%
\pgfsetlinewidth{0.308762pt}%
\definecolor{currentstroke}{rgb}{0.268510,0.009605,0.335427}%
\pgfsetstrokecolor{currentstroke}%
\pgfsetdash{}{0pt}%
\pgfpathmoveto{\pgfqpoint{8.427457in}{1.852948in}}%
\pgfpathlineto{\pgfqpoint{8.427783in}{1.852930in}}%
\pgfusepath{stroke}%
\end{pgfscope}%
\begin{pgfscope}%
\pgfpathrectangle{\pgfqpoint{6.720588in}{1.750000in}}{\pgfqpoint{2.279412in}{2.004545in}}%
\pgfusepath{clip}%
\pgfsetbuttcap%
\pgfsetroundjoin%
\pgfsetlinewidth{0.308861pt}%
\definecolor{currentstroke}{rgb}{0.268510,0.009605,0.335427}%
\pgfsetstrokecolor{currentstroke}%
\pgfsetdash{}{0pt}%
\pgfpathmoveto{\pgfqpoint{8.427783in}{1.852930in}}%
\pgfpathlineto{\pgfqpoint{8.428110in}{1.852915in}}%
\pgfusepath{stroke}%
\end{pgfscope}%
\begin{pgfscope}%
\pgfpathrectangle{\pgfqpoint{6.720588in}{1.750000in}}{\pgfqpoint{2.279412in}{2.004545in}}%
\pgfusepath{clip}%
\pgfsetbuttcap%
\pgfsetroundjoin%
\pgfsetlinewidth{0.308961pt}%
\definecolor{currentstroke}{rgb}{0.268510,0.009605,0.335427}%
\pgfsetstrokecolor{currentstroke}%
\pgfsetdash{}{0pt}%
\pgfpathmoveto{\pgfqpoint{8.428110in}{1.852915in}}%
\pgfpathlineto{\pgfqpoint{8.428153in}{1.852914in}}%
\pgfusepath{stroke}%
\end{pgfscope}%
\begin{pgfscope}%
\pgfpathrectangle{\pgfqpoint{6.720588in}{1.750000in}}{\pgfqpoint{2.279412in}{2.004545in}}%
\pgfusepath{clip}%
\pgfsetbuttcap%
\pgfsetroundjoin%
\pgfsetlinewidth{0.308974pt}%
\definecolor{currentstroke}{rgb}{0.268510,0.009605,0.335427}%
\pgfsetstrokecolor{currentstroke}%
\pgfsetdash{}{0pt}%
\pgfpathmoveto{\pgfqpoint{8.428153in}{1.852914in}}%
\pgfpathlineto{\pgfqpoint{8.427839in}{1.852931in}}%
\pgfusepath{stroke}%
\end{pgfscope}%
\begin{pgfscope}%
\pgfpathrectangle{\pgfqpoint{6.720588in}{1.750000in}}{\pgfqpoint{2.279412in}{2.004545in}}%
\pgfusepath{clip}%
\pgfsetbuttcap%
\pgfsetroundjoin%
\pgfsetlinewidth{0.308879pt}%
\definecolor{currentstroke}{rgb}{0.268510,0.009605,0.335427}%
\pgfsetstrokecolor{currentstroke}%
\pgfsetdash{}{0pt}%
\pgfpathmoveto{\pgfqpoint{8.427839in}{1.852931in}}%
\pgfpathlineto{\pgfqpoint{8.427393in}{1.852953in}}%
\pgfusepath{stroke}%
\end{pgfscope}%
\begin{pgfscope}%
\pgfpathrectangle{\pgfqpoint{6.720588in}{1.750000in}}{\pgfqpoint{2.279412in}{2.004545in}}%
\pgfusepath{clip}%
\pgfsetbuttcap%
\pgfsetroundjoin%
\pgfsetlinewidth{0.308742pt}%
\definecolor{currentstroke}{rgb}{0.268510,0.009605,0.335427}%
\pgfsetstrokecolor{currentstroke}%
\pgfsetdash{}{0pt}%
\pgfpathmoveto{\pgfqpoint{8.427393in}{1.852953in}}%
\pgfpathlineto{\pgfqpoint{8.427351in}{1.852952in}}%
\pgfusepath{stroke}%
\end{pgfscope}%
\begin{pgfscope}%
\pgfpathrectangle{\pgfqpoint{6.720588in}{1.750000in}}{\pgfqpoint{2.279412in}{2.004545in}}%
\pgfusepath{clip}%
\pgfsetbuttcap%
\pgfsetroundjoin%
\pgfsetlinewidth{0.308729pt}%
\definecolor{currentstroke}{rgb}{0.268510,0.009605,0.335427}%
\pgfsetstrokecolor{currentstroke}%
\pgfsetdash{}{0pt}%
\pgfpathmoveto{\pgfqpoint{8.427351in}{1.852952in}}%
\pgfpathlineto{\pgfqpoint{8.427776in}{1.852930in}}%
\pgfusepath{stroke}%
\end{pgfscope}%
\begin{pgfscope}%
\pgfpathrectangle{\pgfqpoint{6.720588in}{1.750000in}}{\pgfqpoint{2.279412in}{2.004545in}}%
\pgfusepath{clip}%
\pgfsetbuttcap%
\pgfsetroundjoin%
\pgfsetlinewidth{0.308859pt}%
\definecolor{currentstroke}{rgb}{0.268510,0.009605,0.335427}%
\pgfsetstrokecolor{currentstroke}%
\pgfsetdash{}{0pt}%
\pgfpathmoveto{\pgfqpoint{8.427776in}{1.852930in}}%
\pgfpathlineto{\pgfqpoint{8.428180in}{1.852910in}}%
\pgfusepath{stroke}%
\end{pgfscope}%
\begin{pgfscope}%
\pgfpathrectangle{\pgfqpoint{6.720588in}{1.750000in}}{\pgfqpoint{2.279412in}{2.004545in}}%
\pgfusepath{clip}%
\pgfsetbuttcap%
\pgfsetroundjoin%
\pgfsetlinewidth{0.308982pt}%
\definecolor{currentstroke}{rgb}{0.268510,0.009605,0.335427}%
\pgfsetstrokecolor{currentstroke}%
\pgfsetdash{}{0pt}%
\pgfpathmoveto{\pgfqpoint{8.428180in}{1.852910in}}%
\pgfpathlineto{\pgfqpoint{8.428235in}{1.852910in}}%
\pgfusepath{stroke}%
\end{pgfscope}%
\begin{pgfscope}%
\pgfpathrectangle{\pgfqpoint{6.720588in}{1.750000in}}{\pgfqpoint{2.279412in}{2.004545in}}%
\pgfusepath{clip}%
\pgfsetbuttcap%
\pgfsetroundjoin%
\pgfsetlinewidth{0.308999pt}%
\definecolor{currentstroke}{rgb}{0.268510,0.009605,0.335427}%
\pgfsetstrokecolor{currentstroke}%
\pgfsetdash{}{0pt}%
\pgfpathmoveto{\pgfqpoint{8.428235in}{1.852910in}}%
\pgfpathlineto{\pgfqpoint{8.427845in}{1.852931in}}%
\pgfusepath{stroke}%
\end{pgfscope}%
\begin{pgfscope}%
\pgfpathrectangle{\pgfqpoint{6.720588in}{1.750000in}}{\pgfqpoint{2.279412in}{2.004545in}}%
\pgfusepath{clip}%
\pgfsetbuttcap%
\pgfsetroundjoin%
\pgfsetlinewidth{0.308880pt}%
\definecolor{currentstroke}{rgb}{0.268510,0.009605,0.335427}%
\pgfsetstrokecolor{currentstroke}%
\pgfsetdash{}{0pt}%
\pgfpathmoveto{\pgfqpoint{8.427845in}{1.852931in}}%
\pgfpathlineto{\pgfqpoint{8.427263in}{1.852959in}}%
\pgfusepath{stroke}%
\end{pgfscope}%
\begin{pgfscope}%
\pgfpathrectangle{\pgfqpoint{6.720588in}{1.750000in}}{\pgfqpoint{2.279412in}{2.004545in}}%
\pgfusepath{clip}%
\pgfsetbuttcap%
\pgfsetroundjoin%
\pgfsetlinewidth{0.308702pt}%
\definecolor{currentstroke}{rgb}{0.268510,0.009605,0.335427}%
\pgfsetstrokecolor{currentstroke}%
\pgfsetdash{}{0pt}%
\pgfpathmoveto{\pgfqpoint{8.427263in}{1.852959in}}%
\pgfpathlineto{\pgfqpoint{8.427216in}{1.852958in}}%
\pgfusepath{stroke}%
\end{pgfscope}%
\begin{pgfscope}%
\pgfpathrectangle{\pgfqpoint{6.720588in}{1.750000in}}{\pgfqpoint{2.279412in}{2.004545in}}%
\pgfusepath{clip}%
\pgfsetbuttcap%
\pgfsetroundjoin%
\pgfsetlinewidth{0.308688pt}%
\definecolor{currentstroke}{rgb}{0.268510,0.009605,0.335427}%
\pgfsetstrokecolor{currentstroke}%
\pgfsetdash{}{0pt}%
\pgfpathmoveto{\pgfqpoint{8.427216in}{1.852958in}}%
\pgfpathlineto{\pgfqpoint{8.427765in}{1.852929in}}%
\pgfusepath{stroke}%
\end{pgfscope}%
\begin{pgfscope}%
\pgfpathrectangle{\pgfqpoint{6.720588in}{1.750000in}}{\pgfqpoint{2.279412in}{2.004545in}}%
\pgfusepath{clip}%
\pgfsetbuttcap%
\pgfsetroundjoin%
\pgfsetlinewidth{0.308856pt}%
\definecolor{currentstroke}{rgb}{0.268510,0.009605,0.335427}%
\pgfsetstrokecolor{currentstroke}%
\pgfsetdash{}{0pt}%
\pgfpathmoveto{\pgfqpoint{8.427765in}{1.852929in}}%
\pgfpathlineto{\pgfqpoint{8.428259in}{1.852905in}}%
\pgfusepath{stroke}%
\end{pgfscope}%
\begin{pgfscope}%
\pgfpathrectangle{\pgfqpoint{6.720588in}{1.750000in}}{\pgfqpoint{2.279412in}{2.004545in}}%
\pgfusepath{clip}%
\pgfsetbuttcap%
\pgfsetroundjoin%
\pgfsetlinewidth{0.309006pt}%
\definecolor{currentstroke}{rgb}{0.268510,0.009605,0.335427}%
\pgfsetstrokecolor{currentstroke}%
\pgfsetdash{}{0pt}%
\pgfpathmoveto{\pgfqpoint{8.428259in}{1.852905in}}%
\pgfpathlineto{\pgfqpoint{8.428327in}{1.852905in}}%
\pgfusepath{stroke}%
\end{pgfscope}%
\begin{pgfscope}%
\pgfpathrectangle{\pgfqpoint{6.720588in}{1.750000in}}{\pgfqpoint{2.279412in}{2.004545in}}%
\pgfusepath{clip}%
\pgfsetbuttcap%
\pgfsetroundjoin%
\pgfsetlinewidth{0.309027pt}%
\definecolor{currentstroke}{rgb}{0.268510,0.009605,0.335427}%
\pgfsetstrokecolor{currentstroke}%
\pgfsetdash{}{0pt}%
\pgfpathmoveto{\pgfqpoint{8.428327in}{1.852905in}}%
\pgfpathlineto{\pgfqpoint{8.427852in}{1.852931in}}%
\pgfusepath{stroke}%
\end{pgfscope}%
\begin{pgfscope}%
\pgfpathrectangle{\pgfqpoint{6.720588in}{1.750000in}}{\pgfqpoint{2.279412in}{2.004545in}}%
\pgfusepath{clip}%
\pgfsetbuttcap%
\pgfsetroundjoin%
\pgfsetlinewidth{0.308883pt}%
\definecolor{currentstroke}{rgb}{0.268510,0.009605,0.335427}%
\pgfsetstrokecolor{currentstroke}%
\pgfsetdash{}{0pt}%
\pgfpathmoveto{\pgfqpoint{8.427852in}{1.852931in}}%
\pgfpathlineto{\pgfqpoint{8.427100in}{1.852967in}}%
\pgfusepath{stroke}%
\end{pgfscope}%
\begin{pgfscope}%
\pgfpathrectangle{\pgfqpoint{6.720588in}{1.750000in}}{\pgfqpoint{2.279412in}{2.004545in}}%
\pgfusepath{clip}%
\pgfsetbuttcap%
\pgfsetroundjoin%
\pgfsetlinewidth{0.308652pt}%
\definecolor{currentstroke}{rgb}{0.268510,0.009605,0.335427}%
\pgfsetstrokecolor{currentstroke}%
\pgfsetdash{}{0pt}%
\pgfpathmoveto{\pgfqpoint{8.427100in}{1.852967in}}%
\pgfpathlineto{\pgfqpoint{8.427100in}{1.852967in}}%
\pgfusepath{stroke}%
\end{pgfscope}%
\begin{pgfscope}%
\pgfpathrectangle{\pgfqpoint{6.720588in}{1.750000in}}{\pgfqpoint{2.279412in}{2.004545in}}%
\pgfusepath{clip}%
\pgfsetbuttcap%
\pgfsetroundjoin%
\pgfsetlinewidth{0.308652pt}%
\definecolor{currentstroke}{rgb}{0.268510,0.009605,0.335427}%
\pgfsetstrokecolor{currentstroke}%
\pgfsetdash{}{0pt}%
\pgfpathmoveto{\pgfqpoint{8.427100in}{1.852967in}}%
\pgfpathlineto{\pgfqpoint{8.427785in}{1.852930in}}%
\pgfusepath{stroke}%
\end{pgfscope}%
\begin{pgfscope}%
\pgfpathrectangle{\pgfqpoint{6.720588in}{1.750000in}}{\pgfqpoint{2.279412in}{2.004545in}}%
\pgfusepath{clip}%
\pgfsetbuttcap%
\pgfsetroundjoin%
\pgfsetlinewidth{0.308862pt}%
\definecolor{currentstroke}{rgb}{0.268510,0.009605,0.335427}%
\pgfsetstrokecolor{currentstroke}%
\pgfsetdash{}{0pt}%
\pgfpathmoveto{\pgfqpoint{8.427785in}{1.852930in}}%
\pgfpathlineto{\pgfqpoint{8.428342in}{1.852903in}}%
\pgfusepath{stroke}%
\end{pgfscope}%
\begin{pgfscope}%
\pgfpathrectangle{\pgfqpoint{6.720588in}{1.750000in}}{\pgfqpoint{2.279412in}{2.004545in}}%
\pgfusepath{clip}%
\pgfsetbuttcap%
\pgfsetroundjoin%
\pgfsetlinewidth{0.309031pt}%
\definecolor{currentstroke}{rgb}{0.268510,0.009605,0.335427}%
\pgfsetstrokecolor{currentstroke}%
\pgfsetdash{}{0pt}%
\pgfpathmoveto{\pgfqpoint{8.428342in}{1.852903in}}%
\pgfpathlineto{\pgfqpoint{8.428398in}{1.852903in}}%
\pgfusepath{stroke}%
\end{pgfscope}%
\begin{pgfscope}%
\pgfpathrectangle{\pgfqpoint{6.720588in}{1.750000in}}{\pgfqpoint{2.279412in}{2.004545in}}%
\pgfusepath{clip}%
\pgfsetbuttcap%
\pgfsetroundjoin%
\pgfsetlinewidth{0.309048pt}%
\definecolor{currentstroke}{rgb}{0.268510,0.009605,0.335427}%
\pgfsetstrokecolor{currentstroke}%
\pgfsetdash{}{0pt}%
\pgfpathmoveto{\pgfqpoint{8.428398in}{1.852903in}}%
\pgfpathlineto{\pgfqpoint{8.427822in}{1.852934in}}%
\pgfusepath{stroke}%
\end{pgfscope}%
\begin{pgfscope}%
\pgfpathrectangle{\pgfqpoint{6.720588in}{1.750000in}}{\pgfqpoint{2.279412in}{2.004545in}}%
\pgfusepath{clip}%
\pgfsetbuttcap%
\pgfsetroundjoin%
\pgfsetlinewidth{0.308873pt}%
\definecolor{currentstroke}{rgb}{0.268510,0.009605,0.335427}%
\pgfsetstrokecolor{currentstroke}%
\pgfsetdash{}{0pt}%
\pgfpathmoveto{\pgfqpoint{8.427822in}{1.852934in}}%
\pgfpathlineto{\pgfqpoint{8.427822in}{1.852934in}}%
\pgfusepath{stroke}%
\end{pgfscope}%
\begin{pgfscope}%
\pgfpathrectangle{\pgfqpoint{6.720588in}{1.750000in}}{\pgfqpoint{2.279412in}{2.004545in}}%
\pgfusepath{clip}%
\pgfsetbuttcap%
\pgfsetroundjoin%
\pgfsetlinewidth{0.308873pt}%
\definecolor{currentstroke}{rgb}{0.268510,0.009605,0.335427}%
\pgfsetstrokecolor{currentstroke}%
\pgfsetdash{}{0pt}%
\pgfpathmoveto{\pgfqpoint{8.427822in}{1.852934in}}%
\pgfpathlineto{\pgfqpoint{8.427709in}{1.852939in}}%
\pgfusepath{stroke}%
\end{pgfscope}%
\begin{pgfscope}%
\pgfpathrectangle{\pgfqpoint{6.720588in}{1.750000in}}{\pgfqpoint{2.279412in}{2.004545in}}%
\pgfusepath{clip}%
\pgfsetbuttcap%
\pgfsetroundjoin%
\pgfsetlinewidth{0.308839pt}%
\definecolor{currentstroke}{rgb}{0.268510,0.009605,0.335427}%
\pgfsetstrokecolor{currentstroke}%
\pgfsetdash{}{0pt}%
\pgfpathmoveto{\pgfqpoint{8.427709in}{1.852939in}}%
\pgfpathlineto{\pgfqpoint{8.427694in}{1.852938in}}%
\pgfusepath{stroke}%
\end{pgfscope}%
\begin{pgfscope}%
\pgfpathrectangle{\pgfqpoint{6.720588in}{1.750000in}}{\pgfqpoint{2.279412in}{2.004545in}}%
\pgfusepath{clip}%
\pgfsetbuttcap%
\pgfsetroundjoin%
\pgfsetlinewidth{0.308834pt}%
\definecolor{currentstroke}{rgb}{0.268510,0.009605,0.335427}%
\pgfsetstrokecolor{currentstroke}%
\pgfsetdash{}{0pt}%
\pgfpathmoveto{\pgfqpoint{8.427694in}{1.852938in}}%
\pgfpathlineto{\pgfqpoint{8.427801in}{1.852932in}}%
\pgfusepath{stroke}%
\end{pgfscope}%
\begin{pgfscope}%
\pgfpathrectangle{\pgfqpoint{6.720588in}{1.750000in}}{\pgfqpoint{2.279412in}{2.004545in}}%
\pgfusepath{clip}%
\pgfsetbuttcap%
\pgfsetroundjoin%
\pgfsetlinewidth{0.308867pt}%
\definecolor{currentstroke}{rgb}{0.268510,0.009605,0.335427}%
\pgfsetstrokecolor{currentstroke}%
\pgfsetdash{}{0pt}%
\pgfpathmoveto{\pgfqpoint{8.427801in}{1.852932in}}%
\pgfpathlineto{\pgfqpoint{8.427922in}{1.852926in}}%
\pgfusepath{stroke}%
\end{pgfscope}%
\begin{pgfscope}%
\pgfpathrectangle{\pgfqpoint{6.720588in}{1.750000in}}{\pgfqpoint{2.279412in}{2.004545in}}%
\pgfusepath{clip}%
\pgfsetbuttcap%
\pgfsetroundjoin%
\pgfsetlinewidth{0.308904pt}%
\definecolor{currentstroke}{rgb}{0.268510,0.009605,0.335427}%
\pgfsetstrokecolor{currentstroke}%
\pgfsetdash{}{0pt}%
\pgfpathmoveto{\pgfqpoint{8.427922in}{1.852926in}}%
\pgfpathlineto{\pgfqpoint{8.427939in}{1.852926in}}%
\pgfusepath{stroke}%
\end{pgfscope}%
\begin{pgfscope}%
\pgfpathrectangle{\pgfqpoint{6.720588in}{1.750000in}}{\pgfqpoint{2.279412in}{2.004545in}}%
\pgfusepath{clip}%
\pgfsetbuttcap%
\pgfsetroundjoin%
\pgfsetlinewidth{0.308909pt}%
\definecolor{currentstroke}{rgb}{0.268510,0.009605,0.335427}%
\pgfsetstrokecolor{currentstroke}%
\pgfsetdash{}{0pt}%
\pgfpathmoveto{\pgfqpoint{8.427939in}{1.852926in}}%
\pgfpathlineto{\pgfqpoint{8.427824in}{1.852932in}}%
\pgfusepath{stroke}%
\end{pgfscope}%
\begin{pgfscope}%
\pgfpathrectangle{\pgfqpoint{6.720588in}{1.750000in}}{\pgfqpoint{2.279412in}{2.004545in}}%
\pgfusepath{clip}%
\pgfsetbuttcap%
\pgfsetroundjoin%
\pgfsetlinewidth{0.308874pt}%
\definecolor{currentstroke}{rgb}{0.268510,0.009605,0.335427}%
\pgfsetstrokecolor{currentstroke}%
\pgfsetdash{}{0pt}%
\pgfpathmoveto{\pgfqpoint{8.427824in}{1.852932in}}%
\pgfpathlineto{\pgfqpoint{8.427678in}{1.852939in}}%
\pgfusepath{stroke}%
\end{pgfscope}%
\begin{pgfscope}%
\pgfpathrectangle{\pgfqpoint{6.720588in}{1.750000in}}{\pgfqpoint{2.279412in}{2.004545in}}%
\pgfusepath{clip}%
\pgfsetbuttcap%
\pgfsetroundjoin%
\pgfsetlinewidth{0.308829pt}%
\definecolor{currentstroke}{rgb}{0.268510,0.009605,0.335427}%
\pgfsetstrokecolor{currentstroke}%
\pgfsetdash{}{0pt}%
\pgfpathmoveto{\pgfqpoint{8.427678in}{1.852939in}}%
\pgfpathlineto{\pgfqpoint{8.427660in}{1.852939in}}%
\pgfusepath{stroke}%
\end{pgfscope}%
\begin{pgfscope}%
\pgfpathrectangle{\pgfqpoint{6.720588in}{1.750000in}}{\pgfqpoint{2.279412in}{2.004545in}}%
\pgfusepath{clip}%
\pgfsetbuttcap%
\pgfsetroundjoin%
\pgfsetlinewidth{0.308824pt}%
\definecolor{currentstroke}{rgb}{0.268510,0.009605,0.335427}%
\pgfsetstrokecolor{currentstroke}%
\pgfsetdash{}{0pt}%
\pgfpathmoveto{\pgfqpoint{8.427660in}{1.852939in}}%
\pgfpathlineto{\pgfqpoint{8.427799in}{1.852931in}}%
\pgfusepath{stroke}%
\end{pgfscope}%
\begin{pgfscope}%
\pgfpathrectangle{\pgfqpoint{6.720588in}{1.750000in}}{\pgfqpoint{2.279412in}{2.004545in}}%
\pgfusepath{clip}%
\pgfsetbuttcap%
\pgfsetroundjoin%
\pgfsetlinewidth{0.308866pt}%
\definecolor{currentstroke}{rgb}{0.268510,0.009605,0.335427}%
\pgfsetstrokecolor{currentstroke}%
\pgfsetdash{}{0pt}%
\pgfpathmoveto{\pgfqpoint{8.427799in}{1.852931in}}%
\pgfpathlineto{\pgfqpoint{8.427953in}{1.852924in}}%
\pgfusepath{stroke}%
\end{pgfscope}%
\begin{pgfscope}%
\pgfpathrectangle{\pgfqpoint{6.720588in}{1.750000in}}{\pgfqpoint{2.279412in}{2.004545in}}%
\pgfusepath{clip}%
\pgfsetbuttcap%
\pgfsetroundjoin%
\pgfsetlinewidth{0.308913pt}%
\definecolor{currentstroke}{rgb}{0.268510,0.009605,0.335427}%
\pgfsetstrokecolor{currentstroke}%
\pgfsetdash{}{0pt}%
\pgfpathmoveto{\pgfqpoint{8.427953in}{1.852924in}}%
\pgfpathlineto{\pgfqpoint{8.427974in}{1.852924in}}%
\pgfusepath{stroke}%
\end{pgfscope}%
\begin{pgfscope}%
\pgfpathrectangle{\pgfqpoint{6.720588in}{1.750000in}}{\pgfqpoint{2.279412in}{2.004545in}}%
\pgfusepath{clip}%
\pgfsetbuttcap%
\pgfsetroundjoin%
\pgfsetlinewidth{0.308920pt}%
\definecolor{currentstroke}{rgb}{0.268510,0.009605,0.335427}%
\pgfsetstrokecolor{currentstroke}%
\pgfsetdash{}{0pt}%
\pgfpathmoveto{\pgfqpoint{8.427974in}{1.852924in}}%
\pgfpathlineto{\pgfqpoint{8.427827in}{1.852932in}}%
\pgfusepath{stroke}%
\end{pgfscope}%
\begin{pgfscope}%
\pgfpathrectangle{\pgfqpoint{6.720588in}{1.750000in}}{\pgfqpoint{2.279412in}{2.004545in}}%
\pgfusepath{clip}%
\pgfsetbuttcap%
\pgfsetroundjoin%
\pgfsetlinewidth{0.308875pt}%
\definecolor{currentstroke}{rgb}{0.268510,0.009605,0.335427}%
\pgfsetstrokecolor{currentstroke}%
\pgfsetdash{}{0pt}%
\pgfpathmoveto{\pgfqpoint{8.427827in}{1.852932in}}%
\pgfpathlineto{\pgfqpoint{8.427636in}{1.852941in}}%
\pgfusepath{stroke}%
\end{pgfscope}%
\begin{pgfscope}%
\pgfpathrectangle{\pgfqpoint{6.720588in}{1.750000in}}{\pgfqpoint{2.279412in}{2.004545in}}%
\pgfusepath{clip}%
\pgfsetbuttcap%
\pgfsetroundjoin%
\pgfsetlinewidth{0.308816pt}%
\definecolor{currentstroke}{rgb}{0.268510,0.009605,0.335427}%
\pgfsetstrokecolor{currentstroke}%
\pgfsetdash{}{0pt}%
\pgfpathmoveto{\pgfqpoint{8.427636in}{1.852941in}}%
\pgfpathlineto{\pgfqpoint{8.427614in}{1.852941in}}%
\pgfusepath{stroke}%
\end{pgfscope}%
\begin{pgfscope}%
\pgfpathrectangle{\pgfqpoint{6.720588in}{1.750000in}}{\pgfqpoint{2.279412in}{2.004545in}}%
\pgfusepath{clip}%
\pgfsetbuttcap%
\pgfsetroundjoin%
\pgfsetlinewidth{0.308810pt}%
\definecolor{currentstroke}{rgb}{0.268510,0.009605,0.335427}%
\pgfsetstrokecolor{currentstroke}%
\pgfsetdash{}{0pt}%
\pgfpathmoveto{\pgfqpoint{8.427614in}{1.852941in}}%
\pgfpathlineto{\pgfqpoint{8.427796in}{1.852931in}}%
\pgfusepath{stroke}%
\end{pgfscope}%
\begin{pgfscope}%
\pgfpathrectangle{\pgfqpoint{6.720588in}{1.750000in}}{\pgfqpoint{2.279412in}{2.004545in}}%
\pgfusepath{clip}%
\pgfsetbuttcap%
\pgfsetroundjoin%
\pgfsetlinewidth{0.308865pt}%
\definecolor{currentstroke}{rgb}{0.268510,0.009605,0.335427}%
\pgfsetstrokecolor{currentstroke}%
\pgfsetdash{}{0pt}%
\pgfpathmoveto{\pgfqpoint{8.427796in}{1.852931in}}%
\pgfpathlineto{\pgfqpoint{8.427992in}{1.852922in}}%
\pgfusepath{stroke}%
\end{pgfscope}%
\begin{pgfscope}%
\pgfpathrectangle{\pgfqpoint{6.720588in}{1.750000in}}{\pgfqpoint{2.279412in}{2.004545in}}%
\pgfusepath{clip}%
\pgfsetbuttcap%
\pgfsetroundjoin%
\pgfsetlinewidth{0.308925pt}%
\definecolor{currentstroke}{rgb}{0.268510,0.009605,0.335427}%
\pgfsetstrokecolor{currentstroke}%
\pgfsetdash{}{0pt}%
\pgfpathmoveto{\pgfqpoint{8.427992in}{1.852922in}}%
\pgfpathlineto{\pgfqpoint{8.428018in}{1.852921in}}%
\pgfusepath{stroke}%
\end{pgfscope}%
\begin{pgfscope}%
\pgfpathrectangle{\pgfqpoint{6.720588in}{1.750000in}}{\pgfqpoint{2.279412in}{2.004545in}}%
\pgfusepath{clip}%
\pgfsetbuttcap%
\pgfsetroundjoin%
\pgfsetlinewidth{0.308933pt}%
\definecolor{currentstroke}{rgb}{0.268510,0.009605,0.335427}%
\pgfsetstrokecolor{currentstroke}%
\pgfsetdash{}{0pt}%
\pgfpathmoveto{\pgfqpoint{8.428018in}{1.852921in}}%
\pgfpathlineto{\pgfqpoint{8.427829in}{1.852932in}}%
\pgfusepath{stroke}%
\end{pgfscope}%
\begin{pgfscope}%
\pgfpathrectangle{\pgfqpoint{6.720588in}{1.750000in}}{\pgfqpoint{2.279412in}{2.004545in}}%
\pgfusepath{clip}%
\pgfsetbuttcap%
\pgfsetroundjoin%
\pgfsetlinewidth{0.308875pt}%
\definecolor{currentstroke}{rgb}{0.268510,0.009605,0.335427}%
\pgfsetstrokecolor{currentstroke}%
\pgfsetdash{}{0pt}%
\pgfpathmoveto{\pgfqpoint{8.427829in}{1.852932in}}%
\pgfpathlineto{\pgfqpoint{8.427581in}{1.852943in}}%
\pgfusepath{stroke}%
\end{pgfscope}%
\begin{pgfscope}%
\pgfpathrectangle{\pgfqpoint{6.720588in}{1.750000in}}{\pgfqpoint{2.279412in}{2.004545in}}%
\pgfusepath{clip}%
\pgfsetbuttcap%
\pgfsetroundjoin%
\pgfsetlinewidth{0.308800pt}%
\definecolor{currentstroke}{rgb}{0.268510,0.009605,0.335427}%
\pgfsetstrokecolor{currentstroke}%
\pgfsetdash{}{0pt}%
\pgfpathmoveto{\pgfqpoint{8.427581in}{1.852943in}}%
\pgfpathlineto{\pgfqpoint{8.427554in}{1.852943in}}%
\pgfusepath{stroke}%
\end{pgfscope}%
\begin{pgfscope}%
\pgfpathrectangle{\pgfqpoint{6.720588in}{1.750000in}}{\pgfqpoint{2.279412in}{2.004545in}}%
\pgfusepath{clip}%
\pgfsetbuttcap%
\pgfsetroundjoin%
\pgfsetlinewidth{0.308791pt}%
\definecolor{currentstroke}{rgb}{0.268510,0.009605,0.335427}%
\pgfsetstrokecolor{currentstroke}%
\pgfsetdash{}{0pt}%
\pgfpathmoveto{\pgfqpoint{8.427554in}{1.852943in}}%
\pgfpathlineto{\pgfqpoint{8.427793in}{1.852931in}}%
\pgfusepath{stroke}%
\end{pgfscope}%
\begin{pgfscope}%
\pgfpathrectangle{\pgfqpoint{6.720588in}{1.750000in}}{\pgfqpoint{2.279412in}{2.004545in}}%
\pgfusepath{clip}%
\pgfsetbuttcap%
\pgfsetroundjoin%
\pgfsetlinewidth{0.308864pt}%
\definecolor{currentstroke}{rgb}{0.268510,0.009605,0.335427}%
\pgfsetstrokecolor{currentstroke}%
\pgfsetdash{}{0pt}%
\pgfpathmoveto{\pgfqpoint{8.427793in}{1.852931in}}%
\pgfpathlineto{\pgfqpoint{8.428040in}{1.852919in}}%
\pgfusepath{stroke}%
\end{pgfscope}%
\begin{pgfscope}%
\pgfpathrectangle{\pgfqpoint{6.720588in}{1.750000in}}{\pgfqpoint{2.279412in}{2.004545in}}%
\pgfusepath{clip}%
\pgfsetbuttcap%
\pgfsetroundjoin%
\pgfsetlinewidth{0.308940pt}%
\definecolor{currentstroke}{rgb}{0.268510,0.009605,0.335427}%
\pgfsetstrokecolor{currentstroke}%
\pgfsetdash{}{0pt}%
\pgfpathmoveto{\pgfqpoint{8.428040in}{1.852919in}}%
\pgfpathlineto{\pgfqpoint{8.428072in}{1.852919in}}%
\pgfusepath{stroke}%
\end{pgfscope}%
\begin{pgfscope}%
\pgfpathrectangle{\pgfqpoint{6.720588in}{1.750000in}}{\pgfqpoint{2.279412in}{2.004545in}}%
\pgfusepath{clip}%
\pgfsetbuttcap%
\pgfsetroundjoin%
\pgfsetlinewidth{0.308949pt}%
\definecolor{currentstroke}{rgb}{0.268510,0.009605,0.335427}%
\pgfsetstrokecolor{currentstroke}%
\pgfsetdash{}{0pt}%
\pgfpathmoveto{\pgfqpoint{8.428072in}{1.852919in}}%
\pgfpathlineto{\pgfqpoint{8.427831in}{1.852932in}}%
\pgfusepath{stroke}%
\end{pgfscope}%
\begin{pgfscope}%
\pgfpathrectangle{\pgfqpoint{6.720588in}{1.750000in}}{\pgfqpoint{2.279412in}{2.004545in}}%
\pgfusepath{clip}%
\pgfsetbuttcap%
\pgfsetroundjoin%
\pgfsetlinewidth{0.308876pt}%
\definecolor{currentstroke}{rgb}{0.268510,0.009605,0.335427}%
\pgfsetstrokecolor{currentstroke}%
\pgfsetdash{}{0pt}%
\pgfpathmoveto{\pgfqpoint{8.427831in}{1.852932in}}%
\pgfpathlineto{\pgfqpoint{8.427507in}{1.852947in}}%
\pgfusepath{stroke}%
\end{pgfscope}%
\begin{pgfscope}%
\pgfpathrectangle{\pgfqpoint{6.720588in}{1.750000in}}{\pgfqpoint{2.279412in}{2.004545in}}%
\pgfusepath{clip}%
\pgfsetbuttcap%
\pgfsetroundjoin%
\pgfsetlinewidth{0.308777pt}%
\definecolor{currentstroke}{rgb}{0.268510,0.009605,0.335427}%
\pgfsetstrokecolor{currentstroke}%
\pgfsetdash{}{0pt}%
\pgfpathmoveto{\pgfqpoint{8.427507in}{1.852947in}}%
\pgfpathlineto{\pgfqpoint{8.427477in}{1.852947in}}%
\pgfusepath{stroke}%
\end{pgfscope}%
\begin{pgfscope}%
\pgfpathrectangle{\pgfqpoint{6.720588in}{1.750000in}}{\pgfqpoint{2.279412in}{2.004545in}}%
\pgfusepath{clip}%
\pgfsetbuttcap%
\pgfsetroundjoin%
\pgfsetlinewidth{0.308768pt}%
\definecolor{currentstroke}{rgb}{0.268510,0.009605,0.335427}%
\pgfsetstrokecolor{currentstroke}%
\pgfsetdash{}{0pt}%
\pgfpathmoveto{\pgfqpoint{8.427477in}{1.852947in}}%
\pgfpathlineto{\pgfqpoint{8.427788in}{1.852930in}}%
\pgfusepath{stroke}%
\end{pgfscope}%
\begin{pgfscope}%
\pgfpathrectangle{\pgfqpoint{6.720588in}{1.750000in}}{\pgfqpoint{2.279412in}{2.004545in}}%
\pgfusepath{clip}%
\pgfsetbuttcap%
\pgfsetroundjoin%
\pgfsetlinewidth{0.308863pt}%
\definecolor{currentstroke}{rgb}{0.268510,0.009605,0.335427}%
\pgfsetstrokecolor{currentstroke}%
\pgfsetdash{}{0pt}%
\pgfpathmoveto{\pgfqpoint{8.427788in}{1.852930in}}%
\pgfpathlineto{\pgfqpoint{8.428099in}{1.852915in}}%
\pgfusepath{stroke}%
\end{pgfscope}%
\begin{pgfscope}%
\pgfpathrectangle{\pgfqpoint{6.720588in}{1.750000in}}{\pgfqpoint{2.279412in}{2.004545in}}%
\pgfusepath{clip}%
\pgfsetbuttcap%
\pgfsetroundjoin%
\pgfsetlinewidth{0.308957pt}%
\definecolor{currentstroke}{rgb}{0.268510,0.009605,0.335427}%
\pgfsetstrokecolor{currentstroke}%
\pgfsetdash{}{0pt}%
\pgfpathmoveto{\pgfqpoint{8.428099in}{1.852915in}}%
\pgfpathlineto{\pgfqpoint{8.428138in}{1.852915in}}%
\pgfusepath{stroke}%
\end{pgfscope}%
\begin{pgfscope}%
\pgfpathrectangle{\pgfqpoint{6.720588in}{1.750000in}}{\pgfqpoint{2.279412in}{2.004545in}}%
\pgfusepath{clip}%
\pgfsetbuttcap%
\pgfsetroundjoin%
\pgfsetlinewidth{0.308969pt}%
\definecolor{currentstroke}{rgb}{0.268510,0.009605,0.335427}%
\pgfsetstrokecolor{currentstroke}%
\pgfsetdash{}{0pt}%
\pgfpathmoveto{\pgfqpoint{8.428138in}{1.852915in}}%
\pgfpathlineto{\pgfqpoint{8.427834in}{1.852932in}}%
\pgfusepath{stroke}%
\end{pgfscope}%
\begin{pgfscope}%
\pgfpathrectangle{\pgfqpoint{6.720588in}{1.750000in}}{\pgfqpoint{2.279412in}{2.004545in}}%
\pgfusepath{clip}%
\pgfsetbuttcap%
\pgfsetroundjoin%
\pgfsetlinewidth{0.308877pt}%
\definecolor{currentstroke}{rgb}{0.268510,0.009605,0.335427}%
\pgfsetstrokecolor{currentstroke}%
\pgfsetdash{}{0pt}%
\pgfpathmoveto{\pgfqpoint{8.427834in}{1.852932in}}%
\pgfpathlineto{\pgfqpoint{8.427411in}{1.852952in}}%
\pgfusepath{stroke}%
\end{pgfscope}%
\begin{pgfscope}%
\pgfpathrectangle{\pgfqpoint{6.720588in}{1.750000in}}{\pgfqpoint{2.279412in}{2.004545in}}%
\pgfusepath{clip}%
\pgfsetbuttcap%
\pgfsetroundjoin%
\pgfsetlinewidth{0.308748pt}%
\definecolor{currentstroke}{rgb}{0.268510,0.009605,0.335427}%
\pgfsetstrokecolor{currentstroke}%
\pgfsetdash{}{0pt}%
\pgfpathmoveto{\pgfqpoint{8.427411in}{1.852952in}}%
\pgfpathlineto{\pgfqpoint{8.427376in}{1.852951in}}%
\pgfusepath{stroke}%
\end{pgfscope}%
\begin{pgfscope}%
\pgfpathrectangle{\pgfqpoint{6.720588in}{1.750000in}}{\pgfqpoint{2.279412in}{2.004545in}}%
\pgfusepath{clip}%
\pgfsetbuttcap%
\pgfsetroundjoin%
\pgfsetlinewidth{0.308737pt}%
\definecolor{currentstroke}{rgb}{0.268510,0.009605,0.335427}%
\pgfsetstrokecolor{currentstroke}%
\pgfsetdash{}{0pt}%
\pgfpathmoveto{\pgfqpoint{8.427376in}{1.852951in}}%
\pgfpathlineto{\pgfqpoint{8.427782in}{1.852930in}}%
\pgfusepath{stroke}%
\end{pgfscope}%
\begin{pgfscope}%
\pgfpathrectangle{\pgfqpoint{6.720588in}{1.750000in}}{\pgfqpoint{2.279412in}{2.004545in}}%
\pgfusepath{clip}%
\pgfsetbuttcap%
\pgfsetroundjoin%
\pgfsetlinewidth{0.308861pt}%
\definecolor{currentstroke}{rgb}{0.268510,0.009605,0.335427}%
\pgfsetstrokecolor{currentstroke}%
\pgfsetdash{}{0pt}%
\pgfpathmoveto{\pgfqpoint{8.427782in}{1.852930in}}%
\pgfpathlineto{\pgfqpoint{8.428168in}{1.852911in}}%
\pgfusepath{stroke}%
\end{pgfscope}%
\begin{pgfscope}%
\pgfpathrectangle{\pgfqpoint{6.720588in}{1.750000in}}{\pgfqpoint{2.279412in}{2.004545in}}%
\pgfusepath{clip}%
\pgfsetbuttcap%
\pgfsetroundjoin%
\pgfsetlinewidth{0.308978pt}%
\definecolor{currentstroke}{rgb}{0.268510,0.009605,0.335427}%
\pgfsetstrokecolor{currentstroke}%
\pgfsetdash{}{0pt}%
\pgfpathmoveto{\pgfqpoint{8.428168in}{1.852911in}}%
\pgfpathlineto{\pgfqpoint{8.428216in}{1.852911in}}%
\pgfusepath{stroke}%
\end{pgfscope}%
\begin{pgfscope}%
\pgfpathrectangle{\pgfqpoint{6.720588in}{1.750000in}}{\pgfqpoint{2.279412in}{2.004545in}}%
\pgfusepath{clip}%
\pgfsetbuttcap%
\pgfsetroundjoin%
\pgfsetlinewidth{0.308993pt}%
\definecolor{currentstroke}{rgb}{0.268510,0.009605,0.335427}%
\pgfsetstrokecolor{currentstroke}%
\pgfsetdash{}{0pt}%
\pgfpathmoveto{\pgfqpoint{8.428216in}{1.852911in}}%
\pgfpathlineto{\pgfqpoint{8.427838in}{1.852931in}}%
\pgfusepath{stroke}%
\end{pgfscope}%
\begin{pgfscope}%
\pgfpathrectangle{\pgfqpoint{6.720588in}{1.750000in}}{\pgfqpoint{2.279412in}{2.004545in}}%
\pgfusepath{clip}%
\pgfsetbuttcap%
\pgfsetroundjoin%
\pgfsetlinewidth{0.308878pt}%
\definecolor{currentstroke}{rgb}{0.268510,0.009605,0.335427}%
\pgfsetstrokecolor{currentstroke}%
\pgfsetdash{}{0pt}%
\pgfpathmoveto{\pgfqpoint{8.427838in}{1.852931in}}%
\pgfpathlineto{\pgfqpoint{8.427287in}{1.852958in}}%
\pgfusepath{stroke}%
\end{pgfscope}%
\begin{pgfscope}%
\pgfpathrectangle{\pgfqpoint{6.720588in}{1.750000in}}{\pgfqpoint{2.279412in}{2.004545in}}%
\pgfusepath{clip}%
\pgfsetbuttcap%
\pgfsetroundjoin%
\pgfsetlinewidth{0.308710pt}%
\definecolor{currentstroke}{rgb}{0.268510,0.009605,0.335427}%
\pgfsetstrokecolor{currentstroke}%
\pgfsetdash{}{0pt}%
\pgfpathmoveto{\pgfqpoint{8.427287in}{1.852958in}}%
\pgfpathlineto{\pgfqpoint{8.427248in}{1.852956in}}%
\pgfusepath{stroke}%
\end{pgfscope}%
\begin{pgfscope}%
\pgfpathrectangle{\pgfqpoint{6.720588in}{1.750000in}}{\pgfqpoint{2.279412in}{2.004545in}}%
\pgfusepath{clip}%
\pgfsetbuttcap%
\pgfsetroundjoin%
\pgfsetlinewidth{0.308697pt}%
\definecolor{currentstroke}{rgb}{0.268510,0.009605,0.335427}%
\pgfsetstrokecolor{currentstroke}%
\pgfsetdash{}{0pt}%
\pgfpathmoveto{\pgfqpoint{8.427248in}{1.852956in}}%
\pgfpathlineto{\pgfqpoint{8.427774in}{1.852929in}}%
\pgfusepath{stroke}%
\end{pgfscope}%
\begin{pgfscope}%
\pgfpathrectangle{\pgfqpoint{6.720588in}{1.750000in}}{\pgfqpoint{2.279412in}{2.004545in}}%
\pgfusepath{clip}%
\pgfsetbuttcap%
\pgfsetroundjoin%
\pgfsetlinewidth{0.308858pt}%
\definecolor{currentstroke}{rgb}{0.268510,0.009605,0.335427}%
\pgfsetstrokecolor{currentstroke}%
\pgfsetdash{}{0pt}%
\pgfpathmoveto{\pgfqpoint{8.427774in}{1.852929in}}%
\pgfpathlineto{\pgfqpoint{8.428247in}{1.852906in}}%
\pgfusepath{stroke}%
\end{pgfscope}%
\begin{pgfscope}%
\pgfpathrectangle{\pgfqpoint{6.720588in}{1.750000in}}{\pgfqpoint{2.279412in}{2.004545in}}%
\pgfusepath{clip}%
\pgfsetbuttcap%
\pgfsetroundjoin%
\pgfsetlinewidth{0.309002pt}%
\definecolor{currentstroke}{rgb}{0.268510,0.009605,0.335427}%
\pgfsetstrokecolor{currentstroke}%
\pgfsetdash{}{0pt}%
\pgfpathmoveto{\pgfqpoint{8.428247in}{1.852906in}}%
\pgfpathlineto{\pgfqpoint{8.428307in}{1.852906in}}%
\pgfusepath{stroke}%
\end{pgfscope}%
\begin{pgfscope}%
\pgfpathrectangle{\pgfqpoint{6.720588in}{1.750000in}}{\pgfqpoint{2.279412in}{2.004545in}}%
\pgfusepath{clip}%
\pgfsetbuttcap%
\pgfsetroundjoin%
\pgfsetlinewidth{0.309021pt}%
\definecolor{currentstroke}{rgb}{0.268510,0.009605,0.335427}%
\pgfsetstrokecolor{currentstroke}%
\pgfsetdash{}{0pt}%
\pgfpathmoveto{\pgfqpoint{8.428307in}{1.852906in}}%
\pgfpathlineto{\pgfqpoint{8.427844in}{1.852931in}}%
\pgfusepath{stroke}%
\end{pgfscope}%
\begin{pgfscope}%
\pgfpathrectangle{\pgfqpoint{6.720588in}{1.750000in}}{\pgfqpoint{2.279412in}{2.004545in}}%
\pgfusepath{clip}%
\pgfsetbuttcap%
\pgfsetroundjoin%
\pgfsetlinewidth{0.308880pt}%
\definecolor{currentstroke}{rgb}{0.268510,0.009605,0.335427}%
\pgfsetstrokecolor{currentstroke}%
\pgfsetdash{}{0pt}%
\pgfpathmoveto{\pgfqpoint{8.427844in}{1.852931in}}%
\pgfpathlineto{\pgfqpoint{8.427129in}{1.852965in}}%
\pgfusepath{stroke}%
\end{pgfscope}%
\begin{pgfscope}%
\pgfpathrectangle{\pgfqpoint{6.720588in}{1.750000in}}{\pgfqpoint{2.279412in}{2.004545in}}%
\pgfusepath{clip}%
\pgfsetbuttcap%
\pgfsetroundjoin%
\pgfsetlinewidth{0.308661pt}%
\definecolor{currentstroke}{rgb}{0.268510,0.009605,0.335427}%
\pgfsetstrokecolor{currentstroke}%
\pgfsetdash{}{0pt}%
\pgfpathmoveto{\pgfqpoint{8.427129in}{1.852965in}}%
\pgfpathlineto{\pgfqpoint{8.427129in}{1.852965in}}%
\pgfusepath{stroke}%
\end{pgfscope}%
\begin{pgfscope}%
\pgfpathrectangle{\pgfqpoint{6.720588in}{1.750000in}}{\pgfqpoint{2.279412in}{2.004545in}}%
\pgfusepath{clip}%
\pgfsetbuttcap%
\pgfsetroundjoin%
\pgfsetlinewidth{0.308661pt}%
\definecolor{currentstroke}{rgb}{0.268510,0.009605,0.335427}%
\pgfsetstrokecolor{currentstroke}%
\pgfsetdash{}{0pt}%
\pgfpathmoveto{\pgfqpoint{8.427129in}{1.852965in}}%
\pgfpathlineto{\pgfqpoint{8.427789in}{1.852930in}}%
\pgfusepath{stroke}%
\end{pgfscope}%
\begin{pgfscope}%
\pgfpathrectangle{\pgfqpoint{6.720588in}{1.750000in}}{\pgfqpoint{2.279412in}{2.004545in}}%
\pgfusepath{clip}%
\pgfsetbuttcap%
\pgfsetroundjoin%
\pgfsetlinewidth{0.308863pt}%
\definecolor{currentstroke}{rgb}{0.268510,0.009605,0.335427}%
\pgfsetstrokecolor{currentstroke}%
\pgfsetdash{}{0pt}%
\pgfpathmoveto{\pgfqpoint{8.427789in}{1.852930in}}%
\pgfpathlineto{\pgfqpoint{8.428329in}{1.852904in}}%
\pgfusepath{stroke}%
\end{pgfscope}%
\begin{pgfscope}%
\pgfpathrectangle{\pgfqpoint{6.720588in}{1.750000in}}{\pgfqpoint{2.279412in}{2.004545in}}%
\pgfusepath{clip}%
\pgfsetbuttcap%
\pgfsetroundjoin%
\pgfsetlinewidth{0.309027pt}%
\definecolor{currentstroke}{rgb}{0.268510,0.009605,0.335427}%
\pgfsetstrokecolor{currentstroke}%
\pgfsetdash{}{0pt}%
\pgfpathmoveto{\pgfqpoint{8.428329in}{1.852904in}}%
\pgfpathlineto{\pgfqpoint{8.428381in}{1.852904in}}%
\pgfusepath{stroke}%
\end{pgfscope}%
\begin{pgfscope}%
\pgfpathrectangle{\pgfqpoint{6.720588in}{1.750000in}}{\pgfqpoint{2.279412in}{2.004545in}}%
\pgfusepath{clip}%
\pgfsetbuttcap%
\pgfsetroundjoin%
\pgfsetlinewidth{0.309043pt}%
\definecolor{currentstroke}{rgb}{0.268510,0.009605,0.335427}%
\pgfsetstrokecolor{currentstroke}%
\pgfsetdash{}{0pt}%
\pgfpathmoveto{\pgfqpoint{8.428381in}{1.852904in}}%
\pgfpathlineto{\pgfqpoint{8.427819in}{1.852934in}}%
\pgfusepath{stroke}%
\end{pgfscope}%
\begin{pgfscope}%
\pgfpathrectangle{\pgfqpoint{6.720588in}{1.750000in}}{\pgfqpoint{2.279412in}{2.004545in}}%
\pgfusepath{clip}%
\pgfsetbuttcap%
\pgfsetroundjoin%
\pgfsetlinewidth{0.308872pt}%
\definecolor{currentstroke}{rgb}{0.268510,0.009605,0.335427}%
\pgfsetstrokecolor{currentstroke}%
\pgfsetdash{}{0pt}%
\pgfpathmoveto{\pgfqpoint{8.427819in}{1.852934in}}%
\pgfpathlineto{\pgfqpoint{8.427819in}{1.852934in}}%
\pgfusepath{stroke}%
\end{pgfscope}%
\begin{pgfscope}%
\pgfpathrectangle{\pgfqpoint{6.720588in}{1.750000in}}{\pgfqpoint{2.279412in}{2.004545in}}%
\pgfusepath{clip}%
\pgfsetbuttcap%
\pgfsetroundjoin%
\pgfsetlinewidth{0.308872pt}%
\definecolor{currentstroke}{rgb}{0.268510,0.009605,0.335427}%
\pgfsetstrokecolor{currentstroke}%
\pgfsetdash{}{0pt}%
\pgfpathmoveto{\pgfqpoint{8.427819in}{1.852934in}}%
\pgfpathlineto{\pgfqpoint{8.427713in}{1.852938in}}%
\pgfusepath{stroke}%
\end{pgfscope}%
\begin{pgfscope}%
\pgfpathrectangle{\pgfqpoint{6.720588in}{1.750000in}}{\pgfqpoint{2.279412in}{2.004545in}}%
\pgfusepath{clip}%
\pgfsetbuttcap%
\pgfsetroundjoin%
\pgfsetlinewidth{0.308840pt}%
\definecolor{currentstroke}{rgb}{0.268510,0.009605,0.335427}%
\pgfsetstrokecolor{currentstroke}%
\pgfsetdash{}{0pt}%
\pgfpathmoveto{\pgfqpoint{8.427713in}{1.852938in}}%
\pgfpathlineto{\pgfqpoint{8.427702in}{1.852938in}}%
\pgfusepath{stroke}%
\end{pgfscope}%
\begin{pgfscope}%
\pgfpathrectangle{\pgfqpoint{6.720588in}{1.750000in}}{\pgfqpoint{2.279412in}{2.004545in}}%
\pgfusepath{clip}%
\pgfsetbuttcap%
\pgfsetroundjoin%
\pgfsetlinewidth{0.308837pt}%
\definecolor{currentstroke}{rgb}{0.268510,0.009605,0.335427}%
\pgfsetstrokecolor{currentstroke}%
\pgfsetdash{}{0pt}%
\pgfpathmoveto{\pgfqpoint{8.427702in}{1.852938in}}%
\pgfpathlineto{\pgfqpoint{8.427805in}{1.852932in}}%
\pgfusepath{stroke}%
\end{pgfscope}%
\begin{pgfscope}%
\pgfpathrectangle{\pgfqpoint{6.720588in}{1.750000in}}{\pgfqpoint{2.279412in}{2.004545in}}%
\pgfusepath{clip}%
\pgfsetbuttcap%
\pgfsetroundjoin%
\pgfsetlinewidth{0.308868pt}%
\definecolor{currentstroke}{rgb}{0.268510,0.009605,0.335427}%
\pgfsetstrokecolor{currentstroke}%
\pgfsetdash{}{0pt}%
\pgfpathmoveto{\pgfqpoint{8.427805in}{1.852932in}}%
\pgfpathlineto{\pgfqpoint{8.427918in}{1.852927in}}%
\pgfusepath{stroke}%
\end{pgfscope}%
\begin{pgfscope}%
\pgfpathrectangle{\pgfqpoint{6.720588in}{1.750000in}}{\pgfqpoint{2.279412in}{2.004545in}}%
\pgfusepath{clip}%
\pgfsetbuttcap%
\pgfsetroundjoin%
\pgfsetlinewidth{0.308903pt}%
\definecolor{currentstroke}{rgb}{0.268510,0.009605,0.335427}%
\pgfsetstrokecolor{currentstroke}%
\pgfsetdash{}{0pt}%
\pgfpathmoveto{\pgfqpoint{8.427918in}{1.852927in}}%
\pgfpathlineto{\pgfqpoint{8.427931in}{1.852926in}}%
\pgfusepath{stroke}%
\end{pgfscope}%
\begin{pgfscope}%
\pgfpathrectangle{\pgfqpoint{6.720588in}{1.750000in}}{\pgfqpoint{2.279412in}{2.004545in}}%
\pgfusepath{clip}%
\pgfsetbuttcap%
\pgfsetroundjoin%
\pgfsetlinewidth{0.308906pt}%
\definecolor{currentstroke}{rgb}{0.268510,0.009605,0.335427}%
\pgfsetstrokecolor{currentstroke}%
\pgfsetdash{}{0pt}%
\pgfpathmoveto{\pgfqpoint{8.427931in}{1.852926in}}%
\pgfpathlineto{\pgfqpoint{8.427820in}{1.852932in}}%
\pgfusepath{stroke}%
\end{pgfscope}%
\begin{pgfscope}%
\pgfpathrectangle{\pgfqpoint{6.720588in}{1.750000in}}{\pgfqpoint{2.279412in}{2.004545in}}%
\pgfusepath{clip}%
\pgfsetbuttcap%
\pgfsetroundjoin%
\pgfsetlinewidth{0.308873pt}%
\definecolor{currentstroke}{rgb}{0.268510,0.009605,0.335427}%
\pgfsetstrokecolor{currentstroke}%
\pgfsetdash{}{0pt}%
\pgfpathmoveto{\pgfqpoint{8.427820in}{1.852932in}}%
\pgfpathlineto{\pgfqpoint{8.427683in}{1.852939in}}%
\pgfusepath{stroke}%
\end{pgfscope}%
\begin{pgfscope}%
\pgfpathrectangle{\pgfqpoint{6.720588in}{1.750000in}}{\pgfqpoint{2.279412in}{2.004545in}}%
\pgfusepath{clip}%
\pgfsetbuttcap%
\pgfsetroundjoin%
\pgfsetlinewidth{0.308831pt}%
\definecolor{currentstroke}{rgb}{0.268510,0.009605,0.335427}%
\pgfsetstrokecolor{currentstroke}%
\pgfsetdash{}{0pt}%
\pgfpathmoveto{\pgfqpoint{8.427683in}{1.852939in}}%
\pgfpathlineto{\pgfqpoint{8.427670in}{1.852938in}}%
\pgfusepath{stroke}%
\end{pgfscope}%
\begin{pgfscope}%
\pgfpathrectangle{\pgfqpoint{6.720588in}{1.750000in}}{\pgfqpoint{2.279412in}{2.004545in}}%
\pgfusepath{clip}%
\pgfsetbuttcap%
\pgfsetroundjoin%
\pgfsetlinewidth{0.308827pt}%
\definecolor{currentstroke}{rgb}{0.268510,0.009605,0.335427}%
\pgfsetstrokecolor{currentstroke}%
\pgfsetdash{}{0pt}%
\pgfpathmoveto{\pgfqpoint{8.427670in}{1.852938in}}%
\pgfpathlineto{\pgfqpoint{8.427804in}{1.852931in}}%
\pgfusepath{stroke}%
\end{pgfscope}%
\begin{pgfscope}%
\pgfpathrectangle{\pgfqpoint{6.720588in}{1.750000in}}{\pgfqpoint{2.279412in}{2.004545in}}%
\pgfusepath{clip}%
\pgfsetbuttcap%
\pgfsetroundjoin%
\pgfsetlinewidth{0.308868pt}%
\definecolor{currentstroke}{rgb}{0.268510,0.009605,0.335427}%
\pgfsetstrokecolor{currentstroke}%
\pgfsetdash{}{0pt}%
\pgfpathmoveto{\pgfqpoint{8.427804in}{1.852931in}}%
\pgfpathlineto{\pgfqpoint{8.427948in}{1.852924in}}%
\pgfusepath{stroke}%
\end{pgfscope}%
\begin{pgfscope}%
\pgfpathrectangle{\pgfqpoint{6.720588in}{1.750000in}}{\pgfqpoint{2.279412in}{2.004545in}}%
\pgfusepath{clip}%
\pgfsetbuttcap%
\pgfsetroundjoin%
\pgfsetlinewidth{0.308912pt}%
\definecolor{currentstroke}{rgb}{0.268510,0.009605,0.335427}%
\pgfsetstrokecolor{currentstroke}%
\pgfsetdash{}{0pt}%
\pgfpathmoveto{\pgfqpoint{8.427948in}{1.852924in}}%
\pgfpathlineto{\pgfqpoint{8.427964in}{1.852924in}}%
\pgfusepath{stroke}%
\end{pgfscope}%
\begin{pgfscope}%
\pgfpathrectangle{\pgfqpoint{6.720588in}{1.750000in}}{\pgfqpoint{2.279412in}{2.004545in}}%
\pgfusepath{clip}%
\pgfsetbuttcap%
\pgfsetroundjoin%
\pgfsetlinewidth{0.308916pt}%
\definecolor{currentstroke}{rgb}{0.268510,0.009605,0.335427}%
\pgfsetstrokecolor{currentstroke}%
\pgfsetdash{}{0pt}%
\pgfpathmoveto{\pgfqpoint{8.427964in}{1.852924in}}%
\pgfpathlineto{\pgfqpoint{8.427821in}{1.852932in}}%
\pgfusepath{stroke}%
\end{pgfscope}%
\begin{pgfscope}%
\pgfpathrectangle{\pgfqpoint{6.720588in}{1.750000in}}{\pgfqpoint{2.279412in}{2.004545in}}%
\pgfusepath{clip}%
\pgfsetbuttcap%
\pgfsetroundjoin%
\pgfsetlinewidth{0.308873pt}%
\definecolor{currentstroke}{rgb}{0.268510,0.009605,0.335427}%
\pgfsetstrokecolor{currentstroke}%
\pgfsetdash{}{0pt}%
\pgfpathmoveto{\pgfqpoint{8.427821in}{1.852932in}}%
\pgfpathlineto{\pgfqpoint{8.427643in}{1.852940in}}%
\pgfusepath{stroke}%
\end{pgfscope}%
\begin{pgfscope}%
\pgfpathrectangle{\pgfqpoint{6.720588in}{1.750000in}}{\pgfqpoint{2.279412in}{2.004545in}}%
\pgfusepath{clip}%
\pgfsetbuttcap%
\pgfsetroundjoin%
\pgfsetlinewidth{0.308819pt}%
\definecolor{currentstroke}{rgb}{0.268510,0.009605,0.335427}%
\pgfsetstrokecolor{currentstroke}%
\pgfsetdash{}{0pt}%
\pgfpathmoveto{\pgfqpoint{8.427643in}{1.852940in}}%
\pgfpathlineto{\pgfqpoint{8.427628in}{1.852940in}}%
\pgfusepath{stroke}%
\end{pgfscope}%
\begin{pgfscope}%
\pgfpathrectangle{\pgfqpoint{6.720588in}{1.750000in}}{\pgfqpoint{2.279412in}{2.004545in}}%
\pgfusepath{clip}%
\pgfsetbuttcap%
\pgfsetroundjoin%
\pgfsetlinewidth{0.308814pt}%
\definecolor{currentstroke}{rgb}{0.268510,0.009605,0.335427}%
\pgfsetstrokecolor{currentstroke}%
\pgfsetdash{}{0pt}%
\pgfpathmoveto{\pgfqpoint{8.427628in}{1.852940in}}%
\pgfpathlineto{\pgfqpoint{8.427803in}{1.852931in}}%
\pgfusepath{stroke}%
\end{pgfscope}%
\begin{pgfscope}%
\pgfpathrectangle{\pgfqpoint{6.720588in}{1.750000in}}{\pgfqpoint{2.279412in}{2.004545in}}%
\pgfusepath{clip}%
\pgfsetbuttcap%
\pgfsetroundjoin%
\pgfsetlinewidth{0.308867pt}%
\definecolor{currentstroke}{rgb}{0.268510,0.009605,0.335427}%
\pgfsetstrokecolor{currentstroke}%
\pgfsetdash{}{0pt}%
\pgfpathmoveto{\pgfqpoint{8.427803in}{1.852931in}}%
\pgfpathlineto{\pgfqpoint{8.427986in}{1.852922in}}%
\pgfusepath{stroke}%
\end{pgfscope}%
\begin{pgfscope}%
\pgfpathrectangle{\pgfqpoint{6.720588in}{1.750000in}}{\pgfqpoint{2.279412in}{2.004545in}}%
\pgfusepath{clip}%
\pgfsetbuttcap%
\pgfsetroundjoin%
\pgfsetlinewidth{0.308923pt}%
\definecolor{currentstroke}{rgb}{0.268510,0.009605,0.335427}%
\pgfsetstrokecolor{currentstroke}%
\pgfsetdash{}{0pt}%
\pgfpathmoveto{\pgfqpoint{8.427986in}{1.852922in}}%
\pgfpathlineto{\pgfqpoint{8.428005in}{1.852922in}}%
\pgfusepath{stroke}%
\end{pgfscope}%
\begin{pgfscope}%
\pgfpathrectangle{\pgfqpoint{6.720588in}{1.750000in}}{\pgfqpoint{2.279412in}{2.004545in}}%
\pgfusepath{clip}%
\pgfsetbuttcap%
\pgfsetroundjoin%
\pgfsetlinewidth{0.308929pt}%
\definecolor{currentstroke}{rgb}{0.268510,0.009605,0.335427}%
\pgfsetstrokecolor{currentstroke}%
\pgfsetdash{}{0pt}%
\pgfpathmoveto{\pgfqpoint{8.428005in}{1.852922in}}%
\pgfpathlineto{\pgfqpoint{8.427821in}{1.852932in}}%
\pgfusepath{stroke}%
\end{pgfscope}%
\begin{pgfscope}%
\pgfpathrectangle{\pgfqpoint{6.720588in}{1.750000in}}{\pgfqpoint{2.279412in}{2.004545in}}%
\pgfusepath{clip}%
\pgfsetbuttcap%
\pgfsetroundjoin%
\pgfsetlinewidth{0.308873pt}%
\definecolor{currentstroke}{rgb}{0.268510,0.009605,0.335427}%
\pgfsetstrokecolor{currentstroke}%
\pgfsetdash{}{0pt}%
\pgfpathmoveto{\pgfqpoint{8.427821in}{1.852932in}}%
\pgfpathlineto{\pgfqpoint{8.427589in}{1.852943in}}%
\pgfusepath{stroke}%
\end{pgfscope}%
\begin{pgfscope}%
\pgfpathrectangle{\pgfqpoint{6.720588in}{1.750000in}}{\pgfqpoint{2.279412in}{2.004545in}}%
\pgfusepath{clip}%
\pgfsetbuttcap%
\pgfsetroundjoin%
\pgfsetlinewidth{0.308802pt}%
\definecolor{currentstroke}{rgb}{0.268510,0.009605,0.335427}%
\pgfsetstrokecolor{currentstroke}%
\pgfsetdash{}{0pt}%
\pgfpathmoveto{\pgfqpoint{8.427589in}{1.852943in}}%
\pgfpathlineto{\pgfqpoint{8.427572in}{1.852942in}}%
\pgfusepath{stroke}%
\end{pgfscope}%
\begin{pgfscope}%
\pgfpathrectangle{\pgfqpoint{6.720588in}{1.750000in}}{\pgfqpoint{2.279412in}{2.004545in}}%
\pgfusepath{clip}%
\pgfsetbuttcap%
\pgfsetroundjoin%
\pgfsetlinewidth{0.308797pt}%
\definecolor{currentstroke}{rgb}{0.268510,0.009605,0.335427}%
\pgfsetstrokecolor{currentstroke}%
\pgfsetdash{}{0pt}%
\pgfpathmoveto{\pgfqpoint{8.427572in}{1.852942in}}%
\pgfpathlineto{\pgfqpoint{8.427801in}{1.852930in}}%
\pgfusepath{stroke}%
\end{pgfscope}%
\begin{pgfscope}%
\pgfpathrectangle{\pgfqpoint{6.720588in}{1.750000in}}{\pgfqpoint{2.279412in}{2.004545in}}%
\pgfusepath{clip}%
\pgfsetbuttcap%
\pgfsetroundjoin%
\pgfsetlinewidth{0.308867pt}%
\definecolor{currentstroke}{rgb}{0.268510,0.009605,0.335427}%
\pgfsetstrokecolor{currentstroke}%
\pgfsetdash{}{0pt}%
\pgfpathmoveto{\pgfqpoint{8.427801in}{1.852930in}}%
\pgfpathlineto{\pgfqpoint{8.428033in}{1.852919in}}%
\pgfusepath{stroke}%
\end{pgfscope}%
\begin{pgfscope}%
\pgfpathrectangle{\pgfqpoint{6.720588in}{1.750000in}}{\pgfqpoint{2.279412in}{2.004545in}}%
\pgfusepath{clip}%
\pgfsetbuttcap%
\pgfsetroundjoin%
\pgfsetlinewidth{0.308937pt}%
\definecolor{currentstroke}{rgb}{0.268510,0.009605,0.335427}%
\pgfsetstrokecolor{currentstroke}%
\pgfsetdash{}{0pt}%
\pgfpathmoveto{\pgfqpoint{8.428033in}{1.852919in}}%
\pgfpathlineto{\pgfqpoint{8.428056in}{1.852920in}}%
\pgfusepath{stroke}%
\end{pgfscope}%
\begin{pgfscope}%
\pgfpathrectangle{\pgfqpoint{6.720588in}{1.750000in}}{\pgfqpoint{2.279412in}{2.004545in}}%
\pgfusepath{clip}%
\pgfsetbuttcap%
\pgfsetroundjoin%
\pgfsetlinewidth{0.308944pt}%
\definecolor{currentstroke}{rgb}{0.268510,0.009605,0.335427}%
\pgfsetstrokecolor{currentstroke}%
\pgfsetdash{}{0pt}%
\pgfpathmoveto{\pgfqpoint{8.428056in}{1.852920in}}%
\pgfpathlineto{\pgfqpoint{8.427822in}{1.852932in}}%
\pgfusepath{stroke}%
\end{pgfscope}%
\begin{pgfscope}%
\pgfpathrectangle{\pgfqpoint{6.720588in}{1.750000in}}{\pgfqpoint{2.279412in}{2.004545in}}%
\pgfusepath{clip}%
\pgfsetbuttcap%
\pgfsetroundjoin%
\pgfsetlinewidth{0.308873pt}%
\definecolor{currentstroke}{rgb}{0.268510,0.009605,0.335427}%
\pgfsetstrokecolor{currentstroke}%
\pgfsetdash{}{0pt}%
\pgfpathmoveto{\pgfqpoint{8.427822in}{1.852932in}}%
\pgfpathlineto{\pgfqpoint{8.427519in}{1.852946in}}%
\pgfusepath{stroke}%
\end{pgfscope}%
\begin{pgfscope}%
\pgfpathrectangle{\pgfqpoint{6.720588in}{1.750000in}}{\pgfqpoint{2.279412in}{2.004545in}}%
\pgfusepath{clip}%
\pgfsetbuttcap%
\pgfsetroundjoin%
\pgfsetlinewidth{0.308781pt}%
\definecolor{currentstroke}{rgb}{0.268510,0.009605,0.335427}%
\pgfsetstrokecolor{currentstroke}%
\pgfsetdash{}{0pt}%
\pgfpathmoveto{\pgfqpoint{8.427519in}{1.852946in}}%
\pgfpathlineto{\pgfqpoint{8.427500in}{1.852946in}}%
\pgfusepath{stroke}%
\end{pgfscope}%
\begin{pgfscope}%
\pgfpathrectangle{\pgfqpoint{6.720588in}{1.750000in}}{\pgfqpoint{2.279412in}{2.004545in}}%
\pgfusepath{clip}%
\pgfsetbuttcap%
\pgfsetroundjoin%
\pgfsetlinewidth{0.308775pt}%
\definecolor{currentstroke}{rgb}{0.268510,0.009605,0.335427}%
\pgfsetstrokecolor{currentstroke}%
\pgfsetdash{}{0pt}%
\pgfpathmoveto{\pgfqpoint{8.427500in}{1.852946in}}%
\pgfpathlineto{\pgfqpoint{8.427799in}{1.852930in}}%
\pgfusepath{stroke}%
\end{pgfscope}%
\begin{pgfscope}%
\pgfpathrectangle{\pgfqpoint{6.720588in}{1.750000in}}{\pgfqpoint{2.279412in}{2.004545in}}%
\pgfusepath{clip}%
\pgfsetbuttcap%
\pgfsetroundjoin%
\pgfsetlinewidth{0.308866pt}%
\definecolor{currentstroke}{rgb}{0.268510,0.009605,0.335427}%
\pgfsetstrokecolor{currentstroke}%
\pgfsetdash{}{0pt}%
\pgfpathmoveto{\pgfqpoint{8.427799in}{1.852930in}}%
\pgfpathlineto{\pgfqpoint{8.428091in}{1.852916in}}%
\pgfusepath{stroke}%
\end{pgfscope}%
\begin{pgfscope}%
\pgfpathrectangle{\pgfqpoint{6.720588in}{1.750000in}}{\pgfqpoint{2.279412in}{2.004545in}}%
\pgfusepath{clip}%
\pgfsetbuttcap%
\pgfsetroundjoin%
\pgfsetlinewidth{0.308955pt}%
\definecolor{currentstroke}{rgb}{0.268510,0.009605,0.335427}%
\pgfsetstrokecolor{currentstroke}%
\pgfsetdash{}{0pt}%
\pgfpathmoveto{\pgfqpoint{8.428091in}{1.852916in}}%
\pgfpathlineto{\pgfqpoint{8.428119in}{1.852916in}}%
\pgfusepath{stroke}%
\end{pgfscope}%
\begin{pgfscope}%
\pgfpathrectangle{\pgfqpoint{6.720588in}{1.750000in}}{\pgfqpoint{2.279412in}{2.004545in}}%
\pgfusepath{clip}%
\pgfsetbuttcap%
\pgfsetroundjoin%
\pgfsetlinewidth{0.308964pt}%
\definecolor{currentstroke}{rgb}{0.268510,0.009605,0.335427}%
\pgfsetstrokecolor{currentstroke}%
\pgfsetdash{}{0pt}%
\pgfpathmoveto{\pgfqpoint{8.428119in}{1.852916in}}%
\pgfpathlineto{\pgfqpoint{8.427822in}{1.852932in}}%
\pgfusepath{stroke}%
\end{pgfscope}%
\begin{pgfscope}%
\pgfpathrectangle{\pgfqpoint{6.720588in}{1.750000in}}{\pgfqpoint{2.279412in}{2.004545in}}%
\pgfusepath{clip}%
\pgfsetbuttcap%
\pgfsetroundjoin%
\pgfsetlinewidth{0.308873pt}%
\definecolor{currentstroke}{rgb}{0.268510,0.009605,0.335427}%
\pgfsetstrokecolor{currentstroke}%
\pgfsetdash{}{0pt}%
\pgfpathmoveto{\pgfqpoint{8.427822in}{1.852932in}}%
\pgfpathlineto{\pgfqpoint{8.427427in}{1.852951in}}%
\pgfusepath{stroke}%
\end{pgfscope}%
\begin{pgfscope}%
\pgfpathrectangle{\pgfqpoint{6.720588in}{1.750000in}}{\pgfqpoint{2.279412in}{2.004545in}}%
\pgfusepath{clip}%
\pgfsetbuttcap%
\pgfsetroundjoin%
\pgfsetlinewidth{0.308752pt}%
\definecolor{currentstroke}{rgb}{0.268510,0.009605,0.335427}%
\pgfsetstrokecolor{currentstroke}%
\pgfsetdash{}{0pt}%
\pgfpathmoveto{\pgfqpoint{8.427427in}{1.852951in}}%
\pgfpathlineto{\pgfqpoint{8.427407in}{1.852950in}}%
\pgfusepath{stroke}%
\end{pgfscope}%
\begin{pgfscope}%
\pgfpathrectangle{\pgfqpoint{6.720588in}{1.750000in}}{\pgfqpoint{2.279412in}{2.004545in}}%
\pgfusepath{clip}%
\pgfsetbuttcap%
\pgfsetroundjoin%
\pgfsetlinewidth{0.308746pt}%
\definecolor{currentstroke}{rgb}{0.268510,0.009605,0.335427}%
\pgfsetstrokecolor{currentstroke}%
\pgfsetdash{}{0pt}%
\pgfpathmoveto{\pgfqpoint{8.427407in}{1.852950in}}%
\pgfpathlineto{\pgfqpoint{8.427796in}{1.852929in}}%
\pgfusepath{stroke}%
\end{pgfscope}%
\begin{pgfscope}%
\pgfpathrectangle{\pgfqpoint{6.720588in}{1.750000in}}{\pgfqpoint{2.279412in}{2.004545in}}%
\pgfusepath{clip}%
\pgfsetbuttcap%
\pgfsetroundjoin%
\pgfsetlinewidth{0.308865pt}%
\definecolor{currentstroke}{rgb}{0.268510,0.009605,0.335427}%
\pgfsetstrokecolor{currentstroke}%
\pgfsetdash{}{0pt}%
\pgfpathmoveto{\pgfqpoint{8.427796in}{1.852929in}}%
\pgfpathlineto{\pgfqpoint{8.428159in}{1.852912in}}%
\pgfusepath{stroke}%
\end{pgfscope}%
\begin{pgfscope}%
\pgfpathrectangle{\pgfqpoint{6.720588in}{1.750000in}}{\pgfqpoint{2.279412in}{2.004545in}}%
\pgfusepath{clip}%
\pgfsetbuttcap%
\pgfsetroundjoin%
\pgfsetlinewidth{0.308976pt}%
\definecolor{currentstroke}{rgb}{0.268510,0.009605,0.335427}%
\pgfsetstrokecolor{currentstroke}%
\pgfsetdash{}{0pt}%
\pgfpathmoveto{\pgfqpoint{8.428159in}{1.852912in}}%
\pgfpathlineto{\pgfqpoint{8.428193in}{1.852912in}}%
\pgfusepath{stroke}%
\end{pgfscope}%
\begin{pgfscope}%
\pgfpathrectangle{\pgfqpoint{6.720588in}{1.750000in}}{\pgfqpoint{2.279412in}{2.004545in}}%
\pgfusepath{clip}%
\pgfsetbuttcap%
\pgfsetroundjoin%
\pgfsetlinewidth{0.308986pt}%
\definecolor{currentstroke}{rgb}{0.268510,0.009605,0.335427}%
\pgfsetstrokecolor{currentstroke}%
\pgfsetdash{}{0pt}%
\pgfpathmoveto{\pgfqpoint{8.428193in}{1.852912in}}%
\pgfpathlineto{\pgfqpoint{8.427822in}{1.852932in}}%
\pgfusepath{stroke}%
\end{pgfscope}%
\begin{pgfscope}%
\pgfpathrectangle{\pgfqpoint{6.720588in}{1.750000in}}{\pgfqpoint{2.279412in}{2.004545in}}%
\pgfusepath{clip}%
\pgfsetbuttcap%
\pgfsetroundjoin%
\pgfsetlinewidth{0.308873pt}%
\definecolor{currentstroke}{rgb}{0.268510,0.009605,0.335427}%
\pgfsetstrokecolor{currentstroke}%
\pgfsetdash{}{0pt}%
\pgfpathmoveto{\pgfqpoint{8.427822in}{1.852932in}}%
\pgfpathlineto{\pgfqpoint{8.427307in}{1.852957in}}%
\pgfusepath{stroke}%
\end{pgfscope}%
\begin{pgfscope}%
\pgfpathrectangle{\pgfqpoint{6.720588in}{1.750000in}}{\pgfqpoint{2.279412in}{2.004545in}}%
\pgfusepath{clip}%
\pgfsetbuttcap%
\pgfsetroundjoin%
\pgfsetlinewidth{0.308716pt}%
\definecolor{currentstroke}{rgb}{0.268510,0.009605,0.335427}%
\pgfsetstrokecolor{currentstroke}%
\pgfsetdash{}{0pt}%
\pgfpathmoveto{\pgfqpoint{8.427307in}{1.852957in}}%
\pgfpathlineto{\pgfqpoint{8.427287in}{1.852954in}}%
\pgfusepath{stroke}%
\end{pgfscope}%
\begin{pgfscope}%
\pgfpathrectangle{\pgfqpoint{6.720588in}{1.750000in}}{\pgfqpoint{2.279412in}{2.004545in}}%
\pgfusepath{clip}%
\pgfsetbuttcap%
\pgfsetroundjoin%
\pgfsetlinewidth{0.308710pt}%
\definecolor{currentstroke}{rgb}{0.268510,0.009605,0.335427}%
\pgfsetstrokecolor{currentstroke}%
\pgfsetdash{}{0pt}%
\pgfpathmoveto{\pgfqpoint{8.427287in}{1.852954in}}%
\pgfpathlineto{\pgfqpoint{8.427791in}{1.852928in}}%
\pgfusepath{stroke}%
\end{pgfscope}%
\begin{pgfscope}%
\pgfpathrectangle{\pgfqpoint{6.720588in}{1.750000in}}{\pgfqpoint{2.279412in}{2.004545in}}%
\pgfusepath{clip}%
\pgfsetbuttcap%
\pgfsetroundjoin%
\pgfsetlinewidth{0.308864pt}%
\definecolor{currentstroke}{rgb}{0.268510,0.009605,0.335427}%
\pgfsetstrokecolor{currentstroke}%
\pgfsetdash{}{0pt}%
\pgfpathmoveto{\pgfqpoint{8.427791in}{1.852928in}}%
\pgfpathlineto{\pgfqpoint{8.428237in}{1.852907in}}%
\pgfusepath{stroke}%
\end{pgfscope}%
\begin{pgfscope}%
\pgfpathrectangle{\pgfqpoint{6.720588in}{1.750000in}}{\pgfqpoint{2.279412in}{2.004545in}}%
\pgfusepath{clip}%
\pgfsetbuttcap%
\pgfsetroundjoin%
\pgfsetlinewidth{0.308999pt}%
\definecolor{currentstroke}{rgb}{0.268510,0.009605,0.335427}%
\pgfsetstrokecolor{currentstroke}%
\pgfsetdash{}{0pt}%
\pgfpathmoveto{\pgfqpoint{8.428237in}{1.852907in}}%
\pgfpathlineto{\pgfqpoint{8.428281in}{1.852907in}}%
\pgfusepath{stroke}%
\end{pgfscope}%
\begin{pgfscope}%
\pgfpathrectangle{\pgfqpoint{6.720588in}{1.750000in}}{\pgfqpoint{2.279412in}{2.004545in}}%
\pgfusepath{clip}%
\pgfsetbuttcap%
\pgfsetroundjoin%
\pgfsetlinewidth{0.309013pt}%
\definecolor{currentstroke}{rgb}{0.268510,0.009605,0.335427}%
\pgfsetstrokecolor{currentstroke}%
\pgfsetdash{}{0pt}%
\pgfpathmoveto{\pgfqpoint{8.428281in}{1.852907in}}%
\pgfpathlineto{\pgfqpoint{8.427824in}{1.852932in}}%
\pgfusepath{stroke}%
\end{pgfscope}%
\begin{pgfscope}%
\pgfpathrectangle{\pgfqpoint{6.720588in}{1.750000in}}{\pgfqpoint{2.279412in}{2.004545in}}%
\pgfusepath{clip}%
\pgfsetbuttcap%
\pgfsetroundjoin%
\pgfsetlinewidth{0.308874pt}%
\definecolor{currentstroke}{rgb}{0.268510,0.009605,0.335427}%
\pgfsetstrokecolor{currentstroke}%
\pgfsetdash{}{0pt}%
\pgfpathmoveto{\pgfqpoint{8.427824in}{1.852932in}}%
\pgfpathlineto{\pgfqpoint{8.427153in}{1.852964in}}%
\pgfusepath{stroke}%
\end{pgfscope}%
\begin{pgfscope}%
\pgfpathrectangle{\pgfqpoint{6.720588in}{1.750000in}}{\pgfqpoint{2.279412in}{2.004545in}}%
\pgfusepath{clip}%
\pgfsetbuttcap%
\pgfsetroundjoin%
\pgfsetlinewidth{0.308669pt}%
\definecolor{currentstroke}{rgb}{0.268510,0.009605,0.335427}%
\pgfsetstrokecolor{currentstroke}%
\pgfsetdash{}{0pt}%
\pgfpathmoveto{\pgfqpoint{8.427153in}{1.852964in}}%
\pgfpathlineto{\pgfqpoint{8.427153in}{1.852964in}}%
\pgfusepath{stroke}%
\end{pgfscope}%
\begin{pgfscope}%
\pgfpathrectangle{\pgfqpoint{6.720588in}{1.750000in}}{\pgfqpoint{2.279412in}{2.004545in}}%
\pgfusepath{clip}%
\pgfsetbuttcap%
\pgfsetroundjoin%
\pgfsetlinewidth{0.308669pt}%
\definecolor{currentstroke}{rgb}{0.268510,0.009605,0.335427}%
\pgfsetstrokecolor{currentstroke}%
\pgfsetdash{}{0pt}%
\pgfpathmoveto{\pgfqpoint{8.427153in}{1.852964in}}%
\pgfpathlineto{\pgfqpoint{8.427794in}{1.852930in}}%
\pgfusepath{stroke}%
\end{pgfscope}%
\begin{pgfscope}%
\pgfpathrectangle{\pgfqpoint{6.720588in}{1.750000in}}{\pgfqpoint{2.279412in}{2.004545in}}%
\pgfusepath{clip}%
\pgfsetbuttcap%
\pgfsetroundjoin%
\pgfsetlinewidth{0.308864pt}%
\definecolor{currentstroke}{rgb}{0.268510,0.009605,0.335427}%
\pgfsetstrokecolor{currentstroke}%
\pgfsetdash{}{0pt}%
\pgfpathmoveto{\pgfqpoint{8.427794in}{1.852930in}}%
\pgfpathlineto{\pgfqpoint{8.428320in}{1.852905in}}%
\pgfusepath{stroke}%
\end{pgfscope}%
\begin{pgfscope}%
\pgfpathrectangle{\pgfqpoint{6.720588in}{1.750000in}}{\pgfqpoint{2.279412in}{2.004545in}}%
\pgfusepath{clip}%
\pgfsetbuttcap%
\pgfsetroundjoin%
\pgfsetlinewidth{0.309024pt}%
\definecolor{currentstroke}{rgb}{0.268510,0.009605,0.335427}%
\pgfsetstrokecolor{currentstroke}%
\pgfsetdash{}{0pt}%
\pgfpathmoveto{\pgfqpoint{8.428320in}{1.852905in}}%
\pgfpathlineto{\pgfqpoint{8.428366in}{1.852905in}}%
\pgfusepath{stroke}%
\end{pgfscope}%
\begin{pgfscope}%
\pgfpathrectangle{\pgfqpoint{6.720588in}{1.750000in}}{\pgfqpoint{2.279412in}{2.004545in}}%
\pgfusepath{clip}%
\pgfsetbuttcap%
\pgfsetroundjoin%
\pgfsetlinewidth{0.309039pt}%
\definecolor{currentstroke}{rgb}{0.268510,0.009605,0.335427}%
\pgfsetstrokecolor{currentstroke}%
\pgfsetdash{}{0pt}%
\pgfpathmoveto{\pgfqpoint{8.428366in}{1.852905in}}%
\pgfpathlineto{\pgfqpoint{8.427814in}{1.852934in}}%
\pgfusepath{stroke}%
\end{pgfscope}%
\begin{pgfscope}%
\pgfpathrectangle{\pgfqpoint{6.720588in}{1.750000in}}{\pgfqpoint{2.279412in}{2.004545in}}%
\pgfusepath{clip}%
\pgfsetbuttcap%
\pgfsetroundjoin%
\pgfsetlinewidth{0.308871pt}%
\definecolor{currentstroke}{rgb}{0.268510,0.009605,0.335427}%
\pgfsetstrokecolor{currentstroke}%
\pgfsetdash{}{0pt}%
\pgfpathmoveto{\pgfqpoint{8.427814in}{1.852934in}}%
\pgfpathlineto{\pgfqpoint{8.426967in}{1.852974in}}%
\pgfusepath{stroke}%
\end{pgfscope}%
\begin{pgfscope}%
\pgfpathrectangle{\pgfqpoint{6.720588in}{1.750000in}}{\pgfqpoint{2.279412in}{2.004545in}}%
\pgfusepath{clip}%
\pgfsetbuttcap%
\pgfsetroundjoin%
\pgfsetlinewidth{0.308611pt}%
\definecolor{currentstroke}{rgb}{0.268510,0.009605,0.335427}%
\pgfsetstrokecolor{currentstroke}%
\pgfsetdash{}{0pt}%
\pgfpathmoveto{\pgfqpoint{8.426967in}{1.852974in}}%
\pgfpathlineto{\pgfqpoint{8.426967in}{1.852974in}}%
\pgfusepath{stroke}%
\end{pgfscope}%
\begin{pgfscope}%
\pgfpathrectangle{\pgfqpoint{6.720588in}{1.750000in}}{\pgfqpoint{2.279412in}{2.004545in}}%
\pgfusepath{clip}%
\pgfsetbuttcap%
\pgfsetroundjoin%
\pgfsetlinewidth{0.308611pt}%
\definecolor{currentstroke}{rgb}{0.268510,0.009605,0.335427}%
\pgfsetstrokecolor{currentstroke}%
\pgfsetdash{}{0pt}%
\pgfpathmoveto{\pgfqpoint{8.426967in}{1.852974in}}%
\pgfpathlineto{\pgfqpoint{8.427777in}{1.852931in}}%
\pgfusepath{stroke}%
\end{pgfscope}%
\begin{pgfscope}%
\pgfpathrectangle{\pgfqpoint{6.720588in}{1.750000in}}{\pgfqpoint{2.279412in}{2.004545in}}%
\pgfusepath{clip}%
\pgfsetbuttcap%
\pgfsetroundjoin%
\pgfsetlinewidth{0.308859pt}%
\definecolor{currentstroke}{rgb}{0.268510,0.009605,0.335427}%
\pgfsetstrokecolor{currentstroke}%
\pgfsetdash{}{0pt}%
\pgfpathmoveto{\pgfqpoint{8.427777in}{1.852931in}}%
\pgfpathlineto{\pgfqpoint{8.428404in}{1.852900in}}%
\pgfusepath{stroke}%
\end{pgfscope}%
\begin{pgfscope}%
\pgfpathrectangle{\pgfqpoint{6.720588in}{1.750000in}}{\pgfqpoint{2.279412in}{2.004545in}}%
\pgfusepath{clip}%
\pgfsetbuttcap%
\pgfsetroundjoin%
\pgfsetlinewidth{0.309050pt}%
\definecolor{currentstroke}{rgb}{0.268510,0.009605,0.335427}%
\pgfsetstrokecolor{currentstroke}%
\pgfsetdash{}{0pt}%
\pgfpathmoveto{\pgfqpoint{8.428404in}{1.852900in}}%
\pgfpathlineto{\pgfqpoint{8.428470in}{1.852900in}}%
\pgfusepath{stroke}%
\end{pgfscope}%
\begin{pgfscope}%
\pgfpathrectangle{\pgfqpoint{6.720588in}{1.750000in}}{\pgfqpoint{2.279412in}{2.004545in}}%
\pgfusepath{clip}%
\pgfsetbuttcap%
\pgfsetroundjoin%
\pgfsetlinewidth{0.309070pt}%
\definecolor{currentstroke}{rgb}{0.268510,0.009605,0.335427}%
\pgfsetstrokecolor{currentstroke}%
\pgfsetdash{}{0pt}%
\pgfpathmoveto{\pgfqpoint{8.428470in}{1.852900in}}%
\pgfpathlineto{\pgfqpoint{8.427826in}{1.852934in}}%
\pgfusepath{stroke}%
\end{pgfscope}%
\begin{pgfscope}%
\pgfpathrectangle{\pgfqpoint{6.720588in}{1.750000in}}{\pgfqpoint{2.279412in}{2.004545in}}%
\pgfusepath{clip}%
\pgfsetbuttcap%
\pgfsetroundjoin%
\pgfsetlinewidth{0.308874pt}%
\definecolor{currentstroke}{rgb}{0.268510,0.009605,0.335427}%
\pgfsetstrokecolor{currentstroke}%
\pgfsetdash{}{0pt}%
\pgfpathmoveto{\pgfqpoint{8.427826in}{1.852934in}}%
\pgfpathlineto{\pgfqpoint{8.427826in}{1.852934in}}%
\pgfusepath{stroke}%
\end{pgfscope}%
\begin{pgfscope}%
\pgfpathrectangle{\pgfqpoint{6.720588in}{1.750000in}}{\pgfqpoint{2.279412in}{2.004545in}}%
\pgfusepath{clip}%
\pgfsetbuttcap%
\pgfsetroundjoin%
\pgfsetlinewidth{0.308874pt}%
\definecolor{currentstroke}{rgb}{0.268510,0.009605,0.335427}%
\pgfsetstrokecolor{currentstroke}%
\pgfsetdash{}{0pt}%
\pgfpathmoveto{\pgfqpoint{8.427826in}{1.852934in}}%
\pgfpathlineto{\pgfqpoint{8.427693in}{1.852940in}}%
\pgfusepath{stroke}%
\end{pgfscope}%
\begin{pgfscope}%
\pgfpathrectangle{\pgfqpoint{6.720588in}{1.750000in}}{\pgfqpoint{2.279412in}{2.004545in}}%
\pgfusepath{clip}%
\pgfsetbuttcap%
\pgfsetroundjoin%
\pgfsetlinewidth{0.308834pt}%
\definecolor{currentstroke}{rgb}{0.268510,0.009605,0.335427}%
\pgfsetstrokecolor{currentstroke}%
\pgfsetdash{}{0pt}%
\pgfpathmoveto{\pgfqpoint{8.427693in}{1.852940in}}%
\pgfpathlineto{\pgfqpoint{8.427673in}{1.852940in}}%
\pgfusepath{stroke}%
\end{pgfscope}%
\begin{pgfscope}%
\pgfpathrectangle{\pgfqpoint{6.720588in}{1.750000in}}{\pgfqpoint{2.279412in}{2.004545in}}%
\pgfusepath{clip}%
\pgfsetbuttcap%
\pgfsetroundjoin%
\pgfsetlinewidth{0.308828pt}%
\definecolor{currentstroke}{rgb}{0.268510,0.009605,0.335427}%
\pgfsetstrokecolor{currentstroke}%
\pgfsetdash{}{0pt}%
\pgfpathmoveto{\pgfqpoint{8.427673in}{1.852940in}}%
\pgfpathlineto{\pgfqpoint{8.427797in}{1.852933in}}%
\pgfusepath{stroke}%
\end{pgfscope}%
\begin{pgfscope}%
\pgfpathrectangle{\pgfqpoint{6.720588in}{1.750000in}}{\pgfqpoint{2.279412in}{2.004545in}}%
\pgfusepath{clip}%
\pgfsetbuttcap%
\pgfsetroundjoin%
\pgfsetlinewidth{0.308866pt}%
\definecolor{currentstroke}{rgb}{0.268510,0.009605,0.335427}%
\pgfsetstrokecolor{currentstroke}%
\pgfsetdash{}{0pt}%
\pgfpathmoveto{\pgfqpoint{8.427797in}{1.852933in}}%
\pgfpathlineto{\pgfqpoint{8.427938in}{1.852926in}}%
\pgfusepath{stroke}%
\end{pgfscope}%
\begin{pgfscope}%
\pgfpathrectangle{\pgfqpoint{6.720588in}{1.750000in}}{\pgfqpoint{2.279412in}{2.004545in}}%
\pgfusepath{clip}%
\pgfsetbuttcap%
\pgfsetroundjoin%
\pgfsetlinewidth{0.308909pt}%
\definecolor{currentstroke}{rgb}{0.268510,0.009605,0.335427}%
\pgfsetstrokecolor{currentstroke}%
\pgfsetdash{}{0pt}%
\pgfpathmoveto{\pgfqpoint{8.427938in}{1.852926in}}%
\pgfpathlineto{\pgfqpoint{8.427960in}{1.852925in}}%
\pgfusepath{stroke}%
\end{pgfscope}%
\begin{pgfscope}%
\pgfpathrectangle{\pgfqpoint{6.720588in}{1.750000in}}{\pgfqpoint{2.279412in}{2.004545in}}%
\pgfusepath{clip}%
\pgfsetbuttcap%
\pgfsetroundjoin%
\pgfsetlinewidth{0.308915pt}%
\definecolor{currentstroke}{rgb}{0.268510,0.009605,0.335427}%
\pgfsetstrokecolor{currentstroke}%
\pgfsetdash{}{0pt}%
\pgfpathmoveto{\pgfqpoint{8.427960in}{1.852925in}}%
\pgfpathlineto{\pgfqpoint{8.427829in}{1.852932in}}%
\pgfusepath{stroke}%
\end{pgfscope}%
\begin{pgfscope}%
\pgfpathrectangle{\pgfqpoint{6.720588in}{1.750000in}}{\pgfqpoint{2.279412in}{2.004545in}}%
\pgfusepath{clip}%
\pgfsetbuttcap%
\pgfsetroundjoin%
\pgfsetlinewidth{0.308875pt}%
\definecolor{currentstroke}{rgb}{0.268510,0.009605,0.335427}%
\pgfsetstrokecolor{currentstroke}%
\pgfsetdash{}{0pt}%
\pgfpathmoveto{\pgfqpoint{8.427829in}{1.852932in}}%
\pgfpathlineto{\pgfqpoint{8.427656in}{1.852940in}}%
\pgfusepath{stroke}%
\end{pgfscope}%
\begin{pgfscope}%
\pgfpathrectangle{\pgfqpoint{6.720588in}{1.750000in}}{\pgfqpoint{2.279412in}{2.004545in}}%
\pgfusepath{clip}%
\pgfsetbuttcap%
\pgfsetroundjoin%
\pgfsetlinewidth{0.308823pt}%
\definecolor{currentstroke}{rgb}{0.268510,0.009605,0.335427}%
\pgfsetstrokecolor{currentstroke}%
\pgfsetdash{}{0pt}%
\pgfpathmoveto{\pgfqpoint{8.427656in}{1.852940in}}%
\pgfpathlineto{\pgfqpoint{8.427632in}{1.852940in}}%
\pgfusepath{stroke}%
\end{pgfscope}%
\begin{pgfscope}%
\pgfpathrectangle{\pgfqpoint{6.720588in}{1.750000in}}{\pgfqpoint{2.279412in}{2.004545in}}%
\pgfusepath{clip}%
\pgfsetbuttcap%
\pgfsetroundjoin%
\pgfsetlinewidth{0.308815pt}%
\definecolor{currentstroke}{rgb}{0.268510,0.009605,0.335427}%
\pgfsetstrokecolor{currentstroke}%
\pgfsetdash{}{0pt}%
\pgfpathmoveto{\pgfqpoint{8.427632in}{1.852940in}}%
\pgfpathlineto{\pgfqpoint{8.427794in}{1.852932in}}%
\pgfusepath{stroke}%
\end{pgfscope}%
\begin{pgfscope}%
\pgfpathrectangle{\pgfqpoint{6.720588in}{1.750000in}}{\pgfqpoint{2.279412in}{2.004545in}}%
\pgfusepath{clip}%
\pgfsetbuttcap%
\pgfsetroundjoin%
\pgfsetlinewidth{0.308865pt}%
\definecolor{currentstroke}{rgb}{0.268510,0.009605,0.335427}%
\pgfsetstrokecolor{currentstroke}%
\pgfsetdash{}{0pt}%
\pgfpathmoveto{\pgfqpoint{8.427794in}{1.852932in}}%
\pgfpathlineto{\pgfqpoint{8.427974in}{1.852923in}}%
\pgfusepath{stroke}%
\end{pgfscope}%
\begin{pgfscope}%
\pgfpathrectangle{\pgfqpoint{6.720588in}{1.750000in}}{\pgfqpoint{2.279412in}{2.004545in}}%
\pgfusepath{clip}%
\pgfsetbuttcap%
\pgfsetroundjoin%
\pgfsetlinewidth{0.308919pt}%
\definecolor{currentstroke}{rgb}{0.268510,0.009605,0.335427}%
\pgfsetstrokecolor{currentstroke}%
\pgfsetdash{}{0pt}%
\pgfpathmoveto{\pgfqpoint{8.427974in}{1.852923in}}%
\pgfpathlineto{\pgfqpoint{8.428001in}{1.852922in}}%
\pgfusepath{stroke}%
\end{pgfscope}%
\begin{pgfscope}%
\pgfpathrectangle{\pgfqpoint{6.720588in}{1.750000in}}{\pgfqpoint{2.279412in}{2.004545in}}%
\pgfusepath{clip}%
\pgfsetbuttcap%
\pgfsetroundjoin%
\pgfsetlinewidth{0.308928pt}%
\definecolor{currentstroke}{rgb}{0.268510,0.009605,0.335427}%
\pgfsetstrokecolor{currentstroke}%
\pgfsetdash{}{0pt}%
\pgfpathmoveto{\pgfqpoint{8.428001in}{1.852922in}}%
\pgfpathlineto{\pgfqpoint{8.427832in}{1.852932in}}%
\pgfusepath{stroke}%
\end{pgfscope}%
\begin{pgfscope}%
\pgfpathrectangle{\pgfqpoint{6.720588in}{1.750000in}}{\pgfqpoint{2.279412in}{2.004545in}}%
\pgfusepath{clip}%
\pgfsetbuttcap%
\pgfsetroundjoin%
\pgfsetlinewidth{0.308876pt}%
\definecolor{currentstroke}{rgb}{0.268510,0.009605,0.335427}%
\pgfsetstrokecolor{currentstroke}%
\pgfsetdash{}{0pt}%
\pgfpathmoveto{\pgfqpoint{8.427832in}{1.852932in}}%
\pgfpathlineto{\pgfqpoint{8.427607in}{1.852942in}}%
\pgfusepath{stroke}%
\end{pgfscope}%
\begin{pgfscope}%
\pgfpathrectangle{\pgfqpoint{6.720588in}{1.750000in}}{\pgfqpoint{2.279412in}{2.004545in}}%
\pgfusepath{clip}%
\pgfsetbuttcap%
\pgfsetroundjoin%
\pgfsetlinewidth{0.308808pt}%
\definecolor{currentstroke}{rgb}{0.268510,0.009605,0.335427}%
\pgfsetstrokecolor{currentstroke}%
\pgfsetdash{}{0pt}%
\pgfpathmoveto{\pgfqpoint{8.427607in}{1.852942in}}%
\pgfpathlineto{\pgfqpoint{8.427578in}{1.852942in}}%
\pgfusepath{stroke}%
\end{pgfscope}%
\begin{pgfscope}%
\pgfpathrectangle{\pgfqpoint{6.720588in}{1.750000in}}{\pgfqpoint{2.279412in}{2.004545in}}%
\pgfusepath{clip}%
\pgfsetbuttcap%
\pgfsetroundjoin%
\pgfsetlinewidth{0.308799pt}%
\definecolor{currentstroke}{rgb}{0.268510,0.009605,0.335427}%
\pgfsetstrokecolor{currentstroke}%
\pgfsetdash{}{0pt}%
\pgfpathmoveto{\pgfqpoint{8.427578in}{1.852942in}}%
\pgfpathlineto{\pgfqpoint{8.427790in}{1.852931in}}%
\pgfusepath{stroke}%
\end{pgfscope}%
\begin{pgfscope}%
\pgfpathrectangle{\pgfqpoint{6.720588in}{1.750000in}}{\pgfqpoint{2.279412in}{2.004545in}}%
\pgfusepath{clip}%
\pgfsetbuttcap%
\pgfsetroundjoin%
\pgfsetlinewidth{0.308863pt}%
\definecolor{currentstroke}{rgb}{0.268510,0.009605,0.335427}%
\pgfsetstrokecolor{currentstroke}%
\pgfsetdash{}{0pt}%
\pgfpathmoveto{\pgfqpoint{8.427790in}{1.852931in}}%
\pgfpathlineto{\pgfqpoint{8.428018in}{1.852920in}}%
\pgfusepath{stroke}%
\end{pgfscope}%
\begin{pgfscope}%
\pgfpathrectangle{\pgfqpoint{6.720588in}{1.750000in}}{\pgfqpoint{2.279412in}{2.004545in}}%
\pgfusepath{clip}%
\pgfsetbuttcap%
\pgfsetroundjoin%
\pgfsetlinewidth{0.308933pt}%
\definecolor{currentstroke}{rgb}{0.268510,0.009605,0.335427}%
\pgfsetstrokecolor{currentstroke}%
\pgfsetdash{}{0pt}%
\pgfpathmoveto{\pgfqpoint{8.428018in}{1.852920in}}%
\pgfpathlineto{\pgfqpoint{8.428051in}{1.852920in}}%
\pgfusepath{stroke}%
\end{pgfscope}%
\begin{pgfscope}%
\pgfpathrectangle{\pgfqpoint{6.720588in}{1.750000in}}{\pgfqpoint{2.279412in}{2.004545in}}%
\pgfusepath{clip}%
\pgfsetbuttcap%
\pgfsetroundjoin%
\pgfsetlinewidth{0.308943pt}%
\definecolor{currentstroke}{rgb}{0.268510,0.009605,0.335427}%
\pgfsetstrokecolor{currentstroke}%
\pgfsetdash{}{0pt}%
\pgfpathmoveto{\pgfqpoint{8.428051in}{1.852920in}}%
\pgfpathlineto{\pgfqpoint{8.427836in}{1.852931in}}%
\pgfusepath{stroke}%
\end{pgfscope}%
\begin{pgfscope}%
\pgfpathrectangle{\pgfqpoint{6.720588in}{1.750000in}}{\pgfqpoint{2.279412in}{2.004545in}}%
\pgfusepath{clip}%
\pgfsetbuttcap%
\pgfsetroundjoin%
\pgfsetlinewidth{0.308877pt}%
\definecolor{currentstroke}{rgb}{0.268510,0.009605,0.335427}%
\pgfsetstrokecolor{currentstroke}%
\pgfsetdash{}{0pt}%
\pgfpathmoveto{\pgfqpoint{8.427836in}{1.852931in}}%
\pgfpathlineto{\pgfqpoint{8.427542in}{1.852945in}}%
\pgfusepath{stroke}%
\end{pgfscope}%
\begin{pgfscope}%
\pgfpathrectangle{\pgfqpoint{6.720588in}{1.750000in}}{\pgfqpoint{2.279412in}{2.004545in}}%
\pgfusepath{clip}%
\pgfsetbuttcap%
\pgfsetroundjoin%
\pgfsetlinewidth{0.308788pt}%
\definecolor{currentstroke}{rgb}{0.268510,0.009605,0.335427}%
\pgfsetstrokecolor{currentstroke}%
\pgfsetdash{}{0pt}%
\pgfpathmoveto{\pgfqpoint{8.427542in}{1.852945in}}%
\pgfpathlineto{\pgfqpoint{8.427507in}{1.852945in}}%
\pgfusepath{stroke}%
\end{pgfscope}%
\begin{pgfscope}%
\pgfpathrectangle{\pgfqpoint{6.720588in}{1.750000in}}{\pgfqpoint{2.279412in}{2.004545in}}%
\pgfusepath{clip}%
\pgfsetbuttcap%
\pgfsetroundjoin%
\pgfsetlinewidth{0.308777pt}%
\definecolor{currentstroke}{rgb}{0.268510,0.009605,0.335427}%
\pgfsetstrokecolor{currentstroke}%
\pgfsetdash{}{0pt}%
\pgfpathmoveto{\pgfqpoint{8.427507in}{1.852945in}}%
\pgfpathlineto{\pgfqpoint{8.427784in}{1.852931in}}%
\pgfusepath{stroke}%
\end{pgfscope}%
\begin{pgfscope}%
\pgfpathrectangle{\pgfqpoint{6.720588in}{1.750000in}}{\pgfqpoint{2.279412in}{2.004545in}}%
\pgfusepath{clip}%
\pgfsetbuttcap%
\pgfsetroundjoin%
\pgfsetlinewidth{0.308862pt}%
\definecolor{currentstroke}{rgb}{0.268510,0.009605,0.335427}%
\pgfsetstrokecolor{currentstroke}%
\pgfsetdash{}{0pt}%
\pgfpathmoveto{\pgfqpoint{8.427784in}{1.852931in}}%
\pgfpathlineto{\pgfqpoint{8.428071in}{1.852917in}}%
\pgfusepath{stroke}%
\end{pgfscope}%
\begin{pgfscope}%
\pgfpathrectangle{\pgfqpoint{6.720588in}{1.750000in}}{\pgfqpoint{2.279412in}{2.004545in}}%
\pgfusepath{clip}%
\pgfsetbuttcap%
\pgfsetroundjoin%
\pgfsetlinewidth{0.308949pt}%
\definecolor{currentstroke}{rgb}{0.268510,0.009605,0.335427}%
\pgfsetstrokecolor{currentstroke}%
\pgfsetdash{}{0pt}%
\pgfpathmoveto{\pgfqpoint{8.428071in}{1.852917in}}%
\pgfpathlineto{\pgfqpoint{8.428112in}{1.852917in}}%
\pgfusepath{stroke}%
\end{pgfscope}%
\begin{pgfscope}%
\pgfpathrectangle{\pgfqpoint{6.720588in}{1.750000in}}{\pgfqpoint{2.279412in}{2.004545in}}%
\pgfusepath{clip}%
\pgfsetbuttcap%
\pgfsetroundjoin%
\pgfsetlinewidth{0.308962pt}%
\definecolor{currentstroke}{rgb}{0.268510,0.009605,0.335427}%
\pgfsetstrokecolor{currentstroke}%
\pgfsetdash{}{0pt}%
\pgfpathmoveto{\pgfqpoint{8.428112in}{1.852917in}}%
\pgfpathlineto{\pgfqpoint{8.427840in}{1.852931in}}%
\pgfusepath{stroke}%
\end{pgfscope}%
\begin{pgfscope}%
\pgfpathrectangle{\pgfqpoint{6.720588in}{1.750000in}}{\pgfqpoint{2.279412in}{2.004545in}}%
\pgfusepath{clip}%
\pgfsetbuttcap%
\pgfsetroundjoin%
\pgfsetlinewidth{0.308879pt}%
\definecolor{currentstroke}{rgb}{0.268510,0.009605,0.335427}%
\pgfsetstrokecolor{currentstroke}%
\pgfsetdash{}{0pt}%
\pgfpathmoveto{\pgfqpoint{8.427840in}{1.852931in}}%
\pgfpathlineto{\pgfqpoint{8.427457in}{1.852950in}}%
\pgfusepath{stroke}%
\end{pgfscope}%
\begin{pgfscope}%
\pgfpathrectangle{\pgfqpoint{6.720588in}{1.750000in}}{\pgfqpoint{2.279412in}{2.004545in}}%
\pgfusepath{clip}%
\pgfsetbuttcap%
\pgfsetroundjoin%
\pgfsetlinewidth{0.308762pt}%
\definecolor{currentstroke}{rgb}{0.268510,0.009605,0.335427}%
\pgfsetstrokecolor{currentstroke}%
\pgfsetdash{}{0pt}%
\pgfpathmoveto{\pgfqpoint{8.427457in}{1.852950in}}%
\pgfpathlineto{\pgfqpoint{8.427415in}{1.852949in}}%
\pgfusepath{stroke}%
\end{pgfscope}%
\begin{pgfscope}%
\pgfpathrectangle{\pgfqpoint{6.720588in}{1.750000in}}{\pgfqpoint{2.279412in}{2.004545in}}%
\pgfusepath{clip}%
\pgfsetbuttcap%
\pgfsetroundjoin%
\pgfsetlinewidth{0.308749pt}%
\definecolor{currentstroke}{rgb}{0.268510,0.009605,0.335427}%
\pgfsetstrokecolor{currentstroke}%
\pgfsetdash{}{0pt}%
\pgfpathmoveto{\pgfqpoint{8.427415in}{1.852949in}}%
\pgfpathlineto{\pgfqpoint{8.427777in}{1.852930in}}%
\pgfusepath{stroke}%
\end{pgfscope}%
\begin{pgfscope}%
\pgfpathrectangle{\pgfqpoint{6.720588in}{1.750000in}}{\pgfqpoint{2.279412in}{2.004545in}}%
\pgfusepath{clip}%
\pgfsetbuttcap%
\pgfsetroundjoin%
\pgfsetlinewidth{0.308859pt}%
\definecolor{currentstroke}{rgb}{0.268510,0.009605,0.335427}%
\pgfsetstrokecolor{currentstroke}%
\pgfsetdash{}{0pt}%
\pgfpathmoveto{\pgfqpoint{8.427777in}{1.852930in}}%
\pgfpathlineto{\pgfqpoint{8.428136in}{1.852913in}}%
\pgfusepath{stroke}%
\end{pgfscope}%
\begin{pgfscope}%
\pgfpathrectangle{\pgfqpoint{6.720588in}{1.750000in}}{\pgfqpoint{2.279412in}{2.004545in}}%
\pgfusepath{clip}%
\pgfsetbuttcap%
\pgfsetroundjoin%
\pgfsetlinewidth{0.308969pt}%
\definecolor{currentstroke}{rgb}{0.268510,0.009605,0.335427}%
\pgfsetstrokecolor{currentstroke}%
\pgfsetdash{}{0pt}%
\pgfpathmoveto{\pgfqpoint{8.428136in}{1.852913in}}%
\pgfpathlineto{\pgfqpoint{8.428186in}{1.852913in}}%
\pgfusepath{stroke}%
\end{pgfscope}%
\begin{pgfscope}%
\pgfpathrectangle{\pgfqpoint{6.720588in}{1.750000in}}{\pgfqpoint{2.279412in}{2.004545in}}%
\pgfusepath{clip}%
\pgfsetbuttcap%
\pgfsetroundjoin%
\pgfsetlinewidth{0.308984pt}%
\definecolor{currentstroke}{rgb}{0.268510,0.009605,0.335427}%
\pgfsetstrokecolor{currentstroke}%
\pgfsetdash{}{0pt}%
\pgfpathmoveto{\pgfqpoint{8.428186in}{1.852913in}}%
\pgfpathlineto{\pgfqpoint{8.427845in}{1.852931in}}%
\pgfusepath{stroke}%
\end{pgfscope}%
\begin{pgfscope}%
\pgfpathrectangle{\pgfqpoint{6.720588in}{1.750000in}}{\pgfqpoint{2.279412in}{2.004545in}}%
\pgfusepath{clip}%
\pgfsetbuttcap%
\pgfsetroundjoin%
\pgfsetlinewidth{0.308880pt}%
\definecolor{currentstroke}{rgb}{0.268510,0.009605,0.335427}%
\pgfsetstrokecolor{currentstroke}%
\pgfsetdash{}{0pt}%
\pgfpathmoveto{\pgfqpoint{8.427845in}{1.852931in}}%
\pgfpathlineto{\pgfqpoint{8.427346in}{1.852955in}}%
\pgfusepath{stroke}%
\end{pgfscope}%
\begin{pgfscope}%
\pgfpathrectangle{\pgfqpoint{6.720588in}{1.750000in}}{\pgfqpoint{2.279412in}{2.004545in}}%
\pgfusepath{clip}%
\pgfsetbuttcap%
\pgfsetroundjoin%
\pgfsetlinewidth{0.308728pt}%
\definecolor{currentstroke}{rgb}{0.268510,0.009605,0.335427}%
\pgfsetstrokecolor{currentstroke}%
\pgfsetdash{}{0pt}%
\pgfpathmoveto{\pgfqpoint{8.427346in}{1.852955in}}%
\pgfpathlineto{\pgfqpoint{8.427297in}{1.852954in}}%
\pgfusepath{stroke}%
\end{pgfscope}%
\begin{pgfscope}%
\pgfpathrectangle{\pgfqpoint{6.720588in}{1.750000in}}{\pgfqpoint{2.279412in}{2.004545in}}%
\pgfusepath{clip}%
\pgfsetbuttcap%
\pgfsetroundjoin%
\pgfsetlinewidth{0.308713pt}%
\definecolor{currentstroke}{rgb}{0.268510,0.009605,0.335427}%
\pgfsetstrokecolor{currentstroke}%
\pgfsetdash{}{0pt}%
\pgfpathmoveto{\pgfqpoint{8.427297in}{1.852954in}}%
\pgfpathlineto{\pgfqpoint{8.427768in}{1.852929in}}%
\pgfusepath{stroke}%
\end{pgfscope}%
\begin{pgfscope}%
\pgfpathrectangle{\pgfqpoint{6.720588in}{1.750000in}}{\pgfqpoint{2.279412in}{2.004545in}}%
\pgfusepath{clip}%
\pgfsetbuttcap%
\pgfsetroundjoin%
\pgfsetlinewidth{0.308856pt}%
\definecolor{currentstroke}{rgb}{0.268510,0.009605,0.335427}%
\pgfsetstrokecolor{currentstroke}%
\pgfsetdash{}{0pt}%
\pgfpathmoveto{\pgfqpoint{8.427768in}{1.852929in}}%
\pgfpathlineto{\pgfqpoint{8.428210in}{1.852909in}}%
\pgfusepath{stroke}%
\end{pgfscope}%
\begin{pgfscope}%
\pgfpathrectangle{\pgfqpoint{6.720588in}{1.750000in}}{\pgfqpoint{2.279412in}{2.004545in}}%
\pgfusepath{clip}%
\pgfsetbuttcap%
\pgfsetroundjoin%
\pgfsetlinewidth{0.308991pt}%
\definecolor{currentstroke}{rgb}{0.268510,0.009605,0.335427}%
\pgfsetstrokecolor{currentstroke}%
\pgfsetdash{}{0pt}%
\pgfpathmoveto{\pgfqpoint{8.428210in}{1.852909in}}%
\pgfpathlineto{\pgfqpoint{8.428273in}{1.852908in}}%
\pgfusepath{stroke}%
\end{pgfscope}%
\begin{pgfscope}%
\pgfpathrectangle{\pgfqpoint{6.720588in}{1.750000in}}{\pgfqpoint{2.279412in}{2.004545in}}%
\pgfusepath{clip}%
\pgfsetbuttcap%
\pgfsetroundjoin%
\pgfsetlinewidth{0.309010pt}%
\definecolor{currentstroke}{rgb}{0.268510,0.009605,0.335427}%
\pgfsetstrokecolor{currentstroke}%
\pgfsetdash{}{0pt}%
\pgfpathmoveto{\pgfqpoint{8.428273in}{1.852908in}}%
\pgfpathlineto{\pgfqpoint{8.427852in}{1.852931in}}%
\pgfusepath{stroke}%
\end{pgfscope}%
\begin{pgfscope}%
\pgfpathrectangle{\pgfqpoint{6.720588in}{1.750000in}}{\pgfqpoint{2.279412in}{2.004545in}}%
\pgfusepath{clip}%
\pgfsetbuttcap%
\pgfsetroundjoin%
\pgfsetlinewidth{0.308882pt}%
\definecolor{currentstroke}{rgb}{0.268510,0.009605,0.335427}%
\pgfsetstrokecolor{currentstroke}%
\pgfsetdash{}{0pt}%
\pgfpathmoveto{\pgfqpoint{8.427852in}{1.852931in}}%
\pgfpathlineto{\pgfqpoint{8.427204in}{1.852962in}}%
\pgfusepath{stroke}%
\end{pgfscope}%
\begin{pgfscope}%
\pgfpathrectangle{\pgfqpoint{6.720588in}{1.750000in}}{\pgfqpoint{2.279412in}{2.004545in}}%
\pgfusepath{clip}%
\pgfsetbuttcap%
\pgfsetroundjoin%
\pgfsetlinewidth{0.308684pt}%
\definecolor{currentstroke}{rgb}{0.268510,0.009605,0.335427}%
\pgfsetstrokecolor{currentstroke}%
\pgfsetdash{}{0pt}%
\pgfpathmoveto{\pgfqpoint{8.427204in}{1.852962in}}%
\pgfpathlineto{\pgfqpoint{8.427204in}{1.852962in}}%
\pgfusepath{stroke}%
\end{pgfscope}%
\begin{pgfscope}%
\pgfpathrectangle{\pgfqpoint{6.720588in}{1.750000in}}{\pgfqpoint{2.279412in}{2.004545in}}%
\pgfusepath{clip}%
\pgfsetbuttcap%
\pgfsetroundjoin%
\pgfsetlinewidth{0.308684pt}%
\definecolor{currentstroke}{rgb}{0.268510,0.009605,0.335427}%
\pgfsetstrokecolor{currentstroke}%
\pgfsetdash{}{0pt}%
\pgfpathmoveto{\pgfqpoint{8.427204in}{1.852962in}}%
\pgfpathlineto{\pgfqpoint{8.427793in}{1.852930in}}%
\pgfusepath{stroke}%
\end{pgfscope}%
\begin{pgfscope}%
\pgfpathrectangle{\pgfqpoint{6.720588in}{1.750000in}}{\pgfqpoint{2.279412in}{2.004545in}}%
\pgfusepath{clip}%
\pgfsetbuttcap%
\pgfsetroundjoin%
\pgfsetlinewidth{0.308864pt}%
\definecolor{currentstroke}{rgb}{0.268510,0.009605,0.335427}%
\pgfsetstrokecolor{currentstroke}%
\pgfsetdash{}{0pt}%
\pgfpathmoveto{\pgfqpoint{8.427793in}{1.852930in}}%
\pgfpathlineto{\pgfqpoint{8.428290in}{1.852906in}}%
\pgfusepath{stroke}%
\end{pgfscope}%
\begin{pgfscope}%
\pgfpathrectangle{\pgfqpoint{6.720588in}{1.750000in}}{\pgfqpoint{2.279412in}{2.004545in}}%
\pgfusepath{clip}%
\pgfsetbuttcap%
\pgfsetroundjoin%
\pgfsetlinewidth{0.309015pt}%
\definecolor{currentstroke}{rgb}{0.268510,0.009605,0.335427}%
\pgfsetstrokecolor{currentstroke}%
\pgfsetdash{}{0pt}%
\pgfpathmoveto{\pgfqpoint{8.428290in}{1.852906in}}%
\pgfpathlineto{\pgfqpoint{8.428335in}{1.852906in}}%
\pgfusepath{stroke}%
\end{pgfscope}%
\begin{pgfscope}%
\pgfpathrectangle{\pgfqpoint{6.720588in}{1.750000in}}{\pgfqpoint{2.279412in}{2.004545in}}%
\pgfusepath{clip}%
\pgfsetbuttcap%
\pgfsetroundjoin%
\pgfsetlinewidth{0.309029pt}%
\definecolor{currentstroke}{rgb}{0.268510,0.009605,0.335427}%
\pgfsetstrokecolor{currentstroke}%
\pgfsetdash{}{0pt}%
\pgfpathmoveto{\pgfqpoint{8.428335in}{1.852906in}}%
\pgfpathlineto{\pgfqpoint{8.427816in}{1.852934in}}%
\pgfusepath{stroke}%
\end{pgfscope}%
\begin{pgfscope}%
\pgfpathrectangle{\pgfqpoint{6.720588in}{1.750000in}}{\pgfqpoint{2.279412in}{2.004545in}}%
\pgfusepath{clip}%
\pgfsetbuttcap%
\pgfsetroundjoin%
\pgfsetlinewidth{0.308872pt}%
\definecolor{currentstroke}{rgb}{0.268510,0.009605,0.335427}%
\pgfsetstrokecolor{currentstroke}%
\pgfsetdash{}{0pt}%
\pgfpathmoveto{\pgfqpoint{8.427816in}{1.852934in}}%
\pgfpathlineto{\pgfqpoint{8.427037in}{1.852971in}}%
\pgfusepath{stroke}%
\end{pgfscope}%
\begin{pgfscope}%
\pgfpathrectangle{\pgfqpoint{6.720588in}{1.750000in}}{\pgfqpoint{2.279412in}{2.004545in}}%
\pgfusepath{clip}%
\pgfsetbuttcap%
\pgfsetroundjoin%
\pgfsetlinewidth{0.308633pt}%
\definecolor{currentstroke}{rgb}{0.268510,0.009605,0.335427}%
\pgfsetstrokecolor{currentstroke}%
\pgfsetdash{}{0pt}%
\pgfpathmoveto{\pgfqpoint{8.427037in}{1.852971in}}%
\pgfpathlineto{\pgfqpoint{8.427037in}{1.852971in}}%
\pgfusepath{stroke}%
\end{pgfscope}%
\begin{pgfscope}%
\pgfpathrectangle{\pgfqpoint{6.720588in}{1.750000in}}{\pgfqpoint{2.279412in}{2.004545in}}%
\pgfusepath{clip}%
\pgfsetbuttcap%
\pgfsetroundjoin%
\pgfsetlinewidth{0.308633pt}%
\definecolor{currentstroke}{rgb}{0.268510,0.009605,0.335427}%
\pgfsetstrokecolor{currentstroke}%
\pgfsetdash{}{0pt}%
\pgfpathmoveto{\pgfqpoint{8.427037in}{1.852971in}}%
\pgfpathlineto{\pgfqpoint{8.427784in}{1.852931in}}%
\pgfusepath{stroke}%
\end{pgfscope}%
\begin{pgfscope}%
\pgfpathrectangle{\pgfqpoint{6.720588in}{1.750000in}}{\pgfqpoint{2.279412in}{2.004545in}}%
\pgfusepath{clip}%
\pgfsetbuttcap%
\pgfsetroundjoin%
\pgfsetlinewidth{0.308861pt}%
\definecolor{currentstroke}{rgb}{0.268510,0.009605,0.335427}%
\pgfsetstrokecolor{currentstroke}%
\pgfsetdash{}{0pt}%
\pgfpathmoveto{\pgfqpoint{8.427784in}{1.852931in}}%
\pgfpathlineto{\pgfqpoint{8.428375in}{1.852902in}}%
\pgfusepath{stroke}%
\end{pgfscope}%
\begin{pgfscope}%
\pgfpathrectangle{\pgfqpoint{6.720588in}{1.750000in}}{\pgfqpoint{2.279412in}{2.004545in}}%
\pgfusepath{clip}%
\pgfsetbuttcap%
\pgfsetroundjoin%
\pgfsetlinewidth{0.309041pt}%
\definecolor{currentstroke}{rgb}{0.268510,0.009605,0.335427}%
\pgfsetstrokecolor{currentstroke}%
\pgfsetdash{}{0pt}%
\pgfpathmoveto{\pgfqpoint{8.428375in}{1.852902in}}%
\pgfpathlineto{\pgfqpoint{8.428433in}{1.852902in}}%
\pgfusepath{stroke}%
\end{pgfscope}%
\begin{pgfscope}%
\pgfpathrectangle{\pgfqpoint{6.720588in}{1.750000in}}{\pgfqpoint{2.279412in}{2.004545in}}%
\pgfusepath{clip}%
\pgfsetbuttcap%
\pgfsetroundjoin%
\pgfsetlinewidth{0.309059pt}%
\definecolor{currentstroke}{rgb}{0.268510,0.009605,0.335427}%
\pgfsetstrokecolor{currentstroke}%
\pgfsetdash{}{0pt}%
\pgfpathmoveto{\pgfqpoint{8.428433in}{1.852902in}}%
\pgfpathlineto{\pgfqpoint{8.427820in}{1.852934in}}%
\pgfusepath{stroke}%
\end{pgfscope}%
\begin{pgfscope}%
\pgfpathrectangle{\pgfqpoint{6.720588in}{1.750000in}}{\pgfqpoint{2.279412in}{2.004545in}}%
\pgfusepath{clip}%
\pgfsetbuttcap%
\pgfsetroundjoin%
\pgfsetlinewidth{0.308873pt}%
\definecolor{currentstroke}{rgb}{0.268510,0.009605,0.335427}%
\pgfsetstrokecolor{currentstroke}%
\pgfsetdash{}{0pt}%
\pgfpathmoveto{\pgfqpoint{8.427820in}{1.852934in}}%
\pgfpathlineto{\pgfqpoint{8.427820in}{1.852934in}}%
\pgfusepath{stroke}%
\end{pgfscope}%
\begin{pgfscope}%
\pgfpathrectangle{\pgfqpoint{6.720588in}{1.750000in}}{\pgfqpoint{2.279412in}{2.004545in}}%
\pgfusepath{clip}%
\pgfsetbuttcap%
\pgfsetroundjoin%
\pgfsetlinewidth{0.308873pt}%
\definecolor{currentstroke}{rgb}{0.268510,0.009605,0.335427}%
\pgfsetstrokecolor{currentstroke}%
\pgfsetdash{}{0pt}%
\pgfpathmoveto{\pgfqpoint{8.427820in}{1.852934in}}%
\pgfpathlineto{\pgfqpoint{8.427701in}{1.852939in}}%
\pgfusepath{stroke}%
\end{pgfscope}%
\begin{pgfscope}%
\pgfpathrectangle{\pgfqpoint{6.720588in}{1.750000in}}{\pgfqpoint{2.279412in}{2.004545in}}%
\pgfusepath{clip}%
\pgfsetbuttcap%
\pgfsetroundjoin%
\pgfsetlinewidth{0.308837pt}%
\definecolor{currentstroke}{rgb}{0.268510,0.009605,0.335427}%
\pgfsetstrokecolor{currentstroke}%
\pgfsetdash{}{0pt}%
\pgfpathmoveto{\pgfqpoint{8.427701in}{1.852939in}}%
\pgfpathlineto{\pgfqpoint{8.427688in}{1.852939in}}%
\pgfusepath{stroke}%
\end{pgfscope}%
\begin{pgfscope}%
\pgfpathrectangle{\pgfqpoint{6.720588in}{1.750000in}}{\pgfqpoint{2.279412in}{2.004545in}}%
\pgfusepath{clip}%
\pgfsetbuttcap%
\pgfsetroundjoin%
\pgfsetlinewidth{0.308833pt}%
\definecolor{currentstroke}{rgb}{0.268510,0.009605,0.335427}%
\pgfsetstrokecolor{currentstroke}%
\pgfsetdash{}{0pt}%
\pgfpathmoveto{\pgfqpoint{8.427688in}{1.852939in}}%
\pgfpathlineto{\pgfqpoint{8.427803in}{1.852932in}}%
\pgfusepath{stroke}%
\end{pgfscope}%
\begin{pgfscope}%
\pgfpathrectangle{\pgfqpoint{6.720588in}{1.750000in}}{\pgfqpoint{2.279412in}{2.004545in}}%
\pgfusepath{clip}%
\pgfsetbuttcap%
\pgfsetroundjoin%
\pgfsetlinewidth{0.308867pt}%
\definecolor{currentstroke}{rgb}{0.268510,0.009605,0.335427}%
\pgfsetstrokecolor{currentstroke}%
\pgfsetdash{}{0pt}%
\pgfpathmoveto{\pgfqpoint{8.427803in}{1.852932in}}%
\pgfpathlineto{\pgfqpoint{8.427930in}{1.852926in}}%
\pgfusepath{stroke}%
\end{pgfscope}%
\begin{pgfscope}%
\pgfpathrectangle{\pgfqpoint{6.720588in}{1.750000in}}{\pgfqpoint{2.279412in}{2.004545in}}%
\pgfusepath{clip}%
\pgfsetbuttcap%
\pgfsetroundjoin%
\pgfsetlinewidth{0.308906pt}%
\definecolor{currentstroke}{rgb}{0.268510,0.009605,0.335427}%
\pgfsetstrokecolor{currentstroke}%
\pgfsetdash{}{0pt}%
\pgfpathmoveto{\pgfqpoint{8.427930in}{1.852926in}}%
\pgfpathlineto{\pgfqpoint{8.427945in}{1.852926in}}%
\pgfusepath{stroke}%
\end{pgfscope}%
\begin{pgfscope}%
\pgfpathrectangle{\pgfqpoint{6.720588in}{1.750000in}}{\pgfqpoint{2.279412in}{2.004545in}}%
\pgfusepath{clip}%
\pgfsetbuttcap%
\pgfsetroundjoin%
\pgfsetlinewidth{0.308911pt}%
\definecolor{currentstroke}{rgb}{0.268510,0.009605,0.335427}%
\pgfsetstrokecolor{currentstroke}%
\pgfsetdash{}{0pt}%
\pgfpathmoveto{\pgfqpoint{8.427945in}{1.852926in}}%
\pgfpathlineto{\pgfqpoint{8.427822in}{1.852932in}}%
\pgfusepath{stroke}%
\end{pgfscope}%
\begin{pgfscope}%
\pgfpathrectangle{\pgfqpoint{6.720588in}{1.750000in}}{\pgfqpoint{2.279412in}{2.004545in}}%
\pgfusepath{clip}%
\pgfsetbuttcap%
\pgfsetroundjoin%
\pgfsetlinewidth{0.308873pt}%
\definecolor{currentstroke}{rgb}{0.268510,0.009605,0.335427}%
\pgfsetstrokecolor{currentstroke}%
\pgfsetdash{}{0pt}%
\pgfpathmoveto{\pgfqpoint{8.427822in}{1.852932in}}%
\pgfpathlineto{\pgfqpoint{8.427667in}{1.852940in}}%
\pgfusepath{stroke}%
\end{pgfscope}%
\begin{pgfscope}%
\pgfpathrectangle{\pgfqpoint{6.720588in}{1.750000in}}{\pgfqpoint{2.279412in}{2.004545in}}%
\pgfusepath{clip}%
\pgfsetbuttcap%
\pgfsetroundjoin%
\pgfsetlinewidth{0.308826pt}%
\definecolor{currentstroke}{rgb}{0.268510,0.009605,0.335427}%
\pgfsetstrokecolor{currentstroke}%
\pgfsetdash{}{0pt}%
\pgfpathmoveto{\pgfqpoint{8.427667in}{1.852940in}}%
\pgfpathlineto{\pgfqpoint{8.427652in}{1.852939in}}%
\pgfusepath{stroke}%
\end{pgfscope}%
\begin{pgfscope}%
\pgfpathrectangle{\pgfqpoint{6.720588in}{1.750000in}}{\pgfqpoint{2.279412in}{2.004545in}}%
\pgfusepath{clip}%
\pgfsetbuttcap%
\pgfsetroundjoin%
\pgfsetlinewidth{0.308821pt}%
\definecolor{currentstroke}{rgb}{0.268510,0.009605,0.335427}%
\pgfsetstrokecolor{currentstroke}%
\pgfsetdash{}{0pt}%
\pgfpathmoveto{\pgfqpoint{8.427652in}{1.852939in}}%
\pgfpathlineto{\pgfqpoint{8.427802in}{1.852931in}}%
\pgfusepath{stroke}%
\end{pgfscope}%
\begin{pgfscope}%
\pgfpathrectangle{\pgfqpoint{6.720588in}{1.750000in}}{\pgfqpoint{2.279412in}{2.004545in}}%
\pgfusepath{clip}%
\pgfsetbuttcap%
\pgfsetroundjoin%
\pgfsetlinewidth{0.308867pt}%
\definecolor{currentstroke}{rgb}{0.268510,0.009605,0.335427}%
\pgfsetstrokecolor{currentstroke}%
\pgfsetdash{}{0pt}%
\pgfpathmoveto{\pgfqpoint{8.427802in}{1.852931in}}%
\pgfpathlineto{\pgfqpoint{8.427963in}{1.852924in}}%
\pgfusepath{stroke}%
\end{pgfscope}%
\begin{pgfscope}%
\pgfpathrectangle{\pgfqpoint{6.720588in}{1.750000in}}{\pgfqpoint{2.279412in}{2.004545in}}%
\pgfusepath{clip}%
\pgfsetbuttcap%
\pgfsetroundjoin%
\pgfsetlinewidth{0.308916pt}%
\definecolor{currentstroke}{rgb}{0.268510,0.009605,0.335427}%
\pgfsetstrokecolor{currentstroke}%
\pgfsetdash{}{0pt}%
\pgfpathmoveto{\pgfqpoint{8.427963in}{1.852924in}}%
\pgfpathlineto{\pgfqpoint{8.427982in}{1.852923in}}%
\pgfusepath{stroke}%
\end{pgfscope}%
\begin{pgfscope}%
\pgfpathrectangle{\pgfqpoint{6.720588in}{1.750000in}}{\pgfqpoint{2.279412in}{2.004545in}}%
\pgfusepath{clip}%
\pgfsetbuttcap%
\pgfsetroundjoin%
\pgfsetlinewidth{0.308922pt}%
\definecolor{currentstroke}{rgb}{0.268510,0.009605,0.335427}%
\pgfsetstrokecolor{currentstroke}%
\pgfsetdash{}{0pt}%
\pgfpathmoveto{\pgfqpoint{8.427982in}{1.852923in}}%
\pgfpathlineto{\pgfqpoint{8.427823in}{1.852932in}}%
\pgfusepath{stroke}%
\end{pgfscope}%
\begin{pgfscope}%
\pgfpathrectangle{\pgfqpoint{6.720588in}{1.750000in}}{\pgfqpoint{2.279412in}{2.004545in}}%
\pgfusepath{clip}%
\pgfsetbuttcap%
\pgfsetroundjoin%
\pgfsetlinewidth{0.308874pt}%
\definecolor{currentstroke}{rgb}{0.268510,0.009605,0.335427}%
\pgfsetstrokecolor{currentstroke}%
\pgfsetdash{}{0pt}%
\pgfpathmoveto{\pgfqpoint{8.427823in}{1.852932in}}%
\pgfpathlineto{\pgfqpoint{8.427622in}{1.852941in}}%
\pgfusepath{stroke}%
\end{pgfscope}%
\begin{pgfscope}%
\pgfpathrectangle{\pgfqpoint{6.720588in}{1.750000in}}{\pgfqpoint{2.279412in}{2.004545in}}%
\pgfusepath{clip}%
\pgfsetbuttcap%
\pgfsetroundjoin%
\pgfsetlinewidth{0.308812pt}%
\definecolor{currentstroke}{rgb}{0.268510,0.009605,0.335427}%
\pgfsetstrokecolor{currentstroke}%
\pgfsetdash{}{0pt}%
\pgfpathmoveto{\pgfqpoint{8.427622in}{1.852941in}}%
\pgfpathlineto{\pgfqpoint{8.427604in}{1.852941in}}%
\pgfusepath{stroke}%
\end{pgfscope}%
\begin{pgfscope}%
\pgfpathrectangle{\pgfqpoint{6.720588in}{1.750000in}}{\pgfqpoint{2.279412in}{2.004545in}}%
\pgfusepath{clip}%
\pgfsetbuttcap%
\pgfsetroundjoin%
\pgfsetlinewidth{0.308807pt}%
\definecolor{currentstroke}{rgb}{0.268510,0.009605,0.335427}%
\pgfsetstrokecolor{currentstroke}%
\pgfsetdash{}{0pt}%
\pgfpathmoveto{\pgfqpoint{8.427604in}{1.852941in}}%
\pgfpathlineto{\pgfqpoint{8.427800in}{1.852931in}}%
\pgfusepath{stroke}%
\end{pgfscope}%
\begin{pgfscope}%
\pgfpathrectangle{\pgfqpoint{6.720588in}{1.750000in}}{\pgfqpoint{2.279412in}{2.004545in}}%
\pgfusepath{clip}%
\pgfsetbuttcap%
\pgfsetroundjoin%
\pgfsetlinewidth{0.308866pt}%
\definecolor{currentstroke}{rgb}{0.268510,0.009605,0.335427}%
\pgfsetstrokecolor{currentstroke}%
\pgfsetdash{}{0pt}%
\pgfpathmoveto{\pgfqpoint{8.427800in}{1.852931in}}%
\pgfpathlineto{\pgfqpoint{8.428005in}{1.852921in}}%
\pgfusepath{stroke}%
\end{pgfscope}%
\begin{pgfscope}%
\pgfpathrectangle{\pgfqpoint{6.720588in}{1.750000in}}{\pgfqpoint{2.279412in}{2.004545in}}%
\pgfusepath{clip}%
\pgfsetbuttcap%
\pgfsetroundjoin%
\pgfsetlinewidth{0.308929pt}%
\definecolor{currentstroke}{rgb}{0.268510,0.009605,0.335427}%
\pgfsetstrokecolor{currentstroke}%
\pgfsetdash{}{0pt}%
\pgfpathmoveto{\pgfqpoint{8.428005in}{1.852921in}}%
\pgfpathlineto{\pgfqpoint{8.428027in}{1.852921in}}%
\pgfusepath{stroke}%
\end{pgfscope}%
\begin{pgfscope}%
\pgfpathrectangle{\pgfqpoint{6.720588in}{1.750000in}}{\pgfqpoint{2.279412in}{2.004545in}}%
\pgfusepath{clip}%
\pgfsetbuttcap%
\pgfsetroundjoin%
\pgfsetlinewidth{0.308936pt}%
\definecolor{currentstroke}{rgb}{0.268510,0.009605,0.335427}%
\pgfsetstrokecolor{currentstroke}%
\pgfsetdash{}{0pt}%
\pgfpathmoveto{\pgfqpoint{8.428027in}{1.852921in}}%
\pgfpathlineto{\pgfqpoint{8.427824in}{1.852932in}}%
\pgfusepath{stroke}%
\end{pgfscope}%
\begin{pgfscope}%
\pgfpathrectangle{\pgfqpoint{6.720588in}{1.750000in}}{\pgfqpoint{2.279412in}{2.004545in}}%
\pgfusepath{clip}%
\pgfsetbuttcap%
\pgfsetroundjoin%
\pgfsetlinewidth{0.308874pt}%
\definecolor{currentstroke}{rgb}{0.268510,0.009605,0.335427}%
\pgfsetstrokecolor{currentstroke}%
\pgfsetdash{}{0pt}%
\pgfpathmoveto{\pgfqpoint{8.427824in}{1.852932in}}%
\pgfpathlineto{\pgfqpoint{8.427562in}{1.852944in}}%
\pgfusepath{stroke}%
\end{pgfscope}%
\begin{pgfscope}%
\pgfpathrectangle{\pgfqpoint{6.720588in}{1.750000in}}{\pgfqpoint{2.279412in}{2.004545in}}%
\pgfusepath{clip}%
\pgfsetbuttcap%
\pgfsetroundjoin%
\pgfsetlinewidth{0.308794pt}%
\definecolor{currentstroke}{rgb}{0.268510,0.009605,0.335427}%
\pgfsetstrokecolor{currentstroke}%
\pgfsetdash{}{0pt}%
\pgfpathmoveto{\pgfqpoint{8.427562in}{1.852944in}}%
\pgfpathlineto{\pgfqpoint{8.427541in}{1.852944in}}%
\pgfusepath{stroke}%
\end{pgfscope}%
\begin{pgfscope}%
\pgfpathrectangle{\pgfqpoint{6.720588in}{1.750000in}}{\pgfqpoint{2.279412in}{2.004545in}}%
\pgfusepath{clip}%
\pgfsetbuttcap%
\pgfsetroundjoin%
\pgfsetlinewidth{0.308787pt}%
\definecolor{currentstroke}{rgb}{0.268510,0.009605,0.335427}%
\pgfsetstrokecolor{currentstroke}%
\pgfsetdash{}{0pt}%
\pgfpathmoveto{\pgfqpoint{8.427541in}{1.852944in}}%
\pgfpathlineto{\pgfqpoint{8.427797in}{1.852930in}}%
\pgfusepath{stroke}%
\end{pgfscope}%
\begin{pgfscope}%
\pgfpathrectangle{\pgfqpoint{6.720588in}{1.750000in}}{\pgfqpoint{2.279412in}{2.004545in}}%
\pgfusepath{clip}%
\pgfsetbuttcap%
\pgfsetroundjoin%
\pgfsetlinewidth{0.308866pt}%
\definecolor{currentstroke}{rgb}{0.268510,0.009605,0.335427}%
\pgfsetstrokecolor{currentstroke}%
\pgfsetdash{}{0pt}%
\pgfpathmoveto{\pgfqpoint{8.427797in}{1.852930in}}%
\pgfpathlineto{\pgfqpoint{8.428056in}{1.852918in}}%
\pgfusepath{stroke}%
\end{pgfscope}%
\begin{pgfscope}%
\pgfpathrectangle{\pgfqpoint{6.720588in}{1.750000in}}{\pgfqpoint{2.279412in}{2.004545in}}%
\pgfusepath{clip}%
\pgfsetbuttcap%
\pgfsetroundjoin%
\pgfsetlinewidth{0.308944pt}%
\definecolor{currentstroke}{rgb}{0.268510,0.009605,0.335427}%
\pgfsetstrokecolor{currentstroke}%
\pgfsetdash{}{0pt}%
\pgfpathmoveto{\pgfqpoint{8.428056in}{1.852918in}}%
\pgfpathlineto{\pgfqpoint{8.428083in}{1.852918in}}%
\pgfusepath{stroke}%
\end{pgfscope}%
\begin{pgfscope}%
\pgfpathrectangle{\pgfqpoint{6.720588in}{1.750000in}}{\pgfqpoint{2.279412in}{2.004545in}}%
\pgfusepath{clip}%
\pgfsetbuttcap%
\pgfsetroundjoin%
\pgfsetlinewidth{1.518339pt}%
\definecolor{currentstroke}{rgb}{0.252899,0.742211,0.448284}%
\pgfsetstrokecolor{currentstroke}%
\pgfsetdash{}{0pt}%
\pgfpathmoveto{\pgfqpoint{6.988332in}{2.752273in}}%
\pgfpathlineto{\pgfqpoint{7.038436in}{2.750428in}}%
\pgfusepath{stroke}%
\end{pgfscope}%
\begin{pgfscope}%
\pgfpathrectangle{\pgfqpoint{6.720588in}{1.750000in}}{\pgfqpoint{2.279412in}{2.004545in}}%
\pgfusepath{clip}%
\pgfsetbuttcap%
\pgfsetroundjoin%
\pgfsetlinewidth{1.479753pt}%
\definecolor{currentstroke}{rgb}{0.214000,0.722114,0.469588}%
\pgfsetstrokecolor{currentstroke}%
\pgfsetdash{}{0pt}%
\pgfpathmoveto{\pgfqpoint{7.038436in}{2.750428in}}%
\pgfpathlineto{\pgfqpoint{7.088558in}{2.749019in}}%
\pgfusepath{stroke}%
\end{pgfscope}%
\begin{pgfscope}%
\pgfpathrectangle{\pgfqpoint{6.720588in}{1.750000in}}{\pgfqpoint{2.279412in}{2.004545in}}%
\pgfusepath{clip}%
\pgfsetbuttcap%
\pgfsetroundjoin%
\pgfsetlinewidth{1.347037pt}%
\definecolor{currentstroke}{rgb}{0.132268,0.655014,0.519661}%
\pgfsetstrokecolor{currentstroke}%
\pgfsetdash{}{0pt}%
\pgfpathmoveto{\pgfqpoint{7.088558in}{2.749019in}}%
\pgfpathlineto{\pgfqpoint{7.138683in}{2.747964in}}%
\pgfusepath{stroke}%
\end{pgfscope}%
\begin{pgfscope}%
\pgfpathrectangle{\pgfqpoint{6.720588in}{1.750000in}}{\pgfqpoint{2.279412in}{2.004545in}}%
\pgfusepath{clip}%
\pgfsetbuttcap%
\pgfsetroundjoin%
\pgfsetlinewidth{1.259718pt}%
\definecolor{currentstroke}{rgb}{0.119512,0.607464,0.540218}%
\pgfsetstrokecolor{currentstroke}%
\pgfsetdash{}{0pt}%
\pgfpathmoveto{\pgfqpoint{7.138683in}{2.747964in}}%
\pgfpathlineto{\pgfqpoint{7.188732in}{2.746085in}}%
\pgfusepath{stroke}%
\end{pgfscope}%
\begin{pgfscope}%
\pgfpathrectangle{\pgfqpoint{6.720588in}{1.750000in}}{\pgfqpoint{2.279412in}{2.004545in}}%
\pgfusepath{clip}%
\pgfsetbuttcap%
\pgfsetroundjoin%
\pgfsetlinewidth{1.173886pt}%
\definecolor{currentstroke}{rgb}{0.128729,0.563265,0.551229}%
\pgfsetstrokecolor{currentstroke}%
\pgfsetdash{}{0pt}%
\pgfpathmoveto{\pgfqpoint{7.188732in}{2.746085in}}%
\pgfpathlineto{\pgfqpoint{7.238779in}{2.743939in}}%
\pgfusepath{stroke}%
\end{pgfscope}%
\begin{pgfscope}%
\pgfpathrectangle{\pgfqpoint{6.720588in}{1.750000in}}{\pgfqpoint{2.279412in}{2.004545in}}%
\pgfusepath{clip}%
\pgfsetbuttcap%
\pgfsetroundjoin%
\pgfsetlinewidth{0.892541pt}%
\definecolor{currentstroke}{rgb}{0.188923,0.410910,0.556326}%
\pgfsetstrokecolor{currentstroke}%
\pgfsetdash{}{0pt}%
\pgfpathmoveto{\pgfqpoint{7.238779in}{2.743939in}}%
\pgfpathlineto{\pgfqpoint{7.288886in}{2.742650in}}%
\pgfusepath{stroke}%
\end{pgfscope}%
\begin{pgfscope}%
\pgfpathrectangle{\pgfqpoint{6.720588in}{1.750000in}}{\pgfqpoint{2.279412in}{2.004545in}}%
\pgfusepath{clip}%
\pgfsetbuttcap%
\pgfsetroundjoin%
\pgfsetlinewidth{0.770040pt}%
\definecolor{currentstroke}{rgb}{0.221989,0.339161,0.548752}%
\pgfsetstrokecolor{currentstroke}%
\pgfsetdash{}{0pt}%
\pgfpathmoveto{\pgfqpoint{7.288886in}{2.742650in}}%
\pgfpathlineto{\pgfqpoint{7.338863in}{2.740677in}}%
\pgfusepath{stroke}%
\end{pgfscope}%
\begin{pgfscope}%
\pgfpathrectangle{\pgfqpoint{6.720588in}{1.750000in}}{\pgfqpoint{2.279412in}{2.004545in}}%
\pgfusepath{clip}%
\pgfsetbuttcap%
\pgfsetroundjoin%
\pgfsetlinewidth{0.639142pt}%
\definecolor{currentstroke}{rgb}{0.257322,0.256130,0.526563}%
\pgfsetstrokecolor{currentstroke}%
\pgfsetdash{}{0pt}%
\pgfpathmoveto{\pgfqpoint{7.338863in}{2.740677in}}%
\pgfpathlineto{\pgfqpoint{7.388685in}{2.738515in}}%
\pgfusepath{stroke}%
\end{pgfscope}%
\begin{pgfscope}%
\pgfpathrectangle{\pgfqpoint{6.720588in}{1.750000in}}{\pgfqpoint{2.279412in}{2.004545in}}%
\pgfusepath{clip}%
\pgfsetbuttcap%
\pgfsetroundjoin%
\pgfsetlinewidth{0.555421pt}%
\definecolor{currentstroke}{rgb}{0.274128,0.199721,0.498911}%
\pgfsetstrokecolor{currentstroke}%
\pgfsetdash{}{0pt}%
\pgfpathmoveto{\pgfqpoint{7.388685in}{2.738515in}}%
\pgfpathlineto{\pgfqpoint{7.388685in}{2.738515in}}%
\pgfusepath{stroke}%
\end{pgfscope}%
\begin{pgfscope}%
\pgfpathrectangle{\pgfqpoint{6.720588in}{1.750000in}}{\pgfqpoint{2.279412in}{2.004545in}}%
\pgfusepath{clip}%
\pgfsetbuttcap%
\pgfsetroundjoin%
\pgfsetlinewidth{0.555421pt}%
\definecolor{currentstroke}{rgb}{0.274128,0.199721,0.498911}%
\pgfsetstrokecolor{currentstroke}%
\pgfsetdash{}{0pt}%
\pgfpathmoveto{\pgfqpoint{7.388685in}{2.738515in}}%
\pgfpathlineto{\pgfqpoint{7.413378in}{2.735121in}}%
\pgfusepath{stroke}%
\end{pgfscope}%
\begin{pgfscope}%
\pgfpathrectangle{\pgfqpoint{6.720588in}{1.750000in}}{\pgfqpoint{2.279412in}{2.004545in}}%
\pgfusepath{clip}%
\pgfsetbuttcap%
\pgfsetroundjoin%
\pgfsetlinewidth{0.384979pt}%
\definecolor{currentstroke}{rgb}{0.280267,0.073417,0.397163}%
\pgfsetstrokecolor{currentstroke}%
\pgfsetdash{}{0pt}%
\pgfpathmoveto{\pgfqpoint{7.413378in}{2.735121in}}%
\pgfpathlineto{\pgfqpoint{7.413378in}{2.735121in}}%
\pgfusepath{stroke}%
\end{pgfscope}%
\begin{pgfscope}%
\pgfpathrectangle{\pgfqpoint{6.720588in}{1.750000in}}{\pgfqpoint{2.279412in}{2.004545in}}%
\pgfusepath{clip}%
\pgfsetbuttcap%
\pgfsetroundjoin%
\pgfsetlinewidth{0.384979pt}%
\definecolor{currentstroke}{rgb}{0.280267,0.073417,0.397163}%
\pgfsetstrokecolor{currentstroke}%
\pgfsetdash{}{0pt}%
\pgfpathmoveto{\pgfqpoint{7.413378in}{2.735121in}}%
\pgfpathlineto{\pgfqpoint{7.428143in}{2.732588in}}%
\pgfusepath{stroke}%
\end{pgfscope}%
\begin{pgfscope}%
\pgfpathrectangle{\pgfqpoint{6.720588in}{1.750000in}}{\pgfqpoint{2.279412in}{2.004545in}}%
\pgfusepath{clip}%
\pgfsetbuttcap%
\pgfsetroundjoin%
\pgfsetlinewidth{0.357020pt}%
\definecolor{currentstroke}{rgb}{0.277018,0.050344,0.375715}%
\pgfsetstrokecolor{currentstroke}%
\pgfsetdash{}{0pt}%
\pgfpathmoveto{\pgfqpoint{7.428143in}{2.732588in}}%
\pgfpathlineto{\pgfqpoint{7.428143in}{2.732588in}}%
\pgfusepath{stroke}%
\end{pgfscope}%
\begin{pgfscope}%
\pgfpathrectangle{\pgfqpoint{6.720588in}{1.750000in}}{\pgfqpoint{2.279412in}{2.004545in}}%
\pgfusepath{clip}%
\pgfsetbuttcap%
\pgfsetroundjoin%
\pgfsetlinewidth{0.357020pt}%
\definecolor{currentstroke}{rgb}{0.277018,0.050344,0.375715}%
\pgfsetstrokecolor{currentstroke}%
\pgfsetdash{}{0pt}%
\pgfpathmoveto{\pgfqpoint{7.428143in}{2.732588in}}%
\pgfpathlineto{\pgfqpoint{7.428143in}{2.732588in}}%
\pgfusepath{stroke}%
\end{pgfscope}%
\begin{pgfscope}%
\pgfpathrectangle{\pgfqpoint{6.720588in}{1.750000in}}{\pgfqpoint{2.279412in}{2.004545in}}%
\pgfusepath{clip}%
\pgfsetbuttcap%
\pgfsetroundjoin%
\pgfsetlinewidth{0.357020pt}%
\definecolor{currentstroke}{rgb}{0.277018,0.050344,0.375715}%
\pgfsetstrokecolor{currentstroke}%
\pgfsetdash{}{0pt}%
\pgfpathmoveto{\pgfqpoint{7.428143in}{2.732588in}}%
\pgfpathlineto{\pgfqpoint{7.433929in}{2.731115in}}%
\pgfusepath{stroke}%
\end{pgfscope}%
\begin{pgfscope}%
\pgfpathrectangle{\pgfqpoint{6.720588in}{1.750000in}}{\pgfqpoint{2.279412in}{2.004545in}}%
\pgfusepath{clip}%
\pgfsetbuttcap%
\pgfsetroundjoin%
\pgfsetlinewidth{0.371726pt}%
\definecolor{currentstroke}{rgb}{0.278791,0.062145,0.386592}%
\pgfsetstrokecolor{currentstroke}%
\pgfsetdash{}{0pt}%
\pgfpathmoveto{\pgfqpoint{7.433929in}{2.731115in}}%
\pgfpathlineto{\pgfqpoint{7.436915in}{2.730025in}}%
\pgfusepath{stroke}%
\end{pgfscope}%
\begin{pgfscope}%
\pgfpathrectangle{\pgfqpoint{6.720588in}{1.750000in}}{\pgfqpoint{2.279412in}{2.004545in}}%
\pgfusepath{clip}%
\pgfsetbuttcap%
\pgfsetroundjoin%
\pgfsetlinewidth{0.391686pt}%
\definecolor{currentstroke}{rgb}{0.280894,0.078907,0.402329}%
\pgfsetstrokecolor{currentstroke}%
\pgfsetdash{}{0pt}%
\pgfpathmoveto{\pgfqpoint{7.436915in}{2.730025in}}%
\pgfpathlineto{\pgfqpoint{7.438476in}{2.729119in}}%
\pgfusepath{stroke}%
\end{pgfscope}%
\begin{pgfscope}%
\pgfpathrectangle{\pgfqpoint{6.720588in}{1.750000in}}{\pgfqpoint{2.279412in}{2.004545in}}%
\pgfusepath{clip}%
\pgfsetbuttcap%
\pgfsetroundjoin%
\pgfsetlinewidth{0.401071pt}%
\definecolor{currentstroke}{rgb}{0.281446,0.084320,0.407414}%
\pgfsetstrokecolor{currentstroke}%
\pgfsetdash{}{0pt}%
\pgfpathmoveto{\pgfqpoint{7.438476in}{2.729119in}}%
\pgfpathlineto{\pgfqpoint{7.439041in}{2.728181in}}%
\pgfusepath{stroke}%
\end{pgfscope}%
\begin{pgfscope}%
\pgfpathrectangle{\pgfqpoint{6.720588in}{1.750000in}}{\pgfqpoint{2.279412in}{2.004545in}}%
\pgfusepath{clip}%
\pgfsetbuttcap%
\pgfsetroundjoin%
\pgfsetlinewidth{0.404063pt}%
\definecolor{currentstroke}{rgb}{0.281924,0.089666,0.412415}%
\pgfsetstrokecolor{currentstroke}%
\pgfsetdash{}{0pt}%
\pgfpathmoveto{\pgfqpoint{7.439041in}{2.728181in}}%
\pgfpathlineto{\pgfqpoint{7.438567in}{2.727432in}}%
\pgfusepath{stroke}%
\end{pgfscope}%
\begin{pgfscope}%
\pgfpathrectangle{\pgfqpoint{6.720588in}{1.750000in}}{\pgfqpoint{2.279412in}{2.004545in}}%
\pgfusepath{clip}%
\pgfsetbuttcap%
\pgfsetroundjoin%
\pgfsetlinewidth{0.401565pt}%
\definecolor{currentstroke}{rgb}{0.281446,0.084320,0.407414}%
\pgfsetstrokecolor{currentstroke}%
\pgfsetdash{}{0pt}%
\pgfpathmoveto{\pgfqpoint{7.438567in}{2.727432in}}%
\pgfpathlineto{\pgfqpoint{7.438567in}{2.727432in}}%
\pgfusepath{stroke}%
\end{pgfscope}%
\begin{pgfscope}%
\pgfpathrectangle{\pgfqpoint{6.720588in}{1.750000in}}{\pgfqpoint{2.279412in}{2.004545in}}%
\pgfusepath{clip}%
\pgfsetbuttcap%
\pgfsetroundjoin%
\pgfsetlinewidth{0.401565pt}%
\definecolor{currentstroke}{rgb}{0.281446,0.084320,0.407414}%
\pgfsetstrokecolor{currentstroke}%
\pgfsetdash{}{0pt}%
\pgfpathmoveto{\pgfqpoint{7.438567in}{2.727432in}}%
\pgfpathlineto{\pgfqpoint{7.437976in}{2.727028in}}%
\pgfusepath{stroke}%
\end{pgfscope}%
\begin{pgfscope}%
\pgfpathrectangle{\pgfqpoint{6.720588in}{1.750000in}}{\pgfqpoint{2.279412in}{2.004545in}}%
\pgfusepath{clip}%
\pgfsetbuttcap%
\pgfsetroundjoin%
\pgfsetlinewidth{0.398731pt}%
\definecolor{currentstroke}{rgb}{0.281446,0.084320,0.407414}%
\pgfsetstrokecolor{currentstroke}%
\pgfsetdash{}{0pt}%
\pgfpathmoveto{\pgfqpoint{7.437976in}{2.727028in}}%
\pgfpathlineto{\pgfqpoint{7.437976in}{2.727028in}}%
\pgfusepath{stroke}%
\end{pgfscope}%
\begin{pgfscope}%
\pgfpathrectangle{\pgfqpoint{6.720588in}{1.750000in}}{\pgfqpoint{2.279412in}{2.004545in}}%
\pgfusepath{clip}%
\pgfsetbuttcap%
\pgfsetroundjoin%
\pgfsetlinewidth{0.398731pt}%
\definecolor{currentstroke}{rgb}{0.281446,0.084320,0.407414}%
\pgfsetstrokecolor{currentstroke}%
\pgfsetdash{}{0pt}%
\pgfpathmoveto{\pgfqpoint{7.437976in}{2.727028in}}%
\pgfpathlineto{\pgfqpoint{7.438100in}{2.726845in}}%
\pgfusepath{stroke}%
\end{pgfscope}%
\begin{pgfscope}%
\pgfpathrectangle{\pgfqpoint{6.720588in}{1.750000in}}{\pgfqpoint{2.279412in}{2.004545in}}%
\pgfusepath{clip}%
\pgfsetbuttcap%
\pgfsetroundjoin%
\pgfsetlinewidth{0.399359pt}%
\definecolor{currentstroke}{rgb}{0.281446,0.084320,0.407414}%
\pgfsetstrokecolor{currentstroke}%
\pgfsetdash{}{0pt}%
\pgfpathmoveto{\pgfqpoint{7.438100in}{2.726845in}}%
\pgfpathlineto{\pgfqpoint{7.438246in}{2.726740in}}%
\pgfusepath{stroke}%
\end{pgfscope}%
\begin{pgfscope}%
\pgfpathrectangle{\pgfqpoint{6.720588in}{1.750000in}}{\pgfqpoint{2.279412in}{2.004545in}}%
\pgfusepath{clip}%
\pgfsetbuttcap%
\pgfsetroundjoin%
\pgfsetlinewidth{0.400059pt}%
\definecolor{currentstroke}{rgb}{0.281446,0.084320,0.407414}%
\pgfsetstrokecolor{currentstroke}%
\pgfsetdash{}{0pt}%
\pgfpathmoveto{\pgfqpoint{7.438246in}{2.726740in}}%
\pgfpathlineto{\pgfqpoint{7.438301in}{2.726677in}}%
\pgfusepath{stroke}%
\end{pgfscope}%
\begin{pgfscope}%
\pgfpathrectangle{\pgfqpoint{6.720588in}{1.750000in}}{\pgfqpoint{2.279412in}{2.004545in}}%
\pgfusepath{clip}%
\pgfsetbuttcap%
\pgfsetroundjoin%
\pgfsetlinewidth{0.400318pt}%
\definecolor{currentstroke}{rgb}{0.281446,0.084320,0.407414}%
\pgfsetstrokecolor{currentstroke}%
\pgfsetdash{}{0pt}%
\pgfpathmoveto{\pgfqpoint{7.438301in}{2.726677in}}%
\pgfpathlineto{\pgfqpoint{7.438224in}{2.726634in}}%
\pgfusepath{stroke}%
\end{pgfscope}%
\begin{pgfscope}%
\pgfpathrectangle{\pgfqpoint{6.720588in}{1.750000in}}{\pgfqpoint{2.279412in}{2.004545in}}%
\pgfusepath{clip}%
\pgfsetbuttcap%
\pgfsetroundjoin%
\pgfsetlinewidth{0.399967pt}%
\definecolor{currentstroke}{rgb}{0.281446,0.084320,0.407414}%
\pgfsetstrokecolor{currentstroke}%
\pgfsetdash{}{0pt}%
\pgfpathmoveto{\pgfqpoint{7.438224in}{2.726634in}}%
\pgfpathlineto{\pgfqpoint{7.438090in}{2.726605in}}%
\pgfusepath{stroke}%
\end{pgfscope}%
\begin{pgfscope}%
\pgfpathrectangle{\pgfqpoint{6.720588in}{1.750000in}}{\pgfqpoint{2.279412in}{2.004545in}}%
\pgfusepath{clip}%
\pgfsetbuttcap%
\pgfsetroundjoin%
\pgfsetlinewidth{0.399354pt}%
\definecolor{currentstroke}{rgb}{0.281446,0.084320,0.407414}%
\pgfsetstrokecolor{currentstroke}%
\pgfsetdash{}{0pt}%
\pgfpathmoveto{\pgfqpoint{7.438090in}{2.726605in}}%
\pgfpathlineto{\pgfqpoint{7.438041in}{2.726590in}}%
\pgfusepath{stroke}%
\end{pgfscope}%
\begin{pgfscope}%
\pgfpathrectangle{\pgfqpoint{6.720588in}{1.750000in}}{\pgfqpoint{2.279412in}{2.004545in}}%
\pgfusepath{clip}%
\pgfsetbuttcap%
\pgfsetroundjoin%
\pgfsetlinewidth{0.399133pt}%
\definecolor{currentstroke}{rgb}{0.281446,0.084320,0.407414}%
\pgfsetstrokecolor{currentstroke}%
\pgfsetdash{}{0pt}%
\pgfpathmoveto{\pgfqpoint{7.438041in}{2.726590in}}%
\pgfpathlineto{\pgfqpoint{7.438137in}{2.726586in}}%
\pgfusepath{stroke}%
\end{pgfscope}%
\begin{pgfscope}%
\pgfpathrectangle{\pgfqpoint{6.720588in}{1.750000in}}{\pgfqpoint{2.279412in}{2.004545in}}%
\pgfusepath{clip}%
\pgfsetbuttcap%
\pgfsetroundjoin%
\pgfsetlinewidth{0.399574pt}%
\definecolor{currentstroke}{rgb}{0.281446,0.084320,0.407414}%
\pgfsetstrokecolor{currentstroke}%
\pgfsetdash{}{0pt}%
\pgfpathmoveto{\pgfqpoint{7.438137in}{2.726586in}}%
\pgfpathlineto{\pgfqpoint{7.438280in}{2.726588in}}%
\pgfusepath{stroke}%
\end{pgfscope}%
\begin{pgfscope}%
\pgfpathrectangle{\pgfqpoint{6.720588in}{1.750000in}}{\pgfqpoint{2.279412in}{2.004545in}}%
\pgfusepath{clip}%
\pgfsetbuttcap%
\pgfsetroundjoin%
\pgfsetlinewidth{0.400233pt}%
\definecolor{currentstroke}{rgb}{0.281446,0.084320,0.407414}%
\pgfsetstrokecolor{currentstroke}%
\pgfsetdash{}{0pt}%
\pgfpathmoveto{\pgfqpoint{7.438280in}{2.726588in}}%
\pgfpathlineto{\pgfqpoint{7.438335in}{2.726588in}}%
\pgfusepath{stroke}%
\end{pgfscope}%
\begin{pgfscope}%
\pgfpathrectangle{\pgfqpoint{6.720588in}{1.750000in}}{\pgfqpoint{2.279412in}{2.004545in}}%
\pgfusepath{clip}%
\pgfsetbuttcap%
\pgfsetroundjoin%
\pgfsetlinewidth{0.400482pt}%
\definecolor{currentstroke}{rgb}{0.281446,0.084320,0.407414}%
\pgfsetstrokecolor{currentstroke}%
\pgfsetdash{}{0pt}%
\pgfpathmoveto{\pgfqpoint{7.438335in}{2.726588in}}%
\pgfpathlineto{\pgfqpoint{7.438247in}{2.726583in}}%
\pgfusepath{stroke}%
\end{pgfscope}%
\begin{pgfscope}%
\pgfpathrectangle{\pgfqpoint{6.720588in}{1.750000in}}{\pgfqpoint{2.279412in}{2.004545in}}%
\pgfusepath{clip}%
\pgfsetbuttcap%
\pgfsetroundjoin%
\pgfsetlinewidth{0.400080pt}%
\definecolor{currentstroke}{rgb}{0.281446,0.084320,0.407414}%
\pgfsetstrokecolor{currentstroke}%
\pgfsetdash{}{0pt}%
\pgfpathmoveto{\pgfqpoint{7.438247in}{2.726583in}}%
\pgfpathlineto{\pgfqpoint{7.438091in}{2.726575in}}%
\pgfusepath{stroke}%
\end{pgfscope}%
\begin{pgfscope}%
\pgfpathrectangle{\pgfqpoint{6.720588in}{1.750000in}}{\pgfqpoint{2.279412in}{2.004545in}}%
\pgfusepath{clip}%
\pgfsetbuttcap%
\pgfsetroundjoin%
\pgfsetlinewidth{0.399364pt}%
\definecolor{currentstroke}{rgb}{0.281446,0.084320,0.407414}%
\pgfsetstrokecolor{currentstroke}%
\pgfsetdash{}{0pt}%
\pgfpathmoveto{\pgfqpoint{7.438091in}{2.726575in}}%
\pgfpathlineto{\pgfqpoint{7.438025in}{2.726571in}}%
\pgfusepath{stroke}%
\end{pgfscope}%
\begin{pgfscope}%
\pgfpathrectangle{\pgfqpoint{6.720588in}{1.750000in}}{\pgfqpoint{2.279412in}{2.004545in}}%
\pgfusepath{clip}%
\pgfsetbuttcap%
\pgfsetroundjoin%
\pgfsetlinewidth{0.399060pt}%
\definecolor{currentstroke}{rgb}{0.281446,0.084320,0.407414}%
\pgfsetstrokecolor{currentstroke}%
\pgfsetdash{}{0pt}%
\pgfpathmoveto{\pgfqpoint{7.438025in}{2.726571in}}%
\pgfpathlineto{\pgfqpoint{7.438124in}{2.726575in}}%
\pgfusepath{stroke}%
\end{pgfscope}%
\begin{pgfscope}%
\pgfpathrectangle{\pgfqpoint{6.720588in}{1.750000in}}{\pgfqpoint{2.279412in}{2.004545in}}%
\pgfusepath{clip}%
\pgfsetbuttcap%
\pgfsetroundjoin%
\pgfsetlinewidth{0.399517pt}%
\definecolor{currentstroke}{rgb}{0.281446,0.084320,0.407414}%
\pgfsetstrokecolor{currentstroke}%
\pgfsetdash{}{0pt}%
\pgfpathmoveto{\pgfqpoint{7.438124in}{2.726575in}}%
\pgfpathlineto{\pgfqpoint{7.438285in}{2.726582in}}%
\pgfusepath{stroke}%
\end{pgfscope}%
\begin{pgfscope}%
\pgfpathrectangle{\pgfqpoint{6.720588in}{1.750000in}}{\pgfqpoint{2.279412in}{2.004545in}}%
\pgfusepath{clip}%
\pgfsetbuttcap%
\pgfsetroundjoin%
\pgfsetlinewidth{0.400256pt}%
\definecolor{currentstroke}{rgb}{0.281446,0.084320,0.407414}%
\pgfsetstrokecolor{currentstroke}%
\pgfsetdash{}{0pt}%
\pgfpathmoveto{\pgfqpoint{7.438285in}{2.726582in}}%
\pgfpathlineto{\pgfqpoint{7.438353in}{2.726585in}}%
\pgfusepath{stroke}%
\end{pgfscope}%
\begin{pgfscope}%
\pgfpathrectangle{\pgfqpoint{6.720588in}{1.750000in}}{\pgfqpoint{2.279412in}{2.004545in}}%
\pgfusepath{clip}%
\pgfsetbuttcap%
\pgfsetroundjoin%
\pgfsetlinewidth{0.400567pt}%
\definecolor{currentstroke}{rgb}{0.281446,0.084320,0.407414}%
\pgfsetstrokecolor{currentstroke}%
\pgfsetdash{}{0pt}%
\pgfpathmoveto{\pgfqpoint{7.438353in}{2.726585in}}%
\pgfpathlineto{\pgfqpoint{7.438262in}{2.726581in}}%
\pgfusepath{stroke}%
\end{pgfscope}%
\begin{pgfscope}%
\pgfpathrectangle{\pgfqpoint{6.720588in}{1.750000in}}{\pgfqpoint{2.279412in}{2.004545in}}%
\pgfusepath{clip}%
\pgfsetbuttcap%
\pgfsetroundjoin%
\pgfsetlinewidth{0.400152pt}%
\definecolor{currentstroke}{rgb}{0.281446,0.084320,0.407414}%
\pgfsetstrokecolor{currentstroke}%
\pgfsetdash{}{0pt}%
\pgfpathmoveto{\pgfqpoint{7.438262in}{2.726581in}}%
\pgfpathlineto{\pgfqpoint{7.438087in}{2.726574in}}%
\pgfusepath{stroke}%
\end{pgfscope}%
\begin{pgfscope}%
\pgfpathrectangle{\pgfqpoint{6.720588in}{1.750000in}}{\pgfqpoint{2.279412in}{2.004545in}}%
\pgfusepath{clip}%
\pgfsetbuttcap%
\pgfsetroundjoin%
\pgfsetlinewidth{0.399345pt}%
\definecolor{currentstroke}{rgb}{0.281446,0.084320,0.407414}%
\pgfsetstrokecolor{currentstroke}%
\pgfsetdash{}{0pt}%
\pgfpathmoveto{\pgfqpoint{7.438087in}{2.726574in}}%
\pgfpathlineto{\pgfqpoint{7.438002in}{2.726570in}}%
\pgfusepath{stroke}%
\end{pgfscope}%
\begin{pgfscope}%
\pgfpathrectangle{\pgfqpoint{6.720588in}{1.750000in}}{\pgfqpoint{2.279412in}{2.004545in}}%
\pgfusepath{clip}%
\pgfsetbuttcap%
\pgfsetroundjoin%
\pgfsetlinewidth{0.398959pt}%
\definecolor{currentstroke}{rgb}{0.281446,0.084320,0.407414}%
\pgfsetstrokecolor{currentstroke}%
\pgfsetdash{}{0pt}%
\pgfpathmoveto{\pgfqpoint{7.438002in}{2.726570in}}%
\pgfpathlineto{\pgfqpoint{7.438106in}{2.726574in}}%
\pgfusepath{stroke}%
\end{pgfscope}%
\begin{pgfscope}%
\pgfpathrectangle{\pgfqpoint{6.720588in}{1.750000in}}{\pgfqpoint{2.279412in}{2.004545in}}%
\pgfusepath{clip}%
\pgfsetbuttcap%
\pgfsetroundjoin%
\pgfsetlinewidth{0.399436pt}%
\definecolor{currentstroke}{rgb}{0.281446,0.084320,0.407414}%
\pgfsetstrokecolor{currentstroke}%
\pgfsetdash{}{0pt}%
\pgfpathmoveto{\pgfqpoint{7.438106in}{2.726574in}}%
\pgfpathlineto{\pgfqpoint{7.438288in}{2.726581in}}%
\pgfusepath{stroke}%
\end{pgfscope}%
\begin{pgfscope}%
\pgfpathrectangle{\pgfqpoint{6.720588in}{1.750000in}}{\pgfqpoint{2.279412in}{2.004545in}}%
\pgfusepath{clip}%
\pgfsetbuttcap%
\pgfsetroundjoin%
\pgfsetlinewidth{0.400269pt}%
\definecolor{currentstroke}{rgb}{0.281446,0.084320,0.407414}%
\pgfsetstrokecolor{currentstroke}%
\pgfsetdash{}{0pt}%
\pgfpathmoveto{\pgfqpoint{7.438288in}{2.726581in}}%
\pgfpathlineto{\pgfqpoint{7.438372in}{2.726585in}}%
\pgfusepath{stroke}%
\end{pgfscope}%
\begin{pgfscope}%
\pgfpathrectangle{\pgfqpoint{6.720588in}{1.750000in}}{\pgfqpoint{2.279412in}{2.004545in}}%
\pgfusepath{clip}%
\pgfsetbuttcap%
\pgfsetroundjoin%
\pgfsetlinewidth{0.400655pt}%
\definecolor{currentstroke}{rgb}{0.281446,0.084320,0.407414}%
\pgfsetstrokecolor{currentstroke}%
\pgfsetdash{}{0pt}%
\pgfpathmoveto{\pgfqpoint{7.438372in}{2.726585in}}%
\pgfpathlineto{\pgfqpoint{7.438280in}{2.726582in}}%
\pgfusepath{stroke}%
\end{pgfscope}%
\begin{pgfscope}%
\pgfpathrectangle{\pgfqpoint{6.720588in}{1.750000in}}{\pgfqpoint{2.279412in}{2.004545in}}%
\pgfusepath{clip}%
\pgfsetbuttcap%
\pgfsetroundjoin%
\pgfsetlinewidth{0.400233pt}%
\definecolor{currentstroke}{rgb}{0.281446,0.084320,0.407414}%
\pgfsetstrokecolor{currentstroke}%
\pgfsetdash{}{0pt}%
\pgfpathmoveto{\pgfqpoint{7.438280in}{2.726582in}}%
\pgfpathlineto{\pgfqpoint{7.438083in}{2.726574in}}%
\pgfusepath{stroke}%
\end{pgfscope}%
\begin{pgfscope}%
\pgfpathrectangle{\pgfqpoint{6.720588in}{1.750000in}}{\pgfqpoint{2.279412in}{2.004545in}}%
\pgfusepath{clip}%
\pgfsetbuttcap%
\pgfsetroundjoin%
\pgfsetlinewidth{0.399330pt}%
\definecolor{currentstroke}{rgb}{0.281446,0.084320,0.407414}%
\pgfsetstrokecolor{currentstroke}%
\pgfsetdash{}{0pt}%
\pgfpathmoveto{\pgfqpoint{7.438083in}{2.726574in}}%
\pgfpathlineto{\pgfqpoint{7.437978in}{2.726569in}}%
\pgfusepath{stroke}%
\end{pgfscope}%
\begin{pgfscope}%
\pgfpathrectangle{\pgfqpoint{6.720588in}{1.750000in}}{\pgfqpoint{2.279412in}{2.004545in}}%
\pgfusepath{clip}%
\pgfsetbuttcap%
\pgfsetroundjoin%
\pgfsetlinewidth{0.398847pt}%
\definecolor{currentstroke}{rgb}{0.281446,0.084320,0.407414}%
\pgfsetstrokecolor{currentstroke}%
\pgfsetdash{}{0pt}%
\pgfpathmoveto{\pgfqpoint{7.437978in}{2.726569in}}%
\pgfpathlineto{\pgfqpoint{7.438086in}{2.726573in}}%
\pgfusepath{stroke}%
\end{pgfscope}%
\begin{pgfscope}%
\pgfpathrectangle{\pgfqpoint{6.720588in}{1.750000in}}{\pgfqpoint{2.279412in}{2.004545in}}%
\pgfusepath{clip}%
\pgfsetbuttcap%
\pgfsetroundjoin%
\pgfsetlinewidth{0.399340pt}%
\definecolor{currentstroke}{rgb}{0.281446,0.084320,0.407414}%
\pgfsetstrokecolor{currentstroke}%
\pgfsetdash{}{0pt}%
\pgfpathmoveto{\pgfqpoint{7.438086in}{2.726573in}}%
\pgfpathlineto{\pgfqpoint{7.438290in}{2.726581in}}%
\pgfusepath{stroke}%
\end{pgfscope}%
\begin{pgfscope}%
\pgfpathrectangle{\pgfqpoint{6.720588in}{1.750000in}}{\pgfqpoint{2.279412in}{2.004545in}}%
\pgfusepath{clip}%
\pgfsetbuttcap%
\pgfsetroundjoin%
\pgfsetlinewidth{0.400276pt}%
\definecolor{currentstroke}{rgb}{0.281446,0.084320,0.407414}%
\pgfsetstrokecolor{currentstroke}%
\pgfsetdash{}{0pt}%
\pgfpathmoveto{\pgfqpoint{7.438290in}{2.726581in}}%
\pgfpathlineto{\pgfqpoint{7.438393in}{2.726586in}}%
\pgfusepath{stroke}%
\end{pgfscope}%
\begin{pgfscope}%
\pgfpathrectangle{\pgfqpoint{6.720588in}{1.750000in}}{\pgfqpoint{2.279412in}{2.004545in}}%
\pgfusepath{clip}%
\pgfsetbuttcap%
\pgfsetroundjoin%
\pgfsetlinewidth{0.400750pt}%
\definecolor{currentstroke}{rgb}{0.281446,0.084320,0.407414}%
\pgfsetstrokecolor{currentstroke}%
\pgfsetdash{}{0pt}%
\pgfpathmoveto{\pgfqpoint{7.438393in}{2.726586in}}%
\pgfpathlineto{\pgfqpoint{7.438301in}{2.726583in}}%
\pgfusepath{stroke}%
\end{pgfscope}%
\begin{pgfscope}%
\pgfpathrectangle{\pgfqpoint{6.720588in}{1.750000in}}{\pgfqpoint{2.279412in}{2.004545in}}%
\pgfusepath{clip}%
\pgfsetbuttcap%
\pgfsetroundjoin%
\pgfsetlinewidth{0.400327pt}%
\definecolor{currentstroke}{rgb}{0.281446,0.084320,0.407414}%
\pgfsetstrokecolor{currentstroke}%
\pgfsetdash{}{0pt}%
\pgfpathmoveto{\pgfqpoint{7.438301in}{2.726583in}}%
\pgfpathlineto{\pgfqpoint{7.438081in}{2.726573in}}%
\pgfusepath{stroke}%
\end{pgfscope}%
\begin{pgfscope}%
\pgfpathrectangle{\pgfqpoint{6.720588in}{1.750000in}}{\pgfqpoint{2.279412in}{2.004545in}}%
\pgfusepath{clip}%
\pgfsetbuttcap%
\pgfsetroundjoin%
\pgfsetlinewidth{0.399320pt}%
\definecolor{currentstroke}{rgb}{0.281446,0.084320,0.407414}%
\pgfsetstrokecolor{currentstroke}%
\pgfsetdash{}{0pt}%
\pgfpathmoveto{\pgfqpoint{7.438081in}{2.726573in}}%
\pgfpathlineto{\pgfqpoint{7.437951in}{2.726567in}}%
\pgfusepath{stroke}%
\end{pgfscope}%
\begin{pgfscope}%
\pgfpathrectangle{\pgfqpoint{6.720588in}{1.750000in}}{\pgfqpoint{2.279412in}{2.004545in}}%
\pgfusepath{clip}%
\pgfsetbuttcap%
\pgfsetroundjoin%
\pgfsetlinewidth{0.398724pt}%
\definecolor{currentstroke}{rgb}{0.281446,0.084320,0.407414}%
\pgfsetstrokecolor{currentstroke}%
\pgfsetdash{}{0pt}%
\pgfpathmoveto{\pgfqpoint{7.437951in}{2.726567in}}%
\pgfpathlineto{\pgfqpoint{7.438061in}{2.726572in}}%
\pgfusepath{stroke}%
\end{pgfscope}%
\begin{pgfscope}%
\pgfpathrectangle{\pgfqpoint{6.720588in}{1.750000in}}{\pgfqpoint{2.279412in}{2.004545in}}%
\pgfusepath{clip}%
\pgfsetbuttcap%
\pgfsetroundjoin%
\pgfsetlinewidth{0.399227pt}%
\definecolor{currentstroke}{rgb}{0.281446,0.084320,0.407414}%
\pgfsetstrokecolor{currentstroke}%
\pgfsetdash{}{0pt}%
\pgfpathmoveto{\pgfqpoint{7.438061in}{2.726572in}}%
\pgfpathlineto{\pgfqpoint{7.438290in}{2.726581in}}%
\pgfusepath{stroke}%
\end{pgfscope}%
\begin{pgfscope}%
\pgfpathrectangle{\pgfqpoint{6.720588in}{1.750000in}}{\pgfqpoint{2.279412in}{2.004545in}}%
\pgfusepath{clip}%
\pgfsetbuttcap%
\pgfsetroundjoin%
\pgfsetlinewidth{0.400277pt}%
\definecolor{currentstroke}{rgb}{0.281446,0.084320,0.407414}%
\pgfsetstrokecolor{currentstroke}%
\pgfsetdash{}{0pt}%
\pgfpathmoveto{\pgfqpoint{7.438290in}{2.726581in}}%
\pgfpathlineto{\pgfqpoint{7.438414in}{2.726587in}}%
\pgfusepath{stroke}%
\end{pgfscope}%
\begin{pgfscope}%
\pgfpathrectangle{\pgfqpoint{6.720588in}{1.750000in}}{\pgfqpoint{2.279412in}{2.004545in}}%
\pgfusepath{clip}%
\pgfsetbuttcap%
\pgfsetroundjoin%
\pgfsetlinewidth{0.400850pt}%
\definecolor{currentstroke}{rgb}{0.281446,0.084320,0.407414}%
\pgfsetstrokecolor{currentstroke}%
\pgfsetdash{}{0pt}%
\pgfpathmoveto{\pgfqpoint{7.438414in}{2.726587in}}%
\pgfpathlineto{\pgfqpoint{7.438324in}{2.726584in}}%
\pgfusepath{stroke}%
\end{pgfscope}%
\begin{pgfscope}%
\pgfpathrectangle{\pgfqpoint{6.720588in}{1.750000in}}{\pgfqpoint{2.279412in}{2.004545in}}%
\pgfusepath{clip}%
\pgfsetbuttcap%
\pgfsetroundjoin%
\pgfsetlinewidth{0.400434pt}%
\definecolor{currentstroke}{rgb}{0.281446,0.084320,0.407414}%
\pgfsetstrokecolor{currentstroke}%
\pgfsetdash{}{0pt}%
\pgfpathmoveto{\pgfqpoint{7.438324in}{2.726584in}}%
\pgfpathlineto{\pgfqpoint{7.438081in}{2.726574in}}%
\pgfusepath{stroke}%
\end{pgfscope}%
\begin{pgfscope}%
\pgfpathrectangle{\pgfqpoint{6.720588in}{1.750000in}}{\pgfqpoint{2.279412in}{2.004545in}}%
\pgfusepath{clip}%
\pgfsetbuttcap%
\pgfsetroundjoin%
\pgfsetlinewidth{0.399318pt}%
\definecolor{currentstroke}{rgb}{0.281446,0.084320,0.407414}%
\pgfsetstrokecolor{currentstroke}%
\pgfsetdash{}{0pt}%
\pgfpathmoveto{\pgfqpoint{7.438081in}{2.726574in}}%
\pgfpathlineto{\pgfqpoint{7.437922in}{2.726566in}}%
\pgfusepath{stroke}%
\end{pgfscope}%
\begin{pgfscope}%
\pgfpathrectangle{\pgfqpoint{6.720588in}{1.750000in}}{\pgfqpoint{2.279412in}{2.004545in}}%
\pgfusepath{clip}%
\pgfsetbuttcap%
\pgfsetroundjoin%
\pgfsetlinewidth{0.398589pt}%
\definecolor{currentstroke}{rgb}{0.281446,0.084320,0.407414}%
\pgfsetstrokecolor{currentstroke}%
\pgfsetdash{}{0pt}%
\pgfpathmoveto{\pgfqpoint{7.437922in}{2.726566in}}%
\pgfpathlineto{\pgfqpoint{7.438032in}{2.726570in}}%
\pgfusepath{stroke}%
\end{pgfscope}%
\begin{pgfscope}%
\pgfpathrectangle{\pgfqpoint{6.720588in}{1.750000in}}{\pgfqpoint{2.279412in}{2.004545in}}%
\pgfusepath{clip}%
\pgfsetbuttcap%
\pgfsetroundjoin%
\pgfsetlinewidth{0.399096pt}%
\definecolor{currentstroke}{rgb}{0.281446,0.084320,0.407414}%
\pgfsetstrokecolor{currentstroke}%
\pgfsetdash{}{0pt}%
\pgfpathmoveto{\pgfqpoint{7.438032in}{2.726570in}}%
\pgfpathlineto{\pgfqpoint{7.438288in}{2.726581in}}%
\pgfusepath{stroke}%
\end{pgfscope}%
\begin{pgfscope}%
\pgfpathrectangle{\pgfqpoint{6.720588in}{1.750000in}}{\pgfqpoint{2.279412in}{2.004545in}}%
\pgfusepath{clip}%
\pgfsetbuttcap%
\pgfsetroundjoin%
\pgfsetlinewidth{0.400268pt}%
\definecolor{currentstroke}{rgb}{0.281446,0.084320,0.407414}%
\pgfsetstrokecolor{currentstroke}%
\pgfsetdash{}{0pt}%
\pgfpathmoveto{\pgfqpoint{7.438288in}{2.726581in}}%
\pgfpathlineto{\pgfqpoint{7.438437in}{2.726588in}}%
\pgfusepath{stroke}%
\end{pgfscope}%
\begin{pgfscope}%
\pgfpathrectangle{\pgfqpoint{6.720588in}{1.750000in}}{\pgfqpoint{2.279412in}{2.004545in}}%
\pgfusepath{clip}%
\pgfsetbuttcap%
\pgfsetroundjoin%
\pgfsetlinewidth{0.400956pt}%
\definecolor{currentstroke}{rgb}{0.281446,0.084320,0.407414}%
\pgfsetstrokecolor{currentstroke}%
\pgfsetdash{}{0pt}%
\pgfpathmoveto{\pgfqpoint{7.438437in}{2.726588in}}%
\pgfpathlineto{\pgfqpoint{7.438351in}{2.726585in}}%
\pgfusepath{stroke}%
\end{pgfscope}%
\begin{pgfscope}%
\pgfpathrectangle{\pgfqpoint{6.720588in}{1.750000in}}{\pgfqpoint{2.279412in}{2.004545in}}%
\pgfusepath{clip}%
\pgfsetbuttcap%
\pgfsetroundjoin%
\pgfsetlinewidth{0.400557pt}%
\definecolor{currentstroke}{rgb}{0.281446,0.084320,0.407414}%
\pgfsetstrokecolor{currentstroke}%
\pgfsetdash{}{0pt}%
\pgfpathmoveto{\pgfqpoint{7.438351in}{2.726585in}}%
\pgfpathlineto{\pgfqpoint{7.438082in}{2.726574in}}%
\pgfusepath{stroke}%
\end{pgfscope}%
\begin{pgfscope}%
\pgfpathrectangle{\pgfqpoint{6.720588in}{1.750000in}}{\pgfqpoint{2.279412in}{2.004545in}}%
\pgfusepath{clip}%
\pgfsetbuttcap%
\pgfsetroundjoin%
\pgfsetlinewidth{0.399326pt}%
\definecolor{currentstroke}{rgb}{0.281446,0.084320,0.407414}%
\pgfsetstrokecolor{currentstroke}%
\pgfsetdash{}{0pt}%
\pgfpathmoveto{\pgfqpoint{7.438082in}{2.726574in}}%
\pgfpathlineto{\pgfqpoint{7.437890in}{2.726565in}}%
\pgfusepath{stroke}%
\end{pgfscope}%
\begin{pgfscope}%
\pgfpathrectangle{\pgfqpoint{6.720588in}{1.750000in}}{\pgfqpoint{2.279412in}{2.004545in}}%
\pgfusepath{clip}%
\pgfsetbuttcap%
\pgfsetroundjoin%
\pgfsetlinewidth{0.398443pt}%
\definecolor{currentstroke}{rgb}{0.281446,0.084320,0.407414}%
\pgfsetstrokecolor{currentstroke}%
\pgfsetdash{}{0pt}%
\pgfpathmoveto{\pgfqpoint{7.437890in}{2.726565in}}%
\pgfpathlineto{\pgfqpoint{7.437999in}{2.726569in}}%
\pgfusepath{stroke}%
\end{pgfscope}%
\begin{pgfscope}%
\pgfpathrectangle{\pgfqpoint{6.720588in}{1.750000in}}{\pgfqpoint{2.279412in}{2.004545in}}%
\pgfusepath{clip}%
\pgfsetbuttcap%
\pgfsetroundjoin%
\pgfsetlinewidth{0.398942pt}%
\definecolor{currentstroke}{rgb}{0.281446,0.084320,0.407414}%
\pgfsetstrokecolor{currentstroke}%
\pgfsetdash{}{0pt}%
\pgfpathmoveto{\pgfqpoint{7.437999in}{2.726569in}}%
\pgfpathlineto{\pgfqpoint{7.438284in}{2.726581in}}%
\pgfusepath{stroke}%
\end{pgfscope}%
\begin{pgfscope}%
\pgfpathrectangle{\pgfqpoint{6.720588in}{1.750000in}}{\pgfqpoint{2.279412in}{2.004545in}}%
\pgfusepath{clip}%
\pgfsetbuttcap%
\pgfsetroundjoin%
\pgfsetlinewidth{0.400249pt}%
\definecolor{currentstroke}{rgb}{0.281446,0.084320,0.407414}%
\pgfsetstrokecolor{currentstroke}%
\pgfsetdash{}{0pt}%
\pgfpathmoveto{\pgfqpoint{7.438284in}{2.726581in}}%
\pgfpathlineto{\pgfqpoint{7.438461in}{2.726589in}}%
\pgfusepath{stroke}%
\end{pgfscope}%
\begin{pgfscope}%
\pgfpathrectangle{\pgfqpoint{6.720588in}{1.750000in}}{\pgfqpoint{2.279412in}{2.004545in}}%
\pgfusepath{clip}%
\pgfsetbuttcap%
\pgfsetroundjoin%
\pgfsetlinewidth{0.401066pt}%
\definecolor{currentstroke}{rgb}{0.281446,0.084320,0.407414}%
\pgfsetstrokecolor{currentstroke}%
\pgfsetdash{}{0pt}%
\pgfpathmoveto{\pgfqpoint{7.438461in}{2.726589in}}%
\pgfpathlineto{\pgfqpoint{7.438381in}{2.726586in}}%
\pgfusepath{stroke}%
\end{pgfscope}%
\begin{pgfscope}%
\pgfpathrectangle{\pgfqpoint{6.720588in}{1.750000in}}{\pgfqpoint{2.279412in}{2.004545in}}%
\pgfusepath{clip}%
\pgfsetbuttcap%
\pgfsetroundjoin%
\pgfsetlinewidth{0.400696pt}%
\definecolor{currentstroke}{rgb}{0.281446,0.084320,0.407414}%
\pgfsetstrokecolor{currentstroke}%
\pgfsetdash{}{0pt}%
\pgfpathmoveto{\pgfqpoint{7.438381in}{2.726586in}}%
\pgfpathlineto{\pgfqpoint{7.438087in}{2.726574in}}%
\pgfusepath{stroke}%
\end{pgfscope}%
\begin{pgfscope}%
\pgfpathrectangle{\pgfqpoint{6.720588in}{1.750000in}}{\pgfqpoint{2.279412in}{2.004545in}}%
\pgfusepath{clip}%
\pgfsetbuttcap%
\pgfsetroundjoin%
\pgfsetlinewidth{0.399346pt}%
\definecolor{currentstroke}{rgb}{0.281446,0.084320,0.407414}%
\pgfsetstrokecolor{currentstroke}%
\pgfsetdash{}{0pt}%
\pgfpathmoveto{\pgfqpoint{7.438087in}{2.726574in}}%
\pgfpathlineto{\pgfqpoint{7.437855in}{2.726563in}}%
\pgfusepath{stroke}%
\end{pgfscope}%
\begin{pgfscope}%
\pgfpathrectangle{\pgfqpoint{6.720588in}{1.750000in}}{\pgfqpoint{2.279412in}{2.004545in}}%
\pgfusepath{clip}%
\pgfsetbuttcap%
\pgfsetroundjoin%
\pgfsetlinewidth{0.398285pt}%
\definecolor{currentstroke}{rgb}{0.281446,0.084320,0.407414}%
\pgfsetstrokecolor{currentstroke}%
\pgfsetdash{}{0pt}%
\pgfpathmoveto{\pgfqpoint{7.437855in}{2.726563in}}%
\pgfpathlineto{\pgfqpoint{7.437960in}{2.726567in}}%
\pgfusepath{stroke}%
\end{pgfscope}%
\begin{pgfscope}%
\pgfpathrectangle{\pgfqpoint{6.720588in}{1.750000in}}{\pgfqpoint{2.279412in}{2.004545in}}%
\pgfusepath{clip}%
\pgfsetbuttcap%
\pgfsetroundjoin%
\pgfsetlinewidth{0.398764pt}%
\definecolor{currentstroke}{rgb}{0.281446,0.084320,0.407414}%
\pgfsetstrokecolor{currentstroke}%
\pgfsetdash{}{0pt}%
\pgfpathmoveto{\pgfqpoint{7.437960in}{2.726567in}}%
\pgfpathlineto{\pgfqpoint{7.438276in}{2.726580in}}%
\pgfusepath{stroke}%
\end{pgfscope}%
\begin{pgfscope}%
\pgfpathrectangle{\pgfqpoint{6.720588in}{1.750000in}}{\pgfqpoint{2.279412in}{2.004545in}}%
\pgfusepath{clip}%
\pgfsetbuttcap%
\pgfsetroundjoin%
\pgfsetlinewidth{0.400215pt}%
\definecolor{currentstroke}{rgb}{0.281446,0.084320,0.407414}%
\pgfsetstrokecolor{currentstroke}%
\pgfsetdash{}{0pt}%
\pgfpathmoveto{\pgfqpoint{7.438276in}{2.726580in}}%
\pgfpathlineto{\pgfqpoint{7.438486in}{2.726590in}}%
\pgfusepath{stroke}%
\end{pgfscope}%
\begin{pgfscope}%
\pgfpathrectangle{\pgfqpoint{6.720588in}{1.750000in}}{\pgfqpoint{2.279412in}{2.004545in}}%
\pgfusepath{clip}%
\pgfsetbuttcap%
\pgfsetroundjoin%
\pgfsetlinewidth{0.401178pt}%
\definecolor{currentstroke}{rgb}{0.281446,0.084320,0.407414}%
\pgfsetstrokecolor{currentstroke}%
\pgfsetdash{}{0pt}%
\pgfpathmoveto{\pgfqpoint{7.438486in}{2.726590in}}%
\pgfpathlineto{\pgfqpoint{7.438415in}{2.726588in}}%
\pgfusepath{stroke}%
\end{pgfscope}%
\begin{pgfscope}%
\pgfpathrectangle{\pgfqpoint{6.720588in}{1.750000in}}{\pgfqpoint{2.279412in}{2.004545in}}%
\pgfusepath{clip}%
\pgfsetbuttcap%
\pgfsetroundjoin%
\pgfsetlinewidth{0.400850pt}%
\definecolor{currentstroke}{rgb}{0.281446,0.084320,0.407414}%
\pgfsetstrokecolor{currentstroke}%
\pgfsetdash{}{0pt}%
\pgfpathmoveto{\pgfqpoint{7.438415in}{2.726588in}}%
\pgfpathlineto{\pgfqpoint{7.438095in}{2.726574in}}%
\pgfusepath{stroke}%
\end{pgfscope}%
\begin{pgfscope}%
\pgfpathrectangle{\pgfqpoint{6.720588in}{1.750000in}}{\pgfqpoint{2.279412in}{2.004545in}}%
\pgfusepath{clip}%
\pgfsetbuttcap%
\pgfsetroundjoin%
\pgfsetlinewidth{0.399383pt}%
\definecolor{currentstroke}{rgb}{0.281446,0.084320,0.407414}%
\pgfsetstrokecolor{currentstroke}%
\pgfsetdash{}{0pt}%
\pgfpathmoveto{\pgfqpoint{7.438095in}{2.726574in}}%
\pgfpathlineto{\pgfqpoint{7.437819in}{2.726562in}}%
\pgfusepath{stroke}%
\end{pgfscope}%
\begin{pgfscope}%
\pgfpathrectangle{\pgfqpoint{6.720588in}{1.750000in}}{\pgfqpoint{2.279412in}{2.004545in}}%
\pgfusepath{clip}%
\pgfsetbuttcap%
\pgfsetroundjoin%
\pgfsetlinewidth{0.398119pt}%
\definecolor{currentstroke}{rgb}{0.281446,0.084320,0.407414}%
\pgfsetstrokecolor{currentstroke}%
\pgfsetdash{}{0pt}%
\pgfpathmoveto{\pgfqpoint{7.437819in}{2.726562in}}%
\pgfpathlineto{\pgfqpoint{7.437915in}{2.726565in}}%
\pgfusepath{stroke}%
\end{pgfscope}%
\begin{pgfscope}%
\pgfpathrectangle{\pgfqpoint{6.720588in}{1.750000in}}{\pgfqpoint{2.279412in}{2.004545in}}%
\pgfusepath{clip}%
\pgfsetbuttcap%
\pgfsetroundjoin%
\pgfsetlinewidth{0.398559pt}%
\definecolor{currentstroke}{rgb}{0.281446,0.084320,0.407414}%
\pgfsetstrokecolor{currentstroke}%
\pgfsetdash{}{0pt}%
\pgfpathmoveto{\pgfqpoint{7.437915in}{2.726565in}}%
\pgfpathlineto{\pgfqpoint{7.438265in}{2.726580in}}%
\pgfusepath{stroke}%
\end{pgfscope}%
\begin{pgfscope}%
\pgfpathrectangle{\pgfqpoint{6.720588in}{1.750000in}}{\pgfqpoint{2.279412in}{2.004545in}}%
\pgfusepath{clip}%
\pgfsetbuttcap%
\pgfsetroundjoin%
\pgfsetlinewidth{0.400164pt}%
\definecolor{currentstroke}{rgb}{0.281446,0.084320,0.407414}%
\pgfsetstrokecolor{currentstroke}%
\pgfsetdash{}{0pt}%
\pgfpathmoveto{\pgfqpoint{7.438265in}{2.726580in}}%
\pgfpathlineto{\pgfqpoint{7.438510in}{2.726591in}}%
\pgfusepath{stroke}%
\end{pgfscope}%
\begin{pgfscope}%
\pgfpathrectangle{\pgfqpoint{6.720588in}{1.750000in}}{\pgfqpoint{2.279412in}{2.004545in}}%
\pgfusepath{clip}%
\pgfsetbuttcap%
\pgfsetroundjoin%
\pgfsetlinewidth{0.401291pt}%
\definecolor{currentstroke}{rgb}{0.281446,0.084320,0.407414}%
\pgfsetstrokecolor{currentstroke}%
\pgfsetdash{}{0pt}%
\pgfpathmoveto{\pgfqpoint{7.438510in}{2.726591in}}%
\pgfpathlineto{\pgfqpoint{7.438452in}{2.726589in}}%
\pgfusepath{stroke}%
\end{pgfscope}%
\begin{pgfscope}%
\pgfpathrectangle{\pgfqpoint{6.720588in}{1.750000in}}{\pgfqpoint{2.279412in}{2.004545in}}%
\pgfusepath{clip}%
\pgfsetbuttcap%
\pgfsetroundjoin%
\pgfsetlinewidth{0.401022pt}%
\definecolor{currentstroke}{rgb}{0.281446,0.084320,0.407414}%
\pgfsetstrokecolor{currentstroke}%
\pgfsetdash{}{0pt}%
\pgfpathmoveto{\pgfqpoint{7.438452in}{2.726589in}}%
\pgfpathlineto{\pgfqpoint{7.438107in}{2.726575in}}%
\pgfusepath{stroke}%
\end{pgfscope}%
\begin{pgfscope}%
\pgfpathrectangle{\pgfqpoint{6.720588in}{1.750000in}}{\pgfqpoint{2.279412in}{2.004545in}}%
\pgfusepath{clip}%
\pgfsetbuttcap%
\pgfsetroundjoin%
\pgfsetlinewidth{0.399439pt}%
\definecolor{currentstroke}{rgb}{0.281446,0.084320,0.407414}%
\pgfsetstrokecolor{currentstroke}%
\pgfsetdash{}{0pt}%
\pgfpathmoveto{\pgfqpoint{7.438107in}{2.726575in}}%
\pgfpathlineto{\pgfqpoint{7.437781in}{2.726560in}}%
\pgfusepath{stroke}%
\end{pgfscope}%
\begin{pgfscope}%
\pgfpathrectangle{\pgfqpoint{6.720588in}{1.750000in}}{\pgfqpoint{2.279412in}{2.004545in}}%
\pgfusepath{clip}%
\pgfsetbuttcap%
\pgfsetroundjoin%
\pgfsetlinewidth{0.397944pt}%
\definecolor{currentstroke}{rgb}{0.281446,0.084320,0.407414}%
\pgfsetstrokecolor{currentstroke}%
\pgfsetdash{}{0pt}%
\pgfpathmoveto{\pgfqpoint{7.437781in}{2.726560in}}%
\pgfpathlineto{\pgfqpoint{7.437863in}{2.726563in}}%
\pgfusepath{stroke}%
\end{pgfscope}%
\begin{pgfscope}%
\pgfpathrectangle{\pgfqpoint{6.720588in}{1.750000in}}{\pgfqpoint{2.279412in}{2.004545in}}%
\pgfusepath{clip}%
\pgfsetbuttcap%
\pgfsetroundjoin%
\pgfsetlinewidth{0.398322pt}%
\definecolor{currentstroke}{rgb}{0.281446,0.084320,0.407414}%
\pgfsetstrokecolor{currentstroke}%
\pgfsetdash{}{0pt}%
\pgfpathmoveto{\pgfqpoint{7.437863in}{2.726563in}}%
\pgfpathlineto{\pgfqpoint{7.438249in}{2.726579in}}%
\pgfusepath{stroke}%
\end{pgfscope}%
\begin{pgfscope}%
\pgfpathrectangle{\pgfqpoint{6.720588in}{1.750000in}}{\pgfqpoint{2.279412in}{2.004545in}}%
\pgfusepath{clip}%
\pgfsetbuttcap%
\pgfsetroundjoin%
\pgfsetlinewidth{0.400091pt}%
\definecolor{currentstroke}{rgb}{0.281446,0.084320,0.407414}%
\pgfsetstrokecolor{currentstroke}%
\pgfsetdash{}{0pt}%
\pgfpathmoveto{\pgfqpoint{7.438249in}{2.726579in}}%
\pgfpathlineto{\pgfqpoint{7.438534in}{2.726592in}}%
\pgfusepath{stroke}%
\end{pgfscope}%
\begin{pgfscope}%
\pgfpathrectangle{\pgfqpoint{6.720588in}{1.750000in}}{\pgfqpoint{2.279412in}{2.004545in}}%
\pgfusepath{clip}%
\pgfsetbuttcap%
\pgfsetroundjoin%
\pgfsetlinewidth{0.401402pt}%
\definecolor{currentstroke}{rgb}{0.281446,0.084320,0.407414}%
\pgfsetstrokecolor{currentstroke}%
\pgfsetdash{}{0pt}%
\pgfpathmoveto{\pgfqpoint{7.438534in}{2.726592in}}%
\pgfpathlineto{\pgfqpoint{7.438493in}{2.726591in}}%
\pgfusepath{stroke}%
\end{pgfscope}%
\begin{pgfscope}%
\pgfpathrectangle{\pgfqpoint{6.720588in}{1.750000in}}{\pgfqpoint{2.279412in}{2.004545in}}%
\pgfusepath{clip}%
\pgfsetbuttcap%
\pgfsetroundjoin%
\pgfsetlinewidth{0.401210pt}%
\definecolor{currentstroke}{rgb}{0.281446,0.084320,0.407414}%
\pgfsetstrokecolor{currentstroke}%
\pgfsetdash{}{0pt}%
\pgfpathmoveto{\pgfqpoint{7.438493in}{2.726591in}}%
\pgfpathlineto{\pgfqpoint{7.438125in}{2.726576in}}%
\pgfusepath{stroke}%
\end{pgfscope}%
\begin{pgfscope}%
\pgfpathrectangle{\pgfqpoint{6.720588in}{1.750000in}}{\pgfqpoint{2.279412in}{2.004545in}}%
\pgfusepath{clip}%
\pgfsetbuttcap%
\pgfsetroundjoin%
\pgfsetlinewidth{0.399519pt}%
\definecolor{currentstroke}{rgb}{0.281446,0.084320,0.407414}%
\pgfsetstrokecolor{currentstroke}%
\pgfsetdash{}{0pt}%
\pgfpathmoveto{\pgfqpoint{7.438125in}{2.726576in}}%
\pgfpathlineto{\pgfqpoint{7.437742in}{2.726559in}}%
\pgfusepath{stroke}%
\end{pgfscope}%
\begin{pgfscope}%
\pgfpathrectangle{\pgfqpoint{6.720588in}{1.750000in}}{\pgfqpoint{2.279412in}{2.004545in}}%
\pgfusepath{clip}%
\pgfsetbuttcap%
\pgfsetroundjoin%
\pgfsetlinewidth{0.397766pt}%
\definecolor{currentstroke}{rgb}{0.281446,0.084320,0.407414}%
\pgfsetstrokecolor{currentstroke}%
\pgfsetdash{}{0pt}%
\pgfpathmoveto{\pgfqpoint{7.437742in}{2.726559in}}%
\pgfpathlineto{\pgfqpoint{7.437805in}{2.726560in}}%
\pgfusepath{stroke}%
\end{pgfscope}%
\begin{pgfscope}%
\pgfpathrectangle{\pgfqpoint{6.720588in}{1.750000in}}{\pgfqpoint{2.279412in}{2.004545in}}%
\pgfusepath{clip}%
\pgfsetbuttcap%
\pgfsetroundjoin%
\pgfsetlinewidth{0.398053pt}%
\definecolor{currentstroke}{rgb}{0.281446,0.084320,0.407414}%
\pgfsetstrokecolor{currentstroke}%
\pgfsetdash{}{0pt}%
\pgfpathmoveto{\pgfqpoint{7.437805in}{2.726560in}}%
\pgfpathlineto{\pgfqpoint{7.438228in}{2.726578in}}%
\pgfusepath{stroke}%
\end{pgfscope}%
\begin{pgfscope}%
\pgfpathrectangle{\pgfqpoint{6.720588in}{1.750000in}}{\pgfqpoint{2.279412in}{2.004545in}}%
\pgfusepath{clip}%
\pgfsetbuttcap%
\pgfsetroundjoin%
\pgfsetlinewidth{0.399992pt}%
\definecolor{currentstroke}{rgb}{0.281446,0.084320,0.407414}%
\pgfsetstrokecolor{currentstroke}%
\pgfsetdash{}{0pt}%
\pgfpathmoveto{\pgfqpoint{7.438228in}{2.726578in}}%
\pgfpathlineto{\pgfqpoint{7.438557in}{2.726592in}}%
\pgfusepath{stroke}%
\end{pgfscope}%
\begin{pgfscope}%
\pgfpathrectangle{\pgfqpoint{6.720588in}{1.750000in}}{\pgfqpoint{2.279412in}{2.004545in}}%
\pgfusepath{clip}%
\pgfsetbuttcap%
\pgfsetroundjoin%
\pgfsetlinewidth{0.401507pt}%
\definecolor{currentstroke}{rgb}{0.281446,0.084320,0.407414}%
\pgfsetstrokecolor{currentstroke}%
\pgfsetdash{}{0pt}%
\pgfpathmoveto{\pgfqpoint{7.438557in}{2.726592in}}%
\pgfpathlineto{\pgfqpoint{7.438537in}{2.726593in}}%
\pgfusepath{stroke}%
\end{pgfscope}%
\begin{pgfscope}%
\pgfpathrectangle{\pgfqpoint{6.720588in}{1.750000in}}{\pgfqpoint{2.279412in}{2.004545in}}%
\pgfusepath{clip}%
\pgfsetbuttcap%
\pgfsetroundjoin%
\pgfsetlinewidth{0.401413pt}%
\definecolor{currentstroke}{rgb}{0.281446,0.084320,0.407414}%
\pgfsetstrokecolor{currentstroke}%
\pgfsetdash{}{0pt}%
\pgfpathmoveto{\pgfqpoint{7.438537in}{2.726593in}}%
\pgfpathlineto{\pgfqpoint{7.438148in}{2.726577in}}%
\pgfusepath{stroke}%
\end{pgfscope}%
\begin{pgfscope}%
\pgfpathrectangle{\pgfqpoint{6.720588in}{1.750000in}}{\pgfqpoint{2.279412in}{2.004545in}}%
\pgfusepath{clip}%
\pgfsetbuttcap%
\pgfsetroundjoin%
\pgfsetlinewidth{0.399626pt}%
\definecolor{currentstroke}{rgb}{0.281446,0.084320,0.407414}%
\pgfsetstrokecolor{currentstroke}%
\pgfsetdash{}{0pt}%
\pgfpathmoveto{\pgfqpoint{7.438148in}{2.726577in}}%
\pgfpathlineto{\pgfqpoint{7.437703in}{2.726557in}}%
\pgfusepath{stroke}%
\end{pgfscope}%
\begin{pgfscope}%
\pgfpathrectangle{\pgfqpoint{6.720588in}{1.750000in}}{\pgfqpoint{2.279412in}{2.004545in}}%
\pgfusepath{clip}%
\pgfsetbuttcap%
\pgfsetroundjoin%
\pgfsetlinewidth{0.397589pt}%
\definecolor{currentstroke}{rgb}{0.281446,0.084320,0.407414}%
\pgfsetstrokecolor{currentstroke}%
\pgfsetdash{}{0pt}%
\pgfpathmoveto{\pgfqpoint{7.437703in}{2.726557in}}%
\pgfpathlineto{\pgfqpoint{7.437738in}{2.726557in}}%
\pgfusepath{stroke}%
\end{pgfscope}%
\begin{pgfscope}%
\pgfpathrectangle{\pgfqpoint{6.720588in}{1.750000in}}{\pgfqpoint{2.279412in}{2.004545in}}%
\pgfusepath{clip}%
\pgfsetbuttcap%
\pgfsetroundjoin%
\pgfsetlinewidth{0.397748pt}%
\definecolor{currentstroke}{rgb}{0.281446,0.084320,0.407414}%
\pgfsetstrokecolor{currentstroke}%
\pgfsetdash{}{0pt}%
\pgfpathmoveto{\pgfqpoint{7.437738in}{2.726557in}}%
\pgfpathlineto{\pgfqpoint{7.438199in}{2.726576in}}%
\pgfusepath{stroke}%
\end{pgfscope}%
\begin{pgfscope}%
\pgfpathrectangle{\pgfqpoint{6.720588in}{1.750000in}}{\pgfqpoint{2.279412in}{2.004545in}}%
\pgfusepath{clip}%
\pgfsetbuttcap%
\pgfsetroundjoin%
\pgfsetlinewidth{0.399862pt}%
\definecolor{currentstroke}{rgb}{0.281446,0.084320,0.407414}%
\pgfsetstrokecolor{currentstroke}%
\pgfsetdash{}{0pt}%
\pgfpathmoveto{\pgfqpoint{7.438199in}{2.726576in}}%
\pgfpathlineto{\pgfqpoint{7.438578in}{2.726593in}}%
\pgfusepath{stroke}%
\end{pgfscope}%
\begin{pgfscope}%
\pgfpathrectangle{\pgfqpoint{6.720588in}{1.750000in}}{\pgfqpoint{2.279412in}{2.004545in}}%
\pgfusepath{clip}%
\pgfsetbuttcap%
\pgfsetroundjoin%
\pgfsetlinewidth{0.401601pt}%
\definecolor{currentstroke}{rgb}{0.281446,0.084320,0.407414}%
\pgfsetstrokecolor{currentstroke}%
\pgfsetdash{}{0pt}%
\pgfpathmoveto{\pgfqpoint{7.438578in}{2.726593in}}%
\pgfpathlineto{\pgfqpoint{7.438584in}{2.726594in}}%
\pgfusepath{stroke}%
\end{pgfscope}%
\begin{pgfscope}%
\pgfpathrectangle{\pgfqpoint{6.720588in}{1.750000in}}{\pgfqpoint{2.279412in}{2.004545in}}%
\pgfusepath{clip}%
\pgfsetbuttcap%
\pgfsetroundjoin%
\pgfsetlinewidth{0.401628pt}%
\definecolor{currentstroke}{rgb}{0.281446,0.084320,0.407414}%
\pgfsetstrokecolor{currentstroke}%
\pgfsetdash{}{0pt}%
\pgfpathmoveto{\pgfqpoint{7.438584in}{2.726594in}}%
\pgfpathlineto{\pgfqpoint{7.438178in}{2.726578in}}%
\pgfusepath{stroke}%
\end{pgfscope}%
\begin{pgfscope}%
\pgfpathrectangle{\pgfqpoint{6.720588in}{1.750000in}}{\pgfqpoint{2.279412in}{2.004545in}}%
\pgfusepath{clip}%
\pgfsetbuttcap%
\pgfsetroundjoin%
\pgfsetlinewidth{0.399765pt}%
\definecolor{currentstroke}{rgb}{0.281446,0.084320,0.407414}%
\pgfsetstrokecolor{currentstroke}%
\pgfsetdash{}{0pt}%
\pgfpathmoveto{\pgfqpoint{7.438178in}{2.726578in}}%
\pgfpathlineto{\pgfqpoint{7.437666in}{2.726556in}}%
\pgfusepath{stroke}%
\end{pgfscope}%
\begin{pgfscope}%
\pgfpathrectangle{\pgfqpoint{6.720588in}{1.750000in}}{\pgfqpoint{2.279412in}{2.004545in}}%
\pgfusepath{clip}%
\pgfsetbuttcap%
\pgfsetroundjoin%
\pgfsetlinewidth{0.397420pt}%
\definecolor{currentstroke}{rgb}{0.281446,0.084320,0.407414}%
\pgfsetstrokecolor{currentstroke}%
\pgfsetdash{}{0pt}%
\pgfpathmoveto{\pgfqpoint{7.437666in}{2.726556in}}%
\pgfpathlineto{\pgfqpoint{7.437663in}{2.726554in}}%
\pgfusepath{stroke}%
\end{pgfscope}%
\begin{pgfscope}%
\pgfpathrectangle{\pgfqpoint{6.720588in}{1.750000in}}{\pgfqpoint{2.279412in}{2.004545in}}%
\pgfusepath{clip}%
\pgfsetbuttcap%
\pgfsetroundjoin%
\pgfsetlinewidth{0.397406pt}%
\definecolor{currentstroke}{rgb}{0.281446,0.084320,0.407414}%
\pgfsetstrokecolor{currentstroke}%
\pgfsetdash{}{0pt}%
\pgfpathmoveto{\pgfqpoint{7.437663in}{2.726554in}}%
\pgfpathlineto{\pgfqpoint{7.438163in}{2.726574in}}%
\pgfusepath{stroke}%
\end{pgfscope}%
\begin{pgfscope}%
\pgfpathrectangle{\pgfqpoint{6.720588in}{1.750000in}}{\pgfqpoint{2.279412in}{2.004545in}}%
\pgfusepath{clip}%
\pgfsetbuttcap%
\pgfsetroundjoin%
\pgfsetlinewidth{0.399694pt}%
\definecolor{currentstroke}{rgb}{0.281446,0.084320,0.407414}%
\pgfsetstrokecolor{currentstroke}%
\pgfsetdash{}{0pt}%
\pgfpathmoveto{\pgfqpoint{7.438163in}{2.726574in}}%
\pgfpathlineto{\pgfqpoint{7.438595in}{2.726593in}}%
\pgfusepath{stroke}%
\end{pgfscope}%
\begin{pgfscope}%
\pgfpathrectangle{\pgfqpoint{6.720588in}{1.750000in}}{\pgfqpoint{2.279412in}{2.004545in}}%
\pgfusepath{clip}%
\pgfsetbuttcap%
\pgfsetroundjoin%
\pgfsetlinewidth{0.401679pt}%
\definecolor{currentstroke}{rgb}{0.281446,0.084320,0.407414}%
\pgfsetstrokecolor{currentstroke}%
\pgfsetdash{}{0pt}%
\pgfpathmoveto{\pgfqpoint{7.438595in}{2.726593in}}%
\pgfpathlineto{\pgfqpoint{7.438633in}{2.726596in}}%
\pgfusepath{stroke}%
\end{pgfscope}%
\begin{pgfscope}%
\pgfpathrectangle{\pgfqpoint{6.720588in}{1.750000in}}{\pgfqpoint{2.279412in}{2.004545in}}%
\pgfusepath{clip}%
\pgfsetbuttcap%
\pgfsetroundjoin%
\pgfsetlinewidth{0.401854pt}%
\definecolor{currentstroke}{rgb}{0.281446,0.084320,0.407414}%
\pgfsetstrokecolor{currentstroke}%
\pgfsetdash{}{0pt}%
\pgfpathmoveto{\pgfqpoint{7.438633in}{2.726596in}}%
\pgfpathlineto{\pgfqpoint{7.438216in}{2.726580in}}%
\pgfusepath{stroke}%
\end{pgfscope}%
\begin{pgfscope}%
\pgfpathrectangle{\pgfqpoint{6.720588in}{1.750000in}}{\pgfqpoint{2.279412in}{2.004545in}}%
\pgfusepath{clip}%
\pgfsetbuttcap%
\pgfsetroundjoin%
\pgfsetlinewidth{0.399940pt}%
\definecolor{currentstroke}{rgb}{0.281446,0.084320,0.407414}%
\pgfsetstrokecolor{currentstroke}%
\pgfsetdash{}{0pt}%
\pgfpathmoveto{\pgfqpoint{7.438216in}{2.726580in}}%
\pgfpathlineto{\pgfqpoint{7.437634in}{2.726554in}}%
\pgfusepath{stroke}%
\end{pgfscope}%
\begin{pgfscope}%
\pgfpathrectangle{\pgfqpoint{6.720588in}{1.750000in}}{\pgfqpoint{2.279412in}{2.004545in}}%
\pgfusepath{clip}%
\pgfsetbuttcap%
\pgfsetroundjoin%
\pgfsetlinewidth{0.397271pt}%
\definecolor{currentstroke}{rgb}{0.281446,0.084320,0.407414}%
\pgfsetstrokecolor{currentstroke}%
\pgfsetdash{}{0pt}%
\pgfpathmoveto{\pgfqpoint{7.437634in}{2.726554in}}%
\pgfpathlineto{\pgfqpoint{7.437580in}{2.726550in}}%
\pgfusepath{stroke}%
\end{pgfscope}%
\begin{pgfscope}%
\pgfpathrectangle{\pgfqpoint{6.720588in}{1.750000in}}{\pgfqpoint{2.279412in}{2.004545in}}%
\pgfusepath{clip}%
\pgfsetbuttcap%
\pgfsetroundjoin%
\pgfsetlinewidth{0.397029pt}%
\definecolor{currentstroke}{rgb}{0.280894,0.078907,0.402329}%
\pgfsetstrokecolor{currentstroke}%
\pgfsetdash{}{0pt}%
\pgfpathmoveto{\pgfqpoint{7.437580in}{2.726550in}}%
\pgfpathlineto{\pgfqpoint{7.438117in}{2.726572in}}%
\pgfusepath{stroke}%
\end{pgfscope}%
\begin{pgfscope}%
\pgfpathrectangle{\pgfqpoint{6.720588in}{1.750000in}}{\pgfqpoint{2.279412in}{2.004545in}}%
\pgfusepath{clip}%
\pgfsetbuttcap%
\pgfsetroundjoin%
\pgfsetlinewidth{0.399484pt}%
\definecolor{currentstroke}{rgb}{0.281446,0.084320,0.407414}%
\pgfsetstrokecolor{currentstroke}%
\pgfsetdash{}{0pt}%
\pgfpathmoveto{\pgfqpoint{7.438117in}{2.726572in}}%
\pgfpathlineto{\pgfqpoint{7.438607in}{2.726594in}}%
\pgfusepath{stroke}%
\end{pgfscope}%
\begin{pgfscope}%
\pgfpathrectangle{\pgfqpoint{6.720588in}{1.750000in}}{\pgfqpoint{2.279412in}{2.004545in}}%
\pgfusepath{clip}%
\pgfsetbuttcap%
\pgfsetroundjoin%
\pgfsetlinewidth{0.401735pt}%
\definecolor{currentstroke}{rgb}{0.281446,0.084320,0.407414}%
\pgfsetstrokecolor{currentstroke}%
\pgfsetdash{}{0pt}%
\pgfpathmoveto{\pgfqpoint{7.438607in}{2.726594in}}%
\pgfpathlineto{\pgfqpoint{7.438682in}{2.726598in}}%
\pgfusepath{stroke}%
\end{pgfscope}%
\begin{pgfscope}%
\pgfpathrectangle{\pgfqpoint{6.720588in}{1.750000in}}{\pgfqpoint{2.279412in}{2.004545in}}%
\pgfusepath{clip}%
\pgfsetbuttcap%
\pgfsetroundjoin%
\pgfsetlinewidth{0.402083pt}%
\definecolor{currentstroke}{rgb}{0.281446,0.084320,0.407414}%
\pgfsetstrokecolor{currentstroke}%
\pgfsetdash{}{0pt}%
\pgfpathmoveto{\pgfqpoint{7.438682in}{2.726598in}}%
\pgfpathlineto{\pgfqpoint{7.438263in}{2.726582in}}%
\pgfusepath{stroke}%
\end{pgfscope}%
\begin{pgfscope}%
\pgfpathrectangle{\pgfqpoint{6.720588in}{1.750000in}}{\pgfqpoint{2.279412in}{2.004545in}}%
\pgfusepath{clip}%
\pgfsetbuttcap%
\pgfsetroundjoin%
\pgfsetlinewidth{0.400154pt}%
\definecolor{currentstroke}{rgb}{0.281446,0.084320,0.407414}%
\pgfsetstrokecolor{currentstroke}%
\pgfsetdash{}{0pt}%
\pgfpathmoveto{\pgfqpoint{7.438263in}{2.726582in}}%
\pgfpathlineto{\pgfqpoint{7.437608in}{2.726553in}}%
\pgfusepath{stroke}%
\end{pgfscope}%
\begin{pgfscope}%
\pgfpathrectangle{\pgfqpoint{6.720588in}{1.750000in}}{\pgfqpoint{2.279412in}{2.004545in}}%
\pgfusepath{clip}%
\pgfsetbuttcap%
\pgfsetroundjoin%
\pgfsetlinewidth{0.397152pt}%
\definecolor{currentstroke}{rgb}{0.280894,0.078907,0.402329}%
\pgfsetstrokecolor{currentstroke}%
\pgfsetdash{}{0pt}%
\pgfpathmoveto{\pgfqpoint{7.437608in}{2.726553in}}%
\pgfpathlineto{\pgfqpoint{7.437491in}{2.726546in}}%
\pgfusepath{stroke}%
\end{pgfscope}%
\begin{pgfscope}%
\pgfpathrectangle{\pgfqpoint{6.720588in}{1.750000in}}{\pgfqpoint{2.279412in}{2.004545in}}%
\pgfusepath{clip}%
\pgfsetbuttcap%
\pgfsetroundjoin%
\pgfsetlinewidth{0.396618pt}%
\definecolor{currentstroke}{rgb}{0.280894,0.078907,0.402329}%
\pgfsetstrokecolor{currentstroke}%
\pgfsetdash{}{0pt}%
\pgfpathmoveto{\pgfqpoint{7.437491in}{2.726546in}}%
\pgfpathlineto{\pgfqpoint{7.438060in}{2.726570in}}%
\pgfusepath{stroke}%
\end{pgfscope}%
\begin{pgfscope}%
\pgfpathrectangle{\pgfqpoint{6.720588in}{1.750000in}}{\pgfqpoint{2.279412in}{2.004545in}}%
\pgfusepath{clip}%
\pgfsetbuttcap%
\pgfsetroundjoin%
\pgfsetlinewidth{0.399224pt}%
\definecolor{currentstroke}{rgb}{0.281446,0.084320,0.407414}%
\pgfsetstrokecolor{currentstroke}%
\pgfsetdash{}{0pt}%
\pgfpathmoveto{\pgfqpoint{7.438060in}{2.726570in}}%
\pgfpathlineto{\pgfqpoint{7.438613in}{2.726594in}}%
\pgfusepath{stroke}%
\end{pgfscope}%
\begin{pgfscope}%
\pgfpathrectangle{\pgfqpoint{6.720588in}{1.750000in}}{\pgfqpoint{2.279412in}{2.004545in}}%
\pgfusepath{clip}%
\pgfsetbuttcap%
\pgfsetroundjoin%
\pgfsetlinewidth{0.401762pt}%
\definecolor{currentstroke}{rgb}{0.281446,0.084320,0.407414}%
\pgfsetstrokecolor{currentstroke}%
\pgfsetdash{}{0pt}%
\pgfpathmoveto{\pgfqpoint{7.438613in}{2.726594in}}%
\pgfpathlineto{\pgfqpoint{7.438732in}{2.726600in}}%
\pgfusepath{stroke}%
\end{pgfscope}%
\begin{pgfscope}%
\pgfpathrectangle{\pgfqpoint{6.720588in}{1.750000in}}{\pgfqpoint{2.279412in}{2.004545in}}%
\pgfusepath{clip}%
\pgfsetbuttcap%
\pgfsetroundjoin%
\pgfsetlinewidth{0.402311pt}%
\definecolor{currentstroke}{rgb}{0.281446,0.084320,0.407414}%
\pgfsetstrokecolor{currentstroke}%
\pgfsetdash{}{0pt}%
\pgfpathmoveto{\pgfqpoint{7.438732in}{2.726600in}}%
\pgfpathlineto{\pgfqpoint{7.438318in}{2.726584in}}%
\pgfusepath{stroke}%
\end{pgfscope}%
\begin{pgfscope}%
\pgfpathrectangle{\pgfqpoint{6.720588in}{1.750000in}}{\pgfqpoint{2.279412in}{2.004545in}}%
\pgfusepath{clip}%
\pgfsetbuttcap%
\pgfsetroundjoin%
\pgfsetlinewidth{0.400407pt}%
\definecolor{currentstroke}{rgb}{0.281446,0.084320,0.407414}%
\pgfsetstrokecolor{currentstroke}%
\pgfsetdash{}{0pt}%
\pgfpathmoveto{\pgfqpoint{7.438318in}{2.726584in}}%
\pgfpathlineto{\pgfqpoint{7.437592in}{2.726553in}}%
\pgfusepath{stroke}%
\end{pgfscope}%
\begin{pgfscope}%
\pgfpathrectangle{\pgfqpoint{6.720588in}{1.750000in}}{\pgfqpoint{2.279412in}{2.004545in}}%
\pgfusepath{clip}%
\pgfsetbuttcap%
\pgfsetroundjoin%
\pgfsetlinewidth{0.397080pt}%
\definecolor{currentstroke}{rgb}{0.280894,0.078907,0.402329}%
\pgfsetstrokecolor{currentstroke}%
\pgfsetdash{}{0pt}%
\pgfpathmoveto{\pgfqpoint{7.437592in}{2.726553in}}%
\pgfpathlineto{\pgfqpoint{7.437395in}{2.726542in}}%
\pgfusepath{stroke}%
\end{pgfscope}%
\begin{pgfscope}%
\pgfpathrectangle{\pgfqpoint{6.720588in}{1.750000in}}{\pgfqpoint{2.279412in}{2.004545in}}%
\pgfusepath{clip}%
\pgfsetbuttcap%
\pgfsetroundjoin%
\pgfsetlinewidth{0.396182pt}%
\definecolor{currentstroke}{rgb}{0.280894,0.078907,0.402329}%
\pgfsetstrokecolor{currentstroke}%
\pgfsetdash{}{0pt}%
\pgfpathmoveto{\pgfqpoint{7.437395in}{2.726542in}}%
\pgfpathlineto{\pgfqpoint{7.437991in}{2.726566in}}%
\pgfusepath{stroke}%
\end{pgfscope}%
\begin{pgfscope}%
\pgfpathrectangle{\pgfqpoint{6.720588in}{1.750000in}}{\pgfqpoint{2.279412in}{2.004545in}}%
\pgfusepath{clip}%
\pgfsetbuttcap%
\pgfsetroundjoin%
\pgfsetlinewidth{0.398909pt}%
\definecolor{currentstroke}{rgb}{0.281446,0.084320,0.407414}%
\pgfsetstrokecolor{currentstroke}%
\pgfsetdash{}{0pt}%
\pgfpathmoveto{\pgfqpoint{7.437991in}{2.726566in}}%
\pgfpathlineto{\pgfqpoint{7.438610in}{2.726593in}}%
\pgfusepath{stroke}%
\end{pgfscope}%
\begin{pgfscope}%
\pgfpathrectangle{\pgfqpoint{6.720588in}{1.750000in}}{\pgfqpoint{2.279412in}{2.004545in}}%
\pgfusepath{clip}%
\pgfsetbuttcap%
\pgfsetroundjoin%
\pgfsetlinewidth{0.401750pt}%
\definecolor{currentstroke}{rgb}{0.281446,0.084320,0.407414}%
\pgfsetstrokecolor{currentstroke}%
\pgfsetdash{}{0pt}%
\pgfpathmoveto{\pgfqpoint{7.438610in}{2.726593in}}%
\pgfpathlineto{\pgfqpoint{7.438780in}{2.726602in}}%
\pgfusepath{stroke}%
\end{pgfscope}%
\begin{pgfscope}%
\pgfpathrectangle{\pgfqpoint{6.720588in}{1.750000in}}{\pgfqpoint{2.279412in}{2.004545in}}%
\pgfusepath{clip}%
\pgfsetbuttcap%
\pgfsetroundjoin%
\pgfsetlinewidth{0.402531pt}%
\definecolor{currentstroke}{rgb}{0.281446,0.084320,0.407414}%
\pgfsetstrokecolor{currentstroke}%
\pgfsetdash{}{0pt}%
\pgfpathmoveto{\pgfqpoint{7.438780in}{2.726602in}}%
\pgfpathlineto{\pgfqpoint{7.438381in}{2.726587in}}%
\pgfusepath{stroke}%
\end{pgfscope}%
\begin{pgfscope}%
\pgfpathrectangle{\pgfqpoint{6.720588in}{1.750000in}}{\pgfqpoint{2.279412in}{2.004545in}}%
\pgfusepath{clip}%
\pgfsetbuttcap%
\pgfsetroundjoin%
\pgfsetlinewidth{0.400697pt}%
\definecolor{currentstroke}{rgb}{0.281446,0.084320,0.407414}%
\pgfsetstrokecolor{currentstroke}%
\pgfsetdash{}{0pt}%
\pgfpathmoveto{\pgfqpoint{7.438381in}{2.726587in}}%
\pgfpathlineto{\pgfqpoint{7.437590in}{2.726553in}}%
\pgfusepath{stroke}%
\end{pgfscope}%
\begin{pgfscope}%
\pgfpathrectangle{\pgfqpoint{6.720588in}{1.750000in}}{\pgfqpoint{2.279412in}{2.004545in}}%
\pgfusepath{clip}%
\pgfsetbuttcap%
\pgfsetroundjoin%
\pgfsetlinewidth{0.397071pt}%
\definecolor{currentstroke}{rgb}{0.280894,0.078907,0.402329}%
\pgfsetstrokecolor{currentstroke}%
\pgfsetdash{}{0pt}%
\pgfpathmoveto{\pgfqpoint{7.437590in}{2.726553in}}%
\pgfpathlineto{\pgfqpoint{7.437590in}{2.726553in}}%
\pgfusepath{stroke}%
\end{pgfscope}%
\begin{pgfscope}%
\pgfpathrectangle{\pgfqpoint{6.720588in}{1.750000in}}{\pgfqpoint{2.279412in}{2.004545in}}%
\pgfusepath{clip}%
\pgfsetbuttcap%
\pgfsetroundjoin%
\pgfsetlinewidth{0.397071pt}%
\definecolor{currentstroke}{rgb}{0.280894,0.078907,0.402329}%
\pgfsetstrokecolor{currentstroke}%
\pgfsetdash{}{0pt}%
\pgfpathmoveto{\pgfqpoint{7.437590in}{2.726553in}}%
\pgfpathlineto{\pgfqpoint{7.438146in}{2.726575in}}%
\pgfusepath{stroke}%
\end{pgfscope}%
\begin{pgfscope}%
\pgfpathrectangle{\pgfqpoint{6.720588in}{1.750000in}}{\pgfqpoint{2.279412in}{2.004545in}}%
\pgfusepath{clip}%
\pgfsetbuttcap%
\pgfsetroundjoin%
\pgfsetlinewidth{0.399618pt}%
\definecolor{currentstroke}{rgb}{0.281446,0.084320,0.407414}%
\pgfsetstrokecolor{currentstroke}%
\pgfsetdash{}{0pt}%
\pgfpathmoveto{\pgfqpoint{7.438146in}{2.726575in}}%
\pgfpathlineto{\pgfqpoint{7.438626in}{2.726595in}}%
\pgfusepath{stroke}%
\end{pgfscope}%
\begin{pgfscope}%
\pgfpathrectangle{\pgfqpoint{6.720588in}{1.750000in}}{\pgfqpoint{2.279412in}{2.004545in}}%
\pgfusepath{clip}%
\pgfsetbuttcap%
\pgfsetroundjoin%
\pgfsetlinewidth{0.401822pt}%
\definecolor{currentstroke}{rgb}{0.281446,0.084320,0.407414}%
\pgfsetstrokecolor{currentstroke}%
\pgfsetdash{}{0pt}%
\pgfpathmoveto{\pgfqpoint{7.438626in}{2.726595in}}%
\pgfpathlineto{\pgfqpoint{7.438678in}{2.726599in}}%
\pgfusepath{stroke}%
\end{pgfscope}%
\begin{pgfscope}%
\pgfpathrectangle{\pgfqpoint{6.720588in}{1.750000in}}{\pgfqpoint{2.279412in}{2.004545in}}%
\pgfusepath{clip}%
\pgfsetbuttcap%
\pgfsetroundjoin%
\pgfsetlinewidth{0.402061pt}%
\definecolor{currentstroke}{rgb}{0.281446,0.084320,0.407414}%
\pgfsetstrokecolor{currentstroke}%
\pgfsetdash{}{0pt}%
\pgfpathmoveto{\pgfqpoint{7.438678in}{2.726599in}}%
\pgfpathlineto{\pgfqpoint{7.438232in}{2.726581in}}%
\pgfusepath{stroke}%
\end{pgfscope}%
\begin{pgfscope}%
\pgfpathrectangle{\pgfqpoint{6.720588in}{1.750000in}}{\pgfqpoint{2.279412in}{2.004545in}}%
\pgfusepath{clip}%
\pgfsetbuttcap%
\pgfsetroundjoin%
\pgfsetlinewidth{0.400010pt}%
\definecolor{currentstroke}{rgb}{0.281446,0.084320,0.407414}%
\pgfsetstrokecolor{currentstroke}%
\pgfsetdash{}{0pt}%
\pgfpathmoveto{\pgfqpoint{7.438232in}{2.726581in}}%
\pgfpathlineto{\pgfqpoint{7.437575in}{2.726552in}}%
\pgfusepath{stroke}%
\end{pgfscope}%
\begin{pgfscope}%
\pgfpathrectangle{\pgfqpoint{6.720588in}{1.750000in}}{\pgfqpoint{2.279412in}{2.004545in}}%
\pgfusepath{clip}%
\pgfsetbuttcap%
\pgfsetroundjoin%
\pgfsetlinewidth{0.397003pt}%
\definecolor{currentstroke}{rgb}{0.280894,0.078907,0.402329}%
\pgfsetstrokecolor{currentstroke}%
\pgfsetdash{}{0pt}%
\pgfpathmoveto{\pgfqpoint{7.437575in}{2.726552in}}%
\pgfpathlineto{\pgfqpoint{7.437499in}{2.726547in}}%
\pgfusepath{stroke}%
\end{pgfscope}%
\begin{pgfscope}%
\pgfpathrectangle{\pgfqpoint{6.720588in}{1.750000in}}{\pgfqpoint{2.279412in}{2.004545in}}%
\pgfusepath{clip}%
\pgfsetbuttcap%
\pgfsetroundjoin%
\pgfsetlinewidth{0.396656pt}%
\definecolor{currentstroke}{rgb}{0.280894,0.078907,0.402329}%
\pgfsetstrokecolor{currentstroke}%
\pgfsetdash{}{0pt}%
\pgfpathmoveto{\pgfqpoint{7.437499in}{2.726547in}}%
\pgfpathlineto{\pgfqpoint{7.438092in}{2.726571in}}%
\pgfusepath{stroke}%
\end{pgfscope}%
\begin{pgfscope}%
\pgfpathrectangle{\pgfqpoint{6.720588in}{1.750000in}}{\pgfqpoint{2.279412in}{2.004545in}}%
\pgfusepath{clip}%
\pgfsetbuttcap%
\pgfsetroundjoin%
\pgfsetlinewidth{0.399370pt}%
\definecolor{currentstroke}{rgb}{0.281446,0.084320,0.407414}%
\pgfsetstrokecolor{currentstroke}%
\pgfsetdash{}{0pt}%
\pgfpathmoveto{\pgfqpoint{7.438092in}{2.726571in}}%
\pgfpathlineto{\pgfqpoint{7.438634in}{2.726595in}}%
\pgfusepath{stroke}%
\end{pgfscope}%
\begin{pgfscope}%
\pgfpathrectangle{\pgfqpoint{6.720588in}{1.750000in}}{\pgfqpoint{2.279412in}{2.004545in}}%
\pgfusepath{clip}%
\pgfsetbuttcap%
\pgfsetroundjoin%
\pgfsetlinewidth{0.401860pt}%
\definecolor{currentstroke}{rgb}{0.281446,0.084320,0.407414}%
\pgfsetstrokecolor{currentstroke}%
\pgfsetdash{}{0pt}%
\pgfpathmoveto{\pgfqpoint{7.438634in}{2.726595in}}%
\pgfpathlineto{\pgfqpoint{7.438729in}{2.726600in}}%
\pgfusepath{stroke}%
\end{pgfscope}%
\begin{pgfscope}%
\pgfpathrectangle{\pgfqpoint{6.720588in}{1.750000in}}{\pgfqpoint{2.279412in}{2.004545in}}%
\pgfusepath{clip}%
\pgfsetbuttcap%
\pgfsetroundjoin%
\pgfsetlinewidth{0.402297pt}%
\definecolor{currentstroke}{rgb}{0.281446,0.084320,0.407414}%
\pgfsetstrokecolor{currentstroke}%
\pgfsetdash{}{0pt}%
\pgfpathmoveto{\pgfqpoint{7.438729in}{2.726600in}}%
\pgfpathlineto{\pgfqpoint{7.438286in}{2.726583in}}%
\pgfusepath{stroke}%
\end{pgfscope}%
\begin{pgfscope}%
\pgfpathrectangle{\pgfqpoint{6.720588in}{1.750000in}}{\pgfqpoint{2.279412in}{2.004545in}}%
\pgfusepath{clip}%
\pgfsetbuttcap%
\pgfsetroundjoin%
\pgfsetlinewidth{0.400260pt}%
\definecolor{currentstroke}{rgb}{0.281446,0.084320,0.407414}%
\pgfsetstrokecolor{currentstroke}%
\pgfsetdash{}{0pt}%
\pgfpathmoveto{\pgfqpoint{7.438286in}{2.726583in}}%
\pgfpathlineto{\pgfqpoint{7.437555in}{2.726551in}}%
\pgfusepath{stroke}%
\end{pgfscope}%
\begin{pgfscope}%
\pgfpathrectangle{\pgfqpoint{6.720588in}{1.750000in}}{\pgfqpoint{2.279412in}{2.004545in}}%
\pgfusepath{clip}%
\pgfsetbuttcap%
\pgfsetroundjoin%
\pgfsetlinewidth{0.396910pt}%
\definecolor{currentstroke}{rgb}{0.280894,0.078907,0.402329}%
\pgfsetstrokecolor{currentstroke}%
\pgfsetdash{}{0pt}%
\pgfpathmoveto{\pgfqpoint{7.437555in}{2.726551in}}%
\pgfpathlineto{\pgfqpoint{7.437401in}{2.726542in}}%
\pgfusepath{stroke}%
\end{pgfscope}%
\begin{pgfscope}%
\pgfpathrectangle{\pgfqpoint{6.720588in}{1.750000in}}{\pgfqpoint{2.279412in}{2.004545in}}%
\pgfusepath{clip}%
\pgfsetbuttcap%
\pgfsetroundjoin%
\pgfsetlinewidth{0.396210pt}%
\definecolor{currentstroke}{rgb}{0.280894,0.078907,0.402329}%
\pgfsetstrokecolor{currentstroke}%
\pgfsetdash{}{0pt}%
\pgfpathmoveto{\pgfqpoint{7.437401in}{2.726542in}}%
\pgfpathlineto{\pgfqpoint{7.438025in}{2.726568in}}%
\pgfusepath{stroke}%
\end{pgfscope}%
\begin{pgfscope}%
\pgfpathrectangle{\pgfqpoint{6.720588in}{1.750000in}}{\pgfqpoint{2.279412in}{2.004545in}}%
\pgfusepath{clip}%
\pgfsetbuttcap%
\pgfsetroundjoin%
\pgfsetlinewidth{0.399064pt}%
\definecolor{currentstroke}{rgb}{0.281446,0.084320,0.407414}%
\pgfsetstrokecolor{currentstroke}%
\pgfsetdash{}{0pt}%
\pgfpathmoveto{\pgfqpoint{7.438025in}{2.726568in}}%
\pgfpathlineto{\pgfqpoint{7.438634in}{2.726594in}}%
\pgfusepath{stroke}%
\end{pgfscope}%
\begin{pgfscope}%
\pgfpathrectangle{\pgfqpoint{6.720588in}{1.750000in}}{\pgfqpoint{2.279412in}{2.004545in}}%
\pgfusepath{clip}%
\pgfsetbuttcap%
\pgfsetroundjoin%
\pgfsetlinewidth{0.401859pt}%
\definecolor{currentstroke}{rgb}{0.281446,0.084320,0.407414}%
\pgfsetstrokecolor{currentstroke}%
\pgfsetdash{}{0pt}%
\pgfpathmoveto{\pgfqpoint{7.438634in}{2.726594in}}%
\pgfpathlineto{\pgfqpoint{7.438778in}{2.726602in}}%
\pgfusepath{stroke}%
\end{pgfscope}%
\begin{pgfscope}%
\pgfpathrectangle{\pgfqpoint{6.720588in}{1.750000in}}{\pgfqpoint{2.279412in}{2.004545in}}%
\pgfusepath{clip}%
\pgfsetbuttcap%
\pgfsetroundjoin%
\pgfsetlinewidth{0.402524pt}%
\definecolor{currentstroke}{rgb}{0.281446,0.084320,0.407414}%
\pgfsetstrokecolor{currentstroke}%
\pgfsetdash{}{0pt}%
\pgfpathmoveto{\pgfqpoint{7.438778in}{2.726602in}}%
\pgfpathlineto{\pgfqpoint{7.438349in}{2.726585in}}%
\pgfusepath{stroke}%
\end{pgfscope}%
\begin{pgfscope}%
\pgfpathrectangle{\pgfqpoint{6.720588in}{1.750000in}}{\pgfqpoint{2.279412in}{2.004545in}}%
\pgfusepath{clip}%
\pgfsetbuttcap%
\pgfsetroundjoin%
\pgfsetlinewidth{0.400550pt}%
\definecolor{currentstroke}{rgb}{0.281446,0.084320,0.407414}%
\pgfsetstrokecolor{currentstroke}%
\pgfsetdash{}{0pt}%
\pgfpathmoveto{\pgfqpoint{7.438349in}{2.726585in}}%
\pgfpathlineto{\pgfqpoint{7.437548in}{2.726551in}}%
\pgfusepath{stroke}%
\end{pgfscope}%
\begin{pgfscope}%
\pgfpathrectangle{\pgfqpoint{6.720588in}{1.750000in}}{\pgfqpoint{2.279412in}{2.004545in}}%
\pgfusepath{clip}%
\pgfsetbuttcap%
\pgfsetroundjoin%
\pgfsetlinewidth{0.396880pt}%
\definecolor{currentstroke}{rgb}{0.280894,0.078907,0.402329}%
\pgfsetstrokecolor{currentstroke}%
\pgfsetdash{}{0pt}%
\pgfpathmoveto{\pgfqpoint{7.437548in}{2.726551in}}%
\pgfpathlineto{\pgfqpoint{7.437548in}{2.726551in}}%
\pgfusepath{stroke}%
\end{pgfscope}%
\begin{pgfscope}%
\pgfpathrectangle{\pgfqpoint{6.720588in}{1.750000in}}{\pgfqpoint{2.279412in}{2.004545in}}%
\pgfusepath{clip}%
\pgfsetbuttcap%
\pgfsetroundjoin%
\pgfsetlinewidth{0.396880pt}%
\definecolor{currentstroke}{rgb}{0.280894,0.078907,0.402329}%
\pgfsetstrokecolor{currentstroke}%
\pgfsetdash{}{0pt}%
\pgfpathmoveto{\pgfqpoint{7.437548in}{2.726551in}}%
\pgfpathlineto{\pgfqpoint{7.438143in}{2.726575in}}%
\pgfusepath{stroke}%
\end{pgfscope}%
\begin{pgfscope}%
\pgfpathrectangle{\pgfqpoint{6.720588in}{1.750000in}}{\pgfqpoint{2.279412in}{2.004545in}}%
\pgfusepath{clip}%
\pgfsetbuttcap%
\pgfsetroundjoin%
\pgfsetlinewidth{0.399605pt}%
\definecolor{currentstroke}{rgb}{0.281446,0.084320,0.407414}%
\pgfsetstrokecolor{currentstroke}%
\pgfsetdash{}{0pt}%
\pgfpathmoveto{\pgfqpoint{7.438143in}{2.726575in}}%
\pgfpathlineto{\pgfqpoint{7.438648in}{2.726596in}}%
\pgfusepath{stroke}%
\end{pgfscope}%
\begin{pgfscope}%
\pgfpathrectangle{\pgfqpoint{6.720588in}{1.750000in}}{\pgfqpoint{2.279412in}{2.004545in}}%
\pgfusepath{clip}%
\pgfsetbuttcap%
\pgfsetroundjoin%
\pgfsetlinewidth{0.401924pt}%
\definecolor{currentstroke}{rgb}{0.281446,0.084320,0.407414}%
\pgfsetstrokecolor{currentstroke}%
\pgfsetdash{}{0pt}%
\pgfpathmoveto{\pgfqpoint{7.438648in}{2.726596in}}%
\pgfpathlineto{\pgfqpoint{7.438702in}{2.726600in}}%
\pgfusepath{stroke}%
\end{pgfscope}%
\begin{pgfscope}%
\pgfpathrectangle{\pgfqpoint{6.720588in}{1.750000in}}{\pgfqpoint{2.279412in}{2.004545in}}%
\pgfusepath{clip}%
\pgfsetbuttcap%
\pgfsetroundjoin%
\pgfsetlinewidth{0.402175pt}%
\definecolor{currentstroke}{rgb}{0.281446,0.084320,0.407414}%
\pgfsetstrokecolor{currentstroke}%
\pgfsetdash{}{0pt}%
\pgfpathmoveto{\pgfqpoint{7.438702in}{2.726600in}}%
\pgfpathlineto{\pgfqpoint{7.438233in}{2.726581in}}%
\pgfusepath{stroke}%
\end{pgfscope}%
\begin{pgfscope}%
\pgfpathrectangle{\pgfqpoint{6.720588in}{1.750000in}}{\pgfqpoint{2.279412in}{2.004545in}}%
\pgfusepath{clip}%
\pgfsetbuttcap%
\pgfsetroundjoin%
\pgfsetlinewidth{0.400017pt}%
\definecolor{currentstroke}{rgb}{0.281446,0.084320,0.407414}%
\pgfsetstrokecolor{currentstroke}%
\pgfsetdash{}{0pt}%
\pgfpathmoveto{\pgfqpoint{7.438233in}{2.726581in}}%
\pgfpathlineto{\pgfqpoint{7.437532in}{2.726550in}}%
\pgfusepath{stroke}%
\end{pgfscope}%
\begin{pgfscope}%
\pgfpathrectangle{\pgfqpoint{6.720588in}{1.750000in}}{\pgfqpoint{2.279412in}{2.004545in}}%
\pgfusepath{clip}%
\pgfsetbuttcap%
\pgfsetroundjoin%
\pgfsetlinewidth{0.396808pt}%
\definecolor{currentstroke}{rgb}{0.280894,0.078907,0.402329}%
\pgfsetstrokecolor{currentstroke}%
\pgfsetdash{}{0pt}%
\pgfpathmoveto{\pgfqpoint{7.437532in}{2.726550in}}%
\pgfpathlineto{\pgfqpoint{7.437452in}{2.726545in}}%
\pgfusepath{stroke}%
\end{pgfscope}%
\begin{pgfscope}%
\pgfpathrectangle{\pgfqpoint{6.720588in}{1.750000in}}{\pgfqpoint{2.279412in}{2.004545in}}%
\pgfusepath{clip}%
\pgfsetbuttcap%
\pgfsetroundjoin%
\pgfsetlinewidth{0.396441pt}%
\definecolor{currentstroke}{rgb}{0.280894,0.078907,0.402329}%
\pgfsetstrokecolor{currentstroke}%
\pgfsetdash{}{0pt}%
\pgfpathmoveto{\pgfqpoint{7.437452in}{2.726545in}}%
\pgfpathlineto{\pgfqpoint{7.438084in}{2.726570in}}%
\pgfusepath{stroke}%
\end{pgfscope}%
\begin{pgfscope}%
\pgfpathrectangle{\pgfqpoint{6.720588in}{1.750000in}}{\pgfqpoint{2.279412in}{2.004545in}}%
\pgfusepath{clip}%
\pgfsetbuttcap%
\pgfsetroundjoin%
\pgfsetlinewidth{0.399335pt}%
\definecolor{currentstroke}{rgb}{0.281446,0.084320,0.407414}%
\pgfsetstrokecolor{currentstroke}%
\pgfsetdash{}{0pt}%
\pgfpathmoveto{\pgfqpoint{7.438084in}{2.726570in}}%
\pgfpathlineto{\pgfqpoint{7.438654in}{2.726595in}}%
\pgfusepath{stroke}%
\end{pgfscope}%
\begin{pgfscope}%
\pgfpathrectangle{\pgfqpoint{6.720588in}{1.750000in}}{\pgfqpoint{2.279412in}{2.004545in}}%
\pgfusepath{clip}%
\pgfsetbuttcap%
\pgfsetroundjoin%
\pgfsetlinewidth{0.401951pt}%
\definecolor{currentstroke}{rgb}{0.281446,0.084320,0.407414}%
\pgfsetstrokecolor{currentstroke}%
\pgfsetdash{}{0pt}%
\pgfpathmoveto{\pgfqpoint{7.438654in}{2.726595in}}%
\pgfpathlineto{\pgfqpoint{7.438754in}{2.726601in}}%
\pgfusepath{stroke}%
\end{pgfscope}%
\begin{pgfscope}%
\pgfpathrectangle{\pgfqpoint{6.720588in}{1.750000in}}{\pgfqpoint{2.279412in}{2.004545in}}%
\pgfusepath{clip}%
\pgfsetbuttcap%
\pgfsetroundjoin%
\pgfsetlinewidth{0.402415pt}%
\definecolor{currentstroke}{rgb}{0.281446,0.084320,0.407414}%
\pgfsetstrokecolor{currentstroke}%
\pgfsetdash{}{0pt}%
\pgfpathmoveto{\pgfqpoint{7.438754in}{2.726601in}}%
\pgfpathlineto{\pgfqpoint{7.438292in}{2.726583in}}%
\pgfusepath{stroke}%
\end{pgfscope}%
\begin{pgfscope}%
\pgfpathrectangle{\pgfqpoint{6.720588in}{1.750000in}}{\pgfqpoint{2.279412in}{2.004545in}}%
\pgfusepath{clip}%
\pgfsetbuttcap%
\pgfsetroundjoin%
\pgfsetlinewidth{0.400288pt}%
\definecolor{currentstroke}{rgb}{0.281446,0.084320,0.407414}%
\pgfsetstrokecolor{currentstroke}%
\pgfsetdash{}{0pt}%
\pgfpathmoveto{\pgfqpoint{7.438292in}{2.726583in}}%
\pgfpathlineto{\pgfqpoint{7.437515in}{2.726550in}}%
\pgfusepath{stroke}%
\end{pgfscope}%
\begin{pgfscope}%
\pgfpathrectangle{\pgfqpoint{6.720588in}{1.750000in}}{\pgfqpoint{2.279412in}{2.004545in}}%
\pgfusepath{clip}%
\pgfsetbuttcap%
\pgfsetroundjoin%
\pgfsetlinewidth{0.396730pt}%
\definecolor{currentstroke}{rgb}{0.280894,0.078907,0.402329}%
\pgfsetstrokecolor{currentstroke}%
\pgfsetdash{}{0pt}%
\pgfpathmoveto{\pgfqpoint{7.437515in}{2.726550in}}%
\pgfpathlineto{\pgfqpoint{7.437349in}{2.726540in}}%
\pgfusepath{stroke}%
\end{pgfscope}%
\begin{pgfscope}%
\pgfpathrectangle{\pgfqpoint{6.720588in}{1.750000in}}{\pgfqpoint{2.279412in}{2.004545in}}%
\pgfusepath{clip}%
\pgfsetbuttcap%
\pgfsetroundjoin%
\pgfsetlinewidth{0.395975pt}%
\definecolor{currentstroke}{rgb}{0.280894,0.078907,0.402329}%
\pgfsetstrokecolor{currentstroke}%
\pgfsetdash{}{0pt}%
\pgfpathmoveto{\pgfqpoint{7.437349in}{2.726540in}}%
\pgfpathlineto{\pgfqpoint{7.438012in}{2.726567in}}%
\pgfusepath{stroke}%
\end{pgfscope}%
\begin{pgfscope}%
\pgfpathrectangle{\pgfqpoint{6.720588in}{1.750000in}}{\pgfqpoint{2.279412in}{2.004545in}}%
\pgfusepath{clip}%
\pgfsetbuttcap%
\pgfsetroundjoin%
\pgfsetlinewidth{0.399004pt}%
\definecolor{currentstroke}{rgb}{0.281446,0.084320,0.407414}%
\pgfsetstrokecolor{currentstroke}%
\pgfsetdash{}{0pt}%
\pgfpathmoveto{\pgfqpoint{7.438012in}{2.726567in}}%
\pgfpathlineto{\pgfqpoint{7.438650in}{2.726595in}}%
\pgfusepath{stroke}%
\end{pgfscope}%
\begin{pgfscope}%
\pgfpathrectangle{\pgfqpoint{6.720588in}{1.750000in}}{\pgfqpoint{2.279412in}{2.004545in}}%
\pgfusepath{clip}%
\pgfsetbuttcap%
\pgfsetroundjoin%
\pgfsetlinewidth{0.401936pt}%
\definecolor{currentstroke}{rgb}{0.281446,0.084320,0.407414}%
\pgfsetstrokecolor{currentstroke}%
\pgfsetdash{}{0pt}%
\pgfpathmoveto{\pgfqpoint{7.438650in}{2.726595in}}%
\pgfpathlineto{\pgfqpoint{7.438804in}{2.726603in}}%
\pgfusepath{stroke}%
\end{pgfscope}%
\begin{pgfscope}%
\pgfpathrectangle{\pgfqpoint{6.720588in}{1.750000in}}{\pgfqpoint{2.279412in}{2.004545in}}%
\pgfusepath{clip}%
\pgfsetbuttcap%
\pgfsetroundjoin%
\pgfsetlinewidth{0.402642pt}%
\definecolor{currentstroke}{rgb}{0.281446,0.084320,0.407414}%
\pgfsetstrokecolor{currentstroke}%
\pgfsetdash{}{0pt}%
\pgfpathmoveto{\pgfqpoint{7.438804in}{2.726603in}}%
\pgfpathlineto{\pgfqpoint{7.438360in}{2.726586in}}%
\pgfusepath{stroke}%
\end{pgfscope}%
\begin{pgfscope}%
\pgfpathrectangle{\pgfqpoint{6.720588in}{1.750000in}}{\pgfqpoint{2.279412in}{2.004545in}}%
\pgfusepath{clip}%
\pgfsetbuttcap%
\pgfsetroundjoin%
\pgfsetlinewidth{0.400601pt}%
\definecolor{currentstroke}{rgb}{0.281446,0.084320,0.407414}%
\pgfsetstrokecolor{currentstroke}%
\pgfsetdash{}{0pt}%
\pgfpathmoveto{\pgfqpoint{7.438360in}{2.726586in}}%
\pgfpathlineto{\pgfqpoint{7.437514in}{2.726550in}}%
\pgfusepath{stroke}%
\end{pgfscope}%
\begin{pgfscope}%
\pgfpathrectangle{\pgfqpoint{6.720588in}{1.750000in}}{\pgfqpoint{2.279412in}{2.004545in}}%
\pgfusepath{clip}%
\pgfsetbuttcap%
\pgfsetroundjoin%
\pgfsetlinewidth{0.396724pt}%
\definecolor{currentstroke}{rgb}{0.280894,0.078907,0.402329}%
\pgfsetstrokecolor{currentstroke}%
\pgfsetdash{}{0pt}%
\pgfpathmoveto{\pgfqpoint{7.437514in}{2.726550in}}%
\pgfpathlineto{\pgfqpoint{7.437514in}{2.726550in}}%
\pgfusepath{stroke}%
\end{pgfscope}%
\begin{pgfscope}%
\pgfpathrectangle{\pgfqpoint{6.720588in}{1.750000in}}{\pgfqpoint{2.279412in}{2.004545in}}%
\pgfusepath{clip}%
\pgfsetbuttcap%
\pgfsetroundjoin%
\pgfsetlinewidth{0.396724pt}%
\definecolor{currentstroke}{rgb}{0.280894,0.078907,0.402329}%
\pgfsetstrokecolor{currentstroke}%
\pgfsetdash{}{0pt}%
\pgfpathmoveto{\pgfqpoint{7.437514in}{2.726550in}}%
\pgfpathlineto{\pgfqpoint{7.438139in}{2.726574in}}%
\pgfusepath{stroke}%
\end{pgfscope}%
\begin{pgfscope}%
\pgfpathrectangle{\pgfqpoint{6.720588in}{1.750000in}}{\pgfqpoint{2.279412in}{2.004545in}}%
\pgfusepath{clip}%
\pgfsetbuttcap%
\pgfsetroundjoin%
\pgfsetlinewidth{0.399583pt}%
\definecolor{currentstroke}{rgb}{0.281446,0.084320,0.407414}%
\pgfsetstrokecolor{currentstroke}%
\pgfsetdash{}{0pt}%
\pgfpathmoveto{\pgfqpoint{7.438139in}{2.726574in}}%
\pgfpathlineto{\pgfqpoint{7.438663in}{2.726597in}}%
\pgfusepath{stroke}%
\end{pgfscope}%
\begin{pgfscope}%
\pgfpathrectangle{\pgfqpoint{6.720588in}{1.750000in}}{\pgfqpoint{2.279412in}{2.004545in}}%
\pgfusepath{clip}%
\pgfsetbuttcap%
\pgfsetroundjoin%
\pgfsetlinewidth{0.401995pt}%
\definecolor{currentstroke}{rgb}{0.281446,0.084320,0.407414}%
\pgfsetstrokecolor{currentstroke}%
\pgfsetdash{}{0pt}%
\pgfpathmoveto{\pgfqpoint{7.438663in}{2.726597in}}%
\pgfpathlineto{\pgfqpoint{7.438722in}{2.726601in}}%
\pgfusepath{stroke}%
\end{pgfscope}%
\begin{pgfscope}%
\pgfpathrectangle{\pgfqpoint{6.720588in}{1.750000in}}{\pgfqpoint{2.279412in}{2.004545in}}%
\pgfusepath{clip}%
\pgfsetbuttcap%
\pgfsetroundjoin%
\pgfsetlinewidth{0.402265pt}%
\definecolor{currentstroke}{rgb}{0.281446,0.084320,0.407414}%
\pgfsetstrokecolor{currentstroke}%
\pgfsetdash{}{0pt}%
\pgfpathmoveto{\pgfqpoint{7.438722in}{2.726601in}}%
\pgfpathlineto{\pgfqpoint{7.438237in}{2.726581in}}%
\pgfusepath{stroke}%
\end{pgfscope}%
\begin{pgfscope}%
\pgfpathrectangle{\pgfqpoint{6.720588in}{1.750000in}}{\pgfqpoint{2.279412in}{2.004545in}}%
\pgfusepath{clip}%
\pgfsetbuttcap%
\pgfsetroundjoin%
\pgfsetlinewidth{0.400034pt}%
\definecolor{currentstroke}{rgb}{0.281446,0.084320,0.407414}%
\pgfsetstrokecolor{currentstroke}%
\pgfsetdash{}{0pt}%
\pgfpathmoveto{\pgfqpoint{7.438237in}{2.726581in}}%
\pgfpathlineto{\pgfqpoint{7.437501in}{2.726549in}}%
\pgfusepath{stroke}%
\end{pgfscope}%
\begin{pgfscope}%
\pgfpathrectangle{\pgfqpoint{6.720588in}{1.750000in}}{\pgfqpoint{2.279412in}{2.004545in}}%
\pgfusepath{clip}%
\pgfsetbuttcap%
\pgfsetroundjoin%
\pgfsetlinewidth{0.396667pt}%
\definecolor{currentstroke}{rgb}{0.280894,0.078907,0.402329}%
\pgfsetstrokecolor{currentstroke}%
\pgfsetdash{}{0pt}%
\pgfpathmoveto{\pgfqpoint{7.437501in}{2.726549in}}%
\pgfpathlineto{\pgfqpoint{7.437414in}{2.726543in}}%
\pgfusepath{stroke}%
\end{pgfscope}%
\begin{pgfscope}%
\pgfpathrectangle{\pgfqpoint{6.720588in}{1.750000in}}{\pgfqpoint{2.279412in}{2.004545in}}%
\pgfusepath{clip}%
\pgfsetbuttcap%
\pgfsetroundjoin%
\pgfsetlinewidth{0.396269pt}%
\definecolor{currentstroke}{rgb}{0.280894,0.078907,0.402329}%
\pgfsetstrokecolor{currentstroke}%
\pgfsetdash{}{0pt}%
\pgfpathmoveto{\pgfqpoint{7.437414in}{2.726543in}}%
\pgfpathlineto{\pgfqpoint{7.438075in}{2.726570in}}%
\pgfusepath{stroke}%
\end{pgfscope}%
\begin{pgfscope}%
\pgfpathrectangle{\pgfqpoint{6.720588in}{1.750000in}}{\pgfqpoint{2.279412in}{2.004545in}}%
\pgfusepath{clip}%
\pgfsetbuttcap%
\pgfsetroundjoin%
\pgfsetlinewidth{0.399294pt}%
\definecolor{currentstroke}{rgb}{0.281446,0.084320,0.407414}%
\pgfsetstrokecolor{currentstroke}%
\pgfsetdash{}{0pt}%
\pgfpathmoveto{\pgfqpoint{7.438075in}{2.726570in}}%
\pgfpathlineto{\pgfqpoint{7.438667in}{2.726596in}}%
\pgfusepath{stroke}%
\end{pgfscope}%
\begin{pgfscope}%
\pgfpathrectangle{\pgfqpoint{6.720588in}{1.750000in}}{\pgfqpoint{2.279412in}{2.004545in}}%
\pgfusepath{clip}%
\pgfsetbuttcap%
\pgfsetroundjoin%
\pgfsetlinewidth{0.402012pt}%
\definecolor{currentstroke}{rgb}{0.281446,0.084320,0.407414}%
\pgfsetstrokecolor{currentstroke}%
\pgfsetdash{}{0pt}%
\pgfpathmoveto{\pgfqpoint{7.438667in}{2.726596in}}%
\pgfpathlineto{\pgfqpoint{7.438774in}{2.726602in}}%
\pgfusepath{stroke}%
\end{pgfscope}%
\begin{pgfscope}%
\pgfpathrectangle{\pgfqpoint{6.720588in}{1.750000in}}{\pgfqpoint{2.279412in}{2.004545in}}%
\pgfusepath{clip}%
\pgfsetbuttcap%
\pgfsetroundjoin%
\pgfsetlinewidth{0.402505pt}%
\definecolor{currentstroke}{rgb}{0.281446,0.084320,0.407414}%
\pgfsetstrokecolor{currentstroke}%
\pgfsetdash{}{0pt}%
\pgfpathmoveto{\pgfqpoint{7.438774in}{2.726602in}}%
\pgfpathlineto{\pgfqpoint{7.438300in}{2.726583in}}%
\pgfusepath{stroke}%
\end{pgfscope}%
\begin{pgfscope}%
\pgfpathrectangle{\pgfqpoint{6.720588in}{1.750000in}}{\pgfqpoint{2.279412in}{2.004545in}}%
\pgfusepath{clip}%
\pgfsetbuttcap%
\pgfsetroundjoin%
\pgfsetlinewidth{0.400323pt}%
\definecolor{currentstroke}{rgb}{0.281446,0.084320,0.407414}%
\pgfsetstrokecolor{currentstroke}%
\pgfsetdash{}{0pt}%
\pgfpathmoveto{\pgfqpoint{7.438300in}{2.726583in}}%
\pgfpathlineto{\pgfqpoint{7.437488in}{2.726548in}}%
\pgfusepath{stroke}%
\end{pgfscope}%
\begin{pgfscope}%
\pgfpathrectangle{\pgfqpoint{6.720588in}{1.750000in}}{\pgfqpoint{2.279412in}{2.004545in}}%
\pgfusepath{clip}%
\pgfsetbuttcap%
\pgfsetroundjoin%
\pgfsetlinewidth{0.396605pt}%
\definecolor{currentstroke}{rgb}{0.280894,0.078907,0.402329}%
\pgfsetstrokecolor{currentstroke}%
\pgfsetdash{}{0pt}%
\pgfpathmoveto{\pgfqpoint{7.437488in}{2.726548in}}%
\pgfpathlineto{\pgfqpoint{7.437488in}{2.726548in}}%
\pgfusepath{stroke}%
\end{pgfscope}%
\begin{pgfscope}%
\pgfpathrectangle{\pgfqpoint{6.720588in}{1.750000in}}{\pgfqpoint{2.279412in}{2.004545in}}%
\pgfusepath{clip}%
\pgfsetbuttcap%
\pgfsetroundjoin%
\pgfsetlinewidth{0.396605pt}%
\definecolor{currentstroke}{rgb}{0.280894,0.078907,0.402329}%
\pgfsetstrokecolor{currentstroke}%
\pgfsetdash{}{0pt}%
\pgfpathmoveto{\pgfqpoint{7.437488in}{2.726548in}}%
\pgfpathlineto{\pgfqpoint{7.438139in}{2.726574in}}%
\pgfusepath{stroke}%
\end{pgfscope}%
\begin{pgfscope}%
\pgfpathrectangle{\pgfqpoint{6.720588in}{1.750000in}}{\pgfqpoint{2.279412in}{2.004545in}}%
\pgfusepath{clip}%
\pgfsetbuttcap%
\pgfsetroundjoin%
\pgfsetlinewidth{0.399584pt}%
\definecolor{currentstroke}{rgb}{0.281446,0.084320,0.407414}%
\pgfsetstrokecolor{currentstroke}%
\pgfsetdash{}{0pt}%
\pgfpathmoveto{\pgfqpoint{7.438139in}{2.726574in}}%
\pgfpathlineto{\pgfqpoint{7.438678in}{2.726597in}}%
\pgfusepath{stroke}%
\end{pgfscope}%
\begin{pgfscope}%
\pgfpathrectangle{\pgfqpoint{6.720588in}{1.750000in}}{\pgfqpoint{2.279412in}{2.004545in}}%
\pgfusepath{clip}%
\pgfsetbuttcap%
\pgfsetroundjoin%
\pgfsetlinewidth{0.402063pt}%
\definecolor{currentstroke}{rgb}{0.281446,0.084320,0.407414}%
\pgfsetstrokecolor{currentstroke}%
\pgfsetdash{}{0pt}%
\pgfpathmoveto{\pgfqpoint{7.438678in}{2.726597in}}%
\pgfpathlineto{\pgfqpoint{7.438737in}{2.726601in}}%
\pgfusepath{stroke}%
\end{pgfscope}%
\begin{pgfscope}%
\pgfpathrectangle{\pgfqpoint{6.720588in}{1.750000in}}{\pgfqpoint{2.279412in}{2.004545in}}%
\pgfusepath{clip}%
\pgfsetbuttcap%
\pgfsetroundjoin%
\pgfsetlinewidth{0.402333pt}%
\definecolor{currentstroke}{rgb}{0.281446,0.084320,0.407414}%
\pgfsetstrokecolor{currentstroke}%
\pgfsetdash{}{0pt}%
\pgfpathmoveto{\pgfqpoint{7.438737in}{2.726601in}}%
\pgfpathlineto{\pgfqpoint{7.438235in}{2.726581in}}%
\pgfusepath{stroke}%
\end{pgfscope}%
\begin{pgfscope}%
\pgfpathrectangle{\pgfqpoint{6.720588in}{1.750000in}}{\pgfqpoint{2.279412in}{2.004545in}}%
\pgfusepath{clip}%
\pgfsetbuttcap%
\pgfsetroundjoin%
\pgfsetlinewidth{0.400027pt}%
\definecolor{currentstroke}{rgb}{0.281446,0.084320,0.407414}%
\pgfsetstrokecolor{currentstroke}%
\pgfsetdash{}{0pt}%
\pgfpathmoveto{\pgfqpoint{7.438235in}{2.726581in}}%
\pgfpathlineto{\pgfqpoint{7.437472in}{2.726548in}}%
\pgfusepath{stroke}%
\end{pgfscope}%
\begin{pgfscope}%
\pgfpathrectangle{\pgfqpoint{6.720588in}{1.750000in}}{\pgfqpoint{2.279412in}{2.004545in}}%
\pgfusepath{clip}%
\pgfsetbuttcap%
\pgfsetroundjoin%
\pgfsetlinewidth{0.396531pt}%
\definecolor{currentstroke}{rgb}{0.280894,0.078907,0.402329}%
\pgfsetstrokecolor{currentstroke}%
\pgfsetdash{}{0pt}%
\pgfpathmoveto{\pgfqpoint{7.437472in}{2.726548in}}%
\pgfpathlineto{\pgfqpoint{7.437385in}{2.726542in}}%
\pgfusepath{stroke}%
\end{pgfscope}%
\begin{pgfscope}%
\pgfpathrectangle{\pgfqpoint{6.720588in}{1.750000in}}{\pgfqpoint{2.279412in}{2.004545in}}%
\pgfusepath{clip}%
\pgfsetbuttcap%
\pgfsetroundjoin%
\pgfsetlinewidth{0.396135pt}%
\definecolor{currentstroke}{rgb}{0.280894,0.078907,0.402329}%
\pgfsetstrokecolor{currentstroke}%
\pgfsetdash{}{0pt}%
\pgfpathmoveto{\pgfqpoint{7.437385in}{2.726542in}}%
\pgfpathlineto{\pgfqpoint{7.438073in}{2.726570in}}%
\pgfusepath{stroke}%
\end{pgfscope}%
\begin{pgfscope}%
\pgfpathrectangle{\pgfqpoint{6.720588in}{1.750000in}}{\pgfqpoint{2.279412in}{2.004545in}}%
\pgfusepath{clip}%
\pgfsetbuttcap%
\pgfsetroundjoin%
\pgfsetlinewidth{0.399281pt}%
\definecolor{currentstroke}{rgb}{0.281446,0.084320,0.407414}%
\pgfsetstrokecolor{currentstroke}%
\pgfsetdash{}{0pt}%
\pgfpathmoveto{\pgfqpoint{7.438073in}{2.726570in}}%
\pgfpathlineto{\pgfqpoint{7.438680in}{2.726596in}}%
\pgfusepath{stroke}%
\end{pgfscope}%
\begin{pgfscope}%
\pgfpathrectangle{\pgfqpoint{6.720588in}{1.750000in}}{\pgfqpoint{2.279412in}{2.004545in}}%
\pgfusepath{clip}%
\pgfsetbuttcap%
\pgfsetroundjoin%
\pgfsetlinewidth{0.402072pt}%
\definecolor{currentstroke}{rgb}{0.281446,0.084320,0.407414}%
\pgfsetstrokecolor{currentstroke}%
\pgfsetdash{}{0pt}%
\pgfpathmoveto{\pgfqpoint{7.438680in}{2.726596in}}%
\pgfpathlineto{\pgfqpoint{7.438789in}{2.726603in}}%
\pgfusepath{stroke}%
\end{pgfscope}%
\begin{pgfscope}%
\pgfpathrectangle{\pgfqpoint{6.720588in}{1.750000in}}{\pgfqpoint{2.279412in}{2.004545in}}%
\pgfusepath{clip}%
\pgfsetbuttcap%
\pgfsetroundjoin%
\pgfsetlinewidth{0.402574pt}%
\definecolor{currentstroke}{rgb}{0.281446,0.084320,0.407414}%
\pgfsetstrokecolor{currentstroke}%
\pgfsetdash{}{0pt}%
\pgfpathmoveto{\pgfqpoint{7.438789in}{2.726603in}}%
\pgfpathlineto{\pgfqpoint{7.438302in}{2.726584in}}%
\pgfusepath{stroke}%
\end{pgfscope}%
\begin{pgfscope}%
\pgfpathrectangle{\pgfqpoint{6.720588in}{1.750000in}}{\pgfqpoint{2.279412in}{2.004545in}}%
\pgfusepath{clip}%
\pgfsetbuttcap%
\pgfsetroundjoin%
\pgfsetlinewidth{0.400332pt}%
\definecolor{currentstroke}{rgb}{0.281446,0.084320,0.407414}%
\pgfsetstrokecolor{currentstroke}%
\pgfsetdash{}{0pt}%
\pgfpathmoveto{\pgfqpoint{7.438302in}{2.726584in}}%
\pgfpathlineto{\pgfqpoint{7.437461in}{2.726547in}}%
\pgfusepath{stroke}%
\end{pgfscope}%
\begin{pgfscope}%
\pgfpathrectangle{\pgfqpoint{6.720588in}{1.750000in}}{\pgfqpoint{2.279412in}{2.004545in}}%
\pgfusepath{clip}%
\pgfsetbuttcap%
\pgfsetroundjoin%
\pgfsetlinewidth{0.396481pt}%
\definecolor{currentstroke}{rgb}{0.280894,0.078907,0.402329}%
\pgfsetstrokecolor{currentstroke}%
\pgfsetdash{}{0pt}%
\pgfpathmoveto{\pgfqpoint{7.437461in}{2.726547in}}%
\pgfpathlineto{\pgfqpoint{7.437461in}{2.726547in}}%
\pgfusepath{stroke}%
\end{pgfscope}%
\begin{pgfscope}%
\pgfpathrectangle{\pgfqpoint{6.720588in}{1.750000in}}{\pgfqpoint{2.279412in}{2.004545in}}%
\pgfusepath{clip}%
\pgfsetbuttcap%
\pgfsetroundjoin%
\pgfsetlinewidth{0.396481pt}%
\definecolor{currentstroke}{rgb}{0.280894,0.078907,0.402329}%
\pgfsetstrokecolor{currentstroke}%
\pgfsetdash{}{0pt}%
\pgfpathmoveto{\pgfqpoint{7.437461in}{2.726547in}}%
\pgfpathlineto{\pgfqpoint{7.438135in}{2.726574in}}%
\pgfusepath{stroke}%
\end{pgfscope}%
\begin{pgfscope}%
\pgfpathrectangle{\pgfqpoint{6.720588in}{1.750000in}}{\pgfqpoint{2.279412in}{2.004545in}}%
\pgfusepath{clip}%
\pgfsetbuttcap%
\pgfsetroundjoin%
\pgfsetlinewidth{0.399568pt}%
\definecolor{currentstroke}{rgb}{0.281446,0.084320,0.407414}%
\pgfsetstrokecolor{currentstroke}%
\pgfsetdash{}{0pt}%
\pgfpathmoveto{\pgfqpoint{7.438135in}{2.726574in}}%
\pgfpathlineto{\pgfqpoint{7.438690in}{2.726598in}}%
\pgfusepath{stroke}%
\end{pgfscope}%
\begin{pgfscope}%
\pgfpathrectangle{\pgfqpoint{6.720588in}{1.750000in}}{\pgfqpoint{2.279412in}{2.004545in}}%
\pgfusepath{clip}%
\pgfsetbuttcap%
\pgfsetroundjoin%
\pgfsetlinewidth{0.402118pt}%
\definecolor{currentstroke}{rgb}{0.281446,0.084320,0.407414}%
\pgfsetstrokecolor{currentstroke}%
\pgfsetdash{}{0pt}%
\pgfpathmoveto{\pgfqpoint{7.438690in}{2.726598in}}%
\pgfpathlineto{\pgfqpoint{7.438751in}{2.726602in}}%
\pgfusepath{stroke}%
\end{pgfscope}%
\begin{pgfscope}%
\pgfpathrectangle{\pgfqpoint{6.720588in}{1.750000in}}{\pgfqpoint{2.279412in}{2.004545in}}%
\pgfusepath{clip}%
\pgfsetbuttcap%
\pgfsetroundjoin%
\pgfsetlinewidth{0.402400pt}%
\definecolor{currentstroke}{rgb}{0.281446,0.084320,0.407414}%
\pgfsetstrokecolor{currentstroke}%
\pgfsetdash{}{0pt}%
\pgfpathmoveto{\pgfqpoint{7.438751in}{2.726602in}}%
\pgfpathlineto{\pgfqpoint{7.438238in}{2.726581in}}%
\pgfusepath{stroke}%
\end{pgfscope}%
\begin{pgfscope}%
\pgfpathrectangle{\pgfqpoint{6.720588in}{1.750000in}}{\pgfqpoint{2.279412in}{2.004545in}}%
\pgfusepath{clip}%
\pgfsetbuttcap%
\pgfsetroundjoin%
\pgfsetlinewidth{0.400040pt}%
\definecolor{currentstroke}{rgb}{0.281446,0.084320,0.407414}%
\pgfsetstrokecolor{currentstroke}%
\pgfsetdash{}{0pt}%
\pgfpathmoveto{\pgfqpoint{7.438238in}{2.726581in}}%
\pgfpathlineto{\pgfqpoint{7.437447in}{2.726547in}}%
\pgfusepath{stroke}%
\end{pgfscope}%
\begin{pgfscope}%
\pgfpathrectangle{\pgfqpoint{6.720588in}{1.750000in}}{\pgfqpoint{2.279412in}{2.004545in}}%
\pgfusepath{clip}%
\pgfsetbuttcap%
\pgfsetroundjoin%
\pgfsetlinewidth{0.396419pt}%
\definecolor{currentstroke}{rgb}{0.280894,0.078907,0.402329}%
\pgfsetstrokecolor{currentstroke}%
\pgfsetdash{}{0pt}%
\pgfpathmoveto{\pgfqpoint{7.437447in}{2.726547in}}%
\pgfpathlineto{\pgfqpoint{7.437355in}{2.726540in}}%
\pgfusepath{stroke}%
\end{pgfscope}%
\begin{pgfscope}%
\pgfpathrectangle{\pgfqpoint{6.720588in}{1.750000in}}{\pgfqpoint{2.279412in}{2.004545in}}%
\pgfusepath{clip}%
\pgfsetbuttcap%
\pgfsetroundjoin%
\pgfsetlinewidth{0.395999pt}%
\definecolor{currentstroke}{rgb}{0.280894,0.078907,0.402329}%
\pgfsetstrokecolor{currentstroke}%
\pgfsetdash{}{0pt}%
\pgfpathmoveto{\pgfqpoint{7.437355in}{2.726540in}}%
\pgfpathlineto{\pgfqpoint{7.438066in}{2.726569in}}%
\pgfusepath{stroke}%
\end{pgfscope}%
\begin{pgfscope}%
\pgfpathrectangle{\pgfqpoint{6.720588in}{1.750000in}}{\pgfqpoint{2.279412in}{2.004545in}}%
\pgfusepath{clip}%
\pgfsetbuttcap%
\pgfsetroundjoin%
\pgfsetlinewidth{0.399249pt}%
\definecolor{currentstroke}{rgb}{0.281446,0.084320,0.407414}%
\pgfsetstrokecolor{currentstroke}%
\pgfsetdash{}{0pt}%
\pgfpathmoveto{\pgfqpoint{7.438066in}{2.726569in}}%
\pgfpathlineto{\pgfqpoint{7.438690in}{2.726596in}}%
\pgfusepath{stroke}%
\end{pgfscope}%
\begin{pgfscope}%
\pgfpathrectangle{\pgfqpoint{6.720588in}{1.750000in}}{\pgfqpoint{2.279412in}{2.004545in}}%
\pgfusepath{clip}%
\pgfsetbuttcap%
\pgfsetroundjoin%
\pgfsetlinewidth{0.402117pt}%
\definecolor{currentstroke}{rgb}{0.281446,0.084320,0.407414}%
\pgfsetstrokecolor{currentstroke}%
\pgfsetdash{}{0pt}%
\pgfpathmoveto{\pgfqpoint{7.438690in}{2.726596in}}%
\pgfpathlineto{\pgfqpoint{7.438804in}{2.726603in}}%
\pgfusepath{stroke}%
\end{pgfscope}%
\begin{pgfscope}%
\pgfpathrectangle{\pgfqpoint{6.720588in}{1.750000in}}{\pgfqpoint{2.279412in}{2.004545in}}%
\pgfusepath{clip}%
\pgfsetbuttcap%
\pgfsetroundjoin%
\pgfsetlinewidth{0.402642pt}%
\definecolor{currentstroke}{rgb}{0.281446,0.084320,0.407414}%
\pgfsetstrokecolor{currentstroke}%
\pgfsetdash{}{0pt}%
\pgfpathmoveto{\pgfqpoint{7.438804in}{2.726603in}}%
\pgfpathlineto{\pgfqpoint{7.438308in}{2.726584in}}%
\pgfusepath{stroke}%
\end{pgfscope}%
\begin{pgfscope}%
\pgfpathrectangle{\pgfqpoint{6.720588in}{1.750000in}}{\pgfqpoint{2.279412in}{2.004545in}}%
\pgfusepath{clip}%
\pgfsetbuttcap%
\pgfsetroundjoin%
\pgfsetlinewidth{0.400359pt}%
\definecolor{currentstroke}{rgb}{0.281446,0.084320,0.407414}%
\pgfsetstrokecolor{currentstroke}%
\pgfsetdash{}{0pt}%
\pgfpathmoveto{\pgfqpoint{7.438308in}{2.726584in}}%
\pgfpathlineto{\pgfqpoint{7.437440in}{2.726546in}}%
\pgfusepath{stroke}%
\end{pgfscope}%
\begin{pgfscope}%
\pgfpathrectangle{\pgfqpoint{6.720588in}{1.750000in}}{\pgfqpoint{2.279412in}{2.004545in}}%
\pgfusepath{clip}%
\pgfsetbuttcap%
\pgfsetroundjoin%
\pgfsetlinewidth{0.396385pt}%
\definecolor{currentstroke}{rgb}{0.280894,0.078907,0.402329}%
\pgfsetstrokecolor{currentstroke}%
\pgfsetdash{}{0pt}%
\pgfpathmoveto{\pgfqpoint{7.437440in}{2.726546in}}%
\pgfpathlineto{\pgfqpoint{7.437440in}{2.726546in}}%
\pgfusepath{stroke}%
\end{pgfscope}%
\begin{pgfscope}%
\pgfpathrectangle{\pgfqpoint{6.720588in}{1.750000in}}{\pgfqpoint{2.279412in}{2.004545in}}%
\pgfusepath{clip}%
\pgfsetbuttcap%
\pgfsetroundjoin%
\pgfsetlinewidth{0.396385pt}%
\definecolor{currentstroke}{rgb}{0.280894,0.078907,0.402329}%
\pgfsetstrokecolor{currentstroke}%
\pgfsetdash{}{0pt}%
\pgfpathmoveto{\pgfqpoint{7.437440in}{2.726546in}}%
\pgfpathlineto{\pgfqpoint{7.438132in}{2.726574in}}%
\pgfusepath{stroke}%
\end{pgfscope}%
\begin{pgfscope}%
\pgfpathrectangle{\pgfqpoint{6.720588in}{1.750000in}}{\pgfqpoint{2.279412in}{2.004545in}}%
\pgfusepath{clip}%
\pgfsetbuttcap%
\pgfsetroundjoin%
\pgfsetlinewidth{0.399554pt}%
\definecolor{currentstroke}{rgb}{0.281446,0.084320,0.407414}%
\pgfsetstrokecolor{currentstroke}%
\pgfsetdash{}{0pt}%
\pgfpathmoveto{\pgfqpoint{7.438132in}{2.726574in}}%
\pgfpathlineto{\pgfqpoint{7.438699in}{2.726598in}}%
\pgfusepath{stroke}%
\end{pgfscope}%
\begin{pgfscope}%
\pgfpathrectangle{\pgfqpoint{6.720588in}{1.750000in}}{\pgfqpoint{2.279412in}{2.004545in}}%
\pgfusepath{clip}%
\pgfsetbuttcap%
\pgfsetroundjoin%
\pgfsetlinewidth{0.402158pt}%
\definecolor{currentstroke}{rgb}{0.281446,0.084320,0.407414}%
\pgfsetstrokecolor{currentstroke}%
\pgfsetdash{}{0pt}%
\pgfpathmoveto{\pgfqpoint{7.438699in}{2.726598in}}%
\pgfpathlineto{\pgfqpoint{7.438763in}{2.726602in}}%
\pgfusepath{stroke}%
\end{pgfscope}%
\begin{pgfscope}%
\pgfpathrectangle{\pgfqpoint{6.720588in}{1.750000in}}{\pgfqpoint{2.279412in}{2.004545in}}%
\pgfusepath{clip}%
\pgfsetbuttcap%
\pgfsetroundjoin%
\pgfsetlinewidth{0.402452pt}%
\definecolor{currentstroke}{rgb}{0.281446,0.084320,0.407414}%
\pgfsetstrokecolor{currentstroke}%
\pgfsetdash{}{0pt}%
\pgfpathmoveto{\pgfqpoint{7.438763in}{2.726602in}}%
\pgfpathlineto{\pgfqpoint{7.438241in}{2.726581in}}%
\pgfusepath{stroke}%
\end{pgfscope}%
\begin{pgfscope}%
\pgfpathrectangle{\pgfqpoint{6.720588in}{1.750000in}}{\pgfqpoint{2.279412in}{2.004545in}}%
\pgfusepath{clip}%
\pgfsetbuttcap%
\pgfsetroundjoin%
\pgfsetlinewidth{0.400051pt}%
\definecolor{currentstroke}{rgb}{0.281446,0.084320,0.407414}%
\pgfsetstrokecolor{currentstroke}%
\pgfsetdash{}{0pt}%
\pgfpathmoveto{\pgfqpoint{7.438241in}{2.726581in}}%
\pgfpathlineto{\pgfqpoint{7.437429in}{2.726546in}}%
\pgfusepath{stroke}%
\end{pgfscope}%
\begin{pgfscope}%
\pgfpathrectangle{\pgfqpoint{6.720588in}{1.750000in}}{\pgfqpoint{2.279412in}{2.004545in}}%
\pgfusepath{clip}%
\pgfsetbuttcap%
\pgfsetroundjoin%
\pgfsetlinewidth{0.396334pt}%
\definecolor{currentstroke}{rgb}{0.280894,0.078907,0.402329}%
\pgfsetstrokecolor{currentstroke}%
\pgfsetdash{}{0pt}%
\pgfpathmoveto{\pgfqpoint{7.437429in}{2.726546in}}%
\pgfpathlineto{\pgfqpoint{7.437429in}{2.726546in}}%
\pgfusepath{stroke}%
\end{pgfscope}%
\begin{pgfscope}%
\pgfpathrectangle{\pgfqpoint{6.720588in}{1.750000in}}{\pgfqpoint{2.279412in}{2.004545in}}%
\pgfusepath{clip}%
\pgfsetbuttcap%
\pgfsetroundjoin%
\pgfsetlinewidth{0.396334pt}%
\definecolor{currentstroke}{rgb}{0.280894,0.078907,0.402329}%
\pgfsetstrokecolor{currentstroke}%
\pgfsetdash{}{0pt}%
\pgfpathmoveto{\pgfqpoint{7.437429in}{2.726546in}}%
\pgfpathlineto{\pgfqpoint{7.438134in}{2.726574in}}%
\pgfusepath{stroke}%
\end{pgfscope}%
\begin{pgfscope}%
\pgfpathrectangle{\pgfqpoint{6.720588in}{1.750000in}}{\pgfqpoint{2.279412in}{2.004545in}}%
\pgfusepath{clip}%
\pgfsetbuttcap%
\pgfsetroundjoin%
\pgfsetlinewidth{0.399564pt}%
\definecolor{currentstroke}{rgb}{0.281446,0.084320,0.407414}%
\pgfsetstrokecolor{currentstroke}%
\pgfsetdash{}{0pt}%
\pgfpathmoveto{\pgfqpoint{7.438134in}{2.726574in}}%
\pgfpathlineto{\pgfqpoint{7.438706in}{2.726598in}}%
\pgfusepath{stroke}%
\end{pgfscope}%
\begin{pgfscope}%
\pgfpathrectangle{\pgfqpoint{6.720588in}{1.750000in}}{\pgfqpoint{2.279412in}{2.004545in}}%
\pgfusepath{clip}%
\pgfsetbuttcap%
\pgfsetroundjoin%
\pgfsetlinewidth{0.402193pt}%
\definecolor{currentstroke}{rgb}{0.281446,0.084320,0.407414}%
\pgfsetstrokecolor{currentstroke}%
\pgfsetdash{}{0pt}%
\pgfpathmoveto{\pgfqpoint{7.438706in}{2.726598in}}%
\pgfpathlineto{\pgfqpoint{7.438768in}{2.726603in}}%
\pgfusepath{stroke}%
\end{pgfscope}%
\begin{pgfscope}%
\pgfpathrectangle{\pgfqpoint{6.720588in}{1.750000in}}{\pgfqpoint{2.279412in}{2.004545in}}%
\pgfusepath{clip}%
\pgfsetbuttcap%
\pgfsetroundjoin%
\pgfsetlinewidth{0.313354pt}%
\definecolor{currentstroke}{rgb}{0.268510,0.009605,0.335427}%
\pgfsetstrokecolor{currentstroke}%
\pgfsetdash{}{0pt}%
\pgfpathmoveto{\pgfqpoint{8.680964in}{2.571846in}}%
\pgfpathlineto{\pgfqpoint{8.631406in}{2.571241in}}%
\pgfusepath{stroke}%
\end{pgfscope}%
\begin{pgfscope}%
\pgfpathrectangle{\pgfqpoint{6.720588in}{1.750000in}}{\pgfqpoint{2.279412in}{2.004545in}}%
\pgfusepath{clip}%
\pgfsetbuttcap%
\pgfsetroundjoin%
\pgfsetlinewidth{0.313952pt}%
\definecolor{currentstroke}{rgb}{0.268510,0.009605,0.335427}%
\pgfsetstrokecolor{currentstroke}%
\pgfsetdash{}{0pt}%
\pgfpathmoveto{\pgfqpoint{8.631406in}{2.571241in}}%
\pgfpathlineto{\pgfqpoint{8.582044in}{2.571025in}}%
\pgfusepath{stroke}%
\end{pgfscope}%
\begin{pgfscope}%
\pgfpathrectangle{\pgfqpoint{6.720588in}{1.750000in}}{\pgfqpoint{2.279412in}{2.004545in}}%
\pgfusepath{clip}%
\pgfsetbuttcap%
\pgfsetroundjoin%
\pgfsetlinewidth{0.311884pt}%
\definecolor{currentstroke}{rgb}{0.268510,0.009605,0.335427}%
\pgfsetstrokecolor{currentstroke}%
\pgfsetdash{}{0pt}%
\pgfpathmoveto{\pgfqpoint{8.582044in}{2.571025in}}%
\pgfpathlineto{\pgfqpoint{8.531910in}{2.570383in}}%
\pgfusepath{stroke}%
\end{pgfscope}%
\begin{pgfscope}%
\pgfpathrectangle{\pgfqpoint{6.720588in}{1.750000in}}{\pgfqpoint{2.279412in}{2.004545in}}%
\pgfusepath{clip}%
\pgfsetbuttcap%
\pgfsetroundjoin%
\pgfsetlinewidth{0.324455pt}%
\definecolor{currentstroke}{rgb}{0.271305,0.019942,0.347269}%
\pgfsetstrokecolor{currentstroke}%
\pgfsetdash{}{0pt}%
\pgfpathmoveto{\pgfqpoint{8.531910in}{2.570383in}}%
\pgfpathlineto{\pgfqpoint{8.481767in}{2.570698in}}%
\pgfusepath{stroke}%
\end{pgfscope}%
\begin{pgfscope}%
\pgfpathrectangle{\pgfqpoint{6.720588in}{1.750000in}}{\pgfqpoint{2.279412in}{2.004545in}}%
\pgfusepath{clip}%
\pgfsetbuttcap%
\pgfsetroundjoin%
\pgfsetlinewidth{0.328289pt}%
\definecolor{currentstroke}{rgb}{0.271305,0.019942,0.347269}%
\pgfsetstrokecolor{currentstroke}%
\pgfsetdash{}{0pt}%
\pgfpathmoveto{\pgfqpoint{8.481767in}{2.570698in}}%
\pgfpathlineto{\pgfqpoint{8.431627in}{2.571609in}}%
\pgfusepath{stroke}%
\end{pgfscope}%
\begin{pgfscope}%
\pgfpathrectangle{\pgfqpoint{6.720588in}{1.750000in}}{\pgfqpoint{2.279412in}{2.004545in}}%
\pgfusepath{clip}%
\pgfsetbuttcap%
\pgfsetroundjoin%
\pgfsetlinewidth{0.330659pt}%
\definecolor{currentstroke}{rgb}{0.272594,0.025563,0.353093}%
\pgfsetstrokecolor{currentstroke}%
\pgfsetdash{}{0pt}%
\pgfpathmoveto{\pgfqpoint{8.431627in}{2.571609in}}%
\pgfpathlineto{\pgfqpoint{8.381491in}{2.572680in}}%
\pgfusepath{stroke}%
\end{pgfscope}%
\begin{pgfscope}%
\pgfpathrectangle{\pgfqpoint{6.720588in}{1.750000in}}{\pgfqpoint{2.279412in}{2.004545in}}%
\pgfusepath{clip}%
\pgfsetbuttcap%
\pgfsetroundjoin%
\pgfsetlinewidth{0.365975pt}%
\definecolor{currentstroke}{rgb}{0.277941,0.056324,0.381191}%
\pgfsetstrokecolor{currentstroke}%
\pgfsetdash{}{0pt}%
\pgfpathmoveto{\pgfqpoint{8.381491in}{2.572680in}}%
\pgfpathlineto{\pgfqpoint{8.331350in}{2.573428in}}%
\pgfusepath{stroke}%
\end{pgfscope}%
\begin{pgfscope}%
\pgfpathrectangle{\pgfqpoint{6.720588in}{1.750000in}}{\pgfqpoint{2.279412in}{2.004545in}}%
\pgfusepath{clip}%
\pgfsetbuttcap%
\pgfsetroundjoin%
\pgfsetlinewidth{0.402346pt}%
\definecolor{currentstroke}{rgb}{0.281446,0.084320,0.407414}%
\pgfsetstrokecolor{currentstroke}%
\pgfsetdash{}{0pt}%
\pgfpathmoveto{\pgfqpoint{8.331350in}{2.573428in}}%
\pgfpathlineto{\pgfqpoint{8.281207in}{2.574152in}}%
\pgfusepath{stroke}%
\end{pgfscope}%
\begin{pgfscope}%
\pgfpathrectangle{\pgfqpoint{6.720588in}{1.750000in}}{\pgfqpoint{2.279412in}{2.004545in}}%
\pgfusepath{clip}%
\pgfsetbuttcap%
\pgfsetroundjoin%
\pgfsetlinewidth{0.437148pt}%
\definecolor{currentstroke}{rgb}{0.283091,0.110553,0.431554}%
\pgfsetstrokecolor{currentstroke}%
\pgfsetdash{}{0pt}%
\pgfpathmoveto{\pgfqpoint{8.281207in}{2.574152in}}%
\pgfpathlineto{\pgfqpoint{8.231072in}{2.575248in}}%
\pgfusepath{stroke}%
\end{pgfscope}%
\begin{pgfscope}%
\pgfpathrectangle{\pgfqpoint{6.720588in}{1.750000in}}{\pgfqpoint{2.279412in}{2.004545in}}%
\pgfusepath{clip}%
\pgfsetbuttcap%
\pgfsetroundjoin%
\pgfsetlinewidth{0.483535pt}%
\definecolor{currentstroke}{rgb}{0.282290,0.145912,0.461510}%
\pgfsetstrokecolor{currentstroke}%
\pgfsetdash{}{0pt}%
\pgfpathmoveto{\pgfqpoint{8.231072in}{2.575248in}}%
\pgfpathlineto{\pgfqpoint{8.180937in}{2.576382in}}%
\pgfusepath{stroke}%
\end{pgfscope}%
\begin{pgfscope}%
\pgfpathrectangle{\pgfqpoint{6.720588in}{1.750000in}}{\pgfqpoint{2.279412in}{2.004545in}}%
\pgfusepath{clip}%
\pgfsetbuttcap%
\pgfsetroundjoin%
\pgfsetlinewidth{0.560767pt}%
\definecolor{currentstroke}{rgb}{0.274128,0.199721,0.498911}%
\pgfsetstrokecolor{currentstroke}%
\pgfsetdash{}{0pt}%
\pgfpathmoveto{\pgfqpoint{8.180937in}{2.576382in}}%
\pgfpathlineto{\pgfqpoint{8.130811in}{2.577797in}}%
\pgfusepath{stroke}%
\end{pgfscope}%
\begin{pgfscope}%
\pgfpathrectangle{\pgfqpoint{6.720588in}{1.750000in}}{\pgfqpoint{2.279412in}{2.004545in}}%
\pgfusepath{clip}%
\pgfsetbuttcap%
\pgfsetroundjoin%
\pgfsetlinewidth{0.641564pt}%
\definecolor{currentstroke}{rgb}{0.257322,0.256130,0.526563}%
\pgfsetstrokecolor{currentstroke}%
\pgfsetdash{}{0pt}%
\pgfpathmoveto{\pgfqpoint{8.130811in}{2.577797in}}%
\pgfpathlineto{\pgfqpoint{8.080706in}{2.579677in}}%
\pgfusepath{stroke}%
\end{pgfscope}%
\begin{pgfscope}%
\pgfpathrectangle{\pgfqpoint{6.720588in}{1.750000in}}{\pgfqpoint{2.279412in}{2.004545in}}%
\pgfusepath{clip}%
\pgfsetbuttcap%
\pgfsetroundjoin%
\pgfsetlinewidth{0.713054pt}%
\definecolor{currentstroke}{rgb}{0.237441,0.305202,0.541921}%
\pgfsetstrokecolor{currentstroke}%
\pgfsetdash{}{0pt}%
\pgfpathmoveto{\pgfqpoint{8.080706in}{2.579677in}}%
\pgfpathlineto{\pgfqpoint{8.030631in}{2.582090in}}%
\pgfusepath{stroke}%
\end{pgfscope}%
\begin{pgfscope}%
\pgfpathrectangle{\pgfqpoint{6.720588in}{1.750000in}}{\pgfqpoint{2.279412in}{2.004545in}}%
\pgfusepath{clip}%
\pgfsetbuttcap%
\pgfsetroundjoin%
\pgfsetlinewidth{0.769418pt}%
\definecolor{currentstroke}{rgb}{0.221989,0.339161,0.548752}%
\pgfsetstrokecolor{currentstroke}%
\pgfsetdash{}{0pt}%
\pgfpathmoveto{\pgfqpoint{8.030631in}{2.582090in}}%
\pgfpathlineto{\pgfqpoint{7.980590in}{2.585003in}}%
\pgfusepath{stroke}%
\end{pgfscope}%
\begin{pgfscope}%
\pgfpathrectangle{\pgfqpoint{6.720588in}{1.750000in}}{\pgfqpoint{2.279412in}{2.004545in}}%
\pgfusepath{clip}%
\pgfsetbuttcap%
\pgfsetroundjoin%
\pgfsetlinewidth{0.808668pt}%
\definecolor{currentstroke}{rgb}{0.212395,0.359683,0.551710}%
\pgfsetstrokecolor{currentstroke}%
\pgfsetdash{}{0pt}%
\pgfpathmoveto{\pgfqpoint{7.980590in}{2.585003in}}%
\pgfpathlineto{\pgfqpoint{7.930610in}{2.588604in}}%
\pgfusepath{stroke}%
\end{pgfscope}%
\begin{pgfscope}%
\pgfpathrectangle{\pgfqpoint{6.720588in}{1.750000in}}{\pgfqpoint{2.279412in}{2.004545in}}%
\pgfusepath{clip}%
\pgfsetbuttcap%
\pgfsetroundjoin%
\pgfsetlinewidth{0.790437pt}%
\definecolor{currentstroke}{rgb}{0.216210,0.351535,0.550627}%
\pgfsetstrokecolor{currentstroke}%
\pgfsetdash{}{0pt}%
\pgfpathmoveto{\pgfqpoint{7.930610in}{2.588604in}}%
\pgfpathlineto{\pgfqpoint{7.880736in}{2.593196in}}%
\pgfusepath{stroke}%
\end{pgfscope}%
\begin{pgfscope}%
\pgfpathrectangle{\pgfqpoint{6.720588in}{1.750000in}}{\pgfqpoint{2.279412in}{2.004545in}}%
\pgfusepath{clip}%
\pgfsetbuttcap%
\pgfsetroundjoin%
\pgfsetlinewidth{0.705961pt}%
\definecolor{currentstroke}{rgb}{0.239346,0.300855,0.540844}%
\pgfsetstrokecolor{currentstroke}%
\pgfsetdash{}{0pt}%
\pgfpathmoveto{\pgfqpoint{7.880736in}{2.593196in}}%
\pgfpathlineto{\pgfqpoint{7.831002in}{2.598834in}}%
\pgfusepath{stroke}%
\end{pgfscope}%
\begin{pgfscope}%
\pgfpathrectangle{\pgfqpoint{6.720588in}{1.750000in}}{\pgfqpoint{2.279412in}{2.004545in}}%
\pgfusepath{clip}%
\pgfsetbuttcap%
\pgfsetroundjoin%
\pgfsetlinewidth{0.789398pt}%
\definecolor{currentstroke}{rgb}{0.216210,0.351535,0.550627}%
\pgfsetstrokecolor{currentstroke}%
\pgfsetdash{}{0pt}%
\pgfpathmoveto{\pgfqpoint{7.831002in}{2.598834in}}%
\pgfpathlineto{\pgfqpoint{7.781490in}{2.605785in}}%
\pgfusepath{stroke}%
\end{pgfscope}%
\begin{pgfscope}%
\pgfpathrectangle{\pgfqpoint{6.720588in}{1.750000in}}{\pgfqpoint{2.279412in}{2.004545in}}%
\pgfusepath{clip}%
\pgfsetbuttcap%
\pgfsetroundjoin%
\pgfsetlinewidth{0.698479pt}%
\definecolor{currentstroke}{rgb}{0.243113,0.292092,0.538516}%
\pgfsetstrokecolor{currentstroke}%
\pgfsetdash{}{0pt}%
\pgfpathmoveto{\pgfqpoint{7.781490in}{2.605785in}}%
\pgfpathlineto{\pgfqpoint{7.732249in}{2.614108in}}%
\pgfusepath{stroke}%
\end{pgfscope}%
\begin{pgfscope}%
\pgfpathrectangle{\pgfqpoint{6.720588in}{1.750000in}}{\pgfqpoint{2.279412in}{2.004545in}}%
\pgfusepath{clip}%
\pgfsetbuttcap%
\pgfsetroundjoin%
\pgfsetlinewidth{0.699337pt}%
\definecolor{currentstroke}{rgb}{0.241237,0.296485,0.539709}%
\pgfsetstrokecolor{currentstroke}%
\pgfsetdash{}{0pt}%
\pgfpathmoveto{\pgfqpoint{7.732249in}{2.614108in}}%
\pgfpathlineto{\pgfqpoint{7.683667in}{2.624785in}}%
\pgfusepath{stroke}%
\end{pgfscope}%
\begin{pgfscope}%
\pgfpathrectangle{\pgfqpoint{6.720588in}{1.750000in}}{\pgfqpoint{2.279412in}{2.004545in}}%
\pgfusepath{clip}%
\pgfsetbuttcap%
\pgfsetroundjoin%
\pgfsetlinewidth{0.604549pt}%
\definecolor{currentstroke}{rgb}{0.265145,0.232956,0.516599}%
\pgfsetstrokecolor{currentstroke}%
\pgfsetdash{}{0pt}%
\pgfpathmoveto{\pgfqpoint{7.683667in}{2.624785in}}%
\pgfpathlineto{\pgfqpoint{7.635592in}{2.637175in}}%
\pgfusepath{stroke}%
\end{pgfscope}%
\begin{pgfscope}%
\pgfpathrectangle{\pgfqpoint{6.720588in}{1.750000in}}{\pgfqpoint{2.279412in}{2.004545in}}%
\pgfusepath{clip}%
\pgfsetbuttcap%
\pgfsetroundjoin%
\pgfsetlinewidth{0.600022pt}%
\definecolor{currentstroke}{rgb}{0.266580,0.228262,0.514349}%
\pgfsetstrokecolor{currentstroke}%
\pgfsetdash{}{0pt}%
\pgfpathmoveto{\pgfqpoint{7.635592in}{2.637175in}}%
\pgfpathlineto{\pgfqpoint{7.587496in}{2.649539in}}%
\pgfusepath{stroke}%
\end{pgfscope}%
\begin{pgfscope}%
\pgfpathrectangle{\pgfqpoint{6.720588in}{1.750000in}}{\pgfqpoint{2.279412in}{2.004545in}}%
\pgfusepath{clip}%
\pgfsetbuttcap%
\pgfsetroundjoin%
\pgfsetlinewidth{0.579703pt}%
\definecolor{currentstroke}{rgb}{0.270595,0.214069,0.507052}%
\pgfsetstrokecolor{currentstroke}%
\pgfsetdash{}{0pt}%
\pgfpathmoveto{\pgfqpoint{7.587496in}{2.649539in}}%
\pgfpathlineto{\pgfqpoint{7.539751in}{2.662622in}}%
\pgfusepath{stroke}%
\end{pgfscope}%
\begin{pgfscope}%
\pgfpathrectangle{\pgfqpoint{6.720588in}{1.750000in}}{\pgfqpoint{2.279412in}{2.004545in}}%
\pgfusepath{clip}%
\pgfsetbuttcap%
\pgfsetroundjoin%
\pgfsetlinewidth{0.525085pt}%
\definecolor{currentstroke}{rgb}{0.278826,0.175490,0.483397}%
\pgfsetstrokecolor{currentstroke}%
\pgfsetdash{}{0pt}%
\pgfpathmoveto{\pgfqpoint{7.539751in}{2.662622in}}%
\pgfpathlineto{\pgfqpoint{7.539751in}{2.662622in}}%
\pgfusepath{stroke}%
\end{pgfscope}%
\begin{pgfscope}%
\pgfpathrectangle{\pgfqpoint{6.720588in}{1.750000in}}{\pgfqpoint{2.279412in}{2.004545in}}%
\pgfusepath{clip}%
\pgfsetbuttcap%
\pgfsetroundjoin%
\pgfsetlinewidth{0.525085pt}%
\definecolor{currentstroke}{rgb}{0.278826,0.175490,0.483397}%
\pgfsetstrokecolor{currentstroke}%
\pgfsetdash{}{0pt}%
\pgfpathmoveto{\pgfqpoint{7.539751in}{2.662622in}}%
\pgfpathlineto{\pgfqpoint{7.509931in}{2.673464in}}%
\pgfusepath{stroke}%
\end{pgfscope}%
\begin{pgfscope}%
\pgfpathrectangle{\pgfqpoint{6.720588in}{1.750000in}}{\pgfqpoint{2.279412in}{2.004545in}}%
\pgfusepath{clip}%
\pgfsetbuttcap%
\pgfsetroundjoin%
\pgfsetlinewidth{0.323323pt}%
\definecolor{currentstroke}{rgb}{0.271305,0.019942,0.347269}%
\pgfsetstrokecolor{currentstroke}%
\pgfsetdash{}{0pt}%
\pgfpathmoveto{\pgfqpoint{8.680964in}{2.616952in}}%
\pgfpathlineto{\pgfqpoint{8.632153in}{2.617371in}}%
\pgfusepath{stroke}%
\end{pgfscope}%
\begin{pgfscope}%
\pgfpathrectangle{\pgfqpoint{6.720588in}{1.750000in}}{\pgfqpoint{2.279412in}{2.004545in}}%
\pgfusepath{clip}%
\pgfsetbuttcap%
\pgfsetroundjoin%
\pgfsetlinewidth{0.316599pt}%
\definecolor{currentstroke}{rgb}{0.269944,0.014625,0.341379}%
\pgfsetstrokecolor{currentstroke}%
\pgfsetdash{}{0pt}%
\pgfpathmoveto{\pgfqpoint{8.632153in}{2.617371in}}%
\pgfpathlineto{\pgfqpoint{8.583924in}{2.617167in}}%
\pgfusepath{stroke}%
\end{pgfscope}%
\begin{pgfscope}%
\pgfpathrectangle{\pgfqpoint{6.720588in}{1.750000in}}{\pgfqpoint{2.279412in}{2.004545in}}%
\pgfusepath{clip}%
\pgfsetbuttcap%
\pgfsetroundjoin%
\pgfsetlinewidth{0.310289pt}%
\definecolor{currentstroke}{rgb}{0.268510,0.009605,0.335427}%
\pgfsetstrokecolor{currentstroke}%
\pgfsetdash{}{0pt}%
\pgfpathmoveto{\pgfqpoint{8.583924in}{2.617167in}}%
\pgfpathlineto{\pgfqpoint{8.533840in}{2.615187in}}%
\pgfusepath{stroke}%
\end{pgfscope}%
\begin{pgfscope}%
\pgfpathrectangle{\pgfqpoint{6.720588in}{1.750000in}}{\pgfqpoint{2.279412in}{2.004545in}}%
\pgfusepath{clip}%
\pgfsetbuttcap%
\pgfsetroundjoin%
\pgfsetlinewidth{0.327896pt}%
\definecolor{currentstroke}{rgb}{0.271305,0.019942,0.347269}%
\pgfsetstrokecolor{currentstroke}%
\pgfsetdash{}{0pt}%
\pgfpathmoveto{\pgfqpoint{8.533840in}{2.615187in}}%
\pgfpathlineto{\pgfqpoint{8.483728in}{2.613825in}}%
\pgfusepath{stroke}%
\end{pgfscope}%
\begin{pgfscope}%
\pgfpathrectangle{\pgfqpoint{6.720588in}{1.750000in}}{\pgfqpoint{2.279412in}{2.004545in}}%
\pgfusepath{clip}%
\pgfsetbuttcap%
\pgfsetroundjoin%
\pgfsetlinewidth{0.331547pt}%
\definecolor{currentstroke}{rgb}{0.272594,0.025563,0.353093}%
\pgfsetstrokecolor{currentstroke}%
\pgfsetdash{}{0pt}%
\pgfpathmoveto{\pgfqpoint{8.483728in}{2.613825in}}%
\pgfpathlineto{\pgfqpoint{8.433590in}{2.613646in}}%
\pgfusepath{stroke}%
\end{pgfscope}%
\begin{pgfscope}%
\pgfpathrectangle{\pgfqpoint{6.720588in}{1.750000in}}{\pgfqpoint{2.279412in}{2.004545in}}%
\pgfusepath{clip}%
\pgfsetbuttcap%
\pgfsetroundjoin%
\pgfsetlinewidth{0.338691pt}%
\definecolor{currentstroke}{rgb}{0.273809,0.031497,0.358853}%
\pgfsetstrokecolor{currentstroke}%
\pgfsetdash{}{0pt}%
\pgfpathmoveto{\pgfqpoint{8.433590in}{2.613646in}}%
\pgfpathlineto{\pgfqpoint{8.383443in}{2.613745in}}%
\pgfusepath{stroke}%
\end{pgfscope}%
\begin{pgfscope}%
\pgfpathrectangle{\pgfqpoint{6.720588in}{1.750000in}}{\pgfqpoint{2.279412in}{2.004545in}}%
\pgfusepath{clip}%
\pgfsetbuttcap%
\pgfsetroundjoin%
\pgfsetlinewidth{0.369827pt}%
\definecolor{currentstroke}{rgb}{0.278791,0.062145,0.386592}%
\pgfsetstrokecolor{currentstroke}%
\pgfsetdash{}{0pt}%
\pgfpathmoveto{\pgfqpoint{8.383443in}{2.613745in}}%
\pgfpathlineto{\pgfqpoint{8.333295in}{2.614120in}}%
\pgfusepath{stroke}%
\end{pgfscope}%
\begin{pgfscope}%
\pgfpathrectangle{\pgfqpoint{6.720588in}{1.750000in}}{\pgfqpoint{2.279412in}{2.004545in}}%
\pgfusepath{clip}%
\pgfsetbuttcap%
\pgfsetroundjoin%
\pgfsetlinewidth{0.397102pt}%
\definecolor{currentstroke}{rgb}{0.280894,0.078907,0.402329}%
\pgfsetstrokecolor{currentstroke}%
\pgfsetdash{}{0pt}%
\pgfpathmoveto{\pgfqpoint{8.333295in}{2.614120in}}%
\pgfpathlineto{\pgfqpoint{8.283150in}{2.614828in}}%
\pgfusepath{stroke}%
\end{pgfscope}%
\begin{pgfscope}%
\pgfpathrectangle{\pgfqpoint{6.720588in}{1.750000in}}{\pgfqpoint{2.279412in}{2.004545in}}%
\pgfusepath{clip}%
\pgfsetbuttcap%
\pgfsetroundjoin%
\pgfsetlinewidth{0.434315pt}%
\definecolor{currentstroke}{rgb}{0.283091,0.110553,0.431554}%
\pgfsetstrokecolor{currentstroke}%
\pgfsetdash{}{0pt}%
\pgfpathmoveto{\pgfqpoint{8.283150in}{2.614828in}}%
\pgfpathlineto{\pgfqpoint{8.233006in}{2.615557in}}%
\pgfusepath{stroke}%
\end{pgfscope}%
\begin{pgfscope}%
\pgfpathrectangle{\pgfqpoint{6.720588in}{1.750000in}}{\pgfqpoint{2.279412in}{2.004545in}}%
\pgfusepath{clip}%
\pgfsetbuttcap%
\pgfsetroundjoin%
\pgfsetlinewidth{0.505739pt}%
\definecolor{currentstroke}{rgb}{0.280868,0.160771,0.472899}%
\pgfsetstrokecolor{currentstroke}%
\pgfsetdash{}{0pt}%
\pgfpathmoveto{\pgfqpoint{8.233006in}{2.615557in}}%
\pgfpathlineto{\pgfqpoint{8.182864in}{2.616454in}}%
\pgfusepath{stroke}%
\end{pgfscope}%
\begin{pgfscope}%
\pgfpathrectangle{\pgfqpoint{6.720588in}{1.750000in}}{\pgfqpoint{2.279412in}{2.004545in}}%
\pgfusepath{clip}%
\pgfsetbuttcap%
\pgfsetroundjoin%
\pgfsetlinewidth{0.580408pt}%
\definecolor{currentstroke}{rgb}{0.270595,0.214069,0.507052}%
\pgfsetstrokecolor{currentstroke}%
\pgfsetdash{}{0pt}%
\pgfpathmoveto{\pgfqpoint{8.182864in}{2.616454in}}%
\pgfpathlineto{\pgfqpoint{8.132725in}{2.617453in}}%
\pgfusepath{stroke}%
\end{pgfscope}%
\begin{pgfscope}%
\pgfpathrectangle{\pgfqpoint{6.720588in}{1.750000in}}{\pgfqpoint{2.279412in}{2.004545in}}%
\pgfusepath{clip}%
\pgfsetbuttcap%
\pgfsetroundjoin%
\pgfsetlinewidth{0.664066pt}%
\definecolor{currentstroke}{rgb}{0.252194,0.269783,0.531579}%
\pgfsetstrokecolor{currentstroke}%
\pgfsetdash{}{0pt}%
\pgfpathmoveto{\pgfqpoint{8.132725in}{2.617453in}}%
\pgfpathlineto{\pgfqpoint{8.082595in}{2.618724in}}%
\pgfusepath{stroke}%
\end{pgfscope}%
\begin{pgfscope}%
\pgfpathrectangle{\pgfqpoint{6.720588in}{1.750000in}}{\pgfqpoint{2.279412in}{2.004545in}}%
\pgfusepath{clip}%
\pgfsetbuttcap%
\pgfsetroundjoin%
\pgfsetlinewidth{0.735743pt}%
\definecolor{currentstroke}{rgb}{0.231674,0.318106,0.544834}%
\pgfsetstrokecolor{currentstroke}%
\pgfsetdash{}{0pt}%
\pgfpathmoveto{\pgfqpoint{8.082595in}{2.618724in}}%
\pgfpathlineto{\pgfqpoint{8.032480in}{2.620406in}}%
\pgfusepath{stroke}%
\end{pgfscope}%
\begin{pgfscope}%
\pgfpathrectangle{\pgfqpoint{6.720588in}{1.750000in}}{\pgfqpoint{2.279412in}{2.004545in}}%
\pgfusepath{clip}%
\pgfsetbuttcap%
\pgfsetroundjoin%
\pgfsetlinewidth{0.797929pt}%
\definecolor{currentstroke}{rgb}{0.214298,0.355619,0.551184}%
\pgfsetstrokecolor{currentstroke}%
\pgfsetdash{}{0pt}%
\pgfpathmoveto{\pgfqpoint{8.032480in}{2.620406in}}%
\pgfpathlineto{\pgfqpoint{7.982386in}{2.622504in}}%
\pgfusepath{stroke}%
\end{pgfscope}%
\begin{pgfscope}%
\pgfpathrectangle{\pgfqpoint{6.720588in}{1.750000in}}{\pgfqpoint{2.279412in}{2.004545in}}%
\pgfusepath{clip}%
\pgfsetbuttcap%
\pgfsetroundjoin%
\pgfsetlinewidth{0.824844pt}%
\definecolor{currentstroke}{rgb}{0.206756,0.371758,0.553117}%
\pgfsetstrokecolor{currentstroke}%
\pgfsetdash{}{0pt}%
\pgfpathmoveto{\pgfqpoint{7.982386in}{2.622504in}}%
\pgfpathlineto{\pgfqpoint{7.932327in}{2.625139in}}%
\pgfusepath{stroke}%
\end{pgfscope}%
\begin{pgfscope}%
\pgfpathrectangle{\pgfqpoint{6.720588in}{1.750000in}}{\pgfqpoint{2.279412in}{2.004545in}}%
\pgfusepath{clip}%
\pgfsetbuttcap%
\pgfsetroundjoin%
\pgfsetlinewidth{0.850054pt}%
\definecolor{currentstroke}{rgb}{0.199430,0.387607,0.554642}%
\pgfsetstrokecolor{currentstroke}%
\pgfsetdash{}{0pt}%
\pgfpathmoveto{\pgfqpoint{7.932327in}{2.625139in}}%
\pgfpathlineto{\pgfqpoint{7.882308in}{2.628278in}}%
\pgfusepath{stroke}%
\end{pgfscope}%
\begin{pgfscope}%
\pgfpathrectangle{\pgfqpoint{6.720588in}{1.750000in}}{\pgfqpoint{2.279412in}{2.004545in}}%
\pgfusepath{clip}%
\pgfsetbuttcap%
\pgfsetroundjoin%
\pgfsetlinewidth{1.461559pt}%
\definecolor{currentstroke}{rgb}{0.202219,0.715272,0.476084}%
\pgfsetstrokecolor{currentstroke}%
\pgfsetdash{}{0pt}%
\pgfpathmoveto{\pgfqpoint{7.039624in}{2.707166in}}%
\pgfpathlineto{\pgfqpoint{7.089741in}{2.708618in}}%
\pgfusepath{stroke}%
\end{pgfscope}%
\begin{pgfscope}%
\pgfpathrectangle{\pgfqpoint{6.720588in}{1.750000in}}{\pgfqpoint{2.279412in}{2.004545in}}%
\pgfusepath{clip}%
\pgfsetbuttcap%
\pgfsetroundjoin%
\pgfsetlinewidth{1.335285pt}%
\definecolor{currentstroke}{rgb}{0.128087,0.647749,0.523491}%
\pgfsetstrokecolor{currentstroke}%
\pgfsetdash{}{0pt}%
\pgfpathmoveto{\pgfqpoint{7.089741in}{2.708618in}}%
\pgfpathlineto{\pgfqpoint{7.139868in}{2.709630in}}%
\pgfusepath{stroke}%
\end{pgfscope}%
\begin{pgfscope}%
\pgfpathrectangle{\pgfqpoint{6.720588in}{1.750000in}}{\pgfqpoint{2.279412in}{2.004545in}}%
\pgfusepath{clip}%
\pgfsetbuttcap%
\pgfsetroundjoin%
\pgfsetlinewidth{1.199924pt}%
\definecolor{currentstroke}{rgb}{0.125394,0.574318,0.549086}%
\pgfsetstrokecolor{currentstroke}%
\pgfsetdash{}{0pt}%
\pgfpathmoveto{\pgfqpoint{7.139868in}{2.709630in}}%
\pgfpathlineto{\pgfqpoint{7.190001in}{2.710582in}}%
\pgfusepath{stroke}%
\end{pgfscope}%
\begin{pgfscope}%
\pgfpathrectangle{\pgfqpoint{6.720588in}{1.750000in}}{\pgfqpoint{2.279412in}{2.004545in}}%
\pgfusepath{clip}%
\pgfsetbuttcap%
\pgfsetroundjoin%
\pgfsetlinewidth{1.074970pt}%
\definecolor{currentstroke}{rgb}{0.149039,0.508051,0.557250}%
\pgfsetstrokecolor{currentstroke}%
\pgfsetdash{}{0pt}%
\pgfpathmoveto{\pgfqpoint{7.190001in}{2.710582in}}%
\pgfpathlineto{\pgfqpoint{7.240118in}{2.712015in}}%
\pgfusepath{stroke}%
\end{pgfscope}%
\begin{pgfscope}%
\pgfpathrectangle{\pgfqpoint{6.720588in}{1.750000in}}{\pgfqpoint{2.279412in}{2.004545in}}%
\pgfusepath{clip}%
\pgfsetbuttcap%
\pgfsetroundjoin%
\pgfsetlinewidth{0.925003pt}%
\definecolor{currentstroke}{rgb}{0.182256,0.426184,0.557120}%
\pgfsetstrokecolor{currentstroke}%
\pgfsetdash{}{0pt}%
\pgfpathmoveto{\pgfqpoint{7.240118in}{2.712015in}}%
\pgfpathlineto{\pgfqpoint{7.290247in}{2.713076in}}%
\pgfusepath{stroke}%
\end{pgfscope}%
\begin{pgfscope}%
\pgfpathrectangle{\pgfqpoint{6.720588in}{1.750000in}}{\pgfqpoint{2.279412in}{2.004545in}}%
\pgfusepath{clip}%
\pgfsetbuttcap%
\pgfsetroundjoin%
\pgfsetlinewidth{0.810865pt}%
\definecolor{currentstroke}{rgb}{0.210503,0.363727,0.552206}%
\pgfsetstrokecolor{currentstroke}%
\pgfsetdash{}{0pt}%
\pgfpathmoveto{\pgfqpoint{7.290247in}{2.713076in}}%
\pgfpathlineto{\pgfqpoint{7.340329in}{2.714905in}}%
\pgfusepath{stroke}%
\end{pgfscope}%
\begin{pgfscope}%
\pgfpathrectangle{\pgfqpoint{6.720588in}{1.750000in}}{\pgfqpoint{2.279412in}{2.004545in}}%
\pgfusepath{clip}%
\pgfsetbuttcap%
\pgfsetroundjoin%
\pgfsetlinewidth{0.576628pt}%
\definecolor{currentstroke}{rgb}{0.270595,0.214069,0.507052}%
\pgfsetstrokecolor{currentstroke}%
\pgfsetdash{}{0pt}%
\pgfpathmoveto{\pgfqpoint{7.340329in}{2.714905in}}%
\pgfpathlineto{\pgfqpoint{7.390398in}{2.717005in}}%
\pgfusepath{stroke}%
\end{pgfscope}%
\begin{pgfscope}%
\pgfpathrectangle{\pgfqpoint{6.720588in}{1.750000in}}{\pgfqpoint{2.279412in}{2.004545in}}%
\pgfusepath{clip}%
\pgfsetbuttcap%
\pgfsetroundjoin%
\pgfsetlinewidth{0.521001pt}%
\definecolor{currentstroke}{rgb}{0.278826,0.175490,0.483397}%
\pgfsetstrokecolor{currentstroke}%
\pgfsetdash{}{0pt}%
\pgfpathmoveto{\pgfqpoint{7.390398in}{2.717005in}}%
\pgfpathlineto{\pgfqpoint{7.390398in}{2.717005in}}%
\pgfusepath{stroke}%
\end{pgfscope}%
\begin{pgfscope}%
\pgfpathrectangle{\pgfqpoint{6.720588in}{1.750000in}}{\pgfqpoint{2.279412in}{2.004545in}}%
\pgfusepath{clip}%
\pgfsetbuttcap%
\pgfsetroundjoin%
\pgfsetlinewidth{0.521001pt}%
\definecolor{currentstroke}{rgb}{0.278826,0.175490,0.483397}%
\pgfsetstrokecolor{currentstroke}%
\pgfsetdash{}{0pt}%
\pgfpathmoveto{\pgfqpoint{7.390398in}{2.717005in}}%
\pgfpathlineto{\pgfqpoint{7.413582in}{2.717406in}}%
\pgfusepath{stroke}%
\end{pgfscope}%
\begin{pgfscope}%
\pgfpathrectangle{\pgfqpoint{6.720588in}{1.750000in}}{\pgfqpoint{2.279412in}{2.004545in}}%
\pgfusepath{clip}%
\pgfsetbuttcap%
\pgfsetroundjoin%
\pgfsetlinewidth{0.398459pt}%
\definecolor{currentstroke}{rgb}{0.281446,0.084320,0.407414}%
\pgfsetstrokecolor{currentstroke}%
\pgfsetdash{}{0pt}%
\pgfpathmoveto{\pgfqpoint{7.413582in}{2.717406in}}%
\pgfpathlineto{\pgfqpoint{7.413582in}{2.717406in}}%
\pgfusepath{stroke}%
\end{pgfscope}%
\begin{pgfscope}%
\pgfpathrectangle{\pgfqpoint{6.720588in}{1.750000in}}{\pgfqpoint{2.279412in}{2.004545in}}%
\pgfusepath{clip}%
\pgfsetbuttcap%
\pgfsetroundjoin%
\pgfsetlinewidth{0.313049pt}%
\definecolor{currentstroke}{rgb}{0.268510,0.009605,0.335427}%
\pgfsetstrokecolor{currentstroke}%
\pgfsetdash{}{0pt}%
\pgfpathmoveto{\pgfqpoint{8.542422in}{2.070183in}}%
\pgfpathlineto{\pgfqpoint{8.497859in}{2.070337in}}%
\pgfusepath{stroke}%
\end{pgfscope}%
\begin{pgfscope}%
\pgfpathrectangle{\pgfqpoint{6.720588in}{1.750000in}}{\pgfqpoint{2.279412in}{2.004545in}}%
\pgfusepath{clip}%
\pgfsetbuttcap%
\pgfsetroundjoin%
\pgfsetlinewidth{0.314302pt}%
\definecolor{currentstroke}{rgb}{0.268510,0.009605,0.335427}%
\pgfsetstrokecolor{currentstroke}%
\pgfsetdash{}{0pt}%
\pgfpathmoveto{\pgfqpoint{8.497859in}{2.070337in}}%
\pgfpathlineto{\pgfqpoint{8.453315in}{2.071368in}}%
\pgfusepath{stroke}%
\end{pgfscope}%
\begin{pgfscope}%
\pgfpathrectangle{\pgfqpoint{6.720588in}{1.750000in}}{\pgfqpoint{2.279412in}{2.004545in}}%
\pgfusepath{clip}%
\pgfsetbuttcap%
\pgfsetroundjoin%
\pgfsetlinewidth{0.315544pt}%
\definecolor{currentstroke}{rgb}{0.269944,0.014625,0.341379}%
\pgfsetstrokecolor{currentstroke}%
\pgfsetdash{}{0pt}%
\pgfpathmoveto{\pgfqpoint{8.453315in}{2.071368in}}%
\pgfpathlineto{\pgfqpoint{8.416568in}{2.072518in}}%
\pgfusepath{stroke}%
\end{pgfscope}%
\begin{pgfscope}%
\pgfpathrectangle{\pgfqpoint{6.720588in}{1.750000in}}{\pgfqpoint{2.279412in}{2.004545in}}%
\pgfusepath{clip}%
\pgfsetbuttcap%
\pgfsetroundjoin%
\pgfsetlinewidth{0.315222pt}%
\definecolor{currentstroke}{rgb}{0.269944,0.014625,0.341379}%
\pgfsetstrokecolor{currentstroke}%
\pgfsetdash{}{0pt}%
\pgfpathmoveto{\pgfqpoint{8.416568in}{2.072518in}}%
\pgfpathlineto{\pgfqpoint{8.373213in}{2.075671in}}%
\pgfusepath{stroke}%
\end{pgfscope}%
\begin{pgfscope}%
\pgfpathrectangle{\pgfqpoint{6.720588in}{1.750000in}}{\pgfqpoint{2.279412in}{2.004545in}}%
\pgfusepath{clip}%
\pgfsetbuttcap%
\pgfsetroundjoin%
\pgfsetlinewidth{0.323073pt}%
\definecolor{currentstroke}{rgb}{0.271305,0.019942,0.347269}%
\pgfsetstrokecolor{currentstroke}%
\pgfsetdash{}{0pt}%
\pgfpathmoveto{\pgfqpoint{8.373213in}{2.075671in}}%
\pgfpathlineto{\pgfqpoint{8.323257in}{2.078301in}}%
\pgfusepath{stroke}%
\end{pgfscope}%
\begin{pgfscope}%
\pgfpathrectangle{\pgfqpoint{6.720588in}{1.750000in}}{\pgfqpoint{2.279412in}{2.004545in}}%
\pgfusepath{clip}%
\pgfsetbuttcap%
\pgfsetroundjoin%
\pgfsetlinewidth{0.340275pt}%
\definecolor{currentstroke}{rgb}{0.273809,0.031497,0.358853}%
\pgfsetstrokecolor{currentstroke}%
\pgfsetdash{}{0pt}%
\pgfpathmoveto{\pgfqpoint{8.323257in}{2.078301in}}%
\pgfpathlineto{\pgfqpoint{8.273204in}{2.079348in}}%
\pgfusepath{stroke}%
\end{pgfscope}%
\begin{pgfscope}%
\pgfpathrectangle{\pgfqpoint{6.720588in}{1.750000in}}{\pgfqpoint{2.279412in}{2.004545in}}%
\pgfusepath{clip}%
\pgfsetbuttcap%
\pgfsetroundjoin%
\pgfsetlinewidth{0.336925pt}%
\definecolor{currentstroke}{rgb}{0.273809,0.031497,0.358853}%
\pgfsetstrokecolor{currentstroke}%
\pgfsetdash{}{0pt}%
\pgfpathmoveto{\pgfqpoint{8.273204in}{2.079348in}}%
\pgfpathlineto{\pgfqpoint{8.223189in}{2.081738in}}%
\pgfusepath{stroke}%
\end{pgfscope}%
\begin{pgfscope}%
\pgfpathrectangle{\pgfqpoint{6.720588in}{1.750000in}}{\pgfqpoint{2.279412in}{2.004545in}}%
\pgfusepath{clip}%
\pgfsetbuttcap%
\pgfsetroundjoin%
\pgfsetlinewidth{0.340709pt}%
\definecolor{currentstroke}{rgb}{0.273809,0.031497,0.358853}%
\pgfsetstrokecolor{currentstroke}%
\pgfsetdash{}{0pt}%
\pgfpathmoveto{\pgfqpoint{8.223189in}{2.081738in}}%
\pgfpathlineto{\pgfqpoint{8.173304in}{2.085354in}}%
\pgfusepath{stroke}%
\end{pgfscope}%
\begin{pgfscope}%
\pgfpathrectangle{\pgfqpoint{6.720588in}{1.750000in}}{\pgfqpoint{2.279412in}{2.004545in}}%
\pgfusepath{clip}%
\pgfsetbuttcap%
\pgfsetroundjoin%
\pgfsetlinewidth{0.357263pt}%
\definecolor{currentstroke}{rgb}{0.277018,0.050344,0.375715}%
\pgfsetstrokecolor{currentstroke}%
\pgfsetdash{}{0pt}%
\pgfpathmoveto{\pgfqpoint{8.173304in}{2.085354in}}%
\pgfpathlineto{\pgfqpoint{8.123648in}{2.090659in}}%
\pgfusepath{stroke}%
\end{pgfscope}%
\begin{pgfscope}%
\pgfpathrectangle{\pgfqpoint{6.720588in}{1.750000in}}{\pgfqpoint{2.279412in}{2.004545in}}%
\pgfusepath{clip}%
\pgfsetbuttcap%
\pgfsetroundjoin%
\pgfsetlinewidth{0.348697pt}%
\definecolor{currentstroke}{rgb}{0.274952,0.037752,0.364543}%
\pgfsetstrokecolor{currentstroke}%
\pgfsetdash{}{0pt}%
\pgfpathmoveto{\pgfqpoint{8.123648in}{2.090659in}}%
\pgfpathlineto{\pgfqpoint{8.074111in}{2.097480in}}%
\pgfusepath{stroke}%
\end{pgfscope}%
\begin{pgfscope}%
\pgfpathrectangle{\pgfqpoint{6.720588in}{1.750000in}}{\pgfqpoint{2.279412in}{2.004545in}}%
\pgfusepath{clip}%
\pgfsetbuttcap%
\pgfsetroundjoin%
\pgfsetlinewidth{0.322248pt}%
\definecolor{currentstroke}{rgb}{0.271305,0.019942,0.347269}%
\pgfsetstrokecolor{currentstroke}%
\pgfsetdash{}{0pt}%
\pgfpathmoveto{\pgfqpoint{8.629673in}{2.707166in}}%
\pgfpathlineto{\pgfqpoint{8.629673in}{2.707166in}}%
\pgfusepath{stroke}%
\end{pgfscope}%
\begin{pgfscope}%
\pgfpathrectangle{\pgfqpoint{6.720588in}{1.750000in}}{\pgfqpoint{2.279412in}{2.004545in}}%
\pgfusepath{clip}%
\pgfsetbuttcap%
\pgfsetroundjoin%
\pgfsetlinewidth{0.322248pt}%
\definecolor{currentstroke}{rgb}{0.271305,0.019942,0.347269}%
\pgfsetstrokecolor{currentstroke}%
\pgfsetdash{}{0pt}%
\pgfpathmoveto{\pgfqpoint{8.629673in}{2.707166in}}%
\pgfpathlineto{\pgfqpoint{8.594412in}{2.707364in}}%
\pgfusepath{stroke}%
\end{pgfscope}%
\begin{pgfscope}%
\pgfpathrectangle{\pgfqpoint{6.720588in}{1.750000in}}{\pgfqpoint{2.279412in}{2.004545in}}%
\pgfusepath{clip}%
\pgfsetbuttcap%
\pgfsetroundjoin%
\pgfsetlinewidth{0.315638pt}%
\definecolor{currentstroke}{rgb}{0.269944,0.014625,0.341379}%
\pgfsetstrokecolor{currentstroke}%
\pgfsetdash{}{0pt}%
\pgfpathmoveto{\pgfqpoint{8.594412in}{2.707364in}}%
\pgfpathlineto{\pgfqpoint{8.558743in}{2.707128in}}%
\pgfusepath{stroke}%
\end{pgfscope}%
\begin{pgfscope}%
\pgfpathrectangle{\pgfqpoint{6.720588in}{1.750000in}}{\pgfqpoint{2.279412in}{2.004545in}}%
\pgfusepath{clip}%
\pgfsetbuttcap%
\pgfsetroundjoin%
\pgfsetlinewidth{0.309852pt}%
\definecolor{currentstroke}{rgb}{0.268510,0.009605,0.335427}%
\pgfsetstrokecolor{currentstroke}%
\pgfsetdash{}{0pt}%
\pgfpathmoveto{\pgfqpoint{8.558743in}{2.707128in}}%
\pgfpathlineto{\pgfqpoint{8.509009in}{2.705296in}}%
\pgfusepath{stroke}%
\end{pgfscope}%
\begin{pgfscope}%
\pgfpathrectangle{\pgfqpoint{6.720588in}{1.750000in}}{\pgfqpoint{2.279412in}{2.004545in}}%
\pgfusepath{clip}%
\pgfsetbuttcap%
\pgfsetroundjoin%
\pgfsetlinewidth{0.323152pt}%
\definecolor{currentstroke}{rgb}{0.271305,0.019942,0.347269}%
\pgfsetstrokecolor{currentstroke}%
\pgfsetdash{}{0pt}%
\pgfpathmoveto{\pgfqpoint{8.509009in}{2.705296in}}%
\pgfpathlineto{\pgfqpoint{8.458870in}{2.704764in}}%
\pgfusepath{stroke}%
\end{pgfscope}%
\begin{pgfscope}%
\pgfpathrectangle{\pgfqpoint{6.720588in}{1.750000in}}{\pgfqpoint{2.279412in}{2.004545in}}%
\pgfusepath{clip}%
\pgfsetbuttcap%
\pgfsetroundjoin%
\pgfsetlinewidth{0.336603pt}%
\definecolor{currentstroke}{rgb}{0.273809,0.031497,0.358853}%
\pgfsetstrokecolor{currentstroke}%
\pgfsetdash{}{0pt}%
\pgfpathmoveto{\pgfqpoint{8.458870in}{2.704764in}}%
\pgfpathlineto{\pgfqpoint{8.408725in}{2.704451in}}%
\pgfusepath{stroke}%
\end{pgfscope}%
\begin{pgfscope}%
\pgfpathrectangle{\pgfqpoint{6.720588in}{1.750000in}}{\pgfqpoint{2.279412in}{2.004545in}}%
\pgfusepath{clip}%
\pgfsetbuttcap%
\pgfsetroundjoin%
\pgfsetlinewidth{0.355007pt}%
\definecolor{currentstroke}{rgb}{0.276022,0.044167,0.370164}%
\pgfsetstrokecolor{currentstroke}%
\pgfsetdash{}{0pt}%
\pgfpathmoveto{\pgfqpoint{8.408725in}{2.704451in}}%
\pgfpathlineto{\pgfqpoint{8.358576in}{2.704417in}}%
\pgfusepath{stroke}%
\end{pgfscope}%
\begin{pgfscope}%
\pgfpathrectangle{\pgfqpoint{6.720588in}{1.750000in}}{\pgfqpoint{2.279412in}{2.004545in}}%
\pgfusepath{clip}%
\pgfsetbuttcap%
\pgfsetroundjoin%
\pgfsetlinewidth{0.375013pt}%
\definecolor{currentstroke}{rgb}{0.278791,0.062145,0.386592}%
\pgfsetstrokecolor{currentstroke}%
\pgfsetdash{}{0pt}%
\pgfpathmoveto{\pgfqpoint{8.358576in}{2.704417in}}%
\pgfpathlineto{\pgfqpoint{8.308425in}{2.704315in}}%
\pgfusepath{stroke}%
\end{pgfscope}%
\begin{pgfscope}%
\pgfpathrectangle{\pgfqpoint{6.720588in}{1.750000in}}{\pgfqpoint{2.279412in}{2.004545in}}%
\pgfusepath{clip}%
\pgfsetbuttcap%
\pgfsetroundjoin%
\pgfsetlinewidth{0.420560pt}%
\definecolor{currentstroke}{rgb}{0.282656,0.100196,0.422160}%
\pgfsetstrokecolor{currentstroke}%
\pgfsetdash{}{0pt}%
\pgfpathmoveto{\pgfqpoint{8.308425in}{2.704315in}}%
\pgfpathlineto{\pgfqpoint{8.258276in}{2.703946in}}%
\pgfusepath{stroke}%
\end{pgfscope}%
\begin{pgfscope}%
\pgfpathrectangle{\pgfqpoint{6.720588in}{1.750000in}}{\pgfqpoint{2.279412in}{2.004545in}}%
\pgfusepath{clip}%
\pgfsetbuttcap%
\pgfsetroundjoin%
\pgfsetlinewidth{0.479605pt}%
\definecolor{currentstroke}{rgb}{0.282290,0.145912,0.461510}%
\pgfsetstrokecolor{currentstroke}%
\pgfsetdash{}{0pt}%
\pgfpathmoveto{\pgfqpoint{8.258276in}{2.703946in}}%
\pgfpathlineto{\pgfqpoint{8.208125in}{2.703742in}}%
\pgfusepath{stroke}%
\end{pgfscope}%
\begin{pgfscope}%
\pgfpathrectangle{\pgfqpoint{6.720588in}{1.750000in}}{\pgfqpoint{2.279412in}{2.004545in}}%
\pgfusepath{clip}%
\pgfsetbuttcap%
\pgfsetroundjoin%
\pgfsetlinewidth{0.551161pt}%
\definecolor{currentstroke}{rgb}{0.275191,0.194905,0.496005}%
\pgfsetstrokecolor{currentstroke}%
\pgfsetdash{}{0pt}%
\pgfpathmoveto{\pgfqpoint{8.208125in}{2.703742in}}%
\pgfpathlineto{\pgfqpoint{8.157973in}{2.703639in}}%
\pgfusepath{stroke}%
\end{pgfscope}%
\begin{pgfscope}%
\pgfpathrectangle{\pgfqpoint{6.720588in}{1.750000in}}{\pgfqpoint{2.279412in}{2.004545in}}%
\pgfusepath{clip}%
\pgfsetbuttcap%
\pgfsetroundjoin%
\pgfsetlinewidth{0.652734pt}%
\definecolor{currentstroke}{rgb}{0.253935,0.265254,0.529983}%
\pgfsetstrokecolor{currentstroke}%
\pgfsetdash{}{0pt}%
\pgfpathmoveto{\pgfqpoint{8.157973in}{2.703639in}}%
\pgfpathlineto{\pgfqpoint{8.107822in}{2.703477in}}%
\pgfusepath{stroke}%
\end{pgfscope}%
\begin{pgfscope}%
\pgfpathrectangle{\pgfqpoint{6.720588in}{1.750000in}}{\pgfqpoint{2.279412in}{2.004545in}}%
\pgfusepath{clip}%
\pgfsetbuttcap%
\pgfsetroundjoin%
\pgfsetlinewidth{0.755468pt}%
\definecolor{currentstroke}{rgb}{0.225863,0.330805,0.547314}%
\pgfsetstrokecolor{currentstroke}%
\pgfsetdash{}{0pt}%
\pgfpathmoveto{\pgfqpoint{8.107822in}{2.703477in}}%
\pgfpathlineto{\pgfqpoint{8.057670in}{2.703430in}}%
\pgfusepath{stroke}%
\end{pgfscope}%
\begin{pgfscope}%
\pgfpathrectangle{\pgfqpoint{6.720588in}{1.750000in}}{\pgfqpoint{2.279412in}{2.004545in}}%
\pgfusepath{clip}%
\pgfsetbuttcap%
\pgfsetroundjoin%
\pgfsetlinewidth{0.797896pt}%
\definecolor{currentstroke}{rgb}{0.214298,0.355619,0.551184}%
\pgfsetstrokecolor{currentstroke}%
\pgfsetdash{}{0pt}%
\pgfpathmoveto{\pgfqpoint{8.057670in}{2.703430in}}%
\pgfpathlineto{\pgfqpoint{8.007519in}{2.703538in}}%
\pgfusepath{stroke}%
\end{pgfscope}%
\begin{pgfscope}%
\pgfpathrectangle{\pgfqpoint{6.720588in}{1.750000in}}{\pgfqpoint{2.279412in}{2.004545in}}%
\pgfusepath{clip}%
\pgfsetbuttcap%
\pgfsetroundjoin%
\pgfsetlinewidth{0.862616pt}%
\definecolor{currentstroke}{rgb}{0.197636,0.391528,0.554969}%
\pgfsetstrokecolor{currentstroke}%
\pgfsetdash{}{0pt}%
\pgfpathmoveto{\pgfqpoint{8.007519in}{2.703538in}}%
\pgfpathlineto{\pgfqpoint{7.957368in}{2.703680in}}%
\pgfusepath{stroke}%
\end{pgfscope}%
\begin{pgfscope}%
\pgfpathrectangle{\pgfqpoint{6.720588in}{1.750000in}}{\pgfqpoint{2.279412in}{2.004545in}}%
\pgfusepath{clip}%
\pgfsetbuttcap%
\pgfsetroundjoin%
\pgfsetlinewidth{0.860686pt}%
\definecolor{currentstroke}{rgb}{0.197636,0.391528,0.554969}%
\pgfsetstrokecolor{currentstroke}%
\pgfsetdash{}{0pt}%
\pgfpathmoveto{\pgfqpoint{7.957368in}{2.703680in}}%
\pgfpathlineto{\pgfqpoint{7.907218in}{2.703940in}}%
\pgfusepath{stroke}%
\end{pgfscope}%
\begin{pgfscope}%
\pgfpathrectangle{\pgfqpoint{6.720588in}{1.750000in}}{\pgfqpoint{2.279412in}{2.004545in}}%
\pgfusepath{clip}%
\pgfsetbuttcap%
\pgfsetroundjoin%
\pgfsetlinewidth{0.864542pt}%
\definecolor{currentstroke}{rgb}{0.195860,0.395433,0.555276}%
\pgfsetstrokecolor{currentstroke}%
\pgfsetdash{}{0pt}%
\pgfpathmoveto{\pgfqpoint{7.907218in}{2.703940in}}%
\pgfpathlineto{\pgfqpoint{7.857068in}{2.704364in}}%
\pgfusepath{stroke}%
\end{pgfscope}%
\begin{pgfscope}%
\pgfpathrectangle{\pgfqpoint{6.720588in}{1.750000in}}{\pgfqpoint{2.279412in}{2.004545in}}%
\pgfusepath{clip}%
\pgfsetbuttcap%
\pgfsetroundjoin%
\pgfsetlinewidth{0.788148pt}%
\definecolor{currentstroke}{rgb}{0.218130,0.347432,0.550038}%
\pgfsetstrokecolor{currentstroke}%
\pgfsetdash{}{0pt}%
\pgfpathmoveto{\pgfqpoint{7.857068in}{2.704364in}}%
\pgfpathlineto{\pgfqpoint{7.806925in}{2.705080in}}%
\pgfusepath{stroke}%
\end{pgfscope}%
\begin{pgfscope}%
\pgfpathrectangle{\pgfqpoint{6.720588in}{1.750000in}}{\pgfqpoint{2.279412in}{2.004545in}}%
\pgfusepath{clip}%
\pgfsetbuttcap%
\pgfsetroundjoin%
\pgfsetlinewidth{0.800799pt}%
\definecolor{currentstroke}{rgb}{0.214298,0.355619,0.551184}%
\pgfsetstrokecolor{currentstroke}%
\pgfsetdash{}{0pt}%
\pgfpathmoveto{\pgfqpoint{7.806925in}{2.705080in}}%
\pgfpathlineto{\pgfqpoint{7.756788in}{2.706070in}}%
\pgfusepath{stroke}%
\end{pgfscope}%
\begin{pgfscope}%
\pgfpathrectangle{\pgfqpoint{6.720588in}{1.750000in}}{\pgfqpoint{2.279412in}{2.004545in}}%
\pgfusepath{clip}%
\pgfsetbuttcap%
\pgfsetroundjoin%
\pgfsetlinewidth{0.724227pt}%
\definecolor{currentstroke}{rgb}{0.235526,0.309527,0.542944}%
\pgfsetstrokecolor{currentstroke}%
\pgfsetdash{}{0pt}%
\pgfpathmoveto{\pgfqpoint{7.756788in}{2.706070in}}%
\pgfpathlineto{\pgfqpoint{7.706660in}{2.707315in}}%
\pgfusepath{stroke}%
\end{pgfscope}%
\begin{pgfscope}%
\pgfpathrectangle{\pgfqpoint{6.720588in}{1.750000in}}{\pgfqpoint{2.279412in}{2.004545in}}%
\pgfusepath{clip}%
\pgfsetbuttcap%
\pgfsetroundjoin%
\pgfsetlinewidth{0.648490pt}%
\definecolor{currentstroke}{rgb}{0.255645,0.260703,0.528312}%
\pgfsetstrokecolor{currentstroke}%
\pgfsetdash{}{0pt}%
\pgfpathmoveto{\pgfqpoint{7.706660in}{2.707315in}}%
\pgfpathlineto{\pgfqpoint{7.656547in}{2.708992in}}%
\pgfusepath{stroke}%
\end{pgfscope}%
\begin{pgfscope}%
\pgfpathrectangle{\pgfqpoint{6.720588in}{1.750000in}}{\pgfqpoint{2.279412in}{2.004545in}}%
\pgfusepath{clip}%
\pgfsetbuttcap%
\pgfsetroundjoin%
\pgfsetlinewidth{0.578755pt}%
\definecolor{currentstroke}{rgb}{0.270595,0.214069,0.507052}%
\pgfsetstrokecolor{currentstroke}%
\pgfsetdash{}{0pt}%
\pgfpathmoveto{\pgfqpoint{7.656547in}{2.708992in}}%
\pgfpathlineto{\pgfqpoint{7.606452in}{2.711047in}}%
\pgfusepath{stroke}%
\end{pgfscope}%
\begin{pgfscope}%
\pgfpathrectangle{\pgfqpoint{6.720588in}{1.750000in}}{\pgfqpoint{2.279412in}{2.004545in}}%
\pgfusepath{clip}%
\pgfsetbuttcap%
\pgfsetroundjoin%
\pgfsetlinewidth{0.577729pt}%
\definecolor{currentstroke}{rgb}{0.270595,0.214069,0.507052}%
\pgfsetstrokecolor{currentstroke}%
\pgfsetdash{}{0pt}%
\pgfpathmoveto{\pgfqpoint{7.606452in}{2.711047in}}%
\pgfpathlineto{\pgfqpoint{7.556364in}{2.713182in}}%
\pgfusepath{stroke}%
\end{pgfscope}%
\begin{pgfscope}%
\pgfpathrectangle{\pgfqpoint{6.720588in}{1.750000in}}{\pgfqpoint{2.279412in}{2.004545in}}%
\pgfusepath{clip}%
\pgfsetbuttcap%
\pgfsetroundjoin%
\pgfsetlinewidth{0.539850pt}%
\definecolor{currentstroke}{rgb}{0.277134,0.185228,0.489898}%
\pgfsetstrokecolor{currentstroke}%
\pgfsetdash{}{0pt}%
\pgfpathmoveto{\pgfqpoint{7.556364in}{2.713182in}}%
\pgfpathlineto{\pgfqpoint{7.506310in}{2.715701in}}%
\pgfusepath{stroke}%
\end{pgfscope}%
\begin{pgfscope}%
\pgfpathrectangle{\pgfqpoint{6.720588in}{1.750000in}}{\pgfqpoint{2.279412in}{2.004545in}}%
\pgfusepath{clip}%
\pgfsetbuttcap%
\pgfsetroundjoin%
\pgfsetlinewidth{0.316636pt}%
\definecolor{currentstroke}{rgb}{0.269944,0.014625,0.341379}%
\pgfsetstrokecolor{currentstroke}%
\pgfsetdash{}{0pt}%
\pgfpathmoveto{\pgfqpoint{8.629673in}{2.752273in}}%
\pgfpathlineto{\pgfqpoint{8.579524in}{2.751819in}}%
\pgfusepath{stroke}%
\end{pgfscope}%
\begin{pgfscope}%
\pgfpathrectangle{\pgfqpoint{6.720588in}{1.750000in}}{\pgfqpoint{2.279412in}{2.004545in}}%
\pgfusepath{clip}%
\pgfsetbuttcap%
\pgfsetroundjoin%
\pgfsetlinewidth{0.313765pt}%
\definecolor{currentstroke}{rgb}{0.268510,0.009605,0.335427}%
\pgfsetstrokecolor{currentstroke}%
\pgfsetdash{}{0pt}%
\pgfpathmoveto{\pgfqpoint{8.579524in}{2.751819in}}%
\pgfpathlineto{\pgfqpoint{8.529432in}{2.751245in}}%
\pgfusepath{stroke}%
\end{pgfscope}%
\begin{pgfscope}%
\pgfpathrectangle{\pgfqpoint{6.720588in}{1.750000in}}{\pgfqpoint{2.279412in}{2.004545in}}%
\pgfusepath{clip}%
\pgfsetbuttcap%
\pgfsetroundjoin%
\pgfsetlinewidth{0.320566pt}%
\definecolor{currentstroke}{rgb}{0.269944,0.014625,0.341379}%
\pgfsetstrokecolor{currentstroke}%
\pgfsetdash{}{0pt}%
\pgfpathmoveto{\pgfqpoint{8.529432in}{2.751245in}}%
\pgfpathlineto{\pgfqpoint{8.479338in}{2.750894in}}%
\pgfusepath{stroke}%
\end{pgfscope}%
\begin{pgfscope}%
\pgfpathrectangle{\pgfqpoint{6.720588in}{1.750000in}}{\pgfqpoint{2.279412in}{2.004545in}}%
\pgfusepath{clip}%
\pgfsetbuttcap%
\pgfsetroundjoin%
\pgfsetlinewidth{0.329863pt}%
\definecolor{currentstroke}{rgb}{0.272594,0.025563,0.353093}%
\pgfsetstrokecolor{currentstroke}%
\pgfsetdash{}{0pt}%
\pgfpathmoveto{\pgfqpoint{8.479338in}{2.750894in}}%
\pgfpathlineto{\pgfqpoint{8.429195in}{2.750316in}}%
\pgfusepath{stroke}%
\end{pgfscope}%
\begin{pgfscope}%
\pgfpathrectangle{\pgfqpoint{6.720588in}{1.750000in}}{\pgfqpoint{2.279412in}{2.004545in}}%
\pgfusepath{clip}%
\pgfsetbuttcap%
\pgfsetroundjoin%
\pgfsetlinewidth{0.338714pt}%
\definecolor{currentstroke}{rgb}{0.273809,0.031497,0.358853}%
\pgfsetstrokecolor{currentstroke}%
\pgfsetdash{}{0pt}%
\pgfpathmoveto{\pgfqpoint{8.429195in}{2.750316in}}%
\pgfpathlineto{\pgfqpoint{8.379051in}{2.749677in}}%
\pgfusepath{stroke}%
\end{pgfscope}%
\begin{pgfscope}%
\pgfpathrectangle{\pgfqpoint{6.720588in}{1.750000in}}{\pgfqpoint{2.279412in}{2.004545in}}%
\pgfusepath{clip}%
\pgfsetbuttcap%
\pgfsetroundjoin%
\pgfsetlinewidth{0.361450pt}%
\definecolor{currentstroke}{rgb}{0.277018,0.050344,0.375715}%
\pgfsetstrokecolor{currentstroke}%
\pgfsetdash{}{0pt}%
\pgfpathmoveto{\pgfqpoint{8.379051in}{2.749677in}}%
\pgfpathlineto{\pgfqpoint{8.328900in}{2.749490in}}%
\pgfusepath{stroke}%
\end{pgfscope}%
\begin{pgfscope}%
\pgfpathrectangle{\pgfqpoint{6.720588in}{1.750000in}}{\pgfqpoint{2.279412in}{2.004545in}}%
\pgfusepath{clip}%
\pgfsetbuttcap%
\pgfsetroundjoin%
\pgfsetlinewidth{0.405281pt}%
\definecolor{currentstroke}{rgb}{0.281924,0.089666,0.412415}%
\pgfsetstrokecolor{currentstroke}%
\pgfsetdash{}{0pt}%
\pgfpathmoveto{\pgfqpoint{8.328900in}{2.749490in}}%
\pgfpathlineto{\pgfqpoint{8.278748in}{2.749340in}}%
\pgfusepath{stroke}%
\end{pgfscope}%
\begin{pgfscope}%
\pgfpathrectangle{\pgfqpoint{6.720588in}{1.750000in}}{\pgfqpoint{2.279412in}{2.004545in}}%
\pgfusepath{clip}%
\pgfsetbuttcap%
\pgfsetroundjoin%
\pgfsetlinewidth{0.445045pt}%
\definecolor{currentstroke}{rgb}{0.283197,0.115680,0.436115}%
\pgfsetstrokecolor{currentstroke}%
\pgfsetdash{}{0pt}%
\pgfpathmoveto{\pgfqpoint{8.278748in}{2.749340in}}%
\pgfpathlineto{\pgfqpoint{8.228597in}{2.749154in}}%
\pgfusepath{stroke}%
\end{pgfscope}%
\begin{pgfscope}%
\pgfpathrectangle{\pgfqpoint{6.720588in}{1.750000in}}{\pgfqpoint{2.279412in}{2.004545in}}%
\pgfusepath{clip}%
\pgfsetbuttcap%
\pgfsetroundjoin%
\pgfsetlinewidth{0.510426pt}%
\definecolor{currentstroke}{rgb}{0.280255,0.165693,0.476498}%
\pgfsetstrokecolor{currentstroke}%
\pgfsetdash{}{0pt}%
\pgfpathmoveto{\pgfqpoint{8.228597in}{2.749154in}}%
\pgfpathlineto{\pgfqpoint{8.178447in}{2.748821in}}%
\pgfusepath{stroke}%
\end{pgfscope}%
\begin{pgfscope}%
\pgfpathrectangle{\pgfqpoint{6.720588in}{1.750000in}}{\pgfqpoint{2.279412in}{2.004545in}}%
\pgfusepath{clip}%
\pgfsetbuttcap%
\pgfsetroundjoin%
\pgfsetlinewidth{0.579044pt}%
\definecolor{currentstroke}{rgb}{0.270595,0.214069,0.507052}%
\pgfsetstrokecolor{currentstroke}%
\pgfsetdash{}{0pt}%
\pgfpathmoveto{\pgfqpoint{8.178447in}{2.748821in}}%
\pgfpathlineto{\pgfqpoint{8.128298in}{2.748388in}}%
\pgfusepath{stroke}%
\end{pgfscope}%
\begin{pgfscope}%
\pgfpathrectangle{\pgfqpoint{6.720588in}{1.750000in}}{\pgfqpoint{2.279412in}{2.004545in}}%
\pgfusepath{clip}%
\pgfsetbuttcap%
\pgfsetroundjoin%
\pgfsetlinewidth{0.702240pt}%
\definecolor{currentstroke}{rgb}{0.241237,0.296485,0.539709}%
\pgfsetstrokecolor{currentstroke}%
\pgfsetdash{}{0pt}%
\pgfpathmoveto{\pgfqpoint{8.128298in}{2.748388in}}%
\pgfpathlineto{\pgfqpoint{8.078148in}{2.748026in}}%
\pgfusepath{stroke}%
\end{pgfscope}%
\begin{pgfscope}%
\pgfpathrectangle{\pgfqpoint{6.720588in}{1.750000in}}{\pgfqpoint{2.279412in}{2.004545in}}%
\pgfusepath{clip}%
\pgfsetbuttcap%
\pgfsetroundjoin%
\pgfsetlinewidth{0.778730pt}%
\definecolor{currentstroke}{rgb}{0.220057,0.343307,0.549413}%
\pgfsetstrokecolor{currentstroke}%
\pgfsetdash{}{0pt}%
\pgfpathmoveto{\pgfqpoint{8.078148in}{2.748026in}}%
\pgfpathlineto{\pgfqpoint{8.027999in}{2.747558in}}%
\pgfusepath{stroke}%
\end{pgfscope}%
\begin{pgfscope}%
\pgfpathrectangle{\pgfqpoint{6.720588in}{1.750000in}}{\pgfqpoint{2.279412in}{2.004545in}}%
\pgfusepath{clip}%
\pgfsetbuttcap%
\pgfsetroundjoin%
\pgfsetlinewidth{0.816439pt}%
\definecolor{currentstroke}{rgb}{0.208623,0.367752,0.552675}%
\pgfsetstrokecolor{currentstroke}%
\pgfsetdash{}{0pt}%
\pgfpathmoveto{\pgfqpoint{8.027999in}{2.747558in}}%
\pgfpathlineto{\pgfqpoint{7.977852in}{2.746940in}}%
\pgfusepath{stroke}%
\end{pgfscope}%
\begin{pgfscope}%
\pgfpathrectangle{\pgfqpoint{6.720588in}{1.750000in}}{\pgfqpoint{2.279412in}{2.004545in}}%
\pgfusepath{clip}%
\pgfsetbuttcap%
\pgfsetroundjoin%
\pgfsetlinewidth{0.875187pt}%
\definecolor{currentstroke}{rgb}{0.194100,0.399323,0.555565}%
\pgfsetstrokecolor{currentstroke}%
\pgfsetdash{}{0pt}%
\pgfpathmoveto{\pgfqpoint{7.977852in}{2.746940in}}%
\pgfpathlineto{\pgfqpoint{7.927706in}{2.746325in}}%
\pgfusepath{stroke}%
\end{pgfscope}%
\begin{pgfscope}%
\pgfpathrectangle{\pgfqpoint{6.720588in}{1.750000in}}{\pgfqpoint{2.279412in}{2.004545in}}%
\pgfusepath{clip}%
\pgfsetbuttcap%
\pgfsetroundjoin%
\pgfsetlinewidth{0.864321pt}%
\definecolor{currentstroke}{rgb}{0.195860,0.395433,0.555276}%
\pgfsetstrokecolor{currentstroke}%
\pgfsetdash{}{0pt}%
\pgfpathmoveto{\pgfqpoint{7.927706in}{2.746325in}}%
\pgfpathlineto{\pgfqpoint{7.877560in}{2.745705in}}%
\pgfusepath{stroke}%
\end{pgfscope}%
\begin{pgfscope}%
\pgfpathrectangle{\pgfqpoint{6.720588in}{1.750000in}}{\pgfqpoint{2.279412in}{2.004545in}}%
\pgfusepath{clip}%
\pgfsetbuttcap%
\pgfsetroundjoin%
\pgfsetlinewidth{0.824265pt}%
\definecolor{currentstroke}{rgb}{0.206756,0.371758,0.553117}%
\pgfsetstrokecolor{currentstroke}%
\pgfsetdash{}{0pt}%
\pgfpathmoveto{\pgfqpoint{7.877560in}{2.745705in}}%
\pgfpathlineto{\pgfqpoint{7.827416in}{2.745033in}}%
\pgfusepath{stroke}%
\end{pgfscope}%
\begin{pgfscope}%
\pgfpathrectangle{\pgfqpoint{6.720588in}{1.750000in}}{\pgfqpoint{2.279412in}{2.004545in}}%
\pgfusepath{clip}%
\pgfsetbuttcap%
\pgfsetroundjoin%
\pgfsetlinewidth{0.809475pt}%
\definecolor{currentstroke}{rgb}{0.210503,0.363727,0.552206}%
\pgfsetstrokecolor{currentstroke}%
\pgfsetdash{}{0pt}%
\pgfpathmoveto{\pgfqpoint{7.827416in}{2.745033in}}%
\pgfpathlineto{\pgfqpoint{7.777272in}{2.744276in}}%
\pgfusepath{stroke}%
\end{pgfscope}%
\begin{pgfscope}%
\pgfpathrectangle{\pgfqpoint{6.720588in}{1.750000in}}{\pgfqpoint{2.279412in}{2.004545in}}%
\pgfusepath{clip}%
\pgfsetbuttcap%
\pgfsetroundjoin%
\pgfsetlinewidth{0.723136pt}%
\definecolor{currentstroke}{rgb}{0.235526,0.309527,0.542944}%
\pgfsetstrokecolor{currentstroke}%
\pgfsetdash{}{0pt}%
\pgfpathmoveto{\pgfqpoint{7.777272in}{2.744276in}}%
\pgfpathlineto{\pgfqpoint{7.727151in}{2.743271in}}%
\pgfusepath{stroke}%
\end{pgfscope}%
\begin{pgfscope}%
\pgfpathrectangle{\pgfqpoint{6.720588in}{1.750000in}}{\pgfqpoint{2.279412in}{2.004545in}}%
\pgfusepath{clip}%
\pgfsetbuttcap%
\pgfsetroundjoin%
\pgfsetlinewidth{0.659719pt}%
\definecolor{currentstroke}{rgb}{0.252194,0.269783,0.531579}%
\pgfsetstrokecolor{currentstroke}%
\pgfsetdash{}{0pt}%
\pgfpathmoveto{\pgfqpoint{7.727151in}{2.743271in}}%
\pgfpathlineto{\pgfqpoint{7.677059in}{2.741798in}}%
\pgfusepath{stroke}%
\end{pgfscope}%
\begin{pgfscope}%
\pgfpathrectangle{\pgfqpoint{6.720588in}{1.750000in}}{\pgfqpoint{2.279412in}{2.004545in}}%
\pgfusepath{clip}%
\pgfsetbuttcap%
\pgfsetroundjoin%
\pgfsetlinewidth{0.607084pt}%
\definecolor{currentstroke}{rgb}{0.265145,0.232956,0.516599}%
\pgfsetstrokecolor{currentstroke}%
\pgfsetdash{}{0pt}%
\pgfpathmoveto{\pgfqpoint{7.677059in}{2.741798in}}%
\pgfpathlineto{\pgfqpoint{7.626977in}{2.740020in}}%
\pgfusepath{stroke}%
\end{pgfscope}%
\begin{pgfscope}%
\pgfpathrectangle{\pgfqpoint{6.720588in}{1.750000in}}{\pgfqpoint{2.279412in}{2.004545in}}%
\pgfusepath{clip}%
\pgfsetbuttcap%
\pgfsetroundjoin%
\pgfsetlinewidth{0.575731pt}%
\definecolor{currentstroke}{rgb}{0.270595,0.214069,0.507052}%
\pgfsetstrokecolor{currentstroke}%
\pgfsetdash{}{0pt}%
\pgfpathmoveto{\pgfqpoint{7.626977in}{2.740020in}}%
\pgfpathlineto{\pgfqpoint{7.576907in}{2.738136in}}%
\pgfusepath{stroke}%
\end{pgfscope}%
\begin{pgfscope}%
\pgfpathrectangle{\pgfqpoint{6.720588in}{1.750000in}}{\pgfqpoint{2.279412in}{2.004545in}}%
\pgfusepath{clip}%
\pgfsetbuttcap%
\pgfsetroundjoin%
\pgfsetlinewidth{0.572076pt}%
\definecolor{currentstroke}{rgb}{0.271828,0.209303,0.504434}%
\pgfsetstrokecolor{currentstroke}%
\pgfsetdash{}{0pt}%
\pgfpathmoveto{\pgfqpoint{7.576907in}{2.738136in}}%
\pgfpathlineto{\pgfqpoint{7.526850in}{2.736300in}}%
\pgfusepath{stroke}%
\end{pgfscope}%
\begin{pgfscope}%
\pgfpathrectangle{\pgfqpoint{6.720588in}{1.750000in}}{\pgfqpoint{2.279412in}{2.004545in}}%
\pgfusepath{clip}%
\pgfsetbuttcap%
\pgfsetroundjoin%
\pgfsetlinewidth{0.507861pt}%
\definecolor{currentstroke}{rgb}{0.280255,0.165693,0.476498}%
\pgfsetstrokecolor{currentstroke}%
\pgfsetdash{}{0pt}%
\pgfpathmoveto{\pgfqpoint{7.526850in}{2.736300in}}%
\pgfpathlineto{\pgfqpoint{7.476910in}{2.735042in}}%
\pgfusepath{stroke}%
\end{pgfscope}%
\begin{pgfscope}%
\pgfpathrectangle{\pgfqpoint{6.720588in}{1.750000in}}{\pgfqpoint{2.279412in}{2.004545in}}%
\pgfusepath{clip}%
\pgfsetbuttcap%
\pgfsetroundjoin%
\pgfsetlinewidth{0.404150pt}%
\definecolor{currentstroke}{rgb}{0.281924,0.089666,0.412415}%
\pgfsetstrokecolor{currentstroke}%
\pgfsetdash{}{0pt}%
\pgfpathmoveto{\pgfqpoint{7.476910in}{2.735042in}}%
\pgfpathlineto{\pgfqpoint{7.476910in}{2.735042in}}%
\pgfusepath{stroke}%
\end{pgfscope}%
\begin{pgfscope}%
\pgfpathrectangle{\pgfqpoint{6.720588in}{1.750000in}}{\pgfqpoint{2.279412in}{2.004545in}}%
\pgfusepath{clip}%
\pgfsetbuttcap%
\pgfsetroundjoin%
\pgfsetlinewidth{0.304353pt}%
\definecolor{currentstroke}{rgb}{0.267004,0.004874,0.329415}%
\pgfsetstrokecolor{currentstroke}%
\pgfsetdash{}{0pt}%
\pgfpathmoveto{\pgfqpoint{8.629673in}{2.932700in}}%
\pgfpathlineto{\pgfqpoint{8.585252in}{2.935656in}}%
\pgfusepath{stroke}%
\end{pgfscope}%
\begin{pgfscope}%
\pgfpathrectangle{\pgfqpoint{6.720588in}{1.750000in}}{\pgfqpoint{2.279412in}{2.004545in}}%
\pgfusepath{clip}%
\pgfsetbuttcap%
\pgfsetroundjoin%
\pgfsetlinewidth{0.326856pt}%
\definecolor{currentstroke}{rgb}{0.271305,0.019942,0.347269}%
\pgfsetstrokecolor{currentstroke}%
\pgfsetdash{}{0pt}%
\pgfpathmoveto{\pgfqpoint{8.585252in}{2.935656in}}%
\pgfpathlineto{\pgfqpoint{8.540726in}{2.936257in}}%
\pgfusepath{stroke}%
\end{pgfscope}%
\begin{pgfscope}%
\pgfpathrectangle{\pgfqpoint{6.720588in}{1.750000in}}{\pgfqpoint{2.279412in}{2.004545in}}%
\pgfusepath{clip}%
\pgfsetbuttcap%
\pgfsetroundjoin%
\pgfsetlinewidth{0.320589pt}%
\definecolor{currentstroke}{rgb}{0.269944,0.014625,0.341379}%
\pgfsetstrokecolor{currentstroke}%
\pgfsetdash{}{0pt}%
\pgfpathmoveto{\pgfqpoint{8.540726in}{2.936257in}}%
\pgfpathlineto{\pgfqpoint{8.490579in}{2.936092in}}%
\pgfusepath{stroke}%
\end{pgfscope}%
\begin{pgfscope}%
\pgfpathrectangle{\pgfqpoint{6.720588in}{1.750000in}}{\pgfqpoint{2.279412in}{2.004545in}}%
\pgfusepath{clip}%
\pgfsetbuttcap%
\pgfsetroundjoin%
\pgfsetlinewidth{0.323365pt}%
\definecolor{currentstroke}{rgb}{0.271305,0.019942,0.347269}%
\pgfsetstrokecolor{currentstroke}%
\pgfsetdash{}{0pt}%
\pgfpathmoveto{\pgfqpoint{8.490579in}{2.936092in}}%
\pgfpathlineto{\pgfqpoint{8.440436in}{2.935604in}}%
\pgfusepath{stroke}%
\end{pgfscope}%
\begin{pgfscope}%
\pgfpathrectangle{\pgfqpoint{6.720588in}{1.750000in}}{\pgfqpoint{2.279412in}{2.004545in}}%
\pgfusepath{clip}%
\pgfsetbuttcap%
\pgfsetroundjoin%
\pgfsetlinewidth{0.336142pt}%
\definecolor{currentstroke}{rgb}{0.273809,0.031497,0.358853}%
\pgfsetstrokecolor{currentstroke}%
\pgfsetdash{}{0pt}%
\pgfpathmoveto{\pgfqpoint{8.440436in}{2.935604in}}%
\pgfpathlineto{\pgfqpoint{8.390295in}{2.935002in}}%
\pgfusepath{stroke}%
\end{pgfscope}%
\begin{pgfscope}%
\pgfpathrectangle{\pgfqpoint{6.720588in}{1.750000in}}{\pgfqpoint{2.279412in}{2.004545in}}%
\pgfusepath{clip}%
\pgfsetbuttcap%
\pgfsetroundjoin%
\pgfsetlinewidth{0.350147pt}%
\definecolor{currentstroke}{rgb}{0.276022,0.044167,0.370164}%
\pgfsetstrokecolor{currentstroke}%
\pgfsetdash{}{0pt}%
\pgfpathmoveto{\pgfqpoint{8.390295in}{2.935002in}}%
\pgfpathlineto{\pgfqpoint{8.340149in}{2.934580in}}%
\pgfusepath{stroke}%
\end{pgfscope}%
\begin{pgfscope}%
\pgfpathrectangle{\pgfqpoint{6.720588in}{1.750000in}}{\pgfqpoint{2.279412in}{2.004545in}}%
\pgfusepath{clip}%
\pgfsetbuttcap%
\pgfsetroundjoin%
\pgfsetlinewidth{0.391805pt}%
\definecolor{currentstroke}{rgb}{0.280894,0.078907,0.402329}%
\pgfsetstrokecolor{currentstroke}%
\pgfsetdash{}{0pt}%
\pgfpathmoveto{\pgfqpoint{8.340149in}{2.934580in}}%
\pgfpathlineto{\pgfqpoint{8.290005in}{2.933873in}}%
\pgfusepath{stroke}%
\end{pgfscope}%
\begin{pgfscope}%
\pgfpathrectangle{\pgfqpoint{6.720588in}{1.750000in}}{\pgfqpoint{2.279412in}{2.004545in}}%
\pgfusepath{clip}%
\pgfsetbuttcap%
\pgfsetroundjoin%
\pgfsetlinewidth{0.428084pt}%
\definecolor{currentstroke}{rgb}{0.282910,0.105393,0.426902}%
\pgfsetstrokecolor{currentstroke}%
\pgfsetdash{}{0pt}%
\pgfpathmoveto{\pgfqpoint{8.290005in}{2.933873in}}%
\pgfpathlineto{\pgfqpoint{8.239862in}{2.933128in}}%
\pgfusepath{stroke}%
\end{pgfscope}%
\begin{pgfscope}%
\pgfpathrectangle{\pgfqpoint{6.720588in}{1.750000in}}{\pgfqpoint{2.279412in}{2.004545in}}%
\pgfusepath{clip}%
\pgfsetbuttcap%
\pgfsetroundjoin%
\pgfsetlinewidth{0.454722pt}%
\definecolor{currentstroke}{rgb}{0.283187,0.125848,0.444960}%
\pgfsetstrokecolor{currentstroke}%
\pgfsetdash{}{0pt}%
\pgfpathmoveto{\pgfqpoint{8.239862in}{2.933128in}}%
\pgfpathlineto{\pgfqpoint{8.189721in}{2.932293in}}%
\pgfusepath{stroke}%
\end{pgfscope}%
\begin{pgfscope}%
\pgfpathrectangle{\pgfqpoint{6.720588in}{1.750000in}}{\pgfqpoint{2.279412in}{2.004545in}}%
\pgfusepath{clip}%
\pgfsetbuttcap%
\pgfsetroundjoin%
\pgfsetlinewidth{0.513485pt}%
\definecolor{currentstroke}{rgb}{0.280255,0.165693,0.476498}%
\pgfsetstrokecolor{currentstroke}%
\pgfsetdash{}{0pt}%
\pgfpathmoveto{\pgfqpoint{8.189721in}{2.932293in}}%
\pgfpathlineto{\pgfqpoint{8.139605in}{2.930678in}}%
\pgfusepath{stroke}%
\end{pgfscope}%
\begin{pgfscope}%
\pgfpathrectangle{\pgfqpoint{6.720588in}{1.750000in}}{\pgfqpoint{2.279412in}{2.004545in}}%
\pgfusepath{clip}%
\pgfsetbuttcap%
\pgfsetroundjoin%
\pgfsetlinewidth{0.568939pt}%
\definecolor{currentstroke}{rgb}{0.271828,0.209303,0.504434}%
\pgfsetstrokecolor{currentstroke}%
\pgfsetdash{}{0pt}%
\pgfpathmoveto{\pgfqpoint{8.139605in}{2.930678in}}%
\pgfpathlineto{\pgfqpoint{8.089510in}{2.928600in}}%
\pgfusepath{stroke}%
\end{pgfscope}%
\begin{pgfscope}%
\pgfpathrectangle{\pgfqpoint{6.720588in}{1.750000in}}{\pgfqpoint{2.279412in}{2.004545in}}%
\pgfusepath{clip}%
\pgfsetbuttcap%
\pgfsetroundjoin%
\pgfsetlinewidth{0.638158pt}%
\definecolor{currentstroke}{rgb}{0.257322,0.256130,0.526563}%
\pgfsetstrokecolor{currentstroke}%
\pgfsetdash{}{0pt}%
\pgfpathmoveto{\pgfqpoint{8.089510in}{2.928600in}}%
\pgfpathlineto{\pgfqpoint{8.039456in}{2.925902in}}%
\pgfusepath{stroke}%
\end{pgfscope}%
\begin{pgfscope}%
\pgfpathrectangle{\pgfqpoint{6.720588in}{1.750000in}}{\pgfqpoint{2.279412in}{2.004545in}}%
\pgfusepath{clip}%
\pgfsetbuttcap%
\pgfsetroundjoin%
\pgfsetlinewidth{0.683123pt}%
\definecolor{currentstroke}{rgb}{0.246811,0.283237,0.535941}%
\pgfsetstrokecolor{currentstroke}%
\pgfsetdash{}{0pt}%
\pgfpathmoveto{\pgfqpoint{8.039456in}{2.925902in}}%
\pgfpathlineto{\pgfqpoint{7.989456in}{2.922487in}}%
\pgfusepath{stroke}%
\end{pgfscope}%
\begin{pgfscope}%
\pgfpathrectangle{\pgfqpoint{6.720588in}{1.750000in}}{\pgfqpoint{2.279412in}{2.004545in}}%
\pgfusepath{clip}%
\pgfsetbuttcap%
\pgfsetroundjoin%
\pgfsetlinewidth{0.704879pt}%
\definecolor{currentstroke}{rgb}{0.241237,0.296485,0.539709}%
\pgfsetstrokecolor{currentstroke}%
\pgfsetdash{}{0pt}%
\pgfpathmoveto{\pgfqpoint{7.989456in}{2.922487in}}%
\pgfpathlineto{\pgfqpoint{7.939519in}{2.918458in}}%
\pgfusepath{stroke}%
\end{pgfscope}%
\begin{pgfscope}%
\pgfpathrectangle{\pgfqpoint{6.720588in}{1.750000in}}{\pgfqpoint{2.279412in}{2.004545in}}%
\pgfusepath{clip}%
\pgfsetbuttcap%
\pgfsetroundjoin%
\pgfsetlinewidth{0.723325pt}%
\definecolor{currentstroke}{rgb}{0.235526,0.309527,0.542944}%
\pgfsetstrokecolor{currentstroke}%
\pgfsetdash{}{0pt}%
\pgfpathmoveto{\pgfqpoint{7.939519in}{2.918458in}}%
\pgfpathlineto{\pgfqpoint{7.889719in}{2.913297in}}%
\pgfusepath{stroke}%
\end{pgfscope}%
\begin{pgfscope}%
\pgfpathrectangle{\pgfqpoint{6.720588in}{1.750000in}}{\pgfqpoint{2.279412in}{2.004545in}}%
\pgfusepath{clip}%
\pgfsetbuttcap%
\pgfsetroundjoin%
\pgfsetlinewidth{0.760612pt}%
\definecolor{currentstroke}{rgb}{0.225863,0.330805,0.547314}%
\pgfsetstrokecolor{currentstroke}%
\pgfsetdash{}{0pt}%
\pgfpathmoveto{\pgfqpoint{7.889719in}{2.913297in}}%
\pgfpathlineto{\pgfqpoint{7.840122in}{2.906797in}}%
\pgfusepath{stroke}%
\end{pgfscope}%
\begin{pgfscope}%
\pgfpathrectangle{\pgfqpoint{6.720588in}{1.750000in}}{\pgfqpoint{2.279412in}{2.004545in}}%
\pgfusepath{clip}%
\pgfsetbuttcap%
\pgfsetroundjoin%
\pgfsetlinewidth{0.709423pt}%
\definecolor{currentstroke}{rgb}{0.239346,0.300855,0.540844}%
\pgfsetstrokecolor{currentstroke}%
\pgfsetdash{}{0pt}%
\pgfpathmoveto{\pgfqpoint{7.840122in}{2.906797in}}%
\pgfpathlineto{\pgfqpoint{7.790797in}{2.898879in}}%
\pgfusepath{stroke}%
\end{pgfscope}%
\begin{pgfscope}%
\pgfpathrectangle{\pgfqpoint{6.720588in}{1.750000in}}{\pgfqpoint{2.279412in}{2.004545in}}%
\pgfusepath{clip}%
\pgfsetbuttcap%
\pgfsetroundjoin%
\pgfsetlinewidth{0.639896pt}%
\definecolor{currentstroke}{rgb}{0.257322,0.256130,0.526563}%
\pgfsetstrokecolor{currentstroke}%
\pgfsetdash{}{0pt}%
\pgfpathmoveto{\pgfqpoint{7.790797in}{2.898879in}}%
\pgfpathlineto{\pgfqpoint{7.741987in}{2.888870in}}%
\pgfusepath{stroke}%
\end{pgfscope}%
\begin{pgfscope}%
\pgfpathrectangle{\pgfqpoint{6.720588in}{1.750000in}}{\pgfqpoint{2.279412in}{2.004545in}}%
\pgfusepath{clip}%
\pgfsetbuttcap%
\pgfsetroundjoin%
\pgfsetlinewidth{0.635352pt}%
\definecolor{currentstroke}{rgb}{0.258965,0.251537,0.524736}%
\pgfsetstrokecolor{currentstroke}%
\pgfsetdash{}{0pt}%
\pgfpathmoveto{\pgfqpoint{7.741987in}{2.888870in}}%
\pgfpathlineto{\pgfqpoint{7.693908in}{2.876439in}}%
\pgfusepath{stroke}%
\end{pgfscope}%
\begin{pgfscope}%
\pgfpathrectangle{\pgfqpoint{6.720588in}{1.750000in}}{\pgfqpoint{2.279412in}{2.004545in}}%
\pgfusepath{clip}%
\pgfsetbuttcap%
\pgfsetroundjoin%
\pgfsetlinewidth{0.615091pt}%
\definecolor{currentstroke}{rgb}{0.263663,0.237631,0.518762}%
\pgfsetstrokecolor{currentstroke}%
\pgfsetdash{}{0pt}%
\pgfpathmoveto{\pgfqpoint{7.693908in}{2.876439in}}%
\pgfpathlineto{\pgfqpoint{7.646835in}{2.861406in}}%
\pgfusepath{stroke}%
\end{pgfscope}%
\begin{pgfscope}%
\pgfpathrectangle{\pgfqpoint{6.720588in}{1.750000in}}{\pgfqpoint{2.279412in}{2.004545in}}%
\pgfusepath{clip}%
\pgfsetbuttcap%
\pgfsetroundjoin%
\pgfsetlinewidth{0.589973pt}%
\definecolor{currentstroke}{rgb}{0.267968,0.223549,0.512008}%
\pgfsetstrokecolor{currentstroke}%
\pgfsetdash{}{0pt}%
\pgfpathmoveto{\pgfqpoint{7.646835in}{2.861406in}}%
\pgfpathlineto{\pgfqpoint{7.600691in}{2.844281in}}%
\pgfusepath{stroke}%
\end{pgfscope}%
\begin{pgfscope}%
\pgfpathrectangle{\pgfqpoint{6.720588in}{1.750000in}}{\pgfqpoint{2.279412in}{2.004545in}}%
\pgfusepath{clip}%
\pgfsetbuttcap%
\pgfsetroundjoin%
\pgfsetlinewidth{0.576546pt}%
\definecolor{currentstroke}{rgb}{0.270595,0.214069,0.507052}%
\pgfsetstrokecolor{currentstroke}%
\pgfsetdash{}{0pt}%
\pgfpathmoveto{\pgfqpoint{7.600691in}{2.844281in}}%
\pgfpathlineto{\pgfqpoint{7.554936in}{2.826423in}}%
\pgfusepath{stroke}%
\end{pgfscope}%
\begin{pgfscope}%
\pgfpathrectangle{\pgfqpoint{6.720588in}{1.750000in}}{\pgfqpoint{2.279412in}{2.004545in}}%
\pgfusepath{clip}%
\pgfsetbuttcap%
\pgfsetroundjoin%
\pgfsetlinewidth{0.568777pt}%
\definecolor{currentstroke}{rgb}{0.271828,0.209303,0.504434}%
\pgfsetstrokecolor{currentstroke}%
\pgfsetdash{}{0pt}%
\pgfpathmoveto{\pgfqpoint{7.554936in}{2.826423in}}%
\pgfpathlineto{\pgfqpoint{7.511409in}{2.805201in}}%
\pgfusepath{stroke}%
\end{pgfscope}%
\begin{pgfscope}%
\pgfpathrectangle{\pgfqpoint{6.720588in}{1.750000in}}{\pgfqpoint{2.279412in}{2.004545in}}%
\pgfusepath{clip}%
\pgfsetbuttcap%
\pgfsetroundjoin%
\pgfsetlinewidth{0.518621pt}%
\definecolor{currentstroke}{rgb}{0.279574,0.170599,0.479997}%
\pgfsetstrokecolor{currentstroke}%
\pgfsetdash{}{0pt}%
\pgfpathmoveto{\pgfqpoint{7.511409in}{2.805201in}}%
\pgfpathlineto{\pgfqpoint{7.511409in}{2.805201in}}%
\pgfusepath{stroke}%
\end{pgfscope}%
\begin{pgfscope}%
\pgfpathrectangle{\pgfqpoint{6.720588in}{1.750000in}}{\pgfqpoint{2.279412in}{2.004545in}}%
\pgfusepath{clip}%
\pgfsetbuttcap%
\pgfsetroundjoin%
\pgfsetlinewidth{0.518621pt}%
\definecolor{currentstroke}{rgb}{0.279574,0.170599,0.479997}%
\pgfsetstrokecolor{currentstroke}%
\pgfsetdash{}{0pt}%
\pgfpathmoveto{\pgfqpoint{7.511409in}{2.805201in}}%
\pgfpathlineto{\pgfqpoint{7.487222in}{2.787890in}}%
\pgfusepath{stroke}%
\end{pgfscope}%
\begin{pgfscope}%
\pgfpathrectangle{\pgfqpoint{6.720588in}{1.750000in}}{\pgfqpoint{2.279412in}{2.004545in}}%
\pgfusepath{clip}%
\pgfsetbuttcap%
\pgfsetroundjoin%
\pgfsetlinewidth{0.489911pt}%
\definecolor{currentstroke}{rgb}{0.281887,0.150881,0.465405}%
\pgfsetstrokecolor{currentstroke}%
\pgfsetdash{}{0pt}%
\pgfpathmoveto{\pgfqpoint{7.487222in}{2.787890in}}%
\pgfpathlineto{\pgfqpoint{7.487222in}{2.787890in}}%
\pgfusepath{stroke}%
\end{pgfscope}%
\begin{pgfscope}%
\pgfpathrectangle{\pgfqpoint{6.720588in}{1.750000in}}{\pgfqpoint{2.279412in}{2.004545in}}%
\pgfusepath{clip}%
\pgfsetbuttcap%
\pgfsetroundjoin%
\pgfsetlinewidth{0.321701pt}%
\definecolor{currentstroke}{rgb}{0.269944,0.014625,0.341379}%
\pgfsetstrokecolor{currentstroke}%
\pgfsetdash{}{0pt}%
\pgfpathmoveto{\pgfqpoint{7.706418in}{2.120778in}}%
\pgfpathlineto{\pgfqpoint{7.732022in}{2.154751in}}%
\pgfusepath{stroke}%
\end{pgfscope}%
\begin{pgfscope}%
\pgfpathrectangle{\pgfqpoint{6.720588in}{1.750000in}}{\pgfqpoint{2.279412in}{2.004545in}}%
\pgfusepath{clip}%
\pgfsetbuttcap%
\pgfsetroundjoin%
\pgfsetlinewidth{0.349871pt}%
\definecolor{currentstroke}{rgb}{0.276022,0.044167,0.370164}%
\pgfsetstrokecolor{currentstroke}%
\pgfsetdash{}{0pt}%
\pgfpathmoveto{\pgfqpoint{7.732022in}{2.154751in}}%
\pgfpathlineto{\pgfqpoint{7.732022in}{2.154751in}}%
\pgfusepath{stroke}%
\end{pgfscope}%
\begin{pgfscope}%
\pgfpathrectangle{\pgfqpoint{6.720588in}{1.750000in}}{\pgfqpoint{2.279412in}{2.004545in}}%
\pgfusepath{clip}%
\pgfsetbuttcap%
\pgfsetroundjoin%
\pgfsetlinewidth{0.349871pt}%
\definecolor{currentstroke}{rgb}{0.276022,0.044167,0.370164}%
\pgfsetstrokecolor{currentstroke}%
\pgfsetdash{}{0pt}%
\pgfpathmoveto{\pgfqpoint{7.732022in}{2.154751in}}%
\pgfpathlineto{\pgfqpoint{7.732022in}{2.154751in}}%
\pgfusepath{stroke}%
\end{pgfscope}%
\begin{pgfscope}%
\pgfpathrectangle{\pgfqpoint{6.720588in}{1.750000in}}{\pgfqpoint{2.279412in}{2.004545in}}%
\pgfusepath{clip}%
\pgfsetbuttcap%
\pgfsetroundjoin%
\pgfsetlinewidth{0.349871pt}%
\definecolor{currentstroke}{rgb}{0.276022,0.044167,0.370164}%
\pgfsetstrokecolor{currentstroke}%
\pgfsetdash{}{0pt}%
\pgfpathmoveto{\pgfqpoint{7.732022in}{2.154751in}}%
\pgfpathlineto{\pgfqpoint{7.736160in}{2.166925in}}%
\pgfusepath{stroke}%
\end{pgfscope}%
\begin{pgfscope}%
\pgfpathrectangle{\pgfqpoint{6.720588in}{1.750000in}}{\pgfqpoint{2.279412in}{2.004545in}}%
\pgfusepath{clip}%
\pgfsetbuttcap%
\pgfsetroundjoin%
\pgfsetlinewidth{0.353565pt}%
\definecolor{currentstroke}{rgb}{0.276022,0.044167,0.370164}%
\pgfsetstrokecolor{currentstroke}%
\pgfsetdash{}{0pt}%
\pgfpathmoveto{\pgfqpoint{7.736160in}{2.166925in}}%
\pgfpathlineto{\pgfqpoint{7.734116in}{2.179903in}}%
\pgfusepath{stroke}%
\end{pgfscope}%
\begin{pgfscope}%
\pgfpathrectangle{\pgfqpoint{6.720588in}{1.750000in}}{\pgfqpoint{2.279412in}{2.004545in}}%
\pgfusepath{clip}%
\pgfsetbuttcap%
\pgfsetroundjoin%
\pgfsetlinewidth{0.349238pt}%
\definecolor{currentstroke}{rgb}{0.276022,0.044167,0.370164}%
\pgfsetstrokecolor{currentstroke}%
\pgfsetdash{}{0pt}%
\pgfpathmoveto{\pgfqpoint{7.734116in}{2.179903in}}%
\pgfpathlineto{\pgfqpoint{7.734666in}{2.193318in}}%
\pgfusepath{stroke}%
\end{pgfscope}%
\begin{pgfscope}%
\pgfpathrectangle{\pgfqpoint{6.720588in}{1.750000in}}{\pgfqpoint{2.279412in}{2.004545in}}%
\pgfusepath{clip}%
\pgfsetbuttcap%
\pgfsetroundjoin%
\pgfsetlinewidth{0.334176pt}%
\definecolor{currentstroke}{rgb}{0.272594,0.025563,0.353093}%
\pgfsetstrokecolor{currentstroke}%
\pgfsetdash{}{0pt}%
\pgfpathmoveto{\pgfqpoint{7.734666in}{2.193318in}}%
\pgfpathlineto{\pgfqpoint{7.741644in}{2.217855in}}%
\pgfusepath{stroke}%
\end{pgfscope}%
\begin{pgfscope}%
\pgfpathrectangle{\pgfqpoint{6.720588in}{1.750000in}}{\pgfqpoint{2.279412in}{2.004545in}}%
\pgfusepath{clip}%
\pgfsetbuttcap%
\pgfsetroundjoin%
\pgfsetlinewidth{0.327938pt}%
\definecolor{currentstroke}{rgb}{0.271305,0.019942,0.347269}%
\pgfsetstrokecolor{currentstroke}%
\pgfsetdash{}{0pt}%
\pgfpathmoveto{\pgfqpoint{7.741644in}{2.217855in}}%
\pgfpathlineto{\pgfqpoint{7.741644in}{2.217855in}}%
\pgfusepath{stroke}%
\end{pgfscope}%
\begin{pgfscope}%
\pgfpathrectangle{\pgfqpoint{6.720588in}{1.750000in}}{\pgfqpoint{2.279412in}{2.004545in}}%
\pgfusepath{clip}%
\pgfsetbuttcap%
\pgfsetroundjoin%
\pgfsetlinewidth{0.327938pt}%
\definecolor{currentstroke}{rgb}{0.271305,0.019942,0.347269}%
\pgfsetstrokecolor{currentstroke}%
\pgfsetdash{}{0pt}%
\pgfpathmoveto{\pgfqpoint{7.741644in}{2.217855in}}%
\pgfpathlineto{\pgfqpoint{7.741644in}{2.217855in}}%
\pgfusepath{stroke}%
\end{pgfscope}%
\begin{pgfscope}%
\pgfpathrectangle{\pgfqpoint{6.720588in}{1.750000in}}{\pgfqpoint{2.279412in}{2.004545in}}%
\pgfusepath{clip}%
\pgfsetbuttcap%
\pgfsetroundjoin%
\pgfsetlinewidth{0.327938pt}%
\definecolor{currentstroke}{rgb}{0.271305,0.019942,0.347269}%
\pgfsetstrokecolor{currentstroke}%
\pgfsetdash{}{0pt}%
\pgfpathmoveto{\pgfqpoint{7.741644in}{2.217855in}}%
\pgfpathlineto{\pgfqpoint{7.741498in}{2.231579in}}%
\pgfusepath{stroke}%
\end{pgfscope}%
\begin{pgfscope}%
\pgfpathrectangle{\pgfqpoint{6.720588in}{1.750000in}}{\pgfqpoint{2.279412in}{2.004545in}}%
\pgfusepath{clip}%
\pgfsetbuttcap%
\pgfsetroundjoin%
\pgfsetlinewidth{0.347164pt}%
\definecolor{currentstroke}{rgb}{0.274952,0.037752,0.364543}%
\pgfsetstrokecolor{currentstroke}%
\pgfsetdash{}{0pt}%
\pgfpathmoveto{\pgfqpoint{7.741498in}{2.231579in}}%
\pgfpathlineto{\pgfqpoint{7.737884in}{2.244851in}}%
\pgfusepath{stroke}%
\end{pgfscope}%
\begin{pgfscope}%
\pgfpathrectangle{\pgfqpoint{6.720588in}{1.750000in}}{\pgfqpoint{2.279412in}{2.004545in}}%
\pgfusepath{clip}%
\pgfsetbuttcap%
\pgfsetroundjoin%
\pgfsetlinewidth{0.357764pt}%
\definecolor{currentstroke}{rgb}{0.277018,0.050344,0.375715}%
\pgfsetstrokecolor{currentstroke}%
\pgfsetdash{}{0pt}%
\pgfpathmoveto{\pgfqpoint{7.737884in}{2.244851in}}%
\pgfpathlineto{\pgfqpoint{7.731615in}{2.286213in}}%
\pgfusepath{stroke}%
\end{pgfscope}%
\begin{pgfscope}%
\pgfpathrectangle{\pgfqpoint{6.720588in}{1.750000in}}{\pgfqpoint{2.279412in}{2.004545in}}%
\pgfusepath{clip}%
\pgfsetbuttcap%
\pgfsetroundjoin%
\pgfsetlinewidth{0.357719pt}%
\definecolor{currentstroke}{rgb}{0.277018,0.050344,0.375715}%
\pgfsetstrokecolor{currentstroke}%
\pgfsetdash{}{0pt}%
\pgfpathmoveto{\pgfqpoint{7.731615in}{2.286213in}}%
\pgfpathlineto{\pgfqpoint{7.731615in}{2.286213in}}%
\pgfusepath{stroke}%
\end{pgfscope}%
\begin{pgfscope}%
\pgfpathrectangle{\pgfqpoint{6.720588in}{1.750000in}}{\pgfqpoint{2.279412in}{2.004545in}}%
\pgfusepath{clip}%
\pgfsetbuttcap%
\pgfsetroundjoin%
\pgfsetlinewidth{0.357719pt}%
\definecolor{currentstroke}{rgb}{0.277018,0.050344,0.375715}%
\pgfsetstrokecolor{currentstroke}%
\pgfsetdash{}{0pt}%
\pgfpathmoveto{\pgfqpoint{7.731615in}{2.286213in}}%
\pgfpathlineto{\pgfqpoint{7.720347in}{2.319284in}}%
\pgfusepath{stroke}%
\end{pgfscope}%
\begin{pgfscope}%
\pgfpathrectangle{\pgfqpoint{6.720588in}{1.750000in}}{\pgfqpoint{2.279412in}{2.004545in}}%
\pgfusepath{clip}%
\pgfsetbuttcap%
\pgfsetroundjoin%
\pgfsetlinewidth{0.433096pt}%
\definecolor{currentstroke}{rgb}{0.283091,0.110553,0.431554}%
\pgfsetstrokecolor{currentstroke}%
\pgfsetdash{}{0pt}%
\pgfpathmoveto{\pgfqpoint{7.720347in}{2.319284in}}%
\pgfpathlineto{\pgfqpoint{7.707634in}{2.348865in}}%
\pgfusepath{stroke}%
\end{pgfscope}%
\begin{pgfscope}%
\pgfpathrectangle{\pgfqpoint{6.720588in}{1.750000in}}{\pgfqpoint{2.279412in}{2.004545in}}%
\pgfusepath{clip}%
\pgfsetbuttcap%
\pgfsetroundjoin%
\pgfsetlinewidth{0.429444pt}%
\definecolor{currentstroke}{rgb}{0.282910,0.105393,0.426902}%
\pgfsetstrokecolor{currentstroke}%
\pgfsetdash{}{0pt}%
\pgfpathmoveto{\pgfqpoint{7.707634in}{2.348865in}}%
\pgfpathlineto{\pgfqpoint{7.688073in}{2.388487in}}%
\pgfusepath{stroke}%
\end{pgfscope}%
\begin{pgfscope}%
\pgfpathrectangle{\pgfqpoint{6.720588in}{1.750000in}}{\pgfqpoint{2.279412in}{2.004545in}}%
\pgfusepath{clip}%
\pgfsetbuttcap%
\pgfsetroundjoin%
\pgfsetlinewidth{0.481633pt}%
\definecolor{currentstroke}{rgb}{0.282290,0.145912,0.461510}%
\pgfsetstrokecolor{currentstroke}%
\pgfsetdash{}{0pt}%
\pgfpathmoveto{\pgfqpoint{7.688073in}{2.388487in}}%
\pgfpathlineto{\pgfqpoint{7.667373in}{2.427962in}}%
\pgfusepath{stroke}%
\end{pgfscope}%
\begin{pgfscope}%
\pgfpathrectangle{\pgfqpoint{6.720588in}{1.750000in}}{\pgfqpoint{2.279412in}{2.004545in}}%
\pgfusepath{clip}%
\pgfsetbuttcap%
\pgfsetroundjoin%
\pgfsetlinewidth{0.511164pt}%
\definecolor{currentstroke}{rgb}{0.280255,0.165693,0.476498}%
\pgfsetstrokecolor{currentstroke}%
\pgfsetdash{}{0pt}%
\pgfpathmoveto{\pgfqpoint{7.667373in}{2.427962in}}%
\pgfpathlineto{\pgfqpoint{7.642207in}{2.465707in}}%
\pgfusepath{stroke}%
\end{pgfscope}%
\begin{pgfscope}%
\pgfpathrectangle{\pgfqpoint{6.720588in}{1.750000in}}{\pgfqpoint{2.279412in}{2.004545in}}%
\pgfusepath{clip}%
\pgfsetbuttcap%
\pgfsetroundjoin%
\pgfsetlinewidth{0.616603pt}%
\definecolor{currentstroke}{rgb}{0.263663,0.237631,0.518762}%
\pgfsetstrokecolor{currentstroke}%
\pgfsetdash{}{0pt}%
\pgfpathmoveto{\pgfqpoint{7.642207in}{2.465707in}}%
\pgfpathlineto{\pgfqpoint{7.614331in}{2.502208in}}%
\pgfusepath{stroke}%
\end{pgfscope}%
\begin{pgfscope}%
\pgfpathrectangle{\pgfqpoint{6.720588in}{1.750000in}}{\pgfqpoint{2.279412in}{2.004545in}}%
\pgfusepath{clip}%
\pgfsetbuttcap%
\pgfsetroundjoin%
\pgfsetlinewidth{0.632848pt}%
\definecolor{currentstroke}{rgb}{0.258965,0.251537,0.524736}%
\pgfsetstrokecolor{currentstroke}%
\pgfsetdash{}{0pt}%
\pgfpathmoveto{\pgfqpoint{7.614331in}{2.502208in}}%
\pgfpathlineto{\pgfqpoint{7.585263in}{2.538042in}}%
\pgfusepath{stroke}%
\end{pgfscope}%
\begin{pgfscope}%
\pgfpathrectangle{\pgfqpoint{6.720588in}{1.750000in}}{\pgfqpoint{2.279412in}{2.004545in}}%
\pgfusepath{clip}%
\pgfsetbuttcap%
\pgfsetroundjoin%
\pgfsetlinewidth{0.637716pt}%
\definecolor{currentstroke}{rgb}{0.257322,0.256130,0.526563}%
\pgfsetstrokecolor{currentstroke}%
\pgfsetdash{}{0pt}%
\pgfpathmoveto{\pgfqpoint{7.585263in}{2.538042in}}%
\pgfpathlineto{\pgfqpoint{7.555448in}{2.573221in}}%
\pgfusepath{stroke}%
\end{pgfscope}%
\begin{pgfscope}%
\pgfpathrectangle{\pgfqpoint{6.720588in}{1.750000in}}{\pgfqpoint{2.279412in}{2.004545in}}%
\pgfusepath{clip}%
\pgfsetbuttcap%
\pgfsetroundjoin%
\pgfsetlinewidth{0.624247pt}%
\definecolor{currentstroke}{rgb}{0.260571,0.246922,0.522828}%
\pgfsetstrokecolor{currentstroke}%
\pgfsetdash{}{0pt}%
\pgfpathmoveto{\pgfqpoint{7.555448in}{2.573221in}}%
\pgfpathlineto{\pgfqpoint{7.524440in}{2.607475in}}%
\pgfusepath{stroke}%
\end{pgfscope}%
\begin{pgfscope}%
\pgfpathrectangle{\pgfqpoint{6.720588in}{1.750000in}}{\pgfqpoint{2.279412in}{2.004545in}}%
\pgfusepath{clip}%
\pgfsetbuttcap%
\pgfsetroundjoin%
\pgfsetlinewidth{0.315579pt}%
\definecolor{currentstroke}{rgb}{0.269944,0.014625,0.341379}%
\pgfsetstrokecolor{currentstroke}%
\pgfsetdash{}{0pt}%
\pgfpathmoveto{\pgfqpoint{8.416169in}{2.099273in}}%
\pgfpathlineto{\pgfqpoint{8.366169in}{2.101967in}}%
\pgfusepath{stroke}%
\end{pgfscope}%
\begin{pgfscope}%
\pgfpathrectangle{\pgfqpoint{6.720588in}{1.750000in}}{\pgfqpoint{2.279412in}{2.004545in}}%
\pgfusepath{clip}%
\pgfsetbuttcap%
\pgfsetroundjoin%
\pgfsetlinewidth{0.321765pt}%
\definecolor{currentstroke}{rgb}{0.271305,0.019942,0.347269}%
\pgfsetstrokecolor{currentstroke}%
\pgfsetdash{}{0pt}%
\pgfpathmoveto{\pgfqpoint{8.366169in}{2.101967in}}%
\pgfpathlineto{\pgfqpoint{8.316094in}{2.103828in}}%
\pgfusepath{stroke}%
\end{pgfscope}%
\begin{pgfscope}%
\pgfpathrectangle{\pgfqpoint{6.720588in}{1.750000in}}{\pgfqpoint{2.279412in}{2.004545in}}%
\pgfusepath{clip}%
\pgfsetbuttcap%
\pgfsetroundjoin%
\pgfsetlinewidth{0.327453pt}%
\definecolor{currentstroke}{rgb}{0.271305,0.019942,0.347269}%
\pgfsetstrokecolor{currentstroke}%
\pgfsetdash{}{0pt}%
\pgfpathmoveto{\pgfqpoint{8.316094in}{2.103828in}}%
\pgfpathlineto{\pgfqpoint{8.266084in}{2.105953in}}%
\pgfusepath{stroke}%
\end{pgfscope}%
\begin{pgfscope}%
\pgfpathrectangle{\pgfqpoint{6.720588in}{1.750000in}}{\pgfqpoint{2.279412in}{2.004545in}}%
\pgfusepath{clip}%
\pgfsetbuttcap%
\pgfsetroundjoin%
\pgfsetlinewidth{0.328680pt}%
\definecolor{currentstroke}{rgb}{0.272594,0.025563,0.353093}%
\pgfsetstrokecolor{currentstroke}%
\pgfsetdash{}{0pt}%
\pgfpathmoveto{\pgfqpoint{8.266084in}{2.105953in}}%
\pgfpathlineto{\pgfqpoint{8.216296in}{2.110661in}}%
\pgfusepath{stroke}%
\end{pgfscope}%
\begin{pgfscope}%
\pgfpathrectangle{\pgfqpoint{6.720588in}{1.750000in}}{\pgfqpoint{2.279412in}{2.004545in}}%
\pgfusepath{clip}%
\pgfsetbuttcap%
\pgfsetroundjoin%
\pgfsetlinewidth{0.342919pt}%
\definecolor{currentstroke}{rgb}{0.274952,0.037752,0.364543}%
\pgfsetstrokecolor{currentstroke}%
\pgfsetdash{}{0pt}%
\pgfpathmoveto{\pgfqpoint{8.216296in}{2.110661in}}%
\pgfpathlineto{\pgfqpoint{8.166482in}{2.115449in}}%
\pgfusepath{stroke}%
\end{pgfscope}%
\begin{pgfscope}%
\pgfpathrectangle{\pgfqpoint{6.720588in}{1.750000in}}{\pgfqpoint{2.279412in}{2.004545in}}%
\pgfusepath{clip}%
\pgfsetbuttcap%
\pgfsetroundjoin%
\pgfsetlinewidth{0.352025pt}%
\definecolor{currentstroke}{rgb}{0.276022,0.044167,0.370164}%
\pgfsetstrokecolor{currentstroke}%
\pgfsetdash{}{0pt}%
\pgfpathmoveto{\pgfqpoint{8.166482in}{2.115449in}}%
\pgfpathlineto{\pgfqpoint{8.116754in}{2.120778in}}%
\pgfusepath{stroke}%
\end{pgfscope}%
\begin{pgfscope}%
\pgfpathrectangle{\pgfqpoint{6.720588in}{1.750000in}}{\pgfqpoint{2.279412in}{2.004545in}}%
\pgfusepath{clip}%
\pgfsetbuttcap%
\pgfsetroundjoin%
\pgfsetlinewidth{0.355761pt}%
\definecolor{currentstroke}{rgb}{0.276022,0.044167,0.370164}%
\pgfsetstrokecolor{currentstroke}%
\pgfsetdash{}{0pt}%
\pgfpathmoveto{\pgfqpoint{8.116754in}{2.120778in}}%
\pgfpathlineto{\pgfqpoint{8.067271in}{2.127716in}}%
\pgfusepath{stroke}%
\end{pgfscope}%
\begin{pgfscope}%
\pgfpathrectangle{\pgfqpoint{6.720588in}{1.750000in}}{\pgfqpoint{2.279412in}{2.004545in}}%
\pgfusepath{clip}%
\pgfsetbuttcap%
\pgfsetroundjoin%
\pgfsetlinewidth{0.341344pt}%
\definecolor{currentstroke}{rgb}{0.273809,0.031497,0.358853}%
\pgfsetstrokecolor{currentstroke}%
\pgfsetdash{}{0pt}%
\pgfpathmoveto{\pgfqpoint{8.067271in}{2.127716in}}%
\pgfpathlineto{\pgfqpoint{8.017882in}{2.135171in}}%
\pgfusepath{stroke}%
\end{pgfscope}%
\begin{pgfscope}%
\pgfpathrectangle{\pgfqpoint{6.720588in}{1.750000in}}{\pgfqpoint{2.279412in}{2.004545in}}%
\pgfusepath{clip}%
\pgfsetbuttcap%
\pgfsetroundjoin%
\pgfsetlinewidth{0.347336pt}%
\definecolor{currentstroke}{rgb}{0.274952,0.037752,0.364543}%
\pgfsetstrokecolor{currentstroke}%
\pgfsetdash{}{0pt}%
\pgfpathmoveto{\pgfqpoint{8.017882in}{2.135171in}}%
\pgfpathlineto{\pgfqpoint{7.968588in}{2.142931in}}%
\pgfusepath{stroke}%
\end{pgfscope}%
\begin{pgfscope}%
\pgfpathrectangle{\pgfqpoint{6.720588in}{1.750000in}}{\pgfqpoint{2.279412in}{2.004545in}}%
\pgfusepath{clip}%
\pgfsetbuttcap%
\pgfsetroundjoin%
\pgfsetlinewidth{0.375670pt}%
\definecolor{currentstroke}{rgb}{0.278791,0.062145,0.386592}%
\pgfsetstrokecolor{currentstroke}%
\pgfsetdash{}{0pt}%
\pgfpathmoveto{\pgfqpoint{7.968588in}{2.142931in}}%
\pgfpathlineto{\pgfqpoint{7.920492in}{2.154223in}}%
\pgfusepath{stroke}%
\end{pgfscope}%
\begin{pgfscope}%
\pgfpathrectangle{\pgfqpoint{6.720588in}{1.750000in}}{\pgfqpoint{2.279412in}{2.004545in}}%
\pgfusepath{clip}%
\pgfsetbuttcap%
\pgfsetroundjoin%
\pgfsetlinewidth{0.367823pt}%
\definecolor{currentstroke}{rgb}{0.277941,0.056324,0.381191}%
\pgfsetstrokecolor{currentstroke}%
\pgfsetdash{}{0pt}%
\pgfpathmoveto{\pgfqpoint{7.920492in}{2.154223in}}%
\pgfpathlineto{\pgfqpoint{7.874713in}{2.171355in}}%
\pgfusepath{stroke}%
\end{pgfscope}%
\begin{pgfscope}%
\pgfpathrectangle{\pgfqpoint{6.720588in}{1.750000in}}{\pgfqpoint{2.279412in}{2.004545in}}%
\pgfusepath{clip}%
\pgfsetbuttcap%
\pgfsetroundjoin%
\pgfsetlinewidth{0.343879pt}%
\definecolor{currentstroke}{rgb}{0.274952,0.037752,0.364543}%
\pgfsetstrokecolor{currentstroke}%
\pgfsetdash{}{0pt}%
\pgfpathmoveto{\pgfqpoint{7.874713in}{2.171355in}}%
\pgfpathlineto{\pgfqpoint{7.830509in}{2.190858in}}%
\pgfusepath{stroke}%
\end{pgfscope}%
\begin{pgfscope}%
\pgfpathrectangle{\pgfqpoint{6.720588in}{1.750000in}}{\pgfqpoint{2.279412in}{2.004545in}}%
\pgfusepath{clip}%
\pgfsetbuttcap%
\pgfsetroundjoin%
\pgfsetlinewidth{0.341083pt}%
\definecolor{currentstroke}{rgb}{0.273809,0.031497,0.358853}%
\pgfsetstrokecolor{currentstroke}%
\pgfsetdash{}{0pt}%
\pgfpathmoveto{\pgfqpoint{7.830509in}{2.190858in}}%
\pgfpathlineto{\pgfqpoint{7.789675in}{2.214232in}}%
\pgfusepath{stroke}%
\end{pgfscope}%
\begin{pgfscope}%
\pgfpathrectangle{\pgfqpoint{6.720588in}{1.750000in}}{\pgfqpoint{2.279412in}{2.004545in}}%
\pgfusepath{clip}%
\pgfsetbuttcap%
\pgfsetroundjoin%
\pgfsetlinewidth{0.318258pt}%
\definecolor{currentstroke}{rgb}{0.269944,0.014625,0.341379}%
\pgfsetstrokecolor{currentstroke}%
\pgfsetdash{}{0pt}%
\pgfpathmoveto{\pgfqpoint{8.578381in}{2.165885in}}%
\pgfpathlineto{\pgfqpoint{8.528597in}{2.166381in}}%
\pgfusepath{stroke}%
\end{pgfscope}%
\begin{pgfscope}%
\pgfpathrectangle{\pgfqpoint{6.720588in}{1.750000in}}{\pgfqpoint{2.279412in}{2.004545in}}%
\pgfusepath{clip}%
\pgfsetbuttcap%
\pgfsetroundjoin%
\pgfsetlinewidth{0.318938pt}%
\definecolor{currentstroke}{rgb}{0.269944,0.014625,0.341379}%
\pgfsetstrokecolor{currentstroke}%
\pgfsetdash{}{0pt}%
\pgfpathmoveto{\pgfqpoint{8.528597in}{2.166381in}}%
\pgfpathlineto{\pgfqpoint{8.478866in}{2.165984in}}%
\pgfusepath{stroke}%
\end{pgfscope}%
\begin{pgfscope}%
\pgfpathrectangle{\pgfqpoint{6.720588in}{1.750000in}}{\pgfqpoint{2.279412in}{2.004545in}}%
\pgfusepath{clip}%
\pgfsetbuttcap%
\pgfsetroundjoin%
\pgfsetlinewidth{0.322322pt}%
\definecolor{currentstroke}{rgb}{0.271305,0.019942,0.347269}%
\pgfsetstrokecolor{currentstroke}%
\pgfsetdash{}{0pt}%
\pgfpathmoveto{\pgfqpoint{8.478866in}{2.165984in}}%
\pgfpathlineto{\pgfqpoint{8.428739in}{2.166391in}}%
\pgfusepath{stroke}%
\end{pgfscope}%
\begin{pgfscope}%
\pgfpathrectangle{\pgfqpoint{6.720588in}{1.750000in}}{\pgfqpoint{2.279412in}{2.004545in}}%
\pgfusepath{clip}%
\pgfsetbuttcap%
\pgfsetroundjoin%
\pgfsetlinewidth{0.318374pt}%
\definecolor{currentstroke}{rgb}{0.269944,0.014625,0.341379}%
\pgfsetstrokecolor{currentstroke}%
\pgfsetdash{}{0pt}%
\pgfpathmoveto{\pgfqpoint{8.428739in}{2.166391in}}%
\pgfpathlineto{\pgfqpoint{8.378612in}{2.166605in}}%
\pgfusepath{stroke}%
\end{pgfscope}%
\begin{pgfscope}%
\pgfpathrectangle{\pgfqpoint{6.720588in}{1.750000in}}{\pgfqpoint{2.279412in}{2.004545in}}%
\pgfusepath{clip}%
\pgfsetbuttcap%
\pgfsetroundjoin%
\pgfsetlinewidth{0.334097pt}%
\definecolor{currentstroke}{rgb}{0.272594,0.025563,0.353093}%
\pgfsetstrokecolor{currentstroke}%
\pgfsetdash{}{0pt}%
\pgfpathmoveto{\pgfqpoint{8.378612in}{2.166605in}}%
\pgfpathlineto{\pgfqpoint{8.328530in}{2.168451in}}%
\pgfusepath{stroke}%
\end{pgfscope}%
\begin{pgfscope}%
\pgfpathrectangle{\pgfqpoint{6.720588in}{1.750000in}}{\pgfqpoint{2.279412in}{2.004545in}}%
\pgfusepath{clip}%
\pgfsetbuttcap%
\pgfsetroundjoin%
\pgfsetlinewidth{0.339042pt}%
\definecolor{currentstroke}{rgb}{0.273809,0.031497,0.358853}%
\pgfsetstrokecolor{currentstroke}%
\pgfsetdash{}{0pt}%
\pgfpathmoveto{\pgfqpoint{8.328530in}{2.168451in}}%
\pgfpathlineto{\pgfqpoint{8.278494in}{2.171041in}}%
\pgfusepath{stroke}%
\end{pgfscope}%
\begin{pgfscope}%
\pgfpathrectangle{\pgfqpoint{6.720588in}{1.750000in}}{\pgfqpoint{2.279412in}{2.004545in}}%
\pgfusepath{clip}%
\pgfsetbuttcap%
\pgfsetroundjoin%
\pgfsetlinewidth{0.333123pt}%
\definecolor{currentstroke}{rgb}{0.272594,0.025563,0.353093}%
\pgfsetstrokecolor{currentstroke}%
\pgfsetdash{}{0pt}%
\pgfpathmoveto{\pgfqpoint{8.278494in}{2.171041in}}%
\pgfpathlineto{\pgfqpoint{8.228555in}{2.174441in}}%
\pgfusepath{stroke}%
\end{pgfscope}%
\begin{pgfscope}%
\pgfpathrectangle{\pgfqpoint{6.720588in}{1.750000in}}{\pgfqpoint{2.279412in}{2.004545in}}%
\pgfusepath{clip}%
\pgfsetbuttcap%
\pgfsetroundjoin%
\pgfsetlinewidth{0.336550pt}%
\definecolor{currentstroke}{rgb}{0.273809,0.031497,0.358853}%
\pgfsetstrokecolor{currentstroke}%
\pgfsetdash{}{0pt}%
\pgfpathmoveto{\pgfqpoint{8.228555in}{2.174441in}}%
\pgfpathlineto{\pgfqpoint{8.178698in}{2.178853in}}%
\pgfusepath{stroke}%
\end{pgfscope}%
\begin{pgfscope}%
\pgfpathrectangle{\pgfqpoint{6.720588in}{1.750000in}}{\pgfqpoint{2.279412in}{2.004545in}}%
\pgfusepath{clip}%
\pgfsetbuttcap%
\pgfsetroundjoin%
\pgfsetlinewidth{0.381913pt}%
\definecolor{currentstroke}{rgb}{0.279566,0.067836,0.391917}%
\pgfsetstrokecolor{currentstroke}%
\pgfsetdash{}{0pt}%
\pgfpathmoveto{\pgfqpoint{8.178698in}{2.178853in}}%
\pgfpathlineto{\pgfqpoint{8.128878in}{2.183698in}}%
\pgfusepath{stroke}%
\end{pgfscope}%
\begin{pgfscope}%
\pgfpathrectangle{\pgfqpoint{6.720588in}{1.750000in}}{\pgfqpoint{2.279412in}{2.004545in}}%
\pgfusepath{clip}%
\pgfsetbuttcap%
\pgfsetroundjoin%
\pgfsetlinewidth{0.360529pt}%
\definecolor{currentstroke}{rgb}{0.277018,0.050344,0.375715}%
\pgfsetstrokecolor{currentstroke}%
\pgfsetdash{}{0pt}%
\pgfpathmoveto{\pgfqpoint{8.128878in}{2.183698in}}%
\pgfpathlineto{\pgfqpoint{8.079210in}{2.189552in}}%
\pgfusepath{stroke}%
\end{pgfscope}%
\begin{pgfscope}%
\pgfpathrectangle{\pgfqpoint{6.720588in}{1.750000in}}{\pgfqpoint{2.279412in}{2.004545in}}%
\pgfusepath{clip}%
\pgfsetbuttcap%
\pgfsetroundjoin%
\pgfsetlinewidth{0.358792pt}%
\definecolor{currentstroke}{rgb}{0.277018,0.050344,0.375715}%
\pgfsetstrokecolor{currentstroke}%
\pgfsetdash{}{0pt}%
\pgfpathmoveto{\pgfqpoint{8.079210in}{2.189552in}}%
\pgfpathlineto{\pgfqpoint{8.029614in}{2.195973in}}%
\pgfusepath{stroke}%
\end{pgfscope}%
\begin{pgfscope}%
\pgfpathrectangle{\pgfqpoint{6.720588in}{1.750000in}}{\pgfqpoint{2.279412in}{2.004545in}}%
\pgfusepath{clip}%
\pgfsetbuttcap%
\pgfsetroundjoin%
\pgfsetlinewidth{0.333610pt}%
\definecolor{currentstroke}{rgb}{0.272594,0.025563,0.353093}%
\pgfsetstrokecolor{currentstroke}%
\pgfsetdash{}{0pt}%
\pgfpathmoveto{\pgfqpoint{8.029614in}{2.195973in}}%
\pgfpathlineto{\pgfqpoint{7.980676in}{2.204999in}}%
\pgfusepath{stroke}%
\end{pgfscope}%
\begin{pgfscope}%
\pgfpathrectangle{\pgfqpoint{6.720588in}{1.750000in}}{\pgfqpoint{2.279412in}{2.004545in}}%
\pgfusepath{clip}%
\pgfsetbuttcap%
\pgfsetroundjoin%
\pgfsetlinewidth{0.370199pt}%
\definecolor{currentstroke}{rgb}{0.278791,0.062145,0.386592}%
\pgfsetstrokecolor{currentstroke}%
\pgfsetdash{}{0pt}%
\pgfpathmoveto{\pgfqpoint{7.980676in}{2.204999in}}%
\pgfpathlineto{\pgfqpoint{7.933072in}{2.218599in}}%
\pgfusepath{stroke}%
\end{pgfscope}%
\begin{pgfscope}%
\pgfpathrectangle{\pgfqpoint{6.720588in}{1.750000in}}{\pgfqpoint{2.279412in}{2.004545in}}%
\pgfusepath{clip}%
\pgfsetbuttcap%
\pgfsetroundjoin%
\pgfsetlinewidth{0.387681pt}%
\definecolor{currentstroke}{rgb}{0.280267,0.073417,0.397163}%
\pgfsetstrokecolor{currentstroke}%
\pgfsetdash{}{0pt}%
\pgfpathmoveto{\pgfqpoint{7.933072in}{2.218599in}}%
\pgfpathlineto{\pgfqpoint{7.886822in}{2.235338in}}%
\pgfusepath{stroke}%
\end{pgfscope}%
\begin{pgfscope}%
\pgfpathrectangle{\pgfqpoint{6.720588in}{1.750000in}}{\pgfqpoint{2.279412in}{2.004545in}}%
\pgfusepath{clip}%
\pgfsetbuttcap%
\pgfsetroundjoin%
\pgfsetlinewidth{0.362707pt}%
\definecolor{currentstroke}{rgb}{0.277018,0.050344,0.375715}%
\pgfsetstrokecolor{currentstroke}%
\pgfsetdash{}{0pt}%
\pgfpathmoveto{\pgfqpoint{7.886822in}{2.235338in}}%
\pgfpathlineto{\pgfqpoint{7.842129in}{2.255235in}}%
\pgfusepath{stroke}%
\end{pgfscope}%
\begin{pgfscope}%
\pgfpathrectangle{\pgfqpoint{6.720588in}{1.750000in}}{\pgfqpoint{2.279412in}{2.004545in}}%
\pgfusepath{clip}%
\pgfsetbuttcap%
\pgfsetroundjoin%
\pgfsetlinewidth{0.367452pt}%
\definecolor{currentstroke}{rgb}{0.277941,0.056324,0.381191}%
\pgfsetstrokecolor{currentstroke}%
\pgfsetdash{}{0pt}%
\pgfpathmoveto{\pgfqpoint{7.842129in}{2.255235in}}%
\pgfpathlineto{\pgfqpoint{7.799587in}{2.278375in}}%
\pgfusepath{stroke}%
\end{pgfscope}%
\begin{pgfscope}%
\pgfpathrectangle{\pgfqpoint{6.720588in}{1.750000in}}{\pgfqpoint{2.279412in}{2.004545in}}%
\pgfusepath{clip}%
\pgfsetbuttcap%
\pgfsetroundjoin%
\pgfsetlinewidth{0.389010pt}%
\definecolor{currentstroke}{rgb}{0.280267,0.073417,0.397163}%
\pgfsetstrokecolor{currentstroke}%
\pgfsetdash{}{0pt}%
\pgfpathmoveto{\pgfqpoint{7.799587in}{2.278375in}}%
\pgfpathlineto{\pgfqpoint{7.799587in}{2.278375in}}%
\pgfusepath{stroke}%
\end{pgfscope}%
\begin{pgfscope}%
\pgfpathrectangle{\pgfqpoint{6.720588in}{1.750000in}}{\pgfqpoint{2.279412in}{2.004545in}}%
\pgfusepath{clip}%
\pgfsetbuttcap%
\pgfsetroundjoin%
\pgfsetlinewidth{0.389010pt}%
\definecolor{currentstroke}{rgb}{0.280267,0.073417,0.397163}%
\pgfsetstrokecolor{currentstroke}%
\pgfsetdash{}{0pt}%
\pgfpathmoveto{\pgfqpoint{7.799587in}{2.278375in}}%
\pgfpathlineto{\pgfqpoint{7.768842in}{2.302756in}}%
\pgfusepath{stroke}%
\end{pgfscope}%
\begin{pgfscope}%
\pgfpathrectangle{\pgfqpoint{6.720588in}{1.750000in}}{\pgfqpoint{2.279412in}{2.004545in}}%
\pgfusepath{clip}%
\pgfsetbuttcap%
\pgfsetroundjoin%
\pgfsetlinewidth{0.366362pt}%
\definecolor{currentstroke}{rgb}{0.277941,0.056324,0.381191}%
\pgfsetstrokecolor{currentstroke}%
\pgfsetdash{}{0pt}%
\pgfpathmoveto{\pgfqpoint{7.768842in}{2.302756in}}%
\pgfpathlineto{\pgfqpoint{7.745619in}{2.331598in}}%
\pgfusepath{stroke}%
\end{pgfscope}%
\begin{pgfscope}%
\pgfpathrectangle{\pgfqpoint{6.720588in}{1.750000in}}{\pgfqpoint{2.279412in}{2.004545in}}%
\pgfusepath{clip}%
\pgfsetbuttcap%
\pgfsetroundjoin%
\pgfsetlinewidth{0.320664pt}%
\definecolor{currentstroke}{rgb}{0.269944,0.014625,0.341379}%
\pgfsetstrokecolor{currentstroke}%
\pgfsetdash{}{0pt}%
\pgfpathmoveto{\pgfqpoint{8.578381in}{2.256098in}}%
\pgfpathlineto{\pgfqpoint{8.528258in}{2.256011in}}%
\pgfusepath{stroke}%
\end{pgfscope}%
\begin{pgfscope}%
\pgfpathrectangle{\pgfqpoint{6.720588in}{1.750000in}}{\pgfqpoint{2.279412in}{2.004545in}}%
\pgfusepath{clip}%
\pgfsetbuttcap%
\pgfsetroundjoin%
\pgfsetlinewidth{0.313446pt}%
\definecolor{currentstroke}{rgb}{0.268510,0.009605,0.335427}%
\pgfsetstrokecolor{currentstroke}%
\pgfsetdash{}{0pt}%
\pgfpathmoveto{\pgfqpoint{8.528258in}{2.256011in}}%
\pgfpathlineto{\pgfqpoint{8.478220in}{2.256744in}}%
\pgfusepath{stroke}%
\end{pgfscope}%
\begin{pgfscope}%
\pgfpathrectangle{\pgfqpoint{6.720588in}{1.750000in}}{\pgfqpoint{2.279412in}{2.004545in}}%
\pgfusepath{clip}%
\pgfsetbuttcap%
\pgfsetroundjoin%
\pgfsetlinewidth{0.319803pt}%
\definecolor{currentstroke}{rgb}{0.269944,0.014625,0.341379}%
\pgfsetstrokecolor{currentstroke}%
\pgfsetdash{}{0pt}%
\pgfpathmoveto{\pgfqpoint{8.478220in}{2.256744in}}%
\pgfpathlineto{\pgfqpoint{8.428183in}{2.258719in}}%
\pgfusepath{stroke}%
\end{pgfscope}%
\begin{pgfscope}%
\pgfpathrectangle{\pgfqpoint{6.720588in}{1.750000in}}{\pgfqpoint{2.279412in}{2.004545in}}%
\pgfusepath{clip}%
\pgfsetbuttcap%
\pgfsetroundjoin%
\pgfsetlinewidth{0.323399pt}%
\definecolor{currentstroke}{rgb}{0.271305,0.019942,0.347269}%
\pgfsetstrokecolor{currentstroke}%
\pgfsetdash{}{0pt}%
\pgfpathmoveto{\pgfqpoint{8.428183in}{2.258719in}}%
\pgfpathlineto{\pgfqpoint{8.378055in}{2.259865in}}%
\pgfusepath{stroke}%
\end{pgfscope}%
\begin{pgfscope}%
\pgfpathrectangle{\pgfqpoint{6.720588in}{1.750000in}}{\pgfqpoint{2.279412in}{2.004545in}}%
\pgfusepath{clip}%
\pgfsetbuttcap%
\pgfsetroundjoin%
\pgfsetlinewidth{0.346094pt}%
\definecolor{currentstroke}{rgb}{0.274952,0.037752,0.364543}%
\pgfsetstrokecolor{currentstroke}%
\pgfsetdash{}{0pt}%
\pgfpathmoveto{\pgfqpoint{8.378055in}{2.259865in}}%
\pgfpathlineto{\pgfqpoint{8.327921in}{2.260935in}}%
\pgfusepath{stroke}%
\end{pgfscope}%
\begin{pgfscope}%
\pgfpathrectangle{\pgfqpoint{6.720588in}{1.750000in}}{\pgfqpoint{2.279412in}{2.004545in}}%
\pgfusepath{clip}%
\pgfsetbuttcap%
\pgfsetroundjoin%
\pgfsetlinewidth{0.347820pt}%
\definecolor{currentstroke}{rgb}{0.274952,0.037752,0.364543}%
\pgfsetstrokecolor{currentstroke}%
\pgfsetdash{}{0pt}%
\pgfpathmoveto{\pgfqpoint{8.327921in}{2.260935in}}%
\pgfpathlineto{\pgfqpoint{8.277806in}{2.262447in}}%
\pgfusepath{stroke}%
\end{pgfscope}%
\begin{pgfscope}%
\pgfpathrectangle{\pgfqpoint{6.720588in}{1.750000in}}{\pgfqpoint{2.279412in}{2.004545in}}%
\pgfusepath{clip}%
\pgfsetbuttcap%
\pgfsetroundjoin%
\pgfsetlinewidth{0.361782pt}%
\definecolor{currentstroke}{rgb}{0.277018,0.050344,0.375715}%
\pgfsetstrokecolor{currentstroke}%
\pgfsetdash{}{0pt}%
\pgfpathmoveto{\pgfqpoint{8.277806in}{2.262447in}}%
\pgfpathlineto{\pgfqpoint{8.227721in}{2.264609in}}%
\pgfusepath{stroke}%
\end{pgfscope}%
\begin{pgfscope}%
\pgfpathrectangle{\pgfqpoint{6.720588in}{1.750000in}}{\pgfqpoint{2.279412in}{2.004545in}}%
\pgfusepath{clip}%
\pgfsetbuttcap%
\pgfsetroundjoin%
\pgfsetlinewidth{0.375632pt}%
\definecolor{currentstroke}{rgb}{0.278791,0.062145,0.386592}%
\pgfsetstrokecolor{currentstroke}%
\pgfsetdash{}{0pt}%
\pgfpathmoveto{\pgfqpoint{8.227721in}{2.264609in}}%
\pgfpathlineto{\pgfqpoint{8.177703in}{2.267714in}}%
\pgfusepath{stroke}%
\end{pgfscope}%
\begin{pgfscope}%
\pgfpathrectangle{\pgfqpoint{6.720588in}{1.750000in}}{\pgfqpoint{2.279412in}{2.004545in}}%
\pgfusepath{clip}%
\pgfsetbuttcap%
\pgfsetroundjoin%
\pgfsetlinewidth{0.387996pt}%
\definecolor{currentstroke}{rgb}{0.280267,0.073417,0.397163}%
\pgfsetstrokecolor{currentstroke}%
\pgfsetdash{}{0pt}%
\pgfpathmoveto{\pgfqpoint{8.177703in}{2.267714in}}%
\pgfpathlineto{\pgfqpoint{8.127785in}{2.271897in}}%
\pgfusepath{stroke}%
\end{pgfscope}%
\begin{pgfscope}%
\pgfpathrectangle{\pgfqpoint{6.720588in}{1.750000in}}{\pgfqpoint{2.279412in}{2.004545in}}%
\pgfusepath{clip}%
\pgfsetbuttcap%
\pgfsetroundjoin%
\pgfsetlinewidth{0.387416pt}%
\definecolor{currentstroke}{rgb}{0.280267,0.073417,0.397163}%
\pgfsetstrokecolor{currentstroke}%
\pgfsetdash{}{0pt}%
\pgfpathmoveto{\pgfqpoint{8.127785in}{2.271897in}}%
\pgfpathlineto{\pgfqpoint{8.077982in}{2.277061in}}%
\pgfusepath{stroke}%
\end{pgfscope}%
\begin{pgfscope}%
\pgfpathrectangle{\pgfqpoint{6.720588in}{1.750000in}}{\pgfqpoint{2.279412in}{2.004545in}}%
\pgfusepath{clip}%
\pgfsetbuttcap%
\pgfsetroundjoin%
\pgfsetlinewidth{0.378658pt}%
\definecolor{currentstroke}{rgb}{0.279566,0.067836,0.391917}%
\pgfsetstrokecolor{currentstroke}%
\pgfsetdash{}{0pt}%
\pgfpathmoveto{\pgfqpoint{8.077982in}{2.277061in}}%
\pgfpathlineto{\pgfqpoint{8.028400in}{2.283585in}}%
\pgfusepath{stroke}%
\end{pgfscope}%
\begin{pgfscope}%
\pgfpathrectangle{\pgfqpoint{6.720588in}{1.750000in}}{\pgfqpoint{2.279412in}{2.004545in}}%
\pgfusepath{clip}%
\pgfsetbuttcap%
\pgfsetroundjoin%
\pgfsetlinewidth{0.407103pt}%
\definecolor{currentstroke}{rgb}{0.281924,0.089666,0.412415}%
\pgfsetstrokecolor{currentstroke}%
\pgfsetdash{}{0pt}%
\pgfpathmoveto{\pgfqpoint{8.028400in}{2.283585in}}%
\pgfpathlineto{\pgfqpoint{7.979317in}{2.292216in}}%
\pgfusepath{stroke}%
\end{pgfscope}%
\begin{pgfscope}%
\pgfpathrectangle{\pgfqpoint{6.720588in}{1.750000in}}{\pgfqpoint{2.279412in}{2.004545in}}%
\pgfusepath{clip}%
\pgfsetbuttcap%
\pgfsetroundjoin%
\pgfsetlinewidth{0.401047pt}%
\definecolor{currentstroke}{rgb}{0.281446,0.084320,0.407414}%
\pgfsetstrokecolor{currentstroke}%
\pgfsetdash{}{0pt}%
\pgfpathmoveto{\pgfqpoint{7.979317in}{2.292216in}}%
\pgfpathlineto{\pgfqpoint{7.930655in}{2.302591in}}%
\pgfusepath{stroke}%
\end{pgfscope}%
\begin{pgfscope}%
\pgfpathrectangle{\pgfqpoint{6.720588in}{1.750000in}}{\pgfqpoint{2.279412in}{2.004545in}}%
\pgfusepath{clip}%
\pgfsetbuttcap%
\pgfsetroundjoin%
\pgfsetlinewidth{0.396294pt}%
\definecolor{currentstroke}{rgb}{0.280894,0.078907,0.402329}%
\pgfsetstrokecolor{currentstroke}%
\pgfsetdash{}{0pt}%
\pgfpathmoveto{\pgfqpoint{7.930655in}{2.302591in}}%
\pgfpathlineto{\pgfqpoint{7.883161in}{2.316117in}}%
\pgfusepath{stroke}%
\end{pgfscope}%
\begin{pgfscope}%
\pgfpathrectangle{\pgfqpoint{6.720588in}{1.750000in}}{\pgfqpoint{2.279412in}{2.004545in}}%
\pgfusepath{clip}%
\pgfsetbuttcap%
\pgfsetroundjoin%
\pgfsetlinewidth{0.436377pt}%
\definecolor{currentstroke}{rgb}{0.283091,0.110553,0.431554}%
\pgfsetstrokecolor{currentstroke}%
\pgfsetdash{}{0pt}%
\pgfpathmoveto{\pgfqpoint{7.883161in}{2.316117in}}%
\pgfpathlineto{\pgfqpoint{7.837756in}{2.334681in}}%
\pgfusepath{stroke}%
\end{pgfscope}%
\begin{pgfscope}%
\pgfpathrectangle{\pgfqpoint{6.720588in}{1.750000in}}{\pgfqpoint{2.279412in}{2.004545in}}%
\pgfusepath{clip}%
\pgfsetbuttcap%
\pgfsetroundjoin%
\pgfsetlinewidth{0.457848pt}%
\definecolor{currentstroke}{rgb}{0.283187,0.125848,0.444960}%
\pgfsetstrokecolor{currentstroke}%
\pgfsetdash{}{0pt}%
\pgfpathmoveto{\pgfqpoint{7.837756in}{2.334681in}}%
\pgfpathlineto{\pgfqpoint{7.793858in}{2.355996in}}%
\pgfusepath{stroke}%
\end{pgfscope}%
\begin{pgfscope}%
\pgfpathrectangle{\pgfqpoint{6.720588in}{1.750000in}}{\pgfqpoint{2.279412in}{2.004545in}}%
\pgfusepath{clip}%
\pgfsetbuttcap%
\pgfsetroundjoin%
\pgfsetlinewidth{0.414996pt}%
\definecolor{currentstroke}{rgb}{0.282327,0.094955,0.417331}%
\pgfsetstrokecolor{currentstroke}%
\pgfsetdash{}{0pt}%
\pgfpathmoveto{\pgfqpoint{7.793858in}{2.355996in}}%
\pgfpathlineto{\pgfqpoint{7.753501in}{2.381562in}}%
\pgfusepath{stroke}%
\end{pgfscope}%
\begin{pgfscope}%
\pgfpathrectangle{\pgfqpoint{6.720588in}{1.750000in}}{\pgfqpoint{2.279412in}{2.004545in}}%
\pgfusepath{clip}%
\pgfsetbuttcap%
\pgfsetroundjoin%
\pgfsetlinewidth{0.317489pt}%
\definecolor{currentstroke}{rgb}{0.269944,0.014625,0.341379}%
\pgfsetstrokecolor{currentstroke}%
\pgfsetdash{}{0pt}%
\pgfpathmoveto{\pgfqpoint{8.644093in}{2.297971in}}%
\pgfpathlineto{\pgfqpoint{8.597076in}{2.300014in}}%
\pgfusepath{stroke}%
\end{pgfscope}%
\begin{pgfscope}%
\pgfpathrectangle{\pgfqpoint{6.720588in}{1.750000in}}{\pgfqpoint{2.279412in}{2.004545in}}%
\pgfusepath{clip}%
\pgfsetbuttcap%
\pgfsetroundjoin%
\pgfsetlinewidth{0.307698pt}%
\definecolor{currentstroke}{rgb}{0.267004,0.004874,0.329415}%
\pgfsetstrokecolor{currentstroke}%
\pgfsetdash{}{0pt}%
\pgfpathmoveto{\pgfqpoint{8.597076in}{2.300014in}}%
\pgfpathlineto{\pgfqpoint{8.578381in}{2.301205in}}%
\pgfusepath{stroke}%
\end{pgfscope}%
\begin{pgfscope}%
\pgfpathrectangle{\pgfqpoint{6.720588in}{1.750000in}}{\pgfqpoint{2.279412in}{2.004545in}}%
\pgfusepath{clip}%
\pgfsetbuttcap%
\pgfsetroundjoin%
\pgfsetlinewidth{0.310594pt}%
\definecolor{currentstroke}{rgb}{0.268510,0.009605,0.335427}%
\pgfsetstrokecolor{currentstroke}%
\pgfsetdash{}{0pt}%
\pgfpathmoveto{\pgfqpoint{8.578381in}{2.301205in}}%
\pgfpathlineto{\pgfqpoint{8.578381in}{2.301205in}}%
\pgfusepath{stroke}%
\end{pgfscope}%
\begin{pgfscope}%
\pgfpathrectangle{\pgfqpoint{6.720588in}{1.750000in}}{\pgfqpoint{2.279412in}{2.004545in}}%
\pgfusepath{clip}%
\pgfsetbuttcap%
\pgfsetroundjoin%
\pgfsetlinewidth{0.310594pt}%
\definecolor{currentstroke}{rgb}{0.268510,0.009605,0.335427}%
\pgfsetstrokecolor{currentstroke}%
\pgfsetdash{}{0pt}%
\pgfpathmoveto{\pgfqpoint{8.578381in}{2.301205in}}%
\pgfpathlineto{\pgfqpoint{8.528460in}{2.303109in}}%
\pgfusepath{stroke}%
\end{pgfscope}%
\begin{pgfscope}%
\pgfpathrectangle{\pgfqpoint{6.720588in}{1.750000in}}{\pgfqpoint{2.279412in}{2.004545in}}%
\pgfusepath{clip}%
\pgfsetbuttcap%
\pgfsetroundjoin%
\pgfsetlinewidth{0.317422pt}%
\definecolor{currentstroke}{rgb}{0.269944,0.014625,0.341379}%
\pgfsetstrokecolor{currentstroke}%
\pgfsetdash{}{0pt}%
\pgfpathmoveto{\pgfqpoint{8.528460in}{2.303109in}}%
\pgfpathlineto{\pgfqpoint{8.478379in}{2.302871in}}%
\pgfusepath{stroke}%
\end{pgfscope}%
\begin{pgfscope}%
\pgfpathrectangle{\pgfqpoint{6.720588in}{1.750000in}}{\pgfqpoint{2.279412in}{2.004545in}}%
\pgfusepath{clip}%
\pgfsetbuttcap%
\pgfsetroundjoin%
\pgfsetlinewidth{0.318434pt}%
\definecolor{currentstroke}{rgb}{0.269944,0.014625,0.341379}%
\pgfsetstrokecolor{currentstroke}%
\pgfsetdash{}{0pt}%
\pgfpathmoveto{\pgfqpoint{8.478379in}{2.302871in}}%
\pgfpathlineto{\pgfqpoint{8.428286in}{2.302733in}}%
\pgfusepath{stroke}%
\end{pgfscope}%
\begin{pgfscope}%
\pgfpathrectangle{\pgfqpoint{6.720588in}{1.750000in}}{\pgfqpoint{2.279412in}{2.004545in}}%
\pgfusepath{clip}%
\pgfsetbuttcap%
\pgfsetroundjoin%
\pgfsetlinewidth{0.328331pt}%
\definecolor{currentstroke}{rgb}{0.271305,0.019942,0.347269}%
\pgfsetstrokecolor{currentstroke}%
\pgfsetdash{}{0pt}%
\pgfpathmoveto{\pgfqpoint{8.428286in}{2.302733in}}%
\pgfpathlineto{\pgfqpoint{8.378233in}{2.304304in}}%
\pgfusepath{stroke}%
\end{pgfscope}%
\begin{pgfscope}%
\pgfpathrectangle{\pgfqpoint{6.720588in}{1.750000in}}{\pgfqpoint{2.279412in}{2.004545in}}%
\pgfusepath{clip}%
\pgfsetbuttcap%
\pgfsetroundjoin%
\pgfsetlinewidth{0.327262pt}%
\definecolor{currentstroke}{rgb}{0.271305,0.019942,0.347269}%
\pgfsetstrokecolor{currentstroke}%
\pgfsetdash{}{0pt}%
\pgfpathmoveto{\pgfqpoint{8.378233in}{2.304304in}}%
\pgfpathlineto{\pgfqpoint{8.328176in}{2.306523in}}%
\pgfusepath{stroke}%
\end{pgfscope}%
\begin{pgfscope}%
\pgfpathrectangle{\pgfqpoint{6.720588in}{1.750000in}}{\pgfqpoint{2.279412in}{2.004545in}}%
\pgfusepath{clip}%
\pgfsetbuttcap%
\pgfsetroundjoin%
\pgfsetlinewidth{0.349177pt}%
\definecolor{currentstroke}{rgb}{0.276022,0.044167,0.370164}%
\pgfsetstrokecolor{currentstroke}%
\pgfsetdash{}{0pt}%
\pgfpathmoveto{\pgfqpoint{8.328176in}{2.306523in}}%
\pgfpathlineto{\pgfqpoint{8.278095in}{2.308758in}}%
\pgfusepath{stroke}%
\end{pgfscope}%
\begin{pgfscope}%
\pgfpathrectangle{\pgfqpoint{6.720588in}{1.750000in}}{\pgfqpoint{2.279412in}{2.004545in}}%
\pgfusepath{clip}%
\pgfsetbuttcap%
\pgfsetroundjoin%
\pgfsetlinewidth{0.365210pt}%
\definecolor{currentstroke}{rgb}{0.277941,0.056324,0.381191}%
\pgfsetstrokecolor{currentstroke}%
\pgfsetdash{}{0pt}%
\pgfpathmoveto{\pgfqpoint{8.278095in}{2.308758in}}%
\pgfpathlineto{\pgfqpoint{8.228008in}{2.310930in}}%
\pgfusepath{stroke}%
\end{pgfscope}%
\begin{pgfscope}%
\pgfpathrectangle{\pgfqpoint{6.720588in}{1.750000in}}{\pgfqpoint{2.279412in}{2.004545in}}%
\pgfusepath{clip}%
\pgfsetbuttcap%
\pgfsetroundjoin%
\pgfsetlinewidth{0.390395pt}%
\definecolor{currentstroke}{rgb}{0.280894,0.078907,0.402329}%
\pgfsetstrokecolor{currentstroke}%
\pgfsetdash{}{0pt}%
\pgfpathmoveto{\pgfqpoint{8.228008in}{2.310930in}}%
\pgfpathlineto{\pgfqpoint{8.177945in}{2.313501in}}%
\pgfusepath{stroke}%
\end{pgfscope}%
\begin{pgfscope}%
\pgfpathrectangle{\pgfqpoint{6.720588in}{1.750000in}}{\pgfqpoint{2.279412in}{2.004545in}}%
\pgfusepath{clip}%
\pgfsetbuttcap%
\pgfsetroundjoin%
\pgfsetlinewidth{0.391466pt}%
\definecolor{currentstroke}{rgb}{0.280894,0.078907,0.402329}%
\pgfsetstrokecolor{currentstroke}%
\pgfsetdash{}{0pt}%
\pgfpathmoveto{\pgfqpoint{8.177945in}{2.313501in}}%
\pgfpathlineto{\pgfqpoint{8.127971in}{2.317127in}}%
\pgfusepath{stroke}%
\end{pgfscope}%
\begin{pgfscope}%
\pgfpathrectangle{\pgfqpoint{6.720588in}{1.750000in}}{\pgfqpoint{2.279412in}{2.004545in}}%
\pgfusepath{clip}%
\pgfsetbuttcap%
\pgfsetroundjoin%
\pgfsetlinewidth{0.404083pt}%
\definecolor{currentstroke}{rgb}{0.281924,0.089666,0.412415}%
\pgfsetstrokecolor{currentstroke}%
\pgfsetdash{}{0pt}%
\pgfpathmoveto{\pgfqpoint{8.127971in}{2.317127in}}%
\pgfpathlineto{\pgfqpoint{8.078093in}{2.321677in}}%
\pgfusepath{stroke}%
\end{pgfscope}%
\begin{pgfscope}%
\pgfpathrectangle{\pgfqpoint{6.720588in}{1.750000in}}{\pgfqpoint{2.279412in}{2.004545in}}%
\pgfusepath{clip}%
\pgfsetbuttcap%
\pgfsetroundjoin%
\pgfsetlinewidth{0.428940pt}%
\definecolor{currentstroke}{rgb}{0.282910,0.105393,0.426902}%
\pgfsetstrokecolor{currentstroke}%
\pgfsetdash{}{0pt}%
\pgfpathmoveto{\pgfqpoint{8.078093in}{2.321677in}}%
\pgfpathlineto{\pgfqpoint{8.028433in}{2.327605in}}%
\pgfusepath{stroke}%
\end{pgfscope}%
\begin{pgfscope}%
\pgfpathrectangle{\pgfqpoint{6.720588in}{1.750000in}}{\pgfqpoint{2.279412in}{2.004545in}}%
\pgfusepath{clip}%
\pgfsetbuttcap%
\pgfsetroundjoin%
\pgfsetlinewidth{0.422048pt}%
\definecolor{currentstroke}{rgb}{0.282656,0.100196,0.422160}%
\pgfsetstrokecolor{currentstroke}%
\pgfsetdash{}{0pt}%
\pgfpathmoveto{\pgfqpoint{8.028433in}{2.327605in}}%
\pgfpathlineto{\pgfqpoint{7.979116in}{2.335518in}}%
\pgfusepath{stroke}%
\end{pgfscope}%
\begin{pgfscope}%
\pgfpathrectangle{\pgfqpoint{6.720588in}{1.750000in}}{\pgfqpoint{2.279412in}{2.004545in}}%
\pgfusepath{clip}%
\pgfsetbuttcap%
\pgfsetroundjoin%
\pgfsetlinewidth{0.465010pt}%
\definecolor{currentstroke}{rgb}{0.283072,0.130895,0.449241}%
\pgfsetstrokecolor{currentstroke}%
\pgfsetdash{}{0pt}%
\pgfpathmoveto{\pgfqpoint{7.979116in}{2.335518in}}%
\pgfpathlineto{\pgfqpoint{7.930280in}{2.345433in}}%
\pgfusepath{stroke}%
\end{pgfscope}%
\begin{pgfscope}%
\pgfpathrectangle{\pgfqpoint{6.720588in}{1.750000in}}{\pgfqpoint{2.279412in}{2.004545in}}%
\pgfusepath{clip}%
\pgfsetbuttcap%
\pgfsetroundjoin%
\pgfsetlinewidth{0.326887pt}%
\definecolor{currentstroke}{rgb}{0.271305,0.019942,0.347269}%
\pgfsetstrokecolor{currentstroke}%
\pgfsetdash{}{0pt}%
\pgfpathmoveto{\pgfqpoint{8.578381in}{2.346312in}}%
\pgfpathlineto{\pgfqpoint{8.528247in}{2.346712in}}%
\pgfusepath{stroke}%
\end{pgfscope}%
\begin{pgfscope}%
\pgfpathrectangle{\pgfqpoint{6.720588in}{1.750000in}}{\pgfqpoint{2.279412in}{2.004545in}}%
\pgfusepath{clip}%
\pgfsetbuttcap%
\pgfsetroundjoin%
\pgfsetlinewidth{0.317416pt}%
\definecolor{currentstroke}{rgb}{0.269944,0.014625,0.341379}%
\pgfsetstrokecolor{currentstroke}%
\pgfsetdash{}{0pt}%
\pgfpathmoveto{\pgfqpoint{8.528247in}{2.346712in}}%
\pgfpathlineto{\pgfqpoint{8.478148in}{2.347563in}}%
\pgfusepath{stroke}%
\end{pgfscope}%
\begin{pgfscope}%
\pgfpathrectangle{\pgfqpoint{6.720588in}{1.750000in}}{\pgfqpoint{2.279412in}{2.004545in}}%
\pgfusepath{clip}%
\pgfsetbuttcap%
\pgfsetroundjoin%
\pgfsetlinewidth{0.321282pt}%
\definecolor{currentstroke}{rgb}{0.269944,0.014625,0.341379}%
\pgfsetstrokecolor{currentstroke}%
\pgfsetdash{}{0pt}%
\pgfpathmoveto{\pgfqpoint{8.478148in}{2.347563in}}%
\pgfpathlineto{\pgfqpoint{8.428047in}{2.348745in}}%
\pgfusepath{stroke}%
\end{pgfscope}%
\begin{pgfscope}%
\pgfpathrectangle{\pgfqpoint{6.720588in}{1.750000in}}{\pgfqpoint{2.279412in}{2.004545in}}%
\pgfusepath{clip}%
\pgfsetbuttcap%
\pgfsetroundjoin%
\pgfsetlinewidth{0.325669pt}%
\definecolor{currentstroke}{rgb}{0.271305,0.019942,0.347269}%
\pgfsetstrokecolor{currentstroke}%
\pgfsetdash{}{0pt}%
\pgfpathmoveto{\pgfqpoint{8.428047in}{2.348745in}}%
\pgfpathlineto{\pgfqpoint{8.377940in}{2.350398in}}%
\pgfusepath{stroke}%
\end{pgfscope}%
\begin{pgfscope}%
\pgfpathrectangle{\pgfqpoint{6.720588in}{1.750000in}}{\pgfqpoint{2.279412in}{2.004545in}}%
\pgfusepath{clip}%
\pgfsetbuttcap%
\pgfsetroundjoin%
\pgfsetlinewidth{0.341560pt}%
\definecolor{currentstroke}{rgb}{0.273809,0.031497,0.358853}%
\pgfsetstrokecolor{currentstroke}%
\pgfsetdash{}{0pt}%
\pgfpathmoveto{\pgfqpoint{8.377940in}{2.350398in}}%
\pgfpathlineto{\pgfqpoint{8.327857in}{2.352667in}}%
\pgfusepath{stroke}%
\end{pgfscope}%
\begin{pgfscope}%
\pgfpathrectangle{\pgfqpoint{6.720588in}{1.750000in}}{\pgfqpoint{2.279412in}{2.004545in}}%
\pgfusepath{clip}%
\pgfsetbuttcap%
\pgfsetroundjoin%
\pgfsetlinewidth{0.354989pt}%
\definecolor{currentstroke}{rgb}{0.276022,0.044167,0.370164}%
\pgfsetstrokecolor{currentstroke}%
\pgfsetdash{}{0pt}%
\pgfpathmoveto{\pgfqpoint{8.327857in}{2.352667in}}%
\pgfpathlineto{\pgfqpoint{8.277773in}{2.354904in}}%
\pgfusepath{stroke}%
\end{pgfscope}%
\begin{pgfscope}%
\pgfpathrectangle{\pgfqpoint{6.720588in}{1.750000in}}{\pgfqpoint{2.279412in}{2.004545in}}%
\pgfusepath{clip}%
\pgfsetbuttcap%
\pgfsetroundjoin%
\pgfsetlinewidth{0.378279pt}%
\definecolor{currentstroke}{rgb}{0.279566,0.067836,0.391917}%
\pgfsetstrokecolor{currentstroke}%
\pgfsetdash{}{0pt}%
\pgfpathmoveto{\pgfqpoint{8.277773in}{2.354904in}}%
\pgfpathlineto{\pgfqpoint{8.227692in}{2.357205in}}%
\pgfusepath{stroke}%
\end{pgfscope}%
\begin{pgfscope}%
\pgfpathrectangle{\pgfqpoint{6.720588in}{1.750000in}}{\pgfqpoint{2.279412in}{2.004545in}}%
\pgfusepath{clip}%
\pgfsetbuttcap%
\pgfsetroundjoin%
\pgfsetlinewidth{0.407372pt}%
\definecolor{currentstroke}{rgb}{0.281924,0.089666,0.412415}%
\pgfsetstrokecolor{currentstroke}%
\pgfsetdash{}{0pt}%
\pgfpathmoveto{\pgfqpoint{8.227692in}{2.357205in}}%
\pgfpathlineto{\pgfqpoint{8.177635in}{2.359898in}}%
\pgfusepath{stroke}%
\end{pgfscope}%
\begin{pgfscope}%
\pgfpathrectangle{\pgfqpoint{6.720588in}{1.750000in}}{\pgfqpoint{2.279412in}{2.004545in}}%
\pgfusepath{clip}%
\pgfsetbuttcap%
\pgfsetroundjoin%
\pgfsetlinewidth{0.401709pt}%
\definecolor{currentstroke}{rgb}{0.281446,0.084320,0.407414}%
\pgfsetstrokecolor{currentstroke}%
\pgfsetdash{}{0pt}%
\pgfpathmoveto{\pgfqpoint{8.177635in}{2.359898in}}%
\pgfpathlineto{\pgfqpoint{8.127604in}{2.362952in}}%
\pgfusepath{stroke}%
\end{pgfscope}%
\begin{pgfscope}%
\pgfpathrectangle{\pgfqpoint{6.720588in}{1.750000in}}{\pgfqpoint{2.279412in}{2.004545in}}%
\pgfusepath{clip}%
\pgfsetbuttcap%
\pgfsetroundjoin%
\pgfsetlinewidth{0.423714pt}%
\definecolor{currentstroke}{rgb}{0.282656,0.100196,0.422160}%
\pgfsetstrokecolor{currentstroke}%
\pgfsetdash{}{0pt}%
\pgfpathmoveto{\pgfqpoint{8.127604in}{2.362952in}}%
\pgfpathlineto{\pgfqpoint{8.077664in}{2.366880in}}%
\pgfusepath{stroke}%
\end{pgfscope}%
\begin{pgfscope}%
\pgfpathrectangle{\pgfqpoint{6.720588in}{1.750000in}}{\pgfqpoint{2.279412in}{2.004545in}}%
\pgfusepath{clip}%
\pgfsetbuttcap%
\pgfsetroundjoin%
\pgfsetlinewidth{0.446275pt}%
\definecolor{currentstroke}{rgb}{0.283229,0.120777,0.440584}%
\pgfsetstrokecolor{currentstroke}%
\pgfsetdash{}{0pt}%
\pgfpathmoveto{\pgfqpoint{8.077664in}{2.366880in}}%
\pgfpathlineto{\pgfqpoint{8.027931in}{2.372477in}}%
\pgfusepath{stroke}%
\end{pgfscope}%
\begin{pgfscope}%
\pgfpathrectangle{\pgfqpoint{6.720588in}{1.750000in}}{\pgfqpoint{2.279412in}{2.004545in}}%
\pgfusepath{clip}%
\pgfsetbuttcap%
\pgfsetroundjoin%
\pgfsetlinewidth{0.483752pt}%
\definecolor{currentstroke}{rgb}{0.282290,0.145912,0.461510}%
\pgfsetstrokecolor{currentstroke}%
\pgfsetdash{}{0pt}%
\pgfpathmoveto{\pgfqpoint{8.027931in}{2.372477in}}%
\pgfpathlineto{\pgfqpoint{7.978448in}{2.379606in}}%
\pgfusepath{stroke}%
\end{pgfscope}%
\begin{pgfscope}%
\pgfpathrectangle{\pgfqpoint{6.720588in}{1.750000in}}{\pgfqpoint{2.279412in}{2.004545in}}%
\pgfusepath{clip}%
\pgfsetbuttcap%
\pgfsetroundjoin%
\pgfsetlinewidth{0.478967pt}%
\definecolor{currentstroke}{rgb}{0.282623,0.140926,0.457517}%
\pgfsetstrokecolor{currentstroke}%
\pgfsetdash{}{0pt}%
\pgfpathmoveto{\pgfqpoint{7.978448in}{2.379606in}}%
\pgfpathlineto{\pgfqpoint{7.929365in}{2.388569in}}%
\pgfusepath{stroke}%
\end{pgfscope}%
\begin{pgfscope}%
\pgfpathrectangle{\pgfqpoint{6.720588in}{1.750000in}}{\pgfqpoint{2.279412in}{2.004545in}}%
\pgfusepath{clip}%
\pgfsetbuttcap%
\pgfsetroundjoin%
\pgfsetlinewidth{0.497443pt}%
\definecolor{currentstroke}{rgb}{0.281412,0.155834,0.469201}%
\pgfsetstrokecolor{currentstroke}%
\pgfsetdash{}{0pt}%
\pgfpathmoveto{\pgfqpoint{7.929365in}{2.388569in}}%
\pgfpathlineto{\pgfqpoint{7.880893in}{2.399809in}}%
\pgfusepath{stroke}%
\end{pgfscope}%
\begin{pgfscope}%
\pgfpathrectangle{\pgfqpoint{6.720588in}{1.750000in}}{\pgfqpoint{2.279412in}{2.004545in}}%
\pgfusepath{clip}%
\pgfsetbuttcap%
\pgfsetroundjoin%
\pgfsetlinewidth{0.497071pt}%
\definecolor{currentstroke}{rgb}{0.281412,0.155834,0.469201}%
\pgfsetstrokecolor{currentstroke}%
\pgfsetdash{}{0pt}%
\pgfpathmoveto{\pgfqpoint{7.880893in}{2.399809in}}%
\pgfpathlineto{\pgfqpoint{7.833159in}{2.413268in}}%
\pgfusepath{stroke}%
\end{pgfscope}%
\begin{pgfscope}%
\pgfpathrectangle{\pgfqpoint{6.720588in}{1.750000in}}{\pgfqpoint{2.279412in}{2.004545in}}%
\pgfusepath{clip}%
\pgfsetbuttcap%
\pgfsetroundjoin%
\pgfsetlinewidth{0.535976pt}%
\definecolor{currentstroke}{rgb}{0.277134,0.185228,0.489898}%
\pgfsetstrokecolor{currentstroke}%
\pgfsetdash{}{0pt}%
\pgfpathmoveto{\pgfqpoint{7.833159in}{2.413268in}}%
\pgfpathlineto{\pgfqpoint{7.786605in}{2.429438in}}%
\pgfusepath{stroke}%
\end{pgfscope}%
\begin{pgfscope}%
\pgfpathrectangle{\pgfqpoint{6.720588in}{1.750000in}}{\pgfqpoint{2.279412in}{2.004545in}}%
\pgfusepath{clip}%
\pgfsetbuttcap%
\pgfsetroundjoin%
\pgfsetlinewidth{0.525679pt}%
\definecolor{currentstroke}{rgb}{0.278826,0.175490,0.483397}%
\pgfsetstrokecolor{currentstroke}%
\pgfsetdash{}{0pt}%
\pgfpathmoveto{\pgfqpoint{7.786605in}{2.429438in}}%
\pgfpathlineto{\pgfqpoint{7.742225in}{2.449625in}}%
\pgfusepath{stroke}%
\end{pgfscope}%
\begin{pgfscope}%
\pgfpathrectangle{\pgfqpoint{6.720588in}{1.750000in}}{\pgfqpoint{2.279412in}{2.004545in}}%
\pgfusepath{clip}%
\pgfsetbuttcap%
\pgfsetroundjoin%
\pgfsetlinewidth{0.520914pt}%
\definecolor{currentstroke}{rgb}{0.278826,0.175490,0.483397}%
\pgfsetstrokecolor{currentstroke}%
\pgfsetdash{}{0pt}%
\pgfpathmoveto{\pgfqpoint{7.742225in}{2.449625in}}%
\pgfpathlineto{\pgfqpoint{7.700602in}{2.474060in}}%
\pgfusepath{stroke}%
\end{pgfscope}%
\begin{pgfscope}%
\pgfpathrectangle{\pgfqpoint{6.720588in}{1.750000in}}{\pgfqpoint{2.279412in}{2.004545in}}%
\pgfusepath{clip}%
\pgfsetbuttcap%
\pgfsetroundjoin%
\pgfsetlinewidth{0.320176pt}%
\definecolor{currentstroke}{rgb}{0.269944,0.014625,0.341379}%
\pgfsetstrokecolor{currentstroke}%
\pgfsetdash{}{0pt}%
\pgfpathmoveto{\pgfqpoint{8.578381in}{2.436525in}}%
\pgfpathlineto{\pgfqpoint{8.528253in}{2.437713in}}%
\pgfusepath{stroke}%
\end{pgfscope}%
\begin{pgfscope}%
\pgfpathrectangle{\pgfqpoint{6.720588in}{1.750000in}}{\pgfqpoint{2.279412in}{2.004545in}}%
\pgfusepath{clip}%
\pgfsetbuttcap%
\pgfsetroundjoin%
\pgfsetlinewidth{0.319231pt}%
\definecolor{currentstroke}{rgb}{0.269944,0.014625,0.341379}%
\pgfsetstrokecolor{currentstroke}%
\pgfsetdash{}{0pt}%
\pgfpathmoveto{\pgfqpoint{8.528253in}{2.437713in}}%
\pgfpathlineto{\pgfqpoint{8.478124in}{2.438092in}}%
\pgfusepath{stroke}%
\end{pgfscope}%
\begin{pgfscope}%
\pgfpathrectangle{\pgfqpoint{6.720588in}{1.750000in}}{\pgfqpoint{2.279412in}{2.004545in}}%
\pgfusepath{clip}%
\pgfsetbuttcap%
\pgfsetroundjoin%
\pgfsetlinewidth{0.316919pt}%
\definecolor{currentstroke}{rgb}{0.269944,0.014625,0.341379}%
\pgfsetstrokecolor{currentstroke}%
\pgfsetdash{}{0pt}%
\pgfpathmoveto{\pgfqpoint{8.478124in}{2.438092in}}%
\pgfpathlineto{\pgfqpoint{8.427994in}{2.438320in}}%
\pgfusepath{stroke}%
\end{pgfscope}%
\begin{pgfscope}%
\pgfpathrectangle{\pgfqpoint{6.720588in}{1.750000in}}{\pgfqpoint{2.279412in}{2.004545in}}%
\pgfusepath{clip}%
\pgfsetbuttcap%
\pgfsetroundjoin%
\pgfsetlinewidth{0.337112pt}%
\definecolor{currentstroke}{rgb}{0.273809,0.031497,0.358853}%
\pgfsetstrokecolor{currentstroke}%
\pgfsetdash{}{0pt}%
\pgfpathmoveto{\pgfqpoint{8.427994in}{2.438320in}}%
\pgfpathlineto{\pgfqpoint{8.377858in}{2.439237in}}%
\pgfusepath{stroke}%
\end{pgfscope}%
\begin{pgfscope}%
\pgfpathrectangle{\pgfqpoint{6.720588in}{1.750000in}}{\pgfqpoint{2.279412in}{2.004545in}}%
\pgfusepath{clip}%
\pgfsetbuttcap%
\pgfsetroundjoin%
\pgfsetlinewidth{0.362245pt}%
\definecolor{currentstroke}{rgb}{0.277018,0.050344,0.375715}%
\pgfsetstrokecolor{currentstroke}%
\pgfsetdash{}{0pt}%
\pgfpathmoveto{\pgfqpoint{8.377858in}{2.439237in}}%
\pgfpathlineto{\pgfqpoint{8.327729in}{2.440522in}}%
\pgfusepath{stroke}%
\end{pgfscope}%
\begin{pgfscope}%
\pgfpathrectangle{\pgfqpoint{6.720588in}{1.750000in}}{\pgfqpoint{2.279412in}{2.004545in}}%
\pgfusepath{clip}%
\pgfsetbuttcap%
\pgfsetroundjoin%
\pgfsetlinewidth{0.377348pt}%
\definecolor{currentstroke}{rgb}{0.279566,0.067836,0.391917}%
\pgfsetstrokecolor{currentstroke}%
\pgfsetdash{}{0pt}%
\pgfpathmoveto{\pgfqpoint{8.327729in}{2.440522in}}%
\pgfpathlineto{\pgfqpoint{8.277609in}{2.442020in}}%
\pgfusepath{stroke}%
\end{pgfscope}%
\begin{pgfscope}%
\pgfpathrectangle{\pgfqpoint{6.720588in}{1.750000in}}{\pgfqpoint{2.279412in}{2.004545in}}%
\pgfusepath{clip}%
\pgfsetbuttcap%
\pgfsetroundjoin%
\pgfsetlinewidth{0.411846pt}%
\definecolor{currentstroke}{rgb}{0.282327,0.094955,0.417331}%
\pgfsetstrokecolor{currentstroke}%
\pgfsetdash{}{0pt}%
\pgfpathmoveto{\pgfqpoint{8.277609in}{2.442020in}}%
\pgfpathlineto{\pgfqpoint{8.227500in}{2.443846in}}%
\pgfusepath{stroke}%
\end{pgfscope}%
\begin{pgfscope}%
\pgfpathrectangle{\pgfqpoint{6.720588in}{1.750000in}}{\pgfqpoint{2.279412in}{2.004545in}}%
\pgfusepath{clip}%
\pgfsetbuttcap%
\pgfsetroundjoin%
\pgfsetlinewidth{0.447010pt}%
\definecolor{currentstroke}{rgb}{0.283229,0.120777,0.440584}%
\pgfsetstrokecolor{currentstroke}%
\pgfsetdash{}{0pt}%
\pgfpathmoveto{\pgfqpoint{8.227500in}{2.443846in}}%
\pgfpathlineto{\pgfqpoint{8.177421in}{2.446168in}}%
\pgfusepath{stroke}%
\end{pgfscope}%
\begin{pgfscope}%
\pgfpathrectangle{\pgfqpoint{6.720588in}{1.750000in}}{\pgfqpoint{2.279412in}{2.004545in}}%
\pgfusepath{clip}%
\pgfsetbuttcap%
\pgfsetroundjoin%
\pgfsetlinewidth{0.478393pt}%
\definecolor{currentstroke}{rgb}{0.282623,0.140926,0.457517}%
\pgfsetstrokecolor{currentstroke}%
\pgfsetdash{}{0pt}%
\pgfpathmoveto{\pgfqpoint{8.177421in}{2.446168in}}%
\pgfpathlineto{\pgfqpoint{8.127384in}{2.449125in}}%
\pgfusepath{stroke}%
\end{pgfscope}%
\begin{pgfscope}%
\pgfpathrectangle{\pgfqpoint{6.720588in}{1.750000in}}{\pgfqpoint{2.279412in}{2.004545in}}%
\pgfusepath{clip}%
\pgfsetbuttcap%
\pgfsetroundjoin%
\pgfsetlinewidth{0.527092pt}%
\definecolor{currentstroke}{rgb}{0.278826,0.175490,0.483397}%
\pgfsetstrokecolor{currentstroke}%
\pgfsetdash{}{0pt}%
\pgfpathmoveto{\pgfqpoint{8.127384in}{2.449125in}}%
\pgfpathlineto{\pgfqpoint{8.077389in}{2.452588in}}%
\pgfusepath{stroke}%
\end{pgfscope}%
\begin{pgfscope}%
\pgfpathrectangle{\pgfqpoint{6.720588in}{1.750000in}}{\pgfqpoint{2.279412in}{2.004545in}}%
\pgfusepath{clip}%
\pgfsetbuttcap%
\pgfsetroundjoin%
\pgfsetlinewidth{0.528569pt}%
\definecolor{currentstroke}{rgb}{0.278012,0.180367,0.486697}%
\pgfsetstrokecolor{currentstroke}%
\pgfsetdash{}{0pt}%
\pgfpathmoveto{\pgfqpoint{8.077389in}{2.452588in}}%
\pgfpathlineto{\pgfqpoint{8.027477in}{2.456862in}}%
\pgfusepath{stroke}%
\end{pgfscope}%
\begin{pgfscope}%
\pgfpathrectangle{\pgfqpoint{6.720588in}{1.750000in}}{\pgfqpoint{2.279412in}{2.004545in}}%
\pgfusepath{clip}%
\pgfsetbuttcap%
\pgfsetroundjoin%
\pgfsetlinewidth{0.573626pt}%
\definecolor{currentstroke}{rgb}{0.271828,0.209303,0.504434}%
\pgfsetstrokecolor{currentstroke}%
\pgfsetdash{}{0pt}%
\pgfpathmoveto{\pgfqpoint{8.027477in}{2.456862in}}%
\pgfpathlineto{\pgfqpoint{7.977688in}{2.462124in}}%
\pgfusepath{stroke}%
\end{pgfscope}%
\begin{pgfscope}%
\pgfpathrectangle{\pgfqpoint{6.720588in}{1.750000in}}{\pgfqpoint{2.279412in}{2.004545in}}%
\pgfusepath{clip}%
\pgfsetbuttcap%
\pgfsetroundjoin%
\pgfsetlinewidth{0.604011pt}%
\definecolor{currentstroke}{rgb}{0.265145,0.232956,0.516599}%
\pgfsetstrokecolor{currentstroke}%
\pgfsetdash{}{0pt}%
\pgfpathmoveto{\pgfqpoint{7.977688in}{2.462124in}}%
\pgfpathlineto{\pgfqpoint{7.928099in}{2.468657in}}%
\pgfusepath{stroke}%
\end{pgfscope}%
\begin{pgfscope}%
\pgfpathrectangle{\pgfqpoint{6.720588in}{1.750000in}}{\pgfqpoint{2.279412in}{2.004545in}}%
\pgfusepath{clip}%
\pgfsetbuttcap%
\pgfsetroundjoin%
\pgfsetlinewidth{0.638196pt}%
\definecolor{currentstroke}{rgb}{0.257322,0.256130,0.526563}%
\pgfsetstrokecolor{currentstroke}%
\pgfsetdash{}{0pt}%
\pgfpathmoveto{\pgfqpoint{7.928099in}{2.468657in}}%
\pgfpathlineto{\pgfqpoint{7.878928in}{2.477224in}}%
\pgfusepath{stroke}%
\end{pgfscope}%
\begin{pgfscope}%
\pgfpathrectangle{\pgfqpoint{6.720588in}{1.750000in}}{\pgfqpoint{2.279412in}{2.004545in}}%
\pgfusepath{clip}%
\pgfsetbuttcap%
\pgfsetroundjoin%
\pgfsetlinewidth{0.597134pt}%
\definecolor{currentstroke}{rgb}{0.266580,0.228262,0.514349}%
\pgfsetstrokecolor{currentstroke}%
\pgfsetdash{}{0pt}%
\pgfpathmoveto{\pgfqpoint{7.878928in}{2.477224in}}%
\pgfpathlineto{\pgfqpoint{7.830319in}{2.488009in}}%
\pgfusepath{stroke}%
\end{pgfscope}%
\begin{pgfscope}%
\pgfpathrectangle{\pgfqpoint{6.720588in}{1.750000in}}{\pgfqpoint{2.279412in}{2.004545in}}%
\pgfusepath{clip}%
\pgfsetbuttcap%
\pgfsetroundjoin%
\pgfsetlinewidth{0.634633pt}%
\definecolor{currentstroke}{rgb}{0.258965,0.251537,0.524736}%
\pgfsetstrokecolor{currentstroke}%
\pgfsetdash{}{0pt}%
\pgfpathmoveto{\pgfqpoint{7.830319in}{2.488009in}}%
\pgfpathlineto{\pgfqpoint{7.782498in}{2.501160in}}%
\pgfusepath{stroke}%
\end{pgfscope}%
\begin{pgfscope}%
\pgfpathrectangle{\pgfqpoint{6.720588in}{1.750000in}}{\pgfqpoint{2.279412in}{2.004545in}}%
\pgfusepath{clip}%
\pgfsetbuttcap%
\pgfsetroundjoin%
\pgfsetlinewidth{0.598576pt}%
\definecolor{currentstroke}{rgb}{0.266580,0.228262,0.514349}%
\pgfsetstrokecolor{currentstroke}%
\pgfsetdash{}{0pt}%
\pgfpathmoveto{\pgfqpoint{7.782498in}{2.501160in}}%
\pgfpathlineto{\pgfqpoint{7.735642in}{2.516782in}}%
\pgfusepath{stroke}%
\end{pgfscope}%
\begin{pgfscope}%
\pgfpathrectangle{\pgfqpoint{6.720588in}{1.750000in}}{\pgfqpoint{2.279412in}{2.004545in}}%
\pgfusepath{clip}%
\pgfsetbuttcap%
\pgfsetroundjoin%
\pgfsetlinewidth{0.617755pt}%
\definecolor{currentstroke}{rgb}{0.262138,0.242286,0.520837}%
\pgfsetstrokecolor{currentstroke}%
\pgfsetdash{}{0pt}%
\pgfpathmoveto{\pgfqpoint{7.735642in}{2.516782in}}%
\pgfpathlineto{\pgfqpoint{7.690809in}{2.536264in}}%
\pgfusepath{stroke}%
\end{pgfscope}%
\begin{pgfscope}%
\pgfpathrectangle{\pgfqpoint{6.720588in}{1.750000in}}{\pgfqpoint{2.279412in}{2.004545in}}%
\pgfusepath{clip}%
\pgfsetbuttcap%
\pgfsetroundjoin%
\pgfsetlinewidth{0.553029pt}%
\definecolor{currentstroke}{rgb}{0.275191,0.194905,0.496005}%
\pgfsetstrokecolor{currentstroke}%
\pgfsetdash{}{0pt}%
\pgfpathmoveto{\pgfqpoint{7.690809in}{2.536264in}}%
\pgfpathlineto{\pgfqpoint{7.647938in}{2.558945in}}%
\pgfusepath{stroke}%
\end{pgfscope}%
\begin{pgfscope}%
\pgfpathrectangle{\pgfqpoint{6.720588in}{1.750000in}}{\pgfqpoint{2.279412in}{2.004545in}}%
\pgfusepath{clip}%
\pgfsetbuttcap%
\pgfsetroundjoin%
\pgfsetlinewidth{0.621867pt}%
\definecolor{currentstroke}{rgb}{0.262138,0.242286,0.520837}%
\pgfsetstrokecolor{currentstroke}%
\pgfsetdash{}{0pt}%
\pgfpathmoveto{\pgfqpoint{7.647938in}{2.558945in}}%
\pgfpathlineto{\pgfqpoint{7.606037in}{2.582968in}}%
\pgfusepath{stroke}%
\end{pgfscope}%
\begin{pgfscope}%
\pgfpathrectangle{\pgfqpoint{6.720588in}{1.750000in}}{\pgfqpoint{2.279412in}{2.004545in}}%
\pgfusepath{clip}%
\pgfsetbuttcap%
\pgfsetroundjoin%
\pgfsetlinewidth{0.618773pt}%
\definecolor{currentstroke}{rgb}{0.262138,0.242286,0.520837}%
\pgfsetstrokecolor{currentstroke}%
\pgfsetdash{}{0pt}%
\pgfpathmoveto{\pgfqpoint{7.606037in}{2.582968in}}%
\pgfpathlineto{\pgfqpoint{7.565372in}{2.608516in}}%
\pgfusepath{stroke}%
\end{pgfscope}%
\begin{pgfscope}%
\pgfpathrectangle{\pgfqpoint{6.720588in}{1.750000in}}{\pgfqpoint{2.279412in}{2.004545in}}%
\pgfusepath{clip}%
\pgfsetbuttcap%
\pgfsetroundjoin%
\pgfsetlinewidth{0.313257pt}%
\definecolor{currentstroke}{rgb}{0.268510,0.009605,0.335427}%
\pgfsetstrokecolor{currentstroke}%
\pgfsetdash{}{0pt}%
\pgfpathmoveto{\pgfqpoint{8.603141in}{2.528385in}}%
\pgfpathlineto{\pgfqpoint{8.578381in}{2.526739in}}%
\pgfusepath{stroke}%
\end{pgfscope}%
\begin{pgfscope}%
\pgfpathrectangle{\pgfqpoint{6.720588in}{1.750000in}}{\pgfqpoint{2.279412in}{2.004545in}}%
\pgfusepath{clip}%
\pgfsetbuttcap%
\pgfsetroundjoin%
\pgfsetlinewidth{0.305613pt}%
\definecolor{currentstroke}{rgb}{0.267004,0.004874,0.329415}%
\pgfsetstrokecolor{currentstroke}%
\pgfsetdash{}{0pt}%
\pgfpathmoveto{\pgfqpoint{8.578381in}{2.526739in}}%
\pgfpathlineto{\pgfqpoint{8.578381in}{2.526739in}}%
\pgfusepath{stroke}%
\end{pgfscope}%
\begin{pgfscope}%
\pgfpathrectangle{\pgfqpoint{6.720588in}{1.750000in}}{\pgfqpoint{2.279412in}{2.004545in}}%
\pgfusepath{clip}%
\pgfsetbuttcap%
\pgfsetroundjoin%
\pgfsetlinewidth{0.305613pt}%
\definecolor{currentstroke}{rgb}{0.267004,0.004874,0.329415}%
\pgfsetstrokecolor{currentstroke}%
\pgfsetdash{}{0pt}%
\pgfpathmoveto{\pgfqpoint{8.578381in}{2.526739in}}%
\pgfpathlineto{\pgfqpoint{8.531360in}{2.524329in}}%
\pgfusepath{stroke}%
\end{pgfscope}%
\begin{pgfscope}%
\pgfpathrectangle{\pgfqpoint{6.720588in}{1.750000in}}{\pgfqpoint{2.279412in}{2.004545in}}%
\pgfusepath{clip}%
\pgfsetbuttcap%
\pgfsetroundjoin%
\pgfsetlinewidth{0.329241pt}%
\definecolor{currentstroke}{rgb}{0.272594,0.025563,0.353093}%
\pgfsetstrokecolor{currentstroke}%
\pgfsetdash{}{0pt}%
\pgfpathmoveto{\pgfqpoint{8.531360in}{2.524329in}}%
\pgfpathlineto{\pgfqpoint{8.481212in}{2.524722in}}%
\pgfusepath{stroke}%
\end{pgfscope}%
\begin{pgfscope}%
\pgfpathrectangle{\pgfqpoint{6.720588in}{1.750000in}}{\pgfqpoint{2.279412in}{2.004545in}}%
\pgfusepath{clip}%
\pgfsetbuttcap%
\pgfsetroundjoin%
\pgfsetlinewidth{0.328978pt}%
\definecolor{currentstroke}{rgb}{0.272594,0.025563,0.353093}%
\pgfsetstrokecolor{currentstroke}%
\pgfsetdash{}{0pt}%
\pgfpathmoveto{\pgfqpoint{8.481212in}{2.524722in}}%
\pgfpathlineto{\pgfqpoint{8.431071in}{2.525293in}}%
\pgfusepath{stroke}%
\end{pgfscope}%
\begin{pgfscope}%
\pgfpathrectangle{\pgfqpoint{6.720588in}{1.750000in}}{\pgfqpoint{2.279412in}{2.004545in}}%
\pgfusepath{clip}%
\pgfsetbuttcap%
\pgfsetroundjoin%
\pgfsetlinewidth{0.339910pt}%
\definecolor{currentstroke}{rgb}{0.273809,0.031497,0.358853}%
\pgfsetstrokecolor{currentstroke}%
\pgfsetdash{}{0pt}%
\pgfpathmoveto{\pgfqpoint{8.431071in}{2.525293in}}%
\pgfpathlineto{\pgfqpoint{8.380929in}{2.525844in}}%
\pgfusepath{stroke}%
\end{pgfscope}%
\begin{pgfscope}%
\pgfpathrectangle{\pgfqpoint{6.720588in}{1.750000in}}{\pgfqpoint{2.279412in}{2.004545in}}%
\pgfusepath{clip}%
\pgfsetbuttcap%
\pgfsetroundjoin%
\pgfsetlinewidth{0.361629pt}%
\definecolor{currentstroke}{rgb}{0.277018,0.050344,0.375715}%
\pgfsetstrokecolor{currentstroke}%
\pgfsetdash{}{0pt}%
\pgfpathmoveto{\pgfqpoint{8.380929in}{2.525844in}}%
\pgfpathlineto{\pgfqpoint{8.330785in}{2.526440in}}%
\pgfusepath{stroke}%
\end{pgfscope}%
\begin{pgfscope}%
\pgfpathrectangle{\pgfqpoint{6.720588in}{1.750000in}}{\pgfqpoint{2.279412in}{2.004545in}}%
\pgfusepath{clip}%
\pgfsetbuttcap%
\pgfsetroundjoin%
\pgfsetlinewidth{0.387158pt}%
\definecolor{currentstroke}{rgb}{0.280267,0.073417,0.397163}%
\pgfsetstrokecolor{currentstroke}%
\pgfsetdash{}{0pt}%
\pgfpathmoveto{\pgfqpoint{8.330785in}{2.526440in}}%
\pgfpathlineto{\pgfqpoint{8.280651in}{2.527596in}}%
\pgfusepath{stroke}%
\end{pgfscope}%
\begin{pgfscope}%
\pgfpathrectangle{\pgfqpoint{6.720588in}{1.750000in}}{\pgfqpoint{2.279412in}{2.004545in}}%
\pgfusepath{clip}%
\pgfsetbuttcap%
\pgfsetroundjoin%
\pgfsetlinewidth{0.424953pt}%
\definecolor{currentstroke}{rgb}{0.282910,0.105393,0.426902}%
\pgfsetstrokecolor{currentstroke}%
\pgfsetdash{}{0pt}%
\pgfpathmoveto{\pgfqpoint{8.280651in}{2.527596in}}%
\pgfpathlineto{\pgfqpoint{8.230525in}{2.528953in}}%
\pgfusepath{stroke}%
\end{pgfscope}%
\begin{pgfscope}%
\pgfpathrectangle{\pgfqpoint{6.720588in}{1.750000in}}{\pgfqpoint{2.279412in}{2.004545in}}%
\pgfusepath{clip}%
\pgfsetbuttcap%
\pgfsetroundjoin%
\pgfsetlinewidth{0.468377pt}%
\definecolor{currentstroke}{rgb}{0.282884,0.135920,0.453427}%
\pgfsetstrokecolor{currentstroke}%
\pgfsetdash{}{0pt}%
\pgfpathmoveto{\pgfqpoint{8.230525in}{2.528953in}}%
\pgfpathlineto{\pgfqpoint{8.180405in}{2.530476in}}%
\pgfusepath{stroke}%
\end{pgfscope}%
\begin{pgfscope}%
\pgfpathrectangle{\pgfqpoint{6.720588in}{1.750000in}}{\pgfqpoint{2.279412in}{2.004545in}}%
\pgfusepath{clip}%
\pgfsetbuttcap%
\pgfsetroundjoin%
\pgfsetlinewidth{0.532652pt}%
\definecolor{currentstroke}{rgb}{0.278012,0.180367,0.486697}%
\pgfsetstrokecolor{currentstroke}%
\pgfsetdash{}{0pt}%
\pgfpathmoveto{\pgfqpoint{8.180405in}{2.530476in}}%
\pgfpathlineto{\pgfqpoint{8.130296in}{2.532280in}}%
\pgfusepath{stroke}%
\end{pgfscope}%
\begin{pgfscope}%
\pgfpathrectangle{\pgfqpoint{6.720588in}{1.750000in}}{\pgfqpoint{2.279412in}{2.004545in}}%
\pgfusepath{clip}%
\pgfsetbuttcap%
\pgfsetroundjoin%
\pgfsetlinewidth{0.616629pt}%
\definecolor{currentstroke}{rgb}{0.263663,0.237631,0.518762}%
\pgfsetstrokecolor{currentstroke}%
\pgfsetdash{}{0pt}%
\pgfpathmoveto{\pgfqpoint{8.130296in}{2.532280in}}%
\pgfpathlineto{\pgfqpoint{8.080220in}{2.534680in}}%
\pgfusepath{stroke}%
\end{pgfscope}%
\begin{pgfscope}%
\pgfpathrectangle{\pgfqpoint{6.720588in}{1.750000in}}{\pgfqpoint{2.279412in}{2.004545in}}%
\pgfusepath{clip}%
\pgfsetbuttcap%
\pgfsetroundjoin%
\pgfsetlinewidth{0.661227pt}%
\definecolor{currentstroke}{rgb}{0.252194,0.269783,0.531579}%
\pgfsetstrokecolor{currentstroke}%
\pgfsetdash{}{0pt}%
\pgfpathmoveto{\pgfqpoint{8.080220in}{2.534680in}}%
\pgfpathlineto{\pgfqpoint{8.030201in}{2.537868in}}%
\pgfusepath{stroke}%
\end{pgfscope}%
\begin{pgfscope}%
\pgfpathrectangle{\pgfqpoint{6.720588in}{1.750000in}}{\pgfqpoint{2.279412in}{2.004545in}}%
\pgfusepath{clip}%
\pgfsetbuttcap%
\pgfsetroundjoin%
\pgfsetlinewidth{0.682459pt}%
\definecolor{currentstroke}{rgb}{0.246811,0.283237,0.535941}%
\pgfsetstrokecolor{currentstroke}%
\pgfsetdash{}{0pt}%
\pgfpathmoveto{\pgfqpoint{8.030201in}{2.537868in}}%
\pgfpathlineto{\pgfqpoint{7.980248in}{2.541773in}}%
\pgfusepath{stroke}%
\end{pgfscope}%
\begin{pgfscope}%
\pgfpathrectangle{\pgfqpoint{6.720588in}{1.750000in}}{\pgfqpoint{2.279412in}{2.004545in}}%
\pgfusepath{clip}%
\pgfsetbuttcap%
\pgfsetroundjoin%
\pgfsetlinewidth{0.733285pt}%
\definecolor{currentstroke}{rgb}{0.233603,0.313828,0.543914}%
\pgfsetstrokecolor{currentstroke}%
\pgfsetdash{}{0pt}%
\pgfpathmoveto{\pgfqpoint{7.980248in}{2.541773in}}%
\pgfpathlineto{\pgfqpoint{7.930398in}{2.546568in}}%
\pgfusepath{stroke}%
\end{pgfscope}%
\begin{pgfscope}%
\pgfpathrectangle{\pgfqpoint{6.720588in}{1.750000in}}{\pgfqpoint{2.279412in}{2.004545in}}%
\pgfusepath{clip}%
\pgfsetbuttcap%
\pgfsetroundjoin%
\pgfsetlinewidth{0.316983pt}%
\definecolor{currentstroke}{rgb}{0.269944,0.014625,0.341379}%
\pgfsetstrokecolor{currentstroke}%
\pgfsetdash{}{0pt}%
\pgfpathmoveto{\pgfqpoint{8.578381in}{2.977807in}}%
\pgfpathlineto{\pgfqpoint{8.528237in}{2.977961in}}%
\pgfusepath{stroke}%
\end{pgfscope}%
\begin{pgfscope}%
\pgfpathrectangle{\pgfqpoint{6.720588in}{1.750000in}}{\pgfqpoint{2.279412in}{2.004545in}}%
\pgfusepath{clip}%
\pgfsetbuttcap%
\pgfsetroundjoin%
\pgfsetlinewidth{0.323491pt}%
\definecolor{currentstroke}{rgb}{0.271305,0.019942,0.347269}%
\pgfsetstrokecolor{currentstroke}%
\pgfsetdash{}{0pt}%
\pgfpathmoveto{\pgfqpoint{8.528237in}{2.977961in}}%
\pgfpathlineto{\pgfqpoint{8.478092in}{2.978522in}}%
\pgfusepath{stroke}%
\end{pgfscope}%
\begin{pgfscope}%
\pgfpathrectangle{\pgfqpoint{6.720588in}{1.750000in}}{\pgfqpoint{2.279412in}{2.004545in}}%
\pgfusepath{clip}%
\pgfsetbuttcap%
\pgfsetroundjoin%
\pgfsetlinewidth{0.328514pt}%
\definecolor{currentstroke}{rgb}{0.271305,0.019942,0.347269}%
\pgfsetstrokecolor{currentstroke}%
\pgfsetdash{}{0pt}%
\pgfpathmoveto{\pgfqpoint{8.478092in}{2.978522in}}%
\pgfpathlineto{\pgfqpoint{8.427947in}{2.978272in}}%
\pgfusepath{stroke}%
\end{pgfscope}%
\begin{pgfscope}%
\pgfpathrectangle{\pgfqpoint{6.720588in}{1.750000in}}{\pgfqpoint{2.279412in}{2.004545in}}%
\pgfusepath{clip}%
\pgfsetbuttcap%
\pgfsetroundjoin%
\pgfsetlinewidth{0.342569pt}%
\definecolor{currentstroke}{rgb}{0.274952,0.037752,0.364543}%
\pgfsetstrokecolor{currentstroke}%
\pgfsetdash{}{0pt}%
\pgfpathmoveto{\pgfqpoint{8.427947in}{2.978272in}}%
\pgfpathlineto{\pgfqpoint{8.377804in}{2.977486in}}%
\pgfusepath{stroke}%
\end{pgfscope}%
\begin{pgfscope}%
\pgfpathrectangle{\pgfqpoint{6.720588in}{1.750000in}}{\pgfqpoint{2.279412in}{2.004545in}}%
\pgfusepath{clip}%
\pgfsetbuttcap%
\pgfsetroundjoin%
\pgfsetlinewidth{0.357382pt}%
\definecolor{currentstroke}{rgb}{0.277018,0.050344,0.375715}%
\pgfsetstrokecolor{currentstroke}%
\pgfsetdash{}{0pt}%
\pgfpathmoveto{\pgfqpoint{8.377804in}{2.977486in}}%
\pgfpathlineto{\pgfqpoint{8.327659in}{2.976789in}}%
\pgfusepath{stroke}%
\end{pgfscope}%
\begin{pgfscope}%
\pgfpathrectangle{\pgfqpoint{6.720588in}{1.750000in}}{\pgfqpoint{2.279412in}{2.004545in}}%
\pgfusepath{clip}%
\pgfsetbuttcap%
\pgfsetroundjoin%
\pgfsetlinewidth{0.386344pt}%
\definecolor{currentstroke}{rgb}{0.280267,0.073417,0.397163}%
\pgfsetstrokecolor{currentstroke}%
\pgfsetdash{}{0pt}%
\pgfpathmoveto{\pgfqpoint{8.327659in}{2.976789in}}%
\pgfpathlineto{\pgfqpoint{8.277514in}{2.976168in}}%
\pgfusepath{stroke}%
\end{pgfscope}%
\begin{pgfscope}%
\pgfpathrectangle{\pgfqpoint{6.720588in}{1.750000in}}{\pgfqpoint{2.279412in}{2.004545in}}%
\pgfusepath{clip}%
\pgfsetbuttcap%
\pgfsetroundjoin%
\pgfsetlinewidth{0.426921pt}%
\definecolor{currentstroke}{rgb}{0.282910,0.105393,0.426902}%
\pgfsetstrokecolor{currentstroke}%
\pgfsetdash{}{0pt}%
\pgfpathmoveto{\pgfqpoint{8.277514in}{2.976168in}}%
\pgfpathlineto{\pgfqpoint{8.227376in}{2.975258in}}%
\pgfusepath{stroke}%
\end{pgfscope}%
\begin{pgfscope}%
\pgfpathrectangle{\pgfqpoint{6.720588in}{1.750000in}}{\pgfqpoint{2.279412in}{2.004545in}}%
\pgfusepath{clip}%
\pgfsetbuttcap%
\pgfsetroundjoin%
\pgfsetlinewidth{0.453322pt}%
\definecolor{currentstroke}{rgb}{0.283187,0.125848,0.444960}%
\pgfsetstrokecolor{currentstroke}%
\pgfsetdash{}{0pt}%
\pgfpathmoveto{\pgfqpoint{8.227376in}{2.975258in}}%
\pgfpathlineto{\pgfqpoint{8.177249in}{2.973879in}}%
\pgfusepath{stroke}%
\end{pgfscope}%
\begin{pgfscope}%
\pgfpathrectangle{\pgfqpoint{6.720588in}{1.750000in}}{\pgfqpoint{2.279412in}{2.004545in}}%
\pgfusepath{clip}%
\pgfsetbuttcap%
\pgfsetroundjoin%
\pgfsetlinewidth{0.505999pt}%
\definecolor{currentstroke}{rgb}{0.280868,0.160771,0.472899}%
\pgfsetstrokecolor{currentstroke}%
\pgfsetdash{}{0pt}%
\pgfpathmoveto{\pgfqpoint{8.177249in}{2.973879in}}%
\pgfpathlineto{\pgfqpoint{8.127140in}{2.972089in}}%
\pgfusepath{stroke}%
\end{pgfscope}%
\begin{pgfscope}%
\pgfpathrectangle{\pgfqpoint{6.720588in}{1.750000in}}{\pgfqpoint{2.279412in}{2.004545in}}%
\pgfusepath{clip}%
\pgfsetbuttcap%
\pgfsetroundjoin%
\pgfsetlinewidth{0.553892pt}%
\definecolor{currentstroke}{rgb}{0.275191,0.194905,0.496005}%
\pgfsetstrokecolor{currentstroke}%
\pgfsetdash{}{0pt}%
\pgfpathmoveto{\pgfqpoint{8.127140in}{2.972089in}}%
\pgfpathlineto{\pgfqpoint{8.077063in}{2.969713in}}%
\pgfusepath{stroke}%
\end{pgfscope}%
\begin{pgfscope}%
\pgfpathrectangle{\pgfqpoint{6.720588in}{1.750000in}}{\pgfqpoint{2.279412in}{2.004545in}}%
\pgfusepath{clip}%
\pgfsetbuttcap%
\pgfsetroundjoin%
\pgfsetlinewidth{0.581712pt}%
\definecolor{currentstroke}{rgb}{0.270595,0.214069,0.507052}%
\pgfsetstrokecolor{currentstroke}%
\pgfsetdash{}{0pt}%
\pgfpathmoveto{\pgfqpoint{8.077063in}{2.969713in}}%
\pgfpathlineto{\pgfqpoint{8.027033in}{2.966670in}}%
\pgfusepath{stroke}%
\end{pgfscope}%
\begin{pgfscope}%
\pgfpathrectangle{\pgfqpoint{6.720588in}{1.750000in}}{\pgfqpoint{2.279412in}{2.004545in}}%
\pgfusepath{clip}%
\pgfsetbuttcap%
\pgfsetroundjoin%
\pgfsetlinewidth{0.604344pt}%
\definecolor{currentstroke}{rgb}{0.265145,0.232956,0.516599}%
\pgfsetstrokecolor{currentstroke}%
\pgfsetdash{}{0pt}%
\pgfpathmoveto{\pgfqpoint{8.027033in}{2.966670in}}%
\pgfpathlineto{\pgfqpoint{7.977073in}{2.962849in}}%
\pgfusepath{stroke}%
\end{pgfscope}%
\begin{pgfscope}%
\pgfpathrectangle{\pgfqpoint{6.720588in}{1.750000in}}{\pgfqpoint{2.279412in}{2.004545in}}%
\pgfusepath{clip}%
\pgfsetbuttcap%
\pgfsetroundjoin%
\pgfsetlinewidth{0.644747pt}%
\definecolor{currentstroke}{rgb}{0.255645,0.260703,0.528312}%
\pgfsetstrokecolor{currentstroke}%
\pgfsetdash{}{0pt}%
\pgfpathmoveto{\pgfqpoint{7.977073in}{2.962849in}}%
\pgfpathlineto{\pgfqpoint{7.927226in}{2.958026in}}%
\pgfusepath{stroke}%
\end{pgfscope}%
\begin{pgfscope}%
\pgfpathrectangle{\pgfqpoint{6.720588in}{1.750000in}}{\pgfqpoint{2.279412in}{2.004545in}}%
\pgfusepath{clip}%
\pgfsetbuttcap%
\pgfsetroundjoin%
\pgfsetlinewidth{0.665906pt}%
\definecolor{currentstroke}{rgb}{0.250425,0.274290,0.533103}%
\pgfsetstrokecolor{currentstroke}%
\pgfsetdash{}{0pt}%
\pgfpathmoveto{\pgfqpoint{7.927226in}{2.958026in}}%
\pgfpathlineto{\pgfqpoint{7.877572in}{2.951888in}}%
\pgfusepath{stroke}%
\end{pgfscope}%
\begin{pgfscope}%
\pgfpathrectangle{\pgfqpoint{6.720588in}{1.750000in}}{\pgfqpoint{2.279412in}{2.004545in}}%
\pgfusepath{clip}%
\pgfsetbuttcap%
\pgfsetroundjoin%
\pgfsetlinewidth{0.689042pt}%
\definecolor{currentstroke}{rgb}{0.244972,0.287675,0.537260}%
\pgfsetstrokecolor{currentstroke}%
\pgfsetdash{}{0pt}%
\pgfpathmoveto{\pgfqpoint{7.877572in}{2.951888in}}%
\pgfpathlineto{\pgfqpoint{7.828291in}{2.943832in}}%
\pgfusepath{stroke}%
\end{pgfscope}%
\begin{pgfscope}%
\pgfpathrectangle{\pgfqpoint{6.720588in}{1.750000in}}{\pgfqpoint{2.279412in}{2.004545in}}%
\pgfusepath{clip}%
\pgfsetbuttcap%
\pgfsetroundjoin%
\pgfsetlinewidth{0.642579pt}%
\definecolor{currentstroke}{rgb}{0.257322,0.256130,0.526563}%
\pgfsetstrokecolor{currentstroke}%
\pgfsetdash{}{0pt}%
\pgfpathmoveto{\pgfqpoint{7.828291in}{2.943832in}}%
\pgfpathlineto{\pgfqpoint{7.779592in}{2.933415in}}%
\pgfusepath{stroke}%
\end{pgfscope}%
\begin{pgfscope}%
\pgfpathrectangle{\pgfqpoint{6.720588in}{1.750000in}}{\pgfqpoint{2.279412in}{2.004545in}}%
\pgfusepath{clip}%
\pgfsetbuttcap%
\pgfsetroundjoin%
\pgfsetlinewidth{0.647166pt}%
\definecolor{currentstroke}{rgb}{0.255645,0.260703,0.528312}%
\pgfsetstrokecolor{currentstroke}%
\pgfsetdash{}{0pt}%
\pgfpathmoveto{\pgfqpoint{7.779592in}{2.933415in}}%
\pgfpathlineto{\pgfqpoint{7.731751in}{2.920322in}}%
\pgfusepath{stroke}%
\end{pgfscope}%
\begin{pgfscope}%
\pgfpathrectangle{\pgfqpoint{6.720588in}{1.750000in}}{\pgfqpoint{2.279412in}{2.004545in}}%
\pgfusepath{clip}%
\pgfsetbuttcap%
\pgfsetroundjoin%
\pgfsetlinewidth{0.319448pt}%
\definecolor{currentstroke}{rgb}{0.269944,0.014625,0.341379}%
\pgfsetstrokecolor{currentstroke}%
\pgfsetdash{}{0pt}%
\pgfpathmoveto{\pgfqpoint{8.578381in}{3.022913in}}%
\pgfpathlineto{\pgfqpoint{8.528259in}{3.022610in}}%
\pgfusepath{stroke}%
\end{pgfscope}%
\begin{pgfscope}%
\pgfpathrectangle{\pgfqpoint{6.720588in}{1.750000in}}{\pgfqpoint{2.279412in}{2.004545in}}%
\pgfusepath{clip}%
\pgfsetbuttcap%
\pgfsetroundjoin%
\pgfsetlinewidth{0.319735pt}%
\definecolor{currentstroke}{rgb}{0.269944,0.014625,0.341379}%
\pgfsetstrokecolor{currentstroke}%
\pgfsetdash{}{0pt}%
\pgfpathmoveto{\pgfqpoint{8.528259in}{3.022610in}}%
\pgfpathlineto{\pgfqpoint{8.478147in}{3.022519in}}%
\pgfusepath{stroke}%
\end{pgfscope}%
\begin{pgfscope}%
\pgfpathrectangle{\pgfqpoint{6.720588in}{1.750000in}}{\pgfqpoint{2.279412in}{2.004545in}}%
\pgfusepath{clip}%
\pgfsetbuttcap%
\pgfsetroundjoin%
\pgfsetlinewidth{0.325995pt}%
\definecolor{currentstroke}{rgb}{0.271305,0.019942,0.347269}%
\pgfsetstrokecolor{currentstroke}%
\pgfsetdash{}{0pt}%
\pgfpathmoveto{\pgfqpoint{8.478147in}{3.022519in}}%
\pgfpathlineto{\pgfqpoint{8.428029in}{3.022655in}}%
\pgfusepath{stroke}%
\end{pgfscope}%
\begin{pgfscope}%
\pgfpathrectangle{\pgfqpoint{6.720588in}{1.750000in}}{\pgfqpoint{2.279412in}{2.004545in}}%
\pgfusepath{clip}%
\pgfsetbuttcap%
\pgfsetroundjoin%
\pgfsetlinewidth{0.339974pt}%
\definecolor{currentstroke}{rgb}{0.273809,0.031497,0.358853}%
\pgfsetstrokecolor{currentstroke}%
\pgfsetdash{}{0pt}%
\pgfpathmoveto{\pgfqpoint{8.428029in}{3.022655in}}%
\pgfpathlineto{\pgfqpoint{8.377899in}{3.021682in}}%
\pgfusepath{stroke}%
\end{pgfscope}%
\begin{pgfscope}%
\pgfpathrectangle{\pgfqpoint{6.720588in}{1.750000in}}{\pgfqpoint{2.279412in}{2.004545in}}%
\pgfusepath{clip}%
\pgfsetbuttcap%
\pgfsetroundjoin%
\pgfsetlinewidth{0.356077pt}%
\definecolor{currentstroke}{rgb}{0.277018,0.050344,0.375715}%
\pgfsetstrokecolor{currentstroke}%
\pgfsetdash{}{0pt}%
\pgfpathmoveto{\pgfqpoint{8.377899in}{3.021682in}}%
\pgfpathlineto{\pgfqpoint{8.327776in}{3.020253in}}%
\pgfusepath{stroke}%
\end{pgfscope}%
\begin{pgfscope}%
\pgfpathrectangle{\pgfqpoint{6.720588in}{1.750000in}}{\pgfqpoint{2.279412in}{2.004545in}}%
\pgfusepath{clip}%
\pgfsetbuttcap%
\pgfsetroundjoin%
\pgfsetlinewidth{0.382162pt}%
\definecolor{currentstroke}{rgb}{0.279566,0.067836,0.391917}%
\pgfsetstrokecolor{currentstroke}%
\pgfsetdash{}{0pt}%
\pgfpathmoveto{\pgfqpoint{8.327776in}{3.020253in}}%
\pgfpathlineto{\pgfqpoint{8.277653in}{3.018833in}}%
\pgfusepath{stroke}%
\end{pgfscope}%
\begin{pgfscope}%
\pgfpathrectangle{\pgfqpoint{6.720588in}{1.750000in}}{\pgfqpoint{2.279412in}{2.004545in}}%
\pgfusepath{clip}%
\pgfsetbuttcap%
\pgfsetroundjoin%
\pgfsetlinewidth{0.410948pt}%
\definecolor{currentstroke}{rgb}{0.282327,0.094955,0.417331}%
\pgfsetstrokecolor{currentstroke}%
\pgfsetdash{}{0pt}%
\pgfpathmoveto{\pgfqpoint{8.277653in}{3.018833in}}%
\pgfpathlineto{\pgfqpoint{8.227549in}{3.016965in}}%
\pgfusepath{stroke}%
\end{pgfscope}%
\begin{pgfscope}%
\pgfpathrectangle{\pgfqpoint{6.720588in}{1.750000in}}{\pgfqpoint{2.279412in}{2.004545in}}%
\pgfusepath{clip}%
\pgfsetbuttcap%
\pgfsetroundjoin%
\pgfsetlinewidth{0.436133pt}%
\definecolor{currentstroke}{rgb}{0.283091,0.110553,0.431554}%
\pgfsetstrokecolor{currentstroke}%
\pgfsetdash{}{0pt}%
\pgfpathmoveto{\pgfqpoint{8.227549in}{3.016965in}}%
\pgfpathlineto{\pgfqpoint{8.177465in}{3.014700in}}%
\pgfusepath{stroke}%
\end{pgfscope}%
\begin{pgfscope}%
\pgfpathrectangle{\pgfqpoint{6.720588in}{1.750000in}}{\pgfqpoint{2.279412in}{2.004545in}}%
\pgfusepath{clip}%
\pgfsetbuttcap%
\pgfsetroundjoin%
\pgfsetlinewidth{0.470256pt}%
\definecolor{currentstroke}{rgb}{0.282884,0.135920,0.453427}%
\pgfsetstrokecolor{currentstroke}%
\pgfsetdash{}{0pt}%
\pgfpathmoveto{\pgfqpoint{8.177465in}{3.014700in}}%
\pgfpathlineto{\pgfqpoint{8.127402in}{3.012092in}}%
\pgfusepath{stroke}%
\end{pgfscope}%
\begin{pgfscope}%
\pgfpathrectangle{\pgfqpoint{6.720588in}{1.750000in}}{\pgfqpoint{2.279412in}{2.004545in}}%
\pgfusepath{clip}%
\pgfsetbuttcap%
\pgfsetroundjoin%
\pgfsetlinewidth{0.514603pt}%
\definecolor{currentstroke}{rgb}{0.279574,0.170599,0.479997}%
\pgfsetstrokecolor{currentstroke}%
\pgfsetdash{}{0pt}%
\pgfpathmoveto{\pgfqpoint{8.127402in}{3.012092in}}%
\pgfpathlineto{\pgfqpoint{8.077384in}{3.008911in}}%
\pgfusepath{stroke}%
\end{pgfscope}%
\begin{pgfscope}%
\pgfpathrectangle{\pgfqpoint{6.720588in}{1.750000in}}{\pgfqpoint{2.279412in}{2.004545in}}%
\pgfusepath{clip}%
\pgfsetbuttcap%
\pgfsetroundjoin%
\pgfsetlinewidth{0.542206pt}%
\definecolor{currentstroke}{rgb}{0.276194,0.190074,0.493001}%
\pgfsetstrokecolor{currentstroke}%
\pgfsetdash{}{0pt}%
\pgfpathmoveto{\pgfqpoint{8.077384in}{3.008911in}}%
\pgfpathlineto{\pgfqpoint{8.027436in}{3.004967in}}%
\pgfusepath{stroke}%
\end{pgfscope}%
\begin{pgfscope}%
\pgfpathrectangle{\pgfqpoint{6.720588in}{1.750000in}}{\pgfqpoint{2.279412in}{2.004545in}}%
\pgfusepath{clip}%
\pgfsetbuttcap%
\pgfsetroundjoin%
\pgfsetlinewidth{0.310955pt}%
\definecolor{currentstroke}{rgb}{0.268510,0.009605,0.335427}%
\pgfsetstrokecolor{currentstroke}%
\pgfsetdash{}{0pt}%
\pgfpathmoveto{\pgfqpoint{8.628398in}{3.070945in}}%
\pgfpathlineto{\pgfqpoint{8.578381in}{3.068020in}}%
\pgfusepath{stroke}%
\end{pgfscope}%
\begin{pgfscope}%
\pgfpathrectangle{\pgfqpoint{6.720588in}{1.750000in}}{\pgfqpoint{2.279412in}{2.004545in}}%
\pgfusepath{clip}%
\pgfsetbuttcap%
\pgfsetroundjoin%
\pgfsetlinewidth{0.316483pt}%
\definecolor{currentstroke}{rgb}{0.269944,0.014625,0.341379}%
\pgfsetstrokecolor{currentstroke}%
\pgfsetdash{}{0pt}%
\pgfpathmoveto{\pgfqpoint{8.578381in}{3.068020in}}%
\pgfpathlineto{\pgfqpoint{8.528282in}{3.066906in}}%
\pgfusepath{stroke}%
\end{pgfscope}%
\begin{pgfscope}%
\pgfpathrectangle{\pgfqpoint{6.720588in}{1.750000in}}{\pgfqpoint{2.279412in}{2.004545in}}%
\pgfusepath{clip}%
\pgfsetbuttcap%
\pgfsetroundjoin%
\pgfsetlinewidth{0.318268pt}%
\definecolor{currentstroke}{rgb}{0.269944,0.014625,0.341379}%
\pgfsetstrokecolor{currentstroke}%
\pgfsetdash{}{0pt}%
\pgfpathmoveto{\pgfqpoint{8.528282in}{3.066906in}}%
\pgfpathlineto{\pgfqpoint{8.478189in}{3.065649in}}%
\pgfusepath{stroke}%
\end{pgfscope}%
\begin{pgfscope}%
\pgfpathrectangle{\pgfqpoint{6.720588in}{1.750000in}}{\pgfqpoint{2.279412in}{2.004545in}}%
\pgfusepath{clip}%
\pgfsetbuttcap%
\pgfsetroundjoin%
\pgfsetlinewidth{0.324323pt}%
\definecolor{currentstroke}{rgb}{0.271305,0.019942,0.347269}%
\pgfsetstrokecolor{currentstroke}%
\pgfsetdash{}{0pt}%
\pgfpathmoveto{\pgfqpoint{8.478189in}{3.065649in}}%
\pgfpathlineto{\pgfqpoint{8.428086in}{3.064374in}}%
\pgfusepath{stroke}%
\end{pgfscope}%
\begin{pgfscope}%
\pgfpathrectangle{\pgfqpoint{6.720588in}{1.750000in}}{\pgfqpoint{2.279412in}{2.004545in}}%
\pgfusepath{clip}%
\pgfsetbuttcap%
\pgfsetroundjoin%
\pgfsetlinewidth{0.335629pt}%
\definecolor{currentstroke}{rgb}{0.273809,0.031497,0.358853}%
\pgfsetstrokecolor{currentstroke}%
\pgfsetdash{}{0pt}%
\pgfpathmoveto{\pgfqpoint{8.428086in}{3.064374in}}%
\pgfpathlineto{\pgfqpoint{8.377952in}{3.063562in}}%
\pgfusepath{stroke}%
\end{pgfscope}%
\begin{pgfscope}%
\pgfpathrectangle{\pgfqpoint{6.720588in}{1.750000in}}{\pgfqpoint{2.279412in}{2.004545in}}%
\pgfusepath{clip}%
\pgfsetbuttcap%
\pgfsetroundjoin%
\pgfsetlinewidth{0.357610pt}%
\definecolor{currentstroke}{rgb}{0.277018,0.050344,0.375715}%
\pgfsetstrokecolor{currentstroke}%
\pgfsetdash{}{0pt}%
\pgfpathmoveto{\pgfqpoint{8.377952in}{3.063562in}}%
\pgfpathlineto{\pgfqpoint{8.327823in}{3.062358in}}%
\pgfusepath{stroke}%
\end{pgfscope}%
\begin{pgfscope}%
\pgfpathrectangle{\pgfqpoint{6.720588in}{1.750000in}}{\pgfqpoint{2.279412in}{2.004545in}}%
\pgfusepath{clip}%
\pgfsetbuttcap%
\pgfsetroundjoin%
\pgfsetlinewidth{0.369196pt}%
\definecolor{currentstroke}{rgb}{0.277941,0.056324,0.381191}%
\pgfsetstrokecolor{currentstroke}%
\pgfsetdash{}{0pt}%
\pgfpathmoveto{\pgfqpoint{8.327823in}{3.062358in}}%
\pgfpathlineto{\pgfqpoint{8.277702in}{3.060986in}}%
\pgfusepath{stroke}%
\end{pgfscope}%
\begin{pgfscope}%
\pgfpathrectangle{\pgfqpoint{6.720588in}{1.750000in}}{\pgfqpoint{2.279412in}{2.004545in}}%
\pgfusepath{clip}%
\pgfsetbuttcap%
\pgfsetroundjoin%
\pgfsetlinewidth{0.400435pt}%
\definecolor{currentstroke}{rgb}{0.281446,0.084320,0.407414}%
\pgfsetstrokecolor{currentstroke}%
\pgfsetdash{}{0pt}%
\pgfpathmoveto{\pgfqpoint{8.277702in}{3.060986in}}%
\pgfpathlineto{\pgfqpoint{8.227599in}{3.059085in}}%
\pgfusepath{stroke}%
\end{pgfscope}%
\begin{pgfscope}%
\pgfpathrectangle{\pgfqpoint{6.720588in}{1.750000in}}{\pgfqpoint{2.279412in}{2.004545in}}%
\pgfusepath{clip}%
\pgfsetbuttcap%
\pgfsetroundjoin%
\pgfsetlinewidth{0.417259pt}%
\definecolor{currentstroke}{rgb}{0.282327,0.094955,0.417331}%
\pgfsetstrokecolor{currentstroke}%
\pgfsetdash{}{0pt}%
\pgfpathmoveto{\pgfqpoint{8.227599in}{3.059085in}}%
\pgfpathlineto{\pgfqpoint{8.177510in}{3.056922in}}%
\pgfusepath{stroke}%
\end{pgfscope}%
\begin{pgfscope}%
\pgfpathrectangle{\pgfqpoint{6.720588in}{1.750000in}}{\pgfqpoint{2.279412in}{2.004545in}}%
\pgfusepath{clip}%
\pgfsetbuttcap%
\pgfsetroundjoin%
\pgfsetlinewidth{0.455859pt}%
\definecolor{currentstroke}{rgb}{0.283187,0.125848,0.444960}%
\pgfsetstrokecolor{currentstroke}%
\pgfsetdash{}{0pt}%
\pgfpathmoveto{\pgfqpoint{8.177510in}{3.056922in}}%
\pgfpathlineto{\pgfqpoint{8.127456in}{3.054197in}}%
\pgfusepath{stroke}%
\end{pgfscope}%
\begin{pgfscope}%
\pgfpathrectangle{\pgfqpoint{6.720588in}{1.750000in}}{\pgfqpoint{2.279412in}{2.004545in}}%
\pgfusepath{clip}%
\pgfsetbuttcap%
\pgfsetroundjoin%
\pgfsetlinewidth{0.476041pt}%
\definecolor{currentstroke}{rgb}{0.282623,0.140926,0.457517}%
\pgfsetstrokecolor{currentstroke}%
\pgfsetdash{}{0pt}%
\pgfpathmoveto{\pgfqpoint{8.127456in}{3.054197in}}%
\pgfpathlineto{\pgfqpoint{8.077469in}{3.050679in}}%
\pgfusepath{stroke}%
\end{pgfscope}%
\begin{pgfscope}%
\pgfpathrectangle{\pgfqpoint{6.720588in}{1.750000in}}{\pgfqpoint{2.279412in}{2.004545in}}%
\pgfusepath{clip}%
\pgfsetbuttcap%
\pgfsetroundjoin%
\pgfsetlinewidth{0.505525pt}%
\definecolor{currentstroke}{rgb}{0.280868,0.160771,0.472899}%
\pgfsetstrokecolor{currentstroke}%
\pgfsetdash{}{0pt}%
\pgfpathmoveto{\pgfqpoint{8.077469in}{3.050679in}}%
\pgfpathlineto{\pgfqpoint{8.027599in}{3.046057in}}%
\pgfusepath{stroke}%
\end{pgfscope}%
\begin{pgfscope}%
\pgfpathrectangle{\pgfqpoint{6.720588in}{1.750000in}}{\pgfqpoint{2.279412in}{2.004545in}}%
\pgfusepath{clip}%
\pgfsetbuttcap%
\pgfsetroundjoin%
\pgfsetlinewidth{0.521339pt}%
\definecolor{currentstroke}{rgb}{0.278826,0.175490,0.483397}%
\pgfsetstrokecolor{currentstroke}%
\pgfsetdash{}{0pt}%
\pgfpathmoveto{\pgfqpoint{8.027599in}{3.046057in}}%
\pgfpathlineto{\pgfqpoint{7.977881in}{3.040307in}}%
\pgfusepath{stroke}%
\end{pgfscope}%
\begin{pgfscope}%
\pgfpathrectangle{\pgfqpoint{6.720588in}{1.750000in}}{\pgfqpoint{2.279412in}{2.004545in}}%
\pgfusepath{clip}%
\pgfsetbuttcap%
\pgfsetroundjoin%
\pgfsetlinewidth{0.552460pt}%
\definecolor{currentstroke}{rgb}{0.275191,0.194905,0.496005}%
\pgfsetstrokecolor{currentstroke}%
\pgfsetdash{}{0pt}%
\pgfpathmoveto{\pgfqpoint{7.977881in}{3.040307in}}%
\pgfpathlineto{\pgfqpoint{7.928502in}{3.032704in}}%
\pgfusepath{stroke}%
\end{pgfscope}%
\begin{pgfscope}%
\pgfpathrectangle{\pgfqpoint{6.720588in}{1.750000in}}{\pgfqpoint{2.279412in}{2.004545in}}%
\pgfusepath{clip}%
\pgfsetbuttcap%
\pgfsetroundjoin%
\pgfsetlinewidth{0.499045pt}%
\definecolor{currentstroke}{rgb}{0.281412,0.155834,0.469201}%
\pgfsetstrokecolor{currentstroke}%
\pgfsetdash{}{0pt}%
\pgfpathmoveto{\pgfqpoint{7.928502in}{3.032704in}}%
\pgfpathlineto{\pgfqpoint{7.879706in}{3.022646in}}%
\pgfusepath{stroke}%
\end{pgfscope}%
\begin{pgfscope}%
\pgfpathrectangle{\pgfqpoint{6.720588in}{1.750000in}}{\pgfqpoint{2.279412in}{2.004545in}}%
\pgfusepath{clip}%
\pgfsetbuttcap%
\pgfsetroundjoin%
\pgfsetlinewidth{0.519017pt}%
\definecolor{currentstroke}{rgb}{0.279574,0.170599,0.479997}%
\pgfsetstrokecolor{currentstroke}%
\pgfsetdash{}{0pt}%
\pgfpathmoveto{\pgfqpoint{7.879706in}{3.022646in}}%
\pgfpathlineto{\pgfqpoint{7.831568in}{3.010370in}}%
\pgfusepath{stroke}%
\end{pgfscope}%
\begin{pgfscope}%
\pgfpathrectangle{\pgfqpoint{6.720588in}{1.750000in}}{\pgfqpoint{2.279412in}{2.004545in}}%
\pgfusepath{clip}%
\pgfsetbuttcap%
\pgfsetroundjoin%
\pgfsetlinewidth{0.552223pt}%
\definecolor{currentstroke}{rgb}{0.275191,0.194905,0.496005}%
\pgfsetstrokecolor{currentstroke}%
\pgfsetdash{}{0pt}%
\pgfpathmoveto{\pgfqpoint{7.831568in}{3.010370in}}%
\pgfpathlineto{\pgfqpoint{7.784248in}{2.995887in}}%
\pgfusepath{stroke}%
\end{pgfscope}%
\begin{pgfscope}%
\pgfpathrectangle{\pgfqpoint{6.720588in}{1.750000in}}{\pgfqpoint{2.279412in}{2.004545in}}%
\pgfusepath{clip}%
\pgfsetbuttcap%
\pgfsetroundjoin%
\pgfsetlinewidth{0.590620pt}%
\definecolor{currentstroke}{rgb}{0.267968,0.223549,0.512008}%
\pgfsetstrokecolor{currentstroke}%
\pgfsetdash{}{0pt}%
\pgfpathmoveto{\pgfqpoint{7.784248in}{2.995887in}}%
\pgfpathlineto{\pgfqpoint{7.737993in}{2.979057in}}%
\pgfusepath{stroke}%
\end{pgfscope}%
\begin{pgfscope}%
\pgfpathrectangle{\pgfqpoint{6.720588in}{1.750000in}}{\pgfqpoint{2.279412in}{2.004545in}}%
\pgfusepath{clip}%
\pgfsetbuttcap%
\pgfsetroundjoin%
\pgfsetlinewidth{0.571470pt}%
\definecolor{currentstroke}{rgb}{0.271828,0.209303,0.504434}%
\pgfsetstrokecolor{currentstroke}%
\pgfsetdash{}{0pt}%
\pgfpathmoveto{\pgfqpoint{7.737993in}{2.979057in}}%
\pgfpathlineto{\pgfqpoint{7.693863in}{2.958487in}}%
\pgfusepath{stroke}%
\end{pgfscope}%
\begin{pgfscope}%
\pgfpathrectangle{\pgfqpoint{6.720588in}{1.750000in}}{\pgfqpoint{2.279412in}{2.004545in}}%
\pgfusepath{clip}%
\pgfsetbuttcap%
\pgfsetroundjoin%
\pgfsetlinewidth{0.571503pt}%
\definecolor{currentstroke}{rgb}{0.271828,0.209303,0.504434}%
\pgfsetstrokecolor{currentstroke}%
\pgfsetdash{}{0pt}%
\pgfpathmoveto{\pgfqpoint{7.693863in}{2.958487in}}%
\pgfpathlineto{\pgfqpoint{7.652383in}{2.934006in}}%
\pgfusepath{stroke}%
\end{pgfscope}%
\begin{pgfscope}%
\pgfpathrectangle{\pgfqpoint{6.720588in}{1.750000in}}{\pgfqpoint{2.279412in}{2.004545in}}%
\pgfusepath{clip}%
\pgfsetbuttcap%
\pgfsetroundjoin%
\pgfsetlinewidth{0.592946pt}%
\definecolor{currentstroke}{rgb}{0.267968,0.223549,0.512008}%
\pgfsetstrokecolor{currentstroke}%
\pgfsetdash{}{0pt}%
\pgfpathmoveto{\pgfqpoint{7.652383in}{2.934006in}}%
\pgfpathlineto{\pgfqpoint{7.612978in}{2.906957in}}%
\pgfusepath{stroke}%
\end{pgfscope}%
\begin{pgfscope}%
\pgfpathrectangle{\pgfqpoint{6.720588in}{1.750000in}}{\pgfqpoint{2.279412in}{2.004545in}}%
\pgfusepath{clip}%
\pgfsetbuttcap%
\pgfsetroundjoin%
\pgfsetlinewidth{0.621067pt}%
\definecolor{currentstroke}{rgb}{0.262138,0.242286,0.520837}%
\pgfsetstrokecolor{currentstroke}%
\pgfsetdash{}{0pt}%
\pgfpathmoveto{\pgfqpoint{7.612978in}{2.906957in}}%
\pgfpathlineto{\pgfqpoint{7.573824in}{2.879668in}}%
\pgfusepath{stroke}%
\end{pgfscope}%
\begin{pgfscope}%
\pgfpathrectangle{\pgfqpoint{6.720588in}{1.750000in}}{\pgfqpoint{2.279412in}{2.004545in}}%
\pgfusepath{clip}%
\pgfsetbuttcap%
\pgfsetroundjoin%
\pgfsetlinewidth{0.319043pt}%
\definecolor{currentstroke}{rgb}{0.269944,0.014625,0.341379}%
\pgfsetstrokecolor{currentstroke}%
\pgfsetdash{}{0pt}%
\pgfpathmoveto{\pgfqpoint{8.578381in}{3.293554in}}%
\pgfpathlineto{\pgfqpoint{8.578381in}{3.293554in}}%
\pgfusepath{stroke}%
\end{pgfscope}%
\begin{pgfscope}%
\pgfpathrectangle{\pgfqpoint{6.720588in}{1.750000in}}{\pgfqpoint{2.279412in}{2.004545in}}%
\pgfusepath{clip}%
\pgfsetbuttcap%
\pgfsetroundjoin%
\pgfsetlinewidth{0.319043pt}%
\definecolor{currentstroke}{rgb}{0.269944,0.014625,0.341379}%
\pgfsetstrokecolor{currentstroke}%
\pgfsetdash{}{0pt}%
\pgfpathmoveto{\pgfqpoint{8.578381in}{3.293554in}}%
\pgfpathlineto{\pgfqpoint{8.578381in}{3.293554in}}%
\pgfusepath{stroke}%
\end{pgfscope}%
\begin{pgfscope}%
\pgfpathrectangle{\pgfqpoint{6.720588in}{1.750000in}}{\pgfqpoint{2.279412in}{2.004545in}}%
\pgfusepath{clip}%
\pgfsetbuttcap%
\pgfsetroundjoin%
\pgfsetlinewidth{0.319043pt}%
\definecolor{currentstroke}{rgb}{0.269944,0.014625,0.341379}%
\pgfsetstrokecolor{currentstroke}%
\pgfsetdash{}{0pt}%
\pgfpathmoveto{\pgfqpoint{8.578381in}{3.293554in}}%
\pgfpathlineto{\pgfqpoint{8.553879in}{3.293774in}}%
\pgfusepath{stroke}%
\end{pgfscope}%
\begin{pgfscope}%
\pgfpathrectangle{\pgfqpoint{6.720588in}{1.750000in}}{\pgfqpoint{2.279412in}{2.004545in}}%
\pgfusepath{clip}%
\pgfsetbuttcap%
\pgfsetroundjoin%
\pgfsetlinewidth{0.316627pt}%
\definecolor{currentstroke}{rgb}{0.269944,0.014625,0.341379}%
\pgfsetstrokecolor{currentstroke}%
\pgfsetdash{}{0pt}%
\pgfpathmoveto{\pgfqpoint{8.553879in}{3.293774in}}%
\pgfpathlineto{\pgfqpoint{8.505662in}{3.294519in}}%
\pgfusepath{stroke}%
\end{pgfscope}%
\begin{pgfscope}%
\pgfpathrectangle{\pgfqpoint{6.720588in}{1.750000in}}{\pgfqpoint{2.279412in}{2.004545in}}%
\pgfusepath{clip}%
\pgfsetbuttcap%
\pgfsetroundjoin%
\pgfsetlinewidth{0.313283pt}%
\definecolor{currentstroke}{rgb}{0.268510,0.009605,0.335427}%
\pgfsetstrokecolor{currentstroke}%
\pgfsetdash{}{0pt}%
\pgfpathmoveto{\pgfqpoint{8.505662in}{3.294519in}}%
\pgfpathlineto{\pgfqpoint{8.455632in}{3.295239in}}%
\pgfusepath{stroke}%
\end{pgfscope}%
\begin{pgfscope}%
\pgfpathrectangle{\pgfqpoint{6.720588in}{1.750000in}}{\pgfqpoint{2.279412in}{2.004545in}}%
\pgfusepath{clip}%
\pgfsetbuttcap%
\pgfsetroundjoin%
\pgfsetlinewidth{0.319466pt}%
\definecolor{currentstroke}{rgb}{0.269944,0.014625,0.341379}%
\pgfsetstrokecolor{currentstroke}%
\pgfsetdash{}{0pt}%
\pgfpathmoveto{\pgfqpoint{8.455632in}{3.295239in}}%
\pgfpathlineto{\pgfqpoint{8.405522in}{3.294577in}}%
\pgfusepath{stroke}%
\end{pgfscope}%
\begin{pgfscope}%
\pgfpathrectangle{\pgfqpoint{6.720588in}{1.750000in}}{\pgfqpoint{2.279412in}{2.004545in}}%
\pgfusepath{clip}%
\pgfsetbuttcap%
\pgfsetroundjoin%
\pgfsetlinewidth{0.326990pt}%
\definecolor{currentstroke}{rgb}{0.271305,0.019942,0.347269}%
\pgfsetstrokecolor{currentstroke}%
\pgfsetdash{}{0pt}%
\pgfpathmoveto{\pgfqpoint{8.405522in}{3.294577in}}%
\pgfpathlineto{\pgfqpoint{8.355421in}{3.293842in}}%
\pgfusepath{stroke}%
\end{pgfscope}%
\begin{pgfscope}%
\pgfpathrectangle{\pgfqpoint{6.720588in}{1.750000in}}{\pgfqpoint{2.279412in}{2.004545in}}%
\pgfusepath{clip}%
\pgfsetbuttcap%
\pgfsetroundjoin%
\pgfsetlinewidth{0.330452pt}%
\definecolor{currentstroke}{rgb}{0.272594,0.025563,0.353093}%
\pgfsetstrokecolor{currentstroke}%
\pgfsetdash{}{0pt}%
\pgfpathmoveto{\pgfqpoint{8.355421in}{3.293842in}}%
\pgfpathlineto{\pgfqpoint{8.305300in}{3.293214in}}%
\pgfusepath{stroke}%
\end{pgfscope}%
\begin{pgfscope}%
\pgfpathrectangle{\pgfqpoint{6.720588in}{1.750000in}}{\pgfqpoint{2.279412in}{2.004545in}}%
\pgfusepath{clip}%
\pgfsetbuttcap%
\pgfsetroundjoin%
\pgfsetlinewidth{0.340097pt}%
\definecolor{currentstroke}{rgb}{0.273809,0.031497,0.358853}%
\pgfsetstrokecolor{currentstroke}%
\pgfsetdash{}{0pt}%
\pgfpathmoveto{\pgfqpoint{8.305300in}{3.293214in}}%
\pgfpathlineto{\pgfqpoint{8.255176in}{3.292437in}}%
\pgfusepath{stroke}%
\end{pgfscope}%
\begin{pgfscope}%
\pgfpathrectangle{\pgfqpoint{6.720588in}{1.750000in}}{\pgfqpoint{2.279412in}{2.004545in}}%
\pgfusepath{clip}%
\pgfsetbuttcap%
\pgfsetroundjoin%
\pgfsetlinewidth{0.337415pt}%
\definecolor{currentstroke}{rgb}{0.273809,0.031497,0.358853}%
\pgfsetstrokecolor{currentstroke}%
\pgfsetdash{}{0pt}%
\pgfpathmoveto{\pgfqpoint{8.255176in}{3.292437in}}%
\pgfpathlineto{\pgfqpoint{8.205090in}{3.290756in}}%
\pgfusepath{stroke}%
\end{pgfscope}%
\begin{pgfscope}%
\pgfpathrectangle{\pgfqpoint{6.720588in}{1.750000in}}{\pgfqpoint{2.279412in}{2.004545in}}%
\pgfusepath{clip}%
\pgfsetbuttcap%
\pgfsetroundjoin%
\pgfsetlinewidth{0.353681pt}%
\definecolor{currentstroke}{rgb}{0.276022,0.044167,0.370164}%
\pgfsetstrokecolor{currentstroke}%
\pgfsetdash{}{0pt}%
\pgfpathmoveto{\pgfqpoint{8.205090in}{3.290756in}}%
\pgfpathlineto{\pgfqpoint{8.155098in}{3.287285in}}%
\pgfusepath{stroke}%
\end{pgfscope}%
\begin{pgfscope}%
\pgfpathrectangle{\pgfqpoint{6.720588in}{1.750000in}}{\pgfqpoint{2.279412in}{2.004545in}}%
\pgfusepath{clip}%
\pgfsetbuttcap%
\pgfsetroundjoin%
\pgfsetlinewidth{0.369912pt}%
\definecolor{currentstroke}{rgb}{0.278791,0.062145,0.386592}%
\pgfsetstrokecolor{currentstroke}%
\pgfsetdash{}{0pt}%
\pgfpathmoveto{\pgfqpoint{8.155098in}{3.287285in}}%
\pgfpathlineto{\pgfqpoint{8.105187in}{3.283035in}}%
\pgfusepath{stroke}%
\end{pgfscope}%
\begin{pgfscope}%
\pgfpathrectangle{\pgfqpoint{6.720588in}{1.750000in}}{\pgfqpoint{2.279412in}{2.004545in}}%
\pgfusepath{clip}%
\pgfsetbuttcap%
\pgfsetroundjoin%
\pgfsetlinewidth{0.322849pt}%
\definecolor{currentstroke}{rgb}{0.271305,0.019942,0.347269}%
\pgfsetstrokecolor{currentstroke}%
\pgfsetdash{}{0pt}%
\pgfpathmoveto{\pgfqpoint{8.219337in}{3.383768in}}%
\pgfpathlineto{\pgfqpoint{8.169363in}{3.381139in}}%
\pgfusepath{stroke}%
\end{pgfscope}%
\begin{pgfscope}%
\pgfpathrectangle{\pgfqpoint{6.720588in}{1.750000in}}{\pgfqpoint{2.279412in}{2.004545in}}%
\pgfusepath{clip}%
\pgfsetbuttcap%
\pgfsetroundjoin%
\pgfsetlinewidth{0.343150pt}%
\definecolor{currentstroke}{rgb}{0.274952,0.037752,0.364543}%
\pgfsetstrokecolor{currentstroke}%
\pgfsetdash{}{0pt}%
\pgfpathmoveto{\pgfqpoint{8.169363in}{3.381139in}}%
\pgfpathlineto{\pgfqpoint{8.119603in}{3.375816in}}%
\pgfusepath{stroke}%
\end{pgfscope}%
\begin{pgfscope}%
\pgfpathrectangle{\pgfqpoint{6.720588in}{1.750000in}}{\pgfqpoint{2.279412in}{2.004545in}}%
\pgfusepath{clip}%
\pgfsetbuttcap%
\pgfsetroundjoin%
\pgfsetlinewidth{0.348245pt}%
\definecolor{currentstroke}{rgb}{0.274952,0.037752,0.364543}%
\pgfsetstrokecolor{currentstroke}%
\pgfsetdash{}{0pt}%
\pgfpathmoveto{\pgfqpoint{8.119603in}{3.375816in}}%
\pgfpathlineto{\pgfqpoint{8.069909in}{3.370104in}}%
\pgfusepath{stroke}%
\end{pgfscope}%
\begin{pgfscope}%
\pgfpathrectangle{\pgfqpoint{6.720588in}{1.750000in}}{\pgfqpoint{2.279412in}{2.004545in}}%
\pgfusepath{clip}%
\pgfsetbuttcap%
\pgfsetroundjoin%
\pgfsetlinewidth{0.331492pt}%
\definecolor{currentstroke}{rgb}{0.272594,0.025563,0.353093}%
\pgfsetstrokecolor{currentstroke}%
\pgfsetdash{}{0pt}%
\pgfpathmoveto{\pgfqpoint{8.069909in}{3.370104in}}%
\pgfpathlineto{\pgfqpoint{8.020228in}{3.364249in}}%
\pgfusepath{stroke}%
\end{pgfscope}%
\begin{pgfscope}%
\pgfpathrectangle{\pgfqpoint{6.720588in}{1.750000in}}{\pgfqpoint{2.279412in}{2.004545in}}%
\pgfusepath{clip}%
\pgfsetbuttcap%
\pgfsetroundjoin%
\pgfsetlinewidth{0.350555pt}%
\definecolor{currentstroke}{rgb}{0.276022,0.044167,0.370164}%
\pgfsetstrokecolor{currentstroke}%
\pgfsetdash{}{0pt}%
\pgfpathmoveto{\pgfqpoint{8.020228in}{3.364249in}}%
\pgfpathlineto{\pgfqpoint{7.970734in}{3.357545in}}%
\pgfusepath{stroke}%
\end{pgfscope}%
\begin{pgfscope}%
\pgfpathrectangle{\pgfqpoint{6.720588in}{1.750000in}}{\pgfqpoint{2.279412in}{2.004545in}}%
\pgfusepath{clip}%
\pgfsetbuttcap%
\pgfsetroundjoin%
\pgfsetlinewidth{0.325921pt}%
\definecolor{currentstroke}{rgb}{0.271305,0.019942,0.347269}%
\pgfsetstrokecolor{currentstroke}%
\pgfsetdash{}{0pt}%
\pgfpathmoveto{\pgfqpoint{8.527089in}{2.391418in}}%
\pgfpathlineto{\pgfqpoint{8.476957in}{2.390646in}}%
\pgfusepath{stroke}%
\end{pgfscope}%
\begin{pgfscope}%
\pgfpathrectangle{\pgfqpoint{6.720588in}{1.750000in}}{\pgfqpoint{2.279412in}{2.004545in}}%
\pgfusepath{clip}%
\pgfsetbuttcap%
\pgfsetroundjoin%
\pgfsetlinewidth{0.325775pt}%
\definecolor{currentstroke}{rgb}{0.271305,0.019942,0.347269}%
\pgfsetstrokecolor{currentstroke}%
\pgfsetdash{}{0pt}%
\pgfpathmoveto{\pgfqpoint{8.476957in}{2.390646in}}%
\pgfpathlineto{\pgfqpoint{8.426825in}{2.390582in}}%
\pgfusepath{stroke}%
\end{pgfscope}%
\begin{pgfscope}%
\pgfpathrectangle{\pgfqpoint{6.720588in}{1.750000in}}{\pgfqpoint{2.279412in}{2.004545in}}%
\pgfusepath{clip}%
\pgfsetbuttcap%
\pgfsetroundjoin%
\pgfsetlinewidth{0.336873pt}%
\definecolor{currentstroke}{rgb}{0.273809,0.031497,0.358853}%
\pgfsetstrokecolor{currentstroke}%
\pgfsetdash{}{0pt}%
\pgfpathmoveto{\pgfqpoint{8.426825in}{2.390582in}}%
\pgfpathlineto{\pgfqpoint{8.376698in}{2.391749in}}%
\pgfusepath{stroke}%
\end{pgfscope}%
\begin{pgfscope}%
\pgfpathrectangle{\pgfqpoint{6.720588in}{1.750000in}}{\pgfqpoint{2.279412in}{2.004545in}}%
\pgfusepath{clip}%
\pgfsetbuttcap%
\pgfsetroundjoin%
\pgfsetlinewidth{0.344878pt}%
\definecolor{currentstroke}{rgb}{0.274952,0.037752,0.364543}%
\pgfsetstrokecolor{currentstroke}%
\pgfsetdash{}{0pt}%
\pgfpathmoveto{\pgfqpoint{8.376698in}{2.391749in}}%
\pgfpathlineto{\pgfqpoint{8.326573in}{2.393104in}}%
\pgfusepath{stroke}%
\end{pgfscope}%
\begin{pgfscope}%
\pgfpathrectangle{\pgfqpoint{6.720588in}{1.750000in}}{\pgfqpoint{2.279412in}{2.004545in}}%
\pgfusepath{clip}%
\pgfsetbuttcap%
\pgfsetroundjoin%
\pgfsetlinewidth{0.363655pt}%
\definecolor{currentstroke}{rgb}{0.277941,0.056324,0.381191}%
\pgfsetstrokecolor{currentstroke}%
\pgfsetdash{}{0pt}%
\pgfpathmoveto{\pgfqpoint{8.326573in}{2.393104in}}%
\pgfpathlineto{\pgfqpoint{8.276469in}{2.394931in}}%
\pgfusepath{stroke}%
\end{pgfscope}%
\begin{pgfscope}%
\pgfpathrectangle{\pgfqpoint{6.720588in}{1.750000in}}{\pgfqpoint{2.279412in}{2.004545in}}%
\pgfusepath{clip}%
\pgfsetbuttcap%
\pgfsetroundjoin%
\pgfsetlinewidth{0.394982pt}%
\definecolor{currentstroke}{rgb}{0.280894,0.078907,0.402329}%
\pgfsetstrokecolor{currentstroke}%
\pgfsetdash{}{0pt}%
\pgfpathmoveto{\pgfqpoint{8.276469in}{2.394931in}}%
\pgfpathlineto{\pgfqpoint{8.226393in}{2.397354in}}%
\pgfusepath{stroke}%
\end{pgfscope}%
\begin{pgfscope}%
\pgfpathrectangle{\pgfqpoint{6.720588in}{1.750000in}}{\pgfqpoint{2.279412in}{2.004545in}}%
\pgfusepath{clip}%
\pgfsetbuttcap%
\pgfsetroundjoin%
\pgfsetlinewidth{0.413690pt}%
\definecolor{currentstroke}{rgb}{0.282327,0.094955,0.417331}%
\pgfsetstrokecolor{currentstroke}%
\pgfsetdash{}{0pt}%
\pgfpathmoveto{\pgfqpoint{8.226393in}{2.397354in}}%
\pgfpathlineto{\pgfqpoint{8.176313in}{2.399707in}}%
\pgfusepath{stroke}%
\end{pgfscope}%
\begin{pgfscope}%
\pgfpathrectangle{\pgfqpoint{6.720588in}{1.750000in}}{\pgfqpoint{2.279412in}{2.004545in}}%
\pgfusepath{clip}%
\pgfsetbuttcap%
\pgfsetroundjoin%
\pgfsetlinewidth{0.467673pt}%
\definecolor{currentstroke}{rgb}{0.282884,0.135920,0.453427}%
\pgfsetstrokecolor{currentstroke}%
\pgfsetdash{}{0pt}%
\pgfpathmoveto{\pgfqpoint{8.176313in}{2.399707in}}%
\pgfpathlineto{\pgfqpoint{8.126288in}{2.402755in}}%
\pgfusepath{stroke}%
\end{pgfscope}%
\begin{pgfscope}%
\pgfpathrectangle{\pgfqpoint{6.720588in}{1.750000in}}{\pgfqpoint{2.279412in}{2.004545in}}%
\pgfusepath{clip}%
\pgfsetbuttcap%
\pgfsetroundjoin%
\pgfsetlinewidth{0.473650pt}%
\definecolor{currentstroke}{rgb}{0.282623,0.140926,0.457517}%
\pgfsetstrokecolor{currentstroke}%
\pgfsetdash{}{0pt}%
\pgfpathmoveto{\pgfqpoint{8.126288in}{2.402755in}}%
\pgfpathlineto{\pgfqpoint{8.076343in}{2.406739in}}%
\pgfusepath{stroke}%
\end{pgfscope}%
\begin{pgfscope}%
\pgfpathrectangle{\pgfqpoint{6.720588in}{1.750000in}}{\pgfqpoint{2.279412in}{2.004545in}}%
\pgfusepath{clip}%
\pgfsetbuttcap%
\pgfsetroundjoin%
\pgfsetlinewidth{0.324475pt}%
\definecolor{currentstroke}{rgb}{0.271305,0.019942,0.347269}%
\pgfsetstrokecolor{currentstroke}%
\pgfsetdash{}{0pt}%
\pgfpathmoveto{\pgfqpoint{8.527089in}{2.662059in}}%
\pgfpathlineto{\pgfqpoint{8.476965in}{2.662153in}}%
\pgfusepath{stroke}%
\end{pgfscope}%
\begin{pgfscope}%
\pgfpathrectangle{\pgfqpoint{6.720588in}{1.750000in}}{\pgfqpoint{2.279412in}{2.004545in}}%
\pgfusepath{clip}%
\pgfsetbuttcap%
\pgfsetroundjoin%
\pgfsetlinewidth{0.335442pt}%
\definecolor{currentstroke}{rgb}{0.273809,0.031497,0.358853}%
\pgfsetstrokecolor{currentstroke}%
\pgfsetdash{}{0pt}%
\pgfpathmoveto{\pgfqpoint{8.476965in}{2.662153in}}%
\pgfpathlineto{\pgfqpoint{8.426830in}{2.662286in}}%
\pgfusepath{stroke}%
\end{pgfscope}%
\begin{pgfscope}%
\pgfpathrectangle{\pgfqpoint{6.720588in}{1.750000in}}{\pgfqpoint{2.279412in}{2.004545in}}%
\pgfusepath{clip}%
\pgfsetbuttcap%
\pgfsetroundjoin%
\pgfsetlinewidth{0.341138pt}%
\definecolor{currentstroke}{rgb}{0.273809,0.031497,0.358853}%
\pgfsetstrokecolor{currentstroke}%
\pgfsetdash{}{0pt}%
\pgfpathmoveto{\pgfqpoint{8.426830in}{2.662286in}}%
\pgfpathlineto{\pgfqpoint{8.376679in}{2.662493in}}%
\pgfusepath{stroke}%
\end{pgfscope}%
\begin{pgfscope}%
\pgfpathrectangle{\pgfqpoint{6.720588in}{1.750000in}}{\pgfqpoint{2.279412in}{2.004545in}}%
\pgfusepath{clip}%
\pgfsetbuttcap%
\pgfsetroundjoin%
\pgfsetlinewidth{0.375301pt}%
\definecolor{currentstroke}{rgb}{0.278791,0.062145,0.386592}%
\pgfsetstrokecolor{currentstroke}%
\pgfsetdash{}{0pt}%
\pgfpathmoveto{\pgfqpoint{8.376679in}{2.662493in}}%
\pgfpathlineto{\pgfqpoint{8.326529in}{2.662425in}}%
\pgfusepath{stroke}%
\end{pgfscope}%
\begin{pgfscope}%
\pgfpathrectangle{\pgfqpoint{6.720588in}{1.750000in}}{\pgfqpoint{2.279412in}{2.004545in}}%
\pgfusepath{clip}%
\pgfsetbuttcap%
\pgfsetroundjoin%
\pgfsetlinewidth{0.400809pt}%
\definecolor{currentstroke}{rgb}{0.281446,0.084320,0.407414}%
\pgfsetstrokecolor{currentstroke}%
\pgfsetdash{}{0pt}%
\pgfpathmoveto{\pgfqpoint{8.326529in}{2.662425in}}%
\pgfpathlineto{\pgfqpoint{8.276379in}{2.662327in}}%
\pgfusepath{stroke}%
\end{pgfscope}%
\begin{pgfscope}%
\pgfpathrectangle{\pgfqpoint{6.720588in}{1.750000in}}{\pgfqpoint{2.279412in}{2.004545in}}%
\pgfusepath{clip}%
\pgfsetbuttcap%
\pgfsetroundjoin%
\pgfsetlinewidth{0.454331pt}%
\definecolor{currentstroke}{rgb}{0.283187,0.125848,0.444960}%
\pgfsetstrokecolor{currentstroke}%
\pgfsetdash{}{0pt}%
\pgfpathmoveto{\pgfqpoint{8.276379in}{2.662327in}}%
\pgfpathlineto{\pgfqpoint{8.226229in}{2.662453in}}%
\pgfusepath{stroke}%
\end{pgfscope}%
\begin{pgfscope}%
\pgfpathrectangle{\pgfqpoint{6.720588in}{1.750000in}}{\pgfqpoint{2.279412in}{2.004545in}}%
\pgfusepath{clip}%
\pgfsetbuttcap%
\pgfsetroundjoin%
\pgfsetlinewidth{0.509482pt}%
\definecolor{currentstroke}{rgb}{0.280255,0.165693,0.476498}%
\pgfsetstrokecolor{currentstroke}%
\pgfsetdash{}{0pt}%
\pgfpathmoveto{\pgfqpoint{8.226229in}{2.662453in}}%
\pgfpathlineto{\pgfqpoint{8.176079in}{2.662766in}}%
\pgfusepath{stroke}%
\end{pgfscope}%
\begin{pgfscope}%
\pgfpathrectangle{\pgfqpoint{6.720588in}{1.750000in}}{\pgfqpoint{2.279412in}{2.004545in}}%
\pgfusepath{clip}%
\pgfsetbuttcap%
\pgfsetroundjoin%
\pgfsetlinewidth{0.618993pt}%
\definecolor{currentstroke}{rgb}{0.262138,0.242286,0.520837}%
\pgfsetstrokecolor{currentstroke}%
\pgfsetdash{}{0pt}%
\pgfpathmoveto{\pgfqpoint{8.176079in}{2.662766in}}%
\pgfpathlineto{\pgfqpoint{8.125929in}{2.663081in}}%
\pgfusepath{stroke}%
\end{pgfscope}%
\begin{pgfscope}%
\pgfpathrectangle{\pgfqpoint{6.720588in}{1.750000in}}{\pgfqpoint{2.279412in}{2.004545in}}%
\pgfusepath{clip}%
\pgfsetbuttcap%
\pgfsetroundjoin%
\pgfsetlinewidth{0.704088pt}%
\definecolor{currentstroke}{rgb}{0.241237,0.296485,0.539709}%
\pgfsetstrokecolor{currentstroke}%
\pgfsetdash{}{0pt}%
\pgfpathmoveto{\pgfqpoint{8.125929in}{2.663081in}}%
\pgfpathlineto{\pgfqpoint{8.075780in}{2.663500in}}%
\pgfusepath{stroke}%
\end{pgfscope}%
\begin{pgfscope}%
\pgfpathrectangle{\pgfqpoint{6.720588in}{1.750000in}}{\pgfqpoint{2.279412in}{2.004545in}}%
\pgfusepath{clip}%
\pgfsetbuttcap%
\pgfsetroundjoin%
\pgfsetlinewidth{0.773698pt}%
\definecolor{currentstroke}{rgb}{0.221989,0.339161,0.548752}%
\pgfsetstrokecolor{currentstroke}%
\pgfsetdash{}{0pt}%
\pgfpathmoveto{\pgfqpoint{8.075780in}{2.663500in}}%
\pgfpathlineto{\pgfqpoint{8.025635in}{2.664149in}}%
\pgfusepath{stroke}%
\end{pgfscope}%
\begin{pgfscope}%
\pgfpathrectangle{\pgfqpoint{6.720588in}{1.750000in}}{\pgfqpoint{2.279412in}{2.004545in}}%
\pgfusepath{clip}%
\pgfsetbuttcap%
\pgfsetroundjoin%
\pgfsetlinewidth{0.853058pt}%
\definecolor{currentstroke}{rgb}{0.199430,0.387607,0.554642}%
\pgfsetstrokecolor{currentstroke}%
\pgfsetdash{}{0pt}%
\pgfpathmoveto{\pgfqpoint{8.025635in}{2.664149in}}%
\pgfpathlineto{\pgfqpoint{7.975493in}{2.664992in}}%
\pgfusepath{stroke}%
\end{pgfscope}%
\begin{pgfscope}%
\pgfpathrectangle{\pgfqpoint{6.720588in}{1.750000in}}{\pgfqpoint{2.279412in}{2.004545in}}%
\pgfusepath{clip}%
\pgfsetbuttcap%
\pgfsetroundjoin%
\pgfsetlinewidth{0.837725pt}%
\definecolor{currentstroke}{rgb}{0.203063,0.379716,0.553925}%
\pgfsetstrokecolor{currentstroke}%
\pgfsetdash{}{0pt}%
\pgfpathmoveto{\pgfqpoint{7.975493in}{2.664992in}}%
\pgfpathlineto{\pgfqpoint{7.925359in}{2.666064in}}%
\pgfusepath{stroke}%
\end{pgfscope}%
\begin{pgfscope}%
\pgfpathrectangle{\pgfqpoint{6.720588in}{1.750000in}}{\pgfqpoint{2.279412in}{2.004545in}}%
\pgfusepath{clip}%
\pgfsetbuttcap%
\pgfsetroundjoin%
\pgfsetlinewidth{0.816669pt}%
\definecolor{currentstroke}{rgb}{0.208623,0.367752,0.552675}%
\pgfsetstrokecolor{currentstroke}%
\pgfsetdash{}{0pt}%
\pgfpathmoveto{\pgfqpoint{7.925359in}{2.666064in}}%
\pgfpathlineto{\pgfqpoint{7.875240in}{2.667514in}}%
\pgfusepath{stroke}%
\end{pgfscope}%
\begin{pgfscope}%
\pgfpathrectangle{\pgfqpoint{6.720588in}{1.750000in}}{\pgfqpoint{2.279412in}{2.004545in}}%
\pgfusepath{clip}%
\pgfsetbuttcap%
\pgfsetroundjoin%
\pgfsetlinewidth{0.794694pt}%
\definecolor{currentstroke}{rgb}{0.216210,0.351535,0.550627}%
\pgfsetstrokecolor{currentstroke}%
\pgfsetdash{}{0pt}%
\pgfpathmoveto{\pgfqpoint{7.875240in}{2.667514in}}%
\pgfpathlineto{\pgfqpoint{7.825144in}{2.669484in}}%
\pgfusepath{stroke}%
\end{pgfscope}%
\begin{pgfscope}%
\pgfpathrectangle{\pgfqpoint{6.720588in}{1.750000in}}{\pgfqpoint{2.279412in}{2.004545in}}%
\pgfusepath{clip}%
\pgfsetbuttcap%
\pgfsetroundjoin%
\pgfsetlinewidth{0.745747pt}%
\definecolor{currentstroke}{rgb}{0.229739,0.322361,0.545706}%
\pgfsetstrokecolor{currentstroke}%
\pgfsetdash{}{0pt}%
\pgfpathmoveto{\pgfqpoint{7.825144in}{2.669484in}}%
\pgfpathlineto{\pgfqpoint{7.775072in}{2.671898in}}%
\pgfusepath{stroke}%
\end{pgfscope}%
\begin{pgfscope}%
\pgfpathrectangle{\pgfqpoint{6.720588in}{1.750000in}}{\pgfqpoint{2.279412in}{2.004545in}}%
\pgfusepath{clip}%
\pgfsetbuttcap%
\pgfsetroundjoin%
\pgfsetlinewidth{0.712348pt}%
\definecolor{currentstroke}{rgb}{0.239346,0.300855,0.540844}%
\pgfsetstrokecolor{currentstroke}%
\pgfsetdash{}{0pt}%
\pgfpathmoveto{\pgfqpoint{7.775072in}{2.671898in}}%
\pgfpathlineto{\pgfqpoint{7.725088in}{2.675305in}}%
\pgfusepath{stroke}%
\end{pgfscope}%
\begin{pgfscope}%
\pgfpathrectangle{\pgfqpoint{6.720588in}{1.750000in}}{\pgfqpoint{2.279412in}{2.004545in}}%
\pgfusepath{clip}%
\pgfsetbuttcap%
\pgfsetroundjoin%
\pgfsetlinewidth{0.611992pt}%
\definecolor{currentstroke}{rgb}{0.263663,0.237631,0.518762}%
\pgfsetstrokecolor{currentstroke}%
\pgfsetdash{}{0pt}%
\pgfpathmoveto{\pgfqpoint{7.725088in}{2.675305in}}%
\pgfpathlineto{\pgfqpoint{7.675161in}{2.679257in}}%
\pgfusepath{stroke}%
\end{pgfscope}%
\begin{pgfscope}%
\pgfpathrectangle{\pgfqpoint{6.720588in}{1.750000in}}{\pgfqpoint{2.279412in}{2.004545in}}%
\pgfusepath{clip}%
\pgfsetbuttcap%
\pgfsetroundjoin%
\pgfsetlinewidth{0.318710pt}%
\definecolor{currentstroke}{rgb}{0.269944,0.014625,0.341379}%
\pgfsetstrokecolor{currentstroke}%
\pgfsetdash{}{0pt}%
\pgfpathmoveto{\pgfqpoint{8.527089in}{2.797380in}}%
\pgfpathlineto{\pgfqpoint{8.477045in}{2.797720in}}%
\pgfusepath{stroke}%
\end{pgfscope}%
\begin{pgfscope}%
\pgfpathrectangle{\pgfqpoint{6.720588in}{1.750000in}}{\pgfqpoint{2.279412in}{2.004545in}}%
\pgfusepath{clip}%
\pgfsetbuttcap%
\pgfsetroundjoin%
\pgfsetlinewidth{0.328028pt}%
\definecolor{currentstroke}{rgb}{0.271305,0.019942,0.347269}%
\pgfsetstrokecolor{currentstroke}%
\pgfsetdash{}{0pt}%
\pgfpathmoveto{\pgfqpoint{8.477045in}{2.797720in}}%
\pgfpathlineto{\pgfqpoint{8.426900in}{2.797296in}}%
\pgfusepath{stroke}%
\end{pgfscope}%
\begin{pgfscope}%
\pgfpathrectangle{\pgfqpoint{6.720588in}{1.750000in}}{\pgfqpoint{2.279412in}{2.004545in}}%
\pgfusepath{clip}%
\pgfsetbuttcap%
\pgfsetroundjoin%
\pgfsetlinewidth{0.344465pt}%
\definecolor{currentstroke}{rgb}{0.274952,0.037752,0.364543}%
\pgfsetstrokecolor{currentstroke}%
\pgfsetdash{}{0pt}%
\pgfpathmoveto{\pgfqpoint{8.426900in}{2.797296in}}%
\pgfpathlineto{\pgfqpoint{8.376752in}{2.796877in}}%
\pgfusepath{stroke}%
\end{pgfscope}%
\begin{pgfscope}%
\pgfpathrectangle{\pgfqpoint{6.720588in}{1.750000in}}{\pgfqpoint{2.279412in}{2.004545in}}%
\pgfusepath{clip}%
\pgfsetbuttcap%
\pgfsetroundjoin%
\pgfsetlinewidth{0.368336pt}%
\definecolor{currentstroke}{rgb}{0.277941,0.056324,0.381191}%
\pgfsetstrokecolor{currentstroke}%
\pgfsetdash{}{0pt}%
\pgfpathmoveto{\pgfqpoint{8.376752in}{2.796877in}}%
\pgfpathlineto{\pgfqpoint{8.326602in}{2.796697in}}%
\pgfusepath{stroke}%
\end{pgfscope}%
\begin{pgfscope}%
\pgfpathrectangle{\pgfqpoint{6.720588in}{1.750000in}}{\pgfqpoint{2.279412in}{2.004545in}}%
\pgfusepath{clip}%
\pgfsetbuttcap%
\pgfsetroundjoin%
\pgfsetlinewidth{0.406127pt}%
\definecolor{currentstroke}{rgb}{0.281924,0.089666,0.412415}%
\pgfsetstrokecolor{currentstroke}%
\pgfsetdash{}{0pt}%
\pgfpathmoveto{\pgfqpoint{8.326602in}{2.796697in}}%
\pgfpathlineto{\pgfqpoint{8.276450in}{2.796600in}}%
\pgfusepath{stroke}%
\end{pgfscope}%
\begin{pgfscope}%
\pgfpathrectangle{\pgfqpoint{6.720588in}{1.750000in}}{\pgfqpoint{2.279412in}{2.004545in}}%
\pgfusepath{clip}%
\pgfsetbuttcap%
\pgfsetroundjoin%
\pgfsetlinewidth{0.447116pt}%
\definecolor{currentstroke}{rgb}{0.283229,0.120777,0.440584}%
\pgfsetstrokecolor{currentstroke}%
\pgfsetdash{}{0pt}%
\pgfpathmoveto{\pgfqpoint{8.276450in}{2.796600in}}%
\pgfpathlineto{\pgfqpoint{8.226300in}{2.796385in}}%
\pgfusepath{stroke}%
\end{pgfscope}%
\begin{pgfscope}%
\pgfpathrectangle{\pgfqpoint{6.720588in}{1.750000in}}{\pgfqpoint{2.279412in}{2.004545in}}%
\pgfusepath{clip}%
\pgfsetbuttcap%
\pgfsetroundjoin%
\pgfsetlinewidth{0.497365pt}%
\definecolor{currentstroke}{rgb}{0.281412,0.155834,0.469201}%
\pgfsetstrokecolor{currentstroke}%
\pgfsetdash{}{0pt}%
\pgfpathmoveto{\pgfqpoint{8.226300in}{2.796385in}}%
\pgfpathlineto{\pgfqpoint{8.176150in}{2.796037in}}%
\pgfusepath{stroke}%
\end{pgfscope}%
\begin{pgfscope}%
\pgfpathrectangle{\pgfqpoint{6.720588in}{1.750000in}}{\pgfqpoint{2.279412in}{2.004545in}}%
\pgfusepath{clip}%
\pgfsetbuttcap%
\pgfsetroundjoin%
\pgfsetlinewidth{0.586111pt}%
\definecolor{currentstroke}{rgb}{0.269308,0.218818,0.509577}%
\pgfsetstrokecolor{currentstroke}%
\pgfsetdash{}{0pt}%
\pgfpathmoveto{\pgfqpoint{8.176150in}{2.796037in}}%
\pgfpathlineto{\pgfqpoint{8.125999in}{2.795686in}}%
\pgfusepath{stroke}%
\end{pgfscope}%
\begin{pgfscope}%
\pgfpathrectangle{\pgfqpoint{6.720588in}{1.750000in}}{\pgfqpoint{2.279412in}{2.004545in}}%
\pgfusepath{clip}%
\pgfsetbuttcap%
\pgfsetroundjoin%
\pgfsetlinewidth{0.704206pt}%
\definecolor{currentstroke}{rgb}{0.241237,0.296485,0.539709}%
\pgfsetstrokecolor{currentstroke}%
\pgfsetdash{}{0pt}%
\pgfpathmoveto{\pgfqpoint{8.125999in}{2.795686in}}%
\pgfpathlineto{\pgfqpoint{8.075851in}{2.795203in}}%
\pgfusepath{stroke}%
\end{pgfscope}%
\begin{pgfscope}%
\pgfpathrectangle{\pgfqpoint{6.720588in}{1.750000in}}{\pgfqpoint{2.279412in}{2.004545in}}%
\pgfusepath{clip}%
\pgfsetbuttcap%
\pgfsetroundjoin%
\pgfsetlinewidth{0.778683pt}%
\definecolor{currentstroke}{rgb}{0.220057,0.343307,0.549413}%
\pgfsetstrokecolor{currentstroke}%
\pgfsetdash{}{0pt}%
\pgfpathmoveto{\pgfqpoint{8.075851in}{2.795203in}}%
\pgfpathlineto{\pgfqpoint{8.025707in}{2.794468in}}%
\pgfusepath{stroke}%
\end{pgfscope}%
\begin{pgfscope}%
\pgfpathrectangle{\pgfqpoint{6.720588in}{1.750000in}}{\pgfqpoint{2.279412in}{2.004545in}}%
\pgfusepath{clip}%
\pgfsetbuttcap%
\pgfsetroundjoin%
\pgfsetlinewidth{0.782071pt}%
\definecolor{currentstroke}{rgb}{0.218130,0.347432,0.550038}%
\pgfsetstrokecolor{currentstroke}%
\pgfsetdash{}{0pt}%
\pgfpathmoveto{\pgfqpoint{8.025707in}{2.794468in}}%
\pgfpathlineto{\pgfqpoint{7.975566in}{2.793535in}}%
\pgfusepath{stroke}%
\end{pgfscope}%
\begin{pgfscope}%
\pgfpathrectangle{\pgfqpoint{6.720588in}{1.750000in}}{\pgfqpoint{2.279412in}{2.004545in}}%
\pgfusepath{clip}%
\pgfsetbuttcap%
\pgfsetroundjoin%
\pgfsetlinewidth{0.840337pt}%
\definecolor{currentstroke}{rgb}{0.203063,0.379716,0.553925}%
\pgfsetstrokecolor{currentstroke}%
\pgfsetdash{}{0pt}%
\pgfpathmoveto{\pgfqpoint{7.975566in}{2.793535in}}%
\pgfpathlineto{\pgfqpoint{7.925433in}{2.792353in}}%
\pgfusepath{stroke}%
\end{pgfscope}%
\begin{pgfscope}%
\pgfpathrectangle{\pgfqpoint{6.720588in}{1.750000in}}{\pgfqpoint{2.279412in}{2.004545in}}%
\pgfusepath{clip}%
\pgfsetbuttcap%
\pgfsetroundjoin%
\pgfsetlinewidth{0.821684pt}%
\definecolor{currentstroke}{rgb}{0.208623,0.367752,0.552675}%
\pgfsetstrokecolor{currentstroke}%
\pgfsetdash{}{0pt}%
\pgfpathmoveto{\pgfqpoint{7.925433in}{2.792353in}}%
\pgfpathlineto{\pgfqpoint{7.875315in}{2.790743in}}%
\pgfusepath{stroke}%
\end{pgfscope}%
\begin{pgfscope}%
\pgfpathrectangle{\pgfqpoint{6.720588in}{1.750000in}}{\pgfqpoint{2.279412in}{2.004545in}}%
\pgfusepath{clip}%
\pgfsetbuttcap%
\pgfsetroundjoin%
\pgfsetlinewidth{0.821935pt}%
\definecolor{currentstroke}{rgb}{0.208623,0.367752,0.552675}%
\pgfsetstrokecolor{currentstroke}%
\pgfsetdash{}{0pt}%
\pgfpathmoveto{\pgfqpoint{7.875315in}{2.790743in}}%
\pgfpathlineto{\pgfqpoint{7.825229in}{2.788532in}}%
\pgfusepath{stroke}%
\end{pgfscope}%
\begin{pgfscope}%
\pgfpathrectangle{\pgfqpoint{6.720588in}{1.750000in}}{\pgfqpoint{2.279412in}{2.004545in}}%
\pgfusepath{clip}%
\pgfsetbuttcap%
\pgfsetroundjoin%
\pgfsetlinewidth{0.825559pt}%
\definecolor{currentstroke}{rgb}{0.206756,0.371758,0.553117}%
\pgfsetstrokecolor{currentstroke}%
\pgfsetdash{}{0pt}%
\pgfpathmoveto{\pgfqpoint{7.825229in}{2.788532in}}%
\pgfpathlineto{\pgfqpoint{7.775173in}{2.785852in}}%
\pgfusepath{stroke}%
\end{pgfscope}%
\begin{pgfscope}%
\pgfpathrectangle{\pgfqpoint{6.720588in}{1.750000in}}{\pgfqpoint{2.279412in}{2.004545in}}%
\pgfusepath{clip}%
\pgfsetbuttcap%
\pgfsetroundjoin%
\pgfsetlinewidth{0.705534pt}%
\definecolor{currentstroke}{rgb}{0.241237,0.296485,0.539709}%
\pgfsetstrokecolor{currentstroke}%
\pgfsetdash{}{0pt}%
\pgfpathmoveto{\pgfqpoint{7.775173in}{2.785852in}}%
\pgfpathlineto{\pgfqpoint{7.725155in}{2.782712in}}%
\pgfusepath{stroke}%
\end{pgfscope}%
\begin{pgfscope}%
\pgfpathrectangle{\pgfqpoint{6.720588in}{1.750000in}}{\pgfqpoint{2.279412in}{2.004545in}}%
\pgfusepath{clip}%
\pgfsetbuttcap%
\pgfsetroundjoin%
\pgfsetlinewidth{0.679029pt}%
\definecolor{currentstroke}{rgb}{0.246811,0.283237,0.535941}%
\pgfsetstrokecolor{currentstroke}%
\pgfsetdash{}{0pt}%
\pgfpathmoveto{\pgfqpoint{7.725155in}{2.782712in}}%
\pgfpathlineto{\pgfqpoint{7.675233in}{2.778623in}}%
\pgfusepath{stroke}%
\end{pgfscope}%
\begin{pgfscope}%
\pgfpathrectangle{\pgfqpoint{6.720588in}{1.750000in}}{\pgfqpoint{2.279412in}{2.004545in}}%
\pgfusepath{clip}%
\pgfsetbuttcap%
\pgfsetroundjoin%
\pgfsetlinewidth{0.321447pt}%
\definecolor{currentstroke}{rgb}{0.269944,0.014625,0.341379}%
\pgfsetstrokecolor{currentstroke}%
\pgfsetdash{}{0pt}%
\pgfpathmoveto{\pgfqpoint{8.527089in}{2.842486in}}%
\pgfpathlineto{\pgfqpoint{8.476960in}{2.842157in}}%
\pgfusepath{stroke}%
\end{pgfscope}%
\begin{pgfscope}%
\pgfpathrectangle{\pgfqpoint{6.720588in}{1.750000in}}{\pgfqpoint{2.279412in}{2.004545in}}%
\pgfusepath{clip}%
\pgfsetbuttcap%
\pgfsetroundjoin%
\pgfsetlinewidth{0.336164pt}%
\definecolor{currentstroke}{rgb}{0.273809,0.031497,0.358853}%
\pgfsetstrokecolor{currentstroke}%
\pgfsetdash{}{0pt}%
\pgfpathmoveto{\pgfqpoint{8.476960in}{2.842157in}}%
\pgfpathlineto{\pgfqpoint{8.426825in}{2.841118in}}%
\pgfusepath{stroke}%
\end{pgfscope}%
\begin{pgfscope}%
\pgfpathrectangle{\pgfqpoint{6.720588in}{1.750000in}}{\pgfqpoint{2.279412in}{2.004545in}}%
\pgfusepath{clip}%
\pgfsetbuttcap%
\pgfsetroundjoin%
\pgfsetlinewidth{0.338366pt}%
\definecolor{currentstroke}{rgb}{0.273809,0.031497,0.358853}%
\pgfsetstrokecolor{currentstroke}%
\pgfsetdash{}{0pt}%
\pgfpathmoveto{\pgfqpoint{8.426825in}{2.841118in}}%
\pgfpathlineto{\pgfqpoint{8.376682in}{2.840682in}}%
\pgfusepath{stroke}%
\end{pgfscope}%
\begin{pgfscope}%
\pgfpathrectangle{\pgfqpoint{6.720588in}{1.750000in}}{\pgfqpoint{2.279412in}{2.004545in}}%
\pgfusepath{clip}%
\pgfsetbuttcap%
\pgfsetroundjoin%
\pgfsetlinewidth{0.364067pt}%
\definecolor{currentstroke}{rgb}{0.277941,0.056324,0.381191}%
\pgfsetstrokecolor{currentstroke}%
\pgfsetdash{}{0pt}%
\pgfpathmoveto{\pgfqpoint{8.376682in}{2.840682in}}%
\pgfpathlineto{\pgfqpoint{8.326533in}{2.840581in}}%
\pgfusepath{stroke}%
\end{pgfscope}%
\begin{pgfscope}%
\pgfpathrectangle{\pgfqpoint{6.720588in}{1.750000in}}{\pgfqpoint{2.279412in}{2.004545in}}%
\pgfusepath{clip}%
\pgfsetbuttcap%
\pgfsetroundjoin%
\pgfsetlinewidth{0.406574pt}%
\definecolor{currentstroke}{rgb}{0.281924,0.089666,0.412415}%
\pgfsetstrokecolor{currentstroke}%
\pgfsetdash{}{0pt}%
\pgfpathmoveto{\pgfqpoint{8.326533in}{2.840581in}}%
\pgfpathlineto{\pgfqpoint{8.276385in}{2.840134in}}%
\pgfusepath{stroke}%
\end{pgfscope}%
\begin{pgfscope}%
\pgfpathrectangle{\pgfqpoint{6.720588in}{1.750000in}}{\pgfqpoint{2.279412in}{2.004545in}}%
\pgfusepath{clip}%
\pgfsetbuttcap%
\pgfsetroundjoin%
\pgfsetlinewidth{0.445556pt}%
\definecolor{currentstroke}{rgb}{0.283229,0.120777,0.440584}%
\pgfsetstrokecolor{currentstroke}%
\pgfsetdash{}{0pt}%
\pgfpathmoveto{\pgfqpoint{8.276385in}{2.840134in}}%
\pgfpathlineto{\pgfqpoint{8.226236in}{2.839656in}}%
\pgfusepath{stroke}%
\end{pgfscope}%
\begin{pgfscope}%
\pgfpathrectangle{\pgfqpoint{6.720588in}{1.750000in}}{\pgfqpoint{2.279412in}{2.004545in}}%
\pgfusepath{clip}%
\pgfsetbuttcap%
\pgfsetroundjoin%
\pgfsetlinewidth{0.505432pt}%
\definecolor{currentstroke}{rgb}{0.280868,0.160771,0.472899}%
\pgfsetstrokecolor{currentstroke}%
\pgfsetdash{}{0pt}%
\pgfpathmoveto{\pgfqpoint{8.226236in}{2.839656in}}%
\pgfpathlineto{\pgfqpoint{8.176086in}{2.839254in}}%
\pgfusepath{stroke}%
\end{pgfscope}%
\begin{pgfscope}%
\pgfpathrectangle{\pgfqpoint{6.720588in}{1.750000in}}{\pgfqpoint{2.279412in}{2.004545in}}%
\pgfusepath{clip}%
\pgfsetbuttcap%
\pgfsetroundjoin%
\pgfsetlinewidth{0.590069pt}%
\definecolor{currentstroke}{rgb}{0.267968,0.223549,0.512008}%
\pgfsetstrokecolor{currentstroke}%
\pgfsetdash{}{0pt}%
\pgfpathmoveto{\pgfqpoint{8.176086in}{2.839254in}}%
\pgfpathlineto{\pgfqpoint{8.125942in}{2.838592in}}%
\pgfusepath{stroke}%
\end{pgfscope}%
\begin{pgfscope}%
\pgfpathrectangle{\pgfqpoint{6.720588in}{1.750000in}}{\pgfqpoint{2.279412in}{2.004545in}}%
\pgfusepath{clip}%
\pgfsetbuttcap%
\pgfsetroundjoin%
\pgfsetlinewidth{0.672618pt}%
\definecolor{currentstroke}{rgb}{0.248629,0.278775,0.534556}%
\pgfsetstrokecolor{currentstroke}%
\pgfsetdash{}{0pt}%
\pgfpathmoveto{\pgfqpoint{8.125942in}{2.838592in}}%
\pgfpathlineto{\pgfqpoint{8.075803in}{2.837568in}}%
\pgfusepath{stroke}%
\end{pgfscope}%
\begin{pgfscope}%
\pgfpathrectangle{\pgfqpoint{6.720588in}{1.750000in}}{\pgfqpoint{2.279412in}{2.004545in}}%
\pgfusepath{clip}%
\pgfsetbuttcap%
\pgfsetroundjoin%
\pgfsetlinewidth{0.712869pt}%
\definecolor{currentstroke}{rgb}{0.237441,0.305202,0.541921}%
\pgfsetstrokecolor{currentstroke}%
\pgfsetdash{}{0pt}%
\pgfpathmoveto{\pgfqpoint{8.075803in}{2.837568in}}%
\pgfpathlineto{\pgfqpoint{8.025671in}{2.836357in}}%
\pgfusepath{stroke}%
\end{pgfscope}%
\begin{pgfscope}%
\pgfpathrectangle{\pgfqpoint{6.720588in}{1.750000in}}{\pgfqpoint{2.279412in}{2.004545in}}%
\pgfusepath{clip}%
\pgfsetbuttcap%
\pgfsetroundjoin%
\pgfsetlinewidth{0.771943pt}%
\definecolor{currentstroke}{rgb}{0.221989,0.339161,0.548752}%
\pgfsetstrokecolor{currentstroke}%
\pgfsetdash{}{0pt}%
\pgfpathmoveto{\pgfqpoint{8.025671in}{2.836357in}}%
\pgfpathlineto{\pgfqpoint{7.975551in}{2.834817in}}%
\pgfusepath{stroke}%
\end{pgfscope}%
\begin{pgfscope}%
\pgfpathrectangle{\pgfqpoint{6.720588in}{1.750000in}}{\pgfqpoint{2.279412in}{2.004545in}}%
\pgfusepath{clip}%
\pgfsetbuttcap%
\pgfsetroundjoin%
\pgfsetlinewidth{0.806332pt}%
\definecolor{currentstroke}{rgb}{0.212395,0.359683,0.551710}%
\pgfsetstrokecolor{currentstroke}%
\pgfsetdash{}{0pt}%
\pgfpathmoveto{\pgfqpoint{7.975551in}{2.834817in}}%
\pgfpathlineto{\pgfqpoint{7.925453in}{2.832809in}}%
\pgfusepath{stroke}%
\end{pgfscope}%
\begin{pgfscope}%
\pgfpathrectangle{\pgfqpoint{6.720588in}{1.750000in}}{\pgfqpoint{2.279412in}{2.004545in}}%
\pgfusepath{clip}%
\pgfsetbuttcap%
\pgfsetroundjoin%
\pgfsetlinewidth{0.802518pt}%
\definecolor{currentstroke}{rgb}{0.212395,0.359683,0.551710}%
\pgfsetstrokecolor{currentstroke}%
\pgfsetdash{}{0pt}%
\pgfpathmoveto{\pgfqpoint{7.925453in}{2.832809in}}%
\pgfpathlineto{\pgfqpoint{7.875390in}{2.830212in}}%
\pgfusepath{stroke}%
\end{pgfscope}%
\begin{pgfscope}%
\pgfpathrectangle{\pgfqpoint{6.720588in}{1.750000in}}{\pgfqpoint{2.279412in}{2.004545in}}%
\pgfusepath{clip}%
\pgfsetbuttcap%
\pgfsetroundjoin%
\pgfsetlinewidth{0.720044pt}%
\definecolor{currentstroke}{rgb}{0.235526,0.309527,0.542944}%
\pgfsetstrokecolor{currentstroke}%
\pgfsetdash{}{0pt}%
\pgfpathmoveto{\pgfqpoint{7.875390in}{2.830212in}}%
\pgfpathlineto{\pgfqpoint{7.825377in}{2.826968in}}%
\pgfusepath{stroke}%
\end{pgfscope}%
\begin{pgfscope}%
\pgfpathrectangle{\pgfqpoint{6.720588in}{1.750000in}}{\pgfqpoint{2.279412in}{2.004545in}}%
\pgfusepath{clip}%
\pgfsetbuttcap%
\pgfsetroundjoin%
\pgfsetlinewidth{0.767860pt}%
\definecolor{currentstroke}{rgb}{0.221989,0.339161,0.548752}%
\pgfsetstrokecolor{currentstroke}%
\pgfsetdash{}{0pt}%
\pgfpathmoveto{\pgfqpoint{7.825377in}{2.826968in}}%
\pgfpathlineto{\pgfqpoint{7.775427in}{2.823065in}}%
\pgfusepath{stroke}%
\end{pgfscope}%
\begin{pgfscope}%
\pgfpathrectangle{\pgfqpoint{6.720588in}{1.750000in}}{\pgfqpoint{2.279412in}{2.004545in}}%
\pgfusepath{clip}%
\pgfsetbuttcap%
\pgfsetroundjoin%
\pgfsetlinewidth{0.311866pt}%
\definecolor{currentstroke}{rgb}{0.268510,0.009605,0.335427}%
\pgfsetstrokecolor{currentstroke}%
\pgfsetdash{}{0pt}%
\pgfpathmoveto{\pgfqpoint{8.527089in}{3.158234in}}%
\pgfpathlineto{\pgfqpoint{8.477399in}{3.156007in}}%
\pgfusepath{stroke}%
\end{pgfscope}%
\begin{pgfscope}%
\pgfpathrectangle{\pgfqpoint{6.720588in}{1.750000in}}{\pgfqpoint{2.279412in}{2.004545in}}%
\pgfusepath{clip}%
\pgfsetbuttcap%
\pgfsetroundjoin%
\pgfsetlinewidth{0.317930pt}%
\definecolor{currentstroke}{rgb}{0.269944,0.014625,0.341379}%
\pgfsetstrokecolor{currentstroke}%
\pgfsetdash{}{0pt}%
\pgfpathmoveto{\pgfqpoint{8.477399in}{3.156007in}}%
\pgfpathlineto{\pgfqpoint{8.427300in}{3.155065in}}%
\pgfusepath{stroke}%
\end{pgfscope}%
\begin{pgfscope}%
\pgfpathrectangle{\pgfqpoint{6.720588in}{1.750000in}}{\pgfqpoint{2.279412in}{2.004545in}}%
\pgfusepath{clip}%
\pgfsetbuttcap%
\pgfsetroundjoin%
\pgfsetlinewidth{0.330865pt}%
\definecolor{currentstroke}{rgb}{0.272594,0.025563,0.353093}%
\pgfsetstrokecolor{currentstroke}%
\pgfsetdash{}{0pt}%
\pgfpathmoveto{\pgfqpoint{8.427300in}{3.155065in}}%
\pgfpathlineto{\pgfqpoint{8.377161in}{3.154251in}}%
\pgfusepath{stroke}%
\end{pgfscope}%
\begin{pgfscope}%
\pgfpathrectangle{\pgfqpoint{6.720588in}{1.750000in}}{\pgfqpoint{2.279412in}{2.004545in}}%
\pgfusepath{clip}%
\pgfsetbuttcap%
\pgfsetroundjoin%
\pgfsetlinewidth{0.357139pt}%
\definecolor{currentstroke}{rgb}{0.277018,0.050344,0.375715}%
\pgfsetstrokecolor{currentstroke}%
\pgfsetdash{}{0pt}%
\pgfpathmoveto{\pgfqpoint{8.377161in}{3.154251in}}%
\pgfpathlineto{\pgfqpoint{8.327069in}{3.152319in}}%
\pgfusepath{stroke}%
\end{pgfscope}%
\begin{pgfscope}%
\pgfpathrectangle{\pgfqpoint{6.720588in}{1.750000in}}{\pgfqpoint{2.279412in}{2.004545in}}%
\pgfusepath{clip}%
\pgfsetbuttcap%
\pgfsetroundjoin%
\pgfsetlinewidth{0.368325pt}%
\definecolor{currentstroke}{rgb}{0.277941,0.056324,0.381191}%
\pgfsetstrokecolor{currentstroke}%
\pgfsetdash{}{0pt}%
\pgfpathmoveto{\pgfqpoint{8.327069in}{3.152319in}}%
\pgfpathlineto{\pgfqpoint{8.277011in}{3.149689in}}%
\pgfusepath{stroke}%
\end{pgfscope}%
\begin{pgfscope}%
\pgfpathrectangle{\pgfqpoint{6.720588in}{1.750000in}}{\pgfqpoint{2.279412in}{2.004545in}}%
\pgfusepath{clip}%
\pgfsetbuttcap%
\pgfsetroundjoin%
\pgfsetlinewidth{0.371214pt}%
\definecolor{currentstroke}{rgb}{0.278791,0.062145,0.386592}%
\pgfsetstrokecolor{currentstroke}%
\pgfsetdash{}{0pt}%
\pgfpathmoveto{\pgfqpoint{8.277011in}{3.149689in}}%
\pgfpathlineto{\pgfqpoint{8.226956in}{3.146991in}}%
\pgfusepath{stroke}%
\end{pgfscope}%
\begin{pgfscope}%
\pgfpathrectangle{\pgfqpoint{6.720588in}{1.750000in}}{\pgfqpoint{2.279412in}{2.004545in}}%
\pgfusepath{clip}%
\pgfsetbuttcap%
\pgfsetroundjoin%
\pgfsetlinewidth{0.390003pt}%
\definecolor{currentstroke}{rgb}{0.280267,0.073417,0.397163}%
\pgfsetstrokecolor{currentstroke}%
\pgfsetdash{}{0pt}%
\pgfpathmoveto{\pgfqpoint{8.226956in}{3.146991in}}%
\pgfpathlineto{\pgfqpoint{8.176917in}{3.144022in}}%
\pgfusepath{stroke}%
\end{pgfscope}%
\begin{pgfscope}%
\pgfpathrectangle{\pgfqpoint{6.720588in}{1.750000in}}{\pgfqpoint{2.279412in}{2.004545in}}%
\pgfusepath{clip}%
\pgfsetbuttcap%
\pgfsetroundjoin%
\pgfsetlinewidth{0.403495pt}%
\definecolor{currentstroke}{rgb}{0.281446,0.084320,0.407414}%
\pgfsetstrokecolor{currentstroke}%
\pgfsetdash{}{0pt}%
\pgfpathmoveto{\pgfqpoint{8.176917in}{3.144022in}}%
\pgfpathlineto{\pgfqpoint{8.126944in}{3.140397in}}%
\pgfusepath{stroke}%
\end{pgfscope}%
\begin{pgfscope}%
\pgfpathrectangle{\pgfqpoint{6.720588in}{1.750000in}}{\pgfqpoint{2.279412in}{2.004545in}}%
\pgfusepath{clip}%
\pgfsetbuttcap%
\pgfsetroundjoin%
\pgfsetlinewidth{0.423719pt}%
\definecolor{currentstroke}{rgb}{0.282656,0.100196,0.422160}%
\pgfsetstrokecolor{currentstroke}%
\pgfsetdash{}{0pt}%
\pgfpathmoveto{\pgfqpoint{8.126944in}{3.140397in}}%
\pgfpathlineto{\pgfqpoint{8.077103in}{3.135596in}}%
\pgfusepath{stroke}%
\end{pgfscope}%
\begin{pgfscope}%
\pgfpathrectangle{\pgfqpoint{6.720588in}{1.750000in}}{\pgfqpoint{2.279412in}{2.004545in}}%
\pgfusepath{clip}%
\pgfsetbuttcap%
\pgfsetroundjoin%
\pgfsetlinewidth{0.435522pt}%
\definecolor{currentstroke}{rgb}{0.283091,0.110553,0.431554}%
\pgfsetstrokecolor{currentstroke}%
\pgfsetdash{}{0pt}%
\pgfpathmoveto{\pgfqpoint{8.077103in}{3.135596in}}%
\pgfpathlineto{\pgfqpoint{8.027516in}{3.129098in}}%
\pgfusepath{stroke}%
\end{pgfscope}%
\begin{pgfscope}%
\pgfpathrectangle{\pgfqpoint{6.720588in}{1.750000in}}{\pgfqpoint{2.279412in}{2.004545in}}%
\pgfusepath{clip}%
\pgfsetbuttcap%
\pgfsetroundjoin%
\pgfsetlinewidth{0.445779pt}%
\definecolor{currentstroke}{rgb}{0.283229,0.120777,0.440584}%
\pgfsetstrokecolor{currentstroke}%
\pgfsetdash{}{0pt}%
\pgfpathmoveto{\pgfqpoint{8.027516in}{3.129098in}}%
\pgfpathlineto{\pgfqpoint{7.978157in}{3.121308in}}%
\pgfusepath{stroke}%
\end{pgfscope}%
\begin{pgfscope}%
\pgfpathrectangle{\pgfqpoint{6.720588in}{1.750000in}}{\pgfqpoint{2.279412in}{2.004545in}}%
\pgfusepath{clip}%
\pgfsetbuttcap%
\pgfsetroundjoin%
\pgfsetlinewidth{0.449660pt}%
\definecolor{currentstroke}{rgb}{0.283229,0.120777,0.440584}%
\pgfsetstrokecolor{currentstroke}%
\pgfsetdash{}{0pt}%
\pgfpathmoveto{\pgfqpoint{7.978157in}{3.121308in}}%
\pgfpathlineto{\pgfqpoint{7.929017in}{3.112535in}}%
\pgfusepath{stroke}%
\end{pgfscope}%
\begin{pgfscope}%
\pgfpathrectangle{\pgfqpoint{6.720588in}{1.750000in}}{\pgfqpoint{2.279412in}{2.004545in}}%
\pgfusepath{clip}%
\pgfsetbuttcap%
\pgfsetroundjoin%
\pgfsetlinewidth{0.454368pt}%
\definecolor{currentstroke}{rgb}{0.283187,0.125848,0.444960}%
\pgfsetstrokecolor{currentstroke}%
\pgfsetdash{}{0pt}%
\pgfpathmoveto{\pgfqpoint{7.929017in}{3.112535in}}%
\pgfpathlineto{\pgfqpoint{7.880508in}{3.101484in}}%
\pgfusepath{stroke}%
\end{pgfscope}%
\begin{pgfscope}%
\pgfpathrectangle{\pgfqpoint{6.720588in}{1.750000in}}{\pgfqpoint{2.279412in}{2.004545in}}%
\pgfusepath{clip}%
\pgfsetbuttcap%
\pgfsetroundjoin%
\pgfsetlinewidth{0.481662pt}%
\definecolor{currentstroke}{rgb}{0.282290,0.145912,0.461510}%
\pgfsetstrokecolor{currentstroke}%
\pgfsetdash{}{0pt}%
\pgfpathmoveto{\pgfqpoint{7.880508in}{3.101484in}}%
\pgfpathlineto{\pgfqpoint{7.832912in}{3.087784in}}%
\pgfusepath{stroke}%
\end{pgfscope}%
\begin{pgfscope}%
\pgfpathrectangle{\pgfqpoint{6.720588in}{1.750000in}}{\pgfqpoint{2.279412in}{2.004545in}}%
\pgfusepath{clip}%
\pgfsetbuttcap%
\pgfsetroundjoin%
\pgfsetlinewidth{0.480220pt}%
\definecolor{currentstroke}{rgb}{0.282290,0.145912,0.461510}%
\pgfsetstrokecolor{currentstroke}%
\pgfsetdash{}{0pt}%
\pgfpathmoveto{\pgfqpoint{7.832912in}{3.087784in}}%
\pgfpathlineto{\pgfqpoint{7.787938in}{3.068903in}}%
\pgfusepath{stroke}%
\end{pgfscope}%
\begin{pgfscope}%
\pgfpathrectangle{\pgfqpoint{6.720588in}{1.750000in}}{\pgfqpoint{2.279412in}{2.004545in}}%
\pgfusepath{clip}%
\pgfsetbuttcap%
\pgfsetroundjoin%
\pgfsetlinewidth{0.434898pt}%
\definecolor{currentstroke}{rgb}{0.283091,0.110553,0.431554}%
\pgfsetstrokecolor{currentstroke}%
\pgfsetdash{}{0pt}%
\pgfpathmoveto{\pgfqpoint{7.787938in}{3.068903in}}%
\pgfpathlineto{\pgfqpoint{7.746884in}{3.045114in}}%
\pgfusepath{stroke}%
\end{pgfscope}%
\begin{pgfscope}%
\pgfpathrectangle{\pgfqpoint{6.720588in}{1.750000in}}{\pgfqpoint{2.279412in}{2.004545in}}%
\pgfusepath{clip}%
\pgfsetbuttcap%
\pgfsetroundjoin%
\pgfsetlinewidth{0.501558pt}%
\definecolor{currentstroke}{rgb}{0.280868,0.160771,0.472899}%
\pgfsetstrokecolor{currentstroke}%
\pgfsetdash{}{0pt}%
\pgfpathmoveto{\pgfqpoint{7.746884in}{3.045114in}}%
\pgfpathlineto{\pgfqpoint{7.709053in}{3.016577in}}%
\pgfusepath{stroke}%
\end{pgfscope}%
\begin{pgfscope}%
\pgfpathrectangle{\pgfqpoint{6.720588in}{1.750000in}}{\pgfqpoint{2.279412in}{2.004545in}}%
\pgfusepath{clip}%
\pgfsetbuttcap%
\pgfsetroundjoin%
\pgfsetlinewidth{0.532767pt}%
\definecolor{currentstroke}{rgb}{0.278012,0.180367,0.486697}%
\pgfsetstrokecolor{currentstroke}%
\pgfsetdash{}{0pt}%
\pgfpathmoveto{\pgfqpoint{7.709053in}{3.016577in}}%
\pgfpathlineto{\pgfqpoint{7.672799in}{2.986383in}}%
\pgfusepath{stroke}%
\end{pgfscope}%
\begin{pgfscope}%
\pgfpathrectangle{\pgfqpoint{6.720588in}{1.750000in}}{\pgfqpoint{2.279412in}{2.004545in}}%
\pgfusepath{clip}%
\pgfsetbuttcap%
\pgfsetroundjoin%
\pgfsetlinewidth{0.558586pt}%
\definecolor{currentstroke}{rgb}{0.274128,0.199721,0.498911}%
\pgfsetstrokecolor{currentstroke}%
\pgfsetdash{}{0pt}%
\pgfpathmoveto{\pgfqpoint{7.672799in}{2.986383in}}%
\pgfpathlineto{\pgfqpoint{7.636854in}{2.955803in}}%
\pgfusepath{stroke}%
\end{pgfscope}%
\begin{pgfscope}%
\pgfpathrectangle{\pgfqpoint{6.720588in}{1.750000in}}{\pgfqpoint{2.279412in}{2.004545in}}%
\pgfusepath{clip}%
\pgfsetbuttcap%
\pgfsetroundjoin%
\pgfsetlinewidth{0.600432pt}%
\definecolor{currentstroke}{rgb}{0.266580,0.228262,0.514349}%
\pgfsetstrokecolor{currentstroke}%
\pgfsetdash{}{0pt}%
\pgfpathmoveto{\pgfqpoint{7.636854in}{2.955803in}}%
\pgfpathlineto{\pgfqpoint{7.601403in}{2.924697in}}%
\pgfusepath{stroke}%
\end{pgfscope}%
\begin{pgfscope}%
\pgfpathrectangle{\pgfqpoint{6.720588in}{1.750000in}}{\pgfqpoint{2.279412in}{2.004545in}}%
\pgfusepath{clip}%
\pgfsetbuttcap%
\pgfsetroundjoin%
\pgfsetlinewidth{0.326548pt}%
\definecolor{currentstroke}{rgb}{0.271305,0.019942,0.347269}%
\pgfsetstrokecolor{currentstroke}%
\pgfsetdash{}{0pt}%
\pgfpathmoveto{\pgfqpoint{8.517200in}{2.194357in}}%
\pgfpathlineto{\pgfqpoint{8.467060in}{2.194781in}}%
\pgfusepath{stroke}%
\end{pgfscope}%
\begin{pgfscope}%
\pgfpathrectangle{\pgfqpoint{6.720588in}{1.750000in}}{\pgfqpoint{2.279412in}{2.004545in}}%
\pgfusepath{clip}%
\pgfsetbuttcap%
\pgfsetroundjoin%
\pgfsetlinewidth{0.324652pt}%
\definecolor{currentstroke}{rgb}{0.271305,0.019942,0.347269}%
\pgfsetstrokecolor{currentstroke}%
\pgfsetdash{}{0pt}%
\pgfpathmoveto{\pgfqpoint{8.467060in}{2.194781in}}%
\pgfpathlineto{\pgfqpoint{8.416915in}{2.194723in}}%
\pgfusepath{stroke}%
\end{pgfscope}%
\begin{pgfscope}%
\pgfpathrectangle{\pgfqpoint{6.720588in}{1.750000in}}{\pgfqpoint{2.279412in}{2.004545in}}%
\pgfusepath{clip}%
\pgfsetbuttcap%
\pgfsetroundjoin%
\pgfsetlinewidth{0.320957pt}%
\definecolor{currentstroke}{rgb}{0.269944,0.014625,0.341379}%
\pgfsetstrokecolor{currentstroke}%
\pgfsetdash{}{0pt}%
\pgfpathmoveto{\pgfqpoint{8.416915in}{2.194723in}}%
\pgfpathlineto{\pgfqpoint{8.366797in}{2.195851in}}%
\pgfusepath{stroke}%
\end{pgfscope}%
\begin{pgfscope}%
\pgfpathrectangle{\pgfqpoint{6.720588in}{1.750000in}}{\pgfqpoint{2.279412in}{2.004545in}}%
\pgfusepath{clip}%
\pgfsetbuttcap%
\pgfsetroundjoin%
\pgfsetlinewidth{0.333149pt}%
\definecolor{currentstroke}{rgb}{0.272594,0.025563,0.353093}%
\pgfsetstrokecolor{currentstroke}%
\pgfsetdash{}{0pt}%
\pgfpathmoveto{\pgfqpoint{8.366797in}{2.195851in}}%
\pgfpathlineto{\pgfqpoint{8.316720in}{2.198208in}}%
\pgfusepath{stroke}%
\end{pgfscope}%
\begin{pgfscope}%
\pgfpathrectangle{\pgfqpoint{6.720588in}{1.750000in}}{\pgfqpoint{2.279412in}{2.004545in}}%
\pgfusepath{clip}%
\pgfsetbuttcap%
\pgfsetroundjoin%
\pgfsetlinewidth{0.344053pt}%
\definecolor{currentstroke}{rgb}{0.274952,0.037752,0.364543}%
\pgfsetstrokecolor{currentstroke}%
\pgfsetdash{}{0pt}%
\pgfpathmoveto{\pgfqpoint{8.316720in}{2.198208in}}%
\pgfpathlineto{\pgfqpoint{8.266635in}{2.200436in}}%
\pgfusepath{stroke}%
\end{pgfscope}%
\begin{pgfscope}%
\pgfpathrectangle{\pgfqpoint{6.720588in}{1.750000in}}{\pgfqpoint{2.279412in}{2.004545in}}%
\pgfusepath{clip}%
\pgfsetbuttcap%
\pgfsetroundjoin%
\pgfsetlinewidth{0.345284pt}%
\definecolor{currentstroke}{rgb}{0.274952,0.037752,0.364543}%
\pgfsetstrokecolor{currentstroke}%
\pgfsetdash{}{0pt}%
\pgfpathmoveto{\pgfqpoint{8.266635in}{2.200436in}}%
\pgfpathlineto{\pgfqpoint{8.216566in}{2.202832in}}%
\pgfusepath{stroke}%
\end{pgfscope}%
\begin{pgfscope}%
\pgfpathrectangle{\pgfqpoint{6.720588in}{1.750000in}}{\pgfqpoint{2.279412in}{2.004545in}}%
\pgfusepath{clip}%
\pgfsetbuttcap%
\pgfsetroundjoin%
\pgfsetlinewidth{0.357384pt}%
\definecolor{currentstroke}{rgb}{0.277018,0.050344,0.375715}%
\pgfsetstrokecolor{currentstroke}%
\pgfsetdash{}{0pt}%
\pgfpathmoveto{\pgfqpoint{8.216566in}{2.202832in}}%
\pgfpathlineto{\pgfqpoint{8.166555in}{2.206059in}}%
\pgfusepath{stroke}%
\end{pgfscope}%
\begin{pgfscope}%
\pgfpathrectangle{\pgfqpoint{6.720588in}{1.750000in}}{\pgfqpoint{2.279412in}{2.004545in}}%
\pgfusepath{clip}%
\pgfsetbuttcap%
\pgfsetroundjoin%
\pgfsetlinewidth{0.376231pt}%
\definecolor{currentstroke}{rgb}{0.278791,0.062145,0.386592}%
\pgfsetstrokecolor{currentstroke}%
\pgfsetdash{}{0pt}%
\pgfpathmoveto{\pgfqpoint{8.166555in}{2.206059in}}%
\pgfpathlineto{\pgfqpoint{8.116754in}{2.210991in}}%
\pgfusepath{stroke}%
\end{pgfscope}%
\begin{pgfscope}%
\pgfpathrectangle{\pgfqpoint{6.720588in}{1.750000in}}{\pgfqpoint{2.279412in}{2.004545in}}%
\pgfusepath{clip}%
\pgfsetbuttcap%
\pgfsetroundjoin%
\pgfsetlinewidth{0.328570pt}%
\definecolor{currentstroke}{rgb}{0.271305,0.019942,0.347269}%
\pgfsetstrokecolor{currentstroke}%
\pgfsetdash{}{0pt}%
\pgfpathmoveto{\pgfqpoint{8.475797in}{2.481632in}}%
\pgfpathlineto{\pgfqpoint{8.425651in}{2.482136in}}%
\pgfusepath{stroke}%
\end{pgfscope}%
\begin{pgfscope}%
\pgfpathrectangle{\pgfqpoint{6.720588in}{1.750000in}}{\pgfqpoint{2.279412in}{2.004545in}}%
\pgfusepath{clip}%
\pgfsetbuttcap%
\pgfsetroundjoin%
\pgfsetlinewidth{0.338940pt}%
\definecolor{currentstroke}{rgb}{0.273809,0.031497,0.358853}%
\pgfsetstrokecolor{currentstroke}%
\pgfsetdash{}{0pt}%
\pgfpathmoveto{\pgfqpoint{8.425651in}{2.482136in}}%
\pgfpathlineto{\pgfqpoint{8.375509in}{2.482847in}}%
\pgfusepath{stroke}%
\end{pgfscope}%
\begin{pgfscope}%
\pgfpathrectangle{\pgfqpoint{6.720588in}{1.750000in}}{\pgfqpoint{2.279412in}{2.004545in}}%
\pgfusepath{clip}%
\pgfsetbuttcap%
\pgfsetroundjoin%
\pgfsetlinewidth{0.357824pt}%
\definecolor{currentstroke}{rgb}{0.277018,0.050344,0.375715}%
\pgfsetstrokecolor{currentstroke}%
\pgfsetdash{}{0pt}%
\pgfpathmoveto{\pgfqpoint{8.375509in}{2.482847in}}%
\pgfpathlineto{\pgfqpoint{8.325374in}{2.483907in}}%
\pgfusepath{stroke}%
\end{pgfscope}%
\begin{pgfscope}%
\pgfpathrectangle{\pgfqpoint{6.720588in}{1.750000in}}{\pgfqpoint{2.279412in}{2.004545in}}%
\pgfusepath{clip}%
\pgfsetbuttcap%
\pgfsetroundjoin%
\pgfsetlinewidth{0.382933pt}%
\definecolor{currentstroke}{rgb}{0.279566,0.067836,0.391917}%
\pgfsetstrokecolor{currentstroke}%
\pgfsetdash{}{0pt}%
\pgfpathmoveto{\pgfqpoint{8.325374in}{2.483907in}}%
\pgfpathlineto{\pgfqpoint{8.275242in}{2.485106in}}%
\pgfusepath{stroke}%
\end{pgfscope}%
\begin{pgfscope}%
\pgfpathrectangle{\pgfqpoint{6.720588in}{1.750000in}}{\pgfqpoint{2.279412in}{2.004545in}}%
\pgfusepath{clip}%
\pgfsetbuttcap%
\pgfsetroundjoin%
\pgfsetlinewidth{0.423402pt}%
\definecolor{currentstroke}{rgb}{0.282656,0.100196,0.422160}%
\pgfsetstrokecolor{currentstroke}%
\pgfsetdash{}{0pt}%
\pgfpathmoveto{\pgfqpoint{8.275242in}{2.485106in}}%
\pgfpathlineto{\pgfqpoint{8.225121in}{2.486637in}}%
\pgfusepath{stroke}%
\end{pgfscope}%
\begin{pgfscope}%
\pgfpathrectangle{\pgfqpoint{6.720588in}{1.750000in}}{\pgfqpoint{2.279412in}{2.004545in}}%
\pgfusepath{clip}%
\pgfsetbuttcap%
\pgfsetroundjoin%
\pgfsetlinewidth{0.460496pt}%
\definecolor{currentstroke}{rgb}{0.283072,0.130895,0.449241}%
\pgfsetstrokecolor{currentstroke}%
\pgfsetdash{}{0pt}%
\pgfpathmoveto{\pgfqpoint{8.225121in}{2.486637in}}%
\pgfpathlineto{\pgfqpoint{8.175024in}{2.488671in}}%
\pgfusepath{stroke}%
\end{pgfscope}%
\begin{pgfscope}%
\pgfpathrectangle{\pgfqpoint{6.720588in}{1.750000in}}{\pgfqpoint{2.279412in}{2.004545in}}%
\pgfusepath{clip}%
\pgfsetbuttcap%
\pgfsetroundjoin%
\pgfsetlinewidth{0.518177pt}%
\definecolor{currentstroke}{rgb}{0.279574,0.170599,0.479997}%
\pgfsetstrokecolor{currentstroke}%
\pgfsetdash{}{0pt}%
\pgfpathmoveto{\pgfqpoint{8.175024in}{2.488671in}}%
\pgfpathlineto{\pgfqpoint{8.124950in}{2.491121in}}%
\pgfusepath{stroke}%
\end{pgfscope}%
\begin{pgfscope}%
\pgfpathrectangle{\pgfqpoint{6.720588in}{1.750000in}}{\pgfqpoint{2.279412in}{2.004545in}}%
\pgfusepath{clip}%
\pgfsetbuttcap%
\pgfsetroundjoin%
\pgfsetlinewidth{0.560406pt}%
\definecolor{currentstroke}{rgb}{0.274128,0.199721,0.498911}%
\pgfsetstrokecolor{currentstroke}%
\pgfsetdash{}{0pt}%
\pgfpathmoveto{\pgfqpoint{8.124950in}{2.491121in}}%
\pgfpathlineto{\pgfqpoint{8.074908in}{2.494013in}}%
\pgfusepath{stroke}%
\end{pgfscope}%
\begin{pgfscope}%
\pgfpathrectangle{\pgfqpoint{6.720588in}{1.750000in}}{\pgfqpoint{2.279412in}{2.004545in}}%
\pgfusepath{clip}%
\pgfsetbuttcap%
\pgfsetroundjoin%
\pgfsetlinewidth{0.601973pt}%
\definecolor{currentstroke}{rgb}{0.266580,0.228262,0.514349}%
\pgfsetstrokecolor{currentstroke}%
\pgfsetdash{}{0pt}%
\pgfpathmoveto{\pgfqpoint{8.074908in}{2.494013in}}%
\pgfpathlineto{\pgfqpoint{8.024944in}{2.497793in}}%
\pgfusepath{stroke}%
\end{pgfscope}%
\begin{pgfscope}%
\pgfpathrectangle{\pgfqpoint{6.720588in}{1.750000in}}{\pgfqpoint{2.279412in}{2.004545in}}%
\pgfusepath{clip}%
\pgfsetbuttcap%
\pgfsetroundjoin%
\pgfsetlinewidth{0.322913pt}%
\definecolor{currentstroke}{rgb}{0.271305,0.019942,0.347269}%
\pgfsetstrokecolor{currentstroke}%
\pgfsetdash{}{0pt}%
\pgfpathmoveto{\pgfqpoint{8.475797in}{2.887593in}}%
\pgfpathlineto{\pgfqpoint{8.425653in}{2.887643in}}%
\pgfusepath{stroke}%
\end{pgfscope}%
\begin{pgfscope}%
\pgfpathrectangle{\pgfqpoint{6.720588in}{1.750000in}}{\pgfqpoint{2.279412in}{2.004545in}}%
\pgfusepath{clip}%
\pgfsetbuttcap%
\pgfsetroundjoin%
\pgfsetlinewidth{0.338614pt}%
\definecolor{currentstroke}{rgb}{0.273809,0.031497,0.358853}%
\pgfsetstrokecolor{currentstroke}%
\pgfsetdash{}{0pt}%
\pgfpathmoveto{\pgfqpoint{8.425653in}{2.887643in}}%
\pgfpathlineto{\pgfqpoint{8.375501in}{2.887738in}}%
\pgfusepath{stroke}%
\end{pgfscope}%
\begin{pgfscope}%
\pgfpathrectangle{\pgfqpoint{6.720588in}{1.750000in}}{\pgfqpoint{2.279412in}{2.004545in}}%
\pgfusepath{clip}%
\pgfsetbuttcap%
\pgfsetroundjoin%
\pgfsetlinewidth{0.372451pt}%
\definecolor{currentstroke}{rgb}{0.278791,0.062145,0.386592}%
\pgfsetstrokecolor{currentstroke}%
\pgfsetdash{}{0pt}%
\pgfpathmoveto{\pgfqpoint{8.375501in}{2.887738in}}%
\pgfpathlineto{\pgfqpoint{8.325352in}{2.887515in}}%
\pgfusepath{stroke}%
\end{pgfscope}%
\begin{pgfscope}%
\pgfpathrectangle{\pgfqpoint{6.720588in}{1.750000in}}{\pgfqpoint{2.279412in}{2.004545in}}%
\pgfusepath{clip}%
\pgfsetbuttcap%
\pgfsetroundjoin%
\pgfsetlinewidth{0.399119pt}%
\definecolor{currentstroke}{rgb}{0.281446,0.084320,0.407414}%
\pgfsetstrokecolor{currentstroke}%
\pgfsetdash{}{0pt}%
\pgfpathmoveto{\pgfqpoint{8.325352in}{2.887515in}}%
\pgfpathlineto{\pgfqpoint{8.275204in}{2.886985in}}%
\pgfusepath{stroke}%
\end{pgfscope}%
\begin{pgfscope}%
\pgfpathrectangle{\pgfqpoint{6.720588in}{1.750000in}}{\pgfqpoint{2.279412in}{2.004545in}}%
\pgfusepath{clip}%
\pgfsetbuttcap%
\pgfsetroundjoin%
\pgfsetlinewidth{0.432833pt}%
\definecolor{currentstroke}{rgb}{0.283091,0.110553,0.431554}%
\pgfsetstrokecolor{currentstroke}%
\pgfsetdash{}{0pt}%
\pgfpathmoveto{\pgfqpoint{8.275204in}{2.886985in}}%
\pgfpathlineto{\pgfqpoint{8.225059in}{2.886299in}}%
\pgfusepath{stroke}%
\end{pgfscope}%
\begin{pgfscope}%
\pgfpathrectangle{\pgfqpoint{6.720588in}{1.750000in}}{\pgfqpoint{2.279412in}{2.004545in}}%
\pgfusepath{clip}%
\pgfsetbuttcap%
\pgfsetroundjoin%
\pgfsetlinewidth{0.500666pt}%
\definecolor{currentstroke}{rgb}{0.280868,0.160771,0.472899}%
\pgfsetstrokecolor{currentstroke}%
\pgfsetdash{}{0pt}%
\pgfpathmoveto{\pgfqpoint{8.225059in}{2.886299in}}%
\pgfpathlineto{\pgfqpoint{8.174919in}{2.885323in}}%
\pgfusepath{stroke}%
\end{pgfscope}%
\begin{pgfscope}%
\pgfpathrectangle{\pgfqpoint{6.720588in}{1.750000in}}{\pgfqpoint{2.279412in}{2.004545in}}%
\pgfusepath{clip}%
\pgfsetbuttcap%
\pgfsetroundjoin%
\pgfsetlinewidth{0.559617pt}%
\definecolor{currentstroke}{rgb}{0.274128,0.199721,0.498911}%
\pgfsetstrokecolor{currentstroke}%
\pgfsetdash{}{0pt}%
\pgfpathmoveto{\pgfqpoint{8.174919in}{2.885323in}}%
\pgfpathlineto{\pgfqpoint{8.124790in}{2.884040in}}%
\pgfusepath{stroke}%
\end{pgfscope}%
\begin{pgfscope}%
\pgfpathrectangle{\pgfqpoint{6.720588in}{1.750000in}}{\pgfqpoint{2.279412in}{2.004545in}}%
\pgfusepath{clip}%
\pgfsetbuttcap%
\pgfsetroundjoin%
\pgfsetlinewidth{0.611546pt}%
\definecolor{currentstroke}{rgb}{0.263663,0.237631,0.518762}%
\pgfsetstrokecolor{currentstroke}%
\pgfsetdash{}{0pt}%
\pgfpathmoveto{\pgfqpoint{8.124790in}{2.884040in}}%
\pgfpathlineto{\pgfqpoint{8.074677in}{2.882316in}}%
\pgfusepath{stroke}%
\end{pgfscope}%
\begin{pgfscope}%
\pgfpathrectangle{\pgfqpoint{6.720588in}{1.750000in}}{\pgfqpoint{2.279412in}{2.004545in}}%
\pgfusepath{clip}%
\pgfsetbuttcap%
\pgfsetroundjoin%
\pgfsetlinewidth{0.716214pt}%
\definecolor{currentstroke}{rgb}{0.237441,0.305202,0.541921}%
\pgfsetstrokecolor{currentstroke}%
\pgfsetdash{}{0pt}%
\pgfpathmoveto{\pgfqpoint{8.074677in}{2.882316in}}%
\pgfpathlineto{\pgfqpoint{8.024588in}{2.880134in}}%
\pgfusepath{stroke}%
\end{pgfscope}%
\begin{pgfscope}%
\pgfpathrectangle{\pgfqpoint{6.720588in}{1.750000in}}{\pgfqpoint{2.279412in}{2.004545in}}%
\pgfusepath{clip}%
\pgfsetbuttcap%
\pgfsetroundjoin%
\pgfsetlinewidth{0.724459pt}%
\definecolor{currentstroke}{rgb}{0.235526,0.309527,0.542944}%
\pgfsetstrokecolor{currentstroke}%
\pgfsetdash{}{0pt}%
\pgfpathmoveto{\pgfqpoint{8.024588in}{2.880134in}}%
\pgfpathlineto{\pgfqpoint{7.974529in}{2.877471in}}%
\pgfusepath{stroke}%
\end{pgfscope}%
\begin{pgfscope}%
\pgfpathrectangle{\pgfqpoint{6.720588in}{1.750000in}}{\pgfqpoint{2.279412in}{2.004545in}}%
\pgfusepath{clip}%
\pgfsetbuttcap%
\pgfsetroundjoin%
\pgfsetlinewidth{0.757878pt}%
\definecolor{currentstroke}{rgb}{0.225863,0.330805,0.547314}%
\pgfsetstrokecolor{currentstroke}%
\pgfsetdash{}{0pt}%
\pgfpathmoveto{\pgfqpoint{7.974529in}{2.877471in}}%
\pgfpathlineto{\pgfqpoint{7.924536in}{2.874012in}}%
\pgfusepath{stroke}%
\end{pgfscope}%
\begin{pgfscope}%
\pgfpathrectangle{\pgfqpoint{6.720588in}{1.750000in}}{\pgfqpoint{2.279412in}{2.004545in}}%
\pgfusepath{clip}%
\pgfsetbuttcap%
\pgfsetroundjoin%
\pgfsetlinewidth{0.320612pt}%
\definecolor{currentstroke}{rgb}{0.269944,0.014625,0.341379}%
\pgfsetstrokecolor{currentstroke}%
\pgfsetdash{}{0pt}%
\pgfpathmoveto{\pgfqpoint{8.475797in}{3.113127in}}%
\pgfpathlineto{\pgfqpoint{8.425743in}{3.110506in}}%
\pgfusepath{stroke}%
\end{pgfscope}%
\begin{pgfscope}%
\pgfpathrectangle{\pgfqpoint{6.720588in}{1.750000in}}{\pgfqpoint{2.279412in}{2.004545in}}%
\pgfusepath{clip}%
\pgfsetbuttcap%
\pgfsetroundjoin%
\pgfsetlinewidth{0.327890pt}%
\definecolor{currentstroke}{rgb}{0.271305,0.019942,0.347269}%
\pgfsetstrokecolor{currentstroke}%
\pgfsetdash{}{0pt}%
\pgfpathmoveto{\pgfqpoint{8.425743in}{3.110506in}}%
\pgfpathlineto{\pgfqpoint{8.375663in}{3.108799in}}%
\pgfusepath{stroke}%
\end{pgfscope}%
\begin{pgfscope}%
\pgfpathrectangle{\pgfqpoint{6.720588in}{1.750000in}}{\pgfqpoint{2.279412in}{2.004545in}}%
\pgfusepath{clip}%
\pgfsetbuttcap%
\pgfsetroundjoin%
\pgfsetlinewidth{0.346370pt}%
\definecolor{currentstroke}{rgb}{0.274952,0.037752,0.364543}%
\pgfsetstrokecolor{currentstroke}%
\pgfsetdash{}{0pt}%
\pgfpathmoveto{\pgfqpoint{8.375663in}{3.108799in}}%
\pgfpathlineto{\pgfqpoint{8.325537in}{3.107828in}}%
\pgfusepath{stroke}%
\end{pgfscope}%
\begin{pgfscope}%
\pgfpathrectangle{\pgfqpoint{6.720588in}{1.750000in}}{\pgfqpoint{2.279412in}{2.004545in}}%
\pgfusepath{clip}%
\pgfsetbuttcap%
\pgfsetroundjoin%
\pgfsetlinewidth{0.367247pt}%
\definecolor{currentstroke}{rgb}{0.277941,0.056324,0.381191}%
\pgfsetstrokecolor{currentstroke}%
\pgfsetdash{}{0pt}%
\pgfpathmoveto{\pgfqpoint{8.325537in}{3.107828in}}%
\pgfpathlineto{\pgfqpoint{8.275436in}{3.105890in}}%
\pgfusepath{stroke}%
\end{pgfscope}%
\begin{pgfscope}%
\pgfpathrectangle{\pgfqpoint{6.720588in}{1.750000in}}{\pgfqpoint{2.279412in}{2.004545in}}%
\pgfusepath{clip}%
\pgfsetbuttcap%
\pgfsetroundjoin%
\pgfsetlinewidth{0.389827pt}%
\definecolor{currentstroke}{rgb}{0.280267,0.073417,0.397163}%
\pgfsetstrokecolor{currentstroke}%
\pgfsetdash{}{0pt}%
\pgfpathmoveto{\pgfqpoint{8.275436in}{3.105890in}}%
\pgfpathlineto{\pgfqpoint{8.225356in}{3.103569in}}%
\pgfusepath{stroke}%
\end{pgfscope}%
\begin{pgfscope}%
\pgfpathrectangle{\pgfqpoint{6.720588in}{1.750000in}}{\pgfqpoint{2.279412in}{2.004545in}}%
\pgfusepath{clip}%
\pgfsetbuttcap%
\pgfsetroundjoin%
\pgfsetlinewidth{0.396960pt}%
\definecolor{currentstroke}{rgb}{0.280894,0.078907,0.402329}%
\pgfsetstrokecolor{currentstroke}%
\pgfsetdash{}{0pt}%
\pgfpathmoveto{\pgfqpoint{8.225356in}{3.103569in}}%
\pgfpathlineto{\pgfqpoint{8.175303in}{3.100813in}}%
\pgfusepath{stroke}%
\end{pgfscope}%
\begin{pgfscope}%
\pgfpathrectangle{\pgfqpoint{6.720588in}{1.750000in}}{\pgfqpoint{2.279412in}{2.004545in}}%
\pgfusepath{clip}%
\pgfsetbuttcap%
\pgfsetroundjoin%
\pgfsetlinewidth{0.407297pt}%
\definecolor{currentstroke}{rgb}{0.281924,0.089666,0.412415}%
\pgfsetstrokecolor{currentstroke}%
\pgfsetdash{}{0pt}%
\pgfpathmoveto{\pgfqpoint{8.175303in}{3.100813in}}%
\pgfpathlineto{\pgfqpoint{8.125284in}{3.097654in}}%
\pgfusepath{stroke}%
\end{pgfscope}%
\begin{pgfscope}%
\pgfpathrectangle{\pgfqpoint{6.720588in}{1.750000in}}{\pgfqpoint{2.279412in}{2.004545in}}%
\pgfusepath{clip}%
\pgfsetbuttcap%
\pgfsetroundjoin%
\pgfsetlinewidth{0.332934pt}%
\definecolor{currentstroke}{rgb}{0.272594,0.025563,0.353093}%
\pgfsetstrokecolor{currentstroke}%
\pgfsetdash{}{0pt}%
\pgfpathmoveto{\pgfqpoint{7.757710in}{3.293554in}}%
\pgfpathlineto{\pgfqpoint{7.745515in}{3.260113in}}%
\pgfusepath{stroke}%
\end{pgfscope}%
\begin{pgfscope}%
\pgfpathrectangle{\pgfqpoint{6.720588in}{1.750000in}}{\pgfqpoint{2.279412in}{2.004545in}}%
\pgfusepath{clip}%
\pgfsetbuttcap%
\pgfsetroundjoin%
\pgfsetlinewidth{0.334022pt}%
\definecolor{currentstroke}{rgb}{0.272594,0.025563,0.353093}%
\pgfsetstrokecolor{currentstroke}%
\pgfsetdash{}{0pt}%
\pgfpathmoveto{\pgfqpoint{7.745515in}{3.260113in}}%
\pgfpathlineto{\pgfqpoint{7.729520in}{3.226837in}}%
\pgfusepath{stroke}%
\end{pgfscope}%
\begin{pgfscope}%
\pgfpathrectangle{\pgfqpoint{6.720588in}{1.750000in}}{\pgfqpoint{2.279412in}{2.004545in}}%
\pgfusepath{clip}%
\pgfsetbuttcap%
\pgfsetroundjoin%
\pgfsetlinewidth{0.369979pt}%
\definecolor{currentstroke}{rgb}{0.278791,0.062145,0.386592}%
\pgfsetstrokecolor{currentstroke}%
\pgfsetdash{}{0pt}%
\pgfpathmoveto{\pgfqpoint{7.729520in}{3.226837in}}%
\pgfpathlineto{\pgfqpoint{7.729520in}{3.226837in}}%
\pgfusepath{stroke}%
\end{pgfscope}%
\begin{pgfscope}%
\pgfpathrectangle{\pgfqpoint{6.720588in}{1.750000in}}{\pgfqpoint{2.279412in}{2.004545in}}%
\pgfusepath{clip}%
\pgfsetbuttcap%
\pgfsetroundjoin%
\pgfsetlinewidth{0.369979pt}%
\definecolor{currentstroke}{rgb}{0.278791,0.062145,0.386592}%
\pgfsetstrokecolor{currentstroke}%
\pgfsetdash{}{0pt}%
\pgfpathmoveto{\pgfqpoint{7.729520in}{3.226837in}}%
\pgfpathlineto{\pgfqpoint{7.729520in}{3.226837in}}%
\pgfusepath{stroke}%
\end{pgfscope}%
\begin{pgfscope}%
\pgfpathrectangle{\pgfqpoint{6.720588in}{1.750000in}}{\pgfqpoint{2.279412in}{2.004545in}}%
\pgfusepath{clip}%
\pgfsetbuttcap%
\pgfsetroundjoin%
\pgfsetlinewidth{0.369979pt}%
\definecolor{currentstroke}{rgb}{0.278791,0.062145,0.386592}%
\pgfsetstrokecolor{currentstroke}%
\pgfsetdash{}{0pt}%
\pgfpathmoveto{\pgfqpoint{7.729520in}{3.226837in}}%
\pgfpathlineto{\pgfqpoint{7.725373in}{3.204051in}}%
\pgfusepath{stroke}%
\end{pgfscope}%
\begin{pgfscope}%
\pgfpathrectangle{\pgfqpoint{6.720588in}{1.750000in}}{\pgfqpoint{2.279412in}{2.004545in}}%
\pgfusepath{clip}%
\pgfsetbuttcap%
\pgfsetroundjoin%
\pgfsetlinewidth{0.384612pt}%
\definecolor{currentstroke}{rgb}{0.280267,0.073417,0.397163}%
\pgfsetstrokecolor{currentstroke}%
\pgfsetdash{}{0pt}%
\pgfpathmoveto{\pgfqpoint{7.725373in}{3.204051in}}%
\pgfpathlineto{\pgfqpoint{7.725966in}{3.182342in}}%
\pgfusepath{stroke}%
\end{pgfscope}%
\begin{pgfscope}%
\pgfpathrectangle{\pgfqpoint{6.720588in}{1.750000in}}{\pgfqpoint{2.279412in}{2.004545in}}%
\pgfusepath{clip}%
\pgfsetbuttcap%
\pgfsetroundjoin%
\pgfsetlinewidth{0.387732pt}%
\definecolor{currentstroke}{rgb}{0.280267,0.073417,0.397163}%
\pgfsetstrokecolor{currentstroke}%
\pgfsetdash{}{0pt}%
\pgfpathmoveto{\pgfqpoint{7.725966in}{3.182342in}}%
\pgfpathlineto{\pgfqpoint{7.725966in}{3.182342in}}%
\pgfusepath{stroke}%
\end{pgfscope}%
\begin{pgfscope}%
\pgfpathrectangle{\pgfqpoint{6.720588in}{1.750000in}}{\pgfqpoint{2.279412in}{2.004545in}}%
\pgfusepath{clip}%
\pgfsetbuttcap%
\pgfsetroundjoin%
\pgfsetlinewidth{0.387732pt}%
\definecolor{currentstroke}{rgb}{0.280267,0.073417,0.397163}%
\pgfsetstrokecolor{currentstroke}%
\pgfsetdash{}{0pt}%
\pgfpathmoveto{\pgfqpoint{7.725966in}{3.182342in}}%
\pgfpathlineto{\pgfqpoint{7.722673in}{3.156996in}}%
\pgfusepath{stroke}%
\end{pgfscope}%
\begin{pgfscope}%
\pgfpathrectangle{\pgfqpoint{6.720588in}{1.750000in}}{\pgfqpoint{2.279412in}{2.004545in}}%
\pgfusepath{clip}%
\pgfsetbuttcap%
\pgfsetroundjoin%
\pgfsetlinewidth{0.396936pt}%
\definecolor{currentstroke}{rgb}{0.280894,0.078907,0.402329}%
\pgfsetstrokecolor{currentstroke}%
\pgfsetdash{}{0pt}%
\pgfpathmoveto{\pgfqpoint{7.722673in}{3.156996in}}%
\pgfpathlineto{\pgfqpoint{7.716390in}{3.132258in}}%
\pgfusepath{stroke}%
\end{pgfscope}%
\begin{pgfscope}%
\pgfpathrectangle{\pgfqpoint{6.720588in}{1.750000in}}{\pgfqpoint{2.279412in}{2.004545in}}%
\pgfusepath{clip}%
\pgfsetbuttcap%
\pgfsetroundjoin%
\pgfsetlinewidth{0.381338pt}%
\definecolor{currentstroke}{rgb}{0.279566,0.067836,0.391917}%
\pgfsetstrokecolor{currentstroke}%
\pgfsetdash{}{0pt}%
\pgfpathmoveto{\pgfqpoint{7.716390in}{3.132258in}}%
\pgfpathlineto{\pgfqpoint{7.709632in}{3.106824in}}%
\pgfusepath{stroke}%
\end{pgfscope}%
\begin{pgfscope}%
\pgfpathrectangle{\pgfqpoint{6.720588in}{1.750000in}}{\pgfqpoint{2.279412in}{2.004545in}}%
\pgfusepath{clip}%
\pgfsetbuttcap%
\pgfsetroundjoin%
\pgfsetlinewidth{1.386205pt}%
\definecolor{currentstroke}{rgb}{0.150148,0.676631,0.506589}%
\pgfsetstrokecolor{currentstroke}%
\pgfsetdash{}{0pt}%
\pgfpathmoveto{\pgfqpoint{7.095839in}{2.815869in}}%
\pgfpathlineto{\pgfqpoint{7.145534in}{2.809966in}}%
\pgfusepath{stroke}%
\end{pgfscope}%
\begin{pgfscope}%
\pgfpathrectangle{\pgfqpoint{6.720588in}{1.750000in}}{\pgfqpoint{2.279412in}{2.004545in}}%
\pgfusepath{clip}%
\pgfsetbuttcap%
\pgfsetroundjoin%
\pgfsetlinewidth{1.211341pt}%
\definecolor{currentstroke}{rgb}{0.123463,0.581687,0.547445}%
\pgfsetstrokecolor{currentstroke}%
\pgfsetdash{}{0pt}%
\pgfpathmoveto{\pgfqpoint{7.145534in}{2.809966in}}%
\pgfpathlineto{\pgfqpoint{7.195216in}{2.803993in}}%
\pgfusepath{stroke}%
\end{pgfscope}%
\begin{pgfscope}%
\pgfpathrectangle{\pgfqpoint{6.720588in}{1.750000in}}{\pgfqpoint{2.279412in}{2.004545in}}%
\pgfusepath{clip}%
\pgfsetbuttcap%
\pgfsetroundjoin%
\pgfsetlinewidth{1.052322pt}%
\definecolor{currentstroke}{rgb}{0.153364,0.497000,0.557724}%
\pgfsetstrokecolor{currentstroke}%
\pgfsetdash{}{0pt}%
\pgfpathmoveto{\pgfqpoint{7.195216in}{2.803993in}}%
\pgfpathlineto{\pgfqpoint{7.244791in}{2.797380in}}%
\pgfusepath{stroke}%
\end{pgfscope}%
\begin{pgfscope}%
\pgfpathrectangle{\pgfqpoint{6.720588in}{1.750000in}}{\pgfqpoint{2.279412in}{2.004545in}}%
\pgfusepath{clip}%
\pgfsetbuttcap%
\pgfsetroundjoin%
\pgfsetlinewidth{0.891593pt}%
\definecolor{currentstroke}{rgb}{0.188923,0.410910,0.556326}%
\pgfsetstrokecolor{currentstroke}%
\pgfsetdash{}{0pt}%
\pgfpathmoveto{\pgfqpoint{7.244791in}{2.797380in}}%
\pgfpathlineto{\pgfqpoint{7.294260in}{2.790198in}}%
\pgfusepath{stroke}%
\end{pgfscope}%
\begin{pgfscope}%
\pgfpathrectangle{\pgfqpoint{6.720588in}{1.750000in}}{\pgfqpoint{2.279412in}{2.004545in}}%
\pgfusepath{clip}%
\pgfsetbuttcap%
\pgfsetroundjoin%
\pgfsetlinewidth{0.797037pt}%
\definecolor{currentstroke}{rgb}{0.214298,0.355619,0.551184}%
\pgfsetstrokecolor{currentstroke}%
\pgfsetdash{}{0pt}%
\pgfpathmoveto{\pgfqpoint{7.294260in}{2.790198in}}%
\pgfpathlineto{\pgfqpoint{7.343196in}{2.781175in}}%
\pgfusepath{stroke}%
\end{pgfscope}%
\begin{pgfscope}%
\pgfpathrectangle{\pgfqpoint{6.720588in}{1.750000in}}{\pgfqpoint{2.279412in}{2.004545in}}%
\pgfusepath{clip}%
\pgfsetbuttcap%
\pgfsetroundjoin%
\pgfsetlinewidth{0.331499pt}%
\definecolor{currentstroke}{rgb}{0.272594,0.025563,0.353093}%
\pgfsetstrokecolor{currentstroke}%
\pgfsetdash{}{0pt}%
\pgfpathmoveto{\pgfqpoint{8.424505in}{3.203341in}}%
\pgfpathlineto{\pgfqpoint{8.374357in}{3.203018in}}%
\pgfusepath{stroke}%
\end{pgfscope}%
\begin{pgfscope}%
\pgfpathrectangle{\pgfqpoint{6.720588in}{1.750000in}}{\pgfqpoint{2.279412in}{2.004545in}}%
\pgfusepath{clip}%
\pgfsetbuttcap%
\pgfsetroundjoin%
\pgfsetlinewidth{0.338183pt}%
\definecolor{currentstroke}{rgb}{0.273809,0.031497,0.358853}%
\pgfsetstrokecolor{currentstroke}%
\pgfsetdash{}{0pt}%
\pgfpathmoveto{\pgfqpoint{8.374357in}{3.203018in}}%
\pgfpathlineto{\pgfqpoint{8.324239in}{3.202155in}}%
\pgfusepath{stroke}%
\end{pgfscope}%
\begin{pgfscope}%
\pgfpathrectangle{\pgfqpoint{6.720588in}{1.750000in}}{\pgfqpoint{2.279412in}{2.004545in}}%
\pgfusepath{clip}%
\pgfsetbuttcap%
\pgfsetroundjoin%
\pgfsetlinewidth{0.343065pt}%
\definecolor{currentstroke}{rgb}{0.274952,0.037752,0.364543}%
\pgfsetstrokecolor{currentstroke}%
\pgfsetdash{}{0pt}%
\pgfpathmoveto{\pgfqpoint{8.324239in}{3.202155in}}%
\pgfpathlineto{\pgfqpoint{8.274143in}{3.200663in}}%
\pgfusepath{stroke}%
\end{pgfscope}%
\begin{pgfscope}%
\pgfpathrectangle{\pgfqpoint{6.720588in}{1.750000in}}{\pgfqpoint{2.279412in}{2.004545in}}%
\pgfusepath{clip}%
\pgfsetbuttcap%
\pgfsetroundjoin%
\pgfsetlinewidth{0.364458pt}%
\definecolor{currentstroke}{rgb}{0.277941,0.056324,0.381191}%
\pgfsetstrokecolor{currentstroke}%
\pgfsetdash{}{0pt}%
\pgfpathmoveto{\pgfqpoint{8.274143in}{3.200663in}}%
\pgfpathlineto{\pgfqpoint{8.224090in}{3.198125in}}%
\pgfusepath{stroke}%
\end{pgfscope}%
\begin{pgfscope}%
\pgfpathrectangle{\pgfqpoint{6.720588in}{1.750000in}}{\pgfqpoint{2.279412in}{2.004545in}}%
\pgfusepath{clip}%
\pgfsetbuttcap%
\pgfsetroundjoin%
\pgfsetlinewidth{0.385535pt}%
\definecolor{currentstroke}{rgb}{0.280267,0.073417,0.397163}%
\pgfsetstrokecolor{currentstroke}%
\pgfsetdash{}{0pt}%
\pgfpathmoveto{\pgfqpoint{8.224090in}{3.198125in}}%
\pgfpathlineto{\pgfqpoint{8.174114in}{3.194478in}}%
\pgfusepath{stroke}%
\end{pgfscope}%
\begin{pgfscope}%
\pgfpathrectangle{\pgfqpoint{6.720588in}{1.750000in}}{\pgfqpoint{2.279412in}{2.004545in}}%
\pgfusepath{clip}%
\pgfsetbuttcap%
\pgfsetroundjoin%
\pgfsetlinewidth{0.392484pt}%
\definecolor{currentstroke}{rgb}{0.280894,0.078907,0.402329}%
\pgfsetstrokecolor{currentstroke}%
\pgfsetdash{}{0pt}%
\pgfpathmoveto{\pgfqpoint{8.174114in}{3.194478in}}%
\pgfpathlineto{\pgfqpoint{8.124166in}{3.190571in}}%
\pgfusepath{stroke}%
\end{pgfscope}%
\begin{pgfscope}%
\pgfpathrectangle{\pgfqpoint{6.720588in}{1.750000in}}{\pgfqpoint{2.279412in}{2.004545in}}%
\pgfusepath{clip}%
\pgfsetbuttcap%
\pgfsetroundjoin%
\pgfsetlinewidth{0.401018pt}%
\definecolor{currentstroke}{rgb}{0.281446,0.084320,0.407414}%
\pgfsetstrokecolor{currentstroke}%
\pgfsetdash{}{0pt}%
\pgfpathmoveto{\pgfqpoint{8.124166in}{3.190571in}}%
\pgfpathlineto{\pgfqpoint{8.074383in}{3.185411in}}%
\pgfusepath{stroke}%
\end{pgfscope}%
\begin{pgfscope}%
\pgfpathrectangle{\pgfqpoint{6.720588in}{1.750000in}}{\pgfqpoint{2.279412in}{2.004545in}}%
\pgfusepath{clip}%
\pgfsetbuttcap%
\pgfsetroundjoin%
\pgfsetlinewidth{0.426624pt}%
\definecolor{currentstroke}{rgb}{0.282910,0.105393,0.426902}%
\pgfsetstrokecolor{currentstroke}%
\pgfsetdash{}{0pt}%
\pgfpathmoveto{\pgfqpoint{8.074383in}{3.185411in}}%
\pgfpathlineto{\pgfqpoint{8.025005in}{3.177854in}}%
\pgfusepath{stroke}%
\end{pgfscope}%
\begin{pgfscope}%
\pgfpathrectangle{\pgfqpoint{6.720588in}{1.750000in}}{\pgfqpoint{2.279412in}{2.004545in}}%
\pgfusepath{clip}%
\pgfsetbuttcap%
\pgfsetroundjoin%
\pgfsetlinewidth{0.391750pt}%
\definecolor{currentstroke}{rgb}{0.280894,0.078907,0.402329}%
\pgfsetstrokecolor{currentstroke}%
\pgfsetdash{}{0pt}%
\pgfpathmoveto{\pgfqpoint{8.025005in}{3.177854in}}%
\pgfpathlineto{\pgfqpoint{7.976104in}{3.168182in}}%
\pgfusepath{stroke}%
\end{pgfscope}%
\begin{pgfscope}%
\pgfpathrectangle{\pgfqpoint{6.720588in}{1.750000in}}{\pgfqpoint{2.279412in}{2.004545in}}%
\pgfusepath{clip}%
\pgfsetbuttcap%
\pgfsetroundjoin%
\pgfsetlinewidth{0.418645pt}%
\definecolor{currentstroke}{rgb}{0.282656,0.100196,0.422160}%
\pgfsetstrokecolor{currentstroke}%
\pgfsetdash{}{0pt}%
\pgfpathmoveto{\pgfqpoint{7.976104in}{3.168182in}}%
\pgfpathlineto{\pgfqpoint{7.927598in}{3.157025in}}%
\pgfusepath{stroke}%
\end{pgfscope}%
\begin{pgfscope}%
\pgfpathrectangle{\pgfqpoint{6.720588in}{1.750000in}}{\pgfqpoint{2.279412in}{2.004545in}}%
\pgfusepath{clip}%
\pgfsetbuttcap%
\pgfsetroundjoin%
\pgfsetlinewidth{0.426612pt}%
\definecolor{currentstroke}{rgb}{0.282910,0.105393,0.426902}%
\pgfsetstrokecolor{currentstroke}%
\pgfsetdash{}{0pt}%
\pgfpathmoveto{\pgfqpoint{7.927598in}{3.157025in}}%
\pgfpathlineto{\pgfqpoint{7.879743in}{3.144003in}}%
\pgfusepath{stroke}%
\end{pgfscope}%
\begin{pgfscope}%
\pgfpathrectangle{\pgfqpoint{6.720588in}{1.750000in}}{\pgfqpoint{2.279412in}{2.004545in}}%
\pgfusepath{clip}%
\pgfsetbuttcap%
\pgfsetroundjoin%
\pgfsetlinewidth{0.418073pt}%
\definecolor{currentstroke}{rgb}{0.282656,0.100196,0.422160}%
\pgfsetstrokecolor{currentstroke}%
\pgfsetdash{}{0pt}%
\pgfpathmoveto{\pgfqpoint{7.879743in}{3.144003in}}%
\pgfpathlineto{\pgfqpoint{7.834529in}{3.125564in}}%
\pgfusepath{stroke}%
\end{pgfscope}%
\begin{pgfscope}%
\pgfpathrectangle{\pgfqpoint{6.720588in}{1.750000in}}{\pgfqpoint{2.279412in}{2.004545in}}%
\pgfusepath{clip}%
\pgfsetbuttcap%
\pgfsetroundjoin%
\pgfsetlinewidth{0.424347pt}%
\definecolor{currentstroke}{rgb}{0.282656,0.100196,0.422160}%
\pgfsetstrokecolor{currentstroke}%
\pgfsetdash{}{0pt}%
\pgfpathmoveto{\pgfqpoint{7.834529in}{3.125564in}}%
\pgfpathlineto{\pgfqpoint{7.793358in}{3.103801in}}%
\pgfusepath{stroke}%
\end{pgfscope}%
\begin{pgfscope}%
\pgfpathrectangle{\pgfqpoint{6.720588in}{1.750000in}}{\pgfqpoint{2.279412in}{2.004545in}}%
\pgfusepath{clip}%
\pgfsetbuttcap%
\pgfsetroundjoin%
\pgfsetlinewidth{0.461936pt}%
\definecolor{currentstroke}{rgb}{0.283072,0.130895,0.449241}%
\pgfsetstrokecolor{currentstroke}%
\pgfsetdash{}{0pt}%
\pgfpathmoveto{\pgfqpoint{7.793358in}{3.103801in}}%
\pgfpathlineto{\pgfqpoint{7.753644in}{3.077269in}}%
\pgfusepath{stroke}%
\end{pgfscope}%
\begin{pgfscope}%
\pgfpathrectangle{\pgfqpoint{6.720588in}{1.750000in}}{\pgfqpoint{2.279412in}{2.004545in}}%
\pgfusepath{clip}%
\pgfsetbuttcap%
\pgfsetroundjoin%
\pgfsetlinewidth{0.481744pt}%
\definecolor{currentstroke}{rgb}{0.282290,0.145912,0.461510}%
\pgfsetstrokecolor{currentstroke}%
\pgfsetdash{}{0pt}%
\pgfpathmoveto{\pgfqpoint{7.753644in}{3.077269in}}%
\pgfpathlineto{\pgfqpoint{7.716144in}{3.048424in}}%
\pgfusepath{stroke}%
\end{pgfscope}%
\begin{pgfscope}%
\pgfpathrectangle{\pgfqpoint{6.720588in}{1.750000in}}{\pgfqpoint{2.279412in}{2.004545in}}%
\pgfusepath{clip}%
\pgfsetbuttcap%
\pgfsetroundjoin%
\pgfsetlinewidth{0.323178pt}%
\definecolor{currentstroke}{rgb}{0.271305,0.019942,0.347269}%
\pgfsetstrokecolor{currentstroke}%
\pgfsetdash{}{0pt}%
\pgfpathmoveto{\pgfqpoint{8.424505in}{3.248447in}}%
\pgfpathlineto{\pgfqpoint{8.374394in}{3.246823in}}%
\pgfusepath{stroke}%
\end{pgfscope}%
\begin{pgfscope}%
\pgfpathrectangle{\pgfqpoint{6.720588in}{1.750000in}}{\pgfqpoint{2.279412in}{2.004545in}}%
\pgfusepath{clip}%
\pgfsetbuttcap%
\pgfsetroundjoin%
\pgfsetlinewidth{0.338091pt}%
\definecolor{currentstroke}{rgb}{0.273809,0.031497,0.358853}%
\pgfsetstrokecolor{currentstroke}%
\pgfsetdash{}{0pt}%
\pgfpathmoveto{\pgfqpoint{8.374394in}{3.246823in}}%
\pgfpathlineto{\pgfqpoint{8.324263in}{3.245859in}}%
\pgfusepath{stroke}%
\end{pgfscope}%
\begin{pgfscope}%
\pgfpathrectangle{\pgfqpoint{6.720588in}{1.750000in}}{\pgfqpoint{2.279412in}{2.004545in}}%
\pgfusepath{clip}%
\pgfsetbuttcap%
\pgfsetroundjoin%
\pgfsetlinewidth{0.350127pt}%
\definecolor{currentstroke}{rgb}{0.276022,0.044167,0.370164}%
\pgfsetstrokecolor{currentstroke}%
\pgfsetdash{}{0pt}%
\pgfpathmoveto{\pgfqpoint{8.324263in}{3.245859in}}%
\pgfpathlineto{\pgfqpoint{8.274146in}{3.244524in}}%
\pgfusepath{stroke}%
\end{pgfscope}%
\begin{pgfscope}%
\pgfpathrectangle{\pgfqpoint{6.720588in}{1.750000in}}{\pgfqpoint{2.279412in}{2.004545in}}%
\pgfusepath{clip}%
\pgfsetbuttcap%
\pgfsetroundjoin%
\pgfsetlinewidth{0.351620pt}%
\definecolor{currentstroke}{rgb}{0.276022,0.044167,0.370164}%
\pgfsetstrokecolor{currentstroke}%
\pgfsetdash{}{0pt}%
\pgfpathmoveto{\pgfqpoint{8.274146in}{3.244524in}}%
\pgfpathlineto{\pgfqpoint{8.224087in}{3.242016in}}%
\pgfusepath{stroke}%
\end{pgfscope}%
\begin{pgfscope}%
\pgfpathrectangle{\pgfqpoint{6.720588in}{1.750000in}}{\pgfqpoint{2.279412in}{2.004545in}}%
\pgfusepath{clip}%
\pgfsetbuttcap%
\pgfsetroundjoin%
\pgfsetlinewidth{0.363601pt}%
\definecolor{currentstroke}{rgb}{0.277941,0.056324,0.381191}%
\pgfsetstrokecolor{currentstroke}%
\pgfsetdash{}{0pt}%
\pgfpathmoveto{\pgfqpoint{8.224087in}{3.242016in}}%
\pgfpathlineto{\pgfqpoint{8.174134in}{3.238380in}}%
\pgfusepath{stroke}%
\end{pgfscope}%
\begin{pgfscope}%
\pgfpathrectangle{\pgfqpoint{6.720588in}{1.750000in}}{\pgfqpoint{2.279412in}{2.004545in}}%
\pgfusepath{clip}%
\pgfsetbuttcap%
\pgfsetroundjoin%
\pgfsetlinewidth{0.380897pt}%
\definecolor{currentstroke}{rgb}{0.279566,0.067836,0.391917}%
\pgfsetstrokecolor{currentstroke}%
\pgfsetdash{}{0pt}%
\pgfpathmoveto{\pgfqpoint{8.174134in}{3.238380in}}%
\pgfpathlineto{\pgfqpoint{8.124357in}{3.233312in}}%
\pgfusepath{stroke}%
\end{pgfscope}%
\begin{pgfscope}%
\pgfpathrectangle{\pgfqpoint{6.720588in}{1.750000in}}{\pgfqpoint{2.279412in}{2.004545in}}%
\pgfusepath{clip}%
\pgfsetbuttcap%
\pgfsetroundjoin%
\pgfsetlinewidth{0.379900pt}%
\definecolor{currentstroke}{rgb}{0.279566,0.067836,0.391917}%
\pgfsetstrokecolor{currentstroke}%
\pgfsetdash{}{0pt}%
\pgfpathmoveto{\pgfqpoint{8.065462in}{3.248447in}}%
\pgfpathlineto{\pgfqpoint{8.016077in}{3.240843in}}%
\pgfusepath{stroke}%
\end{pgfscope}%
\begin{pgfscope}%
\pgfpathrectangle{\pgfqpoint{6.720588in}{1.750000in}}{\pgfqpoint{2.279412in}{2.004545in}}%
\pgfusepath{clip}%
\pgfsetbuttcap%
\pgfsetroundjoin%
\pgfsetlinewidth{0.397009pt}%
\definecolor{currentstroke}{rgb}{0.280894,0.078907,0.402329}%
\pgfsetstrokecolor{currentstroke}%
\pgfsetdash{}{0pt}%
\pgfpathmoveto{\pgfqpoint{8.016077in}{3.240843in}}%
\pgfpathlineto{\pgfqpoint{7.967874in}{3.229285in}}%
\pgfusepath{stroke}%
\end{pgfscope}%
\begin{pgfscope}%
\pgfpathrectangle{\pgfqpoint{6.720588in}{1.750000in}}{\pgfqpoint{2.279412in}{2.004545in}}%
\pgfusepath{clip}%
\pgfsetbuttcap%
\pgfsetroundjoin%
\pgfsetlinewidth{0.389370pt}%
\definecolor{currentstroke}{rgb}{0.280267,0.073417,0.397163}%
\pgfsetstrokecolor{currentstroke}%
\pgfsetdash{}{0pt}%
\pgfpathmoveto{\pgfqpoint{7.967874in}{3.229285in}}%
\pgfpathlineto{\pgfqpoint{7.920662in}{3.214610in}}%
\pgfusepath{stroke}%
\end{pgfscope}%
\begin{pgfscope}%
\pgfpathrectangle{\pgfqpoint{6.720588in}{1.750000in}}{\pgfqpoint{2.279412in}{2.004545in}}%
\pgfusepath{clip}%
\pgfsetbuttcap%
\pgfsetroundjoin%
\pgfsetlinewidth{0.390290pt}%
\definecolor{currentstroke}{rgb}{0.280267,0.073417,0.397163}%
\pgfsetstrokecolor{currentstroke}%
\pgfsetdash{}{0pt}%
\pgfpathmoveto{\pgfqpoint{7.920662in}{3.214610in}}%
\pgfpathlineto{\pgfqpoint{7.873778in}{3.199282in}}%
\pgfusepath{stroke}%
\end{pgfscope}%
\begin{pgfscope}%
\pgfpathrectangle{\pgfqpoint{6.720588in}{1.750000in}}{\pgfqpoint{2.279412in}{2.004545in}}%
\pgfusepath{clip}%
\pgfsetbuttcap%
\pgfsetroundjoin%
\pgfsetlinewidth{0.376689pt}%
\definecolor{currentstroke}{rgb}{0.279566,0.067836,0.391917}%
\pgfsetstrokecolor{currentstroke}%
\pgfsetdash{}{0pt}%
\pgfpathmoveto{\pgfqpoint{7.873778in}{3.199282in}}%
\pgfpathlineto{\pgfqpoint{7.829303in}{3.179880in}}%
\pgfusepath{stroke}%
\end{pgfscope}%
\begin{pgfscope}%
\pgfpathrectangle{\pgfqpoint{6.720588in}{1.750000in}}{\pgfqpoint{2.279412in}{2.004545in}}%
\pgfusepath{clip}%
\pgfsetbuttcap%
\pgfsetroundjoin%
\pgfsetlinewidth{0.372502pt}%
\definecolor{currentstroke}{rgb}{0.278791,0.062145,0.386592}%
\pgfsetstrokecolor{currentstroke}%
\pgfsetdash{}{0pt}%
\pgfpathmoveto{\pgfqpoint{7.829303in}{3.179880in}}%
\pgfpathlineto{\pgfqpoint{7.788752in}{3.155199in}}%
\pgfusepath{stroke}%
\end{pgfscope}%
\begin{pgfscope}%
\pgfpathrectangle{\pgfqpoint{6.720588in}{1.750000in}}{\pgfqpoint{2.279412in}{2.004545in}}%
\pgfusepath{clip}%
\pgfsetbuttcap%
\pgfsetroundjoin%
\pgfsetlinewidth{0.395876pt}%
\definecolor{currentstroke}{rgb}{0.280894,0.078907,0.402329}%
\pgfsetstrokecolor{currentstroke}%
\pgfsetdash{}{0pt}%
\pgfpathmoveto{\pgfqpoint{7.788752in}{3.155199in}}%
\pgfpathlineto{\pgfqpoint{7.751101in}{3.128339in}}%
\pgfusepath{stroke}%
\end{pgfscope}%
\begin{pgfscope}%
\pgfpathrectangle{\pgfqpoint{6.720588in}{1.750000in}}{\pgfqpoint{2.279412in}{2.004545in}}%
\pgfusepath{clip}%
\pgfsetbuttcap%
\pgfsetroundjoin%
\pgfsetlinewidth{0.392835pt}%
\definecolor{currentstroke}{rgb}{0.280894,0.078907,0.402329}%
\pgfsetstrokecolor{currentstroke}%
\pgfsetdash{}{0pt}%
\pgfpathmoveto{\pgfqpoint{7.751101in}{3.128339in}}%
\pgfpathlineto{\pgfqpoint{7.751101in}{3.128339in}}%
\pgfusepath{stroke}%
\end{pgfscope}%
\begin{pgfscope}%
\pgfpathrectangle{\pgfqpoint{6.720588in}{1.750000in}}{\pgfqpoint{2.279412in}{2.004545in}}%
\pgfusepath{clip}%
\pgfsetbuttcap%
\pgfsetroundjoin%
\pgfsetlinewidth{0.339876pt}%
\definecolor{currentstroke}{rgb}{0.273809,0.031497,0.358853}%
\pgfsetstrokecolor{currentstroke}%
\pgfsetdash{}{0pt}%
\pgfpathmoveto{\pgfqpoint{8.005012in}{3.282031in}}%
\pgfpathlineto{\pgfqpoint{7.955703in}{3.274872in}}%
\pgfusepath{stroke}%
\end{pgfscope}%
\begin{pgfscope}%
\pgfpathrectangle{\pgfqpoint{6.720588in}{1.750000in}}{\pgfqpoint{2.279412in}{2.004545in}}%
\pgfusepath{clip}%
\pgfsetbuttcap%
\pgfsetroundjoin%
\pgfsetlinewidth{0.358083pt}%
\definecolor{currentstroke}{rgb}{0.277018,0.050344,0.375715}%
\pgfsetstrokecolor{currentstroke}%
\pgfsetdash{}{0pt}%
\pgfpathmoveto{\pgfqpoint{7.955703in}{3.274872in}}%
\pgfpathlineto{\pgfqpoint{7.907647in}{3.262840in}}%
\pgfusepath{stroke}%
\end{pgfscope}%
\begin{pgfscope}%
\pgfpathrectangle{\pgfqpoint{6.720588in}{1.750000in}}{\pgfqpoint{2.279412in}{2.004545in}}%
\pgfusepath{clip}%
\pgfsetbuttcap%
\pgfsetroundjoin%
\pgfsetlinewidth{0.359353pt}%
\definecolor{currentstroke}{rgb}{0.277018,0.050344,0.375715}%
\pgfsetstrokecolor{currentstroke}%
\pgfsetdash{}{0pt}%
\pgfpathmoveto{\pgfqpoint{7.907647in}{3.262840in}}%
\pgfpathlineto{\pgfqpoint{7.860294in}{3.248447in}}%
\pgfusepath{stroke}%
\end{pgfscope}%
\begin{pgfscope}%
\pgfpathrectangle{\pgfqpoint{6.720588in}{1.750000in}}{\pgfqpoint{2.279412in}{2.004545in}}%
\pgfusepath{clip}%
\pgfsetbuttcap%
\pgfsetroundjoin%
\pgfsetlinewidth{0.379629pt}%
\definecolor{currentstroke}{rgb}{0.279566,0.067836,0.391917}%
\pgfsetstrokecolor{currentstroke}%
\pgfsetdash{}{0pt}%
\pgfpathmoveto{\pgfqpoint{7.860294in}{3.248447in}}%
\pgfpathlineto{\pgfqpoint{7.860294in}{3.248447in}}%
\pgfusepath{stroke}%
\end{pgfscope}%
\begin{pgfscope}%
\pgfpathrectangle{\pgfqpoint{6.720588in}{1.750000in}}{\pgfqpoint{2.279412in}{2.004545in}}%
\pgfusepath{clip}%
\pgfsetbuttcap%
\pgfsetroundjoin%
\pgfsetlinewidth{0.379629pt}%
\definecolor{currentstroke}{rgb}{0.279566,0.067836,0.391917}%
\pgfsetstrokecolor{currentstroke}%
\pgfsetdash{}{0pt}%
\pgfpathmoveto{\pgfqpoint{7.860294in}{3.248447in}}%
\pgfpathlineto{\pgfqpoint{7.829523in}{3.233368in}}%
\pgfusepath{stroke}%
\end{pgfscope}%
\begin{pgfscope}%
\pgfpathrectangle{\pgfqpoint{6.720588in}{1.750000in}}{\pgfqpoint{2.279412in}{2.004545in}}%
\pgfusepath{clip}%
\pgfsetbuttcap%
\pgfsetroundjoin%
\pgfsetlinewidth{0.367284pt}%
\definecolor{currentstroke}{rgb}{0.277941,0.056324,0.381191}%
\pgfsetstrokecolor{currentstroke}%
\pgfsetdash{}{0pt}%
\pgfpathmoveto{\pgfqpoint{7.829523in}{3.233368in}}%
\pgfpathlineto{\pgfqpoint{7.805012in}{3.215602in}}%
\pgfusepath{stroke}%
\end{pgfscope}%
\begin{pgfscope}%
\pgfpathrectangle{\pgfqpoint{6.720588in}{1.750000in}}{\pgfqpoint{2.279412in}{2.004545in}}%
\pgfusepath{clip}%
\pgfsetbuttcap%
\pgfsetroundjoin%
\pgfsetlinewidth{0.367493pt}%
\definecolor{currentstroke}{rgb}{0.277941,0.056324,0.381191}%
\pgfsetstrokecolor{currentstroke}%
\pgfsetdash{}{0pt}%
\pgfpathmoveto{\pgfqpoint{7.805012in}{3.215602in}}%
\pgfpathlineto{\pgfqpoint{7.776758in}{3.187482in}}%
\pgfusepath{stroke}%
\end{pgfscope}%
\begin{pgfscope}%
\pgfpathrectangle{\pgfqpoint{6.720588in}{1.750000in}}{\pgfqpoint{2.279412in}{2.004545in}}%
\pgfusepath{clip}%
\pgfsetbuttcap%
\pgfsetroundjoin%
\pgfsetlinewidth{0.420282pt}%
\definecolor{currentstroke}{rgb}{0.282656,0.100196,0.422160}%
\pgfsetstrokecolor{currentstroke}%
\pgfsetdash{}{0pt}%
\pgfpathmoveto{\pgfqpoint{7.776758in}{3.187482in}}%
\pgfpathlineto{\pgfqpoint{7.751533in}{3.150574in}}%
\pgfusepath{stroke}%
\end{pgfscope}%
\begin{pgfscope}%
\pgfpathrectangle{\pgfqpoint{6.720588in}{1.750000in}}{\pgfqpoint{2.279412in}{2.004545in}}%
\pgfusepath{clip}%
\pgfsetbuttcap%
\pgfsetroundjoin%
\pgfsetlinewidth{0.417665pt}%
\definecolor{currentstroke}{rgb}{0.282327,0.094955,0.417331}%
\pgfsetstrokecolor{currentstroke}%
\pgfsetdash{}{0pt}%
\pgfpathmoveto{\pgfqpoint{7.751533in}{3.150574in}}%
\pgfpathlineto{\pgfqpoint{7.751533in}{3.150574in}}%
\pgfusepath{stroke}%
\end{pgfscope}%
\begin{pgfscope}%
\pgfpathrectangle{\pgfqpoint{6.720588in}{1.750000in}}{\pgfqpoint{2.279412in}{2.004545in}}%
\pgfusepath{clip}%
\pgfsetbuttcap%
\pgfsetroundjoin%
\pgfsetlinewidth{0.936083pt}%
\definecolor{currentstroke}{rgb}{0.179019,0.433756,0.557430}%
\pgfsetstrokecolor{currentstroke}%
\pgfsetdash{}{0pt}%
\pgfpathmoveto{\pgfqpoint{7.296083in}{2.887593in}}%
\pgfpathlineto{\pgfqpoint{7.343122in}{2.872504in}}%
\pgfusepath{stroke}%
\end{pgfscope}%
\begin{pgfscope}%
\pgfpathrectangle{\pgfqpoint{6.720588in}{1.750000in}}{\pgfqpoint{2.279412in}{2.004545in}}%
\pgfusepath{clip}%
\pgfsetbuttcap%
\pgfsetroundjoin%
\pgfsetlinewidth{0.864030pt}%
\definecolor{currentstroke}{rgb}{0.195860,0.395433,0.555276}%
\pgfsetstrokecolor{currentstroke}%
\pgfsetdash{}{0pt}%
\pgfpathmoveto{\pgfqpoint{7.343122in}{2.872504in}}%
\pgfpathlineto{\pgfqpoint{7.388368in}{2.853695in}}%
\pgfusepath{stroke}%
\end{pgfscope}%
\begin{pgfscope}%
\pgfpathrectangle{\pgfqpoint{6.720588in}{1.750000in}}{\pgfqpoint{2.279412in}{2.004545in}}%
\pgfusepath{clip}%
\pgfsetbuttcap%
\pgfsetroundjoin%
\pgfsetlinewidth{0.664473pt}%
\definecolor{currentstroke}{rgb}{0.252194,0.269783,0.531579}%
\pgfsetstrokecolor{currentstroke}%
\pgfsetdash{}{0pt}%
\pgfpathmoveto{\pgfqpoint{7.388368in}{2.853695in}}%
\pgfpathlineto{\pgfqpoint{7.431229in}{2.831423in}}%
\pgfusepath{stroke}%
\end{pgfscope}%
\begin{pgfscope}%
\pgfpathrectangle{\pgfqpoint{6.720588in}{1.750000in}}{\pgfqpoint{2.279412in}{2.004545in}}%
\pgfusepath{clip}%
\pgfsetbuttcap%
\pgfsetroundjoin%
\pgfsetlinewidth{0.511082pt}%
\definecolor{currentstroke}{rgb}{0.280255,0.165693,0.476498}%
\pgfsetstrokecolor{currentstroke}%
\pgfsetdash{}{0pt}%
\pgfpathmoveto{\pgfqpoint{7.431229in}{2.831423in}}%
\pgfpathlineto{\pgfqpoint{7.431229in}{2.831423in}}%
\pgfusepath{stroke}%
\end{pgfscope}%
\begin{pgfscope}%
\pgfpathrectangle{\pgfqpoint{6.720588in}{1.750000in}}{\pgfqpoint{2.279412in}{2.004545in}}%
\pgfusepath{clip}%
\pgfsetbuttcap%
\pgfsetroundjoin%
\pgfsetlinewidth{0.511082pt}%
\definecolor{currentstroke}{rgb}{0.280255,0.165693,0.476498}%
\pgfsetstrokecolor{currentstroke}%
\pgfsetdash{}{0pt}%
\pgfpathmoveto{\pgfqpoint{7.431229in}{2.831423in}}%
\pgfpathlineto{\pgfqpoint{7.445235in}{2.819871in}}%
\pgfusepath{stroke}%
\end{pgfscope}%
\begin{pgfscope}%
\pgfpathrectangle{\pgfqpoint{6.720588in}{1.750000in}}{\pgfqpoint{2.279412in}{2.004545in}}%
\pgfusepath{clip}%
\pgfsetbuttcap%
\pgfsetroundjoin%
\pgfsetlinewidth{0.523688pt}%
\definecolor{currentstroke}{rgb}{0.278826,0.175490,0.483397}%
\pgfsetstrokecolor{currentstroke}%
\pgfsetdash{}{0pt}%
\pgfpathmoveto{\pgfqpoint{7.445235in}{2.819871in}}%
\pgfpathlineto{\pgfqpoint{7.455090in}{2.803297in}}%
\pgfusepath{stroke}%
\end{pgfscope}%
\begin{pgfscope}%
\pgfpathrectangle{\pgfqpoint{6.720588in}{1.750000in}}{\pgfqpoint{2.279412in}{2.004545in}}%
\pgfusepath{clip}%
\pgfsetbuttcap%
\pgfsetroundjoin%
\pgfsetlinewidth{0.490055pt}%
\definecolor{currentstroke}{rgb}{0.281887,0.150881,0.465405}%
\pgfsetstrokecolor{currentstroke}%
\pgfsetdash{}{0pt}%
\pgfpathmoveto{\pgfqpoint{7.455090in}{2.803297in}}%
\pgfpathlineto{\pgfqpoint{7.455090in}{2.803297in}}%
\pgfusepath{stroke}%
\end{pgfscope}%
\begin{pgfscope}%
\pgfpathrectangle{\pgfqpoint{6.720588in}{1.750000in}}{\pgfqpoint{2.279412in}{2.004545in}}%
\pgfusepath{clip}%
\pgfsetbuttcap%
\pgfsetroundjoin%
\pgfsetlinewidth{0.490055pt}%
\definecolor{currentstroke}{rgb}{0.281887,0.150881,0.465405}%
\pgfsetstrokecolor{currentstroke}%
\pgfsetdash{}{0pt}%
\pgfpathmoveto{\pgfqpoint{7.455090in}{2.803297in}}%
\pgfpathlineto{\pgfqpoint{7.458467in}{2.789544in}}%
\pgfusepath{stroke}%
\end{pgfscope}%
\begin{pgfscope}%
\pgfpathrectangle{\pgfqpoint{6.720588in}{1.750000in}}{\pgfqpoint{2.279412in}{2.004545in}}%
\pgfusepath{clip}%
\pgfsetbuttcap%
\pgfsetroundjoin%
\pgfsetlinewidth{0.479145pt}%
\definecolor{currentstroke}{rgb}{0.282623,0.140926,0.457517}%
\pgfsetstrokecolor{currentstroke}%
\pgfsetdash{}{0pt}%
\pgfpathmoveto{\pgfqpoint{7.458467in}{2.789544in}}%
\pgfpathlineto{\pgfqpoint{7.458798in}{2.775868in}}%
\pgfusepath{stroke}%
\end{pgfscope}%
\begin{pgfscope}%
\pgfpathrectangle{\pgfqpoint{6.720588in}{1.750000in}}{\pgfqpoint{2.279412in}{2.004545in}}%
\pgfusepath{clip}%
\pgfsetbuttcap%
\pgfsetroundjoin%
\pgfsetlinewidth{0.469090pt}%
\definecolor{currentstroke}{rgb}{0.282884,0.135920,0.453427}%
\pgfsetstrokecolor{currentstroke}%
\pgfsetdash{}{0pt}%
\pgfpathmoveto{\pgfqpoint{7.458798in}{2.775868in}}%
\pgfpathlineto{\pgfqpoint{7.458798in}{2.775868in}}%
\pgfusepath{stroke}%
\end{pgfscope}%
\begin{pgfscope}%
\pgfpathrectangle{\pgfqpoint{6.720588in}{1.750000in}}{\pgfqpoint{2.279412in}{2.004545in}}%
\pgfusepath{clip}%
\pgfsetbuttcap%
\pgfsetroundjoin%
\pgfsetlinewidth{0.874369pt}%
\definecolor{currentstroke}{rgb}{0.194100,0.399323,0.555565}%
\pgfsetstrokecolor{currentstroke}%
\pgfsetdash{}{0pt}%
\pgfpathmoveto{\pgfqpoint{7.299860in}{2.558216in}}%
\pgfpathlineto{\pgfqpoint{7.347375in}{2.571846in}}%
\pgfusepath{stroke}%
\end{pgfscope}%
\begin{pgfscope}%
\pgfpathrectangle{\pgfqpoint{6.720588in}{1.750000in}}{\pgfqpoint{2.279412in}{2.004545in}}%
\pgfusepath{clip}%
\pgfsetbuttcap%
\pgfsetroundjoin%
\pgfsetlinewidth{0.784771pt}%
\definecolor{currentstroke}{rgb}{0.218130,0.347432,0.550038}%
\pgfsetstrokecolor{currentstroke}%
\pgfsetdash{}{0pt}%
\pgfpathmoveto{\pgfqpoint{7.347375in}{2.571846in}}%
\pgfpathlineto{\pgfqpoint{7.392362in}{2.590386in}}%
\pgfusepath{stroke}%
\end{pgfscope}%
\begin{pgfscope}%
\pgfpathrectangle{\pgfqpoint{6.720588in}{1.750000in}}{\pgfqpoint{2.279412in}{2.004545in}}%
\pgfusepath{clip}%
\pgfsetbuttcap%
\pgfsetroundjoin%
\pgfsetlinewidth{0.704424pt}%
\definecolor{currentstroke}{rgb}{0.241237,0.296485,0.539709}%
\pgfsetstrokecolor{currentstroke}%
\pgfsetdash{}{0pt}%
\pgfpathmoveto{\pgfqpoint{7.392362in}{2.590386in}}%
\pgfpathlineto{\pgfqpoint{7.392362in}{2.590386in}}%
\pgfusepath{stroke}%
\end{pgfscope}%
\begin{pgfscope}%
\pgfpathrectangle{\pgfqpoint{6.720588in}{1.750000in}}{\pgfqpoint{2.279412in}{2.004545in}}%
\pgfusepath{clip}%
\pgfsetbuttcap%
\pgfsetroundjoin%
\pgfsetlinewidth{0.704424pt}%
\definecolor{currentstroke}{rgb}{0.241237,0.296485,0.539709}%
\pgfsetstrokecolor{currentstroke}%
\pgfsetdash{}{0pt}%
\pgfpathmoveto{\pgfqpoint{7.392362in}{2.590386in}}%
\pgfpathlineto{\pgfqpoint{7.414502in}{2.605198in}}%
\pgfusepath{stroke}%
\end{pgfscope}%
\begin{pgfscope}%
\pgfpathrectangle{\pgfqpoint{6.720588in}{1.750000in}}{\pgfqpoint{2.279412in}{2.004545in}}%
\pgfusepath{clip}%
\pgfsetbuttcap%
\pgfsetroundjoin%
\pgfsetlinewidth{0.667022pt}%
\definecolor{currentstroke}{rgb}{0.250425,0.274290,0.533103}%
\pgfsetstrokecolor{currentstroke}%
\pgfsetdash{}{0pt}%
\pgfpathmoveto{\pgfqpoint{7.414502in}{2.605198in}}%
\pgfpathlineto{\pgfqpoint{7.439143in}{2.625248in}}%
\pgfusepath{stroke}%
\end{pgfscope}%
\begin{pgfscope}%
\pgfpathrectangle{\pgfqpoint{6.720588in}{1.750000in}}{\pgfqpoint{2.279412in}{2.004545in}}%
\pgfusepath{clip}%
\pgfsetbuttcap%
\pgfsetroundjoin%
\pgfsetlinewidth{0.631666pt}%
\definecolor{currentstroke}{rgb}{0.258965,0.251537,0.524736}%
\pgfsetstrokecolor{currentstroke}%
\pgfsetdash{}{0pt}%
\pgfpathmoveto{\pgfqpoint{7.439143in}{2.625248in}}%
\pgfpathlineto{\pgfqpoint{7.439143in}{2.625248in}}%
\pgfusepath{stroke}%
\end{pgfscope}%
\begin{pgfscope}%
\pgfpathrectangle{\pgfqpoint{6.720588in}{1.750000in}}{\pgfqpoint{2.279412in}{2.004545in}}%
\pgfusepath{clip}%
\pgfsetbuttcap%
\pgfsetroundjoin%
\pgfsetlinewidth{0.631666pt}%
\definecolor{currentstroke}{rgb}{0.258965,0.251537,0.524736}%
\pgfsetstrokecolor{currentstroke}%
\pgfsetdash{}{0pt}%
\pgfpathmoveto{\pgfqpoint{7.439143in}{2.625248in}}%
\pgfpathlineto{\pgfqpoint{7.453590in}{2.641912in}}%
\pgfusepath{stroke}%
\end{pgfscope}%
\begin{pgfscope}%
\pgfpathrectangle{\pgfqpoint{6.720588in}{1.750000in}}{\pgfqpoint{2.279412in}{2.004545in}}%
\pgfusepath{clip}%
\pgfsetbuttcap%
\pgfsetroundjoin%
\pgfsetlinewidth{0.600587pt}%
\definecolor{currentstroke}{rgb}{0.266580,0.228262,0.514349}%
\pgfsetstrokecolor{currentstroke}%
\pgfsetdash{}{0pt}%
\pgfpathmoveto{\pgfqpoint{7.453590in}{2.641912in}}%
\pgfpathlineto{\pgfqpoint{7.453590in}{2.641912in}}%
\pgfusepath{stroke}%
\end{pgfscope}%
\begin{pgfscope}%
\pgfpathrectangle{\pgfqpoint{6.720588in}{1.750000in}}{\pgfqpoint{2.279412in}{2.004545in}}%
\pgfusepath{clip}%
\pgfsetbuttcap%
\pgfsetroundjoin%
\pgfsetlinewidth{0.600587pt}%
\definecolor{currentstroke}{rgb}{0.266580,0.228262,0.514349}%
\pgfsetstrokecolor{currentstroke}%
\pgfsetdash{}{0pt}%
\pgfpathmoveto{\pgfqpoint{7.453590in}{2.641912in}}%
\pgfpathlineto{\pgfqpoint{7.457390in}{2.659299in}}%
\pgfusepath{stroke}%
\end{pgfscope}%
\begin{pgfscope}%
\pgfpathrectangle{\pgfqpoint{6.720588in}{1.750000in}}{\pgfqpoint{2.279412in}{2.004545in}}%
\pgfusepath{clip}%
\pgfsetbuttcap%
\pgfsetroundjoin%
\pgfsetlinewidth{0.515682pt}%
\definecolor{currentstroke}{rgb}{0.279574,0.170599,0.479997}%
\pgfsetstrokecolor{currentstroke}%
\pgfsetdash{}{0pt}%
\pgfpathmoveto{\pgfqpoint{7.457390in}{2.659299in}}%
\pgfpathlineto{\pgfqpoint{7.452367in}{2.676387in}}%
\pgfusepath{stroke}%
\end{pgfscope}%
\begin{pgfscope}%
\pgfpathrectangle{\pgfqpoint{6.720588in}{1.750000in}}{\pgfqpoint{2.279412in}{2.004545in}}%
\pgfusepath{clip}%
\pgfsetbuttcap%
\pgfsetroundjoin%
\pgfsetlinewidth{0.621999pt}%
\definecolor{currentstroke}{rgb}{0.262138,0.242286,0.520837}%
\pgfsetstrokecolor{currentstroke}%
\pgfsetdash{}{0pt}%
\pgfpathmoveto{\pgfqpoint{7.398667in}{2.436525in}}%
\pgfpathlineto{\pgfqpoint{7.442987in}{2.456176in}}%
\pgfusepath{stroke}%
\end{pgfscope}%
\begin{pgfscope}%
\pgfpathrectangle{\pgfqpoint{6.720588in}{1.750000in}}{\pgfqpoint{2.279412in}{2.004545in}}%
\pgfusepath{clip}%
\pgfsetbuttcap%
\pgfsetroundjoin%
\pgfsetlinewidth{0.628394pt}%
\definecolor{currentstroke}{rgb}{0.260571,0.246922,0.522828}%
\pgfsetstrokecolor{currentstroke}%
\pgfsetdash{}{0pt}%
\pgfpathmoveto{\pgfqpoint{7.442987in}{2.456176in}}%
\pgfpathlineto{\pgfqpoint{7.480582in}{2.483551in}}%
\pgfusepath{stroke}%
\end{pgfscope}%
\begin{pgfscope}%
\pgfpathrectangle{\pgfqpoint{6.720588in}{1.750000in}}{\pgfqpoint{2.279412in}{2.004545in}}%
\pgfusepath{clip}%
\pgfsetbuttcap%
\pgfsetroundjoin%
\pgfsetlinewidth{0.574882pt}%
\definecolor{currentstroke}{rgb}{0.271828,0.209303,0.504434}%
\pgfsetstrokecolor{currentstroke}%
\pgfsetdash{}{0pt}%
\pgfpathmoveto{\pgfqpoint{7.480582in}{2.483551in}}%
\pgfpathlineto{\pgfqpoint{7.480582in}{2.483551in}}%
\pgfusepath{stroke}%
\end{pgfscope}%
\begin{pgfscope}%
\pgfpathrectangle{\pgfqpoint{6.720588in}{1.750000in}}{\pgfqpoint{2.279412in}{2.004545in}}%
\pgfusepath{clip}%
\pgfsetbuttcap%
\pgfsetroundjoin%
\pgfsetlinewidth{0.574882pt}%
\definecolor{currentstroke}{rgb}{0.271828,0.209303,0.504434}%
\pgfsetstrokecolor{currentstroke}%
\pgfsetdash{}{0pt}%
\pgfpathmoveto{\pgfqpoint{7.480582in}{2.483551in}}%
\pgfpathlineto{\pgfqpoint{7.498690in}{2.506417in}}%
\pgfusepath{stroke}%
\end{pgfscope}%
\begin{pgfscope}%
\pgfpathrectangle{\pgfqpoint{6.720588in}{1.750000in}}{\pgfqpoint{2.279412in}{2.004545in}}%
\pgfusepath{clip}%
\pgfsetbuttcap%
\pgfsetroundjoin%
\pgfsetlinewidth{0.667721pt}%
\definecolor{currentstroke}{rgb}{0.250425,0.274290,0.533103}%
\pgfsetstrokecolor{currentstroke}%
\pgfsetdash{}{0pt}%
\pgfpathmoveto{\pgfqpoint{7.498690in}{2.506417in}}%
\pgfpathlineto{\pgfqpoint{7.507139in}{2.530272in}}%
\pgfusepath{stroke}%
\end{pgfscope}%
\begin{pgfscope}%
\pgfpathrectangle{\pgfqpoint{6.720588in}{1.750000in}}{\pgfqpoint{2.279412in}{2.004545in}}%
\pgfusepath{clip}%
\pgfsetbuttcap%
\pgfsetroundjoin%
\pgfsetlinewidth{0.612854pt}%
\definecolor{currentstroke}{rgb}{0.263663,0.237631,0.518762}%
\pgfsetstrokecolor{currentstroke}%
\pgfsetdash{}{0pt}%
\pgfpathmoveto{\pgfqpoint{7.507139in}{2.530272in}}%
\pgfpathlineto{\pgfqpoint{7.510855in}{2.560200in}}%
\pgfusepath{stroke}%
\end{pgfscope}%
\begin{pgfscope}%
\pgfpathrectangle{\pgfqpoint{6.720588in}{1.750000in}}{\pgfqpoint{2.279412in}{2.004545in}}%
\pgfusepath{clip}%
\pgfsetbuttcap%
\pgfsetroundjoin%
\pgfsetlinewidth{0.626117pt}%
\definecolor{currentstroke}{rgb}{0.260571,0.246922,0.522828}%
\pgfsetstrokecolor{currentstroke}%
\pgfsetdash{}{0pt}%
\pgfpathmoveto{\pgfqpoint{7.510855in}{2.560200in}}%
\pgfpathlineto{\pgfqpoint{7.504717in}{2.590762in}}%
\pgfusepath{stroke}%
\end{pgfscope}%
\begin{pgfscope}%
\pgfpathrectangle{\pgfqpoint{6.720588in}{1.750000in}}{\pgfqpoint{2.279412in}{2.004545in}}%
\pgfusepath{clip}%
\pgfsetbuttcap%
\pgfsetroundjoin%
\pgfsetlinewidth{0.570805pt}%
\definecolor{currentstroke}{rgb}{0.271828,0.209303,0.504434}%
\pgfsetstrokecolor{currentstroke}%
\pgfsetdash{}{0pt}%
\pgfpathmoveto{\pgfqpoint{7.405053in}{2.372337in}}%
\pgfpathlineto{\pgfqpoint{7.449959in}{2.391418in}}%
\pgfusepath{stroke}%
\end{pgfscope}%
\begin{pgfscope}%
\pgfpathrectangle{\pgfqpoint{6.720588in}{1.750000in}}{\pgfqpoint{2.279412in}{2.004545in}}%
\pgfusepath{clip}%
\pgfsetbuttcap%
\pgfsetroundjoin%
\pgfsetlinewidth{0.638772pt}%
\definecolor{currentstroke}{rgb}{0.257322,0.256130,0.526563}%
\pgfsetstrokecolor{currentstroke}%
\pgfsetdash{}{0pt}%
\pgfpathmoveto{\pgfqpoint{7.449959in}{2.391418in}}%
\pgfpathlineto{\pgfqpoint{7.492290in}{2.414568in}}%
\pgfusepath{stroke}%
\end{pgfscope}%
\begin{pgfscope}%
\pgfpathrectangle{\pgfqpoint{6.720588in}{1.750000in}}{\pgfqpoint{2.279412in}{2.004545in}}%
\pgfusepath{clip}%
\pgfsetbuttcap%
\pgfsetroundjoin%
\pgfsetlinewidth{0.551090pt}%
\definecolor{currentstroke}{rgb}{0.275191,0.194905,0.496005}%
\pgfsetstrokecolor{currentstroke}%
\pgfsetdash{}{0pt}%
\pgfpathmoveto{\pgfqpoint{7.492290in}{2.414568in}}%
\pgfpathlineto{\pgfqpoint{7.492290in}{2.414568in}}%
\pgfusepath{stroke}%
\end{pgfscope}%
\begin{pgfscope}%
\pgfpathrectangle{\pgfqpoint{6.720588in}{1.750000in}}{\pgfqpoint{2.279412in}{2.004545in}}%
\pgfusepath{clip}%
\pgfsetbuttcap%
\pgfsetroundjoin%
\pgfsetlinewidth{0.551090pt}%
\definecolor{currentstroke}{rgb}{0.275191,0.194905,0.496005}%
\pgfsetstrokecolor{currentstroke}%
\pgfsetdash{}{0pt}%
\pgfpathmoveto{\pgfqpoint{7.492290in}{2.414568in}}%
\pgfpathlineto{\pgfqpoint{7.522067in}{2.435565in}}%
\pgfusepath{stroke}%
\end{pgfscope}%
\begin{pgfscope}%
\pgfpathrectangle{\pgfqpoint{6.720588in}{1.750000in}}{\pgfqpoint{2.279412in}{2.004545in}}%
\pgfusepath{clip}%
\pgfsetbuttcap%
\pgfsetroundjoin%
\pgfsetlinewidth{0.494696pt}%
\definecolor{currentstroke}{rgb}{0.281412,0.155834,0.469201}%
\pgfsetstrokecolor{currentstroke}%
\pgfsetdash{}{0pt}%
\pgfpathmoveto{\pgfqpoint{7.522067in}{2.435565in}}%
\pgfpathlineto{\pgfqpoint{7.522067in}{2.435565in}}%
\pgfusepath{stroke}%
\end{pgfscope}%
\begin{pgfscope}%
\pgfpathrectangle{\pgfqpoint{6.720588in}{1.750000in}}{\pgfqpoint{2.279412in}{2.004545in}}%
\pgfusepath{clip}%
\pgfsetbuttcap%
\pgfsetroundjoin%
\pgfsetlinewidth{0.494696pt}%
\definecolor{currentstroke}{rgb}{0.281412,0.155834,0.469201}%
\pgfsetstrokecolor{currentstroke}%
\pgfsetdash{}{0pt}%
\pgfpathmoveto{\pgfqpoint{7.522067in}{2.435565in}}%
\pgfpathlineto{\pgfqpoint{7.539731in}{2.455824in}}%
\pgfusepath{stroke}%
\end{pgfscope}%
\begin{pgfscope}%
\pgfpathrectangle{\pgfqpoint{6.720588in}{1.750000in}}{\pgfqpoint{2.279412in}{2.004545in}}%
\pgfusepath{clip}%
\pgfsetbuttcap%
\pgfsetroundjoin%
\pgfsetlinewidth{0.555094pt}%
\definecolor{currentstroke}{rgb}{0.274128,0.199721,0.498911}%
\pgfsetstrokecolor{currentstroke}%
\pgfsetdash{}{0pt}%
\pgfpathmoveto{\pgfqpoint{7.539731in}{2.455824in}}%
\pgfpathlineto{\pgfqpoint{7.548420in}{2.479495in}}%
\pgfusepath{stroke}%
\end{pgfscope}%
\begin{pgfscope}%
\pgfpathrectangle{\pgfqpoint{6.720588in}{1.750000in}}{\pgfqpoint{2.279412in}{2.004545in}}%
\pgfusepath{clip}%
\pgfsetbuttcap%
\pgfsetroundjoin%
\pgfsetlinewidth{0.630657pt}%
\definecolor{currentstroke}{rgb}{0.258965,0.251537,0.524736}%
\pgfsetstrokecolor{currentstroke}%
\pgfsetdash{}{0pt}%
\pgfpathmoveto{\pgfqpoint{7.548420in}{2.479495in}}%
\pgfpathlineto{\pgfqpoint{7.549195in}{2.502568in}}%
\pgfusepath{stroke}%
\end{pgfscope}%
\begin{pgfscope}%
\pgfpathrectangle{\pgfqpoint{6.720588in}{1.750000in}}{\pgfqpoint{2.279412in}{2.004545in}}%
\pgfusepath{clip}%
\pgfsetbuttcap%
\pgfsetroundjoin%
\pgfsetlinewidth{0.637273pt}%
\definecolor{currentstroke}{rgb}{0.258965,0.251537,0.524736}%
\pgfsetstrokecolor{currentstroke}%
\pgfsetdash{}{0pt}%
\pgfpathmoveto{\pgfqpoint{7.549195in}{2.502568in}}%
\pgfpathlineto{\pgfqpoint{7.541882in}{2.533203in}}%
\pgfusepath{stroke}%
\end{pgfscope}%
\begin{pgfscope}%
\pgfpathrectangle{\pgfqpoint{6.720588in}{1.750000in}}{\pgfqpoint{2.279412in}{2.004545in}}%
\pgfusepath{clip}%
\pgfsetbuttcap%
\pgfsetroundjoin%
\pgfsetlinewidth{0.556971pt}%
\definecolor{currentstroke}{rgb}{0.274128,0.199721,0.498911}%
\pgfsetstrokecolor{currentstroke}%
\pgfsetdash{}{0pt}%
\pgfpathmoveto{\pgfqpoint{7.470828in}{3.042106in}}%
\pgfpathlineto{\pgfqpoint{7.501251in}{3.022913in}}%
\pgfusepath{stroke}%
\end{pgfscope}%
\begin{pgfscope}%
\pgfpathrectangle{\pgfqpoint{6.720588in}{1.750000in}}{\pgfqpoint{2.279412in}{2.004545in}}%
\pgfusepath{clip}%
\pgfsetbuttcap%
\pgfsetroundjoin%
\pgfsetlinewidth{0.625852pt}%
\definecolor{currentstroke}{rgb}{0.260571,0.246922,0.522828}%
\pgfsetstrokecolor{currentstroke}%
\pgfsetdash{}{0pt}%
\pgfpathmoveto{\pgfqpoint{7.501251in}{3.022913in}}%
\pgfpathlineto{\pgfqpoint{7.501251in}{3.022913in}}%
\pgfusepath{stroke}%
\end{pgfscope}%
\begin{pgfscope}%
\pgfpathrectangle{\pgfqpoint{6.720588in}{1.750000in}}{\pgfqpoint{2.279412in}{2.004545in}}%
\pgfusepath{clip}%
\pgfsetbuttcap%
\pgfsetroundjoin%
\pgfsetlinewidth{0.625852pt}%
\definecolor{currentstroke}{rgb}{0.260571,0.246922,0.522828}%
\pgfsetstrokecolor{currentstroke}%
\pgfsetdash{}{0pt}%
\pgfpathmoveto{\pgfqpoint{7.501251in}{3.022913in}}%
\pgfpathlineto{\pgfqpoint{7.501251in}{3.022913in}}%
\pgfusepath{stroke}%
\end{pgfscope}%
\begin{pgfscope}%
\pgfpathrectangle{\pgfqpoint{6.720588in}{1.750000in}}{\pgfqpoint{2.279412in}{2.004545in}}%
\pgfusepath{clip}%
\pgfsetbuttcap%
\pgfsetroundjoin%
\pgfsetlinewidth{0.625852pt}%
\definecolor{currentstroke}{rgb}{0.260571,0.246922,0.522828}%
\pgfsetstrokecolor{currentstroke}%
\pgfsetdash{}{0pt}%
\pgfpathmoveto{\pgfqpoint{7.501251in}{3.022913in}}%
\pgfpathlineto{\pgfqpoint{7.522373in}{3.000319in}}%
\pgfusepath{stroke}%
\end{pgfscope}%
\begin{pgfscope}%
\pgfpathrectangle{\pgfqpoint{6.720588in}{1.750000in}}{\pgfqpoint{2.279412in}{2.004545in}}%
\pgfusepath{clip}%
\pgfsetbuttcap%
\pgfsetroundjoin%
\pgfsetlinewidth{0.596013pt}%
\definecolor{currentstroke}{rgb}{0.267968,0.223549,0.512008}%
\pgfsetstrokecolor{currentstroke}%
\pgfsetdash{}{0pt}%
\pgfpathmoveto{\pgfqpoint{7.522373in}{3.000319in}}%
\pgfpathlineto{\pgfqpoint{7.522373in}{3.000319in}}%
\pgfusepath{stroke}%
\end{pgfscope}%
\begin{pgfscope}%
\pgfpathrectangle{\pgfqpoint{6.720588in}{1.750000in}}{\pgfqpoint{2.279412in}{2.004545in}}%
\pgfusepath{clip}%
\pgfsetbuttcap%
\pgfsetroundjoin%
\pgfsetlinewidth{0.596013pt}%
\definecolor{currentstroke}{rgb}{0.267968,0.223549,0.512008}%
\pgfsetstrokecolor{currentstroke}%
\pgfsetdash{}{0pt}%
\pgfpathmoveto{\pgfqpoint{7.522373in}{3.000319in}}%
\pgfpathlineto{\pgfqpoint{7.533440in}{2.980310in}}%
\pgfusepath{stroke}%
\end{pgfscope}%
\begin{pgfscope}%
\pgfpathrectangle{\pgfqpoint{6.720588in}{1.750000in}}{\pgfqpoint{2.279412in}{2.004545in}}%
\pgfusepath{clip}%
\pgfsetbuttcap%
\pgfsetroundjoin%
\pgfsetlinewidth{0.613832pt}%
\definecolor{currentstroke}{rgb}{0.263663,0.237631,0.518762}%
\pgfsetstrokecolor{currentstroke}%
\pgfsetdash{}{0pt}%
\pgfpathmoveto{\pgfqpoint{7.533440in}{2.980310in}}%
\pgfpathlineto{\pgfqpoint{7.537833in}{2.958789in}}%
\pgfusepath{stroke}%
\end{pgfscope}%
\begin{pgfscope}%
\pgfpathrectangle{\pgfqpoint{6.720588in}{1.750000in}}{\pgfqpoint{2.279412in}{2.004545in}}%
\pgfusepath{clip}%
\pgfsetbuttcap%
\pgfsetroundjoin%
\pgfsetlinewidth{0.638816pt}%
\definecolor{currentstroke}{rgb}{0.257322,0.256130,0.526563}%
\pgfsetstrokecolor{currentstroke}%
\pgfsetdash{}{0pt}%
\pgfpathmoveto{\pgfqpoint{7.537833in}{2.958789in}}%
\pgfpathlineto{\pgfqpoint{7.535814in}{2.936450in}}%
\pgfusepath{stroke}%
\end{pgfscope}%
\begin{pgfscope}%
\pgfpathrectangle{\pgfqpoint{6.720588in}{1.750000in}}{\pgfqpoint{2.279412in}{2.004545in}}%
\pgfusepath{clip}%
\pgfsetbuttcap%
\pgfsetroundjoin%
\pgfsetlinewidth{0.641897pt}%
\definecolor{currentstroke}{rgb}{0.257322,0.256130,0.526563}%
\pgfsetstrokecolor{currentstroke}%
\pgfsetdash{}{0pt}%
\pgfpathmoveto{\pgfqpoint{7.535814in}{2.936450in}}%
\pgfpathlineto{\pgfqpoint{7.531354in}{2.910832in}}%
\pgfusepath{stroke}%
\end{pgfscope}%
\begin{pgfscope}%
\pgfpathrectangle{\pgfqpoint{6.720588in}{1.750000in}}{\pgfqpoint{2.279412in}{2.004545in}}%
\pgfusepath{clip}%
\pgfsetbuttcap%
\pgfsetroundjoin%
\pgfsetlinewidth{0.587531pt}%
\definecolor{currentstroke}{rgb}{0.269308,0.218818,0.509577}%
\pgfsetstrokecolor{currentstroke}%
\pgfsetdash{}{0pt}%
\pgfpathmoveto{\pgfqpoint{7.531354in}{2.910832in}}%
\pgfpathlineto{\pgfqpoint{7.531354in}{2.910832in}}%
\pgfusepath{stroke}%
\end{pgfscope}%
\begin{pgfscope}%
\pgfpathrectangle{\pgfqpoint{6.720588in}{1.750000in}}{\pgfqpoint{2.279412in}{2.004545in}}%
\pgfusepath{clip}%
\pgfsetbuttcap%
\pgfsetroundjoin%
\pgfsetlinewidth{0.587531pt}%
\definecolor{currentstroke}{rgb}{0.269308,0.218818,0.509577}%
\pgfsetstrokecolor{currentstroke}%
\pgfsetdash{}{0pt}%
\pgfpathmoveto{\pgfqpoint{7.531354in}{2.910832in}}%
\pgfpathlineto{\pgfqpoint{7.521548in}{2.884682in}}%
\pgfusepath{stroke}%
\end{pgfscope}%
\begin{pgfscope}%
\pgfpathrectangle{\pgfqpoint{6.720588in}{1.750000in}}{\pgfqpoint{2.279412in}{2.004545in}}%
\pgfusepath{clip}%
\pgfsetbuttcap%
\pgfsetroundjoin%
\pgfsetlinewidth{0.578335pt}%
\definecolor{currentstroke}{rgb}{0.270595,0.214069,0.507052}%
\pgfsetstrokecolor{currentstroke}%
\pgfsetdash{}{0pt}%
\pgfpathmoveto{\pgfqpoint{7.521548in}{2.884682in}}%
\pgfpathlineto{\pgfqpoint{7.510363in}{2.860787in}}%
\pgfusepath{stroke}%
\end{pgfscope}%
\begin{pgfscope}%
\pgfpathrectangle{\pgfqpoint{6.720588in}{1.750000in}}{\pgfqpoint{2.279412in}{2.004545in}}%
\pgfusepath{clip}%
\pgfsetbuttcap%
\pgfsetroundjoin%
\pgfsetlinewidth{0.530358pt}%
\definecolor{currentstroke}{rgb}{0.278012,0.180367,0.486697}%
\pgfsetstrokecolor{currentstroke}%
\pgfsetdash{}{0pt}%
\pgfpathmoveto{\pgfqpoint{7.510363in}{2.860787in}}%
\pgfpathlineto{\pgfqpoint{7.487307in}{2.823570in}}%
\pgfusepath{stroke}%
\end{pgfscope}%
\begin{pgfscope}%
\pgfpathrectangle{\pgfqpoint{6.720588in}{1.750000in}}{\pgfqpoint{2.279412in}{2.004545in}}%
\pgfusepath{clip}%
\pgfsetbuttcap%
\pgfsetroundjoin%
\pgfsetlinewidth{0.525488pt}%
\definecolor{currentstroke}{rgb}{0.278826,0.175490,0.483397}%
\pgfsetstrokecolor{currentstroke}%
\pgfsetdash{}{0pt}%
\pgfpathmoveto{\pgfqpoint{7.487307in}{2.823570in}}%
\pgfpathlineto{\pgfqpoint{7.487307in}{2.823570in}}%
\pgfusepath{stroke}%
\end{pgfscope}%
\begin{pgfscope}%
\pgfpathrectangle{\pgfqpoint{6.720588in}{1.750000in}}{\pgfqpoint{2.279412in}{2.004545in}}%
\pgfusepath{clip}%
\pgfsetroundcap%
\pgfsetroundjoin%
\pgfsetlinewidth{0.308875pt}%
\definecolor{currentstroke}{rgb}{0.268510,0.009605,0.335427}%
\pgfsetstrokecolor{currentstroke}%
\pgfsetdash{}{0pt}%
\pgfpathmoveto{\pgfqpoint{8.427829in}{1.852932in}}%
\pgfpathquadraticcurveto{\pgfqpoint{8.427767in}{1.852935in}}{\pgfqpoint{8.432478in}{1.852709in}}%
\pgfusepath{stroke}%
\end{pgfscope}%
\begin{pgfscope}%
\pgfpathrectangle{\pgfqpoint{6.720588in}{1.750000in}}{\pgfqpoint{2.279412in}{2.004545in}}%
\pgfusepath{clip}%
\pgfsetroundcap%
\pgfsetroundjoin%
\definecolor{currentfill}{rgb}{0.268510,0.009605,0.335427}%
\pgfsetfillcolor{currentfill}%
\pgfsetlinewidth{0.308875pt}%
\definecolor{currentstroke}{rgb}{0.268510,0.009605,0.335427}%
\pgfsetstrokecolor{currentstroke}%
\pgfsetdash{}{0pt}%
\pgfpathmoveto{\pgfqpoint{8.486644in}{1.822312in}}%
\pgfpathlineto{\pgfqpoint{8.432478in}{1.852709in}}%
\pgfpathlineto{\pgfqpoint{8.489296in}{1.877804in}}%
\pgfpathlineto{\pgfqpoint{8.486644in}{1.822312in}}%
\pgfpathlineto{\pgfqpoint{8.486644in}{1.822312in}}%
\pgfpathclose%
\pgfusepath{stroke,fill}%
\end{pgfscope}%
\begin{pgfscope}%
\pgfpathrectangle{\pgfqpoint{6.720588in}{1.750000in}}{\pgfqpoint{2.279412in}{2.004545in}}%
\pgfusepath{clip}%
\pgfsetroundcap%
\pgfsetroundjoin%
\pgfsetlinewidth{0.892541pt}%
\definecolor{currentstroke}{rgb}{0.188923,0.410910,0.556326}%
\pgfsetstrokecolor{currentstroke}%
\pgfsetdash{}{0pt}%
\pgfpathmoveto{\pgfqpoint{7.238779in}{2.743939in}}%
\pgfpathquadraticcurveto{\pgfqpoint{7.251306in}{2.743617in}}{\pgfqpoint{7.250029in}{2.743649in}}%
\pgfusepath{stroke}%
\end{pgfscope}%
\begin{pgfscope}%
\pgfpathrectangle{\pgfqpoint{6.720588in}{1.750000in}}{\pgfqpoint{2.279412in}{2.004545in}}%
\pgfusepath{clip}%
\pgfsetroundcap%
\pgfsetroundjoin%
\definecolor{currentfill}{rgb}{0.188923,0.410910,0.556326}%
\pgfsetfillcolor{currentfill}%
\pgfsetlinewidth{0.892541pt}%
\definecolor{currentstroke}{rgb}{0.188923,0.410910,0.556326}%
\pgfsetstrokecolor{currentstroke}%
\pgfsetdash{}{0pt}%
\pgfpathmoveto{\pgfqpoint{7.195206in}{2.772846in}}%
\pgfpathlineto{\pgfqpoint{7.250029in}{2.743649in}}%
\pgfpathlineto{\pgfqpoint{7.193778in}{2.717309in}}%
\pgfpathlineto{\pgfqpoint{7.195206in}{2.772846in}}%
\pgfpathlineto{\pgfqpoint{7.195206in}{2.772846in}}%
\pgfpathclose%
\pgfusepath{stroke,fill}%
\end{pgfscope}%
\begin{pgfscope}%
\pgfpathrectangle{\pgfqpoint{6.720588in}{1.750000in}}{\pgfqpoint{2.279412in}{2.004545in}}%
\pgfusepath{clip}%
\pgfsetroundcap%
\pgfsetroundjoin%
\pgfsetlinewidth{0.713054pt}%
\definecolor{currentstroke}{rgb}{0.237441,0.305202,0.541921}%
\pgfsetstrokecolor{currentstroke}%
\pgfsetdash{}{0pt}%
\pgfpathmoveto{\pgfqpoint{8.080706in}{2.579677in}}%
\pgfpathquadraticcurveto{\pgfqpoint{8.068187in}{2.580280in}}{\pgfqpoint{8.066687in}{2.580352in}}%
\pgfusepath{stroke}%
\end{pgfscope}%
\begin{pgfscope}%
\pgfpathrectangle{\pgfqpoint{6.720588in}{1.750000in}}{\pgfqpoint{2.279412in}{2.004545in}}%
\pgfusepath{clip}%
\pgfsetroundcap%
\pgfsetroundjoin%
\definecolor{currentfill}{rgb}{0.237441,0.305202,0.541921}%
\pgfsetfillcolor{currentfill}%
\pgfsetlinewidth{0.713054pt}%
\definecolor{currentstroke}{rgb}{0.237441,0.305202,0.541921}%
\pgfsetstrokecolor{currentstroke}%
\pgfsetdash{}{0pt}%
\pgfpathmoveto{\pgfqpoint{8.120841in}{2.549933in}}%
\pgfpathlineto{\pgfqpoint{8.066687in}{2.580352in}}%
\pgfpathlineto{\pgfqpoint{8.123515in}{2.605424in}}%
\pgfpathlineto{\pgfqpoint{8.120841in}{2.549933in}}%
\pgfpathlineto{\pgfqpoint{8.120841in}{2.549933in}}%
\pgfpathclose%
\pgfusepath{stroke,fill}%
\end{pgfscope}%
\begin{pgfscope}%
\pgfpathrectangle{\pgfqpoint{6.720588in}{1.750000in}}{\pgfqpoint{2.279412in}{2.004545in}}%
\pgfusepath{clip}%
\pgfsetroundcap%
\pgfsetroundjoin%
\pgfsetlinewidth{0.434315pt}%
\definecolor{currentstroke}{rgb}{0.283091,0.110553,0.431554}%
\pgfsetstrokecolor{currentstroke}%
\pgfsetdash{}{0pt}%
\pgfpathmoveto{\pgfqpoint{8.283150in}{2.614828in}}%
\pgfpathquadraticcurveto{\pgfqpoint{8.270614in}{2.615010in}}{\pgfqpoint{8.264796in}{2.615095in}}%
\pgfusepath{stroke}%
\end{pgfscope}%
\begin{pgfscope}%
\pgfpathrectangle{\pgfqpoint{6.720588in}{1.750000in}}{\pgfqpoint{2.279412in}{2.004545in}}%
\pgfusepath{clip}%
\pgfsetroundcap%
\pgfsetroundjoin%
\definecolor{currentfill}{rgb}{0.283091,0.110553,0.431554}%
\pgfsetfillcolor{currentfill}%
\pgfsetlinewidth{0.434315pt}%
\definecolor{currentstroke}{rgb}{0.283091,0.110553,0.431554}%
\pgfsetstrokecolor{currentstroke}%
\pgfsetdash{}{0pt}%
\pgfpathmoveto{\pgfqpoint{8.319942in}{2.586513in}}%
\pgfpathlineto{\pgfqpoint{8.264796in}{2.615095in}}%
\pgfpathlineto{\pgfqpoint{8.320750in}{2.642062in}}%
\pgfpathlineto{\pgfqpoint{8.319942in}{2.586513in}}%
\pgfpathlineto{\pgfqpoint{8.319942in}{2.586513in}}%
\pgfpathclose%
\pgfusepath{stroke,fill}%
\end{pgfscope}%
\begin{pgfscope}%
\pgfpathrectangle{\pgfqpoint{6.720588in}{1.750000in}}{\pgfqpoint{2.279412in}{2.004545in}}%
\pgfusepath{clip}%
\pgfsetroundcap%
\pgfsetroundjoin%
\pgfsetlinewidth{1.074970pt}%
\definecolor{currentstroke}{rgb}{0.149039,0.508051,0.557250}%
\pgfsetstrokecolor{currentstroke}%
\pgfsetdash{}{0pt}%
\pgfpathmoveto{\pgfqpoint{7.190001in}{2.710582in}}%
\pgfpathquadraticcurveto{\pgfqpoint{7.202530in}{2.710940in}}{\pgfqpoint{7.198436in}{2.710823in}}%
\pgfusepath{stroke}%
\end{pgfscope}%
\begin{pgfscope}%
\pgfpathrectangle{\pgfqpoint{6.720588in}{1.750000in}}{\pgfqpoint{2.279412in}{2.004545in}}%
\pgfusepath{clip}%
\pgfsetroundcap%
\pgfsetroundjoin%
\definecolor{currentfill}{rgb}{0.149039,0.508051,0.557250}%
\pgfsetfillcolor{currentfill}%
\pgfsetlinewidth{1.074970pt}%
\definecolor{currentstroke}{rgb}{0.149039,0.508051,0.557250}%
\pgfsetstrokecolor{currentstroke}%
\pgfsetdash{}{0pt}%
\pgfpathmoveto{\pgfqpoint{7.142109in}{2.737001in}}%
\pgfpathlineto{\pgfqpoint{7.198436in}{2.710823in}}%
\pgfpathlineto{\pgfqpoint{7.143697in}{2.681469in}}%
\pgfpathlineto{\pgfqpoint{7.142109in}{2.737001in}}%
\pgfpathlineto{\pgfqpoint{7.142109in}{2.737001in}}%
\pgfpathclose%
\pgfusepath{stroke,fill}%
\end{pgfscope}%
\begin{pgfscope}%
\pgfpathrectangle{\pgfqpoint{6.720588in}{1.750000in}}{\pgfqpoint{2.279412in}{2.004545in}}%
\pgfusepath{clip}%
\pgfsetroundcap%
\pgfsetroundjoin%
\pgfsetlinewidth{0.340275pt}%
\definecolor{currentstroke}{rgb}{0.273809,0.031497,0.358853}%
\pgfsetstrokecolor{currentstroke}%
\pgfsetdash{}{0pt}%
\pgfpathmoveto{\pgfqpoint{8.323257in}{2.078301in}}%
\pgfpathquadraticcurveto{\pgfqpoint{8.310743in}{2.078562in}}{\pgfqpoint{8.303493in}{2.078714in}}%
\pgfusepath{stroke}%
\end{pgfscope}%
\begin{pgfscope}%
\pgfpathrectangle{\pgfqpoint{6.720588in}{1.750000in}}{\pgfqpoint{2.279412in}{2.004545in}}%
\pgfusepath{clip}%
\pgfsetroundcap%
\pgfsetroundjoin%
\definecolor{currentfill}{rgb}{0.273809,0.031497,0.358853}%
\pgfsetfillcolor{currentfill}%
\pgfsetlinewidth{0.340275pt}%
\definecolor{currentstroke}{rgb}{0.273809,0.031497,0.358853}%
\pgfsetstrokecolor{currentstroke}%
\pgfsetdash{}{0pt}%
\pgfpathmoveto{\pgfqpoint{8.358456in}{2.049780in}}%
\pgfpathlineto{\pgfqpoint{8.303493in}{2.078714in}}%
\pgfpathlineto{\pgfqpoint{8.359618in}{2.105324in}}%
\pgfpathlineto{\pgfqpoint{8.358456in}{2.049780in}}%
\pgfpathlineto{\pgfqpoint{8.358456in}{2.049780in}}%
\pgfpathclose%
\pgfusepath{stroke,fill}%
\end{pgfscope}%
\begin{pgfscope}%
\pgfpathrectangle{\pgfqpoint{6.720588in}{1.750000in}}{\pgfqpoint{2.279412in}{2.004545in}}%
\pgfusepath{clip}%
\pgfsetroundcap%
\pgfsetroundjoin%
\pgfsetlinewidth{0.755468pt}%
\definecolor{currentstroke}{rgb}{0.225863,0.330805,0.547314}%
\pgfsetstrokecolor{currentstroke}%
\pgfsetdash{}{0pt}%
\pgfpathmoveto{\pgfqpoint{8.107822in}{2.703477in}}%
\pgfpathquadraticcurveto{\pgfqpoint{8.095284in}{2.703465in}}{\pgfqpoint{8.094433in}{2.703464in}}%
\pgfusepath{stroke}%
\end{pgfscope}%
\begin{pgfscope}%
\pgfpathrectangle{\pgfqpoint{6.720588in}{1.750000in}}{\pgfqpoint{2.279412in}{2.004545in}}%
\pgfusepath{clip}%
\pgfsetroundcap%
\pgfsetroundjoin%
\definecolor{currentfill}{rgb}{0.225863,0.330805,0.547314}%
\pgfsetfillcolor{currentfill}%
\pgfsetlinewidth{0.755468pt}%
\definecolor{currentstroke}{rgb}{0.225863,0.330805,0.547314}%
\pgfsetstrokecolor{currentstroke}%
\pgfsetdash{}{0pt}%
\pgfpathmoveto{\pgfqpoint{8.150015in}{2.675739in}}%
\pgfpathlineto{\pgfqpoint{8.094433in}{2.703464in}}%
\pgfpathlineto{\pgfqpoint{8.149963in}{2.731294in}}%
\pgfpathlineto{\pgfqpoint{8.150015in}{2.675739in}}%
\pgfpathlineto{\pgfqpoint{8.150015in}{2.675739in}}%
\pgfpathclose%
\pgfusepath{stroke,fill}%
\end{pgfscope}%
\begin{pgfscope}%
\pgfpathrectangle{\pgfqpoint{6.720588in}{1.750000in}}{\pgfqpoint{2.279412in}{2.004545in}}%
\pgfusepath{clip}%
\pgfsetroundcap%
\pgfsetroundjoin%
\pgfsetlinewidth{0.778730pt}%
\definecolor{currentstroke}{rgb}{0.220057,0.343307,0.549413}%
\pgfsetstrokecolor{currentstroke}%
\pgfsetdash{}{0pt}%
\pgfpathmoveto{\pgfqpoint{8.078148in}{2.748026in}}%
\pgfpathquadraticcurveto{\pgfqpoint{8.065611in}{2.747909in}}{\pgfqpoint{8.065120in}{2.747904in}}%
\pgfusepath{stroke}%
\end{pgfscope}%
\begin{pgfscope}%
\pgfpathrectangle{\pgfqpoint{6.720588in}{1.750000in}}{\pgfqpoint{2.279412in}{2.004545in}}%
\pgfusepath{clip}%
\pgfsetroundcap%
\pgfsetroundjoin%
\definecolor{currentfill}{rgb}{0.220057,0.343307,0.549413}%
\pgfsetfillcolor{currentfill}%
\pgfsetlinewidth{0.778730pt}%
\definecolor{currentstroke}{rgb}{0.220057,0.343307,0.549413}%
\pgfsetstrokecolor{currentstroke}%
\pgfsetdash{}{0pt}%
\pgfpathmoveto{\pgfqpoint{8.120933in}{2.720647in}}%
\pgfpathlineto{\pgfqpoint{8.065120in}{2.747904in}}%
\pgfpathlineto{\pgfqpoint{8.120414in}{2.776200in}}%
\pgfpathlineto{\pgfqpoint{8.120933in}{2.720647in}}%
\pgfpathlineto{\pgfqpoint{8.120933in}{2.720647in}}%
\pgfpathclose%
\pgfusepath{stroke,fill}%
\end{pgfscope}%
\begin{pgfscope}%
\pgfpathrectangle{\pgfqpoint{6.720588in}{1.750000in}}{\pgfqpoint{2.279412in}{2.004545in}}%
\pgfusepath{clip}%
\pgfsetroundcap%
\pgfsetroundjoin%
\pgfsetlinewidth{0.704879pt}%
\definecolor{currentstroke}{rgb}{0.241237,0.296485,0.539709}%
\pgfsetstrokecolor{currentstroke}%
\pgfsetdash{}{0pt}%
\pgfpathmoveto{\pgfqpoint{7.989456in}{2.922487in}}%
\pgfpathquadraticcurveto{\pgfqpoint{7.976972in}{2.921480in}}{\pgfqpoint{7.975357in}{2.921350in}}%
\pgfusepath{stroke}%
\end{pgfscope}%
\begin{pgfscope}%
\pgfpathrectangle{\pgfqpoint{6.720588in}{1.750000in}}{\pgfqpoint{2.279412in}{2.004545in}}%
\pgfusepath{clip}%
\pgfsetroundcap%
\pgfsetroundjoin%
\definecolor{currentfill}{rgb}{0.241237,0.296485,0.539709}%
\pgfsetfillcolor{currentfill}%
\pgfsetlinewidth{0.704879pt}%
\definecolor{currentstroke}{rgb}{0.241237,0.296485,0.539709}%
\pgfsetstrokecolor{currentstroke}%
\pgfsetdash{}{0pt}%
\pgfpathmoveto{\pgfqpoint{8.032966in}{2.898129in}}%
\pgfpathlineto{\pgfqpoint{7.975357in}{2.921350in}}%
\pgfpathlineto{\pgfqpoint{8.028499in}{2.953505in}}%
\pgfpathlineto{\pgfqpoint{8.032966in}{2.898129in}}%
\pgfpathlineto{\pgfqpoint{8.032966in}{2.898129in}}%
\pgfpathclose%
\pgfusepath{stroke,fill}%
\end{pgfscope}%
\begin{pgfscope}%
\pgfpathrectangle{\pgfqpoint{6.720588in}{1.750000in}}{\pgfqpoint{2.279412in}{2.004545in}}%
\pgfusepath{clip}%
\pgfsetroundcap%
\pgfsetroundjoin%
\pgfsetlinewidth{0.429444pt}%
\definecolor{currentstroke}{rgb}{0.282910,0.105393,0.426902}%
\pgfsetstrokecolor{currentstroke}%
\pgfsetdash{}{0pt}%
\pgfpathmoveto{\pgfqpoint{7.707634in}{2.348865in}}%
\pgfpathquadraticcurveto{\pgfqpoint{7.702744in}{2.358770in}}{\pgfqpoint{7.700795in}{2.362719in}}%
\pgfusepath{stroke}%
\end{pgfscope}%
\begin{pgfscope}%
\pgfpathrectangle{\pgfqpoint{6.720588in}{1.750000in}}{\pgfqpoint{2.279412in}{2.004545in}}%
\pgfusepath{clip}%
\pgfsetroundcap%
\pgfsetroundjoin%
\definecolor{currentfill}{rgb}{0.282910,0.105393,0.426902}%
\pgfsetfillcolor{currentfill}%
\pgfsetlinewidth{0.429444pt}%
\definecolor{currentstroke}{rgb}{0.282910,0.105393,0.426902}%
\pgfsetstrokecolor{currentstroke}%
\pgfsetdash{}{0pt}%
\pgfpathmoveto{\pgfqpoint{7.700480in}{2.300606in}}%
\pgfpathlineto{\pgfqpoint{7.700795in}{2.362719in}}%
\pgfpathlineto{\pgfqpoint{7.750296in}{2.325200in}}%
\pgfpathlineto{\pgfqpoint{7.700480in}{2.300606in}}%
\pgfpathlineto{\pgfqpoint{7.700480in}{2.300606in}}%
\pgfpathclose%
\pgfusepath{stroke,fill}%
\end{pgfscope}%
\begin{pgfscope}%
\pgfpathrectangle{\pgfqpoint{6.720588in}{1.750000in}}{\pgfqpoint{2.279412in}{2.004545in}}%
\pgfusepath{clip}%
\pgfsetroundcap%
\pgfsetroundjoin%
\pgfsetlinewidth{0.341344pt}%
\definecolor{currentstroke}{rgb}{0.273809,0.031497,0.358853}%
\pgfsetstrokecolor{currentstroke}%
\pgfsetdash{}{0pt}%
\pgfpathmoveto{\pgfqpoint{8.067271in}{2.127716in}}%
\pgfpathquadraticcurveto{\pgfqpoint{8.054924in}{2.129580in}}{\pgfqpoint{8.047798in}{2.130655in}}%
\pgfusepath{stroke}%
\end{pgfscope}%
\begin{pgfscope}%
\pgfpathrectangle{\pgfqpoint{6.720588in}{1.750000in}}{\pgfqpoint{2.279412in}{2.004545in}}%
\pgfusepath{clip}%
\pgfsetroundcap%
\pgfsetroundjoin%
\definecolor{currentfill}{rgb}{0.273809,0.031497,0.358853}%
\pgfsetfillcolor{currentfill}%
\pgfsetlinewidth{0.341344pt}%
\definecolor{currentstroke}{rgb}{0.273809,0.031497,0.358853}%
\pgfsetstrokecolor{currentstroke}%
\pgfsetdash{}{0pt}%
\pgfpathmoveto{\pgfqpoint{8.098585in}{2.094897in}}%
\pgfpathlineto{\pgfqpoint{8.047798in}{2.130655in}}%
\pgfpathlineto{\pgfqpoint{8.106877in}{2.149830in}}%
\pgfpathlineto{\pgfqpoint{8.098585in}{2.094897in}}%
\pgfpathlineto{\pgfqpoint{8.098585in}{2.094897in}}%
\pgfpathclose%
\pgfusepath{stroke,fill}%
\end{pgfscope}%
\begin{pgfscope}%
\pgfpathrectangle{\pgfqpoint{6.720588in}{1.750000in}}{\pgfqpoint{2.279412in}{2.004545in}}%
\pgfusepath{clip}%
\pgfsetroundcap%
\pgfsetroundjoin%
\pgfsetlinewidth{0.358792pt}%
\definecolor{currentstroke}{rgb}{0.277018,0.050344,0.375715}%
\pgfsetstrokecolor{currentstroke}%
\pgfsetdash{}{0pt}%
\pgfpathmoveto{\pgfqpoint{8.079210in}{2.189552in}}%
\pgfpathquadraticcurveto{\pgfqpoint{8.066811in}{2.191157in}}{\pgfqpoint{8.059916in}{2.192050in}}%
\pgfusepath{stroke}%
\end{pgfscope}%
\begin{pgfscope}%
\pgfpathrectangle{\pgfqpoint{6.720588in}{1.750000in}}{\pgfqpoint{2.279412in}{2.004545in}}%
\pgfusepath{clip}%
\pgfsetroundcap%
\pgfsetroundjoin%
\definecolor{currentfill}{rgb}{0.277018,0.050344,0.375715}%
\pgfsetfillcolor{currentfill}%
\pgfsetlinewidth{0.358792pt}%
\definecolor{currentstroke}{rgb}{0.277018,0.050344,0.375715}%
\pgfsetstrokecolor{currentstroke}%
\pgfsetdash{}{0pt}%
\pgfpathmoveto{\pgfqpoint{8.111445in}{2.157369in}}%
\pgfpathlineto{\pgfqpoint{8.059916in}{2.192050in}}%
\pgfpathlineto{\pgfqpoint{8.118578in}{2.212464in}}%
\pgfpathlineto{\pgfqpoint{8.111445in}{2.157369in}}%
\pgfpathlineto{\pgfqpoint{8.111445in}{2.157369in}}%
\pgfpathclose%
\pgfusepath{stroke,fill}%
\end{pgfscope}%
\begin{pgfscope}%
\pgfpathrectangle{\pgfqpoint{6.720588in}{1.750000in}}{\pgfqpoint{2.279412in}{2.004545in}}%
\pgfusepath{clip}%
\pgfsetroundcap%
\pgfsetroundjoin%
\pgfsetlinewidth{0.387416pt}%
\definecolor{currentstroke}{rgb}{0.280267,0.073417,0.397163}%
\pgfsetstrokecolor{currentstroke}%
\pgfsetdash{}{0pt}%
\pgfpathmoveto{\pgfqpoint{8.127785in}{2.271897in}}%
\pgfpathquadraticcurveto{\pgfqpoint{8.115334in}{2.273188in}}{\pgfqpoint{8.108845in}{2.273861in}}%
\pgfusepath{stroke}%
\end{pgfscope}%
\begin{pgfscope}%
\pgfpathrectangle{\pgfqpoint{6.720588in}{1.750000in}}{\pgfqpoint{2.279412in}{2.004545in}}%
\pgfusepath{clip}%
\pgfsetroundcap%
\pgfsetroundjoin%
\definecolor{currentfill}{rgb}{0.280267,0.073417,0.397163}%
\pgfsetfillcolor{currentfill}%
\pgfsetlinewidth{0.387416pt}%
\definecolor{currentstroke}{rgb}{0.280267,0.073417,0.397163}%
\pgfsetstrokecolor{currentstroke}%
\pgfsetdash{}{0pt}%
\pgfpathmoveto{\pgfqpoint{8.161239in}{2.240501in}}%
\pgfpathlineto{\pgfqpoint{8.108845in}{2.273861in}}%
\pgfpathlineto{\pgfqpoint{8.166969in}{2.295760in}}%
\pgfpathlineto{\pgfqpoint{8.161239in}{2.240501in}}%
\pgfpathlineto{\pgfqpoint{8.161239in}{2.240501in}}%
\pgfpathclose%
\pgfusepath{stroke,fill}%
\end{pgfscope}%
\begin{pgfscope}%
\pgfpathrectangle{\pgfqpoint{6.720588in}{1.750000in}}{\pgfqpoint{2.279412in}{2.004545in}}%
\pgfusepath{clip}%
\pgfsetroundcap%
\pgfsetroundjoin%
\pgfsetlinewidth{0.349177pt}%
\definecolor{currentstroke}{rgb}{0.276022,0.044167,0.370164}%
\pgfsetstrokecolor{currentstroke}%
\pgfsetdash{}{0pt}%
\pgfpathmoveto{\pgfqpoint{8.328176in}{2.306523in}}%
\pgfpathquadraticcurveto{\pgfqpoint{8.315656in}{2.307082in}}{\pgfqpoint{8.308532in}{2.307400in}}%
\pgfusepath{stroke}%
\end{pgfscope}%
\begin{pgfscope}%
\pgfpathrectangle{\pgfqpoint{6.720588in}{1.750000in}}{\pgfqpoint{2.279412in}{2.004545in}}%
\pgfusepath{clip}%
\pgfsetroundcap%
\pgfsetroundjoin%
\definecolor{currentfill}{rgb}{0.276022,0.044167,0.370164}%
\pgfsetfillcolor{currentfill}%
\pgfsetlinewidth{0.349177pt}%
\definecolor{currentstroke}{rgb}{0.276022,0.044167,0.370164}%
\pgfsetstrokecolor{currentstroke}%
\pgfsetdash{}{0pt}%
\pgfpathmoveto{\pgfqpoint{8.362794in}{2.277173in}}%
\pgfpathlineto{\pgfqpoint{8.308532in}{2.307400in}}%
\pgfpathlineto{\pgfqpoint{8.365271in}{2.332674in}}%
\pgfpathlineto{\pgfqpoint{8.362794in}{2.277173in}}%
\pgfpathlineto{\pgfqpoint{8.362794in}{2.277173in}}%
\pgfpathclose%
\pgfusepath{stroke,fill}%
\end{pgfscope}%
\begin{pgfscope}%
\pgfpathrectangle{\pgfqpoint{6.720588in}{1.750000in}}{\pgfqpoint{2.279412in}{2.004545in}}%
\pgfusepath{clip}%
\pgfsetroundcap%
\pgfsetroundjoin%
\pgfsetlinewidth{0.423714pt}%
\definecolor{currentstroke}{rgb}{0.282656,0.100196,0.422160}%
\pgfsetstrokecolor{currentstroke}%
\pgfsetdash{}{0pt}%
\pgfpathmoveto{\pgfqpoint{8.127604in}{2.362952in}}%
\pgfpathquadraticcurveto{\pgfqpoint{8.115119in}{2.363934in}}{\pgfqpoint{8.109169in}{2.364402in}}%
\pgfusepath{stroke}%
\end{pgfscope}%
\begin{pgfscope}%
\pgfpathrectangle{\pgfqpoint{6.720588in}{1.750000in}}{\pgfqpoint{2.279412in}{2.004545in}}%
\pgfusepath{clip}%
\pgfsetroundcap%
\pgfsetroundjoin%
\definecolor{currentfill}{rgb}{0.282656,0.100196,0.422160}%
\pgfsetfillcolor{currentfill}%
\pgfsetlinewidth{0.423714pt}%
\definecolor{currentstroke}{rgb}{0.282656,0.100196,0.422160}%
\pgfsetstrokecolor{currentstroke}%
\pgfsetdash{}{0pt}%
\pgfpathmoveto{\pgfqpoint{8.162375in}{2.332353in}}%
\pgfpathlineto{\pgfqpoint{8.109169in}{2.364402in}}%
\pgfpathlineto{\pgfqpoint{8.166732in}{2.387737in}}%
\pgfpathlineto{\pgfqpoint{8.162375in}{2.332353in}}%
\pgfpathlineto{\pgfqpoint{8.162375in}{2.332353in}}%
\pgfpathclose%
\pgfusepath{stroke,fill}%
\end{pgfscope}%
\begin{pgfscope}%
\pgfpathrectangle{\pgfqpoint{6.720588in}{1.750000in}}{\pgfqpoint{2.279412in}{2.004545in}}%
\pgfusepath{clip}%
\pgfsetroundcap%
\pgfsetroundjoin%
\pgfsetlinewidth{0.604011pt}%
\definecolor{currentstroke}{rgb}{0.265145,0.232956,0.516599}%
\pgfsetstrokecolor{currentstroke}%
\pgfsetdash{}{0pt}%
\pgfpathmoveto{\pgfqpoint{7.977688in}{2.462124in}}%
\pgfpathquadraticcurveto{\pgfqpoint{7.965290in}{2.463757in}}{\pgfqpoint{7.962157in}{2.464170in}}%
\pgfusepath{stroke}%
\end{pgfscope}%
\begin{pgfscope}%
\pgfpathrectangle{\pgfqpoint{6.720588in}{1.750000in}}{\pgfqpoint{2.279412in}{2.004545in}}%
\pgfusepath{clip}%
\pgfsetroundcap%
\pgfsetroundjoin%
\definecolor{currentfill}{rgb}{0.265145,0.232956,0.516599}%
\pgfsetfillcolor{currentfill}%
\pgfsetlinewidth{0.604011pt}%
\definecolor{currentstroke}{rgb}{0.265145,0.232956,0.516599}%
\pgfsetstrokecolor{currentstroke}%
\pgfsetdash{}{0pt}%
\pgfpathmoveto{\pgfqpoint{8.013609in}{2.429374in}}%
\pgfpathlineto{\pgfqpoint{7.962157in}{2.464170in}}%
\pgfpathlineto{\pgfqpoint{8.020865in}{2.484454in}}%
\pgfpathlineto{\pgfqpoint{8.013609in}{2.429374in}}%
\pgfpathlineto{\pgfqpoint{8.013609in}{2.429374in}}%
\pgfpathclose%
\pgfusepath{stroke,fill}%
\end{pgfscope}%
\begin{pgfscope}%
\pgfpathrectangle{\pgfqpoint{6.720588in}{1.750000in}}{\pgfqpoint{2.279412in}{2.004545in}}%
\pgfusepath{clip}%
\pgfsetroundcap%
\pgfsetroundjoin%
\pgfsetlinewidth{0.424953pt}%
\definecolor{currentstroke}{rgb}{0.282910,0.105393,0.426902}%
\pgfsetstrokecolor{currentstroke}%
\pgfsetdash{}{0pt}%
\pgfpathmoveto{\pgfqpoint{8.280651in}{2.527596in}}%
\pgfpathquadraticcurveto{\pgfqpoint{8.268119in}{2.527935in}}{\pgfqpoint{8.262160in}{2.528097in}}%
\pgfusepath{stroke}%
\end{pgfscope}%
\begin{pgfscope}%
\pgfpathrectangle{\pgfqpoint{6.720588in}{1.750000in}}{\pgfqpoint{2.279412in}{2.004545in}}%
\pgfusepath{clip}%
\pgfsetroundcap%
\pgfsetroundjoin%
\definecolor{currentfill}{rgb}{0.282910,0.105393,0.426902}%
\pgfsetfillcolor{currentfill}%
\pgfsetlinewidth{0.424953pt}%
\definecolor{currentstroke}{rgb}{0.282910,0.105393,0.426902}%
\pgfsetstrokecolor{currentstroke}%
\pgfsetdash{}{0pt}%
\pgfpathmoveto{\pgfqpoint{8.316943in}{2.498826in}}%
\pgfpathlineto{\pgfqpoint{8.262160in}{2.528097in}}%
\pgfpathlineto{\pgfqpoint{8.318446in}{2.554361in}}%
\pgfpathlineto{\pgfqpoint{8.316943in}{2.498826in}}%
\pgfpathlineto{\pgfqpoint{8.316943in}{2.498826in}}%
\pgfpathclose%
\pgfusepath{stroke,fill}%
\end{pgfscope}%
\begin{pgfscope}%
\pgfpathrectangle{\pgfqpoint{6.720588in}{1.750000in}}{\pgfqpoint{2.279412in}{2.004545in}}%
\pgfusepath{clip}%
\pgfsetroundcap%
\pgfsetroundjoin%
\pgfsetlinewidth{0.505999pt}%
\definecolor{currentstroke}{rgb}{0.280868,0.160771,0.472899}%
\pgfsetstrokecolor{currentstroke}%
\pgfsetdash{}{0pt}%
\pgfpathmoveto{\pgfqpoint{8.177249in}{2.973879in}}%
\pgfpathquadraticcurveto{\pgfqpoint{8.164722in}{2.973432in}}{\pgfqpoint{8.160018in}{2.973263in}}%
\pgfusepath{stroke}%
\end{pgfscope}%
\begin{pgfscope}%
\pgfpathrectangle{\pgfqpoint{6.720588in}{1.750000in}}{\pgfqpoint{2.279412in}{2.004545in}}%
\pgfusepath{clip}%
\pgfsetroundcap%
\pgfsetroundjoin%
\definecolor{currentfill}{rgb}{0.280868,0.160771,0.472899}%
\pgfsetfillcolor{currentfill}%
\pgfsetlinewidth{0.505999pt}%
\definecolor{currentstroke}{rgb}{0.280868,0.160771,0.472899}%
\pgfsetstrokecolor{currentstroke}%
\pgfsetdash{}{0pt}%
\pgfpathmoveto{\pgfqpoint{8.216529in}{2.947487in}}%
\pgfpathlineto{\pgfqpoint{8.160018in}{2.973263in}}%
\pgfpathlineto{\pgfqpoint{8.214546in}{3.003007in}}%
\pgfpathlineto{\pgfqpoint{8.216529in}{2.947487in}}%
\pgfpathlineto{\pgfqpoint{8.216529in}{2.947487in}}%
\pgfpathclose%
\pgfusepath{stroke,fill}%
\end{pgfscope}%
\begin{pgfscope}%
\pgfpathrectangle{\pgfqpoint{6.720588in}{1.750000in}}{\pgfqpoint{2.279412in}{2.004545in}}%
\pgfusepath{clip}%
\pgfsetroundcap%
\pgfsetroundjoin%
\pgfsetlinewidth{0.382162pt}%
\definecolor{currentstroke}{rgb}{0.279566,0.067836,0.391917}%
\pgfsetstrokecolor{currentstroke}%
\pgfsetdash{}{0pt}%
\pgfpathmoveto{\pgfqpoint{8.327776in}{3.020253in}}%
\pgfpathquadraticcurveto{\pgfqpoint{8.315245in}{3.019898in}}{\pgfqpoint{8.308624in}{3.019711in}}%
\pgfusepath{stroke}%
\end{pgfscope}%
\begin{pgfscope}%
\pgfpathrectangle{\pgfqpoint{6.720588in}{1.750000in}}{\pgfqpoint{2.279412in}{2.004545in}}%
\pgfusepath{clip}%
\pgfsetroundcap%
\pgfsetroundjoin%
\definecolor{currentfill}{rgb}{0.279566,0.067836,0.391917}%
\pgfsetfillcolor{currentfill}%
\pgfsetlinewidth{0.382162pt}%
\definecolor{currentstroke}{rgb}{0.279566,0.067836,0.391917}%
\pgfsetstrokecolor{currentstroke}%
\pgfsetdash{}{0pt}%
\pgfpathmoveto{\pgfqpoint{8.364944in}{2.993518in}}%
\pgfpathlineto{\pgfqpoint{8.308624in}{3.019711in}}%
\pgfpathlineto{\pgfqpoint{8.363371in}{3.049051in}}%
\pgfpathlineto{\pgfqpoint{8.364944in}{2.993518in}}%
\pgfpathlineto{\pgfqpoint{8.364944in}{2.993518in}}%
\pgfpathclose%
\pgfusepath{stroke,fill}%
\end{pgfscope}%
\begin{pgfscope}%
\pgfpathrectangle{\pgfqpoint{6.720588in}{1.750000in}}{\pgfqpoint{2.279412in}{2.004545in}}%
\pgfusepath{clip}%
\pgfsetroundcap%
\pgfsetroundjoin%
\pgfsetlinewidth{0.521339pt}%
\definecolor{currentstroke}{rgb}{0.278826,0.175490,0.483397}%
\pgfsetstrokecolor{currentstroke}%
\pgfsetdash{}{0pt}%
\pgfpathmoveto{\pgfqpoint{8.027599in}{3.046057in}}%
\pgfpathquadraticcurveto{\pgfqpoint{8.015169in}{3.044620in}}{\pgfqpoint{8.010752in}{3.044109in}}%
\pgfusepath{stroke}%
\end{pgfscope}%
\begin{pgfscope}%
\pgfpathrectangle{\pgfqpoint{6.720588in}{1.750000in}}{\pgfqpoint{2.279412in}{2.004545in}}%
\pgfusepath{clip}%
\pgfsetroundcap%
\pgfsetroundjoin%
\definecolor{currentfill}{rgb}{0.278826,0.175490,0.483397}%
\pgfsetfillcolor{currentfill}%
\pgfsetlinewidth{0.521339pt}%
\definecolor{currentstroke}{rgb}{0.278826,0.175490,0.483397}%
\pgfsetstrokecolor{currentstroke}%
\pgfsetdash{}{0pt}%
\pgfpathmoveto{\pgfqpoint{8.069131in}{3.022898in}}%
\pgfpathlineto{\pgfqpoint{8.010752in}{3.044109in}}%
\pgfpathlineto{\pgfqpoint{8.062748in}{3.078086in}}%
\pgfpathlineto{\pgfqpoint{8.069131in}{3.022898in}}%
\pgfpathlineto{\pgfqpoint{8.069131in}{3.022898in}}%
\pgfpathclose%
\pgfusepath{stroke,fill}%
\end{pgfscope}%
\begin{pgfscope}%
\pgfpathrectangle{\pgfqpoint{6.720588in}{1.750000in}}{\pgfqpoint{2.279412in}{2.004545in}}%
\pgfusepath{clip}%
\pgfsetroundcap%
\pgfsetroundjoin%
\pgfsetlinewidth{0.330452pt}%
\definecolor{currentstroke}{rgb}{0.272594,0.025563,0.353093}%
\pgfsetstrokecolor{currentstroke}%
\pgfsetdash{}{0pt}%
\pgfpathmoveto{\pgfqpoint{8.355421in}{3.293842in}}%
\pgfpathquadraticcurveto{\pgfqpoint{8.342890in}{3.293685in}}{\pgfqpoint{8.335472in}{3.293592in}}%
\pgfusepath{stroke}%
\end{pgfscope}%
\begin{pgfscope}%
\pgfpathrectangle{\pgfqpoint{6.720588in}{1.750000in}}{\pgfqpoint{2.279412in}{2.004545in}}%
\pgfusepath{clip}%
\pgfsetroundcap%
\pgfsetroundjoin%
\definecolor{currentfill}{rgb}{0.272594,0.025563,0.353093}%
\pgfsetfillcolor{currentfill}%
\pgfsetlinewidth{0.330452pt}%
\definecolor{currentstroke}{rgb}{0.272594,0.025563,0.353093}%
\pgfsetstrokecolor{currentstroke}%
\pgfsetdash{}{0pt}%
\pgfpathmoveto{\pgfqpoint{8.391371in}{3.266512in}}%
\pgfpathlineto{\pgfqpoint{8.335472in}{3.293592in}}%
\pgfpathlineto{\pgfqpoint{8.390675in}{3.322063in}}%
\pgfpathlineto{\pgfqpoint{8.391371in}{3.266512in}}%
\pgfpathlineto{\pgfqpoint{8.391371in}{3.266512in}}%
\pgfpathclose%
\pgfusepath{stroke,fill}%
\end{pgfscope}%
\begin{pgfscope}%
\pgfpathrectangle{\pgfqpoint{6.720588in}{1.750000in}}{\pgfqpoint{2.279412in}{2.004545in}}%
\pgfusepath{clip}%
\pgfsetroundcap%
\pgfsetroundjoin%
\pgfsetlinewidth{0.348245pt}%
\definecolor{currentstroke}{rgb}{0.274952,0.037752,0.364543}%
\pgfsetstrokecolor{currentstroke}%
\pgfsetdash{}{0pt}%
\pgfpathmoveto{\pgfqpoint{8.119603in}{3.375816in}}%
\pgfpathquadraticcurveto{\pgfqpoint{8.107179in}{3.374388in}}{\pgfqpoint{8.100108in}{3.373575in}}%
\pgfusepath{stroke}%
\end{pgfscope}%
\begin{pgfscope}%
\pgfpathrectangle{\pgfqpoint{6.720588in}{1.750000in}}{\pgfqpoint{2.279412in}{2.004545in}}%
\pgfusepath{clip}%
\pgfsetroundcap%
\pgfsetroundjoin%
\definecolor{currentfill}{rgb}{0.274952,0.037752,0.364543}%
\pgfsetfillcolor{currentfill}%
\pgfsetlinewidth{0.348245pt}%
\definecolor{currentstroke}{rgb}{0.274952,0.037752,0.364543}%
\pgfsetstrokecolor{currentstroke}%
\pgfsetdash{}{0pt}%
\pgfpathmoveto{\pgfqpoint{8.158472in}{3.352323in}}%
\pgfpathlineto{\pgfqpoint{8.100108in}{3.373575in}}%
\pgfpathlineto{\pgfqpoint{8.152128in}{3.407515in}}%
\pgfpathlineto{\pgfqpoint{8.158472in}{3.352323in}}%
\pgfpathlineto{\pgfqpoint{8.158472in}{3.352323in}}%
\pgfpathclose%
\pgfusepath{stroke,fill}%
\end{pgfscope}%
\begin{pgfscope}%
\pgfpathrectangle{\pgfqpoint{6.720588in}{1.750000in}}{\pgfqpoint{2.279412in}{2.004545in}}%
\pgfusepath{clip}%
\pgfsetroundcap%
\pgfsetroundjoin%
\pgfsetlinewidth{0.363655pt}%
\definecolor{currentstroke}{rgb}{0.277941,0.056324,0.381191}%
\pgfsetstrokecolor{currentstroke}%
\pgfsetdash{}{0pt}%
\pgfpathmoveto{\pgfqpoint{8.326573in}{2.393104in}}%
\pgfpathquadraticcurveto{\pgfqpoint{8.314047in}{2.393561in}}{\pgfqpoint{8.307143in}{2.393812in}}%
\pgfusepath{stroke}%
\end{pgfscope}%
\begin{pgfscope}%
\pgfpathrectangle{\pgfqpoint{6.720588in}{1.750000in}}{\pgfqpoint{2.279412in}{2.004545in}}%
\pgfusepath{clip}%
\pgfsetroundcap%
\pgfsetroundjoin%
\definecolor{currentfill}{rgb}{0.277941,0.056324,0.381191}%
\pgfsetfillcolor{currentfill}%
\pgfsetlinewidth{0.363655pt}%
\definecolor{currentstroke}{rgb}{0.277941,0.056324,0.381191}%
\pgfsetstrokecolor{currentstroke}%
\pgfsetdash{}{0pt}%
\pgfpathmoveto{\pgfqpoint{8.361649in}{2.364028in}}%
\pgfpathlineto{\pgfqpoint{8.307143in}{2.393812in}}%
\pgfpathlineto{\pgfqpoint{8.363674in}{2.419546in}}%
\pgfpathlineto{\pgfqpoint{8.361649in}{2.364028in}}%
\pgfpathlineto{\pgfqpoint{8.361649in}{2.364028in}}%
\pgfpathclose%
\pgfusepath{stroke,fill}%
\end{pgfscope}%
\begin{pgfscope}%
\pgfpathrectangle{\pgfqpoint{6.720588in}{1.750000in}}{\pgfqpoint{2.279412in}{2.004545in}}%
\pgfusepath{clip}%
\pgfsetroundcap%
\pgfsetroundjoin%
\pgfsetlinewidth{0.704088pt}%
\definecolor{currentstroke}{rgb}{0.241237,0.296485,0.539709}%
\pgfsetstrokecolor{currentstroke}%
\pgfsetdash{}{0pt}%
\pgfpathmoveto{\pgfqpoint{8.125929in}{2.663081in}}%
\pgfpathquadraticcurveto{\pgfqpoint{8.113392in}{2.663186in}}{\pgfqpoint{8.111746in}{2.663200in}}%
\pgfusepath{stroke}%
\end{pgfscope}%
\begin{pgfscope}%
\pgfpathrectangle{\pgfqpoint{6.720588in}{1.750000in}}{\pgfqpoint{2.279412in}{2.004545in}}%
\pgfusepath{clip}%
\pgfsetroundcap%
\pgfsetroundjoin%
\definecolor{currentfill}{rgb}{0.241237,0.296485,0.539709}%
\pgfsetfillcolor{currentfill}%
\pgfsetlinewidth{0.704088pt}%
\definecolor{currentstroke}{rgb}{0.241237,0.296485,0.539709}%
\pgfsetstrokecolor{currentstroke}%
\pgfsetdash{}{0pt}%
\pgfpathmoveto{\pgfqpoint{8.167068in}{2.634959in}}%
\pgfpathlineto{\pgfqpoint{8.111746in}{2.663200in}}%
\pgfpathlineto{\pgfqpoint{8.167532in}{2.690512in}}%
\pgfpathlineto{\pgfqpoint{8.167068in}{2.634959in}}%
\pgfpathlineto{\pgfqpoint{8.167068in}{2.634959in}}%
\pgfpathclose%
\pgfusepath{stroke,fill}%
\end{pgfscope}%
\begin{pgfscope}%
\pgfpathrectangle{\pgfqpoint{6.720588in}{1.750000in}}{\pgfqpoint{2.279412in}{2.004545in}}%
\pgfusepath{clip}%
\pgfsetroundcap%
\pgfsetroundjoin%
\pgfsetlinewidth{0.704206pt}%
\definecolor{currentstroke}{rgb}{0.241237,0.296485,0.539709}%
\pgfsetstrokecolor{currentstroke}%
\pgfsetdash{}{0pt}%
\pgfpathmoveto{\pgfqpoint{8.125999in}{2.795686in}}%
\pgfpathquadraticcurveto{\pgfqpoint{8.113462in}{2.795565in}}{\pgfqpoint{8.111819in}{2.795550in}}%
\pgfusepath{stroke}%
\end{pgfscope}%
\begin{pgfscope}%
\pgfpathrectangle{\pgfqpoint{6.720588in}{1.750000in}}{\pgfqpoint{2.279412in}{2.004545in}}%
\pgfusepath{clip}%
\pgfsetroundcap%
\pgfsetroundjoin%
\definecolor{currentfill}{rgb}{0.241237,0.296485,0.539709}%
\pgfsetfillcolor{currentfill}%
\pgfsetlinewidth{0.704206pt}%
\definecolor{currentstroke}{rgb}{0.241237,0.296485,0.539709}%
\pgfsetstrokecolor{currentstroke}%
\pgfsetdash{}{0pt}%
\pgfpathmoveto{\pgfqpoint{8.167639in}{2.768308in}}%
\pgfpathlineto{\pgfqpoint{8.111819in}{2.795550in}}%
\pgfpathlineto{\pgfqpoint{8.167104in}{2.823861in}}%
\pgfpathlineto{\pgfqpoint{8.167639in}{2.768308in}}%
\pgfpathlineto{\pgfqpoint{8.167639in}{2.768308in}}%
\pgfpathclose%
\pgfusepath{stroke,fill}%
\end{pgfscope}%
\begin{pgfscope}%
\pgfpathrectangle{\pgfqpoint{6.720588in}{1.750000in}}{\pgfqpoint{2.279412in}{2.004545in}}%
\pgfusepath{clip}%
\pgfsetroundcap%
\pgfsetroundjoin%
\pgfsetlinewidth{0.590069pt}%
\definecolor{currentstroke}{rgb}{0.267968,0.223549,0.512008}%
\pgfsetstrokecolor{currentstroke}%
\pgfsetdash{}{0pt}%
\pgfpathmoveto{\pgfqpoint{8.176086in}{2.839254in}}%
\pgfpathquadraticcurveto{\pgfqpoint{8.163550in}{2.839089in}}{\pgfqpoint{8.160142in}{2.839044in}}%
\pgfusepath{stroke}%
\end{pgfscope}%
\begin{pgfscope}%
\pgfpathrectangle{\pgfqpoint{6.720588in}{1.750000in}}{\pgfqpoint{2.279412in}{2.004545in}}%
\pgfusepath{clip}%
\pgfsetroundcap%
\pgfsetroundjoin%
\definecolor{currentfill}{rgb}{0.267968,0.223549,0.512008}%
\pgfsetfillcolor{currentfill}%
\pgfsetlinewidth{0.590069pt}%
\definecolor{currentstroke}{rgb}{0.267968,0.223549,0.512008}%
\pgfsetstrokecolor{currentstroke}%
\pgfsetdash{}{0pt}%
\pgfpathmoveto{\pgfqpoint{8.216059in}{2.812002in}}%
\pgfpathlineto{\pgfqpoint{8.160142in}{2.839044in}}%
\pgfpathlineto{\pgfqpoint{8.215325in}{2.867553in}}%
\pgfpathlineto{\pgfqpoint{8.216059in}{2.812002in}}%
\pgfpathlineto{\pgfqpoint{8.216059in}{2.812002in}}%
\pgfpathclose%
\pgfusepath{stroke,fill}%
\end{pgfscope}%
\begin{pgfscope}%
\pgfpathrectangle{\pgfqpoint{6.720588in}{1.750000in}}{\pgfqpoint{2.279412in}{2.004545in}}%
\pgfusepath{clip}%
\pgfsetroundcap%
\pgfsetroundjoin%
\pgfsetlinewidth{0.454368pt}%
\definecolor{currentstroke}{rgb}{0.283187,0.125848,0.444960}%
\pgfsetstrokecolor{currentstroke}%
\pgfsetdash{}{0pt}%
\pgfpathmoveto{\pgfqpoint{7.929017in}{3.112535in}}%
\pgfpathquadraticcurveto{\pgfqpoint{7.916890in}{3.109772in}}{\pgfqpoint{7.911616in}{3.108571in}}%
\pgfusepath{stroke}%
\end{pgfscope}%
\begin{pgfscope}%
\pgfpathrectangle{\pgfqpoint{6.720588in}{1.750000in}}{\pgfqpoint{2.279412in}{2.004545in}}%
\pgfusepath{clip}%
\pgfsetroundcap%
\pgfsetroundjoin%
\definecolor{currentfill}{rgb}{0.283187,0.125848,0.444960}%
\pgfsetfillcolor{currentfill}%
\pgfsetlinewidth{0.454368pt}%
\definecolor{currentstroke}{rgb}{0.283187,0.125848,0.444960}%
\pgfsetstrokecolor{currentstroke}%
\pgfsetdash{}{0pt}%
\pgfpathmoveto{\pgfqpoint{7.971954in}{3.093828in}}%
\pgfpathlineto{\pgfqpoint{7.911616in}{3.108571in}}%
\pgfpathlineto{\pgfqpoint{7.959613in}{3.147995in}}%
\pgfpathlineto{\pgfqpoint{7.971954in}{3.093828in}}%
\pgfpathlineto{\pgfqpoint{7.971954in}{3.093828in}}%
\pgfpathclose%
\pgfusepath{stroke,fill}%
\end{pgfscope}%
\begin{pgfscope}%
\pgfpathrectangle{\pgfqpoint{6.720588in}{1.750000in}}{\pgfqpoint{2.279412in}{2.004545in}}%
\pgfusepath{clip}%
\pgfsetroundcap%
\pgfsetroundjoin%
\pgfsetlinewidth{0.344053pt}%
\definecolor{currentstroke}{rgb}{0.274952,0.037752,0.364543}%
\pgfsetstrokecolor{currentstroke}%
\pgfsetdash{}{0pt}%
\pgfpathmoveto{\pgfqpoint{8.316720in}{2.198208in}}%
\pgfpathquadraticcurveto{\pgfqpoint{8.304198in}{2.198765in}}{\pgfqpoint{8.296995in}{2.199085in}}%
\pgfusepath{stroke}%
\end{pgfscope}%
\begin{pgfscope}%
\pgfpathrectangle{\pgfqpoint{6.720588in}{1.750000in}}{\pgfqpoint{2.279412in}{2.004545in}}%
\pgfusepath{clip}%
\pgfsetroundcap%
\pgfsetroundjoin%
\definecolor{currentfill}{rgb}{0.274952,0.037752,0.364543}%
\pgfsetfillcolor{currentfill}%
\pgfsetlinewidth{0.344053pt}%
\definecolor{currentstroke}{rgb}{0.274952,0.037752,0.364543}%
\pgfsetstrokecolor{currentstroke}%
\pgfsetdash{}{0pt}%
\pgfpathmoveto{\pgfqpoint{8.351261in}{2.168866in}}%
\pgfpathlineto{\pgfqpoint{8.296995in}{2.199085in}}%
\pgfpathlineto{\pgfqpoint{8.353730in}{2.224367in}}%
\pgfpathlineto{\pgfqpoint{8.351261in}{2.168866in}}%
\pgfpathlineto{\pgfqpoint{8.351261in}{2.168866in}}%
\pgfpathclose%
\pgfusepath{stroke,fill}%
\end{pgfscope}%
\begin{pgfscope}%
\pgfpathrectangle{\pgfqpoint{6.720588in}{1.750000in}}{\pgfqpoint{2.279412in}{2.004545in}}%
\pgfusepath{clip}%
\pgfsetroundcap%
\pgfsetroundjoin%
\pgfsetlinewidth{0.423402pt}%
\definecolor{currentstroke}{rgb}{0.282656,0.100196,0.422160}%
\pgfsetstrokecolor{currentstroke}%
\pgfsetdash{}{0pt}%
\pgfpathmoveto{\pgfqpoint{8.275242in}{2.485106in}}%
\pgfpathquadraticcurveto{\pgfqpoint{8.262711in}{2.485489in}}{\pgfqpoint{8.256728in}{2.485671in}}%
\pgfusepath{stroke}%
\end{pgfscope}%
\begin{pgfscope}%
\pgfpathrectangle{\pgfqpoint{6.720588in}{1.750000in}}{\pgfqpoint{2.279412in}{2.004545in}}%
\pgfusepath{clip}%
\pgfsetroundcap%
\pgfsetroundjoin%
\definecolor{currentfill}{rgb}{0.282656,0.100196,0.422160}%
\pgfsetfillcolor{currentfill}%
\pgfsetlinewidth{0.423402pt}%
\definecolor{currentstroke}{rgb}{0.282656,0.100196,0.422160}%
\pgfsetstrokecolor{currentstroke}%
\pgfsetdash{}{0pt}%
\pgfpathmoveto{\pgfqpoint{8.311410in}{2.456211in}}%
\pgfpathlineto{\pgfqpoint{8.256728in}{2.485671in}}%
\pgfpathlineto{\pgfqpoint{8.313106in}{2.511741in}}%
\pgfpathlineto{\pgfqpoint{8.311410in}{2.456211in}}%
\pgfpathlineto{\pgfqpoint{8.311410in}{2.456211in}}%
\pgfpathclose%
\pgfusepath{stroke,fill}%
\end{pgfscope}%
\begin{pgfscope}%
\pgfpathrectangle{\pgfqpoint{6.720588in}{1.750000in}}{\pgfqpoint{2.279412in}{2.004545in}}%
\pgfusepath{clip}%
\pgfsetroundcap%
\pgfsetroundjoin%
\pgfsetlinewidth{0.500666pt}%
\definecolor{currentstroke}{rgb}{0.280868,0.160771,0.472899}%
\pgfsetstrokecolor{currentstroke}%
\pgfsetdash{}{0pt}%
\pgfpathmoveto{\pgfqpoint{8.225059in}{2.886299in}}%
\pgfpathquadraticcurveto{\pgfqpoint{8.212524in}{2.886055in}}{\pgfqpoint{8.207733in}{2.885961in}}%
\pgfusepath{stroke}%
\end{pgfscope}%
\begin{pgfscope}%
\pgfpathrectangle{\pgfqpoint{6.720588in}{1.750000in}}{\pgfqpoint{2.279412in}{2.004545in}}%
\pgfusepath{clip}%
\pgfsetroundcap%
\pgfsetroundjoin%
\definecolor{currentfill}{rgb}{0.280868,0.160771,0.472899}%
\pgfsetfillcolor{currentfill}%
\pgfsetlinewidth{0.500666pt}%
\definecolor{currentstroke}{rgb}{0.280868,0.160771,0.472899}%
\pgfsetstrokecolor{currentstroke}%
\pgfsetdash{}{0pt}%
\pgfpathmoveto{\pgfqpoint{8.263818in}{2.859270in}}%
\pgfpathlineto{\pgfqpoint{8.207733in}{2.885961in}}%
\pgfpathlineto{\pgfqpoint{8.262737in}{2.914815in}}%
\pgfpathlineto{\pgfqpoint{8.263818in}{2.859270in}}%
\pgfpathlineto{\pgfqpoint{8.263818in}{2.859270in}}%
\pgfpathclose%
\pgfusepath{stroke,fill}%
\end{pgfscope}%
\begin{pgfscope}%
\pgfpathrectangle{\pgfqpoint{6.720588in}{1.750000in}}{\pgfqpoint{2.279412in}{2.004545in}}%
\pgfusepath{clip}%
\pgfsetroundcap%
\pgfsetroundjoin%
\pgfsetlinewidth{0.367247pt}%
\definecolor{currentstroke}{rgb}{0.277941,0.056324,0.381191}%
\pgfsetstrokecolor{currentstroke}%
\pgfsetdash{}{0pt}%
\pgfpathmoveto{\pgfqpoint{8.325537in}{3.107828in}}%
\pgfpathquadraticcurveto{\pgfqpoint{8.313011in}{3.107344in}}{\pgfqpoint{8.306163in}{3.107079in}}%
\pgfusepath{stroke}%
\end{pgfscope}%
\begin{pgfscope}%
\pgfpathrectangle{\pgfqpoint{6.720588in}{1.750000in}}{\pgfqpoint{2.279412in}{2.004545in}}%
\pgfusepath{clip}%
\pgfsetroundcap%
\pgfsetroundjoin%
\definecolor{currentfill}{rgb}{0.277941,0.056324,0.381191}%
\pgfsetfillcolor{currentfill}%
\pgfsetlinewidth{0.367247pt}%
\definecolor{currentstroke}{rgb}{0.277941,0.056324,0.381191}%
\pgfsetstrokecolor{currentstroke}%
\pgfsetdash{}{0pt}%
\pgfpathmoveto{\pgfqpoint{8.362751in}{3.081469in}}%
\pgfpathlineto{\pgfqpoint{8.306163in}{3.107079in}}%
\pgfpathlineto{\pgfqpoint{8.360603in}{3.136983in}}%
\pgfpathlineto{\pgfqpoint{8.362751in}{3.081469in}}%
\pgfpathlineto{\pgfqpoint{8.362751in}{3.081469in}}%
\pgfpathclose%
\pgfusepath{stroke,fill}%
\end{pgfscope}%
\begin{pgfscope}%
\pgfpathrectangle{\pgfqpoint{6.720588in}{1.750000in}}{\pgfqpoint{2.279412in}{2.004545in}}%
\pgfusepath{clip}%
\pgfsetroundcap%
\pgfsetroundjoin%
\pgfsetlinewidth{0.384612pt}%
\definecolor{currentstroke}{rgb}{0.280267,0.073417,0.397163}%
\pgfsetstrokecolor{currentstroke}%
\pgfsetdash{}{0pt}%
\pgfpathmoveto{\pgfqpoint{7.725373in}{3.204051in}}%
\pgfpathquadraticcurveto{\pgfqpoint{7.725521in}{3.198624in}}{\pgfqpoint{7.725507in}{3.199144in}}%
\pgfusepath{stroke}%
\end{pgfscope}%
\begin{pgfscope}%
\pgfpathrectangle{\pgfqpoint{6.720588in}{1.750000in}}{\pgfqpoint{2.279412in}{2.004545in}}%
\pgfusepath{clip}%
\pgfsetroundcap%
\pgfsetroundjoin%
\definecolor{currentfill}{rgb}{0.280267,0.073417,0.397163}%
\pgfsetfillcolor{currentfill}%
\pgfsetlinewidth{0.384612pt}%
\definecolor{currentstroke}{rgb}{0.280267,0.073417,0.397163}%
\pgfsetstrokecolor{currentstroke}%
\pgfsetdash{}{0pt}%
\pgfpathmoveto{\pgfqpoint{7.751756in}{3.255438in}}%
\pgfpathlineto{\pgfqpoint{7.725507in}{3.199144in}}%
\pgfpathlineto{\pgfqpoint{7.696221in}{3.253920in}}%
\pgfpathlineto{\pgfqpoint{7.751756in}{3.255438in}}%
\pgfpathlineto{\pgfqpoint{7.751756in}{3.255438in}}%
\pgfpathclose%
\pgfusepath{stroke,fill}%
\end{pgfscope}%
\begin{pgfscope}%
\pgfpathrectangle{\pgfqpoint{6.720588in}{1.750000in}}{\pgfqpoint{2.279412in}{2.004545in}}%
\pgfusepath{clip}%
\pgfsetroundcap%
\pgfsetroundjoin%
\pgfsetlinewidth{1.052322pt}%
\definecolor{currentstroke}{rgb}{0.153364,0.497000,0.557724}%
\pgfsetstrokecolor{currentstroke}%
\pgfsetdash{}{0pt}%
\pgfpathmoveto{\pgfqpoint{7.195216in}{2.803993in}}%
\pgfpathquadraticcurveto{\pgfqpoint{7.207610in}{2.802339in}}{\pgfqpoint{7.203867in}{2.802839in}}%
\pgfusepath{stroke}%
\end{pgfscope}%
\begin{pgfscope}%
\pgfpathrectangle{\pgfqpoint{6.720588in}{1.750000in}}{\pgfqpoint{2.279412in}{2.004545in}}%
\pgfusepath{clip}%
\pgfsetroundcap%
\pgfsetroundjoin%
\definecolor{currentfill}{rgb}{0.153364,0.497000,0.557724}%
\pgfsetfillcolor{currentfill}%
\pgfsetlinewidth{1.052322pt}%
\definecolor{currentstroke}{rgb}{0.153364,0.497000,0.557724}%
\pgfsetstrokecolor{currentstroke}%
\pgfsetdash{}{0pt}%
\pgfpathmoveto{\pgfqpoint{7.152472in}{2.837719in}}%
\pgfpathlineto{\pgfqpoint{7.203867in}{2.802839in}}%
\pgfpathlineto{\pgfqpoint{7.145126in}{2.782651in}}%
\pgfpathlineto{\pgfqpoint{7.152472in}{2.837719in}}%
\pgfpathlineto{\pgfqpoint{7.152472in}{2.837719in}}%
\pgfpathclose%
\pgfusepath{stroke,fill}%
\end{pgfscope}%
\begin{pgfscope}%
\pgfpathrectangle{\pgfqpoint{6.720588in}{1.750000in}}{\pgfqpoint{2.279412in}{2.004545in}}%
\pgfusepath{clip}%
\pgfsetroundcap%
\pgfsetroundjoin%
\pgfsetlinewidth{0.391750pt}%
\definecolor{currentstroke}{rgb}{0.280894,0.078907,0.402329}%
\pgfsetstrokecolor{currentstroke}%
\pgfsetdash{}{0pt}%
\pgfpathmoveto{\pgfqpoint{8.025005in}{3.177854in}}%
\pgfpathquadraticcurveto{\pgfqpoint{8.012780in}{3.175436in}}{\pgfqpoint{8.006500in}{3.174194in}}%
\pgfusepath{stroke}%
\end{pgfscope}%
\begin{pgfscope}%
\pgfpathrectangle{\pgfqpoint{6.720588in}{1.750000in}}{\pgfqpoint{2.279412in}{2.004545in}}%
\pgfusepath{clip}%
\pgfsetroundcap%
\pgfsetroundjoin%
\definecolor{currentfill}{rgb}{0.280894,0.078907,0.402329}%
\pgfsetfillcolor{currentfill}%
\pgfsetlinewidth{0.391750pt}%
\definecolor{currentstroke}{rgb}{0.280894,0.078907,0.402329}%
\pgfsetstrokecolor{currentstroke}%
\pgfsetdash{}{0pt}%
\pgfpathmoveto{\pgfqpoint{8.066389in}{3.157723in}}%
\pgfpathlineto{\pgfqpoint{8.006500in}{3.174194in}}%
\pgfpathlineto{\pgfqpoint{8.055610in}{3.212222in}}%
\pgfpathlineto{\pgfqpoint{8.066389in}{3.157723in}}%
\pgfpathlineto{\pgfqpoint{8.066389in}{3.157723in}}%
\pgfpathclose%
\pgfusepath{stroke,fill}%
\end{pgfscope}%
\begin{pgfscope}%
\pgfpathrectangle{\pgfqpoint{6.720588in}{1.750000in}}{\pgfqpoint{2.279412in}{2.004545in}}%
\pgfusepath{clip}%
\pgfsetroundcap%
\pgfsetroundjoin%
\pgfsetlinewidth{0.351620pt}%
\definecolor{currentstroke}{rgb}{0.276022,0.044167,0.370164}%
\pgfsetstrokecolor{currentstroke}%
\pgfsetdash{}{0pt}%
\pgfpathmoveto{\pgfqpoint{8.274146in}{3.244524in}}%
\pgfpathquadraticcurveto{\pgfqpoint{8.261631in}{3.243897in}}{\pgfqpoint{8.254549in}{3.243542in}}%
\pgfusepath{stroke}%
\end{pgfscope}%
\begin{pgfscope}%
\pgfpathrectangle{\pgfqpoint{6.720588in}{1.750000in}}{\pgfqpoint{2.279412in}{2.004545in}}%
\pgfusepath{clip}%
\pgfsetroundcap%
\pgfsetroundjoin%
\definecolor{currentfill}{rgb}{0.276022,0.044167,0.370164}%
\pgfsetfillcolor{currentfill}%
\pgfsetlinewidth{0.351620pt}%
\definecolor{currentstroke}{rgb}{0.276022,0.044167,0.370164}%
\pgfsetstrokecolor{currentstroke}%
\pgfsetdash{}{0pt}%
\pgfpathmoveto{\pgfqpoint{8.311426in}{3.218580in}}%
\pgfpathlineto{\pgfqpoint{8.254549in}{3.243542in}}%
\pgfpathlineto{\pgfqpoint{8.308645in}{3.274066in}}%
\pgfpathlineto{\pgfqpoint{8.311426in}{3.218580in}}%
\pgfpathlineto{\pgfqpoint{8.311426in}{3.218580in}}%
\pgfpathclose%
\pgfusepath{stroke,fill}%
\end{pgfscope}%
\begin{pgfscope}%
\pgfpathrectangle{\pgfqpoint{6.720588in}{1.750000in}}{\pgfqpoint{2.279412in}{2.004545in}}%
\pgfusepath{clip}%
\pgfsetroundcap%
\pgfsetroundjoin%
\pgfsetlinewidth{0.390290pt}%
\definecolor{currentstroke}{rgb}{0.280267,0.073417,0.397163}%
\pgfsetstrokecolor{currentstroke}%
\pgfsetdash{}{0pt}%
\pgfpathmoveto{\pgfqpoint{7.920662in}{3.214610in}}%
\pgfpathquadraticcurveto{\pgfqpoint{7.908941in}{3.210778in}}{\pgfqpoint{7.902959in}{3.208823in}}%
\pgfusepath{stroke}%
\end{pgfscope}%
\begin{pgfscope}%
\pgfpathrectangle{\pgfqpoint{6.720588in}{1.750000in}}{\pgfqpoint{2.279412in}{2.004545in}}%
\pgfusepath{clip}%
\pgfsetroundcap%
\pgfsetroundjoin%
\definecolor{currentfill}{rgb}{0.280267,0.073417,0.397163}%
\pgfsetfillcolor{currentfill}%
\pgfsetlinewidth{0.390290pt}%
\definecolor{currentstroke}{rgb}{0.280267,0.073417,0.397163}%
\pgfsetstrokecolor{currentstroke}%
\pgfsetdash{}{0pt}%
\pgfpathmoveto{\pgfqpoint{7.964396in}{3.199685in}}%
\pgfpathlineto{\pgfqpoint{7.902959in}{3.208823in}}%
\pgfpathlineto{\pgfqpoint{7.947132in}{3.252490in}}%
\pgfpathlineto{\pgfqpoint{7.964396in}{3.199685in}}%
\pgfpathlineto{\pgfqpoint{7.964396in}{3.199685in}}%
\pgfpathclose%
\pgfusepath{stroke,fill}%
\end{pgfscope}%
\begin{pgfscope}%
\pgfpathrectangle{\pgfqpoint{6.720588in}{1.750000in}}{\pgfqpoint{2.279412in}{2.004545in}}%
\pgfusepath{clip}%
\pgfsetroundcap%
\pgfsetroundjoin%
\pgfsetlinewidth{0.367284pt}%
\definecolor{currentstroke}{rgb}{0.277941,0.056324,0.381191}%
\pgfsetstrokecolor{currentstroke}%
\pgfsetdash{}{0pt}%
\pgfpathmoveto{\pgfqpoint{7.829523in}{3.233368in}}%
\pgfpathquadraticcurveto{\pgfqpoint{7.823395in}{3.228927in}}{\pgfqpoint{7.821868in}{3.227820in}}%
\pgfusepath{stroke}%
\end{pgfscope}%
\begin{pgfscope}%
\pgfpathrectangle{\pgfqpoint{6.720588in}{1.750000in}}{\pgfqpoint{2.279412in}{2.004545in}}%
\pgfusepath{clip}%
\pgfsetroundcap%
\pgfsetroundjoin%
\definecolor{currentfill}{rgb}{0.277941,0.056324,0.381191}%
\pgfsetfillcolor{currentfill}%
\pgfsetlinewidth{0.367284pt}%
\definecolor{currentstroke}{rgb}{0.277941,0.056324,0.381191}%
\pgfsetstrokecolor{currentstroke}%
\pgfsetdash{}{0pt}%
\pgfpathmoveto{\pgfqpoint{7.883152in}{3.237934in}}%
\pgfpathlineto{\pgfqpoint{7.821868in}{3.227820in}}%
\pgfpathlineto{\pgfqpoint{7.850547in}{3.282916in}}%
\pgfpathlineto{\pgfqpoint{7.883152in}{3.237934in}}%
\pgfpathlineto{\pgfqpoint{7.883152in}{3.237934in}}%
\pgfpathclose%
\pgfusepath{stroke,fill}%
\end{pgfscope}%
\begin{pgfscope}%
\pgfpathrectangle{\pgfqpoint{6.720588in}{1.750000in}}{\pgfqpoint{2.279412in}{2.004545in}}%
\pgfusepath{clip}%
\pgfsetroundcap%
\pgfsetroundjoin%
\pgfsetlinewidth{0.664473pt}%
\definecolor{currentstroke}{rgb}{0.252194,0.269783,0.531579}%
\pgfsetstrokecolor{currentstroke}%
\pgfsetdash{}{0pt}%
\pgfpathmoveto{\pgfqpoint{7.388368in}{2.853695in}}%
\pgfpathquadraticcurveto{\pgfqpoint{7.399083in}{2.848127in}}{\pgfqpoint{7.400677in}{2.847299in}}%
\pgfusepath{stroke}%
\end{pgfscope}%
\begin{pgfscope}%
\pgfpathrectangle{\pgfqpoint{6.720588in}{1.750000in}}{\pgfqpoint{2.279412in}{2.004545in}}%
\pgfusepath{clip}%
\pgfsetroundcap%
\pgfsetroundjoin%
\definecolor{currentfill}{rgb}{0.252194,0.269783,0.531579}%
\pgfsetfillcolor{currentfill}%
\pgfsetlinewidth{0.664473pt}%
\definecolor{currentstroke}{rgb}{0.252194,0.269783,0.531579}%
\pgfsetstrokecolor{currentstroke}%
\pgfsetdash{}{0pt}%
\pgfpathmoveto{\pgfqpoint{7.364187in}{2.897564in}}%
\pgfpathlineto{\pgfqpoint{7.400677in}{2.847299in}}%
\pgfpathlineto{\pgfqpoint{7.338571in}{2.848267in}}%
\pgfpathlineto{\pgfqpoint{7.364187in}{2.897564in}}%
\pgfpathlineto{\pgfqpoint{7.364187in}{2.897564in}}%
\pgfpathclose%
\pgfusepath{stroke,fill}%
\end{pgfscope}%
\begin{pgfscope}%
\pgfpathrectangle{\pgfqpoint{6.720588in}{1.750000in}}{\pgfqpoint{2.279412in}{2.004545in}}%
\pgfusepath{clip}%
\pgfsetroundcap%
\pgfsetroundjoin%
\pgfsetlinewidth{0.667022pt}%
\definecolor{currentstroke}{rgb}{0.250425,0.274290,0.533103}%
\pgfsetstrokecolor{currentstroke}%
\pgfsetdash{}{0pt}%
\pgfpathmoveto{\pgfqpoint{7.414502in}{2.605198in}}%
\pgfpathquadraticcurveto{\pgfqpoint{7.420662in}{2.610211in}}{\pgfqpoint{7.418818in}{2.608710in}}%
\pgfusepath{stroke}%
\end{pgfscope}%
\begin{pgfscope}%
\pgfpathrectangle{\pgfqpoint{6.720588in}{1.750000in}}{\pgfqpoint{2.279412in}{2.004545in}}%
\pgfusepath{clip}%
\pgfsetroundcap%
\pgfsetroundjoin%
\definecolor{currentfill}{rgb}{0.250425,0.274290,0.533103}%
\pgfsetfillcolor{currentfill}%
\pgfsetlinewidth{0.667022pt}%
\definecolor{currentstroke}{rgb}{0.250425,0.274290,0.533103}%
\pgfsetstrokecolor{currentstroke}%
\pgfsetdash{}{0pt}%
\pgfpathmoveto{\pgfqpoint{7.358194in}{2.595194in}}%
\pgfpathlineto{\pgfqpoint{7.418818in}{2.608710in}}%
\pgfpathlineto{\pgfqpoint{7.393257in}{2.552101in}}%
\pgfpathlineto{\pgfqpoint{7.358194in}{2.595194in}}%
\pgfpathlineto{\pgfqpoint{7.358194in}{2.595194in}}%
\pgfpathclose%
\pgfusepath{stroke,fill}%
\end{pgfscope}%
\begin{pgfscope}%
\pgfpathrectangle{\pgfqpoint{6.720588in}{1.750000in}}{\pgfqpoint{2.279412in}{2.004545in}}%
\pgfusepath{clip}%
\pgfsetroundcap%
\pgfsetroundjoin%
\pgfsetlinewidth{0.667721pt}%
\definecolor{currentstroke}{rgb}{0.250425,0.274290,0.533103}%
\pgfsetstrokecolor{currentstroke}%
\pgfsetdash{}{0pt}%
\pgfpathmoveto{\pgfqpoint{7.498690in}{2.506417in}}%
\pgfpathquadraticcurveto{\pgfqpoint{7.500803in}{2.512381in}}{\pgfqpoint{7.499466in}{2.508608in}}%
\pgfusepath{stroke}%
\end{pgfscope}%
\begin{pgfscope}%
\pgfpathrectangle{\pgfqpoint{6.720588in}{1.750000in}}{\pgfqpoint{2.279412in}{2.004545in}}%
\pgfusepath{clip}%
\pgfsetroundcap%
\pgfsetroundjoin%
\definecolor{currentfill}{rgb}{0.250425,0.274290,0.533103}%
\pgfsetfillcolor{currentfill}%
\pgfsetlinewidth{0.667721pt}%
\definecolor{currentstroke}{rgb}{0.250425,0.274290,0.533103}%
\pgfsetstrokecolor{currentstroke}%
\pgfsetdash{}{0pt}%
\pgfpathmoveto{\pgfqpoint{7.454734in}{2.465513in}}%
\pgfpathlineto{\pgfqpoint{7.499466in}{2.508608in}}%
\pgfpathlineto{\pgfqpoint{7.507102in}{2.446966in}}%
\pgfpathlineto{\pgfqpoint{7.454734in}{2.465513in}}%
\pgfpathlineto{\pgfqpoint{7.454734in}{2.465513in}}%
\pgfpathclose%
\pgfusepath{stroke,fill}%
\end{pgfscope}%
\begin{pgfscope}%
\pgfpathrectangle{\pgfqpoint{6.720588in}{1.750000in}}{\pgfqpoint{2.279412in}{2.004545in}}%
\pgfusepath{clip}%
\pgfsetroundcap%
\pgfsetroundjoin%
\pgfsetlinewidth{0.494696pt}%
\definecolor{currentstroke}{rgb}{0.281412,0.155834,0.469201}%
\pgfsetstrokecolor{currentstroke}%
\pgfsetdash{}{0pt}%
\pgfpathmoveto{\pgfqpoint{7.522067in}{2.435565in}}%
\pgfpathquadraticcurveto{\pgfqpoint{7.526483in}{2.440630in}}{\pgfqpoint{7.525869in}{2.439926in}}%
\pgfusepath{stroke}%
\end{pgfscope}%
\begin{pgfscope}%
\pgfpathrectangle{\pgfqpoint{6.720588in}{1.750000in}}{\pgfqpoint{2.279412in}{2.004545in}}%
\pgfusepath{clip}%
\pgfsetroundcap%
\pgfsetroundjoin%
\definecolor{currentfill}{rgb}{0.281412,0.155834,0.469201}%
\pgfsetfillcolor{currentfill}%
\pgfsetlinewidth{0.494696pt}%
\definecolor{currentstroke}{rgb}{0.281412,0.155834,0.469201}%
\pgfsetstrokecolor{currentstroke}%
\pgfsetdash{}{0pt}%
\pgfpathmoveto{\pgfqpoint{7.468422in}{2.416307in}}%
\pgfpathlineto{\pgfqpoint{7.525869in}{2.439926in}}%
\pgfpathlineto{\pgfqpoint{7.510297in}{2.379797in}}%
\pgfpathlineto{\pgfqpoint{7.468422in}{2.416307in}}%
\pgfpathlineto{\pgfqpoint{7.468422in}{2.416307in}}%
\pgfpathclose%
\pgfusepath{stroke,fill}%
\end{pgfscope}%
\begin{pgfscope}%
\pgfpathrectangle{\pgfqpoint{6.720588in}{1.750000in}}{\pgfqpoint{2.279412in}{2.004545in}}%
\pgfusepath{clip}%
\pgfsetroundcap%
\pgfsetroundjoin%
\pgfsetlinewidth{0.641897pt}%
\definecolor{currentstroke}{rgb}{0.257322,0.256130,0.526563}%
\pgfsetstrokecolor{currentstroke}%
\pgfsetdash{}{0pt}%
\pgfpathmoveto{\pgfqpoint{7.535814in}{2.936450in}}%
\pgfpathquadraticcurveto{\pgfqpoint{7.534699in}{2.930046in}}{\pgfqpoint{7.535287in}{2.933424in}}%
\pgfusepath{stroke}%
\end{pgfscope}%
\begin{pgfscope}%
\pgfpathrectangle{\pgfqpoint{6.720588in}{1.750000in}}{\pgfqpoint{2.279412in}{2.004545in}}%
\pgfusepath{clip}%
\pgfsetroundcap%
\pgfsetroundjoin%
\definecolor{currentfill}{rgb}{0.257322,0.256130,0.526563}%
\pgfsetfillcolor{currentfill}%
\pgfsetlinewidth{0.641897pt}%
\definecolor{currentstroke}{rgb}{0.257322,0.256130,0.526563}%
\pgfsetstrokecolor{currentstroke}%
\pgfsetdash{}{0pt}%
\pgfpathmoveto{\pgfqpoint{7.572182in}{2.983392in}}%
\pgfpathlineto{\pgfqpoint{7.535287in}{2.933424in}}%
\pgfpathlineto{\pgfqpoint{7.517450in}{2.992921in}}%
\pgfpathlineto{\pgfqpoint{7.572182in}{2.983392in}}%
\pgfpathlineto{\pgfqpoint{7.572182in}{2.983392in}}%
\pgfpathclose%
\pgfusepath{stroke,fill}%
\end{pgfscope}%
\begin{pgfscope}%
\pgfpathrectangle{\pgfqpoint{6.720588in}{1.750000in}}{\pgfqpoint{2.279412in}{2.004545in}}%
\pgfusepath{clip}%
\pgfsetbuttcap%
\pgfsetroundjoin%
\pgfsetlinewidth{1.505625pt}%
\definecolor{currentstroke}{rgb}{0.000000,0.000000,0.000000}%
\pgfsetstrokecolor{currentstroke}%
\pgfsetdash{}{0pt}%
\pgfpathmoveto{\pgfqpoint{7.513972in}{2.089441in}}%
\pgfpathlineto{\pgfqpoint{7.513972in}{3.415105in}}%
\pgfusepath{stroke}%
\end{pgfscope}%
\begin{pgfscope}%
\pgfpathrectangle{\pgfqpoint{6.720588in}{1.750000in}}{\pgfqpoint{2.279412in}{2.004545in}}%
\pgfusepath{clip}%
\pgfsetbuttcap%
\pgfsetroundjoin%
\pgfsetlinewidth{1.505625pt}%
\definecolor{currentstroke}{rgb}{0.000000,0.000000,0.000000}%
\pgfsetstrokecolor{currentstroke}%
\pgfsetdash{}{0pt}%
\pgfpathmoveto{\pgfqpoint{8.662499in}{2.089441in}}%
\pgfpathlineto{\pgfqpoint{8.662499in}{3.415105in}}%
\pgfusepath{stroke}%
\end{pgfscope}%
\begin{pgfscope}%
\pgfsetrectcap%
\pgfsetmiterjoin%
\pgfsetlinewidth{0.803000pt}%
\definecolor{currentstroke}{rgb}{0.000000,0.000000,0.000000}%
\pgfsetstrokecolor{currentstroke}%
\pgfsetdash{}{0pt}%
\pgfpathmoveto{\pgfqpoint{6.720588in}{1.750000in}}%
\pgfpathlineto{\pgfqpoint{6.720588in}{3.754545in}}%
\pgfusepath{stroke}%
\end{pgfscope}%
\begin{pgfscope}%
\pgfsetrectcap%
\pgfsetmiterjoin%
\pgfsetlinewidth{0.803000pt}%
\definecolor{currentstroke}{rgb}{0.000000,0.000000,0.000000}%
\pgfsetstrokecolor{currentstroke}%
\pgfsetdash{}{0pt}%
\pgfpathmoveto{\pgfqpoint{9.000000in}{1.750000in}}%
\pgfpathlineto{\pgfqpoint{9.000000in}{3.754545in}}%
\pgfusepath{stroke}%
\end{pgfscope}%
\begin{pgfscope}%
\pgfsetrectcap%
\pgfsetmiterjoin%
\pgfsetlinewidth{0.803000pt}%
\definecolor{currentstroke}{rgb}{0.000000,0.000000,0.000000}%
\pgfsetstrokecolor{currentstroke}%
\pgfsetdash{}{0pt}%
\pgfpathmoveto{\pgfqpoint{6.720588in}{1.750000in}}%
\pgfpathlineto{\pgfqpoint{9.000000in}{1.750000in}}%
\pgfusepath{stroke}%
\end{pgfscope}%
\begin{pgfscope}%
\pgfsetrectcap%
\pgfsetmiterjoin%
\pgfsetlinewidth{0.803000pt}%
\definecolor{currentstroke}{rgb}{0.000000,0.000000,0.000000}%
\pgfsetstrokecolor{currentstroke}%
\pgfsetdash{}{0pt}%
\pgfpathmoveto{\pgfqpoint{6.720588in}{3.754545in}}%
\pgfpathlineto{\pgfqpoint{9.000000in}{3.754545in}}%
\pgfusepath{stroke}%
\end{pgfscope}%
\begin{pgfscope}%
\definecolor{textcolor}{rgb}{0.000000,0.000000,0.000000}%
\pgfsetstrokecolor{textcolor}%
\pgfsetfillcolor{textcolor}%
\pgftext[x=7.860294in,y=3.837879in,,base]{\color{textcolor}\sffamily\fontsize{12.000000}{14.400000}\selectfont f)}%
\end{pgfscope}%
\begin{pgfscope}%
\pgfsetbuttcap%
\pgfsetmiterjoin%
\definecolor{currentfill}{rgb}{1.000000,1.000000,1.000000}%
\pgfsetfillcolor{currentfill}%
\pgfsetlinewidth{0.000000pt}%
\definecolor{currentstroke}{rgb}{0.000000,0.000000,0.000000}%
\pgfsetstrokecolor{currentstroke}%
\pgfsetstrokeopacity{0.000000}%
\pgfsetdash{}{0pt}%
\pgfpathmoveto{\pgfqpoint{3.000000in}{0.700000in}}%
\pgfpathlineto{\pgfqpoint{5.000000in}{0.700000in}}%
\pgfpathlineto{\pgfqpoint{5.000000in}{1.050000in}}%
\pgfpathlineto{\pgfqpoint{3.000000in}{1.050000in}}%
\pgfpathlineto{\pgfqpoint{3.000000in}{0.700000in}}%
\pgfpathclose%
\pgfusepath{fill}%
\end{pgfscope}%
\begin{pgfscope}%
\pgfpathrectangle{\pgfqpoint{3.000000in}{0.700000in}}{\pgfqpoint{2.000000in}{0.350000in}}%
\pgfusepath{clip}%
\pgfsetbuttcap%
\pgfsetmiterjoin%
\definecolor{currentfill}{rgb}{1.000000,1.000000,1.000000}%
\pgfsetfillcolor{currentfill}%
\pgfsetlinewidth{0.010037pt}%
\definecolor{currentstroke}{rgb}{1.000000,1.000000,1.000000}%
\pgfsetstrokecolor{currentstroke}%
\pgfsetdash{}{0pt}%
\pgfusepath{stroke,fill}%
\end{pgfscope}%
\begin{pgfscope}%
\pgfsys@transformshift{3.000000in}{0.694444in}%
\pgftext[left,bottom]{\includegraphics[interpolate=true,width=2.000000in,height=0.361111in]{q_series-img6.png}}%
\end{pgfscope}%
\begin{pgfscope}%
\pgfsetbuttcap%
\pgfsetroundjoin%
\definecolor{currentfill}{rgb}{0.000000,0.000000,0.000000}%
\pgfsetfillcolor{currentfill}%
\pgfsetlinewidth{0.803000pt}%
\definecolor{currentstroke}{rgb}{0.000000,0.000000,0.000000}%
\pgfsetstrokecolor{currentstroke}%
\pgfsetdash{}{0pt}%
\pgfsys@defobject{currentmarker}{\pgfqpoint{0.000000in}{-0.048611in}}{\pgfqpoint{0.000000in}{0.000000in}}{%
\pgfpathmoveto{\pgfqpoint{0.000000in}{0.000000in}}%
\pgfpathlineto{\pgfqpoint{0.000000in}{-0.048611in}}%
\pgfusepath{stroke,fill}%
}%
\begin{pgfscope}%
\pgfsys@transformshift{3.000000in}{0.700000in}%
\pgfsys@useobject{currentmarker}{}%
\end{pgfscope}%
\end{pgfscope}%
\begin{pgfscope}%
\definecolor{textcolor}{rgb}{0.000000,0.000000,0.000000}%
\pgfsetstrokecolor{textcolor}%
\pgfsetfillcolor{textcolor}%
\pgftext[x=3.000000in,y=0.602778in,,top]{\color{textcolor}\sffamily\fontsize{10.000000}{12.000000}\selectfont \(\displaystyle {10^{-1}}\)}%
\end{pgfscope}%
\begin{pgfscope}%
\pgfsetbuttcap%
\pgfsetroundjoin%
\definecolor{currentfill}{rgb}{0.000000,0.000000,0.000000}%
\pgfsetfillcolor{currentfill}%
\pgfsetlinewidth{0.803000pt}%
\definecolor{currentstroke}{rgb}{0.000000,0.000000,0.000000}%
\pgfsetstrokecolor{currentstroke}%
\pgfsetdash{}{0pt}%
\pgfsys@defobject{currentmarker}{\pgfqpoint{0.000000in}{-0.048611in}}{\pgfqpoint{0.000000in}{0.000000in}}{%
\pgfpathmoveto{\pgfqpoint{0.000000in}{0.000000in}}%
\pgfpathlineto{\pgfqpoint{0.000000in}{-0.048611in}}%
\pgfusepath{stroke,fill}%
}%
\begin{pgfscope}%
\pgfsys@transformshift{4.000000in}{0.700000in}%
\pgfsys@useobject{currentmarker}{}%
\end{pgfscope}%
\end{pgfscope}%
\begin{pgfscope}%
\definecolor{textcolor}{rgb}{0.000000,0.000000,0.000000}%
\pgfsetstrokecolor{textcolor}%
\pgfsetfillcolor{textcolor}%
\pgftext[x=4.000000in,y=0.602778in,,top]{\color{textcolor}\sffamily\fontsize{10.000000}{12.000000}\selectfont \(\displaystyle {10^{0}}\)}%
\end{pgfscope}%
\begin{pgfscope}%
\pgfsetbuttcap%
\pgfsetroundjoin%
\definecolor{currentfill}{rgb}{0.000000,0.000000,0.000000}%
\pgfsetfillcolor{currentfill}%
\pgfsetlinewidth{0.803000pt}%
\definecolor{currentstroke}{rgb}{0.000000,0.000000,0.000000}%
\pgfsetstrokecolor{currentstroke}%
\pgfsetdash{}{0pt}%
\pgfsys@defobject{currentmarker}{\pgfqpoint{0.000000in}{-0.048611in}}{\pgfqpoint{0.000000in}{0.000000in}}{%
\pgfpathmoveto{\pgfqpoint{0.000000in}{0.000000in}}%
\pgfpathlineto{\pgfqpoint{0.000000in}{-0.048611in}}%
\pgfusepath{stroke,fill}%
}%
\begin{pgfscope}%
\pgfsys@transformshift{5.000000in}{0.700000in}%
\pgfsys@useobject{currentmarker}{}%
\end{pgfscope}%
\end{pgfscope}%
\begin{pgfscope}%
\definecolor{textcolor}{rgb}{0.000000,0.000000,0.000000}%
\pgfsetstrokecolor{textcolor}%
\pgfsetfillcolor{textcolor}%
\pgftext[x=5.000000in,y=0.602778in,,top]{\color{textcolor}\sffamily\fontsize{10.000000}{12.000000}\selectfont \(\displaystyle {10^{1}}\)}%
\end{pgfscope}%
\begin{pgfscope}%
\pgfsetbuttcap%
\pgfsetroundjoin%
\definecolor{currentfill}{rgb}{0.000000,0.000000,0.000000}%
\pgfsetfillcolor{currentfill}%
\pgfsetlinewidth{0.602250pt}%
\definecolor{currentstroke}{rgb}{0.000000,0.000000,0.000000}%
\pgfsetstrokecolor{currentstroke}%
\pgfsetdash{}{0pt}%
\pgfsys@defobject{currentmarker}{\pgfqpoint{0.000000in}{-0.027778in}}{\pgfqpoint{0.000000in}{0.000000in}}{%
\pgfpathmoveto{\pgfqpoint{0.000000in}{0.000000in}}%
\pgfpathlineto{\pgfqpoint{0.000000in}{-0.027778in}}%
\pgfusepath{stroke,fill}%
}%
\begin{pgfscope}%
\pgfsys@transformshift{3.301030in}{0.700000in}%
\pgfsys@useobject{currentmarker}{}%
\end{pgfscope}%
\end{pgfscope}%
\begin{pgfscope}%
\pgfsetbuttcap%
\pgfsetroundjoin%
\definecolor{currentfill}{rgb}{0.000000,0.000000,0.000000}%
\pgfsetfillcolor{currentfill}%
\pgfsetlinewidth{0.602250pt}%
\definecolor{currentstroke}{rgb}{0.000000,0.000000,0.000000}%
\pgfsetstrokecolor{currentstroke}%
\pgfsetdash{}{0pt}%
\pgfsys@defobject{currentmarker}{\pgfqpoint{0.000000in}{-0.027778in}}{\pgfqpoint{0.000000in}{0.000000in}}{%
\pgfpathmoveto{\pgfqpoint{0.000000in}{0.000000in}}%
\pgfpathlineto{\pgfqpoint{0.000000in}{-0.027778in}}%
\pgfusepath{stroke,fill}%
}%
\begin{pgfscope}%
\pgfsys@transformshift{3.477121in}{0.700000in}%
\pgfsys@useobject{currentmarker}{}%
\end{pgfscope}%
\end{pgfscope}%
\begin{pgfscope}%
\pgfsetbuttcap%
\pgfsetroundjoin%
\definecolor{currentfill}{rgb}{0.000000,0.000000,0.000000}%
\pgfsetfillcolor{currentfill}%
\pgfsetlinewidth{0.602250pt}%
\definecolor{currentstroke}{rgb}{0.000000,0.000000,0.000000}%
\pgfsetstrokecolor{currentstroke}%
\pgfsetdash{}{0pt}%
\pgfsys@defobject{currentmarker}{\pgfqpoint{0.000000in}{-0.027778in}}{\pgfqpoint{0.000000in}{0.000000in}}{%
\pgfpathmoveto{\pgfqpoint{0.000000in}{0.000000in}}%
\pgfpathlineto{\pgfqpoint{0.000000in}{-0.027778in}}%
\pgfusepath{stroke,fill}%
}%
\begin{pgfscope}%
\pgfsys@transformshift{3.602060in}{0.700000in}%
\pgfsys@useobject{currentmarker}{}%
\end{pgfscope}%
\end{pgfscope}%
\begin{pgfscope}%
\pgfsetbuttcap%
\pgfsetroundjoin%
\definecolor{currentfill}{rgb}{0.000000,0.000000,0.000000}%
\pgfsetfillcolor{currentfill}%
\pgfsetlinewidth{0.602250pt}%
\definecolor{currentstroke}{rgb}{0.000000,0.000000,0.000000}%
\pgfsetstrokecolor{currentstroke}%
\pgfsetdash{}{0pt}%
\pgfsys@defobject{currentmarker}{\pgfqpoint{0.000000in}{-0.027778in}}{\pgfqpoint{0.000000in}{0.000000in}}{%
\pgfpathmoveto{\pgfqpoint{0.000000in}{0.000000in}}%
\pgfpathlineto{\pgfqpoint{0.000000in}{-0.027778in}}%
\pgfusepath{stroke,fill}%
}%
\begin{pgfscope}%
\pgfsys@transformshift{3.698970in}{0.700000in}%
\pgfsys@useobject{currentmarker}{}%
\end{pgfscope}%
\end{pgfscope}%
\begin{pgfscope}%
\pgfsetbuttcap%
\pgfsetroundjoin%
\definecolor{currentfill}{rgb}{0.000000,0.000000,0.000000}%
\pgfsetfillcolor{currentfill}%
\pgfsetlinewidth{0.602250pt}%
\definecolor{currentstroke}{rgb}{0.000000,0.000000,0.000000}%
\pgfsetstrokecolor{currentstroke}%
\pgfsetdash{}{0pt}%
\pgfsys@defobject{currentmarker}{\pgfqpoint{0.000000in}{-0.027778in}}{\pgfqpoint{0.000000in}{0.000000in}}{%
\pgfpathmoveto{\pgfqpoint{0.000000in}{0.000000in}}%
\pgfpathlineto{\pgfqpoint{0.000000in}{-0.027778in}}%
\pgfusepath{stroke,fill}%
}%
\begin{pgfscope}%
\pgfsys@transformshift{3.778151in}{0.700000in}%
\pgfsys@useobject{currentmarker}{}%
\end{pgfscope}%
\end{pgfscope}%
\begin{pgfscope}%
\pgfsetbuttcap%
\pgfsetroundjoin%
\definecolor{currentfill}{rgb}{0.000000,0.000000,0.000000}%
\pgfsetfillcolor{currentfill}%
\pgfsetlinewidth{0.602250pt}%
\definecolor{currentstroke}{rgb}{0.000000,0.000000,0.000000}%
\pgfsetstrokecolor{currentstroke}%
\pgfsetdash{}{0pt}%
\pgfsys@defobject{currentmarker}{\pgfqpoint{0.000000in}{-0.027778in}}{\pgfqpoint{0.000000in}{0.000000in}}{%
\pgfpathmoveto{\pgfqpoint{0.000000in}{0.000000in}}%
\pgfpathlineto{\pgfqpoint{0.000000in}{-0.027778in}}%
\pgfusepath{stroke,fill}%
}%
\begin{pgfscope}%
\pgfsys@transformshift{3.845098in}{0.700000in}%
\pgfsys@useobject{currentmarker}{}%
\end{pgfscope}%
\end{pgfscope}%
\begin{pgfscope}%
\pgfsetbuttcap%
\pgfsetroundjoin%
\definecolor{currentfill}{rgb}{0.000000,0.000000,0.000000}%
\pgfsetfillcolor{currentfill}%
\pgfsetlinewidth{0.602250pt}%
\definecolor{currentstroke}{rgb}{0.000000,0.000000,0.000000}%
\pgfsetstrokecolor{currentstroke}%
\pgfsetdash{}{0pt}%
\pgfsys@defobject{currentmarker}{\pgfqpoint{0.000000in}{-0.027778in}}{\pgfqpoint{0.000000in}{0.000000in}}{%
\pgfpathmoveto{\pgfqpoint{0.000000in}{0.000000in}}%
\pgfpathlineto{\pgfqpoint{0.000000in}{-0.027778in}}%
\pgfusepath{stroke,fill}%
}%
\begin{pgfscope}%
\pgfsys@transformshift{3.903090in}{0.700000in}%
\pgfsys@useobject{currentmarker}{}%
\end{pgfscope}%
\end{pgfscope}%
\begin{pgfscope}%
\pgfsetbuttcap%
\pgfsetroundjoin%
\definecolor{currentfill}{rgb}{0.000000,0.000000,0.000000}%
\pgfsetfillcolor{currentfill}%
\pgfsetlinewidth{0.602250pt}%
\definecolor{currentstroke}{rgb}{0.000000,0.000000,0.000000}%
\pgfsetstrokecolor{currentstroke}%
\pgfsetdash{}{0pt}%
\pgfsys@defobject{currentmarker}{\pgfqpoint{0.000000in}{-0.027778in}}{\pgfqpoint{0.000000in}{0.000000in}}{%
\pgfpathmoveto{\pgfqpoint{0.000000in}{0.000000in}}%
\pgfpathlineto{\pgfqpoint{0.000000in}{-0.027778in}}%
\pgfusepath{stroke,fill}%
}%
\begin{pgfscope}%
\pgfsys@transformshift{3.954243in}{0.700000in}%
\pgfsys@useobject{currentmarker}{}%
\end{pgfscope}%
\end{pgfscope}%
\begin{pgfscope}%
\pgfsetbuttcap%
\pgfsetroundjoin%
\definecolor{currentfill}{rgb}{0.000000,0.000000,0.000000}%
\pgfsetfillcolor{currentfill}%
\pgfsetlinewidth{0.602250pt}%
\definecolor{currentstroke}{rgb}{0.000000,0.000000,0.000000}%
\pgfsetstrokecolor{currentstroke}%
\pgfsetdash{}{0pt}%
\pgfsys@defobject{currentmarker}{\pgfqpoint{0.000000in}{-0.027778in}}{\pgfqpoint{0.000000in}{0.000000in}}{%
\pgfpathmoveto{\pgfqpoint{0.000000in}{0.000000in}}%
\pgfpathlineto{\pgfqpoint{0.000000in}{-0.027778in}}%
\pgfusepath{stroke,fill}%
}%
\begin{pgfscope}%
\pgfsys@transformshift{4.301030in}{0.700000in}%
\pgfsys@useobject{currentmarker}{}%
\end{pgfscope}%
\end{pgfscope}%
\begin{pgfscope}%
\pgfsetbuttcap%
\pgfsetroundjoin%
\definecolor{currentfill}{rgb}{0.000000,0.000000,0.000000}%
\pgfsetfillcolor{currentfill}%
\pgfsetlinewidth{0.602250pt}%
\definecolor{currentstroke}{rgb}{0.000000,0.000000,0.000000}%
\pgfsetstrokecolor{currentstroke}%
\pgfsetdash{}{0pt}%
\pgfsys@defobject{currentmarker}{\pgfqpoint{0.000000in}{-0.027778in}}{\pgfqpoint{0.000000in}{0.000000in}}{%
\pgfpathmoveto{\pgfqpoint{0.000000in}{0.000000in}}%
\pgfpathlineto{\pgfqpoint{0.000000in}{-0.027778in}}%
\pgfusepath{stroke,fill}%
}%
\begin{pgfscope}%
\pgfsys@transformshift{4.477121in}{0.700000in}%
\pgfsys@useobject{currentmarker}{}%
\end{pgfscope}%
\end{pgfscope}%
\begin{pgfscope}%
\pgfsetbuttcap%
\pgfsetroundjoin%
\definecolor{currentfill}{rgb}{0.000000,0.000000,0.000000}%
\pgfsetfillcolor{currentfill}%
\pgfsetlinewidth{0.602250pt}%
\definecolor{currentstroke}{rgb}{0.000000,0.000000,0.000000}%
\pgfsetstrokecolor{currentstroke}%
\pgfsetdash{}{0pt}%
\pgfsys@defobject{currentmarker}{\pgfqpoint{0.000000in}{-0.027778in}}{\pgfqpoint{0.000000in}{0.000000in}}{%
\pgfpathmoveto{\pgfqpoint{0.000000in}{0.000000in}}%
\pgfpathlineto{\pgfqpoint{0.000000in}{-0.027778in}}%
\pgfusepath{stroke,fill}%
}%
\begin{pgfscope}%
\pgfsys@transformshift{4.602060in}{0.700000in}%
\pgfsys@useobject{currentmarker}{}%
\end{pgfscope}%
\end{pgfscope}%
\begin{pgfscope}%
\pgfsetbuttcap%
\pgfsetroundjoin%
\definecolor{currentfill}{rgb}{0.000000,0.000000,0.000000}%
\pgfsetfillcolor{currentfill}%
\pgfsetlinewidth{0.602250pt}%
\definecolor{currentstroke}{rgb}{0.000000,0.000000,0.000000}%
\pgfsetstrokecolor{currentstroke}%
\pgfsetdash{}{0pt}%
\pgfsys@defobject{currentmarker}{\pgfqpoint{0.000000in}{-0.027778in}}{\pgfqpoint{0.000000in}{0.000000in}}{%
\pgfpathmoveto{\pgfqpoint{0.000000in}{0.000000in}}%
\pgfpathlineto{\pgfqpoint{0.000000in}{-0.027778in}}%
\pgfusepath{stroke,fill}%
}%
\begin{pgfscope}%
\pgfsys@transformshift{4.698970in}{0.700000in}%
\pgfsys@useobject{currentmarker}{}%
\end{pgfscope}%
\end{pgfscope}%
\begin{pgfscope}%
\pgfsetbuttcap%
\pgfsetroundjoin%
\definecolor{currentfill}{rgb}{0.000000,0.000000,0.000000}%
\pgfsetfillcolor{currentfill}%
\pgfsetlinewidth{0.602250pt}%
\definecolor{currentstroke}{rgb}{0.000000,0.000000,0.000000}%
\pgfsetstrokecolor{currentstroke}%
\pgfsetdash{}{0pt}%
\pgfsys@defobject{currentmarker}{\pgfqpoint{0.000000in}{-0.027778in}}{\pgfqpoint{0.000000in}{0.000000in}}{%
\pgfpathmoveto{\pgfqpoint{0.000000in}{0.000000in}}%
\pgfpathlineto{\pgfqpoint{0.000000in}{-0.027778in}}%
\pgfusepath{stroke,fill}%
}%
\begin{pgfscope}%
\pgfsys@transformshift{4.778151in}{0.700000in}%
\pgfsys@useobject{currentmarker}{}%
\end{pgfscope}%
\end{pgfscope}%
\begin{pgfscope}%
\pgfsetbuttcap%
\pgfsetroundjoin%
\definecolor{currentfill}{rgb}{0.000000,0.000000,0.000000}%
\pgfsetfillcolor{currentfill}%
\pgfsetlinewidth{0.602250pt}%
\definecolor{currentstroke}{rgb}{0.000000,0.000000,0.000000}%
\pgfsetstrokecolor{currentstroke}%
\pgfsetdash{}{0pt}%
\pgfsys@defobject{currentmarker}{\pgfqpoint{0.000000in}{-0.027778in}}{\pgfqpoint{0.000000in}{0.000000in}}{%
\pgfpathmoveto{\pgfqpoint{0.000000in}{0.000000in}}%
\pgfpathlineto{\pgfqpoint{0.000000in}{-0.027778in}}%
\pgfusepath{stroke,fill}%
}%
\begin{pgfscope}%
\pgfsys@transformshift{4.845098in}{0.700000in}%
\pgfsys@useobject{currentmarker}{}%
\end{pgfscope}%
\end{pgfscope}%
\begin{pgfscope}%
\pgfsetbuttcap%
\pgfsetroundjoin%
\definecolor{currentfill}{rgb}{0.000000,0.000000,0.000000}%
\pgfsetfillcolor{currentfill}%
\pgfsetlinewidth{0.602250pt}%
\definecolor{currentstroke}{rgb}{0.000000,0.000000,0.000000}%
\pgfsetstrokecolor{currentstroke}%
\pgfsetdash{}{0pt}%
\pgfsys@defobject{currentmarker}{\pgfqpoint{0.000000in}{-0.027778in}}{\pgfqpoint{0.000000in}{0.000000in}}{%
\pgfpathmoveto{\pgfqpoint{0.000000in}{0.000000in}}%
\pgfpathlineto{\pgfqpoint{0.000000in}{-0.027778in}}%
\pgfusepath{stroke,fill}%
}%
\begin{pgfscope}%
\pgfsys@transformshift{4.903090in}{0.700000in}%
\pgfsys@useobject{currentmarker}{}%
\end{pgfscope}%
\end{pgfscope}%
\begin{pgfscope}%
\pgfsetbuttcap%
\pgfsetroundjoin%
\definecolor{currentfill}{rgb}{0.000000,0.000000,0.000000}%
\pgfsetfillcolor{currentfill}%
\pgfsetlinewidth{0.602250pt}%
\definecolor{currentstroke}{rgb}{0.000000,0.000000,0.000000}%
\pgfsetstrokecolor{currentstroke}%
\pgfsetdash{}{0pt}%
\pgfsys@defobject{currentmarker}{\pgfqpoint{0.000000in}{-0.027778in}}{\pgfqpoint{0.000000in}{0.000000in}}{%
\pgfpathmoveto{\pgfqpoint{0.000000in}{0.000000in}}%
\pgfpathlineto{\pgfqpoint{0.000000in}{-0.027778in}}%
\pgfusepath{stroke,fill}%
}%
\begin{pgfscope}%
\pgfsys@transformshift{4.954243in}{0.700000in}%
\pgfsys@useobject{currentmarker}{}%
\end{pgfscope}%
\end{pgfscope}%
\begin{pgfscope}%
\definecolor{textcolor}{rgb}{0.000000,0.000000,0.000000}%
\pgfsetstrokecolor{textcolor}%
\pgfsetfillcolor{textcolor}%
\pgftext[x=4.000000in,y=0.423766in,,top]{\color{textcolor}\sffamily\fontsize{10.000000}{12.000000}\selectfont \(\displaystyle dQ/dy/dz \, \mathrm{[pC/\mu m^2]}\)}%
\end{pgfscope}%
\begin{pgfscope}%
\pgfsetrectcap%
\pgfsetmiterjoin%
\pgfsetlinewidth{0.803000pt}%
\definecolor{currentstroke}{rgb}{0.000000,0.000000,0.000000}%
\pgfsetstrokecolor{currentstroke}%
\pgfsetdash{}{0pt}%
\pgfpathmoveto{\pgfqpoint{3.000000in}{0.700000in}}%
\pgfpathlineto{\pgfqpoint{3.000000in}{0.875000in}}%
\pgfpathlineto{\pgfqpoint{3.000000in}{1.050000in}}%
\pgfpathlineto{\pgfqpoint{5.000000in}{1.050000in}}%
\pgfpathlineto{\pgfqpoint{5.000000in}{0.875000in}}%
\pgfpathlineto{\pgfqpoint{5.000000in}{0.700000in}}%
\pgfpathlineto{\pgfqpoint{3.000000in}{0.700000in}}%
\pgfpathclose%
\pgfusepath{stroke}%
\end{pgfscope}%
\begin{pgfscope}%
\pgfsetbuttcap%
\pgfsetmiterjoin%
\definecolor{currentfill}{rgb}{1.000000,1.000000,1.000000}%
\pgfsetfillcolor{currentfill}%
\pgfsetlinewidth{0.000000pt}%
\definecolor{currentstroke}{rgb}{0.000000,0.000000,0.000000}%
\pgfsetstrokecolor{currentstroke}%
\pgfsetstrokeopacity{0.000000}%
\pgfsetdash{}{0pt}%
\pgfpathmoveto{\pgfqpoint{5.500000in}{0.700000in}}%
\pgfpathlineto{\pgfqpoint{7.500000in}{0.700000in}}%
\pgfpathlineto{\pgfqpoint{7.500000in}{1.050000in}}%
\pgfpathlineto{\pgfqpoint{5.500000in}{1.050000in}}%
\pgfpathlineto{\pgfqpoint{5.500000in}{0.700000in}}%
\pgfpathclose%
\pgfusepath{fill}%
\end{pgfscope}%
\begin{pgfscope}%
\pgfpathrectangle{\pgfqpoint{5.500000in}{0.700000in}}{\pgfqpoint{2.000000in}{0.350000in}}%
\pgfusepath{clip}%
\pgfsetbuttcap%
\pgfsetmiterjoin%
\definecolor{currentfill}{rgb}{1.000000,1.000000,1.000000}%
\pgfsetfillcolor{currentfill}%
\pgfsetlinewidth{0.010037pt}%
\definecolor{currentstroke}{rgb}{1.000000,1.000000,1.000000}%
\pgfsetstrokecolor{currentstroke}%
\pgfsetdash{}{0pt}%
\pgfusepath{stroke,fill}%
\end{pgfscope}%
\begin{pgfscope}%
\pgfsys@transformshift{5.500000in}{0.694444in}%
\pgftext[left,bottom]{\includegraphics[interpolate=true,width=2.000000in,height=0.361111in]{q_series-img7.png}}%
\end{pgfscope}%
\begin{pgfscope}%
\pgfsetbuttcap%
\pgfsetroundjoin%
\definecolor{currentfill}{rgb}{0.000000,0.000000,0.000000}%
\pgfsetfillcolor{currentfill}%
\pgfsetlinewidth{0.803000pt}%
\definecolor{currentstroke}{rgb}{0.000000,0.000000,0.000000}%
\pgfsetstrokecolor{currentstroke}%
\pgfsetdash{}{0pt}%
\pgfsys@defobject{currentmarker}{\pgfqpoint{0.000000in}{-0.048611in}}{\pgfqpoint{0.000000in}{0.000000in}}{%
\pgfpathmoveto{\pgfqpoint{0.000000in}{0.000000in}}%
\pgfpathlineto{\pgfqpoint{0.000000in}{-0.048611in}}%
\pgfusepath{stroke,fill}%
}%
\begin{pgfscope}%
\pgfsys@transformshift{5.500000in}{0.700000in}%
\pgfsys@useobject{currentmarker}{}%
\end{pgfscope}%
\end{pgfscope}%
\begin{pgfscope}%
\definecolor{textcolor}{rgb}{0.000000,0.000000,0.000000}%
\pgfsetstrokecolor{textcolor}%
\pgfsetfillcolor{textcolor}%
\pgftext[x=5.500000in,y=0.602778in,,top]{\color{textcolor}\sffamily\fontsize{10.000000}{12.000000}\selectfont \(\displaystyle {0}\)}%
\end{pgfscope}%
\begin{pgfscope}%
\pgfsetbuttcap%
\pgfsetroundjoin%
\definecolor{currentfill}{rgb}{0.000000,0.000000,0.000000}%
\pgfsetfillcolor{currentfill}%
\pgfsetlinewidth{0.803000pt}%
\definecolor{currentstroke}{rgb}{0.000000,0.000000,0.000000}%
\pgfsetstrokecolor{currentstroke}%
\pgfsetdash{}{0pt}%
\pgfsys@defobject{currentmarker}{\pgfqpoint{0.000000in}{-0.048611in}}{\pgfqpoint{0.000000in}{0.000000in}}{%
\pgfpathmoveto{\pgfqpoint{0.000000in}{0.000000in}}%
\pgfpathlineto{\pgfqpoint{0.000000in}{-0.048611in}}%
\pgfusepath{stroke,fill}%
}%
\begin{pgfscope}%
\pgfsys@transformshift{6.071429in}{0.700000in}%
\pgfsys@useobject{currentmarker}{}%
\end{pgfscope}%
\end{pgfscope}%
\begin{pgfscope}%
\definecolor{textcolor}{rgb}{0.000000,0.000000,0.000000}%
\pgfsetstrokecolor{textcolor}%
\pgfsetfillcolor{textcolor}%
\pgftext[x=6.071429in,y=0.602778in,,top]{\color{textcolor}\sffamily\fontsize{10.000000}{12.000000}\selectfont \(\displaystyle {200}\)}%
\end{pgfscope}%
\begin{pgfscope}%
\pgfsetbuttcap%
\pgfsetroundjoin%
\definecolor{currentfill}{rgb}{0.000000,0.000000,0.000000}%
\pgfsetfillcolor{currentfill}%
\pgfsetlinewidth{0.803000pt}%
\definecolor{currentstroke}{rgb}{0.000000,0.000000,0.000000}%
\pgfsetstrokecolor{currentstroke}%
\pgfsetdash{}{0pt}%
\pgfsys@defobject{currentmarker}{\pgfqpoint{0.000000in}{-0.048611in}}{\pgfqpoint{0.000000in}{0.000000in}}{%
\pgfpathmoveto{\pgfqpoint{0.000000in}{0.000000in}}%
\pgfpathlineto{\pgfqpoint{0.000000in}{-0.048611in}}%
\pgfusepath{stroke,fill}%
}%
\begin{pgfscope}%
\pgfsys@transformshift{6.642857in}{0.700000in}%
\pgfsys@useobject{currentmarker}{}%
\end{pgfscope}%
\end{pgfscope}%
\begin{pgfscope}%
\definecolor{textcolor}{rgb}{0.000000,0.000000,0.000000}%
\pgfsetstrokecolor{textcolor}%
\pgfsetfillcolor{textcolor}%
\pgftext[x=6.642857in,y=0.602778in,,top]{\color{textcolor}\sffamily\fontsize{10.000000}{12.000000}\selectfont \(\displaystyle {400}\)}%
\end{pgfscope}%
\begin{pgfscope}%
\pgfsetbuttcap%
\pgfsetroundjoin%
\definecolor{currentfill}{rgb}{0.000000,0.000000,0.000000}%
\pgfsetfillcolor{currentfill}%
\pgfsetlinewidth{0.803000pt}%
\definecolor{currentstroke}{rgb}{0.000000,0.000000,0.000000}%
\pgfsetstrokecolor{currentstroke}%
\pgfsetdash{}{0pt}%
\pgfsys@defobject{currentmarker}{\pgfqpoint{0.000000in}{-0.048611in}}{\pgfqpoint{0.000000in}{0.000000in}}{%
\pgfpathmoveto{\pgfqpoint{0.000000in}{0.000000in}}%
\pgfpathlineto{\pgfqpoint{0.000000in}{-0.048611in}}%
\pgfusepath{stroke,fill}%
}%
\begin{pgfscope}%
\pgfsys@transformshift{7.214286in}{0.700000in}%
\pgfsys@useobject{currentmarker}{}%
\end{pgfscope}%
\end{pgfscope}%
\begin{pgfscope}%
\definecolor{textcolor}{rgb}{0.000000,0.000000,0.000000}%
\pgfsetstrokecolor{textcolor}%
\pgfsetfillcolor{textcolor}%
\pgftext[x=7.214286in,y=0.602778in,,top]{\color{textcolor}\sffamily\fontsize{10.000000}{12.000000}\selectfont \(\displaystyle {600}\)}%
\end{pgfscope}%
\begin{pgfscope}%
\definecolor{textcolor}{rgb}{0.000000,0.000000,0.000000}%
\pgfsetstrokecolor{textcolor}%
\pgfsetfillcolor{textcolor}%
\pgftext[x=6.500000in,y=0.423766in,,top]{\color{textcolor}\sffamily\fontsize{10.000000}{12.000000}\selectfont \(\displaystyle F_{Lorentz} \, \mathrm{[mN]}\)}%
\end{pgfscope}%
\begin{pgfscope}%
\pgfsetrectcap%
\pgfsetmiterjoin%
\pgfsetlinewidth{0.803000pt}%
\definecolor{currentstroke}{rgb}{0.000000,0.000000,0.000000}%
\pgfsetstrokecolor{currentstroke}%
\pgfsetdash{}{0pt}%
\pgfpathmoveto{\pgfqpoint{5.500000in}{0.700000in}}%
\pgfpathlineto{\pgfqpoint{5.500000in}{0.875000in}}%
\pgfpathlineto{\pgfqpoint{5.500000in}{1.050000in}}%
\pgfpathlineto{\pgfqpoint{7.500000in}{1.050000in}}%
\pgfpathlineto{\pgfqpoint{7.500000in}{0.875000in}}%
\pgfpathlineto{\pgfqpoint{7.500000in}{0.700000in}}%
\pgfpathlineto{\pgfqpoint{5.500000in}{0.700000in}}%
\pgfpathclose%
\pgfusepath{stroke}%
\end{pgfscope}%
\end{pgfpicture}%
\makeatother%
\endgroup%

	\caption{Time series of a charge density histogram of the driver electrons in Log scale with the acting Lorentz Force drawn on top as vector lines. Vertical lines are drawn in to make change in length of the bunch better visible. 
	\textbf{a)} ($y=\qty{0.04}{mm}$) When entering the plasma. Still a Gaussian distribution, weak forces focus the backside.
	\textbf{b)} ($y=\qty{0.36}{mm}$) Formation of the tail from focusing forces. Decelerating forces grow.
	\textbf{c)} ($y=\qty{0.76}{mm}$) First wing spreads from tail. Strong decelerating forces on the backside.
	\textbf{d)} ($y=\qty{1.08}{mm}$) More wings emerge and spread. Still strong decelerating forces.
	\textbf{e)} ($y=\qty{2.90}{mm}$) Shortly before bunch breakup with visible elongation of the bunch. Only weak forces act.
	\textbf{f)} ($y=\qty{3.38}{mm}$) Bunch after breakup. The fallen back part gets accelerated from the backside of the first cavity.}
	\label{fig:q_series}
\end{figure}
A log scale is chosen to make the borders with low density visible. Additionally, the Lorentz force is layered over the histogram to visualize the cause of the transformation of the driver. It is retrieved from the experienced $\vec{E}$- and $\vec{B}$-field, that every macroparticle stores.
The window is then separated into bins and for every bin the mean of the force has been calculated and plotted as a force field, with the color and width of the lines quantifying the absolute force. 
This corresponds to a case where there only is one macroparticle per bin and the force acting on this particle is plotted.

At start, the spacial distribution still follows a 2D Gauss, as set in \autoref{chap:init}. After some distance was traveled in the plasma, the first cavities start to arise. The focusing Lorentz force of these cavities forms a tail at the end of the driver while the cavities are still in the linear regime.
Only small forces act on the front of the beam, so the front part can diverge freely to the borders. In the center and back there are great acting forces, pushing the particles back and simultaneously centering them, resulting in the creation of the tail.
These forces result from the formation of the first cavities. As can be seen in \autoref{fig:cavity}b, the first cavity forms directly behind the center of the driver so the backside already experiences the decelerating and focusing fields of the cavity.
\begin{figure}
	\centering
	\missingfigure{}
	\caption{Time series of the charge density of the plasma and the driver.
	\textbf{a)} ($y=\qty{0.04}{mm}$) After entering plasma. The vacuum can still be seen on the left border. Weak cavity can already be seen.
	\textbf{b)} ($y=\qty{0.36}{mm}$) Cavities forming together with the tail of the driver.
	\textbf{c)} ($y=\qty{0.68}{mm}$) Blowout regime during spread of the wings on the backside of the driver.
	\textbf{d)} ($y=\qty{2.27}{mm}$) Wakefield returns to linear regime before bunch breakup.}
	\label{fig:cavity}
\end{figure}
Comparing the length of the beam over time (see vertical lines in \autoref{fig:q_series}) shows that the tail is not an elongation of the driver backwards but the backside experiencing focusing forces, which is narrowing it.
Looking back at the cavity formation shows that with tail formation, cavities with similar width emerge and start to form the blowout regime.

The focus forces in the back of the driver cause the electrons to overshoot. This results in widening of the tail and the formation of wing-like structures. The cavities also start to widen and form the strongest electric fields of the whole \gls{pwfa} stage during this process.
The formed wings are diverging while new wings form behind them by particles which oscillate as the got pulled back by the focusing force, causing them to overshoot again. This forms a chain of smaller wings, all spreading with time and broadening the tail further.
In this stage, the strength of the electric fields in the cavities already decreases while the electron density in the wakefield rises.

The backside of the driver spreads further as the impact of the focusing fields is left by the particles. This causes it to grow in tranversal direction together with the front of the bunch. 
Meanwhile the $\vec{E}$-fields of the wakfields lose in strength and the cavities are flood by electrons again. The blowout regime goes over into the linear regime, as seen in \autoref{fig:cavity}d.
Still there are strong forces acting on the center of the bunch, causing it to lose energy. The discussion of this energy loss is continued in \autoref{chap:E_shift}, for now we only look at the distribution of these particles.

In \autoref{fig:force_time} the strength and and position change of the longitudinal Lorentz-force can be seen over traversed distance. It shows how the force pushing the driver back first builds up and then loses it's strength after having traveled long enough.
\begin{figure}
	\centering
	\missingfigure{}
	\caption{Histogram of the longitudinal part of the Lorentz force. The force is sampled over the $\zeta$-direction at a slice in the middle in $z$-direction for every 2000 timesteps. This slice is here plotted over the distance of the driver from the start of the plasma, showing how it changes in position and strength.}
	\label{fig:force_time}
\end{figure}
A forward pushing force on the backside of the driver can also be seen, as the Gaussian distribution of the driver reaches so far back, that some particles are positioned in the accelerating part of the first cavity.
Notable is the part around $y=\qty{3}{mm}$, where the bunch collapses and particles start to fall back rapidly, so they get accelerated again in the back of the first cavity. These particles stem mostly from the middle of the driver,
where the strongest backwards-pushing forces acted before, causing them to rapidly lose energy. 

Here the bunch breakup has no big effect, as the cavities already resumed to the linear regime and only a small part of the driver did fall back. Only small forces remain to act on the driver, so it slowly diverges to the borders of the box.

To support the claims made about the particle movement , particle tracing was used. In \autoref{fig:y_hist_0} the particle distribution, shortly after entering the plasma, can be seen.
\begin{figure}
	\centering
	\begin{subfigure}{0.45\textwidth}
	\centering
	\missingfigure{}
	\caption{} \label{fig:y_hist_0}
	\end{subfigure}
	\begin{subfigure}{0.45\textwidth}
	\centering
	\missingfigure{}
	\caption{} \label{fig:y_hist_time}
	\end{subfigure}
	\caption{Position of the particles in bins after bunch breakup. Each bin shows the minimal $\zeta$-position of the particles in it to visualize how it will move over time.
	\textbf{a)} 2D distribution of the bins for $\zeta=\qty{0.60}{\mm}$. The blue particles in the middle will fall back to the backside of the driver with time.
	\textbf{a)} The bins centered at $z=0$ are sampled every 2000 timesteps and plotted over the traversed distance $y$ in the plasma. Around $y=\qty{3}{mm}$ the particles in the middle fall behind.}
	\label{fig:y_hist}
\end{figure}
The colors show the $\zeta$-position of the respective particles after bunch breakup. Driver electrons, which fall behind during bunch breakup originate from a small region right behind the center of the bunch. 
As seen in \autoref{fig:q_series}, this is the region where the strongest Lorentz force acted.
The $\zeta$ position change of other particles is neglectable small, as only weak forces act elsewhere.
Additionally in \autoref{fig:y_hist_time} the middle slice of this distribution was captured every 2000 timesteps and plotted over the traversed distance. The fall back of the particles in the middle during bunch breakup can be clearly seen.

The same tracking was done for the $z$-direction, seen in the time series in \autoref{fig:z_hist_0}.
\begin{figure}
	\centering
	\missingfigure{}
	\caption{$z$-positions of the particles before entering plasma plotted over there current position. Each bin represents the mean $z$-value at a timestep. }
	\label{fig:z_hist_0}
\end{figure}
As the movement in z-direction is symmetric, this results in bins with equal amount of particles from the top and bottom half to be displayed as middle white. Still the effect is clear, as the wings can clearly be seen at the end of the driver.
The alternating colors can be explained by the oscillation of the electrons, as they get pulled back to the center and form smaller wings. No such motion is observed in the front part in the driver.


\subsection{Parameter comparison}
The transformation of the driver beam form and subsequent changes of quality of the wakefield were compared for different initial parameters. To quantify this quality, the maximal gained energy 
for a potential witness beam is considered. In \autoref{fig:E_y_hist} the charge distribution and $\zeta$-direction of the electric field of the plasma electrons are compared over time. 
\begin{figure}
	\centering
	\missingfigure{}
	\caption{
	\textbf{a)} The charge distribution for all grid points at $x=z=0$ is plotted over the distance in plasma $y$. Clearly visible is a change in the width of the cavities.
	\textbf{b)}The electric field in $\zeta$ direction for all grid points at $x=z=0$ is plotted over the distance in plasma $y$. On the right side the gained energy for a theoretical
	witness beam is plotted too over $\zeta$}
	\label{fig:E_y_hist}
\end{figure}
Especially in a), the change in width of the cavities is visible. This effect was measured in experiment, as seen in \cite{Schoebel2022}. A witness beam which is phase locked with the driver, meaning it is constant in $\zeta$,
would therefore not get maximal energy gain when placed at a position where the accelerating field is maximal as it would later experience a decelerating force, when the cavities shrink again.
In b) this is visible, as the decelerating blue field moves to the right over the position of the strongest red accelerating field. Drawn in is also the expected energy gain for a potential witness beam.
This gain was calculated by integrating the Lorentz Force, created by the fields, over the traversed distance. The peak is not over the position with the highest field but at the position where the highest field is after the cavities shrank.
This peak gained energy can now be used to compare the wakefields for different drivers.

\paragraph*{Energy comparison}\hspace{0pt} \\
First, a comparison between different initial mean kinetic energies of the drivers with same divergence were made. We compare the three energies \qtylist{250; 300; 350}{\MeV} under the aspect how much more energy can be gained with higher initial energy.
Only small qualitative differences exist in the wakefields between the three energies. Generally, the blowout regime can be achieved over a longer distance for higher energy drivers.
This results in slightly increased energy gains, as seen in \autoref{fig:gain_E}. The maximal gainable energies can be found in \autoref{tab:gain_E}.
\begin{table}[h]
\begin{center}
\begin{tabular}{|c|c|} 
	\hline
 	$E_{kin} \, \mathrm{[MeV]}$ & $E_{gain} \, \mathrm{[MeV]}$ \\ 
 	\hline
	250 & 439.7 \\ 
 	300 & 502.6 \\
	350 & 518.9 \\
	\hline
\end{tabular}
\caption{Maximal possible energy gain for different initial kinetic energies.}\label{tab:gain_E}
\end{center}
\end{table}

\begin{figure}
	\centering
	\missingfigure{}
	\caption{Energy gain curves for three different initial kinetic energies.}
	\label{fig:gain_E}
\end{figure}

While the jump from \qtyrange{250}{300}{\MeV} results in \qty{60}{\MeV} higher gains, there is a diminishing return, as only \qty{20}{\MeV} more are reached when increasing from \qtyrange{300}{350}{\MeV}. Real experiments are restricted by the length of the plasma jet, 
resulting in even smaller gain increase, as the main difference between the energies is the duration of the cavities. Therefore, there may be no need for higher energy drivers, as the advantages vanish.
The values make also apparent, that there are only small differences in the energy loss behind the driver (first minimum behind \qty{0}{\um}). The driver with higher energy lives only longer, because it has more energy to lose, not because it looses less per time.

\paragraph*{Divergence comparison}\label{para:div_comp}\hspace{0pt} \\
Next, the driver qualities for three different divergences after passing the metal foil were compared. The energy gains over $\zeta$ for a high divergence ($\theta=\qty{8.7}{\mrad}$), middle divergence ($\theta=\qty{4.2}{\mrad}$, as seen in experiments) 
and low divergence ($\theta=\qty{1.7}{\mrad}$) beam are shown in \autoref{fig:gain_div}.
\begin{figure}
	\centering
	\missingfigure{}
	\caption{Energy gain curves for three different initial divergences.}
	\label{fig:gain_div}
\end{figure}
For all curves a normal maximum like in the graphs before can be seen. Additionally, there is a second much higher peak for the low divergence curve in the backside of the cavity with blowout length.
This results from the long standing blowout with extreme fields and the following smaller blowout, as seen in the charge density graphs \autoref{fig:cavity_low}.

\begin{figure}
	\centering
	\missingfigure{todo: make a to density over time }
	\caption{Charge density of driver and plasma for a low divergent driver for different time steps.
	\textbf{a)} ($y=\qty{1.31}{mm}$) Blowout regime with extremely low plasma densities.
	\textbf{b)} ($y=\qty{3.22}{mm}$) Small cavity formed behind the broken up part of the driver.
	}
	\label{fig:cavity_low}
\end{figure}
Even before the cavities fill and a linear regime sets in, the driver breaks up as the strong forces cause fast energy drain (see the high energy loss in the first minimum behind the driver in \autoref{fig:gain_dive}. After the breakup, the linear regime starts but a small cavity still remains in blowout, induced by a fallen back part of the driver.
This so called beam loading\todo{citation needed} allows for high accelerating fields even after breakup.

For a driver with high divergence, no blowout is achieved so the wakefield remains in the linear regime with weak electric fields. This weak fields are not able to cause a bunch breakup, resulting in the bunch just diverging to the border with time.
The resulting maximal gained energies can be found in \autoref{tab:gain_div}.
\begin{table}[h]
\begin{center}
\begin{tabular}{|c|c|} 
	\hline
 	$divergence \, \mathrm{[mrad]}$ & $E_{gain} \, \mathrm{[MeV]}$ \\ 
 	\hline
	1.7 & 359.3 \\ 
	1.7 (sec. peak) & 840.7 \\ 
 	4.2 & 439.7 \\
	8.7 & 176.9 \\
	\hline
\end{tabular}
\caption{Maximal possible energy gain for different initial divergences.}\label{tab:gain_div}
\end{center}
\end{table}
While the normal maximum is highest for the \qty{4.2}{\mrad} driver, the second maximum of the low divergence driver achieves nearly double the energy gain, giving good reasons to strive for low divergent beams.
 
The question arises, why the change in divergence causes such a extreme difference for drivers. Possible is, that the induced current of the low divergence driver is higher, as the velocity in propagation direction is higher and the spatial distribution
is denser, as the particles don't diverge as much in the time before entering the plasma. 


\paragraph*{Distribution comparison}\hspace{0pt} \\
Even though it is hard to control the from of a beam leaving the \gls{lwfa} for our \gls{pwfa}, a comparison between different distributions can give new insights into the properties that are needed from a driver
to form high accelerating fields. Besides the driver with Gaussian distribution in all spatial direction, a driver with Gaussian distribution in transversal direction and uniform charge distribution in propagation direction 
was therefore simulated.

The energy gain graph \todo{add graph} shows only a small win of \qty{14}{\MeV} in maximal energy can be achieved compared to our Gaussian driver. The biggest difference in the two curves lies
in the higher minimum positioned at the backside of the driver, that results in smaller energy loss and therefore a \qty{0.7}{mm} longer traversed distance of before breakup.

In fact, \todo{add energy and force plot} shows that the uniform driver survives a millimeter longer in the plasma before breakup.
Also no particles which gain energy in the back can be seen, as for the uniform driver there are no particles so far back, that they get accelerated. This results from the driver having a high density in the $x$-$z$-plane center over the whole distribution and not just
in the middle in $\zeta$-direction. Still the later breakup gains only a small plus in maximal energy gain.

\section{Peak Energy shift} \label{chap:E_shift}
In this the chapter the change in energy of the \gls{pwfa} driver is discussed, as this gives further insights into the its stability and therefore also the stability of the wakefield.
The change of the energy over time can be seen in \autoref{fig:E_hist_time}. It shows how the energy-histogram changes over time. As set up, the energy starts as a Gaussian distribution at around \qty{250}{\MeV}.
\begin{figure}
	\centering
	\begin{subfigure}{0.5\textwidth}
	\centering
	\missingfigure{}
	\caption{} \label{fig:E_hist_time}
	\end{subfigure}
	\hfill
	\begin{subfigure}{0.5\textwidth}
	\centering
	\missingfigure{}
	\caption{} \label{fig:E_peak}
	\end{subfigure}
	\caption{(a) Energy histogram over time. Every 2000 timesteps a histogram of the charge over the energy is made and plotted here over the time.
	(b) Peak energy plotted over time.}
	\label{fig:Energy}
\end{figure}
When inducing the wakefield, the Lorentz force, created by the fields of the first cavity, act on the driver causing it to lose energy. As it acts only on parts of the driver, the energy distribution is not moving to lower energies but instead growing in size.
The plot shows the growth in low energy electrons until $E=0$ is reached. In orange, the mean is also plotted in,  as well as the electrons with the minimal energy, both sinking with time. Also shown is the maximum energy, which actually grows.
These are particles, which gain energy, as they are so far back in the bunch, that they get pushed by the Lorentz forces in the middle of the first cavity.

Notable is the fact, that the histogram for every timestep isn't uniform but has visible maximas and minimas. The maximum with the highest energy is called the peak energy, which is also plotted in red.
This energy is notable, as the fact, that it stays constant during the \gls{pwfa} is used in experiments to calculate the initial charge the driver had, before entering the \gls{pwfa} (see \cite{Schoebel2022}.
In \autoref{fig:E_peak} only the peak energy is plotted over time, showing that it isn't constant. Directly after entering the plasma, the peak energy drops from \qtyrange{250}{244}{\MeV}.
Here, it plateaus until bunch breakup, after which it further drops.

The systematic uncertainty of the peak energy is given by the size of the bins while the statistical uncertainty results from the uncertainty of my fit. \todo{add fit image somewhere} The fit assumes two summed Gaussian distributions as as
simplification of the real distribution, which consists of multiple peaks with different heights and widths as well as a unknown background noise. Only the peak energy and the second peak are fitted.
Also drawn in is the curve for a simulation with a down ramp for the plasma after \qty{2.7}{\mm} of traversal, as peak energy is generally measured after leaving the plasma jet. The drop off of \qty{5}{\MeV} remains after 
the downramp and could therefore be measured in experiment. Even after the plasma was left by the driver, the peak energy further dropped. No significant decelerating forces act in this phase so the loss may result
from peak energy particles leaving the simulation box perpendicular to propagation direction. Even when the peak in the energy histogram should only decrease in size but not position, this would result in inaccuracies of the energy fit,
causing the energy loss.

Currently the uncertainty on the peak energy measurement in experiments is higher than the \qty{5}{\MeV} jump. Still this indicates, caution is needed when assuming a constant peak energy, especially after bunch breakup.

\subsection{Parameter comparison}
This results are again compared to different initial conditions of the driver. In \todo{add peak energy over time plot but only for energy loss} the loss of peak energy over traveled distance is compared again for three different initial energies.
For all energies the initial energy drop of \qtyrange{6}{7}{\MeV} can be seen, as well as the slower loss of peak energy afterwards. The initial drop off ends for all energies after a plasma distance of \qty{1}{\mm}, with the high energy drivers having higher energy losses. 
While the \qty{250}{\MeV} driver has a visible plateau, there are smaller or no plateaus for the other two energies. Still the energy loss happens slower at the point after the initial drop off and only increases with progressing distance in the plasma.
The slow energy loss afterwards exists for all drivers. While the distance between the \qty{250}{\MeV} and \qty{300}{\MeV} curve stays roughly the same except for some fluctuations, 
the distance between the \qty{300}{\MeV} and \qty{350}{\MeV} curve increases with time, meaning that the \qty{350}{\MeV} peak loses its energy faster. Drawn in are also the point of bunch breakup, which seems to be uncorrelated to the progression of the curve.

Even when there is no plateau for higher initial energies, the energy loss beyond the initial drop off is comparably small before the end of the gas jet would be achieved in experiments at around \qty{3}{\mm} \cite{Schoebel2022}.


When the same analysis is done on the different initial divergences, as seen in \todo{add figure}, big qualitative differences become apparent. The low divergence driver has the same drop off as the one with normal divergence but afterwards
there is no further energy loss but instead a small increase of roughly \qty{0.5}{\MeV}. For the high divergence driver the initial drop off is smaller with only \qty{4}{\MeV} but afterwards the peak energy decreases linear. After \qty{3}{\mm}
it already has lost \qty{9}{\MeV} of energy. \todo{higher than experimental uncertainty?}

\section{Locality of the energy}\label{chap:loc_E}
The energy loss and spacial locality of the energy peaks can be visualized by again binning the macroparticles in space and analyze the mean energy in every bin. This mean energy is then plotted over the positions of the bins,
an example of this can be seen in \autoref{fig:E_map}. 
\begin{figure}
	\centering
	\missingfigure{}
	\caption{Mean energy of the particles in the bins. Every bin has the size .... Every bin with at least one particle is plotted, so to derive information about the energy density, the charge density from \autoref{fig:q_series} has to be considered to.}
	\label{fig:E_map}
\end{figure}

Parts of the bunch where a strong decelerating force acts have lost most of the energy while the front part, where only weak forces act, retained their peak energy. The energy loss in the back is also not a steady gradient but instead bigger areas where the energy is more or less constant, 
for example the yellow area marking the second energy peak.

The complete time series of the energy is plotted in \autoref{fig:E_time}.
\begin{figure}
	\centering
	\missingfigure{}
	\caption{Mean energy of the particles in the bins over time. Only the bins around $z=0$ are plotted for every timestep, showing the gradually loss of energy in the middle of the driver until bunch breakup.}
	\label{fig:E_time}
\end{figure}
Notable is the constant energy in the front half of the driver and the strong energy loss in the middle. When energies around zero are reached, the electrons fall back, where they get accelerated again from the first cavity.
Also notable is, that the highest energies are achieved at the back of the driver, where the accelerating part of the cavity starts and pushes the electrons there.

\end{document}