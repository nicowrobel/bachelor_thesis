\documentclass[bachelor_thesis]{subfiles}

\begin{document}
\chapter{Analysis of the bunch characteristics}\label{chap:analysis}
In this thesis, the analysis of the bunch and the resulting wakefields are split into two parts. In \autoref{chap:distro_change} the change of the spatial charge distribution of the driver is discussed 
as well as its effect on the quality of the produced wakefield. The latter is quantified by the maximal gainable energy for a potential witness beam.

The second part in \autoref{chap:E_shift} focuses on the energy distribution of the driver, especially on changes to the peak energy, 
as larger changes would violate the assumption of constant peak energy over the course of the \gls{pwfa}, made in experiments \cite{Schoebel2022}.

In both sections, analysis is initially done for a driver with Gaussian charge distribution, an initial kinetic energy of \qty{250}{\MeV} and a divergence of \qty{4.2}{\mrad}. 
This driver is then compared to drivers with different charge distributions, initial energy or divergence, highlighting the effects on the created wakefield. 

\begin{figure}
	\centering
	%% Creator: Matplotlib, PGF backend
%%
%% To include the figure in your LaTeX document, write
%%   \input{<filename>.pgf}
%%
%% Make sure the required packages are loaded in your preamble
%%   \usepackage{pgf}
%%
%% Also ensure that all the required font packages are loaded; for instance,
%% the lmodern package is sometimes necessary when using math font.
%%   \usepackage{lmodern}
%%
%% Figures using additional raster images can only be included by \input if
%% they are in the same directory as the main LaTeX file. For loading figures
%% from other directories you can use the `import` package
%%   \usepackage{import}
%%
%% and then include the figures with
%%   \import{<path to file>}{<filename>.pgf}
%%
%% Matplotlib used the following preamble
%%
\begingroup%
\makeatletter%
\begin{pgfpicture}%
\pgfpathrectangle{\pgfpointorigin}{\pgfqpoint{6.400000in}{3.000000in}}%
\pgfusepath{use as bounding box, clip}%
\begin{pgfscope}%
\pgfsetbuttcap%
\pgfsetmiterjoin%
\pgfsetlinewidth{0.000000pt}%
\definecolor{currentstroke}{rgb}{1.000000,1.000000,1.000000}%
\pgfsetstrokecolor{currentstroke}%
\pgfsetstrokeopacity{0.000000}%
\pgfsetdash{}{0pt}%
\pgfpathmoveto{\pgfqpoint{0.000000in}{0.000000in}}%
\pgfpathlineto{\pgfqpoint{6.400000in}{0.000000in}}%
\pgfpathlineto{\pgfqpoint{6.400000in}{3.000000in}}%
\pgfpathlineto{\pgfqpoint{0.000000in}{3.000000in}}%
\pgfpathlineto{\pgfqpoint{0.000000in}{0.000000in}}%
\pgfpathclose%
\pgfusepath{}%
\end{pgfscope}%
\begin{pgfscope}%
\pgfsetbuttcap%
\pgfsetmiterjoin%
\definecolor{currentfill}{rgb}{1.000000,1.000000,1.000000}%
\pgfsetfillcolor{currentfill}%
\pgfsetlinewidth{0.000000pt}%
\definecolor{currentstroke}{rgb}{0.000000,0.000000,0.000000}%
\pgfsetstrokecolor{currentstroke}%
\pgfsetstrokeopacity{0.000000}%
\pgfsetdash{}{0pt}%
\pgfpathmoveto{\pgfqpoint{0.640000in}{0.420000in}}%
\pgfpathlineto{\pgfqpoint{5.760000in}{0.420000in}}%
\pgfpathlineto{\pgfqpoint{5.760000in}{2.820000in}}%
\pgfpathlineto{\pgfqpoint{0.640000in}{2.820000in}}%
\pgfpathlineto{\pgfqpoint{0.640000in}{0.420000in}}%
\pgfpathclose%
\pgfusepath{fill}%
\end{pgfscope}%
\begin{pgfscope}%
\pgfpathrectangle{\pgfqpoint{0.640000in}{0.420000in}}{\pgfqpoint{5.120000in}{2.400000in}}%
\pgfusepath{clip}%
\pgfsetbuttcap%
\pgfsetmiterjoin%
\definecolor{currentfill}{rgb}{0.121569,0.466667,0.705882}%
\pgfsetfillcolor{currentfill}%
\pgfsetlinewidth{1.003750pt}%
\definecolor{currentstroke}{rgb}{0.000000,0.000000,0.000000}%
\pgfsetstrokecolor{currentstroke}%
\pgfsetdash{}{0pt}%
\pgfpathmoveto{\pgfqpoint{4.684800in}{1.620000in}}%
\pgfpathlineto{\pgfqpoint{4.582400in}{1.512000in}}%
\pgfpathlineto{\pgfqpoint{4.582400in}{1.584000in}}%
\pgfpathlineto{\pgfqpoint{3.814400in}{1.584000in}}%
\pgfpathlineto{\pgfqpoint{3.814400in}{1.656000in}}%
\pgfpathlineto{\pgfqpoint{4.582400in}{1.656000in}}%
\pgfpathlineto{\pgfqpoint{4.582400in}{1.728000in}}%
\pgfpathlineto{\pgfqpoint{4.684800in}{1.620000in}}%
\pgfpathclose%
\pgfusepath{stroke,fill}%
\end{pgfscope}%
\begin{pgfscope}%
\pgfsetbuttcap%
\pgfsetroundjoin%
\definecolor{currentfill}{rgb}{0.000000,0.000000,0.000000}%
\pgfsetfillcolor{currentfill}%
\pgfsetlinewidth{0.803000pt}%
\definecolor{currentstroke}{rgb}{0.000000,0.000000,0.000000}%
\pgfsetstrokecolor{currentstroke}%
\pgfsetdash{}{0pt}%
\pgfsys@defobject{currentmarker}{\pgfqpoint{0.000000in}{-0.048611in}}{\pgfqpoint{0.000000in}{0.000000in}}{%
\pgfpathmoveto{\pgfqpoint{0.000000in}{0.000000in}}%
\pgfpathlineto{\pgfqpoint{0.000000in}{-0.048611in}}%
\pgfusepath{stroke,fill}%
}%
\begin{pgfscope}%
\pgfsys@transformshift{1.152000in}{0.420000in}%
\pgfsys@useobject{currentmarker}{}%
\end{pgfscope}%
\end{pgfscope}%
\begin{pgfscope}%
\definecolor{textcolor}{rgb}{0.000000,0.000000,0.000000}%
\pgfsetstrokecolor{textcolor}%
\pgfsetfillcolor{textcolor}%
\pgftext[x=1.152000in,y=0.322778in,,top]{\color{textcolor}\sffamily\fontsize{10.000000}{12.000000}\selectfont \(\displaystyle {0.80}\)}%
\end{pgfscope}%
\begin{pgfscope}%
\pgfsetbuttcap%
\pgfsetroundjoin%
\definecolor{currentfill}{rgb}{0.000000,0.000000,0.000000}%
\pgfsetfillcolor{currentfill}%
\pgfsetlinewidth{0.803000pt}%
\definecolor{currentstroke}{rgb}{0.000000,0.000000,0.000000}%
\pgfsetstrokecolor{currentstroke}%
\pgfsetdash{}{0pt}%
\pgfsys@defobject{currentmarker}{\pgfqpoint{0.000000in}{-0.048611in}}{\pgfqpoint{0.000000in}{0.000000in}}{%
\pgfpathmoveto{\pgfqpoint{0.000000in}{0.000000in}}%
\pgfpathlineto{\pgfqpoint{0.000000in}{-0.048611in}}%
\pgfusepath{stroke,fill}%
}%
\begin{pgfscope}%
\pgfsys@transformshift{2.176000in}{0.420000in}%
\pgfsys@useobject{currentmarker}{}%
\end{pgfscope}%
\end{pgfscope}%
\begin{pgfscope}%
\definecolor{textcolor}{rgb}{0.000000,0.000000,0.000000}%
\pgfsetstrokecolor{textcolor}%
\pgfsetfillcolor{textcolor}%
\pgftext[x=2.176000in,y=0.322778in,,top]{\color{textcolor}\sffamily\fontsize{10.000000}{12.000000}\selectfont \(\displaystyle {0.82}\)}%
\end{pgfscope}%
\begin{pgfscope}%
\pgfsetbuttcap%
\pgfsetroundjoin%
\definecolor{currentfill}{rgb}{0.000000,0.000000,0.000000}%
\pgfsetfillcolor{currentfill}%
\pgfsetlinewidth{0.803000pt}%
\definecolor{currentstroke}{rgb}{0.000000,0.000000,0.000000}%
\pgfsetstrokecolor{currentstroke}%
\pgfsetdash{}{0pt}%
\pgfsys@defobject{currentmarker}{\pgfqpoint{0.000000in}{-0.048611in}}{\pgfqpoint{0.000000in}{0.000000in}}{%
\pgfpathmoveto{\pgfqpoint{0.000000in}{0.000000in}}%
\pgfpathlineto{\pgfqpoint{0.000000in}{-0.048611in}}%
\pgfusepath{stroke,fill}%
}%
\begin{pgfscope}%
\pgfsys@transformshift{3.200000in}{0.420000in}%
\pgfsys@useobject{currentmarker}{}%
\end{pgfscope}%
\end{pgfscope}%
\begin{pgfscope}%
\definecolor{textcolor}{rgb}{0.000000,0.000000,0.000000}%
\pgfsetstrokecolor{textcolor}%
\pgfsetfillcolor{textcolor}%
\pgftext[x=3.200000in,y=0.322778in,,top]{\color{textcolor}\sffamily\fontsize{10.000000}{12.000000}\selectfont \(\displaystyle {0.84}\)}%
\end{pgfscope}%
\begin{pgfscope}%
\pgfsetbuttcap%
\pgfsetroundjoin%
\definecolor{currentfill}{rgb}{0.000000,0.000000,0.000000}%
\pgfsetfillcolor{currentfill}%
\pgfsetlinewidth{0.803000pt}%
\definecolor{currentstroke}{rgb}{0.000000,0.000000,0.000000}%
\pgfsetstrokecolor{currentstroke}%
\pgfsetdash{}{0pt}%
\pgfsys@defobject{currentmarker}{\pgfqpoint{0.000000in}{-0.048611in}}{\pgfqpoint{0.000000in}{0.000000in}}{%
\pgfpathmoveto{\pgfqpoint{0.000000in}{0.000000in}}%
\pgfpathlineto{\pgfqpoint{0.000000in}{-0.048611in}}%
\pgfusepath{stroke,fill}%
}%
\begin{pgfscope}%
\pgfsys@transformshift{4.224000in}{0.420000in}%
\pgfsys@useobject{currentmarker}{}%
\end{pgfscope}%
\end{pgfscope}%
\begin{pgfscope}%
\definecolor{textcolor}{rgb}{0.000000,0.000000,0.000000}%
\pgfsetstrokecolor{textcolor}%
\pgfsetfillcolor{textcolor}%
\pgftext[x=4.224000in,y=0.322778in,,top]{\color{textcolor}\sffamily\fontsize{10.000000}{12.000000}\selectfont \(\displaystyle {0.86}\)}%
\end{pgfscope}%
\begin{pgfscope}%
\pgfsetbuttcap%
\pgfsetroundjoin%
\definecolor{currentfill}{rgb}{0.000000,0.000000,0.000000}%
\pgfsetfillcolor{currentfill}%
\pgfsetlinewidth{0.803000pt}%
\definecolor{currentstroke}{rgb}{0.000000,0.000000,0.000000}%
\pgfsetstrokecolor{currentstroke}%
\pgfsetdash{}{0pt}%
\pgfsys@defobject{currentmarker}{\pgfqpoint{0.000000in}{-0.048611in}}{\pgfqpoint{0.000000in}{0.000000in}}{%
\pgfpathmoveto{\pgfqpoint{0.000000in}{0.000000in}}%
\pgfpathlineto{\pgfqpoint{0.000000in}{-0.048611in}}%
\pgfusepath{stroke,fill}%
}%
\begin{pgfscope}%
\pgfsys@transformshift{5.248000in}{0.420000in}%
\pgfsys@useobject{currentmarker}{}%
\end{pgfscope}%
\end{pgfscope}%
\begin{pgfscope}%
\definecolor{textcolor}{rgb}{0.000000,0.000000,0.000000}%
\pgfsetstrokecolor{textcolor}%
\pgfsetfillcolor{textcolor}%
\pgftext[x=5.248000in,y=0.322778in,,top]{\color{textcolor}\sffamily\fontsize{10.000000}{12.000000}\selectfont \(\displaystyle {0.88}\)}%
\end{pgfscope}%
\begin{pgfscope}%
\definecolor{textcolor}{rgb}{0.000000,0.000000,0.000000}%
\pgfsetstrokecolor{textcolor}%
\pgfsetfillcolor{textcolor}%
\pgftext[x=3.200000in,y=0.143766in,,top]{\color{textcolor}\sffamily\fontsize{10.000000}{12.000000}\selectfont \(\displaystyle y \, \mathrm{[mm]}\)}%
\end{pgfscope}%
\begin{pgfscope}%
\pgfsetrectcap%
\pgfsetmiterjoin%
\pgfsetlinewidth{0.803000pt}%
\definecolor{currentstroke}{rgb}{0.000000,0.000000,0.000000}%
\pgfsetstrokecolor{currentstroke}%
\pgfsetdash{}{0pt}%
\pgfpathmoveto{\pgfqpoint{0.640000in}{0.420000in}}%
\pgfpathlineto{\pgfqpoint{0.640000in}{2.820000in}}%
\pgfusepath{stroke}%
\end{pgfscope}%
\begin{pgfscope}%
\pgfsetrectcap%
\pgfsetmiterjoin%
\pgfsetlinewidth{0.803000pt}%
\definecolor{currentstroke}{rgb}{0.000000,0.000000,0.000000}%
\pgfsetstrokecolor{currentstroke}%
\pgfsetdash{}{0pt}%
\pgfpathmoveto{\pgfqpoint{5.760000in}{0.420000in}}%
\pgfpathlineto{\pgfqpoint{5.760000in}{2.820000in}}%
\pgfusepath{stroke}%
\end{pgfscope}%
\begin{pgfscope}%
\pgfsetrectcap%
\pgfsetmiterjoin%
\pgfsetlinewidth{0.803000pt}%
\definecolor{currentstroke}{rgb}{0.000000,0.000000,0.000000}%
\pgfsetstrokecolor{currentstroke}%
\pgfsetdash{}{0pt}%
\pgfpathmoveto{\pgfqpoint{0.640000in}{0.420000in}}%
\pgfpathlineto{\pgfqpoint{5.760000in}{0.420000in}}%
\pgfusepath{stroke}%
\end{pgfscope}%
\begin{pgfscope}%
\pgfsetrectcap%
\pgfsetmiterjoin%
\pgfsetlinewidth{0.803000pt}%
\definecolor{currentstroke}{rgb}{0.000000,0.000000,0.000000}%
\pgfsetstrokecolor{currentstroke}%
\pgfsetdash{}{0pt}%
\pgfpathmoveto{\pgfqpoint{0.640000in}{2.820000in}}%
\pgfpathlineto{\pgfqpoint{5.760000in}{2.820000in}}%
\pgfusepath{stroke}%
\end{pgfscope}%
\begin{pgfscope}%
\definecolor{textcolor}{rgb}{0.000000,0.000000,0.000000}%
\pgfsetstrokecolor{textcolor}%
\pgfsetfillcolor{textcolor}%
\pgftext[x=3.814400in,y=1.788000in,left,base]{\color{textcolor}\sffamily\fontsize{14.000000}{16.800000}\selectfont \(\displaystyle \zeta = y-ct\)}%
\end{pgfscope}%
\begin{pgfscope}%
\definecolor{textcolor}{rgb}{0.000000,0.000000,0.000000}%
\pgfsetstrokecolor{textcolor}%
\pgfsetfillcolor{textcolor}%
\pgftext[x=2.636800in,y=2.460000in,,base]{\color{textcolor}\sffamily\fontsize{14.000000}{16.800000}\selectfont Simulation window}%
\end{pgfscope}%
\begin{pgfscope}%
\pgfsetbuttcap%
\pgfsetmiterjoin%
\definecolor{currentfill}{rgb}{1.000000,1.000000,1.000000}%
\pgfsetfillcolor{currentfill}%
\pgfsetlinewidth{0.000000pt}%
\definecolor{currentstroke}{rgb}{0.000000,0.000000,0.000000}%
\pgfsetstrokecolor{currentstroke}%
\pgfsetstrokeopacity{0.000000}%
\pgfsetdash{}{0pt}%
\pgfpathmoveto{\pgfqpoint{1.664000in}{0.900000in}}%
\pgfpathlineto{\pgfqpoint{3.584000in}{0.900000in}}%
\pgfpathlineto{\pgfqpoint{3.584000in}{2.400000in}}%
\pgfpathlineto{\pgfqpoint{1.664000in}{2.400000in}}%
\pgfpathlineto{\pgfqpoint{1.664000in}{0.900000in}}%
\pgfpathclose%
\pgfusepath{fill}%
\end{pgfscope}%
\begin{pgfscope}%
\pgfpathrectangle{\pgfqpoint{1.664000in}{0.900000in}}{\pgfqpoint{1.920000in}{1.500000in}}%
\pgfusepath{clip}%
\pgfsys@transformcm{1.930000}{0.000000}{0.000000}{1.500000}{1.664000in}{0.900000in}%
\pgftext[left,bottom]{\includegraphics[interpolate=false,width=1.000000in,height=1.000000in]{zeta-img0.png}}%
\end{pgfscope}%
\begin{pgfscope}%
\pgfsetbuttcap%
\pgfsetroundjoin%
\definecolor{currentfill}{rgb}{0.000000,0.000000,0.000000}%
\pgfsetfillcolor{currentfill}%
\pgfsetlinewidth{0.803000pt}%
\definecolor{currentstroke}{rgb}{0.000000,0.000000,0.000000}%
\pgfsetstrokecolor{currentstroke}%
\pgfsetdash{}{0pt}%
\pgfsys@defobject{currentmarker}{\pgfqpoint{0.000000in}{-0.048611in}}{\pgfqpoint{0.000000in}{0.000000in}}{%
\pgfpathmoveto{\pgfqpoint{0.000000in}{0.000000in}}%
\pgfpathlineto{\pgfqpoint{0.000000in}{-0.048611in}}%
\pgfusepath{stroke,fill}%
}%
\begin{pgfscope}%
\pgfsys@transformshift{2.211356in}{0.900000in}%
\pgfsys@useobject{currentmarker}{}%
\end{pgfscope}%
\end{pgfscope}%
\begin{pgfscope}%
\definecolor{textcolor}{rgb}{0.000000,0.000000,0.000000}%
\pgfsetstrokecolor{textcolor}%
\pgfsetfillcolor{textcolor}%
\pgftext[x=2.211356in,y=0.802778in,,top]{\color{textcolor}\sffamily\fontsize{10.000000}{12.000000}\selectfont \(\displaystyle {\ensuremath{-}10}\)}%
\end{pgfscope}%
\begin{pgfscope}%
\pgfsetbuttcap%
\pgfsetroundjoin%
\definecolor{currentfill}{rgb}{0.000000,0.000000,0.000000}%
\pgfsetfillcolor{currentfill}%
\pgfsetlinewidth{0.803000pt}%
\definecolor{currentstroke}{rgb}{0.000000,0.000000,0.000000}%
\pgfsetstrokecolor{currentstroke}%
\pgfsetdash{}{0pt}%
\pgfsys@defobject{currentmarker}{\pgfqpoint{0.000000in}{-0.048611in}}{\pgfqpoint{0.000000in}{0.000000in}}{%
\pgfpathmoveto{\pgfqpoint{0.000000in}{0.000000in}}%
\pgfpathlineto{\pgfqpoint{0.000000in}{-0.048611in}}%
\pgfusepath{stroke,fill}%
}%
\begin{pgfscope}%
\pgfsys@transformshift{2.816000in}{0.900000in}%
\pgfsys@useobject{currentmarker}{}%
\end{pgfscope}%
\end{pgfscope}%
\begin{pgfscope}%
\definecolor{textcolor}{rgb}{0.000000,0.000000,0.000000}%
\pgfsetstrokecolor{textcolor}%
\pgfsetfillcolor{textcolor}%
\pgftext[x=2.816000in,y=0.802778in,,top]{\color{textcolor}\sffamily\fontsize{10.000000}{12.000000}\selectfont \(\displaystyle {0}\)}%
\end{pgfscope}%
\begin{pgfscope}%
\pgfsetbuttcap%
\pgfsetroundjoin%
\definecolor{currentfill}{rgb}{0.000000,0.000000,0.000000}%
\pgfsetfillcolor{currentfill}%
\pgfsetlinewidth{0.803000pt}%
\definecolor{currentstroke}{rgb}{0.000000,0.000000,0.000000}%
\pgfsetstrokecolor{currentstroke}%
\pgfsetdash{}{0pt}%
\pgfsys@defobject{currentmarker}{\pgfqpoint{0.000000in}{-0.048611in}}{\pgfqpoint{0.000000in}{0.000000in}}{%
\pgfpathmoveto{\pgfqpoint{0.000000in}{0.000000in}}%
\pgfpathlineto{\pgfqpoint{0.000000in}{-0.048611in}}%
\pgfusepath{stroke,fill}%
}%
\begin{pgfscope}%
\pgfsys@transformshift{3.420644in}{0.900000in}%
\pgfsys@useobject{currentmarker}{}%
\end{pgfscope}%
\end{pgfscope}%
\begin{pgfscope}%
\definecolor{textcolor}{rgb}{0.000000,0.000000,0.000000}%
\pgfsetstrokecolor{textcolor}%
\pgfsetfillcolor{textcolor}%
\pgftext[x=3.420644in,y=0.802778in,,top]{\color{textcolor}\sffamily\fontsize{10.000000}{12.000000}\selectfont \(\displaystyle {10}\)}%
\end{pgfscope}%
\begin{pgfscope}%
\definecolor{textcolor}{rgb}{0.000000,0.000000,0.000000}%
\pgfsetstrokecolor{textcolor}%
\pgfsetfillcolor{textcolor}%
\pgftext[x=2.624000in,y=0.623766in,,top]{\color{textcolor}\sffamily\fontsize{10.000000}{12.000000}\selectfont \(\displaystyle \zeta \, \mathrm{[\mu m]}\)}%
\end{pgfscope}%
\begin{pgfscope}%
\pgfsetbuttcap%
\pgfsetroundjoin%
\definecolor{currentfill}{rgb}{0.000000,0.000000,0.000000}%
\pgfsetfillcolor{currentfill}%
\pgfsetlinewidth{0.803000pt}%
\definecolor{currentstroke}{rgb}{0.000000,0.000000,0.000000}%
\pgfsetstrokecolor{currentstroke}%
\pgfsetdash{}{0pt}%
\pgfsys@defobject{currentmarker}{\pgfqpoint{-0.048611in}{0.000000in}}{\pgfqpoint{-0.000000in}{0.000000in}}{%
\pgfpathmoveto{\pgfqpoint{-0.000000in}{0.000000in}}%
\pgfpathlineto{\pgfqpoint{-0.048611in}{0.000000in}}%
\pgfusepath{stroke,fill}%
}%
\begin{pgfscope}%
\pgfsys@transformshift{1.664000in}{0.988671in}%
\pgfsys@useobject{currentmarker}{}%
\end{pgfscope}%
\end{pgfscope}%
\begin{pgfscope}%
\definecolor{textcolor}{rgb}{0.000000,0.000000,0.000000}%
\pgfsetstrokecolor{textcolor}%
\pgfsetfillcolor{textcolor}%
\pgftext[x=1.319863in, y=0.940446in, left, base]{\color{textcolor}\sffamily\fontsize{10.000000}{12.000000}\selectfont \(\displaystyle {\ensuremath{-}40}\)}%
\end{pgfscope}%
\begin{pgfscope}%
\pgfsetbuttcap%
\pgfsetroundjoin%
\definecolor{currentfill}{rgb}{0.000000,0.000000,0.000000}%
\pgfsetfillcolor{currentfill}%
\pgfsetlinewidth{0.803000pt}%
\definecolor{currentstroke}{rgb}{0.000000,0.000000,0.000000}%
\pgfsetstrokecolor{currentstroke}%
\pgfsetdash{}{0pt}%
\pgfsys@defobject{currentmarker}{\pgfqpoint{-0.048611in}{0.000000in}}{\pgfqpoint{-0.000000in}{0.000000in}}{%
\pgfpathmoveto{\pgfqpoint{-0.000000in}{0.000000in}}%
\pgfpathlineto{\pgfqpoint{-0.048611in}{0.000000in}}%
\pgfusepath{stroke,fill}%
}%
\begin{pgfscope}%
\pgfsys@transformshift{1.664000in}{1.319335in}%
\pgfsys@useobject{currentmarker}{}%
\end{pgfscope}%
\end{pgfscope}%
\begin{pgfscope}%
\definecolor{textcolor}{rgb}{0.000000,0.000000,0.000000}%
\pgfsetstrokecolor{textcolor}%
\pgfsetfillcolor{textcolor}%
\pgftext[x=1.319863in, y=1.271110in, left, base]{\color{textcolor}\sffamily\fontsize{10.000000}{12.000000}\selectfont \(\displaystyle {\ensuremath{-}20}\)}%
\end{pgfscope}%
\begin{pgfscope}%
\pgfsetbuttcap%
\pgfsetroundjoin%
\definecolor{currentfill}{rgb}{0.000000,0.000000,0.000000}%
\pgfsetfillcolor{currentfill}%
\pgfsetlinewidth{0.803000pt}%
\definecolor{currentstroke}{rgb}{0.000000,0.000000,0.000000}%
\pgfsetstrokecolor{currentstroke}%
\pgfsetdash{}{0pt}%
\pgfsys@defobject{currentmarker}{\pgfqpoint{-0.048611in}{0.000000in}}{\pgfqpoint{-0.000000in}{0.000000in}}{%
\pgfpathmoveto{\pgfqpoint{-0.000000in}{0.000000in}}%
\pgfpathlineto{\pgfqpoint{-0.048611in}{0.000000in}}%
\pgfusepath{stroke,fill}%
}%
\begin{pgfscope}%
\pgfsys@transformshift{1.664000in}{1.650000in}%
\pgfsys@useobject{currentmarker}{}%
\end{pgfscope}%
\end{pgfscope}%
\begin{pgfscope}%
\definecolor{textcolor}{rgb}{0.000000,0.000000,0.000000}%
\pgfsetstrokecolor{textcolor}%
\pgfsetfillcolor{textcolor}%
\pgftext[x=1.497333in, y=1.601775in, left, base]{\color{textcolor}\sffamily\fontsize{10.000000}{12.000000}\selectfont \(\displaystyle {0}\)}%
\end{pgfscope}%
\begin{pgfscope}%
\pgfsetbuttcap%
\pgfsetroundjoin%
\definecolor{currentfill}{rgb}{0.000000,0.000000,0.000000}%
\pgfsetfillcolor{currentfill}%
\pgfsetlinewidth{0.803000pt}%
\definecolor{currentstroke}{rgb}{0.000000,0.000000,0.000000}%
\pgfsetstrokecolor{currentstroke}%
\pgfsetdash{}{0pt}%
\pgfsys@defobject{currentmarker}{\pgfqpoint{-0.048611in}{0.000000in}}{\pgfqpoint{-0.000000in}{0.000000in}}{%
\pgfpathmoveto{\pgfqpoint{-0.000000in}{0.000000in}}%
\pgfpathlineto{\pgfqpoint{-0.048611in}{0.000000in}}%
\pgfusepath{stroke,fill}%
}%
\begin{pgfscope}%
\pgfsys@transformshift{1.664000in}{1.980665in}%
\pgfsys@useobject{currentmarker}{}%
\end{pgfscope}%
\end{pgfscope}%
\begin{pgfscope}%
\definecolor{textcolor}{rgb}{0.000000,0.000000,0.000000}%
\pgfsetstrokecolor{textcolor}%
\pgfsetfillcolor{textcolor}%
\pgftext[x=1.427888in, y=1.932439in, left, base]{\color{textcolor}\sffamily\fontsize{10.000000}{12.000000}\selectfont \(\displaystyle {20}\)}%
\end{pgfscope}%
\begin{pgfscope}%
\pgfsetbuttcap%
\pgfsetroundjoin%
\definecolor{currentfill}{rgb}{0.000000,0.000000,0.000000}%
\pgfsetfillcolor{currentfill}%
\pgfsetlinewidth{0.803000pt}%
\definecolor{currentstroke}{rgb}{0.000000,0.000000,0.000000}%
\pgfsetstrokecolor{currentstroke}%
\pgfsetdash{}{0pt}%
\pgfsys@defobject{currentmarker}{\pgfqpoint{-0.048611in}{0.000000in}}{\pgfqpoint{-0.000000in}{0.000000in}}{%
\pgfpathmoveto{\pgfqpoint{-0.000000in}{0.000000in}}%
\pgfpathlineto{\pgfqpoint{-0.048611in}{0.000000in}}%
\pgfusepath{stroke,fill}%
}%
\begin{pgfscope}%
\pgfsys@transformshift{1.664000in}{2.311329in}%
\pgfsys@useobject{currentmarker}{}%
\end{pgfscope}%
\end{pgfscope}%
\begin{pgfscope}%
\definecolor{textcolor}{rgb}{0.000000,0.000000,0.000000}%
\pgfsetstrokecolor{textcolor}%
\pgfsetfillcolor{textcolor}%
\pgftext[x=1.427888in, y=2.263104in, left, base]{\color{textcolor}\sffamily\fontsize{10.000000}{12.000000}\selectfont \(\displaystyle {40}\)}%
\end{pgfscope}%
\begin{pgfscope}%
\definecolor{textcolor}{rgb}{0.000000,0.000000,0.000000}%
\pgfsetstrokecolor{textcolor}%
\pgfsetfillcolor{textcolor}%
\pgftext[x=1.264308in,y=1.650000in,,bottom,rotate=90.000000]{\color{textcolor}\sffamily\fontsize{10.000000}{12.000000}\selectfont \(\displaystyle z \, \mathrm{[\mu m]}\)}%
\end{pgfscope}%
\begin{pgfscope}%
\pgfsetrectcap%
\pgfsetmiterjoin%
\pgfsetlinewidth{0.803000pt}%
\definecolor{currentstroke}{rgb}{0.000000,0.000000,0.000000}%
\pgfsetstrokecolor{currentstroke}%
\pgfsetdash{}{0pt}%
\pgfpathmoveto{\pgfqpoint{1.664000in}{0.900000in}}%
\pgfpathlineto{\pgfqpoint{1.664000in}{2.400000in}}%
\pgfusepath{stroke}%
\end{pgfscope}%
\begin{pgfscope}%
\pgfsetrectcap%
\pgfsetmiterjoin%
\pgfsetlinewidth{0.803000pt}%
\definecolor{currentstroke}{rgb}{0.000000,0.000000,0.000000}%
\pgfsetstrokecolor{currentstroke}%
\pgfsetdash{}{0pt}%
\pgfpathmoveto{\pgfqpoint{3.584000in}{0.900000in}}%
\pgfpathlineto{\pgfqpoint{3.584000in}{2.400000in}}%
\pgfusepath{stroke}%
\end{pgfscope}%
\begin{pgfscope}%
\pgfsetrectcap%
\pgfsetmiterjoin%
\pgfsetlinewidth{0.803000pt}%
\definecolor{currentstroke}{rgb}{0.000000,0.000000,0.000000}%
\pgfsetstrokecolor{currentstroke}%
\pgfsetdash{}{0pt}%
\pgfpathmoveto{\pgfqpoint{1.664000in}{0.900000in}}%
\pgfpathlineto{\pgfqpoint{3.584000in}{0.900000in}}%
\pgfusepath{stroke}%
\end{pgfscope}%
\begin{pgfscope}%
\pgfsetrectcap%
\pgfsetmiterjoin%
\pgfsetlinewidth{0.803000pt}%
\definecolor{currentstroke}{rgb}{0.000000,0.000000,0.000000}%
\pgfsetstrokecolor{currentstroke}%
\pgfsetdash{}{0pt}%
\pgfpathmoveto{\pgfqpoint{1.664000in}{2.400000in}}%
\pgfpathlineto{\pgfqpoint{3.584000in}{2.400000in}}%
\pgfusepath{stroke}%
\end{pgfscope}%
\begin{pgfscope}%
\definecolor{textcolor}{rgb}{0.000000,0.000000,0.000000}%
\pgfsetstrokecolor{textcolor}%
\pgfsetfillcolor{textcolor}%
\pgftext[x=2.816000in,y=1.154003in,,base]{\color{textcolor}\sffamily\fontsize{12.000000}{14.400000}\selectfont Driver}%
\end{pgfscope}%
\end{pgfpicture}%
\makeatother%
\endgroup%

	\caption{Relation between the co-moving coordinate $\zeta$ and the traversed distance $y$. The position of the driver is constant in $\zeta$ as the axis moves with it.}
	\label{fig:zeta}
\end{figure}
Many figures in this chapter show a spatial distribution in the $\zeta$-$z$-plane. $\zeta$ is  an axis that moves at the speed of light parallel to the driver bunch. It is defined as $\zeta=y-ct$ with the speed of light $c$, time $t$ and propagation axis $y$.
The $y$-coordinate has the meaning of the distance of the bunch from the start of the plasma and is given in the description of the figures. The connection of the coordinates is also depicted in \autoref{fig:zeta}. 

 
\section{Transformation of the driver distribution} \label{chap:distro_change}
In this section, the movement of driver-particles is discussed and the changes of their position with respect to each other are shown. As such tracking of individual particles is not possible in experiment,
this will give further insight into the effects of the \gls{pwfa} on the drive beam. Additionally, the created wakefields are analyzed with respect to their formed electric fields and the energy a potential witness can gain from these fields.

In \autoref{fig:q_series} a time series of 2D histograms, showing the charge density of the driver after entering the plasma, can be seen for the $\zeta$-$z$-plane, with $\zeta$ being the axis, which moves with the driver bunch. Also given is $y$, the distance to the start of the plasma upramp.
\begin{figure}
	\centering
	%% Creator: Matplotlib, PGF backend
%%
%% To include the figure in your LaTeX document, write
%%   \input{<filename>.pgf}
%%
%% Make sure the required packages are loaded in your preamble
%%   \usepackage{pgf}
%%
%% Also ensure that all the required font packages are loaded; for instance,
%% the lmodern package is sometimes necessary when using math font.
%%   \usepackage{lmodern}
%%
%% Figures using additional raster images can only be included by \input if
%% they are in the same directory as the main LaTeX file. For loading figures
%% from other directories you can use the `import` package
%%   \usepackage{import}
%%
%% and then include the figures with
%%   \import{<path to file>}{<filename>.pgf}
%%
%% Matplotlib used the following preamble
%%
\begingroup%
\makeatletter%
\begin{pgfpicture}%
\pgfpathrectangle{\pgfpointorigin}{\pgfqpoint{10.000000in}{7.000000in}}%
\pgfusepath{use as bounding box, clip}%
\begin{pgfscope}%
\pgfsetbuttcap%
\pgfsetmiterjoin%
\pgfsetlinewidth{0.000000pt}%
\definecolor{currentstroke}{rgb}{1.000000,1.000000,1.000000}%
\pgfsetstrokecolor{currentstroke}%
\pgfsetstrokeopacity{0.000000}%
\pgfsetdash{}{0pt}%
\pgfpathmoveto{\pgfqpoint{0.000000in}{0.000000in}}%
\pgfpathlineto{\pgfqpoint{10.000000in}{0.000000in}}%
\pgfpathlineto{\pgfqpoint{10.000000in}{7.000000in}}%
\pgfpathlineto{\pgfqpoint{0.000000in}{7.000000in}}%
\pgfpathlineto{\pgfqpoint{0.000000in}{0.000000in}}%
\pgfpathclose%
\pgfusepath{}%
\end{pgfscope}%
\begin{pgfscope}%
\pgfsetbuttcap%
\pgfsetmiterjoin%
\definecolor{currentfill}{rgb}{1.000000,1.000000,1.000000}%
\pgfsetfillcolor{currentfill}%
\pgfsetlinewidth{0.000000pt}%
\definecolor{currentstroke}{rgb}{0.000000,0.000000,0.000000}%
\pgfsetstrokecolor{currentstroke}%
\pgfsetstrokeopacity{0.000000}%
\pgfsetdash{}{0pt}%
\pgfpathmoveto{\pgfqpoint{1.250000in}{4.155455in}}%
\pgfpathlineto{\pgfqpoint{3.529412in}{4.155455in}}%
\pgfpathlineto{\pgfqpoint{3.529412in}{6.160000in}}%
\pgfpathlineto{\pgfqpoint{1.250000in}{6.160000in}}%
\pgfpathlineto{\pgfqpoint{1.250000in}{4.155455in}}%
\pgfpathclose%
\pgfusepath{fill}%
\end{pgfscope}%
\begin{pgfscope}%
\pgfpathrectangle{\pgfqpoint{1.250000in}{4.155455in}}{\pgfqpoint{2.279412in}{2.004545in}}%
\pgfusepath{clip}%
\pgfsys@transformcm{2.291667}{0.000000}{0.000000}{2.013889}{1.250000in}{4.155455in}%
\pgftext[left,bottom]{\includegraphics[interpolate=false,width=1.000000in,height=1.000000in]{q_series-img0.png}}%
\end{pgfscope}%
\begin{pgfscope}%
\pgfsetbuttcap%
\pgfsetroundjoin%
\definecolor{currentfill}{rgb}{0.000000,0.000000,0.000000}%
\pgfsetfillcolor{currentfill}%
\pgfsetlinewidth{0.803000pt}%
\definecolor{currentstroke}{rgb}{0.000000,0.000000,0.000000}%
\pgfsetstrokecolor{currentstroke}%
\pgfsetdash{}{0pt}%
\pgfsys@defobject{currentmarker}{\pgfqpoint{0.000000in}{-0.048611in}}{\pgfqpoint{0.000000in}{0.000000in}}{%
\pgfpathmoveto{\pgfqpoint{0.000000in}{0.000000in}}%
\pgfpathlineto{\pgfqpoint{0.000000in}{-0.048611in}}%
\pgfusepath{stroke,fill}%
}%
\begin{pgfscope}%
\pgfsys@transformshift{1.660542in}{4.155455in}%
\pgfsys@useobject{currentmarker}{}%
\end{pgfscope}%
\end{pgfscope}%
\begin{pgfscope}%
\pgfsetbuttcap%
\pgfsetroundjoin%
\definecolor{currentfill}{rgb}{0.000000,0.000000,0.000000}%
\pgfsetfillcolor{currentfill}%
\pgfsetlinewidth{0.803000pt}%
\definecolor{currentstroke}{rgb}{0.000000,0.000000,0.000000}%
\pgfsetstrokecolor{currentstroke}%
\pgfsetdash{}{0pt}%
\pgfsys@defobject{currentmarker}{\pgfqpoint{0.000000in}{-0.048611in}}{\pgfqpoint{0.000000in}{0.000000in}}{%
\pgfpathmoveto{\pgfqpoint{0.000000in}{0.000000in}}%
\pgfpathlineto{\pgfqpoint{0.000000in}{-0.048611in}}%
\pgfusepath{stroke,fill}%
}%
\begin{pgfscope}%
\pgfsys@transformshift{2.139094in}{4.155455in}%
\pgfsys@useobject{currentmarker}{}%
\end{pgfscope}%
\end{pgfscope}%
\begin{pgfscope}%
\pgfsetbuttcap%
\pgfsetroundjoin%
\definecolor{currentfill}{rgb}{0.000000,0.000000,0.000000}%
\pgfsetfillcolor{currentfill}%
\pgfsetlinewidth{0.803000pt}%
\definecolor{currentstroke}{rgb}{0.000000,0.000000,0.000000}%
\pgfsetstrokecolor{currentstroke}%
\pgfsetdash{}{0pt}%
\pgfsys@defobject{currentmarker}{\pgfqpoint{0.000000in}{-0.048611in}}{\pgfqpoint{0.000000in}{0.000000in}}{%
\pgfpathmoveto{\pgfqpoint{0.000000in}{0.000000in}}%
\pgfpathlineto{\pgfqpoint{0.000000in}{-0.048611in}}%
\pgfusepath{stroke,fill}%
}%
\begin{pgfscope}%
\pgfsys@transformshift{2.617647in}{4.155455in}%
\pgfsys@useobject{currentmarker}{}%
\end{pgfscope}%
\end{pgfscope}%
\begin{pgfscope}%
\pgfsetbuttcap%
\pgfsetroundjoin%
\definecolor{currentfill}{rgb}{0.000000,0.000000,0.000000}%
\pgfsetfillcolor{currentfill}%
\pgfsetlinewidth{0.803000pt}%
\definecolor{currentstroke}{rgb}{0.000000,0.000000,0.000000}%
\pgfsetstrokecolor{currentstroke}%
\pgfsetdash{}{0pt}%
\pgfsys@defobject{currentmarker}{\pgfqpoint{0.000000in}{-0.048611in}}{\pgfqpoint{0.000000in}{0.000000in}}{%
\pgfpathmoveto{\pgfqpoint{0.000000in}{0.000000in}}%
\pgfpathlineto{\pgfqpoint{0.000000in}{-0.048611in}}%
\pgfusepath{stroke,fill}%
}%
\begin{pgfscope}%
\pgfsys@transformshift{3.096200in}{4.155455in}%
\pgfsys@useobject{currentmarker}{}%
\end{pgfscope}%
\end{pgfscope}%
\begin{pgfscope}%
\definecolor{textcolor}{rgb}{0.000000,0.000000,0.000000}%
\pgfsetstrokecolor{textcolor}%
\pgfsetfillcolor{textcolor}%
\pgftext[x=2.389706in,y=4.099899in,,top]{\color{textcolor}\sffamily\fontsize{10.000000}{12.000000}\selectfont \(\displaystyle \zeta \, \mathrm{[\mu m]}\)}%
\end{pgfscope}%
\begin{pgfscope}%
\pgfsetbuttcap%
\pgfsetroundjoin%
\definecolor{currentfill}{rgb}{0.000000,0.000000,0.000000}%
\pgfsetfillcolor{currentfill}%
\pgfsetlinewidth{0.803000pt}%
\definecolor{currentstroke}{rgb}{0.000000,0.000000,0.000000}%
\pgfsetstrokecolor{currentstroke}%
\pgfsetdash{}{0pt}%
\pgfsys@defobject{currentmarker}{\pgfqpoint{-0.048611in}{0.000000in}}{\pgfqpoint{-0.000000in}{0.000000in}}{%
\pgfpathmoveto{\pgfqpoint{-0.000000in}{0.000000in}}%
\pgfpathlineto{\pgfqpoint{-0.048611in}{0.000000in}}%
\pgfusepath{stroke,fill}%
}%
\begin{pgfscope}%
\pgfsys@transformshift{1.250000in}{4.163479in}%
\pgfsys@useobject{currentmarker}{}%
\end{pgfscope}%
\end{pgfscope}%
\begin{pgfscope}%
\definecolor{textcolor}{rgb}{0.000000,0.000000,0.000000}%
\pgfsetstrokecolor{textcolor}%
\pgfsetfillcolor{textcolor}%
\pgftext[x=0.905863in, y=4.115254in, left, base]{\color{textcolor}\sffamily\fontsize{10.000000}{12.000000}\selectfont \(\displaystyle {\ensuremath{-}30}\)}%
\end{pgfscope}%
\begin{pgfscope}%
\pgfsetbuttcap%
\pgfsetroundjoin%
\definecolor{currentfill}{rgb}{0.000000,0.000000,0.000000}%
\pgfsetfillcolor{currentfill}%
\pgfsetlinewidth{0.803000pt}%
\definecolor{currentstroke}{rgb}{0.000000,0.000000,0.000000}%
\pgfsetstrokecolor{currentstroke}%
\pgfsetdash{}{0pt}%
\pgfsys@defobject{currentmarker}{\pgfqpoint{-0.048611in}{0.000000in}}{\pgfqpoint{-0.000000in}{0.000000in}}{%
\pgfpathmoveto{\pgfqpoint{-0.000000in}{0.000000in}}%
\pgfpathlineto{\pgfqpoint{-0.048611in}{0.000000in}}%
\pgfusepath{stroke,fill}%
}%
\begin{pgfscope}%
\pgfsys@transformshift{1.250000in}{4.494895in}%
\pgfsys@useobject{currentmarker}{}%
\end{pgfscope}%
\end{pgfscope}%
\begin{pgfscope}%
\definecolor{textcolor}{rgb}{0.000000,0.000000,0.000000}%
\pgfsetstrokecolor{textcolor}%
\pgfsetfillcolor{textcolor}%
\pgftext[x=0.905863in, y=4.446670in, left, base]{\color{textcolor}\sffamily\fontsize{10.000000}{12.000000}\selectfont \(\displaystyle {\ensuremath{-}20}\)}%
\end{pgfscope}%
\begin{pgfscope}%
\pgfsetbuttcap%
\pgfsetroundjoin%
\definecolor{currentfill}{rgb}{0.000000,0.000000,0.000000}%
\pgfsetfillcolor{currentfill}%
\pgfsetlinewidth{0.803000pt}%
\definecolor{currentstroke}{rgb}{0.000000,0.000000,0.000000}%
\pgfsetstrokecolor{currentstroke}%
\pgfsetdash{}{0pt}%
\pgfsys@defobject{currentmarker}{\pgfqpoint{-0.048611in}{0.000000in}}{\pgfqpoint{-0.000000in}{0.000000in}}{%
\pgfpathmoveto{\pgfqpoint{-0.000000in}{0.000000in}}%
\pgfpathlineto{\pgfqpoint{-0.048611in}{0.000000in}}%
\pgfusepath{stroke,fill}%
}%
\begin{pgfscope}%
\pgfsys@transformshift{1.250000in}{4.826311in}%
\pgfsys@useobject{currentmarker}{}%
\end{pgfscope}%
\end{pgfscope}%
\begin{pgfscope}%
\definecolor{textcolor}{rgb}{0.000000,0.000000,0.000000}%
\pgfsetstrokecolor{textcolor}%
\pgfsetfillcolor{textcolor}%
\pgftext[x=0.905863in, y=4.778086in, left, base]{\color{textcolor}\sffamily\fontsize{10.000000}{12.000000}\selectfont \(\displaystyle {\ensuremath{-}10}\)}%
\end{pgfscope}%
\begin{pgfscope}%
\pgfsetbuttcap%
\pgfsetroundjoin%
\definecolor{currentfill}{rgb}{0.000000,0.000000,0.000000}%
\pgfsetfillcolor{currentfill}%
\pgfsetlinewidth{0.803000pt}%
\definecolor{currentstroke}{rgb}{0.000000,0.000000,0.000000}%
\pgfsetstrokecolor{currentstroke}%
\pgfsetdash{}{0pt}%
\pgfsys@defobject{currentmarker}{\pgfqpoint{-0.048611in}{0.000000in}}{\pgfqpoint{-0.000000in}{0.000000in}}{%
\pgfpathmoveto{\pgfqpoint{-0.000000in}{0.000000in}}%
\pgfpathlineto{\pgfqpoint{-0.048611in}{0.000000in}}%
\pgfusepath{stroke,fill}%
}%
\begin{pgfscope}%
\pgfsys@transformshift{1.250000in}{5.157727in}%
\pgfsys@useobject{currentmarker}{}%
\end{pgfscope}%
\end{pgfscope}%
\begin{pgfscope}%
\definecolor{textcolor}{rgb}{0.000000,0.000000,0.000000}%
\pgfsetstrokecolor{textcolor}%
\pgfsetfillcolor{textcolor}%
\pgftext[x=1.083333in, y=5.109502in, left, base]{\color{textcolor}\sffamily\fontsize{10.000000}{12.000000}\selectfont \(\displaystyle {0}\)}%
\end{pgfscope}%
\begin{pgfscope}%
\pgfsetbuttcap%
\pgfsetroundjoin%
\definecolor{currentfill}{rgb}{0.000000,0.000000,0.000000}%
\pgfsetfillcolor{currentfill}%
\pgfsetlinewidth{0.803000pt}%
\definecolor{currentstroke}{rgb}{0.000000,0.000000,0.000000}%
\pgfsetstrokecolor{currentstroke}%
\pgfsetdash{}{0pt}%
\pgfsys@defobject{currentmarker}{\pgfqpoint{-0.048611in}{0.000000in}}{\pgfqpoint{-0.000000in}{0.000000in}}{%
\pgfpathmoveto{\pgfqpoint{-0.000000in}{0.000000in}}%
\pgfpathlineto{\pgfqpoint{-0.048611in}{0.000000in}}%
\pgfusepath{stroke,fill}%
}%
\begin{pgfscope}%
\pgfsys@transformshift{1.250000in}{5.489143in}%
\pgfsys@useobject{currentmarker}{}%
\end{pgfscope}%
\end{pgfscope}%
\begin{pgfscope}%
\definecolor{textcolor}{rgb}{0.000000,0.000000,0.000000}%
\pgfsetstrokecolor{textcolor}%
\pgfsetfillcolor{textcolor}%
\pgftext[x=1.013888in, y=5.440918in, left, base]{\color{textcolor}\sffamily\fontsize{10.000000}{12.000000}\selectfont \(\displaystyle {10}\)}%
\end{pgfscope}%
\begin{pgfscope}%
\pgfsetbuttcap%
\pgfsetroundjoin%
\definecolor{currentfill}{rgb}{0.000000,0.000000,0.000000}%
\pgfsetfillcolor{currentfill}%
\pgfsetlinewidth{0.803000pt}%
\definecolor{currentstroke}{rgb}{0.000000,0.000000,0.000000}%
\pgfsetstrokecolor{currentstroke}%
\pgfsetdash{}{0pt}%
\pgfsys@defobject{currentmarker}{\pgfqpoint{-0.048611in}{0.000000in}}{\pgfqpoint{-0.000000in}{0.000000in}}{%
\pgfpathmoveto{\pgfqpoint{-0.000000in}{0.000000in}}%
\pgfpathlineto{\pgfqpoint{-0.048611in}{0.000000in}}%
\pgfusepath{stroke,fill}%
}%
\begin{pgfscope}%
\pgfsys@transformshift{1.250000in}{5.820559in}%
\pgfsys@useobject{currentmarker}{}%
\end{pgfscope}%
\end{pgfscope}%
\begin{pgfscope}%
\definecolor{textcolor}{rgb}{0.000000,0.000000,0.000000}%
\pgfsetstrokecolor{textcolor}%
\pgfsetfillcolor{textcolor}%
\pgftext[x=1.013888in, y=5.772334in, left, base]{\color{textcolor}\sffamily\fontsize{10.000000}{12.000000}\selectfont \(\displaystyle {20}\)}%
\end{pgfscope}%
\begin{pgfscope}%
\pgfsetbuttcap%
\pgfsetroundjoin%
\definecolor{currentfill}{rgb}{0.000000,0.000000,0.000000}%
\pgfsetfillcolor{currentfill}%
\pgfsetlinewidth{0.803000pt}%
\definecolor{currentstroke}{rgb}{0.000000,0.000000,0.000000}%
\pgfsetstrokecolor{currentstroke}%
\pgfsetdash{}{0pt}%
\pgfsys@defobject{currentmarker}{\pgfqpoint{-0.048611in}{0.000000in}}{\pgfqpoint{-0.000000in}{0.000000in}}{%
\pgfpathmoveto{\pgfqpoint{-0.000000in}{0.000000in}}%
\pgfpathlineto{\pgfqpoint{-0.048611in}{0.000000in}}%
\pgfusepath{stroke,fill}%
}%
\begin{pgfscope}%
\pgfsys@transformshift{1.250000in}{6.151975in}%
\pgfsys@useobject{currentmarker}{}%
\end{pgfscope}%
\end{pgfscope}%
\begin{pgfscope}%
\definecolor{textcolor}{rgb}{0.000000,0.000000,0.000000}%
\pgfsetstrokecolor{textcolor}%
\pgfsetfillcolor{textcolor}%
\pgftext[x=1.013888in, y=6.103750in, left, base]{\color{textcolor}\sffamily\fontsize{10.000000}{12.000000}\selectfont \(\displaystyle {30}\)}%
\end{pgfscope}%
\begin{pgfscope}%
\definecolor{textcolor}{rgb}{0.000000,0.000000,0.000000}%
\pgfsetstrokecolor{textcolor}%
\pgfsetfillcolor{textcolor}%
\pgftext[x=0.850308in,y=5.157727in,,bottom,rotate=90.000000]{\color{textcolor}\sffamily\fontsize{10.000000}{12.000000}\selectfont \(\displaystyle z \, \mathrm{[\mu m]}\)}%
\end{pgfscope}%
\begin{pgfscope}%
\pgfpathrectangle{\pgfqpoint{1.250000in}{4.155455in}}{\pgfqpoint{2.279412in}{2.004545in}}%
\pgfusepath{clip}%
\pgfsetbuttcap%
\pgfsetroundjoin%
\pgfsetlinewidth{0.314766pt}%
\definecolor{currentstroke}{rgb}{0.268510,0.009605,0.335427}%
\pgfsetstrokecolor{currentstroke}%
\pgfsetdash{}{0pt}%
\pgfpathmoveto{\pgfqpoint{3.261668in}{5.067514in}}%
\pgfpathlineto{\pgfqpoint{3.211678in}{5.066024in}}%
\pgfusepath{stroke}%
\end{pgfscope}%
\begin{pgfscope}%
\pgfpathrectangle{\pgfqpoint{1.250000in}{4.155455in}}{\pgfqpoint{2.279412in}{2.004545in}}%
\pgfusepath{clip}%
\pgfsetbuttcap%
\pgfsetroundjoin%
\pgfsetlinewidth{0.320833pt}%
\definecolor{currentstroke}{rgb}{0.269944,0.014625,0.341379}%
\pgfsetstrokecolor{currentstroke}%
\pgfsetdash{}{0pt}%
\pgfpathmoveto{\pgfqpoint{3.211678in}{5.066024in}}%
\pgfpathlineto{\pgfqpoint{3.161670in}{5.065633in}}%
\pgfusepath{stroke}%
\end{pgfscope}%
\begin{pgfscope}%
\pgfpathrectangle{\pgfqpoint{1.250000in}{4.155455in}}{\pgfqpoint{2.279412in}{2.004545in}}%
\pgfusepath{clip}%
\pgfsetbuttcap%
\pgfsetroundjoin%
\pgfsetlinewidth{0.332228pt}%
\definecolor{currentstroke}{rgb}{0.272594,0.025563,0.353093}%
\pgfsetstrokecolor{currentstroke}%
\pgfsetdash{}{0pt}%
\pgfpathmoveto{\pgfqpoint{3.161670in}{5.065633in}}%
\pgfpathlineto{\pgfqpoint{3.111528in}{5.065873in}}%
\pgfusepath{stroke}%
\end{pgfscope}%
\begin{pgfscope}%
\pgfpathrectangle{\pgfqpoint{1.250000in}{4.155455in}}{\pgfqpoint{2.279412in}{2.004545in}}%
\pgfusepath{clip}%
\pgfsetbuttcap%
\pgfsetroundjoin%
\pgfsetlinewidth{0.342263pt}%
\definecolor{currentstroke}{rgb}{0.273809,0.031497,0.358853}%
\pgfsetstrokecolor{currentstroke}%
\pgfsetdash{}{0pt}%
\pgfpathmoveto{\pgfqpoint{3.111528in}{5.065873in}}%
\pgfpathlineto{\pgfqpoint{3.061390in}{5.066243in}}%
\pgfusepath{stroke}%
\end{pgfscope}%
\begin{pgfscope}%
\pgfpathrectangle{\pgfqpoint{1.250000in}{4.155455in}}{\pgfqpoint{2.279412in}{2.004545in}}%
\pgfusepath{clip}%
\pgfsetbuttcap%
\pgfsetroundjoin%
\pgfsetlinewidth{0.351884pt}%
\definecolor{currentstroke}{rgb}{0.276022,0.044167,0.370164}%
\pgfsetstrokecolor{currentstroke}%
\pgfsetdash{}{0pt}%
\pgfpathmoveto{\pgfqpoint{3.061390in}{5.066243in}}%
\pgfpathlineto{\pgfqpoint{3.011249in}{5.066629in}}%
\pgfusepath{stroke}%
\end{pgfscope}%
\begin{pgfscope}%
\pgfpathrectangle{\pgfqpoint{1.250000in}{4.155455in}}{\pgfqpoint{2.279412in}{2.004545in}}%
\pgfusepath{clip}%
\pgfsetbuttcap%
\pgfsetroundjoin%
\pgfsetlinewidth{0.379649pt}%
\definecolor{currentstroke}{rgb}{0.279566,0.067836,0.391917}%
\pgfsetstrokecolor{currentstroke}%
\pgfsetdash{}{0pt}%
\pgfpathmoveto{\pgfqpoint{3.011249in}{5.066629in}}%
\pgfpathlineto{\pgfqpoint{2.961101in}{5.066845in}}%
\pgfusepath{stroke}%
\end{pgfscope}%
\begin{pgfscope}%
\pgfpathrectangle{\pgfqpoint{1.250000in}{4.155455in}}{\pgfqpoint{2.279412in}{2.004545in}}%
\pgfusepath{clip}%
\pgfsetbuttcap%
\pgfsetroundjoin%
\pgfsetlinewidth{0.417723pt}%
\definecolor{currentstroke}{rgb}{0.282327,0.094955,0.417331}%
\pgfsetstrokecolor{currentstroke}%
\pgfsetdash{}{0pt}%
\pgfpathmoveto{\pgfqpoint{2.961101in}{5.066845in}}%
\pgfpathlineto{\pgfqpoint{2.910955in}{5.067481in}}%
\pgfusepath{stroke}%
\end{pgfscope}%
\begin{pgfscope}%
\pgfpathrectangle{\pgfqpoint{1.250000in}{4.155455in}}{\pgfqpoint{2.279412in}{2.004545in}}%
\pgfusepath{clip}%
\pgfsetbuttcap%
\pgfsetroundjoin%
\pgfsetlinewidth{0.450946pt}%
\definecolor{currentstroke}{rgb}{0.283229,0.120777,0.440584}%
\pgfsetstrokecolor{currentstroke}%
\pgfsetdash{}{0pt}%
\pgfpathmoveto{\pgfqpoint{2.910955in}{5.067481in}}%
\pgfpathlineto{\pgfqpoint{2.860811in}{5.068226in}}%
\pgfusepath{stroke}%
\end{pgfscope}%
\begin{pgfscope}%
\pgfpathrectangle{\pgfqpoint{1.250000in}{4.155455in}}{\pgfqpoint{2.279412in}{2.004545in}}%
\pgfusepath{clip}%
\pgfsetbuttcap%
\pgfsetroundjoin%
\pgfsetlinewidth{0.494280pt}%
\definecolor{currentstroke}{rgb}{0.281412,0.155834,0.469201}%
\pgfsetstrokecolor{currentstroke}%
\pgfsetdash{}{0pt}%
\pgfpathmoveto{\pgfqpoint{2.860811in}{5.068226in}}%
\pgfpathlineto{\pgfqpoint{2.810668in}{5.069040in}}%
\pgfusepath{stroke}%
\end{pgfscope}%
\begin{pgfscope}%
\pgfpathrectangle{\pgfqpoint{1.250000in}{4.155455in}}{\pgfqpoint{2.279412in}{2.004545in}}%
\pgfusepath{clip}%
\pgfsetbuttcap%
\pgfsetroundjoin%
\pgfsetlinewidth{0.577859pt}%
\definecolor{currentstroke}{rgb}{0.270595,0.214069,0.507052}%
\pgfsetstrokecolor{currentstroke}%
\pgfsetdash{}{0pt}%
\pgfpathmoveto{\pgfqpoint{2.810668in}{5.069040in}}%
\pgfpathlineto{\pgfqpoint{2.760526in}{5.069867in}}%
\pgfusepath{stroke}%
\end{pgfscope}%
\begin{pgfscope}%
\pgfpathrectangle{\pgfqpoint{1.250000in}{4.155455in}}{\pgfqpoint{2.279412in}{2.004545in}}%
\pgfusepath{clip}%
\pgfsetbuttcap%
\pgfsetroundjoin%
\pgfsetlinewidth{0.650131pt}%
\definecolor{currentstroke}{rgb}{0.255645,0.260703,0.528312}%
\pgfsetstrokecolor{currentstroke}%
\pgfsetdash{}{0pt}%
\pgfpathmoveto{\pgfqpoint{2.760526in}{5.069867in}}%
\pgfpathlineto{\pgfqpoint{2.710387in}{5.070846in}}%
\pgfusepath{stroke}%
\end{pgfscope}%
\begin{pgfscope}%
\pgfpathrectangle{\pgfqpoint{1.250000in}{4.155455in}}{\pgfqpoint{2.279412in}{2.004545in}}%
\pgfusepath{clip}%
\pgfsetbuttcap%
\pgfsetroundjoin%
\pgfsetlinewidth{0.736934pt}%
\definecolor{currentstroke}{rgb}{0.231674,0.318106,0.544834}%
\pgfsetstrokecolor{currentstroke}%
\pgfsetdash{}{0pt}%
\pgfpathmoveto{\pgfqpoint{2.710387in}{5.070846in}}%
\pgfpathlineto{\pgfqpoint{2.660254in}{5.072045in}}%
\pgfusepath{stroke}%
\end{pgfscope}%
\begin{pgfscope}%
\pgfpathrectangle{\pgfqpoint{1.250000in}{4.155455in}}{\pgfqpoint{2.279412in}{2.004545in}}%
\pgfusepath{clip}%
\pgfsetbuttcap%
\pgfsetroundjoin%
\pgfsetlinewidth{0.772768pt}%
\definecolor{currentstroke}{rgb}{0.221989,0.339161,0.548752}%
\pgfsetstrokecolor{currentstroke}%
\pgfsetdash{}{0pt}%
\pgfpathmoveto{\pgfqpoint{2.660254in}{5.072045in}}%
\pgfpathlineto{\pgfqpoint{2.610125in}{5.073372in}}%
\pgfusepath{stroke}%
\end{pgfscope}%
\begin{pgfscope}%
\pgfpathrectangle{\pgfqpoint{1.250000in}{4.155455in}}{\pgfqpoint{2.279412in}{2.004545in}}%
\pgfusepath{clip}%
\pgfsetbuttcap%
\pgfsetroundjoin%
\pgfsetlinewidth{0.844684pt}%
\definecolor{currentstroke}{rgb}{0.201239,0.383670,0.554294}%
\pgfsetstrokecolor{currentstroke}%
\pgfsetdash{}{0pt}%
\pgfpathmoveto{\pgfqpoint{2.610125in}{5.073372in}}%
\pgfpathlineto{\pgfqpoint{2.560002in}{5.074834in}}%
\pgfusepath{stroke}%
\end{pgfscope}%
\begin{pgfscope}%
\pgfpathrectangle{\pgfqpoint{1.250000in}{4.155455in}}{\pgfqpoint{2.279412in}{2.004545in}}%
\pgfusepath{clip}%
\pgfsetbuttcap%
\pgfsetroundjoin%
\pgfsetlinewidth{0.910444pt}%
\definecolor{currentstroke}{rgb}{0.185556,0.418570,0.556753}%
\pgfsetstrokecolor{currentstroke}%
\pgfsetdash{}{0pt}%
\pgfpathmoveto{\pgfqpoint{2.560002in}{5.074834in}}%
\pgfpathlineto{\pgfqpoint{2.509886in}{5.076499in}}%
\pgfusepath{stroke}%
\end{pgfscope}%
\begin{pgfscope}%
\pgfpathrectangle{\pgfqpoint{1.250000in}{4.155455in}}{\pgfqpoint{2.279412in}{2.004545in}}%
\pgfusepath{clip}%
\pgfsetbuttcap%
\pgfsetroundjoin%
\pgfsetlinewidth{0.922568pt}%
\definecolor{currentstroke}{rgb}{0.182256,0.426184,0.557120}%
\pgfsetstrokecolor{currentstroke}%
\pgfsetdash{}{0pt}%
\pgfpathmoveto{\pgfqpoint{2.509886in}{5.076499in}}%
\pgfpathlineto{\pgfqpoint{2.459782in}{5.078402in}}%
\pgfusepath{stroke}%
\end{pgfscope}%
\begin{pgfscope}%
\pgfpathrectangle{\pgfqpoint{1.250000in}{4.155455in}}{\pgfqpoint{2.279412in}{2.004545in}}%
\pgfusepath{clip}%
\pgfsetbuttcap%
\pgfsetroundjoin%
\pgfsetlinewidth{0.937628pt}%
\definecolor{currentstroke}{rgb}{0.179019,0.433756,0.557430}%
\pgfsetstrokecolor{currentstroke}%
\pgfsetdash{}{0pt}%
\pgfpathmoveto{\pgfqpoint{2.459782in}{5.078402in}}%
\pgfpathlineto{\pgfqpoint{2.409700in}{5.080713in}}%
\pgfusepath{stroke}%
\end{pgfscope}%
\begin{pgfscope}%
\pgfpathrectangle{\pgfqpoint{1.250000in}{4.155455in}}{\pgfqpoint{2.279412in}{2.004545in}}%
\pgfusepath{clip}%
\pgfsetbuttcap%
\pgfsetroundjoin%
\pgfsetlinewidth{0.937728pt}%
\definecolor{currentstroke}{rgb}{0.179019,0.433756,0.557430}%
\pgfsetstrokecolor{currentstroke}%
\pgfsetdash{}{0pt}%
\pgfpathmoveto{\pgfqpoint{2.409700in}{5.080713in}}%
\pgfpathlineto{\pgfqpoint{2.359652in}{5.083516in}}%
\pgfusepath{stroke}%
\end{pgfscope}%
\begin{pgfscope}%
\pgfpathrectangle{\pgfqpoint{1.250000in}{4.155455in}}{\pgfqpoint{2.279412in}{2.004545in}}%
\pgfusepath{clip}%
\pgfsetbuttcap%
\pgfsetroundjoin%
\pgfsetlinewidth{0.792661pt}%
\definecolor{currentstroke}{rgb}{0.216210,0.351535,0.550627}%
\pgfsetstrokecolor{currentstroke}%
\pgfsetdash{}{0pt}%
\pgfpathmoveto{\pgfqpoint{2.359652in}{5.083516in}}%
\pgfpathlineto{\pgfqpoint{2.309688in}{5.087205in}}%
\pgfusepath{stroke}%
\end{pgfscope}%
\begin{pgfscope}%
\pgfpathrectangle{\pgfqpoint{1.250000in}{4.155455in}}{\pgfqpoint{2.279412in}{2.004545in}}%
\pgfusepath{clip}%
\pgfsetbuttcap%
\pgfsetroundjoin%
\pgfsetlinewidth{0.761476pt}%
\definecolor{currentstroke}{rgb}{0.223925,0.334994,0.548053}%
\pgfsetstrokecolor{currentstroke}%
\pgfsetdash{}{0pt}%
\pgfpathmoveto{\pgfqpoint{2.309688in}{5.087205in}}%
\pgfpathlineto{\pgfqpoint{2.259835in}{5.091895in}}%
\pgfusepath{stroke}%
\end{pgfscope}%
\begin{pgfscope}%
\pgfpathrectangle{\pgfqpoint{1.250000in}{4.155455in}}{\pgfqpoint{2.279412in}{2.004545in}}%
\pgfusepath{clip}%
\pgfsetbuttcap%
\pgfsetroundjoin%
\pgfsetlinewidth{0.631319pt}%
\definecolor{currentstroke}{rgb}{0.258965,0.251537,0.524736}%
\pgfsetstrokecolor{currentstroke}%
\pgfsetdash{}{0pt}%
\pgfpathmoveto{\pgfqpoint{2.259835in}{5.091895in}}%
\pgfpathlineto{\pgfqpoint{2.210326in}{5.098640in}}%
\pgfusepath{stroke}%
\end{pgfscope}%
\begin{pgfscope}%
\pgfpathrectangle{\pgfqpoint{1.250000in}{4.155455in}}{\pgfqpoint{2.279412in}{2.004545in}}%
\pgfusepath{clip}%
\pgfsetbuttcap%
\pgfsetroundjoin%
\pgfsetlinewidth{0.544955pt}%
\definecolor{currentstroke}{rgb}{0.276194,0.190074,0.493001}%
\pgfsetstrokecolor{currentstroke}%
\pgfsetdash{}{0pt}%
\pgfpathmoveto{\pgfqpoint{2.210326in}{5.098640in}}%
\pgfpathlineto{\pgfqpoint{2.210326in}{5.098640in}}%
\pgfusepath{stroke}%
\end{pgfscope}%
\begin{pgfscope}%
\pgfpathrectangle{\pgfqpoint{1.250000in}{4.155455in}}{\pgfqpoint{2.279412in}{2.004545in}}%
\pgfusepath{clip}%
\pgfsetbuttcap%
\pgfsetroundjoin%
\pgfsetlinewidth{0.544955pt}%
\definecolor{currentstroke}{rgb}{0.276194,0.190074,0.493001}%
\pgfsetstrokecolor{currentstroke}%
\pgfsetdash{}{0pt}%
\pgfpathmoveto{\pgfqpoint{2.210326in}{5.098640in}}%
\pgfpathlineto{\pgfqpoint{2.174776in}{5.107413in}}%
\pgfusepath{stroke}%
\end{pgfscope}%
\begin{pgfscope}%
\pgfpathrectangle{\pgfqpoint{1.250000in}{4.155455in}}{\pgfqpoint{2.279412in}{2.004545in}}%
\pgfusepath{clip}%
\pgfsetbuttcap%
\pgfsetroundjoin%
\pgfsetlinewidth{0.450733pt}%
\definecolor{currentstroke}{rgb}{0.283229,0.120777,0.440584}%
\pgfsetstrokecolor{currentstroke}%
\pgfsetdash{}{0pt}%
\pgfpathmoveto{\pgfqpoint{2.174776in}{5.107413in}}%
\pgfpathlineto{\pgfqpoint{2.174776in}{5.107413in}}%
\pgfusepath{stroke}%
\end{pgfscope}%
\begin{pgfscope}%
\pgfpathrectangle{\pgfqpoint{1.250000in}{4.155455in}}{\pgfqpoint{2.279412in}{2.004545in}}%
\pgfusepath{clip}%
\pgfsetbuttcap%
\pgfsetroundjoin%
\pgfsetlinewidth{0.450733pt}%
\definecolor{currentstroke}{rgb}{0.283229,0.120777,0.440584}%
\pgfsetstrokecolor{currentstroke}%
\pgfsetdash{}{0pt}%
\pgfpathmoveto{\pgfqpoint{2.174776in}{5.107413in}}%
\pgfpathlineto{\pgfqpoint{2.161238in}{5.112875in}}%
\pgfusepath{stroke}%
\end{pgfscope}%
\begin{pgfscope}%
\pgfpathrectangle{\pgfqpoint{1.250000in}{4.155455in}}{\pgfqpoint{2.279412in}{2.004545in}}%
\pgfusepath{clip}%
\pgfsetbuttcap%
\pgfsetroundjoin%
\pgfsetlinewidth{0.422897pt}%
\definecolor{currentstroke}{rgb}{0.282656,0.100196,0.422160}%
\pgfsetstrokecolor{currentstroke}%
\pgfsetdash{}{0pt}%
\pgfpathmoveto{\pgfqpoint{2.161238in}{5.112875in}}%
\pgfpathlineto{\pgfqpoint{2.161238in}{5.112875in}}%
\pgfusepath{stroke}%
\end{pgfscope}%
\begin{pgfscope}%
\pgfpathrectangle{\pgfqpoint{1.250000in}{4.155455in}}{\pgfqpoint{2.279412in}{2.004545in}}%
\pgfusepath{clip}%
\pgfsetbuttcap%
\pgfsetroundjoin%
\pgfsetlinewidth{0.422897pt}%
\definecolor{currentstroke}{rgb}{0.282656,0.100196,0.422160}%
\pgfsetstrokecolor{currentstroke}%
\pgfsetdash{}{0pt}%
\pgfpathmoveto{\pgfqpoint{2.161238in}{5.112875in}}%
\pgfpathlineto{\pgfqpoint{2.154652in}{5.117740in}}%
\pgfusepath{stroke}%
\end{pgfscope}%
\begin{pgfscope}%
\pgfpathrectangle{\pgfqpoint{1.250000in}{4.155455in}}{\pgfqpoint{2.279412in}{2.004545in}}%
\pgfusepath{clip}%
\pgfsetbuttcap%
\pgfsetroundjoin%
\pgfsetlinewidth{0.399226pt}%
\definecolor{currentstroke}{rgb}{0.281446,0.084320,0.407414}%
\pgfsetstrokecolor{currentstroke}%
\pgfsetdash{}{0pt}%
\pgfpathmoveto{\pgfqpoint{2.154652in}{5.117740in}}%
\pgfpathlineto{\pgfqpoint{2.151654in}{5.122910in}}%
\pgfusepath{stroke}%
\end{pgfscope}%
\begin{pgfscope}%
\pgfpathrectangle{\pgfqpoint{1.250000in}{4.155455in}}{\pgfqpoint{2.279412in}{2.004545in}}%
\pgfusepath{clip}%
\pgfsetbuttcap%
\pgfsetroundjoin%
\pgfsetlinewidth{0.389221pt}%
\definecolor{currentstroke}{rgb}{0.280267,0.073417,0.397163}%
\pgfsetstrokecolor{currentstroke}%
\pgfsetdash{}{0pt}%
\pgfpathmoveto{\pgfqpoint{2.151654in}{5.122910in}}%
\pgfpathlineto{\pgfqpoint{2.150420in}{5.127633in}}%
\pgfusepath{stroke}%
\end{pgfscope}%
\begin{pgfscope}%
\pgfpathrectangle{\pgfqpoint{1.250000in}{4.155455in}}{\pgfqpoint{2.279412in}{2.004545in}}%
\pgfusepath{clip}%
\pgfsetbuttcap%
\pgfsetroundjoin%
\pgfsetlinewidth{0.384643pt}%
\definecolor{currentstroke}{rgb}{0.280267,0.073417,0.397163}%
\pgfsetstrokecolor{currentstroke}%
\pgfsetdash{}{0pt}%
\pgfpathmoveto{\pgfqpoint{2.150420in}{5.127633in}}%
\pgfpathlineto{\pgfqpoint{2.150290in}{5.132189in}}%
\pgfusepath{stroke}%
\end{pgfscope}%
\begin{pgfscope}%
\pgfpathrectangle{\pgfqpoint{1.250000in}{4.155455in}}{\pgfqpoint{2.279412in}{2.004545in}}%
\pgfusepath{clip}%
\pgfsetbuttcap%
\pgfsetroundjoin%
\pgfsetlinewidth{0.382712pt}%
\definecolor{currentstroke}{rgb}{0.279566,0.067836,0.391917}%
\pgfsetstrokecolor{currentstroke}%
\pgfsetdash{}{0pt}%
\pgfpathmoveto{\pgfqpoint{2.150290in}{5.132189in}}%
\pgfpathlineto{\pgfqpoint{2.151053in}{5.136250in}}%
\pgfusepath{stroke}%
\end{pgfscope}%
\begin{pgfscope}%
\pgfpathrectangle{\pgfqpoint{1.250000in}{4.155455in}}{\pgfqpoint{2.279412in}{2.004545in}}%
\pgfusepath{clip}%
\pgfsetbuttcap%
\pgfsetroundjoin%
\pgfsetlinewidth{0.382720pt}%
\definecolor{currentstroke}{rgb}{0.279566,0.067836,0.391917}%
\pgfsetstrokecolor{currentstroke}%
\pgfsetdash{}{0pt}%
\pgfpathmoveto{\pgfqpoint{2.151053in}{5.136250in}}%
\pgfpathlineto{\pgfqpoint{2.152204in}{5.139287in}}%
\pgfusepath{stroke}%
\end{pgfscope}%
\begin{pgfscope}%
\pgfpathrectangle{\pgfqpoint{1.250000in}{4.155455in}}{\pgfqpoint{2.279412in}{2.004545in}}%
\pgfusepath{clip}%
\pgfsetbuttcap%
\pgfsetroundjoin%
\pgfsetlinewidth{0.384585pt}%
\definecolor{currentstroke}{rgb}{0.280267,0.073417,0.397163}%
\pgfsetstrokecolor{currentstroke}%
\pgfsetdash{}{0pt}%
\pgfpathmoveto{\pgfqpoint{2.152204in}{5.139287in}}%
\pgfpathlineto{\pgfqpoint{2.152060in}{5.141274in}}%
\pgfusepath{stroke}%
\end{pgfscope}%
\begin{pgfscope}%
\pgfpathrectangle{\pgfqpoint{1.250000in}{4.155455in}}{\pgfqpoint{2.279412in}{2.004545in}}%
\pgfusepath{clip}%
\pgfsetbuttcap%
\pgfsetroundjoin%
\pgfsetlinewidth{0.384857pt}%
\definecolor{currentstroke}{rgb}{0.280267,0.073417,0.397163}%
\pgfsetstrokecolor{currentstroke}%
\pgfsetdash{}{0pt}%
\pgfpathmoveto{\pgfqpoint{2.152060in}{5.141274in}}%
\pgfpathlineto{\pgfqpoint{2.151069in}{5.142644in}}%
\pgfusepath{stroke}%
\end{pgfscope}%
\begin{pgfscope}%
\pgfpathrectangle{\pgfqpoint{1.250000in}{4.155455in}}{\pgfqpoint{2.279412in}{2.004545in}}%
\pgfusepath{clip}%
\pgfsetbuttcap%
\pgfsetroundjoin%
\pgfsetlinewidth{0.384261pt}%
\definecolor{currentstroke}{rgb}{0.280267,0.073417,0.397163}%
\pgfsetstrokecolor{currentstroke}%
\pgfsetdash{}{0pt}%
\pgfpathmoveto{\pgfqpoint{2.151069in}{5.142644in}}%
\pgfpathlineto{\pgfqpoint{2.150506in}{5.143723in}}%
\pgfusepath{stroke}%
\end{pgfscope}%
\begin{pgfscope}%
\pgfpathrectangle{\pgfqpoint{1.250000in}{4.155455in}}{\pgfqpoint{2.279412in}{2.004545in}}%
\pgfusepath{clip}%
\pgfsetbuttcap%
\pgfsetroundjoin%
\pgfsetlinewidth{0.384328pt}%
\definecolor{currentstroke}{rgb}{0.280267,0.073417,0.397163}%
\pgfsetstrokecolor{currentstroke}%
\pgfsetdash{}{0pt}%
\pgfpathmoveto{\pgfqpoint{2.150506in}{5.143723in}}%
\pgfpathlineto{\pgfqpoint{2.150571in}{5.144600in}}%
\pgfusepath{stroke}%
\end{pgfscope}%
\begin{pgfscope}%
\pgfpathrectangle{\pgfqpoint{1.250000in}{4.155455in}}{\pgfqpoint{2.279412in}{2.004545in}}%
\pgfusepath{clip}%
\pgfsetbuttcap%
\pgfsetroundjoin%
\pgfsetlinewidth{0.384856pt}%
\definecolor{currentstroke}{rgb}{0.280267,0.073417,0.397163}%
\pgfsetstrokecolor{currentstroke}%
\pgfsetdash{}{0pt}%
\pgfpathmoveto{\pgfqpoint{2.150571in}{5.144600in}}%
\pgfpathlineto{\pgfqpoint{2.151180in}{5.145247in}}%
\pgfusepath{stroke}%
\end{pgfscope}%
\begin{pgfscope}%
\pgfpathrectangle{\pgfqpoint{1.250000in}{4.155455in}}{\pgfqpoint{2.279412in}{2.004545in}}%
\pgfusepath{clip}%
\pgfsetbuttcap%
\pgfsetroundjoin%
\pgfsetlinewidth{0.385478pt}%
\definecolor{currentstroke}{rgb}{0.280267,0.073417,0.397163}%
\pgfsetstrokecolor{currentstroke}%
\pgfsetdash{}{0pt}%
\pgfpathmoveto{\pgfqpoint{2.151180in}{5.145247in}}%
\pgfpathlineto{\pgfqpoint{2.151759in}{5.145673in}}%
\pgfusepath{stroke}%
\end{pgfscope}%
\begin{pgfscope}%
\pgfpathrectangle{\pgfqpoint{1.250000in}{4.155455in}}{\pgfqpoint{2.279412in}{2.004545in}}%
\pgfusepath{clip}%
\pgfsetbuttcap%
\pgfsetroundjoin%
\pgfsetlinewidth{0.385867pt}%
\definecolor{currentstroke}{rgb}{0.280267,0.073417,0.397163}%
\pgfsetstrokecolor{currentstroke}%
\pgfsetdash{}{0pt}%
\pgfpathmoveto{\pgfqpoint{2.151759in}{5.145673in}}%
\pgfpathlineto{\pgfqpoint{2.151522in}{5.145976in}}%
\pgfusepath{stroke}%
\end{pgfscope}%
\begin{pgfscope}%
\pgfpathrectangle{\pgfqpoint{1.250000in}{4.155455in}}{\pgfqpoint{2.279412in}{2.004545in}}%
\pgfusepath{clip}%
\pgfsetbuttcap%
\pgfsetroundjoin%
\pgfsetlinewidth{0.385891pt}%
\definecolor{currentstroke}{rgb}{0.280267,0.073417,0.397163}%
\pgfsetstrokecolor{currentstroke}%
\pgfsetdash{}{0pt}%
\pgfpathmoveto{\pgfqpoint{2.151522in}{5.145976in}}%
\pgfpathlineto{\pgfqpoint{2.150929in}{5.146205in}}%
\pgfusepath{stroke}%
\end{pgfscope}%
\begin{pgfscope}%
\pgfpathrectangle{\pgfqpoint{1.250000in}{4.155455in}}{\pgfqpoint{2.279412in}{2.004545in}}%
\pgfusepath{clip}%
\pgfsetbuttcap%
\pgfsetroundjoin%
\pgfsetlinewidth{0.385823pt}%
\definecolor{currentstroke}{rgb}{0.280267,0.073417,0.397163}%
\pgfsetstrokecolor{currentstroke}%
\pgfsetdash{}{0pt}%
\pgfpathmoveto{\pgfqpoint{2.150929in}{5.146205in}}%
\pgfpathlineto{\pgfqpoint{2.150636in}{5.146368in}}%
\pgfusepath{stroke}%
\end{pgfscope}%
\begin{pgfscope}%
\pgfpathrectangle{\pgfqpoint{1.250000in}{4.155455in}}{\pgfqpoint{2.279412in}{2.004545in}}%
\pgfusepath{clip}%
\pgfsetbuttcap%
\pgfsetroundjoin%
\pgfsetlinewidth{0.385835pt}%
\definecolor{currentstroke}{rgb}{0.280267,0.073417,0.397163}%
\pgfsetstrokecolor{currentstroke}%
\pgfsetdash{}{0pt}%
\pgfpathmoveto{\pgfqpoint{2.150636in}{5.146368in}}%
\pgfpathlineto{\pgfqpoint{2.150792in}{5.146469in}}%
\pgfusepath{stroke}%
\end{pgfscope}%
\begin{pgfscope}%
\pgfpathrectangle{\pgfqpoint{1.250000in}{4.155455in}}{\pgfqpoint{2.279412in}{2.004545in}}%
\pgfusepath{clip}%
\pgfsetbuttcap%
\pgfsetroundjoin%
\pgfsetlinewidth{0.385921pt}%
\definecolor{currentstroke}{rgb}{0.280267,0.073417,0.397163}%
\pgfsetstrokecolor{currentstroke}%
\pgfsetdash{}{0pt}%
\pgfpathmoveto{\pgfqpoint{2.150792in}{5.146469in}}%
\pgfpathlineto{\pgfqpoint{2.151250in}{5.146522in}}%
\pgfusepath{stroke}%
\end{pgfscope}%
\begin{pgfscope}%
\pgfpathrectangle{\pgfqpoint{1.250000in}{4.155455in}}{\pgfqpoint{2.279412in}{2.004545in}}%
\pgfusepath{clip}%
\pgfsetbuttcap%
\pgfsetroundjoin%
\pgfsetlinewidth{0.386035pt}%
\definecolor{currentstroke}{rgb}{0.280267,0.073417,0.397163}%
\pgfsetstrokecolor{currentstroke}%
\pgfsetdash{}{0pt}%
\pgfpathmoveto{\pgfqpoint{2.151250in}{5.146522in}}%
\pgfpathlineto{\pgfqpoint{2.151512in}{5.146560in}}%
\pgfusepath{stroke}%
\end{pgfscope}%
\begin{pgfscope}%
\pgfpathrectangle{\pgfqpoint{1.250000in}{4.155455in}}{\pgfqpoint{2.279412in}{2.004545in}}%
\pgfusepath{clip}%
\pgfsetbuttcap%
\pgfsetroundjoin%
\pgfsetlinewidth{0.386099pt}%
\definecolor{currentstroke}{rgb}{0.280267,0.073417,0.397163}%
\pgfsetstrokecolor{currentstroke}%
\pgfsetdash{}{0pt}%
\pgfpathmoveto{\pgfqpoint{2.151512in}{5.146560in}}%
\pgfpathlineto{\pgfqpoint{2.151264in}{5.146609in}}%
\pgfusepath{stroke}%
\end{pgfscope}%
\begin{pgfscope}%
\pgfpathrectangle{\pgfqpoint{1.250000in}{4.155455in}}{\pgfqpoint{2.279412in}{2.004545in}}%
\pgfusepath{clip}%
\pgfsetbuttcap%
\pgfsetroundjoin%
\pgfsetlinewidth{0.386074pt}%
\definecolor{currentstroke}{rgb}{0.280267,0.073417,0.397163}%
\pgfsetstrokecolor{currentstroke}%
\pgfsetdash{}{0pt}%
\pgfpathmoveto{\pgfqpoint{2.151264in}{5.146609in}}%
\pgfpathlineto{\pgfqpoint{2.150888in}{5.146655in}}%
\pgfusepath{stroke}%
\end{pgfscope}%
\begin{pgfscope}%
\pgfpathrectangle{\pgfqpoint{1.250000in}{4.155455in}}{\pgfqpoint{2.279412in}{2.004545in}}%
\pgfusepath{clip}%
\pgfsetbuttcap%
\pgfsetroundjoin%
\pgfsetlinewidth{0.386030pt}%
\definecolor{currentstroke}{rgb}{0.280267,0.073417,0.397163}%
\pgfsetstrokecolor{currentstroke}%
\pgfsetdash{}{0pt}%
\pgfpathmoveto{\pgfqpoint{2.150888in}{5.146655in}}%
\pgfpathlineto{\pgfqpoint{2.150759in}{5.146680in}}%
\pgfusepath{stroke}%
\end{pgfscope}%
\begin{pgfscope}%
\pgfpathrectangle{\pgfqpoint{1.250000in}{4.155455in}}{\pgfqpoint{2.279412in}{2.004545in}}%
\pgfusepath{clip}%
\pgfsetbuttcap%
\pgfsetroundjoin%
\pgfsetlinewidth{0.386022pt}%
\definecolor{currentstroke}{rgb}{0.280267,0.073417,0.397163}%
\pgfsetstrokecolor{currentstroke}%
\pgfsetdash{}{0pt}%
\pgfpathmoveto{\pgfqpoint{2.150759in}{5.146680in}}%
\pgfpathlineto{\pgfqpoint{2.150943in}{5.146681in}}%
\pgfusepath{stroke}%
\end{pgfscope}%
\begin{pgfscope}%
\pgfpathrectangle{\pgfqpoint{1.250000in}{4.155455in}}{\pgfqpoint{2.279412in}{2.004545in}}%
\pgfusepath{clip}%
\pgfsetbuttcap%
\pgfsetroundjoin%
\pgfsetlinewidth{0.386052pt}%
\definecolor{currentstroke}{rgb}{0.280267,0.073417,0.397163}%
\pgfsetstrokecolor{currentstroke}%
\pgfsetdash{}{0pt}%
\pgfpathmoveto{\pgfqpoint{2.150943in}{5.146681in}}%
\pgfpathlineto{\pgfqpoint{2.151255in}{5.146672in}}%
\pgfusepath{stroke}%
\end{pgfscope}%
\begin{pgfscope}%
\pgfpathrectangle{\pgfqpoint{1.250000in}{4.155455in}}{\pgfqpoint{2.279412in}{2.004545in}}%
\pgfusepath{clip}%
\pgfsetbuttcap%
\pgfsetroundjoin%
\pgfsetlinewidth{0.386098pt}%
\definecolor{currentstroke}{rgb}{0.280267,0.073417,0.397163}%
\pgfsetstrokecolor{currentstroke}%
\pgfsetdash{}{0pt}%
\pgfpathmoveto{\pgfqpoint{2.151255in}{5.146672in}}%
\pgfpathlineto{\pgfqpoint{2.151342in}{5.146673in}}%
\pgfusepath{stroke}%
\end{pgfscope}%
\begin{pgfscope}%
\pgfpathrectangle{\pgfqpoint{1.250000in}{4.155455in}}{\pgfqpoint{2.279412in}{2.004545in}}%
\pgfusepath{clip}%
\pgfsetbuttcap%
\pgfsetroundjoin%
\pgfsetlinewidth{0.386113pt}%
\definecolor{currentstroke}{rgb}{0.280267,0.073417,0.397163}%
\pgfsetstrokecolor{currentstroke}%
\pgfsetdash{}{0pt}%
\pgfpathmoveto{\pgfqpoint{2.151342in}{5.146673in}}%
\pgfpathlineto{\pgfqpoint{2.151127in}{5.146690in}}%
\pgfusepath{stroke}%
\end{pgfscope}%
\begin{pgfscope}%
\pgfpathrectangle{\pgfqpoint{1.250000in}{4.155455in}}{\pgfqpoint{2.279412in}{2.004545in}}%
\pgfusepath{clip}%
\pgfsetbuttcap%
\pgfsetroundjoin%
\pgfsetlinewidth{0.386085pt}%
\definecolor{currentstroke}{rgb}{0.280267,0.073417,0.397163}%
\pgfsetstrokecolor{currentstroke}%
\pgfsetdash{}{0pt}%
\pgfpathmoveto{\pgfqpoint{2.151127in}{5.146690in}}%
\pgfpathlineto{\pgfqpoint{2.150895in}{5.146706in}}%
\pgfusepath{stroke}%
\end{pgfscope}%
\begin{pgfscope}%
\pgfpathrectangle{\pgfqpoint{1.250000in}{4.155455in}}{\pgfqpoint{2.279412in}{2.004545in}}%
\pgfusepath{clip}%
\pgfsetbuttcap%
\pgfsetroundjoin%
\pgfsetlinewidth{0.386056pt}%
\definecolor{currentstroke}{rgb}{0.280267,0.073417,0.397163}%
\pgfsetstrokecolor{currentstroke}%
\pgfsetdash{}{0pt}%
\pgfpathmoveto{\pgfqpoint{2.150895in}{5.146706in}}%
\pgfpathlineto{\pgfqpoint{2.150862in}{5.146710in}}%
\pgfusepath{stroke}%
\end{pgfscope}%
\begin{pgfscope}%
\pgfpathrectangle{\pgfqpoint{1.250000in}{4.155455in}}{\pgfqpoint{2.279412in}{2.004545in}}%
\pgfusepath{clip}%
\pgfsetbuttcap%
\pgfsetroundjoin%
\pgfsetlinewidth{0.386053pt}%
\definecolor{currentstroke}{rgb}{0.280267,0.073417,0.397163}%
\pgfsetstrokecolor{currentstroke}%
\pgfsetdash{}{0pt}%
\pgfpathmoveto{\pgfqpoint{2.150862in}{5.146710in}}%
\pgfpathlineto{\pgfqpoint{2.151034in}{5.146701in}}%
\pgfusepath{stroke}%
\end{pgfscope}%
\begin{pgfscope}%
\pgfpathrectangle{\pgfqpoint{1.250000in}{4.155455in}}{\pgfqpoint{2.279412in}{2.004545in}}%
\pgfusepath{clip}%
\pgfsetbuttcap%
\pgfsetroundjoin%
\pgfsetlinewidth{0.386075pt}%
\definecolor{currentstroke}{rgb}{0.280267,0.073417,0.397163}%
\pgfsetstrokecolor{currentstroke}%
\pgfsetdash{}{0pt}%
\pgfpathmoveto{\pgfqpoint{2.151034in}{5.146701in}}%
\pgfpathlineto{\pgfqpoint{2.151227in}{5.146691in}}%
\pgfusepath{stroke}%
\end{pgfscope}%
\begin{pgfscope}%
\pgfpathrectangle{\pgfqpoint{1.250000in}{4.155455in}}{\pgfqpoint{2.279412in}{2.004545in}}%
\pgfusepath{clip}%
\pgfsetbuttcap%
\pgfsetroundjoin%
\pgfsetlinewidth{0.386101pt}%
\definecolor{currentstroke}{rgb}{0.280267,0.073417,0.397163}%
\pgfsetstrokecolor{currentstroke}%
\pgfsetdash{}{0pt}%
\pgfpathmoveto{\pgfqpoint{2.151227in}{5.146691in}}%
\pgfpathlineto{\pgfqpoint{2.151226in}{5.146691in}}%
\pgfusepath{stroke}%
\end{pgfscope}%
\begin{pgfscope}%
\pgfpathrectangle{\pgfqpoint{1.250000in}{4.155455in}}{\pgfqpoint{2.279412in}{2.004545in}}%
\pgfusepath{clip}%
\pgfsetbuttcap%
\pgfsetroundjoin%
\pgfsetlinewidth{0.386102pt}%
\definecolor{currentstroke}{rgb}{0.280267,0.073417,0.397163}%
\pgfsetstrokecolor{currentstroke}%
\pgfsetdash{}{0pt}%
\pgfpathmoveto{\pgfqpoint{2.151226in}{5.146691in}}%
\pgfpathlineto{\pgfqpoint{2.151057in}{5.146702in}}%
\pgfusepath{stroke}%
\end{pgfscope}%
\begin{pgfscope}%
\pgfpathrectangle{\pgfqpoint{1.250000in}{4.155455in}}{\pgfqpoint{2.279412in}{2.004545in}}%
\pgfusepath{clip}%
\pgfsetbuttcap%
\pgfsetroundjoin%
\pgfsetlinewidth{0.386080pt}%
\definecolor{currentstroke}{rgb}{0.280267,0.073417,0.397163}%
\pgfsetstrokecolor{currentstroke}%
\pgfsetdash{}{0pt}%
\pgfpathmoveto{\pgfqpoint{2.151057in}{5.146702in}}%
\pgfpathlineto{\pgfqpoint{2.150922in}{5.146710in}}%
\pgfusepath{stroke}%
\end{pgfscope}%
\begin{pgfscope}%
\pgfpathrectangle{\pgfqpoint{1.250000in}{4.155455in}}{\pgfqpoint{2.279412in}{2.004545in}}%
\pgfusepath{clip}%
\pgfsetbuttcap%
\pgfsetroundjoin%
\pgfsetlinewidth{0.386062pt}%
\definecolor{currentstroke}{rgb}{0.280267,0.073417,0.397163}%
\pgfsetstrokecolor{currentstroke}%
\pgfsetdash{}{0pt}%
\pgfpathmoveto{\pgfqpoint{2.150922in}{5.146710in}}%
\pgfpathlineto{\pgfqpoint{2.150942in}{5.146709in}}%
\pgfusepath{stroke}%
\end{pgfscope}%
\begin{pgfscope}%
\pgfpathrectangle{\pgfqpoint{1.250000in}{4.155455in}}{\pgfqpoint{2.279412in}{2.004545in}}%
\pgfusepath{clip}%
\pgfsetbuttcap%
\pgfsetroundjoin%
\pgfsetlinewidth{0.386065pt}%
\definecolor{currentstroke}{rgb}{0.280267,0.073417,0.397163}%
\pgfsetstrokecolor{currentstroke}%
\pgfsetdash{}{0pt}%
\pgfpathmoveto{\pgfqpoint{2.150942in}{5.146709in}}%
\pgfpathlineto{\pgfqpoint{2.151081in}{5.146701in}}%
\pgfusepath{stroke}%
\end{pgfscope}%
\begin{pgfscope}%
\pgfpathrectangle{\pgfqpoint{1.250000in}{4.155455in}}{\pgfqpoint{2.279412in}{2.004545in}}%
\pgfusepath{clip}%
\pgfsetbuttcap%
\pgfsetroundjoin%
\pgfsetlinewidth{0.386083pt}%
\definecolor{currentstroke}{rgb}{0.280267,0.073417,0.397163}%
\pgfsetstrokecolor{currentstroke}%
\pgfsetdash{}{0pt}%
\pgfpathmoveto{\pgfqpoint{2.151081in}{5.146701in}}%
\pgfpathlineto{\pgfqpoint{2.151188in}{5.146695in}}%
\pgfusepath{stroke}%
\end{pgfscope}%
\begin{pgfscope}%
\pgfpathrectangle{\pgfqpoint{1.250000in}{4.155455in}}{\pgfqpoint{2.279412in}{2.004545in}}%
\pgfusepath{clip}%
\pgfsetbuttcap%
\pgfsetroundjoin%
\pgfsetlinewidth{0.386097pt}%
\definecolor{currentstroke}{rgb}{0.280267,0.073417,0.397163}%
\pgfsetstrokecolor{currentstroke}%
\pgfsetdash{}{0pt}%
\pgfpathmoveto{\pgfqpoint{2.151188in}{5.146695in}}%
\pgfpathlineto{\pgfqpoint{2.151149in}{5.146697in}}%
\pgfusepath{stroke}%
\end{pgfscope}%
\begin{pgfscope}%
\pgfpathrectangle{\pgfqpoint{1.250000in}{4.155455in}}{\pgfqpoint{2.279412in}{2.004545in}}%
\pgfusepath{clip}%
\pgfsetbuttcap%
\pgfsetroundjoin%
\pgfsetlinewidth{0.386092pt}%
\definecolor{currentstroke}{rgb}{0.280267,0.073417,0.397163}%
\pgfsetstrokecolor{currentstroke}%
\pgfsetdash{}{0pt}%
\pgfpathmoveto{\pgfqpoint{2.151149in}{5.146697in}}%
\pgfpathlineto{\pgfqpoint{2.151027in}{5.146705in}}%
\pgfusepath{stroke}%
\end{pgfscope}%
\begin{pgfscope}%
\pgfpathrectangle{\pgfqpoint{1.250000in}{4.155455in}}{\pgfqpoint{2.279412in}{2.004545in}}%
\pgfusepath{clip}%
\pgfsetbuttcap%
\pgfsetroundjoin%
\pgfsetlinewidth{0.386076pt}%
\definecolor{currentstroke}{rgb}{0.280267,0.073417,0.397163}%
\pgfsetstrokecolor{currentstroke}%
\pgfsetdash{}{0pt}%
\pgfpathmoveto{\pgfqpoint{2.151027in}{5.146705in}}%
\pgfpathlineto{\pgfqpoint{2.150956in}{5.146709in}}%
\pgfusepath{stroke}%
\end{pgfscope}%
\begin{pgfscope}%
\pgfpathrectangle{\pgfqpoint{1.250000in}{4.155455in}}{\pgfqpoint{2.279412in}{2.004545in}}%
\pgfusepath{clip}%
\pgfsetbuttcap%
\pgfsetroundjoin%
\pgfsetlinewidth{0.386067pt}%
\definecolor{currentstroke}{rgb}{0.280267,0.073417,0.397163}%
\pgfsetstrokecolor{currentstroke}%
\pgfsetdash{}{0pt}%
\pgfpathmoveto{\pgfqpoint{2.150956in}{5.146709in}}%
\pgfpathlineto{\pgfqpoint{2.150998in}{5.146706in}}%
\pgfusepath{stroke}%
\end{pgfscope}%
\begin{pgfscope}%
\pgfpathrectangle{\pgfqpoint{1.250000in}{4.155455in}}{\pgfqpoint{2.279412in}{2.004545in}}%
\pgfusepath{clip}%
\pgfsetbuttcap%
\pgfsetroundjoin%
\pgfsetlinewidth{0.386072pt}%
\definecolor{currentstroke}{rgb}{0.280267,0.073417,0.397163}%
\pgfsetstrokecolor{currentstroke}%
\pgfsetdash{}{0pt}%
\pgfpathmoveto{\pgfqpoint{2.150998in}{5.146706in}}%
\pgfpathlineto{\pgfqpoint{2.151099in}{5.146700in}}%
\pgfusepath{stroke}%
\end{pgfscope}%
\begin{pgfscope}%
\pgfpathrectangle{\pgfqpoint{1.250000in}{4.155455in}}{\pgfqpoint{2.279412in}{2.004545in}}%
\pgfusepath{clip}%
\pgfsetbuttcap%
\pgfsetroundjoin%
\pgfsetlinewidth{0.386085pt}%
\definecolor{currentstroke}{rgb}{0.280267,0.073417,0.397163}%
\pgfsetstrokecolor{currentstroke}%
\pgfsetdash{}{0pt}%
\pgfpathmoveto{\pgfqpoint{2.151099in}{5.146700in}}%
\pgfpathlineto{\pgfqpoint{2.151149in}{5.146697in}}%
\pgfusepath{stroke}%
\end{pgfscope}%
\begin{pgfscope}%
\pgfpathrectangle{\pgfqpoint{1.250000in}{4.155455in}}{\pgfqpoint{2.279412in}{2.004545in}}%
\pgfusepath{clip}%
\pgfsetbuttcap%
\pgfsetroundjoin%
\pgfsetlinewidth{0.386092pt}%
\definecolor{currentstroke}{rgb}{0.280267,0.073417,0.397163}%
\pgfsetstrokecolor{currentstroke}%
\pgfsetdash{}{0pt}%
\pgfpathmoveto{\pgfqpoint{2.151149in}{5.146697in}}%
\pgfpathlineto{\pgfqpoint{2.151101in}{5.146700in}}%
\pgfusepath{stroke}%
\end{pgfscope}%
\begin{pgfscope}%
\pgfpathrectangle{\pgfqpoint{1.250000in}{4.155455in}}{\pgfqpoint{2.279412in}{2.004545in}}%
\pgfusepath{clip}%
\pgfsetbuttcap%
\pgfsetroundjoin%
\pgfsetlinewidth{0.386086pt}%
\definecolor{currentstroke}{rgb}{0.280267,0.073417,0.397163}%
\pgfsetstrokecolor{currentstroke}%
\pgfsetdash{}{0pt}%
\pgfpathmoveto{\pgfqpoint{2.151101in}{5.146700in}}%
\pgfpathlineto{\pgfqpoint{2.151017in}{5.146705in}}%
\pgfusepath{stroke}%
\end{pgfscope}%
\begin{pgfscope}%
\pgfpathrectangle{\pgfqpoint{1.250000in}{4.155455in}}{\pgfqpoint{2.279412in}{2.004545in}}%
\pgfusepath{clip}%
\pgfsetbuttcap%
\pgfsetroundjoin%
\pgfsetlinewidth{0.386075pt}%
\definecolor{currentstroke}{rgb}{0.280267,0.073417,0.397163}%
\pgfsetstrokecolor{currentstroke}%
\pgfsetdash{}{0pt}%
\pgfpathmoveto{\pgfqpoint{2.151017in}{5.146705in}}%
\pgfpathlineto{\pgfqpoint{2.150988in}{5.146707in}}%
\pgfusepath{stroke}%
\end{pgfscope}%
\begin{pgfscope}%
\pgfpathrectangle{\pgfqpoint{1.250000in}{4.155455in}}{\pgfqpoint{2.279412in}{2.004545in}}%
\pgfusepath{clip}%
\pgfsetbuttcap%
\pgfsetroundjoin%
\pgfsetlinewidth{0.386071pt}%
\definecolor{currentstroke}{rgb}{0.280267,0.073417,0.397163}%
\pgfsetstrokecolor{currentstroke}%
\pgfsetdash{}{0pt}%
\pgfpathmoveto{\pgfqpoint{2.150988in}{5.146707in}}%
\pgfpathlineto{\pgfqpoint{2.151034in}{5.146704in}}%
\pgfusepath{stroke}%
\end{pgfscope}%
\begin{pgfscope}%
\pgfpathrectangle{\pgfqpoint{1.250000in}{4.155455in}}{\pgfqpoint{2.279412in}{2.004545in}}%
\pgfusepath{clip}%
\pgfsetbuttcap%
\pgfsetroundjoin%
\pgfsetlinewidth{0.386077pt}%
\definecolor{currentstroke}{rgb}{0.280267,0.073417,0.397163}%
\pgfsetstrokecolor{currentstroke}%
\pgfsetdash{}{0pt}%
\pgfpathmoveto{\pgfqpoint{2.151034in}{5.146704in}}%
\pgfpathlineto{\pgfqpoint{2.151101in}{5.146700in}}%
\pgfusepath{stroke}%
\end{pgfscope}%
\begin{pgfscope}%
\pgfpathrectangle{\pgfqpoint{1.250000in}{4.155455in}}{\pgfqpoint{2.279412in}{2.004545in}}%
\pgfusepath{clip}%
\pgfsetbuttcap%
\pgfsetroundjoin%
\pgfsetlinewidth{0.386086pt}%
\definecolor{currentstroke}{rgb}{0.280267,0.073417,0.397163}%
\pgfsetstrokecolor{currentstroke}%
\pgfsetdash{}{0pt}%
\pgfpathmoveto{\pgfqpoint{2.151101in}{5.146700in}}%
\pgfpathlineto{\pgfqpoint{2.151118in}{5.146699in}}%
\pgfusepath{stroke}%
\end{pgfscope}%
\begin{pgfscope}%
\pgfpathrectangle{\pgfqpoint{1.250000in}{4.155455in}}{\pgfqpoint{2.279412in}{2.004545in}}%
\pgfusepath{clip}%
\pgfsetbuttcap%
\pgfsetroundjoin%
\pgfsetlinewidth{0.386088pt}%
\definecolor{currentstroke}{rgb}{0.280267,0.073417,0.397163}%
\pgfsetstrokecolor{currentstroke}%
\pgfsetdash{}{0pt}%
\pgfpathmoveto{\pgfqpoint{2.151118in}{5.146699in}}%
\pgfpathlineto{\pgfqpoint{2.151072in}{5.146702in}}%
\pgfusepath{stroke}%
\end{pgfscope}%
\begin{pgfscope}%
\pgfpathrectangle{\pgfqpoint{1.250000in}{4.155455in}}{\pgfqpoint{2.279412in}{2.004545in}}%
\pgfusepath{clip}%
\pgfsetbuttcap%
\pgfsetroundjoin%
\pgfsetlinewidth{0.386082pt}%
\definecolor{currentstroke}{rgb}{0.280267,0.073417,0.397163}%
\pgfsetstrokecolor{currentstroke}%
\pgfsetdash{}{0pt}%
\pgfpathmoveto{\pgfqpoint{2.151072in}{5.146702in}}%
\pgfpathlineto{\pgfqpoint{2.151020in}{5.146705in}}%
\pgfusepath{stroke}%
\end{pgfscope}%
\begin{pgfscope}%
\pgfpathrectangle{\pgfqpoint{1.250000in}{4.155455in}}{\pgfqpoint{2.279412in}{2.004545in}}%
\pgfusepath{clip}%
\pgfsetbuttcap%
\pgfsetroundjoin%
\pgfsetlinewidth{0.386075pt}%
\definecolor{currentstroke}{rgb}{0.280267,0.073417,0.397163}%
\pgfsetstrokecolor{currentstroke}%
\pgfsetdash{}{0pt}%
\pgfpathmoveto{\pgfqpoint{2.151020in}{5.146705in}}%
\pgfpathlineto{\pgfqpoint{2.151013in}{5.146706in}}%
\pgfusepath{stroke}%
\end{pgfscope}%
\begin{pgfscope}%
\pgfpathrectangle{\pgfqpoint{1.250000in}{4.155455in}}{\pgfqpoint{2.279412in}{2.004545in}}%
\pgfusepath{clip}%
\pgfsetbuttcap%
\pgfsetroundjoin%
\pgfsetlinewidth{0.386074pt}%
\definecolor{currentstroke}{rgb}{0.280267,0.073417,0.397163}%
\pgfsetstrokecolor{currentstroke}%
\pgfsetdash{}{0pt}%
\pgfpathmoveto{\pgfqpoint{2.151013in}{5.146706in}}%
\pgfpathlineto{\pgfqpoint{2.151054in}{5.146703in}}%
\pgfusepath{stroke}%
\end{pgfscope}%
\begin{pgfscope}%
\pgfpathrectangle{\pgfqpoint{1.250000in}{4.155455in}}{\pgfqpoint{2.279412in}{2.004545in}}%
\pgfusepath{clip}%
\pgfsetbuttcap%
\pgfsetroundjoin%
\pgfsetlinewidth{0.386080pt}%
\definecolor{currentstroke}{rgb}{0.280267,0.073417,0.397163}%
\pgfsetstrokecolor{currentstroke}%
\pgfsetdash{}{0pt}%
\pgfpathmoveto{\pgfqpoint{2.151054in}{5.146703in}}%
\pgfpathlineto{\pgfqpoint{2.151095in}{5.146701in}}%
\pgfusepath{stroke}%
\end{pgfscope}%
\begin{pgfscope}%
\pgfpathrectangle{\pgfqpoint{1.250000in}{4.155455in}}{\pgfqpoint{2.279412in}{2.004545in}}%
\pgfusepath{clip}%
\pgfsetbuttcap%
\pgfsetroundjoin%
\pgfsetlinewidth{0.386085pt}%
\definecolor{currentstroke}{rgb}{0.280267,0.073417,0.397163}%
\pgfsetstrokecolor{currentstroke}%
\pgfsetdash{}{0pt}%
\pgfpathmoveto{\pgfqpoint{2.151095in}{5.146701in}}%
\pgfpathlineto{\pgfqpoint{2.151094in}{5.146701in}}%
\pgfusepath{stroke}%
\end{pgfscope}%
\begin{pgfscope}%
\pgfpathrectangle{\pgfqpoint{1.250000in}{4.155455in}}{\pgfqpoint{2.279412in}{2.004545in}}%
\pgfusepath{clip}%
\pgfsetbuttcap%
\pgfsetroundjoin%
\pgfsetlinewidth{0.386085pt}%
\definecolor{currentstroke}{rgb}{0.280267,0.073417,0.397163}%
\pgfsetstrokecolor{currentstroke}%
\pgfsetdash{}{0pt}%
\pgfpathmoveto{\pgfqpoint{2.151094in}{5.146701in}}%
\pgfpathlineto{\pgfqpoint{2.151057in}{5.146703in}}%
\pgfusepath{stroke}%
\end{pgfscope}%
\begin{pgfscope}%
\pgfpathrectangle{\pgfqpoint{1.250000in}{4.155455in}}{\pgfqpoint{2.279412in}{2.004545in}}%
\pgfusepath{clip}%
\pgfsetbuttcap%
\pgfsetroundjoin%
\pgfsetlinewidth{0.386080pt}%
\definecolor{currentstroke}{rgb}{0.280267,0.073417,0.397163}%
\pgfsetstrokecolor{currentstroke}%
\pgfsetdash{}{0pt}%
\pgfpathmoveto{\pgfqpoint{2.151057in}{5.146703in}}%
\pgfpathlineto{\pgfqpoint{2.151027in}{5.146705in}}%
\pgfusepath{stroke}%
\end{pgfscope}%
\begin{pgfscope}%
\pgfpathrectangle{\pgfqpoint{1.250000in}{4.155455in}}{\pgfqpoint{2.279412in}{2.004545in}}%
\pgfusepath{clip}%
\pgfsetbuttcap%
\pgfsetroundjoin%
\pgfsetlinewidth{0.386076pt}%
\definecolor{currentstroke}{rgb}{0.280267,0.073417,0.397163}%
\pgfsetstrokecolor{currentstroke}%
\pgfsetdash{}{0pt}%
\pgfpathmoveto{\pgfqpoint{2.151027in}{5.146705in}}%
\pgfpathlineto{\pgfqpoint{2.151032in}{5.146704in}}%
\pgfusepath{stroke}%
\end{pgfscope}%
\begin{pgfscope}%
\pgfpathrectangle{\pgfqpoint{1.250000in}{4.155455in}}{\pgfqpoint{2.279412in}{2.004545in}}%
\pgfusepath{clip}%
\pgfsetbuttcap%
\pgfsetroundjoin%
\pgfsetlinewidth{0.386077pt}%
\definecolor{currentstroke}{rgb}{0.280267,0.073417,0.397163}%
\pgfsetstrokecolor{currentstroke}%
\pgfsetdash{}{0pt}%
\pgfpathmoveto{\pgfqpoint{2.151032in}{5.146704in}}%
\pgfpathlineto{\pgfqpoint{2.151064in}{5.146702in}}%
\pgfusepath{stroke}%
\end{pgfscope}%
\begin{pgfscope}%
\pgfpathrectangle{\pgfqpoint{1.250000in}{4.155455in}}{\pgfqpoint{2.279412in}{2.004545in}}%
\pgfusepath{clip}%
\pgfsetbuttcap%
\pgfsetroundjoin%
\pgfsetlinewidth{0.386081pt}%
\definecolor{currentstroke}{rgb}{0.280267,0.073417,0.397163}%
\pgfsetstrokecolor{currentstroke}%
\pgfsetdash{}{0pt}%
\pgfpathmoveto{\pgfqpoint{2.151064in}{5.146702in}}%
\pgfpathlineto{\pgfqpoint{2.151087in}{5.146701in}}%
\pgfusepath{stroke}%
\end{pgfscope}%
\begin{pgfscope}%
\pgfpathrectangle{\pgfqpoint{1.250000in}{4.155455in}}{\pgfqpoint{2.279412in}{2.004545in}}%
\pgfusepath{clip}%
\pgfsetbuttcap%
\pgfsetroundjoin%
\pgfsetlinewidth{0.386084pt}%
\definecolor{currentstroke}{rgb}{0.280267,0.073417,0.397163}%
\pgfsetstrokecolor{currentstroke}%
\pgfsetdash{}{0pt}%
\pgfpathmoveto{\pgfqpoint{2.151087in}{5.146701in}}%
\pgfpathlineto{\pgfqpoint{2.151078in}{5.146702in}}%
\pgfusepath{stroke}%
\end{pgfscope}%
\begin{pgfscope}%
\pgfpathrectangle{\pgfqpoint{1.250000in}{4.155455in}}{\pgfqpoint{2.279412in}{2.004545in}}%
\pgfusepath{clip}%
\pgfsetbuttcap%
\pgfsetroundjoin%
\pgfsetlinewidth{0.386083pt}%
\definecolor{currentstroke}{rgb}{0.280267,0.073417,0.397163}%
\pgfsetstrokecolor{currentstroke}%
\pgfsetdash{}{0pt}%
\pgfpathmoveto{\pgfqpoint{2.151078in}{5.146702in}}%
\pgfpathlineto{\pgfqpoint{2.151051in}{5.146703in}}%
\pgfusepath{stroke}%
\end{pgfscope}%
\begin{pgfscope}%
\pgfpathrectangle{\pgfqpoint{1.250000in}{4.155455in}}{\pgfqpoint{2.279412in}{2.004545in}}%
\pgfusepath{clip}%
\pgfsetbuttcap%
\pgfsetroundjoin%
\pgfsetlinewidth{0.386079pt}%
\definecolor{currentstroke}{rgb}{0.280267,0.073417,0.397163}%
\pgfsetstrokecolor{currentstroke}%
\pgfsetdash{}{0pt}%
\pgfpathmoveto{\pgfqpoint{2.151051in}{5.146703in}}%
\pgfpathlineto{\pgfqpoint{2.151035in}{5.146704in}}%
\pgfusepath{stroke}%
\end{pgfscope}%
\begin{pgfscope}%
\pgfpathrectangle{\pgfqpoint{1.250000in}{4.155455in}}{\pgfqpoint{2.279412in}{2.004545in}}%
\pgfusepath{clip}%
\pgfsetbuttcap%
\pgfsetroundjoin%
\pgfsetlinewidth{0.386077pt}%
\definecolor{currentstroke}{rgb}{0.280267,0.073417,0.397163}%
\pgfsetstrokecolor{currentstroke}%
\pgfsetdash{}{0pt}%
\pgfpathmoveto{\pgfqpoint{2.151035in}{5.146704in}}%
\pgfpathlineto{\pgfqpoint{2.151045in}{5.146704in}}%
\pgfusepath{stroke}%
\end{pgfscope}%
\begin{pgfscope}%
\pgfpathrectangle{\pgfqpoint{1.250000in}{4.155455in}}{\pgfqpoint{2.279412in}{2.004545in}}%
\pgfusepath{clip}%
\pgfsetbuttcap%
\pgfsetroundjoin%
\pgfsetlinewidth{0.386078pt}%
\definecolor{currentstroke}{rgb}{0.280267,0.073417,0.397163}%
\pgfsetstrokecolor{currentstroke}%
\pgfsetdash{}{0pt}%
\pgfpathmoveto{\pgfqpoint{2.151045in}{5.146704in}}%
\pgfpathlineto{\pgfqpoint{2.151068in}{5.146702in}}%
\pgfusepath{stroke}%
\end{pgfscope}%
\begin{pgfscope}%
\pgfpathrectangle{\pgfqpoint{1.250000in}{4.155455in}}{\pgfqpoint{2.279412in}{2.004545in}}%
\pgfusepath{clip}%
\pgfsetbuttcap%
\pgfsetroundjoin%
\pgfsetlinewidth{0.386081pt}%
\definecolor{currentstroke}{rgb}{0.280267,0.073417,0.397163}%
\pgfsetstrokecolor{currentstroke}%
\pgfsetdash{}{0pt}%
\pgfpathmoveto{\pgfqpoint{2.151068in}{5.146702in}}%
\pgfpathlineto{\pgfqpoint{2.151079in}{5.146702in}}%
\pgfusepath{stroke}%
\end{pgfscope}%
\begin{pgfscope}%
\pgfpathrectangle{\pgfqpoint{1.250000in}{4.155455in}}{\pgfqpoint{2.279412in}{2.004545in}}%
\pgfusepath{clip}%
\pgfsetbuttcap%
\pgfsetroundjoin%
\pgfsetlinewidth{0.386083pt}%
\definecolor{currentstroke}{rgb}{0.280267,0.073417,0.397163}%
\pgfsetstrokecolor{currentstroke}%
\pgfsetdash{}{0pt}%
\pgfpathmoveto{\pgfqpoint{2.151079in}{5.146702in}}%
\pgfpathlineto{\pgfqpoint{2.151068in}{5.146702in}}%
\pgfusepath{stroke}%
\end{pgfscope}%
\begin{pgfscope}%
\pgfpathrectangle{\pgfqpoint{1.250000in}{4.155455in}}{\pgfqpoint{2.279412in}{2.004545in}}%
\pgfusepath{clip}%
\pgfsetbuttcap%
\pgfsetroundjoin%
\pgfsetlinewidth{0.386081pt}%
\definecolor{currentstroke}{rgb}{0.280267,0.073417,0.397163}%
\pgfsetstrokecolor{currentstroke}%
\pgfsetdash{}{0pt}%
\pgfpathmoveto{\pgfqpoint{2.151068in}{5.146702in}}%
\pgfpathlineto{\pgfqpoint{2.151049in}{5.146703in}}%
\pgfusepath{stroke}%
\end{pgfscope}%
\begin{pgfscope}%
\pgfpathrectangle{\pgfqpoint{1.250000in}{4.155455in}}{\pgfqpoint{2.279412in}{2.004545in}}%
\pgfusepath{clip}%
\pgfsetbuttcap%
\pgfsetroundjoin%
\pgfsetlinewidth{0.386079pt}%
\definecolor{currentstroke}{rgb}{0.280267,0.073417,0.397163}%
\pgfsetstrokecolor{currentstroke}%
\pgfsetdash{}{0pt}%
\pgfpathmoveto{\pgfqpoint{2.151049in}{5.146703in}}%
\pgfpathlineto{\pgfqpoint{2.151043in}{5.146704in}}%
\pgfusepath{stroke}%
\end{pgfscope}%
\begin{pgfscope}%
\pgfpathrectangle{\pgfqpoint{1.250000in}{4.155455in}}{\pgfqpoint{2.279412in}{2.004545in}}%
\pgfusepath{clip}%
\pgfsetbuttcap%
\pgfsetroundjoin%
\pgfsetlinewidth{0.386078pt}%
\definecolor{currentstroke}{rgb}{0.280267,0.073417,0.397163}%
\pgfsetstrokecolor{currentstroke}%
\pgfsetdash{}{0pt}%
\pgfpathmoveto{\pgfqpoint{2.151043in}{5.146704in}}%
\pgfpathlineto{\pgfqpoint{2.151053in}{5.146703in}}%
\pgfusepath{stroke}%
\end{pgfscope}%
\begin{pgfscope}%
\pgfpathrectangle{\pgfqpoint{1.250000in}{4.155455in}}{\pgfqpoint{2.279412in}{2.004545in}}%
\pgfusepath{clip}%
\pgfsetbuttcap%
\pgfsetroundjoin%
\pgfsetlinewidth{0.386080pt}%
\definecolor{currentstroke}{rgb}{0.280267,0.073417,0.397163}%
\pgfsetstrokecolor{currentstroke}%
\pgfsetdash{}{0pt}%
\pgfpathmoveto{\pgfqpoint{2.151053in}{5.146703in}}%
\pgfpathlineto{\pgfqpoint{2.151068in}{5.146702in}}%
\pgfusepath{stroke}%
\end{pgfscope}%
\begin{pgfscope}%
\pgfpathrectangle{\pgfqpoint{1.250000in}{4.155455in}}{\pgfqpoint{2.279412in}{2.004545in}}%
\pgfusepath{clip}%
\pgfsetbuttcap%
\pgfsetroundjoin%
\pgfsetlinewidth{0.386081pt}%
\definecolor{currentstroke}{rgb}{0.280267,0.073417,0.397163}%
\pgfsetstrokecolor{currentstroke}%
\pgfsetdash{}{0pt}%
\pgfpathmoveto{\pgfqpoint{2.151068in}{5.146702in}}%
\pgfpathlineto{\pgfqpoint{2.151072in}{5.146702in}}%
\pgfusepath{stroke}%
\end{pgfscope}%
\begin{pgfscope}%
\pgfpathrectangle{\pgfqpoint{1.250000in}{4.155455in}}{\pgfqpoint{2.279412in}{2.004545in}}%
\pgfusepath{clip}%
\pgfsetbuttcap%
\pgfsetroundjoin%
\pgfsetlinewidth{0.386082pt}%
\definecolor{currentstroke}{rgb}{0.280267,0.073417,0.397163}%
\pgfsetstrokecolor{currentstroke}%
\pgfsetdash{}{0pt}%
\pgfpathmoveto{\pgfqpoint{2.151072in}{5.146702in}}%
\pgfpathlineto{\pgfqpoint{2.151061in}{5.146703in}}%
\pgfusepath{stroke}%
\end{pgfscope}%
\begin{pgfscope}%
\pgfpathrectangle{\pgfqpoint{1.250000in}{4.155455in}}{\pgfqpoint{2.279412in}{2.004545in}}%
\pgfusepath{clip}%
\pgfsetbuttcap%
\pgfsetroundjoin%
\pgfsetlinewidth{0.386081pt}%
\definecolor{currentstroke}{rgb}{0.280267,0.073417,0.397163}%
\pgfsetstrokecolor{currentstroke}%
\pgfsetdash{}{0pt}%
\pgfpathmoveto{\pgfqpoint{2.151061in}{5.146703in}}%
\pgfpathlineto{\pgfqpoint{2.151050in}{5.146703in}}%
\pgfusepath{stroke}%
\end{pgfscope}%
\begin{pgfscope}%
\pgfpathrectangle{\pgfqpoint{1.250000in}{4.155455in}}{\pgfqpoint{2.279412in}{2.004545in}}%
\pgfusepath{clip}%
\pgfsetbuttcap%
\pgfsetroundjoin%
\pgfsetlinewidth{0.386079pt}%
\definecolor{currentstroke}{rgb}{0.280267,0.073417,0.397163}%
\pgfsetstrokecolor{currentstroke}%
\pgfsetdash{}{0pt}%
\pgfpathmoveto{\pgfqpoint{2.151050in}{5.146703in}}%
\pgfpathlineto{\pgfqpoint{2.151049in}{5.146703in}}%
\pgfusepath{stroke}%
\end{pgfscope}%
\begin{pgfscope}%
\pgfpathrectangle{\pgfqpoint{1.250000in}{4.155455in}}{\pgfqpoint{2.279412in}{2.004545in}}%
\pgfusepath{clip}%
\pgfsetbuttcap%
\pgfsetroundjoin%
\pgfsetlinewidth{0.386079pt}%
\definecolor{currentstroke}{rgb}{0.280267,0.073417,0.397163}%
\pgfsetstrokecolor{currentstroke}%
\pgfsetdash{}{0pt}%
\pgfpathmoveto{\pgfqpoint{2.151049in}{5.146703in}}%
\pgfpathlineto{\pgfqpoint{2.151058in}{5.146703in}}%
\pgfusepath{stroke}%
\end{pgfscope}%
\begin{pgfscope}%
\pgfpathrectangle{\pgfqpoint{1.250000in}{4.155455in}}{\pgfqpoint{2.279412in}{2.004545in}}%
\pgfusepath{clip}%
\pgfsetbuttcap%
\pgfsetroundjoin%
\pgfsetlinewidth{0.386080pt}%
\definecolor{currentstroke}{rgb}{0.280267,0.073417,0.397163}%
\pgfsetstrokecolor{currentstroke}%
\pgfsetdash{}{0pt}%
\pgfpathmoveto{\pgfqpoint{2.151058in}{5.146703in}}%
\pgfpathlineto{\pgfqpoint{2.151067in}{5.146702in}}%
\pgfusepath{stroke}%
\end{pgfscope}%
\begin{pgfscope}%
\pgfpathrectangle{\pgfqpoint{1.250000in}{4.155455in}}{\pgfqpoint{2.279412in}{2.004545in}}%
\pgfusepath{clip}%
\pgfsetbuttcap%
\pgfsetroundjoin%
\pgfsetlinewidth{0.386081pt}%
\definecolor{currentstroke}{rgb}{0.280267,0.073417,0.397163}%
\pgfsetstrokecolor{currentstroke}%
\pgfsetdash{}{0pt}%
\pgfpathmoveto{\pgfqpoint{2.151067in}{5.146702in}}%
\pgfpathlineto{\pgfqpoint{2.151067in}{5.146702in}}%
\pgfusepath{stroke}%
\end{pgfscope}%
\begin{pgfscope}%
\pgfpathrectangle{\pgfqpoint{1.250000in}{4.155455in}}{\pgfqpoint{2.279412in}{2.004545in}}%
\pgfusepath{clip}%
\pgfsetbuttcap%
\pgfsetroundjoin%
\pgfsetlinewidth{0.386081pt}%
\definecolor{currentstroke}{rgb}{0.280267,0.073417,0.397163}%
\pgfsetstrokecolor{currentstroke}%
\pgfsetdash{}{0pt}%
\pgfpathmoveto{\pgfqpoint{2.151067in}{5.146702in}}%
\pgfpathlineto{\pgfqpoint{2.151058in}{5.146703in}}%
\pgfusepath{stroke}%
\end{pgfscope}%
\begin{pgfscope}%
\pgfpathrectangle{\pgfqpoint{1.250000in}{4.155455in}}{\pgfqpoint{2.279412in}{2.004545in}}%
\pgfusepath{clip}%
\pgfsetbuttcap%
\pgfsetroundjoin%
\pgfsetlinewidth{0.386080pt}%
\definecolor{currentstroke}{rgb}{0.280267,0.073417,0.397163}%
\pgfsetstrokecolor{currentstroke}%
\pgfsetdash{}{0pt}%
\pgfpathmoveto{\pgfqpoint{2.151058in}{5.146703in}}%
\pgfpathlineto{\pgfqpoint{2.151051in}{5.146703in}}%
\pgfusepath{stroke}%
\end{pgfscope}%
\begin{pgfscope}%
\pgfpathrectangle{\pgfqpoint{1.250000in}{4.155455in}}{\pgfqpoint{2.279412in}{2.004545in}}%
\pgfusepath{clip}%
\pgfsetbuttcap%
\pgfsetroundjoin%
\pgfsetlinewidth{0.386079pt}%
\definecolor{currentstroke}{rgb}{0.280267,0.073417,0.397163}%
\pgfsetstrokecolor{currentstroke}%
\pgfsetdash{}{0pt}%
\pgfpathmoveto{\pgfqpoint{2.151051in}{5.146703in}}%
\pgfpathlineto{\pgfqpoint{2.151053in}{5.146703in}}%
\pgfusepath{stroke}%
\end{pgfscope}%
\begin{pgfscope}%
\pgfpathrectangle{\pgfqpoint{1.250000in}{4.155455in}}{\pgfqpoint{2.279412in}{2.004545in}}%
\pgfusepath{clip}%
\pgfsetbuttcap%
\pgfsetroundjoin%
\pgfsetlinewidth{0.386079pt}%
\definecolor{currentstroke}{rgb}{0.280267,0.073417,0.397163}%
\pgfsetstrokecolor{currentstroke}%
\pgfsetdash{}{0pt}%
\pgfpathmoveto{\pgfqpoint{2.151053in}{5.146703in}}%
\pgfpathlineto{\pgfqpoint{2.151060in}{5.146703in}}%
\pgfusepath{stroke}%
\end{pgfscope}%
\begin{pgfscope}%
\pgfpathrectangle{\pgfqpoint{1.250000in}{4.155455in}}{\pgfqpoint{2.279412in}{2.004545in}}%
\pgfusepath{clip}%
\pgfsetbuttcap%
\pgfsetroundjoin%
\pgfsetlinewidth{0.386080pt}%
\definecolor{currentstroke}{rgb}{0.280267,0.073417,0.397163}%
\pgfsetstrokecolor{currentstroke}%
\pgfsetdash{}{0pt}%
\pgfpathmoveto{\pgfqpoint{2.151060in}{5.146703in}}%
\pgfpathlineto{\pgfqpoint{2.151065in}{5.146702in}}%
\pgfusepath{stroke}%
\end{pgfscope}%
\begin{pgfscope}%
\pgfpathrectangle{\pgfqpoint{1.250000in}{4.155455in}}{\pgfqpoint{2.279412in}{2.004545in}}%
\pgfusepath{clip}%
\pgfsetbuttcap%
\pgfsetroundjoin%
\pgfsetlinewidth{0.386081pt}%
\definecolor{currentstroke}{rgb}{0.280267,0.073417,0.397163}%
\pgfsetstrokecolor{currentstroke}%
\pgfsetdash{}{0pt}%
\pgfpathmoveto{\pgfqpoint{2.151065in}{5.146702in}}%
\pgfpathlineto{\pgfqpoint{2.151063in}{5.146703in}}%
\pgfusepath{stroke}%
\end{pgfscope}%
\begin{pgfscope}%
\pgfpathrectangle{\pgfqpoint{1.250000in}{4.155455in}}{\pgfqpoint{2.279412in}{2.004545in}}%
\pgfusepath{clip}%
\pgfsetbuttcap%
\pgfsetroundjoin%
\pgfsetlinewidth{0.386081pt}%
\definecolor{currentstroke}{rgb}{0.280267,0.073417,0.397163}%
\pgfsetstrokecolor{currentstroke}%
\pgfsetdash{}{0pt}%
\pgfpathmoveto{\pgfqpoint{2.151063in}{5.146703in}}%
\pgfpathlineto{\pgfqpoint{2.151057in}{5.146703in}}%
\pgfusepath{stroke}%
\end{pgfscope}%
\begin{pgfscope}%
\pgfpathrectangle{\pgfqpoint{1.250000in}{4.155455in}}{\pgfqpoint{2.279412in}{2.004545in}}%
\pgfusepath{clip}%
\pgfsetbuttcap%
\pgfsetroundjoin%
\pgfsetlinewidth{0.386080pt}%
\definecolor{currentstroke}{rgb}{0.280267,0.073417,0.397163}%
\pgfsetstrokecolor{currentstroke}%
\pgfsetdash{}{0pt}%
\pgfpathmoveto{\pgfqpoint{2.151057in}{5.146703in}}%
\pgfpathlineto{\pgfqpoint{2.151053in}{5.146703in}}%
\pgfusepath{stroke}%
\end{pgfscope}%
\begin{pgfscope}%
\pgfpathrectangle{\pgfqpoint{1.250000in}{4.155455in}}{\pgfqpoint{2.279412in}{2.004545in}}%
\pgfusepath{clip}%
\pgfsetbuttcap%
\pgfsetroundjoin%
\pgfsetlinewidth{0.386080pt}%
\definecolor{currentstroke}{rgb}{0.280267,0.073417,0.397163}%
\pgfsetstrokecolor{currentstroke}%
\pgfsetdash{}{0pt}%
\pgfpathmoveto{\pgfqpoint{2.151053in}{5.146703in}}%
\pgfpathlineto{\pgfqpoint{2.151056in}{5.146703in}}%
\pgfusepath{stroke}%
\end{pgfscope}%
\begin{pgfscope}%
\pgfpathrectangle{\pgfqpoint{1.250000in}{4.155455in}}{\pgfqpoint{2.279412in}{2.004545in}}%
\pgfusepath{clip}%
\pgfsetbuttcap%
\pgfsetroundjoin%
\pgfsetlinewidth{0.386080pt}%
\definecolor{currentstroke}{rgb}{0.280267,0.073417,0.397163}%
\pgfsetstrokecolor{currentstroke}%
\pgfsetdash{}{0pt}%
\pgfpathmoveto{\pgfqpoint{2.151056in}{5.146703in}}%
\pgfpathlineto{\pgfqpoint{2.151061in}{5.146703in}}%
\pgfusepath{stroke}%
\end{pgfscope}%
\begin{pgfscope}%
\pgfpathrectangle{\pgfqpoint{1.250000in}{4.155455in}}{\pgfqpoint{2.279412in}{2.004545in}}%
\pgfusepath{clip}%
\pgfsetbuttcap%
\pgfsetroundjoin%
\pgfsetlinewidth{0.386080pt}%
\definecolor{currentstroke}{rgb}{0.280267,0.073417,0.397163}%
\pgfsetstrokecolor{currentstroke}%
\pgfsetdash{}{0pt}%
\pgfpathmoveto{\pgfqpoint{2.151061in}{5.146703in}}%
\pgfpathlineto{\pgfqpoint{2.151063in}{5.146703in}}%
\pgfusepath{stroke}%
\end{pgfscope}%
\begin{pgfscope}%
\pgfpathrectangle{\pgfqpoint{1.250000in}{4.155455in}}{\pgfqpoint{2.279412in}{2.004545in}}%
\pgfusepath{clip}%
\pgfsetbuttcap%
\pgfsetroundjoin%
\pgfsetlinewidth{0.386081pt}%
\definecolor{currentstroke}{rgb}{0.280267,0.073417,0.397163}%
\pgfsetstrokecolor{currentstroke}%
\pgfsetdash{}{0pt}%
\pgfpathmoveto{\pgfqpoint{2.151063in}{5.146703in}}%
\pgfpathlineto{\pgfqpoint{2.151061in}{5.146703in}}%
\pgfusepath{stroke}%
\end{pgfscope}%
\begin{pgfscope}%
\pgfpathrectangle{\pgfqpoint{1.250000in}{4.155455in}}{\pgfqpoint{2.279412in}{2.004545in}}%
\pgfusepath{clip}%
\pgfsetbuttcap%
\pgfsetroundjoin%
\pgfsetlinewidth{0.386080pt}%
\definecolor{currentstroke}{rgb}{0.280267,0.073417,0.397163}%
\pgfsetstrokecolor{currentstroke}%
\pgfsetdash{}{0pt}%
\pgfpathmoveto{\pgfqpoint{2.151061in}{5.146703in}}%
\pgfpathlineto{\pgfqpoint{2.151056in}{5.146703in}}%
\pgfusepath{stroke}%
\end{pgfscope}%
\begin{pgfscope}%
\pgfpathrectangle{\pgfqpoint{1.250000in}{4.155455in}}{\pgfqpoint{2.279412in}{2.004545in}}%
\pgfusepath{clip}%
\pgfsetbuttcap%
\pgfsetroundjoin%
\pgfsetlinewidth{0.386080pt}%
\definecolor{currentstroke}{rgb}{0.280267,0.073417,0.397163}%
\pgfsetstrokecolor{currentstroke}%
\pgfsetdash{}{0pt}%
\pgfpathmoveto{\pgfqpoint{2.151056in}{5.146703in}}%
\pgfpathlineto{\pgfqpoint{2.151055in}{5.146703in}}%
\pgfusepath{stroke}%
\end{pgfscope}%
\begin{pgfscope}%
\pgfpathrectangle{\pgfqpoint{1.250000in}{4.155455in}}{\pgfqpoint{2.279412in}{2.004545in}}%
\pgfusepath{clip}%
\pgfsetbuttcap%
\pgfsetroundjoin%
\pgfsetlinewidth{0.386080pt}%
\definecolor{currentstroke}{rgb}{0.280267,0.073417,0.397163}%
\pgfsetstrokecolor{currentstroke}%
\pgfsetdash{}{0pt}%
\pgfpathmoveto{\pgfqpoint{2.151055in}{5.146703in}}%
\pgfpathlineto{\pgfqpoint{2.151058in}{5.146703in}}%
\pgfusepath{stroke}%
\end{pgfscope}%
\begin{pgfscope}%
\pgfpathrectangle{\pgfqpoint{1.250000in}{4.155455in}}{\pgfqpoint{2.279412in}{2.004545in}}%
\pgfusepath{clip}%
\pgfsetbuttcap%
\pgfsetroundjoin%
\pgfsetlinewidth{0.386080pt}%
\definecolor{currentstroke}{rgb}{0.280267,0.073417,0.397163}%
\pgfsetstrokecolor{currentstroke}%
\pgfsetdash{}{0pt}%
\pgfpathmoveto{\pgfqpoint{2.151058in}{5.146703in}}%
\pgfpathlineto{\pgfqpoint{2.151061in}{5.146703in}}%
\pgfusepath{stroke}%
\end{pgfscope}%
\begin{pgfscope}%
\pgfpathrectangle{\pgfqpoint{1.250000in}{4.155455in}}{\pgfqpoint{2.279412in}{2.004545in}}%
\pgfusepath{clip}%
\pgfsetbuttcap%
\pgfsetroundjoin%
\pgfsetlinewidth{0.386080pt}%
\definecolor{currentstroke}{rgb}{0.280267,0.073417,0.397163}%
\pgfsetstrokecolor{currentstroke}%
\pgfsetdash{}{0pt}%
\pgfpathmoveto{\pgfqpoint{2.151061in}{5.146703in}}%
\pgfpathlineto{\pgfqpoint{2.151062in}{5.146703in}}%
\pgfusepath{stroke}%
\end{pgfscope}%
\begin{pgfscope}%
\pgfpathrectangle{\pgfqpoint{1.250000in}{4.155455in}}{\pgfqpoint{2.279412in}{2.004545in}}%
\pgfusepath{clip}%
\pgfsetbuttcap%
\pgfsetroundjoin%
\pgfsetlinewidth{0.386081pt}%
\definecolor{currentstroke}{rgb}{0.280267,0.073417,0.397163}%
\pgfsetstrokecolor{currentstroke}%
\pgfsetdash{}{0pt}%
\pgfpathmoveto{\pgfqpoint{2.151062in}{5.146703in}}%
\pgfpathlineto{\pgfqpoint{2.151059in}{5.146703in}}%
\pgfusepath{stroke}%
\end{pgfscope}%
\begin{pgfscope}%
\pgfpathrectangle{\pgfqpoint{1.250000in}{4.155455in}}{\pgfqpoint{2.279412in}{2.004545in}}%
\pgfusepath{clip}%
\pgfsetbuttcap%
\pgfsetroundjoin%
\pgfsetlinewidth{0.386080pt}%
\definecolor{currentstroke}{rgb}{0.280267,0.073417,0.397163}%
\pgfsetstrokecolor{currentstroke}%
\pgfsetdash{}{0pt}%
\pgfpathmoveto{\pgfqpoint{2.151059in}{5.146703in}}%
\pgfpathlineto{\pgfqpoint{2.151057in}{5.146703in}}%
\pgfusepath{stroke}%
\end{pgfscope}%
\begin{pgfscope}%
\pgfpathrectangle{\pgfqpoint{1.250000in}{4.155455in}}{\pgfqpoint{2.279412in}{2.004545in}}%
\pgfusepath{clip}%
\pgfsetbuttcap%
\pgfsetroundjoin%
\pgfsetlinewidth{0.386080pt}%
\definecolor{currentstroke}{rgb}{0.280267,0.073417,0.397163}%
\pgfsetstrokecolor{currentstroke}%
\pgfsetdash{}{0pt}%
\pgfpathmoveto{\pgfqpoint{2.151057in}{5.146703in}}%
\pgfpathlineto{\pgfqpoint{2.151057in}{5.146703in}}%
\pgfusepath{stroke}%
\end{pgfscope}%
\begin{pgfscope}%
\pgfpathrectangle{\pgfqpoint{1.250000in}{4.155455in}}{\pgfqpoint{2.279412in}{2.004545in}}%
\pgfusepath{clip}%
\pgfsetbuttcap%
\pgfsetroundjoin%
\pgfsetlinewidth{0.386080pt}%
\definecolor{currentstroke}{rgb}{0.280267,0.073417,0.397163}%
\pgfsetstrokecolor{currentstroke}%
\pgfsetdash{}{0pt}%
\pgfpathmoveto{\pgfqpoint{2.151057in}{5.146703in}}%
\pgfpathlineto{\pgfqpoint{2.151059in}{5.146703in}}%
\pgfusepath{stroke}%
\end{pgfscope}%
\begin{pgfscope}%
\pgfpathrectangle{\pgfqpoint{1.250000in}{4.155455in}}{\pgfqpoint{2.279412in}{2.004545in}}%
\pgfusepath{clip}%
\pgfsetbuttcap%
\pgfsetroundjoin%
\pgfsetlinewidth{0.386080pt}%
\definecolor{currentstroke}{rgb}{0.280267,0.073417,0.397163}%
\pgfsetstrokecolor{currentstroke}%
\pgfsetdash{}{0pt}%
\pgfpathmoveto{\pgfqpoint{2.151059in}{5.146703in}}%
\pgfpathlineto{\pgfqpoint{2.151061in}{5.146703in}}%
\pgfusepath{stroke}%
\end{pgfscope}%
\begin{pgfscope}%
\pgfpathrectangle{\pgfqpoint{1.250000in}{4.155455in}}{\pgfqpoint{2.279412in}{2.004545in}}%
\pgfusepath{clip}%
\pgfsetbuttcap%
\pgfsetroundjoin%
\pgfsetlinewidth{0.386080pt}%
\definecolor{currentstroke}{rgb}{0.280267,0.073417,0.397163}%
\pgfsetstrokecolor{currentstroke}%
\pgfsetdash{}{0pt}%
\pgfpathmoveto{\pgfqpoint{2.151061in}{5.146703in}}%
\pgfpathlineto{\pgfqpoint{2.151061in}{5.146703in}}%
\pgfusepath{stroke}%
\end{pgfscope}%
\begin{pgfscope}%
\pgfpathrectangle{\pgfqpoint{1.250000in}{4.155455in}}{\pgfqpoint{2.279412in}{2.004545in}}%
\pgfusepath{clip}%
\pgfsetbuttcap%
\pgfsetroundjoin%
\pgfsetlinewidth{0.386080pt}%
\definecolor{currentstroke}{rgb}{0.280267,0.073417,0.397163}%
\pgfsetstrokecolor{currentstroke}%
\pgfsetdash{}{0pt}%
\pgfpathmoveto{\pgfqpoint{2.151061in}{5.146703in}}%
\pgfpathlineto{\pgfqpoint{2.151059in}{5.146703in}}%
\pgfusepath{stroke}%
\end{pgfscope}%
\begin{pgfscope}%
\pgfpathrectangle{\pgfqpoint{1.250000in}{4.155455in}}{\pgfqpoint{2.279412in}{2.004545in}}%
\pgfusepath{clip}%
\pgfsetbuttcap%
\pgfsetroundjoin%
\pgfsetlinewidth{0.386080pt}%
\definecolor{currentstroke}{rgb}{0.280267,0.073417,0.397163}%
\pgfsetstrokecolor{currentstroke}%
\pgfsetdash{}{0pt}%
\pgfpathmoveto{\pgfqpoint{2.151059in}{5.146703in}}%
\pgfpathlineto{\pgfqpoint{2.151057in}{5.146703in}}%
\pgfusepath{stroke}%
\end{pgfscope}%
\begin{pgfscope}%
\pgfpathrectangle{\pgfqpoint{1.250000in}{4.155455in}}{\pgfqpoint{2.279412in}{2.004545in}}%
\pgfusepath{clip}%
\pgfsetbuttcap%
\pgfsetroundjoin%
\pgfsetlinewidth{0.386080pt}%
\definecolor{currentstroke}{rgb}{0.280267,0.073417,0.397163}%
\pgfsetstrokecolor{currentstroke}%
\pgfsetdash{}{0pt}%
\pgfpathmoveto{\pgfqpoint{2.151057in}{5.146703in}}%
\pgfpathlineto{\pgfqpoint{2.151058in}{5.146703in}}%
\pgfusepath{stroke}%
\end{pgfscope}%
\begin{pgfscope}%
\pgfpathrectangle{\pgfqpoint{1.250000in}{4.155455in}}{\pgfqpoint{2.279412in}{2.004545in}}%
\pgfusepath{clip}%
\pgfsetbuttcap%
\pgfsetroundjoin%
\pgfsetlinewidth{0.386080pt}%
\definecolor{currentstroke}{rgb}{0.280267,0.073417,0.397163}%
\pgfsetstrokecolor{currentstroke}%
\pgfsetdash{}{0pt}%
\pgfpathmoveto{\pgfqpoint{2.151058in}{5.146703in}}%
\pgfpathlineto{\pgfqpoint{2.151059in}{5.146703in}}%
\pgfusepath{stroke}%
\end{pgfscope}%
\begin{pgfscope}%
\pgfpathrectangle{\pgfqpoint{1.250000in}{4.155455in}}{\pgfqpoint{2.279412in}{2.004545in}}%
\pgfusepath{clip}%
\pgfsetbuttcap%
\pgfsetroundjoin%
\pgfsetlinewidth{0.386080pt}%
\definecolor{currentstroke}{rgb}{0.280267,0.073417,0.397163}%
\pgfsetstrokecolor{currentstroke}%
\pgfsetdash{}{0pt}%
\pgfpathmoveto{\pgfqpoint{2.151059in}{5.146703in}}%
\pgfpathlineto{\pgfqpoint{2.151060in}{5.146703in}}%
\pgfusepath{stroke}%
\end{pgfscope}%
\begin{pgfscope}%
\pgfpathrectangle{\pgfqpoint{1.250000in}{4.155455in}}{\pgfqpoint{2.279412in}{2.004545in}}%
\pgfusepath{clip}%
\pgfsetbuttcap%
\pgfsetroundjoin%
\pgfsetlinewidth{0.386080pt}%
\definecolor{currentstroke}{rgb}{0.280267,0.073417,0.397163}%
\pgfsetstrokecolor{currentstroke}%
\pgfsetdash{}{0pt}%
\pgfpathmoveto{\pgfqpoint{2.151060in}{5.146703in}}%
\pgfpathlineto{\pgfqpoint{2.151060in}{5.146703in}}%
\pgfusepath{stroke}%
\end{pgfscope}%
\begin{pgfscope}%
\pgfpathrectangle{\pgfqpoint{1.250000in}{4.155455in}}{\pgfqpoint{2.279412in}{2.004545in}}%
\pgfusepath{clip}%
\pgfsetbuttcap%
\pgfsetroundjoin%
\pgfsetlinewidth{0.386080pt}%
\definecolor{currentstroke}{rgb}{0.280267,0.073417,0.397163}%
\pgfsetstrokecolor{currentstroke}%
\pgfsetdash{}{0pt}%
\pgfpathmoveto{\pgfqpoint{2.151060in}{5.146703in}}%
\pgfpathlineto{\pgfqpoint{2.151058in}{5.146703in}}%
\pgfusepath{stroke}%
\end{pgfscope}%
\begin{pgfscope}%
\pgfpathrectangle{\pgfqpoint{1.250000in}{4.155455in}}{\pgfqpoint{2.279412in}{2.004545in}}%
\pgfusepath{clip}%
\pgfsetbuttcap%
\pgfsetroundjoin%
\pgfsetlinewidth{0.386080pt}%
\definecolor{currentstroke}{rgb}{0.280267,0.073417,0.397163}%
\pgfsetstrokecolor{currentstroke}%
\pgfsetdash{}{0pt}%
\pgfpathmoveto{\pgfqpoint{2.151058in}{5.146703in}}%
\pgfpathlineto{\pgfqpoint{2.151058in}{5.146703in}}%
\pgfusepath{stroke}%
\end{pgfscope}%
\begin{pgfscope}%
\pgfpathrectangle{\pgfqpoint{1.250000in}{4.155455in}}{\pgfqpoint{2.279412in}{2.004545in}}%
\pgfusepath{clip}%
\pgfsetbuttcap%
\pgfsetroundjoin%
\pgfsetlinewidth{0.386080pt}%
\definecolor{currentstroke}{rgb}{0.280267,0.073417,0.397163}%
\pgfsetstrokecolor{currentstroke}%
\pgfsetdash{}{0pt}%
\pgfpathmoveto{\pgfqpoint{2.151058in}{5.146703in}}%
\pgfpathlineto{\pgfqpoint{2.151058in}{5.146703in}}%
\pgfusepath{stroke}%
\end{pgfscope}%
\begin{pgfscope}%
\pgfpathrectangle{\pgfqpoint{1.250000in}{4.155455in}}{\pgfqpoint{2.279412in}{2.004545in}}%
\pgfusepath{clip}%
\pgfsetbuttcap%
\pgfsetroundjoin%
\pgfsetlinewidth{0.386080pt}%
\definecolor{currentstroke}{rgb}{0.280267,0.073417,0.397163}%
\pgfsetstrokecolor{currentstroke}%
\pgfsetdash{}{0pt}%
\pgfpathmoveto{\pgfqpoint{2.151058in}{5.146703in}}%
\pgfpathlineto{\pgfqpoint{2.151059in}{5.146703in}}%
\pgfusepath{stroke}%
\end{pgfscope}%
\begin{pgfscope}%
\pgfpathrectangle{\pgfqpoint{1.250000in}{4.155455in}}{\pgfqpoint{2.279412in}{2.004545in}}%
\pgfusepath{clip}%
\pgfsetbuttcap%
\pgfsetroundjoin%
\pgfsetlinewidth{0.386080pt}%
\definecolor{currentstroke}{rgb}{0.280267,0.073417,0.397163}%
\pgfsetstrokecolor{currentstroke}%
\pgfsetdash{}{0pt}%
\pgfpathmoveto{\pgfqpoint{2.151059in}{5.146703in}}%
\pgfpathlineto{\pgfqpoint{2.151060in}{5.146703in}}%
\pgfusepath{stroke}%
\end{pgfscope}%
\begin{pgfscope}%
\pgfpathrectangle{\pgfqpoint{1.250000in}{4.155455in}}{\pgfqpoint{2.279412in}{2.004545in}}%
\pgfusepath{clip}%
\pgfsetbuttcap%
\pgfsetroundjoin%
\pgfsetlinewidth{0.386080pt}%
\definecolor{currentstroke}{rgb}{0.280267,0.073417,0.397163}%
\pgfsetstrokecolor{currentstroke}%
\pgfsetdash{}{0pt}%
\pgfpathmoveto{\pgfqpoint{2.151060in}{5.146703in}}%
\pgfpathlineto{\pgfqpoint{2.151059in}{5.146703in}}%
\pgfusepath{stroke}%
\end{pgfscope}%
\begin{pgfscope}%
\pgfpathrectangle{\pgfqpoint{1.250000in}{4.155455in}}{\pgfqpoint{2.279412in}{2.004545in}}%
\pgfusepath{clip}%
\pgfsetbuttcap%
\pgfsetroundjoin%
\pgfsetlinewidth{0.386080pt}%
\definecolor{currentstroke}{rgb}{0.280267,0.073417,0.397163}%
\pgfsetstrokecolor{currentstroke}%
\pgfsetdash{}{0pt}%
\pgfpathmoveto{\pgfqpoint{2.151059in}{5.146703in}}%
\pgfpathlineto{\pgfqpoint{2.151058in}{5.146703in}}%
\pgfusepath{stroke}%
\end{pgfscope}%
\begin{pgfscope}%
\pgfpathrectangle{\pgfqpoint{1.250000in}{4.155455in}}{\pgfqpoint{2.279412in}{2.004545in}}%
\pgfusepath{clip}%
\pgfsetbuttcap%
\pgfsetroundjoin%
\pgfsetlinewidth{0.386080pt}%
\definecolor{currentstroke}{rgb}{0.280267,0.073417,0.397163}%
\pgfsetstrokecolor{currentstroke}%
\pgfsetdash{}{0pt}%
\pgfpathmoveto{\pgfqpoint{2.151058in}{5.146703in}}%
\pgfpathlineto{\pgfqpoint{2.151058in}{5.146703in}}%
\pgfusepath{stroke}%
\end{pgfscope}%
\begin{pgfscope}%
\pgfpathrectangle{\pgfqpoint{1.250000in}{4.155455in}}{\pgfqpoint{2.279412in}{2.004545in}}%
\pgfusepath{clip}%
\pgfsetbuttcap%
\pgfsetroundjoin%
\pgfsetlinewidth{0.386080pt}%
\definecolor{currentstroke}{rgb}{0.280267,0.073417,0.397163}%
\pgfsetstrokecolor{currentstroke}%
\pgfsetdash{}{0pt}%
\pgfpathmoveto{\pgfqpoint{2.151058in}{5.146703in}}%
\pgfpathlineto{\pgfqpoint{2.151059in}{5.146703in}}%
\pgfusepath{stroke}%
\end{pgfscope}%
\begin{pgfscope}%
\pgfpathrectangle{\pgfqpoint{1.250000in}{4.155455in}}{\pgfqpoint{2.279412in}{2.004545in}}%
\pgfusepath{clip}%
\pgfsetbuttcap%
\pgfsetroundjoin%
\pgfsetlinewidth{0.386080pt}%
\definecolor{currentstroke}{rgb}{0.280267,0.073417,0.397163}%
\pgfsetstrokecolor{currentstroke}%
\pgfsetdash{}{0pt}%
\pgfpathmoveto{\pgfqpoint{2.151059in}{5.146703in}}%
\pgfpathlineto{\pgfqpoint{2.151059in}{5.146703in}}%
\pgfusepath{stroke}%
\end{pgfscope}%
\begin{pgfscope}%
\pgfpathrectangle{\pgfqpoint{1.250000in}{4.155455in}}{\pgfqpoint{2.279412in}{2.004545in}}%
\pgfusepath{clip}%
\pgfsetbuttcap%
\pgfsetroundjoin%
\pgfsetlinewidth{0.386080pt}%
\definecolor{currentstroke}{rgb}{0.280267,0.073417,0.397163}%
\pgfsetstrokecolor{currentstroke}%
\pgfsetdash{}{0pt}%
\pgfpathmoveto{\pgfqpoint{2.151059in}{5.146703in}}%
\pgfpathlineto{\pgfqpoint{2.151059in}{5.146703in}}%
\pgfusepath{stroke}%
\end{pgfscope}%
\begin{pgfscope}%
\pgfpathrectangle{\pgfqpoint{1.250000in}{4.155455in}}{\pgfqpoint{2.279412in}{2.004545in}}%
\pgfusepath{clip}%
\pgfsetbuttcap%
\pgfsetroundjoin%
\pgfsetlinewidth{0.386080pt}%
\definecolor{currentstroke}{rgb}{0.280267,0.073417,0.397163}%
\pgfsetstrokecolor{currentstroke}%
\pgfsetdash{}{0pt}%
\pgfpathmoveto{\pgfqpoint{2.151059in}{5.146703in}}%
\pgfpathlineto{\pgfqpoint{2.151059in}{5.146703in}}%
\pgfusepath{stroke}%
\end{pgfscope}%
\begin{pgfscope}%
\pgfpathrectangle{\pgfqpoint{1.250000in}{4.155455in}}{\pgfqpoint{2.279412in}{2.004545in}}%
\pgfusepath{clip}%
\pgfsetbuttcap%
\pgfsetroundjoin%
\pgfsetlinewidth{0.386080pt}%
\definecolor{currentstroke}{rgb}{0.280267,0.073417,0.397163}%
\pgfsetstrokecolor{currentstroke}%
\pgfsetdash{}{0pt}%
\pgfpathmoveto{\pgfqpoint{2.151059in}{5.146703in}}%
\pgfpathlineto{\pgfqpoint{2.151058in}{5.146703in}}%
\pgfusepath{stroke}%
\end{pgfscope}%
\begin{pgfscope}%
\pgfpathrectangle{\pgfqpoint{1.250000in}{4.155455in}}{\pgfqpoint{2.279412in}{2.004545in}}%
\pgfusepath{clip}%
\pgfsetbuttcap%
\pgfsetroundjoin%
\pgfsetlinewidth{0.386080pt}%
\definecolor{currentstroke}{rgb}{0.280267,0.073417,0.397163}%
\pgfsetstrokecolor{currentstroke}%
\pgfsetdash{}{0pt}%
\pgfpathmoveto{\pgfqpoint{2.151058in}{5.146703in}}%
\pgfpathlineto{\pgfqpoint{2.151058in}{5.146703in}}%
\pgfusepath{stroke}%
\end{pgfscope}%
\begin{pgfscope}%
\pgfpathrectangle{\pgfqpoint{1.250000in}{4.155455in}}{\pgfqpoint{2.279412in}{2.004545in}}%
\pgfusepath{clip}%
\pgfsetbuttcap%
\pgfsetroundjoin%
\pgfsetlinewidth{0.386080pt}%
\definecolor{currentstroke}{rgb}{0.280267,0.073417,0.397163}%
\pgfsetstrokecolor{currentstroke}%
\pgfsetdash{}{0pt}%
\pgfpathmoveto{\pgfqpoint{2.151058in}{5.146703in}}%
\pgfpathlineto{\pgfqpoint{2.151059in}{5.146703in}}%
\pgfusepath{stroke}%
\end{pgfscope}%
\begin{pgfscope}%
\pgfpathrectangle{\pgfqpoint{1.250000in}{4.155455in}}{\pgfqpoint{2.279412in}{2.004545in}}%
\pgfusepath{clip}%
\pgfsetbuttcap%
\pgfsetroundjoin%
\pgfsetlinewidth{0.386080pt}%
\definecolor{currentstroke}{rgb}{0.280267,0.073417,0.397163}%
\pgfsetstrokecolor{currentstroke}%
\pgfsetdash{}{0pt}%
\pgfpathmoveto{\pgfqpoint{2.151059in}{5.146703in}}%
\pgfpathlineto{\pgfqpoint{2.151059in}{5.146703in}}%
\pgfusepath{stroke}%
\end{pgfscope}%
\begin{pgfscope}%
\pgfpathrectangle{\pgfqpoint{1.250000in}{4.155455in}}{\pgfqpoint{2.279412in}{2.004545in}}%
\pgfusepath{clip}%
\pgfsetbuttcap%
\pgfsetroundjoin%
\pgfsetlinewidth{0.386080pt}%
\definecolor{currentstroke}{rgb}{0.280267,0.073417,0.397163}%
\pgfsetstrokecolor{currentstroke}%
\pgfsetdash{}{0pt}%
\pgfpathmoveto{\pgfqpoint{2.151059in}{5.146703in}}%
\pgfpathlineto{\pgfqpoint{2.151059in}{5.146703in}}%
\pgfusepath{stroke}%
\end{pgfscope}%
\begin{pgfscope}%
\pgfpathrectangle{\pgfqpoint{1.250000in}{4.155455in}}{\pgfqpoint{2.279412in}{2.004545in}}%
\pgfusepath{clip}%
\pgfsetbuttcap%
\pgfsetroundjoin%
\pgfsetlinewidth{0.386080pt}%
\definecolor{currentstroke}{rgb}{0.280267,0.073417,0.397163}%
\pgfsetstrokecolor{currentstroke}%
\pgfsetdash{}{0pt}%
\pgfpathmoveto{\pgfqpoint{2.151059in}{5.146703in}}%
\pgfpathlineto{\pgfqpoint{2.151059in}{5.146703in}}%
\pgfusepath{stroke}%
\end{pgfscope}%
\begin{pgfscope}%
\pgfpathrectangle{\pgfqpoint{1.250000in}{4.155455in}}{\pgfqpoint{2.279412in}{2.004545in}}%
\pgfusepath{clip}%
\pgfsetbuttcap%
\pgfsetroundjoin%
\pgfsetlinewidth{0.386080pt}%
\definecolor{currentstroke}{rgb}{0.280267,0.073417,0.397163}%
\pgfsetstrokecolor{currentstroke}%
\pgfsetdash{}{0pt}%
\pgfpathmoveto{\pgfqpoint{2.151059in}{5.146703in}}%
\pgfpathlineto{\pgfqpoint{2.151058in}{5.146703in}}%
\pgfusepath{stroke}%
\end{pgfscope}%
\begin{pgfscope}%
\pgfpathrectangle{\pgfqpoint{1.250000in}{4.155455in}}{\pgfqpoint{2.279412in}{2.004545in}}%
\pgfusepath{clip}%
\pgfsetbuttcap%
\pgfsetroundjoin%
\pgfsetlinewidth{0.386080pt}%
\definecolor{currentstroke}{rgb}{0.280267,0.073417,0.397163}%
\pgfsetstrokecolor{currentstroke}%
\pgfsetdash{}{0pt}%
\pgfpathmoveto{\pgfqpoint{2.151058in}{5.146703in}}%
\pgfpathlineto{\pgfqpoint{2.151059in}{5.146703in}}%
\pgfusepath{stroke}%
\end{pgfscope}%
\begin{pgfscope}%
\pgfpathrectangle{\pgfqpoint{1.250000in}{4.155455in}}{\pgfqpoint{2.279412in}{2.004545in}}%
\pgfusepath{clip}%
\pgfsetbuttcap%
\pgfsetroundjoin%
\pgfsetlinewidth{0.386080pt}%
\definecolor{currentstroke}{rgb}{0.280267,0.073417,0.397163}%
\pgfsetstrokecolor{currentstroke}%
\pgfsetdash{}{0pt}%
\pgfpathmoveto{\pgfqpoint{2.151059in}{5.146703in}}%
\pgfpathlineto{\pgfqpoint{2.151059in}{5.146703in}}%
\pgfusepath{stroke}%
\end{pgfscope}%
\begin{pgfscope}%
\pgfpathrectangle{\pgfqpoint{1.250000in}{4.155455in}}{\pgfqpoint{2.279412in}{2.004545in}}%
\pgfusepath{clip}%
\pgfsetbuttcap%
\pgfsetroundjoin%
\pgfsetlinewidth{0.386080pt}%
\definecolor{currentstroke}{rgb}{0.280267,0.073417,0.397163}%
\pgfsetstrokecolor{currentstroke}%
\pgfsetdash{}{0pt}%
\pgfpathmoveto{\pgfqpoint{2.151059in}{5.146703in}}%
\pgfpathlineto{\pgfqpoint{2.151059in}{5.146703in}}%
\pgfusepath{stroke}%
\end{pgfscope}%
\begin{pgfscope}%
\pgfpathrectangle{\pgfqpoint{1.250000in}{4.155455in}}{\pgfqpoint{2.279412in}{2.004545in}}%
\pgfusepath{clip}%
\pgfsetbuttcap%
\pgfsetroundjoin%
\pgfsetlinewidth{0.386080pt}%
\definecolor{currentstroke}{rgb}{0.280267,0.073417,0.397163}%
\pgfsetstrokecolor{currentstroke}%
\pgfsetdash{}{0pt}%
\pgfpathmoveto{\pgfqpoint{2.151059in}{5.146703in}}%
\pgfpathlineto{\pgfqpoint{2.151059in}{5.146703in}}%
\pgfusepath{stroke}%
\end{pgfscope}%
\begin{pgfscope}%
\pgfpathrectangle{\pgfqpoint{1.250000in}{4.155455in}}{\pgfqpoint{2.279412in}{2.004545in}}%
\pgfusepath{clip}%
\pgfsetbuttcap%
\pgfsetroundjoin%
\pgfsetlinewidth{0.386080pt}%
\definecolor{currentstroke}{rgb}{0.280267,0.073417,0.397163}%
\pgfsetstrokecolor{currentstroke}%
\pgfsetdash{}{0pt}%
\pgfpathmoveto{\pgfqpoint{2.151059in}{5.146703in}}%
\pgfpathlineto{\pgfqpoint{2.151059in}{5.146703in}}%
\pgfusepath{stroke}%
\end{pgfscope}%
\begin{pgfscope}%
\pgfpathrectangle{\pgfqpoint{1.250000in}{4.155455in}}{\pgfqpoint{2.279412in}{2.004545in}}%
\pgfusepath{clip}%
\pgfsetbuttcap%
\pgfsetroundjoin%
\pgfsetlinewidth{0.386080pt}%
\definecolor{currentstroke}{rgb}{0.280267,0.073417,0.397163}%
\pgfsetstrokecolor{currentstroke}%
\pgfsetdash{}{0pt}%
\pgfpathmoveto{\pgfqpoint{2.151059in}{5.146703in}}%
\pgfpathlineto{\pgfqpoint{2.151059in}{5.146703in}}%
\pgfusepath{stroke}%
\end{pgfscope}%
\begin{pgfscope}%
\pgfpathrectangle{\pgfqpoint{1.250000in}{4.155455in}}{\pgfqpoint{2.279412in}{2.004545in}}%
\pgfusepath{clip}%
\pgfsetbuttcap%
\pgfsetroundjoin%
\pgfsetlinewidth{0.386080pt}%
\definecolor{currentstroke}{rgb}{0.280267,0.073417,0.397163}%
\pgfsetstrokecolor{currentstroke}%
\pgfsetdash{}{0pt}%
\pgfpathmoveto{\pgfqpoint{2.151059in}{5.146703in}}%
\pgfpathlineto{\pgfqpoint{2.151059in}{5.146703in}}%
\pgfusepath{stroke}%
\end{pgfscope}%
\begin{pgfscope}%
\pgfpathrectangle{\pgfqpoint{1.250000in}{4.155455in}}{\pgfqpoint{2.279412in}{2.004545in}}%
\pgfusepath{clip}%
\pgfsetbuttcap%
\pgfsetroundjoin%
\pgfsetlinewidth{0.386080pt}%
\definecolor{currentstroke}{rgb}{0.280267,0.073417,0.397163}%
\pgfsetstrokecolor{currentstroke}%
\pgfsetdash{}{0pt}%
\pgfpathmoveto{\pgfqpoint{2.151059in}{5.146703in}}%
\pgfpathlineto{\pgfqpoint{2.151059in}{5.146703in}}%
\pgfusepath{stroke}%
\end{pgfscope}%
\begin{pgfscope}%
\pgfpathrectangle{\pgfqpoint{1.250000in}{4.155455in}}{\pgfqpoint{2.279412in}{2.004545in}}%
\pgfusepath{clip}%
\pgfsetbuttcap%
\pgfsetroundjoin%
\pgfsetlinewidth{0.386080pt}%
\definecolor{currentstroke}{rgb}{0.280267,0.073417,0.397163}%
\pgfsetstrokecolor{currentstroke}%
\pgfsetdash{}{0pt}%
\pgfpathmoveto{\pgfqpoint{2.151059in}{5.146703in}}%
\pgfpathlineto{\pgfqpoint{2.151059in}{5.146703in}}%
\pgfusepath{stroke}%
\end{pgfscope}%
\begin{pgfscope}%
\pgfpathrectangle{\pgfqpoint{1.250000in}{4.155455in}}{\pgfqpoint{2.279412in}{2.004545in}}%
\pgfusepath{clip}%
\pgfsetbuttcap%
\pgfsetroundjoin%
\pgfsetlinewidth{0.386080pt}%
\definecolor{currentstroke}{rgb}{0.280267,0.073417,0.397163}%
\pgfsetstrokecolor{currentstroke}%
\pgfsetdash{}{0pt}%
\pgfpathmoveto{\pgfqpoint{2.151059in}{5.146703in}}%
\pgfpathlineto{\pgfqpoint{2.151059in}{5.146703in}}%
\pgfusepath{stroke}%
\end{pgfscope}%
\begin{pgfscope}%
\pgfpathrectangle{\pgfqpoint{1.250000in}{4.155455in}}{\pgfqpoint{2.279412in}{2.004545in}}%
\pgfusepath{clip}%
\pgfsetbuttcap%
\pgfsetroundjoin%
\pgfsetlinewidth{0.386080pt}%
\definecolor{currentstroke}{rgb}{0.280267,0.073417,0.397163}%
\pgfsetstrokecolor{currentstroke}%
\pgfsetdash{}{0pt}%
\pgfpathmoveto{\pgfqpoint{2.151059in}{5.146703in}}%
\pgfpathlineto{\pgfqpoint{2.151059in}{5.146703in}}%
\pgfusepath{stroke}%
\end{pgfscope}%
\begin{pgfscope}%
\pgfpathrectangle{\pgfqpoint{1.250000in}{4.155455in}}{\pgfqpoint{2.279412in}{2.004545in}}%
\pgfusepath{clip}%
\pgfsetbuttcap%
\pgfsetroundjoin%
\pgfsetlinewidth{0.386080pt}%
\definecolor{currentstroke}{rgb}{0.280267,0.073417,0.397163}%
\pgfsetstrokecolor{currentstroke}%
\pgfsetdash{}{0pt}%
\pgfpathmoveto{\pgfqpoint{2.151059in}{5.146703in}}%
\pgfpathlineto{\pgfqpoint{2.151059in}{5.146703in}}%
\pgfusepath{stroke}%
\end{pgfscope}%
\begin{pgfscope}%
\pgfpathrectangle{\pgfqpoint{1.250000in}{4.155455in}}{\pgfqpoint{2.279412in}{2.004545in}}%
\pgfusepath{clip}%
\pgfsetbuttcap%
\pgfsetroundjoin%
\pgfsetlinewidth{0.386080pt}%
\definecolor{currentstroke}{rgb}{0.280267,0.073417,0.397163}%
\pgfsetstrokecolor{currentstroke}%
\pgfsetdash{}{0pt}%
\pgfpathmoveto{\pgfqpoint{2.151059in}{5.146703in}}%
\pgfpathlineto{\pgfqpoint{2.151059in}{5.146703in}}%
\pgfusepath{stroke}%
\end{pgfscope}%
\begin{pgfscope}%
\pgfpathrectangle{\pgfqpoint{1.250000in}{4.155455in}}{\pgfqpoint{2.279412in}{2.004545in}}%
\pgfusepath{clip}%
\pgfsetbuttcap%
\pgfsetroundjoin%
\pgfsetlinewidth{0.386080pt}%
\definecolor{currentstroke}{rgb}{0.280267,0.073417,0.397163}%
\pgfsetstrokecolor{currentstroke}%
\pgfsetdash{}{0pt}%
\pgfpathmoveto{\pgfqpoint{2.151059in}{5.146703in}}%
\pgfpathlineto{\pgfqpoint{2.151059in}{5.146703in}}%
\pgfusepath{stroke}%
\end{pgfscope}%
\begin{pgfscope}%
\pgfpathrectangle{\pgfqpoint{1.250000in}{4.155455in}}{\pgfqpoint{2.279412in}{2.004545in}}%
\pgfusepath{clip}%
\pgfsetbuttcap%
\pgfsetroundjoin%
\pgfsetlinewidth{0.386080pt}%
\definecolor{currentstroke}{rgb}{0.280267,0.073417,0.397163}%
\pgfsetstrokecolor{currentstroke}%
\pgfsetdash{}{0pt}%
\pgfpathmoveto{\pgfqpoint{2.151059in}{5.146703in}}%
\pgfpathlineto{\pgfqpoint{2.151059in}{5.146703in}}%
\pgfusepath{stroke}%
\end{pgfscope}%
\begin{pgfscope}%
\pgfpathrectangle{\pgfqpoint{1.250000in}{4.155455in}}{\pgfqpoint{2.279412in}{2.004545in}}%
\pgfusepath{clip}%
\pgfsetbuttcap%
\pgfsetroundjoin%
\pgfsetlinewidth{0.386080pt}%
\definecolor{currentstroke}{rgb}{0.280267,0.073417,0.397163}%
\pgfsetstrokecolor{currentstroke}%
\pgfsetdash{}{0pt}%
\pgfpathmoveto{\pgfqpoint{2.151059in}{5.146703in}}%
\pgfpathlineto{\pgfqpoint{2.151059in}{5.146703in}}%
\pgfusepath{stroke}%
\end{pgfscope}%
\begin{pgfscope}%
\pgfpathrectangle{\pgfqpoint{1.250000in}{4.155455in}}{\pgfqpoint{2.279412in}{2.004545in}}%
\pgfusepath{clip}%
\pgfsetbuttcap%
\pgfsetroundjoin%
\pgfsetlinewidth{0.386080pt}%
\definecolor{currentstroke}{rgb}{0.280267,0.073417,0.397163}%
\pgfsetstrokecolor{currentstroke}%
\pgfsetdash{}{0pt}%
\pgfpathmoveto{\pgfqpoint{2.151059in}{5.146703in}}%
\pgfpathlineto{\pgfqpoint{2.151059in}{5.146703in}}%
\pgfusepath{stroke}%
\end{pgfscope}%
\begin{pgfscope}%
\pgfpathrectangle{\pgfqpoint{1.250000in}{4.155455in}}{\pgfqpoint{2.279412in}{2.004545in}}%
\pgfusepath{clip}%
\pgfsetbuttcap%
\pgfsetroundjoin%
\pgfsetlinewidth{0.386080pt}%
\definecolor{currentstroke}{rgb}{0.280267,0.073417,0.397163}%
\pgfsetstrokecolor{currentstroke}%
\pgfsetdash{}{0pt}%
\pgfpathmoveto{\pgfqpoint{2.151059in}{5.146703in}}%
\pgfpathlineto{\pgfqpoint{2.151059in}{5.146703in}}%
\pgfusepath{stroke}%
\end{pgfscope}%
\begin{pgfscope}%
\pgfpathrectangle{\pgfqpoint{1.250000in}{4.155455in}}{\pgfqpoint{2.279412in}{2.004545in}}%
\pgfusepath{clip}%
\pgfsetbuttcap%
\pgfsetroundjoin%
\pgfsetlinewidth{0.386080pt}%
\definecolor{currentstroke}{rgb}{0.280267,0.073417,0.397163}%
\pgfsetstrokecolor{currentstroke}%
\pgfsetdash{}{0pt}%
\pgfpathmoveto{\pgfqpoint{2.151059in}{5.146703in}}%
\pgfpathlineto{\pgfqpoint{2.151059in}{5.146703in}}%
\pgfusepath{stroke}%
\end{pgfscope}%
\begin{pgfscope}%
\pgfpathrectangle{\pgfqpoint{1.250000in}{4.155455in}}{\pgfqpoint{2.279412in}{2.004545in}}%
\pgfusepath{clip}%
\pgfsetbuttcap%
\pgfsetroundjoin%
\pgfsetlinewidth{0.386080pt}%
\definecolor{currentstroke}{rgb}{0.280267,0.073417,0.397163}%
\pgfsetstrokecolor{currentstroke}%
\pgfsetdash{}{0pt}%
\pgfpathmoveto{\pgfqpoint{2.151059in}{5.146703in}}%
\pgfpathlineto{\pgfqpoint{2.151059in}{5.146703in}}%
\pgfusepath{stroke}%
\end{pgfscope}%
\begin{pgfscope}%
\pgfpathrectangle{\pgfqpoint{1.250000in}{4.155455in}}{\pgfqpoint{2.279412in}{2.004545in}}%
\pgfusepath{clip}%
\pgfsetbuttcap%
\pgfsetroundjoin%
\pgfsetlinewidth{0.386080pt}%
\definecolor{currentstroke}{rgb}{0.280267,0.073417,0.397163}%
\pgfsetstrokecolor{currentstroke}%
\pgfsetdash{}{0pt}%
\pgfpathmoveto{\pgfqpoint{2.151059in}{5.146703in}}%
\pgfpathlineto{\pgfqpoint{2.151059in}{5.146703in}}%
\pgfusepath{stroke}%
\end{pgfscope}%
\begin{pgfscope}%
\pgfpathrectangle{\pgfqpoint{1.250000in}{4.155455in}}{\pgfqpoint{2.279412in}{2.004545in}}%
\pgfusepath{clip}%
\pgfsetbuttcap%
\pgfsetroundjoin%
\pgfsetlinewidth{0.386080pt}%
\definecolor{currentstroke}{rgb}{0.280267,0.073417,0.397163}%
\pgfsetstrokecolor{currentstroke}%
\pgfsetdash{}{0pt}%
\pgfpathmoveto{\pgfqpoint{2.151059in}{5.146703in}}%
\pgfpathlineto{\pgfqpoint{2.151059in}{5.146703in}}%
\pgfusepath{stroke}%
\end{pgfscope}%
\begin{pgfscope}%
\pgfpathrectangle{\pgfqpoint{1.250000in}{4.155455in}}{\pgfqpoint{2.279412in}{2.004545in}}%
\pgfusepath{clip}%
\pgfsetbuttcap%
\pgfsetroundjoin%
\pgfsetlinewidth{0.386080pt}%
\definecolor{currentstroke}{rgb}{0.280267,0.073417,0.397163}%
\pgfsetstrokecolor{currentstroke}%
\pgfsetdash{}{0pt}%
\pgfpathmoveto{\pgfqpoint{2.151059in}{5.146703in}}%
\pgfpathlineto{\pgfqpoint{2.151059in}{5.146703in}}%
\pgfusepath{stroke}%
\end{pgfscope}%
\begin{pgfscope}%
\pgfpathrectangle{\pgfqpoint{1.250000in}{4.155455in}}{\pgfqpoint{2.279412in}{2.004545in}}%
\pgfusepath{clip}%
\pgfsetbuttcap%
\pgfsetroundjoin%
\pgfsetlinewidth{0.386080pt}%
\definecolor{currentstroke}{rgb}{0.280267,0.073417,0.397163}%
\pgfsetstrokecolor{currentstroke}%
\pgfsetdash{}{0pt}%
\pgfpathmoveto{\pgfqpoint{2.151059in}{5.146703in}}%
\pgfpathlineto{\pgfqpoint{2.151059in}{5.146703in}}%
\pgfusepath{stroke}%
\end{pgfscope}%
\begin{pgfscope}%
\pgfpathrectangle{\pgfqpoint{1.250000in}{4.155455in}}{\pgfqpoint{2.279412in}{2.004545in}}%
\pgfusepath{clip}%
\pgfsetbuttcap%
\pgfsetroundjoin%
\pgfsetlinewidth{0.386080pt}%
\definecolor{currentstroke}{rgb}{0.280267,0.073417,0.397163}%
\pgfsetstrokecolor{currentstroke}%
\pgfsetdash{}{0pt}%
\pgfpathmoveto{\pgfqpoint{2.151059in}{5.146703in}}%
\pgfpathlineto{\pgfqpoint{2.151059in}{5.146703in}}%
\pgfusepath{stroke}%
\end{pgfscope}%
\begin{pgfscope}%
\pgfpathrectangle{\pgfqpoint{1.250000in}{4.155455in}}{\pgfqpoint{2.279412in}{2.004545in}}%
\pgfusepath{clip}%
\pgfsetbuttcap%
\pgfsetroundjoin%
\pgfsetlinewidth{0.386080pt}%
\definecolor{currentstroke}{rgb}{0.280267,0.073417,0.397163}%
\pgfsetstrokecolor{currentstroke}%
\pgfsetdash{}{0pt}%
\pgfpathmoveto{\pgfqpoint{2.151059in}{5.146703in}}%
\pgfpathlineto{\pgfqpoint{2.151059in}{5.146703in}}%
\pgfusepath{stroke}%
\end{pgfscope}%
\begin{pgfscope}%
\pgfpathrectangle{\pgfqpoint{1.250000in}{4.155455in}}{\pgfqpoint{2.279412in}{2.004545in}}%
\pgfusepath{clip}%
\pgfsetbuttcap%
\pgfsetroundjoin%
\pgfsetlinewidth{0.386080pt}%
\definecolor{currentstroke}{rgb}{0.280267,0.073417,0.397163}%
\pgfsetstrokecolor{currentstroke}%
\pgfsetdash{}{0pt}%
\pgfpathmoveto{\pgfqpoint{2.151059in}{5.146703in}}%
\pgfpathlineto{\pgfqpoint{2.151059in}{5.146703in}}%
\pgfusepath{stroke}%
\end{pgfscope}%
\begin{pgfscope}%
\pgfpathrectangle{\pgfqpoint{1.250000in}{4.155455in}}{\pgfqpoint{2.279412in}{2.004545in}}%
\pgfusepath{clip}%
\pgfsetbuttcap%
\pgfsetroundjoin%
\pgfsetlinewidth{0.386080pt}%
\definecolor{currentstroke}{rgb}{0.280267,0.073417,0.397163}%
\pgfsetstrokecolor{currentstroke}%
\pgfsetdash{}{0pt}%
\pgfpathmoveto{\pgfqpoint{2.151059in}{5.146703in}}%
\pgfpathlineto{\pgfqpoint{2.151059in}{5.146703in}}%
\pgfusepath{stroke}%
\end{pgfscope}%
\begin{pgfscope}%
\pgfpathrectangle{\pgfqpoint{1.250000in}{4.155455in}}{\pgfqpoint{2.279412in}{2.004545in}}%
\pgfusepath{clip}%
\pgfsetbuttcap%
\pgfsetroundjoin%
\pgfsetlinewidth{0.386080pt}%
\definecolor{currentstroke}{rgb}{0.280267,0.073417,0.397163}%
\pgfsetstrokecolor{currentstroke}%
\pgfsetdash{}{0pt}%
\pgfpathmoveto{\pgfqpoint{2.151059in}{5.146703in}}%
\pgfpathlineto{\pgfqpoint{2.151059in}{5.146703in}}%
\pgfusepath{stroke}%
\end{pgfscope}%
\begin{pgfscope}%
\pgfpathrectangle{\pgfqpoint{1.250000in}{4.155455in}}{\pgfqpoint{2.279412in}{2.004545in}}%
\pgfusepath{clip}%
\pgfsetbuttcap%
\pgfsetroundjoin%
\pgfsetlinewidth{0.386080pt}%
\definecolor{currentstroke}{rgb}{0.280267,0.073417,0.397163}%
\pgfsetstrokecolor{currentstroke}%
\pgfsetdash{}{0pt}%
\pgfpathmoveto{\pgfqpoint{2.151059in}{5.146703in}}%
\pgfpathlineto{\pgfqpoint{2.151059in}{5.146703in}}%
\pgfusepath{stroke}%
\end{pgfscope}%
\begin{pgfscope}%
\pgfpathrectangle{\pgfqpoint{1.250000in}{4.155455in}}{\pgfqpoint{2.279412in}{2.004545in}}%
\pgfusepath{clip}%
\pgfsetbuttcap%
\pgfsetroundjoin%
\pgfsetlinewidth{0.386080pt}%
\definecolor{currentstroke}{rgb}{0.280267,0.073417,0.397163}%
\pgfsetstrokecolor{currentstroke}%
\pgfsetdash{}{0pt}%
\pgfpathmoveto{\pgfqpoint{2.151059in}{5.146703in}}%
\pgfpathlineto{\pgfqpoint{2.151059in}{5.146703in}}%
\pgfusepath{stroke}%
\end{pgfscope}%
\begin{pgfscope}%
\pgfpathrectangle{\pgfqpoint{1.250000in}{4.155455in}}{\pgfqpoint{2.279412in}{2.004545in}}%
\pgfusepath{clip}%
\pgfsetbuttcap%
\pgfsetroundjoin%
\pgfsetlinewidth{0.386080pt}%
\definecolor{currentstroke}{rgb}{0.280267,0.073417,0.397163}%
\pgfsetstrokecolor{currentstroke}%
\pgfsetdash{}{0pt}%
\pgfpathmoveto{\pgfqpoint{2.151059in}{5.146703in}}%
\pgfpathlineto{\pgfqpoint{2.151059in}{5.146703in}}%
\pgfusepath{stroke}%
\end{pgfscope}%
\begin{pgfscope}%
\pgfpathrectangle{\pgfqpoint{1.250000in}{4.155455in}}{\pgfqpoint{2.279412in}{2.004545in}}%
\pgfusepath{clip}%
\pgfsetbuttcap%
\pgfsetroundjoin%
\pgfsetlinewidth{0.386080pt}%
\definecolor{currentstroke}{rgb}{0.280267,0.073417,0.397163}%
\pgfsetstrokecolor{currentstroke}%
\pgfsetdash{}{0pt}%
\pgfpathmoveto{\pgfqpoint{2.151059in}{5.146703in}}%
\pgfpathlineto{\pgfqpoint{2.151059in}{5.146703in}}%
\pgfusepath{stroke}%
\end{pgfscope}%
\begin{pgfscope}%
\pgfpathrectangle{\pgfqpoint{1.250000in}{4.155455in}}{\pgfqpoint{2.279412in}{2.004545in}}%
\pgfusepath{clip}%
\pgfsetbuttcap%
\pgfsetroundjoin%
\pgfsetlinewidth{0.386080pt}%
\definecolor{currentstroke}{rgb}{0.280267,0.073417,0.397163}%
\pgfsetstrokecolor{currentstroke}%
\pgfsetdash{}{0pt}%
\pgfpathmoveto{\pgfqpoint{2.151059in}{5.146703in}}%
\pgfpathlineto{\pgfqpoint{2.151059in}{5.146703in}}%
\pgfusepath{stroke}%
\end{pgfscope}%
\begin{pgfscope}%
\pgfpathrectangle{\pgfqpoint{1.250000in}{4.155455in}}{\pgfqpoint{2.279412in}{2.004545in}}%
\pgfusepath{clip}%
\pgfsetbuttcap%
\pgfsetroundjoin%
\pgfsetlinewidth{0.386080pt}%
\definecolor{currentstroke}{rgb}{0.280267,0.073417,0.397163}%
\pgfsetstrokecolor{currentstroke}%
\pgfsetdash{}{0pt}%
\pgfpathmoveto{\pgfqpoint{2.151059in}{5.146703in}}%
\pgfpathlineto{\pgfqpoint{2.151059in}{5.146703in}}%
\pgfusepath{stroke}%
\end{pgfscope}%
\begin{pgfscope}%
\pgfpathrectangle{\pgfqpoint{1.250000in}{4.155455in}}{\pgfqpoint{2.279412in}{2.004545in}}%
\pgfusepath{clip}%
\pgfsetbuttcap%
\pgfsetroundjoin%
\pgfsetlinewidth{0.386080pt}%
\definecolor{currentstroke}{rgb}{0.280267,0.073417,0.397163}%
\pgfsetstrokecolor{currentstroke}%
\pgfsetdash{}{0pt}%
\pgfpathmoveto{\pgfqpoint{2.151059in}{5.146703in}}%
\pgfpathlineto{\pgfqpoint{2.151059in}{5.146703in}}%
\pgfusepath{stroke}%
\end{pgfscope}%
\begin{pgfscope}%
\pgfpathrectangle{\pgfqpoint{1.250000in}{4.155455in}}{\pgfqpoint{2.279412in}{2.004545in}}%
\pgfusepath{clip}%
\pgfsetbuttcap%
\pgfsetroundjoin%
\pgfsetlinewidth{0.386080pt}%
\definecolor{currentstroke}{rgb}{0.280267,0.073417,0.397163}%
\pgfsetstrokecolor{currentstroke}%
\pgfsetdash{}{0pt}%
\pgfpathmoveto{\pgfqpoint{2.151059in}{5.146703in}}%
\pgfpathlineto{\pgfqpoint{2.151059in}{5.146703in}}%
\pgfusepath{stroke}%
\end{pgfscope}%
\begin{pgfscope}%
\pgfpathrectangle{\pgfqpoint{1.250000in}{4.155455in}}{\pgfqpoint{2.279412in}{2.004545in}}%
\pgfusepath{clip}%
\pgfsetbuttcap%
\pgfsetroundjoin%
\pgfsetlinewidth{0.386080pt}%
\definecolor{currentstroke}{rgb}{0.280267,0.073417,0.397163}%
\pgfsetstrokecolor{currentstroke}%
\pgfsetdash{}{0pt}%
\pgfpathmoveto{\pgfqpoint{2.151059in}{5.146703in}}%
\pgfpathlineto{\pgfqpoint{2.151059in}{5.146703in}}%
\pgfusepath{stroke}%
\end{pgfscope}%
\begin{pgfscope}%
\pgfpathrectangle{\pgfqpoint{1.250000in}{4.155455in}}{\pgfqpoint{2.279412in}{2.004545in}}%
\pgfusepath{clip}%
\pgfsetbuttcap%
\pgfsetroundjoin%
\pgfsetlinewidth{0.386080pt}%
\definecolor{currentstroke}{rgb}{0.280267,0.073417,0.397163}%
\pgfsetstrokecolor{currentstroke}%
\pgfsetdash{}{0pt}%
\pgfpathmoveto{\pgfqpoint{2.151059in}{5.146703in}}%
\pgfpathlineto{\pgfqpoint{2.151059in}{5.146703in}}%
\pgfusepath{stroke}%
\end{pgfscope}%
\begin{pgfscope}%
\pgfpathrectangle{\pgfqpoint{1.250000in}{4.155455in}}{\pgfqpoint{2.279412in}{2.004545in}}%
\pgfusepath{clip}%
\pgfsetbuttcap%
\pgfsetroundjoin%
\pgfsetlinewidth{0.386080pt}%
\definecolor{currentstroke}{rgb}{0.280267,0.073417,0.397163}%
\pgfsetstrokecolor{currentstroke}%
\pgfsetdash{}{0pt}%
\pgfpathmoveto{\pgfqpoint{2.151059in}{5.146703in}}%
\pgfpathlineto{\pgfqpoint{2.151059in}{5.146703in}}%
\pgfusepath{stroke}%
\end{pgfscope}%
\begin{pgfscope}%
\pgfpathrectangle{\pgfqpoint{1.250000in}{4.155455in}}{\pgfqpoint{2.279412in}{2.004545in}}%
\pgfusepath{clip}%
\pgfsetbuttcap%
\pgfsetroundjoin%
\pgfsetlinewidth{0.386080pt}%
\definecolor{currentstroke}{rgb}{0.280267,0.073417,0.397163}%
\pgfsetstrokecolor{currentstroke}%
\pgfsetdash{}{0pt}%
\pgfpathmoveto{\pgfqpoint{2.151059in}{5.146703in}}%
\pgfpathlineto{\pgfqpoint{2.151059in}{5.146703in}}%
\pgfusepath{stroke}%
\end{pgfscope}%
\begin{pgfscope}%
\pgfpathrectangle{\pgfqpoint{1.250000in}{4.155455in}}{\pgfqpoint{2.279412in}{2.004545in}}%
\pgfusepath{clip}%
\pgfsetbuttcap%
\pgfsetroundjoin%
\pgfsetlinewidth{0.386080pt}%
\definecolor{currentstroke}{rgb}{0.280267,0.073417,0.397163}%
\pgfsetstrokecolor{currentstroke}%
\pgfsetdash{}{0pt}%
\pgfpathmoveto{\pgfqpoint{2.151059in}{5.146703in}}%
\pgfpathlineto{\pgfqpoint{2.151059in}{5.146703in}}%
\pgfusepath{stroke}%
\end{pgfscope}%
\begin{pgfscope}%
\pgfpathrectangle{\pgfqpoint{1.250000in}{4.155455in}}{\pgfqpoint{2.279412in}{2.004545in}}%
\pgfusepath{clip}%
\pgfsetbuttcap%
\pgfsetroundjoin%
\pgfsetlinewidth{0.386080pt}%
\definecolor{currentstroke}{rgb}{0.280267,0.073417,0.397163}%
\pgfsetstrokecolor{currentstroke}%
\pgfsetdash{}{0pt}%
\pgfpathmoveto{\pgfqpoint{2.151059in}{5.146703in}}%
\pgfpathlineto{\pgfqpoint{2.151059in}{5.146703in}}%
\pgfusepath{stroke}%
\end{pgfscope}%
\begin{pgfscope}%
\pgfpathrectangle{\pgfqpoint{1.250000in}{4.155455in}}{\pgfqpoint{2.279412in}{2.004545in}}%
\pgfusepath{clip}%
\pgfsetbuttcap%
\pgfsetroundjoin%
\pgfsetlinewidth{0.386080pt}%
\definecolor{currentstroke}{rgb}{0.280267,0.073417,0.397163}%
\pgfsetstrokecolor{currentstroke}%
\pgfsetdash{}{0pt}%
\pgfpathmoveto{\pgfqpoint{2.151059in}{5.146703in}}%
\pgfpathlineto{\pgfqpoint{2.151059in}{5.146703in}}%
\pgfusepath{stroke}%
\end{pgfscope}%
\begin{pgfscope}%
\pgfpathrectangle{\pgfqpoint{1.250000in}{4.155455in}}{\pgfqpoint{2.279412in}{2.004545in}}%
\pgfusepath{clip}%
\pgfsetbuttcap%
\pgfsetroundjoin%
\pgfsetlinewidth{0.386080pt}%
\definecolor{currentstroke}{rgb}{0.280267,0.073417,0.397163}%
\pgfsetstrokecolor{currentstroke}%
\pgfsetdash{}{0pt}%
\pgfpathmoveto{\pgfqpoint{2.151059in}{5.146703in}}%
\pgfpathlineto{\pgfqpoint{2.151059in}{5.146703in}}%
\pgfusepath{stroke}%
\end{pgfscope}%
\begin{pgfscope}%
\pgfpathrectangle{\pgfqpoint{1.250000in}{4.155455in}}{\pgfqpoint{2.279412in}{2.004545in}}%
\pgfusepath{clip}%
\pgfsetbuttcap%
\pgfsetroundjoin%
\pgfsetlinewidth{0.386080pt}%
\definecolor{currentstroke}{rgb}{0.280267,0.073417,0.397163}%
\pgfsetstrokecolor{currentstroke}%
\pgfsetdash{}{0pt}%
\pgfpathmoveto{\pgfqpoint{2.151059in}{5.146703in}}%
\pgfpathlineto{\pgfqpoint{2.151059in}{5.146703in}}%
\pgfusepath{stroke}%
\end{pgfscope}%
\begin{pgfscope}%
\pgfpathrectangle{\pgfqpoint{1.250000in}{4.155455in}}{\pgfqpoint{2.279412in}{2.004545in}}%
\pgfusepath{clip}%
\pgfsetbuttcap%
\pgfsetroundjoin%
\pgfsetlinewidth{0.386080pt}%
\definecolor{currentstroke}{rgb}{0.280267,0.073417,0.397163}%
\pgfsetstrokecolor{currentstroke}%
\pgfsetdash{}{0pt}%
\pgfpathmoveto{\pgfqpoint{2.151059in}{5.146703in}}%
\pgfpathlineto{\pgfqpoint{2.151059in}{5.146703in}}%
\pgfusepath{stroke}%
\end{pgfscope}%
\begin{pgfscope}%
\pgfpathrectangle{\pgfqpoint{1.250000in}{4.155455in}}{\pgfqpoint{2.279412in}{2.004545in}}%
\pgfusepath{clip}%
\pgfsetbuttcap%
\pgfsetroundjoin%
\pgfsetlinewidth{0.000000pt}%
\definecolor{currentstroke}{rgb}{0.280267,0.073417,0.397163}%
\pgfsetstrokecolor{currentstroke}%
\pgfsetdash{}{0pt}%
\pgfpathmoveto{\pgfqpoint{3.310736in}{5.160505in}}%
\pgfpathlineto{\pgfqpoint{3.261668in}{5.157727in}}%
\pgfusepath{}%
\end{pgfscope}%
\begin{pgfscope}%
\pgfpathrectangle{\pgfqpoint{1.250000in}{4.155455in}}{\pgfqpoint{2.279412in}{2.004545in}}%
\pgfusepath{clip}%
\pgfsetbuttcap%
\pgfsetroundjoin%
\pgfsetlinewidth{0.316510pt}%
\definecolor{currentstroke}{rgb}{0.269944,0.014625,0.341379}%
\pgfsetstrokecolor{currentstroke}%
\pgfsetdash{}{0pt}%
\pgfpathmoveto{\pgfqpoint{3.261668in}{5.157727in}}%
\pgfpathlineto{\pgfqpoint{3.211586in}{5.157822in}}%
\pgfusepath{stroke}%
\end{pgfscope}%
\begin{pgfscope}%
\pgfpathrectangle{\pgfqpoint{1.250000in}{4.155455in}}{\pgfqpoint{2.279412in}{2.004545in}}%
\pgfusepath{clip}%
\pgfsetbuttcap%
\pgfsetroundjoin%
\pgfsetlinewidth{0.324450pt}%
\definecolor{currentstroke}{rgb}{0.271305,0.019942,0.347269}%
\pgfsetstrokecolor{currentstroke}%
\pgfsetdash{}{0pt}%
\pgfpathmoveto{\pgfqpoint{3.211586in}{5.157822in}}%
\pgfpathlineto{\pgfqpoint{3.161448in}{5.156906in}}%
\pgfusepath{stroke}%
\end{pgfscope}%
\begin{pgfscope}%
\pgfpathrectangle{\pgfqpoint{1.250000in}{4.155455in}}{\pgfqpoint{2.279412in}{2.004545in}}%
\pgfusepath{clip}%
\pgfsetbuttcap%
\pgfsetroundjoin%
\pgfsetlinewidth{0.324392pt}%
\definecolor{currentstroke}{rgb}{0.271305,0.019942,0.347269}%
\pgfsetstrokecolor{currentstroke}%
\pgfsetdash{}{0pt}%
\pgfpathmoveto{\pgfqpoint{3.161448in}{5.156906in}}%
\pgfpathlineto{\pgfqpoint{3.111314in}{5.156607in}}%
\pgfusepath{stroke}%
\end{pgfscope}%
\begin{pgfscope}%
\pgfpathrectangle{\pgfqpoint{1.250000in}{4.155455in}}{\pgfqpoint{2.279412in}{2.004545in}}%
\pgfusepath{clip}%
\pgfsetbuttcap%
\pgfsetroundjoin%
\pgfsetlinewidth{0.339024pt}%
\definecolor{currentstroke}{rgb}{0.273809,0.031497,0.358853}%
\pgfsetstrokecolor{currentstroke}%
\pgfsetdash{}{0pt}%
\pgfpathmoveto{\pgfqpoint{3.111314in}{5.156607in}}%
\pgfpathlineto{\pgfqpoint{3.061165in}{5.156906in}}%
\pgfusepath{stroke}%
\end{pgfscope}%
\begin{pgfscope}%
\pgfpathrectangle{\pgfqpoint{1.250000in}{4.155455in}}{\pgfqpoint{2.279412in}{2.004545in}}%
\pgfusepath{clip}%
\pgfsetbuttcap%
\pgfsetroundjoin%
\pgfsetlinewidth{0.355958pt}%
\definecolor{currentstroke}{rgb}{0.276022,0.044167,0.370164}%
\pgfsetstrokecolor{currentstroke}%
\pgfsetdash{}{0pt}%
\pgfpathmoveto{\pgfqpoint{3.061165in}{5.156906in}}%
\pgfpathlineto{\pgfqpoint{3.011016in}{5.156845in}}%
\pgfusepath{stroke}%
\end{pgfscope}%
\begin{pgfscope}%
\pgfpathrectangle{\pgfqpoint{1.250000in}{4.155455in}}{\pgfqpoint{2.279412in}{2.004545in}}%
\pgfusepath{clip}%
\pgfsetbuttcap%
\pgfsetroundjoin%
\pgfsetlinewidth{0.382949pt}%
\definecolor{currentstroke}{rgb}{0.279566,0.067836,0.391917}%
\pgfsetstrokecolor{currentstroke}%
\pgfsetdash{}{0pt}%
\pgfpathmoveto{\pgfqpoint{3.011016in}{5.156845in}}%
\pgfpathlineto{\pgfqpoint{2.960866in}{5.156789in}}%
\pgfusepath{stroke}%
\end{pgfscope}%
\begin{pgfscope}%
\pgfpathrectangle{\pgfqpoint{1.250000in}{4.155455in}}{\pgfqpoint{2.279412in}{2.004545in}}%
\pgfusepath{clip}%
\pgfsetbuttcap%
\pgfsetroundjoin%
\pgfsetlinewidth{0.411358pt}%
\definecolor{currentstroke}{rgb}{0.282327,0.094955,0.417331}%
\pgfsetstrokecolor{currentstroke}%
\pgfsetdash{}{0pt}%
\pgfpathmoveto{\pgfqpoint{2.960866in}{5.156789in}}%
\pgfpathlineto{\pgfqpoint{2.910715in}{5.156918in}}%
\pgfusepath{stroke}%
\end{pgfscope}%
\begin{pgfscope}%
\pgfpathrectangle{\pgfqpoint{1.250000in}{4.155455in}}{\pgfqpoint{2.279412in}{2.004545in}}%
\pgfusepath{clip}%
\pgfsetbuttcap%
\pgfsetroundjoin%
\pgfsetlinewidth{0.458282pt}%
\definecolor{currentstroke}{rgb}{0.283187,0.125848,0.444960}%
\pgfsetstrokecolor{currentstroke}%
\pgfsetdash{}{0pt}%
\pgfpathmoveto{\pgfqpoint{2.910715in}{5.156918in}}%
\pgfpathlineto{\pgfqpoint{2.860564in}{5.157139in}}%
\pgfusepath{stroke}%
\end{pgfscope}%
\begin{pgfscope}%
\pgfpathrectangle{\pgfqpoint{1.250000in}{4.155455in}}{\pgfqpoint{2.279412in}{2.004545in}}%
\pgfusepath{clip}%
\pgfsetbuttcap%
\pgfsetroundjoin%
\pgfsetlinewidth{0.526522pt}%
\definecolor{currentstroke}{rgb}{0.278826,0.175490,0.483397}%
\pgfsetstrokecolor{currentstroke}%
\pgfsetdash{}{0pt}%
\pgfpathmoveto{\pgfqpoint{2.860564in}{5.157139in}}%
\pgfpathlineto{\pgfqpoint{2.810412in}{5.157159in}}%
\pgfusepath{stroke}%
\end{pgfscope}%
\begin{pgfscope}%
\pgfpathrectangle{\pgfqpoint{1.250000in}{4.155455in}}{\pgfqpoint{2.279412in}{2.004545in}}%
\pgfusepath{clip}%
\pgfsetbuttcap%
\pgfsetroundjoin%
\pgfsetlinewidth{0.581152pt}%
\definecolor{currentstroke}{rgb}{0.270595,0.214069,0.507052}%
\pgfsetstrokecolor{currentstroke}%
\pgfsetdash{}{0pt}%
\pgfpathmoveto{\pgfqpoint{2.810412in}{5.157159in}}%
\pgfpathlineto{\pgfqpoint{2.760260in}{5.157111in}}%
\pgfusepath{stroke}%
\end{pgfscope}%
\begin{pgfscope}%
\pgfpathrectangle{\pgfqpoint{1.250000in}{4.155455in}}{\pgfqpoint{2.279412in}{2.004545in}}%
\pgfusepath{clip}%
\pgfsetbuttcap%
\pgfsetroundjoin%
\pgfsetlinewidth{0.657848pt}%
\definecolor{currentstroke}{rgb}{0.253935,0.265254,0.529983}%
\pgfsetstrokecolor{currentstroke}%
\pgfsetdash{}{0pt}%
\pgfpathmoveto{\pgfqpoint{2.760260in}{5.157111in}}%
\pgfpathlineto{\pgfqpoint{2.710108in}{5.157020in}}%
\pgfusepath{stroke}%
\end{pgfscope}%
\begin{pgfscope}%
\pgfpathrectangle{\pgfqpoint{1.250000in}{4.155455in}}{\pgfqpoint{2.279412in}{2.004545in}}%
\pgfusepath{clip}%
\pgfsetbuttcap%
\pgfsetroundjoin%
\pgfsetlinewidth{0.774935pt}%
\definecolor{currentstroke}{rgb}{0.220057,0.343307,0.549413}%
\pgfsetstrokecolor{currentstroke}%
\pgfsetdash{}{0pt}%
\pgfpathmoveto{\pgfqpoint{2.710108in}{5.157020in}}%
\pgfpathlineto{\pgfqpoint{2.659957in}{5.156798in}}%
\pgfusepath{stroke}%
\end{pgfscope}%
\begin{pgfscope}%
\pgfpathrectangle{\pgfqpoint{1.250000in}{4.155455in}}{\pgfqpoint{2.279412in}{2.004545in}}%
\pgfusepath{clip}%
\pgfsetbuttcap%
\pgfsetroundjoin%
\pgfsetlinewidth{0.806102pt}%
\definecolor{currentstroke}{rgb}{0.212395,0.359683,0.551710}%
\pgfsetstrokecolor{currentstroke}%
\pgfsetdash{}{0pt}%
\pgfpathmoveto{\pgfqpoint{2.659957in}{5.156798in}}%
\pgfpathlineto{\pgfqpoint{2.609806in}{5.156687in}}%
\pgfusepath{stroke}%
\end{pgfscope}%
\begin{pgfscope}%
\pgfpathrectangle{\pgfqpoint{1.250000in}{4.155455in}}{\pgfqpoint{2.279412in}{2.004545in}}%
\pgfusepath{clip}%
\pgfsetbuttcap%
\pgfsetroundjoin%
\pgfsetlinewidth{0.917923pt}%
\definecolor{currentstroke}{rgb}{0.183898,0.422383,0.556944}%
\pgfsetstrokecolor{currentstroke}%
\pgfsetdash{}{0pt}%
\pgfpathmoveto{\pgfqpoint{2.609806in}{5.156687in}}%
\pgfpathlineto{\pgfqpoint{2.559654in}{5.156731in}}%
\pgfusepath{stroke}%
\end{pgfscope}%
\begin{pgfscope}%
\pgfpathrectangle{\pgfqpoint{1.250000in}{4.155455in}}{\pgfqpoint{2.279412in}{2.004545in}}%
\pgfusepath{clip}%
\pgfsetbuttcap%
\pgfsetroundjoin%
\pgfsetlinewidth{0.916783pt}%
\definecolor{currentstroke}{rgb}{0.183898,0.422383,0.556944}%
\pgfsetstrokecolor{currentstroke}%
\pgfsetdash{}{0pt}%
\pgfpathmoveto{\pgfqpoint{2.559654in}{5.156731in}}%
\pgfpathlineto{\pgfqpoint{2.509502in}{5.156696in}}%
\pgfusepath{stroke}%
\end{pgfscope}%
\begin{pgfscope}%
\pgfpathrectangle{\pgfqpoint{1.250000in}{4.155455in}}{\pgfqpoint{2.279412in}{2.004545in}}%
\pgfusepath{clip}%
\pgfsetbuttcap%
\pgfsetroundjoin%
\pgfsetlinewidth{0.949927pt}%
\definecolor{currentstroke}{rgb}{0.175841,0.441290,0.557685}%
\pgfsetstrokecolor{currentstroke}%
\pgfsetdash{}{0pt}%
\pgfpathmoveto{\pgfqpoint{2.509502in}{5.156696in}}%
\pgfpathlineto{\pgfqpoint{2.459350in}{5.156643in}}%
\pgfusepath{stroke}%
\end{pgfscope}%
\begin{pgfscope}%
\pgfpathrectangle{\pgfqpoint{1.250000in}{4.155455in}}{\pgfqpoint{2.279412in}{2.004545in}}%
\pgfusepath{clip}%
\pgfsetbuttcap%
\pgfsetroundjoin%
\pgfsetlinewidth{0.956837pt}%
\definecolor{currentstroke}{rgb}{0.174274,0.445044,0.557792}%
\pgfsetstrokecolor{currentstroke}%
\pgfsetdash{}{0pt}%
\pgfpathmoveto{\pgfqpoint{2.459350in}{5.156643in}}%
\pgfpathlineto{\pgfqpoint{2.409199in}{5.156602in}}%
\pgfusepath{stroke}%
\end{pgfscope}%
\begin{pgfscope}%
\pgfpathrectangle{\pgfqpoint{1.250000in}{4.155455in}}{\pgfqpoint{2.279412in}{2.004545in}}%
\pgfusepath{clip}%
\pgfsetbuttcap%
\pgfsetroundjoin%
\pgfsetlinewidth{0.879839pt}%
\definecolor{currentstroke}{rgb}{0.192357,0.403199,0.555836}%
\pgfsetstrokecolor{currentstroke}%
\pgfsetdash{}{0pt}%
\pgfpathmoveto{\pgfqpoint{2.409199in}{5.156602in}}%
\pgfpathlineto{\pgfqpoint{2.359047in}{5.156479in}}%
\pgfusepath{stroke}%
\end{pgfscope}%
\begin{pgfscope}%
\pgfpathrectangle{\pgfqpoint{1.250000in}{4.155455in}}{\pgfqpoint{2.279412in}{2.004545in}}%
\pgfusepath{clip}%
\pgfsetbuttcap%
\pgfsetroundjoin%
\pgfsetlinewidth{0.888691pt}%
\definecolor{currentstroke}{rgb}{0.190631,0.407061,0.556089}%
\pgfsetstrokecolor{currentstroke}%
\pgfsetdash{}{0pt}%
\pgfpathmoveto{\pgfqpoint{2.359047in}{5.156479in}}%
\pgfpathlineto{\pgfqpoint{2.308898in}{5.156155in}}%
\pgfusepath{stroke}%
\end{pgfscope}%
\begin{pgfscope}%
\pgfpathrectangle{\pgfqpoint{1.250000in}{4.155455in}}{\pgfqpoint{2.279412in}{2.004545in}}%
\pgfusepath{clip}%
\pgfsetbuttcap%
\pgfsetroundjoin%
\pgfsetlinewidth{0.751988pt}%
\definecolor{currentstroke}{rgb}{0.227802,0.326594,0.546532}%
\pgfsetstrokecolor{currentstroke}%
\pgfsetdash{}{0pt}%
\pgfpathmoveto{\pgfqpoint{2.308898in}{5.156155in}}%
\pgfpathlineto{\pgfqpoint{2.258751in}{5.155711in}}%
\pgfusepath{stroke}%
\end{pgfscope}%
\begin{pgfscope}%
\pgfpathrectangle{\pgfqpoint{1.250000in}{4.155455in}}{\pgfqpoint{2.279412in}{2.004545in}}%
\pgfusepath{clip}%
\pgfsetbuttcap%
\pgfsetroundjoin%
\pgfsetlinewidth{0.659276pt}%
\definecolor{currentstroke}{rgb}{0.252194,0.269783,0.531579}%
\pgfsetstrokecolor{currentstroke}%
\pgfsetdash{}{0pt}%
\pgfpathmoveto{\pgfqpoint{2.258751in}{5.155711in}}%
\pgfpathlineto{\pgfqpoint{2.208606in}{5.155395in}}%
\pgfusepath{stroke}%
\end{pgfscope}%
\begin{pgfscope}%
\pgfpathrectangle{\pgfqpoint{1.250000in}{4.155455in}}{\pgfqpoint{2.279412in}{2.004545in}}%
\pgfusepath{clip}%
\pgfsetbuttcap%
\pgfsetroundjoin%
\pgfsetlinewidth{0.516727pt}%
\definecolor{currentstroke}{rgb}{0.279574,0.170599,0.479997}%
\pgfsetstrokecolor{currentstroke}%
\pgfsetdash{}{0pt}%
\pgfpathmoveto{\pgfqpoint{2.208606in}{5.155395in}}%
\pgfpathlineto{\pgfqpoint{2.208606in}{5.155395in}}%
\pgfusepath{stroke}%
\end{pgfscope}%
\begin{pgfscope}%
\pgfpathrectangle{\pgfqpoint{1.250000in}{4.155455in}}{\pgfqpoint{2.279412in}{2.004545in}}%
\pgfusepath{clip}%
\pgfsetbuttcap%
\pgfsetroundjoin%
\pgfsetlinewidth{0.516727pt}%
\definecolor{currentstroke}{rgb}{0.279574,0.170599,0.479997}%
\pgfsetstrokecolor{currentstroke}%
\pgfsetdash{}{0pt}%
\pgfpathmoveto{\pgfqpoint{2.208606in}{5.155395in}}%
\pgfpathlineto{\pgfqpoint{2.168719in}{5.153614in}}%
\pgfusepath{stroke}%
\end{pgfscope}%
\begin{pgfscope}%
\pgfpathrectangle{\pgfqpoint{1.250000in}{4.155455in}}{\pgfqpoint{2.279412in}{2.004545in}}%
\pgfusepath{clip}%
\pgfsetbuttcap%
\pgfsetroundjoin%
\pgfsetlinewidth{0.410249pt}%
\definecolor{currentstroke}{rgb}{0.281924,0.089666,0.412415}%
\pgfsetstrokecolor{currentstroke}%
\pgfsetdash{}{0pt}%
\pgfpathmoveto{\pgfqpoint{2.168719in}{5.153614in}}%
\pgfpathlineto{\pgfqpoint{2.168719in}{5.153614in}}%
\pgfusepath{stroke}%
\end{pgfscope}%
\begin{pgfscope}%
\pgfpathrectangle{\pgfqpoint{1.250000in}{4.155455in}}{\pgfqpoint{2.279412in}{2.004545in}}%
\pgfusepath{clip}%
\pgfsetbuttcap%
\pgfsetroundjoin%
\pgfsetlinewidth{0.315957pt}%
\definecolor{currentstroke}{rgb}{0.269944,0.014625,0.341379}%
\pgfsetstrokecolor{currentstroke}%
\pgfsetdash{}{0pt}%
\pgfpathmoveto{\pgfqpoint{3.261668in}{5.202834in}}%
\pgfpathlineto{\pgfqpoint{3.211981in}{5.204770in}}%
\pgfusepath{stroke}%
\end{pgfscope}%
\begin{pgfscope}%
\pgfpathrectangle{\pgfqpoint{1.250000in}{4.155455in}}{\pgfqpoint{2.279412in}{2.004545in}}%
\pgfusepath{clip}%
\pgfsetbuttcap%
\pgfsetroundjoin%
\pgfsetlinewidth{0.321343pt}%
\definecolor{currentstroke}{rgb}{0.269944,0.014625,0.341379}%
\pgfsetstrokecolor{currentstroke}%
\pgfsetdash{}{0pt}%
\pgfpathmoveto{\pgfqpoint{3.211981in}{5.204770in}}%
\pgfpathlineto{\pgfqpoint{3.161848in}{5.205205in}}%
\pgfusepath{stroke}%
\end{pgfscope}%
\begin{pgfscope}%
\pgfpathrectangle{\pgfqpoint{1.250000in}{4.155455in}}{\pgfqpoint{2.279412in}{2.004545in}}%
\pgfusepath{clip}%
\pgfsetbuttcap%
\pgfsetroundjoin%
\pgfsetlinewidth{0.337568pt}%
\definecolor{currentstroke}{rgb}{0.273809,0.031497,0.358853}%
\pgfsetstrokecolor{currentstroke}%
\pgfsetdash{}{0pt}%
\pgfpathmoveto{\pgfqpoint{3.161848in}{5.205205in}}%
\pgfpathlineto{\pgfqpoint{3.111722in}{5.205900in}}%
\pgfusepath{stroke}%
\end{pgfscope}%
\begin{pgfscope}%
\pgfpathrectangle{\pgfqpoint{1.250000in}{4.155455in}}{\pgfqpoint{2.279412in}{2.004545in}}%
\pgfusepath{clip}%
\pgfsetbuttcap%
\pgfsetroundjoin%
\pgfsetlinewidth{0.342734pt}%
\definecolor{currentstroke}{rgb}{0.274952,0.037752,0.364543}%
\pgfsetstrokecolor{currentstroke}%
\pgfsetdash{}{0pt}%
\pgfpathmoveto{\pgfqpoint{3.111722in}{5.205900in}}%
\pgfpathlineto{\pgfqpoint{3.061586in}{5.206522in}}%
\pgfusepath{stroke}%
\end{pgfscope}%
\begin{pgfscope}%
\pgfpathrectangle{\pgfqpoint{1.250000in}{4.155455in}}{\pgfqpoint{2.279412in}{2.004545in}}%
\pgfusepath{clip}%
\pgfsetbuttcap%
\pgfsetroundjoin%
\pgfsetlinewidth{0.351816pt}%
\definecolor{currentstroke}{rgb}{0.276022,0.044167,0.370164}%
\pgfsetstrokecolor{currentstroke}%
\pgfsetdash{}{0pt}%
\pgfpathmoveto{\pgfqpoint{3.061586in}{5.206522in}}%
\pgfpathlineto{\pgfqpoint{3.011436in}{5.206568in}}%
\pgfusepath{stroke}%
\end{pgfscope}%
\begin{pgfscope}%
\pgfpathrectangle{\pgfqpoint{1.250000in}{4.155455in}}{\pgfqpoint{2.279412in}{2.004545in}}%
\pgfusepath{clip}%
\pgfsetbuttcap%
\pgfsetroundjoin%
\pgfsetlinewidth{0.378744pt}%
\definecolor{currentstroke}{rgb}{0.279566,0.067836,0.391917}%
\pgfsetstrokecolor{currentstroke}%
\pgfsetdash{}{0pt}%
\pgfpathmoveto{\pgfqpoint{3.011436in}{5.206568in}}%
\pgfpathlineto{\pgfqpoint{2.961288in}{5.206340in}}%
\pgfusepath{stroke}%
\end{pgfscope}%
\begin{pgfscope}%
\pgfpathrectangle{\pgfqpoint{1.250000in}{4.155455in}}{\pgfqpoint{2.279412in}{2.004545in}}%
\pgfusepath{clip}%
\pgfsetbuttcap%
\pgfsetroundjoin%
\pgfsetlinewidth{0.412531pt}%
\definecolor{currentstroke}{rgb}{0.282327,0.094955,0.417331}%
\pgfsetstrokecolor{currentstroke}%
\pgfsetdash{}{0pt}%
\pgfpathmoveto{\pgfqpoint{2.961288in}{5.206340in}}%
\pgfpathlineto{\pgfqpoint{2.911140in}{5.205876in}}%
\pgfusepath{stroke}%
\end{pgfscope}%
\begin{pgfscope}%
\pgfpathrectangle{\pgfqpoint{1.250000in}{4.155455in}}{\pgfqpoint{2.279412in}{2.004545in}}%
\pgfusepath{clip}%
\pgfsetbuttcap%
\pgfsetroundjoin%
\pgfsetlinewidth{0.452040pt}%
\definecolor{currentstroke}{rgb}{0.283229,0.120777,0.440584}%
\pgfsetstrokecolor{currentstroke}%
\pgfsetdash{}{0pt}%
\pgfpathmoveto{\pgfqpoint{2.911140in}{5.205876in}}%
\pgfpathlineto{\pgfqpoint{2.860991in}{5.205378in}}%
\pgfusepath{stroke}%
\end{pgfscope}%
\begin{pgfscope}%
\pgfpathrectangle{\pgfqpoint{1.250000in}{4.155455in}}{\pgfqpoint{2.279412in}{2.004545in}}%
\pgfusepath{clip}%
\pgfsetbuttcap%
\pgfsetroundjoin%
\pgfsetlinewidth{0.509855pt}%
\definecolor{currentstroke}{rgb}{0.280255,0.165693,0.476498}%
\pgfsetstrokecolor{currentstroke}%
\pgfsetdash{}{0pt}%
\pgfpathmoveto{\pgfqpoint{2.860991in}{5.205378in}}%
\pgfpathlineto{\pgfqpoint{2.810844in}{5.204817in}}%
\pgfusepath{stroke}%
\end{pgfscope}%
\begin{pgfscope}%
\pgfpathrectangle{\pgfqpoint{1.250000in}{4.155455in}}{\pgfqpoint{2.279412in}{2.004545in}}%
\pgfusepath{clip}%
\pgfsetbuttcap%
\pgfsetroundjoin%
\pgfsetlinewidth{0.570096pt}%
\definecolor{currentstroke}{rgb}{0.271828,0.209303,0.504434}%
\pgfsetstrokecolor{currentstroke}%
\pgfsetdash{}{0pt}%
\pgfpathmoveto{\pgfqpoint{2.810844in}{5.204817in}}%
\pgfpathlineto{\pgfqpoint{2.760698in}{5.204200in}}%
\pgfusepath{stroke}%
\end{pgfscope}%
\begin{pgfscope}%
\pgfpathrectangle{\pgfqpoint{1.250000in}{4.155455in}}{\pgfqpoint{2.279412in}{2.004545in}}%
\pgfusepath{clip}%
\pgfsetbuttcap%
\pgfsetroundjoin%
\pgfsetlinewidth{0.675813pt}%
\definecolor{currentstroke}{rgb}{0.248629,0.278775,0.534556}%
\pgfsetstrokecolor{currentstroke}%
\pgfsetdash{}{0pt}%
\pgfpathmoveto{\pgfqpoint{2.760698in}{5.204200in}}%
\pgfpathlineto{\pgfqpoint{2.710552in}{5.203557in}}%
\pgfusepath{stroke}%
\end{pgfscope}%
\begin{pgfscope}%
\pgfpathrectangle{\pgfqpoint{1.250000in}{4.155455in}}{\pgfqpoint{2.279412in}{2.004545in}}%
\pgfusepath{clip}%
\pgfsetbuttcap%
\pgfsetroundjoin%
\pgfsetlinewidth{0.737203pt}%
\definecolor{currentstroke}{rgb}{0.231674,0.318106,0.544834}%
\pgfsetstrokecolor{currentstroke}%
\pgfsetdash{}{0pt}%
\pgfpathmoveto{\pgfqpoint{2.710552in}{5.203557in}}%
\pgfpathlineto{\pgfqpoint{2.660407in}{5.202860in}}%
\pgfusepath{stroke}%
\end{pgfscope}%
\begin{pgfscope}%
\pgfpathrectangle{\pgfqpoint{1.250000in}{4.155455in}}{\pgfqpoint{2.279412in}{2.004545in}}%
\pgfusepath{clip}%
\pgfsetbuttcap%
\pgfsetroundjoin%
\pgfsetlinewidth{0.818811pt}%
\definecolor{currentstroke}{rgb}{0.208623,0.367752,0.552675}%
\pgfsetstrokecolor{currentstroke}%
\pgfsetdash{}{0pt}%
\pgfpathmoveto{\pgfqpoint{2.660407in}{5.202860in}}%
\pgfpathlineto{\pgfqpoint{2.610265in}{5.202001in}}%
\pgfusepath{stroke}%
\end{pgfscope}%
\begin{pgfscope}%
\pgfpathrectangle{\pgfqpoint{1.250000in}{4.155455in}}{\pgfqpoint{2.279412in}{2.004545in}}%
\pgfusepath{clip}%
\pgfsetbuttcap%
\pgfsetroundjoin%
\pgfsetlinewidth{0.863410pt}%
\definecolor{currentstroke}{rgb}{0.197636,0.391528,0.554969}%
\pgfsetstrokecolor{currentstroke}%
\pgfsetdash{}{0pt}%
\pgfpathmoveto{\pgfqpoint{2.610265in}{5.202001in}}%
\pgfpathlineto{\pgfqpoint{2.560127in}{5.200998in}}%
\pgfusepath{stroke}%
\end{pgfscope}%
\begin{pgfscope}%
\pgfpathrectangle{\pgfqpoint{1.250000in}{4.155455in}}{\pgfqpoint{2.279412in}{2.004545in}}%
\pgfusepath{clip}%
\pgfsetbuttcap%
\pgfsetroundjoin%
\pgfsetlinewidth{0.904552pt}%
\definecolor{currentstroke}{rgb}{0.187231,0.414746,0.556547}%
\pgfsetstrokecolor{currentstroke}%
\pgfsetdash{}{0pt}%
\pgfpathmoveto{\pgfqpoint{2.560127in}{5.200998in}}%
\pgfpathlineto{\pgfqpoint{2.509994in}{5.199818in}}%
\pgfusepath{stroke}%
\end{pgfscope}%
\begin{pgfscope}%
\pgfpathrectangle{\pgfqpoint{1.250000in}{4.155455in}}{\pgfqpoint{2.279412in}{2.004545in}}%
\pgfusepath{clip}%
\pgfsetbuttcap%
\pgfsetroundjoin%
\pgfsetlinewidth{0.955884pt}%
\definecolor{currentstroke}{rgb}{0.174274,0.445044,0.557792}%
\pgfsetstrokecolor{currentstroke}%
\pgfsetdash{}{0pt}%
\pgfpathmoveto{\pgfqpoint{2.509994in}{5.199818in}}%
\pgfpathlineto{\pgfqpoint{2.459868in}{5.198403in}}%
\pgfusepath{stroke}%
\end{pgfscope}%
\begin{pgfscope}%
\pgfpathrectangle{\pgfqpoint{1.250000in}{4.155455in}}{\pgfqpoint{2.279412in}{2.004545in}}%
\pgfusepath{clip}%
\pgfsetbuttcap%
\pgfsetroundjoin%
\pgfsetlinewidth{0.933120pt}%
\definecolor{currentstroke}{rgb}{0.179019,0.433756,0.557430}%
\pgfsetstrokecolor{currentstroke}%
\pgfsetdash{}{0pt}%
\pgfpathmoveto{\pgfqpoint{2.459868in}{5.198403in}}%
\pgfpathlineto{\pgfqpoint{2.409752in}{5.196752in}}%
\pgfusepath{stroke}%
\end{pgfscope}%
\begin{pgfscope}%
\pgfpathrectangle{\pgfqpoint{1.250000in}{4.155455in}}{\pgfqpoint{2.279412in}{2.004545in}}%
\pgfusepath{clip}%
\pgfsetbuttcap%
\pgfsetroundjoin%
\pgfsetlinewidth{0.891611pt}%
\definecolor{currentstroke}{rgb}{0.188923,0.410910,0.556326}%
\pgfsetstrokecolor{currentstroke}%
\pgfsetdash{}{0pt}%
\pgfpathmoveto{\pgfqpoint{2.409752in}{5.196752in}}%
\pgfpathlineto{\pgfqpoint{2.359648in}{5.194853in}}%
\pgfusepath{stroke}%
\end{pgfscope}%
\begin{pgfscope}%
\pgfpathrectangle{\pgfqpoint{1.250000in}{4.155455in}}{\pgfqpoint{2.279412in}{2.004545in}}%
\pgfusepath{clip}%
\pgfsetbuttcap%
\pgfsetroundjoin%
\pgfsetlinewidth{0.842831pt}%
\definecolor{currentstroke}{rgb}{0.203063,0.379716,0.553925}%
\pgfsetstrokecolor{currentstroke}%
\pgfsetdash{}{0pt}%
\pgfpathmoveto{\pgfqpoint{2.359648in}{5.194853in}}%
\pgfpathlineto{\pgfqpoint{2.309569in}{5.192578in}}%
\pgfusepath{stroke}%
\end{pgfscope}%
\begin{pgfscope}%
\pgfpathrectangle{\pgfqpoint{1.250000in}{4.155455in}}{\pgfqpoint{2.279412in}{2.004545in}}%
\pgfusepath{clip}%
\pgfsetbuttcap%
\pgfsetroundjoin%
\pgfsetlinewidth{0.778124pt}%
\definecolor{currentstroke}{rgb}{0.220057,0.343307,0.549413}%
\pgfsetstrokecolor{currentstroke}%
\pgfsetdash{}{0pt}%
\pgfpathmoveto{\pgfqpoint{2.309569in}{5.192578in}}%
\pgfpathlineto{\pgfqpoint{2.259559in}{5.189401in}}%
\pgfusepath{stroke}%
\end{pgfscope}%
\begin{pgfscope}%
\pgfpathrectangle{\pgfqpoint{1.250000in}{4.155455in}}{\pgfqpoint{2.279412in}{2.004545in}}%
\pgfusepath{clip}%
\pgfsetbuttcap%
\pgfsetroundjoin%
\pgfsetlinewidth{0.673240pt}%
\definecolor{currentstroke}{rgb}{0.248629,0.278775,0.534556}%
\pgfsetstrokecolor{currentstroke}%
\pgfsetdash{}{0pt}%
\pgfpathmoveto{\pgfqpoint{2.259559in}{5.189401in}}%
\pgfpathlineto{\pgfqpoint{2.209740in}{5.184519in}}%
\pgfusepath{stroke}%
\end{pgfscope}%
\begin{pgfscope}%
\pgfpathrectangle{\pgfqpoint{1.250000in}{4.155455in}}{\pgfqpoint{2.279412in}{2.004545in}}%
\pgfusepath{clip}%
\pgfsetbuttcap%
\pgfsetroundjoin%
\pgfsetlinewidth{0.531114pt}%
\definecolor{currentstroke}{rgb}{0.278012,0.180367,0.486697}%
\pgfsetstrokecolor{currentstroke}%
\pgfsetdash{}{0pt}%
\pgfpathmoveto{\pgfqpoint{2.209740in}{5.184519in}}%
\pgfpathlineto{\pgfqpoint{2.209740in}{5.184519in}}%
\pgfusepath{stroke}%
\end{pgfscope}%
\begin{pgfscope}%
\pgfpathrectangle{\pgfqpoint{1.250000in}{4.155455in}}{\pgfqpoint{2.279412in}{2.004545in}}%
\pgfusepath{clip}%
\pgfsetbuttcap%
\pgfsetroundjoin%
\pgfsetlinewidth{0.316003pt}%
\definecolor{currentstroke}{rgb}{0.269944,0.014625,0.341379}%
\pgfsetstrokecolor{currentstroke}%
\pgfsetdash{}{0pt}%
\pgfpathmoveto{\pgfqpoint{3.308228in}{5.297635in}}%
\pgfpathlineto{\pgfqpoint{3.261668in}{5.293048in}}%
\pgfusepath{stroke}%
\end{pgfscope}%
\begin{pgfscope}%
\pgfpathrectangle{\pgfqpoint{1.250000in}{4.155455in}}{\pgfqpoint{2.279412in}{2.004545in}}%
\pgfusepath{clip}%
\pgfsetbuttcap%
\pgfsetroundjoin%
\pgfsetlinewidth{0.307472pt}%
\definecolor{currentstroke}{rgb}{0.267004,0.004874,0.329415}%
\pgfsetstrokecolor{currentstroke}%
\pgfsetdash{}{0pt}%
\pgfpathmoveto{\pgfqpoint{3.261668in}{5.293048in}}%
\pgfpathlineto{\pgfqpoint{3.212910in}{5.289157in}}%
\pgfusepath{stroke}%
\end{pgfscope}%
\begin{pgfscope}%
\pgfpathrectangle{\pgfqpoint{1.250000in}{4.155455in}}{\pgfqpoint{2.279412in}{2.004545in}}%
\pgfusepath{clip}%
\pgfsetbuttcap%
\pgfsetroundjoin%
\pgfsetlinewidth{0.322418pt}%
\definecolor{currentstroke}{rgb}{0.271305,0.019942,0.347269}%
\pgfsetstrokecolor{currentstroke}%
\pgfsetdash{}{0pt}%
\pgfpathmoveto{\pgfqpoint{3.212910in}{5.289157in}}%
\pgfpathlineto{\pgfqpoint{3.164108in}{5.288957in}}%
\pgfusepath{stroke}%
\end{pgfscope}%
\begin{pgfscope}%
\pgfpathrectangle{\pgfqpoint{1.250000in}{4.155455in}}{\pgfqpoint{2.279412in}{2.004545in}}%
\pgfusepath{clip}%
\pgfsetbuttcap%
\pgfsetroundjoin%
\pgfsetlinewidth{0.322336pt}%
\definecolor{currentstroke}{rgb}{0.271305,0.019942,0.347269}%
\pgfsetstrokecolor{currentstroke}%
\pgfsetdash{}{0pt}%
\pgfpathmoveto{\pgfqpoint{3.164108in}{5.288957in}}%
\pgfpathlineto{\pgfqpoint{3.114000in}{5.288004in}}%
\pgfusepath{stroke}%
\end{pgfscope}%
\begin{pgfscope}%
\pgfpathrectangle{\pgfqpoint{1.250000in}{4.155455in}}{\pgfqpoint{2.279412in}{2.004545in}}%
\pgfusepath{clip}%
\pgfsetbuttcap%
\pgfsetroundjoin%
\pgfsetlinewidth{0.337190pt}%
\definecolor{currentstroke}{rgb}{0.273809,0.031497,0.358853}%
\pgfsetstrokecolor{currentstroke}%
\pgfsetdash{}{0pt}%
\pgfpathmoveto{\pgfqpoint{3.114000in}{5.288004in}}%
\pgfpathlineto{\pgfqpoint{3.063867in}{5.287036in}}%
\pgfusepath{stroke}%
\end{pgfscope}%
\begin{pgfscope}%
\pgfpathrectangle{\pgfqpoint{1.250000in}{4.155455in}}{\pgfqpoint{2.279412in}{2.004545in}}%
\pgfusepath{clip}%
\pgfsetbuttcap%
\pgfsetroundjoin%
\pgfsetlinewidth{0.348377pt}%
\definecolor{currentstroke}{rgb}{0.274952,0.037752,0.364543}%
\pgfsetstrokecolor{currentstroke}%
\pgfsetdash{}{0pt}%
\pgfpathmoveto{\pgfqpoint{3.063867in}{5.287036in}}%
\pgfpathlineto{\pgfqpoint{3.013725in}{5.286268in}}%
\pgfusepath{stroke}%
\end{pgfscope}%
\begin{pgfscope}%
\pgfpathrectangle{\pgfqpoint{1.250000in}{4.155455in}}{\pgfqpoint{2.279412in}{2.004545in}}%
\pgfusepath{clip}%
\pgfsetbuttcap%
\pgfsetroundjoin%
\pgfsetlinewidth{0.371419pt}%
\definecolor{currentstroke}{rgb}{0.278791,0.062145,0.386592}%
\pgfsetstrokecolor{currentstroke}%
\pgfsetdash{}{0pt}%
\pgfpathmoveto{\pgfqpoint{3.013725in}{5.286268in}}%
\pgfpathlineto{\pgfqpoint{2.963581in}{5.285592in}}%
\pgfusepath{stroke}%
\end{pgfscope}%
\begin{pgfscope}%
\pgfpathrectangle{\pgfqpoint{1.250000in}{4.155455in}}{\pgfqpoint{2.279412in}{2.004545in}}%
\pgfusepath{clip}%
\pgfsetbuttcap%
\pgfsetroundjoin%
\pgfsetlinewidth{0.396002pt}%
\definecolor{currentstroke}{rgb}{0.280894,0.078907,0.402329}%
\pgfsetstrokecolor{currentstroke}%
\pgfsetdash{}{0pt}%
\pgfpathmoveto{\pgfqpoint{2.963581in}{5.285592in}}%
\pgfpathlineto{\pgfqpoint{2.913438in}{5.284812in}}%
\pgfusepath{stroke}%
\end{pgfscope}%
\begin{pgfscope}%
\pgfpathrectangle{\pgfqpoint{1.250000in}{4.155455in}}{\pgfqpoint{2.279412in}{2.004545in}}%
\pgfusepath{clip}%
\pgfsetbuttcap%
\pgfsetroundjoin%
\pgfsetlinewidth{0.448721pt}%
\definecolor{currentstroke}{rgb}{0.283229,0.120777,0.440584}%
\pgfsetstrokecolor{currentstroke}%
\pgfsetdash{}{0pt}%
\pgfpathmoveto{\pgfqpoint{2.913438in}{5.284812in}}%
\pgfpathlineto{\pgfqpoint{2.863296in}{5.283955in}}%
\pgfusepath{stroke}%
\end{pgfscope}%
\begin{pgfscope}%
\pgfpathrectangle{\pgfqpoint{1.250000in}{4.155455in}}{\pgfqpoint{2.279412in}{2.004545in}}%
\pgfusepath{clip}%
\pgfsetbuttcap%
\pgfsetroundjoin%
\pgfsetlinewidth{0.496969pt}%
\definecolor{currentstroke}{rgb}{0.281412,0.155834,0.469201}%
\pgfsetstrokecolor{currentstroke}%
\pgfsetdash{}{0pt}%
\pgfpathmoveto{\pgfqpoint{2.863296in}{5.283955in}}%
\pgfpathlineto{\pgfqpoint{2.813157in}{5.282993in}}%
\pgfusepath{stroke}%
\end{pgfscope}%
\begin{pgfscope}%
\pgfpathrectangle{\pgfqpoint{1.250000in}{4.155455in}}{\pgfqpoint{2.279412in}{2.004545in}}%
\pgfusepath{clip}%
\pgfsetbuttcap%
\pgfsetroundjoin%
\pgfsetlinewidth{0.551764pt}%
\definecolor{currentstroke}{rgb}{0.275191,0.194905,0.496005}%
\pgfsetstrokecolor{currentstroke}%
\pgfsetdash{}{0pt}%
\pgfpathmoveto{\pgfqpoint{2.813157in}{5.282993in}}%
\pgfpathlineto{\pgfqpoint{2.763028in}{5.281668in}}%
\pgfusepath{stroke}%
\end{pgfscope}%
\begin{pgfscope}%
\pgfpathrectangle{\pgfqpoint{1.250000in}{4.155455in}}{\pgfqpoint{2.279412in}{2.004545in}}%
\pgfusepath{clip}%
\pgfsetbuttcap%
\pgfsetroundjoin%
\pgfsetlinewidth{0.602633pt}%
\definecolor{currentstroke}{rgb}{0.266580,0.228262,0.514349}%
\pgfsetstrokecolor{currentstroke}%
\pgfsetdash{}{0pt}%
\pgfpathmoveto{\pgfqpoint{2.763028in}{5.281668in}}%
\pgfpathlineto{\pgfqpoint{2.712908in}{5.280097in}}%
\pgfusepath{stroke}%
\end{pgfscope}%
\begin{pgfscope}%
\pgfpathrectangle{\pgfqpoint{1.250000in}{4.155455in}}{\pgfqpoint{2.279412in}{2.004545in}}%
\pgfusepath{clip}%
\pgfsetbuttcap%
\pgfsetroundjoin%
\pgfsetlinewidth{0.691756pt}%
\definecolor{currentstroke}{rgb}{0.244972,0.287675,0.537260}%
\pgfsetstrokecolor{currentstroke}%
\pgfsetdash{}{0pt}%
\pgfpathmoveto{\pgfqpoint{2.712908in}{5.280097in}}%
\pgfpathlineto{\pgfqpoint{2.662797in}{5.278326in}}%
\pgfusepath{stroke}%
\end{pgfscope}%
\begin{pgfscope}%
\pgfpathrectangle{\pgfqpoint{1.250000in}{4.155455in}}{\pgfqpoint{2.279412in}{2.004545in}}%
\pgfusepath{clip}%
\pgfsetbuttcap%
\pgfsetroundjoin%
\pgfsetlinewidth{0.753269pt}%
\definecolor{currentstroke}{rgb}{0.227802,0.326594,0.546532}%
\pgfsetstrokecolor{currentstroke}%
\pgfsetdash{}{0pt}%
\pgfpathmoveto{\pgfqpoint{2.662797in}{5.278326in}}%
\pgfpathlineto{\pgfqpoint{2.612698in}{5.276308in}}%
\pgfusepath{stroke}%
\end{pgfscope}%
\begin{pgfscope}%
\pgfpathrectangle{\pgfqpoint{1.250000in}{4.155455in}}{\pgfqpoint{2.279412in}{2.004545in}}%
\pgfusepath{clip}%
\pgfsetbuttcap%
\pgfsetroundjoin%
\pgfsetlinewidth{0.837951pt}%
\definecolor{currentstroke}{rgb}{0.203063,0.379716,0.553925}%
\pgfsetstrokecolor{currentstroke}%
\pgfsetdash{}{0pt}%
\pgfpathmoveto{\pgfqpoint{2.612698in}{5.276308in}}%
\pgfpathlineto{\pgfqpoint{2.562614in}{5.274027in}}%
\pgfusepath{stroke}%
\end{pgfscope}%
\begin{pgfscope}%
\pgfpathrectangle{\pgfqpoint{1.250000in}{4.155455in}}{\pgfqpoint{2.279412in}{2.004545in}}%
\pgfusepath{clip}%
\pgfsetbuttcap%
\pgfsetroundjoin%
\pgfsetlinewidth{0.873015pt}%
\definecolor{currentstroke}{rgb}{0.194100,0.399323,0.555565}%
\pgfsetstrokecolor{currentstroke}%
\pgfsetdash{}{0pt}%
\pgfpathmoveto{\pgfqpoint{2.562614in}{5.274027in}}%
\pgfpathlineto{\pgfqpoint{2.512549in}{5.271433in}}%
\pgfusepath{stroke}%
\end{pgfscope}%
\begin{pgfscope}%
\pgfpathrectangle{\pgfqpoint{1.250000in}{4.155455in}}{\pgfqpoint{2.279412in}{2.004545in}}%
\pgfusepath{clip}%
\pgfsetbuttcap%
\pgfsetroundjoin%
\pgfsetlinewidth{0.888508pt}%
\definecolor{currentstroke}{rgb}{0.190631,0.407061,0.556089}%
\pgfsetstrokecolor{currentstroke}%
\pgfsetdash{}{0pt}%
\pgfpathmoveto{\pgfqpoint{2.512549in}{5.271433in}}%
\pgfpathlineto{\pgfqpoint{2.462511in}{5.268482in}}%
\pgfusepath{stroke}%
\end{pgfscope}%
\begin{pgfscope}%
\pgfpathrectangle{\pgfqpoint{1.250000in}{4.155455in}}{\pgfqpoint{2.279412in}{2.004545in}}%
\pgfusepath{clip}%
\pgfsetbuttcap%
\pgfsetroundjoin%
\pgfsetlinewidth{0.887245pt}%
\definecolor{currentstroke}{rgb}{0.190631,0.407061,0.556089}%
\pgfsetstrokecolor{currentstroke}%
\pgfsetdash{}{0pt}%
\pgfpathmoveto{\pgfqpoint{2.462511in}{5.268482in}}%
\pgfpathlineto{\pgfqpoint{2.412517in}{5.265022in}}%
\pgfusepath{stroke}%
\end{pgfscope}%
\begin{pgfscope}%
\pgfpathrectangle{\pgfqpoint{1.250000in}{4.155455in}}{\pgfqpoint{2.279412in}{2.004545in}}%
\pgfusepath{clip}%
\pgfsetbuttcap%
\pgfsetroundjoin%
\pgfsetlinewidth{0.843980pt}%
\definecolor{currentstroke}{rgb}{0.201239,0.383670,0.554294}%
\pgfsetstrokecolor{currentstroke}%
\pgfsetdash{}{0pt}%
\pgfpathmoveto{\pgfqpoint{2.412517in}{5.265022in}}%
\pgfpathlineto{\pgfqpoint{2.362584in}{5.260948in}}%
\pgfusepath{stroke}%
\end{pgfscope}%
\begin{pgfscope}%
\pgfpathrectangle{\pgfqpoint{1.250000in}{4.155455in}}{\pgfqpoint{2.279412in}{2.004545in}}%
\pgfusepath{clip}%
\pgfsetbuttcap%
\pgfsetroundjoin%
\pgfsetlinewidth{0.841402pt}%
\definecolor{currentstroke}{rgb}{0.203063,0.379716,0.553925}%
\pgfsetstrokecolor{currentstroke}%
\pgfsetdash{}{0pt}%
\pgfpathmoveto{\pgfqpoint{2.362584in}{5.260948in}}%
\pgfpathlineto{\pgfqpoint{2.312763in}{5.255954in}}%
\pgfusepath{stroke}%
\end{pgfscope}%
\begin{pgfscope}%
\pgfpathrectangle{\pgfqpoint{1.250000in}{4.155455in}}{\pgfqpoint{2.279412in}{2.004545in}}%
\pgfusepath{clip}%
\pgfsetbuttcap%
\pgfsetroundjoin%
\pgfsetlinewidth{0.750001pt}%
\definecolor{currentstroke}{rgb}{0.227802,0.326594,0.546532}%
\pgfsetstrokecolor{currentstroke}%
\pgfsetdash{}{0pt}%
\pgfpathmoveto{\pgfqpoint{2.312763in}{5.255954in}}%
\pgfpathlineto{\pgfqpoint{2.263261in}{5.249049in}}%
\pgfusepath{stroke}%
\end{pgfscope}%
\begin{pgfscope}%
\pgfpathrectangle{\pgfqpoint{1.250000in}{4.155455in}}{\pgfqpoint{2.279412in}{2.004545in}}%
\pgfusepath{clip}%
\pgfsetbuttcap%
\pgfsetroundjoin%
\pgfsetlinewidth{0.636592pt}%
\definecolor{currentstroke}{rgb}{0.258965,0.251537,0.524736}%
\pgfsetstrokecolor{currentstroke}%
\pgfsetdash{}{0pt}%
\pgfpathmoveto{\pgfqpoint{2.263261in}{5.249049in}}%
\pgfpathlineto{\pgfqpoint{2.214557in}{5.238805in}}%
\pgfusepath{stroke}%
\end{pgfscope}%
\begin{pgfscope}%
\pgfpathrectangle{\pgfqpoint{1.250000in}{4.155455in}}{\pgfqpoint{2.279412in}{2.004545in}}%
\pgfusepath{clip}%
\pgfsetbuttcap%
\pgfsetroundjoin%
\pgfsetlinewidth{0.553080pt}%
\definecolor{currentstroke}{rgb}{0.275191,0.194905,0.496005}%
\pgfsetstrokecolor{currentstroke}%
\pgfsetdash{}{0pt}%
\pgfpathmoveto{\pgfqpoint{2.214557in}{5.238805in}}%
\pgfpathlineto{\pgfqpoint{2.214557in}{5.238805in}}%
\pgfusepath{stroke}%
\end{pgfscope}%
\begin{pgfscope}%
\pgfpathrectangle{\pgfqpoint{1.250000in}{4.155455in}}{\pgfqpoint{2.279412in}{2.004545in}}%
\pgfusepath{clip}%
\pgfsetbuttcap%
\pgfsetroundjoin%
\pgfsetlinewidth{0.553080pt}%
\definecolor{currentstroke}{rgb}{0.275191,0.194905,0.496005}%
\pgfsetstrokecolor{currentstroke}%
\pgfsetdash{}{0pt}%
\pgfpathmoveto{\pgfqpoint{2.214557in}{5.238805in}}%
\pgfpathlineto{\pgfqpoint{2.183145in}{5.228287in}}%
\pgfusepath{stroke}%
\end{pgfscope}%
\begin{pgfscope}%
\pgfpathrectangle{\pgfqpoint{1.250000in}{4.155455in}}{\pgfqpoint{2.279412in}{2.004545in}}%
\pgfusepath{clip}%
\pgfsetbuttcap%
\pgfsetroundjoin%
\pgfsetlinewidth{0.504134pt}%
\definecolor{currentstroke}{rgb}{0.280868,0.160771,0.472899}%
\pgfsetstrokecolor{currentstroke}%
\pgfsetdash{}{0pt}%
\pgfpathmoveto{\pgfqpoint{2.183145in}{5.228287in}}%
\pgfpathlineto{\pgfqpoint{2.183145in}{5.228287in}}%
\pgfusepath{stroke}%
\end{pgfscope}%
\begin{pgfscope}%
\pgfpathrectangle{\pgfqpoint{1.250000in}{4.155455in}}{\pgfqpoint{2.279412in}{2.004545in}}%
\pgfusepath{clip}%
\pgfsetbuttcap%
\pgfsetroundjoin%
\pgfsetlinewidth{0.314519pt}%
\definecolor{currentstroke}{rgb}{0.268510,0.009605,0.335427}%
\pgfsetstrokecolor{currentstroke}%
\pgfsetdash{}{0pt}%
\pgfpathmoveto{\pgfqpoint{3.285241in}{5.338870in}}%
\pgfpathlineto{\pgfqpoint{3.261668in}{5.338154in}}%
\pgfusepath{stroke}%
\end{pgfscope}%
\begin{pgfscope}%
\pgfpathrectangle{\pgfqpoint{1.250000in}{4.155455in}}{\pgfqpoint{2.279412in}{2.004545in}}%
\pgfusepath{clip}%
\pgfsetbuttcap%
\pgfsetroundjoin%
\pgfsetlinewidth{0.311532pt}%
\definecolor{currentstroke}{rgb}{0.268510,0.009605,0.335427}%
\pgfsetstrokecolor{currentstroke}%
\pgfsetdash{}{0pt}%
\pgfpathmoveto{\pgfqpoint{3.261668in}{5.338154in}}%
\pgfpathlineto{\pgfqpoint{3.261668in}{5.338154in}}%
\pgfusepath{stroke}%
\end{pgfscope}%
\begin{pgfscope}%
\pgfpathrectangle{\pgfqpoint{1.250000in}{4.155455in}}{\pgfqpoint{2.279412in}{2.004545in}}%
\pgfusepath{clip}%
\pgfsetbuttcap%
\pgfsetroundjoin%
\pgfsetlinewidth{0.311532pt}%
\definecolor{currentstroke}{rgb}{0.268510,0.009605,0.335427}%
\pgfsetstrokecolor{currentstroke}%
\pgfsetdash{}{0pt}%
\pgfpathmoveto{\pgfqpoint{3.261668in}{5.338154in}}%
\pgfpathlineto{\pgfqpoint{3.261668in}{5.338154in}}%
\pgfusepath{stroke}%
\end{pgfscope}%
\begin{pgfscope}%
\pgfpathrectangle{\pgfqpoint{1.250000in}{4.155455in}}{\pgfqpoint{2.279412in}{2.004545in}}%
\pgfusepath{clip}%
\pgfsetbuttcap%
\pgfsetroundjoin%
\pgfsetlinewidth{0.311532pt}%
\definecolor{currentstroke}{rgb}{0.268510,0.009605,0.335427}%
\pgfsetstrokecolor{currentstroke}%
\pgfsetdash{}{0pt}%
\pgfpathmoveto{\pgfqpoint{3.261668in}{5.338154in}}%
\pgfpathlineto{\pgfqpoint{3.226608in}{5.336542in}}%
\pgfusepath{stroke}%
\end{pgfscope}%
\begin{pgfscope}%
\pgfpathrectangle{\pgfqpoint{1.250000in}{4.155455in}}{\pgfqpoint{2.279412in}{2.004545in}}%
\pgfusepath{clip}%
\pgfsetbuttcap%
\pgfsetroundjoin%
\pgfsetlinewidth{0.315923pt}%
\definecolor{currentstroke}{rgb}{0.269944,0.014625,0.341379}%
\pgfsetstrokecolor{currentstroke}%
\pgfsetdash{}{0pt}%
\pgfpathmoveto{\pgfqpoint{3.226608in}{5.336542in}}%
\pgfpathlineto{\pgfqpoint{3.177907in}{5.335774in}}%
\pgfusepath{stroke}%
\end{pgfscope}%
\begin{pgfscope}%
\pgfpathrectangle{\pgfqpoint{1.250000in}{4.155455in}}{\pgfqpoint{2.279412in}{2.004545in}}%
\pgfusepath{clip}%
\pgfsetbuttcap%
\pgfsetroundjoin%
\pgfsetlinewidth{0.329728pt}%
\definecolor{currentstroke}{rgb}{0.272594,0.025563,0.353093}%
\pgfsetstrokecolor{currentstroke}%
\pgfsetdash{}{0pt}%
\pgfpathmoveto{\pgfqpoint{3.177907in}{5.335774in}}%
\pgfpathlineto{\pgfqpoint{3.127854in}{5.333289in}}%
\pgfusepath{stroke}%
\end{pgfscope}%
\begin{pgfscope}%
\pgfpathrectangle{\pgfqpoint{1.250000in}{4.155455in}}{\pgfqpoint{2.279412in}{2.004545in}}%
\pgfusepath{clip}%
\pgfsetbuttcap%
\pgfsetroundjoin%
\pgfsetlinewidth{0.326670pt}%
\definecolor{currentstroke}{rgb}{0.271305,0.019942,0.347269}%
\pgfsetstrokecolor{currentstroke}%
\pgfsetdash{}{0pt}%
\pgfpathmoveto{\pgfqpoint{3.127854in}{5.333289in}}%
\pgfpathlineto{\pgfqpoint{3.077830in}{5.330538in}}%
\pgfusepath{stroke}%
\end{pgfscope}%
\begin{pgfscope}%
\pgfpathrectangle{\pgfqpoint{1.250000in}{4.155455in}}{\pgfqpoint{2.279412in}{2.004545in}}%
\pgfusepath{clip}%
\pgfsetbuttcap%
\pgfsetroundjoin%
\pgfsetlinewidth{0.344946pt}%
\definecolor{currentstroke}{rgb}{0.274952,0.037752,0.364543}%
\pgfsetstrokecolor{currentstroke}%
\pgfsetdash{}{0pt}%
\pgfpathmoveto{\pgfqpoint{3.077830in}{5.330538in}}%
\pgfpathlineto{\pgfqpoint{3.027764in}{5.328542in}}%
\pgfusepath{stroke}%
\end{pgfscope}%
\begin{pgfscope}%
\pgfpathrectangle{\pgfqpoint{1.250000in}{4.155455in}}{\pgfqpoint{2.279412in}{2.004545in}}%
\pgfusepath{clip}%
\pgfsetbuttcap%
\pgfsetroundjoin%
\pgfsetlinewidth{0.366837pt}%
\definecolor{currentstroke}{rgb}{0.277941,0.056324,0.381191}%
\pgfsetstrokecolor{currentstroke}%
\pgfsetdash{}{0pt}%
\pgfpathmoveto{\pgfqpoint{3.027764in}{5.328542in}}%
\pgfpathlineto{\pgfqpoint{2.977632in}{5.327326in}}%
\pgfusepath{stroke}%
\end{pgfscope}%
\begin{pgfscope}%
\pgfpathrectangle{\pgfqpoint{1.250000in}{4.155455in}}{\pgfqpoint{2.279412in}{2.004545in}}%
\pgfusepath{clip}%
\pgfsetbuttcap%
\pgfsetroundjoin%
\pgfsetlinewidth{0.391606pt}%
\definecolor{currentstroke}{rgb}{0.280894,0.078907,0.402329}%
\pgfsetstrokecolor{currentstroke}%
\pgfsetdash{}{0pt}%
\pgfpathmoveto{\pgfqpoint{2.977632in}{5.327326in}}%
\pgfpathlineto{\pgfqpoint{2.927500in}{5.326114in}}%
\pgfusepath{stroke}%
\end{pgfscope}%
\begin{pgfscope}%
\pgfpathrectangle{\pgfqpoint{1.250000in}{4.155455in}}{\pgfqpoint{2.279412in}{2.004545in}}%
\pgfusepath{clip}%
\pgfsetbuttcap%
\pgfsetroundjoin%
\pgfsetlinewidth{0.417674pt}%
\definecolor{currentstroke}{rgb}{0.282327,0.094955,0.417331}%
\pgfsetstrokecolor{currentstroke}%
\pgfsetdash{}{0pt}%
\pgfpathmoveto{\pgfqpoint{2.927500in}{5.326114in}}%
\pgfpathlineto{\pgfqpoint{2.877372in}{5.324779in}}%
\pgfusepath{stroke}%
\end{pgfscope}%
\begin{pgfscope}%
\pgfpathrectangle{\pgfqpoint{1.250000in}{4.155455in}}{\pgfqpoint{2.279412in}{2.004545in}}%
\pgfusepath{clip}%
\pgfsetbuttcap%
\pgfsetroundjoin%
\pgfsetlinewidth{0.467020pt}%
\definecolor{currentstroke}{rgb}{0.282884,0.135920,0.453427}%
\pgfsetstrokecolor{currentstroke}%
\pgfsetdash{}{0pt}%
\pgfpathmoveto{\pgfqpoint{2.877372in}{5.324779in}}%
\pgfpathlineto{\pgfqpoint{2.827254in}{5.323185in}}%
\pgfusepath{stroke}%
\end{pgfscope}%
\begin{pgfscope}%
\pgfpathrectangle{\pgfqpoint{1.250000in}{4.155455in}}{\pgfqpoint{2.279412in}{2.004545in}}%
\pgfusepath{clip}%
\pgfsetbuttcap%
\pgfsetroundjoin%
\pgfsetlinewidth{0.517446pt}%
\definecolor{currentstroke}{rgb}{0.279574,0.170599,0.479997}%
\pgfsetstrokecolor{currentstroke}%
\pgfsetdash{}{0pt}%
\pgfpathmoveto{\pgfqpoint{2.827254in}{5.323185in}}%
\pgfpathlineto{\pgfqpoint{2.777146in}{5.321356in}}%
\pgfusepath{stroke}%
\end{pgfscope}%
\begin{pgfscope}%
\pgfpathrectangle{\pgfqpoint{1.250000in}{4.155455in}}{\pgfqpoint{2.279412in}{2.004545in}}%
\pgfusepath{clip}%
\pgfsetbuttcap%
\pgfsetroundjoin%
\pgfsetlinewidth{0.582352pt}%
\definecolor{currentstroke}{rgb}{0.270595,0.214069,0.507052}%
\pgfsetstrokecolor{currentstroke}%
\pgfsetdash{}{0pt}%
\pgfpathmoveto{\pgfqpoint{2.777146in}{5.321356in}}%
\pgfpathlineto{\pgfqpoint{2.727049in}{5.319311in}}%
\pgfusepath{stroke}%
\end{pgfscope}%
\begin{pgfscope}%
\pgfpathrectangle{\pgfqpoint{1.250000in}{4.155455in}}{\pgfqpoint{2.279412in}{2.004545in}}%
\pgfusepath{clip}%
\pgfsetbuttcap%
\pgfsetroundjoin%
\pgfsetlinewidth{0.642484pt}%
\definecolor{currentstroke}{rgb}{0.257322,0.256130,0.526563}%
\pgfsetstrokecolor{currentstroke}%
\pgfsetdash{}{0pt}%
\pgfpathmoveto{\pgfqpoint{2.727049in}{5.319311in}}%
\pgfpathlineto{\pgfqpoint{2.676970in}{5.316949in}}%
\pgfusepath{stroke}%
\end{pgfscope}%
\begin{pgfscope}%
\pgfpathrectangle{\pgfqpoint{1.250000in}{4.155455in}}{\pgfqpoint{2.279412in}{2.004545in}}%
\pgfusepath{clip}%
\pgfsetbuttcap%
\pgfsetroundjoin%
\pgfsetlinewidth{0.324699pt}%
\definecolor{currentstroke}{rgb}{0.271305,0.019942,0.347269}%
\pgfsetstrokecolor{currentstroke}%
\pgfsetdash{}{0pt}%
\pgfpathmoveto{\pgfqpoint{3.210376in}{5.022407in}}%
\pgfpathlineto{\pgfqpoint{3.160255in}{5.023055in}}%
\pgfusepath{stroke}%
\end{pgfscope}%
\begin{pgfscope}%
\pgfpathrectangle{\pgfqpoint{1.250000in}{4.155455in}}{\pgfqpoint{2.279412in}{2.004545in}}%
\pgfusepath{clip}%
\pgfsetbuttcap%
\pgfsetroundjoin%
\pgfsetlinewidth{0.331751pt}%
\definecolor{currentstroke}{rgb}{0.272594,0.025563,0.353093}%
\pgfsetstrokecolor{currentstroke}%
\pgfsetdash{}{0pt}%
\pgfpathmoveto{\pgfqpoint{3.160255in}{5.023055in}}%
\pgfpathlineto{\pgfqpoint{3.110112in}{5.023650in}}%
\pgfusepath{stroke}%
\end{pgfscope}%
\begin{pgfscope}%
\pgfpathrectangle{\pgfqpoint{1.250000in}{4.155455in}}{\pgfqpoint{2.279412in}{2.004545in}}%
\pgfusepath{clip}%
\pgfsetbuttcap%
\pgfsetroundjoin%
\pgfsetlinewidth{0.337042pt}%
\definecolor{currentstroke}{rgb}{0.273809,0.031497,0.358853}%
\pgfsetstrokecolor{currentstroke}%
\pgfsetdash{}{0pt}%
\pgfpathmoveto{\pgfqpoint{3.110112in}{5.023650in}}%
\pgfpathlineto{\pgfqpoint{3.059972in}{5.024378in}}%
\pgfusepath{stroke}%
\end{pgfscope}%
\begin{pgfscope}%
\pgfpathrectangle{\pgfqpoint{1.250000in}{4.155455in}}{\pgfqpoint{2.279412in}{2.004545in}}%
\pgfusepath{clip}%
\pgfsetbuttcap%
\pgfsetroundjoin%
\pgfsetlinewidth{0.357037pt}%
\definecolor{currentstroke}{rgb}{0.277018,0.050344,0.375715}%
\pgfsetstrokecolor{currentstroke}%
\pgfsetdash{}{0pt}%
\pgfpathmoveto{\pgfqpoint{3.059972in}{5.024378in}}%
\pgfpathlineto{\pgfqpoint{3.009832in}{5.025219in}}%
\pgfusepath{stroke}%
\end{pgfscope}%
\begin{pgfscope}%
\pgfpathrectangle{\pgfqpoint{1.250000in}{4.155455in}}{\pgfqpoint{2.279412in}{2.004545in}}%
\pgfusepath{clip}%
\pgfsetbuttcap%
\pgfsetroundjoin%
\pgfsetlinewidth{0.375027pt}%
\definecolor{currentstroke}{rgb}{0.278791,0.062145,0.386592}%
\pgfsetstrokecolor{currentstroke}%
\pgfsetdash{}{0pt}%
\pgfpathmoveto{\pgfqpoint{3.009832in}{5.025219in}}%
\pgfpathlineto{\pgfqpoint{2.959696in}{5.026301in}}%
\pgfusepath{stroke}%
\end{pgfscope}%
\begin{pgfscope}%
\pgfpathrectangle{\pgfqpoint{1.250000in}{4.155455in}}{\pgfqpoint{2.279412in}{2.004545in}}%
\pgfusepath{clip}%
\pgfsetbuttcap%
\pgfsetroundjoin%
\pgfsetlinewidth{0.407157pt}%
\definecolor{currentstroke}{rgb}{0.281924,0.089666,0.412415}%
\pgfsetstrokecolor{currentstroke}%
\pgfsetdash{}{0pt}%
\pgfpathmoveto{\pgfqpoint{2.959696in}{5.026301in}}%
\pgfpathlineto{\pgfqpoint{2.909561in}{5.027438in}}%
\pgfusepath{stroke}%
\end{pgfscope}%
\begin{pgfscope}%
\pgfpathrectangle{\pgfqpoint{1.250000in}{4.155455in}}{\pgfqpoint{2.279412in}{2.004545in}}%
\pgfusepath{clip}%
\pgfsetbuttcap%
\pgfsetroundjoin%
\pgfsetlinewidth{0.446113pt}%
\definecolor{currentstroke}{rgb}{0.283229,0.120777,0.440584}%
\pgfsetstrokecolor{currentstroke}%
\pgfsetdash{}{0pt}%
\pgfpathmoveto{\pgfqpoint{2.909561in}{5.027438in}}%
\pgfpathlineto{\pgfqpoint{2.859428in}{5.028649in}}%
\pgfusepath{stroke}%
\end{pgfscope}%
\begin{pgfscope}%
\pgfpathrectangle{\pgfqpoint{1.250000in}{4.155455in}}{\pgfqpoint{2.279412in}{2.004545in}}%
\pgfusepath{clip}%
\pgfsetbuttcap%
\pgfsetroundjoin%
\pgfsetlinewidth{0.488952pt}%
\definecolor{currentstroke}{rgb}{0.281887,0.150881,0.465405}%
\pgfsetstrokecolor{currentstroke}%
\pgfsetdash{}{0pt}%
\pgfpathmoveto{\pgfqpoint{2.859428in}{5.028649in}}%
\pgfpathlineto{\pgfqpoint{2.809299in}{5.029953in}}%
\pgfusepath{stroke}%
\end{pgfscope}%
\begin{pgfscope}%
\pgfpathrectangle{\pgfqpoint{1.250000in}{4.155455in}}{\pgfqpoint{2.279412in}{2.004545in}}%
\pgfusepath{clip}%
\pgfsetbuttcap%
\pgfsetroundjoin%
\pgfsetlinewidth{0.558410pt}%
\definecolor{currentstroke}{rgb}{0.274128,0.199721,0.498911}%
\pgfsetstrokecolor{currentstroke}%
\pgfsetdash{}{0pt}%
\pgfpathmoveto{\pgfqpoint{2.809299in}{5.029953in}}%
\pgfpathlineto{\pgfqpoint{2.759172in}{5.031320in}}%
\pgfusepath{stroke}%
\end{pgfscope}%
\begin{pgfscope}%
\pgfpathrectangle{\pgfqpoint{1.250000in}{4.155455in}}{\pgfqpoint{2.279412in}{2.004545in}}%
\pgfusepath{clip}%
\pgfsetbuttcap%
\pgfsetroundjoin%
\pgfsetlinewidth{0.628460pt}%
\definecolor{currentstroke}{rgb}{0.260571,0.246922,0.522828}%
\pgfsetstrokecolor{currentstroke}%
\pgfsetdash{}{0pt}%
\pgfpathmoveto{\pgfqpoint{2.759172in}{5.031320in}}%
\pgfpathlineto{\pgfqpoint{2.709054in}{5.032949in}}%
\pgfusepath{stroke}%
\end{pgfscope}%
\begin{pgfscope}%
\pgfpathrectangle{\pgfqpoint{1.250000in}{4.155455in}}{\pgfqpoint{2.279412in}{2.004545in}}%
\pgfusepath{clip}%
\pgfsetbuttcap%
\pgfsetroundjoin%
\pgfsetlinewidth{0.719122pt}%
\definecolor{currentstroke}{rgb}{0.237441,0.305202,0.541921}%
\pgfsetstrokecolor{currentstroke}%
\pgfsetdash{}{0pt}%
\pgfpathmoveto{\pgfqpoint{2.709054in}{5.032949in}}%
\pgfpathlineto{\pgfqpoint{2.658941in}{5.034681in}}%
\pgfusepath{stroke}%
\end{pgfscope}%
\begin{pgfscope}%
\pgfpathrectangle{\pgfqpoint{1.250000in}{4.155455in}}{\pgfqpoint{2.279412in}{2.004545in}}%
\pgfusepath{clip}%
\pgfsetbuttcap%
\pgfsetroundjoin%
\pgfsetlinewidth{0.747712pt}%
\definecolor{currentstroke}{rgb}{0.227802,0.326594,0.546532}%
\pgfsetstrokecolor{currentstroke}%
\pgfsetdash{}{0pt}%
\pgfpathmoveto{\pgfqpoint{2.658941in}{5.034681in}}%
\pgfpathlineto{\pgfqpoint{2.608837in}{5.036575in}}%
\pgfusepath{stroke}%
\end{pgfscope}%
\begin{pgfscope}%
\pgfpathrectangle{\pgfqpoint{1.250000in}{4.155455in}}{\pgfqpoint{2.279412in}{2.004545in}}%
\pgfusepath{clip}%
\pgfsetbuttcap%
\pgfsetroundjoin%
\pgfsetlinewidth{0.814573pt}%
\definecolor{currentstroke}{rgb}{0.210503,0.363727,0.552206}%
\pgfsetstrokecolor{currentstroke}%
\pgfsetdash{}{0pt}%
\pgfpathmoveto{\pgfqpoint{2.608837in}{5.036575in}}%
\pgfpathlineto{\pgfqpoint{2.558747in}{5.038757in}}%
\pgfusepath{stroke}%
\end{pgfscope}%
\begin{pgfscope}%
\pgfpathrectangle{\pgfqpoint{1.250000in}{4.155455in}}{\pgfqpoint{2.279412in}{2.004545in}}%
\pgfusepath{clip}%
\pgfsetbuttcap%
\pgfsetroundjoin%
\pgfsetlinewidth{0.890641pt}%
\definecolor{currentstroke}{rgb}{0.190631,0.407061,0.556089}%
\pgfsetstrokecolor{currentstroke}%
\pgfsetdash{}{0pt}%
\pgfpathmoveto{\pgfqpoint{2.558747in}{5.038757in}}%
\pgfpathlineto{\pgfqpoint{2.508675in}{5.041241in}}%
\pgfusepath{stroke}%
\end{pgfscope}%
\begin{pgfscope}%
\pgfpathrectangle{\pgfqpoint{1.250000in}{4.155455in}}{\pgfqpoint{2.279412in}{2.004545in}}%
\pgfusepath{clip}%
\pgfsetbuttcap%
\pgfsetroundjoin%
\pgfsetlinewidth{0.902870pt}%
\definecolor{currentstroke}{rgb}{0.187231,0.414746,0.556547}%
\pgfsetstrokecolor{currentstroke}%
\pgfsetdash{}{0pt}%
\pgfpathmoveto{\pgfqpoint{2.508675in}{5.041241in}}%
\pgfpathlineto{\pgfqpoint{2.458628in}{5.044070in}}%
\pgfusepath{stroke}%
\end{pgfscope}%
\begin{pgfscope}%
\pgfpathrectangle{\pgfqpoint{1.250000in}{4.155455in}}{\pgfqpoint{2.279412in}{2.004545in}}%
\pgfusepath{clip}%
\pgfsetbuttcap%
\pgfsetroundjoin%
\pgfsetlinewidth{0.321240pt}%
\definecolor{currentstroke}{rgb}{0.269944,0.014625,0.341379}%
\pgfsetstrokecolor{currentstroke}%
\pgfsetdash{}{0pt}%
\pgfpathmoveto{\pgfqpoint{3.210376in}{5.247941in}}%
\pgfpathlineto{\pgfqpoint{3.160442in}{5.248879in}}%
\pgfusepath{stroke}%
\end{pgfscope}%
\begin{pgfscope}%
\pgfpathrectangle{\pgfqpoint{1.250000in}{4.155455in}}{\pgfqpoint{2.279412in}{2.004545in}}%
\pgfusepath{clip}%
\pgfsetbuttcap%
\pgfsetroundjoin%
\pgfsetlinewidth{0.323699pt}%
\definecolor{currentstroke}{rgb}{0.271305,0.019942,0.347269}%
\pgfsetstrokecolor{currentstroke}%
\pgfsetdash{}{0pt}%
\pgfpathmoveto{\pgfqpoint{3.160442in}{5.248879in}}%
\pgfpathlineto{\pgfqpoint{3.110371in}{5.247779in}}%
\pgfusepath{stroke}%
\end{pgfscope}%
\begin{pgfscope}%
\pgfpathrectangle{\pgfqpoint{1.250000in}{4.155455in}}{\pgfqpoint{2.279412in}{2.004545in}}%
\pgfusepath{clip}%
\pgfsetbuttcap%
\pgfsetroundjoin%
\pgfsetlinewidth{0.342306pt}%
\definecolor{currentstroke}{rgb}{0.274952,0.037752,0.364543}%
\pgfsetstrokecolor{currentstroke}%
\pgfsetdash{}{0pt}%
\pgfpathmoveto{\pgfqpoint{3.110371in}{5.247779in}}%
\pgfpathlineto{\pgfqpoint{3.060231in}{5.247249in}}%
\pgfusepath{stroke}%
\end{pgfscope}%
\begin{pgfscope}%
\pgfpathrectangle{\pgfqpoint{1.250000in}{4.155455in}}{\pgfqpoint{2.279412in}{2.004545in}}%
\pgfusepath{clip}%
\pgfsetbuttcap%
\pgfsetroundjoin%
\pgfsetlinewidth{0.353022pt}%
\definecolor{currentstroke}{rgb}{0.276022,0.044167,0.370164}%
\pgfsetstrokecolor{currentstroke}%
\pgfsetdash{}{0pt}%
\pgfpathmoveto{\pgfqpoint{3.060231in}{5.247249in}}%
\pgfpathlineto{\pgfqpoint{3.010088in}{5.246805in}}%
\pgfusepath{stroke}%
\end{pgfscope}%
\begin{pgfscope}%
\pgfpathrectangle{\pgfqpoint{1.250000in}{4.155455in}}{\pgfqpoint{2.279412in}{2.004545in}}%
\pgfusepath{clip}%
\pgfsetbuttcap%
\pgfsetroundjoin%
\pgfsetlinewidth{0.368456pt}%
\definecolor{currentstroke}{rgb}{0.277941,0.056324,0.381191}%
\pgfsetstrokecolor{currentstroke}%
\pgfsetdash{}{0pt}%
\pgfpathmoveto{\pgfqpoint{3.010088in}{5.246805in}}%
\pgfpathlineto{\pgfqpoint{2.959945in}{5.246061in}}%
\pgfusepath{stroke}%
\end{pgfscope}%
\begin{pgfscope}%
\pgfpathrectangle{\pgfqpoint{1.250000in}{4.155455in}}{\pgfqpoint{2.279412in}{2.004545in}}%
\pgfusepath{clip}%
\pgfsetbuttcap%
\pgfsetroundjoin%
\pgfsetlinewidth{0.408726pt}%
\definecolor{currentstroke}{rgb}{0.281924,0.089666,0.412415}%
\pgfsetstrokecolor{currentstroke}%
\pgfsetdash{}{0pt}%
\pgfpathmoveto{\pgfqpoint{2.959945in}{5.246061in}}%
\pgfpathlineto{\pgfqpoint{2.909806in}{5.245056in}}%
\pgfusepath{stroke}%
\end{pgfscope}%
\begin{pgfscope}%
\pgfpathrectangle{\pgfqpoint{1.250000in}{4.155455in}}{\pgfqpoint{2.279412in}{2.004545in}}%
\pgfusepath{clip}%
\pgfsetbuttcap%
\pgfsetroundjoin%
\pgfsetlinewidth{0.442135pt}%
\definecolor{currentstroke}{rgb}{0.283197,0.115680,0.436115}%
\pgfsetstrokecolor{currentstroke}%
\pgfsetdash{}{0pt}%
\pgfpathmoveto{\pgfqpoint{2.909806in}{5.245056in}}%
\pgfpathlineto{\pgfqpoint{2.859666in}{5.244118in}}%
\pgfusepath{stroke}%
\end{pgfscope}%
\begin{pgfscope}%
\pgfpathrectangle{\pgfqpoint{1.250000in}{4.155455in}}{\pgfqpoint{2.279412in}{2.004545in}}%
\pgfusepath{clip}%
\pgfsetbuttcap%
\pgfsetroundjoin%
\pgfsetlinewidth{0.509446pt}%
\definecolor{currentstroke}{rgb}{0.280255,0.165693,0.476498}%
\pgfsetstrokecolor{currentstroke}%
\pgfsetdash{}{0pt}%
\pgfpathmoveto{\pgfqpoint{2.859666in}{5.244118in}}%
\pgfpathlineto{\pgfqpoint{2.809528in}{5.243075in}}%
\pgfusepath{stroke}%
\end{pgfscope}%
\begin{pgfscope}%
\pgfpathrectangle{\pgfqpoint{1.250000in}{4.155455in}}{\pgfqpoint{2.279412in}{2.004545in}}%
\pgfusepath{clip}%
\pgfsetbuttcap%
\pgfsetroundjoin%
\pgfsetlinewidth{0.561379pt}%
\definecolor{currentstroke}{rgb}{0.274128,0.199721,0.498911}%
\pgfsetstrokecolor{currentstroke}%
\pgfsetdash{}{0pt}%
\pgfpathmoveto{\pgfqpoint{2.809528in}{5.243075in}}%
\pgfpathlineto{\pgfqpoint{2.759393in}{5.241929in}}%
\pgfusepath{stroke}%
\end{pgfscope}%
\begin{pgfscope}%
\pgfpathrectangle{\pgfqpoint{1.250000in}{4.155455in}}{\pgfqpoint{2.279412in}{2.004545in}}%
\pgfusepath{clip}%
\pgfsetbuttcap%
\pgfsetroundjoin%
\pgfsetlinewidth{0.634518pt}%
\definecolor{currentstroke}{rgb}{0.258965,0.251537,0.524736}%
\pgfsetstrokecolor{currentstroke}%
\pgfsetdash{}{0pt}%
\pgfpathmoveto{\pgfqpoint{2.759393in}{5.241929in}}%
\pgfpathlineto{\pgfqpoint{2.709260in}{5.240746in}}%
\pgfusepath{stroke}%
\end{pgfscope}%
\begin{pgfscope}%
\pgfpathrectangle{\pgfqpoint{1.250000in}{4.155455in}}{\pgfqpoint{2.279412in}{2.004545in}}%
\pgfusepath{clip}%
\pgfsetbuttcap%
\pgfsetroundjoin%
\pgfsetlinewidth{0.703703pt}%
\definecolor{currentstroke}{rgb}{0.241237,0.296485,0.539709}%
\pgfsetstrokecolor{currentstroke}%
\pgfsetdash{}{0pt}%
\pgfpathmoveto{\pgfqpoint{2.709260in}{5.240746in}}%
\pgfpathlineto{\pgfqpoint{2.659130in}{5.239424in}}%
\pgfusepath{stroke}%
\end{pgfscope}%
\begin{pgfscope}%
\pgfpathrectangle{\pgfqpoint{1.250000in}{4.155455in}}{\pgfqpoint{2.279412in}{2.004545in}}%
\pgfusepath{clip}%
\pgfsetbuttcap%
\pgfsetroundjoin%
\pgfsetlinewidth{0.812220pt}%
\definecolor{currentstroke}{rgb}{0.210503,0.363727,0.552206}%
\pgfsetstrokecolor{currentstroke}%
\pgfsetdash{}{0pt}%
\pgfpathmoveto{\pgfqpoint{2.659130in}{5.239424in}}%
\pgfpathlineto{\pgfqpoint{2.609010in}{5.237881in}}%
\pgfusepath{stroke}%
\end{pgfscope}%
\begin{pgfscope}%
\pgfpathrectangle{\pgfqpoint{1.250000in}{4.155455in}}{\pgfqpoint{2.279412in}{2.004545in}}%
\pgfusepath{clip}%
\pgfsetbuttcap%
\pgfsetroundjoin%
\pgfsetlinewidth{0.864620pt}%
\definecolor{currentstroke}{rgb}{0.195860,0.395433,0.555276}%
\pgfsetstrokecolor{currentstroke}%
\pgfsetdash{}{0pt}%
\pgfpathmoveto{\pgfqpoint{2.609010in}{5.237881in}}%
\pgfpathlineto{\pgfqpoint{2.558898in}{5.236112in}}%
\pgfusepath{stroke}%
\end{pgfscope}%
\begin{pgfscope}%
\pgfpathrectangle{\pgfqpoint{1.250000in}{4.155455in}}{\pgfqpoint{2.279412in}{2.004545in}}%
\pgfusepath{clip}%
\pgfsetbuttcap%
\pgfsetroundjoin%
\pgfsetlinewidth{0.897347pt}%
\definecolor{currentstroke}{rgb}{0.188923,0.410910,0.556326}%
\pgfsetstrokecolor{currentstroke}%
\pgfsetdash{}{0pt}%
\pgfpathmoveto{\pgfqpoint{2.558898in}{5.236112in}}%
\pgfpathlineto{\pgfqpoint{2.508797in}{5.234134in}}%
\pgfusepath{stroke}%
\end{pgfscope}%
\begin{pgfscope}%
\pgfpathrectangle{\pgfqpoint{1.250000in}{4.155455in}}{\pgfqpoint{2.279412in}{2.004545in}}%
\pgfusepath{clip}%
\pgfsetbuttcap%
\pgfsetroundjoin%
\pgfsetlinewidth{0.320788pt}%
\definecolor{currentstroke}{rgb}{0.269944,0.014625,0.341379}%
\pgfsetstrokecolor{currentstroke}%
\pgfsetdash{}{0pt}%
\pgfpathmoveto{\pgfqpoint{3.210376in}{5.383261in}}%
\pgfpathlineto{\pgfqpoint{3.160274in}{5.382614in}}%
\pgfusepath{stroke}%
\end{pgfscope}%
\begin{pgfscope}%
\pgfpathrectangle{\pgfqpoint{1.250000in}{4.155455in}}{\pgfqpoint{2.279412in}{2.004545in}}%
\pgfusepath{clip}%
\pgfsetbuttcap%
\pgfsetroundjoin%
\pgfsetlinewidth{0.324309pt}%
\definecolor{currentstroke}{rgb}{0.271305,0.019942,0.347269}%
\pgfsetstrokecolor{currentstroke}%
\pgfsetdash{}{0pt}%
\pgfpathmoveto{\pgfqpoint{3.160274in}{5.382614in}}%
\pgfpathlineto{\pgfqpoint{3.110192in}{5.380481in}}%
\pgfusepath{stroke}%
\end{pgfscope}%
\begin{pgfscope}%
\pgfpathrectangle{\pgfqpoint{1.250000in}{4.155455in}}{\pgfqpoint{2.279412in}{2.004545in}}%
\pgfusepath{clip}%
\pgfsetbuttcap%
\pgfsetroundjoin%
\pgfsetlinewidth{0.329792pt}%
\definecolor{currentstroke}{rgb}{0.272594,0.025563,0.353093}%
\pgfsetstrokecolor{currentstroke}%
\pgfsetdash{}{0pt}%
\pgfpathmoveto{\pgfqpoint{3.110192in}{5.380481in}}%
\pgfpathlineto{\pgfqpoint{3.060106in}{5.378599in}}%
\pgfusepath{stroke}%
\end{pgfscope}%
\begin{pgfscope}%
\pgfpathrectangle{\pgfqpoint{1.250000in}{4.155455in}}{\pgfqpoint{2.279412in}{2.004545in}}%
\pgfusepath{clip}%
\pgfsetbuttcap%
\pgfsetroundjoin%
\pgfsetlinewidth{0.341741pt}%
\definecolor{currentstroke}{rgb}{0.273809,0.031497,0.358853}%
\pgfsetstrokecolor{currentstroke}%
\pgfsetdash{}{0pt}%
\pgfpathmoveto{\pgfqpoint{3.060106in}{5.378599in}}%
\pgfpathlineto{\pgfqpoint{3.009991in}{5.377207in}}%
\pgfusepath{stroke}%
\end{pgfscope}%
\begin{pgfscope}%
\pgfpathrectangle{\pgfqpoint{1.250000in}{4.155455in}}{\pgfqpoint{2.279412in}{2.004545in}}%
\pgfusepath{clip}%
\pgfsetbuttcap%
\pgfsetroundjoin%
\pgfsetlinewidth{0.352792pt}%
\definecolor{currentstroke}{rgb}{0.276022,0.044167,0.370164}%
\pgfsetstrokecolor{currentstroke}%
\pgfsetdash{}{0pt}%
\pgfpathmoveto{\pgfqpoint{3.009991in}{5.377207in}}%
\pgfpathlineto{\pgfqpoint{2.959904in}{5.375022in}}%
\pgfusepath{stroke}%
\end{pgfscope}%
\begin{pgfscope}%
\pgfpathrectangle{\pgfqpoint{1.250000in}{4.155455in}}{\pgfqpoint{2.279412in}{2.004545in}}%
\pgfusepath{clip}%
\pgfsetbuttcap%
\pgfsetroundjoin%
\pgfsetlinewidth{0.392128pt}%
\definecolor{currentstroke}{rgb}{0.280894,0.078907,0.402329}%
\pgfsetstrokecolor{currentstroke}%
\pgfsetdash{}{0pt}%
\pgfpathmoveto{\pgfqpoint{2.959904in}{5.375022in}}%
\pgfpathlineto{\pgfqpoint{2.909836in}{5.372480in}}%
\pgfusepath{stroke}%
\end{pgfscope}%
\begin{pgfscope}%
\pgfpathrectangle{\pgfqpoint{1.250000in}{4.155455in}}{\pgfqpoint{2.279412in}{2.004545in}}%
\pgfusepath{clip}%
\pgfsetbuttcap%
\pgfsetroundjoin%
\pgfsetlinewidth{0.427931pt}%
\definecolor{currentstroke}{rgb}{0.282910,0.105393,0.426902}%
\pgfsetstrokecolor{currentstroke}%
\pgfsetdash{}{0pt}%
\pgfpathmoveto{\pgfqpoint{2.909836in}{5.372480in}}%
\pgfpathlineto{\pgfqpoint{2.859762in}{5.370030in}}%
\pgfusepath{stroke}%
\end{pgfscope}%
\begin{pgfscope}%
\pgfpathrectangle{\pgfqpoint{1.250000in}{4.155455in}}{\pgfqpoint{2.279412in}{2.004545in}}%
\pgfusepath{clip}%
\pgfsetbuttcap%
\pgfsetroundjoin%
\pgfsetlinewidth{0.460156pt}%
\definecolor{currentstroke}{rgb}{0.283072,0.130895,0.449241}%
\pgfsetstrokecolor{currentstroke}%
\pgfsetdash{}{0pt}%
\pgfpathmoveto{\pgfqpoint{2.859762in}{5.370030in}}%
\pgfpathlineto{\pgfqpoint{2.809692in}{5.367528in}}%
\pgfusepath{stroke}%
\end{pgfscope}%
\begin{pgfscope}%
\pgfpathrectangle{\pgfqpoint{1.250000in}{4.155455in}}{\pgfqpoint{2.279412in}{2.004545in}}%
\pgfusepath{clip}%
\pgfsetbuttcap%
\pgfsetroundjoin%
\pgfsetlinewidth{0.509038pt}%
\definecolor{currentstroke}{rgb}{0.280255,0.165693,0.476498}%
\pgfsetstrokecolor{currentstroke}%
\pgfsetdash{}{0pt}%
\pgfpathmoveto{\pgfqpoint{2.809692in}{5.367528in}}%
\pgfpathlineto{\pgfqpoint{2.759629in}{5.364913in}}%
\pgfusepath{stroke}%
\end{pgfscope}%
\begin{pgfscope}%
\pgfpathrectangle{\pgfqpoint{1.250000in}{4.155455in}}{\pgfqpoint{2.279412in}{2.004545in}}%
\pgfusepath{clip}%
\pgfsetbuttcap%
\pgfsetroundjoin%
\pgfsetlinewidth{0.548112pt}%
\definecolor{currentstroke}{rgb}{0.276194,0.190074,0.493001}%
\pgfsetstrokecolor{currentstroke}%
\pgfsetdash{}{0pt}%
\pgfpathmoveto{\pgfqpoint{2.759629in}{5.364913in}}%
\pgfpathlineto{\pgfqpoint{2.709579in}{5.362100in}}%
\pgfusepath{stroke}%
\end{pgfscope}%
\begin{pgfscope}%
\pgfpathrectangle{\pgfqpoint{1.250000in}{4.155455in}}{\pgfqpoint{2.279412in}{2.004545in}}%
\pgfusepath{clip}%
\pgfsetbuttcap%
\pgfsetroundjoin%
\pgfsetlinewidth{0.625166pt}%
\definecolor{currentstroke}{rgb}{0.260571,0.246922,0.522828}%
\pgfsetstrokecolor{currentstroke}%
\pgfsetdash{}{0pt}%
\pgfpathmoveto{\pgfqpoint{2.709579in}{5.362100in}}%
\pgfpathlineto{\pgfqpoint{2.659546in}{5.359075in}}%
\pgfusepath{stroke}%
\end{pgfscope}%
\begin{pgfscope}%
\pgfpathrectangle{\pgfqpoint{1.250000in}{4.155455in}}{\pgfqpoint{2.279412in}{2.004545in}}%
\pgfusepath{clip}%
\pgfsetbuttcap%
\pgfsetroundjoin%
\pgfsetlinewidth{0.676486pt}%
\definecolor{currentstroke}{rgb}{0.248629,0.278775,0.534556}%
\pgfsetstrokecolor{currentstroke}%
\pgfsetdash{}{0pt}%
\pgfpathmoveto{\pgfqpoint{2.659546in}{5.359075in}}%
\pgfpathlineto{\pgfqpoint{2.609542in}{5.355694in}}%
\pgfusepath{stroke}%
\end{pgfscope}%
\begin{pgfscope}%
\pgfpathrectangle{\pgfqpoint{1.250000in}{4.155455in}}{\pgfqpoint{2.279412in}{2.004545in}}%
\pgfusepath{clip}%
\pgfsetbuttcap%
\pgfsetroundjoin%
\pgfsetlinewidth{0.742192pt}%
\definecolor{currentstroke}{rgb}{0.229739,0.322361,0.545706}%
\pgfsetstrokecolor{currentstroke}%
\pgfsetdash{}{0pt}%
\pgfpathmoveto{\pgfqpoint{2.609542in}{5.355694in}}%
\pgfpathlineto{\pgfqpoint{2.559586in}{5.351811in}}%
\pgfusepath{stroke}%
\end{pgfscope}%
\begin{pgfscope}%
\pgfpathrectangle{\pgfqpoint{1.250000in}{4.155455in}}{\pgfqpoint{2.279412in}{2.004545in}}%
\pgfusepath{clip}%
\pgfsetbuttcap%
\pgfsetroundjoin%
\pgfsetlinewidth{0.761618pt}%
\definecolor{currentstroke}{rgb}{0.223925,0.334994,0.548053}%
\pgfsetstrokecolor{currentstroke}%
\pgfsetdash{}{0pt}%
\pgfpathmoveto{\pgfqpoint{2.559586in}{5.351811in}}%
\pgfpathlineto{\pgfqpoint{2.509683in}{5.347442in}}%
\pgfusepath{stroke}%
\end{pgfscope}%
\begin{pgfscope}%
\pgfpathrectangle{\pgfqpoint{1.250000in}{4.155455in}}{\pgfqpoint{2.279412in}{2.004545in}}%
\pgfusepath{clip}%
\pgfsetbuttcap%
\pgfsetroundjoin%
\pgfsetlinewidth{0.817042pt}%
\definecolor{currentstroke}{rgb}{0.208623,0.367752,0.552675}%
\pgfsetstrokecolor{currentstroke}%
\pgfsetdash{}{0pt}%
\pgfpathmoveto{\pgfqpoint{2.509683in}{5.347442in}}%
\pgfpathlineto{\pgfqpoint{2.459854in}{5.342463in}}%
\pgfusepath{stroke}%
\end{pgfscope}%
\begin{pgfscope}%
\pgfpathrectangle{\pgfqpoint{1.250000in}{4.155455in}}{\pgfqpoint{2.279412in}{2.004545in}}%
\pgfusepath{clip}%
\pgfsetbuttcap%
\pgfsetroundjoin%
\pgfsetlinewidth{0.801558pt}%
\definecolor{currentstroke}{rgb}{0.214298,0.355619,0.551184}%
\pgfsetstrokecolor{currentstroke}%
\pgfsetdash{}{0pt}%
\pgfpathmoveto{\pgfqpoint{2.459854in}{5.342463in}}%
\pgfpathlineto{\pgfqpoint{2.410158in}{5.336563in}}%
\pgfusepath{stroke}%
\end{pgfscope}%
\begin{pgfscope}%
\pgfpathrectangle{\pgfqpoint{1.250000in}{4.155455in}}{\pgfqpoint{2.279412in}{2.004545in}}%
\pgfusepath{clip}%
\pgfsetbuttcap%
\pgfsetroundjoin%
\pgfsetlinewidth{0.793496pt}%
\definecolor{currentstroke}{rgb}{0.216210,0.351535,0.550627}%
\pgfsetstrokecolor{currentstroke}%
\pgfsetdash{}{0pt}%
\pgfpathmoveto{\pgfqpoint{2.410158in}{5.336563in}}%
\pgfpathlineto{\pgfqpoint{2.360689in}{5.329371in}}%
\pgfusepath{stroke}%
\end{pgfscope}%
\begin{pgfscope}%
\pgfpathrectangle{\pgfqpoint{1.250000in}{4.155455in}}{\pgfqpoint{2.279412in}{2.004545in}}%
\pgfusepath{clip}%
\pgfsetbuttcap%
\pgfsetroundjoin%
\pgfsetlinewidth{0.802958pt}%
\definecolor{currentstroke}{rgb}{0.212395,0.359683,0.551710}%
\pgfsetstrokecolor{currentstroke}%
\pgfsetdash{}{0pt}%
\pgfpathmoveto{\pgfqpoint{2.360689in}{5.329371in}}%
\pgfpathlineto{\pgfqpoint{2.311621in}{5.320348in}}%
\pgfusepath{stroke}%
\end{pgfscope}%
\begin{pgfscope}%
\pgfpathrectangle{\pgfqpoint{1.250000in}{4.155455in}}{\pgfqpoint{2.279412in}{2.004545in}}%
\pgfusepath{clip}%
\pgfsetbuttcap%
\pgfsetroundjoin%
\pgfsetlinewidth{0.728022pt}%
\definecolor{currentstroke}{rgb}{0.233603,0.313828,0.543914}%
\pgfsetstrokecolor{currentstroke}%
\pgfsetdash{}{0pt}%
\pgfpathmoveto{\pgfqpoint{2.311621in}{5.320348in}}%
\pgfpathlineto{\pgfqpoint{2.263424in}{5.308414in}}%
\pgfusepath{stroke}%
\end{pgfscope}%
\begin{pgfscope}%
\pgfpathrectangle{\pgfqpoint{1.250000in}{4.155455in}}{\pgfqpoint{2.279412in}{2.004545in}}%
\pgfusepath{clip}%
\pgfsetbuttcap%
\pgfsetroundjoin%
\pgfsetlinewidth{0.657693pt}%
\definecolor{currentstroke}{rgb}{0.253935,0.265254,0.529983}%
\pgfsetstrokecolor{currentstroke}%
\pgfsetdash{}{0pt}%
\pgfpathmoveto{\pgfqpoint{2.263424in}{5.308414in}}%
\pgfpathlineto{\pgfqpoint{2.217704in}{5.290984in}}%
\pgfusepath{stroke}%
\end{pgfscope}%
\begin{pgfscope}%
\pgfpathrectangle{\pgfqpoint{1.250000in}{4.155455in}}{\pgfqpoint{2.279412in}{2.004545in}}%
\pgfusepath{clip}%
\pgfsetbuttcap%
\pgfsetroundjoin%
\pgfsetlinewidth{0.567064pt}%
\definecolor{currentstroke}{rgb}{0.273006,0.204520,0.501721}%
\pgfsetstrokecolor{currentstroke}%
\pgfsetdash{}{0pt}%
\pgfpathmoveto{\pgfqpoint{2.217704in}{5.290984in}}%
\pgfpathlineto{\pgfqpoint{2.217704in}{5.290984in}}%
\pgfusepath{stroke}%
\end{pgfscope}%
\begin{pgfscope}%
\pgfpathrectangle{\pgfqpoint{1.250000in}{4.155455in}}{\pgfqpoint{2.279412in}{2.004545in}}%
\pgfusepath{clip}%
\pgfsetbuttcap%
\pgfsetroundjoin%
\pgfsetlinewidth{0.567064pt}%
\definecolor{currentstroke}{rgb}{0.273006,0.204520,0.501721}%
\pgfsetstrokecolor{currentstroke}%
\pgfsetdash{}{0pt}%
\pgfpathmoveto{\pgfqpoint{2.217704in}{5.290984in}}%
\pgfpathlineto{\pgfqpoint{2.182527in}{5.271342in}}%
\pgfusepath{stroke}%
\end{pgfscope}%
\begin{pgfscope}%
\pgfpathrectangle{\pgfqpoint{1.250000in}{4.155455in}}{\pgfqpoint{2.279412in}{2.004545in}}%
\pgfusepath{clip}%
\pgfsetbuttcap%
\pgfsetroundjoin%
\pgfsetlinewidth{0.548898pt}%
\definecolor{currentstroke}{rgb}{0.275191,0.194905,0.496005}%
\pgfsetstrokecolor{currentstroke}%
\pgfsetdash{}{0pt}%
\pgfpathmoveto{\pgfqpoint{2.182527in}{5.271342in}}%
\pgfpathlineto{\pgfqpoint{2.182527in}{5.271342in}}%
\pgfusepath{stroke}%
\end{pgfscope}%
\begin{pgfscope}%
\pgfpathrectangle{\pgfqpoint{1.250000in}{4.155455in}}{\pgfqpoint{2.279412in}{2.004545in}}%
\pgfusepath{clip}%
\pgfsetbuttcap%
\pgfsetroundjoin%
\pgfsetlinewidth{0.323032pt}%
\definecolor{currentstroke}{rgb}{0.271305,0.019942,0.347269}%
\pgfsetstrokecolor{currentstroke}%
\pgfsetdash{}{0pt}%
\pgfpathmoveto{\pgfqpoint{3.159084in}{4.751766in}}%
\pgfpathlineto{\pgfqpoint{3.109053in}{4.754152in}}%
\pgfusepath{stroke}%
\end{pgfscope}%
\begin{pgfscope}%
\pgfpathrectangle{\pgfqpoint{1.250000in}{4.155455in}}{\pgfqpoint{2.279412in}{2.004545in}}%
\pgfusepath{clip}%
\pgfsetbuttcap%
\pgfsetroundjoin%
\pgfsetlinewidth{0.326368pt}%
\definecolor{currentstroke}{rgb}{0.271305,0.019942,0.347269}%
\pgfsetstrokecolor{currentstroke}%
\pgfsetdash{}{0pt}%
\pgfpathmoveto{\pgfqpoint{3.109053in}{4.754152in}}%
\pgfpathlineto{\pgfqpoint{3.059015in}{4.756737in}}%
\pgfusepath{stroke}%
\end{pgfscope}%
\begin{pgfscope}%
\pgfpathrectangle{\pgfqpoint{1.250000in}{4.155455in}}{\pgfqpoint{2.279412in}{2.004545in}}%
\pgfusepath{clip}%
\pgfsetbuttcap%
\pgfsetroundjoin%
\pgfsetlinewidth{0.329019pt}%
\definecolor{currentstroke}{rgb}{0.272594,0.025563,0.353093}%
\pgfsetstrokecolor{currentstroke}%
\pgfsetdash{}{0pt}%
\pgfpathmoveto{\pgfqpoint{3.059015in}{4.756737in}}%
\pgfpathlineto{\pgfqpoint{3.008950in}{4.759114in}}%
\pgfusepath{stroke}%
\end{pgfscope}%
\begin{pgfscope}%
\pgfpathrectangle{\pgfqpoint{1.250000in}{4.155455in}}{\pgfqpoint{2.279412in}{2.004545in}}%
\pgfusepath{clip}%
\pgfsetbuttcap%
\pgfsetroundjoin%
\pgfsetlinewidth{0.341983pt}%
\definecolor{currentstroke}{rgb}{0.273809,0.031497,0.358853}%
\pgfsetstrokecolor{currentstroke}%
\pgfsetdash{}{0pt}%
\pgfpathmoveto{\pgfqpoint{3.008950in}{4.759114in}}%
\pgfpathlineto{\pgfqpoint{2.958877in}{4.761503in}}%
\pgfusepath{stroke}%
\end{pgfscope}%
\begin{pgfscope}%
\pgfpathrectangle{\pgfqpoint{1.250000in}{4.155455in}}{\pgfqpoint{2.279412in}{2.004545in}}%
\pgfusepath{clip}%
\pgfsetbuttcap%
\pgfsetroundjoin%
\pgfsetlinewidth{0.345691pt}%
\definecolor{currentstroke}{rgb}{0.274952,0.037752,0.364543}%
\pgfsetstrokecolor{currentstroke}%
\pgfsetdash{}{0pt}%
\pgfpathmoveto{\pgfqpoint{2.958877in}{4.761503in}}%
\pgfpathlineto{\pgfqpoint{2.908803in}{4.763888in}}%
\pgfusepath{stroke}%
\end{pgfscope}%
\begin{pgfscope}%
\pgfpathrectangle{\pgfqpoint{1.250000in}{4.155455in}}{\pgfqpoint{2.279412in}{2.004545in}}%
\pgfusepath{clip}%
\pgfsetbuttcap%
\pgfsetroundjoin%
\pgfsetlinewidth{0.360770pt}%
\definecolor{currentstroke}{rgb}{0.277018,0.050344,0.375715}%
\pgfsetstrokecolor{currentstroke}%
\pgfsetdash{}{0pt}%
\pgfpathmoveto{\pgfqpoint{2.908803in}{4.763888in}}%
\pgfpathlineto{\pgfqpoint{2.858755in}{4.766603in}}%
\pgfusepath{stroke}%
\end{pgfscope}%
\begin{pgfscope}%
\pgfpathrectangle{\pgfqpoint{1.250000in}{4.155455in}}{\pgfqpoint{2.279412in}{2.004545in}}%
\pgfusepath{clip}%
\pgfsetbuttcap%
\pgfsetroundjoin%
\pgfsetlinewidth{0.381301pt}%
\definecolor{currentstroke}{rgb}{0.279566,0.067836,0.391917}%
\pgfsetstrokecolor{currentstroke}%
\pgfsetdash{}{0pt}%
\pgfpathmoveto{\pgfqpoint{2.858755in}{4.766603in}}%
\pgfpathlineto{\pgfqpoint{2.808796in}{4.770419in}}%
\pgfusepath{stroke}%
\end{pgfscope}%
\begin{pgfscope}%
\pgfpathrectangle{\pgfqpoint{1.250000in}{4.155455in}}{\pgfqpoint{2.279412in}{2.004545in}}%
\pgfusepath{clip}%
\pgfsetbuttcap%
\pgfsetroundjoin%
\pgfsetlinewidth{0.414541pt}%
\definecolor{currentstroke}{rgb}{0.282327,0.094955,0.417331}%
\pgfsetstrokecolor{currentstroke}%
\pgfsetdash{}{0pt}%
\pgfpathmoveto{\pgfqpoint{2.808796in}{4.770419in}}%
\pgfpathlineto{\pgfqpoint{2.758868in}{4.774578in}}%
\pgfusepath{stroke}%
\end{pgfscope}%
\begin{pgfscope}%
\pgfpathrectangle{\pgfqpoint{1.250000in}{4.155455in}}{\pgfqpoint{2.279412in}{2.004545in}}%
\pgfusepath{clip}%
\pgfsetbuttcap%
\pgfsetroundjoin%
\pgfsetlinewidth{0.440418pt}%
\definecolor{currentstroke}{rgb}{0.283197,0.115680,0.436115}%
\pgfsetstrokecolor{currentstroke}%
\pgfsetdash{}{0pt}%
\pgfpathmoveto{\pgfqpoint{2.758868in}{4.774578in}}%
\pgfpathlineto{\pgfqpoint{2.709017in}{4.779348in}}%
\pgfusepath{stroke}%
\end{pgfscope}%
\begin{pgfscope}%
\pgfpathrectangle{\pgfqpoint{1.250000in}{4.155455in}}{\pgfqpoint{2.279412in}{2.004545in}}%
\pgfusepath{clip}%
\pgfsetbuttcap%
\pgfsetroundjoin%
\pgfsetlinewidth{0.485036pt}%
\definecolor{currentstroke}{rgb}{0.282290,0.145912,0.461510}%
\pgfsetstrokecolor{currentstroke}%
\pgfsetdash{}{0pt}%
\pgfpathmoveto{\pgfqpoint{2.709017in}{4.779348in}}%
\pgfpathlineto{\pgfqpoint{2.659252in}{4.784808in}}%
\pgfusepath{stroke}%
\end{pgfscope}%
\begin{pgfscope}%
\pgfpathrectangle{\pgfqpoint{1.250000in}{4.155455in}}{\pgfqpoint{2.279412in}{2.004545in}}%
\pgfusepath{clip}%
\pgfsetbuttcap%
\pgfsetroundjoin%
\pgfsetlinewidth{0.499330pt}%
\definecolor{currentstroke}{rgb}{0.281412,0.155834,0.469201}%
\pgfsetstrokecolor{currentstroke}%
\pgfsetdash{}{0pt}%
\pgfpathmoveto{\pgfqpoint{2.659252in}{4.784808in}}%
\pgfpathlineto{\pgfqpoint{2.609538in}{4.790614in}}%
\pgfusepath{stroke}%
\end{pgfscope}%
\begin{pgfscope}%
\pgfpathrectangle{\pgfqpoint{1.250000in}{4.155455in}}{\pgfqpoint{2.279412in}{2.004545in}}%
\pgfusepath{clip}%
\pgfsetbuttcap%
\pgfsetroundjoin%
\pgfsetlinewidth{0.549193pt}%
\definecolor{currentstroke}{rgb}{0.275191,0.194905,0.496005}%
\pgfsetstrokecolor{currentstroke}%
\pgfsetdash{}{0pt}%
\pgfpathmoveto{\pgfqpoint{2.609538in}{4.790614in}}%
\pgfpathlineto{\pgfqpoint{2.559934in}{4.797101in}}%
\pgfusepath{stroke}%
\end{pgfscope}%
\begin{pgfscope}%
\pgfpathrectangle{\pgfqpoint{1.250000in}{4.155455in}}{\pgfqpoint{2.279412in}{2.004545in}}%
\pgfusepath{clip}%
\pgfsetbuttcap%
\pgfsetroundjoin%
\pgfsetlinewidth{0.593245pt}%
\definecolor{currentstroke}{rgb}{0.267968,0.223549,0.512008}%
\pgfsetstrokecolor{currentstroke}%
\pgfsetdash{}{0pt}%
\pgfpathmoveto{\pgfqpoint{2.559934in}{4.797101in}}%
\pgfpathlineto{\pgfqpoint{2.510514in}{4.804568in}}%
\pgfusepath{stroke}%
\end{pgfscope}%
\begin{pgfscope}%
\pgfpathrectangle{\pgfqpoint{1.250000in}{4.155455in}}{\pgfqpoint{2.279412in}{2.004545in}}%
\pgfusepath{clip}%
\pgfsetbuttcap%
\pgfsetroundjoin%
\pgfsetlinewidth{0.579972pt}%
\definecolor{currentstroke}{rgb}{0.270595,0.214069,0.507052}%
\pgfsetstrokecolor{currentstroke}%
\pgfsetdash{}{0pt}%
\pgfpathmoveto{\pgfqpoint{2.510514in}{4.804568in}}%
\pgfpathlineto{\pgfqpoint{2.461457in}{4.813669in}}%
\pgfusepath{stroke}%
\end{pgfscope}%
\begin{pgfscope}%
\pgfpathrectangle{\pgfqpoint{1.250000in}{4.155455in}}{\pgfqpoint{2.279412in}{2.004545in}}%
\pgfusepath{clip}%
\pgfsetbuttcap%
\pgfsetroundjoin%
\pgfsetlinewidth{0.609228pt}%
\definecolor{currentstroke}{rgb}{0.265145,0.232956,0.516599}%
\pgfsetstrokecolor{currentstroke}%
\pgfsetdash{}{0pt}%
\pgfpathmoveto{\pgfqpoint{2.461457in}{4.813669in}}%
\pgfpathlineto{\pgfqpoint{2.412881in}{4.824586in}}%
\pgfusepath{stroke}%
\end{pgfscope}%
\begin{pgfscope}%
\pgfpathrectangle{\pgfqpoint{1.250000in}{4.155455in}}{\pgfqpoint{2.279412in}{2.004545in}}%
\pgfusepath{clip}%
\pgfsetbuttcap%
\pgfsetroundjoin%
\pgfsetlinewidth{0.631326pt}%
\definecolor{currentstroke}{rgb}{0.258965,0.251537,0.524736}%
\pgfsetstrokecolor{currentstroke}%
\pgfsetdash{}{0pt}%
\pgfpathmoveto{\pgfqpoint{2.412881in}{4.824586in}}%
\pgfpathlineto{\pgfqpoint{2.364760in}{4.836953in}}%
\pgfusepath{stroke}%
\end{pgfscope}%
\begin{pgfscope}%
\pgfpathrectangle{\pgfqpoint{1.250000in}{4.155455in}}{\pgfqpoint{2.279412in}{2.004545in}}%
\pgfusepath{clip}%
\pgfsetbuttcap%
\pgfsetroundjoin%
\pgfsetlinewidth{0.629490pt}%
\definecolor{currentstroke}{rgb}{0.260571,0.246922,0.522828}%
\pgfsetstrokecolor{currentstroke}%
\pgfsetdash{}{0pt}%
\pgfpathmoveto{\pgfqpoint{2.364760in}{4.836953in}}%
\pgfpathlineto{\pgfqpoint{2.318023in}{4.852622in}}%
\pgfusepath{stroke}%
\end{pgfscope}%
\begin{pgfscope}%
\pgfpathrectangle{\pgfqpoint{1.250000in}{4.155455in}}{\pgfqpoint{2.279412in}{2.004545in}}%
\pgfusepath{clip}%
\pgfsetbuttcap%
\pgfsetroundjoin%
\pgfsetlinewidth{0.609203pt}%
\definecolor{currentstroke}{rgb}{0.265145,0.232956,0.516599}%
\pgfsetstrokecolor{currentstroke}%
\pgfsetdash{}{0pt}%
\pgfpathmoveto{\pgfqpoint{2.318023in}{4.852622in}}%
\pgfpathlineto{\pgfqpoint{2.273949in}{4.873316in}}%
\pgfusepath{stroke}%
\end{pgfscope}%
\begin{pgfscope}%
\pgfpathrectangle{\pgfqpoint{1.250000in}{4.155455in}}{\pgfqpoint{2.279412in}{2.004545in}}%
\pgfusepath{clip}%
\pgfsetbuttcap%
\pgfsetroundjoin%
\pgfsetlinewidth{0.627607pt}%
\definecolor{currentstroke}{rgb}{0.260571,0.246922,0.522828}%
\pgfsetstrokecolor{currentstroke}%
\pgfsetdash{}{0pt}%
\pgfpathmoveto{\pgfqpoint{2.273949in}{4.873316in}}%
\pgfpathlineto{\pgfqpoint{2.234794in}{4.900176in}}%
\pgfusepath{stroke}%
\end{pgfscope}%
\begin{pgfscope}%
\pgfpathrectangle{\pgfqpoint{1.250000in}{4.155455in}}{\pgfqpoint{2.279412in}{2.004545in}}%
\pgfusepath{clip}%
\pgfsetbuttcap%
\pgfsetroundjoin%
\pgfsetlinewidth{0.594539pt}%
\definecolor{currentstroke}{rgb}{0.267968,0.223549,0.512008}%
\pgfsetstrokecolor{currentstroke}%
\pgfsetdash{}{0pt}%
\pgfpathmoveto{\pgfqpoint{2.234794in}{4.900176in}}%
\pgfpathlineto{\pgfqpoint{2.205311in}{4.928557in}}%
\pgfusepath{stroke}%
\end{pgfscope}%
\begin{pgfscope}%
\pgfpathrectangle{\pgfqpoint{1.250000in}{4.155455in}}{\pgfqpoint{2.279412in}{2.004545in}}%
\pgfusepath{clip}%
\pgfsetbuttcap%
\pgfsetroundjoin%
\pgfsetlinewidth{0.624445pt}%
\definecolor{currentstroke}{rgb}{0.260571,0.246922,0.522828}%
\pgfsetstrokecolor{currentstroke}%
\pgfsetdash{}{0pt}%
\pgfpathmoveto{\pgfqpoint{2.205311in}{4.928557in}}%
\pgfpathlineto{\pgfqpoint{2.205311in}{4.928557in}}%
\pgfusepath{stroke}%
\end{pgfscope}%
\begin{pgfscope}%
\pgfpathrectangle{\pgfqpoint{1.250000in}{4.155455in}}{\pgfqpoint{2.279412in}{2.004545in}}%
\pgfusepath{clip}%
\pgfsetbuttcap%
\pgfsetroundjoin%
\pgfsetlinewidth{0.624445pt}%
\definecolor{currentstroke}{rgb}{0.260571,0.246922,0.522828}%
\pgfsetstrokecolor{currentstroke}%
\pgfsetdash{}{0pt}%
\pgfpathmoveto{\pgfqpoint{2.205311in}{4.928557in}}%
\pgfpathlineto{\pgfqpoint{2.188867in}{4.954081in}}%
\pgfusepath{stroke}%
\end{pgfscope}%
\begin{pgfscope}%
\pgfpathrectangle{\pgfqpoint{1.250000in}{4.155455in}}{\pgfqpoint{2.279412in}{2.004545in}}%
\pgfusepath{clip}%
\pgfsetbuttcap%
\pgfsetroundjoin%
\pgfsetlinewidth{0.588791pt}%
\definecolor{currentstroke}{rgb}{0.269308,0.218818,0.509577}%
\pgfsetstrokecolor{currentstroke}%
\pgfsetdash{}{0pt}%
\pgfpathmoveto{\pgfqpoint{2.188867in}{4.954081in}}%
\pgfpathlineto{\pgfqpoint{2.176716in}{4.978638in}}%
\pgfusepath{stroke}%
\end{pgfscope}%
\begin{pgfscope}%
\pgfpathrectangle{\pgfqpoint{1.250000in}{4.155455in}}{\pgfqpoint{2.279412in}{2.004545in}}%
\pgfusepath{clip}%
\pgfsetbuttcap%
\pgfsetroundjoin%
\pgfsetlinewidth{0.605817pt}%
\definecolor{currentstroke}{rgb}{0.265145,0.232956,0.516599}%
\pgfsetstrokecolor{currentstroke}%
\pgfsetdash{}{0pt}%
\pgfpathmoveto{\pgfqpoint{2.176716in}{4.978638in}}%
\pgfpathlineto{\pgfqpoint{2.176716in}{4.978638in}}%
\pgfusepath{stroke}%
\end{pgfscope}%
\begin{pgfscope}%
\pgfpathrectangle{\pgfqpoint{1.250000in}{4.155455in}}{\pgfqpoint{2.279412in}{2.004545in}}%
\pgfusepath{clip}%
\pgfsetbuttcap%
\pgfsetroundjoin%
\pgfsetlinewidth{0.605817pt}%
\definecolor{currentstroke}{rgb}{0.265145,0.232956,0.516599}%
\pgfsetstrokecolor{currentstroke}%
\pgfsetdash{}{0pt}%
\pgfpathmoveto{\pgfqpoint{2.176716in}{4.978638in}}%
\pgfpathlineto{\pgfqpoint{2.170915in}{5.002117in}}%
\pgfusepath{stroke}%
\end{pgfscope}%
\begin{pgfscope}%
\pgfpathrectangle{\pgfqpoint{1.250000in}{4.155455in}}{\pgfqpoint{2.279412in}{2.004545in}}%
\pgfusepath{clip}%
\pgfsetbuttcap%
\pgfsetroundjoin%
\pgfsetlinewidth{0.538949pt}%
\definecolor{currentstroke}{rgb}{0.277134,0.185228,0.489898}%
\pgfsetstrokecolor{currentstroke}%
\pgfsetdash{}{0pt}%
\pgfpathmoveto{\pgfqpoint{2.170915in}{5.002117in}}%
\pgfpathlineto{\pgfqpoint{2.166014in}{5.024319in}}%
\pgfusepath{stroke}%
\end{pgfscope}%
\begin{pgfscope}%
\pgfpathrectangle{\pgfqpoint{1.250000in}{4.155455in}}{\pgfqpoint{2.279412in}{2.004545in}}%
\pgfusepath{clip}%
\pgfsetbuttcap%
\pgfsetroundjoin%
\pgfsetlinewidth{0.488867pt}%
\definecolor{currentstroke}{rgb}{0.281887,0.150881,0.465405}%
\pgfsetstrokecolor{currentstroke}%
\pgfsetdash{}{0pt}%
\pgfpathmoveto{\pgfqpoint{2.166014in}{5.024319in}}%
\pgfpathlineto{\pgfqpoint{2.154033in}{5.052150in}}%
\pgfusepath{stroke}%
\end{pgfscope}%
\begin{pgfscope}%
\pgfpathrectangle{\pgfqpoint{1.250000in}{4.155455in}}{\pgfqpoint{2.279412in}{2.004545in}}%
\pgfusepath{clip}%
\pgfsetbuttcap%
\pgfsetroundjoin%
\pgfsetlinewidth{0.540836pt}%
\definecolor{currentstroke}{rgb}{0.277134,0.185228,0.489898}%
\pgfsetstrokecolor{currentstroke}%
\pgfsetdash{}{0pt}%
\pgfpathmoveto{\pgfqpoint{2.154033in}{5.052150in}}%
\pgfpathlineto{\pgfqpoint{2.154033in}{5.052150in}}%
\pgfusepath{stroke}%
\end{pgfscope}%
\begin{pgfscope}%
\pgfpathrectangle{\pgfqpoint{1.250000in}{4.155455in}}{\pgfqpoint{2.279412in}{2.004545in}}%
\pgfusepath{clip}%
\pgfsetbuttcap%
\pgfsetroundjoin%
\pgfsetlinewidth{0.540836pt}%
\definecolor{currentstroke}{rgb}{0.277134,0.185228,0.489898}%
\pgfsetstrokecolor{currentstroke}%
\pgfsetdash{}{0pt}%
\pgfpathmoveto{\pgfqpoint{2.154033in}{5.052150in}}%
\pgfpathlineto{\pgfqpoint{2.154033in}{5.052150in}}%
\pgfusepath{stroke}%
\end{pgfscope}%
\begin{pgfscope}%
\pgfpathrectangle{\pgfqpoint{1.250000in}{4.155455in}}{\pgfqpoint{2.279412in}{2.004545in}}%
\pgfusepath{clip}%
\pgfsetbuttcap%
\pgfsetroundjoin%
\pgfsetlinewidth{0.540836pt}%
\definecolor{currentstroke}{rgb}{0.277134,0.185228,0.489898}%
\pgfsetstrokecolor{currentstroke}%
\pgfsetdash{}{0pt}%
\pgfpathmoveto{\pgfqpoint{2.154033in}{5.052150in}}%
\pgfpathlineto{\pgfqpoint{2.150856in}{5.069324in}}%
\pgfusepath{stroke}%
\end{pgfscope}%
\begin{pgfscope}%
\pgfpathrectangle{\pgfqpoint{1.250000in}{4.155455in}}{\pgfqpoint{2.279412in}{2.004545in}}%
\pgfusepath{clip}%
\pgfsetbuttcap%
\pgfsetroundjoin%
\pgfsetlinewidth{0.503411pt}%
\definecolor{currentstroke}{rgb}{0.280868,0.160771,0.472899}%
\pgfsetstrokecolor{currentstroke}%
\pgfsetdash{}{0pt}%
\pgfpathmoveto{\pgfqpoint{2.150856in}{5.069324in}}%
\pgfpathlineto{\pgfqpoint{2.147243in}{5.086513in}}%
\pgfusepath{stroke}%
\end{pgfscope}%
\begin{pgfscope}%
\pgfpathrectangle{\pgfqpoint{1.250000in}{4.155455in}}{\pgfqpoint{2.279412in}{2.004545in}}%
\pgfusepath{clip}%
\pgfsetbuttcap%
\pgfsetroundjoin%
\pgfsetlinewidth{0.485904pt}%
\definecolor{currentstroke}{rgb}{0.282290,0.145912,0.461510}%
\pgfsetstrokecolor{currentstroke}%
\pgfsetdash{}{0pt}%
\pgfpathmoveto{\pgfqpoint{2.147243in}{5.086513in}}%
\pgfpathlineto{\pgfqpoint{2.147243in}{5.086513in}}%
\pgfusepath{stroke}%
\end{pgfscope}%
\begin{pgfscope}%
\pgfpathrectangle{\pgfqpoint{1.250000in}{4.155455in}}{\pgfqpoint{2.279412in}{2.004545in}}%
\pgfusepath{clip}%
\pgfsetbuttcap%
\pgfsetroundjoin%
\pgfsetlinewidth{0.320553pt}%
\definecolor{currentstroke}{rgb}{0.269944,0.014625,0.341379}%
\pgfsetstrokecolor{currentstroke}%
\pgfsetdash{}{0pt}%
\pgfpathmoveto{\pgfqpoint{3.204942in}{4.839585in}}%
\pgfpathlineto{\pgfqpoint{3.159084in}{4.841980in}}%
\pgfusepath{stroke}%
\end{pgfscope}%
\begin{pgfscope}%
\pgfpathrectangle{\pgfqpoint{1.250000in}{4.155455in}}{\pgfqpoint{2.279412in}{2.004545in}}%
\pgfusepath{clip}%
\pgfsetbuttcap%
\pgfsetroundjoin%
\pgfsetlinewidth{0.325248pt}%
\definecolor{currentstroke}{rgb}{0.271305,0.019942,0.347269}%
\pgfsetstrokecolor{currentstroke}%
\pgfsetdash{}{0pt}%
\pgfpathmoveto{\pgfqpoint{3.159084in}{4.841980in}}%
\pgfpathlineto{\pgfqpoint{3.108985in}{4.843554in}}%
\pgfusepath{stroke}%
\end{pgfscope}%
\begin{pgfscope}%
\pgfpathrectangle{\pgfqpoint{1.250000in}{4.155455in}}{\pgfqpoint{2.279412in}{2.004545in}}%
\pgfusepath{clip}%
\pgfsetbuttcap%
\pgfsetroundjoin%
\pgfsetlinewidth{0.327628pt}%
\definecolor{currentstroke}{rgb}{0.271305,0.019942,0.347269}%
\pgfsetstrokecolor{currentstroke}%
\pgfsetdash{}{0pt}%
\pgfpathmoveto{\pgfqpoint{3.108985in}{4.843554in}}%
\pgfpathlineto{\pgfqpoint{3.058865in}{4.844858in}}%
\pgfusepath{stroke}%
\end{pgfscope}%
\begin{pgfscope}%
\pgfpathrectangle{\pgfqpoint{1.250000in}{4.155455in}}{\pgfqpoint{2.279412in}{2.004545in}}%
\pgfusepath{clip}%
\pgfsetbuttcap%
\pgfsetroundjoin%
\pgfsetlinewidth{0.338137pt}%
\definecolor{currentstroke}{rgb}{0.273809,0.031497,0.358853}%
\pgfsetstrokecolor{currentstroke}%
\pgfsetdash{}{0pt}%
\pgfpathmoveto{\pgfqpoint{3.058865in}{4.844858in}}%
\pgfpathlineto{\pgfqpoint{3.008771in}{4.846819in}}%
\pgfusepath{stroke}%
\end{pgfscope}%
\begin{pgfscope}%
\pgfpathrectangle{\pgfqpoint{1.250000in}{4.155455in}}{\pgfqpoint{2.279412in}{2.004545in}}%
\pgfusepath{clip}%
\pgfsetbuttcap%
\pgfsetroundjoin%
\pgfsetlinewidth{0.348726pt}%
\definecolor{currentstroke}{rgb}{0.274952,0.037752,0.364543}%
\pgfsetstrokecolor{currentstroke}%
\pgfsetdash{}{0pt}%
\pgfpathmoveto{\pgfqpoint{3.008771in}{4.846819in}}%
\pgfpathlineto{\pgfqpoint{2.958693in}{4.849081in}}%
\pgfusepath{stroke}%
\end{pgfscope}%
\begin{pgfscope}%
\pgfpathrectangle{\pgfqpoint{1.250000in}{4.155455in}}{\pgfqpoint{2.279412in}{2.004545in}}%
\pgfusepath{clip}%
\pgfsetbuttcap%
\pgfsetroundjoin%
\pgfsetlinewidth{0.382530pt}%
\definecolor{currentstroke}{rgb}{0.279566,0.067836,0.391917}%
\pgfsetstrokecolor{currentstroke}%
\pgfsetdash{}{0pt}%
\pgfpathmoveto{\pgfqpoint{2.958693in}{4.849081in}}%
\pgfpathlineto{\pgfqpoint{2.908663in}{4.852109in}}%
\pgfusepath{stroke}%
\end{pgfscope}%
\begin{pgfscope}%
\pgfpathrectangle{\pgfqpoint{1.250000in}{4.155455in}}{\pgfqpoint{2.279412in}{2.004545in}}%
\pgfusepath{clip}%
\pgfsetbuttcap%
\pgfsetroundjoin%
\pgfsetlinewidth{0.397028pt}%
\definecolor{currentstroke}{rgb}{0.280894,0.078907,0.402329}%
\pgfsetstrokecolor{currentstroke}%
\pgfsetdash{}{0pt}%
\pgfpathmoveto{\pgfqpoint{2.908663in}{4.852109in}}%
\pgfpathlineto{\pgfqpoint{2.858667in}{4.855576in}}%
\pgfusepath{stroke}%
\end{pgfscope}%
\begin{pgfscope}%
\pgfpathrectangle{\pgfqpoint{1.250000in}{4.155455in}}{\pgfqpoint{2.279412in}{2.004545in}}%
\pgfusepath{clip}%
\pgfsetbuttcap%
\pgfsetroundjoin%
\pgfsetlinewidth{0.431225pt}%
\definecolor{currentstroke}{rgb}{0.282910,0.105393,0.426902}%
\pgfsetstrokecolor{currentstroke}%
\pgfsetdash{}{0pt}%
\pgfpathmoveto{\pgfqpoint{2.858667in}{4.855576in}}%
\pgfpathlineto{\pgfqpoint{2.808679in}{4.859135in}}%
\pgfusepath{stroke}%
\end{pgfscope}%
\begin{pgfscope}%
\pgfpathrectangle{\pgfqpoint{1.250000in}{4.155455in}}{\pgfqpoint{2.279412in}{2.004545in}}%
\pgfusepath{clip}%
\pgfsetbuttcap%
\pgfsetroundjoin%
\pgfsetlinewidth{0.453274pt}%
\definecolor{currentstroke}{rgb}{0.283187,0.125848,0.444960}%
\pgfsetstrokecolor{currentstroke}%
\pgfsetdash{}{0pt}%
\pgfpathmoveto{\pgfqpoint{2.808679in}{4.859135in}}%
\pgfpathlineto{\pgfqpoint{2.758691in}{4.862684in}}%
\pgfusepath{stroke}%
\end{pgfscope}%
\begin{pgfscope}%
\pgfpathrectangle{\pgfqpoint{1.250000in}{4.155455in}}{\pgfqpoint{2.279412in}{2.004545in}}%
\pgfusepath{clip}%
\pgfsetbuttcap%
\pgfsetroundjoin%
\pgfsetlinewidth{0.505809pt}%
\definecolor{currentstroke}{rgb}{0.280868,0.160771,0.472899}%
\pgfsetstrokecolor{currentstroke}%
\pgfsetdash{}{0pt}%
\pgfpathmoveto{\pgfqpoint{2.758691in}{4.862684in}}%
\pgfpathlineto{\pgfqpoint{2.708720in}{4.866409in}}%
\pgfusepath{stroke}%
\end{pgfscope}%
\begin{pgfscope}%
\pgfpathrectangle{\pgfqpoint{1.250000in}{4.155455in}}{\pgfqpoint{2.279412in}{2.004545in}}%
\pgfusepath{clip}%
\pgfsetbuttcap%
\pgfsetroundjoin%
\pgfsetlinewidth{0.548635pt}%
\definecolor{currentstroke}{rgb}{0.275191,0.194905,0.496005}%
\pgfsetstrokecolor{currentstroke}%
\pgfsetdash{}{0pt}%
\pgfpathmoveto{\pgfqpoint{2.708720in}{4.866409in}}%
\pgfpathlineto{\pgfqpoint{2.658790in}{4.870540in}}%
\pgfusepath{stroke}%
\end{pgfscope}%
\begin{pgfscope}%
\pgfpathrectangle{\pgfqpoint{1.250000in}{4.155455in}}{\pgfqpoint{2.279412in}{2.004545in}}%
\pgfusepath{clip}%
\pgfsetbuttcap%
\pgfsetroundjoin%
\pgfsetlinewidth{0.592659pt}%
\definecolor{currentstroke}{rgb}{0.267968,0.223549,0.512008}%
\pgfsetstrokecolor{currentstroke}%
\pgfsetdash{}{0pt}%
\pgfpathmoveto{\pgfqpoint{2.658790in}{4.870540in}}%
\pgfpathlineto{\pgfqpoint{2.608905in}{4.875072in}}%
\pgfusepath{stroke}%
\end{pgfscope}%
\begin{pgfscope}%
\pgfpathrectangle{\pgfqpoint{1.250000in}{4.155455in}}{\pgfqpoint{2.279412in}{2.004545in}}%
\pgfusepath{clip}%
\pgfsetbuttcap%
\pgfsetroundjoin%
\pgfsetlinewidth{0.635047pt}%
\definecolor{currentstroke}{rgb}{0.258965,0.251537,0.524736}%
\pgfsetstrokecolor{currentstroke}%
\pgfsetdash{}{0pt}%
\pgfpathmoveto{\pgfqpoint{2.608905in}{4.875072in}}%
\pgfpathlineto{\pgfqpoint{2.559072in}{4.880024in}}%
\pgfusepath{stroke}%
\end{pgfscope}%
\begin{pgfscope}%
\pgfpathrectangle{\pgfqpoint{1.250000in}{4.155455in}}{\pgfqpoint{2.279412in}{2.004545in}}%
\pgfusepath{clip}%
\pgfsetbuttcap%
\pgfsetroundjoin%
\pgfsetlinewidth{0.674378pt}%
\definecolor{currentstroke}{rgb}{0.248629,0.278775,0.534556}%
\pgfsetstrokecolor{currentstroke}%
\pgfsetdash{}{0pt}%
\pgfpathmoveto{\pgfqpoint{2.559072in}{4.880024in}}%
\pgfpathlineto{\pgfqpoint{2.509324in}{4.885592in}}%
\pgfusepath{stroke}%
\end{pgfscope}%
\begin{pgfscope}%
\pgfpathrectangle{\pgfqpoint{1.250000in}{4.155455in}}{\pgfqpoint{2.279412in}{2.004545in}}%
\pgfusepath{clip}%
\pgfsetbuttcap%
\pgfsetroundjoin%
\pgfsetlinewidth{0.686240pt}%
\definecolor{currentstroke}{rgb}{0.244972,0.287675,0.537260}%
\pgfsetstrokecolor{currentstroke}%
\pgfsetdash{}{0pt}%
\pgfpathmoveto{\pgfqpoint{2.509324in}{4.885592in}}%
\pgfpathlineto{\pgfqpoint{2.459764in}{4.892303in}}%
\pgfusepath{stroke}%
\end{pgfscope}%
\begin{pgfscope}%
\pgfpathrectangle{\pgfqpoint{1.250000in}{4.155455in}}{\pgfqpoint{2.279412in}{2.004545in}}%
\pgfusepath{clip}%
\pgfsetbuttcap%
\pgfsetroundjoin%
\pgfsetlinewidth{0.712501pt}%
\definecolor{currentstroke}{rgb}{0.239346,0.300855,0.540844}%
\pgfsetstrokecolor{currentstroke}%
\pgfsetdash{}{0pt}%
\pgfpathmoveto{\pgfqpoint{2.459764in}{4.892303in}}%
\pgfpathlineto{\pgfqpoint{2.410427in}{4.900192in}}%
\pgfusepath{stroke}%
\end{pgfscope}%
\begin{pgfscope}%
\pgfpathrectangle{\pgfqpoint{1.250000in}{4.155455in}}{\pgfqpoint{2.279412in}{2.004545in}}%
\pgfusepath{clip}%
\pgfsetbuttcap%
\pgfsetroundjoin%
\pgfsetlinewidth{0.757959pt}%
\definecolor{currentstroke}{rgb}{0.225863,0.330805,0.547314}%
\pgfsetstrokecolor{currentstroke}%
\pgfsetdash{}{0pt}%
\pgfpathmoveto{\pgfqpoint{2.410427in}{4.900192in}}%
\pgfpathlineto{\pgfqpoint{2.361495in}{4.909765in}}%
\pgfusepath{stroke}%
\end{pgfscope}%
\begin{pgfscope}%
\pgfpathrectangle{\pgfqpoint{1.250000in}{4.155455in}}{\pgfqpoint{2.279412in}{2.004545in}}%
\pgfusepath{clip}%
\pgfsetbuttcap%
\pgfsetroundjoin%
\pgfsetlinewidth{0.708797pt}%
\definecolor{currentstroke}{rgb}{0.239346,0.300855,0.540844}%
\pgfsetstrokecolor{currentstroke}%
\pgfsetdash{}{0pt}%
\pgfpathmoveto{\pgfqpoint{2.361495in}{4.909765in}}%
\pgfpathlineto{\pgfqpoint{2.313425in}{4.922162in}}%
\pgfusepath{stroke}%
\end{pgfscope}%
\begin{pgfscope}%
\pgfpathrectangle{\pgfqpoint{1.250000in}{4.155455in}}{\pgfqpoint{2.279412in}{2.004545in}}%
\pgfusepath{clip}%
\pgfsetbuttcap%
\pgfsetroundjoin%
\pgfsetlinewidth{0.699175pt}%
\definecolor{currentstroke}{rgb}{0.241237,0.296485,0.539709}%
\pgfsetstrokecolor{currentstroke}%
\pgfsetdash{}{0pt}%
\pgfpathmoveto{\pgfqpoint{2.313425in}{4.922162in}}%
\pgfpathlineto{\pgfqpoint{2.266920in}{4.938403in}}%
\pgfusepath{stroke}%
\end{pgfscope}%
\begin{pgfscope}%
\pgfpathrectangle{\pgfqpoint{1.250000in}{4.155455in}}{\pgfqpoint{2.279412in}{2.004545in}}%
\pgfusepath{clip}%
\pgfsetbuttcap%
\pgfsetroundjoin%
\pgfsetlinewidth{0.646964pt}%
\definecolor{currentstroke}{rgb}{0.255645,0.260703,0.528312}%
\pgfsetstrokecolor{currentstroke}%
\pgfsetdash{}{0pt}%
\pgfpathmoveto{\pgfqpoint{2.266920in}{4.938403in}}%
\pgfpathlineto{\pgfqpoint{2.224271in}{4.960585in}}%
\pgfusepath{stroke}%
\end{pgfscope}%
\begin{pgfscope}%
\pgfpathrectangle{\pgfqpoint{1.250000in}{4.155455in}}{\pgfqpoint{2.279412in}{2.004545in}}%
\pgfusepath{clip}%
\pgfsetbuttcap%
\pgfsetroundjoin%
\pgfsetlinewidth{0.601611pt}%
\definecolor{currentstroke}{rgb}{0.266580,0.228262,0.514349}%
\pgfsetstrokecolor{currentstroke}%
\pgfsetdash{}{0pt}%
\pgfpathmoveto{\pgfqpoint{2.224271in}{4.960585in}}%
\pgfpathlineto{\pgfqpoint{2.224271in}{4.960585in}}%
\pgfusepath{stroke}%
\end{pgfscope}%
\begin{pgfscope}%
\pgfpathrectangle{\pgfqpoint{1.250000in}{4.155455in}}{\pgfqpoint{2.279412in}{2.004545in}}%
\pgfusepath{clip}%
\pgfsetbuttcap%
\pgfsetroundjoin%
\pgfsetlinewidth{0.330032pt}%
\definecolor{currentstroke}{rgb}{0.272594,0.025563,0.353093}%
\pgfsetstrokecolor{currentstroke}%
\pgfsetdash{}{0pt}%
\pgfpathmoveto{\pgfqpoint{3.159084in}{4.887087in}}%
\pgfpathlineto{\pgfqpoint{3.108958in}{4.888298in}}%
\pgfusepath{stroke}%
\end{pgfscope}%
\begin{pgfscope}%
\pgfpathrectangle{\pgfqpoint{1.250000in}{4.155455in}}{\pgfqpoint{2.279412in}{2.004545in}}%
\pgfusepath{clip}%
\pgfsetbuttcap%
\pgfsetroundjoin%
\pgfsetlinewidth{0.331108pt}%
\definecolor{currentstroke}{rgb}{0.272594,0.025563,0.353093}%
\pgfsetstrokecolor{currentstroke}%
\pgfsetdash{}{0pt}%
\pgfpathmoveto{\pgfqpoint{3.108958in}{4.888298in}}%
\pgfpathlineto{\pgfqpoint{3.058843in}{4.889946in}}%
\pgfusepath{stroke}%
\end{pgfscope}%
\begin{pgfscope}%
\pgfpathrectangle{\pgfqpoint{1.250000in}{4.155455in}}{\pgfqpoint{2.279412in}{2.004545in}}%
\pgfusepath{clip}%
\pgfsetbuttcap%
\pgfsetroundjoin%
\pgfsetlinewidth{0.343678pt}%
\definecolor{currentstroke}{rgb}{0.274952,0.037752,0.364543}%
\pgfsetstrokecolor{currentstroke}%
\pgfsetdash{}{0pt}%
\pgfpathmoveto{\pgfqpoint{3.058843in}{4.889946in}}%
\pgfpathlineto{\pgfqpoint{3.008757in}{4.892149in}}%
\pgfusepath{stroke}%
\end{pgfscope}%
\begin{pgfscope}%
\pgfpathrectangle{\pgfqpoint{1.250000in}{4.155455in}}{\pgfqpoint{2.279412in}{2.004545in}}%
\pgfusepath{clip}%
\pgfsetbuttcap%
\pgfsetroundjoin%
\pgfsetlinewidth{0.359608pt}%
\definecolor{currentstroke}{rgb}{0.277018,0.050344,0.375715}%
\pgfsetstrokecolor{currentstroke}%
\pgfsetdash{}{0pt}%
\pgfpathmoveto{\pgfqpoint{3.008757in}{4.892149in}}%
\pgfpathlineto{\pgfqpoint{2.958677in}{4.894420in}}%
\pgfusepath{stroke}%
\end{pgfscope}%
\begin{pgfscope}%
\pgfpathrectangle{\pgfqpoint{1.250000in}{4.155455in}}{\pgfqpoint{2.279412in}{2.004545in}}%
\pgfusepath{clip}%
\pgfsetbuttcap%
\pgfsetroundjoin%
\pgfsetlinewidth{0.390832pt}%
\definecolor{currentstroke}{rgb}{0.280894,0.078907,0.402329}%
\pgfsetstrokecolor{currentstroke}%
\pgfsetdash{}{0pt}%
\pgfpathmoveto{\pgfqpoint{2.958677in}{4.894420in}}%
\pgfpathlineto{\pgfqpoint{2.908595in}{4.896662in}}%
\pgfusepath{stroke}%
\end{pgfscope}%
\begin{pgfscope}%
\pgfpathrectangle{\pgfqpoint{1.250000in}{4.155455in}}{\pgfqpoint{2.279412in}{2.004545in}}%
\pgfusepath{clip}%
\pgfsetbuttcap%
\pgfsetroundjoin%
\pgfsetlinewidth{0.402445pt}%
\definecolor{currentstroke}{rgb}{0.281446,0.084320,0.407414}%
\pgfsetstrokecolor{currentstroke}%
\pgfsetdash{}{0pt}%
\pgfpathmoveto{\pgfqpoint{2.908595in}{4.896662in}}%
\pgfpathlineto{\pgfqpoint{2.858520in}{4.899098in}}%
\pgfusepath{stroke}%
\end{pgfscope}%
\begin{pgfscope}%
\pgfpathrectangle{\pgfqpoint{1.250000in}{4.155455in}}{\pgfqpoint{2.279412in}{2.004545in}}%
\pgfusepath{clip}%
\pgfsetbuttcap%
\pgfsetroundjoin%
\pgfsetlinewidth{0.433310pt}%
\definecolor{currentstroke}{rgb}{0.283091,0.110553,0.431554}%
\pgfsetstrokecolor{currentstroke}%
\pgfsetdash{}{0pt}%
\pgfpathmoveto{\pgfqpoint{2.858520in}{4.899098in}}%
\pgfpathlineto{\pgfqpoint{2.808441in}{4.901477in}}%
\pgfusepath{stroke}%
\end{pgfscope}%
\begin{pgfscope}%
\pgfpathrectangle{\pgfqpoint{1.250000in}{4.155455in}}{\pgfqpoint{2.279412in}{2.004545in}}%
\pgfusepath{clip}%
\pgfsetbuttcap%
\pgfsetroundjoin%
\pgfsetlinewidth{0.497023pt}%
\definecolor{currentstroke}{rgb}{0.281412,0.155834,0.469201}%
\pgfsetstrokecolor{currentstroke}%
\pgfsetdash{}{0pt}%
\pgfpathmoveto{\pgfqpoint{2.808441in}{4.901477in}}%
\pgfpathlineto{\pgfqpoint{2.758377in}{4.904084in}}%
\pgfusepath{stroke}%
\end{pgfscope}%
\begin{pgfscope}%
\pgfpathrectangle{\pgfqpoint{1.250000in}{4.155455in}}{\pgfqpoint{2.279412in}{2.004545in}}%
\pgfusepath{clip}%
\pgfsetbuttcap%
\pgfsetroundjoin%
\pgfsetlinewidth{0.314823pt}%
\definecolor{currentstroke}{rgb}{0.268510,0.009605,0.335427}%
\pgfsetstrokecolor{currentstroke}%
\pgfsetdash{}{0pt}%
\pgfpathmoveto{\pgfqpoint{3.159084in}{4.932193in}}%
\pgfpathlineto{\pgfqpoint{3.109409in}{4.935552in}}%
\pgfusepath{stroke}%
\end{pgfscope}%
\begin{pgfscope}%
\pgfpathrectangle{\pgfqpoint{1.250000in}{4.155455in}}{\pgfqpoint{2.279412in}{2.004545in}}%
\pgfusepath{clip}%
\pgfsetbuttcap%
\pgfsetroundjoin%
\pgfsetlinewidth{0.337719pt}%
\definecolor{currentstroke}{rgb}{0.273809,0.031497,0.358853}%
\pgfsetstrokecolor{currentstroke}%
\pgfsetdash{}{0pt}%
\pgfpathmoveto{\pgfqpoint{3.109409in}{4.935552in}}%
\pgfpathlineto{\pgfqpoint{3.059267in}{4.935832in}}%
\pgfusepath{stroke}%
\end{pgfscope}%
\begin{pgfscope}%
\pgfpathrectangle{\pgfqpoint{1.250000in}{4.155455in}}{\pgfqpoint{2.279412in}{2.004545in}}%
\pgfusepath{clip}%
\pgfsetbuttcap%
\pgfsetroundjoin%
\pgfsetlinewidth{0.343544pt}%
\definecolor{currentstroke}{rgb}{0.274952,0.037752,0.364543}%
\pgfsetstrokecolor{currentstroke}%
\pgfsetdash{}{0pt}%
\pgfpathmoveto{\pgfqpoint{3.059267in}{4.935832in}}%
\pgfpathlineto{\pgfqpoint{3.009164in}{4.937105in}}%
\pgfusepath{stroke}%
\end{pgfscope}%
\begin{pgfscope}%
\pgfpathrectangle{\pgfqpoint{1.250000in}{4.155455in}}{\pgfqpoint{2.279412in}{2.004545in}}%
\pgfusepath{clip}%
\pgfsetbuttcap%
\pgfsetroundjoin%
\pgfsetlinewidth{0.361672pt}%
\definecolor{currentstroke}{rgb}{0.277018,0.050344,0.375715}%
\pgfsetstrokecolor{currentstroke}%
\pgfsetdash{}{0pt}%
\pgfpathmoveto{\pgfqpoint{3.009164in}{4.937105in}}%
\pgfpathlineto{\pgfqpoint{2.959079in}{4.939328in}}%
\pgfusepath{stroke}%
\end{pgfscope}%
\begin{pgfscope}%
\pgfpathrectangle{\pgfqpoint{1.250000in}{4.155455in}}{\pgfqpoint{2.279412in}{2.004545in}}%
\pgfusepath{clip}%
\pgfsetbuttcap%
\pgfsetroundjoin%
\pgfsetlinewidth{0.389859pt}%
\definecolor{currentstroke}{rgb}{0.280267,0.073417,0.397163}%
\pgfsetstrokecolor{currentstroke}%
\pgfsetdash{}{0pt}%
\pgfpathmoveto{\pgfqpoint{2.959079in}{4.939328in}}%
\pgfpathlineto{\pgfqpoint{2.908994in}{4.941551in}}%
\pgfusepath{stroke}%
\end{pgfscope}%
\begin{pgfscope}%
\pgfpathrectangle{\pgfqpoint{1.250000in}{4.155455in}}{\pgfqpoint{2.279412in}{2.004545in}}%
\pgfusepath{clip}%
\pgfsetbuttcap%
\pgfsetroundjoin%
\pgfsetlinewidth{0.426814pt}%
\definecolor{currentstroke}{rgb}{0.282910,0.105393,0.426902}%
\pgfsetstrokecolor{currentstroke}%
\pgfsetdash{}{0pt}%
\pgfpathmoveto{\pgfqpoint{2.908994in}{4.941551in}}%
\pgfpathlineto{\pgfqpoint{2.858912in}{4.943851in}}%
\pgfusepath{stroke}%
\end{pgfscope}%
\begin{pgfscope}%
\pgfpathrectangle{\pgfqpoint{1.250000in}{4.155455in}}{\pgfqpoint{2.279412in}{2.004545in}}%
\pgfusepath{clip}%
\pgfsetbuttcap%
\pgfsetroundjoin%
\pgfsetlinewidth{0.464310pt}%
\definecolor{currentstroke}{rgb}{0.283072,0.130895,0.449241}%
\pgfsetstrokecolor{currentstroke}%
\pgfsetdash{}{0pt}%
\pgfpathmoveto{\pgfqpoint{2.858912in}{4.943851in}}%
\pgfpathlineto{\pgfqpoint{2.808821in}{4.946028in}}%
\pgfusepath{stroke}%
\end{pgfscope}%
\begin{pgfscope}%
\pgfpathrectangle{\pgfqpoint{1.250000in}{4.155455in}}{\pgfqpoint{2.279412in}{2.004545in}}%
\pgfusepath{clip}%
\pgfsetbuttcap%
\pgfsetroundjoin%
\pgfsetlinewidth{0.517837pt}%
\definecolor{currentstroke}{rgb}{0.279574,0.170599,0.479997}%
\pgfsetstrokecolor{currentstroke}%
\pgfsetdash{}{0pt}%
\pgfpathmoveto{\pgfqpoint{2.808821in}{4.946028in}}%
\pgfpathlineto{\pgfqpoint{2.758742in}{4.948408in}}%
\pgfusepath{stroke}%
\end{pgfscope}%
\begin{pgfscope}%
\pgfpathrectangle{\pgfqpoint{1.250000in}{4.155455in}}{\pgfqpoint{2.279412in}{2.004545in}}%
\pgfusepath{clip}%
\pgfsetbuttcap%
\pgfsetroundjoin%
\pgfsetlinewidth{0.566584pt}%
\definecolor{currentstroke}{rgb}{0.273006,0.204520,0.501721}%
\pgfsetstrokecolor{currentstroke}%
\pgfsetdash{}{0pt}%
\pgfpathmoveto{\pgfqpoint{2.758742in}{4.948408in}}%
\pgfpathlineto{\pgfqpoint{2.708678in}{4.951013in}}%
\pgfusepath{stroke}%
\end{pgfscope}%
\begin{pgfscope}%
\pgfpathrectangle{\pgfqpoint{1.250000in}{4.155455in}}{\pgfqpoint{2.279412in}{2.004545in}}%
\pgfusepath{clip}%
\pgfsetbuttcap%
\pgfsetroundjoin%
\pgfsetlinewidth{0.629383pt}%
\definecolor{currentstroke}{rgb}{0.260571,0.246922,0.522828}%
\pgfsetstrokecolor{currentstroke}%
\pgfsetdash{}{0pt}%
\pgfpathmoveto{\pgfqpoint{2.708678in}{4.951013in}}%
\pgfpathlineto{\pgfqpoint{2.658630in}{4.953836in}}%
\pgfusepath{stroke}%
\end{pgfscope}%
\begin{pgfscope}%
\pgfpathrectangle{\pgfqpoint{1.250000in}{4.155455in}}{\pgfqpoint{2.279412in}{2.004545in}}%
\pgfusepath{clip}%
\pgfsetbuttcap%
\pgfsetroundjoin%
\pgfsetlinewidth{0.675546pt}%
\definecolor{currentstroke}{rgb}{0.248629,0.278775,0.534556}%
\pgfsetstrokecolor{currentstroke}%
\pgfsetdash{}{0pt}%
\pgfpathmoveto{\pgfqpoint{2.658630in}{4.953836in}}%
\pgfpathlineto{\pgfqpoint{2.608607in}{4.956972in}}%
\pgfusepath{stroke}%
\end{pgfscope}%
\begin{pgfscope}%
\pgfpathrectangle{\pgfqpoint{1.250000in}{4.155455in}}{\pgfqpoint{2.279412in}{2.004545in}}%
\pgfusepath{clip}%
\pgfsetbuttcap%
\pgfsetroundjoin%
\pgfsetlinewidth{0.758248pt}%
\definecolor{currentstroke}{rgb}{0.225863,0.330805,0.547314}%
\pgfsetstrokecolor{currentstroke}%
\pgfsetdash{}{0pt}%
\pgfpathmoveto{\pgfqpoint{2.608607in}{4.956972in}}%
\pgfpathlineto{\pgfqpoint{2.558629in}{4.960628in}}%
\pgfusepath{stroke}%
\end{pgfscope}%
\begin{pgfscope}%
\pgfpathrectangle{\pgfqpoint{1.250000in}{4.155455in}}{\pgfqpoint{2.279412in}{2.004545in}}%
\pgfusepath{clip}%
\pgfsetbuttcap%
\pgfsetroundjoin%
\pgfsetlinewidth{0.791234pt}%
\definecolor{currentstroke}{rgb}{0.216210,0.351535,0.550627}%
\pgfsetstrokecolor{currentstroke}%
\pgfsetdash{}{0pt}%
\pgfpathmoveto{\pgfqpoint{2.558629in}{4.960628in}}%
\pgfpathlineto{\pgfqpoint{2.508703in}{4.964804in}}%
\pgfusepath{stroke}%
\end{pgfscope}%
\begin{pgfscope}%
\pgfpathrectangle{\pgfqpoint{1.250000in}{4.155455in}}{\pgfqpoint{2.279412in}{2.004545in}}%
\pgfusepath{clip}%
\pgfsetbuttcap%
\pgfsetroundjoin%
\pgfsetlinewidth{0.816887pt}%
\definecolor{currentstroke}{rgb}{0.208623,0.367752,0.552675}%
\pgfsetstrokecolor{currentstroke}%
\pgfsetdash{}{0pt}%
\pgfpathmoveto{\pgfqpoint{2.508703in}{4.964804in}}%
\pgfpathlineto{\pgfqpoint{2.458868in}{4.969725in}}%
\pgfusepath{stroke}%
\end{pgfscope}%
\begin{pgfscope}%
\pgfpathrectangle{\pgfqpoint{1.250000in}{4.155455in}}{\pgfqpoint{2.279412in}{2.004545in}}%
\pgfusepath{clip}%
\pgfsetbuttcap%
\pgfsetroundjoin%
\pgfsetlinewidth{0.818276pt}%
\definecolor{currentstroke}{rgb}{0.208623,0.367752,0.552675}%
\pgfsetstrokecolor{currentstroke}%
\pgfsetdash{}{0pt}%
\pgfpathmoveto{\pgfqpoint{2.458868in}{4.969725in}}%
\pgfpathlineto{\pgfqpoint{2.409204in}{4.975824in}}%
\pgfusepath{stroke}%
\end{pgfscope}%
\begin{pgfscope}%
\pgfpathrectangle{\pgfqpoint{1.250000in}{4.155455in}}{\pgfqpoint{2.279412in}{2.004545in}}%
\pgfusepath{clip}%
\pgfsetbuttcap%
\pgfsetroundjoin%
\pgfsetlinewidth{0.819577pt}%
\definecolor{currentstroke}{rgb}{0.208623,0.367752,0.552675}%
\pgfsetstrokecolor{currentstroke}%
\pgfsetdash{}{0pt}%
\pgfpathmoveto{\pgfqpoint{2.409204in}{4.975824in}}%
\pgfpathlineto{\pgfqpoint{2.359822in}{4.983452in}}%
\pgfusepath{stroke}%
\end{pgfscope}%
\begin{pgfscope}%
\pgfpathrectangle{\pgfqpoint{1.250000in}{4.155455in}}{\pgfqpoint{2.279412in}{2.004545in}}%
\pgfusepath{clip}%
\pgfsetbuttcap%
\pgfsetroundjoin%
\pgfsetlinewidth{0.793900pt}%
\definecolor{currentstroke}{rgb}{0.216210,0.351535,0.550627}%
\pgfsetstrokecolor{currentstroke}%
\pgfsetdash{}{0pt}%
\pgfpathmoveto{\pgfqpoint{2.359822in}{4.983452in}}%
\pgfpathlineto{\pgfqpoint{2.310767in}{4.992566in}}%
\pgfusepath{stroke}%
\end{pgfscope}%
\begin{pgfscope}%
\pgfpathrectangle{\pgfqpoint{1.250000in}{4.155455in}}{\pgfqpoint{2.279412in}{2.004545in}}%
\pgfusepath{clip}%
\pgfsetbuttcap%
\pgfsetroundjoin%
\pgfsetlinewidth{0.722427pt}%
\definecolor{currentstroke}{rgb}{0.235526,0.309527,0.542944}%
\pgfsetstrokecolor{currentstroke}%
\pgfsetdash{}{0pt}%
\pgfpathmoveto{\pgfqpoint{2.310767in}{4.992566in}}%
\pgfpathlineto{\pgfqpoint{2.262434in}{5.004133in}}%
\pgfusepath{stroke}%
\end{pgfscope}%
\begin{pgfscope}%
\pgfpathrectangle{\pgfqpoint{1.250000in}{4.155455in}}{\pgfqpoint{2.279412in}{2.004545in}}%
\pgfusepath{clip}%
\pgfsetbuttcap%
\pgfsetroundjoin%
\pgfsetlinewidth{0.694980pt}%
\definecolor{currentstroke}{rgb}{0.243113,0.292092,0.538516}%
\pgfsetstrokecolor{currentstroke}%
\pgfsetdash{}{0pt}%
\pgfpathmoveto{\pgfqpoint{2.262434in}{5.004133in}}%
\pgfpathlineto{\pgfqpoint{2.216337in}{5.020835in}}%
\pgfusepath{stroke}%
\end{pgfscope}%
\begin{pgfscope}%
\pgfpathrectangle{\pgfqpoint{1.250000in}{4.155455in}}{\pgfqpoint{2.279412in}{2.004545in}}%
\pgfusepath{clip}%
\pgfsetbuttcap%
\pgfsetroundjoin%
\pgfsetlinewidth{0.559614pt}%
\definecolor{currentstroke}{rgb}{0.274128,0.199721,0.498911}%
\pgfsetstrokecolor{currentstroke}%
\pgfsetdash{}{0pt}%
\pgfpathmoveto{\pgfqpoint{2.216337in}{5.020835in}}%
\pgfpathlineto{\pgfqpoint{2.216337in}{5.020835in}}%
\pgfusepath{stroke}%
\end{pgfscope}%
\begin{pgfscope}%
\pgfpathrectangle{\pgfqpoint{1.250000in}{4.155455in}}{\pgfqpoint{2.279412in}{2.004545in}}%
\pgfusepath{clip}%
\pgfsetbuttcap%
\pgfsetroundjoin%
\pgfsetlinewidth{0.333280pt}%
\definecolor{currentstroke}{rgb}{0.272594,0.025563,0.353093}%
\pgfsetstrokecolor{currentstroke}%
\pgfsetdash{}{0pt}%
\pgfpathmoveto{\pgfqpoint{3.159084in}{4.977300in}}%
\pgfpathlineto{\pgfqpoint{3.108936in}{4.977675in}}%
\pgfusepath{stroke}%
\end{pgfscope}%
\begin{pgfscope}%
\pgfpathrectangle{\pgfqpoint{1.250000in}{4.155455in}}{\pgfqpoint{2.279412in}{2.004545in}}%
\pgfusepath{clip}%
\pgfsetbuttcap%
\pgfsetroundjoin%
\pgfsetlinewidth{0.334996pt}%
\definecolor{currentstroke}{rgb}{0.272594,0.025563,0.353093}%
\pgfsetstrokecolor{currentstroke}%
\pgfsetdash{}{0pt}%
\pgfpathmoveto{\pgfqpoint{3.108936in}{4.977675in}}%
\pgfpathlineto{\pgfqpoint{3.058813in}{4.978908in}}%
\pgfusepath{stroke}%
\end{pgfscope}%
\begin{pgfscope}%
\pgfpathrectangle{\pgfqpoint{1.250000in}{4.155455in}}{\pgfqpoint{2.279412in}{2.004545in}}%
\pgfusepath{clip}%
\pgfsetbuttcap%
\pgfsetroundjoin%
\pgfsetlinewidth{0.342670pt}%
\definecolor{currentstroke}{rgb}{0.274952,0.037752,0.364543}%
\pgfsetstrokecolor{currentstroke}%
\pgfsetdash{}{0pt}%
\pgfpathmoveto{\pgfqpoint{3.058813in}{4.978908in}}%
\pgfpathlineto{\pgfqpoint{3.008718in}{4.980798in}}%
\pgfusepath{stroke}%
\end{pgfscope}%
\begin{pgfscope}%
\pgfpathrectangle{\pgfqpoint{1.250000in}{4.155455in}}{\pgfqpoint{2.279412in}{2.004545in}}%
\pgfusepath{clip}%
\pgfsetbuttcap%
\pgfsetroundjoin%
\pgfsetlinewidth{0.376266pt}%
\definecolor{currentstroke}{rgb}{0.278791,0.062145,0.386592}%
\pgfsetstrokecolor{currentstroke}%
\pgfsetdash{}{0pt}%
\pgfpathmoveto{\pgfqpoint{3.008718in}{4.980798in}}%
\pgfpathlineto{\pgfqpoint{2.958625in}{4.982764in}}%
\pgfusepath{stroke}%
\end{pgfscope}%
\begin{pgfscope}%
\pgfpathrectangle{\pgfqpoint{1.250000in}{4.155455in}}{\pgfqpoint{2.279412in}{2.004545in}}%
\pgfusepath{clip}%
\pgfsetbuttcap%
\pgfsetroundjoin%
\pgfsetlinewidth{0.406572pt}%
\definecolor{currentstroke}{rgb}{0.281924,0.089666,0.412415}%
\pgfsetstrokecolor{currentstroke}%
\pgfsetdash{}{0pt}%
\pgfpathmoveto{\pgfqpoint{2.958625in}{4.982764in}}%
\pgfpathlineto{\pgfqpoint{2.908533in}{4.984916in}}%
\pgfusepath{stroke}%
\end{pgfscope}%
\begin{pgfscope}%
\pgfpathrectangle{\pgfqpoint{1.250000in}{4.155455in}}{\pgfqpoint{2.279412in}{2.004545in}}%
\pgfusepath{clip}%
\pgfsetbuttcap%
\pgfsetroundjoin%
\pgfsetlinewidth{0.434830pt}%
\definecolor{currentstroke}{rgb}{0.283091,0.110553,0.431554}%
\pgfsetstrokecolor{currentstroke}%
\pgfsetdash{}{0pt}%
\pgfpathmoveto{\pgfqpoint{2.908533in}{4.984916in}}%
\pgfpathlineto{\pgfqpoint{2.858441in}{4.987066in}}%
\pgfusepath{stroke}%
\end{pgfscope}%
\begin{pgfscope}%
\pgfpathrectangle{\pgfqpoint{1.250000in}{4.155455in}}{\pgfqpoint{2.279412in}{2.004545in}}%
\pgfusepath{clip}%
\pgfsetbuttcap%
\pgfsetroundjoin%
\pgfsetlinewidth{0.484595pt}%
\definecolor{currentstroke}{rgb}{0.282290,0.145912,0.461510}%
\pgfsetstrokecolor{currentstroke}%
\pgfsetdash{}{0pt}%
\pgfpathmoveto{\pgfqpoint{2.858441in}{4.987066in}}%
\pgfpathlineto{\pgfqpoint{2.808346in}{4.989150in}}%
\pgfusepath{stroke}%
\end{pgfscope}%
\begin{pgfscope}%
\pgfpathrectangle{\pgfqpoint{1.250000in}{4.155455in}}{\pgfqpoint{2.279412in}{2.004545in}}%
\pgfusepath{clip}%
\pgfsetbuttcap%
\pgfsetroundjoin%
\pgfsetlinewidth{0.531508pt}%
\definecolor{currentstroke}{rgb}{0.278012,0.180367,0.486697}%
\pgfsetstrokecolor{currentstroke}%
\pgfsetdash{}{0pt}%
\pgfpathmoveto{\pgfqpoint{2.808346in}{4.989150in}}%
\pgfpathlineto{\pgfqpoint{2.758257in}{4.991351in}}%
\pgfusepath{stroke}%
\end{pgfscope}%
\begin{pgfscope}%
\pgfpathrectangle{\pgfqpoint{1.250000in}{4.155455in}}{\pgfqpoint{2.279412in}{2.004545in}}%
\pgfusepath{clip}%
\pgfsetbuttcap%
\pgfsetroundjoin%
\pgfsetlinewidth{0.607451pt}%
\definecolor{currentstroke}{rgb}{0.265145,0.232956,0.516599}%
\pgfsetstrokecolor{currentstroke}%
\pgfsetdash{}{0pt}%
\pgfpathmoveto{\pgfqpoint{2.758257in}{4.991351in}}%
\pgfpathlineto{\pgfqpoint{2.708177in}{4.993698in}}%
\pgfusepath{stroke}%
\end{pgfscope}%
\begin{pgfscope}%
\pgfpathrectangle{\pgfqpoint{1.250000in}{4.155455in}}{\pgfqpoint{2.279412in}{2.004545in}}%
\pgfusepath{clip}%
\pgfsetbuttcap%
\pgfsetroundjoin%
\pgfsetlinewidth{0.682555pt}%
\definecolor{currentstroke}{rgb}{0.246811,0.283237,0.535941}%
\pgfsetstrokecolor{currentstroke}%
\pgfsetdash{}{0pt}%
\pgfpathmoveto{\pgfqpoint{2.708177in}{4.993698in}}%
\pgfpathlineto{\pgfqpoint{2.658111in}{4.996286in}}%
\pgfusepath{stroke}%
\end{pgfscope}%
\begin{pgfscope}%
\pgfpathrectangle{\pgfqpoint{1.250000in}{4.155455in}}{\pgfqpoint{2.279412in}{2.004545in}}%
\pgfusepath{clip}%
\pgfsetbuttcap%
\pgfsetroundjoin%
\pgfsetlinewidth{0.333297pt}%
\definecolor{currentstroke}{rgb}{0.272594,0.025563,0.353093}%
\pgfsetstrokecolor{currentstroke}%
\pgfsetdash{}{0pt}%
\pgfpathmoveto{\pgfqpoint{3.159084in}{5.112620in}}%
\pgfpathlineto{\pgfqpoint{3.108941in}{5.112868in}}%
\pgfusepath{stroke}%
\end{pgfscope}%
\begin{pgfscope}%
\pgfpathrectangle{\pgfqpoint{1.250000in}{4.155455in}}{\pgfqpoint{2.279412in}{2.004545in}}%
\pgfusepath{clip}%
\pgfsetbuttcap%
\pgfsetroundjoin%
\pgfsetlinewidth{0.334709pt}%
\definecolor{currentstroke}{rgb}{0.272594,0.025563,0.353093}%
\pgfsetstrokecolor{currentstroke}%
\pgfsetdash{}{0pt}%
\pgfpathmoveto{\pgfqpoint{3.108941in}{5.112868in}}%
\pgfpathlineto{\pgfqpoint{3.058795in}{5.113316in}}%
\pgfusepath{stroke}%
\end{pgfscope}%
\begin{pgfscope}%
\pgfpathrectangle{\pgfqpoint{1.250000in}{4.155455in}}{\pgfqpoint{2.279412in}{2.004545in}}%
\pgfusepath{clip}%
\pgfsetbuttcap%
\pgfsetroundjoin%
\pgfsetlinewidth{0.352597pt}%
\definecolor{currentstroke}{rgb}{0.276022,0.044167,0.370164}%
\pgfsetstrokecolor{currentstroke}%
\pgfsetdash{}{0pt}%
\pgfpathmoveto{\pgfqpoint{3.058795in}{5.113316in}}%
\pgfpathlineto{\pgfqpoint{3.008644in}{5.113531in}}%
\pgfusepath{stroke}%
\end{pgfscope}%
\begin{pgfscope}%
\pgfpathrectangle{\pgfqpoint{1.250000in}{4.155455in}}{\pgfqpoint{2.279412in}{2.004545in}}%
\pgfusepath{clip}%
\pgfsetbuttcap%
\pgfsetroundjoin%
\pgfsetlinewidth{0.375216pt}%
\definecolor{currentstroke}{rgb}{0.278791,0.062145,0.386592}%
\pgfsetstrokecolor{currentstroke}%
\pgfsetdash{}{0pt}%
\pgfpathmoveto{\pgfqpoint{3.008644in}{5.113531in}}%
\pgfpathlineto{\pgfqpoint{2.958492in}{5.113639in}}%
\pgfusepath{stroke}%
\end{pgfscope}%
\begin{pgfscope}%
\pgfpathrectangle{\pgfqpoint{1.250000in}{4.155455in}}{\pgfqpoint{2.279412in}{2.004545in}}%
\pgfusepath{clip}%
\pgfsetbuttcap%
\pgfsetroundjoin%
\pgfsetlinewidth{0.422357pt}%
\definecolor{currentstroke}{rgb}{0.282656,0.100196,0.422160}%
\pgfsetstrokecolor{currentstroke}%
\pgfsetdash{}{0pt}%
\pgfpathmoveto{\pgfqpoint{2.958492in}{5.113639in}}%
\pgfpathlineto{\pgfqpoint{2.908341in}{5.113871in}}%
\pgfusepath{stroke}%
\end{pgfscope}%
\begin{pgfscope}%
\pgfpathrectangle{\pgfqpoint{1.250000in}{4.155455in}}{\pgfqpoint{2.279412in}{2.004545in}}%
\pgfusepath{clip}%
\pgfsetbuttcap%
\pgfsetroundjoin%
\pgfsetlinewidth{0.462827pt}%
\definecolor{currentstroke}{rgb}{0.283072,0.130895,0.449241}%
\pgfsetstrokecolor{currentstroke}%
\pgfsetdash{}{0pt}%
\pgfpathmoveto{\pgfqpoint{2.908341in}{5.113871in}}%
\pgfpathlineto{\pgfqpoint{2.858191in}{5.114157in}}%
\pgfusepath{stroke}%
\end{pgfscope}%
\begin{pgfscope}%
\pgfpathrectangle{\pgfqpoint{1.250000in}{4.155455in}}{\pgfqpoint{2.279412in}{2.004545in}}%
\pgfusepath{clip}%
\pgfsetbuttcap%
\pgfsetroundjoin%
\pgfsetlinewidth{0.519326pt}%
\definecolor{currentstroke}{rgb}{0.279574,0.170599,0.479997}%
\pgfsetstrokecolor{currentstroke}%
\pgfsetdash{}{0pt}%
\pgfpathmoveto{\pgfqpoint{2.858191in}{5.114157in}}%
\pgfpathlineto{\pgfqpoint{2.808041in}{5.114504in}}%
\pgfusepath{stroke}%
\end{pgfscope}%
\begin{pgfscope}%
\pgfpathrectangle{\pgfqpoint{1.250000in}{4.155455in}}{\pgfqpoint{2.279412in}{2.004545in}}%
\pgfusepath{clip}%
\pgfsetbuttcap%
\pgfsetroundjoin%
\pgfsetlinewidth{0.577849pt}%
\definecolor{currentstroke}{rgb}{0.270595,0.214069,0.507052}%
\pgfsetstrokecolor{currentstroke}%
\pgfsetdash{}{0pt}%
\pgfpathmoveto{\pgfqpoint{2.808041in}{5.114504in}}%
\pgfpathlineto{\pgfqpoint{2.757892in}{5.114974in}}%
\pgfusepath{stroke}%
\end{pgfscope}%
\begin{pgfscope}%
\pgfpathrectangle{\pgfqpoint{1.250000in}{4.155455in}}{\pgfqpoint{2.279412in}{2.004545in}}%
\pgfusepath{clip}%
\pgfsetbuttcap%
\pgfsetroundjoin%
\pgfsetlinewidth{0.662941pt}%
\definecolor{currentstroke}{rgb}{0.252194,0.269783,0.531579}%
\pgfsetstrokecolor{currentstroke}%
\pgfsetdash{}{0pt}%
\pgfpathmoveto{\pgfqpoint{2.757892in}{5.114974in}}%
\pgfpathlineto{\pgfqpoint{2.707742in}{5.115371in}}%
\pgfusepath{stroke}%
\end{pgfscope}%
\begin{pgfscope}%
\pgfpathrectangle{\pgfqpoint{1.250000in}{4.155455in}}{\pgfqpoint{2.279412in}{2.004545in}}%
\pgfusepath{clip}%
\pgfsetbuttcap%
\pgfsetroundjoin%
\pgfsetlinewidth{0.751400pt}%
\definecolor{currentstroke}{rgb}{0.227802,0.326594,0.546532}%
\pgfsetstrokecolor{currentstroke}%
\pgfsetdash{}{0pt}%
\pgfpathmoveto{\pgfqpoint{2.707742in}{5.115371in}}%
\pgfpathlineto{\pgfqpoint{2.657593in}{5.115848in}}%
\pgfusepath{stroke}%
\end{pgfscope}%
\begin{pgfscope}%
\pgfpathrectangle{\pgfqpoint{1.250000in}{4.155455in}}{\pgfqpoint{2.279412in}{2.004545in}}%
\pgfusepath{clip}%
\pgfsetbuttcap%
\pgfsetroundjoin%
\pgfsetlinewidth{0.809179pt}%
\definecolor{currentstroke}{rgb}{0.210503,0.363727,0.552206}%
\pgfsetstrokecolor{currentstroke}%
\pgfsetdash{}{0pt}%
\pgfpathmoveto{\pgfqpoint{2.657593in}{5.115848in}}%
\pgfpathlineto{\pgfqpoint{2.607446in}{5.116436in}}%
\pgfusepath{stroke}%
\end{pgfscope}%
\begin{pgfscope}%
\pgfpathrectangle{\pgfqpoint{1.250000in}{4.155455in}}{\pgfqpoint{2.279412in}{2.004545in}}%
\pgfusepath{clip}%
\pgfsetbuttcap%
\pgfsetroundjoin%
\pgfsetlinewidth{0.866505pt}%
\definecolor{currentstroke}{rgb}{0.195860,0.395433,0.555276}%
\pgfsetstrokecolor{currentstroke}%
\pgfsetdash{}{0pt}%
\pgfpathmoveto{\pgfqpoint{2.607446in}{5.116436in}}%
\pgfpathlineto{\pgfqpoint{2.557299in}{5.117046in}}%
\pgfusepath{stroke}%
\end{pgfscope}%
\begin{pgfscope}%
\pgfpathrectangle{\pgfqpoint{1.250000in}{4.155455in}}{\pgfqpoint{2.279412in}{2.004545in}}%
\pgfusepath{clip}%
\pgfsetbuttcap%
\pgfsetroundjoin%
\pgfsetlinewidth{0.920814pt}%
\definecolor{currentstroke}{rgb}{0.182256,0.426184,0.557120}%
\pgfsetstrokecolor{currentstroke}%
\pgfsetdash{}{0pt}%
\pgfpathmoveto{\pgfqpoint{2.557299in}{5.117046in}}%
\pgfpathlineto{\pgfqpoint{2.507154in}{5.117804in}}%
\pgfusepath{stroke}%
\end{pgfscope}%
\begin{pgfscope}%
\pgfpathrectangle{\pgfqpoint{1.250000in}{4.155455in}}{\pgfqpoint{2.279412in}{2.004545in}}%
\pgfusepath{clip}%
\pgfsetbuttcap%
\pgfsetroundjoin%
\pgfsetlinewidth{0.943248pt}%
\definecolor{currentstroke}{rgb}{0.177423,0.437527,0.557565}%
\pgfsetstrokecolor{currentstroke}%
\pgfsetdash{}{0pt}%
\pgfpathmoveto{\pgfqpoint{2.507154in}{5.117804in}}%
\pgfpathlineto{\pgfqpoint{2.457011in}{5.118623in}}%
\pgfusepath{stroke}%
\end{pgfscope}%
\begin{pgfscope}%
\pgfpathrectangle{\pgfqpoint{1.250000in}{4.155455in}}{\pgfqpoint{2.279412in}{2.004545in}}%
\pgfusepath{clip}%
\pgfsetbuttcap%
\pgfsetroundjoin%
\pgfsetlinewidth{0.885662pt}%
\definecolor{currentstroke}{rgb}{0.190631,0.407061,0.556089}%
\pgfsetstrokecolor{currentstroke}%
\pgfsetdash{}{0pt}%
\pgfpathmoveto{\pgfqpoint{2.457011in}{5.118623in}}%
\pgfpathlineto{\pgfqpoint{2.406872in}{5.119607in}}%
\pgfusepath{stroke}%
\end{pgfscope}%
\begin{pgfscope}%
\pgfpathrectangle{\pgfqpoint{1.250000in}{4.155455in}}{\pgfqpoint{2.279412in}{2.004545in}}%
\pgfusepath{clip}%
\pgfsetbuttcap%
\pgfsetroundjoin%
\pgfsetlinewidth{0.903143pt}%
\definecolor{currentstroke}{rgb}{0.187231,0.414746,0.556547}%
\pgfsetstrokecolor{currentstroke}%
\pgfsetdash{}{0pt}%
\pgfpathmoveto{\pgfqpoint{2.406872in}{5.119607in}}%
\pgfpathlineto{\pgfqpoint{2.356747in}{5.120968in}}%
\pgfusepath{stroke}%
\end{pgfscope}%
\begin{pgfscope}%
\pgfpathrectangle{\pgfqpoint{1.250000in}{4.155455in}}{\pgfqpoint{2.279412in}{2.004545in}}%
\pgfusepath{clip}%
\pgfsetbuttcap%
\pgfsetroundjoin%
\pgfsetlinewidth{0.804110pt}%
\definecolor{currentstroke}{rgb}{0.212395,0.359683,0.551710}%
\pgfsetstrokecolor{currentstroke}%
\pgfsetdash{}{0pt}%
\pgfpathmoveto{\pgfqpoint{2.356747in}{5.120968in}}%
\pgfpathlineto{\pgfqpoint{2.306629in}{5.122523in}}%
\pgfusepath{stroke}%
\end{pgfscope}%
\begin{pgfscope}%
\pgfpathrectangle{\pgfqpoint{1.250000in}{4.155455in}}{\pgfqpoint{2.279412in}{2.004545in}}%
\pgfusepath{clip}%
\pgfsetbuttcap%
\pgfsetroundjoin%
\pgfsetlinewidth{0.314223pt}%
\definecolor{currentstroke}{rgb}{0.268510,0.009605,0.335427}%
\pgfsetstrokecolor{currentstroke}%
\pgfsetdash{}{0pt}%
\pgfpathmoveto{\pgfqpoint{3.159084in}{5.428368in}}%
\pgfpathlineto{\pgfqpoint{3.109127in}{5.425100in}}%
\pgfusepath{stroke}%
\end{pgfscope}%
\begin{pgfscope}%
\pgfpathrectangle{\pgfqpoint{1.250000in}{4.155455in}}{\pgfqpoint{2.279412in}{2.004545in}}%
\pgfusepath{clip}%
\pgfsetbuttcap%
\pgfsetroundjoin%
\pgfsetlinewidth{0.331358pt}%
\definecolor{currentstroke}{rgb}{0.272594,0.025563,0.353093}%
\pgfsetstrokecolor{currentstroke}%
\pgfsetdash{}{0pt}%
\pgfpathmoveto{\pgfqpoint{3.109127in}{5.425100in}}%
\pgfpathlineto{\pgfqpoint{3.059124in}{5.421936in}}%
\pgfusepath{stroke}%
\end{pgfscope}%
\begin{pgfscope}%
\pgfpathrectangle{\pgfqpoint{1.250000in}{4.155455in}}{\pgfqpoint{2.279412in}{2.004545in}}%
\pgfusepath{clip}%
\pgfsetbuttcap%
\pgfsetroundjoin%
\pgfsetlinewidth{0.333971pt}%
\definecolor{currentstroke}{rgb}{0.272594,0.025563,0.353093}%
\pgfsetstrokecolor{currentstroke}%
\pgfsetdash{}{0pt}%
\pgfpathmoveto{\pgfqpoint{3.059124in}{5.421936in}}%
\pgfpathlineto{\pgfqpoint{3.009112in}{5.418861in}}%
\pgfusepath{stroke}%
\end{pgfscope}%
\begin{pgfscope}%
\pgfpathrectangle{\pgfqpoint{1.250000in}{4.155455in}}{\pgfqpoint{2.279412in}{2.004545in}}%
\pgfusepath{clip}%
\pgfsetbuttcap%
\pgfsetroundjoin%
\pgfsetlinewidth{0.365128pt}%
\definecolor{currentstroke}{rgb}{0.277941,0.056324,0.381191}%
\pgfsetstrokecolor{currentstroke}%
\pgfsetdash{}{0pt}%
\pgfpathmoveto{\pgfqpoint{3.009112in}{5.418861in}}%
\pgfpathlineto{\pgfqpoint{2.959024in}{5.416697in}}%
\pgfusepath{stroke}%
\end{pgfscope}%
\begin{pgfscope}%
\pgfpathrectangle{\pgfqpoint{1.250000in}{4.155455in}}{\pgfqpoint{2.279412in}{2.004545in}}%
\pgfusepath{clip}%
\pgfsetbuttcap%
\pgfsetroundjoin%
\pgfsetlinewidth{0.382935pt}%
\definecolor{currentstroke}{rgb}{0.279566,0.067836,0.391917}%
\pgfsetstrokecolor{currentstroke}%
\pgfsetdash{}{0pt}%
\pgfpathmoveto{\pgfqpoint{2.959024in}{5.416697in}}%
\pgfpathlineto{\pgfqpoint{2.908928in}{5.414642in}}%
\pgfusepath{stroke}%
\end{pgfscope}%
\begin{pgfscope}%
\pgfpathrectangle{\pgfqpoint{1.250000in}{4.155455in}}{\pgfqpoint{2.279412in}{2.004545in}}%
\pgfusepath{clip}%
\pgfsetbuttcap%
\pgfsetroundjoin%
\pgfsetlinewidth{0.428965pt}%
\definecolor{currentstroke}{rgb}{0.282910,0.105393,0.426902}%
\pgfsetstrokecolor{currentstroke}%
\pgfsetdash{}{0pt}%
\pgfpathmoveto{\pgfqpoint{2.908928in}{5.414642in}}%
\pgfpathlineto{\pgfqpoint{2.858875in}{5.411975in}}%
\pgfusepath{stroke}%
\end{pgfscope}%
\begin{pgfscope}%
\pgfpathrectangle{\pgfqpoint{1.250000in}{4.155455in}}{\pgfqpoint{2.279412in}{2.004545in}}%
\pgfusepath{clip}%
\pgfsetbuttcap%
\pgfsetroundjoin%
\pgfsetlinewidth{0.437536pt}%
\definecolor{currentstroke}{rgb}{0.283091,0.110553,0.431554}%
\pgfsetstrokecolor{currentstroke}%
\pgfsetdash{}{0pt}%
\pgfpathmoveto{\pgfqpoint{2.858875in}{5.411975in}}%
\pgfpathlineto{\pgfqpoint{2.808866in}{5.408656in}}%
\pgfusepath{stroke}%
\end{pgfscope}%
\begin{pgfscope}%
\pgfpathrectangle{\pgfqpoint{1.250000in}{4.155455in}}{\pgfqpoint{2.279412in}{2.004545in}}%
\pgfusepath{clip}%
\pgfsetbuttcap%
\pgfsetroundjoin%
\pgfsetlinewidth{0.310922pt}%
\definecolor{currentstroke}{rgb}{0.268510,0.009605,0.335427}%
\pgfsetstrokecolor{currentstroke}%
\pgfsetdash{}{0pt}%
\pgfpathmoveto{\pgfqpoint{3.159084in}{5.473475in}}%
\pgfpathlineto{\pgfqpoint{3.109012in}{5.472769in}}%
\pgfusepath{stroke}%
\end{pgfscope}%
\begin{pgfscope}%
\pgfpathrectangle{\pgfqpoint{1.250000in}{4.155455in}}{\pgfqpoint{2.279412in}{2.004545in}}%
\pgfusepath{clip}%
\pgfsetbuttcap%
\pgfsetroundjoin%
\pgfsetlinewidth{0.330315pt}%
\definecolor{currentstroke}{rgb}{0.272594,0.025563,0.353093}%
\pgfsetstrokecolor{currentstroke}%
\pgfsetdash{}{0pt}%
\pgfpathmoveto{\pgfqpoint{3.109012in}{5.472769in}}%
\pgfpathlineto{\pgfqpoint{3.058969in}{5.469975in}}%
\pgfusepath{stroke}%
\end{pgfscope}%
\begin{pgfscope}%
\pgfpathrectangle{\pgfqpoint{1.250000in}{4.155455in}}{\pgfqpoint{2.279412in}{2.004545in}}%
\pgfusepath{clip}%
\pgfsetbuttcap%
\pgfsetroundjoin%
\pgfsetlinewidth{0.334415pt}%
\definecolor{currentstroke}{rgb}{0.272594,0.025563,0.353093}%
\pgfsetstrokecolor{currentstroke}%
\pgfsetdash{}{0pt}%
\pgfpathmoveto{\pgfqpoint{3.058969in}{5.469975in}}%
\pgfpathlineto{\pgfqpoint{3.008937in}{5.466992in}}%
\pgfusepath{stroke}%
\end{pgfscope}%
\begin{pgfscope}%
\pgfpathrectangle{\pgfqpoint{1.250000in}{4.155455in}}{\pgfqpoint{2.279412in}{2.004545in}}%
\pgfusepath{clip}%
\pgfsetbuttcap%
\pgfsetroundjoin%
\pgfsetlinewidth{0.349184pt}%
\definecolor{currentstroke}{rgb}{0.276022,0.044167,0.370164}%
\pgfsetstrokecolor{currentstroke}%
\pgfsetdash{}{0pt}%
\pgfpathmoveto{\pgfqpoint{3.008937in}{5.466992in}}%
\pgfpathlineto{\pgfqpoint{2.958904in}{5.463984in}}%
\pgfusepath{stroke}%
\end{pgfscope}%
\begin{pgfscope}%
\pgfpathrectangle{\pgfqpoint{1.250000in}{4.155455in}}{\pgfqpoint{2.279412in}{2.004545in}}%
\pgfusepath{clip}%
\pgfsetbuttcap%
\pgfsetroundjoin%
\pgfsetlinewidth{0.362529pt}%
\definecolor{currentstroke}{rgb}{0.277018,0.050344,0.375715}%
\pgfsetstrokecolor{currentstroke}%
\pgfsetdash{}{0pt}%
\pgfpathmoveto{\pgfqpoint{2.958904in}{5.463984in}}%
\pgfpathlineto{\pgfqpoint{2.908846in}{5.461314in}}%
\pgfusepath{stroke}%
\end{pgfscope}%
\begin{pgfscope}%
\pgfpathrectangle{\pgfqpoint{1.250000in}{4.155455in}}{\pgfqpoint{2.279412in}{2.004545in}}%
\pgfusepath{clip}%
\pgfsetbuttcap%
\pgfsetroundjoin%
\pgfsetlinewidth{0.394136pt}%
\definecolor{currentstroke}{rgb}{0.280894,0.078907,0.402329}%
\pgfsetstrokecolor{currentstroke}%
\pgfsetdash{}{0pt}%
\pgfpathmoveto{\pgfqpoint{2.908846in}{5.461314in}}%
\pgfpathlineto{\pgfqpoint{2.858825in}{5.458205in}}%
\pgfusepath{stroke}%
\end{pgfscope}%
\begin{pgfscope}%
\pgfpathrectangle{\pgfqpoint{1.250000in}{4.155455in}}{\pgfqpoint{2.279412in}{2.004545in}}%
\pgfusepath{clip}%
\pgfsetbuttcap%
\pgfsetroundjoin%
\pgfsetlinewidth{0.418513pt}%
\definecolor{currentstroke}{rgb}{0.282656,0.100196,0.422160}%
\pgfsetstrokecolor{currentstroke}%
\pgfsetdash{}{0pt}%
\pgfpathmoveto{\pgfqpoint{2.858825in}{5.458205in}}%
\pgfpathlineto{\pgfqpoint{2.808863in}{5.454376in}}%
\pgfusepath{stroke}%
\end{pgfscope}%
\begin{pgfscope}%
\pgfpathrectangle{\pgfqpoint{1.250000in}{4.155455in}}{\pgfqpoint{2.279412in}{2.004545in}}%
\pgfusepath{clip}%
\pgfsetbuttcap%
\pgfsetroundjoin%
\pgfsetlinewidth{0.455375pt}%
\definecolor{currentstroke}{rgb}{0.283187,0.125848,0.444960}%
\pgfsetstrokecolor{currentstroke}%
\pgfsetdash{}{0pt}%
\pgfpathmoveto{\pgfqpoint{2.808863in}{5.454376in}}%
\pgfpathlineto{\pgfqpoint{2.758911in}{5.450443in}}%
\pgfusepath{stroke}%
\end{pgfscope}%
\begin{pgfscope}%
\pgfpathrectangle{\pgfqpoint{1.250000in}{4.155455in}}{\pgfqpoint{2.279412in}{2.004545in}}%
\pgfusepath{clip}%
\pgfsetbuttcap%
\pgfsetroundjoin%
\pgfsetlinewidth{0.497820pt}%
\definecolor{currentstroke}{rgb}{0.281412,0.155834,0.469201}%
\pgfsetstrokecolor{currentstroke}%
\pgfsetdash{}{0pt}%
\pgfpathmoveto{\pgfqpoint{2.758911in}{5.450443in}}%
\pgfpathlineto{\pgfqpoint{2.708979in}{5.446325in}}%
\pgfusepath{stroke}%
\end{pgfscope}%
\begin{pgfscope}%
\pgfpathrectangle{\pgfqpoint{1.250000in}{4.155455in}}{\pgfqpoint{2.279412in}{2.004545in}}%
\pgfusepath{clip}%
\pgfsetbuttcap%
\pgfsetroundjoin%
\pgfsetlinewidth{0.551962pt}%
\definecolor{currentstroke}{rgb}{0.275191,0.194905,0.496005}%
\pgfsetstrokecolor{currentstroke}%
\pgfsetdash{}{0pt}%
\pgfpathmoveto{\pgfqpoint{2.708979in}{5.446325in}}%
\pgfpathlineto{\pgfqpoint{2.659078in}{5.441920in}}%
\pgfusepath{stroke}%
\end{pgfscope}%
\begin{pgfscope}%
\pgfpathrectangle{\pgfqpoint{1.250000in}{4.155455in}}{\pgfqpoint{2.279412in}{2.004545in}}%
\pgfusepath{clip}%
\pgfsetbuttcap%
\pgfsetroundjoin%
\pgfsetlinewidth{0.599244pt}%
\definecolor{currentstroke}{rgb}{0.266580,0.228262,0.514349}%
\pgfsetstrokecolor{currentstroke}%
\pgfsetdash{}{0pt}%
\pgfpathmoveto{\pgfqpoint{2.659078in}{5.441920in}}%
\pgfpathlineto{\pgfqpoint{2.609230in}{5.437083in}}%
\pgfusepath{stroke}%
\end{pgfscope}%
\begin{pgfscope}%
\pgfpathrectangle{\pgfqpoint{1.250000in}{4.155455in}}{\pgfqpoint{2.279412in}{2.004545in}}%
\pgfusepath{clip}%
\pgfsetbuttcap%
\pgfsetroundjoin%
\pgfsetlinewidth{0.639855pt}%
\definecolor{currentstroke}{rgb}{0.257322,0.256130,0.526563}%
\pgfsetstrokecolor{currentstroke}%
\pgfsetdash{}{0pt}%
\pgfpathmoveto{\pgfqpoint{2.609230in}{5.437083in}}%
\pgfpathlineto{\pgfqpoint{2.559459in}{5.431670in}}%
\pgfusepath{stroke}%
\end{pgfscope}%
\begin{pgfscope}%
\pgfpathrectangle{\pgfqpoint{1.250000in}{4.155455in}}{\pgfqpoint{2.279412in}{2.004545in}}%
\pgfusepath{clip}%
\pgfsetbuttcap%
\pgfsetroundjoin%
\pgfsetlinewidth{0.685214pt}%
\definecolor{currentstroke}{rgb}{0.246811,0.283237,0.535941}%
\pgfsetstrokecolor{currentstroke}%
\pgfsetdash{}{0pt}%
\pgfpathmoveto{\pgfqpoint{2.559459in}{5.431670in}}%
\pgfpathlineto{\pgfqpoint{2.509792in}{5.425575in}}%
\pgfusepath{stroke}%
\end{pgfscope}%
\begin{pgfscope}%
\pgfpathrectangle{\pgfqpoint{1.250000in}{4.155455in}}{\pgfqpoint{2.279412in}{2.004545in}}%
\pgfusepath{clip}%
\pgfsetbuttcap%
\pgfsetroundjoin%
\pgfsetlinewidth{0.693941pt}%
\definecolor{currentstroke}{rgb}{0.243113,0.292092,0.538516}%
\pgfsetstrokecolor{currentstroke}%
\pgfsetdash{}{0pt}%
\pgfpathmoveto{\pgfqpoint{2.509792in}{5.425575in}}%
\pgfpathlineto{\pgfqpoint{2.460294in}{5.418500in}}%
\pgfusepath{stroke}%
\end{pgfscope}%
\begin{pgfscope}%
\pgfpathrectangle{\pgfqpoint{1.250000in}{4.155455in}}{\pgfqpoint{2.279412in}{2.004545in}}%
\pgfusepath{clip}%
\pgfsetbuttcap%
\pgfsetroundjoin%
\pgfsetlinewidth{0.719532pt}%
\definecolor{currentstroke}{rgb}{0.237441,0.305202,0.541921}%
\pgfsetstrokecolor{currentstroke}%
\pgfsetdash{}{0pt}%
\pgfpathmoveto{\pgfqpoint{2.460294in}{5.418500in}}%
\pgfpathlineto{\pgfqpoint{2.411052in}{5.410175in}}%
\pgfusepath{stroke}%
\end{pgfscope}%
\begin{pgfscope}%
\pgfpathrectangle{\pgfqpoint{1.250000in}{4.155455in}}{\pgfqpoint{2.279412in}{2.004545in}}%
\pgfusepath{clip}%
\pgfsetbuttcap%
\pgfsetroundjoin%
\pgfsetlinewidth{0.734508pt}%
\definecolor{currentstroke}{rgb}{0.231674,0.318106,0.544834}%
\pgfsetstrokecolor{currentstroke}%
\pgfsetdash{}{0pt}%
\pgfpathmoveto{\pgfqpoint{2.411052in}{5.410175in}}%
\pgfpathlineto{\pgfqpoint{2.362206in}{5.400246in}}%
\pgfusepath{stroke}%
\end{pgfscope}%
\begin{pgfscope}%
\pgfpathrectangle{\pgfqpoint{1.250000in}{4.155455in}}{\pgfqpoint{2.279412in}{2.004545in}}%
\pgfusepath{clip}%
\pgfsetbuttcap%
\pgfsetroundjoin%
\pgfsetlinewidth{0.678714pt}%
\definecolor{currentstroke}{rgb}{0.246811,0.283237,0.535941}%
\pgfsetstrokecolor{currentstroke}%
\pgfsetdash{}{0pt}%
\pgfpathmoveto{\pgfqpoint{2.362206in}{5.400246in}}%
\pgfpathlineto{\pgfqpoint{2.314104in}{5.387934in}}%
\pgfusepath{stroke}%
\end{pgfscope}%
\begin{pgfscope}%
\pgfpathrectangle{\pgfqpoint{1.250000in}{4.155455in}}{\pgfqpoint{2.279412in}{2.004545in}}%
\pgfusepath{clip}%
\pgfsetbuttcap%
\pgfsetroundjoin%
\pgfsetlinewidth{0.681150pt}%
\definecolor{currentstroke}{rgb}{0.246811,0.283237,0.535941}%
\pgfsetstrokecolor{currentstroke}%
\pgfsetdash{}{0pt}%
\pgfpathmoveto{\pgfqpoint{2.314104in}{5.387934in}}%
\pgfpathlineto{\pgfqpoint{2.267452in}{5.372003in}}%
\pgfusepath{stroke}%
\end{pgfscope}%
\begin{pgfscope}%
\pgfpathrectangle{\pgfqpoint{1.250000in}{4.155455in}}{\pgfqpoint{2.279412in}{2.004545in}}%
\pgfusepath{clip}%
\pgfsetbuttcap%
\pgfsetroundjoin%
\pgfsetlinewidth{0.633347pt}%
\definecolor{currentstroke}{rgb}{0.258965,0.251537,0.524736}%
\pgfsetstrokecolor{currentstroke}%
\pgfsetdash{}{0pt}%
\pgfpathmoveto{\pgfqpoint{2.267452in}{5.372003in}}%
\pgfpathlineto{\pgfqpoint{2.224502in}{5.350044in}}%
\pgfusepath{stroke}%
\end{pgfscope}%
\begin{pgfscope}%
\pgfpathrectangle{\pgfqpoint{1.250000in}{4.155455in}}{\pgfqpoint{2.279412in}{2.004545in}}%
\pgfusepath{clip}%
\pgfsetbuttcap%
\pgfsetroundjoin%
\pgfsetlinewidth{0.596158pt}%
\definecolor{currentstroke}{rgb}{0.267968,0.223549,0.512008}%
\pgfsetstrokecolor{currentstroke}%
\pgfsetdash{}{0pt}%
\pgfpathmoveto{\pgfqpoint{2.224502in}{5.350044in}}%
\pgfpathlineto{\pgfqpoint{2.224502in}{5.350044in}}%
\pgfusepath{stroke}%
\end{pgfscope}%
\begin{pgfscope}%
\pgfpathrectangle{\pgfqpoint{1.250000in}{4.155455in}}{\pgfqpoint{2.279412in}{2.004545in}}%
\pgfusepath{clip}%
\pgfsetbuttcap%
\pgfsetroundjoin%
\pgfsetlinewidth{0.596158pt}%
\definecolor{currentstroke}{rgb}{0.267968,0.223549,0.512008}%
\pgfsetstrokecolor{currentstroke}%
\pgfsetdash{}{0pt}%
\pgfpathmoveto{\pgfqpoint{2.224502in}{5.350044in}}%
\pgfpathlineto{\pgfqpoint{2.195245in}{5.328436in}}%
\pgfusepath{stroke}%
\end{pgfscope}%
\begin{pgfscope}%
\pgfpathrectangle{\pgfqpoint{1.250000in}{4.155455in}}{\pgfqpoint{2.279412in}{2.004545in}}%
\pgfusepath{clip}%
\pgfsetbuttcap%
\pgfsetroundjoin%
\pgfsetlinewidth{0.576076pt}%
\definecolor{currentstroke}{rgb}{0.270595,0.214069,0.507052}%
\pgfsetstrokecolor{currentstroke}%
\pgfsetdash{}{0pt}%
\pgfpathmoveto{\pgfqpoint{2.195245in}{5.328436in}}%
\pgfpathlineto{\pgfqpoint{2.195245in}{5.328436in}}%
\pgfusepath{stroke}%
\end{pgfscope}%
\begin{pgfscope}%
\pgfpathrectangle{\pgfqpoint{1.250000in}{4.155455in}}{\pgfqpoint{2.279412in}{2.004545in}}%
\pgfusepath{clip}%
\pgfsetbuttcap%
\pgfsetroundjoin%
\pgfsetlinewidth{0.576076pt}%
\definecolor{currentstroke}{rgb}{0.270595,0.214069,0.507052}%
\pgfsetstrokecolor{currentstroke}%
\pgfsetdash{}{0pt}%
\pgfpathmoveto{\pgfqpoint{2.195245in}{5.328436in}}%
\pgfpathlineto{\pgfqpoint{2.195245in}{5.328436in}}%
\pgfusepath{stroke}%
\end{pgfscope}%
\begin{pgfscope}%
\pgfpathrectangle{\pgfqpoint{1.250000in}{4.155455in}}{\pgfqpoint{2.279412in}{2.004545in}}%
\pgfusepath{clip}%
\pgfsetbuttcap%
\pgfsetroundjoin%
\pgfsetlinewidth{0.332944pt}%
\definecolor{currentstroke}{rgb}{0.272594,0.025563,0.353093}%
\pgfsetstrokecolor{currentstroke}%
\pgfsetdash{}{0pt}%
\pgfpathmoveto{\pgfqpoint{3.107792in}{4.796873in}}%
\pgfpathlineto{\pgfqpoint{3.057765in}{4.798331in}}%
\pgfusepath{stroke}%
\end{pgfscope}%
\begin{pgfscope}%
\pgfpathrectangle{\pgfqpoint{1.250000in}{4.155455in}}{\pgfqpoint{2.279412in}{2.004545in}}%
\pgfusepath{clip}%
\pgfsetbuttcap%
\pgfsetroundjoin%
\pgfsetlinewidth{0.317212pt}%
\definecolor{currentstroke}{rgb}{0.269944,0.014625,0.341379}%
\pgfsetstrokecolor{currentstroke}%
\pgfsetdash{}{0pt}%
\pgfpathmoveto{\pgfqpoint{3.057765in}{4.798331in}}%
\pgfpathlineto{\pgfqpoint{3.007696in}{4.800357in}}%
\pgfusepath{stroke}%
\end{pgfscope}%
\begin{pgfscope}%
\pgfpathrectangle{\pgfqpoint{1.250000in}{4.155455in}}{\pgfqpoint{2.279412in}{2.004545in}}%
\pgfusepath{clip}%
\pgfsetbuttcap%
\pgfsetroundjoin%
\pgfsetlinewidth{0.344765pt}%
\definecolor{currentstroke}{rgb}{0.274952,0.037752,0.364543}%
\pgfsetstrokecolor{currentstroke}%
\pgfsetdash{}{0pt}%
\pgfpathmoveto{\pgfqpoint{3.007696in}{4.800357in}}%
\pgfpathlineto{\pgfqpoint{2.957630in}{4.802755in}}%
\pgfusepath{stroke}%
\end{pgfscope}%
\begin{pgfscope}%
\pgfpathrectangle{\pgfqpoint{1.250000in}{4.155455in}}{\pgfqpoint{2.279412in}{2.004545in}}%
\pgfusepath{clip}%
\pgfsetbuttcap%
\pgfsetroundjoin%
\pgfsetlinewidth{0.356068pt}%
\definecolor{currentstroke}{rgb}{0.277018,0.050344,0.375715}%
\pgfsetstrokecolor{currentstroke}%
\pgfsetdash{}{0pt}%
\pgfpathmoveto{\pgfqpoint{2.957630in}{4.802755in}}%
\pgfpathlineto{\pgfqpoint{2.907578in}{4.805430in}}%
\pgfusepath{stroke}%
\end{pgfscope}%
\begin{pgfscope}%
\pgfpathrectangle{\pgfqpoint{1.250000in}{4.155455in}}{\pgfqpoint{2.279412in}{2.004545in}}%
\pgfusepath{clip}%
\pgfsetbuttcap%
\pgfsetroundjoin%
\pgfsetlinewidth{0.380700pt}%
\definecolor{currentstroke}{rgb}{0.279566,0.067836,0.391917}%
\pgfsetstrokecolor{currentstroke}%
\pgfsetdash{}{0pt}%
\pgfpathmoveto{\pgfqpoint{2.907578in}{4.805430in}}%
\pgfpathlineto{\pgfqpoint{2.857578in}{4.808858in}}%
\pgfusepath{stroke}%
\end{pgfscope}%
\begin{pgfscope}%
\pgfpathrectangle{\pgfqpoint{1.250000in}{4.155455in}}{\pgfqpoint{2.279412in}{2.004545in}}%
\pgfusepath{clip}%
\pgfsetbuttcap%
\pgfsetroundjoin%
\pgfsetlinewidth{0.409336pt}%
\definecolor{currentstroke}{rgb}{0.281924,0.089666,0.412415}%
\pgfsetstrokecolor{currentstroke}%
\pgfsetdash{}{0pt}%
\pgfpathmoveto{\pgfqpoint{2.857578in}{4.808858in}}%
\pgfpathlineto{\pgfqpoint{2.807603in}{4.812545in}}%
\pgfusepath{stroke}%
\end{pgfscope}%
\begin{pgfscope}%
\pgfpathrectangle{\pgfqpoint{1.250000in}{4.155455in}}{\pgfqpoint{2.279412in}{2.004545in}}%
\pgfusepath{clip}%
\pgfsetbuttcap%
\pgfsetroundjoin%
\pgfsetlinewidth{0.327702pt}%
\definecolor{currentstroke}{rgb}{0.271305,0.019942,0.347269}%
\pgfsetstrokecolor{currentstroke}%
\pgfsetdash{}{0pt}%
\pgfpathmoveto{\pgfqpoint{2.846180in}{4.552263in}}%
\pgfpathlineto{\pgfqpoint{2.796698in}{4.559078in}}%
\pgfusepath{stroke}%
\end{pgfscope}%
\begin{pgfscope}%
\pgfpathrectangle{\pgfqpoint{1.250000in}{4.155455in}}{\pgfqpoint{2.279412in}{2.004545in}}%
\pgfusepath{clip}%
\pgfsetbuttcap%
\pgfsetroundjoin%
\pgfsetlinewidth{0.330157pt}%
\definecolor{currentstroke}{rgb}{0.272594,0.025563,0.353093}%
\pgfsetstrokecolor{currentstroke}%
\pgfsetdash{}{0pt}%
\pgfpathmoveto{\pgfqpoint{2.796698in}{4.559078in}}%
\pgfpathlineto{\pgfqpoint{2.747218in}{4.565881in}}%
\pgfusepath{stroke}%
\end{pgfscope}%
\begin{pgfscope}%
\pgfpathrectangle{\pgfqpoint{1.250000in}{4.155455in}}{\pgfqpoint{2.279412in}{2.004545in}}%
\pgfusepath{clip}%
\pgfsetbuttcap%
\pgfsetroundjoin%
\pgfsetlinewidth{0.348333pt}%
\definecolor{currentstroke}{rgb}{0.274952,0.037752,0.364543}%
\pgfsetstrokecolor{currentstroke}%
\pgfsetdash{}{0pt}%
\pgfpathmoveto{\pgfqpoint{2.747218in}{4.565881in}}%
\pgfpathlineto{\pgfqpoint{2.697457in}{4.571339in}}%
\pgfusepath{stroke}%
\end{pgfscope}%
\begin{pgfscope}%
\pgfpathrectangle{\pgfqpoint{1.250000in}{4.155455in}}{\pgfqpoint{2.279412in}{2.004545in}}%
\pgfusepath{clip}%
\pgfsetbuttcap%
\pgfsetroundjoin%
\pgfsetlinewidth{0.355939pt}%
\definecolor{currentstroke}{rgb}{0.276022,0.044167,0.370164}%
\pgfsetstrokecolor{currentstroke}%
\pgfsetdash{}{0pt}%
\pgfpathmoveto{\pgfqpoint{2.697457in}{4.571339in}}%
\pgfpathlineto{\pgfqpoint{2.647872in}{4.577888in}}%
\pgfusepath{stroke}%
\end{pgfscope}%
\begin{pgfscope}%
\pgfpathrectangle{\pgfqpoint{1.250000in}{4.155455in}}{\pgfqpoint{2.279412in}{2.004545in}}%
\pgfusepath{clip}%
\pgfsetbuttcap%
\pgfsetroundjoin%
\pgfsetlinewidth{0.372349pt}%
\definecolor{currentstroke}{rgb}{0.278791,0.062145,0.386592}%
\pgfsetstrokecolor{currentstroke}%
\pgfsetdash{}{0pt}%
\pgfpathmoveto{\pgfqpoint{2.647872in}{4.577888in}}%
\pgfpathlineto{\pgfqpoint{2.598440in}{4.585269in}}%
\pgfusepath{stroke}%
\end{pgfscope}%
\begin{pgfscope}%
\pgfpathrectangle{\pgfqpoint{1.250000in}{4.155455in}}{\pgfqpoint{2.279412in}{2.004545in}}%
\pgfusepath{clip}%
\pgfsetbuttcap%
\pgfsetroundjoin%
\pgfsetlinewidth{0.382761pt}%
\definecolor{currentstroke}{rgb}{0.279566,0.067836,0.391917}%
\pgfsetstrokecolor{currentstroke}%
\pgfsetdash{}{0pt}%
\pgfpathmoveto{\pgfqpoint{2.598440in}{4.585269in}}%
\pgfpathlineto{\pgfqpoint{2.549185in}{4.593496in}}%
\pgfusepath{stroke}%
\end{pgfscope}%
\begin{pgfscope}%
\pgfpathrectangle{\pgfqpoint{1.250000in}{4.155455in}}{\pgfqpoint{2.279412in}{2.004545in}}%
\pgfusepath{clip}%
\pgfsetbuttcap%
\pgfsetroundjoin%
\pgfsetlinewidth{0.406391pt}%
\definecolor{currentstroke}{rgb}{0.281924,0.089666,0.412415}%
\pgfsetstrokecolor{currentstroke}%
\pgfsetdash{}{0pt}%
\pgfpathmoveto{\pgfqpoint{2.549185in}{4.593496in}}%
\pgfpathlineto{\pgfqpoint{2.499995in}{4.601947in}}%
\pgfusepath{stroke}%
\end{pgfscope}%
\begin{pgfscope}%
\pgfpathrectangle{\pgfqpoint{1.250000in}{4.155455in}}{\pgfqpoint{2.279412in}{2.004545in}}%
\pgfusepath{clip}%
\pgfsetbuttcap%
\pgfsetroundjoin%
\pgfsetlinewidth{0.304530pt}%
\definecolor{currentstroke}{rgb}{0.267004,0.004874,0.329415}%
\pgfsetstrokecolor{currentstroke}%
\pgfsetdash{}{0pt}%
\pgfpathmoveto{\pgfqpoint{3.056501in}{4.661553in}}%
\pgfpathlineto{\pgfqpoint{3.011719in}{4.666772in}}%
\pgfusepath{stroke}%
\end{pgfscope}%
\begin{pgfscope}%
\pgfpathrectangle{\pgfqpoint{1.250000in}{4.155455in}}{\pgfqpoint{2.279412in}{2.004545in}}%
\pgfusepath{clip}%
\pgfsetbuttcap%
\pgfsetroundjoin%
\pgfsetlinewidth{0.323576pt}%
\definecolor{currentstroke}{rgb}{0.271305,0.019942,0.347269}%
\pgfsetstrokecolor{currentstroke}%
\pgfsetdash{}{0pt}%
\pgfpathmoveto{\pgfqpoint{3.011719in}{4.666772in}}%
\pgfpathlineto{\pgfqpoint{2.962316in}{4.672072in}}%
\pgfusepath{stroke}%
\end{pgfscope}%
\begin{pgfscope}%
\pgfpathrectangle{\pgfqpoint{1.250000in}{4.155455in}}{\pgfqpoint{2.279412in}{2.004545in}}%
\pgfusepath{clip}%
\pgfsetbuttcap%
\pgfsetroundjoin%
\pgfsetlinewidth{0.337483pt}%
\definecolor{currentstroke}{rgb}{0.273809,0.031497,0.358853}%
\pgfsetstrokecolor{currentstroke}%
\pgfsetdash{}{0pt}%
\pgfpathmoveto{\pgfqpoint{2.962316in}{4.672072in}}%
\pgfpathlineto{\pgfqpoint{2.912615in}{4.677244in}}%
\pgfusepath{stroke}%
\end{pgfscope}%
\begin{pgfscope}%
\pgfpathrectangle{\pgfqpoint{1.250000in}{4.155455in}}{\pgfqpoint{2.279412in}{2.004545in}}%
\pgfusepath{clip}%
\pgfsetbuttcap%
\pgfsetroundjoin%
\pgfsetlinewidth{0.345754pt}%
\definecolor{currentstroke}{rgb}{0.274952,0.037752,0.364543}%
\pgfsetstrokecolor{currentstroke}%
\pgfsetdash{}{0pt}%
\pgfpathmoveto{\pgfqpoint{2.912615in}{4.677244in}}%
\pgfpathlineto{\pgfqpoint{2.862856in}{4.682631in}}%
\pgfusepath{stroke}%
\end{pgfscope}%
\begin{pgfscope}%
\pgfpathrectangle{\pgfqpoint{1.250000in}{4.155455in}}{\pgfqpoint{2.279412in}{2.004545in}}%
\pgfusepath{clip}%
\pgfsetbuttcap%
\pgfsetroundjoin%
\pgfsetlinewidth{0.340239pt}%
\definecolor{currentstroke}{rgb}{0.273809,0.031497,0.358853}%
\pgfsetstrokecolor{currentstroke}%
\pgfsetdash{}{0pt}%
\pgfpathmoveto{\pgfqpoint{2.862856in}{4.682631in}}%
\pgfpathlineto{\pgfqpoint{2.813143in}{4.688412in}}%
\pgfusepath{stroke}%
\end{pgfscope}%
\begin{pgfscope}%
\pgfpathrectangle{\pgfqpoint{1.250000in}{4.155455in}}{\pgfqpoint{2.279412in}{2.004545in}}%
\pgfusepath{clip}%
\pgfsetbuttcap%
\pgfsetroundjoin%
\pgfsetlinewidth{0.377104pt}%
\definecolor{currentstroke}{rgb}{0.279566,0.067836,0.391917}%
\pgfsetstrokecolor{currentstroke}%
\pgfsetdash{}{0pt}%
\pgfpathmoveto{\pgfqpoint{2.813143in}{4.688412in}}%
\pgfpathlineto{\pgfqpoint{2.763384in}{4.693882in}}%
\pgfusepath{stroke}%
\end{pgfscope}%
\begin{pgfscope}%
\pgfpathrectangle{\pgfqpoint{1.250000in}{4.155455in}}{\pgfqpoint{2.279412in}{2.004545in}}%
\pgfusepath{clip}%
\pgfsetbuttcap%
\pgfsetroundjoin%
\pgfsetlinewidth{0.389555pt}%
\definecolor{currentstroke}{rgb}{0.280267,0.073417,0.397163}%
\pgfsetstrokecolor{currentstroke}%
\pgfsetdash{}{0pt}%
\pgfpathmoveto{\pgfqpoint{2.763384in}{4.693882in}}%
\pgfpathlineto{\pgfqpoint{2.713761in}{4.700197in}}%
\pgfusepath{stroke}%
\end{pgfscope}%
\begin{pgfscope}%
\pgfpathrectangle{\pgfqpoint{1.250000in}{4.155455in}}{\pgfqpoint{2.279412in}{2.004545in}}%
\pgfusepath{clip}%
\pgfsetbuttcap%
\pgfsetroundjoin%
\pgfsetlinewidth{0.420901pt}%
\definecolor{currentstroke}{rgb}{0.282656,0.100196,0.422160}%
\pgfsetstrokecolor{currentstroke}%
\pgfsetdash{}{0pt}%
\pgfpathmoveto{\pgfqpoint{2.713761in}{4.700197in}}%
\pgfpathlineto{\pgfqpoint{2.664266in}{4.707301in}}%
\pgfusepath{stroke}%
\end{pgfscope}%
\begin{pgfscope}%
\pgfpathrectangle{\pgfqpoint{1.250000in}{4.155455in}}{\pgfqpoint{2.279412in}{2.004545in}}%
\pgfusepath{clip}%
\pgfsetbuttcap%
\pgfsetroundjoin%
\pgfsetlinewidth{0.441219pt}%
\definecolor{currentstroke}{rgb}{0.283197,0.115680,0.436115}%
\pgfsetstrokecolor{currentstroke}%
\pgfsetdash{}{0pt}%
\pgfpathmoveto{\pgfqpoint{2.664266in}{4.707301in}}%
\pgfpathlineto{\pgfqpoint{2.614795in}{4.714535in}}%
\pgfusepath{stroke}%
\end{pgfscope}%
\begin{pgfscope}%
\pgfpathrectangle{\pgfqpoint{1.250000in}{4.155455in}}{\pgfqpoint{2.279412in}{2.004545in}}%
\pgfusepath{clip}%
\pgfsetbuttcap%
\pgfsetroundjoin%
\pgfsetlinewidth{0.472823pt}%
\definecolor{currentstroke}{rgb}{0.282623,0.140926,0.457517}%
\pgfsetstrokecolor{currentstroke}%
\pgfsetdash{}{0pt}%
\pgfpathmoveto{\pgfqpoint{2.614795in}{4.714535in}}%
\pgfpathlineto{\pgfqpoint{2.565468in}{4.722467in}}%
\pgfusepath{stroke}%
\end{pgfscope}%
\begin{pgfscope}%
\pgfpathrectangle{\pgfqpoint{1.250000in}{4.155455in}}{\pgfqpoint{2.279412in}{2.004545in}}%
\pgfusepath{clip}%
\pgfsetbuttcap%
\pgfsetroundjoin%
\pgfsetlinewidth{0.499202pt}%
\definecolor{currentstroke}{rgb}{0.281412,0.155834,0.469201}%
\pgfsetstrokecolor{currentstroke}%
\pgfsetdash{}{0pt}%
\pgfpathmoveto{\pgfqpoint{2.565468in}{4.722467in}}%
\pgfpathlineto{\pgfqpoint{2.516377in}{4.731453in}}%
\pgfusepath{stroke}%
\end{pgfscope}%
\begin{pgfscope}%
\pgfpathrectangle{\pgfqpoint{1.250000in}{4.155455in}}{\pgfqpoint{2.279412in}{2.004545in}}%
\pgfusepath{clip}%
\pgfsetbuttcap%
\pgfsetroundjoin%
\pgfsetlinewidth{0.527259pt}%
\definecolor{currentstroke}{rgb}{0.278826,0.175490,0.483397}%
\pgfsetstrokecolor{currentstroke}%
\pgfsetdash{}{0pt}%
\pgfpathmoveto{\pgfqpoint{2.516377in}{4.731453in}}%
\pgfpathlineto{\pgfqpoint{2.467713in}{4.742054in}}%
\pgfusepath{stroke}%
\end{pgfscope}%
\begin{pgfscope}%
\pgfpathrectangle{\pgfqpoint{1.250000in}{4.155455in}}{\pgfqpoint{2.279412in}{2.004545in}}%
\pgfusepath{clip}%
\pgfsetbuttcap%
\pgfsetroundjoin%
\pgfsetlinewidth{0.495905pt}%
\definecolor{currentstroke}{rgb}{0.281412,0.155834,0.469201}%
\pgfsetstrokecolor{currentstroke}%
\pgfsetdash{}{0pt}%
\pgfpathmoveto{\pgfqpoint{2.467713in}{4.742054in}}%
\pgfpathlineto{\pgfqpoint{2.419548in}{4.754310in}}%
\pgfusepath{stroke}%
\end{pgfscope}%
\begin{pgfscope}%
\pgfpathrectangle{\pgfqpoint{1.250000in}{4.155455in}}{\pgfqpoint{2.279412in}{2.004545in}}%
\pgfusepath{clip}%
\pgfsetbuttcap%
\pgfsetroundjoin%
\pgfsetlinewidth{0.545531pt}%
\definecolor{currentstroke}{rgb}{0.276194,0.190074,0.493001}%
\pgfsetstrokecolor{currentstroke}%
\pgfsetdash{}{0pt}%
\pgfpathmoveto{\pgfqpoint{2.419548in}{4.754310in}}%
\pgfpathlineto{\pgfqpoint{2.372274in}{4.768898in}}%
\pgfusepath{stroke}%
\end{pgfscope}%
\begin{pgfscope}%
\pgfpathrectangle{\pgfqpoint{1.250000in}{4.155455in}}{\pgfqpoint{2.279412in}{2.004545in}}%
\pgfusepath{clip}%
\pgfsetbuttcap%
\pgfsetroundjoin%
\pgfsetlinewidth{0.523651pt}%
\definecolor{currentstroke}{rgb}{0.278826,0.175490,0.483397}%
\pgfsetstrokecolor{currentstroke}%
\pgfsetdash{}{0pt}%
\pgfpathmoveto{\pgfqpoint{2.372274in}{4.768898in}}%
\pgfpathlineto{\pgfqpoint{2.327256in}{4.788029in}}%
\pgfusepath{stroke}%
\end{pgfscope}%
\begin{pgfscope}%
\pgfpathrectangle{\pgfqpoint{1.250000in}{4.155455in}}{\pgfqpoint{2.279412in}{2.004545in}}%
\pgfusepath{clip}%
\pgfsetbuttcap%
\pgfsetroundjoin%
\pgfsetlinewidth{0.542862pt}%
\definecolor{currentstroke}{rgb}{0.276194,0.190074,0.493001}%
\pgfsetstrokecolor{currentstroke}%
\pgfsetdash{}{0pt}%
\pgfpathmoveto{\pgfqpoint{2.327256in}{4.788029in}}%
\pgfpathlineto{\pgfqpoint{2.285490in}{4.812073in}}%
\pgfusepath{stroke}%
\end{pgfscope}%
\begin{pgfscope}%
\pgfpathrectangle{\pgfqpoint{1.250000in}{4.155455in}}{\pgfqpoint{2.279412in}{2.004545in}}%
\pgfusepath{clip}%
\pgfsetbuttcap%
\pgfsetroundjoin%
\pgfsetlinewidth{0.540651pt}%
\definecolor{currentstroke}{rgb}{0.277134,0.185228,0.489898}%
\pgfsetstrokecolor{currentstroke}%
\pgfsetdash{}{0pt}%
\pgfpathmoveto{\pgfqpoint{2.285490in}{4.812073in}}%
\pgfpathlineto{\pgfqpoint{2.247408in}{4.840351in}}%
\pgfusepath{stroke}%
\end{pgfscope}%
\begin{pgfscope}%
\pgfpathrectangle{\pgfqpoint{1.250000in}{4.155455in}}{\pgfqpoint{2.279412in}{2.004545in}}%
\pgfusepath{clip}%
\pgfsetbuttcap%
\pgfsetroundjoin%
\pgfsetlinewidth{0.630094pt}%
\definecolor{currentstroke}{rgb}{0.260571,0.246922,0.522828}%
\pgfsetstrokecolor{currentstroke}%
\pgfsetdash{}{0pt}%
\pgfpathmoveto{\pgfqpoint{2.247408in}{4.840351in}}%
\pgfpathlineto{\pgfqpoint{2.247408in}{4.840351in}}%
\pgfusepath{stroke}%
\end{pgfscope}%
\begin{pgfscope}%
\pgfpathrectangle{\pgfqpoint{1.250000in}{4.155455in}}{\pgfqpoint{2.279412in}{2.004545in}}%
\pgfusepath{clip}%
\pgfsetbuttcap%
\pgfsetroundjoin%
\pgfsetlinewidth{0.331395pt}%
\definecolor{currentstroke}{rgb}{0.272594,0.025563,0.353093}%
\pgfsetstrokecolor{currentstroke}%
\pgfsetdash{}{0pt}%
\pgfpathmoveto{\pgfqpoint{3.056501in}{5.518582in}}%
\pgfpathlineto{\pgfqpoint{3.006459in}{5.515801in}}%
\pgfusepath{stroke}%
\end{pgfscope}%
\begin{pgfscope}%
\pgfpathrectangle{\pgfqpoint{1.250000in}{4.155455in}}{\pgfqpoint{2.279412in}{2.004545in}}%
\pgfusepath{clip}%
\pgfsetbuttcap%
\pgfsetroundjoin%
\pgfsetlinewidth{0.340862pt}%
\definecolor{currentstroke}{rgb}{0.273809,0.031497,0.358853}%
\pgfsetstrokecolor{currentstroke}%
\pgfsetdash{}{0pt}%
\pgfpathmoveto{\pgfqpoint{3.006459in}{5.515801in}}%
\pgfpathlineto{\pgfqpoint{2.956423in}{5.512840in}}%
\pgfusepath{stroke}%
\end{pgfscope}%
\begin{pgfscope}%
\pgfpathrectangle{\pgfqpoint{1.250000in}{4.155455in}}{\pgfqpoint{2.279412in}{2.004545in}}%
\pgfusepath{clip}%
\pgfsetbuttcap%
\pgfsetroundjoin%
\pgfsetlinewidth{0.359491pt}%
\definecolor{currentstroke}{rgb}{0.277018,0.050344,0.375715}%
\pgfsetstrokecolor{currentstroke}%
\pgfsetdash{}{0pt}%
\pgfpathmoveto{\pgfqpoint{2.956423in}{5.512840in}}%
\pgfpathlineto{\pgfqpoint{2.906426in}{5.509426in}}%
\pgfusepath{stroke}%
\end{pgfscope}%
\begin{pgfscope}%
\pgfpathrectangle{\pgfqpoint{1.250000in}{4.155455in}}{\pgfqpoint{2.279412in}{2.004545in}}%
\pgfusepath{clip}%
\pgfsetbuttcap%
\pgfsetroundjoin%
\pgfsetlinewidth{0.367463pt}%
\definecolor{currentstroke}{rgb}{0.277941,0.056324,0.381191}%
\pgfsetstrokecolor{currentstroke}%
\pgfsetdash{}{0pt}%
\pgfpathmoveto{\pgfqpoint{2.906426in}{5.509426in}}%
\pgfpathlineto{\pgfqpoint{2.856487in}{5.505398in}}%
\pgfusepath{stroke}%
\end{pgfscope}%
\begin{pgfscope}%
\pgfpathrectangle{\pgfqpoint{1.250000in}{4.155455in}}{\pgfqpoint{2.279412in}{2.004545in}}%
\pgfusepath{clip}%
\pgfsetbuttcap%
\pgfsetroundjoin%
\pgfsetlinewidth{0.400774pt}%
\definecolor{currentstroke}{rgb}{0.281446,0.084320,0.407414}%
\pgfsetstrokecolor{currentstroke}%
\pgfsetdash{}{0pt}%
\pgfpathmoveto{\pgfqpoint{2.856487in}{5.505398in}}%
\pgfpathlineto{\pgfqpoint{2.806562in}{5.501227in}}%
\pgfusepath{stroke}%
\end{pgfscope}%
\begin{pgfscope}%
\pgfpathrectangle{\pgfqpoint{1.250000in}{4.155455in}}{\pgfqpoint{2.279412in}{2.004545in}}%
\pgfusepath{clip}%
\pgfsetbuttcap%
\pgfsetroundjoin%
\pgfsetlinewidth{0.427710pt}%
\definecolor{currentstroke}{rgb}{0.282910,0.105393,0.426902}%
\pgfsetstrokecolor{currentstroke}%
\pgfsetdash{}{0pt}%
\pgfpathmoveto{\pgfqpoint{2.806562in}{5.501227in}}%
\pgfpathlineto{\pgfqpoint{2.756636in}{5.497053in}}%
\pgfusepath{stroke}%
\end{pgfscope}%
\begin{pgfscope}%
\pgfpathrectangle{\pgfqpoint{1.250000in}{4.155455in}}{\pgfqpoint{2.279412in}{2.004545in}}%
\pgfusepath{clip}%
\pgfsetbuttcap%
\pgfsetroundjoin%
\pgfsetlinewidth{0.462036pt}%
\definecolor{currentstroke}{rgb}{0.283072,0.130895,0.449241}%
\pgfsetstrokecolor{currentstroke}%
\pgfsetdash{}{0pt}%
\pgfpathmoveto{\pgfqpoint{2.756636in}{5.497053in}}%
\pgfpathlineto{\pgfqpoint{2.706750in}{5.492539in}}%
\pgfusepath{stroke}%
\end{pgfscope}%
\begin{pgfscope}%
\pgfpathrectangle{\pgfqpoint{1.250000in}{4.155455in}}{\pgfqpoint{2.279412in}{2.004545in}}%
\pgfusepath{clip}%
\pgfsetbuttcap%
\pgfsetroundjoin%
\pgfsetlinewidth{0.507614pt}%
\definecolor{currentstroke}{rgb}{0.280255,0.165693,0.476498}%
\pgfsetstrokecolor{currentstroke}%
\pgfsetdash{}{0pt}%
\pgfpathmoveto{\pgfqpoint{2.706750in}{5.492539in}}%
\pgfpathlineto{\pgfqpoint{2.656909in}{5.487641in}}%
\pgfusepath{stroke}%
\end{pgfscope}%
\begin{pgfscope}%
\pgfpathrectangle{\pgfqpoint{1.250000in}{4.155455in}}{\pgfqpoint{2.279412in}{2.004545in}}%
\pgfusepath{clip}%
\pgfsetbuttcap%
\pgfsetroundjoin%
\pgfsetlinewidth{0.560449pt}%
\definecolor{currentstroke}{rgb}{0.274128,0.199721,0.498911}%
\pgfsetstrokecolor{currentstroke}%
\pgfsetdash{}{0pt}%
\pgfpathmoveto{\pgfqpoint{2.656909in}{5.487641in}}%
\pgfpathlineto{\pgfqpoint{2.607144in}{5.482209in}}%
\pgfusepath{stroke}%
\end{pgfscope}%
\begin{pgfscope}%
\pgfpathrectangle{\pgfqpoint{1.250000in}{4.155455in}}{\pgfqpoint{2.279412in}{2.004545in}}%
\pgfusepath{clip}%
\pgfsetbuttcap%
\pgfsetroundjoin%
\pgfsetlinewidth{0.577294pt}%
\definecolor{currentstroke}{rgb}{0.270595,0.214069,0.507052}%
\pgfsetstrokecolor{currentstroke}%
\pgfsetdash{}{0pt}%
\pgfpathmoveto{\pgfqpoint{2.607144in}{5.482209in}}%
\pgfpathlineto{\pgfqpoint{2.557478in}{5.476083in}}%
\pgfusepath{stroke}%
\end{pgfscope}%
\begin{pgfscope}%
\pgfpathrectangle{\pgfqpoint{1.250000in}{4.155455in}}{\pgfqpoint{2.279412in}{2.004545in}}%
\pgfusepath{clip}%
\pgfsetbuttcap%
\pgfsetroundjoin%
\pgfsetlinewidth{0.610231pt}%
\definecolor{currentstroke}{rgb}{0.263663,0.237631,0.518762}%
\pgfsetstrokecolor{currentstroke}%
\pgfsetdash{}{0pt}%
\pgfpathmoveto{\pgfqpoint{2.557478in}{5.476083in}}%
\pgfpathlineto{\pgfqpoint{2.507896in}{5.469463in}}%
\pgfusepath{stroke}%
\end{pgfscope}%
\begin{pgfscope}%
\pgfpathrectangle{\pgfqpoint{1.250000in}{4.155455in}}{\pgfqpoint{2.279412in}{2.004545in}}%
\pgfusepath{clip}%
\pgfsetbuttcap%
\pgfsetroundjoin%
\pgfsetlinewidth{0.627557pt}%
\definecolor{currentstroke}{rgb}{0.260571,0.246922,0.522828}%
\pgfsetstrokecolor{currentstroke}%
\pgfsetdash{}{0pt}%
\pgfpathmoveto{\pgfqpoint{2.507896in}{5.469463in}}%
\pgfpathlineto{\pgfqpoint{2.458536in}{5.461705in}}%
\pgfusepath{stroke}%
\end{pgfscope}%
\begin{pgfscope}%
\pgfpathrectangle{\pgfqpoint{1.250000in}{4.155455in}}{\pgfqpoint{2.279412in}{2.004545in}}%
\pgfusepath{clip}%
\pgfsetbuttcap%
\pgfsetroundjoin%
\pgfsetlinewidth{0.652811pt}%
\definecolor{currentstroke}{rgb}{0.253935,0.265254,0.529983}%
\pgfsetstrokecolor{currentstroke}%
\pgfsetdash{}{0pt}%
\pgfpathmoveto{\pgfqpoint{2.458536in}{5.461705in}}%
\pgfpathlineto{\pgfqpoint{2.409513in}{5.452441in}}%
\pgfusepath{stroke}%
\end{pgfscope}%
\begin{pgfscope}%
\pgfpathrectangle{\pgfqpoint{1.250000in}{4.155455in}}{\pgfqpoint{2.279412in}{2.004545in}}%
\pgfusepath{clip}%
\pgfsetbuttcap%
\pgfsetroundjoin%
\pgfsetlinewidth{0.680768pt}%
\definecolor{currentstroke}{rgb}{0.246811,0.283237,0.535941}%
\pgfsetstrokecolor{currentstroke}%
\pgfsetdash{}{0pt}%
\pgfpathmoveto{\pgfqpoint{2.409513in}{5.452441in}}%
\pgfpathlineto{\pgfqpoint{2.361116in}{5.441025in}}%
\pgfusepath{stroke}%
\end{pgfscope}%
\begin{pgfscope}%
\pgfpathrectangle{\pgfqpoint{1.250000in}{4.155455in}}{\pgfqpoint{2.279412in}{2.004545in}}%
\pgfusepath{clip}%
\pgfsetbuttcap%
\pgfsetroundjoin%
\pgfsetlinewidth{0.662322pt}%
\definecolor{currentstroke}{rgb}{0.252194,0.269783,0.531579}%
\pgfsetstrokecolor{currentstroke}%
\pgfsetdash{}{0pt}%
\pgfpathmoveto{\pgfqpoint{2.361116in}{5.441025in}}%
\pgfpathlineto{\pgfqpoint{2.313852in}{5.426495in}}%
\pgfusepath{stroke}%
\end{pgfscope}%
\begin{pgfscope}%
\pgfpathrectangle{\pgfqpoint{1.250000in}{4.155455in}}{\pgfqpoint{2.279412in}{2.004545in}}%
\pgfusepath{clip}%
\pgfsetbuttcap%
\pgfsetroundjoin%
\pgfsetlinewidth{0.624273pt}%
\definecolor{currentstroke}{rgb}{0.260571,0.246922,0.522828}%
\pgfsetstrokecolor{currentstroke}%
\pgfsetdash{}{0pt}%
\pgfpathmoveto{\pgfqpoint{2.313852in}{5.426495in}}%
\pgfpathlineto{\pgfqpoint{2.268685in}{5.407714in}}%
\pgfusepath{stroke}%
\end{pgfscope}%
\begin{pgfscope}%
\pgfpathrectangle{\pgfqpoint{1.250000in}{4.155455in}}{\pgfqpoint{2.279412in}{2.004545in}}%
\pgfusepath{clip}%
\pgfsetbuttcap%
\pgfsetroundjoin%
\pgfsetlinewidth{0.623648pt}%
\definecolor{currentstroke}{rgb}{0.260571,0.246922,0.522828}%
\pgfsetstrokecolor{currentstroke}%
\pgfsetdash{}{0pt}%
\pgfpathmoveto{\pgfqpoint{2.268685in}{5.407714in}}%
\pgfpathlineto{\pgfqpoint{2.227273in}{5.383420in}}%
\pgfusepath{stroke}%
\end{pgfscope}%
\begin{pgfscope}%
\pgfpathrectangle{\pgfqpoint{1.250000in}{4.155455in}}{\pgfqpoint{2.279412in}{2.004545in}}%
\pgfusepath{clip}%
\pgfsetbuttcap%
\pgfsetroundjoin%
\pgfsetlinewidth{0.578625pt}%
\definecolor{currentstroke}{rgb}{0.270595,0.214069,0.507052}%
\pgfsetstrokecolor{currentstroke}%
\pgfsetdash{}{0pt}%
\pgfpathmoveto{\pgfqpoint{2.227273in}{5.383420in}}%
\pgfpathlineto{\pgfqpoint{2.227273in}{5.383420in}}%
\pgfusepath{stroke}%
\end{pgfscope}%
\begin{pgfscope}%
\pgfpathrectangle{\pgfqpoint{1.250000in}{4.155455in}}{\pgfqpoint{2.279412in}{2.004545in}}%
\pgfusepath{clip}%
\pgfsetbuttcap%
\pgfsetroundjoin%
\pgfsetlinewidth{0.328277pt}%
\definecolor{currentstroke}{rgb}{0.271305,0.019942,0.347269}%
\pgfsetstrokecolor{currentstroke}%
\pgfsetdash{}{0pt}%
\pgfpathmoveto{\pgfqpoint{3.056501in}{5.563688in}}%
\pgfpathlineto{\pgfqpoint{3.006485in}{5.561576in}}%
\pgfusepath{stroke}%
\end{pgfscope}%
\begin{pgfscope}%
\pgfpathrectangle{\pgfqpoint{1.250000in}{4.155455in}}{\pgfqpoint{2.279412in}{2.004545in}}%
\pgfusepath{clip}%
\pgfsetbuttcap%
\pgfsetroundjoin%
\pgfsetlinewidth{0.331322pt}%
\definecolor{currentstroke}{rgb}{0.272594,0.025563,0.353093}%
\pgfsetstrokecolor{currentstroke}%
\pgfsetdash{}{0pt}%
\pgfpathmoveto{\pgfqpoint{3.006485in}{5.561576in}}%
\pgfpathlineto{\pgfqpoint{2.956444in}{5.559019in}}%
\pgfusepath{stroke}%
\end{pgfscope}%
\begin{pgfscope}%
\pgfpathrectangle{\pgfqpoint{1.250000in}{4.155455in}}{\pgfqpoint{2.279412in}{2.004545in}}%
\pgfusepath{clip}%
\pgfsetbuttcap%
\pgfsetroundjoin%
\pgfsetlinewidth{0.356250pt}%
\definecolor{currentstroke}{rgb}{0.277018,0.050344,0.375715}%
\pgfsetstrokecolor{currentstroke}%
\pgfsetdash{}{0pt}%
\pgfpathmoveto{\pgfqpoint{2.956444in}{5.559019in}}%
\pgfpathlineto{\pgfqpoint{2.906447in}{5.555655in}}%
\pgfusepath{stroke}%
\end{pgfscope}%
\begin{pgfscope}%
\pgfpathrectangle{\pgfqpoint{1.250000in}{4.155455in}}{\pgfqpoint{2.279412in}{2.004545in}}%
\pgfusepath{clip}%
\pgfsetbuttcap%
\pgfsetroundjoin%
\pgfsetlinewidth{0.364894pt}%
\definecolor{currentstroke}{rgb}{0.277941,0.056324,0.381191}%
\pgfsetstrokecolor{currentstroke}%
\pgfsetdash{}{0pt}%
\pgfpathmoveto{\pgfqpoint{2.906447in}{5.555655in}}%
\pgfpathlineto{\pgfqpoint{2.856453in}{5.552170in}}%
\pgfusepath{stroke}%
\end{pgfscope}%
\begin{pgfscope}%
\pgfpathrectangle{\pgfqpoint{1.250000in}{4.155455in}}{\pgfqpoint{2.279412in}{2.004545in}}%
\pgfusepath{clip}%
\pgfsetbuttcap%
\pgfsetroundjoin%
\pgfsetlinewidth{0.386079pt}%
\definecolor{currentstroke}{rgb}{0.280267,0.073417,0.397163}%
\pgfsetstrokecolor{currentstroke}%
\pgfsetdash{}{0pt}%
\pgfpathmoveto{\pgfqpoint{2.856453in}{5.552170in}}%
\pgfpathlineto{\pgfqpoint{2.806533in}{5.547979in}}%
\pgfusepath{stroke}%
\end{pgfscope}%
\begin{pgfscope}%
\pgfpathrectangle{\pgfqpoint{1.250000in}{4.155455in}}{\pgfqpoint{2.279412in}{2.004545in}}%
\pgfusepath{clip}%
\pgfsetbuttcap%
\pgfsetroundjoin%
\pgfsetlinewidth{0.404844pt}%
\definecolor{currentstroke}{rgb}{0.281924,0.089666,0.412415}%
\pgfsetstrokecolor{currentstroke}%
\pgfsetdash{}{0pt}%
\pgfpathmoveto{\pgfqpoint{2.806533in}{5.547979in}}%
\pgfpathlineto{\pgfqpoint{2.756687in}{5.543183in}}%
\pgfusepath{stroke}%
\end{pgfscope}%
\begin{pgfscope}%
\pgfpathrectangle{\pgfqpoint{1.250000in}{4.155455in}}{\pgfqpoint{2.279412in}{2.004545in}}%
\pgfusepath{clip}%
\pgfsetbuttcap%
\pgfsetroundjoin%
\pgfsetlinewidth{0.443362pt}%
\definecolor{currentstroke}{rgb}{0.283197,0.115680,0.436115}%
\pgfsetstrokecolor{currentstroke}%
\pgfsetdash{}{0pt}%
\pgfpathmoveto{\pgfqpoint{2.756687in}{5.543183in}}%
\pgfpathlineto{\pgfqpoint{2.706867in}{5.538171in}}%
\pgfusepath{stroke}%
\end{pgfscope}%
\begin{pgfscope}%
\pgfpathrectangle{\pgfqpoint{1.250000in}{4.155455in}}{\pgfqpoint{2.279412in}{2.004545in}}%
\pgfusepath{clip}%
\pgfsetbuttcap%
\pgfsetroundjoin%
\pgfsetlinewidth{0.468666pt}%
\definecolor{currentstroke}{rgb}{0.282884,0.135920,0.453427}%
\pgfsetstrokecolor{currentstroke}%
\pgfsetdash{}{0pt}%
\pgfpathmoveto{\pgfqpoint{2.706867in}{5.538171in}}%
\pgfpathlineto{\pgfqpoint{2.657052in}{5.533072in}}%
\pgfusepath{stroke}%
\end{pgfscope}%
\begin{pgfscope}%
\pgfpathrectangle{\pgfqpoint{1.250000in}{4.155455in}}{\pgfqpoint{2.279412in}{2.004545in}}%
\pgfusepath{clip}%
\pgfsetbuttcap%
\pgfsetroundjoin%
\pgfsetlinewidth{0.498708pt}%
\definecolor{currentstroke}{rgb}{0.281412,0.155834,0.469201}%
\pgfsetstrokecolor{currentstroke}%
\pgfsetdash{}{0pt}%
\pgfpathmoveto{\pgfqpoint{2.657052in}{5.533072in}}%
\pgfpathlineto{\pgfqpoint{2.607302in}{5.527533in}}%
\pgfusepath{stroke}%
\end{pgfscope}%
\begin{pgfscope}%
\pgfpathrectangle{\pgfqpoint{1.250000in}{4.155455in}}{\pgfqpoint{2.279412in}{2.004545in}}%
\pgfusepath{clip}%
\pgfsetbuttcap%
\pgfsetroundjoin%
\pgfsetlinewidth{0.534302pt}%
\definecolor{currentstroke}{rgb}{0.278012,0.180367,0.486697}%
\pgfsetstrokecolor{currentstroke}%
\pgfsetdash{}{0pt}%
\pgfpathmoveto{\pgfqpoint{2.607302in}{5.527533in}}%
\pgfpathlineto{\pgfqpoint{2.557675in}{5.521216in}}%
\pgfusepath{stroke}%
\end{pgfscope}%
\begin{pgfscope}%
\pgfpathrectangle{\pgfqpoint{1.250000in}{4.155455in}}{\pgfqpoint{2.279412in}{2.004545in}}%
\pgfusepath{clip}%
\pgfsetbuttcap%
\pgfsetroundjoin%
\pgfsetlinewidth{0.550647pt}%
\definecolor{currentstroke}{rgb}{0.275191,0.194905,0.496005}%
\pgfsetstrokecolor{currentstroke}%
\pgfsetdash{}{0pt}%
\pgfpathmoveto{\pgfqpoint{2.557675in}{5.521216in}}%
\pgfpathlineto{\pgfqpoint{2.508201in}{5.514045in}}%
\pgfusepath{stroke}%
\end{pgfscope}%
\begin{pgfscope}%
\pgfpathrectangle{\pgfqpoint{1.250000in}{4.155455in}}{\pgfqpoint{2.279412in}{2.004545in}}%
\pgfusepath{clip}%
\pgfsetbuttcap%
\pgfsetroundjoin%
\pgfsetlinewidth{0.571081pt}%
\definecolor{currentstroke}{rgb}{0.271828,0.209303,0.504434}%
\pgfsetstrokecolor{currentstroke}%
\pgfsetdash{}{0pt}%
\pgfpathmoveto{\pgfqpoint{2.508201in}{5.514045in}}%
\pgfpathlineto{\pgfqpoint{2.459130in}{5.505026in}}%
\pgfusepath{stroke}%
\end{pgfscope}%
\begin{pgfscope}%
\pgfpathrectangle{\pgfqpoint{1.250000in}{4.155455in}}{\pgfqpoint{2.279412in}{2.004545in}}%
\pgfusepath{clip}%
\pgfsetbuttcap%
\pgfsetroundjoin%
\pgfsetlinewidth{0.342589pt}%
\definecolor{currentstroke}{rgb}{0.274952,0.037752,0.364543}%
\pgfsetstrokecolor{currentstroke}%
\pgfsetdash{}{0pt}%
\pgfpathmoveto{\pgfqpoint{2.800041in}{5.744115in}}%
\pgfpathlineto{\pgfqpoint{2.750177in}{5.739815in}}%
\pgfusepath{stroke}%
\end{pgfscope}%
\begin{pgfscope}%
\pgfpathrectangle{\pgfqpoint{1.250000in}{4.155455in}}{\pgfqpoint{2.279412in}{2.004545in}}%
\pgfusepath{clip}%
\pgfsetbuttcap%
\pgfsetroundjoin%
\pgfsetlinewidth{0.360991pt}%
\definecolor{currentstroke}{rgb}{0.277018,0.050344,0.375715}%
\pgfsetstrokecolor{currentstroke}%
\pgfsetdash{}{0pt}%
\pgfpathmoveto{\pgfqpoint{2.750177in}{5.739815in}}%
\pgfpathlineto{\pgfqpoint{2.701161in}{5.731552in}}%
\pgfusepath{stroke}%
\end{pgfscope}%
\begin{pgfscope}%
\pgfpathrectangle{\pgfqpoint{1.250000in}{4.155455in}}{\pgfqpoint{2.279412in}{2.004545in}}%
\pgfusepath{clip}%
\pgfsetbuttcap%
\pgfsetroundjoin%
\pgfsetlinewidth{0.349347pt}%
\definecolor{currentstroke}{rgb}{0.276022,0.044167,0.370164}%
\pgfsetstrokecolor{currentstroke}%
\pgfsetdash{}{0pt}%
\pgfpathmoveto{\pgfqpoint{2.701161in}{5.731552in}}%
\pgfpathlineto{\pgfqpoint{2.652307in}{5.722121in}}%
\pgfusepath{stroke}%
\end{pgfscope}%
\begin{pgfscope}%
\pgfpathrectangle{\pgfqpoint{1.250000in}{4.155455in}}{\pgfqpoint{2.279412in}{2.004545in}}%
\pgfusepath{clip}%
\pgfsetbuttcap%
\pgfsetroundjoin%
\pgfsetlinewidth{0.378044pt}%
\definecolor{currentstroke}{rgb}{0.279566,0.067836,0.391917}%
\pgfsetstrokecolor{currentstroke}%
\pgfsetdash{}{0pt}%
\pgfpathmoveto{\pgfqpoint{2.652307in}{5.722121in}}%
\pgfpathlineto{\pgfqpoint{2.603064in}{5.713942in}}%
\pgfusepath{stroke}%
\end{pgfscope}%
\begin{pgfscope}%
\pgfpathrectangle{\pgfqpoint{1.250000in}{4.155455in}}{\pgfqpoint{2.279412in}{2.004545in}}%
\pgfusepath{clip}%
\pgfsetbuttcap%
\pgfsetroundjoin%
\pgfsetlinewidth{0.384910pt}%
\definecolor{currentstroke}{rgb}{0.280267,0.073417,0.397163}%
\pgfsetstrokecolor{currentstroke}%
\pgfsetdash{}{0pt}%
\pgfpathmoveto{\pgfqpoint{2.603064in}{5.713942in}}%
\pgfpathlineto{\pgfqpoint{2.553997in}{5.704841in}}%
\pgfusepath{stroke}%
\end{pgfscope}%
\begin{pgfscope}%
\pgfpathrectangle{\pgfqpoint{1.250000in}{4.155455in}}{\pgfqpoint{2.279412in}{2.004545in}}%
\pgfusepath{clip}%
\pgfsetbuttcap%
\pgfsetroundjoin%
\pgfsetlinewidth{0.396258pt}%
\definecolor{currentstroke}{rgb}{0.280894,0.078907,0.402329}%
\pgfsetstrokecolor{currentstroke}%
\pgfsetdash{}{0pt}%
\pgfpathmoveto{\pgfqpoint{2.553997in}{5.704841in}}%
\pgfpathlineto{\pgfqpoint{2.505199in}{5.694770in}}%
\pgfusepath{stroke}%
\end{pgfscope}%
\begin{pgfscope}%
\pgfpathrectangle{\pgfqpoint{1.250000in}{4.155455in}}{\pgfqpoint{2.279412in}{2.004545in}}%
\pgfusepath{clip}%
\pgfsetbuttcap%
\pgfsetroundjoin%
\pgfsetlinewidth{0.404512pt}%
\definecolor{currentstroke}{rgb}{0.281924,0.089666,0.412415}%
\pgfsetstrokecolor{currentstroke}%
\pgfsetdash{}{0pt}%
\pgfpathmoveto{\pgfqpoint{2.505199in}{5.694770in}}%
\pgfpathlineto{\pgfqpoint{2.456904in}{5.682898in}}%
\pgfusepath{stroke}%
\end{pgfscope}%
\begin{pgfscope}%
\pgfpathrectangle{\pgfqpoint{1.250000in}{4.155455in}}{\pgfqpoint{2.279412in}{2.004545in}}%
\pgfusepath{clip}%
\pgfsetbuttcap%
\pgfsetroundjoin%
\pgfsetlinewidth{0.409619pt}%
\definecolor{currentstroke}{rgb}{0.281924,0.089666,0.412415}%
\pgfsetstrokecolor{currentstroke}%
\pgfsetdash{}{0pt}%
\pgfpathmoveto{\pgfqpoint{2.456904in}{5.682898in}}%
\pgfpathlineto{\pgfqpoint{2.409339in}{5.669036in}}%
\pgfusepath{stroke}%
\end{pgfscope}%
\begin{pgfscope}%
\pgfpathrectangle{\pgfqpoint{1.250000in}{4.155455in}}{\pgfqpoint{2.279412in}{2.004545in}}%
\pgfusepath{clip}%
\pgfsetbuttcap%
\pgfsetroundjoin%
\pgfsetlinewidth{0.417164pt}%
\definecolor{currentstroke}{rgb}{0.282327,0.094955,0.417331}%
\pgfsetstrokecolor{currentstroke}%
\pgfsetdash{}{0pt}%
\pgfpathmoveto{\pgfqpoint{2.409339in}{5.669036in}}%
\pgfpathlineto{\pgfqpoint{2.362685in}{5.652917in}}%
\pgfusepath{stroke}%
\end{pgfscope}%
\begin{pgfscope}%
\pgfpathrectangle{\pgfqpoint{1.250000in}{4.155455in}}{\pgfqpoint{2.279412in}{2.004545in}}%
\pgfusepath{clip}%
\pgfsetbuttcap%
\pgfsetroundjoin%
\pgfsetlinewidth{0.446092pt}%
\definecolor{currentstroke}{rgb}{0.283229,0.120777,0.440584}%
\pgfsetstrokecolor{currentstroke}%
\pgfsetdash{}{0pt}%
\pgfpathmoveto{\pgfqpoint{2.362685in}{5.652917in}}%
\pgfpathlineto{\pgfqpoint{2.318392in}{5.632632in}}%
\pgfusepath{stroke}%
\end{pgfscope}%
\begin{pgfscope}%
\pgfpathrectangle{\pgfqpoint{1.250000in}{4.155455in}}{\pgfqpoint{2.279412in}{2.004545in}}%
\pgfusepath{clip}%
\pgfsetbuttcap%
\pgfsetroundjoin%
\pgfsetlinewidth{0.463586pt}%
\definecolor{currentstroke}{rgb}{0.283072,0.130895,0.449241}%
\pgfsetstrokecolor{currentstroke}%
\pgfsetdash{}{0pt}%
\pgfpathmoveto{\pgfqpoint{2.318392in}{5.632632in}}%
\pgfpathlineto{\pgfqpoint{2.279342in}{5.605522in}}%
\pgfusepath{stroke}%
\end{pgfscope}%
\begin{pgfscope}%
\pgfpathrectangle{\pgfqpoint{1.250000in}{4.155455in}}{\pgfqpoint{2.279412in}{2.004545in}}%
\pgfusepath{clip}%
\pgfsetbuttcap%
\pgfsetroundjoin%
\pgfsetlinewidth{0.462631pt}%
\definecolor{currentstroke}{rgb}{0.283072,0.130895,0.449241}%
\pgfsetstrokecolor{currentstroke}%
\pgfsetdash{}{0pt}%
\pgfpathmoveto{\pgfqpoint{2.279342in}{5.605522in}}%
\pgfpathlineto{\pgfqpoint{2.279342in}{5.605522in}}%
\pgfusepath{stroke}%
\end{pgfscope}%
\begin{pgfscope}%
\pgfpathrectangle{\pgfqpoint{1.250000in}{4.155455in}}{\pgfqpoint{2.279412in}{2.004545in}}%
\pgfusepath{clip}%
\pgfsetbuttcap%
\pgfsetroundjoin%
\pgfsetlinewidth{0.462631pt}%
\definecolor{currentstroke}{rgb}{0.283072,0.130895,0.449241}%
\pgfsetstrokecolor{currentstroke}%
\pgfsetdash{}{0pt}%
\pgfpathmoveto{\pgfqpoint{2.279342in}{5.605522in}}%
\pgfpathlineto{\pgfqpoint{2.259276in}{5.580812in}}%
\pgfusepath{stroke}%
\end{pgfscope}%
\begin{pgfscope}%
\pgfpathrectangle{\pgfqpoint{1.250000in}{4.155455in}}{\pgfqpoint{2.279412in}{2.004545in}}%
\pgfusepath{clip}%
\pgfsetbuttcap%
\pgfsetroundjoin%
\pgfsetlinewidth{0.498823pt}%
\definecolor{currentstroke}{rgb}{0.281412,0.155834,0.469201}%
\pgfsetstrokecolor{currentstroke}%
\pgfsetdash{}{0pt}%
\pgfpathmoveto{\pgfqpoint{2.259276in}{5.580812in}}%
\pgfpathlineto{\pgfqpoint{2.244517in}{5.557811in}}%
\pgfusepath{stroke}%
\end{pgfscope}%
\begin{pgfscope}%
\pgfpathrectangle{\pgfqpoint{1.250000in}{4.155455in}}{\pgfqpoint{2.279412in}{2.004545in}}%
\pgfusepath{clip}%
\pgfsetbuttcap%
\pgfsetroundjoin%
\pgfsetlinewidth{0.490348pt}%
\definecolor{currentstroke}{rgb}{0.281887,0.150881,0.465405}%
\pgfsetstrokecolor{currentstroke}%
\pgfsetdash{}{0pt}%
\pgfpathmoveto{\pgfqpoint{2.244517in}{5.557811in}}%
\pgfpathlineto{\pgfqpoint{2.225922in}{5.521858in}}%
\pgfusepath{stroke}%
\end{pgfscope}%
\begin{pgfscope}%
\pgfpathrectangle{\pgfqpoint{1.250000in}{4.155455in}}{\pgfqpoint{2.279412in}{2.004545in}}%
\pgfusepath{clip}%
\pgfsetbuttcap%
\pgfsetroundjoin%
\pgfsetlinewidth{0.570913pt}%
\definecolor{currentstroke}{rgb}{0.271828,0.209303,0.504434}%
\pgfsetstrokecolor{currentstroke}%
\pgfsetdash{}{0pt}%
\pgfpathmoveto{\pgfqpoint{2.225922in}{5.521858in}}%
\pgfpathlineto{\pgfqpoint{2.207541in}{5.481471in}}%
\pgfusepath{stroke}%
\end{pgfscope}%
\begin{pgfscope}%
\pgfpathrectangle{\pgfqpoint{1.250000in}{4.155455in}}{\pgfqpoint{2.279412in}{2.004545in}}%
\pgfusepath{clip}%
\pgfsetbuttcap%
\pgfsetroundjoin%
\pgfsetlinewidth{0.599937pt}%
\definecolor{currentstroke}{rgb}{0.266580,0.228262,0.514349}%
\pgfsetstrokecolor{currentstroke}%
\pgfsetdash{}{0pt}%
\pgfpathmoveto{\pgfqpoint{2.207541in}{5.481471in}}%
\pgfpathlineto{\pgfqpoint{2.194423in}{5.439652in}}%
\pgfusepath{stroke}%
\end{pgfscope}%
\begin{pgfscope}%
\pgfpathrectangle{\pgfqpoint{1.250000in}{4.155455in}}{\pgfqpoint{2.279412in}{2.004545in}}%
\pgfusepath{clip}%
\pgfsetbuttcap%
\pgfsetroundjoin%
\pgfsetlinewidth{0.577379pt}%
\definecolor{currentstroke}{rgb}{0.270595,0.214069,0.507052}%
\pgfsetstrokecolor{currentstroke}%
\pgfsetdash{}{0pt}%
\pgfpathmoveto{\pgfqpoint{2.194423in}{5.439652in}}%
\pgfpathlineto{\pgfqpoint{2.184065in}{5.398957in}}%
\pgfusepath{stroke}%
\end{pgfscope}%
\begin{pgfscope}%
\pgfpathrectangle{\pgfqpoint{1.250000in}{4.155455in}}{\pgfqpoint{2.279412in}{2.004545in}}%
\pgfusepath{clip}%
\pgfsetbuttcap%
\pgfsetroundjoin%
\pgfsetlinewidth{0.373801pt}%
\definecolor{currentstroke}{rgb}{0.278791,0.062145,0.386592}%
\pgfsetstrokecolor{currentstroke}%
\pgfsetdash{}{0pt}%
\pgfpathmoveto{\pgfqpoint{2.436644in}{4.602607in}}%
\pgfpathlineto{\pgfqpoint{2.389706in}{4.616446in}}%
\pgfusepath{stroke}%
\end{pgfscope}%
\begin{pgfscope}%
\pgfpathrectangle{\pgfqpoint{1.250000in}{4.155455in}}{\pgfqpoint{2.279412in}{2.004545in}}%
\pgfusepath{clip}%
\pgfsetbuttcap%
\pgfsetroundjoin%
\pgfsetlinewidth{0.416665pt}%
\definecolor{currentstroke}{rgb}{0.282327,0.094955,0.417331}%
\pgfsetstrokecolor{currentstroke}%
\pgfsetdash{}{0pt}%
\pgfpathmoveto{\pgfqpoint{2.389706in}{4.616446in}}%
\pgfpathlineto{\pgfqpoint{2.345831in}{4.637314in}}%
\pgfusepath{stroke}%
\end{pgfscope}%
\begin{pgfscope}%
\pgfpathrectangle{\pgfqpoint{1.250000in}{4.155455in}}{\pgfqpoint{2.279412in}{2.004545in}}%
\pgfusepath{clip}%
\pgfsetbuttcap%
\pgfsetroundjoin%
\pgfsetlinewidth{0.412347pt}%
\definecolor{currentstroke}{rgb}{0.282327,0.094955,0.417331}%
\pgfsetstrokecolor{currentstroke}%
\pgfsetdash{}{0pt}%
\pgfpathmoveto{\pgfqpoint{2.345831in}{4.637314in}}%
\pgfpathlineto{\pgfqpoint{2.345831in}{4.637314in}}%
\pgfusepath{stroke}%
\end{pgfscope}%
\begin{pgfscope}%
\pgfpathrectangle{\pgfqpoint{1.250000in}{4.155455in}}{\pgfqpoint{2.279412in}{2.004545in}}%
\pgfusepath{clip}%
\pgfsetbuttcap%
\pgfsetroundjoin%
\pgfsetlinewidth{0.412347pt}%
\definecolor{currentstroke}{rgb}{0.282327,0.094955,0.417331}%
\pgfsetstrokecolor{currentstroke}%
\pgfsetdash{}{0pt}%
\pgfpathmoveto{\pgfqpoint{2.345831in}{4.637314in}}%
\pgfpathlineto{\pgfqpoint{2.313435in}{4.659322in}}%
\pgfusepath{stroke}%
\end{pgfscope}%
\begin{pgfscope}%
\pgfpathrectangle{\pgfqpoint{1.250000in}{4.155455in}}{\pgfqpoint{2.279412in}{2.004545in}}%
\pgfusepath{clip}%
\pgfsetbuttcap%
\pgfsetroundjoin%
\pgfsetlinewidth{0.432400pt}%
\definecolor{currentstroke}{rgb}{0.283091,0.110553,0.431554}%
\pgfsetstrokecolor{currentstroke}%
\pgfsetdash{}{0pt}%
\pgfpathmoveto{\pgfqpoint{2.313435in}{4.659322in}}%
\pgfpathlineto{\pgfqpoint{2.283923in}{4.684851in}}%
\pgfusepath{stroke}%
\end{pgfscope}%
\begin{pgfscope}%
\pgfpathrectangle{\pgfqpoint{1.250000in}{4.155455in}}{\pgfqpoint{2.279412in}{2.004545in}}%
\pgfusepath{clip}%
\pgfsetbuttcap%
\pgfsetroundjoin%
\pgfsetlinewidth{0.453675pt}%
\definecolor{currentstroke}{rgb}{0.283187,0.125848,0.444960}%
\pgfsetstrokecolor{currentstroke}%
\pgfsetdash{}{0pt}%
\pgfpathmoveto{\pgfqpoint{2.283923in}{4.684851in}}%
\pgfpathlineto{\pgfqpoint{2.283923in}{4.684851in}}%
\pgfusepath{stroke}%
\end{pgfscope}%
\begin{pgfscope}%
\pgfpathrectangle{\pgfqpoint{1.250000in}{4.155455in}}{\pgfqpoint{2.279412in}{2.004545in}}%
\pgfusepath{clip}%
\pgfsetbuttcap%
\pgfsetroundjoin%
\pgfsetlinewidth{0.453675pt}%
\definecolor{currentstroke}{rgb}{0.283187,0.125848,0.444960}%
\pgfsetstrokecolor{currentstroke}%
\pgfsetdash{}{0pt}%
\pgfpathmoveto{\pgfqpoint{2.283923in}{4.684851in}}%
\pgfpathlineto{\pgfqpoint{2.260525in}{4.708508in}}%
\pgfusepath{stroke}%
\end{pgfscope}%
\begin{pgfscope}%
\pgfpathrectangle{\pgfqpoint{1.250000in}{4.155455in}}{\pgfqpoint{2.279412in}{2.004545in}}%
\pgfusepath{clip}%
\pgfsetbuttcap%
\pgfsetroundjoin%
\pgfsetlinewidth{0.476667pt}%
\definecolor{currentstroke}{rgb}{0.282623,0.140926,0.457517}%
\pgfsetstrokecolor{currentstroke}%
\pgfsetdash{}{0pt}%
\pgfpathmoveto{\pgfqpoint{2.260525in}{4.708508in}}%
\pgfpathlineto{\pgfqpoint{2.260525in}{4.708508in}}%
\pgfusepath{stroke}%
\end{pgfscope}%
\begin{pgfscope}%
\pgfpathrectangle{\pgfqpoint{1.250000in}{4.155455in}}{\pgfqpoint{2.279412in}{2.004545in}}%
\pgfusepath{clip}%
\pgfsetbuttcap%
\pgfsetroundjoin%
\pgfsetlinewidth{0.476667pt}%
\definecolor{currentstroke}{rgb}{0.282623,0.140926,0.457517}%
\pgfsetstrokecolor{currentstroke}%
\pgfsetdash{}{0pt}%
\pgfpathmoveto{\pgfqpoint{2.260525in}{4.708508in}}%
\pgfpathlineto{\pgfqpoint{2.244865in}{4.734108in}}%
\pgfusepath{stroke}%
\end{pgfscope}%
\begin{pgfscope}%
\pgfpathrectangle{\pgfqpoint{1.250000in}{4.155455in}}{\pgfqpoint{2.279412in}{2.004545in}}%
\pgfusepath{clip}%
\pgfsetbuttcap%
\pgfsetroundjoin%
\pgfsetlinewidth{0.474515pt}%
\definecolor{currentstroke}{rgb}{0.282623,0.140926,0.457517}%
\pgfsetstrokecolor{currentstroke}%
\pgfsetdash{}{0pt}%
\pgfpathmoveto{\pgfqpoint{2.244865in}{4.734108in}}%
\pgfpathlineto{\pgfqpoint{2.231112in}{4.760486in}}%
\pgfusepath{stroke}%
\end{pgfscope}%
\begin{pgfscope}%
\pgfpathrectangle{\pgfqpoint{1.250000in}{4.155455in}}{\pgfqpoint{2.279412in}{2.004545in}}%
\pgfusepath{clip}%
\pgfsetbuttcap%
\pgfsetroundjoin%
\pgfsetlinewidth{0.497693pt}%
\definecolor{currentstroke}{rgb}{0.281412,0.155834,0.469201}%
\pgfsetstrokecolor{currentstroke}%
\pgfsetdash{}{0pt}%
\pgfpathmoveto{\pgfqpoint{2.231112in}{4.760486in}}%
\pgfpathlineto{\pgfqpoint{2.231112in}{4.760486in}}%
\pgfusepath{stroke}%
\end{pgfscope}%
\begin{pgfscope}%
\pgfpathrectangle{\pgfqpoint{1.250000in}{4.155455in}}{\pgfqpoint{2.279412in}{2.004545in}}%
\pgfusepath{clip}%
\pgfsetbuttcap%
\pgfsetroundjoin%
\pgfsetlinewidth{0.497693pt}%
\definecolor{currentstroke}{rgb}{0.281412,0.155834,0.469201}%
\pgfsetstrokecolor{currentstroke}%
\pgfsetdash{}{0pt}%
\pgfpathmoveto{\pgfqpoint{2.231112in}{4.760486in}}%
\pgfpathlineto{\pgfqpoint{2.217456in}{4.790397in}}%
\pgfusepath{stroke}%
\end{pgfscope}%
\begin{pgfscope}%
\pgfpathrectangle{\pgfqpoint{1.250000in}{4.155455in}}{\pgfqpoint{2.279412in}{2.004545in}}%
\pgfusepath{clip}%
\pgfsetbuttcap%
\pgfsetroundjoin%
\pgfsetlinewidth{0.505059pt}%
\definecolor{currentstroke}{rgb}{0.280868,0.160771,0.472899}%
\pgfsetstrokecolor{currentstroke}%
\pgfsetdash{}{0pt}%
\pgfpathmoveto{\pgfqpoint{2.217456in}{4.790397in}}%
\pgfpathlineto{\pgfqpoint{2.208180in}{4.823645in}}%
\pgfusepath{stroke}%
\end{pgfscope}%
\begin{pgfscope}%
\pgfpathrectangle{\pgfqpoint{1.250000in}{4.155455in}}{\pgfqpoint{2.279412in}{2.004545in}}%
\pgfusepath{clip}%
\pgfsetbuttcap%
\pgfsetroundjoin%
\pgfsetlinewidth{0.565136pt}%
\definecolor{currentstroke}{rgb}{0.273006,0.204520,0.501721}%
\pgfsetstrokecolor{currentstroke}%
\pgfsetdash{}{0pt}%
\pgfpathmoveto{\pgfqpoint{2.208180in}{4.823645in}}%
\pgfpathlineto{\pgfqpoint{2.195073in}{4.865293in}}%
\pgfusepath{stroke}%
\end{pgfscope}%
\begin{pgfscope}%
\pgfpathrectangle{\pgfqpoint{1.250000in}{4.155455in}}{\pgfqpoint{2.279412in}{2.004545in}}%
\pgfusepath{clip}%
\pgfsetbuttcap%
\pgfsetroundjoin%
\pgfsetlinewidth{0.572638pt}%
\definecolor{currentstroke}{rgb}{0.271828,0.209303,0.504434}%
\pgfsetstrokecolor{currentstroke}%
\pgfsetdash{}{0pt}%
\pgfpathmoveto{\pgfqpoint{2.195073in}{4.865293in}}%
\pgfpathlineto{\pgfqpoint{2.182757in}{4.906526in}}%
\pgfusepath{stroke}%
\end{pgfscope}%
\begin{pgfscope}%
\pgfpathrectangle{\pgfqpoint{1.250000in}{4.155455in}}{\pgfqpoint{2.279412in}{2.004545in}}%
\pgfusepath{clip}%
\pgfsetbuttcap%
\pgfsetroundjoin%
\pgfsetlinewidth{0.353689pt}%
\definecolor{currentstroke}{rgb}{0.276022,0.044167,0.370164}%
\pgfsetstrokecolor{currentstroke}%
\pgfsetdash{}{0pt}%
\pgfpathmoveto{\pgfqpoint{2.745269in}{4.602894in}}%
\pgfpathlineto{\pgfqpoint{2.695719in}{4.609688in}}%
\pgfusepath{stroke}%
\end{pgfscope}%
\begin{pgfscope}%
\pgfpathrectangle{\pgfqpoint{1.250000in}{4.155455in}}{\pgfqpoint{2.279412in}{2.004545in}}%
\pgfusepath{clip}%
\pgfsetbuttcap%
\pgfsetroundjoin%
\pgfsetlinewidth{0.379545pt}%
\definecolor{currentstroke}{rgb}{0.279566,0.067836,0.391917}%
\pgfsetstrokecolor{currentstroke}%
\pgfsetdash{}{0pt}%
\pgfpathmoveto{\pgfqpoint{2.695719in}{4.609688in}}%
\pgfpathlineto{\pgfqpoint{2.646165in}{4.616446in}}%
\pgfusepath{stroke}%
\end{pgfscope}%
\begin{pgfscope}%
\pgfpathrectangle{\pgfqpoint{1.250000in}{4.155455in}}{\pgfqpoint{2.279412in}{2.004545in}}%
\pgfusepath{clip}%
\pgfsetbuttcap%
\pgfsetroundjoin%
\pgfsetlinewidth{0.377201pt}%
\definecolor{currentstroke}{rgb}{0.279566,0.067836,0.391917}%
\pgfsetstrokecolor{currentstroke}%
\pgfsetdash{}{0pt}%
\pgfpathmoveto{\pgfqpoint{2.646165in}{4.616446in}}%
\pgfpathlineto{\pgfqpoint{2.596803in}{4.624168in}}%
\pgfusepath{stroke}%
\end{pgfscope}%
\begin{pgfscope}%
\pgfpathrectangle{\pgfqpoint{1.250000in}{4.155455in}}{\pgfqpoint{2.279412in}{2.004545in}}%
\pgfusepath{clip}%
\pgfsetbuttcap%
\pgfsetroundjoin%
\pgfsetlinewidth{0.389478pt}%
\definecolor{currentstroke}{rgb}{0.280267,0.073417,0.397163}%
\pgfsetstrokecolor{currentstroke}%
\pgfsetdash{}{0pt}%
\pgfpathmoveto{\pgfqpoint{2.596803in}{4.624168in}}%
\pgfpathlineto{\pgfqpoint{2.547703in}{4.633127in}}%
\pgfusepath{stroke}%
\end{pgfscope}%
\begin{pgfscope}%
\pgfpathrectangle{\pgfqpoint{1.250000in}{4.155455in}}{\pgfqpoint{2.279412in}{2.004545in}}%
\pgfusepath{clip}%
\pgfsetbuttcap%
\pgfsetroundjoin%
\pgfsetlinewidth{0.412063pt}%
\definecolor{currentstroke}{rgb}{0.282327,0.094955,0.417331}%
\pgfsetstrokecolor{currentstroke}%
\pgfsetdash{}{0pt}%
\pgfpathmoveto{\pgfqpoint{2.547703in}{4.633127in}}%
\pgfpathlineto{\pgfqpoint{2.499072in}{4.643819in}}%
\pgfusepath{stroke}%
\end{pgfscope}%
\begin{pgfscope}%
\pgfpathrectangle{\pgfqpoint{1.250000in}{4.155455in}}{\pgfqpoint{2.279412in}{2.004545in}}%
\pgfusepath{clip}%
\pgfsetbuttcap%
\pgfsetroundjoin%
\pgfsetlinewidth{0.451495pt}%
\definecolor{currentstroke}{rgb}{0.283229,0.120777,0.440584}%
\pgfsetstrokecolor{currentstroke}%
\pgfsetdash{}{0pt}%
\pgfpathmoveto{\pgfqpoint{2.499072in}{4.643819in}}%
\pgfpathlineto{\pgfqpoint{2.451154in}{4.656741in}}%
\pgfusepath{stroke}%
\end{pgfscope}%
\begin{pgfscope}%
\pgfpathrectangle{\pgfqpoint{1.250000in}{4.155455in}}{\pgfqpoint{2.279412in}{2.004545in}}%
\pgfusepath{clip}%
\pgfsetbuttcap%
\pgfsetroundjoin%
\pgfsetlinewidth{0.441727pt}%
\definecolor{currentstroke}{rgb}{0.283197,0.115680,0.436115}%
\pgfsetstrokecolor{currentstroke}%
\pgfsetdash{}{0pt}%
\pgfpathmoveto{\pgfqpoint{2.451154in}{4.656741in}}%
\pgfpathlineto{\pgfqpoint{2.404092in}{4.671843in}}%
\pgfusepath{stroke}%
\end{pgfscope}%
\begin{pgfscope}%
\pgfpathrectangle{\pgfqpoint{1.250000in}{4.155455in}}{\pgfqpoint{2.279412in}{2.004545in}}%
\pgfusepath{clip}%
\pgfsetbuttcap%
\pgfsetroundjoin%
\pgfsetlinewidth{0.483024pt}%
\definecolor{currentstroke}{rgb}{0.282290,0.145912,0.461510}%
\pgfsetstrokecolor{currentstroke}%
\pgfsetdash{}{0pt}%
\pgfpathmoveto{\pgfqpoint{2.404092in}{4.671843in}}%
\pgfpathlineto{\pgfqpoint{2.358540in}{4.690111in}}%
\pgfusepath{stroke}%
\end{pgfscope}%
\begin{pgfscope}%
\pgfpathrectangle{\pgfqpoint{1.250000in}{4.155455in}}{\pgfqpoint{2.279412in}{2.004545in}}%
\pgfusepath{clip}%
\pgfsetbuttcap%
\pgfsetroundjoin%
\pgfsetlinewidth{0.458802pt}%
\definecolor{currentstroke}{rgb}{0.283187,0.125848,0.444960}%
\pgfsetstrokecolor{currentstroke}%
\pgfsetdash{}{0pt}%
\pgfpathmoveto{\pgfqpoint{2.358540in}{4.690111in}}%
\pgfpathlineto{\pgfqpoint{2.316524in}{4.713742in}}%
\pgfusepath{stroke}%
\end{pgfscope}%
\begin{pgfscope}%
\pgfpathrectangle{\pgfqpoint{1.250000in}{4.155455in}}{\pgfqpoint{2.279412in}{2.004545in}}%
\pgfusepath{clip}%
\pgfsetbuttcap%
\pgfsetroundjoin%
\pgfsetlinewidth{0.500943pt}%
\definecolor{currentstroke}{rgb}{0.280868,0.160771,0.472899}%
\pgfsetstrokecolor{currentstroke}%
\pgfsetdash{}{0pt}%
\pgfpathmoveto{\pgfqpoint{2.316524in}{4.713742in}}%
\pgfpathlineto{\pgfqpoint{2.280249in}{4.743645in}}%
\pgfusepath{stroke}%
\end{pgfscope}%
\begin{pgfscope}%
\pgfpathrectangle{\pgfqpoint{1.250000in}{4.155455in}}{\pgfqpoint{2.279412in}{2.004545in}}%
\pgfusepath{clip}%
\pgfsetbuttcap%
\pgfsetroundjoin%
\pgfsetlinewidth{0.320464pt}%
\definecolor{currentstroke}{rgb}{0.269944,0.014625,0.341379}%
\pgfsetstrokecolor{currentstroke}%
\pgfsetdash{}{0pt}%
\pgfpathmoveto{\pgfqpoint{3.005209in}{5.699009in}}%
\pgfpathlineto{\pgfqpoint{2.955322in}{5.695074in}}%
\pgfusepath{stroke}%
\end{pgfscope}%
\begin{pgfscope}%
\pgfpathrectangle{\pgfqpoint{1.250000in}{4.155455in}}{\pgfqpoint{2.279412in}{2.004545in}}%
\pgfusepath{clip}%
\pgfsetbuttcap%
\pgfsetroundjoin%
\pgfsetlinewidth{0.328374pt}%
\definecolor{currentstroke}{rgb}{0.271305,0.019942,0.347269}%
\pgfsetstrokecolor{currentstroke}%
\pgfsetdash{}{0pt}%
\pgfpathmoveto{\pgfqpoint{2.955322in}{5.695074in}}%
\pgfpathlineto{\pgfqpoint{2.905631in}{5.689246in}}%
\pgfusepath{stroke}%
\end{pgfscope}%
\begin{pgfscope}%
\pgfpathrectangle{\pgfqpoint{1.250000in}{4.155455in}}{\pgfqpoint{2.279412in}{2.004545in}}%
\pgfusepath{clip}%
\pgfsetbuttcap%
\pgfsetroundjoin%
\pgfsetlinewidth{0.339895pt}%
\definecolor{currentstroke}{rgb}{0.273809,0.031497,0.358853}%
\pgfsetstrokecolor{currentstroke}%
\pgfsetdash{}{0pt}%
\pgfpathmoveto{\pgfqpoint{2.905631in}{5.689246in}}%
\pgfpathlineto{\pgfqpoint{2.855855in}{5.684012in}}%
\pgfusepath{stroke}%
\end{pgfscope}%
\begin{pgfscope}%
\pgfpathrectangle{\pgfqpoint{1.250000in}{4.155455in}}{\pgfqpoint{2.279412in}{2.004545in}}%
\pgfusepath{clip}%
\pgfsetbuttcap%
\pgfsetroundjoin%
\pgfsetlinewidth{0.341424pt}%
\definecolor{currentstroke}{rgb}{0.273809,0.031497,0.358853}%
\pgfsetstrokecolor{currentstroke}%
\pgfsetdash{}{0pt}%
\pgfpathmoveto{\pgfqpoint{2.855855in}{5.684012in}}%
\pgfpathlineto{\pgfqpoint{2.806086in}{5.678681in}}%
\pgfusepath{stroke}%
\end{pgfscope}%
\begin{pgfscope}%
\pgfpathrectangle{\pgfqpoint{1.250000in}{4.155455in}}{\pgfqpoint{2.279412in}{2.004545in}}%
\pgfusepath{clip}%
\pgfsetbuttcap%
\pgfsetroundjoin%
\pgfsetlinewidth{0.350428pt}%
\definecolor{currentstroke}{rgb}{0.276022,0.044167,0.370164}%
\pgfsetstrokecolor{currentstroke}%
\pgfsetdash{}{0pt}%
\pgfpathmoveto{\pgfqpoint{2.806086in}{5.678681in}}%
\pgfpathlineto{\pgfqpoint{2.756388in}{5.672818in}}%
\pgfusepath{stroke}%
\end{pgfscope}%
\begin{pgfscope}%
\pgfpathrectangle{\pgfqpoint{1.250000in}{4.155455in}}{\pgfqpoint{2.279412in}{2.004545in}}%
\pgfusepath{clip}%
\pgfsetbuttcap%
\pgfsetroundjoin%
\pgfsetlinewidth{0.359144pt}%
\definecolor{currentstroke}{rgb}{0.277018,0.050344,0.375715}%
\pgfsetstrokecolor{currentstroke}%
\pgfsetdash{}{0pt}%
\pgfpathmoveto{\pgfqpoint{2.756388in}{5.672818in}}%
\pgfpathlineto{\pgfqpoint{2.706676in}{5.667035in}}%
\pgfusepath{stroke}%
\end{pgfscope}%
\begin{pgfscope}%
\pgfpathrectangle{\pgfqpoint{1.250000in}{4.155455in}}{\pgfqpoint{2.279412in}{2.004545in}}%
\pgfusepath{clip}%
\pgfsetbuttcap%
\pgfsetroundjoin%
\pgfsetlinewidth{0.385223pt}%
\definecolor{currentstroke}{rgb}{0.280267,0.073417,0.397163}%
\pgfsetstrokecolor{currentstroke}%
\pgfsetdash{}{0pt}%
\pgfpathmoveto{\pgfqpoint{2.706676in}{5.667035in}}%
\pgfpathlineto{\pgfqpoint{2.657004in}{5.661005in}}%
\pgfusepath{stroke}%
\end{pgfscope}%
\begin{pgfscope}%
\pgfpathrectangle{\pgfqpoint{1.250000in}{4.155455in}}{\pgfqpoint{2.279412in}{2.004545in}}%
\pgfusepath{clip}%
\pgfsetbuttcap%
\pgfsetroundjoin%
\pgfsetlinewidth{0.405847pt}%
\definecolor{currentstroke}{rgb}{0.281924,0.089666,0.412415}%
\pgfsetstrokecolor{currentstroke}%
\pgfsetdash{}{0pt}%
\pgfpathmoveto{\pgfqpoint{2.657004in}{5.661005in}}%
\pgfpathlineto{\pgfqpoint{2.607466in}{5.654204in}}%
\pgfusepath{stroke}%
\end{pgfscope}%
\begin{pgfscope}%
\pgfpathrectangle{\pgfqpoint{1.250000in}{4.155455in}}{\pgfqpoint{2.279412in}{2.004545in}}%
\pgfusepath{clip}%
\pgfsetbuttcap%
\pgfsetroundjoin%
\pgfsetlinewidth{0.385369pt}%
\definecolor{currentstroke}{rgb}{0.280267,0.073417,0.397163}%
\pgfsetstrokecolor{currentstroke}%
\pgfsetdash{}{0pt}%
\pgfpathmoveto{\pgfqpoint{2.607466in}{5.654204in}}%
\pgfpathlineto{\pgfqpoint{2.558063in}{5.646706in}}%
\pgfusepath{stroke}%
\end{pgfscope}%
\begin{pgfscope}%
\pgfpathrectangle{\pgfqpoint{1.250000in}{4.155455in}}{\pgfqpoint{2.279412in}{2.004545in}}%
\pgfusepath{clip}%
\pgfsetbuttcap%
\pgfsetroundjoin%
\pgfsetlinewidth{0.437468pt}%
\definecolor{currentstroke}{rgb}{0.283091,0.110553,0.431554}%
\pgfsetstrokecolor{currentstroke}%
\pgfsetdash{}{0pt}%
\pgfpathmoveto{\pgfqpoint{2.558063in}{5.646706in}}%
\pgfpathlineto{\pgfqpoint{2.508963in}{5.637821in}}%
\pgfusepath{stroke}%
\end{pgfscope}%
\begin{pgfscope}%
\pgfpathrectangle{\pgfqpoint{1.250000in}{4.155455in}}{\pgfqpoint{2.279412in}{2.004545in}}%
\pgfusepath{clip}%
\pgfsetbuttcap%
\pgfsetroundjoin%
\pgfsetlinewidth{0.448635pt}%
\definecolor{currentstroke}{rgb}{0.283229,0.120777,0.440584}%
\pgfsetstrokecolor{currentstroke}%
\pgfsetdash{}{0pt}%
\pgfpathmoveto{\pgfqpoint{2.508963in}{5.637821in}}%
\pgfpathlineto{\pgfqpoint{2.460357in}{5.626994in}}%
\pgfusepath{stroke}%
\end{pgfscope}%
\begin{pgfscope}%
\pgfpathrectangle{\pgfqpoint{1.250000in}{4.155455in}}{\pgfqpoint{2.279412in}{2.004545in}}%
\pgfusepath{clip}%
\pgfsetbuttcap%
\pgfsetroundjoin%
\pgfsetlinewidth{0.462999pt}%
\definecolor{currentstroke}{rgb}{0.283072,0.130895,0.449241}%
\pgfsetstrokecolor{currentstroke}%
\pgfsetdash{}{0pt}%
\pgfpathmoveto{\pgfqpoint{2.460357in}{5.626994in}}%
\pgfpathlineto{\pgfqpoint{2.412450in}{5.614019in}}%
\pgfusepath{stroke}%
\end{pgfscope}%
\begin{pgfscope}%
\pgfpathrectangle{\pgfqpoint{1.250000in}{4.155455in}}{\pgfqpoint{2.279412in}{2.004545in}}%
\pgfusepath{clip}%
\pgfsetbuttcap%
\pgfsetroundjoin%
\pgfsetlinewidth{0.516544pt}%
\definecolor{currentstroke}{rgb}{0.279574,0.170599,0.479997}%
\pgfsetstrokecolor{currentstroke}%
\pgfsetdash{}{0pt}%
\pgfpathmoveto{\pgfqpoint{2.412450in}{5.614019in}}%
\pgfpathlineto{\pgfqpoint{2.365726in}{5.598119in}}%
\pgfusepath{stroke}%
\end{pgfscope}%
\begin{pgfscope}%
\pgfpathrectangle{\pgfqpoint{1.250000in}{4.155455in}}{\pgfqpoint{2.279412in}{2.004545in}}%
\pgfusepath{clip}%
\pgfsetbuttcap%
\pgfsetroundjoin%
\pgfsetlinewidth{0.475636pt}%
\definecolor{currentstroke}{rgb}{0.282623,0.140926,0.457517}%
\pgfsetstrokecolor{currentstroke}%
\pgfsetdash{}{0pt}%
\pgfpathmoveto{\pgfqpoint{2.365726in}{5.598119in}}%
\pgfpathlineto{\pgfqpoint{2.322034in}{5.576948in}}%
\pgfusepath{stroke}%
\end{pgfscope}%
\begin{pgfscope}%
\pgfpathrectangle{\pgfqpoint{1.250000in}{4.155455in}}{\pgfqpoint{2.279412in}{2.004545in}}%
\pgfusepath{clip}%
\pgfsetbuttcap%
\pgfsetroundjoin%
\pgfsetlinewidth{0.483086pt}%
\definecolor{currentstroke}{rgb}{0.282290,0.145912,0.461510}%
\pgfsetstrokecolor{currentstroke}%
\pgfsetdash{}{0pt}%
\pgfpathmoveto{\pgfqpoint{2.322034in}{5.576948in}}%
\pgfpathlineto{\pgfqpoint{2.282535in}{5.550328in}}%
\pgfusepath{stroke}%
\end{pgfscope}%
\begin{pgfscope}%
\pgfpathrectangle{\pgfqpoint{1.250000in}{4.155455in}}{\pgfqpoint{2.279412in}{2.004545in}}%
\pgfusepath{clip}%
\pgfsetbuttcap%
\pgfsetroundjoin%
\pgfsetlinewidth{0.533677pt}%
\definecolor{currentstroke}{rgb}{0.278012,0.180367,0.486697}%
\pgfsetstrokecolor{currentstroke}%
\pgfsetdash{}{0pt}%
\pgfpathmoveto{\pgfqpoint{2.282535in}{5.550328in}}%
\pgfpathlineto{\pgfqpoint{2.282535in}{5.550328in}}%
\pgfusepath{stroke}%
\end{pgfscope}%
\begin{pgfscope}%
\pgfpathrectangle{\pgfqpoint{1.250000in}{4.155455in}}{\pgfqpoint{2.279412in}{2.004545in}}%
\pgfusepath{clip}%
\pgfsetbuttcap%
\pgfsetroundjoin%
\pgfsetlinewidth{0.334901pt}%
\definecolor{currentstroke}{rgb}{0.272594,0.025563,0.353093}%
\pgfsetstrokecolor{currentstroke}%
\pgfsetdash{}{0pt}%
\pgfpathmoveto{\pgfqpoint{2.892336in}{4.622722in}}%
\pgfpathlineto{\pgfqpoint{2.842614in}{4.628110in}}%
\pgfusepath{stroke}%
\end{pgfscope}%
\begin{pgfscope}%
\pgfpathrectangle{\pgfqpoint{1.250000in}{4.155455in}}{\pgfqpoint{2.279412in}{2.004545in}}%
\pgfusepath{clip}%
\pgfsetbuttcap%
\pgfsetroundjoin%
\pgfsetlinewidth{0.331048pt}%
\definecolor{currentstroke}{rgb}{0.272594,0.025563,0.353093}%
\pgfsetstrokecolor{currentstroke}%
\pgfsetdash{}{0pt}%
\pgfpathmoveto{\pgfqpoint{2.842614in}{4.628110in}}%
\pgfpathlineto{\pgfqpoint{2.792961in}{4.634243in}}%
\pgfusepath{stroke}%
\end{pgfscope}%
\begin{pgfscope}%
\pgfpathrectangle{\pgfqpoint{1.250000in}{4.155455in}}{\pgfqpoint{2.279412in}{2.004545in}}%
\pgfusepath{clip}%
\pgfsetbuttcap%
\pgfsetroundjoin%
\pgfsetlinewidth{0.354169pt}%
\definecolor{currentstroke}{rgb}{0.276022,0.044167,0.370164}%
\pgfsetstrokecolor{currentstroke}%
\pgfsetdash{}{0pt}%
\pgfpathmoveto{\pgfqpoint{2.792961in}{4.634243in}}%
\pgfpathlineto{\pgfqpoint{2.743216in}{4.639833in}}%
\pgfusepath{stroke}%
\end{pgfscope}%
\begin{pgfscope}%
\pgfpathrectangle{\pgfqpoint{1.250000in}{4.155455in}}{\pgfqpoint{2.279412in}{2.004545in}}%
\pgfusepath{clip}%
\pgfsetbuttcap%
\pgfsetroundjoin%
\pgfsetlinewidth{0.370426pt}%
\definecolor{currentstroke}{rgb}{0.278791,0.062145,0.386592}%
\pgfsetstrokecolor{currentstroke}%
\pgfsetdash{}{0pt}%
\pgfpathmoveto{\pgfqpoint{2.743216in}{4.639833in}}%
\pgfpathlineto{\pgfqpoint{2.693584in}{4.646117in}}%
\pgfusepath{stroke}%
\end{pgfscope}%
\begin{pgfscope}%
\pgfpathrectangle{\pgfqpoint{1.250000in}{4.155455in}}{\pgfqpoint{2.279412in}{2.004545in}}%
\pgfusepath{clip}%
\pgfsetbuttcap%
\pgfsetroundjoin%
\pgfsetlinewidth{0.382164pt}%
\definecolor{currentstroke}{rgb}{0.279566,0.067836,0.391917}%
\pgfsetstrokecolor{currentstroke}%
\pgfsetdash{}{0pt}%
\pgfpathmoveto{\pgfqpoint{2.693584in}{4.646117in}}%
\pgfpathlineto{\pgfqpoint{2.644074in}{4.653124in}}%
\pgfusepath{stroke}%
\end{pgfscope}%
\begin{pgfscope}%
\pgfpathrectangle{\pgfqpoint{1.250000in}{4.155455in}}{\pgfqpoint{2.279412in}{2.004545in}}%
\pgfusepath{clip}%
\pgfsetbuttcap%
\pgfsetroundjoin%
\pgfsetlinewidth{0.398587pt}%
\definecolor{currentstroke}{rgb}{0.281446,0.084320,0.407414}%
\pgfsetstrokecolor{currentstroke}%
\pgfsetdash{}{0pt}%
\pgfpathmoveto{\pgfqpoint{2.644074in}{4.653124in}}%
\pgfpathlineto{\pgfqpoint{2.594873in}{4.661553in}}%
\pgfusepath{stroke}%
\end{pgfscope}%
\begin{pgfscope}%
\pgfpathrectangle{\pgfqpoint{1.250000in}{4.155455in}}{\pgfqpoint{2.279412in}{2.004545in}}%
\pgfusepath{clip}%
\pgfsetbuttcap%
\pgfsetroundjoin%
\pgfsetlinewidth{0.433347pt}%
\definecolor{currentstroke}{rgb}{0.283091,0.110553,0.431554}%
\pgfsetstrokecolor{currentstroke}%
\pgfsetdash{}{0pt}%
\pgfpathmoveto{\pgfqpoint{2.594873in}{4.661553in}}%
\pgfpathlineto{\pgfqpoint{2.545805in}{4.670634in}}%
\pgfusepath{stroke}%
\end{pgfscope}%
\begin{pgfscope}%
\pgfpathrectangle{\pgfqpoint{1.250000in}{4.155455in}}{\pgfqpoint{2.279412in}{2.004545in}}%
\pgfusepath{clip}%
\pgfsetbuttcap%
\pgfsetroundjoin%
\pgfsetlinewidth{0.321757pt}%
\definecolor{currentstroke}{rgb}{0.271305,0.019942,0.347269}%
\pgfsetstrokecolor{currentstroke}%
\pgfsetdash{}{0pt}%
\pgfpathmoveto{\pgfqpoint{3.053047in}{5.619320in}}%
\pgfpathlineto{\pgfqpoint{3.003340in}{5.615595in}}%
\pgfusepath{stroke}%
\end{pgfscope}%
\begin{pgfscope}%
\pgfpathrectangle{\pgfqpoint{1.250000in}{4.155455in}}{\pgfqpoint{2.279412in}{2.004545in}}%
\pgfusepath{clip}%
\pgfsetbuttcap%
\pgfsetroundjoin%
\pgfsetlinewidth{0.326451pt}%
\definecolor{currentstroke}{rgb}{0.271305,0.019942,0.347269}%
\pgfsetstrokecolor{currentstroke}%
\pgfsetdash{}{0pt}%
\pgfpathmoveto{\pgfqpoint{3.003340in}{5.615595in}}%
\pgfpathlineto{\pgfqpoint{2.953917in}{5.608795in}}%
\pgfusepath{stroke}%
\end{pgfscope}%
\begin{pgfscope}%
\pgfpathrectangle{\pgfqpoint{1.250000in}{4.155455in}}{\pgfqpoint{2.279412in}{2.004545in}}%
\pgfusepath{clip}%
\pgfsetbuttcap%
\pgfsetroundjoin%
\pgfsetlinewidth{0.334746pt}%
\definecolor{currentstroke}{rgb}{0.272594,0.025563,0.353093}%
\pgfsetstrokecolor{currentstroke}%
\pgfsetdash{}{0pt}%
\pgfpathmoveto{\pgfqpoint{2.953917in}{5.608795in}}%
\pgfpathlineto{\pgfqpoint{2.904150in}{5.603887in}}%
\pgfusepath{stroke}%
\end{pgfscope}%
\begin{pgfscope}%
\pgfpathrectangle{\pgfqpoint{1.250000in}{4.155455in}}{\pgfqpoint{2.279412in}{2.004545in}}%
\pgfusepath{clip}%
\pgfsetbuttcap%
\pgfsetroundjoin%
\pgfsetlinewidth{0.351177pt}%
\definecolor{currentstroke}{rgb}{0.276022,0.044167,0.370164}%
\pgfsetstrokecolor{currentstroke}%
\pgfsetdash{}{0pt}%
\pgfpathmoveto{\pgfqpoint{2.904150in}{5.603887in}}%
\pgfpathlineto{\pgfqpoint{2.854178in}{5.600308in}}%
\pgfusepath{stroke}%
\end{pgfscope}%
\begin{pgfscope}%
\pgfpathrectangle{\pgfqpoint{1.250000in}{4.155455in}}{\pgfqpoint{2.279412in}{2.004545in}}%
\pgfusepath{clip}%
\pgfsetbuttcap%
\pgfsetroundjoin%
\pgfsetlinewidth{0.363681pt}%
\definecolor{currentstroke}{rgb}{0.277941,0.056324,0.381191}%
\pgfsetstrokecolor{currentstroke}%
\pgfsetdash{}{0pt}%
\pgfpathmoveto{\pgfqpoint{2.854178in}{5.600308in}}%
\pgfpathlineto{\pgfqpoint{2.804328in}{5.595601in}}%
\pgfusepath{stroke}%
\end{pgfscope}%
\begin{pgfscope}%
\pgfpathrectangle{\pgfqpoint{1.250000in}{4.155455in}}{\pgfqpoint{2.279412in}{2.004545in}}%
\pgfusepath{clip}%
\pgfsetbuttcap%
\pgfsetroundjoin%
\pgfsetlinewidth{0.384830pt}%
\definecolor{currentstroke}{rgb}{0.280267,0.073417,0.397163}%
\pgfsetstrokecolor{currentstroke}%
\pgfsetdash{}{0pt}%
\pgfpathmoveto{\pgfqpoint{2.804328in}{5.595601in}}%
\pgfpathlineto{\pgfqpoint{2.754530in}{5.590399in}}%
\pgfusepath{stroke}%
\end{pgfscope}%
\begin{pgfscope}%
\pgfpathrectangle{\pgfqpoint{1.250000in}{4.155455in}}{\pgfqpoint{2.279412in}{2.004545in}}%
\pgfusepath{clip}%
\pgfsetbuttcap%
\pgfsetroundjoin%
\pgfsetlinewidth{0.415982pt}%
\definecolor{currentstroke}{rgb}{0.282327,0.094955,0.417331}%
\pgfsetstrokecolor{currentstroke}%
\pgfsetdash{}{0pt}%
\pgfpathmoveto{\pgfqpoint{2.754530in}{5.590399in}}%
\pgfpathlineto{\pgfqpoint{2.704728in}{5.585227in}}%
\pgfusepath{stroke}%
\end{pgfscope}%
\begin{pgfscope}%
\pgfpathrectangle{\pgfqpoint{1.250000in}{4.155455in}}{\pgfqpoint{2.279412in}{2.004545in}}%
\pgfusepath{clip}%
\pgfsetbuttcap%
\pgfsetroundjoin%
\pgfsetlinewidth{0.432081pt}%
\definecolor{currentstroke}{rgb}{0.283091,0.110553,0.431554}%
\pgfsetstrokecolor{currentstroke}%
\pgfsetdash{}{0pt}%
\pgfpathmoveto{\pgfqpoint{2.704728in}{5.585227in}}%
\pgfpathlineto{\pgfqpoint{2.655001in}{5.579509in}}%
\pgfusepath{stroke}%
\end{pgfscope}%
\begin{pgfscope}%
\pgfpathrectangle{\pgfqpoint{1.250000in}{4.155455in}}{\pgfqpoint{2.279412in}{2.004545in}}%
\pgfusepath{clip}%
\pgfsetbuttcap%
\pgfsetroundjoin%
\pgfsetlinewidth{0.474489pt}%
\definecolor{currentstroke}{rgb}{0.282623,0.140926,0.457517}%
\pgfsetstrokecolor{currentstroke}%
\pgfsetdash{}{0pt}%
\pgfpathmoveto{\pgfqpoint{2.655001in}{5.579509in}}%
\pgfpathlineto{\pgfqpoint{2.605354in}{5.573289in}}%
\pgfusepath{stroke}%
\end{pgfscope}%
\begin{pgfscope}%
\pgfpathrectangle{\pgfqpoint{1.250000in}{4.155455in}}{\pgfqpoint{2.279412in}{2.004545in}}%
\pgfusepath{clip}%
\pgfsetbuttcap%
\pgfsetroundjoin%
\pgfsetlinewidth{0.476344pt}%
\definecolor{currentstroke}{rgb}{0.282623,0.140926,0.457517}%
\pgfsetstrokecolor{currentstroke}%
\pgfsetdash{}{0pt}%
\pgfpathmoveto{\pgfqpoint{2.605354in}{5.573289in}}%
\pgfpathlineto{\pgfqpoint{2.555881in}{5.566084in}}%
\pgfusepath{stroke}%
\end{pgfscope}%
\begin{pgfscope}%
\pgfpathrectangle{\pgfqpoint{1.250000in}{4.155455in}}{\pgfqpoint{2.279412in}{2.004545in}}%
\pgfusepath{clip}%
\pgfsetbuttcap%
\pgfsetroundjoin%
\pgfsetlinewidth{0.517767pt}%
\definecolor{currentstroke}{rgb}{0.279574,0.170599,0.479997}%
\pgfsetstrokecolor{currentstroke}%
\pgfsetdash{}{0pt}%
\pgfpathmoveto{\pgfqpoint{2.555881in}{5.566084in}}%
\pgfpathlineto{\pgfqpoint{2.506604in}{5.557897in}}%
\pgfusepath{stroke}%
\end{pgfscope}%
\begin{pgfscope}%
\pgfpathrectangle{\pgfqpoint{1.250000in}{4.155455in}}{\pgfqpoint{2.279412in}{2.004545in}}%
\pgfusepath{clip}%
\pgfsetbuttcap%
\pgfsetroundjoin%
\pgfsetlinewidth{0.535662pt}%
\definecolor{currentstroke}{rgb}{0.277134,0.185228,0.489898}%
\pgfsetstrokecolor{currentstroke}%
\pgfsetdash{}{0pt}%
\pgfpathmoveto{\pgfqpoint{2.506604in}{5.557897in}}%
\pgfpathlineto{\pgfqpoint{2.457589in}{5.548592in}}%
\pgfusepath{stroke}%
\end{pgfscope}%
\begin{pgfscope}%
\pgfpathrectangle{\pgfqpoint{1.250000in}{4.155455in}}{\pgfqpoint{2.279412in}{2.004545in}}%
\pgfusepath{clip}%
\pgfsetbuttcap%
\pgfsetroundjoin%
\pgfsetlinewidth{0.531960pt}%
\definecolor{currentstroke}{rgb}{0.278012,0.180367,0.486697}%
\pgfsetstrokecolor{currentstroke}%
\pgfsetdash{}{0pt}%
\pgfpathmoveto{\pgfqpoint{2.457589in}{5.548592in}}%
\pgfpathlineto{\pgfqpoint{2.409139in}{5.537283in}}%
\pgfusepath{stroke}%
\end{pgfscope}%
\begin{pgfscope}%
\pgfpathrectangle{\pgfqpoint{1.250000in}{4.155455in}}{\pgfqpoint{2.279412in}{2.004545in}}%
\pgfusepath{clip}%
\pgfsetbuttcap%
\pgfsetroundjoin%
\pgfsetlinewidth{0.514896pt}%
\definecolor{currentstroke}{rgb}{0.279574,0.170599,0.479997}%
\pgfsetstrokecolor{currentstroke}%
\pgfsetdash{}{0pt}%
\pgfpathmoveto{\pgfqpoint{2.409139in}{5.537283in}}%
\pgfpathlineto{\pgfqpoint{2.361680in}{5.523154in}}%
\pgfusepath{stroke}%
\end{pgfscope}%
\begin{pgfscope}%
\pgfpathrectangle{\pgfqpoint{1.250000in}{4.155455in}}{\pgfqpoint{2.279412in}{2.004545in}}%
\pgfusepath{clip}%
\pgfsetbuttcap%
\pgfsetroundjoin%
\pgfsetlinewidth{0.550825pt}%
\definecolor{currentstroke}{rgb}{0.275191,0.194905,0.496005}%
\pgfsetstrokecolor{currentstroke}%
\pgfsetdash{}{0pt}%
\pgfpathmoveto{\pgfqpoint{2.361680in}{5.523154in}}%
\pgfpathlineto{\pgfqpoint{2.316023in}{5.505113in}}%
\pgfusepath{stroke}%
\end{pgfscope}%
\begin{pgfscope}%
\pgfpathrectangle{\pgfqpoint{1.250000in}{4.155455in}}{\pgfqpoint{2.279412in}{2.004545in}}%
\pgfusepath{clip}%
\pgfsetbuttcap%
\pgfsetroundjoin%
\pgfsetlinewidth{0.567040pt}%
\definecolor{currentstroke}{rgb}{0.273006,0.204520,0.501721}%
\pgfsetstrokecolor{currentstroke}%
\pgfsetdash{}{0pt}%
\pgfpathmoveto{\pgfqpoint{2.316023in}{5.505113in}}%
\pgfpathlineto{\pgfqpoint{2.273794in}{5.481625in}}%
\pgfusepath{stroke}%
\end{pgfscope}%
\begin{pgfscope}%
\pgfpathrectangle{\pgfqpoint{1.250000in}{4.155455in}}{\pgfqpoint{2.279412in}{2.004545in}}%
\pgfusepath{clip}%
\pgfsetbuttcap%
\pgfsetroundjoin%
\pgfsetlinewidth{0.573473pt}%
\definecolor{currentstroke}{rgb}{0.271828,0.209303,0.504434}%
\pgfsetstrokecolor{currentstroke}%
\pgfsetdash{}{0pt}%
\pgfpathmoveto{\pgfqpoint{2.273794in}{5.481625in}}%
\pgfpathlineto{\pgfqpoint{2.236712in}{5.452216in}}%
\pgfusepath{stroke}%
\end{pgfscope}%
\begin{pgfscope}%
\pgfpathrectangle{\pgfqpoint{1.250000in}{4.155455in}}{\pgfqpoint{2.279412in}{2.004545in}}%
\pgfusepath{clip}%
\pgfsetbuttcap%
\pgfsetroundjoin%
\pgfsetlinewidth{0.529007pt}%
\definecolor{currentstroke}{rgb}{0.278012,0.180367,0.486697}%
\pgfsetstrokecolor{currentstroke}%
\pgfsetdash{}{0pt}%
\pgfpathmoveto{\pgfqpoint{2.236712in}{5.452216in}}%
\pgfpathlineto{\pgfqpoint{2.236712in}{5.452216in}}%
\pgfusepath{stroke}%
\end{pgfscope}%
\begin{pgfscope}%
\pgfpathrectangle{\pgfqpoint{1.250000in}{4.155455in}}{\pgfqpoint{2.279412in}{2.004545in}}%
\pgfusepath{clip}%
\pgfsetbuttcap%
\pgfsetroundjoin%
\pgfsetlinewidth{0.529007pt}%
\definecolor{currentstroke}{rgb}{0.278012,0.180367,0.486697}%
\pgfsetstrokecolor{currentstroke}%
\pgfsetdash{}{0pt}%
\pgfpathmoveto{\pgfqpoint{2.236712in}{5.452216in}}%
\pgfpathlineto{\pgfqpoint{2.213781in}{5.426123in}}%
\pgfusepath{stroke}%
\end{pgfscope}%
\begin{pgfscope}%
\pgfpathrectangle{\pgfqpoint{1.250000in}{4.155455in}}{\pgfqpoint{2.279412in}{2.004545in}}%
\pgfusepath{clip}%
\pgfsetbuttcap%
\pgfsetroundjoin%
\pgfsetlinewidth{0.307390pt}%
\definecolor{currentstroke}{rgb}{0.267004,0.004874,0.329415}%
\pgfsetstrokecolor{currentstroke}%
\pgfsetdash{}{0pt}%
\pgfpathmoveto{\pgfqpoint{3.101036in}{5.666502in}}%
\pgfpathlineto{\pgfqpoint{3.053041in}{5.663899in}}%
\pgfusepath{stroke}%
\end{pgfscope}%
\begin{pgfscope}%
\pgfpathrectangle{\pgfqpoint{1.250000in}{4.155455in}}{\pgfqpoint{2.279412in}{2.004545in}}%
\pgfusepath{clip}%
\pgfsetbuttcap%
\pgfsetroundjoin%
\pgfsetlinewidth{0.319411pt}%
\definecolor{currentstroke}{rgb}{0.269944,0.014625,0.341379}%
\pgfsetstrokecolor{currentstroke}%
\pgfsetdash{}{0pt}%
\pgfpathmoveto{\pgfqpoint{3.053041in}{5.663899in}}%
\pgfpathlineto{\pgfqpoint{3.003482in}{5.660110in}}%
\pgfusepath{stroke}%
\end{pgfscope}%
\begin{pgfscope}%
\pgfpathrectangle{\pgfqpoint{1.250000in}{4.155455in}}{\pgfqpoint{2.279412in}{2.004545in}}%
\pgfusepath{clip}%
\pgfsetbuttcap%
\pgfsetroundjoin%
\pgfsetlinewidth{0.323742pt}%
\definecolor{currentstroke}{rgb}{0.271305,0.019942,0.347269}%
\pgfsetstrokecolor{currentstroke}%
\pgfsetdash{}{0pt}%
\pgfpathmoveto{\pgfqpoint{3.003482in}{5.660110in}}%
\pgfpathlineto{\pgfqpoint{2.953917in}{5.653902in}}%
\pgfusepath{stroke}%
\end{pgfscope}%
\begin{pgfscope}%
\pgfpathrectangle{\pgfqpoint{1.250000in}{4.155455in}}{\pgfqpoint{2.279412in}{2.004545in}}%
\pgfusepath{clip}%
\pgfsetbuttcap%
\pgfsetroundjoin%
\pgfsetlinewidth{0.327907pt}%
\definecolor{currentstroke}{rgb}{0.271305,0.019942,0.347269}%
\pgfsetstrokecolor{currentstroke}%
\pgfsetdash{}{0pt}%
\pgfpathmoveto{\pgfqpoint{2.953917in}{5.653902in}}%
\pgfpathlineto{\pgfqpoint{2.904337in}{5.647766in}}%
\pgfusepath{stroke}%
\end{pgfscope}%
\begin{pgfscope}%
\pgfpathrectangle{\pgfqpoint{1.250000in}{4.155455in}}{\pgfqpoint{2.279412in}{2.004545in}}%
\pgfusepath{clip}%
\pgfsetbuttcap%
\pgfsetroundjoin%
\pgfsetlinewidth{0.342704pt}%
\definecolor{currentstroke}{rgb}{0.274952,0.037752,0.364543}%
\pgfsetstrokecolor{currentstroke}%
\pgfsetdash{}{0pt}%
\pgfpathmoveto{\pgfqpoint{2.904337in}{5.647766in}}%
\pgfpathlineto{\pgfqpoint{2.854371in}{5.644010in}}%
\pgfusepath{stroke}%
\end{pgfscope}%
\begin{pgfscope}%
\pgfpathrectangle{\pgfqpoint{1.250000in}{4.155455in}}{\pgfqpoint{2.279412in}{2.004545in}}%
\pgfusepath{clip}%
\pgfsetbuttcap%
\pgfsetroundjoin%
\pgfsetlinewidth{0.353509pt}%
\definecolor{currentstroke}{rgb}{0.276022,0.044167,0.370164}%
\pgfsetstrokecolor{currentstroke}%
\pgfsetdash{}{0pt}%
\pgfpathmoveto{\pgfqpoint{2.854371in}{5.644010in}}%
\pgfpathlineto{\pgfqpoint{2.804491in}{5.639493in}}%
\pgfusepath{stroke}%
\end{pgfscope}%
\begin{pgfscope}%
\pgfpathrectangle{\pgfqpoint{1.250000in}{4.155455in}}{\pgfqpoint{2.279412in}{2.004545in}}%
\pgfusepath{clip}%
\pgfsetbuttcap%
\pgfsetroundjoin%
\pgfsetlinewidth{0.540455pt}%
\definecolor{currentstroke}{rgb}{0.277134,0.185228,0.489898}%
\pgfsetstrokecolor{currentstroke}%
\pgfsetdash{}{0pt}%
\pgfpathmoveto{\pgfqpoint{2.692807in}{4.825180in}}%
\pgfpathlineto{\pgfqpoint{2.642997in}{4.830313in}}%
\pgfusepath{stroke}%
\end{pgfscope}%
\begin{pgfscope}%
\pgfpathrectangle{\pgfqpoint{1.250000in}{4.155455in}}{\pgfqpoint{2.279412in}{2.004545in}}%
\pgfusepath{clip}%
\pgfsetbuttcap%
\pgfsetroundjoin%
\pgfsetlinewidth{0.569045pt}%
\definecolor{currentstroke}{rgb}{0.271828,0.209303,0.504434}%
\pgfsetstrokecolor{currentstroke}%
\pgfsetdash{}{0pt}%
\pgfpathmoveto{\pgfqpoint{2.642997in}{4.830313in}}%
\pgfpathlineto{\pgfqpoint{2.593241in}{4.835830in}}%
\pgfusepath{stroke}%
\end{pgfscope}%
\begin{pgfscope}%
\pgfpathrectangle{\pgfqpoint{1.250000in}{4.155455in}}{\pgfqpoint{2.279412in}{2.004545in}}%
\pgfusepath{clip}%
\pgfsetbuttcap%
\pgfsetroundjoin%
\pgfsetlinewidth{0.590181pt}%
\definecolor{currentstroke}{rgb}{0.267968,0.223549,0.512008}%
\pgfsetstrokecolor{currentstroke}%
\pgfsetdash{}{0pt}%
\pgfpathmoveto{\pgfqpoint{2.593241in}{4.835830in}}%
\pgfpathlineto{\pgfqpoint{2.543582in}{4.841980in}}%
\pgfusepath{stroke}%
\end{pgfscope}%
\begin{pgfscope}%
\pgfpathrectangle{\pgfqpoint{1.250000in}{4.155455in}}{\pgfqpoint{2.279412in}{2.004545in}}%
\pgfusepath{clip}%
\pgfsetbuttcap%
\pgfsetroundjoin%
\pgfsetlinewidth{0.644432pt}%
\definecolor{currentstroke}{rgb}{0.255645,0.260703,0.528312}%
\pgfsetstrokecolor{currentstroke}%
\pgfsetdash{}{0pt}%
\pgfpathmoveto{\pgfqpoint{2.543582in}{4.841980in}}%
\pgfpathlineto{\pgfqpoint{2.494079in}{4.849007in}}%
\pgfusepath{stroke}%
\end{pgfscope}%
\begin{pgfscope}%
\pgfpathrectangle{\pgfqpoint{1.250000in}{4.155455in}}{\pgfqpoint{2.279412in}{2.004545in}}%
\pgfusepath{clip}%
\pgfsetbuttcap%
\pgfsetroundjoin%
\pgfsetlinewidth{0.672187pt}%
\definecolor{currentstroke}{rgb}{0.248629,0.278775,0.534556}%
\pgfsetstrokecolor{currentstroke}%
\pgfsetdash{}{0pt}%
\pgfpathmoveto{\pgfqpoint{2.494079in}{4.849007in}}%
\pgfpathlineto{\pgfqpoint{2.444853in}{4.857400in}}%
\pgfusepath{stroke}%
\end{pgfscope}%
\begin{pgfscope}%
\pgfpathrectangle{\pgfqpoint{1.250000in}{4.155455in}}{\pgfqpoint{2.279412in}{2.004545in}}%
\pgfusepath{clip}%
\pgfsetbuttcap%
\pgfsetroundjoin%
\pgfsetlinewidth{0.736253pt}%
\definecolor{currentstroke}{rgb}{0.231674,0.318106,0.544834}%
\pgfsetstrokecolor{currentstroke}%
\pgfsetdash{}{0pt}%
\pgfpathmoveto{\pgfqpoint{2.030663in}{5.428368in}}%
\pgfpathlineto{\pgfqpoint{2.074598in}{5.407662in}}%
\pgfusepath{stroke}%
\end{pgfscope}%
\begin{pgfscope}%
\pgfpathrectangle{\pgfqpoint{1.250000in}{4.155455in}}{\pgfqpoint{2.279412in}{2.004545in}}%
\pgfusepath{clip}%
\pgfsetbuttcap%
\pgfsetroundjoin%
\pgfsetlinewidth{0.718333pt}%
\definecolor{currentstroke}{rgb}{0.237441,0.305202,0.541921}%
\pgfsetstrokecolor{currentstroke}%
\pgfsetdash{}{0pt}%
\pgfpathmoveto{\pgfqpoint{2.074598in}{5.407662in}}%
\pgfpathlineto{\pgfqpoint{2.074598in}{5.407662in}}%
\pgfusepath{stroke}%
\end{pgfscope}%
\begin{pgfscope}%
\pgfpathrectangle{\pgfqpoint{1.250000in}{4.155455in}}{\pgfqpoint{2.279412in}{2.004545in}}%
\pgfusepath{clip}%
\pgfsetbuttcap%
\pgfsetroundjoin%
\pgfsetlinewidth{0.718333pt}%
\definecolor{currentstroke}{rgb}{0.237441,0.305202,0.541921}%
\pgfsetstrokecolor{currentstroke}%
\pgfsetdash{}{0pt}%
\pgfpathmoveto{\pgfqpoint{2.074598in}{5.407662in}}%
\pgfpathlineto{\pgfqpoint{2.107334in}{5.385856in}}%
\pgfusepath{stroke}%
\end{pgfscope}%
\begin{pgfscope}%
\pgfpathrectangle{\pgfqpoint{1.250000in}{4.155455in}}{\pgfqpoint{2.279412in}{2.004545in}}%
\pgfusepath{clip}%
\pgfsetbuttcap%
\pgfsetroundjoin%
\pgfsetlinewidth{0.623968pt}%
\definecolor{currentstroke}{rgb}{0.260571,0.246922,0.522828}%
\pgfsetstrokecolor{currentstroke}%
\pgfsetdash{}{0pt}%
\pgfpathmoveto{\pgfqpoint{2.107334in}{5.385856in}}%
\pgfpathlineto{\pgfqpoint{2.107334in}{5.385856in}}%
\pgfusepath{stroke}%
\end{pgfscope}%
\begin{pgfscope}%
\pgfpathrectangle{\pgfqpoint{1.250000in}{4.155455in}}{\pgfqpoint{2.279412in}{2.004545in}}%
\pgfusepath{clip}%
\pgfsetbuttcap%
\pgfsetroundjoin%
\pgfsetlinewidth{0.623968pt}%
\definecolor{currentstroke}{rgb}{0.260571,0.246922,0.522828}%
\pgfsetstrokecolor{currentstroke}%
\pgfsetdash{}{0pt}%
\pgfpathmoveto{\pgfqpoint{2.107334in}{5.385856in}}%
\pgfpathlineto{\pgfqpoint{2.125461in}{5.367792in}}%
\pgfusepath{stroke}%
\end{pgfscope}%
\begin{pgfscope}%
\pgfpathrectangle{\pgfqpoint{1.250000in}{4.155455in}}{\pgfqpoint{2.279412in}{2.004545in}}%
\pgfusepath{clip}%
\pgfsetbuttcap%
\pgfsetroundjoin%
\pgfsetlinewidth{0.595593pt}%
\definecolor{currentstroke}{rgb}{0.267968,0.223549,0.512008}%
\pgfsetstrokecolor{currentstroke}%
\pgfsetdash{}{0pt}%
\pgfpathmoveto{\pgfqpoint{2.125461in}{5.367792in}}%
\pgfpathlineto{\pgfqpoint{2.140041in}{5.343796in}}%
\pgfusepath{stroke}%
\end{pgfscope}%
\begin{pgfscope}%
\pgfpathrectangle{\pgfqpoint{1.250000in}{4.155455in}}{\pgfqpoint{2.279412in}{2.004545in}}%
\pgfusepath{clip}%
\pgfsetbuttcap%
\pgfsetroundjoin%
\pgfsetlinewidth{0.640444pt}%
\definecolor{currentstroke}{rgb}{0.257322,0.256130,0.526563}%
\pgfsetstrokecolor{currentstroke}%
\pgfsetdash{}{0pt}%
\pgfpathmoveto{\pgfqpoint{2.140041in}{5.343796in}}%
\pgfpathlineto{\pgfqpoint{2.147872in}{5.316665in}}%
\pgfusepath{stroke}%
\end{pgfscope}%
\begin{pgfscope}%
\pgfpathrectangle{\pgfqpoint{1.250000in}{4.155455in}}{\pgfqpoint{2.279412in}{2.004545in}}%
\pgfusepath{clip}%
\pgfsetbuttcap%
\pgfsetroundjoin%
\pgfsetlinewidth{0.666141pt}%
\definecolor{currentstroke}{rgb}{0.250425,0.274290,0.533103}%
\pgfsetstrokecolor{currentstroke}%
\pgfsetdash{}{0pt}%
\pgfpathmoveto{\pgfqpoint{2.147872in}{5.316665in}}%
\pgfpathlineto{\pgfqpoint{2.148339in}{5.291142in}}%
\pgfusepath{stroke}%
\end{pgfscope}%
\begin{pgfscope}%
\pgfpathrectangle{\pgfqpoint{1.250000in}{4.155455in}}{\pgfqpoint{2.279412in}{2.004545in}}%
\pgfusepath{clip}%
\pgfsetbuttcap%
\pgfsetroundjoin%
\pgfsetlinewidth{0.568752pt}%
\definecolor{currentstroke}{rgb}{0.271828,0.209303,0.504434}%
\pgfsetstrokecolor{currentstroke}%
\pgfsetdash{}{0pt}%
\pgfpathmoveto{\pgfqpoint{2.148339in}{5.291142in}}%
\pgfpathlineto{\pgfqpoint{2.144148in}{5.251281in}}%
\pgfusepath{stroke}%
\end{pgfscope}%
\begin{pgfscope}%
\pgfpathrectangle{\pgfqpoint{1.250000in}{4.155455in}}{\pgfqpoint{2.279412in}{2.004545in}}%
\pgfusepath{clip}%
\pgfsetbuttcap%
\pgfsetroundjoin%
\pgfsetlinewidth{0.501284pt}%
\definecolor{currentstroke}{rgb}{0.280868,0.160771,0.472899}%
\pgfsetstrokecolor{currentstroke}%
\pgfsetdash{}{0pt}%
\pgfpathmoveto{\pgfqpoint{2.144148in}{5.251281in}}%
\pgfpathlineto{\pgfqpoint{2.141746in}{5.215466in}}%
\pgfusepath{stroke}%
\end{pgfscope}%
\begin{pgfscope}%
\pgfpathrectangle{\pgfqpoint{1.250000in}{4.155455in}}{\pgfqpoint{2.279412in}{2.004545in}}%
\pgfusepath{clip}%
\pgfsetbuttcap%
\pgfsetroundjoin%
\pgfsetlinewidth{0.467723pt}%
\definecolor{currentstroke}{rgb}{0.282884,0.135920,0.453427}%
\pgfsetstrokecolor{currentstroke}%
\pgfsetdash{}{0pt}%
\pgfpathmoveto{\pgfqpoint{2.141746in}{5.215466in}}%
\pgfpathlineto{\pgfqpoint{2.141746in}{5.215466in}}%
\pgfusepath{stroke}%
\end{pgfscope}%
\begin{pgfscope}%
\pgfpathrectangle{\pgfqpoint{1.250000in}{4.155455in}}{\pgfqpoint{2.279412in}{2.004545in}}%
\pgfusepath{clip}%
\pgfsetbuttcap%
\pgfsetroundjoin%
\pgfsetlinewidth{0.467723pt}%
\definecolor{currentstroke}{rgb}{0.282884,0.135920,0.453427}%
\pgfsetstrokecolor{currentstroke}%
\pgfsetdash{}{0pt}%
\pgfpathmoveto{\pgfqpoint{2.141746in}{5.215466in}}%
\pgfpathlineto{\pgfqpoint{2.142444in}{5.195952in}}%
\pgfusepath{stroke}%
\end{pgfscope}%
\begin{pgfscope}%
\pgfpathrectangle{\pgfqpoint{1.250000in}{4.155455in}}{\pgfqpoint{2.279412in}{2.004545in}}%
\pgfusepath{clip}%
\pgfsetbuttcap%
\pgfsetroundjoin%
\pgfsetlinewidth{0.389647pt}%
\definecolor{currentstroke}{rgb}{0.280267,0.073417,0.397163}%
\pgfsetstrokecolor{currentstroke}%
\pgfsetdash{}{0pt}%
\pgfpathmoveto{\pgfqpoint{2.142444in}{5.195952in}}%
\pgfpathlineto{\pgfqpoint{2.142444in}{5.195952in}}%
\pgfusepath{stroke}%
\end{pgfscope}%
\begin{pgfscope}%
\pgfpathrectangle{\pgfqpoint{1.250000in}{4.155455in}}{\pgfqpoint{2.279412in}{2.004545in}}%
\pgfusepath{clip}%
\pgfsetbuttcap%
\pgfsetroundjoin%
\pgfsetlinewidth{0.389647pt}%
\definecolor{currentstroke}{rgb}{0.280267,0.073417,0.397163}%
\pgfsetstrokecolor{currentstroke}%
\pgfsetdash{}{0pt}%
\pgfpathmoveto{\pgfqpoint{2.142444in}{5.195952in}}%
\pgfpathlineto{\pgfqpoint{2.143126in}{5.188047in}}%
\pgfusepath{stroke}%
\end{pgfscope}%
\begin{pgfscope}%
\pgfpathrectangle{\pgfqpoint{1.250000in}{4.155455in}}{\pgfqpoint{2.279412in}{2.004545in}}%
\pgfusepath{clip}%
\pgfsetbuttcap%
\pgfsetroundjoin%
\pgfsetlinewidth{0.868268pt}%
\definecolor{currentstroke}{rgb}{0.195860,0.395433,0.555276}%
\pgfsetstrokecolor{currentstroke}%
\pgfsetdash{}{0pt}%
\pgfpathmoveto{\pgfqpoint{1.983758in}{4.916790in}}%
\pgfpathlineto{\pgfqpoint{2.030663in}{4.932193in}}%
\pgfusepath{stroke}%
\end{pgfscope}%
\begin{pgfscope}%
\pgfpathrectangle{\pgfqpoint{1.250000in}{4.155455in}}{\pgfqpoint{2.279412in}{2.004545in}}%
\pgfusepath{clip}%
\pgfsetbuttcap%
\pgfsetroundjoin%
\pgfsetlinewidth{0.684650pt}%
\definecolor{currentstroke}{rgb}{0.246811,0.283237,0.535941}%
\pgfsetstrokecolor{currentstroke}%
\pgfsetdash{}{0pt}%
\pgfpathmoveto{\pgfqpoint{2.030663in}{4.932193in}}%
\pgfpathlineto{\pgfqpoint{2.075039in}{4.951878in}}%
\pgfusepath{stroke}%
\end{pgfscope}%
\begin{pgfscope}%
\pgfpathrectangle{\pgfqpoint{1.250000in}{4.155455in}}{\pgfqpoint{2.279412in}{2.004545in}}%
\pgfusepath{clip}%
\pgfsetbuttcap%
\pgfsetroundjoin%
\pgfsetlinewidth{0.734637pt}%
\definecolor{currentstroke}{rgb}{0.231674,0.318106,0.544834}%
\pgfsetstrokecolor{currentstroke}%
\pgfsetdash{}{0pt}%
\pgfpathmoveto{\pgfqpoint{2.075039in}{4.951878in}}%
\pgfpathlineto{\pgfqpoint{2.075039in}{4.951878in}}%
\pgfusepath{stroke}%
\end{pgfscope}%
\begin{pgfscope}%
\pgfpathrectangle{\pgfqpoint{1.250000in}{4.155455in}}{\pgfqpoint{2.279412in}{2.004545in}}%
\pgfusepath{clip}%
\pgfsetbuttcap%
\pgfsetroundjoin%
\pgfsetlinewidth{0.734637pt}%
\definecolor{currentstroke}{rgb}{0.231674,0.318106,0.544834}%
\pgfsetstrokecolor{currentstroke}%
\pgfsetdash{}{0pt}%
\pgfpathmoveto{\pgfqpoint{2.075039in}{4.951878in}}%
\pgfpathlineto{\pgfqpoint{2.102782in}{4.968454in}}%
\pgfusepath{stroke}%
\end{pgfscope}%
\begin{pgfscope}%
\pgfpathrectangle{\pgfqpoint{1.250000in}{4.155455in}}{\pgfqpoint{2.279412in}{2.004545in}}%
\pgfusepath{clip}%
\pgfsetbuttcap%
\pgfsetroundjoin%
\pgfsetlinewidth{0.589586pt}%
\definecolor{currentstroke}{rgb}{0.267968,0.223549,0.512008}%
\pgfsetstrokecolor{currentstroke}%
\pgfsetdash{}{0pt}%
\pgfpathmoveto{\pgfqpoint{2.102782in}{4.968454in}}%
\pgfpathlineto{\pgfqpoint{2.102782in}{4.968454in}}%
\pgfusepath{stroke}%
\end{pgfscope}%
\begin{pgfscope}%
\pgfpathrectangle{\pgfqpoint{1.250000in}{4.155455in}}{\pgfqpoint{2.279412in}{2.004545in}}%
\pgfusepath{clip}%
\pgfsetbuttcap%
\pgfsetroundjoin%
\pgfsetlinewidth{0.589586pt}%
\definecolor{currentstroke}{rgb}{0.267968,0.223549,0.512008}%
\pgfsetstrokecolor{currentstroke}%
\pgfsetdash{}{0pt}%
\pgfpathmoveto{\pgfqpoint{2.102782in}{4.968454in}}%
\pgfpathlineto{\pgfqpoint{2.102782in}{4.968454in}}%
\pgfusepath{stroke}%
\end{pgfscope}%
\begin{pgfscope}%
\pgfpathrectangle{\pgfqpoint{1.250000in}{4.155455in}}{\pgfqpoint{2.279412in}{2.004545in}}%
\pgfusepath{clip}%
\pgfsetbuttcap%
\pgfsetroundjoin%
\pgfsetlinewidth{0.589586pt}%
\definecolor{currentstroke}{rgb}{0.267968,0.223549,0.512008}%
\pgfsetstrokecolor{currentstroke}%
\pgfsetdash{}{0pt}%
\pgfpathmoveto{\pgfqpoint{2.102782in}{4.968454in}}%
\pgfpathlineto{\pgfqpoint{2.118714in}{4.985469in}}%
\pgfusepath{stroke}%
\end{pgfscope}%
\begin{pgfscope}%
\pgfpathrectangle{\pgfqpoint{1.250000in}{4.155455in}}{\pgfqpoint{2.279412in}{2.004545in}}%
\pgfusepath{clip}%
\pgfsetbuttcap%
\pgfsetroundjoin%
\pgfsetlinewidth{0.684853pt}%
\definecolor{currentstroke}{rgb}{0.246811,0.283237,0.535941}%
\pgfsetstrokecolor{currentstroke}%
\pgfsetdash{}{0pt}%
\pgfpathmoveto{\pgfqpoint{2.118714in}{4.985469in}}%
\pgfpathlineto{\pgfqpoint{2.130413in}{5.002863in}}%
\pgfusepath{stroke}%
\end{pgfscope}%
\begin{pgfscope}%
\pgfpathrectangle{\pgfqpoint{1.250000in}{4.155455in}}{\pgfqpoint{2.279412in}{2.004545in}}%
\pgfusepath{clip}%
\pgfsetbuttcap%
\pgfsetroundjoin%
\pgfsetlinewidth{0.602730pt}%
\definecolor{currentstroke}{rgb}{0.266580,0.228262,0.514349}%
\pgfsetstrokecolor{currentstroke}%
\pgfsetdash{}{0pt}%
\pgfpathmoveto{\pgfqpoint{2.130413in}{5.002863in}}%
\pgfpathlineto{\pgfqpoint{2.130413in}{5.002863in}}%
\pgfusepath{stroke}%
\end{pgfscope}%
\begin{pgfscope}%
\pgfpathrectangle{\pgfqpoint{1.250000in}{4.155455in}}{\pgfqpoint{2.279412in}{2.004545in}}%
\pgfusepath{clip}%
\pgfsetbuttcap%
\pgfsetroundjoin%
\pgfsetlinewidth{0.602730pt}%
\definecolor{currentstroke}{rgb}{0.266580,0.228262,0.514349}%
\pgfsetstrokecolor{currentstroke}%
\pgfsetdash{}{0pt}%
\pgfpathmoveto{\pgfqpoint{2.130413in}{5.002863in}}%
\pgfpathlineto{\pgfqpoint{2.140975in}{5.021723in}}%
\pgfusepath{stroke}%
\end{pgfscope}%
\begin{pgfscope}%
\pgfpathrectangle{\pgfqpoint{1.250000in}{4.155455in}}{\pgfqpoint{2.279412in}{2.004545in}}%
\pgfusepath{clip}%
\pgfsetbuttcap%
\pgfsetroundjoin%
\pgfsetlinewidth{0.593750pt}%
\definecolor{currentstroke}{rgb}{0.267968,0.223549,0.512008}%
\pgfsetstrokecolor{currentstroke}%
\pgfsetdash{}{0pt}%
\pgfpathmoveto{\pgfqpoint{2.140975in}{5.021723in}}%
\pgfpathlineto{\pgfqpoint{2.140975in}{5.021723in}}%
\pgfusepath{stroke}%
\end{pgfscope}%
\begin{pgfscope}%
\pgfpathrectangle{\pgfqpoint{1.250000in}{4.155455in}}{\pgfqpoint{2.279412in}{2.004545in}}%
\pgfusepath{clip}%
\pgfsetbuttcap%
\pgfsetroundjoin%
\pgfsetlinewidth{0.593750pt}%
\definecolor{currentstroke}{rgb}{0.267968,0.223549,0.512008}%
\pgfsetstrokecolor{currentstroke}%
\pgfsetdash{}{0pt}%
\pgfpathmoveto{\pgfqpoint{2.140975in}{5.021723in}}%
\pgfpathlineto{\pgfqpoint{2.144215in}{5.040934in}}%
\pgfusepath{stroke}%
\end{pgfscope}%
\begin{pgfscope}%
\pgfpathrectangle{\pgfqpoint{1.250000in}{4.155455in}}{\pgfqpoint{2.279412in}{2.004545in}}%
\pgfusepath{clip}%
\pgfsetbuttcap%
\pgfsetroundjoin%
\pgfsetlinewidth{0.684284pt}%
\definecolor{currentstroke}{rgb}{0.246811,0.283237,0.535941}%
\pgfsetstrokecolor{currentstroke}%
\pgfsetdash{}{0pt}%
\pgfpathmoveto{\pgfqpoint{2.644824in}{5.387186in}}%
\pgfpathlineto{\pgfqpoint{2.594873in}{5.383261in}}%
\pgfusepath{stroke}%
\end{pgfscope}%
\begin{pgfscope}%
\pgfpathrectangle{\pgfqpoint{1.250000in}{4.155455in}}{\pgfqpoint{2.279412in}{2.004545in}}%
\pgfusepath{clip}%
\pgfsetbuttcap%
\pgfsetroundjoin%
\pgfsetlinewidth{0.706442pt}%
\definecolor{currentstroke}{rgb}{0.239346,0.300855,0.540844}%
\pgfsetstrokecolor{currentstroke}%
\pgfsetdash{}{0pt}%
\pgfpathmoveto{\pgfqpoint{2.594873in}{5.383261in}}%
\pgfpathlineto{\pgfqpoint{2.544982in}{5.378797in}}%
\pgfusepath{stroke}%
\end{pgfscope}%
\begin{pgfscope}%
\pgfpathrectangle{\pgfqpoint{1.250000in}{4.155455in}}{\pgfqpoint{2.279412in}{2.004545in}}%
\pgfusepath{clip}%
\pgfsetbuttcap%
\pgfsetroundjoin%
\pgfsetlinewidth{0.746721pt}%
\definecolor{currentstroke}{rgb}{0.229739,0.322361,0.545706}%
\pgfsetstrokecolor{currentstroke}%
\pgfsetdash{}{0pt}%
\pgfpathmoveto{\pgfqpoint{2.544982in}{5.378797in}}%
\pgfpathlineto{\pgfqpoint{2.495150in}{5.373830in}}%
\pgfusepath{stroke}%
\end{pgfscope}%
\begin{pgfscope}%
\pgfpathrectangle{\pgfqpoint{1.250000in}{4.155455in}}{\pgfqpoint{2.279412in}{2.004545in}}%
\pgfusepath{clip}%
\pgfsetbuttcap%
\pgfsetroundjoin%
\pgfsetlinewidth{0.765026pt}%
\definecolor{currentstroke}{rgb}{0.223925,0.334994,0.548053}%
\pgfsetstrokecolor{currentstroke}%
\pgfsetdash{}{0pt}%
\pgfpathmoveto{\pgfqpoint{2.495150in}{5.373830in}}%
\pgfpathlineto{\pgfqpoint{2.445413in}{5.368199in}}%
\pgfusepath{stroke}%
\end{pgfscope}%
\begin{pgfscope}%
\pgfpathrectangle{\pgfqpoint{1.250000in}{4.155455in}}{\pgfqpoint{2.279412in}{2.004545in}}%
\pgfusepath{clip}%
\pgfsetbuttcap%
\pgfsetroundjoin%
\pgfsetlinewidth{0.773978pt}%
\definecolor{currentstroke}{rgb}{0.221989,0.339161,0.548752}%
\pgfsetstrokecolor{currentstroke}%
\pgfsetdash{}{0pt}%
\pgfpathmoveto{\pgfqpoint{2.445413in}{5.368199in}}%
\pgfpathlineto{\pgfqpoint{2.395882in}{5.361333in}}%
\pgfusepath{stroke}%
\end{pgfscope}%
\begin{pgfscope}%
\pgfpathrectangle{\pgfqpoint{1.250000in}{4.155455in}}{\pgfqpoint{2.279412in}{2.004545in}}%
\pgfusepath{clip}%
\pgfsetroundcap%
\pgfsetroundjoin%
\pgfsetlinewidth{0.772768pt}%
\definecolor{currentstroke}{rgb}{0.221989,0.339161,0.548752}%
\pgfsetstrokecolor{currentstroke}%
\pgfsetdash{}{0pt}%
\pgfpathmoveto{\pgfqpoint{2.660254in}{5.072045in}}%
\pgfpathquadraticcurveto{\pgfqpoint{2.647722in}{5.072377in}}{\pgfqpoint{2.647141in}{5.072392in}}%
\pgfusepath{stroke}%
\end{pgfscope}%
\begin{pgfscope}%
\pgfpathrectangle{\pgfqpoint{1.250000in}{4.155455in}}{\pgfqpoint{2.279412in}{2.004545in}}%
\pgfusepath{clip}%
\pgfsetroundcap%
\pgfsetroundjoin%
\definecolor{currentfill}{rgb}{0.221989,0.339161,0.548752}%
\pgfsetfillcolor{currentfill}%
\pgfsetlinewidth{0.772768pt}%
\definecolor{currentstroke}{rgb}{0.221989,0.339161,0.548752}%
\pgfsetstrokecolor{currentstroke}%
\pgfsetdash{}{0pt}%
\pgfpathmoveto{\pgfqpoint{2.701942in}{5.043154in}}%
\pgfpathlineto{\pgfqpoint{2.647141in}{5.072392in}}%
\pgfpathlineto{\pgfqpoint{2.703412in}{5.098690in}}%
\pgfpathlineto{\pgfqpoint{2.701942in}{5.043154in}}%
\pgfpathlineto{\pgfqpoint{2.701942in}{5.043154in}}%
\pgfpathclose%
\pgfusepath{stroke,fill}%
\end{pgfscope}%
\begin{pgfscope}%
\pgfpathrectangle{\pgfqpoint{1.250000in}{4.155455in}}{\pgfqpoint{2.279412in}{2.004545in}}%
\pgfusepath{clip}%
\pgfsetroundcap%
\pgfsetroundjoin%
\pgfsetlinewidth{0.657848pt}%
\definecolor{currentstroke}{rgb}{0.253935,0.265254,0.529983}%
\pgfsetstrokecolor{currentstroke}%
\pgfsetdash{}{0pt}%
\pgfpathmoveto{\pgfqpoint{2.760260in}{5.157111in}}%
\pgfpathquadraticcurveto{\pgfqpoint{2.747722in}{5.157088in}}{\pgfqpoint{2.745361in}{5.157084in}}%
\pgfusepath{stroke}%
\end{pgfscope}%
\begin{pgfscope}%
\pgfpathrectangle{\pgfqpoint{1.250000in}{4.155455in}}{\pgfqpoint{2.279412in}{2.004545in}}%
\pgfusepath{clip}%
\pgfsetroundcap%
\pgfsetroundjoin%
\definecolor{currentfill}{rgb}{0.253935,0.265254,0.529983}%
\pgfsetfillcolor{currentfill}%
\pgfsetlinewidth{0.657848pt}%
\definecolor{currentstroke}{rgb}{0.253935,0.265254,0.529983}%
\pgfsetstrokecolor{currentstroke}%
\pgfsetdash{}{0pt}%
\pgfpathmoveto{\pgfqpoint{2.800967in}{5.129407in}}%
\pgfpathlineto{\pgfqpoint{2.745361in}{5.157084in}}%
\pgfpathlineto{\pgfqpoint{2.800866in}{5.184963in}}%
\pgfpathlineto{\pgfqpoint{2.800967in}{5.129407in}}%
\pgfpathlineto{\pgfqpoint{2.800967in}{5.129407in}}%
\pgfpathclose%
\pgfusepath{stroke,fill}%
\end{pgfscope}%
\begin{pgfscope}%
\pgfpathrectangle{\pgfqpoint{1.250000in}{4.155455in}}{\pgfqpoint{2.279412in}{2.004545in}}%
\pgfusepath{clip}%
\pgfsetroundcap%
\pgfsetroundjoin%
\pgfsetlinewidth{0.675813pt}%
\definecolor{currentstroke}{rgb}{0.248629,0.278775,0.534556}%
\pgfsetstrokecolor{currentstroke}%
\pgfsetdash{}{0pt}%
\pgfpathmoveto{\pgfqpoint{2.760698in}{5.204200in}}%
\pgfpathquadraticcurveto{\pgfqpoint{2.748161in}{5.204039in}}{\pgfqpoint{2.746079in}{5.204012in}}%
\pgfusepath{stroke}%
\end{pgfscope}%
\begin{pgfscope}%
\pgfpathrectangle{\pgfqpoint{1.250000in}{4.155455in}}{\pgfqpoint{2.279412in}{2.004545in}}%
\pgfusepath{clip}%
\pgfsetroundcap%
\pgfsetroundjoin%
\definecolor{currentfill}{rgb}{0.248629,0.278775,0.534556}%
\pgfsetfillcolor{currentfill}%
\pgfsetlinewidth{0.675813pt}%
\definecolor{currentstroke}{rgb}{0.248629,0.278775,0.534556}%
\pgfsetstrokecolor{currentstroke}%
\pgfsetdash{}{0pt}%
\pgfpathmoveto{\pgfqpoint{2.801986in}{5.176949in}}%
\pgfpathlineto{\pgfqpoint{2.746079in}{5.204012in}}%
\pgfpathlineto{\pgfqpoint{2.801274in}{5.232500in}}%
\pgfpathlineto{\pgfqpoint{2.801986in}{5.176949in}}%
\pgfpathlineto{\pgfqpoint{2.801986in}{5.176949in}}%
\pgfpathclose%
\pgfusepath{stroke,fill}%
\end{pgfscope}%
\begin{pgfscope}%
\pgfpathrectangle{\pgfqpoint{1.250000in}{4.155455in}}{\pgfqpoint{2.279412in}{2.004545in}}%
\pgfusepath{clip}%
\pgfsetroundcap%
\pgfsetroundjoin%
\pgfsetlinewidth{0.602633pt}%
\definecolor{currentstroke}{rgb}{0.266580,0.228262,0.514349}%
\pgfsetstrokecolor{currentstroke}%
\pgfsetdash{}{0pt}%
\pgfpathmoveto{\pgfqpoint{2.763028in}{5.281668in}}%
\pgfpathquadraticcurveto{\pgfqpoint{2.750498in}{5.281275in}}{\pgfqpoint{2.747287in}{5.281174in}}%
\pgfusepath{stroke}%
\end{pgfscope}%
\begin{pgfscope}%
\pgfpathrectangle{\pgfqpoint{1.250000in}{4.155455in}}{\pgfqpoint{2.279412in}{2.004545in}}%
\pgfusepath{clip}%
\pgfsetroundcap%
\pgfsetroundjoin%
\definecolor{currentfill}{rgb}{0.266580,0.228262,0.514349}%
\pgfsetfillcolor{currentfill}%
\pgfsetlinewidth{0.602633pt}%
\definecolor{currentstroke}{rgb}{0.266580,0.228262,0.514349}%
\pgfsetstrokecolor{currentstroke}%
\pgfsetdash{}{0pt}%
\pgfpathmoveto{\pgfqpoint{2.803685in}{5.255151in}}%
\pgfpathlineto{\pgfqpoint{2.747287in}{5.281174in}}%
\pgfpathlineto{\pgfqpoint{2.801944in}{5.310680in}}%
\pgfpathlineto{\pgfqpoint{2.803685in}{5.255151in}}%
\pgfpathlineto{\pgfqpoint{2.803685in}{5.255151in}}%
\pgfpathclose%
\pgfusepath{stroke,fill}%
\end{pgfscope}%
\begin{pgfscope}%
\pgfpathrectangle{\pgfqpoint{1.250000in}{4.155455in}}{\pgfqpoint{2.279412in}{2.004545in}}%
\pgfusepath{clip}%
\pgfsetroundcap%
\pgfsetroundjoin%
\pgfsetlinewidth{0.366837pt}%
\definecolor{currentstroke}{rgb}{0.277941,0.056324,0.381191}%
\pgfsetstrokecolor{currentstroke}%
\pgfsetdash{}{0pt}%
\pgfpathmoveto{\pgfqpoint{3.027764in}{5.328542in}}%
\pgfpathquadraticcurveto{\pgfqpoint{3.015231in}{5.328238in}}{\pgfqpoint{3.008371in}{5.328071in}}%
\pgfusepath{stroke}%
\end{pgfscope}%
\begin{pgfscope}%
\pgfpathrectangle{\pgfqpoint{1.250000in}{4.155455in}}{\pgfqpoint{2.279412in}{2.004545in}}%
\pgfusepath{clip}%
\pgfsetroundcap%
\pgfsetroundjoin%
\definecolor{currentfill}{rgb}{0.277941,0.056324,0.381191}%
\pgfsetfillcolor{currentfill}%
\pgfsetlinewidth{0.366837pt}%
\definecolor{currentstroke}{rgb}{0.277941,0.056324,0.381191}%
\pgfsetstrokecolor{currentstroke}%
\pgfsetdash{}{0pt}%
\pgfpathmoveto{\pgfqpoint{3.064584in}{5.301648in}}%
\pgfpathlineto{\pgfqpoint{3.008371in}{5.328071in}}%
\pgfpathlineto{\pgfqpoint{3.063237in}{5.357188in}}%
\pgfpathlineto{\pgfqpoint{3.064584in}{5.301648in}}%
\pgfpathlineto{\pgfqpoint{3.064584in}{5.301648in}}%
\pgfpathclose%
\pgfusepath{stroke,fill}%
\end{pgfscope}%
\begin{pgfscope}%
\pgfpathrectangle{\pgfqpoint{1.250000in}{4.155455in}}{\pgfqpoint{2.279412in}{2.004545in}}%
\pgfusepath{clip}%
\pgfsetroundcap%
\pgfsetroundjoin%
\pgfsetlinewidth{0.488952pt}%
\definecolor{currentstroke}{rgb}{0.281887,0.150881,0.465405}%
\pgfsetstrokecolor{currentstroke}%
\pgfsetdash{}{0pt}%
\pgfpathmoveto{\pgfqpoint{2.859428in}{5.028649in}}%
\pgfpathquadraticcurveto{\pgfqpoint{2.846896in}{5.028975in}}{\pgfqpoint{2.841925in}{5.029105in}}%
\pgfusepath{stroke}%
\end{pgfscope}%
\begin{pgfscope}%
\pgfpathrectangle{\pgfqpoint{1.250000in}{4.155455in}}{\pgfqpoint{2.279412in}{2.004545in}}%
\pgfusepath{clip}%
\pgfsetroundcap%
\pgfsetroundjoin%
\definecolor{currentfill}{rgb}{0.281887,0.150881,0.465405}%
\pgfsetfillcolor{currentfill}%
\pgfsetlinewidth{0.488952pt}%
\definecolor{currentstroke}{rgb}{0.281887,0.150881,0.465405}%
\pgfsetstrokecolor{currentstroke}%
\pgfsetdash{}{0pt}%
\pgfpathmoveto{\pgfqpoint{2.896740in}{4.999892in}}%
\pgfpathlineto{\pgfqpoint{2.841925in}{5.029105in}}%
\pgfpathlineto{\pgfqpoint{2.898184in}{5.055429in}}%
\pgfpathlineto{\pgfqpoint{2.896740in}{4.999892in}}%
\pgfpathlineto{\pgfqpoint{2.896740in}{4.999892in}}%
\pgfpathclose%
\pgfusepath{stroke,fill}%
\end{pgfscope}%
\begin{pgfscope}%
\pgfpathrectangle{\pgfqpoint{1.250000in}{4.155455in}}{\pgfqpoint{2.279412in}{2.004545in}}%
\pgfusepath{clip}%
\pgfsetroundcap%
\pgfsetroundjoin%
\pgfsetlinewidth{0.509446pt}%
\definecolor{currentstroke}{rgb}{0.280255,0.165693,0.476498}%
\pgfsetstrokecolor{currentstroke}%
\pgfsetdash{}{0pt}%
\pgfpathmoveto{\pgfqpoint{2.859666in}{5.244118in}}%
\pgfpathquadraticcurveto{\pgfqpoint{2.847131in}{5.243857in}}{\pgfqpoint{2.842477in}{5.243760in}}%
\pgfusepath{stroke}%
\end{pgfscope}%
\begin{pgfscope}%
\pgfpathrectangle{\pgfqpoint{1.250000in}{4.155455in}}{\pgfqpoint{2.279412in}{2.004545in}}%
\pgfusepath{clip}%
\pgfsetroundcap%
\pgfsetroundjoin%
\definecolor{currentfill}{rgb}{0.280255,0.165693,0.476498}%
\pgfsetfillcolor{currentfill}%
\pgfsetlinewidth{0.509446pt}%
\definecolor{currentstroke}{rgb}{0.280255,0.165693,0.476498}%
\pgfsetstrokecolor{currentstroke}%
\pgfsetdash{}{0pt}%
\pgfpathmoveto{\pgfqpoint{2.898597in}{5.217143in}}%
\pgfpathlineto{\pgfqpoint{2.842477in}{5.243760in}}%
\pgfpathlineto{\pgfqpoint{2.897443in}{5.272687in}}%
\pgfpathlineto{\pgfqpoint{2.898597in}{5.217143in}}%
\pgfpathlineto{\pgfqpoint{2.898597in}{5.217143in}}%
\pgfpathclose%
\pgfusepath{stroke,fill}%
\end{pgfscope}%
\begin{pgfscope}%
\pgfpathrectangle{\pgfqpoint{1.250000in}{4.155455in}}{\pgfqpoint{2.279412in}{2.004545in}}%
\pgfusepath{clip}%
\pgfsetroundcap%
\pgfsetroundjoin%
\pgfsetlinewidth{0.676486pt}%
\definecolor{currentstroke}{rgb}{0.248629,0.278775,0.534556}%
\pgfsetstrokecolor{currentstroke}%
\pgfsetdash{}{0pt}%
\pgfpathmoveto{\pgfqpoint{2.659546in}{5.359075in}}%
\pgfpathquadraticcurveto{\pgfqpoint{2.647045in}{5.358230in}}{\pgfqpoint{2.644986in}{5.358090in}}%
\pgfusepath{stroke}%
\end{pgfscope}%
\begin{pgfscope}%
\pgfpathrectangle{\pgfqpoint{1.250000in}{4.155455in}}{\pgfqpoint{2.279412in}{2.004545in}}%
\pgfusepath{clip}%
\pgfsetroundcap%
\pgfsetroundjoin%
\definecolor{currentfill}{rgb}{0.248629,0.278775,0.534556}%
\pgfsetfillcolor{currentfill}%
\pgfsetlinewidth{0.676486pt}%
\definecolor{currentstroke}{rgb}{0.248629,0.278775,0.534556}%
\pgfsetstrokecolor{currentstroke}%
\pgfsetdash{}{0pt}%
\pgfpathmoveto{\pgfqpoint{2.702288in}{5.334123in}}%
\pgfpathlineto{\pgfqpoint{2.644986in}{5.358090in}}%
\pgfpathlineto{\pgfqpoint{2.698541in}{5.389552in}}%
\pgfpathlineto{\pgfqpoint{2.702288in}{5.334123in}}%
\pgfpathlineto{\pgfqpoint{2.702288in}{5.334123in}}%
\pgfpathclose%
\pgfusepath{stroke,fill}%
\end{pgfscope}%
\begin{pgfscope}%
\pgfpathrectangle{\pgfqpoint{1.250000in}{4.155455in}}{\pgfqpoint{2.279412in}{2.004545in}}%
\pgfusepath{clip}%
\pgfsetroundcap%
\pgfsetroundjoin%
\pgfsetlinewidth{0.631326pt}%
\definecolor{currentstroke}{rgb}{0.258965,0.251537,0.524736}%
\pgfsetstrokecolor{currentstroke}%
\pgfsetdash{}{0pt}%
\pgfpathmoveto{\pgfqpoint{2.412881in}{4.824586in}}%
\pgfpathquadraticcurveto{\pgfqpoint{2.400851in}{4.827678in}}{\pgfqpoint{2.398280in}{4.828339in}}%
\pgfusepath{stroke}%
\end{pgfscope}%
\begin{pgfscope}%
\pgfpathrectangle{\pgfqpoint{1.250000in}{4.155455in}}{\pgfqpoint{2.279412in}{2.004545in}}%
\pgfusepath{clip}%
\pgfsetroundcap%
\pgfsetroundjoin%
\definecolor{currentfill}{rgb}{0.258965,0.251537,0.524736}%
\pgfsetfillcolor{currentfill}%
\pgfsetlinewidth{0.631326pt}%
\definecolor{currentstroke}{rgb}{0.258965,0.251537,0.524736}%
\pgfsetstrokecolor{currentstroke}%
\pgfsetdash{}{0pt}%
\pgfpathmoveto{\pgfqpoint{2.445173in}{4.787607in}}%
\pgfpathlineto{\pgfqpoint{2.398280in}{4.828339in}}%
\pgfpathlineto{\pgfqpoint{2.459001in}{4.841415in}}%
\pgfpathlineto{\pgfqpoint{2.445173in}{4.787607in}}%
\pgfpathlineto{\pgfqpoint{2.445173in}{4.787607in}}%
\pgfpathclose%
\pgfusepath{stroke,fill}%
\end{pgfscope}%
\begin{pgfscope}%
\pgfpathrectangle{\pgfqpoint{1.250000in}{4.155455in}}{\pgfqpoint{2.279412in}{2.004545in}}%
\pgfusepath{clip}%
\pgfsetroundcap%
\pgfsetroundjoin%
\pgfsetlinewidth{0.548635pt}%
\definecolor{currentstroke}{rgb}{0.275191,0.194905,0.496005}%
\pgfsetstrokecolor{currentstroke}%
\pgfsetdash{}{0pt}%
\pgfpathmoveto{\pgfqpoint{2.708720in}{4.866409in}}%
\pgfpathquadraticcurveto{\pgfqpoint{2.696237in}{4.867442in}}{\pgfqpoint{2.692213in}{4.867775in}}%
\pgfusepath{stroke}%
\end{pgfscope}%
\begin{pgfscope}%
\pgfpathrectangle{\pgfqpoint{1.250000in}{4.155455in}}{\pgfqpoint{2.279412in}{2.004545in}}%
\pgfusepath{clip}%
\pgfsetroundcap%
\pgfsetroundjoin%
\definecolor{currentfill}{rgb}{0.275191,0.194905,0.496005}%
\pgfsetfillcolor{currentfill}%
\pgfsetlinewidth{0.548635pt}%
\definecolor{currentstroke}{rgb}{0.275191,0.194905,0.496005}%
\pgfsetstrokecolor{currentstroke}%
\pgfsetdash{}{0pt}%
\pgfpathmoveto{\pgfqpoint{2.745289in}{4.835510in}}%
\pgfpathlineto{\pgfqpoint{2.692213in}{4.867775in}}%
\pgfpathlineto{\pgfqpoint{2.749870in}{4.890876in}}%
\pgfpathlineto{\pgfqpoint{2.745289in}{4.835510in}}%
\pgfpathlineto{\pgfqpoint{2.745289in}{4.835510in}}%
\pgfpathclose%
\pgfusepath{stroke,fill}%
\end{pgfscope}%
\begin{pgfscope}%
\pgfpathrectangle{\pgfqpoint{1.250000in}{4.155455in}}{\pgfqpoint{2.279412in}{2.004545in}}%
\pgfusepath{clip}%
\pgfsetroundcap%
\pgfsetroundjoin%
\pgfsetlinewidth{0.390832pt}%
\definecolor{currentstroke}{rgb}{0.280894,0.078907,0.402329}%
\pgfsetstrokecolor{currentstroke}%
\pgfsetdash{}{0pt}%
\pgfpathmoveto{\pgfqpoint{2.958677in}{4.894420in}}%
\pgfpathquadraticcurveto{\pgfqpoint{2.946156in}{4.894980in}}{\pgfqpoint{2.939676in}{4.895270in}}%
\pgfusepath{stroke}%
\end{pgfscope}%
\begin{pgfscope}%
\pgfpathrectangle{\pgfqpoint{1.250000in}{4.155455in}}{\pgfqpoint{2.279412in}{2.004545in}}%
\pgfusepath{clip}%
\pgfsetroundcap%
\pgfsetroundjoin%
\definecolor{currentfill}{rgb}{0.280894,0.078907,0.402329}%
\pgfsetfillcolor{currentfill}%
\pgfsetlinewidth{0.390832pt}%
\definecolor{currentstroke}{rgb}{0.280894,0.078907,0.402329}%
\pgfsetstrokecolor{currentstroke}%
\pgfsetdash{}{0pt}%
\pgfpathmoveto{\pgfqpoint{2.993933in}{4.865035in}}%
\pgfpathlineto{\pgfqpoint{2.939676in}{4.895270in}}%
\pgfpathlineto{\pgfqpoint{2.996418in}{4.920535in}}%
\pgfpathlineto{\pgfqpoint{2.993933in}{4.865035in}}%
\pgfpathlineto{\pgfqpoint{2.993933in}{4.865035in}}%
\pgfpathclose%
\pgfusepath{stroke,fill}%
\end{pgfscope}%
\begin{pgfscope}%
\pgfpathrectangle{\pgfqpoint{1.250000in}{4.155455in}}{\pgfqpoint{2.279412in}{2.004545in}}%
\pgfusepath{clip}%
\pgfsetroundcap%
\pgfsetroundjoin%
\pgfsetlinewidth{0.629383pt}%
\definecolor{currentstroke}{rgb}{0.260571,0.246922,0.522828}%
\pgfsetstrokecolor{currentstroke}%
\pgfsetdash{}{0pt}%
\pgfpathmoveto{\pgfqpoint{2.708678in}{4.951013in}}%
\pgfpathquadraticcurveto{\pgfqpoint{2.696166in}{4.951719in}}{\pgfqpoint{2.693375in}{4.951876in}}%
\pgfusepath{stroke}%
\end{pgfscope}%
\begin{pgfscope}%
\pgfpathrectangle{\pgfqpoint{1.250000in}{4.155455in}}{\pgfqpoint{2.279412in}{2.004545in}}%
\pgfusepath{clip}%
\pgfsetroundcap%
\pgfsetroundjoin%
\definecolor{currentfill}{rgb}{0.260571,0.246922,0.522828}%
\pgfsetfillcolor{currentfill}%
\pgfsetlinewidth{0.629383pt}%
\definecolor{currentstroke}{rgb}{0.260571,0.246922,0.522828}%
\pgfsetstrokecolor{currentstroke}%
\pgfsetdash{}{0pt}%
\pgfpathmoveto{\pgfqpoint{2.747278in}{4.921014in}}%
\pgfpathlineto{\pgfqpoint{2.693375in}{4.951876in}}%
\pgfpathlineto{\pgfqpoint{2.750407in}{4.976481in}}%
\pgfpathlineto{\pgfqpoint{2.747278in}{4.921014in}}%
\pgfpathlineto{\pgfqpoint{2.747278in}{4.921014in}}%
\pgfpathclose%
\pgfusepath{stroke,fill}%
\end{pgfscope}%
\begin{pgfscope}%
\pgfpathrectangle{\pgfqpoint{1.250000in}{4.155455in}}{\pgfqpoint{2.279412in}{2.004545in}}%
\pgfusepath{clip}%
\pgfsetroundcap%
\pgfsetroundjoin%
\pgfsetlinewidth{0.434830pt}%
\definecolor{currentstroke}{rgb}{0.283091,0.110553,0.431554}%
\pgfsetstrokecolor{currentstroke}%
\pgfsetdash{}{0pt}%
\pgfpathmoveto{\pgfqpoint{2.908533in}{4.984916in}}%
\pgfpathquadraticcurveto{\pgfqpoint{2.896010in}{4.985454in}}{\pgfqpoint{2.890208in}{4.985703in}}%
\pgfusepath{stroke}%
\end{pgfscope}%
\begin{pgfscope}%
\pgfpathrectangle{\pgfqpoint{1.250000in}{4.155455in}}{\pgfqpoint{2.279412in}{2.004545in}}%
\pgfusepath{clip}%
\pgfsetroundcap%
\pgfsetroundjoin%
\definecolor{currentfill}{rgb}{0.283091,0.110553,0.431554}%
\pgfsetfillcolor{currentfill}%
\pgfsetlinewidth{0.434830pt}%
\definecolor{currentstroke}{rgb}{0.283091,0.110553,0.431554}%
\pgfsetstrokecolor{currentstroke}%
\pgfsetdash{}{0pt}%
\pgfpathmoveto{\pgfqpoint{2.944521in}{4.955569in}}%
\pgfpathlineto{\pgfqpoint{2.890208in}{4.985703in}}%
\pgfpathlineto{\pgfqpoint{2.946903in}{5.011073in}}%
\pgfpathlineto{\pgfqpoint{2.944521in}{4.955569in}}%
\pgfpathlineto{\pgfqpoint{2.944521in}{4.955569in}}%
\pgfpathclose%
\pgfusepath{stroke,fill}%
\end{pgfscope}%
\begin{pgfscope}%
\pgfpathrectangle{\pgfqpoint{1.250000in}{4.155455in}}{\pgfqpoint{2.279412in}{2.004545in}}%
\pgfusepath{clip}%
\pgfsetroundcap%
\pgfsetroundjoin%
\pgfsetlinewidth{0.662941pt}%
\definecolor{currentstroke}{rgb}{0.252194,0.269783,0.531579}%
\pgfsetstrokecolor{currentstroke}%
\pgfsetdash{}{0pt}%
\pgfpathmoveto{\pgfqpoint{2.757892in}{5.114974in}}%
\pgfpathquadraticcurveto{\pgfqpoint{2.745354in}{5.115073in}}{\pgfqpoint{2.743072in}{5.115091in}}%
\pgfusepath{stroke}%
\end{pgfscope}%
\begin{pgfscope}%
\pgfpathrectangle{\pgfqpoint{1.250000in}{4.155455in}}{\pgfqpoint{2.279412in}{2.004545in}}%
\pgfusepath{clip}%
\pgfsetroundcap%
\pgfsetroundjoin%
\definecolor{currentfill}{rgb}{0.252194,0.269783,0.531579}%
\pgfsetfillcolor{currentfill}%
\pgfsetlinewidth{0.662941pt}%
\definecolor{currentstroke}{rgb}{0.252194,0.269783,0.531579}%
\pgfsetstrokecolor{currentstroke}%
\pgfsetdash{}{0pt}%
\pgfpathmoveto{\pgfqpoint{2.798406in}{5.086874in}}%
\pgfpathlineto{\pgfqpoint{2.743072in}{5.115091in}}%
\pgfpathlineto{\pgfqpoint{2.798846in}{5.142427in}}%
\pgfpathlineto{\pgfqpoint{2.798406in}{5.086874in}}%
\pgfpathlineto{\pgfqpoint{2.798406in}{5.086874in}}%
\pgfpathclose%
\pgfusepath{stroke,fill}%
\end{pgfscope}%
\begin{pgfscope}%
\pgfpathrectangle{\pgfqpoint{1.250000in}{4.155455in}}{\pgfqpoint{2.279412in}{2.004545in}}%
\pgfusepath{clip}%
\pgfsetroundcap%
\pgfsetroundjoin%
\pgfsetlinewidth{0.365128pt}%
\definecolor{currentstroke}{rgb}{0.277941,0.056324,0.381191}%
\pgfsetstrokecolor{currentstroke}%
\pgfsetdash{}{0pt}%
\pgfpathmoveto{\pgfqpoint{3.009112in}{5.418861in}}%
\pgfpathquadraticcurveto{\pgfqpoint{2.996590in}{5.418320in}}{\pgfqpoint{2.989711in}{5.418023in}}%
\pgfusepath{stroke}%
\end{pgfscope}%
\begin{pgfscope}%
\pgfpathrectangle{\pgfqpoint{1.250000in}{4.155455in}}{\pgfqpoint{2.279412in}{2.004545in}}%
\pgfusepath{clip}%
\pgfsetroundcap%
\pgfsetroundjoin%
\definecolor{currentfill}{rgb}{0.277941,0.056324,0.381191}%
\pgfsetfillcolor{currentfill}%
\pgfsetlinewidth{0.365128pt}%
\definecolor{currentstroke}{rgb}{0.277941,0.056324,0.381191}%
\pgfsetstrokecolor{currentstroke}%
\pgfsetdash{}{0pt}%
\pgfpathmoveto{\pgfqpoint{3.046414in}{5.392669in}}%
\pgfpathlineto{\pgfqpoint{2.989711in}{5.418023in}}%
\pgfpathlineto{\pgfqpoint{3.044016in}{5.448173in}}%
\pgfpathlineto{\pgfqpoint{3.046414in}{5.392669in}}%
\pgfpathlineto{\pgfqpoint{3.046414in}{5.392669in}}%
\pgfpathclose%
\pgfusepath{stroke,fill}%
\end{pgfscope}%
\begin{pgfscope}%
\pgfpathrectangle{\pgfqpoint{1.250000in}{4.155455in}}{\pgfqpoint{2.279412in}{2.004545in}}%
\pgfusepath{clip}%
\pgfsetroundcap%
\pgfsetroundjoin%
\pgfsetlinewidth{0.599244pt}%
\definecolor{currentstroke}{rgb}{0.266580,0.228262,0.514349}%
\pgfsetstrokecolor{currentstroke}%
\pgfsetdash{}{0pt}%
\pgfpathmoveto{\pgfqpoint{2.659078in}{5.441920in}}%
\pgfpathquadraticcurveto{\pgfqpoint{2.646616in}{5.440711in}}{\pgfqpoint{2.643381in}{5.440397in}}%
\pgfusepath{stroke}%
\end{pgfscope}%
\begin{pgfscope}%
\pgfpathrectangle{\pgfqpoint{1.250000in}{4.155455in}}{\pgfqpoint{2.279412in}{2.004545in}}%
\pgfusepath{clip}%
\pgfsetroundcap%
\pgfsetroundjoin%
\definecolor{currentfill}{rgb}{0.266580,0.228262,0.514349}%
\pgfsetfillcolor{currentfill}%
\pgfsetlinewidth{0.599244pt}%
\definecolor{currentstroke}{rgb}{0.266580,0.228262,0.514349}%
\pgfsetstrokecolor{currentstroke}%
\pgfsetdash{}{0pt}%
\pgfpathmoveto{\pgfqpoint{2.701360in}{5.418115in}}%
\pgfpathlineto{\pgfqpoint{2.643381in}{5.440397in}}%
\pgfpathlineto{\pgfqpoint{2.695994in}{5.473411in}}%
\pgfpathlineto{\pgfqpoint{2.701360in}{5.418115in}}%
\pgfpathlineto{\pgfqpoint{2.701360in}{5.418115in}}%
\pgfpathclose%
\pgfusepath{stroke,fill}%
\end{pgfscope}%
\begin{pgfscope}%
\pgfpathrectangle{\pgfqpoint{1.250000in}{4.155455in}}{\pgfqpoint{2.279412in}{2.004545in}}%
\pgfusepath{clip}%
\pgfsetroundcap%
\pgfsetroundjoin%
\pgfsetlinewidth{0.356068pt}%
\definecolor{currentstroke}{rgb}{0.277018,0.050344,0.375715}%
\pgfsetstrokecolor{currentstroke}%
\pgfsetdash{}{0pt}%
\pgfpathmoveto{\pgfqpoint{2.957630in}{4.802755in}}%
\pgfpathquadraticcurveto{\pgfqpoint{2.945117in}{4.803423in}}{\pgfqpoint{2.938104in}{4.803798in}}%
\pgfusepath{stroke}%
\end{pgfscope}%
\begin{pgfscope}%
\pgfpathrectangle{\pgfqpoint{1.250000in}{4.155455in}}{\pgfqpoint{2.279412in}{2.004545in}}%
\pgfusepath{clip}%
\pgfsetroundcap%
\pgfsetroundjoin%
\definecolor{currentfill}{rgb}{0.277018,0.050344,0.375715}%
\pgfsetfillcolor{currentfill}%
\pgfsetlinewidth{0.356068pt}%
\definecolor{currentstroke}{rgb}{0.277018,0.050344,0.375715}%
\pgfsetstrokecolor{currentstroke}%
\pgfsetdash{}{0pt}%
\pgfpathmoveto{\pgfqpoint{2.992098in}{4.773095in}}%
\pgfpathlineto{\pgfqpoint{2.938104in}{4.803798in}}%
\pgfpathlineto{\pgfqpoint{2.995063in}{4.828571in}}%
\pgfpathlineto{\pgfqpoint{2.992098in}{4.773095in}}%
\pgfpathlineto{\pgfqpoint{2.992098in}{4.773095in}}%
\pgfpathclose%
\pgfusepath{stroke,fill}%
\end{pgfscope}%
\begin{pgfscope}%
\pgfpathrectangle{\pgfqpoint{1.250000in}{4.155455in}}{\pgfqpoint{2.279412in}{2.004545in}}%
\pgfusepath{clip}%
\pgfsetroundcap%
\pgfsetroundjoin%
\pgfsetlinewidth{0.355939pt}%
\definecolor{currentstroke}{rgb}{0.276022,0.044167,0.370164}%
\pgfsetstrokecolor{currentstroke}%
\pgfsetdash{}{0pt}%
\pgfpathmoveto{\pgfqpoint{2.697457in}{4.571339in}}%
\pgfpathquadraticcurveto{\pgfqpoint{2.685061in}{4.572976in}}{\pgfqpoint{2.678124in}{4.573892in}}%
\pgfusepath{stroke}%
\end{pgfscope}%
\begin{pgfscope}%
\pgfpathrectangle{\pgfqpoint{1.250000in}{4.155455in}}{\pgfqpoint{2.279412in}{2.004545in}}%
\pgfusepath{clip}%
\pgfsetroundcap%
\pgfsetroundjoin%
\definecolor{currentfill}{rgb}{0.276022,0.044167,0.370164}%
\pgfsetfillcolor{currentfill}%
\pgfsetlinewidth{0.355939pt}%
\definecolor{currentstroke}{rgb}{0.276022,0.044167,0.370164}%
\pgfsetstrokecolor{currentstroke}%
\pgfsetdash{}{0pt}%
\pgfpathmoveto{\pgfqpoint{2.729564in}{4.539080in}}%
\pgfpathlineto{\pgfqpoint{2.678124in}{4.573892in}}%
\pgfpathlineto{\pgfqpoint{2.736838in}{4.594157in}}%
\pgfpathlineto{\pgfqpoint{2.729564in}{4.539080in}}%
\pgfpathlineto{\pgfqpoint{2.729564in}{4.539080in}}%
\pgfpathclose%
\pgfusepath{stroke,fill}%
\end{pgfscope}%
\begin{pgfscope}%
\pgfpathrectangle{\pgfqpoint{1.250000in}{4.155455in}}{\pgfqpoint{2.279412in}{2.004545in}}%
\pgfusepath{clip}%
\pgfsetroundcap%
\pgfsetroundjoin%
\pgfsetlinewidth{0.472823pt}%
\definecolor{currentstroke}{rgb}{0.282623,0.140926,0.457517}%
\pgfsetstrokecolor{currentstroke}%
\pgfsetdash{}{0pt}%
\pgfpathmoveto{\pgfqpoint{2.614795in}{4.714535in}}%
\pgfpathquadraticcurveto{\pgfqpoint{2.602464in}{4.716518in}}{\pgfqpoint{2.597354in}{4.717340in}}%
\pgfusepath{stroke}%
\end{pgfscope}%
\begin{pgfscope}%
\pgfpathrectangle{\pgfqpoint{1.250000in}{4.155455in}}{\pgfqpoint{2.279412in}{2.004545in}}%
\pgfusepath{clip}%
\pgfsetroundcap%
\pgfsetroundjoin%
\definecolor{currentfill}{rgb}{0.282623,0.140926,0.457517}%
\pgfsetfillcolor{currentfill}%
\pgfsetlinewidth{0.472823pt}%
\definecolor{currentstroke}{rgb}{0.282623,0.140926,0.457517}%
\pgfsetstrokecolor{currentstroke}%
\pgfsetdash{}{0pt}%
\pgfpathmoveto{\pgfqpoint{2.647795in}{4.681094in}}%
\pgfpathlineto{\pgfqpoint{2.597354in}{4.717340in}}%
\pgfpathlineto{\pgfqpoint{2.656615in}{4.735945in}}%
\pgfpathlineto{\pgfqpoint{2.647795in}{4.681094in}}%
\pgfpathlineto{\pgfqpoint{2.647795in}{4.681094in}}%
\pgfpathclose%
\pgfusepath{stroke,fill}%
\end{pgfscope}%
\begin{pgfscope}%
\pgfpathrectangle{\pgfqpoint{1.250000in}{4.155455in}}{\pgfqpoint{2.279412in}{2.004545in}}%
\pgfusepath{clip}%
\pgfsetroundcap%
\pgfsetroundjoin%
\pgfsetlinewidth{0.577294pt}%
\definecolor{currentstroke}{rgb}{0.270595,0.214069,0.507052}%
\pgfsetstrokecolor{currentstroke}%
\pgfsetdash{}{0pt}%
\pgfpathmoveto{\pgfqpoint{2.607144in}{5.482209in}}%
\pgfpathquadraticcurveto{\pgfqpoint{2.594727in}{5.480678in}}{\pgfqpoint{2.591175in}{5.480240in}}%
\pgfusepath{stroke}%
\end{pgfscope}%
\begin{pgfscope}%
\pgfpathrectangle{\pgfqpoint{1.250000in}{4.155455in}}{\pgfqpoint{2.279412in}{2.004545in}}%
\pgfusepath{clip}%
\pgfsetroundcap%
\pgfsetroundjoin%
\definecolor{currentfill}{rgb}{0.270595,0.214069,0.507052}%
\pgfsetfillcolor{currentfill}%
\pgfsetlinewidth{0.577294pt}%
\definecolor{currentstroke}{rgb}{0.270595,0.214069,0.507052}%
\pgfsetstrokecolor{currentstroke}%
\pgfsetdash{}{0pt}%
\pgfpathmoveto{\pgfqpoint{2.649713in}{5.459472in}}%
\pgfpathlineto{\pgfqpoint{2.591175in}{5.480240in}}%
\pgfpathlineto{\pgfqpoint{2.642912in}{5.514610in}}%
\pgfpathlineto{\pgfqpoint{2.649713in}{5.459472in}}%
\pgfpathlineto{\pgfqpoint{2.649713in}{5.459472in}}%
\pgfpathclose%
\pgfusepath{stroke,fill}%
\end{pgfscope}%
\begin{pgfscope}%
\pgfpathrectangle{\pgfqpoint{1.250000in}{4.155455in}}{\pgfqpoint{2.279412in}{2.004545in}}%
\pgfusepath{clip}%
\pgfsetroundcap%
\pgfsetroundjoin%
\pgfsetlinewidth{0.443362pt}%
\definecolor{currentstroke}{rgb}{0.283197,0.115680,0.436115}%
\pgfsetstrokecolor{currentstroke}%
\pgfsetdash{}{0pt}%
\pgfpathmoveto{\pgfqpoint{2.756687in}{5.543183in}}%
\pgfpathquadraticcurveto{\pgfqpoint{2.744232in}{5.541930in}}{\pgfqpoint{2.738601in}{5.541363in}}%
\pgfusepath{stroke}%
\end{pgfscope}%
\begin{pgfscope}%
\pgfpathrectangle{\pgfqpoint{1.250000in}{4.155455in}}{\pgfqpoint{2.279412in}{2.004545in}}%
\pgfusepath{clip}%
\pgfsetroundcap%
\pgfsetroundjoin%
\definecolor{currentfill}{rgb}{0.283197,0.115680,0.436115}%
\pgfsetfillcolor{currentfill}%
\pgfsetlinewidth{0.443362pt}%
\definecolor{currentstroke}{rgb}{0.283197,0.115680,0.436115}%
\pgfsetstrokecolor{currentstroke}%
\pgfsetdash{}{0pt}%
\pgfpathmoveto{\pgfqpoint{2.796658in}{5.519286in}}%
\pgfpathlineto{\pgfqpoint{2.738601in}{5.541363in}}%
\pgfpathlineto{\pgfqpoint{2.791097in}{5.574562in}}%
\pgfpathlineto{\pgfqpoint{2.796658in}{5.519286in}}%
\pgfpathlineto{\pgfqpoint{2.796658in}{5.519286in}}%
\pgfpathclose%
\pgfusepath{stroke,fill}%
\end{pgfscope}%
\begin{pgfscope}%
\pgfpathrectangle{\pgfqpoint{1.250000in}{4.155455in}}{\pgfqpoint{2.279412in}{2.004545in}}%
\pgfusepath{clip}%
\pgfsetroundcap%
\pgfsetroundjoin%
\pgfsetlinewidth{0.463586pt}%
\definecolor{currentstroke}{rgb}{0.283072,0.130895,0.449241}%
\pgfsetstrokecolor{currentstroke}%
\pgfsetdash{}{0pt}%
\pgfpathmoveto{\pgfqpoint{2.318392in}{5.632632in}}%
\pgfpathquadraticcurveto{\pgfqpoint{2.308630in}{5.625855in}}{\pgfqpoint{2.304759in}{5.623167in}}%
\pgfusepath{stroke}%
\end{pgfscope}%
\begin{pgfscope}%
\pgfpathrectangle{\pgfqpoint{1.250000in}{4.155455in}}{\pgfqpoint{2.279412in}{2.004545in}}%
\pgfusepath{clip}%
\pgfsetroundcap%
\pgfsetroundjoin%
\definecolor{currentfill}{rgb}{0.283072,0.130895,0.449241}%
\pgfsetfillcolor{currentfill}%
\pgfsetlinewidth{0.463586pt}%
\definecolor{currentstroke}{rgb}{0.283072,0.130895,0.449241}%
\pgfsetstrokecolor{currentstroke}%
\pgfsetdash{}{0pt}%
\pgfpathmoveto{\pgfqpoint{2.366236in}{5.632032in}}%
\pgfpathlineto{\pgfqpoint{2.304759in}{5.623167in}}%
\pgfpathlineto{\pgfqpoint{2.334553in}{5.677668in}}%
\pgfpathlineto{\pgfqpoint{2.366236in}{5.632032in}}%
\pgfpathlineto{\pgfqpoint{2.366236in}{5.632032in}}%
\pgfpathclose%
\pgfusepath{stroke,fill}%
\end{pgfscope}%
\begin{pgfscope}%
\pgfpathrectangle{\pgfqpoint{1.250000in}{4.155455in}}{\pgfqpoint{2.279412in}{2.004545in}}%
\pgfusepath{clip}%
\pgfsetroundcap%
\pgfsetroundjoin%
\pgfsetlinewidth{0.474515pt}%
\definecolor{currentstroke}{rgb}{0.282623,0.140926,0.457517}%
\pgfsetstrokecolor{currentstroke}%
\pgfsetdash{}{0pt}%
\pgfpathmoveto{\pgfqpoint{2.244865in}{4.734108in}}%
\pgfpathquadraticcurveto{\pgfqpoint{2.241427in}{4.740703in}}{\pgfqpoint{2.241383in}{4.740788in}}%
\pgfusepath{stroke}%
\end{pgfscope}%
\begin{pgfscope}%
\pgfpathrectangle{\pgfqpoint{1.250000in}{4.155455in}}{\pgfqpoint{2.279412in}{2.004545in}}%
\pgfusepath{clip}%
\pgfsetroundcap%
\pgfsetroundjoin%
\definecolor{currentfill}{rgb}{0.282623,0.140926,0.457517}%
\pgfsetfillcolor{currentfill}%
\pgfsetlinewidth{0.474515pt}%
\definecolor{currentstroke}{rgb}{0.282623,0.140926,0.457517}%
\pgfsetstrokecolor{currentstroke}%
\pgfsetdash{}{0pt}%
\pgfpathmoveto{\pgfqpoint{2.242438in}{4.678684in}}%
\pgfpathlineto{\pgfqpoint{2.241383in}{4.740788in}}%
\pgfpathlineto{\pgfqpoint{2.291699in}{4.704370in}}%
\pgfpathlineto{\pgfqpoint{2.242438in}{4.678684in}}%
\pgfpathlineto{\pgfqpoint{2.242438in}{4.678684in}}%
\pgfpathclose%
\pgfusepath{stroke,fill}%
\end{pgfscope}%
\begin{pgfscope}%
\pgfpathrectangle{\pgfqpoint{1.250000in}{4.155455in}}{\pgfqpoint{2.279412in}{2.004545in}}%
\pgfusepath{clip}%
\pgfsetroundcap%
\pgfsetroundjoin%
\pgfsetlinewidth{0.451495pt}%
\definecolor{currentstroke}{rgb}{0.283229,0.120777,0.440584}%
\pgfsetstrokecolor{currentstroke}%
\pgfsetdash{}{0pt}%
\pgfpathmoveto{\pgfqpoint{2.499072in}{4.643819in}}%
\pgfpathquadraticcurveto{\pgfqpoint{2.487092in}{4.647049in}}{\pgfqpoint{2.481857in}{4.648461in}}%
\pgfusepath{stroke}%
\end{pgfscope}%
\begin{pgfscope}%
\pgfpathrectangle{\pgfqpoint{1.250000in}{4.155455in}}{\pgfqpoint{2.279412in}{2.004545in}}%
\pgfusepath{clip}%
\pgfsetroundcap%
\pgfsetroundjoin%
\definecolor{currentfill}{rgb}{0.283229,0.120777,0.440584}%
\pgfsetfillcolor{currentfill}%
\pgfsetlinewidth{0.451495pt}%
\definecolor{currentstroke}{rgb}{0.283229,0.120777,0.440584}%
\pgfsetstrokecolor{currentstroke}%
\pgfsetdash{}{0pt}%
\pgfpathmoveto{\pgfqpoint{2.528264in}{4.607176in}}%
\pgfpathlineto{\pgfqpoint{2.481857in}{4.648461in}}%
\pgfpathlineto{\pgfqpoint{2.542729in}{4.660816in}}%
\pgfpathlineto{\pgfqpoint{2.528264in}{4.607176in}}%
\pgfpathlineto{\pgfqpoint{2.528264in}{4.607176in}}%
\pgfpathclose%
\pgfusepath{stroke,fill}%
\end{pgfscope}%
\begin{pgfscope}%
\pgfpathrectangle{\pgfqpoint{1.250000in}{4.155455in}}{\pgfqpoint{2.279412in}{2.004545in}}%
\pgfusepath{clip}%
\pgfsetroundcap%
\pgfsetroundjoin%
\pgfsetlinewidth{0.385369pt}%
\definecolor{currentstroke}{rgb}{0.280267,0.073417,0.397163}%
\pgfsetstrokecolor{currentstroke}%
\pgfsetdash{}{0pt}%
\pgfpathmoveto{\pgfqpoint{2.607466in}{5.654204in}}%
\pgfpathquadraticcurveto{\pgfqpoint{2.595115in}{5.652330in}}{\pgfqpoint{2.588659in}{5.651350in}}%
\pgfusepath{stroke}%
\end{pgfscope}%
\begin{pgfscope}%
\pgfpathrectangle{\pgfqpoint{1.250000in}{4.155455in}}{\pgfqpoint{2.279412in}{2.004545in}}%
\pgfusepath{clip}%
\pgfsetroundcap%
\pgfsetroundjoin%
\definecolor{currentfill}{rgb}{0.280267,0.073417,0.397163}%
\pgfsetfillcolor{currentfill}%
\pgfsetlinewidth{0.385369pt}%
\definecolor{currentstroke}{rgb}{0.280267,0.073417,0.397163}%
\pgfsetstrokecolor{currentstroke}%
\pgfsetdash{}{0pt}%
\pgfpathmoveto{\pgfqpoint{2.647753in}{5.632223in}}%
\pgfpathlineto{\pgfqpoint{2.588659in}{5.651350in}}%
\pgfpathlineto{\pgfqpoint{2.639417in}{5.687149in}}%
\pgfpathlineto{\pgfqpoint{2.647753in}{5.632223in}}%
\pgfpathlineto{\pgfqpoint{2.647753in}{5.632223in}}%
\pgfpathclose%
\pgfusepath{stroke,fill}%
\end{pgfscope}%
\begin{pgfscope}%
\pgfpathrectangle{\pgfqpoint{1.250000in}{4.155455in}}{\pgfqpoint{2.279412in}{2.004545in}}%
\pgfusepath{clip}%
\pgfsetroundcap%
\pgfsetroundjoin%
\pgfsetlinewidth{0.370426pt}%
\definecolor{currentstroke}{rgb}{0.278791,0.062145,0.386592}%
\pgfsetstrokecolor{currentstroke}%
\pgfsetdash{}{0pt}%
\pgfpathmoveto{\pgfqpoint{2.743216in}{4.639833in}}%
\pgfpathquadraticcurveto{\pgfqpoint{2.730808in}{4.641404in}}{\pgfqpoint{2.724085in}{4.642255in}}%
\pgfusepath{stroke}%
\end{pgfscope}%
\begin{pgfscope}%
\pgfpathrectangle{\pgfqpoint{1.250000in}{4.155455in}}{\pgfqpoint{2.279412in}{2.004545in}}%
\pgfusepath{clip}%
\pgfsetroundcap%
\pgfsetroundjoin%
\definecolor{currentfill}{rgb}{0.278791,0.062145,0.386592}%
\pgfsetfillcolor{currentfill}%
\pgfsetlinewidth{0.370426pt}%
\definecolor{currentstroke}{rgb}{0.278791,0.062145,0.386592}%
\pgfsetstrokecolor{currentstroke}%
\pgfsetdash{}{0pt}%
\pgfpathmoveto{\pgfqpoint{2.775711in}{4.607718in}}%
\pgfpathlineto{\pgfqpoint{2.724085in}{4.642255in}}%
\pgfpathlineto{\pgfqpoint{2.782690in}{4.662834in}}%
\pgfpathlineto{\pgfqpoint{2.775711in}{4.607718in}}%
\pgfpathlineto{\pgfqpoint{2.775711in}{4.607718in}}%
\pgfpathclose%
\pgfusepath{stroke,fill}%
\end{pgfscope}%
\begin{pgfscope}%
\pgfpathrectangle{\pgfqpoint{1.250000in}{4.155455in}}{\pgfqpoint{2.279412in}{2.004545in}}%
\pgfusepath{clip}%
\pgfsetroundcap%
\pgfsetroundjoin%
\pgfsetlinewidth{0.517767pt}%
\definecolor{currentstroke}{rgb}{0.279574,0.170599,0.479997}%
\pgfsetstrokecolor{currentstroke}%
\pgfsetdash{}{0pt}%
\pgfpathmoveto{\pgfqpoint{2.555881in}{5.566084in}}%
\pgfpathquadraticcurveto{\pgfqpoint{2.543562in}{5.564037in}}{\pgfqpoint{2.539144in}{5.563303in}}%
\pgfusepath{stroke}%
\end{pgfscope}%
\begin{pgfscope}%
\pgfpathrectangle{\pgfqpoint{1.250000in}{4.155455in}}{\pgfqpoint{2.279412in}{2.004545in}}%
\pgfusepath{clip}%
\pgfsetroundcap%
\pgfsetroundjoin%
\definecolor{currentfill}{rgb}{0.279574,0.170599,0.479997}%
\pgfsetfillcolor{currentfill}%
\pgfsetlinewidth{0.517767pt}%
\definecolor{currentstroke}{rgb}{0.279574,0.170599,0.479997}%
\pgfsetstrokecolor{currentstroke}%
\pgfsetdash{}{0pt}%
\pgfpathmoveto{\pgfqpoint{2.598501in}{5.545006in}}%
\pgfpathlineto{\pgfqpoint{2.539144in}{5.563303in}}%
\pgfpathlineto{\pgfqpoint{2.589396in}{5.599810in}}%
\pgfpathlineto{\pgfqpoint{2.598501in}{5.545006in}}%
\pgfpathlineto{\pgfqpoint{2.598501in}{5.545006in}}%
\pgfpathclose%
\pgfusepath{stroke,fill}%
\end{pgfscope}%
\begin{pgfscope}%
\pgfpathrectangle{\pgfqpoint{1.250000in}{4.155455in}}{\pgfqpoint{2.279412in}{2.004545in}}%
\pgfusepath{clip}%
\pgfsetroundcap%
\pgfsetroundjoin%
\pgfsetlinewidth{0.327907pt}%
\definecolor{currentstroke}{rgb}{0.271305,0.019942,0.347269}%
\pgfsetstrokecolor{currentstroke}%
\pgfsetdash{}{0pt}%
\pgfpathmoveto{\pgfqpoint{2.953917in}{5.653902in}}%
\pgfpathquadraticcurveto{\pgfqpoint{2.941522in}{5.652368in}}{\pgfqpoint{2.934161in}{5.651457in}}%
\pgfusepath{stroke}%
\end{pgfscope}%
\begin{pgfscope}%
\pgfpathrectangle{\pgfqpoint{1.250000in}{4.155455in}}{\pgfqpoint{2.279412in}{2.004545in}}%
\pgfusepath{clip}%
\pgfsetroundcap%
\pgfsetroundjoin%
\definecolor{currentfill}{rgb}{0.271305,0.019942,0.347269}%
\pgfsetfillcolor{currentfill}%
\pgfsetlinewidth{0.327907pt}%
\definecolor{currentstroke}{rgb}{0.271305,0.019942,0.347269}%
\pgfsetstrokecolor{currentstroke}%
\pgfsetdash{}{0pt}%
\pgfpathmoveto{\pgfqpoint{2.992708in}{5.630713in}}%
\pgfpathlineto{\pgfqpoint{2.934161in}{5.651457in}}%
\pgfpathlineto{\pgfqpoint{2.985885in}{5.685848in}}%
\pgfpathlineto{\pgfqpoint{2.992708in}{5.630713in}}%
\pgfpathlineto{\pgfqpoint{2.992708in}{5.630713in}}%
\pgfpathclose%
\pgfusepath{stroke,fill}%
\end{pgfscope}%
\begin{pgfscope}%
\pgfpathrectangle{\pgfqpoint{1.250000in}{4.155455in}}{\pgfqpoint{2.279412in}{2.004545in}}%
\pgfusepath{clip}%
\pgfsetroundcap%
\pgfsetroundjoin%
\pgfsetlinewidth{0.590181pt}%
\definecolor{currentstroke}{rgb}{0.267968,0.223549,0.512008}%
\pgfsetstrokecolor{currentstroke}%
\pgfsetdash{}{0pt}%
\pgfpathmoveto{\pgfqpoint{2.593241in}{4.835830in}}%
\pgfpathquadraticcurveto{\pgfqpoint{2.580826in}{4.837367in}}{\pgfqpoint{2.577473in}{4.837783in}}%
\pgfusepath{stroke}%
\end{pgfscope}%
\begin{pgfscope}%
\pgfpathrectangle{\pgfqpoint{1.250000in}{4.155455in}}{\pgfqpoint{2.279412in}{2.004545in}}%
\pgfusepath{clip}%
\pgfsetroundcap%
\pgfsetroundjoin%
\definecolor{currentfill}{rgb}{0.267968,0.223549,0.512008}%
\pgfsetfillcolor{currentfill}%
\pgfsetlinewidth{0.590181pt}%
\definecolor{currentstroke}{rgb}{0.267968,0.223549,0.512008}%
\pgfsetstrokecolor{currentstroke}%
\pgfsetdash{}{0pt}%
\pgfpathmoveto{\pgfqpoint{2.629193in}{4.803387in}}%
\pgfpathlineto{\pgfqpoint{2.577473in}{4.837783in}}%
\pgfpathlineto{\pgfqpoint{2.636021in}{4.858522in}}%
\pgfpathlineto{\pgfqpoint{2.629193in}{4.803387in}}%
\pgfpathlineto{\pgfqpoint{2.629193in}{4.803387in}}%
\pgfpathclose%
\pgfusepath{stroke,fill}%
\end{pgfscope}%
\begin{pgfscope}%
\pgfpathrectangle{\pgfqpoint{1.250000in}{4.155455in}}{\pgfqpoint{2.279412in}{2.004545in}}%
\pgfusepath{clip}%
\pgfsetroundcap%
\pgfsetroundjoin%
\pgfsetlinewidth{0.666141pt}%
\definecolor{currentstroke}{rgb}{0.250425,0.274290,0.533103}%
\pgfsetstrokecolor{currentstroke}%
\pgfsetdash{}{0pt}%
\pgfpathmoveto{\pgfqpoint{2.147872in}{5.316665in}}%
\pgfpathquadraticcurveto{\pgfqpoint{2.147989in}{5.310284in}}{\pgfqpoint{2.147917in}{5.314207in}}%
\pgfusepath{stroke}%
\end{pgfscope}%
\begin{pgfscope}%
\pgfpathrectangle{\pgfqpoint{1.250000in}{4.155455in}}{\pgfqpoint{2.279412in}{2.004545in}}%
\pgfusepath{clip}%
\pgfsetroundcap%
\pgfsetroundjoin%
\definecolor{currentfill}{rgb}{0.250425,0.274290,0.533103}%
\pgfsetfillcolor{currentfill}%
\pgfsetlinewidth{0.666141pt}%
\definecolor{currentstroke}{rgb}{0.250425,0.274290,0.533103}%
\pgfsetstrokecolor{currentstroke}%
\pgfsetdash{}{0pt}%
\pgfpathmoveto{\pgfqpoint{2.174673in}{5.370262in}}%
\pgfpathlineto{\pgfqpoint{2.147917in}{5.314207in}}%
\pgfpathlineto{\pgfqpoint{2.119127in}{5.369245in}}%
\pgfpathlineto{\pgfqpoint{2.174673in}{5.370262in}}%
\pgfpathlineto{\pgfqpoint{2.174673in}{5.370262in}}%
\pgfpathclose%
\pgfusepath{stroke,fill}%
\end{pgfscope}%
\begin{pgfscope}%
\pgfpathrectangle{\pgfqpoint{1.250000in}{4.155455in}}{\pgfqpoint{2.279412in}{2.004545in}}%
\pgfusepath{clip}%
\pgfsetroundcap%
\pgfsetroundjoin%
\pgfsetlinewidth{0.589586pt}%
\definecolor{currentstroke}{rgb}{0.267968,0.223549,0.512008}%
\pgfsetstrokecolor{currentstroke}%
\pgfsetdash{}{0pt}%
\pgfpathmoveto{\pgfqpoint{2.102782in}{4.968454in}}%
\pgfpathquadraticcurveto{\pgfqpoint{2.106765in}{4.972708in}}{\pgfqpoint{2.104514in}{4.970303in}}%
\pgfusepath{stroke}%
\end{pgfscope}%
\begin{pgfscope}%
\pgfpathrectangle{\pgfqpoint{1.250000in}{4.155455in}}{\pgfqpoint{2.279412in}{2.004545in}}%
\pgfusepath{clip}%
\pgfsetroundcap%
\pgfsetroundjoin%
\definecolor{currentfill}{rgb}{0.267968,0.223549,0.512008}%
\pgfsetfillcolor{currentfill}%
\pgfsetlinewidth{0.589586pt}%
\definecolor{currentstroke}{rgb}{0.267968,0.223549,0.512008}%
\pgfsetstrokecolor{currentstroke}%
\pgfsetdash{}{0pt}%
\pgfpathmoveto{\pgfqpoint{2.046266in}{4.948735in}}%
\pgfpathlineto{\pgfqpoint{2.104514in}{4.970303in}}%
\pgfpathlineto{\pgfqpoint{2.086820in}{4.910764in}}%
\pgfpathlineto{\pgfqpoint{2.046266in}{4.948735in}}%
\pgfpathlineto{\pgfqpoint{2.046266in}{4.948735in}}%
\pgfpathclose%
\pgfusepath{stroke,fill}%
\end{pgfscope}%
\begin{pgfscope}%
\pgfpathrectangle{\pgfqpoint{1.250000in}{4.155455in}}{\pgfqpoint{2.279412in}{2.004545in}}%
\pgfusepath{clip}%
\pgfsetroundcap%
\pgfsetroundjoin%
\pgfsetlinewidth{0.746721pt}%
\definecolor{currentstroke}{rgb}{0.229739,0.322361,0.545706}%
\pgfsetstrokecolor{currentstroke}%
\pgfsetdash{}{0pt}%
\pgfpathmoveto{\pgfqpoint{2.544982in}{5.378797in}}%
\pgfpathquadraticcurveto{\pgfqpoint{2.532524in}{5.377555in}}{\pgfqpoint{2.531561in}{5.377459in}}%
\pgfusepath{stroke}%
\end{pgfscope}%
\begin{pgfscope}%
\pgfpathrectangle{\pgfqpoint{1.250000in}{4.155455in}}{\pgfqpoint{2.279412in}{2.004545in}}%
\pgfusepath{clip}%
\pgfsetroundcap%
\pgfsetroundjoin%
\definecolor{currentfill}{rgb}{0.229739,0.322361,0.545706}%
\pgfsetfillcolor{currentfill}%
\pgfsetlinewidth{0.746721pt}%
\definecolor{currentstroke}{rgb}{0.229739,0.322361,0.545706}%
\pgfsetstrokecolor{currentstroke}%
\pgfsetdash{}{0pt}%
\pgfpathmoveto{\pgfqpoint{2.589598in}{5.355329in}}%
\pgfpathlineto{\pgfqpoint{2.531561in}{5.377459in}}%
\pgfpathlineto{\pgfqpoint{2.584087in}{5.410610in}}%
\pgfpathlineto{\pgfqpoint{2.589598in}{5.355329in}}%
\pgfpathlineto{\pgfqpoint{2.589598in}{5.355329in}}%
\pgfpathclose%
\pgfusepath{stroke,fill}%
\end{pgfscope}%
\begin{pgfscope}%
\pgfpathrectangle{\pgfqpoint{1.250000in}{4.155455in}}{\pgfqpoint{2.279412in}{2.004545in}}%
\pgfusepath{clip}%
\pgfsetbuttcap%
\pgfsetroundjoin%
\pgfsetlinewidth{1.505625pt}%
\definecolor{currentstroke}{rgb}{0.000000,0.000000,0.000000}%
\pgfsetstrokecolor{currentstroke}%
\pgfsetdash{}{0pt}%
\pgfpathmoveto{\pgfqpoint{2.043384in}{4.494895in}}%
\pgfpathlineto{\pgfqpoint{2.043384in}{5.820559in}}%
\pgfusepath{stroke}%
\end{pgfscope}%
\begin{pgfscope}%
\pgfpathrectangle{\pgfqpoint{1.250000in}{4.155455in}}{\pgfqpoint{2.279412in}{2.004545in}}%
\pgfusepath{clip}%
\pgfsetbuttcap%
\pgfsetroundjoin%
\pgfsetlinewidth{1.505625pt}%
\definecolor{currentstroke}{rgb}{0.000000,0.000000,0.000000}%
\pgfsetstrokecolor{currentstroke}%
\pgfsetdash{}{0pt}%
\pgfpathmoveto{\pgfqpoint{3.191910in}{4.494895in}}%
\pgfpathlineto{\pgfqpoint{3.191910in}{5.820559in}}%
\pgfusepath{stroke}%
\end{pgfscope}%
\begin{pgfscope}%
\pgfsetrectcap%
\pgfsetmiterjoin%
\pgfsetlinewidth{0.803000pt}%
\definecolor{currentstroke}{rgb}{0.000000,0.000000,0.000000}%
\pgfsetstrokecolor{currentstroke}%
\pgfsetdash{}{0pt}%
\pgfpathmoveto{\pgfqpoint{1.250000in}{4.155455in}}%
\pgfpathlineto{\pgfqpoint{1.250000in}{6.160000in}}%
\pgfusepath{stroke}%
\end{pgfscope}%
\begin{pgfscope}%
\pgfsetrectcap%
\pgfsetmiterjoin%
\pgfsetlinewidth{0.803000pt}%
\definecolor{currentstroke}{rgb}{0.000000,0.000000,0.000000}%
\pgfsetstrokecolor{currentstroke}%
\pgfsetdash{}{0pt}%
\pgfpathmoveto{\pgfqpoint{3.529412in}{4.155455in}}%
\pgfpathlineto{\pgfqpoint{3.529412in}{6.160000in}}%
\pgfusepath{stroke}%
\end{pgfscope}%
\begin{pgfscope}%
\pgfsetrectcap%
\pgfsetmiterjoin%
\pgfsetlinewidth{0.803000pt}%
\definecolor{currentstroke}{rgb}{0.000000,0.000000,0.000000}%
\pgfsetstrokecolor{currentstroke}%
\pgfsetdash{}{0pt}%
\pgfpathmoveto{\pgfqpoint{1.250000in}{4.155455in}}%
\pgfpathlineto{\pgfqpoint{3.529412in}{4.155455in}}%
\pgfusepath{stroke}%
\end{pgfscope}%
\begin{pgfscope}%
\pgfsetrectcap%
\pgfsetmiterjoin%
\pgfsetlinewidth{0.803000pt}%
\definecolor{currentstroke}{rgb}{0.000000,0.000000,0.000000}%
\pgfsetstrokecolor{currentstroke}%
\pgfsetdash{}{0pt}%
\pgfpathmoveto{\pgfqpoint{1.250000in}{6.160000in}}%
\pgfpathlineto{\pgfqpoint{3.529412in}{6.160000in}}%
\pgfusepath{stroke}%
\end{pgfscope}%
\begin{pgfscope}%
\definecolor{textcolor}{rgb}{0.000000,0.000000,0.000000}%
\pgfsetstrokecolor{textcolor}%
\pgfsetfillcolor{textcolor}%
\pgftext[x=2.389706in,y=6.243333in,,base]{\color{textcolor}\sffamily\fontsize{12.000000}{14.400000}\selectfont a)}%
\end{pgfscope}%
\begin{pgfscope}%
\pgfsetbuttcap%
\pgfsetmiterjoin%
\definecolor{currentfill}{rgb}{1.000000,1.000000,1.000000}%
\pgfsetfillcolor{currentfill}%
\pgfsetlinewidth{0.000000pt}%
\definecolor{currentstroke}{rgb}{0.000000,0.000000,0.000000}%
\pgfsetstrokecolor{currentstroke}%
\pgfsetstrokeopacity{0.000000}%
\pgfsetdash{}{0pt}%
\pgfpathmoveto{\pgfqpoint{3.985294in}{4.155455in}}%
\pgfpathlineto{\pgfqpoint{6.264706in}{4.155455in}}%
\pgfpathlineto{\pgfqpoint{6.264706in}{6.160000in}}%
\pgfpathlineto{\pgfqpoint{3.985294in}{6.160000in}}%
\pgfpathlineto{\pgfqpoint{3.985294in}{4.155455in}}%
\pgfpathclose%
\pgfusepath{fill}%
\end{pgfscope}%
\begin{pgfscope}%
\pgfpathrectangle{\pgfqpoint{3.985294in}{4.155455in}}{\pgfqpoint{2.279412in}{2.004545in}}%
\pgfusepath{clip}%
\pgfsys@transformcm{2.291667}{0.000000}{0.000000}{2.013889}{3.985294in}{4.155455in}%
\pgftext[left,bottom]{\includegraphics[interpolate=false,width=1.000000in,height=1.000000in]{q_series-img1.png}}%
\end{pgfscope}%
\begin{pgfscope}%
\pgfsetbuttcap%
\pgfsetroundjoin%
\definecolor{currentfill}{rgb}{0.000000,0.000000,0.000000}%
\pgfsetfillcolor{currentfill}%
\pgfsetlinewidth{0.803000pt}%
\definecolor{currentstroke}{rgb}{0.000000,0.000000,0.000000}%
\pgfsetstrokecolor{currentstroke}%
\pgfsetdash{}{0pt}%
\pgfsys@defobject{currentmarker}{\pgfqpoint{0.000000in}{-0.048611in}}{\pgfqpoint{0.000000in}{0.000000in}}{%
\pgfpathmoveto{\pgfqpoint{0.000000in}{0.000000in}}%
\pgfpathlineto{\pgfqpoint{0.000000in}{-0.048611in}}%
\pgfusepath{stroke,fill}%
}%
\begin{pgfscope}%
\pgfsys@transformshift{4.395836in}{4.155455in}%
\pgfsys@useobject{currentmarker}{}%
\end{pgfscope}%
\end{pgfscope}%
\begin{pgfscope}%
\pgfsetbuttcap%
\pgfsetroundjoin%
\definecolor{currentfill}{rgb}{0.000000,0.000000,0.000000}%
\pgfsetfillcolor{currentfill}%
\pgfsetlinewidth{0.803000pt}%
\definecolor{currentstroke}{rgb}{0.000000,0.000000,0.000000}%
\pgfsetstrokecolor{currentstroke}%
\pgfsetdash{}{0pt}%
\pgfsys@defobject{currentmarker}{\pgfqpoint{0.000000in}{-0.048611in}}{\pgfqpoint{0.000000in}{0.000000in}}{%
\pgfpathmoveto{\pgfqpoint{0.000000in}{0.000000in}}%
\pgfpathlineto{\pgfqpoint{0.000000in}{-0.048611in}}%
\pgfusepath{stroke,fill}%
}%
\begin{pgfscope}%
\pgfsys@transformshift{4.874388in}{4.155455in}%
\pgfsys@useobject{currentmarker}{}%
\end{pgfscope}%
\end{pgfscope}%
\begin{pgfscope}%
\pgfsetbuttcap%
\pgfsetroundjoin%
\definecolor{currentfill}{rgb}{0.000000,0.000000,0.000000}%
\pgfsetfillcolor{currentfill}%
\pgfsetlinewidth{0.803000pt}%
\definecolor{currentstroke}{rgb}{0.000000,0.000000,0.000000}%
\pgfsetstrokecolor{currentstroke}%
\pgfsetdash{}{0pt}%
\pgfsys@defobject{currentmarker}{\pgfqpoint{0.000000in}{-0.048611in}}{\pgfqpoint{0.000000in}{0.000000in}}{%
\pgfpathmoveto{\pgfqpoint{0.000000in}{0.000000in}}%
\pgfpathlineto{\pgfqpoint{0.000000in}{-0.048611in}}%
\pgfusepath{stroke,fill}%
}%
\begin{pgfscope}%
\pgfsys@transformshift{5.352941in}{4.155455in}%
\pgfsys@useobject{currentmarker}{}%
\end{pgfscope}%
\end{pgfscope}%
\begin{pgfscope}%
\pgfsetbuttcap%
\pgfsetroundjoin%
\definecolor{currentfill}{rgb}{0.000000,0.000000,0.000000}%
\pgfsetfillcolor{currentfill}%
\pgfsetlinewidth{0.803000pt}%
\definecolor{currentstroke}{rgb}{0.000000,0.000000,0.000000}%
\pgfsetstrokecolor{currentstroke}%
\pgfsetdash{}{0pt}%
\pgfsys@defobject{currentmarker}{\pgfqpoint{0.000000in}{-0.048611in}}{\pgfqpoint{0.000000in}{0.000000in}}{%
\pgfpathmoveto{\pgfqpoint{0.000000in}{0.000000in}}%
\pgfpathlineto{\pgfqpoint{0.000000in}{-0.048611in}}%
\pgfusepath{stroke,fill}%
}%
\begin{pgfscope}%
\pgfsys@transformshift{5.831494in}{4.155455in}%
\pgfsys@useobject{currentmarker}{}%
\end{pgfscope}%
\end{pgfscope}%
\begin{pgfscope}%
\definecolor{textcolor}{rgb}{0.000000,0.000000,0.000000}%
\pgfsetstrokecolor{textcolor}%
\pgfsetfillcolor{textcolor}%
\pgftext[x=5.125000in,y=4.099899in,,top]{\color{textcolor}\sffamily\fontsize{10.000000}{12.000000}\selectfont \(\displaystyle \zeta \, \mathrm{[\mu m]}\)}%
\end{pgfscope}%
\begin{pgfscope}%
\pgfsetbuttcap%
\pgfsetroundjoin%
\definecolor{currentfill}{rgb}{0.000000,0.000000,0.000000}%
\pgfsetfillcolor{currentfill}%
\pgfsetlinewidth{0.803000pt}%
\definecolor{currentstroke}{rgb}{0.000000,0.000000,0.000000}%
\pgfsetstrokecolor{currentstroke}%
\pgfsetdash{}{0pt}%
\pgfsys@defobject{currentmarker}{\pgfqpoint{-0.048611in}{0.000000in}}{\pgfqpoint{-0.000000in}{0.000000in}}{%
\pgfpathmoveto{\pgfqpoint{-0.000000in}{0.000000in}}%
\pgfpathlineto{\pgfqpoint{-0.048611in}{0.000000in}}%
\pgfusepath{stroke,fill}%
}%
\begin{pgfscope}%
\pgfsys@transformshift{3.985294in}{4.163479in}%
\pgfsys@useobject{currentmarker}{}%
\end{pgfscope}%
\end{pgfscope}%
\begin{pgfscope}%
\pgfsetbuttcap%
\pgfsetroundjoin%
\definecolor{currentfill}{rgb}{0.000000,0.000000,0.000000}%
\pgfsetfillcolor{currentfill}%
\pgfsetlinewidth{0.803000pt}%
\definecolor{currentstroke}{rgb}{0.000000,0.000000,0.000000}%
\pgfsetstrokecolor{currentstroke}%
\pgfsetdash{}{0pt}%
\pgfsys@defobject{currentmarker}{\pgfqpoint{-0.048611in}{0.000000in}}{\pgfqpoint{-0.000000in}{0.000000in}}{%
\pgfpathmoveto{\pgfqpoint{-0.000000in}{0.000000in}}%
\pgfpathlineto{\pgfqpoint{-0.048611in}{0.000000in}}%
\pgfusepath{stroke,fill}%
}%
\begin{pgfscope}%
\pgfsys@transformshift{3.985294in}{4.494895in}%
\pgfsys@useobject{currentmarker}{}%
\end{pgfscope}%
\end{pgfscope}%
\begin{pgfscope}%
\pgfsetbuttcap%
\pgfsetroundjoin%
\definecolor{currentfill}{rgb}{0.000000,0.000000,0.000000}%
\pgfsetfillcolor{currentfill}%
\pgfsetlinewidth{0.803000pt}%
\definecolor{currentstroke}{rgb}{0.000000,0.000000,0.000000}%
\pgfsetstrokecolor{currentstroke}%
\pgfsetdash{}{0pt}%
\pgfsys@defobject{currentmarker}{\pgfqpoint{-0.048611in}{0.000000in}}{\pgfqpoint{-0.000000in}{0.000000in}}{%
\pgfpathmoveto{\pgfqpoint{-0.000000in}{0.000000in}}%
\pgfpathlineto{\pgfqpoint{-0.048611in}{0.000000in}}%
\pgfusepath{stroke,fill}%
}%
\begin{pgfscope}%
\pgfsys@transformshift{3.985294in}{4.826311in}%
\pgfsys@useobject{currentmarker}{}%
\end{pgfscope}%
\end{pgfscope}%
\begin{pgfscope}%
\pgfsetbuttcap%
\pgfsetroundjoin%
\definecolor{currentfill}{rgb}{0.000000,0.000000,0.000000}%
\pgfsetfillcolor{currentfill}%
\pgfsetlinewidth{0.803000pt}%
\definecolor{currentstroke}{rgb}{0.000000,0.000000,0.000000}%
\pgfsetstrokecolor{currentstroke}%
\pgfsetdash{}{0pt}%
\pgfsys@defobject{currentmarker}{\pgfqpoint{-0.048611in}{0.000000in}}{\pgfqpoint{-0.000000in}{0.000000in}}{%
\pgfpathmoveto{\pgfqpoint{-0.000000in}{0.000000in}}%
\pgfpathlineto{\pgfqpoint{-0.048611in}{0.000000in}}%
\pgfusepath{stroke,fill}%
}%
\begin{pgfscope}%
\pgfsys@transformshift{3.985294in}{5.157727in}%
\pgfsys@useobject{currentmarker}{}%
\end{pgfscope}%
\end{pgfscope}%
\begin{pgfscope}%
\pgfsetbuttcap%
\pgfsetroundjoin%
\definecolor{currentfill}{rgb}{0.000000,0.000000,0.000000}%
\pgfsetfillcolor{currentfill}%
\pgfsetlinewidth{0.803000pt}%
\definecolor{currentstroke}{rgb}{0.000000,0.000000,0.000000}%
\pgfsetstrokecolor{currentstroke}%
\pgfsetdash{}{0pt}%
\pgfsys@defobject{currentmarker}{\pgfqpoint{-0.048611in}{0.000000in}}{\pgfqpoint{-0.000000in}{0.000000in}}{%
\pgfpathmoveto{\pgfqpoint{-0.000000in}{0.000000in}}%
\pgfpathlineto{\pgfqpoint{-0.048611in}{0.000000in}}%
\pgfusepath{stroke,fill}%
}%
\begin{pgfscope}%
\pgfsys@transformshift{3.985294in}{5.489143in}%
\pgfsys@useobject{currentmarker}{}%
\end{pgfscope}%
\end{pgfscope}%
\begin{pgfscope}%
\pgfsetbuttcap%
\pgfsetroundjoin%
\definecolor{currentfill}{rgb}{0.000000,0.000000,0.000000}%
\pgfsetfillcolor{currentfill}%
\pgfsetlinewidth{0.803000pt}%
\definecolor{currentstroke}{rgb}{0.000000,0.000000,0.000000}%
\pgfsetstrokecolor{currentstroke}%
\pgfsetdash{}{0pt}%
\pgfsys@defobject{currentmarker}{\pgfqpoint{-0.048611in}{0.000000in}}{\pgfqpoint{-0.000000in}{0.000000in}}{%
\pgfpathmoveto{\pgfqpoint{-0.000000in}{0.000000in}}%
\pgfpathlineto{\pgfqpoint{-0.048611in}{0.000000in}}%
\pgfusepath{stroke,fill}%
}%
\begin{pgfscope}%
\pgfsys@transformshift{3.985294in}{5.820559in}%
\pgfsys@useobject{currentmarker}{}%
\end{pgfscope}%
\end{pgfscope}%
\begin{pgfscope}%
\pgfsetbuttcap%
\pgfsetroundjoin%
\definecolor{currentfill}{rgb}{0.000000,0.000000,0.000000}%
\pgfsetfillcolor{currentfill}%
\pgfsetlinewidth{0.803000pt}%
\definecolor{currentstroke}{rgb}{0.000000,0.000000,0.000000}%
\pgfsetstrokecolor{currentstroke}%
\pgfsetdash{}{0pt}%
\pgfsys@defobject{currentmarker}{\pgfqpoint{-0.048611in}{0.000000in}}{\pgfqpoint{-0.000000in}{0.000000in}}{%
\pgfpathmoveto{\pgfqpoint{-0.000000in}{0.000000in}}%
\pgfpathlineto{\pgfqpoint{-0.048611in}{0.000000in}}%
\pgfusepath{stroke,fill}%
}%
\begin{pgfscope}%
\pgfsys@transformshift{3.985294in}{6.151975in}%
\pgfsys@useobject{currentmarker}{}%
\end{pgfscope}%
\end{pgfscope}%
\begin{pgfscope}%
\definecolor{textcolor}{rgb}{0.000000,0.000000,0.000000}%
\pgfsetstrokecolor{textcolor}%
\pgfsetfillcolor{textcolor}%
\pgftext[x=3.929739in,y=5.157727in,,bottom,rotate=90.000000]{\color{textcolor}\sffamily\fontsize{10.000000}{12.000000}\selectfont \(\displaystyle z \, \mathrm{[\mu m]}\)}%
\end{pgfscope}%
\begin{pgfscope}%
\pgfpathrectangle{\pgfqpoint{3.985294in}{4.155455in}}{\pgfqpoint{2.279412in}{2.004545in}}%
\pgfusepath{clip}%
\pgfsetbuttcap%
\pgfsetroundjoin%
\pgfsetlinewidth{0.000000pt}%
\definecolor{currentstroke}{rgb}{0.000000,0.000000,0.000000}%
\pgfsetstrokecolor{currentstroke}%
\pgfsetdash{}{0pt}%
\pgfpathmoveto{\pgfqpoint{6.028836in}{5.061185in}}%
\pgfpathlineto{\pgfqpoint{6.027320in}{5.060322in}}%
\pgfusepath{}%
\end{pgfscope}%
\begin{pgfscope}%
\pgfpathrectangle{\pgfqpoint{3.985294in}{4.155455in}}{\pgfqpoint{2.279412in}{2.004545in}}%
\pgfusepath{clip}%
\pgfsetbuttcap%
\pgfsetroundjoin%
\pgfsetlinewidth{0.305143pt}%
\definecolor{currentstroke}{rgb}{0.267004,0.004874,0.329415}%
\pgfsetstrokecolor{currentstroke}%
\pgfsetdash{}{0pt}%
\pgfpathmoveto{\pgfqpoint{6.027320in}{5.060322in}}%
\pgfpathlineto{\pgfqpoint{6.026713in}{5.059604in}}%
\pgfusepath{stroke}%
\end{pgfscope}%
\begin{pgfscope}%
\pgfpathrectangle{\pgfqpoint{3.985294in}{4.155455in}}{\pgfqpoint{2.279412in}{2.004545in}}%
\pgfusepath{clip}%
\pgfsetbuttcap%
\pgfsetroundjoin%
\pgfsetlinewidth{0.304854pt}%
\definecolor{currentstroke}{rgb}{0.267004,0.004874,0.329415}%
\pgfsetstrokecolor{currentstroke}%
\pgfsetdash{}{0pt}%
\pgfpathmoveto{\pgfqpoint{6.026713in}{5.059604in}}%
\pgfpathlineto{\pgfqpoint{6.026378in}{5.058832in}}%
\pgfusepath{stroke}%
\end{pgfscope}%
\begin{pgfscope}%
\pgfpathrectangle{\pgfqpoint{3.985294in}{4.155455in}}{\pgfqpoint{2.279412in}{2.004545in}}%
\pgfusepath{clip}%
\pgfsetbuttcap%
\pgfsetroundjoin%
\pgfsetlinewidth{0.304440pt}%
\definecolor{currentstroke}{rgb}{0.267004,0.004874,0.329415}%
\pgfsetstrokecolor{currentstroke}%
\pgfsetdash{}{0pt}%
\pgfpathmoveto{\pgfqpoint{6.026378in}{5.058832in}}%
\pgfpathlineto{\pgfqpoint{6.026006in}{5.057898in}}%
\pgfusepath{stroke}%
\end{pgfscope}%
\begin{pgfscope}%
\pgfpathrectangle{\pgfqpoint{3.985294in}{4.155455in}}{\pgfqpoint{2.279412in}{2.004545in}}%
\pgfusepath{clip}%
\pgfsetbuttcap%
\pgfsetroundjoin%
\pgfsetlinewidth{0.304459pt}%
\definecolor{currentstroke}{rgb}{0.267004,0.004874,0.329415}%
\pgfsetstrokecolor{currentstroke}%
\pgfsetdash{}{0pt}%
\pgfpathmoveto{\pgfqpoint{6.026006in}{5.057898in}}%
\pgfpathlineto{\pgfqpoint{6.025764in}{5.056788in}}%
\pgfusepath{stroke}%
\end{pgfscope}%
\begin{pgfscope}%
\pgfpathrectangle{\pgfqpoint{3.985294in}{4.155455in}}{\pgfqpoint{2.279412in}{2.004545in}}%
\pgfusepath{clip}%
\pgfsetbuttcap%
\pgfsetroundjoin%
\pgfsetlinewidth{0.304748pt}%
\definecolor{currentstroke}{rgb}{0.267004,0.004874,0.329415}%
\pgfsetstrokecolor{currentstroke}%
\pgfsetdash{}{0pt}%
\pgfpathmoveto{\pgfqpoint{6.025764in}{5.056788in}}%
\pgfpathlineto{\pgfqpoint{6.025517in}{5.055557in}}%
\pgfusepath{stroke}%
\end{pgfscope}%
\begin{pgfscope}%
\pgfpathrectangle{\pgfqpoint{3.985294in}{4.155455in}}{\pgfqpoint{2.279412in}{2.004545in}}%
\pgfusepath{clip}%
\pgfsetbuttcap%
\pgfsetroundjoin%
\pgfsetlinewidth{0.305057pt}%
\definecolor{currentstroke}{rgb}{0.267004,0.004874,0.329415}%
\pgfsetstrokecolor{currentstroke}%
\pgfsetdash{}{0pt}%
\pgfpathmoveto{\pgfqpoint{6.025517in}{5.055557in}}%
\pgfpathlineto{\pgfqpoint{6.025390in}{5.054233in}}%
\pgfusepath{stroke}%
\end{pgfscope}%
\begin{pgfscope}%
\pgfpathrectangle{\pgfqpoint{3.985294in}{4.155455in}}{\pgfqpoint{2.279412in}{2.004545in}}%
\pgfusepath{clip}%
\pgfsetbuttcap%
\pgfsetroundjoin%
\pgfsetlinewidth{0.305329pt}%
\definecolor{currentstroke}{rgb}{0.267004,0.004874,0.329415}%
\pgfsetstrokecolor{currentstroke}%
\pgfsetdash{}{0pt}%
\pgfpathmoveto{\pgfqpoint{6.025390in}{5.054233in}}%
\pgfpathlineto{\pgfqpoint{6.025365in}{5.052802in}}%
\pgfusepath{stroke}%
\end{pgfscope}%
\begin{pgfscope}%
\pgfpathrectangle{\pgfqpoint{3.985294in}{4.155455in}}{\pgfqpoint{2.279412in}{2.004545in}}%
\pgfusepath{clip}%
\pgfsetbuttcap%
\pgfsetroundjoin%
\pgfsetlinewidth{0.305578pt}%
\definecolor{currentstroke}{rgb}{0.267004,0.004874,0.329415}%
\pgfsetstrokecolor{currentstroke}%
\pgfsetdash{}{0pt}%
\pgfpathmoveto{\pgfqpoint{6.025365in}{5.052802in}}%
\pgfpathlineto{\pgfqpoint{6.025318in}{5.051218in}}%
\pgfusepath{stroke}%
\end{pgfscope}%
\begin{pgfscope}%
\pgfpathrectangle{\pgfqpoint{3.985294in}{4.155455in}}{\pgfqpoint{2.279412in}{2.004545in}}%
\pgfusepath{clip}%
\pgfsetbuttcap%
\pgfsetroundjoin%
\pgfsetlinewidth{0.305860pt}%
\definecolor{currentstroke}{rgb}{0.267004,0.004874,0.329415}%
\pgfsetstrokecolor{currentstroke}%
\pgfsetdash{}{0pt}%
\pgfpathmoveto{\pgfqpoint{6.025318in}{5.051218in}}%
\pgfpathlineto{\pgfqpoint{6.025129in}{5.049445in}}%
\pgfusepath{stroke}%
\end{pgfscope}%
\begin{pgfscope}%
\pgfpathrectangle{\pgfqpoint{3.985294in}{4.155455in}}{\pgfqpoint{2.279412in}{2.004545in}}%
\pgfusepath{clip}%
\pgfsetbuttcap%
\pgfsetroundjoin%
\pgfsetlinewidth{0.306224pt}%
\definecolor{currentstroke}{rgb}{0.267004,0.004874,0.329415}%
\pgfsetstrokecolor{currentstroke}%
\pgfsetdash{}{0pt}%
\pgfpathmoveto{\pgfqpoint{6.025129in}{5.049445in}}%
\pgfpathlineto{\pgfqpoint{6.024823in}{5.047492in}}%
\pgfusepath{stroke}%
\end{pgfscope}%
\begin{pgfscope}%
\pgfpathrectangle{\pgfqpoint{3.985294in}{4.155455in}}{\pgfqpoint{2.279412in}{2.004545in}}%
\pgfusepath{clip}%
\pgfsetbuttcap%
\pgfsetroundjoin%
\pgfsetlinewidth{0.306657pt}%
\definecolor{currentstroke}{rgb}{0.267004,0.004874,0.329415}%
\pgfsetstrokecolor{currentstroke}%
\pgfsetdash{}{0pt}%
\pgfpathmoveto{\pgfqpoint{6.024823in}{5.047492in}}%
\pgfpathlineto{\pgfqpoint{6.024823in}{5.047492in}}%
\pgfusepath{stroke}%
\end{pgfscope}%
\begin{pgfscope}%
\pgfpathrectangle{\pgfqpoint{3.985294in}{4.155455in}}{\pgfqpoint{2.279412in}{2.004545in}}%
\pgfusepath{clip}%
\pgfsetbuttcap%
\pgfsetroundjoin%
\pgfsetlinewidth{0.306657pt}%
\definecolor{currentstroke}{rgb}{0.267004,0.004874,0.329415}%
\pgfsetstrokecolor{currentstroke}%
\pgfsetdash{}{0pt}%
\pgfpathmoveto{\pgfqpoint{6.024823in}{5.047492in}}%
\pgfpathlineto{\pgfqpoint{6.023615in}{5.044909in}}%
\pgfusepath{stroke}%
\end{pgfscope}%
\begin{pgfscope}%
\pgfpathrectangle{\pgfqpoint{3.985294in}{4.155455in}}{\pgfqpoint{2.279412in}{2.004545in}}%
\pgfusepath{clip}%
\pgfsetbuttcap%
\pgfsetroundjoin%
\pgfsetlinewidth{0.307496pt}%
\definecolor{currentstroke}{rgb}{0.267004,0.004874,0.329415}%
\pgfsetstrokecolor{currentstroke}%
\pgfsetdash{}{0pt}%
\pgfpathmoveto{\pgfqpoint{6.023615in}{5.044909in}}%
\pgfpathlineto{\pgfqpoint{6.023526in}{5.042806in}}%
\pgfusepath{stroke}%
\end{pgfscope}%
\begin{pgfscope}%
\pgfpathrectangle{\pgfqpoint{3.985294in}{4.155455in}}{\pgfqpoint{2.279412in}{2.004545in}}%
\pgfusepath{clip}%
\pgfsetbuttcap%
\pgfsetroundjoin%
\pgfsetlinewidth{0.307855pt}%
\definecolor{currentstroke}{rgb}{0.267004,0.004874,0.329415}%
\pgfsetstrokecolor{currentstroke}%
\pgfsetdash{}{0pt}%
\pgfpathmoveto{\pgfqpoint{6.023526in}{5.042806in}}%
\pgfpathlineto{\pgfqpoint{6.024314in}{5.041070in}}%
\pgfusepath{stroke}%
\end{pgfscope}%
\begin{pgfscope}%
\pgfpathrectangle{\pgfqpoint{3.985294in}{4.155455in}}{\pgfqpoint{2.279412in}{2.004545in}}%
\pgfusepath{clip}%
\pgfsetbuttcap%
\pgfsetroundjoin%
\pgfsetlinewidth{0.307872pt}%
\definecolor{currentstroke}{rgb}{0.267004,0.004874,0.329415}%
\pgfsetstrokecolor{currentstroke}%
\pgfsetdash{}{0pt}%
\pgfpathmoveto{\pgfqpoint{6.024314in}{5.041070in}}%
\pgfpathlineto{\pgfqpoint{6.025719in}{5.039112in}}%
\pgfusepath{stroke}%
\end{pgfscope}%
\begin{pgfscope}%
\pgfpathrectangle{\pgfqpoint{3.985294in}{4.155455in}}{\pgfqpoint{2.279412in}{2.004545in}}%
\pgfusepath{clip}%
\pgfsetbuttcap%
\pgfsetroundjoin%
\pgfsetlinewidth{0.307750pt}%
\definecolor{currentstroke}{rgb}{0.267004,0.004874,0.329415}%
\pgfsetstrokecolor{currentstroke}%
\pgfsetdash{}{0pt}%
\pgfpathmoveto{\pgfqpoint{6.025719in}{5.039112in}}%
\pgfpathlineto{\pgfqpoint{6.025719in}{5.039112in}}%
\pgfusepath{stroke}%
\end{pgfscope}%
\begin{pgfscope}%
\pgfpathrectangle{\pgfqpoint{3.985294in}{4.155455in}}{\pgfqpoint{2.279412in}{2.004545in}}%
\pgfusepath{clip}%
\pgfsetbuttcap%
\pgfsetroundjoin%
\pgfsetlinewidth{0.307750pt}%
\definecolor{currentstroke}{rgb}{0.267004,0.004874,0.329415}%
\pgfsetstrokecolor{currentstroke}%
\pgfsetdash{}{0pt}%
\pgfpathmoveto{\pgfqpoint{6.025719in}{5.039112in}}%
\pgfpathlineto{\pgfqpoint{6.028397in}{5.036421in}}%
\pgfusepath{stroke}%
\end{pgfscope}%
\begin{pgfscope}%
\pgfpathrectangle{\pgfqpoint{3.985294in}{4.155455in}}{\pgfqpoint{2.279412in}{2.004545in}}%
\pgfusepath{clip}%
\pgfsetbuttcap%
\pgfsetroundjoin%
\pgfsetlinewidth{0.307912pt}%
\definecolor{currentstroke}{rgb}{0.267004,0.004874,0.329415}%
\pgfsetstrokecolor{currentstroke}%
\pgfsetdash{}{0pt}%
\pgfpathmoveto{\pgfqpoint{6.028397in}{5.036421in}}%
\pgfpathlineto{\pgfqpoint{6.028397in}{5.036421in}}%
\pgfusepath{stroke}%
\end{pgfscope}%
\begin{pgfscope}%
\pgfpathrectangle{\pgfqpoint{3.985294in}{4.155455in}}{\pgfqpoint{2.279412in}{2.004545in}}%
\pgfusepath{clip}%
\pgfsetbuttcap%
\pgfsetroundjoin%
\pgfsetlinewidth{0.307912pt}%
\definecolor{currentstroke}{rgb}{0.267004,0.004874,0.329415}%
\pgfsetstrokecolor{currentstroke}%
\pgfsetdash{}{0pt}%
\pgfpathmoveto{\pgfqpoint{6.028397in}{5.036421in}}%
\pgfpathlineto{\pgfqpoint{6.035927in}{5.032747in}}%
\pgfusepath{stroke}%
\end{pgfscope}%
\begin{pgfscope}%
\pgfpathrectangle{\pgfqpoint{3.985294in}{4.155455in}}{\pgfqpoint{2.279412in}{2.004545in}}%
\pgfusepath{clip}%
\pgfsetbuttcap%
\pgfsetroundjoin%
\pgfsetlinewidth{0.307884pt}%
\definecolor{currentstroke}{rgb}{0.267004,0.004874,0.329415}%
\pgfsetstrokecolor{currentstroke}%
\pgfsetdash{}{0pt}%
\pgfpathmoveto{\pgfqpoint{6.035927in}{5.032747in}}%
\pgfpathlineto{\pgfqpoint{6.035927in}{5.032747in}}%
\pgfusepath{stroke}%
\end{pgfscope}%
\begin{pgfscope}%
\pgfpathrectangle{\pgfqpoint{3.985294in}{4.155455in}}{\pgfqpoint{2.279412in}{2.004545in}}%
\pgfusepath{clip}%
\pgfsetbuttcap%
\pgfsetroundjoin%
\pgfsetlinewidth{0.307884pt}%
\definecolor{currentstroke}{rgb}{0.267004,0.004874,0.329415}%
\pgfsetstrokecolor{currentstroke}%
\pgfsetdash{}{0pt}%
\pgfpathmoveto{\pgfqpoint{6.035927in}{5.032747in}}%
\pgfpathlineto{\pgfqpoint{6.040657in}{5.029197in}}%
\pgfusepath{stroke}%
\end{pgfscope}%
\begin{pgfscope}%
\pgfpathrectangle{\pgfqpoint{3.985294in}{4.155455in}}{\pgfqpoint{2.279412in}{2.004545in}}%
\pgfusepath{clip}%
\pgfsetbuttcap%
\pgfsetroundjoin%
\pgfsetlinewidth{0.307931pt}%
\definecolor{currentstroke}{rgb}{0.267004,0.004874,0.329415}%
\pgfsetstrokecolor{currentstroke}%
\pgfsetdash{}{0pt}%
\pgfpathmoveto{\pgfqpoint{6.040657in}{5.029197in}}%
\pgfpathlineto{\pgfqpoint{6.042409in}{5.026700in}}%
\pgfusepath{stroke}%
\end{pgfscope}%
\begin{pgfscope}%
\pgfpathrectangle{\pgfqpoint{3.985294in}{4.155455in}}{\pgfqpoint{2.279412in}{2.004545in}}%
\pgfusepath{clip}%
\pgfsetbuttcap%
\pgfsetroundjoin%
\pgfsetlinewidth{0.308150pt}%
\definecolor{currentstroke}{rgb}{0.268510,0.009605,0.335427}%
\pgfsetstrokecolor{currentstroke}%
\pgfsetdash{}{0pt}%
\pgfpathmoveto{\pgfqpoint{6.042409in}{5.026700in}}%
\pgfpathlineto{\pgfqpoint{6.043603in}{5.025079in}}%
\pgfusepath{stroke}%
\end{pgfscope}%
\begin{pgfscope}%
\pgfpathrectangle{\pgfqpoint{3.985294in}{4.155455in}}{\pgfqpoint{2.279412in}{2.004545in}}%
\pgfusepath{clip}%
\pgfsetbuttcap%
\pgfsetroundjoin%
\pgfsetlinewidth{0.308238pt}%
\definecolor{currentstroke}{rgb}{0.268510,0.009605,0.335427}%
\pgfsetstrokecolor{currentstroke}%
\pgfsetdash{}{0pt}%
\pgfpathmoveto{\pgfqpoint{6.043603in}{5.025079in}}%
\pgfpathlineto{\pgfqpoint{6.045274in}{5.023924in}}%
\pgfusepath{stroke}%
\end{pgfscope}%
\begin{pgfscope}%
\pgfpathrectangle{\pgfqpoint{3.985294in}{4.155455in}}{\pgfqpoint{2.279412in}{2.004545in}}%
\pgfusepath{clip}%
\pgfsetbuttcap%
\pgfsetroundjoin%
\pgfsetlinewidth{0.308074pt}%
\definecolor{currentstroke}{rgb}{0.268510,0.009605,0.335427}%
\pgfsetstrokecolor{currentstroke}%
\pgfsetdash{}{0pt}%
\pgfpathmoveto{\pgfqpoint{6.045274in}{5.023924in}}%
\pgfpathlineto{\pgfqpoint{6.048254in}{5.022407in}}%
\pgfusepath{stroke}%
\end{pgfscope}%
\begin{pgfscope}%
\pgfpathrectangle{\pgfqpoint{3.985294in}{4.155455in}}{\pgfqpoint{2.279412in}{2.004545in}}%
\pgfusepath{clip}%
\pgfsetbuttcap%
\pgfsetroundjoin%
\pgfsetlinewidth{0.307598pt}%
\definecolor{currentstroke}{rgb}{0.267004,0.004874,0.329415}%
\pgfsetstrokecolor{currentstroke}%
\pgfsetdash{}{0pt}%
\pgfpathmoveto{\pgfqpoint{6.048254in}{5.022407in}}%
\pgfpathlineto{\pgfqpoint{6.048254in}{5.022407in}}%
\pgfusepath{stroke}%
\end{pgfscope}%
\begin{pgfscope}%
\pgfpathrectangle{\pgfqpoint{3.985294in}{4.155455in}}{\pgfqpoint{2.279412in}{2.004545in}}%
\pgfusepath{clip}%
\pgfsetbuttcap%
\pgfsetroundjoin%
\pgfsetlinewidth{0.307598pt}%
\definecolor{currentstroke}{rgb}{0.267004,0.004874,0.329415}%
\pgfsetstrokecolor{currentstroke}%
\pgfsetdash{}{0pt}%
\pgfpathmoveto{\pgfqpoint{6.048254in}{5.022407in}}%
\pgfpathlineto{\pgfqpoint{6.048254in}{5.022407in}}%
\pgfusepath{stroke}%
\end{pgfscope}%
\begin{pgfscope}%
\pgfpathrectangle{\pgfqpoint{3.985294in}{4.155455in}}{\pgfqpoint{2.279412in}{2.004545in}}%
\pgfusepath{clip}%
\pgfsetbuttcap%
\pgfsetroundjoin%
\pgfsetlinewidth{0.308931pt}%
\definecolor{currentstroke}{rgb}{0.268510,0.009605,0.335427}%
\pgfsetstrokecolor{currentstroke}%
\pgfsetdash{}{0pt}%
\pgfpathmoveto{\pgfqpoint{6.014399in}{5.022290in}}%
\pgfpathlineto{\pgfqpoint{5.996962in}{5.022407in}}%
\pgfusepath{stroke}%
\end{pgfscope}%
\begin{pgfscope}%
\pgfpathrectangle{\pgfqpoint{3.985294in}{4.155455in}}{\pgfqpoint{2.279412in}{2.004545in}}%
\pgfusepath{clip}%
\pgfsetbuttcap%
\pgfsetroundjoin%
\pgfsetlinewidth{0.310171pt}%
\definecolor{currentstroke}{rgb}{0.268510,0.009605,0.335427}%
\pgfsetstrokecolor{currentstroke}%
\pgfsetdash{}{0pt}%
\pgfpathmoveto{\pgfqpoint{5.996962in}{5.022407in}}%
\pgfpathlineto{\pgfqpoint{5.996962in}{5.022407in}}%
\pgfusepath{stroke}%
\end{pgfscope}%
\begin{pgfscope}%
\pgfpathrectangle{\pgfqpoint{3.985294in}{4.155455in}}{\pgfqpoint{2.279412in}{2.004545in}}%
\pgfusepath{clip}%
\pgfsetbuttcap%
\pgfsetroundjoin%
\pgfsetlinewidth{0.310171pt}%
\definecolor{currentstroke}{rgb}{0.268510,0.009605,0.335427}%
\pgfsetstrokecolor{currentstroke}%
\pgfsetdash{}{0pt}%
\pgfpathmoveto{\pgfqpoint{5.996962in}{5.022407in}}%
\pgfpathlineto{\pgfqpoint{5.946864in}{5.022777in}}%
\pgfusepath{stroke}%
\end{pgfscope}%
\begin{pgfscope}%
\pgfpathrectangle{\pgfqpoint{3.985294in}{4.155455in}}{\pgfqpoint{2.279412in}{2.004545in}}%
\pgfusepath{clip}%
\pgfsetbuttcap%
\pgfsetroundjoin%
\pgfsetlinewidth{0.320468pt}%
\definecolor{currentstroke}{rgb}{0.269944,0.014625,0.341379}%
\pgfsetstrokecolor{currentstroke}%
\pgfsetdash{}{0pt}%
\pgfpathmoveto{\pgfqpoint{5.946864in}{5.022777in}}%
\pgfpathlineto{\pgfqpoint{5.896745in}{5.023449in}}%
\pgfusepath{stroke}%
\end{pgfscope}%
\begin{pgfscope}%
\pgfpathrectangle{\pgfqpoint{3.985294in}{4.155455in}}{\pgfqpoint{2.279412in}{2.004545in}}%
\pgfusepath{clip}%
\pgfsetbuttcap%
\pgfsetroundjoin%
\pgfsetlinewidth{0.322000pt}%
\definecolor{currentstroke}{rgb}{0.271305,0.019942,0.347269}%
\pgfsetstrokecolor{currentstroke}%
\pgfsetdash{}{0pt}%
\pgfpathmoveto{\pgfqpoint{5.896745in}{5.023449in}}%
\pgfpathlineto{\pgfqpoint{5.846630in}{5.024493in}}%
\pgfusepath{stroke}%
\end{pgfscope}%
\begin{pgfscope}%
\pgfpathrectangle{\pgfqpoint{3.985294in}{4.155455in}}{\pgfqpoint{2.279412in}{2.004545in}}%
\pgfusepath{clip}%
\pgfsetbuttcap%
\pgfsetroundjoin%
\pgfsetlinewidth{0.342656pt}%
\definecolor{currentstroke}{rgb}{0.274952,0.037752,0.364543}%
\pgfsetstrokecolor{currentstroke}%
\pgfsetdash{}{0pt}%
\pgfpathmoveto{\pgfqpoint{5.846630in}{5.024493in}}%
\pgfpathlineto{\pgfqpoint{5.796518in}{5.026111in}}%
\pgfusepath{stroke}%
\end{pgfscope}%
\begin{pgfscope}%
\pgfpathrectangle{\pgfqpoint{3.985294in}{4.155455in}}{\pgfqpoint{2.279412in}{2.004545in}}%
\pgfusepath{clip}%
\pgfsetbuttcap%
\pgfsetroundjoin%
\pgfsetlinewidth{0.350010pt}%
\definecolor{currentstroke}{rgb}{0.276022,0.044167,0.370164}%
\pgfsetstrokecolor{currentstroke}%
\pgfsetdash{}{0pt}%
\pgfpathmoveto{\pgfqpoint{5.796518in}{5.026111in}}%
\pgfpathlineto{\pgfqpoint{5.746420in}{5.027875in}}%
\pgfusepath{stroke}%
\end{pgfscope}%
\begin{pgfscope}%
\pgfpathrectangle{\pgfqpoint{3.985294in}{4.155455in}}{\pgfqpoint{2.279412in}{2.004545in}}%
\pgfusepath{clip}%
\pgfsetbuttcap%
\pgfsetroundjoin%
\pgfsetlinewidth{0.371405pt}%
\definecolor{currentstroke}{rgb}{0.278791,0.062145,0.386592}%
\pgfsetstrokecolor{currentstroke}%
\pgfsetdash{}{0pt}%
\pgfpathmoveto{\pgfqpoint{5.746420in}{5.027875in}}%
\pgfpathlineto{\pgfqpoint{5.696308in}{5.029401in}}%
\pgfusepath{stroke}%
\end{pgfscope}%
\begin{pgfscope}%
\pgfpathrectangle{\pgfqpoint{3.985294in}{4.155455in}}{\pgfqpoint{2.279412in}{2.004545in}}%
\pgfusepath{clip}%
\pgfsetbuttcap%
\pgfsetroundjoin%
\pgfsetlinewidth{0.395365pt}%
\definecolor{currentstroke}{rgb}{0.280894,0.078907,0.402329}%
\pgfsetstrokecolor{currentstroke}%
\pgfsetdash{}{0pt}%
\pgfpathmoveto{\pgfqpoint{5.696308in}{5.029401in}}%
\pgfpathlineto{\pgfqpoint{5.646176in}{5.030608in}}%
\pgfusepath{stroke}%
\end{pgfscope}%
\begin{pgfscope}%
\pgfpathrectangle{\pgfqpoint{3.985294in}{4.155455in}}{\pgfqpoint{2.279412in}{2.004545in}}%
\pgfusepath{clip}%
\pgfsetbuttcap%
\pgfsetroundjoin%
\pgfsetlinewidth{0.443340pt}%
\definecolor{currentstroke}{rgb}{0.283197,0.115680,0.436115}%
\pgfsetstrokecolor{currentstroke}%
\pgfsetdash{}{0pt}%
\pgfpathmoveto{\pgfqpoint{5.646176in}{5.030608in}}%
\pgfpathlineto{\pgfqpoint{5.596043in}{5.031841in}}%
\pgfusepath{stroke}%
\end{pgfscope}%
\begin{pgfscope}%
\pgfpathrectangle{\pgfqpoint{3.985294in}{4.155455in}}{\pgfqpoint{2.279412in}{2.004545in}}%
\pgfusepath{clip}%
\pgfsetbuttcap%
\pgfsetroundjoin%
\pgfsetlinewidth{0.495012pt}%
\definecolor{currentstroke}{rgb}{0.281412,0.155834,0.469201}%
\pgfsetstrokecolor{currentstroke}%
\pgfsetdash{}{0pt}%
\pgfpathmoveto{\pgfqpoint{5.596043in}{5.031841in}}%
\pgfpathlineto{\pgfqpoint{5.545915in}{5.033165in}}%
\pgfusepath{stroke}%
\end{pgfscope}%
\begin{pgfscope}%
\pgfpathrectangle{\pgfqpoint{3.985294in}{4.155455in}}{\pgfqpoint{2.279412in}{2.004545in}}%
\pgfusepath{clip}%
\pgfsetbuttcap%
\pgfsetroundjoin%
\pgfsetlinewidth{0.545425pt}%
\definecolor{currentstroke}{rgb}{0.276194,0.190074,0.493001}%
\pgfsetstrokecolor{currentstroke}%
\pgfsetdash{}{0pt}%
\pgfpathmoveto{\pgfqpoint{5.545915in}{5.033165in}}%
\pgfpathlineto{\pgfqpoint{5.495793in}{5.034657in}}%
\pgfusepath{stroke}%
\end{pgfscope}%
\begin{pgfscope}%
\pgfpathrectangle{\pgfqpoint{3.985294in}{4.155455in}}{\pgfqpoint{2.279412in}{2.004545in}}%
\pgfusepath{clip}%
\pgfsetbuttcap%
\pgfsetroundjoin%
\pgfsetlinewidth{0.631651pt}%
\definecolor{currentstroke}{rgb}{0.258965,0.251537,0.524736}%
\pgfsetstrokecolor{currentstroke}%
\pgfsetdash{}{0pt}%
\pgfpathmoveto{\pgfqpoint{5.495793in}{5.034657in}}%
\pgfpathlineto{\pgfqpoint{5.445683in}{5.036451in}}%
\pgfusepath{stroke}%
\end{pgfscope}%
\begin{pgfscope}%
\pgfpathrectangle{\pgfqpoint{3.985294in}{4.155455in}}{\pgfqpoint{2.279412in}{2.004545in}}%
\pgfusepath{clip}%
\pgfsetbuttcap%
\pgfsetroundjoin%
\pgfsetlinewidth{0.710791pt}%
\definecolor{currentstroke}{rgb}{0.239346,0.300855,0.540844}%
\pgfsetstrokecolor{currentstroke}%
\pgfsetdash{}{0pt}%
\pgfpathmoveto{\pgfqpoint{5.445683in}{5.036451in}}%
\pgfpathlineto{\pgfqpoint{5.395588in}{5.038548in}}%
\pgfusepath{stroke}%
\end{pgfscope}%
\begin{pgfscope}%
\pgfpathrectangle{\pgfqpoint{3.985294in}{4.155455in}}{\pgfqpoint{2.279412in}{2.004545in}}%
\pgfusepath{clip}%
\pgfsetbuttcap%
\pgfsetroundjoin%
\pgfsetlinewidth{0.796738pt}%
\definecolor{currentstroke}{rgb}{0.214298,0.355619,0.551184}%
\pgfsetstrokecolor{currentstroke}%
\pgfsetdash{}{0pt}%
\pgfpathmoveto{\pgfqpoint{5.395588in}{5.038548in}}%
\pgfpathlineto{\pgfqpoint{5.345506in}{5.040865in}}%
\pgfusepath{stroke}%
\end{pgfscope}%
\begin{pgfscope}%
\pgfpathrectangle{\pgfqpoint{3.985294in}{4.155455in}}{\pgfqpoint{2.279412in}{2.004545in}}%
\pgfusepath{clip}%
\pgfsetbuttcap%
\pgfsetroundjoin%
\pgfsetlinewidth{0.911975pt}%
\definecolor{currentstroke}{rgb}{0.183898,0.422383,0.556944}%
\pgfsetstrokecolor{currentstroke}%
\pgfsetdash{}{0pt}%
\pgfpathmoveto{\pgfqpoint{5.345506in}{5.040865in}}%
\pgfpathlineto{\pgfqpoint{5.295448in}{5.043548in}}%
\pgfusepath{stroke}%
\end{pgfscope}%
\begin{pgfscope}%
\pgfpathrectangle{\pgfqpoint{3.985294in}{4.155455in}}{\pgfqpoint{2.279412in}{2.004545in}}%
\pgfusepath{clip}%
\pgfsetbuttcap%
\pgfsetroundjoin%
\pgfsetlinewidth{1.023449pt}%
\definecolor{currentstroke}{rgb}{0.159194,0.482237,0.558073}%
\pgfsetstrokecolor{currentstroke}%
\pgfsetdash{}{0pt}%
\pgfpathmoveto{\pgfqpoint{5.295448in}{5.043548in}}%
\pgfpathlineto{\pgfqpoint{5.245429in}{5.046726in}}%
\pgfusepath{stroke}%
\end{pgfscope}%
\begin{pgfscope}%
\pgfpathrectangle{\pgfqpoint{3.985294in}{4.155455in}}{\pgfqpoint{2.279412in}{2.004545in}}%
\pgfusepath{clip}%
\pgfsetbuttcap%
\pgfsetroundjoin%
\pgfsetlinewidth{1.070889pt}%
\definecolor{currentstroke}{rgb}{0.149039,0.508051,0.557250}%
\pgfsetstrokecolor{currentstroke}%
\pgfsetdash{}{0pt}%
\pgfpathmoveto{\pgfqpoint{5.245429in}{5.046726in}}%
\pgfpathlineto{\pgfqpoint{5.195455in}{5.050389in}}%
\pgfusepath{stroke}%
\end{pgfscope}%
\begin{pgfscope}%
\pgfpathrectangle{\pgfqpoint{3.985294in}{4.155455in}}{\pgfqpoint{2.279412in}{2.004545in}}%
\pgfusepath{clip}%
\pgfsetbuttcap%
\pgfsetroundjoin%
\pgfsetlinewidth{1.146455pt}%
\definecolor{currentstroke}{rgb}{0.133743,0.548535,0.553541}%
\pgfsetstrokecolor{currentstroke}%
\pgfsetdash{}{0pt}%
\pgfpathmoveto{\pgfqpoint{5.195455in}{5.050389in}}%
\pgfpathlineto{\pgfqpoint{5.145544in}{5.054646in}}%
\pgfusepath{stroke}%
\end{pgfscope}%
\begin{pgfscope}%
\pgfpathrectangle{\pgfqpoint{3.985294in}{4.155455in}}{\pgfqpoint{2.279412in}{2.004545in}}%
\pgfusepath{clip}%
\pgfsetbuttcap%
\pgfsetroundjoin%
\pgfsetlinewidth{1.222406pt}%
\definecolor{currentstroke}{rgb}{0.121831,0.589055,0.545623}%
\pgfsetstrokecolor{currentstroke}%
\pgfsetdash{}{0pt}%
\pgfpathmoveto{\pgfqpoint{5.145544in}{5.054646in}}%
\pgfpathlineto{\pgfqpoint{5.095722in}{5.059660in}}%
\pgfusepath{stroke}%
\end{pgfscope}%
\begin{pgfscope}%
\pgfpathrectangle{\pgfqpoint{3.985294in}{4.155455in}}{\pgfqpoint{2.279412in}{2.004545in}}%
\pgfusepath{clip}%
\pgfsetbuttcap%
\pgfsetroundjoin%
\pgfsetlinewidth{1.176525pt}%
\definecolor{currentstroke}{rgb}{0.128729,0.563265,0.551229}%
\pgfsetstrokecolor{currentstroke}%
\pgfsetdash{}{0pt}%
\pgfpathmoveto{\pgfqpoint{5.095722in}{5.059660in}}%
\pgfpathlineto{\pgfqpoint{5.046040in}{5.065624in}}%
\pgfusepath{stroke}%
\end{pgfscope}%
\begin{pgfscope}%
\pgfpathrectangle{\pgfqpoint{3.985294in}{4.155455in}}{\pgfqpoint{2.279412in}{2.004545in}}%
\pgfusepath{clip}%
\pgfsetbuttcap%
\pgfsetroundjoin%
\pgfsetlinewidth{1.284555pt}%
\definecolor{currentstroke}{rgb}{0.120081,0.622161,0.534946}%
\pgfsetstrokecolor{currentstroke}%
\pgfsetdash{}{0pt}%
\pgfpathmoveto{\pgfqpoint{5.046040in}{5.065624in}}%
\pgfpathlineto{\pgfqpoint{4.996559in}{5.072723in}}%
\pgfusepath{stroke}%
\end{pgfscope}%
\begin{pgfscope}%
\pgfpathrectangle{\pgfqpoint{3.985294in}{4.155455in}}{\pgfqpoint{2.279412in}{2.004545in}}%
\pgfusepath{clip}%
\pgfsetbuttcap%
\pgfsetroundjoin%
\pgfsetlinewidth{1.255930pt}%
\definecolor{currentstroke}{rgb}{0.119512,0.607464,0.540218}%
\pgfsetstrokecolor{currentstroke}%
\pgfsetdash{}{0pt}%
\pgfpathmoveto{\pgfqpoint{4.996559in}{5.072723in}}%
\pgfpathlineto{\pgfqpoint{4.947372in}{5.081230in}}%
\pgfusepath{stroke}%
\end{pgfscope}%
\begin{pgfscope}%
\pgfpathrectangle{\pgfqpoint{3.985294in}{4.155455in}}{\pgfqpoint{2.279412in}{2.004545in}}%
\pgfusepath{clip}%
\pgfsetbuttcap%
\pgfsetroundjoin%
\pgfsetlinewidth{1.313789pt}%
\definecolor{currentstroke}{rgb}{0.123444,0.636809,0.528763}%
\pgfsetstrokecolor{currentstroke}%
\pgfsetdash{}{0pt}%
\pgfpathmoveto{\pgfqpoint{4.947372in}{5.081230in}}%
\pgfpathlineto{\pgfqpoint{4.898560in}{5.091211in}}%
\pgfusepath{stroke}%
\end{pgfscope}%
\begin{pgfscope}%
\pgfpathrectangle{\pgfqpoint{3.985294in}{4.155455in}}{\pgfqpoint{2.279412in}{2.004545in}}%
\pgfusepath{clip}%
\pgfsetbuttcap%
\pgfsetroundjoin%
\pgfsetlinewidth{1.353322pt}%
\definecolor{currentstroke}{rgb}{0.134692,0.658636,0.517649}%
\pgfsetstrokecolor{currentstroke}%
\pgfsetdash{}{0pt}%
\pgfpathmoveto{\pgfqpoint{4.898560in}{5.091211in}}%
\pgfpathlineto{\pgfqpoint{4.850240in}{5.102772in}}%
\pgfusepath{stroke}%
\end{pgfscope}%
\begin{pgfscope}%
\pgfpathrectangle{\pgfqpoint{3.985294in}{4.155455in}}{\pgfqpoint{2.279412in}{2.004545in}}%
\pgfusepath{clip}%
\pgfsetbuttcap%
\pgfsetroundjoin%
\pgfsetlinewidth{1.177334pt}%
\definecolor{currentstroke}{rgb}{0.128729,0.563265,0.551229}%
\pgfsetstrokecolor{currentstroke}%
\pgfsetdash{}{0pt}%
\pgfpathmoveto{\pgfqpoint{4.850240in}{5.102772in}}%
\pgfpathlineto{\pgfqpoint{4.802113in}{5.114742in}}%
\pgfusepath{stroke}%
\end{pgfscope}%
\begin{pgfscope}%
\pgfpathrectangle{\pgfqpoint{3.985294in}{4.155455in}}{\pgfqpoint{2.279412in}{2.004545in}}%
\pgfusepath{clip}%
\pgfsetbuttcap%
\pgfsetroundjoin%
\pgfsetlinewidth{1.053311pt}%
\definecolor{currentstroke}{rgb}{0.153364,0.497000,0.557724}%
\pgfsetstrokecolor{currentstroke}%
\pgfsetdash{}{0pt}%
\pgfpathmoveto{\pgfqpoint{4.802113in}{5.114742in}}%
\pgfpathlineto{\pgfqpoint{4.754363in}{5.127107in}}%
\pgfusepath{stroke}%
\end{pgfscope}%
\begin{pgfscope}%
\pgfpathrectangle{\pgfqpoint{3.985294in}{4.155455in}}{\pgfqpoint{2.279412in}{2.004545in}}%
\pgfusepath{clip}%
\pgfsetbuttcap%
\pgfsetroundjoin%
\pgfsetlinewidth{0.892051pt}%
\definecolor{currentstroke}{rgb}{0.188923,0.410910,0.556326}%
\pgfsetstrokecolor{currentstroke}%
\pgfsetdash{}{0pt}%
\pgfpathmoveto{\pgfqpoint{4.754363in}{5.127107in}}%
\pgfpathlineto{\pgfqpoint{4.707121in}{5.139953in}}%
\pgfusepath{stroke}%
\end{pgfscope}%
\begin{pgfscope}%
\pgfpathrectangle{\pgfqpoint{3.985294in}{4.155455in}}{\pgfqpoint{2.279412in}{2.004545in}}%
\pgfusepath{clip}%
\pgfsetbuttcap%
\pgfsetroundjoin%
\pgfsetlinewidth{0.593420pt}%
\definecolor{currentstroke}{rgb}{0.267968,0.223549,0.512008}%
\pgfsetstrokecolor{currentstroke}%
\pgfsetdash{}{0pt}%
\pgfpathmoveto{\pgfqpoint{4.707121in}{5.139953in}}%
\pgfpathlineto{\pgfqpoint{4.662349in}{5.150514in}}%
\pgfusepath{stroke}%
\end{pgfscope}%
\begin{pgfscope}%
\pgfpathrectangle{\pgfqpoint{3.985294in}{4.155455in}}{\pgfqpoint{2.279412in}{2.004545in}}%
\pgfusepath{clip}%
\pgfsetbuttcap%
\pgfsetroundjoin%
\pgfsetlinewidth{0.570283pt}%
\definecolor{currentstroke}{rgb}{0.271828,0.209303,0.504434}%
\pgfsetstrokecolor{currentstroke}%
\pgfsetdash{}{0pt}%
\pgfpathmoveto{\pgfqpoint{4.662349in}{5.150514in}}%
\pgfpathlineto{\pgfqpoint{4.662349in}{5.150514in}}%
\pgfusepath{stroke}%
\end{pgfscope}%
\begin{pgfscope}%
\pgfpathrectangle{\pgfqpoint{3.985294in}{4.155455in}}{\pgfqpoint{2.279412in}{2.004545in}}%
\pgfusepath{clip}%
\pgfsetbuttcap%
\pgfsetroundjoin%
\pgfsetlinewidth{0.570283pt}%
\definecolor{currentstroke}{rgb}{0.271828,0.209303,0.504434}%
\pgfsetstrokecolor{currentstroke}%
\pgfsetdash{}{0pt}%
\pgfpathmoveto{\pgfqpoint{4.662349in}{5.150514in}}%
\pgfpathlineto{\pgfqpoint{4.654512in}{5.151881in}}%
\pgfusepath{stroke}%
\end{pgfscope}%
\begin{pgfscope}%
\pgfpathrectangle{\pgfqpoint{3.985294in}{4.155455in}}{\pgfqpoint{2.279412in}{2.004545in}}%
\pgfusepath{clip}%
\pgfsetbuttcap%
\pgfsetroundjoin%
\pgfsetlinewidth{0.541384pt}%
\definecolor{currentstroke}{rgb}{0.276194,0.190074,0.493001}%
\pgfsetstrokecolor{currentstroke}%
\pgfsetdash{}{0pt}%
\pgfpathmoveto{\pgfqpoint{4.654512in}{5.151881in}}%
\pgfpathlineto{\pgfqpoint{4.654512in}{5.151881in}}%
\pgfusepath{stroke}%
\end{pgfscope}%
\begin{pgfscope}%
\pgfpathrectangle{\pgfqpoint{3.985294in}{4.155455in}}{\pgfqpoint{2.279412in}{2.004545in}}%
\pgfusepath{clip}%
\pgfsetbuttcap%
\pgfsetroundjoin%
\pgfsetlinewidth{0.541384pt}%
\definecolor{currentstroke}{rgb}{0.276194,0.190074,0.493001}%
\pgfsetstrokecolor{currentstroke}%
\pgfsetdash{}{0pt}%
\pgfpathmoveto{\pgfqpoint{4.654512in}{5.151881in}}%
\pgfpathlineto{\pgfqpoint{4.651144in}{5.152659in}}%
\pgfusepath{stroke}%
\end{pgfscope}%
\begin{pgfscope}%
\pgfpathrectangle{\pgfqpoint{3.985294in}{4.155455in}}{\pgfqpoint{2.279412in}{2.004545in}}%
\pgfusepath{clip}%
\pgfsetbuttcap%
\pgfsetroundjoin%
\pgfsetlinewidth{0.524874pt}%
\definecolor{currentstroke}{rgb}{0.278826,0.175490,0.483397}%
\pgfsetstrokecolor{currentstroke}%
\pgfsetdash{}{0pt}%
\pgfpathmoveto{\pgfqpoint{4.651144in}{5.152659in}}%
\pgfpathlineto{\pgfqpoint{4.649490in}{5.152968in}}%
\pgfusepath{stroke}%
\end{pgfscope}%
\begin{pgfscope}%
\pgfpathrectangle{\pgfqpoint{3.985294in}{4.155455in}}{\pgfqpoint{2.279412in}{2.004545in}}%
\pgfusepath{clip}%
\pgfsetbuttcap%
\pgfsetroundjoin%
\pgfsetlinewidth{0.526226pt}%
\definecolor{currentstroke}{rgb}{0.278826,0.175490,0.483397}%
\pgfsetstrokecolor{currentstroke}%
\pgfsetdash{}{0pt}%
\pgfpathmoveto{\pgfqpoint{4.649490in}{5.152968in}}%
\pgfpathlineto{\pgfqpoint{4.648531in}{5.153173in}}%
\pgfusepath{stroke}%
\end{pgfscope}%
\begin{pgfscope}%
\pgfpathrectangle{\pgfqpoint{3.985294in}{4.155455in}}{\pgfqpoint{2.279412in}{2.004545in}}%
\pgfusepath{clip}%
\pgfsetbuttcap%
\pgfsetroundjoin%
\pgfsetlinewidth{0.527379pt}%
\definecolor{currentstroke}{rgb}{0.278826,0.175490,0.483397}%
\pgfsetstrokecolor{currentstroke}%
\pgfsetdash{}{0pt}%
\pgfpathmoveto{\pgfqpoint{4.648531in}{5.153173in}}%
\pgfpathlineto{\pgfqpoint{4.648352in}{5.153177in}}%
\pgfusepath{stroke}%
\end{pgfscope}%
\begin{pgfscope}%
\pgfpathrectangle{\pgfqpoint{3.985294in}{4.155455in}}{\pgfqpoint{2.279412in}{2.004545in}}%
\pgfusepath{clip}%
\pgfsetbuttcap%
\pgfsetroundjoin%
\pgfsetlinewidth{0.527956pt}%
\definecolor{currentstroke}{rgb}{0.278012,0.180367,0.486697}%
\pgfsetstrokecolor{currentstroke}%
\pgfsetdash{}{0pt}%
\pgfpathmoveto{\pgfqpoint{4.648352in}{5.153177in}}%
\pgfpathlineto{\pgfqpoint{4.649119in}{5.152914in}}%
\pgfusepath{stroke}%
\end{pgfscope}%
\begin{pgfscope}%
\pgfpathrectangle{\pgfqpoint{3.985294in}{4.155455in}}{\pgfqpoint{2.279412in}{2.004545in}}%
\pgfusepath{clip}%
\pgfsetbuttcap%
\pgfsetroundjoin%
\pgfsetlinewidth{0.527998pt}%
\definecolor{currentstroke}{rgb}{0.278012,0.180367,0.486697}%
\pgfsetstrokecolor{currentstroke}%
\pgfsetdash{}{0pt}%
\pgfpathmoveto{\pgfqpoint{4.649119in}{5.152914in}}%
\pgfpathlineto{\pgfqpoint{4.649119in}{5.152914in}}%
\pgfusepath{stroke}%
\end{pgfscope}%
\begin{pgfscope}%
\pgfpathrectangle{\pgfqpoint{3.985294in}{4.155455in}}{\pgfqpoint{2.279412in}{2.004545in}}%
\pgfusepath{clip}%
\pgfsetbuttcap%
\pgfsetroundjoin%
\pgfsetlinewidth{0.527998pt}%
\definecolor{currentstroke}{rgb}{0.278012,0.180367,0.486697}%
\pgfsetstrokecolor{currentstroke}%
\pgfsetdash{}{0pt}%
\pgfpathmoveto{\pgfqpoint{4.649119in}{5.152914in}}%
\pgfpathlineto{\pgfqpoint{4.649443in}{5.152794in}}%
\pgfusepath{stroke}%
\end{pgfscope}%
\begin{pgfscope}%
\pgfpathrectangle{\pgfqpoint{3.985294in}{4.155455in}}{\pgfqpoint{2.279412in}{2.004545in}}%
\pgfusepath{clip}%
\pgfsetbuttcap%
\pgfsetroundjoin%
\pgfsetlinewidth{0.528210pt}%
\definecolor{currentstroke}{rgb}{0.278012,0.180367,0.486697}%
\pgfsetstrokecolor{currentstroke}%
\pgfsetdash{}{0pt}%
\pgfpathmoveto{\pgfqpoint{4.649443in}{5.152794in}}%
\pgfpathlineto{\pgfqpoint{4.649619in}{5.152726in}}%
\pgfusepath{stroke}%
\end{pgfscope}%
\begin{pgfscope}%
\pgfpathrectangle{\pgfqpoint{3.985294in}{4.155455in}}{\pgfqpoint{2.279412in}{2.004545in}}%
\pgfusepath{clip}%
\pgfsetbuttcap%
\pgfsetroundjoin%
\pgfsetlinewidth{0.528375pt}%
\definecolor{currentstroke}{rgb}{0.278012,0.180367,0.486697}%
\pgfsetstrokecolor{currentstroke}%
\pgfsetdash{}{0pt}%
\pgfpathmoveto{\pgfqpoint{4.649619in}{5.152726in}}%
\pgfpathlineto{\pgfqpoint{4.649421in}{5.152782in}}%
\pgfusepath{stroke}%
\end{pgfscope}%
\begin{pgfscope}%
\pgfpathrectangle{\pgfqpoint{3.985294in}{4.155455in}}{\pgfqpoint{2.279412in}{2.004545in}}%
\pgfusepath{clip}%
\pgfsetbuttcap%
\pgfsetroundjoin%
\pgfsetlinewidth{0.528402pt}%
\definecolor{currentstroke}{rgb}{0.278012,0.180367,0.486697}%
\pgfsetstrokecolor{currentstroke}%
\pgfsetdash{}{0pt}%
\pgfpathmoveto{\pgfqpoint{4.649421in}{5.152782in}}%
\pgfpathlineto{\pgfqpoint{4.649089in}{5.152886in}}%
\pgfusepath{stroke}%
\end{pgfscope}%
\begin{pgfscope}%
\pgfpathrectangle{\pgfqpoint{3.985294in}{4.155455in}}{\pgfqpoint{2.279412in}{2.004545in}}%
\pgfusepath{clip}%
\pgfsetbuttcap%
\pgfsetroundjoin%
\pgfsetlinewidth{0.528385pt}%
\definecolor{currentstroke}{rgb}{0.278012,0.180367,0.486697}%
\pgfsetstrokecolor{currentstroke}%
\pgfsetdash{}{0pt}%
\pgfpathmoveto{\pgfqpoint{4.649089in}{5.152886in}}%
\pgfpathlineto{\pgfqpoint{4.648935in}{5.152936in}}%
\pgfusepath{stroke}%
\end{pgfscope}%
\begin{pgfscope}%
\pgfpathrectangle{\pgfqpoint{3.985294in}{4.155455in}}{\pgfqpoint{2.279412in}{2.004545in}}%
\pgfusepath{clip}%
\pgfsetbuttcap%
\pgfsetroundjoin%
\pgfsetlinewidth{0.528376pt}%
\definecolor{currentstroke}{rgb}{0.278012,0.180367,0.486697}%
\pgfsetstrokecolor{currentstroke}%
\pgfsetdash{}{0pt}%
\pgfpathmoveto{\pgfqpoint{4.648935in}{5.152936in}}%
\pgfpathlineto{\pgfqpoint{4.649088in}{5.152889in}}%
\pgfusepath{stroke}%
\end{pgfscope}%
\begin{pgfscope}%
\pgfpathrectangle{\pgfqpoint{3.985294in}{4.155455in}}{\pgfqpoint{2.279412in}{2.004545in}}%
\pgfusepath{clip}%
\pgfsetbuttcap%
\pgfsetroundjoin%
\pgfsetlinewidth{0.528354pt}%
\definecolor{currentstroke}{rgb}{0.278012,0.180367,0.486697}%
\pgfsetstrokecolor{currentstroke}%
\pgfsetdash{}{0pt}%
\pgfpathmoveto{\pgfqpoint{4.649088in}{5.152889in}}%
\pgfpathlineto{\pgfqpoint{4.649459in}{5.152771in}}%
\pgfusepath{stroke}%
\end{pgfscope}%
\begin{pgfscope}%
\pgfpathrectangle{\pgfqpoint{3.985294in}{4.155455in}}{\pgfqpoint{2.279412in}{2.004545in}}%
\pgfusepath{clip}%
\pgfsetbuttcap%
\pgfsetroundjoin%
\pgfsetlinewidth{0.528394pt}%
\definecolor{currentstroke}{rgb}{0.278012,0.180367,0.486697}%
\pgfsetstrokecolor{currentstroke}%
\pgfsetdash{}{0pt}%
\pgfpathmoveto{\pgfqpoint{4.649459in}{5.152771in}}%
\pgfpathlineto{\pgfqpoint{4.649662in}{5.152703in}}%
\pgfusepath{stroke}%
\end{pgfscope}%
\begin{pgfscope}%
\pgfpathrectangle{\pgfqpoint{3.985294in}{4.155455in}}{\pgfqpoint{2.279412in}{2.004545in}}%
\pgfusepath{clip}%
\pgfsetbuttcap%
\pgfsetroundjoin%
\pgfsetlinewidth{0.528481pt}%
\definecolor{currentstroke}{rgb}{0.278012,0.180367,0.486697}%
\pgfsetstrokecolor{currentstroke}%
\pgfsetdash{}{0pt}%
\pgfpathmoveto{\pgfqpoint{4.649662in}{5.152703in}}%
\pgfpathlineto{\pgfqpoint{4.649441in}{5.152771in}}%
\pgfusepath{stroke}%
\end{pgfscope}%
\begin{pgfscope}%
\pgfpathrectangle{\pgfqpoint{3.985294in}{4.155455in}}{\pgfqpoint{2.279412in}{2.004545in}}%
\pgfusepath{clip}%
\pgfsetbuttcap%
\pgfsetroundjoin%
\pgfsetlinewidth{0.528455pt}%
\definecolor{currentstroke}{rgb}{0.278012,0.180367,0.486697}%
\pgfsetstrokecolor{currentstroke}%
\pgfsetdash{}{0pt}%
\pgfpathmoveto{\pgfqpoint{4.649441in}{5.152771in}}%
\pgfpathlineto{\pgfqpoint{4.649073in}{5.152888in}}%
\pgfusepath{stroke}%
\end{pgfscope}%
\begin{pgfscope}%
\pgfpathrectangle{\pgfqpoint{3.985294in}{4.155455in}}{\pgfqpoint{2.279412in}{2.004545in}}%
\pgfusepath{clip}%
\pgfsetbuttcap%
\pgfsetroundjoin%
\pgfsetlinewidth{0.528410pt}%
\definecolor{currentstroke}{rgb}{0.278012,0.180367,0.486697}%
\pgfsetstrokecolor{currentstroke}%
\pgfsetdash{}{0pt}%
\pgfpathmoveto{\pgfqpoint{4.649073in}{5.152888in}}%
\pgfpathlineto{\pgfqpoint{4.648902in}{5.152945in}}%
\pgfusepath{stroke}%
\end{pgfscope}%
\begin{pgfscope}%
\pgfpathrectangle{\pgfqpoint{3.985294in}{4.155455in}}{\pgfqpoint{2.279412in}{2.004545in}}%
\pgfusepath{clip}%
\pgfsetbuttcap%
\pgfsetroundjoin%
\pgfsetlinewidth{0.528390pt}%
\definecolor{currentstroke}{rgb}{0.278012,0.180367,0.486697}%
\pgfsetstrokecolor{currentstroke}%
\pgfsetdash{}{0pt}%
\pgfpathmoveto{\pgfqpoint{4.648902in}{5.152945in}}%
\pgfpathlineto{\pgfqpoint{4.649069in}{5.152895in}}%
\pgfusepath{stroke}%
\end{pgfscope}%
\begin{pgfscope}%
\pgfpathrectangle{\pgfqpoint{3.985294in}{4.155455in}}{\pgfqpoint{2.279412in}{2.004545in}}%
\pgfusepath{clip}%
\pgfsetbuttcap%
\pgfsetroundjoin%
\pgfsetlinewidth{0.528358pt}%
\definecolor{currentstroke}{rgb}{0.278012,0.180367,0.486697}%
\pgfsetstrokecolor{currentstroke}%
\pgfsetdash{}{0pt}%
\pgfpathmoveto{\pgfqpoint{4.649069in}{5.152895in}}%
\pgfpathlineto{\pgfqpoint{4.649482in}{5.152764in}}%
\pgfusepath{stroke}%
\end{pgfscope}%
\begin{pgfscope}%
\pgfpathrectangle{\pgfqpoint{3.985294in}{4.155455in}}{\pgfqpoint{2.279412in}{2.004545in}}%
\pgfusepath{clip}%
\pgfsetbuttcap%
\pgfsetroundjoin%
\pgfsetlinewidth{0.528399pt}%
\definecolor{currentstroke}{rgb}{0.278012,0.180367,0.486697}%
\pgfsetstrokecolor{currentstroke}%
\pgfsetdash{}{0pt}%
\pgfpathmoveto{\pgfqpoint{4.649482in}{5.152764in}}%
\pgfpathlineto{\pgfqpoint{4.649712in}{5.152687in}}%
\pgfusepath{stroke}%
\end{pgfscope}%
\begin{pgfscope}%
\pgfpathrectangle{\pgfqpoint{3.985294in}{4.155455in}}{\pgfqpoint{2.279412in}{2.004545in}}%
\pgfusepath{clip}%
\pgfsetbuttcap%
\pgfsetroundjoin%
\pgfsetlinewidth{0.528501pt}%
\definecolor{currentstroke}{rgb}{0.278012,0.180367,0.486697}%
\pgfsetstrokecolor{currentstroke}%
\pgfsetdash{}{0pt}%
\pgfpathmoveto{\pgfqpoint{4.649712in}{5.152687in}}%
\pgfpathlineto{\pgfqpoint{4.649464in}{5.152763in}}%
\pgfusepath{stroke}%
\end{pgfscope}%
\begin{pgfscope}%
\pgfpathrectangle{\pgfqpoint{3.985294in}{4.155455in}}{\pgfqpoint{2.279412in}{2.004545in}}%
\pgfusepath{clip}%
\pgfsetbuttcap%
\pgfsetroundjoin%
\pgfsetlinewidth{0.528466pt}%
\definecolor{currentstroke}{rgb}{0.278012,0.180367,0.486697}%
\pgfsetstrokecolor{currentstroke}%
\pgfsetdash{}{0pt}%
\pgfpathmoveto{\pgfqpoint{4.649464in}{5.152763in}}%
\pgfpathlineto{\pgfqpoint{4.649057in}{5.152893in}}%
\pgfusepath{stroke}%
\end{pgfscope}%
\begin{pgfscope}%
\pgfpathrectangle{\pgfqpoint{3.985294in}{4.155455in}}{\pgfqpoint{2.279412in}{2.004545in}}%
\pgfusepath{clip}%
\pgfsetbuttcap%
\pgfsetroundjoin%
\pgfsetlinewidth{0.528415pt}%
\definecolor{currentstroke}{rgb}{0.278012,0.180367,0.486697}%
\pgfsetstrokecolor{currentstroke}%
\pgfsetdash{}{0pt}%
\pgfpathmoveto{\pgfqpoint{4.649057in}{5.152893in}}%
\pgfpathlineto{\pgfqpoint{4.648869in}{5.152955in}}%
\pgfusepath{stroke}%
\end{pgfscope}%
\begin{pgfscope}%
\pgfpathrectangle{\pgfqpoint{3.985294in}{4.155455in}}{\pgfqpoint{2.279412in}{2.004545in}}%
\pgfusepath{clip}%
\pgfsetbuttcap%
\pgfsetroundjoin%
\pgfsetlinewidth{0.528395pt}%
\definecolor{currentstroke}{rgb}{0.278012,0.180367,0.486697}%
\pgfsetstrokecolor{currentstroke}%
\pgfsetdash{}{0pt}%
\pgfpathmoveto{\pgfqpoint{4.648869in}{5.152955in}}%
\pgfpathlineto{\pgfqpoint{4.649048in}{5.152901in}}%
\pgfusepath{stroke}%
\end{pgfscope}%
\begin{pgfscope}%
\pgfpathrectangle{\pgfqpoint{3.985294in}{4.155455in}}{\pgfqpoint{2.279412in}{2.004545in}}%
\pgfusepath{clip}%
\pgfsetbuttcap%
\pgfsetroundjoin%
\pgfsetlinewidth{0.528357pt}%
\definecolor{currentstroke}{rgb}{0.278012,0.180367,0.486697}%
\pgfsetstrokecolor{currentstroke}%
\pgfsetdash{}{0pt}%
\pgfpathmoveto{\pgfqpoint{4.649048in}{5.152901in}}%
\pgfpathlineto{\pgfqpoint{4.649507in}{5.152756in}}%
\pgfusepath{stroke}%
\end{pgfscope}%
\begin{pgfscope}%
\pgfpathrectangle{\pgfqpoint{3.985294in}{4.155455in}}{\pgfqpoint{2.279412in}{2.004545in}}%
\pgfusepath{clip}%
\pgfsetbuttcap%
\pgfsetroundjoin%
\pgfsetlinewidth{0.528402pt}%
\definecolor{currentstroke}{rgb}{0.278012,0.180367,0.486697}%
\pgfsetstrokecolor{currentstroke}%
\pgfsetdash{}{0pt}%
\pgfpathmoveto{\pgfqpoint{4.649507in}{5.152756in}}%
\pgfpathlineto{\pgfqpoint{4.649767in}{5.152669in}}%
\pgfusepath{stroke}%
\end{pgfscope}%
\begin{pgfscope}%
\pgfpathrectangle{\pgfqpoint{3.985294in}{4.155455in}}{\pgfqpoint{2.279412in}{2.004545in}}%
\pgfusepath{clip}%
\pgfsetbuttcap%
\pgfsetroundjoin%
\pgfsetlinewidth{0.528523pt}%
\definecolor{currentstroke}{rgb}{0.278012,0.180367,0.486697}%
\pgfsetstrokecolor{currentstroke}%
\pgfsetdash{}{0pt}%
\pgfpathmoveto{\pgfqpoint{4.649767in}{5.152669in}}%
\pgfpathlineto{\pgfqpoint{4.649489in}{5.152754in}}%
\pgfusepath{stroke}%
\end{pgfscope}%
\begin{pgfscope}%
\pgfpathrectangle{\pgfqpoint{3.985294in}{4.155455in}}{\pgfqpoint{2.279412in}{2.004545in}}%
\pgfusepath{clip}%
\pgfsetbuttcap%
\pgfsetroundjoin%
\pgfsetlinewidth{0.528479pt}%
\definecolor{currentstroke}{rgb}{0.278012,0.180367,0.486697}%
\pgfsetstrokecolor{currentstroke}%
\pgfsetdash{}{0pt}%
\pgfpathmoveto{\pgfqpoint{4.649489in}{5.152754in}}%
\pgfpathlineto{\pgfqpoint{4.649041in}{5.152897in}}%
\pgfusepath{stroke}%
\end{pgfscope}%
\begin{pgfscope}%
\pgfpathrectangle{\pgfqpoint{3.985294in}{4.155455in}}{\pgfqpoint{2.279412in}{2.004545in}}%
\pgfusepath{clip}%
\pgfsetbuttcap%
\pgfsetroundjoin%
\pgfsetlinewidth{0.528420pt}%
\definecolor{currentstroke}{rgb}{0.278012,0.180367,0.486697}%
\pgfsetstrokecolor{currentstroke}%
\pgfsetdash{}{0pt}%
\pgfpathmoveto{\pgfqpoint{4.649041in}{5.152897in}}%
\pgfpathlineto{\pgfqpoint{4.648835in}{5.152966in}}%
\pgfusepath{stroke}%
\end{pgfscope}%
\begin{pgfscope}%
\pgfpathrectangle{\pgfqpoint{3.985294in}{4.155455in}}{\pgfqpoint{2.279412in}{2.004545in}}%
\pgfusepath{clip}%
\pgfsetbuttcap%
\pgfsetroundjoin%
\pgfsetlinewidth{0.528401pt}%
\definecolor{currentstroke}{rgb}{0.278012,0.180367,0.486697}%
\pgfsetstrokecolor{currentstroke}%
\pgfsetdash{}{0pt}%
\pgfpathmoveto{\pgfqpoint{4.648835in}{5.152966in}}%
\pgfpathlineto{\pgfqpoint{4.649026in}{5.152909in}}%
\pgfusepath{stroke}%
\end{pgfscope}%
\begin{pgfscope}%
\pgfpathrectangle{\pgfqpoint{3.985294in}{4.155455in}}{\pgfqpoint{2.279412in}{2.004545in}}%
\pgfusepath{clip}%
\pgfsetbuttcap%
\pgfsetroundjoin%
\pgfsetlinewidth{0.528356pt}%
\definecolor{currentstroke}{rgb}{0.278012,0.180367,0.486697}%
\pgfsetstrokecolor{currentstroke}%
\pgfsetdash{}{0pt}%
\pgfpathmoveto{\pgfqpoint{4.649026in}{5.152909in}}%
\pgfpathlineto{\pgfqpoint{4.649532in}{5.152748in}}%
\pgfusepath{stroke}%
\end{pgfscope}%
\begin{pgfscope}%
\pgfpathrectangle{\pgfqpoint{3.985294in}{4.155455in}}{\pgfqpoint{2.279412in}{2.004545in}}%
\pgfusepath{clip}%
\pgfsetbuttcap%
\pgfsetroundjoin%
\pgfsetlinewidth{0.528406pt}%
\definecolor{currentstroke}{rgb}{0.278012,0.180367,0.486697}%
\pgfsetstrokecolor{currentstroke}%
\pgfsetdash{}{0pt}%
\pgfpathmoveto{\pgfqpoint{4.649532in}{5.152748in}}%
\pgfpathlineto{\pgfqpoint{4.649826in}{5.152650in}}%
\pgfusepath{stroke}%
\end{pgfscope}%
\begin{pgfscope}%
\pgfpathrectangle{\pgfqpoint{3.985294in}{4.155455in}}{\pgfqpoint{2.279412in}{2.004545in}}%
\pgfusepath{clip}%
\pgfsetbuttcap%
\pgfsetroundjoin%
\pgfsetlinewidth{0.528549pt}%
\definecolor{currentstroke}{rgb}{0.278012,0.180367,0.486697}%
\pgfsetstrokecolor{currentstroke}%
\pgfsetdash{}{0pt}%
\pgfpathmoveto{\pgfqpoint{4.649826in}{5.152650in}}%
\pgfpathlineto{\pgfqpoint{4.649518in}{5.152744in}}%
\pgfusepath{stroke}%
\end{pgfscope}%
\begin{pgfscope}%
\pgfpathrectangle{\pgfqpoint{3.985294in}{4.155455in}}{\pgfqpoint{2.279412in}{2.004545in}}%
\pgfusepath{clip}%
\pgfsetbuttcap%
\pgfsetroundjoin%
\pgfsetlinewidth{0.528493pt}%
\definecolor{currentstroke}{rgb}{0.278012,0.180367,0.486697}%
\pgfsetstrokecolor{currentstroke}%
\pgfsetdash{}{0pt}%
\pgfpathmoveto{\pgfqpoint{4.649518in}{5.152744in}}%
\pgfpathlineto{\pgfqpoint{4.649026in}{5.152902in}}%
\pgfusepath{stroke}%
\end{pgfscope}%
\begin{pgfscope}%
\pgfpathrectangle{\pgfqpoint{3.985294in}{4.155455in}}{\pgfqpoint{2.279412in}{2.004545in}}%
\pgfusepath{clip}%
\pgfsetbuttcap%
\pgfsetroundjoin%
\pgfsetlinewidth{0.528426pt}%
\definecolor{currentstroke}{rgb}{0.278012,0.180367,0.486697}%
\pgfsetstrokecolor{currentstroke}%
\pgfsetdash{}{0pt}%
\pgfpathmoveto{\pgfqpoint{4.649026in}{5.152902in}}%
\pgfpathlineto{\pgfqpoint{4.648800in}{5.152977in}}%
\pgfusepath{stroke}%
\end{pgfscope}%
\begin{pgfscope}%
\pgfpathrectangle{\pgfqpoint{3.985294in}{4.155455in}}{\pgfqpoint{2.279412in}{2.004545in}}%
\pgfusepath{clip}%
\pgfsetbuttcap%
\pgfsetroundjoin%
\pgfsetlinewidth{0.528408pt}%
\definecolor{currentstroke}{rgb}{0.278012,0.180367,0.486697}%
\pgfsetstrokecolor{currentstroke}%
\pgfsetdash{}{0pt}%
\pgfpathmoveto{\pgfqpoint{4.648800in}{5.152977in}}%
\pgfpathlineto{\pgfqpoint{4.649002in}{5.152916in}}%
\pgfusepath{stroke}%
\end{pgfscope}%
\begin{pgfscope}%
\pgfpathrectangle{\pgfqpoint{3.985294in}{4.155455in}}{\pgfqpoint{2.279412in}{2.004545in}}%
\pgfusepath{clip}%
\pgfsetbuttcap%
\pgfsetroundjoin%
\pgfsetlinewidth{0.528357pt}%
\definecolor{currentstroke}{rgb}{0.278012,0.180367,0.486697}%
\pgfsetstrokecolor{currentstroke}%
\pgfsetdash{}{0pt}%
\pgfpathmoveto{\pgfqpoint{4.649002in}{5.152916in}}%
\pgfpathlineto{\pgfqpoint{4.649558in}{5.152740in}}%
\pgfusepath{stroke}%
\end{pgfscope}%
\begin{pgfscope}%
\pgfpathrectangle{\pgfqpoint{3.985294in}{4.155455in}}{\pgfqpoint{2.279412in}{2.004545in}}%
\pgfusepath{clip}%
\pgfsetbuttcap%
\pgfsetroundjoin%
\pgfsetlinewidth{0.528410pt}%
\definecolor{currentstroke}{rgb}{0.278012,0.180367,0.486697}%
\pgfsetstrokecolor{currentstroke}%
\pgfsetdash{}{0pt}%
\pgfpathmoveto{\pgfqpoint{4.649558in}{5.152740in}}%
\pgfpathlineto{\pgfqpoint{4.649889in}{5.152629in}}%
\pgfusepath{stroke}%
\end{pgfscope}%
\begin{pgfscope}%
\pgfpathrectangle{\pgfqpoint{3.985294in}{4.155455in}}{\pgfqpoint{2.279412in}{2.004545in}}%
\pgfusepath{clip}%
\pgfsetbuttcap%
\pgfsetroundjoin%
\pgfsetlinewidth{0.528579pt}%
\definecolor{currentstroke}{rgb}{0.278012,0.180367,0.486697}%
\pgfsetstrokecolor{currentstroke}%
\pgfsetdash{}{0pt}%
\pgfpathmoveto{\pgfqpoint{4.649889in}{5.152629in}}%
\pgfpathlineto{\pgfqpoint{4.649549in}{5.152734in}}%
\pgfusepath{stroke}%
\end{pgfscope}%
\begin{pgfscope}%
\pgfpathrectangle{\pgfqpoint{3.985294in}{4.155455in}}{\pgfqpoint{2.279412in}{2.004545in}}%
\pgfusepath{clip}%
\pgfsetbuttcap%
\pgfsetroundjoin%
\pgfsetlinewidth{0.528510pt}%
\definecolor{currentstroke}{rgb}{0.278012,0.180367,0.486697}%
\pgfsetstrokecolor{currentstroke}%
\pgfsetdash{}{0pt}%
\pgfpathmoveto{\pgfqpoint{4.649549in}{5.152734in}}%
\pgfpathlineto{\pgfqpoint{4.649012in}{5.152905in}}%
\pgfusepath{stroke}%
\end{pgfscope}%
\begin{pgfscope}%
\pgfpathrectangle{\pgfqpoint{3.985294in}{4.155455in}}{\pgfqpoint{2.279412in}{2.004545in}}%
\pgfusepath{clip}%
\pgfsetbuttcap%
\pgfsetroundjoin%
\pgfsetlinewidth{0.528433pt}%
\definecolor{currentstroke}{rgb}{0.278012,0.180367,0.486697}%
\pgfsetstrokecolor{currentstroke}%
\pgfsetdash{}{0pt}%
\pgfpathmoveto{\pgfqpoint{4.649012in}{5.152905in}}%
\pgfpathlineto{\pgfqpoint{4.648765in}{5.152988in}}%
\pgfusepath{stroke}%
\end{pgfscope}%
\begin{pgfscope}%
\pgfpathrectangle{\pgfqpoint{3.985294in}{4.155455in}}{\pgfqpoint{2.279412in}{2.004545in}}%
\pgfusepath{clip}%
\pgfsetbuttcap%
\pgfsetroundjoin%
\pgfsetlinewidth{0.528416pt}%
\definecolor{currentstroke}{rgb}{0.278012,0.180367,0.486697}%
\pgfsetstrokecolor{currentstroke}%
\pgfsetdash{}{0pt}%
\pgfpathmoveto{\pgfqpoint{4.648765in}{5.152988in}}%
\pgfpathlineto{\pgfqpoint{4.648977in}{5.152924in}}%
\pgfusepath{stroke}%
\end{pgfscope}%
\begin{pgfscope}%
\pgfpathrectangle{\pgfqpoint{3.985294in}{4.155455in}}{\pgfqpoint{2.279412in}{2.004545in}}%
\pgfusepath{clip}%
\pgfsetbuttcap%
\pgfsetroundjoin%
\pgfsetlinewidth{0.528357pt}%
\definecolor{currentstroke}{rgb}{0.278012,0.180367,0.486697}%
\pgfsetstrokecolor{currentstroke}%
\pgfsetdash{}{0pt}%
\pgfpathmoveto{\pgfqpoint{4.648977in}{5.152924in}}%
\pgfpathlineto{\pgfqpoint{4.649583in}{5.152733in}}%
\pgfusepath{stroke}%
\end{pgfscope}%
\begin{pgfscope}%
\pgfpathrectangle{\pgfqpoint{3.985294in}{4.155455in}}{\pgfqpoint{2.279412in}{2.004545in}}%
\pgfusepath{clip}%
\pgfsetbuttcap%
\pgfsetroundjoin%
\pgfsetlinewidth{0.528414pt}%
\definecolor{currentstroke}{rgb}{0.278012,0.180367,0.486697}%
\pgfsetstrokecolor{currentstroke}%
\pgfsetdash{}{0pt}%
\pgfpathmoveto{\pgfqpoint{4.649583in}{5.152733in}}%
\pgfpathlineto{\pgfqpoint{4.649583in}{5.152733in}}%
\pgfusepath{stroke}%
\end{pgfscope}%
\begin{pgfscope}%
\pgfpathrectangle{\pgfqpoint{3.985294in}{4.155455in}}{\pgfqpoint{2.279412in}{2.004545in}}%
\pgfusepath{clip}%
\pgfsetbuttcap%
\pgfsetroundjoin%
\pgfsetlinewidth{0.528414pt}%
\definecolor{currentstroke}{rgb}{0.278012,0.180367,0.486697}%
\pgfsetstrokecolor{currentstroke}%
\pgfsetdash{}{0pt}%
\pgfpathmoveto{\pgfqpoint{4.649583in}{5.152733in}}%
\pgfpathlineto{\pgfqpoint{4.649300in}{5.152819in}}%
\pgfusepath{stroke}%
\end{pgfscope}%
\begin{pgfscope}%
\pgfpathrectangle{\pgfqpoint{3.985294in}{4.155455in}}{\pgfqpoint{2.279412in}{2.004545in}}%
\pgfusepath{clip}%
\pgfsetbuttcap%
\pgfsetroundjoin%
\pgfsetlinewidth{0.528402pt}%
\definecolor{currentstroke}{rgb}{0.278012,0.180367,0.486697}%
\pgfsetstrokecolor{currentstroke}%
\pgfsetdash{}{0pt}%
\pgfpathmoveto{\pgfqpoint{4.649300in}{5.152819in}}%
\pgfpathlineto{\pgfqpoint{4.649006in}{5.152912in}}%
\pgfusepath{stroke}%
\end{pgfscope}%
\begin{pgfscope}%
\pgfpathrectangle{\pgfqpoint{3.985294in}{4.155455in}}{\pgfqpoint{2.279412in}{2.004545in}}%
\pgfusepath{clip}%
\pgfsetbuttcap%
\pgfsetroundjoin%
\pgfsetlinewidth{0.528387pt}%
\definecolor{currentstroke}{rgb}{0.278012,0.180367,0.486697}%
\pgfsetstrokecolor{currentstroke}%
\pgfsetdash{}{0pt}%
\pgfpathmoveto{\pgfqpoint{4.649006in}{5.152912in}}%
\pgfpathlineto{\pgfqpoint{4.648955in}{5.152930in}}%
\pgfusepath{stroke}%
\end{pgfscope}%
\begin{pgfscope}%
\pgfpathrectangle{\pgfqpoint{3.985294in}{4.155455in}}{\pgfqpoint{2.279412in}{2.004545in}}%
\pgfusepath{clip}%
\pgfsetbuttcap%
\pgfsetroundjoin%
\pgfsetlinewidth{0.528370pt}%
\definecolor{currentstroke}{rgb}{0.278012,0.180367,0.486697}%
\pgfsetstrokecolor{currentstroke}%
\pgfsetdash{}{0pt}%
\pgfpathmoveto{\pgfqpoint{4.648955in}{5.152930in}}%
\pgfpathlineto{\pgfqpoint{4.649208in}{5.152851in}}%
\pgfusepath{stroke}%
\end{pgfscope}%
\begin{pgfscope}%
\pgfpathrectangle{\pgfqpoint{3.985294in}{4.155455in}}{\pgfqpoint{2.279412in}{2.004545in}}%
\pgfusepath{clip}%
\pgfsetbuttcap%
\pgfsetroundjoin%
\pgfsetlinewidth{0.528356pt}%
\definecolor{currentstroke}{rgb}{0.278012,0.180367,0.486697}%
\pgfsetstrokecolor{currentstroke}%
\pgfsetdash{}{0pt}%
\pgfpathmoveto{\pgfqpoint{4.649208in}{5.152851in}}%
\pgfpathlineto{\pgfqpoint{4.649569in}{5.152735in}}%
\pgfusepath{stroke}%
\end{pgfscope}%
\begin{pgfscope}%
\pgfpathrectangle{\pgfqpoint{3.985294in}{4.155455in}}{\pgfqpoint{2.279412in}{2.004545in}}%
\pgfusepath{clip}%
\pgfsetbuttcap%
\pgfsetroundjoin%
\pgfsetlinewidth{0.528429pt}%
\definecolor{currentstroke}{rgb}{0.278012,0.180367,0.486697}%
\pgfsetstrokecolor{currentstroke}%
\pgfsetdash{}{0pt}%
\pgfpathmoveto{\pgfqpoint{4.649569in}{5.152735in}}%
\pgfpathlineto{\pgfqpoint{4.649622in}{5.152715in}}%
\pgfusepath{stroke}%
\end{pgfscope}%
\begin{pgfscope}%
\pgfpathrectangle{\pgfqpoint{3.985294in}{4.155455in}}{\pgfqpoint{2.279412in}{2.004545in}}%
\pgfusepath{clip}%
\pgfsetbuttcap%
\pgfsetroundjoin%
\pgfsetlinewidth{0.528483pt}%
\definecolor{currentstroke}{rgb}{0.278012,0.180367,0.486697}%
\pgfsetstrokecolor{currentstroke}%
\pgfsetdash{}{0pt}%
\pgfpathmoveto{\pgfqpoint{4.649622in}{5.152715in}}%
\pgfpathlineto{\pgfqpoint{4.649306in}{5.152814in}}%
\pgfusepath{stroke}%
\end{pgfscope}%
\begin{pgfscope}%
\pgfpathrectangle{\pgfqpoint{3.985294in}{4.155455in}}{\pgfqpoint{2.279412in}{2.004545in}}%
\pgfusepath{clip}%
\pgfsetbuttcap%
\pgfsetroundjoin%
\pgfsetlinewidth{0.528434pt}%
\definecolor{currentstroke}{rgb}{0.278012,0.180367,0.486697}%
\pgfsetstrokecolor{currentstroke}%
\pgfsetdash{}{0pt}%
\pgfpathmoveto{\pgfqpoint{4.649306in}{5.152814in}}%
\pgfpathlineto{\pgfqpoint{4.648982in}{5.152918in}}%
\pgfusepath{stroke}%
\end{pgfscope}%
\begin{pgfscope}%
\pgfpathrectangle{\pgfqpoint{3.985294in}{4.155455in}}{\pgfqpoint{2.279412in}{2.004545in}}%
\pgfusepath{clip}%
\pgfsetbuttcap%
\pgfsetroundjoin%
\pgfsetlinewidth{0.528403pt}%
\definecolor{currentstroke}{rgb}{0.278012,0.180367,0.486697}%
\pgfsetstrokecolor{currentstroke}%
\pgfsetdash{}{0pt}%
\pgfpathmoveto{\pgfqpoint{4.648982in}{5.152918in}}%
\pgfpathlineto{\pgfqpoint{4.648924in}{5.152939in}}%
\pgfusepath{stroke}%
\end{pgfscope}%
\begin{pgfscope}%
\pgfpathrectangle{\pgfqpoint{3.985294in}{4.155455in}}{\pgfqpoint{2.279412in}{2.004545in}}%
\pgfusepath{clip}%
\pgfsetbuttcap%
\pgfsetroundjoin%
\pgfsetlinewidth{0.528378pt}%
\definecolor{currentstroke}{rgb}{0.278012,0.180367,0.486697}%
\pgfsetstrokecolor{currentstroke}%
\pgfsetdash{}{0pt}%
\pgfpathmoveto{\pgfqpoint{4.648924in}{5.152939in}}%
\pgfpathlineto{\pgfqpoint{4.649201in}{5.152853in}}%
\pgfusepath{stroke}%
\end{pgfscope}%
\begin{pgfscope}%
\pgfpathrectangle{\pgfqpoint{3.985294in}{4.155455in}}{\pgfqpoint{2.279412in}{2.004545in}}%
\pgfusepath{clip}%
\pgfsetbuttcap%
\pgfsetroundjoin%
\pgfsetlinewidth{0.528357pt}%
\definecolor{currentstroke}{rgb}{0.278012,0.180367,0.486697}%
\pgfsetstrokecolor{currentstroke}%
\pgfsetdash{}{0pt}%
\pgfpathmoveto{\pgfqpoint{4.649201in}{5.152853in}}%
\pgfpathlineto{\pgfqpoint{4.649607in}{5.152723in}}%
\pgfusepath{stroke}%
\end{pgfscope}%
\begin{pgfscope}%
\pgfpathrectangle{\pgfqpoint{3.985294in}{4.155455in}}{\pgfqpoint{2.279412in}{2.004545in}}%
\pgfusepath{clip}%
\pgfsetbuttcap%
\pgfsetroundjoin%
\pgfsetlinewidth{0.528440pt}%
\definecolor{currentstroke}{rgb}{0.278012,0.180367,0.486697}%
\pgfsetstrokecolor{currentstroke}%
\pgfsetdash{}{0pt}%
\pgfpathmoveto{\pgfqpoint{4.649607in}{5.152723in}}%
\pgfpathlineto{\pgfqpoint{4.649667in}{5.152700in}}%
\pgfusepath{stroke}%
\end{pgfscope}%
\begin{pgfscope}%
\pgfpathrectangle{\pgfqpoint{3.985294in}{4.155455in}}{\pgfqpoint{2.279412in}{2.004545in}}%
\pgfusepath{clip}%
\pgfsetbuttcap%
\pgfsetroundjoin%
\pgfsetlinewidth{0.528502pt}%
\definecolor{currentstroke}{rgb}{0.278012,0.180367,0.486697}%
\pgfsetstrokecolor{currentstroke}%
\pgfsetdash{}{0pt}%
\pgfpathmoveto{\pgfqpoint{4.649667in}{5.152700in}}%
\pgfpathlineto{\pgfqpoint{4.649314in}{5.152811in}}%
\pgfusepath{stroke}%
\end{pgfscope}%
\begin{pgfscope}%
\pgfpathrectangle{\pgfqpoint{3.985294in}{4.155455in}}{\pgfqpoint{2.279412in}{2.004545in}}%
\pgfusepath{clip}%
\pgfsetbuttcap%
\pgfsetroundjoin%
\pgfsetlinewidth{0.528441pt}%
\definecolor{currentstroke}{rgb}{0.278012,0.180367,0.486697}%
\pgfsetstrokecolor{currentstroke}%
\pgfsetdash{}{0pt}%
\pgfpathmoveto{\pgfqpoint{4.649314in}{5.152811in}}%
\pgfpathlineto{\pgfqpoint{4.648957in}{5.152926in}}%
\pgfusepath{stroke}%
\end{pgfscope}%
\begin{pgfscope}%
\pgfpathrectangle{\pgfqpoint{3.985294in}{4.155455in}}{\pgfqpoint{2.279412in}{2.004545in}}%
\pgfusepath{clip}%
\pgfsetbuttcap%
\pgfsetroundjoin%
\pgfsetlinewidth{0.528408pt}%
\definecolor{currentstroke}{rgb}{0.278012,0.180367,0.486697}%
\pgfsetstrokecolor{currentstroke}%
\pgfsetdash{}{0pt}%
\pgfpathmoveto{\pgfqpoint{4.648957in}{5.152926in}}%
\pgfpathlineto{\pgfqpoint{4.648892in}{5.152949in}}%
\pgfusepath{stroke}%
\end{pgfscope}%
\begin{pgfscope}%
\pgfpathrectangle{\pgfqpoint{3.985294in}{4.155455in}}{\pgfqpoint{2.279412in}{2.004545in}}%
\pgfusepath{clip}%
\pgfsetbuttcap%
\pgfsetroundjoin%
\pgfsetlinewidth{0.528382pt}%
\definecolor{currentstroke}{rgb}{0.278012,0.180367,0.486697}%
\pgfsetstrokecolor{currentstroke}%
\pgfsetdash{}{0pt}%
\pgfpathmoveto{\pgfqpoint{4.648892in}{5.152949in}}%
\pgfpathlineto{\pgfqpoint{4.649194in}{5.152856in}}%
\pgfusepath{stroke}%
\end{pgfscope}%
\begin{pgfscope}%
\pgfpathrectangle{\pgfqpoint{3.985294in}{4.155455in}}{\pgfqpoint{2.279412in}{2.004545in}}%
\pgfusepath{clip}%
\pgfsetbuttcap%
\pgfsetroundjoin%
\pgfsetlinewidth{0.528355pt}%
\definecolor{currentstroke}{rgb}{0.278012,0.180367,0.486697}%
\pgfsetstrokecolor{currentstroke}%
\pgfsetdash{}{0pt}%
\pgfpathmoveto{\pgfqpoint{4.649194in}{5.152856in}}%
\pgfpathlineto{\pgfqpoint{4.649648in}{5.152710in}}%
\pgfusepath{stroke}%
\end{pgfscope}%
\begin{pgfscope}%
\pgfpathrectangle{\pgfqpoint{3.985294in}{4.155455in}}{\pgfqpoint{2.279412in}{2.004545in}}%
\pgfusepath{clip}%
\pgfsetbuttcap%
\pgfsetroundjoin%
\pgfsetlinewidth{0.528450pt}%
\definecolor{currentstroke}{rgb}{0.278012,0.180367,0.486697}%
\pgfsetstrokecolor{currentstroke}%
\pgfsetdash{}{0pt}%
\pgfpathmoveto{\pgfqpoint{4.649648in}{5.152710in}}%
\pgfpathlineto{\pgfqpoint{4.649717in}{5.152683in}}%
\pgfusepath{stroke}%
\end{pgfscope}%
\begin{pgfscope}%
\pgfpathrectangle{\pgfqpoint{3.985294in}{4.155455in}}{\pgfqpoint{2.279412in}{2.004545in}}%
\pgfusepath{clip}%
\pgfsetbuttcap%
\pgfsetroundjoin%
\pgfsetlinewidth{0.528524pt}%
\definecolor{currentstroke}{rgb}{0.278012,0.180367,0.486697}%
\pgfsetstrokecolor{currentstroke}%
\pgfsetdash{}{0pt}%
\pgfpathmoveto{\pgfqpoint{4.649717in}{5.152683in}}%
\pgfpathlineto{\pgfqpoint{4.649323in}{5.152807in}}%
\pgfusepath{stroke}%
\end{pgfscope}%
\begin{pgfscope}%
\pgfpathrectangle{\pgfqpoint{3.985294in}{4.155455in}}{\pgfqpoint{2.279412in}{2.004545in}}%
\pgfusepath{clip}%
\pgfsetbuttcap%
\pgfsetroundjoin%
\pgfsetlinewidth{0.528449pt}%
\definecolor{currentstroke}{rgb}{0.278012,0.180367,0.486697}%
\pgfsetstrokecolor{currentstroke}%
\pgfsetdash{}{0pt}%
\pgfpathmoveto{\pgfqpoint{4.649323in}{5.152807in}}%
\pgfpathlineto{\pgfqpoint{4.648932in}{5.152933in}}%
\pgfusepath{stroke}%
\end{pgfscope}%
\begin{pgfscope}%
\pgfpathrectangle{\pgfqpoint{3.985294in}{4.155455in}}{\pgfqpoint{2.279412in}{2.004545in}}%
\pgfusepath{clip}%
\pgfsetbuttcap%
\pgfsetroundjoin%
\pgfsetlinewidth{0.528414pt}%
\definecolor{currentstroke}{rgb}{0.278012,0.180367,0.486697}%
\pgfsetstrokecolor{currentstroke}%
\pgfsetdash{}{0pt}%
\pgfpathmoveto{\pgfqpoint{4.648932in}{5.152933in}}%
\pgfpathlineto{\pgfqpoint{4.648858in}{5.152960in}}%
\pgfusepath{stroke}%
\end{pgfscope}%
\begin{pgfscope}%
\pgfpathrectangle{\pgfqpoint{3.985294in}{4.155455in}}{\pgfqpoint{2.279412in}{2.004545in}}%
\pgfusepath{clip}%
\pgfsetbuttcap%
\pgfsetroundjoin%
\pgfsetlinewidth{0.528386pt}%
\definecolor{currentstroke}{rgb}{0.278012,0.180367,0.486697}%
\pgfsetstrokecolor{currentstroke}%
\pgfsetdash{}{0pt}%
\pgfpathmoveto{\pgfqpoint{4.648858in}{5.152960in}}%
\pgfpathlineto{\pgfqpoint{4.649185in}{5.152859in}}%
\pgfusepath{stroke}%
\end{pgfscope}%
\begin{pgfscope}%
\pgfpathrectangle{\pgfqpoint{3.985294in}{4.155455in}}{\pgfqpoint{2.279412in}{2.004545in}}%
\pgfusepath{clip}%
\pgfsetbuttcap%
\pgfsetroundjoin%
\pgfsetlinewidth{0.528353pt}%
\definecolor{currentstroke}{rgb}{0.278012,0.180367,0.486697}%
\pgfsetstrokecolor{currentstroke}%
\pgfsetdash{}{0pt}%
\pgfpathmoveto{\pgfqpoint{4.649185in}{5.152859in}}%
\pgfpathlineto{\pgfqpoint{4.649692in}{5.152696in}}%
\pgfusepath{stroke}%
\end{pgfscope}%
\begin{pgfscope}%
\pgfpathrectangle{\pgfqpoint{3.985294in}{4.155455in}}{\pgfqpoint{2.279412in}{2.004545in}}%
\pgfusepath{clip}%
\pgfsetbuttcap%
\pgfsetroundjoin%
\pgfsetlinewidth{0.528463pt}%
\definecolor{currentstroke}{rgb}{0.278012,0.180367,0.486697}%
\pgfsetstrokecolor{currentstroke}%
\pgfsetdash{}{0pt}%
\pgfpathmoveto{\pgfqpoint{4.649692in}{5.152696in}}%
\pgfpathlineto{\pgfqpoint{4.649771in}{5.152666in}}%
\pgfusepath{stroke}%
\end{pgfscope}%
\begin{pgfscope}%
\pgfpathrectangle{\pgfqpoint{3.985294in}{4.155455in}}{\pgfqpoint{2.279412in}{2.004545in}}%
\pgfusepath{clip}%
\pgfsetbuttcap%
\pgfsetroundjoin%
\pgfsetlinewidth{0.528548pt}%
\definecolor{currentstroke}{rgb}{0.278012,0.180367,0.486697}%
\pgfsetstrokecolor{currentstroke}%
\pgfsetdash{}{0pt}%
\pgfpathmoveto{\pgfqpoint{4.649771in}{5.152666in}}%
\pgfpathlineto{\pgfqpoint{4.649334in}{5.152803in}}%
\pgfusepath{stroke}%
\end{pgfscope}%
\begin{pgfscope}%
\pgfpathrectangle{\pgfqpoint{3.985294in}{4.155455in}}{\pgfqpoint{2.279412in}{2.004545in}}%
\pgfusepath{clip}%
\pgfsetbuttcap%
\pgfsetroundjoin%
\pgfsetlinewidth{0.528458pt}%
\definecolor{currentstroke}{rgb}{0.278012,0.180367,0.486697}%
\pgfsetstrokecolor{currentstroke}%
\pgfsetdash{}{0pt}%
\pgfpathmoveto{\pgfqpoint{4.649334in}{5.152803in}}%
\pgfpathlineto{\pgfqpoint{4.648906in}{5.152941in}}%
\pgfusepath{stroke}%
\end{pgfscope}%
\begin{pgfscope}%
\pgfpathrectangle{\pgfqpoint{3.985294in}{4.155455in}}{\pgfqpoint{2.279412in}{2.004545in}}%
\pgfusepath{clip}%
\pgfsetbuttcap%
\pgfsetroundjoin%
\pgfsetlinewidth{0.528420pt}%
\definecolor{currentstroke}{rgb}{0.278012,0.180367,0.486697}%
\pgfsetstrokecolor{currentstroke}%
\pgfsetdash{}{0pt}%
\pgfpathmoveto{\pgfqpoint{4.648906in}{5.152941in}}%
\pgfpathlineto{\pgfqpoint{4.648823in}{5.152971in}}%
\pgfusepath{stroke}%
\end{pgfscope}%
\begin{pgfscope}%
\pgfpathrectangle{\pgfqpoint{3.985294in}{4.155455in}}{\pgfqpoint{2.279412in}{2.004545in}}%
\pgfusepath{clip}%
\pgfsetbuttcap%
\pgfsetroundjoin%
\pgfsetlinewidth{0.528392pt}%
\definecolor{currentstroke}{rgb}{0.278012,0.180367,0.486697}%
\pgfsetstrokecolor{currentstroke}%
\pgfsetdash{}{0pt}%
\pgfpathmoveto{\pgfqpoint{4.648823in}{5.152971in}}%
\pgfpathlineto{\pgfqpoint{4.649175in}{5.152862in}}%
\pgfusepath{stroke}%
\end{pgfscope}%
\begin{pgfscope}%
\pgfpathrectangle{\pgfqpoint{3.985294in}{4.155455in}}{\pgfqpoint{2.279412in}{2.004545in}}%
\pgfusepath{clip}%
\pgfsetbuttcap%
\pgfsetroundjoin%
\pgfsetlinewidth{0.528351pt}%
\definecolor{currentstroke}{rgb}{0.278012,0.180367,0.486697}%
\pgfsetstrokecolor{currentstroke}%
\pgfsetdash{}{0pt}%
\pgfpathmoveto{\pgfqpoint{4.649175in}{5.152862in}}%
\pgfpathlineto{\pgfqpoint{4.649737in}{5.152682in}}%
\pgfusepath{stroke}%
\end{pgfscope}%
\begin{pgfscope}%
\pgfpathrectangle{\pgfqpoint{3.985294in}{4.155455in}}{\pgfqpoint{2.279412in}{2.004545in}}%
\pgfusepath{clip}%
\pgfsetbuttcap%
\pgfsetroundjoin%
\pgfsetlinewidth{0.528477pt}%
\definecolor{currentstroke}{rgb}{0.278012,0.180367,0.486697}%
\pgfsetstrokecolor{currentstroke}%
\pgfsetdash{}{0pt}%
\pgfpathmoveto{\pgfqpoint{4.649737in}{5.152682in}}%
\pgfpathlineto{\pgfqpoint{4.649828in}{5.152647in}}%
\pgfusepath{stroke}%
\end{pgfscope}%
\begin{pgfscope}%
\pgfpathrectangle{\pgfqpoint{3.985294in}{4.155455in}}{\pgfqpoint{2.279412in}{2.004545in}}%
\pgfusepath{clip}%
\pgfsetbuttcap%
\pgfsetroundjoin%
\pgfsetlinewidth{0.528577pt}%
\definecolor{currentstroke}{rgb}{0.278012,0.180367,0.486697}%
\pgfsetstrokecolor{currentstroke}%
\pgfsetdash{}{0pt}%
\pgfpathmoveto{\pgfqpoint{4.649828in}{5.152647in}}%
\pgfpathlineto{\pgfqpoint{4.649347in}{5.152798in}}%
\pgfusepath{stroke}%
\end{pgfscope}%
\begin{pgfscope}%
\pgfpathrectangle{\pgfqpoint{3.985294in}{4.155455in}}{\pgfqpoint{2.279412in}{2.004545in}}%
\pgfusepath{clip}%
\pgfsetbuttcap%
\pgfsetroundjoin%
\pgfsetlinewidth{0.528468pt}%
\definecolor{currentstroke}{rgb}{0.278012,0.180367,0.486697}%
\pgfsetstrokecolor{currentstroke}%
\pgfsetdash{}{0pt}%
\pgfpathmoveto{\pgfqpoint{4.649347in}{5.152798in}}%
\pgfpathlineto{\pgfqpoint{4.648882in}{5.152948in}}%
\pgfusepath{stroke}%
\end{pgfscope}%
\begin{pgfscope}%
\pgfpathrectangle{\pgfqpoint{3.985294in}{4.155455in}}{\pgfqpoint{2.279412in}{2.004545in}}%
\pgfusepath{clip}%
\pgfsetbuttcap%
\pgfsetroundjoin%
\pgfsetlinewidth{0.528427pt}%
\definecolor{currentstroke}{rgb}{0.278012,0.180367,0.486697}%
\pgfsetstrokecolor{currentstroke}%
\pgfsetdash{}{0pt}%
\pgfpathmoveto{\pgfqpoint{4.648882in}{5.152948in}}%
\pgfpathlineto{\pgfqpoint{4.648788in}{5.152982in}}%
\pgfusepath{stroke}%
\end{pgfscope}%
\begin{pgfscope}%
\pgfpathrectangle{\pgfqpoint{3.985294in}{4.155455in}}{\pgfqpoint{2.279412in}{2.004545in}}%
\pgfusepath{clip}%
\pgfsetbuttcap%
\pgfsetroundjoin%
\pgfsetlinewidth{0.528398pt}%
\definecolor{currentstroke}{rgb}{0.278012,0.180367,0.486697}%
\pgfsetstrokecolor{currentstroke}%
\pgfsetdash{}{0pt}%
\pgfpathmoveto{\pgfqpoint{4.648788in}{5.152982in}}%
\pgfpathlineto{\pgfqpoint{4.649163in}{5.152866in}}%
\pgfusepath{stroke}%
\end{pgfscope}%
\begin{pgfscope}%
\pgfpathrectangle{\pgfqpoint{3.985294in}{4.155455in}}{\pgfqpoint{2.279412in}{2.004545in}}%
\pgfusepath{clip}%
\pgfsetbuttcap%
\pgfsetroundjoin%
\pgfsetlinewidth{0.528349pt}%
\definecolor{currentstroke}{rgb}{0.278012,0.180367,0.486697}%
\pgfsetstrokecolor{currentstroke}%
\pgfsetdash{}{0pt}%
\pgfpathmoveto{\pgfqpoint{4.649163in}{5.152866in}}%
\pgfpathlineto{\pgfqpoint{4.649783in}{5.152667in}}%
\pgfusepath{stroke}%
\end{pgfscope}%
\begin{pgfscope}%
\pgfpathrectangle{\pgfqpoint{3.985294in}{4.155455in}}{\pgfqpoint{2.279412in}{2.004545in}}%
\pgfusepath{clip}%
\pgfsetbuttcap%
\pgfsetroundjoin%
\pgfsetlinewidth{0.528492pt}%
\definecolor{currentstroke}{rgb}{0.278012,0.180367,0.486697}%
\pgfsetstrokecolor{currentstroke}%
\pgfsetdash{}{0pt}%
\pgfpathmoveto{\pgfqpoint{4.649783in}{5.152667in}}%
\pgfpathlineto{\pgfqpoint{4.649889in}{5.152627in}}%
\pgfusepath{stroke}%
\end{pgfscope}%
\begin{pgfscope}%
\pgfpathrectangle{\pgfqpoint{3.985294in}{4.155455in}}{\pgfqpoint{2.279412in}{2.004545in}}%
\pgfusepath{clip}%
\pgfsetbuttcap%
\pgfsetroundjoin%
\pgfsetlinewidth{0.528610pt}%
\definecolor{currentstroke}{rgb}{0.278012,0.180367,0.486697}%
\pgfsetstrokecolor{currentstroke}%
\pgfsetdash{}{0pt}%
\pgfpathmoveto{\pgfqpoint{4.649889in}{5.152627in}}%
\pgfpathlineto{\pgfqpoint{4.649363in}{5.152792in}}%
\pgfusepath{stroke}%
\end{pgfscope}%
\begin{pgfscope}%
\pgfpathrectangle{\pgfqpoint{3.985294in}{4.155455in}}{\pgfqpoint{2.279412in}{2.004545in}}%
\pgfusepath{clip}%
\pgfsetbuttcap%
\pgfsetroundjoin%
\pgfsetlinewidth{0.528479pt}%
\definecolor{currentstroke}{rgb}{0.278012,0.180367,0.486697}%
\pgfsetstrokecolor{currentstroke}%
\pgfsetdash{}{0pt}%
\pgfpathmoveto{\pgfqpoint{4.649363in}{5.152792in}}%
\pgfpathlineto{\pgfqpoint{4.648859in}{5.152955in}}%
\pgfusepath{stroke}%
\end{pgfscope}%
\begin{pgfscope}%
\pgfpathrectangle{\pgfqpoint{3.985294in}{4.155455in}}{\pgfqpoint{2.279412in}{2.004545in}}%
\pgfusepath{clip}%
\pgfsetbuttcap%
\pgfsetroundjoin%
\pgfsetlinewidth{0.528435pt}%
\definecolor{currentstroke}{rgb}{0.278012,0.180367,0.486697}%
\pgfsetstrokecolor{currentstroke}%
\pgfsetdash{}{0pt}%
\pgfpathmoveto{\pgfqpoint{4.648859in}{5.152955in}}%
\pgfpathlineto{\pgfqpoint{4.648753in}{5.152993in}}%
\pgfusepath{stroke}%
\end{pgfscope}%
\begin{pgfscope}%
\pgfpathrectangle{\pgfqpoint{3.985294in}{4.155455in}}{\pgfqpoint{2.279412in}{2.004545in}}%
\pgfusepath{clip}%
\pgfsetbuttcap%
\pgfsetroundjoin%
\pgfsetlinewidth{0.528406pt}%
\definecolor{currentstroke}{rgb}{0.278012,0.180367,0.486697}%
\pgfsetstrokecolor{currentstroke}%
\pgfsetdash{}{0pt}%
\pgfpathmoveto{\pgfqpoint{4.648753in}{5.152993in}}%
\pgfpathlineto{\pgfqpoint{4.649149in}{5.152871in}}%
\pgfusepath{stroke}%
\end{pgfscope}%
\begin{pgfscope}%
\pgfpathrectangle{\pgfqpoint{3.985294in}{4.155455in}}{\pgfqpoint{2.279412in}{2.004545in}}%
\pgfusepath{clip}%
\pgfsetbuttcap%
\pgfsetroundjoin%
\pgfsetlinewidth{0.528347pt}%
\definecolor{currentstroke}{rgb}{0.278012,0.180367,0.486697}%
\pgfsetstrokecolor{currentstroke}%
\pgfsetdash{}{0pt}%
\pgfpathmoveto{\pgfqpoint{4.649149in}{5.152871in}}%
\pgfpathlineto{\pgfqpoint{4.649828in}{5.152653in}}%
\pgfusepath{stroke}%
\end{pgfscope}%
\begin{pgfscope}%
\pgfpathrectangle{\pgfqpoint{3.985294in}{4.155455in}}{\pgfqpoint{2.279412in}{2.004545in}}%
\pgfusepath{clip}%
\pgfsetbuttcap%
\pgfsetroundjoin%
\pgfsetlinewidth{0.528508pt}%
\definecolor{currentstroke}{rgb}{0.278012,0.180367,0.486697}%
\pgfsetstrokecolor{currentstroke}%
\pgfsetdash{}{0pt}%
\pgfpathmoveto{\pgfqpoint{4.649828in}{5.152653in}}%
\pgfpathlineto{\pgfqpoint{4.649828in}{5.152653in}}%
\pgfusepath{stroke}%
\end{pgfscope}%
\begin{pgfscope}%
\pgfpathrectangle{\pgfqpoint{3.985294in}{4.155455in}}{\pgfqpoint{2.279412in}{2.004545in}}%
\pgfusepath{clip}%
\pgfsetbuttcap%
\pgfsetroundjoin%
\pgfsetlinewidth{0.528508pt}%
\definecolor{currentstroke}{rgb}{0.278012,0.180367,0.486697}%
\pgfsetstrokecolor{currentstroke}%
\pgfsetdash{}{0pt}%
\pgfpathmoveto{\pgfqpoint{4.649828in}{5.152653in}}%
\pgfpathlineto{\pgfqpoint{4.649293in}{5.152818in}}%
\pgfusepath{stroke}%
\end{pgfscope}%
\begin{pgfscope}%
\pgfpathrectangle{\pgfqpoint{3.985294in}{4.155455in}}{\pgfqpoint{2.279412in}{2.004545in}}%
\pgfusepath{clip}%
\pgfsetbuttcap%
\pgfsetroundjoin%
\pgfsetlinewidth{0.528426pt}%
\definecolor{currentstroke}{rgb}{0.278012,0.180367,0.486697}%
\pgfsetstrokecolor{currentstroke}%
\pgfsetdash{}{0pt}%
\pgfpathmoveto{\pgfqpoint{4.649293in}{5.152818in}}%
\pgfpathlineto{\pgfqpoint{4.648837in}{5.152964in}}%
\pgfusepath{stroke}%
\end{pgfscope}%
\begin{pgfscope}%
\pgfpathrectangle{\pgfqpoint{3.985294in}{4.155455in}}{\pgfqpoint{2.279412in}{2.004545in}}%
\pgfusepath{clip}%
\pgfsetbuttcap%
\pgfsetroundjoin%
\pgfsetlinewidth{0.528412pt}%
\definecolor{currentstroke}{rgb}{0.278012,0.180367,0.486697}%
\pgfsetstrokecolor{currentstroke}%
\pgfsetdash{}{0pt}%
\pgfpathmoveto{\pgfqpoint{4.648837in}{5.152964in}}%
\pgfpathlineto{\pgfqpoint{4.648787in}{5.152984in}}%
\pgfusepath{stroke}%
\end{pgfscope}%
\begin{pgfscope}%
\pgfpathrectangle{\pgfqpoint{3.985294in}{4.155455in}}{\pgfqpoint{2.279412in}{2.004545in}}%
\pgfusepath{clip}%
\pgfsetbuttcap%
\pgfsetroundjoin%
\pgfsetlinewidth{0.528386pt}%
\definecolor{currentstroke}{rgb}{0.278012,0.180367,0.486697}%
\pgfsetstrokecolor{currentstroke}%
\pgfsetdash{}{0pt}%
\pgfpathmoveto{\pgfqpoint{4.648787in}{5.152984in}}%
\pgfpathlineto{\pgfqpoint{4.649223in}{5.152848in}}%
\pgfusepath{stroke}%
\end{pgfscope}%
\begin{pgfscope}%
\pgfpathrectangle{\pgfqpoint{3.985294in}{4.155455in}}{\pgfqpoint{2.279412in}{2.004545in}}%
\pgfusepath{clip}%
\pgfsetbuttcap%
\pgfsetroundjoin%
\pgfsetlinewidth{0.528346pt}%
\definecolor{currentstroke}{rgb}{0.278012,0.180367,0.486697}%
\pgfsetstrokecolor{currentstroke}%
\pgfsetdash{}{0pt}%
\pgfpathmoveto{\pgfqpoint{4.649223in}{5.152848in}}%
\pgfpathlineto{\pgfqpoint{4.649861in}{5.152642in}}%
\pgfusepath{stroke}%
\end{pgfscope}%
\begin{pgfscope}%
\pgfpathrectangle{\pgfqpoint{3.985294in}{4.155455in}}{\pgfqpoint{2.279412in}{2.004545in}}%
\pgfusepath{clip}%
\pgfsetbuttcap%
\pgfsetroundjoin%
\pgfsetlinewidth{0.528526pt}%
\definecolor{currentstroke}{rgb}{0.278012,0.180367,0.486697}%
\pgfsetstrokecolor{currentstroke}%
\pgfsetdash{}{0pt}%
\pgfpathmoveto{\pgfqpoint{4.649861in}{5.152642in}}%
\pgfpathlineto{\pgfqpoint{4.649861in}{5.152642in}}%
\pgfusepath{stroke}%
\end{pgfscope}%
\begin{pgfscope}%
\pgfpathrectangle{\pgfqpoint{3.985294in}{4.155455in}}{\pgfqpoint{2.279412in}{2.004545in}}%
\pgfusepath{clip}%
\pgfsetbuttcap%
\pgfsetroundjoin%
\pgfsetlinewidth{0.528526pt}%
\definecolor{currentstroke}{rgb}{0.278012,0.180367,0.486697}%
\pgfsetstrokecolor{currentstroke}%
\pgfsetdash{}{0pt}%
\pgfpathmoveto{\pgfqpoint{4.649861in}{5.152642in}}%
\pgfpathlineto{\pgfqpoint{4.649287in}{5.152820in}}%
\pgfusepath{stroke}%
\end{pgfscope}%
\begin{pgfscope}%
\pgfpathrectangle{\pgfqpoint{3.985294in}{4.155455in}}{\pgfqpoint{2.279412in}{2.004545in}}%
\pgfusepath{clip}%
\pgfsetbuttcap%
\pgfsetroundjoin%
\pgfsetlinewidth{0.528430pt}%
\definecolor{currentstroke}{rgb}{0.278012,0.180367,0.486697}%
\pgfsetstrokecolor{currentstroke}%
\pgfsetdash{}{0pt}%
\pgfpathmoveto{\pgfqpoint{4.649287in}{5.152820in}}%
\pgfpathlineto{\pgfqpoint{4.648812in}{5.152972in}}%
\pgfusepath{stroke}%
\end{pgfscope}%
\begin{pgfscope}%
\pgfpathrectangle{\pgfqpoint{3.985294in}{4.155455in}}{\pgfqpoint{2.279412in}{2.004545in}}%
\pgfusepath{clip}%
\pgfsetbuttcap%
\pgfsetroundjoin%
\pgfsetlinewidth{0.528417pt}%
\definecolor{currentstroke}{rgb}{0.278012,0.180367,0.486697}%
\pgfsetstrokecolor{currentstroke}%
\pgfsetdash{}{0pt}%
\pgfpathmoveto{\pgfqpoint{4.648812in}{5.152972in}}%
\pgfpathlineto{\pgfqpoint{4.648766in}{5.152990in}}%
\pgfusepath{stroke}%
\end{pgfscope}%
\begin{pgfscope}%
\pgfpathrectangle{\pgfqpoint{3.985294in}{4.155455in}}{\pgfqpoint{2.279412in}{2.004545in}}%
\pgfusepath{clip}%
\pgfsetbuttcap%
\pgfsetroundjoin%
\pgfsetlinewidth{0.528389pt}%
\definecolor{currentstroke}{rgb}{0.278012,0.180367,0.486697}%
\pgfsetstrokecolor{currentstroke}%
\pgfsetdash{}{0pt}%
\pgfpathmoveto{\pgfqpoint{4.648766in}{5.152990in}}%
\pgfpathlineto{\pgfqpoint{4.649232in}{5.152845in}}%
\pgfusepath{stroke}%
\end{pgfscope}%
\begin{pgfscope}%
\pgfpathrectangle{\pgfqpoint{3.985294in}{4.155455in}}{\pgfqpoint{2.279412in}{2.004545in}}%
\pgfusepath{clip}%
\pgfsetbuttcap%
\pgfsetroundjoin%
\pgfsetlinewidth{0.528345pt}%
\definecolor{currentstroke}{rgb}{0.278012,0.180367,0.486697}%
\pgfsetstrokecolor{currentstroke}%
\pgfsetdash{}{0pt}%
\pgfpathmoveto{\pgfqpoint{4.649232in}{5.152845in}}%
\pgfpathlineto{\pgfqpoint{4.649910in}{5.152626in}}%
\pgfusepath{stroke}%
\end{pgfscope}%
\begin{pgfscope}%
\pgfpathrectangle{\pgfqpoint{3.985294in}{4.155455in}}{\pgfqpoint{2.279412in}{2.004545in}}%
\pgfusepath{clip}%
\pgfsetbuttcap%
\pgfsetroundjoin%
\pgfsetlinewidth{0.528548pt}%
\definecolor{currentstroke}{rgb}{0.278012,0.180367,0.486697}%
\pgfsetstrokecolor{currentstroke}%
\pgfsetdash{}{0pt}%
\pgfpathmoveto{\pgfqpoint{4.649910in}{5.152626in}}%
\pgfpathlineto{\pgfqpoint{4.649910in}{5.152626in}}%
\pgfusepath{stroke}%
\end{pgfscope}%
\begin{pgfscope}%
\pgfpathrectangle{\pgfqpoint{3.985294in}{4.155455in}}{\pgfqpoint{2.279412in}{2.004545in}}%
\pgfusepath{clip}%
\pgfsetbuttcap%
\pgfsetroundjoin%
\pgfsetlinewidth{0.528548pt}%
\definecolor{currentstroke}{rgb}{0.278012,0.180367,0.486697}%
\pgfsetstrokecolor{currentstroke}%
\pgfsetdash{}{0pt}%
\pgfpathmoveto{\pgfqpoint{4.649910in}{5.152626in}}%
\pgfpathlineto{\pgfqpoint{4.649291in}{5.152818in}}%
\pgfusepath{stroke}%
\end{pgfscope}%
\begin{pgfscope}%
\pgfpathrectangle{\pgfqpoint{3.985294in}{4.155455in}}{\pgfqpoint{2.279412in}{2.004545in}}%
\pgfusepath{clip}%
\pgfsetbuttcap%
\pgfsetroundjoin%
\pgfsetlinewidth{0.528435pt}%
\definecolor{currentstroke}{rgb}{0.278012,0.180367,0.486697}%
\pgfsetstrokecolor{currentstroke}%
\pgfsetdash{}{0pt}%
\pgfpathmoveto{\pgfqpoint{4.649291in}{5.152818in}}%
\pgfpathlineto{\pgfqpoint{4.648789in}{5.152979in}}%
\pgfusepath{stroke}%
\end{pgfscope}%
\begin{pgfscope}%
\pgfpathrectangle{\pgfqpoint{3.985294in}{4.155455in}}{\pgfqpoint{2.279412in}{2.004545in}}%
\pgfusepath{clip}%
\pgfsetbuttcap%
\pgfsetroundjoin%
\pgfsetlinewidth{0.528423pt}%
\definecolor{currentstroke}{rgb}{0.278012,0.180367,0.486697}%
\pgfsetstrokecolor{currentstroke}%
\pgfsetdash{}{0pt}%
\pgfpathmoveto{\pgfqpoint{4.648789in}{5.152979in}}%
\pgfpathlineto{\pgfqpoint{4.648738in}{5.152999in}}%
\pgfusepath{stroke}%
\end{pgfscope}%
\begin{pgfscope}%
\pgfpathrectangle{\pgfqpoint{3.985294in}{4.155455in}}{\pgfqpoint{2.279412in}{2.004545in}}%
\pgfusepath{clip}%
\pgfsetbuttcap%
\pgfsetroundjoin%
\pgfsetlinewidth{0.528395pt}%
\definecolor{currentstroke}{rgb}{0.278012,0.180367,0.486697}%
\pgfsetstrokecolor{currentstroke}%
\pgfsetdash{}{0pt}%
\pgfpathmoveto{\pgfqpoint{4.648738in}{5.152999in}}%
\pgfpathlineto{\pgfqpoint{4.649230in}{5.152846in}}%
\pgfusepath{stroke}%
\end{pgfscope}%
\begin{pgfscope}%
\pgfpathrectangle{\pgfqpoint{3.985294in}{4.155455in}}{\pgfqpoint{2.279412in}{2.004545in}}%
\pgfusepath{clip}%
\pgfsetbuttcap%
\pgfsetroundjoin%
\pgfsetlinewidth{0.528343pt}%
\definecolor{currentstroke}{rgb}{0.278012,0.180367,0.486697}%
\pgfsetstrokecolor{currentstroke}%
\pgfsetdash{}{0pt}%
\pgfpathmoveto{\pgfqpoint{4.649230in}{5.152846in}}%
\pgfpathlineto{\pgfqpoint{4.649961in}{5.152610in}}%
\pgfusepath{stroke}%
\end{pgfscope}%
\begin{pgfscope}%
\pgfpathrectangle{\pgfqpoint{3.985294in}{4.155455in}}{\pgfqpoint{2.279412in}{2.004545in}}%
\pgfusepath{clip}%
\pgfsetbuttcap%
\pgfsetroundjoin%
\pgfsetlinewidth{0.528570pt}%
\definecolor{currentstroke}{rgb}{0.278012,0.180367,0.486697}%
\pgfsetstrokecolor{currentstroke}%
\pgfsetdash{}{0pt}%
\pgfpathmoveto{\pgfqpoint{4.649961in}{5.152610in}}%
\pgfpathlineto{\pgfqpoint{4.649961in}{5.152610in}}%
\pgfusepath{stroke}%
\end{pgfscope}%
\begin{pgfscope}%
\pgfpathrectangle{\pgfqpoint{3.985294in}{4.155455in}}{\pgfqpoint{2.279412in}{2.004545in}}%
\pgfusepath{clip}%
\pgfsetbuttcap%
\pgfsetroundjoin%
\pgfsetlinewidth{0.528570pt}%
\definecolor{currentstroke}{rgb}{0.278012,0.180367,0.486697}%
\pgfsetstrokecolor{currentstroke}%
\pgfsetdash{}{0pt}%
\pgfpathmoveto{\pgfqpoint{4.649961in}{5.152610in}}%
\pgfpathlineto{\pgfqpoint{4.649297in}{5.152816in}}%
\pgfusepath{stroke}%
\end{pgfscope}%
\begin{pgfscope}%
\pgfpathrectangle{\pgfqpoint{3.985294in}{4.155455in}}{\pgfqpoint{2.279412in}{2.004545in}}%
\pgfusepath{clip}%
\pgfsetbuttcap%
\pgfsetroundjoin%
\pgfsetlinewidth{0.528440pt}%
\definecolor{currentstroke}{rgb}{0.278012,0.180367,0.486697}%
\pgfsetstrokecolor{currentstroke}%
\pgfsetdash{}{0pt}%
\pgfpathmoveto{\pgfqpoint{4.649297in}{5.152816in}}%
\pgfpathlineto{\pgfqpoint{4.648767in}{5.152986in}}%
\pgfusepath{stroke}%
\end{pgfscope}%
\begin{pgfscope}%
\pgfpathrectangle{\pgfqpoint{3.985294in}{4.155455in}}{\pgfqpoint{2.279412in}{2.004545in}}%
\pgfusepath{clip}%
\pgfsetbuttcap%
\pgfsetroundjoin%
\pgfsetlinewidth{0.528429pt}%
\definecolor{currentstroke}{rgb}{0.278012,0.180367,0.486697}%
\pgfsetstrokecolor{currentstroke}%
\pgfsetdash{}{0pt}%
\pgfpathmoveto{\pgfqpoint{4.648767in}{5.152986in}}%
\pgfpathlineto{\pgfqpoint{4.648711in}{5.153007in}}%
\pgfusepath{stroke}%
\end{pgfscope}%
\begin{pgfscope}%
\pgfpathrectangle{\pgfqpoint{3.985294in}{4.155455in}}{\pgfqpoint{2.279412in}{2.004545in}}%
\pgfusepath{clip}%
\pgfsetbuttcap%
\pgfsetroundjoin%
\pgfsetlinewidth{0.528400pt}%
\definecolor{currentstroke}{rgb}{0.278012,0.180367,0.486697}%
\pgfsetstrokecolor{currentstroke}%
\pgfsetdash{}{0pt}%
\pgfpathmoveto{\pgfqpoint{4.648711in}{5.153007in}}%
\pgfpathlineto{\pgfqpoint{4.649224in}{5.152848in}}%
\pgfusepath{stroke}%
\end{pgfscope}%
\begin{pgfscope}%
\pgfpathrectangle{\pgfqpoint{3.985294in}{4.155455in}}{\pgfqpoint{2.279412in}{2.004545in}}%
\pgfusepath{clip}%
\pgfsetbuttcap%
\pgfsetroundjoin%
\pgfsetlinewidth{0.528342pt}%
\definecolor{currentstroke}{rgb}{0.278012,0.180367,0.486697}%
\pgfsetstrokecolor{currentstroke}%
\pgfsetdash{}{0pt}%
\pgfpathmoveto{\pgfqpoint{4.649224in}{5.152848in}}%
\pgfpathlineto{\pgfqpoint{4.650008in}{5.152595in}}%
\pgfusepath{stroke}%
\end{pgfscope}%
\begin{pgfscope}%
\pgfpathrectangle{\pgfqpoint{3.985294in}{4.155455in}}{\pgfqpoint{2.279412in}{2.004545in}}%
\pgfusepath{clip}%
\pgfsetbuttcap%
\pgfsetroundjoin%
\pgfsetlinewidth{0.528593pt}%
\definecolor{currentstroke}{rgb}{0.278012,0.180367,0.486697}%
\pgfsetstrokecolor{currentstroke}%
\pgfsetdash{}{0pt}%
\pgfpathmoveto{\pgfqpoint{4.650008in}{5.152595in}}%
\pgfpathlineto{\pgfqpoint{4.650008in}{5.152595in}}%
\pgfusepath{stroke}%
\end{pgfscope}%
\begin{pgfscope}%
\pgfpathrectangle{\pgfqpoint{3.985294in}{4.155455in}}{\pgfqpoint{2.279412in}{2.004545in}}%
\pgfusepath{clip}%
\pgfsetbuttcap%
\pgfsetroundjoin%
\pgfsetlinewidth{0.528593pt}%
\definecolor{currentstroke}{rgb}{0.278012,0.180367,0.486697}%
\pgfsetstrokecolor{currentstroke}%
\pgfsetdash{}{0pt}%
\pgfpathmoveto{\pgfqpoint{4.650008in}{5.152595in}}%
\pgfpathlineto{\pgfqpoint{4.649304in}{5.152813in}}%
\pgfusepath{stroke}%
\end{pgfscope}%
\begin{pgfscope}%
\pgfpathrectangle{\pgfqpoint{3.985294in}{4.155455in}}{\pgfqpoint{2.279412in}{2.004545in}}%
\pgfusepath{clip}%
\pgfsetbuttcap%
\pgfsetroundjoin%
\pgfsetlinewidth{0.528445pt}%
\definecolor{currentstroke}{rgb}{0.278012,0.180367,0.486697}%
\pgfsetstrokecolor{currentstroke}%
\pgfsetdash{}{0pt}%
\pgfpathmoveto{\pgfqpoint{4.649304in}{5.152813in}}%
\pgfpathlineto{\pgfqpoint{4.648748in}{5.152991in}}%
\pgfusepath{stroke}%
\end{pgfscope}%
\begin{pgfscope}%
\pgfpathrectangle{\pgfqpoint{3.985294in}{4.155455in}}{\pgfqpoint{2.279412in}{2.004545in}}%
\pgfusepath{clip}%
\pgfsetbuttcap%
\pgfsetroundjoin%
\pgfsetlinewidth{0.528435pt}%
\definecolor{currentstroke}{rgb}{0.278012,0.180367,0.486697}%
\pgfsetstrokecolor{currentstroke}%
\pgfsetdash{}{0pt}%
\pgfpathmoveto{\pgfqpoint{4.648748in}{5.152991in}}%
\pgfpathlineto{\pgfqpoint{4.648686in}{5.153015in}}%
\pgfusepath{stroke}%
\end{pgfscope}%
\begin{pgfscope}%
\pgfpathrectangle{\pgfqpoint{3.985294in}{4.155455in}}{\pgfqpoint{2.279412in}{2.004545in}}%
\pgfusepath{clip}%
\pgfsetbuttcap%
\pgfsetroundjoin%
\pgfsetlinewidth{0.528406pt}%
\definecolor{currentstroke}{rgb}{0.278012,0.180367,0.486697}%
\pgfsetstrokecolor{currentstroke}%
\pgfsetdash{}{0pt}%
\pgfpathmoveto{\pgfqpoint{4.648686in}{5.153015in}}%
\pgfpathlineto{\pgfqpoint{4.649219in}{5.152850in}}%
\pgfusepath{stroke}%
\end{pgfscope}%
\begin{pgfscope}%
\pgfpathrectangle{\pgfqpoint{3.985294in}{4.155455in}}{\pgfqpoint{2.279412in}{2.004545in}}%
\pgfusepath{clip}%
\pgfsetbuttcap%
\pgfsetroundjoin%
\pgfsetlinewidth{0.528340pt}%
\definecolor{currentstroke}{rgb}{0.278012,0.180367,0.486697}%
\pgfsetstrokecolor{currentstroke}%
\pgfsetdash{}{0pt}%
\pgfpathmoveto{\pgfqpoint{4.649219in}{5.152850in}}%
\pgfpathlineto{\pgfqpoint{4.650051in}{5.152582in}}%
\pgfusepath{stroke}%
\end{pgfscope}%
\begin{pgfscope}%
\pgfpathrectangle{\pgfqpoint{3.985294in}{4.155455in}}{\pgfqpoint{2.279412in}{2.004545in}}%
\pgfusepath{clip}%
\pgfsetbuttcap%
\pgfsetroundjoin%
\pgfsetlinewidth{0.528614pt}%
\definecolor{currentstroke}{rgb}{0.278012,0.180367,0.486697}%
\pgfsetstrokecolor{currentstroke}%
\pgfsetdash{}{0pt}%
\pgfpathmoveto{\pgfqpoint{4.650051in}{5.152582in}}%
\pgfpathlineto{\pgfqpoint{4.650051in}{5.152582in}}%
\pgfusepath{stroke}%
\end{pgfscope}%
\begin{pgfscope}%
\pgfpathrectangle{\pgfqpoint{3.985294in}{4.155455in}}{\pgfqpoint{2.279412in}{2.004545in}}%
\pgfusepath{clip}%
\pgfsetbuttcap%
\pgfsetroundjoin%
\pgfsetlinewidth{0.528614pt}%
\definecolor{currentstroke}{rgb}{0.278012,0.180367,0.486697}%
\pgfsetstrokecolor{currentstroke}%
\pgfsetdash{}{0pt}%
\pgfpathmoveto{\pgfqpoint{4.650051in}{5.152582in}}%
\pgfpathlineto{\pgfqpoint{4.649311in}{5.152811in}}%
\pgfusepath{stroke}%
\end{pgfscope}%
\begin{pgfscope}%
\pgfpathrectangle{\pgfqpoint{3.985294in}{4.155455in}}{\pgfqpoint{2.279412in}{2.004545in}}%
\pgfusepath{clip}%
\pgfsetbuttcap%
\pgfsetroundjoin%
\pgfsetlinewidth{0.528449pt}%
\definecolor{currentstroke}{rgb}{0.278012,0.180367,0.486697}%
\pgfsetstrokecolor{currentstroke}%
\pgfsetdash{}{0pt}%
\pgfpathmoveto{\pgfqpoint{4.649311in}{5.152811in}}%
\pgfpathlineto{\pgfqpoint{4.648732in}{5.152996in}}%
\pgfusepath{stroke}%
\end{pgfscope}%
\begin{pgfscope}%
\pgfpathrectangle{\pgfqpoint{3.985294in}{4.155455in}}{\pgfqpoint{2.279412in}{2.004545in}}%
\pgfusepath{clip}%
\pgfsetbuttcap%
\pgfsetroundjoin%
\pgfsetlinewidth{0.528440pt}%
\definecolor{currentstroke}{rgb}{0.278012,0.180367,0.486697}%
\pgfsetstrokecolor{currentstroke}%
\pgfsetdash{}{0pt}%
\pgfpathmoveto{\pgfqpoint{4.648732in}{5.152996in}}%
\pgfpathlineto{\pgfqpoint{4.648664in}{5.153022in}}%
\pgfusepath{stroke}%
\end{pgfscope}%
\begin{pgfscope}%
\pgfpathrectangle{\pgfqpoint{3.985294in}{4.155455in}}{\pgfqpoint{2.279412in}{2.004545in}}%
\pgfusepath{clip}%
\pgfsetbuttcap%
\pgfsetroundjoin%
\pgfsetlinewidth{0.528411pt}%
\definecolor{currentstroke}{rgb}{0.278012,0.180367,0.486697}%
\pgfsetstrokecolor{currentstroke}%
\pgfsetdash{}{0pt}%
\pgfpathmoveto{\pgfqpoint{4.648664in}{5.153022in}}%
\pgfpathlineto{\pgfqpoint{4.649213in}{5.152851in}}%
\pgfusepath{stroke}%
\end{pgfscope}%
\begin{pgfscope}%
\pgfpathrectangle{\pgfqpoint{3.985294in}{4.155455in}}{\pgfqpoint{2.279412in}{2.004545in}}%
\pgfusepath{clip}%
\pgfsetbuttcap%
\pgfsetroundjoin%
\pgfsetlinewidth{0.528339pt}%
\definecolor{currentstroke}{rgb}{0.278012,0.180367,0.486697}%
\pgfsetstrokecolor{currentstroke}%
\pgfsetdash{}{0pt}%
\pgfpathmoveto{\pgfqpoint{4.649213in}{5.152851in}}%
\pgfpathlineto{\pgfqpoint{4.650089in}{5.152569in}}%
\pgfusepath{stroke}%
\end{pgfscope}%
\begin{pgfscope}%
\pgfpathrectangle{\pgfqpoint{3.985294in}{4.155455in}}{\pgfqpoint{2.279412in}{2.004545in}}%
\pgfusepath{clip}%
\pgfsetbuttcap%
\pgfsetroundjoin%
\pgfsetlinewidth{0.528633pt}%
\definecolor{currentstroke}{rgb}{0.278012,0.180367,0.486697}%
\pgfsetstrokecolor{currentstroke}%
\pgfsetdash{}{0pt}%
\pgfpathmoveto{\pgfqpoint{4.650089in}{5.152569in}}%
\pgfpathlineto{\pgfqpoint{4.650089in}{5.152569in}}%
\pgfusepath{stroke}%
\end{pgfscope}%
\begin{pgfscope}%
\pgfpathrectangle{\pgfqpoint{3.985294in}{4.155455in}}{\pgfqpoint{2.279412in}{2.004545in}}%
\pgfusepath{clip}%
\pgfsetbuttcap%
\pgfsetroundjoin%
\pgfsetlinewidth{0.528633pt}%
\definecolor{currentstroke}{rgb}{0.278012,0.180367,0.486697}%
\pgfsetstrokecolor{currentstroke}%
\pgfsetdash{}{0pt}%
\pgfpathmoveto{\pgfqpoint{4.650089in}{5.152569in}}%
\pgfpathlineto{\pgfqpoint{4.649317in}{5.152808in}}%
\pgfusepath{stroke}%
\end{pgfscope}%
\begin{pgfscope}%
\pgfpathrectangle{\pgfqpoint{3.985294in}{4.155455in}}{\pgfqpoint{2.279412in}{2.004545in}}%
\pgfusepath{clip}%
\pgfsetbuttcap%
\pgfsetroundjoin%
\pgfsetlinewidth{0.528453pt}%
\definecolor{currentstroke}{rgb}{0.278012,0.180367,0.486697}%
\pgfsetstrokecolor{currentstroke}%
\pgfsetdash{}{0pt}%
\pgfpathmoveto{\pgfqpoint{4.649317in}{5.152808in}}%
\pgfpathlineto{\pgfqpoint{4.648719in}{5.153000in}}%
\pgfusepath{stroke}%
\end{pgfscope}%
\begin{pgfscope}%
\pgfpathrectangle{\pgfqpoint{3.985294in}{4.155455in}}{\pgfqpoint{2.279412in}{2.004545in}}%
\pgfusepath{clip}%
\pgfsetbuttcap%
\pgfsetroundjoin%
\pgfsetlinewidth{0.528445pt}%
\definecolor{currentstroke}{rgb}{0.278012,0.180367,0.486697}%
\pgfsetstrokecolor{currentstroke}%
\pgfsetdash{}{0pt}%
\pgfpathmoveto{\pgfqpoint{4.648719in}{5.153000in}}%
\pgfpathlineto{\pgfqpoint{4.648646in}{5.153028in}}%
\pgfusepath{stroke}%
\end{pgfscope}%
\begin{pgfscope}%
\pgfpathrectangle{\pgfqpoint{3.985294in}{4.155455in}}{\pgfqpoint{2.279412in}{2.004545in}}%
\pgfusepath{clip}%
\pgfsetbuttcap%
\pgfsetroundjoin%
\pgfsetlinewidth{0.528416pt}%
\definecolor{currentstroke}{rgb}{0.278012,0.180367,0.486697}%
\pgfsetstrokecolor{currentstroke}%
\pgfsetdash{}{0pt}%
\pgfpathmoveto{\pgfqpoint{4.648646in}{5.153028in}}%
\pgfpathlineto{\pgfqpoint{4.649207in}{5.152853in}}%
\pgfusepath{stroke}%
\end{pgfscope}%
\begin{pgfscope}%
\pgfpathrectangle{\pgfqpoint{3.985294in}{4.155455in}}{\pgfqpoint{2.279412in}{2.004545in}}%
\pgfusepath{clip}%
\pgfsetbuttcap%
\pgfsetroundjoin%
\pgfsetlinewidth{0.528338pt}%
\definecolor{currentstroke}{rgb}{0.278012,0.180367,0.486697}%
\pgfsetstrokecolor{currentstroke}%
\pgfsetdash{}{0pt}%
\pgfpathmoveto{\pgfqpoint{4.649207in}{5.152853in}}%
\pgfpathlineto{\pgfqpoint{4.649207in}{5.152853in}}%
\pgfusepath{stroke}%
\end{pgfscope}%
\begin{pgfscope}%
\pgfpathrectangle{\pgfqpoint{3.985294in}{4.155455in}}{\pgfqpoint{2.279412in}{2.004545in}}%
\pgfusepath{clip}%
\pgfsetbuttcap%
\pgfsetroundjoin%
\pgfsetlinewidth{0.528338pt}%
\definecolor{currentstroke}{rgb}{0.278012,0.180367,0.486697}%
\pgfsetstrokecolor{currentstroke}%
\pgfsetdash{}{0pt}%
\pgfpathmoveto{\pgfqpoint{4.649207in}{5.152853in}}%
\pgfpathlineto{\pgfqpoint{4.649352in}{5.152805in}}%
\pgfusepath{stroke}%
\end{pgfscope}%
\begin{pgfscope}%
\pgfpathrectangle{\pgfqpoint{3.985294in}{4.155455in}}{\pgfqpoint{2.279412in}{2.004545in}}%
\pgfusepath{clip}%
\pgfsetbuttcap%
\pgfsetroundjoin%
\pgfsetlinewidth{0.528373pt}%
\definecolor{currentstroke}{rgb}{0.278012,0.180367,0.486697}%
\pgfsetstrokecolor{currentstroke}%
\pgfsetdash{}{0pt}%
\pgfpathmoveto{\pgfqpoint{4.649352in}{5.152805in}}%
\pgfpathlineto{\pgfqpoint{4.649404in}{5.152787in}}%
\pgfusepath{stroke}%
\end{pgfscope}%
\begin{pgfscope}%
\pgfpathrectangle{\pgfqpoint{3.985294in}{4.155455in}}{\pgfqpoint{2.279412in}{2.004545in}}%
\pgfusepath{clip}%
\pgfsetbuttcap%
\pgfsetroundjoin%
\pgfsetlinewidth{0.528404pt}%
\definecolor{currentstroke}{rgb}{0.278012,0.180367,0.486697}%
\pgfsetstrokecolor{currentstroke}%
\pgfsetdash{}{0pt}%
\pgfpathmoveto{\pgfqpoint{4.649404in}{5.152787in}}%
\pgfpathlineto{\pgfqpoint{4.649303in}{5.152818in}}%
\pgfusepath{stroke}%
\end{pgfscope}%
\begin{pgfscope}%
\pgfpathrectangle{\pgfqpoint{3.985294in}{4.155455in}}{\pgfqpoint{2.279412in}{2.004545in}}%
\pgfusepath{clip}%
\pgfsetbuttcap%
\pgfsetroundjoin%
\pgfsetlinewidth{0.528402pt}%
\definecolor{currentstroke}{rgb}{0.278012,0.180367,0.486697}%
\pgfsetstrokecolor{currentstroke}%
\pgfsetdash{}{0pt}%
\pgfpathmoveto{\pgfqpoint{4.649303in}{5.152818in}}%
\pgfpathlineto{\pgfqpoint{4.649155in}{5.152865in}}%
\pgfusepath{stroke}%
\end{pgfscope}%
\begin{pgfscope}%
\pgfpathrectangle{\pgfqpoint{3.985294in}{4.155455in}}{\pgfqpoint{2.279412in}{2.004545in}}%
\pgfusepath{clip}%
\pgfsetbuttcap%
\pgfsetroundjoin%
\pgfsetlinewidth{0.528388pt}%
\definecolor{currentstroke}{rgb}{0.278012,0.180367,0.486697}%
\pgfsetstrokecolor{currentstroke}%
\pgfsetdash{}{0pt}%
\pgfpathmoveto{\pgfqpoint{4.649155in}{5.152865in}}%
\pgfpathlineto{\pgfqpoint{4.649104in}{5.152882in}}%
\pgfusepath{stroke}%
\end{pgfscope}%
\begin{pgfscope}%
\pgfpathrectangle{\pgfqpoint{3.985294in}{4.155455in}}{\pgfqpoint{2.279412in}{2.004545in}}%
\pgfusepath{clip}%
\pgfsetbuttcap%
\pgfsetroundjoin%
\pgfsetlinewidth{0.528375pt}%
\definecolor{currentstroke}{rgb}{0.278012,0.180367,0.486697}%
\pgfsetstrokecolor{currentstroke}%
\pgfsetdash{}{0pt}%
\pgfpathmoveto{\pgfqpoint{4.649104in}{5.152882in}}%
\pgfpathlineto{\pgfqpoint{4.649200in}{5.152852in}}%
\pgfusepath{stroke}%
\end{pgfscope}%
\begin{pgfscope}%
\pgfpathrectangle{\pgfqpoint{3.985294in}{4.155455in}}{\pgfqpoint{2.279412in}{2.004545in}}%
\pgfusepath{clip}%
\pgfsetbuttcap%
\pgfsetroundjoin%
\pgfsetlinewidth{0.528371pt}%
\definecolor{currentstroke}{rgb}{0.278012,0.180367,0.486697}%
\pgfsetstrokecolor{currentstroke}%
\pgfsetdash{}{0pt}%
\pgfpathmoveto{\pgfqpoint{4.649200in}{5.152852in}}%
\pgfpathlineto{\pgfqpoint{4.649364in}{5.152800in}}%
\pgfusepath{stroke}%
\end{pgfscope}%
\begin{pgfscope}%
\pgfpathrectangle{\pgfqpoint{3.985294in}{4.155455in}}{\pgfqpoint{2.279412in}{2.004545in}}%
\pgfusepath{clip}%
\pgfsetbuttcap%
\pgfsetroundjoin%
\pgfsetlinewidth{0.528392pt}%
\definecolor{currentstroke}{rgb}{0.278012,0.180367,0.486697}%
\pgfsetstrokecolor{currentstroke}%
\pgfsetdash{}{0pt}%
\pgfpathmoveto{\pgfqpoint{4.649364in}{5.152800in}}%
\pgfpathlineto{\pgfqpoint{4.649424in}{5.152779in}}%
\pgfusepath{stroke}%
\end{pgfscope}%
\begin{pgfscope}%
\pgfpathrectangle{\pgfqpoint{3.985294in}{4.155455in}}{\pgfqpoint{2.279412in}{2.004545in}}%
\pgfusepath{clip}%
\pgfsetbuttcap%
\pgfsetroundjoin%
\pgfsetlinewidth{0.528417pt}%
\definecolor{currentstroke}{rgb}{0.278012,0.180367,0.486697}%
\pgfsetstrokecolor{currentstroke}%
\pgfsetdash{}{0pt}%
\pgfpathmoveto{\pgfqpoint{4.649424in}{5.152779in}}%
\pgfpathlineto{\pgfqpoint{4.649309in}{5.152815in}}%
\pgfusepath{stroke}%
\end{pgfscope}%
\begin{pgfscope}%
\pgfpathrectangle{\pgfqpoint{3.985294in}{4.155455in}}{\pgfqpoint{2.279412in}{2.004545in}}%
\pgfusepath{clip}%
\pgfsetbuttcap%
\pgfsetroundjoin%
\pgfsetlinewidth{0.528409pt}%
\definecolor{currentstroke}{rgb}{0.278012,0.180367,0.486697}%
\pgfsetstrokecolor{currentstroke}%
\pgfsetdash{}{0pt}%
\pgfpathmoveto{\pgfqpoint{4.649309in}{5.152815in}}%
\pgfpathlineto{\pgfqpoint{4.649144in}{5.152868in}}%
\pgfusepath{stroke}%
\end{pgfscope}%
\begin{pgfscope}%
\pgfpathrectangle{\pgfqpoint{3.985294in}{4.155455in}}{\pgfqpoint{2.279412in}{2.004545in}}%
\pgfusepath{clip}%
\pgfsetbuttcap%
\pgfsetroundjoin%
\pgfsetlinewidth{0.528391pt}%
\definecolor{currentstroke}{rgb}{0.278012,0.180367,0.486697}%
\pgfsetstrokecolor{currentstroke}%
\pgfsetdash{}{0pt}%
\pgfpathmoveto{\pgfqpoint{4.649144in}{5.152868in}}%
\pgfpathlineto{\pgfqpoint{4.649086in}{5.152888in}}%
\pgfusepath{stroke}%
\end{pgfscope}%
\begin{pgfscope}%
\pgfpathrectangle{\pgfqpoint{3.985294in}{4.155455in}}{\pgfqpoint{2.279412in}{2.004545in}}%
\pgfusepath{clip}%
\pgfsetbuttcap%
\pgfsetroundjoin%
\pgfsetlinewidth{0.528376pt}%
\definecolor{currentstroke}{rgb}{0.278012,0.180367,0.486697}%
\pgfsetstrokecolor{currentstroke}%
\pgfsetdash{}{0pt}%
\pgfpathmoveto{\pgfqpoint{4.649086in}{5.152888in}}%
\pgfpathlineto{\pgfqpoint{4.649194in}{5.152854in}}%
\pgfusepath{stroke}%
\end{pgfscope}%
\begin{pgfscope}%
\pgfpathrectangle{\pgfqpoint{3.985294in}{4.155455in}}{\pgfqpoint{2.279412in}{2.004545in}}%
\pgfusepath{clip}%
\pgfsetbuttcap%
\pgfsetroundjoin%
\pgfsetlinewidth{0.528369pt}%
\definecolor{currentstroke}{rgb}{0.278012,0.180367,0.486697}%
\pgfsetstrokecolor{currentstroke}%
\pgfsetdash{}{0pt}%
\pgfpathmoveto{\pgfqpoint{4.649194in}{5.152854in}}%
\pgfpathlineto{\pgfqpoint{4.649379in}{5.152795in}}%
\pgfusepath{stroke}%
\end{pgfscope}%
\begin{pgfscope}%
\pgfpathrectangle{\pgfqpoint{3.985294in}{4.155455in}}{\pgfqpoint{2.279412in}{2.004545in}}%
\pgfusepath{clip}%
\pgfsetbuttcap%
\pgfsetroundjoin%
\pgfsetlinewidth{0.528393pt}%
\definecolor{currentstroke}{rgb}{0.278012,0.180367,0.486697}%
\pgfsetstrokecolor{currentstroke}%
\pgfsetdash{}{0pt}%
\pgfpathmoveto{\pgfqpoint{4.649379in}{5.152795in}}%
\pgfpathlineto{\pgfqpoint{4.649446in}{5.152772in}}%
\pgfusepath{stroke}%
\end{pgfscope}%
\begin{pgfscope}%
\pgfpathrectangle{\pgfqpoint{3.985294in}{4.155455in}}{\pgfqpoint{2.279412in}{2.004545in}}%
\pgfusepath{clip}%
\pgfsetbuttcap%
\pgfsetroundjoin%
\pgfsetlinewidth{0.528422pt}%
\definecolor{currentstroke}{rgb}{0.278012,0.180367,0.486697}%
\pgfsetstrokecolor{currentstroke}%
\pgfsetdash{}{0pt}%
\pgfpathmoveto{\pgfqpoint{4.649446in}{5.152772in}}%
\pgfpathlineto{\pgfqpoint{4.649316in}{5.152812in}}%
\pgfusepath{stroke}%
\end{pgfscope}%
\begin{pgfscope}%
\pgfpathrectangle{\pgfqpoint{3.985294in}{4.155455in}}{\pgfqpoint{2.279412in}{2.004545in}}%
\pgfusepath{clip}%
\pgfsetbuttcap%
\pgfsetroundjoin%
\pgfsetlinewidth{0.528413pt}%
\definecolor{currentstroke}{rgb}{0.278012,0.180367,0.486697}%
\pgfsetstrokecolor{currentstroke}%
\pgfsetdash{}{0pt}%
\pgfpathmoveto{\pgfqpoint{4.649316in}{5.152812in}}%
\pgfpathlineto{\pgfqpoint{4.649131in}{5.152872in}}%
\pgfusepath{stroke}%
\end{pgfscope}%
\begin{pgfscope}%
\pgfpathrectangle{\pgfqpoint{3.985294in}{4.155455in}}{\pgfqpoint{2.279412in}{2.004545in}}%
\pgfusepath{clip}%
\pgfsetbuttcap%
\pgfsetroundjoin%
\pgfsetlinewidth{0.528392pt}%
\definecolor{currentstroke}{rgb}{0.278012,0.180367,0.486697}%
\pgfsetstrokecolor{currentstroke}%
\pgfsetdash{}{0pt}%
\pgfpathmoveto{\pgfqpoint{4.649131in}{5.152872in}}%
\pgfpathlineto{\pgfqpoint{4.649066in}{5.152894in}}%
\pgfusepath{stroke}%
\end{pgfscope}%
\begin{pgfscope}%
\pgfpathrectangle{\pgfqpoint{3.985294in}{4.155455in}}{\pgfqpoint{2.279412in}{2.004545in}}%
\pgfusepath{clip}%
\pgfsetbuttcap%
\pgfsetroundjoin%
\pgfsetlinewidth{0.528376pt}%
\definecolor{currentstroke}{rgb}{0.278012,0.180367,0.486697}%
\pgfsetstrokecolor{currentstroke}%
\pgfsetdash{}{0pt}%
\pgfpathmoveto{\pgfqpoint{4.649066in}{5.152894in}}%
\pgfpathlineto{\pgfqpoint{4.649187in}{5.152857in}}%
\pgfusepath{stroke}%
\end{pgfscope}%
\begin{pgfscope}%
\pgfpathrectangle{\pgfqpoint{3.985294in}{4.155455in}}{\pgfqpoint{2.279412in}{2.004545in}}%
\pgfusepath{clip}%
\pgfsetbuttcap%
\pgfsetroundjoin%
\pgfsetlinewidth{0.528368pt}%
\definecolor{currentstroke}{rgb}{0.278012,0.180367,0.486697}%
\pgfsetstrokecolor{currentstroke}%
\pgfsetdash{}{0pt}%
\pgfpathmoveto{\pgfqpoint{4.649187in}{5.152857in}}%
\pgfpathlineto{\pgfqpoint{4.649396in}{5.152790in}}%
\pgfusepath{stroke}%
\end{pgfscope}%
\begin{pgfscope}%
\pgfpathrectangle{\pgfqpoint{3.985294in}{4.155455in}}{\pgfqpoint{2.279412in}{2.004545in}}%
\pgfusepath{clip}%
\pgfsetbuttcap%
\pgfsetroundjoin%
\pgfsetlinewidth{0.528395pt}%
\definecolor{currentstroke}{rgb}{0.278012,0.180367,0.486697}%
\pgfsetstrokecolor{currentstroke}%
\pgfsetdash{}{0pt}%
\pgfpathmoveto{\pgfqpoint{4.649396in}{5.152790in}}%
\pgfpathlineto{\pgfqpoint{4.649472in}{5.152764in}}%
\pgfusepath{stroke}%
\end{pgfscope}%
\begin{pgfscope}%
\pgfpathrectangle{\pgfqpoint{3.985294in}{4.155455in}}{\pgfqpoint{2.279412in}{2.004545in}}%
\pgfusepath{clip}%
\pgfsetbuttcap%
\pgfsetroundjoin%
\pgfsetlinewidth{0.528428pt}%
\definecolor{currentstroke}{rgb}{0.278012,0.180367,0.486697}%
\pgfsetstrokecolor{currentstroke}%
\pgfsetdash{}{0pt}%
\pgfpathmoveto{\pgfqpoint{4.649472in}{5.152764in}}%
\pgfpathlineto{\pgfqpoint{4.649324in}{5.152810in}}%
\pgfusepath{stroke}%
\end{pgfscope}%
\begin{pgfscope}%
\pgfpathrectangle{\pgfqpoint{3.985294in}{4.155455in}}{\pgfqpoint{2.279412in}{2.004545in}}%
\pgfusepath{clip}%
\pgfsetbuttcap%
\pgfsetroundjoin%
\pgfsetlinewidth{0.528416pt}%
\definecolor{currentstroke}{rgb}{0.278012,0.180367,0.486697}%
\pgfsetstrokecolor{currentstroke}%
\pgfsetdash{}{0pt}%
\pgfpathmoveto{\pgfqpoint{4.649324in}{5.152810in}}%
\pgfpathlineto{\pgfqpoint{4.649117in}{5.152876in}}%
\pgfusepath{stroke}%
\end{pgfscope}%
\begin{pgfscope}%
\pgfpathrectangle{\pgfqpoint{3.985294in}{4.155455in}}{\pgfqpoint{2.279412in}{2.004545in}}%
\pgfusepath{clip}%
\pgfsetbuttcap%
\pgfsetroundjoin%
\pgfsetlinewidth{0.528394pt}%
\definecolor{currentstroke}{rgb}{0.278012,0.180367,0.486697}%
\pgfsetstrokecolor{currentstroke}%
\pgfsetdash{}{0pt}%
\pgfpathmoveto{\pgfqpoint{4.649117in}{5.152876in}}%
\pgfpathlineto{\pgfqpoint{4.649045in}{5.152901in}}%
\pgfusepath{stroke}%
\end{pgfscope}%
\begin{pgfscope}%
\pgfpathrectangle{\pgfqpoint{3.985294in}{4.155455in}}{\pgfqpoint{2.279412in}{2.004545in}}%
\pgfusepath{clip}%
\pgfsetbuttcap%
\pgfsetroundjoin%
\pgfsetlinewidth{0.528376pt}%
\definecolor{currentstroke}{rgb}{0.278012,0.180367,0.486697}%
\pgfsetstrokecolor{currentstroke}%
\pgfsetdash{}{0pt}%
\pgfpathmoveto{\pgfqpoint{4.649045in}{5.152901in}}%
\pgfpathlineto{\pgfqpoint{4.649179in}{5.152859in}}%
\pgfusepath{stroke}%
\end{pgfscope}%
\begin{pgfscope}%
\pgfpathrectangle{\pgfqpoint{3.985294in}{4.155455in}}{\pgfqpoint{2.279412in}{2.004545in}}%
\pgfusepath{clip}%
\pgfsetbuttcap%
\pgfsetroundjoin%
\pgfsetlinewidth{0.528366pt}%
\definecolor{currentstroke}{rgb}{0.278012,0.180367,0.486697}%
\pgfsetstrokecolor{currentstroke}%
\pgfsetdash{}{0pt}%
\pgfpathmoveto{\pgfqpoint{4.649179in}{5.152859in}}%
\pgfpathlineto{\pgfqpoint{4.649414in}{5.152784in}}%
\pgfusepath{stroke}%
\end{pgfscope}%
\begin{pgfscope}%
\pgfpathrectangle{\pgfqpoint{3.985294in}{4.155455in}}{\pgfqpoint{2.279412in}{2.004545in}}%
\pgfusepath{clip}%
\pgfsetbuttcap%
\pgfsetroundjoin%
\pgfsetlinewidth{0.528397pt}%
\definecolor{currentstroke}{rgb}{0.278012,0.180367,0.486697}%
\pgfsetstrokecolor{currentstroke}%
\pgfsetdash{}{0pt}%
\pgfpathmoveto{\pgfqpoint{4.649414in}{5.152784in}}%
\pgfpathlineto{\pgfqpoint{4.649500in}{5.152755in}}%
\pgfusepath{stroke}%
\end{pgfscope}%
\begin{pgfscope}%
\pgfpathrectangle{\pgfqpoint{3.985294in}{4.155455in}}{\pgfqpoint{2.279412in}{2.004545in}}%
\pgfusepath{clip}%
\pgfsetbuttcap%
\pgfsetroundjoin%
\pgfsetlinewidth{0.528436pt}%
\definecolor{currentstroke}{rgb}{0.278012,0.180367,0.486697}%
\pgfsetstrokecolor{currentstroke}%
\pgfsetdash{}{0pt}%
\pgfpathmoveto{\pgfqpoint{4.649500in}{5.152755in}}%
\pgfpathlineto{\pgfqpoint{4.649334in}{5.152806in}}%
\pgfusepath{stroke}%
\end{pgfscope}%
\begin{pgfscope}%
\pgfpathrectangle{\pgfqpoint{3.985294in}{4.155455in}}{\pgfqpoint{2.279412in}{2.004545in}}%
\pgfusepath{clip}%
\pgfsetbuttcap%
\pgfsetroundjoin%
\pgfsetlinewidth{0.528421pt}%
\definecolor{currentstroke}{rgb}{0.278012,0.180367,0.486697}%
\pgfsetstrokecolor{currentstroke}%
\pgfsetdash{}{0pt}%
\pgfpathmoveto{\pgfqpoint{4.649334in}{5.152806in}}%
\pgfpathlineto{\pgfqpoint{4.649101in}{5.152881in}}%
\pgfusepath{stroke}%
\end{pgfscope}%
\begin{pgfscope}%
\pgfpathrectangle{\pgfqpoint{3.985294in}{4.155455in}}{\pgfqpoint{2.279412in}{2.004545in}}%
\pgfusepath{clip}%
\pgfsetbuttcap%
\pgfsetroundjoin%
\pgfsetlinewidth{0.528395pt}%
\definecolor{currentstroke}{rgb}{0.278012,0.180367,0.486697}%
\pgfsetstrokecolor{currentstroke}%
\pgfsetdash{}{0pt}%
\pgfpathmoveto{\pgfqpoint{4.649101in}{5.152881in}}%
\pgfpathlineto{\pgfqpoint{4.649021in}{5.152908in}}%
\pgfusepath{stroke}%
\end{pgfscope}%
\begin{pgfscope}%
\pgfpathrectangle{\pgfqpoint{3.985294in}{4.155455in}}{\pgfqpoint{2.279412in}{2.004545in}}%
\pgfusepath{clip}%
\pgfsetbuttcap%
\pgfsetroundjoin%
\pgfsetlinewidth{0.528377pt}%
\definecolor{currentstroke}{rgb}{0.278012,0.180367,0.486697}%
\pgfsetstrokecolor{currentstroke}%
\pgfsetdash{}{0pt}%
\pgfpathmoveto{\pgfqpoint{4.649021in}{5.152908in}}%
\pgfpathlineto{\pgfqpoint{4.649170in}{5.152862in}}%
\pgfusepath{stroke}%
\end{pgfscope}%
\begin{pgfscope}%
\pgfpathrectangle{\pgfqpoint{3.985294in}{4.155455in}}{\pgfqpoint{2.279412in}{2.004545in}}%
\pgfusepath{clip}%
\pgfsetbuttcap%
\pgfsetroundjoin%
\pgfsetlinewidth{0.528364pt}%
\definecolor{currentstroke}{rgb}{0.278012,0.180367,0.486697}%
\pgfsetstrokecolor{currentstroke}%
\pgfsetdash{}{0pt}%
\pgfpathmoveto{\pgfqpoint{4.649170in}{5.152862in}}%
\pgfpathlineto{\pgfqpoint{4.649435in}{5.152778in}}%
\pgfusepath{stroke}%
\end{pgfscope}%
\begin{pgfscope}%
\pgfpathrectangle{\pgfqpoint{3.985294in}{4.155455in}}{\pgfqpoint{2.279412in}{2.004545in}}%
\pgfusepath{clip}%
\pgfsetbuttcap%
\pgfsetroundjoin%
\pgfsetlinewidth{0.528399pt}%
\definecolor{currentstroke}{rgb}{0.278012,0.180367,0.486697}%
\pgfsetstrokecolor{currentstroke}%
\pgfsetdash{}{0pt}%
\pgfpathmoveto{\pgfqpoint{4.649435in}{5.152778in}}%
\pgfpathlineto{\pgfqpoint{4.649533in}{5.152744in}}%
\pgfusepath{stroke}%
\end{pgfscope}%
\begin{pgfscope}%
\pgfpathrectangle{\pgfqpoint{3.985294in}{4.155455in}}{\pgfqpoint{2.279412in}{2.004545in}}%
\pgfusepath{clip}%
\pgfsetbuttcap%
\pgfsetroundjoin%
\pgfsetlinewidth{0.528445pt}%
\definecolor{currentstroke}{rgb}{0.278012,0.180367,0.486697}%
\pgfsetstrokecolor{currentstroke}%
\pgfsetdash{}{0pt}%
\pgfpathmoveto{\pgfqpoint{4.649533in}{5.152744in}}%
\pgfpathlineto{\pgfqpoint{4.649344in}{5.152803in}}%
\pgfusepath{stroke}%
\end{pgfscope}%
\begin{pgfscope}%
\pgfpathrectangle{\pgfqpoint{3.985294in}{4.155455in}}{\pgfqpoint{2.279412in}{2.004545in}}%
\pgfusepath{clip}%
\pgfsetbuttcap%
\pgfsetroundjoin%
\pgfsetlinewidth{0.528426pt}%
\definecolor{currentstroke}{rgb}{0.278012,0.180367,0.486697}%
\pgfsetstrokecolor{currentstroke}%
\pgfsetdash{}{0pt}%
\pgfpathmoveto{\pgfqpoint{4.649344in}{5.152803in}}%
\pgfpathlineto{\pgfqpoint{4.649085in}{5.152886in}}%
\pgfusepath{stroke}%
\end{pgfscope}%
\begin{pgfscope}%
\pgfpathrectangle{\pgfqpoint{3.985294in}{4.155455in}}{\pgfqpoint{2.279412in}{2.004545in}}%
\pgfusepath{clip}%
\pgfsetbuttcap%
\pgfsetroundjoin%
\pgfsetlinewidth{0.528397pt}%
\definecolor{currentstroke}{rgb}{0.278012,0.180367,0.486697}%
\pgfsetstrokecolor{currentstroke}%
\pgfsetdash{}{0pt}%
\pgfpathmoveto{\pgfqpoint{4.649085in}{5.152886in}}%
\pgfpathlineto{\pgfqpoint{4.648995in}{5.152916in}}%
\pgfusepath{stroke}%
\end{pgfscope}%
\begin{pgfscope}%
\pgfpathrectangle{\pgfqpoint{3.985294in}{4.155455in}}{\pgfqpoint{2.279412in}{2.004545in}}%
\pgfusepath{clip}%
\pgfsetbuttcap%
\pgfsetroundjoin%
\pgfsetlinewidth{0.528378pt}%
\definecolor{currentstroke}{rgb}{0.278012,0.180367,0.486697}%
\pgfsetstrokecolor{currentstroke}%
\pgfsetdash{}{0pt}%
\pgfpathmoveto{\pgfqpoint{4.648995in}{5.152916in}}%
\pgfpathlineto{\pgfqpoint{4.649159in}{5.152866in}}%
\pgfusepath{stroke}%
\end{pgfscope}%
\begin{pgfscope}%
\pgfpathrectangle{\pgfqpoint{3.985294in}{4.155455in}}{\pgfqpoint{2.279412in}{2.004545in}}%
\pgfusepath{clip}%
\pgfsetbuttcap%
\pgfsetroundjoin%
\pgfsetlinewidth{0.528362pt}%
\definecolor{currentstroke}{rgb}{0.278012,0.180367,0.486697}%
\pgfsetstrokecolor{currentstroke}%
\pgfsetdash{}{0pt}%
\pgfpathmoveto{\pgfqpoint{4.649159in}{5.152866in}}%
\pgfpathlineto{\pgfqpoint{4.649458in}{5.152771in}}%
\pgfusepath{stroke}%
\end{pgfscope}%
\begin{pgfscope}%
\pgfpathrectangle{\pgfqpoint{3.985294in}{4.155455in}}{\pgfqpoint{2.279412in}{2.004545in}}%
\pgfusepath{clip}%
\pgfsetbuttcap%
\pgfsetroundjoin%
\pgfsetlinewidth{0.528402pt}%
\definecolor{currentstroke}{rgb}{0.278012,0.180367,0.486697}%
\pgfsetstrokecolor{currentstroke}%
\pgfsetdash{}{0pt}%
\pgfpathmoveto{\pgfqpoint{4.649458in}{5.152771in}}%
\pgfpathlineto{\pgfqpoint{4.649569in}{5.152733in}}%
\pgfusepath{stroke}%
\end{pgfscope}%
\begin{pgfscope}%
\pgfpathrectangle{\pgfqpoint{3.985294in}{4.155455in}}{\pgfqpoint{2.279412in}{2.004545in}}%
\pgfusepath{clip}%
\pgfsetbuttcap%
\pgfsetroundjoin%
\pgfsetlinewidth{0.528457pt}%
\definecolor{currentstroke}{rgb}{0.278012,0.180367,0.486697}%
\pgfsetstrokecolor{currentstroke}%
\pgfsetdash{}{0pt}%
\pgfpathmoveto{\pgfqpoint{4.649569in}{5.152733in}}%
\pgfpathlineto{\pgfqpoint{4.649356in}{5.152799in}}%
\pgfusepath{stroke}%
\end{pgfscope}%
\begin{pgfscope}%
\pgfpathrectangle{\pgfqpoint{3.985294in}{4.155455in}}{\pgfqpoint{2.279412in}{2.004545in}}%
\pgfusepath{clip}%
\pgfsetbuttcap%
\pgfsetroundjoin%
\pgfsetlinewidth{0.528432pt}%
\definecolor{currentstroke}{rgb}{0.278012,0.180367,0.486697}%
\pgfsetstrokecolor{currentstroke}%
\pgfsetdash{}{0pt}%
\pgfpathmoveto{\pgfqpoint{4.649356in}{5.152799in}}%
\pgfpathlineto{\pgfqpoint{4.649067in}{5.152891in}}%
\pgfusepath{stroke}%
\end{pgfscope}%
\begin{pgfscope}%
\pgfpathrectangle{\pgfqpoint{3.985294in}{4.155455in}}{\pgfqpoint{2.279412in}{2.004545in}}%
\pgfusepath{clip}%
\pgfsetbuttcap%
\pgfsetroundjoin%
\pgfsetlinewidth{0.528400pt}%
\definecolor{currentstroke}{rgb}{0.278012,0.180367,0.486697}%
\pgfsetstrokecolor{currentstroke}%
\pgfsetdash{}{0pt}%
\pgfpathmoveto{\pgfqpoint{4.649067in}{5.152891in}}%
\pgfpathlineto{\pgfqpoint{4.648967in}{5.152925in}}%
\pgfusepath{stroke}%
\end{pgfscope}%
\begin{pgfscope}%
\pgfpathrectangle{\pgfqpoint{3.985294in}{4.155455in}}{\pgfqpoint{2.279412in}{2.004545in}}%
\pgfusepath{clip}%
\pgfsetbuttcap%
\pgfsetroundjoin%
\pgfsetlinewidth{0.528380pt}%
\definecolor{currentstroke}{rgb}{0.278012,0.180367,0.486697}%
\pgfsetstrokecolor{currentstroke}%
\pgfsetdash{}{0pt}%
\pgfpathmoveto{\pgfqpoint{4.648967in}{5.152925in}}%
\pgfpathlineto{\pgfqpoint{4.649148in}{5.152870in}}%
\pgfusepath{stroke}%
\end{pgfscope}%
\begin{pgfscope}%
\pgfpathrectangle{\pgfqpoint{3.985294in}{4.155455in}}{\pgfqpoint{2.279412in}{2.004545in}}%
\pgfusepath{clip}%
\pgfsetbuttcap%
\pgfsetroundjoin%
\pgfsetlinewidth{0.528360pt}%
\definecolor{currentstroke}{rgb}{0.278012,0.180367,0.486697}%
\pgfsetstrokecolor{currentstroke}%
\pgfsetdash{}{0pt}%
\pgfpathmoveto{\pgfqpoint{4.649148in}{5.152870in}}%
\pgfpathlineto{\pgfqpoint{4.649483in}{5.152763in}}%
\pgfusepath{stroke}%
\end{pgfscope}%
\begin{pgfscope}%
\pgfpathrectangle{\pgfqpoint{3.985294in}{4.155455in}}{\pgfqpoint{2.279412in}{2.004545in}}%
\pgfusepath{clip}%
\pgfsetbuttcap%
\pgfsetroundjoin%
\pgfsetlinewidth{0.528406pt}%
\definecolor{currentstroke}{rgb}{0.278012,0.180367,0.486697}%
\pgfsetstrokecolor{currentstroke}%
\pgfsetdash{}{0pt}%
\pgfpathmoveto{\pgfqpoint{4.649483in}{5.152763in}}%
\pgfpathlineto{\pgfqpoint{4.649609in}{5.152720in}}%
\pgfusepath{stroke}%
\end{pgfscope}%
\begin{pgfscope}%
\pgfpathrectangle{\pgfqpoint{3.985294in}{4.155455in}}{\pgfqpoint{2.279412in}{2.004545in}}%
\pgfusepath{clip}%
\pgfsetbuttcap%
\pgfsetroundjoin%
\pgfsetlinewidth{0.528470pt}%
\definecolor{currentstroke}{rgb}{0.278012,0.180367,0.486697}%
\pgfsetstrokecolor{currentstroke}%
\pgfsetdash{}{0pt}%
\pgfpathmoveto{\pgfqpoint{4.649609in}{5.152720in}}%
\pgfpathlineto{\pgfqpoint{4.649369in}{5.152794in}}%
\pgfusepath{stroke}%
\end{pgfscope}%
\begin{pgfscope}%
\pgfpathrectangle{\pgfqpoint{3.985294in}{4.155455in}}{\pgfqpoint{2.279412in}{2.004545in}}%
\pgfusepath{clip}%
\pgfsetbuttcap%
\pgfsetroundjoin%
\pgfsetlinewidth{0.528438pt}%
\definecolor{currentstroke}{rgb}{0.278012,0.180367,0.486697}%
\pgfsetstrokecolor{currentstroke}%
\pgfsetdash{}{0pt}%
\pgfpathmoveto{\pgfqpoint{4.649369in}{5.152794in}}%
\pgfpathlineto{\pgfqpoint{4.649049in}{5.152897in}}%
\pgfusepath{stroke}%
\end{pgfscope}%
\begin{pgfscope}%
\pgfpathrectangle{\pgfqpoint{3.985294in}{4.155455in}}{\pgfqpoint{2.279412in}{2.004545in}}%
\pgfusepath{clip}%
\pgfsetbuttcap%
\pgfsetroundjoin%
\pgfsetlinewidth{0.528403pt}%
\definecolor{currentstroke}{rgb}{0.278012,0.180367,0.486697}%
\pgfsetstrokecolor{currentstroke}%
\pgfsetdash{}{0pt}%
\pgfpathmoveto{\pgfqpoint{4.649049in}{5.152897in}}%
\pgfpathlineto{\pgfqpoint{4.648937in}{5.152935in}}%
\pgfusepath{stroke}%
\end{pgfscope}%
\begin{pgfscope}%
\pgfpathrectangle{\pgfqpoint{3.985294in}{4.155455in}}{\pgfqpoint{2.279412in}{2.004545in}}%
\pgfusepath{clip}%
\pgfsetbuttcap%
\pgfsetroundjoin%
\pgfsetlinewidth{0.528382pt}%
\definecolor{currentstroke}{rgb}{0.278012,0.180367,0.486697}%
\pgfsetstrokecolor{currentstroke}%
\pgfsetdash{}{0pt}%
\pgfpathmoveto{\pgfqpoint{4.648937in}{5.152935in}}%
\pgfpathlineto{\pgfqpoint{4.649136in}{5.152874in}}%
\pgfusepath{stroke}%
\end{pgfscope}%
\begin{pgfscope}%
\pgfpathrectangle{\pgfqpoint{3.985294in}{4.155455in}}{\pgfqpoint{2.279412in}{2.004545in}}%
\pgfusepath{clip}%
\pgfsetbuttcap%
\pgfsetroundjoin%
\pgfsetlinewidth{0.528358pt}%
\definecolor{currentstroke}{rgb}{0.278012,0.180367,0.486697}%
\pgfsetstrokecolor{currentstroke}%
\pgfsetdash{}{0pt}%
\pgfpathmoveto{\pgfqpoint{4.649136in}{5.152874in}}%
\pgfpathlineto{\pgfqpoint{4.649511in}{5.152754in}}%
\pgfusepath{stroke}%
\end{pgfscope}%
\begin{pgfscope}%
\pgfpathrectangle{\pgfqpoint{3.985294in}{4.155455in}}{\pgfqpoint{2.279412in}{2.004545in}}%
\pgfusepath{clip}%
\pgfsetbuttcap%
\pgfsetroundjoin%
\pgfsetlinewidth{0.528411pt}%
\definecolor{currentstroke}{rgb}{0.278012,0.180367,0.486697}%
\pgfsetstrokecolor{currentstroke}%
\pgfsetdash{}{0pt}%
\pgfpathmoveto{\pgfqpoint{4.649511in}{5.152754in}}%
\pgfpathlineto{\pgfqpoint{4.649654in}{5.152705in}}%
\pgfusepath{stroke}%
\end{pgfscope}%
\begin{pgfscope}%
\pgfpathrectangle{\pgfqpoint{3.985294in}{4.155455in}}{\pgfqpoint{2.279412in}{2.004545in}}%
\pgfusepath{clip}%
\pgfsetbuttcap%
\pgfsetroundjoin%
\pgfsetlinewidth{0.528486pt}%
\definecolor{currentstroke}{rgb}{0.278012,0.180367,0.486697}%
\pgfsetstrokecolor{currentstroke}%
\pgfsetdash{}{0pt}%
\pgfpathmoveto{\pgfqpoint{4.649654in}{5.152705in}}%
\pgfpathlineto{\pgfqpoint{4.649384in}{5.152789in}}%
\pgfusepath{stroke}%
\end{pgfscope}%
\begin{pgfscope}%
\pgfpathrectangle{\pgfqpoint{3.985294in}{4.155455in}}{\pgfqpoint{2.279412in}{2.004545in}}%
\pgfusepath{clip}%
\pgfsetbuttcap%
\pgfsetroundjoin%
\pgfsetlinewidth{0.528446pt}%
\definecolor{currentstroke}{rgb}{0.278012,0.180367,0.486697}%
\pgfsetstrokecolor{currentstroke}%
\pgfsetdash{}{0pt}%
\pgfpathmoveto{\pgfqpoint{4.649384in}{5.152789in}}%
\pgfpathlineto{\pgfqpoint{4.649029in}{5.152903in}}%
\pgfusepath{stroke}%
\end{pgfscope}%
\begin{pgfscope}%
\pgfpathrectangle{\pgfqpoint{3.985294in}{4.155455in}}{\pgfqpoint{2.279412in}{2.004545in}}%
\pgfusepath{clip}%
\pgfsetbuttcap%
\pgfsetroundjoin%
\pgfsetlinewidth{0.528407pt}%
\definecolor{currentstroke}{rgb}{0.278012,0.180367,0.486697}%
\pgfsetstrokecolor{currentstroke}%
\pgfsetdash{}{0pt}%
\pgfpathmoveto{\pgfqpoint{4.649029in}{5.152903in}}%
\pgfpathlineto{\pgfqpoint{4.648905in}{5.152945in}}%
\pgfusepath{stroke}%
\end{pgfscope}%
\begin{pgfscope}%
\pgfpathrectangle{\pgfqpoint{3.985294in}{4.155455in}}{\pgfqpoint{2.279412in}{2.004545in}}%
\pgfusepath{clip}%
\pgfsetbuttcap%
\pgfsetroundjoin%
\pgfsetlinewidth{0.528386pt}%
\definecolor{currentstroke}{rgb}{0.278012,0.180367,0.486697}%
\pgfsetstrokecolor{currentstroke}%
\pgfsetdash{}{0pt}%
\pgfpathmoveto{\pgfqpoint{4.648905in}{5.152945in}}%
\pgfpathlineto{\pgfqpoint{4.649122in}{5.152878in}}%
\pgfusepath{stroke}%
\end{pgfscope}%
\begin{pgfscope}%
\pgfpathrectangle{\pgfqpoint{3.985294in}{4.155455in}}{\pgfqpoint{2.279412in}{2.004545in}}%
\pgfusepath{clip}%
\pgfsetbuttcap%
\pgfsetroundjoin%
\pgfsetlinewidth{0.528356pt}%
\definecolor{currentstroke}{rgb}{0.278012,0.180367,0.486697}%
\pgfsetstrokecolor{currentstroke}%
\pgfsetdash{}{0pt}%
\pgfpathmoveto{\pgfqpoint{4.649122in}{5.152878in}}%
\pgfpathlineto{\pgfqpoint{4.649541in}{5.152745in}}%
\pgfusepath{stroke}%
\end{pgfscope}%
\begin{pgfscope}%
\pgfpathrectangle{\pgfqpoint{3.985294in}{4.155455in}}{\pgfqpoint{2.279412in}{2.004545in}}%
\pgfusepath{clip}%
\pgfsetbuttcap%
\pgfsetroundjoin%
\pgfsetlinewidth{0.528416pt}%
\definecolor{currentstroke}{rgb}{0.278012,0.180367,0.486697}%
\pgfsetstrokecolor{currentstroke}%
\pgfsetdash{}{0pt}%
\pgfpathmoveto{\pgfqpoint{4.649541in}{5.152745in}}%
\pgfpathlineto{\pgfqpoint{4.649703in}{5.152689in}}%
\pgfusepath{stroke}%
\end{pgfscope}%
\begin{pgfscope}%
\pgfpathrectangle{\pgfqpoint{3.985294in}{4.155455in}}{\pgfqpoint{2.279412in}{2.004545in}}%
\pgfusepath{clip}%
\pgfsetbuttcap%
\pgfsetroundjoin%
\pgfsetlinewidth{0.528505pt}%
\definecolor{currentstroke}{rgb}{0.278012,0.180367,0.486697}%
\pgfsetstrokecolor{currentstroke}%
\pgfsetdash{}{0pt}%
\pgfpathmoveto{\pgfqpoint{4.649703in}{5.152689in}}%
\pgfpathlineto{\pgfqpoint{4.649401in}{5.152783in}}%
\pgfusepath{stroke}%
\end{pgfscope}%
\begin{pgfscope}%
\pgfpathrectangle{\pgfqpoint{3.985294in}{4.155455in}}{\pgfqpoint{2.279412in}{2.004545in}}%
\pgfusepath{clip}%
\pgfsetbuttcap%
\pgfsetroundjoin%
\pgfsetlinewidth{0.528455pt}%
\definecolor{currentstroke}{rgb}{0.278012,0.180367,0.486697}%
\pgfsetstrokecolor{currentstroke}%
\pgfsetdash{}{0pt}%
\pgfpathmoveto{\pgfqpoint{4.649401in}{5.152783in}}%
\pgfpathlineto{\pgfqpoint{4.649009in}{5.152909in}}%
\pgfusepath{stroke}%
\end{pgfscope}%
\begin{pgfscope}%
\pgfpathrectangle{\pgfqpoint{3.985294in}{4.155455in}}{\pgfqpoint{2.279412in}{2.004545in}}%
\pgfusepath{clip}%
\pgfsetbuttcap%
\pgfsetroundjoin%
\pgfsetlinewidth{0.528412pt}%
\definecolor{currentstroke}{rgb}{0.278012,0.180367,0.486697}%
\pgfsetstrokecolor{currentstroke}%
\pgfsetdash{}{0pt}%
\pgfpathmoveto{\pgfqpoint{4.649009in}{5.152909in}}%
\pgfpathlineto{\pgfqpoint{4.648872in}{5.152955in}}%
\pgfusepath{stroke}%
\end{pgfscope}%
\begin{pgfscope}%
\pgfpathrectangle{\pgfqpoint{3.985294in}{4.155455in}}{\pgfqpoint{2.279412in}{2.004545in}}%
\pgfusepath{clip}%
\pgfsetbuttcap%
\pgfsetroundjoin%
\pgfsetlinewidth{0.528390pt}%
\definecolor{currentstroke}{rgb}{0.278012,0.180367,0.486697}%
\pgfsetstrokecolor{currentstroke}%
\pgfsetdash{}{0pt}%
\pgfpathmoveto{\pgfqpoint{4.648872in}{5.152955in}}%
\pgfpathlineto{\pgfqpoint{4.649106in}{5.152883in}}%
\pgfusepath{stroke}%
\end{pgfscope}%
\begin{pgfscope}%
\pgfpathrectangle{\pgfqpoint{3.985294in}{4.155455in}}{\pgfqpoint{2.279412in}{2.004545in}}%
\pgfusepath{clip}%
\pgfsetbuttcap%
\pgfsetroundjoin%
\pgfsetlinewidth{0.528354pt}%
\definecolor{currentstroke}{rgb}{0.278012,0.180367,0.486697}%
\pgfsetstrokecolor{currentstroke}%
\pgfsetdash{}{0pt}%
\pgfpathmoveto{\pgfqpoint{4.649106in}{5.152883in}}%
\pgfpathlineto{\pgfqpoint{4.649574in}{5.152734in}}%
\pgfusepath{stroke}%
\end{pgfscope}%
\begin{pgfscope}%
\pgfpathrectangle{\pgfqpoint{3.985294in}{4.155455in}}{\pgfqpoint{2.279412in}{2.004545in}}%
\pgfusepath{clip}%
\pgfsetbuttcap%
\pgfsetroundjoin%
\pgfsetlinewidth{0.528422pt}%
\definecolor{currentstroke}{rgb}{0.278012,0.180367,0.486697}%
\pgfsetstrokecolor{currentstroke}%
\pgfsetdash{}{0pt}%
\pgfpathmoveto{\pgfqpoint{4.649574in}{5.152734in}}%
\pgfpathlineto{\pgfqpoint{4.649757in}{5.152672in}}%
\pgfusepath{stroke}%
\end{pgfscope}%
\begin{pgfscope}%
\pgfpathrectangle{\pgfqpoint{3.985294in}{4.155455in}}{\pgfqpoint{2.279412in}{2.004545in}}%
\pgfusepath{clip}%
\pgfsetbuttcap%
\pgfsetroundjoin%
\pgfsetlinewidth{0.528528pt}%
\definecolor{currentstroke}{rgb}{0.278012,0.180367,0.486697}%
\pgfsetstrokecolor{currentstroke}%
\pgfsetdash{}{0pt}%
\pgfpathmoveto{\pgfqpoint{4.649757in}{5.152672in}}%
\pgfpathlineto{\pgfqpoint{4.649420in}{5.152776in}}%
\pgfusepath{stroke}%
\end{pgfscope}%
\begin{pgfscope}%
\pgfpathrectangle{\pgfqpoint{3.985294in}{4.155455in}}{\pgfqpoint{2.279412in}{2.004545in}}%
\pgfusepath{clip}%
\pgfsetbuttcap%
\pgfsetroundjoin%
\pgfsetlinewidth{0.528466pt}%
\definecolor{currentstroke}{rgb}{0.278012,0.180367,0.486697}%
\pgfsetstrokecolor{currentstroke}%
\pgfsetdash{}{0pt}%
\pgfpathmoveto{\pgfqpoint{4.649420in}{5.152776in}}%
\pgfpathlineto{\pgfqpoint{4.648989in}{5.152915in}}%
\pgfusepath{stroke}%
\end{pgfscope}%
\begin{pgfscope}%
\pgfpathrectangle{\pgfqpoint{3.985294in}{4.155455in}}{\pgfqpoint{2.279412in}{2.004545in}}%
\pgfusepath{clip}%
\pgfsetbuttcap%
\pgfsetroundjoin%
\pgfsetlinewidth{0.528417pt}%
\definecolor{currentstroke}{rgb}{0.278012,0.180367,0.486697}%
\pgfsetstrokecolor{currentstroke}%
\pgfsetdash{}{0pt}%
\pgfpathmoveto{\pgfqpoint{4.648989in}{5.152915in}}%
\pgfpathlineto{\pgfqpoint{4.648837in}{5.152966in}}%
\pgfusepath{stroke}%
\end{pgfscope}%
\begin{pgfscope}%
\pgfpathrectangle{\pgfqpoint{3.985294in}{4.155455in}}{\pgfqpoint{2.279412in}{2.004545in}}%
\pgfusepath{clip}%
\pgfsetbuttcap%
\pgfsetroundjoin%
\pgfsetlinewidth{0.528395pt}%
\definecolor{currentstroke}{rgb}{0.278012,0.180367,0.486697}%
\pgfsetstrokecolor{currentstroke}%
\pgfsetdash{}{0pt}%
\pgfpathmoveto{\pgfqpoint{4.648837in}{5.152966in}}%
\pgfpathlineto{\pgfqpoint{4.649089in}{5.152889in}}%
\pgfusepath{stroke}%
\end{pgfscope}%
\begin{pgfscope}%
\pgfpathrectangle{\pgfqpoint{3.985294in}{4.155455in}}{\pgfqpoint{2.279412in}{2.004545in}}%
\pgfusepath{clip}%
\pgfsetbuttcap%
\pgfsetroundjoin%
\pgfsetlinewidth{0.528353pt}%
\definecolor{currentstroke}{rgb}{0.278012,0.180367,0.486697}%
\pgfsetstrokecolor{currentstroke}%
\pgfsetdash{}{0pt}%
\pgfpathmoveto{\pgfqpoint{4.649089in}{5.152889in}}%
\pgfpathlineto{\pgfqpoint{4.649607in}{5.152724in}}%
\pgfusepath{stroke}%
\end{pgfscope}%
\begin{pgfscope}%
\pgfpathrectangle{\pgfqpoint{3.985294in}{4.155455in}}{\pgfqpoint{2.279412in}{2.004545in}}%
\pgfusepath{clip}%
\pgfsetbuttcap%
\pgfsetroundjoin%
\pgfsetlinewidth{0.528430pt}%
\definecolor{currentstroke}{rgb}{0.278012,0.180367,0.486697}%
\pgfsetstrokecolor{currentstroke}%
\pgfsetdash{}{0pt}%
\pgfpathmoveto{\pgfqpoint{4.649607in}{5.152724in}}%
\pgfpathlineto{\pgfqpoint{4.649815in}{5.152653in}}%
\pgfusepath{stroke}%
\end{pgfscope}%
\begin{pgfscope}%
\pgfpathrectangle{\pgfqpoint{3.985294in}{4.155455in}}{\pgfqpoint{2.279412in}{2.004545in}}%
\pgfusepath{clip}%
\pgfsetbuttcap%
\pgfsetroundjoin%
\pgfsetlinewidth{0.528554pt}%
\definecolor{currentstroke}{rgb}{0.278012,0.180367,0.486697}%
\pgfsetstrokecolor{currentstroke}%
\pgfsetdash{}{0pt}%
\pgfpathmoveto{\pgfqpoint{4.649815in}{5.152653in}}%
\pgfpathlineto{\pgfqpoint{4.649441in}{5.152769in}}%
\pgfusepath{stroke}%
\end{pgfscope}%
\begin{pgfscope}%
\pgfpathrectangle{\pgfqpoint{3.985294in}{4.155455in}}{\pgfqpoint{2.279412in}{2.004545in}}%
\pgfusepath{clip}%
\pgfsetbuttcap%
\pgfsetroundjoin%
\pgfsetlinewidth{0.528478pt}%
\definecolor{currentstroke}{rgb}{0.278012,0.180367,0.486697}%
\pgfsetstrokecolor{currentstroke}%
\pgfsetdash{}{0pt}%
\pgfpathmoveto{\pgfqpoint{4.649441in}{5.152769in}}%
\pgfpathlineto{\pgfqpoint{4.648969in}{5.152920in}}%
\pgfusepath{stroke}%
\end{pgfscope}%
\begin{pgfscope}%
\pgfpathrectangle{\pgfqpoint{3.985294in}{4.155455in}}{\pgfqpoint{2.279412in}{2.004545in}}%
\pgfusepath{clip}%
\pgfsetbuttcap%
\pgfsetroundjoin%
\pgfsetlinewidth{0.528423pt}%
\definecolor{currentstroke}{rgb}{0.278012,0.180367,0.486697}%
\pgfsetstrokecolor{currentstroke}%
\pgfsetdash{}{0pt}%
\pgfpathmoveto{\pgfqpoint{4.648969in}{5.152920in}}%
\pgfpathlineto{\pgfqpoint{4.648802in}{5.152977in}}%
\pgfusepath{stroke}%
\end{pgfscope}%
\begin{pgfscope}%
\pgfpathrectangle{\pgfqpoint{3.985294in}{4.155455in}}{\pgfqpoint{2.279412in}{2.004545in}}%
\pgfusepath{clip}%
\pgfsetbuttcap%
\pgfsetroundjoin%
\pgfsetlinewidth{0.528402pt}%
\definecolor{currentstroke}{rgb}{0.278012,0.180367,0.486697}%
\pgfsetstrokecolor{currentstroke}%
\pgfsetdash{}{0pt}%
\pgfpathmoveto{\pgfqpoint{4.648802in}{5.152977in}}%
\pgfpathlineto{\pgfqpoint{4.649070in}{5.152895in}}%
\pgfusepath{stroke}%
\end{pgfscope}%
\begin{pgfscope}%
\pgfpathrectangle{\pgfqpoint{3.985294in}{4.155455in}}{\pgfqpoint{2.279412in}{2.004545in}}%
\pgfusepath{clip}%
\pgfsetbuttcap%
\pgfsetroundjoin%
\pgfsetlinewidth{0.528352pt}%
\definecolor{currentstroke}{rgb}{0.278012,0.180367,0.486697}%
\pgfsetstrokecolor{currentstroke}%
\pgfsetdash{}{0pt}%
\pgfpathmoveto{\pgfqpoint{4.649070in}{5.152895in}}%
\pgfpathlineto{\pgfqpoint{4.649642in}{5.152713in}}%
\pgfusepath{stroke}%
\end{pgfscope}%
\begin{pgfscope}%
\pgfpathrectangle{\pgfqpoint{3.985294in}{4.155455in}}{\pgfqpoint{2.279412in}{2.004545in}}%
\pgfusepath{clip}%
\pgfsetbuttcap%
\pgfsetroundjoin%
\pgfsetlinewidth{0.528438pt}%
\definecolor{currentstroke}{rgb}{0.278012,0.180367,0.486697}%
\pgfsetstrokecolor{currentstroke}%
\pgfsetdash{}{0pt}%
\pgfpathmoveto{\pgfqpoint{4.649642in}{5.152713in}}%
\pgfpathlineto{\pgfqpoint{4.649877in}{5.152632in}}%
\pgfusepath{stroke}%
\end{pgfscope}%
\begin{pgfscope}%
\pgfpathrectangle{\pgfqpoint{3.985294in}{4.155455in}}{\pgfqpoint{2.279412in}{2.004545in}}%
\pgfusepath{clip}%
\pgfsetbuttcap%
\pgfsetroundjoin%
\pgfsetlinewidth{0.528585pt}%
\definecolor{currentstroke}{rgb}{0.278012,0.180367,0.486697}%
\pgfsetstrokecolor{currentstroke}%
\pgfsetdash{}{0pt}%
\pgfpathmoveto{\pgfqpoint{4.649877in}{5.152632in}}%
\pgfpathlineto{\pgfqpoint{4.649464in}{5.152760in}}%
\pgfusepath{stroke}%
\end{pgfscope}%
\begin{pgfscope}%
\pgfpathrectangle{\pgfqpoint{3.985294in}{4.155455in}}{\pgfqpoint{2.279412in}{2.004545in}}%
\pgfusepath{clip}%
\pgfsetbuttcap%
\pgfsetroundjoin%
\pgfsetlinewidth{0.528491pt}%
\definecolor{currentstroke}{rgb}{0.278012,0.180367,0.486697}%
\pgfsetstrokecolor{currentstroke}%
\pgfsetdash{}{0pt}%
\pgfpathmoveto{\pgfqpoint{4.649464in}{5.152760in}}%
\pgfpathlineto{\pgfqpoint{4.648950in}{5.152926in}}%
\pgfusepath{stroke}%
\end{pgfscope}%
\begin{pgfscope}%
\pgfpathrectangle{\pgfqpoint{3.985294in}{4.155455in}}{\pgfqpoint{2.279412in}{2.004545in}}%
\pgfusepath{clip}%
\pgfsetbuttcap%
\pgfsetroundjoin%
\pgfsetlinewidth{0.528430pt}%
\definecolor{currentstroke}{rgb}{0.278012,0.180367,0.486697}%
\pgfsetstrokecolor{currentstroke}%
\pgfsetdash{}{0pt}%
\pgfpathmoveto{\pgfqpoint{4.648950in}{5.152926in}}%
\pgfpathlineto{\pgfqpoint{4.648766in}{5.152988in}}%
\pgfusepath{stroke}%
\end{pgfscope}%
\begin{pgfscope}%
\pgfpathrectangle{\pgfqpoint{3.985294in}{4.155455in}}{\pgfqpoint{2.279412in}{2.004545in}}%
\pgfusepath{clip}%
\pgfsetbuttcap%
\pgfsetroundjoin%
\pgfsetlinewidth{0.528409pt}%
\definecolor{currentstroke}{rgb}{0.278012,0.180367,0.486697}%
\pgfsetstrokecolor{currentstroke}%
\pgfsetdash{}{0pt}%
\pgfpathmoveto{\pgfqpoint{4.648766in}{5.152988in}}%
\pgfpathlineto{\pgfqpoint{4.649050in}{5.152902in}}%
\pgfusepath{stroke}%
\end{pgfscope}%
\begin{pgfscope}%
\pgfpathrectangle{\pgfqpoint{3.985294in}{4.155455in}}{\pgfqpoint{2.279412in}{2.004545in}}%
\pgfusepath{clip}%
\pgfsetbuttcap%
\pgfsetroundjoin%
\pgfsetlinewidth{0.528351pt}%
\definecolor{currentstroke}{rgb}{0.278012,0.180367,0.486697}%
\pgfsetstrokecolor{currentstroke}%
\pgfsetdash{}{0pt}%
\pgfpathmoveto{\pgfqpoint{4.649050in}{5.152902in}}%
\pgfpathlineto{\pgfqpoint{4.649676in}{5.152702in}}%
\pgfusepath{stroke}%
\end{pgfscope}%
\begin{pgfscope}%
\pgfpathrectangle{\pgfqpoint{3.985294in}{4.155455in}}{\pgfqpoint{2.279412in}{2.004545in}}%
\pgfusepath{clip}%
\pgfsetbuttcap%
\pgfsetroundjoin%
\pgfsetlinewidth{0.528446pt}%
\definecolor{currentstroke}{rgb}{0.278012,0.180367,0.486697}%
\pgfsetstrokecolor{currentstroke}%
\pgfsetdash{}{0pt}%
\pgfpathmoveto{\pgfqpoint{4.649676in}{5.152702in}}%
\pgfpathlineto{\pgfqpoint{4.649676in}{5.152702in}}%
\pgfusepath{stroke}%
\end{pgfscope}%
\begin{pgfscope}%
\pgfpathrectangle{\pgfqpoint{3.985294in}{4.155455in}}{\pgfqpoint{2.279412in}{2.004545in}}%
\pgfusepath{clip}%
\pgfsetbuttcap%
\pgfsetroundjoin%
\pgfsetlinewidth{0.528446pt}%
\definecolor{currentstroke}{rgb}{0.278012,0.180367,0.486697}%
\pgfsetstrokecolor{currentstroke}%
\pgfsetdash{}{0pt}%
\pgfpathmoveto{\pgfqpoint{4.649676in}{5.152702in}}%
\pgfpathlineto{\pgfqpoint{4.649294in}{5.152819in}}%
\pgfusepath{stroke}%
\end{pgfscope}%
\begin{pgfscope}%
\pgfpathrectangle{\pgfqpoint{3.985294in}{4.155455in}}{\pgfqpoint{2.279412in}{2.004545in}}%
\pgfusepath{clip}%
\pgfsetbuttcap%
\pgfsetroundjoin%
\pgfsetlinewidth{0.528411pt}%
\definecolor{currentstroke}{rgb}{0.278012,0.180367,0.486697}%
\pgfsetstrokecolor{currentstroke}%
\pgfsetdash{}{0pt}%
\pgfpathmoveto{\pgfqpoint{4.649294in}{5.152819in}}%
\pgfpathlineto{\pgfqpoint{4.648935in}{5.152934in}}%
\pgfusepath{stroke}%
\end{pgfscope}%
\begin{pgfscope}%
\pgfpathrectangle{\pgfqpoint{3.985294in}{4.155455in}}{\pgfqpoint{2.279412in}{2.004545in}}%
\pgfusepath{clip}%
\pgfsetbuttcap%
\pgfsetroundjoin%
\pgfsetlinewidth{0.528395pt}%
\definecolor{currentstroke}{rgb}{0.278012,0.180367,0.486697}%
\pgfsetstrokecolor{currentstroke}%
\pgfsetdash{}{0pt}%
\pgfpathmoveto{\pgfqpoint{4.648935in}{5.152934in}}%
\pgfpathlineto{\pgfqpoint{4.648887in}{5.152952in}}%
\pgfusepath{stroke}%
\end{pgfscope}%
\begin{pgfscope}%
\pgfpathrectangle{\pgfqpoint{3.985294in}{4.155455in}}{\pgfqpoint{2.279412in}{2.004545in}}%
\pgfusepath{clip}%
\pgfsetbuttcap%
\pgfsetroundjoin%
\pgfsetlinewidth{0.528374pt}%
\definecolor{currentstroke}{rgb}{0.278012,0.180367,0.486697}%
\pgfsetstrokecolor{currentstroke}%
\pgfsetdash{}{0pt}%
\pgfpathmoveto{\pgfqpoint{4.648887in}{5.152952in}}%
\pgfpathlineto{\pgfqpoint{4.649216in}{5.152849in}}%
\pgfusepath{stroke}%
\end{pgfscope}%
\begin{pgfscope}%
\pgfpathrectangle{\pgfqpoint{3.985294in}{4.155455in}}{\pgfqpoint{2.279412in}{2.004545in}}%
\pgfusepath{clip}%
\pgfsetbuttcap%
\pgfsetroundjoin%
\pgfsetlinewidth{0.314895pt}%
\definecolor{currentstroke}{rgb}{0.269944,0.014625,0.341379}%
\pgfsetstrokecolor{currentstroke}%
\pgfsetdash{}{0pt}%
\pgfpathmoveto{\pgfqpoint{5.996962in}{5.067514in}}%
\pgfpathlineto{\pgfqpoint{5.946864in}{5.067801in}}%
\pgfusepath{stroke}%
\end{pgfscope}%
\begin{pgfscope}%
\pgfpathrectangle{\pgfqpoint{3.985294in}{4.155455in}}{\pgfqpoint{2.279412in}{2.004545in}}%
\pgfusepath{clip}%
\pgfsetbuttcap%
\pgfsetroundjoin%
\pgfsetlinewidth{0.316764pt}%
\definecolor{currentstroke}{rgb}{0.269944,0.014625,0.341379}%
\pgfsetstrokecolor{currentstroke}%
\pgfsetdash{}{0pt}%
\pgfpathmoveto{\pgfqpoint{5.946864in}{5.067801in}}%
\pgfpathlineto{\pgfqpoint{5.896737in}{5.067112in}}%
\pgfusepath{stroke}%
\end{pgfscope}%
\begin{pgfscope}%
\pgfpathrectangle{\pgfqpoint{3.985294in}{4.155455in}}{\pgfqpoint{2.279412in}{2.004545in}}%
\pgfusepath{clip}%
\pgfsetbuttcap%
\pgfsetroundjoin%
\pgfsetlinewidth{0.322364pt}%
\definecolor{currentstroke}{rgb}{0.271305,0.019942,0.347269}%
\pgfsetstrokecolor{currentstroke}%
\pgfsetdash{}{0pt}%
\pgfpathmoveto{\pgfqpoint{5.896737in}{5.067112in}}%
\pgfpathlineto{\pgfqpoint{5.846612in}{5.067206in}}%
\pgfusepath{stroke}%
\end{pgfscope}%
\begin{pgfscope}%
\pgfpathrectangle{\pgfqpoint{3.985294in}{4.155455in}}{\pgfqpoint{2.279412in}{2.004545in}}%
\pgfusepath{clip}%
\pgfsetbuttcap%
\pgfsetroundjoin%
\pgfsetlinewidth{0.334407pt}%
\definecolor{currentstroke}{rgb}{0.272594,0.025563,0.353093}%
\pgfsetstrokecolor{currentstroke}%
\pgfsetdash{}{0pt}%
\pgfpathmoveto{\pgfqpoint{5.846612in}{5.067206in}}%
\pgfpathlineto{\pgfqpoint{5.796483in}{5.067697in}}%
\pgfusepath{stroke}%
\end{pgfscope}%
\begin{pgfscope}%
\pgfpathrectangle{\pgfqpoint{3.985294in}{4.155455in}}{\pgfqpoint{2.279412in}{2.004545in}}%
\pgfusepath{clip}%
\pgfsetbuttcap%
\pgfsetroundjoin%
\pgfsetlinewidth{0.355299pt}%
\definecolor{currentstroke}{rgb}{0.276022,0.044167,0.370164}%
\pgfsetstrokecolor{currentstroke}%
\pgfsetdash{}{0pt}%
\pgfpathmoveto{\pgfqpoint{5.796483in}{5.067697in}}%
\pgfpathlineto{\pgfqpoint{5.746348in}{5.068307in}}%
\pgfusepath{stroke}%
\end{pgfscope}%
\begin{pgfscope}%
\pgfpathrectangle{\pgfqpoint{3.985294in}{4.155455in}}{\pgfqpoint{2.279412in}{2.004545in}}%
\pgfusepath{clip}%
\pgfsetbuttcap%
\pgfsetroundjoin%
\pgfsetlinewidth{0.375569pt}%
\definecolor{currentstroke}{rgb}{0.278791,0.062145,0.386592}%
\pgfsetstrokecolor{currentstroke}%
\pgfsetdash{}{0pt}%
\pgfpathmoveto{\pgfqpoint{5.746348in}{5.068307in}}%
\pgfpathlineto{\pgfqpoint{5.696206in}{5.069109in}}%
\pgfusepath{stroke}%
\end{pgfscope}%
\begin{pgfscope}%
\pgfpathrectangle{\pgfqpoint{3.985294in}{4.155455in}}{\pgfqpoint{2.279412in}{2.004545in}}%
\pgfusepath{clip}%
\pgfsetbuttcap%
\pgfsetroundjoin%
\pgfsetlinewidth{0.413006pt}%
\definecolor{currentstroke}{rgb}{0.282327,0.094955,0.417331}%
\pgfsetstrokecolor{currentstroke}%
\pgfsetdash{}{0pt}%
\pgfpathmoveto{\pgfqpoint{5.696206in}{5.069109in}}%
\pgfpathlineto{\pgfqpoint{5.646067in}{5.069907in}}%
\pgfusepath{stroke}%
\end{pgfscope}%
\begin{pgfscope}%
\pgfpathrectangle{\pgfqpoint{3.985294in}{4.155455in}}{\pgfqpoint{2.279412in}{2.004545in}}%
\pgfusepath{clip}%
\pgfsetbuttcap%
\pgfsetroundjoin%
\pgfsetlinewidth{0.449938pt}%
\definecolor{currentstroke}{rgb}{0.283229,0.120777,0.440584}%
\pgfsetstrokecolor{currentstroke}%
\pgfsetdash{}{0pt}%
\pgfpathmoveto{\pgfqpoint{5.646067in}{5.069907in}}%
\pgfpathlineto{\pgfqpoint{5.595929in}{5.070821in}}%
\pgfusepath{stroke}%
\end{pgfscope}%
\begin{pgfscope}%
\pgfpathrectangle{\pgfqpoint{3.985294in}{4.155455in}}{\pgfqpoint{2.279412in}{2.004545in}}%
\pgfusepath{clip}%
\pgfsetbuttcap%
\pgfsetroundjoin%
\pgfsetlinewidth{0.491048pt}%
\definecolor{currentstroke}{rgb}{0.281887,0.150881,0.465405}%
\pgfsetstrokecolor{currentstroke}%
\pgfsetdash{}{0pt}%
\pgfpathmoveto{\pgfqpoint{5.595929in}{5.070821in}}%
\pgfpathlineto{\pgfqpoint{5.545789in}{5.071658in}}%
\pgfusepath{stroke}%
\end{pgfscope}%
\begin{pgfscope}%
\pgfpathrectangle{\pgfqpoint{3.985294in}{4.155455in}}{\pgfqpoint{2.279412in}{2.004545in}}%
\pgfusepath{clip}%
\pgfsetbuttcap%
\pgfsetroundjoin%
\pgfsetlinewidth{0.579984pt}%
\definecolor{currentstroke}{rgb}{0.270595,0.214069,0.507052}%
\pgfsetstrokecolor{currentstroke}%
\pgfsetdash{}{0pt}%
\pgfpathmoveto{\pgfqpoint{5.545789in}{5.071658in}}%
\pgfpathlineto{\pgfqpoint{5.495648in}{5.072538in}}%
\pgfusepath{stroke}%
\end{pgfscope}%
\begin{pgfscope}%
\pgfpathrectangle{\pgfqpoint{3.985294in}{4.155455in}}{\pgfqpoint{2.279412in}{2.004545in}}%
\pgfusepath{clip}%
\pgfsetbuttcap%
\pgfsetroundjoin%
\pgfsetlinewidth{0.644897pt}%
\definecolor{currentstroke}{rgb}{0.255645,0.260703,0.528312}%
\pgfsetstrokecolor{currentstroke}%
\pgfsetdash{}{0pt}%
\pgfpathmoveto{\pgfqpoint{5.495648in}{5.072538in}}%
\pgfpathlineto{\pgfqpoint{5.445514in}{5.073693in}}%
\pgfusepath{stroke}%
\end{pgfscope}%
\begin{pgfscope}%
\pgfpathrectangle{\pgfqpoint{3.985294in}{4.155455in}}{\pgfqpoint{2.279412in}{2.004545in}}%
\pgfusepath{clip}%
\pgfsetbuttcap%
\pgfsetroundjoin%
\pgfsetlinewidth{0.761355pt}%
\definecolor{currentstroke}{rgb}{0.223925,0.334994,0.548053}%
\pgfsetstrokecolor{currentstroke}%
\pgfsetdash{}{0pt}%
\pgfpathmoveto{\pgfqpoint{5.445514in}{5.073693in}}%
\pgfpathlineto{\pgfqpoint{5.395383in}{5.074974in}}%
\pgfusepath{stroke}%
\end{pgfscope}%
\begin{pgfscope}%
\pgfpathrectangle{\pgfqpoint{3.985294in}{4.155455in}}{\pgfqpoint{2.279412in}{2.004545in}}%
\pgfusepath{clip}%
\pgfsetbuttcap%
\pgfsetroundjoin%
\pgfsetlinewidth{0.858954pt}%
\definecolor{currentstroke}{rgb}{0.197636,0.391528,0.554969}%
\pgfsetstrokecolor{currentstroke}%
\pgfsetdash{}{0pt}%
\pgfpathmoveto{\pgfqpoint{5.395383in}{5.074974in}}%
\pgfpathlineto{\pgfqpoint{5.345258in}{5.076394in}}%
\pgfusepath{stroke}%
\end{pgfscope}%
\begin{pgfscope}%
\pgfpathrectangle{\pgfqpoint{3.985294in}{4.155455in}}{\pgfqpoint{2.279412in}{2.004545in}}%
\pgfusepath{clip}%
\pgfsetbuttcap%
\pgfsetroundjoin%
\pgfsetlinewidth{0.965048pt}%
\definecolor{currentstroke}{rgb}{0.172719,0.448791,0.557885}%
\pgfsetstrokecolor{currentstroke}%
\pgfsetdash{}{0pt}%
\pgfpathmoveto{\pgfqpoint{5.345258in}{5.076394in}}%
\pgfpathlineto{\pgfqpoint{5.295143in}{5.078077in}}%
\pgfusepath{stroke}%
\end{pgfscope}%
\begin{pgfscope}%
\pgfpathrectangle{\pgfqpoint{3.985294in}{4.155455in}}{\pgfqpoint{2.279412in}{2.004545in}}%
\pgfusepath{clip}%
\pgfsetbuttcap%
\pgfsetroundjoin%
\pgfsetlinewidth{0.304887pt}%
\definecolor{currentstroke}{rgb}{0.267004,0.004874,0.329415}%
\pgfsetstrokecolor{currentstroke}%
\pgfsetdash{}{0pt}%
\pgfpathmoveto{\pgfqpoint{5.996962in}{5.157727in}}%
\pgfpathlineto{\pgfqpoint{5.950134in}{5.151860in}}%
\pgfusepath{stroke}%
\end{pgfscope}%
\begin{pgfscope}%
\pgfpathrectangle{\pgfqpoint{3.985294in}{4.155455in}}{\pgfqpoint{2.279412in}{2.004545in}}%
\pgfusepath{clip}%
\pgfsetbuttcap%
\pgfsetroundjoin%
\pgfsetlinewidth{0.315480pt}%
\definecolor{currentstroke}{rgb}{0.269944,0.014625,0.341379}%
\pgfsetstrokecolor{currentstroke}%
\pgfsetdash{}{0pt}%
\pgfpathmoveto{\pgfqpoint{5.950134in}{5.151860in}}%
\pgfpathlineto{\pgfqpoint{5.905772in}{5.152066in}}%
\pgfusepath{stroke}%
\end{pgfscope}%
\begin{pgfscope}%
\pgfpathrectangle{\pgfqpoint{3.985294in}{4.155455in}}{\pgfqpoint{2.279412in}{2.004545in}}%
\pgfusepath{clip}%
\pgfsetbuttcap%
\pgfsetroundjoin%
\pgfsetlinewidth{0.327626pt}%
\definecolor{currentstroke}{rgb}{0.271305,0.019942,0.347269}%
\pgfsetstrokecolor{currentstroke}%
\pgfsetdash{}{0pt}%
\pgfpathmoveto{\pgfqpoint{5.905772in}{5.152066in}}%
\pgfpathlineto{\pgfqpoint{5.855629in}{5.152522in}}%
\pgfusepath{stroke}%
\end{pgfscope}%
\begin{pgfscope}%
\pgfpathrectangle{\pgfqpoint{3.985294in}{4.155455in}}{\pgfqpoint{2.279412in}{2.004545in}}%
\pgfusepath{clip}%
\pgfsetbuttcap%
\pgfsetroundjoin%
\pgfsetlinewidth{0.335010pt}%
\definecolor{currentstroke}{rgb}{0.272594,0.025563,0.353093}%
\pgfsetstrokecolor{currentstroke}%
\pgfsetdash{}{0pt}%
\pgfpathmoveto{\pgfqpoint{5.855629in}{5.152522in}}%
\pgfpathlineto{\pgfqpoint{5.805487in}{5.153165in}}%
\pgfusepath{stroke}%
\end{pgfscope}%
\begin{pgfscope}%
\pgfpathrectangle{\pgfqpoint{3.985294in}{4.155455in}}{\pgfqpoint{2.279412in}{2.004545in}}%
\pgfusepath{clip}%
\pgfsetbuttcap%
\pgfsetroundjoin%
\pgfsetlinewidth{0.357534pt}%
\definecolor{currentstroke}{rgb}{0.277018,0.050344,0.375715}%
\pgfsetstrokecolor{currentstroke}%
\pgfsetdash{}{0pt}%
\pgfpathmoveto{\pgfqpoint{5.805487in}{5.153165in}}%
\pgfpathlineto{\pgfqpoint{5.755341in}{5.153310in}}%
\pgfusepath{stroke}%
\end{pgfscope}%
\begin{pgfscope}%
\pgfpathrectangle{\pgfqpoint{3.985294in}{4.155455in}}{\pgfqpoint{2.279412in}{2.004545in}}%
\pgfusepath{clip}%
\pgfsetbuttcap%
\pgfsetroundjoin%
\pgfsetlinewidth{0.377219pt}%
\definecolor{currentstroke}{rgb}{0.279566,0.067836,0.391917}%
\pgfsetstrokecolor{currentstroke}%
\pgfsetdash{}{0pt}%
\pgfpathmoveto{\pgfqpoint{5.755341in}{5.153310in}}%
\pgfpathlineto{\pgfqpoint{5.705192in}{5.153196in}}%
\pgfusepath{stroke}%
\end{pgfscope}%
\begin{pgfscope}%
\pgfpathrectangle{\pgfqpoint{3.985294in}{4.155455in}}{\pgfqpoint{2.279412in}{2.004545in}}%
\pgfusepath{clip}%
\pgfsetbuttcap%
\pgfsetroundjoin%
\pgfsetlinewidth{0.408067pt}%
\definecolor{currentstroke}{rgb}{0.281924,0.089666,0.412415}%
\pgfsetstrokecolor{currentstroke}%
\pgfsetdash{}{0pt}%
\pgfpathmoveto{\pgfqpoint{5.705192in}{5.153196in}}%
\pgfpathlineto{\pgfqpoint{5.655040in}{5.153213in}}%
\pgfusepath{stroke}%
\end{pgfscope}%
\begin{pgfscope}%
\pgfpathrectangle{\pgfqpoint{3.985294in}{4.155455in}}{\pgfqpoint{2.279412in}{2.004545in}}%
\pgfusepath{clip}%
\pgfsetbuttcap%
\pgfsetroundjoin%
\pgfsetlinewidth{0.439536pt}%
\definecolor{currentstroke}{rgb}{0.283197,0.115680,0.436115}%
\pgfsetstrokecolor{currentstroke}%
\pgfsetdash{}{0pt}%
\pgfpathmoveto{\pgfqpoint{5.655040in}{5.153213in}}%
\pgfpathlineto{\pgfqpoint{5.604890in}{5.153191in}}%
\pgfusepath{stroke}%
\end{pgfscope}%
\begin{pgfscope}%
\pgfpathrectangle{\pgfqpoint{3.985294in}{4.155455in}}{\pgfqpoint{2.279412in}{2.004545in}}%
\pgfusepath{clip}%
\pgfsetbuttcap%
\pgfsetroundjoin%
\pgfsetlinewidth{0.504137pt}%
\definecolor{currentstroke}{rgb}{0.280868,0.160771,0.472899}%
\pgfsetstrokecolor{currentstroke}%
\pgfsetdash{}{0pt}%
\pgfpathmoveto{\pgfqpoint{5.604890in}{5.153191in}}%
\pgfpathlineto{\pgfqpoint{5.554738in}{5.153201in}}%
\pgfusepath{stroke}%
\end{pgfscope}%
\begin{pgfscope}%
\pgfpathrectangle{\pgfqpoint{3.985294in}{4.155455in}}{\pgfqpoint{2.279412in}{2.004545in}}%
\pgfusepath{clip}%
\pgfsetbuttcap%
\pgfsetroundjoin%
\pgfsetlinewidth{0.559889pt}%
\definecolor{currentstroke}{rgb}{0.274128,0.199721,0.498911}%
\pgfsetstrokecolor{currentstroke}%
\pgfsetdash{}{0pt}%
\pgfpathmoveto{\pgfqpoint{5.554738in}{5.153201in}}%
\pgfpathlineto{\pgfqpoint{5.504587in}{5.153137in}}%
\pgfusepath{stroke}%
\end{pgfscope}%
\begin{pgfscope}%
\pgfpathrectangle{\pgfqpoint{3.985294in}{4.155455in}}{\pgfqpoint{2.279412in}{2.004545in}}%
\pgfusepath{clip}%
\pgfsetbuttcap%
\pgfsetroundjoin%
\pgfsetlinewidth{0.677445pt}%
\definecolor{currentstroke}{rgb}{0.248629,0.278775,0.534556}%
\pgfsetstrokecolor{currentstroke}%
\pgfsetdash{}{0pt}%
\pgfpathmoveto{\pgfqpoint{5.504587in}{5.153137in}}%
\pgfpathlineto{\pgfqpoint{5.454436in}{5.152934in}}%
\pgfusepath{stroke}%
\end{pgfscope}%
\begin{pgfscope}%
\pgfpathrectangle{\pgfqpoint{3.985294in}{4.155455in}}{\pgfqpoint{2.279412in}{2.004545in}}%
\pgfusepath{clip}%
\pgfsetbuttcap%
\pgfsetroundjoin%
\pgfsetlinewidth{0.772743pt}%
\definecolor{currentstroke}{rgb}{0.221989,0.339161,0.548752}%
\pgfsetstrokecolor{currentstroke}%
\pgfsetdash{}{0pt}%
\pgfpathmoveto{\pgfqpoint{5.454436in}{5.152934in}}%
\pgfpathlineto{\pgfqpoint{5.404285in}{5.152784in}}%
\pgfusepath{stroke}%
\end{pgfscope}%
\begin{pgfscope}%
\pgfpathrectangle{\pgfqpoint{3.985294in}{4.155455in}}{\pgfqpoint{2.279412in}{2.004545in}}%
\pgfusepath{clip}%
\pgfsetbuttcap%
\pgfsetroundjoin%
\pgfsetlinewidth{0.883034pt}%
\definecolor{currentstroke}{rgb}{0.192357,0.403199,0.555836}%
\pgfsetstrokecolor{currentstroke}%
\pgfsetdash{}{0pt}%
\pgfpathmoveto{\pgfqpoint{5.404285in}{5.152784in}}%
\pgfpathlineto{\pgfqpoint{5.354134in}{5.152612in}}%
\pgfusepath{stroke}%
\end{pgfscope}%
\begin{pgfscope}%
\pgfpathrectangle{\pgfqpoint{3.985294in}{4.155455in}}{\pgfqpoint{2.279412in}{2.004545in}}%
\pgfusepath{clip}%
\pgfsetbuttcap%
\pgfsetroundjoin%
\pgfsetlinewidth{1.011035pt}%
\definecolor{currentstroke}{rgb}{0.162142,0.474838,0.558140}%
\pgfsetstrokecolor{currentstroke}%
\pgfsetdash{}{0pt}%
\pgfpathmoveto{\pgfqpoint{5.354134in}{5.152612in}}%
\pgfpathlineto{\pgfqpoint{5.303985in}{5.152341in}}%
\pgfusepath{stroke}%
\end{pgfscope}%
\begin{pgfscope}%
\pgfpathrectangle{\pgfqpoint{3.985294in}{4.155455in}}{\pgfqpoint{2.279412in}{2.004545in}}%
\pgfusepath{clip}%
\pgfsetbuttcap%
\pgfsetroundjoin%
\pgfsetlinewidth{1.164974pt}%
\definecolor{currentstroke}{rgb}{0.131172,0.555899,0.552459}%
\pgfsetstrokecolor{currentstroke}%
\pgfsetdash{}{0pt}%
\pgfpathmoveto{\pgfqpoint{5.303985in}{5.152341in}}%
\pgfpathlineto{\pgfqpoint{5.253835in}{5.152119in}}%
\pgfusepath{stroke}%
\end{pgfscope}%
\begin{pgfscope}%
\pgfpathrectangle{\pgfqpoint{3.985294in}{4.155455in}}{\pgfqpoint{2.279412in}{2.004545in}}%
\pgfusepath{clip}%
\pgfsetbuttcap%
\pgfsetroundjoin%
\pgfsetlinewidth{1.274030pt}%
\definecolor{currentstroke}{rgb}{0.119483,0.614817,0.537692}%
\pgfsetstrokecolor{currentstroke}%
\pgfsetdash{}{0pt}%
\pgfpathmoveto{\pgfqpoint{5.253835in}{5.152119in}}%
\pgfpathlineto{\pgfqpoint{5.203685in}{5.151998in}}%
\pgfusepath{stroke}%
\end{pgfscope}%
\begin{pgfscope}%
\pgfpathrectangle{\pgfqpoint{3.985294in}{4.155455in}}{\pgfqpoint{2.279412in}{2.004545in}}%
\pgfusepath{clip}%
\pgfsetbuttcap%
\pgfsetroundjoin%
\pgfsetlinewidth{1.412911pt}%
\definecolor{currentstroke}{rgb}{0.166383,0.690856,0.496502}%
\pgfsetstrokecolor{currentstroke}%
\pgfsetdash{}{0pt}%
\pgfpathmoveto{\pgfqpoint{5.203685in}{5.151998in}}%
\pgfpathlineto{\pgfqpoint{5.153536in}{5.151907in}}%
\pgfusepath{stroke}%
\end{pgfscope}%
\begin{pgfscope}%
\pgfpathrectangle{\pgfqpoint{3.985294in}{4.155455in}}{\pgfqpoint{2.279412in}{2.004545in}}%
\pgfusepath{clip}%
\pgfsetbuttcap%
\pgfsetroundjoin%
\pgfsetlinewidth{1.534934pt}%
\definecolor{currentstroke}{rgb}{0.266941,0.748751,0.440573}%
\pgfsetstrokecolor{currentstroke}%
\pgfsetdash{}{0pt}%
\pgfpathmoveto{\pgfqpoint{5.153536in}{5.151907in}}%
\pgfpathlineto{\pgfqpoint{5.103388in}{5.151745in}}%
\pgfusepath{stroke}%
\end{pgfscope}%
\begin{pgfscope}%
\pgfpathrectangle{\pgfqpoint{3.985294in}{4.155455in}}{\pgfqpoint{2.279412in}{2.004545in}}%
\pgfusepath{clip}%
\pgfsetbuttcap%
\pgfsetroundjoin%
\pgfsetlinewidth{1.664811pt}%
\definecolor{currentstroke}{rgb}{0.421908,0.805774,0.351910}%
\pgfsetstrokecolor{currentstroke}%
\pgfsetdash{}{0pt}%
\pgfpathmoveto{\pgfqpoint{5.103388in}{5.151745in}}%
\pgfpathlineto{\pgfqpoint{5.053243in}{5.151649in}}%
\pgfusepath{stroke}%
\end{pgfscope}%
\begin{pgfscope}%
\pgfpathrectangle{\pgfqpoint{3.985294in}{4.155455in}}{\pgfqpoint{2.279412in}{2.004545in}}%
\pgfusepath{clip}%
\pgfsetbuttcap%
\pgfsetroundjoin%
\pgfsetlinewidth{1.611233pt}%
\definecolor{currentstroke}{rgb}{0.352360,0.783011,0.392636}%
\pgfsetstrokecolor{currentstroke}%
\pgfsetdash{}{0pt}%
\pgfpathmoveto{\pgfqpoint{5.053243in}{5.151649in}}%
\pgfpathlineto{\pgfqpoint{5.003102in}{5.151604in}}%
\pgfusepath{stroke}%
\end{pgfscope}%
\begin{pgfscope}%
\pgfpathrectangle{\pgfqpoint{3.985294in}{4.155455in}}{\pgfqpoint{2.279412in}{2.004545in}}%
\pgfusepath{clip}%
\pgfsetbuttcap%
\pgfsetroundjoin%
\pgfsetlinewidth{1.620421pt}%
\definecolor{currentstroke}{rgb}{0.369214,0.788888,0.382914}%
\pgfsetstrokecolor{currentstroke}%
\pgfsetdash{}{0pt}%
\pgfpathmoveto{\pgfqpoint{5.003102in}{5.151604in}}%
\pgfpathlineto{\pgfqpoint{4.952968in}{5.151561in}}%
\pgfusepath{stroke}%
\end{pgfscope}%
\begin{pgfscope}%
\pgfpathrectangle{\pgfqpoint{3.985294in}{4.155455in}}{\pgfqpoint{2.279412in}{2.004545in}}%
\pgfusepath{clip}%
\pgfsetbuttcap%
\pgfsetroundjoin%
\pgfsetlinewidth{1.498576pt}%
\definecolor{currentstroke}{rgb}{0.232815,0.732247,0.459277}%
\pgfsetstrokecolor{currentstroke}%
\pgfsetdash{}{0pt}%
\pgfpathmoveto{\pgfqpoint{4.952968in}{5.151561in}}%
\pgfpathlineto{\pgfqpoint{4.902841in}{5.151673in}}%
\pgfusepath{stroke}%
\end{pgfscope}%
\begin{pgfscope}%
\pgfpathrectangle{\pgfqpoint{3.985294in}{4.155455in}}{\pgfqpoint{2.279412in}{2.004545in}}%
\pgfusepath{clip}%
\pgfsetbuttcap%
\pgfsetroundjoin%
\pgfsetlinewidth{1.368795pt}%
\definecolor{currentstroke}{rgb}{0.140210,0.665859,0.513427}%
\pgfsetstrokecolor{currentstroke}%
\pgfsetdash{}{0pt}%
\pgfpathmoveto{\pgfqpoint{4.902841in}{5.151673in}}%
\pgfpathlineto{\pgfqpoint{4.852732in}{5.151755in}}%
\pgfusepath{stroke}%
\end{pgfscope}%
\begin{pgfscope}%
\pgfpathrectangle{\pgfqpoint{3.985294in}{4.155455in}}{\pgfqpoint{2.279412in}{2.004545in}}%
\pgfusepath{clip}%
\pgfsetbuttcap%
\pgfsetroundjoin%
\pgfsetlinewidth{1.160367pt}%
\definecolor{currentstroke}{rgb}{0.131172,0.555899,0.552459}%
\pgfsetstrokecolor{currentstroke}%
\pgfsetdash{}{0pt}%
\pgfpathmoveto{\pgfqpoint{4.852732in}{5.151755in}}%
\pgfpathlineto{\pgfqpoint{4.802667in}{5.151964in}}%
\pgfusepath{stroke}%
\end{pgfscope}%
\begin{pgfscope}%
\pgfpathrectangle{\pgfqpoint{3.985294in}{4.155455in}}{\pgfqpoint{2.279412in}{2.004545in}}%
\pgfusepath{clip}%
\pgfsetbuttcap%
\pgfsetroundjoin%
\pgfsetlinewidth{1.000225pt}%
\definecolor{currentstroke}{rgb}{0.165117,0.467423,0.558141}%
\pgfsetstrokecolor{currentstroke}%
\pgfsetdash{}{0pt}%
\pgfpathmoveto{\pgfqpoint{4.802667in}{5.151964in}}%
\pgfpathlineto{\pgfqpoint{4.752712in}{5.151988in}}%
\pgfusepath{stroke}%
\end{pgfscope}%
\begin{pgfscope}%
\pgfpathrectangle{\pgfqpoint{3.985294in}{4.155455in}}{\pgfqpoint{2.279412in}{2.004545in}}%
\pgfusepath{clip}%
\pgfsetbuttcap%
\pgfsetroundjoin%
\pgfsetlinewidth{0.318163pt}%
\definecolor{currentstroke}{rgb}{0.269944,0.014625,0.341379}%
\pgfsetstrokecolor{currentstroke}%
\pgfsetdash{}{0pt}%
\pgfpathmoveto{\pgfqpoint{5.996962in}{5.202834in}}%
\pgfpathlineto{\pgfqpoint{5.946877in}{5.203440in}}%
\pgfusepath{stroke}%
\end{pgfscope}%
\begin{pgfscope}%
\pgfpathrectangle{\pgfqpoint{3.985294in}{4.155455in}}{\pgfqpoint{2.279412in}{2.004545in}}%
\pgfusepath{clip}%
\pgfsetbuttcap%
\pgfsetroundjoin%
\pgfsetlinewidth{0.321276pt}%
\definecolor{currentstroke}{rgb}{0.269944,0.014625,0.341379}%
\pgfsetstrokecolor{currentstroke}%
\pgfsetdash{}{0pt}%
\pgfpathmoveto{\pgfqpoint{5.946877in}{5.203440in}}%
\pgfpathlineto{\pgfqpoint{5.896774in}{5.203984in}}%
\pgfusepath{stroke}%
\end{pgfscope}%
\begin{pgfscope}%
\pgfpathrectangle{\pgfqpoint{3.985294in}{4.155455in}}{\pgfqpoint{2.279412in}{2.004545in}}%
\pgfusepath{clip}%
\pgfsetbuttcap%
\pgfsetroundjoin%
\pgfsetlinewidth{0.340688pt}%
\definecolor{currentstroke}{rgb}{0.273809,0.031497,0.358853}%
\pgfsetstrokecolor{currentstroke}%
\pgfsetdash{}{0pt}%
\pgfpathmoveto{\pgfqpoint{5.896774in}{5.203984in}}%
\pgfpathlineto{\pgfqpoint{5.846634in}{5.204539in}}%
\pgfusepath{stroke}%
\end{pgfscope}%
\begin{pgfscope}%
\pgfpathrectangle{\pgfqpoint{3.985294in}{4.155455in}}{\pgfqpoint{2.279412in}{2.004545in}}%
\pgfusepath{clip}%
\pgfsetbuttcap%
\pgfsetroundjoin%
\pgfsetlinewidth{0.329763pt}%
\definecolor{currentstroke}{rgb}{0.272594,0.025563,0.353093}%
\pgfsetstrokecolor{currentstroke}%
\pgfsetdash{}{0pt}%
\pgfpathmoveto{\pgfqpoint{5.846634in}{5.204539in}}%
\pgfpathlineto{\pgfqpoint{5.796496in}{5.205090in}}%
\pgfusepath{stroke}%
\end{pgfscope}%
\begin{pgfscope}%
\pgfpathrectangle{\pgfqpoint{3.985294in}{4.155455in}}{\pgfqpoint{2.279412in}{2.004545in}}%
\pgfusepath{clip}%
\pgfsetbuttcap%
\pgfsetroundjoin%
\pgfsetlinewidth{0.347732pt}%
\definecolor{currentstroke}{rgb}{0.274952,0.037752,0.364543}%
\pgfsetstrokecolor{currentstroke}%
\pgfsetdash{}{0pt}%
\pgfpathmoveto{\pgfqpoint{5.796496in}{5.205090in}}%
\pgfpathlineto{\pgfqpoint{5.746350in}{5.205503in}}%
\pgfusepath{stroke}%
\end{pgfscope}%
\begin{pgfscope}%
\pgfpathrectangle{\pgfqpoint{3.985294in}{4.155455in}}{\pgfqpoint{2.279412in}{2.004545in}}%
\pgfusepath{clip}%
\pgfsetbuttcap%
\pgfsetroundjoin%
\pgfsetlinewidth{0.370771pt}%
\definecolor{currentstroke}{rgb}{0.278791,0.062145,0.386592}%
\pgfsetstrokecolor{currentstroke}%
\pgfsetdash{}{0pt}%
\pgfpathmoveto{\pgfqpoint{5.746350in}{5.205503in}}%
\pgfpathlineto{\pgfqpoint{5.696200in}{5.205455in}}%
\pgfusepath{stroke}%
\end{pgfscope}%
\begin{pgfscope}%
\pgfpathrectangle{\pgfqpoint{3.985294in}{4.155455in}}{\pgfqpoint{2.279412in}{2.004545in}}%
\pgfusepath{clip}%
\pgfsetbuttcap%
\pgfsetroundjoin%
\pgfsetlinewidth{0.398948pt}%
\definecolor{currentstroke}{rgb}{0.281446,0.084320,0.407414}%
\pgfsetstrokecolor{currentstroke}%
\pgfsetdash{}{0pt}%
\pgfpathmoveto{\pgfqpoint{5.696200in}{5.205455in}}%
\pgfpathlineto{\pgfqpoint{5.646050in}{5.205200in}}%
\pgfusepath{stroke}%
\end{pgfscope}%
\begin{pgfscope}%
\pgfpathrectangle{\pgfqpoint{3.985294in}{4.155455in}}{\pgfqpoint{2.279412in}{2.004545in}}%
\pgfusepath{clip}%
\pgfsetbuttcap%
\pgfsetroundjoin%
\pgfsetlinewidth{0.462930pt}%
\definecolor{currentstroke}{rgb}{0.283072,0.130895,0.449241}%
\pgfsetstrokecolor{currentstroke}%
\pgfsetdash{}{0pt}%
\pgfpathmoveto{\pgfqpoint{5.646050in}{5.205200in}}%
\pgfpathlineto{\pgfqpoint{5.595902in}{5.204843in}}%
\pgfusepath{stroke}%
\end{pgfscope}%
\begin{pgfscope}%
\pgfpathrectangle{\pgfqpoint{3.985294in}{4.155455in}}{\pgfqpoint{2.279412in}{2.004545in}}%
\pgfusepath{clip}%
\pgfsetbuttcap%
\pgfsetroundjoin%
\pgfsetlinewidth{0.499202pt}%
\definecolor{currentstroke}{rgb}{0.281412,0.155834,0.469201}%
\pgfsetstrokecolor{currentstroke}%
\pgfsetdash{}{0pt}%
\pgfpathmoveto{\pgfqpoint{5.595902in}{5.204843in}}%
\pgfpathlineto{\pgfqpoint{5.545755in}{5.204349in}}%
\pgfusepath{stroke}%
\end{pgfscope}%
\begin{pgfscope}%
\pgfpathrectangle{\pgfqpoint{3.985294in}{4.155455in}}{\pgfqpoint{2.279412in}{2.004545in}}%
\pgfusepath{clip}%
\pgfsetbuttcap%
\pgfsetroundjoin%
\pgfsetlinewidth{0.577181pt}%
\definecolor{currentstroke}{rgb}{0.270595,0.214069,0.507052}%
\pgfsetstrokecolor{currentstroke}%
\pgfsetdash{}{0pt}%
\pgfpathmoveto{\pgfqpoint{5.545755in}{5.204349in}}%
\pgfpathlineto{\pgfqpoint{5.495607in}{5.203794in}}%
\pgfusepath{stroke}%
\end{pgfscope}%
\begin{pgfscope}%
\pgfpathrectangle{\pgfqpoint{3.985294in}{4.155455in}}{\pgfqpoint{2.279412in}{2.004545in}}%
\pgfusepath{clip}%
\pgfsetbuttcap%
\pgfsetroundjoin%
\pgfsetlinewidth{0.672362pt}%
\definecolor{currentstroke}{rgb}{0.248629,0.278775,0.534556}%
\pgfsetstrokecolor{currentstroke}%
\pgfsetdash{}{0pt}%
\pgfpathmoveto{\pgfqpoint{5.495607in}{5.203794in}}%
\pgfpathlineto{\pgfqpoint{5.445462in}{5.203067in}}%
\pgfusepath{stroke}%
\end{pgfscope}%
\begin{pgfscope}%
\pgfpathrectangle{\pgfqpoint{3.985294in}{4.155455in}}{\pgfqpoint{2.279412in}{2.004545in}}%
\pgfusepath{clip}%
\pgfsetbuttcap%
\pgfsetroundjoin%
\pgfsetlinewidth{0.774705pt}%
\definecolor{currentstroke}{rgb}{0.220057,0.343307,0.549413}%
\pgfsetstrokecolor{currentstroke}%
\pgfsetdash{}{0pt}%
\pgfpathmoveto{\pgfqpoint{5.445462in}{5.203067in}}%
\pgfpathlineto{\pgfqpoint{5.395321in}{5.202145in}}%
\pgfusepath{stroke}%
\end{pgfscope}%
\begin{pgfscope}%
\pgfpathrectangle{\pgfqpoint{3.985294in}{4.155455in}}{\pgfqpoint{2.279412in}{2.004545in}}%
\pgfusepath{clip}%
\pgfsetbuttcap%
\pgfsetroundjoin%
\pgfsetlinewidth{0.880753pt}%
\definecolor{currentstroke}{rgb}{0.192357,0.403199,0.555836}%
\pgfsetstrokecolor{currentstroke}%
\pgfsetdash{}{0pt}%
\pgfpathmoveto{\pgfqpoint{5.395321in}{5.202145in}}%
\pgfpathlineto{\pgfqpoint{5.345184in}{5.201098in}}%
\pgfusepath{stroke}%
\end{pgfscope}%
\begin{pgfscope}%
\pgfpathrectangle{\pgfqpoint{3.985294in}{4.155455in}}{\pgfqpoint{2.279412in}{2.004545in}}%
\pgfusepath{clip}%
\pgfsetbuttcap%
\pgfsetroundjoin%
\pgfsetlinewidth{0.989033pt}%
\definecolor{currentstroke}{rgb}{0.166617,0.463708,0.558119}%
\pgfsetstrokecolor{currentstroke}%
\pgfsetdash{}{0pt}%
\pgfpathmoveto{\pgfqpoint{5.345184in}{5.201098in}}%
\pgfpathlineto{\pgfqpoint{5.295052in}{5.199854in}}%
\pgfusepath{stroke}%
\end{pgfscope}%
\begin{pgfscope}%
\pgfpathrectangle{\pgfqpoint{3.985294in}{4.155455in}}{\pgfqpoint{2.279412in}{2.004545in}}%
\pgfusepath{clip}%
\pgfsetbuttcap%
\pgfsetroundjoin%
\pgfsetlinewidth{1.140974pt}%
\definecolor{currentstroke}{rgb}{0.135066,0.544853,0.554029}%
\pgfsetstrokecolor{currentstroke}%
\pgfsetdash{}{0pt}%
\pgfpathmoveto{\pgfqpoint{5.295052in}{5.199854in}}%
\pgfpathlineto{\pgfqpoint{5.244931in}{5.198317in}}%
\pgfusepath{stroke}%
\end{pgfscope}%
\begin{pgfscope}%
\pgfpathrectangle{\pgfqpoint{3.985294in}{4.155455in}}{\pgfqpoint{2.279412in}{2.004545in}}%
\pgfusepath{clip}%
\pgfsetbuttcap%
\pgfsetroundjoin%
\pgfsetlinewidth{1.207102pt}%
\definecolor{currentstroke}{rgb}{0.123463,0.581687,0.547445}%
\pgfsetstrokecolor{currentstroke}%
\pgfsetdash{}{0pt}%
\pgfpathmoveto{\pgfqpoint{5.244931in}{5.198317in}}%
\pgfpathlineto{\pgfqpoint{5.194818in}{5.196573in}}%
\pgfusepath{stroke}%
\end{pgfscope}%
\begin{pgfscope}%
\pgfpathrectangle{\pgfqpoint{3.985294in}{4.155455in}}{\pgfqpoint{2.279412in}{2.004545in}}%
\pgfusepath{clip}%
\pgfsetbuttcap%
\pgfsetroundjoin%
\pgfsetlinewidth{1.314564pt}%
\definecolor{currentstroke}{rgb}{0.123444,0.636809,0.528763}%
\pgfsetstrokecolor{currentstroke}%
\pgfsetdash{}{0pt}%
\pgfpathmoveto{\pgfqpoint{5.194818in}{5.196573in}}%
\pgfpathlineto{\pgfqpoint{5.144727in}{5.194441in}}%
\pgfusepath{stroke}%
\end{pgfscope}%
\begin{pgfscope}%
\pgfpathrectangle{\pgfqpoint{3.985294in}{4.155455in}}{\pgfqpoint{2.279412in}{2.004545in}}%
\pgfusepath{clip}%
\pgfsetbuttcap%
\pgfsetroundjoin%
\pgfsetlinewidth{1.436925pt}%
\definecolor{currentstroke}{rgb}{0.180653,0.701402,0.488189}%
\pgfsetstrokecolor{currentstroke}%
\pgfsetdash{}{0pt}%
\pgfpathmoveto{\pgfqpoint{5.144727in}{5.194441in}}%
\pgfpathlineto{\pgfqpoint{5.094663in}{5.191867in}}%
\pgfusepath{stroke}%
\end{pgfscope}%
\begin{pgfscope}%
\pgfpathrectangle{\pgfqpoint{3.985294in}{4.155455in}}{\pgfqpoint{2.279412in}{2.004545in}}%
\pgfusepath{clip}%
\pgfsetbuttcap%
\pgfsetroundjoin%
\pgfsetlinewidth{1.522158pt}%
\definecolor{currentstroke}{rgb}{0.252899,0.742211,0.448284}%
\pgfsetstrokecolor{currentstroke}%
\pgfsetdash{}{0pt}%
\pgfpathmoveto{\pgfqpoint{5.094663in}{5.191867in}}%
\pgfpathlineto{\pgfqpoint{5.044631in}{5.188882in}}%
\pgfusepath{stroke}%
\end{pgfscope}%
\begin{pgfscope}%
\pgfpathrectangle{\pgfqpoint{3.985294in}{4.155455in}}{\pgfqpoint{2.279412in}{2.004545in}}%
\pgfusepath{clip}%
\pgfsetbuttcap%
\pgfsetroundjoin%
\pgfsetlinewidth{1.498547pt}%
\definecolor{currentstroke}{rgb}{0.232815,0.732247,0.459277}%
\pgfsetstrokecolor{currentstroke}%
\pgfsetdash{}{0pt}%
\pgfpathmoveto{\pgfqpoint{5.044631in}{5.188882in}}%
\pgfpathlineto{\pgfqpoint{4.994647in}{5.185370in}}%
\pgfusepath{stroke}%
\end{pgfscope}%
\begin{pgfscope}%
\pgfpathrectangle{\pgfqpoint{3.985294in}{4.155455in}}{\pgfqpoint{2.279412in}{2.004545in}}%
\pgfusepath{clip}%
\pgfsetbuttcap%
\pgfsetroundjoin%
\pgfsetlinewidth{1.504095pt}%
\definecolor{currentstroke}{rgb}{0.239374,0.735588,0.455688}%
\pgfsetstrokecolor{currentstroke}%
\pgfsetdash{}{0pt}%
\pgfpathmoveto{\pgfqpoint{4.994647in}{5.185370in}}%
\pgfpathlineto{\pgfqpoint{4.944728in}{5.181224in}}%
\pgfusepath{stroke}%
\end{pgfscope}%
\begin{pgfscope}%
\pgfpathrectangle{\pgfqpoint{3.985294in}{4.155455in}}{\pgfqpoint{2.279412in}{2.004545in}}%
\pgfusepath{clip}%
\pgfsetbuttcap%
\pgfsetroundjoin%
\pgfsetlinewidth{0.313669pt}%
\definecolor{currentstroke}{rgb}{0.268510,0.009605,0.335427}%
\pgfsetstrokecolor{currentstroke}%
\pgfsetdash{}{0pt}%
\pgfpathmoveto{\pgfqpoint{5.996962in}{5.293048in}}%
\pgfpathlineto{\pgfqpoint{5.947063in}{5.291326in}}%
\pgfusepath{stroke}%
\end{pgfscope}%
\begin{pgfscope}%
\pgfpathrectangle{\pgfqpoint{3.985294in}{4.155455in}}{\pgfqpoint{2.279412in}{2.004545in}}%
\pgfusepath{clip}%
\pgfsetbuttcap%
\pgfsetroundjoin%
\pgfsetlinewidth{0.318996pt}%
\definecolor{currentstroke}{rgb}{0.269944,0.014625,0.341379}%
\pgfsetstrokecolor{currentstroke}%
\pgfsetdash{}{0pt}%
\pgfpathmoveto{\pgfqpoint{5.947063in}{5.291326in}}%
\pgfpathlineto{\pgfqpoint{5.896935in}{5.290978in}}%
\pgfusepath{stroke}%
\end{pgfscope}%
\begin{pgfscope}%
\pgfpathrectangle{\pgfqpoint{3.985294in}{4.155455in}}{\pgfqpoint{2.279412in}{2.004545in}}%
\pgfusepath{clip}%
\pgfsetbuttcap%
\pgfsetroundjoin%
\pgfsetlinewidth{0.333422pt}%
\definecolor{currentstroke}{rgb}{0.272594,0.025563,0.353093}%
\pgfsetstrokecolor{currentstroke}%
\pgfsetdash{}{0pt}%
\pgfpathmoveto{\pgfqpoint{5.896935in}{5.290978in}}%
\pgfpathlineto{\pgfqpoint{5.846801in}{5.290198in}}%
\pgfusepath{stroke}%
\end{pgfscope}%
\begin{pgfscope}%
\pgfpathrectangle{\pgfqpoint{3.985294in}{4.155455in}}{\pgfqpoint{2.279412in}{2.004545in}}%
\pgfusepath{clip}%
\pgfsetbuttcap%
\pgfsetroundjoin%
\pgfsetlinewidth{0.334839pt}%
\definecolor{currentstroke}{rgb}{0.272594,0.025563,0.353093}%
\pgfsetstrokecolor{currentstroke}%
\pgfsetdash{}{0pt}%
\pgfpathmoveto{\pgfqpoint{5.846801in}{5.290198in}}%
\pgfpathlineto{\pgfqpoint{5.796652in}{5.290024in}}%
\pgfusepath{stroke}%
\end{pgfscope}%
\begin{pgfscope}%
\pgfpathrectangle{\pgfqpoint{3.985294in}{4.155455in}}{\pgfqpoint{2.279412in}{2.004545in}}%
\pgfusepath{clip}%
\pgfsetbuttcap%
\pgfsetroundjoin%
\pgfsetlinewidth{0.346950pt}%
\definecolor{currentstroke}{rgb}{0.274952,0.037752,0.364543}%
\pgfsetstrokecolor{currentstroke}%
\pgfsetdash{}{0pt}%
\pgfpathmoveto{\pgfqpoint{5.796652in}{5.290024in}}%
\pgfpathlineto{\pgfqpoint{5.746508in}{5.289578in}}%
\pgfusepath{stroke}%
\end{pgfscope}%
\begin{pgfscope}%
\pgfpathrectangle{\pgfqpoint{3.985294in}{4.155455in}}{\pgfqpoint{2.279412in}{2.004545in}}%
\pgfusepath{clip}%
\pgfsetbuttcap%
\pgfsetroundjoin%
\pgfsetlinewidth{0.366202pt}%
\definecolor{currentstroke}{rgb}{0.277941,0.056324,0.381191}%
\pgfsetstrokecolor{currentstroke}%
\pgfsetdash{}{0pt}%
\pgfpathmoveto{\pgfqpoint{5.746508in}{5.289578in}}%
\pgfpathlineto{\pgfqpoint{5.696368in}{5.288795in}}%
\pgfusepath{stroke}%
\end{pgfscope}%
\begin{pgfscope}%
\pgfpathrectangle{\pgfqpoint{3.985294in}{4.155455in}}{\pgfqpoint{2.279412in}{2.004545in}}%
\pgfusepath{clip}%
\pgfsetbuttcap%
\pgfsetroundjoin%
\pgfsetlinewidth{0.400406pt}%
\definecolor{currentstroke}{rgb}{0.281446,0.084320,0.407414}%
\pgfsetstrokecolor{currentstroke}%
\pgfsetdash{}{0pt}%
\pgfpathmoveto{\pgfqpoint{5.696368in}{5.288795in}}%
\pgfpathlineto{\pgfqpoint{5.646233in}{5.287811in}}%
\pgfusepath{stroke}%
\end{pgfscope}%
\begin{pgfscope}%
\pgfpathrectangle{\pgfqpoint{3.985294in}{4.155455in}}{\pgfqpoint{2.279412in}{2.004545in}}%
\pgfusepath{clip}%
\pgfsetbuttcap%
\pgfsetroundjoin%
\pgfsetlinewidth{0.432981pt}%
\definecolor{currentstroke}{rgb}{0.283091,0.110553,0.431554}%
\pgfsetstrokecolor{currentstroke}%
\pgfsetdash{}{0pt}%
\pgfpathmoveto{\pgfqpoint{5.646233in}{5.287811in}}%
\pgfpathlineto{\pgfqpoint{5.596100in}{5.286682in}}%
\pgfusepath{stroke}%
\end{pgfscope}%
\begin{pgfscope}%
\pgfpathrectangle{\pgfqpoint{3.985294in}{4.155455in}}{\pgfqpoint{2.279412in}{2.004545in}}%
\pgfusepath{clip}%
\pgfsetbuttcap%
\pgfsetroundjoin%
\pgfsetlinewidth{0.489770pt}%
\definecolor{currentstroke}{rgb}{0.281887,0.150881,0.465405}%
\pgfsetstrokecolor{currentstroke}%
\pgfsetdash{}{0pt}%
\pgfpathmoveto{\pgfqpoint{5.596100in}{5.286682in}}%
\pgfpathlineto{\pgfqpoint{5.545970in}{5.285408in}}%
\pgfusepath{stroke}%
\end{pgfscope}%
\begin{pgfscope}%
\pgfpathrectangle{\pgfqpoint{3.985294in}{4.155455in}}{\pgfqpoint{2.279412in}{2.004545in}}%
\pgfusepath{clip}%
\pgfsetbuttcap%
\pgfsetroundjoin%
\pgfsetlinewidth{0.538401pt}%
\definecolor{currentstroke}{rgb}{0.277134,0.185228,0.489898}%
\pgfsetstrokecolor{currentstroke}%
\pgfsetdash{}{0pt}%
\pgfpathmoveto{\pgfqpoint{5.545970in}{5.285408in}}%
\pgfpathlineto{\pgfqpoint{5.495853in}{5.283786in}}%
\pgfusepath{stroke}%
\end{pgfscope}%
\begin{pgfscope}%
\pgfpathrectangle{\pgfqpoint{3.985294in}{4.155455in}}{\pgfqpoint{2.279412in}{2.004545in}}%
\pgfusepath{clip}%
\pgfsetbuttcap%
\pgfsetroundjoin%
\pgfsetlinewidth{0.625824pt}%
\definecolor{currentstroke}{rgb}{0.260571,0.246922,0.522828}%
\pgfsetstrokecolor{currentstroke}%
\pgfsetdash{}{0pt}%
\pgfpathmoveto{\pgfqpoint{5.495853in}{5.283786in}}%
\pgfpathlineto{\pgfqpoint{5.445742in}{5.282034in}}%
\pgfusepath{stroke}%
\end{pgfscope}%
\begin{pgfscope}%
\pgfpathrectangle{\pgfqpoint{3.985294in}{4.155455in}}{\pgfqpoint{2.279412in}{2.004545in}}%
\pgfusepath{clip}%
\pgfsetbuttcap%
\pgfsetroundjoin%
\pgfsetlinewidth{0.702021pt}%
\definecolor{currentstroke}{rgb}{0.241237,0.296485,0.539709}%
\pgfsetstrokecolor{currentstroke}%
\pgfsetdash{}{0pt}%
\pgfpathmoveto{\pgfqpoint{5.445742in}{5.282034in}}%
\pgfpathlineto{\pgfqpoint{5.395650in}{5.279927in}}%
\pgfusepath{stroke}%
\end{pgfscope}%
\begin{pgfscope}%
\pgfpathrectangle{\pgfqpoint{3.985294in}{4.155455in}}{\pgfqpoint{2.279412in}{2.004545in}}%
\pgfusepath{clip}%
\pgfsetbuttcap%
\pgfsetroundjoin%
\pgfsetlinewidth{0.787501pt}%
\definecolor{currentstroke}{rgb}{0.218130,0.347432,0.550038}%
\pgfsetstrokecolor{currentstroke}%
\pgfsetdash{}{0pt}%
\pgfpathmoveto{\pgfqpoint{5.395650in}{5.279927in}}%
\pgfpathlineto{\pgfqpoint{5.345578in}{5.277440in}}%
\pgfusepath{stroke}%
\end{pgfscope}%
\begin{pgfscope}%
\pgfpathrectangle{\pgfqpoint{3.985294in}{4.155455in}}{\pgfqpoint{2.279412in}{2.004545in}}%
\pgfusepath{clip}%
\pgfsetbuttcap%
\pgfsetroundjoin%
\pgfsetlinewidth{0.881833pt}%
\definecolor{currentstroke}{rgb}{0.192357,0.403199,0.555836}%
\pgfsetstrokecolor{currentstroke}%
\pgfsetdash{}{0pt}%
\pgfpathmoveto{\pgfqpoint{5.345578in}{5.277440in}}%
\pgfpathlineto{\pgfqpoint{5.295516in}{5.274801in}}%
\pgfusepath{stroke}%
\end{pgfscope}%
\begin{pgfscope}%
\pgfpathrectangle{\pgfqpoint{3.985294in}{4.155455in}}{\pgfqpoint{2.279412in}{2.004545in}}%
\pgfusepath{clip}%
\pgfsetbuttcap%
\pgfsetroundjoin%
\pgfsetlinewidth{0.952325pt}%
\definecolor{currentstroke}{rgb}{0.175841,0.441290,0.557685}%
\pgfsetstrokecolor{currentstroke}%
\pgfsetdash{}{0pt}%
\pgfpathmoveto{\pgfqpoint{5.295516in}{5.274801in}}%
\pgfpathlineto{\pgfqpoint{5.245483in}{5.271790in}}%
\pgfusepath{stroke}%
\end{pgfscope}%
\begin{pgfscope}%
\pgfpathrectangle{\pgfqpoint{3.985294in}{4.155455in}}{\pgfqpoint{2.279412in}{2.004545in}}%
\pgfusepath{clip}%
\pgfsetbuttcap%
\pgfsetroundjoin%
\pgfsetlinewidth{1.066179pt}%
\definecolor{currentstroke}{rgb}{0.150476,0.504369,0.557430}%
\pgfsetstrokecolor{currentstroke}%
\pgfsetdash{}{0pt}%
\pgfpathmoveto{\pgfqpoint{5.245483in}{5.271790in}}%
\pgfpathlineto{\pgfqpoint{5.195519in}{5.268027in}}%
\pgfusepath{stroke}%
\end{pgfscope}%
\begin{pgfscope}%
\pgfpathrectangle{\pgfqpoint{3.985294in}{4.155455in}}{\pgfqpoint{2.279412in}{2.004545in}}%
\pgfusepath{clip}%
\pgfsetbuttcap%
\pgfsetroundjoin%
\pgfsetlinewidth{1.099460pt}%
\definecolor{currentstroke}{rgb}{0.143343,0.522773,0.556295}%
\pgfsetstrokecolor{currentstroke}%
\pgfsetdash{}{0pt}%
\pgfpathmoveto{\pgfqpoint{5.195519in}{5.268027in}}%
\pgfpathlineto{\pgfqpoint{5.145632in}{5.263536in}}%
\pgfusepath{stroke}%
\end{pgfscope}%
\begin{pgfscope}%
\pgfpathrectangle{\pgfqpoint{3.985294in}{4.155455in}}{\pgfqpoint{2.279412in}{2.004545in}}%
\pgfusepath{clip}%
\pgfsetbuttcap%
\pgfsetroundjoin%
\pgfsetlinewidth{1.140245pt}%
\definecolor{currentstroke}{rgb}{0.135066,0.544853,0.554029}%
\pgfsetstrokecolor{currentstroke}%
\pgfsetdash{}{0pt}%
\pgfpathmoveto{\pgfqpoint{5.145632in}{5.263536in}}%
\pgfpathlineto{\pgfqpoint{5.095843in}{5.258283in}}%
\pgfusepath{stroke}%
\end{pgfscope}%
\begin{pgfscope}%
\pgfpathrectangle{\pgfqpoint{3.985294in}{4.155455in}}{\pgfqpoint{2.279412in}{2.004545in}}%
\pgfusepath{clip}%
\pgfsetbuttcap%
\pgfsetroundjoin%
\pgfsetlinewidth{1.262622pt}%
\definecolor{currentstroke}{rgb}{0.119423,0.611141,0.538982}%
\pgfsetstrokecolor{currentstroke}%
\pgfsetdash{}{0pt}%
\pgfpathmoveto{\pgfqpoint{5.095843in}{5.258283in}}%
\pgfpathlineto{\pgfqpoint{5.046237in}{5.251866in}}%
\pgfusepath{stroke}%
\end{pgfscope}%
\begin{pgfscope}%
\pgfpathrectangle{\pgfqpoint{3.985294in}{4.155455in}}{\pgfqpoint{2.279412in}{2.004545in}}%
\pgfusepath{clip}%
\pgfsetbuttcap%
\pgfsetroundjoin%
\pgfsetlinewidth{1.204529pt}%
\definecolor{currentstroke}{rgb}{0.124395,0.578002,0.548287}%
\pgfsetstrokecolor{currentstroke}%
\pgfsetdash{}{0pt}%
\pgfpathmoveto{\pgfqpoint{5.046237in}{5.251866in}}%
\pgfpathlineto{\pgfqpoint{4.996939in}{5.243883in}}%
\pgfusepath{stroke}%
\end{pgfscope}%
\begin{pgfscope}%
\pgfpathrectangle{\pgfqpoint{3.985294in}{4.155455in}}{\pgfqpoint{2.279412in}{2.004545in}}%
\pgfusepath{clip}%
\pgfsetbuttcap%
\pgfsetroundjoin%
\pgfsetlinewidth{1.175857pt}%
\definecolor{currentstroke}{rgb}{0.128729,0.563265,0.551229}%
\pgfsetstrokecolor{currentstroke}%
\pgfsetdash{}{0pt}%
\pgfpathmoveto{\pgfqpoint{4.996939in}{5.243883in}}%
\pgfpathlineto{\pgfqpoint{4.948039in}{5.234176in}}%
\pgfusepath{stroke}%
\end{pgfscope}%
\begin{pgfscope}%
\pgfpathrectangle{\pgfqpoint{3.985294in}{4.155455in}}{\pgfqpoint{2.279412in}{2.004545in}}%
\pgfusepath{clip}%
\pgfsetbuttcap%
\pgfsetroundjoin%
\pgfsetlinewidth{0.308226pt}%
\definecolor{currentstroke}{rgb}{0.268510,0.009605,0.335427}%
\pgfsetstrokecolor{currentstroke}%
\pgfsetdash{}{0pt}%
\pgfpathmoveto{\pgfqpoint{5.996962in}{5.338154in}}%
\pgfpathlineto{\pgfqpoint{5.947059in}{5.336327in}}%
\pgfusepath{stroke}%
\end{pgfscope}%
\begin{pgfscope}%
\pgfpathrectangle{\pgfqpoint{3.985294in}{4.155455in}}{\pgfqpoint{2.279412in}{2.004545in}}%
\pgfusepath{clip}%
\pgfsetbuttcap%
\pgfsetroundjoin%
\pgfsetlinewidth{0.314576pt}%
\definecolor{currentstroke}{rgb}{0.268510,0.009605,0.335427}%
\pgfsetstrokecolor{currentstroke}%
\pgfsetdash{}{0pt}%
\pgfpathmoveto{\pgfqpoint{5.947059in}{5.336327in}}%
\pgfpathlineto{\pgfqpoint{5.897016in}{5.334800in}}%
\pgfusepath{stroke}%
\end{pgfscope}%
\begin{pgfscope}%
\pgfpathrectangle{\pgfqpoint{3.985294in}{4.155455in}}{\pgfqpoint{2.279412in}{2.004545in}}%
\pgfusepath{clip}%
\pgfsetbuttcap%
\pgfsetroundjoin%
\pgfsetlinewidth{0.321728pt}%
\definecolor{currentstroke}{rgb}{0.271305,0.019942,0.347269}%
\pgfsetstrokecolor{currentstroke}%
\pgfsetdash{}{0pt}%
\pgfpathmoveto{\pgfqpoint{5.897016in}{5.334800in}}%
\pgfpathlineto{\pgfqpoint{5.846929in}{5.332851in}}%
\pgfusepath{stroke}%
\end{pgfscope}%
\begin{pgfscope}%
\pgfpathrectangle{\pgfqpoint{3.985294in}{4.155455in}}{\pgfqpoint{2.279412in}{2.004545in}}%
\pgfusepath{clip}%
\pgfsetbuttcap%
\pgfsetroundjoin%
\pgfsetlinewidth{0.342718pt}%
\definecolor{currentstroke}{rgb}{0.274952,0.037752,0.364543}%
\pgfsetstrokecolor{currentstroke}%
\pgfsetdash{}{0pt}%
\pgfpathmoveto{\pgfqpoint{5.846929in}{5.332851in}}%
\pgfpathlineto{\pgfqpoint{5.796816in}{5.331206in}}%
\pgfusepath{stroke}%
\end{pgfscope}%
\begin{pgfscope}%
\pgfpathrectangle{\pgfqpoint{3.985294in}{4.155455in}}{\pgfqpoint{2.279412in}{2.004545in}}%
\pgfusepath{clip}%
\pgfsetbuttcap%
\pgfsetroundjoin%
\pgfsetlinewidth{0.352134pt}%
\definecolor{currentstroke}{rgb}{0.276022,0.044167,0.370164}%
\pgfsetstrokecolor{currentstroke}%
\pgfsetdash{}{0pt}%
\pgfpathmoveto{\pgfqpoint{5.796816in}{5.331206in}}%
\pgfpathlineto{\pgfqpoint{5.746687in}{5.330236in}}%
\pgfusepath{stroke}%
\end{pgfscope}%
\begin{pgfscope}%
\pgfpathrectangle{\pgfqpoint{3.985294in}{4.155455in}}{\pgfqpoint{2.279412in}{2.004545in}}%
\pgfusepath{clip}%
\pgfsetbuttcap%
\pgfsetroundjoin%
\pgfsetlinewidth{0.367105pt}%
\definecolor{currentstroke}{rgb}{0.277941,0.056324,0.381191}%
\pgfsetstrokecolor{currentstroke}%
\pgfsetdash{}{0pt}%
\pgfpathmoveto{\pgfqpoint{5.746687in}{5.330236in}}%
\pgfpathlineto{\pgfqpoint{5.696559in}{5.329208in}}%
\pgfusepath{stroke}%
\end{pgfscope}%
\begin{pgfscope}%
\pgfpathrectangle{\pgfqpoint{3.985294in}{4.155455in}}{\pgfqpoint{2.279412in}{2.004545in}}%
\pgfusepath{clip}%
\pgfsetbuttcap%
\pgfsetroundjoin%
\pgfsetlinewidth{0.391386pt}%
\definecolor{currentstroke}{rgb}{0.280894,0.078907,0.402329}%
\pgfsetstrokecolor{currentstroke}%
\pgfsetdash{}{0pt}%
\pgfpathmoveto{\pgfqpoint{5.696559in}{5.329208in}}%
\pgfpathlineto{\pgfqpoint{5.646434in}{5.327852in}}%
\pgfusepath{stroke}%
\end{pgfscope}%
\begin{pgfscope}%
\pgfpathrectangle{\pgfqpoint{3.985294in}{4.155455in}}{\pgfqpoint{2.279412in}{2.004545in}}%
\pgfusepath{clip}%
\pgfsetbuttcap%
\pgfsetroundjoin%
\pgfsetlinewidth{0.425059pt}%
\definecolor{currentstroke}{rgb}{0.282910,0.105393,0.426902}%
\pgfsetstrokecolor{currentstroke}%
\pgfsetdash{}{0pt}%
\pgfpathmoveto{\pgfqpoint{5.646434in}{5.327852in}}%
\pgfpathlineto{\pgfqpoint{5.596308in}{5.326497in}}%
\pgfusepath{stroke}%
\end{pgfscope}%
\begin{pgfscope}%
\pgfpathrectangle{\pgfqpoint{3.985294in}{4.155455in}}{\pgfqpoint{2.279412in}{2.004545in}}%
\pgfusepath{clip}%
\pgfsetbuttcap%
\pgfsetroundjoin%
\pgfsetlinewidth{0.470696pt}%
\definecolor{currentstroke}{rgb}{0.282884,0.135920,0.453427}%
\pgfsetstrokecolor{currentstroke}%
\pgfsetdash{}{0pt}%
\pgfpathmoveto{\pgfqpoint{5.596308in}{5.326497in}}%
\pgfpathlineto{\pgfqpoint{5.546202in}{5.324628in}}%
\pgfusepath{stroke}%
\end{pgfscope}%
\begin{pgfscope}%
\pgfpathrectangle{\pgfqpoint{3.985294in}{4.155455in}}{\pgfqpoint{2.279412in}{2.004545in}}%
\pgfusepath{clip}%
\pgfsetbuttcap%
\pgfsetroundjoin%
\pgfsetlinewidth{0.523522pt}%
\definecolor{currentstroke}{rgb}{0.278826,0.175490,0.483397}%
\pgfsetstrokecolor{currentstroke}%
\pgfsetdash{}{0pt}%
\pgfpathmoveto{\pgfqpoint{5.546202in}{5.324628in}}%
\pgfpathlineto{\pgfqpoint{5.496105in}{5.322573in}}%
\pgfusepath{stroke}%
\end{pgfscope}%
\begin{pgfscope}%
\pgfpathrectangle{\pgfqpoint{3.985294in}{4.155455in}}{\pgfqpoint{2.279412in}{2.004545in}}%
\pgfusepath{clip}%
\pgfsetbuttcap%
\pgfsetroundjoin%
\pgfsetlinewidth{0.569014pt}%
\definecolor{currentstroke}{rgb}{0.271828,0.209303,0.504434}%
\pgfsetstrokecolor{currentstroke}%
\pgfsetdash{}{0pt}%
\pgfpathmoveto{\pgfqpoint{5.496105in}{5.322573in}}%
\pgfpathlineto{\pgfqpoint{5.446023in}{5.320267in}}%
\pgfusepath{stroke}%
\end{pgfscope}%
\begin{pgfscope}%
\pgfpathrectangle{\pgfqpoint{3.985294in}{4.155455in}}{\pgfqpoint{2.279412in}{2.004545in}}%
\pgfusepath{clip}%
\pgfsetbuttcap%
\pgfsetroundjoin%
\pgfsetlinewidth{0.654261pt}%
\definecolor{currentstroke}{rgb}{0.253935,0.265254,0.529983}%
\pgfsetstrokecolor{currentstroke}%
\pgfsetdash{}{0pt}%
\pgfpathmoveto{\pgfqpoint{5.446023in}{5.320267in}}%
\pgfpathlineto{\pgfqpoint{5.395962in}{5.317607in}}%
\pgfusepath{stroke}%
\end{pgfscope}%
\begin{pgfscope}%
\pgfpathrectangle{\pgfqpoint{3.985294in}{4.155455in}}{\pgfqpoint{2.279412in}{2.004545in}}%
\pgfusepath{clip}%
\pgfsetbuttcap%
\pgfsetroundjoin%
\pgfsetlinewidth{0.315650pt}%
\definecolor{currentstroke}{rgb}{0.269944,0.014625,0.341379}%
\pgfsetstrokecolor{currentstroke}%
\pgfsetdash{}{0pt}%
\pgfpathmoveto{\pgfqpoint{5.945670in}{4.841980in}}%
\pgfpathlineto{\pgfqpoint{5.900564in}{4.842537in}}%
\pgfusepath{stroke}%
\end{pgfscope}%
\begin{pgfscope}%
\pgfpathrectangle{\pgfqpoint{3.985294in}{4.155455in}}{\pgfqpoint{2.279412in}{2.004545in}}%
\pgfusepath{clip}%
\pgfsetbuttcap%
\pgfsetroundjoin%
\pgfsetlinewidth{0.323755pt}%
\definecolor{currentstroke}{rgb}{0.271305,0.019942,0.347269}%
\pgfsetstrokecolor{currentstroke}%
\pgfsetdash{}{0pt}%
\pgfpathmoveto{\pgfqpoint{5.900564in}{4.842537in}}%
\pgfpathlineto{\pgfqpoint{5.851156in}{4.843061in}}%
\pgfusepath{stroke}%
\end{pgfscope}%
\begin{pgfscope}%
\pgfpathrectangle{\pgfqpoint{3.985294in}{4.155455in}}{\pgfqpoint{2.279412in}{2.004545in}}%
\pgfusepath{clip}%
\pgfsetbuttcap%
\pgfsetroundjoin%
\pgfsetlinewidth{0.330659pt}%
\definecolor{currentstroke}{rgb}{0.272594,0.025563,0.353093}%
\pgfsetstrokecolor{currentstroke}%
\pgfsetdash{}{0pt}%
\pgfpathmoveto{\pgfqpoint{5.851156in}{4.843061in}}%
\pgfpathlineto{\pgfqpoint{5.801100in}{4.844655in}}%
\pgfusepath{stroke}%
\end{pgfscope}%
\begin{pgfscope}%
\pgfpathrectangle{\pgfqpoint{3.985294in}{4.155455in}}{\pgfqpoint{2.279412in}{2.004545in}}%
\pgfusepath{clip}%
\pgfsetbuttcap%
\pgfsetroundjoin%
\pgfsetlinewidth{0.331070pt}%
\definecolor{currentstroke}{rgb}{0.272594,0.025563,0.353093}%
\pgfsetstrokecolor{currentstroke}%
\pgfsetdash{}{0pt}%
\pgfpathmoveto{\pgfqpoint{5.801100in}{4.844655in}}%
\pgfpathlineto{\pgfqpoint{5.751054in}{4.846864in}}%
\pgfusepath{stroke}%
\end{pgfscope}%
\begin{pgfscope}%
\pgfpathrectangle{\pgfqpoint{3.985294in}{4.155455in}}{\pgfqpoint{2.279412in}{2.004545in}}%
\pgfusepath{clip}%
\pgfsetbuttcap%
\pgfsetroundjoin%
\pgfsetlinewidth{0.352729pt}%
\definecolor{currentstroke}{rgb}{0.276022,0.044167,0.370164}%
\pgfsetstrokecolor{currentstroke}%
\pgfsetdash{}{0pt}%
\pgfpathmoveto{\pgfqpoint{5.751054in}{4.846864in}}%
\pgfpathlineto{\pgfqpoint{5.700952in}{4.848414in}}%
\pgfusepath{stroke}%
\end{pgfscope}%
\begin{pgfscope}%
\pgfpathrectangle{\pgfqpoint{3.985294in}{4.155455in}}{\pgfqpoint{2.279412in}{2.004545in}}%
\pgfusepath{clip}%
\pgfsetbuttcap%
\pgfsetroundjoin%
\pgfsetlinewidth{0.363616pt}%
\definecolor{currentstroke}{rgb}{0.277941,0.056324,0.381191}%
\pgfsetstrokecolor{currentstroke}%
\pgfsetdash{}{0pt}%
\pgfpathmoveto{\pgfqpoint{5.700952in}{4.848414in}}%
\pgfpathlineto{\pgfqpoint{5.650868in}{4.850520in}}%
\pgfusepath{stroke}%
\end{pgfscope}%
\begin{pgfscope}%
\pgfpathrectangle{\pgfqpoint{3.985294in}{4.155455in}}{\pgfqpoint{2.279412in}{2.004545in}}%
\pgfusepath{clip}%
\pgfsetbuttcap%
\pgfsetroundjoin%
\pgfsetlinewidth{0.387641pt}%
\definecolor{currentstroke}{rgb}{0.280267,0.073417,0.397163}%
\pgfsetstrokecolor{currentstroke}%
\pgfsetdash{}{0pt}%
\pgfpathmoveto{\pgfqpoint{5.650868in}{4.850520in}}%
\pgfpathlineto{\pgfqpoint{5.600798in}{4.852988in}}%
\pgfusepath{stroke}%
\end{pgfscope}%
\begin{pgfscope}%
\pgfpathrectangle{\pgfqpoint{3.985294in}{4.155455in}}{\pgfqpoint{2.279412in}{2.004545in}}%
\pgfusepath{clip}%
\pgfsetbuttcap%
\pgfsetroundjoin%
\pgfsetlinewidth{0.407921pt}%
\definecolor{currentstroke}{rgb}{0.281924,0.089666,0.412415}%
\pgfsetstrokecolor{currentstroke}%
\pgfsetdash{}{0pt}%
\pgfpathmoveto{\pgfqpoint{5.600798in}{4.852988in}}%
\pgfpathlineto{\pgfqpoint{5.550741in}{4.855683in}}%
\pgfusepath{stroke}%
\end{pgfscope}%
\begin{pgfscope}%
\pgfpathrectangle{\pgfqpoint{3.985294in}{4.155455in}}{\pgfqpoint{2.279412in}{2.004545in}}%
\pgfusepath{clip}%
\pgfsetbuttcap%
\pgfsetroundjoin%
\pgfsetlinewidth{0.472312pt}%
\definecolor{currentstroke}{rgb}{0.282884,0.135920,0.453427}%
\pgfsetstrokecolor{currentstroke}%
\pgfsetdash{}{0pt}%
\pgfpathmoveto{\pgfqpoint{5.550741in}{4.855683in}}%
\pgfpathlineto{\pgfqpoint{5.500712in}{4.858768in}}%
\pgfusepath{stroke}%
\end{pgfscope}%
\begin{pgfscope}%
\pgfpathrectangle{\pgfqpoint{3.985294in}{4.155455in}}{\pgfqpoint{2.279412in}{2.004545in}}%
\pgfusepath{clip}%
\pgfsetbuttcap%
\pgfsetroundjoin%
\pgfsetlinewidth{0.486071pt}%
\definecolor{currentstroke}{rgb}{0.282290,0.145912,0.461510}%
\pgfsetstrokecolor{currentstroke}%
\pgfsetdash{}{0pt}%
\pgfpathmoveto{\pgfqpoint{5.500712in}{4.858768in}}%
\pgfpathlineto{\pgfqpoint{5.450733in}{4.862406in}}%
\pgfusepath{stroke}%
\end{pgfscope}%
\begin{pgfscope}%
\pgfpathrectangle{\pgfqpoint{3.985294in}{4.155455in}}{\pgfqpoint{2.279412in}{2.004545in}}%
\pgfusepath{clip}%
\pgfsetbuttcap%
\pgfsetroundjoin%
\pgfsetlinewidth{0.529770pt}%
\definecolor{currentstroke}{rgb}{0.278012,0.180367,0.486697}%
\pgfsetstrokecolor{currentstroke}%
\pgfsetdash{}{0pt}%
\pgfpathmoveto{\pgfqpoint{5.450733in}{4.862406in}}%
\pgfpathlineto{\pgfqpoint{5.400810in}{4.866609in}}%
\pgfusepath{stroke}%
\end{pgfscope}%
\begin{pgfscope}%
\pgfpathrectangle{\pgfqpoint{3.985294in}{4.155455in}}{\pgfqpoint{2.279412in}{2.004545in}}%
\pgfusepath{clip}%
\pgfsetbuttcap%
\pgfsetroundjoin%
\pgfsetlinewidth{0.588247pt}%
\definecolor{currentstroke}{rgb}{0.269308,0.218818,0.509577}%
\pgfsetstrokecolor{currentstroke}%
\pgfsetdash{}{0pt}%
\pgfpathmoveto{\pgfqpoint{5.400810in}{4.866609in}}%
\pgfpathlineto{\pgfqpoint{5.350969in}{4.871477in}}%
\pgfusepath{stroke}%
\end{pgfscope}%
\begin{pgfscope}%
\pgfpathrectangle{\pgfqpoint{3.985294in}{4.155455in}}{\pgfqpoint{2.279412in}{2.004545in}}%
\pgfusepath{clip}%
\pgfsetbuttcap%
\pgfsetroundjoin%
\pgfsetlinewidth{0.646512pt}%
\definecolor{currentstroke}{rgb}{0.255645,0.260703,0.528312}%
\pgfsetstrokecolor{currentstroke}%
\pgfsetdash{}{0pt}%
\pgfpathmoveto{\pgfqpoint{5.350969in}{4.871477in}}%
\pgfpathlineto{\pgfqpoint{5.301232in}{4.877118in}}%
\pgfusepath{stroke}%
\end{pgfscope}%
\begin{pgfscope}%
\pgfpathrectangle{\pgfqpoint{3.985294in}{4.155455in}}{\pgfqpoint{2.279412in}{2.004545in}}%
\pgfusepath{clip}%
\pgfsetbuttcap%
\pgfsetroundjoin%
\pgfsetlinewidth{0.682462pt}%
\definecolor{currentstroke}{rgb}{0.246811,0.283237,0.535941}%
\pgfsetstrokecolor{currentstroke}%
\pgfsetdash{}{0pt}%
\pgfpathmoveto{\pgfqpoint{5.301232in}{4.877118in}}%
\pgfpathlineto{\pgfqpoint{5.251608in}{4.883480in}}%
\pgfusepath{stroke}%
\end{pgfscope}%
\begin{pgfscope}%
\pgfpathrectangle{\pgfqpoint{3.985294in}{4.155455in}}{\pgfqpoint{2.279412in}{2.004545in}}%
\pgfusepath{clip}%
\pgfsetbuttcap%
\pgfsetroundjoin%
\pgfsetlinewidth{0.697437pt}%
\definecolor{currentstroke}{rgb}{0.243113,0.292092,0.538516}%
\pgfsetstrokecolor{currentstroke}%
\pgfsetdash{}{0pt}%
\pgfpathmoveto{\pgfqpoint{5.251608in}{4.883480in}}%
\pgfpathlineto{\pgfqpoint{5.202203in}{4.891004in}}%
\pgfusepath{stroke}%
\end{pgfscope}%
\begin{pgfscope}%
\pgfpathrectangle{\pgfqpoint{3.985294in}{4.155455in}}{\pgfqpoint{2.279412in}{2.004545in}}%
\pgfusepath{clip}%
\pgfsetbuttcap%
\pgfsetroundjoin%
\pgfsetlinewidth{0.709698pt}%
\definecolor{currentstroke}{rgb}{0.239346,0.300855,0.540844}%
\pgfsetstrokecolor{currentstroke}%
\pgfsetdash{}{0pt}%
\pgfpathmoveto{\pgfqpoint{5.202203in}{4.891004in}}%
\pgfpathlineto{\pgfqpoint{5.153080in}{4.899853in}}%
\pgfusepath{stroke}%
\end{pgfscope}%
\begin{pgfscope}%
\pgfpathrectangle{\pgfqpoint{3.985294in}{4.155455in}}{\pgfqpoint{2.279412in}{2.004545in}}%
\pgfusepath{clip}%
\pgfsetbuttcap%
\pgfsetroundjoin%
\pgfsetlinewidth{0.756834pt}%
\definecolor{currentstroke}{rgb}{0.225863,0.330805,0.547314}%
\pgfsetstrokecolor{currentstroke}%
\pgfsetdash{}{0pt}%
\pgfpathmoveto{\pgfqpoint{5.153080in}{4.899853in}}%
\pgfpathlineto{\pgfqpoint{5.104394in}{4.910371in}}%
\pgfusepath{stroke}%
\end{pgfscope}%
\begin{pgfscope}%
\pgfpathrectangle{\pgfqpoint{3.985294in}{4.155455in}}{\pgfqpoint{2.279412in}{2.004545in}}%
\pgfusepath{clip}%
\pgfsetbuttcap%
\pgfsetroundjoin%
\pgfsetlinewidth{0.796932pt}%
\definecolor{currentstroke}{rgb}{0.214298,0.355619,0.551184}%
\pgfsetstrokecolor{currentstroke}%
\pgfsetdash{}{0pt}%
\pgfpathmoveto{\pgfqpoint{5.104394in}{4.910371in}}%
\pgfpathlineto{\pgfqpoint{5.056816in}{4.924121in}}%
\pgfusepath{stroke}%
\end{pgfscope}%
\begin{pgfscope}%
\pgfpathrectangle{\pgfqpoint{3.985294in}{4.155455in}}{\pgfqpoint{2.279412in}{2.004545in}}%
\pgfusepath{clip}%
\pgfsetbuttcap%
\pgfsetroundjoin%
\pgfsetlinewidth{0.706691pt}%
\definecolor{currentstroke}{rgb}{0.239346,0.300855,0.540844}%
\pgfsetstrokecolor{currentstroke}%
\pgfsetdash{}{0pt}%
\pgfpathmoveto{\pgfqpoint{5.056816in}{4.924121in}}%
\pgfpathlineto{\pgfqpoint{5.010893in}{4.941687in}}%
\pgfusepath{stroke}%
\end{pgfscope}%
\begin{pgfscope}%
\pgfpathrectangle{\pgfqpoint{3.985294in}{4.155455in}}{\pgfqpoint{2.279412in}{2.004545in}}%
\pgfusepath{clip}%
\pgfsetbuttcap%
\pgfsetroundjoin%
\pgfsetlinewidth{0.810312pt}%
\definecolor{currentstroke}{rgb}{0.210503,0.363727,0.552206}%
\pgfsetstrokecolor{currentstroke}%
\pgfsetdash{}{0pt}%
\pgfpathmoveto{\pgfqpoint{5.010893in}{4.941687in}}%
\pgfpathlineto{\pgfqpoint{4.966653in}{4.962383in}}%
\pgfusepath{stroke}%
\end{pgfscope}%
\begin{pgfscope}%
\pgfpathrectangle{\pgfqpoint{3.985294in}{4.155455in}}{\pgfqpoint{2.279412in}{2.004545in}}%
\pgfusepath{clip}%
\pgfsetbuttcap%
\pgfsetroundjoin%
\pgfsetlinewidth{0.822161pt}%
\definecolor{currentstroke}{rgb}{0.208623,0.367752,0.552675}%
\pgfsetstrokecolor{currentstroke}%
\pgfsetdash{}{0pt}%
\pgfpathmoveto{\pgfqpoint{4.966653in}{4.962383in}}%
\pgfpathlineto{\pgfqpoint{4.924848in}{4.986556in}}%
\pgfusepath{stroke}%
\end{pgfscope}%
\begin{pgfscope}%
\pgfpathrectangle{\pgfqpoint{3.985294in}{4.155455in}}{\pgfqpoint{2.279412in}{2.004545in}}%
\pgfusepath{clip}%
\pgfsetbuttcap%
\pgfsetroundjoin%
\pgfsetlinewidth{1.006728pt}%
\definecolor{currentstroke}{rgb}{0.163625,0.471133,0.558148}%
\pgfsetstrokecolor{currentstroke}%
\pgfsetdash{}{0pt}%
\pgfpathmoveto{\pgfqpoint{4.924848in}{4.986556in}}%
\pgfpathlineto{\pgfqpoint{4.889016in}{5.016109in}}%
\pgfusepath{stroke}%
\end{pgfscope}%
\begin{pgfscope}%
\pgfpathrectangle{\pgfqpoint{3.985294in}{4.155455in}}{\pgfqpoint{2.279412in}{2.004545in}}%
\pgfusepath{clip}%
\pgfsetbuttcap%
\pgfsetroundjoin%
\pgfsetlinewidth{0.797530pt}%
\definecolor{currentstroke}{rgb}{0.214298,0.355619,0.551184}%
\pgfsetstrokecolor{currentstroke}%
\pgfsetdash{}{0pt}%
\pgfpathmoveto{\pgfqpoint{4.889016in}{5.016109in}}%
\pgfpathlineto{\pgfqpoint{4.857852in}{5.042943in}}%
\pgfusepath{stroke}%
\end{pgfscope}%
\begin{pgfscope}%
\pgfpathrectangle{\pgfqpoint{3.985294in}{4.155455in}}{\pgfqpoint{2.279412in}{2.004545in}}%
\pgfusepath{clip}%
\pgfsetbuttcap%
\pgfsetroundjoin%
\pgfsetlinewidth{0.865021pt}%
\definecolor{currentstroke}{rgb}{0.195860,0.395433,0.555276}%
\pgfsetstrokecolor{currentstroke}%
\pgfsetdash{}{0pt}%
\pgfpathmoveto{\pgfqpoint{4.857852in}{5.042943in}}%
\pgfpathlineto{\pgfqpoint{4.823491in}{5.070571in}}%
\pgfusepath{stroke}%
\end{pgfscope}%
\begin{pgfscope}%
\pgfpathrectangle{\pgfqpoint{3.985294in}{4.155455in}}{\pgfqpoint{2.279412in}{2.004545in}}%
\pgfusepath{clip}%
\pgfsetbuttcap%
\pgfsetroundjoin%
\pgfsetlinewidth{0.326963pt}%
\definecolor{currentstroke}{rgb}{0.271305,0.019942,0.347269}%
\pgfsetstrokecolor{currentstroke}%
\pgfsetdash{}{0pt}%
\pgfpathmoveto{\pgfqpoint{5.894378in}{4.977300in}}%
\pgfpathlineto{\pgfqpoint{5.844249in}{4.978562in}}%
\pgfusepath{stroke}%
\end{pgfscope}%
\begin{pgfscope}%
\pgfpathrectangle{\pgfqpoint{3.985294in}{4.155455in}}{\pgfqpoint{2.279412in}{2.004545in}}%
\pgfusepath{clip}%
\pgfsetbuttcap%
\pgfsetroundjoin%
\pgfsetlinewidth{0.344188pt}%
\definecolor{currentstroke}{rgb}{0.274952,0.037752,0.364543}%
\pgfsetstrokecolor{currentstroke}%
\pgfsetdash{}{0pt}%
\pgfpathmoveto{\pgfqpoint{5.844249in}{4.978562in}}%
\pgfpathlineto{\pgfqpoint{5.794126in}{4.979990in}}%
\pgfusepath{stroke}%
\end{pgfscope}%
\begin{pgfscope}%
\pgfpathrectangle{\pgfqpoint{3.985294in}{4.155455in}}{\pgfqpoint{2.279412in}{2.004545in}}%
\pgfusepath{clip}%
\pgfsetbuttcap%
\pgfsetroundjoin%
\pgfsetlinewidth{0.350286pt}%
\definecolor{currentstroke}{rgb}{0.276022,0.044167,0.370164}%
\pgfsetstrokecolor{currentstroke}%
\pgfsetdash{}{0pt}%
\pgfpathmoveto{\pgfqpoint{5.794126in}{4.979990in}}%
\pgfpathlineto{\pgfqpoint{5.744012in}{4.981603in}}%
\pgfusepath{stroke}%
\end{pgfscope}%
\begin{pgfscope}%
\pgfpathrectangle{\pgfqpoint{3.985294in}{4.155455in}}{\pgfqpoint{2.279412in}{2.004545in}}%
\pgfusepath{clip}%
\pgfsetbuttcap%
\pgfsetroundjoin%
\pgfsetlinewidth{0.356182pt}%
\definecolor{currentstroke}{rgb}{0.277018,0.050344,0.375715}%
\pgfsetstrokecolor{currentstroke}%
\pgfsetdash{}{0pt}%
\pgfpathmoveto{\pgfqpoint{5.744012in}{4.981603in}}%
\pgfpathlineto{\pgfqpoint{5.693898in}{4.983278in}}%
\pgfusepath{stroke}%
\end{pgfscope}%
\begin{pgfscope}%
\pgfpathrectangle{\pgfqpoint{3.985294in}{4.155455in}}{\pgfqpoint{2.279412in}{2.004545in}}%
\pgfusepath{clip}%
\pgfsetbuttcap%
\pgfsetroundjoin%
\pgfsetlinewidth{0.395104pt}%
\definecolor{currentstroke}{rgb}{0.280894,0.078907,0.402329}%
\pgfsetstrokecolor{currentstroke}%
\pgfsetdash{}{0pt}%
\pgfpathmoveto{\pgfqpoint{5.693898in}{4.983278in}}%
\pgfpathlineto{\pgfqpoint{5.643781in}{4.984906in}}%
\pgfusepath{stroke}%
\end{pgfscope}%
\begin{pgfscope}%
\pgfpathrectangle{\pgfqpoint{3.985294in}{4.155455in}}{\pgfqpoint{2.279412in}{2.004545in}}%
\pgfusepath{clip}%
\pgfsetbuttcap%
\pgfsetroundjoin%
\pgfsetlinewidth{0.428818pt}%
\definecolor{currentstroke}{rgb}{0.282910,0.105393,0.426902}%
\pgfsetstrokecolor{currentstroke}%
\pgfsetdash{}{0pt}%
\pgfpathmoveto{\pgfqpoint{5.643781in}{4.984906in}}%
\pgfpathlineto{\pgfqpoint{5.593672in}{4.986732in}}%
\pgfusepath{stroke}%
\end{pgfscope}%
\begin{pgfscope}%
\pgfpathrectangle{\pgfqpoint{3.985294in}{4.155455in}}{\pgfqpoint{2.279412in}{2.004545in}}%
\pgfusepath{clip}%
\pgfsetbuttcap%
\pgfsetroundjoin%
\pgfsetlinewidth{0.467141pt}%
\definecolor{currentstroke}{rgb}{0.282884,0.135920,0.453427}%
\pgfsetstrokecolor{currentstroke}%
\pgfsetdash{}{0pt}%
\pgfpathmoveto{\pgfqpoint{5.593672in}{4.986732in}}%
\pgfpathlineto{\pgfqpoint{5.543574in}{4.988760in}}%
\pgfusepath{stroke}%
\end{pgfscope}%
\begin{pgfscope}%
\pgfpathrectangle{\pgfqpoint{3.985294in}{4.155455in}}{\pgfqpoint{2.279412in}{2.004545in}}%
\pgfusepath{clip}%
\pgfsetbuttcap%
\pgfsetroundjoin%
\pgfsetlinewidth{0.535697pt}%
\definecolor{currentstroke}{rgb}{0.277134,0.185228,0.489898}%
\pgfsetstrokecolor{currentstroke}%
\pgfsetdash{}{0pt}%
\pgfpathmoveto{\pgfqpoint{5.543574in}{4.988760in}}%
\pgfpathlineto{\pgfqpoint{5.493481in}{4.990898in}}%
\pgfusepath{stroke}%
\end{pgfscope}%
\begin{pgfscope}%
\pgfpathrectangle{\pgfqpoint{3.985294in}{4.155455in}}{\pgfqpoint{2.279412in}{2.004545in}}%
\pgfusepath{clip}%
\pgfsetbuttcap%
\pgfsetroundjoin%
\pgfsetlinewidth{0.582445pt}%
\definecolor{currentstroke}{rgb}{0.270595,0.214069,0.507052}%
\pgfsetstrokecolor{currentstroke}%
\pgfsetdash{}{0pt}%
\pgfpathmoveto{\pgfqpoint{5.493481in}{4.990898in}}%
\pgfpathlineto{\pgfqpoint{5.443396in}{4.993167in}}%
\pgfusepath{stroke}%
\end{pgfscope}%
\begin{pgfscope}%
\pgfpathrectangle{\pgfqpoint{3.985294in}{4.155455in}}{\pgfqpoint{2.279412in}{2.004545in}}%
\pgfusepath{clip}%
\pgfsetbuttcap%
\pgfsetroundjoin%
\pgfsetlinewidth{0.692424pt}%
\definecolor{currentstroke}{rgb}{0.243113,0.292092,0.538516}%
\pgfsetstrokecolor{currentstroke}%
\pgfsetdash{}{0pt}%
\pgfpathmoveto{\pgfqpoint{5.443396in}{4.993167in}}%
\pgfpathlineto{\pgfqpoint{5.393327in}{4.995690in}}%
\pgfusepath{stroke}%
\end{pgfscope}%
\begin{pgfscope}%
\pgfpathrectangle{\pgfqpoint{3.985294in}{4.155455in}}{\pgfqpoint{2.279412in}{2.004545in}}%
\pgfusepath{clip}%
\pgfsetbuttcap%
\pgfsetroundjoin%
\pgfsetlinewidth{0.714902pt}%
\definecolor{currentstroke}{rgb}{0.237441,0.305202,0.541921}%
\pgfsetstrokecolor{currentstroke}%
\pgfsetdash{}{0pt}%
\pgfpathmoveto{\pgfqpoint{5.393327in}{4.995690in}}%
\pgfpathlineto{\pgfqpoint{5.343284in}{4.998582in}}%
\pgfusepath{stroke}%
\end{pgfscope}%
\begin{pgfscope}%
\pgfpathrectangle{\pgfqpoint{3.985294in}{4.155455in}}{\pgfqpoint{2.279412in}{2.004545in}}%
\pgfusepath{clip}%
\pgfsetbuttcap%
\pgfsetroundjoin%
\pgfsetlinewidth{0.332470pt}%
\definecolor{currentstroke}{rgb}{0.272594,0.025563,0.353093}%
\pgfsetstrokecolor{currentstroke}%
\pgfsetdash{}{0pt}%
\pgfpathmoveto{\pgfqpoint{5.894378in}{5.112620in}}%
\pgfpathlineto{\pgfqpoint{5.844240in}{5.111982in}}%
\pgfusepath{stroke}%
\end{pgfscope}%
\begin{pgfscope}%
\pgfpathrectangle{\pgfqpoint{3.985294in}{4.155455in}}{\pgfqpoint{2.279412in}{2.004545in}}%
\pgfusepath{clip}%
\pgfsetbuttcap%
\pgfsetroundjoin%
\pgfsetlinewidth{0.338307pt}%
\definecolor{currentstroke}{rgb}{0.273809,0.031497,0.358853}%
\pgfsetstrokecolor{currentstroke}%
\pgfsetdash{}{0pt}%
\pgfpathmoveto{\pgfqpoint{5.844240in}{5.111982in}}%
\pgfpathlineto{\pgfqpoint{5.794097in}{5.111911in}}%
\pgfusepath{stroke}%
\end{pgfscope}%
\begin{pgfscope}%
\pgfpathrectangle{\pgfqpoint{3.985294in}{4.155455in}}{\pgfqpoint{2.279412in}{2.004545in}}%
\pgfusepath{clip}%
\pgfsetbuttcap%
\pgfsetroundjoin%
\pgfsetlinewidth{0.350460pt}%
\definecolor{currentstroke}{rgb}{0.276022,0.044167,0.370164}%
\pgfsetstrokecolor{currentstroke}%
\pgfsetdash{}{0pt}%
\pgfpathmoveto{\pgfqpoint{5.794097in}{5.111911in}}%
\pgfpathlineto{\pgfqpoint{5.743949in}{5.112173in}}%
\pgfusepath{stroke}%
\end{pgfscope}%
\begin{pgfscope}%
\pgfpathrectangle{\pgfqpoint{3.985294in}{4.155455in}}{\pgfqpoint{2.279412in}{2.004545in}}%
\pgfusepath{clip}%
\pgfsetbuttcap%
\pgfsetroundjoin%
\pgfsetlinewidth{0.378910pt}%
\definecolor{currentstroke}{rgb}{0.279566,0.067836,0.391917}%
\pgfsetstrokecolor{currentstroke}%
\pgfsetdash{}{0pt}%
\pgfpathmoveto{\pgfqpoint{5.743949in}{5.112173in}}%
\pgfpathlineto{\pgfqpoint{5.693802in}{5.112621in}}%
\pgfusepath{stroke}%
\end{pgfscope}%
\begin{pgfscope}%
\pgfpathrectangle{\pgfqpoint{3.985294in}{4.155455in}}{\pgfqpoint{2.279412in}{2.004545in}}%
\pgfusepath{clip}%
\pgfsetbuttcap%
\pgfsetroundjoin%
\pgfsetlinewidth{0.416722pt}%
\definecolor{currentstroke}{rgb}{0.282327,0.094955,0.417331}%
\pgfsetstrokecolor{currentstroke}%
\pgfsetdash{}{0pt}%
\pgfpathmoveto{\pgfqpoint{5.693802in}{5.112621in}}%
\pgfpathlineto{\pgfqpoint{5.643655in}{5.113155in}}%
\pgfusepath{stroke}%
\end{pgfscope}%
\begin{pgfscope}%
\pgfpathrectangle{\pgfqpoint{3.985294in}{4.155455in}}{\pgfqpoint{2.279412in}{2.004545in}}%
\pgfusepath{clip}%
\pgfsetbuttcap%
\pgfsetroundjoin%
\pgfsetlinewidth{0.470418pt}%
\definecolor{currentstroke}{rgb}{0.282884,0.135920,0.453427}%
\pgfsetstrokecolor{currentstroke}%
\pgfsetdash{}{0pt}%
\pgfpathmoveto{\pgfqpoint{5.643655in}{5.113155in}}%
\pgfpathlineto{\pgfqpoint{5.593506in}{5.113530in}}%
\pgfusepath{stroke}%
\end{pgfscope}%
\begin{pgfscope}%
\pgfpathrectangle{\pgfqpoint{3.985294in}{4.155455in}}{\pgfqpoint{2.279412in}{2.004545in}}%
\pgfusepath{clip}%
\pgfsetbuttcap%
\pgfsetroundjoin%
\pgfsetlinewidth{0.524568pt}%
\definecolor{currentstroke}{rgb}{0.278826,0.175490,0.483397}%
\pgfsetstrokecolor{currentstroke}%
\pgfsetdash{}{0pt}%
\pgfpathmoveto{\pgfqpoint{5.593506in}{5.113530in}}%
\pgfpathlineto{\pgfqpoint{5.543357in}{5.113974in}}%
\pgfusepath{stroke}%
\end{pgfscope}%
\begin{pgfscope}%
\pgfpathrectangle{\pgfqpoint{3.985294in}{4.155455in}}{\pgfqpoint{2.279412in}{2.004545in}}%
\pgfusepath{clip}%
\pgfsetbuttcap%
\pgfsetroundjoin%
\pgfsetlinewidth{0.568974pt}%
\definecolor{currentstroke}{rgb}{0.271828,0.209303,0.504434}%
\pgfsetstrokecolor{currentstroke}%
\pgfsetdash{}{0pt}%
\pgfpathmoveto{\pgfqpoint{5.543357in}{5.113974in}}%
\pgfpathlineto{\pgfqpoint{5.493207in}{5.114362in}}%
\pgfusepath{stroke}%
\end{pgfscope}%
\begin{pgfscope}%
\pgfpathrectangle{\pgfqpoint{3.985294in}{4.155455in}}{\pgfqpoint{2.279412in}{2.004545in}}%
\pgfusepath{clip}%
\pgfsetbuttcap%
\pgfsetroundjoin%
\pgfsetlinewidth{0.699173pt}%
\definecolor{currentstroke}{rgb}{0.241237,0.296485,0.539709}%
\pgfsetstrokecolor{currentstroke}%
\pgfsetdash{}{0pt}%
\pgfpathmoveto{\pgfqpoint{5.493207in}{5.114362in}}%
\pgfpathlineto{\pgfqpoint{5.443058in}{5.114826in}}%
\pgfusepath{stroke}%
\end{pgfscope}%
\begin{pgfscope}%
\pgfpathrectangle{\pgfqpoint{3.985294in}{4.155455in}}{\pgfqpoint{2.279412in}{2.004545in}}%
\pgfusepath{clip}%
\pgfsetbuttcap%
\pgfsetroundjoin%
\pgfsetlinewidth{0.793142pt}%
\definecolor{currentstroke}{rgb}{0.216210,0.351535,0.550627}%
\pgfsetstrokecolor{currentstroke}%
\pgfsetdash{}{0pt}%
\pgfpathmoveto{\pgfqpoint{5.443058in}{5.114826in}}%
\pgfpathlineto{\pgfqpoint{5.392911in}{5.115455in}}%
\pgfusepath{stroke}%
\end{pgfscope}%
\begin{pgfscope}%
\pgfpathrectangle{\pgfqpoint{3.985294in}{4.155455in}}{\pgfqpoint{2.279412in}{2.004545in}}%
\pgfusepath{clip}%
\pgfsetbuttcap%
\pgfsetroundjoin%
\pgfsetlinewidth{0.905987pt}%
\definecolor{currentstroke}{rgb}{0.185556,0.418570,0.556753}%
\pgfsetstrokecolor{currentstroke}%
\pgfsetdash{}{0pt}%
\pgfpathmoveto{\pgfqpoint{5.392911in}{5.115455in}}%
\pgfpathlineto{\pgfqpoint{5.342769in}{5.116294in}}%
\pgfusepath{stroke}%
\end{pgfscope}%
\begin{pgfscope}%
\pgfpathrectangle{\pgfqpoint{3.985294in}{4.155455in}}{\pgfqpoint{2.279412in}{2.004545in}}%
\pgfusepath{clip}%
\pgfsetbuttcap%
\pgfsetroundjoin%
\pgfsetlinewidth{1.007675pt}%
\definecolor{currentstroke}{rgb}{0.163625,0.471133,0.558148}%
\pgfsetstrokecolor{currentstroke}%
\pgfsetdash{}{0pt}%
\pgfpathmoveto{\pgfqpoint{5.342769in}{5.116294in}}%
\pgfpathlineto{\pgfqpoint{5.292628in}{5.117232in}}%
\pgfusepath{stroke}%
\end{pgfscope}%
\begin{pgfscope}%
\pgfpathrectangle{\pgfqpoint{3.985294in}{4.155455in}}{\pgfqpoint{2.279412in}{2.004545in}}%
\pgfusepath{clip}%
\pgfsetbuttcap%
\pgfsetroundjoin%
\pgfsetlinewidth{1.136319pt}%
\definecolor{currentstroke}{rgb}{0.136408,0.541173,0.554483}%
\pgfsetstrokecolor{currentstroke}%
\pgfsetdash{}{0pt}%
\pgfpathmoveto{\pgfqpoint{5.292628in}{5.117232in}}%
\pgfpathlineto{\pgfqpoint{5.242491in}{5.118286in}}%
\pgfusepath{stroke}%
\end{pgfscope}%
\begin{pgfscope}%
\pgfpathrectangle{\pgfqpoint{3.985294in}{4.155455in}}{\pgfqpoint{2.279412in}{2.004545in}}%
\pgfusepath{clip}%
\pgfsetbuttcap%
\pgfsetroundjoin%
\pgfsetlinewidth{1.316148pt}%
\definecolor{currentstroke}{rgb}{0.123444,0.636809,0.528763}%
\pgfsetstrokecolor{currentstroke}%
\pgfsetdash{}{0pt}%
\pgfpathmoveto{\pgfqpoint{5.242491in}{5.118286in}}%
\pgfpathlineto{\pgfqpoint{5.192359in}{5.119521in}}%
\pgfusepath{stroke}%
\end{pgfscope}%
\begin{pgfscope}%
\pgfpathrectangle{\pgfqpoint{3.985294in}{4.155455in}}{\pgfqpoint{2.279412in}{2.004545in}}%
\pgfusepath{clip}%
\pgfsetbuttcap%
\pgfsetroundjoin%
\pgfsetlinewidth{1.409475pt}%
\definecolor{currentstroke}{rgb}{0.162016,0.687316,0.499129}%
\pgfsetstrokecolor{currentstroke}%
\pgfsetdash{}{0pt}%
\pgfpathmoveto{\pgfqpoint{5.192359in}{5.119521in}}%
\pgfpathlineto{\pgfqpoint{5.142235in}{5.120974in}}%
\pgfusepath{stroke}%
\end{pgfscope}%
\begin{pgfscope}%
\pgfpathrectangle{\pgfqpoint{3.985294in}{4.155455in}}{\pgfqpoint{2.279412in}{2.004545in}}%
\pgfusepath{clip}%
\pgfsetbuttcap%
\pgfsetroundjoin%
\pgfsetlinewidth{1.533328pt}%
\definecolor{currentstroke}{rgb}{0.266941,0.748751,0.440573}%
\pgfsetstrokecolor{currentstroke}%
\pgfsetdash{}{0pt}%
\pgfpathmoveto{\pgfqpoint{5.142235in}{5.120974in}}%
\pgfpathlineto{\pgfqpoint{5.092129in}{5.122775in}}%
\pgfusepath{stroke}%
\end{pgfscope}%
\begin{pgfscope}%
\pgfpathrectangle{\pgfqpoint{3.985294in}{4.155455in}}{\pgfqpoint{2.279412in}{2.004545in}}%
\pgfusepath{clip}%
\pgfsetbuttcap%
\pgfsetroundjoin%
\pgfsetlinewidth{1.578885pt}%
\definecolor{currentstroke}{rgb}{0.319809,0.770914,0.411152}%
\pgfsetstrokecolor{currentstroke}%
\pgfsetdash{}{0pt}%
\pgfpathmoveto{\pgfqpoint{5.092129in}{5.122775in}}%
\pgfpathlineto{\pgfqpoint{5.042041in}{5.124892in}}%
\pgfusepath{stroke}%
\end{pgfscope}%
\begin{pgfscope}%
\pgfpathrectangle{\pgfqpoint{3.985294in}{4.155455in}}{\pgfqpoint{2.279412in}{2.004545in}}%
\pgfusepath{clip}%
\pgfsetbuttcap%
\pgfsetroundjoin%
\pgfsetlinewidth{1.577983pt}%
\definecolor{currentstroke}{rgb}{0.319809,0.770914,0.411152}%
\pgfsetstrokecolor{currentstroke}%
\pgfsetdash{}{0pt}%
\pgfpathmoveto{\pgfqpoint{5.042041in}{5.124892in}}%
\pgfpathlineto{\pgfqpoint{4.991988in}{5.127473in}}%
\pgfusepath{stroke}%
\end{pgfscope}%
\begin{pgfscope}%
\pgfpathrectangle{\pgfqpoint{3.985294in}{4.155455in}}{\pgfqpoint{2.279412in}{2.004545in}}%
\pgfusepath{clip}%
\pgfsetbuttcap%
\pgfsetroundjoin%
\pgfsetlinewidth{0.328728pt}%
\definecolor{currentstroke}{rgb}{0.272594,0.025563,0.353093}%
\pgfsetstrokecolor{currentstroke}%
\pgfsetdash{}{0pt}%
\pgfpathmoveto{\pgfqpoint{5.894378in}{5.247941in}}%
\pgfpathlineto{\pgfqpoint{5.844276in}{5.248447in}}%
\pgfusepath{stroke}%
\end{pgfscope}%
\begin{pgfscope}%
\pgfpathrectangle{\pgfqpoint{3.985294in}{4.155455in}}{\pgfqpoint{2.279412in}{2.004545in}}%
\pgfusepath{clip}%
\pgfsetbuttcap%
\pgfsetroundjoin%
\pgfsetlinewidth{0.337263pt}%
\definecolor{currentstroke}{rgb}{0.273809,0.031497,0.358853}%
\pgfsetstrokecolor{currentstroke}%
\pgfsetdash{}{0pt}%
\pgfpathmoveto{\pgfqpoint{5.844276in}{5.248447in}}%
\pgfpathlineto{\pgfqpoint{5.794144in}{5.247800in}}%
\pgfusepath{stroke}%
\end{pgfscope}%
\begin{pgfscope}%
\pgfpathrectangle{\pgfqpoint{3.985294in}{4.155455in}}{\pgfqpoint{2.279412in}{2.004545in}}%
\pgfusepath{clip}%
\pgfsetbuttcap%
\pgfsetroundjoin%
\pgfsetlinewidth{0.353037pt}%
\definecolor{currentstroke}{rgb}{0.276022,0.044167,0.370164}%
\pgfsetstrokecolor{currentstroke}%
\pgfsetdash{}{0pt}%
\pgfpathmoveto{\pgfqpoint{5.794144in}{5.247800in}}%
\pgfpathlineto{\pgfqpoint{5.744000in}{5.247461in}}%
\pgfusepath{stroke}%
\end{pgfscope}%
\begin{pgfscope}%
\pgfpathrectangle{\pgfqpoint{3.985294in}{4.155455in}}{\pgfqpoint{2.279412in}{2.004545in}}%
\pgfusepath{clip}%
\pgfsetbuttcap%
\pgfsetroundjoin%
\pgfsetlinewidth{0.373645pt}%
\definecolor{currentstroke}{rgb}{0.278791,0.062145,0.386592}%
\pgfsetstrokecolor{currentstroke}%
\pgfsetdash{}{0pt}%
\pgfpathmoveto{\pgfqpoint{5.744000in}{5.247461in}}%
\pgfpathlineto{\pgfqpoint{5.693861in}{5.246668in}}%
\pgfusepath{stroke}%
\end{pgfscope}%
\begin{pgfscope}%
\pgfpathrectangle{\pgfqpoint{3.985294in}{4.155455in}}{\pgfqpoint{2.279412in}{2.004545in}}%
\pgfusepath{clip}%
\pgfsetbuttcap%
\pgfsetroundjoin%
\pgfsetlinewidth{0.393793pt}%
\definecolor{currentstroke}{rgb}{0.280894,0.078907,0.402329}%
\pgfsetstrokecolor{currentstroke}%
\pgfsetdash{}{0pt}%
\pgfpathmoveto{\pgfqpoint{5.693861in}{5.246668in}}%
\pgfpathlineto{\pgfqpoint{5.643725in}{5.245642in}}%
\pgfusepath{stroke}%
\end{pgfscope}%
\begin{pgfscope}%
\pgfpathrectangle{\pgfqpoint{3.985294in}{4.155455in}}{\pgfqpoint{2.279412in}{2.004545in}}%
\pgfusepath{clip}%
\pgfsetbuttcap%
\pgfsetroundjoin%
\pgfsetlinewidth{0.443937pt}%
\definecolor{currentstroke}{rgb}{0.283197,0.115680,0.436115}%
\pgfsetstrokecolor{currentstroke}%
\pgfsetdash{}{0pt}%
\pgfpathmoveto{\pgfqpoint{5.643725in}{5.245642in}}%
\pgfpathlineto{\pgfqpoint{5.593583in}{5.244794in}}%
\pgfusepath{stroke}%
\end{pgfscope}%
\begin{pgfscope}%
\pgfpathrectangle{\pgfqpoint{3.985294in}{4.155455in}}{\pgfqpoint{2.279412in}{2.004545in}}%
\pgfusepath{clip}%
\pgfsetbuttcap%
\pgfsetroundjoin%
\pgfsetlinewidth{0.510461pt}%
\definecolor{currentstroke}{rgb}{0.280255,0.165693,0.476498}%
\pgfsetstrokecolor{currentstroke}%
\pgfsetdash{}{0pt}%
\pgfpathmoveto{\pgfqpoint{5.593583in}{5.244794in}}%
\pgfpathlineto{\pgfqpoint{5.543446in}{5.243777in}}%
\pgfusepath{stroke}%
\end{pgfscope}%
\begin{pgfscope}%
\pgfpathrectangle{\pgfqpoint{3.985294in}{4.155455in}}{\pgfqpoint{2.279412in}{2.004545in}}%
\pgfusepath{clip}%
\pgfsetbuttcap%
\pgfsetroundjoin%
\pgfsetlinewidth{0.566474pt}%
\definecolor{currentstroke}{rgb}{0.273006,0.204520,0.501721}%
\pgfsetstrokecolor{currentstroke}%
\pgfsetdash{}{0pt}%
\pgfpathmoveto{\pgfqpoint{5.543446in}{5.243777in}}%
\pgfpathlineto{\pgfqpoint{5.493315in}{5.242496in}}%
\pgfusepath{stroke}%
\end{pgfscope}%
\begin{pgfscope}%
\pgfpathrectangle{\pgfqpoint{3.985294in}{4.155455in}}{\pgfqpoint{2.279412in}{2.004545in}}%
\pgfusepath{clip}%
\pgfsetbuttcap%
\pgfsetroundjoin%
\pgfsetlinewidth{0.658898pt}%
\definecolor{currentstroke}{rgb}{0.252194,0.269783,0.531579}%
\pgfsetstrokecolor{currentstroke}%
\pgfsetdash{}{0pt}%
\pgfpathmoveto{\pgfqpoint{5.493315in}{5.242496in}}%
\pgfpathlineto{\pgfqpoint{5.443185in}{5.241190in}}%
\pgfusepath{stroke}%
\end{pgfscope}%
\begin{pgfscope}%
\pgfpathrectangle{\pgfqpoint{3.985294in}{4.155455in}}{\pgfqpoint{2.279412in}{2.004545in}}%
\pgfusepath{clip}%
\pgfsetbuttcap%
\pgfsetroundjoin%
\pgfsetlinewidth{0.730542pt}%
\definecolor{currentstroke}{rgb}{0.233603,0.313828,0.543914}%
\pgfsetstrokecolor{currentstroke}%
\pgfsetdash{}{0pt}%
\pgfpathmoveto{\pgfqpoint{5.443185in}{5.241190in}}%
\pgfpathlineto{\pgfqpoint{5.393064in}{5.239665in}}%
\pgfusepath{stroke}%
\end{pgfscope}%
\begin{pgfscope}%
\pgfpathrectangle{\pgfqpoint{3.985294in}{4.155455in}}{\pgfqpoint{2.279412in}{2.004545in}}%
\pgfusepath{clip}%
\pgfsetbuttcap%
\pgfsetroundjoin%
\pgfsetlinewidth{0.816320pt}%
\definecolor{currentstroke}{rgb}{0.208623,0.367752,0.552675}%
\pgfsetstrokecolor{currentstroke}%
\pgfsetdash{}{0pt}%
\pgfpathmoveto{\pgfqpoint{5.393064in}{5.239665in}}%
\pgfpathlineto{\pgfqpoint{5.342954in}{5.237858in}}%
\pgfusepath{stroke}%
\end{pgfscope}%
\begin{pgfscope}%
\pgfpathrectangle{\pgfqpoint{3.985294in}{4.155455in}}{\pgfqpoint{2.279412in}{2.004545in}}%
\pgfusepath{clip}%
\pgfsetbuttcap%
\pgfsetroundjoin%
\pgfsetlinewidth{0.946282pt}%
\definecolor{currentstroke}{rgb}{0.175841,0.441290,0.557685}%
\pgfsetstrokecolor{currentstroke}%
\pgfsetdash{}{0pt}%
\pgfpathmoveto{\pgfqpoint{5.342954in}{5.237858in}}%
\pgfpathlineto{\pgfqpoint{5.292854in}{5.235870in}}%
\pgfusepath{stroke}%
\end{pgfscope}%
\begin{pgfscope}%
\pgfpathrectangle{\pgfqpoint{3.985294in}{4.155455in}}{\pgfqpoint{2.279412in}{2.004545in}}%
\pgfusepath{clip}%
\pgfsetbuttcap%
\pgfsetroundjoin%
\pgfsetlinewidth{1.036774pt}%
\definecolor{currentstroke}{rgb}{0.156270,0.489624,0.557936}%
\pgfsetstrokecolor{currentstroke}%
\pgfsetdash{}{0pt}%
\pgfpathmoveto{\pgfqpoint{5.292854in}{5.235870in}}%
\pgfpathlineto{\pgfqpoint{5.242775in}{5.233503in}}%
\pgfusepath{stroke}%
\end{pgfscope}%
\begin{pgfscope}%
\pgfpathrectangle{\pgfqpoint{3.985294in}{4.155455in}}{\pgfqpoint{2.279412in}{2.004545in}}%
\pgfusepath{clip}%
\pgfsetbuttcap%
\pgfsetroundjoin%
\pgfsetlinewidth{0.320856pt}%
\definecolor{currentstroke}{rgb}{0.269944,0.014625,0.341379}%
\pgfsetstrokecolor{currentstroke}%
\pgfsetdash{}{0pt}%
\pgfpathmoveto{\pgfqpoint{5.894378in}{5.383261in}}%
\pgfpathlineto{\pgfqpoint{5.844277in}{5.381382in}}%
\pgfusepath{stroke}%
\end{pgfscope}%
\begin{pgfscope}%
\pgfpathrectangle{\pgfqpoint{3.985294in}{4.155455in}}{\pgfqpoint{2.279412in}{2.004545in}}%
\pgfusepath{clip}%
\pgfsetbuttcap%
\pgfsetroundjoin%
\pgfsetlinewidth{0.335084pt}%
\definecolor{currentstroke}{rgb}{0.272594,0.025563,0.353093}%
\pgfsetstrokecolor{currentstroke}%
\pgfsetdash{}{0pt}%
\pgfpathmoveto{\pgfqpoint{5.844277in}{5.381382in}}%
\pgfpathlineto{\pgfqpoint{5.794169in}{5.379629in}}%
\pgfusepath{stroke}%
\end{pgfscope}%
\begin{pgfscope}%
\pgfpathrectangle{\pgfqpoint{3.985294in}{4.155455in}}{\pgfqpoint{2.279412in}{2.004545in}}%
\pgfusepath{clip}%
\pgfsetbuttcap%
\pgfsetroundjoin%
\pgfsetlinewidth{0.343990pt}%
\definecolor{currentstroke}{rgb}{0.274952,0.037752,0.364543}%
\pgfsetstrokecolor{currentstroke}%
\pgfsetdash{}{0pt}%
\pgfpathmoveto{\pgfqpoint{5.794169in}{5.379629in}}%
\pgfpathlineto{\pgfqpoint{5.744046in}{5.378237in}}%
\pgfusepath{stroke}%
\end{pgfscope}%
\begin{pgfscope}%
\pgfpathrectangle{\pgfqpoint{3.985294in}{4.155455in}}{\pgfqpoint{2.279412in}{2.004545in}}%
\pgfusepath{clip}%
\pgfsetbuttcap%
\pgfsetroundjoin%
\pgfsetlinewidth{0.355653pt}%
\definecolor{currentstroke}{rgb}{0.276022,0.044167,0.370164}%
\pgfsetstrokecolor{currentstroke}%
\pgfsetdash{}{0pt}%
\pgfpathmoveto{\pgfqpoint{5.744046in}{5.378237in}}%
\pgfpathlineto{\pgfqpoint{5.693935in}{5.376527in}}%
\pgfusepath{stroke}%
\end{pgfscope}%
\begin{pgfscope}%
\pgfpathrectangle{\pgfqpoint{3.985294in}{4.155455in}}{\pgfqpoint{2.279412in}{2.004545in}}%
\pgfusepath{clip}%
\pgfsetbuttcap%
\pgfsetroundjoin%
\pgfsetlinewidth{0.386694pt}%
\definecolor{currentstroke}{rgb}{0.280267,0.073417,0.397163}%
\pgfsetstrokecolor{currentstroke}%
\pgfsetdash{}{0pt}%
\pgfpathmoveto{\pgfqpoint{5.693935in}{5.376527in}}%
\pgfpathlineto{\pgfqpoint{5.643844in}{5.374361in}}%
\pgfusepath{stroke}%
\end{pgfscope}%
\begin{pgfscope}%
\pgfpathrectangle{\pgfqpoint{3.985294in}{4.155455in}}{\pgfqpoint{2.279412in}{2.004545in}}%
\pgfusepath{clip}%
\pgfsetbuttcap%
\pgfsetroundjoin%
\pgfsetlinewidth{0.421717pt}%
\definecolor{currentstroke}{rgb}{0.282656,0.100196,0.422160}%
\pgfsetstrokecolor{currentstroke}%
\pgfsetdash{}{0pt}%
\pgfpathmoveto{\pgfqpoint{5.643844in}{5.374361in}}%
\pgfpathlineto{\pgfqpoint{5.593759in}{5.372077in}}%
\pgfusepath{stroke}%
\end{pgfscope}%
\begin{pgfscope}%
\pgfpathrectangle{\pgfqpoint{3.985294in}{4.155455in}}{\pgfqpoint{2.279412in}{2.004545in}}%
\pgfusepath{clip}%
\pgfsetbuttcap%
\pgfsetroundjoin%
\pgfsetlinewidth{0.449624pt}%
\definecolor{currentstroke}{rgb}{0.283229,0.120777,0.440584}%
\pgfsetstrokecolor{currentstroke}%
\pgfsetdash{}{0pt}%
\pgfpathmoveto{\pgfqpoint{5.593759in}{5.372077in}}%
\pgfpathlineto{\pgfqpoint{5.543680in}{5.369705in}}%
\pgfusepath{stroke}%
\end{pgfscope}%
\begin{pgfscope}%
\pgfpathrectangle{\pgfqpoint{3.985294in}{4.155455in}}{\pgfqpoint{2.279412in}{2.004545in}}%
\pgfusepath{clip}%
\pgfsetbuttcap%
\pgfsetroundjoin%
\pgfsetlinewidth{0.495056pt}%
\definecolor{currentstroke}{rgb}{0.281412,0.155834,0.469201}%
\pgfsetstrokecolor{currentstroke}%
\pgfsetdash{}{0pt}%
\pgfpathmoveto{\pgfqpoint{5.543680in}{5.369705in}}%
\pgfpathlineto{\pgfqpoint{5.493605in}{5.367279in}}%
\pgfusepath{stroke}%
\end{pgfscope}%
\begin{pgfscope}%
\pgfpathrectangle{\pgfqpoint{3.985294in}{4.155455in}}{\pgfqpoint{2.279412in}{2.004545in}}%
\pgfusepath{clip}%
\pgfsetbuttcap%
\pgfsetroundjoin%
\pgfsetlinewidth{0.549845pt}%
\definecolor{currentstroke}{rgb}{0.275191,0.194905,0.496005}%
\pgfsetstrokecolor{currentstroke}%
\pgfsetdash{}{0pt}%
\pgfpathmoveto{\pgfqpoint{5.493605in}{5.367279in}}%
\pgfpathlineto{\pgfqpoint{5.443547in}{5.364613in}}%
\pgfusepath{stroke}%
\end{pgfscope}%
\begin{pgfscope}%
\pgfpathrectangle{\pgfqpoint{3.985294in}{4.155455in}}{\pgfqpoint{2.279412in}{2.004545in}}%
\pgfusepath{clip}%
\pgfsetbuttcap%
\pgfsetroundjoin%
\pgfsetlinewidth{0.619339pt}%
\definecolor{currentstroke}{rgb}{0.262138,0.242286,0.520837}%
\pgfsetstrokecolor{currentstroke}%
\pgfsetdash{}{0pt}%
\pgfpathmoveto{\pgfqpoint{5.443547in}{5.364613in}}%
\pgfpathlineto{\pgfqpoint{5.393534in}{5.361342in}}%
\pgfusepath{stroke}%
\end{pgfscope}%
\begin{pgfscope}%
\pgfpathrectangle{\pgfqpoint{3.985294in}{4.155455in}}{\pgfqpoint{2.279412in}{2.004545in}}%
\pgfusepath{clip}%
\pgfsetbuttcap%
\pgfsetroundjoin%
\pgfsetlinewidth{0.650271pt}%
\definecolor{currentstroke}{rgb}{0.255645,0.260703,0.528312}%
\pgfsetstrokecolor{currentstroke}%
\pgfsetdash{}{0pt}%
\pgfpathmoveto{\pgfqpoint{5.393534in}{5.361342in}}%
\pgfpathlineto{\pgfqpoint{5.343573in}{5.357511in}}%
\pgfusepath{stroke}%
\end{pgfscope}%
\begin{pgfscope}%
\pgfpathrectangle{\pgfqpoint{3.985294in}{4.155455in}}{\pgfqpoint{2.279412in}{2.004545in}}%
\pgfusepath{clip}%
\pgfsetbuttcap%
\pgfsetroundjoin%
\pgfsetlinewidth{0.739091pt}%
\definecolor{currentstroke}{rgb}{0.231674,0.318106,0.544834}%
\pgfsetstrokecolor{currentstroke}%
\pgfsetdash{}{0pt}%
\pgfpathmoveto{\pgfqpoint{5.343573in}{5.357511in}}%
\pgfpathlineto{\pgfqpoint{5.293690in}{5.352985in}}%
\pgfusepath{stroke}%
\end{pgfscope}%
\begin{pgfscope}%
\pgfpathrectangle{\pgfqpoint{3.985294in}{4.155455in}}{\pgfqpoint{2.279412in}{2.004545in}}%
\pgfusepath{clip}%
\pgfsetbuttcap%
\pgfsetroundjoin%
\pgfsetlinewidth{0.790335pt}%
\definecolor{currentstroke}{rgb}{0.216210,0.351535,0.550627}%
\pgfsetstrokecolor{currentstroke}%
\pgfsetdash{}{0pt}%
\pgfpathmoveto{\pgfqpoint{5.293690in}{5.352985in}}%
\pgfpathlineto{\pgfqpoint{5.243914in}{5.347625in}}%
\pgfusepath{stroke}%
\end{pgfscope}%
\begin{pgfscope}%
\pgfpathrectangle{\pgfqpoint{3.985294in}{4.155455in}}{\pgfqpoint{2.279412in}{2.004545in}}%
\pgfusepath{clip}%
\pgfsetbuttcap%
\pgfsetroundjoin%
\pgfsetlinewidth{0.805512pt}%
\definecolor{currentstroke}{rgb}{0.212395,0.359683,0.551710}%
\pgfsetstrokecolor{currentstroke}%
\pgfsetdash{}{0pt}%
\pgfpathmoveto{\pgfqpoint{5.243914in}{5.347625in}}%
\pgfpathlineto{\pgfqpoint{5.194291in}{5.341270in}}%
\pgfusepath{stroke}%
\end{pgfscope}%
\begin{pgfscope}%
\pgfpathrectangle{\pgfqpoint{3.985294in}{4.155455in}}{\pgfqpoint{2.279412in}{2.004545in}}%
\pgfusepath{clip}%
\pgfsetbuttcap%
\pgfsetroundjoin%
\pgfsetlinewidth{0.907815pt}%
\definecolor{currentstroke}{rgb}{0.185556,0.418570,0.556753}%
\pgfsetstrokecolor{currentstroke}%
\pgfsetdash{}{0pt}%
\pgfpathmoveto{\pgfqpoint{5.194291in}{5.341270in}}%
\pgfpathlineto{\pgfqpoint{5.144897in}{5.333669in}}%
\pgfusepath{stroke}%
\end{pgfscope}%
\begin{pgfscope}%
\pgfpathrectangle{\pgfqpoint{3.985294in}{4.155455in}}{\pgfqpoint{2.279412in}{2.004545in}}%
\pgfusepath{clip}%
\pgfsetbuttcap%
\pgfsetroundjoin%
\pgfsetlinewidth{0.927536pt}%
\definecolor{currentstroke}{rgb}{0.180629,0.429975,0.557282}%
\pgfsetstrokecolor{currentstroke}%
\pgfsetdash{}{0pt}%
\pgfpathmoveto{\pgfqpoint{5.144897in}{5.333669in}}%
\pgfpathlineto{\pgfqpoint{5.095807in}{5.324691in}}%
\pgfusepath{stroke}%
\end{pgfscope}%
\begin{pgfscope}%
\pgfpathrectangle{\pgfqpoint{3.985294in}{4.155455in}}{\pgfqpoint{2.279412in}{2.004545in}}%
\pgfusepath{clip}%
\pgfsetbuttcap%
\pgfsetroundjoin%
\pgfsetlinewidth{0.958271pt}%
\definecolor{currentstroke}{rgb}{0.174274,0.445044,0.557792}%
\pgfsetstrokecolor{currentstroke}%
\pgfsetdash{}{0pt}%
\pgfpathmoveto{\pgfqpoint{5.095807in}{5.324691in}}%
\pgfpathlineto{\pgfqpoint{5.047141in}{5.314106in}}%
\pgfusepath{stroke}%
\end{pgfscope}%
\begin{pgfscope}%
\pgfpathrectangle{\pgfqpoint{3.985294in}{4.155455in}}{\pgfqpoint{2.279412in}{2.004545in}}%
\pgfusepath{clip}%
\pgfsetbuttcap%
\pgfsetroundjoin%
\pgfsetlinewidth{1.012551pt}%
\definecolor{currentstroke}{rgb}{0.162142,0.474838,0.558140}%
\pgfsetstrokecolor{currentstroke}%
\pgfsetdash{}{0pt}%
\pgfpathmoveto{\pgfqpoint{5.047141in}{5.314106in}}%
\pgfpathlineto{\pgfqpoint{4.999217in}{5.301277in}}%
\pgfusepath{stroke}%
\end{pgfscope}%
\begin{pgfscope}%
\pgfpathrectangle{\pgfqpoint{3.985294in}{4.155455in}}{\pgfqpoint{2.279412in}{2.004545in}}%
\pgfusepath{clip}%
\pgfsetbuttcap%
\pgfsetroundjoin%
\pgfsetlinewidth{0.954570pt}%
\definecolor{currentstroke}{rgb}{0.174274,0.445044,0.557792}%
\pgfsetstrokecolor{currentstroke}%
\pgfsetdash{}{0pt}%
\pgfpathmoveto{\pgfqpoint{4.999217in}{5.301277in}}%
\pgfpathlineto{\pgfqpoint{4.952521in}{5.285497in}}%
\pgfusepath{stroke}%
\end{pgfscope}%
\begin{pgfscope}%
\pgfpathrectangle{\pgfqpoint{3.985294in}{4.155455in}}{\pgfqpoint{2.279412in}{2.004545in}}%
\pgfusepath{clip}%
\pgfsetbuttcap%
\pgfsetroundjoin%
\pgfsetlinewidth{0.898657pt}%
\definecolor{currentstroke}{rgb}{0.187231,0.414746,0.556547}%
\pgfsetstrokecolor{currentstroke}%
\pgfsetdash{}{0pt}%
\pgfpathmoveto{\pgfqpoint{4.952521in}{5.285497in}}%
\pgfpathlineto{\pgfqpoint{4.907308in}{5.266756in}}%
\pgfusepath{stroke}%
\end{pgfscope}%
\begin{pgfscope}%
\pgfpathrectangle{\pgfqpoint{3.985294in}{4.155455in}}{\pgfqpoint{2.279412in}{2.004545in}}%
\pgfusepath{clip}%
\pgfsetbuttcap%
\pgfsetroundjoin%
\pgfsetlinewidth{0.947468pt}%
\definecolor{currentstroke}{rgb}{0.175841,0.441290,0.557685}%
\pgfsetstrokecolor{currentstroke}%
\pgfsetdash{}{0pt}%
\pgfpathmoveto{\pgfqpoint{4.907308in}{5.266756in}}%
\pgfpathlineto{\pgfqpoint{4.862550in}{5.247237in}}%
\pgfusepath{stroke}%
\end{pgfscope}%
\begin{pgfscope}%
\pgfpathrectangle{\pgfqpoint{3.985294in}{4.155455in}}{\pgfqpoint{2.279412in}{2.004545in}}%
\pgfusepath{clip}%
\pgfsetbuttcap%
\pgfsetroundjoin%
\pgfsetlinewidth{1.110340pt}%
\definecolor{currentstroke}{rgb}{0.141935,0.526453,0.555991}%
\pgfsetstrokecolor{currentstroke}%
\pgfsetdash{}{0pt}%
\pgfpathmoveto{\pgfqpoint{4.862550in}{5.247237in}}%
\pgfpathlineto{\pgfqpoint{4.819097in}{5.225878in}}%
\pgfusepath{stroke}%
\end{pgfscope}%
\begin{pgfscope}%
\pgfpathrectangle{\pgfqpoint{3.985294in}{4.155455in}}{\pgfqpoint{2.279412in}{2.004545in}}%
\pgfusepath{clip}%
\pgfsetbuttcap%
\pgfsetroundjoin%
\pgfsetlinewidth{1.095859pt}%
\definecolor{currentstroke}{rgb}{0.144759,0.519093,0.556572}%
\pgfsetstrokecolor{currentstroke}%
\pgfsetdash{}{0pt}%
\pgfpathmoveto{\pgfqpoint{4.819097in}{5.225878in}}%
\pgfpathlineto{\pgfqpoint{4.776411in}{5.203717in}}%
\pgfusepath{stroke}%
\end{pgfscope}%
\begin{pgfscope}%
\pgfpathrectangle{\pgfqpoint{3.985294in}{4.155455in}}{\pgfqpoint{2.279412in}{2.004545in}}%
\pgfusepath{clip}%
\pgfsetbuttcap%
\pgfsetroundjoin%
\pgfsetlinewidth{0.937661pt}%
\definecolor{currentstroke}{rgb}{0.179019,0.433756,0.557430}%
\pgfsetstrokecolor{currentstroke}%
\pgfsetdash{}{0pt}%
\pgfpathmoveto{\pgfqpoint{4.776411in}{5.203717in}}%
\pgfpathlineto{\pgfqpoint{4.733122in}{5.183231in}}%
\pgfusepath{stroke}%
\end{pgfscope}%
\begin{pgfscope}%
\pgfpathrectangle{\pgfqpoint{3.985294in}{4.155455in}}{\pgfqpoint{2.279412in}{2.004545in}}%
\pgfusepath{clip}%
\pgfsetbuttcap%
\pgfsetroundjoin%
\pgfsetlinewidth{0.322833pt}%
\definecolor{currentstroke}{rgb}{0.271305,0.019942,0.347269}%
\pgfsetstrokecolor{currentstroke}%
\pgfsetdash{}{0pt}%
\pgfpathmoveto{\pgfqpoint{5.894378in}{5.428368in}}%
\pgfpathlineto{\pgfqpoint{5.844264in}{5.427046in}}%
\pgfusepath{stroke}%
\end{pgfscope}%
\begin{pgfscope}%
\pgfpathrectangle{\pgfqpoint{3.985294in}{4.155455in}}{\pgfqpoint{2.279412in}{2.004545in}}%
\pgfusepath{clip}%
\pgfsetbuttcap%
\pgfsetroundjoin%
\pgfsetlinewidth{0.327222pt}%
\definecolor{currentstroke}{rgb}{0.271305,0.019942,0.347269}%
\pgfsetstrokecolor{currentstroke}%
\pgfsetdash{}{0pt}%
\pgfpathmoveto{\pgfqpoint{5.844264in}{5.427046in}}%
\pgfpathlineto{\pgfqpoint{5.794127in}{5.426351in}}%
\pgfusepath{stroke}%
\end{pgfscope}%
\begin{pgfscope}%
\pgfpathrectangle{\pgfqpoint{3.985294in}{4.155455in}}{\pgfqpoint{2.279412in}{2.004545in}}%
\pgfusepath{clip}%
\pgfsetbuttcap%
\pgfsetroundjoin%
\pgfsetlinewidth{0.337054pt}%
\definecolor{currentstroke}{rgb}{0.273809,0.031497,0.358853}%
\pgfsetstrokecolor{currentstroke}%
\pgfsetdash{}{0pt}%
\pgfpathmoveto{\pgfqpoint{5.794127in}{5.426351in}}%
\pgfpathlineto{\pgfqpoint{5.744035in}{5.424633in}}%
\pgfusepath{stroke}%
\end{pgfscope}%
\begin{pgfscope}%
\pgfpathrectangle{\pgfqpoint{3.985294in}{4.155455in}}{\pgfqpoint{2.279412in}{2.004545in}}%
\pgfusepath{clip}%
\pgfsetbuttcap%
\pgfsetroundjoin%
\pgfsetlinewidth{0.357269pt}%
\definecolor{currentstroke}{rgb}{0.277018,0.050344,0.375715}%
\pgfsetstrokecolor{currentstroke}%
\pgfsetdash{}{0pt}%
\pgfpathmoveto{\pgfqpoint{5.744035in}{5.424633in}}%
\pgfpathlineto{\pgfqpoint{5.693957in}{5.422301in}}%
\pgfusepath{stroke}%
\end{pgfscope}%
\begin{pgfscope}%
\pgfpathrectangle{\pgfqpoint{3.985294in}{4.155455in}}{\pgfqpoint{2.279412in}{2.004545in}}%
\pgfusepath{clip}%
\pgfsetbuttcap%
\pgfsetroundjoin%
\pgfsetlinewidth{0.375300pt}%
\definecolor{currentstroke}{rgb}{0.278791,0.062145,0.386592}%
\pgfsetstrokecolor{currentstroke}%
\pgfsetdash{}{0pt}%
\pgfpathmoveto{\pgfqpoint{5.693957in}{5.422301in}}%
\pgfpathlineto{\pgfqpoint{5.643865in}{5.420263in}}%
\pgfusepath{stroke}%
\end{pgfscope}%
\begin{pgfscope}%
\pgfpathrectangle{\pgfqpoint{3.985294in}{4.155455in}}{\pgfqpoint{2.279412in}{2.004545in}}%
\pgfusepath{clip}%
\pgfsetbuttcap%
\pgfsetroundjoin%
\pgfsetlinewidth{0.410446pt}%
\definecolor{currentstroke}{rgb}{0.281924,0.089666,0.412415}%
\pgfsetstrokecolor{currentstroke}%
\pgfsetdash{}{0pt}%
\pgfpathmoveto{\pgfqpoint{5.643865in}{5.420263in}}%
\pgfpathlineto{\pgfqpoint{5.593780in}{5.418067in}}%
\pgfusepath{stroke}%
\end{pgfscope}%
\begin{pgfscope}%
\pgfpathrectangle{\pgfqpoint{3.985294in}{4.155455in}}{\pgfqpoint{2.279412in}{2.004545in}}%
\pgfusepath{clip}%
\pgfsetbuttcap%
\pgfsetroundjoin%
\pgfsetlinewidth{0.442758pt}%
\definecolor{currentstroke}{rgb}{0.283197,0.115680,0.436115}%
\pgfsetstrokecolor{currentstroke}%
\pgfsetdash{}{0pt}%
\pgfpathmoveto{\pgfqpoint{5.593780in}{5.418067in}}%
\pgfpathlineto{\pgfqpoint{5.543709in}{5.415578in}}%
\pgfusepath{stroke}%
\end{pgfscope}%
\begin{pgfscope}%
\pgfpathrectangle{\pgfqpoint{3.985294in}{4.155455in}}{\pgfqpoint{2.279412in}{2.004545in}}%
\pgfusepath{clip}%
\pgfsetbuttcap%
\pgfsetroundjoin%
\pgfsetlinewidth{0.465127pt}%
\definecolor{currentstroke}{rgb}{0.283072,0.130895,0.449241}%
\pgfsetstrokecolor{currentstroke}%
\pgfsetdash{}{0pt}%
\pgfpathmoveto{\pgfqpoint{5.543709in}{5.415578in}}%
\pgfpathlineto{\pgfqpoint{5.493655in}{5.412834in}}%
\pgfusepath{stroke}%
\end{pgfscope}%
\begin{pgfscope}%
\pgfpathrectangle{\pgfqpoint{3.985294in}{4.155455in}}{\pgfqpoint{2.279412in}{2.004545in}}%
\pgfusepath{clip}%
\pgfsetbuttcap%
\pgfsetroundjoin%
\pgfsetlinewidth{0.524295pt}%
\definecolor{currentstroke}{rgb}{0.278826,0.175490,0.483397}%
\pgfsetstrokecolor{currentstroke}%
\pgfsetdash{}{0pt}%
\pgfpathmoveto{\pgfqpoint{5.493655in}{5.412834in}}%
\pgfpathlineto{\pgfqpoint{5.443634in}{5.409682in}}%
\pgfusepath{stroke}%
\end{pgfscope}%
\begin{pgfscope}%
\pgfpathrectangle{\pgfqpoint{3.985294in}{4.155455in}}{\pgfqpoint{2.279412in}{2.004545in}}%
\pgfusepath{clip}%
\pgfsetbuttcap%
\pgfsetroundjoin%
\pgfsetlinewidth{0.561883pt}%
\definecolor{currentstroke}{rgb}{0.273006,0.204520,0.501721}%
\pgfsetstrokecolor{currentstroke}%
\pgfsetdash{}{0pt}%
\pgfpathmoveto{\pgfqpoint{5.443634in}{5.409682in}}%
\pgfpathlineto{\pgfqpoint{5.393662in}{5.405968in}}%
\pgfusepath{stroke}%
\end{pgfscope}%
\begin{pgfscope}%
\pgfpathrectangle{\pgfqpoint{3.985294in}{4.155455in}}{\pgfqpoint{2.279412in}{2.004545in}}%
\pgfusepath{clip}%
\pgfsetbuttcap%
\pgfsetroundjoin%
\pgfsetlinewidth{0.614183pt}%
\definecolor{currentstroke}{rgb}{0.263663,0.237631,0.518762}%
\pgfsetstrokecolor{currentstroke}%
\pgfsetdash{}{0pt}%
\pgfpathmoveto{\pgfqpoint{5.393662in}{5.405968in}}%
\pgfpathlineto{\pgfqpoint{5.343752in}{5.401662in}}%
\pgfusepath{stroke}%
\end{pgfscope}%
\begin{pgfscope}%
\pgfpathrectangle{\pgfqpoint{3.985294in}{4.155455in}}{\pgfqpoint{2.279412in}{2.004545in}}%
\pgfusepath{clip}%
\pgfsetbuttcap%
\pgfsetroundjoin%
\pgfsetlinewidth{0.650809pt}%
\definecolor{currentstroke}{rgb}{0.255645,0.260703,0.528312}%
\pgfsetstrokecolor{currentstroke}%
\pgfsetdash{}{0pt}%
\pgfpathmoveto{\pgfqpoint{5.343752in}{5.401662in}}%
\pgfpathlineto{\pgfqpoint{5.293914in}{5.396751in}}%
\pgfusepath{stroke}%
\end{pgfscope}%
\begin{pgfscope}%
\pgfpathrectangle{\pgfqpoint{3.985294in}{4.155455in}}{\pgfqpoint{2.279412in}{2.004545in}}%
\pgfusepath{clip}%
\pgfsetbuttcap%
\pgfsetroundjoin%
\pgfsetlinewidth{0.733456pt}%
\definecolor{currentstroke}{rgb}{0.231674,0.318106,0.544834}%
\pgfsetstrokecolor{currentstroke}%
\pgfsetdash{}{0pt}%
\pgfpathmoveto{\pgfqpoint{5.293914in}{5.396751in}}%
\pgfpathlineto{\pgfqpoint{5.244199in}{5.390973in}}%
\pgfusepath{stroke}%
\end{pgfscope}%
\begin{pgfscope}%
\pgfpathrectangle{\pgfqpoint{3.985294in}{4.155455in}}{\pgfqpoint{2.279412in}{2.004545in}}%
\pgfusepath{clip}%
\pgfsetbuttcap%
\pgfsetroundjoin%
\pgfsetlinewidth{0.725150pt}%
\definecolor{currentstroke}{rgb}{0.235526,0.309527,0.542944}%
\pgfsetstrokecolor{currentstroke}%
\pgfsetdash{}{0pt}%
\pgfpathmoveto{\pgfqpoint{5.244199in}{5.390973in}}%
\pgfpathlineto{\pgfqpoint{5.194743in}{5.383722in}}%
\pgfusepath{stroke}%
\end{pgfscope}%
\begin{pgfscope}%
\pgfpathrectangle{\pgfqpoint{3.985294in}{4.155455in}}{\pgfqpoint{2.279412in}{2.004545in}}%
\pgfusepath{clip}%
\pgfsetbuttcap%
\pgfsetroundjoin%
\pgfsetlinewidth{0.798808pt}%
\definecolor{currentstroke}{rgb}{0.214298,0.355619,0.551184}%
\pgfsetstrokecolor{currentstroke}%
\pgfsetdash{}{0pt}%
\pgfpathmoveto{\pgfqpoint{5.194743in}{5.383722in}}%
\pgfpathlineto{\pgfqpoint{5.145612in}{5.374906in}}%
\pgfusepath{stroke}%
\end{pgfscope}%
\begin{pgfscope}%
\pgfpathrectangle{\pgfqpoint{3.985294in}{4.155455in}}{\pgfqpoint{2.279412in}{2.004545in}}%
\pgfusepath{clip}%
\pgfsetbuttcap%
\pgfsetroundjoin%
\pgfsetlinewidth{0.794592pt}%
\definecolor{currentstroke}{rgb}{0.216210,0.351535,0.550627}%
\pgfsetstrokecolor{currentstroke}%
\pgfsetdash{}{0pt}%
\pgfpathmoveto{\pgfqpoint{5.145612in}{5.374906in}}%
\pgfpathlineto{\pgfqpoint{5.096961in}{5.364284in}}%
\pgfusepath{stroke}%
\end{pgfscope}%
\begin{pgfscope}%
\pgfpathrectangle{\pgfqpoint{3.985294in}{4.155455in}}{\pgfqpoint{2.279412in}{2.004545in}}%
\pgfusepath{clip}%
\pgfsetbuttcap%
\pgfsetroundjoin%
\pgfsetlinewidth{0.327013pt}%
\definecolor{currentstroke}{rgb}{0.271305,0.019942,0.347269}%
\pgfsetstrokecolor{currentstroke}%
\pgfsetdash{}{0pt}%
\pgfpathmoveto{\pgfqpoint{5.843087in}{4.796873in}}%
\pgfpathlineto{\pgfqpoint{5.792964in}{4.797567in}}%
\pgfusepath{stroke}%
\end{pgfscope}%
\begin{pgfscope}%
\pgfpathrectangle{\pgfqpoint{3.985294in}{4.155455in}}{\pgfqpoint{2.279412in}{2.004545in}}%
\pgfusepath{clip}%
\pgfsetbuttcap%
\pgfsetroundjoin%
\pgfsetlinewidth{0.317025pt}%
\definecolor{currentstroke}{rgb}{0.269944,0.014625,0.341379}%
\pgfsetstrokecolor{currentstroke}%
\pgfsetdash{}{0pt}%
\pgfpathmoveto{\pgfqpoint{5.792964in}{4.797567in}}%
\pgfpathlineto{\pgfqpoint{5.742866in}{4.798860in}}%
\pgfusepath{stroke}%
\end{pgfscope}%
\begin{pgfscope}%
\pgfpathrectangle{\pgfqpoint{3.985294in}{4.155455in}}{\pgfqpoint{2.279412in}{2.004545in}}%
\pgfusepath{clip}%
\pgfsetbuttcap%
\pgfsetroundjoin%
\pgfsetlinewidth{0.338551pt}%
\definecolor{currentstroke}{rgb}{0.273809,0.031497,0.358853}%
\pgfsetstrokecolor{currentstroke}%
\pgfsetdash{}{0pt}%
\pgfpathmoveto{\pgfqpoint{5.742866in}{4.798860in}}%
\pgfpathlineto{\pgfqpoint{5.692764in}{4.800288in}}%
\pgfusepath{stroke}%
\end{pgfscope}%
\begin{pgfscope}%
\pgfpathrectangle{\pgfqpoint{3.985294in}{4.155455in}}{\pgfqpoint{2.279412in}{2.004545in}}%
\pgfusepath{clip}%
\pgfsetbuttcap%
\pgfsetroundjoin%
\pgfsetlinewidth{0.340832pt}%
\definecolor{currentstroke}{rgb}{0.273809,0.031497,0.358853}%
\pgfsetstrokecolor{currentstroke}%
\pgfsetdash{}{0pt}%
\pgfpathmoveto{\pgfqpoint{5.692764in}{4.800288in}}%
\pgfpathlineto{\pgfqpoint{5.642681in}{4.802392in}}%
\pgfusepath{stroke}%
\end{pgfscope}%
\begin{pgfscope}%
\pgfpathrectangle{\pgfqpoint{3.985294in}{4.155455in}}{\pgfqpoint{2.279412in}{2.004545in}}%
\pgfusepath{clip}%
\pgfsetbuttcap%
\pgfsetroundjoin%
\pgfsetlinewidth{0.392745pt}%
\definecolor{currentstroke}{rgb}{0.280894,0.078907,0.402329}%
\pgfsetstrokecolor{currentstroke}%
\pgfsetdash{}{0pt}%
\pgfpathmoveto{\pgfqpoint{5.642681in}{4.802392in}}%
\pgfpathlineto{\pgfqpoint{5.592636in}{4.805262in}}%
\pgfusepath{stroke}%
\end{pgfscope}%
\begin{pgfscope}%
\pgfpathrectangle{\pgfqpoint{3.985294in}{4.155455in}}{\pgfqpoint{2.279412in}{2.004545in}}%
\pgfusepath{clip}%
\pgfsetbuttcap%
\pgfsetroundjoin%
\pgfsetlinewidth{0.402144pt}%
\definecolor{currentstroke}{rgb}{0.281446,0.084320,0.407414}%
\pgfsetstrokecolor{currentstroke}%
\pgfsetdash{}{0pt}%
\pgfpathmoveto{\pgfqpoint{5.592636in}{4.805262in}}%
\pgfpathlineto{\pgfqpoint{5.542632in}{4.808584in}}%
\pgfusepath{stroke}%
\end{pgfscope}%
\begin{pgfscope}%
\pgfpathrectangle{\pgfqpoint{3.985294in}{4.155455in}}{\pgfqpoint{2.279412in}{2.004545in}}%
\pgfusepath{clip}%
\pgfsetbuttcap%
\pgfsetroundjoin%
\pgfsetlinewidth{0.428696pt}%
\definecolor{currentstroke}{rgb}{0.282910,0.105393,0.426902}%
\pgfsetstrokecolor{currentstroke}%
\pgfsetdash{}{0pt}%
\pgfpathmoveto{\pgfqpoint{5.542632in}{4.808584in}}%
\pgfpathlineto{\pgfqpoint{5.492677in}{4.812467in}}%
\pgfusepath{stroke}%
\end{pgfscope}%
\begin{pgfscope}%
\pgfpathrectangle{\pgfqpoint{3.985294in}{4.155455in}}{\pgfqpoint{2.279412in}{2.004545in}}%
\pgfusepath{clip}%
\pgfsetbuttcap%
\pgfsetroundjoin%
\pgfsetlinewidth{0.476352pt}%
\definecolor{currentstroke}{rgb}{0.282623,0.140926,0.457517}%
\pgfsetstrokecolor{currentstroke}%
\pgfsetdash{}{0pt}%
\pgfpathmoveto{\pgfqpoint{5.492677in}{4.812467in}}%
\pgfpathlineto{\pgfqpoint{5.442796in}{4.817000in}}%
\pgfusepath{stroke}%
\end{pgfscope}%
\begin{pgfscope}%
\pgfpathrectangle{\pgfqpoint{3.985294in}{4.155455in}}{\pgfqpoint{2.279412in}{2.004545in}}%
\pgfusepath{clip}%
\pgfsetbuttcap%
\pgfsetroundjoin%
\pgfsetlinewidth{0.496276pt}%
\definecolor{currentstroke}{rgb}{0.281412,0.155834,0.469201}%
\pgfsetstrokecolor{currentstroke}%
\pgfsetdash{}{0pt}%
\pgfpathmoveto{\pgfqpoint{5.442796in}{4.817000in}}%
\pgfpathlineto{\pgfqpoint{5.392978in}{4.822082in}}%
\pgfusepath{stroke}%
\end{pgfscope}%
\begin{pgfscope}%
\pgfpathrectangle{\pgfqpoint{3.985294in}{4.155455in}}{\pgfqpoint{2.279412in}{2.004545in}}%
\pgfusepath{clip}%
\pgfsetbuttcap%
\pgfsetroundjoin%
\pgfsetlinewidth{0.514545pt}%
\definecolor{currentstroke}{rgb}{0.279574,0.170599,0.479997}%
\pgfsetstrokecolor{currentstroke}%
\pgfsetdash{}{0pt}%
\pgfpathmoveto{\pgfqpoint{5.392978in}{4.822082in}}%
\pgfpathlineto{\pgfqpoint{5.343241in}{4.827711in}}%
\pgfusepath{stroke}%
\end{pgfscope}%
\begin{pgfscope}%
\pgfpathrectangle{\pgfqpoint{3.985294in}{4.155455in}}{\pgfqpoint{2.279412in}{2.004545in}}%
\pgfusepath{clip}%
\pgfsetbuttcap%
\pgfsetroundjoin%
\pgfsetlinewidth{0.565794pt}%
\definecolor{currentstroke}{rgb}{0.273006,0.204520,0.501721}%
\pgfsetstrokecolor{currentstroke}%
\pgfsetdash{}{0pt}%
\pgfpathmoveto{\pgfqpoint{5.343241in}{4.827711in}}%
\pgfpathlineto{\pgfqpoint{5.293640in}{4.834218in}}%
\pgfusepath{stroke}%
\end{pgfscope}%
\begin{pgfscope}%
\pgfpathrectangle{\pgfqpoint{3.985294in}{4.155455in}}{\pgfqpoint{2.279412in}{2.004545in}}%
\pgfusepath{clip}%
\pgfsetbuttcap%
\pgfsetroundjoin%
\pgfsetlinewidth{0.610820pt}%
\definecolor{currentstroke}{rgb}{0.263663,0.237631,0.518762}%
\pgfsetstrokecolor{currentstroke}%
\pgfsetdash{}{0pt}%
\pgfpathmoveto{\pgfqpoint{5.293640in}{4.834218in}}%
\pgfpathlineto{\pgfqpoint{5.244299in}{4.842028in}}%
\pgfusepath{stroke}%
\end{pgfscope}%
\begin{pgfscope}%
\pgfpathrectangle{\pgfqpoint{3.985294in}{4.155455in}}{\pgfqpoint{2.279412in}{2.004545in}}%
\pgfusepath{clip}%
\pgfsetbuttcap%
\pgfsetroundjoin%
\pgfsetlinewidth{0.633633pt}%
\definecolor{currentstroke}{rgb}{0.258965,0.251537,0.524736}%
\pgfsetstrokecolor{currentstroke}%
\pgfsetdash{}{0pt}%
\pgfpathmoveto{\pgfqpoint{5.244299in}{4.842028in}}%
\pgfpathlineto{\pgfqpoint{5.195322in}{4.851474in}}%
\pgfusepath{stroke}%
\end{pgfscope}%
\begin{pgfscope}%
\pgfpathrectangle{\pgfqpoint{3.985294in}{4.155455in}}{\pgfqpoint{2.279412in}{2.004545in}}%
\pgfusepath{clip}%
\pgfsetbuttcap%
\pgfsetroundjoin%
\pgfsetlinewidth{0.610492pt}%
\definecolor{currentstroke}{rgb}{0.263663,0.237631,0.518762}%
\pgfsetstrokecolor{currentstroke}%
\pgfsetdash{}{0pt}%
\pgfpathmoveto{\pgfqpoint{5.195322in}{4.851474in}}%
\pgfpathlineto{\pgfqpoint{5.146825in}{4.862647in}}%
\pgfusepath{stroke}%
\end{pgfscope}%
\begin{pgfscope}%
\pgfpathrectangle{\pgfqpoint{3.985294in}{4.155455in}}{\pgfqpoint{2.279412in}{2.004545in}}%
\pgfusepath{clip}%
\pgfsetbuttcap%
\pgfsetroundjoin%
\pgfsetlinewidth{0.618173pt}%
\definecolor{currentstroke}{rgb}{0.262138,0.242286,0.520837}%
\pgfsetstrokecolor{currentstroke}%
\pgfsetdash{}{0pt}%
\pgfpathmoveto{\pgfqpoint{5.146825in}{4.862647in}}%
\pgfpathlineto{\pgfqpoint{5.099193in}{4.876341in}}%
\pgfusepath{stroke}%
\end{pgfscope}%
\begin{pgfscope}%
\pgfpathrectangle{\pgfqpoint{3.985294in}{4.155455in}}{\pgfqpoint{2.279412in}{2.004545in}}%
\pgfusepath{clip}%
\pgfsetbuttcap%
\pgfsetroundjoin%
\pgfsetlinewidth{0.645396pt}%
\definecolor{currentstroke}{rgb}{0.255645,0.260703,0.528312}%
\pgfsetstrokecolor{currentstroke}%
\pgfsetdash{}{0pt}%
\pgfpathmoveto{\pgfqpoint{5.099193in}{4.876341in}}%
\pgfpathlineto{\pgfqpoint{5.052898in}{4.893161in}}%
\pgfusepath{stroke}%
\end{pgfscope}%
\begin{pgfscope}%
\pgfpathrectangle{\pgfqpoint{3.985294in}{4.155455in}}{\pgfqpoint{2.279412in}{2.004545in}}%
\pgfusepath{clip}%
\pgfsetbuttcap%
\pgfsetroundjoin%
\pgfsetlinewidth{0.333335pt}%
\definecolor{currentstroke}{rgb}{0.272594,0.025563,0.353093}%
\pgfsetstrokecolor{currentstroke}%
\pgfsetdash{}{0pt}%
\pgfpathmoveto{\pgfqpoint{5.843087in}{4.887087in}}%
\pgfpathlineto{\pgfqpoint{5.792980in}{4.888670in}}%
\pgfusepath{stroke}%
\end{pgfscope}%
\begin{pgfscope}%
\pgfpathrectangle{\pgfqpoint{3.985294in}{4.155455in}}{\pgfqpoint{2.279412in}{2.004545in}}%
\pgfusepath{clip}%
\pgfsetbuttcap%
\pgfsetroundjoin%
\pgfsetlinewidth{0.341085pt}%
\definecolor{currentstroke}{rgb}{0.273809,0.031497,0.358853}%
\pgfsetstrokecolor{currentstroke}%
\pgfsetdash{}{0pt}%
\pgfpathmoveto{\pgfqpoint{5.792980in}{4.888670in}}%
\pgfpathlineto{\pgfqpoint{5.742867in}{4.889924in}}%
\pgfusepath{stroke}%
\end{pgfscope}%
\begin{pgfscope}%
\pgfpathrectangle{\pgfqpoint{3.985294in}{4.155455in}}{\pgfqpoint{2.279412in}{2.004545in}}%
\pgfusepath{clip}%
\pgfsetbuttcap%
\pgfsetroundjoin%
\pgfsetlinewidth{0.350158pt}%
\definecolor{currentstroke}{rgb}{0.276022,0.044167,0.370164}%
\pgfsetstrokecolor{currentstroke}%
\pgfsetdash{}{0pt}%
\pgfpathmoveto{\pgfqpoint{5.742867in}{4.889924in}}%
\pgfpathlineto{\pgfqpoint{5.692796in}{4.892109in}}%
\pgfusepath{stroke}%
\end{pgfscope}%
\begin{pgfscope}%
\pgfpathrectangle{\pgfqpoint{3.985294in}{4.155455in}}{\pgfqpoint{2.279412in}{2.004545in}}%
\pgfusepath{clip}%
\pgfsetbuttcap%
\pgfsetroundjoin%
\pgfsetlinewidth{0.381470pt}%
\definecolor{currentstroke}{rgb}{0.279566,0.067836,0.391917}%
\pgfsetstrokecolor{currentstroke}%
\pgfsetdash{}{0pt}%
\pgfpathmoveto{\pgfqpoint{5.692796in}{4.892109in}}%
\pgfpathlineto{\pgfqpoint{5.642739in}{4.894810in}}%
\pgfusepath{stroke}%
\end{pgfscope}%
\begin{pgfscope}%
\pgfpathrectangle{\pgfqpoint{3.985294in}{4.155455in}}{\pgfqpoint{2.279412in}{2.004545in}}%
\pgfusepath{clip}%
\pgfsetbuttcap%
\pgfsetroundjoin%
\pgfsetlinewidth{0.396765pt}%
\definecolor{currentstroke}{rgb}{0.280894,0.078907,0.402329}%
\pgfsetstrokecolor{currentstroke}%
\pgfsetdash{}{0pt}%
\pgfpathmoveto{\pgfqpoint{5.642739in}{4.894810in}}%
\pgfpathlineto{\pgfqpoint{5.592659in}{4.897142in}}%
\pgfusepath{stroke}%
\end{pgfscope}%
\begin{pgfscope}%
\pgfpathrectangle{\pgfqpoint{3.985294in}{4.155455in}}{\pgfqpoint{2.279412in}{2.004545in}}%
\pgfusepath{clip}%
\pgfsetbuttcap%
\pgfsetroundjoin%
\pgfsetlinewidth{0.439251pt}%
\definecolor{currentstroke}{rgb}{0.283197,0.115680,0.436115}%
\pgfsetstrokecolor{currentstroke}%
\pgfsetdash{}{0pt}%
\pgfpathmoveto{\pgfqpoint{5.592659in}{4.897142in}}%
\pgfpathlineto{\pgfqpoint{5.542585in}{4.899564in}}%
\pgfusepath{stroke}%
\end{pgfscope}%
\begin{pgfscope}%
\pgfpathrectangle{\pgfqpoint{3.985294in}{4.155455in}}{\pgfqpoint{2.279412in}{2.004545in}}%
\pgfusepath{clip}%
\pgfsetbuttcap%
\pgfsetroundjoin%
\pgfsetlinewidth{0.481617pt}%
\definecolor{currentstroke}{rgb}{0.282290,0.145912,0.461510}%
\pgfsetstrokecolor{currentstroke}%
\pgfsetdash{}{0pt}%
\pgfpathmoveto{\pgfqpoint{5.542585in}{4.899564in}}%
\pgfpathlineto{\pgfqpoint{5.492536in}{4.902390in}}%
\pgfusepath{stroke}%
\end{pgfscope}%
\begin{pgfscope}%
\pgfpathrectangle{\pgfqpoint{3.985294in}{4.155455in}}{\pgfqpoint{2.279412in}{2.004545in}}%
\pgfusepath{clip}%
\pgfsetbuttcap%
\pgfsetroundjoin%
\pgfsetlinewidth{0.515974pt}%
\definecolor{currentstroke}{rgb}{0.279574,0.170599,0.479997}%
\pgfsetstrokecolor{currentstroke}%
\pgfsetdash{}{0pt}%
\pgfpathmoveto{\pgfqpoint{5.492536in}{4.902390in}}%
\pgfpathlineto{\pgfqpoint{5.442513in}{4.905539in}}%
\pgfusepath{stroke}%
\end{pgfscope}%
\begin{pgfscope}%
\pgfpathrectangle{\pgfqpoint{3.985294in}{4.155455in}}{\pgfqpoint{2.279412in}{2.004545in}}%
\pgfusepath{clip}%
\pgfsetbuttcap%
\pgfsetroundjoin%
\pgfsetlinewidth{0.334960pt}%
\definecolor{currentstroke}{rgb}{0.272594,0.025563,0.353093}%
\pgfsetstrokecolor{currentstroke}%
\pgfsetdash{}{0pt}%
\pgfpathmoveto{\pgfqpoint{5.843087in}{4.932193in}}%
\pgfpathlineto{\pgfqpoint{5.793015in}{4.934485in}}%
\pgfusepath{stroke}%
\end{pgfscope}%
\begin{pgfscope}%
\pgfpathrectangle{\pgfqpoint{3.985294in}{4.155455in}}{\pgfqpoint{2.279412in}{2.004545in}}%
\pgfusepath{clip}%
\pgfsetbuttcap%
\pgfsetroundjoin%
\pgfsetlinewidth{0.345322pt}%
\definecolor{currentstroke}{rgb}{0.274952,0.037752,0.364543}%
\pgfsetstrokecolor{currentstroke}%
\pgfsetdash{}{0pt}%
\pgfpathmoveto{\pgfqpoint{5.793015in}{4.934485in}}%
\pgfpathlineto{\pgfqpoint{5.742906in}{4.936260in}}%
\pgfusepath{stroke}%
\end{pgfscope}%
\begin{pgfscope}%
\pgfpathrectangle{\pgfqpoint{3.985294in}{4.155455in}}{\pgfqpoint{2.279412in}{2.004545in}}%
\pgfusepath{clip}%
\pgfsetbuttcap%
\pgfsetroundjoin%
\pgfsetlinewidth{0.356467pt}%
\definecolor{currentstroke}{rgb}{0.277018,0.050344,0.375715}%
\pgfsetstrokecolor{currentstroke}%
\pgfsetdash{}{0pt}%
\pgfpathmoveto{\pgfqpoint{5.742906in}{4.936260in}}%
\pgfpathlineto{\pgfqpoint{5.692795in}{4.938010in}}%
\pgfusepath{stroke}%
\end{pgfscope}%
\begin{pgfscope}%
\pgfpathrectangle{\pgfqpoint{3.985294in}{4.155455in}}{\pgfqpoint{2.279412in}{2.004545in}}%
\pgfusepath{clip}%
\pgfsetbuttcap%
\pgfsetroundjoin%
\pgfsetlinewidth{0.384589pt}%
\definecolor{currentstroke}{rgb}{0.280267,0.073417,0.397163}%
\pgfsetstrokecolor{currentstroke}%
\pgfsetdash{}{0pt}%
\pgfpathmoveto{\pgfqpoint{5.692795in}{4.938010in}}%
\pgfpathlineto{\pgfqpoint{5.642686in}{4.939834in}}%
\pgfusepath{stroke}%
\end{pgfscope}%
\begin{pgfscope}%
\pgfpathrectangle{\pgfqpoint{3.985294in}{4.155455in}}{\pgfqpoint{2.279412in}{2.004545in}}%
\pgfusepath{clip}%
\pgfsetbuttcap%
\pgfsetroundjoin%
\pgfsetlinewidth{0.414712pt}%
\definecolor{currentstroke}{rgb}{0.282327,0.094955,0.417331}%
\pgfsetstrokecolor{currentstroke}%
\pgfsetdash{}{0pt}%
\pgfpathmoveto{\pgfqpoint{5.642686in}{4.939834in}}%
\pgfpathlineto{\pgfqpoint{5.592598in}{4.942044in}}%
\pgfusepath{stroke}%
\end{pgfscope}%
\begin{pgfscope}%
\pgfpathrectangle{\pgfqpoint{3.985294in}{4.155455in}}{\pgfqpoint{2.279412in}{2.004545in}}%
\pgfusepath{clip}%
\pgfsetbuttcap%
\pgfsetroundjoin%
\pgfsetlinewidth{0.442910pt}%
\definecolor{currentstroke}{rgb}{0.283197,0.115680,0.436115}%
\pgfsetstrokecolor{currentstroke}%
\pgfsetdash{}{0pt}%
\pgfpathmoveto{\pgfqpoint{5.592598in}{4.942044in}}%
\pgfpathlineto{\pgfqpoint{5.542511in}{4.944264in}}%
\pgfusepath{stroke}%
\end{pgfscope}%
\begin{pgfscope}%
\pgfpathrectangle{\pgfqpoint{3.985294in}{4.155455in}}{\pgfqpoint{2.279412in}{2.004545in}}%
\pgfusepath{clip}%
\pgfsetbuttcap%
\pgfsetroundjoin%
\pgfsetlinewidth{0.500851pt}%
\definecolor{currentstroke}{rgb}{0.280868,0.160771,0.472899}%
\pgfsetstrokecolor{currentstroke}%
\pgfsetdash{}{0pt}%
\pgfpathmoveto{\pgfqpoint{5.542511in}{4.944264in}}%
\pgfpathlineto{\pgfqpoint{5.492426in}{4.946526in}}%
\pgfusepath{stroke}%
\end{pgfscope}%
\begin{pgfscope}%
\pgfpathrectangle{\pgfqpoint{3.985294in}{4.155455in}}{\pgfqpoint{2.279412in}{2.004545in}}%
\pgfusepath{clip}%
\pgfsetbuttcap%
\pgfsetroundjoin%
\pgfsetlinewidth{0.565303pt}%
\definecolor{currentstroke}{rgb}{0.273006,0.204520,0.501721}%
\pgfsetstrokecolor{currentstroke}%
\pgfsetdash{}{0pt}%
\pgfpathmoveto{\pgfqpoint{5.492426in}{4.946526in}}%
\pgfpathlineto{\pgfqpoint{5.442367in}{4.949198in}}%
\pgfusepath{stroke}%
\end{pgfscope}%
\begin{pgfscope}%
\pgfpathrectangle{\pgfqpoint{3.985294in}{4.155455in}}{\pgfqpoint{2.279412in}{2.004545in}}%
\pgfusepath{clip}%
\pgfsetbuttcap%
\pgfsetroundjoin%
\pgfsetlinewidth{0.606262pt}%
\definecolor{currentstroke}{rgb}{0.265145,0.232956,0.516599}%
\pgfsetstrokecolor{currentstroke}%
\pgfsetdash{}{0pt}%
\pgfpathmoveto{\pgfqpoint{5.442367in}{4.949198in}}%
\pgfpathlineto{\pgfqpoint{5.392343in}{4.952342in}}%
\pgfusepath{stroke}%
\end{pgfscope}%
\begin{pgfscope}%
\pgfpathrectangle{\pgfqpoint{3.985294in}{4.155455in}}{\pgfqpoint{2.279412in}{2.004545in}}%
\pgfusepath{clip}%
\pgfsetbuttcap%
\pgfsetroundjoin%
\pgfsetlinewidth{0.321368pt}%
\definecolor{currentstroke}{rgb}{0.269944,0.014625,0.341379}%
\pgfsetstrokecolor{currentstroke}%
\pgfsetdash{}{0pt}%
\pgfpathmoveto{\pgfqpoint{5.843087in}{5.473475in}}%
\pgfpathlineto{\pgfqpoint{5.793017in}{5.471338in}}%
\pgfusepath{stroke}%
\end{pgfscope}%
\begin{pgfscope}%
\pgfpathrectangle{\pgfqpoint{3.985294in}{4.155455in}}{\pgfqpoint{2.279412in}{2.004545in}}%
\pgfusepath{clip}%
\pgfsetbuttcap%
\pgfsetroundjoin%
\pgfsetlinewidth{0.332259pt}%
\definecolor{currentstroke}{rgb}{0.272594,0.025563,0.353093}%
\pgfsetstrokecolor{currentstroke}%
\pgfsetdash{}{0pt}%
\pgfpathmoveto{\pgfqpoint{5.793017in}{5.471338in}}%
\pgfpathlineto{\pgfqpoint{5.742960in}{5.469072in}}%
\pgfusepath{stroke}%
\end{pgfscope}%
\begin{pgfscope}%
\pgfpathrectangle{\pgfqpoint{3.985294in}{4.155455in}}{\pgfqpoint{2.279412in}{2.004545in}}%
\pgfusepath{clip}%
\pgfsetbuttcap%
\pgfsetroundjoin%
\pgfsetlinewidth{0.343228pt}%
\definecolor{currentstroke}{rgb}{0.274952,0.037752,0.364543}%
\pgfsetstrokecolor{currentstroke}%
\pgfsetdash{}{0pt}%
\pgfpathmoveto{\pgfqpoint{5.742960in}{5.469072in}}%
\pgfpathlineto{\pgfqpoint{5.692929in}{5.466155in}}%
\pgfusepath{stroke}%
\end{pgfscope}%
\begin{pgfscope}%
\pgfpathrectangle{\pgfqpoint{3.985294in}{4.155455in}}{\pgfqpoint{2.279412in}{2.004545in}}%
\pgfusepath{clip}%
\pgfsetbuttcap%
\pgfsetroundjoin%
\pgfsetlinewidth{0.365977pt}%
\definecolor{currentstroke}{rgb}{0.277941,0.056324,0.381191}%
\pgfsetstrokecolor{currentstroke}%
\pgfsetdash{}{0pt}%
\pgfpathmoveto{\pgfqpoint{5.692929in}{5.466155in}}%
\pgfpathlineto{\pgfqpoint{5.642894in}{5.463158in}}%
\pgfusepath{stroke}%
\end{pgfscope}%
\begin{pgfscope}%
\pgfpathrectangle{\pgfqpoint{3.985294in}{4.155455in}}{\pgfqpoint{2.279412in}{2.004545in}}%
\pgfusepath{clip}%
\pgfsetbuttcap%
\pgfsetroundjoin%
\pgfsetlinewidth{0.384460pt}%
\definecolor{currentstroke}{rgb}{0.280267,0.073417,0.397163}%
\pgfsetstrokecolor{currentstroke}%
\pgfsetdash{}{0pt}%
\pgfpathmoveto{\pgfqpoint{5.642894in}{5.463158in}}%
\pgfpathlineto{\pgfqpoint{5.592875in}{5.459987in}}%
\pgfusepath{stroke}%
\end{pgfscope}%
\begin{pgfscope}%
\pgfpathrectangle{\pgfqpoint{3.985294in}{4.155455in}}{\pgfqpoint{2.279412in}{2.004545in}}%
\pgfusepath{clip}%
\pgfsetbuttcap%
\pgfsetroundjoin%
\pgfsetlinewidth{0.410041pt}%
\definecolor{currentstroke}{rgb}{0.281924,0.089666,0.412415}%
\pgfsetstrokecolor{currentstroke}%
\pgfsetdash{}{0pt}%
\pgfpathmoveto{\pgfqpoint{5.592875in}{5.459987in}}%
\pgfpathlineto{\pgfqpoint{5.542906in}{5.456231in}}%
\pgfusepath{stroke}%
\end{pgfscope}%
\begin{pgfscope}%
\pgfpathrectangle{\pgfqpoint{3.985294in}{4.155455in}}{\pgfqpoint{2.279412in}{2.004545in}}%
\pgfusepath{clip}%
\pgfsetbuttcap%
\pgfsetroundjoin%
\pgfsetlinewidth{0.461001pt}%
\definecolor{currentstroke}{rgb}{0.283072,0.130895,0.449241}%
\pgfsetstrokecolor{currentstroke}%
\pgfsetdash{}{0pt}%
\pgfpathmoveto{\pgfqpoint{5.542906in}{5.456231in}}%
\pgfpathlineto{\pgfqpoint{5.492966in}{5.452189in}}%
\pgfusepath{stroke}%
\end{pgfscope}%
\begin{pgfscope}%
\pgfpathrectangle{\pgfqpoint{3.985294in}{4.155455in}}{\pgfqpoint{2.279412in}{2.004545in}}%
\pgfusepath{clip}%
\pgfsetbuttcap%
\pgfsetroundjoin%
\pgfsetlinewidth{0.340407pt}%
\definecolor{currentstroke}{rgb}{0.273809,0.031497,0.358853}%
\pgfsetstrokecolor{currentstroke}%
\pgfsetdash{}{0pt}%
\pgfpathmoveto{\pgfqpoint{5.843087in}{5.518582in}}%
\pgfpathlineto{\pgfqpoint{5.792972in}{5.517161in}}%
\pgfusepath{stroke}%
\end{pgfscope}%
\begin{pgfscope}%
\pgfpathrectangle{\pgfqpoint{3.985294in}{4.155455in}}{\pgfqpoint{2.279412in}{2.004545in}}%
\pgfusepath{clip}%
\pgfsetbuttcap%
\pgfsetroundjoin%
\pgfsetlinewidth{0.328814pt}%
\definecolor{currentstroke}{rgb}{0.272594,0.025563,0.353093}%
\pgfsetstrokecolor{currentstroke}%
\pgfsetdash{}{0pt}%
\pgfpathmoveto{\pgfqpoint{5.792972in}{5.517161in}}%
\pgfpathlineto{\pgfqpoint{5.742903in}{5.514705in}}%
\pgfusepath{stroke}%
\end{pgfscope}%
\begin{pgfscope}%
\pgfpathrectangle{\pgfqpoint{3.985294in}{4.155455in}}{\pgfqpoint{2.279412in}{2.004545in}}%
\pgfusepath{clip}%
\pgfsetbuttcap%
\pgfsetroundjoin%
\pgfsetlinewidth{0.339568pt}%
\definecolor{currentstroke}{rgb}{0.273809,0.031497,0.358853}%
\pgfsetstrokecolor{currentstroke}%
\pgfsetdash{}{0pt}%
\pgfpathmoveto{\pgfqpoint{5.742903in}{5.514705in}}%
\pgfpathlineto{\pgfqpoint{5.692847in}{5.512076in}}%
\pgfusepath{stroke}%
\end{pgfscope}%
\begin{pgfscope}%
\pgfpathrectangle{\pgfqpoint{3.985294in}{4.155455in}}{\pgfqpoint{2.279412in}{2.004545in}}%
\pgfusepath{clip}%
\pgfsetbuttcap%
\pgfsetroundjoin%
\pgfsetlinewidth{0.359015pt}%
\definecolor{currentstroke}{rgb}{0.277018,0.050344,0.375715}%
\pgfsetstrokecolor{currentstroke}%
\pgfsetdash{}{0pt}%
\pgfpathmoveto{\pgfqpoint{5.692847in}{5.512076in}}%
\pgfpathlineto{\pgfqpoint{5.642842in}{5.508870in}}%
\pgfusepath{stroke}%
\end{pgfscope}%
\begin{pgfscope}%
\pgfpathrectangle{\pgfqpoint{3.985294in}{4.155455in}}{\pgfqpoint{2.279412in}{2.004545in}}%
\pgfusepath{clip}%
\pgfsetbuttcap%
\pgfsetroundjoin%
\pgfsetlinewidth{0.358898pt}%
\definecolor{currentstroke}{rgb}{0.277018,0.050344,0.375715}%
\pgfsetstrokecolor{currentstroke}%
\pgfsetdash{}{0pt}%
\pgfpathmoveto{\pgfqpoint{5.642842in}{5.508870in}}%
\pgfpathlineto{\pgfqpoint{5.592874in}{5.505133in}}%
\pgfusepath{stroke}%
\end{pgfscope}%
\begin{pgfscope}%
\pgfpathrectangle{\pgfqpoint{3.985294in}{4.155455in}}{\pgfqpoint{2.279412in}{2.004545in}}%
\pgfusepath{clip}%
\pgfsetbuttcap%
\pgfsetroundjoin%
\pgfsetlinewidth{0.394781pt}%
\definecolor{currentstroke}{rgb}{0.280894,0.078907,0.402329}%
\pgfsetstrokecolor{currentstroke}%
\pgfsetdash{}{0pt}%
\pgfpathmoveto{\pgfqpoint{5.592874in}{5.505133in}}%
\pgfpathlineto{\pgfqpoint{5.542924in}{5.501208in}}%
\pgfusepath{stroke}%
\end{pgfscope}%
\begin{pgfscope}%
\pgfpathrectangle{\pgfqpoint{3.985294in}{4.155455in}}{\pgfqpoint{2.279412in}{2.004545in}}%
\pgfusepath{clip}%
\pgfsetbuttcap%
\pgfsetroundjoin%
\pgfsetlinewidth{0.425580pt}%
\definecolor{currentstroke}{rgb}{0.282910,0.105393,0.426902}%
\pgfsetstrokecolor{currentstroke}%
\pgfsetdash{}{0pt}%
\pgfpathmoveto{\pgfqpoint{5.542924in}{5.501208in}}%
\pgfpathlineto{\pgfqpoint{5.493023in}{5.496808in}}%
\pgfusepath{stroke}%
\end{pgfscope}%
\begin{pgfscope}%
\pgfpathrectangle{\pgfqpoint{3.985294in}{4.155455in}}{\pgfqpoint{2.279412in}{2.004545in}}%
\pgfusepath{clip}%
\pgfsetbuttcap%
\pgfsetroundjoin%
\pgfsetlinewidth{0.459905pt}%
\definecolor{currentstroke}{rgb}{0.283072,0.130895,0.449241}%
\pgfsetstrokecolor{currentstroke}%
\pgfsetdash{}{0pt}%
\pgfpathmoveto{\pgfqpoint{5.493023in}{5.496808in}}%
\pgfpathlineto{\pgfqpoint{5.443190in}{5.491872in}}%
\pgfusepath{stroke}%
\end{pgfscope}%
\begin{pgfscope}%
\pgfpathrectangle{\pgfqpoint{3.985294in}{4.155455in}}{\pgfqpoint{2.279412in}{2.004545in}}%
\pgfusepath{clip}%
\pgfsetbuttcap%
\pgfsetroundjoin%
\pgfsetlinewidth{0.503417pt}%
\definecolor{currentstroke}{rgb}{0.280868,0.160771,0.472899}%
\pgfsetstrokecolor{currentstroke}%
\pgfsetdash{}{0pt}%
\pgfpathmoveto{\pgfqpoint{5.443190in}{5.491872in}}%
\pgfpathlineto{\pgfqpoint{5.393409in}{5.486521in}}%
\pgfusepath{stroke}%
\end{pgfscope}%
\begin{pgfscope}%
\pgfpathrectangle{\pgfqpoint{3.985294in}{4.155455in}}{\pgfqpoint{2.279412in}{2.004545in}}%
\pgfusepath{clip}%
\pgfsetbuttcap%
\pgfsetroundjoin%
\pgfsetlinewidth{0.515285pt}%
\definecolor{currentstroke}{rgb}{0.279574,0.170599,0.479997}%
\pgfsetstrokecolor{currentstroke}%
\pgfsetdash{}{0pt}%
\pgfpathmoveto{\pgfqpoint{5.393409in}{5.486521in}}%
\pgfpathlineto{\pgfqpoint{5.343725in}{5.480537in}}%
\pgfusepath{stroke}%
\end{pgfscope}%
\begin{pgfscope}%
\pgfpathrectangle{\pgfqpoint{3.985294in}{4.155455in}}{\pgfqpoint{2.279412in}{2.004545in}}%
\pgfusepath{clip}%
\pgfsetbuttcap%
\pgfsetroundjoin%
\pgfsetlinewidth{0.551844pt}%
\definecolor{currentstroke}{rgb}{0.275191,0.194905,0.496005}%
\pgfsetstrokecolor{currentstroke}%
\pgfsetdash{}{0pt}%
\pgfpathmoveto{\pgfqpoint{5.343725in}{5.480537in}}%
\pgfpathlineto{\pgfqpoint{5.294212in}{5.473538in}}%
\pgfusepath{stroke}%
\end{pgfscope}%
\begin{pgfscope}%
\pgfpathrectangle{\pgfqpoint{3.985294in}{4.155455in}}{\pgfqpoint{2.279412in}{2.004545in}}%
\pgfusepath{clip}%
\pgfsetbuttcap%
\pgfsetroundjoin%
\pgfsetlinewidth{0.594172pt}%
\definecolor{currentstroke}{rgb}{0.267968,0.223549,0.512008}%
\pgfsetstrokecolor{currentstroke}%
\pgfsetdash{}{0pt}%
\pgfpathmoveto{\pgfqpoint{5.294212in}{5.473538in}}%
\pgfpathlineto{\pgfqpoint{5.244882in}{5.465614in}}%
\pgfusepath{stroke}%
\end{pgfscope}%
\begin{pgfscope}%
\pgfpathrectangle{\pgfqpoint{3.985294in}{4.155455in}}{\pgfqpoint{2.279412in}{2.004545in}}%
\pgfusepath{clip}%
\pgfsetbuttcap%
\pgfsetroundjoin%
\pgfsetlinewidth{0.606916pt}%
\definecolor{currentstroke}{rgb}{0.265145,0.232956,0.516599}%
\pgfsetstrokecolor{currentstroke}%
\pgfsetdash{}{0pt}%
\pgfpathmoveto{\pgfqpoint{5.244882in}{5.465614in}}%
\pgfpathlineto{\pgfqpoint{5.195813in}{5.456527in}}%
\pgfusepath{stroke}%
\end{pgfscope}%
\begin{pgfscope}%
\pgfpathrectangle{\pgfqpoint{3.985294in}{4.155455in}}{\pgfqpoint{2.279412in}{2.004545in}}%
\pgfusepath{clip}%
\pgfsetbuttcap%
\pgfsetroundjoin%
\pgfsetlinewidth{0.644168pt}%
\definecolor{currentstroke}{rgb}{0.257322,0.256130,0.526563}%
\pgfsetstrokecolor{currentstroke}%
\pgfsetdash{}{0pt}%
\pgfpathmoveto{\pgfqpoint{5.195813in}{5.456527in}}%
\pgfpathlineto{\pgfqpoint{5.147201in}{5.445737in}}%
\pgfusepath{stroke}%
\end{pgfscope}%
\begin{pgfscope}%
\pgfpathrectangle{\pgfqpoint{3.985294in}{4.155455in}}{\pgfqpoint{2.279412in}{2.004545in}}%
\pgfusepath{clip}%
\pgfsetbuttcap%
\pgfsetroundjoin%
\pgfsetlinewidth{0.673727pt}%
\definecolor{currentstroke}{rgb}{0.248629,0.278775,0.534556}%
\pgfsetstrokecolor{currentstroke}%
\pgfsetdash{}{0pt}%
\pgfpathmoveto{\pgfqpoint{5.147201in}{5.445737in}}%
\pgfpathlineto{\pgfqpoint{5.099632in}{5.431938in}}%
\pgfusepath{stroke}%
\end{pgfscope}%
\begin{pgfscope}%
\pgfpathrectangle{\pgfqpoint{3.985294in}{4.155455in}}{\pgfqpoint{2.279412in}{2.004545in}}%
\pgfusepath{clip}%
\pgfsetbuttcap%
\pgfsetroundjoin%
\pgfsetlinewidth{0.692063pt}%
\definecolor{currentstroke}{rgb}{0.244972,0.287675,0.537260}%
\pgfsetstrokecolor{currentstroke}%
\pgfsetdash{}{0pt}%
\pgfpathmoveto{\pgfqpoint{5.099632in}{5.431938in}}%
\pgfpathlineto{\pgfqpoint{5.053415in}{5.414917in}}%
\pgfusepath{stroke}%
\end{pgfscope}%
\begin{pgfscope}%
\pgfpathrectangle{\pgfqpoint{3.985294in}{4.155455in}}{\pgfqpoint{2.279412in}{2.004545in}}%
\pgfusepath{clip}%
\pgfsetbuttcap%
\pgfsetroundjoin%
\pgfsetlinewidth{0.627671pt}%
\definecolor{currentstroke}{rgb}{0.260571,0.246922,0.522828}%
\pgfsetstrokecolor{currentstroke}%
\pgfsetdash{}{0pt}%
\pgfpathmoveto{\pgfqpoint{5.053415in}{5.414917in}}%
\pgfpathlineto{\pgfqpoint{5.008633in}{5.395142in}}%
\pgfusepath{stroke}%
\end{pgfscope}%
\begin{pgfscope}%
\pgfpathrectangle{\pgfqpoint{3.985294in}{4.155455in}}{\pgfqpoint{2.279412in}{2.004545in}}%
\pgfusepath{clip}%
\pgfsetbuttcap%
\pgfsetroundjoin%
\pgfsetlinewidth{0.732619pt}%
\definecolor{currentstroke}{rgb}{0.233603,0.313828,0.543914}%
\pgfsetstrokecolor{currentstroke}%
\pgfsetdash{}{0pt}%
\pgfpathmoveto{\pgfqpoint{5.008633in}{5.395142in}}%
\pgfpathlineto{\pgfqpoint{4.966364in}{5.371670in}}%
\pgfusepath{stroke}%
\end{pgfscope}%
\begin{pgfscope}%
\pgfpathrectangle{\pgfqpoint{3.985294in}{4.155455in}}{\pgfqpoint{2.279412in}{2.004545in}}%
\pgfusepath{clip}%
\pgfsetbuttcap%
\pgfsetroundjoin%
\pgfsetlinewidth{0.812944pt}%
\definecolor{currentstroke}{rgb}{0.210503,0.363727,0.552206}%
\pgfsetstrokecolor{currentstroke}%
\pgfsetdash{}{0pt}%
\pgfpathmoveto{\pgfqpoint{4.966364in}{5.371670in}}%
\pgfpathlineto{\pgfqpoint{4.928032in}{5.343467in}}%
\pgfusepath{stroke}%
\end{pgfscope}%
\begin{pgfscope}%
\pgfpathrectangle{\pgfqpoint{3.985294in}{4.155455in}}{\pgfqpoint{2.279412in}{2.004545in}}%
\pgfusepath{clip}%
\pgfsetbuttcap%
\pgfsetroundjoin%
\pgfsetlinewidth{0.807264pt}%
\definecolor{currentstroke}{rgb}{0.212395,0.359683,0.551710}%
\pgfsetstrokecolor{currentstroke}%
\pgfsetdash{}{0pt}%
\pgfpathmoveto{\pgfqpoint{4.928032in}{5.343467in}}%
\pgfpathlineto{\pgfqpoint{4.894383in}{5.311051in}}%
\pgfusepath{stroke}%
\end{pgfscope}%
\begin{pgfscope}%
\pgfpathrectangle{\pgfqpoint{3.985294in}{4.155455in}}{\pgfqpoint{2.279412in}{2.004545in}}%
\pgfusepath{clip}%
\pgfsetbuttcap%
\pgfsetroundjoin%
\pgfsetlinewidth{0.861486pt}%
\definecolor{currentstroke}{rgb}{0.197636,0.391528,0.554969}%
\pgfsetstrokecolor{currentstroke}%
\pgfsetdash{}{0pt}%
\pgfpathmoveto{\pgfqpoint{4.894383in}{5.311051in}}%
\pgfpathlineto{\pgfqpoint{4.859210in}{5.280089in}}%
\pgfusepath{stroke}%
\end{pgfscope}%
\begin{pgfscope}%
\pgfpathrectangle{\pgfqpoint{3.985294in}{4.155455in}}{\pgfqpoint{2.279412in}{2.004545in}}%
\pgfusepath{clip}%
\pgfsetbuttcap%
\pgfsetroundjoin%
\pgfsetlinewidth{0.327599pt}%
\definecolor{currentstroke}{rgb}{0.271305,0.019942,0.347269}%
\pgfsetstrokecolor{currentstroke}%
\pgfsetdash{}{0pt}%
\pgfpathmoveto{\pgfqpoint{5.843087in}{5.563688in}}%
\pgfpathlineto{\pgfqpoint{5.792964in}{5.562417in}}%
\pgfusepath{stroke}%
\end{pgfscope}%
\begin{pgfscope}%
\pgfpathrectangle{\pgfqpoint{3.985294in}{4.155455in}}{\pgfqpoint{2.279412in}{2.004545in}}%
\pgfusepath{clip}%
\pgfsetbuttcap%
\pgfsetroundjoin%
\pgfsetlinewidth{0.327808pt}%
\definecolor{currentstroke}{rgb}{0.271305,0.019942,0.347269}%
\pgfsetstrokecolor{currentstroke}%
\pgfsetdash{}{0pt}%
\pgfpathmoveto{\pgfqpoint{5.792964in}{5.562417in}}%
\pgfpathlineto{\pgfqpoint{5.742868in}{5.560699in}}%
\pgfusepath{stroke}%
\end{pgfscope}%
\begin{pgfscope}%
\pgfpathrectangle{\pgfqpoint{3.985294in}{4.155455in}}{\pgfqpoint{2.279412in}{2.004545in}}%
\pgfusepath{clip}%
\pgfsetbuttcap%
\pgfsetroundjoin%
\pgfsetlinewidth{0.331063pt}%
\definecolor{currentstroke}{rgb}{0.272594,0.025563,0.353093}%
\pgfsetstrokecolor{currentstroke}%
\pgfsetdash{}{0pt}%
\pgfpathmoveto{\pgfqpoint{5.742868in}{5.560699in}}%
\pgfpathlineto{\pgfqpoint{5.692819in}{5.557949in}}%
\pgfusepath{stroke}%
\end{pgfscope}%
\begin{pgfscope}%
\pgfpathrectangle{\pgfqpoint{3.985294in}{4.155455in}}{\pgfqpoint{2.279412in}{2.004545in}}%
\pgfusepath{clip}%
\pgfsetbuttcap%
\pgfsetroundjoin%
\pgfsetlinewidth{0.347110pt}%
\definecolor{currentstroke}{rgb}{0.274952,0.037752,0.364543}%
\pgfsetstrokecolor{currentstroke}%
\pgfsetdash{}{0pt}%
\pgfpathmoveto{\pgfqpoint{5.692819in}{5.557949in}}%
\pgfpathlineto{\pgfqpoint{5.642790in}{5.554890in}}%
\pgfusepath{stroke}%
\end{pgfscope}%
\begin{pgfscope}%
\pgfpathrectangle{\pgfqpoint{3.985294in}{4.155455in}}{\pgfqpoint{2.279412in}{2.004545in}}%
\pgfusepath{clip}%
\pgfsetbuttcap%
\pgfsetroundjoin%
\pgfsetlinewidth{0.367111pt}%
\definecolor{currentstroke}{rgb}{0.277941,0.056324,0.381191}%
\pgfsetstrokecolor{currentstroke}%
\pgfsetdash{}{0pt}%
\pgfpathmoveto{\pgfqpoint{5.642790in}{5.554890in}}%
\pgfpathlineto{\pgfqpoint{5.592816in}{5.551221in}}%
\pgfusepath{stroke}%
\end{pgfscope}%
\begin{pgfscope}%
\pgfpathrectangle{\pgfqpoint{3.985294in}{4.155455in}}{\pgfqpoint{2.279412in}{2.004545in}}%
\pgfusepath{clip}%
\pgfsetbuttcap%
\pgfsetroundjoin%
\pgfsetlinewidth{0.378307pt}%
\definecolor{currentstroke}{rgb}{0.279566,0.067836,0.391917}%
\pgfsetstrokecolor{currentstroke}%
\pgfsetdash{}{0pt}%
\pgfpathmoveto{\pgfqpoint{5.592816in}{5.551221in}}%
\pgfpathlineto{\pgfqpoint{5.542915in}{5.546855in}}%
\pgfusepath{stroke}%
\end{pgfscope}%
\begin{pgfscope}%
\pgfpathrectangle{\pgfqpoint{3.985294in}{4.155455in}}{\pgfqpoint{2.279412in}{2.004545in}}%
\pgfusepath{clip}%
\pgfsetbuttcap%
\pgfsetroundjoin%
\pgfsetlinewidth{0.401847pt}%
\definecolor{currentstroke}{rgb}{0.281446,0.084320,0.407414}%
\pgfsetstrokecolor{currentstroke}%
\pgfsetdash{}{0pt}%
\pgfpathmoveto{\pgfqpoint{5.542915in}{5.546855in}}%
\pgfpathlineto{\pgfqpoint{5.493072in}{5.542034in}}%
\pgfusepath{stroke}%
\end{pgfscope}%
\begin{pgfscope}%
\pgfpathrectangle{\pgfqpoint{3.985294in}{4.155455in}}{\pgfqpoint{2.279412in}{2.004545in}}%
\pgfusepath{clip}%
\pgfsetbuttcap%
\pgfsetroundjoin%
\pgfsetlinewidth{0.433171pt}%
\definecolor{currentstroke}{rgb}{0.283091,0.110553,0.431554}%
\pgfsetstrokecolor{currentstroke}%
\pgfsetdash{}{0pt}%
\pgfpathmoveto{\pgfqpoint{5.493072in}{5.542034in}}%
\pgfpathlineto{\pgfqpoint{5.443260in}{5.536938in}}%
\pgfusepath{stroke}%
\end{pgfscope}%
\begin{pgfscope}%
\pgfpathrectangle{\pgfqpoint{3.985294in}{4.155455in}}{\pgfqpoint{2.279412in}{2.004545in}}%
\pgfusepath{clip}%
\pgfsetbuttcap%
\pgfsetroundjoin%
\pgfsetlinewidth{0.451262pt}%
\definecolor{currentstroke}{rgb}{0.283229,0.120777,0.440584}%
\pgfsetstrokecolor{currentstroke}%
\pgfsetdash{}{0pt}%
\pgfpathmoveto{\pgfqpoint{5.443260in}{5.536938in}}%
\pgfpathlineto{\pgfqpoint{5.393499in}{5.531458in}}%
\pgfusepath{stroke}%
\end{pgfscope}%
\begin{pgfscope}%
\pgfpathrectangle{\pgfqpoint{3.985294in}{4.155455in}}{\pgfqpoint{2.279412in}{2.004545in}}%
\pgfusepath{clip}%
\pgfsetbuttcap%
\pgfsetroundjoin%
\pgfsetlinewidth{0.471024pt}%
\definecolor{currentstroke}{rgb}{0.282884,0.135920,0.453427}%
\pgfsetstrokecolor{currentstroke}%
\pgfsetdash{}{0pt}%
\pgfpathmoveto{\pgfqpoint{5.393499in}{5.531458in}}%
\pgfpathlineto{\pgfqpoint{5.343926in}{5.524839in}}%
\pgfusepath{stroke}%
\end{pgfscope}%
\begin{pgfscope}%
\pgfpathrectangle{\pgfqpoint{3.985294in}{4.155455in}}{\pgfqpoint{2.279412in}{2.004545in}}%
\pgfusepath{clip}%
\pgfsetbuttcap%
\pgfsetroundjoin%
\pgfsetlinewidth{0.514398pt}%
\definecolor{currentstroke}{rgb}{0.279574,0.170599,0.479997}%
\pgfsetstrokecolor{currentstroke}%
\pgfsetdash{}{0pt}%
\pgfpathmoveto{\pgfqpoint{5.343926in}{5.524839in}}%
\pgfpathlineto{\pgfqpoint{5.294525in}{5.517266in}}%
\pgfusepath{stroke}%
\end{pgfscope}%
\begin{pgfscope}%
\pgfpathrectangle{\pgfqpoint{3.985294in}{4.155455in}}{\pgfqpoint{2.279412in}{2.004545in}}%
\pgfusepath{clip}%
\pgfsetbuttcap%
\pgfsetroundjoin%
\pgfsetlinewidth{0.518872pt}%
\definecolor{currentstroke}{rgb}{0.279574,0.170599,0.479997}%
\pgfsetstrokecolor{currentstroke}%
\pgfsetdash{}{0pt}%
\pgfpathmoveto{\pgfqpoint{5.294525in}{5.517266in}}%
\pgfpathlineto{\pgfqpoint{5.245255in}{5.509057in}}%
\pgfusepath{stroke}%
\end{pgfscope}%
\begin{pgfscope}%
\pgfpathrectangle{\pgfqpoint{3.985294in}{4.155455in}}{\pgfqpoint{2.279412in}{2.004545in}}%
\pgfusepath{clip}%
\pgfsetbuttcap%
\pgfsetroundjoin%
\pgfsetlinewidth{0.542862pt}%
\definecolor{currentstroke}{rgb}{0.276194,0.190074,0.493001}%
\pgfsetstrokecolor{currentstroke}%
\pgfsetdash{}{0pt}%
\pgfpathmoveto{\pgfqpoint{5.245255in}{5.509057in}}%
\pgfpathlineto{\pgfqpoint{5.196426in}{5.499101in}}%
\pgfusepath{stroke}%
\end{pgfscope}%
\begin{pgfscope}%
\pgfpathrectangle{\pgfqpoint{3.985294in}{4.155455in}}{\pgfqpoint{2.279412in}{2.004545in}}%
\pgfusepath{clip}%
\pgfsetbuttcap%
\pgfsetroundjoin%
\pgfsetlinewidth{0.534298pt}%
\definecolor{currentstroke}{rgb}{0.278012,0.180367,0.486697}%
\pgfsetstrokecolor{currentstroke}%
\pgfsetdash{}{0pt}%
\pgfpathmoveto{\pgfqpoint{5.196426in}{5.499101in}}%
\pgfpathlineto{\pgfqpoint{5.148208in}{5.487007in}}%
\pgfusepath{stroke}%
\end{pgfscope}%
\begin{pgfscope}%
\pgfpathrectangle{\pgfqpoint{3.985294in}{4.155455in}}{\pgfqpoint{2.279412in}{2.004545in}}%
\pgfusepath{clip}%
\pgfsetbuttcap%
\pgfsetroundjoin%
\pgfsetlinewidth{0.520454pt}%
\definecolor{currentstroke}{rgb}{0.279574,0.170599,0.479997}%
\pgfsetstrokecolor{currentstroke}%
\pgfsetdash{}{0pt}%
\pgfpathmoveto{\pgfqpoint{5.148208in}{5.487007in}}%
\pgfpathlineto{\pgfqpoint{5.100938in}{5.472412in}}%
\pgfusepath{stroke}%
\end{pgfscope}%
\begin{pgfscope}%
\pgfpathrectangle{\pgfqpoint{3.985294in}{4.155455in}}{\pgfqpoint{2.279412in}{2.004545in}}%
\pgfusepath{clip}%
\pgfsetbuttcap%
\pgfsetroundjoin%
\pgfsetlinewidth{0.587172pt}%
\definecolor{currentstroke}{rgb}{0.269308,0.218818,0.509577}%
\pgfsetstrokecolor{currentstroke}%
\pgfsetdash{}{0pt}%
\pgfpathmoveto{\pgfqpoint{5.100938in}{5.472412in}}%
\pgfpathlineto{\pgfqpoint{5.056133in}{5.452968in}}%
\pgfusepath{stroke}%
\end{pgfscope}%
\begin{pgfscope}%
\pgfpathrectangle{\pgfqpoint{3.985294in}{4.155455in}}{\pgfqpoint{2.279412in}{2.004545in}}%
\pgfusepath{clip}%
\pgfsetbuttcap%
\pgfsetroundjoin%
\pgfsetlinewidth{0.576283pt}%
\definecolor{currentstroke}{rgb}{0.270595,0.214069,0.507052}%
\pgfsetstrokecolor{currentstroke}%
\pgfsetdash{}{0pt}%
\pgfpathmoveto{\pgfqpoint{5.056133in}{5.452968in}}%
\pgfpathlineto{\pgfqpoint{5.013715in}{5.429606in}}%
\pgfusepath{stroke}%
\end{pgfscope}%
\begin{pgfscope}%
\pgfpathrectangle{\pgfqpoint{3.985294in}{4.155455in}}{\pgfqpoint{2.279412in}{2.004545in}}%
\pgfusepath{clip}%
\pgfsetbuttcap%
\pgfsetroundjoin%
\pgfsetlinewidth{0.318516pt}%
\definecolor{currentstroke}{rgb}{0.269944,0.014625,0.341379}%
\pgfsetstrokecolor{currentstroke}%
\pgfsetdash{}{0pt}%
\pgfpathmoveto{\pgfqpoint{5.791795in}{4.706659in}}%
\pgfpathlineto{\pgfqpoint{5.741712in}{4.708663in}}%
\pgfusepath{stroke}%
\end{pgfscope}%
\begin{pgfscope}%
\pgfpathrectangle{\pgfqpoint{3.985294in}{4.155455in}}{\pgfqpoint{2.279412in}{2.004545in}}%
\pgfusepath{clip}%
\pgfsetbuttcap%
\pgfsetroundjoin%
\pgfsetlinewidth{0.330957pt}%
\definecolor{currentstroke}{rgb}{0.272594,0.025563,0.353093}%
\pgfsetstrokecolor{currentstroke}%
\pgfsetdash{}{0pt}%
\pgfpathmoveto{\pgfqpoint{5.741712in}{4.708663in}}%
\pgfpathlineto{\pgfqpoint{5.691646in}{4.711018in}}%
\pgfusepath{stroke}%
\end{pgfscope}%
\begin{pgfscope}%
\pgfpathrectangle{\pgfqpoint{3.985294in}{4.155455in}}{\pgfqpoint{2.279412in}{2.004545in}}%
\pgfusepath{clip}%
\pgfsetbuttcap%
\pgfsetroundjoin%
\pgfsetlinewidth{0.346540pt}%
\definecolor{currentstroke}{rgb}{0.274952,0.037752,0.364543}%
\pgfsetstrokecolor{currentstroke}%
\pgfsetdash{}{0pt}%
\pgfpathmoveto{\pgfqpoint{5.691646in}{4.711018in}}%
\pgfpathlineto{\pgfqpoint{5.641630in}{4.714070in}}%
\pgfusepath{stroke}%
\end{pgfscope}%
\begin{pgfscope}%
\pgfpathrectangle{\pgfqpoint{3.985294in}{4.155455in}}{\pgfqpoint{2.279412in}{2.004545in}}%
\pgfusepath{clip}%
\pgfsetbuttcap%
\pgfsetroundjoin%
\pgfsetlinewidth{0.346036pt}%
\definecolor{currentstroke}{rgb}{0.274952,0.037752,0.364543}%
\pgfsetstrokecolor{currentstroke}%
\pgfsetdash{}{0pt}%
\pgfpathmoveto{\pgfqpoint{5.641630in}{4.714070in}}%
\pgfpathlineto{\pgfqpoint{5.591710in}{4.718169in}}%
\pgfusepath{stroke}%
\end{pgfscope}%
\begin{pgfscope}%
\pgfpathrectangle{\pgfqpoint{3.985294in}{4.155455in}}{\pgfqpoint{2.279412in}{2.004545in}}%
\pgfusepath{clip}%
\pgfsetbuttcap%
\pgfsetroundjoin%
\pgfsetlinewidth{0.373803pt}%
\definecolor{currentstroke}{rgb}{0.278791,0.062145,0.386592}%
\pgfsetstrokecolor{currentstroke}%
\pgfsetdash{}{0pt}%
\pgfpathmoveto{\pgfqpoint{5.591710in}{4.718169in}}%
\pgfpathlineto{\pgfqpoint{5.541900in}{4.723222in}}%
\pgfusepath{stroke}%
\end{pgfscope}%
\begin{pgfscope}%
\pgfpathrectangle{\pgfqpoint{3.985294in}{4.155455in}}{\pgfqpoint{2.279412in}{2.004545in}}%
\pgfusepath{clip}%
\pgfsetbuttcap%
\pgfsetroundjoin%
\pgfsetlinewidth{0.389306pt}%
\definecolor{currentstroke}{rgb}{0.280267,0.073417,0.397163}%
\pgfsetstrokecolor{currentstroke}%
\pgfsetdash{}{0pt}%
\pgfpathmoveto{\pgfqpoint{5.541900in}{4.723222in}}%
\pgfpathlineto{\pgfqpoint{5.492128in}{4.728608in}}%
\pgfusepath{stroke}%
\end{pgfscope}%
\begin{pgfscope}%
\pgfpathrectangle{\pgfqpoint{3.985294in}{4.155455in}}{\pgfqpoint{2.279412in}{2.004545in}}%
\pgfusepath{clip}%
\pgfsetbuttcap%
\pgfsetroundjoin%
\pgfsetlinewidth{0.398488pt}%
\definecolor{currentstroke}{rgb}{0.281446,0.084320,0.407414}%
\pgfsetstrokecolor{currentstroke}%
\pgfsetdash{}{0pt}%
\pgfpathmoveto{\pgfqpoint{5.492128in}{4.728608in}}%
\pgfpathlineto{\pgfqpoint{5.442354in}{4.733993in}}%
\pgfusepath{stroke}%
\end{pgfscope}%
\begin{pgfscope}%
\pgfpathrectangle{\pgfqpoint{3.985294in}{4.155455in}}{\pgfqpoint{2.279412in}{2.004545in}}%
\pgfusepath{clip}%
\pgfsetbuttcap%
\pgfsetroundjoin%
\pgfsetlinewidth{0.432059pt}%
\definecolor{currentstroke}{rgb}{0.283091,0.110553,0.431554}%
\pgfsetstrokecolor{currentstroke}%
\pgfsetdash{}{0pt}%
\pgfpathmoveto{\pgfqpoint{5.442354in}{4.733993in}}%
\pgfpathlineto{\pgfqpoint{5.392675in}{4.740007in}}%
\pgfusepath{stroke}%
\end{pgfscope}%
\begin{pgfscope}%
\pgfpathrectangle{\pgfqpoint{3.985294in}{4.155455in}}{\pgfqpoint{2.279412in}{2.004545in}}%
\pgfusepath{clip}%
\pgfsetbuttcap%
\pgfsetroundjoin%
\pgfsetlinewidth{0.447987pt}%
\definecolor{currentstroke}{rgb}{0.283229,0.120777,0.440584}%
\pgfsetstrokecolor{currentstroke}%
\pgfsetdash{}{0pt}%
\pgfpathmoveto{\pgfqpoint{5.392675in}{4.740007in}}%
\pgfpathlineto{\pgfqpoint{5.343175in}{4.747049in}}%
\pgfusepath{stroke}%
\end{pgfscope}%
\begin{pgfscope}%
\pgfpathrectangle{\pgfqpoint{3.985294in}{4.155455in}}{\pgfqpoint{2.279412in}{2.004545in}}%
\pgfusepath{clip}%
\pgfsetbuttcap%
\pgfsetroundjoin%
\pgfsetlinewidth{0.483902pt}%
\definecolor{currentstroke}{rgb}{0.282290,0.145912,0.461510}%
\pgfsetstrokecolor{currentstroke}%
\pgfsetdash{}{0pt}%
\pgfpathmoveto{\pgfqpoint{5.343175in}{4.747049in}}%
\pgfpathlineto{\pgfqpoint{5.293862in}{4.755023in}}%
\pgfusepath{stroke}%
\end{pgfscope}%
\begin{pgfscope}%
\pgfpathrectangle{\pgfqpoint{3.985294in}{4.155455in}}{\pgfqpoint{2.279412in}{2.004545in}}%
\pgfusepath{clip}%
\pgfsetbuttcap%
\pgfsetroundjoin%
\pgfsetlinewidth{0.504077pt}%
\definecolor{currentstroke}{rgb}{0.280868,0.160771,0.472899}%
\pgfsetstrokecolor{currentstroke}%
\pgfsetdash{}{0pt}%
\pgfpathmoveto{\pgfqpoint{5.293862in}{4.755023in}}%
\pgfpathlineto{\pgfqpoint{5.244836in}{4.764224in}}%
\pgfusepath{stroke}%
\end{pgfscope}%
\begin{pgfscope}%
\pgfpathrectangle{\pgfqpoint{3.985294in}{4.155455in}}{\pgfqpoint{2.279412in}{2.004545in}}%
\pgfusepath{clip}%
\pgfsetbuttcap%
\pgfsetroundjoin%
\pgfsetlinewidth{0.473964pt}%
\definecolor{currentstroke}{rgb}{0.282623,0.140926,0.457517}%
\pgfsetstrokecolor{currentstroke}%
\pgfsetdash{}{0pt}%
\pgfpathmoveto{\pgfqpoint{5.244836in}{4.764224in}}%
\pgfpathlineto{\pgfqpoint{5.196550in}{4.775949in}}%
\pgfusepath{stroke}%
\end{pgfscope}%
\begin{pgfscope}%
\pgfpathrectangle{\pgfqpoint{3.985294in}{4.155455in}}{\pgfqpoint{2.279412in}{2.004545in}}%
\pgfusepath{clip}%
\pgfsetbuttcap%
\pgfsetroundjoin%
\pgfsetlinewidth{0.472725pt}%
\definecolor{currentstroke}{rgb}{0.282623,0.140926,0.457517}%
\pgfsetstrokecolor{currentstroke}%
\pgfsetdash{}{0pt}%
\pgfpathmoveto{\pgfqpoint{5.196550in}{4.775949in}}%
\pgfpathlineto{\pgfqpoint{5.149179in}{4.790300in}}%
\pgfusepath{stroke}%
\end{pgfscope}%
\begin{pgfscope}%
\pgfpathrectangle{\pgfqpoint{3.985294in}{4.155455in}}{\pgfqpoint{2.279412in}{2.004545in}}%
\pgfusepath{clip}%
\pgfsetbuttcap%
\pgfsetroundjoin%
\pgfsetlinewidth{0.515725pt}%
\definecolor{currentstroke}{rgb}{0.279574,0.170599,0.479997}%
\pgfsetstrokecolor{currentstroke}%
\pgfsetdash{}{0pt}%
\pgfpathmoveto{\pgfqpoint{5.149179in}{4.790300in}}%
\pgfpathlineto{\pgfqpoint{5.102859in}{4.807071in}}%
\pgfusepath{stroke}%
\end{pgfscope}%
\begin{pgfscope}%
\pgfpathrectangle{\pgfqpoint{3.985294in}{4.155455in}}{\pgfqpoint{2.279412in}{2.004545in}}%
\pgfusepath{clip}%
\pgfsetbuttcap%
\pgfsetroundjoin%
\pgfsetlinewidth{0.577690pt}%
\definecolor{currentstroke}{rgb}{0.270595,0.214069,0.507052}%
\pgfsetstrokecolor{currentstroke}%
\pgfsetdash{}{0pt}%
\pgfpathmoveto{\pgfqpoint{5.102859in}{4.807071in}}%
\pgfpathlineto{\pgfqpoint{5.058867in}{4.827942in}}%
\pgfusepath{stroke}%
\end{pgfscope}%
\begin{pgfscope}%
\pgfpathrectangle{\pgfqpoint{3.985294in}{4.155455in}}{\pgfqpoint{2.279412in}{2.004545in}}%
\pgfusepath{clip}%
\pgfsetbuttcap%
\pgfsetroundjoin%
\pgfsetlinewidth{0.510573pt}%
\definecolor{currentstroke}{rgb}{0.280255,0.165693,0.476498}%
\pgfsetstrokecolor{currentstroke}%
\pgfsetdash{}{0pt}%
\pgfpathmoveto{\pgfqpoint{5.058867in}{4.827942in}}%
\pgfpathlineto{\pgfqpoint{5.017821in}{4.853094in}}%
\pgfusepath{stroke}%
\end{pgfscope}%
\begin{pgfscope}%
\pgfpathrectangle{\pgfqpoint{3.985294in}{4.155455in}}{\pgfqpoint{2.279412in}{2.004545in}}%
\pgfusepath{clip}%
\pgfsetbuttcap%
\pgfsetroundjoin%
\pgfsetlinewidth{0.608166pt}%
\definecolor{currentstroke}{rgb}{0.265145,0.232956,0.516599}%
\pgfsetstrokecolor{currentstroke}%
\pgfsetdash{}{0pt}%
\pgfpathmoveto{\pgfqpoint{5.017821in}{4.853094in}}%
\pgfpathlineto{\pgfqpoint{4.980290in}{4.882046in}}%
\pgfusepath{stroke}%
\end{pgfscope}%
\begin{pgfscope}%
\pgfpathrectangle{\pgfqpoint{3.985294in}{4.155455in}}{\pgfqpoint{2.279412in}{2.004545in}}%
\pgfusepath{clip}%
\pgfsetbuttcap%
\pgfsetroundjoin%
\pgfsetlinewidth{0.699106pt}%
\definecolor{currentstroke}{rgb}{0.241237,0.296485,0.539709}%
\pgfsetstrokecolor{currentstroke}%
\pgfsetdash{}{0pt}%
\pgfpathmoveto{\pgfqpoint{4.980290in}{4.882046in}}%
\pgfpathlineto{\pgfqpoint{4.945610in}{4.913673in}}%
\pgfusepath{stroke}%
\end{pgfscope}%
\begin{pgfscope}%
\pgfpathrectangle{\pgfqpoint{3.985294in}{4.155455in}}{\pgfqpoint{2.279412in}{2.004545in}}%
\pgfusepath{clip}%
\pgfsetbuttcap%
\pgfsetroundjoin%
\pgfsetlinewidth{0.766848pt}%
\definecolor{currentstroke}{rgb}{0.223925,0.334994,0.548053}%
\pgfsetstrokecolor{currentstroke}%
\pgfsetdash{}{0pt}%
\pgfpathmoveto{\pgfqpoint{4.945610in}{4.913673in}}%
\pgfpathlineto{\pgfqpoint{4.911852in}{4.945548in}}%
\pgfusepath{stroke}%
\end{pgfscope}%
\begin{pgfscope}%
\pgfpathrectangle{\pgfqpoint{3.985294in}{4.155455in}}{\pgfqpoint{2.279412in}{2.004545in}}%
\pgfusepath{clip}%
\pgfsetbuttcap%
\pgfsetroundjoin%
\pgfsetlinewidth{0.940412pt}%
\definecolor{currentstroke}{rgb}{0.177423,0.437527,0.557565}%
\pgfsetstrokecolor{currentstroke}%
\pgfsetdash{}{0pt}%
\pgfpathmoveto{\pgfqpoint{4.911852in}{4.945548in}}%
\pgfpathlineto{\pgfqpoint{4.911852in}{4.945548in}}%
\pgfusepath{stroke}%
\end{pgfscope}%
\begin{pgfscope}%
\pgfpathrectangle{\pgfqpoint{3.985294in}{4.155455in}}{\pgfqpoint{2.279412in}{2.004545in}}%
\pgfusepath{clip}%
\pgfsetbuttcap%
\pgfsetroundjoin%
\pgfsetlinewidth{0.940412pt}%
\definecolor{currentstroke}{rgb}{0.177423,0.437527,0.557565}%
\pgfsetstrokecolor{currentstroke}%
\pgfsetdash{}{0pt}%
\pgfpathmoveto{\pgfqpoint{4.911852in}{4.945548in}}%
\pgfpathlineto{\pgfqpoint{4.911852in}{4.945548in}}%
\pgfusepath{stroke}%
\end{pgfscope}%
\begin{pgfscope}%
\pgfpathrectangle{\pgfqpoint{3.985294in}{4.155455in}}{\pgfqpoint{2.279412in}{2.004545in}}%
\pgfusepath{clip}%
\pgfsetbuttcap%
\pgfsetroundjoin%
\pgfsetlinewidth{0.940412pt}%
\definecolor{currentstroke}{rgb}{0.177423,0.437527,0.557565}%
\pgfsetstrokecolor{currentstroke}%
\pgfsetdash{}{0pt}%
\pgfpathmoveto{\pgfqpoint{4.911852in}{4.945548in}}%
\pgfpathlineto{\pgfqpoint{4.911852in}{4.945548in}}%
\pgfusepath{stroke}%
\end{pgfscope}%
\begin{pgfscope}%
\pgfpathrectangle{\pgfqpoint{3.985294in}{4.155455in}}{\pgfqpoint{2.279412in}{2.004545in}}%
\pgfusepath{clip}%
\pgfsetbuttcap%
\pgfsetroundjoin%
\pgfsetlinewidth{0.310931pt}%
\definecolor{currentstroke}{rgb}{0.268510,0.009605,0.335427}%
\pgfsetstrokecolor{currentstroke}%
\pgfsetdash{}{0pt}%
\pgfpathmoveto{\pgfqpoint{5.876105in}{5.706395in}}%
\pgfpathlineto{\pgfqpoint{5.833987in}{5.703725in}}%
\pgfusepath{stroke}%
\end{pgfscope}%
\begin{pgfscope}%
\pgfpathrectangle{\pgfqpoint{3.985294in}{4.155455in}}{\pgfqpoint{2.279412in}{2.004545in}}%
\pgfusepath{clip}%
\pgfsetbuttcap%
\pgfsetroundjoin%
\pgfsetlinewidth{0.317385pt}%
\definecolor{currentstroke}{rgb}{0.269944,0.014625,0.341379}%
\pgfsetstrokecolor{currentstroke}%
\pgfsetdash{}{0pt}%
\pgfpathmoveto{\pgfqpoint{5.833987in}{5.703725in}}%
\pgfpathlineto{\pgfqpoint{5.791795in}{5.699009in}}%
\pgfusepath{stroke}%
\end{pgfscope}%
\begin{pgfscope}%
\pgfpathrectangle{\pgfqpoint{3.985294in}{4.155455in}}{\pgfqpoint{2.279412in}{2.004545in}}%
\pgfusepath{clip}%
\pgfsetbuttcap%
\pgfsetroundjoin%
\pgfsetlinewidth{0.317927pt}%
\definecolor{currentstroke}{rgb}{0.269944,0.014625,0.341379}%
\pgfsetstrokecolor{currentstroke}%
\pgfsetdash{}{0pt}%
\pgfpathmoveto{\pgfqpoint{5.791795in}{5.699009in}}%
\pgfpathlineto{\pgfqpoint{5.743137in}{5.696620in}}%
\pgfusepath{stroke}%
\end{pgfscope}%
\begin{pgfscope}%
\pgfpathrectangle{\pgfqpoint{3.985294in}{4.155455in}}{\pgfqpoint{2.279412in}{2.004545in}}%
\pgfusepath{clip}%
\pgfsetbuttcap%
\pgfsetroundjoin%
\pgfsetlinewidth{0.324859pt}%
\definecolor{currentstroke}{rgb}{0.271305,0.019942,0.347269}%
\pgfsetstrokecolor{currentstroke}%
\pgfsetdash{}{0pt}%
\pgfpathmoveto{\pgfqpoint{5.743137in}{5.696620in}}%
\pgfpathlineto{\pgfqpoint{5.693890in}{5.690483in}}%
\pgfusepath{stroke}%
\end{pgfscope}%
\begin{pgfscope}%
\pgfpathrectangle{\pgfqpoint{3.985294in}{4.155455in}}{\pgfqpoint{2.279412in}{2.004545in}}%
\pgfusepath{clip}%
\pgfsetbuttcap%
\pgfsetroundjoin%
\pgfsetlinewidth{0.321607pt}%
\definecolor{currentstroke}{rgb}{0.269944,0.014625,0.341379}%
\pgfsetstrokecolor{currentstroke}%
\pgfsetdash{}{0pt}%
\pgfpathmoveto{\pgfqpoint{5.693890in}{5.690483in}}%
\pgfpathlineto{\pgfqpoint{5.644501in}{5.684682in}}%
\pgfusepath{stroke}%
\end{pgfscope}%
\begin{pgfscope}%
\pgfpathrectangle{\pgfqpoint{3.985294in}{4.155455in}}{\pgfqpoint{2.279412in}{2.004545in}}%
\pgfusepath{clip}%
\pgfsetbuttcap%
\pgfsetroundjoin%
\pgfsetlinewidth{0.333702pt}%
\definecolor{currentstroke}{rgb}{0.272594,0.025563,0.353093}%
\pgfsetstrokecolor{currentstroke}%
\pgfsetdash{}{0pt}%
\pgfpathmoveto{\pgfqpoint{5.644501in}{5.684682in}}%
\pgfpathlineto{\pgfqpoint{5.594501in}{5.681329in}}%
\pgfusepath{stroke}%
\end{pgfscope}%
\begin{pgfscope}%
\pgfpathrectangle{\pgfqpoint{3.985294in}{4.155455in}}{\pgfqpoint{2.279412in}{2.004545in}}%
\pgfusepath{clip}%
\pgfsetbuttcap%
\pgfsetroundjoin%
\pgfsetlinewidth{0.335115pt}%
\definecolor{currentstroke}{rgb}{0.272594,0.025563,0.353093}%
\pgfsetstrokecolor{currentstroke}%
\pgfsetdash{}{0pt}%
\pgfpathmoveto{\pgfqpoint{5.594501in}{5.681329in}}%
\pgfpathlineto{\pgfqpoint{5.544537in}{5.677704in}}%
\pgfusepath{stroke}%
\end{pgfscope}%
\begin{pgfscope}%
\pgfpathrectangle{\pgfqpoint{3.985294in}{4.155455in}}{\pgfqpoint{2.279412in}{2.004545in}}%
\pgfusepath{clip}%
\pgfsetbuttcap%
\pgfsetroundjoin%
\pgfsetlinewidth{0.348179pt}%
\definecolor{currentstroke}{rgb}{0.274952,0.037752,0.364543}%
\pgfsetstrokecolor{currentstroke}%
\pgfsetdash{}{0pt}%
\pgfpathmoveto{\pgfqpoint{5.544537in}{5.677704in}}%
\pgfpathlineto{\pgfqpoint{5.494595in}{5.673832in}}%
\pgfusepath{stroke}%
\end{pgfscope}%
\begin{pgfscope}%
\pgfpathrectangle{\pgfqpoint{3.985294in}{4.155455in}}{\pgfqpoint{2.279412in}{2.004545in}}%
\pgfusepath{clip}%
\pgfsetbuttcap%
\pgfsetroundjoin%
\pgfsetlinewidth{0.375515pt}%
\definecolor{currentstroke}{rgb}{0.278791,0.062145,0.386592}%
\pgfsetstrokecolor{currentstroke}%
\pgfsetdash{}{0pt}%
\pgfpathmoveto{\pgfqpoint{5.494595in}{5.673832in}}%
\pgfpathlineto{\pgfqpoint{5.444771in}{5.668909in}}%
\pgfusepath{stroke}%
\end{pgfscope}%
\begin{pgfscope}%
\pgfpathrectangle{\pgfqpoint{3.985294in}{4.155455in}}{\pgfqpoint{2.279412in}{2.004545in}}%
\pgfusepath{clip}%
\pgfsetbuttcap%
\pgfsetroundjoin%
\pgfsetlinewidth{0.380611pt}%
\definecolor{currentstroke}{rgb}{0.279566,0.067836,0.391917}%
\pgfsetstrokecolor{currentstroke}%
\pgfsetdash{}{0pt}%
\pgfpathmoveto{\pgfqpoint{5.444771in}{5.668909in}}%
\pgfpathlineto{\pgfqpoint{5.395189in}{5.662336in}}%
\pgfusepath{stroke}%
\end{pgfscope}%
\begin{pgfscope}%
\pgfpathrectangle{\pgfqpoint{3.985294in}{4.155455in}}{\pgfqpoint{2.279412in}{2.004545in}}%
\pgfusepath{clip}%
\pgfsetbuttcap%
\pgfsetroundjoin%
\pgfsetlinewidth{0.394710pt}%
\definecolor{currentstroke}{rgb}{0.280894,0.078907,0.402329}%
\pgfsetstrokecolor{currentstroke}%
\pgfsetdash{}{0pt}%
\pgfpathmoveto{\pgfqpoint{5.395189in}{5.662336in}}%
\pgfpathlineto{\pgfqpoint{5.345972in}{5.653971in}}%
\pgfusepath{stroke}%
\end{pgfscope}%
\begin{pgfscope}%
\pgfpathrectangle{\pgfqpoint{3.985294in}{4.155455in}}{\pgfqpoint{2.279412in}{2.004545in}}%
\pgfusepath{clip}%
\pgfsetbuttcap%
\pgfsetroundjoin%
\pgfsetlinewidth{0.400815pt}%
\definecolor{currentstroke}{rgb}{0.281446,0.084320,0.407414}%
\pgfsetstrokecolor{currentstroke}%
\pgfsetdash{}{0pt}%
\pgfpathmoveto{\pgfqpoint{5.345972in}{5.653971in}}%
\pgfpathlineto{\pgfqpoint{5.296982in}{5.644604in}}%
\pgfusepath{stroke}%
\end{pgfscope}%
\begin{pgfscope}%
\pgfpathrectangle{\pgfqpoint{3.985294in}{4.155455in}}{\pgfqpoint{2.279412in}{2.004545in}}%
\pgfusepath{clip}%
\pgfsetbuttcap%
\pgfsetroundjoin%
\pgfsetlinewidth{0.417331pt}%
\definecolor{currentstroke}{rgb}{0.282327,0.094955,0.417331}%
\pgfsetstrokecolor{currentstroke}%
\pgfsetdash{}{0pt}%
\pgfpathmoveto{\pgfqpoint{5.296982in}{5.644604in}}%
\pgfpathlineto{\pgfqpoint{5.248191in}{5.634491in}}%
\pgfusepath{stroke}%
\end{pgfscope}%
\begin{pgfscope}%
\pgfpathrectangle{\pgfqpoint{3.985294in}{4.155455in}}{\pgfqpoint{2.279412in}{2.004545in}}%
\pgfusepath{clip}%
\pgfsetbuttcap%
\pgfsetroundjoin%
\pgfsetlinewidth{0.437539pt}%
\definecolor{currentstroke}{rgb}{0.283091,0.110553,0.431554}%
\pgfsetstrokecolor{currentstroke}%
\pgfsetdash{}{0pt}%
\pgfpathmoveto{\pgfqpoint{5.248191in}{5.634491in}}%
\pgfpathlineto{\pgfqpoint{5.200048in}{5.622260in}}%
\pgfusepath{stroke}%
\end{pgfscope}%
\begin{pgfscope}%
\pgfpathrectangle{\pgfqpoint{3.985294in}{4.155455in}}{\pgfqpoint{2.279412in}{2.004545in}}%
\pgfusepath{clip}%
\pgfsetbuttcap%
\pgfsetroundjoin%
\pgfsetlinewidth{0.419820pt}%
\definecolor{currentstroke}{rgb}{0.282656,0.100196,0.422160}%
\pgfsetstrokecolor{currentstroke}%
\pgfsetdash{}{0pt}%
\pgfpathmoveto{\pgfqpoint{5.200048in}{5.622260in}}%
\pgfpathlineto{\pgfqpoint{5.154122in}{5.605350in}}%
\pgfusepath{stroke}%
\end{pgfscope}%
\begin{pgfscope}%
\pgfpathrectangle{\pgfqpoint{3.985294in}{4.155455in}}{\pgfqpoint{2.279412in}{2.004545in}}%
\pgfusepath{clip}%
\pgfsetbuttcap%
\pgfsetroundjoin%
\pgfsetlinewidth{0.427521pt}%
\definecolor{currentstroke}{rgb}{0.282910,0.105393,0.426902}%
\pgfsetstrokecolor{currentstroke}%
\pgfsetdash{}{0pt}%
\pgfpathmoveto{\pgfqpoint{5.154122in}{5.605350in}}%
\pgfpathlineto{\pgfqpoint{5.109312in}{5.585966in}}%
\pgfusepath{stroke}%
\end{pgfscope}%
\begin{pgfscope}%
\pgfpathrectangle{\pgfqpoint{3.985294in}{4.155455in}}{\pgfqpoint{2.279412in}{2.004545in}}%
\pgfusepath{clip}%
\pgfsetbuttcap%
\pgfsetroundjoin%
\pgfsetlinewidth{0.493509pt}%
\definecolor{currentstroke}{rgb}{0.281412,0.155834,0.469201}%
\pgfsetstrokecolor{currentstroke}%
\pgfsetdash{}{0pt}%
\pgfpathmoveto{\pgfqpoint{5.109312in}{5.585966in}}%
\pgfpathlineto{\pgfqpoint{5.066402in}{5.563609in}}%
\pgfusepath{stroke}%
\end{pgfscope}%
\begin{pgfscope}%
\pgfpathrectangle{\pgfqpoint{3.985294in}{4.155455in}}{\pgfqpoint{2.279412in}{2.004545in}}%
\pgfusepath{clip}%
\pgfsetbuttcap%
\pgfsetroundjoin%
\pgfsetlinewidth{0.464177pt}%
\definecolor{currentstroke}{rgb}{0.283072,0.130895,0.449241}%
\pgfsetstrokecolor{currentstroke}%
\pgfsetdash{}{0pt}%
\pgfpathmoveto{\pgfqpoint{5.066402in}{5.563609in}}%
\pgfpathlineto{\pgfqpoint{5.027627in}{5.536171in}}%
\pgfusepath{stroke}%
\end{pgfscope}%
\begin{pgfscope}%
\pgfpathrectangle{\pgfqpoint{3.985294in}{4.155455in}}{\pgfqpoint{2.279412in}{2.004545in}}%
\pgfusepath{clip}%
\pgfsetbuttcap%
\pgfsetroundjoin%
\pgfsetlinewidth{0.532574pt}%
\definecolor{currentstroke}{rgb}{0.278012,0.180367,0.486697}%
\pgfsetstrokecolor{currentstroke}%
\pgfsetdash{}{0pt}%
\pgfpathmoveto{\pgfqpoint{5.027627in}{5.536171in}}%
\pgfpathlineto{\pgfqpoint{5.027627in}{5.536171in}}%
\pgfusepath{stroke}%
\end{pgfscope}%
\begin{pgfscope}%
\pgfpathrectangle{\pgfqpoint{3.985294in}{4.155455in}}{\pgfqpoint{2.279412in}{2.004545in}}%
\pgfusepath{clip}%
\pgfsetbuttcap%
\pgfsetroundjoin%
\pgfsetlinewidth{0.532574pt}%
\definecolor{currentstroke}{rgb}{0.278012,0.180367,0.486697}%
\pgfsetstrokecolor{currentstroke}%
\pgfsetdash{}{0pt}%
\pgfpathmoveto{\pgfqpoint{5.027627in}{5.536171in}}%
\pgfpathlineto{\pgfqpoint{5.003372in}{5.510144in}}%
\pgfusepath{stroke}%
\end{pgfscope}%
\begin{pgfscope}%
\pgfpathrectangle{\pgfqpoint{3.985294in}{4.155455in}}{\pgfqpoint{2.279412in}{2.004545in}}%
\pgfusepath{clip}%
\pgfsetbuttcap%
\pgfsetroundjoin%
\pgfsetlinewidth{0.572507pt}%
\definecolor{currentstroke}{rgb}{0.271828,0.209303,0.504434}%
\pgfsetstrokecolor{currentstroke}%
\pgfsetdash{}{0pt}%
\pgfpathmoveto{\pgfqpoint{5.003372in}{5.510144in}}%
\pgfpathlineto{\pgfqpoint{4.979499in}{5.481632in}}%
\pgfusepath{stroke}%
\end{pgfscope}%
\begin{pgfscope}%
\pgfpathrectangle{\pgfqpoint{3.985294in}{4.155455in}}{\pgfqpoint{2.279412in}{2.004545in}}%
\pgfusepath{clip}%
\pgfsetbuttcap%
\pgfsetroundjoin%
\pgfsetlinewidth{0.547992pt}%
\definecolor{currentstroke}{rgb}{0.276194,0.190074,0.493001}%
\pgfsetstrokecolor{currentstroke}%
\pgfsetdash{}{0pt}%
\pgfpathmoveto{\pgfqpoint{4.979499in}{5.481632in}}%
\pgfpathlineto{\pgfqpoint{4.953959in}{5.445063in}}%
\pgfusepath{stroke}%
\end{pgfscope}%
\begin{pgfscope}%
\pgfpathrectangle{\pgfqpoint{3.985294in}{4.155455in}}{\pgfqpoint{2.279412in}{2.004545in}}%
\pgfusepath{clip}%
\pgfsetbuttcap%
\pgfsetroundjoin%
\pgfsetlinewidth{0.638273pt}%
\definecolor{currentstroke}{rgb}{0.257322,0.256130,0.526563}%
\pgfsetstrokecolor{currentstroke}%
\pgfsetdash{}{0pt}%
\pgfpathmoveto{\pgfqpoint{4.953959in}{5.445063in}}%
\pgfpathlineto{\pgfqpoint{4.930649in}{5.406866in}}%
\pgfusepath{stroke}%
\end{pgfscope}%
\begin{pgfscope}%
\pgfpathrectangle{\pgfqpoint{3.985294in}{4.155455in}}{\pgfqpoint{2.279412in}{2.004545in}}%
\pgfusepath{clip}%
\pgfsetbuttcap%
\pgfsetroundjoin%
\pgfsetlinewidth{0.706192pt}%
\definecolor{currentstroke}{rgb}{0.239346,0.300855,0.540844}%
\pgfsetstrokecolor{currentstroke}%
\pgfsetdash{}{0pt}%
\pgfpathmoveto{\pgfqpoint{4.930649in}{5.406866in}}%
\pgfpathlineto{\pgfqpoint{4.905638in}{5.369056in}}%
\pgfusepath{stroke}%
\end{pgfscope}%
\begin{pgfscope}%
\pgfpathrectangle{\pgfqpoint{3.985294in}{4.155455in}}{\pgfqpoint{2.279412in}{2.004545in}}%
\pgfusepath{clip}%
\pgfsetbuttcap%
\pgfsetroundjoin%
\pgfsetlinewidth{0.808125pt}%
\definecolor{currentstroke}{rgb}{0.212395,0.359683,0.551710}%
\pgfsetstrokecolor{currentstroke}%
\pgfsetdash{}{0pt}%
\pgfpathmoveto{\pgfqpoint{4.905638in}{5.369056in}}%
\pgfpathlineto{\pgfqpoint{4.878884in}{5.331946in}}%
\pgfusepath{stroke}%
\end{pgfscope}%
\begin{pgfscope}%
\pgfpathrectangle{\pgfqpoint{3.985294in}{4.155455in}}{\pgfqpoint{2.279412in}{2.004545in}}%
\pgfusepath{clip}%
\pgfsetbuttcap%
\pgfsetroundjoin%
\pgfsetlinewidth{0.350920pt}%
\definecolor{currentstroke}{rgb}{0.276022,0.044167,0.370164}%
\pgfsetstrokecolor{currentstroke}%
\pgfsetdash{}{0pt}%
\pgfpathmoveto{\pgfqpoint{5.484043in}{5.744115in}}%
\pgfpathlineto{\pgfqpoint{5.434820in}{5.738140in}}%
\pgfusepath{stroke}%
\end{pgfscope}%
\begin{pgfscope}%
\pgfpathrectangle{\pgfqpoint{3.985294in}{4.155455in}}{\pgfqpoint{2.279412in}{2.004545in}}%
\pgfusepath{clip}%
\pgfsetbuttcap%
\pgfsetroundjoin%
\pgfsetlinewidth{0.344779pt}%
\definecolor{currentstroke}{rgb}{0.274952,0.037752,0.364543}%
\pgfsetstrokecolor{currentstroke}%
\pgfsetdash{}{0pt}%
\pgfpathmoveto{\pgfqpoint{5.434820in}{5.738140in}}%
\pgfpathlineto{\pgfqpoint{5.385706in}{5.732053in}}%
\pgfusepath{stroke}%
\end{pgfscope}%
\begin{pgfscope}%
\pgfpathrectangle{\pgfqpoint{3.985294in}{4.155455in}}{\pgfqpoint{2.279412in}{2.004545in}}%
\pgfusepath{clip}%
\pgfsetbuttcap%
\pgfsetroundjoin%
\pgfsetlinewidth{0.357717pt}%
\definecolor{currentstroke}{rgb}{0.277018,0.050344,0.375715}%
\pgfsetstrokecolor{currentstroke}%
\pgfsetdash{}{0pt}%
\pgfpathmoveto{\pgfqpoint{5.385706in}{5.732053in}}%
\pgfpathlineto{\pgfqpoint{5.336571in}{5.723646in}}%
\pgfusepath{stroke}%
\end{pgfscope}%
\begin{pgfscope}%
\pgfpathrectangle{\pgfqpoint{3.985294in}{4.155455in}}{\pgfqpoint{2.279412in}{2.004545in}}%
\pgfusepath{clip}%
\pgfsetbuttcap%
\pgfsetroundjoin%
\pgfsetlinewidth{0.377136pt}%
\definecolor{currentstroke}{rgb}{0.279566,0.067836,0.391917}%
\pgfsetstrokecolor{currentstroke}%
\pgfsetdash{}{0pt}%
\pgfpathmoveto{\pgfqpoint{5.336571in}{5.723646in}}%
\pgfpathlineto{\pgfqpoint{5.288162in}{5.712319in}}%
\pgfusepath{stroke}%
\end{pgfscope}%
\begin{pgfscope}%
\pgfpathrectangle{\pgfqpoint{3.985294in}{4.155455in}}{\pgfqpoint{2.279412in}{2.004545in}}%
\pgfusepath{clip}%
\pgfsetbuttcap%
\pgfsetroundjoin%
\pgfsetlinewidth{0.363276pt}%
\definecolor{currentstroke}{rgb}{0.277941,0.056324,0.381191}%
\pgfsetstrokecolor{currentstroke}%
\pgfsetdash{}{0pt}%
\pgfpathmoveto{\pgfqpoint{5.288162in}{5.712319in}}%
\pgfpathlineto{\pgfqpoint{5.240023in}{5.700168in}}%
\pgfusepath{stroke}%
\end{pgfscope}%
\begin{pgfscope}%
\pgfpathrectangle{\pgfqpoint{3.985294in}{4.155455in}}{\pgfqpoint{2.279412in}{2.004545in}}%
\pgfusepath{clip}%
\pgfsetbuttcap%
\pgfsetroundjoin%
\pgfsetlinewidth{0.400106pt}%
\definecolor{currentstroke}{rgb}{0.281446,0.084320,0.407414}%
\pgfsetstrokecolor{currentstroke}%
\pgfsetdash{}{0pt}%
\pgfpathmoveto{\pgfqpoint{5.240023in}{5.700168in}}%
\pgfpathlineto{\pgfqpoint{5.192316in}{5.686806in}}%
\pgfusepath{stroke}%
\end{pgfscope}%
\begin{pgfscope}%
\pgfpathrectangle{\pgfqpoint{3.985294in}{4.155455in}}{\pgfqpoint{2.279412in}{2.004545in}}%
\pgfusepath{clip}%
\pgfsetbuttcap%
\pgfsetroundjoin%
\pgfsetlinewidth{0.368507pt}%
\definecolor{currentstroke}{rgb}{0.277941,0.056324,0.381191}%
\pgfsetstrokecolor{currentstroke}%
\pgfsetdash{}{0pt}%
\pgfpathmoveto{\pgfqpoint{5.275078in}{4.602878in}}%
\pgfpathlineto{\pgfqpoint{5.227584in}{4.616446in}}%
\pgfusepath{stroke}%
\end{pgfscope}%
\begin{pgfscope}%
\pgfpathrectangle{\pgfqpoint{3.985294in}{4.155455in}}{\pgfqpoint{2.279412in}{2.004545in}}%
\pgfusepath{clip}%
\pgfsetbuttcap%
\pgfsetroundjoin%
\pgfsetlinewidth{0.382702pt}%
\definecolor{currentstroke}{rgb}{0.279566,0.067836,0.391917}%
\pgfsetstrokecolor{currentstroke}%
\pgfsetdash{}{0pt}%
\pgfpathmoveto{\pgfqpoint{5.227584in}{4.616446in}}%
\pgfpathlineto{\pgfqpoint{5.181825in}{4.634231in}}%
\pgfusepath{stroke}%
\end{pgfscope}%
\begin{pgfscope}%
\pgfpathrectangle{\pgfqpoint{3.985294in}{4.155455in}}{\pgfqpoint{2.279412in}{2.004545in}}%
\pgfusepath{clip}%
\pgfsetbuttcap%
\pgfsetroundjoin%
\pgfsetlinewidth{0.395621pt}%
\definecolor{currentstroke}{rgb}{0.280894,0.078907,0.402329}%
\pgfsetstrokecolor{currentstroke}%
\pgfsetdash{}{0pt}%
\pgfpathmoveto{\pgfqpoint{5.181825in}{4.634231in}}%
\pgfpathlineto{\pgfqpoint{5.137310in}{4.654331in}}%
\pgfusepath{stroke}%
\end{pgfscope}%
\begin{pgfscope}%
\pgfpathrectangle{\pgfqpoint{3.985294in}{4.155455in}}{\pgfqpoint{2.279412in}{2.004545in}}%
\pgfusepath{clip}%
\pgfsetbuttcap%
\pgfsetroundjoin%
\pgfsetlinewidth{0.426845pt}%
\definecolor{currentstroke}{rgb}{0.282910,0.105393,0.426902}%
\pgfsetstrokecolor{currentstroke}%
\pgfsetdash{}{0pt}%
\pgfpathmoveto{\pgfqpoint{5.137310in}{4.654331in}}%
\pgfpathlineto{\pgfqpoint{5.095748in}{4.678639in}}%
\pgfusepath{stroke}%
\end{pgfscope}%
\begin{pgfscope}%
\pgfpathrectangle{\pgfqpoint{3.985294in}{4.155455in}}{\pgfqpoint{2.279412in}{2.004545in}}%
\pgfusepath{clip}%
\pgfsetbuttcap%
\pgfsetroundjoin%
\pgfsetlinewidth{0.404975pt}%
\definecolor{currentstroke}{rgb}{0.281924,0.089666,0.412415}%
\pgfsetstrokecolor{currentstroke}%
\pgfsetdash{}{0pt}%
\pgfpathmoveto{\pgfqpoint{5.095748in}{4.678639in}}%
\pgfpathlineto{\pgfqpoint{5.056782in}{4.706345in}}%
\pgfusepath{stroke}%
\end{pgfscope}%
\begin{pgfscope}%
\pgfpathrectangle{\pgfqpoint{3.985294in}{4.155455in}}{\pgfqpoint{2.279412in}{2.004545in}}%
\pgfusepath{clip}%
\pgfsetbuttcap%
\pgfsetroundjoin%
\pgfsetlinewidth{0.360400pt}%
\definecolor{currentstroke}{rgb}{0.277018,0.050344,0.375715}%
\pgfsetstrokecolor{currentstroke}%
\pgfsetdash{}{0pt}%
\pgfpathmoveto{\pgfqpoint{5.577718in}{4.582166in}}%
\pgfpathlineto{\pgfqpoint{5.527952in}{4.587528in}}%
\pgfusepath{stroke}%
\end{pgfscope}%
\begin{pgfscope}%
\pgfpathrectangle{\pgfqpoint{3.985294in}{4.155455in}}{\pgfqpoint{2.279412in}{2.004545in}}%
\pgfusepath{clip}%
\pgfsetbuttcap%
\pgfsetroundjoin%
\pgfsetlinewidth{0.347088pt}%
\definecolor{currentstroke}{rgb}{0.274952,0.037752,0.364543}%
\pgfsetstrokecolor{currentstroke}%
\pgfsetdash{}{0pt}%
\pgfpathmoveto{\pgfqpoint{5.527952in}{4.587528in}}%
\pgfpathlineto{\pgfqpoint{5.478280in}{4.593489in}}%
\pgfusepath{stroke}%
\end{pgfscope}%
\begin{pgfscope}%
\pgfpathrectangle{\pgfqpoint{3.985294in}{4.155455in}}{\pgfqpoint{2.279412in}{2.004545in}}%
\pgfusepath{clip}%
\pgfsetbuttcap%
\pgfsetroundjoin%
\pgfsetlinewidth{0.355842pt}%
\definecolor{currentstroke}{rgb}{0.276022,0.044167,0.370164}%
\pgfsetstrokecolor{currentstroke}%
\pgfsetdash{}{0pt}%
\pgfpathmoveto{\pgfqpoint{5.478280in}{4.593489in}}%
\pgfpathlineto{\pgfqpoint{5.428738in}{4.600267in}}%
\pgfusepath{stroke}%
\end{pgfscope}%
\begin{pgfscope}%
\pgfpathrectangle{\pgfqpoint{3.985294in}{4.155455in}}{\pgfqpoint{2.279412in}{2.004545in}}%
\pgfusepath{clip}%
\pgfsetbuttcap%
\pgfsetroundjoin%
\pgfsetlinewidth{0.373865pt}%
\definecolor{currentstroke}{rgb}{0.278791,0.062145,0.386592}%
\pgfsetstrokecolor{currentstroke}%
\pgfsetdash{}{0pt}%
\pgfpathmoveto{\pgfqpoint{5.428738in}{4.600267in}}%
\pgfpathlineto{\pgfqpoint{5.379431in}{4.608252in}}%
\pgfusepath{stroke}%
\end{pgfscope}%
\begin{pgfscope}%
\pgfpathrectangle{\pgfqpoint{3.985294in}{4.155455in}}{\pgfqpoint{2.279412in}{2.004545in}}%
\pgfusepath{clip}%
\pgfsetbuttcap%
\pgfsetroundjoin%
\pgfsetlinewidth{0.378832pt}%
\definecolor{currentstroke}{rgb}{0.279566,0.067836,0.391917}%
\pgfsetstrokecolor{currentstroke}%
\pgfsetdash{}{0pt}%
\pgfpathmoveto{\pgfqpoint{5.379431in}{4.608252in}}%
\pgfpathlineto{\pgfqpoint{5.330168in}{4.616446in}}%
\pgfusepath{stroke}%
\end{pgfscope}%
\begin{pgfscope}%
\pgfpathrectangle{\pgfqpoint{3.985294in}{4.155455in}}{\pgfqpoint{2.279412in}{2.004545in}}%
\pgfusepath{clip}%
\pgfsetbuttcap%
\pgfsetroundjoin%
\pgfsetlinewidth{0.340923pt}%
\definecolor{currentstroke}{rgb}{0.273809,0.031497,0.358853}%
\pgfsetstrokecolor{currentstroke}%
\pgfsetdash{}{0pt}%
\pgfpathmoveto{\pgfqpoint{5.740503in}{4.751766in}}%
\pgfpathlineto{\pgfqpoint{5.690481in}{4.752424in}}%
\pgfusepath{stroke}%
\end{pgfscope}%
\begin{pgfscope}%
\pgfpathrectangle{\pgfqpoint{3.985294in}{4.155455in}}{\pgfqpoint{2.279412in}{2.004545in}}%
\pgfusepath{clip}%
\pgfsetbuttcap%
\pgfsetroundjoin%
\pgfsetlinewidth{0.340914pt}%
\definecolor{currentstroke}{rgb}{0.273809,0.031497,0.358853}%
\pgfsetstrokecolor{currentstroke}%
\pgfsetdash{}{0pt}%
\pgfpathmoveto{\pgfqpoint{5.690481in}{4.752424in}}%
\pgfpathlineto{\pgfqpoint{5.640508in}{4.753907in}}%
\pgfusepath{stroke}%
\end{pgfscope}%
\begin{pgfscope}%
\pgfpathrectangle{\pgfqpoint{3.985294in}{4.155455in}}{\pgfqpoint{2.279412in}{2.004545in}}%
\pgfusepath{clip}%
\pgfsetbuttcap%
\pgfsetroundjoin%
\pgfsetlinewidth{0.361167pt}%
\definecolor{currentstroke}{rgb}{0.277018,0.050344,0.375715}%
\pgfsetstrokecolor{currentstroke}%
\pgfsetdash{}{0pt}%
\pgfpathmoveto{\pgfqpoint{5.640508in}{4.753907in}}%
\pgfpathlineto{\pgfqpoint{5.590537in}{4.757584in}}%
\pgfusepath{stroke}%
\end{pgfscope}%
\begin{pgfscope}%
\pgfpathrectangle{\pgfqpoint{3.985294in}{4.155455in}}{\pgfqpoint{2.279412in}{2.004545in}}%
\pgfusepath{clip}%
\pgfsetbuttcap%
\pgfsetroundjoin%
\pgfsetlinewidth{0.382251pt}%
\definecolor{currentstroke}{rgb}{0.279566,0.067836,0.391917}%
\pgfsetstrokecolor{currentstroke}%
\pgfsetdash{}{0pt}%
\pgfpathmoveto{\pgfqpoint{5.590537in}{4.757584in}}%
\pgfpathlineto{\pgfqpoint{5.540596in}{4.761562in}}%
\pgfusepath{stroke}%
\end{pgfscope}%
\begin{pgfscope}%
\pgfpathrectangle{\pgfqpoint{3.985294in}{4.155455in}}{\pgfqpoint{2.279412in}{2.004545in}}%
\pgfusepath{clip}%
\pgfsetbuttcap%
\pgfsetroundjoin%
\pgfsetlinewidth{0.404529pt}%
\definecolor{currentstroke}{rgb}{0.281924,0.089666,0.412415}%
\pgfsetstrokecolor{currentstroke}%
\pgfsetdash{}{0pt}%
\pgfpathmoveto{\pgfqpoint{5.540596in}{4.761562in}}%
\pgfpathlineto{\pgfqpoint{5.490713in}{4.766109in}}%
\pgfusepath{stroke}%
\end{pgfscope}%
\begin{pgfscope}%
\pgfpathrectangle{\pgfqpoint{3.985294in}{4.155455in}}{\pgfqpoint{2.279412in}{2.004545in}}%
\pgfusepath{clip}%
\pgfsetbuttcap%
\pgfsetroundjoin%
\pgfsetlinewidth{0.335516pt}%
\definecolor{currentstroke}{rgb}{0.273809,0.031497,0.358853}%
\pgfsetstrokecolor{currentstroke}%
\pgfsetdash{}{0pt}%
\pgfpathmoveto{\pgfqpoint{5.627138in}{4.621141in}}%
\pgfpathlineto{\pgfqpoint{5.577149in}{4.624685in}}%
\pgfusepath{stroke}%
\end{pgfscope}%
\begin{pgfscope}%
\pgfpathrectangle{\pgfqpoint{3.985294in}{4.155455in}}{\pgfqpoint{2.279412in}{2.004545in}}%
\pgfusepath{clip}%
\pgfsetbuttcap%
\pgfsetroundjoin%
\pgfsetlinewidth{0.342960pt}%
\definecolor{currentstroke}{rgb}{0.274952,0.037752,0.364543}%
\pgfsetstrokecolor{currentstroke}%
\pgfsetdash{}{0pt}%
\pgfpathmoveto{\pgfqpoint{5.577149in}{4.624685in}}%
\pgfpathlineto{\pgfqpoint{5.527400in}{4.629921in}}%
\pgfusepath{stroke}%
\end{pgfscope}%
\begin{pgfscope}%
\pgfpathrectangle{\pgfqpoint{3.985294in}{4.155455in}}{\pgfqpoint{2.279412in}{2.004545in}}%
\pgfusepath{clip}%
\pgfsetbuttcap%
\pgfsetroundjoin%
\pgfsetlinewidth{0.353718pt}%
\definecolor{currentstroke}{rgb}{0.276022,0.044167,0.370164}%
\pgfsetstrokecolor{currentstroke}%
\pgfsetdash{}{0pt}%
\pgfpathmoveto{\pgfqpoint{5.527400in}{4.629921in}}%
\pgfpathlineto{\pgfqpoint{5.477923in}{4.636958in}}%
\pgfusepath{stroke}%
\end{pgfscope}%
\begin{pgfscope}%
\pgfpathrectangle{\pgfqpoint{3.985294in}{4.155455in}}{\pgfqpoint{2.279412in}{2.004545in}}%
\pgfusepath{clip}%
\pgfsetbuttcap%
\pgfsetroundjoin%
\pgfsetlinewidth{0.365177pt}%
\definecolor{currentstroke}{rgb}{0.277941,0.056324,0.381191}%
\pgfsetstrokecolor{currentstroke}%
\pgfsetdash{}{0pt}%
\pgfpathmoveto{\pgfqpoint{5.477923in}{4.636958in}}%
\pgfpathlineto{\pgfqpoint{5.428506in}{4.644391in}}%
\pgfusepath{stroke}%
\end{pgfscope}%
\begin{pgfscope}%
\pgfpathrectangle{\pgfqpoint{3.985294in}{4.155455in}}{\pgfqpoint{2.279412in}{2.004545in}}%
\pgfusepath{clip}%
\pgfsetbuttcap%
\pgfsetroundjoin%
\pgfsetlinewidth{0.377199pt}%
\definecolor{currentstroke}{rgb}{0.279566,0.067836,0.391917}%
\pgfsetstrokecolor{currentstroke}%
\pgfsetdash{}{0pt}%
\pgfpathmoveto{\pgfqpoint{5.428506in}{4.644391in}}%
\pgfpathlineto{\pgfqpoint{5.379156in}{4.652216in}}%
\pgfusepath{stroke}%
\end{pgfscope}%
\begin{pgfscope}%
\pgfpathrectangle{\pgfqpoint{3.985294in}{4.155455in}}{\pgfqpoint{2.279412in}{2.004545in}}%
\pgfusepath{clip}%
\pgfsetbuttcap%
\pgfsetroundjoin%
\pgfsetlinewidth{0.384966pt}%
\definecolor{currentstroke}{rgb}{0.280267,0.073417,0.397163}%
\pgfsetstrokecolor{currentstroke}%
\pgfsetdash{}{0pt}%
\pgfpathmoveto{\pgfqpoint{5.379156in}{4.652216in}}%
\pgfpathlineto{\pgfqpoint{5.330168in}{4.661553in}}%
\pgfusepath{stroke}%
\end{pgfscope}%
\begin{pgfscope}%
\pgfpathrectangle{\pgfqpoint{3.985294in}{4.155455in}}{\pgfqpoint{2.279412in}{2.004545in}}%
\pgfusepath{clip}%
\pgfsetbuttcap%
\pgfsetroundjoin%
\pgfsetlinewidth{0.403744pt}%
\definecolor{currentstroke}{rgb}{0.281446,0.084320,0.407414}%
\pgfsetstrokecolor{currentstroke}%
\pgfsetdash{}{0pt}%
\pgfpathmoveto{\pgfqpoint{5.330168in}{4.661553in}}%
\pgfpathlineto{\pgfqpoint{5.281496in}{4.672166in}}%
\pgfusepath{stroke}%
\end{pgfscope}%
\begin{pgfscope}%
\pgfpathrectangle{\pgfqpoint{3.985294in}{4.155455in}}{\pgfqpoint{2.279412in}{2.004545in}}%
\pgfusepath{clip}%
\pgfsetbuttcap%
\pgfsetroundjoin%
\pgfsetlinewidth{0.427791pt}%
\definecolor{currentstroke}{rgb}{0.282910,0.105393,0.426902}%
\pgfsetstrokecolor{currentstroke}%
\pgfsetdash{}{0pt}%
\pgfpathmoveto{\pgfqpoint{5.281496in}{4.672166in}}%
\pgfpathlineto{\pgfqpoint{5.233210in}{4.683988in}}%
\pgfusepath{stroke}%
\end{pgfscope}%
\begin{pgfscope}%
\pgfpathrectangle{\pgfqpoint{3.985294in}{4.155455in}}{\pgfqpoint{2.279412in}{2.004545in}}%
\pgfusepath{clip}%
\pgfsetbuttcap%
\pgfsetroundjoin%
\pgfsetlinewidth{0.322204pt}%
\definecolor{currentstroke}{rgb}{0.271305,0.019942,0.347269}%
\pgfsetstrokecolor{currentstroke}%
\pgfsetdash{}{0pt}%
\pgfpathmoveto{\pgfqpoint{5.823455in}{4.635851in}}%
\pgfpathlineto{\pgfqpoint{5.773360in}{4.637322in}}%
\pgfusepath{stroke}%
\end{pgfscope}%
\begin{pgfscope}%
\pgfpathrectangle{\pgfqpoint{3.985294in}{4.155455in}}{\pgfqpoint{2.279412in}{2.004545in}}%
\pgfusepath{clip}%
\pgfsetbuttcap%
\pgfsetroundjoin%
\pgfsetlinewidth{0.326500pt}%
\definecolor{currentstroke}{rgb}{0.271305,0.019942,0.347269}%
\pgfsetstrokecolor{currentstroke}%
\pgfsetdash{}{0pt}%
\pgfpathmoveto{\pgfqpoint{5.773360in}{4.637322in}}%
\pgfpathlineto{\pgfqpoint{5.723824in}{4.640806in}}%
\pgfusepath{stroke}%
\end{pgfscope}%
\begin{pgfscope}%
\pgfpathrectangle{\pgfqpoint{3.985294in}{4.155455in}}{\pgfqpoint{2.279412in}{2.004545in}}%
\pgfusepath{clip}%
\pgfsetbuttcap%
\pgfsetroundjoin%
\pgfsetlinewidth{0.316606pt}%
\definecolor{currentstroke}{rgb}{0.269944,0.014625,0.341379}%
\pgfsetstrokecolor{currentstroke}%
\pgfsetdash{}{0pt}%
\pgfpathmoveto{\pgfqpoint{5.723824in}{4.640806in}}%
\pgfpathlineto{\pgfqpoint{5.683584in}{4.644145in}}%
\pgfusepath{stroke}%
\end{pgfscope}%
\begin{pgfscope}%
\pgfpathrectangle{\pgfqpoint{3.985294in}{4.155455in}}{\pgfqpoint{2.279412in}{2.004545in}}%
\pgfusepath{clip}%
\pgfsetbuttcap%
\pgfsetroundjoin%
\pgfsetlinewidth{0.331704pt}%
\definecolor{currentstroke}{rgb}{0.272594,0.025563,0.353093}%
\pgfsetstrokecolor{currentstroke}%
\pgfsetdash{}{0pt}%
\pgfpathmoveto{\pgfqpoint{5.683584in}{4.644145in}}%
\pgfpathlineto{\pgfqpoint{5.683584in}{4.644145in}}%
\pgfusepath{stroke}%
\end{pgfscope}%
\begin{pgfscope}%
\pgfpathrectangle{\pgfqpoint{3.985294in}{4.155455in}}{\pgfqpoint{2.279412in}{2.004545in}}%
\pgfusepath{clip}%
\pgfsetbuttcap%
\pgfsetroundjoin%
\pgfsetlinewidth{0.331704pt}%
\definecolor{currentstroke}{rgb}{0.272594,0.025563,0.353093}%
\pgfsetstrokecolor{currentstroke}%
\pgfsetdash{}{0pt}%
\pgfpathmoveto{\pgfqpoint{5.683584in}{4.644145in}}%
\pgfpathlineto{\pgfqpoint{5.633625in}{4.647604in}}%
\pgfusepath{stroke}%
\end{pgfscope}%
\begin{pgfscope}%
\pgfpathrectangle{\pgfqpoint{3.985294in}{4.155455in}}{\pgfqpoint{2.279412in}{2.004545in}}%
\pgfusepath{clip}%
\pgfsetbuttcap%
\pgfsetroundjoin%
\pgfsetlinewidth{0.353609pt}%
\definecolor{currentstroke}{rgb}{0.276022,0.044167,0.370164}%
\pgfsetstrokecolor{currentstroke}%
\pgfsetdash{}{0pt}%
\pgfpathmoveto{\pgfqpoint{5.633625in}{4.647604in}}%
\pgfpathlineto{\pgfqpoint{5.583688in}{4.651542in}}%
\pgfusepath{stroke}%
\end{pgfscope}%
\begin{pgfscope}%
\pgfpathrectangle{\pgfqpoint{3.985294in}{4.155455in}}{\pgfqpoint{2.279412in}{2.004545in}}%
\pgfusepath{clip}%
\pgfsetbuttcap%
\pgfsetroundjoin%
\pgfsetlinewidth{0.355482pt}%
\definecolor{currentstroke}{rgb}{0.276022,0.044167,0.370164}%
\pgfsetstrokecolor{currentstroke}%
\pgfsetdash{}{0pt}%
\pgfpathmoveto{\pgfqpoint{5.583688in}{4.651542in}}%
\pgfpathlineto{\pgfqpoint{5.533837in}{4.656327in}}%
\pgfusepath{stroke}%
\end{pgfscope}%
\begin{pgfscope}%
\pgfpathrectangle{\pgfqpoint{3.985294in}{4.155455in}}{\pgfqpoint{2.279412in}{2.004545in}}%
\pgfusepath{clip}%
\pgfsetbuttcap%
\pgfsetroundjoin%
\pgfsetlinewidth{0.368525pt}%
\definecolor{currentstroke}{rgb}{0.277941,0.056324,0.381191}%
\pgfsetstrokecolor{currentstroke}%
\pgfsetdash{}{0pt}%
\pgfpathmoveto{\pgfqpoint{5.533837in}{4.656327in}}%
\pgfpathlineto{\pgfqpoint{5.484043in}{4.661553in}}%
\pgfusepath{stroke}%
\end{pgfscope}%
\begin{pgfscope}%
\pgfpathrectangle{\pgfqpoint{3.985294in}{4.155455in}}{\pgfqpoint{2.279412in}{2.004545in}}%
\pgfusepath{clip}%
\pgfsetbuttcap%
\pgfsetroundjoin%
\pgfsetlinewidth{0.440873pt}%
\definecolor{currentstroke}{rgb}{0.283197,0.115680,0.436115}%
\pgfsetstrokecolor{currentstroke}%
\pgfsetdash{}{0pt}%
\pgfpathmoveto{\pgfqpoint{5.176292in}{5.653902in}}%
\pgfpathlineto{\pgfqpoint{5.130437in}{5.636131in}}%
\pgfusepath{stroke}%
\end{pgfscope}%
\begin{pgfscope}%
\pgfpathrectangle{\pgfqpoint{3.985294in}{4.155455in}}{\pgfqpoint{2.279412in}{2.004545in}}%
\pgfusepath{clip}%
\pgfsetbuttcap%
\pgfsetroundjoin%
\pgfsetlinewidth{0.433881pt}%
\definecolor{currentstroke}{rgb}{0.283091,0.110553,0.431554}%
\pgfsetstrokecolor{currentstroke}%
\pgfsetdash{}{0pt}%
\pgfpathmoveto{\pgfqpoint{5.130437in}{5.636131in}}%
\pgfpathlineto{\pgfqpoint{5.086501in}{5.615132in}}%
\pgfusepath{stroke}%
\end{pgfscope}%
\begin{pgfscope}%
\pgfpathrectangle{\pgfqpoint{3.985294in}{4.155455in}}{\pgfqpoint{2.279412in}{2.004545in}}%
\pgfusepath{clip}%
\pgfsetbuttcap%
\pgfsetroundjoin%
\pgfsetlinewidth{0.397800pt}%
\definecolor{currentstroke}{rgb}{0.281446,0.084320,0.407414}%
\pgfsetstrokecolor{currentstroke}%
\pgfsetdash{}{0pt}%
\pgfpathmoveto{\pgfqpoint{5.086501in}{5.615132in}}%
\pgfpathlineto{\pgfqpoint{5.046347in}{5.589189in}}%
\pgfusepath{stroke}%
\end{pgfscope}%
\begin{pgfscope}%
\pgfpathrectangle{\pgfqpoint{3.985294in}{4.155455in}}{\pgfqpoint{2.279412in}{2.004545in}}%
\pgfusepath{clip}%
\pgfsetbuttcap%
\pgfsetroundjoin%
\pgfsetlinewidth{0.478409pt}%
\definecolor{currentstroke}{rgb}{0.282623,0.140926,0.457517}%
\pgfsetstrokecolor{currentstroke}%
\pgfsetdash{}{0pt}%
\pgfpathmoveto{\pgfqpoint{5.046347in}{5.589189in}}%
\pgfpathlineto{\pgfqpoint{5.010100in}{5.559938in}}%
\pgfusepath{stroke}%
\end{pgfscope}%
\begin{pgfscope}%
\pgfpathrectangle{\pgfqpoint{3.985294in}{4.155455in}}{\pgfqpoint{2.279412in}{2.004545in}}%
\pgfusepath{clip}%
\pgfsetbuttcap%
\pgfsetroundjoin%
\pgfsetlinewidth{0.446839pt}%
\definecolor{currentstroke}{rgb}{0.283229,0.120777,0.440584}%
\pgfsetstrokecolor{currentstroke}%
\pgfsetdash{}{0pt}%
\pgfpathmoveto{\pgfqpoint{5.010100in}{5.559938in}}%
\pgfpathlineto{\pgfqpoint{5.010100in}{5.559938in}}%
\pgfusepath{stroke}%
\end{pgfscope}%
\begin{pgfscope}%
\pgfpathrectangle{\pgfqpoint{3.985294in}{4.155455in}}{\pgfqpoint{2.279412in}{2.004545in}}%
\pgfusepath{clip}%
\pgfsetbuttcap%
\pgfsetroundjoin%
\pgfsetlinewidth{0.446839pt}%
\definecolor{currentstroke}{rgb}{0.283229,0.120777,0.440584}%
\pgfsetstrokecolor{currentstroke}%
\pgfsetdash{}{0pt}%
\pgfpathmoveto{\pgfqpoint{5.010100in}{5.559938in}}%
\pgfpathlineto{\pgfqpoint{4.994188in}{5.540465in}}%
\pgfusepath{stroke}%
\end{pgfscope}%
\begin{pgfscope}%
\pgfpathrectangle{\pgfqpoint{3.985294in}{4.155455in}}{\pgfqpoint{2.279412in}{2.004545in}}%
\pgfusepath{clip}%
\pgfsetbuttcap%
\pgfsetroundjoin%
\pgfsetlinewidth{0.526062pt}%
\definecolor{currentstroke}{rgb}{0.278826,0.175490,0.483397}%
\pgfsetstrokecolor{currentstroke}%
\pgfsetdash{}{0pt}%
\pgfpathmoveto{\pgfqpoint{4.994188in}{5.540465in}}%
\pgfpathlineto{\pgfqpoint{4.983554in}{5.519677in}}%
\pgfusepath{stroke}%
\end{pgfscope}%
\begin{pgfscope}%
\pgfpathrectangle{\pgfqpoint{3.985294in}{4.155455in}}{\pgfqpoint{2.279412in}{2.004545in}}%
\pgfusepath{clip}%
\pgfsetbuttcap%
\pgfsetroundjoin%
\pgfsetlinewidth{0.407421pt}%
\definecolor{currentstroke}{rgb}{0.281924,0.089666,0.412415}%
\pgfsetstrokecolor{currentstroke}%
\pgfsetdash{}{0pt}%
\pgfpathmoveto{\pgfqpoint{5.324671in}{4.684625in}}%
\pgfpathlineto{\pgfqpoint{5.276010in}{4.695264in}}%
\pgfusepath{stroke}%
\end{pgfscope}%
\begin{pgfscope}%
\pgfpathrectangle{\pgfqpoint{3.985294in}{4.155455in}}{\pgfqpoint{2.279412in}{2.004545in}}%
\pgfusepath{clip}%
\pgfsetbuttcap%
\pgfsetroundjoin%
\pgfsetlinewidth{0.438419pt}%
\definecolor{currentstroke}{rgb}{0.283197,0.115680,0.436115}%
\pgfsetstrokecolor{currentstroke}%
\pgfsetdash{}{0pt}%
\pgfpathmoveto{\pgfqpoint{5.276010in}{4.695264in}}%
\pgfpathlineto{\pgfqpoint{5.227584in}{4.706659in}}%
\pgfusepath{stroke}%
\end{pgfscope}%
\begin{pgfscope}%
\pgfpathrectangle{\pgfqpoint{3.985294in}{4.155455in}}{\pgfqpoint{2.279412in}{2.004545in}}%
\pgfusepath{clip}%
\pgfsetbuttcap%
\pgfsetroundjoin%
\pgfsetlinewidth{0.457470pt}%
\definecolor{currentstroke}{rgb}{0.283187,0.125848,0.444960}%
\pgfsetstrokecolor{currentstroke}%
\pgfsetdash{}{0pt}%
\pgfpathmoveto{\pgfqpoint{5.227584in}{4.706659in}}%
\pgfpathlineto{\pgfqpoint{5.179875in}{4.720170in}}%
\pgfusepath{stroke}%
\end{pgfscope}%
\begin{pgfscope}%
\pgfpathrectangle{\pgfqpoint{3.985294in}{4.155455in}}{\pgfqpoint{2.279412in}{2.004545in}}%
\pgfusepath{clip}%
\pgfsetbuttcap%
\pgfsetroundjoin%
\pgfsetlinewidth{0.473288pt}%
\definecolor{currentstroke}{rgb}{0.282623,0.140926,0.457517}%
\pgfsetstrokecolor{currentstroke}%
\pgfsetdash{}{0pt}%
\pgfpathmoveto{\pgfqpoint{5.179875in}{4.720170in}}%
\pgfpathlineto{\pgfqpoint{5.133239in}{4.736253in}}%
\pgfusepath{stroke}%
\end{pgfscope}%
\begin{pgfscope}%
\pgfpathrectangle{\pgfqpoint{3.985294in}{4.155455in}}{\pgfqpoint{2.279412in}{2.004545in}}%
\pgfusepath{clip}%
\pgfsetbuttcap%
\pgfsetroundjoin%
\pgfsetlinewidth{0.463304pt}%
\definecolor{currentstroke}{rgb}{0.283072,0.130895,0.449241}%
\pgfsetstrokecolor{currentstroke}%
\pgfsetdash{}{0pt}%
\pgfpathmoveto{\pgfqpoint{5.133239in}{4.736253in}}%
\pgfpathlineto{\pgfqpoint{5.088608in}{4.756085in}}%
\pgfusepath{stroke}%
\end{pgfscope}%
\begin{pgfscope}%
\pgfpathrectangle{\pgfqpoint{3.985294in}{4.155455in}}{\pgfqpoint{2.279412in}{2.004545in}}%
\pgfusepath{clip}%
\pgfsetbuttcap%
\pgfsetroundjoin%
\pgfsetlinewidth{0.496788pt}%
\definecolor{currentstroke}{rgb}{0.281412,0.155834,0.469201}%
\pgfsetstrokecolor{currentstroke}%
\pgfsetdash{}{0pt}%
\pgfpathmoveto{\pgfqpoint{5.088608in}{4.756085in}}%
\pgfpathlineto{\pgfqpoint{5.048002in}{4.781332in}}%
\pgfusepath{stroke}%
\end{pgfscope}%
\begin{pgfscope}%
\pgfpathrectangle{\pgfqpoint{3.985294in}{4.155455in}}{\pgfqpoint{2.279412in}{2.004545in}}%
\pgfusepath{clip}%
\pgfsetbuttcap%
\pgfsetroundjoin%
\pgfsetlinewidth{0.518354pt}%
\definecolor{currentstroke}{rgb}{0.279574,0.170599,0.479997}%
\pgfsetstrokecolor{currentstroke}%
\pgfsetdash{}{0pt}%
\pgfpathmoveto{\pgfqpoint{5.048002in}{4.781332in}}%
\pgfpathlineto{\pgfqpoint{5.011877in}{4.811372in}}%
\pgfusepath{stroke}%
\end{pgfscope}%
\begin{pgfscope}%
\pgfpathrectangle{\pgfqpoint{3.985294in}{4.155455in}}{\pgfqpoint{2.279412in}{2.004545in}}%
\pgfusepath{clip}%
\pgfsetbuttcap%
\pgfsetroundjoin%
\pgfsetlinewidth{0.601501pt}%
\definecolor{currentstroke}{rgb}{0.266580,0.228262,0.514349}%
\pgfsetstrokecolor{currentstroke}%
\pgfsetdash{}{0pt}%
\pgfpathmoveto{\pgfqpoint{5.011877in}{4.811372in}}%
\pgfpathlineto{\pgfqpoint{4.982813in}{4.845481in}}%
\pgfusepath{stroke}%
\end{pgfscope}%
\begin{pgfscope}%
\pgfpathrectangle{\pgfqpoint{3.985294in}{4.155455in}}{\pgfqpoint{2.279412in}{2.004545in}}%
\pgfusepath{clip}%
\pgfsetbuttcap%
\pgfsetroundjoin%
\pgfsetlinewidth{0.350167pt}%
\definecolor{currentstroke}{rgb}{0.276022,0.044167,0.370164}%
\pgfsetstrokecolor{currentstroke}%
\pgfsetdash{}{0pt}%
\pgfpathmoveto{\pgfqpoint{5.637919in}{5.608795in}}%
\pgfpathlineto{\pgfqpoint{5.587947in}{5.605142in}}%
\pgfusepath{stroke}%
\end{pgfscope}%
\begin{pgfscope}%
\pgfpathrectangle{\pgfqpoint{3.985294in}{4.155455in}}{\pgfqpoint{2.279412in}{2.004545in}}%
\pgfusepath{clip}%
\pgfsetbuttcap%
\pgfsetroundjoin%
\pgfsetlinewidth{0.361011pt}%
\definecolor{currentstroke}{rgb}{0.277018,0.050344,0.375715}%
\pgfsetstrokecolor{currentstroke}%
\pgfsetdash{}{0pt}%
\pgfpathmoveto{\pgfqpoint{5.587947in}{5.605142in}}%
\pgfpathlineto{\pgfqpoint{5.538055in}{5.600720in}}%
\pgfusepath{stroke}%
\end{pgfscope}%
\begin{pgfscope}%
\pgfpathrectangle{\pgfqpoint{3.985294in}{4.155455in}}{\pgfqpoint{2.279412in}{2.004545in}}%
\pgfusepath{clip}%
\pgfsetbuttcap%
\pgfsetroundjoin%
\pgfsetlinewidth{0.377966pt}%
\definecolor{currentstroke}{rgb}{0.279566,0.067836,0.391917}%
\pgfsetstrokecolor{currentstroke}%
\pgfsetdash{}{0pt}%
\pgfpathmoveto{\pgfqpoint{5.538055in}{5.600720in}}%
\pgfpathlineto{\pgfqpoint{5.488199in}{5.595955in}}%
\pgfusepath{stroke}%
\end{pgfscope}%
\begin{pgfscope}%
\pgfpathrectangle{\pgfqpoint{3.985294in}{4.155455in}}{\pgfqpoint{2.279412in}{2.004545in}}%
\pgfusepath{clip}%
\pgfsetbuttcap%
\pgfsetroundjoin%
\pgfsetlinewidth{0.403108pt}%
\definecolor{currentstroke}{rgb}{0.281446,0.084320,0.407414}%
\pgfsetstrokecolor{currentstroke}%
\pgfsetdash{}{0pt}%
\pgfpathmoveto{\pgfqpoint{5.488199in}{5.595955in}}%
\pgfpathlineto{\pgfqpoint{5.438423in}{5.590624in}}%
\pgfusepath{stroke}%
\end{pgfscope}%
\begin{pgfscope}%
\pgfpathrectangle{\pgfqpoint{3.985294in}{4.155455in}}{\pgfqpoint{2.279412in}{2.004545in}}%
\pgfusepath{clip}%
\pgfsetbuttcap%
\pgfsetroundjoin%
\pgfsetlinewidth{0.426416pt}%
\definecolor{currentstroke}{rgb}{0.282910,0.105393,0.426902}%
\pgfsetstrokecolor{currentstroke}%
\pgfsetdash{}{0pt}%
\pgfpathmoveto{\pgfqpoint{5.438423in}{5.590624in}}%
\pgfpathlineto{\pgfqpoint{5.388787in}{5.584325in}}%
\pgfusepath{stroke}%
\end{pgfscope}%
\begin{pgfscope}%
\pgfpathrectangle{\pgfqpoint{3.985294in}{4.155455in}}{\pgfqpoint{2.279412in}{2.004545in}}%
\pgfusepath{clip}%
\pgfsetbuttcap%
\pgfsetroundjoin%
\pgfsetlinewidth{0.439890pt}%
\definecolor{currentstroke}{rgb}{0.283197,0.115680,0.436115}%
\pgfsetstrokecolor{currentstroke}%
\pgfsetdash{}{0pt}%
\pgfpathmoveto{\pgfqpoint{5.388787in}{5.584325in}}%
\pgfpathlineto{\pgfqpoint{5.339370in}{5.576853in}}%
\pgfusepath{stroke}%
\end{pgfscope}%
\begin{pgfscope}%
\pgfpathrectangle{\pgfqpoint{3.985294in}{4.155455in}}{\pgfqpoint{2.279412in}{2.004545in}}%
\pgfusepath{clip}%
\pgfsetbuttcap%
\pgfsetroundjoin%
\pgfsetlinewidth{0.448162pt}%
\definecolor{currentstroke}{rgb}{0.283229,0.120777,0.440584}%
\pgfsetstrokecolor{currentstroke}%
\pgfsetdash{}{0pt}%
\pgfpathmoveto{\pgfqpoint{5.339370in}{5.576853in}}%
\pgfpathlineto{\pgfqpoint{5.290172in}{5.568307in}}%
\pgfusepath{stroke}%
\end{pgfscope}%
\begin{pgfscope}%
\pgfpathrectangle{\pgfqpoint{3.985294in}{4.155455in}}{\pgfqpoint{2.279412in}{2.004545in}}%
\pgfusepath{clip}%
\pgfsetbuttcap%
\pgfsetroundjoin%
\pgfsetlinewidth{0.465788pt}%
\definecolor{currentstroke}{rgb}{0.283072,0.130895,0.449241}%
\pgfsetstrokecolor{currentstroke}%
\pgfsetdash{}{0pt}%
\pgfpathmoveto{\pgfqpoint{5.290172in}{5.568307in}}%
\pgfpathlineto{\pgfqpoint{5.241316in}{5.558423in}}%
\pgfusepath{stroke}%
\end{pgfscope}%
\begin{pgfscope}%
\pgfpathrectangle{\pgfqpoint{3.985294in}{4.155455in}}{\pgfqpoint{2.279412in}{2.004545in}}%
\pgfusepath{clip}%
\pgfsetbuttcap%
\pgfsetroundjoin%
\pgfsetlinewidth{0.503949pt}%
\definecolor{currentstroke}{rgb}{0.280868,0.160771,0.472899}%
\pgfsetstrokecolor{currentstroke}%
\pgfsetdash{}{0pt}%
\pgfpathmoveto{\pgfqpoint{5.241316in}{5.558423in}}%
\pgfpathlineto{\pgfqpoint{5.193001in}{5.546628in}}%
\pgfusepath{stroke}%
\end{pgfscope}%
\begin{pgfscope}%
\pgfpathrectangle{\pgfqpoint{3.985294in}{4.155455in}}{\pgfqpoint{2.279412in}{2.004545in}}%
\pgfusepath{clip}%
\pgfsetbuttcap%
\pgfsetroundjoin%
\pgfsetlinewidth{0.521465pt}%
\definecolor{currentstroke}{rgb}{0.278826,0.175490,0.483397}%
\pgfsetstrokecolor{currentstroke}%
\pgfsetdash{}{0pt}%
\pgfpathmoveto{\pgfqpoint{5.193001in}{5.546628in}}%
\pgfpathlineto{\pgfqpoint{5.145507in}{5.532549in}}%
\pgfusepath{stroke}%
\end{pgfscope}%
\begin{pgfscope}%
\pgfpathrectangle{\pgfqpoint{3.985294in}{4.155455in}}{\pgfqpoint{2.279412in}{2.004545in}}%
\pgfusepath{clip}%
\pgfsetbuttcap%
\pgfsetroundjoin%
\pgfsetlinewidth{0.546342pt}%
\definecolor{currentstroke}{rgb}{0.276194,0.190074,0.493001}%
\pgfsetstrokecolor{currentstroke}%
\pgfsetdash{}{0pt}%
\pgfpathmoveto{\pgfqpoint{5.145507in}{5.532549in}}%
\pgfpathlineto{\pgfqpoint{5.099283in}{5.515536in}}%
\pgfusepath{stroke}%
\end{pgfscope}%
\begin{pgfscope}%
\pgfpathrectangle{\pgfqpoint{3.985294in}{4.155455in}}{\pgfqpoint{2.279412in}{2.004545in}}%
\pgfusepath{clip}%
\pgfsetbuttcap%
\pgfsetroundjoin%
\pgfsetlinewidth{0.000000pt}%
\definecolor{currentstroke}{rgb}{0.276194,0.190074,0.493001}%
\pgfsetstrokecolor{currentstroke}%
\pgfsetdash{}{0pt}%
\pgfpathmoveto{\pgfqpoint{4.828028in}{4.817346in}}%
\pgfpathlineto{\pgfqpoint{4.868541in}{4.841980in}}%
\pgfusepath{}%
\end{pgfscope}%
\begin{pgfscope}%
\pgfpathrectangle{\pgfqpoint{3.985294in}{4.155455in}}{\pgfqpoint{2.279412in}{2.004545in}}%
\pgfusepath{clip}%
\pgfsetbuttcap%
\pgfsetroundjoin%
\pgfsetlinewidth{0.634477pt}%
\definecolor{currentstroke}{rgb}{0.258965,0.251537,0.524736}%
\pgfsetstrokecolor{currentstroke}%
\pgfsetdash{}{0pt}%
\pgfpathmoveto{\pgfqpoint{4.868541in}{4.841980in}}%
\pgfpathlineto{\pgfqpoint{4.868541in}{4.841980in}}%
\pgfusepath{stroke}%
\end{pgfscope}%
\begin{pgfscope}%
\pgfpathrectangle{\pgfqpoint{3.985294in}{4.155455in}}{\pgfqpoint{2.279412in}{2.004545in}}%
\pgfusepath{clip}%
\pgfsetbuttcap%
\pgfsetroundjoin%
\pgfsetlinewidth{0.634477pt}%
\definecolor{currentstroke}{rgb}{0.258965,0.251537,0.524736}%
\pgfsetstrokecolor{currentstroke}%
\pgfsetdash{}{0pt}%
\pgfpathmoveto{\pgfqpoint{4.868541in}{4.841980in}}%
\pgfpathlineto{\pgfqpoint{4.883376in}{4.858290in}}%
\pgfusepath{stroke}%
\end{pgfscope}%
\begin{pgfscope}%
\pgfpathrectangle{\pgfqpoint{3.985294in}{4.155455in}}{\pgfqpoint{2.279412in}{2.004545in}}%
\pgfusepath{clip}%
\pgfsetbuttcap%
\pgfsetroundjoin%
\pgfsetlinewidth{0.574842pt}%
\definecolor{currentstroke}{rgb}{0.271828,0.209303,0.504434}%
\pgfsetstrokecolor{currentstroke}%
\pgfsetdash{}{0pt}%
\pgfpathmoveto{\pgfqpoint{4.883376in}{4.858290in}}%
\pgfpathlineto{\pgfqpoint{4.883376in}{4.858290in}}%
\pgfusepath{stroke}%
\end{pgfscope}%
\begin{pgfscope}%
\pgfpathrectangle{\pgfqpoint{3.985294in}{4.155455in}}{\pgfqpoint{2.279412in}{2.004545in}}%
\pgfusepath{clip}%
\pgfsetbuttcap%
\pgfsetroundjoin%
\pgfsetlinewidth{0.574842pt}%
\definecolor{currentstroke}{rgb}{0.271828,0.209303,0.504434}%
\pgfsetstrokecolor{currentstroke}%
\pgfsetdash{}{0pt}%
\pgfpathmoveto{\pgfqpoint{4.883376in}{4.858290in}}%
\pgfpathlineto{\pgfqpoint{4.888982in}{4.874060in}}%
\pgfusepath{stroke}%
\end{pgfscope}%
\begin{pgfscope}%
\pgfpathrectangle{\pgfqpoint{3.985294in}{4.155455in}}{\pgfqpoint{2.279412in}{2.004545in}}%
\pgfusepath{clip}%
\pgfsetbuttcap%
\pgfsetroundjoin%
\pgfsetlinewidth{0.583899pt}%
\definecolor{currentstroke}{rgb}{0.269308,0.218818,0.509577}%
\pgfsetstrokecolor{currentstroke}%
\pgfsetdash{}{0pt}%
\pgfpathmoveto{\pgfqpoint{4.888982in}{4.874060in}}%
\pgfpathlineto{\pgfqpoint{4.889760in}{4.889934in}}%
\pgfusepath{stroke}%
\end{pgfscope}%
\begin{pgfscope}%
\pgfpathrectangle{\pgfqpoint{3.985294in}{4.155455in}}{\pgfqpoint{2.279412in}{2.004545in}}%
\pgfusepath{clip}%
\pgfsetbuttcap%
\pgfsetroundjoin%
\pgfsetlinewidth{0.643711pt}%
\definecolor{currentstroke}{rgb}{0.257322,0.256130,0.526563}%
\pgfsetstrokecolor{currentstroke}%
\pgfsetdash{}{0pt}%
\pgfpathmoveto{\pgfqpoint{4.889760in}{4.889934in}}%
\pgfpathlineto{\pgfqpoint{4.890756in}{4.917333in}}%
\pgfusepath{stroke}%
\end{pgfscope}%
\begin{pgfscope}%
\pgfpathrectangle{\pgfqpoint{3.985294in}{4.155455in}}{\pgfqpoint{2.279412in}{2.004545in}}%
\pgfusepath{clip}%
\pgfsetbuttcap%
\pgfsetroundjoin%
\pgfsetlinewidth{0.640524pt}%
\definecolor{currentstroke}{rgb}{0.257322,0.256130,0.526563}%
\pgfsetstrokecolor{currentstroke}%
\pgfsetdash{}{0pt}%
\pgfpathmoveto{\pgfqpoint{4.890756in}{4.917333in}}%
\pgfpathlineto{\pgfqpoint{4.889561in}{4.960071in}}%
\pgfusepath{stroke}%
\end{pgfscope}%
\begin{pgfscope}%
\pgfpathrectangle{\pgfqpoint{3.985294in}{4.155455in}}{\pgfqpoint{2.279412in}{2.004545in}}%
\pgfusepath{clip}%
\pgfsetbuttcap%
\pgfsetroundjoin%
\pgfsetlinewidth{0.694629pt}%
\definecolor{currentstroke}{rgb}{0.243113,0.292092,0.538516}%
\pgfsetstrokecolor{currentstroke}%
\pgfsetdash{}{0pt}%
\pgfpathmoveto{\pgfqpoint{4.889561in}{4.960071in}}%
\pgfpathlineto{\pgfqpoint{4.889561in}{4.960071in}}%
\pgfusepath{stroke}%
\end{pgfscope}%
\begin{pgfscope}%
\pgfpathrectangle{\pgfqpoint{3.985294in}{4.155455in}}{\pgfqpoint{2.279412in}{2.004545in}}%
\pgfusepath{clip}%
\pgfsetbuttcap%
\pgfsetroundjoin%
\pgfsetlinewidth{0.694629pt}%
\definecolor{currentstroke}{rgb}{0.243113,0.292092,0.538516}%
\pgfsetstrokecolor{currentstroke}%
\pgfsetdash{}{0pt}%
\pgfpathmoveto{\pgfqpoint{4.889561in}{4.960071in}}%
\pgfpathlineto{\pgfqpoint{4.883523in}{4.980422in}}%
\pgfusepath{stroke}%
\end{pgfscope}%
\begin{pgfscope}%
\pgfpathrectangle{\pgfqpoint{3.985294in}{4.155455in}}{\pgfqpoint{2.279412in}{2.004545in}}%
\pgfusepath{clip}%
\pgfsetbuttcap%
\pgfsetroundjoin%
\pgfsetlinewidth{0.939730pt}%
\definecolor{currentstroke}{rgb}{0.177423,0.437527,0.557565}%
\pgfsetstrokecolor{currentstroke}%
\pgfsetdash{}{0pt}%
\pgfpathmoveto{\pgfqpoint{4.883523in}{4.980422in}}%
\pgfpathlineto{\pgfqpoint{4.872063in}{4.999380in}}%
\pgfusepath{stroke}%
\end{pgfscope}%
\begin{pgfscope}%
\pgfpathrectangle{\pgfqpoint{3.985294in}{4.155455in}}{\pgfqpoint{2.279412in}{2.004545in}}%
\pgfusepath{clip}%
\pgfsetbuttcap%
\pgfsetroundjoin%
\pgfsetlinewidth{0.703759pt}%
\definecolor{currentstroke}{rgb}{0.241237,0.296485,0.539709}%
\pgfsetstrokecolor{currentstroke}%
\pgfsetdash{}{0pt}%
\pgfpathmoveto{\pgfqpoint{5.330168in}{4.932193in}}%
\pgfpathlineto{\pgfqpoint{5.280313in}{4.936955in}}%
\pgfusepath{stroke}%
\end{pgfscope}%
\begin{pgfscope}%
\pgfpathrectangle{\pgfqpoint{3.985294in}{4.155455in}}{\pgfqpoint{2.279412in}{2.004545in}}%
\pgfusepath{clip}%
\pgfsetbuttcap%
\pgfsetroundjoin%
\pgfsetlinewidth{0.792613pt}%
\definecolor{currentstroke}{rgb}{0.216210,0.351535,0.550627}%
\pgfsetstrokecolor{currentstroke}%
\pgfsetdash{}{0pt}%
\pgfpathmoveto{\pgfqpoint{5.280313in}{4.936955in}}%
\pgfpathlineto{\pgfqpoint{5.230594in}{4.942713in}}%
\pgfusepath{stroke}%
\end{pgfscope}%
\begin{pgfscope}%
\pgfpathrectangle{\pgfqpoint{3.985294in}{4.155455in}}{\pgfqpoint{2.279412in}{2.004545in}}%
\pgfusepath{clip}%
\pgfsetbuttcap%
\pgfsetroundjoin%
\pgfsetlinewidth{0.805336pt}%
\definecolor{currentstroke}{rgb}{0.212395,0.359683,0.551710}%
\pgfsetstrokecolor{currentstroke}%
\pgfsetdash{}{0pt}%
\pgfpathmoveto{\pgfqpoint{5.230594in}{4.942713in}}%
\pgfpathlineto{\pgfqpoint{5.181062in}{4.949577in}}%
\pgfusepath{stroke}%
\end{pgfscope}%
\begin{pgfscope}%
\pgfpathrectangle{\pgfqpoint{3.985294in}{4.155455in}}{\pgfqpoint{2.279412in}{2.004545in}}%
\pgfusepath{clip}%
\pgfsetbuttcap%
\pgfsetroundjoin%
\pgfsetlinewidth{0.858018pt}%
\definecolor{currentstroke}{rgb}{0.197636,0.391528,0.554969}%
\pgfsetstrokecolor{currentstroke}%
\pgfsetdash{}{0pt}%
\pgfpathmoveto{\pgfqpoint{5.181062in}{4.949577in}}%
\pgfpathlineto{\pgfqpoint{5.131779in}{4.957704in}}%
\pgfusepath{stroke}%
\end{pgfscope}%
\begin{pgfscope}%
\pgfpathrectangle{\pgfqpoint{3.985294in}{4.155455in}}{\pgfqpoint{2.279412in}{2.004545in}}%
\pgfusepath{clip}%
\pgfsetbuttcap%
\pgfsetroundjoin%
\pgfsetlinewidth{0.833414pt}%
\definecolor{currentstroke}{rgb}{0.204903,0.375746,0.553533}%
\pgfsetstrokecolor{currentstroke}%
\pgfsetdash{}{0pt}%
\pgfpathmoveto{\pgfqpoint{5.131779in}{4.957704in}}%
\pgfpathlineto{\pgfqpoint{5.082923in}{4.967582in}}%
\pgfusepath{stroke}%
\end{pgfscope}%
\begin{pgfscope}%
\pgfpathrectangle{\pgfqpoint{3.985294in}{4.155455in}}{\pgfqpoint{2.279412in}{2.004545in}}%
\pgfusepath{clip}%
\pgfsetbuttcap%
\pgfsetroundjoin%
\pgfsetlinewidth{0.947737pt}%
\definecolor{currentstroke}{rgb}{0.175841,0.441290,0.557685}%
\pgfsetstrokecolor{currentstroke}%
\pgfsetdash{}{0pt}%
\pgfpathmoveto{\pgfqpoint{5.082923in}{4.967582in}}%
\pgfpathlineto{\pgfqpoint{5.034783in}{4.979860in}}%
\pgfusepath{stroke}%
\end{pgfscope}%
\begin{pgfscope}%
\pgfpathrectangle{\pgfqpoint{3.985294in}{4.155455in}}{\pgfqpoint{2.279412in}{2.004545in}}%
\pgfusepath{clip}%
\pgfsetbuttcap%
\pgfsetroundjoin%
\pgfsetlinewidth{0.864446pt}%
\definecolor{currentstroke}{rgb}{0.195860,0.395433,0.555276}%
\pgfsetstrokecolor{currentstroke}%
\pgfsetdash{}{0pt}%
\pgfpathmoveto{\pgfqpoint{5.255262in}{4.980680in}}%
\pgfpathlineto{\pgfqpoint{5.205503in}{4.986163in}}%
\pgfusepath{stroke}%
\end{pgfscope}%
\begin{pgfscope}%
\pgfpathrectangle{\pgfqpoint{3.985294in}{4.155455in}}{\pgfqpoint{2.279412in}{2.004545in}}%
\pgfusepath{clip}%
\pgfsetbuttcap%
\pgfsetroundjoin%
\pgfsetlinewidth{0.955237pt}%
\definecolor{currentstroke}{rgb}{0.174274,0.445044,0.557792}%
\pgfsetstrokecolor{currentstroke}%
\pgfsetdash{}{0pt}%
\pgfpathmoveto{\pgfqpoint{5.205503in}{4.986163in}}%
\pgfpathlineto{\pgfqpoint{5.155888in}{4.992569in}}%
\pgfusepath{stroke}%
\end{pgfscope}%
\begin{pgfscope}%
\pgfpathrectangle{\pgfqpoint{3.985294in}{4.155455in}}{\pgfqpoint{2.279412in}{2.004545in}}%
\pgfusepath{clip}%
\pgfsetbuttcap%
\pgfsetroundjoin%
\pgfsetlinewidth{1.029814pt}%
\definecolor{currentstroke}{rgb}{0.157729,0.485932,0.558013}%
\pgfsetstrokecolor{currentstroke}%
\pgfsetdash{}{0pt}%
\pgfpathmoveto{\pgfqpoint{5.155888in}{4.992569in}}%
\pgfpathlineto{\pgfqpoint{5.106534in}{5.000342in}}%
\pgfusepath{stroke}%
\end{pgfscope}%
\begin{pgfscope}%
\pgfpathrectangle{\pgfqpoint{3.985294in}{4.155455in}}{\pgfqpoint{2.279412in}{2.004545in}}%
\pgfusepath{clip}%
\pgfsetbuttcap%
\pgfsetroundjoin%
\pgfsetlinewidth{1.012791pt}%
\definecolor{currentstroke}{rgb}{0.162142,0.474838,0.558140}%
\pgfsetstrokecolor{currentstroke}%
\pgfsetdash{}{0pt}%
\pgfpathmoveto{\pgfqpoint{5.106534in}{5.000342in}}%
\pgfpathlineto{\pgfqpoint{5.057597in}{5.009929in}}%
\pgfusepath{stroke}%
\end{pgfscope}%
\begin{pgfscope}%
\pgfpathrectangle{\pgfqpoint{3.985294in}{4.155455in}}{\pgfqpoint{2.279412in}{2.004545in}}%
\pgfusepath{clip}%
\pgfsetbuttcap%
\pgfsetroundjoin%
\pgfsetlinewidth{1.026264pt}%
\definecolor{currentstroke}{rgb}{0.159194,0.482237,0.558073}%
\pgfsetstrokecolor{currentstroke}%
\pgfsetdash{}{0pt}%
\pgfpathmoveto{\pgfqpoint{5.057597in}{5.009929in}}%
\pgfpathlineto{\pgfqpoint{5.009155in}{5.021284in}}%
\pgfusepath{stroke}%
\end{pgfscope}%
\begin{pgfscope}%
\pgfpathrectangle{\pgfqpoint{3.985294in}{4.155455in}}{\pgfqpoint{2.279412in}{2.004545in}}%
\pgfusepath{clip}%
\pgfsetbuttcap%
\pgfsetroundjoin%
\pgfsetlinewidth{1.169510pt}%
\definecolor{currentstroke}{rgb}{0.129933,0.559582,0.551864}%
\pgfsetstrokecolor{currentstroke}%
\pgfsetdash{}{0pt}%
\pgfpathmoveto{\pgfqpoint{5.009155in}{5.021284in}}%
\pgfpathlineto{\pgfqpoint{4.961410in}{5.034702in}}%
\pgfusepath{stroke}%
\end{pgfscope}%
\begin{pgfscope}%
\pgfpathrectangle{\pgfqpoint{3.985294in}{4.155455in}}{\pgfqpoint{2.279412in}{2.004545in}}%
\pgfusepath{clip}%
\pgfsetbuttcap%
\pgfsetroundjoin%
\pgfsetlinewidth{1.095416pt}%
\definecolor{currentstroke}{rgb}{0.144759,0.519093,0.556572}%
\pgfsetstrokecolor{currentstroke}%
\pgfsetdash{}{0pt}%
\pgfpathmoveto{\pgfqpoint{4.961410in}{5.034702in}}%
\pgfpathlineto{\pgfqpoint{4.914523in}{5.050196in}}%
\pgfusepath{stroke}%
\end{pgfscope}%
\begin{pgfscope}%
\pgfpathrectangle{\pgfqpoint{3.985294in}{4.155455in}}{\pgfqpoint{2.279412in}{2.004545in}}%
\pgfusepath{clip}%
\pgfsetbuttcap%
\pgfsetroundjoin%
\pgfsetlinewidth{1.115895pt}%
\definecolor{currentstroke}{rgb}{0.140536,0.530132,0.555659}%
\pgfsetstrokecolor{currentstroke}%
\pgfsetdash{}{0pt}%
\pgfpathmoveto{\pgfqpoint{4.914523in}{5.050196in}}%
\pgfpathlineto{\pgfqpoint{4.868541in}{5.067514in}}%
\pgfusepath{stroke}%
\end{pgfscope}%
\begin{pgfscope}%
\pgfpathrectangle{\pgfqpoint{3.985294in}{4.155455in}}{\pgfqpoint{2.279412in}{2.004545in}}%
\pgfusepath{clip}%
\pgfsetroundcap%
\pgfsetroundjoin%
\pgfsetlinewidth{0.307872pt}%
\definecolor{currentstroke}{rgb}{0.267004,0.004874,0.329415}%
\pgfsetstrokecolor{currentstroke}%
\pgfsetdash{}{0pt}%
\pgfpathmoveto{\pgfqpoint{6.024314in}{5.041070in}}%
\pgfpathquadraticcurveto{\pgfqpoint{6.024665in}{5.040580in}}{\pgfqpoint{6.022239in}{5.043960in}}%
\pgfusepath{stroke}%
\end{pgfscope}%
\begin{pgfscope}%
\pgfpathrectangle{\pgfqpoint{3.985294in}{4.155455in}}{\pgfqpoint{2.279412in}{2.004545in}}%
\pgfusepath{clip}%
\pgfsetroundcap%
\pgfsetroundjoin%
\definecolor{currentfill}{rgb}{0.267004,0.004874,0.329415}%
\pgfsetfillcolor{currentfill}%
\pgfsetlinewidth{0.307872pt}%
\definecolor{currentstroke}{rgb}{0.267004,0.004874,0.329415}%
\pgfsetstrokecolor{currentstroke}%
\pgfsetdash{}{0pt}%
\pgfpathmoveto{\pgfqpoint{6.012401in}{5.105289in}}%
\pgfpathlineto{\pgfqpoint{6.022239in}{5.043960in}}%
\pgfpathlineto{\pgfqpoint{5.967273in}{5.072887in}}%
\pgfpathlineto{\pgfqpoint{6.012401in}{5.105289in}}%
\pgfpathlineto{\pgfqpoint{6.012401in}{5.105289in}}%
\pgfpathclose%
\pgfusepath{stroke,fill}%
\end{pgfscope}%
\begin{pgfscope}%
\pgfpathrectangle{\pgfqpoint{3.985294in}{4.155455in}}{\pgfqpoint{2.279412in}{2.004545in}}%
\pgfusepath{clip}%
\pgfsetroundcap%
\pgfsetroundjoin%
\pgfsetlinewidth{1.023449pt}%
\definecolor{currentstroke}{rgb}{0.159194,0.482237,0.558073}%
\pgfsetstrokecolor{currentstroke}%
\pgfsetdash{}{0pt}%
\pgfpathmoveto{\pgfqpoint{5.295448in}{5.043548in}}%
\pgfpathquadraticcurveto{\pgfqpoint{5.282943in}{5.044343in}}{\pgfqpoint{5.286240in}{5.044133in}}%
\pgfusepath{stroke}%
\end{pgfscope}%
\begin{pgfscope}%
\pgfpathrectangle{\pgfqpoint{3.985294in}{4.155455in}}{\pgfqpoint{2.279412in}{2.004545in}}%
\pgfusepath{clip}%
\pgfsetroundcap%
\pgfsetroundjoin%
\definecolor{currentfill}{rgb}{0.159194,0.482237,0.558073}%
\pgfsetfillcolor{currentfill}%
\pgfsetlinewidth{1.023449pt}%
\definecolor{currentstroke}{rgb}{0.159194,0.482237,0.558073}%
\pgfsetstrokecolor{currentstroke}%
\pgfsetdash{}{0pt}%
\pgfpathmoveto{\pgfqpoint{5.339922in}{5.012889in}}%
\pgfpathlineto{\pgfqpoint{5.286240in}{5.044133in}}%
\pgfpathlineto{\pgfqpoint{5.343445in}{5.068333in}}%
\pgfpathlineto{\pgfqpoint{5.339922in}{5.012889in}}%
\pgfpathlineto{\pgfqpoint{5.339922in}{5.012889in}}%
\pgfpathclose%
\pgfusepath{stroke,fill}%
\end{pgfscope}%
\begin{pgfscope}%
\pgfpathrectangle{\pgfqpoint{3.985294in}{4.155455in}}{\pgfqpoint{2.279412in}{2.004545in}}%
\pgfusepath{clip}%
\pgfsetroundcap%
\pgfsetroundjoin%
\pgfsetlinewidth{0.449938pt}%
\definecolor{currentstroke}{rgb}{0.283229,0.120777,0.440584}%
\pgfsetstrokecolor{currentstroke}%
\pgfsetdash{}{0pt}%
\pgfpathmoveto{\pgfqpoint{5.646067in}{5.069907in}}%
\pgfpathquadraticcurveto{\pgfqpoint{5.633532in}{5.070135in}}{\pgfqpoint{5.627958in}{5.070237in}}%
\pgfusepath{stroke}%
\end{pgfscope}%
\begin{pgfscope}%
\pgfpathrectangle{\pgfqpoint{3.985294in}{4.155455in}}{\pgfqpoint{2.279412in}{2.004545in}}%
\pgfusepath{clip}%
\pgfsetroundcap%
\pgfsetroundjoin%
\definecolor{currentfill}{rgb}{0.283229,0.120777,0.440584}%
\pgfsetfillcolor{currentfill}%
\pgfsetlinewidth{0.449938pt}%
\definecolor{currentstroke}{rgb}{0.283229,0.120777,0.440584}%
\pgfsetstrokecolor{currentstroke}%
\pgfsetdash{}{0pt}%
\pgfpathmoveto{\pgfqpoint{5.682998in}{5.041451in}}%
\pgfpathlineto{\pgfqpoint{5.627958in}{5.070237in}}%
\pgfpathlineto{\pgfqpoint{5.684010in}{5.096997in}}%
\pgfpathlineto{\pgfqpoint{5.682998in}{5.041451in}}%
\pgfpathlineto{\pgfqpoint{5.682998in}{5.041451in}}%
\pgfpathclose%
\pgfusepath{stroke,fill}%
\end{pgfscope}%
\begin{pgfscope}%
\pgfpathrectangle{\pgfqpoint{3.985294in}{4.155455in}}{\pgfqpoint{2.279412in}{2.004545in}}%
\pgfusepath{clip}%
\pgfsetroundcap%
\pgfsetroundjoin%
\pgfsetlinewidth{0.883034pt}%
\definecolor{currentstroke}{rgb}{0.192357,0.403199,0.555836}%
\pgfsetstrokecolor{currentstroke}%
\pgfsetdash{}{0pt}%
\pgfpathmoveto{\pgfqpoint{5.404285in}{5.152784in}}%
\pgfpathquadraticcurveto{\pgfqpoint{5.391747in}{5.152741in}}{\pgfqpoint{5.392870in}{5.152745in}}%
\pgfusepath{stroke}%
\end{pgfscope}%
\begin{pgfscope}%
\pgfpathrectangle{\pgfqpoint{3.985294in}{4.155455in}}{\pgfqpoint{2.279412in}{2.004545in}}%
\pgfusepath{clip}%
\pgfsetroundcap%
\pgfsetroundjoin%
\definecolor{currentfill}{rgb}{0.192357,0.403199,0.555836}%
\pgfsetfillcolor{currentfill}%
\pgfsetlinewidth{0.883034pt}%
\definecolor{currentstroke}{rgb}{0.192357,0.403199,0.555836}%
\pgfsetstrokecolor{currentstroke}%
\pgfsetdash{}{0pt}%
\pgfpathmoveto{\pgfqpoint{5.448521in}{5.125158in}}%
\pgfpathlineto{\pgfqpoint{5.392870in}{5.152745in}}%
\pgfpathlineto{\pgfqpoint{5.448330in}{5.180713in}}%
\pgfpathlineto{\pgfqpoint{5.448521in}{5.125158in}}%
\pgfpathlineto{\pgfqpoint{5.448521in}{5.125158in}}%
\pgfpathclose%
\pgfusepath{stroke,fill}%
\end{pgfscope}%
\begin{pgfscope}%
\pgfpathrectangle{\pgfqpoint{3.985294in}{4.155455in}}{\pgfqpoint{2.279412in}{2.004545in}}%
\pgfusepath{clip}%
\pgfsetroundcap%
\pgfsetroundjoin%
\pgfsetlinewidth{0.672362pt}%
\definecolor{currentstroke}{rgb}{0.248629,0.278775,0.534556}%
\pgfsetstrokecolor{currentstroke}%
\pgfsetdash{}{0pt}%
\pgfpathmoveto{\pgfqpoint{5.495607in}{5.203794in}}%
\pgfpathquadraticcurveto{\pgfqpoint{5.483071in}{5.203613in}}{\pgfqpoint{5.480935in}{5.203582in}}%
\pgfusepath{stroke}%
\end{pgfscope}%
\begin{pgfscope}%
\pgfpathrectangle{\pgfqpoint{3.985294in}{4.155455in}}{\pgfqpoint{2.279412in}{2.004545in}}%
\pgfusepath{clip}%
\pgfsetroundcap%
\pgfsetroundjoin%
\definecolor{currentfill}{rgb}{0.248629,0.278775,0.534556}%
\pgfsetfillcolor{currentfill}%
\pgfsetlinewidth{0.672362pt}%
\definecolor{currentstroke}{rgb}{0.248629,0.278775,0.534556}%
\pgfsetstrokecolor{currentstroke}%
\pgfsetdash{}{0pt}%
\pgfpathmoveto{\pgfqpoint{5.536888in}{5.176612in}}%
\pgfpathlineto{\pgfqpoint{5.480935in}{5.203582in}}%
\pgfpathlineto{\pgfqpoint{5.536082in}{5.232162in}}%
\pgfpathlineto{\pgfqpoint{5.536888in}{5.176612in}}%
\pgfpathlineto{\pgfqpoint{5.536888in}{5.176612in}}%
\pgfpathclose%
\pgfusepath{stroke,fill}%
\end{pgfscope}%
\begin{pgfscope}%
\pgfpathrectangle{\pgfqpoint{3.985294in}{4.155455in}}{\pgfqpoint{2.279412in}{2.004545in}}%
\pgfusepath{clip}%
\pgfsetroundcap%
\pgfsetroundjoin%
\pgfsetlinewidth{0.625824pt}%
\definecolor{currentstroke}{rgb}{0.260571,0.246922,0.522828}%
\pgfsetstrokecolor{currentstroke}%
\pgfsetdash{}{0pt}%
\pgfpathmoveto{\pgfqpoint{5.495853in}{5.283786in}}%
\pgfpathquadraticcurveto{\pgfqpoint{5.483326in}{5.283348in}}{\pgfqpoint{5.480474in}{5.283248in}}%
\pgfusepath{stroke}%
\end{pgfscope}%
\begin{pgfscope}%
\pgfpathrectangle{\pgfqpoint{3.985294in}{4.155455in}}{\pgfqpoint{2.279412in}{2.004545in}}%
\pgfusepath{clip}%
\pgfsetroundcap%
\pgfsetroundjoin%
\definecolor{currentfill}{rgb}{0.260571,0.246922,0.522828}%
\pgfsetfillcolor{currentfill}%
\pgfsetlinewidth{0.625824pt}%
\definecolor{currentstroke}{rgb}{0.260571,0.246922,0.522828}%
\pgfsetstrokecolor{currentstroke}%
\pgfsetdash{}{0pt}%
\pgfpathmoveto{\pgfqpoint{5.536966in}{5.257428in}}%
\pgfpathlineto{\pgfqpoint{5.480474in}{5.283248in}}%
\pgfpathlineto{\pgfqpoint{5.535025in}{5.312950in}}%
\pgfpathlineto{\pgfqpoint{5.536966in}{5.257428in}}%
\pgfpathlineto{\pgfqpoint{5.536966in}{5.257428in}}%
\pgfpathclose%
\pgfusepath{stroke,fill}%
\end{pgfscope}%
\begin{pgfscope}%
\pgfpathrectangle{\pgfqpoint{3.985294in}{4.155455in}}{\pgfqpoint{2.279412in}{2.004545in}}%
\pgfusepath{clip}%
\pgfsetroundcap%
\pgfsetroundjoin%
\pgfsetlinewidth{0.391386pt}%
\definecolor{currentstroke}{rgb}{0.280894,0.078907,0.402329}%
\pgfsetstrokecolor{currentstroke}%
\pgfsetdash{}{0pt}%
\pgfpathmoveto{\pgfqpoint{5.696559in}{5.329208in}}%
\pgfpathquadraticcurveto{\pgfqpoint{5.684027in}{5.328869in}}{\pgfqpoint{5.677549in}{5.328693in}}%
\pgfusepath{stroke}%
\end{pgfscope}%
\begin{pgfscope}%
\pgfpathrectangle{\pgfqpoint{3.985294in}{4.155455in}}{\pgfqpoint{2.279412in}{2.004545in}}%
\pgfusepath{clip}%
\pgfsetroundcap%
\pgfsetroundjoin%
\definecolor{currentfill}{rgb}{0.280894,0.078907,0.402329}%
\pgfsetfillcolor{currentfill}%
\pgfsetlinewidth{0.391386pt}%
\definecolor{currentstroke}{rgb}{0.280894,0.078907,0.402329}%
\pgfsetstrokecolor{currentstroke}%
\pgfsetdash{}{0pt}%
\pgfpathmoveto{\pgfqpoint{5.733835in}{5.302428in}}%
\pgfpathlineto{\pgfqpoint{5.677549in}{5.328693in}}%
\pgfpathlineto{\pgfqpoint{5.732333in}{5.357963in}}%
\pgfpathlineto{\pgfqpoint{5.733835in}{5.302428in}}%
\pgfpathlineto{\pgfqpoint{5.733835in}{5.302428in}}%
\pgfpathclose%
\pgfusepath{stroke,fill}%
\end{pgfscope}%
\begin{pgfscope}%
\pgfpathrectangle{\pgfqpoint{3.985294in}{4.155455in}}{\pgfqpoint{2.279412in}{2.004545in}}%
\pgfusepath{clip}%
\pgfsetroundcap%
\pgfsetroundjoin%
\pgfsetlinewidth{0.697437pt}%
\definecolor{currentstroke}{rgb}{0.243113,0.292092,0.538516}%
\pgfsetstrokecolor{currentstroke}%
\pgfsetdash{}{0pt}%
\pgfpathmoveto{\pgfqpoint{5.251608in}{4.883480in}}%
\pgfpathquadraticcurveto{\pgfqpoint{5.239257in}{4.885361in}}{\pgfqpoint{5.237572in}{4.885618in}}%
\pgfusepath{stroke}%
\end{pgfscope}%
\begin{pgfscope}%
\pgfpathrectangle{\pgfqpoint{3.985294in}{4.155455in}}{\pgfqpoint{2.279412in}{2.004545in}}%
\pgfusepath{clip}%
\pgfsetroundcap%
\pgfsetroundjoin%
\definecolor{currentfill}{rgb}{0.243113,0.292092,0.538516}%
\pgfsetfillcolor{currentfill}%
\pgfsetlinewidth{0.697437pt}%
\definecolor{currentstroke}{rgb}{0.243113,0.292092,0.538516}%
\pgfsetstrokecolor{currentstroke}%
\pgfsetdash{}{0pt}%
\pgfpathmoveto{\pgfqpoint{5.288312in}{4.849792in}}%
\pgfpathlineto{\pgfqpoint{5.237572in}{4.885618in}}%
\pgfpathlineto{\pgfqpoint{5.296676in}{4.904715in}}%
\pgfpathlineto{\pgfqpoint{5.288312in}{4.849792in}}%
\pgfpathlineto{\pgfqpoint{5.288312in}{4.849792in}}%
\pgfpathclose%
\pgfusepath{stroke,fill}%
\end{pgfscope}%
\begin{pgfscope}%
\pgfpathrectangle{\pgfqpoint{3.985294in}{4.155455in}}{\pgfqpoint{2.279412in}{2.004545in}}%
\pgfusepath{clip}%
\pgfsetroundcap%
\pgfsetroundjoin%
\pgfsetlinewidth{0.428818pt}%
\definecolor{currentstroke}{rgb}{0.282910,0.105393,0.426902}%
\pgfsetstrokecolor{currentstroke}%
\pgfsetdash{}{0pt}%
\pgfpathmoveto{\pgfqpoint{5.643781in}{4.984906in}}%
\pgfpathquadraticcurveto{\pgfqpoint{5.631254in}{4.985362in}}{\pgfqpoint{5.625356in}{4.985577in}}%
\pgfusepath{stroke}%
\end{pgfscope}%
\begin{pgfscope}%
\pgfpathrectangle{\pgfqpoint{3.985294in}{4.155455in}}{\pgfqpoint{2.279412in}{2.004545in}}%
\pgfusepath{clip}%
\pgfsetroundcap%
\pgfsetroundjoin%
\definecolor{currentfill}{rgb}{0.282910,0.105393,0.426902}%
\pgfsetfillcolor{currentfill}%
\pgfsetlinewidth{0.428818pt}%
\definecolor{currentstroke}{rgb}{0.282910,0.105393,0.426902}%
\pgfsetstrokecolor{currentstroke}%
\pgfsetdash{}{0pt}%
\pgfpathmoveto{\pgfqpoint{5.679863in}{4.955795in}}%
\pgfpathlineto{\pgfqpoint{5.625356in}{4.985577in}}%
\pgfpathlineto{\pgfqpoint{5.681886in}{5.011313in}}%
\pgfpathlineto{\pgfqpoint{5.679863in}{4.955795in}}%
\pgfpathlineto{\pgfqpoint{5.679863in}{4.955795in}}%
\pgfpathclose%
\pgfusepath{stroke,fill}%
\end{pgfscope}%
\begin{pgfscope}%
\pgfpathrectangle{\pgfqpoint{3.985294in}{4.155455in}}{\pgfqpoint{2.279412in}{2.004545in}}%
\pgfusepath{clip}%
\pgfsetroundcap%
\pgfsetroundjoin%
\pgfsetlinewidth{0.793142pt}%
\definecolor{currentstroke}{rgb}{0.216210,0.351535,0.550627}%
\pgfsetstrokecolor{currentstroke}%
\pgfsetdash{}{0pt}%
\pgfpathmoveto{\pgfqpoint{5.443058in}{5.114826in}}%
\pgfpathquadraticcurveto{\pgfqpoint{5.430521in}{5.114984in}}{\pgfqpoint{5.430254in}{5.114987in}}%
\pgfusepath{stroke}%
\end{pgfscope}%
\begin{pgfscope}%
\pgfpathrectangle{\pgfqpoint{3.985294in}{4.155455in}}{\pgfqpoint{2.279412in}{2.004545in}}%
\pgfusepath{clip}%
\pgfsetroundcap%
\pgfsetroundjoin%
\definecolor{currentfill}{rgb}{0.216210,0.351535,0.550627}%
\pgfsetfillcolor{currentfill}%
\pgfsetlinewidth{0.793142pt}%
\definecolor{currentstroke}{rgb}{0.216210,0.351535,0.550627}%
\pgfsetstrokecolor{currentstroke}%
\pgfsetdash{}{0pt}%
\pgfpathmoveto{\pgfqpoint{5.485457in}{5.086514in}}%
\pgfpathlineto{\pgfqpoint{5.430254in}{5.114987in}}%
\pgfpathlineto{\pgfqpoint{5.486153in}{5.142066in}}%
\pgfpathlineto{\pgfqpoint{5.485457in}{5.086514in}}%
\pgfpathlineto{\pgfqpoint{5.485457in}{5.086514in}}%
\pgfpathclose%
\pgfusepath{stroke,fill}%
\end{pgfscope}%
\begin{pgfscope}%
\pgfpathrectangle{\pgfqpoint{3.985294in}{4.155455in}}{\pgfqpoint{2.279412in}{2.004545in}}%
\pgfusepath{clip}%
\pgfsetroundcap%
\pgfsetroundjoin%
\pgfsetlinewidth{0.510461pt}%
\definecolor{currentstroke}{rgb}{0.280255,0.165693,0.476498}%
\pgfsetstrokecolor{currentstroke}%
\pgfsetdash{}{0pt}%
\pgfpathmoveto{\pgfqpoint{5.593583in}{5.244794in}}%
\pgfpathquadraticcurveto{\pgfqpoint{5.581049in}{5.244540in}}{\pgfqpoint{5.576410in}{5.244446in}}%
\pgfusepath{stroke}%
\end{pgfscope}%
\begin{pgfscope}%
\pgfpathrectangle{\pgfqpoint{3.985294in}{4.155455in}}{\pgfqpoint{2.279412in}{2.004545in}}%
\pgfusepath{clip}%
\pgfsetroundcap%
\pgfsetroundjoin%
\definecolor{currentfill}{rgb}{0.280255,0.165693,0.476498}%
\pgfsetfillcolor{currentfill}%
\pgfsetlinewidth{0.510461pt}%
\definecolor{currentstroke}{rgb}{0.280255,0.165693,0.476498}%
\pgfsetstrokecolor{currentstroke}%
\pgfsetdash{}{0pt}%
\pgfpathmoveto{\pgfqpoint{5.632517in}{5.217800in}}%
\pgfpathlineto{\pgfqpoint{5.576410in}{5.244446in}}%
\pgfpathlineto{\pgfqpoint{5.631391in}{5.273344in}}%
\pgfpathlineto{\pgfqpoint{5.632517in}{5.217800in}}%
\pgfpathlineto{\pgfqpoint{5.632517in}{5.217800in}}%
\pgfpathclose%
\pgfusepath{stroke,fill}%
\end{pgfscope}%
\begin{pgfscope}%
\pgfpathrectangle{\pgfqpoint{3.985294in}{4.155455in}}{\pgfqpoint{2.279412in}{2.004545in}}%
\pgfusepath{clip}%
\pgfsetroundcap%
\pgfsetroundjoin%
\pgfsetlinewidth{0.805512pt}%
\definecolor{currentstroke}{rgb}{0.212395,0.359683,0.551710}%
\pgfsetstrokecolor{currentstroke}%
\pgfsetdash{}{0pt}%
\pgfpathmoveto{\pgfqpoint{5.243914in}{5.347625in}}%
\pgfpathquadraticcurveto{\pgfqpoint{5.231508in}{5.346037in}}{\pgfqpoint{5.231463in}{5.346031in}}%
\pgfusepath{stroke}%
\end{pgfscope}%
\begin{pgfscope}%
\pgfpathrectangle{\pgfqpoint{3.985294in}{4.155455in}}{\pgfqpoint{2.279412in}{2.004545in}}%
\pgfusepath{clip}%
\pgfsetroundcap%
\pgfsetroundjoin%
\definecolor{currentfill}{rgb}{0.212395,0.359683,0.551710}%
\pgfsetfillcolor{currentfill}%
\pgfsetlinewidth{0.805512pt}%
\definecolor{currentstroke}{rgb}{0.212395,0.359683,0.551710}%
\pgfsetstrokecolor{currentstroke}%
\pgfsetdash{}{0pt}%
\pgfpathmoveto{\pgfqpoint{5.290097in}{5.325536in}}%
\pgfpathlineto{\pgfqpoint{5.231463in}{5.346031in}}%
\pgfpathlineto{\pgfqpoint{5.283039in}{5.380642in}}%
\pgfpathlineto{\pgfqpoint{5.290097in}{5.325536in}}%
\pgfpathlineto{\pgfqpoint{5.290097in}{5.325536in}}%
\pgfpathclose%
\pgfusepath{stroke,fill}%
\end{pgfscope}%
\begin{pgfscope}%
\pgfpathrectangle{\pgfqpoint{3.985294in}{4.155455in}}{\pgfqpoint{2.279412in}{2.004545in}}%
\pgfusepath{clip}%
\pgfsetroundcap%
\pgfsetroundjoin%
\pgfsetlinewidth{0.524295pt}%
\definecolor{currentstroke}{rgb}{0.278826,0.175490,0.483397}%
\pgfsetstrokecolor{currentstroke}%
\pgfsetdash{}{0pt}%
\pgfpathmoveto{\pgfqpoint{5.493655in}{5.412834in}}%
\pgfpathquadraticcurveto{\pgfqpoint{5.481149in}{5.412046in}}{\pgfqpoint{5.476739in}{5.411768in}}%
\pgfusepath{stroke}%
\end{pgfscope}%
\begin{pgfscope}%
\pgfpathrectangle{\pgfqpoint{3.985294in}{4.155455in}}{\pgfqpoint{2.279412in}{2.004545in}}%
\pgfusepath{clip}%
\pgfsetroundcap%
\pgfsetroundjoin%
\definecolor{currentfill}{rgb}{0.278826,0.175490,0.483397}%
\pgfsetfillcolor{currentfill}%
\pgfsetlinewidth{0.524295pt}%
\definecolor{currentstroke}{rgb}{0.278826,0.175490,0.483397}%
\pgfsetstrokecolor{currentstroke}%
\pgfsetdash{}{0pt}%
\pgfpathmoveto{\pgfqpoint{5.533932in}{5.387539in}}%
\pgfpathlineto{\pgfqpoint{5.476739in}{5.411768in}}%
\pgfpathlineto{\pgfqpoint{5.530438in}{5.442985in}}%
\pgfpathlineto{\pgfqpoint{5.533932in}{5.387539in}}%
\pgfpathlineto{\pgfqpoint{5.533932in}{5.387539in}}%
\pgfpathclose%
\pgfusepath{stroke,fill}%
\end{pgfscope}%
\begin{pgfscope}%
\pgfpathrectangle{\pgfqpoint{3.985294in}{4.155455in}}{\pgfqpoint{2.279412in}{2.004545in}}%
\pgfusepath{clip}%
\pgfsetroundcap%
\pgfsetroundjoin%
\pgfsetlinewidth{0.496276pt}%
\definecolor{currentstroke}{rgb}{0.281412,0.155834,0.469201}%
\pgfsetstrokecolor{currentstroke}%
\pgfsetdash{}{0pt}%
\pgfpathmoveto{\pgfqpoint{5.442796in}{4.817000in}}%
\pgfpathquadraticcurveto{\pgfqpoint{5.430341in}{4.818271in}}{\pgfqpoint{5.425525in}{4.818762in}}%
\pgfusepath{stroke}%
\end{pgfscope}%
\begin{pgfscope}%
\pgfpathrectangle{\pgfqpoint{3.985294in}{4.155455in}}{\pgfqpoint{2.279412in}{2.004545in}}%
\pgfusepath{clip}%
\pgfsetroundcap%
\pgfsetroundjoin%
\definecolor{currentfill}{rgb}{0.281412,0.155834,0.469201}%
\pgfsetfillcolor{currentfill}%
\pgfsetlinewidth{0.496276pt}%
\definecolor{currentstroke}{rgb}{0.281412,0.155834,0.469201}%
\pgfsetstrokecolor{currentstroke}%
\pgfsetdash{}{0pt}%
\pgfpathmoveto{\pgfqpoint{5.477975in}{4.785490in}}%
\pgfpathlineto{\pgfqpoint{5.425525in}{4.818762in}}%
\pgfpathlineto{\pgfqpoint{5.483612in}{4.840759in}}%
\pgfpathlineto{\pgfqpoint{5.477975in}{4.785490in}}%
\pgfpathlineto{\pgfqpoint{5.477975in}{4.785490in}}%
\pgfpathclose%
\pgfusepath{stroke,fill}%
\end{pgfscope}%
\begin{pgfscope}%
\pgfpathrectangle{\pgfqpoint{3.985294in}{4.155455in}}{\pgfqpoint{2.279412in}{2.004545in}}%
\pgfusepath{clip}%
\pgfsetroundcap%
\pgfsetroundjoin%
\pgfsetlinewidth{0.396765pt}%
\definecolor{currentstroke}{rgb}{0.280894,0.078907,0.402329}%
\pgfsetstrokecolor{currentstroke}%
\pgfsetdash{}{0pt}%
\pgfpathmoveto{\pgfqpoint{5.642739in}{4.894810in}}%
\pgfpathquadraticcurveto{\pgfqpoint{5.630219in}{4.895393in}}{\pgfqpoint{5.623831in}{4.895691in}}%
\pgfusepath{stroke}%
\end{pgfscope}%
\begin{pgfscope}%
\pgfpathrectangle{\pgfqpoint{3.985294in}{4.155455in}}{\pgfqpoint{2.279412in}{2.004545in}}%
\pgfusepath{clip}%
\pgfsetroundcap%
\pgfsetroundjoin%
\definecolor{currentfill}{rgb}{0.280894,0.078907,0.402329}%
\pgfsetfillcolor{currentfill}%
\pgfsetlinewidth{0.396765pt}%
\definecolor{currentstroke}{rgb}{0.280894,0.078907,0.402329}%
\pgfsetstrokecolor{currentstroke}%
\pgfsetdash{}{0pt}%
\pgfpathmoveto{\pgfqpoint{5.678034in}{4.865360in}}%
\pgfpathlineto{\pgfqpoint{5.623831in}{4.895691in}}%
\pgfpathlineto{\pgfqpoint{5.680618in}{4.920855in}}%
\pgfpathlineto{\pgfqpoint{5.678034in}{4.865360in}}%
\pgfpathlineto{\pgfqpoint{5.678034in}{4.865360in}}%
\pgfpathclose%
\pgfusepath{stroke,fill}%
\end{pgfscope}%
\begin{pgfscope}%
\pgfpathrectangle{\pgfqpoint{3.985294in}{4.155455in}}{\pgfqpoint{2.279412in}{2.004545in}}%
\pgfusepath{clip}%
\pgfsetroundcap%
\pgfsetroundjoin%
\pgfsetlinewidth{0.414712pt}%
\definecolor{currentstroke}{rgb}{0.282327,0.094955,0.417331}%
\pgfsetstrokecolor{currentstroke}%
\pgfsetdash{}{0pt}%
\pgfpathmoveto{\pgfqpoint{5.642686in}{4.939834in}}%
\pgfpathquadraticcurveto{\pgfqpoint{5.630164in}{4.940387in}}{\pgfqpoint{5.624051in}{4.940656in}}%
\pgfusepath{stroke}%
\end{pgfscope}%
\begin{pgfscope}%
\pgfpathrectangle{\pgfqpoint{3.985294in}{4.155455in}}{\pgfqpoint{2.279412in}{2.004545in}}%
\pgfusepath{clip}%
\pgfsetroundcap%
\pgfsetroundjoin%
\definecolor{currentfill}{rgb}{0.282327,0.094955,0.417331}%
\pgfsetfillcolor{currentfill}%
\pgfsetlinewidth{0.414712pt}%
\definecolor{currentstroke}{rgb}{0.282327,0.094955,0.417331}%
\pgfsetstrokecolor{currentstroke}%
\pgfsetdash{}{0pt}%
\pgfpathmoveto{\pgfqpoint{5.678328in}{4.910456in}}%
\pgfpathlineto{\pgfqpoint{5.624051in}{4.940656in}}%
\pgfpathlineto{\pgfqpoint{5.680778in}{4.965958in}}%
\pgfpathlineto{\pgfqpoint{5.678328in}{4.910456in}}%
\pgfpathlineto{\pgfqpoint{5.678328in}{4.910456in}}%
\pgfpathclose%
\pgfusepath{stroke,fill}%
\end{pgfscope}%
\begin{pgfscope}%
\pgfpathrectangle{\pgfqpoint{3.985294in}{4.155455in}}{\pgfqpoint{2.279412in}{2.004545in}}%
\pgfusepath{clip}%
\pgfsetroundcap%
\pgfsetroundjoin%
\pgfsetlinewidth{0.365977pt}%
\definecolor{currentstroke}{rgb}{0.277941,0.056324,0.381191}%
\pgfsetstrokecolor{currentstroke}%
\pgfsetdash{}{0pt}%
\pgfpathmoveto{\pgfqpoint{5.692929in}{5.466155in}}%
\pgfpathquadraticcurveto{\pgfqpoint{5.680420in}{5.465406in}}{\pgfqpoint{5.673563in}{5.464995in}}%
\pgfusepath{stroke}%
\end{pgfscope}%
\begin{pgfscope}%
\pgfpathrectangle{\pgfqpoint{3.985294in}{4.155455in}}{\pgfqpoint{2.279412in}{2.004545in}}%
\pgfusepath{clip}%
\pgfsetroundcap%
\pgfsetroundjoin%
\definecolor{currentfill}{rgb}{0.277941,0.056324,0.381191}%
\pgfsetfillcolor{currentfill}%
\pgfsetlinewidth{0.365977pt}%
\definecolor{currentstroke}{rgb}{0.277941,0.056324,0.381191}%
\pgfsetstrokecolor{currentstroke}%
\pgfsetdash{}{0pt}%
\pgfpathmoveto{\pgfqpoint{5.730680in}{5.440589in}}%
\pgfpathlineto{\pgfqpoint{5.673563in}{5.464995in}}%
\pgfpathlineto{\pgfqpoint{5.727358in}{5.496045in}}%
\pgfpathlineto{\pgfqpoint{5.730680in}{5.440589in}}%
\pgfpathlineto{\pgfqpoint{5.730680in}{5.440589in}}%
\pgfpathclose%
\pgfusepath{stroke,fill}%
\end{pgfscope}%
\begin{pgfscope}%
\pgfpathrectangle{\pgfqpoint{3.985294in}{4.155455in}}{\pgfqpoint{2.279412in}{2.004545in}}%
\pgfusepath{clip}%
\pgfsetroundcap%
\pgfsetroundjoin%
\pgfsetlinewidth{0.606916pt}%
\definecolor{currentstroke}{rgb}{0.265145,0.232956,0.516599}%
\pgfsetstrokecolor{currentstroke}%
\pgfsetdash{}{0pt}%
\pgfpathmoveto{\pgfqpoint{5.244882in}{5.465614in}}%
\pgfpathquadraticcurveto{\pgfqpoint{5.232614in}{5.463342in}}{\pgfqpoint{5.229579in}{5.462780in}}%
\pgfusepath{stroke}%
\end{pgfscope}%
\begin{pgfscope}%
\pgfpathrectangle{\pgfqpoint{3.985294in}{4.155455in}}{\pgfqpoint{2.279412in}{2.004545in}}%
\pgfusepath{clip}%
\pgfsetroundcap%
\pgfsetroundjoin%
\definecolor{currentfill}{rgb}{0.265145,0.232956,0.516599}%
\pgfsetfillcolor{currentfill}%
\pgfsetlinewidth{0.606916pt}%
\definecolor{currentstroke}{rgb}{0.265145,0.232956,0.516599}%
\pgfsetstrokecolor{currentstroke}%
\pgfsetdash{}{0pt}%
\pgfpathmoveto{\pgfqpoint{5.289264in}{5.445582in}}%
\pgfpathlineto{\pgfqpoint{5.229579in}{5.462780in}}%
\pgfpathlineto{\pgfqpoint{5.279148in}{5.500209in}}%
\pgfpathlineto{\pgfqpoint{5.289264in}{5.445582in}}%
\pgfpathlineto{\pgfqpoint{5.289264in}{5.445582in}}%
\pgfpathclose%
\pgfusepath{stroke,fill}%
\end{pgfscope}%
\begin{pgfscope}%
\pgfpathrectangle{\pgfqpoint{3.985294in}{4.155455in}}{\pgfqpoint{2.279412in}{2.004545in}}%
\pgfusepath{clip}%
\pgfsetroundcap%
\pgfsetroundjoin%
\pgfsetlinewidth{0.471024pt}%
\definecolor{currentstroke}{rgb}{0.282884,0.135920,0.453427}%
\pgfsetstrokecolor{currentstroke}%
\pgfsetdash{}{0pt}%
\pgfpathmoveto{\pgfqpoint{5.393499in}{5.531458in}}%
\pgfpathquadraticcurveto{\pgfqpoint{5.381106in}{5.529803in}}{\pgfqpoint{5.375935in}{5.529113in}}%
\pgfusepath{stroke}%
\end{pgfscope}%
\begin{pgfscope}%
\pgfpathrectangle{\pgfqpoint{3.985294in}{4.155455in}}{\pgfqpoint{2.279412in}{2.004545in}}%
\pgfusepath{clip}%
\pgfsetroundcap%
\pgfsetroundjoin%
\definecolor{currentfill}{rgb}{0.282884,0.135920,0.453427}%
\pgfsetfillcolor{currentfill}%
\pgfsetlinewidth{0.471024pt}%
\definecolor{currentstroke}{rgb}{0.282884,0.135920,0.453427}%
\pgfsetstrokecolor{currentstroke}%
\pgfsetdash{}{0pt}%
\pgfpathmoveto{\pgfqpoint{5.434678in}{5.508932in}}%
\pgfpathlineto{\pgfqpoint{5.375935in}{5.529113in}}%
\pgfpathlineto{\pgfqpoint{5.427326in}{5.563999in}}%
\pgfpathlineto{\pgfqpoint{5.434678in}{5.508932in}}%
\pgfpathlineto{\pgfqpoint{5.434678in}{5.508932in}}%
\pgfpathclose%
\pgfusepath{stroke,fill}%
\end{pgfscope}%
\begin{pgfscope}%
\pgfpathrectangle{\pgfqpoint{3.985294in}{4.155455in}}{\pgfqpoint{2.279412in}{2.004545in}}%
\pgfusepath{clip}%
\pgfsetroundcap%
\pgfsetroundjoin%
\pgfsetlinewidth{0.473964pt}%
\definecolor{currentstroke}{rgb}{0.282623,0.140926,0.457517}%
\pgfsetstrokecolor{currentstroke}%
\pgfsetdash{}{0pt}%
\pgfpathmoveto{\pgfqpoint{5.244836in}{4.764224in}}%
\pgfpathquadraticcurveto{\pgfqpoint{5.232764in}{4.767155in}}{\pgfqpoint{5.227818in}{4.768356in}}%
\pgfusepath{stroke}%
\end{pgfscope}%
\begin{pgfscope}%
\pgfpathrectangle{\pgfqpoint{3.985294in}{4.155455in}}{\pgfqpoint{2.279412in}{2.004545in}}%
\pgfusepath{clip}%
\pgfsetroundcap%
\pgfsetroundjoin%
\definecolor{currentfill}{rgb}{0.282623,0.140926,0.457517}%
\pgfsetfillcolor{currentfill}%
\pgfsetlinewidth{0.473964pt}%
\definecolor{currentstroke}{rgb}{0.282623,0.140926,0.457517}%
\pgfsetstrokecolor{currentstroke}%
\pgfsetdash{}{0pt}%
\pgfpathmoveto{\pgfqpoint{5.275250in}{4.728253in}}%
\pgfpathlineto{\pgfqpoint{5.227818in}{4.768356in}}%
\pgfpathlineto{\pgfqpoint{5.288360in}{4.782240in}}%
\pgfpathlineto{\pgfqpoint{5.275250in}{4.728253in}}%
\pgfpathlineto{\pgfqpoint{5.275250in}{4.728253in}}%
\pgfpathclose%
\pgfusepath{stroke,fill}%
\end{pgfscope}%
\begin{pgfscope}%
\pgfpathrectangle{\pgfqpoint{3.985294in}{4.155455in}}{\pgfqpoint{2.279412in}{2.004545in}}%
\pgfusepath{clip}%
\pgfsetroundcap%
\pgfsetroundjoin%
\pgfsetlinewidth{0.427521pt}%
\definecolor{currentstroke}{rgb}{0.282910,0.105393,0.426902}%
\pgfsetstrokecolor{currentstroke}%
\pgfsetdash{}{0pt}%
\pgfpathmoveto{\pgfqpoint{5.154122in}{5.605350in}}%
\pgfpathquadraticcurveto{\pgfqpoint{5.142920in}{5.600504in}}{\pgfqpoint{5.137787in}{5.598284in}}%
\pgfusepath{stroke}%
\end{pgfscope}%
\begin{pgfscope}%
\pgfpathrectangle{\pgfqpoint{3.985294in}{4.155455in}}{\pgfqpoint{2.279412in}{2.004545in}}%
\pgfusepath{clip}%
\pgfsetroundcap%
\pgfsetroundjoin%
\definecolor{currentfill}{rgb}{0.282910,0.105393,0.426902}%
\pgfsetfillcolor{currentfill}%
\pgfsetlinewidth{0.427521pt}%
\definecolor{currentstroke}{rgb}{0.282910,0.105393,0.426902}%
\pgfsetstrokecolor{currentstroke}%
\pgfsetdash{}{0pt}%
\pgfpathmoveto{\pgfqpoint{5.199805in}{5.594846in}}%
\pgfpathlineto{\pgfqpoint{5.137787in}{5.598284in}}%
\pgfpathlineto{\pgfqpoint{5.177749in}{5.645835in}}%
\pgfpathlineto{\pgfqpoint{5.199805in}{5.594846in}}%
\pgfpathlineto{\pgfqpoint{5.199805in}{5.594846in}}%
\pgfpathclose%
\pgfusepath{stroke,fill}%
\end{pgfscope}%
\begin{pgfscope}%
\pgfpathrectangle{\pgfqpoint{3.985294in}{4.155455in}}{\pgfqpoint{2.279412in}{2.004545in}}%
\pgfusepath{clip}%
\pgfsetroundcap%
\pgfsetroundjoin%
\pgfsetlinewidth{0.377136pt}%
\definecolor{currentstroke}{rgb}{0.279566,0.067836,0.391917}%
\pgfsetstrokecolor{currentstroke}%
\pgfsetdash{}{0pt}%
\pgfpathmoveto{\pgfqpoint{5.336571in}{5.723646in}}%
\pgfpathquadraticcurveto{\pgfqpoint{5.324469in}{5.720815in}}{\pgfqpoint{5.318048in}{5.719312in}}%
\pgfusepath{stroke}%
\end{pgfscope}%
\begin{pgfscope}%
\pgfpathrectangle{\pgfqpoint{3.985294in}{4.155455in}}{\pgfqpoint{2.279412in}{2.004545in}}%
\pgfusepath{clip}%
\pgfsetroundcap%
\pgfsetroundjoin%
\definecolor{currentfill}{rgb}{0.279566,0.067836,0.391917}%
\pgfsetfillcolor{currentfill}%
\pgfsetlinewidth{0.377136pt}%
\definecolor{currentstroke}{rgb}{0.279566,0.067836,0.391917}%
\pgfsetstrokecolor{currentstroke}%
\pgfsetdash{}{0pt}%
\pgfpathmoveto{\pgfqpoint{5.378471in}{5.704922in}}%
\pgfpathlineto{\pgfqpoint{5.318048in}{5.719312in}}%
\pgfpathlineto{\pgfqpoint{5.365814in}{5.759016in}}%
\pgfpathlineto{\pgfqpoint{5.378471in}{5.704922in}}%
\pgfpathlineto{\pgfqpoint{5.378471in}{5.704922in}}%
\pgfpathclose%
\pgfusepath{stroke,fill}%
\end{pgfscope}%
\begin{pgfscope}%
\pgfpathrectangle{\pgfqpoint{3.985294in}{4.155455in}}{\pgfqpoint{2.279412in}{2.004545in}}%
\pgfusepath{clip}%
\pgfsetroundcap%
\pgfsetroundjoin%
\pgfsetlinewidth{0.395621pt}%
\definecolor{currentstroke}{rgb}{0.280894,0.078907,0.402329}%
\pgfsetstrokecolor{currentstroke}%
\pgfsetdash{}{0pt}%
\pgfpathmoveto{\pgfqpoint{5.181825in}{4.634231in}}%
\pgfpathquadraticcurveto{\pgfqpoint{5.170697in}{4.639256in}}{\pgfqpoint{5.165146in}{4.641762in}}%
\pgfusepath{stroke}%
\end{pgfscope}%
\begin{pgfscope}%
\pgfpathrectangle{\pgfqpoint{3.985294in}{4.155455in}}{\pgfqpoint{2.279412in}{2.004545in}}%
\pgfusepath{clip}%
\pgfsetroundcap%
\pgfsetroundjoin%
\definecolor{currentfill}{rgb}{0.280894,0.078907,0.402329}%
\pgfsetfillcolor{currentfill}%
\pgfsetlinewidth{0.395621pt}%
\definecolor{currentstroke}{rgb}{0.280894,0.078907,0.402329}%
\pgfsetstrokecolor{currentstroke}%
\pgfsetdash{}{0pt}%
\pgfpathmoveto{\pgfqpoint{5.204348in}{4.593583in}}%
\pgfpathlineto{\pgfqpoint{5.165146in}{4.641762in}}%
\pgfpathlineto{\pgfqpoint{5.227210in}{4.644216in}}%
\pgfpathlineto{\pgfqpoint{5.204348in}{4.593583in}}%
\pgfpathlineto{\pgfqpoint{5.204348in}{4.593583in}}%
\pgfpathclose%
\pgfusepath{stroke,fill}%
\end{pgfscope}%
\begin{pgfscope}%
\pgfpathrectangle{\pgfqpoint{3.985294in}{4.155455in}}{\pgfqpoint{2.279412in}{2.004545in}}%
\pgfusepath{clip}%
\pgfsetroundcap%
\pgfsetroundjoin%
\pgfsetlinewidth{0.355842pt}%
\definecolor{currentstroke}{rgb}{0.276022,0.044167,0.370164}%
\pgfsetstrokecolor{currentstroke}%
\pgfsetdash{}{0pt}%
\pgfpathmoveto{\pgfqpoint{5.478280in}{4.593489in}}%
\pgfpathquadraticcurveto{\pgfqpoint{5.465894in}{4.595183in}}{\pgfqpoint{5.458963in}{4.596132in}}%
\pgfusepath{stroke}%
\end{pgfscope}%
\begin{pgfscope}%
\pgfpathrectangle{\pgfqpoint{3.985294in}{4.155455in}}{\pgfqpoint{2.279412in}{2.004545in}}%
\pgfusepath{clip}%
\pgfsetroundcap%
\pgfsetroundjoin%
\definecolor{currentfill}{rgb}{0.276022,0.044167,0.370164}%
\pgfsetfillcolor{currentfill}%
\pgfsetlinewidth{0.355842pt}%
\definecolor{currentstroke}{rgb}{0.276022,0.044167,0.370164}%
\pgfsetstrokecolor{currentstroke}%
\pgfsetdash{}{0pt}%
\pgfpathmoveto{\pgfqpoint{5.510240in}{4.561079in}}%
\pgfpathlineto{\pgfqpoint{5.458963in}{4.596132in}}%
\pgfpathlineto{\pgfqpoint{5.517771in}{4.616122in}}%
\pgfpathlineto{\pgfqpoint{5.510240in}{4.561079in}}%
\pgfpathlineto{\pgfqpoint{5.510240in}{4.561079in}}%
\pgfpathclose%
\pgfusepath{stroke,fill}%
\end{pgfscope}%
\begin{pgfscope}%
\pgfpathrectangle{\pgfqpoint{3.985294in}{4.155455in}}{\pgfqpoint{2.279412in}{2.004545in}}%
\pgfusepath{clip}%
\pgfsetroundcap%
\pgfsetroundjoin%
\pgfsetlinewidth{0.361167pt}%
\definecolor{currentstroke}{rgb}{0.277018,0.050344,0.375715}%
\pgfsetstrokecolor{currentstroke}%
\pgfsetdash{}{0pt}%
\pgfpathmoveto{\pgfqpoint{5.640508in}{4.753907in}}%
\pgfpathquadraticcurveto{\pgfqpoint{5.628015in}{4.754827in}}{\pgfqpoint{5.621095in}{4.755336in}}%
\pgfusepath{stroke}%
\end{pgfscope}%
\begin{pgfscope}%
\pgfpathrectangle{\pgfqpoint{3.985294in}{4.155455in}}{\pgfqpoint{2.279412in}{2.004545in}}%
\pgfusepath{clip}%
\pgfsetroundcap%
\pgfsetroundjoin%
\definecolor{currentfill}{rgb}{0.277018,0.050344,0.375715}%
\pgfsetfillcolor{currentfill}%
\pgfsetlinewidth{0.361167pt}%
\definecolor{currentstroke}{rgb}{0.277018,0.050344,0.375715}%
\pgfsetstrokecolor{currentstroke}%
\pgfsetdash{}{0pt}%
\pgfpathmoveto{\pgfqpoint{5.674462in}{4.723556in}}%
\pgfpathlineto{\pgfqpoint{5.621095in}{4.755336in}}%
\pgfpathlineto{\pgfqpoint{5.678539in}{4.778962in}}%
\pgfpathlineto{\pgfqpoint{5.674462in}{4.723556in}}%
\pgfpathlineto{\pgfqpoint{5.674462in}{4.723556in}}%
\pgfpathclose%
\pgfusepath{stroke,fill}%
\end{pgfscope}%
\begin{pgfscope}%
\pgfpathrectangle{\pgfqpoint{3.985294in}{4.155455in}}{\pgfqpoint{2.279412in}{2.004545in}}%
\pgfusepath{clip}%
\pgfsetroundcap%
\pgfsetroundjoin%
\pgfsetlinewidth{0.377199pt}%
\definecolor{currentstroke}{rgb}{0.279566,0.067836,0.391917}%
\pgfsetstrokecolor{currentstroke}%
\pgfsetdash{}{0pt}%
\pgfpathmoveto{\pgfqpoint{5.428506in}{4.644391in}}%
\pgfpathquadraticcurveto{\pgfqpoint{5.416168in}{4.646347in}}{\pgfqpoint{5.409594in}{4.647390in}}%
\pgfusepath{stroke}%
\end{pgfscope}%
\begin{pgfscope}%
\pgfpathrectangle{\pgfqpoint{3.985294in}{4.155455in}}{\pgfqpoint{2.279412in}{2.004545in}}%
\pgfusepath{clip}%
\pgfsetroundcap%
\pgfsetroundjoin%
\definecolor{currentfill}{rgb}{0.279566,0.067836,0.391917}%
\pgfsetfillcolor{currentfill}%
\pgfsetlinewidth{0.377199pt}%
\definecolor{currentstroke}{rgb}{0.279566,0.067836,0.391917}%
\pgfsetstrokecolor{currentstroke}%
\pgfsetdash{}{0pt}%
\pgfpathmoveto{\pgfqpoint{5.460114in}{4.611255in}}%
\pgfpathlineto{\pgfqpoint{5.409594in}{4.647390in}}%
\pgfpathlineto{\pgfqpoint{5.468814in}{4.666125in}}%
\pgfpathlineto{\pgfqpoint{5.460114in}{4.611255in}}%
\pgfpathlineto{\pgfqpoint{5.460114in}{4.611255in}}%
\pgfpathclose%
\pgfusepath{stroke,fill}%
\end{pgfscope}%
\begin{pgfscope}%
\pgfpathrectangle{\pgfqpoint{3.985294in}{4.155455in}}{\pgfqpoint{2.279412in}{2.004545in}}%
\pgfusepath{clip}%
\pgfsetroundcap%
\pgfsetroundjoin%
\pgfsetlinewidth{0.331704pt}%
\definecolor{currentstroke}{rgb}{0.272594,0.025563,0.353093}%
\pgfsetstrokecolor{currentstroke}%
\pgfsetdash{}{0pt}%
\pgfpathmoveto{\pgfqpoint{5.683584in}{4.644145in}}%
\pgfpathquadraticcurveto{\pgfqpoint{5.671094in}{4.645010in}}{\pgfqpoint{5.663724in}{4.645520in}}%
\pgfusepath{stroke}%
\end{pgfscope}%
\begin{pgfscope}%
\pgfpathrectangle{\pgfqpoint{3.985294in}{4.155455in}}{\pgfqpoint{2.279412in}{2.004545in}}%
\pgfusepath{clip}%
\pgfsetroundcap%
\pgfsetroundjoin%
\definecolor{currentfill}{rgb}{0.272594,0.025563,0.353093}%
\pgfsetfillcolor{currentfill}%
\pgfsetlinewidth{0.331704pt}%
\definecolor{currentstroke}{rgb}{0.272594,0.025563,0.353093}%
\pgfsetstrokecolor{currentstroke}%
\pgfsetdash{}{0pt}%
\pgfpathmoveto{\pgfqpoint{5.717228in}{4.613971in}}%
\pgfpathlineto{\pgfqpoint{5.663724in}{4.645520in}}%
\pgfpathlineto{\pgfqpoint{5.721066in}{4.669394in}}%
\pgfpathlineto{\pgfqpoint{5.717228in}{4.613971in}}%
\pgfpathlineto{\pgfqpoint{5.717228in}{4.613971in}}%
\pgfpathclose%
\pgfusepath{stroke,fill}%
\end{pgfscope}%
\begin{pgfscope}%
\pgfpathrectangle{\pgfqpoint{3.985294in}{4.155455in}}{\pgfqpoint{2.279412in}{2.004545in}}%
\pgfusepath{clip}%
\pgfsetroundcap%
\pgfsetroundjoin%
\pgfsetlinewidth{0.397800pt}%
\definecolor{currentstroke}{rgb}{0.281446,0.084320,0.407414}%
\pgfsetstrokecolor{currentstroke}%
\pgfsetdash{}{0pt}%
\pgfpathmoveto{\pgfqpoint{5.086501in}{5.615132in}}%
\pgfpathquadraticcurveto{\pgfqpoint{5.076463in}{5.608647in}}{\pgfqpoint{5.071593in}{5.605500in}}%
\pgfusepath{stroke}%
\end{pgfscope}%
\begin{pgfscope}%
\pgfpathrectangle{\pgfqpoint{3.985294in}{4.155455in}}{\pgfqpoint{2.279412in}{2.004545in}}%
\pgfusepath{clip}%
\pgfsetroundcap%
\pgfsetroundjoin%
\definecolor{currentfill}{rgb}{0.281446,0.084320,0.407414}%
\pgfsetfillcolor{currentfill}%
\pgfsetlinewidth{0.397800pt}%
\definecolor{currentstroke}{rgb}{0.281446,0.084320,0.407414}%
\pgfsetstrokecolor{currentstroke}%
\pgfsetdash{}{0pt}%
\pgfpathmoveto{\pgfqpoint{5.133331in}{5.612318in}}%
\pgfpathlineto{\pgfqpoint{5.071593in}{5.605500in}}%
\pgfpathlineto{\pgfqpoint{5.103182in}{5.658981in}}%
\pgfpathlineto{\pgfqpoint{5.133331in}{5.612318in}}%
\pgfpathlineto{\pgfqpoint{5.133331in}{5.612318in}}%
\pgfpathclose%
\pgfusepath{stroke,fill}%
\end{pgfscope}%
\begin{pgfscope}%
\pgfpathrectangle{\pgfqpoint{3.985294in}{4.155455in}}{\pgfqpoint{2.279412in}{2.004545in}}%
\pgfusepath{clip}%
\pgfsetroundcap%
\pgfsetroundjoin%
\pgfsetlinewidth{0.463304pt}%
\definecolor{currentstroke}{rgb}{0.283072,0.130895,0.449241}%
\pgfsetstrokecolor{currentstroke}%
\pgfsetdash{}{0pt}%
\pgfpathmoveto{\pgfqpoint{5.133239in}{4.736253in}}%
\pgfpathquadraticcurveto{\pgfqpoint{5.122081in}{4.741211in}}{\pgfqpoint{5.117473in}{4.743258in}}%
\pgfusepath{stroke}%
\end{pgfscope}%
\begin{pgfscope}%
\pgfpathrectangle{\pgfqpoint{3.985294in}{4.155455in}}{\pgfqpoint{2.279412in}{2.004545in}}%
\pgfusepath{clip}%
\pgfsetroundcap%
\pgfsetroundjoin%
\definecolor{currentfill}{rgb}{0.283072,0.130895,0.449241}%
\pgfsetfillcolor{currentfill}%
\pgfsetlinewidth{0.463304pt}%
\definecolor{currentstroke}{rgb}{0.283072,0.130895,0.449241}%
\pgfsetstrokecolor{currentstroke}%
\pgfsetdash{}{0pt}%
\pgfpathmoveto{\pgfqpoint{5.156963in}{4.695315in}}%
\pgfpathlineto{\pgfqpoint{5.117473in}{4.743258in}}%
\pgfpathlineto{\pgfqpoint{5.179522in}{4.746084in}}%
\pgfpathlineto{\pgfqpoint{5.156963in}{4.695315in}}%
\pgfpathlineto{\pgfqpoint{5.156963in}{4.695315in}}%
\pgfpathclose%
\pgfusepath{stroke,fill}%
\end{pgfscope}%
\begin{pgfscope}%
\pgfpathrectangle{\pgfqpoint{3.985294in}{4.155455in}}{\pgfqpoint{2.279412in}{2.004545in}}%
\pgfusepath{clip}%
\pgfsetroundcap%
\pgfsetroundjoin%
\pgfsetlinewidth{0.439890pt}%
\definecolor{currentstroke}{rgb}{0.283197,0.115680,0.436115}%
\pgfsetstrokecolor{currentstroke}%
\pgfsetdash{}{0pt}%
\pgfpathmoveto{\pgfqpoint{5.388787in}{5.584325in}}%
\pgfpathquadraticcurveto{\pgfqpoint{5.376433in}{5.582457in}}{\pgfqpoint{5.370807in}{5.581606in}}%
\pgfusepath{stroke}%
\end{pgfscope}%
\begin{pgfscope}%
\pgfpathrectangle{\pgfqpoint{3.985294in}{4.155455in}}{\pgfqpoint{2.279412in}{2.004545in}}%
\pgfusepath{clip}%
\pgfsetroundcap%
\pgfsetroundjoin%
\definecolor{currentfill}{rgb}{0.283197,0.115680,0.436115}%
\pgfsetfillcolor{currentfill}%
\pgfsetlinewidth{0.439890pt}%
\definecolor{currentstroke}{rgb}{0.283197,0.115680,0.436115}%
\pgfsetstrokecolor{currentstroke}%
\pgfsetdash{}{0pt}%
\pgfpathmoveto{\pgfqpoint{5.429891in}{5.562447in}}%
\pgfpathlineto{\pgfqpoint{5.370807in}{5.581606in}}%
\pgfpathlineto{\pgfqpoint{5.421585in}{5.617378in}}%
\pgfpathlineto{\pgfqpoint{5.429891in}{5.562447in}}%
\pgfpathlineto{\pgfqpoint{5.429891in}{5.562447in}}%
\pgfpathclose%
\pgfusepath{stroke,fill}%
\end{pgfscope}%
\begin{pgfscope}%
\pgfpathrectangle{\pgfqpoint{3.985294in}{4.155455in}}{\pgfqpoint{2.279412in}{2.004545in}}%
\pgfusepath{clip}%
\pgfsetroundcap%
\pgfsetroundjoin%
\pgfsetlinewidth{0.643711pt}%
\definecolor{currentstroke}{rgb}{0.257322,0.256130,0.526563}%
\pgfsetstrokecolor{currentstroke}%
\pgfsetdash{}{0pt}%
\pgfpathmoveto{\pgfqpoint{4.889760in}{4.889934in}}%
\pgfpathquadraticcurveto{\pgfqpoint{4.890009in}{4.896783in}}{\pgfqpoint{4.889896in}{4.893681in}}%
\pgfusepath{stroke}%
\end{pgfscope}%
\begin{pgfscope}%
\pgfpathrectangle{\pgfqpoint{3.985294in}{4.155455in}}{\pgfqpoint{2.279412in}{2.004545in}}%
\pgfusepath{clip}%
\pgfsetroundcap%
\pgfsetroundjoin%
\definecolor{currentfill}{rgb}{0.257322,0.256130,0.526563}%
\pgfsetfillcolor{currentfill}%
\pgfsetlinewidth{0.643711pt}%
\definecolor{currentstroke}{rgb}{0.257322,0.256130,0.526563}%
\pgfsetstrokecolor{currentstroke}%
\pgfsetdash{}{0pt}%
\pgfpathmoveto{\pgfqpoint{4.860119in}{4.839171in}}%
\pgfpathlineto{\pgfqpoint{4.889896in}{4.893681in}}%
\pgfpathlineto{\pgfqpoint{4.915638in}{4.837154in}}%
\pgfpathlineto{\pgfqpoint{4.860119in}{4.839171in}}%
\pgfpathlineto{\pgfqpoint{4.860119in}{4.839171in}}%
\pgfpathclose%
\pgfusepath{stroke,fill}%
\end{pgfscope}%
\begin{pgfscope}%
\pgfpathrectangle{\pgfqpoint{3.985294in}{4.155455in}}{\pgfqpoint{2.279412in}{2.004545in}}%
\pgfusepath{clip}%
\pgfsetroundcap%
\pgfsetroundjoin%
\pgfsetlinewidth{0.858018pt}%
\definecolor{currentstroke}{rgb}{0.197636,0.391528,0.554969}%
\pgfsetstrokecolor{currentstroke}%
\pgfsetdash{}{0pt}%
\pgfpathmoveto{\pgfqpoint{5.181062in}{4.949577in}}%
\pgfpathquadraticcurveto{\pgfqpoint{5.168742in}{4.951609in}}{\pgfqpoint{5.169517in}{4.951481in}}%
\pgfusepath{stroke}%
\end{pgfscope}%
\begin{pgfscope}%
\pgfpathrectangle{\pgfqpoint{3.985294in}{4.155455in}}{\pgfqpoint{2.279412in}{2.004545in}}%
\pgfusepath{clip}%
\pgfsetroundcap%
\pgfsetroundjoin%
\definecolor{currentfill}{rgb}{0.197636,0.391528,0.554969}%
\pgfsetfillcolor{currentfill}%
\pgfsetlinewidth{0.858018pt}%
\definecolor{currentstroke}{rgb}{0.197636,0.391528,0.554969}%
\pgfsetstrokecolor{currentstroke}%
\pgfsetdash{}{0pt}%
\pgfpathmoveto{\pgfqpoint{5.219814in}{4.915035in}}%
\pgfpathlineto{\pgfqpoint{5.169517in}{4.951481in}}%
\pgfpathlineto{\pgfqpoint{5.228852in}{4.969850in}}%
\pgfpathlineto{\pgfqpoint{5.219814in}{4.915035in}}%
\pgfpathlineto{\pgfqpoint{5.219814in}{4.915035in}}%
\pgfpathclose%
\pgfusepath{stroke,fill}%
\end{pgfscope}%
\begin{pgfscope}%
\pgfpathrectangle{\pgfqpoint{3.985294in}{4.155455in}}{\pgfqpoint{2.279412in}{2.004545in}}%
\pgfusepath{clip}%
\pgfsetroundcap%
\pgfsetroundjoin%
\pgfsetlinewidth{1.026264pt}%
\definecolor{currentstroke}{rgb}{0.159194,0.482237,0.558073}%
\pgfsetstrokecolor{currentstroke}%
\pgfsetdash{}{0pt}%
\pgfpathmoveto{\pgfqpoint{5.057597in}{5.009929in}}%
\pgfpathquadraticcurveto{\pgfqpoint{5.045486in}{5.012768in}}{\pgfqpoint{5.048833in}{5.011983in}}%
\pgfusepath{stroke}%
\end{pgfscope}%
\begin{pgfscope}%
\pgfpathrectangle{\pgfqpoint{3.985294in}{4.155455in}}{\pgfqpoint{2.279412in}{2.004545in}}%
\pgfusepath{clip}%
\pgfsetroundcap%
\pgfsetroundjoin%
\definecolor{currentfill}{rgb}{0.159194,0.482237,0.558073}%
\pgfsetfillcolor{currentfill}%
\pgfsetlinewidth{1.026264pt}%
\definecolor{currentstroke}{rgb}{0.159194,0.482237,0.558073}%
\pgfsetstrokecolor{currentstroke}%
\pgfsetdash{}{0pt}%
\pgfpathmoveto{\pgfqpoint{5.096584in}{4.972260in}}%
\pgfpathlineto{\pgfqpoint{5.048833in}{5.011983in}}%
\pgfpathlineto{\pgfqpoint{5.109262in}{5.026350in}}%
\pgfpathlineto{\pgfqpoint{5.096584in}{4.972260in}}%
\pgfpathlineto{\pgfqpoint{5.096584in}{4.972260in}}%
\pgfpathclose%
\pgfusepath{stroke,fill}%
\end{pgfscope}%
\begin{pgfscope}%
\pgfpathrectangle{\pgfqpoint{3.985294in}{4.155455in}}{\pgfqpoint{2.279412in}{2.004545in}}%
\pgfusepath{clip}%
\pgfsetbuttcap%
\pgfsetroundjoin%
\pgfsetlinewidth{1.505625pt}%
\definecolor{currentstroke}{rgb}{0.000000,0.000000,0.000000}%
\pgfsetstrokecolor{currentstroke}%
\pgfsetdash{}{0pt}%
\pgfpathmoveto{\pgfqpoint{4.778678in}{4.494895in}}%
\pgfpathlineto{\pgfqpoint{4.778678in}{5.820559in}}%
\pgfusepath{stroke}%
\end{pgfscope}%
\begin{pgfscope}%
\pgfpathrectangle{\pgfqpoint{3.985294in}{4.155455in}}{\pgfqpoint{2.279412in}{2.004545in}}%
\pgfusepath{clip}%
\pgfsetbuttcap%
\pgfsetroundjoin%
\pgfsetlinewidth{1.505625pt}%
\definecolor{currentstroke}{rgb}{0.000000,0.000000,0.000000}%
\pgfsetstrokecolor{currentstroke}%
\pgfsetdash{}{0pt}%
\pgfpathmoveto{\pgfqpoint{5.927204in}{4.494895in}}%
\pgfpathlineto{\pgfqpoint{5.927204in}{5.820559in}}%
\pgfusepath{stroke}%
\end{pgfscope}%
\begin{pgfscope}%
\pgfsetrectcap%
\pgfsetmiterjoin%
\pgfsetlinewidth{0.803000pt}%
\definecolor{currentstroke}{rgb}{0.000000,0.000000,0.000000}%
\pgfsetstrokecolor{currentstroke}%
\pgfsetdash{}{0pt}%
\pgfpathmoveto{\pgfqpoint{3.985294in}{4.155455in}}%
\pgfpathlineto{\pgfqpoint{3.985294in}{6.160000in}}%
\pgfusepath{stroke}%
\end{pgfscope}%
\begin{pgfscope}%
\pgfsetrectcap%
\pgfsetmiterjoin%
\pgfsetlinewidth{0.803000pt}%
\definecolor{currentstroke}{rgb}{0.000000,0.000000,0.000000}%
\pgfsetstrokecolor{currentstroke}%
\pgfsetdash{}{0pt}%
\pgfpathmoveto{\pgfqpoint{6.264706in}{4.155455in}}%
\pgfpathlineto{\pgfqpoint{6.264706in}{6.160000in}}%
\pgfusepath{stroke}%
\end{pgfscope}%
\begin{pgfscope}%
\pgfsetrectcap%
\pgfsetmiterjoin%
\pgfsetlinewidth{0.803000pt}%
\definecolor{currentstroke}{rgb}{0.000000,0.000000,0.000000}%
\pgfsetstrokecolor{currentstroke}%
\pgfsetdash{}{0pt}%
\pgfpathmoveto{\pgfqpoint{3.985294in}{4.155455in}}%
\pgfpathlineto{\pgfqpoint{6.264706in}{4.155455in}}%
\pgfusepath{stroke}%
\end{pgfscope}%
\begin{pgfscope}%
\pgfsetrectcap%
\pgfsetmiterjoin%
\pgfsetlinewidth{0.803000pt}%
\definecolor{currentstroke}{rgb}{0.000000,0.000000,0.000000}%
\pgfsetstrokecolor{currentstroke}%
\pgfsetdash{}{0pt}%
\pgfpathmoveto{\pgfqpoint{3.985294in}{6.160000in}}%
\pgfpathlineto{\pgfqpoint{6.264706in}{6.160000in}}%
\pgfusepath{stroke}%
\end{pgfscope}%
\begin{pgfscope}%
\definecolor{textcolor}{rgb}{0.000000,0.000000,0.000000}%
\pgfsetstrokecolor{textcolor}%
\pgfsetfillcolor{textcolor}%
\pgftext[x=5.125000in,y=6.243333in,,base]{\color{textcolor}\sffamily\fontsize{12.000000}{14.400000}\selectfont b)}%
\end{pgfscope}%
\begin{pgfscope}%
\pgfsetbuttcap%
\pgfsetmiterjoin%
\definecolor{currentfill}{rgb}{1.000000,1.000000,1.000000}%
\pgfsetfillcolor{currentfill}%
\pgfsetlinewidth{0.000000pt}%
\definecolor{currentstroke}{rgb}{0.000000,0.000000,0.000000}%
\pgfsetstrokecolor{currentstroke}%
\pgfsetstrokeopacity{0.000000}%
\pgfsetdash{}{0pt}%
\pgfpathmoveto{\pgfqpoint{6.720588in}{4.155455in}}%
\pgfpathlineto{\pgfqpoint{9.000000in}{4.155455in}}%
\pgfpathlineto{\pgfqpoint{9.000000in}{6.160000in}}%
\pgfpathlineto{\pgfqpoint{6.720588in}{6.160000in}}%
\pgfpathlineto{\pgfqpoint{6.720588in}{4.155455in}}%
\pgfpathclose%
\pgfusepath{fill}%
\end{pgfscope}%
\begin{pgfscope}%
\pgfpathrectangle{\pgfqpoint{6.720588in}{4.155455in}}{\pgfqpoint{2.279412in}{2.004545in}}%
\pgfusepath{clip}%
\pgfsys@transformcm{2.291667}{0.000000}{0.000000}{2.013889}{6.720588in}{4.155455in}%
\pgftext[left,bottom]{\includegraphics[interpolate=false,width=1.000000in,height=1.000000in]{q_series-img2.png}}%
\end{pgfscope}%
\begin{pgfscope}%
\pgfsetbuttcap%
\pgfsetroundjoin%
\definecolor{currentfill}{rgb}{0.000000,0.000000,0.000000}%
\pgfsetfillcolor{currentfill}%
\pgfsetlinewidth{0.803000pt}%
\definecolor{currentstroke}{rgb}{0.000000,0.000000,0.000000}%
\pgfsetstrokecolor{currentstroke}%
\pgfsetdash{}{0pt}%
\pgfsys@defobject{currentmarker}{\pgfqpoint{0.000000in}{-0.048611in}}{\pgfqpoint{0.000000in}{0.000000in}}{%
\pgfpathmoveto{\pgfqpoint{0.000000in}{0.000000in}}%
\pgfpathlineto{\pgfqpoint{0.000000in}{-0.048611in}}%
\pgfusepath{stroke,fill}%
}%
\begin{pgfscope}%
\pgfsys@transformshift{7.131130in}{4.155455in}%
\pgfsys@useobject{currentmarker}{}%
\end{pgfscope}%
\end{pgfscope}%
\begin{pgfscope}%
\pgfsetbuttcap%
\pgfsetroundjoin%
\definecolor{currentfill}{rgb}{0.000000,0.000000,0.000000}%
\pgfsetfillcolor{currentfill}%
\pgfsetlinewidth{0.803000pt}%
\definecolor{currentstroke}{rgb}{0.000000,0.000000,0.000000}%
\pgfsetstrokecolor{currentstroke}%
\pgfsetdash{}{0pt}%
\pgfsys@defobject{currentmarker}{\pgfqpoint{0.000000in}{-0.048611in}}{\pgfqpoint{0.000000in}{0.000000in}}{%
\pgfpathmoveto{\pgfqpoint{0.000000in}{0.000000in}}%
\pgfpathlineto{\pgfqpoint{0.000000in}{-0.048611in}}%
\pgfusepath{stroke,fill}%
}%
\begin{pgfscope}%
\pgfsys@transformshift{7.609683in}{4.155455in}%
\pgfsys@useobject{currentmarker}{}%
\end{pgfscope}%
\end{pgfscope}%
\begin{pgfscope}%
\pgfsetbuttcap%
\pgfsetroundjoin%
\definecolor{currentfill}{rgb}{0.000000,0.000000,0.000000}%
\pgfsetfillcolor{currentfill}%
\pgfsetlinewidth{0.803000pt}%
\definecolor{currentstroke}{rgb}{0.000000,0.000000,0.000000}%
\pgfsetstrokecolor{currentstroke}%
\pgfsetdash{}{0pt}%
\pgfsys@defobject{currentmarker}{\pgfqpoint{0.000000in}{-0.048611in}}{\pgfqpoint{0.000000in}{0.000000in}}{%
\pgfpathmoveto{\pgfqpoint{0.000000in}{0.000000in}}%
\pgfpathlineto{\pgfqpoint{0.000000in}{-0.048611in}}%
\pgfusepath{stroke,fill}%
}%
\begin{pgfscope}%
\pgfsys@transformshift{8.088235in}{4.155455in}%
\pgfsys@useobject{currentmarker}{}%
\end{pgfscope}%
\end{pgfscope}%
\begin{pgfscope}%
\pgfsetbuttcap%
\pgfsetroundjoin%
\definecolor{currentfill}{rgb}{0.000000,0.000000,0.000000}%
\pgfsetfillcolor{currentfill}%
\pgfsetlinewidth{0.803000pt}%
\definecolor{currentstroke}{rgb}{0.000000,0.000000,0.000000}%
\pgfsetstrokecolor{currentstroke}%
\pgfsetdash{}{0pt}%
\pgfsys@defobject{currentmarker}{\pgfqpoint{0.000000in}{-0.048611in}}{\pgfqpoint{0.000000in}{0.000000in}}{%
\pgfpathmoveto{\pgfqpoint{0.000000in}{0.000000in}}%
\pgfpathlineto{\pgfqpoint{0.000000in}{-0.048611in}}%
\pgfusepath{stroke,fill}%
}%
\begin{pgfscope}%
\pgfsys@transformshift{8.566788in}{4.155455in}%
\pgfsys@useobject{currentmarker}{}%
\end{pgfscope}%
\end{pgfscope}%
\begin{pgfscope}%
\definecolor{textcolor}{rgb}{0.000000,0.000000,0.000000}%
\pgfsetstrokecolor{textcolor}%
\pgfsetfillcolor{textcolor}%
\pgftext[x=7.860294in,y=4.099899in,,top]{\color{textcolor}\sffamily\fontsize{10.000000}{12.000000}\selectfont \(\displaystyle \zeta \, \mathrm{[\mu m]}\)}%
\end{pgfscope}%
\begin{pgfscope}%
\pgfsetbuttcap%
\pgfsetroundjoin%
\definecolor{currentfill}{rgb}{0.000000,0.000000,0.000000}%
\pgfsetfillcolor{currentfill}%
\pgfsetlinewidth{0.803000pt}%
\definecolor{currentstroke}{rgb}{0.000000,0.000000,0.000000}%
\pgfsetstrokecolor{currentstroke}%
\pgfsetdash{}{0pt}%
\pgfsys@defobject{currentmarker}{\pgfqpoint{-0.048611in}{0.000000in}}{\pgfqpoint{-0.000000in}{0.000000in}}{%
\pgfpathmoveto{\pgfqpoint{-0.000000in}{0.000000in}}%
\pgfpathlineto{\pgfqpoint{-0.048611in}{0.000000in}}%
\pgfusepath{stroke,fill}%
}%
\begin{pgfscope}%
\pgfsys@transformshift{6.720588in}{4.163479in}%
\pgfsys@useobject{currentmarker}{}%
\end{pgfscope}%
\end{pgfscope}%
\begin{pgfscope}%
\pgfsetbuttcap%
\pgfsetroundjoin%
\definecolor{currentfill}{rgb}{0.000000,0.000000,0.000000}%
\pgfsetfillcolor{currentfill}%
\pgfsetlinewidth{0.803000pt}%
\definecolor{currentstroke}{rgb}{0.000000,0.000000,0.000000}%
\pgfsetstrokecolor{currentstroke}%
\pgfsetdash{}{0pt}%
\pgfsys@defobject{currentmarker}{\pgfqpoint{-0.048611in}{0.000000in}}{\pgfqpoint{-0.000000in}{0.000000in}}{%
\pgfpathmoveto{\pgfqpoint{-0.000000in}{0.000000in}}%
\pgfpathlineto{\pgfqpoint{-0.048611in}{0.000000in}}%
\pgfusepath{stroke,fill}%
}%
\begin{pgfscope}%
\pgfsys@transformshift{6.720588in}{4.494895in}%
\pgfsys@useobject{currentmarker}{}%
\end{pgfscope}%
\end{pgfscope}%
\begin{pgfscope}%
\pgfsetbuttcap%
\pgfsetroundjoin%
\definecolor{currentfill}{rgb}{0.000000,0.000000,0.000000}%
\pgfsetfillcolor{currentfill}%
\pgfsetlinewidth{0.803000pt}%
\definecolor{currentstroke}{rgb}{0.000000,0.000000,0.000000}%
\pgfsetstrokecolor{currentstroke}%
\pgfsetdash{}{0pt}%
\pgfsys@defobject{currentmarker}{\pgfqpoint{-0.048611in}{0.000000in}}{\pgfqpoint{-0.000000in}{0.000000in}}{%
\pgfpathmoveto{\pgfqpoint{-0.000000in}{0.000000in}}%
\pgfpathlineto{\pgfqpoint{-0.048611in}{0.000000in}}%
\pgfusepath{stroke,fill}%
}%
\begin{pgfscope}%
\pgfsys@transformshift{6.720588in}{4.826311in}%
\pgfsys@useobject{currentmarker}{}%
\end{pgfscope}%
\end{pgfscope}%
\begin{pgfscope}%
\pgfsetbuttcap%
\pgfsetroundjoin%
\definecolor{currentfill}{rgb}{0.000000,0.000000,0.000000}%
\pgfsetfillcolor{currentfill}%
\pgfsetlinewidth{0.803000pt}%
\definecolor{currentstroke}{rgb}{0.000000,0.000000,0.000000}%
\pgfsetstrokecolor{currentstroke}%
\pgfsetdash{}{0pt}%
\pgfsys@defobject{currentmarker}{\pgfqpoint{-0.048611in}{0.000000in}}{\pgfqpoint{-0.000000in}{0.000000in}}{%
\pgfpathmoveto{\pgfqpoint{-0.000000in}{0.000000in}}%
\pgfpathlineto{\pgfqpoint{-0.048611in}{0.000000in}}%
\pgfusepath{stroke,fill}%
}%
\begin{pgfscope}%
\pgfsys@transformshift{6.720588in}{5.157727in}%
\pgfsys@useobject{currentmarker}{}%
\end{pgfscope}%
\end{pgfscope}%
\begin{pgfscope}%
\pgfsetbuttcap%
\pgfsetroundjoin%
\definecolor{currentfill}{rgb}{0.000000,0.000000,0.000000}%
\pgfsetfillcolor{currentfill}%
\pgfsetlinewidth{0.803000pt}%
\definecolor{currentstroke}{rgb}{0.000000,0.000000,0.000000}%
\pgfsetstrokecolor{currentstroke}%
\pgfsetdash{}{0pt}%
\pgfsys@defobject{currentmarker}{\pgfqpoint{-0.048611in}{0.000000in}}{\pgfqpoint{-0.000000in}{0.000000in}}{%
\pgfpathmoveto{\pgfqpoint{-0.000000in}{0.000000in}}%
\pgfpathlineto{\pgfqpoint{-0.048611in}{0.000000in}}%
\pgfusepath{stroke,fill}%
}%
\begin{pgfscope}%
\pgfsys@transformshift{6.720588in}{5.489143in}%
\pgfsys@useobject{currentmarker}{}%
\end{pgfscope}%
\end{pgfscope}%
\begin{pgfscope}%
\pgfsetbuttcap%
\pgfsetroundjoin%
\definecolor{currentfill}{rgb}{0.000000,0.000000,0.000000}%
\pgfsetfillcolor{currentfill}%
\pgfsetlinewidth{0.803000pt}%
\definecolor{currentstroke}{rgb}{0.000000,0.000000,0.000000}%
\pgfsetstrokecolor{currentstroke}%
\pgfsetdash{}{0pt}%
\pgfsys@defobject{currentmarker}{\pgfqpoint{-0.048611in}{0.000000in}}{\pgfqpoint{-0.000000in}{0.000000in}}{%
\pgfpathmoveto{\pgfqpoint{-0.000000in}{0.000000in}}%
\pgfpathlineto{\pgfqpoint{-0.048611in}{0.000000in}}%
\pgfusepath{stroke,fill}%
}%
\begin{pgfscope}%
\pgfsys@transformshift{6.720588in}{5.820559in}%
\pgfsys@useobject{currentmarker}{}%
\end{pgfscope}%
\end{pgfscope}%
\begin{pgfscope}%
\pgfsetbuttcap%
\pgfsetroundjoin%
\definecolor{currentfill}{rgb}{0.000000,0.000000,0.000000}%
\pgfsetfillcolor{currentfill}%
\pgfsetlinewidth{0.803000pt}%
\definecolor{currentstroke}{rgb}{0.000000,0.000000,0.000000}%
\pgfsetstrokecolor{currentstroke}%
\pgfsetdash{}{0pt}%
\pgfsys@defobject{currentmarker}{\pgfqpoint{-0.048611in}{0.000000in}}{\pgfqpoint{-0.000000in}{0.000000in}}{%
\pgfpathmoveto{\pgfqpoint{-0.000000in}{0.000000in}}%
\pgfpathlineto{\pgfqpoint{-0.048611in}{0.000000in}}%
\pgfusepath{stroke,fill}%
}%
\begin{pgfscope}%
\pgfsys@transformshift{6.720588in}{6.151975in}%
\pgfsys@useobject{currentmarker}{}%
\end{pgfscope}%
\end{pgfscope}%
\begin{pgfscope}%
\definecolor{textcolor}{rgb}{0.000000,0.000000,0.000000}%
\pgfsetstrokecolor{textcolor}%
\pgfsetfillcolor{textcolor}%
\pgftext[x=6.665033in,y=5.157727in,,bottom,rotate=90.000000]{\color{textcolor}\sffamily\fontsize{10.000000}{12.000000}\selectfont \(\displaystyle z \, \mathrm{[\mu m]}\)}%
\end{pgfscope}%
\begin{pgfscope}%
\pgfpathrectangle{\pgfqpoint{6.720588in}{4.155455in}}{\pgfqpoint{2.279412in}{2.004545in}}%
\pgfusepath{clip}%
\pgfsetbuttcap%
\pgfsetroundjoin%
\pgfsetlinewidth{0.320723pt}%
\definecolor{currentstroke}{rgb}{0.269944,0.014625,0.341379}%
\pgfsetstrokecolor{currentstroke}%
\pgfsetdash{}{0pt}%
\pgfpathmoveto{\pgfqpoint{8.732256in}{4.977300in}}%
\pgfpathlineto{\pgfqpoint{8.683755in}{4.977786in}}%
\pgfusepath{stroke}%
\end{pgfscope}%
\begin{pgfscope}%
\pgfpathrectangle{\pgfqpoint{6.720588in}{4.155455in}}{\pgfqpoint{2.279412in}{2.004545in}}%
\pgfusepath{clip}%
\pgfsetbuttcap%
\pgfsetroundjoin%
\pgfsetlinewidth{0.321799pt}%
\definecolor{currentstroke}{rgb}{0.271305,0.019942,0.347269}%
\pgfsetstrokecolor{currentstroke}%
\pgfsetdash{}{0pt}%
\pgfpathmoveto{\pgfqpoint{8.683755in}{4.977786in}}%
\pgfpathlineto{\pgfqpoint{8.633756in}{4.979149in}}%
\pgfusepath{stroke}%
\end{pgfscope}%
\begin{pgfscope}%
\pgfpathrectangle{\pgfqpoint{6.720588in}{4.155455in}}{\pgfqpoint{2.279412in}{2.004545in}}%
\pgfusepath{clip}%
\pgfsetbuttcap%
\pgfsetroundjoin%
\pgfsetlinewidth{0.323030pt}%
\definecolor{currentstroke}{rgb}{0.271305,0.019942,0.347269}%
\pgfsetstrokecolor{currentstroke}%
\pgfsetdash{}{0pt}%
\pgfpathmoveto{\pgfqpoint{8.633756in}{4.979149in}}%
\pgfpathlineto{\pgfqpoint{8.583829in}{4.981575in}}%
\pgfusepath{stroke}%
\end{pgfscope}%
\begin{pgfscope}%
\pgfpathrectangle{\pgfqpoint{6.720588in}{4.155455in}}{\pgfqpoint{2.279412in}{2.004545in}}%
\pgfusepath{clip}%
\pgfsetbuttcap%
\pgfsetroundjoin%
\pgfsetlinewidth{0.340082pt}%
\definecolor{currentstroke}{rgb}{0.273809,0.031497,0.358853}%
\pgfsetstrokecolor{currentstroke}%
\pgfsetdash{}{0pt}%
\pgfpathmoveto{\pgfqpoint{8.583829in}{4.981575in}}%
\pgfpathlineto{\pgfqpoint{8.533778in}{4.984069in}}%
\pgfusepath{stroke}%
\end{pgfscope}%
\begin{pgfscope}%
\pgfpathrectangle{\pgfqpoint{6.720588in}{4.155455in}}{\pgfqpoint{2.279412in}{2.004545in}}%
\pgfusepath{clip}%
\pgfsetbuttcap%
\pgfsetroundjoin%
\pgfsetlinewidth{0.342932pt}%
\definecolor{currentstroke}{rgb}{0.274952,0.037752,0.364543}%
\pgfsetstrokecolor{currentstroke}%
\pgfsetdash{}{0pt}%
\pgfpathmoveto{\pgfqpoint{8.533778in}{4.984069in}}%
\pgfpathlineto{\pgfqpoint{8.483672in}{4.985818in}}%
\pgfusepath{stroke}%
\end{pgfscope}%
\begin{pgfscope}%
\pgfpathrectangle{\pgfqpoint{6.720588in}{4.155455in}}{\pgfqpoint{2.279412in}{2.004545in}}%
\pgfusepath{clip}%
\pgfsetbuttcap%
\pgfsetroundjoin%
\pgfsetlinewidth{0.364065pt}%
\definecolor{currentstroke}{rgb}{0.277941,0.056324,0.381191}%
\pgfsetstrokecolor{currentstroke}%
\pgfsetdash{}{0pt}%
\pgfpathmoveto{\pgfqpoint{8.483672in}{4.985818in}}%
\pgfpathlineto{\pgfqpoint{8.433564in}{4.987502in}}%
\pgfusepath{stroke}%
\end{pgfscope}%
\begin{pgfscope}%
\pgfpathrectangle{\pgfqpoint{6.720588in}{4.155455in}}{\pgfqpoint{2.279412in}{2.004545in}}%
\pgfusepath{clip}%
\pgfsetbuttcap%
\pgfsetroundjoin%
\pgfsetlinewidth{0.385518pt}%
\definecolor{currentstroke}{rgb}{0.280267,0.073417,0.397163}%
\pgfsetstrokecolor{currentstroke}%
\pgfsetdash{}{0pt}%
\pgfpathmoveto{\pgfqpoint{8.433564in}{4.987502in}}%
\pgfpathlineto{\pgfqpoint{8.383452in}{4.989203in}}%
\pgfusepath{stroke}%
\end{pgfscope}%
\begin{pgfscope}%
\pgfpathrectangle{\pgfqpoint{6.720588in}{4.155455in}}{\pgfqpoint{2.279412in}{2.004545in}}%
\pgfusepath{clip}%
\pgfsetbuttcap%
\pgfsetroundjoin%
\pgfsetlinewidth{0.432187pt}%
\definecolor{currentstroke}{rgb}{0.283091,0.110553,0.431554}%
\pgfsetstrokecolor{currentstroke}%
\pgfsetdash{}{0pt}%
\pgfpathmoveto{\pgfqpoint{8.383452in}{4.989203in}}%
\pgfpathlineto{\pgfqpoint{8.333344in}{4.990992in}}%
\pgfusepath{stroke}%
\end{pgfscope}%
\begin{pgfscope}%
\pgfpathrectangle{\pgfqpoint{6.720588in}{4.155455in}}{\pgfqpoint{2.279412in}{2.004545in}}%
\pgfusepath{clip}%
\pgfsetbuttcap%
\pgfsetroundjoin%
\pgfsetlinewidth{0.472810pt}%
\definecolor{currentstroke}{rgb}{0.282623,0.140926,0.457517}%
\pgfsetstrokecolor{currentstroke}%
\pgfsetdash{}{0pt}%
\pgfpathmoveto{\pgfqpoint{8.333344in}{4.990992in}}%
\pgfpathlineto{\pgfqpoint{8.283238in}{4.992872in}}%
\pgfusepath{stroke}%
\end{pgfscope}%
\begin{pgfscope}%
\pgfpathrectangle{\pgfqpoint{6.720588in}{4.155455in}}{\pgfqpoint{2.279412in}{2.004545in}}%
\pgfusepath{clip}%
\pgfsetbuttcap%
\pgfsetroundjoin%
\pgfsetlinewidth{0.549242pt}%
\definecolor{currentstroke}{rgb}{0.275191,0.194905,0.496005}%
\pgfsetstrokecolor{currentstroke}%
\pgfsetdash{}{0pt}%
\pgfpathmoveto{\pgfqpoint{8.283238in}{4.992872in}}%
\pgfpathlineto{\pgfqpoint{8.233144in}{4.994989in}}%
\pgfusepath{stroke}%
\end{pgfscope}%
\begin{pgfscope}%
\pgfpathrectangle{\pgfqpoint{6.720588in}{4.155455in}}{\pgfqpoint{2.279412in}{2.004545in}}%
\pgfusepath{clip}%
\pgfsetbuttcap%
\pgfsetroundjoin%
\pgfsetlinewidth{0.593635pt}%
\definecolor{currentstroke}{rgb}{0.267968,0.223549,0.512008}%
\pgfsetstrokecolor{currentstroke}%
\pgfsetdash{}{0pt}%
\pgfpathmoveto{\pgfqpoint{8.233144in}{4.994989in}}%
\pgfpathlineto{\pgfqpoint{8.183069in}{4.997416in}}%
\pgfusepath{stroke}%
\end{pgfscope}%
\begin{pgfscope}%
\pgfpathrectangle{\pgfqpoint{6.720588in}{4.155455in}}{\pgfqpoint{2.279412in}{2.004545in}}%
\pgfusepath{clip}%
\pgfsetbuttcap%
\pgfsetroundjoin%
\pgfsetlinewidth{0.699427pt}%
\definecolor{currentstroke}{rgb}{0.241237,0.296485,0.539709}%
\pgfsetstrokecolor{currentstroke}%
\pgfsetdash{}{0pt}%
\pgfpathmoveto{\pgfqpoint{8.183069in}{4.997416in}}%
\pgfpathlineto{\pgfqpoint{8.133027in}{5.000317in}}%
\pgfusepath{stroke}%
\end{pgfscope}%
\begin{pgfscope}%
\pgfpathrectangle{\pgfqpoint{6.720588in}{4.155455in}}{\pgfqpoint{2.279412in}{2.004545in}}%
\pgfusepath{clip}%
\pgfsetbuttcap%
\pgfsetroundjoin%
\pgfsetlinewidth{0.807666pt}%
\definecolor{currentstroke}{rgb}{0.212395,0.359683,0.551710}%
\pgfsetstrokecolor{currentstroke}%
\pgfsetdash{}{0pt}%
\pgfpathmoveto{\pgfqpoint{8.133027in}{5.000317in}}%
\pgfpathlineto{\pgfqpoint{8.083044in}{5.003908in}}%
\pgfusepath{stroke}%
\end{pgfscope}%
\begin{pgfscope}%
\pgfpathrectangle{\pgfqpoint{6.720588in}{4.155455in}}{\pgfqpoint{2.279412in}{2.004545in}}%
\pgfusepath{clip}%
\pgfsetbuttcap%
\pgfsetroundjoin%
\pgfsetlinewidth{0.944019pt}%
\definecolor{currentstroke}{rgb}{0.177423,0.437527,0.557565}%
\pgfsetstrokecolor{currentstroke}%
\pgfsetdash{}{0pt}%
\pgfpathmoveto{\pgfqpoint{8.083044in}{5.003908in}}%
\pgfpathlineto{\pgfqpoint{8.033161in}{5.008423in}}%
\pgfusepath{stroke}%
\end{pgfscope}%
\begin{pgfscope}%
\pgfpathrectangle{\pgfqpoint{6.720588in}{4.155455in}}{\pgfqpoint{2.279412in}{2.004545in}}%
\pgfusepath{clip}%
\pgfsetbuttcap%
\pgfsetroundjoin%
\pgfsetlinewidth{1.130353pt}%
\definecolor{currentstroke}{rgb}{0.137770,0.537492,0.554906}%
\pgfsetstrokecolor{currentstroke}%
\pgfsetdash{}{0pt}%
\pgfpathmoveto{\pgfqpoint{8.033161in}{5.008423in}}%
\pgfpathlineto{\pgfqpoint{7.983451in}{5.014191in}}%
\pgfusepath{stroke}%
\end{pgfscope}%
\begin{pgfscope}%
\pgfpathrectangle{\pgfqpoint{6.720588in}{4.155455in}}{\pgfqpoint{2.279412in}{2.004545in}}%
\pgfusepath{clip}%
\pgfsetbuttcap%
\pgfsetroundjoin%
\pgfsetlinewidth{1.575947pt}%
\definecolor{currentstroke}{rgb}{0.311925,0.767822,0.415586}%
\pgfsetstrokecolor{currentstroke}%
\pgfsetdash{}{0pt}%
\pgfpathmoveto{\pgfqpoint{7.983451in}{5.014191in}}%
\pgfpathlineto{\pgfqpoint{7.933954in}{5.021264in}}%
\pgfusepath{stroke}%
\end{pgfscope}%
\begin{pgfscope}%
\pgfpathrectangle{\pgfqpoint{6.720588in}{4.155455in}}{\pgfqpoint{2.279412in}{2.004545in}}%
\pgfusepath{clip}%
\pgfsetbuttcap%
\pgfsetroundjoin%
\pgfsetlinewidth{1.777832pt}%
\definecolor{currentstroke}{rgb}{0.585678,0.846661,0.249897}%
\pgfsetstrokecolor{currentstroke}%
\pgfsetdash{}{0pt}%
\pgfpathmoveto{\pgfqpoint{7.933954in}{5.021264in}}%
\pgfpathlineto{\pgfqpoint{7.884639in}{5.029253in}}%
\pgfusepath{stroke}%
\end{pgfscope}%
\begin{pgfscope}%
\pgfpathrectangle{\pgfqpoint{6.720588in}{4.155455in}}{\pgfqpoint{2.279412in}{2.004545in}}%
\pgfusepath{clip}%
\pgfsetbuttcap%
\pgfsetroundjoin%
\pgfsetlinewidth{1.951114pt}%
\definecolor{currentstroke}{rgb}{0.845561,0.887322,0.099702}%
\pgfsetstrokecolor{currentstroke}%
\pgfsetdash{}{0pt}%
\pgfpathmoveto{\pgfqpoint{7.884639in}{5.029253in}}%
\pgfpathlineto{\pgfqpoint{7.835474in}{5.037929in}}%
\pgfusepath{stroke}%
\end{pgfscope}%
\begin{pgfscope}%
\pgfpathrectangle{\pgfqpoint{6.720588in}{4.155455in}}{\pgfqpoint{2.279412in}{2.004545in}}%
\pgfusepath{clip}%
\pgfsetbuttcap%
\pgfsetroundjoin%
\pgfsetlinewidth{2.010161pt}%
\definecolor{currentstroke}{rgb}{0.935904,0.898570,0.108131}%
\pgfsetstrokecolor{currentstroke}%
\pgfsetdash{}{0pt}%
\pgfpathmoveto{\pgfqpoint{7.835474in}{5.037929in}}%
\pgfpathlineto{\pgfqpoint{7.786508in}{5.047411in}}%
\pgfusepath{stroke}%
\end{pgfscope}%
\begin{pgfscope}%
\pgfpathrectangle{\pgfqpoint{6.720588in}{4.155455in}}{\pgfqpoint{2.279412in}{2.004545in}}%
\pgfusepath{clip}%
\pgfsetbuttcap%
\pgfsetroundjoin%
\pgfsetlinewidth{2.020870pt}%
\definecolor{currentstroke}{rgb}{0.945636,0.899815,0.112838}%
\pgfsetstrokecolor{currentstroke}%
\pgfsetdash{}{0pt}%
\pgfpathmoveto{\pgfqpoint{7.786508in}{5.047411in}}%
\pgfpathlineto{\pgfqpoint{7.737731in}{5.057626in}}%
\pgfusepath{stroke}%
\end{pgfscope}%
\begin{pgfscope}%
\pgfpathrectangle{\pgfqpoint{6.720588in}{4.155455in}}{\pgfqpoint{2.279412in}{2.004545in}}%
\pgfusepath{clip}%
\pgfsetbuttcap%
\pgfsetroundjoin%
\pgfsetlinewidth{2.085110pt}%
\definecolor{currentstroke}{rgb}{0.993248,0.906157,0.143936}%
\pgfsetstrokecolor{currentstroke}%
\pgfsetdash{}{0pt}%
\pgfpathmoveto{\pgfqpoint{7.737731in}{5.057626in}}%
\pgfpathlineto{\pgfqpoint{7.689066in}{5.068226in}}%
\pgfusepath{stroke}%
\end{pgfscope}%
\begin{pgfscope}%
\pgfpathrectangle{\pgfqpoint{6.720588in}{4.155455in}}{\pgfqpoint{2.279412in}{2.004545in}}%
\pgfusepath{clip}%
\pgfsetbuttcap%
\pgfsetroundjoin%
\pgfsetlinewidth{1.988105pt}%
\definecolor{currentstroke}{rgb}{0.896320,0.893616,0.096335}%
\pgfsetstrokecolor{currentstroke}%
\pgfsetdash{}{0pt}%
\pgfpathmoveto{\pgfqpoint{7.689066in}{5.068226in}}%
\pgfpathlineto{\pgfqpoint{7.640447in}{5.078985in}}%
\pgfusepath{stroke}%
\end{pgfscope}%
\begin{pgfscope}%
\pgfpathrectangle{\pgfqpoint{6.720588in}{4.155455in}}{\pgfqpoint{2.279412in}{2.004545in}}%
\pgfusepath{clip}%
\pgfsetbuttcap%
\pgfsetroundjoin%
\pgfsetlinewidth{2.129298pt}%
\definecolor{currentstroke}{rgb}{0.993248,0.906157,0.143936}%
\pgfsetstrokecolor{currentstroke}%
\pgfsetdash{}{0pt}%
\pgfpathmoveto{\pgfqpoint{7.640447in}{5.078985in}}%
\pgfpathlineto{\pgfqpoint{7.592004in}{5.090335in}}%
\pgfusepath{stroke}%
\end{pgfscope}%
\begin{pgfscope}%
\pgfpathrectangle{\pgfqpoint{6.720588in}{4.155455in}}{\pgfqpoint{2.279412in}{2.004545in}}%
\pgfusepath{clip}%
\pgfsetbuttcap%
\pgfsetroundjoin%
\pgfsetlinewidth{1.807633pt}%
\definecolor{currentstroke}{rgb}{0.626579,0.854645,0.223353}%
\pgfsetstrokecolor{currentstroke}%
\pgfsetdash{}{0pt}%
\pgfpathmoveto{\pgfqpoint{7.592004in}{5.090335in}}%
\pgfpathlineto{\pgfqpoint{7.543549in}{5.101600in}}%
\pgfusepath{stroke}%
\end{pgfscope}%
\begin{pgfscope}%
\pgfpathrectangle{\pgfqpoint{6.720588in}{4.155455in}}{\pgfqpoint{2.279412in}{2.004545in}}%
\pgfusepath{clip}%
\pgfsetbuttcap%
\pgfsetroundjoin%
\pgfsetlinewidth{1.833775pt}%
\definecolor{currentstroke}{rgb}{0.668054,0.861999,0.196293}%
\pgfsetstrokecolor{currentstroke}%
\pgfsetdash{}{0pt}%
\pgfpathmoveto{\pgfqpoint{7.543549in}{5.101600in}}%
\pgfpathlineto{\pgfqpoint{7.495109in}{5.112770in}}%
\pgfusepath{stroke}%
\end{pgfscope}%
\begin{pgfscope}%
\pgfpathrectangle{\pgfqpoint{6.720588in}{4.155455in}}{\pgfqpoint{2.279412in}{2.004545in}}%
\pgfusepath{clip}%
\pgfsetbuttcap%
\pgfsetroundjoin%
\pgfsetlinewidth{1.524042pt}%
\definecolor{currentstroke}{rgb}{0.259857,0.745492,0.444467}%
\pgfsetstrokecolor{currentstroke}%
\pgfsetdash{}{0pt}%
\pgfpathmoveto{\pgfqpoint{7.495109in}{5.112770in}}%
\pgfpathlineto{\pgfqpoint{7.446802in}{5.124220in}}%
\pgfusepath{stroke}%
\end{pgfscope}%
\begin{pgfscope}%
\pgfpathrectangle{\pgfqpoint{6.720588in}{4.155455in}}{\pgfqpoint{2.279412in}{2.004545in}}%
\pgfusepath{clip}%
\pgfsetbuttcap%
\pgfsetroundjoin%
\pgfsetlinewidth{1.380752pt}%
\definecolor{currentstroke}{rgb}{0.146616,0.673050,0.508936}%
\pgfsetstrokecolor{currentstroke}%
\pgfsetdash{}{0pt}%
\pgfpathmoveto{\pgfqpoint{7.446802in}{5.124220in}}%
\pgfpathlineto{\pgfqpoint{7.398396in}{5.134983in}}%
\pgfusepath{stroke}%
\end{pgfscope}%
\begin{pgfscope}%
\pgfpathrectangle{\pgfqpoint{6.720588in}{4.155455in}}{\pgfqpoint{2.279412in}{2.004545in}}%
\pgfusepath{clip}%
\pgfsetbuttcap%
\pgfsetroundjoin%
\pgfsetlinewidth{1.186528pt}%
\definecolor{currentstroke}{rgb}{0.126453,0.570633,0.549841}%
\pgfsetstrokecolor{currentstroke}%
\pgfsetdash{}{0pt}%
\pgfpathmoveto{\pgfqpoint{7.398396in}{5.134983in}}%
\pgfpathlineto{\pgfqpoint{7.349742in}{5.142316in}}%
\pgfusepath{stroke}%
\end{pgfscope}%
\begin{pgfscope}%
\pgfpathrectangle{\pgfqpoint{6.720588in}{4.155455in}}{\pgfqpoint{2.279412in}{2.004545in}}%
\pgfusepath{clip}%
\pgfsetbuttcap%
\pgfsetroundjoin%
\pgfsetlinewidth{0.311984pt}%
\definecolor{currentstroke}{rgb}{0.268510,0.009605,0.335427}%
\pgfsetstrokecolor{currentstroke}%
\pgfsetdash{}{0pt}%
\pgfpathmoveto{\pgfqpoint{8.732256in}{5.247941in}}%
\pgfpathlineto{\pgfqpoint{8.682175in}{5.247574in}}%
\pgfusepath{stroke}%
\end{pgfscope}%
\begin{pgfscope}%
\pgfpathrectangle{\pgfqpoint{6.720588in}{4.155455in}}{\pgfqpoint{2.279412in}{2.004545in}}%
\pgfusepath{clip}%
\pgfsetbuttcap%
\pgfsetroundjoin%
\pgfsetlinewidth{0.318798pt}%
\definecolor{currentstroke}{rgb}{0.269944,0.014625,0.341379}%
\pgfsetstrokecolor{currentstroke}%
\pgfsetdash{}{0pt}%
\pgfpathmoveto{\pgfqpoint{8.682175in}{5.247574in}}%
\pgfpathlineto{\pgfqpoint{8.632052in}{5.248322in}}%
\pgfusepath{stroke}%
\end{pgfscope}%
\begin{pgfscope}%
\pgfpathrectangle{\pgfqpoint{6.720588in}{4.155455in}}{\pgfqpoint{2.279412in}{2.004545in}}%
\pgfusepath{clip}%
\pgfsetbuttcap%
\pgfsetroundjoin%
\pgfsetlinewidth{0.330351pt}%
\definecolor{currentstroke}{rgb}{0.272594,0.025563,0.353093}%
\pgfsetstrokecolor{currentstroke}%
\pgfsetdash{}{0pt}%
\pgfpathmoveto{\pgfqpoint{8.632052in}{5.248322in}}%
\pgfpathlineto{\pgfqpoint{8.581920in}{5.248613in}}%
\pgfusepath{stroke}%
\end{pgfscope}%
\begin{pgfscope}%
\pgfpathrectangle{\pgfqpoint{6.720588in}{4.155455in}}{\pgfqpoint{2.279412in}{2.004545in}}%
\pgfusepath{clip}%
\pgfsetbuttcap%
\pgfsetroundjoin%
\pgfsetlinewidth{0.331845pt}%
\definecolor{currentstroke}{rgb}{0.272594,0.025563,0.353093}%
\pgfsetstrokecolor{currentstroke}%
\pgfsetdash{}{0pt}%
\pgfpathmoveto{\pgfqpoint{8.581920in}{5.248613in}}%
\pgfpathlineto{\pgfqpoint{8.531785in}{5.247899in}}%
\pgfusepath{stroke}%
\end{pgfscope}%
\begin{pgfscope}%
\pgfpathrectangle{\pgfqpoint{6.720588in}{4.155455in}}{\pgfqpoint{2.279412in}{2.004545in}}%
\pgfusepath{clip}%
\pgfsetbuttcap%
\pgfsetroundjoin%
\pgfsetlinewidth{0.347175pt}%
\definecolor{currentstroke}{rgb}{0.274952,0.037752,0.364543}%
\pgfsetstrokecolor{currentstroke}%
\pgfsetdash{}{0pt}%
\pgfpathmoveto{\pgfqpoint{8.531785in}{5.247899in}}%
\pgfpathlineto{\pgfqpoint{8.481642in}{5.247230in}}%
\pgfusepath{stroke}%
\end{pgfscope}%
\begin{pgfscope}%
\pgfpathrectangle{\pgfqpoint{6.720588in}{4.155455in}}{\pgfqpoint{2.279412in}{2.004545in}}%
\pgfusepath{clip}%
\pgfsetbuttcap%
\pgfsetroundjoin%
\pgfsetlinewidth{0.372784pt}%
\definecolor{currentstroke}{rgb}{0.278791,0.062145,0.386592}%
\pgfsetstrokecolor{currentstroke}%
\pgfsetdash{}{0pt}%
\pgfpathmoveto{\pgfqpoint{8.481642in}{5.247230in}}%
\pgfpathlineto{\pgfqpoint{8.431497in}{5.246739in}}%
\pgfusepath{stroke}%
\end{pgfscope}%
\begin{pgfscope}%
\pgfpathrectangle{\pgfqpoint{6.720588in}{4.155455in}}{\pgfqpoint{2.279412in}{2.004545in}}%
\pgfusepath{clip}%
\pgfsetbuttcap%
\pgfsetroundjoin%
\pgfsetlinewidth{0.392605pt}%
\definecolor{currentstroke}{rgb}{0.280894,0.078907,0.402329}%
\pgfsetstrokecolor{currentstroke}%
\pgfsetdash{}{0pt}%
\pgfpathmoveto{\pgfqpoint{8.431497in}{5.246739in}}%
\pgfpathlineto{\pgfqpoint{8.381354in}{5.246000in}}%
\pgfusepath{stroke}%
\end{pgfscope}%
\begin{pgfscope}%
\pgfpathrectangle{\pgfqpoint{6.720588in}{4.155455in}}{\pgfqpoint{2.279412in}{2.004545in}}%
\pgfusepath{clip}%
\pgfsetbuttcap%
\pgfsetroundjoin%
\pgfsetlinewidth{0.441200pt}%
\definecolor{currentstroke}{rgb}{0.283197,0.115680,0.436115}%
\pgfsetstrokecolor{currentstroke}%
\pgfsetdash{}{0pt}%
\pgfpathmoveto{\pgfqpoint{8.381354in}{5.246000in}}%
\pgfpathlineto{\pgfqpoint{8.331213in}{5.245086in}}%
\pgfusepath{stroke}%
\end{pgfscope}%
\begin{pgfscope}%
\pgfpathrectangle{\pgfqpoint{6.720588in}{4.155455in}}{\pgfqpoint{2.279412in}{2.004545in}}%
\pgfusepath{clip}%
\pgfsetbuttcap%
\pgfsetroundjoin%
\pgfsetlinewidth{0.494482pt}%
\definecolor{currentstroke}{rgb}{0.281412,0.155834,0.469201}%
\pgfsetstrokecolor{currentstroke}%
\pgfsetdash{}{0pt}%
\pgfpathmoveto{\pgfqpoint{8.331213in}{5.245086in}}%
\pgfpathlineto{\pgfqpoint{8.281075in}{5.244071in}}%
\pgfusepath{stroke}%
\end{pgfscope}%
\begin{pgfscope}%
\pgfpathrectangle{\pgfqpoint{6.720588in}{4.155455in}}{\pgfqpoint{2.279412in}{2.004545in}}%
\pgfusepath{clip}%
\pgfsetbuttcap%
\pgfsetroundjoin%
\pgfsetlinewidth{0.575201pt}%
\definecolor{currentstroke}{rgb}{0.271828,0.209303,0.504434}%
\pgfsetstrokecolor{currentstroke}%
\pgfsetdash{}{0pt}%
\pgfpathmoveto{\pgfqpoint{8.281075in}{5.244071in}}%
\pgfpathlineto{\pgfqpoint{8.230941in}{5.242908in}}%
\pgfusepath{stroke}%
\end{pgfscope}%
\begin{pgfscope}%
\pgfpathrectangle{\pgfqpoint{6.720588in}{4.155455in}}{\pgfqpoint{2.279412in}{2.004545in}}%
\pgfusepath{clip}%
\pgfsetbuttcap%
\pgfsetroundjoin%
\pgfsetlinewidth{0.692210pt}%
\definecolor{currentstroke}{rgb}{0.244972,0.287675,0.537260}%
\pgfsetstrokecolor{currentstroke}%
\pgfsetdash{}{0pt}%
\pgfpathmoveto{\pgfqpoint{8.230941in}{5.242908in}}%
\pgfpathlineto{\pgfqpoint{8.180817in}{5.241480in}}%
\pgfusepath{stroke}%
\end{pgfscope}%
\begin{pgfscope}%
\pgfpathrectangle{\pgfqpoint{6.720588in}{4.155455in}}{\pgfqpoint{2.279412in}{2.004545in}}%
\pgfusepath{clip}%
\pgfsetbuttcap%
\pgfsetroundjoin%
\pgfsetlinewidth{0.852124pt}%
\definecolor{currentstroke}{rgb}{0.199430,0.387607,0.554642}%
\pgfsetstrokecolor{currentstroke}%
\pgfsetdash{}{0pt}%
\pgfpathmoveto{\pgfqpoint{8.180817in}{5.241480in}}%
\pgfpathlineto{\pgfqpoint{8.130706in}{5.239711in}}%
\pgfusepath{stroke}%
\end{pgfscope}%
\begin{pgfscope}%
\pgfpathrectangle{\pgfqpoint{6.720588in}{4.155455in}}{\pgfqpoint{2.279412in}{2.004545in}}%
\pgfusepath{clip}%
\pgfsetbuttcap%
\pgfsetroundjoin%
\pgfsetlinewidth{1.030497pt}%
\definecolor{currentstroke}{rgb}{0.157729,0.485932,0.558013}%
\pgfsetstrokecolor{currentstroke}%
\pgfsetdash{}{0pt}%
\pgfpathmoveto{\pgfqpoint{8.130706in}{5.239711in}}%
\pgfpathlineto{\pgfqpoint{8.080619in}{5.237486in}}%
\pgfusepath{stroke}%
\end{pgfscope}%
\begin{pgfscope}%
\pgfpathrectangle{\pgfqpoint{6.720588in}{4.155455in}}{\pgfqpoint{2.279412in}{2.004545in}}%
\pgfusepath{clip}%
\pgfsetbuttcap%
\pgfsetroundjoin%
\pgfsetlinewidth{1.304331pt}%
\definecolor{currentstroke}{rgb}{0.122312,0.633153,0.530398}%
\pgfsetstrokecolor{currentstroke}%
\pgfsetdash{}{0pt}%
\pgfpathmoveto{\pgfqpoint{8.080619in}{5.237486in}}%
\pgfpathlineto{\pgfqpoint{8.030578in}{5.234581in}}%
\pgfusepath{stroke}%
\end{pgfscope}%
\begin{pgfscope}%
\pgfpathrectangle{\pgfqpoint{6.720588in}{4.155455in}}{\pgfqpoint{2.279412in}{2.004545in}}%
\pgfusepath{clip}%
\pgfsetbuttcap%
\pgfsetroundjoin%
\pgfsetlinewidth{1.688733pt}%
\definecolor{currentstroke}{rgb}{0.458674,0.816363,0.329727}%
\pgfsetstrokecolor{currentstroke}%
\pgfsetdash{}{0pt}%
\pgfpathmoveto{\pgfqpoint{8.030578in}{5.234581in}}%
\pgfpathlineto{\pgfqpoint{7.980595in}{5.230986in}}%
\pgfusepath{stroke}%
\end{pgfscope}%
\begin{pgfscope}%
\pgfpathrectangle{\pgfqpoint{6.720588in}{4.155455in}}{\pgfqpoint{2.279412in}{2.004545in}}%
\pgfusepath{clip}%
\pgfsetbuttcap%
\pgfsetroundjoin%
\pgfsetlinewidth{1.923966pt}%
\definecolor{currentstroke}{rgb}{0.804182,0.882046,0.114965}%
\pgfsetstrokecolor{currentstroke}%
\pgfsetdash{}{0pt}%
\pgfpathmoveto{\pgfqpoint{7.980595in}{5.230986in}}%
\pgfpathlineto{\pgfqpoint{7.930691in}{5.226649in}}%
\pgfusepath{stroke}%
\end{pgfscope}%
\begin{pgfscope}%
\pgfpathrectangle{\pgfqpoint{6.720588in}{4.155455in}}{\pgfqpoint{2.279412in}{2.004545in}}%
\pgfusepath{clip}%
\pgfsetbuttcap%
\pgfsetroundjoin%
\pgfsetlinewidth{2.095647pt}%
\definecolor{currentstroke}{rgb}{0.993248,0.906157,0.143936}%
\pgfsetstrokecolor{currentstroke}%
\pgfsetdash{}{0pt}%
\pgfpathmoveto{\pgfqpoint{7.930691in}{5.226649in}}%
\pgfpathlineto{\pgfqpoint{7.880858in}{5.221703in}}%
\pgfusepath{stroke}%
\end{pgfscope}%
\begin{pgfscope}%
\pgfpathrectangle{\pgfqpoint{6.720588in}{4.155455in}}{\pgfqpoint{2.279412in}{2.004545in}}%
\pgfusepath{clip}%
\pgfsetbuttcap%
\pgfsetroundjoin%
\pgfsetlinewidth{2.157884pt}%
\definecolor{currentstroke}{rgb}{0.993248,0.906157,0.143936}%
\pgfsetstrokecolor{currentstroke}%
\pgfsetdash{}{0pt}%
\pgfpathmoveto{\pgfqpoint{7.880858in}{5.221703in}}%
\pgfpathlineto{\pgfqpoint{7.831095in}{5.216249in}}%
\pgfusepath{stroke}%
\end{pgfscope}%
\begin{pgfscope}%
\pgfpathrectangle{\pgfqpoint{6.720588in}{4.155455in}}{\pgfqpoint{2.279412in}{2.004545in}}%
\pgfusepath{clip}%
\pgfsetbuttcap%
\pgfsetroundjoin%
\pgfsetlinewidth{2.262567pt}%
\definecolor{currentstroke}{rgb}{0.993248,0.906157,0.143936}%
\pgfsetstrokecolor{currentstroke}%
\pgfsetdash{}{0pt}%
\pgfpathmoveto{\pgfqpoint{7.831095in}{5.216249in}}%
\pgfpathlineto{\pgfqpoint{7.781390in}{5.210411in}}%
\pgfusepath{stroke}%
\end{pgfscope}%
\begin{pgfscope}%
\pgfpathrectangle{\pgfqpoint{6.720588in}{4.155455in}}{\pgfqpoint{2.279412in}{2.004545in}}%
\pgfusepath{clip}%
\pgfsetbuttcap%
\pgfsetroundjoin%
\pgfsetlinewidth{2.302523pt}%
\definecolor{currentstroke}{rgb}{0.993248,0.906157,0.143936}%
\pgfsetstrokecolor{currentstroke}%
\pgfsetdash{}{0pt}%
\pgfpathmoveto{\pgfqpoint{7.781390in}{5.210411in}}%
\pgfpathlineto{\pgfqpoint{7.731758in}{5.204165in}}%
\pgfusepath{stroke}%
\end{pgfscope}%
\begin{pgfscope}%
\pgfpathrectangle{\pgfqpoint{6.720588in}{4.155455in}}{\pgfqpoint{2.279412in}{2.004545in}}%
\pgfusepath{clip}%
\pgfsetbuttcap%
\pgfsetroundjoin%
\pgfsetlinewidth{2.356476pt}%
\definecolor{currentstroke}{rgb}{0.993248,0.906157,0.143936}%
\pgfsetstrokecolor{currentstroke}%
\pgfsetdash{}{0pt}%
\pgfpathmoveto{\pgfqpoint{7.731758in}{5.204165in}}%
\pgfpathlineto{\pgfqpoint{7.682164in}{5.197665in}}%
\pgfusepath{stroke}%
\end{pgfscope}%
\begin{pgfscope}%
\pgfpathrectangle{\pgfqpoint{6.720588in}{4.155455in}}{\pgfqpoint{2.279412in}{2.004545in}}%
\pgfusepath{clip}%
\pgfsetbuttcap%
\pgfsetroundjoin%
\pgfsetlinewidth{2.122661pt}%
\definecolor{currentstroke}{rgb}{0.993248,0.906157,0.143936}%
\pgfsetstrokecolor{currentstroke}%
\pgfsetdash{}{0pt}%
\pgfpathmoveto{\pgfqpoint{7.682164in}{5.197665in}}%
\pgfpathlineto{\pgfqpoint{7.632621in}{5.190940in}}%
\pgfusepath{stroke}%
\end{pgfscope}%
\begin{pgfscope}%
\pgfpathrectangle{\pgfqpoint{6.720588in}{4.155455in}}{\pgfqpoint{2.279412in}{2.004545in}}%
\pgfusepath{clip}%
\pgfsetbuttcap%
\pgfsetroundjoin%
\pgfsetlinewidth{1.972738pt}%
\definecolor{currentstroke}{rgb}{0.876168,0.891125,0.095250}%
\pgfsetstrokecolor{currentstroke}%
\pgfsetdash{}{0pt}%
\pgfpathmoveto{\pgfqpoint{7.632621in}{5.190940in}}%
\pgfpathlineto{\pgfqpoint{7.583126in}{5.184128in}}%
\pgfusepath{stroke}%
\end{pgfscope}%
\begin{pgfscope}%
\pgfpathrectangle{\pgfqpoint{6.720588in}{4.155455in}}{\pgfqpoint{2.279412in}{2.004545in}}%
\pgfusepath{clip}%
\pgfsetbuttcap%
\pgfsetroundjoin%
\pgfsetlinewidth{1.876535pt}%
\definecolor{currentstroke}{rgb}{0.730889,0.871916,0.156029}%
\pgfsetstrokecolor{currentstroke}%
\pgfsetdash{}{0pt}%
\pgfpathmoveto{\pgfqpoint{7.583126in}{5.184128in}}%
\pgfpathlineto{\pgfqpoint{7.533642in}{5.177170in}}%
\pgfusepath{stroke}%
\end{pgfscope}%
\begin{pgfscope}%
\pgfpathrectangle{\pgfqpoint{6.720588in}{4.155455in}}{\pgfqpoint{2.279412in}{2.004545in}}%
\pgfusepath{clip}%
\pgfsetbuttcap%
\pgfsetroundjoin%
\pgfsetlinewidth{1.569561pt}%
\definecolor{currentstroke}{rgb}{0.304148,0.764704,0.419943}%
\pgfsetstrokecolor{currentstroke}%
\pgfsetdash{}{0pt}%
\pgfpathmoveto{\pgfqpoint{7.533642in}{5.177170in}}%
\pgfpathlineto{\pgfqpoint{7.484109in}{5.170852in}}%
\pgfusepath{stroke}%
\end{pgfscope}%
\begin{pgfscope}%
\pgfpathrectangle{\pgfqpoint{6.720588in}{4.155455in}}{\pgfqpoint{2.279412in}{2.004545in}}%
\pgfusepath{clip}%
\pgfsetbuttcap%
\pgfsetroundjoin%
\pgfsetlinewidth{1.359678pt}%
\definecolor{currentstroke}{rgb}{0.137339,0.662252,0.515571}%
\pgfsetstrokecolor{currentstroke}%
\pgfsetdash{}{0pt}%
\pgfpathmoveto{\pgfqpoint{7.484109in}{5.170852in}}%
\pgfpathlineto{\pgfqpoint{7.434741in}{5.164547in}}%
\pgfusepath{stroke}%
\end{pgfscope}%
\begin{pgfscope}%
\pgfpathrectangle{\pgfqpoint{6.720588in}{4.155455in}}{\pgfqpoint{2.279412in}{2.004545in}}%
\pgfusepath{clip}%
\pgfsetbuttcap%
\pgfsetroundjoin%
\pgfsetlinewidth{1.228794pt}%
\definecolor{currentstroke}{rgb}{0.121148,0.592739,0.544641}%
\pgfsetstrokecolor{currentstroke}%
\pgfsetdash{}{0pt}%
\pgfpathmoveto{\pgfqpoint{7.434741in}{5.164547in}}%
\pgfpathlineto{\pgfqpoint{7.385082in}{5.160671in}}%
\pgfusepath{stroke}%
\end{pgfscope}%
\begin{pgfscope}%
\pgfpathrectangle{\pgfqpoint{6.720588in}{4.155455in}}{\pgfqpoint{2.279412in}{2.004545in}}%
\pgfusepath{clip}%
\pgfsetbuttcap%
\pgfsetroundjoin%
\pgfsetlinewidth{0.309954pt}%
\definecolor{currentstroke}{rgb}{0.268510,0.009605,0.335427}%
\pgfsetstrokecolor{currentstroke}%
\pgfsetdash{}{0pt}%
\pgfpathmoveto{\pgfqpoint{8.732256in}{5.293048in}}%
\pgfpathlineto{\pgfqpoint{8.682228in}{5.294577in}}%
\pgfusepath{stroke}%
\end{pgfscope}%
\begin{pgfscope}%
\pgfpathrectangle{\pgfqpoint{6.720588in}{4.155455in}}{\pgfqpoint{2.279412in}{2.004545in}}%
\pgfusepath{clip}%
\pgfsetbuttcap%
\pgfsetroundjoin%
\pgfsetlinewidth{0.326466pt}%
\definecolor{currentstroke}{rgb}{0.271305,0.019942,0.347269}%
\pgfsetstrokecolor{currentstroke}%
\pgfsetdash{}{0pt}%
\pgfpathmoveto{\pgfqpoint{8.682228in}{5.294577in}}%
\pgfpathlineto{\pgfqpoint{8.632094in}{5.294340in}}%
\pgfusepath{stroke}%
\end{pgfscope}%
\begin{pgfscope}%
\pgfpathrectangle{\pgfqpoint{6.720588in}{4.155455in}}{\pgfqpoint{2.279412in}{2.004545in}}%
\pgfusepath{clip}%
\pgfsetbuttcap%
\pgfsetroundjoin%
\pgfsetlinewidth{0.328203pt}%
\definecolor{currentstroke}{rgb}{0.271305,0.019942,0.347269}%
\pgfsetstrokecolor{currentstroke}%
\pgfsetdash{}{0pt}%
\pgfpathmoveto{\pgfqpoint{8.632094in}{5.294340in}}%
\pgfpathlineto{\pgfqpoint{8.581960in}{5.293449in}}%
\pgfusepath{stroke}%
\end{pgfscope}%
\begin{pgfscope}%
\pgfpathrectangle{\pgfqpoint{6.720588in}{4.155455in}}{\pgfqpoint{2.279412in}{2.004545in}}%
\pgfusepath{clip}%
\pgfsetbuttcap%
\pgfsetroundjoin%
\pgfsetlinewidth{0.332265pt}%
\definecolor{currentstroke}{rgb}{0.272594,0.025563,0.353093}%
\pgfsetstrokecolor{currentstroke}%
\pgfsetdash{}{0pt}%
\pgfpathmoveto{\pgfqpoint{8.581960in}{5.293449in}}%
\pgfpathlineto{\pgfqpoint{8.531820in}{5.292720in}}%
\pgfusepath{stroke}%
\end{pgfscope}%
\begin{pgfscope}%
\pgfpathrectangle{\pgfqpoint{6.720588in}{4.155455in}}{\pgfqpoint{2.279412in}{2.004545in}}%
\pgfusepath{clip}%
\pgfsetbuttcap%
\pgfsetroundjoin%
\pgfsetlinewidth{0.348154pt}%
\definecolor{currentstroke}{rgb}{0.274952,0.037752,0.364543}%
\pgfsetstrokecolor{currentstroke}%
\pgfsetdash{}{0pt}%
\pgfpathmoveto{\pgfqpoint{8.531820in}{5.292720in}}%
\pgfpathlineto{\pgfqpoint{8.481676in}{5.292055in}}%
\pgfusepath{stroke}%
\end{pgfscope}%
\begin{pgfscope}%
\pgfpathrectangle{\pgfqpoint{6.720588in}{4.155455in}}{\pgfqpoint{2.279412in}{2.004545in}}%
\pgfusepath{clip}%
\pgfsetbuttcap%
\pgfsetroundjoin%
\pgfsetlinewidth{0.366666pt}%
\definecolor{currentstroke}{rgb}{0.277941,0.056324,0.381191}%
\pgfsetstrokecolor{currentstroke}%
\pgfsetdash{}{0pt}%
\pgfpathmoveto{\pgfqpoint{8.481676in}{5.292055in}}%
\pgfpathlineto{\pgfqpoint{8.431538in}{5.291017in}}%
\pgfusepath{stroke}%
\end{pgfscope}%
\begin{pgfscope}%
\pgfpathrectangle{\pgfqpoint{6.720588in}{4.155455in}}{\pgfqpoint{2.279412in}{2.004545in}}%
\pgfusepath{clip}%
\pgfsetbuttcap%
\pgfsetroundjoin%
\pgfsetlinewidth{0.395579pt}%
\definecolor{currentstroke}{rgb}{0.280894,0.078907,0.402329}%
\pgfsetstrokecolor{currentstroke}%
\pgfsetdash{}{0pt}%
\pgfpathmoveto{\pgfqpoint{8.431538in}{5.291017in}}%
\pgfpathlineto{\pgfqpoint{8.381407in}{5.289785in}}%
\pgfusepath{stroke}%
\end{pgfscope}%
\begin{pgfscope}%
\pgfpathrectangle{\pgfqpoint{6.720588in}{4.155455in}}{\pgfqpoint{2.279412in}{2.004545in}}%
\pgfusepath{clip}%
\pgfsetbuttcap%
\pgfsetroundjoin%
\pgfsetlinewidth{0.423890pt}%
\definecolor{currentstroke}{rgb}{0.282656,0.100196,0.422160}%
\pgfsetstrokecolor{currentstroke}%
\pgfsetdash{}{0pt}%
\pgfpathmoveto{\pgfqpoint{8.381407in}{5.289785in}}%
\pgfpathlineto{\pgfqpoint{8.331288in}{5.288252in}}%
\pgfusepath{stroke}%
\end{pgfscope}%
\begin{pgfscope}%
\pgfpathrectangle{\pgfqpoint{6.720588in}{4.155455in}}{\pgfqpoint{2.279412in}{2.004545in}}%
\pgfusepath{clip}%
\pgfsetbuttcap%
\pgfsetroundjoin%
\pgfsetlinewidth{0.474252pt}%
\definecolor{currentstroke}{rgb}{0.282623,0.140926,0.457517}%
\pgfsetstrokecolor{currentstroke}%
\pgfsetdash{}{0pt}%
\pgfpathmoveto{\pgfqpoint{8.331288in}{5.288252in}}%
\pgfpathlineto{\pgfqpoint{8.281172in}{5.286610in}}%
\pgfusepath{stroke}%
\end{pgfscope}%
\begin{pgfscope}%
\pgfpathrectangle{\pgfqpoint{6.720588in}{4.155455in}}{\pgfqpoint{2.279412in}{2.004545in}}%
\pgfusepath{clip}%
\pgfsetbuttcap%
\pgfsetroundjoin%
\pgfsetlinewidth{0.547323pt}%
\definecolor{currentstroke}{rgb}{0.276194,0.190074,0.493001}%
\pgfsetstrokecolor{currentstroke}%
\pgfsetdash{}{0pt}%
\pgfpathmoveto{\pgfqpoint{8.281172in}{5.286610in}}%
\pgfpathlineto{\pgfqpoint{8.231059in}{5.284897in}}%
\pgfusepath{stroke}%
\end{pgfscope}%
\begin{pgfscope}%
\pgfpathrectangle{\pgfqpoint{6.720588in}{4.155455in}}{\pgfqpoint{2.279412in}{2.004545in}}%
\pgfusepath{clip}%
\pgfsetbuttcap%
\pgfsetroundjoin%
\pgfsetlinewidth{0.643836pt}%
\definecolor{currentstroke}{rgb}{0.257322,0.256130,0.526563}%
\pgfsetstrokecolor{currentstroke}%
\pgfsetdash{}{0pt}%
\pgfpathmoveto{\pgfqpoint{8.231059in}{5.284897in}}%
\pgfpathlineto{\pgfqpoint{8.180964in}{5.282827in}}%
\pgfusepath{stroke}%
\end{pgfscope}%
\begin{pgfscope}%
\pgfpathrectangle{\pgfqpoint{6.720588in}{4.155455in}}{\pgfqpoint{2.279412in}{2.004545in}}%
\pgfusepath{clip}%
\pgfsetbuttcap%
\pgfsetroundjoin%
\pgfsetlinewidth{0.731130pt}%
\definecolor{currentstroke}{rgb}{0.233603,0.313828,0.543914}%
\pgfsetstrokecolor{currentstroke}%
\pgfsetdash{}{0pt}%
\pgfpathmoveto{\pgfqpoint{8.180964in}{5.282827in}}%
\pgfpathlineto{\pgfqpoint{8.130894in}{5.280313in}}%
\pgfusepath{stroke}%
\end{pgfscope}%
\begin{pgfscope}%
\pgfpathrectangle{\pgfqpoint{6.720588in}{4.155455in}}{\pgfqpoint{2.279412in}{2.004545in}}%
\pgfusepath{clip}%
\pgfsetbuttcap%
\pgfsetroundjoin%
\pgfsetlinewidth{0.880983pt}%
\definecolor{currentstroke}{rgb}{0.192357,0.403199,0.555836}%
\pgfsetstrokecolor{currentstroke}%
\pgfsetdash{}{0pt}%
\pgfpathmoveto{\pgfqpoint{8.130894in}{5.280313in}}%
\pgfpathlineto{\pgfqpoint{8.080873in}{5.277169in}}%
\pgfusepath{stroke}%
\end{pgfscope}%
\begin{pgfscope}%
\pgfpathrectangle{\pgfqpoint{6.720588in}{4.155455in}}{\pgfqpoint{2.279412in}{2.004545in}}%
\pgfusepath{clip}%
\pgfsetbuttcap%
\pgfsetroundjoin%
\pgfsetlinewidth{1.094085pt}%
\definecolor{currentstroke}{rgb}{0.144759,0.519093,0.556572}%
\pgfsetstrokecolor{currentstroke}%
\pgfsetdash{}{0pt}%
\pgfpathmoveto{\pgfqpoint{8.080873in}{5.277169in}}%
\pgfpathlineto{\pgfqpoint{8.030940in}{5.273097in}}%
\pgfusepath{stroke}%
\end{pgfscope}%
\begin{pgfscope}%
\pgfpathrectangle{\pgfqpoint{6.720588in}{4.155455in}}{\pgfqpoint{2.279412in}{2.004545in}}%
\pgfusepath{clip}%
\pgfsetbuttcap%
\pgfsetroundjoin%
\pgfsetlinewidth{0.313200pt}%
\definecolor{currentstroke}{rgb}{0.268510,0.009605,0.335427}%
\pgfsetstrokecolor{currentstroke}%
\pgfsetdash{}{0pt}%
\pgfpathmoveto{\pgfqpoint{8.680964in}{4.932193in}}%
\pgfpathlineto{\pgfqpoint{8.632352in}{4.932963in}}%
\pgfusepath{stroke}%
\end{pgfscope}%
\begin{pgfscope}%
\pgfpathrectangle{\pgfqpoint{6.720588in}{4.155455in}}{\pgfqpoint{2.279412in}{2.004545in}}%
\pgfusepath{clip}%
\pgfsetbuttcap%
\pgfsetroundjoin%
\pgfsetlinewidth{0.318671pt}%
\definecolor{currentstroke}{rgb}{0.269944,0.014625,0.341379}%
\pgfsetstrokecolor{currentstroke}%
\pgfsetdash{}{0pt}%
\pgfpathmoveto{\pgfqpoint{8.632352in}{4.932963in}}%
\pgfpathlineto{\pgfqpoint{8.582213in}{4.933271in}}%
\pgfusepath{stroke}%
\end{pgfscope}%
\begin{pgfscope}%
\pgfpathrectangle{\pgfqpoint{6.720588in}{4.155455in}}{\pgfqpoint{2.279412in}{2.004545in}}%
\pgfusepath{clip}%
\pgfsetbuttcap%
\pgfsetroundjoin%
\pgfsetlinewidth{0.324895pt}%
\definecolor{currentstroke}{rgb}{0.271305,0.019942,0.347269}%
\pgfsetstrokecolor{currentstroke}%
\pgfsetdash{}{0pt}%
\pgfpathmoveto{\pgfqpoint{8.582213in}{4.933271in}}%
\pgfpathlineto{\pgfqpoint{8.532091in}{4.934246in}}%
\pgfusepath{stroke}%
\end{pgfscope}%
\begin{pgfscope}%
\pgfpathrectangle{\pgfqpoint{6.720588in}{4.155455in}}{\pgfqpoint{2.279412in}{2.004545in}}%
\pgfusepath{clip}%
\pgfsetbuttcap%
\pgfsetroundjoin%
\pgfsetlinewidth{0.339833pt}%
\definecolor{currentstroke}{rgb}{0.273809,0.031497,0.358853}%
\pgfsetstrokecolor{currentstroke}%
\pgfsetdash{}{0pt}%
\pgfpathmoveto{\pgfqpoint{8.532091in}{4.934246in}}%
\pgfpathlineto{\pgfqpoint{8.482023in}{4.936669in}}%
\pgfusepath{stroke}%
\end{pgfscope}%
\begin{pgfscope}%
\pgfpathrectangle{\pgfqpoint{6.720588in}{4.155455in}}{\pgfqpoint{2.279412in}{2.004545in}}%
\pgfusepath{clip}%
\pgfsetbuttcap%
\pgfsetroundjoin%
\pgfsetlinewidth{0.351874pt}%
\definecolor{currentstroke}{rgb}{0.276022,0.044167,0.370164}%
\pgfsetstrokecolor{currentstroke}%
\pgfsetdash{}{0pt}%
\pgfpathmoveto{\pgfqpoint{8.482023in}{4.936669in}}%
\pgfpathlineto{\pgfqpoint{8.431944in}{4.938904in}}%
\pgfusepath{stroke}%
\end{pgfscope}%
\begin{pgfscope}%
\pgfpathrectangle{\pgfqpoint{6.720588in}{4.155455in}}{\pgfqpoint{2.279412in}{2.004545in}}%
\pgfusepath{clip}%
\pgfsetbuttcap%
\pgfsetroundjoin%
\pgfsetlinewidth{0.374677pt}%
\definecolor{currentstroke}{rgb}{0.278791,0.062145,0.386592}%
\pgfsetstrokecolor{currentstroke}%
\pgfsetdash{}{0pt}%
\pgfpathmoveto{\pgfqpoint{8.431944in}{4.938904in}}%
\pgfpathlineto{\pgfqpoint{8.381844in}{4.940863in}}%
\pgfusepath{stroke}%
\end{pgfscope}%
\begin{pgfscope}%
\pgfpathrectangle{\pgfqpoint{6.720588in}{4.155455in}}{\pgfqpoint{2.279412in}{2.004545in}}%
\pgfusepath{clip}%
\pgfsetbuttcap%
\pgfsetroundjoin%
\pgfsetlinewidth{0.405932pt}%
\definecolor{currentstroke}{rgb}{0.281924,0.089666,0.412415}%
\pgfsetstrokecolor{currentstroke}%
\pgfsetdash{}{0pt}%
\pgfpathmoveto{\pgfqpoint{8.381844in}{4.940863in}}%
\pgfpathlineto{\pgfqpoint{8.331761in}{4.943184in}}%
\pgfusepath{stroke}%
\end{pgfscope}%
\begin{pgfscope}%
\pgfpathrectangle{\pgfqpoint{6.720588in}{4.155455in}}{\pgfqpoint{2.279412in}{2.004545in}}%
\pgfusepath{clip}%
\pgfsetbuttcap%
\pgfsetroundjoin%
\pgfsetlinewidth{0.456233pt}%
\definecolor{currentstroke}{rgb}{0.283187,0.125848,0.444960}%
\pgfsetstrokecolor{currentstroke}%
\pgfsetdash{}{0pt}%
\pgfpathmoveto{\pgfqpoint{8.331761in}{4.943184in}}%
\pgfpathlineto{\pgfqpoint{8.281679in}{4.945505in}}%
\pgfusepath{stroke}%
\end{pgfscope}%
\begin{pgfscope}%
\pgfpathrectangle{\pgfqpoint{6.720588in}{4.155455in}}{\pgfqpoint{2.279412in}{2.004545in}}%
\pgfusepath{clip}%
\pgfsetbuttcap%
\pgfsetroundjoin%
\pgfsetlinewidth{0.505737pt}%
\definecolor{currentstroke}{rgb}{0.280868,0.160771,0.472899}%
\pgfsetstrokecolor{currentstroke}%
\pgfsetdash{}{0pt}%
\pgfpathmoveto{\pgfqpoint{8.281679in}{4.945505in}}%
\pgfpathlineto{\pgfqpoint{8.231614in}{4.948075in}}%
\pgfusepath{stroke}%
\end{pgfscope}%
\begin{pgfscope}%
\pgfpathrectangle{\pgfqpoint{6.720588in}{4.155455in}}{\pgfqpoint{2.279412in}{2.004545in}}%
\pgfusepath{clip}%
\pgfsetbuttcap%
\pgfsetroundjoin%
\pgfsetlinewidth{0.557660pt}%
\definecolor{currentstroke}{rgb}{0.274128,0.199721,0.498911}%
\pgfsetstrokecolor{currentstroke}%
\pgfsetdash{}{0pt}%
\pgfpathmoveto{\pgfqpoint{8.231614in}{4.948075in}}%
\pgfpathlineto{\pgfqpoint{8.181593in}{4.951228in}}%
\pgfusepath{stroke}%
\end{pgfscope}%
\begin{pgfscope}%
\pgfpathrectangle{\pgfqpoint{6.720588in}{4.155455in}}{\pgfqpoint{2.279412in}{2.004545in}}%
\pgfusepath{clip}%
\pgfsetbuttcap%
\pgfsetroundjoin%
\pgfsetlinewidth{0.638000pt}%
\definecolor{currentstroke}{rgb}{0.257322,0.256130,0.526563}%
\pgfsetstrokecolor{currentstroke}%
\pgfsetdash{}{0pt}%
\pgfpathmoveto{\pgfqpoint{8.181593in}{4.951228in}}%
\pgfpathlineto{\pgfqpoint{8.131631in}{4.955041in}}%
\pgfusepath{stroke}%
\end{pgfscope}%
\begin{pgfscope}%
\pgfpathrectangle{\pgfqpoint{6.720588in}{4.155455in}}{\pgfqpoint{2.279412in}{2.004545in}}%
\pgfusepath{clip}%
\pgfsetbuttcap%
\pgfsetroundjoin%
\pgfsetlinewidth{0.736629pt}%
\definecolor{currentstroke}{rgb}{0.231674,0.318106,0.544834}%
\pgfsetstrokecolor{currentstroke}%
\pgfsetdash{}{0pt}%
\pgfpathmoveto{\pgfqpoint{8.131631in}{4.955041in}}%
\pgfpathlineto{\pgfqpoint{8.081760in}{4.959679in}}%
\pgfusepath{stroke}%
\end{pgfscope}%
\begin{pgfscope}%
\pgfpathrectangle{\pgfqpoint{6.720588in}{4.155455in}}{\pgfqpoint{2.279412in}{2.004545in}}%
\pgfusepath{clip}%
\pgfsetbuttcap%
\pgfsetroundjoin%
\pgfsetlinewidth{0.815715pt}%
\definecolor{currentstroke}{rgb}{0.210503,0.363727,0.552206}%
\pgfsetstrokecolor{currentstroke}%
\pgfsetdash{}{0pt}%
\pgfpathmoveto{\pgfqpoint{8.081760in}{4.959679in}}%
\pgfpathlineto{\pgfqpoint{8.032016in}{4.965259in}}%
\pgfusepath{stroke}%
\end{pgfscope}%
\begin{pgfscope}%
\pgfpathrectangle{\pgfqpoint{6.720588in}{4.155455in}}{\pgfqpoint{2.279412in}{2.004545in}}%
\pgfusepath{clip}%
\pgfsetbuttcap%
\pgfsetroundjoin%
\pgfsetlinewidth{0.808555pt}%
\definecolor{currentstroke}{rgb}{0.212395,0.359683,0.551710}%
\pgfsetstrokecolor{currentstroke}%
\pgfsetdash{}{0pt}%
\pgfpathmoveto{\pgfqpoint{8.032016in}{4.965259in}}%
\pgfpathlineto{\pgfqpoint{7.982467in}{4.972025in}}%
\pgfusepath{stroke}%
\end{pgfscope}%
\begin{pgfscope}%
\pgfpathrectangle{\pgfqpoint{6.720588in}{4.155455in}}{\pgfqpoint{2.279412in}{2.004545in}}%
\pgfusepath{clip}%
\pgfsetbuttcap%
\pgfsetroundjoin%
\pgfsetlinewidth{0.984415pt}%
\definecolor{currentstroke}{rgb}{0.168126,0.459988,0.558082}%
\pgfsetstrokecolor{currentstroke}%
\pgfsetdash{}{0pt}%
\pgfpathmoveto{\pgfqpoint{7.982467in}{4.972025in}}%
\pgfpathlineto{\pgfqpoint{7.933214in}{4.980285in}}%
\pgfusepath{stroke}%
\end{pgfscope}%
\begin{pgfscope}%
\pgfpathrectangle{\pgfqpoint{6.720588in}{4.155455in}}{\pgfqpoint{2.279412in}{2.004545in}}%
\pgfusepath{clip}%
\pgfsetbuttcap%
\pgfsetroundjoin%
\pgfsetlinewidth{1.297118pt}%
\definecolor{currentstroke}{rgb}{0.121380,0.629492,0.531973}%
\pgfsetstrokecolor{currentstroke}%
\pgfsetdash{}{0pt}%
\pgfpathmoveto{\pgfqpoint{7.933214in}{4.980285in}}%
\pgfpathlineto{\pgfqpoint{7.884336in}{4.990121in}}%
\pgfusepath{stroke}%
\end{pgfscope}%
\begin{pgfscope}%
\pgfpathrectangle{\pgfqpoint{6.720588in}{4.155455in}}{\pgfqpoint{2.279412in}{2.004545in}}%
\pgfusepath{clip}%
\pgfsetbuttcap%
\pgfsetroundjoin%
\pgfsetlinewidth{0.320983pt}%
\definecolor{currentstroke}{rgb}{0.269944,0.014625,0.341379}%
\pgfsetstrokecolor{currentstroke}%
\pgfsetdash{}{0pt}%
\pgfpathmoveto{\pgfqpoint{8.680964in}{5.022407in}}%
\pgfpathlineto{\pgfqpoint{8.630902in}{5.024210in}}%
\pgfusepath{stroke}%
\end{pgfscope}%
\begin{pgfscope}%
\pgfpathrectangle{\pgfqpoint{6.720588in}{4.155455in}}{\pgfqpoint{2.279412in}{2.004545in}}%
\pgfusepath{clip}%
\pgfsetbuttcap%
\pgfsetroundjoin%
\pgfsetlinewidth{0.332038pt}%
\definecolor{currentstroke}{rgb}{0.272594,0.025563,0.353093}%
\pgfsetstrokecolor{currentstroke}%
\pgfsetdash{}{0pt}%
\pgfpathmoveto{\pgfqpoint{8.630902in}{5.024210in}}%
\pgfpathlineto{\pgfqpoint{8.580764in}{5.024635in}}%
\pgfusepath{stroke}%
\end{pgfscope}%
\begin{pgfscope}%
\pgfpathrectangle{\pgfqpoint{6.720588in}{4.155455in}}{\pgfqpoint{2.279412in}{2.004545in}}%
\pgfusepath{clip}%
\pgfsetbuttcap%
\pgfsetroundjoin%
\pgfsetlinewidth{0.337136pt}%
\definecolor{currentstroke}{rgb}{0.273809,0.031497,0.358853}%
\pgfsetstrokecolor{currentstroke}%
\pgfsetdash{}{0pt}%
\pgfpathmoveto{\pgfqpoint{8.580764in}{5.024635in}}%
\pgfpathlineto{\pgfqpoint{8.530618in}{5.024760in}}%
\pgfusepath{stroke}%
\end{pgfscope}%
\begin{pgfscope}%
\pgfpathrectangle{\pgfqpoint{6.720588in}{4.155455in}}{\pgfqpoint{2.279412in}{2.004545in}}%
\pgfusepath{clip}%
\pgfsetbuttcap%
\pgfsetroundjoin%
\pgfsetlinewidth{0.347779pt}%
\definecolor{currentstroke}{rgb}{0.274952,0.037752,0.364543}%
\pgfsetstrokecolor{currentstroke}%
\pgfsetdash{}{0pt}%
\pgfpathmoveto{\pgfqpoint{8.530618in}{5.024760in}}%
\pgfpathlineto{\pgfqpoint{8.480474in}{5.025293in}}%
\pgfusepath{stroke}%
\end{pgfscope}%
\begin{pgfscope}%
\pgfpathrectangle{\pgfqpoint{6.720588in}{4.155455in}}{\pgfqpoint{2.279412in}{2.004545in}}%
\pgfusepath{clip}%
\pgfsetbuttcap%
\pgfsetroundjoin%
\pgfsetlinewidth{0.367421pt}%
\definecolor{currentstroke}{rgb}{0.277941,0.056324,0.381191}%
\pgfsetstrokecolor{currentstroke}%
\pgfsetdash{}{0pt}%
\pgfpathmoveto{\pgfqpoint{8.480474in}{5.025293in}}%
\pgfpathlineto{\pgfqpoint{8.430331in}{5.025995in}}%
\pgfusepath{stroke}%
\end{pgfscope}%
\begin{pgfscope}%
\pgfpathrectangle{\pgfqpoint{6.720588in}{4.155455in}}{\pgfqpoint{2.279412in}{2.004545in}}%
\pgfusepath{clip}%
\pgfsetbuttcap%
\pgfsetroundjoin%
\pgfsetlinewidth{0.402634pt}%
\definecolor{currentstroke}{rgb}{0.281446,0.084320,0.407414}%
\pgfsetstrokecolor{currentstroke}%
\pgfsetdash{}{0pt}%
\pgfpathmoveto{\pgfqpoint{8.430331in}{5.025995in}}%
\pgfpathlineto{\pgfqpoint{8.380192in}{5.026890in}}%
\pgfusepath{stroke}%
\end{pgfscope}%
\begin{pgfscope}%
\pgfpathrectangle{\pgfqpoint{6.720588in}{4.155455in}}{\pgfqpoint{2.279412in}{2.004545in}}%
\pgfusepath{clip}%
\pgfsetbuttcap%
\pgfsetroundjoin%
\pgfsetlinewidth{0.428825pt}%
\definecolor{currentstroke}{rgb}{0.282910,0.105393,0.426902}%
\pgfsetstrokecolor{currentstroke}%
\pgfsetdash{}{0pt}%
\pgfpathmoveto{\pgfqpoint{8.380192in}{5.026890in}}%
\pgfpathlineto{\pgfqpoint{8.330054in}{5.027931in}}%
\pgfusepath{stroke}%
\end{pgfscope}%
\begin{pgfscope}%
\pgfpathrectangle{\pgfqpoint{6.720588in}{4.155455in}}{\pgfqpoint{2.279412in}{2.004545in}}%
\pgfusepath{clip}%
\pgfsetbuttcap%
\pgfsetroundjoin%
\pgfsetlinewidth{0.485426pt}%
\definecolor{currentstroke}{rgb}{0.282290,0.145912,0.461510}%
\pgfsetstrokecolor{currentstroke}%
\pgfsetdash{}{0pt}%
\pgfpathmoveto{\pgfqpoint{8.330054in}{5.027931in}}%
\pgfpathlineto{\pgfqpoint{8.279919in}{5.029103in}}%
\pgfusepath{stroke}%
\end{pgfscope}%
\begin{pgfscope}%
\pgfpathrectangle{\pgfqpoint{6.720588in}{4.155455in}}{\pgfqpoint{2.279412in}{2.004545in}}%
\pgfusepath{clip}%
\pgfsetbuttcap%
\pgfsetroundjoin%
\pgfsetlinewidth{0.559520pt}%
\definecolor{currentstroke}{rgb}{0.274128,0.199721,0.498911}%
\pgfsetstrokecolor{currentstroke}%
\pgfsetdash{}{0pt}%
\pgfpathmoveto{\pgfqpoint{8.279919in}{5.029103in}}%
\pgfpathlineto{\pgfqpoint{8.229792in}{5.030476in}}%
\pgfusepath{stroke}%
\end{pgfscope}%
\begin{pgfscope}%
\pgfpathrectangle{\pgfqpoint{6.720588in}{4.155455in}}{\pgfqpoint{2.279412in}{2.004545in}}%
\pgfusepath{clip}%
\pgfsetbuttcap%
\pgfsetroundjoin%
\pgfsetlinewidth{0.683910pt}%
\definecolor{currentstroke}{rgb}{0.246811,0.283237,0.535941}%
\pgfsetstrokecolor{currentstroke}%
\pgfsetdash{}{0pt}%
\pgfpathmoveto{\pgfqpoint{8.229792in}{5.030476in}}%
\pgfpathlineto{\pgfqpoint{8.179679in}{5.032175in}}%
\pgfusepath{stroke}%
\end{pgfscope}%
\begin{pgfscope}%
\pgfpathrectangle{\pgfqpoint{6.720588in}{4.155455in}}{\pgfqpoint{2.279412in}{2.004545in}}%
\pgfusepath{clip}%
\pgfsetbuttcap%
\pgfsetroundjoin%
\pgfsetlinewidth{0.310521pt}%
\definecolor{currentstroke}{rgb}{0.268510,0.009605,0.335427}%
\pgfsetstrokecolor{currentstroke}%
\pgfsetdash{}{0pt}%
\pgfpathmoveto{\pgfqpoint{8.680964in}{5.067514in}}%
\pgfpathlineto{\pgfqpoint{8.631933in}{5.071808in}}%
\pgfusepath{stroke}%
\end{pgfscope}%
\begin{pgfscope}%
\pgfpathrectangle{\pgfqpoint{6.720588in}{4.155455in}}{\pgfqpoint{2.279412in}{2.004545in}}%
\pgfusepath{clip}%
\pgfsetbuttcap%
\pgfsetroundjoin%
\pgfsetlinewidth{0.321239pt}%
\definecolor{currentstroke}{rgb}{0.269944,0.014625,0.341379}%
\pgfsetstrokecolor{currentstroke}%
\pgfsetdash{}{0pt}%
\pgfpathmoveto{\pgfqpoint{8.631933in}{5.071808in}}%
\pgfpathlineto{\pgfqpoint{8.581804in}{5.072721in}}%
\pgfusepath{stroke}%
\end{pgfscope}%
\begin{pgfscope}%
\pgfpathrectangle{\pgfqpoint{6.720588in}{4.155455in}}{\pgfqpoint{2.279412in}{2.004545in}}%
\pgfusepath{clip}%
\pgfsetbuttcap%
\pgfsetroundjoin%
\pgfsetlinewidth{0.326684pt}%
\definecolor{currentstroke}{rgb}{0.271305,0.019942,0.347269}%
\pgfsetstrokecolor{currentstroke}%
\pgfsetdash{}{0pt}%
\pgfpathmoveto{\pgfqpoint{8.581804in}{5.072721in}}%
\pgfpathlineto{\pgfqpoint{8.531679in}{5.073306in}}%
\pgfusepath{stroke}%
\end{pgfscope}%
\begin{pgfscope}%
\pgfpathrectangle{\pgfqpoint{6.720588in}{4.155455in}}{\pgfqpoint{2.279412in}{2.004545in}}%
\pgfusepath{clip}%
\pgfsetbuttcap%
\pgfsetroundjoin%
\pgfsetlinewidth{0.344202pt}%
\definecolor{currentstroke}{rgb}{0.274952,0.037752,0.364543}%
\pgfsetstrokecolor{currentstroke}%
\pgfsetdash{}{0pt}%
\pgfpathmoveto{\pgfqpoint{8.531679in}{5.073306in}}%
\pgfpathlineto{\pgfqpoint{8.481534in}{5.073347in}}%
\pgfusepath{stroke}%
\end{pgfscope}%
\begin{pgfscope}%
\pgfpathrectangle{\pgfqpoint{6.720588in}{4.155455in}}{\pgfqpoint{2.279412in}{2.004545in}}%
\pgfusepath{clip}%
\pgfsetbuttcap%
\pgfsetroundjoin%
\pgfsetlinewidth{0.371211pt}%
\definecolor{currentstroke}{rgb}{0.278791,0.062145,0.386592}%
\pgfsetstrokecolor{currentstroke}%
\pgfsetdash{}{0pt}%
\pgfpathmoveto{\pgfqpoint{8.481534in}{5.073347in}}%
\pgfpathlineto{\pgfqpoint{8.431386in}{5.073631in}}%
\pgfusepath{stroke}%
\end{pgfscope}%
\begin{pgfscope}%
\pgfpathrectangle{\pgfqpoint{6.720588in}{4.155455in}}{\pgfqpoint{2.279412in}{2.004545in}}%
\pgfusepath{clip}%
\pgfsetbuttcap%
\pgfsetroundjoin%
\pgfsetlinewidth{0.400179pt}%
\definecolor{currentstroke}{rgb}{0.281446,0.084320,0.407414}%
\pgfsetstrokecolor{currentstroke}%
\pgfsetdash{}{0pt}%
\pgfpathmoveto{\pgfqpoint{8.431386in}{5.073631in}}%
\pgfpathlineto{\pgfqpoint{8.381238in}{5.074009in}}%
\pgfusepath{stroke}%
\end{pgfscope}%
\begin{pgfscope}%
\pgfpathrectangle{\pgfqpoint{6.720588in}{4.155455in}}{\pgfqpoint{2.279412in}{2.004545in}}%
\pgfusepath{clip}%
\pgfsetbuttcap%
\pgfsetroundjoin%
\pgfsetlinewidth{0.435678pt}%
\definecolor{currentstroke}{rgb}{0.283091,0.110553,0.431554}%
\pgfsetstrokecolor{currentstroke}%
\pgfsetdash{}{0pt}%
\pgfpathmoveto{\pgfqpoint{8.381238in}{5.074009in}}%
\pgfpathlineto{\pgfqpoint{8.331092in}{5.074689in}}%
\pgfusepath{stroke}%
\end{pgfscope}%
\begin{pgfscope}%
\pgfpathrectangle{\pgfqpoint{6.720588in}{4.155455in}}{\pgfqpoint{2.279412in}{2.004545in}}%
\pgfusepath{clip}%
\pgfsetbuttcap%
\pgfsetroundjoin%
\pgfsetlinewidth{0.507141pt}%
\definecolor{currentstroke}{rgb}{0.280255,0.165693,0.476498}%
\pgfsetstrokecolor{currentstroke}%
\pgfsetdash{}{0pt}%
\pgfpathmoveto{\pgfqpoint{8.331092in}{5.074689in}}%
\pgfpathlineto{\pgfqpoint{8.280946in}{5.075337in}}%
\pgfusepath{stroke}%
\end{pgfscope}%
\begin{pgfscope}%
\pgfpathrectangle{\pgfqpoint{6.720588in}{4.155455in}}{\pgfqpoint{2.279412in}{2.004545in}}%
\pgfusepath{clip}%
\pgfsetbuttcap%
\pgfsetroundjoin%
\pgfsetlinewidth{0.582539pt}%
\definecolor{currentstroke}{rgb}{0.269308,0.218818,0.509577}%
\pgfsetstrokecolor{currentstroke}%
\pgfsetdash{}{0pt}%
\pgfpathmoveto{\pgfqpoint{8.280946in}{5.075337in}}%
\pgfpathlineto{\pgfqpoint{8.230801in}{5.076039in}}%
\pgfusepath{stroke}%
\end{pgfscope}%
\begin{pgfscope}%
\pgfpathrectangle{\pgfqpoint{6.720588in}{4.155455in}}{\pgfqpoint{2.279412in}{2.004545in}}%
\pgfusepath{clip}%
\pgfsetbuttcap%
\pgfsetroundjoin%
\pgfsetlinewidth{0.720570pt}%
\definecolor{currentstroke}{rgb}{0.235526,0.309527,0.542944}%
\pgfsetstrokecolor{currentstroke}%
\pgfsetdash{}{0pt}%
\pgfpathmoveto{\pgfqpoint{8.230801in}{5.076039in}}%
\pgfpathlineto{\pgfqpoint{8.180660in}{5.076971in}}%
\pgfusepath{stroke}%
\end{pgfscope}%
\begin{pgfscope}%
\pgfpathrectangle{\pgfqpoint{6.720588in}{4.155455in}}{\pgfqpoint{2.279412in}{2.004545in}}%
\pgfusepath{clip}%
\pgfsetbuttcap%
\pgfsetroundjoin%
\pgfsetlinewidth{0.896089pt}%
\definecolor{currentstroke}{rgb}{0.188923,0.410910,0.556326}%
\pgfsetstrokecolor{currentstroke}%
\pgfsetdash{}{0pt}%
\pgfpathmoveto{\pgfqpoint{8.180660in}{5.076971in}}%
\pgfpathlineto{\pgfqpoint{8.130532in}{5.078301in}}%
\pgfusepath{stroke}%
\end{pgfscope}%
\begin{pgfscope}%
\pgfpathrectangle{\pgfqpoint{6.720588in}{4.155455in}}{\pgfqpoint{2.279412in}{2.004545in}}%
\pgfusepath{clip}%
\pgfsetbuttcap%
\pgfsetroundjoin%
\pgfsetlinewidth{1.169811pt}%
\definecolor{currentstroke}{rgb}{0.129933,0.559582,0.551864}%
\pgfsetstrokecolor{currentstroke}%
\pgfsetdash{}{0pt}%
\pgfpathmoveto{\pgfqpoint{8.130532in}{5.078301in}}%
\pgfpathlineto{\pgfqpoint{8.080427in}{5.080174in}}%
\pgfusepath{stroke}%
\end{pgfscope}%
\begin{pgfscope}%
\pgfpathrectangle{\pgfqpoint{6.720588in}{4.155455in}}{\pgfqpoint{2.279412in}{2.004545in}}%
\pgfusepath{clip}%
\pgfsetbuttcap%
\pgfsetroundjoin%
\pgfsetlinewidth{1.490603pt}%
\definecolor{currentstroke}{rgb}{0.226397,0.728888,0.462789}%
\pgfsetstrokecolor{currentstroke}%
\pgfsetdash{}{0pt}%
\pgfpathmoveto{\pgfqpoint{8.080427in}{5.080174in}}%
\pgfpathlineto{\pgfqpoint{8.030350in}{5.082567in}}%
\pgfusepath{stroke}%
\end{pgfscope}%
\begin{pgfscope}%
\pgfpathrectangle{\pgfqpoint{6.720588in}{4.155455in}}{\pgfqpoint{2.279412in}{2.004545in}}%
\pgfusepath{clip}%
\pgfsetbuttcap%
\pgfsetroundjoin%
\pgfsetlinewidth{1.820453pt}%
\definecolor{currentstroke}{rgb}{0.647257,0.858400,0.209861}%
\pgfsetstrokecolor{currentstroke}%
\pgfsetdash{}{0pt}%
\pgfpathmoveto{\pgfqpoint{8.030350in}{5.082567in}}%
\pgfpathlineto{\pgfqpoint{7.980316in}{5.085526in}}%
\pgfusepath{stroke}%
\end{pgfscope}%
\begin{pgfscope}%
\pgfpathrectangle{\pgfqpoint{6.720588in}{4.155455in}}{\pgfqpoint{2.279412in}{2.004545in}}%
\pgfusepath{clip}%
\pgfsetbuttcap%
\pgfsetroundjoin%
\pgfsetlinewidth{2.094366pt}%
\definecolor{currentstroke}{rgb}{0.993248,0.906157,0.143936}%
\pgfsetstrokecolor{currentstroke}%
\pgfsetdash{}{0pt}%
\pgfpathmoveto{\pgfqpoint{7.980316in}{5.085526in}}%
\pgfpathlineto{\pgfqpoint{7.930325in}{5.088999in}}%
\pgfusepath{stroke}%
\end{pgfscope}%
\begin{pgfscope}%
\pgfpathrectangle{\pgfqpoint{6.720588in}{4.155455in}}{\pgfqpoint{2.279412in}{2.004545in}}%
\pgfusepath{clip}%
\pgfsetbuttcap%
\pgfsetroundjoin%
\pgfsetlinewidth{2.189055pt}%
\definecolor{currentstroke}{rgb}{0.993248,0.906157,0.143936}%
\pgfsetstrokecolor{currentstroke}%
\pgfsetdash{}{0pt}%
\pgfpathmoveto{\pgfqpoint{7.930325in}{5.088999in}}%
\pgfpathlineto{\pgfqpoint{7.880369in}{5.092830in}}%
\pgfusepath{stroke}%
\end{pgfscope}%
\begin{pgfscope}%
\pgfpathrectangle{\pgfqpoint{6.720588in}{4.155455in}}{\pgfqpoint{2.279412in}{2.004545in}}%
\pgfusepath{clip}%
\pgfsetbuttcap%
\pgfsetroundjoin%
\pgfsetlinewidth{2.306179pt}%
\definecolor{currentstroke}{rgb}{0.993248,0.906157,0.143936}%
\pgfsetstrokecolor{currentstroke}%
\pgfsetdash{}{0pt}%
\pgfpathmoveto{\pgfqpoint{7.880369in}{5.092830in}}%
\pgfpathlineto{\pgfqpoint{7.830451in}{5.097033in}}%
\pgfusepath{stroke}%
\end{pgfscope}%
\begin{pgfscope}%
\pgfpathrectangle{\pgfqpoint{6.720588in}{4.155455in}}{\pgfqpoint{2.279412in}{2.004545in}}%
\pgfusepath{clip}%
\pgfsetbuttcap%
\pgfsetroundjoin%
\pgfsetlinewidth{2.369909pt}%
\definecolor{currentstroke}{rgb}{0.993248,0.906157,0.143936}%
\pgfsetstrokecolor{currentstroke}%
\pgfsetdash{}{0pt}%
\pgfpathmoveto{\pgfqpoint{7.830451in}{5.097033in}}%
\pgfpathlineto{\pgfqpoint{7.780569in}{5.101575in}}%
\pgfusepath{stroke}%
\end{pgfscope}%
\begin{pgfscope}%
\pgfpathrectangle{\pgfqpoint{6.720588in}{4.155455in}}{\pgfqpoint{2.279412in}{2.004545in}}%
\pgfusepath{clip}%
\pgfsetbuttcap%
\pgfsetroundjoin%
\pgfsetlinewidth{2.389047pt}%
\definecolor{currentstroke}{rgb}{0.993248,0.906157,0.143936}%
\pgfsetstrokecolor{currentstroke}%
\pgfsetdash{}{0pt}%
\pgfpathmoveto{\pgfqpoint{7.780569in}{5.101575in}}%
\pgfpathlineto{\pgfqpoint{7.730734in}{5.106441in}}%
\pgfusepath{stroke}%
\end{pgfscope}%
\begin{pgfscope}%
\pgfpathrectangle{\pgfqpoint{6.720588in}{4.155455in}}{\pgfqpoint{2.279412in}{2.004545in}}%
\pgfusepath{clip}%
\pgfsetbuttcap%
\pgfsetroundjoin%
\pgfsetlinewidth{2.338154pt}%
\definecolor{currentstroke}{rgb}{0.993248,0.906157,0.143936}%
\pgfsetstrokecolor{currentstroke}%
\pgfsetdash{}{0pt}%
\pgfpathmoveto{\pgfqpoint{7.730734in}{5.106441in}}%
\pgfpathlineto{\pgfqpoint{7.680948in}{5.111620in}}%
\pgfusepath{stroke}%
\end{pgfscope}%
\begin{pgfscope}%
\pgfpathrectangle{\pgfqpoint{6.720588in}{4.155455in}}{\pgfqpoint{2.279412in}{2.004545in}}%
\pgfusepath{clip}%
\pgfsetbuttcap%
\pgfsetroundjoin%
\pgfsetlinewidth{2.188187pt}%
\definecolor{currentstroke}{rgb}{0.993248,0.906157,0.143936}%
\pgfsetstrokecolor{currentstroke}%
\pgfsetdash{}{0pt}%
\pgfpathmoveto{\pgfqpoint{7.680948in}{5.111620in}}%
\pgfpathlineto{\pgfqpoint{7.631173in}{5.116897in}}%
\pgfusepath{stroke}%
\end{pgfscope}%
\begin{pgfscope}%
\pgfpathrectangle{\pgfqpoint{6.720588in}{4.155455in}}{\pgfqpoint{2.279412in}{2.004545in}}%
\pgfusepath{clip}%
\pgfsetbuttcap%
\pgfsetroundjoin%
\pgfsetlinewidth{0.319456pt}%
\definecolor{currentstroke}{rgb}{0.269944,0.014625,0.341379}%
\pgfsetstrokecolor{currentstroke}%
\pgfsetdash{}{0pt}%
\pgfpathmoveto{\pgfqpoint{8.680964in}{5.157727in}}%
\pgfpathlineto{\pgfqpoint{8.630832in}{5.158008in}}%
\pgfusepath{stroke}%
\end{pgfscope}%
\begin{pgfscope}%
\pgfpathrectangle{\pgfqpoint{6.720588in}{4.155455in}}{\pgfqpoint{2.279412in}{2.004545in}}%
\pgfusepath{clip}%
\pgfsetbuttcap%
\pgfsetroundjoin%
\pgfsetlinewidth{0.333371pt}%
\definecolor{currentstroke}{rgb}{0.272594,0.025563,0.353093}%
\pgfsetstrokecolor{currentstroke}%
\pgfsetdash{}{0pt}%
\pgfpathmoveto{\pgfqpoint{8.630832in}{5.158008in}}%
\pgfpathlineto{\pgfqpoint{8.580692in}{5.157241in}}%
\pgfusepath{stroke}%
\end{pgfscope}%
\begin{pgfscope}%
\pgfpathrectangle{\pgfqpoint{6.720588in}{4.155455in}}{\pgfqpoint{2.279412in}{2.004545in}}%
\pgfusepath{clip}%
\pgfsetbuttcap%
\pgfsetroundjoin%
\pgfsetlinewidth{0.332842pt}%
\definecolor{currentstroke}{rgb}{0.272594,0.025563,0.353093}%
\pgfsetstrokecolor{currentstroke}%
\pgfsetdash{}{0pt}%
\pgfpathmoveto{\pgfqpoint{8.580692in}{5.157241in}}%
\pgfpathlineto{\pgfqpoint{8.530551in}{5.157124in}}%
\pgfusepath{stroke}%
\end{pgfscope}%
\begin{pgfscope}%
\pgfpathrectangle{\pgfqpoint{6.720588in}{4.155455in}}{\pgfqpoint{2.279412in}{2.004545in}}%
\pgfusepath{clip}%
\pgfsetbuttcap%
\pgfsetroundjoin%
\pgfsetlinewidth{0.344839pt}%
\definecolor{currentstroke}{rgb}{0.274952,0.037752,0.364543}%
\pgfsetstrokecolor{currentstroke}%
\pgfsetdash{}{0pt}%
\pgfpathmoveto{\pgfqpoint{8.530551in}{5.157124in}}%
\pgfpathlineto{\pgfqpoint{8.480403in}{5.157550in}}%
\pgfusepath{stroke}%
\end{pgfscope}%
\begin{pgfscope}%
\pgfpathrectangle{\pgfqpoint{6.720588in}{4.155455in}}{\pgfqpoint{2.279412in}{2.004545in}}%
\pgfusepath{clip}%
\pgfsetbuttcap%
\pgfsetroundjoin%
\pgfsetlinewidth{0.368345pt}%
\definecolor{currentstroke}{rgb}{0.277941,0.056324,0.381191}%
\pgfsetstrokecolor{currentstroke}%
\pgfsetdash{}{0pt}%
\pgfpathmoveto{\pgfqpoint{8.480403in}{5.157550in}}%
\pgfpathlineto{\pgfqpoint{8.430253in}{5.157764in}}%
\pgfusepath{stroke}%
\end{pgfscope}%
\begin{pgfscope}%
\pgfpathrectangle{\pgfqpoint{6.720588in}{4.155455in}}{\pgfqpoint{2.279412in}{2.004545in}}%
\pgfusepath{clip}%
\pgfsetbuttcap%
\pgfsetroundjoin%
\pgfsetlinewidth{0.394473pt}%
\definecolor{currentstroke}{rgb}{0.280894,0.078907,0.402329}%
\pgfsetstrokecolor{currentstroke}%
\pgfsetdash{}{0pt}%
\pgfpathmoveto{\pgfqpoint{8.430253in}{5.157764in}}%
\pgfpathlineto{\pgfqpoint{8.380108in}{5.158379in}}%
\pgfusepath{stroke}%
\end{pgfscope}%
\begin{pgfscope}%
\pgfpathrectangle{\pgfqpoint{6.720588in}{4.155455in}}{\pgfqpoint{2.279412in}{2.004545in}}%
\pgfusepath{clip}%
\pgfsetbuttcap%
\pgfsetroundjoin%
\pgfsetlinewidth{0.445773pt}%
\definecolor{currentstroke}{rgb}{0.283229,0.120777,0.440584}%
\pgfsetstrokecolor{currentstroke}%
\pgfsetdash{}{0pt}%
\pgfpathmoveto{\pgfqpoint{8.380108in}{5.158379in}}%
\pgfpathlineto{\pgfqpoint{8.329965in}{5.159121in}}%
\pgfusepath{stroke}%
\end{pgfscope}%
\begin{pgfscope}%
\pgfpathrectangle{\pgfqpoint{6.720588in}{4.155455in}}{\pgfqpoint{2.279412in}{2.004545in}}%
\pgfusepath{clip}%
\pgfsetbuttcap%
\pgfsetroundjoin%
\pgfsetlinewidth{0.506978pt}%
\definecolor{currentstroke}{rgb}{0.280255,0.165693,0.476498}%
\pgfsetstrokecolor{currentstroke}%
\pgfsetdash{}{0pt}%
\pgfpathmoveto{\pgfqpoint{8.329965in}{5.159121in}}%
\pgfpathlineto{\pgfqpoint{8.279815in}{5.159381in}}%
\pgfusepath{stroke}%
\end{pgfscope}%
\begin{pgfscope}%
\pgfpathrectangle{\pgfqpoint{6.720588in}{4.155455in}}{\pgfqpoint{2.279412in}{2.004545in}}%
\pgfusepath{clip}%
\pgfsetbuttcap%
\pgfsetroundjoin%
\pgfsetlinewidth{0.629521pt}%
\definecolor{currentstroke}{rgb}{0.260571,0.246922,0.522828}%
\pgfsetstrokecolor{currentstroke}%
\pgfsetdash{}{0pt}%
\pgfpathmoveto{\pgfqpoint{8.279815in}{5.159381in}}%
\pgfpathlineto{\pgfqpoint{8.229663in}{5.159357in}}%
\pgfusepath{stroke}%
\end{pgfscope}%
\begin{pgfscope}%
\pgfpathrectangle{\pgfqpoint{6.720588in}{4.155455in}}{\pgfqpoint{2.279412in}{2.004545in}}%
\pgfusepath{clip}%
\pgfsetbuttcap%
\pgfsetroundjoin%
\pgfsetlinewidth{0.746714pt}%
\definecolor{currentstroke}{rgb}{0.229739,0.322361,0.545706}%
\pgfsetstrokecolor{currentstroke}%
\pgfsetdash{}{0pt}%
\pgfpathmoveto{\pgfqpoint{8.229663in}{5.159357in}}%
\pgfpathlineto{\pgfqpoint{8.179511in}{5.159322in}}%
\pgfusepath{stroke}%
\end{pgfscope}%
\begin{pgfscope}%
\pgfpathrectangle{\pgfqpoint{6.720588in}{4.155455in}}{\pgfqpoint{2.279412in}{2.004545in}}%
\pgfusepath{clip}%
\pgfsetbuttcap%
\pgfsetroundjoin%
\pgfsetlinewidth{0.963546pt}%
\definecolor{currentstroke}{rgb}{0.172719,0.448791,0.557885}%
\pgfsetstrokecolor{currentstroke}%
\pgfsetdash{}{0pt}%
\pgfpathmoveto{\pgfqpoint{8.179511in}{5.159322in}}%
\pgfpathlineto{\pgfqpoint{8.129360in}{5.159248in}}%
\pgfusepath{stroke}%
\end{pgfscope}%
\begin{pgfscope}%
\pgfpathrectangle{\pgfqpoint{6.720588in}{4.155455in}}{\pgfqpoint{2.279412in}{2.004545in}}%
\pgfusepath{clip}%
\pgfsetbuttcap%
\pgfsetroundjoin%
\pgfsetlinewidth{1.321034pt}%
\definecolor{currentstroke}{rgb}{0.124780,0.640461,0.527068}%
\pgfsetstrokecolor{currentstroke}%
\pgfsetdash{}{0pt}%
\pgfpathmoveto{\pgfqpoint{8.129360in}{5.159248in}}%
\pgfpathlineto{\pgfqpoint{8.079208in}{5.159036in}}%
\pgfusepath{stroke}%
\end{pgfscope}%
\begin{pgfscope}%
\pgfpathrectangle{\pgfqpoint{6.720588in}{4.155455in}}{\pgfqpoint{2.279412in}{2.004545in}}%
\pgfusepath{clip}%
\pgfsetbuttcap%
\pgfsetroundjoin%
\pgfsetlinewidth{1.637098pt}%
\definecolor{currentstroke}{rgb}{0.386433,0.794644,0.372886}%
\pgfsetstrokecolor{currentstroke}%
\pgfsetdash{}{0pt}%
\pgfpathmoveto{\pgfqpoint{8.079208in}{5.159036in}}%
\pgfpathlineto{\pgfqpoint{8.029058in}{5.158682in}}%
\pgfusepath{stroke}%
\end{pgfscope}%
\begin{pgfscope}%
\pgfpathrectangle{\pgfqpoint{6.720588in}{4.155455in}}{\pgfqpoint{2.279412in}{2.004545in}}%
\pgfusepath{clip}%
\pgfsetbuttcap%
\pgfsetroundjoin%
\pgfsetlinewidth{2.064650pt}%
\definecolor{currentstroke}{rgb}{0.993248,0.906157,0.143936}%
\pgfsetstrokecolor{currentstroke}%
\pgfsetdash{}{0pt}%
\pgfpathmoveto{\pgfqpoint{8.029058in}{5.158682in}}%
\pgfpathlineto{\pgfqpoint{7.978909in}{5.158286in}}%
\pgfusepath{stroke}%
\end{pgfscope}%
\begin{pgfscope}%
\pgfpathrectangle{\pgfqpoint{6.720588in}{4.155455in}}{\pgfqpoint{2.279412in}{2.004545in}}%
\pgfusepath{clip}%
\pgfsetbuttcap%
\pgfsetroundjoin%
\pgfsetlinewidth{2.228268pt}%
\definecolor{currentstroke}{rgb}{0.993248,0.906157,0.143936}%
\pgfsetstrokecolor{currentstroke}%
\pgfsetdash{}{0pt}%
\pgfpathmoveto{\pgfqpoint{7.978909in}{5.158286in}}%
\pgfpathlineto{\pgfqpoint{7.928759in}{5.157924in}}%
\pgfusepath{stroke}%
\end{pgfscope}%
\begin{pgfscope}%
\pgfpathrectangle{\pgfqpoint{6.720588in}{4.155455in}}{\pgfqpoint{2.279412in}{2.004545in}}%
\pgfusepath{clip}%
\pgfsetbuttcap%
\pgfsetroundjoin%
\pgfsetlinewidth{2.349178pt}%
\definecolor{currentstroke}{rgb}{0.993248,0.906157,0.143936}%
\pgfsetstrokecolor{currentstroke}%
\pgfsetdash{}{0pt}%
\pgfpathmoveto{\pgfqpoint{7.928759in}{5.157924in}}%
\pgfpathlineto{\pgfqpoint{7.878609in}{5.157500in}}%
\pgfusepath{stroke}%
\end{pgfscope}%
\begin{pgfscope}%
\pgfpathrectangle{\pgfqpoint{6.720588in}{4.155455in}}{\pgfqpoint{2.279412in}{2.004545in}}%
\pgfusepath{clip}%
\pgfsetbuttcap%
\pgfsetroundjoin%
\pgfsetlinewidth{2.449764pt}%
\definecolor{currentstroke}{rgb}{0.993248,0.906157,0.143936}%
\pgfsetstrokecolor{currentstroke}%
\pgfsetdash{}{0pt}%
\pgfpathmoveto{\pgfqpoint{7.878609in}{5.157500in}}%
\pgfpathlineto{\pgfqpoint{7.828461in}{5.157030in}}%
\pgfusepath{stroke}%
\end{pgfscope}%
\begin{pgfscope}%
\pgfpathrectangle{\pgfqpoint{6.720588in}{4.155455in}}{\pgfqpoint{2.279412in}{2.004545in}}%
\pgfusepath{clip}%
\pgfsetbuttcap%
\pgfsetroundjoin%
\pgfsetlinewidth{2.547319pt}%
\definecolor{currentstroke}{rgb}{0.993248,0.906157,0.143936}%
\pgfsetstrokecolor{currentstroke}%
\pgfsetdash{}{0pt}%
\pgfpathmoveto{\pgfqpoint{7.828461in}{5.157030in}}%
\pgfpathlineto{\pgfqpoint{7.778314in}{5.156540in}}%
\pgfusepath{stroke}%
\end{pgfscope}%
\begin{pgfscope}%
\pgfpathrectangle{\pgfqpoint{6.720588in}{4.155455in}}{\pgfqpoint{2.279412in}{2.004545in}}%
\pgfusepath{clip}%
\pgfsetbuttcap%
\pgfsetroundjoin%
\pgfsetlinewidth{2.487285pt}%
\definecolor{currentstroke}{rgb}{0.993248,0.906157,0.143936}%
\pgfsetstrokecolor{currentstroke}%
\pgfsetdash{}{0pt}%
\pgfpathmoveto{\pgfqpoint{7.778314in}{5.156540in}}%
\pgfpathlineto{\pgfqpoint{7.728170in}{5.156009in}}%
\pgfusepath{stroke}%
\end{pgfscope}%
\begin{pgfscope}%
\pgfpathrectangle{\pgfqpoint{6.720588in}{4.155455in}}{\pgfqpoint{2.279412in}{2.004545in}}%
\pgfusepath{clip}%
\pgfsetbuttcap%
\pgfsetroundjoin%
\pgfsetlinewidth{2.340463pt}%
\definecolor{currentstroke}{rgb}{0.993248,0.906157,0.143936}%
\pgfsetstrokecolor{currentstroke}%
\pgfsetdash{}{0pt}%
\pgfpathmoveto{\pgfqpoint{7.728170in}{5.156009in}}%
\pgfpathlineto{\pgfqpoint{7.678027in}{5.155581in}}%
\pgfusepath{stroke}%
\end{pgfscope}%
\begin{pgfscope}%
\pgfpathrectangle{\pgfqpoint{6.720588in}{4.155455in}}{\pgfqpoint{2.279412in}{2.004545in}}%
\pgfusepath{clip}%
\pgfsetbuttcap%
\pgfsetroundjoin%
\pgfsetlinewidth{2.179625pt}%
\definecolor{currentstroke}{rgb}{0.993248,0.906157,0.143936}%
\pgfsetstrokecolor{currentstroke}%
\pgfsetdash{}{0pt}%
\pgfpathmoveto{\pgfqpoint{7.678027in}{5.155581in}}%
\pgfpathlineto{\pgfqpoint{7.627892in}{5.155089in}}%
\pgfusepath{stroke}%
\end{pgfscope}%
\begin{pgfscope}%
\pgfpathrectangle{\pgfqpoint{6.720588in}{4.155455in}}{\pgfqpoint{2.279412in}{2.004545in}}%
\pgfusepath{clip}%
\pgfsetbuttcap%
\pgfsetroundjoin%
\pgfsetlinewidth{0.323545pt}%
\definecolor{currentstroke}{rgb}{0.271305,0.019942,0.347269}%
\pgfsetstrokecolor{currentstroke}%
\pgfsetdash{}{0pt}%
\pgfpathmoveto{\pgfqpoint{8.680964in}{5.202834in}}%
\pgfpathlineto{\pgfqpoint{8.630887in}{5.201602in}}%
\pgfusepath{stroke}%
\end{pgfscope}%
\begin{pgfscope}%
\pgfpathrectangle{\pgfqpoint{6.720588in}{4.155455in}}{\pgfqpoint{2.279412in}{2.004545in}}%
\pgfusepath{clip}%
\pgfsetbuttcap%
\pgfsetroundjoin%
\pgfsetlinewidth{0.329608pt}%
\definecolor{currentstroke}{rgb}{0.272594,0.025563,0.353093}%
\pgfsetstrokecolor{currentstroke}%
\pgfsetdash{}{0pt}%
\pgfpathmoveto{\pgfqpoint{8.630887in}{5.201602in}}%
\pgfpathlineto{\pgfqpoint{8.580745in}{5.201453in}}%
\pgfusepath{stroke}%
\end{pgfscope}%
\begin{pgfscope}%
\pgfpathrectangle{\pgfqpoint{6.720588in}{4.155455in}}{\pgfqpoint{2.279412in}{2.004545in}}%
\pgfusepath{clip}%
\pgfsetbuttcap%
\pgfsetroundjoin%
\pgfsetlinewidth{0.350223pt}%
\definecolor{currentstroke}{rgb}{0.276022,0.044167,0.370164}%
\pgfsetstrokecolor{currentstroke}%
\pgfsetdash{}{0pt}%
\pgfpathmoveto{\pgfqpoint{8.580745in}{5.201453in}}%
\pgfpathlineto{\pgfqpoint{8.530604in}{5.201549in}}%
\pgfusepath{stroke}%
\end{pgfscope}%
\begin{pgfscope}%
\pgfpathrectangle{\pgfqpoint{6.720588in}{4.155455in}}{\pgfqpoint{2.279412in}{2.004545in}}%
\pgfusepath{clip}%
\pgfsetbuttcap%
\pgfsetroundjoin%
\pgfsetlinewidth{0.345792pt}%
\definecolor{currentstroke}{rgb}{0.274952,0.037752,0.364543}%
\pgfsetstrokecolor{currentstroke}%
\pgfsetdash{}{0pt}%
\pgfpathmoveto{\pgfqpoint{8.530604in}{5.201549in}}%
\pgfpathlineto{\pgfqpoint{8.480462in}{5.201579in}}%
\pgfusepath{stroke}%
\end{pgfscope}%
\begin{pgfscope}%
\pgfpathrectangle{\pgfqpoint{6.720588in}{4.155455in}}{\pgfqpoint{2.279412in}{2.004545in}}%
\pgfusepath{clip}%
\pgfsetbuttcap%
\pgfsetroundjoin%
\pgfsetlinewidth{0.373848pt}%
\definecolor{currentstroke}{rgb}{0.278791,0.062145,0.386592}%
\pgfsetstrokecolor{currentstroke}%
\pgfsetdash{}{0pt}%
\pgfpathmoveto{\pgfqpoint{8.480462in}{5.201579in}}%
\pgfpathlineto{\pgfqpoint{8.430313in}{5.201551in}}%
\pgfusepath{stroke}%
\end{pgfscope}%
\begin{pgfscope}%
\pgfpathrectangle{\pgfqpoint{6.720588in}{4.155455in}}{\pgfqpoint{2.279412in}{2.004545in}}%
\pgfusepath{clip}%
\pgfsetbuttcap%
\pgfsetroundjoin%
\pgfsetlinewidth{0.398029pt}%
\definecolor{currentstroke}{rgb}{0.281446,0.084320,0.407414}%
\pgfsetstrokecolor{currentstroke}%
\pgfsetdash{}{0pt}%
\pgfpathmoveto{\pgfqpoint{8.430313in}{5.201551in}}%
\pgfpathlineto{\pgfqpoint{8.380162in}{5.201396in}}%
\pgfusepath{stroke}%
\end{pgfscope}%
\begin{pgfscope}%
\pgfpathrectangle{\pgfqpoint{6.720588in}{4.155455in}}{\pgfqpoint{2.279412in}{2.004545in}}%
\pgfusepath{clip}%
\pgfsetbuttcap%
\pgfsetroundjoin%
\pgfsetlinewidth{0.449112pt}%
\definecolor{currentstroke}{rgb}{0.283229,0.120777,0.440584}%
\pgfsetstrokecolor{currentstroke}%
\pgfsetdash{}{0pt}%
\pgfpathmoveto{\pgfqpoint{8.380162in}{5.201396in}}%
\pgfpathlineto{\pgfqpoint{8.330012in}{5.201088in}}%
\pgfusepath{stroke}%
\end{pgfscope}%
\begin{pgfscope}%
\pgfpathrectangle{\pgfqpoint{6.720588in}{4.155455in}}{\pgfqpoint{2.279412in}{2.004545in}}%
\pgfusepath{clip}%
\pgfsetbuttcap%
\pgfsetroundjoin%
\pgfsetlinewidth{0.509566pt}%
\definecolor{currentstroke}{rgb}{0.280255,0.165693,0.476498}%
\pgfsetstrokecolor{currentstroke}%
\pgfsetdash{}{0pt}%
\pgfpathmoveto{\pgfqpoint{8.330012in}{5.201088in}}%
\pgfpathlineto{\pgfqpoint{8.279863in}{5.200632in}}%
\pgfusepath{stroke}%
\end{pgfscope}%
\begin{pgfscope}%
\pgfpathrectangle{\pgfqpoint{6.720588in}{4.155455in}}{\pgfqpoint{2.279412in}{2.004545in}}%
\pgfusepath{clip}%
\pgfsetbuttcap%
\pgfsetroundjoin%
\pgfsetlinewidth{0.591869pt}%
\definecolor{currentstroke}{rgb}{0.267968,0.223549,0.512008}%
\pgfsetstrokecolor{currentstroke}%
\pgfsetdash{}{0pt}%
\pgfpathmoveto{\pgfqpoint{8.279863in}{5.200632in}}%
\pgfpathlineto{\pgfqpoint{8.229715in}{5.200099in}}%
\pgfusepath{stroke}%
\end{pgfscope}%
\begin{pgfscope}%
\pgfpathrectangle{\pgfqpoint{6.720588in}{4.155455in}}{\pgfqpoint{2.279412in}{2.004545in}}%
\pgfusepath{clip}%
\pgfsetbuttcap%
\pgfsetroundjoin%
\pgfsetlinewidth{0.740481pt}%
\definecolor{currentstroke}{rgb}{0.229739,0.322361,0.545706}%
\pgfsetstrokecolor{currentstroke}%
\pgfsetdash{}{0pt}%
\pgfpathmoveto{\pgfqpoint{8.229715in}{5.200099in}}%
\pgfpathlineto{\pgfqpoint{8.179570in}{5.199352in}}%
\pgfusepath{stroke}%
\end{pgfscope}%
\begin{pgfscope}%
\pgfpathrectangle{\pgfqpoint{6.720588in}{4.155455in}}{\pgfqpoint{2.279412in}{2.004545in}}%
\pgfusepath{clip}%
\pgfsetbuttcap%
\pgfsetroundjoin%
\pgfsetlinewidth{0.937432pt}%
\definecolor{currentstroke}{rgb}{0.179019,0.433756,0.557430}%
\pgfsetstrokecolor{currentstroke}%
\pgfsetdash{}{0pt}%
\pgfpathmoveto{\pgfqpoint{8.179570in}{5.199352in}}%
\pgfpathlineto{\pgfqpoint{8.129434in}{5.198279in}}%
\pgfusepath{stroke}%
\end{pgfscope}%
\begin{pgfscope}%
\pgfpathrectangle{\pgfqpoint{6.720588in}{4.155455in}}{\pgfqpoint{2.279412in}{2.004545in}}%
\pgfusepath{clip}%
\pgfsetbuttcap%
\pgfsetroundjoin%
\pgfsetlinewidth{1.206388pt}%
\definecolor{currentstroke}{rgb}{0.124395,0.578002,0.548287}%
\pgfsetstrokecolor{currentstroke}%
\pgfsetdash{}{0pt}%
\pgfpathmoveto{\pgfqpoint{8.129434in}{5.198279in}}%
\pgfpathlineto{\pgfqpoint{8.079308in}{5.196866in}}%
\pgfusepath{stroke}%
\end{pgfscope}%
\begin{pgfscope}%
\pgfpathrectangle{\pgfqpoint{6.720588in}{4.155455in}}{\pgfqpoint{2.279412in}{2.004545in}}%
\pgfusepath{clip}%
\pgfsetbuttcap%
\pgfsetroundjoin%
\pgfsetlinewidth{1.572263pt}%
\definecolor{currentstroke}{rgb}{0.311925,0.767822,0.415586}%
\pgfsetstrokecolor{currentstroke}%
\pgfsetdash{}{0pt}%
\pgfpathmoveto{\pgfqpoint{8.079308in}{5.196866in}}%
\pgfpathlineto{\pgfqpoint{8.029199in}{5.195058in}}%
\pgfusepath{stroke}%
\end{pgfscope}%
\begin{pgfscope}%
\pgfpathrectangle{\pgfqpoint{6.720588in}{4.155455in}}{\pgfqpoint{2.279412in}{2.004545in}}%
\pgfusepath{clip}%
\pgfsetbuttcap%
\pgfsetroundjoin%
\pgfsetlinewidth{1.896808pt}%
\definecolor{currentstroke}{rgb}{0.762373,0.876424,0.137064}%
\pgfsetstrokecolor{currentstroke}%
\pgfsetdash{}{0pt}%
\pgfpathmoveto{\pgfqpoint{8.029199in}{5.195058in}}%
\pgfpathlineto{\pgfqpoint{7.979112in}{5.192862in}}%
\pgfusepath{stroke}%
\end{pgfscope}%
\begin{pgfscope}%
\pgfpathrectangle{\pgfqpoint{6.720588in}{4.155455in}}{\pgfqpoint{2.279412in}{2.004545in}}%
\pgfusepath{clip}%
\pgfsetbuttcap%
\pgfsetroundjoin%
\pgfsetlinewidth{2.136414pt}%
\definecolor{currentstroke}{rgb}{0.993248,0.906157,0.143936}%
\pgfsetstrokecolor{currentstroke}%
\pgfsetdash{}{0pt}%
\pgfpathmoveto{\pgfqpoint{7.979112in}{5.192862in}}%
\pgfpathlineto{\pgfqpoint{7.929047in}{5.190319in}}%
\pgfusepath{stroke}%
\end{pgfscope}%
\begin{pgfscope}%
\pgfpathrectangle{\pgfqpoint{6.720588in}{4.155455in}}{\pgfqpoint{2.279412in}{2.004545in}}%
\pgfusepath{clip}%
\pgfsetbuttcap%
\pgfsetroundjoin%
\pgfsetlinewidth{0.316801pt}%
\definecolor{currentstroke}{rgb}{0.269944,0.014625,0.341379}%
\pgfsetstrokecolor{currentstroke}%
\pgfsetdash{}{0pt}%
\pgfpathmoveto{\pgfqpoint{8.680964in}{5.338154in}}%
\pgfpathlineto{\pgfqpoint{8.631013in}{5.339550in}}%
\pgfusepath{stroke}%
\end{pgfscope}%
\begin{pgfscope}%
\pgfpathrectangle{\pgfqpoint{6.720588in}{4.155455in}}{\pgfqpoint{2.279412in}{2.004545in}}%
\pgfusepath{clip}%
\pgfsetbuttcap%
\pgfsetroundjoin%
\pgfsetlinewidth{0.328058pt}%
\definecolor{currentstroke}{rgb}{0.271305,0.019942,0.347269}%
\pgfsetstrokecolor{currentstroke}%
\pgfsetdash{}{0pt}%
\pgfpathmoveto{\pgfqpoint{8.631013in}{5.339550in}}%
\pgfpathlineto{\pgfqpoint{8.580876in}{5.338803in}}%
\pgfusepath{stroke}%
\end{pgfscope}%
\begin{pgfscope}%
\pgfpathrectangle{\pgfqpoint{6.720588in}{4.155455in}}{\pgfqpoint{2.279412in}{2.004545in}}%
\pgfusepath{clip}%
\pgfsetbuttcap%
\pgfsetroundjoin%
\pgfsetlinewidth{0.329268pt}%
\definecolor{currentstroke}{rgb}{0.272594,0.025563,0.353093}%
\pgfsetstrokecolor{currentstroke}%
\pgfsetdash{}{0pt}%
\pgfpathmoveto{\pgfqpoint{8.580876in}{5.338803in}}%
\pgfpathlineto{\pgfqpoint{8.530755in}{5.337396in}}%
\pgfusepath{stroke}%
\end{pgfscope}%
\begin{pgfscope}%
\pgfpathrectangle{\pgfqpoint{6.720588in}{4.155455in}}{\pgfqpoint{2.279412in}{2.004545in}}%
\pgfusepath{clip}%
\pgfsetbuttcap%
\pgfsetroundjoin%
\pgfsetlinewidth{0.337879pt}%
\definecolor{currentstroke}{rgb}{0.273809,0.031497,0.358853}%
\pgfsetstrokecolor{currentstroke}%
\pgfsetdash{}{0pt}%
\pgfpathmoveto{\pgfqpoint{8.530755in}{5.337396in}}%
\pgfpathlineto{\pgfqpoint{8.480628in}{5.336102in}}%
\pgfusepath{stroke}%
\end{pgfscope}%
\begin{pgfscope}%
\pgfpathrectangle{\pgfqpoint{6.720588in}{4.155455in}}{\pgfqpoint{2.279412in}{2.004545in}}%
\pgfusepath{clip}%
\pgfsetbuttcap%
\pgfsetroundjoin%
\pgfsetlinewidth{0.353350pt}%
\definecolor{currentstroke}{rgb}{0.276022,0.044167,0.370164}%
\pgfsetstrokecolor{currentstroke}%
\pgfsetdash{}{0pt}%
\pgfpathmoveto{\pgfqpoint{8.480628in}{5.336102in}}%
\pgfpathlineto{\pgfqpoint{8.430486in}{5.335278in}}%
\pgfusepath{stroke}%
\end{pgfscope}%
\begin{pgfscope}%
\pgfpathrectangle{\pgfqpoint{6.720588in}{4.155455in}}{\pgfqpoint{2.279412in}{2.004545in}}%
\pgfusepath{clip}%
\pgfsetbuttcap%
\pgfsetroundjoin%
\pgfsetlinewidth{0.384662pt}%
\definecolor{currentstroke}{rgb}{0.280267,0.073417,0.397163}%
\pgfsetstrokecolor{currentstroke}%
\pgfsetdash{}{0pt}%
\pgfpathmoveto{\pgfqpoint{8.430486in}{5.335278in}}%
\pgfpathlineto{\pgfqpoint{8.380352in}{5.334174in}}%
\pgfusepath{stroke}%
\end{pgfscope}%
\begin{pgfscope}%
\pgfpathrectangle{\pgfqpoint{6.720588in}{4.155455in}}{\pgfqpoint{2.279412in}{2.004545in}}%
\pgfusepath{clip}%
\pgfsetbuttcap%
\pgfsetroundjoin%
\pgfsetlinewidth{0.422772pt}%
\definecolor{currentstroke}{rgb}{0.282656,0.100196,0.422160}%
\pgfsetstrokecolor{currentstroke}%
\pgfsetdash{}{0pt}%
\pgfpathmoveto{\pgfqpoint{8.380352in}{5.334174in}}%
\pgfpathlineto{\pgfqpoint{8.330226in}{5.332789in}}%
\pgfusepath{stroke}%
\end{pgfscope}%
\begin{pgfscope}%
\pgfpathrectangle{\pgfqpoint{6.720588in}{4.155455in}}{\pgfqpoint{2.279412in}{2.004545in}}%
\pgfusepath{clip}%
\pgfsetbuttcap%
\pgfsetroundjoin%
\pgfsetlinewidth{0.466906pt}%
\definecolor{currentstroke}{rgb}{0.282884,0.135920,0.453427}%
\pgfsetstrokecolor{currentstroke}%
\pgfsetdash{}{0pt}%
\pgfpathmoveto{\pgfqpoint{8.330226in}{5.332789in}}%
\pgfpathlineto{\pgfqpoint{8.280103in}{5.331295in}}%
\pgfusepath{stroke}%
\end{pgfscope}%
\begin{pgfscope}%
\pgfpathrectangle{\pgfqpoint{6.720588in}{4.155455in}}{\pgfqpoint{2.279412in}{2.004545in}}%
\pgfusepath{clip}%
\pgfsetbuttcap%
\pgfsetroundjoin%
\pgfsetlinewidth{0.523430pt}%
\definecolor{currentstroke}{rgb}{0.278826,0.175490,0.483397}%
\pgfsetstrokecolor{currentstroke}%
\pgfsetdash{}{0pt}%
\pgfpathmoveto{\pgfqpoint{8.280103in}{5.331295in}}%
\pgfpathlineto{\pgfqpoint{8.229998in}{5.329411in}}%
\pgfusepath{stroke}%
\end{pgfscope}%
\begin{pgfscope}%
\pgfpathrectangle{\pgfqpoint{6.720588in}{4.155455in}}{\pgfqpoint{2.279412in}{2.004545in}}%
\pgfusepath{clip}%
\pgfsetbuttcap%
\pgfsetroundjoin%
\pgfsetlinewidth{0.588780pt}%
\definecolor{currentstroke}{rgb}{0.269308,0.218818,0.509577}%
\pgfsetstrokecolor{currentstroke}%
\pgfsetdash{}{0pt}%
\pgfpathmoveto{\pgfqpoint{8.229998in}{5.329411in}}%
\pgfpathlineto{\pgfqpoint{8.179921in}{5.327023in}}%
\pgfusepath{stroke}%
\end{pgfscope}%
\begin{pgfscope}%
\pgfpathrectangle{\pgfqpoint{6.720588in}{4.155455in}}{\pgfqpoint{2.279412in}{2.004545in}}%
\pgfusepath{clip}%
\pgfsetbuttcap%
\pgfsetroundjoin%
\pgfsetlinewidth{0.695193pt}%
\definecolor{currentstroke}{rgb}{0.243113,0.292092,0.538516}%
\pgfsetstrokecolor{currentstroke}%
\pgfsetdash{}{0pt}%
\pgfpathmoveto{\pgfqpoint{8.179921in}{5.327023in}}%
\pgfpathlineto{\pgfqpoint{8.129891in}{5.323988in}}%
\pgfusepath{stroke}%
\end{pgfscope}%
\begin{pgfscope}%
\pgfpathrectangle{\pgfqpoint{6.720588in}{4.155455in}}{\pgfqpoint{2.279412in}{2.004545in}}%
\pgfusepath{clip}%
\pgfsetbuttcap%
\pgfsetroundjoin%
\pgfsetlinewidth{0.726318pt}%
\definecolor{currentstroke}{rgb}{0.235526,0.309527,0.542944}%
\pgfsetstrokecolor{currentstroke}%
\pgfsetdash{}{0pt}%
\pgfpathmoveto{\pgfqpoint{8.129891in}{5.323988in}}%
\pgfpathlineto{\pgfqpoint{8.079947in}{5.320020in}}%
\pgfusepath{stroke}%
\end{pgfscope}%
\begin{pgfscope}%
\pgfpathrectangle{\pgfqpoint{6.720588in}{4.155455in}}{\pgfqpoint{2.279412in}{2.004545in}}%
\pgfusepath{clip}%
\pgfsetbuttcap%
\pgfsetroundjoin%
\pgfsetlinewidth{0.307436pt}%
\definecolor{currentstroke}{rgb}{0.267004,0.004874,0.329415}%
\pgfsetstrokecolor{currentstroke}%
\pgfsetdash{}{0pt}%
\pgfpathmoveto{\pgfqpoint{8.629673in}{4.796873in}}%
\pgfpathlineto{\pgfqpoint{8.581614in}{4.797026in}}%
\pgfusepath{stroke}%
\end{pgfscope}%
\begin{pgfscope}%
\pgfpathrectangle{\pgfqpoint{6.720588in}{4.155455in}}{\pgfqpoint{2.279412in}{2.004545in}}%
\pgfusepath{clip}%
\pgfsetbuttcap%
\pgfsetroundjoin%
\pgfsetlinewidth{0.318523pt}%
\definecolor{currentstroke}{rgb}{0.269944,0.014625,0.341379}%
\pgfsetstrokecolor{currentstroke}%
\pgfsetdash{}{0pt}%
\pgfpathmoveto{\pgfqpoint{8.581614in}{4.797026in}}%
\pgfpathlineto{\pgfqpoint{8.538883in}{4.799762in}}%
\pgfusepath{stroke}%
\end{pgfscope}%
\begin{pgfscope}%
\pgfpathrectangle{\pgfqpoint{6.720588in}{4.155455in}}{\pgfqpoint{2.279412in}{2.004545in}}%
\pgfusepath{clip}%
\pgfsetbuttcap%
\pgfsetroundjoin%
\pgfsetlinewidth{0.331069pt}%
\definecolor{currentstroke}{rgb}{0.272594,0.025563,0.353093}%
\pgfsetstrokecolor{currentstroke}%
\pgfsetdash{}{0pt}%
\pgfpathmoveto{\pgfqpoint{8.538883in}{4.799762in}}%
\pgfpathlineto{\pgfqpoint{8.488807in}{4.801509in}}%
\pgfusepath{stroke}%
\end{pgfscope}%
\begin{pgfscope}%
\pgfpathrectangle{\pgfqpoint{6.720588in}{4.155455in}}{\pgfqpoint{2.279412in}{2.004545in}}%
\pgfusepath{clip}%
\pgfsetbuttcap%
\pgfsetroundjoin%
\pgfsetlinewidth{0.330209pt}%
\definecolor{currentstroke}{rgb}{0.272594,0.025563,0.353093}%
\pgfsetstrokecolor{currentstroke}%
\pgfsetdash{}{0pt}%
\pgfpathmoveto{\pgfqpoint{8.488807in}{4.801509in}}%
\pgfpathlineto{\pgfqpoint{8.438795in}{4.804558in}}%
\pgfusepath{stroke}%
\end{pgfscope}%
\begin{pgfscope}%
\pgfpathrectangle{\pgfqpoint{6.720588in}{4.155455in}}{\pgfqpoint{2.279412in}{2.004545in}}%
\pgfusepath{clip}%
\pgfsetbuttcap%
\pgfsetroundjoin%
\pgfsetlinewidth{0.342854pt}%
\definecolor{currentstroke}{rgb}{0.274952,0.037752,0.364543}%
\pgfsetstrokecolor{currentstroke}%
\pgfsetdash{}{0pt}%
\pgfpathmoveto{\pgfqpoint{8.438795in}{4.804558in}}%
\pgfpathlineto{\pgfqpoint{8.388741in}{4.807034in}}%
\pgfusepath{stroke}%
\end{pgfscope}%
\begin{pgfscope}%
\pgfpathrectangle{\pgfqpoint{6.720588in}{4.155455in}}{\pgfqpoint{2.279412in}{2.004545in}}%
\pgfusepath{clip}%
\pgfsetbuttcap%
\pgfsetroundjoin%
\pgfsetlinewidth{0.367776pt}%
\definecolor{currentstroke}{rgb}{0.277941,0.056324,0.381191}%
\pgfsetstrokecolor{currentstroke}%
\pgfsetdash{}{0pt}%
\pgfpathmoveto{\pgfqpoint{8.388741in}{4.807034in}}%
\pgfpathlineto{\pgfqpoint{8.338664in}{4.809404in}}%
\pgfusepath{stroke}%
\end{pgfscope}%
\begin{pgfscope}%
\pgfpathrectangle{\pgfqpoint{6.720588in}{4.155455in}}{\pgfqpoint{2.279412in}{2.004545in}}%
\pgfusepath{clip}%
\pgfsetbuttcap%
\pgfsetroundjoin%
\pgfsetlinewidth{0.389477pt}%
\definecolor{currentstroke}{rgb}{0.280267,0.073417,0.397163}%
\pgfsetstrokecolor{currentstroke}%
\pgfsetdash{}{0pt}%
\pgfpathmoveto{\pgfqpoint{8.338664in}{4.809404in}}%
\pgfpathlineto{\pgfqpoint{8.288613in}{4.812180in}}%
\pgfusepath{stroke}%
\end{pgfscope}%
\begin{pgfscope}%
\pgfpathrectangle{\pgfqpoint{6.720588in}{4.155455in}}{\pgfqpoint{2.279412in}{2.004545in}}%
\pgfusepath{clip}%
\pgfsetbuttcap%
\pgfsetroundjoin%
\pgfsetlinewidth{0.427987pt}%
\definecolor{currentstroke}{rgb}{0.282910,0.105393,0.426902}%
\pgfsetstrokecolor{currentstroke}%
\pgfsetdash{}{0pt}%
\pgfpathmoveto{\pgfqpoint{8.288613in}{4.812180in}}%
\pgfpathlineto{\pgfqpoint{8.238606in}{4.815505in}}%
\pgfusepath{stroke}%
\end{pgfscope}%
\begin{pgfscope}%
\pgfpathrectangle{\pgfqpoint{6.720588in}{4.155455in}}{\pgfqpoint{2.279412in}{2.004545in}}%
\pgfusepath{clip}%
\pgfsetbuttcap%
\pgfsetroundjoin%
\pgfsetlinewidth{0.454375pt}%
\definecolor{currentstroke}{rgb}{0.283187,0.125848,0.444960}%
\pgfsetstrokecolor{currentstroke}%
\pgfsetdash{}{0pt}%
\pgfpathmoveto{\pgfqpoint{8.238606in}{4.815505in}}%
\pgfpathlineto{\pgfqpoint{8.188659in}{4.819429in}}%
\pgfusepath{stroke}%
\end{pgfscope}%
\begin{pgfscope}%
\pgfpathrectangle{\pgfqpoint{6.720588in}{4.155455in}}{\pgfqpoint{2.279412in}{2.004545in}}%
\pgfusepath{clip}%
\pgfsetbuttcap%
\pgfsetroundjoin%
\pgfsetlinewidth{0.463344pt}%
\definecolor{currentstroke}{rgb}{0.283072,0.130895,0.449241}%
\pgfsetstrokecolor{currentstroke}%
\pgfsetdash{}{0pt}%
\pgfpathmoveto{\pgfqpoint{8.188659in}{4.819429in}}%
\pgfpathlineto{\pgfqpoint{8.138786in}{4.824012in}}%
\pgfusepath{stroke}%
\end{pgfscope}%
\begin{pgfscope}%
\pgfpathrectangle{\pgfqpoint{6.720588in}{4.155455in}}{\pgfqpoint{2.279412in}{2.004545in}}%
\pgfusepath{clip}%
\pgfsetbuttcap%
\pgfsetroundjoin%
\pgfsetlinewidth{0.506488pt}%
\definecolor{currentstroke}{rgb}{0.280868,0.160771,0.472899}%
\pgfsetstrokecolor{currentstroke}%
\pgfsetdash{}{0pt}%
\pgfpathmoveto{\pgfqpoint{8.138786in}{4.824012in}}%
\pgfpathlineto{\pgfqpoint{8.089100in}{4.829892in}}%
\pgfusepath{stroke}%
\end{pgfscope}%
\begin{pgfscope}%
\pgfpathrectangle{\pgfqpoint{6.720588in}{4.155455in}}{\pgfqpoint{2.279412in}{2.004545in}}%
\pgfusepath{clip}%
\pgfsetbuttcap%
\pgfsetroundjoin%
\pgfsetlinewidth{0.530222pt}%
\definecolor{currentstroke}{rgb}{0.278012,0.180367,0.486697}%
\pgfsetstrokecolor{currentstroke}%
\pgfsetdash{}{0pt}%
\pgfpathmoveto{\pgfqpoint{8.089100in}{4.829892in}}%
\pgfpathlineto{\pgfqpoint{8.039616in}{4.836992in}}%
\pgfusepath{stroke}%
\end{pgfscope}%
\begin{pgfscope}%
\pgfpathrectangle{\pgfqpoint{6.720588in}{4.155455in}}{\pgfqpoint{2.279412in}{2.004545in}}%
\pgfusepath{clip}%
\pgfsetbuttcap%
\pgfsetroundjoin%
\pgfsetlinewidth{0.490843pt}%
\definecolor{currentstroke}{rgb}{0.281887,0.150881,0.465405}%
\pgfsetstrokecolor{currentstroke}%
\pgfsetdash{}{0pt}%
\pgfpathmoveto{\pgfqpoint{8.039616in}{4.836992in}}%
\pgfpathlineto{\pgfqpoint{7.990513in}{4.845819in}}%
\pgfusepath{stroke}%
\end{pgfscope}%
\begin{pgfscope}%
\pgfpathrectangle{\pgfqpoint{6.720588in}{4.155455in}}{\pgfqpoint{2.279412in}{2.004545in}}%
\pgfusepath{clip}%
\pgfsetbuttcap%
\pgfsetroundjoin%
\pgfsetlinewidth{0.548378pt}%
\definecolor{currentstroke}{rgb}{0.275191,0.194905,0.496005}%
\pgfsetstrokecolor{currentstroke}%
\pgfsetdash{}{0pt}%
\pgfpathmoveto{\pgfqpoint{7.990513in}{4.845819in}}%
\pgfpathlineto{\pgfqpoint{7.942008in}{4.856965in}}%
\pgfusepath{stroke}%
\end{pgfscope}%
\begin{pgfscope}%
\pgfpathrectangle{\pgfqpoint{6.720588in}{4.155455in}}{\pgfqpoint{2.279412in}{2.004545in}}%
\pgfusepath{clip}%
\pgfsetbuttcap%
\pgfsetroundjoin%
\pgfsetlinewidth{0.522149pt}%
\definecolor{currentstroke}{rgb}{0.278826,0.175490,0.483397}%
\pgfsetstrokecolor{currentstroke}%
\pgfsetdash{}{0pt}%
\pgfpathmoveto{\pgfqpoint{7.942008in}{4.856965in}}%
\pgfpathlineto{\pgfqpoint{7.894123in}{4.870018in}}%
\pgfusepath{stroke}%
\end{pgfscope}%
\begin{pgfscope}%
\pgfpathrectangle{\pgfqpoint{6.720588in}{4.155455in}}{\pgfqpoint{2.279412in}{2.004545in}}%
\pgfusepath{clip}%
\pgfsetbuttcap%
\pgfsetroundjoin%
\pgfsetlinewidth{0.659858pt}%
\definecolor{currentstroke}{rgb}{0.252194,0.269783,0.531579}%
\pgfsetstrokecolor{currentstroke}%
\pgfsetdash{}{0pt}%
\pgfpathmoveto{\pgfqpoint{7.894123in}{4.870018in}}%
\pgfpathlineto{\pgfqpoint{7.847710in}{4.886518in}}%
\pgfusepath{stroke}%
\end{pgfscope}%
\begin{pgfscope}%
\pgfpathrectangle{\pgfqpoint{6.720588in}{4.155455in}}{\pgfqpoint{2.279412in}{2.004545in}}%
\pgfusepath{clip}%
\pgfsetbuttcap%
\pgfsetroundjoin%
\pgfsetlinewidth{0.674110pt}%
\definecolor{currentstroke}{rgb}{0.248629,0.278775,0.534556}%
\pgfsetstrokecolor{currentstroke}%
\pgfsetdash{}{0pt}%
\pgfpathmoveto{\pgfqpoint{7.847710in}{4.886518in}}%
\pgfpathlineto{\pgfqpoint{7.803358in}{4.906650in}}%
\pgfusepath{stroke}%
\end{pgfscope}%
\begin{pgfscope}%
\pgfpathrectangle{\pgfqpoint{6.720588in}{4.155455in}}{\pgfqpoint{2.279412in}{2.004545in}}%
\pgfusepath{clip}%
\pgfsetbuttcap%
\pgfsetroundjoin%
\pgfsetlinewidth{0.545389pt}%
\definecolor{currentstroke}{rgb}{0.276194,0.190074,0.493001}%
\pgfsetstrokecolor{currentstroke}%
\pgfsetdash{}{0pt}%
\pgfpathmoveto{\pgfqpoint{7.803358in}{4.906650in}}%
\pgfpathlineto{\pgfqpoint{7.759809in}{4.928147in}}%
\pgfusepath{stroke}%
\end{pgfscope}%
\begin{pgfscope}%
\pgfpathrectangle{\pgfqpoint{6.720588in}{4.155455in}}{\pgfqpoint{2.279412in}{2.004545in}}%
\pgfusepath{clip}%
\pgfsetbuttcap%
\pgfsetroundjoin%
\pgfsetlinewidth{1.295505pt}%
\definecolor{currentstroke}{rgb}{0.120638,0.625828,0.533488}%
\pgfsetstrokecolor{currentstroke}%
\pgfsetdash{}{0pt}%
\pgfpathmoveto{\pgfqpoint{7.759809in}{4.928147in}}%
\pgfpathlineto{\pgfqpoint{7.716871in}{4.950881in}}%
\pgfusepath{stroke}%
\end{pgfscope}%
\begin{pgfscope}%
\pgfpathrectangle{\pgfqpoint{6.720588in}{4.155455in}}{\pgfqpoint{2.279412in}{2.004545in}}%
\pgfusepath{clip}%
\pgfsetbuttcap%
\pgfsetroundjoin%
\pgfsetlinewidth{1.482175pt}%
\definecolor{currentstroke}{rgb}{0.220124,0.725509,0.466226}%
\pgfsetstrokecolor{currentstroke}%
\pgfsetdash{}{0pt}%
\pgfpathmoveto{\pgfqpoint{7.716871in}{4.950881in}}%
\pgfpathlineto{\pgfqpoint{7.675235in}{4.975388in}}%
\pgfusepath{stroke}%
\end{pgfscope}%
\begin{pgfscope}%
\pgfpathrectangle{\pgfqpoint{6.720588in}{4.155455in}}{\pgfqpoint{2.279412in}{2.004545in}}%
\pgfusepath{clip}%
\pgfsetbuttcap%
\pgfsetroundjoin%
\pgfsetlinewidth{1.400540pt}%
\definecolor{currentstroke}{rgb}{0.157851,0.683765,0.501686}%
\pgfsetstrokecolor{currentstroke}%
\pgfsetdash{}{0pt}%
\pgfpathmoveto{\pgfqpoint{7.675235in}{4.975388in}}%
\pgfpathlineto{\pgfqpoint{7.634535in}{5.001114in}}%
\pgfusepath{stroke}%
\end{pgfscope}%
\begin{pgfscope}%
\pgfpathrectangle{\pgfqpoint{6.720588in}{4.155455in}}{\pgfqpoint{2.279412in}{2.004545in}}%
\pgfusepath{clip}%
\pgfsetbuttcap%
\pgfsetroundjoin%
\pgfsetlinewidth{1.534453pt}%
\definecolor{currentstroke}{rgb}{0.266941,0.748751,0.440573}%
\pgfsetstrokecolor{currentstroke}%
\pgfsetdash{}{0pt}%
\pgfpathmoveto{\pgfqpoint{7.634535in}{5.001114in}}%
\pgfpathlineto{\pgfqpoint{7.592831in}{5.025458in}}%
\pgfusepath{stroke}%
\end{pgfscope}%
\begin{pgfscope}%
\pgfpathrectangle{\pgfqpoint{6.720588in}{4.155455in}}{\pgfqpoint{2.279412in}{2.004545in}}%
\pgfusepath{clip}%
\pgfsetbuttcap%
\pgfsetroundjoin%
\pgfsetlinewidth{0.325777pt}%
\definecolor{currentstroke}{rgb}{0.271305,0.019942,0.347269}%
\pgfsetstrokecolor{currentstroke}%
\pgfsetdash{}{0pt}%
\pgfpathmoveto{\pgfqpoint{8.629673in}{5.112620in}}%
\pgfpathlineto{\pgfqpoint{8.579804in}{5.115288in}}%
\pgfusepath{stroke}%
\end{pgfscope}%
\begin{pgfscope}%
\pgfpathrectangle{\pgfqpoint{6.720588in}{4.155455in}}{\pgfqpoint{2.279412in}{2.004545in}}%
\pgfusepath{clip}%
\pgfsetbuttcap%
\pgfsetroundjoin%
\pgfsetlinewidth{0.315741pt}%
\definecolor{currentstroke}{rgb}{0.269944,0.014625,0.341379}%
\pgfsetstrokecolor{currentstroke}%
\pgfsetdash{}{0pt}%
\pgfpathmoveto{\pgfqpoint{8.579804in}{5.115288in}}%
\pgfpathlineto{\pgfqpoint{8.529878in}{5.117183in}}%
\pgfusepath{stroke}%
\end{pgfscope}%
\begin{pgfscope}%
\pgfpathrectangle{\pgfqpoint{6.720588in}{4.155455in}}{\pgfqpoint{2.279412in}{2.004545in}}%
\pgfusepath{clip}%
\pgfsetbuttcap%
\pgfsetroundjoin%
\pgfsetlinewidth{0.348984pt}%
\definecolor{currentstroke}{rgb}{0.274952,0.037752,0.364543}%
\pgfsetstrokecolor{currentstroke}%
\pgfsetdash{}{0pt}%
\pgfpathmoveto{\pgfqpoint{8.529878in}{5.117183in}}%
\pgfpathlineto{\pgfqpoint{8.479729in}{5.116951in}}%
\pgfusepath{stroke}%
\end{pgfscope}%
\begin{pgfscope}%
\pgfpathrectangle{\pgfqpoint{6.720588in}{4.155455in}}{\pgfqpoint{2.279412in}{2.004545in}}%
\pgfusepath{clip}%
\pgfsetbuttcap%
\pgfsetroundjoin%
\pgfsetlinewidth{0.373856pt}%
\definecolor{currentstroke}{rgb}{0.278791,0.062145,0.386592}%
\pgfsetstrokecolor{currentstroke}%
\pgfsetdash{}{0pt}%
\pgfpathmoveto{\pgfqpoint{8.479729in}{5.116951in}}%
\pgfpathlineto{\pgfqpoint{8.429581in}{5.116829in}}%
\pgfusepath{stroke}%
\end{pgfscope}%
\begin{pgfscope}%
\pgfpathrectangle{\pgfqpoint{6.720588in}{4.155455in}}{\pgfqpoint{2.279412in}{2.004545in}}%
\pgfusepath{clip}%
\pgfsetbuttcap%
\pgfsetroundjoin%
\pgfsetlinewidth{0.398590pt}%
\definecolor{currentstroke}{rgb}{0.281446,0.084320,0.407414}%
\pgfsetstrokecolor{currentstroke}%
\pgfsetdash{}{0pt}%
\pgfpathmoveto{\pgfqpoint{8.429581in}{5.116829in}}%
\pgfpathlineto{\pgfqpoint{8.379432in}{5.117217in}}%
\pgfusepath{stroke}%
\end{pgfscope}%
\begin{pgfscope}%
\pgfpathrectangle{\pgfqpoint{6.720588in}{4.155455in}}{\pgfqpoint{2.279412in}{2.004545in}}%
\pgfusepath{clip}%
\pgfsetbuttcap%
\pgfsetroundjoin%
\pgfsetlinewidth{0.454951pt}%
\definecolor{currentstroke}{rgb}{0.283187,0.125848,0.444960}%
\pgfsetstrokecolor{currentstroke}%
\pgfsetdash{}{0pt}%
\pgfpathmoveto{\pgfqpoint{8.379432in}{5.117217in}}%
\pgfpathlineto{\pgfqpoint{8.329282in}{5.117490in}}%
\pgfusepath{stroke}%
\end{pgfscope}%
\begin{pgfscope}%
\pgfpathrectangle{\pgfqpoint{6.720588in}{4.155455in}}{\pgfqpoint{2.279412in}{2.004545in}}%
\pgfusepath{clip}%
\pgfsetbuttcap%
\pgfsetroundjoin%
\pgfsetlinewidth{0.510605pt}%
\definecolor{currentstroke}{rgb}{0.280255,0.165693,0.476498}%
\pgfsetstrokecolor{currentstroke}%
\pgfsetdash{}{0pt}%
\pgfpathmoveto{\pgfqpoint{8.329282in}{5.117490in}}%
\pgfpathlineto{\pgfqpoint{8.279130in}{5.117648in}}%
\pgfusepath{stroke}%
\end{pgfscope}%
\begin{pgfscope}%
\pgfpathrectangle{\pgfqpoint{6.720588in}{4.155455in}}{\pgfqpoint{2.279412in}{2.004545in}}%
\pgfusepath{clip}%
\pgfsetbuttcap%
\pgfsetroundjoin%
\pgfsetlinewidth{0.593142pt}%
\definecolor{currentstroke}{rgb}{0.267968,0.223549,0.512008}%
\pgfsetstrokecolor{currentstroke}%
\pgfsetdash{}{0pt}%
\pgfpathmoveto{\pgfqpoint{8.279130in}{5.117648in}}%
\pgfpathlineto{\pgfqpoint{8.228979in}{5.117877in}}%
\pgfusepath{stroke}%
\end{pgfscope}%
\begin{pgfscope}%
\pgfpathrectangle{\pgfqpoint{6.720588in}{4.155455in}}{\pgfqpoint{2.279412in}{2.004545in}}%
\pgfusepath{clip}%
\pgfsetbuttcap%
\pgfsetroundjoin%
\pgfsetlinewidth{0.761016pt}%
\definecolor{currentstroke}{rgb}{0.223925,0.334994,0.548053}%
\pgfsetstrokecolor{currentstroke}%
\pgfsetdash{}{0pt}%
\pgfpathmoveto{\pgfqpoint{8.228979in}{5.117877in}}%
\pgfpathlineto{\pgfqpoint{8.178828in}{5.118156in}}%
\pgfusepath{stroke}%
\end{pgfscope}%
\begin{pgfscope}%
\pgfpathrectangle{\pgfqpoint{6.720588in}{4.155455in}}{\pgfqpoint{2.279412in}{2.004545in}}%
\pgfusepath{clip}%
\pgfsetbuttcap%
\pgfsetroundjoin%
\pgfsetlinewidth{0.961852pt}%
\definecolor{currentstroke}{rgb}{0.172719,0.448791,0.557885}%
\pgfsetstrokecolor{currentstroke}%
\pgfsetdash{}{0pt}%
\pgfpathmoveto{\pgfqpoint{8.178828in}{5.118156in}}%
\pgfpathlineto{\pgfqpoint{8.128680in}{5.118673in}}%
\pgfusepath{stroke}%
\end{pgfscope}%
\begin{pgfscope}%
\pgfpathrectangle{\pgfqpoint{6.720588in}{4.155455in}}{\pgfqpoint{2.279412in}{2.004545in}}%
\pgfusepath{clip}%
\pgfsetbuttcap%
\pgfsetroundjoin%
\pgfsetlinewidth{1.262428pt}%
\definecolor{currentstroke}{rgb}{0.119423,0.611141,0.538982}%
\pgfsetstrokecolor{currentstroke}%
\pgfsetdash{}{0pt}%
\pgfpathmoveto{\pgfqpoint{8.128680in}{5.118673in}}%
\pgfpathlineto{\pgfqpoint{8.078536in}{5.119447in}}%
\pgfusepath{stroke}%
\end{pgfscope}%
\begin{pgfscope}%
\pgfpathrectangle{\pgfqpoint{6.720588in}{4.155455in}}{\pgfqpoint{2.279412in}{2.004545in}}%
\pgfusepath{clip}%
\pgfsetbuttcap%
\pgfsetroundjoin%
\pgfsetlinewidth{1.632054pt}%
\definecolor{currentstroke}{rgb}{0.377779,0.791781,0.377939}%
\pgfsetstrokecolor{currentstroke}%
\pgfsetdash{}{0pt}%
\pgfpathmoveto{\pgfqpoint{8.078536in}{5.119447in}}%
\pgfpathlineto{\pgfqpoint{8.028398in}{5.120474in}}%
\pgfusepath{stroke}%
\end{pgfscope}%
\begin{pgfscope}%
\pgfpathrectangle{\pgfqpoint{6.720588in}{4.155455in}}{\pgfqpoint{2.279412in}{2.004545in}}%
\pgfusepath{clip}%
\pgfsetbuttcap%
\pgfsetroundjoin%
\pgfsetlinewidth{1.981989pt}%
\definecolor{currentstroke}{rgb}{0.886271,0.892374,0.095374}%
\pgfsetstrokecolor{currentstroke}%
\pgfsetdash{}{0pt}%
\pgfpathmoveto{\pgfqpoint{8.028398in}{5.120474in}}%
\pgfpathlineto{\pgfqpoint{7.978270in}{5.121794in}}%
\pgfusepath{stroke}%
\end{pgfscope}%
\begin{pgfscope}%
\pgfpathrectangle{\pgfqpoint{6.720588in}{4.155455in}}{\pgfqpoint{2.279412in}{2.004545in}}%
\pgfusepath{clip}%
\pgfsetbuttcap%
\pgfsetroundjoin%
\pgfsetlinewidth{2.180273pt}%
\definecolor{currentstroke}{rgb}{0.993248,0.906157,0.143936}%
\pgfsetstrokecolor{currentstroke}%
\pgfsetdash{}{0pt}%
\pgfpathmoveto{\pgfqpoint{7.978270in}{5.121794in}}%
\pgfpathlineto{\pgfqpoint{7.928154in}{5.123381in}}%
\pgfusepath{stroke}%
\end{pgfscope}%
\begin{pgfscope}%
\pgfpathrectangle{\pgfqpoint{6.720588in}{4.155455in}}{\pgfqpoint{2.279412in}{2.004545in}}%
\pgfusepath{clip}%
\pgfsetbuttcap%
\pgfsetroundjoin%
\pgfsetlinewidth{0.314399pt}%
\definecolor{currentstroke}{rgb}{0.268510,0.009605,0.335427}%
\pgfsetstrokecolor{currentstroke}%
\pgfsetdash{}{0pt}%
\pgfpathmoveto{\pgfqpoint{8.629673in}{5.428368in}}%
\pgfpathlineto{\pgfqpoint{8.579644in}{5.426530in}}%
\pgfusepath{stroke}%
\end{pgfscope}%
\begin{pgfscope}%
\pgfpathrectangle{\pgfqpoint{6.720588in}{4.155455in}}{\pgfqpoint{2.279412in}{2.004545in}}%
\pgfusepath{clip}%
\pgfsetbuttcap%
\pgfsetroundjoin%
\pgfsetlinewidth{0.324575pt}%
\definecolor{currentstroke}{rgb}{0.271305,0.019942,0.347269}%
\pgfsetstrokecolor{currentstroke}%
\pgfsetdash{}{0pt}%
\pgfpathmoveto{\pgfqpoint{8.579644in}{5.426530in}}%
\pgfpathlineto{\pgfqpoint{8.529557in}{5.424533in}}%
\pgfusepath{stroke}%
\end{pgfscope}%
\begin{pgfscope}%
\pgfpathrectangle{\pgfqpoint{6.720588in}{4.155455in}}{\pgfqpoint{2.279412in}{2.004545in}}%
\pgfusepath{clip}%
\pgfsetbuttcap%
\pgfsetroundjoin%
\pgfsetlinewidth{0.332560pt}%
\definecolor{currentstroke}{rgb}{0.272594,0.025563,0.353093}%
\pgfsetstrokecolor{currentstroke}%
\pgfsetdash{}{0pt}%
\pgfpathmoveto{\pgfqpoint{8.529557in}{5.424533in}}%
\pgfpathlineto{\pgfqpoint{8.479434in}{5.423425in}}%
\pgfusepath{stroke}%
\end{pgfscope}%
\begin{pgfscope}%
\pgfpathrectangle{\pgfqpoint{6.720588in}{4.155455in}}{\pgfqpoint{2.279412in}{2.004545in}}%
\pgfusepath{clip}%
\pgfsetbuttcap%
\pgfsetroundjoin%
\pgfsetlinewidth{0.350807pt}%
\definecolor{currentstroke}{rgb}{0.276022,0.044167,0.370164}%
\pgfsetstrokecolor{currentstroke}%
\pgfsetdash{}{0pt}%
\pgfpathmoveto{\pgfqpoint{8.479434in}{5.423425in}}%
\pgfpathlineto{\pgfqpoint{8.429302in}{5.422439in}}%
\pgfusepath{stroke}%
\end{pgfscope}%
\begin{pgfscope}%
\pgfpathrectangle{\pgfqpoint{6.720588in}{4.155455in}}{\pgfqpoint{2.279412in}{2.004545in}}%
\pgfusepath{clip}%
\pgfsetbuttcap%
\pgfsetroundjoin%
\pgfsetlinewidth{0.371469pt}%
\definecolor{currentstroke}{rgb}{0.278791,0.062145,0.386592}%
\pgfsetstrokecolor{currentstroke}%
\pgfsetdash{}{0pt}%
\pgfpathmoveto{\pgfqpoint{8.429302in}{5.422439in}}%
\pgfpathlineto{\pgfqpoint{8.379203in}{5.420614in}}%
\pgfusepath{stroke}%
\end{pgfscope}%
\begin{pgfscope}%
\pgfpathrectangle{\pgfqpoint{6.720588in}{4.155455in}}{\pgfqpoint{2.279412in}{2.004545in}}%
\pgfusepath{clip}%
\pgfsetbuttcap%
\pgfsetroundjoin%
\pgfsetlinewidth{0.401446pt}%
\definecolor{currentstroke}{rgb}{0.281446,0.084320,0.407414}%
\pgfsetstrokecolor{currentstroke}%
\pgfsetdash{}{0pt}%
\pgfpathmoveto{\pgfqpoint{8.379203in}{5.420614in}}%
\pgfpathlineto{\pgfqpoint{8.329134in}{5.418105in}}%
\pgfusepath{stroke}%
\end{pgfscope}%
\begin{pgfscope}%
\pgfpathrectangle{\pgfqpoint{6.720588in}{4.155455in}}{\pgfqpoint{2.279412in}{2.004545in}}%
\pgfusepath{clip}%
\pgfsetbuttcap%
\pgfsetroundjoin%
\pgfsetlinewidth{0.417860pt}%
\definecolor{currentstroke}{rgb}{0.282656,0.100196,0.422160}%
\pgfsetstrokecolor{currentstroke}%
\pgfsetdash{}{0pt}%
\pgfpathmoveto{\pgfqpoint{8.329134in}{5.418105in}}%
\pgfpathlineto{\pgfqpoint{8.279073in}{5.415467in}}%
\pgfusepath{stroke}%
\end{pgfscope}%
\begin{pgfscope}%
\pgfpathrectangle{\pgfqpoint{6.720588in}{4.155455in}}{\pgfqpoint{2.279412in}{2.004545in}}%
\pgfusepath{clip}%
\pgfsetbuttcap%
\pgfsetroundjoin%
\pgfsetlinewidth{0.471943pt}%
\definecolor{currentstroke}{rgb}{0.282884,0.135920,0.453427}%
\pgfsetstrokecolor{currentstroke}%
\pgfsetdash{}{0pt}%
\pgfpathmoveto{\pgfqpoint{8.279073in}{5.415467in}}%
\pgfpathlineto{\pgfqpoint{8.229032in}{5.412537in}}%
\pgfusepath{stroke}%
\end{pgfscope}%
\begin{pgfscope}%
\pgfpathrectangle{\pgfqpoint{6.720588in}{4.155455in}}{\pgfqpoint{2.279412in}{2.004545in}}%
\pgfusepath{clip}%
\pgfsetbuttcap%
\pgfsetroundjoin%
\pgfsetlinewidth{0.491934pt}%
\definecolor{currentstroke}{rgb}{0.281887,0.150881,0.465405}%
\pgfsetstrokecolor{currentstroke}%
\pgfsetdash{}{0pt}%
\pgfpathmoveto{\pgfqpoint{8.229032in}{5.412537in}}%
\pgfpathlineto{\pgfqpoint{8.179045in}{5.408994in}}%
\pgfusepath{stroke}%
\end{pgfscope}%
\begin{pgfscope}%
\pgfpathrectangle{\pgfqpoint{6.720588in}{4.155455in}}{\pgfqpoint{2.279412in}{2.004545in}}%
\pgfusepath{clip}%
\pgfsetbuttcap%
\pgfsetroundjoin%
\pgfsetlinewidth{0.556807pt}%
\definecolor{currentstroke}{rgb}{0.274128,0.199721,0.498911}%
\pgfsetstrokecolor{currentstroke}%
\pgfsetdash{}{0pt}%
\pgfpathmoveto{\pgfqpoint{8.179045in}{5.408994in}}%
\pgfpathlineto{\pgfqpoint{8.129130in}{5.404731in}}%
\pgfusepath{stroke}%
\end{pgfscope}%
\begin{pgfscope}%
\pgfpathrectangle{\pgfqpoint{6.720588in}{4.155455in}}{\pgfqpoint{2.279412in}{2.004545in}}%
\pgfusepath{clip}%
\pgfsetbuttcap%
\pgfsetroundjoin%
\pgfsetlinewidth{0.629860pt}%
\definecolor{currentstroke}{rgb}{0.260571,0.246922,0.522828}%
\pgfsetstrokecolor{currentstroke}%
\pgfsetdash{}{0pt}%
\pgfpathmoveto{\pgfqpoint{8.129130in}{5.404731in}}%
\pgfpathlineto{\pgfqpoint{8.079325in}{5.399607in}}%
\pgfusepath{stroke}%
\end{pgfscope}%
\begin{pgfscope}%
\pgfpathrectangle{\pgfqpoint{6.720588in}{4.155455in}}{\pgfqpoint{2.279412in}{2.004545in}}%
\pgfusepath{clip}%
\pgfsetbuttcap%
\pgfsetroundjoin%
\pgfsetlinewidth{0.607965pt}%
\definecolor{currentstroke}{rgb}{0.265145,0.232956,0.516599}%
\pgfsetstrokecolor{currentstroke}%
\pgfsetdash{}{0pt}%
\pgfpathmoveto{\pgfqpoint{8.079325in}{5.399607in}}%
\pgfpathlineto{\pgfqpoint{8.029723in}{5.393171in}}%
\pgfusepath{stroke}%
\end{pgfscope}%
\begin{pgfscope}%
\pgfpathrectangle{\pgfqpoint{6.720588in}{4.155455in}}{\pgfqpoint{2.279412in}{2.004545in}}%
\pgfusepath{clip}%
\pgfsetbuttcap%
\pgfsetroundjoin%
\pgfsetlinewidth{0.638447pt}%
\definecolor{currentstroke}{rgb}{0.257322,0.256130,0.526563}%
\pgfsetstrokecolor{currentstroke}%
\pgfsetdash{}{0pt}%
\pgfpathmoveto{\pgfqpoint{8.029723in}{5.393171in}}%
\pgfpathlineto{\pgfqpoint{7.980388in}{5.385319in}}%
\pgfusepath{stroke}%
\end{pgfscope}%
\begin{pgfscope}%
\pgfpathrectangle{\pgfqpoint{6.720588in}{4.155455in}}{\pgfqpoint{2.279412in}{2.004545in}}%
\pgfusepath{clip}%
\pgfsetbuttcap%
\pgfsetroundjoin%
\pgfsetlinewidth{0.632013pt}%
\definecolor{currentstroke}{rgb}{0.258965,0.251537,0.524736}%
\pgfsetstrokecolor{currentstroke}%
\pgfsetdash{}{0pt}%
\pgfpathmoveto{\pgfqpoint{7.980388in}{5.385319in}}%
\pgfpathlineto{\pgfqpoint{7.931481in}{5.375649in}}%
\pgfusepath{stroke}%
\end{pgfscope}%
\begin{pgfscope}%
\pgfpathrectangle{\pgfqpoint{6.720588in}{4.155455in}}{\pgfqpoint{2.279412in}{2.004545in}}%
\pgfusepath{clip}%
\pgfsetbuttcap%
\pgfsetroundjoin%
\pgfsetlinewidth{0.804295pt}%
\definecolor{currentstroke}{rgb}{0.212395,0.359683,0.551710}%
\pgfsetstrokecolor{currentstroke}%
\pgfsetdash{}{0pt}%
\pgfpathmoveto{\pgfqpoint{7.931481in}{5.375649in}}%
\pgfpathlineto{\pgfqpoint{7.883234in}{5.363676in}}%
\pgfusepath{stroke}%
\end{pgfscope}%
\begin{pgfscope}%
\pgfpathrectangle{\pgfqpoint{6.720588in}{4.155455in}}{\pgfqpoint{2.279412in}{2.004545in}}%
\pgfusepath{clip}%
\pgfsetbuttcap%
\pgfsetroundjoin%
\pgfsetlinewidth{0.801582pt}%
\definecolor{currentstroke}{rgb}{0.214298,0.355619,0.551184}%
\pgfsetstrokecolor{currentstroke}%
\pgfsetdash{}{0pt}%
\pgfpathmoveto{\pgfqpoint{7.883234in}{5.363676in}}%
\pgfpathlineto{\pgfqpoint{7.835717in}{5.349616in}}%
\pgfusepath{stroke}%
\end{pgfscope}%
\begin{pgfscope}%
\pgfpathrectangle{\pgfqpoint{6.720588in}{4.155455in}}{\pgfqpoint{2.279412in}{2.004545in}}%
\pgfusepath{clip}%
\pgfsetbuttcap%
\pgfsetroundjoin%
\pgfsetlinewidth{1.141484pt}%
\definecolor{currentstroke}{rgb}{0.135066,0.544853,0.554029}%
\pgfsetstrokecolor{currentstroke}%
\pgfsetdash{}{0pt}%
\pgfpathmoveto{\pgfqpoint{7.835717in}{5.349616in}}%
\pgfpathlineto{\pgfqpoint{7.788940in}{5.333766in}}%
\pgfusepath{stroke}%
\end{pgfscope}%
\begin{pgfscope}%
\pgfpathrectangle{\pgfqpoint{6.720588in}{4.155455in}}{\pgfqpoint{2.279412in}{2.004545in}}%
\pgfusepath{clip}%
\pgfsetbuttcap%
\pgfsetroundjoin%
\pgfsetlinewidth{1.533022pt}%
\definecolor{currentstroke}{rgb}{0.266941,0.748751,0.440573}%
\pgfsetstrokecolor{currentstroke}%
\pgfsetdash{}{0pt}%
\pgfpathmoveto{\pgfqpoint{7.788940in}{5.333766in}}%
\pgfpathlineto{\pgfqpoint{7.742972in}{5.316172in}}%
\pgfusepath{stroke}%
\end{pgfscope}%
\begin{pgfscope}%
\pgfpathrectangle{\pgfqpoint{6.720588in}{4.155455in}}{\pgfqpoint{2.279412in}{2.004545in}}%
\pgfusepath{clip}%
\pgfsetbuttcap%
\pgfsetroundjoin%
\pgfsetlinewidth{1.706212pt}%
\definecolor{currentstroke}{rgb}{0.477504,0.821444,0.318195}%
\pgfsetstrokecolor{currentstroke}%
\pgfsetdash{}{0pt}%
\pgfpathmoveto{\pgfqpoint{7.742972in}{5.316172in}}%
\pgfpathlineto{\pgfqpoint{7.697652in}{5.297320in}}%
\pgfusepath{stroke}%
\end{pgfscope}%
\begin{pgfscope}%
\pgfpathrectangle{\pgfqpoint{6.720588in}{4.155455in}}{\pgfqpoint{2.279412in}{2.004545in}}%
\pgfusepath{clip}%
\pgfsetbuttcap%
\pgfsetroundjoin%
\pgfsetlinewidth{1.826763pt}%
\definecolor{currentstroke}{rgb}{0.657642,0.860219,0.203082}%
\pgfsetstrokecolor{currentstroke}%
\pgfsetdash{}{0pt}%
\pgfpathmoveto{\pgfqpoint{7.697652in}{5.297320in}}%
\pgfpathlineto{\pgfqpoint{7.652546in}{5.278095in}}%
\pgfusepath{stroke}%
\end{pgfscope}%
\begin{pgfscope}%
\pgfpathrectangle{\pgfqpoint{6.720588in}{4.155455in}}{\pgfqpoint{2.279412in}{2.004545in}}%
\pgfusepath{clip}%
\pgfsetbuttcap%
\pgfsetroundjoin%
\pgfsetlinewidth{2.084222pt}%
\definecolor{currentstroke}{rgb}{0.993248,0.906157,0.143936}%
\pgfsetstrokecolor{currentstroke}%
\pgfsetdash{}{0pt}%
\pgfpathmoveto{\pgfqpoint{7.652546in}{5.278095in}}%
\pgfpathlineto{\pgfqpoint{7.607320in}{5.259232in}}%
\pgfusepath{stroke}%
\end{pgfscope}%
\begin{pgfscope}%
\pgfpathrectangle{\pgfqpoint{6.720588in}{4.155455in}}{\pgfqpoint{2.279412in}{2.004545in}}%
\pgfusepath{clip}%
\pgfsetbuttcap%
\pgfsetroundjoin%
\pgfsetlinewidth{1.798254pt}%
\definecolor{currentstroke}{rgb}{0.616293,0.852709,0.230052}%
\pgfsetstrokecolor{currentstroke}%
\pgfsetdash{}{0pt}%
\pgfpathmoveto{\pgfqpoint{7.607320in}{5.259232in}}%
\pgfpathlineto{\pgfqpoint{7.562156in}{5.240253in}}%
\pgfusepath{stroke}%
\end{pgfscope}%
\begin{pgfscope}%
\pgfpathrectangle{\pgfqpoint{6.720588in}{4.155455in}}{\pgfqpoint{2.279412in}{2.004545in}}%
\pgfusepath{clip}%
\pgfsetbuttcap%
\pgfsetroundjoin%
\pgfsetlinewidth{2.003580pt}%
\definecolor{currentstroke}{rgb}{0.926106,0.897330,0.104071}%
\pgfsetstrokecolor{currentstroke}%
\pgfsetdash{}{0pt}%
\pgfpathmoveto{\pgfqpoint{7.562156in}{5.240253in}}%
\pgfpathlineto{\pgfqpoint{7.517083in}{5.221011in}}%
\pgfusepath{stroke}%
\end{pgfscope}%
\begin{pgfscope}%
\pgfpathrectangle{\pgfqpoint{6.720588in}{4.155455in}}{\pgfqpoint{2.279412in}{2.004545in}}%
\pgfusepath{clip}%
\pgfsetbuttcap%
\pgfsetroundjoin%
\pgfsetlinewidth{1.788611pt}%
\definecolor{currentstroke}{rgb}{0.595839,0.848717,0.243329}%
\pgfsetstrokecolor{currentstroke}%
\pgfsetdash{}{0pt}%
\pgfpathmoveto{\pgfqpoint{7.517083in}{5.221011in}}%
\pgfpathlineto{\pgfqpoint{7.471804in}{5.202205in}}%
\pgfusepath{stroke}%
\end{pgfscope}%
\begin{pgfscope}%
\pgfpathrectangle{\pgfqpoint{6.720588in}{4.155455in}}{\pgfqpoint{2.279412in}{2.004545in}}%
\pgfusepath{clip}%
\pgfsetbuttcap%
\pgfsetroundjoin%
\pgfsetlinewidth{1.398741pt}%
\definecolor{currentstroke}{rgb}{0.153894,0.680203,0.504172}%
\pgfsetstrokecolor{currentstroke}%
\pgfsetdash{}{0pt}%
\pgfpathmoveto{\pgfqpoint{7.471804in}{5.202205in}}%
\pgfpathlineto{\pgfqpoint{7.426131in}{5.184281in}}%
\pgfusepath{stroke}%
\end{pgfscope}%
\begin{pgfscope}%
\pgfpathrectangle{\pgfqpoint{6.720588in}{4.155455in}}{\pgfqpoint{2.279412in}{2.004545in}}%
\pgfusepath{clip}%
\pgfsetbuttcap%
\pgfsetroundjoin%
\pgfsetlinewidth{0.313881pt}%
\definecolor{currentstroke}{rgb}{0.268510,0.009605,0.335427}%
\pgfsetstrokecolor{currentstroke}%
\pgfsetdash{}{0pt}%
\pgfpathmoveto{\pgfqpoint{8.578381in}{4.706659in}}%
\pgfpathlineto{\pgfqpoint{8.528292in}{4.707183in}}%
\pgfusepath{stroke}%
\end{pgfscope}%
\begin{pgfscope}%
\pgfpathrectangle{\pgfqpoint{6.720588in}{4.155455in}}{\pgfqpoint{2.279412in}{2.004545in}}%
\pgfusepath{clip}%
\pgfsetbuttcap%
\pgfsetroundjoin%
\pgfsetlinewidth{0.328420pt}%
\definecolor{currentstroke}{rgb}{0.271305,0.019942,0.347269}%
\pgfsetstrokecolor{currentstroke}%
\pgfsetdash{}{0pt}%
\pgfpathmoveto{\pgfqpoint{8.528292in}{4.707183in}}%
\pgfpathlineto{\pgfqpoint{8.478285in}{4.709304in}}%
\pgfusepath{stroke}%
\end{pgfscope}%
\begin{pgfscope}%
\pgfpathrectangle{\pgfqpoint{6.720588in}{4.155455in}}{\pgfqpoint{2.279412in}{2.004545in}}%
\pgfusepath{clip}%
\pgfsetbuttcap%
\pgfsetroundjoin%
\pgfsetlinewidth{0.330226pt}%
\definecolor{currentstroke}{rgb}{0.272594,0.025563,0.353093}%
\pgfsetstrokecolor{currentstroke}%
\pgfsetdash{}{0pt}%
\pgfpathmoveto{\pgfqpoint{8.478285in}{4.709304in}}%
\pgfpathlineto{\pgfqpoint{8.428313in}{4.712304in}}%
\pgfusepath{stroke}%
\end{pgfscope}%
\begin{pgfscope}%
\pgfpathrectangle{\pgfqpoint{6.720588in}{4.155455in}}{\pgfqpoint{2.279412in}{2.004545in}}%
\pgfusepath{clip}%
\pgfsetbuttcap%
\pgfsetroundjoin%
\pgfsetlinewidth{0.330844pt}%
\definecolor{currentstroke}{rgb}{0.272594,0.025563,0.353093}%
\pgfsetstrokecolor{currentstroke}%
\pgfsetdash{}{0pt}%
\pgfpathmoveto{\pgfqpoint{8.428313in}{4.712304in}}%
\pgfpathlineto{\pgfqpoint{8.378246in}{4.714550in}}%
\pgfusepath{stroke}%
\end{pgfscope}%
\begin{pgfscope}%
\pgfpathrectangle{\pgfqpoint{6.720588in}{4.155455in}}{\pgfqpoint{2.279412in}{2.004545in}}%
\pgfusepath{clip}%
\pgfsetbuttcap%
\pgfsetroundjoin%
\pgfsetlinewidth{0.339517pt}%
\definecolor{currentstroke}{rgb}{0.273809,0.031497,0.358853}%
\pgfsetstrokecolor{currentstroke}%
\pgfsetdash{}{0pt}%
\pgfpathmoveto{\pgfqpoint{8.378246in}{4.714550in}}%
\pgfpathlineto{\pgfqpoint{8.328220in}{4.717428in}}%
\pgfusepath{stroke}%
\end{pgfscope}%
\begin{pgfscope}%
\pgfpathrectangle{\pgfqpoint{6.720588in}{4.155455in}}{\pgfqpoint{2.279412in}{2.004545in}}%
\pgfusepath{clip}%
\pgfsetbuttcap%
\pgfsetroundjoin%
\pgfsetlinewidth{0.357629pt}%
\definecolor{currentstroke}{rgb}{0.277018,0.050344,0.375715}%
\pgfsetstrokecolor{currentstroke}%
\pgfsetdash{}{0pt}%
\pgfpathmoveto{\pgfqpoint{8.328220in}{4.717428in}}%
\pgfpathlineto{\pgfqpoint{8.278282in}{4.721469in}}%
\pgfusepath{stroke}%
\end{pgfscope}%
\begin{pgfscope}%
\pgfpathrectangle{\pgfqpoint{6.720588in}{4.155455in}}{\pgfqpoint{2.279412in}{2.004545in}}%
\pgfusepath{clip}%
\pgfsetbuttcap%
\pgfsetroundjoin%
\pgfsetlinewidth{0.386404pt}%
\definecolor{currentstroke}{rgb}{0.280267,0.073417,0.397163}%
\pgfsetstrokecolor{currentstroke}%
\pgfsetdash{}{0pt}%
\pgfpathmoveto{\pgfqpoint{8.278282in}{4.721469in}}%
\pgfpathlineto{\pgfqpoint{8.228454in}{4.726374in}}%
\pgfusepath{stroke}%
\end{pgfscope}%
\begin{pgfscope}%
\pgfpathrectangle{\pgfqpoint{6.720588in}{4.155455in}}{\pgfqpoint{2.279412in}{2.004545in}}%
\pgfusepath{clip}%
\pgfsetbuttcap%
\pgfsetroundjoin%
\pgfsetlinewidth{0.387552pt}%
\definecolor{currentstroke}{rgb}{0.280267,0.073417,0.397163}%
\pgfsetstrokecolor{currentstroke}%
\pgfsetdash{}{0pt}%
\pgfpathmoveto{\pgfqpoint{8.228454in}{4.726374in}}%
\pgfpathlineto{\pgfqpoint{8.178750in}{4.732204in}}%
\pgfusepath{stroke}%
\end{pgfscope}%
\begin{pgfscope}%
\pgfpathrectangle{\pgfqpoint{6.720588in}{4.155455in}}{\pgfqpoint{2.279412in}{2.004545in}}%
\pgfusepath{clip}%
\pgfsetbuttcap%
\pgfsetroundjoin%
\pgfsetlinewidth{0.420780pt}%
\definecolor{currentstroke}{rgb}{0.282656,0.100196,0.422160}%
\pgfsetstrokecolor{currentstroke}%
\pgfsetdash{}{0pt}%
\pgfpathmoveto{\pgfqpoint{8.178750in}{4.732204in}}%
\pgfpathlineto{\pgfqpoint{8.129239in}{4.739147in}}%
\pgfusepath{stroke}%
\end{pgfscope}%
\begin{pgfscope}%
\pgfpathrectangle{\pgfqpoint{6.720588in}{4.155455in}}{\pgfqpoint{2.279412in}{2.004545in}}%
\pgfusepath{clip}%
\pgfsetbuttcap%
\pgfsetroundjoin%
\pgfsetlinewidth{0.398536pt}%
\definecolor{currentstroke}{rgb}{0.281446,0.084320,0.407414}%
\pgfsetstrokecolor{currentstroke}%
\pgfsetdash{}{0pt}%
\pgfpathmoveto{\pgfqpoint{8.129239in}{4.739147in}}%
\pgfpathlineto{\pgfqpoint{8.080060in}{4.747747in}}%
\pgfusepath{stroke}%
\end{pgfscope}%
\begin{pgfscope}%
\pgfpathrectangle{\pgfqpoint{6.720588in}{4.155455in}}{\pgfqpoint{2.279412in}{2.004545in}}%
\pgfusepath{clip}%
\pgfsetbuttcap%
\pgfsetroundjoin%
\pgfsetlinewidth{0.424005pt}%
\definecolor{currentstroke}{rgb}{0.282656,0.100196,0.422160}%
\pgfsetstrokecolor{currentstroke}%
\pgfsetdash{}{0pt}%
\pgfpathmoveto{\pgfqpoint{8.080060in}{4.747747in}}%
\pgfpathlineto{\pgfqpoint{8.031026in}{4.756937in}}%
\pgfusepath{stroke}%
\end{pgfscope}%
\begin{pgfscope}%
\pgfpathrectangle{\pgfqpoint{6.720588in}{4.155455in}}{\pgfqpoint{2.279412in}{2.004545in}}%
\pgfusepath{clip}%
\pgfsetbuttcap%
\pgfsetroundjoin%
\pgfsetlinewidth{0.419875pt}%
\definecolor{currentstroke}{rgb}{0.282656,0.100196,0.422160}%
\pgfsetstrokecolor{currentstroke}%
\pgfsetdash{}{0pt}%
\pgfpathmoveto{\pgfqpoint{8.031026in}{4.756937in}}%
\pgfpathlineto{\pgfqpoint{7.982656in}{4.768150in}}%
\pgfusepath{stroke}%
\end{pgfscope}%
\begin{pgfscope}%
\pgfpathrectangle{\pgfqpoint{6.720588in}{4.155455in}}{\pgfqpoint{2.279412in}{2.004545in}}%
\pgfusepath{clip}%
\pgfsetbuttcap%
\pgfsetroundjoin%
\pgfsetlinewidth{0.427464pt}%
\definecolor{currentstroke}{rgb}{0.282910,0.105393,0.426902}%
\pgfsetstrokecolor{currentstroke}%
\pgfsetdash{}{0pt}%
\pgfpathmoveto{\pgfqpoint{7.982656in}{4.768150in}}%
\pgfpathlineto{\pgfqpoint{7.937210in}{4.785738in}}%
\pgfusepath{stroke}%
\end{pgfscope}%
\begin{pgfscope}%
\pgfpathrectangle{\pgfqpoint{6.720588in}{4.155455in}}{\pgfqpoint{2.279412in}{2.004545in}}%
\pgfusepath{clip}%
\pgfsetbuttcap%
\pgfsetroundjoin%
\pgfsetlinewidth{0.462706pt}%
\definecolor{currentstroke}{rgb}{0.283072,0.130895,0.449241}%
\pgfsetstrokecolor{currentstroke}%
\pgfsetdash{}{0pt}%
\pgfpathmoveto{\pgfqpoint{7.937210in}{4.785738in}}%
\pgfpathlineto{\pgfqpoint{7.896410in}{4.802988in}}%
\pgfusepath{stroke}%
\end{pgfscope}%
\begin{pgfscope}%
\pgfpathrectangle{\pgfqpoint{6.720588in}{4.155455in}}{\pgfqpoint{2.279412in}{2.004545in}}%
\pgfusepath{clip}%
\pgfsetbuttcap%
\pgfsetroundjoin%
\pgfsetlinewidth{0.462831pt}%
\definecolor{currentstroke}{rgb}{0.283072,0.130895,0.449241}%
\pgfsetstrokecolor{currentstroke}%
\pgfsetdash{}{0pt}%
\pgfpathmoveto{\pgfqpoint{7.896410in}{4.802988in}}%
\pgfpathlineto{\pgfqpoint{7.852430in}{4.823883in}}%
\pgfusepath{stroke}%
\end{pgfscope}%
\begin{pgfscope}%
\pgfpathrectangle{\pgfqpoint{6.720588in}{4.155455in}}{\pgfqpoint{2.279412in}{2.004545in}}%
\pgfusepath{clip}%
\pgfsetbuttcap%
\pgfsetroundjoin%
\pgfsetlinewidth{0.570023pt}%
\definecolor{currentstroke}{rgb}{0.271828,0.209303,0.504434}%
\pgfsetstrokecolor{currentstroke}%
\pgfsetdash{}{0pt}%
\pgfpathmoveto{\pgfqpoint{7.852430in}{4.823883in}}%
\pgfpathlineto{\pgfqpoint{7.810245in}{4.847554in}}%
\pgfusepath{stroke}%
\end{pgfscope}%
\begin{pgfscope}%
\pgfpathrectangle{\pgfqpoint{6.720588in}{4.155455in}}{\pgfqpoint{2.279412in}{2.004545in}}%
\pgfusepath{clip}%
\pgfsetbuttcap%
\pgfsetroundjoin%
\pgfsetlinewidth{0.325698pt}%
\definecolor{currentstroke}{rgb}{0.271305,0.019942,0.347269}%
\pgfsetstrokecolor{currentstroke}%
\pgfsetdash{}{0pt}%
\pgfpathmoveto{\pgfqpoint{8.578381in}{4.887087in}}%
\pgfpathlineto{\pgfqpoint{8.528267in}{4.888554in}}%
\pgfusepath{stroke}%
\end{pgfscope}%
\begin{pgfscope}%
\pgfpathrectangle{\pgfqpoint{6.720588in}{4.155455in}}{\pgfqpoint{2.279412in}{2.004545in}}%
\pgfusepath{clip}%
\pgfsetbuttcap%
\pgfsetroundjoin%
\pgfsetlinewidth{0.340609pt}%
\definecolor{currentstroke}{rgb}{0.273809,0.031497,0.358853}%
\pgfsetstrokecolor{currentstroke}%
\pgfsetdash{}{0pt}%
\pgfpathmoveto{\pgfqpoint{8.528267in}{4.888554in}}%
\pgfpathlineto{\pgfqpoint{8.478158in}{4.890248in}}%
\pgfusepath{stroke}%
\end{pgfscope}%
\begin{pgfscope}%
\pgfpathrectangle{\pgfqpoint{6.720588in}{4.155455in}}{\pgfqpoint{2.279412in}{2.004545in}}%
\pgfusepath{clip}%
\pgfsetbuttcap%
\pgfsetroundjoin%
\pgfsetlinewidth{0.347737pt}%
\definecolor{currentstroke}{rgb}{0.274952,0.037752,0.364543}%
\pgfsetstrokecolor{currentstroke}%
\pgfsetdash{}{0pt}%
\pgfpathmoveto{\pgfqpoint{8.478158in}{4.890248in}}%
\pgfpathlineto{\pgfqpoint{8.428062in}{4.892189in}}%
\pgfusepath{stroke}%
\end{pgfscope}%
\begin{pgfscope}%
\pgfpathrectangle{\pgfqpoint{6.720588in}{4.155455in}}{\pgfqpoint{2.279412in}{2.004545in}}%
\pgfusepath{clip}%
\pgfsetbuttcap%
\pgfsetroundjoin%
\pgfsetlinewidth{0.375640pt}%
\definecolor{currentstroke}{rgb}{0.278791,0.062145,0.386592}%
\pgfsetstrokecolor{currentstroke}%
\pgfsetdash{}{0pt}%
\pgfpathmoveto{\pgfqpoint{8.428062in}{4.892189in}}%
\pgfpathlineto{\pgfqpoint{8.377986in}{4.894601in}}%
\pgfusepath{stroke}%
\end{pgfscope}%
\begin{pgfscope}%
\pgfpathrectangle{\pgfqpoint{6.720588in}{4.155455in}}{\pgfqpoint{2.279412in}{2.004545in}}%
\pgfusepath{clip}%
\pgfsetbuttcap%
\pgfsetroundjoin%
\pgfsetlinewidth{0.395463pt}%
\definecolor{currentstroke}{rgb}{0.280894,0.078907,0.402329}%
\pgfsetstrokecolor{currentstroke}%
\pgfsetdash{}{0pt}%
\pgfpathmoveto{\pgfqpoint{8.377986in}{4.894601in}}%
\pgfpathlineto{\pgfqpoint{8.327903in}{4.896868in}}%
\pgfusepath{stroke}%
\end{pgfscope}%
\begin{pgfscope}%
\pgfpathrectangle{\pgfqpoint{6.720588in}{4.155455in}}{\pgfqpoint{2.279412in}{2.004545in}}%
\pgfusepath{clip}%
\pgfsetbuttcap%
\pgfsetroundjoin%
\pgfsetlinewidth{0.421448pt}%
\definecolor{currentstroke}{rgb}{0.282656,0.100196,0.422160}%
\pgfsetstrokecolor{currentstroke}%
\pgfsetdash{}{0pt}%
\pgfpathmoveto{\pgfqpoint{8.327903in}{4.896868in}}%
\pgfpathlineto{\pgfqpoint{8.277834in}{4.899348in}}%
\pgfusepath{stroke}%
\end{pgfscope}%
\begin{pgfscope}%
\pgfpathrectangle{\pgfqpoint{6.720588in}{4.155455in}}{\pgfqpoint{2.279412in}{2.004545in}}%
\pgfusepath{clip}%
\pgfsetbuttcap%
\pgfsetroundjoin%
\pgfsetlinewidth{0.466138pt}%
\definecolor{currentstroke}{rgb}{0.282884,0.135920,0.453427}%
\pgfsetstrokecolor{currentstroke}%
\pgfsetdash{}{0pt}%
\pgfpathmoveto{\pgfqpoint{8.277834in}{4.899348in}}%
\pgfpathlineto{\pgfqpoint{8.227798in}{4.902329in}}%
\pgfusepath{stroke}%
\end{pgfscope}%
\begin{pgfscope}%
\pgfpathrectangle{\pgfqpoint{6.720588in}{4.155455in}}{\pgfqpoint{2.279412in}{2.004545in}}%
\pgfusepath{clip}%
\pgfsetbuttcap%
\pgfsetroundjoin%
\pgfsetlinewidth{0.482657pt}%
\definecolor{currentstroke}{rgb}{0.282290,0.145912,0.461510}%
\pgfsetstrokecolor{currentstroke}%
\pgfsetdash{}{0pt}%
\pgfpathmoveto{\pgfqpoint{8.227798in}{4.902329in}}%
\pgfpathlineto{\pgfqpoint{8.177801in}{4.905778in}}%
\pgfusepath{stroke}%
\end{pgfscope}%
\begin{pgfscope}%
\pgfpathrectangle{\pgfqpoint{6.720588in}{4.155455in}}{\pgfqpoint{2.279412in}{2.004545in}}%
\pgfusepath{clip}%
\pgfsetbuttcap%
\pgfsetroundjoin%
\pgfsetlinewidth{0.554707pt}%
\definecolor{currentstroke}{rgb}{0.275191,0.194905,0.496005}%
\pgfsetstrokecolor{currentstroke}%
\pgfsetdash{}{0pt}%
\pgfpathmoveto{\pgfqpoint{8.177801in}{4.905778in}}%
\pgfpathlineto{\pgfqpoint{8.127879in}{4.909968in}}%
\pgfusepath{stroke}%
\end{pgfscope}%
\begin{pgfscope}%
\pgfpathrectangle{\pgfqpoint{6.720588in}{4.155455in}}{\pgfqpoint{2.279412in}{2.004545in}}%
\pgfusepath{clip}%
\pgfsetbuttcap%
\pgfsetroundjoin%
\pgfsetlinewidth{0.568344pt}%
\definecolor{currentstroke}{rgb}{0.273006,0.204520,0.501721}%
\pgfsetstrokecolor{currentstroke}%
\pgfsetdash{}{0pt}%
\pgfpathmoveto{\pgfqpoint{8.127879in}{4.909968in}}%
\pgfpathlineto{\pgfqpoint{8.078088in}{4.915206in}}%
\pgfusepath{stroke}%
\end{pgfscope}%
\begin{pgfscope}%
\pgfpathrectangle{\pgfqpoint{6.720588in}{4.155455in}}{\pgfqpoint{2.279412in}{2.004545in}}%
\pgfusepath{clip}%
\pgfsetbuttcap%
\pgfsetroundjoin%
\pgfsetlinewidth{0.601530pt}%
\definecolor{currentstroke}{rgb}{0.266580,0.228262,0.514349}%
\pgfsetstrokecolor{currentstroke}%
\pgfsetdash{}{0pt}%
\pgfpathmoveto{\pgfqpoint{8.078088in}{4.915206in}}%
\pgfpathlineto{\pgfqpoint{8.028509in}{4.921779in}}%
\pgfusepath{stroke}%
\end{pgfscope}%
\begin{pgfscope}%
\pgfpathrectangle{\pgfqpoint{6.720588in}{4.155455in}}{\pgfqpoint{2.279412in}{2.004545in}}%
\pgfusepath{clip}%
\pgfsetbuttcap%
\pgfsetroundjoin%
\pgfsetlinewidth{0.664419pt}%
\definecolor{currentstroke}{rgb}{0.252194,0.269783,0.531579}%
\pgfsetstrokecolor{currentstroke}%
\pgfsetdash{}{0pt}%
\pgfpathmoveto{\pgfqpoint{8.028509in}{4.921779in}}%
\pgfpathlineto{\pgfqpoint{7.979172in}{4.929664in}}%
\pgfusepath{stroke}%
\end{pgfscope}%
\begin{pgfscope}%
\pgfpathrectangle{\pgfqpoint{6.720588in}{4.155455in}}{\pgfqpoint{2.279412in}{2.004545in}}%
\pgfusepath{clip}%
\pgfsetbuttcap%
\pgfsetroundjoin%
\pgfsetlinewidth{0.683618pt}%
\definecolor{currentstroke}{rgb}{0.246811,0.283237,0.535941}%
\pgfsetstrokecolor{currentstroke}%
\pgfsetdash{}{0pt}%
\pgfpathmoveto{\pgfqpoint{7.979172in}{4.929664in}}%
\pgfpathlineto{\pgfqpoint{7.930198in}{4.939105in}}%
\pgfusepath{stroke}%
\end{pgfscope}%
\begin{pgfscope}%
\pgfpathrectangle{\pgfqpoint{6.720588in}{4.155455in}}{\pgfqpoint{2.279412in}{2.004545in}}%
\pgfusepath{clip}%
\pgfsetbuttcap%
\pgfsetroundjoin%
\pgfsetlinewidth{0.573759pt}%
\definecolor{currentstroke}{rgb}{0.271828,0.209303,0.504434}%
\pgfsetstrokecolor{currentstroke}%
\pgfsetdash{}{0pt}%
\pgfpathmoveto{\pgfqpoint{7.930198in}{4.939105in}}%
\pgfpathlineto{\pgfqpoint{7.881790in}{4.950539in}}%
\pgfusepath{stroke}%
\end{pgfscope}%
\begin{pgfscope}%
\pgfpathrectangle{\pgfqpoint{6.720588in}{4.155455in}}{\pgfqpoint{2.279412in}{2.004545in}}%
\pgfusepath{clip}%
\pgfsetbuttcap%
\pgfsetroundjoin%
\pgfsetlinewidth{0.924094pt}%
\definecolor{currentstroke}{rgb}{0.182256,0.426184,0.557120}%
\pgfsetstrokecolor{currentstroke}%
\pgfsetdash{}{0pt}%
\pgfpathmoveto{\pgfqpoint{7.881790in}{4.950539in}}%
\pgfpathlineto{\pgfqpoint{7.834081in}{4.964043in}}%
\pgfusepath{stroke}%
\end{pgfscope}%
\begin{pgfscope}%
\pgfpathrectangle{\pgfqpoint{6.720588in}{4.155455in}}{\pgfqpoint{2.279412in}{2.004545in}}%
\pgfusepath{clip}%
\pgfsetbuttcap%
\pgfsetroundjoin%
\pgfsetlinewidth{1.578202pt}%
\definecolor{currentstroke}{rgb}{0.319809,0.770914,0.411152}%
\pgfsetstrokecolor{currentstroke}%
\pgfsetdash{}{0pt}%
\pgfpathmoveto{\pgfqpoint{7.834081in}{4.964043in}}%
\pgfpathlineto{\pgfqpoint{7.787098in}{4.979427in}}%
\pgfusepath{stroke}%
\end{pgfscope}%
\begin{pgfscope}%
\pgfpathrectangle{\pgfqpoint{6.720588in}{4.155455in}}{\pgfqpoint{2.279412in}{2.004545in}}%
\pgfusepath{clip}%
\pgfsetbuttcap%
\pgfsetroundjoin%
\pgfsetlinewidth{1.708293pt}%
\definecolor{currentstroke}{rgb}{0.487026,0.823929,0.312321}%
\pgfsetstrokecolor{currentstroke}%
\pgfsetdash{}{0pt}%
\pgfpathmoveto{\pgfqpoint{7.787098in}{4.979427in}}%
\pgfpathlineto{\pgfqpoint{7.740686in}{4.996107in}}%
\pgfusepath{stroke}%
\end{pgfscope}%
\begin{pgfscope}%
\pgfpathrectangle{\pgfqpoint{6.720588in}{4.155455in}}{\pgfqpoint{2.279412in}{2.004545in}}%
\pgfusepath{clip}%
\pgfsetbuttcap%
\pgfsetroundjoin%
\pgfsetlinewidth{1.831409pt}%
\definecolor{currentstroke}{rgb}{0.668054,0.861999,0.196293}%
\pgfsetstrokecolor{currentstroke}%
\pgfsetdash{}{0pt}%
\pgfpathmoveto{\pgfqpoint{7.740686in}{4.996107in}}%
\pgfpathlineto{\pgfqpoint{7.694654in}{5.013558in}}%
\pgfusepath{stroke}%
\end{pgfscope}%
\begin{pgfscope}%
\pgfpathrectangle{\pgfqpoint{6.720588in}{4.155455in}}{\pgfqpoint{2.279412in}{2.004545in}}%
\pgfusepath{clip}%
\pgfsetbuttcap%
\pgfsetroundjoin%
\pgfsetlinewidth{0.321897pt}%
\definecolor{currentstroke}{rgb}{0.271305,0.019942,0.347269}%
\pgfsetstrokecolor{currentstroke}%
\pgfsetdash{}{0pt}%
\pgfpathmoveto{\pgfqpoint{8.578381in}{5.383261in}}%
\pgfpathlineto{\pgfqpoint{8.528275in}{5.381966in}}%
\pgfusepath{stroke}%
\end{pgfscope}%
\begin{pgfscope}%
\pgfpathrectangle{\pgfqpoint{6.720588in}{4.155455in}}{\pgfqpoint{2.279412in}{2.004545in}}%
\pgfusepath{clip}%
\pgfsetbuttcap%
\pgfsetroundjoin%
\pgfsetlinewidth{0.336339pt}%
\definecolor{currentstroke}{rgb}{0.273809,0.031497,0.358853}%
\pgfsetstrokecolor{currentstroke}%
\pgfsetdash{}{0pt}%
\pgfpathmoveto{\pgfqpoint{8.528275in}{5.381966in}}%
\pgfpathlineto{\pgfqpoint{8.478157in}{5.380388in}}%
\pgfusepath{stroke}%
\end{pgfscope}%
\begin{pgfscope}%
\pgfpathrectangle{\pgfqpoint{6.720588in}{4.155455in}}{\pgfqpoint{2.279412in}{2.004545in}}%
\pgfusepath{clip}%
\pgfsetbuttcap%
\pgfsetroundjoin%
\pgfsetlinewidth{0.363674pt}%
\definecolor{currentstroke}{rgb}{0.277941,0.056324,0.381191}%
\pgfsetstrokecolor{currentstroke}%
\pgfsetdash{}{0pt}%
\pgfpathmoveto{\pgfqpoint{8.478157in}{5.380388in}}%
\pgfpathlineto{\pgfqpoint{8.428042in}{5.378736in}}%
\pgfusepath{stroke}%
\end{pgfscope}%
\begin{pgfscope}%
\pgfpathrectangle{\pgfqpoint{6.720588in}{4.155455in}}{\pgfqpoint{2.279412in}{2.004545in}}%
\pgfusepath{clip}%
\pgfsetbuttcap%
\pgfsetroundjoin%
\pgfsetlinewidth{0.397265pt}%
\definecolor{currentstroke}{rgb}{0.281446,0.084320,0.407414}%
\pgfsetstrokecolor{currentstroke}%
\pgfsetdash{}{0pt}%
\pgfpathmoveto{\pgfqpoint{8.428042in}{5.378736in}}%
\pgfpathlineto{\pgfqpoint{8.377935in}{5.376878in}}%
\pgfusepath{stroke}%
\end{pgfscope}%
\begin{pgfscope}%
\pgfpathrectangle{\pgfqpoint{6.720588in}{4.155455in}}{\pgfqpoint{2.279412in}{2.004545in}}%
\pgfusepath{clip}%
\pgfsetbuttcap%
\pgfsetroundjoin%
\pgfsetlinewidth{0.410694pt}%
\definecolor{currentstroke}{rgb}{0.281924,0.089666,0.412415}%
\pgfsetstrokecolor{currentstroke}%
\pgfsetdash{}{0pt}%
\pgfpathmoveto{\pgfqpoint{8.377935in}{5.376878in}}%
\pgfpathlineto{\pgfqpoint{8.327840in}{5.374789in}}%
\pgfusepath{stroke}%
\end{pgfscope}%
\begin{pgfscope}%
\pgfpathrectangle{\pgfqpoint{6.720588in}{4.155455in}}{\pgfqpoint{2.279412in}{2.004545in}}%
\pgfusepath{clip}%
\pgfsetbuttcap%
\pgfsetroundjoin%
\pgfsetlinewidth{0.432487pt}%
\definecolor{currentstroke}{rgb}{0.283091,0.110553,0.431554}%
\pgfsetstrokecolor{currentstroke}%
\pgfsetdash{}{0pt}%
\pgfpathmoveto{\pgfqpoint{8.327840in}{5.374789in}}%
\pgfpathlineto{\pgfqpoint{8.277765in}{5.372384in}}%
\pgfusepath{stroke}%
\end{pgfscope}%
\begin{pgfscope}%
\pgfpathrectangle{\pgfqpoint{6.720588in}{4.155455in}}{\pgfqpoint{2.279412in}{2.004545in}}%
\pgfusepath{clip}%
\pgfsetbuttcap%
\pgfsetroundjoin%
\pgfsetlinewidth{0.496796pt}%
\definecolor{currentstroke}{rgb}{0.281412,0.155834,0.469201}%
\pgfsetstrokecolor{currentstroke}%
\pgfsetdash{}{0pt}%
\pgfpathmoveto{\pgfqpoint{8.277765in}{5.372384in}}%
\pgfpathlineto{\pgfqpoint{8.227702in}{5.369784in}}%
\pgfusepath{stroke}%
\end{pgfscope}%
\begin{pgfscope}%
\pgfpathrectangle{\pgfqpoint{6.720588in}{4.155455in}}{\pgfqpoint{2.279412in}{2.004545in}}%
\pgfusepath{clip}%
\pgfsetbuttcap%
\pgfsetroundjoin%
\pgfsetlinewidth{0.551588pt}%
\definecolor{currentstroke}{rgb}{0.275191,0.194905,0.496005}%
\pgfsetstrokecolor{currentstroke}%
\pgfsetdash{}{0pt}%
\pgfpathmoveto{\pgfqpoint{8.227702in}{5.369784in}}%
\pgfpathlineto{\pgfqpoint{8.177663in}{5.366853in}}%
\pgfusepath{stroke}%
\end{pgfscope}%
\begin{pgfscope}%
\pgfpathrectangle{\pgfqpoint{6.720588in}{4.155455in}}{\pgfqpoint{2.279412in}{2.004545in}}%
\pgfusepath{clip}%
\pgfsetbuttcap%
\pgfsetroundjoin%
\pgfsetlinewidth{0.327884pt}%
\definecolor{currentstroke}{rgb}{0.271305,0.019942,0.347269}%
\pgfsetstrokecolor{currentstroke}%
\pgfsetdash{}{0pt}%
\pgfpathmoveto{\pgfqpoint{8.578381in}{5.473475in}}%
\pgfpathlineto{\pgfqpoint{8.528269in}{5.472299in}}%
\pgfusepath{stroke}%
\end{pgfscope}%
\begin{pgfscope}%
\pgfpathrectangle{\pgfqpoint{6.720588in}{4.155455in}}{\pgfqpoint{2.279412in}{2.004545in}}%
\pgfusepath{clip}%
\pgfsetbuttcap%
\pgfsetroundjoin%
\pgfsetlinewidth{0.331990pt}%
\definecolor{currentstroke}{rgb}{0.272594,0.025563,0.353093}%
\pgfsetstrokecolor{currentstroke}%
\pgfsetdash{}{0pt}%
\pgfpathmoveto{\pgfqpoint{8.528269in}{5.472299in}}%
\pgfpathlineto{\pgfqpoint{8.478186in}{5.470139in}}%
\pgfusepath{stroke}%
\end{pgfscope}%
\begin{pgfscope}%
\pgfpathrectangle{\pgfqpoint{6.720588in}{4.155455in}}{\pgfqpoint{2.279412in}{2.004545in}}%
\pgfusepath{clip}%
\pgfsetbuttcap%
\pgfsetroundjoin%
\pgfsetlinewidth{0.341269pt}%
\definecolor{currentstroke}{rgb}{0.273809,0.031497,0.358853}%
\pgfsetstrokecolor{currentstroke}%
\pgfsetdash{}{0pt}%
\pgfpathmoveto{\pgfqpoint{8.478186in}{5.470139in}}%
\pgfpathlineto{\pgfqpoint{8.428111in}{5.467783in}}%
\pgfusepath{stroke}%
\end{pgfscope}%
\begin{pgfscope}%
\pgfpathrectangle{\pgfqpoint{6.720588in}{4.155455in}}{\pgfqpoint{2.279412in}{2.004545in}}%
\pgfusepath{clip}%
\pgfsetbuttcap%
\pgfsetroundjoin%
\pgfsetlinewidth{0.363470pt}%
\definecolor{currentstroke}{rgb}{0.277941,0.056324,0.381191}%
\pgfsetstrokecolor{currentstroke}%
\pgfsetdash{}{0pt}%
\pgfpathmoveto{\pgfqpoint{8.428111in}{5.467783in}}%
\pgfpathlineto{\pgfqpoint{8.378052in}{5.465208in}}%
\pgfusepath{stroke}%
\end{pgfscope}%
\begin{pgfscope}%
\pgfpathrectangle{\pgfqpoint{6.720588in}{4.155455in}}{\pgfqpoint{2.279412in}{2.004545in}}%
\pgfusepath{clip}%
\pgfsetbuttcap%
\pgfsetroundjoin%
\pgfsetlinewidth{0.386992pt}%
\definecolor{currentstroke}{rgb}{0.280267,0.073417,0.397163}%
\pgfsetstrokecolor{currentstroke}%
\pgfsetdash{}{0pt}%
\pgfpathmoveto{\pgfqpoint{8.378052in}{5.465208in}}%
\pgfpathlineto{\pgfqpoint{8.328036in}{5.462017in}}%
\pgfusepath{stroke}%
\end{pgfscope}%
\begin{pgfscope}%
\pgfpathrectangle{\pgfqpoint{6.720588in}{4.155455in}}{\pgfqpoint{2.279412in}{2.004545in}}%
\pgfusepath{clip}%
\pgfsetbuttcap%
\pgfsetroundjoin%
\pgfsetlinewidth{0.409994pt}%
\definecolor{currentstroke}{rgb}{0.281924,0.089666,0.412415}%
\pgfsetstrokecolor{currentstroke}%
\pgfsetdash{}{0pt}%
\pgfpathmoveto{\pgfqpoint{8.328036in}{5.462017in}}%
\pgfpathlineto{\pgfqpoint{8.278033in}{5.458630in}}%
\pgfusepath{stroke}%
\end{pgfscope}%
\begin{pgfscope}%
\pgfpathrectangle{\pgfqpoint{6.720588in}{4.155455in}}{\pgfqpoint{2.279412in}{2.004545in}}%
\pgfusepath{clip}%
\pgfsetbuttcap%
\pgfsetroundjoin%
\pgfsetlinewidth{0.449286pt}%
\definecolor{currentstroke}{rgb}{0.283229,0.120777,0.440584}%
\pgfsetstrokecolor{currentstroke}%
\pgfsetdash{}{0pt}%
\pgfpathmoveto{\pgfqpoint{8.278033in}{5.458630in}}%
\pgfpathlineto{\pgfqpoint{8.228082in}{5.454719in}}%
\pgfusepath{stroke}%
\end{pgfscope}%
\begin{pgfscope}%
\pgfpathrectangle{\pgfqpoint{6.720588in}{4.155455in}}{\pgfqpoint{2.279412in}{2.004545in}}%
\pgfusepath{clip}%
\pgfsetbuttcap%
\pgfsetroundjoin%
\pgfsetlinewidth{0.322942pt}%
\definecolor{currentstroke}{rgb}{0.271305,0.019942,0.347269}%
\pgfsetstrokecolor{currentstroke}%
\pgfsetdash{}{0pt}%
\pgfpathmoveto{\pgfqpoint{8.578381in}{5.518582in}}%
\pgfpathlineto{\pgfqpoint{8.528352in}{5.517716in}}%
\pgfusepath{stroke}%
\end{pgfscope}%
\begin{pgfscope}%
\pgfpathrectangle{\pgfqpoint{6.720588in}{4.155455in}}{\pgfqpoint{2.279412in}{2.004545in}}%
\pgfusepath{clip}%
\pgfsetbuttcap%
\pgfsetroundjoin%
\pgfsetlinewidth{0.327334pt}%
\definecolor{currentstroke}{rgb}{0.271305,0.019942,0.347269}%
\pgfsetstrokecolor{currentstroke}%
\pgfsetdash{}{0pt}%
\pgfpathmoveto{\pgfqpoint{8.528352in}{5.517716in}}%
\pgfpathlineto{\pgfqpoint{8.478342in}{5.515207in}}%
\pgfusepath{stroke}%
\end{pgfscope}%
\begin{pgfscope}%
\pgfpathrectangle{\pgfqpoint{6.720588in}{4.155455in}}{\pgfqpoint{2.279412in}{2.004545in}}%
\pgfusepath{clip}%
\pgfsetbuttcap%
\pgfsetroundjoin%
\pgfsetlinewidth{0.338451pt}%
\definecolor{currentstroke}{rgb}{0.273809,0.031497,0.358853}%
\pgfsetstrokecolor{currentstroke}%
\pgfsetdash{}{0pt}%
\pgfpathmoveto{\pgfqpoint{8.478342in}{5.515207in}}%
\pgfpathlineto{\pgfqpoint{8.428342in}{5.512248in}}%
\pgfusepath{stroke}%
\end{pgfscope}%
\begin{pgfscope}%
\pgfpathrectangle{\pgfqpoint{6.720588in}{4.155455in}}{\pgfqpoint{2.279412in}{2.004545in}}%
\pgfusepath{clip}%
\pgfsetbuttcap%
\pgfsetroundjoin%
\pgfsetlinewidth{0.355846pt}%
\definecolor{currentstroke}{rgb}{0.276022,0.044167,0.370164}%
\pgfsetstrokecolor{currentstroke}%
\pgfsetdash{}{0pt}%
\pgfpathmoveto{\pgfqpoint{8.428342in}{5.512248in}}%
\pgfpathlineto{\pgfqpoint{8.378282in}{5.509777in}}%
\pgfusepath{stroke}%
\end{pgfscope}%
\begin{pgfscope}%
\pgfpathrectangle{\pgfqpoint{6.720588in}{4.155455in}}{\pgfqpoint{2.279412in}{2.004545in}}%
\pgfusepath{clip}%
\pgfsetbuttcap%
\pgfsetroundjoin%
\pgfsetlinewidth{0.375126pt}%
\definecolor{currentstroke}{rgb}{0.278791,0.062145,0.386592}%
\pgfsetstrokecolor{currentstroke}%
\pgfsetdash{}{0pt}%
\pgfpathmoveto{\pgfqpoint{8.378282in}{5.509777in}}%
\pgfpathlineto{\pgfqpoint{8.328250in}{5.507153in}}%
\pgfusepath{stroke}%
\end{pgfscope}%
\begin{pgfscope}%
\pgfpathrectangle{\pgfqpoint{6.720588in}{4.155455in}}{\pgfqpoint{2.279412in}{2.004545in}}%
\pgfusepath{clip}%
\pgfsetbuttcap%
\pgfsetroundjoin%
\pgfsetlinewidth{0.372567pt}%
\definecolor{currentstroke}{rgb}{0.278791,0.062145,0.386592}%
\pgfsetstrokecolor{currentstroke}%
\pgfsetdash{}{0pt}%
\pgfpathmoveto{\pgfqpoint{8.328250in}{5.507153in}}%
\pgfpathlineto{\pgfqpoint{8.278317in}{5.503255in}}%
\pgfusepath{stroke}%
\end{pgfscope}%
\begin{pgfscope}%
\pgfpathrectangle{\pgfqpoint{6.720588in}{4.155455in}}{\pgfqpoint{2.279412in}{2.004545in}}%
\pgfusepath{clip}%
\pgfsetbuttcap%
\pgfsetroundjoin%
\pgfsetlinewidth{0.415036pt}%
\definecolor{currentstroke}{rgb}{0.282327,0.094955,0.417331}%
\pgfsetstrokecolor{currentstroke}%
\pgfsetdash{}{0pt}%
\pgfpathmoveto{\pgfqpoint{8.278317in}{5.503255in}}%
\pgfpathlineto{\pgfqpoint{8.228443in}{5.498660in}}%
\pgfusepath{stroke}%
\end{pgfscope}%
\begin{pgfscope}%
\pgfpathrectangle{\pgfqpoint{6.720588in}{4.155455in}}{\pgfqpoint{2.279412in}{2.004545in}}%
\pgfusepath{clip}%
\pgfsetbuttcap%
\pgfsetroundjoin%
\pgfsetlinewidth{0.435126pt}%
\definecolor{currentstroke}{rgb}{0.283091,0.110553,0.431554}%
\pgfsetstrokecolor{currentstroke}%
\pgfsetdash{}{0pt}%
\pgfpathmoveto{\pgfqpoint{8.228443in}{5.498660in}}%
\pgfpathlineto{\pgfqpoint{8.178647in}{5.493425in}}%
\pgfusepath{stroke}%
\end{pgfscope}%
\begin{pgfscope}%
\pgfpathrectangle{\pgfqpoint{6.720588in}{4.155455in}}{\pgfqpoint{2.279412in}{2.004545in}}%
\pgfusepath{clip}%
\pgfsetbuttcap%
\pgfsetroundjoin%
\pgfsetlinewidth{0.467618pt}%
\definecolor{currentstroke}{rgb}{0.282884,0.135920,0.453427}%
\pgfsetstrokecolor{currentstroke}%
\pgfsetdash{}{0pt}%
\pgfpathmoveto{\pgfqpoint{8.178647in}{5.493425in}}%
\pgfpathlineto{\pgfqpoint{8.128945in}{5.487547in}}%
\pgfusepath{stroke}%
\end{pgfscope}%
\begin{pgfscope}%
\pgfpathrectangle{\pgfqpoint{6.720588in}{4.155455in}}{\pgfqpoint{2.279412in}{2.004545in}}%
\pgfusepath{clip}%
\pgfsetbuttcap%
\pgfsetroundjoin%
\pgfsetlinewidth{0.487938pt}%
\definecolor{currentstroke}{rgb}{0.281887,0.150881,0.465405}%
\pgfsetstrokecolor{currentstroke}%
\pgfsetdash{}{0pt}%
\pgfpathmoveto{\pgfqpoint{8.128945in}{5.487547in}}%
\pgfpathlineto{\pgfqpoint{8.079458in}{5.480461in}}%
\pgfusepath{stroke}%
\end{pgfscope}%
\begin{pgfscope}%
\pgfpathrectangle{\pgfqpoint{6.720588in}{4.155455in}}{\pgfqpoint{2.279412in}{2.004545in}}%
\pgfusepath{clip}%
\pgfsetbuttcap%
\pgfsetroundjoin%
\pgfsetlinewidth{0.502495pt}%
\definecolor{currentstroke}{rgb}{0.280868,0.160771,0.472899}%
\pgfsetstrokecolor{currentstroke}%
\pgfsetdash{}{0pt}%
\pgfpathmoveto{\pgfqpoint{8.079458in}{5.480461in}}%
\pgfpathlineto{\pgfqpoint{8.030336in}{5.471658in}}%
\pgfusepath{stroke}%
\end{pgfscope}%
\begin{pgfscope}%
\pgfpathrectangle{\pgfqpoint{6.720588in}{4.155455in}}{\pgfqpoint{2.279412in}{2.004545in}}%
\pgfusepath{clip}%
\pgfsetbuttcap%
\pgfsetroundjoin%
\pgfsetlinewidth{0.496115pt}%
\definecolor{currentstroke}{rgb}{0.281412,0.155834,0.469201}%
\pgfsetstrokecolor{currentstroke}%
\pgfsetdash{}{0pt}%
\pgfpathmoveto{\pgfqpoint{8.030336in}{5.471658in}}%
\pgfpathlineto{\pgfqpoint{7.981634in}{5.461175in}}%
\pgfusepath{stroke}%
\end{pgfscope}%
\begin{pgfscope}%
\pgfpathrectangle{\pgfqpoint{6.720588in}{4.155455in}}{\pgfqpoint{2.279412in}{2.004545in}}%
\pgfusepath{clip}%
\pgfsetbuttcap%
\pgfsetroundjoin%
\pgfsetlinewidth{0.484528pt}%
\definecolor{currentstroke}{rgb}{0.282290,0.145912,0.461510}%
\pgfsetstrokecolor{currentstroke}%
\pgfsetdash{}{0pt}%
\pgfpathmoveto{\pgfqpoint{7.981634in}{5.461175in}}%
\pgfpathlineto{\pgfqpoint{7.933594in}{5.448619in}}%
\pgfusepath{stroke}%
\end{pgfscope}%
\begin{pgfscope}%
\pgfpathrectangle{\pgfqpoint{6.720588in}{4.155455in}}{\pgfqpoint{2.279412in}{2.004545in}}%
\pgfusepath{clip}%
\pgfsetbuttcap%
\pgfsetroundjoin%
\pgfsetlinewidth{0.587401pt}%
\definecolor{currentstroke}{rgb}{0.269308,0.218818,0.509577}%
\pgfsetstrokecolor{currentstroke}%
\pgfsetdash{}{0pt}%
\pgfpathmoveto{\pgfqpoint{7.933594in}{5.448619in}}%
\pgfpathlineto{\pgfqpoint{7.886585in}{5.433364in}}%
\pgfusepath{stroke}%
\end{pgfscope}%
\begin{pgfscope}%
\pgfpathrectangle{\pgfqpoint{6.720588in}{4.155455in}}{\pgfqpoint{2.279412in}{2.004545in}}%
\pgfusepath{clip}%
\pgfsetbuttcap%
\pgfsetroundjoin%
\pgfsetlinewidth{0.320129pt}%
\definecolor{currentstroke}{rgb}{0.269944,0.014625,0.341379}%
\pgfsetstrokecolor{currentstroke}%
\pgfsetdash{}{0pt}%
\pgfpathmoveto{\pgfqpoint{8.578381in}{5.563688in}}%
\pgfpathlineto{\pgfqpoint{8.528299in}{5.561761in}}%
\pgfusepath{stroke}%
\end{pgfscope}%
\begin{pgfscope}%
\pgfpathrectangle{\pgfqpoint{6.720588in}{4.155455in}}{\pgfqpoint{2.279412in}{2.004545in}}%
\pgfusepath{clip}%
\pgfsetbuttcap%
\pgfsetroundjoin%
\pgfsetlinewidth{0.327463pt}%
\definecolor{currentstroke}{rgb}{0.271305,0.019942,0.347269}%
\pgfsetstrokecolor{currentstroke}%
\pgfsetdash{}{0pt}%
\pgfpathmoveto{\pgfqpoint{8.528299in}{5.561761in}}%
\pgfpathlineto{\pgfqpoint{8.478210in}{5.559983in}}%
\pgfusepath{stroke}%
\end{pgfscope}%
\begin{pgfscope}%
\pgfpathrectangle{\pgfqpoint{6.720588in}{4.155455in}}{\pgfqpoint{2.279412in}{2.004545in}}%
\pgfusepath{clip}%
\pgfsetbuttcap%
\pgfsetroundjoin%
\pgfsetlinewidth{0.330622pt}%
\definecolor{currentstroke}{rgb}{0.272594,0.025563,0.353093}%
\pgfsetstrokecolor{currentstroke}%
\pgfsetdash{}{0pt}%
\pgfpathmoveto{\pgfqpoint{8.478210in}{5.559983in}}%
\pgfpathlineto{\pgfqpoint{8.428122in}{5.557990in}}%
\pgfusepath{stroke}%
\end{pgfscope}%
\begin{pgfscope}%
\pgfpathrectangle{\pgfqpoint{6.720588in}{4.155455in}}{\pgfqpoint{2.279412in}{2.004545in}}%
\pgfusepath{clip}%
\pgfsetbuttcap%
\pgfsetroundjoin%
\pgfsetlinewidth{0.353175pt}%
\definecolor{currentstroke}{rgb}{0.276022,0.044167,0.370164}%
\pgfsetstrokecolor{currentstroke}%
\pgfsetdash{}{0pt}%
\pgfpathmoveto{\pgfqpoint{8.428122in}{5.557990in}}%
\pgfpathlineto{\pgfqpoint{8.378066in}{5.555367in}}%
\pgfusepath{stroke}%
\end{pgfscope}%
\begin{pgfscope}%
\pgfpathrectangle{\pgfqpoint{6.720588in}{4.155455in}}{\pgfqpoint{2.279412in}{2.004545in}}%
\pgfusepath{clip}%
\pgfsetbuttcap%
\pgfsetroundjoin%
\pgfsetlinewidth{0.357655pt}%
\definecolor{currentstroke}{rgb}{0.277018,0.050344,0.375715}%
\pgfsetstrokecolor{currentstroke}%
\pgfsetdash{}{0pt}%
\pgfpathmoveto{\pgfqpoint{8.378066in}{5.555367in}}%
\pgfpathlineto{\pgfqpoint{8.328086in}{5.551848in}}%
\pgfusepath{stroke}%
\end{pgfscope}%
\begin{pgfscope}%
\pgfpathrectangle{\pgfqpoint{6.720588in}{4.155455in}}{\pgfqpoint{2.279412in}{2.004545in}}%
\pgfusepath{clip}%
\pgfsetbuttcap%
\pgfsetroundjoin%
\pgfsetlinewidth{0.372149pt}%
\definecolor{currentstroke}{rgb}{0.278791,0.062145,0.386592}%
\pgfsetstrokecolor{currentstroke}%
\pgfsetdash{}{0pt}%
\pgfpathmoveto{\pgfqpoint{8.328086in}{5.551848in}}%
\pgfpathlineto{\pgfqpoint{8.278158in}{5.547741in}}%
\pgfusepath{stroke}%
\end{pgfscope}%
\begin{pgfscope}%
\pgfpathrectangle{\pgfqpoint{6.720588in}{4.155455in}}{\pgfqpoint{2.279412in}{2.004545in}}%
\pgfusepath{clip}%
\pgfsetbuttcap%
\pgfsetroundjoin%
\pgfsetlinewidth{0.386692pt}%
\definecolor{currentstroke}{rgb}{0.280267,0.073417,0.397163}%
\pgfsetstrokecolor{currentstroke}%
\pgfsetdash{}{0pt}%
\pgfpathmoveto{\pgfqpoint{8.278158in}{5.547741in}}%
\pgfpathlineto{\pgfqpoint{8.228258in}{5.543359in}}%
\pgfusepath{stroke}%
\end{pgfscope}%
\begin{pgfscope}%
\pgfpathrectangle{\pgfqpoint{6.720588in}{4.155455in}}{\pgfqpoint{2.279412in}{2.004545in}}%
\pgfusepath{clip}%
\pgfsetbuttcap%
\pgfsetroundjoin%
\pgfsetlinewidth{0.408485pt}%
\definecolor{currentstroke}{rgb}{0.281924,0.089666,0.412415}%
\pgfsetstrokecolor{currentstroke}%
\pgfsetdash{}{0pt}%
\pgfpathmoveto{\pgfqpoint{8.228258in}{5.543359in}}%
\pgfpathlineto{\pgfqpoint{8.178420in}{5.538446in}}%
\pgfusepath{stroke}%
\end{pgfscope}%
\begin{pgfscope}%
\pgfpathrectangle{\pgfqpoint{6.720588in}{4.155455in}}{\pgfqpoint{2.279412in}{2.004545in}}%
\pgfusepath{clip}%
\pgfsetbuttcap%
\pgfsetroundjoin%
\pgfsetlinewidth{0.418058pt}%
\definecolor{currentstroke}{rgb}{0.282656,0.100196,0.422160}%
\pgfsetstrokecolor{currentstroke}%
\pgfsetdash{}{0pt}%
\pgfpathmoveto{\pgfqpoint{8.178420in}{5.538446in}}%
\pgfpathlineto{\pgfqpoint{8.128671in}{5.532895in}}%
\pgfusepath{stroke}%
\end{pgfscope}%
\begin{pgfscope}%
\pgfpathrectangle{\pgfqpoint{6.720588in}{4.155455in}}{\pgfqpoint{2.279412in}{2.004545in}}%
\pgfusepath{clip}%
\pgfsetbuttcap%
\pgfsetroundjoin%
\pgfsetlinewidth{0.410583pt}%
\definecolor{currentstroke}{rgb}{0.281924,0.089666,0.412415}%
\pgfsetstrokecolor{currentstroke}%
\pgfsetdash{}{0pt}%
\pgfpathmoveto{\pgfqpoint{8.128671in}{5.532895in}}%
\pgfpathlineto{\pgfqpoint{8.079231in}{5.525630in}}%
\pgfusepath{stroke}%
\end{pgfscope}%
\begin{pgfscope}%
\pgfpathrectangle{\pgfqpoint{6.720588in}{4.155455in}}{\pgfqpoint{2.279412in}{2.004545in}}%
\pgfusepath{clip}%
\pgfsetbuttcap%
\pgfsetroundjoin%
\pgfsetlinewidth{0.466259pt}%
\definecolor{currentstroke}{rgb}{0.282884,0.135920,0.453427}%
\pgfsetstrokecolor{currentstroke}%
\pgfsetdash{}{0pt}%
\pgfpathmoveto{\pgfqpoint{8.079231in}{5.525630in}}%
\pgfpathlineto{\pgfqpoint{8.030229in}{5.516339in}}%
\pgfusepath{stroke}%
\end{pgfscope}%
\begin{pgfscope}%
\pgfpathrectangle{\pgfqpoint{6.720588in}{4.155455in}}{\pgfqpoint{2.279412in}{2.004545in}}%
\pgfusepath{clip}%
\pgfsetbuttcap%
\pgfsetroundjoin%
\pgfsetlinewidth{0.471948pt}%
\definecolor{currentstroke}{rgb}{0.282884,0.135920,0.453427}%
\pgfsetstrokecolor{currentstroke}%
\pgfsetdash{}{0pt}%
\pgfpathmoveto{\pgfqpoint{8.030229in}{5.516339in}}%
\pgfpathlineto{\pgfqpoint{7.981779in}{5.505038in}}%
\pgfusepath{stroke}%
\end{pgfscope}%
\begin{pgfscope}%
\pgfpathrectangle{\pgfqpoint{6.720588in}{4.155455in}}{\pgfqpoint{2.279412in}{2.004545in}}%
\pgfusepath{clip}%
\pgfsetbuttcap%
\pgfsetroundjoin%
\pgfsetlinewidth{0.438123pt}%
\definecolor{currentstroke}{rgb}{0.283091,0.110553,0.431554}%
\pgfsetstrokecolor{currentstroke}%
\pgfsetdash{}{0pt}%
\pgfpathmoveto{\pgfqpoint{7.981779in}{5.505038in}}%
\pgfpathlineto{\pgfqpoint{7.934667in}{5.490240in}}%
\pgfusepath{stroke}%
\end{pgfscope}%
\begin{pgfscope}%
\pgfpathrectangle{\pgfqpoint{6.720588in}{4.155455in}}{\pgfqpoint{2.279412in}{2.004545in}}%
\pgfusepath{clip}%
\pgfsetbuttcap%
\pgfsetroundjoin%
\pgfsetlinewidth{0.339579pt}%
\definecolor{currentstroke}{rgb}{0.273809,0.031497,0.358853}%
\pgfsetstrokecolor{currentstroke}%
\pgfsetdash{}{0pt}%
\pgfpathmoveto{\pgfqpoint{8.314272in}{5.814716in}}%
\pgfpathlineto{\pgfqpoint{8.265133in}{5.809807in}}%
\pgfusepath{stroke}%
\end{pgfscope}%
\begin{pgfscope}%
\pgfpathrectangle{\pgfqpoint{6.720588in}{4.155455in}}{\pgfqpoint{2.279412in}{2.004545in}}%
\pgfusepath{clip}%
\pgfsetbuttcap%
\pgfsetroundjoin%
\pgfsetlinewidth{0.329552pt}%
\definecolor{currentstroke}{rgb}{0.272594,0.025563,0.353093}%
\pgfsetstrokecolor{currentstroke}%
\pgfsetdash{}{0pt}%
\pgfpathmoveto{\pgfqpoint{8.265133in}{5.809807in}}%
\pgfpathlineto{\pgfqpoint{8.215742in}{5.802216in}}%
\pgfusepath{stroke}%
\end{pgfscope}%
\begin{pgfscope}%
\pgfpathrectangle{\pgfqpoint{6.720588in}{4.155455in}}{\pgfqpoint{2.279412in}{2.004545in}}%
\pgfusepath{clip}%
\pgfsetbuttcap%
\pgfsetroundjoin%
\pgfsetlinewidth{0.331855pt}%
\definecolor{currentstroke}{rgb}{0.272594,0.025563,0.353093}%
\pgfsetstrokecolor{currentstroke}%
\pgfsetdash{}{0pt}%
\pgfpathmoveto{\pgfqpoint{8.215742in}{5.802216in}}%
\pgfpathlineto{\pgfqpoint{8.166113in}{5.795989in}}%
\pgfusepath{stroke}%
\end{pgfscope}%
\begin{pgfscope}%
\pgfpathrectangle{\pgfqpoint{6.720588in}{4.155455in}}{\pgfqpoint{2.279412in}{2.004545in}}%
\pgfusepath{clip}%
\pgfsetbuttcap%
\pgfsetroundjoin%
\pgfsetlinewidth{0.343132pt}%
\definecolor{currentstroke}{rgb}{0.274952,0.037752,0.364543}%
\pgfsetstrokecolor{currentstroke}%
\pgfsetdash{}{0pt}%
\pgfpathmoveto{\pgfqpoint{8.166113in}{5.795989in}}%
\pgfpathlineto{\pgfqpoint{8.116754in}{5.789222in}}%
\pgfusepath{stroke}%
\end{pgfscope}%
\begin{pgfscope}%
\pgfpathrectangle{\pgfqpoint{6.720588in}{4.155455in}}{\pgfqpoint{2.279412in}{2.004545in}}%
\pgfusepath{clip}%
\pgfsetbuttcap%
\pgfsetroundjoin%
\pgfsetlinewidth{0.350865pt}%
\definecolor{currentstroke}{rgb}{0.276022,0.044167,0.370164}%
\pgfsetstrokecolor{currentstroke}%
\pgfsetdash{}{0pt}%
\pgfpathmoveto{\pgfqpoint{8.116754in}{5.789222in}}%
\pgfpathlineto{\pgfqpoint{8.067918in}{5.780186in}}%
\pgfusepath{stroke}%
\end{pgfscope}%
\begin{pgfscope}%
\pgfpathrectangle{\pgfqpoint{6.720588in}{4.155455in}}{\pgfqpoint{2.279412in}{2.004545in}}%
\pgfusepath{clip}%
\pgfsetbuttcap%
\pgfsetroundjoin%
\pgfsetlinewidth{0.346609pt}%
\definecolor{currentstroke}{rgb}{0.274952,0.037752,0.364543}%
\pgfsetstrokecolor{currentstroke}%
\pgfsetdash{}{0pt}%
\pgfpathmoveto{\pgfqpoint{8.154043in}{4.544970in}}%
\pgfpathlineto{\pgfqpoint{8.110719in}{4.549714in}}%
\pgfusepath{stroke}%
\end{pgfscope}%
\begin{pgfscope}%
\pgfpathrectangle{\pgfqpoint{6.720588in}{4.155455in}}{\pgfqpoint{2.279412in}{2.004545in}}%
\pgfusepath{clip}%
\pgfsetbuttcap%
\pgfsetroundjoin%
\pgfsetlinewidth{0.343683pt}%
\definecolor{currentstroke}{rgb}{0.274952,0.037752,0.364543}%
\pgfsetstrokecolor{currentstroke}%
\pgfsetdash{}{0pt}%
\pgfpathmoveto{\pgfqpoint{8.110719in}{4.549714in}}%
\pgfpathlineto{\pgfqpoint{8.062403in}{4.559914in}}%
\pgfusepath{stroke}%
\end{pgfscope}%
\begin{pgfscope}%
\pgfpathrectangle{\pgfqpoint{6.720588in}{4.155455in}}{\pgfqpoint{2.279412in}{2.004545in}}%
\pgfusepath{clip}%
\pgfsetbuttcap%
\pgfsetroundjoin%
\pgfsetlinewidth{0.327050pt}%
\definecolor{currentstroke}{rgb}{0.271305,0.019942,0.347269}%
\pgfsetstrokecolor{currentstroke}%
\pgfsetdash{}{0pt}%
\pgfpathmoveto{\pgfqpoint{8.062403in}{4.559914in}}%
\pgfpathlineto{\pgfqpoint{8.014170in}{4.571339in}}%
\pgfusepath{stroke}%
\end{pgfscope}%
\begin{pgfscope}%
\pgfpathrectangle{\pgfqpoint{6.720588in}{4.155455in}}{\pgfqpoint{2.279412in}{2.004545in}}%
\pgfusepath{clip}%
\pgfsetbuttcap%
\pgfsetroundjoin%
\pgfsetlinewidth{0.336916pt}%
\definecolor{currentstroke}{rgb}{0.273809,0.031497,0.358853}%
\pgfsetstrokecolor{currentstroke}%
\pgfsetdash{}{0pt}%
\pgfpathmoveto{\pgfqpoint{8.014170in}{4.571339in}}%
\pgfpathlineto{\pgfqpoint{7.967225in}{4.584854in}}%
\pgfusepath{stroke}%
\end{pgfscope}%
\begin{pgfscope}%
\pgfpathrectangle{\pgfqpoint{6.720588in}{4.155455in}}{\pgfqpoint{2.279412in}{2.004545in}}%
\pgfusepath{clip}%
\pgfsetbuttcap%
\pgfsetroundjoin%
\pgfsetlinewidth{0.344672pt}%
\definecolor{currentstroke}{rgb}{0.274952,0.037752,0.364543}%
\pgfsetstrokecolor{currentstroke}%
\pgfsetdash{}{0pt}%
\pgfpathmoveto{\pgfqpoint{7.967225in}{4.584854in}}%
\pgfpathlineto{\pgfqpoint{7.926492in}{4.598605in}}%
\pgfusepath{stroke}%
\end{pgfscope}%
\begin{pgfscope}%
\pgfpathrectangle{\pgfqpoint{6.720588in}{4.155455in}}{\pgfqpoint{2.279412in}{2.004545in}}%
\pgfusepath{clip}%
\pgfsetbuttcap%
\pgfsetroundjoin%
\pgfsetlinewidth{0.336748pt}%
\definecolor{currentstroke}{rgb}{0.273809,0.031497,0.358853}%
\pgfsetstrokecolor{currentstroke}%
\pgfsetdash{}{0pt}%
\pgfpathmoveto{\pgfqpoint{8.527089in}{4.751766in}}%
\pgfpathlineto{\pgfqpoint{8.477338in}{4.755562in}}%
\pgfusepath{stroke}%
\end{pgfscope}%
\begin{pgfscope}%
\pgfpathrectangle{\pgfqpoint{6.720588in}{4.155455in}}{\pgfqpoint{2.279412in}{2.004545in}}%
\pgfusepath{clip}%
\pgfsetbuttcap%
\pgfsetroundjoin%
\pgfsetlinewidth{0.334940pt}%
\definecolor{currentstroke}{rgb}{0.272594,0.025563,0.353093}%
\pgfsetstrokecolor{currentstroke}%
\pgfsetdash{}{0pt}%
\pgfpathmoveto{\pgfqpoint{8.477338in}{4.755562in}}%
\pgfpathlineto{\pgfqpoint{8.427634in}{4.759518in}}%
\pgfusepath{stroke}%
\end{pgfscope}%
\begin{pgfscope}%
\pgfpathrectangle{\pgfqpoint{6.720588in}{4.155455in}}{\pgfqpoint{2.279412in}{2.004545in}}%
\pgfusepath{clip}%
\pgfsetbuttcap%
\pgfsetroundjoin%
\pgfsetlinewidth{0.333227pt}%
\definecolor{currentstroke}{rgb}{0.272594,0.025563,0.353093}%
\pgfsetstrokecolor{currentstroke}%
\pgfsetdash{}{0pt}%
\pgfpathmoveto{\pgfqpoint{8.427634in}{4.759518in}}%
\pgfpathlineto{\pgfqpoint{8.377545in}{4.761529in}}%
\pgfusepath{stroke}%
\end{pgfscope}%
\begin{pgfscope}%
\pgfpathrectangle{\pgfqpoint{6.720588in}{4.155455in}}{\pgfqpoint{2.279412in}{2.004545in}}%
\pgfusepath{clip}%
\pgfsetbuttcap%
\pgfsetroundjoin%
\pgfsetlinewidth{0.359474pt}%
\definecolor{currentstroke}{rgb}{0.277018,0.050344,0.375715}%
\pgfsetstrokecolor{currentstroke}%
\pgfsetdash{}{0pt}%
\pgfpathmoveto{\pgfqpoint{8.377545in}{4.761529in}}%
\pgfpathlineto{\pgfqpoint{8.327497in}{4.764337in}}%
\pgfusepath{stroke}%
\end{pgfscope}%
\begin{pgfscope}%
\pgfpathrectangle{\pgfqpoint{6.720588in}{4.155455in}}{\pgfqpoint{2.279412in}{2.004545in}}%
\pgfusepath{clip}%
\pgfsetbuttcap%
\pgfsetroundjoin%
\pgfsetlinewidth{0.376508pt}%
\definecolor{currentstroke}{rgb}{0.278791,0.062145,0.386592}%
\pgfsetstrokecolor{currentstroke}%
\pgfsetdash{}{0pt}%
\pgfpathmoveto{\pgfqpoint{8.327497in}{4.764337in}}%
\pgfpathlineto{\pgfqpoint{8.277498in}{4.767636in}}%
\pgfusepath{stroke}%
\end{pgfscope}%
\begin{pgfscope}%
\pgfpathrectangle{\pgfqpoint{6.720588in}{4.155455in}}{\pgfqpoint{2.279412in}{2.004545in}}%
\pgfusepath{clip}%
\pgfsetbuttcap%
\pgfsetroundjoin%
\pgfsetlinewidth{0.389342pt}%
\definecolor{currentstroke}{rgb}{0.280267,0.073417,0.397163}%
\pgfsetstrokecolor{currentstroke}%
\pgfsetdash{}{0pt}%
\pgfpathmoveto{\pgfqpoint{8.277498in}{4.767636in}}%
\pgfpathlineto{\pgfqpoint{8.227586in}{4.771808in}}%
\pgfusepath{stroke}%
\end{pgfscope}%
\begin{pgfscope}%
\pgfpathrectangle{\pgfqpoint{6.720588in}{4.155455in}}{\pgfqpoint{2.279412in}{2.004545in}}%
\pgfusepath{clip}%
\pgfsetbuttcap%
\pgfsetroundjoin%
\pgfsetlinewidth{0.420497pt}%
\definecolor{currentstroke}{rgb}{0.282656,0.100196,0.422160}%
\pgfsetstrokecolor{currentstroke}%
\pgfsetdash{}{0pt}%
\pgfpathmoveto{\pgfqpoint{8.227586in}{4.771808in}}%
\pgfpathlineto{\pgfqpoint{8.177794in}{4.777055in}}%
\pgfusepath{stroke}%
\end{pgfscope}%
\begin{pgfscope}%
\pgfpathrectangle{\pgfqpoint{6.720588in}{4.155455in}}{\pgfqpoint{2.279412in}{2.004545in}}%
\pgfusepath{clip}%
\pgfsetbuttcap%
\pgfsetroundjoin%
\pgfsetlinewidth{0.432331pt}%
\definecolor{currentstroke}{rgb}{0.283091,0.110553,0.431554}%
\pgfsetstrokecolor{currentstroke}%
\pgfsetdash{}{0pt}%
\pgfpathmoveto{\pgfqpoint{8.177794in}{4.777055in}}%
\pgfpathlineto{\pgfqpoint{8.128153in}{4.783310in}}%
\pgfusepath{stroke}%
\end{pgfscope}%
\begin{pgfscope}%
\pgfpathrectangle{\pgfqpoint{6.720588in}{4.155455in}}{\pgfqpoint{2.279412in}{2.004545in}}%
\pgfusepath{clip}%
\pgfsetbuttcap%
\pgfsetroundjoin%
\pgfsetlinewidth{0.438527pt}%
\definecolor{currentstroke}{rgb}{0.283197,0.115680,0.436115}%
\pgfsetstrokecolor{currentstroke}%
\pgfsetdash{}{0pt}%
\pgfpathmoveto{\pgfqpoint{8.128153in}{4.783310in}}%
\pgfpathlineto{\pgfqpoint{8.078743in}{4.790783in}}%
\pgfusepath{stroke}%
\end{pgfscope}%
\begin{pgfscope}%
\pgfpathrectangle{\pgfqpoint{6.720588in}{4.155455in}}{\pgfqpoint{2.279412in}{2.004545in}}%
\pgfusepath{clip}%
\pgfsetbuttcap%
\pgfsetroundjoin%
\pgfsetlinewidth{0.480344pt}%
\definecolor{currentstroke}{rgb}{0.282290,0.145912,0.461510}%
\pgfsetstrokecolor{currentstroke}%
\pgfsetdash{}{0pt}%
\pgfpathmoveto{\pgfqpoint{8.078743in}{4.790783in}}%
\pgfpathlineto{\pgfqpoint{8.029598in}{4.799480in}}%
\pgfusepath{stroke}%
\end{pgfscope}%
\begin{pgfscope}%
\pgfpathrectangle{\pgfqpoint{6.720588in}{4.155455in}}{\pgfqpoint{2.279412in}{2.004545in}}%
\pgfusepath{clip}%
\pgfsetbuttcap%
\pgfsetroundjoin%
\pgfsetlinewidth{0.490314pt}%
\definecolor{currentstroke}{rgb}{0.281887,0.150881,0.465405}%
\pgfsetstrokecolor{currentstroke}%
\pgfsetdash{}{0pt}%
\pgfpathmoveto{\pgfqpoint{8.029598in}{4.799480in}}%
\pgfpathlineto{\pgfqpoint{7.981076in}{4.810391in}}%
\pgfusepath{stroke}%
\end{pgfscope}%
\begin{pgfscope}%
\pgfpathrectangle{\pgfqpoint{6.720588in}{4.155455in}}{\pgfqpoint{2.279412in}{2.004545in}}%
\pgfusepath{clip}%
\pgfsetbuttcap%
\pgfsetroundjoin%
\pgfsetlinewidth{0.486882pt}%
\definecolor{currentstroke}{rgb}{0.281887,0.150881,0.465405}%
\pgfsetstrokecolor{currentstroke}%
\pgfsetdash{}{0pt}%
\pgfpathmoveto{\pgfqpoint{7.981076in}{4.810391in}}%
\pgfpathlineto{\pgfqpoint{7.933529in}{4.824180in}}%
\pgfusepath{stroke}%
\end{pgfscope}%
\begin{pgfscope}%
\pgfpathrectangle{\pgfqpoint{6.720588in}{4.155455in}}{\pgfqpoint{2.279412in}{2.004545in}}%
\pgfusepath{clip}%
\pgfsetbuttcap%
\pgfsetroundjoin%
\pgfsetlinewidth{0.321546pt}%
\definecolor{currentstroke}{rgb}{0.269944,0.014625,0.341379}%
\pgfsetstrokecolor{currentstroke}%
\pgfsetdash{}{0pt}%
\pgfpathmoveto{\pgfqpoint{8.527089in}{5.608795in}}%
\pgfpathlineto{\pgfqpoint{8.476992in}{5.607548in}}%
\pgfusepath{stroke}%
\end{pgfscope}%
\begin{pgfscope}%
\pgfpathrectangle{\pgfqpoint{6.720588in}{4.155455in}}{\pgfqpoint{2.279412in}{2.004545in}}%
\pgfusepath{clip}%
\pgfsetbuttcap%
\pgfsetroundjoin%
\pgfsetlinewidth{0.331632pt}%
\definecolor{currentstroke}{rgb}{0.272594,0.025563,0.353093}%
\pgfsetstrokecolor{currentstroke}%
\pgfsetdash{}{0pt}%
\pgfpathmoveto{\pgfqpoint{8.476992in}{5.607548in}}%
\pgfpathlineto{\pgfqpoint{8.426927in}{5.605307in}}%
\pgfusepath{stroke}%
\end{pgfscope}%
\begin{pgfscope}%
\pgfpathrectangle{\pgfqpoint{6.720588in}{4.155455in}}{\pgfqpoint{2.279412in}{2.004545in}}%
\pgfusepath{clip}%
\pgfsetbuttcap%
\pgfsetroundjoin%
\pgfsetlinewidth{0.341576pt}%
\definecolor{currentstroke}{rgb}{0.273809,0.031497,0.358853}%
\pgfsetstrokecolor{currentstroke}%
\pgfsetdash{}{0pt}%
\pgfpathmoveto{\pgfqpoint{8.426927in}{5.605307in}}%
\pgfpathlineto{\pgfqpoint{8.376931in}{5.601931in}}%
\pgfusepath{stroke}%
\end{pgfscope}%
\begin{pgfscope}%
\pgfpathrectangle{\pgfqpoint{6.720588in}{4.155455in}}{\pgfqpoint{2.279412in}{2.004545in}}%
\pgfusepath{clip}%
\pgfsetbuttcap%
\pgfsetroundjoin%
\pgfsetlinewidth{0.352230pt}%
\definecolor{currentstroke}{rgb}{0.276022,0.044167,0.370164}%
\pgfsetstrokecolor{currentstroke}%
\pgfsetdash{}{0pt}%
\pgfpathmoveto{\pgfqpoint{8.376931in}{5.601931in}}%
\pgfpathlineto{\pgfqpoint{8.326950in}{5.598327in}}%
\pgfusepath{stroke}%
\end{pgfscope}%
\begin{pgfscope}%
\pgfpathrectangle{\pgfqpoint{6.720588in}{4.155455in}}{\pgfqpoint{2.279412in}{2.004545in}}%
\pgfusepath{clip}%
\pgfsetbuttcap%
\pgfsetroundjoin%
\pgfsetlinewidth{0.362761pt}%
\definecolor{currentstroke}{rgb}{0.277018,0.050344,0.375715}%
\pgfsetstrokecolor{currentstroke}%
\pgfsetdash{}{0pt}%
\pgfpathmoveto{\pgfqpoint{8.326950in}{5.598327in}}%
\pgfpathlineto{\pgfqpoint{8.276984in}{5.594602in}}%
\pgfusepath{stroke}%
\end{pgfscope}%
\begin{pgfscope}%
\pgfpathrectangle{\pgfqpoint{6.720588in}{4.155455in}}{\pgfqpoint{2.279412in}{2.004545in}}%
\pgfusepath{clip}%
\pgfsetbuttcap%
\pgfsetroundjoin%
\pgfsetlinewidth{0.367713pt}%
\definecolor{currentstroke}{rgb}{0.277941,0.056324,0.381191}%
\pgfsetstrokecolor{currentstroke}%
\pgfsetdash{}{0pt}%
\pgfpathmoveto{\pgfqpoint{8.276984in}{5.594602in}}%
\pgfpathlineto{\pgfqpoint{8.227093in}{5.590197in}}%
\pgfusepath{stroke}%
\end{pgfscope}%
\begin{pgfscope}%
\pgfpathrectangle{\pgfqpoint{6.720588in}{4.155455in}}{\pgfqpoint{2.279412in}{2.004545in}}%
\pgfusepath{clip}%
\pgfsetbuttcap%
\pgfsetroundjoin%
\pgfsetlinewidth{0.384605pt}%
\definecolor{currentstroke}{rgb}{0.280267,0.073417,0.397163}%
\pgfsetstrokecolor{currentstroke}%
\pgfsetdash{}{0pt}%
\pgfpathmoveto{\pgfqpoint{8.227093in}{5.590197in}}%
\pgfpathlineto{\pgfqpoint{8.177360in}{5.584580in}}%
\pgfusepath{stroke}%
\end{pgfscope}%
\begin{pgfscope}%
\pgfpathrectangle{\pgfqpoint{6.720588in}{4.155455in}}{\pgfqpoint{2.279412in}{2.004545in}}%
\pgfusepath{clip}%
\pgfsetbuttcap%
\pgfsetroundjoin%
\pgfsetlinewidth{0.392481pt}%
\definecolor{currentstroke}{rgb}{0.280894,0.078907,0.402329}%
\pgfsetstrokecolor{currentstroke}%
\pgfsetdash{}{0pt}%
\pgfpathmoveto{\pgfqpoint{8.177360in}{5.584580in}}%
\pgfpathlineto{\pgfqpoint{8.127882in}{5.577449in}}%
\pgfusepath{stroke}%
\end{pgfscope}%
\begin{pgfscope}%
\pgfpathrectangle{\pgfqpoint{6.720588in}{4.155455in}}{\pgfqpoint{2.279412in}{2.004545in}}%
\pgfusepath{clip}%
\pgfsetbuttcap%
\pgfsetroundjoin%
\pgfsetlinewidth{0.429400pt}%
\definecolor{currentstroke}{rgb}{0.282910,0.105393,0.426902}%
\pgfsetstrokecolor{currentstroke}%
\pgfsetdash{}{0pt}%
\pgfpathmoveto{\pgfqpoint{8.127882in}{5.577449in}}%
\pgfpathlineto{\pgfqpoint{8.078632in}{5.569178in}}%
\pgfusepath{stroke}%
\end{pgfscope}%
\begin{pgfscope}%
\pgfpathrectangle{\pgfqpoint{6.720588in}{4.155455in}}{\pgfqpoint{2.279412in}{2.004545in}}%
\pgfusepath{clip}%
\pgfsetbuttcap%
\pgfsetroundjoin%
\pgfsetlinewidth{0.425429pt}%
\definecolor{currentstroke}{rgb}{0.282910,0.105393,0.426902}%
\pgfsetstrokecolor{currentstroke}%
\pgfsetdash{}{0pt}%
\pgfpathmoveto{\pgfqpoint{8.078632in}{5.569178in}}%
\pgfpathlineto{\pgfqpoint{8.029840in}{5.559172in}}%
\pgfusepath{stroke}%
\end{pgfscope}%
\begin{pgfscope}%
\pgfpathrectangle{\pgfqpoint{6.720588in}{4.155455in}}{\pgfqpoint{2.279412in}{2.004545in}}%
\pgfusepath{clip}%
\pgfsetbuttcap%
\pgfsetroundjoin%
\pgfsetlinewidth{0.384798pt}%
\definecolor{currentstroke}{rgb}{0.280267,0.073417,0.397163}%
\pgfsetstrokecolor{currentstroke}%
\pgfsetdash{}{0pt}%
\pgfpathmoveto{\pgfqpoint{8.029840in}{5.559172in}}%
\pgfpathlineto{\pgfqpoint{7.981765in}{5.546786in}}%
\pgfusepath{stroke}%
\end{pgfscope}%
\begin{pgfscope}%
\pgfpathrectangle{\pgfqpoint{6.720588in}{4.155455in}}{\pgfqpoint{2.279412in}{2.004545in}}%
\pgfusepath{clip}%
\pgfsetbuttcap%
\pgfsetroundjoin%
\pgfsetlinewidth{0.429082pt}%
\definecolor{currentstroke}{rgb}{0.282910,0.105393,0.426902}%
\pgfsetstrokecolor{currentstroke}%
\pgfsetdash{}{0pt}%
\pgfpathmoveto{\pgfqpoint{7.981765in}{5.546786in}}%
\pgfpathlineto{\pgfqpoint{7.934846in}{5.531543in}}%
\pgfusepath{stroke}%
\end{pgfscope}%
\begin{pgfscope}%
\pgfpathrectangle{\pgfqpoint{6.720588in}{4.155455in}}{\pgfqpoint{2.279412in}{2.004545in}}%
\pgfusepath{clip}%
\pgfsetbuttcap%
\pgfsetroundjoin%
\pgfsetlinewidth{0.333171pt}%
\definecolor{currentstroke}{rgb}{0.272594,0.025563,0.353093}%
\pgfsetstrokecolor{currentstroke}%
\pgfsetdash{}{0pt}%
\pgfpathmoveto{\pgfqpoint{8.475797in}{4.841980in}}%
\pgfpathlineto{\pgfqpoint{8.425694in}{4.843666in}}%
\pgfusepath{stroke}%
\end{pgfscope}%
\begin{pgfscope}%
\pgfpathrectangle{\pgfqpoint{6.720588in}{4.155455in}}{\pgfqpoint{2.279412in}{2.004545in}}%
\pgfusepath{clip}%
\pgfsetbuttcap%
\pgfsetroundjoin%
\pgfsetlinewidth{0.364284pt}%
\definecolor{currentstroke}{rgb}{0.277941,0.056324,0.381191}%
\pgfsetstrokecolor{currentstroke}%
\pgfsetdash{}{0pt}%
\pgfpathmoveto{\pgfqpoint{8.425694in}{4.843666in}}%
\pgfpathlineto{\pgfqpoint{8.375580in}{4.845333in}}%
\pgfusepath{stroke}%
\end{pgfscope}%
\begin{pgfscope}%
\pgfpathrectangle{\pgfqpoint{6.720588in}{4.155455in}}{\pgfqpoint{2.279412in}{2.004545in}}%
\pgfusepath{clip}%
\pgfsetbuttcap%
\pgfsetroundjoin%
\pgfsetlinewidth{0.372541pt}%
\definecolor{currentstroke}{rgb}{0.278791,0.062145,0.386592}%
\pgfsetstrokecolor{currentstroke}%
\pgfsetdash{}{0pt}%
\pgfpathmoveto{\pgfqpoint{8.375580in}{4.845333in}}%
\pgfpathlineto{\pgfqpoint{8.325489in}{4.847392in}}%
\pgfusepath{stroke}%
\end{pgfscope}%
\begin{pgfscope}%
\pgfpathrectangle{\pgfqpoint{6.720588in}{4.155455in}}{\pgfqpoint{2.279412in}{2.004545in}}%
\pgfusepath{clip}%
\pgfsetbuttcap%
\pgfsetroundjoin%
\pgfsetlinewidth{0.418000pt}%
\definecolor{currentstroke}{rgb}{0.282656,0.100196,0.422160}%
\pgfsetstrokecolor{currentstroke}%
\pgfsetdash{}{0pt}%
\pgfpathmoveto{\pgfqpoint{8.325489in}{4.847392in}}%
\pgfpathlineto{\pgfqpoint{8.275432in}{4.850063in}}%
\pgfusepath{stroke}%
\end{pgfscope}%
\begin{pgfscope}%
\pgfpathrectangle{\pgfqpoint{6.720588in}{4.155455in}}{\pgfqpoint{2.279412in}{2.004545in}}%
\pgfusepath{clip}%
\pgfsetbuttcap%
\pgfsetroundjoin%
\pgfsetlinewidth{0.448029pt}%
\definecolor{currentstroke}{rgb}{0.283229,0.120777,0.440584}%
\pgfsetstrokecolor{currentstroke}%
\pgfsetdash{}{0pt}%
\pgfpathmoveto{\pgfqpoint{8.275432in}{4.850063in}}%
\pgfpathlineto{\pgfqpoint{8.225417in}{4.853282in}}%
\pgfusepath{stroke}%
\end{pgfscope}%
\begin{pgfscope}%
\pgfpathrectangle{\pgfqpoint{6.720588in}{4.155455in}}{\pgfqpoint{2.279412in}{2.004545in}}%
\pgfusepath{clip}%
\pgfsetbuttcap%
\pgfsetroundjoin%
\pgfsetlinewidth{0.330828pt}%
\definecolor{currentstroke}{rgb}{0.272594,0.025563,0.353093}%
\pgfsetstrokecolor{currentstroke}%
\pgfsetdash{}{0pt}%
\pgfpathmoveto{\pgfqpoint{8.373213in}{5.699009in}}%
\pgfpathlineto{\pgfqpoint{8.323504in}{5.693438in}}%
\pgfusepath{stroke}%
\end{pgfscope}%
\begin{pgfscope}%
\pgfpathrectangle{\pgfqpoint{6.720588in}{4.155455in}}{\pgfqpoint{2.279412in}{2.004545in}}%
\pgfusepath{clip}%
\pgfsetbuttcap%
\pgfsetroundjoin%
\pgfsetlinewidth{0.335356pt}%
\definecolor{currentstroke}{rgb}{0.272594,0.025563,0.353093}%
\pgfsetstrokecolor{currentstroke}%
\pgfsetdash{}{0pt}%
\pgfpathmoveto{\pgfqpoint{8.323504in}{5.693438in}}%
\pgfpathlineto{\pgfqpoint{8.273916in}{5.687051in}}%
\pgfusepath{stroke}%
\end{pgfscope}%
\begin{pgfscope}%
\pgfpathrectangle{\pgfqpoint{6.720588in}{4.155455in}}{\pgfqpoint{2.279412in}{2.004545in}}%
\pgfusepath{clip}%
\pgfsetbuttcap%
\pgfsetroundjoin%
\pgfsetlinewidth{0.355540pt}%
\definecolor{currentstroke}{rgb}{0.276022,0.044167,0.370164}%
\pgfsetstrokecolor{currentstroke}%
\pgfsetdash{}{0pt}%
\pgfpathmoveto{\pgfqpoint{8.273916in}{5.687051in}}%
\pgfpathlineto{\pgfqpoint{8.224285in}{5.681005in}}%
\pgfusepath{stroke}%
\end{pgfscope}%
\begin{pgfscope}%
\pgfpathrectangle{\pgfqpoint{6.720588in}{4.155455in}}{\pgfqpoint{2.279412in}{2.004545in}}%
\pgfusepath{clip}%
\pgfsetbuttcap%
\pgfsetroundjoin%
\pgfsetlinewidth{0.366859pt}%
\definecolor{currentstroke}{rgb}{0.277941,0.056324,0.381191}%
\pgfsetstrokecolor{currentstroke}%
\pgfsetdash{}{0pt}%
\pgfpathmoveto{\pgfqpoint{8.224285in}{5.681005in}}%
\pgfpathlineto{\pgfqpoint{8.175044in}{5.673007in}}%
\pgfusepath{stroke}%
\end{pgfscope}%
\begin{pgfscope}%
\pgfpathrectangle{\pgfqpoint{6.720588in}{4.155455in}}{\pgfqpoint{2.279412in}{2.004545in}}%
\pgfusepath{clip}%
\pgfsetbuttcap%
\pgfsetroundjoin%
\pgfsetlinewidth{0.327758pt}%
\definecolor{currentstroke}{rgb}{0.271305,0.019942,0.347269}%
\pgfsetstrokecolor{currentstroke}%
\pgfsetdash{}{0pt}%
\pgfpathmoveto{\pgfqpoint{8.175044in}{5.673007in}}%
\pgfpathlineto{\pgfqpoint{8.126011in}{5.663885in}}%
\pgfusepath{stroke}%
\end{pgfscope}%
\begin{pgfscope}%
\pgfpathrectangle{\pgfqpoint{6.720588in}{4.155455in}}{\pgfqpoint{2.279412in}{2.004545in}}%
\pgfusepath{clip}%
\pgfsetbuttcap%
\pgfsetroundjoin%
\pgfsetlinewidth{0.383011pt}%
\definecolor{currentstroke}{rgb}{0.279566,0.067836,0.391917}%
\pgfsetstrokecolor{currentstroke}%
\pgfsetdash{}{0pt}%
\pgfpathmoveto{\pgfqpoint{8.126011in}{5.663885in}}%
\pgfpathlineto{\pgfqpoint{8.077124in}{5.654252in}}%
\pgfusepath{stroke}%
\end{pgfscope}%
\begin{pgfscope}%
\pgfpathrectangle{\pgfqpoint{6.720588in}{4.155455in}}{\pgfqpoint{2.279412in}{2.004545in}}%
\pgfusepath{clip}%
\pgfsetbuttcap%
\pgfsetroundjoin%
\pgfsetlinewidth{0.356543pt}%
\definecolor{currentstroke}{rgb}{0.277018,0.050344,0.375715}%
\pgfsetstrokecolor{currentstroke}%
\pgfsetdash{}{0pt}%
\pgfpathmoveto{\pgfqpoint{8.077124in}{5.654252in}}%
\pgfpathlineto{\pgfqpoint{8.028816in}{5.642491in}}%
\pgfusepath{stroke}%
\end{pgfscope}%
\begin{pgfscope}%
\pgfpathrectangle{\pgfqpoint{6.720588in}{4.155455in}}{\pgfqpoint{2.279412in}{2.004545in}}%
\pgfusepath{clip}%
\pgfsetbuttcap%
\pgfsetroundjoin%
\pgfsetlinewidth{0.367428pt}%
\definecolor{currentstroke}{rgb}{0.277941,0.056324,0.381191}%
\pgfsetstrokecolor{currentstroke}%
\pgfsetdash{}{0pt}%
\pgfpathmoveto{\pgfqpoint{8.028816in}{5.642491in}}%
\pgfpathlineto{\pgfqpoint{7.981100in}{5.629004in}}%
\pgfusepath{stroke}%
\end{pgfscope}%
\begin{pgfscope}%
\pgfpathrectangle{\pgfqpoint{6.720588in}{4.155455in}}{\pgfqpoint{2.279412in}{2.004545in}}%
\pgfusepath{clip}%
\pgfsetbuttcap%
\pgfsetroundjoin%
\pgfsetlinewidth{0.370196pt}%
\definecolor{currentstroke}{rgb}{0.278791,0.062145,0.386592}%
\pgfsetstrokecolor{currentstroke}%
\pgfsetdash{}{0pt}%
\pgfpathmoveto{\pgfqpoint{7.981100in}{5.629004in}}%
\pgfpathlineto{\pgfqpoint{7.934997in}{5.611923in}}%
\pgfusepath{stroke}%
\end{pgfscope}%
\begin{pgfscope}%
\pgfpathrectangle{\pgfqpoint{6.720588in}{4.155455in}}{\pgfqpoint{2.279412in}{2.004545in}}%
\pgfusepath{clip}%
\pgfsetbuttcap%
\pgfsetroundjoin%
\pgfsetlinewidth{0.408245pt}%
\definecolor{currentstroke}{rgb}{0.281924,0.089666,0.412415}%
\pgfsetstrokecolor{currentstroke}%
\pgfsetdash{}{0pt}%
\pgfpathmoveto{\pgfqpoint{7.934997in}{5.611923in}}%
\pgfpathlineto{\pgfqpoint{7.890220in}{5.593222in}}%
\pgfusepath{stroke}%
\end{pgfscope}%
\begin{pgfscope}%
\pgfpathrectangle{\pgfqpoint{6.720588in}{4.155455in}}{\pgfqpoint{2.279412in}{2.004545in}}%
\pgfusepath{clip}%
\pgfsetbuttcap%
\pgfsetroundjoin%
\pgfsetlinewidth{0.369317pt}%
\definecolor{currentstroke}{rgb}{0.277941,0.056324,0.381191}%
\pgfsetstrokecolor{currentstroke}%
\pgfsetdash{}{0pt}%
\pgfpathmoveto{\pgfqpoint{8.165934in}{5.707330in}}%
\pgfpathlineto{\pgfqpoint{8.116754in}{5.699009in}}%
\pgfusepath{stroke}%
\end{pgfscope}%
\begin{pgfscope}%
\pgfpathrectangle{\pgfqpoint{6.720588in}{4.155455in}}{\pgfqpoint{2.279412in}{2.004545in}}%
\pgfusepath{clip}%
\pgfsetbuttcap%
\pgfsetroundjoin%
\pgfsetlinewidth{0.364308pt}%
\definecolor{currentstroke}{rgb}{0.277941,0.056324,0.381191}%
\pgfsetstrokecolor{currentstroke}%
\pgfsetdash{}{0pt}%
\pgfpathmoveto{\pgfqpoint{8.116754in}{5.699009in}}%
\pgfpathlineto{\pgfqpoint{8.068034in}{5.688682in}}%
\pgfusepath{stroke}%
\end{pgfscope}%
\begin{pgfscope}%
\pgfpathrectangle{\pgfqpoint{6.720588in}{4.155455in}}{\pgfqpoint{2.279412in}{2.004545in}}%
\pgfusepath{clip}%
\pgfsetbuttcap%
\pgfsetroundjoin%
\pgfsetlinewidth{0.357904pt}%
\definecolor{currentstroke}{rgb}{0.277018,0.050344,0.375715}%
\pgfsetstrokecolor{currentstroke}%
\pgfsetdash{}{0pt}%
\pgfpathmoveto{\pgfqpoint{8.068034in}{5.688682in}}%
\pgfpathlineto{\pgfqpoint{8.019416in}{5.678091in}}%
\pgfusepath{stroke}%
\end{pgfscope}%
\begin{pgfscope}%
\pgfpathrectangle{\pgfqpoint{6.720588in}{4.155455in}}{\pgfqpoint{2.279412in}{2.004545in}}%
\pgfusepath{clip}%
\pgfsetbuttcap%
\pgfsetroundjoin%
\pgfsetlinewidth{0.353716pt}%
\definecolor{currentstroke}{rgb}{0.276022,0.044167,0.370164}%
\pgfsetstrokecolor{currentstroke}%
\pgfsetdash{}{0pt}%
\pgfpathmoveto{\pgfqpoint{8.019416in}{5.678091in}}%
\pgfpathlineto{\pgfqpoint{7.973044in}{5.662418in}}%
\pgfusepath{stroke}%
\end{pgfscope}%
\begin{pgfscope}%
\pgfpathrectangle{\pgfqpoint{6.720588in}{4.155455in}}{\pgfqpoint{2.279412in}{2.004545in}}%
\pgfusepath{clip}%
\pgfsetbuttcap%
\pgfsetroundjoin%
\pgfsetlinewidth{0.397387pt}%
\definecolor{currentstroke}{rgb}{0.281446,0.084320,0.407414}%
\pgfsetstrokecolor{currentstroke}%
\pgfsetdash{}{0pt}%
\pgfpathmoveto{\pgfqpoint{7.973044in}{5.662418in}}%
\pgfpathlineto{\pgfqpoint{7.933223in}{5.646745in}}%
\pgfusepath{stroke}%
\end{pgfscope}%
\begin{pgfscope}%
\pgfpathrectangle{\pgfqpoint{6.720588in}{4.155455in}}{\pgfqpoint{2.279412in}{2.004545in}}%
\pgfusepath{clip}%
\pgfsetbuttcap%
\pgfsetroundjoin%
\pgfsetlinewidth{0.324255pt}%
\definecolor{currentstroke}{rgb}{0.271305,0.019942,0.347269}%
\pgfsetstrokecolor{currentstroke}%
\pgfsetdash{}{0pt}%
\pgfpathmoveto{\pgfqpoint{8.465158in}{4.627407in}}%
\pgfpathlineto{\pgfqpoint{8.415215in}{4.631005in}}%
\pgfusepath{stroke}%
\end{pgfscope}%
\begin{pgfscope}%
\pgfpathrectangle{\pgfqpoint{6.720588in}{4.155455in}}{\pgfqpoint{2.279412in}{2.004545in}}%
\pgfusepath{clip}%
\pgfsetbuttcap%
\pgfsetroundjoin%
\pgfsetlinewidth{0.327498pt}%
\definecolor{currentstroke}{rgb}{0.271305,0.019942,0.347269}%
\pgfsetstrokecolor{currentstroke}%
\pgfsetdash{}{0pt}%
\pgfpathmoveto{\pgfqpoint{8.415215in}{4.631005in}}%
\pgfpathlineto{\pgfqpoint{8.365221in}{4.633720in}}%
\pgfusepath{stroke}%
\end{pgfscope}%
\begin{pgfscope}%
\pgfpathrectangle{\pgfqpoint{6.720588in}{4.155455in}}{\pgfqpoint{2.279412in}{2.004545in}}%
\pgfusepath{clip}%
\pgfsetbuttcap%
\pgfsetroundjoin%
\pgfsetlinewidth{0.338093pt}%
\definecolor{currentstroke}{rgb}{0.273809,0.031497,0.358853}%
\pgfsetstrokecolor{currentstroke}%
\pgfsetdash{}{0pt}%
\pgfpathmoveto{\pgfqpoint{8.365221in}{4.633720in}}%
\pgfpathlineto{\pgfqpoint{8.315211in}{4.636565in}}%
\pgfusepath{stroke}%
\end{pgfscope}%
\begin{pgfscope}%
\pgfpathrectangle{\pgfqpoint{6.720588in}{4.155455in}}{\pgfqpoint{2.279412in}{2.004545in}}%
\pgfusepath{clip}%
\pgfsetbuttcap%
\pgfsetroundjoin%
\pgfsetlinewidth{0.349303pt}%
\definecolor{currentstroke}{rgb}{0.276022,0.044167,0.370164}%
\pgfsetstrokecolor{currentstroke}%
\pgfsetdash{}{0pt}%
\pgfpathmoveto{\pgfqpoint{8.315211in}{4.636565in}}%
\pgfpathlineto{\pgfqpoint{8.265436in}{4.641832in}}%
\pgfusepath{stroke}%
\end{pgfscope}%
\begin{pgfscope}%
\pgfpathrectangle{\pgfqpoint{6.720588in}{4.155455in}}{\pgfqpoint{2.279412in}{2.004545in}}%
\pgfusepath{clip}%
\pgfsetbuttcap%
\pgfsetroundjoin%
\pgfsetlinewidth{0.374347pt}%
\definecolor{currentstroke}{rgb}{0.278791,0.062145,0.386592}%
\pgfsetstrokecolor{currentstroke}%
\pgfsetdash{}{0pt}%
\pgfpathmoveto{\pgfqpoint{8.265436in}{4.641832in}}%
\pgfpathlineto{\pgfqpoint{8.215755in}{4.647793in}}%
\pgfusepath{stroke}%
\end{pgfscope}%
\begin{pgfscope}%
\pgfpathrectangle{\pgfqpoint{6.720588in}{4.155455in}}{\pgfqpoint{2.279412in}{2.004545in}}%
\pgfusepath{clip}%
\pgfsetbuttcap%
\pgfsetroundjoin%
\pgfsetlinewidth{0.367686pt}%
\definecolor{currentstroke}{rgb}{0.277941,0.056324,0.381191}%
\pgfsetstrokecolor{currentstroke}%
\pgfsetdash{}{0pt}%
\pgfpathmoveto{\pgfqpoint{8.215755in}{4.647793in}}%
\pgfpathlineto{\pgfqpoint{8.166125in}{4.654027in}}%
\pgfusepath{stroke}%
\end{pgfscope}%
\begin{pgfscope}%
\pgfpathrectangle{\pgfqpoint{6.720588in}{4.155455in}}{\pgfqpoint{2.279412in}{2.004545in}}%
\pgfusepath{clip}%
\pgfsetbuttcap%
\pgfsetroundjoin%
\pgfsetlinewidth{0.382533pt}%
\definecolor{currentstroke}{rgb}{0.279566,0.067836,0.391917}%
\pgfsetstrokecolor{currentstroke}%
\pgfsetdash{}{0pt}%
\pgfpathmoveto{\pgfqpoint{8.166125in}{4.654027in}}%
\pgfpathlineto{\pgfqpoint{8.116754in}{4.661553in}}%
\pgfusepath{stroke}%
\end{pgfscope}%
\begin{pgfscope}%
\pgfpathrectangle{\pgfqpoint{6.720588in}{4.155455in}}{\pgfqpoint{2.279412in}{2.004545in}}%
\pgfusepath{clip}%
\pgfsetbuttcap%
\pgfsetroundjoin%
\pgfsetlinewidth{0.387584pt}%
\definecolor{currentstroke}{rgb}{0.280267,0.073417,0.397163}%
\pgfsetstrokecolor{currentstroke}%
\pgfsetdash{}{0pt}%
\pgfpathmoveto{\pgfqpoint{8.116754in}{4.661553in}}%
\pgfpathlineto{\pgfqpoint{8.067633in}{4.670266in}}%
\pgfusepath{stroke}%
\end{pgfscope}%
\begin{pgfscope}%
\pgfpathrectangle{\pgfqpoint{6.720588in}{4.155455in}}{\pgfqpoint{2.279412in}{2.004545in}}%
\pgfusepath{clip}%
\pgfsetbuttcap%
\pgfsetroundjoin%
\pgfsetlinewidth{0.328017pt}%
\definecolor{currentstroke}{rgb}{0.271305,0.019942,0.347269}%
\pgfsetstrokecolor{currentstroke}%
\pgfsetdash{}{0pt}%
\pgfpathmoveto{\pgfqpoint{8.468707in}{5.674783in}}%
\pgfpathlineto{\pgfqpoint{8.418762in}{5.671547in}}%
\pgfusepath{stroke}%
\end{pgfscope}%
\begin{pgfscope}%
\pgfpathrectangle{\pgfqpoint{6.720588in}{4.155455in}}{\pgfqpoint{2.279412in}{2.004545in}}%
\pgfusepath{clip}%
\pgfsetbuttcap%
\pgfsetroundjoin%
\pgfsetlinewidth{0.331448pt}%
\definecolor{currentstroke}{rgb}{0.272594,0.025563,0.353093}%
\pgfsetstrokecolor{currentstroke}%
\pgfsetdash{}{0pt}%
\pgfpathmoveto{\pgfqpoint{8.418762in}{5.671547in}}%
\pgfpathlineto{\pgfqpoint{8.368871in}{5.667434in}}%
\pgfusepath{stroke}%
\end{pgfscope}%
\begin{pgfscope}%
\pgfpathrectangle{\pgfqpoint{6.720588in}{4.155455in}}{\pgfqpoint{2.279412in}{2.004545in}}%
\pgfusepath{clip}%
\pgfsetbuttcap%
\pgfsetroundjoin%
\pgfsetlinewidth{0.334284pt}%
\definecolor{currentstroke}{rgb}{0.272594,0.025563,0.353093}%
\pgfsetstrokecolor{currentstroke}%
\pgfsetdash{}{0pt}%
\pgfpathmoveto{\pgfqpoint{8.368871in}{5.667434in}}%
\pgfpathlineto{\pgfqpoint{8.319051in}{5.662934in}}%
\pgfusepath{stroke}%
\end{pgfscope}%
\begin{pgfscope}%
\pgfpathrectangle{\pgfqpoint{6.720588in}{4.155455in}}{\pgfqpoint{2.279412in}{2.004545in}}%
\pgfusepath{clip}%
\pgfsetbuttcap%
\pgfsetroundjoin%
\pgfsetlinewidth{0.332550pt}%
\definecolor{currentstroke}{rgb}{0.272594,0.025563,0.353093}%
\pgfsetstrokecolor{currentstroke}%
\pgfsetdash{}{0pt}%
\pgfpathmoveto{\pgfqpoint{8.319051in}{5.662934in}}%
\pgfpathlineto{\pgfqpoint{8.269289in}{5.657763in}}%
\pgfusepath{stroke}%
\end{pgfscope}%
\begin{pgfscope}%
\pgfpathrectangle{\pgfqpoint{6.720588in}{4.155455in}}{\pgfqpoint{2.279412in}{2.004545in}}%
\pgfusepath{clip}%
\pgfsetbuttcap%
\pgfsetroundjoin%
\pgfsetlinewidth{0.348352pt}%
\definecolor{currentstroke}{rgb}{0.274952,0.037752,0.364543}%
\pgfsetstrokecolor{currentstroke}%
\pgfsetdash{}{0pt}%
\pgfpathmoveto{\pgfqpoint{8.269289in}{5.657763in}}%
\pgfpathlineto{\pgfqpoint{8.219337in}{5.653902in}}%
\pgfusepath{stroke}%
\end{pgfscope}%
\begin{pgfscope}%
\pgfpathrectangle{\pgfqpoint{6.720588in}{4.155455in}}{\pgfqpoint{2.279412in}{2.004545in}}%
\pgfusepath{clip}%
\pgfsetbuttcap%
\pgfsetroundjoin%
\pgfsetlinewidth{0.394580pt}%
\definecolor{currentstroke}{rgb}{0.280894,0.078907,0.402329}%
\pgfsetstrokecolor{currentstroke}%
\pgfsetdash{}{0pt}%
\pgfpathmoveto{\pgfqpoint{8.010169in}{4.692222in}}%
\pgfpathlineto{\pgfqpoint{7.962878in}{4.706659in}}%
\pgfusepath{stroke}%
\end{pgfscope}%
\begin{pgfscope}%
\pgfpathrectangle{\pgfqpoint{6.720588in}{4.155455in}}{\pgfqpoint{2.279412in}{2.004545in}}%
\pgfusepath{clip}%
\pgfsetbuttcap%
\pgfsetroundjoin%
\pgfsetlinewidth{0.398277pt}%
\definecolor{currentstroke}{rgb}{0.281446,0.084320,0.407414}%
\pgfsetstrokecolor{currentstroke}%
\pgfsetdash{}{0pt}%
\pgfpathmoveto{\pgfqpoint{7.962878in}{4.706659in}}%
\pgfpathlineto{\pgfqpoint{7.917023in}{4.724172in}}%
\pgfusepath{stroke}%
\end{pgfscope}%
\begin{pgfscope}%
\pgfpathrectangle{\pgfqpoint{6.720588in}{4.155455in}}{\pgfqpoint{2.279412in}{2.004545in}}%
\pgfusepath{clip}%
\pgfsetbuttcap%
\pgfsetroundjoin%
\pgfsetlinewidth{0.390755pt}%
\definecolor{currentstroke}{rgb}{0.280894,0.078907,0.402329}%
\pgfsetstrokecolor{currentstroke}%
\pgfsetdash{}{0pt}%
\pgfpathmoveto{\pgfqpoint{7.917023in}{4.724172in}}%
\pgfpathlineto{\pgfqpoint{7.872799in}{4.744530in}}%
\pgfusepath{stroke}%
\end{pgfscope}%
\begin{pgfscope}%
\pgfpathrectangle{\pgfqpoint{6.720588in}{4.155455in}}{\pgfqpoint{2.279412in}{2.004545in}}%
\pgfusepath{clip}%
\pgfsetbuttcap%
\pgfsetroundjoin%
\pgfsetlinewidth{0.412902pt}%
\definecolor{currentstroke}{rgb}{0.282327,0.094955,0.417331}%
\pgfsetstrokecolor{currentstroke}%
\pgfsetdash{}{0pt}%
\pgfpathmoveto{\pgfqpoint{7.872799in}{4.744530in}}%
\pgfpathlineto{\pgfqpoint{7.831576in}{4.768900in}}%
\pgfusepath{stroke}%
\end{pgfscope}%
\begin{pgfscope}%
\pgfpathrectangle{\pgfqpoint{6.720588in}{4.155455in}}{\pgfqpoint{2.279412in}{2.004545in}}%
\pgfusepath{clip}%
\pgfsetbuttcap%
\pgfsetroundjoin%
\pgfsetlinewidth{0.408013pt}%
\definecolor{currentstroke}{rgb}{0.281924,0.089666,0.412415}%
\pgfsetstrokecolor{currentstroke}%
\pgfsetdash{}{0pt}%
\pgfpathmoveto{\pgfqpoint{7.831576in}{4.768900in}}%
\pgfpathlineto{\pgfqpoint{7.831576in}{4.768900in}}%
\pgfusepath{stroke}%
\end{pgfscope}%
\begin{pgfscope}%
\pgfpathrectangle{\pgfqpoint{6.720588in}{4.155455in}}{\pgfqpoint{2.279412in}{2.004545in}}%
\pgfusepath{clip}%
\pgfsetbuttcap%
\pgfsetroundjoin%
\pgfsetlinewidth{0.408013pt}%
\definecolor{currentstroke}{rgb}{0.281924,0.089666,0.412415}%
\pgfsetstrokecolor{currentstroke}%
\pgfsetdash{}{0pt}%
\pgfpathmoveto{\pgfqpoint{7.831576in}{4.768900in}}%
\pgfpathlineto{\pgfqpoint{7.808428in}{4.789937in}}%
\pgfusepath{stroke}%
\end{pgfscope}%
\begin{pgfscope}%
\pgfpathrectangle{\pgfqpoint{6.720588in}{4.155455in}}{\pgfqpoint{2.279412in}{2.004545in}}%
\pgfusepath{clip}%
\pgfsetbuttcap%
\pgfsetroundjoin%
\pgfsetlinewidth{0.513738pt}%
\definecolor{currentstroke}{rgb}{0.280255,0.165693,0.476498}%
\pgfsetstrokecolor{currentstroke}%
\pgfsetdash{}{0pt}%
\pgfpathmoveto{\pgfqpoint{7.808428in}{4.789937in}}%
\pgfpathlineto{\pgfqpoint{7.787348in}{4.812027in}}%
\pgfusepath{stroke}%
\end{pgfscope}%
\begin{pgfscope}%
\pgfpathrectangle{\pgfqpoint{6.720588in}{4.155455in}}{\pgfqpoint{2.279412in}{2.004545in}}%
\pgfusepath{clip}%
\pgfsetbuttcap%
\pgfsetroundjoin%
\pgfsetlinewidth{0.618147pt}%
\definecolor{currentstroke}{rgb}{0.262138,0.242286,0.520837}%
\pgfsetstrokecolor{currentstroke}%
\pgfsetdash{}{0pt}%
\pgfpathmoveto{\pgfqpoint{7.757710in}{5.428368in}}%
\pgfpathlineto{\pgfqpoint{7.723022in}{5.397320in}}%
\pgfusepath{stroke}%
\end{pgfscope}%
\begin{pgfscope}%
\pgfpathrectangle{\pgfqpoint{6.720588in}{4.155455in}}{\pgfqpoint{2.279412in}{2.004545in}}%
\pgfusepath{clip}%
\pgfsetbuttcap%
\pgfsetroundjoin%
\pgfsetlinewidth{1.227616pt}%
\definecolor{currentstroke}{rgb}{0.121148,0.592739,0.544641}%
\pgfsetstrokecolor{currentstroke}%
\pgfsetdash{}{0pt}%
\pgfpathmoveto{\pgfqpoint{7.723022in}{5.397320in}}%
\pgfpathlineto{\pgfqpoint{7.684888in}{5.368714in}}%
\pgfusepath{stroke}%
\end{pgfscope}%
\begin{pgfscope}%
\pgfpathrectangle{\pgfqpoint{6.720588in}{4.155455in}}{\pgfqpoint{2.279412in}{2.004545in}}%
\pgfusepath{clip}%
\pgfsetbuttcap%
\pgfsetroundjoin%
\pgfsetlinewidth{1.326505pt}%
\definecolor{currentstroke}{rgb}{0.126326,0.644107,0.525311}%
\pgfsetstrokecolor{currentstroke}%
\pgfsetdash{}{0pt}%
\pgfpathmoveto{\pgfqpoint{7.684888in}{5.368714in}}%
\pgfpathlineto{\pgfqpoint{7.647286in}{5.339613in}}%
\pgfusepath{stroke}%
\end{pgfscope}%
\begin{pgfscope}%
\pgfpathrectangle{\pgfqpoint{6.720588in}{4.155455in}}{\pgfqpoint{2.279412in}{2.004545in}}%
\pgfusepath{clip}%
\pgfsetbuttcap%
\pgfsetroundjoin%
\pgfsetlinewidth{1.520294pt}%
\definecolor{currentstroke}{rgb}{0.252899,0.742211,0.448284}%
\pgfsetstrokecolor{currentstroke}%
\pgfsetdash{}{0pt}%
\pgfpathmoveto{\pgfqpoint{7.647286in}{5.339613in}}%
\pgfpathlineto{\pgfqpoint{7.609442in}{5.310796in}}%
\pgfusepath{stroke}%
\end{pgfscope}%
\begin{pgfscope}%
\pgfpathrectangle{\pgfqpoint{6.720588in}{4.155455in}}{\pgfqpoint{2.279412in}{2.004545in}}%
\pgfusepath{clip}%
\pgfsetbuttcap%
\pgfsetroundjoin%
\pgfsetlinewidth{1.998693pt}%
\definecolor{currentstroke}{rgb}{0.916242,0.896091,0.100717}%
\pgfsetstrokecolor{currentstroke}%
\pgfsetdash{}{0pt}%
\pgfpathmoveto{\pgfqpoint{7.609442in}{5.310796in}}%
\pgfpathlineto{\pgfqpoint{7.571175in}{5.282398in}}%
\pgfusepath{stroke}%
\end{pgfscope}%
\begin{pgfscope}%
\pgfpathrectangle{\pgfqpoint{6.720588in}{4.155455in}}{\pgfqpoint{2.279412in}{2.004545in}}%
\pgfusepath{clip}%
\pgfsetbuttcap%
\pgfsetroundjoin%
\pgfsetlinewidth{0.900607pt}%
\definecolor{currentstroke}{rgb}{0.187231,0.414746,0.556547}%
\pgfsetstrokecolor{currentstroke}%
\pgfsetdash{}{0pt}%
\pgfpathmoveto{\pgfqpoint{8.014170in}{5.338154in}}%
\pgfpathlineto{\pgfqpoint{7.964734in}{5.330802in}}%
\pgfusepath{stroke}%
\end{pgfscope}%
\begin{pgfscope}%
\pgfpathrectangle{\pgfqpoint{6.720588in}{4.155455in}}{\pgfqpoint{2.279412in}{2.004545in}}%
\pgfusepath{clip}%
\pgfsetbuttcap%
\pgfsetroundjoin%
\pgfsetlinewidth{1.005876pt}%
\definecolor{currentstroke}{rgb}{0.163625,0.471133,0.558148}%
\pgfsetstrokecolor{currentstroke}%
\pgfsetdash{}{0pt}%
\pgfpathmoveto{\pgfqpoint{7.964734in}{5.330802in}}%
\pgfpathlineto{\pgfqpoint{7.915681in}{5.321677in}}%
\pgfusepath{stroke}%
\end{pgfscope}%
\begin{pgfscope}%
\pgfpathrectangle{\pgfqpoint{6.720588in}{4.155455in}}{\pgfqpoint{2.279412in}{2.004545in}}%
\pgfusepath{clip}%
\pgfsetbuttcap%
\pgfsetroundjoin%
\pgfsetlinewidth{1.298551pt}%
\definecolor{currentstroke}{rgb}{0.121380,0.629492,0.531973}%
\pgfsetstrokecolor{currentstroke}%
\pgfsetdash{}{0pt}%
\pgfpathmoveto{\pgfqpoint{7.915681in}{5.321677in}}%
\pgfpathlineto{\pgfqpoint{7.867032in}{5.311007in}}%
\pgfusepath{stroke}%
\end{pgfscope}%
\begin{pgfscope}%
\pgfpathrectangle{\pgfqpoint{6.720588in}{4.155455in}}{\pgfqpoint{2.279412in}{2.004545in}}%
\pgfusepath{clip}%
\pgfsetbuttcap%
\pgfsetroundjoin%
\pgfsetlinewidth{1.677197pt}%
\definecolor{currentstroke}{rgb}{0.440137,0.811138,0.340967}%
\pgfsetstrokecolor{currentstroke}%
\pgfsetdash{}{0pt}%
\pgfpathmoveto{\pgfqpoint{7.867032in}{5.311007in}}%
\pgfpathlineto{\pgfqpoint{7.818767in}{5.299059in}}%
\pgfusepath{stroke}%
\end{pgfscope}%
\begin{pgfscope}%
\pgfpathrectangle{\pgfqpoint{6.720588in}{4.155455in}}{\pgfqpoint{2.279412in}{2.004545in}}%
\pgfusepath{clip}%
\pgfsetbuttcap%
\pgfsetroundjoin%
\pgfsetlinewidth{1.799357pt}%
\definecolor{currentstroke}{rgb}{0.616293,0.852709,0.230052}%
\pgfsetstrokecolor{currentstroke}%
\pgfsetdash{}{0pt}%
\pgfpathmoveto{\pgfqpoint{7.818767in}{5.299059in}}%
\pgfpathlineto{\pgfqpoint{7.770856in}{5.286053in}}%
\pgfusepath{stroke}%
\end{pgfscope}%
\begin{pgfscope}%
\pgfpathrectangle{\pgfqpoint{6.720588in}{4.155455in}}{\pgfqpoint{2.279412in}{2.004545in}}%
\pgfusepath{clip}%
\pgfsetbuttcap%
\pgfsetroundjoin%
\pgfsetlinewidth{1.644951pt}%
\definecolor{currentstroke}{rgb}{0.395174,0.797475,0.367757}%
\pgfsetstrokecolor{currentstroke}%
\pgfsetdash{}{0pt}%
\pgfpathmoveto{\pgfqpoint{7.954177in}{5.283216in}}%
\pgfpathlineto{\pgfqpoint{7.904763in}{5.275697in}}%
\pgfusepath{stroke}%
\end{pgfscope}%
\begin{pgfscope}%
\pgfpathrectangle{\pgfqpoint{6.720588in}{4.155455in}}{\pgfqpoint{2.279412in}{2.004545in}}%
\pgfusepath{clip}%
\pgfsetbuttcap%
\pgfsetroundjoin%
\pgfsetlinewidth{1.866655pt}%
\definecolor{currentstroke}{rgb}{0.720391,0.870350,0.162603}%
\pgfsetstrokecolor{currentstroke}%
\pgfsetdash{}{0pt}%
\pgfpathmoveto{\pgfqpoint{7.904763in}{5.275697in}}%
\pgfpathlineto{\pgfqpoint{7.855531in}{5.267323in}}%
\pgfusepath{stroke}%
\end{pgfscope}%
\begin{pgfscope}%
\pgfpathrectangle{\pgfqpoint{6.720588in}{4.155455in}}{\pgfqpoint{2.279412in}{2.004545in}}%
\pgfusepath{clip}%
\pgfsetbuttcap%
\pgfsetroundjoin%
\pgfsetlinewidth{1.830566pt}%
\definecolor{currentstroke}{rgb}{0.657642,0.860219,0.203082}%
\pgfsetstrokecolor{currentstroke}%
\pgfsetdash{}{0pt}%
\pgfpathmoveto{\pgfqpoint{7.855531in}{5.267323in}}%
\pgfpathlineto{\pgfqpoint{7.806511in}{5.258045in}}%
\pgfusepath{stroke}%
\end{pgfscope}%
\begin{pgfscope}%
\pgfpathrectangle{\pgfqpoint{6.720588in}{4.155455in}}{\pgfqpoint{2.279412in}{2.004545in}}%
\pgfusepath{clip}%
\pgfsetbuttcap%
\pgfsetroundjoin%
\pgfsetlinewidth{1.993114pt}%
\definecolor{currentstroke}{rgb}{0.906311,0.894855,0.098125}%
\pgfsetstrokecolor{currentstroke}%
\pgfsetdash{}{0pt}%
\pgfpathmoveto{\pgfqpoint{7.806511in}{5.258045in}}%
\pgfpathlineto{\pgfqpoint{7.757710in}{5.247941in}}%
\pgfusepath{stroke}%
\end{pgfscope}%
\begin{pgfscope}%
\pgfpathrectangle{\pgfqpoint{6.720588in}{4.155455in}}{\pgfqpoint{2.279412in}{2.004545in}}%
\pgfusepath{clip}%
\pgfsetbuttcap%
\pgfsetroundjoin%
\pgfsetlinewidth{1.880991pt}%
\definecolor{currentstroke}{rgb}{0.741388,0.873449,0.149561}%
\pgfsetstrokecolor{currentstroke}%
\pgfsetdash{}{0pt}%
\pgfpathmoveto{\pgfqpoint{7.757710in}{5.247941in}}%
\pgfpathlineto{\pgfqpoint{7.709063in}{5.237275in}}%
\pgfusepath{stroke}%
\end{pgfscope}%
\begin{pgfscope}%
\pgfpathrectangle{\pgfqpoint{6.720588in}{4.155455in}}{\pgfqpoint{2.279412in}{2.004545in}}%
\pgfusepath{clip}%
\pgfsetbuttcap%
\pgfsetroundjoin%
\pgfsetlinewidth{2.080433pt}%
\definecolor{currentstroke}{rgb}{0.993248,0.906157,0.143936}%
\pgfsetstrokecolor{currentstroke}%
\pgfsetdash{}{0pt}%
\pgfpathmoveto{\pgfqpoint{7.709063in}{5.237275in}}%
\pgfpathlineto{\pgfqpoint{7.660537in}{5.226195in}}%
\pgfusepath{stroke}%
\end{pgfscope}%
\begin{pgfscope}%
\pgfpathrectangle{\pgfqpoint{6.720588in}{4.155455in}}{\pgfqpoint{2.279412in}{2.004545in}}%
\pgfusepath{clip}%
\pgfsetroundcap%
\pgfsetroundjoin%
\pgfsetlinewidth{1.130353pt}%
\definecolor{currentstroke}{rgb}{0.137770,0.537492,0.554906}%
\pgfsetstrokecolor{currentstroke}%
\pgfsetdash{}{0pt}%
\pgfpathmoveto{\pgfqpoint{8.033161in}{5.008423in}}%
\pgfpathquadraticcurveto{\pgfqpoint{8.020734in}{5.009865in}}{\pgfqpoint{8.025676in}{5.009292in}}%
\pgfusepath{stroke}%
\end{pgfscope}%
\begin{pgfscope}%
\pgfpathrectangle{\pgfqpoint{6.720588in}{4.155455in}}{\pgfqpoint{2.279412in}{2.004545in}}%
\pgfusepath{clip}%
\pgfsetroundcap%
\pgfsetroundjoin%
\definecolor{currentfill}{rgb}{0.137770,0.537492,0.554906}%
\pgfsetfillcolor{currentfill}%
\pgfsetlinewidth{1.130353pt}%
\definecolor{currentstroke}{rgb}{0.137770,0.537492,0.554906}%
\pgfsetstrokecolor{currentstroke}%
\pgfsetdash{}{0pt}%
\pgfpathmoveto{\pgfqpoint{8.077660in}{4.975296in}}%
\pgfpathlineto{\pgfqpoint{8.025676in}{5.009292in}}%
\pgfpathlineto{\pgfqpoint{8.084063in}{5.030482in}}%
\pgfpathlineto{\pgfqpoint{8.077660in}{4.975296in}}%
\pgfpathlineto{\pgfqpoint{8.077660in}{4.975296in}}%
\pgfpathclose%
\pgfusepath{stroke,fill}%
\end{pgfscope}%
\begin{pgfscope}%
\pgfpathrectangle{\pgfqpoint{6.720588in}{4.155455in}}{\pgfqpoint{2.279412in}{2.004545in}}%
\pgfusepath{clip}%
\pgfsetroundcap%
\pgfsetroundjoin%
\pgfsetlinewidth{1.304331pt}%
\definecolor{currentstroke}{rgb}{0.122312,0.633153,0.530398}%
\pgfsetstrokecolor{currentstroke}%
\pgfsetdash{}{0pt}%
\pgfpathmoveto{\pgfqpoint{8.080619in}{5.237486in}}%
\pgfpathquadraticcurveto{\pgfqpoint{8.068109in}{5.236760in}}{\pgfqpoint{8.075743in}{5.237203in}}%
\pgfusepath{stroke}%
\end{pgfscope}%
\begin{pgfscope}%
\pgfpathrectangle{\pgfqpoint{6.720588in}{4.155455in}}{\pgfqpoint{2.279412in}{2.004545in}}%
\pgfusepath{clip}%
\pgfsetroundcap%
\pgfsetroundjoin%
\definecolor{currentfill}{rgb}{0.122312,0.633153,0.530398}%
\pgfsetfillcolor{currentfill}%
\pgfsetlinewidth{1.304331pt}%
\definecolor{currentstroke}{rgb}{0.122312,0.633153,0.530398}%
\pgfsetstrokecolor{currentstroke}%
\pgfsetdash{}{0pt}%
\pgfpathmoveto{\pgfqpoint{8.132815in}{5.212691in}}%
\pgfpathlineto{\pgfqpoint{8.075743in}{5.237203in}}%
\pgfpathlineto{\pgfqpoint{8.129596in}{5.268153in}}%
\pgfpathlineto{\pgfqpoint{8.132815in}{5.212691in}}%
\pgfpathlineto{\pgfqpoint{8.132815in}{5.212691in}}%
\pgfpathclose%
\pgfusepath{stroke,fill}%
\end{pgfscope}%
\begin{pgfscope}%
\pgfpathrectangle{\pgfqpoint{6.720588in}{4.155455in}}{\pgfqpoint{2.279412in}{2.004545in}}%
\pgfusepath{clip}%
\pgfsetroundcap%
\pgfsetroundjoin%
\pgfsetlinewidth{0.423890pt}%
\definecolor{currentstroke}{rgb}{0.282656,0.100196,0.422160}%
\pgfsetstrokecolor{currentstroke}%
\pgfsetdash{}{0pt}%
\pgfpathmoveto{\pgfqpoint{8.381407in}{5.289785in}}%
\pgfpathquadraticcurveto{\pgfqpoint{8.368877in}{5.289402in}}{\pgfqpoint{8.362902in}{5.289219in}}%
\pgfusepath{stroke}%
\end{pgfscope}%
\begin{pgfscope}%
\pgfpathrectangle{\pgfqpoint{6.720588in}{4.155455in}}{\pgfqpoint{2.279412in}{2.004545in}}%
\pgfusepath{clip}%
\pgfsetroundcap%
\pgfsetroundjoin%
\definecolor{currentfill}{rgb}{0.282656,0.100196,0.422160}%
\pgfsetfillcolor{currentfill}%
\pgfsetlinewidth{0.423890pt}%
\definecolor{currentstroke}{rgb}{0.282656,0.100196,0.422160}%
\pgfsetstrokecolor{currentstroke}%
\pgfsetdash{}{0pt}%
\pgfpathmoveto{\pgfqpoint{8.419281in}{5.263153in}}%
\pgfpathlineto{\pgfqpoint{8.362902in}{5.289219in}}%
\pgfpathlineto{\pgfqpoint{8.417582in}{5.318683in}}%
\pgfpathlineto{\pgfqpoint{8.419281in}{5.263153in}}%
\pgfpathlineto{\pgfqpoint{8.419281in}{5.263153in}}%
\pgfpathclose%
\pgfusepath{stroke,fill}%
\end{pgfscope}%
\begin{pgfscope}%
\pgfpathrectangle{\pgfqpoint{6.720588in}{4.155455in}}{\pgfqpoint{2.279412in}{2.004545in}}%
\pgfusepath{clip}%
\pgfsetroundcap%
\pgfsetroundjoin%
\pgfsetlinewidth{0.505737pt}%
\definecolor{currentstroke}{rgb}{0.280868,0.160771,0.472899}%
\pgfsetstrokecolor{currentstroke}%
\pgfsetdash{}{0pt}%
\pgfpathmoveto{\pgfqpoint{8.281679in}{4.945505in}}%
\pgfpathquadraticcurveto{\pgfqpoint{8.269163in}{4.946148in}}{\pgfqpoint{8.264460in}{4.946389in}}%
\pgfusepath{stroke}%
\end{pgfscope}%
\begin{pgfscope}%
\pgfpathrectangle{\pgfqpoint{6.720588in}{4.155455in}}{\pgfqpoint{2.279412in}{2.004545in}}%
\pgfusepath{clip}%
\pgfsetroundcap%
\pgfsetroundjoin%
\definecolor{currentfill}{rgb}{0.280868,0.160771,0.472899}%
\pgfsetfillcolor{currentfill}%
\pgfsetlinewidth{0.505737pt}%
\definecolor{currentstroke}{rgb}{0.280868,0.160771,0.472899}%
\pgfsetstrokecolor{currentstroke}%
\pgfsetdash{}{0pt}%
\pgfpathmoveto{\pgfqpoint{8.318518in}{4.915799in}}%
\pgfpathlineto{\pgfqpoint{8.264460in}{4.946389in}}%
\pgfpathlineto{\pgfqpoint{8.321367in}{4.971282in}}%
\pgfpathlineto{\pgfqpoint{8.318518in}{4.915799in}}%
\pgfpathlineto{\pgfqpoint{8.318518in}{4.915799in}}%
\pgfpathclose%
\pgfusepath{stroke,fill}%
\end{pgfscope}%
\begin{pgfscope}%
\pgfpathrectangle{\pgfqpoint{6.720588in}{4.155455in}}{\pgfqpoint{2.279412in}{2.004545in}}%
\pgfusepath{clip}%
\pgfsetroundcap%
\pgfsetroundjoin%
\pgfsetlinewidth{0.402634pt}%
\definecolor{currentstroke}{rgb}{0.281446,0.084320,0.407414}%
\pgfsetstrokecolor{currentstroke}%
\pgfsetdash{}{0pt}%
\pgfpathmoveto{\pgfqpoint{8.430331in}{5.025995in}}%
\pgfpathquadraticcurveto{\pgfqpoint{8.417796in}{5.026219in}}{\pgfqpoint{8.411489in}{5.026331in}}%
\pgfusepath{stroke}%
\end{pgfscope}%
\begin{pgfscope}%
\pgfpathrectangle{\pgfqpoint{6.720588in}{4.155455in}}{\pgfqpoint{2.279412in}{2.004545in}}%
\pgfusepath{clip}%
\pgfsetroundcap%
\pgfsetroundjoin%
\definecolor{currentfill}{rgb}{0.281446,0.084320,0.407414}%
\pgfsetfillcolor{currentfill}%
\pgfsetlinewidth{0.402634pt}%
\definecolor{currentstroke}{rgb}{0.281446,0.084320,0.407414}%
\pgfsetstrokecolor{currentstroke}%
\pgfsetdash{}{0pt}%
\pgfpathmoveto{\pgfqpoint{8.466540in}{4.997566in}}%
\pgfpathlineto{\pgfqpoint{8.411489in}{5.026331in}}%
\pgfpathlineto{\pgfqpoint{8.467532in}{5.053113in}}%
\pgfpathlineto{\pgfqpoint{8.466540in}{4.997566in}}%
\pgfpathlineto{\pgfqpoint{8.466540in}{4.997566in}}%
\pgfpathclose%
\pgfusepath{stroke,fill}%
\end{pgfscope}%
\begin{pgfscope}%
\pgfpathrectangle{\pgfqpoint{6.720588in}{4.155455in}}{\pgfqpoint{2.279412in}{2.004545in}}%
\pgfusepath{clip}%
\pgfsetroundcap%
\pgfsetroundjoin%
\pgfsetlinewidth{0.896089pt}%
\definecolor{currentstroke}{rgb}{0.188923,0.410910,0.556326}%
\pgfsetstrokecolor{currentstroke}%
\pgfsetdash{}{0pt}%
\pgfpathmoveto{\pgfqpoint{8.180660in}{5.076971in}}%
\pgfpathquadraticcurveto{\pgfqpoint{8.168128in}{5.077304in}}{\pgfqpoint{8.169454in}{5.077269in}}%
\pgfusepath{stroke}%
\end{pgfscope}%
\begin{pgfscope}%
\pgfpathrectangle{\pgfqpoint{6.720588in}{4.155455in}}{\pgfqpoint{2.279412in}{2.004545in}}%
\pgfusepath{clip}%
\pgfsetroundcap%
\pgfsetroundjoin%
\definecolor{currentfill}{rgb}{0.188923,0.410910,0.556326}%
\pgfsetfillcolor{currentfill}%
\pgfsetlinewidth{0.896089pt}%
\definecolor{currentstroke}{rgb}{0.188923,0.410910,0.556326}%
\pgfsetstrokecolor{currentstroke}%
\pgfsetdash{}{0pt}%
\pgfpathmoveto{\pgfqpoint{8.224254in}{5.048028in}}%
\pgfpathlineto{\pgfqpoint{8.169454in}{5.077269in}}%
\pgfpathlineto{\pgfqpoint{8.225726in}{5.103564in}}%
\pgfpathlineto{\pgfqpoint{8.224254in}{5.048028in}}%
\pgfpathlineto{\pgfqpoint{8.224254in}{5.048028in}}%
\pgfpathclose%
\pgfusepath{stroke,fill}%
\end{pgfscope}%
\begin{pgfscope}%
\pgfpathrectangle{\pgfqpoint{6.720588in}{4.155455in}}{\pgfqpoint{2.279412in}{2.004545in}}%
\pgfusepath{clip}%
\pgfsetroundcap%
\pgfsetroundjoin%
\pgfsetlinewidth{0.963546pt}%
\definecolor{currentstroke}{rgb}{0.172719,0.448791,0.557885}%
\pgfsetstrokecolor{currentstroke}%
\pgfsetdash{}{0pt}%
\pgfpathmoveto{\pgfqpoint{8.179511in}{5.159322in}}%
\pgfpathquadraticcurveto{\pgfqpoint{8.166973in}{5.159303in}}{\pgfqpoint{8.169342in}{5.159307in}}%
\pgfusepath{stroke}%
\end{pgfscope}%
\begin{pgfscope}%
\pgfpathrectangle{\pgfqpoint{6.720588in}{4.155455in}}{\pgfqpoint{2.279412in}{2.004545in}}%
\pgfusepath{clip}%
\pgfsetroundcap%
\pgfsetroundjoin%
\definecolor{currentfill}{rgb}{0.172719,0.448791,0.557885}%
\pgfsetfillcolor{currentfill}%
\pgfsetlinewidth{0.963546pt}%
\definecolor{currentstroke}{rgb}{0.172719,0.448791,0.557885}%
\pgfsetstrokecolor{currentstroke}%
\pgfsetdash{}{0pt}%
\pgfpathmoveto{\pgfqpoint{8.224938in}{5.131611in}}%
\pgfpathlineto{\pgfqpoint{8.169342in}{5.159307in}}%
\pgfpathlineto{\pgfqpoint{8.224856in}{5.187167in}}%
\pgfpathlineto{\pgfqpoint{8.224938in}{5.131611in}}%
\pgfpathlineto{\pgfqpoint{8.224938in}{5.131611in}}%
\pgfpathclose%
\pgfusepath{stroke,fill}%
\end{pgfscope}%
\begin{pgfscope}%
\pgfpathrectangle{\pgfqpoint{6.720588in}{4.155455in}}{\pgfqpoint{2.279412in}{2.004545in}}%
\pgfusepath{clip}%
\pgfsetroundcap%
\pgfsetroundjoin%
\pgfsetlinewidth{0.509566pt}%
\definecolor{currentstroke}{rgb}{0.280255,0.165693,0.476498}%
\pgfsetstrokecolor{currentstroke}%
\pgfsetdash{}{0pt}%
\pgfpathmoveto{\pgfqpoint{8.330012in}{5.201088in}}%
\pgfpathquadraticcurveto{\pgfqpoint{8.317475in}{5.200974in}}{\pgfqpoint{8.312820in}{5.200932in}}%
\pgfusepath{stroke}%
\end{pgfscope}%
\begin{pgfscope}%
\pgfpathrectangle{\pgfqpoint{6.720588in}{4.155455in}}{\pgfqpoint{2.279412in}{2.004545in}}%
\pgfusepath{clip}%
\pgfsetroundcap%
\pgfsetroundjoin%
\definecolor{currentfill}{rgb}{0.280255,0.165693,0.476498}%
\pgfsetfillcolor{currentfill}%
\pgfsetlinewidth{0.509566pt}%
\definecolor{currentstroke}{rgb}{0.280255,0.165693,0.476498}%
\pgfsetstrokecolor{currentstroke}%
\pgfsetdash{}{0pt}%
\pgfpathmoveto{\pgfqpoint{8.368627in}{5.173661in}}%
\pgfpathlineto{\pgfqpoint{8.312820in}{5.200932in}}%
\pgfpathlineto{\pgfqpoint{8.368121in}{5.229215in}}%
\pgfpathlineto{\pgfqpoint{8.368627in}{5.173661in}}%
\pgfpathlineto{\pgfqpoint{8.368627in}{5.173661in}}%
\pgfpathclose%
\pgfusepath{stroke,fill}%
\end{pgfscope}%
\begin{pgfscope}%
\pgfpathrectangle{\pgfqpoint{6.720588in}{4.155455in}}{\pgfqpoint{2.279412in}{2.004545in}}%
\pgfusepath{clip}%
\pgfsetroundcap%
\pgfsetroundjoin%
\pgfsetlinewidth{0.422772pt}%
\definecolor{currentstroke}{rgb}{0.282656,0.100196,0.422160}%
\pgfsetstrokecolor{currentstroke}%
\pgfsetdash{}{0pt}%
\pgfpathmoveto{\pgfqpoint{8.380352in}{5.334174in}}%
\pgfpathquadraticcurveto{\pgfqpoint{8.367821in}{5.333828in}}{\pgfqpoint{8.361827in}{5.333662in}}%
\pgfusepath{stroke}%
\end{pgfscope}%
\begin{pgfscope}%
\pgfpathrectangle{\pgfqpoint{6.720588in}{4.155455in}}{\pgfqpoint{2.279412in}{2.004545in}}%
\pgfusepath{clip}%
\pgfsetroundcap%
\pgfsetroundjoin%
\definecolor{currentfill}{rgb}{0.282656,0.100196,0.422160}%
\pgfsetfillcolor{currentfill}%
\pgfsetlinewidth{0.422772pt}%
\definecolor{currentstroke}{rgb}{0.282656,0.100196,0.422160}%
\pgfsetstrokecolor{currentstroke}%
\pgfsetdash{}{0pt}%
\pgfpathmoveto{\pgfqpoint{8.418128in}{5.307429in}}%
\pgfpathlineto{\pgfqpoint{8.361827in}{5.333662in}}%
\pgfpathlineto{\pgfqpoint{8.416594in}{5.362963in}}%
\pgfpathlineto{\pgfqpoint{8.418128in}{5.307429in}}%
\pgfpathlineto{\pgfqpoint{8.418128in}{5.307429in}}%
\pgfpathclose%
\pgfusepath{stroke,fill}%
\end{pgfscope}%
\begin{pgfscope}%
\pgfpathrectangle{\pgfqpoint{6.720588in}{4.155455in}}{\pgfqpoint{2.279412in}{2.004545in}}%
\pgfusepath{clip}%
\pgfsetroundcap%
\pgfsetroundjoin%
\pgfsetlinewidth{0.548378pt}%
\definecolor{currentstroke}{rgb}{0.275191,0.194905,0.496005}%
\pgfsetstrokecolor{currentstroke}%
\pgfsetdash{}{0pt}%
\pgfpathmoveto{\pgfqpoint{7.990513in}{4.845819in}}%
\pgfpathquadraticcurveto{\pgfqpoint{7.978387in}{4.848606in}}{\pgfqpoint{7.974528in}{4.849492in}}%
\pgfusepath{stroke}%
\end{pgfscope}%
\begin{pgfscope}%
\pgfpathrectangle{\pgfqpoint{6.720588in}{4.155455in}}{\pgfqpoint{2.279412in}{2.004545in}}%
\pgfusepath{clip}%
\pgfsetroundcap%
\pgfsetroundjoin%
\definecolor{currentfill}{rgb}{0.275191,0.194905,0.496005}%
\pgfsetfillcolor{currentfill}%
\pgfsetlinewidth{0.548378pt}%
\definecolor{currentstroke}{rgb}{0.275191,0.194905,0.496005}%
\pgfsetstrokecolor{currentstroke}%
\pgfsetdash{}{0pt}%
\pgfpathmoveto{\pgfqpoint{8.022452in}{4.809979in}}%
\pgfpathlineto{\pgfqpoint{7.974528in}{4.849492in}}%
\pgfpathlineto{\pgfqpoint{8.034894in}{4.864123in}}%
\pgfpathlineto{\pgfqpoint{8.022452in}{4.809979in}}%
\pgfpathlineto{\pgfqpoint{8.022452in}{4.809979in}}%
\pgfpathclose%
\pgfusepath{stroke,fill}%
\end{pgfscope}%
\begin{pgfscope}%
\pgfpathrectangle{\pgfqpoint{6.720588in}{4.155455in}}{\pgfqpoint{2.279412in}{2.004545in}}%
\pgfusepath{clip}%
\pgfsetroundcap%
\pgfsetroundjoin%
\pgfsetlinewidth{0.593142pt}%
\definecolor{currentstroke}{rgb}{0.267968,0.223549,0.512008}%
\pgfsetstrokecolor{currentstroke}%
\pgfsetdash{}{0pt}%
\pgfpathmoveto{\pgfqpoint{8.279130in}{5.117648in}}%
\pgfpathquadraticcurveto{\pgfqpoint{8.266592in}{5.117705in}}{\pgfqpoint{8.263231in}{5.117720in}}%
\pgfusepath{stroke}%
\end{pgfscope}%
\begin{pgfscope}%
\pgfpathrectangle{\pgfqpoint{6.720588in}{4.155455in}}{\pgfqpoint{2.279412in}{2.004545in}}%
\pgfusepath{clip}%
\pgfsetroundcap%
\pgfsetroundjoin%
\definecolor{currentfill}{rgb}{0.267968,0.223549,0.512008}%
\pgfsetfillcolor{currentfill}%
\pgfsetlinewidth{0.593142pt}%
\definecolor{currentstroke}{rgb}{0.267968,0.223549,0.512008}%
\pgfsetstrokecolor{currentstroke}%
\pgfsetdash{}{0pt}%
\pgfpathmoveto{\pgfqpoint{8.318658in}{5.089688in}}%
\pgfpathlineto{\pgfqpoint{8.263231in}{5.117720in}}%
\pgfpathlineto{\pgfqpoint{8.318913in}{5.145243in}}%
\pgfpathlineto{\pgfqpoint{8.318658in}{5.089688in}}%
\pgfpathlineto{\pgfqpoint{8.318658in}{5.089688in}}%
\pgfpathclose%
\pgfusepath{stroke,fill}%
\end{pgfscope}%
\begin{pgfscope}%
\pgfpathrectangle{\pgfqpoint{6.720588in}{4.155455in}}{\pgfqpoint{2.279412in}{2.004545in}}%
\pgfusepath{clip}%
\pgfsetroundcap%
\pgfsetroundjoin%
\pgfsetlinewidth{0.804295pt}%
\definecolor{currentstroke}{rgb}{0.212395,0.359683,0.551710}%
\pgfsetstrokecolor{currentstroke}%
\pgfsetdash{}{0pt}%
\pgfpathmoveto{\pgfqpoint{7.931481in}{5.375649in}}%
\pgfpathquadraticcurveto{\pgfqpoint{7.919419in}{5.372656in}}{\pgfqpoint{7.919434in}{5.372660in}}%
\pgfusepath{stroke}%
\end{pgfscope}%
\begin{pgfscope}%
\pgfpathrectangle{\pgfqpoint{6.720588in}{4.155455in}}{\pgfqpoint{2.279412in}{2.004545in}}%
\pgfusepath{clip}%
\pgfsetroundcap%
\pgfsetroundjoin%
\definecolor{currentfill}{rgb}{0.212395,0.359683,0.551710}%
\pgfsetfillcolor{currentfill}%
\pgfsetlinewidth{0.804295pt}%
\definecolor{currentstroke}{rgb}{0.212395,0.359683,0.551710}%
\pgfsetstrokecolor{currentstroke}%
\pgfsetdash{}{0pt}%
\pgfpathmoveto{\pgfqpoint{7.980044in}{5.359081in}}%
\pgfpathlineto{\pgfqpoint{7.919434in}{5.372660in}}%
\pgfpathlineto{\pgfqpoint{7.966663in}{5.413001in}}%
\pgfpathlineto{\pgfqpoint{7.980044in}{5.359081in}}%
\pgfpathlineto{\pgfqpoint{7.980044in}{5.359081in}}%
\pgfpathclose%
\pgfusepath{stroke,fill}%
\end{pgfscope}%
\begin{pgfscope}%
\pgfpathrectangle{\pgfqpoint{6.720588in}{4.155455in}}{\pgfqpoint{2.279412in}{2.004545in}}%
\pgfusepath{clip}%
\pgfsetroundcap%
\pgfsetroundjoin%
\pgfsetlinewidth{0.420780pt}%
\definecolor{currentstroke}{rgb}{0.282656,0.100196,0.422160}%
\pgfsetstrokecolor{currentstroke}%
\pgfsetdash{}{0pt}%
\pgfpathmoveto{\pgfqpoint{8.178750in}{4.732204in}}%
\pgfpathquadraticcurveto{\pgfqpoint{8.166372in}{4.733939in}}{\pgfqpoint{8.160441in}{4.734771in}}%
\pgfusepath{stroke}%
\end{pgfscope}%
\begin{pgfscope}%
\pgfpathrectangle{\pgfqpoint{6.720588in}{4.155455in}}{\pgfqpoint{2.279412in}{2.004545in}}%
\pgfusepath{clip}%
\pgfsetroundcap%
\pgfsetroundjoin%
\definecolor{currentfill}{rgb}{0.282656,0.100196,0.422160}%
\pgfsetfillcolor{currentfill}%
\pgfsetlinewidth{0.420780pt}%
\definecolor{currentstroke}{rgb}{0.282656,0.100196,0.422160}%
\pgfsetstrokecolor{currentstroke}%
\pgfsetdash{}{0pt}%
\pgfpathmoveto{\pgfqpoint{8.211600in}{4.699547in}}%
\pgfpathlineto{\pgfqpoint{8.160441in}{4.734771in}}%
\pgfpathlineto{\pgfqpoint{8.219316in}{4.754565in}}%
\pgfpathlineto{\pgfqpoint{8.211600in}{4.699547in}}%
\pgfpathlineto{\pgfqpoint{8.211600in}{4.699547in}}%
\pgfpathclose%
\pgfusepath{stroke,fill}%
\end{pgfscope}%
\begin{pgfscope}%
\pgfpathrectangle{\pgfqpoint{6.720588in}{4.155455in}}{\pgfqpoint{2.279412in}{2.004545in}}%
\pgfusepath{clip}%
\pgfsetroundcap%
\pgfsetroundjoin%
\pgfsetlinewidth{0.568344pt}%
\definecolor{currentstroke}{rgb}{0.273006,0.204520,0.501721}%
\pgfsetstrokecolor{currentstroke}%
\pgfsetdash{}{0pt}%
\pgfpathmoveto{\pgfqpoint{8.127879in}{4.909968in}}%
\pgfpathquadraticcurveto{\pgfqpoint{8.115431in}{4.911278in}}{\pgfqpoint{8.111728in}{4.911667in}}%
\pgfusepath{stroke}%
\end{pgfscope}%
\begin{pgfscope}%
\pgfpathrectangle{\pgfqpoint{6.720588in}{4.155455in}}{\pgfqpoint{2.279412in}{2.004545in}}%
\pgfusepath{clip}%
\pgfsetroundcap%
\pgfsetroundjoin%
\definecolor{currentfill}{rgb}{0.273006,0.204520,0.501721}%
\pgfsetfillcolor{currentfill}%
\pgfsetlinewidth{0.568344pt}%
\definecolor{currentstroke}{rgb}{0.273006,0.204520,0.501721}%
\pgfsetstrokecolor{currentstroke}%
\pgfsetdash{}{0pt}%
\pgfpathmoveto{\pgfqpoint{8.164072in}{4.878229in}}%
\pgfpathlineto{\pgfqpoint{8.111728in}{4.911667in}}%
\pgfpathlineto{\pgfqpoint{8.169885in}{4.933480in}}%
\pgfpathlineto{\pgfqpoint{8.164072in}{4.878229in}}%
\pgfpathlineto{\pgfqpoint{8.164072in}{4.878229in}}%
\pgfpathclose%
\pgfusepath{stroke,fill}%
\end{pgfscope}%
\begin{pgfscope}%
\pgfpathrectangle{\pgfqpoint{6.720588in}{4.155455in}}{\pgfqpoint{2.279412in}{2.004545in}}%
\pgfusepath{clip}%
\pgfsetroundcap%
\pgfsetroundjoin%
\pgfsetlinewidth{0.410694pt}%
\definecolor{currentstroke}{rgb}{0.281924,0.089666,0.412415}%
\pgfsetstrokecolor{currentstroke}%
\pgfsetdash{}{0pt}%
\pgfpathmoveto{\pgfqpoint{8.377935in}{5.376878in}}%
\pgfpathquadraticcurveto{\pgfqpoint{8.365411in}{5.376356in}}{\pgfqpoint{8.359235in}{5.376098in}}%
\pgfusepath{stroke}%
\end{pgfscope}%
\begin{pgfscope}%
\pgfpathrectangle{\pgfqpoint{6.720588in}{4.155455in}}{\pgfqpoint{2.279412in}{2.004545in}}%
\pgfusepath{clip}%
\pgfsetroundcap%
\pgfsetroundjoin%
\definecolor{currentfill}{rgb}{0.281924,0.089666,0.412415}%
\pgfsetfillcolor{currentfill}%
\pgfsetlinewidth{0.410694pt}%
\definecolor{currentstroke}{rgb}{0.281924,0.089666,0.412415}%
\pgfsetstrokecolor{currentstroke}%
\pgfsetdash{}{0pt}%
\pgfpathmoveto{\pgfqpoint{8.415900in}{5.350659in}}%
\pgfpathlineto{\pgfqpoint{8.359235in}{5.376098in}}%
\pgfpathlineto{\pgfqpoint{8.413586in}{5.406166in}}%
\pgfpathlineto{\pgfqpoint{8.415900in}{5.350659in}}%
\pgfpathlineto{\pgfqpoint{8.415900in}{5.350659in}}%
\pgfpathclose%
\pgfusepath{stroke,fill}%
\end{pgfscope}%
\begin{pgfscope}%
\pgfpathrectangle{\pgfqpoint{6.720588in}{4.155455in}}{\pgfqpoint{2.279412in}{2.004545in}}%
\pgfusepath{clip}%
\pgfsetroundcap%
\pgfsetroundjoin%
\pgfsetlinewidth{0.363470pt}%
\definecolor{currentstroke}{rgb}{0.277941,0.056324,0.381191}%
\pgfsetstrokecolor{currentstroke}%
\pgfsetdash{}{0pt}%
\pgfpathmoveto{\pgfqpoint{8.428111in}{5.467783in}}%
\pgfpathquadraticcurveto{\pgfqpoint{8.415597in}{5.467139in}}{\pgfqpoint{8.408697in}{5.466784in}}%
\pgfusepath{stroke}%
\end{pgfscope}%
\begin{pgfscope}%
\pgfpathrectangle{\pgfqpoint{6.720588in}{4.155455in}}{\pgfqpoint{2.279412in}{2.004545in}}%
\pgfusepath{clip}%
\pgfsetroundcap%
\pgfsetroundjoin%
\definecolor{currentfill}{rgb}{0.277941,0.056324,0.381191}%
\pgfsetfillcolor{currentfill}%
\pgfsetlinewidth{0.363470pt}%
\definecolor{currentstroke}{rgb}{0.277941,0.056324,0.381191}%
\pgfsetstrokecolor{currentstroke}%
\pgfsetdash{}{0pt}%
\pgfpathmoveto{\pgfqpoint{8.465606in}{5.441896in}}%
\pgfpathlineto{\pgfqpoint{8.408697in}{5.466784in}}%
\pgfpathlineto{\pgfqpoint{8.462753in}{5.497379in}}%
\pgfpathlineto{\pgfqpoint{8.465606in}{5.441896in}}%
\pgfpathlineto{\pgfqpoint{8.465606in}{5.441896in}}%
\pgfpathclose%
\pgfusepath{stroke,fill}%
\end{pgfscope}%
\begin{pgfscope}%
\pgfpathrectangle{\pgfqpoint{6.720588in}{4.155455in}}{\pgfqpoint{2.279412in}{2.004545in}}%
\pgfusepath{clip}%
\pgfsetroundcap%
\pgfsetroundjoin%
\pgfsetlinewidth{0.435126pt}%
\definecolor{currentstroke}{rgb}{0.283091,0.110553,0.431554}%
\pgfsetstrokecolor{currentstroke}%
\pgfsetdash{}{0pt}%
\pgfpathmoveto{\pgfqpoint{8.228443in}{5.498660in}}%
\pgfpathquadraticcurveto{\pgfqpoint{8.215994in}{5.497351in}}{\pgfqpoint{8.210240in}{5.496746in}}%
\pgfusepath{stroke}%
\end{pgfscope}%
\begin{pgfscope}%
\pgfpathrectangle{\pgfqpoint{6.720588in}{4.155455in}}{\pgfqpoint{2.279412in}{2.004545in}}%
\pgfusepath{clip}%
\pgfsetroundcap%
\pgfsetroundjoin%
\definecolor{currentfill}{rgb}{0.283091,0.110553,0.431554}%
\pgfsetfillcolor{currentfill}%
\pgfsetlinewidth{0.435126pt}%
\definecolor{currentstroke}{rgb}{0.283091,0.110553,0.431554}%
\pgfsetstrokecolor{currentstroke}%
\pgfsetdash{}{0pt}%
\pgfpathmoveto{\pgfqpoint{8.268395in}{5.474929in}}%
\pgfpathlineto{\pgfqpoint{8.210240in}{5.496746in}}%
\pgfpathlineto{\pgfqpoint{8.262587in}{5.530180in}}%
\pgfpathlineto{\pgfqpoint{8.268395in}{5.474929in}}%
\pgfpathlineto{\pgfqpoint{8.268395in}{5.474929in}}%
\pgfpathclose%
\pgfusepath{stroke,fill}%
\end{pgfscope}%
\begin{pgfscope}%
\pgfpathrectangle{\pgfqpoint{6.720588in}{4.155455in}}{\pgfqpoint{2.279412in}{2.004545in}}%
\pgfusepath{clip}%
\pgfsetroundcap%
\pgfsetroundjoin%
\pgfsetlinewidth{0.386692pt}%
\definecolor{currentstroke}{rgb}{0.280267,0.073417,0.397163}%
\pgfsetstrokecolor{currentstroke}%
\pgfsetdash{}{0pt}%
\pgfpathmoveto{\pgfqpoint{8.278158in}{5.547741in}}%
\pgfpathquadraticcurveto{\pgfqpoint{8.265683in}{5.546645in}}{\pgfqpoint{8.259167in}{5.546073in}}%
\pgfusepath{stroke}%
\end{pgfscope}%
\begin{pgfscope}%
\pgfpathrectangle{\pgfqpoint{6.720588in}{4.155455in}}{\pgfqpoint{2.279412in}{2.004545in}}%
\pgfusepath{clip}%
\pgfsetroundcap%
\pgfsetroundjoin%
\definecolor{currentfill}{rgb}{0.280267,0.073417,0.397163}%
\pgfsetfillcolor{currentfill}%
\pgfsetlinewidth{0.386692pt}%
\definecolor{currentstroke}{rgb}{0.280267,0.073417,0.397163}%
\pgfsetstrokecolor{currentstroke}%
\pgfsetdash{}{0pt}%
\pgfpathmoveto{\pgfqpoint{8.316940in}{5.523262in}}%
\pgfpathlineto{\pgfqpoint{8.259167in}{5.546073in}}%
\pgfpathlineto{\pgfqpoint{8.312080in}{5.578604in}}%
\pgfpathlineto{\pgfqpoint{8.316940in}{5.523262in}}%
\pgfpathlineto{\pgfqpoint{8.316940in}{5.523262in}}%
\pgfpathclose%
\pgfusepath{stroke,fill}%
\end{pgfscope}%
\begin{pgfscope}%
\pgfpathrectangle{\pgfqpoint{6.720588in}{4.155455in}}{\pgfqpoint{2.279412in}{2.004545in}}%
\pgfusepath{clip}%
\pgfsetroundcap%
\pgfsetroundjoin%
\pgfsetlinewidth{0.331855pt}%
\definecolor{currentstroke}{rgb}{0.272594,0.025563,0.353093}%
\pgfsetstrokecolor{currentstroke}%
\pgfsetdash{}{0pt}%
\pgfpathmoveto{\pgfqpoint{8.215742in}{5.802216in}}%
\pgfpathquadraticcurveto{\pgfqpoint{8.203335in}{5.800659in}}{\pgfqpoint{8.196021in}{5.799742in}}%
\pgfusepath{stroke}%
\end{pgfscope}%
\begin{pgfscope}%
\pgfpathrectangle{\pgfqpoint{6.720588in}{4.155455in}}{\pgfqpoint{2.279412in}{2.004545in}}%
\pgfusepath{clip}%
\pgfsetroundcap%
\pgfsetroundjoin%
\definecolor{currentfill}{rgb}{0.272594,0.025563,0.353093}%
\pgfsetfillcolor{currentfill}%
\pgfsetlinewidth{0.331855pt}%
\definecolor{currentstroke}{rgb}{0.272594,0.025563,0.353093}%
\pgfsetstrokecolor{currentstroke}%
\pgfsetdash{}{0pt}%
\pgfpathmoveto{\pgfqpoint{8.254603in}{5.779096in}}%
\pgfpathlineto{\pgfqpoint{8.196021in}{5.799742in}}%
\pgfpathlineto{\pgfqpoint{8.247687in}{5.834219in}}%
\pgfpathlineto{\pgfqpoint{8.254603in}{5.779096in}}%
\pgfpathlineto{\pgfqpoint{8.254603in}{5.779096in}}%
\pgfpathclose%
\pgfusepath{stroke,fill}%
\end{pgfscope}%
\begin{pgfscope}%
\pgfpathrectangle{\pgfqpoint{6.720588in}{4.155455in}}{\pgfqpoint{2.279412in}{2.004545in}}%
\pgfusepath{clip}%
\pgfsetroundcap%
\pgfsetroundjoin%
\pgfsetlinewidth{0.327050pt}%
\definecolor{currentstroke}{rgb}{0.271305,0.019942,0.347269}%
\pgfsetstrokecolor{currentstroke}%
\pgfsetdash{}{0pt}%
\pgfpathmoveto{\pgfqpoint{8.062403in}{4.559914in}}%
\pgfpathquadraticcurveto{\pgfqpoint{8.050345in}{4.562771in}}{\pgfqpoint{8.043210in}{4.564461in}}%
\pgfusepath{stroke}%
\end{pgfscope}%
\begin{pgfscope}%
\pgfpathrectangle{\pgfqpoint{6.720588in}{4.155455in}}{\pgfqpoint{2.279412in}{2.004545in}}%
\pgfusepath{clip}%
\pgfsetroundcap%
\pgfsetroundjoin%
\definecolor{currentfill}{rgb}{0.271305,0.019942,0.347269}%
\pgfsetfillcolor{currentfill}%
\pgfsetlinewidth{0.327050pt}%
\definecolor{currentstroke}{rgb}{0.271305,0.019942,0.347269}%
\pgfsetstrokecolor{currentstroke}%
\pgfsetdash{}{0pt}%
\pgfpathmoveto{\pgfqpoint{8.090867in}{4.524626in}}%
\pgfpathlineto{\pgfqpoint{8.043210in}{4.564461in}}%
\pgfpathlineto{\pgfqpoint{8.103672in}{4.578686in}}%
\pgfpathlineto{\pgfqpoint{8.090867in}{4.524626in}}%
\pgfpathlineto{\pgfqpoint{8.090867in}{4.524626in}}%
\pgfpathclose%
\pgfusepath{stroke,fill}%
\end{pgfscope}%
\begin{pgfscope}%
\pgfpathrectangle{\pgfqpoint{6.720588in}{4.155455in}}{\pgfqpoint{2.279412in}{2.004545in}}%
\pgfusepath{clip}%
\pgfsetroundcap%
\pgfsetroundjoin%
\pgfsetlinewidth{0.420497pt}%
\definecolor{currentstroke}{rgb}{0.282656,0.100196,0.422160}%
\pgfsetstrokecolor{currentstroke}%
\pgfsetdash{}{0pt}%
\pgfpathmoveto{\pgfqpoint{8.227586in}{4.771808in}}%
\pgfpathquadraticcurveto{\pgfqpoint{8.215138in}{4.773120in}}{\pgfqpoint{8.209159in}{4.773750in}}%
\pgfusepath{stroke}%
\end{pgfscope}%
\begin{pgfscope}%
\pgfpathrectangle{\pgfqpoint{6.720588in}{4.155455in}}{\pgfqpoint{2.279412in}{2.004545in}}%
\pgfusepath{clip}%
\pgfsetroundcap%
\pgfsetroundjoin%
\definecolor{currentfill}{rgb}{0.282656,0.100196,0.422160}%
\pgfsetfillcolor{currentfill}%
\pgfsetlinewidth{0.420497pt}%
\definecolor{currentstroke}{rgb}{0.282656,0.100196,0.422160}%
\pgfsetstrokecolor{currentstroke}%
\pgfsetdash{}{0pt}%
\pgfpathmoveto{\pgfqpoint{8.261498in}{4.740303in}}%
\pgfpathlineto{\pgfqpoint{8.209159in}{4.773750in}}%
\pgfpathlineto{\pgfqpoint{8.267320in}{4.795553in}}%
\pgfpathlineto{\pgfqpoint{8.261498in}{4.740303in}}%
\pgfpathlineto{\pgfqpoint{8.261498in}{4.740303in}}%
\pgfpathclose%
\pgfusepath{stroke,fill}%
\end{pgfscope}%
\begin{pgfscope}%
\pgfpathrectangle{\pgfqpoint{6.720588in}{4.155455in}}{\pgfqpoint{2.279412in}{2.004545in}}%
\pgfusepath{clip}%
\pgfsetroundcap%
\pgfsetroundjoin%
\pgfsetlinewidth{0.384605pt}%
\definecolor{currentstroke}{rgb}{0.280267,0.073417,0.397163}%
\pgfsetstrokecolor{currentstroke}%
\pgfsetdash{}{0pt}%
\pgfpathmoveto{\pgfqpoint{8.227093in}{5.590197in}}%
\pgfpathquadraticcurveto{\pgfqpoint{8.214660in}{5.588793in}}{\pgfqpoint{8.208139in}{5.588056in}}%
\pgfusepath{stroke}%
\end{pgfscope}%
\begin{pgfscope}%
\pgfpathrectangle{\pgfqpoint{6.720588in}{4.155455in}}{\pgfqpoint{2.279412in}{2.004545in}}%
\pgfusepath{clip}%
\pgfsetroundcap%
\pgfsetroundjoin%
\definecolor{currentfill}{rgb}{0.280267,0.073417,0.397163}%
\pgfsetfillcolor{currentfill}%
\pgfsetlinewidth{0.384605pt}%
\definecolor{currentstroke}{rgb}{0.280267,0.073417,0.397163}%
\pgfsetstrokecolor{currentstroke}%
\pgfsetdash{}{0pt}%
\pgfpathmoveto{\pgfqpoint{8.266461in}{5.566688in}}%
\pgfpathlineto{\pgfqpoint{8.208139in}{5.588056in}}%
\pgfpathlineto{\pgfqpoint{8.260226in}{5.621893in}}%
\pgfpathlineto{\pgfqpoint{8.266461in}{5.566688in}}%
\pgfpathlineto{\pgfqpoint{8.266461in}{5.566688in}}%
\pgfpathclose%
\pgfusepath{stroke,fill}%
\end{pgfscope}%
\begin{pgfscope}%
\pgfpathrectangle{\pgfqpoint{6.720588in}{4.155455in}}{\pgfqpoint{2.279412in}{2.004545in}}%
\pgfusepath{clip}%
\pgfsetroundcap%
\pgfsetroundjoin%
\pgfsetlinewidth{0.372541pt}%
\definecolor{currentstroke}{rgb}{0.278791,0.062145,0.386592}%
\pgfsetstrokecolor{currentstroke}%
\pgfsetdash{}{0pt}%
\pgfpathmoveto{\pgfqpoint{8.375580in}{4.845333in}}%
\pgfpathquadraticcurveto{\pgfqpoint{8.363057in}{4.845848in}}{\pgfqpoint{8.356293in}{4.846126in}}%
\pgfusepath{stroke}%
\end{pgfscope}%
\begin{pgfscope}%
\pgfpathrectangle{\pgfqpoint{6.720588in}{4.155455in}}{\pgfqpoint{2.279412in}{2.004545in}}%
\pgfusepath{clip}%
\pgfsetroundcap%
\pgfsetroundjoin%
\definecolor{currentfill}{rgb}{0.278791,0.062145,0.386592}%
\pgfsetfillcolor{currentfill}%
\pgfsetlinewidth{0.372541pt}%
\definecolor{currentstroke}{rgb}{0.278791,0.062145,0.386592}%
\pgfsetstrokecolor{currentstroke}%
\pgfsetdash{}{0pt}%
\pgfpathmoveto{\pgfqpoint{8.410660in}{4.816089in}}%
\pgfpathlineto{\pgfqpoint{8.356293in}{4.846126in}}%
\pgfpathlineto{\pgfqpoint{8.412943in}{4.871598in}}%
\pgfpathlineto{\pgfqpoint{8.410660in}{4.816089in}}%
\pgfpathlineto{\pgfqpoint{8.410660in}{4.816089in}}%
\pgfpathclose%
\pgfusepath{stroke,fill}%
\end{pgfscope}%
\begin{pgfscope}%
\pgfpathrectangle{\pgfqpoint{6.720588in}{4.155455in}}{\pgfqpoint{2.279412in}{2.004545in}}%
\pgfusepath{clip}%
\pgfsetroundcap%
\pgfsetroundjoin%
\pgfsetlinewidth{0.383011pt}%
\definecolor{currentstroke}{rgb}{0.279566,0.067836,0.391917}%
\pgfsetstrokecolor{currentstroke}%
\pgfsetdash{}{0pt}%
\pgfpathmoveto{\pgfqpoint{8.126011in}{5.663885in}}%
\pgfpathquadraticcurveto{\pgfqpoint{8.113789in}{5.661477in}}{\pgfqpoint{8.107381in}{5.660214in}}%
\pgfusepath{stroke}%
\end{pgfscope}%
\begin{pgfscope}%
\pgfpathrectangle{\pgfqpoint{6.720588in}{4.155455in}}{\pgfqpoint{2.279412in}{2.004545in}}%
\pgfusepath{clip}%
\pgfsetroundcap%
\pgfsetroundjoin%
\definecolor{currentfill}{rgb}{0.279566,0.067836,0.391917}%
\pgfsetfillcolor{currentfill}%
\pgfsetlinewidth{0.383011pt}%
\definecolor{currentstroke}{rgb}{0.279566,0.067836,0.391917}%
\pgfsetstrokecolor{currentstroke}%
\pgfsetdash{}{0pt}%
\pgfpathmoveto{\pgfqpoint{8.167259in}{5.643701in}}%
\pgfpathlineto{\pgfqpoint{8.107381in}{5.660214in}}%
\pgfpathlineto{\pgfqpoint{8.156518in}{5.698209in}}%
\pgfpathlineto{\pgfqpoint{8.167259in}{5.643701in}}%
\pgfpathlineto{\pgfqpoint{8.167259in}{5.643701in}}%
\pgfpathclose%
\pgfusepath{stroke,fill}%
\end{pgfscope}%
\begin{pgfscope}%
\pgfpathrectangle{\pgfqpoint{6.720588in}{4.155455in}}{\pgfqpoint{2.279412in}{2.004545in}}%
\pgfusepath{clip}%
\pgfsetroundcap%
\pgfsetroundjoin%
\pgfsetlinewidth{0.357904pt}%
\definecolor{currentstroke}{rgb}{0.277018,0.050344,0.375715}%
\pgfsetstrokecolor{currentstroke}%
\pgfsetdash{}{0pt}%
\pgfpathmoveto{\pgfqpoint{8.068034in}{5.688682in}}%
\pgfpathquadraticcurveto{\pgfqpoint{8.055880in}{5.686034in}}{\pgfqpoint{8.049135in}{5.684565in}}%
\pgfusepath{stroke}%
\end{pgfscope}%
\begin{pgfscope}%
\pgfpathrectangle{\pgfqpoint{6.720588in}{4.155455in}}{\pgfqpoint{2.279412in}{2.004545in}}%
\pgfusepath{clip}%
\pgfsetroundcap%
\pgfsetroundjoin%
\definecolor{currentfill}{rgb}{0.277018,0.050344,0.375715}%
\pgfsetfillcolor{currentfill}%
\pgfsetlinewidth{0.357904pt}%
\definecolor{currentstroke}{rgb}{0.277018,0.050344,0.375715}%
\pgfsetstrokecolor{currentstroke}%
\pgfsetdash{}{0pt}%
\pgfpathmoveto{\pgfqpoint{8.109330in}{5.669248in}}%
\pgfpathlineto{\pgfqpoint{8.049135in}{5.684565in}}%
\pgfpathlineto{\pgfqpoint{8.097505in}{5.723530in}}%
\pgfpathlineto{\pgfqpoint{8.109330in}{5.669248in}}%
\pgfpathlineto{\pgfqpoint{8.109330in}{5.669248in}}%
\pgfpathclose%
\pgfusepath{stroke,fill}%
\end{pgfscope}%
\begin{pgfscope}%
\pgfpathrectangle{\pgfqpoint{6.720588in}{4.155455in}}{\pgfqpoint{2.279412in}{2.004545in}}%
\pgfusepath{clip}%
\pgfsetroundcap%
\pgfsetroundjoin%
\pgfsetlinewidth{0.374347pt}%
\definecolor{currentstroke}{rgb}{0.278791,0.062145,0.386592}%
\pgfsetstrokecolor{currentstroke}%
\pgfsetdash{}{0pt}%
\pgfpathmoveto{\pgfqpoint{8.265436in}{4.641832in}}%
\pgfpathquadraticcurveto{\pgfqpoint{8.253016in}{4.643322in}}{\pgfqpoint{8.246346in}{4.644123in}}%
\pgfusepath{stroke}%
\end{pgfscope}%
\begin{pgfscope}%
\pgfpathrectangle{\pgfqpoint{6.720588in}{4.155455in}}{\pgfqpoint{2.279412in}{2.004545in}}%
\pgfusepath{clip}%
\pgfsetroundcap%
\pgfsetroundjoin%
\definecolor{currentfill}{rgb}{0.278791,0.062145,0.386592}%
\pgfsetfillcolor{currentfill}%
\pgfsetlinewidth{0.374347pt}%
\definecolor{currentstroke}{rgb}{0.278791,0.062145,0.386592}%
\pgfsetstrokecolor{currentstroke}%
\pgfsetdash{}{0pt}%
\pgfpathmoveto{\pgfqpoint{8.298196in}{4.609924in}}%
\pgfpathlineto{\pgfqpoint{8.246346in}{4.644123in}}%
\pgfpathlineto{\pgfqpoint{8.304815in}{4.665084in}}%
\pgfpathlineto{\pgfqpoint{8.298196in}{4.609924in}}%
\pgfpathlineto{\pgfqpoint{8.298196in}{4.609924in}}%
\pgfpathclose%
\pgfusepath{stroke,fill}%
\end{pgfscope}%
\begin{pgfscope}%
\pgfpathrectangle{\pgfqpoint{6.720588in}{4.155455in}}{\pgfqpoint{2.279412in}{2.004545in}}%
\pgfusepath{clip}%
\pgfsetroundcap%
\pgfsetroundjoin%
\pgfsetlinewidth{0.334284pt}%
\definecolor{currentstroke}{rgb}{0.272594,0.025563,0.353093}%
\pgfsetstrokecolor{currentstroke}%
\pgfsetdash{}{0pt}%
\pgfpathmoveto{\pgfqpoint{8.368871in}{5.667434in}}%
\pgfpathquadraticcurveto{\pgfqpoint{8.356416in}{5.666309in}}{\pgfqpoint{8.349111in}{5.665649in}}%
\pgfusepath{stroke}%
\end{pgfscope}%
\begin{pgfscope}%
\pgfpathrectangle{\pgfqpoint{6.720588in}{4.155455in}}{\pgfqpoint{2.279412in}{2.004545in}}%
\pgfusepath{clip}%
\pgfsetroundcap%
\pgfsetroundjoin%
\definecolor{currentfill}{rgb}{0.272594,0.025563,0.353093}%
\pgfsetfillcolor{currentfill}%
\pgfsetlinewidth{0.334284pt}%
\definecolor{currentstroke}{rgb}{0.272594,0.025563,0.353093}%
\pgfsetstrokecolor{currentstroke}%
\pgfsetdash{}{0pt}%
\pgfpathmoveto{\pgfqpoint{8.406940in}{5.642982in}}%
\pgfpathlineto{\pgfqpoint{8.349111in}{5.665649in}}%
\pgfpathlineto{\pgfqpoint{8.401943in}{5.698312in}}%
\pgfpathlineto{\pgfqpoint{8.406940in}{5.642982in}}%
\pgfpathlineto{\pgfqpoint{8.406940in}{5.642982in}}%
\pgfpathclose%
\pgfusepath{stroke,fill}%
\end{pgfscope}%
\begin{pgfscope}%
\pgfpathrectangle{\pgfqpoint{6.720588in}{4.155455in}}{\pgfqpoint{2.279412in}{2.004545in}}%
\pgfusepath{clip}%
\pgfsetroundcap%
\pgfsetroundjoin%
\pgfsetlinewidth{0.412902pt}%
\definecolor{currentstroke}{rgb}{0.282327,0.094955,0.417331}%
\pgfsetstrokecolor{currentstroke}%
\pgfsetdash{}{0pt}%
\pgfpathmoveto{\pgfqpoint{7.872799in}{4.744530in}}%
\pgfpathquadraticcurveto{\pgfqpoint{7.862493in}{4.750622in}}{\pgfqpoint{7.857686in}{4.753464in}}%
\pgfusepath{stroke}%
\end{pgfscope}%
\begin{pgfscope}%
\pgfpathrectangle{\pgfqpoint{6.720588in}{4.155455in}}{\pgfqpoint{2.279412in}{2.004545in}}%
\pgfusepath{clip}%
\pgfsetroundcap%
\pgfsetroundjoin%
\definecolor{currentfill}{rgb}{0.282327,0.094955,0.417331}%
\pgfsetfillcolor{currentfill}%
\pgfsetlinewidth{0.412902pt}%
\definecolor{currentstroke}{rgb}{0.282327,0.094955,0.417331}%
\pgfsetstrokecolor{currentstroke}%
\pgfsetdash{}{0pt}%
\pgfpathmoveto{\pgfqpoint{7.891373in}{4.701280in}}%
\pgfpathlineto{\pgfqpoint{7.857686in}{4.753464in}}%
\pgfpathlineto{\pgfqpoint{7.919646in}{4.749104in}}%
\pgfpathlineto{\pgfqpoint{7.891373in}{4.701280in}}%
\pgfpathlineto{\pgfqpoint{7.891373in}{4.701280in}}%
\pgfpathclose%
\pgfusepath{stroke,fill}%
\end{pgfscope}%
\begin{pgfscope}%
\pgfpathrectangle{\pgfqpoint{6.720588in}{4.155455in}}{\pgfqpoint{2.279412in}{2.004545in}}%
\pgfusepath{clip}%
\pgfsetroundcap%
\pgfsetroundjoin%
\pgfsetlinewidth{1.326505pt}%
\definecolor{currentstroke}{rgb}{0.126326,0.644107,0.525311}%
\pgfsetstrokecolor{currentstroke}%
\pgfsetdash{}{0pt}%
\pgfpathmoveto{\pgfqpoint{7.684888in}{5.368714in}}%
\pgfpathquadraticcurveto{\pgfqpoint{7.675488in}{5.361439in}}{\pgfqpoint{7.682316in}{5.366723in}}%
\pgfusepath{stroke}%
\end{pgfscope}%
\begin{pgfscope}%
\pgfpathrectangle{\pgfqpoint{6.720588in}{4.155455in}}{\pgfqpoint{2.279412in}{2.004545in}}%
\pgfusepath{clip}%
\pgfsetroundcap%
\pgfsetroundjoin%
\definecolor{currentfill}{rgb}{0.126326,0.644107,0.525311}%
\pgfsetfillcolor{currentfill}%
\pgfsetlinewidth{1.326505pt}%
\definecolor{currentstroke}{rgb}{0.126326,0.644107,0.525311}%
\pgfsetstrokecolor{currentstroke}%
\pgfsetdash{}{0pt}%
\pgfpathmoveto{\pgfqpoint{7.743252in}{5.378758in}}%
\pgfpathlineto{\pgfqpoint{7.682316in}{5.366723in}}%
\pgfpathlineto{\pgfqpoint{7.709249in}{5.422693in}}%
\pgfpathlineto{\pgfqpoint{7.743252in}{5.378758in}}%
\pgfpathlineto{\pgfqpoint{7.743252in}{5.378758in}}%
\pgfpathclose%
\pgfusepath{stroke,fill}%
\end{pgfscope}%
\begin{pgfscope}%
\pgfpathrectangle{\pgfqpoint{6.720588in}{4.155455in}}{\pgfqpoint{2.279412in}{2.004545in}}%
\pgfusepath{clip}%
\pgfsetroundcap%
\pgfsetroundjoin%
\pgfsetlinewidth{1.298551pt}%
\definecolor{currentstroke}{rgb}{0.121380,0.629492,0.531973}%
\pgfsetstrokecolor{currentstroke}%
\pgfsetdash{}{0pt}%
\pgfpathmoveto{\pgfqpoint{7.915681in}{5.321677in}}%
\pgfpathquadraticcurveto{\pgfqpoint{7.903519in}{5.319009in}}{\pgfqpoint{7.910979in}{5.320646in}}%
\pgfusepath{stroke}%
\end{pgfscope}%
\begin{pgfscope}%
\pgfpathrectangle{\pgfqpoint{6.720588in}{4.155455in}}{\pgfqpoint{2.279412in}{2.004545in}}%
\pgfusepath{clip}%
\pgfsetroundcap%
\pgfsetroundjoin%
\definecolor{currentfill}{rgb}{0.121380,0.629492,0.531973}%
\pgfsetfillcolor{currentfill}%
\pgfsetlinewidth{1.298551pt}%
\definecolor{currentstroke}{rgb}{0.121380,0.629492,0.531973}%
\pgfsetstrokecolor{currentstroke}%
\pgfsetdash{}{0pt}%
\pgfpathmoveto{\pgfqpoint{7.971196in}{5.305414in}}%
\pgfpathlineto{\pgfqpoint{7.910979in}{5.320646in}}%
\pgfpathlineto{\pgfqpoint{7.959294in}{5.359680in}}%
\pgfpathlineto{\pgfqpoint{7.971196in}{5.305414in}}%
\pgfpathlineto{\pgfqpoint{7.971196in}{5.305414in}}%
\pgfpathclose%
\pgfusepath{stroke,fill}%
\end{pgfscope}%
\begin{pgfscope}%
\pgfpathrectangle{\pgfqpoint{6.720588in}{4.155455in}}{\pgfqpoint{2.279412in}{2.004545in}}%
\pgfusepath{clip}%
\pgfsetroundcap%
\pgfsetroundjoin%
\pgfsetlinewidth{1.993114pt}%
\definecolor{currentstroke}{rgb}{0.906311,0.894855,0.098125}%
\pgfsetstrokecolor{currentstroke}%
\pgfsetdash{}{0pt}%
\pgfpathmoveto{\pgfqpoint{7.806511in}{5.258045in}}%
\pgfpathquadraticcurveto{\pgfqpoint{7.794311in}{5.255519in}}{\pgfqpoint{7.812304in}{5.259245in}}%
\pgfusepath{stroke}%
\end{pgfscope}%
\begin{pgfscope}%
\pgfpathrectangle{\pgfqpoint{6.720588in}{4.155455in}}{\pgfqpoint{2.279412in}{2.004545in}}%
\pgfusepath{clip}%
\pgfsetroundcap%
\pgfsetroundjoin%
\definecolor{currentfill}{rgb}{0.906311,0.894855,0.098125}%
\pgfsetfillcolor{currentfill}%
\pgfsetlinewidth{1.993114pt}%
\definecolor{currentstroke}{rgb}{0.906311,0.894855,0.098125}%
\pgfsetstrokecolor{currentstroke}%
\pgfsetdash{}{0pt}%
\pgfpathmoveto{\pgfqpoint{7.872338in}{5.243308in}}%
\pgfpathlineto{\pgfqpoint{7.812304in}{5.259245in}}%
\pgfpathlineto{\pgfqpoint{7.861074in}{5.297710in}}%
\pgfpathlineto{\pgfqpoint{7.872338in}{5.243308in}}%
\pgfpathlineto{\pgfqpoint{7.872338in}{5.243308in}}%
\pgfpathclose%
\pgfusepath{stroke,fill}%
\end{pgfscope}%
\begin{pgfscope}%
\pgfpathrectangle{\pgfqpoint{6.720588in}{4.155455in}}{\pgfqpoint{2.279412in}{2.004545in}}%
\pgfusepath{clip}%
\pgfsetbuttcap%
\pgfsetroundjoin%
\pgfsetlinewidth{1.505625pt}%
\definecolor{currentstroke}{rgb}{0.000000,0.000000,0.000000}%
\pgfsetstrokecolor{currentstroke}%
\pgfsetdash{}{0pt}%
\pgfpathmoveto{\pgfqpoint{7.513972in}{4.494895in}}%
\pgfpathlineto{\pgfqpoint{7.513972in}{5.820559in}}%
\pgfusepath{stroke}%
\end{pgfscope}%
\begin{pgfscope}%
\pgfpathrectangle{\pgfqpoint{6.720588in}{4.155455in}}{\pgfqpoint{2.279412in}{2.004545in}}%
\pgfusepath{clip}%
\pgfsetbuttcap%
\pgfsetroundjoin%
\pgfsetlinewidth{1.505625pt}%
\definecolor{currentstroke}{rgb}{0.000000,0.000000,0.000000}%
\pgfsetstrokecolor{currentstroke}%
\pgfsetdash{}{0pt}%
\pgfpathmoveto{\pgfqpoint{8.662499in}{4.494895in}}%
\pgfpathlineto{\pgfqpoint{8.662499in}{5.820559in}}%
\pgfusepath{stroke}%
\end{pgfscope}%
\begin{pgfscope}%
\pgfsetrectcap%
\pgfsetmiterjoin%
\pgfsetlinewidth{0.803000pt}%
\definecolor{currentstroke}{rgb}{0.000000,0.000000,0.000000}%
\pgfsetstrokecolor{currentstroke}%
\pgfsetdash{}{0pt}%
\pgfpathmoveto{\pgfqpoint{6.720588in}{4.155455in}}%
\pgfpathlineto{\pgfqpoint{6.720588in}{6.160000in}}%
\pgfusepath{stroke}%
\end{pgfscope}%
\begin{pgfscope}%
\pgfsetrectcap%
\pgfsetmiterjoin%
\pgfsetlinewidth{0.803000pt}%
\definecolor{currentstroke}{rgb}{0.000000,0.000000,0.000000}%
\pgfsetstrokecolor{currentstroke}%
\pgfsetdash{}{0pt}%
\pgfpathmoveto{\pgfqpoint{9.000000in}{4.155455in}}%
\pgfpathlineto{\pgfqpoint{9.000000in}{6.160000in}}%
\pgfusepath{stroke}%
\end{pgfscope}%
\begin{pgfscope}%
\pgfsetrectcap%
\pgfsetmiterjoin%
\pgfsetlinewidth{0.803000pt}%
\definecolor{currentstroke}{rgb}{0.000000,0.000000,0.000000}%
\pgfsetstrokecolor{currentstroke}%
\pgfsetdash{}{0pt}%
\pgfpathmoveto{\pgfqpoint{6.720588in}{4.155455in}}%
\pgfpathlineto{\pgfqpoint{9.000000in}{4.155455in}}%
\pgfusepath{stroke}%
\end{pgfscope}%
\begin{pgfscope}%
\pgfsetrectcap%
\pgfsetmiterjoin%
\pgfsetlinewidth{0.803000pt}%
\definecolor{currentstroke}{rgb}{0.000000,0.000000,0.000000}%
\pgfsetstrokecolor{currentstroke}%
\pgfsetdash{}{0pt}%
\pgfpathmoveto{\pgfqpoint{6.720588in}{6.160000in}}%
\pgfpathlineto{\pgfqpoint{9.000000in}{6.160000in}}%
\pgfusepath{stroke}%
\end{pgfscope}%
\begin{pgfscope}%
\definecolor{textcolor}{rgb}{0.000000,0.000000,0.000000}%
\pgfsetstrokecolor{textcolor}%
\pgfsetfillcolor{textcolor}%
\pgftext[x=7.860294in,y=6.243333in,,base]{\color{textcolor}\sffamily\fontsize{12.000000}{14.400000}\selectfont c)}%
\end{pgfscope}%
\begin{pgfscope}%
\pgfsetbuttcap%
\pgfsetmiterjoin%
\definecolor{currentfill}{rgb}{1.000000,1.000000,1.000000}%
\pgfsetfillcolor{currentfill}%
\pgfsetlinewidth{0.000000pt}%
\definecolor{currentstroke}{rgb}{0.000000,0.000000,0.000000}%
\pgfsetstrokecolor{currentstroke}%
\pgfsetstrokeopacity{0.000000}%
\pgfsetdash{}{0pt}%
\pgfpathmoveto{\pgfqpoint{1.250000in}{1.750000in}}%
\pgfpathlineto{\pgfqpoint{3.529412in}{1.750000in}}%
\pgfpathlineto{\pgfqpoint{3.529412in}{3.754545in}}%
\pgfpathlineto{\pgfqpoint{1.250000in}{3.754545in}}%
\pgfpathlineto{\pgfqpoint{1.250000in}{1.750000in}}%
\pgfpathclose%
\pgfusepath{fill}%
\end{pgfscope}%
\begin{pgfscope}%
\pgfpathrectangle{\pgfqpoint{1.250000in}{1.750000in}}{\pgfqpoint{2.279412in}{2.004545in}}%
\pgfusepath{clip}%
\pgfsys@transformcm{2.291667}{0.000000}{0.000000}{2.013889}{1.250000in}{1.750000in}%
\pgftext[left,bottom]{\includegraphics[interpolate=false,width=1.000000in,height=1.000000in]{q_series-img3.png}}%
\end{pgfscope}%
\begin{pgfscope}%
\pgfsetbuttcap%
\pgfsetroundjoin%
\definecolor{currentfill}{rgb}{0.000000,0.000000,0.000000}%
\pgfsetfillcolor{currentfill}%
\pgfsetlinewidth{0.803000pt}%
\definecolor{currentstroke}{rgb}{0.000000,0.000000,0.000000}%
\pgfsetstrokecolor{currentstroke}%
\pgfsetdash{}{0pt}%
\pgfsys@defobject{currentmarker}{\pgfqpoint{0.000000in}{-0.048611in}}{\pgfqpoint{0.000000in}{0.000000in}}{%
\pgfpathmoveto{\pgfqpoint{0.000000in}{0.000000in}}%
\pgfpathlineto{\pgfqpoint{0.000000in}{-0.048611in}}%
\pgfusepath{stroke,fill}%
}%
\begin{pgfscope}%
\pgfsys@transformshift{1.660542in}{1.750000in}%
\pgfsys@useobject{currentmarker}{}%
\end{pgfscope}%
\end{pgfscope}%
\begin{pgfscope}%
\definecolor{textcolor}{rgb}{0.000000,0.000000,0.000000}%
\pgfsetstrokecolor{textcolor}%
\pgfsetfillcolor{textcolor}%
\pgftext[x=1.660542in,y=1.652778in,,top]{\color{textcolor}\sffamily\fontsize{10.000000}{12.000000}\selectfont \(\displaystyle {\ensuremath{-}10}\)}%
\end{pgfscope}%
\begin{pgfscope}%
\pgfsetbuttcap%
\pgfsetroundjoin%
\definecolor{currentfill}{rgb}{0.000000,0.000000,0.000000}%
\pgfsetfillcolor{currentfill}%
\pgfsetlinewidth{0.803000pt}%
\definecolor{currentstroke}{rgb}{0.000000,0.000000,0.000000}%
\pgfsetstrokecolor{currentstroke}%
\pgfsetdash{}{0pt}%
\pgfsys@defobject{currentmarker}{\pgfqpoint{0.000000in}{-0.048611in}}{\pgfqpoint{0.000000in}{0.000000in}}{%
\pgfpathmoveto{\pgfqpoint{0.000000in}{0.000000in}}%
\pgfpathlineto{\pgfqpoint{0.000000in}{-0.048611in}}%
\pgfusepath{stroke,fill}%
}%
\begin{pgfscope}%
\pgfsys@transformshift{2.139094in}{1.750000in}%
\pgfsys@useobject{currentmarker}{}%
\end{pgfscope}%
\end{pgfscope}%
\begin{pgfscope}%
\definecolor{textcolor}{rgb}{0.000000,0.000000,0.000000}%
\pgfsetstrokecolor{textcolor}%
\pgfsetfillcolor{textcolor}%
\pgftext[x=2.139094in,y=1.652778in,,top]{\color{textcolor}\sffamily\fontsize{10.000000}{12.000000}\selectfont \(\displaystyle {\ensuremath{-}5}\)}%
\end{pgfscope}%
\begin{pgfscope}%
\pgfsetbuttcap%
\pgfsetroundjoin%
\definecolor{currentfill}{rgb}{0.000000,0.000000,0.000000}%
\pgfsetfillcolor{currentfill}%
\pgfsetlinewidth{0.803000pt}%
\definecolor{currentstroke}{rgb}{0.000000,0.000000,0.000000}%
\pgfsetstrokecolor{currentstroke}%
\pgfsetdash{}{0pt}%
\pgfsys@defobject{currentmarker}{\pgfqpoint{0.000000in}{-0.048611in}}{\pgfqpoint{0.000000in}{0.000000in}}{%
\pgfpathmoveto{\pgfqpoint{0.000000in}{0.000000in}}%
\pgfpathlineto{\pgfqpoint{0.000000in}{-0.048611in}}%
\pgfusepath{stroke,fill}%
}%
\begin{pgfscope}%
\pgfsys@transformshift{2.617647in}{1.750000in}%
\pgfsys@useobject{currentmarker}{}%
\end{pgfscope}%
\end{pgfscope}%
\begin{pgfscope}%
\definecolor{textcolor}{rgb}{0.000000,0.000000,0.000000}%
\pgfsetstrokecolor{textcolor}%
\pgfsetfillcolor{textcolor}%
\pgftext[x=2.617647in,y=1.652778in,,top]{\color{textcolor}\sffamily\fontsize{10.000000}{12.000000}\selectfont \(\displaystyle {0}\)}%
\end{pgfscope}%
\begin{pgfscope}%
\pgfsetbuttcap%
\pgfsetroundjoin%
\definecolor{currentfill}{rgb}{0.000000,0.000000,0.000000}%
\pgfsetfillcolor{currentfill}%
\pgfsetlinewidth{0.803000pt}%
\definecolor{currentstroke}{rgb}{0.000000,0.000000,0.000000}%
\pgfsetstrokecolor{currentstroke}%
\pgfsetdash{}{0pt}%
\pgfsys@defobject{currentmarker}{\pgfqpoint{0.000000in}{-0.048611in}}{\pgfqpoint{0.000000in}{0.000000in}}{%
\pgfpathmoveto{\pgfqpoint{0.000000in}{0.000000in}}%
\pgfpathlineto{\pgfqpoint{0.000000in}{-0.048611in}}%
\pgfusepath{stroke,fill}%
}%
\begin{pgfscope}%
\pgfsys@transformshift{3.096200in}{1.750000in}%
\pgfsys@useobject{currentmarker}{}%
\end{pgfscope}%
\end{pgfscope}%
\begin{pgfscope}%
\definecolor{textcolor}{rgb}{0.000000,0.000000,0.000000}%
\pgfsetstrokecolor{textcolor}%
\pgfsetfillcolor{textcolor}%
\pgftext[x=3.096200in,y=1.652778in,,top]{\color{textcolor}\sffamily\fontsize{10.000000}{12.000000}\selectfont \(\displaystyle {5}\)}%
\end{pgfscope}%
\begin{pgfscope}%
\definecolor{textcolor}{rgb}{0.000000,0.000000,0.000000}%
\pgfsetstrokecolor{textcolor}%
\pgfsetfillcolor{textcolor}%
\pgftext[x=2.389706in,y=1.473766in,,top]{\color{textcolor}\sffamily\fontsize{10.000000}{12.000000}\selectfont \(\displaystyle \zeta \, \mathrm{[\mu m]}\)}%
\end{pgfscope}%
\begin{pgfscope}%
\pgfsetbuttcap%
\pgfsetroundjoin%
\definecolor{currentfill}{rgb}{0.000000,0.000000,0.000000}%
\pgfsetfillcolor{currentfill}%
\pgfsetlinewidth{0.803000pt}%
\definecolor{currentstroke}{rgb}{0.000000,0.000000,0.000000}%
\pgfsetstrokecolor{currentstroke}%
\pgfsetdash{}{0pt}%
\pgfsys@defobject{currentmarker}{\pgfqpoint{-0.048611in}{0.000000in}}{\pgfqpoint{-0.000000in}{0.000000in}}{%
\pgfpathmoveto{\pgfqpoint{-0.000000in}{0.000000in}}%
\pgfpathlineto{\pgfqpoint{-0.048611in}{0.000000in}}%
\pgfusepath{stroke,fill}%
}%
\begin{pgfscope}%
\pgfsys@transformshift{1.250000in}{1.758025in}%
\pgfsys@useobject{currentmarker}{}%
\end{pgfscope}%
\end{pgfscope}%
\begin{pgfscope}%
\definecolor{textcolor}{rgb}{0.000000,0.000000,0.000000}%
\pgfsetstrokecolor{textcolor}%
\pgfsetfillcolor{textcolor}%
\pgftext[x=0.905863in, y=1.709799in, left, base]{\color{textcolor}\sffamily\fontsize{10.000000}{12.000000}\selectfont \(\displaystyle {\ensuremath{-}30}\)}%
\end{pgfscope}%
\begin{pgfscope}%
\pgfsetbuttcap%
\pgfsetroundjoin%
\definecolor{currentfill}{rgb}{0.000000,0.000000,0.000000}%
\pgfsetfillcolor{currentfill}%
\pgfsetlinewidth{0.803000pt}%
\definecolor{currentstroke}{rgb}{0.000000,0.000000,0.000000}%
\pgfsetstrokecolor{currentstroke}%
\pgfsetdash{}{0pt}%
\pgfsys@defobject{currentmarker}{\pgfqpoint{-0.048611in}{0.000000in}}{\pgfqpoint{-0.000000in}{0.000000in}}{%
\pgfpathmoveto{\pgfqpoint{-0.000000in}{0.000000in}}%
\pgfpathlineto{\pgfqpoint{-0.048611in}{0.000000in}}%
\pgfusepath{stroke,fill}%
}%
\begin{pgfscope}%
\pgfsys@transformshift{1.250000in}{2.089441in}%
\pgfsys@useobject{currentmarker}{}%
\end{pgfscope}%
\end{pgfscope}%
\begin{pgfscope}%
\definecolor{textcolor}{rgb}{0.000000,0.000000,0.000000}%
\pgfsetstrokecolor{textcolor}%
\pgfsetfillcolor{textcolor}%
\pgftext[x=0.905863in, y=2.041215in, left, base]{\color{textcolor}\sffamily\fontsize{10.000000}{12.000000}\selectfont \(\displaystyle {\ensuremath{-}20}\)}%
\end{pgfscope}%
\begin{pgfscope}%
\pgfsetbuttcap%
\pgfsetroundjoin%
\definecolor{currentfill}{rgb}{0.000000,0.000000,0.000000}%
\pgfsetfillcolor{currentfill}%
\pgfsetlinewidth{0.803000pt}%
\definecolor{currentstroke}{rgb}{0.000000,0.000000,0.000000}%
\pgfsetstrokecolor{currentstroke}%
\pgfsetdash{}{0pt}%
\pgfsys@defobject{currentmarker}{\pgfqpoint{-0.048611in}{0.000000in}}{\pgfqpoint{-0.000000in}{0.000000in}}{%
\pgfpathmoveto{\pgfqpoint{-0.000000in}{0.000000in}}%
\pgfpathlineto{\pgfqpoint{-0.048611in}{0.000000in}}%
\pgfusepath{stroke,fill}%
}%
\begin{pgfscope}%
\pgfsys@transformshift{1.250000in}{2.420857in}%
\pgfsys@useobject{currentmarker}{}%
\end{pgfscope}%
\end{pgfscope}%
\begin{pgfscope}%
\definecolor{textcolor}{rgb}{0.000000,0.000000,0.000000}%
\pgfsetstrokecolor{textcolor}%
\pgfsetfillcolor{textcolor}%
\pgftext[x=0.905863in, y=2.372631in, left, base]{\color{textcolor}\sffamily\fontsize{10.000000}{12.000000}\selectfont \(\displaystyle {\ensuremath{-}10}\)}%
\end{pgfscope}%
\begin{pgfscope}%
\pgfsetbuttcap%
\pgfsetroundjoin%
\definecolor{currentfill}{rgb}{0.000000,0.000000,0.000000}%
\pgfsetfillcolor{currentfill}%
\pgfsetlinewidth{0.803000pt}%
\definecolor{currentstroke}{rgb}{0.000000,0.000000,0.000000}%
\pgfsetstrokecolor{currentstroke}%
\pgfsetdash{}{0pt}%
\pgfsys@defobject{currentmarker}{\pgfqpoint{-0.048611in}{0.000000in}}{\pgfqpoint{-0.000000in}{0.000000in}}{%
\pgfpathmoveto{\pgfqpoint{-0.000000in}{0.000000in}}%
\pgfpathlineto{\pgfqpoint{-0.048611in}{0.000000in}}%
\pgfusepath{stroke,fill}%
}%
\begin{pgfscope}%
\pgfsys@transformshift{1.250000in}{2.752273in}%
\pgfsys@useobject{currentmarker}{}%
\end{pgfscope}%
\end{pgfscope}%
\begin{pgfscope}%
\definecolor{textcolor}{rgb}{0.000000,0.000000,0.000000}%
\pgfsetstrokecolor{textcolor}%
\pgfsetfillcolor{textcolor}%
\pgftext[x=1.083333in, y=2.704047in, left, base]{\color{textcolor}\sffamily\fontsize{10.000000}{12.000000}\selectfont \(\displaystyle {0}\)}%
\end{pgfscope}%
\begin{pgfscope}%
\pgfsetbuttcap%
\pgfsetroundjoin%
\definecolor{currentfill}{rgb}{0.000000,0.000000,0.000000}%
\pgfsetfillcolor{currentfill}%
\pgfsetlinewidth{0.803000pt}%
\definecolor{currentstroke}{rgb}{0.000000,0.000000,0.000000}%
\pgfsetstrokecolor{currentstroke}%
\pgfsetdash{}{0pt}%
\pgfsys@defobject{currentmarker}{\pgfqpoint{-0.048611in}{0.000000in}}{\pgfqpoint{-0.000000in}{0.000000in}}{%
\pgfpathmoveto{\pgfqpoint{-0.000000in}{0.000000in}}%
\pgfpathlineto{\pgfqpoint{-0.048611in}{0.000000in}}%
\pgfusepath{stroke,fill}%
}%
\begin{pgfscope}%
\pgfsys@transformshift{1.250000in}{3.083689in}%
\pgfsys@useobject{currentmarker}{}%
\end{pgfscope}%
\end{pgfscope}%
\begin{pgfscope}%
\definecolor{textcolor}{rgb}{0.000000,0.000000,0.000000}%
\pgfsetstrokecolor{textcolor}%
\pgfsetfillcolor{textcolor}%
\pgftext[x=1.013888in, y=3.035463in, left, base]{\color{textcolor}\sffamily\fontsize{10.000000}{12.000000}\selectfont \(\displaystyle {10}\)}%
\end{pgfscope}%
\begin{pgfscope}%
\pgfsetbuttcap%
\pgfsetroundjoin%
\definecolor{currentfill}{rgb}{0.000000,0.000000,0.000000}%
\pgfsetfillcolor{currentfill}%
\pgfsetlinewidth{0.803000pt}%
\definecolor{currentstroke}{rgb}{0.000000,0.000000,0.000000}%
\pgfsetstrokecolor{currentstroke}%
\pgfsetdash{}{0pt}%
\pgfsys@defobject{currentmarker}{\pgfqpoint{-0.048611in}{0.000000in}}{\pgfqpoint{-0.000000in}{0.000000in}}{%
\pgfpathmoveto{\pgfqpoint{-0.000000in}{0.000000in}}%
\pgfpathlineto{\pgfqpoint{-0.048611in}{0.000000in}}%
\pgfusepath{stroke,fill}%
}%
\begin{pgfscope}%
\pgfsys@transformshift{1.250000in}{3.415105in}%
\pgfsys@useobject{currentmarker}{}%
\end{pgfscope}%
\end{pgfscope}%
\begin{pgfscope}%
\definecolor{textcolor}{rgb}{0.000000,0.000000,0.000000}%
\pgfsetstrokecolor{textcolor}%
\pgfsetfillcolor{textcolor}%
\pgftext[x=1.013888in, y=3.366879in, left, base]{\color{textcolor}\sffamily\fontsize{10.000000}{12.000000}\selectfont \(\displaystyle {20}\)}%
\end{pgfscope}%
\begin{pgfscope}%
\pgfsetbuttcap%
\pgfsetroundjoin%
\definecolor{currentfill}{rgb}{0.000000,0.000000,0.000000}%
\pgfsetfillcolor{currentfill}%
\pgfsetlinewidth{0.803000pt}%
\definecolor{currentstroke}{rgb}{0.000000,0.000000,0.000000}%
\pgfsetstrokecolor{currentstroke}%
\pgfsetdash{}{0pt}%
\pgfsys@defobject{currentmarker}{\pgfqpoint{-0.048611in}{0.000000in}}{\pgfqpoint{-0.000000in}{0.000000in}}{%
\pgfpathmoveto{\pgfqpoint{-0.000000in}{0.000000in}}%
\pgfpathlineto{\pgfqpoint{-0.048611in}{0.000000in}}%
\pgfusepath{stroke,fill}%
}%
\begin{pgfscope}%
\pgfsys@transformshift{1.250000in}{3.746521in}%
\pgfsys@useobject{currentmarker}{}%
\end{pgfscope}%
\end{pgfscope}%
\begin{pgfscope}%
\definecolor{textcolor}{rgb}{0.000000,0.000000,0.000000}%
\pgfsetstrokecolor{textcolor}%
\pgfsetfillcolor{textcolor}%
\pgftext[x=1.013888in, y=3.698295in, left, base]{\color{textcolor}\sffamily\fontsize{10.000000}{12.000000}\selectfont \(\displaystyle {30}\)}%
\end{pgfscope}%
\begin{pgfscope}%
\definecolor{textcolor}{rgb}{0.000000,0.000000,0.000000}%
\pgfsetstrokecolor{textcolor}%
\pgfsetfillcolor{textcolor}%
\pgftext[x=0.850308in,y=2.752273in,,bottom,rotate=90.000000]{\color{textcolor}\sffamily\fontsize{10.000000}{12.000000}\selectfont \(\displaystyle z \, \mathrm{[\mu m]}\)}%
\end{pgfscope}%
\begin{pgfscope}%
\pgfpathrectangle{\pgfqpoint{1.250000in}{1.750000in}}{\pgfqpoint{2.279412in}{2.004545in}}%
\pgfusepath{clip}%
\pgfsetbuttcap%
\pgfsetroundjoin%
\pgfsetlinewidth{0.312301pt}%
\definecolor{currentstroke}{rgb}{0.268510,0.009605,0.335427}%
\pgfsetstrokecolor{currentstroke}%
\pgfsetdash{}{0pt}%
\pgfpathmoveto{\pgfqpoint{3.261668in}{2.842486in}}%
\pgfpathlineto{\pgfqpoint{3.261668in}{2.842486in}}%
\pgfusepath{stroke}%
\end{pgfscope}%
\begin{pgfscope}%
\pgfpathrectangle{\pgfqpoint{1.250000in}{1.750000in}}{\pgfqpoint{2.279412in}{2.004545in}}%
\pgfusepath{clip}%
\pgfsetbuttcap%
\pgfsetroundjoin%
\pgfsetlinewidth{0.312301pt}%
\definecolor{currentstroke}{rgb}{0.268510,0.009605,0.335427}%
\pgfsetstrokecolor{currentstroke}%
\pgfsetdash{}{0pt}%
\pgfpathmoveto{\pgfqpoint{3.261668in}{2.842486in}}%
\pgfpathlineto{\pgfqpoint{3.261668in}{2.842486in}}%
\pgfusepath{stroke}%
\end{pgfscope}%
\begin{pgfscope}%
\pgfpathrectangle{\pgfqpoint{1.250000in}{1.750000in}}{\pgfqpoint{2.279412in}{2.004545in}}%
\pgfusepath{clip}%
\pgfsetbuttcap%
\pgfsetroundjoin%
\pgfsetlinewidth{0.312301pt}%
\definecolor{currentstroke}{rgb}{0.268510,0.009605,0.335427}%
\pgfsetstrokecolor{currentstroke}%
\pgfsetdash{}{0pt}%
\pgfpathmoveto{\pgfqpoint{3.261668in}{2.842486in}}%
\pgfpathlineto{\pgfqpoint{3.246937in}{2.841358in}}%
\pgfusepath{stroke}%
\end{pgfscope}%
\begin{pgfscope}%
\pgfpathrectangle{\pgfqpoint{1.250000in}{1.750000in}}{\pgfqpoint{2.279412in}{2.004545in}}%
\pgfusepath{clip}%
\pgfsetbuttcap%
\pgfsetroundjoin%
\pgfsetlinewidth{0.320187pt}%
\definecolor{currentstroke}{rgb}{0.269944,0.014625,0.341379}%
\pgfsetstrokecolor{currentstroke}%
\pgfsetdash{}{0pt}%
\pgfpathmoveto{\pgfqpoint{3.246937in}{2.841358in}}%
\pgfpathlineto{\pgfqpoint{3.224977in}{2.840191in}}%
\pgfusepath{stroke}%
\end{pgfscope}%
\begin{pgfscope}%
\pgfpathrectangle{\pgfqpoint{1.250000in}{1.750000in}}{\pgfqpoint{2.279412in}{2.004545in}}%
\pgfusepath{clip}%
\pgfsetbuttcap%
\pgfsetroundjoin%
\pgfsetlinewidth{0.316338pt}%
\definecolor{currentstroke}{rgb}{0.269944,0.014625,0.341379}%
\pgfsetstrokecolor{currentstroke}%
\pgfsetdash{}{0pt}%
\pgfpathmoveto{\pgfqpoint{3.224977in}{2.840191in}}%
\pgfpathlineto{\pgfqpoint{3.174943in}{2.837755in}}%
\pgfusepath{stroke}%
\end{pgfscope}%
\begin{pgfscope}%
\pgfpathrectangle{\pgfqpoint{1.250000in}{1.750000in}}{\pgfqpoint{2.279412in}{2.004545in}}%
\pgfusepath{clip}%
\pgfsetbuttcap%
\pgfsetroundjoin%
\pgfsetlinewidth{0.321250pt}%
\definecolor{currentstroke}{rgb}{0.269944,0.014625,0.341379}%
\pgfsetstrokecolor{currentstroke}%
\pgfsetdash{}{0pt}%
\pgfpathmoveto{\pgfqpoint{3.174943in}{2.837755in}}%
\pgfpathlineto{\pgfqpoint{3.124818in}{2.836800in}}%
\pgfusepath{stroke}%
\end{pgfscope}%
\begin{pgfscope}%
\pgfpathrectangle{\pgfqpoint{1.250000in}{1.750000in}}{\pgfqpoint{2.279412in}{2.004545in}}%
\pgfusepath{clip}%
\pgfsetbuttcap%
\pgfsetroundjoin%
\pgfsetlinewidth{0.332243pt}%
\definecolor{currentstroke}{rgb}{0.272594,0.025563,0.353093}%
\pgfsetstrokecolor{currentstroke}%
\pgfsetdash{}{0pt}%
\pgfpathmoveto{\pgfqpoint{3.124818in}{2.836800in}}%
\pgfpathlineto{\pgfqpoint{3.074686in}{2.836038in}}%
\pgfusepath{stroke}%
\end{pgfscope}%
\begin{pgfscope}%
\pgfpathrectangle{\pgfqpoint{1.250000in}{1.750000in}}{\pgfqpoint{2.279412in}{2.004545in}}%
\pgfusepath{clip}%
\pgfsetbuttcap%
\pgfsetroundjoin%
\pgfsetlinewidth{0.335165pt}%
\definecolor{currentstroke}{rgb}{0.272594,0.025563,0.353093}%
\pgfsetstrokecolor{currentstroke}%
\pgfsetdash{}{0pt}%
\pgfpathmoveto{\pgfqpoint{3.074686in}{2.836038in}}%
\pgfpathlineto{\pgfqpoint{3.024546in}{2.835376in}}%
\pgfusepath{stroke}%
\end{pgfscope}%
\begin{pgfscope}%
\pgfpathrectangle{\pgfqpoint{1.250000in}{1.750000in}}{\pgfqpoint{2.279412in}{2.004545in}}%
\pgfusepath{clip}%
\pgfsetbuttcap%
\pgfsetroundjoin%
\pgfsetlinewidth{0.357648pt}%
\definecolor{currentstroke}{rgb}{0.277018,0.050344,0.375715}%
\pgfsetstrokecolor{currentstroke}%
\pgfsetdash{}{0pt}%
\pgfpathmoveto{\pgfqpoint{3.024546in}{2.835376in}}%
\pgfpathlineto{\pgfqpoint{2.974410in}{2.834395in}}%
\pgfusepath{stroke}%
\end{pgfscope}%
\begin{pgfscope}%
\pgfpathrectangle{\pgfqpoint{1.250000in}{1.750000in}}{\pgfqpoint{2.279412in}{2.004545in}}%
\pgfusepath{clip}%
\pgfsetbuttcap%
\pgfsetroundjoin%
\pgfsetlinewidth{0.376431pt}%
\definecolor{currentstroke}{rgb}{0.278791,0.062145,0.386592}%
\pgfsetstrokecolor{currentstroke}%
\pgfsetdash{}{0pt}%
\pgfpathmoveto{\pgfqpoint{2.974410in}{2.834395in}}%
\pgfpathlineto{\pgfqpoint{2.924269in}{2.833590in}}%
\pgfusepath{stroke}%
\end{pgfscope}%
\begin{pgfscope}%
\pgfpathrectangle{\pgfqpoint{1.250000in}{1.750000in}}{\pgfqpoint{2.279412in}{2.004545in}}%
\pgfusepath{clip}%
\pgfsetbuttcap%
\pgfsetroundjoin%
\pgfsetlinewidth{0.418115pt}%
\definecolor{currentstroke}{rgb}{0.282656,0.100196,0.422160}%
\pgfsetstrokecolor{currentstroke}%
\pgfsetdash{}{0pt}%
\pgfpathmoveto{\pgfqpoint{2.924269in}{2.833590in}}%
\pgfpathlineto{\pgfqpoint{2.874127in}{2.832762in}}%
\pgfusepath{stroke}%
\end{pgfscope}%
\begin{pgfscope}%
\pgfpathrectangle{\pgfqpoint{1.250000in}{1.750000in}}{\pgfqpoint{2.279412in}{2.004545in}}%
\pgfusepath{clip}%
\pgfsetbuttcap%
\pgfsetroundjoin%
\pgfsetlinewidth{0.482349pt}%
\definecolor{currentstroke}{rgb}{0.282290,0.145912,0.461510}%
\pgfsetstrokecolor{currentstroke}%
\pgfsetdash{}{0pt}%
\pgfpathmoveto{\pgfqpoint{2.874127in}{2.832762in}}%
\pgfpathlineto{\pgfqpoint{2.823988in}{2.831767in}}%
\pgfusepath{stroke}%
\end{pgfscope}%
\begin{pgfscope}%
\pgfpathrectangle{\pgfqpoint{1.250000in}{1.750000in}}{\pgfqpoint{2.279412in}{2.004545in}}%
\pgfusepath{clip}%
\pgfsetbuttcap%
\pgfsetroundjoin%
\pgfsetlinewidth{0.558793pt}%
\definecolor{currentstroke}{rgb}{0.274128,0.199721,0.498911}%
\pgfsetstrokecolor{currentstroke}%
\pgfsetdash{}{0pt}%
\pgfpathmoveto{\pgfqpoint{2.823988in}{2.831767in}}%
\pgfpathlineto{\pgfqpoint{2.773850in}{2.830732in}}%
\pgfusepath{stroke}%
\end{pgfscope}%
\begin{pgfscope}%
\pgfpathrectangle{\pgfqpoint{1.250000in}{1.750000in}}{\pgfqpoint{2.279412in}{2.004545in}}%
\pgfusepath{clip}%
\pgfsetbuttcap%
\pgfsetroundjoin%
\pgfsetlinewidth{0.710282pt}%
\definecolor{currentstroke}{rgb}{0.239346,0.300855,0.540844}%
\pgfsetstrokecolor{currentstroke}%
\pgfsetdash{}{0pt}%
\pgfpathmoveto{\pgfqpoint{2.773850in}{2.830732in}}%
\pgfpathlineto{\pgfqpoint{2.723717in}{2.829542in}}%
\pgfusepath{stroke}%
\end{pgfscope}%
\begin{pgfscope}%
\pgfpathrectangle{\pgfqpoint{1.250000in}{1.750000in}}{\pgfqpoint{2.279412in}{2.004545in}}%
\pgfusepath{clip}%
\pgfsetbuttcap%
\pgfsetroundjoin%
\pgfsetlinewidth{0.899260pt}%
\definecolor{currentstroke}{rgb}{0.187231,0.414746,0.556547}%
\pgfsetstrokecolor{currentstroke}%
\pgfsetdash{}{0pt}%
\pgfpathmoveto{\pgfqpoint{2.723717in}{2.829542in}}%
\pgfpathlineto{\pgfqpoint{2.673602in}{2.827900in}}%
\pgfusepath{stroke}%
\end{pgfscope}%
\begin{pgfscope}%
\pgfpathrectangle{\pgfqpoint{1.250000in}{1.750000in}}{\pgfqpoint{2.279412in}{2.004545in}}%
\pgfusepath{clip}%
\pgfsetbuttcap%
\pgfsetroundjoin%
\pgfsetlinewidth{1.204249pt}%
\definecolor{currentstroke}{rgb}{0.124395,0.578002,0.548287}%
\pgfsetstrokecolor{currentstroke}%
\pgfsetdash{}{0pt}%
\pgfpathmoveto{\pgfqpoint{2.673602in}{2.827900in}}%
\pgfpathlineto{\pgfqpoint{2.623520in}{2.825612in}}%
\pgfusepath{stroke}%
\end{pgfscope}%
\begin{pgfscope}%
\pgfpathrectangle{\pgfqpoint{1.250000in}{1.750000in}}{\pgfqpoint{2.279412in}{2.004545in}}%
\pgfusepath{clip}%
\pgfsetbuttcap%
\pgfsetroundjoin%
\pgfsetlinewidth{1.410375pt}%
\definecolor{currentstroke}{rgb}{0.162016,0.687316,0.499129}%
\pgfsetstrokecolor{currentstroke}%
\pgfsetdash{}{0pt}%
\pgfpathmoveto{\pgfqpoint{2.623520in}{2.825612in}}%
\pgfpathlineto{\pgfqpoint{2.573469in}{2.822846in}}%
\pgfusepath{stroke}%
\end{pgfscope}%
\begin{pgfscope}%
\pgfpathrectangle{\pgfqpoint{1.250000in}{1.750000in}}{\pgfqpoint{2.279412in}{2.004545in}}%
\pgfusepath{clip}%
\pgfsetbuttcap%
\pgfsetroundjoin%
\pgfsetlinewidth{1.600860pt}%
\definecolor{currentstroke}{rgb}{0.344074,0.780029,0.397381}%
\pgfsetstrokecolor{currentstroke}%
\pgfsetdash{}{0pt}%
\pgfpathmoveto{\pgfqpoint{2.573469in}{2.822846in}}%
\pgfpathlineto{\pgfqpoint{2.523460in}{2.819576in}}%
\pgfusepath{stroke}%
\end{pgfscope}%
\begin{pgfscope}%
\pgfpathrectangle{\pgfqpoint{1.250000in}{1.750000in}}{\pgfqpoint{2.279412in}{2.004545in}}%
\pgfusepath{clip}%
\pgfsetbuttcap%
\pgfsetroundjoin%
\pgfsetlinewidth{1.713784pt}%
\definecolor{currentstroke}{rgb}{0.487026,0.823929,0.312321}%
\pgfsetstrokecolor{currentstroke}%
\pgfsetdash{}{0pt}%
\pgfpathmoveto{\pgfqpoint{2.523460in}{2.819576in}}%
\pgfpathlineto{\pgfqpoint{2.473500in}{2.815755in}}%
\pgfusepath{stroke}%
\end{pgfscope}%
\begin{pgfscope}%
\pgfpathrectangle{\pgfqpoint{1.250000in}{1.750000in}}{\pgfqpoint{2.279412in}{2.004545in}}%
\pgfusepath{clip}%
\pgfsetbuttcap%
\pgfsetroundjoin%
\pgfsetlinewidth{1.979392pt}%
\definecolor{currentstroke}{rgb}{0.886271,0.892374,0.095374}%
\pgfsetstrokecolor{currentstroke}%
\pgfsetdash{}{0pt}%
\pgfpathmoveto{\pgfqpoint{2.473500in}{2.815755in}}%
\pgfpathlineto{\pgfqpoint{2.423587in}{2.811482in}}%
\pgfusepath{stroke}%
\end{pgfscope}%
\begin{pgfscope}%
\pgfpathrectangle{\pgfqpoint{1.250000in}{1.750000in}}{\pgfqpoint{2.279412in}{2.004545in}}%
\pgfusepath{clip}%
\pgfsetbuttcap%
\pgfsetroundjoin%
\pgfsetlinewidth{2.151564pt}%
\definecolor{currentstroke}{rgb}{0.993248,0.906157,0.143936}%
\pgfsetstrokecolor{currentstroke}%
\pgfsetdash{}{0pt}%
\pgfpathmoveto{\pgfqpoint{2.423587in}{2.811482in}}%
\pgfpathlineto{\pgfqpoint{2.373719in}{2.806821in}}%
\pgfusepath{stroke}%
\end{pgfscope}%
\begin{pgfscope}%
\pgfpathrectangle{\pgfqpoint{1.250000in}{1.750000in}}{\pgfqpoint{2.279412in}{2.004545in}}%
\pgfusepath{clip}%
\pgfsetbuttcap%
\pgfsetroundjoin%
\pgfsetlinewidth{2.076310pt}%
\definecolor{currentstroke}{rgb}{0.993248,0.906157,0.143936}%
\pgfsetstrokecolor{currentstroke}%
\pgfsetdash{}{0pt}%
\pgfpathmoveto{\pgfqpoint{2.373719in}{2.806821in}}%
\pgfpathlineto{\pgfqpoint{2.323909in}{2.801751in}}%
\pgfusepath{stroke}%
\end{pgfscope}%
\begin{pgfscope}%
\pgfpathrectangle{\pgfqpoint{1.250000in}{1.750000in}}{\pgfqpoint{2.279412in}{2.004545in}}%
\pgfusepath{clip}%
\pgfsetbuttcap%
\pgfsetroundjoin%
\pgfsetlinewidth{2.153260pt}%
\definecolor{currentstroke}{rgb}{0.993248,0.906157,0.143936}%
\pgfsetstrokecolor{currentstroke}%
\pgfsetdash{}{0pt}%
\pgfpathmoveto{\pgfqpoint{2.323909in}{2.801751in}}%
\pgfpathlineto{\pgfqpoint{2.274154in}{2.796284in}}%
\pgfusepath{stroke}%
\end{pgfscope}%
\begin{pgfscope}%
\pgfpathrectangle{\pgfqpoint{1.250000in}{1.750000in}}{\pgfqpoint{2.279412in}{2.004545in}}%
\pgfusepath{clip}%
\pgfsetbuttcap%
\pgfsetroundjoin%
\pgfsetlinewidth{2.182352pt}%
\definecolor{currentstroke}{rgb}{0.993248,0.906157,0.143936}%
\pgfsetstrokecolor{currentstroke}%
\pgfsetdash{}{0pt}%
\pgfpathmoveto{\pgfqpoint{2.274154in}{2.796284in}}%
\pgfpathlineto{\pgfqpoint{2.224461in}{2.790396in}}%
\pgfusepath{stroke}%
\end{pgfscope}%
\begin{pgfscope}%
\pgfpathrectangle{\pgfqpoint{1.250000in}{1.750000in}}{\pgfqpoint{2.279412in}{2.004545in}}%
\pgfusepath{clip}%
\pgfsetbuttcap%
\pgfsetroundjoin%
\pgfsetlinewidth{1.991862pt}%
\definecolor{currentstroke}{rgb}{0.906311,0.894855,0.098125}%
\pgfsetstrokecolor{currentstroke}%
\pgfsetdash{}{0pt}%
\pgfpathmoveto{\pgfqpoint{2.224461in}{2.790396in}}%
\pgfpathlineto{\pgfqpoint{2.174822in}{2.784232in}}%
\pgfusepath{stroke}%
\end{pgfscope}%
\begin{pgfscope}%
\pgfpathrectangle{\pgfqpoint{1.250000in}{1.750000in}}{\pgfqpoint{2.279412in}{2.004545in}}%
\pgfusepath{clip}%
\pgfsetbuttcap%
\pgfsetroundjoin%
\pgfsetlinewidth{1.925530pt}%
\definecolor{currentstroke}{rgb}{0.804182,0.882046,0.114965}%
\pgfsetstrokecolor{currentstroke}%
\pgfsetdash{}{0pt}%
\pgfpathmoveto{\pgfqpoint{2.174822in}{2.784232in}}%
\pgfpathlineto{\pgfqpoint{2.125238in}{2.777843in}}%
\pgfusepath{stroke}%
\end{pgfscope}%
\begin{pgfscope}%
\pgfpathrectangle{\pgfqpoint{1.250000in}{1.750000in}}{\pgfqpoint{2.279412in}{2.004545in}}%
\pgfusepath{clip}%
\pgfsetbuttcap%
\pgfsetroundjoin%
\pgfsetlinewidth{1.698256pt}%
\definecolor{currentstroke}{rgb}{0.468053,0.818921,0.323998}%
\pgfsetstrokecolor{currentstroke}%
\pgfsetdash{}{0pt}%
\pgfpathmoveto{\pgfqpoint{2.125238in}{2.777843in}}%
\pgfpathlineto{\pgfqpoint{2.075697in}{2.771142in}}%
\pgfusepath{stroke}%
\end{pgfscope}%
\begin{pgfscope}%
\pgfpathrectangle{\pgfqpoint{1.250000in}{1.750000in}}{\pgfqpoint{2.279412in}{2.004545in}}%
\pgfusepath{clip}%
\pgfsetbuttcap%
\pgfsetroundjoin%
\pgfsetlinewidth{1.499752pt}%
\definecolor{currentstroke}{rgb}{0.232815,0.732247,0.459277}%
\pgfsetstrokecolor{currentstroke}%
\pgfsetdash{}{0pt}%
\pgfpathmoveto{\pgfqpoint{2.075697in}{2.771142in}}%
\pgfpathlineto{\pgfqpoint{2.026230in}{2.764312in}}%
\pgfusepath{stroke}%
\end{pgfscope}%
\begin{pgfscope}%
\pgfpathrectangle{\pgfqpoint{1.250000in}{1.750000in}}{\pgfqpoint{2.279412in}{2.004545in}}%
\pgfusepath{clip}%
\pgfsetbuttcap%
\pgfsetroundjoin%
\pgfsetlinewidth{1.135869pt}%
\definecolor{currentstroke}{rgb}{0.136408,0.541173,0.554483}%
\pgfsetstrokecolor{currentstroke}%
\pgfsetdash{}{0pt}%
\pgfpathmoveto{\pgfqpoint{2.026230in}{2.764312in}}%
\pgfpathlineto{\pgfqpoint{1.976903in}{2.757656in}}%
\pgfusepath{stroke}%
\end{pgfscope}%
\begin{pgfscope}%
\pgfpathrectangle{\pgfqpoint{1.250000in}{1.750000in}}{\pgfqpoint{2.279412in}{2.004545in}}%
\pgfusepath{clip}%
\pgfsetbuttcap%
\pgfsetroundjoin%
\pgfsetlinewidth{1.240380pt}%
\definecolor{currentstroke}{rgb}{0.120565,0.596422,0.543611}%
\pgfsetstrokecolor{currentstroke}%
\pgfsetdash{}{0pt}%
\pgfpathmoveto{\pgfqpoint{1.976903in}{2.757656in}}%
\pgfpathlineto{\pgfqpoint{1.927546in}{2.751125in}}%
\pgfusepath{stroke}%
\end{pgfscope}%
\begin{pgfscope}%
\pgfpathrectangle{\pgfqpoint{1.250000in}{1.750000in}}{\pgfqpoint{2.279412in}{2.004545in}}%
\pgfusepath{clip}%
\pgfsetbuttcap%
\pgfsetroundjoin%
\pgfsetlinewidth{0.315376pt}%
\definecolor{currentstroke}{rgb}{0.269944,0.014625,0.341379}%
\pgfsetstrokecolor{currentstroke}%
\pgfsetdash{}{0pt}%
\pgfpathmoveto{\pgfqpoint{3.261668in}{2.887593in}}%
\pgfpathlineto{\pgfqpoint{3.261668in}{2.887593in}}%
\pgfusepath{stroke}%
\end{pgfscope}%
\begin{pgfscope}%
\pgfpathrectangle{\pgfqpoint{1.250000in}{1.750000in}}{\pgfqpoint{2.279412in}{2.004545in}}%
\pgfusepath{clip}%
\pgfsetbuttcap%
\pgfsetroundjoin%
\pgfsetlinewidth{0.315376pt}%
\definecolor{currentstroke}{rgb}{0.269944,0.014625,0.341379}%
\pgfsetstrokecolor{currentstroke}%
\pgfsetdash{}{0pt}%
\pgfpathmoveto{\pgfqpoint{3.261668in}{2.887593in}}%
\pgfpathlineto{\pgfqpoint{3.261668in}{2.887593in}}%
\pgfusepath{stroke}%
\end{pgfscope}%
\begin{pgfscope}%
\pgfpathrectangle{\pgfqpoint{1.250000in}{1.750000in}}{\pgfqpoint{2.279412in}{2.004545in}}%
\pgfusepath{clip}%
\pgfsetbuttcap%
\pgfsetroundjoin%
\pgfsetlinewidth{0.315376pt}%
\definecolor{currentstroke}{rgb}{0.269944,0.014625,0.341379}%
\pgfsetstrokecolor{currentstroke}%
\pgfsetdash{}{0pt}%
\pgfpathmoveto{\pgfqpoint{3.261668in}{2.887593in}}%
\pgfpathlineto{\pgfqpoint{3.247445in}{2.887001in}}%
\pgfusepath{stroke}%
\end{pgfscope}%
\begin{pgfscope}%
\pgfpathrectangle{\pgfqpoint{1.250000in}{1.750000in}}{\pgfqpoint{2.279412in}{2.004545in}}%
\pgfusepath{clip}%
\pgfsetbuttcap%
\pgfsetroundjoin%
\pgfsetlinewidth{0.324049pt}%
\definecolor{currentstroke}{rgb}{0.271305,0.019942,0.347269}%
\pgfsetstrokecolor{currentstroke}%
\pgfsetdash{}{0pt}%
\pgfpathmoveto{\pgfqpoint{3.247445in}{2.887001in}}%
\pgfpathlineto{\pgfqpoint{3.225216in}{2.886755in}}%
\pgfusepath{stroke}%
\end{pgfscope}%
\begin{pgfscope}%
\pgfpathrectangle{\pgfqpoint{1.250000in}{1.750000in}}{\pgfqpoint{2.279412in}{2.004545in}}%
\pgfusepath{clip}%
\pgfsetbuttcap%
\pgfsetroundjoin%
\pgfsetlinewidth{0.316195pt}%
\definecolor{currentstroke}{rgb}{0.269944,0.014625,0.341379}%
\pgfsetstrokecolor{currentstroke}%
\pgfsetdash{}{0pt}%
\pgfpathmoveto{\pgfqpoint{3.225216in}{2.886755in}}%
\pgfpathlineto{\pgfqpoint{3.178477in}{2.887528in}}%
\pgfusepath{stroke}%
\end{pgfscope}%
\begin{pgfscope}%
\pgfpathrectangle{\pgfqpoint{1.250000in}{1.750000in}}{\pgfqpoint{2.279412in}{2.004545in}}%
\pgfusepath{clip}%
\pgfsetbuttcap%
\pgfsetroundjoin%
\pgfsetlinewidth{0.305500pt}%
\definecolor{currentstroke}{rgb}{0.267004,0.004874,0.329415}%
\pgfsetstrokecolor{currentstroke}%
\pgfsetdash{}{0pt}%
\pgfpathmoveto{\pgfqpoint{3.178477in}{2.887528in}}%
\pgfpathlineto{\pgfqpoint{3.133098in}{2.886789in}}%
\pgfusepath{stroke}%
\end{pgfscope}%
\begin{pgfscope}%
\pgfpathrectangle{\pgfqpoint{1.250000in}{1.750000in}}{\pgfqpoint{2.279412in}{2.004545in}}%
\pgfusepath{clip}%
\pgfsetbuttcap%
\pgfsetroundjoin%
\pgfsetlinewidth{0.317350pt}%
\definecolor{currentstroke}{rgb}{0.269944,0.014625,0.341379}%
\pgfsetstrokecolor{currentstroke}%
\pgfsetdash{}{0pt}%
\pgfpathmoveto{\pgfqpoint{3.133098in}{2.886789in}}%
\pgfpathlineto{\pgfqpoint{3.083327in}{2.885373in}}%
\pgfusepath{stroke}%
\end{pgfscope}%
\begin{pgfscope}%
\pgfpathrectangle{\pgfqpoint{1.250000in}{1.750000in}}{\pgfqpoint{2.279412in}{2.004545in}}%
\pgfusepath{clip}%
\pgfsetbuttcap%
\pgfsetroundjoin%
\pgfsetlinewidth{0.336222pt}%
\definecolor{currentstroke}{rgb}{0.273809,0.031497,0.358853}%
\pgfsetstrokecolor{currentstroke}%
\pgfsetdash{}{0pt}%
\pgfpathmoveto{\pgfqpoint{3.083327in}{2.885373in}}%
\pgfpathlineto{\pgfqpoint{3.033188in}{2.884740in}}%
\pgfusepath{stroke}%
\end{pgfscope}%
\begin{pgfscope}%
\pgfpathrectangle{\pgfqpoint{1.250000in}{1.750000in}}{\pgfqpoint{2.279412in}{2.004545in}}%
\pgfusepath{clip}%
\pgfsetbuttcap%
\pgfsetroundjoin%
\pgfsetlinewidth{0.349370pt}%
\definecolor{currentstroke}{rgb}{0.276022,0.044167,0.370164}%
\pgfsetstrokecolor{currentstroke}%
\pgfsetdash{}{0pt}%
\pgfpathmoveto{\pgfqpoint{3.033188in}{2.884740in}}%
\pgfpathlineto{\pgfqpoint{2.983056in}{2.883681in}}%
\pgfusepath{stroke}%
\end{pgfscope}%
\begin{pgfscope}%
\pgfpathrectangle{\pgfqpoint{1.250000in}{1.750000in}}{\pgfqpoint{2.279412in}{2.004545in}}%
\pgfusepath{clip}%
\pgfsetbuttcap%
\pgfsetroundjoin%
\pgfsetlinewidth{0.367987pt}%
\definecolor{currentstroke}{rgb}{0.277941,0.056324,0.381191}%
\pgfsetstrokecolor{currentstroke}%
\pgfsetdash{}{0pt}%
\pgfpathmoveto{\pgfqpoint{2.983056in}{2.883681in}}%
\pgfpathlineto{\pgfqpoint{2.932925in}{2.882526in}}%
\pgfusepath{stroke}%
\end{pgfscope}%
\begin{pgfscope}%
\pgfpathrectangle{\pgfqpoint{1.250000in}{1.750000in}}{\pgfqpoint{2.279412in}{2.004545in}}%
\pgfusepath{clip}%
\pgfsetbuttcap%
\pgfsetroundjoin%
\pgfsetlinewidth{0.394733pt}%
\definecolor{currentstroke}{rgb}{0.280894,0.078907,0.402329}%
\pgfsetstrokecolor{currentstroke}%
\pgfsetdash{}{0pt}%
\pgfpathmoveto{\pgfqpoint{2.932925in}{2.882526in}}%
\pgfpathlineto{\pgfqpoint{2.882786in}{2.881606in}}%
\pgfusepath{stroke}%
\end{pgfscope}%
\begin{pgfscope}%
\pgfpathrectangle{\pgfqpoint{1.250000in}{1.750000in}}{\pgfqpoint{2.279412in}{2.004545in}}%
\pgfusepath{clip}%
\pgfsetbuttcap%
\pgfsetroundjoin%
\pgfsetlinewidth{0.453713pt}%
\definecolor{currentstroke}{rgb}{0.283187,0.125848,0.444960}%
\pgfsetstrokecolor{currentstroke}%
\pgfsetdash{}{0pt}%
\pgfpathmoveto{\pgfqpoint{2.882786in}{2.881606in}}%
\pgfpathlineto{\pgfqpoint{2.832655in}{2.880387in}}%
\pgfusepath{stroke}%
\end{pgfscope}%
\begin{pgfscope}%
\pgfpathrectangle{\pgfqpoint{1.250000in}{1.750000in}}{\pgfqpoint{2.279412in}{2.004545in}}%
\pgfusepath{clip}%
\pgfsetbuttcap%
\pgfsetroundjoin%
\pgfsetlinewidth{0.510378pt}%
\definecolor{currentstroke}{rgb}{0.280255,0.165693,0.476498}%
\pgfsetstrokecolor{currentstroke}%
\pgfsetdash{}{0pt}%
\pgfpathmoveto{\pgfqpoint{2.832655in}{2.880387in}}%
\pgfpathlineto{\pgfqpoint{2.782535in}{2.878836in}}%
\pgfusepath{stroke}%
\end{pgfscope}%
\begin{pgfscope}%
\pgfpathrectangle{\pgfqpoint{1.250000in}{1.750000in}}{\pgfqpoint{2.279412in}{2.004545in}}%
\pgfusepath{clip}%
\pgfsetbuttcap%
\pgfsetroundjoin%
\pgfsetlinewidth{0.612056pt}%
\definecolor{currentstroke}{rgb}{0.263663,0.237631,0.518762}%
\pgfsetstrokecolor{currentstroke}%
\pgfsetdash{}{0pt}%
\pgfpathmoveto{\pgfqpoint{2.782535in}{2.878836in}}%
\pgfpathlineto{\pgfqpoint{2.732430in}{2.876947in}}%
\pgfusepath{stroke}%
\end{pgfscope}%
\begin{pgfscope}%
\pgfpathrectangle{\pgfqpoint{1.250000in}{1.750000in}}{\pgfqpoint{2.279412in}{2.004545in}}%
\pgfusepath{clip}%
\pgfsetbuttcap%
\pgfsetroundjoin%
\pgfsetlinewidth{0.777335pt}%
\definecolor{currentstroke}{rgb}{0.220057,0.343307,0.549413}%
\pgfsetstrokecolor{currentstroke}%
\pgfsetdash{}{0pt}%
\pgfpathmoveto{\pgfqpoint{2.732430in}{2.876947in}}%
\pgfpathlineto{\pgfqpoint{2.682360in}{2.874459in}}%
\pgfusepath{stroke}%
\end{pgfscope}%
\begin{pgfscope}%
\pgfpathrectangle{\pgfqpoint{1.250000in}{1.750000in}}{\pgfqpoint{2.279412in}{2.004545in}}%
\pgfusepath{clip}%
\pgfsetbuttcap%
\pgfsetroundjoin%
\pgfsetlinewidth{0.948281pt}%
\definecolor{currentstroke}{rgb}{0.175841,0.441290,0.557685}%
\pgfsetstrokecolor{currentstroke}%
\pgfsetdash{}{0pt}%
\pgfpathmoveto{\pgfqpoint{2.682360in}{2.874459in}}%
\pgfpathlineto{\pgfqpoint{2.632341in}{2.871266in}}%
\pgfusepath{stroke}%
\end{pgfscope}%
\begin{pgfscope}%
\pgfpathrectangle{\pgfqpoint{1.250000in}{1.750000in}}{\pgfqpoint{2.279412in}{2.004545in}}%
\pgfusepath{clip}%
\pgfsetbuttcap%
\pgfsetroundjoin%
\pgfsetlinewidth{1.131860pt}%
\definecolor{currentstroke}{rgb}{0.136408,0.541173,0.554483}%
\pgfsetstrokecolor{currentstroke}%
\pgfsetdash{}{0pt}%
\pgfpathmoveto{\pgfqpoint{2.632341in}{2.871266in}}%
\pgfpathlineto{\pgfqpoint{2.582394in}{2.867313in}}%
\pgfusepath{stroke}%
\end{pgfscope}%
\begin{pgfscope}%
\pgfpathrectangle{\pgfqpoint{1.250000in}{1.750000in}}{\pgfqpoint{2.279412in}{2.004545in}}%
\pgfusepath{clip}%
\pgfsetbuttcap%
\pgfsetroundjoin%
\pgfsetlinewidth{1.373917pt}%
\definecolor{currentstroke}{rgb}{0.143303,0.669459,0.511215}%
\pgfsetstrokecolor{currentstroke}%
\pgfsetdash{}{0pt}%
\pgfpathmoveto{\pgfqpoint{2.582394in}{2.867313in}}%
\pgfpathlineto{\pgfqpoint{2.532537in}{2.862580in}}%
\pgfusepath{stroke}%
\end{pgfscope}%
\begin{pgfscope}%
\pgfpathrectangle{\pgfqpoint{1.250000in}{1.750000in}}{\pgfqpoint{2.279412in}{2.004545in}}%
\pgfusepath{clip}%
\pgfsetbuttcap%
\pgfsetroundjoin%
\pgfsetlinewidth{1.450399pt}%
\definecolor{currentstroke}{rgb}{0.191090,0.708366,0.482284}%
\pgfsetstrokecolor{currentstroke}%
\pgfsetdash{}{0pt}%
\pgfpathmoveto{\pgfqpoint{2.532537in}{2.862580in}}%
\pgfpathlineto{\pgfqpoint{2.482781in}{2.857084in}}%
\pgfusepath{stroke}%
\end{pgfscope}%
\begin{pgfscope}%
\pgfpathrectangle{\pgfqpoint{1.250000in}{1.750000in}}{\pgfqpoint{2.279412in}{2.004545in}}%
\pgfusepath{clip}%
\pgfsetbuttcap%
\pgfsetroundjoin%
\pgfsetlinewidth{1.544243pt}%
\definecolor{currentstroke}{rgb}{0.281477,0.755203,0.432552}%
\pgfsetstrokecolor{currentstroke}%
\pgfsetdash{}{0pt}%
\pgfpathmoveto{\pgfqpoint{2.482781in}{2.857084in}}%
\pgfpathlineto{\pgfqpoint{2.433158in}{2.850723in}}%
\pgfusepath{stroke}%
\end{pgfscope}%
\begin{pgfscope}%
\pgfpathrectangle{\pgfqpoint{1.250000in}{1.750000in}}{\pgfqpoint{2.279412in}{2.004545in}}%
\pgfusepath{clip}%
\pgfsetbuttcap%
\pgfsetroundjoin%
\pgfsetlinewidth{1.871190pt}%
\definecolor{currentstroke}{rgb}{0.720391,0.870350,0.162603}%
\pgfsetstrokecolor{currentstroke}%
\pgfsetdash{}{0pt}%
\pgfpathmoveto{\pgfqpoint{2.433158in}{2.850723in}}%
\pgfpathlineto{\pgfqpoint{2.383682in}{2.843542in}}%
\pgfusepath{stroke}%
\end{pgfscope}%
\begin{pgfscope}%
\pgfpathrectangle{\pgfqpoint{1.250000in}{1.750000in}}{\pgfqpoint{2.279412in}{2.004545in}}%
\pgfusepath{clip}%
\pgfsetbuttcap%
\pgfsetroundjoin%
\pgfsetlinewidth{1.942274pt}%
\definecolor{currentstroke}{rgb}{0.835270,0.886029,0.102646}%
\pgfsetstrokecolor{currentstroke}%
\pgfsetdash{}{0pt}%
\pgfpathmoveto{\pgfqpoint{2.383682in}{2.843542in}}%
\pgfpathlineto{\pgfqpoint{2.334364in}{2.835579in}}%
\pgfusepath{stroke}%
\end{pgfscope}%
\begin{pgfscope}%
\pgfpathrectangle{\pgfqpoint{1.250000in}{1.750000in}}{\pgfqpoint{2.279412in}{2.004545in}}%
\pgfusepath{clip}%
\pgfsetbuttcap%
\pgfsetroundjoin%
\pgfsetlinewidth{1.903922pt}%
\definecolor{currentstroke}{rgb}{0.772852,0.877868,0.131109}%
\pgfsetstrokecolor{currentstroke}%
\pgfsetdash{}{0pt}%
\pgfpathmoveto{\pgfqpoint{2.334364in}{2.835579in}}%
\pgfpathlineto{\pgfqpoint{2.285157in}{2.827103in}}%
\pgfusepath{stroke}%
\end{pgfscope}%
\begin{pgfscope}%
\pgfpathrectangle{\pgfqpoint{1.250000in}{1.750000in}}{\pgfqpoint{2.279412in}{2.004545in}}%
\pgfusepath{clip}%
\pgfsetbuttcap%
\pgfsetroundjoin%
\pgfsetlinewidth{0.323460pt}%
\definecolor{currentstroke}{rgb}{0.271305,0.019942,0.347269}%
\pgfsetstrokecolor{currentstroke}%
\pgfsetdash{}{0pt}%
\pgfpathmoveto{\pgfqpoint{3.210376in}{2.481632in}}%
\pgfpathlineto{\pgfqpoint{3.160243in}{2.482569in}}%
\pgfusepath{stroke}%
\end{pgfscope}%
\begin{pgfscope}%
\pgfpathrectangle{\pgfqpoint{1.250000in}{1.750000in}}{\pgfqpoint{2.279412in}{2.004545in}}%
\pgfusepath{clip}%
\pgfsetbuttcap%
\pgfsetroundjoin%
\pgfsetlinewidth{0.323893pt}%
\definecolor{currentstroke}{rgb}{0.271305,0.019942,0.347269}%
\pgfsetstrokecolor{currentstroke}%
\pgfsetdash{}{0pt}%
\pgfpathmoveto{\pgfqpoint{3.160243in}{2.482569in}}%
\pgfpathlineto{\pgfqpoint{3.110116in}{2.482864in}}%
\pgfusepath{stroke}%
\end{pgfscope}%
\begin{pgfscope}%
\pgfpathrectangle{\pgfqpoint{1.250000in}{1.750000in}}{\pgfqpoint{2.279412in}{2.004545in}}%
\pgfusepath{clip}%
\pgfsetbuttcap%
\pgfsetroundjoin%
\pgfsetlinewidth{0.316869pt}%
\definecolor{currentstroke}{rgb}{0.269944,0.014625,0.341379}%
\pgfsetstrokecolor{currentstroke}%
\pgfsetdash{}{0pt}%
\pgfpathmoveto{\pgfqpoint{3.110116in}{2.482864in}}%
\pgfpathlineto{\pgfqpoint{3.060068in}{2.484435in}}%
\pgfusepath{stroke}%
\end{pgfscope}%
\begin{pgfscope}%
\pgfpathrectangle{\pgfqpoint{1.250000in}{1.750000in}}{\pgfqpoint{2.279412in}{2.004545in}}%
\pgfusepath{clip}%
\pgfsetbuttcap%
\pgfsetroundjoin%
\pgfsetlinewidth{0.327763pt}%
\definecolor{currentstroke}{rgb}{0.271305,0.019942,0.347269}%
\pgfsetstrokecolor{currentstroke}%
\pgfsetdash{}{0pt}%
\pgfpathmoveto{\pgfqpoint{3.060068in}{2.484435in}}%
\pgfpathlineto{\pgfqpoint{3.010019in}{2.486817in}}%
\pgfusepath{stroke}%
\end{pgfscope}%
\begin{pgfscope}%
\pgfpathrectangle{\pgfqpoint{1.250000in}{1.750000in}}{\pgfqpoint{2.279412in}{2.004545in}}%
\pgfusepath{clip}%
\pgfsetbuttcap%
\pgfsetroundjoin%
\pgfsetlinewidth{0.343346pt}%
\definecolor{currentstroke}{rgb}{0.274952,0.037752,0.364543}%
\pgfsetstrokecolor{currentstroke}%
\pgfsetdash{}{0pt}%
\pgfpathmoveto{\pgfqpoint{3.010019in}{2.486817in}}%
\pgfpathlineto{\pgfqpoint{2.959917in}{2.488788in}}%
\pgfusepath{stroke}%
\end{pgfscope}%
\begin{pgfscope}%
\pgfpathrectangle{\pgfqpoint{1.250000in}{1.750000in}}{\pgfqpoint{2.279412in}{2.004545in}}%
\pgfusepath{clip}%
\pgfsetbuttcap%
\pgfsetroundjoin%
\pgfsetlinewidth{0.366440pt}%
\definecolor{currentstroke}{rgb}{0.277941,0.056324,0.381191}%
\pgfsetstrokecolor{currentstroke}%
\pgfsetdash{}{0pt}%
\pgfpathmoveto{\pgfqpoint{2.959917in}{2.488788in}}%
\pgfpathlineto{\pgfqpoint{2.909846in}{2.491249in}}%
\pgfusepath{stroke}%
\end{pgfscope}%
\begin{pgfscope}%
\pgfpathrectangle{\pgfqpoint{1.250000in}{1.750000in}}{\pgfqpoint{2.279412in}{2.004545in}}%
\pgfusepath{clip}%
\pgfsetbuttcap%
\pgfsetroundjoin%
\pgfsetlinewidth{0.403134pt}%
\definecolor{currentstroke}{rgb}{0.281446,0.084320,0.407414}%
\pgfsetstrokecolor{currentstroke}%
\pgfsetdash{}{0pt}%
\pgfpathmoveto{\pgfqpoint{2.909846in}{2.491249in}}%
\pgfpathlineto{\pgfqpoint{2.859777in}{2.493754in}}%
\pgfusepath{stroke}%
\end{pgfscope}%
\begin{pgfscope}%
\pgfpathrectangle{\pgfqpoint{1.250000in}{1.750000in}}{\pgfqpoint{2.279412in}{2.004545in}}%
\pgfusepath{clip}%
\pgfsetbuttcap%
\pgfsetroundjoin%
\pgfsetlinewidth{0.411007pt}%
\definecolor{currentstroke}{rgb}{0.282327,0.094955,0.417331}%
\pgfsetstrokecolor{currentstroke}%
\pgfsetdash{}{0pt}%
\pgfpathmoveto{\pgfqpoint{2.859777in}{2.493754in}}%
\pgfpathlineto{\pgfqpoint{2.809704in}{2.496188in}}%
\pgfusepath{stroke}%
\end{pgfscope}%
\begin{pgfscope}%
\pgfpathrectangle{\pgfqpoint{1.250000in}{1.750000in}}{\pgfqpoint{2.279412in}{2.004545in}}%
\pgfusepath{clip}%
\pgfsetbuttcap%
\pgfsetroundjoin%
\pgfsetlinewidth{0.469112pt}%
\definecolor{currentstroke}{rgb}{0.282884,0.135920,0.453427}%
\pgfsetstrokecolor{currentstroke}%
\pgfsetdash{}{0pt}%
\pgfpathmoveto{\pgfqpoint{2.809704in}{2.496188in}}%
\pgfpathlineto{\pgfqpoint{2.759671in}{2.499205in}}%
\pgfusepath{stroke}%
\end{pgfscope}%
\begin{pgfscope}%
\pgfpathrectangle{\pgfqpoint{1.250000in}{1.750000in}}{\pgfqpoint{2.279412in}{2.004545in}}%
\pgfusepath{clip}%
\pgfsetbuttcap%
\pgfsetroundjoin%
\pgfsetlinewidth{0.502734pt}%
\definecolor{currentstroke}{rgb}{0.280868,0.160771,0.472899}%
\pgfsetstrokecolor{currentstroke}%
\pgfsetdash{}{0pt}%
\pgfpathmoveto{\pgfqpoint{2.759671in}{2.499205in}}%
\pgfpathlineto{\pgfqpoint{2.709703in}{2.502948in}}%
\pgfusepath{stroke}%
\end{pgfscope}%
\begin{pgfscope}%
\pgfpathrectangle{\pgfqpoint{1.250000in}{1.750000in}}{\pgfqpoint{2.279412in}{2.004545in}}%
\pgfusepath{clip}%
\pgfsetbuttcap%
\pgfsetroundjoin%
\pgfsetlinewidth{0.583507pt}%
\definecolor{currentstroke}{rgb}{0.269308,0.218818,0.509577}%
\pgfsetstrokecolor{currentstroke}%
\pgfsetdash{}{0pt}%
\pgfpathmoveto{\pgfqpoint{2.709703in}{2.502948in}}%
\pgfpathlineto{\pgfqpoint{2.659836in}{2.507615in}}%
\pgfusepath{stroke}%
\end{pgfscope}%
\begin{pgfscope}%
\pgfpathrectangle{\pgfqpoint{1.250000in}{1.750000in}}{\pgfqpoint{2.279412in}{2.004545in}}%
\pgfusepath{clip}%
\pgfsetbuttcap%
\pgfsetroundjoin%
\pgfsetlinewidth{0.606711pt}%
\definecolor{currentstroke}{rgb}{0.265145,0.232956,0.516599}%
\pgfsetstrokecolor{currentstroke}%
\pgfsetdash{}{0pt}%
\pgfpathmoveto{\pgfqpoint{2.659836in}{2.507615in}}%
\pgfpathlineto{\pgfqpoint{2.610154in}{2.513581in}}%
\pgfusepath{stroke}%
\end{pgfscope}%
\begin{pgfscope}%
\pgfpathrectangle{\pgfqpoint{1.250000in}{1.750000in}}{\pgfqpoint{2.279412in}{2.004545in}}%
\pgfusepath{clip}%
\pgfsetbuttcap%
\pgfsetroundjoin%
\pgfsetlinewidth{0.721397pt}%
\definecolor{currentstroke}{rgb}{0.235526,0.309527,0.542944}%
\pgfsetstrokecolor{currentstroke}%
\pgfsetdash{}{0pt}%
\pgfpathmoveto{\pgfqpoint{2.610154in}{2.513581in}}%
\pgfpathlineto{\pgfqpoint{2.560697in}{2.520874in}}%
\pgfusepath{stroke}%
\end{pgfscope}%
\begin{pgfscope}%
\pgfpathrectangle{\pgfqpoint{1.250000in}{1.750000in}}{\pgfqpoint{2.279412in}{2.004545in}}%
\pgfusepath{clip}%
\pgfsetbuttcap%
\pgfsetroundjoin%
\pgfsetlinewidth{0.886117pt}%
\definecolor{currentstroke}{rgb}{0.190631,0.407061,0.556089}%
\pgfsetstrokecolor{currentstroke}%
\pgfsetdash{}{0pt}%
\pgfpathmoveto{\pgfqpoint{2.560697in}{2.520874in}}%
\pgfpathlineto{\pgfqpoint{2.511536in}{2.529545in}}%
\pgfusepath{stroke}%
\end{pgfscope}%
\begin{pgfscope}%
\pgfpathrectangle{\pgfqpoint{1.250000in}{1.750000in}}{\pgfqpoint{2.279412in}{2.004545in}}%
\pgfusepath{clip}%
\pgfsetbuttcap%
\pgfsetroundjoin%
\pgfsetlinewidth{1.084088pt}%
\definecolor{currentstroke}{rgb}{0.146180,0.515413,0.556823}%
\pgfsetstrokecolor{currentstroke}%
\pgfsetdash{}{0pt}%
\pgfpathmoveto{\pgfqpoint{2.511536in}{2.529545in}}%
\pgfpathlineto{\pgfqpoint{2.462759in}{2.539768in}}%
\pgfusepath{stroke}%
\end{pgfscope}%
\begin{pgfscope}%
\pgfpathrectangle{\pgfqpoint{1.250000in}{1.750000in}}{\pgfqpoint{2.279412in}{2.004545in}}%
\pgfusepath{clip}%
\pgfsetbuttcap%
\pgfsetroundjoin%
\pgfsetlinewidth{1.080757pt}%
\definecolor{currentstroke}{rgb}{0.147607,0.511733,0.557049}%
\pgfsetstrokecolor{currentstroke}%
\pgfsetdash{}{0pt}%
\pgfpathmoveto{\pgfqpoint{2.462759in}{2.539768in}}%
\pgfpathlineto{\pgfqpoint{2.414446in}{2.551558in}}%
\pgfusepath{stroke}%
\end{pgfscope}%
\begin{pgfscope}%
\pgfpathrectangle{\pgfqpoint{1.250000in}{1.750000in}}{\pgfqpoint{2.279412in}{2.004545in}}%
\pgfusepath{clip}%
\pgfsetbuttcap%
\pgfsetroundjoin%
\pgfsetlinewidth{1.096219pt}%
\definecolor{currentstroke}{rgb}{0.144759,0.519093,0.556572}%
\pgfsetstrokecolor{currentstroke}%
\pgfsetdash{}{0pt}%
\pgfpathmoveto{\pgfqpoint{2.414446in}{2.551558in}}%
\pgfpathlineto{\pgfqpoint{2.366556in}{2.564631in}}%
\pgfusepath{stroke}%
\end{pgfscope}%
\begin{pgfscope}%
\pgfpathrectangle{\pgfqpoint{1.250000in}{1.750000in}}{\pgfqpoint{2.279412in}{2.004545in}}%
\pgfusepath{clip}%
\pgfsetbuttcap%
\pgfsetroundjoin%
\pgfsetlinewidth{1.313659pt}%
\definecolor{currentstroke}{rgb}{0.123444,0.636809,0.528763}%
\pgfsetstrokecolor{currentstroke}%
\pgfsetdash{}{0pt}%
\pgfpathmoveto{\pgfqpoint{2.366556in}{2.564631in}}%
\pgfpathlineto{\pgfqpoint{2.319108in}{2.578863in}}%
\pgfusepath{stroke}%
\end{pgfscope}%
\begin{pgfscope}%
\pgfpathrectangle{\pgfqpoint{1.250000in}{1.750000in}}{\pgfqpoint{2.279412in}{2.004545in}}%
\pgfusepath{clip}%
\pgfsetbuttcap%
\pgfsetroundjoin%
\pgfsetlinewidth{1.596481pt}%
\definecolor{currentstroke}{rgb}{0.335885,0.777018,0.402049}%
\pgfsetstrokecolor{currentstroke}%
\pgfsetdash{}{0pt}%
\pgfpathmoveto{\pgfqpoint{2.319108in}{2.578863in}}%
\pgfpathlineto{\pgfqpoint{2.272104in}{2.594209in}}%
\pgfusepath{stroke}%
\end{pgfscope}%
\begin{pgfscope}%
\pgfpathrectangle{\pgfqpoint{1.250000in}{1.750000in}}{\pgfqpoint{2.279412in}{2.004545in}}%
\pgfusepath{clip}%
\pgfsetbuttcap%
\pgfsetroundjoin%
\pgfsetlinewidth{1.810939pt}%
\definecolor{currentstroke}{rgb}{0.636902,0.856542,0.216620}%
\pgfsetstrokecolor{currentstroke}%
\pgfsetdash{}{0pt}%
\pgfpathmoveto{\pgfqpoint{2.272104in}{2.594209in}}%
\pgfpathlineto{\pgfqpoint{2.225444in}{2.610351in}}%
\pgfusepath{stroke}%
\end{pgfscope}%
\begin{pgfscope}%
\pgfpathrectangle{\pgfqpoint{1.250000in}{1.750000in}}{\pgfqpoint{2.279412in}{2.004545in}}%
\pgfusepath{clip}%
\pgfsetbuttcap%
\pgfsetroundjoin%
\pgfsetlinewidth{1.764458pt}%
\definecolor{currentstroke}{rgb}{0.565498,0.842430,0.262877}%
\pgfsetstrokecolor{currentstroke}%
\pgfsetdash{}{0pt}%
\pgfpathmoveto{\pgfqpoint{2.225444in}{2.610351in}}%
\pgfpathlineto{\pgfqpoint{2.179008in}{2.626947in}}%
\pgfusepath{stroke}%
\end{pgfscope}%
\begin{pgfscope}%
\pgfpathrectangle{\pgfqpoint{1.250000in}{1.750000in}}{\pgfqpoint{2.279412in}{2.004545in}}%
\pgfusepath{clip}%
\pgfsetbuttcap%
\pgfsetroundjoin%
\pgfsetlinewidth{2.092217pt}%
\definecolor{currentstroke}{rgb}{0.993248,0.906157,0.143936}%
\pgfsetstrokecolor{currentstroke}%
\pgfsetdash{}{0pt}%
\pgfpathmoveto{\pgfqpoint{2.179008in}{2.626947in}}%
\pgfpathlineto{\pgfqpoint{2.132793in}{2.643976in}}%
\pgfusepath{stroke}%
\end{pgfscope}%
\begin{pgfscope}%
\pgfpathrectangle{\pgfqpoint{1.250000in}{1.750000in}}{\pgfqpoint{2.279412in}{2.004545in}}%
\pgfusepath{clip}%
\pgfsetbuttcap%
\pgfsetroundjoin%
\pgfsetlinewidth{1.856792pt}%
\definecolor{currentstroke}{rgb}{0.699415,0.867117,0.175971}%
\pgfsetstrokecolor{currentstroke}%
\pgfsetdash{}{0pt}%
\pgfpathmoveto{\pgfqpoint{2.132793in}{2.643976in}}%
\pgfpathlineto{\pgfqpoint{2.086711in}{2.661278in}}%
\pgfusepath{stroke}%
\end{pgfscope}%
\begin{pgfscope}%
\pgfpathrectangle{\pgfqpoint{1.250000in}{1.750000in}}{\pgfqpoint{2.279412in}{2.004545in}}%
\pgfusepath{clip}%
\pgfsetbuttcap%
\pgfsetroundjoin%
\pgfsetlinewidth{1.838868pt}%
\definecolor{currentstroke}{rgb}{0.678489,0.863742,0.189503}%
\pgfsetstrokecolor{currentstroke}%
\pgfsetdash{}{0pt}%
\pgfpathmoveto{\pgfqpoint{2.086711in}{2.661278in}}%
\pgfpathlineto{\pgfqpoint{2.040886in}{2.679023in}}%
\pgfusepath{stroke}%
\end{pgfscope}%
\begin{pgfscope}%
\pgfpathrectangle{\pgfqpoint{1.250000in}{1.750000in}}{\pgfqpoint{2.279412in}{2.004545in}}%
\pgfusepath{clip}%
\pgfsetbuttcap%
\pgfsetroundjoin%
\pgfsetlinewidth{1.654215pt}%
\definecolor{currentstroke}{rgb}{0.412913,0.803041,0.357269}%
\pgfsetstrokecolor{currentstroke}%
\pgfsetdash{}{0pt}%
\pgfpathmoveto{\pgfqpoint{2.040886in}{2.679023in}}%
\pgfpathlineto{\pgfqpoint{1.994925in}{2.696370in}}%
\pgfusepath{stroke}%
\end{pgfscope}%
\begin{pgfscope}%
\pgfpathrectangle{\pgfqpoint{1.250000in}{1.750000in}}{\pgfqpoint{2.279412in}{2.004545in}}%
\pgfusepath{clip}%
\pgfsetbuttcap%
\pgfsetroundjoin%
\pgfsetlinewidth{0.313659pt}%
\definecolor{currentstroke}{rgb}{0.268510,0.009605,0.335427}%
\pgfsetstrokecolor{currentstroke}%
\pgfsetdash{}{0pt}%
\pgfpathmoveto{\pgfqpoint{3.210376in}{2.526739in}}%
\pgfpathlineto{\pgfqpoint{3.160384in}{2.527775in}}%
\pgfusepath{stroke}%
\end{pgfscope}%
\begin{pgfscope}%
\pgfpathrectangle{\pgfqpoint{1.250000in}{1.750000in}}{\pgfqpoint{2.279412in}{2.004545in}}%
\pgfusepath{clip}%
\pgfsetbuttcap%
\pgfsetroundjoin%
\pgfsetlinewidth{0.332586pt}%
\definecolor{currentstroke}{rgb}{0.272594,0.025563,0.353093}%
\pgfsetstrokecolor{currentstroke}%
\pgfsetdash{}{0pt}%
\pgfpathmoveto{\pgfqpoint{3.160384in}{2.527775in}}%
\pgfpathlineto{\pgfqpoint{3.110272in}{2.529191in}}%
\pgfusepath{stroke}%
\end{pgfscope}%
\begin{pgfscope}%
\pgfpathrectangle{\pgfqpoint{1.250000in}{1.750000in}}{\pgfqpoint{2.279412in}{2.004545in}}%
\pgfusepath{clip}%
\pgfsetbuttcap%
\pgfsetroundjoin%
\pgfsetlinewidth{0.329216pt}%
\definecolor{currentstroke}{rgb}{0.272594,0.025563,0.353093}%
\pgfsetstrokecolor{currentstroke}%
\pgfsetdash{}{0pt}%
\pgfpathmoveto{\pgfqpoint{3.110272in}{2.529191in}}%
\pgfpathlineto{\pgfqpoint{3.060148in}{2.529986in}}%
\pgfusepath{stroke}%
\end{pgfscope}%
\begin{pgfscope}%
\pgfpathrectangle{\pgfqpoint{1.250000in}{1.750000in}}{\pgfqpoint{2.279412in}{2.004545in}}%
\pgfusepath{clip}%
\pgfsetbuttcap%
\pgfsetroundjoin%
\pgfsetlinewidth{0.331598pt}%
\definecolor{currentstroke}{rgb}{0.272594,0.025563,0.353093}%
\pgfsetstrokecolor{currentstroke}%
\pgfsetdash{}{0pt}%
\pgfpathmoveto{\pgfqpoint{3.060148in}{2.529986in}}%
\pgfpathlineto{\pgfqpoint{3.010037in}{2.531289in}}%
\pgfusepath{stroke}%
\end{pgfscope}%
\begin{pgfscope}%
\pgfpathrectangle{\pgfqpoint{1.250000in}{1.750000in}}{\pgfqpoint{2.279412in}{2.004545in}}%
\pgfusepath{clip}%
\pgfsetbuttcap%
\pgfsetroundjoin%
\pgfsetlinewidth{0.358347pt}%
\definecolor{currentstroke}{rgb}{0.277018,0.050344,0.375715}%
\pgfsetstrokecolor{currentstroke}%
\pgfsetdash{}{0pt}%
\pgfpathmoveto{\pgfqpoint{3.010037in}{2.531289in}}%
\pgfpathlineto{\pgfqpoint{2.959943in}{2.533348in}}%
\pgfusepath{stroke}%
\end{pgfscope}%
\begin{pgfscope}%
\pgfpathrectangle{\pgfqpoint{1.250000in}{1.750000in}}{\pgfqpoint{2.279412in}{2.004545in}}%
\pgfusepath{clip}%
\pgfsetbuttcap%
\pgfsetroundjoin%
\pgfsetlinewidth{0.369778pt}%
\definecolor{currentstroke}{rgb}{0.278791,0.062145,0.386592}%
\pgfsetstrokecolor{currentstroke}%
\pgfsetdash{}{0pt}%
\pgfpathmoveto{\pgfqpoint{2.959943in}{2.533348in}}%
\pgfpathlineto{\pgfqpoint{2.909831in}{2.535048in}}%
\pgfusepath{stroke}%
\end{pgfscope}%
\begin{pgfscope}%
\pgfpathrectangle{\pgfqpoint{1.250000in}{1.750000in}}{\pgfqpoint{2.279412in}{2.004545in}}%
\pgfusepath{clip}%
\pgfsetbuttcap%
\pgfsetroundjoin%
\pgfsetlinewidth{0.399920pt}%
\definecolor{currentstroke}{rgb}{0.281446,0.084320,0.407414}%
\pgfsetstrokecolor{currentstroke}%
\pgfsetdash{}{0pt}%
\pgfpathmoveto{\pgfqpoint{2.909831in}{2.535048in}}%
\pgfpathlineto{\pgfqpoint{2.859707in}{2.536486in}}%
\pgfusepath{stroke}%
\end{pgfscope}%
\begin{pgfscope}%
\pgfpathrectangle{\pgfqpoint{1.250000in}{1.750000in}}{\pgfqpoint{2.279412in}{2.004545in}}%
\pgfusepath{clip}%
\pgfsetbuttcap%
\pgfsetroundjoin%
\pgfsetlinewidth{0.441067pt}%
\definecolor{currentstroke}{rgb}{0.283197,0.115680,0.436115}%
\pgfsetstrokecolor{currentstroke}%
\pgfsetdash{}{0pt}%
\pgfpathmoveto{\pgfqpoint{2.859707in}{2.536486in}}%
\pgfpathlineto{\pgfqpoint{2.809616in}{2.538570in}}%
\pgfusepath{stroke}%
\end{pgfscope}%
\begin{pgfscope}%
\pgfpathrectangle{\pgfqpoint{1.250000in}{1.750000in}}{\pgfqpoint{2.279412in}{2.004545in}}%
\pgfusepath{clip}%
\pgfsetbuttcap%
\pgfsetroundjoin%
\pgfsetlinewidth{0.514578pt}%
\definecolor{currentstroke}{rgb}{0.279574,0.170599,0.479997}%
\pgfsetstrokecolor{currentstroke}%
\pgfsetdash{}{0pt}%
\pgfpathmoveto{\pgfqpoint{2.809616in}{2.538570in}}%
\pgfpathlineto{\pgfqpoint{2.759558in}{2.541227in}}%
\pgfusepath{stroke}%
\end{pgfscope}%
\begin{pgfscope}%
\pgfpathrectangle{\pgfqpoint{1.250000in}{1.750000in}}{\pgfqpoint{2.279412in}{2.004545in}}%
\pgfusepath{clip}%
\pgfsetbuttcap%
\pgfsetroundjoin%
\pgfsetlinewidth{0.543635pt}%
\definecolor{currentstroke}{rgb}{0.276194,0.190074,0.493001}%
\pgfsetstrokecolor{currentstroke}%
\pgfsetdash{}{0pt}%
\pgfpathmoveto{\pgfqpoint{2.759558in}{2.541227in}}%
\pgfpathlineto{\pgfqpoint{2.709550in}{2.544522in}}%
\pgfusepath{stroke}%
\end{pgfscope}%
\begin{pgfscope}%
\pgfpathrectangle{\pgfqpoint{1.250000in}{1.750000in}}{\pgfqpoint{2.279412in}{2.004545in}}%
\pgfusepath{clip}%
\pgfsetbuttcap%
\pgfsetroundjoin%
\pgfsetlinewidth{0.311252pt}%
\definecolor{currentstroke}{rgb}{0.268510,0.009605,0.335427}%
\pgfsetstrokecolor{currentstroke}%
\pgfsetdash{}{0pt}%
\pgfpathmoveto{\pgfqpoint{3.210376in}{2.616952in}}%
\pgfpathlineto{\pgfqpoint{3.160280in}{2.617333in}}%
\pgfusepath{stroke}%
\end{pgfscope}%
\begin{pgfscope}%
\pgfpathrectangle{\pgfqpoint{1.250000in}{1.750000in}}{\pgfqpoint{2.279412in}{2.004545in}}%
\pgfusepath{clip}%
\pgfsetbuttcap%
\pgfsetroundjoin%
\pgfsetlinewidth{0.326309pt}%
\definecolor{currentstroke}{rgb}{0.271305,0.019942,0.347269}%
\pgfsetstrokecolor{currentstroke}%
\pgfsetdash{}{0pt}%
\pgfpathmoveto{\pgfqpoint{3.160280in}{2.617333in}}%
\pgfpathlineto{\pgfqpoint{3.110186in}{2.618890in}}%
\pgfusepath{stroke}%
\end{pgfscope}%
\begin{pgfscope}%
\pgfpathrectangle{\pgfqpoint{1.250000in}{1.750000in}}{\pgfqpoint{2.279412in}{2.004545in}}%
\pgfusepath{clip}%
\pgfsetbuttcap%
\pgfsetroundjoin%
\pgfsetlinewidth{0.330197pt}%
\definecolor{currentstroke}{rgb}{0.272594,0.025563,0.353093}%
\pgfsetstrokecolor{currentstroke}%
\pgfsetdash{}{0pt}%
\pgfpathmoveto{\pgfqpoint{3.110186in}{2.618890in}}%
\pgfpathlineto{\pgfqpoint{3.060060in}{2.619629in}}%
\pgfusepath{stroke}%
\end{pgfscope}%
\begin{pgfscope}%
\pgfpathrectangle{\pgfqpoint{1.250000in}{1.750000in}}{\pgfqpoint{2.279412in}{2.004545in}}%
\pgfusepath{clip}%
\pgfsetbuttcap%
\pgfsetroundjoin%
\pgfsetlinewidth{0.343220pt}%
\definecolor{currentstroke}{rgb}{0.274952,0.037752,0.364543}%
\pgfsetstrokecolor{currentstroke}%
\pgfsetdash{}{0pt}%
\pgfpathmoveto{\pgfqpoint{3.060060in}{2.619629in}}%
\pgfpathlineto{\pgfqpoint{3.009924in}{2.620429in}}%
\pgfusepath{stroke}%
\end{pgfscope}%
\begin{pgfscope}%
\pgfpathrectangle{\pgfqpoint{1.250000in}{1.750000in}}{\pgfqpoint{2.279412in}{2.004545in}}%
\pgfusepath{clip}%
\pgfsetbuttcap%
\pgfsetroundjoin%
\pgfsetlinewidth{0.365675pt}%
\definecolor{currentstroke}{rgb}{0.277941,0.056324,0.381191}%
\pgfsetstrokecolor{currentstroke}%
\pgfsetdash{}{0pt}%
\pgfpathmoveto{\pgfqpoint{3.009924in}{2.620429in}}%
\pgfpathlineto{\pgfqpoint{2.959788in}{2.621336in}}%
\pgfusepath{stroke}%
\end{pgfscope}%
\begin{pgfscope}%
\pgfpathrectangle{\pgfqpoint{1.250000in}{1.750000in}}{\pgfqpoint{2.279412in}{2.004545in}}%
\pgfusepath{clip}%
\pgfsetbuttcap%
\pgfsetroundjoin%
\pgfsetlinewidth{0.386221pt}%
\definecolor{currentstroke}{rgb}{0.280267,0.073417,0.397163}%
\pgfsetstrokecolor{currentstroke}%
\pgfsetdash{}{0pt}%
\pgfpathmoveto{\pgfqpoint{2.959788in}{2.621336in}}%
\pgfpathlineto{\pgfqpoint{2.909646in}{2.622129in}}%
\pgfusepath{stroke}%
\end{pgfscope}%
\begin{pgfscope}%
\pgfpathrectangle{\pgfqpoint{1.250000in}{1.750000in}}{\pgfqpoint{2.279412in}{2.004545in}}%
\pgfusepath{clip}%
\pgfsetbuttcap%
\pgfsetroundjoin%
\pgfsetlinewidth{0.421665pt}%
\definecolor{currentstroke}{rgb}{0.282656,0.100196,0.422160}%
\pgfsetstrokecolor{currentstroke}%
\pgfsetdash{}{0pt}%
\pgfpathmoveto{\pgfqpoint{2.909646in}{2.622129in}}%
\pgfpathlineto{\pgfqpoint{2.859507in}{2.623084in}}%
\pgfusepath{stroke}%
\end{pgfscope}%
\begin{pgfscope}%
\pgfpathrectangle{\pgfqpoint{1.250000in}{1.750000in}}{\pgfqpoint{2.279412in}{2.004545in}}%
\pgfusepath{clip}%
\pgfsetbuttcap%
\pgfsetroundjoin%
\pgfsetlinewidth{0.491170pt}%
\definecolor{currentstroke}{rgb}{0.281887,0.150881,0.465405}%
\pgfsetstrokecolor{currentstroke}%
\pgfsetdash{}{0pt}%
\pgfpathmoveto{\pgfqpoint{2.859507in}{2.623084in}}%
\pgfpathlineto{\pgfqpoint{2.809370in}{2.624132in}}%
\pgfusepath{stroke}%
\end{pgfscope}%
\begin{pgfscope}%
\pgfpathrectangle{\pgfqpoint{1.250000in}{1.750000in}}{\pgfqpoint{2.279412in}{2.004545in}}%
\pgfusepath{clip}%
\pgfsetbuttcap%
\pgfsetroundjoin%
\pgfsetlinewidth{0.585347pt}%
\definecolor{currentstroke}{rgb}{0.269308,0.218818,0.509577}%
\pgfsetstrokecolor{currentstroke}%
\pgfsetdash{}{0pt}%
\pgfpathmoveto{\pgfqpoint{2.809370in}{2.624132in}}%
\pgfpathlineto{\pgfqpoint{2.759243in}{2.625496in}}%
\pgfusepath{stroke}%
\end{pgfscope}%
\begin{pgfscope}%
\pgfpathrectangle{\pgfqpoint{1.250000in}{1.750000in}}{\pgfqpoint{2.279412in}{2.004545in}}%
\pgfusepath{clip}%
\pgfsetbuttcap%
\pgfsetroundjoin%
\pgfsetlinewidth{0.724120pt}%
\definecolor{currentstroke}{rgb}{0.235526,0.309527,0.542944}%
\pgfsetstrokecolor{currentstroke}%
\pgfsetdash{}{0pt}%
\pgfpathmoveto{\pgfqpoint{2.759243in}{2.625496in}}%
\pgfpathlineto{\pgfqpoint{2.709135in}{2.627294in}}%
\pgfusepath{stroke}%
\end{pgfscope}%
\begin{pgfscope}%
\pgfpathrectangle{\pgfqpoint{1.250000in}{1.750000in}}{\pgfqpoint{2.279412in}{2.004545in}}%
\pgfusepath{clip}%
\pgfsetbuttcap%
\pgfsetroundjoin%
\pgfsetlinewidth{0.918855pt}%
\definecolor{currentstroke}{rgb}{0.182256,0.426184,0.557120}%
\pgfsetstrokecolor{currentstroke}%
\pgfsetdash{}{0pt}%
\pgfpathmoveto{\pgfqpoint{2.709135in}{2.627294in}}%
\pgfpathlineto{\pgfqpoint{2.659059in}{2.629694in}}%
\pgfusepath{stroke}%
\end{pgfscope}%
\begin{pgfscope}%
\pgfpathrectangle{\pgfqpoint{1.250000in}{1.750000in}}{\pgfqpoint{2.279412in}{2.004545in}}%
\pgfusepath{clip}%
\pgfsetbuttcap%
\pgfsetroundjoin%
\pgfsetlinewidth{1.120220pt}%
\definecolor{currentstroke}{rgb}{0.139147,0.533812,0.555298}%
\pgfsetstrokecolor{currentstroke}%
\pgfsetdash{}{0pt}%
\pgfpathmoveto{\pgfqpoint{2.659059in}{2.629694in}}%
\pgfpathlineto{\pgfqpoint{2.609034in}{2.632793in}}%
\pgfusepath{stroke}%
\end{pgfscope}%
\begin{pgfscope}%
\pgfpathrectangle{\pgfqpoint{1.250000in}{1.750000in}}{\pgfqpoint{2.279412in}{2.004545in}}%
\pgfusepath{clip}%
\pgfsetbuttcap%
\pgfsetroundjoin%
\pgfsetlinewidth{1.297197pt}%
\definecolor{currentstroke}{rgb}{0.121380,0.629492,0.531973}%
\pgfsetstrokecolor{currentstroke}%
\pgfsetdash{}{0pt}%
\pgfpathmoveto{\pgfqpoint{2.609034in}{2.632793in}}%
\pgfpathlineto{\pgfqpoint{2.559069in}{2.636573in}}%
\pgfusepath{stroke}%
\end{pgfscope}%
\begin{pgfscope}%
\pgfpathrectangle{\pgfqpoint{1.250000in}{1.750000in}}{\pgfqpoint{2.279412in}{2.004545in}}%
\pgfusepath{clip}%
\pgfsetbuttcap%
\pgfsetroundjoin%
\pgfsetlinewidth{1.514642pt}%
\definecolor{currentstroke}{rgb}{0.246070,0.738910,0.452024}%
\pgfsetstrokecolor{currentstroke}%
\pgfsetdash{}{0pt}%
\pgfpathmoveto{\pgfqpoint{2.559069in}{2.636573in}}%
\pgfpathlineto{\pgfqpoint{2.509170in}{2.640966in}}%
\pgfusepath{stroke}%
\end{pgfscope}%
\begin{pgfscope}%
\pgfpathrectangle{\pgfqpoint{1.250000in}{1.750000in}}{\pgfqpoint{2.279412in}{2.004545in}}%
\pgfusepath{clip}%
\pgfsetbuttcap%
\pgfsetroundjoin%
\pgfsetlinewidth{1.634007pt}%
\definecolor{currentstroke}{rgb}{0.386433,0.794644,0.372886}%
\pgfsetstrokecolor{currentstroke}%
\pgfsetdash{}{0pt}%
\pgfpathmoveto{\pgfqpoint{2.509170in}{2.640966in}}%
\pgfpathlineto{\pgfqpoint{2.459354in}{2.646020in}}%
\pgfusepath{stroke}%
\end{pgfscope}%
\begin{pgfscope}%
\pgfpathrectangle{\pgfqpoint{1.250000in}{1.750000in}}{\pgfqpoint{2.279412in}{2.004545in}}%
\pgfusepath{clip}%
\pgfsetbuttcap%
\pgfsetroundjoin%
\pgfsetlinewidth{1.875591pt}%
\definecolor{currentstroke}{rgb}{0.730889,0.871916,0.156029}%
\pgfsetstrokecolor{currentstroke}%
\pgfsetdash{}{0pt}%
\pgfpathmoveto{\pgfqpoint{2.459354in}{2.646020in}}%
\pgfpathlineto{\pgfqpoint{2.409635in}{2.651764in}}%
\pgfusepath{stroke}%
\end{pgfscope}%
\begin{pgfscope}%
\pgfpathrectangle{\pgfqpoint{1.250000in}{1.750000in}}{\pgfqpoint{2.279412in}{2.004545in}}%
\pgfusepath{clip}%
\pgfsetbuttcap%
\pgfsetroundjoin%
\pgfsetlinewidth{2.090758pt}%
\definecolor{currentstroke}{rgb}{0.993248,0.906157,0.143936}%
\pgfsetstrokecolor{currentstroke}%
\pgfsetdash{}{0pt}%
\pgfpathmoveto{\pgfqpoint{2.409635in}{2.651764in}}%
\pgfpathlineto{\pgfqpoint{2.360009in}{2.658104in}}%
\pgfusepath{stroke}%
\end{pgfscope}%
\begin{pgfscope}%
\pgfpathrectangle{\pgfqpoint{1.250000in}{1.750000in}}{\pgfqpoint{2.279412in}{2.004545in}}%
\pgfusepath{clip}%
\pgfsetbuttcap%
\pgfsetroundjoin%
\pgfsetlinewidth{2.079396pt}%
\definecolor{currentstroke}{rgb}{0.993248,0.906157,0.143936}%
\pgfsetstrokecolor{currentstroke}%
\pgfsetdash{}{0pt}%
\pgfpathmoveto{\pgfqpoint{2.360009in}{2.658104in}}%
\pgfpathlineto{\pgfqpoint{2.310474in}{2.664953in}}%
\pgfusepath{stroke}%
\end{pgfscope}%
\begin{pgfscope}%
\pgfpathrectangle{\pgfqpoint{1.250000in}{1.750000in}}{\pgfqpoint{2.279412in}{2.004545in}}%
\pgfusepath{clip}%
\pgfsetbuttcap%
\pgfsetroundjoin%
\pgfsetlinewidth{2.165650pt}%
\definecolor{currentstroke}{rgb}{0.993248,0.906157,0.143936}%
\pgfsetstrokecolor{currentstroke}%
\pgfsetdash{}{0pt}%
\pgfpathmoveto{\pgfqpoint{2.310474in}{2.664953in}}%
\pgfpathlineto{\pgfqpoint{2.261065in}{2.672481in}}%
\pgfusepath{stroke}%
\end{pgfscope}%
\begin{pgfscope}%
\pgfpathrectangle{\pgfqpoint{1.250000in}{1.750000in}}{\pgfqpoint{2.279412in}{2.004545in}}%
\pgfusepath{clip}%
\pgfsetbuttcap%
\pgfsetroundjoin%
\pgfsetlinewidth{2.104936pt}%
\definecolor{currentstroke}{rgb}{0.993248,0.906157,0.143936}%
\pgfsetstrokecolor{currentstroke}%
\pgfsetdash{}{0pt}%
\pgfpathmoveto{\pgfqpoint{2.261065in}{2.672481in}}%
\pgfpathlineto{\pgfqpoint{2.211803in}{2.680701in}}%
\pgfusepath{stroke}%
\end{pgfscope}%
\begin{pgfscope}%
\pgfpathrectangle{\pgfqpoint{1.250000in}{1.750000in}}{\pgfqpoint{2.279412in}{2.004545in}}%
\pgfusepath{clip}%
\pgfsetbuttcap%
\pgfsetroundjoin%
\pgfsetlinewidth{2.026138pt}%
\definecolor{currentstroke}{rgb}{0.955300,0.901065,0.118128}%
\pgfsetstrokecolor{currentstroke}%
\pgfsetdash{}{0pt}%
\pgfpathmoveto{\pgfqpoint{2.211803in}{2.680701in}}%
\pgfpathlineto{\pgfqpoint{2.162596in}{2.689110in}}%
\pgfusepath{stroke}%
\end{pgfscope}%
\begin{pgfscope}%
\pgfpathrectangle{\pgfqpoint{1.250000in}{1.750000in}}{\pgfqpoint{2.279412in}{2.004545in}}%
\pgfusepath{clip}%
\pgfsetbuttcap%
\pgfsetroundjoin%
\pgfsetlinewidth{1.998195pt}%
\definecolor{currentstroke}{rgb}{0.916242,0.896091,0.100717}%
\pgfsetstrokecolor{currentstroke}%
\pgfsetdash{}{0pt}%
\pgfpathmoveto{\pgfqpoint{2.162596in}{2.689110in}}%
\pgfpathlineto{\pgfqpoint{2.113504in}{2.697959in}}%
\pgfusepath{stroke}%
\end{pgfscope}%
\begin{pgfscope}%
\pgfpathrectangle{\pgfqpoint{1.250000in}{1.750000in}}{\pgfqpoint{2.279412in}{2.004545in}}%
\pgfusepath{clip}%
\pgfsetbuttcap%
\pgfsetroundjoin%
\pgfsetlinewidth{1.688900pt}%
\definecolor{currentstroke}{rgb}{0.458674,0.816363,0.329727}%
\pgfsetstrokecolor{currentstroke}%
\pgfsetdash{}{0pt}%
\pgfpathmoveto{\pgfqpoint{2.113504in}{2.697959in}}%
\pgfpathlineto{\pgfqpoint{2.064567in}{2.707360in}}%
\pgfusepath{stroke}%
\end{pgfscope}%
\begin{pgfscope}%
\pgfpathrectangle{\pgfqpoint{1.250000in}{1.750000in}}{\pgfqpoint{2.279412in}{2.004545in}}%
\pgfusepath{clip}%
\pgfsetbuttcap%
\pgfsetroundjoin%
\pgfsetlinewidth{1.431389pt}%
\definecolor{currentstroke}{rgb}{0.175707,0.697900,0.491033}%
\pgfsetstrokecolor{currentstroke}%
\pgfsetdash{}{0pt}%
\pgfpathmoveto{\pgfqpoint{2.064567in}{2.707360in}}%
\pgfpathlineto{\pgfqpoint{2.015589in}{2.716452in}}%
\pgfusepath{stroke}%
\end{pgfscope}%
\begin{pgfscope}%
\pgfpathrectangle{\pgfqpoint{1.250000in}{1.750000in}}{\pgfqpoint{2.279412in}{2.004545in}}%
\pgfusepath{clip}%
\pgfsetbuttcap%
\pgfsetroundjoin%
\pgfsetlinewidth{0.318818pt}%
\definecolor{currentstroke}{rgb}{0.269944,0.014625,0.341379}%
\pgfsetstrokecolor{currentstroke}%
\pgfsetdash{}{0pt}%
\pgfpathmoveto{\pgfqpoint{3.159084in}{2.346312in}}%
\pgfpathlineto{\pgfqpoint{3.109474in}{2.348675in}}%
\pgfusepath{stroke}%
\end{pgfscope}%
\begin{pgfscope}%
\pgfpathrectangle{\pgfqpoint{1.250000in}{1.750000in}}{\pgfqpoint{2.279412in}{2.004545in}}%
\pgfusepath{clip}%
\pgfsetbuttcap%
\pgfsetroundjoin%
\pgfsetlinewidth{0.313336pt}%
\definecolor{currentstroke}{rgb}{0.268510,0.009605,0.335427}%
\pgfsetstrokecolor{currentstroke}%
\pgfsetdash{}{0pt}%
\pgfpathmoveto{\pgfqpoint{3.109474in}{2.348675in}}%
\pgfpathlineto{\pgfqpoint{3.059836in}{2.351203in}}%
\pgfusepath{stroke}%
\end{pgfscope}%
\begin{pgfscope}%
\pgfpathrectangle{\pgfqpoint{1.250000in}{1.750000in}}{\pgfqpoint{2.279412in}{2.004545in}}%
\pgfusepath{clip}%
\pgfsetbuttcap%
\pgfsetroundjoin%
\pgfsetlinewidth{0.334472pt}%
\definecolor{currentstroke}{rgb}{0.272594,0.025563,0.353093}%
\pgfsetstrokecolor{currentstroke}%
\pgfsetdash{}{0pt}%
\pgfpathmoveto{\pgfqpoint{3.059836in}{2.351203in}}%
\pgfpathlineto{\pgfqpoint{3.009768in}{2.353682in}}%
\pgfusepath{stroke}%
\end{pgfscope}%
\begin{pgfscope}%
\pgfpathrectangle{\pgfqpoint{1.250000in}{1.750000in}}{\pgfqpoint{2.279412in}{2.004545in}}%
\pgfusepath{clip}%
\pgfsetbuttcap%
\pgfsetroundjoin%
\pgfsetlinewidth{0.334261pt}%
\definecolor{currentstroke}{rgb}{0.272594,0.025563,0.353093}%
\pgfsetstrokecolor{currentstroke}%
\pgfsetdash{}{0pt}%
\pgfpathmoveto{\pgfqpoint{3.009768in}{2.353682in}}%
\pgfpathlineto{\pgfqpoint{2.959679in}{2.355688in}}%
\pgfusepath{stroke}%
\end{pgfscope}%
\begin{pgfscope}%
\pgfpathrectangle{\pgfqpoint{1.250000in}{1.750000in}}{\pgfqpoint{2.279412in}{2.004545in}}%
\pgfusepath{clip}%
\pgfsetbuttcap%
\pgfsetroundjoin%
\pgfsetlinewidth{0.348659pt}%
\definecolor{currentstroke}{rgb}{0.274952,0.037752,0.364543}%
\pgfsetstrokecolor{currentstroke}%
\pgfsetdash{}{0pt}%
\pgfpathmoveto{\pgfqpoint{2.959679in}{2.355688in}}%
\pgfpathlineto{\pgfqpoint{2.909640in}{2.358323in}}%
\pgfusepath{stroke}%
\end{pgfscope}%
\begin{pgfscope}%
\pgfpathrectangle{\pgfqpoint{1.250000in}{1.750000in}}{\pgfqpoint{2.279412in}{2.004545in}}%
\pgfusepath{clip}%
\pgfsetbuttcap%
\pgfsetroundjoin%
\pgfsetlinewidth{0.359401pt}%
\definecolor{currentstroke}{rgb}{0.277018,0.050344,0.375715}%
\pgfsetstrokecolor{currentstroke}%
\pgfsetdash{}{0pt}%
\pgfpathmoveto{\pgfqpoint{2.909640in}{2.358323in}}%
\pgfpathlineto{\pgfqpoint{2.859621in}{2.361454in}}%
\pgfusepath{stroke}%
\end{pgfscope}%
\begin{pgfscope}%
\pgfpathrectangle{\pgfqpoint{1.250000in}{1.750000in}}{\pgfqpoint{2.279412in}{2.004545in}}%
\pgfusepath{clip}%
\pgfsetbuttcap%
\pgfsetroundjoin%
\pgfsetlinewidth{0.379432pt}%
\definecolor{currentstroke}{rgb}{0.279566,0.067836,0.391917}%
\pgfsetstrokecolor{currentstroke}%
\pgfsetdash{}{0pt}%
\pgfpathmoveto{\pgfqpoint{2.859621in}{2.361454in}}%
\pgfpathlineto{\pgfqpoint{2.809556in}{2.363881in}}%
\pgfusepath{stroke}%
\end{pgfscope}%
\begin{pgfscope}%
\pgfpathrectangle{\pgfqpoint{1.250000in}{1.750000in}}{\pgfqpoint{2.279412in}{2.004545in}}%
\pgfusepath{clip}%
\pgfsetbuttcap%
\pgfsetroundjoin%
\pgfsetlinewidth{0.386348pt}%
\definecolor{currentstroke}{rgb}{0.280267,0.073417,0.397163}%
\pgfsetstrokecolor{currentstroke}%
\pgfsetdash{}{0pt}%
\pgfpathmoveto{\pgfqpoint{2.809556in}{2.363881in}}%
\pgfpathlineto{\pgfqpoint{2.759658in}{2.367810in}}%
\pgfusepath{stroke}%
\end{pgfscope}%
\begin{pgfscope}%
\pgfpathrectangle{\pgfqpoint{1.250000in}{1.750000in}}{\pgfqpoint{2.279412in}{2.004545in}}%
\pgfusepath{clip}%
\pgfsetbuttcap%
\pgfsetroundjoin%
\pgfsetlinewidth{0.417744pt}%
\definecolor{currentstroke}{rgb}{0.282327,0.094955,0.417331}%
\pgfsetstrokecolor{currentstroke}%
\pgfsetdash{}{0pt}%
\pgfpathmoveto{\pgfqpoint{2.759658in}{2.367810in}}%
\pgfpathlineto{\pgfqpoint{2.709938in}{2.373515in}}%
\pgfusepath{stroke}%
\end{pgfscope}%
\begin{pgfscope}%
\pgfpathrectangle{\pgfqpoint{1.250000in}{1.750000in}}{\pgfqpoint{2.279412in}{2.004545in}}%
\pgfusepath{clip}%
\pgfsetbuttcap%
\pgfsetroundjoin%
\pgfsetlinewidth{0.433725pt}%
\definecolor{currentstroke}{rgb}{0.283091,0.110553,0.431554}%
\pgfsetstrokecolor{currentstroke}%
\pgfsetdash{}{0pt}%
\pgfpathmoveto{\pgfqpoint{2.709938in}{2.373515in}}%
\pgfpathlineto{\pgfqpoint{2.660232in}{2.379375in}}%
\pgfusepath{stroke}%
\end{pgfscope}%
\begin{pgfscope}%
\pgfpathrectangle{\pgfqpoint{1.250000in}{1.750000in}}{\pgfqpoint{2.279412in}{2.004545in}}%
\pgfusepath{clip}%
\pgfsetbuttcap%
\pgfsetroundjoin%
\pgfsetlinewidth{0.410104pt}%
\definecolor{currentstroke}{rgb}{0.281924,0.089666,0.412415}%
\pgfsetstrokecolor{currentstroke}%
\pgfsetdash{}{0pt}%
\pgfpathmoveto{\pgfqpoint{2.660232in}{2.379375in}}%
\pgfpathlineto{\pgfqpoint{2.610756in}{2.386402in}}%
\pgfusepath{stroke}%
\end{pgfscope}%
\begin{pgfscope}%
\pgfpathrectangle{\pgfqpoint{1.250000in}{1.750000in}}{\pgfqpoint{2.279412in}{2.004545in}}%
\pgfusepath{clip}%
\pgfsetbuttcap%
\pgfsetroundjoin%
\pgfsetlinewidth{0.450017pt}%
\definecolor{currentstroke}{rgb}{0.283229,0.120777,0.440584}%
\pgfsetstrokecolor{currentstroke}%
\pgfsetdash{}{0pt}%
\pgfpathmoveto{\pgfqpoint{2.610756in}{2.386402in}}%
\pgfpathlineto{\pgfqpoint{2.561705in}{2.395503in}}%
\pgfusepath{stroke}%
\end{pgfscope}%
\begin{pgfscope}%
\pgfpathrectangle{\pgfqpoint{1.250000in}{1.750000in}}{\pgfqpoint{2.279412in}{2.004545in}}%
\pgfusepath{clip}%
\pgfsetbuttcap%
\pgfsetroundjoin%
\pgfsetlinewidth{0.516129pt}%
\definecolor{currentstroke}{rgb}{0.279574,0.170599,0.479997}%
\pgfsetstrokecolor{currentstroke}%
\pgfsetdash{}{0pt}%
\pgfpathmoveto{\pgfqpoint{2.561705in}{2.395503in}}%
\pgfpathlineto{\pgfqpoint{2.513326in}{2.406989in}}%
\pgfusepath{stroke}%
\end{pgfscope}%
\begin{pgfscope}%
\pgfpathrectangle{\pgfqpoint{1.250000in}{1.750000in}}{\pgfqpoint{2.279412in}{2.004545in}}%
\pgfusepath{clip}%
\pgfsetbuttcap%
\pgfsetroundjoin%
\pgfsetlinewidth{0.422664pt}%
\definecolor{currentstroke}{rgb}{0.282656,0.100196,0.422160}%
\pgfsetstrokecolor{currentstroke}%
\pgfsetdash{}{0pt}%
\pgfpathmoveto{\pgfqpoint{2.513326in}{2.406989in}}%
\pgfpathlineto{\pgfqpoint{2.466004in}{2.421370in}}%
\pgfusepath{stroke}%
\end{pgfscope}%
\begin{pgfscope}%
\pgfpathrectangle{\pgfqpoint{1.250000in}{1.750000in}}{\pgfqpoint{2.279412in}{2.004545in}}%
\pgfusepath{clip}%
\pgfsetbuttcap%
\pgfsetroundjoin%
\pgfsetlinewidth{0.465663pt}%
\definecolor{currentstroke}{rgb}{0.283072,0.130895,0.449241}%
\pgfsetstrokecolor{currentstroke}%
\pgfsetdash{}{0pt}%
\pgfpathmoveto{\pgfqpoint{2.466004in}{2.421370in}}%
\pgfpathlineto{\pgfqpoint{2.419637in}{2.437967in}}%
\pgfusepath{stroke}%
\end{pgfscope}%
\begin{pgfscope}%
\pgfpathrectangle{\pgfqpoint{1.250000in}{1.750000in}}{\pgfqpoint{2.279412in}{2.004545in}}%
\pgfusepath{clip}%
\pgfsetbuttcap%
\pgfsetroundjoin%
\pgfsetlinewidth{0.632789pt}%
\definecolor{currentstroke}{rgb}{0.258965,0.251537,0.524736}%
\pgfsetstrokecolor{currentstroke}%
\pgfsetdash{}{0pt}%
\pgfpathmoveto{\pgfqpoint{2.419637in}{2.437967in}}%
\pgfpathlineto{\pgfqpoint{2.374541in}{2.457154in}}%
\pgfusepath{stroke}%
\end{pgfscope}%
\begin{pgfscope}%
\pgfpathrectangle{\pgfqpoint{1.250000in}{1.750000in}}{\pgfqpoint{2.279412in}{2.004545in}}%
\pgfusepath{clip}%
\pgfsetbuttcap%
\pgfsetroundjoin%
\pgfsetlinewidth{0.757511pt}%
\definecolor{currentstroke}{rgb}{0.225863,0.330805,0.547314}%
\pgfsetstrokecolor{currentstroke}%
\pgfsetdash{}{0pt}%
\pgfpathmoveto{\pgfqpoint{2.374541in}{2.457154in}}%
\pgfpathlineto{\pgfqpoint{2.330736in}{2.478541in}}%
\pgfusepath{stroke}%
\end{pgfscope}%
\begin{pgfscope}%
\pgfpathrectangle{\pgfqpoint{1.250000in}{1.750000in}}{\pgfqpoint{2.279412in}{2.004545in}}%
\pgfusepath{clip}%
\pgfsetbuttcap%
\pgfsetroundjoin%
\pgfsetlinewidth{0.767693pt}%
\definecolor{currentstroke}{rgb}{0.223925,0.334994,0.548053}%
\pgfsetstrokecolor{currentstroke}%
\pgfsetdash{}{0pt}%
\pgfpathmoveto{\pgfqpoint{2.330736in}{2.478541in}}%
\pgfpathlineto{\pgfqpoint{2.288388in}{2.502064in}}%
\pgfusepath{stroke}%
\end{pgfscope}%
\begin{pgfscope}%
\pgfpathrectangle{\pgfqpoint{1.250000in}{1.750000in}}{\pgfqpoint{2.279412in}{2.004545in}}%
\pgfusepath{clip}%
\pgfsetbuttcap%
\pgfsetroundjoin%
\pgfsetlinewidth{0.921152pt}%
\definecolor{currentstroke}{rgb}{0.182256,0.426184,0.557120}%
\pgfsetstrokecolor{currentstroke}%
\pgfsetdash{}{0pt}%
\pgfpathmoveto{\pgfqpoint{2.288388in}{2.502064in}}%
\pgfpathlineto{\pgfqpoint{2.247150in}{2.527103in}}%
\pgfusepath{stroke}%
\end{pgfscope}%
\begin{pgfscope}%
\pgfpathrectangle{\pgfqpoint{1.250000in}{1.750000in}}{\pgfqpoint{2.279412in}{2.004545in}}%
\pgfusepath{clip}%
\pgfsetbuttcap%
\pgfsetroundjoin%
\pgfsetlinewidth{1.216397pt}%
\definecolor{currentstroke}{rgb}{0.122606,0.585371,0.546557}%
\pgfsetstrokecolor{currentstroke}%
\pgfsetdash{}{0pt}%
\pgfpathmoveto{\pgfqpoint{2.247150in}{2.527103in}}%
\pgfpathlineto{\pgfqpoint{2.206228in}{2.552564in}}%
\pgfusepath{stroke}%
\end{pgfscope}%
\begin{pgfscope}%
\pgfpathrectangle{\pgfqpoint{1.250000in}{1.750000in}}{\pgfqpoint{2.279412in}{2.004545in}}%
\pgfusepath{clip}%
\pgfsetbuttcap%
\pgfsetroundjoin%
\pgfsetlinewidth{1.541118pt}%
\definecolor{currentstroke}{rgb}{0.274149,0.751988,0.436601}%
\pgfsetstrokecolor{currentstroke}%
\pgfsetdash{}{0pt}%
\pgfpathmoveto{\pgfqpoint{2.206228in}{2.552564in}}%
\pgfpathlineto{\pgfqpoint{2.165448in}{2.578220in}}%
\pgfusepath{stroke}%
\end{pgfscope}%
\begin{pgfscope}%
\pgfpathrectangle{\pgfqpoint{1.250000in}{1.750000in}}{\pgfqpoint{2.279412in}{2.004545in}}%
\pgfusepath{clip}%
\pgfsetbuttcap%
\pgfsetroundjoin%
\pgfsetlinewidth{1.956508pt}%
\definecolor{currentstroke}{rgb}{0.855810,0.888601,0.097452}%
\pgfsetstrokecolor{currentstroke}%
\pgfsetdash{}{0pt}%
\pgfpathmoveto{\pgfqpoint{2.165448in}{2.578220in}}%
\pgfpathlineto{\pgfqpoint{2.124609in}{2.603755in}}%
\pgfusepath{stroke}%
\end{pgfscope}%
\begin{pgfscope}%
\pgfpathrectangle{\pgfqpoint{1.250000in}{1.750000in}}{\pgfqpoint{2.279412in}{2.004545in}}%
\pgfusepath{clip}%
\pgfsetbuttcap%
\pgfsetroundjoin%
\pgfsetlinewidth{1.972809pt}%
\definecolor{currentstroke}{rgb}{0.876168,0.891125,0.095250}%
\pgfsetstrokecolor{currentstroke}%
\pgfsetdash{}{0pt}%
\pgfpathmoveto{\pgfqpoint{2.124609in}{2.603755in}}%
\pgfpathlineto{\pgfqpoint{2.082861in}{2.628024in}}%
\pgfusepath{stroke}%
\end{pgfscope}%
\begin{pgfscope}%
\pgfpathrectangle{\pgfqpoint{1.250000in}{1.750000in}}{\pgfqpoint{2.279412in}{2.004545in}}%
\pgfusepath{clip}%
\pgfsetbuttcap%
\pgfsetroundjoin%
\pgfsetlinewidth{0.330086pt}%
\definecolor{currentstroke}{rgb}{0.272594,0.025563,0.353093}%
\pgfsetstrokecolor{currentstroke}%
\pgfsetdash{}{0pt}%
\pgfpathmoveto{\pgfqpoint{3.159084in}{2.436525in}}%
\pgfpathlineto{\pgfqpoint{3.108948in}{2.436291in}}%
\pgfusepath{stroke}%
\end{pgfscope}%
\begin{pgfscope}%
\pgfpathrectangle{\pgfqpoint{1.250000in}{1.750000in}}{\pgfqpoint{2.279412in}{2.004545in}}%
\pgfusepath{clip}%
\pgfsetbuttcap%
\pgfsetroundjoin%
\pgfsetlinewidth{0.325685pt}%
\definecolor{currentstroke}{rgb}{0.271305,0.019942,0.347269}%
\pgfsetstrokecolor{currentstroke}%
\pgfsetdash{}{0pt}%
\pgfpathmoveto{\pgfqpoint{3.108948in}{2.436291in}}%
\pgfpathlineto{\pgfqpoint{3.058833in}{2.437424in}}%
\pgfusepath{stroke}%
\end{pgfscope}%
\begin{pgfscope}%
\pgfpathrectangle{\pgfqpoint{1.250000in}{1.750000in}}{\pgfqpoint{2.279412in}{2.004545in}}%
\pgfusepath{clip}%
\pgfsetbuttcap%
\pgfsetroundjoin%
\pgfsetlinewidth{0.325605pt}%
\definecolor{currentstroke}{rgb}{0.271305,0.019942,0.347269}%
\pgfsetstrokecolor{currentstroke}%
\pgfsetdash{}{0pt}%
\pgfpathmoveto{\pgfqpoint{3.058833in}{2.437424in}}%
\pgfpathlineto{\pgfqpoint{3.008724in}{2.439116in}}%
\pgfusepath{stroke}%
\end{pgfscope}%
\begin{pgfscope}%
\pgfpathrectangle{\pgfqpoint{1.250000in}{1.750000in}}{\pgfqpoint{2.279412in}{2.004545in}}%
\pgfusepath{clip}%
\pgfsetbuttcap%
\pgfsetroundjoin%
\pgfsetlinewidth{0.338586pt}%
\definecolor{currentstroke}{rgb}{0.273809,0.031497,0.358853}%
\pgfsetstrokecolor{currentstroke}%
\pgfsetdash{}{0pt}%
\pgfpathmoveto{\pgfqpoint{3.008724in}{2.439116in}}%
\pgfpathlineto{\pgfqpoint{2.958616in}{2.440787in}}%
\pgfusepath{stroke}%
\end{pgfscope}%
\begin{pgfscope}%
\pgfpathrectangle{\pgfqpoint{1.250000in}{1.750000in}}{\pgfqpoint{2.279412in}{2.004545in}}%
\pgfusepath{clip}%
\pgfsetbuttcap%
\pgfsetroundjoin%
\pgfsetlinewidth{0.365243pt}%
\definecolor{currentstroke}{rgb}{0.277941,0.056324,0.381191}%
\pgfsetstrokecolor{currentstroke}%
\pgfsetdash{}{0pt}%
\pgfpathmoveto{\pgfqpoint{2.958616in}{2.440787in}}%
\pgfpathlineto{\pgfqpoint{2.908546in}{2.443214in}}%
\pgfusepath{stroke}%
\end{pgfscope}%
\begin{pgfscope}%
\pgfpathrectangle{\pgfqpoint{1.250000in}{1.750000in}}{\pgfqpoint{2.279412in}{2.004545in}}%
\pgfusepath{clip}%
\pgfsetbuttcap%
\pgfsetroundjoin%
\pgfsetlinewidth{0.373135pt}%
\definecolor{currentstroke}{rgb}{0.278791,0.062145,0.386592}%
\pgfsetstrokecolor{currentstroke}%
\pgfsetdash{}{0pt}%
\pgfpathmoveto{\pgfqpoint{2.908546in}{2.443214in}}%
\pgfpathlineto{\pgfqpoint{2.858484in}{2.445826in}}%
\pgfusepath{stroke}%
\end{pgfscope}%
\begin{pgfscope}%
\pgfpathrectangle{\pgfqpoint{1.250000in}{1.750000in}}{\pgfqpoint{2.279412in}{2.004545in}}%
\pgfusepath{clip}%
\pgfsetbuttcap%
\pgfsetroundjoin%
\pgfsetlinewidth{0.399679pt}%
\definecolor{currentstroke}{rgb}{0.281446,0.084320,0.407414}%
\pgfsetstrokecolor{currentstroke}%
\pgfsetdash{}{0pt}%
\pgfpathmoveto{\pgfqpoint{2.858484in}{2.445826in}}%
\pgfpathlineto{\pgfqpoint{2.808419in}{2.448416in}}%
\pgfusepath{stroke}%
\end{pgfscope}%
\begin{pgfscope}%
\pgfpathrectangle{\pgfqpoint{1.250000in}{1.750000in}}{\pgfqpoint{2.279412in}{2.004545in}}%
\pgfusepath{clip}%
\pgfsetbuttcap%
\pgfsetroundjoin%
\pgfsetlinewidth{0.427753pt}%
\definecolor{currentstroke}{rgb}{0.282910,0.105393,0.426902}%
\pgfsetstrokecolor{currentstroke}%
\pgfsetdash{}{0pt}%
\pgfpathmoveto{\pgfqpoint{2.808419in}{2.448416in}}%
\pgfpathlineto{\pgfqpoint{2.758425in}{2.451841in}}%
\pgfusepath{stroke}%
\end{pgfscope}%
\begin{pgfscope}%
\pgfpathrectangle{\pgfqpoint{1.250000in}{1.750000in}}{\pgfqpoint{2.279412in}{2.004545in}}%
\pgfusepath{clip}%
\pgfsetbuttcap%
\pgfsetroundjoin%
\pgfsetlinewidth{0.464249pt}%
\definecolor{currentstroke}{rgb}{0.283072,0.130895,0.449241}%
\pgfsetstrokecolor{currentstroke}%
\pgfsetdash{}{0pt}%
\pgfpathmoveto{\pgfqpoint{2.758425in}{2.451841in}}%
\pgfpathlineto{\pgfqpoint{2.708517in}{2.456146in}}%
\pgfusepath{stroke}%
\end{pgfscope}%
\begin{pgfscope}%
\pgfpathrectangle{\pgfqpoint{1.250000in}{1.750000in}}{\pgfqpoint{2.279412in}{2.004545in}}%
\pgfusepath{clip}%
\pgfsetbuttcap%
\pgfsetroundjoin%
\pgfsetlinewidth{0.495057pt}%
\definecolor{currentstroke}{rgb}{0.281412,0.155834,0.469201}%
\pgfsetstrokecolor{currentstroke}%
\pgfsetdash{}{0pt}%
\pgfpathmoveto{\pgfqpoint{2.708517in}{2.456146in}}%
\pgfpathlineto{\pgfqpoint{2.658724in}{2.461378in}}%
\pgfusepath{stroke}%
\end{pgfscope}%
\begin{pgfscope}%
\pgfpathrectangle{\pgfqpoint{1.250000in}{1.750000in}}{\pgfqpoint{2.279412in}{2.004545in}}%
\pgfusepath{clip}%
\pgfsetbuttcap%
\pgfsetroundjoin%
\pgfsetlinewidth{0.503802pt}%
\definecolor{currentstroke}{rgb}{0.280868,0.160771,0.472899}%
\pgfsetstrokecolor{currentstroke}%
\pgfsetdash{}{0pt}%
\pgfpathmoveto{\pgfqpoint{2.658724in}{2.461378in}}%
\pgfpathlineto{\pgfqpoint{2.609159in}{2.468049in}}%
\pgfusepath{stroke}%
\end{pgfscope}%
\begin{pgfscope}%
\pgfpathrectangle{\pgfqpoint{1.250000in}{1.750000in}}{\pgfqpoint{2.279412in}{2.004545in}}%
\pgfusepath{clip}%
\pgfsetbuttcap%
\pgfsetroundjoin%
\pgfsetlinewidth{0.519015pt}%
\definecolor{currentstroke}{rgb}{0.279574,0.170599,0.479997}%
\pgfsetstrokecolor{currentstroke}%
\pgfsetdash{}{0pt}%
\pgfpathmoveto{\pgfqpoint{2.609159in}{2.468049in}}%
\pgfpathlineto{\pgfqpoint{2.559825in}{2.475962in}}%
\pgfusepath{stroke}%
\end{pgfscope}%
\begin{pgfscope}%
\pgfpathrectangle{\pgfqpoint{1.250000in}{1.750000in}}{\pgfqpoint{2.279412in}{2.004545in}}%
\pgfusepath{clip}%
\pgfsetbuttcap%
\pgfsetroundjoin%
\pgfsetlinewidth{0.501999pt}%
\definecolor{currentstroke}{rgb}{0.280868,0.160771,0.472899}%
\pgfsetstrokecolor{currentstroke}%
\pgfsetdash{}{0pt}%
\pgfpathmoveto{\pgfqpoint{2.559825in}{2.475962in}}%
\pgfpathlineto{\pgfqpoint{2.510903in}{2.485571in}}%
\pgfusepath{stroke}%
\end{pgfscope}%
\begin{pgfscope}%
\pgfpathrectangle{\pgfqpoint{1.250000in}{1.750000in}}{\pgfqpoint{2.279412in}{2.004545in}}%
\pgfusepath{clip}%
\pgfsetbuttcap%
\pgfsetroundjoin%
\pgfsetlinewidth{0.771507pt}%
\definecolor{currentstroke}{rgb}{0.221989,0.339161,0.548752}%
\pgfsetstrokecolor{currentstroke}%
\pgfsetdash{}{0pt}%
\pgfpathmoveto{\pgfqpoint{2.510903in}{2.485571in}}%
\pgfpathlineto{\pgfqpoint{2.462584in}{2.497336in}}%
\pgfusepath{stroke}%
\end{pgfscope}%
\begin{pgfscope}%
\pgfpathrectangle{\pgfqpoint{1.250000in}{1.750000in}}{\pgfqpoint{2.279412in}{2.004545in}}%
\pgfusepath{clip}%
\pgfsetbuttcap%
\pgfsetroundjoin%
\pgfsetlinewidth{0.953561pt}%
\definecolor{currentstroke}{rgb}{0.174274,0.445044,0.557792}%
\pgfsetstrokecolor{currentstroke}%
\pgfsetdash{}{0pt}%
\pgfpathmoveto{\pgfqpoint{2.462584in}{2.497336in}}%
\pgfpathlineto{\pgfqpoint{2.414883in}{2.510902in}}%
\pgfusepath{stroke}%
\end{pgfscope}%
\begin{pgfscope}%
\pgfpathrectangle{\pgfqpoint{1.250000in}{1.750000in}}{\pgfqpoint{2.279412in}{2.004545in}}%
\pgfusepath{clip}%
\pgfsetbuttcap%
\pgfsetroundjoin%
\pgfsetlinewidth{0.957089pt}%
\definecolor{currentstroke}{rgb}{0.174274,0.445044,0.557792}%
\pgfsetstrokecolor{currentstroke}%
\pgfsetdash{}{0pt}%
\pgfpathmoveto{\pgfqpoint{2.414883in}{2.510902in}}%
\pgfpathlineto{\pgfqpoint{2.367817in}{2.526102in}}%
\pgfusepath{stroke}%
\end{pgfscope}%
\begin{pgfscope}%
\pgfpathrectangle{\pgfqpoint{1.250000in}{1.750000in}}{\pgfqpoint{2.279412in}{2.004545in}}%
\pgfusepath{clip}%
\pgfsetbuttcap%
\pgfsetroundjoin%
\pgfsetlinewidth{0.919488pt}%
\definecolor{currentstroke}{rgb}{0.182256,0.426184,0.557120}%
\pgfsetstrokecolor{currentstroke}%
\pgfsetdash{}{0pt}%
\pgfpathmoveto{\pgfqpoint{2.367817in}{2.526102in}}%
\pgfpathlineto{\pgfqpoint{2.321501in}{2.542969in}}%
\pgfusepath{stroke}%
\end{pgfscope}%
\begin{pgfscope}%
\pgfpathrectangle{\pgfqpoint{1.250000in}{1.750000in}}{\pgfqpoint{2.279412in}{2.004545in}}%
\pgfusepath{clip}%
\pgfsetbuttcap%
\pgfsetroundjoin%
\pgfsetlinewidth{0.325625pt}%
\definecolor{currentstroke}{rgb}{0.271305,0.019942,0.347269}%
\pgfsetstrokecolor{currentstroke}%
\pgfsetdash{}{0pt}%
\pgfpathmoveto{\pgfqpoint{3.159084in}{2.571846in}}%
\pgfpathlineto{\pgfqpoint{3.108937in}{2.572434in}}%
\pgfusepath{stroke}%
\end{pgfscope}%
\begin{pgfscope}%
\pgfpathrectangle{\pgfqpoint{1.250000in}{1.750000in}}{\pgfqpoint{2.279412in}{2.004545in}}%
\pgfusepath{clip}%
\pgfsetbuttcap%
\pgfsetroundjoin%
\pgfsetlinewidth{0.336896pt}%
\definecolor{currentstroke}{rgb}{0.273809,0.031497,0.358853}%
\pgfsetstrokecolor{currentstroke}%
\pgfsetdash{}{0pt}%
\pgfpathmoveto{\pgfqpoint{3.108937in}{2.572434in}}%
\pgfpathlineto{\pgfqpoint{3.058797in}{2.573323in}}%
\pgfusepath{stroke}%
\end{pgfscope}%
\begin{pgfscope}%
\pgfpathrectangle{\pgfqpoint{1.250000in}{1.750000in}}{\pgfqpoint{2.279412in}{2.004545in}}%
\pgfusepath{clip}%
\pgfsetbuttcap%
\pgfsetroundjoin%
\pgfsetlinewidth{0.339672pt}%
\definecolor{currentstroke}{rgb}{0.273809,0.031497,0.358853}%
\pgfsetstrokecolor{currentstroke}%
\pgfsetdash{}{0pt}%
\pgfpathmoveto{\pgfqpoint{3.058797in}{2.573323in}}%
\pgfpathlineto{\pgfqpoint{3.008673in}{2.574647in}}%
\pgfusepath{stroke}%
\end{pgfscope}%
\begin{pgfscope}%
\pgfpathrectangle{\pgfqpoint{1.250000in}{1.750000in}}{\pgfqpoint{2.279412in}{2.004545in}}%
\pgfusepath{clip}%
\pgfsetbuttcap%
\pgfsetroundjoin%
\pgfsetlinewidth{0.353253pt}%
\definecolor{currentstroke}{rgb}{0.276022,0.044167,0.370164}%
\pgfsetstrokecolor{currentstroke}%
\pgfsetdash{}{0pt}%
\pgfpathmoveto{\pgfqpoint{3.008673in}{2.574647in}}%
\pgfpathlineto{\pgfqpoint{2.958552in}{2.576077in}}%
\pgfusepath{stroke}%
\end{pgfscope}%
\begin{pgfscope}%
\pgfpathrectangle{\pgfqpoint{1.250000in}{1.750000in}}{\pgfqpoint{2.279412in}{2.004545in}}%
\pgfusepath{clip}%
\pgfsetbuttcap%
\pgfsetroundjoin%
\pgfsetlinewidth{0.384848pt}%
\definecolor{currentstroke}{rgb}{0.280267,0.073417,0.397163}%
\pgfsetstrokecolor{currentstroke}%
\pgfsetdash{}{0pt}%
\pgfpathmoveto{\pgfqpoint{2.958552in}{2.576077in}}%
\pgfpathlineto{\pgfqpoint{2.908432in}{2.577626in}}%
\pgfusepath{stroke}%
\end{pgfscope}%
\begin{pgfscope}%
\pgfpathrectangle{\pgfqpoint{1.250000in}{1.750000in}}{\pgfqpoint{2.279412in}{2.004545in}}%
\pgfusepath{clip}%
\pgfsetbuttcap%
\pgfsetroundjoin%
\pgfsetlinewidth{0.407229pt}%
\definecolor{currentstroke}{rgb}{0.281924,0.089666,0.412415}%
\pgfsetstrokecolor{currentstroke}%
\pgfsetdash{}{0pt}%
\pgfpathmoveto{\pgfqpoint{2.908432in}{2.577626in}}%
\pgfpathlineto{\pgfqpoint{2.858312in}{2.579166in}}%
\pgfusepath{stroke}%
\end{pgfscope}%
\begin{pgfscope}%
\pgfpathrectangle{\pgfqpoint{1.250000in}{1.750000in}}{\pgfqpoint{2.279412in}{2.004545in}}%
\pgfusepath{clip}%
\pgfsetbuttcap%
\pgfsetroundjoin%
\pgfsetlinewidth{0.475888pt}%
\definecolor{currentstroke}{rgb}{0.282623,0.140926,0.457517}%
\pgfsetstrokecolor{currentstroke}%
\pgfsetdash{}{0pt}%
\pgfpathmoveto{\pgfqpoint{2.858312in}{2.579166in}}%
\pgfpathlineto{\pgfqpoint{2.808195in}{2.580766in}}%
\pgfusepath{stroke}%
\end{pgfscope}%
\begin{pgfscope}%
\pgfpathrectangle{\pgfqpoint{1.250000in}{1.750000in}}{\pgfqpoint{2.279412in}{2.004545in}}%
\pgfusepath{clip}%
\pgfsetbuttcap%
\pgfsetroundjoin%
\pgfsetlinewidth{0.551251pt}%
\definecolor{currentstroke}{rgb}{0.275191,0.194905,0.496005}%
\pgfsetstrokecolor{currentstroke}%
\pgfsetdash{}{0pt}%
\pgfpathmoveto{\pgfqpoint{2.808195in}{2.580766in}}%
\pgfpathlineto{\pgfqpoint{2.758106in}{2.582949in}}%
\pgfusepath{stroke}%
\end{pgfscope}%
\begin{pgfscope}%
\pgfpathrectangle{\pgfqpoint{1.250000in}{1.750000in}}{\pgfqpoint{2.279412in}{2.004545in}}%
\pgfusepath{clip}%
\pgfsetbuttcap%
\pgfsetroundjoin%
\pgfsetlinewidth{0.637944pt}%
\definecolor{currentstroke}{rgb}{0.257322,0.256130,0.526563}%
\pgfsetstrokecolor{currentstroke}%
\pgfsetdash{}{0pt}%
\pgfpathmoveto{\pgfqpoint{2.758106in}{2.582949in}}%
\pgfpathlineto{\pgfqpoint{2.708048in}{2.585634in}}%
\pgfusepath{stroke}%
\end{pgfscope}%
\begin{pgfscope}%
\pgfpathrectangle{\pgfqpoint{1.250000in}{1.750000in}}{\pgfqpoint{2.279412in}{2.004545in}}%
\pgfusepath{clip}%
\pgfsetbuttcap%
\pgfsetroundjoin%
\pgfsetlinewidth{0.765092pt}%
\definecolor{currentstroke}{rgb}{0.223925,0.334994,0.548053}%
\pgfsetstrokecolor{currentstroke}%
\pgfsetdash{}{0pt}%
\pgfpathmoveto{\pgfqpoint{2.708048in}{2.585634in}}%
\pgfpathlineto{\pgfqpoint{2.658038in}{2.588926in}}%
\pgfusepath{stroke}%
\end{pgfscope}%
\begin{pgfscope}%
\pgfpathrectangle{\pgfqpoint{1.250000in}{1.750000in}}{\pgfqpoint{2.279412in}{2.004545in}}%
\pgfusepath{clip}%
\pgfsetbuttcap%
\pgfsetroundjoin%
\pgfsetlinewidth{0.927791pt}%
\definecolor{currentstroke}{rgb}{0.180629,0.429975,0.557282}%
\pgfsetstrokecolor{currentstroke}%
\pgfsetdash{}{0pt}%
\pgfpathmoveto{\pgfqpoint{2.658038in}{2.588926in}}%
\pgfpathlineto{\pgfqpoint{2.608115in}{2.593104in}}%
\pgfusepath{stroke}%
\end{pgfscope}%
\begin{pgfscope}%
\pgfpathrectangle{\pgfqpoint{1.250000in}{1.750000in}}{\pgfqpoint{2.279412in}{2.004545in}}%
\pgfusepath{clip}%
\pgfsetbuttcap%
\pgfsetroundjoin%
\pgfsetlinewidth{0.322942pt}%
\definecolor{currentstroke}{rgb}{0.271305,0.019942,0.347269}%
\pgfsetstrokecolor{currentstroke}%
\pgfsetdash{}{0pt}%
\pgfpathmoveto{\pgfqpoint{3.159084in}{2.662059in}}%
\pgfpathlineto{\pgfqpoint{3.108952in}{2.662249in}}%
\pgfusepath{stroke}%
\end{pgfscope}%
\begin{pgfscope}%
\pgfpathrectangle{\pgfqpoint{1.250000in}{1.750000in}}{\pgfqpoint{2.279412in}{2.004545in}}%
\pgfusepath{clip}%
\pgfsetbuttcap%
\pgfsetroundjoin%
\pgfsetlinewidth{0.332303pt}%
\definecolor{currentstroke}{rgb}{0.272594,0.025563,0.353093}%
\pgfsetstrokecolor{currentstroke}%
\pgfsetdash{}{0pt}%
\pgfpathmoveto{\pgfqpoint{3.108952in}{2.662249in}}%
\pgfpathlineto{\pgfqpoint{3.058816in}{2.662221in}}%
\pgfusepath{stroke}%
\end{pgfscope}%
\begin{pgfscope}%
\pgfpathrectangle{\pgfqpoint{1.250000in}{1.750000in}}{\pgfqpoint{2.279412in}{2.004545in}}%
\pgfusepath{clip}%
\pgfsetbuttcap%
\pgfsetroundjoin%
\pgfsetlinewidth{0.339987pt}%
\definecolor{currentstroke}{rgb}{0.273809,0.031497,0.358853}%
\pgfsetstrokecolor{currentstroke}%
\pgfsetdash{}{0pt}%
\pgfpathmoveto{\pgfqpoint{3.058816in}{2.662221in}}%
\pgfpathlineto{\pgfqpoint{3.008668in}{2.662235in}}%
\pgfusepath{stroke}%
\end{pgfscope}%
\begin{pgfscope}%
\pgfpathrectangle{\pgfqpoint{1.250000in}{1.750000in}}{\pgfqpoint{2.279412in}{2.004545in}}%
\pgfusepath{clip}%
\pgfsetbuttcap%
\pgfsetroundjoin%
\pgfsetlinewidth{0.362895pt}%
\definecolor{currentstroke}{rgb}{0.277941,0.056324,0.381191}%
\pgfsetstrokecolor{currentstroke}%
\pgfsetdash{}{0pt}%
\pgfpathmoveto{\pgfqpoint{3.008668in}{2.662235in}}%
\pgfpathlineto{\pgfqpoint{2.958521in}{2.662051in}}%
\pgfusepath{stroke}%
\end{pgfscope}%
\begin{pgfscope}%
\pgfpathrectangle{\pgfqpoint{1.250000in}{1.750000in}}{\pgfqpoint{2.279412in}{2.004545in}}%
\pgfusepath{clip}%
\pgfsetbuttcap%
\pgfsetroundjoin%
\pgfsetlinewidth{0.393917pt}%
\definecolor{currentstroke}{rgb}{0.280894,0.078907,0.402329}%
\pgfsetstrokecolor{currentstroke}%
\pgfsetdash{}{0pt}%
\pgfpathmoveto{\pgfqpoint{2.958521in}{2.662051in}}%
\pgfpathlineto{\pgfqpoint{2.908373in}{2.662076in}}%
\pgfusepath{stroke}%
\end{pgfscope}%
\begin{pgfscope}%
\pgfpathrectangle{\pgfqpoint{1.250000in}{1.750000in}}{\pgfqpoint{2.279412in}{2.004545in}}%
\pgfusepath{clip}%
\pgfsetbuttcap%
\pgfsetroundjoin%
\pgfsetlinewidth{0.442860pt}%
\definecolor{currentstroke}{rgb}{0.283197,0.115680,0.436115}%
\pgfsetstrokecolor{currentstroke}%
\pgfsetdash{}{0pt}%
\pgfpathmoveto{\pgfqpoint{2.908373in}{2.662076in}}%
\pgfpathlineto{\pgfqpoint{2.858222in}{2.662221in}}%
\pgfusepath{stroke}%
\end{pgfscope}%
\begin{pgfscope}%
\pgfpathrectangle{\pgfqpoint{1.250000in}{1.750000in}}{\pgfqpoint{2.279412in}{2.004545in}}%
\pgfusepath{clip}%
\pgfsetbuttcap%
\pgfsetroundjoin%
\pgfsetlinewidth{0.509195pt}%
\definecolor{currentstroke}{rgb}{0.280255,0.165693,0.476498}%
\pgfsetstrokecolor{currentstroke}%
\pgfsetdash{}{0pt}%
\pgfpathmoveto{\pgfqpoint{2.858222in}{2.662221in}}%
\pgfpathlineto{\pgfqpoint{2.808073in}{2.662596in}}%
\pgfusepath{stroke}%
\end{pgfscope}%
\begin{pgfscope}%
\pgfpathrectangle{\pgfqpoint{1.250000in}{1.750000in}}{\pgfqpoint{2.279412in}{2.004545in}}%
\pgfusepath{clip}%
\pgfsetbuttcap%
\pgfsetroundjoin%
\pgfsetlinewidth{0.605382pt}%
\definecolor{currentstroke}{rgb}{0.265145,0.232956,0.516599}%
\pgfsetstrokecolor{currentstroke}%
\pgfsetdash{}{0pt}%
\pgfpathmoveto{\pgfqpoint{2.808073in}{2.662596in}}%
\pgfpathlineto{\pgfqpoint{2.757926in}{2.663192in}}%
\pgfusepath{stroke}%
\end{pgfscope}%
\begin{pgfscope}%
\pgfpathrectangle{\pgfqpoint{1.250000in}{1.750000in}}{\pgfqpoint{2.279412in}{2.004545in}}%
\pgfusepath{clip}%
\pgfsetbuttcap%
\pgfsetroundjoin%
\pgfsetlinewidth{0.765677pt}%
\definecolor{currentstroke}{rgb}{0.223925,0.334994,0.548053}%
\pgfsetstrokecolor{currentstroke}%
\pgfsetdash{}{0pt}%
\pgfpathmoveto{\pgfqpoint{2.757926in}{2.663192in}}%
\pgfpathlineto{\pgfqpoint{2.707787in}{2.664144in}}%
\pgfusepath{stroke}%
\end{pgfscope}%
\begin{pgfscope}%
\pgfpathrectangle{\pgfqpoint{1.250000in}{1.750000in}}{\pgfqpoint{2.279412in}{2.004545in}}%
\pgfusepath{clip}%
\pgfsetbuttcap%
\pgfsetroundjoin%
\pgfsetlinewidth{0.997099pt}%
\definecolor{currentstroke}{rgb}{0.165117,0.467423,0.558141}%
\pgfsetstrokecolor{currentstroke}%
\pgfsetdash{}{0pt}%
\pgfpathmoveto{\pgfqpoint{2.707787in}{2.664144in}}%
\pgfpathlineto{\pgfqpoint{2.657665in}{2.665638in}}%
\pgfusepath{stroke}%
\end{pgfscope}%
\begin{pgfscope}%
\pgfpathrectangle{\pgfqpoint{1.250000in}{1.750000in}}{\pgfqpoint{2.279412in}{2.004545in}}%
\pgfusepath{clip}%
\pgfsetbuttcap%
\pgfsetroundjoin%
\pgfsetlinewidth{1.252314pt}%
\definecolor{currentstroke}{rgb}{0.119738,0.603785,0.541400}%
\pgfsetstrokecolor{currentstroke}%
\pgfsetdash{}{0pt}%
\pgfpathmoveto{\pgfqpoint{2.657665in}{2.665638in}}%
\pgfpathlineto{\pgfqpoint{2.607566in}{2.667638in}}%
\pgfusepath{stroke}%
\end{pgfscope}%
\begin{pgfscope}%
\pgfpathrectangle{\pgfqpoint{1.250000in}{1.750000in}}{\pgfqpoint{2.279412in}{2.004545in}}%
\pgfusepath{clip}%
\pgfsetbuttcap%
\pgfsetroundjoin%
\pgfsetlinewidth{1.540257pt}%
\definecolor{currentstroke}{rgb}{0.274149,0.751988,0.436601}%
\pgfsetstrokecolor{currentstroke}%
\pgfsetdash{}{0pt}%
\pgfpathmoveto{\pgfqpoint{2.607566in}{2.667638in}}%
\pgfpathlineto{\pgfqpoint{2.557496in}{2.670123in}}%
\pgfusepath{stroke}%
\end{pgfscope}%
\begin{pgfscope}%
\pgfpathrectangle{\pgfqpoint{1.250000in}{1.750000in}}{\pgfqpoint{2.279412in}{2.004545in}}%
\pgfusepath{clip}%
\pgfsetbuttcap%
\pgfsetroundjoin%
\pgfsetlinewidth{0.337443pt}%
\definecolor{currentstroke}{rgb}{0.273809,0.031497,0.358853}%
\pgfsetstrokecolor{currentstroke}%
\pgfsetdash{}{0pt}%
\pgfpathmoveto{\pgfqpoint{3.159084in}{2.707166in}}%
\pgfpathlineto{\pgfqpoint{3.108952in}{2.708195in}}%
\pgfusepath{stroke}%
\end{pgfscope}%
\begin{pgfscope}%
\pgfpathrectangle{\pgfqpoint{1.250000in}{1.750000in}}{\pgfqpoint{2.279412in}{2.004545in}}%
\pgfusepath{clip}%
\pgfsetbuttcap%
\pgfsetroundjoin%
\pgfsetlinewidth{0.326715pt}%
\definecolor{currentstroke}{rgb}{0.271305,0.019942,0.347269}%
\pgfsetstrokecolor{currentstroke}%
\pgfsetdash{}{0pt}%
\pgfpathmoveto{\pgfqpoint{3.108952in}{2.708195in}}%
\pgfpathlineto{\pgfqpoint{3.058815in}{2.708701in}}%
\pgfusepath{stroke}%
\end{pgfscope}%
\begin{pgfscope}%
\pgfpathrectangle{\pgfqpoint{1.250000in}{1.750000in}}{\pgfqpoint{2.279412in}{2.004545in}}%
\pgfusepath{clip}%
\pgfsetbuttcap%
\pgfsetroundjoin%
\pgfsetlinewidth{0.340646pt}%
\definecolor{currentstroke}{rgb}{0.273809,0.031497,0.358853}%
\pgfsetstrokecolor{currentstroke}%
\pgfsetdash{}{0pt}%
\pgfpathmoveto{\pgfqpoint{3.058815in}{2.708701in}}%
\pgfpathlineto{\pgfqpoint{3.008671in}{2.708844in}}%
\pgfusepath{stroke}%
\end{pgfscope}%
\begin{pgfscope}%
\pgfpathrectangle{\pgfqpoint{1.250000in}{1.750000in}}{\pgfqpoint{2.279412in}{2.004545in}}%
\pgfusepath{clip}%
\pgfsetbuttcap%
\pgfsetroundjoin%
\pgfsetlinewidth{0.360706pt}%
\definecolor{currentstroke}{rgb}{0.277018,0.050344,0.375715}%
\pgfsetstrokecolor{currentstroke}%
\pgfsetdash{}{0pt}%
\pgfpathmoveto{\pgfqpoint{3.008671in}{2.708844in}}%
\pgfpathlineto{\pgfqpoint{2.958529in}{2.709189in}}%
\pgfusepath{stroke}%
\end{pgfscope}%
\begin{pgfscope}%
\pgfpathrectangle{\pgfqpoint{1.250000in}{1.750000in}}{\pgfqpoint{2.279412in}{2.004545in}}%
\pgfusepath{clip}%
\pgfsetbuttcap%
\pgfsetroundjoin%
\pgfsetlinewidth{0.388634pt}%
\definecolor{currentstroke}{rgb}{0.280267,0.073417,0.397163}%
\pgfsetstrokecolor{currentstroke}%
\pgfsetdash{}{0pt}%
\pgfpathmoveto{\pgfqpoint{2.958529in}{2.709189in}}%
\pgfpathlineto{\pgfqpoint{2.908383in}{2.709640in}}%
\pgfusepath{stroke}%
\end{pgfscope}%
\begin{pgfscope}%
\pgfpathrectangle{\pgfqpoint{1.250000in}{1.750000in}}{\pgfqpoint{2.279412in}{2.004545in}}%
\pgfusepath{clip}%
\pgfsetbuttcap%
\pgfsetroundjoin%
\pgfsetlinewidth{0.447725pt}%
\definecolor{currentstroke}{rgb}{0.283229,0.120777,0.440584}%
\pgfsetstrokecolor{currentstroke}%
\pgfsetdash{}{0pt}%
\pgfpathmoveto{\pgfqpoint{2.908383in}{2.709640in}}%
\pgfpathlineto{\pgfqpoint{2.858235in}{2.710177in}}%
\pgfusepath{stroke}%
\end{pgfscope}%
\begin{pgfscope}%
\pgfpathrectangle{\pgfqpoint{1.250000in}{1.750000in}}{\pgfqpoint{2.279412in}{2.004545in}}%
\pgfusepath{clip}%
\pgfsetbuttcap%
\pgfsetroundjoin%
\pgfsetlinewidth{0.490309pt}%
\definecolor{currentstroke}{rgb}{0.281887,0.150881,0.465405}%
\pgfsetstrokecolor{currentstroke}%
\pgfsetdash{}{0pt}%
\pgfpathmoveto{\pgfqpoint{2.858235in}{2.710177in}}%
\pgfpathlineto{\pgfqpoint{2.808086in}{2.710591in}}%
\pgfusepath{stroke}%
\end{pgfscope}%
\begin{pgfscope}%
\pgfpathrectangle{\pgfqpoint{1.250000in}{1.750000in}}{\pgfqpoint{2.279412in}{2.004545in}}%
\pgfusepath{clip}%
\pgfsetbuttcap%
\pgfsetroundjoin%
\pgfsetlinewidth{0.621973pt}%
\definecolor{currentstroke}{rgb}{0.262138,0.242286,0.520837}%
\pgfsetstrokecolor{currentstroke}%
\pgfsetdash{}{0pt}%
\pgfpathmoveto{\pgfqpoint{2.808086in}{2.710591in}}%
\pgfpathlineto{\pgfqpoint{2.757938in}{2.711154in}}%
\pgfusepath{stroke}%
\end{pgfscope}%
\begin{pgfscope}%
\pgfpathrectangle{\pgfqpoint{1.250000in}{1.750000in}}{\pgfqpoint{2.279412in}{2.004545in}}%
\pgfusepath{clip}%
\pgfsetbuttcap%
\pgfsetroundjoin%
\pgfsetlinewidth{0.826513pt}%
\definecolor{currentstroke}{rgb}{0.206756,0.371758,0.553117}%
\pgfsetstrokecolor{currentstroke}%
\pgfsetdash{}{0pt}%
\pgfpathmoveto{\pgfqpoint{2.757938in}{2.711154in}}%
\pgfpathlineto{\pgfqpoint{2.707792in}{2.711862in}}%
\pgfusepath{stroke}%
\end{pgfscope}%
\begin{pgfscope}%
\pgfpathrectangle{\pgfqpoint{1.250000in}{1.750000in}}{\pgfqpoint{2.279412in}{2.004545in}}%
\pgfusepath{clip}%
\pgfsetbuttcap%
\pgfsetroundjoin%
\pgfsetlinewidth{1.077231pt}%
\definecolor{currentstroke}{rgb}{0.147607,0.511733,0.557049}%
\pgfsetstrokecolor{currentstroke}%
\pgfsetdash{}{0pt}%
\pgfpathmoveto{\pgfqpoint{2.707792in}{2.711862in}}%
\pgfpathlineto{\pgfqpoint{2.657649in}{2.712696in}}%
\pgfusepath{stroke}%
\end{pgfscope}%
\begin{pgfscope}%
\pgfpathrectangle{\pgfqpoint{1.250000in}{1.750000in}}{\pgfqpoint{2.279412in}{2.004545in}}%
\pgfusepath{clip}%
\pgfsetbuttcap%
\pgfsetroundjoin%
\pgfsetlinewidth{1.409190pt}%
\definecolor{currentstroke}{rgb}{0.162016,0.687316,0.499129}%
\pgfsetstrokecolor{currentstroke}%
\pgfsetdash{}{0pt}%
\pgfpathmoveto{\pgfqpoint{2.657649in}{2.712696in}}%
\pgfpathlineto{\pgfqpoint{2.607511in}{2.713708in}}%
\pgfusepath{stroke}%
\end{pgfscope}%
\begin{pgfscope}%
\pgfpathrectangle{\pgfqpoint{1.250000in}{1.750000in}}{\pgfqpoint{2.279412in}{2.004545in}}%
\pgfusepath{clip}%
\pgfsetbuttcap%
\pgfsetroundjoin%
\pgfsetlinewidth{1.618036pt}%
\definecolor{currentstroke}{rgb}{0.360741,0.785964,0.387814}%
\pgfsetstrokecolor{currentstroke}%
\pgfsetdash{}{0pt}%
\pgfpathmoveto{\pgfqpoint{2.607511in}{2.713708in}}%
\pgfpathlineto{\pgfqpoint{2.557378in}{2.714899in}}%
\pgfusepath{stroke}%
\end{pgfscope}%
\begin{pgfscope}%
\pgfpathrectangle{\pgfqpoint{1.250000in}{1.750000in}}{\pgfqpoint{2.279412in}{2.004545in}}%
\pgfusepath{clip}%
\pgfsetbuttcap%
\pgfsetroundjoin%
\pgfsetlinewidth{1.846552pt}%
\definecolor{currentstroke}{rgb}{0.688944,0.865448,0.182725}%
\pgfsetstrokecolor{currentstroke}%
\pgfsetdash{}{0pt}%
\pgfpathmoveto{\pgfqpoint{2.557378in}{2.714899in}}%
\pgfpathlineto{\pgfqpoint{2.507250in}{2.716229in}}%
\pgfusepath{stroke}%
\end{pgfscope}%
\begin{pgfscope}%
\pgfpathrectangle{\pgfqpoint{1.250000in}{1.750000in}}{\pgfqpoint{2.279412in}{2.004545in}}%
\pgfusepath{clip}%
\pgfsetbuttcap%
\pgfsetroundjoin%
\pgfsetlinewidth{2.123266pt}%
\definecolor{currentstroke}{rgb}{0.993248,0.906157,0.143936}%
\pgfsetstrokecolor{currentstroke}%
\pgfsetdash{}{0pt}%
\pgfpathmoveto{\pgfqpoint{2.507250in}{2.716229in}}%
\pgfpathlineto{\pgfqpoint{2.457130in}{2.717730in}}%
\pgfusepath{stroke}%
\end{pgfscope}%
\begin{pgfscope}%
\pgfpathrectangle{\pgfqpoint{1.250000in}{1.750000in}}{\pgfqpoint{2.279412in}{2.004545in}}%
\pgfusepath{clip}%
\pgfsetbuttcap%
\pgfsetroundjoin%
\pgfsetlinewidth{2.302834pt}%
\definecolor{currentstroke}{rgb}{0.993248,0.906157,0.143936}%
\pgfsetstrokecolor{currentstroke}%
\pgfsetdash{}{0pt}%
\pgfpathmoveto{\pgfqpoint{2.457130in}{2.717730in}}%
\pgfpathlineto{\pgfqpoint{2.407018in}{2.719399in}}%
\pgfusepath{stroke}%
\end{pgfscope}%
\begin{pgfscope}%
\pgfpathrectangle{\pgfqpoint{1.250000in}{1.750000in}}{\pgfqpoint{2.279412in}{2.004545in}}%
\pgfusepath{clip}%
\pgfsetbuttcap%
\pgfsetroundjoin%
\pgfsetlinewidth{2.370850pt}%
\definecolor{currentstroke}{rgb}{0.993248,0.906157,0.143936}%
\pgfsetstrokecolor{currentstroke}%
\pgfsetdash{}{0pt}%
\pgfpathmoveto{\pgfqpoint{2.407018in}{2.719399in}}%
\pgfpathlineto{\pgfqpoint{2.356917in}{2.721244in}}%
\pgfusepath{stroke}%
\end{pgfscope}%
\begin{pgfscope}%
\pgfpathrectangle{\pgfqpoint{1.250000in}{1.750000in}}{\pgfqpoint{2.279412in}{2.004545in}}%
\pgfusepath{clip}%
\pgfsetbuttcap%
\pgfsetroundjoin%
\pgfsetlinewidth{2.339015pt}%
\definecolor{currentstroke}{rgb}{0.993248,0.906157,0.143936}%
\pgfsetstrokecolor{currentstroke}%
\pgfsetdash{}{0pt}%
\pgfpathmoveto{\pgfqpoint{2.356917in}{2.721244in}}%
\pgfpathlineto{\pgfqpoint{2.306825in}{2.723205in}}%
\pgfusepath{stroke}%
\end{pgfscope}%
\begin{pgfscope}%
\pgfpathrectangle{\pgfqpoint{1.250000in}{1.750000in}}{\pgfqpoint{2.279412in}{2.004545in}}%
\pgfusepath{clip}%
\pgfsetbuttcap%
\pgfsetroundjoin%
\pgfsetlinewidth{2.277781pt}%
\definecolor{currentstroke}{rgb}{0.993248,0.906157,0.143936}%
\pgfsetstrokecolor{currentstroke}%
\pgfsetdash{}{0pt}%
\pgfpathmoveto{\pgfqpoint{2.306825in}{2.723205in}}%
\pgfpathlineto{\pgfqpoint{2.256743in}{2.725312in}}%
\pgfusepath{stroke}%
\end{pgfscope}%
\begin{pgfscope}%
\pgfpathrectangle{\pgfqpoint{1.250000in}{1.750000in}}{\pgfqpoint{2.279412in}{2.004545in}}%
\pgfusepath{clip}%
\pgfsetbuttcap%
\pgfsetroundjoin%
\pgfsetlinewidth{0.327881pt}%
\definecolor{currentstroke}{rgb}{0.271305,0.019942,0.347269}%
\pgfsetstrokecolor{currentstroke}%
\pgfsetdash{}{0pt}%
\pgfpathmoveto{\pgfqpoint{3.159084in}{2.797380in}}%
\pgfpathlineto{\pgfqpoint{3.108942in}{2.796819in}}%
\pgfusepath{stroke}%
\end{pgfscope}%
\begin{pgfscope}%
\pgfpathrectangle{\pgfqpoint{1.250000in}{1.750000in}}{\pgfqpoint{2.279412in}{2.004545in}}%
\pgfusepath{clip}%
\pgfsetbuttcap%
\pgfsetroundjoin%
\pgfsetlinewidth{0.332204pt}%
\definecolor{currentstroke}{rgb}{0.272594,0.025563,0.353093}%
\pgfsetstrokecolor{currentstroke}%
\pgfsetdash{}{0pt}%
\pgfpathmoveto{\pgfqpoint{3.108942in}{2.796819in}}%
\pgfpathlineto{\pgfqpoint{3.058795in}{2.796324in}}%
\pgfusepath{stroke}%
\end{pgfscope}%
\begin{pgfscope}%
\pgfpathrectangle{\pgfqpoint{1.250000in}{1.750000in}}{\pgfqpoint{2.279412in}{2.004545in}}%
\pgfusepath{clip}%
\pgfsetbuttcap%
\pgfsetroundjoin%
\pgfsetlinewidth{0.338061pt}%
\definecolor{currentstroke}{rgb}{0.273809,0.031497,0.358853}%
\pgfsetstrokecolor{currentstroke}%
\pgfsetdash{}{0pt}%
\pgfpathmoveto{\pgfqpoint{3.058795in}{2.796324in}}%
\pgfpathlineto{\pgfqpoint{3.008645in}{2.796064in}}%
\pgfusepath{stroke}%
\end{pgfscope}%
\begin{pgfscope}%
\pgfpathrectangle{\pgfqpoint{1.250000in}{1.750000in}}{\pgfqpoint{2.279412in}{2.004545in}}%
\pgfusepath{clip}%
\pgfsetbuttcap%
\pgfsetroundjoin%
\pgfsetlinewidth{0.362589pt}%
\definecolor{currentstroke}{rgb}{0.277018,0.050344,0.375715}%
\pgfsetstrokecolor{currentstroke}%
\pgfsetdash{}{0pt}%
\pgfpathmoveto{\pgfqpoint{3.008645in}{2.796064in}}%
\pgfpathlineto{\pgfqpoint{2.958494in}{2.796051in}}%
\pgfusepath{stroke}%
\end{pgfscope}%
\begin{pgfscope}%
\pgfpathrectangle{\pgfqpoint{1.250000in}{1.750000in}}{\pgfqpoint{2.279412in}{2.004545in}}%
\pgfusepath{clip}%
\pgfsetbuttcap%
\pgfsetroundjoin%
\pgfsetlinewidth{0.395319pt}%
\definecolor{currentstroke}{rgb}{0.280894,0.078907,0.402329}%
\pgfsetstrokecolor{currentstroke}%
\pgfsetdash{}{0pt}%
\pgfpathmoveto{\pgfqpoint{2.958494in}{2.796051in}}%
\pgfpathlineto{\pgfqpoint{2.908345in}{2.795939in}}%
\pgfusepath{stroke}%
\end{pgfscope}%
\begin{pgfscope}%
\pgfpathrectangle{\pgfqpoint{1.250000in}{1.750000in}}{\pgfqpoint{2.279412in}{2.004545in}}%
\pgfusepath{clip}%
\pgfsetbuttcap%
\pgfsetroundjoin%
\pgfsetlinewidth{0.440638pt}%
\definecolor{currentstroke}{rgb}{0.283197,0.115680,0.436115}%
\pgfsetstrokecolor{currentstroke}%
\pgfsetdash{}{0pt}%
\pgfpathmoveto{\pgfqpoint{2.908345in}{2.795939in}}%
\pgfpathlineto{\pgfqpoint{2.858196in}{2.795562in}}%
\pgfusepath{stroke}%
\end{pgfscope}%
\begin{pgfscope}%
\pgfpathrectangle{\pgfqpoint{1.250000in}{1.750000in}}{\pgfqpoint{2.279412in}{2.004545in}}%
\pgfusepath{clip}%
\pgfsetbuttcap%
\pgfsetroundjoin%
\pgfsetlinewidth{0.510114pt}%
\definecolor{currentstroke}{rgb}{0.280255,0.165693,0.476498}%
\pgfsetstrokecolor{currentstroke}%
\pgfsetdash{}{0pt}%
\pgfpathmoveto{\pgfqpoint{2.858196in}{2.795562in}}%
\pgfpathlineto{\pgfqpoint{2.808046in}{2.795197in}}%
\pgfusepath{stroke}%
\end{pgfscope}%
\begin{pgfscope}%
\pgfpathrectangle{\pgfqpoint{1.250000in}{1.750000in}}{\pgfqpoint{2.279412in}{2.004545in}}%
\pgfusepath{clip}%
\pgfsetbuttcap%
\pgfsetroundjoin%
\pgfsetlinewidth{0.619922pt}%
\definecolor{currentstroke}{rgb}{0.262138,0.242286,0.520837}%
\pgfsetstrokecolor{currentstroke}%
\pgfsetdash{}{0pt}%
\pgfpathmoveto{\pgfqpoint{2.808046in}{2.795197in}}%
\pgfpathlineto{\pgfqpoint{2.757897in}{2.794731in}}%
\pgfusepath{stroke}%
\end{pgfscope}%
\begin{pgfscope}%
\pgfpathrectangle{\pgfqpoint{1.250000in}{1.750000in}}{\pgfqpoint{2.279412in}{2.004545in}}%
\pgfusepath{clip}%
\pgfsetbuttcap%
\pgfsetroundjoin%
\pgfsetlinewidth{0.797722pt}%
\definecolor{currentstroke}{rgb}{0.214298,0.355619,0.551184}%
\pgfsetstrokecolor{currentstroke}%
\pgfsetdash{}{0pt}%
\pgfpathmoveto{\pgfqpoint{2.757897in}{2.794731in}}%
\pgfpathlineto{\pgfqpoint{2.707750in}{2.794110in}}%
\pgfusepath{stroke}%
\end{pgfscope}%
\begin{pgfscope}%
\pgfpathrectangle{\pgfqpoint{1.250000in}{1.750000in}}{\pgfqpoint{2.279412in}{2.004545in}}%
\pgfusepath{clip}%
\pgfsetbuttcap%
\pgfsetroundjoin%
\pgfsetlinewidth{1.071473pt}%
\definecolor{currentstroke}{rgb}{0.149039,0.508051,0.557250}%
\pgfsetstrokecolor{currentstroke}%
\pgfsetdash{}{0pt}%
\pgfpathmoveto{\pgfqpoint{2.707750in}{2.794110in}}%
\pgfpathlineto{\pgfqpoint{2.657611in}{2.793139in}}%
\pgfusepath{stroke}%
\end{pgfscope}%
\begin{pgfscope}%
\pgfpathrectangle{\pgfqpoint{1.250000in}{1.750000in}}{\pgfqpoint{2.279412in}{2.004545in}}%
\pgfusepath{clip}%
\pgfsetbuttcap%
\pgfsetroundjoin%
\pgfsetlinewidth{1.333162pt}%
\definecolor{currentstroke}{rgb}{0.128087,0.647749,0.523491}%
\pgfsetstrokecolor{currentstroke}%
\pgfsetdash{}{0pt}%
\pgfpathmoveto{\pgfqpoint{2.657611in}{2.793139in}}%
\pgfpathlineto{\pgfqpoint{2.607482in}{2.791806in}}%
\pgfusepath{stroke}%
\end{pgfscope}%
\begin{pgfscope}%
\pgfpathrectangle{\pgfqpoint{1.250000in}{1.750000in}}{\pgfqpoint{2.279412in}{2.004545in}}%
\pgfusepath{clip}%
\pgfsetbuttcap%
\pgfsetroundjoin%
\pgfsetlinewidth{0.314298pt}%
\definecolor{currentstroke}{rgb}{0.268510,0.009605,0.335427}%
\pgfsetstrokecolor{currentstroke}%
\pgfsetdash{}{0pt}%
\pgfpathmoveto{\pgfqpoint{3.159084in}{2.932700in}}%
\pgfpathlineto{\pgfqpoint{3.109043in}{2.934071in}}%
\pgfusepath{stroke}%
\end{pgfscope}%
\begin{pgfscope}%
\pgfpathrectangle{\pgfqpoint{1.250000in}{1.750000in}}{\pgfqpoint{2.279412in}{2.004545in}}%
\pgfusepath{clip}%
\pgfsetbuttcap%
\pgfsetroundjoin%
\pgfsetlinewidth{0.328112pt}%
\definecolor{currentstroke}{rgb}{0.271305,0.019942,0.347269}%
\pgfsetstrokecolor{currentstroke}%
\pgfsetdash{}{0pt}%
\pgfpathmoveto{\pgfqpoint{3.109043in}{2.934071in}}%
\pgfpathlineto{\pgfqpoint{3.058925in}{2.933217in}}%
\pgfusepath{stroke}%
\end{pgfscope}%
\begin{pgfscope}%
\pgfpathrectangle{\pgfqpoint{1.250000in}{1.750000in}}{\pgfqpoint{2.279412in}{2.004545in}}%
\pgfusepath{clip}%
\pgfsetbuttcap%
\pgfsetroundjoin%
\pgfsetlinewidth{0.345706pt}%
\definecolor{currentstroke}{rgb}{0.274952,0.037752,0.364543}%
\pgfsetstrokecolor{currentstroke}%
\pgfsetdash{}{0pt}%
\pgfpathmoveto{\pgfqpoint{3.058925in}{2.933217in}}%
\pgfpathlineto{\pgfqpoint{3.008826in}{2.931363in}}%
\pgfusepath{stroke}%
\end{pgfscope}%
\begin{pgfscope}%
\pgfpathrectangle{\pgfqpoint{1.250000in}{1.750000in}}{\pgfqpoint{2.279412in}{2.004545in}}%
\pgfusepath{clip}%
\pgfsetbuttcap%
\pgfsetroundjoin%
\pgfsetlinewidth{0.359223pt}%
\definecolor{currentstroke}{rgb}{0.277018,0.050344,0.375715}%
\pgfsetstrokecolor{currentstroke}%
\pgfsetdash{}{0pt}%
\pgfpathmoveto{\pgfqpoint{3.008826in}{2.931363in}}%
\pgfpathlineto{\pgfqpoint{2.958711in}{2.929818in}}%
\pgfusepath{stroke}%
\end{pgfscope}%
\begin{pgfscope}%
\pgfpathrectangle{\pgfqpoint{1.250000in}{1.750000in}}{\pgfqpoint{2.279412in}{2.004545in}}%
\pgfusepath{clip}%
\pgfsetbuttcap%
\pgfsetroundjoin%
\pgfsetlinewidth{0.383343pt}%
\definecolor{currentstroke}{rgb}{0.279566,0.067836,0.391917}%
\pgfsetstrokecolor{currentstroke}%
\pgfsetdash{}{0pt}%
\pgfpathmoveto{\pgfqpoint{2.958711in}{2.929818in}}%
\pgfpathlineto{\pgfqpoint{2.908573in}{2.928796in}}%
\pgfusepath{stroke}%
\end{pgfscope}%
\begin{pgfscope}%
\pgfpathrectangle{\pgfqpoint{1.250000in}{1.750000in}}{\pgfqpoint{2.279412in}{2.004545in}}%
\pgfusepath{clip}%
\pgfsetbuttcap%
\pgfsetroundjoin%
\pgfsetlinewidth{0.415120pt}%
\definecolor{currentstroke}{rgb}{0.282327,0.094955,0.417331}%
\pgfsetstrokecolor{currentstroke}%
\pgfsetdash{}{0pt}%
\pgfpathmoveto{\pgfqpoint{2.908573in}{2.928796in}}%
\pgfpathlineto{\pgfqpoint{2.858447in}{2.927405in}}%
\pgfusepath{stroke}%
\end{pgfscope}%
\begin{pgfscope}%
\pgfpathrectangle{\pgfqpoint{1.250000in}{1.750000in}}{\pgfqpoint{2.279412in}{2.004545in}}%
\pgfusepath{clip}%
\pgfsetbuttcap%
\pgfsetroundjoin%
\pgfsetlinewidth{0.446007pt}%
\definecolor{currentstroke}{rgb}{0.283229,0.120777,0.440584}%
\pgfsetstrokecolor{currentstroke}%
\pgfsetdash{}{0pt}%
\pgfpathmoveto{\pgfqpoint{2.858447in}{2.927405in}}%
\pgfpathlineto{\pgfqpoint{2.808334in}{2.925671in}}%
\pgfusepath{stroke}%
\end{pgfscope}%
\begin{pgfscope}%
\pgfpathrectangle{\pgfqpoint{1.250000in}{1.750000in}}{\pgfqpoint{2.279412in}{2.004545in}}%
\pgfusepath{clip}%
\pgfsetbuttcap%
\pgfsetroundjoin%
\pgfsetlinewidth{0.514840pt}%
\definecolor{currentstroke}{rgb}{0.279574,0.170599,0.479997}%
\pgfsetstrokecolor{currentstroke}%
\pgfsetdash{}{0pt}%
\pgfpathmoveto{\pgfqpoint{2.808334in}{2.925671in}}%
\pgfpathlineto{\pgfqpoint{2.758232in}{2.923727in}}%
\pgfusepath{stroke}%
\end{pgfscope}%
\begin{pgfscope}%
\pgfpathrectangle{\pgfqpoint{1.250000in}{1.750000in}}{\pgfqpoint{2.279412in}{2.004545in}}%
\pgfusepath{clip}%
\pgfsetbuttcap%
\pgfsetroundjoin%
\pgfsetlinewidth{0.594575pt}%
\definecolor{currentstroke}{rgb}{0.267968,0.223549,0.512008}%
\pgfsetstrokecolor{currentstroke}%
\pgfsetdash{}{0pt}%
\pgfpathmoveto{\pgfqpoint{2.758232in}{2.923727in}}%
\pgfpathlineto{\pgfqpoint{2.708161in}{2.921253in}}%
\pgfusepath{stroke}%
\end{pgfscope}%
\begin{pgfscope}%
\pgfpathrectangle{\pgfqpoint{1.250000in}{1.750000in}}{\pgfqpoint{2.279412in}{2.004545in}}%
\pgfusepath{clip}%
\pgfsetbuttcap%
\pgfsetroundjoin%
\pgfsetlinewidth{0.708632pt}%
\definecolor{currentstroke}{rgb}{0.239346,0.300855,0.540844}%
\pgfsetstrokecolor{currentstroke}%
\pgfsetdash{}{0pt}%
\pgfpathmoveto{\pgfqpoint{2.708161in}{2.921253in}}%
\pgfpathlineto{\pgfqpoint{2.658161in}{2.917868in}}%
\pgfusepath{stroke}%
\end{pgfscope}%
\begin{pgfscope}%
\pgfpathrectangle{\pgfqpoint{1.250000in}{1.750000in}}{\pgfqpoint{2.279412in}{2.004545in}}%
\pgfusepath{clip}%
\pgfsetbuttcap%
\pgfsetroundjoin%
\pgfsetlinewidth{0.870849pt}%
\definecolor{currentstroke}{rgb}{0.194100,0.399323,0.555565}%
\pgfsetstrokecolor{currentstroke}%
\pgfsetdash{}{0pt}%
\pgfpathmoveto{\pgfqpoint{2.658161in}{2.917868in}}%
\pgfpathlineto{\pgfqpoint{2.608282in}{2.913319in}}%
\pgfusepath{stroke}%
\end{pgfscope}%
\begin{pgfscope}%
\pgfpathrectangle{\pgfqpoint{1.250000in}{1.750000in}}{\pgfqpoint{2.279412in}{2.004545in}}%
\pgfusepath{clip}%
\pgfsetbuttcap%
\pgfsetroundjoin%
\pgfsetlinewidth{1.045579pt}%
\definecolor{currentstroke}{rgb}{0.154815,0.493313,0.557840}%
\pgfsetstrokecolor{currentstroke}%
\pgfsetdash{}{0pt}%
\pgfpathmoveto{\pgfqpoint{2.608282in}{2.913319in}}%
\pgfpathlineto{\pgfqpoint{2.558537in}{2.907733in}}%
\pgfusepath{stroke}%
\end{pgfscope}%
\begin{pgfscope}%
\pgfpathrectangle{\pgfqpoint{1.250000in}{1.750000in}}{\pgfqpoint{2.279412in}{2.004545in}}%
\pgfusepath{clip}%
\pgfsetbuttcap%
\pgfsetroundjoin%
\pgfsetlinewidth{1.252847pt}%
\definecolor{currentstroke}{rgb}{0.119738,0.603785,0.541400}%
\pgfsetstrokecolor{currentstroke}%
\pgfsetdash{}{0pt}%
\pgfpathmoveto{\pgfqpoint{2.558537in}{2.907733in}}%
\pgfpathlineto{\pgfqpoint{2.508944in}{2.901200in}}%
\pgfusepath{stroke}%
\end{pgfscope}%
\begin{pgfscope}%
\pgfpathrectangle{\pgfqpoint{1.250000in}{1.750000in}}{\pgfqpoint{2.279412in}{2.004545in}}%
\pgfusepath{clip}%
\pgfsetbuttcap%
\pgfsetroundjoin%
\pgfsetlinewidth{1.265483pt}%
\definecolor{currentstroke}{rgb}{0.119423,0.611141,0.538982}%
\pgfsetstrokecolor{currentstroke}%
\pgfsetdash{}{0pt}%
\pgfpathmoveto{\pgfqpoint{2.508944in}{2.901200in}}%
\pgfpathlineto{\pgfqpoint{2.459570in}{2.893503in}}%
\pgfusepath{stroke}%
\end{pgfscope}%
\begin{pgfscope}%
\pgfpathrectangle{\pgfqpoint{1.250000in}{1.750000in}}{\pgfqpoint{2.279412in}{2.004545in}}%
\pgfusepath{clip}%
\pgfsetbuttcap%
\pgfsetroundjoin%
\pgfsetlinewidth{1.254705pt}%
\definecolor{currentstroke}{rgb}{0.119738,0.603785,0.541400}%
\pgfsetstrokecolor{currentstroke}%
\pgfsetdash{}{0pt}%
\pgfpathmoveto{\pgfqpoint{2.459570in}{2.893503in}}%
\pgfpathlineto{\pgfqpoint{2.410435in}{2.884690in}}%
\pgfusepath{stroke}%
\end{pgfscope}%
\begin{pgfscope}%
\pgfpathrectangle{\pgfqpoint{1.250000in}{1.750000in}}{\pgfqpoint{2.279412in}{2.004545in}}%
\pgfusepath{clip}%
\pgfsetbuttcap%
\pgfsetroundjoin%
\pgfsetlinewidth{1.650091pt}%
\definecolor{currentstroke}{rgb}{0.404001,0.800275,0.362552}%
\pgfsetstrokecolor{currentstroke}%
\pgfsetdash{}{0pt}%
\pgfpathmoveto{\pgfqpoint{2.410435in}{2.884690in}}%
\pgfpathlineto{\pgfqpoint{2.361531in}{2.874940in}}%
\pgfusepath{stroke}%
\end{pgfscope}%
\begin{pgfscope}%
\pgfpathrectangle{\pgfqpoint{1.250000in}{1.750000in}}{\pgfqpoint{2.279412in}{2.004545in}}%
\pgfusepath{clip}%
\pgfsetbuttcap%
\pgfsetroundjoin%
\pgfsetlinewidth{0.312280pt}%
\definecolor{currentstroke}{rgb}{0.268510,0.009605,0.335427}%
\pgfsetstrokecolor{currentstroke}%
\pgfsetdash{}{0pt}%
\pgfpathmoveto{\pgfqpoint{3.159084in}{2.977807in}}%
\pgfpathlineto{\pgfqpoint{3.108995in}{2.976406in}}%
\pgfusepath{stroke}%
\end{pgfscope}%
\begin{pgfscope}%
\pgfpathrectangle{\pgfqpoint{1.250000in}{1.750000in}}{\pgfqpoint{2.279412in}{2.004545in}}%
\pgfusepath{clip}%
\pgfsetbuttcap%
\pgfsetroundjoin%
\pgfsetlinewidth{0.328858pt}%
\definecolor{currentstroke}{rgb}{0.272594,0.025563,0.353093}%
\pgfsetstrokecolor{currentstroke}%
\pgfsetdash{}{0pt}%
\pgfpathmoveto{\pgfqpoint{3.108995in}{2.976406in}}%
\pgfpathlineto{\pgfqpoint{3.058857in}{2.975895in}}%
\pgfusepath{stroke}%
\end{pgfscope}%
\begin{pgfscope}%
\pgfpathrectangle{\pgfqpoint{1.250000in}{1.750000in}}{\pgfqpoint{2.279412in}{2.004545in}}%
\pgfusepath{clip}%
\pgfsetbuttcap%
\pgfsetroundjoin%
\pgfsetlinewidth{0.335713pt}%
\definecolor{currentstroke}{rgb}{0.273809,0.031497,0.358853}%
\pgfsetstrokecolor{currentstroke}%
\pgfsetdash{}{0pt}%
\pgfpathmoveto{\pgfqpoint{3.058857in}{2.975895in}}%
\pgfpathlineto{\pgfqpoint{3.008735in}{2.974478in}}%
\pgfusepath{stroke}%
\end{pgfscope}%
\begin{pgfscope}%
\pgfpathrectangle{\pgfqpoint{1.250000in}{1.750000in}}{\pgfqpoint{2.279412in}{2.004545in}}%
\pgfusepath{clip}%
\pgfsetbuttcap%
\pgfsetroundjoin%
\pgfsetlinewidth{0.354652pt}%
\definecolor{currentstroke}{rgb}{0.276022,0.044167,0.370164}%
\pgfsetstrokecolor{currentstroke}%
\pgfsetdash{}{0pt}%
\pgfpathmoveto{\pgfqpoint{3.008735in}{2.974478in}}%
\pgfpathlineto{\pgfqpoint{2.958606in}{2.973282in}}%
\pgfusepath{stroke}%
\end{pgfscope}%
\begin{pgfscope}%
\pgfpathrectangle{\pgfqpoint{1.250000in}{1.750000in}}{\pgfqpoint{2.279412in}{2.004545in}}%
\pgfusepath{clip}%
\pgfsetbuttcap%
\pgfsetroundjoin%
\pgfsetlinewidth{0.370246pt}%
\definecolor{currentstroke}{rgb}{0.278791,0.062145,0.386592}%
\pgfsetstrokecolor{currentstroke}%
\pgfsetdash{}{0pt}%
\pgfpathmoveto{\pgfqpoint{2.958606in}{2.973282in}}%
\pgfpathlineto{\pgfqpoint{2.908474in}{2.972112in}}%
\pgfusepath{stroke}%
\end{pgfscope}%
\begin{pgfscope}%
\pgfpathrectangle{\pgfqpoint{1.250000in}{1.750000in}}{\pgfqpoint{2.279412in}{2.004545in}}%
\pgfusepath{clip}%
\pgfsetbuttcap%
\pgfsetroundjoin%
\pgfsetlinewidth{0.403051pt}%
\definecolor{currentstroke}{rgb}{0.281446,0.084320,0.407414}%
\pgfsetstrokecolor{currentstroke}%
\pgfsetdash{}{0pt}%
\pgfpathmoveto{\pgfqpoint{2.908474in}{2.972112in}}%
\pgfpathlineto{\pgfqpoint{2.858362in}{2.970377in}}%
\pgfusepath{stroke}%
\end{pgfscope}%
\begin{pgfscope}%
\pgfpathrectangle{\pgfqpoint{1.250000in}{1.750000in}}{\pgfqpoint{2.279412in}{2.004545in}}%
\pgfusepath{clip}%
\pgfsetbuttcap%
\pgfsetroundjoin%
\pgfsetlinewidth{0.434540pt}%
\definecolor{currentstroke}{rgb}{0.283091,0.110553,0.431554}%
\pgfsetstrokecolor{currentstroke}%
\pgfsetdash{}{0pt}%
\pgfpathmoveto{\pgfqpoint{2.858362in}{2.970377in}}%
\pgfpathlineto{\pgfqpoint{2.808263in}{2.968361in}}%
\pgfusepath{stroke}%
\end{pgfscope}%
\begin{pgfscope}%
\pgfpathrectangle{\pgfqpoint{1.250000in}{1.750000in}}{\pgfqpoint{2.279412in}{2.004545in}}%
\pgfusepath{clip}%
\pgfsetbuttcap%
\pgfsetroundjoin%
\pgfsetlinewidth{0.482235pt}%
\definecolor{currentstroke}{rgb}{0.282290,0.145912,0.461510}%
\pgfsetstrokecolor{currentstroke}%
\pgfsetdash{}{0pt}%
\pgfpathmoveto{\pgfqpoint{2.808263in}{2.968361in}}%
\pgfpathlineto{\pgfqpoint{2.758192in}{2.965884in}}%
\pgfusepath{stroke}%
\end{pgfscope}%
\begin{pgfscope}%
\pgfpathrectangle{\pgfqpoint{1.250000in}{1.750000in}}{\pgfqpoint{2.279412in}{2.004545in}}%
\pgfusepath{clip}%
\pgfsetbuttcap%
\pgfsetroundjoin%
\pgfsetlinewidth{0.554103pt}%
\definecolor{currentstroke}{rgb}{0.275191,0.194905,0.496005}%
\pgfsetstrokecolor{currentstroke}%
\pgfsetdash{}{0pt}%
\pgfpathmoveto{\pgfqpoint{2.758192in}{2.965884in}}%
\pgfpathlineto{\pgfqpoint{2.708160in}{2.962856in}}%
\pgfusepath{stroke}%
\end{pgfscope}%
\begin{pgfscope}%
\pgfpathrectangle{\pgfqpoint{1.250000in}{1.750000in}}{\pgfqpoint{2.279412in}{2.004545in}}%
\pgfusepath{clip}%
\pgfsetbuttcap%
\pgfsetroundjoin%
\pgfsetlinewidth{0.619956pt}%
\definecolor{currentstroke}{rgb}{0.262138,0.242286,0.520837}%
\pgfsetstrokecolor{currentstroke}%
\pgfsetdash{}{0pt}%
\pgfpathmoveto{\pgfqpoint{2.708160in}{2.962856in}}%
\pgfpathlineto{\pgfqpoint{2.658204in}{2.959012in}}%
\pgfusepath{stroke}%
\end{pgfscope}%
\begin{pgfscope}%
\pgfpathrectangle{\pgfqpoint{1.250000in}{1.750000in}}{\pgfqpoint{2.279412in}{2.004545in}}%
\pgfusepath{clip}%
\pgfsetbuttcap%
\pgfsetroundjoin%
\pgfsetlinewidth{0.321538pt}%
\definecolor{currentstroke}{rgb}{0.269944,0.014625,0.341379}%
\pgfsetstrokecolor{currentstroke}%
\pgfsetdash{}{0pt}%
\pgfpathmoveto{\pgfqpoint{3.107792in}{2.301205in}}%
\pgfpathlineto{\pgfqpoint{3.057727in}{2.303051in}}%
\pgfusepath{stroke}%
\end{pgfscope}%
\begin{pgfscope}%
\pgfpathrectangle{\pgfqpoint{1.250000in}{1.750000in}}{\pgfqpoint{2.279412in}{2.004545in}}%
\pgfusepath{clip}%
\pgfsetbuttcap%
\pgfsetroundjoin%
\pgfsetlinewidth{0.324639pt}%
\definecolor{currentstroke}{rgb}{0.271305,0.019942,0.347269}%
\pgfsetstrokecolor{currentstroke}%
\pgfsetdash{}{0pt}%
\pgfpathmoveto{\pgfqpoint{3.057727in}{2.303051in}}%
\pgfpathlineto{\pgfqpoint{3.007662in}{2.304856in}}%
\pgfusepath{stroke}%
\end{pgfscope}%
\begin{pgfscope}%
\pgfpathrectangle{\pgfqpoint{1.250000in}{1.750000in}}{\pgfqpoint{2.279412in}{2.004545in}}%
\pgfusepath{clip}%
\pgfsetbuttcap%
\pgfsetroundjoin%
\pgfsetlinewidth{0.326631pt}%
\definecolor{currentstroke}{rgb}{0.271305,0.019942,0.347269}%
\pgfsetstrokecolor{currentstroke}%
\pgfsetdash{}{0pt}%
\pgfpathmoveto{\pgfqpoint{3.007662in}{2.304856in}}%
\pgfpathlineto{\pgfqpoint{2.957554in}{2.306464in}}%
\pgfusepath{stroke}%
\end{pgfscope}%
\begin{pgfscope}%
\pgfpathrectangle{\pgfqpoint{1.250000in}{1.750000in}}{\pgfqpoint{2.279412in}{2.004545in}}%
\pgfusepath{clip}%
\pgfsetbuttcap%
\pgfsetroundjoin%
\pgfsetlinewidth{0.326567pt}%
\definecolor{currentstroke}{rgb}{0.271305,0.019942,0.347269}%
\pgfsetstrokecolor{currentstroke}%
\pgfsetdash{}{0pt}%
\pgfpathmoveto{\pgfqpoint{2.957554in}{2.306464in}}%
\pgfpathlineto{\pgfqpoint{2.907480in}{2.308767in}}%
\pgfusepath{stroke}%
\end{pgfscope}%
\begin{pgfscope}%
\pgfpathrectangle{\pgfqpoint{1.250000in}{1.750000in}}{\pgfqpoint{2.279412in}{2.004545in}}%
\pgfusepath{clip}%
\pgfsetbuttcap%
\pgfsetroundjoin%
\pgfsetlinewidth{0.345446pt}%
\definecolor{currentstroke}{rgb}{0.274952,0.037752,0.364543}%
\pgfsetstrokecolor{currentstroke}%
\pgfsetdash{}{0pt}%
\pgfpathmoveto{\pgfqpoint{2.907480in}{2.308767in}}%
\pgfpathlineto{\pgfqpoint{2.857488in}{2.312160in}}%
\pgfusepath{stroke}%
\end{pgfscope}%
\begin{pgfscope}%
\pgfpathrectangle{\pgfqpoint{1.250000in}{1.750000in}}{\pgfqpoint{2.279412in}{2.004545in}}%
\pgfusepath{clip}%
\pgfsetbuttcap%
\pgfsetroundjoin%
\pgfsetlinewidth{0.318152pt}%
\definecolor{currentstroke}{rgb}{0.269944,0.014625,0.341379}%
\pgfsetstrokecolor{currentstroke}%
\pgfsetdash{}{0pt}%
\pgfpathmoveto{\pgfqpoint{3.107792in}{2.391418in}}%
\pgfpathlineto{\pgfqpoint{3.057738in}{2.393953in}}%
\pgfusepath{stroke}%
\end{pgfscope}%
\begin{pgfscope}%
\pgfpathrectangle{\pgfqpoint{1.250000in}{1.750000in}}{\pgfqpoint{2.279412in}{2.004545in}}%
\pgfusepath{clip}%
\pgfsetbuttcap%
\pgfsetroundjoin%
\pgfsetlinewidth{0.324016pt}%
\definecolor{currentstroke}{rgb}{0.271305,0.019942,0.347269}%
\pgfsetstrokecolor{currentstroke}%
\pgfsetdash{}{0pt}%
\pgfpathmoveto{\pgfqpoint{3.057738in}{2.393953in}}%
\pgfpathlineto{\pgfqpoint{3.007646in}{2.395855in}}%
\pgfusepath{stroke}%
\end{pgfscope}%
\begin{pgfscope}%
\pgfpathrectangle{\pgfqpoint{1.250000in}{1.750000in}}{\pgfqpoint{2.279412in}{2.004545in}}%
\pgfusepath{clip}%
\pgfsetbuttcap%
\pgfsetroundjoin%
\pgfsetlinewidth{0.341073pt}%
\definecolor{currentstroke}{rgb}{0.273809,0.031497,0.358853}%
\pgfsetstrokecolor{currentstroke}%
\pgfsetdash{}{0pt}%
\pgfpathmoveto{\pgfqpoint{3.007646in}{2.395855in}}%
\pgfpathlineto{\pgfqpoint{2.957524in}{2.396522in}}%
\pgfusepath{stroke}%
\end{pgfscope}%
\begin{pgfscope}%
\pgfpathrectangle{\pgfqpoint{1.250000in}{1.750000in}}{\pgfqpoint{2.279412in}{2.004545in}}%
\pgfusepath{clip}%
\pgfsetbuttcap%
\pgfsetroundjoin%
\pgfsetlinewidth{0.350454pt}%
\definecolor{currentstroke}{rgb}{0.276022,0.044167,0.370164}%
\pgfsetstrokecolor{currentstroke}%
\pgfsetdash{}{0pt}%
\pgfpathmoveto{\pgfqpoint{2.957524in}{2.396522in}}%
\pgfpathlineto{\pgfqpoint{2.907400in}{2.397277in}}%
\pgfusepath{stroke}%
\end{pgfscope}%
\begin{pgfscope}%
\pgfpathrectangle{\pgfqpoint{1.250000in}{1.750000in}}{\pgfqpoint{2.279412in}{2.004545in}}%
\pgfusepath{clip}%
\pgfsetbuttcap%
\pgfsetroundjoin%
\pgfsetlinewidth{0.368328pt}%
\definecolor{currentstroke}{rgb}{0.277941,0.056324,0.381191}%
\pgfsetstrokecolor{currentstroke}%
\pgfsetdash{}{0pt}%
\pgfpathmoveto{\pgfqpoint{2.907400in}{2.397277in}}%
\pgfpathlineto{\pgfqpoint{2.857300in}{2.399232in}}%
\pgfusepath{stroke}%
\end{pgfscope}%
\begin{pgfscope}%
\pgfpathrectangle{\pgfqpoint{1.250000in}{1.750000in}}{\pgfqpoint{2.279412in}{2.004545in}}%
\pgfusepath{clip}%
\pgfsetbuttcap%
\pgfsetroundjoin%
\pgfsetlinewidth{0.379754pt}%
\definecolor{currentstroke}{rgb}{0.279566,0.067836,0.391917}%
\pgfsetstrokecolor{currentstroke}%
\pgfsetdash{}{0pt}%
\pgfpathmoveto{\pgfqpoint{2.857300in}{2.399232in}}%
\pgfpathlineto{\pgfqpoint{2.807238in}{2.401843in}}%
\pgfusepath{stroke}%
\end{pgfscope}%
\begin{pgfscope}%
\pgfpathrectangle{\pgfqpoint{1.250000in}{1.750000in}}{\pgfqpoint{2.279412in}{2.004545in}}%
\pgfusepath{clip}%
\pgfsetbuttcap%
\pgfsetroundjoin%
\pgfsetlinewidth{0.417507pt}%
\definecolor{currentstroke}{rgb}{0.282327,0.094955,0.417331}%
\pgfsetstrokecolor{currentstroke}%
\pgfsetdash{}{0pt}%
\pgfpathmoveto{\pgfqpoint{2.807238in}{2.401843in}}%
\pgfpathlineto{\pgfqpoint{2.757267in}{2.405427in}}%
\pgfusepath{stroke}%
\end{pgfscope}%
\begin{pgfscope}%
\pgfpathrectangle{\pgfqpoint{1.250000in}{1.750000in}}{\pgfqpoint{2.279412in}{2.004545in}}%
\pgfusepath{clip}%
\pgfsetbuttcap%
\pgfsetroundjoin%
\pgfsetlinewidth{0.332518pt}%
\definecolor{currentstroke}{rgb}{0.272594,0.025563,0.353093}%
\pgfsetstrokecolor{currentstroke}%
\pgfsetdash{}{0pt}%
\pgfpathmoveto{\pgfqpoint{3.107792in}{2.752273in}}%
\pgfpathlineto{\pgfqpoint{3.057656in}{2.751713in}}%
\pgfusepath{stroke}%
\end{pgfscope}%
\begin{pgfscope}%
\pgfpathrectangle{\pgfqpoint{1.250000in}{1.750000in}}{\pgfqpoint{2.279412in}{2.004545in}}%
\pgfusepath{clip}%
\pgfsetbuttcap%
\pgfsetroundjoin%
\pgfsetlinewidth{0.344625pt}%
\definecolor{currentstroke}{rgb}{0.274952,0.037752,0.364543}%
\pgfsetstrokecolor{currentstroke}%
\pgfsetdash{}{0pt}%
\pgfpathmoveto{\pgfqpoint{3.057656in}{2.751713in}}%
\pgfpathlineto{\pgfqpoint{3.007512in}{2.751494in}}%
\pgfusepath{stroke}%
\end{pgfscope}%
\begin{pgfscope}%
\pgfpathrectangle{\pgfqpoint{1.250000in}{1.750000in}}{\pgfqpoint{2.279412in}{2.004545in}}%
\pgfusepath{clip}%
\pgfsetbuttcap%
\pgfsetroundjoin%
\pgfsetlinewidth{0.366257pt}%
\definecolor{currentstroke}{rgb}{0.277941,0.056324,0.381191}%
\pgfsetstrokecolor{currentstroke}%
\pgfsetdash{}{0pt}%
\pgfpathmoveto{\pgfqpoint{3.007512in}{2.751494in}}%
\pgfpathlineto{\pgfqpoint{2.957364in}{2.751594in}}%
\pgfusepath{stroke}%
\end{pgfscope}%
\begin{pgfscope}%
\pgfpathrectangle{\pgfqpoint{1.250000in}{1.750000in}}{\pgfqpoint{2.279412in}{2.004545in}}%
\pgfusepath{clip}%
\pgfsetbuttcap%
\pgfsetroundjoin%
\pgfsetlinewidth{0.393770pt}%
\definecolor{currentstroke}{rgb}{0.280894,0.078907,0.402329}%
\pgfsetstrokecolor{currentstroke}%
\pgfsetdash{}{0pt}%
\pgfpathmoveto{\pgfqpoint{2.957364in}{2.751594in}}%
\pgfpathlineto{\pgfqpoint{2.907214in}{2.751722in}}%
\pgfusepath{stroke}%
\end{pgfscope}%
\begin{pgfscope}%
\pgfpathrectangle{\pgfqpoint{1.250000in}{1.750000in}}{\pgfqpoint{2.279412in}{2.004545in}}%
\pgfusepath{clip}%
\pgfsetbuttcap%
\pgfsetroundjoin%
\pgfsetlinewidth{0.454441pt}%
\definecolor{currentstroke}{rgb}{0.283187,0.125848,0.444960}%
\pgfsetstrokecolor{currentstroke}%
\pgfsetdash{}{0pt}%
\pgfpathmoveto{\pgfqpoint{2.907214in}{2.751722in}}%
\pgfpathlineto{\pgfqpoint{2.857062in}{2.751583in}}%
\pgfusepath{stroke}%
\end{pgfscope}%
\begin{pgfscope}%
\pgfpathrectangle{\pgfqpoint{1.250000in}{1.750000in}}{\pgfqpoint{2.279412in}{2.004545in}}%
\pgfusepath{clip}%
\pgfsetbuttcap%
\pgfsetroundjoin%
\pgfsetlinewidth{0.506277pt}%
\definecolor{currentstroke}{rgb}{0.280868,0.160771,0.472899}%
\pgfsetstrokecolor{currentstroke}%
\pgfsetdash{}{0pt}%
\pgfpathmoveto{\pgfqpoint{2.857062in}{2.751583in}}%
\pgfpathlineto{\pgfqpoint{2.806910in}{2.751517in}}%
\pgfusepath{stroke}%
\end{pgfscope}%
\begin{pgfscope}%
\pgfpathrectangle{\pgfqpoint{1.250000in}{1.750000in}}{\pgfqpoint{2.279412in}{2.004545in}}%
\pgfusepath{clip}%
\pgfsetbuttcap%
\pgfsetroundjoin%
\pgfsetlinewidth{0.638080pt}%
\definecolor{currentstroke}{rgb}{0.257322,0.256130,0.526563}%
\pgfsetstrokecolor{currentstroke}%
\pgfsetdash{}{0pt}%
\pgfpathmoveto{\pgfqpoint{2.806910in}{2.751517in}}%
\pgfpathlineto{\pgfqpoint{2.756759in}{2.751528in}}%
\pgfusepath{stroke}%
\end{pgfscope}%
\begin{pgfscope}%
\pgfpathrectangle{\pgfqpoint{1.250000in}{1.750000in}}{\pgfqpoint{2.279412in}{2.004545in}}%
\pgfusepath{clip}%
\pgfsetbuttcap%
\pgfsetroundjoin%
\pgfsetlinewidth{0.847703pt}%
\definecolor{currentstroke}{rgb}{0.201239,0.383670,0.554294}%
\pgfsetstrokecolor{currentstroke}%
\pgfsetdash{}{0pt}%
\pgfpathmoveto{\pgfqpoint{2.756759in}{2.751528in}}%
\pgfpathlineto{\pgfqpoint{2.706607in}{2.751420in}}%
\pgfusepath{stroke}%
\end{pgfscope}%
\begin{pgfscope}%
\pgfpathrectangle{\pgfqpoint{1.250000in}{1.750000in}}{\pgfqpoint{2.279412in}{2.004545in}}%
\pgfusepath{clip}%
\pgfsetbuttcap%
\pgfsetroundjoin%
\pgfsetlinewidth{1.123203pt}%
\definecolor{currentstroke}{rgb}{0.139147,0.533812,0.555298}%
\pgfsetstrokecolor{currentstroke}%
\pgfsetdash{}{0pt}%
\pgfpathmoveto{\pgfqpoint{2.706607in}{2.751420in}}%
\pgfpathlineto{\pgfqpoint{2.656456in}{2.751222in}}%
\pgfusepath{stroke}%
\end{pgfscope}%
\begin{pgfscope}%
\pgfpathrectangle{\pgfqpoint{1.250000in}{1.750000in}}{\pgfqpoint{2.279412in}{2.004545in}}%
\pgfusepath{clip}%
\pgfsetbuttcap%
\pgfsetroundjoin%
\pgfsetlinewidth{1.421565pt}%
\definecolor{currentstroke}{rgb}{0.170948,0.694384,0.493803}%
\pgfsetstrokecolor{currentstroke}%
\pgfsetdash{}{0pt}%
\pgfpathmoveto{\pgfqpoint{2.656456in}{2.751222in}}%
\pgfpathlineto{\pgfqpoint{2.606304in}{2.751076in}}%
\pgfusepath{stroke}%
\end{pgfscope}%
\begin{pgfscope}%
\pgfpathrectangle{\pgfqpoint{1.250000in}{1.750000in}}{\pgfqpoint{2.279412in}{2.004545in}}%
\pgfusepath{clip}%
\pgfsetbuttcap%
\pgfsetroundjoin%
\pgfsetlinewidth{1.656000pt}%
\definecolor{currentstroke}{rgb}{0.412913,0.803041,0.357269}%
\pgfsetstrokecolor{currentstroke}%
\pgfsetdash{}{0pt}%
\pgfpathmoveto{\pgfqpoint{2.606304in}{2.751076in}}%
\pgfpathlineto{\pgfqpoint{2.556153in}{2.750867in}}%
\pgfusepath{stroke}%
\end{pgfscope}%
\begin{pgfscope}%
\pgfpathrectangle{\pgfqpoint{1.250000in}{1.750000in}}{\pgfqpoint{2.279412in}{2.004545in}}%
\pgfusepath{clip}%
\pgfsetbuttcap%
\pgfsetroundjoin%
\pgfsetlinewidth{1.934087pt}%
\definecolor{currentstroke}{rgb}{0.814576,0.883393,0.110347}%
\pgfsetstrokecolor{currentstroke}%
\pgfsetdash{}{0pt}%
\pgfpathmoveto{\pgfqpoint{2.556153in}{2.750867in}}%
\pgfpathlineto{\pgfqpoint{2.506003in}{2.750626in}}%
\pgfusepath{stroke}%
\end{pgfscope}%
\begin{pgfscope}%
\pgfpathrectangle{\pgfqpoint{1.250000in}{1.750000in}}{\pgfqpoint{2.279412in}{2.004545in}}%
\pgfusepath{clip}%
\pgfsetbuttcap%
\pgfsetroundjoin%
\pgfsetlinewidth{2.211943pt}%
\definecolor{currentstroke}{rgb}{0.993248,0.906157,0.143936}%
\pgfsetstrokecolor{currentstroke}%
\pgfsetdash{}{0pt}%
\pgfpathmoveto{\pgfqpoint{2.506003in}{2.750626in}}%
\pgfpathlineto{\pgfqpoint{2.455853in}{2.750321in}}%
\pgfusepath{stroke}%
\end{pgfscope}%
\begin{pgfscope}%
\pgfpathrectangle{\pgfqpoint{1.250000in}{1.750000in}}{\pgfqpoint{2.279412in}{2.004545in}}%
\pgfusepath{clip}%
\pgfsetbuttcap%
\pgfsetroundjoin%
\pgfsetlinewidth{2.260436pt}%
\definecolor{currentstroke}{rgb}{0.993248,0.906157,0.143936}%
\pgfsetstrokecolor{currentstroke}%
\pgfsetdash{}{0pt}%
\pgfpathmoveto{\pgfqpoint{2.455853in}{2.750321in}}%
\pgfpathlineto{\pgfqpoint{2.405705in}{2.749947in}}%
\pgfusepath{stroke}%
\end{pgfscope}%
\begin{pgfscope}%
\pgfpathrectangle{\pgfqpoint{1.250000in}{1.750000in}}{\pgfqpoint{2.279412in}{2.004545in}}%
\pgfusepath{clip}%
\pgfsetbuttcap%
\pgfsetroundjoin%
\pgfsetlinewidth{2.356766pt}%
\definecolor{currentstroke}{rgb}{0.993248,0.906157,0.143936}%
\pgfsetstrokecolor{currentstroke}%
\pgfsetdash{}{0pt}%
\pgfpathmoveto{\pgfqpoint{2.405705in}{2.749947in}}%
\pgfpathlineto{\pgfqpoint{2.355559in}{2.749532in}}%
\pgfusepath{stroke}%
\end{pgfscope}%
\begin{pgfscope}%
\pgfpathrectangle{\pgfqpoint{1.250000in}{1.750000in}}{\pgfqpoint{2.279412in}{2.004545in}}%
\pgfusepath{clip}%
\pgfsetbuttcap%
\pgfsetroundjoin%
\pgfsetlinewidth{2.338725pt}%
\definecolor{currentstroke}{rgb}{0.993248,0.906157,0.143936}%
\pgfsetstrokecolor{currentstroke}%
\pgfsetdash{}{0pt}%
\pgfpathmoveto{\pgfqpoint{2.355559in}{2.749532in}}%
\pgfpathlineto{\pgfqpoint{2.305415in}{2.749084in}}%
\pgfusepath{stroke}%
\end{pgfscope}%
\begin{pgfscope}%
\pgfpathrectangle{\pgfqpoint{1.250000in}{1.750000in}}{\pgfqpoint{2.279412in}{2.004545in}}%
\pgfusepath{clip}%
\pgfsetbuttcap%
\pgfsetroundjoin%
\pgfsetlinewidth{2.358918pt}%
\definecolor{currentstroke}{rgb}{0.993248,0.906157,0.143936}%
\pgfsetstrokecolor{currentstroke}%
\pgfsetdash{}{0pt}%
\pgfpathmoveto{\pgfqpoint{2.305415in}{2.749084in}}%
\pgfpathlineto{\pgfqpoint{2.255274in}{2.748631in}}%
\pgfusepath{stroke}%
\end{pgfscope}%
\begin{pgfscope}%
\pgfpathrectangle{\pgfqpoint{1.250000in}{1.750000in}}{\pgfqpoint{2.279412in}{2.004545in}}%
\pgfusepath{clip}%
\pgfsetbuttcap%
\pgfsetroundjoin%
\pgfsetlinewidth{2.199408pt}%
\definecolor{currentstroke}{rgb}{0.993248,0.906157,0.143936}%
\pgfsetstrokecolor{currentstroke}%
\pgfsetdash{}{0pt}%
\pgfpathmoveto{\pgfqpoint{2.255274in}{2.748631in}}%
\pgfpathlineto{\pgfqpoint{2.205140in}{2.748070in}}%
\pgfusepath{stroke}%
\end{pgfscope}%
\begin{pgfscope}%
\pgfpathrectangle{\pgfqpoint{1.250000in}{1.750000in}}{\pgfqpoint{2.279412in}{2.004545in}}%
\pgfusepath{clip}%
\pgfsetbuttcap%
\pgfsetroundjoin%
\pgfsetlinewidth{2.036367pt}%
\definecolor{currentstroke}{rgb}{0.964894,0.902323,0.123941}%
\pgfsetstrokecolor{currentstroke}%
\pgfsetdash{}{0pt}%
\pgfpathmoveto{\pgfqpoint{2.205140in}{2.748070in}}%
\pgfpathlineto{\pgfqpoint{2.155015in}{2.747475in}}%
\pgfusepath{stroke}%
\end{pgfscope}%
\begin{pgfscope}%
\pgfpathrectangle{\pgfqpoint{1.250000in}{1.750000in}}{\pgfqpoint{2.279412in}{2.004545in}}%
\pgfusepath{clip}%
\pgfsetbuttcap%
\pgfsetroundjoin%
\pgfsetlinewidth{0.326157pt}%
\definecolor{currentstroke}{rgb}{0.271305,0.019942,0.347269}%
\pgfsetstrokecolor{currentstroke}%
\pgfsetdash{}{0pt}%
\pgfpathmoveto{\pgfqpoint{3.107792in}{3.022913in}}%
\pgfpathlineto{\pgfqpoint{3.057656in}{3.022321in}}%
\pgfusepath{stroke}%
\end{pgfscope}%
\begin{pgfscope}%
\pgfpathrectangle{\pgfqpoint{1.250000in}{1.750000in}}{\pgfqpoint{2.279412in}{2.004545in}}%
\pgfusepath{clip}%
\pgfsetbuttcap%
\pgfsetroundjoin%
\pgfsetlinewidth{0.329213pt}%
\definecolor{currentstroke}{rgb}{0.272594,0.025563,0.353093}%
\pgfsetstrokecolor{currentstroke}%
\pgfsetdash{}{0pt}%
\pgfpathmoveto{\pgfqpoint{3.057656in}{3.022321in}}%
\pgfpathlineto{\pgfqpoint{3.007539in}{3.020927in}}%
\pgfusepath{stroke}%
\end{pgfscope}%
\begin{pgfscope}%
\pgfpathrectangle{\pgfqpoint{1.250000in}{1.750000in}}{\pgfqpoint{2.279412in}{2.004545in}}%
\pgfusepath{clip}%
\pgfsetbuttcap%
\pgfsetroundjoin%
\pgfsetlinewidth{0.345420pt}%
\definecolor{currentstroke}{rgb}{0.274952,0.037752,0.364543}%
\pgfsetstrokecolor{currentstroke}%
\pgfsetdash{}{0pt}%
\pgfpathmoveto{\pgfqpoint{3.007539in}{3.020927in}}%
\pgfpathlineto{\pgfqpoint{2.957419in}{3.019532in}}%
\pgfusepath{stroke}%
\end{pgfscope}%
\begin{pgfscope}%
\pgfpathrectangle{\pgfqpoint{1.250000in}{1.750000in}}{\pgfqpoint{2.279412in}{2.004545in}}%
\pgfusepath{clip}%
\pgfsetbuttcap%
\pgfsetroundjoin%
\pgfsetlinewidth{0.373715pt}%
\definecolor{currentstroke}{rgb}{0.278791,0.062145,0.386592}%
\pgfsetstrokecolor{currentstroke}%
\pgfsetdash{}{0pt}%
\pgfpathmoveto{\pgfqpoint{2.957419in}{3.019532in}}%
\pgfpathlineto{\pgfqpoint{2.907317in}{3.017698in}}%
\pgfusepath{stroke}%
\end{pgfscope}%
\begin{pgfscope}%
\pgfpathrectangle{\pgfqpoint{1.250000in}{1.750000in}}{\pgfqpoint{2.279412in}{2.004545in}}%
\pgfusepath{clip}%
\pgfsetbuttcap%
\pgfsetroundjoin%
\pgfsetlinewidth{0.403580pt}%
\definecolor{currentstroke}{rgb}{0.281446,0.084320,0.407414}%
\pgfsetstrokecolor{currentstroke}%
\pgfsetdash{}{0pt}%
\pgfpathmoveto{\pgfqpoint{2.907317in}{3.017698in}}%
\pgfpathlineto{\pgfqpoint{2.857219in}{3.015864in}}%
\pgfusepath{stroke}%
\end{pgfscope}%
\begin{pgfscope}%
\pgfpathrectangle{\pgfqpoint{1.250000in}{1.750000in}}{\pgfqpoint{2.279412in}{2.004545in}}%
\pgfusepath{clip}%
\pgfsetbuttcap%
\pgfsetroundjoin%
\pgfsetlinewidth{0.419582pt}%
\definecolor{currentstroke}{rgb}{0.282656,0.100196,0.422160}%
\pgfsetstrokecolor{currentstroke}%
\pgfsetdash{}{0pt}%
\pgfpathmoveto{\pgfqpoint{2.857219in}{3.015864in}}%
\pgfpathlineto{\pgfqpoint{2.807128in}{3.013911in}}%
\pgfusepath{stroke}%
\end{pgfscope}%
\begin{pgfscope}%
\pgfpathrectangle{\pgfqpoint{1.250000in}{1.750000in}}{\pgfqpoint{2.279412in}{2.004545in}}%
\pgfusepath{clip}%
\pgfsetbuttcap%
\pgfsetroundjoin%
\pgfsetlinewidth{0.458439pt}%
\definecolor{currentstroke}{rgb}{0.283187,0.125848,0.444960}%
\pgfsetstrokecolor{currentstroke}%
\pgfsetdash{}{0pt}%
\pgfpathmoveto{\pgfqpoint{2.807128in}{3.013911in}}%
\pgfpathlineto{\pgfqpoint{2.757083in}{3.011046in}}%
\pgfusepath{stroke}%
\end{pgfscope}%
\begin{pgfscope}%
\pgfpathrectangle{\pgfqpoint{1.250000in}{1.750000in}}{\pgfqpoint{2.279412in}{2.004545in}}%
\pgfusepath{clip}%
\pgfsetbuttcap%
\pgfsetroundjoin%
\pgfsetlinewidth{0.483881pt}%
\definecolor{currentstroke}{rgb}{0.282290,0.145912,0.461510}%
\pgfsetstrokecolor{currentstroke}%
\pgfsetdash{}{0pt}%
\pgfpathmoveto{\pgfqpoint{2.757083in}{3.011046in}}%
\pgfpathlineto{\pgfqpoint{2.707057in}{3.007929in}}%
\pgfusepath{stroke}%
\end{pgfscope}%
\begin{pgfscope}%
\pgfpathrectangle{\pgfqpoint{1.250000in}{1.750000in}}{\pgfqpoint{2.279412in}{2.004545in}}%
\pgfusepath{clip}%
\pgfsetbuttcap%
\pgfsetroundjoin%
\pgfsetlinewidth{0.448980pt}%
\definecolor{currentstroke}{rgb}{0.283229,0.120777,0.440584}%
\pgfsetstrokecolor{currentstroke}%
\pgfsetdash{}{0pt}%
\pgfpathmoveto{\pgfqpoint{2.707057in}{3.007929in}}%
\pgfpathlineto{\pgfqpoint{2.657175in}{3.003546in}}%
\pgfusepath{stroke}%
\end{pgfscope}%
\begin{pgfscope}%
\pgfpathrectangle{\pgfqpoint{1.250000in}{1.750000in}}{\pgfqpoint{2.279412in}{2.004545in}}%
\pgfusepath{clip}%
\pgfsetbuttcap%
\pgfsetroundjoin%
\pgfsetlinewidth{0.611878pt}%
\definecolor{currentstroke}{rgb}{0.263663,0.237631,0.518762}%
\pgfsetstrokecolor{currentstroke}%
\pgfsetdash{}{0pt}%
\pgfpathmoveto{\pgfqpoint{2.657175in}{3.003546in}}%
\pgfpathlineto{\pgfqpoint{2.607516in}{2.997433in}}%
\pgfusepath{stroke}%
\end{pgfscope}%
\begin{pgfscope}%
\pgfpathrectangle{\pgfqpoint{1.250000in}{1.750000in}}{\pgfqpoint{2.279412in}{2.004545in}}%
\pgfusepath{clip}%
\pgfsetbuttcap%
\pgfsetroundjoin%
\pgfsetlinewidth{0.603921pt}%
\definecolor{currentstroke}{rgb}{0.265145,0.232956,0.516599}%
\pgfsetstrokecolor{currentstroke}%
\pgfsetdash{}{0pt}%
\pgfpathmoveto{\pgfqpoint{2.607516in}{2.997433in}}%
\pgfpathlineto{\pgfqpoint{2.558107in}{2.989957in}}%
\pgfusepath{stroke}%
\end{pgfscope}%
\begin{pgfscope}%
\pgfpathrectangle{\pgfqpoint{1.250000in}{1.750000in}}{\pgfqpoint{2.279412in}{2.004545in}}%
\pgfusepath{clip}%
\pgfsetbuttcap%
\pgfsetroundjoin%
\pgfsetlinewidth{0.732528pt}%
\definecolor{currentstroke}{rgb}{0.233603,0.313828,0.543914}%
\pgfsetstrokecolor{currentstroke}%
\pgfsetdash{}{0pt}%
\pgfpathmoveto{\pgfqpoint{2.558107in}{2.989957in}}%
\pgfpathlineto{\pgfqpoint{2.509052in}{2.980836in}}%
\pgfusepath{stroke}%
\end{pgfscope}%
\begin{pgfscope}%
\pgfpathrectangle{\pgfqpoint{1.250000in}{1.750000in}}{\pgfqpoint{2.279412in}{2.004545in}}%
\pgfusepath{clip}%
\pgfsetbuttcap%
\pgfsetroundjoin%
\pgfsetlinewidth{0.863142pt}%
\definecolor{currentstroke}{rgb}{0.197636,0.391528,0.554969}%
\pgfsetstrokecolor{currentstroke}%
\pgfsetdash{}{0pt}%
\pgfpathmoveto{\pgfqpoint{2.509052in}{2.980836in}}%
\pgfpathlineto{\pgfqpoint{2.460439in}{2.970049in}}%
\pgfusepath{stroke}%
\end{pgfscope}%
\begin{pgfscope}%
\pgfpathrectangle{\pgfqpoint{1.250000in}{1.750000in}}{\pgfqpoint{2.279412in}{2.004545in}}%
\pgfusepath{clip}%
\pgfsetbuttcap%
\pgfsetroundjoin%
\pgfsetlinewidth{1.065430pt}%
\definecolor{currentstroke}{rgb}{0.150476,0.504369,0.557430}%
\pgfsetstrokecolor{currentstroke}%
\pgfsetdash{}{0pt}%
\pgfpathmoveto{\pgfqpoint{2.460439in}{2.970049in}}%
\pgfpathlineto{\pgfqpoint{2.412295in}{2.957723in}}%
\pgfusepath{stroke}%
\end{pgfscope}%
\begin{pgfscope}%
\pgfpathrectangle{\pgfqpoint{1.250000in}{1.750000in}}{\pgfqpoint{2.279412in}{2.004545in}}%
\pgfusepath{clip}%
\pgfsetbuttcap%
\pgfsetroundjoin%
\pgfsetlinewidth{1.034784pt}%
\definecolor{currentstroke}{rgb}{0.157729,0.485932,0.558013}%
\pgfsetstrokecolor{currentstroke}%
\pgfsetdash{}{0pt}%
\pgfpathmoveto{\pgfqpoint{2.412295in}{2.957723in}}%
\pgfpathlineto{\pgfqpoint{2.364840in}{2.943531in}}%
\pgfusepath{stroke}%
\end{pgfscope}%
\begin{pgfscope}%
\pgfpathrectangle{\pgfqpoint{1.250000in}{1.750000in}}{\pgfqpoint{2.279412in}{2.004545in}}%
\pgfusepath{clip}%
\pgfsetbuttcap%
\pgfsetroundjoin%
\pgfsetlinewidth{1.005665pt}%
\definecolor{currentstroke}{rgb}{0.163625,0.471133,0.558148}%
\pgfsetstrokecolor{currentstroke}%
\pgfsetdash{}{0pt}%
\pgfpathmoveto{\pgfqpoint{2.364840in}{2.943531in}}%
\pgfpathlineto{\pgfqpoint{2.318085in}{2.927615in}}%
\pgfusepath{stroke}%
\end{pgfscope}%
\begin{pgfscope}%
\pgfpathrectangle{\pgfqpoint{1.250000in}{1.750000in}}{\pgfqpoint{2.279412in}{2.004545in}}%
\pgfusepath{clip}%
\pgfsetbuttcap%
\pgfsetroundjoin%
\pgfsetlinewidth{1.259150pt}%
\definecolor{currentstroke}{rgb}{0.119512,0.607464,0.540218}%
\pgfsetstrokecolor{currentstroke}%
\pgfsetdash{}{0pt}%
\pgfpathmoveto{\pgfqpoint{2.318085in}{2.927615in}}%
\pgfpathlineto{\pgfqpoint{2.271853in}{2.910551in}}%
\pgfusepath{stroke}%
\end{pgfscope}%
\begin{pgfscope}%
\pgfpathrectangle{\pgfqpoint{1.250000in}{1.750000in}}{\pgfqpoint{2.279412in}{2.004545in}}%
\pgfusepath{clip}%
\pgfsetbuttcap%
\pgfsetroundjoin%
\pgfsetlinewidth{1.728984pt}%
\definecolor{currentstroke}{rgb}{0.515992,0.831158,0.294279}%
\pgfsetstrokecolor{currentstroke}%
\pgfsetdash{}{0pt}%
\pgfpathmoveto{\pgfqpoint{2.271853in}{2.910551in}}%
\pgfpathlineto{\pgfqpoint{2.226497in}{2.891818in}}%
\pgfusepath{stroke}%
\end{pgfscope}%
\begin{pgfscope}%
\pgfpathrectangle{\pgfqpoint{1.250000in}{1.750000in}}{\pgfqpoint{2.279412in}{2.004545in}}%
\pgfusepath{clip}%
\pgfsetbuttcap%
\pgfsetroundjoin%
\pgfsetlinewidth{1.394549pt}%
\definecolor{currentstroke}{rgb}{0.153894,0.680203,0.504172}%
\pgfsetstrokecolor{currentstroke}%
\pgfsetdash{}{0pt}%
\pgfpathmoveto{\pgfqpoint{2.226497in}{2.891818in}}%
\pgfpathlineto{\pgfqpoint{2.182002in}{2.871519in}}%
\pgfusepath{stroke}%
\end{pgfscope}%
\begin{pgfscope}%
\pgfpathrectangle{\pgfqpoint{1.250000in}{1.750000in}}{\pgfqpoint{2.279412in}{2.004545in}}%
\pgfusepath{clip}%
\pgfsetbuttcap%
\pgfsetroundjoin%
\pgfsetlinewidth{1.578883pt}%
\definecolor{currentstroke}{rgb}{0.319809,0.770914,0.411152}%
\pgfsetstrokecolor{currentstroke}%
\pgfsetdash{}{0pt}%
\pgfpathmoveto{\pgfqpoint{2.182002in}{2.871519in}}%
\pgfpathlineto{\pgfqpoint{2.137210in}{2.851728in}}%
\pgfusepath{stroke}%
\end{pgfscope}%
\begin{pgfscope}%
\pgfpathrectangle{\pgfqpoint{1.250000in}{1.750000in}}{\pgfqpoint{2.279412in}{2.004545in}}%
\pgfusepath{clip}%
\pgfsetbuttcap%
\pgfsetroundjoin%
\pgfsetlinewidth{2.161473pt}%
\definecolor{currentstroke}{rgb}{0.993248,0.906157,0.143936}%
\pgfsetstrokecolor{currentstroke}%
\pgfsetdash{}{0pt}%
\pgfpathmoveto{\pgfqpoint{2.137210in}{2.851728in}}%
\pgfpathlineto{\pgfqpoint{2.092129in}{2.832442in}}%
\pgfusepath{stroke}%
\end{pgfscope}%
\begin{pgfscope}%
\pgfpathrectangle{\pgfqpoint{1.250000in}{1.750000in}}{\pgfqpoint{2.279412in}{2.004545in}}%
\pgfusepath{clip}%
\pgfsetbuttcap%
\pgfsetroundjoin%
\pgfsetlinewidth{1.767649pt}%
\definecolor{currentstroke}{rgb}{0.565498,0.842430,0.262877}%
\pgfsetstrokecolor{currentstroke}%
\pgfsetdash{}{0pt}%
\pgfpathmoveto{\pgfqpoint{2.092129in}{2.832442in}}%
\pgfpathlineto{\pgfqpoint{2.046758in}{2.813728in}}%
\pgfusepath{stroke}%
\end{pgfscope}%
\begin{pgfscope}%
\pgfpathrectangle{\pgfqpoint{1.250000in}{1.750000in}}{\pgfqpoint{2.279412in}{2.004545in}}%
\pgfusepath{clip}%
\pgfsetbuttcap%
\pgfsetroundjoin%
\pgfsetlinewidth{1.792309pt}%
\definecolor{currentstroke}{rgb}{0.606045,0.850733,0.236712}%
\pgfsetstrokecolor{currentstroke}%
\pgfsetdash{}{0pt}%
\pgfpathmoveto{\pgfqpoint{2.046758in}{2.813728in}}%
\pgfpathlineto{\pgfqpoint{2.001348in}{2.795165in}}%
\pgfusepath{stroke}%
\end{pgfscope}%
\begin{pgfscope}%
\pgfpathrectangle{\pgfqpoint{1.250000in}{1.750000in}}{\pgfqpoint{2.279412in}{2.004545in}}%
\pgfusepath{clip}%
\pgfsetbuttcap%
\pgfsetroundjoin%
\pgfsetlinewidth{1.554934pt}%
\definecolor{currentstroke}{rgb}{0.288921,0.758394,0.428426}%
\pgfsetstrokecolor{currentstroke}%
\pgfsetdash{}{0pt}%
\pgfpathmoveto{\pgfqpoint{2.001348in}{2.795165in}}%
\pgfpathlineto{\pgfqpoint{1.955622in}{2.777631in}}%
\pgfusepath{stroke}%
\end{pgfscope}%
\begin{pgfscope}%
\pgfpathrectangle{\pgfqpoint{1.250000in}{1.750000in}}{\pgfqpoint{2.279412in}{2.004545in}}%
\pgfusepath{clip}%
\pgfsetbuttcap%
\pgfsetroundjoin%
\pgfsetlinewidth{0.315374pt}%
\definecolor{currentstroke}{rgb}{0.269944,0.014625,0.341379}%
\pgfsetstrokecolor{currentstroke}%
\pgfsetdash{}{0pt}%
\pgfpathmoveto{\pgfqpoint{3.107792in}{3.068020in}}%
\pgfpathlineto{\pgfqpoint{3.057947in}{3.065440in}}%
\pgfusepath{stroke}%
\end{pgfscope}%
\begin{pgfscope}%
\pgfpathrectangle{\pgfqpoint{1.250000in}{1.750000in}}{\pgfqpoint{2.279412in}{2.004545in}}%
\pgfusepath{clip}%
\pgfsetbuttcap%
\pgfsetroundjoin%
\pgfsetlinewidth{0.321889pt}%
\definecolor{currentstroke}{rgb}{0.271305,0.019942,0.347269}%
\pgfsetstrokecolor{currentstroke}%
\pgfsetdash{}{0pt}%
\pgfpathmoveto{\pgfqpoint{3.057947in}{3.065440in}}%
\pgfpathlineto{\pgfqpoint{3.007826in}{3.064308in}}%
\pgfusepath{stroke}%
\end{pgfscope}%
\begin{pgfscope}%
\pgfpathrectangle{\pgfqpoint{1.250000in}{1.750000in}}{\pgfqpoint{2.279412in}{2.004545in}}%
\pgfusepath{clip}%
\pgfsetbuttcap%
\pgfsetroundjoin%
\pgfsetlinewidth{0.338438pt}%
\definecolor{currentstroke}{rgb}{0.273809,0.031497,0.358853}%
\pgfsetstrokecolor{currentstroke}%
\pgfsetdash{}{0pt}%
\pgfpathmoveto{\pgfqpoint{3.007826in}{3.064308in}}%
\pgfpathlineto{\pgfqpoint{2.957748in}{3.062384in}}%
\pgfusepath{stroke}%
\end{pgfscope}%
\begin{pgfscope}%
\pgfpathrectangle{\pgfqpoint{1.250000in}{1.750000in}}{\pgfqpoint{2.279412in}{2.004545in}}%
\pgfusepath{clip}%
\pgfsetbuttcap%
\pgfsetroundjoin%
\pgfsetlinewidth{0.352825pt}%
\definecolor{currentstroke}{rgb}{0.276022,0.044167,0.370164}%
\pgfsetstrokecolor{currentstroke}%
\pgfsetdash{}{0pt}%
\pgfpathmoveto{\pgfqpoint{2.957748in}{3.062384in}}%
\pgfpathlineto{\pgfqpoint{2.907706in}{3.059675in}}%
\pgfusepath{stroke}%
\end{pgfscope}%
\begin{pgfscope}%
\pgfpathrectangle{\pgfqpoint{1.250000in}{1.750000in}}{\pgfqpoint{2.279412in}{2.004545in}}%
\pgfusepath{clip}%
\pgfsetbuttcap%
\pgfsetroundjoin%
\pgfsetlinewidth{0.385099pt}%
\definecolor{currentstroke}{rgb}{0.280267,0.073417,0.397163}%
\pgfsetstrokecolor{currentstroke}%
\pgfsetdash{}{0pt}%
\pgfpathmoveto{\pgfqpoint{2.907706in}{3.059675in}}%
\pgfpathlineto{\pgfqpoint{2.857656in}{3.056910in}}%
\pgfusepath{stroke}%
\end{pgfscope}%
\begin{pgfscope}%
\pgfpathrectangle{\pgfqpoint{1.250000in}{1.750000in}}{\pgfqpoint{2.279412in}{2.004545in}}%
\pgfusepath{clip}%
\pgfsetbuttcap%
\pgfsetroundjoin%
\pgfsetlinewidth{0.398495pt}%
\definecolor{currentstroke}{rgb}{0.281446,0.084320,0.407414}%
\pgfsetstrokecolor{currentstroke}%
\pgfsetdash{}{0pt}%
\pgfpathmoveto{\pgfqpoint{2.857656in}{3.056910in}}%
\pgfpathlineto{\pgfqpoint{2.807624in}{3.053936in}}%
\pgfusepath{stroke}%
\end{pgfscope}%
\begin{pgfscope}%
\pgfpathrectangle{\pgfqpoint{1.250000in}{1.750000in}}{\pgfqpoint{2.279412in}{2.004545in}}%
\pgfusepath{clip}%
\pgfsetbuttcap%
\pgfsetroundjoin%
\pgfsetlinewidth{0.436622pt}%
\definecolor{currentstroke}{rgb}{0.283091,0.110553,0.431554}%
\pgfsetstrokecolor{currentstroke}%
\pgfsetdash{}{0pt}%
\pgfpathmoveto{\pgfqpoint{2.807624in}{3.053936in}}%
\pgfpathlineto{\pgfqpoint{2.757663in}{3.050122in}}%
\pgfusepath{stroke}%
\end{pgfscope}%
\begin{pgfscope}%
\pgfpathrectangle{\pgfqpoint{1.250000in}{1.750000in}}{\pgfqpoint{2.279412in}{2.004545in}}%
\pgfusepath{clip}%
\pgfsetbuttcap%
\pgfsetroundjoin%
\pgfsetlinewidth{0.461249pt}%
\definecolor{currentstroke}{rgb}{0.283072,0.130895,0.449241}%
\pgfsetstrokecolor{currentstroke}%
\pgfsetdash{}{0pt}%
\pgfpathmoveto{\pgfqpoint{2.757663in}{3.050122in}}%
\pgfpathlineto{\pgfqpoint{2.707786in}{3.045536in}}%
\pgfusepath{stroke}%
\end{pgfscope}%
\begin{pgfscope}%
\pgfpathrectangle{\pgfqpoint{1.250000in}{1.750000in}}{\pgfqpoint{2.279412in}{2.004545in}}%
\pgfusepath{clip}%
\pgfsetbuttcap%
\pgfsetroundjoin%
\pgfsetlinewidth{0.318419pt}%
\definecolor{currentstroke}{rgb}{0.269944,0.014625,0.341379}%
\pgfsetstrokecolor{currentstroke}%
\pgfsetdash{}{0pt}%
\pgfpathmoveto{\pgfqpoint{3.107792in}{3.113127in}}%
\pgfpathlineto{\pgfqpoint{3.057674in}{3.111987in}}%
\pgfusepath{stroke}%
\end{pgfscope}%
\begin{pgfscope}%
\pgfpathrectangle{\pgfqpoint{1.250000in}{1.750000in}}{\pgfqpoint{2.279412in}{2.004545in}}%
\pgfusepath{clip}%
\pgfsetbuttcap%
\pgfsetroundjoin%
\pgfsetlinewidth{0.332076pt}%
\definecolor{currentstroke}{rgb}{0.272594,0.025563,0.353093}%
\pgfsetstrokecolor{currentstroke}%
\pgfsetdash{}{0pt}%
\pgfpathmoveto{\pgfqpoint{3.057674in}{3.111987in}}%
\pgfpathlineto{\pgfqpoint{3.007587in}{3.110278in}}%
\pgfusepath{stroke}%
\end{pgfscope}%
\begin{pgfscope}%
\pgfpathrectangle{\pgfqpoint{1.250000in}{1.750000in}}{\pgfqpoint{2.279412in}{2.004545in}}%
\pgfusepath{clip}%
\pgfsetbuttcap%
\pgfsetroundjoin%
\pgfsetlinewidth{0.342549pt}%
\definecolor{currentstroke}{rgb}{0.274952,0.037752,0.364543}%
\pgfsetstrokecolor{currentstroke}%
\pgfsetdash{}{0pt}%
\pgfpathmoveto{\pgfqpoint{3.007587in}{3.110278in}}%
\pgfpathlineto{\pgfqpoint{2.957493in}{3.108668in}}%
\pgfusepath{stroke}%
\end{pgfscope}%
\begin{pgfscope}%
\pgfpathrectangle{\pgfqpoint{1.250000in}{1.750000in}}{\pgfqpoint{2.279412in}{2.004545in}}%
\pgfusepath{clip}%
\pgfsetbuttcap%
\pgfsetroundjoin%
\pgfsetlinewidth{0.352555pt}%
\definecolor{currentstroke}{rgb}{0.276022,0.044167,0.370164}%
\pgfsetstrokecolor{currentstroke}%
\pgfsetdash{}{0pt}%
\pgfpathmoveto{\pgfqpoint{2.957493in}{3.108668in}}%
\pgfpathlineto{\pgfqpoint{2.907452in}{3.106495in}}%
\pgfusepath{stroke}%
\end{pgfscope}%
\begin{pgfscope}%
\pgfpathrectangle{\pgfqpoint{1.250000in}{1.750000in}}{\pgfqpoint{2.279412in}{2.004545in}}%
\pgfusepath{clip}%
\pgfsetbuttcap%
\pgfsetroundjoin%
\pgfsetlinewidth{0.372187pt}%
\definecolor{currentstroke}{rgb}{0.278791,0.062145,0.386592}%
\pgfsetstrokecolor{currentstroke}%
\pgfsetdash{}{0pt}%
\pgfpathmoveto{\pgfqpoint{2.907452in}{3.106495in}}%
\pgfpathlineto{\pgfqpoint{2.857463in}{3.103023in}}%
\pgfusepath{stroke}%
\end{pgfscope}%
\begin{pgfscope}%
\pgfpathrectangle{\pgfqpoint{1.250000in}{1.750000in}}{\pgfqpoint{2.279412in}{2.004545in}}%
\pgfusepath{clip}%
\pgfsetbuttcap%
\pgfsetroundjoin%
\pgfsetlinewidth{0.375675pt}%
\definecolor{currentstroke}{rgb}{0.278791,0.062145,0.386592}%
\pgfsetstrokecolor{currentstroke}%
\pgfsetdash{}{0pt}%
\pgfpathmoveto{\pgfqpoint{2.857463in}{3.103023in}}%
\pgfpathlineto{\pgfqpoint{2.807473in}{3.099514in}}%
\pgfusepath{stroke}%
\end{pgfscope}%
\begin{pgfscope}%
\pgfpathrectangle{\pgfqpoint{1.250000in}{1.750000in}}{\pgfqpoint{2.279412in}{2.004545in}}%
\pgfusepath{clip}%
\pgfsetbuttcap%
\pgfsetroundjoin%
\pgfsetlinewidth{0.402319pt}%
\definecolor{currentstroke}{rgb}{0.281446,0.084320,0.407414}%
\pgfsetstrokecolor{currentstroke}%
\pgfsetdash{}{0pt}%
\pgfpathmoveto{\pgfqpoint{2.807473in}{3.099514in}}%
\pgfpathlineto{\pgfqpoint{2.757605in}{3.094907in}}%
\pgfusepath{stroke}%
\end{pgfscope}%
\begin{pgfscope}%
\pgfpathrectangle{\pgfqpoint{1.250000in}{1.750000in}}{\pgfqpoint{2.279412in}{2.004545in}}%
\pgfusepath{clip}%
\pgfsetbuttcap%
\pgfsetroundjoin%
\pgfsetlinewidth{0.325537pt}%
\definecolor{currentstroke}{rgb}{0.271305,0.019942,0.347269}%
\pgfsetstrokecolor{currentstroke}%
\pgfsetdash{}{0pt}%
\pgfpathmoveto{\pgfqpoint{3.107792in}{3.158234in}}%
\pgfpathlineto{\pgfqpoint{3.057664in}{3.158475in}}%
\pgfusepath{stroke}%
\end{pgfscope}%
\begin{pgfscope}%
\pgfpathrectangle{\pgfqpoint{1.250000in}{1.750000in}}{\pgfqpoint{2.279412in}{2.004545in}}%
\pgfusepath{clip}%
\pgfsetbuttcap%
\pgfsetroundjoin%
\pgfsetlinewidth{0.328219pt}%
\definecolor{currentstroke}{rgb}{0.271305,0.019942,0.347269}%
\pgfsetstrokecolor{currentstroke}%
\pgfsetdash{}{0pt}%
\pgfpathmoveto{\pgfqpoint{3.057664in}{3.158475in}}%
\pgfpathlineto{\pgfqpoint{3.007597in}{3.157482in}}%
\pgfusepath{stroke}%
\end{pgfscope}%
\begin{pgfscope}%
\pgfpathrectangle{\pgfqpoint{1.250000in}{1.750000in}}{\pgfqpoint{2.279412in}{2.004545in}}%
\pgfusepath{clip}%
\pgfsetbuttcap%
\pgfsetroundjoin%
\pgfsetlinewidth{0.331031pt}%
\definecolor{currentstroke}{rgb}{0.272594,0.025563,0.353093}%
\pgfsetstrokecolor{currentstroke}%
\pgfsetdash{}{0pt}%
\pgfpathmoveto{\pgfqpoint{3.007597in}{3.157482in}}%
\pgfpathlineto{\pgfqpoint{2.957517in}{3.155783in}}%
\pgfusepath{stroke}%
\end{pgfscope}%
\begin{pgfscope}%
\pgfpathrectangle{\pgfqpoint{1.250000in}{1.750000in}}{\pgfqpoint{2.279412in}{2.004545in}}%
\pgfusepath{clip}%
\pgfsetbuttcap%
\pgfsetroundjoin%
\pgfsetlinewidth{0.354432pt}%
\definecolor{currentstroke}{rgb}{0.276022,0.044167,0.370164}%
\pgfsetstrokecolor{currentstroke}%
\pgfsetdash{}{0pt}%
\pgfpathmoveto{\pgfqpoint{2.957517in}{3.155783in}}%
\pgfpathlineto{\pgfqpoint{2.907412in}{3.154198in}}%
\pgfusepath{stroke}%
\end{pgfscope}%
\begin{pgfscope}%
\pgfpathrectangle{\pgfqpoint{1.250000in}{1.750000in}}{\pgfqpoint{2.279412in}{2.004545in}}%
\pgfusepath{clip}%
\pgfsetbuttcap%
\pgfsetroundjoin%
\pgfsetlinewidth{0.354569pt}%
\definecolor{currentstroke}{rgb}{0.276022,0.044167,0.370164}%
\pgfsetstrokecolor{currentstroke}%
\pgfsetdash{}{0pt}%
\pgfpathmoveto{\pgfqpoint{2.907412in}{3.154198in}}%
\pgfpathlineto{\pgfqpoint{2.857355in}{3.151555in}}%
\pgfusepath{stroke}%
\end{pgfscope}%
\begin{pgfscope}%
\pgfpathrectangle{\pgfqpoint{1.250000in}{1.750000in}}{\pgfqpoint{2.279412in}{2.004545in}}%
\pgfusepath{clip}%
\pgfsetbuttcap%
\pgfsetroundjoin%
\pgfsetlinewidth{0.368546pt}%
\definecolor{currentstroke}{rgb}{0.277941,0.056324,0.381191}%
\pgfsetstrokecolor{currentstroke}%
\pgfsetdash{}{0pt}%
\pgfpathmoveto{\pgfqpoint{2.857355in}{3.151555in}}%
\pgfpathlineto{\pgfqpoint{2.807382in}{3.147956in}}%
\pgfusepath{stroke}%
\end{pgfscope}%
\begin{pgfscope}%
\pgfpathrectangle{\pgfqpoint{1.250000in}{1.750000in}}{\pgfqpoint{2.279412in}{2.004545in}}%
\pgfusepath{clip}%
\pgfsetbuttcap%
\pgfsetroundjoin%
\pgfsetlinewidth{0.384009pt}%
\definecolor{currentstroke}{rgb}{0.280267,0.073417,0.397163}%
\pgfsetstrokecolor{currentstroke}%
\pgfsetdash{}{0pt}%
\pgfpathmoveto{\pgfqpoint{2.807382in}{3.147956in}}%
\pgfpathlineto{\pgfqpoint{2.757543in}{3.143099in}}%
\pgfusepath{stroke}%
\end{pgfscope}%
\begin{pgfscope}%
\pgfpathrectangle{\pgfqpoint{1.250000in}{1.750000in}}{\pgfqpoint{2.279412in}{2.004545in}}%
\pgfusepath{clip}%
\pgfsetbuttcap%
\pgfsetroundjoin%
\pgfsetlinewidth{0.392659pt}%
\definecolor{currentstroke}{rgb}{0.280894,0.078907,0.402329}%
\pgfsetstrokecolor{currentstroke}%
\pgfsetdash{}{0pt}%
\pgfpathmoveto{\pgfqpoint{2.757543in}{3.143099in}}%
\pgfpathlineto{\pgfqpoint{2.707829in}{3.137341in}}%
\pgfusepath{stroke}%
\end{pgfscope}%
\begin{pgfscope}%
\pgfpathrectangle{\pgfqpoint{1.250000in}{1.750000in}}{\pgfqpoint{2.279412in}{2.004545in}}%
\pgfusepath{clip}%
\pgfsetbuttcap%
\pgfsetroundjoin%
\pgfsetlinewidth{0.391705pt}%
\definecolor{currentstroke}{rgb}{0.280894,0.078907,0.402329}%
\pgfsetstrokecolor{currentstroke}%
\pgfsetdash{}{0pt}%
\pgfpathmoveto{\pgfqpoint{2.707829in}{3.137341in}}%
\pgfpathlineto{\pgfqpoint{2.658359in}{3.130227in}}%
\pgfusepath{stroke}%
\end{pgfscope}%
\begin{pgfscope}%
\pgfpathrectangle{\pgfqpoint{1.250000in}{1.750000in}}{\pgfqpoint{2.279412in}{2.004545in}}%
\pgfusepath{clip}%
\pgfsetbuttcap%
\pgfsetroundjoin%
\pgfsetlinewidth{0.407281pt}%
\definecolor{currentstroke}{rgb}{0.281924,0.089666,0.412415}%
\pgfsetstrokecolor{currentstroke}%
\pgfsetdash{}{0pt}%
\pgfpathmoveto{\pgfqpoint{2.658359in}{3.130227in}}%
\pgfpathlineto{\pgfqpoint{2.609151in}{3.121760in}}%
\pgfusepath{stroke}%
\end{pgfscope}%
\begin{pgfscope}%
\pgfpathrectangle{\pgfqpoint{1.250000in}{1.750000in}}{\pgfqpoint{2.279412in}{2.004545in}}%
\pgfusepath{clip}%
\pgfsetbuttcap%
\pgfsetroundjoin%
\pgfsetlinewidth{0.433833pt}%
\definecolor{currentstroke}{rgb}{0.283091,0.110553,0.431554}%
\pgfsetstrokecolor{currentstroke}%
\pgfsetdash{}{0pt}%
\pgfpathmoveto{\pgfqpoint{2.609151in}{3.121760in}}%
\pgfpathlineto{\pgfqpoint{2.560246in}{3.112090in}}%
\pgfusepath{stroke}%
\end{pgfscope}%
\begin{pgfscope}%
\pgfpathrectangle{\pgfqpoint{1.250000in}{1.750000in}}{\pgfqpoint{2.279412in}{2.004545in}}%
\pgfusepath{clip}%
\pgfsetbuttcap%
\pgfsetroundjoin%
\pgfsetlinewidth{0.434481pt}%
\definecolor{currentstroke}{rgb}{0.283091,0.110553,0.431554}%
\pgfsetstrokecolor{currentstroke}%
\pgfsetdash{}{0pt}%
\pgfpathmoveto{\pgfqpoint{2.560246in}{3.112090in}}%
\pgfpathlineto{\pgfqpoint{2.512089in}{3.099967in}}%
\pgfusepath{stroke}%
\end{pgfscope}%
\begin{pgfscope}%
\pgfpathrectangle{\pgfqpoint{1.250000in}{1.750000in}}{\pgfqpoint{2.279412in}{2.004545in}}%
\pgfusepath{clip}%
\pgfsetbuttcap%
\pgfsetroundjoin%
\pgfsetlinewidth{0.432016pt}%
\definecolor{currentstroke}{rgb}{0.283091,0.110553,0.431554}%
\pgfsetstrokecolor{currentstroke}%
\pgfsetdash{}{0pt}%
\pgfpathmoveto{\pgfqpoint{2.512089in}{3.099967in}}%
\pgfpathlineto{\pgfqpoint{2.465219in}{3.084443in}}%
\pgfusepath{stroke}%
\end{pgfscope}%
\begin{pgfscope}%
\pgfpathrectangle{\pgfqpoint{1.250000in}{1.750000in}}{\pgfqpoint{2.279412in}{2.004545in}}%
\pgfusepath{clip}%
\pgfsetbuttcap%
\pgfsetroundjoin%
\pgfsetlinewidth{0.525397pt}%
\definecolor{currentstroke}{rgb}{0.278826,0.175490,0.483397}%
\pgfsetstrokecolor{currentstroke}%
\pgfsetdash{}{0pt}%
\pgfpathmoveto{\pgfqpoint{2.465219in}{3.084443in}}%
\pgfpathlineto{\pgfqpoint{2.419809in}{3.065944in}}%
\pgfusepath{stroke}%
\end{pgfscope}%
\begin{pgfscope}%
\pgfpathrectangle{\pgfqpoint{1.250000in}{1.750000in}}{\pgfqpoint{2.279412in}{2.004545in}}%
\pgfusepath{clip}%
\pgfsetbuttcap%
\pgfsetroundjoin%
\pgfsetlinewidth{0.519969pt}%
\definecolor{currentstroke}{rgb}{0.279574,0.170599,0.479997}%
\pgfsetstrokecolor{currentstroke}%
\pgfsetdash{}{0pt}%
\pgfpathmoveto{\pgfqpoint{2.419809in}{3.065944in}}%
\pgfpathlineto{\pgfqpoint{2.375612in}{3.045293in}}%
\pgfusepath{stroke}%
\end{pgfscope}%
\begin{pgfscope}%
\pgfpathrectangle{\pgfqpoint{1.250000in}{1.750000in}}{\pgfqpoint{2.279412in}{2.004545in}}%
\pgfusepath{clip}%
\pgfsetbuttcap%
\pgfsetroundjoin%
\pgfsetlinewidth{0.765358pt}%
\definecolor{currentstroke}{rgb}{0.223925,0.334994,0.548053}%
\pgfsetstrokecolor{currentstroke}%
\pgfsetdash{}{0pt}%
\pgfpathmoveto{\pgfqpoint{2.375612in}{3.045293in}}%
\pgfpathlineto{\pgfqpoint{2.332569in}{3.022719in}}%
\pgfusepath{stroke}%
\end{pgfscope}%
\begin{pgfscope}%
\pgfpathrectangle{\pgfqpoint{1.250000in}{1.750000in}}{\pgfqpoint{2.279412in}{2.004545in}}%
\pgfusepath{clip}%
\pgfsetbuttcap%
\pgfsetroundjoin%
\pgfsetlinewidth{0.922381pt}%
\definecolor{currentstroke}{rgb}{0.182256,0.426184,0.557120}%
\pgfsetstrokecolor{currentstroke}%
\pgfsetdash{}{0pt}%
\pgfpathmoveto{\pgfqpoint{2.332569in}{3.022719in}}%
\pgfpathlineto{\pgfqpoint{2.291074in}{2.998071in}}%
\pgfusepath{stroke}%
\end{pgfscope}%
\begin{pgfscope}%
\pgfpathrectangle{\pgfqpoint{1.250000in}{1.750000in}}{\pgfqpoint{2.279412in}{2.004545in}}%
\pgfusepath{clip}%
\pgfsetbuttcap%
\pgfsetroundjoin%
\pgfsetlinewidth{0.876047pt}%
\definecolor{currentstroke}{rgb}{0.194100,0.399323,0.555565}%
\pgfsetstrokecolor{currentstroke}%
\pgfsetdash{}{0pt}%
\pgfpathmoveto{\pgfqpoint{2.291074in}{2.998071in}}%
\pgfpathlineto{\pgfqpoint{2.251651in}{2.970915in}}%
\pgfusepath{stroke}%
\end{pgfscope}%
\begin{pgfscope}%
\pgfpathrectangle{\pgfqpoint{1.250000in}{1.750000in}}{\pgfqpoint{2.279412in}{2.004545in}}%
\pgfusepath{clip}%
\pgfsetbuttcap%
\pgfsetroundjoin%
\pgfsetlinewidth{0.831986pt}%
\definecolor{currentstroke}{rgb}{0.204903,0.375746,0.553533}%
\pgfsetstrokecolor{currentstroke}%
\pgfsetdash{}{0pt}%
\pgfpathmoveto{\pgfqpoint{2.251651in}{2.970915in}}%
\pgfpathlineto{\pgfqpoint{2.212149in}{2.943829in}}%
\pgfusepath{stroke}%
\end{pgfscope}%
\begin{pgfscope}%
\pgfpathrectangle{\pgfqpoint{1.250000in}{1.750000in}}{\pgfqpoint{2.279412in}{2.004545in}}%
\pgfusepath{clip}%
\pgfsetbuttcap%
\pgfsetroundjoin%
\pgfsetlinewidth{1.384768pt}%
\definecolor{currentstroke}{rgb}{0.146616,0.673050,0.508936}%
\pgfsetstrokecolor{currentstroke}%
\pgfsetdash{}{0pt}%
\pgfpathmoveto{\pgfqpoint{2.212149in}{2.943829in}}%
\pgfpathlineto{\pgfqpoint{2.172652in}{2.916741in}}%
\pgfusepath{stroke}%
\end{pgfscope}%
\begin{pgfscope}%
\pgfpathrectangle{\pgfqpoint{1.250000in}{1.750000in}}{\pgfqpoint{2.279412in}{2.004545in}}%
\pgfusepath{clip}%
\pgfsetbuttcap%
\pgfsetroundjoin%
\pgfsetlinewidth{1.497277pt}%
\definecolor{currentstroke}{rgb}{0.232815,0.732247,0.459277}%
\pgfsetstrokecolor{currentstroke}%
\pgfsetdash{}{0pt}%
\pgfpathmoveto{\pgfqpoint{2.172652in}{2.916741in}}%
\pgfpathlineto{\pgfqpoint{2.133353in}{2.889407in}}%
\pgfusepath{stroke}%
\end{pgfscope}%
\begin{pgfscope}%
\pgfpathrectangle{\pgfqpoint{1.250000in}{1.750000in}}{\pgfqpoint{2.279412in}{2.004545in}}%
\pgfusepath{clip}%
\pgfsetbuttcap%
\pgfsetroundjoin%
\pgfsetlinewidth{0.317110pt}%
\definecolor{currentstroke}{rgb}{0.269944,0.014625,0.341379}%
\pgfsetstrokecolor{currentstroke}%
\pgfsetdash{}{0pt}%
\pgfpathmoveto{\pgfqpoint{3.056501in}{2.256098in}}%
\pgfpathlineto{\pgfqpoint{3.006590in}{2.257762in}}%
\pgfusepath{stroke}%
\end{pgfscope}%
\begin{pgfscope}%
\pgfpathrectangle{\pgfqpoint{1.250000in}{1.750000in}}{\pgfqpoint{2.279412in}{2.004545in}}%
\pgfusepath{clip}%
\pgfsetbuttcap%
\pgfsetroundjoin%
\pgfsetlinewidth{0.316278pt}%
\definecolor{currentstroke}{rgb}{0.269944,0.014625,0.341379}%
\pgfsetstrokecolor{currentstroke}%
\pgfsetdash{}{0pt}%
\pgfpathmoveto{\pgfqpoint{3.006590in}{2.257762in}}%
\pgfpathlineto{\pgfqpoint{2.956602in}{2.260122in}}%
\pgfusepath{stroke}%
\end{pgfscope}%
\begin{pgfscope}%
\pgfpathrectangle{\pgfqpoint{1.250000in}{1.750000in}}{\pgfqpoint{2.279412in}{2.004545in}}%
\pgfusepath{clip}%
\pgfsetbuttcap%
\pgfsetroundjoin%
\pgfsetlinewidth{0.327154pt}%
\definecolor{currentstroke}{rgb}{0.271305,0.019942,0.347269}%
\pgfsetstrokecolor{currentstroke}%
\pgfsetdash{}{0pt}%
\pgfpathmoveto{\pgfqpoint{2.956602in}{2.260122in}}%
\pgfpathlineto{\pgfqpoint{2.906551in}{2.262696in}}%
\pgfusepath{stroke}%
\end{pgfscope}%
\begin{pgfscope}%
\pgfpathrectangle{\pgfqpoint{1.250000in}{1.750000in}}{\pgfqpoint{2.279412in}{2.004545in}}%
\pgfusepath{clip}%
\pgfsetbuttcap%
\pgfsetroundjoin%
\pgfsetlinewidth{0.338910pt}%
\definecolor{currentstroke}{rgb}{0.273809,0.031497,0.358853}%
\pgfsetstrokecolor{currentstroke}%
\pgfsetdash{}{0pt}%
\pgfpathmoveto{\pgfqpoint{2.906551in}{2.262696in}}%
\pgfpathlineto{\pgfqpoint{2.856667in}{2.266965in}}%
\pgfusepath{stroke}%
\end{pgfscope}%
\begin{pgfscope}%
\pgfpathrectangle{\pgfqpoint{1.250000in}{1.750000in}}{\pgfqpoint{2.279412in}{2.004545in}}%
\pgfusepath{clip}%
\pgfsetbuttcap%
\pgfsetroundjoin%
\pgfsetlinewidth{0.349828pt}%
\definecolor{currentstroke}{rgb}{0.276022,0.044167,0.370164}%
\pgfsetstrokecolor{currentstroke}%
\pgfsetdash{}{0pt}%
\pgfpathmoveto{\pgfqpoint{2.856667in}{2.266965in}}%
\pgfpathlineto{\pgfqpoint{2.806856in}{2.271929in}}%
\pgfusepath{stroke}%
\end{pgfscope}%
\begin{pgfscope}%
\pgfpathrectangle{\pgfqpoint{1.250000in}{1.750000in}}{\pgfqpoint{2.279412in}{2.004545in}}%
\pgfusepath{clip}%
\pgfsetbuttcap%
\pgfsetroundjoin%
\pgfsetlinewidth{0.360658pt}%
\definecolor{currentstroke}{rgb}{0.277018,0.050344,0.375715}%
\pgfsetstrokecolor{currentstroke}%
\pgfsetdash{}{0pt}%
\pgfpathmoveto{\pgfqpoint{2.806856in}{2.271929in}}%
\pgfpathlineto{\pgfqpoint{2.757006in}{2.276692in}}%
\pgfusepath{stroke}%
\end{pgfscope}%
\begin{pgfscope}%
\pgfpathrectangle{\pgfqpoint{1.250000in}{1.750000in}}{\pgfqpoint{2.279412in}{2.004545in}}%
\pgfusepath{clip}%
\pgfsetbuttcap%
\pgfsetroundjoin%
\pgfsetlinewidth{0.374608pt}%
\definecolor{currentstroke}{rgb}{0.278791,0.062145,0.386592}%
\pgfsetstrokecolor{currentstroke}%
\pgfsetdash{}{0pt}%
\pgfpathmoveto{\pgfqpoint{2.757006in}{2.276692in}}%
\pgfpathlineto{\pgfqpoint{2.707210in}{2.281872in}}%
\pgfusepath{stroke}%
\end{pgfscope}%
\begin{pgfscope}%
\pgfpathrectangle{\pgfqpoint{1.250000in}{1.750000in}}{\pgfqpoint{2.279412in}{2.004545in}}%
\pgfusepath{clip}%
\pgfsetbuttcap%
\pgfsetroundjoin%
\pgfsetlinewidth{0.378945pt}%
\definecolor{currentstroke}{rgb}{0.279566,0.067836,0.391917}%
\pgfsetstrokecolor{currentstroke}%
\pgfsetdash{}{0pt}%
\pgfpathmoveto{\pgfqpoint{2.707210in}{2.281872in}}%
\pgfpathlineto{\pgfqpoint{2.657858in}{2.289306in}}%
\pgfusepath{stroke}%
\end{pgfscope}%
\begin{pgfscope}%
\pgfpathrectangle{\pgfqpoint{1.250000in}{1.750000in}}{\pgfqpoint{2.279412in}{2.004545in}}%
\pgfusepath{clip}%
\pgfsetbuttcap%
\pgfsetroundjoin%
\pgfsetlinewidth{0.386684pt}%
\definecolor{currentstroke}{rgb}{0.280267,0.073417,0.397163}%
\pgfsetstrokecolor{currentstroke}%
\pgfsetdash{}{0pt}%
\pgfpathmoveto{\pgfqpoint{2.657858in}{2.289306in}}%
\pgfpathlineto{\pgfqpoint{2.608891in}{2.298690in}}%
\pgfusepath{stroke}%
\end{pgfscope}%
\begin{pgfscope}%
\pgfpathrectangle{\pgfqpoint{1.250000in}{1.750000in}}{\pgfqpoint{2.279412in}{2.004545in}}%
\pgfusepath{clip}%
\pgfsetbuttcap%
\pgfsetroundjoin%
\pgfsetlinewidth{0.323382pt}%
\definecolor{currentstroke}{rgb}{0.271305,0.019942,0.347269}%
\pgfsetstrokecolor{currentstroke}%
\pgfsetdash{}{0pt}%
\pgfpathmoveto{\pgfqpoint{3.056501in}{3.203341in}}%
\pgfpathlineto{\pgfqpoint{3.006514in}{3.200802in}}%
\pgfusepath{stroke}%
\end{pgfscope}%
\begin{pgfscope}%
\pgfpathrectangle{\pgfqpoint{1.250000in}{1.750000in}}{\pgfqpoint{2.279412in}{2.004545in}}%
\pgfusepath{clip}%
\pgfsetbuttcap%
\pgfsetroundjoin%
\pgfsetlinewidth{0.326274pt}%
\definecolor{currentstroke}{rgb}{0.271305,0.019942,0.347269}%
\pgfsetstrokecolor{currentstroke}%
\pgfsetdash{}{0pt}%
\pgfpathmoveto{\pgfqpoint{3.006514in}{3.200802in}}%
\pgfpathlineto{\pgfqpoint{2.956422in}{3.198997in}}%
\pgfusepath{stroke}%
\end{pgfscope}%
\begin{pgfscope}%
\pgfpathrectangle{\pgfqpoint{1.250000in}{1.750000in}}{\pgfqpoint{2.279412in}{2.004545in}}%
\pgfusepath{clip}%
\pgfsetbuttcap%
\pgfsetroundjoin%
\pgfsetlinewidth{0.338646pt}%
\definecolor{currentstroke}{rgb}{0.273809,0.031497,0.358853}%
\pgfsetstrokecolor{currentstroke}%
\pgfsetdash{}{0pt}%
\pgfpathmoveto{\pgfqpoint{2.956422in}{3.198997in}}%
\pgfpathlineto{\pgfqpoint{2.906342in}{3.196825in}}%
\pgfusepath{stroke}%
\end{pgfscope}%
\begin{pgfscope}%
\pgfpathrectangle{\pgfqpoint{1.250000in}{1.750000in}}{\pgfqpoint{2.279412in}{2.004545in}}%
\pgfusepath{clip}%
\pgfsetbuttcap%
\pgfsetroundjoin%
\pgfsetlinewidth{0.344738pt}%
\definecolor{currentstroke}{rgb}{0.274952,0.037752,0.364543}%
\pgfsetstrokecolor{currentstroke}%
\pgfsetdash{}{0pt}%
\pgfpathmoveto{\pgfqpoint{2.906342in}{3.196825in}}%
\pgfpathlineto{\pgfqpoint{2.856337in}{3.193632in}}%
\pgfusepath{stroke}%
\end{pgfscope}%
\begin{pgfscope}%
\pgfpathrectangle{\pgfqpoint{1.250000in}{1.750000in}}{\pgfqpoint{2.279412in}{2.004545in}}%
\pgfusepath{clip}%
\pgfsetbuttcap%
\pgfsetroundjoin%
\pgfsetlinewidth{0.362418pt}%
\definecolor{currentstroke}{rgb}{0.277018,0.050344,0.375715}%
\pgfsetstrokecolor{currentstroke}%
\pgfsetdash{}{0pt}%
\pgfpathmoveto{\pgfqpoint{2.856337in}{3.193632in}}%
\pgfpathlineto{\pgfqpoint{2.806525in}{3.188681in}}%
\pgfusepath{stroke}%
\end{pgfscope}%
\begin{pgfscope}%
\pgfpathrectangle{\pgfqpoint{1.250000in}{1.750000in}}{\pgfqpoint{2.279412in}{2.004545in}}%
\pgfusepath{clip}%
\pgfsetbuttcap%
\pgfsetroundjoin%
\pgfsetlinewidth{0.339778pt}%
\definecolor{currentstroke}{rgb}{0.273809,0.031497,0.358853}%
\pgfsetstrokecolor{currentstroke}%
\pgfsetdash{}{0pt}%
\pgfpathmoveto{\pgfqpoint{2.806525in}{3.188681in}}%
\pgfpathlineto{\pgfqpoint{2.756774in}{3.183266in}}%
\pgfusepath{stroke}%
\end{pgfscope}%
\begin{pgfscope}%
\pgfpathrectangle{\pgfqpoint{1.250000in}{1.750000in}}{\pgfqpoint{2.279412in}{2.004545in}}%
\pgfusepath{clip}%
\pgfsetbuttcap%
\pgfsetroundjoin%
\pgfsetlinewidth{0.318889pt}%
\definecolor{currentstroke}{rgb}{0.269944,0.014625,0.341379}%
\pgfsetstrokecolor{currentstroke}%
\pgfsetdash{}{0pt}%
\pgfpathmoveto{\pgfqpoint{3.056501in}{3.248447in}}%
\pgfpathlineto{\pgfqpoint{3.006689in}{3.244878in}}%
\pgfusepath{stroke}%
\end{pgfscope}%
\begin{pgfscope}%
\pgfpathrectangle{\pgfqpoint{1.250000in}{1.750000in}}{\pgfqpoint{2.279412in}{2.004545in}}%
\pgfusepath{clip}%
\pgfsetbuttcap%
\pgfsetroundjoin%
\pgfsetlinewidth{0.325662pt}%
\definecolor{currentstroke}{rgb}{0.271305,0.019942,0.347269}%
\pgfsetstrokecolor{currentstroke}%
\pgfsetdash{}{0pt}%
\pgfpathmoveto{\pgfqpoint{3.006689in}{3.244878in}}%
\pgfpathlineto{\pgfqpoint{2.956917in}{3.240122in}}%
\pgfusepath{stroke}%
\end{pgfscope}%
\begin{pgfscope}%
\pgfpathrectangle{\pgfqpoint{1.250000in}{1.750000in}}{\pgfqpoint{2.279412in}{2.004545in}}%
\pgfusepath{clip}%
\pgfsetbuttcap%
\pgfsetroundjoin%
\pgfsetlinewidth{0.340760pt}%
\definecolor{currentstroke}{rgb}{0.273809,0.031497,0.358853}%
\pgfsetstrokecolor{currentstroke}%
\pgfsetdash{}{0pt}%
\pgfpathmoveto{\pgfqpoint{2.956917in}{3.240122in}}%
\pgfpathlineto{\pgfqpoint{2.906978in}{3.236492in}}%
\pgfusepath{stroke}%
\end{pgfscope}%
\begin{pgfscope}%
\pgfpathrectangle{\pgfqpoint{1.250000in}{1.750000in}}{\pgfqpoint{2.279412in}{2.004545in}}%
\pgfusepath{clip}%
\pgfsetbuttcap%
\pgfsetroundjoin%
\pgfsetlinewidth{0.334152pt}%
\definecolor{currentstroke}{rgb}{0.272594,0.025563,0.353093}%
\pgfsetstrokecolor{currentstroke}%
\pgfsetdash{}{0pt}%
\pgfpathmoveto{\pgfqpoint{2.906978in}{3.236492in}}%
\pgfpathlineto{\pgfqpoint{2.857119in}{3.232025in}}%
\pgfusepath{stroke}%
\end{pgfscope}%
\begin{pgfscope}%
\pgfpathrectangle{\pgfqpoint{1.250000in}{1.750000in}}{\pgfqpoint{2.279412in}{2.004545in}}%
\pgfusepath{clip}%
\pgfsetbuttcap%
\pgfsetroundjoin%
\pgfsetlinewidth{0.345433pt}%
\definecolor{currentstroke}{rgb}{0.274952,0.037752,0.364543}%
\pgfsetstrokecolor{currentstroke}%
\pgfsetdash{}{0pt}%
\pgfpathmoveto{\pgfqpoint{2.857119in}{3.232025in}}%
\pgfpathlineto{\pgfqpoint{2.807304in}{3.227004in}}%
\pgfusepath{stroke}%
\end{pgfscope}%
\begin{pgfscope}%
\pgfpathrectangle{\pgfqpoint{1.250000in}{1.750000in}}{\pgfqpoint{2.279412in}{2.004545in}}%
\pgfusepath{clip}%
\pgfsetbuttcap%
\pgfsetroundjoin%
\pgfsetlinewidth{0.318203pt}%
\definecolor{currentstroke}{rgb}{0.269944,0.014625,0.341379}%
\pgfsetstrokecolor{currentstroke}%
\pgfsetdash{}{0pt}%
\pgfpathmoveto{\pgfqpoint{3.056501in}{3.293554in}}%
\pgfpathlineto{\pgfqpoint{3.006868in}{3.291340in}}%
\pgfusepath{stroke}%
\end{pgfscope}%
\begin{pgfscope}%
\pgfpathrectangle{\pgfqpoint{1.250000in}{1.750000in}}{\pgfqpoint{2.279412in}{2.004545in}}%
\pgfusepath{clip}%
\pgfsetbuttcap%
\pgfsetroundjoin%
\pgfsetlinewidth{0.323240pt}%
\definecolor{currentstroke}{rgb}{0.271305,0.019942,0.347269}%
\pgfsetstrokecolor{currentstroke}%
\pgfsetdash{}{0pt}%
\pgfpathmoveto{\pgfqpoint{3.006868in}{3.291340in}}%
\pgfpathlineto{\pgfqpoint{2.957191in}{3.287323in}}%
\pgfusepath{stroke}%
\end{pgfscope}%
\begin{pgfscope}%
\pgfpathrectangle{\pgfqpoint{1.250000in}{1.750000in}}{\pgfqpoint{2.279412in}{2.004545in}}%
\pgfusepath{clip}%
\pgfsetbuttcap%
\pgfsetroundjoin%
\pgfsetlinewidth{0.316501pt}%
\definecolor{currentstroke}{rgb}{0.269944,0.014625,0.341379}%
\pgfsetstrokecolor{currentstroke}%
\pgfsetdash{}{0pt}%
\pgfpathmoveto{\pgfqpoint{2.957191in}{3.287323in}}%
\pgfpathlineto{\pgfqpoint{2.907659in}{3.281367in}}%
\pgfusepath{stroke}%
\end{pgfscope}%
\begin{pgfscope}%
\pgfpathrectangle{\pgfqpoint{1.250000in}{1.750000in}}{\pgfqpoint{2.279412in}{2.004545in}}%
\pgfusepath{clip}%
\pgfsetbuttcap%
\pgfsetroundjoin%
\pgfsetlinewidth{0.326744pt}%
\definecolor{currentstroke}{rgb}{0.271305,0.019942,0.347269}%
\pgfsetstrokecolor{currentstroke}%
\pgfsetdash{}{0pt}%
\pgfpathmoveto{\pgfqpoint{2.907659in}{3.281367in}}%
\pgfpathlineto{\pgfqpoint{2.857815in}{3.276595in}}%
\pgfusepath{stroke}%
\end{pgfscope}%
\begin{pgfscope}%
\pgfpathrectangle{\pgfqpoint{1.250000in}{1.750000in}}{\pgfqpoint{2.279412in}{2.004545in}}%
\pgfusepath{clip}%
\pgfsetbuttcap%
\pgfsetroundjoin%
\pgfsetlinewidth{0.340024pt}%
\definecolor{currentstroke}{rgb}{0.273809,0.031497,0.358853}%
\pgfsetstrokecolor{currentstroke}%
\pgfsetdash{}{0pt}%
\pgfpathmoveto{\pgfqpoint{2.857815in}{3.276595in}}%
\pgfpathlineto{\pgfqpoint{2.807952in}{3.272043in}}%
\pgfusepath{stroke}%
\end{pgfscope}%
\begin{pgfscope}%
\pgfpathrectangle{\pgfqpoint{1.250000in}{1.750000in}}{\pgfqpoint{2.279412in}{2.004545in}}%
\pgfusepath{clip}%
\pgfsetbuttcap%
\pgfsetroundjoin%
\pgfsetlinewidth{0.343765pt}%
\definecolor{currentstroke}{rgb}{0.274952,0.037752,0.364543}%
\pgfsetstrokecolor{currentstroke}%
\pgfsetdash{}{0pt}%
\pgfpathmoveto{\pgfqpoint{2.807952in}{3.272043in}}%
\pgfpathlineto{\pgfqpoint{2.758087in}{3.267667in}}%
\pgfusepath{stroke}%
\end{pgfscope}%
\begin{pgfscope}%
\pgfpathrectangle{\pgfqpoint{1.250000in}{1.750000in}}{\pgfqpoint{2.279412in}{2.004545in}}%
\pgfusepath{clip}%
\pgfsetbuttcap%
\pgfsetroundjoin%
\pgfsetlinewidth{0.322446pt}%
\definecolor{currentstroke}{rgb}{0.271305,0.019942,0.347269}%
\pgfsetstrokecolor{currentstroke}%
\pgfsetdash{}{0pt}%
\pgfpathmoveto{\pgfqpoint{2.945372in}{2.181084in}}%
\pgfpathlineto{\pgfqpoint{2.895548in}{2.185015in}}%
\pgfusepath{stroke}%
\end{pgfscope}%
\begin{pgfscope}%
\pgfpathrectangle{\pgfqpoint{1.250000in}{1.750000in}}{\pgfqpoint{2.279412in}{2.004545in}}%
\pgfusepath{clip}%
\pgfsetbuttcap%
\pgfsetroundjoin%
\pgfsetlinewidth{0.328434pt}%
\definecolor{currentstroke}{rgb}{0.271305,0.019942,0.347269}%
\pgfsetstrokecolor{currentstroke}%
\pgfsetdash{}{0pt}%
\pgfpathmoveto{\pgfqpoint{2.895548in}{2.185015in}}%
\pgfpathlineto{\pgfqpoint{2.845793in}{2.189731in}}%
\pgfusepath{stroke}%
\end{pgfscope}%
\begin{pgfscope}%
\pgfpathrectangle{\pgfqpoint{1.250000in}{1.750000in}}{\pgfqpoint{2.279412in}{2.004545in}}%
\pgfusepath{clip}%
\pgfsetbuttcap%
\pgfsetroundjoin%
\pgfsetlinewidth{0.336762pt}%
\definecolor{currentstroke}{rgb}{0.273809,0.031497,0.358853}%
\pgfsetstrokecolor{currentstroke}%
\pgfsetdash{}{0pt}%
\pgfpathmoveto{\pgfqpoint{2.845793in}{2.189731in}}%
\pgfpathlineto{\pgfqpoint{2.796203in}{2.195951in}}%
\pgfusepath{stroke}%
\end{pgfscope}%
\begin{pgfscope}%
\pgfpathrectangle{\pgfqpoint{1.250000in}{1.750000in}}{\pgfqpoint{2.279412in}{2.004545in}}%
\pgfusepath{clip}%
\pgfsetbuttcap%
\pgfsetroundjoin%
\pgfsetlinewidth{0.356941pt}%
\definecolor{currentstroke}{rgb}{0.277018,0.050344,0.375715}%
\pgfsetstrokecolor{currentstroke}%
\pgfsetdash{}{0pt}%
\pgfpathmoveto{\pgfqpoint{2.796203in}{2.195951in}}%
\pgfpathlineto{\pgfqpoint{2.746858in}{2.203620in}}%
\pgfusepath{stroke}%
\end{pgfscope}%
\begin{pgfscope}%
\pgfpathrectangle{\pgfqpoint{1.250000in}{1.750000in}}{\pgfqpoint{2.279412in}{2.004545in}}%
\pgfusepath{clip}%
\pgfsetbuttcap%
\pgfsetroundjoin%
\pgfsetlinewidth{0.356794pt}%
\definecolor{currentstroke}{rgb}{0.277018,0.050344,0.375715}%
\pgfsetstrokecolor{currentstroke}%
\pgfsetdash{}{0pt}%
\pgfpathmoveto{\pgfqpoint{2.746858in}{2.203620in}}%
\pgfpathlineto{\pgfqpoint{2.697457in}{2.210991in}}%
\pgfusepath{stroke}%
\end{pgfscope}%
\begin{pgfscope}%
\pgfpathrectangle{\pgfqpoint{1.250000in}{1.750000in}}{\pgfqpoint{2.279412in}{2.004545in}}%
\pgfusepath{clip}%
\pgfsetbuttcap%
\pgfsetroundjoin%
\pgfsetlinewidth{0.351657pt}%
\definecolor{currentstroke}{rgb}{0.276022,0.044167,0.370164}%
\pgfsetstrokecolor{currentstroke}%
\pgfsetdash{}{0pt}%
\pgfpathmoveto{\pgfqpoint{2.697457in}{2.210991in}}%
\pgfpathlineto{\pgfqpoint{2.647973in}{2.218055in}}%
\pgfusepath{stroke}%
\end{pgfscope}%
\begin{pgfscope}%
\pgfpathrectangle{\pgfqpoint{1.250000in}{1.750000in}}{\pgfqpoint{2.279412in}{2.004545in}}%
\pgfusepath{clip}%
\pgfsetbuttcap%
\pgfsetroundjoin%
\pgfsetlinewidth{0.360877pt}%
\definecolor{currentstroke}{rgb}{0.277018,0.050344,0.375715}%
\pgfsetstrokecolor{currentstroke}%
\pgfsetdash{}{0pt}%
\pgfpathmoveto{\pgfqpoint{2.693104in}{2.238360in}}%
\pgfpathlineto{\pgfqpoint{2.643916in}{2.246917in}}%
\pgfusepath{stroke}%
\end{pgfscope}%
\begin{pgfscope}%
\pgfpathrectangle{\pgfqpoint{1.250000in}{1.750000in}}{\pgfqpoint{2.279412in}{2.004545in}}%
\pgfusepath{clip}%
\pgfsetbuttcap%
\pgfsetroundjoin%
\pgfsetlinewidth{0.331253pt}%
\definecolor{currentstroke}{rgb}{0.272594,0.025563,0.353093}%
\pgfsetstrokecolor{currentstroke}%
\pgfsetdash{}{0pt}%
\pgfpathmoveto{\pgfqpoint{2.643916in}{2.246917in}}%
\pgfpathlineto{\pgfqpoint{2.594873in}{2.256098in}}%
\pgfusepath{stroke}%
\end{pgfscope}%
\begin{pgfscope}%
\pgfpathrectangle{\pgfqpoint{1.250000in}{1.750000in}}{\pgfqpoint{2.279412in}{2.004545in}}%
\pgfusepath{clip}%
\pgfsetbuttcap%
\pgfsetroundjoin%
\pgfsetlinewidth{0.363590pt}%
\definecolor{currentstroke}{rgb}{0.277941,0.056324,0.381191}%
\pgfsetstrokecolor{currentstroke}%
\pgfsetdash{}{0pt}%
\pgfpathmoveto{\pgfqpoint{2.594873in}{2.256098in}}%
\pgfpathlineto{\pgfqpoint{2.546385in}{2.267223in}}%
\pgfusepath{stroke}%
\end{pgfscope}%
\begin{pgfscope}%
\pgfpathrectangle{\pgfqpoint{1.250000in}{1.750000in}}{\pgfqpoint{2.279412in}{2.004545in}}%
\pgfusepath{clip}%
\pgfsetbuttcap%
\pgfsetroundjoin%
\pgfsetlinewidth{0.391029pt}%
\definecolor{currentstroke}{rgb}{0.280894,0.078907,0.402329}%
\pgfsetstrokecolor{currentstroke}%
\pgfsetdash{}{0pt}%
\pgfpathmoveto{\pgfqpoint{2.546385in}{2.267223in}}%
\pgfpathlineto{\pgfqpoint{2.499251in}{2.281927in}}%
\pgfusepath{stroke}%
\end{pgfscope}%
\begin{pgfscope}%
\pgfpathrectangle{\pgfqpoint{1.250000in}{1.750000in}}{\pgfqpoint{2.279412in}{2.004545in}}%
\pgfusepath{clip}%
\pgfsetbuttcap%
\pgfsetroundjoin%
\pgfsetlinewidth{0.398366pt}%
\definecolor{currentstroke}{rgb}{0.281446,0.084320,0.407414}%
\pgfsetstrokecolor{currentstroke}%
\pgfsetdash{}{0pt}%
\pgfpathmoveto{\pgfqpoint{2.499251in}{2.281927in}}%
\pgfpathlineto{\pgfqpoint{2.453557in}{2.299759in}}%
\pgfusepath{stroke}%
\end{pgfscope}%
\begin{pgfscope}%
\pgfpathrectangle{\pgfqpoint{1.250000in}{1.750000in}}{\pgfqpoint{2.279412in}{2.004545in}}%
\pgfusepath{clip}%
\pgfsetbuttcap%
\pgfsetroundjoin%
\pgfsetlinewidth{0.361789pt}%
\definecolor{currentstroke}{rgb}{0.277018,0.050344,0.375715}%
\pgfsetstrokecolor{currentstroke}%
\pgfsetdash{}{0pt}%
\pgfpathmoveto{\pgfqpoint{2.690339in}{3.231318in}}%
\pgfpathlineto{\pgfqpoint{2.640874in}{3.224323in}}%
\pgfusepath{stroke}%
\end{pgfscope}%
\begin{pgfscope}%
\pgfpathrectangle{\pgfqpoint{1.250000in}{1.750000in}}{\pgfqpoint{2.279412in}{2.004545in}}%
\pgfusepath{clip}%
\pgfsetbuttcap%
\pgfsetroundjoin%
\pgfsetlinewidth{0.370121pt}%
\definecolor{currentstroke}{rgb}{0.278791,0.062145,0.386592}%
\pgfsetstrokecolor{currentstroke}%
\pgfsetdash{}{0pt}%
\pgfpathmoveto{\pgfqpoint{2.640874in}{3.224323in}}%
\pgfpathlineto{\pgfqpoint{2.591978in}{3.214766in}}%
\pgfusepath{stroke}%
\end{pgfscope}%
\begin{pgfscope}%
\pgfpathrectangle{\pgfqpoint{1.250000in}{1.750000in}}{\pgfqpoint{2.279412in}{2.004545in}}%
\pgfusepath{clip}%
\pgfsetbuttcap%
\pgfsetroundjoin%
\pgfsetlinewidth{0.370805pt}%
\definecolor{currentstroke}{rgb}{0.278791,0.062145,0.386592}%
\pgfsetstrokecolor{currentstroke}%
\pgfsetdash{}{0pt}%
\pgfpathmoveto{\pgfqpoint{2.591978in}{3.214766in}}%
\pgfpathlineto{\pgfqpoint{2.543582in}{3.203341in}}%
\pgfusepath{stroke}%
\end{pgfscope}%
\begin{pgfscope}%
\pgfpathrectangle{\pgfqpoint{1.250000in}{1.750000in}}{\pgfqpoint{2.279412in}{2.004545in}}%
\pgfusepath{clip}%
\pgfsetbuttcap%
\pgfsetroundjoin%
\pgfsetlinewidth{0.394037pt}%
\definecolor{currentstroke}{rgb}{0.280894,0.078907,0.402329}%
\pgfsetstrokecolor{currentstroke}%
\pgfsetdash{}{0pt}%
\pgfpathmoveto{\pgfqpoint{2.543582in}{3.203341in}}%
\pgfpathlineto{\pgfqpoint{2.495827in}{3.190304in}}%
\pgfusepath{stroke}%
\end{pgfscope}%
\begin{pgfscope}%
\pgfpathrectangle{\pgfqpoint{1.250000in}{1.750000in}}{\pgfqpoint{2.279412in}{2.004545in}}%
\pgfusepath{clip}%
\pgfsetbuttcap%
\pgfsetroundjoin%
\pgfsetlinewidth{0.347209pt}%
\definecolor{currentstroke}{rgb}{0.274952,0.037752,0.364543}%
\pgfsetstrokecolor{currentstroke}%
\pgfsetdash{}{0pt}%
\pgfpathmoveto{\pgfqpoint{2.495827in}{3.190304in}}%
\pgfpathlineto{\pgfqpoint{2.450745in}{3.171902in}}%
\pgfusepath{stroke}%
\end{pgfscope}%
\begin{pgfscope}%
\pgfpathrectangle{\pgfqpoint{1.250000in}{1.750000in}}{\pgfqpoint{2.279412in}{2.004545in}}%
\pgfusepath{clip}%
\pgfsetbuttcap%
\pgfsetroundjoin%
\pgfsetlinewidth{0.728689pt}%
\definecolor{currentstroke}{rgb}{0.233603,0.313828,0.543914}%
\pgfsetstrokecolor{currentstroke}%
\pgfsetdash{}{0pt}%
\pgfpathmoveto{\pgfqpoint{2.287122in}{2.436525in}}%
\pgfpathlineto{\pgfqpoint{2.251431in}{2.467453in}}%
\pgfusepath{stroke}%
\end{pgfscope}%
\begin{pgfscope}%
\pgfpathrectangle{\pgfqpoint{1.250000in}{1.750000in}}{\pgfqpoint{2.279412in}{2.004545in}}%
\pgfusepath{clip}%
\pgfsetbuttcap%
\pgfsetroundjoin%
\pgfsetlinewidth{0.965641pt}%
\definecolor{currentstroke}{rgb}{0.172719,0.448791,0.557885}%
\pgfsetstrokecolor{currentstroke}%
\pgfsetdash{}{0pt}%
\pgfpathmoveto{\pgfqpoint{2.251431in}{2.467453in}}%
\pgfpathlineto{\pgfqpoint{2.221269in}{2.501016in}}%
\pgfusepath{stroke}%
\end{pgfscope}%
\begin{pgfscope}%
\pgfpathrectangle{\pgfqpoint{1.250000in}{1.750000in}}{\pgfqpoint{2.279412in}{2.004545in}}%
\pgfusepath{clip}%
\pgfsetbuttcap%
\pgfsetroundjoin%
\pgfsetlinewidth{0.698171pt}%
\definecolor{currentstroke}{rgb}{0.243113,0.292092,0.538516}%
\pgfsetstrokecolor{currentstroke}%
\pgfsetdash{}{0pt}%
\pgfpathmoveto{\pgfqpoint{2.221269in}{2.501016in}}%
\pgfpathlineto{\pgfqpoint{2.189060in}{2.530376in}}%
\pgfusepath{stroke}%
\end{pgfscope}%
\begin{pgfscope}%
\pgfpathrectangle{\pgfqpoint{1.250000in}{1.750000in}}{\pgfqpoint{2.279412in}{2.004545in}}%
\pgfusepath{clip}%
\pgfsetbuttcap%
\pgfsetroundjoin%
\pgfsetlinewidth{1.236329pt}%
\definecolor{currentstroke}{rgb}{0.120565,0.596422,0.543611}%
\pgfsetstrokecolor{currentstroke}%
\pgfsetdash{}{0pt}%
\pgfpathmoveto{\pgfqpoint{2.189060in}{2.530376in}}%
\pgfpathlineto{\pgfqpoint{2.152327in}{2.560229in}}%
\pgfusepath{stroke}%
\end{pgfscope}%
\begin{pgfscope}%
\pgfpathrectangle{\pgfqpoint{1.250000in}{1.750000in}}{\pgfqpoint{2.279412in}{2.004545in}}%
\pgfusepath{clip}%
\pgfsetbuttcap%
\pgfsetroundjoin%
\pgfsetlinewidth{1.787770pt}%
\definecolor{currentstroke}{rgb}{0.595839,0.848717,0.243329}%
\pgfsetstrokecolor{currentstroke}%
\pgfsetdash{}{0pt}%
\pgfpathmoveto{\pgfqpoint{2.152327in}{2.560229in}}%
\pgfpathlineto{\pgfqpoint{2.114821in}{2.589345in}}%
\pgfusepath{stroke}%
\end{pgfscope}%
\begin{pgfscope}%
\pgfpathrectangle{\pgfqpoint{1.250000in}{1.750000in}}{\pgfqpoint{2.279412in}{2.004545in}}%
\pgfusepath{clip}%
\pgfsetbuttcap%
\pgfsetroundjoin%
\pgfsetlinewidth{0.434665pt}%
\definecolor{currentstroke}{rgb}{0.283091,0.110553,0.431554}%
\pgfsetstrokecolor{currentstroke}%
\pgfsetdash{}{0pt}%
\pgfpathmoveto{\pgfqpoint{2.691051in}{3.093301in}}%
\pgfpathlineto{\pgfqpoint{2.641407in}{3.087063in}}%
\pgfusepath{stroke}%
\end{pgfscope}%
\begin{pgfscope}%
\pgfpathrectangle{\pgfqpoint{1.250000in}{1.750000in}}{\pgfqpoint{2.279412in}{2.004545in}}%
\pgfusepath{clip}%
\pgfsetbuttcap%
\pgfsetroundjoin%
\pgfsetlinewidth{0.451864pt}%
\definecolor{currentstroke}{rgb}{0.283229,0.120777,0.440584}%
\pgfsetstrokecolor{currentstroke}%
\pgfsetdash{}{0pt}%
\pgfpathmoveto{\pgfqpoint{2.641407in}{3.087063in}}%
\pgfpathlineto{\pgfqpoint{2.592182in}{3.078845in}}%
\pgfusepath{stroke}%
\end{pgfscope}%
\begin{pgfscope}%
\pgfpathrectangle{\pgfqpoint{1.250000in}{1.750000in}}{\pgfqpoint{2.279412in}{2.004545in}}%
\pgfusepath{clip}%
\pgfsetbuttcap%
\pgfsetroundjoin%
\pgfsetlinewidth{0.502956pt}%
\definecolor{currentstroke}{rgb}{0.280868,0.160771,0.472899}%
\pgfsetstrokecolor{currentstroke}%
\pgfsetdash{}{0pt}%
\pgfpathmoveto{\pgfqpoint{2.592182in}{3.078845in}}%
\pgfpathlineto{\pgfqpoint{2.543582in}{3.068020in}}%
\pgfusepath{stroke}%
\end{pgfscope}%
\begin{pgfscope}%
\pgfpathrectangle{\pgfqpoint{1.250000in}{1.750000in}}{\pgfqpoint{2.279412in}{2.004545in}}%
\pgfusepath{clip}%
\pgfsetbuttcap%
\pgfsetroundjoin%
\pgfsetlinewidth{0.497151pt}%
\definecolor{currentstroke}{rgb}{0.281412,0.155834,0.469201}%
\pgfsetstrokecolor{currentstroke}%
\pgfsetdash{}{0pt}%
\pgfpathmoveto{\pgfqpoint{2.543582in}{3.068020in}}%
\pgfpathlineto{\pgfqpoint{2.495987in}{3.054566in}}%
\pgfusepath{stroke}%
\end{pgfscope}%
\begin{pgfscope}%
\pgfpathrectangle{\pgfqpoint{1.250000in}{1.750000in}}{\pgfqpoint{2.279412in}{2.004545in}}%
\pgfusepath{clip}%
\pgfsetbuttcap%
\pgfsetroundjoin%
\pgfsetlinewidth{0.432602pt}%
\definecolor{currentstroke}{rgb}{0.283091,0.110553,0.431554}%
\pgfsetstrokecolor{currentstroke}%
\pgfsetdash{}{0pt}%
\pgfpathmoveto{\pgfqpoint{2.495987in}{3.054566in}}%
\pgfpathlineto{\pgfqpoint{2.449233in}{3.038872in}}%
\pgfusepath{stroke}%
\end{pgfscope}%
\begin{pgfscope}%
\pgfpathrectangle{\pgfqpoint{1.250000in}{1.750000in}}{\pgfqpoint{2.279412in}{2.004545in}}%
\pgfusepath{clip}%
\pgfsetbuttcap%
\pgfsetroundjoin%
\pgfsetlinewidth{0.791914pt}%
\definecolor{currentstroke}{rgb}{0.216210,0.351535,0.550627}%
\pgfsetstrokecolor{currentstroke}%
\pgfsetdash{}{0pt}%
\pgfpathmoveto{\pgfqpoint{2.133246in}{3.068020in}}%
\pgfpathlineto{\pgfqpoint{2.133246in}{3.068020in}}%
\pgfusepath{stroke}%
\end{pgfscope}%
\begin{pgfscope}%
\pgfpathrectangle{\pgfqpoint{1.250000in}{1.750000in}}{\pgfqpoint{2.279412in}{2.004545in}}%
\pgfusepath{clip}%
\pgfsetbuttcap%
\pgfsetroundjoin%
\pgfsetlinewidth{0.791914pt}%
\definecolor{currentstroke}{rgb}{0.216210,0.351535,0.550627}%
\pgfsetstrokecolor{currentstroke}%
\pgfsetdash{}{0pt}%
\pgfpathmoveto{\pgfqpoint{2.133246in}{3.068020in}}%
\pgfpathlineto{\pgfqpoint{2.135228in}{3.036278in}}%
\pgfusepath{stroke}%
\end{pgfscope}%
\begin{pgfscope}%
\pgfpathrectangle{\pgfqpoint{1.250000in}{1.750000in}}{\pgfqpoint{2.279412in}{2.004545in}}%
\pgfusepath{clip}%
\pgfsetbuttcap%
\pgfsetroundjoin%
\pgfsetlinewidth{1.017424pt}%
\definecolor{currentstroke}{rgb}{0.160665,0.478540,0.558115}%
\pgfsetstrokecolor{currentstroke}%
\pgfsetdash{}{0pt}%
\pgfpathmoveto{\pgfqpoint{2.135228in}{3.036278in}}%
\pgfpathlineto{\pgfqpoint{2.132559in}{3.007114in}}%
\pgfusepath{stroke}%
\end{pgfscope}%
\begin{pgfscope}%
\pgfpathrectangle{\pgfqpoint{1.250000in}{1.750000in}}{\pgfqpoint{2.279412in}{2.004545in}}%
\pgfusepath{clip}%
\pgfsetbuttcap%
\pgfsetroundjoin%
\pgfsetlinewidth{0.966536pt}%
\definecolor{currentstroke}{rgb}{0.172719,0.448791,0.557885}%
\pgfsetstrokecolor{currentstroke}%
\pgfsetdash{}{0pt}%
\pgfpathmoveto{\pgfqpoint{2.132559in}{3.007114in}}%
\pgfpathlineto{\pgfqpoint{2.132559in}{3.007114in}}%
\pgfusepath{stroke}%
\end{pgfscope}%
\begin{pgfscope}%
\pgfpathrectangle{\pgfqpoint{1.250000in}{1.750000in}}{\pgfqpoint{2.279412in}{2.004545in}}%
\pgfusepath{clip}%
\pgfsetbuttcap%
\pgfsetroundjoin%
\pgfsetlinewidth{0.966536pt}%
\definecolor{currentstroke}{rgb}{0.172719,0.448791,0.557885}%
\pgfsetstrokecolor{currentstroke}%
\pgfsetdash{}{0pt}%
\pgfpathmoveto{\pgfqpoint{2.132559in}{3.007114in}}%
\pgfpathlineto{\pgfqpoint{2.128517in}{2.977625in}}%
\pgfusepath{stroke}%
\end{pgfscope}%
\begin{pgfscope}%
\pgfpathrectangle{\pgfqpoint{1.250000in}{1.750000in}}{\pgfqpoint{2.279412in}{2.004545in}}%
\pgfusepath{clip}%
\pgfsetbuttcap%
\pgfsetroundjoin%
\pgfsetlinewidth{0.846787pt}%
\definecolor{currentstroke}{rgb}{0.201239,0.383670,0.554294}%
\pgfsetstrokecolor{currentstroke}%
\pgfsetdash{}{0pt}%
\pgfpathmoveto{\pgfqpoint{2.128517in}{2.977625in}}%
\pgfpathlineto{\pgfqpoint{2.128517in}{2.977625in}}%
\pgfusepath{stroke}%
\end{pgfscope}%
\begin{pgfscope}%
\pgfpathrectangle{\pgfqpoint{1.250000in}{1.750000in}}{\pgfqpoint{2.279412in}{2.004545in}}%
\pgfusepath{clip}%
\pgfsetbuttcap%
\pgfsetroundjoin%
\pgfsetlinewidth{0.846787pt}%
\definecolor{currentstroke}{rgb}{0.201239,0.383670,0.554294}%
\pgfsetstrokecolor{currentstroke}%
\pgfsetdash{}{0pt}%
\pgfpathmoveto{\pgfqpoint{2.128517in}{2.977625in}}%
\pgfpathlineto{\pgfqpoint{2.118483in}{2.955174in}}%
\pgfusepath{stroke}%
\end{pgfscope}%
\begin{pgfscope}%
\pgfpathrectangle{\pgfqpoint{1.250000in}{1.750000in}}{\pgfqpoint{2.279412in}{2.004545in}}%
\pgfusepath{clip}%
\pgfsetbuttcap%
\pgfsetroundjoin%
\pgfsetlinewidth{1.381056pt}%
\definecolor{currentstroke}{rgb}{0.146616,0.673050,0.508936}%
\pgfsetstrokecolor{currentstroke}%
\pgfsetdash{}{0pt}%
\pgfpathmoveto{\pgfqpoint{2.118483in}{2.955174in}}%
\pgfpathlineto{\pgfqpoint{2.103924in}{2.934830in}}%
\pgfusepath{stroke}%
\end{pgfscope}%
\begin{pgfscope}%
\pgfpathrectangle{\pgfqpoint{1.250000in}{1.750000in}}{\pgfqpoint{2.279412in}{2.004545in}}%
\pgfusepath{clip}%
\pgfsetbuttcap%
\pgfsetroundjoin%
\pgfsetlinewidth{1.512066pt}%
\definecolor{currentstroke}{rgb}{0.246070,0.738910,0.452024}%
\pgfsetstrokecolor{currentstroke}%
\pgfsetdash{}{0pt}%
\pgfpathmoveto{\pgfqpoint{2.103924in}{2.934830in}}%
\pgfpathlineto{\pgfqpoint{2.074391in}{2.899508in}}%
\pgfusepath{stroke}%
\end{pgfscope}%
\begin{pgfscope}%
\pgfpathrectangle{\pgfqpoint{1.250000in}{1.750000in}}{\pgfqpoint{2.279412in}{2.004545in}}%
\pgfusepath{clip}%
\pgfsetbuttcap%
\pgfsetroundjoin%
\pgfsetlinewidth{2.045854pt}%
\definecolor{currentstroke}{rgb}{0.983868,0.904867,0.136897}%
\pgfsetstrokecolor{currentstroke}%
\pgfsetdash{}{0pt}%
\pgfpathmoveto{\pgfqpoint{2.074391in}{2.899508in}}%
\pgfpathlineto{\pgfqpoint{2.045979in}{2.864745in}}%
\pgfusepath{stroke}%
\end{pgfscope}%
\begin{pgfscope}%
\pgfpathrectangle{\pgfqpoint{1.250000in}{1.750000in}}{\pgfqpoint{2.279412in}{2.004545in}}%
\pgfusepath{clip}%
\pgfsetbuttcap%
\pgfsetroundjoin%
\pgfsetlinewidth{1.355683pt}%
\definecolor{currentstroke}{rgb}{0.134692,0.658636,0.517649}%
\pgfsetstrokecolor{currentstroke}%
\pgfsetdash{}{0pt}%
\pgfpathmoveto{\pgfqpoint{2.045979in}{2.864745in}}%
\pgfpathlineto{\pgfqpoint{2.045979in}{2.864745in}}%
\pgfusepath{stroke}%
\end{pgfscope}%
\begin{pgfscope}%
\pgfpathrectangle{\pgfqpoint{1.250000in}{1.750000in}}{\pgfqpoint{2.279412in}{2.004545in}}%
\pgfusepath{clip}%
\pgfsetbuttcap%
\pgfsetroundjoin%
\pgfsetlinewidth{1.355683pt}%
\definecolor{currentstroke}{rgb}{0.134692,0.658636,0.517649}%
\pgfsetstrokecolor{currentstroke}%
\pgfsetdash{}{0pt}%
\pgfpathmoveto{\pgfqpoint{2.045979in}{2.864745in}}%
\pgfpathlineto{\pgfqpoint{2.045979in}{2.864745in}}%
\pgfusepath{stroke}%
\end{pgfscope}%
\begin{pgfscope}%
\pgfpathrectangle{\pgfqpoint{1.250000in}{1.750000in}}{\pgfqpoint{2.279412in}{2.004545in}}%
\pgfusepath{clip}%
\pgfsetbuttcap%
\pgfsetroundjoin%
\pgfsetlinewidth{1.355683pt}%
\definecolor{currentstroke}{rgb}{0.134692,0.658636,0.517649}%
\pgfsetstrokecolor{currentstroke}%
\pgfsetdash{}{0pt}%
\pgfpathmoveto{\pgfqpoint{2.045979in}{2.864745in}}%
\pgfpathlineto{\pgfqpoint{2.025647in}{2.846319in}}%
\pgfusepath{stroke}%
\end{pgfscope}%
\begin{pgfscope}%
\pgfpathrectangle{\pgfqpoint{1.250000in}{1.750000in}}{\pgfqpoint{2.279412in}{2.004545in}}%
\pgfusepath{clip}%
\pgfsetbuttcap%
\pgfsetroundjoin%
\pgfsetlinewidth{2.117182pt}%
\definecolor{currentstroke}{rgb}{0.993248,0.906157,0.143936}%
\pgfsetstrokecolor{currentstroke}%
\pgfsetdash{}{0pt}%
\pgfpathmoveto{\pgfqpoint{2.025647in}{2.846319in}}%
\pgfpathlineto{\pgfqpoint{2.003578in}{2.830534in}}%
\pgfusepath{stroke}%
\end{pgfscope}%
\begin{pgfscope}%
\pgfpathrectangle{\pgfqpoint{1.250000in}{1.750000in}}{\pgfqpoint{2.279412in}{2.004545in}}%
\pgfusepath{clip}%
\pgfsetbuttcap%
\pgfsetroundjoin%
\pgfsetlinewidth{1.222151pt}%
\definecolor{currentstroke}{rgb}{0.121831,0.589055,0.545623}%
\pgfsetstrokecolor{currentstroke}%
\pgfsetdash{}{0pt}%
\pgfpathmoveto{\pgfqpoint{2.543582in}{2.571846in}}%
\pgfpathlineto{\pgfqpoint{2.494176in}{2.579390in}}%
\pgfusepath{stroke}%
\end{pgfscope}%
\begin{pgfscope}%
\pgfpathrectangle{\pgfqpoint{1.250000in}{1.750000in}}{\pgfqpoint{2.279412in}{2.004545in}}%
\pgfusepath{clip}%
\pgfsetbuttcap%
\pgfsetroundjoin%
\pgfsetlinewidth{1.292544pt}%
\definecolor{currentstroke}{rgb}{0.120638,0.625828,0.533488}%
\pgfsetstrokecolor{currentstroke}%
\pgfsetdash{}{0pt}%
\pgfpathmoveto{\pgfqpoint{2.494176in}{2.579390in}}%
\pgfpathlineto{\pgfqpoint{2.444979in}{2.587928in}}%
\pgfusepath{stroke}%
\end{pgfscope}%
\begin{pgfscope}%
\pgfpathrectangle{\pgfqpoint{1.250000in}{1.750000in}}{\pgfqpoint{2.279412in}{2.004545in}}%
\pgfusepath{clip}%
\pgfsetbuttcap%
\pgfsetroundjoin%
\pgfsetlinewidth{1.298395pt}%
\definecolor{currentstroke}{rgb}{0.121380,0.629492,0.531973}%
\pgfsetstrokecolor{currentstroke}%
\pgfsetdash{}{0pt}%
\pgfpathmoveto{\pgfqpoint{2.444979in}{2.587928in}}%
\pgfpathlineto{\pgfqpoint{2.396011in}{2.597418in}}%
\pgfusepath{stroke}%
\end{pgfscope}%
\begin{pgfscope}%
\pgfpathrectangle{\pgfqpoint{1.250000in}{1.750000in}}{\pgfqpoint{2.279412in}{2.004545in}}%
\pgfusepath{clip}%
\pgfsetbuttcap%
\pgfsetroundjoin%
\pgfsetlinewidth{1.573272pt}%
\definecolor{currentstroke}{rgb}{0.311925,0.767822,0.415586}%
\pgfsetstrokecolor{currentstroke}%
\pgfsetdash{}{0pt}%
\pgfpathmoveto{\pgfqpoint{2.396011in}{2.597418in}}%
\pgfpathlineto{\pgfqpoint{2.347306in}{2.607900in}}%
\pgfusepath{stroke}%
\end{pgfscope}%
\begin{pgfscope}%
\pgfpathrectangle{\pgfqpoint{1.250000in}{1.750000in}}{\pgfqpoint{2.279412in}{2.004545in}}%
\pgfusepath{clip}%
\pgfsetbuttcap%
\pgfsetroundjoin%
\pgfsetlinewidth{1.856345pt}%
\definecolor{currentstroke}{rgb}{0.699415,0.867117,0.175971}%
\pgfsetstrokecolor{currentstroke}%
\pgfsetdash{}{0pt}%
\pgfpathmoveto{\pgfqpoint{2.347306in}{2.607900in}}%
\pgfpathlineto{\pgfqpoint{2.298819in}{2.619137in}}%
\pgfusepath{stroke}%
\end{pgfscope}%
\begin{pgfscope}%
\pgfpathrectangle{\pgfqpoint{1.250000in}{1.750000in}}{\pgfqpoint{2.279412in}{2.004545in}}%
\pgfusepath{clip}%
\pgfsetroundcap%
\pgfsetroundjoin%
\pgfsetlinewidth{1.410375pt}%
\definecolor{currentstroke}{rgb}{0.162016,0.687316,0.499129}%
\pgfsetstrokecolor{currentstroke}%
\pgfsetdash{}{0pt}%
\pgfpathmoveto{\pgfqpoint{2.623520in}{2.825612in}}%
\pgfpathquadraticcurveto{\pgfqpoint{2.611007in}{2.824920in}}{\pgfqpoint{2.620280in}{2.825433in}}%
\pgfusepath{stroke}%
\end{pgfscope}%
\begin{pgfscope}%
\pgfpathrectangle{\pgfqpoint{1.250000in}{1.750000in}}{\pgfqpoint{2.279412in}{2.004545in}}%
\pgfusepath{clip}%
\pgfsetroundcap%
\pgfsetroundjoin%
\definecolor{currentfill}{rgb}{0.162016,0.687316,0.499129}%
\pgfsetfillcolor{currentfill}%
\pgfsetlinewidth{1.410375pt}%
\definecolor{currentstroke}{rgb}{0.162016,0.687316,0.499129}%
\pgfsetstrokecolor{currentstroke}%
\pgfsetdash{}{0pt}%
\pgfpathmoveto{\pgfqpoint{2.677284in}{2.800762in}}%
\pgfpathlineto{\pgfqpoint{2.620280in}{2.825433in}}%
\pgfpathlineto{\pgfqpoint{2.674219in}{2.856233in}}%
\pgfpathlineto{\pgfqpoint{2.677284in}{2.800762in}}%
\pgfpathlineto{\pgfqpoint{2.677284in}{2.800762in}}%
\pgfpathclose%
\pgfusepath{stroke,fill}%
\end{pgfscope}%
\begin{pgfscope}%
\pgfpathrectangle{\pgfqpoint{1.250000in}{1.750000in}}{\pgfqpoint{2.279412in}{2.004545in}}%
\pgfusepath{clip}%
\pgfsetroundcap%
\pgfsetroundjoin%
\pgfsetlinewidth{0.612056pt}%
\definecolor{currentstroke}{rgb}{0.263663,0.237631,0.518762}%
\pgfsetstrokecolor{currentstroke}%
\pgfsetdash{}{0pt}%
\pgfpathmoveto{\pgfqpoint{2.782535in}{2.878836in}}%
\pgfpathquadraticcurveto{\pgfqpoint{2.770008in}{2.878364in}}{\pgfqpoint{2.766944in}{2.878249in}}%
\pgfusepath{stroke}%
\end{pgfscope}%
\begin{pgfscope}%
\pgfpathrectangle{\pgfqpoint{1.250000in}{1.750000in}}{\pgfqpoint{2.279412in}{2.004545in}}%
\pgfusepath{clip}%
\pgfsetroundcap%
\pgfsetroundjoin%
\definecolor{currentfill}{rgb}{0.263663,0.237631,0.518762}%
\pgfsetfillcolor{currentfill}%
\pgfsetlinewidth{0.612056pt}%
\definecolor{currentstroke}{rgb}{0.263663,0.237631,0.518762}%
\pgfsetstrokecolor{currentstroke}%
\pgfsetdash{}{0pt}%
\pgfpathmoveto{\pgfqpoint{2.823507in}{2.852584in}}%
\pgfpathlineto{\pgfqpoint{2.766944in}{2.878249in}}%
\pgfpathlineto{\pgfqpoint{2.821414in}{2.908100in}}%
\pgfpathlineto{\pgfqpoint{2.823507in}{2.852584in}}%
\pgfpathlineto{\pgfqpoint{2.823507in}{2.852584in}}%
\pgfpathclose%
\pgfusepath{stroke,fill}%
\end{pgfscope}%
\begin{pgfscope}%
\pgfpathrectangle{\pgfqpoint{1.250000in}{1.750000in}}{\pgfqpoint{2.279412in}{2.004545in}}%
\pgfusepath{clip}%
\pgfsetroundcap%
\pgfsetroundjoin%
\pgfsetlinewidth{0.886117pt}%
\definecolor{currentstroke}{rgb}{0.190631,0.407061,0.556089}%
\pgfsetstrokecolor{currentstroke}%
\pgfsetdash{}{0pt}%
\pgfpathmoveto{\pgfqpoint{2.560697in}{2.520874in}}%
\pgfpathquadraticcurveto{\pgfqpoint{2.548407in}{2.523042in}}{\pgfqpoint{2.549617in}{2.522829in}}%
\pgfusepath{stroke}%
\end{pgfscope}%
\begin{pgfscope}%
\pgfpathrectangle{\pgfqpoint{1.250000in}{1.750000in}}{\pgfqpoint{2.279412in}{2.004545in}}%
\pgfusepath{clip}%
\pgfsetroundcap%
\pgfsetroundjoin%
\definecolor{currentfill}{rgb}{0.190631,0.407061,0.556089}%
\pgfsetfillcolor{currentfill}%
\pgfsetlinewidth{0.886117pt}%
\definecolor{currentstroke}{rgb}{0.190631,0.407061,0.556089}%
\pgfsetstrokecolor{currentstroke}%
\pgfsetdash{}{0pt}%
\pgfpathmoveto{\pgfqpoint{2.599503in}{2.485823in}}%
\pgfpathlineto{\pgfqpoint{2.549617in}{2.522829in}}%
\pgfpathlineto{\pgfqpoint{2.609152in}{2.540534in}}%
\pgfpathlineto{\pgfqpoint{2.599503in}{2.485823in}}%
\pgfpathlineto{\pgfqpoint{2.599503in}{2.485823in}}%
\pgfpathclose%
\pgfusepath{stroke,fill}%
\end{pgfscope}%
\begin{pgfscope}%
\pgfpathrectangle{\pgfqpoint{1.250000in}{1.750000in}}{\pgfqpoint{2.279412in}{2.004545in}}%
\pgfusepath{clip}%
\pgfsetroundcap%
\pgfsetroundjoin%
\pgfsetlinewidth{0.369778pt}%
\definecolor{currentstroke}{rgb}{0.278791,0.062145,0.386592}%
\pgfsetstrokecolor{currentstroke}%
\pgfsetdash{}{0pt}%
\pgfpathmoveto{\pgfqpoint{2.959943in}{2.533348in}}%
\pgfpathquadraticcurveto{\pgfqpoint{2.947415in}{2.533773in}}{\pgfqpoint{2.940604in}{2.534004in}}%
\pgfusepath{stroke}%
\end{pgfscope}%
\begin{pgfscope}%
\pgfpathrectangle{\pgfqpoint{1.250000in}{1.750000in}}{\pgfqpoint{2.279412in}{2.004545in}}%
\pgfusepath{clip}%
\pgfsetroundcap%
\pgfsetroundjoin%
\definecolor{currentfill}{rgb}{0.278791,0.062145,0.386592}%
\pgfsetfillcolor{currentfill}%
\pgfsetlinewidth{0.369778pt}%
\definecolor{currentstroke}{rgb}{0.278791,0.062145,0.386592}%
\pgfsetstrokecolor{currentstroke}%
\pgfsetdash{}{0pt}%
\pgfpathmoveto{\pgfqpoint{2.995186in}{2.504358in}}%
\pgfpathlineto{\pgfqpoint{2.940604in}{2.534004in}}%
\pgfpathlineto{\pgfqpoint{2.997070in}{2.559882in}}%
\pgfpathlineto{\pgfqpoint{2.995186in}{2.504358in}}%
\pgfpathlineto{\pgfqpoint{2.995186in}{2.504358in}}%
\pgfpathclose%
\pgfusepath{stroke,fill}%
\end{pgfscope}%
\begin{pgfscope}%
\pgfpathrectangle{\pgfqpoint{1.250000in}{1.750000in}}{\pgfqpoint{2.279412in}{2.004545in}}%
\pgfusepath{clip}%
\pgfsetroundcap%
\pgfsetroundjoin%
\pgfsetlinewidth{1.297197pt}%
\definecolor{currentstroke}{rgb}{0.121380,0.629492,0.531973}%
\pgfsetstrokecolor{currentstroke}%
\pgfsetdash{}{0pt}%
\pgfpathmoveto{\pgfqpoint{2.609034in}{2.632793in}}%
\pgfpathquadraticcurveto{\pgfqpoint{2.596542in}{2.633738in}}{\pgfqpoint{2.604062in}{2.633169in}}%
\pgfusepath{stroke}%
\end{pgfscope}%
\begin{pgfscope}%
\pgfpathrectangle{\pgfqpoint{1.250000in}{1.750000in}}{\pgfqpoint{2.279412in}{2.004545in}}%
\pgfusepath{clip}%
\pgfsetroundcap%
\pgfsetroundjoin%
\definecolor{currentfill}{rgb}{0.121380,0.629492,0.531973}%
\pgfsetfillcolor{currentfill}%
\pgfsetlinewidth{1.297197pt}%
\definecolor{currentstroke}{rgb}{0.121380,0.629492,0.531973}%
\pgfsetstrokecolor{currentstroke}%
\pgfsetdash{}{0pt}%
\pgfpathmoveto{\pgfqpoint{2.657363in}{2.601280in}}%
\pgfpathlineto{\pgfqpoint{2.604062in}{2.633169in}}%
\pgfpathlineto{\pgfqpoint{2.661555in}{2.656677in}}%
\pgfpathlineto{\pgfqpoint{2.657363in}{2.601280in}}%
\pgfpathlineto{\pgfqpoint{2.657363in}{2.601280in}}%
\pgfpathclose%
\pgfusepath{stroke,fill}%
\end{pgfscope}%
\begin{pgfscope}%
\pgfpathrectangle{\pgfqpoint{1.250000in}{1.750000in}}{\pgfqpoint{2.279412in}{2.004545in}}%
\pgfusepath{clip}%
\pgfsetroundcap%
\pgfsetroundjoin%
\pgfsetlinewidth{0.422664pt}%
\definecolor{currentstroke}{rgb}{0.282656,0.100196,0.422160}%
\pgfsetstrokecolor{currentstroke}%
\pgfsetdash{}{0pt}%
\pgfpathmoveto{\pgfqpoint{2.513326in}{2.406989in}}%
\pgfpathquadraticcurveto{\pgfqpoint{2.501496in}{2.410584in}}{\pgfqpoint{2.495922in}{2.412278in}}%
\pgfusepath{stroke}%
\end{pgfscope}%
\begin{pgfscope}%
\pgfpathrectangle{\pgfqpoint{1.250000in}{1.750000in}}{\pgfqpoint{2.279412in}{2.004545in}}%
\pgfusepath{clip}%
\pgfsetroundcap%
\pgfsetroundjoin%
\definecolor{currentfill}{rgb}{0.282656,0.100196,0.422160}%
\pgfsetfillcolor{currentfill}%
\pgfsetlinewidth{0.422664pt}%
\definecolor{currentstroke}{rgb}{0.282656,0.100196,0.422160}%
\pgfsetstrokecolor{currentstroke}%
\pgfsetdash{}{0pt}%
\pgfpathmoveto{\pgfqpoint{2.541000in}{2.369547in}}%
\pgfpathlineto{\pgfqpoint{2.495922in}{2.412278in}}%
\pgfpathlineto{\pgfqpoint{2.557154in}{2.422702in}}%
\pgfpathlineto{\pgfqpoint{2.541000in}{2.369547in}}%
\pgfpathlineto{\pgfqpoint{2.541000in}{2.369547in}}%
\pgfpathclose%
\pgfusepath{stroke,fill}%
\end{pgfscope}%
\begin{pgfscope}%
\pgfpathrectangle{\pgfqpoint{1.250000in}{1.750000in}}{\pgfqpoint{2.279412in}{2.004545in}}%
\pgfusepath{clip}%
\pgfsetroundcap%
\pgfsetroundjoin%
\pgfsetlinewidth{0.495057pt}%
\definecolor{currentstroke}{rgb}{0.281412,0.155834,0.469201}%
\pgfsetstrokecolor{currentstroke}%
\pgfsetdash{}{0pt}%
\pgfpathmoveto{\pgfqpoint{2.708517in}{2.456146in}}%
\pgfpathquadraticcurveto{\pgfqpoint{2.696068in}{2.457454in}}{\pgfqpoint{2.691237in}{2.457962in}}%
\pgfusepath{stroke}%
\end{pgfscope}%
\begin{pgfscope}%
\pgfpathrectangle{\pgfqpoint{1.250000in}{1.750000in}}{\pgfqpoint{2.279412in}{2.004545in}}%
\pgfusepath{clip}%
\pgfsetroundcap%
\pgfsetroundjoin%
\definecolor{currentfill}{rgb}{0.281412,0.155834,0.469201}%
\pgfsetfillcolor{currentfill}%
\pgfsetlinewidth{0.495057pt}%
\definecolor{currentstroke}{rgb}{0.281412,0.155834,0.469201}%
\pgfsetstrokecolor{currentstroke}%
\pgfsetdash{}{0pt}%
\pgfpathmoveto{\pgfqpoint{2.743585in}{2.424530in}}%
\pgfpathlineto{\pgfqpoint{2.691237in}{2.457962in}}%
\pgfpathlineto{\pgfqpoint{2.749391in}{2.479781in}}%
\pgfpathlineto{\pgfqpoint{2.743585in}{2.424530in}}%
\pgfpathlineto{\pgfqpoint{2.743585in}{2.424530in}}%
\pgfpathclose%
\pgfusepath{stroke,fill}%
\end{pgfscope}%
\begin{pgfscope}%
\pgfpathrectangle{\pgfqpoint{1.250000in}{1.750000in}}{\pgfqpoint{2.279412in}{2.004545in}}%
\pgfusepath{clip}%
\pgfsetroundcap%
\pgfsetroundjoin%
\pgfsetlinewidth{0.407229pt}%
\definecolor{currentstroke}{rgb}{0.281924,0.089666,0.412415}%
\pgfsetstrokecolor{currentstroke}%
\pgfsetdash{}{0pt}%
\pgfpathmoveto{\pgfqpoint{2.908432in}{2.577626in}}%
\pgfpathquadraticcurveto{\pgfqpoint{2.895902in}{2.578011in}}{\pgfqpoint{2.889669in}{2.578203in}}%
\pgfusepath{stroke}%
\end{pgfscope}%
\begin{pgfscope}%
\pgfpathrectangle{\pgfqpoint{1.250000in}{1.750000in}}{\pgfqpoint{2.279412in}{2.004545in}}%
\pgfusepath{clip}%
\pgfsetroundcap%
\pgfsetroundjoin%
\definecolor{currentfill}{rgb}{0.281924,0.089666,0.412415}%
\pgfsetfillcolor{currentfill}%
\pgfsetlinewidth{0.407229pt}%
\definecolor{currentstroke}{rgb}{0.281924,0.089666,0.412415}%
\pgfsetstrokecolor{currentstroke}%
\pgfsetdash{}{0pt}%
\pgfpathmoveto{\pgfqpoint{2.944346in}{2.548732in}}%
\pgfpathlineto{\pgfqpoint{2.889669in}{2.578203in}}%
\pgfpathlineto{\pgfqpoint{2.946052in}{2.604262in}}%
\pgfpathlineto{\pgfqpoint{2.944346in}{2.548732in}}%
\pgfpathlineto{\pgfqpoint{2.944346in}{2.548732in}}%
\pgfpathclose%
\pgfusepath{stroke,fill}%
\end{pgfscope}%
\begin{pgfscope}%
\pgfpathrectangle{\pgfqpoint{1.250000in}{1.750000in}}{\pgfqpoint{2.279412in}{2.004545in}}%
\pgfusepath{clip}%
\pgfsetroundcap%
\pgfsetroundjoin%
\pgfsetlinewidth{0.509195pt}%
\definecolor{currentstroke}{rgb}{0.280255,0.165693,0.476498}%
\pgfsetstrokecolor{currentstroke}%
\pgfsetdash{}{0pt}%
\pgfpathmoveto{\pgfqpoint{2.858222in}{2.662221in}}%
\pgfpathquadraticcurveto{\pgfqpoint{2.845685in}{2.662315in}}{\pgfqpoint{2.841025in}{2.662349in}}%
\pgfusepath{stroke}%
\end{pgfscope}%
\begin{pgfscope}%
\pgfpathrectangle{\pgfqpoint{1.250000in}{1.750000in}}{\pgfqpoint{2.279412in}{2.004545in}}%
\pgfusepath{clip}%
\pgfsetroundcap%
\pgfsetroundjoin%
\definecolor{currentfill}{rgb}{0.280255,0.165693,0.476498}%
\pgfsetfillcolor{currentfill}%
\pgfsetlinewidth{0.509195pt}%
\definecolor{currentstroke}{rgb}{0.280255,0.165693,0.476498}%
\pgfsetstrokecolor{currentstroke}%
\pgfsetdash{}{0pt}%
\pgfpathmoveto{\pgfqpoint{2.896370in}{2.634156in}}%
\pgfpathlineto{\pgfqpoint{2.841025in}{2.662349in}}%
\pgfpathlineto{\pgfqpoint{2.896787in}{2.689710in}}%
\pgfpathlineto{\pgfqpoint{2.896370in}{2.634156in}}%
\pgfpathlineto{\pgfqpoint{2.896370in}{2.634156in}}%
\pgfpathclose%
\pgfusepath{stroke,fill}%
\end{pgfscope}%
\begin{pgfscope}%
\pgfpathrectangle{\pgfqpoint{1.250000in}{1.750000in}}{\pgfqpoint{2.279412in}{2.004545in}}%
\pgfusepath{clip}%
\pgfsetroundcap%
\pgfsetroundjoin%
\pgfsetlinewidth{1.077231pt}%
\definecolor{currentstroke}{rgb}{0.147607,0.511733,0.557049}%
\pgfsetstrokecolor{currentstroke}%
\pgfsetdash{}{0pt}%
\pgfpathmoveto{\pgfqpoint{2.707792in}{2.711862in}}%
\pgfpathquadraticcurveto{\pgfqpoint{2.695256in}{2.712070in}}{\pgfqpoint{2.699383in}{2.712002in}}%
\pgfusepath{stroke}%
\end{pgfscope}%
\begin{pgfscope}%
\pgfpathrectangle{\pgfqpoint{1.250000in}{1.750000in}}{\pgfqpoint{2.279412in}{2.004545in}}%
\pgfusepath{clip}%
\pgfsetroundcap%
\pgfsetroundjoin%
\definecolor{currentfill}{rgb}{0.147607,0.511733,0.557049}%
\pgfsetfillcolor{currentfill}%
\pgfsetlinewidth{1.077231pt}%
\definecolor{currentstroke}{rgb}{0.147607,0.511733,0.557049}%
\pgfsetstrokecolor{currentstroke}%
\pgfsetdash{}{0pt}%
\pgfpathmoveto{\pgfqpoint{2.754469in}{2.683304in}}%
\pgfpathlineto{\pgfqpoint{2.699383in}{2.712002in}}%
\pgfpathlineto{\pgfqpoint{2.755393in}{2.738852in}}%
\pgfpathlineto{\pgfqpoint{2.754469in}{2.683304in}}%
\pgfpathlineto{\pgfqpoint{2.754469in}{2.683304in}}%
\pgfpathclose%
\pgfusepath{stroke,fill}%
\end{pgfscope}%
\begin{pgfscope}%
\pgfpathrectangle{\pgfqpoint{1.250000in}{1.750000in}}{\pgfqpoint{2.279412in}{2.004545in}}%
\pgfusepath{clip}%
\pgfsetroundcap%
\pgfsetroundjoin%
\pgfsetlinewidth{0.440638pt}%
\definecolor{currentstroke}{rgb}{0.283197,0.115680,0.436115}%
\pgfsetstrokecolor{currentstroke}%
\pgfsetdash{}{0pt}%
\pgfpathmoveto{\pgfqpoint{2.908345in}{2.795939in}}%
\pgfpathquadraticcurveto{\pgfqpoint{2.895808in}{2.795844in}}{\pgfqpoint{2.890087in}{2.795801in}}%
\pgfusepath{stroke}%
\end{pgfscope}%
\begin{pgfscope}%
\pgfpathrectangle{\pgfqpoint{1.250000in}{1.750000in}}{\pgfqpoint{2.279412in}{2.004545in}}%
\pgfusepath{clip}%
\pgfsetroundcap%
\pgfsetroundjoin%
\definecolor{currentfill}{rgb}{0.283197,0.115680,0.436115}%
\pgfsetfillcolor{currentfill}%
\pgfsetlinewidth{0.440638pt}%
\definecolor{currentstroke}{rgb}{0.283197,0.115680,0.436115}%
\pgfsetstrokecolor{currentstroke}%
\pgfsetdash{}{0pt}%
\pgfpathmoveto{\pgfqpoint{2.945850in}{2.768442in}}%
\pgfpathlineto{\pgfqpoint{2.890087in}{2.795801in}}%
\pgfpathlineto{\pgfqpoint{2.945432in}{2.823996in}}%
\pgfpathlineto{\pgfqpoint{2.945850in}{2.768442in}}%
\pgfpathlineto{\pgfqpoint{2.945850in}{2.768442in}}%
\pgfpathclose%
\pgfusepath{stroke,fill}%
\end{pgfscope}%
\begin{pgfscope}%
\pgfpathrectangle{\pgfqpoint{1.250000in}{1.750000in}}{\pgfqpoint{2.279412in}{2.004545in}}%
\pgfusepath{clip}%
\pgfsetroundcap%
\pgfsetroundjoin%
\pgfsetlinewidth{0.594575pt}%
\definecolor{currentstroke}{rgb}{0.267968,0.223549,0.512008}%
\pgfsetstrokecolor{currentstroke}%
\pgfsetdash{}{0pt}%
\pgfpathmoveto{\pgfqpoint{2.758232in}{2.923727in}}%
\pgfpathquadraticcurveto{\pgfqpoint{2.745714in}{2.923109in}}{\pgfqpoint{2.742384in}{2.922944in}}%
\pgfusepath{stroke}%
\end{pgfscope}%
\begin{pgfscope}%
\pgfpathrectangle{\pgfqpoint{1.250000in}{1.750000in}}{\pgfqpoint{2.279412in}{2.004545in}}%
\pgfusepath{clip}%
\pgfsetroundcap%
\pgfsetroundjoin%
\definecolor{currentfill}{rgb}{0.267968,0.223549,0.512008}%
\pgfsetfillcolor{currentfill}%
\pgfsetlinewidth{0.594575pt}%
\definecolor{currentstroke}{rgb}{0.267968,0.223549,0.512008}%
\pgfsetstrokecolor{currentstroke}%
\pgfsetdash{}{0pt}%
\pgfpathmoveto{\pgfqpoint{2.799242in}{2.897942in}}%
\pgfpathlineto{\pgfqpoint{2.742384in}{2.922944in}}%
\pgfpathlineto{\pgfqpoint{2.796501in}{2.953430in}}%
\pgfpathlineto{\pgfqpoint{2.799242in}{2.897942in}}%
\pgfpathlineto{\pgfqpoint{2.799242in}{2.897942in}}%
\pgfpathclose%
\pgfusepath{stroke,fill}%
\end{pgfscope}%
\begin{pgfscope}%
\pgfpathrectangle{\pgfqpoint{1.250000in}{1.750000in}}{\pgfqpoint{2.279412in}{2.004545in}}%
\pgfusepath{clip}%
\pgfsetroundcap%
\pgfsetroundjoin%
\pgfsetlinewidth{0.403051pt}%
\definecolor{currentstroke}{rgb}{0.281446,0.084320,0.407414}%
\pgfsetstrokecolor{currentstroke}%
\pgfsetdash{}{0pt}%
\pgfpathmoveto{\pgfqpoint{2.908474in}{2.972112in}}%
\pgfpathquadraticcurveto{\pgfqpoint{2.895946in}{2.971678in}}{\pgfqpoint{2.889650in}{2.971460in}}%
\pgfusepath{stroke}%
\end{pgfscope}%
\begin{pgfscope}%
\pgfpathrectangle{\pgfqpoint{1.250000in}{1.750000in}}{\pgfqpoint{2.279412in}{2.004545in}}%
\pgfusepath{clip}%
\pgfsetroundcap%
\pgfsetroundjoin%
\definecolor{currentfill}{rgb}{0.281446,0.084320,0.407414}%
\pgfsetfillcolor{currentfill}%
\pgfsetlinewidth{0.403051pt}%
\definecolor{currentstroke}{rgb}{0.281446,0.084320,0.407414}%
\pgfsetstrokecolor{currentstroke}%
\pgfsetdash{}{0pt}%
\pgfpathmoveto{\pgfqpoint{2.946133in}{2.945621in}}%
\pgfpathlineto{\pgfqpoint{2.889650in}{2.971460in}}%
\pgfpathlineto{\pgfqpoint{2.944211in}{3.001143in}}%
\pgfpathlineto{\pgfqpoint{2.946133in}{2.945621in}}%
\pgfpathlineto{\pgfqpoint{2.946133in}{2.945621in}}%
\pgfpathclose%
\pgfusepath{stroke,fill}%
\end{pgfscope}%
\begin{pgfscope}%
\pgfpathrectangle{\pgfqpoint{1.250000in}{1.750000in}}{\pgfqpoint{2.279412in}{2.004545in}}%
\pgfusepath{clip}%
\pgfsetroundcap%
\pgfsetroundjoin%
\pgfsetlinewidth{0.326631pt}%
\definecolor{currentstroke}{rgb}{0.271305,0.019942,0.347269}%
\pgfsetstrokecolor{currentstroke}%
\pgfsetdash{}{0pt}%
\pgfpathmoveto{\pgfqpoint{3.007662in}{2.304856in}}%
\pgfpathquadraticcurveto{\pgfqpoint{2.995135in}{2.305258in}}{\pgfqpoint{2.987659in}{2.305498in}}%
\pgfusepath{stroke}%
\end{pgfscope}%
\begin{pgfscope}%
\pgfpathrectangle{\pgfqpoint{1.250000in}{1.750000in}}{\pgfqpoint{2.279412in}{2.004545in}}%
\pgfusepath{clip}%
\pgfsetroundcap%
\pgfsetroundjoin%
\definecolor{currentfill}{rgb}{0.271305,0.019942,0.347269}%
\pgfsetfillcolor{currentfill}%
\pgfsetlinewidth{0.326631pt}%
\definecolor{currentstroke}{rgb}{0.271305,0.019942,0.347269}%
\pgfsetstrokecolor{currentstroke}%
\pgfsetdash{}{0pt}%
\pgfpathmoveto{\pgfqpoint{3.042295in}{2.275953in}}%
\pgfpathlineto{\pgfqpoint{2.987659in}{2.305498in}}%
\pgfpathlineto{\pgfqpoint{3.044077in}{2.331480in}}%
\pgfpathlineto{\pgfqpoint{3.042295in}{2.275953in}}%
\pgfpathlineto{\pgfqpoint{3.042295in}{2.275953in}}%
\pgfpathclose%
\pgfusepath{stroke,fill}%
\end{pgfscope}%
\begin{pgfscope}%
\pgfpathrectangle{\pgfqpoint{1.250000in}{1.750000in}}{\pgfqpoint{2.279412in}{2.004545in}}%
\pgfusepath{clip}%
\pgfsetroundcap%
\pgfsetroundjoin%
\pgfsetlinewidth{0.350454pt}%
\definecolor{currentstroke}{rgb}{0.276022,0.044167,0.370164}%
\pgfsetstrokecolor{currentstroke}%
\pgfsetdash{}{0pt}%
\pgfpathmoveto{\pgfqpoint{2.957524in}{2.396522in}}%
\pgfpathquadraticcurveto{\pgfqpoint{2.944993in}{2.396710in}}{\pgfqpoint{2.937883in}{2.396818in}}%
\pgfusepath{stroke}%
\end{pgfscope}%
\begin{pgfscope}%
\pgfpathrectangle{\pgfqpoint{1.250000in}{1.750000in}}{\pgfqpoint{2.279412in}{2.004545in}}%
\pgfusepath{clip}%
\pgfsetroundcap%
\pgfsetroundjoin%
\definecolor{currentfill}{rgb}{0.276022,0.044167,0.370164}%
\pgfsetfillcolor{currentfill}%
\pgfsetlinewidth{0.350454pt}%
\definecolor{currentstroke}{rgb}{0.276022,0.044167,0.370164}%
\pgfsetstrokecolor{currentstroke}%
\pgfsetdash{}{0pt}%
\pgfpathmoveto{\pgfqpoint{2.993014in}{2.368206in}}%
\pgfpathlineto{\pgfqpoint{2.937883in}{2.396818in}}%
\pgfpathlineto{\pgfqpoint{2.993851in}{2.423755in}}%
\pgfpathlineto{\pgfqpoint{2.993014in}{2.368206in}}%
\pgfpathlineto{\pgfqpoint{2.993014in}{2.368206in}}%
\pgfpathclose%
\pgfusepath{stroke,fill}%
\end{pgfscope}%
\begin{pgfscope}%
\pgfpathrectangle{\pgfqpoint{1.250000in}{1.750000in}}{\pgfqpoint{2.279412in}{2.004545in}}%
\pgfusepath{clip}%
\pgfsetroundcap%
\pgfsetroundjoin%
\pgfsetlinewidth{1.421565pt}%
\definecolor{currentstroke}{rgb}{0.170948,0.694384,0.493803}%
\pgfsetstrokecolor{currentstroke}%
\pgfsetdash{}{0pt}%
\pgfpathmoveto{\pgfqpoint{2.656456in}{2.751222in}}%
\pgfpathquadraticcurveto{\pgfqpoint{2.643918in}{2.751185in}}{\pgfqpoint{2.653372in}{2.751213in}}%
\pgfusepath{stroke}%
\end{pgfscope}%
\begin{pgfscope}%
\pgfpathrectangle{\pgfqpoint{1.250000in}{1.750000in}}{\pgfqpoint{2.279412in}{2.004545in}}%
\pgfusepath{clip}%
\pgfsetroundcap%
\pgfsetroundjoin%
\definecolor{currentfill}{rgb}{0.170948,0.694384,0.493803}%
\pgfsetfillcolor{currentfill}%
\pgfsetlinewidth{1.421565pt}%
\definecolor{currentstroke}{rgb}{0.170948,0.694384,0.493803}%
\pgfsetstrokecolor{currentstroke}%
\pgfsetdash{}{0pt}%
\pgfpathmoveto{\pgfqpoint{2.709008in}{2.723597in}}%
\pgfpathlineto{\pgfqpoint{2.653372in}{2.751213in}}%
\pgfpathlineto{\pgfqpoint{2.708847in}{2.779152in}}%
\pgfpathlineto{\pgfqpoint{2.709008in}{2.723597in}}%
\pgfpathlineto{\pgfqpoint{2.709008in}{2.723597in}}%
\pgfpathclose%
\pgfusepath{stroke,fill}%
\end{pgfscope}%
\begin{pgfscope}%
\pgfpathrectangle{\pgfqpoint{1.250000in}{1.750000in}}{\pgfqpoint{2.279412in}{2.004545in}}%
\pgfusepath{clip}%
\pgfsetroundcap%
\pgfsetroundjoin%
\pgfsetlinewidth{1.065430pt}%
\definecolor{currentstroke}{rgb}{0.150476,0.504369,0.557430}%
\pgfsetstrokecolor{currentstroke}%
\pgfsetdash{}{0pt}%
\pgfpathmoveto{\pgfqpoint{2.460439in}{2.970049in}}%
\pgfpathquadraticcurveto{\pgfqpoint{2.448403in}{2.966967in}}{\pgfqpoint{2.452334in}{2.967974in}}%
\pgfusepath{stroke}%
\end{pgfscope}%
\begin{pgfscope}%
\pgfpathrectangle{\pgfqpoint{1.250000in}{1.750000in}}{\pgfqpoint{2.279412in}{2.004545in}}%
\pgfusepath{clip}%
\pgfsetroundcap%
\pgfsetroundjoin%
\definecolor{currentfill}{rgb}{0.150476,0.504369,0.557430}%
\pgfsetfillcolor{currentfill}%
\pgfsetlinewidth{1.065430pt}%
\definecolor{currentstroke}{rgb}{0.150476,0.504369,0.557430}%
\pgfsetstrokecolor{currentstroke}%
\pgfsetdash{}{0pt}%
\pgfpathmoveto{\pgfqpoint{2.513043in}{2.954843in}}%
\pgfpathlineto{\pgfqpoint{2.452334in}{2.967974in}}%
\pgfpathlineto{\pgfqpoint{2.499264in}{3.008663in}}%
\pgfpathlineto{\pgfqpoint{2.513043in}{2.954843in}}%
\pgfpathlineto{\pgfqpoint{2.513043in}{2.954843in}}%
\pgfpathclose%
\pgfusepath{stroke,fill}%
\end{pgfscope}%
\begin{pgfscope}%
\pgfpathrectangle{\pgfqpoint{1.250000in}{1.750000in}}{\pgfqpoint{2.279412in}{2.004545in}}%
\pgfusepath{clip}%
\pgfsetroundcap%
\pgfsetroundjoin%
\pgfsetlinewidth{0.385099pt}%
\definecolor{currentstroke}{rgb}{0.280267,0.073417,0.397163}%
\pgfsetstrokecolor{currentstroke}%
\pgfsetdash{}{0pt}%
\pgfpathmoveto{\pgfqpoint{2.907706in}{3.059675in}}%
\pgfpathquadraticcurveto{\pgfqpoint{2.895193in}{3.058984in}}{\pgfqpoint{2.888629in}{3.058621in}}%
\pgfusepath{stroke}%
\end{pgfscope}%
\begin{pgfscope}%
\pgfpathrectangle{\pgfqpoint{1.250000in}{1.750000in}}{\pgfqpoint{2.279412in}{2.004545in}}%
\pgfusepath{clip}%
\pgfsetroundcap%
\pgfsetroundjoin%
\definecolor{currentfill}{rgb}{0.280267,0.073417,0.397163}%
\pgfsetfillcolor{currentfill}%
\pgfsetlinewidth{0.385099pt}%
\definecolor{currentstroke}{rgb}{0.280267,0.073417,0.397163}%
\pgfsetstrokecolor{currentstroke}%
\pgfsetdash{}{0pt}%
\pgfpathmoveto{\pgfqpoint{2.945632in}{3.033950in}}%
\pgfpathlineto{\pgfqpoint{2.888629in}{3.058621in}}%
\pgfpathlineto{\pgfqpoint{2.942568in}{3.089421in}}%
\pgfpathlineto{\pgfqpoint{2.945632in}{3.033950in}}%
\pgfpathlineto{\pgfqpoint{2.945632in}{3.033950in}}%
\pgfpathclose%
\pgfusepath{stroke,fill}%
\end{pgfscope}%
\begin{pgfscope}%
\pgfpathrectangle{\pgfqpoint{1.250000in}{1.750000in}}{\pgfqpoint{2.279412in}{2.004545in}}%
\pgfusepath{clip}%
\pgfsetroundcap%
\pgfsetroundjoin%
\pgfsetlinewidth{0.352555pt}%
\definecolor{currentstroke}{rgb}{0.276022,0.044167,0.370164}%
\pgfsetstrokecolor{currentstroke}%
\pgfsetdash{}{0pt}%
\pgfpathmoveto{\pgfqpoint{2.957493in}{3.108668in}}%
\pgfpathquadraticcurveto{\pgfqpoint{2.944983in}{3.108125in}}{\pgfqpoint{2.937921in}{3.107818in}}%
\pgfusepath{stroke}%
\end{pgfscope}%
\begin{pgfscope}%
\pgfpathrectangle{\pgfqpoint{1.250000in}{1.750000in}}{\pgfqpoint{2.279412in}{2.004545in}}%
\pgfusepath{clip}%
\pgfsetroundcap%
\pgfsetroundjoin%
\definecolor{currentfill}{rgb}{0.276022,0.044167,0.370164}%
\pgfsetfillcolor{currentfill}%
\pgfsetlinewidth{0.352555pt}%
\definecolor{currentstroke}{rgb}{0.276022,0.044167,0.370164}%
\pgfsetstrokecolor{currentstroke}%
\pgfsetdash{}{0pt}%
\pgfpathmoveto{\pgfqpoint{2.994630in}{3.082477in}}%
\pgfpathlineto{\pgfqpoint{2.937921in}{3.107818in}}%
\pgfpathlineto{\pgfqpoint{2.992220in}{3.137980in}}%
\pgfpathlineto{\pgfqpoint{2.994630in}{3.082477in}}%
\pgfpathlineto{\pgfqpoint{2.994630in}{3.082477in}}%
\pgfpathclose%
\pgfusepath{stroke,fill}%
\end{pgfscope}%
\begin{pgfscope}%
\pgfpathrectangle{\pgfqpoint{1.250000in}{1.750000in}}{\pgfqpoint{2.279412in}{2.004545in}}%
\pgfusepath{clip}%
\pgfsetroundcap%
\pgfsetroundjoin%
\pgfsetlinewidth{0.432016pt}%
\definecolor{currentstroke}{rgb}{0.283091,0.110553,0.431554}%
\pgfsetstrokecolor{currentstroke}%
\pgfsetdash{}{0pt}%
\pgfpathmoveto{\pgfqpoint{2.512089in}{3.099967in}}%
\pgfpathquadraticcurveto{\pgfqpoint{2.500372in}{3.096086in}}{\pgfqpoint{2.494999in}{3.094306in}}%
\pgfusepath{stroke}%
\end{pgfscope}%
\begin{pgfscope}%
\pgfpathrectangle{\pgfqpoint{1.250000in}{1.750000in}}{\pgfqpoint{2.279412in}{2.004545in}}%
\pgfusepath{clip}%
\pgfsetroundcap%
\pgfsetroundjoin%
\definecolor{currentfill}{rgb}{0.283091,0.110553,0.431554}%
\pgfsetfillcolor{currentfill}%
\pgfsetlinewidth{0.432016pt}%
\definecolor{currentstroke}{rgb}{0.283091,0.110553,0.431554}%
\pgfsetstrokecolor{currentstroke}%
\pgfsetdash{}{0pt}%
\pgfpathmoveto{\pgfqpoint{2.556471in}{3.085405in}}%
\pgfpathlineto{\pgfqpoint{2.494999in}{3.094306in}}%
\pgfpathlineto{\pgfqpoint{2.539003in}{3.138143in}}%
\pgfpathlineto{\pgfqpoint{2.556471in}{3.085405in}}%
\pgfpathlineto{\pgfqpoint{2.556471in}{3.085405in}}%
\pgfpathclose%
\pgfusepath{stroke,fill}%
\end{pgfscope}%
\begin{pgfscope}%
\pgfpathrectangle{\pgfqpoint{1.250000in}{1.750000in}}{\pgfqpoint{2.279412in}{2.004545in}}%
\pgfusepath{clip}%
\pgfsetroundcap%
\pgfsetroundjoin%
\pgfsetlinewidth{0.349828pt}%
\definecolor{currentstroke}{rgb}{0.276022,0.044167,0.370164}%
\pgfsetstrokecolor{currentstroke}%
\pgfsetdash{}{0pt}%
\pgfpathmoveto{\pgfqpoint{2.856667in}{2.266965in}}%
\pgfpathquadraticcurveto{\pgfqpoint{2.844214in}{2.268206in}}{\pgfqpoint{2.837147in}{2.268910in}}%
\pgfusepath{stroke}%
\end{pgfscope}%
\begin{pgfscope}%
\pgfpathrectangle{\pgfqpoint{1.250000in}{1.750000in}}{\pgfqpoint{2.279412in}{2.004545in}}%
\pgfusepath{clip}%
\pgfsetroundcap%
\pgfsetroundjoin%
\definecolor{currentfill}{rgb}{0.276022,0.044167,0.370164}%
\pgfsetfillcolor{currentfill}%
\pgfsetlinewidth{0.349828pt}%
\definecolor{currentstroke}{rgb}{0.276022,0.044167,0.370164}%
\pgfsetstrokecolor{currentstroke}%
\pgfsetdash{}{0pt}%
\pgfpathmoveto{\pgfqpoint{2.889673in}{2.235759in}}%
\pgfpathlineto{\pgfqpoint{2.837147in}{2.268910in}}%
\pgfpathlineto{\pgfqpoint{2.895184in}{2.291041in}}%
\pgfpathlineto{\pgfqpoint{2.889673in}{2.235759in}}%
\pgfpathlineto{\pgfqpoint{2.889673in}{2.235759in}}%
\pgfpathclose%
\pgfusepath{stroke,fill}%
\end{pgfscope}%
\begin{pgfscope}%
\pgfpathrectangle{\pgfqpoint{1.250000in}{1.750000in}}{\pgfqpoint{2.279412in}{2.004545in}}%
\pgfusepath{clip}%
\pgfsetroundcap%
\pgfsetroundjoin%
\pgfsetlinewidth{0.344738pt}%
\definecolor{currentstroke}{rgb}{0.274952,0.037752,0.364543}%
\pgfsetstrokecolor{currentstroke}%
\pgfsetdash{}{0pt}%
\pgfpathmoveto{\pgfqpoint{2.906342in}{3.196825in}}%
\pgfpathquadraticcurveto{\pgfqpoint{2.893841in}{3.196027in}}{\pgfqpoint{2.886662in}{3.195568in}}%
\pgfusepath{stroke}%
\end{pgfscope}%
\begin{pgfscope}%
\pgfpathrectangle{\pgfqpoint{1.250000in}{1.750000in}}{\pgfqpoint{2.279412in}{2.004545in}}%
\pgfusepath{clip}%
\pgfsetroundcap%
\pgfsetroundjoin%
\definecolor{currentfill}{rgb}{0.274952,0.037752,0.364543}%
\pgfsetfillcolor{currentfill}%
\pgfsetlinewidth{0.344738pt}%
\definecolor{currentstroke}{rgb}{0.274952,0.037752,0.364543}%
\pgfsetstrokecolor{currentstroke}%
\pgfsetdash{}{0pt}%
\pgfpathmoveto{\pgfqpoint{2.943875in}{3.171387in}}%
\pgfpathlineto{\pgfqpoint{2.886662in}{3.195568in}}%
\pgfpathlineto{\pgfqpoint{2.940335in}{3.226830in}}%
\pgfpathlineto{\pgfqpoint{2.943875in}{3.171387in}}%
\pgfpathlineto{\pgfqpoint{2.943875in}{3.171387in}}%
\pgfpathclose%
\pgfusepath{stroke,fill}%
\end{pgfscope}%
\begin{pgfscope}%
\pgfpathrectangle{\pgfqpoint{1.250000in}{1.750000in}}{\pgfqpoint{2.279412in}{2.004545in}}%
\pgfusepath{clip}%
\pgfsetroundcap%
\pgfsetroundjoin%
\pgfsetlinewidth{0.340760pt}%
\definecolor{currentstroke}{rgb}{0.273809,0.031497,0.358853}%
\pgfsetstrokecolor{currentstroke}%
\pgfsetdash{}{0pt}%
\pgfpathmoveto{\pgfqpoint{2.956917in}{3.240122in}}%
\pgfpathquadraticcurveto{\pgfqpoint{2.944432in}{3.239215in}}{\pgfqpoint{2.937205in}{3.238689in}}%
\pgfusepath{stroke}%
\end{pgfscope}%
\begin{pgfscope}%
\pgfpathrectangle{\pgfqpoint{1.250000in}{1.750000in}}{\pgfqpoint{2.279412in}{2.004545in}}%
\pgfusepath{clip}%
\pgfsetroundcap%
\pgfsetroundjoin%
\definecolor{currentfill}{rgb}{0.273809,0.031497,0.358853}%
\pgfsetfillcolor{currentfill}%
\pgfsetlinewidth{0.340760pt}%
\definecolor{currentstroke}{rgb}{0.273809,0.031497,0.358853}%
\pgfsetstrokecolor{currentstroke}%
\pgfsetdash{}{0pt}%
\pgfpathmoveto{\pgfqpoint{2.994629in}{3.215012in}}%
\pgfpathlineto{\pgfqpoint{2.937205in}{3.238689in}}%
\pgfpathlineto{\pgfqpoint{2.990601in}{3.270422in}}%
\pgfpathlineto{\pgfqpoint{2.994629in}{3.215012in}}%
\pgfpathlineto{\pgfqpoint{2.994629in}{3.215012in}}%
\pgfpathclose%
\pgfusepath{stroke,fill}%
\end{pgfscope}%
\begin{pgfscope}%
\pgfpathrectangle{\pgfqpoint{1.250000in}{1.750000in}}{\pgfqpoint{2.279412in}{2.004545in}}%
\pgfusepath{clip}%
\pgfsetroundcap%
\pgfsetroundjoin%
\pgfsetlinewidth{0.326744pt}%
\definecolor{currentstroke}{rgb}{0.271305,0.019942,0.347269}%
\pgfsetstrokecolor{currentstroke}%
\pgfsetdash{}{0pt}%
\pgfpathmoveto{\pgfqpoint{2.907659in}{3.281367in}}%
\pgfpathquadraticcurveto{\pgfqpoint{2.895198in}{3.280174in}}{\pgfqpoint{2.887769in}{3.279463in}}%
\pgfusepath{stroke}%
\end{pgfscope}%
\begin{pgfscope}%
\pgfpathrectangle{\pgfqpoint{1.250000in}{1.750000in}}{\pgfqpoint{2.279412in}{2.004545in}}%
\pgfusepath{clip}%
\pgfsetroundcap%
\pgfsetroundjoin%
\definecolor{currentfill}{rgb}{0.271305,0.019942,0.347269}%
\pgfsetfillcolor{currentfill}%
\pgfsetlinewidth{0.326744pt}%
\definecolor{currentstroke}{rgb}{0.271305,0.019942,0.347269}%
\pgfsetstrokecolor{currentstroke}%
\pgfsetdash{}{0pt}%
\pgfpathmoveto{\pgfqpoint{2.945719in}{3.257106in}}%
\pgfpathlineto{\pgfqpoint{2.887769in}{3.279463in}}%
\pgfpathlineto{\pgfqpoint{2.940424in}{3.312409in}}%
\pgfpathlineto{\pgfqpoint{2.945719in}{3.257106in}}%
\pgfpathlineto{\pgfqpoint{2.945719in}{3.257106in}}%
\pgfpathclose%
\pgfusepath{stroke,fill}%
\end{pgfscope}%
\begin{pgfscope}%
\pgfpathrectangle{\pgfqpoint{1.250000in}{1.750000in}}{\pgfqpoint{2.279412in}{2.004545in}}%
\pgfusepath{clip}%
\pgfsetroundcap%
\pgfsetroundjoin%
\pgfsetlinewidth{0.356941pt}%
\definecolor{currentstroke}{rgb}{0.277018,0.050344,0.375715}%
\pgfsetstrokecolor{currentstroke}%
\pgfsetdash{}{0pt}%
\pgfpathmoveto{\pgfqpoint{2.796203in}{2.195951in}}%
\pgfpathquadraticcurveto{\pgfqpoint{2.783867in}{2.197868in}}{\pgfqpoint{2.776987in}{2.198937in}}%
\pgfusepath{stroke}%
\end{pgfscope}%
\begin{pgfscope}%
\pgfpathrectangle{\pgfqpoint{1.250000in}{1.750000in}}{\pgfqpoint{2.279412in}{2.004545in}}%
\pgfusepath{clip}%
\pgfsetroundcap%
\pgfsetroundjoin%
\definecolor{currentfill}{rgb}{0.277018,0.050344,0.375715}%
\pgfsetfillcolor{currentfill}%
\pgfsetlinewidth{0.356941pt}%
\definecolor{currentstroke}{rgb}{0.277018,0.050344,0.375715}%
\pgfsetstrokecolor{currentstroke}%
\pgfsetdash{}{0pt}%
\pgfpathmoveto{\pgfqpoint{2.827618in}{2.162957in}}%
\pgfpathlineto{\pgfqpoint{2.776987in}{2.198937in}}%
\pgfpathlineto{\pgfqpoint{2.836150in}{2.217854in}}%
\pgfpathlineto{\pgfqpoint{2.827618in}{2.162957in}}%
\pgfpathlineto{\pgfqpoint{2.827618in}{2.162957in}}%
\pgfpathclose%
\pgfusepath{stroke,fill}%
\end{pgfscope}%
\begin{pgfscope}%
\pgfpathrectangle{\pgfqpoint{1.250000in}{1.750000in}}{\pgfqpoint{2.279412in}{2.004545in}}%
\pgfusepath{clip}%
\pgfsetroundcap%
\pgfsetroundjoin%
\pgfsetlinewidth{0.363590pt}%
\definecolor{currentstroke}{rgb}{0.277941,0.056324,0.381191}%
\pgfsetstrokecolor{currentstroke}%
\pgfsetdash{}{0pt}%
\pgfpathmoveto{\pgfqpoint{2.594873in}{2.256098in}}%
\pgfpathquadraticcurveto{\pgfqpoint{2.582751in}{2.258879in}}{\pgfqpoint{2.576111in}{2.260403in}}%
\pgfusepath{stroke}%
\end{pgfscope}%
\begin{pgfscope}%
\pgfpathrectangle{\pgfqpoint{1.250000in}{1.750000in}}{\pgfqpoint{2.279412in}{2.004545in}}%
\pgfusepath{clip}%
\pgfsetroundcap%
\pgfsetroundjoin%
\definecolor{currentfill}{rgb}{0.277941,0.056324,0.381191}%
\pgfsetfillcolor{currentfill}%
\pgfsetlinewidth{0.363590pt}%
\definecolor{currentstroke}{rgb}{0.277941,0.056324,0.381191}%
\pgfsetstrokecolor{currentstroke}%
\pgfsetdash{}{0pt}%
\pgfpathmoveto{\pgfqpoint{2.624049in}{2.220905in}}%
\pgfpathlineto{\pgfqpoint{2.576111in}{2.260403in}}%
\pgfpathlineto{\pgfqpoint{2.636472in}{2.275054in}}%
\pgfpathlineto{\pgfqpoint{2.624049in}{2.220905in}}%
\pgfpathlineto{\pgfqpoint{2.624049in}{2.220905in}}%
\pgfpathclose%
\pgfusepath{stroke,fill}%
\end{pgfscope}%
\begin{pgfscope}%
\pgfpathrectangle{\pgfqpoint{1.250000in}{1.750000in}}{\pgfqpoint{2.279412in}{2.004545in}}%
\pgfusepath{clip}%
\pgfsetroundcap%
\pgfsetroundjoin%
\pgfsetlinewidth{0.370805pt}%
\definecolor{currentstroke}{rgb}{0.278791,0.062145,0.386592}%
\pgfsetstrokecolor{currentstroke}%
\pgfsetdash{}{0pt}%
\pgfpathmoveto{\pgfqpoint{2.591978in}{3.214766in}}%
\pgfpathquadraticcurveto{\pgfqpoint{2.579879in}{3.211910in}}{\pgfqpoint{2.573363in}{3.210371in}}%
\pgfusepath{stroke}%
\end{pgfscope}%
\begin{pgfscope}%
\pgfpathrectangle{\pgfqpoint{1.250000in}{1.750000in}}{\pgfqpoint{2.279412in}{2.004545in}}%
\pgfusepath{clip}%
\pgfsetroundcap%
\pgfsetroundjoin%
\definecolor{currentfill}{rgb}{0.278791,0.062145,0.386592}%
\pgfsetfillcolor{currentfill}%
\pgfsetlinewidth{0.370805pt}%
\definecolor{currentstroke}{rgb}{0.278791,0.062145,0.386592}%
\pgfsetstrokecolor{currentstroke}%
\pgfsetdash{}{0pt}%
\pgfpathmoveto{\pgfqpoint{2.633814in}{3.196102in}}%
\pgfpathlineto{\pgfqpoint{2.573363in}{3.210371in}}%
\pgfpathlineto{\pgfqpoint{2.621049in}{3.250171in}}%
\pgfpathlineto{\pgfqpoint{2.633814in}{3.196102in}}%
\pgfpathlineto{\pgfqpoint{2.633814in}{3.196102in}}%
\pgfpathclose%
\pgfusepath{stroke,fill}%
\end{pgfscope}%
\begin{pgfscope}%
\pgfpathrectangle{\pgfqpoint{1.250000in}{1.750000in}}{\pgfqpoint{2.279412in}{2.004545in}}%
\pgfusepath{clip}%
\pgfsetroundcap%
\pgfsetroundjoin%
\pgfsetlinewidth{0.698171pt}%
\definecolor{currentstroke}{rgb}{0.243113,0.292092,0.538516}%
\pgfsetstrokecolor{currentstroke}%
\pgfsetdash{}{0pt}%
\pgfpathmoveto{\pgfqpoint{2.221269in}{2.501016in}}%
\pgfpathquadraticcurveto{\pgfqpoint{2.213217in}{2.508356in}}{\pgfqpoint{2.213147in}{2.508420in}}%
\pgfusepath{stroke}%
\end{pgfscope}%
\begin{pgfscope}%
\pgfpathrectangle{\pgfqpoint{1.250000in}{1.750000in}}{\pgfqpoint{2.279412in}{2.004545in}}%
\pgfusepath{clip}%
\pgfsetroundcap%
\pgfsetroundjoin%
\definecolor{currentfill}{rgb}{0.243113,0.292092,0.538516}%
\pgfsetfillcolor{currentfill}%
\pgfsetlinewidth{0.698171pt}%
\definecolor{currentstroke}{rgb}{0.243113,0.292092,0.538516}%
\pgfsetstrokecolor{currentstroke}%
\pgfsetdash{}{0pt}%
\pgfpathmoveto{\pgfqpoint{2.235492in}{2.450465in}}%
\pgfpathlineto{\pgfqpoint{2.213147in}{2.508420in}}%
\pgfpathlineto{\pgfqpoint{2.272917in}{2.491523in}}%
\pgfpathlineto{\pgfqpoint{2.235492in}{2.450465in}}%
\pgfpathlineto{\pgfqpoint{2.235492in}{2.450465in}}%
\pgfpathclose%
\pgfusepath{stroke,fill}%
\end{pgfscope}%
\begin{pgfscope}%
\pgfpathrectangle{\pgfqpoint{1.250000in}{1.750000in}}{\pgfqpoint{2.279412in}{2.004545in}}%
\pgfusepath{clip}%
\pgfsetroundcap%
\pgfsetroundjoin%
\pgfsetlinewidth{0.502956pt}%
\definecolor{currentstroke}{rgb}{0.280868,0.160771,0.472899}%
\pgfsetstrokecolor{currentstroke}%
\pgfsetdash{}{0pt}%
\pgfpathmoveto{\pgfqpoint{2.592182in}{3.078845in}}%
\pgfpathquadraticcurveto{\pgfqpoint{2.580032in}{3.076139in}}{\pgfqpoint{2.575477in}{3.075124in}}%
\pgfusepath{stroke}%
\end{pgfscope}%
\begin{pgfscope}%
\pgfpathrectangle{\pgfqpoint{1.250000in}{1.750000in}}{\pgfqpoint{2.279412in}{2.004545in}}%
\pgfusepath{clip}%
\pgfsetroundcap%
\pgfsetroundjoin%
\definecolor{currentfill}{rgb}{0.280868,0.160771,0.472899}%
\pgfsetfillcolor{currentfill}%
\pgfsetlinewidth{0.502956pt}%
\definecolor{currentstroke}{rgb}{0.280868,0.160771,0.472899}%
\pgfsetstrokecolor{currentstroke}%
\pgfsetdash{}{0pt}%
\pgfpathmoveto{\pgfqpoint{2.635742in}{3.060089in}}%
\pgfpathlineto{\pgfqpoint{2.575477in}{3.075124in}}%
\pgfpathlineto{\pgfqpoint{2.623664in}{3.114316in}}%
\pgfpathlineto{\pgfqpoint{2.635742in}{3.060089in}}%
\pgfpathlineto{\pgfqpoint{2.635742in}{3.060089in}}%
\pgfpathclose%
\pgfusepath{stroke,fill}%
\end{pgfscope}%
\begin{pgfscope}%
\pgfpathrectangle{\pgfqpoint{1.250000in}{1.750000in}}{\pgfqpoint{2.279412in}{2.004545in}}%
\pgfusepath{clip}%
\pgfsetroundcap%
\pgfsetroundjoin%
\pgfsetlinewidth{1.381056pt}%
\definecolor{currentstroke}{rgb}{0.146616,0.673050,0.508936}%
\pgfsetstrokecolor{currentstroke}%
\pgfsetdash{}{0pt}%
\pgfpathmoveto{\pgfqpoint{2.118483in}{2.955174in}}%
\pgfpathquadraticcurveto{\pgfqpoint{2.114843in}{2.950088in}}{\pgfqpoint{2.123638in}{2.962376in}}%
\pgfusepath{stroke}%
\end{pgfscope}%
\begin{pgfscope}%
\pgfpathrectangle{\pgfqpoint{1.250000in}{1.750000in}}{\pgfqpoint{2.279412in}{2.004545in}}%
\pgfusepath{clip}%
\pgfsetroundcap%
\pgfsetroundjoin%
\definecolor{currentfill}{rgb}{0.146616,0.673050,0.508936}%
\pgfsetfillcolor{currentfill}%
\pgfsetlinewidth{1.381056pt}%
\definecolor{currentstroke}{rgb}{0.146616,0.673050,0.508936}%
\pgfsetstrokecolor{currentstroke}%
\pgfsetdash{}{0pt}%
\pgfpathmoveto{\pgfqpoint{2.178559in}{2.991388in}}%
\pgfpathlineto{\pgfqpoint{2.123638in}{2.962376in}}%
\pgfpathlineto{\pgfqpoint{2.133381in}{3.023720in}}%
\pgfpathlineto{\pgfqpoint{2.178559in}{2.991388in}}%
\pgfpathlineto{\pgfqpoint{2.178559in}{2.991388in}}%
\pgfpathclose%
\pgfusepath{stroke,fill}%
\end{pgfscope}%
\begin{pgfscope}%
\pgfpathrectangle{\pgfqpoint{1.250000in}{1.750000in}}{\pgfqpoint{2.279412in}{2.004545in}}%
\pgfusepath{clip}%
\pgfsetroundcap%
\pgfsetroundjoin%
\pgfsetlinewidth{1.298395pt}%
\definecolor{currentstroke}{rgb}{0.121380,0.629492,0.531973}%
\pgfsetstrokecolor{currentstroke}%
\pgfsetdash{}{0pt}%
\pgfpathmoveto{\pgfqpoint{2.444979in}{2.587928in}}%
\pgfpathquadraticcurveto{\pgfqpoint{2.432737in}{2.590301in}}{\pgfqpoint{2.440214in}{2.588852in}}%
\pgfusepath{stroke}%
\end{pgfscope}%
\begin{pgfscope}%
\pgfpathrectangle{\pgfqpoint{1.250000in}{1.750000in}}{\pgfqpoint{2.279412in}{2.004545in}}%
\pgfusepath{clip}%
\pgfsetroundcap%
\pgfsetroundjoin%
\definecolor{currentfill}{rgb}{0.121380,0.629492,0.531973}%
\pgfsetfillcolor{currentfill}%
\pgfsetlinewidth{1.298395pt}%
\definecolor{currentstroke}{rgb}{0.121380,0.629492,0.531973}%
\pgfsetstrokecolor{currentstroke}%
\pgfsetdash{}{0pt}%
\pgfpathmoveto{\pgfqpoint{2.489470in}{2.551011in}}%
\pgfpathlineto{\pgfqpoint{2.440214in}{2.588852in}}%
\pgfpathlineto{\pgfqpoint{2.500040in}{2.605552in}}%
\pgfpathlineto{\pgfqpoint{2.489470in}{2.551011in}}%
\pgfpathlineto{\pgfqpoint{2.489470in}{2.551011in}}%
\pgfpathclose%
\pgfusepath{stroke,fill}%
\end{pgfscope}%
\begin{pgfscope}%
\pgfpathrectangle{\pgfqpoint{1.250000in}{1.750000in}}{\pgfqpoint{2.279412in}{2.004545in}}%
\pgfusepath{clip}%
\pgfsetbuttcap%
\pgfsetroundjoin%
\pgfsetlinewidth{1.505625pt}%
\definecolor{currentstroke}{rgb}{0.000000,0.000000,0.000000}%
\pgfsetstrokecolor{currentstroke}%
\pgfsetdash{}{0pt}%
\pgfpathmoveto{\pgfqpoint{2.043384in}{2.089441in}}%
\pgfpathlineto{\pgfqpoint{2.043384in}{3.415105in}}%
\pgfusepath{stroke}%
\end{pgfscope}%
\begin{pgfscope}%
\pgfpathrectangle{\pgfqpoint{1.250000in}{1.750000in}}{\pgfqpoint{2.279412in}{2.004545in}}%
\pgfusepath{clip}%
\pgfsetbuttcap%
\pgfsetroundjoin%
\pgfsetlinewidth{1.505625pt}%
\definecolor{currentstroke}{rgb}{0.000000,0.000000,0.000000}%
\pgfsetstrokecolor{currentstroke}%
\pgfsetdash{}{0pt}%
\pgfpathmoveto{\pgfqpoint{3.191910in}{2.089441in}}%
\pgfpathlineto{\pgfqpoint{3.191910in}{3.415105in}}%
\pgfusepath{stroke}%
\end{pgfscope}%
\begin{pgfscope}%
\pgfsetrectcap%
\pgfsetmiterjoin%
\pgfsetlinewidth{0.803000pt}%
\definecolor{currentstroke}{rgb}{0.000000,0.000000,0.000000}%
\pgfsetstrokecolor{currentstroke}%
\pgfsetdash{}{0pt}%
\pgfpathmoveto{\pgfqpoint{1.250000in}{1.750000in}}%
\pgfpathlineto{\pgfqpoint{1.250000in}{3.754545in}}%
\pgfusepath{stroke}%
\end{pgfscope}%
\begin{pgfscope}%
\pgfsetrectcap%
\pgfsetmiterjoin%
\pgfsetlinewidth{0.803000pt}%
\definecolor{currentstroke}{rgb}{0.000000,0.000000,0.000000}%
\pgfsetstrokecolor{currentstroke}%
\pgfsetdash{}{0pt}%
\pgfpathmoveto{\pgfqpoint{3.529412in}{1.750000in}}%
\pgfpathlineto{\pgfqpoint{3.529412in}{3.754545in}}%
\pgfusepath{stroke}%
\end{pgfscope}%
\begin{pgfscope}%
\pgfsetrectcap%
\pgfsetmiterjoin%
\pgfsetlinewidth{0.803000pt}%
\definecolor{currentstroke}{rgb}{0.000000,0.000000,0.000000}%
\pgfsetstrokecolor{currentstroke}%
\pgfsetdash{}{0pt}%
\pgfpathmoveto{\pgfqpoint{1.250000in}{1.750000in}}%
\pgfpathlineto{\pgfqpoint{3.529412in}{1.750000in}}%
\pgfusepath{stroke}%
\end{pgfscope}%
\begin{pgfscope}%
\pgfsetrectcap%
\pgfsetmiterjoin%
\pgfsetlinewidth{0.803000pt}%
\definecolor{currentstroke}{rgb}{0.000000,0.000000,0.000000}%
\pgfsetstrokecolor{currentstroke}%
\pgfsetdash{}{0pt}%
\pgfpathmoveto{\pgfqpoint{1.250000in}{3.754545in}}%
\pgfpathlineto{\pgfqpoint{3.529412in}{3.754545in}}%
\pgfusepath{stroke}%
\end{pgfscope}%
\begin{pgfscope}%
\definecolor{textcolor}{rgb}{0.000000,0.000000,0.000000}%
\pgfsetstrokecolor{textcolor}%
\pgfsetfillcolor{textcolor}%
\pgftext[x=2.389706in,y=3.837879in,,base]{\color{textcolor}\sffamily\fontsize{12.000000}{14.400000}\selectfont d)}%
\end{pgfscope}%
\begin{pgfscope}%
\pgfsetbuttcap%
\pgfsetmiterjoin%
\definecolor{currentfill}{rgb}{1.000000,1.000000,1.000000}%
\pgfsetfillcolor{currentfill}%
\pgfsetlinewidth{0.000000pt}%
\definecolor{currentstroke}{rgb}{0.000000,0.000000,0.000000}%
\pgfsetstrokecolor{currentstroke}%
\pgfsetstrokeopacity{0.000000}%
\pgfsetdash{}{0pt}%
\pgfpathmoveto{\pgfqpoint{3.985294in}{1.750000in}}%
\pgfpathlineto{\pgfqpoint{6.264706in}{1.750000in}}%
\pgfpathlineto{\pgfqpoint{6.264706in}{3.754545in}}%
\pgfpathlineto{\pgfqpoint{3.985294in}{3.754545in}}%
\pgfpathlineto{\pgfqpoint{3.985294in}{1.750000in}}%
\pgfpathclose%
\pgfusepath{fill}%
\end{pgfscope}%
\begin{pgfscope}%
\pgfpathrectangle{\pgfqpoint{3.985294in}{1.750000in}}{\pgfqpoint{2.279412in}{2.004545in}}%
\pgfusepath{clip}%
\pgfsys@transformcm{2.291667}{0.000000}{0.000000}{2.013889}{3.985294in}{1.750000in}%
\pgftext[left,bottom]{\includegraphics[interpolate=false,width=1.000000in,height=1.000000in]{q_series-img4.png}}%
\end{pgfscope}%
\begin{pgfscope}%
\pgfsetbuttcap%
\pgfsetroundjoin%
\definecolor{currentfill}{rgb}{0.000000,0.000000,0.000000}%
\pgfsetfillcolor{currentfill}%
\pgfsetlinewidth{0.803000pt}%
\definecolor{currentstroke}{rgb}{0.000000,0.000000,0.000000}%
\pgfsetstrokecolor{currentstroke}%
\pgfsetdash{}{0pt}%
\pgfsys@defobject{currentmarker}{\pgfqpoint{0.000000in}{-0.048611in}}{\pgfqpoint{0.000000in}{0.000000in}}{%
\pgfpathmoveto{\pgfqpoint{0.000000in}{0.000000in}}%
\pgfpathlineto{\pgfqpoint{0.000000in}{-0.048611in}}%
\pgfusepath{stroke,fill}%
}%
\begin{pgfscope}%
\pgfsys@transformshift{4.395836in}{1.750000in}%
\pgfsys@useobject{currentmarker}{}%
\end{pgfscope}%
\end{pgfscope}%
\begin{pgfscope}%
\definecolor{textcolor}{rgb}{0.000000,0.000000,0.000000}%
\pgfsetstrokecolor{textcolor}%
\pgfsetfillcolor{textcolor}%
\pgftext[x=4.395836in,y=1.652778in,,top]{\color{textcolor}\sffamily\fontsize{10.000000}{12.000000}\selectfont \(\displaystyle {\ensuremath{-}10}\)}%
\end{pgfscope}%
\begin{pgfscope}%
\pgfsetbuttcap%
\pgfsetroundjoin%
\definecolor{currentfill}{rgb}{0.000000,0.000000,0.000000}%
\pgfsetfillcolor{currentfill}%
\pgfsetlinewidth{0.803000pt}%
\definecolor{currentstroke}{rgb}{0.000000,0.000000,0.000000}%
\pgfsetstrokecolor{currentstroke}%
\pgfsetdash{}{0pt}%
\pgfsys@defobject{currentmarker}{\pgfqpoint{0.000000in}{-0.048611in}}{\pgfqpoint{0.000000in}{0.000000in}}{%
\pgfpathmoveto{\pgfqpoint{0.000000in}{0.000000in}}%
\pgfpathlineto{\pgfqpoint{0.000000in}{-0.048611in}}%
\pgfusepath{stroke,fill}%
}%
\begin{pgfscope}%
\pgfsys@transformshift{4.874388in}{1.750000in}%
\pgfsys@useobject{currentmarker}{}%
\end{pgfscope}%
\end{pgfscope}%
\begin{pgfscope}%
\definecolor{textcolor}{rgb}{0.000000,0.000000,0.000000}%
\pgfsetstrokecolor{textcolor}%
\pgfsetfillcolor{textcolor}%
\pgftext[x=4.874388in,y=1.652778in,,top]{\color{textcolor}\sffamily\fontsize{10.000000}{12.000000}\selectfont \(\displaystyle {\ensuremath{-}5}\)}%
\end{pgfscope}%
\begin{pgfscope}%
\pgfsetbuttcap%
\pgfsetroundjoin%
\definecolor{currentfill}{rgb}{0.000000,0.000000,0.000000}%
\pgfsetfillcolor{currentfill}%
\pgfsetlinewidth{0.803000pt}%
\definecolor{currentstroke}{rgb}{0.000000,0.000000,0.000000}%
\pgfsetstrokecolor{currentstroke}%
\pgfsetdash{}{0pt}%
\pgfsys@defobject{currentmarker}{\pgfqpoint{0.000000in}{-0.048611in}}{\pgfqpoint{0.000000in}{0.000000in}}{%
\pgfpathmoveto{\pgfqpoint{0.000000in}{0.000000in}}%
\pgfpathlineto{\pgfqpoint{0.000000in}{-0.048611in}}%
\pgfusepath{stroke,fill}%
}%
\begin{pgfscope}%
\pgfsys@transformshift{5.352941in}{1.750000in}%
\pgfsys@useobject{currentmarker}{}%
\end{pgfscope}%
\end{pgfscope}%
\begin{pgfscope}%
\definecolor{textcolor}{rgb}{0.000000,0.000000,0.000000}%
\pgfsetstrokecolor{textcolor}%
\pgfsetfillcolor{textcolor}%
\pgftext[x=5.352941in,y=1.652778in,,top]{\color{textcolor}\sffamily\fontsize{10.000000}{12.000000}\selectfont \(\displaystyle {0}\)}%
\end{pgfscope}%
\begin{pgfscope}%
\pgfsetbuttcap%
\pgfsetroundjoin%
\definecolor{currentfill}{rgb}{0.000000,0.000000,0.000000}%
\pgfsetfillcolor{currentfill}%
\pgfsetlinewidth{0.803000pt}%
\definecolor{currentstroke}{rgb}{0.000000,0.000000,0.000000}%
\pgfsetstrokecolor{currentstroke}%
\pgfsetdash{}{0pt}%
\pgfsys@defobject{currentmarker}{\pgfqpoint{0.000000in}{-0.048611in}}{\pgfqpoint{0.000000in}{0.000000in}}{%
\pgfpathmoveto{\pgfqpoint{0.000000in}{0.000000in}}%
\pgfpathlineto{\pgfqpoint{0.000000in}{-0.048611in}}%
\pgfusepath{stroke,fill}%
}%
\begin{pgfscope}%
\pgfsys@transformshift{5.831494in}{1.750000in}%
\pgfsys@useobject{currentmarker}{}%
\end{pgfscope}%
\end{pgfscope}%
\begin{pgfscope}%
\definecolor{textcolor}{rgb}{0.000000,0.000000,0.000000}%
\pgfsetstrokecolor{textcolor}%
\pgfsetfillcolor{textcolor}%
\pgftext[x=5.831494in,y=1.652778in,,top]{\color{textcolor}\sffamily\fontsize{10.000000}{12.000000}\selectfont \(\displaystyle {5}\)}%
\end{pgfscope}%
\begin{pgfscope}%
\definecolor{textcolor}{rgb}{0.000000,0.000000,0.000000}%
\pgfsetstrokecolor{textcolor}%
\pgfsetfillcolor{textcolor}%
\pgftext[x=5.125000in,y=1.473766in,,top]{\color{textcolor}\sffamily\fontsize{10.000000}{12.000000}\selectfont \(\displaystyle \zeta \, \mathrm{[\mu m]}\)}%
\end{pgfscope}%
\begin{pgfscope}%
\pgfsetbuttcap%
\pgfsetroundjoin%
\definecolor{currentfill}{rgb}{0.000000,0.000000,0.000000}%
\pgfsetfillcolor{currentfill}%
\pgfsetlinewidth{0.803000pt}%
\definecolor{currentstroke}{rgb}{0.000000,0.000000,0.000000}%
\pgfsetstrokecolor{currentstroke}%
\pgfsetdash{}{0pt}%
\pgfsys@defobject{currentmarker}{\pgfqpoint{-0.048611in}{0.000000in}}{\pgfqpoint{-0.000000in}{0.000000in}}{%
\pgfpathmoveto{\pgfqpoint{-0.000000in}{0.000000in}}%
\pgfpathlineto{\pgfqpoint{-0.048611in}{0.000000in}}%
\pgfusepath{stroke,fill}%
}%
\begin{pgfscope}%
\pgfsys@transformshift{3.985294in}{1.758025in}%
\pgfsys@useobject{currentmarker}{}%
\end{pgfscope}%
\end{pgfscope}%
\begin{pgfscope}%
\pgfsetbuttcap%
\pgfsetroundjoin%
\definecolor{currentfill}{rgb}{0.000000,0.000000,0.000000}%
\pgfsetfillcolor{currentfill}%
\pgfsetlinewidth{0.803000pt}%
\definecolor{currentstroke}{rgb}{0.000000,0.000000,0.000000}%
\pgfsetstrokecolor{currentstroke}%
\pgfsetdash{}{0pt}%
\pgfsys@defobject{currentmarker}{\pgfqpoint{-0.048611in}{0.000000in}}{\pgfqpoint{-0.000000in}{0.000000in}}{%
\pgfpathmoveto{\pgfqpoint{-0.000000in}{0.000000in}}%
\pgfpathlineto{\pgfqpoint{-0.048611in}{0.000000in}}%
\pgfusepath{stroke,fill}%
}%
\begin{pgfscope}%
\pgfsys@transformshift{3.985294in}{2.089441in}%
\pgfsys@useobject{currentmarker}{}%
\end{pgfscope}%
\end{pgfscope}%
\begin{pgfscope}%
\pgfsetbuttcap%
\pgfsetroundjoin%
\definecolor{currentfill}{rgb}{0.000000,0.000000,0.000000}%
\pgfsetfillcolor{currentfill}%
\pgfsetlinewidth{0.803000pt}%
\definecolor{currentstroke}{rgb}{0.000000,0.000000,0.000000}%
\pgfsetstrokecolor{currentstroke}%
\pgfsetdash{}{0pt}%
\pgfsys@defobject{currentmarker}{\pgfqpoint{-0.048611in}{0.000000in}}{\pgfqpoint{-0.000000in}{0.000000in}}{%
\pgfpathmoveto{\pgfqpoint{-0.000000in}{0.000000in}}%
\pgfpathlineto{\pgfqpoint{-0.048611in}{0.000000in}}%
\pgfusepath{stroke,fill}%
}%
\begin{pgfscope}%
\pgfsys@transformshift{3.985294in}{2.420857in}%
\pgfsys@useobject{currentmarker}{}%
\end{pgfscope}%
\end{pgfscope}%
\begin{pgfscope}%
\pgfsetbuttcap%
\pgfsetroundjoin%
\definecolor{currentfill}{rgb}{0.000000,0.000000,0.000000}%
\pgfsetfillcolor{currentfill}%
\pgfsetlinewidth{0.803000pt}%
\definecolor{currentstroke}{rgb}{0.000000,0.000000,0.000000}%
\pgfsetstrokecolor{currentstroke}%
\pgfsetdash{}{0pt}%
\pgfsys@defobject{currentmarker}{\pgfqpoint{-0.048611in}{0.000000in}}{\pgfqpoint{-0.000000in}{0.000000in}}{%
\pgfpathmoveto{\pgfqpoint{-0.000000in}{0.000000in}}%
\pgfpathlineto{\pgfqpoint{-0.048611in}{0.000000in}}%
\pgfusepath{stroke,fill}%
}%
\begin{pgfscope}%
\pgfsys@transformshift{3.985294in}{2.752273in}%
\pgfsys@useobject{currentmarker}{}%
\end{pgfscope}%
\end{pgfscope}%
\begin{pgfscope}%
\pgfsetbuttcap%
\pgfsetroundjoin%
\definecolor{currentfill}{rgb}{0.000000,0.000000,0.000000}%
\pgfsetfillcolor{currentfill}%
\pgfsetlinewidth{0.803000pt}%
\definecolor{currentstroke}{rgb}{0.000000,0.000000,0.000000}%
\pgfsetstrokecolor{currentstroke}%
\pgfsetdash{}{0pt}%
\pgfsys@defobject{currentmarker}{\pgfqpoint{-0.048611in}{0.000000in}}{\pgfqpoint{-0.000000in}{0.000000in}}{%
\pgfpathmoveto{\pgfqpoint{-0.000000in}{0.000000in}}%
\pgfpathlineto{\pgfqpoint{-0.048611in}{0.000000in}}%
\pgfusepath{stroke,fill}%
}%
\begin{pgfscope}%
\pgfsys@transformshift{3.985294in}{3.083689in}%
\pgfsys@useobject{currentmarker}{}%
\end{pgfscope}%
\end{pgfscope}%
\begin{pgfscope}%
\pgfsetbuttcap%
\pgfsetroundjoin%
\definecolor{currentfill}{rgb}{0.000000,0.000000,0.000000}%
\pgfsetfillcolor{currentfill}%
\pgfsetlinewidth{0.803000pt}%
\definecolor{currentstroke}{rgb}{0.000000,0.000000,0.000000}%
\pgfsetstrokecolor{currentstroke}%
\pgfsetdash{}{0pt}%
\pgfsys@defobject{currentmarker}{\pgfqpoint{-0.048611in}{0.000000in}}{\pgfqpoint{-0.000000in}{0.000000in}}{%
\pgfpathmoveto{\pgfqpoint{-0.000000in}{0.000000in}}%
\pgfpathlineto{\pgfqpoint{-0.048611in}{0.000000in}}%
\pgfusepath{stroke,fill}%
}%
\begin{pgfscope}%
\pgfsys@transformshift{3.985294in}{3.415105in}%
\pgfsys@useobject{currentmarker}{}%
\end{pgfscope}%
\end{pgfscope}%
\begin{pgfscope}%
\pgfsetbuttcap%
\pgfsetroundjoin%
\definecolor{currentfill}{rgb}{0.000000,0.000000,0.000000}%
\pgfsetfillcolor{currentfill}%
\pgfsetlinewidth{0.803000pt}%
\definecolor{currentstroke}{rgb}{0.000000,0.000000,0.000000}%
\pgfsetstrokecolor{currentstroke}%
\pgfsetdash{}{0pt}%
\pgfsys@defobject{currentmarker}{\pgfqpoint{-0.048611in}{0.000000in}}{\pgfqpoint{-0.000000in}{0.000000in}}{%
\pgfpathmoveto{\pgfqpoint{-0.000000in}{0.000000in}}%
\pgfpathlineto{\pgfqpoint{-0.048611in}{0.000000in}}%
\pgfusepath{stroke,fill}%
}%
\begin{pgfscope}%
\pgfsys@transformshift{3.985294in}{3.746521in}%
\pgfsys@useobject{currentmarker}{}%
\end{pgfscope}%
\end{pgfscope}%
\begin{pgfscope}%
\definecolor{textcolor}{rgb}{0.000000,0.000000,0.000000}%
\pgfsetstrokecolor{textcolor}%
\pgfsetfillcolor{textcolor}%
\pgftext[x=3.929739in,y=2.752273in,,bottom,rotate=90.000000]{\color{textcolor}\sffamily\fontsize{10.000000}{12.000000}\selectfont \(\displaystyle z \, \mathrm{[\mu m]}\)}%
\end{pgfscope}%
\begin{pgfscope}%
\pgfpathrectangle{\pgfqpoint{3.985294in}{1.750000in}}{\pgfqpoint{2.279412in}{2.004545in}}%
\pgfusepath{clip}%
\pgfsetbuttcap%
\pgfsetroundjoin%
\pgfsetlinewidth{0.312828pt}%
\definecolor{currentstroke}{rgb}{0.268510,0.009605,0.335427}%
\pgfsetstrokecolor{currentstroke}%
\pgfsetdash{}{0pt}%
\pgfpathmoveto{\pgfqpoint{5.945670in}{2.797380in}}%
\pgfpathlineto{\pgfqpoint{5.895578in}{2.797547in}}%
\pgfusepath{stroke}%
\end{pgfscope}%
\begin{pgfscope}%
\pgfpathrectangle{\pgfqpoint{3.985294in}{1.750000in}}{\pgfqpoint{2.279412in}{2.004545in}}%
\pgfusepath{clip}%
\pgfsetbuttcap%
\pgfsetroundjoin%
\pgfsetlinewidth{0.310269pt}%
\definecolor{currentstroke}{rgb}{0.268510,0.009605,0.335427}%
\pgfsetstrokecolor{currentstroke}%
\pgfsetdash{}{0pt}%
\pgfpathmoveto{\pgfqpoint{5.895578in}{2.797547in}}%
\pgfpathlineto{\pgfqpoint{5.845497in}{2.797272in}}%
\pgfusepath{stroke}%
\end{pgfscope}%
\begin{pgfscope}%
\pgfpathrectangle{\pgfqpoint{3.985294in}{1.750000in}}{\pgfqpoint{2.279412in}{2.004545in}}%
\pgfusepath{clip}%
\pgfsetbuttcap%
\pgfsetroundjoin%
\pgfsetlinewidth{0.319025pt}%
\definecolor{currentstroke}{rgb}{0.269944,0.014625,0.341379}%
\pgfsetstrokecolor{currentstroke}%
\pgfsetdash{}{0pt}%
\pgfpathmoveto{\pgfqpoint{5.845497in}{2.797272in}}%
\pgfpathlineto{\pgfqpoint{5.795424in}{2.796711in}}%
\pgfusepath{stroke}%
\end{pgfscope}%
\begin{pgfscope}%
\pgfpathrectangle{\pgfqpoint{3.985294in}{1.750000in}}{\pgfqpoint{2.279412in}{2.004545in}}%
\pgfusepath{clip}%
\pgfsetbuttcap%
\pgfsetroundjoin%
\pgfsetlinewidth{0.321601pt}%
\definecolor{currentstroke}{rgb}{0.269944,0.014625,0.341379}%
\pgfsetstrokecolor{currentstroke}%
\pgfsetdash{}{0pt}%
\pgfpathmoveto{\pgfqpoint{5.795424in}{2.796711in}}%
\pgfpathlineto{\pgfqpoint{5.745304in}{2.796056in}}%
\pgfusepath{stroke}%
\end{pgfscope}%
\begin{pgfscope}%
\pgfpathrectangle{\pgfqpoint{3.985294in}{1.750000in}}{\pgfqpoint{2.279412in}{2.004545in}}%
\pgfusepath{clip}%
\pgfsetbuttcap%
\pgfsetroundjoin%
\pgfsetlinewidth{0.333302pt}%
\definecolor{currentstroke}{rgb}{0.272594,0.025563,0.353093}%
\pgfsetstrokecolor{currentstroke}%
\pgfsetdash{}{0pt}%
\pgfpathmoveto{\pgfqpoint{5.745304in}{2.796056in}}%
\pgfpathlineto{\pgfqpoint{5.695157in}{2.795917in}}%
\pgfusepath{stroke}%
\end{pgfscope}%
\begin{pgfscope}%
\pgfpathrectangle{\pgfqpoint{3.985294in}{1.750000in}}{\pgfqpoint{2.279412in}{2.004545in}}%
\pgfusepath{clip}%
\pgfsetbuttcap%
\pgfsetroundjoin%
\pgfsetlinewidth{0.342905pt}%
\definecolor{currentstroke}{rgb}{0.274952,0.037752,0.364543}%
\pgfsetstrokecolor{currentstroke}%
\pgfsetdash{}{0pt}%
\pgfpathmoveto{\pgfqpoint{5.695157in}{2.795917in}}%
\pgfpathlineto{\pgfqpoint{5.645006in}{2.795948in}}%
\pgfusepath{stroke}%
\end{pgfscope}%
\begin{pgfscope}%
\pgfpathrectangle{\pgfqpoint{3.985294in}{1.750000in}}{\pgfqpoint{2.279412in}{2.004545in}}%
\pgfusepath{clip}%
\pgfsetbuttcap%
\pgfsetroundjoin%
\pgfsetlinewidth{0.377669pt}%
\definecolor{currentstroke}{rgb}{0.279566,0.067836,0.391917}%
\pgfsetstrokecolor{currentstroke}%
\pgfsetdash{}{0pt}%
\pgfpathmoveto{\pgfqpoint{5.645006in}{2.795948in}}%
\pgfpathlineto{\pgfqpoint{5.594855in}{2.795778in}}%
\pgfusepath{stroke}%
\end{pgfscope}%
\begin{pgfscope}%
\pgfpathrectangle{\pgfqpoint{3.985294in}{1.750000in}}{\pgfqpoint{2.279412in}{2.004545in}}%
\pgfusepath{clip}%
\pgfsetbuttcap%
\pgfsetroundjoin%
\pgfsetlinewidth{0.430685pt}%
\definecolor{currentstroke}{rgb}{0.282910,0.105393,0.426902}%
\pgfsetstrokecolor{currentstroke}%
\pgfsetdash{}{0pt}%
\pgfpathmoveto{\pgfqpoint{5.594855in}{2.795778in}}%
\pgfpathlineto{\pgfqpoint{5.544705in}{2.795438in}}%
\pgfusepath{stroke}%
\end{pgfscope}%
\begin{pgfscope}%
\pgfpathrectangle{\pgfqpoint{3.985294in}{1.750000in}}{\pgfqpoint{2.279412in}{2.004545in}}%
\pgfusepath{clip}%
\pgfsetbuttcap%
\pgfsetroundjoin%
\pgfsetlinewidth{0.476832pt}%
\definecolor{currentstroke}{rgb}{0.282623,0.140926,0.457517}%
\pgfsetstrokecolor{currentstroke}%
\pgfsetdash{}{0pt}%
\pgfpathmoveto{\pgfqpoint{5.544705in}{2.795438in}}%
\pgfpathlineto{\pgfqpoint{5.494554in}{2.795154in}}%
\pgfusepath{stroke}%
\end{pgfscope}%
\begin{pgfscope}%
\pgfpathrectangle{\pgfqpoint{3.985294in}{1.750000in}}{\pgfqpoint{2.279412in}{2.004545in}}%
\pgfusepath{clip}%
\pgfsetbuttcap%
\pgfsetroundjoin%
\pgfsetlinewidth{0.538580pt}%
\definecolor{currentstroke}{rgb}{0.277134,0.185228,0.489898}%
\pgfsetstrokecolor{currentstroke}%
\pgfsetdash{}{0pt}%
\pgfpathmoveto{\pgfqpoint{5.494554in}{2.795154in}}%
\pgfpathlineto{\pgfqpoint{5.444404in}{2.794861in}}%
\pgfusepath{stroke}%
\end{pgfscope}%
\begin{pgfscope}%
\pgfpathrectangle{\pgfqpoint{3.985294in}{1.750000in}}{\pgfqpoint{2.279412in}{2.004545in}}%
\pgfusepath{clip}%
\pgfsetbuttcap%
\pgfsetroundjoin%
\pgfsetlinewidth{0.629436pt}%
\definecolor{currentstroke}{rgb}{0.260571,0.246922,0.522828}%
\pgfsetstrokecolor{currentstroke}%
\pgfsetdash{}{0pt}%
\pgfpathmoveto{\pgfqpoint{5.444404in}{2.794861in}}%
\pgfpathlineto{\pgfqpoint{5.394253in}{2.794519in}}%
\pgfusepath{stroke}%
\end{pgfscope}%
\begin{pgfscope}%
\pgfpathrectangle{\pgfqpoint{3.985294in}{1.750000in}}{\pgfqpoint{2.279412in}{2.004545in}}%
\pgfusepath{clip}%
\pgfsetbuttcap%
\pgfsetroundjoin%
\pgfsetlinewidth{0.778921pt}%
\definecolor{currentstroke}{rgb}{0.220057,0.343307,0.549413}%
\pgfsetstrokecolor{currentstroke}%
\pgfsetdash{}{0pt}%
\pgfpathmoveto{\pgfqpoint{5.394253in}{2.794519in}}%
\pgfpathlineto{\pgfqpoint{5.344106in}{2.793973in}}%
\pgfusepath{stroke}%
\end{pgfscope}%
\begin{pgfscope}%
\pgfpathrectangle{\pgfqpoint{3.985294in}{1.750000in}}{\pgfqpoint{2.279412in}{2.004545in}}%
\pgfusepath{clip}%
\pgfsetbuttcap%
\pgfsetroundjoin%
\pgfsetlinewidth{0.872830pt}%
\definecolor{currentstroke}{rgb}{0.194100,0.399323,0.555565}%
\pgfsetstrokecolor{currentstroke}%
\pgfsetdash{}{0pt}%
\pgfpathmoveto{\pgfqpoint{5.344106in}{2.793973in}}%
\pgfpathlineto{\pgfqpoint{5.293965in}{2.793064in}}%
\pgfusepath{stroke}%
\end{pgfscope}%
\begin{pgfscope}%
\pgfpathrectangle{\pgfqpoint{3.985294in}{1.750000in}}{\pgfqpoint{2.279412in}{2.004545in}}%
\pgfusepath{clip}%
\pgfsetbuttcap%
\pgfsetroundjoin%
\pgfsetlinewidth{0.935027pt}%
\definecolor{currentstroke}{rgb}{0.179019,0.433756,0.557430}%
\pgfsetstrokecolor{currentstroke}%
\pgfsetdash{}{0pt}%
\pgfpathmoveto{\pgfqpoint{5.293965in}{2.793064in}}%
\pgfpathlineto{\pgfqpoint{5.243833in}{2.791850in}}%
\pgfusepath{stroke}%
\end{pgfscope}%
\begin{pgfscope}%
\pgfpathrectangle{\pgfqpoint{3.985294in}{1.750000in}}{\pgfqpoint{2.279412in}{2.004545in}}%
\pgfusepath{clip}%
\pgfsetbuttcap%
\pgfsetroundjoin%
\pgfsetlinewidth{0.997916pt}%
\definecolor{currentstroke}{rgb}{0.165117,0.467423,0.558141}%
\pgfsetstrokecolor{currentstroke}%
\pgfsetdash{}{0pt}%
\pgfpathmoveto{\pgfqpoint{5.243833in}{2.791850in}}%
\pgfpathlineto{\pgfqpoint{5.193711in}{2.790332in}}%
\pgfusepath{stroke}%
\end{pgfscope}%
\begin{pgfscope}%
\pgfpathrectangle{\pgfqpoint{3.985294in}{1.750000in}}{\pgfqpoint{2.279412in}{2.004545in}}%
\pgfusepath{clip}%
\pgfsetbuttcap%
\pgfsetroundjoin%
\pgfsetlinewidth{0.989865pt}%
\definecolor{currentstroke}{rgb}{0.166617,0.463708,0.558119}%
\pgfsetstrokecolor{currentstroke}%
\pgfsetdash{}{0pt}%
\pgfpathmoveto{\pgfqpoint{5.193711in}{2.790332in}}%
\pgfpathlineto{\pgfqpoint{5.143613in}{2.788340in}}%
\pgfusepath{stroke}%
\end{pgfscope}%
\begin{pgfscope}%
\pgfpathrectangle{\pgfqpoint{3.985294in}{1.750000in}}{\pgfqpoint{2.279412in}{2.004545in}}%
\pgfusepath{clip}%
\pgfsetbuttcap%
\pgfsetroundjoin%
\pgfsetlinewidth{0.958060pt}%
\definecolor{currentstroke}{rgb}{0.174274,0.445044,0.557792}%
\pgfsetstrokecolor{currentstroke}%
\pgfsetdash{}{0pt}%
\pgfpathmoveto{\pgfqpoint{5.143613in}{2.788340in}}%
\pgfpathlineto{\pgfqpoint{5.093546in}{2.785823in}}%
\pgfusepath{stroke}%
\end{pgfscope}%
\begin{pgfscope}%
\pgfpathrectangle{\pgfqpoint{3.985294in}{1.750000in}}{\pgfqpoint{2.279412in}{2.004545in}}%
\pgfusepath{clip}%
\pgfsetbuttcap%
\pgfsetroundjoin%
\pgfsetlinewidth{1.028129pt}%
\definecolor{currentstroke}{rgb}{0.159194,0.482237,0.558073}%
\pgfsetstrokecolor{currentstroke}%
\pgfsetdash{}{0pt}%
\pgfpathmoveto{\pgfqpoint{5.093546in}{2.785823in}}%
\pgfpathlineto{\pgfqpoint{5.043511in}{2.782872in}}%
\pgfusepath{stroke}%
\end{pgfscope}%
\begin{pgfscope}%
\pgfpathrectangle{\pgfqpoint{3.985294in}{1.750000in}}{\pgfqpoint{2.279412in}{2.004545in}}%
\pgfusepath{clip}%
\pgfsetbuttcap%
\pgfsetroundjoin%
\pgfsetlinewidth{0.949756pt}%
\definecolor{currentstroke}{rgb}{0.175841,0.441290,0.557685}%
\pgfsetstrokecolor{currentstroke}%
\pgfsetdash{}{0pt}%
\pgfpathmoveto{\pgfqpoint{5.043511in}{2.782872in}}%
\pgfpathlineto{\pgfqpoint{4.993511in}{2.779513in}}%
\pgfusepath{stroke}%
\end{pgfscope}%
\begin{pgfscope}%
\pgfpathrectangle{\pgfqpoint{3.985294in}{1.750000in}}{\pgfqpoint{2.279412in}{2.004545in}}%
\pgfusepath{clip}%
\pgfsetbuttcap%
\pgfsetroundjoin%
\pgfsetlinewidth{0.963076pt}%
\definecolor{currentstroke}{rgb}{0.172719,0.448791,0.557885}%
\pgfsetstrokecolor{currentstroke}%
\pgfsetdash{}{0pt}%
\pgfpathmoveto{\pgfqpoint{4.993511in}{2.779513in}}%
\pgfpathlineto{\pgfqpoint{4.943565in}{2.775603in}}%
\pgfusepath{stroke}%
\end{pgfscope}%
\begin{pgfscope}%
\pgfpathrectangle{\pgfqpoint{3.985294in}{1.750000in}}{\pgfqpoint{2.279412in}{2.004545in}}%
\pgfusepath{clip}%
\pgfsetbuttcap%
\pgfsetroundjoin%
\pgfsetlinewidth{0.926902pt}%
\definecolor{currentstroke}{rgb}{0.180629,0.429975,0.557282}%
\pgfsetstrokecolor{currentstroke}%
\pgfsetdash{}{0pt}%
\pgfpathmoveto{\pgfqpoint{4.943565in}{2.775603in}}%
\pgfpathlineto{\pgfqpoint{4.893659in}{2.771283in}}%
\pgfusepath{stroke}%
\end{pgfscope}%
\begin{pgfscope}%
\pgfpathrectangle{\pgfqpoint{3.985294in}{1.750000in}}{\pgfqpoint{2.279412in}{2.004545in}}%
\pgfusepath{clip}%
\pgfsetbuttcap%
\pgfsetroundjoin%
\pgfsetlinewidth{0.781780pt}%
\definecolor{currentstroke}{rgb}{0.218130,0.347432,0.550038}%
\pgfsetstrokecolor{currentstroke}%
\pgfsetdash{}{0pt}%
\pgfpathmoveto{\pgfqpoint{4.893659in}{2.771283in}}%
\pgfpathlineto{\pgfqpoint{4.843832in}{2.766430in}}%
\pgfusepath{stroke}%
\end{pgfscope}%
\begin{pgfscope}%
\pgfpathrectangle{\pgfqpoint{3.985294in}{1.750000in}}{\pgfqpoint{2.279412in}{2.004545in}}%
\pgfusepath{clip}%
\pgfsetbuttcap%
\pgfsetroundjoin%
\pgfsetlinewidth{0.770538pt}%
\definecolor{currentstroke}{rgb}{0.221989,0.339161,0.548752}%
\pgfsetstrokecolor{currentstroke}%
\pgfsetdash{}{0pt}%
\pgfpathmoveto{\pgfqpoint{4.843832in}{2.766430in}}%
\pgfpathlineto{\pgfqpoint{4.794211in}{2.760317in}}%
\pgfusepath{stroke}%
\end{pgfscope}%
\begin{pgfscope}%
\pgfpathrectangle{\pgfqpoint{3.985294in}{1.750000in}}{\pgfqpoint{2.279412in}{2.004545in}}%
\pgfusepath{clip}%
\pgfsetbuttcap%
\pgfsetroundjoin%
\pgfsetlinewidth{0.675290pt}%
\definecolor{currentstroke}{rgb}{0.248629,0.278775,0.534556}%
\pgfsetstrokecolor{currentstroke}%
\pgfsetdash{}{0pt}%
\pgfpathmoveto{\pgfqpoint{4.794211in}{2.760317in}}%
\pgfpathlineto{\pgfqpoint{4.744990in}{2.752312in}}%
\pgfusepath{stroke}%
\end{pgfscope}%
\begin{pgfscope}%
\pgfpathrectangle{\pgfqpoint{3.985294in}{1.750000in}}{\pgfqpoint{2.279412in}{2.004545in}}%
\pgfusepath{clip}%
\pgfsetbuttcap%
\pgfsetroundjoin%
\pgfsetlinewidth{0.593919pt}%
\definecolor{currentstroke}{rgb}{0.267968,0.223549,0.512008}%
\pgfsetstrokecolor{currentstroke}%
\pgfsetdash{}{0pt}%
\pgfpathmoveto{\pgfqpoint{4.744990in}{2.752312in}}%
\pgfpathlineto{\pgfqpoint{4.697476in}{2.740838in}}%
\pgfusepath{stroke}%
\end{pgfscope}%
\begin{pgfscope}%
\pgfpathrectangle{\pgfqpoint{3.985294in}{1.750000in}}{\pgfqpoint{2.279412in}{2.004545in}}%
\pgfusepath{clip}%
\pgfsetbuttcap%
\pgfsetroundjoin%
\pgfsetlinewidth{0.469707pt}%
\definecolor{currentstroke}{rgb}{0.282884,0.135920,0.453427}%
\pgfsetstrokecolor{currentstroke}%
\pgfsetdash{}{0pt}%
\pgfpathmoveto{\pgfqpoint{4.697476in}{2.740838in}}%
\pgfpathlineto{\pgfqpoint{4.697476in}{2.740838in}}%
\pgfusepath{stroke}%
\end{pgfscope}%
\begin{pgfscope}%
\pgfpathrectangle{\pgfqpoint{3.985294in}{1.750000in}}{\pgfqpoint{2.279412in}{2.004545in}}%
\pgfusepath{clip}%
\pgfsetbuttcap%
\pgfsetroundjoin%
\pgfsetlinewidth{0.469707pt}%
\definecolor{currentstroke}{rgb}{0.282884,0.135920,0.453427}%
\pgfsetstrokecolor{currentstroke}%
\pgfsetdash{}{0pt}%
\pgfpathmoveto{\pgfqpoint{4.697476in}{2.740838in}}%
\pgfpathlineto{\pgfqpoint{4.688732in}{2.736616in}}%
\pgfusepath{stroke}%
\end{pgfscope}%
\begin{pgfscope}%
\pgfpathrectangle{\pgfqpoint{3.985294in}{1.750000in}}{\pgfqpoint{2.279412in}{2.004545in}}%
\pgfusepath{clip}%
\pgfsetbuttcap%
\pgfsetroundjoin%
\pgfsetlinewidth{0.440900pt}%
\definecolor{currentstroke}{rgb}{0.283197,0.115680,0.436115}%
\pgfsetstrokecolor{currentstroke}%
\pgfsetdash{}{0pt}%
\pgfpathmoveto{\pgfqpoint{4.688732in}{2.736616in}}%
\pgfpathlineto{\pgfqpoint{4.685177in}{2.733211in}}%
\pgfusepath{stroke}%
\end{pgfscope}%
\begin{pgfscope}%
\pgfpathrectangle{\pgfqpoint{3.985294in}{1.750000in}}{\pgfqpoint{2.279412in}{2.004545in}}%
\pgfusepath{clip}%
\pgfsetbuttcap%
\pgfsetroundjoin%
\pgfsetlinewidth{0.420652pt}%
\definecolor{currentstroke}{rgb}{0.282656,0.100196,0.422160}%
\pgfsetstrokecolor{currentstroke}%
\pgfsetdash{}{0pt}%
\pgfpathmoveto{\pgfqpoint{4.685177in}{2.733211in}}%
\pgfpathlineto{\pgfqpoint{4.684051in}{2.730690in}}%
\pgfusepath{stroke}%
\end{pgfscope}%
\begin{pgfscope}%
\pgfpathrectangle{\pgfqpoint{3.985294in}{1.750000in}}{\pgfqpoint{2.279412in}{2.004545in}}%
\pgfusepath{clip}%
\pgfsetbuttcap%
\pgfsetroundjoin%
\pgfsetlinewidth{0.414309pt}%
\definecolor{currentstroke}{rgb}{0.282327,0.094955,0.417331}%
\pgfsetstrokecolor{currentstroke}%
\pgfsetdash{}{0pt}%
\pgfpathmoveto{\pgfqpoint{4.684051in}{2.730690in}}%
\pgfpathlineto{\pgfqpoint{4.684108in}{2.728896in}}%
\pgfusepath{stroke}%
\end{pgfscope}%
\begin{pgfscope}%
\pgfpathrectangle{\pgfqpoint{3.985294in}{1.750000in}}{\pgfqpoint{2.279412in}{2.004545in}}%
\pgfusepath{clip}%
\pgfsetbuttcap%
\pgfsetroundjoin%
\pgfsetlinewidth{0.414242pt}%
\definecolor{currentstroke}{rgb}{0.282327,0.094955,0.417331}%
\pgfsetstrokecolor{currentstroke}%
\pgfsetdash{}{0pt}%
\pgfpathmoveto{\pgfqpoint{4.684108in}{2.728896in}}%
\pgfpathlineto{\pgfqpoint{4.684800in}{2.727678in}}%
\pgfusepath{stroke}%
\end{pgfscope}%
\begin{pgfscope}%
\pgfpathrectangle{\pgfqpoint{3.985294in}{1.750000in}}{\pgfqpoint{2.279412in}{2.004545in}}%
\pgfusepath{clip}%
\pgfsetbuttcap%
\pgfsetroundjoin%
\pgfsetlinewidth{0.415561pt}%
\definecolor{currentstroke}{rgb}{0.282327,0.094955,0.417331}%
\pgfsetstrokecolor{currentstroke}%
\pgfsetdash{}{0pt}%
\pgfpathmoveto{\pgfqpoint{4.684800in}{2.727678in}}%
\pgfpathlineto{\pgfqpoint{4.686282in}{2.726872in}}%
\pgfusepath{stroke}%
\end{pgfscope}%
\begin{pgfscope}%
\pgfpathrectangle{\pgfqpoint{3.985294in}{1.750000in}}{\pgfqpoint{2.279412in}{2.004545in}}%
\pgfusepath{clip}%
\pgfsetbuttcap%
\pgfsetroundjoin%
\pgfsetlinewidth{0.418112pt}%
\definecolor{currentstroke}{rgb}{0.282656,0.100196,0.422160}%
\pgfsetstrokecolor{currentstroke}%
\pgfsetdash{}{0pt}%
\pgfpathmoveto{\pgfqpoint{4.686282in}{2.726872in}}%
\pgfpathlineto{\pgfqpoint{4.686282in}{2.726872in}}%
\pgfusepath{stroke}%
\end{pgfscope}%
\begin{pgfscope}%
\pgfpathrectangle{\pgfqpoint{3.985294in}{1.750000in}}{\pgfqpoint{2.279412in}{2.004545in}}%
\pgfusepath{clip}%
\pgfsetbuttcap%
\pgfsetroundjoin%
\pgfsetlinewidth{0.418112pt}%
\definecolor{currentstroke}{rgb}{0.282656,0.100196,0.422160}%
\pgfsetstrokecolor{currentstroke}%
\pgfsetdash{}{0pt}%
\pgfpathmoveto{\pgfqpoint{4.686282in}{2.726872in}}%
\pgfpathlineto{\pgfqpoint{4.686372in}{2.726564in}}%
\pgfusepath{stroke}%
\end{pgfscope}%
\begin{pgfscope}%
\pgfpathrectangle{\pgfqpoint{3.985294in}{1.750000in}}{\pgfqpoint{2.279412in}{2.004545in}}%
\pgfusepath{clip}%
\pgfsetbuttcap%
\pgfsetroundjoin%
\pgfsetlinewidth{0.418037pt}%
\definecolor{currentstroke}{rgb}{0.282656,0.100196,0.422160}%
\pgfsetstrokecolor{currentstroke}%
\pgfsetdash{}{0pt}%
\pgfpathmoveto{\pgfqpoint{4.686372in}{2.726564in}}%
\pgfpathlineto{\pgfqpoint{4.686369in}{2.726420in}}%
\pgfusepath{stroke}%
\end{pgfscope}%
\begin{pgfscope}%
\pgfpathrectangle{\pgfqpoint{3.985294in}{1.750000in}}{\pgfqpoint{2.279412in}{2.004545in}}%
\pgfusepath{clip}%
\pgfsetbuttcap%
\pgfsetroundjoin%
\pgfsetlinewidth{0.417916pt}%
\definecolor{currentstroke}{rgb}{0.282656,0.100196,0.422160}%
\pgfsetstrokecolor{currentstroke}%
\pgfsetdash{}{0pt}%
\pgfpathmoveto{\pgfqpoint{4.686369in}{2.726420in}}%
\pgfpathlineto{\pgfqpoint{4.686403in}{2.726341in}}%
\pgfusepath{stroke}%
\end{pgfscope}%
\begin{pgfscope}%
\pgfpathrectangle{\pgfqpoint{3.985294in}{1.750000in}}{\pgfqpoint{2.279412in}{2.004545in}}%
\pgfusepath{clip}%
\pgfsetbuttcap%
\pgfsetroundjoin%
\pgfsetlinewidth{0.417910pt}%
\definecolor{currentstroke}{rgb}{0.282656,0.100196,0.422160}%
\pgfsetstrokecolor{currentstroke}%
\pgfsetdash{}{0pt}%
\pgfpathmoveto{\pgfqpoint{4.686403in}{2.726341in}}%
\pgfpathlineto{\pgfqpoint{4.686501in}{2.726287in}}%
\pgfusepath{stroke}%
\end{pgfscope}%
\begin{pgfscope}%
\pgfpathrectangle{\pgfqpoint{3.985294in}{1.750000in}}{\pgfqpoint{2.279412in}{2.004545in}}%
\pgfusepath{clip}%
\pgfsetbuttcap%
\pgfsetroundjoin%
\pgfsetlinewidth{0.418036pt}%
\definecolor{currentstroke}{rgb}{0.282656,0.100196,0.422160}%
\pgfsetstrokecolor{currentstroke}%
\pgfsetdash{}{0pt}%
\pgfpathmoveto{\pgfqpoint{4.686501in}{2.726287in}}%
\pgfpathlineto{\pgfqpoint{4.686600in}{2.726250in}}%
\pgfusepath{stroke}%
\end{pgfscope}%
\begin{pgfscope}%
\pgfpathrectangle{\pgfqpoint{3.985294in}{1.750000in}}{\pgfqpoint{2.279412in}{2.004545in}}%
\pgfusepath{clip}%
\pgfsetbuttcap%
\pgfsetroundjoin%
\pgfsetlinewidth{0.418176pt}%
\definecolor{currentstroke}{rgb}{0.282656,0.100196,0.422160}%
\pgfsetstrokecolor{currentstroke}%
\pgfsetdash{}{0pt}%
\pgfpathmoveto{\pgfqpoint{4.686600in}{2.726250in}}%
\pgfpathlineto{\pgfqpoint{4.686612in}{2.726239in}}%
\pgfusepath{stroke}%
\end{pgfscope}%
\begin{pgfscope}%
\pgfpathrectangle{\pgfqpoint{3.985294in}{1.750000in}}{\pgfqpoint{2.279412in}{2.004545in}}%
\pgfusepath{clip}%
\pgfsetbuttcap%
\pgfsetroundjoin%
\pgfsetlinewidth{0.418187pt}%
\definecolor{currentstroke}{rgb}{0.282656,0.100196,0.422160}%
\pgfsetstrokecolor{currentstroke}%
\pgfsetdash{}{0pt}%
\pgfpathmoveto{\pgfqpoint{4.686612in}{2.726239in}}%
\pgfpathlineto{\pgfqpoint{4.686534in}{2.726249in}}%
\pgfusepath{stroke}%
\end{pgfscope}%
\begin{pgfscope}%
\pgfpathrectangle{\pgfqpoint{3.985294in}{1.750000in}}{\pgfqpoint{2.279412in}{2.004545in}}%
\pgfusepath{clip}%
\pgfsetbuttcap%
\pgfsetroundjoin%
\pgfsetlinewidth{0.418061pt}%
\definecolor{currentstroke}{rgb}{0.282656,0.100196,0.422160}%
\pgfsetstrokecolor{currentstroke}%
\pgfsetdash{}{0pt}%
\pgfpathmoveto{\pgfqpoint{4.686534in}{2.726249in}}%
\pgfpathlineto{\pgfqpoint{4.686451in}{2.726262in}}%
\pgfusepath{stroke}%
\end{pgfscope}%
\begin{pgfscope}%
\pgfpathrectangle{\pgfqpoint{3.985294in}{1.750000in}}{\pgfqpoint{2.279412in}{2.004545in}}%
\pgfusepath{clip}%
\pgfsetbuttcap%
\pgfsetroundjoin%
\pgfsetlinewidth{0.417930pt}%
\definecolor{currentstroke}{rgb}{0.282656,0.100196,0.422160}%
\pgfsetstrokecolor{currentstroke}%
\pgfsetdash{}{0pt}%
\pgfpathmoveto{\pgfqpoint{4.686451in}{2.726262in}}%
\pgfpathlineto{\pgfqpoint{4.686446in}{2.726262in}}%
\pgfusepath{stroke}%
\end{pgfscope}%
\begin{pgfscope}%
\pgfpathrectangle{\pgfqpoint{3.985294in}{1.750000in}}{\pgfqpoint{2.279412in}{2.004545in}}%
\pgfusepath{clip}%
\pgfsetbuttcap%
\pgfsetroundjoin%
\pgfsetlinewidth{0.417920pt}%
\definecolor{currentstroke}{rgb}{0.282656,0.100196,0.422160}%
\pgfsetstrokecolor{currentstroke}%
\pgfsetdash{}{0pt}%
\pgfpathmoveto{\pgfqpoint{4.686446in}{2.726262in}}%
\pgfpathlineto{\pgfqpoint{4.686523in}{2.726247in}}%
\pgfusepath{stroke}%
\end{pgfscope}%
\begin{pgfscope}%
\pgfpathrectangle{\pgfqpoint{3.985294in}{1.750000in}}{\pgfqpoint{2.279412in}{2.004545in}}%
\pgfusepath{clip}%
\pgfsetbuttcap%
\pgfsetroundjoin%
\pgfsetlinewidth{0.418041pt}%
\definecolor{currentstroke}{rgb}{0.282656,0.100196,0.422160}%
\pgfsetstrokecolor{currentstroke}%
\pgfsetdash{}{0pt}%
\pgfpathmoveto{\pgfqpoint{4.686523in}{2.726247in}}%
\pgfpathlineto{\pgfqpoint{4.686610in}{2.726231in}}%
\pgfusepath{stroke}%
\end{pgfscope}%
\begin{pgfscope}%
\pgfpathrectangle{\pgfqpoint{3.985294in}{1.750000in}}{\pgfqpoint{2.279412in}{2.004545in}}%
\pgfusepath{clip}%
\pgfsetbuttcap%
\pgfsetroundjoin%
\pgfsetlinewidth{0.418176pt}%
\definecolor{currentstroke}{rgb}{0.282656,0.100196,0.422160}%
\pgfsetstrokecolor{currentstroke}%
\pgfsetdash{}{0pt}%
\pgfpathmoveto{\pgfqpoint{4.686610in}{2.726231in}}%
\pgfpathlineto{\pgfqpoint{4.686615in}{2.726229in}}%
\pgfusepath{stroke}%
\end{pgfscope}%
\begin{pgfscope}%
\pgfpathrectangle{\pgfqpoint{3.985294in}{1.750000in}}{\pgfqpoint{2.279412in}{2.004545in}}%
\pgfusepath{clip}%
\pgfsetbuttcap%
\pgfsetroundjoin%
\pgfsetlinewidth{0.418184pt}%
\definecolor{currentstroke}{rgb}{0.282656,0.100196,0.422160}%
\pgfsetstrokecolor{currentstroke}%
\pgfsetdash{}{0pt}%
\pgfpathmoveto{\pgfqpoint{4.686615in}{2.726229in}}%
\pgfpathlineto{\pgfqpoint{4.686535in}{2.726244in}}%
\pgfusepath{stroke}%
\end{pgfscope}%
\begin{pgfscope}%
\pgfpathrectangle{\pgfqpoint{3.985294in}{1.750000in}}{\pgfqpoint{2.279412in}{2.004545in}}%
\pgfusepath{clip}%
\pgfsetbuttcap%
\pgfsetroundjoin%
\pgfsetlinewidth{0.418058pt}%
\definecolor{currentstroke}{rgb}{0.282656,0.100196,0.422160}%
\pgfsetstrokecolor{currentstroke}%
\pgfsetdash{}{0pt}%
\pgfpathmoveto{\pgfqpoint{4.686535in}{2.726244in}}%
\pgfpathlineto{\pgfqpoint{4.686453in}{2.726259in}}%
\pgfusepath{stroke}%
\end{pgfscope}%
\begin{pgfscope}%
\pgfpathrectangle{\pgfqpoint{3.985294in}{1.750000in}}{\pgfqpoint{2.279412in}{2.004545in}}%
\pgfusepath{clip}%
\pgfsetbuttcap%
\pgfsetroundjoin%
\pgfsetlinewidth{0.417930pt}%
\definecolor{currentstroke}{rgb}{0.282656,0.100196,0.422160}%
\pgfsetstrokecolor{currentstroke}%
\pgfsetdash{}{0pt}%
\pgfpathmoveto{\pgfqpoint{4.686453in}{2.726259in}}%
\pgfpathlineto{\pgfqpoint{4.686448in}{2.726260in}}%
\pgfusepath{stroke}%
\end{pgfscope}%
\begin{pgfscope}%
\pgfpathrectangle{\pgfqpoint{3.985294in}{1.750000in}}{\pgfqpoint{2.279412in}{2.004545in}}%
\pgfusepath{clip}%
\pgfsetbuttcap%
\pgfsetroundjoin%
\pgfsetlinewidth{0.417923pt}%
\definecolor{currentstroke}{rgb}{0.282656,0.100196,0.422160}%
\pgfsetstrokecolor{currentstroke}%
\pgfsetdash{}{0pt}%
\pgfpathmoveto{\pgfqpoint{4.686448in}{2.726260in}}%
\pgfpathlineto{\pgfqpoint{4.686526in}{2.726246in}}%
\pgfusepath{stroke}%
\end{pgfscope}%
\begin{pgfscope}%
\pgfpathrectangle{\pgfqpoint{3.985294in}{1.750000in}}{\pgfqpoint{2.279412in}{2.004545in}}%
\pgfusepath{clip}%
\pgfsetbuttcap%
\pgfsetroundjoin%
\pgfsetlinewidth{0.418045pt}%
\definecolor{currentstroke}{rgb}{0.282656,0.100196,0.422160}%
\pgfsetstrokecolor{currentstroke}%
\pgfsetdash{}{0pt}%
\pgfpathmoveto{\pgfqpoint{4.686526in}{2.726246in}}%
\pgfpathlineto{\pgfqpoint{4.686610in}{2.726231in}}%
\pgfusepath{stroke}%
\end{pgfscope}%
\begin{pgfscope}%
\pgfpathrectangle{\pgfqpoint{3.985294in}{1.750000in}}{\pgfqpoint{2.279412in}{2.004545in}}%
\pgfusepath{clip}%
\pgfsetbuttcap%
\pgfsetroundjoin%
\pgfsetlinewidth{0.418176pt}%
\definecolor{currentstroke}{rgb}{0.282656,0.100196,0.422160}%
\pgfsetstrokecolor{currentstroke}%
\pgfsetdash{}{0pt}%
\pgfpathmoveto{\pgfqpoint{4.686610in}{2.726231in}}%
\pgfpathlineto{\pgfqpoint{4.686613in}{2.726230in}}%
\pgfusepath{stroke}%
\end{pgfscope}%
\begin{pgfscope}%
\pgfpathrectangle{\pgfqpoint{3.985294in}{1.750000in}}{\pgfqpoint{2.279412in}{2.004545in}}%
\pgfusepath{clip}%
\pgfsetbuttcap%
\pgfsetroundjoin%
\pgfsetlinewidth{0.418181pt}%
\definecolor{currentstroke}{rgb}{0.282656,0.100196,0.422160}%
\pgfsetstrokecolor{currentstroke}%
\pgfsetdash{}{0pt}%
\pgfpathmoveto{\pgfqpoint{4.686613in}{2.726230in}}%
\pgfpathlineto{\pgfqpoint{4.686533in}{2.726244in}}%
\pgfusepath{stroke}%
\end{pgfscope}%
\begin{pgfscope}%
\pgfpathrectangle{\pgfqpoint{3.985294in}{1.750000in}}{\pgfqpoint{2.279412in}{2.004545in}}%
\pgfusepath{clip}%
\pgfsetbuttcap%
\pgfsetroundjoin%
\pgfsetlinewidth{0.418055pt}%
\definecolor{currentstroke}{rgb}{0.282656,0.100196,0.422160}%
\pgfsetstrokecolor{currentstroke}%
\pgfsetdash{}{0pt}%
\pgfpathmoveto{\pgfqpoint{4.686533in}{2.726244in}}%
\pgfpathlineto{\pgfqpoint{4.686453in}{2.726259in}}%
\pgfusepath{stroke}%
\end{pgfscope}%
\begin{pgfscope}%
\pgfpathrectangle{\pgfqpoint{3.985294in}{1.750000in}}{\pgfqpoint{2.279412in}{2.004545in}}%
\pgfusepath{clip}%
\pgfsetbuttcap%
\pgfsetroundjoin%
\pgfsetlinewidth{0.417930pt}%
\definecolor{currentstroke}{rgb}{0.282656,0.100196,0.422160}%
\pgfsetstrokecolor{currentstroke}%
\pgfsetdash{}{0pt}%
\pgfpathmoveto{\pgfqpoint{4.686453in}{2.726259in}}%
\pgfpathlineto{\pgfqpoint{4.686450in}{2.726260in}}%
\pgfusepath{stroke}%
\end{pgfscope}%
\begin{pgfscope}%
\pgfpathrectangle{\pgfqpoint{3.985294in}{1.750000in}}{\pgfqpoint{2.279412in}{2.004545in}}%
\pgfusepath{clip}%
\pgfsetbuttcap%
\pgfsetroundjoin%
\pgfsetlinewidth{0.417926pt}%
\definecolor{currentstroke}{rgb}{0.282656,0.100196,0.422160}%
\pgfsetstrokecolor{currentstroke}%
\pgfsetdash{}{0pt}%
\pgfpathmoveto{\pgfqpoint{4.686450in}{2.726260in}}%
\pgfpathlineto{\pgfqpoint{4.686528in}{2.726246in}}%
\pgfusepath{stroke}%
\end{pgfscope}%
\begin{pgfscope}%
\pgfpathrectangle{\pgfqpoint{3.985294in}{1.750000in}}{\pgfqpoint{2.279412in}{2.004545in}}%
\pgfusepath{clip}%
\pgfsetbuttcap%
\pgfsetroundjoin%
\pgfsetlinewidth{0.418047pt}%
\definecolor{currentstroke}{rgb}{0.282656,0.100196,0.422160}%
\pgfsetstrokecolor{currentstroke}%
\pgfsetdash{}{0pt}%
\pgfpathmoveto{\pgfqpoint{4.686528in}{2.726246in}}%
\pgfpathlineto{\pgfqpoint{4.686610in}{2.726231in}}%
\pgfusepath{stroke}%
\end{pgfscope}%
\begin{pgfscope}%
\pgfpathrectangle{\pgfqpoint{3.985294in}{1.750000in}}{\pgfqpoint{2.279412in}{2.004545in}}%
\pgfusepath{clip}%
\pgfsetbuttcap%
\pgfsetroundjoin%
\pgfsetlinewidth{0.418176pt}%
\definecolor{currentstroke}{rgb}{0.282656,0.100196,0.422160}%
\pgfsetstrokecolor{currentstroke}%
\pgfsetdash{}{0pt}%
\pgfpathmoveto{\pgfqpoint{4.686610in}{2.726231in}}%
\pgfpathlineto{\pgfqpoint{4.686611in}{2.726230in}}%
\pgfusepath{stroke}%
\end{pgfscope}%
\begin{pgfscope}%
\pgfpathrectangle{\pgfqpoint{3.985294in}{1.750000in}}{\pgfqpoint{2.279412in}{2.004545in}}%
\pgfusepath{clip}%
\pgfsetbuttcap%
\pgfsetroundjoin%
\pgfsetlinewidth{0.418177pt}%
\definecolor{currentstroke}{rgb}{0.282656,0.100196,0.422160}%
\pgfsetstrokecolor{currentstroke}%
\pgfsetdash{}{0pt}%
\pgfpathmoveto{\pgfqpoint{4.686611in}{2.726230in}}%
\pgfpathlineto{\pgfqpoint{4.686531in}{2.726245in}}%
\pgfusepath{stroke}%
\end{pgfscope}%
\begin{pgfscope}%
\pgfpathrectangle{\pgfqpoint{3.985294in}{1.750000in}}{\pgfqpoint{2.279412in}{2.004545in}}%
\pgfusepath{clip}%
\pgfsetbuttcap%
\pgfsetroundjoin%
\pgfsetlinewidth{0.418052pt}%
\definecolor{currentstroke}{rgb}{0.282656,0.100196,0.422160}%
\pgfsetstrokecolor{currentstroke}%
\pgfsetdash{}{0pt}%
\pgfpathmoveto{\pgfqpoint{4.686531in}{2.726245in}}%
\pgfpathlineto{\pgfqpoint{4.686453in}{2.726259in}}%
\pgfusepath{stroke}%
\end{pgfscope}%
\begin{pgfscope}%
\pgfpathrectangle{\pgfqpoint{3.985294in}{1.750000in}}{\pgfqpoint{2.279412in}{2.004545in}}%
\pgfusepath{clip}%
\pgfsetbuttcap%
\pgfsetroundjoin%
\pgfsetlinewidth{0.417930pt}%
\definecolor{currentstroke}{rgb}{0.282656,0.100196,0.422160}%
\pgfsetstrokecolor{currentstroke}%
\pgfsetdash{}{0pt}%
\pgfpathmoveto{\pgfqpoint{4.686453in}{2.726259in}}%
\pgfpathlineto{\pgfqpoint{4.686452in}{2.726260in}}%
\pgfusepath{stroke}%
\end{pgfscope}%
\begin{pgfscope}%
\pgfpathrectangle{\pgfqpoint{3.985294in}{1.750000in}}{\pgfqpoint{2.279412in}{2.004545in}}%
\pgfusepath{clip}%
\pgfsetbuttcap%
\pgfsetroundjoin%
\pgfsetlinewidth{0.417929pt}%
\definecolor{currentstroke}{rgb}{0.282656,0.100196,0.422160}%
\pgfsetstrokecolor{currentstroke}%
\pgfsetdash{}{0pt}%
\pgfpathmoveto{\pgfqpoint{4.686452in}{2.726260in}}%
\pgfpathlineto{\pgfqpoint{4.686529in}{2.726246in}}%
\pgfusepath{stroke}%
\end{pgfscope}%
\begin{pgfscope}%
\pgfpathrectangle{\pgfqpoint{3.985294in}{1.750000in}}{\pgfqpoint{2.279412in}{2.004545in}}%
\pgfusepath{clip}%
\pgfsetbuttcap%
\pgfsetroundjoin%
\pgfsetlinewidth{0.418050pt}%
\definecolor{currentstroke}{rgb}{0.282656,0.100196,0.422160}%
\pgfsetstrokecolor{currentstroke}%
\pgfsetdash{}{0pt}%
\pgfpathmoveto{\pgfqpoint{4.686529in}{2.726246in}}%
\pgfpathlineto{\pgfqpoint{4.686610in}{2.726231in}}%
\pgfusepath{stroke}%
\end{pgfscope}%
\begin{pgfscope}%
\pgfpathrectangle{\pgfqpoint{3.985294in}{1.750000in}}{\pgfqpoint{2.279412in}{2.004545in}}%
\pgfusepath{clip}%
\pgfsetbuttcap%
\pgfsetroundjoin%
\pgfsetlinewidth{0.418175pt}%
\definecolor{currentstroke}{rgb}{0.282656,0.100196,0.422160}%
\pgfsetstrokecolor{currentstroke}%
\pgfsetdash{}{0pt}%
\pgfpathmoveto{\pgfqpoint{4.686610in}{2.726231in}}%
\pgfpathlineto{\pgfqpoint{4.686609in}{2.726230in}}%
\pgfusepath{stroke}%
\end{pgfscope}%
\begin{pgfscope}%
\pgfpathrectangle{\pgfqpoint{3.985294in}{1.750000in}}{\pgfqpoint{2.279412in}{2.004545in}}%
\pgfusepath{clip}%
\pgfsetbuttcap%
\pgfsetroundjoin%
\pgfsetlinewidth{0.418174pt}%
\definecolor{currentstroke}{rgb}{0.282656,0.100196,0.422160}%
\pgfsetstrokecolor{currentstroke}%
\pgfsetdash{}{0pt}%
\pgfpathmoveto{\pgfqpoint{4.686609in}{2.726230in}}%
\pgfpathlineto{\pgfqpoint{4.686529in}{2.726245in}}%
\pgfusepath{stroke}%
\end{pgfscope}%
\begin{pgfscope}%
\pgfpathrectangle{\pgfqpoint{3.985294in}{1.750000in}}{\pgfqpoint{2.279412in}{2.004545in}}%
\pgfusepath{clip}%
\pgfsetbuttcap%
\pgfsetroundjoin%
\pgfsetlinewidth{0.418049pt}%
\definecolor{currentstroke}{rgb}{0.282656,0.100196,0.422160}%
\pgfsetstrokecolor{currentstroke}%
\pgfsetdash{}{0pt}%
\pgfpathmoveto{\pgfqpoint{4.686529in}{2.726245in}}%
\pgfpathlineto{\pgfqpoint{4.686453in}{2.726259in}}%
\pgfusepath{stroke}%
\end{pgfscope}%
\begin{pgfscope}%
\pgfpathrectangle{\pgfqpoint{3.985294in}{1.750000in}}{\pgfqpoint{2.279412in}{2.004545in}}%
\pgfusepath{clip}%
\pgfsetbuttcap%
\pgfsetroundjoin%
\pgfsetlinewidth{0.417930pt}%
\definecolor{currentstroke}{rgb}{0.282656,0.100196,0.422160}%
\pgfsetstrokecolor{currentstroke}%
\pgfsetdash{}{0pt}%
\pgfpathmoveto{\pgfqpoint{4.686453in}{2.726259in}}%
\pgfpathlineto{\pgfqpoint{4.686454in}{2.726259in}}%
\pgfusepath{stroke}%
\end{pgfscope}%
\begin{pgfscope}%
\pgfpathrectangle{\pgfqpoint{3.985294in}{1.750000in}}{\pgfqpoint{2.279412in}{2.004545in}}%
\pgfusepath{clip}%
\pgfsetbuttcap%
\pgfsetroundjoin%
\pgfsetlinewidth{0.417932pt}%
\definecolor{currentstroke}{rgb}{0.282656,0.100196,0.422160}%
\pgfsetstrokecolor{currentstroke}%
\pgfsetdash{}{0pt}%
\pgfpathmoveto{\pgfqpoint{4.686454in}{2.726259in}}%
\pgfpathlineto{\pgfqpoint{4.686531in}{2.726245in}}%
\pgfusepath{stroke}%
\end{pgfscope}%
\begin{pgfscope}%
\pgfpathrectangle{\pgfqpoint{3.985294in}{1.750000in}}{\pgfqpoint{2.279412in}{2.004545in}}%
\pgfusepath{clip}%
\pgfsetbuttcap%
\pgfsetroundjoin%
\pgfsetlinewidth{0.418053pt}%
\definecolor{currentstroke}{rgb}{0.282656,0.100196,0.422160}%
\pgfsetstrokecolor{currentstroke}%
\pgfsetdash{}{0pt}%
\pgfpathmoveto{\pgfqpoint{4.686531in}{2.726245in}}%
\pgfpathlineto{\pgfqpoint{4.686610in}{2.726231in}}%
\pgfusepath{stroke}%
\end{pgfscope}%
\begin{pgfscope}%
\pgfpathrectangle{\pgfqpoint{3.985294in}{1.750000in}}{\pgfqpoint{2.279412in}{2.004545in}}%
\pgfusepath{clip}%
\pgfsetbuttcap%
\pgfsetroundjoin%
\pgfsetlinewidth{0.418175pt}%
\definecolor{currentstroke}{rgb}{0.282656,0.100196,0.422160}%
\pgfsetstrokecolor{currentstroke}%
\pgfsetdash{}{0pt}%
\pgfpathmoveto{\pgfqpoint{4.686610in}{2.726231in}}%
\pgfpathlineto{\pgfqpoint{4.686607in}{2.726231in}}%
\pgfusepath{stroke}%
\end{pgfscope}%
\begin{pgfscope}%
\pgfpathrectangle{\pgfqpoint{3.985294in}{1.750000in}}{\pgfqpoint{2.279412in}{2.004545in}}%
\pgfusepath{clip}%
\pgfsetbuttcap%
\pgfsetroundjoin%
\pgfsetlinewidth{0.418171pt}%
\definecolor{currentstroke}{rgb}{0.282656,0.100196,0.422160}%
\pgfsetstrokecolor{currentstroke}%
\pgfsetdash{}{0pt}%
\pgfpathmoveto{\pgfqpoint{4.686607in}{2.726231in}}%
\pgfpathlineto{\pgfqpoint{4.686527in}{2.726245in}}%
\pgfusepath{stroke}%
\end{pgfscope}%
\begin{pgfscope}%
\pgfpathrectangle{\pgfqpoint{3.985294in}{1.750000in}}{\pgfqpoint{2.279412in}{2.004545in}}%
\pgfusepath{clip}%
\pgfsetbuttcap%
\pgfsetroundjoin%
\pgfsetlinewidth{0.418047pt}%
\definecolor{currentstroke}{rgb}{0.282656,0.100196,0.422160}%
\pgfsetstrokecolor{currentstroke}%
\pgfsetdash{}{0pt}%
\pgfpathmoveto{\pgfqpoint{4.686527in}{2.726245in}}%
\pgfpathlineto{\pgfqpoint{4.686453in}{2.726259in}}%
\pgfusepath{stroke}%
\end{pgfscope}%
\begin{pgfscope}%
\pgfpathrectangle{\pgfqpoint{3.985294in}{1.750000in}}{\pgfqpoint{2.279412in}{2.004545in}}%
\pgfusepath{clip}%
\pgfsetbuttcap%
\pgfsetroundjoin%
\pgfsetlinewidth{0.417930pt}%
\definecolor{currentstroke}{rgb}{0.282656,0.100196,0.422160}%
\pgfsetstrokecolor{currentstroke}%
\pgfsetdash{}{0pt}%
\pgfpathmoveto{\pgfqpoint{4.686453in}{2.726259in}}%
\pgfpathlineto{\pgfqpoint{4.686456in}{2.726259in}}%
\pgfusepath{stroke}%
\end{pgfscope}%
\begin{pgfscope}%
\pgfpathrectangle{\pgfqpoint{3.985294in}{1.750000in}}{\pgfqpoint{2.279412in}{2.004545in}}%
\pgfusepath{clip}%
\pgfsetbuttcap%
\pgfsetroundjoin%
\pgfsetlinewidth{0.417935pt}%
\definecolor{currentstroke}{rgb}{0.282656,0.100196,0.422160}%
\pgfsetstrokecolor{currentstroke}%
\pgfsetdash{}{0pt}%
\pgfpathmoveto{\pgfqpoint{4.686456in}{2.726259in}}%
\pgfpathlineto{\pgfqpoint{4.686533in}{2.726245in}}%
\pgfusepath{stroke}%
\end{pgfscope}%
\begin{pgfscope}%
\pgfpathrectangle{\pgfqpoint{3.985294in}{1.750000in}}{\pgfqpoint{2.279412in}{2.004545in}}%
\pgfusepath{clip}%
\pgfsetbuttcap%
\pgfsetroundjoin%
\pgfsetlinewidth{0.418056pt}%
\definecolor{currentstroke}{rgb}{0.282656,0.100196,0.422160}%
\pgfsetstrokecolor{currentstroke}%
\pgfsetdash{}{0pt}%
\pgfpathmoveto{\pgfqpoint{4.686533in}{2.726245in}}%
\pgfpathlineto{\pgfqpoint{4.686609in}{2.726231in}}%
\pgfusepath{stroke}%
\end{pgfscope}%
\begin{pgfscope}%
\pgfpathrectangle{\pgfqpoint{3.985294in}{1.750000in}}{\pgfqpoint{2.279412in}{2.004545in}}%
\pgfusepath{clip}%
\pgfsetbuttcap%
\pgfsetroundjoin%
\pgfsetlinewidth{0.418175pt}%
\definecolor{currentstroke}{rgb}{0.282656,0.100196,0.422160}%
\pgfsetstrokecolor{currentstroke}%
\pgfsetdash{}{0pt}%
\pgfpathmoveto{\pgfqpoint{4.686609in}{2.726231in}}%
\pgfpathlineto{\pgfqpoint{4.686605in}{2.726231in}}%
\pgfusepath{stroke}%
\end{pgfscope}%
\begin{pgfscope}%
\pgfpathrectangle{\pgfqpoint{3.985294in}{1.750000in}}{\pgfqpoint{2.279412in}{2.004545in}}%
\pgfusepath{clip}%
\pgfsetbuttcap%
\pgfsetroundjoin%
\pgfsetlinewidth{0.418168pt}%
\definecolor{currentstroke}{rgb}{0.282656,0.100196,0.422160}%
\pgfsetstrokecolor{currentstroke}%
\pgfsetdash{}{0pt}%
\pgfpathmoveto{\pgfqpoint{4.686605in}{2.726231in}}%
\pgfpathlineto{\pgfqpoint{4.686526in}{2.726246in}}%
\pgfusepath{stroke}%
\end{pgfscope}%
\begin{pgfscope}%
\pgfpathrectangle{\pgfqpoint{3.985294in}{1.750000in}}{\pgfqpoint{2.279412in}{2.004545in}}%
\pgfusepath{clip}%
\pgfsetbuttcap%
\pgfsetroundjoin%
\pgfsetlinewidth{0.418044pt}%
\definecolor{currentstroke}{rgb}{0.282656,0.100196,0.422160}%
\pgfsetstrokecolor{currentstroke}%
\pgfsetdash{}{0pt}%
\pgfpathmoveto{\pgfqpoint{4.686526in}{2.726246in}}%
\pgfpathlineto{\pgfqpoint{4.686453in}{2.726259in}}%
\pgfusepath{stroke}%
\end{pgfscope}%
\begin{pgfscope}%
\pgfpathrectangle{\pgfqpoint{3.985294in}{1.750000in}}{\pgfqpoint{2.279412in}{2.004545in}}%
\pgfusepath{clip}%
\pgfsetbuttcap%
\pgfsetroundjoin%
\pgfsetlinewidth{0.417931pt}%
\definecolor{currentstroke}{rgb}{0.282656,0.100196,0.422160}%
\pgfsetstrokecolor{currentstroke}%
\pgfsetdash{}{0pt}%
\pgfpathmoveto{\pgfqpoint{4.686453in}{2.726259in}}%
\pgfpathlineto{\pgfqpoint{4.686458in}{2.726259in}}%
\pgfusepath{stroke}%
\end{pgfscope}%
\begin{pgfscope}%
\pgfpathrectangle{\pgfqpoint{3.985294in}{1.750000in}}{\pgfqpoint{2.279412in}{2.004545in}}%
\pgfusepath{clip}%
\pgfsetbuttcap%
\pgfsetroundjoin%
\pgfsetlinewidth{0.417938pt}%
\definecolor{currentstroke}{rgb}{0.282656,0.100196,0.422160}%
\pgfsetstrokecolor{currentstroke}%
\pgfsetdash{}{0pt}%
\pgfpathmoveto{\pgfqpoint{4.686458in}{2.726259in}}%
\pgfpathlineto{\pgfqpoint{4.686535in}{2.726245in}}%
\pgfusepath{stroke}%
\end{pgfscope}%
\begin{pgfscope}%
\pgfpathrectangle{\pgfqpoint{3.985294in}{1.750000in}}{\pgfqpoint{2.279412in}{2.004545in}}%
\pgfusepath{clip}%
\pgfsetbuttcap%
\pgfsetroundjoin%
\pgfsetlinewidth{0.418059pt}%
\definecolor{currentstroke}{rgb}{0.282656,0.100196,0.422160}%
\pgfsetstrokecolor{currentstroke}%
\pgfsetdash{}{0pt}%
\pgfpathmoveto{\pgfqpoint{4.686535in}{2.726245in}}%
\pgfpathlineto{\pgfqpoint{4.686609in}{2.726231in}}%
\pgfusepath{stroke}%
\end{pgfscope}%
\begin{pgfscope}%
\pgfpathrectangle{\pgfqpoint{3.985294in}{1.750000in}}{\pgfqpoint{2.279412in}{2.004545in}}%
\pgfusepath{clip}%
\pgfsetbuttcap%
\pgfsetroundjoin%
\pgfsetlinewidth{0.418174pt}%
\definecolor{currentstroke}{rgb}{0.282656,0.100196,0.422160}%
\pgfsetstrokecolor{currentstroke}%
\pgfsetdash{}{0pt}%
\pgfpathmoveto{\pgfqpoint{4.686609in}{2.726231in}}%
\pgfpathlineto{\pgfqpoint{4.686603in}{2.726232in}}%
\pgfusepath{stroke}%
\end{pgfscope}%
\begin{pgfscope}%
\pgfpathrectangle{\pgfqpoint{3.985294in}{1.750000in}}{\pgfqpoint{2.279412in}{2.004545in}}%
\pgfusepath{clip}%
\pgfsetbuttcap%
\pgfsetroundjoin%
\pgfsetlinewidth{0.418165pt}%
\definecolor{currentstroke}{rgb}{0.282656,0.100196,0.422160}%
\pgfsetstrokecolor{currentstroke}%
\pgfsetdash{}{0pt}%
\pgfpathmoveto{\pgfqpoint{4.686603in}{2.726232in}}%
\pgfpathlineto{\pgfqpoint{4.686524in}{2.726246in}}%
\pgfusepath{stroke}%
\end{pgfscope}%
\begin{pgfscope}%
\pgfpathrectangle{\pgfqpoint{3.985294in}{1.750000in}}{\pgfqpoint{2.279412in}{2.004545in}}%
\pgfusepath{clip}%
\pgfsetbuttcap%
\pgfsetroundjoin%
\pgfsetlinewidth{0.418041pt}%
\definecolor{currentstroke}{rgb}{0.282656,0.100196,0.422160}%
\pgfsetstrokecolor{currentstroke}%
\pgfsetdash{}{0pt}%
\pgfpathmoveto{\pgfqpoint{4.686524in}{2.726246in}}%
\pgfpathlineto{\pgfqpoint{4.686453in}{2.726259in}}%
\pgfusepath{stroke}%
\end{pgfscope}%
\begin{pgfscope}%
\pgfpathrectangle{\pgfqpoint{3.985294in}{1.750000in}}{\pgfqpoint{2.279412in}{2.004545in}}%
\pgfusepath{clip}%
\pgfsetbuttcap%
\pgfsetroundjoin%
\pgfsetlinewidth{0.417931pt}%
\definecolor{currentstroke}{rgb}{0.282656,0.100196,0.422160}%
\pgfsetstrokecolor{currentstroke}%
\pgfsetdash{}{0pt}%
\pgfpathmoveto{\pgfqpoint{4.686453in}{2.726259in}}%
\pgfpathlineto{\pgfqpoint{4.686459in}{2.726258in}}%
\pgfusepath{stroke}%
\end{pgfscope}%
\begin{pgfscope}%
\pgfpathrectangle{\pgfqpoint{3.985294in}{1.750000in}}{\pgfqpoint{2.279412in}{2.004545in}}%
\pgfusepath{clip}%
\pgfsetbuttcap%
\pgfsetroundjoin%
\pgfsetlinewidth{0.417941pt}%
\definecolor{currentstroke}{rgb}{0.282656,0.100196,0.422160}%
\pgfsetstrokecolor{currentstroke}%
\pgfsetdash{}{0pt}%
\pgfpathmoveto{\pgfqpoint{4.686459in}{2.726258in}}%
\pgfpathlineto{\pgfqpoint{4.686536in}{2.726244in}}%
\pgfusepath{stroke}%
\end{pgfscope}%
\begin{pgfscope}%
\pgfpathrectangle{\pgfqpoint{3.985294in}{1.750000in}}{\pgfqpoint{2.279412in}{2.004545in}}%
\pgfusepath{clip}%
\pgfsetbuttcap%
\pgfsetroundjoin%
\pgfsetlinewidth{0.418061pt}%
\definecolor{currentstroke}{rgb}{0.282656,0.100196,0.422160}%
\pgfsetstrokecolor{currentstroke}%
\pgfsetdash{}{0pt}%
\pgfpathmoveto{\pgfqpoint{4.686536in}{2.726244in}}%
\pgfpathlineto{\pgfqpoint{4.686609in}{2.726231in}}%
\pgfusepath{stroke}%
\end{pgfscope}%
\begin{pgfscope}%
\pgfpathrectangle{\pgfqpoint{3.985294in}{1.750000in}}{\pgfqpoint{2.279412in}{2.004545in}}%
\pgfusepath{clip}%
\pgfsetbuttcap%
\pgfsetroundjoin%
\pgfsetlinewidth{0.418174pt}%
\definecolor{currentstroke}{rgb}{0.282656,0.100196,0.422160}%
\pgfsetstrokecolor{currentstroke}%
\pgfsetdash{}{0pt}%
\pgfpathmoveto{\pgfqpoint{4.686609in}{2.726231in}}%
\pgfpathlineto{\pgfqpoint{4.686601in}{2.726232in}}%
\pgfusepath{stroke}%
\end{pgfscope}%
\begin{pgfscope}%
\pgfpathrectangle{\pgfqpoint{3.985294in}{1.750000in}}{\pgfqpoint{2.279412in}{2.004545in}}%
\pgfusepath{clip}%
\pgfsetbuttcap%
\pgfsetroundjoin%
\pgfsetlinewidth{0.418162pt}%
\definecolor{currentstroke}{rgb}{0.282656,0.100196,0.422160}%
\pgfsetstrokecolor{currentstroke}%
\pgfsetdash{}{0pt}%
\pgfpathmoveto{\pgfqpoint{4.686601in}{2.726232in}}%
\pgfpathlineto{\pgfqpoint{4.686522in}{2.726246in}}%
\pgfusepath{stroke}%
\end{pgfscope}%
\begin{pgfscope}%
\pgfpathrectangle{\pgfqpoint{3.985294in}{1.750000in}}{\pgfqpoint{2.279412in}{2.004545in}}%
\pgfusepath{clip}%
\pgfsetbuttcap%
\pgfsetroundjoin%
\pgfsetlinewidth{0.418039pt}%
\definecolor{currentstroke}{rgb}{0.282656,0.100196,0.422160}%
\pgfsetstrokecolor{currentstroke}%
\pgfsetdash{}{0pt}%
\pgfpathmoveto{\pgfqpoint{4.686522in}{2.726246in}}%
\pgfpathlineto{\pgfqpoint{4.686454in}{2.726259in}}%
\pgfusepath{stroke}%
\end{pgfscope}%
\begin{pgfscope}%
\pgfpathrectangle{\pgfqpoint{3.985294in}{1.750000in}}{\pgfqpoint{2.279412in}{2.004545in}}%
\pgfusepath{clip}%
\pgfsetbuttcap%
\pgfsetroundjoin%
\pgfsetlinewidth{0.417931pt}%
\definecolor{currentstroke}{rgb}{0.282656,0.100196,0.422160}%
\pgfsetstrokecolor{currentstroke}%
\pgfsetdash{}{0pt}%
\pgfpathmoveto{\pgfqpoint{4.686454in}{2.726259in}}%
\pgfpathlineto{\pgfqpoint{4.686461in}{2.726258in}}%
\pgfusepath{stroke}%
\end{pgfscope}%
\begin{pgfscope}%
\pgfpathrectangle{\pgfqpoint{3.985294in}{1.750000in}}{\pgfqpoint{2.279412in}{2.004545in}}%
\pgfusepath{clip}%
\pgfsetbuttcap%
\pgfsetroundjoin%
\pgfsetlinewidth{0.417944pt}%
\definecolor{currentstroke}{rgb}{0.282656,0.100196,0.422160}%
\pgfsetstrokecolor{currentstroke}%
\pgfsetdash{}{0pt}%
\pgfpathmoveto{\pgfqpoint{4.686461in}{2.726258in}}%
\pgfpathlineto{\pgfqpoint{4.686538in}{2.726244in}}%
\pgfusepath{stroke}%
\end{pgfscope}%
\begin{pgfscope}%
\pgfpathrectangle{\pgfqpoint{3.985294in}{1.750000in}}{\pgfqpoint{2.279412in}{2.004545in}}%
\pgfusepath{clip}%
\pgfsetbuttcap%
\pgfsetroundjoin%
\pgfsetlinewidth{0.418064pt}%
\definecolor{currentstroke}{rgb}{0.282656,0.100196,0.422160}%
\pgfsetstrokecolor{currentstroke}%
\pgfsetdash{}{0pt}%
\pgfpathmoveto{\pgfqpoint{4.686538in}{2.726244in}}%
\pgfpathlineto{\pgfqpoint{4.686608in}{2.726231in}}%
\pgfusepath{stroke}%
\end{pgfscope}%
\begin{pgfscope}%
\pgfpathrectangle{\pgfqpoint{3.985294in}{1.750000in}}{\pgfqpoint{2.279412in}{2.004545in}}%
\pgfusepath{clip}%
\pgfsetbuttcap%
\pgfsetroundjoin%
\pgfsetlinewidth{0.418173pt}%
\definecolor{currentstroke}{rgb}{0.282656,0.100196,0.422160}%
\pgfsetstrokecolor{currentstroke}%
\pgfsetdash{}{0pt}%
\pgfpathmoveto{\pgfqpoint{4.686608in}{2.726231in}}%
\pgfpathlineto{\pgfqpoint{4.686599in}{2.726232in}}%
\pgfusepath{stroke}%
\end{pgfscope}%
\begin{pgfscope}%
\pgfpathrectangle{\pgfqpoint{3.985294in}{1.750000in}}{\pgfqpoint{2.279412in}{2.004545in}}%
\pgfusepath{clip}%
\pgfsetbuttcap%
\pgfsetroundjoin%
\pgfsetlinewidth{0.418159pt}%
\definecolor{currentstroke}{rgb}{0.282656,0.100196,0.422160}%
\pgfsetstrokecolor{currentstroke}%
\pgfsetdash{}{0pt}%
\pgfpathmoveto{\pgfqpoint{4.686599in}{2.726232in}}%
\pgfpathlineto{\pgfqpoint{4.686521in}{2.726247in}}%
\pgfusepath{stroke}%
\end{pgfscope}%
\begin{pgfscope}%
\pgfpathrectangle{\pgfqpoint{3.985294in}{1.750000in}}{\pgfqpoint{2.279412in}{2.004545in}}%
\pgfusepath{clip}%
\pgfsetbuttcap%
\pgfsetroundjoin%
\pgfsetlinewidth{0.418036pt}%
\definecolor{currentstroke}{rgb}{0.282656,0.100196,0.422160}%
\pgfsetstrokecolor{currentstroke}%
\pgfsetdash{}{0pt}%
\pgfpathmoveto{\pgfqpoint{4.686521in}{2.726247in}}%
\pgfpathlineto{\pgfqpoint{4.686454in}{2.726259in}}%
\pgfusepath{stroke}%
\end{pgfscope}%
\begin{pgfscope}%
\pgfpathrectangle{\pgfqpoint{3.985294in}{1.750000in}}{\pgfqpoint{2.279412in}{2.004545in}}%
\pgfusepath{clip}%
\pgfsetbuttcap%
\pgfsetroundjoin%
\pgfsetlinewidth{0.417932pt}%
\definecolor{currentstroke}{rgb}{0.282656,0.100196,0.422160}%
\pgfsetstrokecolor{currentstroke}%
\pgfsetdash{}{0pt}%
\pgfpathmoveto{\pgfqpoint{4.686454in}{2.726259in}}%
\pgfpathlineto{\pgfqpoint{4.686463in}{2.726258in}}%
\pgfusepath{stroke}%
\end{pgfscope}%
\begin{pgfscope}%
\pgfpathrectangle{\pgfqpoint{3.985294in}{1.750000in}}{\pgfqpoint{2.279412in}{2.004545in}}%
\pgfusepath{clip}%
\pgfsetbuttcap%
\pgfsetroundjoin%
\pgfsetlinewidth{0.417946pt}%
\definecolor{currentstroke}{rgb}{0.282656,0.100196,0.422160}%
\pgfsetstrokecolor{currentstroke}%
\pgfsetdash{}{0pt}%
\pgfpathmoveto{\pgfqpoint{4.686463in}{2.726258in}}%
\pgfpathlineto{\pgfqpoint{4.686540in}{2.726244in}}%
\pgfusepath{stroke}%
\end{pgfscope}%
\begin{pgfscope}%
\pgfpathrectangle{\pgfqpoint{3.985294in}{1.750000in}}{\pgfqpoint{2.279412in}{2.004545in}}%
\pgfusepath{clip}%
\pgfsetbuttcap%
\pgfsetroundjoin%
\pgfsetlinewidth{0.418066pt}%
\definecolor{currentstroke}{rgb}{0.282656,0.100196,0.422160}%
\pgfsetstrokecolor{currentstroke}%
\pgfsetdash{}{0pt}%
\pgfpathmoveto{\pgfqpoint{4.686540in}{2.726244in}}%
\pgfpathlineto{\pgfqpoint{4.686608in}{2.726231in}}%
\pgfusepath{stroke}%
\end{pgfscope}%
\begin{pgfscope}%
\pgfpathrectangle{\pgfqpoint{3.985294in}{1.750000in}}{\pgfqpoint{2.279412in}{2.004545in}}%
\pgfusepath{clip}%
\pgfsetbuttcap%
\pgfsetroundjoin%
\pgfsetlinewidth{0.418173pt}%
\definecolor{currentstroke}{rgb}{0.282656,0.100196,0.422160}%
\pgfsetstrokecolor{currentstroke}%
\pgfsetdash{}{0pt}%
\pgfpathmoveto{\pgfqpoint{4.686608in}{2.726231in}}%
\pgfpathlineto{\pgfqpoint{4.686597in}{2.726233in}}%
\pgfusepath{stroke}%
\end{pgfscope}%
\begin{pgfscope}%
\pgfpathrectangle{\pgfqpoint{3.985294in}{1.750000in}}{\pgfqpoint{2.279412in}{2.004545in}}%
\pgfusepath{clip}%
\pgfsetbuttcap%
\pgfsetroundjoin%
\pgfsetlinewidth{0.418155pt}%
\definecolor{currentstroke}{rgb}{0.282656,0.100196,0.422160}%
\pgfsetstrokecolor{currentstroke}%
\pgfsetdash{}{0pt}%
\pgfpathmoveto{\pgfqpoint{4.686597in}{2.726233in}}%
\pgfpathlineto{\pgfqpoint{4.686519in}{2.726247in}}%
\pgfusepath{stroke}%
\end{pgfscope}%
\begin{pgfscope}%
\pgfpathrectangle{\pgfqpoint{3.985294in}{1.750000in}}{\pgfqpoint{2.279412in}{2.004545in}}%
\pgfusepath{clip}%
\pgfsetbuttcap%
\pgfsetroundjoin%
\pgfsetlinewidth{0.418034pt}%
\definecolor{currentstroke}{rgb}{0.282656,0.100196,0.422160}%
\pgfsetstrokecolor{currentstroke}%
\pgfsetdash{}{0pt}%
\pgfpathmoveto{\pgfqpoint{4.686519in}{2.726247in}}%
\pgfpathlineto{\pgfqpoint{4.686454in}{2.726259in}}%
\pgfusepath{stroke}%
\end{pgfscope}%
\begin{pgfscope}%
\pgfpathrectangle{\pgfqpoint{3.985294in}{1.750000in}}{\pgfqpoint{2.279412in}{2.004545in}}%
\pgfusepath{clip}%
\pgfsetbuttcap%
\pgfsetroundjoin%
\pgfsetlinewidth{0.417932pt}%
\definecolor{currentstroke}{rgb}{0.282656,0.100196,0.422160}%
\pgfsetstrokecolor{currentstroke}%
\pgfsetdash{}{0pt}%
\pgfpathmoveto{\pgfqpoint{4.686454in}{2.726259in}}%
\pgfpathlineto{\pgfqpoint{4.686465in}{2.726257in}}%
\pgfusepath{stroke}%
\end{pgfscope}%
\begin{pgfscope}%
\pgfpathrectangle{\pgfqpoint{3.985294in}{1.750000in}}{\pgfqpoint{2.279412in}{2.004545in}}%
\pgfusepath{clip}%
\pgfsetbuttcap%
\pgfsetroundjoin%
\pgfsetlinewidth{0.417949pt}%
\definecolor{currentstroke}{rgb}{0.282656,0.100196,0.422160}%
\pgfsetstrokecolor{currentstroke}%
\pgfsetdash{}{0pt}%
\pgfpathmoveto{\pgfqpoint{4.686465in}{2.726257in}}%
\pgfpathlineto{\pgfqpoint{4.686541in}{2.726243in}}%
\pgfusepath{stroke}%
\end{pgfscope}%
\begin{pgfscope}%
\pgfpathrectangle{\pgfqpoint{3.985294in}{1.750000in}}{\pgfqpoint{2.279412in}{2.004545in}}%
\pgfusepath{clip}%
\pgfsetbuttcap%
\pgfsetroundjoin%
\pgfsetlinewidth{0.418069pt}%
\definecolor{currentstroke}{rgb}{0.282656,0.100196,0.422160}%
\pgfsetstrokecolor{currentstroke}%
\pgfsetdash{}{0pt}%
\pgfpathmoveto{\pgfqpoint{4.686541in}{2.726243in}}%
\pgfpathlineto{\pgfqpoint{4.686608in}{2.726231in}}%
\pgfusepath{stroke}%
\end{pgfscope}%
\begin{pgfscope}%
\pgfpathrectangle{\pgfqpoint{3.985294in}{1.750000in}}{\pgfqpoint{2.279412in}{2.004545in}}%
\pgfusepath{clip}%
\pgfsetbuttcap%
\pgfsetroundjoin%
\pgfsetlinewidth{0.418172pt}%
\definecolor{currentstroke}{rgb}{0.282656,0.100196,0.422160}%
\pgfsetstrokecolor{currentstroke}%
\pgfsetdash{}{0pt}%
\pgfpathmoveto{\pgfqpoint{4.686608in}{2.726231in}}%
\pgfpathlineto{\pgfqpoint{4.686595in}{2.726233in}}%
\pgfusepath{stroke}%
\end{pgfscope}%
\begin{pgfscope}%
\pgfpathrectangle{\pgfqpoint{3.985294in}{1.750000in}}{\pgfqpoint{2.279412in}{2.004545in}}%
\pgfusepath{clip}%
\pgfsetbuttcap%
\pgfsetroundjoin%
\pgfsetlinewidth{0.418152pt}%
\definecolor{currentstroke}{rgb}{0.282656,0.100196,0.422160}%
\pgfsetstrokecolor{currentstroke}%
\pgfsetdash{}{0pt}%
\pgfpathmoveto{\pgfqpoint{4.686595in}{2.726233in}}%
\pgfpathlineto{\pgfqpoint{4.686518in}{2.726247in}}%
\pgfusepath{stroke}%
\end{pgfscope}%
\begin{pgfscope}%
\pgfpathrectangle{\pgfqpoint{3.985294in}{1.750000in}}{\pgfqpoint{2.279412in}{2.004545in}}%
\pgfusepath{clip}%
\pgfsetbuttcap%
\pgfsetroundjoin%
\pgfsetlinewidth{0.418031pt}%
\definecolor{currentstroke}{rgb}{0.282656,0.100196,0.422160}%
\pgfsetstrokecolor{currentstroke}%
\pgfsetdash{}{0pt}%
\pgfpathmoveto{\pgfqpoint{4.686518in}{2.726247in}}%
\pgfpathlineto{\pgfqpoint{4.686455in}{2.726259in}}%
\pgfusepath{stroke}%
\end{pgfscope}%
\begin{pgfscope}%
\pgfpathrectangle{\pgfqpoint{3.985294in}{1.750000in}}{\pgfqpoint{2.279412in}{2.004545in}}%
\pgfusepath{clip}%
\pgfsetbuttcap%
\pgfsetroundjoin%
\pgfsetlinewidth{0.417933pt}%
\definecolor{currentstroke}{rgb}{0.282656,0.100196,0.422160}%
\pgfsetstrokecolor{currentstroke}%
\pgfsetdash{}{0pt}%
\pgfpathmoveto{\pgfqpoint{4.686455in}{2.726259in}}%
\pgfpathlineto{\pgfqpoint{4.686467in}{2.726257in}}%
\pgfusepath{stroke}%
\end{pgfscope}%
\begin{pgfscope}%
\pgfpathrectangle{\pgfqpoint{3.985294in}{1.750000in}}{\pgfqpoint{2.279412in}{2.004545in}}%
\pgfusepath{clip}%
\pgfsetbuttcap%
\pgfsetroundjoin%
\pgfsetlinewidth{0.417952pt}%
\definecolor{currentstroke}{rgb}{0.282656,0.100196,0.422160}%
\pgfsetstrokecolor{currentstroke}%
\pgfsetdash{}{0pt}%
\pgfpathmoveto{\pgfqpoint{4.686467in}{2.726257in}}%
\pgfpathlineto{\pgfqpoint{4.686543in}{2.726243in}}%
\pgfusepath{stroke}%
\end{pgfscope}%
\begin{pgfscope}%
\pgfpathrectangle{\pgfqpoint{3.985294in}{1.750000in}}{\pgfqpoint{2.279412in}{2.004545in}}%
\pgfusepath{clip}%
\pgfsetbuttcap%
\pgfsetroundjoin%
\pgfsetlinewidth{0.418071pt}%
\definecolor{currentstroke}{rgb}{0.282656,0.100196,0.422160}%
\pgfsetstrokecolor{currentstroke}%
\pgfsetdash{}{0pt}%
\pgfpathmoveto{\pgfqpoint{4.686543in}{2.726243in}}%
\pgfpathlineto{\pgfqpoint{4.686607in}{2.726231in}}%
\pgfusepath{stroke}%
\end{pgfscope}%
\begin{pgfscope}%
\pgfpathrectangle{\pgfqpoint{3.985294in}{1.750000in}}{\pgfqpoint{2.279412in}{2.004545in}}%
\pgfusepath{clip}%
\pgfsetbuttcap%
\pgfsetroundjoin%
\pgfsetlinewidth{0.418171pt}%
\definecolor{currentstroke}{rgb}{0.282656,0.100196,0.422160}%
\pgfsetstrokecolor{currentstroke}%
\pgfsetdash{}{0pt}%
\pgfpathmoveto{\pgfqpoint{4.686607in}{2.726231in}}%
\pgfpathlineto{\pgfqpoint{4.686593in}{2.726233in}}%
\pgfusepath{stroke}%
\end{pgfscope}%
\begin{pgfscope}%
\pgfpathrectangle{\pgfqpoint{3.985294in}{1.750000in}}{\pgfqpoint{2.279412in}{2.004545in}}%
\pgfusepath{clip}%
\pgfsetbuttcap%
\pgfsetroundjoin%
\pgfsetlinewidth{0.418149pt}%
\definecolor{currentstroke}{rgb}{0.282656,0.100196,0.422160}%
\pgfsetstrokecolor{currentstroke}%
\pgfsetdash{}{0pt}%
\pgfpathmoveto{\pgfqpoint{4.686593in}{2.726233in}}%
\pgfpathlineto{\pgfqpoint{4.686516in}{2.726247in}}%
\pgfusepath{stroke}%
\end{pgfscope}%
\begin{pgfscope}%
\pgfpathrectangle{\pgfqpoint{3.985294in}{1.750000in}}{\pgfqpoint{2.279412in}{2.004545in}}%
\pgfusepath{clip}%
\pgfsetbuttcap%
\pgfsetroundjoin%
\pgfsetlinewidth{0.418029pt}%
\definecolor{currentstroke}{rgb}{0.282656,0.100196,0.422160}%
\pgfsetstrokecolor{currentstroke}%
\pgfsetdash{}{0pt}%
\pgfpathmoveto{\pgfqpoint{4.686516in}{2.726247in}}%
\pgfpathlineto{\pgfqpoint{4.686455in}{2.726259in}}%
\pgfusepath{stroke}%
\end{pgfscope}%
\begin{pgfscope}%
\pgfpathrectangle{\pgfqpoint{3.985294in}{1.750000in}}{\pgfqpoint{2.279412in}{2.004545in}}%
\pgfusepath{clip}%
\pgfsetbuttcap%
\pgfsetroundjoin%
\pgfsetlinewidth{0.417934pt}%
\definecolor{currentstroke}{rgb}{0.282656,0.100196,0.422160}%
\pgfsetstrokecolor{currentstroke}%
\pgfsetdash{}{0pt}%
\pgfpathmoveto{\pgfqpoint{4.686455in}{2.726259in}}%
\pgfpathlineto{\pgfqpoint{4.686469in}{2.726257in}}%
\pgfusepath{stroke}%
\end{pgfscope}%
\begin{pgfscope}%
\pgfpathrectangle{\pgfqpoint{3.985294in}{1.750000in}}{\pgfqpoint{2.279412in}{2.004545in}}%
\pgfusepath{clip}%
\pgfsetbuttcap%
\pgfsetroundjoin%
\pgfsetlinewidth{0.417955pt}%
\definecolor{currentstroke}{rgb}{0.282656,0.100196,0.422160}%
\pgfsetstrokecolor{currentstroke}%
\pgfsetdash{}{0pt}%
\pgfpathmoveto{\pgfqpoint{4.686469in}{2.726257in}}%
\pgfpathlineto{\pgfqpoint{4.686544in}{2.726243in}}%
\pgfusepath{stroke}%
\end{pgfscope}%
\begin{pgfscope}%
\pgfpathrectangle{\pgfqpoint{3.985294in}{1.750000in}}{\pgfqpoint{2.279412in}{2.004545in}}%
\pgfusepath{clip}%
\pgfsetbuttcap%
\pgfsetroundjoin%
\pgfsetlinewidth{0.418073pt}%
\definecolor{currentstroke}{rgb}{0.282656,0.100196,0.422160}%
\pgfsetstrokecolor{currentstroke}%
\pgfsetdash{}{0pt}%
\pgfpathmoveto{\pgfqpoint{4.686544in}{2.726243in}}%
\pgfpathlineto{\pgfqpoint{4.686607in}{2.726231in}}%
\pgfusepath{stroke}%
\end{pgfscope}%
\begin{pgfscope}%
\pgfpathrectangle{\pgfqpoint{3.985294in}{1.750000in}}{\pgfqpoint{2.279412in}{2.004545in}}%
\pgfusepath{clip}%
\pgfsetbuttcap%
\pgfsetroundjoin%
\pgfsetlinewidth{0.418171pt}%
\definecolor{currentstroke}{rgb}{0.282656,0.100196,0.422160}%
\pgfsetstrokecolor{currentstroke}%
\pgfsetdash{}{0pt}%
\pgfpathmoveto{\pgfqpoint{4.686607in}{2.726231in}}%
\pgfpathlineto{\pgfqpoint{4.686591in}{2.726234in}}%
\pgfusepath{stroke}%
\end{pgfscope}%
\begin{pgfscope}%
\pgfpathrectangle{\pgfqpoint{3.985294in}{1.750000in}}{\pgfqpoint{2.279412in}{2.004545in}}%
\pgfusepath{clip}%
\pgfsetbuttcap%
\pgfsetroundjoin%
\pgfsetlinewidth{0.418146pt}%
\definecolor{currentstroke}{rgb}{0.282656,0.100196,0.422160}%
\pgfsetstrokecolor{currentstroke}%
\pgfsetdash{}{0pt}%
\pgfpathmoveto{\pgfqpoint{4.686591in}{2.726234in}}%
\pgfpathlineto{\pgfqpoint{4.686515in}{2.726248in}}%
\pgfusepath{stroke}%
\end{pgfscope}%
\begin{pgfscope}%
\pgfpathrectangle{\pgfqpoint{3.985294in}{1.750000in}}{\pgfqpoint{2.279412in}{2.004545in}}%
\pgfusepath{clip}%
\pgfsetbuttcap%
\pgfsetroundjoin%
\pgfsetlinewidth{0.418027pt}%
\definecolor{currentstroke}{rgb}{0.282656,0.100196,0.422160}%
\pgfsetstrokecolor{currentstroke}%
\pgfsetdash{}{0pt}%
\pgfpathmoveto{\pgfqpoint{4.686515in}{2.726248in}}%
\pgfpathlineto{\pgfqpoint{4.686456in}{2.726259in}}%
\pgfusepath{stroke}%
\end{pgfscope}%
\begin{pgfscope}%
\pgfpathrectangle{\pgfqpoint{3.985294in}{1.750000in}}{\pgfqpoint{2.279412in}{2.004545in}}%
\pgfusepath{clip}%
\pgfsetbuttcap%
\pgfsetroundjoin%
\pgfsetlinewidth{0.417934pt}%
\definecolor{currentstroke}{rgb}{0.282656,0.100196,0.422160}%
\pgfsetstrokecolor{currentstroke}%
\pgfsetdash{}{0pt}%
\pgfpathmoveto{\pgfqpoint{4.686456in}{2.726259in}}%
\pgfpathlineto{\pgfqpoint{4.686470in}{2.726256in}}%
\pgfusepath{stroke}%
\end{pgfscope}%
\begin{pgfscope}%
\pgfpathrectangle{\pgfqpoint{3.985294in}{1.750000in}}{\pgfqpoint{2.279412in}{2.004545in}}%
\pgfusepath{clip}%
\pgfsetbuttcap%
\pgfsetroundjoin%
\pgfsetlinewidth{0.417958pt}%
\definecolor{currentstroke}{rgb}{0.282656,0.100196,0.422160}%
\pgfsetstrokecolor{currentstroke}%
\pgfsetdash{}{0pt}%
\pgfpathmoveto{\pgfqpoint{4.686470in}{2.726256in}}%
\pgfpathlineto{\pgfqpoint{4.686545in}{2.726243in}}%
\pgfusepath{stroke}%
\end{pgfscope}%
\begin{pgfscope}%
\pgfpathrectangle{\pgfqpoint{3.985294in}{1.750000in}}{\pgfqpoint{2.279412in}{2.004545in}}%
\pgfusepath{clip}%
\pgfsetbuttcap%
\pgfsetroundjoin%
\pgfsetlinewidth{0.418075pt}%
\definecolor{currentstroke}{rgb}{0.282656,0.100196,0.422160}%
\pgfsetstrokecolor{currentstroke}%
\pgfsetdash{}{0pt}%
\pgfpathmoveto{\pgfqpoint{4.686545in}{2.726243in}}%
\pgfpathlineto{\pgfqpoint{4.686606in}{2.726231in}}%
\pgfusepath{stroke}%
\end{pgfscope}%
\begin{pgfscope}%
\pgfpathrectangle{\pgfqpoint{3.985294in}{1.750000in}}{\pgfqpoint{2.279412in}{2.004545in}}%
\pgfusepath{clip}%
\pgfsetbuttcap%
\pgfsetroundjoin%
\pgfsetlinewidth{0.418170pt}%
\definecolor{currentstroke}{rgb}{0.282656,0.100196,0.422160}%
\pgfsetstrokecolor{currentstroke}%
\pgfsetdash{}{0pt}%
\pgfpathmoveto{\pgfqpoint{4.686606in}{2.726231in}}%
\pgfpathlineto{\pgfqpoint{4.686589in}{2.726234in}}%
\pgfusepath{stroke}%
\end{pgfscope}%
\begin{pgfscope}%
\pgfpathrectangle{\pgfqpoint{3.985294in}{1.750000in}}{\pgfqpoint{2.279412in}{2.004545in}}%
\pgfusepath{clip}%
\pgfsetbuttcap%
\pgfsetroundjoin%
\pgfsetlinewidth{0.418143pt}%
\definecolor{currentstroke}{rgb}{0.282656,0.100196,0.422160}%
\pgfsetstrokecolor{currentstroke}%
\pgfsetdash{}{0pt}%
\pgfpathmoveto{\pgfqpoint{4.686589in}{2.726234in}}%
\pgfpathlineto{\pgfqpoint{4.686513in}{2.726248in}}%
\pgfusepath{stroke}%
\end{pgfscope}%
\begin{pgfscope}%
\pgfpathrectangle{\pgfqpoint{3.985294in}{1.750000in}}{\pgfqpoint{2.279412in}{2.004545in}}%
\pgfusepath{clip}%
\pgfsetbuttcap%
\pgfsetroundjoin%
\pgfsetlinewidth{0.418025pt}%
\definecolor{currentstroke}{rgb}{0.282656,0.100196,0.422160}%
\pgfsetstrokecolor{currentstroke}%
\pgfsetdash{}{0pt}%
\pgfpathmoveto{\pgfqpoint{4.686513in}{2.726248in}}%
\pgfpathlineto{\pgfqpoint{4.686456in}{2.726259in}}%
\pgfusepath{stroke}%
\end{pgfscope}%
\begin{pgfscope}%
\pgfpathrectangle{\pgfqpoint{3.985294in}{1.750000in}}{\pgfqpoint{2.279412in}{2.004545in}}%
\pgfusepath{clip}%
\pgfsetbuttcap%
\pgfsetroundjoin%
\pgfsetlinewidth{0.417935pt}%
\definecolor{currentstroke}{rgb}{0.282656,0.100196,0.422160}%
\pgfsetstrokecolor{currentstroke}%
\pgfsetdash{}{0pt}%
\pgfpathmoveto{\pgfqpoint{4.686456in}{2.726259in}}%
\pgfpathlineto{\pgfqpoint{4.686472in}{2.726256in}}%
\pgfusepath{stroke}%
\end{pgfscope}%
\begin{pgfscope}%
\pgfpathrectangle{\pgfqpoint{3.985294in}{1.750000in}}{\pgfqpoint{2.279412in}{2.004545in}}%
\pgfusepath{clip}%
\pgfsetbuttcap%
\pgfsetroundjoin%
\pgfsetlinewidth{0.417961pt}%
\definecolor{currentstroke}{rgb}{0.282656,0.100196,0.422160}%
\pgfsetstrokecolor{currentstroke}%
\pgfsetdash{}{0pt}%
\pgfpathmoveto{\pgfqpoint{4.686472in}{2.726256in}}%
\pgfpathlineto{\pgfqpoint{4.686547in}{2.726242in}}%
\pgfusepath{stroke}%
\end{pgfscope}%
\begin{pgfscope}%
\pgfpathrectangle{\pgfqpoint{3.985294in}{1.750000in}}{\pgfqpoint{2.279412in}{2.004545in}}%
\pgfusepath{clip}%
\pgfsetbuttcap%
\pgfsetroundjoin%
\pgfsetlinewidth{0.418077pt}%
\definecolor{currentstroke}{rgb}{0.282656,0.100196,0.422160}%
\pgfsetstrokecolor{currentstroke}%
\pgfsetdash{}{0pt}%
\pgfpathmoveto{\pgfqpoint{4.686547in}{2.726242in}}%
\pgfpathlineto{\pgfqpoint{4.686606in}{2.726231in}}%
\pgfusepath{stroke}%
\end{pgfscope}%
\begin{pgfscope}%
\pgfpathrectangle{\pgfqpoint{3.985294in}{1.750000in}}{\pgfqpoint{2.279412in}{2.004545in}}%
\pgfusepath{clip}%
\pgfsetbuttcap%
\pgfsetroundjoin%
\pgfsetlinewidth{0.418169pt}%
\definecolor{currentstroke}{rgb}{0.282656,0.100196,0.422160}%
\pgfsetstrokecolor{currentstroke}%
\pgfsetdash{}{0pt}%
\pgfpathmoveto{\pgfqpoint{4.686606in}{2.726231in}}%
\pgfpathlineto{\pgfqpoint{4.686587in}{2.726234in}}%
\pgfusepath{stroke}%
\end{pgfscope}%
\begin{pgfscope}%
\pgfpathrectangle{\pgfqpoint{3.985294in}{1.750000in}}{\pgfqpoint{2.279412in}{2.004545in}}%
\pgfusepath{clip}%
\pgfsetbuttcap%
\pgfsetroundjoin%
\pgfsetlinewidth{0.418140pt}%
\definecolor{currentstroke}{rgb}{0.282656,0.100196,0.422160}%
\pgfsetstrokecolor{currentstroke}%
\pgfsetdash{}{0pt}%
\pgfpathmoveto{\pgfqpoint{4.686587in}{2.726234in}}%
\pgfpathlineto{\pgfqpoint{4.686512in}{2.726248in}}%
\pgfusepath{stroke}%
\end{pgfscope}%
\begin{pgfscope}%
\pgfpathrectangle{\pgfqpoint{3.985294in}{1.750000in}}{\pgfqpoint{2.279412in}{2.004545in}}%
\pgfusepath{clip}%
\pgfsetbuttcap%
\pgfsetroundjoin%
\pgfsetlinewidth{0.418023pt}%
\definecolor{currentstroke}{rgb}{0.282656,0.100196,0.422160}%
\pgfsetstrokecolor{currentstroke}%
\pgfsetdash{}{0pt}%
\pgfpathmoveto{\pgfqpoint{4.686512in}{2.726248in}}%
\pgfpathlineto{\pgfqpoint{4.686457in}{2.726259in}}%
\pgfusepath{stroke}%
\end{pgfscope}%
\begin{pgfscope}%
\pgfpathrectangle{\pgfqpoint{3.985294in}{1.750000in}}{\pgfqpoint{2.279412in}{2.004545in}}%
\pgfusepath{clip}%
\pgfsetbuttcap%
\pgfsetroundjoin%
\pgfsetlinewidth{0.417936pt}%
\definecolor{currentstroke}{rgb}{0.282656,0.100196,0.422160}%
\pgfsetstrokecolor{currentstroke}%
\pgfsetdash{}{0pt}%
\pgfpathmoveto{\pgfqpoint{4.686457in}{2.726259in}}%
\pgfpathlineto{\pgfqpoint{4.686474in}{2.726256in}}%
\pgfusepath{stroke}%
\end{pgfscope}%
\begin{pgfscope}%
\pgfpathrectangle{\pgfqpoint{3.985294in}{1.750000in}}{\pgfqpoint{2.279412in}{2.004545in}}%
\pgfusepath{clip}%
\pgfsetbuttcap%
\pgfsetroundjoin%
\pgfsetlinewidth{0.417963pt}%
\definecolor{currentstroke}{rgb}{0.282656,0.100196,0.422160}%
\pgfsetstrokecolor{currentstroke}%
\pgfsetdash{}{0pt}%
\pgfpathmoveto{\pgfqpoint{4.686474in}{2.726256in}}%
\pgfpathlineto{\pgfqpoint{4.686548in}{2.726242in}}%
\pgfusepath{stroke}%
\end{pgfscope}%
\begin{pgfscope}%
\pgfpathrectangle{\pgfqpoint{3.985294in}{1.750000in}}{\pgfqpoint{2.279412in}{2.004545in}}%
\pgfusepath{clip}%
\pgfsetbuttcap%
\pgfsetroundjoin%
\pgfsetlinewidth{0.418079pt}%
\definecolor{currentstroke}{rgb}{0.282656,0.100196,0.422160}%
\pgfsetstrokecolor{currentstroke}%
\pgfsetdash{}{0pt}%
\pgfpathmoveto{\pgfqpoint{4.686548in}{2.726242in}}%
\pgfpathlineto{\pgfqpoint{4.686605in}{2.726231in}}%
\pgfusepath{stroke}%
\end{pgfscope}%
\begin{pgfscope}%
\pgfpathrectangle{\pgfqpoint{3.985294in}{1.750000in}}{\pgfqpoint{2.279412in}{2.004545in}}%
\pgfusepath{clip}%
\pgfsetbuttcap%
\pgfsetroundjoin%
\pgfsetlinewidth{0.418168pt}%
\definecolor{currentstroke}{rgb}{0.282656,0.100196,0.422160}%
\pgfsetstrokecolor{currentstroke}%
\pgfsetdash{}{0pt}%
\pgfpathmoveto{\pgfqpoint{4.686605in}{2.726231in}}%
\pgfpathlineto{\pgfqpoint{4.686586in}{2.726235in}}%
\pgfusepath{stroke}%
\end{pgfscope}%
\begin{pgfscope}%
\pgfpathrectangle{\pgfqpoint{3.985294in}{1.750000in}}{\pgfqpoint{2.279412in}{2.004545in}}%
\pgfusepath{clip}%
\pgfsetbuttcap%
\pgfsetroundjoin%
\pgfsetlinewidth{0.418138pt}%
\definecolor{currentstroke}{rgb}{0.282656,0.100196,0.422160}%
\pgfsetstrokecolor{currentstroke}%
\pgfsetdash{}{0pt}%
\pgfpathmoveto{\pgfqpoint{4.686586in}{2.726235in}}%
\pgfpathlineto{\pgfqpoint{4.686511in}{2.726249in}}%
\pgfusepath{stroke}%
\end{pgfscope}%
\begin{pgfscope}%
\pgfpathrectangle{\pgfqpoint{3.985294in}{1.750000in}}{\pgfqpoint{2.279412in}{2.004545in}}%
\pgfusepath{clip}%
\pgfsetbuttcap%
\pgfsetroundjoin%
\pgfsetlinewidth{0.418021pt}%
\definecolor{currentstroke}{rgb}{0.282656,0.100196,0.422160}%
\pgfsetstrokecolor{currentstroke}%
\pgfsetdash{}{0pt}%
\pgfpathmoveto{\pgfqpoint{4.686511in}{2.726249in}}%
\pgfpathlineto{\pgfqpoint{4.686457in}{2.726259in}}%
\pgfusepath{stroke}%
\end{pgfscope}%
\begin{pgfscope}%
\pgfpathrectangle{\pgfqpoint{3.985294in}{1.750000in}}{\pgfqpoint{2.279412in}{2.004545in}}%
\pgfusepath{clip}%
\pgfsetbuttcap%
\pgfsetroundjoin%
\pgfsetlinewidth{0.417937pt}%
\definecolor{currentstroke}{rgb}{0.282656,0.100196,0.422160}%
\pgfsetstrokecolor{currentstroke}%
\pgfsetdash{}{0pt}%
\pgfpathmoveto{\pgfqpoint{4.686457in}{2.726259in}}%
\pgfpathlineto{\pgfqpoint{4.686476in}{2.726255in}}%
\pgfusepath{stroke}%
\end{pgfscope}%
\begin{pgfscope}%
\pgfpathrectangle{\pgfqpoint{3.985294in}{1.750000in}}{\pgfqpoint{2.279412in}{2.004545in}}%
\pgfusepath{clip}%
\pgfsetbuttcap%
\pgfsetroundjoin%
\pgfsetlinewidth{0.417966pt}%
\definecolor{currentstroke}{rgb}{0.282656,0.100196,0.422160}%
\pgfsetstrokecolor{currentstroke}%
\pgfsetdash{}{0pt}%
\pgfpathmoveto{\pgfqpoint{4.686476in}{2.726255in}}%
\pgfpathlineto{\pgfqpoint{4.686549in}{2.726242in}}%
\pgfusepath{stroke}%
\end{pgfscope}%
\begin{pgfscope}%
\pgfpathrectangle{\pgfqpoint{3.985294in}{1.750000in}}{\pgfqpoint{2.279412in}{2.004545in}}%
\pgfusepath{clip}%
\pgfsetbuttcap%
\pgfsetroundjoin%
\pgfsetlinewidth{0.418081pt}%
\definecolor{currentstroke}{rgb}{0.282656,0.100196,0.422160}%
\pgfsetstrokecolor{currentstroke}%
\pgfsetdash{}{0pt}%
\pgfpathmoveto{\pgfqpoint{4.686549in}{2.726242in}}%
\pgfpathlineto{\pgfqpoint{4.686605in}{2.726231in}}%
\pgfusepath{stroke}%
\end{pgfscope}%
\begin{pgfscope}%
\pgfpathrectangle{\pgfqpoint{3.985294in}{1.750000in}}{\pgfqpoint{2.279412in}{2.004545in}}%
\pgfusepath{clip}%
\pgfsetbuttcap%
\pgfsetroundjoin%
\pgfsetlinewidth{0.418167pt}%
\definecolor{currentstroke}{rgb}{0.282656,0.100196,0.422160}%
\pgfsetstrokecolor{currentstroke}%
\pgfsetdash{}{0pt}%
\pgfpathmoveto{\pgfqpoint{4.686605in}{2.726231in}}%
\pgfpathlineto{\pgfqpoint{4.686584in}{2.726235in}}%
\pgfusepath{stroke}%
\end{pgfscope}%
\begin{pgfscope}%
\pgfpathrectangle{\pgfqpoint{3.985294in}{1.750000in}}{\pgfqpoint{2.279412in}{2.004545in}}%
\pgfusepath{clip}%
\pgfsetbuttcap%
\pgfsetroundjoin%
\pgfsetlinewidth{0.418135pt}%
\definecolor{currentstroke}{rgb}{0.282656,0.100196,0.422160}%
\pgfsetstrokecolor{currentstroke}%
\pgfsetdash{}{0pt}%
\pgfpathmoveto{\pgfqpoint{4.686584in}{2.726235in}}%
\pgfpathlineto{\pgfqpoint{4.686510in}{2.726249in}}%
\pgfusepath{stroke}%
\end{pgfscope}%
\begin{pgfscope}%
\pgfpathrectangle{\pgfqpoint{3.985294in}{1.750000in}}{\pgfqpoint{2.279412in}{2.004545in}}%
\pgfusepath{clip}%
\pgfsetbuttcap%
\pgfsetroundjoin%
\pgfsetlinewidth{0.418019pt}%
\definecolor{currentstroke}{rgb}{0.282656,0.100196,0.422160}%
\pgfsetstrokecolor{currentstroke}%
\pgfsetdash{}{0pt}%
\pgfpathmoveto{\pgfqpoint{4.686510in}{2.726249in}}%
\pgfpathlineto{\pgfqpoint{4.686458in}{2.726259in}}%
\pgfusepath{stroke}%
\end{pgfscope}%
\begin{pgfscope}%
\pgfpathrectangle{\pgfqpoint{3.985294in}{1.750000in}}{\pgfqpoint{2.279412in}{2.004545in}}%
\pgfusepath{clip}%
\pgfsetbuttcap%
\pgfsetroundjoin%
\pgfsetlinewidth{0.417938pt}%
\definecolor{currentstroke}{rgb}{0.282656,0.100196,0.422160}%
\pgfsetstrokecolor{currentstroke}%
\pgfsetdash{}{0pt}%
\pgfpathmoveto{\pgfqpoint{4.686458in}{2.726259in}}%
\pgfpathlineto{\pgfqpoint{4.686478in}{2.726255in}}%
\pgfusepath{stroke}%
\end{pgfscope}%
\begin{pgfscope}%
\pgfpathrectangle{\pgfqpoint{3.985294in}{1.750000in}}{\pgfqpoint{2.279412in}{2.004545in}}%
\pgfusepath{clip}%
\pgfsetbuttcap%
\pgfsetroundjoin%
\pgfsetlinewidth{0.417969pt}%
\definecolor{currentstroke}{rgb}{0.282656,0.100196,0.422160}%
\pgfsetstrokecolor{currentstroke}%
\pgfsetdash{}{0pt}%
\pgfpathmoveto{\pgfqpoint{4.686478in}{2.726255in}}%
\pgfpathlineto{\pgfqpoint{4.686551in}{2.726242in}}%
\pgfusepath{stroke}%
\end{pgfscope}%
\begin{pgfscope}%
\pgfpathrectangle{\pgfqpoint{3.985294in}{1.750000in}}{\pgfqpoint{2.279412in}{2.004545in}}%
\pgfusepath{clip}%
\pgfsetbuttcap%
\pgfsetroundjoin%
\pgfsetlinewidth{0.418083pt}%
\definecolor{currentstroke}{rgb}{0.282656,0.100196,0.422160}%
\pgfsetstrokecolor{currentstroke}%
\pgfsetdash{}{0pt}%
\pgfpathmoveto{\pgfqpoint{4.686551in}{2.726242in}}%
\pgfpathlineto{\pgfqpoint{4.686604in}{2.726232in}}%
\pgfusepath{stroke}%
\end{pgfscope}%
\begin{pgfscope}%
\pgfpathrectangle{\pgfqpoint{3.985294in}{1.750000in}}{\pgfqpoint{2.279412in}{2.004545in}}%
\pgfusepath{clip}%
\pgfsetbuttcap%
\pgfsetroundjoin%
\pgfsetlinewidth{0.418166pt}%
\definecolor{currentstroke}{rgb}{0.282656,0.100196,0.422160}%
\pgfsetstrokecolor{currentstroke}%
\pgfsetdash{}{0pt}%
\pgfpathmoveto{\pgfqpoint{4.686604in}{2.726232in}}%
\pgfpathlineto{\pgfqpoint{4.686582in}{2.726235in}}%
\pgfusepath{stroke}%
\end{pgfscope}%
\begin{pgfscope}%
\pgfpathrectangle{\pgfqpoint{3.985294in}{1.750000in}}{\pgfqpoint{2.279412in}{2.004545in}}%
\pgfusepath{clip}%
\pgfsetbuttcap%
\pgfsetroundjoin%
\pgfsetlinewidth{0.418132pt}%
\definecolor{currentstroke}{rgb}{0.282656,0.100196,0.422160}%
\pgfsetstrokecolor{currentstroke}%
\pgfsetdash{}{0pt}%
\pgfpathmoveto{\pgfqpoint{4.686582in}{2.726235in}}%
\pgfpathlineto{\pgfqpoint{4.686508in}{2.726249in}}%
\pgfusepath{stroke}%
\end{pgfscope}%
\begin{pgfscope}%
\pgfpathrectangle{\pgfqpoint{3.985294in}{1.750000in}}{\pgfqpoint{2.279412in}{2.004545in}}%
\pgfusepath{clip}%
\pgfsetbuttcap%
\pgfsetroundjoin%
\pgfsetlinewidth{0.418017pt}%
\definecolor{currentstroke}{rgb}{0.282656,0.100196,0.422160}%
\pgfsetstrokecolor{currentstroke}%
\pgfsetdash{}{0pt}%
\pgfpathmoveto{\pgfqpoint{4.686508in}{2.726249in}}%
\pgfpathlineto{\pgfqpoint{4.686458in}{2.726258in}}%
\pgfusepath{stroke}%
\end{pgfscope}%
\begin{pgfscope}%
\pgfpathrectangle{\pgfqpoint{3.985294in}{1.750000in}}{\pgfqpoint{2.279412in}{2.004545in}}%
\pgfusepath{clip}%
\pgfsetbuttcap%
\pgfsetroundjoin%
\pgfsetlinewidth{0.417938pt}%
\definecolor{currentstroke}{rgb}{0.282656,0.100196,0.422160}%
\pgfsetstrokecolor{currentstroke}%
\pgfsetdash{}{0pt}%
\pgfpathmoveto{\pgfqpoint{4.686458in}{2.726258in}}%
\pgfpathlineto{\pgfqpoint{4.686479in}{2.726255in}}%
\pgfusepath{stroke}%
\end{pgfscope}%
\begin{pgfscope}%
\pgfpathrectangle{\pgfqpoint{3.985294in}{1.750000in}}{\pgfqpoint{2.279412in}{2.004545in}}%
\pgfusepath{clip}%
\pgfsetbuttcap%
\pgfsetroundjoin%
\pgfsetlinewidth{0.417972pt}%
\definecolor{currentstroke}{rgb}{0.282656,0.100196,0.422160}%
\pgfsetstrokecolor{currentstroke}%
\pgfsetdash{}{0pt}%
\pgfpathmoveto{\pgfqpoint{4.686479in}{2.726255in}}%
\pgfpathlineto{\pgfqpoint{4.686552in}{2.726241in}}%
\pgfusepath{stroke}%
\end{pgfscope}%
\begin{pgfscope}%
\pgfpathrectangle{\pgfqpoint{3.985294in}{1.750000in}}{\pgfqpoint{2.279412in}{2.004545in}}%
\pgfusepath{clip}%
\pgfsetbuttcap%
\pgfsetroundjoin%
\pgfsetlinewidth{0.418085pt}%
\definecolor{currentstroke}{rgb}{0.282656,0.100196,0.422160}%
\pgfsetstrokecolor{currentstroke}%
\pgfsetdash{}{0pt}%
\pgfpathmoveto{\pgfqpoint{4.686552in}{2.726241in}}%
\pgfpathlineto{\pgfqpoint{4.686603in}{2.726232in}}%
\pgfusepath{stroke}%
\end{pgfscope}%
\begin{pgfscope}%
\pgfpathrectangle{\pgfqpoint{3.985294in}{1.750000in}}{\pgfqpoint{2.279412in}{2.004545in}}%
\pgfusepath{clip}%
\pgfsetbuttcap%
\pgfsetroundjoin%
\pgfsetlinewidth{0.418165pt}%
\definecolor{currentstroke}{rgb}{0.282656,0.100196,0.422160}%
\pgfsetstrokecolor{currentstroke}%
\pgfsetdash{}{0pt}%
\pgfpathmoveto{\pgfqpoint{4.686603in}{2.726232in}}%
\pgfpathlineto{\pgfqpoint{4.686580in}{2.726236in}}%
\pgfusepath{stroke}%
\end{pgfscope}%
\begin{pgfscope}%
\pgfpathrectangle{\pgfqpoint{3.985294in}{1.750000in}}{\pgfqpoint{2.279412in}{2.004545in}}%
\pgfusepath{clip}%
\pgfsetbuttcap%
\pgfsetroundjoin%
\pgfsetlinewidth{0.311077pt}%
\definecolor{currentstroke}{rgb}{0.268510,0.009605,0.335427}%
\pgfsetstrokecolor{currentstroke}%
\pgfsetdash{}{0pt}%
\pgfpathmoveto{\pgfqpoint{5.945670in}{2.842486in}}%
\pgfpathlineto{\pgfqpoint{5.927266in}{2.843711in}}%
\pgfusepath{stroke}%
\end{pgfscope}%
\begin{pgfscope}%
\pgfpathrectangle{\pgfqpoint{3.985294in}{1.750000in}}{\pgfqpoint{2.279412in}{2.004545in}}%
\pgfusepath{clip}%
\pgfsetbuttcap%
\pgfsetroundjoin%
\pgfsetlinewidth{0.317529pt}%
\definecolor{currentstroke}{rgb}{0.269944,0.014625,0.341379}%
\pgfsetstrokecolor{currentstroke}%
\pgfsetdash{}{0pt}%
\pgfpathmoveto{\pgfqpoint{5.927266in}{2.843711in}}%
\pgfpathlineto{\pgfqpoint{5.927266in}{2.843711in}}%
\pgfusepath{stroke}%
\end{pgfscope}%
\begin{pgfscope}%
\pgfpathrectangle{\pgfqpoint{3.985294in}{1.750000in}}{\pgfqpoint{2.279412in}{2.004545in}}%
\pgfusepath{clip}%
\pgfsetbuttcap%
\pgfsetroundjoin%
\pgfsetlinewidth{0.317529pt}%
\definecolor{currentstroke}{rgb}{0.269944,0.014625,0.341379}%
\pgfsetstrokecolor{currentstroke}%
\pgfsetdash{}{0pt}%
\pgfpathmoveto{\pgfqpoint{5.927266in}{2.843711in}}%
\pgfpathlineto{\pgfqpoint{5.927266in}{2.843711in}}%
\pgfusepath{stroke}%
\end{pgfscope}%
\begin{pgfscope}%
\pgfpathrectangle{\pgfqpoint{3.985294in}{1.750000in}}{\pgfqpoint{2.279412in}{2.004545in}}%
\pgfusepath{clip}%
\pgfsetbuttcap%
\pgfsetroundjoin%
\pgfsetlinewidth{0.317529pt}%
\definecolor{currentstroke}{rgb}{0.269944,0.014625,0.341379}%
\pgfsetstrokecolor{currentstroke}%
\pgfsetdash{}{0pt}%
\pgfpathmoveto{\pgfqpoint{5.927266in}{2.843711in}}%
\pgfpathlineto{\pgfqpoint{5.903828in}{2.843715in}}%
\pgfusepath{stroke}%
\end{pgfscope}%
\begin{pgfscope}%
\pgfpathrectangle{\pgfqpoint{3.985294in}{1.750000in}}{\pgfqpoint{2.279412in}{2.004545in}}%
\pgfusepath{clip}%
\pgfsetbuttcap%
\pgfsetroundjoin%
\pgfsetlinewidth{0.320596pt}%
\definecolor{currentstroke}{rgb}{0.269944,0.014625,0.341379}%
\pgfsetstrokecolor{currentstroke}%
\pgfsetdash{}{0pt}%
\pgfpathmoveto{\pgfqpoint{5.903828in}{2.843715in}}%
\pgfpathlineto{\pgfqpoint{5.877253in}{2.843105in}}%
\pgfusepath{stroke}%
\end{pgfscope}%
\begin{pgfscope}%
\pgfpathrectangle{\pgfqpoint{3.985294in}{1.750000in}}{\pgfqpoint{2.279412in}{2.004545in}}%
\pgfusepath{clip}%
\pgfsetbuttcap%
\pgfsetroundjoin%
\pgfsetlinewidth{0.308808pt}%
\definecolor{currentstroke}{rgb}{0.268510,0.009605,0.335427}%
\pgfsetstrokecolor{currentstroke}%
\pgfsetdash{}{0pt}%
\pgfpathmoveto{\pgfqpoint{5.877253in}{2.843105in}}%
\pgfpathlineto{\pgfqpoint{5.827603in}{2.843653in}}%
\pgfusepath{stroke}%
\end{pgfscope}%
\begin{pgfscope}%
\pgfpathrectangle{\pgfqpoint{3.985294in}{1.750000in}}{\pgfqpoint{2.279412in}{2.004545in}}%
\pgfusepath{clip}%
\pgfsetbuttcap%
\pgfsetroundjoin%
\pgfsetlinewidth{0.328558pt}%
\definecolor{currentstroke}{rgb}{0.271305,0.019942,0.347269}%
\pgfsetstrokecolor{currentstroke}%
\pgfsetdash{}{0pt}%
\pgfpathmoveto{\pgfqpoint{5.827603in}{2.843653in}}%
\pgfpathlineto{\pgfqpoint{5.777590in}{2.844917in}}%
\pgfusepath{stroke}%
\end{pgfscope}%
\begin{pgfscope}%
\pgfpathrectangle{\pgfqpoint{3.985294in}{1.750000in}}{\pgfqpoint{2.279412in}{2.004545in}}%
\pgfusepath{clip}%
\pgfsetbuttcap%
\pgfsetroundjoin%
\pgfsetlinewidth{0.324377pt}%
\definecolor{currentstroke}{rgb}{0.271305,0.019942,0.347269}%
\pgfsetstrokecolor{currentstroke}%
\pgfsetdash{}{0pt}%
\pgfpathmoveto{\pgfqpoint{5.777590in}{2.844917in}}%
\pgfpathlineto{\pgfqpoint{5.727457in}{2.844873in}}%
\pgfusepath{stroke}%
\end{pgfscope}%
\begin{pgfscope}%
\pgfpathrectangle{\pgfqpoint{3.985294in}{1.750000in}}{\pgfqpoint{2.279412in}{2.004545in}}%
\pgfusepath{clip}%
\pgfsetbuttcap%
\pgfsetroundjoin%
\pgfsetlinewidth{0.335559pt}%
\definecolor{currentstroke}{rgb}{0.273809,0.031497,0.358853}%
\pgfsetstrokecolor{currentstroke}%
\pgfsetdash{}{0pt}%
\pgfpathmoveto{\pgfqpoint{5.727457in}{2.844873in}}%
\pgfpathlineto{\pgfqpoint{5.677314in}{2.844194in}}%
\pgfusepath{stroke}%
\end{pgfscope}%
\begin{pgfscope}%
\pgfpathrectangle{\pgfqpoint{3.985294in}{1.750000in}}{\pgfqpoint{2.279412in}{2.004545in}}%
\pgfusepath{clip}%
\pgfsetbuttcap%
\pgfsetroundjoin%
\pgfsetlinewidth{0.357765pt}%
\definecolor{currentstroke}{rgb}{0.277018,0.050344,0.375715}%
\pgfsetstrokecolor{currentstroke}%
\pgfsetdash{}{0pt}%
\pgfpathmoveto{\pgfqpoint{5.677314in}{2.844194in}}%
\pgfpathlineto{\pgfqpoint{5.627172in}{2.843452in}}%
\pgfusepath{stroke}%
\end{pgfscope}%
\begin{pgfscope}%
\pgfpathrectangle{\pgfqpoint{3.985294in}{1.750000in}}{\pgfqpoint{2.279412in}{2.004545in}}%
\pgfusepath{clip}%
\pgfsetbuttcap%
\pgfsetroundjoin%
\pgfsetlinewidth{0.391448pt}%
\definecolor{currentstroke}{rgb}{0.280894,0.078907,0.402329}%
\pgfsetstrokecolor{currentstroke}%
\pgfsetdash{}{0pt}%
\pgfpathmoveto{\pgfqpoint{5.627172in}{2.843452in}}%
\pgfpathlineto{\pgfqpoint{5.577024in}{2.842909in}}%
\pgfusepath{stroke}%
\end{pgfscope}%
\begin{pgfscope}%
\pgfpathrectangle{\pgfqpoint{3.985294in}{1.750000in}}{\pgfqpoint{2.279412in}{2.004545in}}%
\pgfusepath{clip}%
\pgfsetbuttcap%
\pgfsetroundjoin%
\pgfsetlinewidth{0.443446pt}%
\definecolor{currentstroke}{rgb}{0.283197,0.115680,0.436115}%
\pgfsetstrokecolor{currentstroke}%
\pgfsetdash{}{0pt}%
\pgfpathmoveto{\pgfqpoint{5.577024in}{2.842909in}}%
\pgfpathlineto{\pgfqpoint{5.526876in}{2.842356in}}%
\pgfusepath{stroke}%
\end{pgfscope}%
\begin{pgfscope}%
\pgfpathrectangle{\pgfqpoint{3.985294in}{1.750000in}}{\pgfqpoint{2.279412in}{2.004545in}}%
\pgfusepath{clip}%
\pgfsetbuttcap%
\pgfsetroundjoin%
\pgfsetlinewidth{0.498873pt}%
\definecolor{currentstroke}{rgb}{0.281412,0.155834,0.469201}%
\pgfsetstrokecolor{currentstroke}%
\pgfsetdash{}{0pt}%
\pgfpathmoveto{\pgfqpoint{5.526876in}{2.842356in}}%
\pgfpathlineto{\pgfqpoint{5.476730in}{2.841696in}}%
\pgfusepath{stroke}%
\end{pgfscope}%
\begin{pgfscope}%
\pgfpathrectangle{\pgfqpoint{3.985294in}{1.750000in}}{\pgfqpoint{2.279412in}{2.004545in}}%
\pgfusepath{clip}%
\pgfsetbuttcap%
\pgfsetroundjoin%
\pgfsetlinewidth{0.557983pt}%
\definecolor{currentstroke}{rgb}{0.274128,0.199721,0.498911}%
\pgfsetstrokecolor{currentstroke}%
\pgfsetdash{}{0pt}%
\pgfpathmoveto{\pgfqpoint{5.476730in}{2.841696in}}%
\pgfpathlineto{\pgfqpoint{5.426588in}{2.840855in}}%
\pgfusepath{stroke}%
\end{pgfscope}%
\begin{pgfscope}%
\pgfpathrectangle{\pgfqpoint{3.985294in}{1.750000in}}{\pgfqpoint{2.279412in}{2.004545in}}%
\pgfusepath{clip}%
\pgfsetbuttcap%
\pgfsetroundjoin%
\pgfsetlinewidth{0.669225pt}%
\definecolor{currentstroke}{rgb}{0.250425,0.274290,0.533103}%
\pgfsetstrokecolor{currentstroke}%
\pgfsetdash{}{0pt}%
\pgfpathmoveto{\pgfqpoint{5.426588in}{2.840855in}}%
\pgfpathlineto{\pgfqpoint{5.376452in}{2.839743in}}%
\pgfusepath{stroke}%
\end{pgfscope}%
\begin{pgfscope}%
\pgfpathrectangle{\pgfqpoint{3.985294in}{1.750000in}}{\pgfqpoint{2.279412in}{2.004545in}}%
\pgfusepath{clip}%
\pgfsetbuttcap%
\pgfsetroundjoin%
\pgfsetlinewidth{0.746227pt}%
\definecolor{currentstroke}{rgb}{0.229739,0.322361,0.545706}%
\pgfsetstrokecolor{currentstroke}%
\pgfsetdash{}{0pt}%
\pgfpathmoveto{\pgfqpoint{5.376452in}{2.839743in}}%
\pgfpathlineto{\pgfqpoint{5.326327in}{2.838329in}}%
\pgfusepath{stroke}%
\end{pgfscope}%
\begin{pgfscope}%
\pgfpathrectangle{\pgfqpoint{3.985294in}{1.750000in}}{\pgfqpoint{2.279412in}{2.004545in}}%
\pgfusepath{clip}%
\pgfsetbuttcap%
\pgfsetroundjoin%
\pgfsetlinewidth{0.815967pt}%
\definecolor{currentstroke}{rgb}{0.208623,0.367752,0.552675}%
\pgfsetstrokecolor{currentstroke}%
\pgfsetdash{}{0pt}%
\pgfpathmoveto{\pgfqpoint{5.326327in}{2.838329in}}%
\pgfpathlineto{\pgfqpoint{5.276221in}{2.836475in}}%
\pgfusepath{stroke}%
\end{pgfscope}%
\begin{pgfscope}%
\pgfpathrectangle{\pgfqpoint{3.985294in}{1.750000in}}{\pgfqpoint{2.279412in}{2.004545in}}%
\pgfusepath{clip}%
\pgfsetbuttcap%
\pgfsetroundjoin%
\pgfsetlinewidth{0.913673pt}%
\definecolor{currentstroke}{rgb}{0.183898,0.422383,0.556944}%
\pgfsetstrokecolor{currentstroke}%
\pgfsetdash{}{0pt}%
\pgfpathmoveto{\pgfqpoint{5.276221in}{2.836475in}}%
\pgfpathlineto{\pgfqpoint{5.226147in}{2.834027in}}%
\pgfusepath{stroke}%
\end{pgfscope}%
\begin{pgfscope}%
\pgfpathrectangle{\pgfqpoint{3.985294in}{1.750000in}}{\pgfqpoint{2.279412in}{2.004545in}}%
\pgfusepath{clip}%
\pgfsetbuttcap%
\pgfsetroundjoin%
\pgfsetlinewidth{0.963045pt}%
\definecolor{currentstroke}{rgb}{0.172719,0.448791,0.557885}%
\pgfsetstrokecolor{currentstroke}%
\pgfsetdash{}{0pt}%
\pgfpathmoveto{\pgfqpoint{5.226147in}{2.834027in}}%
\pgfpathlineto{\pgfqpoint{5.176106in}{2.831110in}}%
\pgfusepath{stroke}%
\end{pgfscope}%
\begin{pgfscope}%
\pgfpathrectangle{\pgfqpoint{3.985294in}{1.750000in}}{\pgfqpoint{2.279412in}{2.004545in}}%
\pgfusepath{clip}%
\pgfsetbuttcap%
\pgfsetroundjoin%
\pgfsetlinewidth{0.909707pt}%
\definecolor{currentstroke}{rgb}{0.185556,0.418570,0.556753}%
\pgfsetstrokecolor{currentstroke}%
\pgfsetdash{}{0pt}%
\pgfpathmoveto{\pgfqpoint{5.176106in}{2.831110in}}%
\pgfpathlineto{\pgfqpoint{5.126123in}{2.827532in}}%
\pgfusepath{stroke}%
\end{pgfscope}%
\begin{pgfscope}%
\pgfpathrectangle{\pgfqpoint{3.985294in}{1.750000in}}{\pgfqpoint{2.279412in}{2.004545in}}%
\pgfusepath{clip}%
\pgfsetbuttcap%
\pgfsetroundjoin%
\pgfsetlinewidth{0.929159pt}%
\definecolor{currentstroke}{rgb}{0.180629,0.429975,0.557282}%
\pgfsetstrokecolor{currentstroke}%
\pgfsetdash{}{0pt}%
\pgfpathmoveto{\pgfqpoint{5.126123in}{2.827532in}}%
\pgfpathlineto{\pgfqpoint{5.076225in}{2.823140in}}%
\pgfusepath{stroke}%
\end{pgfscope}%
\begin{pgfscope}%
\pgfpathrectangle{\pgfqpoint{3.985294in}{1.750000in}}{\pgfqpoint{2.279412in}{2.004545in}}%
\pgfusepath{clip}%
\pgfsetbuttcap%
\pgfsetroundjoin%
\pgfsetlinewidth{0.305703pt}%
\definecolor{currentstroke}{rgb}{0.267004,0.004874,0.329415}%
\pgfsetstrokecolor{currentstroke}%
\pgfsetdash{}{0pt}%
\pgfpathmoveto{\pgfqpoint{5.945670in}{2.932700in}}%
\pgfpathlineto{\pgfqpoint{5.899362in}{2.935319in}}%
\pgfusepath{stroke}%
\end{pgfscope}%
\begin{pgfscope}%
\pgfpathrectangle{\pgfqpoint{3.985294in}{1.750000in}}{\pgfqpoint{2.279412in}{2.004545in}}%
\pgfusepath{clip}%
\pgfsetbuttcap%
\pgfsetroundjoin%
\pgfsetlinewidth{0.318105pt}%
\definecolor{currentstroke}{rgb}{0.269944,0.014625,0.341379}%
\pgfsetstrokecolor{currentstroke}%
\pgfsetdash{}{0pt}%
\pgfpathmoveto{\pgfqpoint{5.899362in}{2.935319in}}%
\pgfpathlineto{\pgfqpoint{5.851513in}{2.935880in}}%
\pgfusepath{stroke}%
\end{pgfscope}%
\begin{pgfscope}%
\pgfpathrectangle{\pgfqpoint{3.985294in}{1.750000in}}{\pgfqpoint{2.279412in}{2.004545in}}%
\pgfusepath{clip}%
\pgfsetbuttcap%
\pgfsetroundjoin%
\pgfsetlinewidth{0.323357pt}%
\definecolor{currentstroke}{rgb}{0.271305,0.019942,0.347269}%
\pgfsetstrokecolor{currentstroke}%
\pgfsetdash{}{0pt}%
\pgfpathmoveto{\pgfqpoint{5.851513in}{2.935880in}}%
\pgfpathlineto{\pgfqpoint{5.801379in}{2.935711in}}%
\pgfusepath{stroke}%
\end{pgfscope}%
\begin{pgfscope}%
\pgfpathrectangle{\pgfqpoint{3.985294in}{1.750000in}}{\pgfqpoint{2.279412in}{2.004545in}}%
\pgfusepath{clip}%
\pgfsetbuttcap%
\pgfsetroundjoin%
\pgfsetlinewidth{0.313582pt}%
\definecolor{currentstroke}{rgb}{0.268510,0.009605,0.335427}%
\pgfsetstrokecolor{currentstroke}%
\pgfsetdash{}{0pt}%
\pgfpathmoveto{\pgfqpoint{5.801379in}{2.935711in}}%
\pgfpathlineto{\pgfqpoint{5.751240in}{2.935646in}}%
\pgfusepath{stroke}%
\end{pgfscope}%
\begin{pgfscope}%
\pgfpathrectangle{\pgfqpoint{3.985294in}{1.750000in}}{\pgfqpoint{2.279412in}{2.004545in}}%
\pgfusepath{clip}%
\pgfsetbuttcap%
\pgfsetroundjoin%
\pgfsetlinewidth{0.334608pt}%
\definecolor{currentstroke}{rgb}{0.272594,0.025563,0.353093}%
\pgfsetstrokecolor{currentstroke}%
\pgfsetdash{}{0pt}%
\pgfpathmoveto{\pgfqpoint{5.751240in}{2.935646in}}%
\pgfpathlineto{\pgfqpoint{5.701098in}{2.934937in}}%
\pgfusepath{stroke}%
\end{pgfscope}%
\begin{pgfscope}%
\pgfpathrectangle{\pgfqpoint{3.985294in}{1.750000in}}{\pgfqpoint{2.279412in}{2.004545in}}%
\pgfusepath{clip}%
\pgfsetbuttcap%
\pgfsetroundjoin%
\pgfsetlinewidth{0.343178pt}%
\definecolor{currentstroke}{rgb}{0.274952,0.037752,0.364543}%
\pgfsetstrokecolor{currentstroke}%
\pgfsetdash{}{0pt}%
\pgfpathmoveto{\pgfqpoint{5.701098in}{2.934937in}}%
\pgfpathlineto{\pgfqpoint{5.650957in}{2.934088in}}%
\pgfusepath{stroke}%
\end{pgfscope}%
\begin{pgfscope}%
\pgfpathrectangle{\pgfqpoint{3.985294in}{1.750000in}}{\pgfqpoint{2.279412in}{2.004545in}}%
\pgfusepath{clip}%
\pgfsetbuttcap%
\pgfsetroundjoin%
\pgfsetlinewidth{0.360611pt}%
\definecolor{currentstroke}{rgb}{0.277018,0.050344,0.375715}%
\pgfsetstrokecolor{currentstroke}%
\pgfsetdash{}{0pt}%
\pgfpathmoveto{\pgfqpoint{5.650957in}{2.934088in}}%
\pgfpathlineto{\pgfqpoint{5.600811in}{2.933501in}}%
\pgfusepath{stroke}%
\end{pgfscope}%
\begin{pgfscope}%
\pgfpathrectangle{\pgfqpoint{3.985294in}{1.750000in}}{\pgfqpoint{2.279412in}{2.004545in}}%
\pgfusepath{clip}%
\pgfsetbuttcap%
\pgfsetroundjoin%
\pgfsetlinewidth{0.402059pt}%
\definecolor{currentstroke}{rgb}{0.281446,0.084320,0.407414}%
\pgfsetstrokecolor{currentstroke}%
\pgfsetdash{}{0pt}%
\pgfpathmoveto{\pgfqpoint{5.600811in}{2.933501in}}%
\pgfpathlineto{\pgfqpoint{5.550665in}{2.932875in}}%
\pgfusepath{stroke}%
\end{pgfscope}%
\begin{pgfscope}%
\pgfpathrectangle{\pgfqpoint{3.985294in}{1.750000in}}{\pgfqpoint{2.279412in}{2.004545in}}%
\pgfusepath{clip}%
\pgfsetbuttcap%
\pgfsetroundjoin%
\pgfsetlinewidth{0.453073pt}%
\definecolor{currentstroke}{rgb}{0.283187,0.125848,0.444960}%
\pgfsetstrokecolor{currentstroke}%
\pgfsetdash{}{0pt}%
\pgfpathmoveto{\pgfqpoint{5.550665in}{2.932875in}}%
\pgfpathlineto{\pgfqpoint{5.500526in}{2.931908in}}%
\pgfusepath{stroke}%
\end{pgfscope}%
\begin{pgfscope}%
\pgfpathrectangle{\pgfqpoint{3.985294in}{1.750000in}}{\pgfqpoint{2.279412in}{2.004545in}}%
\pgfusepath{clip}%
\pgfsetbuttcap%
\pgfsetroundjoin%
\pgfsetlinewidth{0.496072pt}%
\definecolor{currentstroke}{rgb}{0.281412,0.155834,0.469201}%
\pgfsetstrokecolor{currentstroke}%
\pgfsetdash{}{0pt}%
\pgfpathmoveto{\pgfqpoint{5.500526in}{2.931908in}}%
\pgfpathlineto{\pgfqpoint{5.450392in}{2.930753in}}%
\pgfusepath{stroke}%
\end{pgfscope}%
\begin{pgfscope}%
\pgfpathrectangle{\pgfqpoint{3.985294in}{1.750000in}}{\pgfqpoint{2.279412in}{2.004545in}}%
\pgfusepath{clip}%
\pgfsetbuttcap%
\pgfsetroundjoin%
\pgfsetlinewidth{0.560874pt}%
\definecolor{currentstroke}{rgb}{0.274128,0.199721,0.498911}%
\pgfsetstrokecolor{currentstroke}%
\pgfsetdash{}{0pt}%
\pgfpathmoveto{\pgfqpoint{5.450392in}{2.930753in}}%
\pgfpathlineto{\pgfqpoint{5.400267in}{2.929319in}}%
\pgfusepath{stroke}%
\end{pgfscope}%
\begin{pgfscope}%
\pgfpathrectangle{\pgfqpoint{3.985294in}{1.750000in}}{\pgfqpoint{2.279412in}{2.004545in}}%
\pgfusepath{clip}%
\pgfsetbuttcap%
\pgfsetroundjoin%
\pgfsetlinewidth{0.620817pt}%
\definecolor{currentstroke}{rgb}{0.262138,0.242286,0.520837}%
\pgfsetstrokecolor{currentstroke}%
\pgfsetdash{}{0pt}%
\pgfpathmoveto{\pgfqpoint{5.400267in}{2.929319in}}%
\pgfpathlineto{\pgfqpoint{5.350173in}{2.927264in}}%
\pgfusepath{stroke}%
\end{pgfscope}%
\begin{pgfscope}%
\pgfpathrectangle{\pgfqpoint{3.985294in}{1.750000in}}{\pgfqpoint{2.279412in}{2.004545in}}%
\pgfusepath{clip}%
\pgfsetbuttcap%
\pgfsetroundjoin%
\pgfsetlinewidth{0.649719pt}%
\definecolor{currentstroke}{rgb}{0.255645,0.260703,0.528312}%
\pgfsetstrokecolor{currentstroke}%
\pgfsetdash{}{0pt}%
\pgfpathmoveto{\pgfqpoint{5.350173in}{2.927264in}}%
\pgfpathlineto{\pgfqpoint{5.300121in}{2.924508in}}%
\pgfusepath{stroke}%
\end{pgfscope}%
\begin{pgfscope}%
\pgfpathrectangle{\pgfqpoint{3.985294in}{1.750000in}}{\pgfqpoint{2.279412in}{2.004545in}}%
\pgfusepath{clip}%
\pgfsetbuttcap%
\pgfsetroundjoin%
\pgfsetlinewidth{0.746195pt}%
\definecolor{currentstroke}{rgb}{0.229739,0.322361,0.545706}%
\pgfsetstrokecolor{currentstroke}%
\pgfsetdash{}{0pt}%
\pgfpathmoveto{\pgfqpoint{5.300121in}{2.924508in}}%
\pgfpathlineto{\pgfqpoint{5.250138in}{2.920950in}}%
\pgfusepath{stroke}%
\end{pgfscope}%
\begin{pgfscope}%
\pgfpathrectangle{\pgfqpoint{3.985294in}{1.750000in}}{\pgfqpoint{2.279412in}{2.004545in}}%
\pgfusepath{clip}%
\pgfsetbuttcap%
\pgfsetroundjoin%
\pgfsetlinewidth{0.774500pt}%
\definecolor{currentstroke}{rgb}{0.221989,0.339161,0.548752}%
\pgfsetstrokecolor{currentstroke}%
\pgfsetdash{}{0pt}%
\pgfpathmoveto{\pgfqpoint{5.250138in}{2.920950in}}%
\pgfpathlineto{\pgfqpoint{5.200269in}{2.916324in}}%
\pgfusepath{stroke}%
\end{pgfscope}%
\begin{pgfscope}%
\pgfpathrectangle{\pgfqpoint{3.985294in}{1.750000in}}{\pgfqpoint{2.279412in}{2.004545in}}%
\pgfusepath{clip}%
\pgfsetbuttcap%
\pgfsetroundjoin%
\pgfsetlinewidth{0.780271pt}%
\definecolor{currentstroke}{rgb}{0.220057,0.343307,0.549413}%
\pgfsetstrokecolor{currentstroke}%
\pgfsetdash{}{0pt}%
\pgfpathmoveto{\pgfqpoint{5.200269in}{2.916324in}}%
\pgfpathlineto{\pgfqpoint{5.150579in}{2.910413in}}%
\pgfusepath{stroke}%
\end{pgfscope}%
\begin{pgfscope}%
\pgfpathrectangle{\pgfqpoint{3.985294in}{1.750000in}}{\pgfqpoint{2.279412in}{2.004545in}}%
\pgfusepath{clip}%
\pgfsetbuttcap%
\pgfsetroundjoin%
\pgfsetlinewidth{0.795525pt}%
\definecolor{currentstroke}{rgb}{0.214298,0.355619,0.551184}%
\pgfsetstrokecolor{currentstroke}%
\pgfsetdash{}{0pt}%
\pgfpathmoveto{\pgfqpoint{5.150579in}{2.910413in}}%
\pgfpathlineto{\pgfqpoint{5.101171in}{2.902916in}}%
\pgfusepath{stroke}%
\end{pgfscope}%
\begin{pgfscope}%
\pgfpathrectangle{\pgfqpoint{3.985294in}{1.750000in}}{\pgfqpoint{2.279412in}{2.004545in}}%
\pgfusepath{clip}%
\pgfsetbuttcap%
\pgfsetroundjoin%
\pgfsetlinewidth{0.812628pt}%
\definecolor{currentstroke}{rgb}{0.210503,0.363727,0.552206}%
\pgfsetstrokecolor{currentstroke}%
\pgfsetdash{}{0pt}%
\pgfpathmoveto{\pgfqpoint{5.101171in}{2.902916in}}%
\pgfpathlineto{\pgfqpoint{5.052093in}{2.893890in}}%
\pgfusepath{stroke}%
\end{pgfscope}%
\begin{pgfscope}%
\pgfpathrectangle{\pgfqpoint{3.985294in}{1.750000in}}{\pgfqpoint{2.279412in}{2.004545in}}%
\pgfusepath{clip}%
\pgfsetbuttcap%
\pgfsetroundjoin%
\pgfsetlinewidth{0.835888pt}%
\definecolor{currentstroke}{rgb}{0.204903,0.375746,0.553533}%
\pgfsetstrokecolor{currentstroke}%
\pgfsetdash{}{0pt}%
\pgfpathmoveto{\pgfqpoint{5.052093in}{2.893890in}}%
\pgfpathlineto{\pgfqpoint{5.003400in}{2.883385in}}%
\pgfusepath{stroke}%
\end{pgfscope}%
\begin{pgfscope}%
\pgfpathrectangle{\pgfqpoint{3.985294in}{1.750000in}}{\pgfqpoint{2.279412in}{2.004545in}}%
\pgfusepath{clip}%
\pgfsetbuttcap%
\pgfsetroundjoin%
\pgfsetlinewidth{0.816415pt}%
\definecolor{currentstroke}{rgb}{0.208623,0.367752,0.552675}%
\pgfsetstrokecolor{currentstroke}%
\pgfsetdash{}{0pt}%
\pgfpathmoveto{\pgfqpoint{5.003400in}{2.883385in}}%
\pgfpathlineto{\pgfqpoint{4.955208in}{2.871241in}}%
\pgfusepath{stroke}%
\end{pgfscope}%
\begin{pgfscope}%
\pgfpathrectangle{\pgfqpoint{3.985294in}{1.750000in}}{\pgfqpoint{2.279412in}{2.004545in}}%
\pgfusepath{clip}%
\pgfsetbuttcap%
\pgfsetroundjoin%
\pgfsetlinewidth{0.824940pt}%
\definecolor{currentstroke}{rgb}{0.206756,0.371758,0.553117}%
\pgfsetstrokecolor{currentstroke}%
\pgfsetdash{}{0pt}%
\pgfpathmoveto{\pgfqpoint{4.955208in}{2.871241in}}%
\pgfpathlineto{\pgfqpoint{4.907641in}{2.857360in}}%
\pgfusepath{stroke}%
\end{pgfscope}%
\begin{pgfscope}%
\pgfpathrectangle{\pgfqpoint{3.985294in}{1.750000in}}{\pgfqpoint{2.279412in}{2.004545in}}%
\pgfusepath{clip}%
\pgfsetbuttcap%
\pgfsetroundjoin%
\pgfsetlinewidth{0.806710pt}%
\definecolor{currentstroke}{rgb}{0.212395,0.359683,0.551710}%
\pgfsetstrokecolor{currentstroke}%
\pgfsetdash{}{0pt}%
\pgfpathmoveto{\pgfqpoint{4.907641in}{2.857360in}}%
\pgfpathlineto{\pgfqpoint{4.861187in}{2.841034in}}%
\pgfusepath{stroke}%
\end{pgfscope}%
\begin{pgfscope}%
\pgfpathrectangle{\pgfqpoint{3.985294in}{1.750000in}}{\pgfqpoint{2.279412in}{2.004545in}}%
\pgfusepath{clip}%
\pgfsetbuttcap%
\pgfsetroundjoin%
\pgfsetlinewidth{0.759001pt}%
\definecolor{currentstroke}{rgb}{0.225863,0.330805,0.547314}%
\pgfsetstrokecolor{currentstroke}%
\pgfsetdash{}{0pt}%
\pgfpathmoveto{\pgfqpoint{4.861187in}{2.841034in}}%
\pgfpathlineto{\pgfqpoint{4.816088in}{2.822174in}}%
\pgfusepath{stroke}%
\end{pgfscope}%
\begin{pgfscope}%
\pgfpathrectangle{\pgfqpoint{3.985294in}{1.750000in}}{\pgfqpoint{2.279412in}{2.004545in}}%
\pgfusepath{clip}%
\pgfsetbuttcap%
\pgfsetroundjoin%
\pgfsetlinewidth{0.760565pt}%
\definecolor{currentstroke}{rgb}{0.225863,0.330805,0.547314}%
\pgfsetstrokecolor{currentstroke}%
\pgfsetdash{}{0pt}%
\pgfpathmoveto{\pgfqpoint{4.816088in}{2.822174in}}%
\pgfpathlineto{\pgfqpoint{4.772151in}{2.801353in}}%
\pgfusepath{stroke}%
\end{pgfscope}%
\begin{pgfscope}%
\pgfpathrectangle{\pgfqpoint{3.985294in}{1.750000in}}{\pgfqpoint{2.279412in}{2.004545in}}%
\pgfusepath{clip}%
\pgfsetbuttcap%
\pgfsetroundjoin%
\pgfsetlinewidth{0.666970pt}%
\definecolor{currentstroke}{rgb}{0.250425,0.274290,0.533103}%
\pgfsetstrokecolor{currentstroke}%
\pgfsetdash{}{0pt}%
\pgfpathmoveto{\pgfqpoint{4.772151in}{2.801353in}}%
\pgfpathlineto{\pgfqpoint{4.772151in}{2.801353in}}%
\pgfusepath{stroke}%
\end{pgfscope}%
\begin{pgfscope}%
\pgfpathrectangle{\pgfqpoint{3.985294in}{1.750000in}}{\pgfqpoint{2.279412in}{2.004545in}}%
\pgfusepath{clip}%
\pgfsetbuttcap%
\pgfsetroundjoin%
\pgfsetlinewidth{0.666970pt}%
\definecolor{currentstroke}{rgb}{0.250425,0.274290,0.533103}%
\pgfsetstrokecolor{currentstroke}%
\pgfsetdash{}{0pt}%
\pgfpathmoveto{\pgfqpoint{4.772151in}{2.801353in}}%
\pgfpathlineto{\pgfqpoint{4.738735in}{2.780393in}}%
\pgfusepath{stroke}%
\end{pgfscope}%
\begin{pgfscope}%
\pgfpathrectangle{\pgfqpoint{3.985294in}{1.750000in}}{\pgfqpoint{2.279412in}{2.004545in}}%
\pgfusepath{clip}%
\pgfsetbuttcap%
\pgfsetroundjoin%
\pgfsetlinewidth{0.314052pt}%
\definecolor{currentstroke}{rgb}{0.268510,0.009605,0.335427}%
\pgfsetstrokecolor{currentstroke}%
\pgfsetdash{}{0pt}%
\pgfpathmoveto{\pgfqpoint{5.421579in}{2.039108in}}%
\pgfpathlineto{\pgfqpoint{5.372191in}{2.044932in}}%
\pgfusepath{stroke}%
\end{pgfscope}%
\begin{pgfscope}%
\pgfpathrectangle{\pgfqpoint{3.985294in}{1.750000in}}{\pgfqpoint{2.279412in}{2.004545in}}%
\pgfusepath{clip}%
\pgfsetbuttcap%
\pgfsetroundjoin%
\pgfsetlinewidth{0.338796pt}%
\definecolor{currentstroke}{rgb}{0.273809,0.031497,0.358853}%
\pgfsetstrokecolor{currentstroke}%
\pgfsetdash{}{0pt}%
\pgfpathmoveto{\pgfqpoint{5.372191in}{2.044932in}}%
\pgfpathlineto{\pgfqpoint{5.322332in}{2.048726in}}%
\pgfusepath{stroke}%
\end{pgfscope}%
\begin{pgfscope}%
\pgfpathrectangle{\pgfqpoint{3.985294in}{1.750000in}}{\pgfqpoint{2.279412in}{2.004545in}}%
\pgfusepath{clip}%
\pgfsetbuttcap%
\pgfsetroundjoin%
\pgfsetlinewidth{0.331335pt}%
\definecolor{currentstroke}{rgb}{0.272594,0.025563,0.353093}%
\pgfsetstrokecolor{currentstroke}%
\pgfsetdash{}{0pt}%
\pgfpathmoveto{\pgfqpoint{5.322332in}{2.048726in}}%
\pgfpathlineto{\pgfqpoint{5.272660in}{2.053255in}}%
\pgfusepath{stroke}%
\end{pgfscope}%
\begin{pgfscope}%
\pgfpathrectangle{\pgfqpoint{3.985294in}{1.750000in}}{\pgfqpoint{2.279412in}{2.004545in}}%
\pgfusepath{clip}%
\pgfsetbuttcap%
\pgfsetroundjoin%
\pgfsetlinewidth{0.342152pt}%
\definecolor{currentstroke}{rgb}{0.273809,0.031497,0.358853}%
\pgfsetstrokecolor{currentstroke}%
\pgfsetdash{}{0pt}%
\pgfpathmoveto{\pgfqpoint{5.272660in}{2.053255in}}%
\pgfpathlineto{\pgfqpoint{5.224136in}{2.063320in}}%
\pgfusepath{stroke}%
\end{pgfscope}%
\begin{pgfscope}%
\pgfpathrectangle{\pgfqpoint{3.985294in}{1.750000in}}{\pgfqpoint{2.279412in}{2.004545in}}%
\pgfusepath{clip}%
\pgfsetbuttcap%
\pgfsetroundjoin%
\pgfsetlinewidth{0.330342pt}%
\definecolor{currentstroke}{rgb}{0.272594,0.025563,0.353093}%
\pgfsetstrokecolor{currentstroke}%
\pgfsetdash{}{0pt}%
\pgfpathmoveto{\pgfqpoint{5.224136in}{2.063320in}}%
\pgfpathlineto{\pgfqpoint{5.176292in}{2.075671in}}%
\pgfusepath{stroke}%
\end{pgfscope}%
\begin{pgfscope}%
\pgfpathrectangle{\pgfqpoint{3.985294in}{1.750000in}}{\pgfqpoint{2.279412in}{2.004545in}}%
\pgfusepath{clip}%
\pgfsetbuttcap%
\pgfsetroundjoin%
\pgfsetlinewidth{0.324495pt}%
\definecolor{currentstroke}{rgb}{0.271305,0.019942,0.347269}%
\pgfsetstrokecolor{currentstroke}%
\pgfsetdash{}{0pt}%
\pgfpathmoveto{\pgfqpoint{5.894378in}{2.301205in}}%
\pgfpathlineto{\pgfqpoint{5.844239in}{2.301848in}}%
\pgfusepath{stroke}%
\end{pgfscope}%
\begin{pgfscope}%
\pgfpathrectangle{\pgfqpoint{3.985294in}{1.750000in}}{\pgfqpoint{2.279412in}{2.004545in}}%
\pgfusepath{clip}%
\pgfsetbuttcap%
\pgfsetroundjoin%
\pgfsetlinewidth{0.315274pt}%
\definecolor{currentstroke}{rgb}{0.269944,0.014625,0.341379}%
\pgfsetstrokecolor{currentstroke}%
\pgfsetdash{}{0pt}%
\pgfpathmoveto{\pgfqpoint{5.844239in}{2.301848in}}%
\pgfpathlineto{\pgfqpoint{5.794175in}{2.301503in}}%
\pgfusepath{stroke}%
\end{pgfscope}%
\begin{pgfscope}%
\pgfpathrectangle{\pgfqpoint{3.985294in}{1.750000in}}{\pgfqpoint{2.279412in}{2.004545in}}%
\pgfusepath{clip}%
\pgfsetbuttcap%
\pgfsetroundjoin%
\pgfsetlinewidth{0.316420pt}%
\definecolor{currentstroke}{rgb}{0.269944,0.014625,0.341379}%
\pgfsetstrokecolor{currentstroke}%
\pgfsetdash{}{0pt}%
\pgfpathmoveto{\pgfqpoint{5.794175in}{2.301503in}}%
\pgfpathlineto{\pgfqpoint{5.744114in}{2.300200in}}%
\pgfusepath{stroke}%
\end{pgfscope}%
\begin{pgfscope}%
\pgfpathrectangle{\pgfqpoint{3.985294in}{1.750000in}}{\pgfqpoint{2.279412in}{2.004545in}}%
\pgfusepath{clip}%
\pgfsetbuttcap%
\pgfsetroundjoin%
\pgfsetlinewidth{0.323885pt}%
\definecolor{currentstroke}{rgb}{0.271305,0.019942,0.347269}%
\pgfsetstrokecolor{currentstroke}%
\pgfsetdash{}{0pt}%
\pgfpathmoveto{\pgfqpoint{5.744114in}{2.300200in}}%
\pgfpathlineto{\pgfqpoint{5.694002in}{2.300757in}}%
\pgfusepath{stroke}%
\end{pgfscope}%
\begin{pgfscope}%
\pgfpathrectangle{\pgfqpoint{3.985294in}{1.750000in}}{\pgfqpoint{2.279412in}{2.004545in}}%
\pgfusepath{clip}%
\pgfsetbuttcap%
\pgfsetroundjoin%
\pgfsetlinewidth{0.329138pt}%
\definecolor{currentstroke}{rgb}{0.272594,0.025563,0.353093}%
\pgfsetstrokecolor{currentstroke}%
\pgfsetdash{}{0pt}%
\pgfpathmoveto{\pgfqpoint{5.694002in}{2.300757in}}%
\pgfpathlineto{\pgfqpoint{5.643889in}{2.302388in}}%
\pgfusepath{stroke}%
\end{pgfscope}%
\begin{pgfscope}%
\pgfpathrectangle{\pgfqpoint{3.985294in}{1.750000in}}{\pgfqpoint{2.279412in}{2.004545in}}%
\pgfusepath{clip}%
\pgfsetbuttcap%
\pgfsetroundjoin%
\pgfsetlinewidth{0.340950pt}%
\definecolor{currentstroke}{rgb}{0.273809,0.031497,0.358853}%
\pgfsetstrokecolor{currentstroke}%
\pgfsetdash{}{0pt}%
\pgfpathmoveto{\pgfqpoint{5.643889in}{2.302388in}}%
\pgfpathlineto{\pgfqpoint{5.593762in}{2.303726in}}%
\pgfusepath{stroke}%
\end{pgfscope}%
\begin{pgfscope}%
\pgfpathrectangle{\pgfqpoint{3.985294in}{1.750000in}}{\pgfqpoint{2.279412in}{2.004545in}}%
\pgfusepath{clip}%
\pgfsetbuttcap%
\pgfsetroundjoin%
\pgfsetlinewidth{0.352242pt}%
\definecolor{currentstroke}{rgb}{0.276022,0.044167,0.370164}%
\pgfsetstrokecolor{currentstroke}%
\pgfsetdash{}{0pt}%
\pgfpathmoveto{\pgfqpoint{5.593762in}{2.303726in}}%
\pgfpathlineto{\pgfqpoint{5.543659in}{2.305508in}}%
\pgfusepath{stroke}%
\end{pgfscope}%
\begin{pgfscope}%
\pgfpathrectangle{\pgfqpoint{3.985294in}{1.750000in}}{\pgfqpoint{2.279412in}{2.004545in}}%
\pgfusepath{clip}%
\pgfsetbuttcap%
\pgfsetroundjoin%
\pgfsetlinewidth{0.363049pt}%
\definecolor{currentstroke}{rgb}{0.277941,0.056324,0.381191}%
\pgfsetstrokecolor{currentstroke}%
\pgfsetdash{}{0pt}%
\pgfpathmoveto{\pgfqpoint{5.543659in}{2.305508in}}%
\pgfpathlineto{\pgfqpoint{5.493589in}{2.307943in}}%
\pgfusepath{stroke}%
\end{pgfscope}%
\begin{pgfscope}%
\pgfpathrectangle{\pgfqpoint{3.985294in}{1.750000in}}{\pgfqpoint{2.279412in}{2.004545in}}%
\pgfusepath{clip}%
\pgfsetbuttcap%
\pgfsetroundjoin%
\pgfsetlinewidth{0.387669pt}%
\definecolor{currentstroke}{rgb}{0.280267,0.073417,0.397163}%
\pgfsetstrokecolor{currentstroke}%
\pgfsetdash{}{0pt}%
\pgfpathmoveto{\pgfqpoint{5.493589in}{2.307943in}}%
\pgfpathlineto{\pgfqpoint{5.443575in}{2.311089in}}%
\pgfusepath{stroke}%
\end{pgfscope}%
\begin{pgfscope}%
\pgfpathrectangle{\pgfqpoint{3.985294in}{1.750000in}}{\pgfqpoint{2.279412in}{2.004545in}}%
\pgfusepath{clip}%
\pgfsetbuttcap%
\pgfsetroundjoin%
\pgfsetlinewidth{0.422992pt}%
\definecolor{currentstroke}{rgb}{0.282656,0.100196,0.422160}%
\pgfsetstrokecolor{currentstroke}%
\pgfsetdash{}{0pt}%
\pgfpathmoveto{\pgfqpoint{5.443575in}{2.311089in}}%
\pgfpathlineto{\pgfqpoint{5.393643in}{2.315192in}}%
\pgfusepath{stroke}%
\end{pgfscope}%
\begin{pgfscope}%
\pgfpathrectangle{\pgfqpoint{3.985294in}{1.750000in}}{\pgfqpoint{2.279412in}{2.004545in}}%
\pgfusepath{clip}%
\pgfsetbuttcap%
\pgfsetroundjoin%
\pgfsetlinewidth{0.408562pt}%
\definecolor{currentstroke}{rgb}{0.281924,0.089666,0.412415}%
\pgfsetstrokecolor{currentstroke}%
\pgfsetdash{}{0pt}%
\pgfpathmoveto{\pgfqpoint{5.393643in}{2.315192in}}%
\pgfpathlineto{\pgfqpoint{5.343833in}{2.320280in}}%
\pgfusepath{stroke}%
\end{pgfscope}%
\begin{pgfscope}%
\pgfpathrectangle{\pgfqpoint{3.985294in}{1.750000in}}{\pgfqpoint{2.279412in}{2.004545in}}%
\pgfusepath{clip}%
\pgfsetbuttcap%
\pgfsetroundjoin%
\pgfsetlinewidth{0.408493pt}%
\definecolor{currentstroke}{rgb}{0.281924,0.089666,0.412415}%
\pgfsetstrokecolor{currentstroke}%
\pgfsetdash{}{0pt}%
\pgfpathmoveto{\pgfqpoint{5.343833in}{2.320280in}}%
\pgfpathlineto{\pgfqpoint{5.294292in}{2.327040in}}%
\pgfusepath{stroke}%
\end{pgfscope}%
\begin{pgfscope}%
\pgfpathrectangle{\pgfqpoint{3.985294in}{1.750000in}}{\pgfqpoint{2.279412in}{2.004545in}}%
\pgfusepath{clip}%
\pgfsetbuttcap%
\pgfsetroundjoin%
\pgfsetlinewidth{0.424931pt}%
\definecolor{currentstroke}{rgb}{0.282910,0.105393,0.426902}%
\pgfsetstrokecolor{currentstroke}%
\pgfsetdash{}{0pt}%
\pgfpathmoveto{\pgfqpoint{5.294292in}{2.327040in}}%
\pgfpathlineto{\pgfqpoint{5.245188in}{2.335907in}}%
\pgfusepath{stroke}%
\end{pgfscope}%
\begin{pgfscope}%
\pgfpathrectangle{\pgfqpoint{3.985294in}{1.750000in}}{\pgfqpoint{2.279412in}{2.004545in}}%
\pgfusepath{clip}%
\pgfsetbuttcap%
\pgfsetroundjoin%
\pgfsetlinewidth{0.446102pt}%
\definecolor{currentstroke}{rgb}{0.283229,0.120777,0.440584}%
\pgfsetstrokecolor{currentstroke}%
\pgfsetdash{}{0pt}%
\pgfpathmoveto{\pgfqpoint{5.245188in}{2.335907in}}%
\pgfpathlineto{\pgfqpoint{5.196679in}{2.347014in}}%
\pgfusepath{stroke}%
\end{pgfscope}%
\begin{pgfscope}%
\pgfpathrectangle{\pgfqpoint{3.985294in}{1.750000in}}{\pgfqpoint{2.279412in}{2.004545in}}%
\pgfusepath{clip}%
\pgfsetbuttcap%
\pgfsetroundjoin%
\pgfsetlinewidth{0.417045pt}%
\definecolor{currentstroke}{rgb}{0.282327,0.094955,0.417331}%
\pgfsetstrokecolor{currentstroke}%
\pgfsetdash{}{0pt}%
\pgfpathmoveto{\pgfqpoint{5.196679in}{2.347014in}}%
\pgfpathlineto{\pgfqpoint{5.148593in}{2.359507in}}%
\pgfusepath{stroke}%
\end{pgfscope}%
\begin{pgfscope}%
\pgfpathrectangle{\pgfqpoint{3.985294in}{1.750000in}}{\pgfqpoint{2.279412in}{2.004545in}}%
\pgfusepath{clip}%
\pgfsetbuttcap%
\pgfsetroundjoin%
\pgfsetlinewidth{0.447524pt}%
\definecolor{currentstroke}{rgb}{0.283229,0.120777,0.440584}%
\pgfsetstrokecolor{currentstroke}%
\pgfsetdash{}{0pt}%
\pgfpathmoveto{\pgfqpoint{5.148593in}{2.359507in}}%
\pgfpathlineto{\pgfqpoint{5.101132in}{2.373642in}}%
\pgfusepath{stroke}%
\end{pgfscope}%
\begin{pgfscope}%
\pgfpathrectangle{\pgfqpoint{3.985294in}{1.750000in}}{\pgfqpoint{2.279412in}{2.004545in}}%
\pgfusepath{clip}%
\pgfsetbuttcap%
\pgfsetroundjoin%
\pgfsetlinewidth{0.479036pt}%
\definecolor{currentstroke}{rgb}{0.282623,0.140926,0.457517}%
\pgfsetstrokecolor{currentstroke}%
\pgfsetdash{}{0pt}%
\pgfpathmoveto{\pgfqpoint{5.101132in}{2.373642in}}%
\pgfpathlineto{\pgfqpoint{5.056514in}{2.393137in}}%
\pgfusepath{stroke}%
\end{pgfscope}%
\begin{pgfscope}%
\pgfpathrectangle{\pgfqpoint{3.985294in}{1.750000in}}{\pgfqpoint{2.279412in}{2.004545in}}%
\pgfusepath{clip}%
\pgfsetbuttcap%
\pgfsetroundjoin%
\pgfsetlinewidth{0.550522pt}%
\definecolor{currentstroke}{rgb}{0.275191,0.194905,0.496005}%
\pgfsetstrokecolor{currentstroke}%
\pgfsetdash{}{0pt}%
\pgfpathmoveto{\pgfqpoint{5.056514in}{2.393137in}}%
\pgfpathlineto{\pgfqpoint{5.018519in}{2.417581in}}%
\pgfusepath{stroke}%
\end{pgfscope}%
\begin{pgfscope}%
\pgfpathrectangle{\pgfqpoint{3.985294in}{1.750000in}}{\pgfqpoint{2.279412in}{2.004545in}}%
\pgfusepath{clip}%
\pgfsetbuttcap%
\pgfsetroundjoin%
\pgfsetlinewidth{0.513729pt}%
\definecolor{currentstroke}{rgb}{0.280255,0.165693,0.476498}%
\pgfsetstrokecolor{currentstroke}%
\pgfsetdash{}{0pt}%
\pgfpathmoveto{\pgfqpoint{5.018519in}{2.417581in}}%
\pgfpathlineto{\pgfqpoint{4.979486in}{2.445020in}}%
\pgfusepath{stroke}%
\end{pgfscope}%
\begin{pgfscope}%
\pgfpathrectangle{\pgfqpoint{3.985294in}{1.750000in}}{\pgfqpoint{2.279412in}{2.004545in}}%
\pgfusepath{clip}%
\pgfsetbuttcap%
\pgfsetroundjoin%
\pgfsetlinewidth{0.588135pt}%
\definecolor{currentstroke}{rgb}{0.269308,0.218818,0.509577}%
\pgfsetstrokecolor{currentstroke}%
\pgfsetdash{}{0pt}%
\pgfpathmoveto{\pgfqpoint{4.979486in}{2.445020in}}%
\pgfpathlineto{\pgfqpoint{4.942550in}{2.474667in}}%
\pgfusepath{stroke}%
\end{pgfscope}%
\begin{pgfscope}%
\pgfpathrectangle{\pgfqpoint{3.985294in}{1.750000in}}{\pgfqpoint{2.279412in}{2.004545in}}%
\pgfusepath{clip}%
\pgfsetbuttcap%
\pgfsetroundjoin%
\pgfsetlinewidth{0.652013pt}%
\definecolor{currentstroke}{rgb}{0.253935,0.265254,0.529983}%
\pgfsetstrokecolor{currentstroke}%
\pgfsetdash{}{0pt}%
\pgfpathmoveto{\pgfqpoint{4.942550in}{2.474667in}}%
\pgfpathlineto{\pgfqpoint{4.906305in}{2.505114in}}%
\pgfusepath{stroke}%
\end{pgfscope}%
\begin{pgfscope}%
\pgfpathrectangle{\pgfqpoint{3.985294in}{1.750000in}}{\pgfqpoint{2.279412in}{2.004545in}}%
\pgfusepath{clip}%
\pgfsetbuttcap%
\pgfsetroundjoin%
\pgfsetlinewidth{0.747438pt}%
\definecolor{currentstroke}{rgb}{0.227802,0.326594,0.546532}%
\pgfsetstrokecolor{currentstroke}%
\pgfsetdash{}{0pt}%
\pgfpathmoveto{\pgfqpoint{4.906305in}{2.505114in}}%
\pgfpathlineto{\pgfqpoint{4.869869in}{2.535346in}}%
\pgfusepath{stroke}%
\end{pgfscope}%
\begin{pgfscope}%
\pgfpathrectangle{\pgfqpoint{3.985294in}{1.750000in}}{\pgfqpoint{2.279412in}{2.004545in}}%
\pgfusepath{clip}%
\pgfsetbuttcap%
\pgfsetroundjoin%
\pgfsetlinewidth{0.849848pt}%
\definecolor{currentstroke}{rgb}{0.201239,0.383670,0.554294}%
\pgfsetstrokecolor{currentstroke}%
\pgfsetdash{}{0pt}%
\pgfpathmoveto{\pgfqpoint{4.869869in}{2.535346in}}%
\pgfpathlineto{\pgfqpoint{4.834601in}{2.566596in}}%
\pgfusepath{stroke}%
\end{pgfscope}%
\begin{pgfscope}%
\pgfpathrectangle{\pgfqpoint{3.985294in}{1.750000in}}{\pgfqpoint{2.279412in}{2.004545in}}%
\pgfusepath{clip}%
\pgfsetbuttcap%
\pgfsetroundjoin%
\pgfsetlinewidth{0.784452pt}%
\definecolor{currentstroke}{rgb}{0.218130,0.347432,0.550038}%
\pgfsetstrokecolor{currentstroke}%
\pgfsetdash{}{0pt}%
\pgfpathmoveto{\pgfqpoint{4.834601in}{2.566596in}}%
\pgfpathlineto{\pgfqpoint{4.801066in}{2.598739in}}%
\pgfusepath{stroke}%
\end{pgfscope}%
\begin{pgfscope}%
\pgfpathrectangle{\pgfqpoint{3.985294in}{1.750000in}}{\pgfqpoint{2.279412in}{2.004545in}}%
\pgfusepath{clip}%
\pgfsetbuttcap%
\pgfsetroundjoin%
\pgfsetlinewidth{0.730644pt}%
\definecolor{currentstroke}{rgb}{0.233603,0.313828,0.543914}%
\pgfsetstrokecolor{currentstroke}%
\pgfsetdash{}{0pt}%
\pgfpathmoveto{\pgfqpoint{4.801066in}{2.598739in}}%
\pgfpathlineto{\pgfqpoint{4.766098in}{2.629610in}}%
\pgfusepath{stroke}%
\end{pgfscope}%
\begin{pgfscope}%
\pgfpathrectangle{\pgfqpoint{3.985294in}{1.750000in}}{\pgfqpoint{2.279412in}{2.004545in}}%
\pgfusepath{clip}%
\pgfsetbuttcap%
\pgfsetroundjoin%
\pgfsetlinewidth{0.740024pt}%
\definecolor{currentstroke}{rgb}{0.231674,0.318106,0.544834}%
\pgfsetstrokecolor{currentstroke}%
\pgfsetdash{}{0pt}%
\pgfpathmoveto{\pgfqpoint{4.766098in}{2.629610in}}%
\pgfpathlineto{\pgfqpoint{4.736219in}{2.662479in}}%
\pgfusepath{stroke}%
\end{pgfscope}%
\begin{pgfscope}%
\pgfpathrectangle{\pgfqpoint{3.985294in}{1.750000in}}{\pgfqpoint{2.279412in}{2.004545in}}%
\pgfusepath{clip}%
\pgfsetbuttcap%
\pgfsetroundjoin%
\pgfsetlinewidth{0.677603pt}%
\definecolor{currentstroke}{rgb}{0.248629,0.278775,0.534556}%
\pgfsetstrokecolor{currentstroke}%
\pgfsetdash{}{0pt}%
\pgfpathmoveto{\pgfqpoint{4.736219in}{2.662479in}}%
\pgfpathlineto{\pgfqpoint{4.707857in}{2.689638in}}%
\pgfusepath{stroke}%
\end{pgfscope}%
\begin{pgfscope}%
\pgfpathrectangle{\pgfqpoint{3.985294in}{1.750000in}}{\pgfqpoint{2.279412in}{2.004545in}}%
\pgfusepath{clip}%
\pgfsetbuttcap%
\pgfsetroundjoin%
\pgfsetlinewidth{0.532116pt}%
\definecolor{currentstroke}{rgb}{0.278012,0.180367,0.486697}%
\pgfsetstrokecolor{currentstroke}%
\pgfsetdash{}{0pt}%
\pgfpathmoveto{\pgfqpoint{4.707857in}{2.689638in}}%
\pgfpathlineto{\pgfqpoint{4.707857in}{2.689638in}}%
\pgfusepath{stroke}%
\end{pgfscope}%
\begin{pgfscope}%
\pgfpathrectangle{\pgfqpoint{3.985294in}{1.750000in}}{\pgfqpoint{2.279412in}{2.004545in}}%
\pgfusepath{clip}%
\pgfsetbuttcap%
\pgfsetroundjoin%
\pgfsetlinewidth{0.532116pt}%
\definecolor{currentstroke}{rgb}{0.278012,0.180367,0.486697}%
\pgfsetstrokecolor{currentstroke}%
\pgfsetdash{}{0pt}%
\pgfpathmoveto{\pgfqpoint{4.707857in}{2.689638in}}%
\pgfpathlineto{\pgfqpoint{4.696138in}{2.700163in}}%
\pgfusepath{stroke}%
\end{pgfscope}%
\begin{pgfscope}%
\pgfpathrectangle{\pgfqpoint{3.985294in}{1.750000in}}{\pgfqpoint{2.279412in}{2.004545in}}%
\pgfusepath{clip}%
\pgfsetbuttcap%
\pgfsetroundjoin%
\pgfsetlinewidth{0.499655pt}%
\definecolor{currentstroke}{rgb}{0.281412,0.155834,0.469201}%
\pgfsetstrokecolor{currentstroke}%
\pgfsetdash{}{0pt}%
\pgfpathmoveto{\pgfqpoint{4.696138in}{2.700163in}}%
\pgfpathlineto{\pgfqpoint{4.696138in}{2.700163in}}%
\pgfusepath{stroke}%
\end{pgfscope}%
\begin{pgfscope}%
\pgfpathrectangle{\pgfqpoint{3.985294in}{1.750000in}}{\pgfqpoint{2.279412in}{2.004545in}}%
\pgfusepath{clip}%
\pgfsetbuttcap%
\pgfsetroundjoin%
\pgfsetlinewidth{0.499655pt}%
\definecolor{currentstroke}{rgb}{0.281412,0.155834,0.469201}%
\pgfsetstrokecolor{currentstroke}%
\pgfsetdash{}{0pt}%
\pgfpathmoveto{\pgfqpoint{4.696138in}{2.700163in}}%
\pgfpathlineto{\pgfqpoint{4.696138in}{2.700163in}}%
\pgfusepath{stroke}%
\end{pgfscope}%
\begin{pgfscope}%
\pgfpathrectangle{\pgfqpoint{3.985294in}{1.750000in}}{\pgfqpoint{2.279412in}{2.004545in}}%
\pgfusepath{clip}%
\pgfsetbuttcap%
\pgfsetroundjoin%
\pgfsetlinewidth{0.499655pt}%
\definecolor{currentstroke}{rgb}{0.281412,0.155834,0.469201}%
\pgfsetstrokecolor{currentstroke}%
\pgfsetdash{}{0pt}%
\pgfpathmoveto{\pgfqpoint{4.696138in}{2.700163in}}%
\pgfpathlineto{\pgfqpoint{4.691957in}{2.707254in}}%
\pgfusepath{stroke}%
\end{pgfscope}%
\begin{pgfscope}%
\pgfpathrectangle{\pgfqpoint{3.985294in}{1.750000in}}{\pgfqpoint{2.279412in}{2.004545in}}%
\pgfusepath{clip}%
\pgfsetbuttcap%
\pgfsetroundjoin%
\pgfsetlinewidth{0.439413pt}%
\definecolor{currentstroke}{rgb}{0.283197,0.115680,0.436115}%
\pgfsetstrokecolor{currentstroke}%
\pgfsetdash{}{0pt}%
\pgfpathmoveto{\pgfqpoint{4.691957in}{2.707254in}}%
\pgfpathlineto{\pgfqpoint{4.690574in}{2.713247in}}%
\pgfusepath{stroke}%
\end{pgfscope}%
\begin{pgfscope}%
\pgfpathrectangle{\pgfqpoint{3.985294in}{1.750000in}}{\pgfqpoint{2.279412in}{2.004545in}}%
\pgfusepath{clip}%
\pgfsetbuttcap%
\pgfsetroundjoin%
\pgfsetlinewidth{0.328396pt}%
\definecolor{currentstroke}{rgb}{0.271305,0.019942,0.347269}%
\pgfsetstrokecolor{currentstroke}%
\pgfsetdash{}{0pt}%
\pgfpathmoveto{\pgfqpoint{5.894378in}{2.346312in}}%
\pgfpathlineto{\pgfqpoint{5.844332in}{2.346694in}}%
\pgfusepath{stroke}%
\end{pgfscope}%
\begin{pgfscope}%
\pgfpathrectangle{\pgfqpoint{3.985294in}{1.750000in}}{\pgfqpoint{2.279412in}{2.004545in}}%
\pgfusepath{clip}%
\pgfsetbuttcap%
\pgfsetroundjoin%
\pgfsetlinewidth{0.314543pt}%
\definecolor{currentstroke}{rgb}{0.268510,0.009605,0.335427}%
\pgfsetstrokecolor{currentstroke}%
\pgfsetdash{}{0pt}%
\pgfpathmoveto{\pgfqpoint{5.844332in}{2.346694in}}%
\pgfpathlineto{\pgfqpoint{5.794292in}{2.347268in}}%
\pgfusepath{stroke}%
\end{pgfscope}%
\begin{pgfscope}%
\pgfpathrectangle{\pgfqpoint{3.985294in}{1.750000in}}{\pgfqpoint{2.279412in}{2.004545in}}%
\pgfusepath{clip}%
\pgfsetbuttcap%
\pgfsetroundjoin%
\pgfsetlinewidth{0.320653pt}%
\definecolor{currentstroke}{rgb}{0.269944,0.014625,0.341379}%
\pgfsetstrokecolor{currentstroke}%
\pgfsetdash{}{0pt}%
\pgfpathmoveto{\pgfqpoint{5.794292in}{2.347268in}}%
\pgfpathlineto{\pgfqpoint{5.744217in}{2.348905in}}%
\pgfusepath{stroke}%
\end{pgfscope}%
\begin{pgfscope}%
\pgfpathrectangle{\pgfqpoint{3.985294in}{1.750000in}}{\pgfqpoint{2.279412in}{2.004545in}}%
\pgfusepath{clip}%
\pgfsetbuttcap%
\pgfsetroundjoin%
\pgfsetlinewidth{0.325720pt}%
\definecolor{currentstroke}{rgb}{0.271305,0.019942,0.347269}%
\pgfsetstrokecolor{currentstroke}%
\pgfsetdash{}{0pt}%
\pgfpathmoveto{\pgfqpoint{5.744217in}{2.348905in}}%
\pgfpathlineto{\pgfqpoint{5.694140in}{2.350180in}}%
\pgfusepath{stroke}%
\end{pgfscope}%
\begin{pgfscope}%
\pgfpathrectangle{\pgfqpoint{3.985294in}{1.750000in}}{\pgfqpoint{2.279412in}{2.004545in}}%
\pgfusepath{clip}%
\pgfsetbuttcap%
\pgfsetroundjoin%
\pgfsetlinewidth{0.331984pt}%
\definecolor{currentstroke}{rgb}{0.272594,0.025563,0.353093}%
\pgfsetstrokecolor{currentstroke}%
\pgfsetdash{}{0pt}%
\pgfpathmoveto{\pgfqpoint{5.694140in}{2.350180in}}%
\pgfpathlineto{\pgfqpoint{5.644003in}{2.350384in}}%
\pgfusepath{stroke}%
\end{pgfscope}%
\begin{pgfscope}%
\pgfpathrectangle{\pgfqpoint{3.985294in}{1.750000in}}{\pgfqpoint{2.279412in}{2.004545in}}%
\pgfusepath{clip}%
\pgfsetbuttcap%
\pgfsetroundjoin%
\pgfsetlinewidth{0.347773pt}%
\definecolor{currentstroke}{rgb}{0.274952,0.037752,0.364543}%
\pgfsetstrokecolor{currentstroke}%
\pgfsetdash{}{0pt}%
\pgfpathmoveto{\pgfqpoint{5.644003in}{2.350384in}}%
\pgfpathlineto{\pgfqpoint{5.593880in}{2.351748in}}%
\pgfusepath{stroke}%
\end{pgfscope}%
\begin{pgfscope}%
\pgfpathrectangle{\pgfqpoint{3.985294in}{1.750000in}}{\pgfqpoint{2.279412in}{2.004545in}}%
\pgfusepath{clip}%
\pgfsetbuttcap%
\pgfsetroundjoin%
\pgfsetlinewidth{0.366458pt}%
\definecolor{currentstroke}{rgb}{0.277941,0.056324,0.381191}%
\pgfsetstrokecolor{currentstroke}%
\pgfsetdash{}{0pt}%
\pgfpathmoveto{\pgfqpoint{5.593880in}{2.351748in}}%
\pgfpathlineto{\pgfqpoint{5.543776in}{2.353611in}}%
\pgfusepath{stroke}%
\end{pgfscope}%
\begin{pgfscope}%
\pgfpathrectangle{\pgfqpoint{3.985294in}{1.750000in}}{\pgfqpoint{2.279412in}{2.004545in}}%
\pgfusepath{clip}%
\pgfsetbuttcap%
\pgfsetroundjoin%
\pgfsetlinewidth{0.380754pt}%
\definecolor{currentstroke}{rgb}{0.279566,0.067836,0.391917}%
\pgfsetstrokecolor{currentstroke}%
\pgfsetdash{}{0pt}%
\pgfpathmoveto{\pgfqpoint{5.543776in}{2.353611in}}%
\pgfpathlineto{\pgfqpoint{5.493702in}{2.355988in}}%
\pgfusepath{stroke}%
\end{pgfscope}%
\begin{pgfscope}%
\pgfpathrectangle{\pgfqpoint{3.985294in}{1.750000in}}{\pgfqpoint{2.279412in}{2.004545in}}%
\pgfusepath{clip}%
\pgfsetbuttcap%
\pgfsetroundjoin%
\pgfsetlinewidth{0.398451pt}%
\definecolor{currentstroke}{rgb}{0.281446,0.084320,0.407414}%
\pgfsetstrokecolor{currentstroke}%
\pgfsetdash{}{0pt}%
\pgfpathmoveto{\pgfqpoint{5.493702in}{2.355988in}}%
\pgfpathlineto{\pgfqpoint{5.443671in}{2.359040in}}%
\pgfusepath{stroke}%
\end{pgfscope}%
\begin{pgfscope}%
\pgfpathrectangle{\pgfqpoint{3.985294in}{1.750000in}}{\pgfqpoint{2.279412in}{2.004545in}}%
\pgfusepath{clip}%
\pgfsetbuttcap%
\pgfsetroundjoin%
\pgfsetlinewidth{0.420717pt}%
\definecolor{currentstroke}{rgb}{0.282656,0.100196,0.422160}%
\pgfsetstrokecolor{currentstroke}%
\pgfsetdash{}{0pt}%
\pgfpathmoveto{\pgfqpoint{5.443671in}{2.359040in}}%
\pgfpathlineto{\pgfqpoint{5.393725in}{2.362966in}}%
\pgfusepath{stroke}%
\end{pgfscope}%
\begin{pgfscope}%
\pgfpathrectangle{\pgfqpoint{3.985294in}{1.750000in}}{\pgfqpoint{2.279412in}{2.004545in}}%
\pgfusepath{clip}%
\pgfsetbuttcap%
\pgfsetroundjoin%
\pgfsetlinewidth{0.438557pt}%
\definecolor{currentstroke}{rgb}{0.283197,0.115680,0.436115}%
\pgfsetstrokecolor{currentstroke}%
\pgfsetdash{}{0pt}%
\pgfpathmoveto{\pgfqpoint{5.393725in}{2.362966in}}%
\pgfpathlineto{\pgfqpoint{5.343884in}{2.367855in}}%
\pgfusepath{stroke}%
\end{pgfscope}%
\begin{pgfscope}%
\pgfpathrectangle{\pgfqpoint{3.985294in}{1.750000in}}{\pgfqpoint{2.279412in}{2.004545in}}%
\pgfusepath{clip}%
\pgfsetbuttcap%
\pgfsetroundjoin%
\pgfsetlinewidth{0.454391pt}%
\definecolor{currentstroke}{rgb}{0.283187,0.125848,0.444960}%
\pgfsetstrokecolor{currentstroke}%
\pgfsetdash{}{0pt}%
\pgfpathmoveto{\pgfqpoint{5.343884in}{2.367855in}}%
\pgfpathlineto{\pgfqpoint{5.294206in}{2.373777in}}%
\pgfusepath{stroke}%
\end{pgfscope}%
\begin{pgfscope}%
\pgfpathrectangle{\pgfqpoint{3.985294in}{1.750000in}}{\pgfqpoint{2.279412in}{2.004545in}}%
\pgfusepath{clip}%
\pgfsetbuttcap%
\pgfsetroundjoin%
\pgfsetlinewidth{0.470802pt}%
\definecolor{currentstroke}{rgb}{0.282884,0.135920,0.453427}%
\pgfsetstrokecolor{currentstroke}%
\pgfsetdash{}{0pt}%
\pgfpathmoveto{\pgfqpoint{5.294206in}{2.373777in}}%
\pgfpathlineto{\pgfqpoint{5.244873in}{2.381576in}}%
\pgfusepath{stroke}%
\end{pgfscope}%
\begin{pgfscope}%
\pgfpathrectangle{\pgfqpoint{3.985294in}{1.750000in}}{\pgfqpoint{2.279412in}{2.004545in}}%
\pgfusepath{clip}%
\pgfsetbuttcap%
\pgfsetroundjoin%
\pgfsetlinewidth{0.487341pt}%
\definecolor{currentstroke}{rgb}{0.281887,0.150881,0.465405}%
\pgfsetstrokecolor{currentstroke}%
\pgfsetdash{}{0pt}%
\pgfpathmoveto{\pgfqpoint{5.244873in}{2.381576in}}%
\pgfpathlineto{\pgfqpoint{5.195883in}{2.390968in}}%
\pgfusepath{stroke}%
\end{pgfscope}%
\begin{pgfscope}%
\pgfpathrectangle{\pgfqpoint{3.985294in}{1.750000in}}{\pgfqpoint{2.279412in}{2.004545in}}%
\pgfusepath{clip}%
\pgfsetbuttcap%
\pgfsetroundjoin%
\pgfsetlinewidth{0.319573pt}%
\definecolor{currentstroke}{rgb}{0.269944,0.014625,0.341379}%
\pgfsetstrokecolor{currentstroke}%
\pgfsetdash{}{0pt}%
\pgfpathmoveto{\pgfqpoint{5.894378in}{2.391418in}}%
\pgfpathlineto{\pgfqpoint{5.844619in}{2.396179in}}%
\pgfusepath{stroke}%
\end{pgfscope}%
\begin{pgfscope}%
\pgfpathrectangle{\pgfqpoint{3.985294in}{1.750000in}}{\pgfqpoint{2.279412in}{2.004545in}}%
\pgfusepath{clip}%
\pgfsetbuttcap%
\pgfsetroundjoin%
\pgfsetlinewidth{0.311921pt}%
\definecolor{currentstroke}{rgb}{0.268510,0.009605,0.335427}%
\pgfsetstrokecolor{currentstroke}%
\pgfsetdash{}{0pt}%
\pgfpathmoveto{\pgfqpoint{5.844619in}{2.396179in}}%
\pgfpathlineto{\pgfqpoint{5.794793in}{2.398574in}}%
\pgfusepath{stroke}%
\end{pgfscope}%
\begin{pgfscope}%
\pgfpathrectangle{\pgfqpoint{3.985294in}{1.750000in}}{\pgfqpoint{2.279412in}{2.004545in}}%
\pgfusepath{clip}%
\pgfsetbuttcap%
\pgfsetroundjoin%
\pgfsetlinewidth{0.332302pt}%
\definecolor{currentstroke}{rgb}{0.272594,0.025563,0.353093}%
\pgfsetstrokecolor{currentstroke}%
\pgfsetdash{}{0pt}%
\pgfpathmoveto{\pgfqpoint{5.794793in}{2.398574in}}%
\pgfpathlineto{\pgfqpoint{5.744663in}{2.397946in}}%
\pgfusepath{stroke}%
\end{pgfscope}%
\begin{pgfscope}%
\pgfpathrectangle{\pgfqpoint{3.985294in}{1.750000in}}{\pgfqpoint{2.279412in}{2.004545in}}%
\pgfusepath{clip}%
\pgfsetbuttcap%
\pgfsetroundjoin%
\pgfsetlinewidth{0.325429pt}%
\definecolor{currentstroke}{rgb}{0.271305,0.019942,0.347269}%
\pgfsetstrokecolor{currentstroke}%
\pgfsetdash{}{0pt}%
\pgfpathmoveto{\pgfqpoint{5.744663in}{2.397946in}}%
\pgfpathlineto{\pgfqpoint{5.694525in}{2.398525in}}%
\pgfusepath{stroke}%
\end{pgfscope}%
\begin{pgfscope}%
\pgfpathrectangle{\pgfqpoint{3.985294in}{1.750000in}}{\pgfqpoint{2.279412in}{2.004545in}}%
\pgfusepath{clip}%
\pgfsetbuttcap%
\pgfsetroundjoin%
\pgfsetlinewidth{0.333796pt}%
\definecolor{currentstroke}{rgb}{0.272594,0.025563,0.353093}%
\pgfsetstrokecolor{currentstroke}%
\pgfsetdash{}{0pt}%
\pgfpathmoveto{\pgfqpoint{5.694525in}{2.398525in}}%
\pgfpathlineto{\pgfqpoint{5.644388in}{2.399487in}}%
\pgfusepath{stroke}%
\end{pgfscope}%
\begin{pgfscope}%
\pgfpathrectangle{\pgfqpoint{3.985294in}{1.750000in}}{\pgfqpoint{2.279412in}{2.004545in}}%
\pgfusepath{clip}%
\pgfsetbuttcap%
\pgfsetroundjoin%
\pgfsetlinewidth{0.347659pt}%
\definecolor{currentstroke}{rgb}{0.274952,0.037752,0.364543}%
\pgfsetstrokecolor{currentstroke}%
\pgfsetdash{}{0pt}%
\pgfpathmoveto{\pgfqpoint{5.644388in}{2.399487in}}%
\pgfpathlineto{\pgfqpoint{5.594264in}{2.400743in}}%
\pgfusepath{stroke}%
\end{pgfscope}%
\begin{pgfscope}%
\pgfpathrectangle{\pgfqpoint{3.985294in}{1.750000in}}{\pgfqpoint{2.279412in}{2.004545in}}%
\pgfusepath{clip}%
\pgfsetbuttcap%
\pgfsetroundjoin%
\pgfsetlinewidth{0.371600pt}%
\definecolor{currentstroke}{rgb}{0.278791,0.062145,0.386592}%
\pgfsetstrokecolor{currentstroke}%
\pgfsetdash{}{0pt}%
\pgfpathmoveto{\pgfqpoint{5.594264in}{2.400743in}}%
\pgfpathlineto{\pgfqpoint{5.544157in}{2.402494in}}%
\pgfusepath{stroke}%
\end{pgfscope}%
\begin{pgfscope}%
\pgfpathrectangle{\pgfqpoint{3.985294in}{1.750000in}}{\pgfqpoint{2.279412in}{2.004545in}}%
\pgfusepath{clip}%
\pgfsetbuttcap%
\pgfsetroundjoin%
\pgfsetlinewidth{0.408713pt}%
\definecolor{currentstroke}{rgb}{0.281924,0.089666,0.412415}%
\pgfsetstrokecolor{currentstroke}%
\pgfsetdash{}{0pt}%
\pgfpathmoveto{\pgfqpoint{5.544157in}{2.402494in}}%
\pgfpathlineto{\pgfqpoint{5.494073in}{2.404708in}}%
\pgfusepath{stroke}%
\end{pgfscope}%
\begin{pgfscope}%
\pgfpathrectangle{\pgfqpoint{3.985294in}{1.750000in}}{\pgfqpoint{2.279412in}{2.004545in}}%
\pgfusepath{clip}%
\pgfsetbuttcap%
\pgfsetroundjoin%
\pgfsetlinewidth{0.439982pt}%
\definecolor{currentstroke}{rgb}{0.283197,0.115680,0.436115}%
\pgfsetstrokecolor{currentstroke}%
\pgfsetdash{}{0pt}%
\pgfpathmoveto{\pgfqpoint{5.494073in}{2.404708in}}%
\pgfpathlineto{\pgfqpoint{5.444023in}{2.407507in}}%
\pgfusepath{stroke}%
\end{pgfscope}%
\begin{pgfscope}%
\pgfpathrectangle{\pgfqpoint{3.985294in}{1.750000in}}{\pgfqpoint{2.279412in}{2.004545in}}%
\pgfusepath{clip}%
\pgfsetbuttcap%
\pgfsetroundjoin%
\pgfsetlinewidth{0.482866pt}%
\definecolor{currentstroke}{rgb}{0.282290,0.145912,0.461510}%
\pgfsetstrokecolor{currentstroke}%
\pgfsetdash{}{0pt}%
\pgfpathmoveto{\pgfqpoint{5.444023in}{2.407507in}}%
\pgfpathlineto{\pgfqpoint{5.394017in}{2.410836in}}%
\pgfusepath{stroke}%
\end{pgfscope}%
\begin{pgfscope}%
\pgfpathrectangle{\pgfqpoint{3.985294in}{1.750000in}}{\pgfqpoint{2.279412in}{2.004545in}}%
\pgfusepath{clip}%
\pgfsetbuttcap%
\pgfsetroundjoin%
\pgfsetlinewidth{0.488200pt}%
\definecolor{currentstroke}{rgb}{0.281887,0.150881,0.465405}%
\pgfsetstrokecolor{currentstroke}%
\pgfsetdash{}{0pt}%
\pgfpathmoveto{\pgfqpoint{5.394017in}{2.410836in}}%
\pgfpathlineto{\pgfqpoint{5.344141in}{2.415355in}}%
\pgfusepath{stroke}%
\end{pgfscope}%
\begin{pgfscope}%
\pgfpathrectangle{\pgfqpoint{3.985294in}{1.750000in}}{\pgfqpoint{2.279412in}{2.004545in}}%
\pgfusepath{clip}%
\pgfsetbuttcap%
\pgfsetroundjoin%
\pgfsetlinewidth{0.533314pt}%
\definecolor{currentstroke}{rgb}{0.278012,0.180367,0.486697}%
\pgfsetstrokecolor{currentstroke}%
\pgfsetdash{}{0pt}%
\pgfpathmoveto{\pgfqpoint{5.344141in}{2.415355in}}%
\pgfpathlineto{\pgfqpoint{5.294420in}{2.421083in}}%
\pgfusepath{stroke}%
\end{pgfscope}%
\begin{pgfscope}%
\pgfpathrectangle{\pgfqpoint{3.985294in}{1.750000in}}{\pgfqpoint{2.279412in}{2.004545in}}%
\pgfusepath{clip}%
\pgfsetbuttcap%
\pgfsetroundjoin%
\pgfsetlinewidth{0.539498pt}%
\definecolor{currentstroke}{rgb}{0.277134,0.185228,0.489898}%
\pgfsetstrokecolor{currentstroke}%
\pgfsetdash{}{0pt}%
\pgfpathmoveto{\pgfqpoint{5.294420in}{2.421083in}}%
\pgfpathlineto{\pgfqpoint{5.244956in}{2.428264in}}%
\pgfusepath{stroke}%
\end{pgfscope}%
\begin{pgfscope}%
\pgfpathrectangle{\pgfqpoint{3.985294in}{1.750000in}}{\pgfqpoint{2.279412in}{2.004545in}}%
\pgfusepath{clip}%
\pgfsetbuttcap%
\pgfsetroundjoin%
\pgfsetlinewidth{0.565446pt}%
\definecolor{currentstroke}{rgb}{0.273006,0.204520,0.501721}%
\pgfsetstrokecolor{currentstroke}%
\pgfsetdash{}{0pt}%
\pgfpathmoveto{\pgfqpoint{5.244956in}{2.428264in}}%
\pgfpathlineto{\pgfqpoint{5.195904in}{2.437394in}}%
\pgfusepath{stroke}%
\end{pgfscope}%
\begin{pgfscope}%
\pgfpathrectangle{\pgfqpoint{3.985294in}{1.750000in}}{\pgfqpoint{2.279412in}{2.004545in}}%
\pgfusepath{clip}%
\pgfsetbuttcap%
\pgfsetroundjoin%
\pgfsetlinewidth{0.602908pt}%
\definecolor{currentstroke}{rgb}{0.266580,0.228262,0.514349}%
\pgfsetstrokecolor{currentstroke}%
\pgfsetdash{}{0pt}%
\pgfpathmoveto{\pgfqpoint{5.195904in}{2.437394in}}%
\pgfpathlineto{\pgfqpoint{5.147361in}{2.448423in}}%
\pgfusepath{stroke}%
\end{pgfscope}%
\begin{pgfscope}%
\pgfpathrectangle{\pgfqpoint{3.985294in}{1.750000in}}{\pgfqpoint{2.279412in}{2.004545in}}%
\pgfusepath{clip}%
\pgfsetbuttcap%
\pgfsetroundjoin%
\pgfsetlinewidth{0.596666pt}%
\definecolor{currentstroke}{rgb}{0.266580,0.228262,0.514349}%
\pgfsetstrokecolor{currentstroke}%
\pgfsetdash{}{0pt}%
\pgfpathmoveto{\pgfqpoint{5.147361in}{2.448423in}}%
\pgfpathlineto{\pgfqpoint{5.099502in}{2.461538in}}%
\pgfusepath{stroke}%
\end{pgfscope}%
\begin{pgfscope}%
\pgfpathrectangle{\pgfqpoint{3.985294in}{1.750000in}}{\pgfqpoint{2.279412in}{2.004545in}}%
\pgfusepath{clip}%
\pgfsetbuttcap%
\pgfsetroundjoin%
\pgfsetlinewidth{0.625277pt}%
\definecolor{currentstroke}{rgb}{0.260571,0.246922,0.522828}%
\pgfsetstrokecolor{currentstroke}%
\pgfsetdash{}{0pt}%
\pgfpathmoveto{\pgfqpoint{5.099502in}{2.461538in}}%
\pgfpathlineto{\pgfqpoint{5.052537in}{2.476932in}}%
\pgfusepath{stroke}%
\end{pgfscope}%
\begin{pgfscope}%
\pgfpathrectangle{\pgfqpoint{3.985294in}{1.750000in}}{\pgfqpoint{2.279412in}{2.004545in}}%
\pgfusepath{clip}%
\pgfsetbuttcap%
\pgfsetroundjoin%
\pgfsetlinewidth{0.646195pt}%
\definecolor{currentstroke}{rgb}{0.255645,0.260703,0.528312}%
\pgfsetstrokecolor{currentstroke}%
\pgfsetdash{}{0pt}%
\pgfpathmoveto{\pgfqpoint{5.052537in}{2.476932in}}%
\pgfpathlineto{\pgfqpoint{5.007165in}{2.495574in}}%
\pgfusepath{stroke}%
\end{pgfscope}%
\begin{pgfscope}%
\pgfpathrectangle{\pgfqpoint{3.985294in}{1.750000in}}{\pgfqpoint{2.279412in}{2.004545in}}%
\pgfusepath{clip}%
\pgfsetbuttcap%
\pgfsetroundjoin%
\pgfsetlinewidth{0.664952pt}%
\definecolor{currentstroke}{rgb}{0.250425,0.274290,0.533103}%
\pgfsetstrokecolor{currentstroke}%
\pgfsetdash{}{0pt}%
\pgfpathmoveto{\pgfqpoint{5.007165in}{2.495574in}}%
\pgfpathlineto{\pgfqpoint{4.963175in}{2.516687in}}%
\pgfusepath{stroke}%
\end{pgfscope}%
\begin{pgfscope}%
\pgfpathrectangle{\pgfqpoint{3.985294in}{1.750000in}}{\pgfqpoint{2.279412in}{2.004545in}}%
\pgfusepath{clip}%
\pgfsetbuttcap%
\pgfsetroundjoin%
\pgfsetlinewidth{0.308846pt}%
\definecolor{currentstroke}{rgb}{0.268510,0.009605,0.335427}%
\pgfsetstrokecolor{currentstroke}%
\pgfsetdash{}{0pt}%
\pgfpathmoveto{\pgfqpoint{5.894378in}{2.571846in}}%
\pgfpathlineto{\pgfqpoint{5.845686in}{2.574329in}}%
\pgfusepath{stroke}%
\end{pgfscope}%
\begin{pgfscope}%
\pgfpathrectangle{\pgfqpoint{3.985294in}{1.750000in}}{\pgfqpoint{2.279412in}{2.004545in}}%
\pgfusepath{clip}%
\pgfsetbuttcap%
\pgfsetroundjoin%
\pgfsetlinewidth{0.315924pt}%
\definecolor{currentstroke}{rgb}{0.269944,0.014625,0.341379}%
\pgfsetstrokecolor{currentstroke}%
\pgfsetdash{}{0pt}%
\pgfpathmoveto{\pgfqpoint{5.845686in}{2.574329in}}%
\pgfpathlineto{\pgfqpoint{5.795950in}{2.577618in}}%
\pgfusepath{stroke}%
\end{pgfscope}%
\begin{pgfscope}%
\pgfpathrectangle{\pgfqpoint{3.985294in}{1.750000in}}{\pgfqpoint{2.279412in}{2.004545in}}%
\pgfusepath{clip}%
\pgfsetbuttcap%
\pgfsetroundjoin%
\pgfsetlinewidth{0.323387pt}%
\definecolor{currentstroke}{rgb}{0.271305,0.019942,0.347269}%
\pgfsetstrokecolor{currentstroke}%
\pgfsetdash{}{0pt}%
\pgfpathmoveto{\pgfqpoint{5.795950in}{2.577618in}}%
\pgfpathlineto{\pgfqpoint{5.745840in}{2.577183in}}%
\pgfusepath{stroke}%
\end{pgfscope}%
\begin{pgfscope}%
\pgfpathrectangle{\pgfqpoint{3.985294in}{1.750000in}}{\pgfqpoint{2.279412in}{2.004545in}}%
\pgfusepath{clip}%
\pgfsetbuttcap%
\pgfsetroundjoin%
\pgfsetlinewidth{0.325670pt}%
\definecolor{currentstroke}{rgb}{0.271305,0.019942,0.347269}%
\pgfsetstrokecolor{currentstroke}%
\pgfsetdash{}{0pt}%
\pgfpathmoveto{\pgfqpoint{5.745840in}{2.577183in}}%
\pgfpathlineto{\pgfqpoint{5.695739in}{2.577382in}}%
\pgfusepath{stroke}%
\end{pgfscope}%
\begin{pgfscope}%
\pgfpathrectangle{\pgfqpoint{3.985294in}{1.750000in}}{\pgfqpoint{2.279412in}{2.004545in}}%
\pgfusepath{clip}%
\pgfsetbuttcap%
\pgfsetroundjoin%
\pgfsetlinewidth{0.346897pt}%
\definecolor{currentstroke}{rgb}{0.274952,0.037752,0.364543}%
\pgfsetstrokecolor{currentstroke}%
\pgfsetdash{}{0pt}%
\pgfpathmoveto{\pgfqpoint{5.695739in}{2.577382in}}%
\pgfpathlineto{\pgfqpoint{5.645601in}{2.578221in}}%
\pgfusepath{stroke}%
\end{pgfscope}%
\begin{pgfscope}%
\pgfpathrectangle{\pgfqpoint{3.985294in}{1.750000in}}{\pgfqpoint{2.279412in}{2.004545in}}%
\pgfusepath{clip}%
\pgfsetbuttcap%
\pgfsetroundjoin%
\pgfsetlinewidth{0.380817pt}%
\definecolor{currentstroke}{rgb}{0.279566,0.067836,0.391917}%
\pgfsetstrokecolor{currentstroke}%
\pgfsetdash{}{0pt}%
\pgfpathmoveto{\pgfqpoint{5.645601in}{2.578221in}}%
\pgfpathlineto{\pgfqpoint{5.595458in}{2.578988in}}%
\pgfusepath{stroke}%
\end{pgfscope}%
\begin{pgfscope}%
\pgfpathrectangle{\pgfqpoint{3.985294in}{1.750000in}}{\pgfqpoint{2.279412in}{2.004545in}}%
\pgfusepath{clip}%
\pgfsetbuttcap%
\pgfsetroundjoin%
\pgfsetlinewidth{0.410853pt}%
\definecolor{currentstroke}{rgb}{0.281924,0.089666,0.412415}%
\pgfsetstrokecolor{currentstroke}%
\pgfsetdash{}{0pt}%
\pgfpathmoveto{\pgfqpoint{5.595458in}{2.578988in}}%
\pgfpathlineto{\pgfqpoint{5.545322in}{2.580084in}}%
\pgfusepath{stroke}%
\end{pgfscope}%
\begin{pgfscope}%
\pgfpathrectangle{\pgfqpoint{3.985294in}{1.750000in}}{\pgfqpoint{2.279412in}{2.004545in}}%
\pgfusepath{clip}%
\pgfsetbuttcap%
\pgfsetroundjoin%
\pgfsetlinewidth{0.470508pt}%
\definecolor{currentstroke}{rgb}{0.282884,0.135920,0.453427}%
\pgfsetstrokecolor{currentstroke}%
\pgfsetdash{}{0pt}%
\pgfpathmoveto{\pgfqpoint{5.545322in}{2.580084in}}%
\pgfpathlineto{\pgfqpoint{5.495188in}{2.581256in}}%
\pgfusepath{stroke}%
\end{pgfscope}%
\begin{pgfscope}%
\pgfpathrectangle{\pgfqpoint{3.985294in}{1.750000in}}{\pgfqpoint{2.279412in}{2.004545in}}%
\pgfusepath{clip}%
\pgfsetbuttcap%
\pgfsetroundjoin%
\pgfsetlinewidth{0.531884pt}%
\definecolor{currentstroke}{rgb}{0.278012,0.180367,0.486697}%
\pgfsetstrokecolor{currentstroke}%
\pgfsetdash{}{0pt}%
\pgfpathmoveto{\pgfqpoint{5.495188in}{2.581256in}}%
\pgfpathlineto{\pgfqpoint{5.445061in}{2.582649in}}%
\pgfusepath{stroke}%
\end{pgfscope}%
\begin{pgfscope}%
\pgfpathrectangle{\pgfqpoint{3.985294in}{1.750000in}}{\pgfqpoint{2.279412in}{2.004545in}}%
\pgfusepath{clip}%
\pgfsetbuttcap%
\pgfsetroundjoin%
\pgfsetlinewidth{0.628070pt}%
\definecolor{currentstroke}{rgb}{0.260571,0.246922,0.522828}%
\pgfsetstrokecolor{currentstroke}%
\pgfsetdash{}{0pt}%
\pgfpathmoveto{\pgfqpoint{5.445061in}{2.582649in}}%
\pgfpathlineto{\pgfqpoint{5.394947in}{2.584338in}}%
\pgfusepath{stroke}%
\end{pgfscope}%
\begin{pgfscope}%
\pgfpathrectangle{\pgfqpoint{3.985294in}{1.750000in}}{\pgfqpoint{2.279412in}{2.004545in}}%
\pgfusepath{clip}%
\pgfsetbuttcap%
\pgfsetroundjoin%
\pgfsetlinewidth{0.706990pt}%
\definecolor{currentstroke}{rgb}{0.239346,0.300855,0.540844}%
\pgfsetstrokecolor{currentstroke}%
\pgfsetdash{}{0pt}%
\pgfpathmoveto{\pgfqpoint{5.394947in}{2.584338in}}%
\pgfpathlineto{\pgfqpoint{5.344854in}{2.586444in}}%
\pgfusepath{stroke}%
\end{pgfscope}%
\begin{pgfscope}%
\pgfpathrectangle{\pgfqpoint{3.985294in}{1.750000in}}{\pgfqpoint{2.279412in}{2.004545in}}%
\pgfusepath{clip}%
\pgfsetbuttcap%
\pgfsetroundjoin%
\pgfsetlinewidth{0.788378pt}%
\definecolor{currentstroke}{rgb}{0.216210,0.351535,0.550627}%
\pgfsetstrokecolor{currentstroke}%
\pgfsetdash{}{0pt}%
\pgfpathmoveto{\pgfqpoint{5.344854in}{2.586444in}}%
\pgfpathlineto{\pgfqpoint{5.294802in}{2.589195in}}%
\pgfusepath{stroke}%
\end{pgfscope}%
\begin{pgfscope}%
\pgfpathrectangle{\pgfqpoint{3.985294in}{1.750000in}}{\pgfqpoint{2.279412in}{2.004545in}}%
\pgfusepath{clip}%
\pgfsetbuttcap%
\pgfsetroundjoin%
\pgfsetlinewidth{0.840385pt}%
\definecolor{currentstroke}{rgb}{0.203063,0.379716,0.553925}%
\pgfsetstrokecolor{currentstroke}%
\pgfsetdash{}{0pt}%
\pgfpathmoveto{\pgfqpoint{5.294802in}{2.589195in}}%
\pgfpathlineto{\pgfqpoint{5.244798in}{2.592562in}}%
\pgfusepath{stroke}%
\end{pgfscope}%
\begin{pgfscope}%
\pgfpathrectangle{\pgfqpoint{3.985294in}{1.750000in}}{\pgfqpoint{2.279412in}{2.004545in}}%
\pgfusepath{clip}%
\pgfsetbuttcap%
\pgfsetroundjoin%
\pgfsetlinewidth{0.876851pt}%
\definecolor{currentstroke}{rgb}{0.194100,0.399323,0.555565}%
\pgfsetstrokecolor{currentstroke}%
\pgfsetdash{}{0pt}%
\pgfpathmoveto{\pgfqpoint{5.244798in}{2.592562in}}%
\pgfpathlineto{\pgfqpoint{5.194869in}{2.596686in}}%
\pgfusepath{stroke}%
\end{pgfscope}%
\begin{pgfscope}%
\pgfpathrectangle{\pgfqpoint{3.985294in}{1.750000in}}{\pgfqpoint{2.279412in}{2.004545in}}%
\pgfusepath{clip}%
\pgfsetbuttcap%
\pgfsetroundjoin%
\pgfsetlinewidth{0.961658pt}%
\definecolor{currentstroke}{rgb}{0.172719,0.448791,0.557885}%
\pgfsetstrokecolor{currentstroke}%
\pgfsetdash{}{0pt}%
\pgfpathmoveto{\pgfqpoint{5.194869in}{2.596686in}}%
\pgfpathlineto{\pgfqpoint{5.145044in}{2.601671in}}%
\pgfusepath{stroke}%
\end{pgfscope}%
\begin{pgfscope}%
\pgfpathrectangle{\pgfqpoint{3.985294in}{1.750000in}}{\pgfqpoint{2.279412in}{2.004545in}}%
\pgfusepath{clip}%
\pgfsetbuttcap%
\pgfsetroundjoin%
\pgfsetlinewidth{0.953343pt}%
\definecolor{currentstroke}{rgb}{0.174274,0.445044,0.557792}%
\pgfsetstrokecolor{currentstroke}%
\pgfsetdash{}{0pt}%
\pgfpathmoveto{\pgfqpoint{5.145044in}{2.601671in}}%
\pgfpathlineto{\pgfqpoint{5.095344in}{2.607520in}}%
\pgfusepath{stroke}%
\end{pgfscope}%
\begin{pgfscope}%
\pgfpathrectangle{\pgfqpoint{3.985294in}{1.750000in}}{\pgfqpoint{2.279412in}{2.004545in}}%
\pgfusepath{clip}%
\pgfsetbuttcap%
\pgfsetroundjoin%
\pgfsetlinewidth{0.919277pt}%
\definecolor{currentstroke}{rgb}{0.182256,0.426184,0.557120}%
\pgfsetstrokecolor{currentstroke}%
\pgfsetdash{}{0pt}%
\pgfpathmoveto{\pgfqpoint{5.095344in}{2.607520in}}%
\pgfpathlineto{\pgfqpoint{5.045797in}{2.614313in}}%
\pgfusepath{stroke}%
\end{pgfscope}%
\begin{pgfscope}%
\pgfpathrectangle{\pgfqpoint{3.985294in}{1.750000in}}{\pgfqpoint{2.279412in}{2.004545in}}%
\pgfusepath{clip}%
\pgfsetbuttcap%
\pgfsetroundjoin%
\pgfsetlinewidth{0.891168pt}%
\definecolor{currentstroke}{rgb}{0.190631,0.407061,0.556089}%
\pgfsetstrokecolor{currentstroke}%
\pgfsetdash{}{0pt}%
\pgfpathmoveto{\pgfqpoint{5.045797in}{2.614313in}}%
\pgfpathlineto{\pgfqpoint{4.996436in}{2.622068in}}%
\pgfusepath{stroke}%
\end{pgfscope}%
\begin{pgfscope}%
\pgfpathrectangle{\pgfqpoint{3.985294in}{1.750000in}}{\pgfqpoint{2.279412in}{2.004545in}}%
\pgfusepath{clip}%
\pgfsetbuttcap%
\pgfsetroundjoin%
\pgfsetlinewidth{0.880827pt}%
\definecolor{currentstroke}{rgb}{0.192357,0.403199,0.555836}%
\pgfsetstrokecolor{currentstroke}%
\pgfsetdash{}{0pt}%
\pgfpathmoveto{\pgfqpoint{4.996436in}{2.622068in}}%
\pgfpathlineto{\pgfqpoint{4.947335in}{2.630991in}}%
\pgfusepath{stroke}%
\end{pgfscope}%
\begin{pgfscope}%
\pgfpathrectangle{\pgfqpoint{3.985294in}{1.750000in}}{\pgfqpoint{2.279412in}{2.004545in}}%
\pgfusepath{clip}%
\pgfsetbuttcap%
\pgfsetroundjoin%
\pgfsetlinewidth{0.874269pt}%
\definecolor{currentstroke}{rgb}{0.194100,0.399323,0.555565}%
\pgfsetstrokecolor{currentstroke}%
\pgfsetdash{}{0pt}%
\pgfpathmoveto{\pgfqpoint{4.947335in}{2.630991in}}%
\pgfpathlineto{\pgfqpoint{4.898503in}{2.640992in}}%
\pgfusepath{stroke}%
\end{pgfscope}%
\begin{pgfscope}%
\pgfpathrectangle{\pgfqpoint{3.985294in}{1.750000in}}{\pgfqpoint{2.279412in}{2.004545in}}%
\pgfusepath{clip}%
\pgfsetbuttcap%
\pgfsetroundjoin%
\pgfsetlinewidth{0.868195pt}%
\definecolor{currentstroke}{rgb}{0.195860,0.395433,0.555276}%
\pgfsetstrokecolor{currentstroke}%
\pgfsetdash{}{0pt}%
\pgfpathmoveto{\pgfqpoint{4.898503in}{2.640992in}}%
\pgfpathlineto{\pgfqpoint{4.849965in}{2.652011in}}%
\pgfusepath{stroke}%
\end{pgfscope}%
\begin{pgfscope}%
\pgfpathrectangle{\pgfqpoint{3.985294in}{1.750000in}}{\pgfqpoint{2.279412in}{2.004545in}}%
\pgfusepath{clip}%
\pgfsetbuttcap%
\pgfsetroundjoin%
\pgfsetlinewidth{0.799530pt}%
\definecolor{currentstroke}{rgb}{0.214298,0.355619,0.551184}%
\pgfsetstrokecolor{currentstroke}%
\pgfsetdash{}{0pt}%
\pgfpathmoveto{\pgfqpoint{4.849965in}{2.652011in}}%
\pgfpathlineto{\pgfqpoint{4.802025in}{2.664745in}}%
\pgfusepath{stroke}%
\end{pgfscope}%
\begin{pgfscope}%
\pgfpathrectangle{\pgfqpoint{3.985294in}{1.750000in}}{\pgfqpoint{2.279412in}{2.004545in}}%
\pgfusepath{clip}%
\pgfsetbuttcap%
\pgfsetroundjoin%
\pgfsetlinewidth{0.681505pt}%
\definecolor{currentstroke}{rgb}{0.246811,0.283237,0.535941}%
\pgfsetstrokecolor{currentstroke}%
\pgfsetdash{}{0pt}%
\pgfpathmoveto{\pgfqpoint{4.802025in}{2.664745in}}%
\pgfpathlineto{\pgfqpoint{4.754940in}{2.679019in}}%
\pgfusepath{stroke}%
\end{pgfscope}%
\begin{pgfscope}%
\pgfpathrectangle{\pgfqpoint{3.985294in}{1.750000in}}{\pgfqpoint{2.279412in}{2.004545in}}%
\pgfusepath{clip}%
\pgfsetbuttcap%
\pgfsetroundjoin%
\pgfsetlinewidth{0.683668pt}%
\definecolor{currentstroke}{rgb}{0.246811,0.283237,0.535941}%
\pgfsetstrokecolor{currentstroke}%
\pgfsetdash{}{0pt}%
\pgfpathmoveto{\pgfqpoint{4.754940in}{2.679019in}}%
\pgfpathlineto{\pgfqpoint{4.754940in}{2.679019in}}%
\pgfusepath{stroke}%
\end{pgfscope}%
\begin{pgfscope}%
\pgfpathrectangle{\pgfqpoint{3.985294in}{1.750000in}}{\pgfqpoint{2.279412in}{2.004545in}}%
\pgfusepath{clip}%
\pgfsetbuttcap%
\pgfsetroundjoin%
\pgfsetlinewidth{0.320760pt}%
\definecolor{currentstroke}{rgb}{0.269944,0.014625,0.341379}%
\pgfsetstrokecolor{currentstroke}%
\pgfsetdash{}{0pt}%
\pgfpathmoveto{\pgfqpoint{5.894378in}{2.616952in}}%
\pgfpathlineto{\pgfqpoint{5.844272in}{2.615712in}}%
\pgfusepath{stroke}%
\end{pgfscope}%
\begin{pgfscope}%
\pgfpathrectangle{\pgfqpoint{3.985294in}{1.750000in}}{\pgfqpoint{2.279412in}{2.004545in}}%
\pgfusepath{clip}%
\pgfsetbuttcap%
\pgfsetroundjoin%
\pgfsetlinewidth{0.317877pt}%
\definecolor{currentstroke}{rgb}{0.269944,0.014625,0.341379}%
\pgfsetstrokecolor{currentstroke}%
\pgfsetdash{}{0pt}%
\pgfpathmoveto{\pgfqpoint{5.844272in}{2.615712in}}%
\pgfpathlineto{\pgfqpoint{5.794138in}{2.615386in}}%
\pgfusepath{stroke}%
\end{pgfscope}%
\begin{pgfscope}%
\pgfpathrectangle{\pgfqpoint{3.985294in}{1.750000in}}{\pgfqpoint{2.279412in}{2.004545in}}%
\pgfusepath{clip}%
\pgfsetbuttcap%
\pgfsetroundjoin%
\pgfsetlinewidth{0.330738pt}%
\definecolor{currentstroke}{rgb}{0.272594,0.025563,0.353093}%
\pgfsetstrokecolor{currentstroke}%
\pgfsetdash{}{0pt}%
\pgfpathmoveto{\pgfqpoint{5.794138in}{2.615386in}}%
\pgfpathlineto{\pgfqpoint{5.743997in}{2.615483in}}%
\pgfusepath{stroke}%
\end{pgfscope}%
\begin{pgfscope}%
\pgfpathrectangle{\pgfqpoint{3.985294in}{1.750000in}}{\pgfqpoint{2.279412in}{2.004545in}}%
\pgfusepath{clip}%
\pgfsetbuttcap%
\pgfsetroundjoin%
\pgfsetlinewidth{0.332748pt}%
\definecolor{currentstroke}{rgb}{0.272594,0.025563,0.353093}%
\pgfsetstrokecolor{currentstroke}%
\pgfsetdash{}{0pt}%
\pgfpathmoveto{\pgfqpoint{5.743997in}{2.615483in}}%
\pgfpathlineto{\pgfqpoint{5.693855in}{2.616023in}}%
\pgfusepath{stroke}%
\end{pgfscope}%
\begin{pgfscope}%
\pgfpathrectangle{\pgfqpoint{3.985294in}{1.750000in}}{\pgfqpoint{2.279412in}{2.004545in}}%
\pgfusepath{clip}%
\pgfsetbuttcap%
\pgfsetroundjoin%
\pgfsetlinewidth{0.348199pt}%
\definecolor{currentstroke}{rgb}{0.274952,0.037752,0.364543}%
\pgfsetstrokecolor{currentstroke}%
\pgfsetdash{}{0pt}%
\pgfpathmoveto{\pgfqpoint{5.693855in}{2.616023in}}%
\pgfpathlineto{\pgfqpoint{5.643716in}{2.616966in}}%
\pgfusepath{stroke}%
\end{pgfscope}%
\begin{pgfscope}%
\pgfpathrectangle{\pgfqpoint{3.985294in}{1.750000in}}{\pgfqpoint{2.279412in}{2.004545in}}%
\pgfusepath{clip}%
\pgfsetbuttcap%
\pgfsetroundjoin%
\pgfsetlinewidth{0.379817pt}%
\definecolor{currentstroke}{rgb}{0.279566,0.067836,0.391917}%
\pgfsetstrokecolor{currentstroke}%
\pgfsetdash{}{0pt}%
\pgfpathmoveto{\pgfqpoint{5.643716in}{2.616966in}}%
\pgfpathlineto{\pgfqpoint{5.593571in}{2.617655in}}%
\pgfusepath{stroke}%
\end{pgfscope}%
\begin{pgfscope}%
\pgfpathrectangle{\pgfqpoint{3.985294in}{1.750000in}}{\pgfqpoint{2.279412in}{2.004545in}}%
\pgfusepath{clip}%
\pgfsetbuttcap%
\pgfsetroundjoin%
\pgfsetlinewidth{0.424067pt}%
\definecolor{currentstroke}{rgb}{0.282656,0.100196,0.422160}%
\pgfsetstrokecolor{currentstroke}%
\pgfsetdash{}{0pt}%
\pgfpathmoveto{\pgfqpoint{5.593571in}{2.617655in}}%
\pgfpathlineto{\pgfqpoint{5.543426in}{2.618294in}}%
\pgfusepath{stroke}%
\end{pgfscope}%
\begin{pgfscope}%
\pgfpathrectangle{\pgfqpoint{3.985294in}{1.750000in}}{\pgfqpoint{2.279412in}{2.004545in}}%
\pgfusepath{clip}%
\pgfsetbuttcap%
\pgfsetroundjoin%
\pgfsetlinewidth{0.488655pt}%
\definecolor{currentstroke}{rgb}{0.281887,0.150881,0.465405}%
\pgfsetstrokecolor{currentstroke}%
\pgfsetdash{}{0pt}%
\pgfpathmoveto{\pgfqpoint{5.543426in}{2.618294in}}%
\pgfpathlineto{\pgfqpoint{5.493283in}{2.619078in}}%
\pgfusepath{stroke}%
\end{pgfscope}%
\begin{pgfscope}%
\pgfpathrectangle{\pgfqpoint{3.985294in}{1.750000in}}{\pgfqpoint{2.279412in}{2.004545in}}%
\pgfusepath{clip}%
\pgfsetbuttcap%
\pgfsetroundjoin%
\pgfsetlinewidth{0.558487pt}%
\definecolor{currentstroke}{rgb}{0.274128,0.199721,0.498911}%
\pgfsetstrokecolor{currentstroke}%
\pgfsetdash{}{0pt}%
\pgfpathmoveto{\pgfqpoint{5.493283in}{2.619078in}}%
\pgfpathlineto{\pgfqpoint{5.443146in}{2.620082in}}%
\pgfusepath{stroke}%
\end{pgfscope}%
\begin{pgfscope}%
\pgfpathrectangle{\pgfqpoint{3.985294in}{1.750000in}}{\pgfqpoint{2.279412in}{2.004545in}}%
\pgfusepath{clip}%
\pgfsetbuttcap%
\pgfsetroundjoin%
\pgfsetlinewidth{0.645230pt}%
\definecolor{currentstroke}{rgb}{0.255645,0.260703,0.528312}%
\pgfsetstrokecolor{currentstroke}%
\pgfsetdash{}{0pt}%
\pgfpathmoveto{\pgfqpoint{5.443146in}{2.620082in}}%
\pgfpathlineto{\pgfqpoint{5.393014in}{2.621298in}}%
\pgfusepath{stroke}%
\end{pgfscope}%
\begin{pgfscope}%
\pgfpathrectangle{\pgfqpoint{3.985294in}{1.750000in}}{\pgfqpoint{2.279412in}{2.004545in}}%
\pgfusepath{clip}%
\pgfsetbuttcap%
\pgfsetroundjoin%
\pgfsetlinewidth{0.734053pt}%
\definecolor{currentstroke}{rgb}{0.231674,0.318106,0.544834}%
\pgfsetstrokecolor{currentstroke}%
\pgfsetdash{}{0pt}%
\pgfpathmoveto{\pgfqpoint{5.393014in}{2.621298in}}%
\pgfpathlineto{\pgfqpoint{5.342893in}{2.622819in}}%
\pgfusepath{stroke}%
\end{pgfscope}%
\begin{pgfscope}%
\pgfpathrectangle{\pgfqpoint{3.985294in}{1.750000in}}{\pgfqpoint{2.279412in}{2.004545in}}%
\pgfusepath{clip}%
\pgfsetbuttcap%
\pgfsetroundjoin%
\pgfsetlinewidth{0.821126pt}%
\definecolor{currentstroke}{rgb}{0.208623,0.367752,0.552675}%
\pgfsetstrokecolor{currentstroke}%
\pgfsetdash{}{0pt}%
\pgfpathmoveto{\pgfqpoint{5.342893in}{2.622819in}}%
\pgfpathlineto{\pgfqpoint{5.292789in}{2.624720in}}%
\pgfusepath{stroke}%
\end{pgfscope}%
\begin{pgfscope}%
\pgfpathrectangle{\pgfqpoint{3.985294in}{1.750000in}}{\pgfqpoint{2.279412in}{2.004545in}}%
\pgfusepath{clip}%
\pgfsetbuttcap%
\pgfsetroundjoin%
\pgfsetlinewidth{0.894935pt}%
\definecolor{currentstroke}{rgb}{0.188923,0.410910,0.556326}%
\pgfsetstrokecolor{currentstroke}%
\pgfsetdash{}{0pt}%
\pgfpathmoveto{\pgfqpoint{5.292789in}{2.624720in}}%
\pgfpathlineto{\pgfqpoint{5.242708in}{2.627029in}}%
\pgfusepath{stroke}%
\end{pgfscope}%
\begin{pgfscope}%
\pgfpathrectangle{\pgfqpoint{3.985294in}{1.750000in}}{\pgfqpoint{2.279412in}{2.004545in}}%
\pgfusepath{clip}%
\pgfsetbuttcap%
\pgfsetroundjoin%
\pgfsetlinewidth{0.317180pt}%
\definecolor{currentstroke}{rgb}{0.269944,0.014625,0.341379}%
\pgfsetstrokecolor{currentstroke}%
\pgfsetdash{}{0pt}%
\pgfpathmoveto{\pgfqpoint{5.894378in}{2.707166in}}%
\pgfpathlineto{\pgfqpoint{5.845085in}{2.706562in}}%
\pgfusepath{stroke}%
\end{pgfscope}%
\begin{pgfscope}%
\pgfpathrectangle{\pgfqpoint{3.985294in}{1.750000in}}{\pgfqpoint{2.279412in}{2.004545in}}%
\pgfusepath{clip}%
\pgfsetbuttcap%
\pgfsetroundjoin%
\pgfsetlinewidth{0.318567pt}%
\definecolor{currentstroke}{rgb}{0.269944,0.014625,0.341379}%
\pgfsetstrokecolor{currentstroke}%
\pgfsetdash{}{0pt}%
\pgfpathmoveto{\pgfqpoint{5.845085in}{2.706562in}}%
\pgfpathlineto{\pgfqpoint{5.796116in}{2.706901in}}%
\pgfusepath{stroke}%
\end{pgfscope}%
\begin{pgfscope}%
\pgfpathrectangle{\pgfqpoint{3.985294in}{1.750000in}}{\pgfqpoint{2.279412in}{2.004545in}}%
\pgfusepath{clip}%
\pgfsetbuttcap%
\pgfsetroundjoin%
\pgfsetlinewidth{0.326923pt}%
\definecolor{currentstroke}{rgb}{0.271305,0.019942,0.347269}%
\pgfsetstrokecolor{currentstroke}%
\pgfsetdash{}{0pt}%
\pgfpathmoveto{\pgfqpoint{5.796116in}{2.706901in}}%
\pgfpathlineto{\pgfqpoint{5.746047in}{2.707629in}}%
\pgfusepath{stroke}%
\end{pgfscope}%
\begin{pgfscope}%
\pgfpathrectangle{\pgfqpoint{3.985294in}{1.750000in}}{\pgfqpoint{2.279412in}{2.004545in}}%
\pgfusepath{clip}%
\pgfsetbuttcap%
\pgfsetroundjoin%
\pgfsetlinewidth{0.331982pt}%
\definecolor{currentstroke}{rgb}{0.272594,0.025563,0.353093}%
\pgfsetstrokecolor{currentstroke}%
\pgfsetdash{}{0pt}%
\pgfpathmoveto{\pgfqpoint{5.746047in}{2.707629in}}%
\pgfpathlineto{\pgfqpoint{5.695916in}{2.707102in}}%
\pgfusepath{stroke}%
\end{pgfscope}%
\begin{pgfscope}%
\pgfpathrectangle{\pgfqpoint{3.985294in}{1.750000in}}{\pgfqpoint{2.279412in}{2.004545in}}%
\pgfusepath{clip}%
\pgfsetbuttcap%
\pgfsetroundjoin%
\pgfsetlinewidth{0.349282pt}%
\definecolor{currentstroke}{rgb}{0.276022,0.044167,0.370164}%
\pgfsetstrokecolor{currentstroke}%
\pgfsetdash{}{0pt}%
\pgfpathmoveto{\pgfqpoint{5.695916in}{2.707102in}}%
\pgfpathlineto{\pgfqpoint{5.645767in}{2.706903in}}%
\pgfusepath{stroke}%
\end{pgfscope}%
\begin{pgfscope}%
\pgfpathrectangle{\pgfqpoint{3.985294in}{1.750000in}}{\pgfqpoint{2.279412in}{2.004545in}}%
\pgfusepath{clip}%
\pgfsetbuttcap%
\pgfsetroundjoin%
\pgfsetlinewidth{0.390397pt}%
\definecolor{currentstroke}{rgb}{0.280894,0.078907,0.402329}%
\pgfsetstrokecolor{currentstroke}%
\pgfsetdash{}{0pt}%
\pgfpathmoveto{\pgfqpoint{5.645767in}{2.706903in}}%
\pgfpathlineto{\pgfqpoint{5.595617in}{2.706637in}}%
\pgfusepath{stroke}%
\end{pgfscope}%
\begin{pgfscope}%
\pgfpathrectangle{\pgfqpoint{3.985294in}{1.750000in}}{\pgfqpoint{2.279412in}{2.004545in}}%
\pgfusepath{clip}%
\pgfsetbuttcap%
\pgfsetroundjoin%
\pgfsetlinewidth{0.419161pt}%
\definecolor{currentstroke}{rgb}{0.282656,0.100196,0.422160}%
\pgfsetstrokecolor{currentstroke}%
\pgfsetdash{}{0pt}%
\pgfpathmoveto{\pgfqpoint{5.595617in}{2.706637in}}%
\pgfpathlineto{\pgfqpoint{5.545466in}{2.706603in}}%
\pgfusepath{stroke}%
\end{pgfscope}%
\begin{pgfscope}%
\pgfpathrectangle{\pgfqpoint{3.985294in}{1.750000in}}{\pgfqpoint{2.279412in}{2.004545in}}%
\pgfusepath{clip}%
\pgfsetbuttcap%
\pgfsetroundjoin%
\pgfsetlinewidth{0.482730pt}%
\definecolor{currentstroke}{rgb}{0.282290,0.145912,0.461510}%
\pgfsetstrokecolor{currentstroke}%
\pgfsetdash{}{0pt}%
\pgfpathmoveto{\pgfqpoint{5.545466in}{2.706603in}}%
\pgfpathlineto{\pgfqpoint{5.495315in}{2.706703in}}%
\pgfusepath{stroke}%
\end{pgfscope}%
\begin{pgfscope}%
\pgfpathrectangle{\pgfqpoint{3.985294in}{1.750000in}}{\pgfqpoint{2.279412in}{2.004545in}}%
\pgfusepath{clip}%
\pgfsetbuttcap%
\pgfsetroundjoin%
\pgfsetlinewidth{0.552856pt}%
\definecolor{currentstroke}{rgb}{0.275191,0.194905,0.496005}%
\pgfsetstrokecolor{currentstroke}%
\pgfsetdash{}{0pt}%
\pgfpathmoveto{\pgfqpoint{5.495315in}{2.706703in}}%
\pgfpathlineto{\pgfqpoint{5.445163in}{2.706701in}}%
\pgfusepath{stroke}%
\end{pgfscope}%
\begin{pgfscope}%
\pgfpathrectangle{\pgfqpoint{3.985294in}{1.750000in}}{\pgfqpoint{2.279412in}{2.004545in}}%
\pgfusepath{clip}%
\pgfsetbuttcap%
\pgfsetroundjoin%
\pgfsetlinewidth{0.660096pt}%
\definecolor{currentstroke}{rgb}{0.252194,0.269783,0.531579}%
\pgfsetstrokecolor{currentstroke}%
\pgfsetdash{}{0pt}%
\pgfpathmoveto{\pgfqpoint{5.445163in}{2.706701in}}%
\pgfpathlineto{\pgfqpoint{5.395011in}{2.706649in}}%
\pgfusepath{stroke}%
\end{pgfscope}%
\begin{pgfscope}%
\pgfpathrectangle{\pgfqpoint{3.985294in}{1.750000in}}{\pgfqpoint{2.279412in}{2.004545in}}%
\pgfusepath{clip}%
\pgfsetbuttcap%
\pgfsetroundjoin%
\pgfsetlinewidth{0.781581pt}%
\definecolor{currentstroke}{rgb}{0.218130,0.347432,0.550038}%
\pgfsetstrokecolor{currentstroke}%
\pgfsetdash{}{0pt}%
\pgfpathmoveto{\pgfqpoint{5.395011in}{2.706649in}}%
\pgfpathlineto{\pgfqpoint{5.344859in}{2.706719in}}%
\pgfusepath{stroke}%
\end{pgfscope}%
\begin{pgfscope}%
\pgfpathrectangle{\pgfqpoint{3.985294in}{1.750000in}}{\pgfqpoint{2.279412in}{2.004545in}}%
\pgfusepath{clip}%
\pgfsetbuttcap%
\pgfsetroundjoin%
\pgfsetlinewidth{0.885654pt}%
\definecolor{currentstroke}{rgb}{0.190631,0.407061,0.556089}%
\pgfsetstrokecolor{currentstroke}%
\pgfsetdash{}{0pt}%
\pgfpathmoveto{\pgfqpoint{5.344859in}{2.706719in}}%
\pgfpathlineto{\pgfqpoint{5.294708in}{2.706910in}}%
\pgfusepath{stroke}%
\end{pgfscope}%
\begin{pgfscope}%
\pgfpathrectangle{\pgfqpoint{3.985294in}{1.750000in}}{\pgfqpoint{2.279412in}{2.004545in}}%
\pgfusepath{clip}%
\pgfsetbuttcap%
\pgfsetroundjoin%
\pgfsetlinewidth{0.993764pt}%
\definecolor{currentstroke}{rgb}{0.166617,0.463708,0.558119}%
\pgfsetstrokecolor{currentstroke}%
\pgfsetdash{}{0pt}%
\pgfpathmoveto{\pgfqpoint{5.294708in}{2.706910in}}%
\pgfpathlineto{\pgfqpoint{5.244558in}{2.707202in}}%
\pgfusepath{stroke}%
\end{pgfscope}%
\begin{pgfscope}%
\pgfpathrectangle{\pgfqpoint{3.985294in}{1.750000in}}{\pgfqpoint{2.279412in}{2.004545in}}%
\pgfusepath{clip}%
\pgfsetbuttcap%
\pgfsetroundjoin%
\pgfsetlinewidth{1.061991pt}%
\definecolor{currentstroke}{rgb}{0.151918,0.500685,0.557587}%
\pgfsetstrokecolor{currentstroke}%
\pgfsetdash{}{0pt}%
\pgfpathmoveto{\pgfqpoint{5.244558in}{2.707202in}}%
\pgfpathlineto{\pgfqpoint{5.194410in}{2.707696in}}%
\pgfusepath{stroke}%
\end{pgfscope}%
\begin{pgfscope}%
\pgfpathrectangle{\pgfqpoint{3.985294in}{1.750000in}}{\pgfqpoint{2.279412in}{2.004545in}}%
\pgfusepath{clip}%
\pgfsetbuttcap%
\pgfsetroundjoin%
\pgfsetlinewidth{1.057819pt}%
\definecolor{currentstroke}{rgb}{0.151918,0.500685,0.557587}%
\pgfsetstrokecolor{currentstroke}%
\pgfsetdash{}{0pt}%
\pgfpathmoveto{\pgfqpoint{5.194410in}{2.707696in}}%
\pgfpathlineto{\pgfqpoint{5.144264in}{2.708331in}}%
\pgfusepath{stroke}%
\end{pgfscope}%
\begin{pgfscope}%
\pgfpathrectangle{\pgfqpoint{3.985294in}{1.750000in}}{\pgfqpoint{2.279412in}{2.004545in}}%
\pgfusepath{clip}%
\pgfsetbuttcap%
\pgfsetroundjoin%
\pgfsetlinewidth{1.079151pt}%
\definecolor{currentstroke}{rgb}{0.147607,0.511733,0.557049}%
\pgfsetstrokecolor{currentstroke}%
\pgfsetdash{}{0pt}%
\pgfpathmoveto{\pgfqpoint{5.144264in}{2.708331in}}%
\pgfpathlineto{\pgfqpoint{5.094121in}{2.709040in}}%
\pgfusepath{stroke}%
\end{pgfscope}%
\begin{pgfscope}%
\pgfpathrectangle{\pgfqpoint{3.985294in}{1.750000in}}{\pgfqpoint{2.279412in}{2.004545in}}%
\pgfusepath{clip}%
\pgfsetbuttcap%
\pgfsetroundjoin%
\pgfsetlinewidth{0.999253pt}%
\definecolor{currentstroke}{rgb}{0.165117,0.467423,0.558141}%
\pgfsetstrokecolor{currentstroke}%
\pgfsetdash{}{0pt}%
\pgfpathmoveto{\pgfqpoint{5.094121in}{2.709040in}}%
\pgfpathlineto{\pgfqpoint{5.043986in}{2.710107in}}%
\pgfusepath{stroke}%
\end{pgfscope}%
\begin{pgfscope}%
\pgfpathrectangle{\pgfqpoint{3.985294in}{1.750000in}}{\pgfqpoint{2.279412in}{2.004545in}}%
\pgfusepath{clip}%
\pgfsetbuttcap%
\pgfsetroundjoin%
\pgfsetlinewidth{0.990668pt}%
\definecolor{currentstroke}{rgb}{0.166617,0.463708,0.558119}%
\pgfsetstrokecolor{currentstroke}%
\pgfsetdash{}{0pt}%
\pgfpathmoveto{\pgfqpoint{5.043986in}{2.710107in}}%
\pgfpathlineto{\pgfqpoint{4.993857in}{2.711391in}}%
\pgfusepath{stroke}%
\end{pgfscope}%
\begin{pgfscope}%
\pgfpathrectangle{\pgfqpoint{3.985294in}{1.750000in}}{\pgfqpoint{2.279412in}{2.004545in}}%
\pgfusepath{clip}%
\pgfsetbuttcap%
\pgfsetroundjoin%
\pgfsetlinewidth{0.942005pt}%
\definecolor{currentstroke}{rgb}{0.177423,0.437527,0.557565}%
\pgfsetstrokecolor{currentstroke}%
\pgfsetdash{}{0pt}%
\pgfpathmoveto{\pgfqpoint{4.993857in}{2.711391in}}%
\pgfpathlineto{\pgfqpoint{4.943737in}{2.712913in}}%
\pgfusepath{stroke}%
\end{pgfscope}%
\begin{pgfscope}%
\pgfpathrectangle{\pgfqpoint{3.985294in}{1.750000in}}{\pgfqpoint{2.279412in}{2.004545in}}%
\pgfusepath{clip}%
\pgfsetbuttcap%
\pgfsetroundjoin%
\pgfsetlinewidth{0.900751pt}%
\definecolor{currentstroke}{rgb}{0.187231,0.414746,0.556547}%
\pgfsetstrokecolor{currentstroke}%
\pgfsetdash{}{0pt}%
\pgfpathmoveto{\pgfqpoint{4.943737in}{2.712913in}}%
\pgfpathlineto{\pgfqpoint{4.893626in}{2.714539in}}%
\pgfusepath{stroke}%
\end{pgfscope}%
\begin{pgfscope}%
\pgfpathrectangle{\pgfqpoint{3.985294in}{1.750000in}}{\pgfqpoint{2.279412in}{2.004545in}}%
\pgfusepath{clip}%
\pgfsetbuttcap%
\pgfsetroundjoin%
\pgfsetlinewidth{0.823160pt}%
\definecolor{currentstroke}{rgb}{0.206756,0.371758,0.553117}%
\pgfsetstrokecolor{currentstroke}%
\pgfsetdash{}{0pt}%
\pgfpathmoveto{\pgfqpoint{4.893626in}{2.714539in}}%
\pgfpathlineto{\pgfqpoint{4.843538in}{2.716367in}}%
\pgfusepath{stroke}%
\end{pgfscope}%
\begin{pgfscope}%
\pgfpathrectangle{\pgfqpoint{3.985294in}{1.750000in}}{\pgfqpoint{2.279412in}{2.004545in}}%
\pgfusepath{clip}%
\pgfsetbuttcap%
\pgfsetroundjoin%
\pgfsetlinewidth{0.767705pt}%
\definecolor{currentstroke}{rgb}{0.223925,0.334994,0.548053}%
\pgfsetstrokecolor{currentstroke}%
\pgfsetdash{}{0pt}%
\pgfpathmoveto{\pgfqpoint{4.843538in}{2.716367in}}%
\pgfpathlineto{\pgfqpoint{4.793474in}{2.718505in}}%
\pgfusepath{stroke}%
\end{pgfscope}%
\begin{pgfscope}%
\pgfpathrectangle{\pgfqpoint{3.985294in}{1.750000in}}{\pgfqpoint{2.279412in}{2.004545in}}%
\pgfusepath{clip}%
\pgfsetbuttcap%
\pgfsetroundjoin%
\pgfsetlinewidth{0.631578pt}%
\definecolor{currentstroke}{rgb}{0.258965,0.251537,0.524736}%
\pgfsetstrokecolor{currentstroke}%
\pgfsetdash{}{0pt}%
\pgfpathmoveto{\pgfqpoint{4.793474in}{2.718505in}}%
\pgfpathlineto{\pgfqpoint{4.743651in}{2.720639in}}%
\pgfusepath{stroke}%
\end{pgfscope}%
\begin{pgfscope}%
\pgfpathrectangle{\pgfqpoint{3.985294in}{1.750000in}}{\pgfqpoint{2.279412in}{2.004545in}}%
\pgfusepath{clip}%
\pgfsetbuttcap%
\pgfsetroundjoin%
\pgfsetlinewidth{0.504082pt}%
\definecolor{currentstroke}{rgb}{0.280868,0.160771,0.472899}%
\pgfsetstrokecolor{currentstroke}%
\pgfsetdash{}{0pt}%
\pgfpathmoveto{\pgfqpoint{4.743651in}{2.720639in}}%
\pgfpathlineto{\pgfqpoint{4.743651in}{2.720639in}}%
\pgfusepath{stroke}%
\end{pgfscope}%
\begin{pgfscope}%
\pgfpathrectangle{\pgfqpoint{3.985294in}{1.750000in}}{\pgfqpoint{2.279412in}{2.004545in}}%
\pgfusepath{clip}%
\pgfsetbuttcap%
\pgfsetroundjoin%
\pgfsetlinewidth{0.313958pt}%
\definecolor{currentstroke}{rgb}{0.268510,0.009605,0.335427}%
\pgfsetstrokecolor{currentstroke}%
\pgfsetdash{}{0pt}%
\pgfpathmoveto{\pgfqpoint{5.915884in}{2.751962in}}%
\pgfpathlineto{\pgfqpoint{5.894378in}{2.752273in}}%
\pgfusepath{stroke}%
\end{pgfscope}%
\begin{pgfscope}%
\pgfpathrectangle{\pgfqpoint{3.985294in}{1.750000in}}{\pgfqpoint{2.279412in}{2.004545in}}%
\pgfusepath{clip}%
\pgfsetbuttcap%
\pgfsetroundjoin%
\pgfsetlinewidth{0.331175pt}%
\definecolor{currentstroke}{rgb}{0.272594,0.025563,0.353093}%
\pgfsetstrokecolor{currentstroke}%
\pgfsetdash{}{0pt}%
\pgfpathmoveto{\pgfqpoint{5.894378in}{2.752273in}}%
\pgfpathlineto{\pgfqpoint{5.894378in}{2.752273in}}%
\pgfusepath{stroke}%
\end{pgfscope}%
\begin{pgfscope}%
\pgfpathrectangle{\pgfqpoint{3.985294in}{1.750000in}}{\pgfqpoint{2.279412in}{2.004545in}}%
\pgfusepath{clip}%
\pgfsetbuttcap%
\pgfsetroundjoin%
\pgfsetlinewidth{0.331175pt}%
\definecolor{currentstroke}{rgb}{0.272594,0.025563,0.353093}%
\pgfsetstrokecolor{currentstroke}%
\pgfsetdash{}{0pt}%
\pgfpathmoveto{\pgfqpoint{5.894378in}{2.752273in}}%
\pgfpathlineto{\pgfqpoint{5.894378in}{2.752273in}}%
\pgfusepath{stroke}%
\end{pgfscope}%
\begin{pgfscope}%
\pgfpathrectangle{\pgfqpoint{3.985294in}{1.750000in}}{\pgfqpoint{2.279412in}{2.004545in}}%
\pgfusepath{clip}%
\pgfsetbuttcap%
\pgfsetroundjoin%
\pgfsetlinewidth{0.331175pt}%
\definecolor{currentstroke}{rgb}{0.272594,0.025563,0.353093}%
\pgfsetstrokecolor{currentstroke}%
\pgfsetdash{}{0pt}%
\pgfpathmoveto{\pgfqpoint{5.894378in}{2.752273in}}%
\pgfpathlineto{\pgfqpoint{5.844229in}{2.752366in}}%
\pgfusepath{stroke}%
\end{pgfscope}%
\begin{pgfscope}%
\pgfpathrectangle{\pgfqpoint{3.985294in}{1.750000in}}{\pgfqpoint{2.279412in}{2.004545in}}%
\pgfusepath{clip}%
\pgfsetbuttcap%
\pgfsetroundjoin%
\pgfsetlinewidth{0.323499pt}%
\definecolor{currentstroke}{rgb}{0.271305,0.019942,0.347269}%
\pgfsetstrokecolor{currentstroke}%
\pgfsetdash{}{0pt}%
\pgfpathmoveto{\pgfqpoint{5.844229in}{2.752366in}}%
\pgfpathlineto{\pgfqpoint{5.794083in}{2.752625in}}%
\pgfusepath{stroke}%
\end{pgfscope}%
\begin{pgfscope}%
\pgfpathrectangle{\pgfqpoint{3.985294in}{1.750000in}}{\pgfqpoint{2.279412in}{2.004545in}}%
\pgfusepath{clip}%
\pgfsetbuttcap%
\pgfsetroundjoin%
\pgfsetlinewidth{0.323585pt}%
\definecolor{currentstroke}{rgb}{0.271305,0.019942,0.347269}%
\pgfsetstrokecolor{currentstroke}%
\pgfsetdash{}{0pt}%
\pgfpathmoveto{\pgfqpoint{5.794083in}{2.752625in}}%
\pgfpathlineto{\pgfqpoint{5.743939in}{2.752661in}}%
\pgfusepath{stroke}%
\end{pgfscope}%
\begin{pgfscope}%
\pgfpathrectangle{\pgfqpoint{3.985294in}{1.750000in}}{\pgfqpoint{2.279412in}{2.004545in}}%
\pgfusepath{clip}%
\pgfsetbuttcap%
\pgfsetroundjoin%
\pgfsetlinewidth{0.332505pt}%
\definecolor{currentstroke}{rgb}{0.272594,0.025563,0.353093}%
\pgfsetstrokecolor{currentstroke}%
\pgfsetdash{}{0pt}%
\pgfpathmoveto{\pgfqpoint{5.743939in}{2.752661in}}%
\pgfpathlineto{\pgfqpoint{5.693791in}{2.752463in}}%
\pgfusepath{stroke}%
\end{pgfscope}%
\begin{pgfscope}%
\pgfpathrectangle{\pgfqpoint{3.985294in}{1.750000in}}{\pgfqpoint{2.279412in}{2.004545in}}%
\pgfusepath{clip}%
\pgfsetbuttcap%
\pgfsetroundjoin%
\pgfsetlinewidth{0.355124pt}%
\definecolor{currentstroke}{rgb}{0.276022,0.044167,0.370164}%
\pgfsetstrokecolor{currentstroke}%
\pgfsetdash{}{0pt}%
\pgfpathmoveto{\pgfqpoint{5.693791in}{2.752463in}}%
\pgfpathlineto{\pgfqpoint{5.643646in}{2.751957in}}%
\pgfusepath{stroke}%
\end{pgfscope}%
\begin{pgfscope}%
\pgfpathrectangle{\pgfqpoint{3.985294in}{1.750000in}}{\pgfqpoint{2.279412in}{2.004545in}}%
\pgfusepath{clip}%
\pgfsetbuttcap%
\pgfsetroundjoin%
\pgfsetlinewidth{0.383702pt}%
\definecolor{currentstroke}{rgb}{0.280267,0.073417,0.397163}%
\pgfsetstrokecolor{currentstroke}%
\pgfsetdash{}{0pt}%
\pgfpathmoveto{\pgfqpoint{5.643646in}{2.751957in}}%
\pgfpathlineto{\pgfqpoint{5.593502in}{2.751406in}}%
\pgfusepath{stroke}%
\end{pgfscope}%
\begin{pgfscope}%
\pgfpathrectangle{\pgfqpoint{3.985294in}{1.750000in}}{\pgfqpoint{2.279412in}{2.004545in}}%
\pgfusepath{clip}%
\pgfsetbuttcap%
\pgfsetroundjoin%
\pgfsetlinewidth{0.409730pt}%
\definecolor{currentstroke}{rgb}{0.281924,0.089666,0.412415}%
\pgfsetstrokecolor{currentstroke}%
\pgfsetdash{}{0pt}%
\pgfpathmoveto{\pgfqpoint{5.593502in}{2.751406in}}%
\pgfpathlineto{\pgfqpoint{5.543351in}{2.751288in}}%
\pgfusepath{stroke}%
\end{pgfscope}%
\begin{pgfscope}%
\pgfpathrectangle{\pgfqpoint{3.985294in}{1.750000in}}{\pgfqpoint{2.279412in}{2.004545in}}%
\pgfusepath{clip}%
\pgfsetbuttcap%
\pgfsetroundjoin%
\pgfsetlinewidth{0.474802pt}%
\definecolor{currentstroke}{rgb}{0.282623,0.140926,0.457517}%
\pgfsetstrokecolor{currentstroke}%
\pgfsetdash{}{0pt}%
\pgfpathmoveto{\pgfqpoint{5.543351in}{2.751288in}}%
\pgfpathlineto{\pgfqpoint{5.493201in}{2.750932in}}%
\pgfusepath{stroke}%
\end{pgfscope}%
\begin{pgfscope}%
\pgfpathrectangle{\pgfqpoint{3.985294in}{1.750000in}}{\pgfqpoint{2.279412in}{2.004545in}}%
\pgfusepath{clip}%
\pgfsetbuttcap%
\pgfsetroundjoin%
\pgfsetlinewidth{0.565857pt}%
\definecolor{currentstroke}{rgb}{0.273006,0.204520,0.501721}%
\pgfsetstrokecolor{currentstroke}%
\pgfsetdash{}{0pt}%
\pgfpathmoveto{\pgfqpoint{5.493201in}{2.750932in}}%
\pgfpathlineto{\pgfqpoint{5.443051in}{2.750661in}}%
\pgfusepath{stroke}%
\end{pgfscope}%
\begin{pgfscope}%
\pgfpathrectangle{\pgfqpoint{3.985294in}{1.750000in}}{\pgfqpoint{2.279412in}{2.004545in}}%
\pgfusepath{clip}%
\pgfsetbuttcap%
\pgfsetroundjoin%
\pgfsetlinewidth{0.658390pt}%
\definecolor{currentstroke}{rgb}{0.252194,0.269783,0.531579}%
\pgfsetstrokecolor{currentstroke}%
\pgfsetdash{}{0pt}%
\pgfpathmoveto{\pgfqpoint{5.443051in}{2.750661in}}%
\pgfpathlineto{\pgfqpoint{5.392899in}{2.750480in}}%
\pgfusepath{stroke}%
\end{pgfscope}%
\begin{pgfscope}%
\pgfpathrectangle{\pgfqpoint{3.985294in}{1.750000in}}{\pgfqpoint{2.279412in}{2.004545in}}%
\pgfusepath{clip}%
\pgfsetbuttcap%
\pgfsetroundjoin%
\pgfsetlinewidth{0.784436pt}%
\definecolor{currentstroke}{rgb}{0.218130,0.347432,0.550038}%
\pgfsetstrokecolor{currentstroke}%
\pgfsetdash{}{0pt}%
\pgfpathmoveto{\pgfqpoint{5.392899in}{2.750480in}}%
\pgfpathlineto{\pgfqpoint{5.342749in}{2.750144in}}%
\pgfusepath{stroke}%
\end{pgfscope}%
\begin{pgfscope}%
\pgfpathrectangle{\pgfqpoint{3.985294in}{1.750000in}}{\pgfqpoint{2.279412in}{2.004545in}}%
\pgfusepath{clip}%
\pgfsetbuttcap%
\pgfsetroundjoin%
\pgfsetlinewidth{0.899794pt}%
\definecolor{currentstroke}{rgb}{0.187231,0.414746,0.556547}%
\pgfsetstrokecolor{currentstroke}%
\pgfsetdash{}{0pt}%
\pgfpathmoveto{\pgfqpoint{5.342749in}{2.750144in}}%
\pgfpathlineto{\pgfqpoint{5.292599in}{2.749771in}}%
\pgfusepath{stroke}%
\end{pgfscope}%
\begin{pgfscope}%
\pgfpathrectangle{\pgfqpoint{3.985294in}{1.750000in}}{\pgfqpoint{2.279412in}{2.004545in}}%
\pgfusepath{clip}%
\pgfsetbuttcap%
\pgfsetroundjoin%
\pgfsetlinewidth{0.977620pt}%
\definecolor{currentstroke}{rgb}{0.169646,0.456262,0.558030}%
\pgfsetstrokecolor{currentstroke}%
\pgfsetdash{}{0pt}%
\pgfpathmoveto{\pgfqpoint{5.292599in}{2.749771in}}%
\pgfpathlineto{\pgfqpoint{5.242451in}{2.749289in}}%
\pgfusepath{stroke}%
\end{pgfscope}%
\begin{pgfscope}%
\pgfpathrectangle{\pgfqpoint{3.985294in}{1.750000in}}{\pgfqpoint{2.279412in}{2.004545in}}%
\pgfusepath{clip}%
\pgfsetbuttcap%
\pgfsetroundjoin%
\pgfsetlinewidth{1.061920pt}%
\definecolor{currentstroke}{rgb}{0.151918,0.500685,0.557587}%
\pgfsetstrokecolor{currentstroke}%
\pgfsetdash{}{0pt}%
\pgfpathmoveto{\pgfqpoint{5.242451in}{2.749289in}}%
\pgfpathlineto{\pgfqpoint{5.192304in}{2.748703in}}%
\pgfusepath{stroke}%
\end{pgfscope}%
\begin{pgfscope}%
\pgfpathrectangle{\pgfqpoint{3.985294in}{1.750000in}}{\pgfqpoint{2.279412in}{2.004545in}}%
\pgfusepath{clip}%
\pgfsetbuttcap%
\pgfsetroundjoin%
\pgfsetlinewidth{1.051635pt}%
\definecolor{currentstroke}{rgb}{0.153364,0.497000,0.557724}%
\pgfsetstrokecolor{currentstroke}%
\pgfsetdash{}{0pt}%
\pgfpathmoveto{\pgfqpoint{5.192304in}{2.748703in}}%
\pgfpathlineto{\pgfqpoint{5.142159in}{2.748006in}}%
\pgfusepath{stroke}%
\end{pgfscope}%
\begin{pgfscope}%
\pgfpathrectangle{\pgfqpoint{3.985294in}{1.750000in}}{\pgfqpoint{2.279412in}{2.004545in}}%
\pgfusepath{clip}%
\pgfsetbuttcap%
\pgfsetroundjoin%
\pgfsetlinewidth{1.051172pt}%
\definecolor{currentstroke}{rgb}{0.153364,0.497000,0.557724}%
\pgfsetstrokecolor{currentstroke}%
\pgfsetdash{}{0pt}%
\pgfpathmoveto{\pgfqpoint{5.142159in}{2.748006in}}%
\pgfpathlineto{\pgfqpoint{5.092018in}{2.747178in}}%
\pgfusepath{stroke}%
\end{pgfscope}%
\begin{pgfscope}%
\pgfpathrectangle{\pgfqpoint{3.985294in}{1.750000in}}{\pgfqpoint{2.279412in}{2.004545in}}%
\pgfusepath{clip}%
\pgfsetbuttcap%
\pgfsetroundjoin%
\pgfsetlinewidth{1.007214pt}%
\definecolor{currentstroke}{rgb}{0.163625,0.471133,0.558148}%
\pgfsetstrokecolor{currentstroke}%
\pgfsetdash{}{0pt}%
\pgfpathmoveto{\pgfqpoint{5.092018in}{2.747178in}}%
\pgfpathlineto{\pgfqpoint{5.041885in}{2.746152in}}%
\pgfusepath{stroke}%
\end{pgfscope}%
\begin{pgfscope}%
\pgfpathrectangle{\pgfqpoint{3.985294in}{1.750000in}}{\pgfqpoint{2.279412in}{2.004545in}}%
\pgfusepath{clip}%
\pgfsetbuttcap%
\pgfsetroundjoin%
\pgfsetlinewidth{0.999881pt}%
\definecolor{currentstroke}{rgb}{0.165117,0.467423,0.558141}%
\pgfsetstrokecolor{currentstroke}%
\pgfsetdash{}{0pt}%
\pgfpathmoveto{\pgfqpoint{5.041885in}{2.746152in}}%
\pgfpathlineto{\pgfqpoint{4.991758in}{2.744939in}}%
\pgfusepath{stroke}%
\end{pgfscope}%
\begin{pgfscope}%
\pgfpathrectangle{\pgfqpoint{3.985294in}{1.750000in}}{\pgfqpoint{2.279412in}{2.004545in}}%
\pgfusepath{clip}%
\pgfsetbuttcap%
\pgfsetroundjoin%
\pgfsetlinewidth{0.911292pt}%
\definecolor{currentstroke}{rgb}{0.185556,0.418570,0.556753}%
\pgfsetstrokecolor{currentstroke}%
\pgfsetdash{}{0pt}%
\pgfpathmoveto{\pgfqpoint{4.991758in}{2.744939in}}%
\pgfpathlineto{\pgfqpoint{4.941636in}{2.743542in}}%
\pgfusepath{stroke}%
\end{pgfscope}%
\begin{pgfscope}%
\pgfpathrectangle{\pgfqpoint{3.985294in}{1.750000in}}{\pgfqpoint{2.279412in}{2.004545in}}%
\pgfusepath{clip}%
\pgfsetbuttcap%
\pgfsetroundjoin%
\pgfsetlinewidth{0.326151pt}%
\definecolor{currentstroke}{rgb}{0.271305,0.019942,0.347269}%
\pgfsetstrokecolor{currentstroke}%
\pgfsetdash{}{0pt}%
\pgfpathmoveto{\pgfqpoint{5.894378in}{3.068020in}}%
\pgfpathlineto{\pgfqpoint{5.844241in}{3.067552in}}%
\pgfusepath{stroke}%
\end{pgfscope}%
\begin{pgfscope}%
\pgfpathrectangle{\pgfqpoint{3.985294in}{1.750000in}}{\pgfqpoint{2.279412in}{2.004545in}}%
\pgfusepath{clip}%
\pgfsetbuttcap%
\pgfsetroundjoin%
\pgfsetlinewidth{0.322636pt}%
\definecolor{currentstroke}{rgb}{0.271305,0.019942,0.347269}%
\pgfsetstrokecolor{currentstroke}%
\pgfsetdash{}{0pt}%
\pgfpathmoveto{\pgfqpoint{5.844241in}{3.067552in}}%
\pgfpathlineto{\pgfqpoint{5.794207in}{3.068197in}}%
\pgfusepath{stroke}%
\end{pgfscope}%
\begin{pgfscope}%
\pgfpathrectangle{\pgfqpoint{3.985294in}{1.750000in}}{\pgfqpoint{2.279412in}{2.004545in}}%
\pgfusepath{clip}%
\pgfsetbuttcap%
\pgfsetroundjoin%
\pgfsetlinewidth{0.316358pt}%
\definecolor{currentstroke}{rgb}{0.269944,0.014625,0.341379}%
\pgfsetstrokecolor{currentstroke}%
\pgfsetdash{}{0pt}%
\pgfpathmoveto{\pgfqpoint{5.794207in}{3.068197in}}%
\pgfpathlineto{\pgfqpoint{5.744188in}{3.068442in}}%
\pgfusepath{stroke}%
\end{pgfscope}%
\begin{pgfscope}%
\pgfpathrectangle{\pgfqpoint{3.985294in}{1.750000in}}{\pgfqpoint{2.279412in}{2.004545in}}%
\pgfusepath{clip}%
\pgfsetbuttcap%
\pgfsetroundjoin%
\pgfsetlinewidth{0.327152pt}%
\definecolor{currentstroke}{rgb}{0.271305,0.019942,0.347269}%
\pgfsetstrokecolor{currentstroke}%
\pgfsetdash{}{0pt}%
\pgfpathmoveto{\pgfqpoint{5.744188in}{3.068442in}}%
\pgfpathlineto{\pgfqpoint{5.694066in}{3.067739in}}%
\pgfusepath{stroke}%
\end{pgfscope}%
\begin{pgfscope}%
\pgfpathrectangle{\pgfqpoint{3.985294in}{1.750000in}}{\pgfqpoint{2.279412in}{2.004545in}}%
\pgfusepath{clip}%
\pgfsetbuttcap%
\pgfsetroundjoin%
\pgfsetlinewidth{0.335220pt}%
\definecolor{currentstroke}{rgb}{0.272594,0.025563,0.353093}%
\pgfsetstrokecolor{currentstroke}%
\pgfsetdash{}{0pt}%
\pgfpathmoveto{\pgfqpoint{5.694066in}{3.067739in}}%
\pgfpathlineto{\pgfqpoint{5.643936in}{3.066584in}}%
\pgfusepath{stroke}%
\end{pgfscope}%
\begin{pgfscope}%
\pgfpathrectangle{\pgfqpoint{3.985294in}{1.750000in}}{\pgfqpoint{2.279412in}{2.004545in}}%
\pgfusepath{clip}%
\pgfsetbuttcap%
\pgfsetroundjoin%
\pgfsetlinewidth{0.355082pt}%
\definecolor{currentstroke}{rgb}{0.276022,0.044167,0.370164}%
\pgfsetstrokecolor{currentstroke}%
\pgfsetdash{}{0pt}%
\pgfpathmoveto{\pgfqpoint{5.643936in}{3.066584in}}%
\pgfpathlineto{\pgfqpoint{5.593818in}{3.065039in}}%
\pgfusepath{stroke}%
\end{pgfscope}%
\begin{pgfscope}%
\pgfpathrectangle{\pgfqpoint{3.985294in}{1.750000in}}{\pgfqpoint{2.279412in}{2.004545in}}%
\pgfusepath{clip}%
\pgfsetbuttcap%
\pgfsetroundjoin%
\pgfsetlinewidth{0.376924pt}%
\definecolor{currentstroke}{rgb}{0.279566,0.067836,0.391917}%
\pgfsetstrokecolor{currentstroke}%
\pgfsetdash{}{0pt}%
\pgfpathmoveto{\pgfqpoint{5.593818in}{3.065039in}}%
\pgfpathlineto{\pgfqpoint{5.543708in}{3.063330in}}%
\pgfusepath{stroke}%
\end{pgfscope}%
\begin{pgfscope}%
\pgfpathrectangle{\pgfqpoint{3.985294in}{1.750000in}}{\pgfqpoint{2.279412in}{2.004545in}}%
\pgfusepath{clip}%
\pgfsetbuttcap%
\pgfsetroundjoin%
\pgfsetlinewidth{0.401396pt}%
\definecolor{currentstroke}{rgb}{0.281446,0.084320,0.407414}%
\pgfsetstrokecolor{currentstroke}%
\pgfsetdash{}{0pt}%
\pgfpathmoveto{\pgfqpoint{5.543708in}{3.063330in}}%
\pgfpathlineto{\pgfqpoint{5.493631in}{3.060957in}}%
\pgfusepath{stroke}%
\end{pgfscope}%
\begin{pgfscope}%
\pgfpathrectangle{\pgfqpoint{3.985294in}{1.750000in}}{\pgfqpoint{2.279412in}{2.004545in}}%
\pgfusepath{clip}%
\pgfsetbuttcap%
\pgfsetroundjoin%
\pgfsetlinewidth{0.445045pt}%
\definecolor{currentstroke}{rgb}{0.283197,0.115680,0.436115}%
\pgfsetstrokecolor{currentstroke}%
\pgfsetdash{}{0pt}%
\pgfpathmoveto{\pgfqpoint{5.493631in}{3.060957in}}%
\pgfpathlineto{\pgfqpoint{5.443607in}{3.057846in}}%
\pgfusepath{stroke}%
\end{pgfscope}%
\begin{pgfscope}%
\pgfpathrectangle{\pgfqpoint{3.985294in}{1.750000in}}{\pgfqpoint{2.279412in}{2.004545in}}%
\pgfusepath{clip}%
\pgfsetbuttcap%
\pgfsetroundjoin%
\pgfsetlinewidth{0.441956pt}%
\definecolor{currentstroke}{rgb}{0.283197,0.115680,0.436115}%
\pgfsetstrokecolor{currentstroke}%
\pgfsetdash{}{0pt}%
\pgfpathmoveto{\pgfqpoint{5.443607in}{3.057846in}}%
\pgfpathlineto{\pgfqpoint{5.393664in}{3.053847in}}%
\pgfusepath{stroke}%
\end{pgfscope}%
\begin{pgfscope}%
\pgfpathrectangle{\pgfqpoint{3.985294in}{1.750000in}}{\pgfqpoint{2.279412in}{2.004545in}}%
\pgfusepath{clip}%
\pgfsetbuttcap%
\pgfsetroundjoin%
\pgfsetlinewidth{0.484216pt}%
\definecolor{currentstroke}{rgb}{0.282290,0.145912,0.461510}%
\pgfsetstrokecolor{currentstroke}%
\pgfsetdash{}{0pt}%
\pgfpathmoveto{\pgfqpoint{5.393664in}{3.053847in}}%
\pgfpathlineto{\pgfqpoint{5.343847in}{3.048804in}}%
\pgfusepath{stroke}%
\end{pgfscope}%
\begin{pgfscope}%
\pgfpathrectangle{\pgfqpoint{3.985294in}{1.750000in}}{\pgfqpoint{2.279412in}{2.004545in}}%
\pgfusepath{clip}%
\pgfsetbuttcap%
\pgfsetroundjoin%
\pgfsetlinewidth{0.554254pt}%
\definecolor{currentstroke}{rgb}{0.275191,0.194905,0.496005}%
\pgfsetstrokecolor{currentstroke}%
\pgfsetdash{}{0pt}%
\pgfpathmoveto{\pgfqpoint{5.343847in}{3.048804in}}%
\pgfpathlineto{\pgfqpoint{5.294152in}{3.042877in}}%
\pgfusepath{stroke}%
\end{pgfscope}%
\begin{pgfscope}%
\pgfpathrectangle{\pgfqpoint{3.985294in}{1.750000in}}{\pgfqpoint{2.279412in}{2.004545in}}%
\pgfusepath{clip}%
\pgfsetbuttcap%
\pgfsetroundjoin%
\pgfsetlinewidth{0.539557pt}%
\definecolor{currentstroke}{rgb}{0.277134,0.185228,0.489898}%
\pgfsetstrokecolor{currentstroke}%
\pgfsetdash{}{0pt}%
\pgfpathmoveto{\pgfqpoint{5.294152in}{3.042877in}}%
\pgfpathlineto{\pgfqpoint{5.244647in}{3.035880in}}%
\pgfusepath{stroke}%
\end{pgfscope}%
\begin{pgfscope}%
\pgfpathrectangle{\pgfqpoint{3.985294in}{1.750000in}}{\pgfqpoint{2.279412in}{2.004545in}}%
\pgfusepath{clip}%
\pgfsetbuttcap%
\pgfsetroundjoin%
\pgfsetlinewidth{0.580949pt}%
\definecolor{currentstroke}{rgb}{0.270595,0.214069,0.507052}%
\pgfsetstrokecolor{currentstroke}%
\pgfsetdash{}{0pt}%
\pgfpathmoveto{\pgfqpoint{5.244647in}{3.035880in}}%
\pgfpathlineto{\pgfqpoint{5.195451in}{3.027362in}}%
\pgfusepath{stroke}%
\end{pgfscope}%
\begin{pgfscope}%
\pgfpathrectangle{\pgfqpoint{3.985294in}{1.750000in}}{\pgfqpoint{2.279412in}{2.004545in}}%
\pgfusepath{clip}%
\pgfsetbuttcap%
\pgfsetroundjoin%
\pgfsetlinewidth{0.571151pt}%
\definecolor{currentstroke}{rgb}{0.271828,0.209303,0.504434}%
\pgfsetstrokecolor{currentstroke}%
\pgfsetdash{}{0pt}%
\pgfpathmoveto{\pgfqpoint{5.195451in}{3.027362in}}%
\pgfpathlineto{\pgfqpoint{5.146825in}{3.016708in}}%
\pgfusepath{stroke}%
\end{pgfscope}%
\begin{pgfscope}%
\pgfpathrectangle{\pgfqpoint{3.985294in}{1.750000in}}{\pgfqpoint{2.279412in}{2.004545in}}%
\pgfusepath{clip}%
\pgfsetbuttcap%
\pgfsetroundjoin%
\pgfsetlinewidth{0.614198pt}%
\definecolor{currentstroke}{rgb}{0.263663,0.237631,0.518762}%
\pgfsetstrokecolor{currentstroke}%
\pgfsetdash{}{0pt}%
\pgfpathmoveto{\pgfqpoint{5.146825in}{3.016708in}}%
\pgfpathlineto{\pgfqpoint{5.099034in}{3.003465in}}%
\pgfusepath{stroke}%
\end{pgfscope}%
\begin{pgfscope}%
\pgfpathrectangle{\pgfqpoint{3.985294in}{1.750000in}}{\pgfqpoint{2.279412in}{2.004545in}}%
\pgfusepath{clip}%
\pgfsetbuttcap%
\pgfsetroundjoin%
\pgfsetlinewidth{0.633116pt}%
\definecolor{currentstroke}{rgb}{0.258965,0.251537,0.524736}%
\pgfsetstrokecolor{currentstroke}%
\pgfsetdash{}{0pt}%
\pgfpathmoveto{\pgfqpoint{5.099034in}{3.003465in}}%
\pgfpathlineto{\pgfqpoint{5.052448in}{2.987281in}}%
\pgfusepath{stroke}%
\end{pgfscope}%
\begin{pgfscope}%
\pgfpathrectangle{\pgfqpoint{3.985294in}{1.750000in}}{\pgfqpoint{2.279412in}{2.004545in}}%
\pgfusepath{clip}%
\pgfsetbuttcap%
\pgfsetroundjoin%
\pgfsetlinewidth{0.633118pt}%
\definecolor{currentstroke}{rgb}{0.258965,0.251537,0.524736}%
\pgfsetstrokecolor{currentstroke}%
\pgfsetdash{}{0pt}%
\pgfpathmoveto{\pgfqpoint{5.052448in}{2.987281in}}%
\pgfpathlineto{\pgfqpoint{5.007278in}{2.968224in}}%
\pgfusepath{stroke}%
\end{pgfscope}%
\begin{pgfscope}%
\pgfpathrectangle{\pgfqpoint{3.985294in}{1.750000in}}{\pgfqpoint{2.279412in}{2.004545in}}%
\pgfusepath{clip}%
\pgfsetbuttcap%
\pgfsetroundjoin%
\pgfsetlinewidth{0.610704pt}%
\definecolor{currentstroke}{rgb}{0.263663,0.237631,0.518762}%
\pgfsetstrokecolor{currentstroke}%
\pgfsetdash{}{0pt}%
\pgfpathmoveto{\pgfqpoint{5.007278in}{2.968224in}}%
\pgfpathlineto{\pgfqpoint{4.963518in}{2.946755in}}%
\pgfusepath{stroke}%
\end{pgfscope}%
\begin{pgfscope}%
\pgfpathrectangle{\pgfqpoint{3.985294in}{1.750000in}}{\pgfqpoint{2.279412in}{2.004545in}}%
\pgfusepath{clip}%
\pgfsetbuttcap%
\pgfsetroundjoin%
\pgfsetlinewidth{0.685661pt}%
\definecolor{currentstroke}{rgb}{0.244972,0.287675,0.537260}%
\pgfsetstrokecolor{currentstroke}%
\pgfsetdash{}{0pt}%
\pgfpathmoveto{\pgfqpoint{4.963518in}{2.946755in}}%
\pgfpathlineto{\pgfqpoint{4.920975in}{2.923517in}}%
\pgfusepath{stroke}%
\end{pgfscope}%
\begin{pgfscope}%
\pgfpathrectangle{\pgfqpoint{3.985294in}{1.750000in}}{\pgfqpoint{2.279412in}{2.004545in}}%
\pgfusepath{clip}%
\pgfsetbuttcap%
\pgfsetroundjoin%
\pgfsetlinewidth{0.751205pt}%
\definecolor{currentstroke}{rgb}{0.227802,0.326594,0.546532}%
\pgfsetstrokecolor{currentstroke}%
\pgfsetdash{}{0pt}%
\pgfpathmoveto{\pgfqpoint{4.920975in}{2.923517in}}%
\pgfpathlineto{\pgfqpoint{4.879834in}{2.898449in}}%
\pgfusepath{stroke}%
\end{pgfscope}%
\begin{pgfscope}%
\pgfpathrectangle{\pgfqpoint{3.985294in}{1.750000in}}{\pgfqpoint{2.279412in}{2.004545in}}%
\pgfusepath{clip}%
\pgfsetbuttcap%
\pgfsetroundjoin%
\pgfsetlinewidth{0.720321pt}%
\definecolor{currentstroke}{rgb}{0.235526,0.309527,0.542944}%
\pgfsetstrokecolor{currentstroke}%
\pgfsetdash{}{0pt}%
\pgfpathmoveto{\pgfqpoint{4.879834in}{2.898449in}}%
\pgfpathlineto{\pgfqpoint{4.839615in}{2.872278in}}%
\pgfusepath{stroke}%
\end{pgfscope}%
\begin{pgfscope}%
\pgfpathrectangle{\pgfqpoint{3.985294in}{1.750000in}}{\pgfqpoint{2.279412in}{2.004545in}}%
\pgfusepath{clip}%
\pgfsetbuttcap%
\pgfsetroundjoin%
\pgfsetlinewidth{0.316890pt}%
\definecolor{currentstroke}{rgb}{0.269944,0.014625,0.341379}%
\pgfsetstrokecolor{currentstroke}%
\pgfsetdash{}{0pt}%
\pgfpathmoveto{\pgfqpoint{5.894378in}{3.113127in}}%
\pgfpathlineto{\pgfqpoint{5.844319in}{3.114894in}}%
\pgfusepath{stroke}%
\end{pgfscope}%
\begin{pgfscope}%
\pgfpathrectangle{\pgfqpoint{3.985294in}{1.750000in}}{\pgfqpoint{2.279412in}{2.004545in}}%
\pgfusepath{clip}%
\pgfsetbuttcap%
\pgfsetroundjoin%
\pgfsetlinewidth{0.322072pt}%
\definecolor{currentstroke}{rgb}{0.271305,0.019942,0.347269}%
\pgfsetstrokecolor{currentstroke}%
\pgfsetdash{}{0pt}%
\pgfpathmoveto{\pgfqpoint{5.844319in}{3.114894in}}%
\pgfpathlineto{\pgfqpoint{5.794307in}{3.114974in}}%
\pgfusepath{stroke}%
\end{pgfscope}%
\begin{pgfscope}%
\pgfpathrectangle{\pgfqpoint{3.985294in}{1.750000in}}{\pgfqpoint{2.279412in}{2.004545in}}%
\pgfusepath{clip}%
\pgfsetbuttcap%
\pgfsetroundjoin%
\pgfsetlinewidth{0.316224pt}%
\definecolor{currentstroke}{rgb}{0.269944,0.014625,0.341379}%
\pgfsetstrokecolor{currentstroke}%
\pgfsetdash{}{0pt}%
\pgfpathmoveto{\pgfqpoint{5.794307in}{3.114974in}}%
\pgfpathlineto{\pgfqpoint{5.744291in}{3.114390in}}%
\pgfusepath{stroke}%
\end{pgfscope}%
\begin{pgfscope}%
\pgfpathrectangle{\pgfqpoint{3.985294in}{1.750000in}}{\pgfqpoint{2.279412in}{2.004545in}}%
\pgfusepath{clip}%
\pgfsetbuttcap%
\pgfsetroundjoin%
\pgfsetlinewidth{0.324814pt}%
\definecolor{currentstroke}{rgb}{0.271305,0.019942,0.347269}%
\pgfsetstrokecolor{currentstroke}%
\pgfsetdash{}{0pt}%
\pgfpathmoveto{\pgfqpoint{5.744291in}{3.114390in}}%
\pgfpathlineto{\pgfqpoint{5.694156in}{3.113703in}}%
\pgfusepath{stroke}%
\end{pgfscope}%
\begin{pgfscope}%
\pgfpathrectangle{\pgfqpoint{3.985294in}{1.750000in}}{\pgfqpoint{2.279412in}{2.004545in}}%
\pgfusepath{clip}%
\pgfsetbuttcap%
\pgfsetroundjoin%
\pgfsetlinewidth{0.339049pt}%
\definecolor{currentstroke}{rgb}{0.273809,0.031497,0.358853}%
\pgfsetstrokecolor{currentstroke}%
\pgfsetdash{}{0pt}%
\pgfpathmoveto{\pgfqpoint{5.694156in}{3.113703in}}%
\pgfpathlineto{\pgfqpoint{5.644026in}{3.112670in}}%
\pgfusepath{stroke}%
\end{pgfscope}%
\begin{pgfscope}%
\pgfpathrectangle{\pgfqpoint{3.985294in}{1.750000in}}{\pgfqpoint{2.279412in}{2.004545in}}%
\pgfusepath{clip}%
\pgfsetbuttcap%
\pgfsetroundjoin%
\pgfsetlinewidth{0.343565pt}%
\definecolor{currentstroke}{rgb}{0.274952,0.037752,0.364543}%
\pgfsetstrokecolor{currentstroke}%
\pgfsetdash{}{0pt}%
\pgfpathmoveto{\pgfqpoint{5.644026in}{3.112670in}}%
\pgfpathlineto{\pgfqpoint{5.593936in}{3.110546in}}%
\pgfusepath{stroke}%
\end{pgfscope}%
\begin{pgfscope}%
\pgfpathrectangle{\pgfqpoint{3.985294in}{1.750000in}}{\pgfqpoint{2.279412in}{2.004545in}}%
\pgfusepath{clip}%
\pgfsetbuttcap%
\pgfsetroundjoin%
\pgfsetlinewidth{0.360487pt}%
\definecolor{currentstroke}{rgb}{0.277018,0.050344,0.375715}%
\pgfsetstrokecolor{currentstroke}%
\pgfsetdash{}{0pt}%
\pgfpathmoveto{\pgfqpoint{5.593936in}{3.110546in}}%
\pgfpathlineto{\pgfqpoint{5.543859in}{3.108163in}}%
\pgfusepath{stroke}%
\end{pgfscope}%
\begin{pgfscope}%
\pgfpathrectangle{\pgfqpoint{3.985294in}{1.750000in}}{\pgfqpoint{2.279412in}{2.004545in}}%
\pgfusepath{clip}%
\pgfsetbuttcap%
\pgfsetroundjoin%
\pgfsetlinewidth{0.391256pt}%
\definecolor{currentstroke}{rgb}{0.280894,0.078907,0.402329}%
\pgfsetstrokecolor{currentstroke}%
\pgfsetdash{}{0pt}%
\pgfpathmoveto{\pgfqpoint{5.543859in}{3.108163in}}%
\pgfpathlineto{\pgfqpoint{5.493798in}{3.105581in}}%
\pgfusepath{stroke}%
\end{pgfscope}%
\begin{pgfscope}%
\pgfpathrectangle{\pgfqpoint{3.985294in}{1.750000in}}{\pgfqpoint{2.279412in}{2.004545in}}%
\pgfusepath{clip}%
\pgfsetbuttcap%
\pgfsetroundjoin%
\pgfsetlinewidth{0.426189pt}%
\definecolor{currentstroke}{rgb}{0.282910,0.105393,0.426902}%
\pgfsetstrokecolor{currentstroke}%
\pgfsetdash{}{0pt}%
\pgfpathmoveto{\pgfqpoint{5.493798in}{3.105581in}}%
\pgfpathlineto{\pgfqpoint{5.443817in}{3.102029in}}%
\pgfusepath{stroke}%
\end{pgfscope}%
\begin{pgfscope}%
\pgfpathrectangle{\pgfqpoint{3.985294in}{1.750000in}}{\pgfqpoint{2.279412in}{2.004545in}}%
\pgfusepath{clip}%
\pgfsetbuttcap%
\pgfsetroundjoin%
\pgfsetlinewidth{0.441710pt}%
\definecolor{currentstroke}{rgb}{0.283197,0.115680,0.436115}%
\pgfsetstrokecolor{currentstroke}%
\pgfsetdash{}{0pt}%
\pgfpathmoveto{\pgfqpoint{5.443817in}{3.102029in}}%
\pgfpathlineto{\pgfqpoint{5.393903in}{3.097786in}}%
\pgfusepath{stroke}%
\end{pgfscope}%
\begin{pgfscope}%
\pgfpathrectangle{\pgfqpoint{3.985294in}{1.750000in}}{\pgfqpoint{2.279412in}{2.004545in}}%
\pgfusepath{clip}%
\pgfsetbuttcap%
\pgfsetroundjoin%
\pgfsetlinewidth{0.469980pt}%
\definecolor{currentstroke}{rgb}{0.282884,0.135920,0.453427}%
\pgfsetstrokecolor{currentstroke}%
\pgfsetdash{}{0pt}%
\pgfpathmoveto{\pgfqpoint{5.393903in}{3.097786in}}%
\pgfpathlineto{\pgfqpoint{5.344082in}{3.092778in}}%
\pgfusepath{stroke}%
\end{pgfscope}%
\begin{pgfscope}%
\pgfpathrectangle{\pgfqpoint{3.985294in}{1.750000in}}{\pgfqpoint{2.279412in}{2.004545in}}%
\pgfusepath{clip}%
\pgfsetbuttcap%
\pgfsetroundjoin%
\pgfsetlinewidth{0.471543pt}%
\definecolor{currentstroke}{rgb}{0.282884,0.135920,0.453427}%
\pgfsetstrokecolor{currentstroke}%
\pgfsetdash{}{0pt}%
\pgfpathmoveto{\pgfqpoint{5.344082in}{3.092778in}}%
\pgfpathlineto{\pgfqpoint{5.294427in}{3.086643in}}%
\pgfusepath{stroke}%
\end{pgfscope}%
\begin{pgfscope}%
\pgfpathrectangle{\pgfqpoint{3.985294in}{1.750000in}}{\pgfqpoint{2.279412in}{2.004545in}}%
\pgfusepath{clip}%
\pgfsetbuttcap%
\pgfsetroundjoin%
\pgfsetlinewidth{0.487467pt}%
\definecolor{currentstroke}{rgb}{0.281887,0.150881,0.465405}%
\pgfsetstrokecolor{currentstroke}%
\pgfsetdash{}{0pt}%
\pgfpathmoveto{\pgfqpoint{5.294427in}{3.086643in}}%
\pgfpathlineto{\pgfqpoint{5.245030in}{3.079086in}}%
\pgfusepath{stroke}%
\end{pgfscope}%
\begin{pgfscope}%
\pgfpathrectangle{\pgfqpoint{3.985294in}{1.750000in}}{\pgfqpoint{2.279412in}{2.004545in}}%
\pgfusepath{clip}%
\pgfsetbuttcap%
\pgfsetroundjoin%
\pgfsetlinewidth{0.510452pt}%
\definecolor{currentstroke}{rgb}{0.280255,0.165693,0.476498}%
\pgfsetstrokecolor{currentstroke}%
\pgfsetdash{}{0pt}%
\pgfpathmoveto{\pgfqpoint{5.245030in}{3.079086in}}%
\pgfpathlineto{\pgfqpoint{5.196199in}{3.069160in}}%
\pgfusepath{stroke}%
\end{pgfscope}%
\begin{pgfscope}%
\pgfpathrectangle{\pgfqpoint{3.985294in}{1.750000in}}{\pgfqpoint{2.279412in}{2.004545in}}%
\pgfusepath{clip}%
\pgfsetbuttcap%
\pgfsetroundjoin%
\pgfsetlinewidth{0.470521pt}%
\definecolor{currentstroke}{rgb}{0.282884,0.135920,0.453427}%
\pgfsetstrokecolor{currentstroke}%
\pgfsetdash{}{0pt}%
\pgfpathmoveto{\pgfqpoint{5.196199in}{3.069160in}}%
\pgfpathlineto{\pgfqpoint{5.148440in}{3.055916in}}%
\pgfusepath{stroke}%
\end{pgfscope}%
\begin{pgfscope}%
\pgfpathrectangle{\pgfqpoint{3.985294in}{1.750000in}}{\pgfqpoint{2.279412in}{2.004545in}}%
\pgfusepath{clip}%
\pgfsetbuttcap%
\pgfsetroundjoin%
\pgfsetlinewidth{0.312622pt}%
\definecolor{currentstroke}{rgb}{0.268510,0.009605,0.335427}%
\pgfsetstrokecolor{currentstroke}%
\pgfsetdash{}{0pt}%
\pgfpathmoveto{\pgfqpoint{5.843087in}{3.428874in}}%
\pgfpathlineto{\pgfqpoint{5.792977in}{3.429329in}}%
\pgfusepath{stroke}%
\end{pgfscope}%
\begin{pgfscope}%
\pgfpathrectangle{\pgfqpoint{3.985294in}{1.750000in}}{\pgfqpoint{2.279412in}{2.004545in}}%
\pgfusepath{clip}%
\pgfsetbuttcap%
\pgfsetroundjoin%
\pgfsetlinewidth{0.319677pt}%
\definecolor{currentstroke}{rgb}{0.269944,0.014625,0.341379}%
\pgfsetstrokecolor{currentstroke}%
\pgfsetdash{}{0pt}%
\pgfpathmoveto{\pgfqpoint{5.792977in}{3.429329in}}%
\pgfpathlineto{\pgfqpoint{5.743331in}{3.425848in}}%
\pgfusepath{stroke}%
\end{pgfscope}%
\begin{pgfscope}%
\pgfpathrectangle{\pgfqpoint{3.985294in}{1.750000in}}{\pgfqpoint{2.279412in}{2.004545in}}%
\pgfusepath{clip}%
\pgfsetbuttcap%
\pgfsetroundjoin%
\pgfsetlinewidth{0.314650pt}%
\definecolor{currentstroke}{rgb}{0.268510,0.009605,0.335427}%
\pgfsetstrokecolor{currentstroke}%
\pgfsetdash{}{0pt}%
\pgfpathmoveto{\pgfqpoint{5.743331in}{3.425848in}}%
\pgfpathlineto{\pgfqpoint{5.693810in}{3.420543in}}%
\pgfusepath{stroke}%
\end{pgfscope}%
\begin{pgfscope}%
\pgfpathrectangle{\pgfqpoint{3.985294in}{1.750000in}}{\pgfqpoint{2.279412in}{2.004545in}}%
\pgfusepath{clip}%
\pgfsetbuttcap%
\pgfsetroundjoin%
\pgfsetlinewidth{0.327560pt}%
\definecolor{currentstroke}{rgb}{0.271305,0.019942,0.347269}%
\pgfsetstrokecolor{currentstroke}%
\pgfsetdash{}{0pt}%
\pgfpathmoveto{\pgfqpoint{5.693810in}{3.420543in}}%
\pgfpathlineto{\pgfqpoint{5.643748in}{3.420629in}}%
\pgfusepath{stroke}%
\end{pgfscope}%
\begin{pgfscope}%
\pgfpathrectangle{\pgfqpoint{3.985294in}{1.750000in}}{\pgfqpoint{2.279412in}{2.004545in}}%
\pgfusepath{clip}%
\pgfsetbuttcap%
\pgfsetroundjoin%
\pgfsetlinewidth{0.319603pt}%
\definecolor{currentstroke}{rgb}{0.269944,0.014625,0.341379}%
\pgfsetstrokecolor{currentstroke}%
\pgfsetdash{}{0pt}%
\pgfpathmoveto{\pgfqpoint{5.643748in}{3.420629in}}%
\pgfpathlineto{\pgfqpoint{5.593833in}{3.418793in}}%
\pgfusepath{stroke}%
\end{pgfscope}%
\begin{pgfscope}%
\pgfpathrectangle{\pgfqpoint{3.985294in}{1.750000in}}{\pgfqpoint{2.279412in}{2.004545in}}%
\pgfusepath{clip}%
\pgfsetbuttcap%
\pgfsetroundjoin%
\pgfsetlinewidth{0.312445pt}%
\definecolor{currentstroke}{rgb}{0.268510,0.009605,0.335427}%
\pgfsetstrokecolor{currentstroke}%
\pgfsetdash{}{0pt}%
\pgfpathmoveto{\pgfqpoint{5.889227in}{2.436758in}}%
\pgfpathlineto{\pgfqpoint{5.843087in}{2.436525in}}%
\pgfusepath{stroke}%
\end{pgfscope}%
\begin{pgfscope}%
\pgfpathrectangle{\pgfqpoint{3.985294in}{1.750000in}}{\pgfqpoint{2.279412in}{2.004545in}}%
\pgfusepath{clip}%
\pgfsetbuttcap%
\pgfsetroundjoin%
\pgfsetlinewidth{0.317222pt}%
\definecolor{currentstroke}{rgb}{0.269944,0.014625,0.341379}%
\pgfsetstrokecolor{currentstroke}%
\pgfsetdash{}{0pt}%
\pgfpathmoveto{\pgfqpoint{5.843087in}{2.436525in}}%
\pgfpathlineto{\pgfqpoint{5.796945in}{2.435863in}}%
\pgfusepath{stroke}%
\end{pgfscope}%
\begin{pgfscope}%
\pgfpathrectangle{\pgfqpoint{3.985294in}{1.750000in}}{\pgfqpoint{2.279412in}{2.004545in}}%
\pgfusepath{clip}%
\pgfsetbuttcap%
\pgfsetroundjoin%
\pgfsetlinewidth{0.323305pt}%
\definecolor{currentstroke}{rgb}{0.271305,0.019942,0.347269}%
\pgfsetstrokecolor{currentstroke}%
\pgfsetdash{}{0pt}%
\pgfpathmoveto{\pgfqpoint{5.796945in}{2.435863in}}%
\pgfpathlineto{\pgfqpoint{5.746817in}{2.435508in}}%
\pgfusepath{stroke}%
\end{pgfscope}%
\begin{pgfscope}%
\pgfpathrectangle{\pgfqpoint{3.985294in}{1.750000in}}{\pgfqpoint{2.279412in}{2.004545in}}%
\pgfusepath{clip}%
\pgfsetbuttcap%
\pgfsetroundjoin%
\pgfsetlinewidth{0.333541pt}%
\definecolor{currentstroke}{rgb}{0.272594,0.025563,0.353093}%
\pgfsetstrokecolor{currentstroke}%
\pgfsetdash{}{0pt}%
\pgfpathmoveto{\pgfqpoint{5.746817in}{2.435508in}}%
\pgfpathlineto{\pgfqpoint{5.696686in}{2.435286in}}%
\pgfusepath{stroke}%
\end{pgfscope}%
\begin{pgfscope}%
\pgfpathrectangle{\pgfqpoint{3.985294in}{1.750000in}}{\pgfqpoint{2.279412in}{2.004545in}}%
\pgfusepath{clip}%
\pgfsetbuttcap%
\pgfsetroundjoin%
\pgfsetlinewidth{0.342604pt}%
\definecolor{currentstroke}{rgb}{0.274952,0.037752,0.364543}%
\pgfsetstrokecolor{currentstroke}%
\pgfsetdash{}{0pt}%
\pgfpathmoveto{\pgfqpoint{5.696686in}{2.435286in}}%
\pgfpathlineto{\pgfqpoint{5.646539in}{2.434918in}}%
\pgfusepath{stroke}%
\end{pgfscope}%
\begin{pgfscope}%
\pgfpathrectangle{\pgfqpoint{3.985294in}{1.750000in}}{\pgfqpoint{2.279412in}{2.004545in}}%
\pgfusepath{clip}%
\pgfsetbuttcap%
\pgfsetroundjoin%
\pgfsetlinewidth{0.356956pt}%
\definecolor{currentstroke}{rgb}{0.277018,0.050344,0.375715}%
\pgfsetstrokecolor{currentstroke}%
\pgfsetdash{}{0pt}%
\pgfpathmoveto{\pgfqpoint{5.646539in}{2.434918in}}%
\pgfpathlineto{\pgfqpoint{5.596408in}{2.435539in}}%
\pgfusepath{stroke}%
\end{pgfscope}%
\begin{pgfscope}%
\pgfpathrectangle{\pgfqpoint{3.985294in}{1.750000in}}{\pgfqpoint{2.279412in}{2.004545in}}%
\pgfusepath{clip}%
\pgfsetbuttcap%
\pgfsetroundjoin%
\pgfsetlinewidth{0.377201pt}%
\definecolor{currentstroke}{rgb}{0.279566,0.067836,0.391917}%
\pgfsetstrokecolor{currentstroke}%
\pgfsetdash{}{0pt}%
\pgfpathmoveto{\pgfqpoint{5.596408in}{2.435539in}}%
\pgfpathlineto{\pgfqpoint{5.546297in}{2.437281in}}%
\pgfusepath{stroke}%
\end{pgfscope}%
\begin{pgfscope}%
\pgfpathrectangle{\pgfqpoint{3.985294in}{1.750000in}}{\pgfqpoint{2.279412in}{2.004545in}}%
\pgfusepath{clip}%
\pgfsetbuttcap%
\pgfsetroundjoin%
\pgfsetlinewidth{0.416547pt}%
\definecolor{currentstroke}{rgb}{0.282327,0.094955,0.417331}%
\pgfsetstrokecolor{currentstroke}%
\pgfsetdash{}{0pt}%
\pgfpathmoveto{\pgfqpoint{5.546297in}{2.437281in}}%
\pgfpathlineto{\pgfqpoint{5.496191in}{2.439142in}}%
\pgfusepath{stroke}%
\end{pgfscope}%
\begin{pgfscope}%
\pgfpathrectangle{\pgfqpoint{3.985294in}{1.750000in}}{\pgfqpoint{2.279412in}{2.004545in}}%
\pgfusepath{clip}%
\pgfsetbuttcap%
\pgfsetroundjoin%
\pgfsetlinewidth{0.459295pt}%
\definecolor{currentstroke}{rgb}{0.283072,0.130895,0.449241}%
\pgfsetstrokecolor{currentstroke}%
\pgfsetdash{}{0pt}%
\pgfpathmoveto{\pgfqpoint{5.496191in}{2.439142in}}%
\pgfpathlineto{\pgfqpoint{5.446139in}{2.441807in}}%
\pgfusepath{stroke}%
\end{pgfscope}%
\begin{pgfscope}%
\pgfpathrectangle{\pgfqpoint{3.985294in}{1.750000in}}{\pgfqpoint{2.279412in}{2.004545in}}%
\pgfusepath{clip}%
\pgfsetbuttcap%
\pgfsetroundjoin%
\pgfsetlinewidth{0.478824pt}%
\definecolor{currentstroke}{rgb}{0.282623,0.140926,0.457517}%
\pgfsetstrokecolor{currentstroke}%
\pgfsetdash{}{0pt}%
\pgfpathmoveto{\pgfqpoint{5.446139in}{2.441807in}}%
\pgfpathlineto{\pgfqpoint{5.396177in}{2.445610in}}%
\pgfusepath{stroke}%
\end{pgfscope}%
\begin{pgfscope}%
\pgfpathrectangle{\pgfqpoint{3.985294in}{1.750000in}}{\pgfqpoint{2.279412in}{2.004545in}}%
\pgfusepath{clip}%
\pgfsetbuttcap%
\pgfsetroundjoin%
\pgfsetlinewidth{0.316556pt}%
\definecolor{currentstroke}{rgb}{0.269944,0.014625,0.341379}%
\pgfsetstrokecolor{currentstroke}%
\pgfsetdash{}{0pt}%
\pgfpathmoveto{\pgfqpoint{5.843087in}{2.481632in}}%
\pgfpathlineto{\pgfqpoint{5.792949in}{2.481802in}}%
\pgfusepath{stroke}%
\end{pgfscope}%
\begin{pgfscope}%
\pgfpathrectangle{\pgfqpoint{3.985294in}{1.750000in}}{\pgfqpoint{2.279412in}{2.004545in}}%
\pgfusepath{clip}%
\pgfsetbuttcap%
\pgfsetroundjoin%
\pgfsetlinewidth{0.323306pt}%
\definecolor{currentstroke}{rgb}{0.271305,0.019942,0.347269}%
\pgfsetstrokecolor{currentstroke}%
\pgfsetdash{}{0pt}%
\pgfpathmoveto{\pgfqpoint{5.792949in}{2.481802in}}%
\pgfpathlineto{\pgfqpoint{5.742816in}{2.481600in}}%
\pgfusepath{stroke}%
\end{pgfscope}%
\begin{pgfscope}%
\pgfpathrectangle{\pgfqpoint{3.985294in}{1.750000in}}{\pgfqpoint{2.279412in}{2.004545in}}%
\pgfusepath{clip}%
\pgfsetbuttcap%
\pgfsetroundjoin%
\pgfsetlinewidth{0.328559pt}%
\definecolor{currentstroke}{rgb}{0.271305,0.019942,0.347269}%
\pgfsetstrokecolor{currentstroke}%
\pgfsetdash{}{0pt}%
\pgfpathmoveto{\pgfqpoint{5.742816in}{2.481600in}}%
\pgfpathlineto{\pgfqpoint{5.692676in}{2.481978in}}%
\pgfusepath{stroke}%
\end{pgfscope}%
\begin{pgfscope}%
\pgfpathrectangle{\pgfqpoint{3.985294in}{1.750000in}}{\pgfqpoint{2.279412in}{2.004545in}}%
\pgfusepath{clip}%
\pgfsetbuttcap%
\pgfsetroundjoin%
\pgfsetlinewidth{0.344703pt}%
\definecolor{currentstroke}{rgb}{0.274952,0.037752,0.364543}%
\pgfsetstrokecolor{currentstroke}%
\pgfsetdash{}{0pt}%
\pgfpathmoveto{\pgfqpoint{5.692676in}{2.481978in}}%
\pgfpathlineto{\pgfqpoint{5.642543in}{2.483033in}}%
\pgfusepath{stroke}%
\end{pgfscope}%
\begin{pgfscope}%
\pgfpathrectangle{\pgfqpoint{3.985294in}{1.750000in}}{\pgfqpoint{2.279412in}{2.004545in}}%
\pgfusepath{clip}%
\pgfsetbuttcap%
\pgfsetroundjoin%
\pgfsetlinewidth{0.367215pt}%
\definecolor{currentstroke}{rgb}{0.277941,0.056324,0.381191}%
\pgfsetstrokecolor{currentstroke}%
\pgfsetdash{}{0pt}%
\pgfpathmoveto{\pgfqpoint{5.642543in}{2.483033in}}%
\pgfpathlineto{\pgfqpoint{5.592418in}{2.484407in}}%
\pgfusepath{stroke}%
\end{pgfscope}%
\begin{pgfscope}%
\pgfpathrectangle{\pgfqpoint{3.985294in}{1.750000in}}{\pgfqpoint{2.279412in}{2.004545in}}%
\pgfusepath{clip}%
\pgfsetbuttcap%
\pgfsetroundjoin%
\pgfsetlinewidth{0.394714pt}%
\definecolor{currentstroke}{rgb}{0.280894,0.078907,0.402329}%
\pgfsetstrokecolor{currentstroke}%
\pgfsetdash{}{0pt}%
\pgfpathmoveto{\pgfqpoint{5.592418in}{2.484407in}}%
\pgfpathlineto{\pgfqpoint{5.542298in}{2.485923in}}%
\pgfusepath{stroke}%
\end{pgfscope}%
\begin{pgfscope}%
\pgfpathrectangle{\pgfqpoint{3.985294in}{1.750000in}}{\pgfqpoint{2.279412in}{2.004545in}}%
\pgfusepath{clip}%
\pgfsetbuttcap%
\pgfsetroundjoin%
\pgfsetlinewidth{0.432137pt}%
\definecolor{currentstroke}{rgb}{0.283091,0.110553,0.431554}%
\pgfsetstrokecolor{currentstroke}%
\pgfsetdash{}{0pt}%
\pgfpathmoveto{\pgfqpoint{5.542298in}{2.485923in}}%
\pgfpathlineto{\pgfqpoint{5.492190in}{2.487759in}}%
\pgfusepath{stroke}%
\end{pgfscope}%
\begin{pgfscope}%
\pgfpathrectangle{\pgfqpoint{3.985294in}{1.750000in}}{\pgfqpoint{2.279412in}{2.004545in}}%
\pgfusepath{clip}%
\pgfsetbuttcap%
\pgfsetroundjoin%
\pgfsetlinewidth{0.482819pt}%
\definecolor{currentstroke}{rgb}{0.282290,0.145912,0.461510}%
\pgfsetstrokecolor{currentstroke}%
\pgfsetdash{}{0pt}%
\pgfpathmoveto{\pgfqpoint{5.492190in}{2.487759in}}%
\pgfpathlineto{\pgfqpoint{5.442108in}{2.490062in}}%
\pgfusepath{stroke}%
\end{pgfscope}%
\begin{pgfscope}%
\pgfpathrectangle{\pgfqpoint{3.985294in}{1.750000in}}{\pgfqpoint{2.279412in}{2.004545in}}%
\pgfusepath{clip}%
\pgfsetbuttcap%
\pgfsetroundjoin%
\pgfsetlinewidth{0.514619pt}%
\definecolor{currentstroke}{rgb}{0.279574,0.170599,0.479997}%
\pgfsetstrokecolor{currentstroke}%
\pgfsetdash{}{0pt}%
\pgfpathmoveto{\pgfqpoint{5.442108in}{2.490062in}}%
\pgfpathlineto{\pgfqpoint{5.392066in}{2.492967in}}%
\pgfusepath{stroke}%
\end{pgfscope}%
\begin{pgfscope}%
\pgfpathrectangle{\pgfqpoint{3.985294in}{1.750000in}}{\pgfqpoint{2.279412in}{2.004545in}}%
\pgfusepath{clip}%
\pgfsetbuttcap%
\pgfsetroundjoin%
\pgfsetlinewidth{0.600859pt}%
\definecolor{currentstroke}{rgb}{0.266580,0.228262,0.514349}%
\pgfsetstrokecolor{currentstroke}%
\pgfsetdash{}{0pt}%
\pgfpathmoveto{\pgfqpoint{5.392066in}{2.492967in}}%
\pgfpathlineto{\pgfqpoint{5.342087in}{2.496580in}}%
\pgfusepath{stroke}%
\end{pgfscope}%
\begin{pgfscope}%
\pgfpathrectangle{\pgfqpoint{3.985294in}{1.750000in}}{\pgfqpoint{2.279412in}{2.004545in}}%
\pgfusepath{clip}%
\pgfsetbuttcap%
\pgfsetroundjoin%
\pgfsetlinewidth{0.642925pt}%
\definecolor{currentstroke}{rgb}{0.257322,0.256130,0.526563}%
\pgfsetstrokecolor{currentstroke}%
\pgfsetdash{}{0pt}%
\pgfpathmoveto{\pgfqpoint{5.342087in}{2.496580in}}%
\pgfpathlineto{\pgfqpoint{5.292206in}{2.501121in}}%
\pgfusepath{stroke}%
\end{pgfscope}%
\begin{pgfscope}%
\pgfpathrectangle{\pgfqpoint{3.985294in}{1.750000in}}{\pgfqpoint{2.279412in}{2.004545in}}%
\pgfusepath{clip}%
\pgfsetbuttcap%
\pgfsetroundjoin%
\pgfsetlinewidth{0.673228pt}%
\definecolor{currentstroke}{rgb}{0.248629,0.278775,0.534556}%
\pgfsetstrokecolor{currentstroke}%
\pgfsetdash{}{0pt}%
\pgfpathmoveto{\pgfqpoint{5.292206in}{2.501121in}}%
\pgfpathlineto{\pgfqpoint{5.242438in}{2.506526in}}%
\pgfusepath{stroke}%
\end{pgfscope}%
\begin{pgfscope}%
\pgfpathrectangle{\pgfqpoint{3.985294in}{1.750000in}}{\pgfqpoint{2.279412in}{2.004545in}}%
\pgfusepath{clip}%
\pgfsetbuttcap%
\pgfsetroundjoin%
\pgfsetlinewidth{0.757968pt}%
\definecolor{currentstroke}{rgb}{0.225863,0.330805,0.547314}%
\pgfsetstrokecolor{currentstroke}%
\pgfsetdash{}{0pt}%
\pgfpathmoveto{\pgfqpoint{5.242438in}{2.506526in}}%
\pgfpathlineto{\pgfqpoint{5.192867in}{2.513159in}}%
\pgfusepath{stroke}%
\end{pgfscope}%
\begin{pgfscope}%
\pgfpathrectangle{\pgfqpoint{3.985294in}{1.750000in}}{\pgfqpoint{2.279412in}{2.004545in}}%
\pgfusepath{clip}%
\pgfsetbuttcap%
\pgfsetroundjoin%
\pgfsetlinewidth{0.725278pt}%
\definecolor{currentstroke}{rgb}{0.235526,0.309527,0.542944}%
\pgfsetstrokecolor{currentstroke}%
\pgfsetdash{}{0pt}%
\pgfpathmoveto{\pgfqpoint{5.192867in}{2.513159in}}%
\pgfpathlineto{\pgfqpoint{5.143577in}{2.521264in}}%
\pgfusepath{stroke}%
\end{pgfscope}%
\begin{pgfscope}%
\pgfpathrectangle{\pgfqpoint{3.985294in}{1.750000in}}{\pgfqpoint{2.279412in}{2.004545in}}%
\pgfusepath{clip}%
\pgfsetbuttcap%
\pgfsetroundjoin%
\pgfsetlinewidth{0.731164pt}%
\definecolor{currentstroke}{rgb}{0.233603,0.313828,0.543914}%
\pgfsetstrokecolor{currentstroke}%
\pgfsetdash{}{0pt}%
\pgfpathmoveto{\pgfqpoint{5.143577in}{2.521264in}}%
\pgfpathlineto{\pgfqpoint{5.094629in}{2.530800in}}%
\pgfusepath{stroke}%
\end{pgfscope}%
\begin{pgfscope}%
\pgfpathrectangle{\pgfqpoint{3.985294in}{1.750000in}}{\pgfqpoint{2.279412in}{2.004545in}}%
\pgfusepath{clip}%
\pgfsetbuttcap%
\pgfsetroundjoin%
\pgfsetlinewidth{0.762922pt}%
\definecolor{currentstroke}{rgb}{0.223925,0.334994,0.548053}%
\pgfsetstrokecolor{currentstroke}%
\pgfsetdash{}{0pt}%
\pgfpathmoveto{\pgfqpoint{5.094629in}{2.530800in}}%
\pgfpathlineto{\pgfqpoint{5.046184in}{2.542122in}}%
\pgfusepath{stroke}%
\end{pgfscope}%
\begin{pgfscope}%
\pgfpathrectangle{\pgfqpoint{3.985294in}{1.750000in}}{\pgfqpoint{2.279412in}{2.004545in}}%
\pgfusepath{clip}%
\pgfsetbuttcap%
\pgfsetroundjoin%
\pgfsetlinewidth{0.747276pt}%
\definecolor{currentstroke}{rgb}{0.227802,0.326594,0.546532}%
\pgfsetstrokecolor{currentstroke}%
\pgfsetdash{}{0pt}%
\pgfpathmoveto{\pgfqpoint{5.046184in}{2.542122in}}%
\pgfpathlineto{\pgfqpoint{4.998523in}{2.555758in}}%
\pgfusepath{stroke}%
\end{pgfscope}%
\begin{pgfscope}%
\pgfpathrectangle{\pgfqpoint{3.985294in}{1.750000in}}{\pgfqpoint{2.279412in}{2.004545in}}%
\pgfusepath{clip}%
\pgfsetbuttcap%
\pgfsetroundjoin%
\pgfsetlinewidth{0.813264pt}%
\definecolor{currentstroke}{rgb}{0.210503,0.363727,0.552206}%
\pgfsetstrokecolor{currentstroke}%
\pgfsetdash{}{0pt}%
\pgfpathmoveto{\pgfqpoint{4.998523in}{2.555758in}}%
\pgfpathlineto{\pgfqpoint{4.951559in}{2.571176in}}%
\pgfusepath{stroke}%
\end{pgfscope}%
\begin{pgfscope}%
\pgfpathrectangle{\pgfqpoint{3.985294in}{1.750000in}}{\pgfqpoint{2.279412in}{2.004545in}}%
\pgfusepath{clip}%
\pgfsetbuttcap%
\pgfsetroundjoin%
\pgfsetlinewidth{0.838826pt}%
\definecolor{currentstroke}{rgb}{0.203063,0.379716,0.553925}%
\pgfsetstrokecolor{currentstroke}%
\pgfsetdash{}{0pt}%
\pgfpathmoveto{\pgfqpoint{4.951559in}{2.571176in}}%
\pgfpathlineto{\pgfqpoint{4.905382in}{2.588282in}}%
\pgfusepath{stroke}%
\end{pgfscope}%
\begin{pgfscope}%
\pgfpathrectangle{\pgfqpoint{3.985294in}{1.750000in}}{\pgfqpoint{2.279412in}{2.004545in}}%
\pgfusepath{clip}%
\pgfsetbuttcap%
\pgfsetroundjoin%
\pgfsetlinewidth{0.811673pt}%
\definecolor{currentstroke}{rgb}{0.210503,0.363727,0.552206}%
\pgfsetstrokecolor{currentstroke}%
\pgfsetdash{}{0pt}%
\pgfpathmoveto{\pgfqpoint{4.905382in}{2.588282in}}%
\pgfpathlineto{\pgfqpoint{4.859564in}{2.606068in}}%
\pgfusepath{stroke}%
\end{pgfscope}%
\begin{pgfscope}%
\pgfpathrectangle{\pgfqpoint{3.985294in}{1.750000in}}{\pgfqpoint{2.279412in}{2.004545in}}%
\pgfusepath{clip}%
\pgfsetbuttcap%
\pgfsetroundjoin%
\pgfsetlinewidth{0.321468pt}%
\definecolor{currentstroke}{rgb}{0.269944,0.014625,0.341379}%
\pgfsetstrokecolor{currentstroke}%
\pgfsetdash{}{0pt}%
\pgfpathmoveto{\pgfqpoint{5.843087in}{2.662059in}}%
\pgfpathlineto{\pgfqpoint{5.792964in}{2.662193in}}%
\pgfusepath{stroke}%
\end{pgfscope}%
\begin{pgfscope}%
\pgfpathrectangle{\pgfqpoint{3.985294in}{1.750000in}}{\pgfqpoint{2.279412in}{2.004545in}}%
\pgfusepath{clip}%
\pgfsetbuttcap%
\pgfsetroundjoin%
\pgfsetlinewidth{0.325606pt}%
\definecolor{currentstroke}{rgb}{0.271305,0.019942,0.347269}%
\pgfsetstrokecolor{currentstroke}%
\pgfsetdash{}{0pt}%
\pgfpathmoveto{\pgfqpoint{5.792964in}{2.662193in}}%
\pgfpathlineto{\pgfqpoint{5.742843in}{2.662295in}}%
\pgfusepath{stroke}%
\end{pgfscope}%
\begin{pgfscope}%
\pgfpathrectangle{\pgfqpoint{3.985294in}{1.750000in}}{\pgfqpoint{2.279412in}{2.004545in}}%
\pgfusepath{clip}%
\pgfsetbuttcap%
\pgfsetroundjoin%
\pgfsetlinewidth{0.332664pt}%
\definecolor{currentstroke}{rgb}{0.272594,0.025563,0.353093}%
\pgfsetstrokecolor{currentstroke}%
\pgfsetdash{}{0pt}%
\pgfpathmoveto{\pgfqpoint{5.742843in}{2.662295in}}%
\pgfpathlineto{\pgfqpoint{5.692712in}{2.662791in}}%
\pgfusepath{stroke}%
\end{pgfscope}%
\begin{pgfscope}%
\pgfpathrectangle{\pgfqpoint{3.985294in}{1.750000in}}{\pgfqpoint{2.279412in}{2.004545in}}%
\pgfusepath{clip}%
\pgfsetbuttcap%
\pgfsetroundjoin%
\pgfsetlinewidth{0.350174pt}%
\definecolor{currentstroke}{rgb}{0.276022,0.044167,0.370164}%
\pgfsetstrokecolor{currentstroke}%
\pgfsetdash{}{0pt}%
\pgfpathmoveto{\pgfqpoint{5.692712in}{2.662791in}}%
\pgfpathlineto{\pgfqpoint{5.642568in}{2.662754in}}%
\pgfusepath{stroke}%
\end{pgfscope}%
\begin{pgfscope}%
\pgfpathrectangle{\pgfqpoint{3.985294in}{1.750000in}}{\pgfqpoint{2.279412in}{2.004545in}}%
\pgfusepath{clip}%
\pgfsetbuttcap%
\pgfsetroundjoin%
\pgfsetlinewidth{0.377938pt}%
\definecolor{currentstroke}{rgb}{0.279566,0.067836,0.391917}%
\pgfsetstrokecolor{currentstroke}%
\pgfsetdash{}{0pt}%
\pgfpathmoveto{\pgfqpoint{5.642568in}{2.662754in}}%
\pgfpathlineto{\pgfqpoint{5.592418in}{2.662874in}}%
\pgfusepath{stroke}%
\end{pgfscope}%
\begin{pgfscope}%
\pgfpathrectangle{\pgfqpoint{3.985294in}{1.750000in}}{\pgfqpoint{2.279412in}{2.004545in}}%
\pgfusepath{clip}%
\pgfsetbuttcap%
\pgfsetroundjoin%
\pgfsetlinewidth{0.419775pt}%
\definecolor{currentstroke}{rgb}{0.282656,0.100196,0.422160}%
\pgfsetstrokecolor{currentstroke}%
\pgfsetdash{}{0pt}%
\pgfpathmoveto{\pgfqpoint{5.592418in}{2.662874in}}%
\pgfpathlineto{\pgfqpoint{5.542267in}{2.663088in}}%
\pgfusepath{stroke}%
\end{pgfscope}%
\begin{pgfscope}%
\pgfpathrectangle{\pgfqpoint{3.985294in}{1.750000in}}{\pgfqpoint{2.279412in}{2.004545in}}%
\pgfusepath{clip}%
\pgfsetbuttcap%
\pgfsetroundjoin%
\pgfsetlinewidth{0.481217pt}%
\definecolor{currentstroke}{rgb}{0.282290,0.145912,0.461510}%
\pgfsetstrokecolor{currentstroke}%
\pgfsetdash{}{0pt}%
\pgfpathmoveto{\pgfqpoint{5.542267in}{2.663088in}}%
\pgfpathlineto{\pgfqpoint{5.492116in}{2.663380in}}%
\pgfusepath{stroke}%
\end{pgfscope}%
\begin{pgfscope}%
\pgfpathrectangle{\pgfqpoint{3.985294in}{1.750000in}}{\pgfqpoint{2.279412in}{2.004545in}}%
\pgfusepath{clip}%
\pgfsetbuttcap%
\pgfsetroundjoin%
\pgfsetlinewidth{0.564277pt}%
\definecolor{currentstroke}{rgb}{0.273006,0.204520,0.501721}%
\pgfsetstrokecolor{currentstroke}%
\pgfsetdash{}{0pt}%
\pgfpathmoveto{\pgfqpoint{5.492116in}{2.663380in}}%
\pgfpathlineto{\pgfqpoint{5.441966in}{2.663778in}}%
\pgfusepath{stroke}%
\end{pgfscope}%
\begin{pgfscope}%
\pgfpathrectangle{\pgfqpoint{3.985294in}{1.750000in}}{\pgfqpoint{2.279412in}{2.004545in}}%
\pgfusepath{clip}%
\pgfsetbuttcap%
\pgfsetroundjoin%
\pgfsetlinewidth{0.675207pt}%
\definecolor{currentstroke}{rgb}{0.248629,0.278775,0.534556}%
\pgfsetstrokecolor{currentstroke}%
\pgfsetdash{}{0pt}%
\pgfpathmoveto{\pgfqpoint{5.441966in}{2.663778in}}%
\pgfpathlineto{\pgfqpoint{5.391817in}{2.664248in}}%
\pgfusepath{stroke}%
\end{pgfscope}%
\begin{pgfscope}%
\pgfpathrectangle{\pgfqpoint{3.985294in}{1.750000in}}{\pgfqpoint{2.279412in}{2.004545in}}%
\pgfusepath{clip}%
\pgfsetbuttcap%
\pgfsetroundjoin%
\pgfsetlinewidth{0.788553pt}%
\definecolor{currentstroke}{rgb}{0.216210,0.351535,0.550627}%
\pgfsetstrokecolor{currentstroke}%
\pgfsetdash{}{0pt}%
\pgfpathmoveto{\pgfqpoint{5.391817in}{2.664248in}}%
\pgfpathlineto{\pgfqpoint{5.341671in}{2.664898in}}%
\pgfusepath{stroke}%
\end{pgfscope}%
\begin{pgfscope}%
\pgfpathrectangle{\pgfqpoint{3.985294in}{1.750000in}}{\pgfqpoint{2.279412in}{2.004545in}}%
\pgfusepath{clip}%
\pgfsetbuttcap%
\pgfsetroundjoin%
\pgfsetlinewidth{0.864729pt}%
\definecolor{currentstroke}{rgb}{0.195860,0.395433,0.555276}%
\pgfsetstrokecolor{currentstroke}%
\pgfsetdash{}{0pt}%
\pgfpathmoveto{\pgfqpoint{5.341671in}{2.664898in}}%
\pgfpathlineto{\pgfqpoint{5.291530in}{2.665785in}}%
\pgfusepath{stroke}%
\end{pgfscope}%
\begin{pgfscope}%
\pgfpathrectangle{\pgfqpoint{3.985294in}{1.750000in}}{\pgfqpoint{2.279412in}{2.004545in}}%
\pgfusepath{clip}%
\pgfsetbuttcap%
\pgfsetroundjoin%
\pgfsetlinewidth{0.951399pt}%
\definecolor{currentstroke}{rgb}{0.175841,0.441290,0.557685}%
\pgfsetstrokecolor{currentstroke}%
\pgfsetdash{}{0pt}%
\pgfpathmoveto{\pgfqpoint{5.291530in}{2.665785in}}%
\pgfpathlineto{\pgfqpoint{5.241396in}{2.666958in}}%
\pgfusepath{stroke}%
\end{pgfscope}%
\begin{pgfscope}%
\pgfpathrectangle{\pgfqpoint{3.985294in}{1.750000in}}{\pgfqpoint{2.279412in}{2.004545in}}%
\pgfusepath{clip}%
\pgfsetbuttcap%
\pgfsetroundjoin%
\pgfsetlinewidth{0.987429pt}%
\definecolor{currentstroke}{rgb}{0.166617,0.463708,0.558119}%
\pgfsetstrokecolor{currentstroke}%
\pgfsetdash{}{0pt}%
\pgfpathmoveto{\pgfqpoint{5.241396in}{2.666958in}}%
\pgfpathlineto{\pgfqpoint{5.191272in}{2.668378in}}%
\pgfusepath{stroke}%
\end{pgfscope}%
\begin{pgfscope}%
\pgfpathrectangle{\pgfqpoint{3.985294in}{1.750000in}}{\pgfqpoint{2.279412in}{2.004545in}}%
\pgfusepath{clip}%
\pgfsetbuttcap%
\pgfsetroundjoin%
\pgfsetlinewidth{0.998662pt}%
\definecolor{currentstroke}{rgb}{0.165117,0.467423,0.558141}%
\pgfsetstrokecolor{currentstroke}%
\pgfsetdash{}{0pt}%
\pgfpathmoveto{\pgfqpoint{5.191272in}{2.668378in}}%
\pgfpathlineto{\pgfqpoint{5.141166in}{2.670214in}}%
\pgfusepath{stroke}%
\end{pgfscope}%
\begin{pgfscope}%
\pgfpathrectangle{\pgfqpoint{3.985294in}{1.750000in}}{\pgfqpoint{2.279412in}{2.004545in}}%
\pgfusepath{clip}%
\pgfsetbuttcap%
\pgfsetroundjoin%
\pgfsetlinewidth{0.991063pt}%
\definecolor{currentstroke}{rgb}{0.166617,0.463708,0.558119}%
\pgfsetstrokecolor{currentstroke}%
\pgfsetdash{}{0pt}%
\pgfpathmoveto{\pgfqpoint{5.141166in}{2.670214in}}%
\pgfpathlineto{\pgfqpoint{5.091086in}{2.672553in}}%
\pgfusepath{stroke}%
\end{pgfscope}%
\begin{pgfscope}%
\pgfpathrectangle{\pgfqpoint{3.985294in}{1.750000in}}{\pgfqpoint{2.279412in}{2.004545in}}%
\pgfusepath{clip}%
\pgfsetbuttcap%
\pgfsetroundjoin%
\pgfsetlinewidth{1.021856pt}%
\definecolor{currentstroke}{rgb}{0.159194,0.482237,0.558073}%
\pgfsetstrokecolor{currentstroke}%
\pgfsetdash{}{0pt}%
\pgfpathmoveto{\pgfqpoint{5.091086in}{2.672553in}}%
\pgfpathlineto{\pgfqpoint{5.041031in}{2.675275in}}%
\pgfusepath{stroke}%
\end{pgfscope}%
\begin{pgfscope}%
\pgfpathrectangle{\pgfqpoint{3.985294in}{1.750000in}}{\pgfqpoint{2.279412in}{2.004545in}}%
\pgfusepath{clip}%
\pgfsetbuttcap%
\pgfsetroundjoin%
\pgfsetlinewidth{0.964873pt}%
\definecolor{currentstroke}{rgb}{0.172719,0.448791,0.557885}%
\pgfsetstrokecolor{currentstroke}%
\pgfsetdash{}{0pt}%
\pgfpathmoveto{\pgfqpoint{5.041031in}{2.675275in}}%
\pgfpathlineto{\pgfqpoint{4.991021in}{2.678524in}}%
\pgfusepath{stroke}%
\end{pgfscope}%
\begin{pgfscope}%
\pgfpathrectangle{\pgfqpoint{3.985294in}{1.750000in}}{\pgfqpoint{2.279412in}{2.004545in}}%
\pgfusepath{clip}%
\pgfsetbuttcap%
\pgfsetroundjoin%
\pgfsetlinewidth{0.323155pt}%
\definecolor{currentstroke}{rgb}{0.271305,0.019942,0.347269}%
\pgfsetstrokecolor{currentstroke}%
\pgfsetdash{}{0pt}%
\pgfpathmoveto{\pgfqpoint{5.843087in}{2.887593in}}%
\pgfpathlineto{\pgfqpoint{5.792951in}{2.886911in}}%
\pgfusepath{stroke}%
\end{pgfscope}%
\begin{pgfscope}%
\pgfpathrectangle{\pgfqpoint{3.985294in}{1.750000in}}{\pgfqpoint{2.279412in}{2.004545in}}%
\pgfusepath{clip}%
\pgfsetbuttcap%
\pgfsetroundjoin%
\pgfsetlinewidth{0.320576pt}%
\definecolor{currentstroke}{rgb}{0.269944,0.014625,0.341379}%
\pgfsetstrokecolor{currentstroke}%
\pgfsetdash{}{0pt}%
\pgfpathmoveto{\pgfqpoint{5.792951in}{2.886911in}}%
\pgfpathlineto{\pgfqpoint{5.742809in}{2.886637in}}%
\pgfusepath{stroke}%
\end{pgfscope}%
\begin{pgfscope}%
\pgfpathrectangle{\pgfqpoint{3.985294in}{1.750000in}}{\pgfqpoint{2.279412in}{2.004545in}}%
\pgfusepath{clip}%
\pgfsetbuttcap%
\pgfsetroundjoin%
\pgfsetlinewidth{0.339115pt}%
\definecolor{currentstroke}{rgb}{0.273809,0.031497,0.358853}%
\pgfsetstrokecolor{currentstroke}%
\pgfsetdash{}{0pt}%
\pgfpathmoveto{\pgfqpoint{5.742809in}{2.886637in}}%
\pgfpathlineto{\pgfqpoint{5.692660in}{2.886926in}}%
\pgfusepath{stroke}%
\end{pgfscope}%
\begin{pgfscope}%
\pgfpathrectangle{\pgfqpoint{3.985294in}{1.750000in}}{\pgfqpoint{2.279412in}{2.004545in}}%
\pgfusepath{clip}%
\pgfsetbuttcap%
\pgfsetroundjoin%
\pgfsetlinewidth{0.344296pt}%
\definecolor{currentstroke}{rgb}{0.274952,0.037752,0.364543}%
\pgfsetstrokecolor{currentstroke}%
\pgfsetdash{}{0pt}%
\pgfpathmoveto{\pgfqpoint{5.692660in}{2.886926in}}%
\pgfpathlineto{\pgfqpoint{5.642521in}{2.887680in}}%
\pgfusepath{stroke}%
\end{pgfscope}%
\begin{pgfscope}%
\pgfpathrectangle{\pgfqpoint{3.985294in}{1.750000in}}{\pgfqpoint{2.279412in}{2.004545in}}%
\pgfusepath{clip}%
\pgfsetbuttcap%
\pgfsetroundjoin%
\pgfsetlinewidth{0.375296pt}%
\definecolor{currentstroke}{rgb}{0.278791,0.062145,0.386592}%
\pgfsetstrokecolor{currentstroke}%
\pgfsetdash{}{0pt}%
\pgfpathmoveto{\pgfqpoint{5.642521in}{2.887680in}}%
\pgfpathlineto{\pgfqpoint{5.592381in}{2.887911in}}%
\pgfusepath{stroke}%
\end{pgfscope}%
\begin{pgfscope}%
\pgfpathrectangle{\pgfqpoint{3.985294in}{1.750000in}}{\pgfqpoint{2.279412in}{2.004545in}}%
\pgfusepath{clip}%
\pgfsetbuttcap%
\pgfsetroundjoin%
\pgfsetlinewidth{0.420584pt}%
\definecolor{currentstroke}{rgb}{0.282656,0.100196,0.422160}%
\pgfsetstrokecolor{currentstroke}%
\pgfsetdash{}{0pt}%
\pgfpathmoveto{\pgfqpoint{5.592381in}{2.887911in}}%
\pgfpathlineto{\pgfqpoint{5.542235in}{2.887284in}}%
\pgfusepath{stroke}%
\end{pgfscope}%
\begin{pgfscope}%
\pgfpathrectangle{\pgfqpoint{3.985294in}{1.750000in}}{\pgfqpoint{2.279412in}{2.004545in}}%
\pgfusepath{clip}%
\pgfsetbuttcap%
\pgfsetroundjoin%
\pgfsetlinewidth{0.464085pt}%
\definecolor{currentstroke}{rgb}{0.283072,0.130895,0.449241}%
\pgfsetstrokecolor{currentstroke}%
\pgfsetdash{}{0pt}%
\pgfpathmoveto{\pgfqpoint{5.542235in}{2.887284in}}%
\pgfpathlineto{\pgfqpoint{5.492093in}{2.886445in}}%
\pgfusepath{stroke}%
\end{pgfscope}%
\begin{pgfscope}%
\pgfpathrectangle{\pgfqpoint{3.985294in}{1.750000in}}{\pgfqpoint{2.279412in}{2.004545in}}%
\pgfusepath{clip}%
\pgfsetbuttcap%
\pgfsetroundjoin%
\pgfsetlinewidth{0.539487pt}%
\definecolor{currentstroke}{rgb}{0.277134,0.185228,0.489898}%
\pgfsetstrokecolor{currentstroke}%
\pgfsetdash{}{0pt}%
\pgfpathmoveto{\pgfqpoint{5.492093in}{2.886445in}}%
\pgfpathlineto{\pgfqpoint{5.441954in}{2.885438in}}%
\pgfusepath{stroke}%
\end{pgfscope}%
\begin{pgfscope}%
\pgfpathrectangle{\pgfqpoint{3.985294in}{1.750000in}}{\pgfqpoint{2.279412in}{2.004545in}}%
\pgfusepath{clip}%
\pgfsetbuttcap%
\pgfsetroundjoin%
\pgfsetlinewidth{0.622699pt}%
\definecolor{currentstroke}{rgb}{0.262138,0.242286,0.520837}%
\pgfsetstrokecolor{currentstroke}%
\pgfsetdash{}{0pt}%
\pgfpathmoveto{\pgfqpoint{5.441954in}{2.885438in}}%
\pgfpathlineto{\pgfqpoint{5.391823in}{2.884175in}}%
\pgfusepath{stroke}%
\end{pgfscope}%
\begin{pgfscope}%
\pgfpathrectangle{\pgfqpoint{3.985294in}{1.750000in}}{\pgfqpoint{2.279412in}{2.004545in}}%
\pgfusepath{clip}%
\pgfsetbuttcap%
\pgfsetroundjoin%
\pgfsetlinewidth{0.676509pt}%
\definecolor{currentstroke}{rgb}{0.248629,0.278775,0.534556}%
\pgfsetstrokecolor{currentstroke}%
\pgfsetdash{}{0pt}%
\pgfpathmoveto{\pgfqpoint{5.391823in}{2.884175in}}%
\pgfpathlineto{\pgfqpoint{5.341712in}{2.882430in}}%
\pgfusepath{stroke}%
\end{pgfscope}%
\begin{pgfscope}%
\pgfpathrectangle{\pgfqpoint{3.985294in}{1.750000in}}{\pgfqpoint{2.279412in}{2.004545in}}%
\pgfusepath{clip}%
\pgfsetbuttcap%
\pgfsetroundjoin%
\pgfsetlinewidth{0.745548pt}%
\definecolor{currentstroke}{rgb}{0.229739,0.322361,0.545706}%
\pgfsetstrokecolor{currentstroke}%
\pgfsetdash{}{0pt}%
\pgfpathmoveto{\pgfqpoint{5.341712in}{2.882430in}}%
\pgfpathlineto{\pgfqpoint{5.291634in}{2.880080in}}%
\pgfusepath{stroke}%
\end{pgfscope}%
\begin{pgfscope}%
\pgfpathrectangle{\pgfqpoint{3.985294in}{1.750000in}}{\pgfqpoint{2.279412in}{2.004545in}}%
\pgfusepath{clip}%
\pgfsetbuttcap%
\pgfsetroundjoin%
\pgfsetlinewidth{0.809140pt}%
\definecolor{currentstroke}{rgb}{0.210503,0.363727,0.552206}%
\pgfsetstrokecolor{currentstroke}%
\pgfsetdash{}{0pt}%
\pgfpathmoveto{\pgfqpoint{5.291634in}{2.880080in}}%
\pgfpathlineto{\pgfqpoint{5.241603in}{2.877057in}}%
\pgfusepath{stroke}%
\end{pgfscope}%
\begin{pgfscope}%
\pgfpathrectangle{\pgfqpoint{3.985294in}{1.750000in}}{\pgfqpoint{2.279412in}{2.004545in}}%
\pgfusepath{clip}%
\pgfsetbuttcap%
\pgfsetroundjoin%
\pgfsetlinewidth{0.872398pt}%
\definecolor{currentstroke}{rgb}{0.194100,0.399323,0.555565}%
\pgfsetstrokecolor{currentstroke}%
\pgfsetdash{}{0pt}%
\pgfpathmoveto{\pgfqpoint{5.241603in}{2.877057in}}%
\pgfpathlineto{\pgfqpoint{5.191622in}{2.873430in}}%
\pgfusepath{stroke}%
\end{pgfscope}%
\begin{pgfscope}%
\pgfpathrectangle{\pgfqpoint{3.985294in}{1.750000in}}{\pgfqpoint{2.279412in}{2.004545in}}%
\pgfusepath{clip}%
\pgfsetbuttcap%
\pgfsetroundjoin%
\pgfsetlinewidth{0.313308pt}%
\definecolor{currentstroke}{rgb}{0.268510,0.009605,0.335427}%
\pgfsetstrokecolor{currentstroke}%
\pgfsetdash{}{0pt}%
\pgfpathmoveto{\pgfqpoint{5.843087in}{2.977807in}}%
\pgfpathlineto{\pgfqpoint{5.792979in}{2.977111in}}%
\pgfusepath{stroke}%
\end{pgfscope}%
\begin{pgfscope}%
\pgfpathrectangle{\pgfqpoint{3.985294in}{1.750000in}}{\pgfqpoint{2.279412in}{2.004545in}}%
\pgfusepath{clip}%
\pgfsetbuttcap%
\pgfsetroundjoin%
\pgfsetlinewidth{0.320968pt}%
\definecolor{currentstroke}{rgb}{0.269944,0.014625,0.341379}%
\pgfsetstrokecolor{currentstroke}%
\pgfsetdash{}{0pt}%
\pgfpathmoveto{\pgfqpoint{5.792979in}{2.977111in}}%
\pgfpathlineto{\pgfqpoint{5.742844in}{2.976303in}}%
\pgfusepath{stroke}%
\end{pgfscope}%
\begin{pgfscope}%
\pgfpathrectangle{\pgfqpoint{3.985294in}{1.750000in}}{\pgfqpoint{2.279412in}{2.004545in}}%
\pgfusepath{clip}%
\pgfsetbuttcap%
\pgfsetroundjoin%
\pgfsetlinewidth{0.329863pt}%
\definecolor{currentstroke}{rgb}{0.272594,0.025563,0.353093}%
\pgfsetstrokecolor{currentstroke}%
\pgfsetdash{}{0pt}%
\pgfpathmoveto{\pgfqpoint{5.742844in}{2.976303in}}%
\pgfpathlineto{\pgfqpoint{5.692701in}{2.975776in}}%
\pgfusepath{stroke}%
\end{pgfscope}%
\begin{pgfscope}%
\pgfpathrectangle{\pgfqpoint{3.985294in}{1.750000in}}{\pgfqpoint{2.279412in}{2.004545in}}%
\pgfusepath{clip}%
\pgfsetbuttcap%
\pgfsetroundjoin%
\pgfsetlinewidth{0.347330pt}%
\definecolor{currentstroke}{rgb}{0.274952,0.037752,0.364543}%
\pgfsetstrokecolor{currentstroke}%
\pgfsetdash{}{0pt}%
\pgfpathmoveto{\pgfqpoint{5.692701in}{2.975776in}}%
\pgfpathlineto{\pgfqpoint{5.642559in}{2.975047in}}%
\pgfusepath{stroke}%
\end{pgfscope}%
\begin{pgfscope}%
\pgfpathrectangle{\pgfqpoint{3.985294in}{1.750000in}}{\pgfqpoint{2.279412in}{2.004545in}}%
\pgfusepath{clip}%
\pgfsetbuttcap%
\pgfsetroundjoin%
\pgfsetlinewidth{0.368063pt}%
\definecolor{currentstroke}{rgb}{0.277941,0.056324,0.381191}%
\pgfsetstrokecolor{currentstroke}%
\pgfsetdash{}{0pt}%
\pgfpathmoveto{\pgfqpoint{5.642559in}{2.975047in}}%
\pgfpathlineto{\pgfqpoint{5.592421in}{2.974043in}}%
\pgfusepath{stroke}%
\end{pgfscope}%
\begin{pgfscope}%
\pgfpathrectangle{\pgfqpoint{3.985294in}{1.750000in}}{\pgfqpoint{2.279412in}{2.004545in}}%
\pgfusepath{clip}%
\pgfsetbuttcap%
\pgfsetroundjoin%
\pgfsetlinewidth{0.391561pt}%
\definecolor{currentstroke}{rgb}{0.280894,0.078907,0.402329}%
\pgfsetstrokecolor{currentstroke}%
\pgfsetdash{}{0pt}%
\pgfpathmoveto{\pgfqpoint{5.592421in}{2.974043in}}%
\pgfpathlineto{\pgfqpoint{5.542287in}{2.972874in}}%
\pgfusepath{stroke}%
\end{pgfscope}%
\begin{pgfscope}%
\pgfpathrectangle{\pgfqpoint{3.985294in}{1.750000in}}{\pgfqpoint{2.279412in}{2.004545in}}%
\pgfusepath{clip}%
\pgfsetbuttcap%
\pgfsetroundjoin%
\pgfsetlinewidth{0.446461pt}%
\definecolor{currentstroke}{rgb}{0.283229,0.120777,0.440584}%
\pgfsetstrokecolor{currentstroke}%
\pgfsetdash{}{0pt}%
\pgfpathmoveto{\pgfqpoint{5.542287in}{2.972874in}}%
\pgfpathlineto{\pgfqpoint{5.492167in}{2.971323in}}%
\pgfusepath{stroke}%
\end{pgfscope}%
\begin{pgfscope}%
\pgfpathrectangle{\pgfqpoint{3.985294in}{1.750000in}}{\pgfqpoint{2.279412in}{2.004545in}}%
\pgfusepath{clip}%
\pgfsetbuttcap%
\pgfsetroundjoin%
\pgfsetlinewidth{0.493124pt}%
\definecolor{currentstroke}{rgb}{0.281887,0.150881,0.465405}%
\pgfsetstrokecolor{currentstroke}%
\pgfsetdash{}{0pt}%
\pgfpathmoveto{\pgfqpoint{5.492167in}{2.971323in}}%
\pgfpathlineto{\pgfqpoint{5.442065in}{2.969359in}}%
\pgfusepath{stroke}%
\end{pgfscope}%
\begin{pgfscope}%
\pgfpathrectangle{\pgfqpoint{3.985294in}{1.750000in}}{\pgfqpoint{2.279412in}{2.004545in}}%
\pgfusepath{clip}%
\pgfsetbuttcap%
\pgfsetroundjoin%
\pgfsetlinewidth{0.536074pt}%
\definecolor{currentstroke}{rgb}{0.277134,0.185228,0.489898}%
\pgfsetstrokecolor{currentstroke}%
\pgfsetdash{}{0pt}%
\pgfpathmoveto{\pgfqpoint{5.442065in}{2.969359in}}%
\pgfpathlineto{\pgfqpoint{5.392002in}{2.966786in}}%
\pgfusepath{stroke}%
\end{pgfscope}%
\begin{pgfscope}%
\pgfpathrectangle{\pgfqpoint{3.985294in}{1.750000in}}{\pgfqpoint{2.279412in}{2.004545in}}%
\pgfusepath{clip}%
\pgfsetbuttcap%
\pgfsetroundjoin%
\pgfsetlinewidth{0.603283pt}%
\definecolor{currentstroke}{rgb}{0.265145,0.232956,0.516599}%
\pgfsetstrokecolor{currentstroke}%
\pgfsetdash{}{0pt}%
\pgfpathmoveto{\pgfqpoint{5.392002in}{2.966786in}}%
\pgfpathlineto{\pgfqpoint{5.341992in}{2.963502in}}%
\pgfusepath{stroke}%
\end{pgfscope}%
\begin{pgfscope}%
\pgfpathrectangle{\pgfqpoint{3.985294in}{1.750000in}}{\pgfqpoint{2.279412in}{2.004545in}}%
\pgfusepath{clip}%
\pgfsetbuttcap%
\pgfsetroundjoin%
\pgfsetlinewidth{0.630663pt}%
\definecolor{currentstroke}{rgb}{0.258965,0.251537,0.524736}%
\pgfsetstrokecolor{currentstroke}%
\pgfsetdash{}{0pt}%
\pgfpathmoveto{\pgfqpoint{5.341992in}{2.963502in}}%
\pgfpathlineto{\pgfqpoint{5.292048in}{2.959532in}}%
\pgfusepath{stroke}%
\end{pgfscope}%
\begin{pgfscope}%
\pgfpathrectangle{\pgfqpoint{3.985294in}{1.750000in}}{\pgfqpoint{2.279412in}{2.004545in}}%
\pgfusepath{clip}%
\pgfsetbuttcap%
\pgfsetroundjoin%
\pgfsetlinewidth{0.320971pt}%
\definecolor{currentstroke}{rgb}{0.269944,0.014625,0.341379}%
\pgfsetstrokecolor{currentstroke}%
\pgfsetdash{}{0pt}%
\pgfpathmoveto{\pgfqpoint{5.843087in}{3.022913in}}%
\pgfpathlineto{\pgfqpoint{5.793026in}{3.021658in}}%
\pgfusepath{stroke}%
\end{pgfscope}%
\begin{pgfscope}%
\pgfpathrectangle{\pgfqpoint{3.985294in}{1.750000in}}{\pgfqpoint{2.279412in}{2.004545in}}%
\pgfusepath{clip}%
\pgfsetbuttcap%
\pgfsetroundjoin%
\pgfsetlinewidth{0.318784pt}%
\definecolor{currentstroke}{rgb}{0.269944,0.014625,0.341379}%
\pgfsetstrokecolor{currentstroke}%
\pgfsetdash{}{0pt}%
\pgfpathmoveto{\pgfqpoint{5.793026in}{3.021658in}}%
\pgfpathlineto{\pgfqpoint{5.742923in}{3.020707in}}%
\pgfusepath{stroke}%
\end{pgfscope}%
\begin{pgfscope}%
\pgfpathrectangle{\pgfqpoint{3.985294in}{1.750000in}}{\pgfqpoint{2.279412in}{2.004545in}}%
\pgfusepath{clip}%
\pgfsetbuttcap%
\pgfsetroundjoin%
\pgfsetlinewidth{0.331222pt}%
\definecolor{currentstroke}{rgb}{0.272594,0.025563,0.353093}%
\pgfsetstrokecolor{currentstroke}%
\pgfsetdash{}{0pt}%
\pgfpathmoveto{\pgfqpoint{5.742923in}{3.020707in}}%
\pgfpathlineto{\pgfqpoint{5.692793in}{3.019718in}}%
\pgfusepath{stroke}%
\end{pgfscope}%
\begin{pgfscope}%
\pgfpathrectangle{\pgfqpoint{3.985294in}{1.750000in}}{\pgfqpoint{2.279412in}{2.004545in}}%
\pgfusepath{clip}%
\pgfsetbuttcap%
\pgfsetroundjoin%
\pgfsetlinewidth{0.344419pt}%
\definecolor{currentstroke}{rgb}{0.274952,0.037752,0.364543}%
\pgfsetstrokecolor{currentstroke}%
\pgfsetdash{}{0pt}%
\pgfpathmoveto{\pgfqpoint{5.692793in}{3.019718in}}%
\pgfpathlineto{\pgfqpoint{5.642671in}{3.018237in}}%
\pgfusepath{stroke}%
\end{pgfscope}%
\begin{pgfscope}%
\pgfpathrectangle{\pgfqpoint{3.985294in}{1.750000in}}{\pgfqpoint{2.279412in}{2.004545in}}%
\pgfusepath{clip}%
\pgfsetbuttcap%
\pgfsetroundjoin%
\pgfsetlinewidth{0.369752pt}%
\definecolor{currentstroke}{rgb}{0.278791,0.062145,0.386592}%
\pgfsetstrokecolor{currentstroke}%
\pgfsetdash{}{0pt}%
\pgfpathmoveto{\pgfqpoint{5.642671in}{3.018237in}}%
\pgfpathlineto{\pgfqpoint{5.592552in}{3.016668in}}%
\pgfusepath{stroke}%
\end{pgfscope}%
\begin{pgfscope}%
\pgfpathrectangle{\pgfqpoint{3.985294in}{1.750000in}}{\pgfqpoint{2.279412in}{2.004545in}}%
\pgfusepath{clip}%
\pgfsetbuttcap%
\pgfsetroundjoin%
\pgfsetlinewidth{0.387876pt}%
\definecolor{currentstroke}{rgb}{0.280267,0.073417,0.397163}%
\pgfsetstrokecolor{currentstroke}%
\pgfsetdash{}{0pt}%
\pgfpathmoveto{\pgfqpoint{5.592552in}{3.016668in}}%
\pgfpathlineto{\pgfqpoint{5.542438in}{3.014999in}}%
\pgfusepath{stroke}%
\end{pgfscope}%
\begin{pgfscope}%
\pgfpathrectangle{\pgfqpoint{3.985294in}{1.750000in}}{\pgfqpoint{2.279412in}{2.004545in}}%
\pgfusepath{clip}%
\pgfsetbuttcap%
\pgfsetroundjoin%
\pgfsetlinewidth{0.428392pt}%
\definecolor{currentstroke}{rgb}{0.282910,0.105393,0.426902}%
\pgfsetstrokecolor{currentstroke}%
\pgfsetdash{}{0pt}%
\pgfpathmoveto{\pgfqpoint{5.542438in}{3.014999in}}%
\pgfpathlineto{\pgfqpoint{5.492351in}{3.012800in}}%
\pgfusepath{stroke}%
\end{pgfscope}%
\begin{pgfscope}%
\pgfpathrectangle{\pgfqpoint{3.985294in}{1.750000in}}{\pgfqpoint{2.279412in}{2.004545in}}%
\pgfusepath{clip}%
\pgfsetbuttcap%
\pgfsetroundjoin%
\pgfsetlinewidth{0.457362pt}%
\definecolor{currentstroke}{rgb}{0.283187,0.125848,0.444960}%
\pgfsetstrokecolor{currentstroke}%
\pgfsetdash{}{0pt}%
\pgfpathmoveto{\pgfqpoint{5.492351in}{3.012800in}}%
\pgfpathlineto{\pgfqpoint{5.442297in}{3.010048in}}%
\pgfusepath{stroke}%
\end{pgfscope}%
\begin{pgfscope}%
\pgfpathrectangle{\pgfqpoint{3.985294in}{1.750000in}}{\pgfqpoint{2.279412in}{2.004545in}}%
\pgfusepath{clip}%
\pgfsetbuttcap%
\pgfsetroundjoin%
\pgfsetlinewidth{0.495927pt}%
\definecolor{currentstroke}{rgb}{0.281412,0.155834,0.469201}%
\pgfsetstrokecolor{currentstroke}%
\pgfsetdash{}{0pt}%
\pgfpathmoveto{\pgfqpoint{5.442297in}{3.010048in}}%
\pgfpathlineto{\pgfqpoint{5.392271in}{3.006940in}}%
\pgfusepath{stroke}%
\end{pgfscope}%
\begin{pgfscope}%
\pgfpathrectangle{\pgfqpoint{3.985294in}{1.750000in}}{\pgfqpoint{2.279412in}{2.004545in}}%
\pgfusepath{clip}%
\pgfsetbuttcap%
\pgfsetroundjoin%
\pgfsetlinewidth{0.557416pt}%
\definecolor{currentstroke}{rgb}{0.274128,0.199721,0.498911}%
\pgfsetstrokecolor{currentstroke}%
\pgfsetdash{}{0pt}%
\pgfpathmoveto{\pgfqpoint{5.392271in}{3.006940in}}%
\pgfpathlineto{\pgfqpoint{5.342311in}{3.003139in}}%
\pgfusepath{stroke}%
\end{pgfscope}%
\begin{pgfscope}%
\pgfpathrectangle{\pgfqpoint{3.985294in}{1.750000in}}{\pgfqpoint{2.279412in}{2.004545in}}%
\pgfusepath{clip}%
\pgfsetbuttcap%
\pgfsetroundjoin%
\pgfsetlinewidth{0.308259pt}%
\definecolor{currentstroke}{rgb}{0.268510,0.009605,0.335427}%
\pgfsetstrokecolor{currentstroke}%
\pgfsetdash{}{0pt}%
\pgfpathmoveto{\pgfqpoint{5.843087in}{3.203341in}}%
\pgfpathlineto{\pgfqpoint{5.843087in}{3.203341in}}%
\pgfusepath{stroke}%
\end{pgfscope}%
\begin{pgfscope}%
\pgfpathrectangle{\pgfqpoint{3.985294in}{1.750000in}}{\pgfqpoint{2.279412in}{2.004545in}}%
\pgfusepath{clip}%
\pgfsetbuttcap%
\pgfsetroundjoin%
\pgfsetlinewidth{0.308259pt}%
\definecolor{currentstroke}{rgb}{0.268510,0.009605,0.335427}%
\pgfsetstrokecolor{currentstroke}%
\pgfsetdash{}{0pt}%
\pgfpathmoveto{\pgfqpoint{5.843087in}{3.203341in}}%
\pgfpathlineto{\pgfqpoint{5.830324in}{3.199484in}}%
\pgfusepath{stroke}%
\end{pgfscope}%
\begin{pgfscope}%
\pgfpathrectangle{\pgfqpoint{3.985294in}{1.750000in}}{\pgfqpoint{2.279412in}{2.004545in}}%
\pgfusepath{clip}%
\pgfsetbuttcap%
\pgfsetroundjoin%
\pgfsetlinewidth{0.309473pt}%
\definecolor{currentstroke}{rgb}{0.268510,0.009605,0.335427}%
\pgfsetstrokecolor{currentstroke}%
\pgfsetdash{}{0pt}%
\pgfpathmoveto{\pgfqpoint{5.830324in}{3.199484in}}%
\pgfpathlineto{\pgfqpoint{5.830324in}{3.199484in}}%
\pgfusepath{stroke}%
\end{pgfscope}%
\begin{pgfscope}%
\pgfpathrectangle{\pgfqpoint{3.985294in}{1.750000in}}{\pgfqpoint{2.279412in}{2.004545in}}%
\pgfusepath{clip}%
\pgfsetbuttcap%
\pgfsetroundjoin%
\pgfsetlinewidth{0.309473pt}%
\definecolor{currentstroke}{rgb}{0.268510,0.009605,0.335427}%
\pgfsetstrokecolor{currentstroke}%
\pgfsetdash{}{0pt}%
\pgfpathmoveto{\pgfqpoint{5.830324in}{3.199484in}}%
\pgfpathlineto{\pgfqpoint{5.811095in}{3.198203in}}%
\pgfusepath{stroke}%
\end{pgfscope}%
\begin{pgfscope}%
\pgfpathrectangle{\pgfqpoint{3.985294in}{1.750000in}}{\pgfqpoint{2.279412in}{2.004545in}}%
\pgfusepath{clip}%
\pgfsetbuttcap%
\pgfsetroundjoin%
\pgfsetlinewidth{0.310875pt}%
\definecolor{currentstroke}{rgb}{0.268510,0.009605,0.335427}%
\pgfsetstrokecolor{currentstroke}%
\pgfsetdash{}{0pt}%
\pgfpathmoveto{\pgfqpoint{5.811095in}{3.198203in}}%
\pgfpathlineto{\pgfqpoint{5.773060in}{3.198553in}}%
\pgfusepath{stroke}%
\end{pgfscope}%
\begin{pgfscope}%
\pgfpathrectangle{\pgfqpoint{3.985294in}{1.750000in}}{\pgfqpoint{2.279412in}{2.004545in}}%
\pgfusepath{clip}%
\pgfsetbuttcap%
\pgfsetroundjoin%
\pgfsetlinewidth{0.315560pt}%
\definecolor{currentstroke}{rgb}{0.269944,0.014625,0.341379}%
\pgfsetstrokecolor{currentstroke}%
\pgfsetdash{}{0pt}%
\pgfpathmoveto{\pgfqpoint{5.773060in}{3.198553in}}%
\pgfpathlineto{\pgfqpoint{5.731039in}{3.198329in}}%
\pgfusepath{stroke}%
\end{pgfscope}%
\begin{pgfscope}%
\pgfpathrectangle{\pgfqpoint{3.985294in}{1.750000in}}{\pgfqpoint{2.279412in}{2.004545in}}%
\pgfusepath{clip}%
\pgfsetbuttcap%
\pgfsetroundjoin%
\pgfsetlinewidth{0.318445pt}%
\definecolor{currentstroke}{rgb}{0.269944,0.014625,0.341379}%
\pgfsetstrokecolor{currentstroke}%
\pgfsetdash{}{0pt}%
\pgfpathmoveto{\pgfqpoint{5.731039in}{3.198329in}}%
\pgfpathlineto{\pgfqpoint{5.680931in}{3.196977in}}%
\pgfusepath{stroke}%
\end{pgfscope}%
\begin{pgfscope}%
\pgfpathrectangle{\pgfqpoint{3.985294in}{1.750000in}}{\pgfqpoint{2.279412in}{2.004545in}}%
\pgfusepath{clip}%
\pgfsetbuttcap%
\pgfsetroundjoin%
\pgfsetlinewidth{0.326132pt}%
\definecolor{currentstroke}{rgb}{0.271305,0.019942,0.347269}%
\pgfsetstrokecolor{currentstroke}%
\pgfsetdash{}{0pt}%
\pgfpathmoveto{\pgfqpoint{5.680931in}{3.196977in}}%
\pgfpathlineto{\pgfqpoint{5.630828in}{3.195400in}}%
\pgfusepath{stroke}%
\end{pgfscope}%
\begin{pgfscope}%
\pgfpathrectangle{\pgfqpoint{3.985294in}{1.750000in}}{\pgfqpoint{2.279412in}{2.004545in}}%
\pgfusepath{clip}%
\pgfsetbuttcap%
\pgfsetroundjoin%
\pgfsetlinewidth{0.345276pt}%
\definecolor{currentstroke}{rgb}{0.274952,0.037752,0.364543}%
\pgfsetstrokecolor{currentstroke}%
\pgfsetdash{}{0pt}%
\pgfpathmoveto{\pgfqpoint{5.630828in}{3.195400in}}%
\pgfpathlineto{\pgfqpoint{5.580748in}{3.193216in}}%
\pgfusepath{stroke}%
\end{pgfscope}%
\begin{pgfscope}%
\pgfpathrectangle{\pgfqpoint{3.985294in}{1.750000in}}{\pgfqpoint{2.279412in}{2.004545in}}%
\pgfusepath{clip}%
\pgfsetbuttcap%
\pgfsetroundjoin%
\pgfsetlinewidth{0.361099pt}%
\definecolor{currentstroke}{rgb}{0.277018,0.050344,0.375715}%
\pgfsetstrokecolor{currentstroke}%
\pgfsetdash{}{0pt}%
\pgfpathmoveto{\pgfqpoint{5.580748in}{3.193216in}}%
\pgfpathlineto{\pgfqpoint{5.530685in}{3.190671in}}%
\pgfusepath{stroke}%
\end{pgfscope}%
\begin{pgfscope}%
\pgfpathrectangle{\pgfqpoint{3.985294in}{1.750000in}}{\pgfqpoint{2.279412in}{2.004545in}}%
\pgfusepath{clip}%
\pgfsetbuttcap%
\pgfsetroundjoin%
\pgfsetlinewidth{0.357739pt}%
\definecolor{currentstroke}{rgb}{0.277018,0.050344,0.375715}%
\pgfsetstrokecolor{currentstroke}%
\pgfsetdash{}{0pt}%
\pgfpathmoveto{\pgfqpoint{5.530685in}{3.190671in}}%
\pgfpathlineto{\pgfqpoint{5.480617in}{3.188137in}}%
\pgfusepath{stroke}%
\end{pgfscope}%
\begin{pgfscope}%
\pgfpathrectangle{\pgfqpoint{3.985294in}{1.750000in}}{\pgfqpoint{2.279412in}{2.004545in}}%
\pgfusepath{clip}%
\pgfsetbuttcap%
\pgfsetroundjoin%
\pgfsetlinewidth{0.377746pt}%
\definecolor{currentstroke}{rgb}{0.279566,0.067836,0.391917}%
\pgfsetstrokecolor{currentstroke}%
\pgfsetdash{}{0pt}%
\pgfpathmoveto{\pgfqpoint{5.480617in}{3.188137in}}%
\pgfpathlineto{\pgfqpoint{5.430621in}{3.184830in}}%
\pgfusepath{stroke}%
\end{pgfscope}%
\begin{pgfscope}%
\pgfpathrectangle{\pgfqpoint{3.985294in}{1.750000in}}{\pgfqpoint{2.279412in}{2.004545in}}%
\pgfusepath{clip}%
\pgfsetbuttcap%
\pgfsetroundjoin%
\pgfsetlinewidth{0.406160pt}%
\definecolor{currentstroke}{rgb}{0.281924,0.089666,0.412415}%
\pgfsetstrokecolor{currentstroke}%
\pgfsetdash{}{0pt}%
\pgfpathmoveto{\pgfqpoint{5.430621in}{3.184830in}}%
\pgfpathlineto{\pgfqpoint{5.380779in}{3.180010in}}%
\pgfusepath{stroke}%
\end{pgfscope}%
\begin{pgfscope}%
\pgfpathrectangle{\pgfqpoint{3.985294in}{1.750000in}}{\pgfqpoint{2.279412in}{2.004545in}}%
\pgfusepath{clip}%
\pgfsetbuttcap%
\pgfsetroundjoin%
\pgfsetlinewidth{0.396644pt}%
\definecolor{currentstroke}{rgb}{0.280894,0.078907,0.402329}%
\pgfsetstrokecolor{currentstroke}%
\pgfsetdash{}{0pt}%
\pgfpathmoveto{\pgfqpoint{5.380779in}{3.180010in}}%
\pgfpathlineto{\pgfqpoint{5.331160in}{3.173700in}}%
\pgfusepath{stroke}%
\end{pgfscope}%
\begin{pgfscope}%
\pgfpathrectangle{\pgfqpoint{3.985294in}{1.750000in}}{\pgfqpoint{2.279412in}{2.004545in}}%
\pgfusepath{clip}%
\pgfsetbuttcap%
\pgfsetroundjoin%
\pgfsetlinewidth{0.387251pt}%
\definecolor{currentstroke}{rgb}{0.280267,0.073417,0.397163}%
\pgfsetstrokecolor{currentstroke}%
\pgfsetdash{}{0pt}%
\pgfpathmoveto{\pgfqpoint{5.331160in}{3.173700in}}%
\pgfpathlineto{\pgfqpoint{5.281796in}{3.165978in}}%
\pgfusepath{stroke}%
\end{pgfscope}%
\begin{pgfscope}%
\pgfpathrectangle{\pgfqpoint{3.985294in}{1.750000in}}{\pgfqpoint{2.279412in}{2.004545in}}%
\pgfusepath{clip}%
\pgfsetbuttcap%
\pgfsetroundjoin%
\pgfsetlinewidth{0.416125pt}%
\definecolor{currentstroke}{rgb}{0.282327,0.094955,0.417331}%
\pgfsetstrokecolor{currentstroke}%
\pgfsetdash{}{0pt}%
\pgfpathmoveto{\pgfqpoint{5.281796in}{3.165978in}}%
\pgfpathlineto{\pgfqpoint{5.232766in}{3.156861in}}%
\pgfusepath{stroke}%
\end{pgfscope}%
\begin{pgfscope}%
\pgfpathrectangle{\pgfqpoint{3.985294in}{1.750000in}}{\pgfqpoint{2.279412in}{2.004545in}}%
\pgfusepath{clip}%
\pgfsetbuttcap%
\pgfsetroundjoin%
\pgfsetlinewidth{0.394633pt}%
\definecolor{currentstroke}{rgb}{0.280894,0.078907,0.402329}%
\pgfsetstrokecolor{currentstroke}%
\pgfsetdash{}{0pt}%
\pgfpathmoveto{\pgfqpoint{5.232766in}{3.156861in}}%
\pgfpathlineto{\pgfqpoint{5.184457in}{3.145253in}}%
\pgfusepath{stroke}%
\end{pgfscope}%
\begin{pgfscope}%
\pgfpathrectangle{\pgfqpoint{3.985294in}{1.750000in}}{\pgfqpoint{2.279412in}{2.004545in}}%
\pgfusepath{clip}%
\pgfsetbuttcap%
\pgfsetroundjoin%
\pgfsetlinewidth{0.399984pt}%
\definecolor{currentstroke}{rgb}{0.281446,0.084320,0.407414}%
\pgfsetstrokecolor{currentstroke}%
\pgfsetdash{}{0pt}%
\pgfpathmoveto{\pgfqpoint{5.184457in}{3.145253in}}%
\pgfpathlineto{\pgfqpoint{5.137574in}{3.129812in}}%
\pgfusepath{stroke}%
\end{pgfscope}%
\begin{pgfscope}%
\pgfpathrectangle{\pgfqpoint{3.985294in}{1.750000in}}{\pgfqpoint{2.279412in}{2.004545in}}%
\pgfusepath{clip}%
\pgfsetbuttcap%
\pgfsetroundjoin%
\pgfsetlinewidth{0.459515pt}%
\definecolor{currentstroke}{rgb}{0.283072,0.130895,0.449241}%
\pgfsetstrokecolor{currentstroke}%
\pgfsetdash{}{0pt}%
\pgfpathmoveto{\pgfqpoint{5.137574in}{3.129812in}}%
\pgfpathlineto{\pgfqpoint{5.092923in}{3.110099in}}%
\pgfusepath{stroke}%
\end{pgfscope}%
\begin{pgfscope}%
\pgfpathrectangle{\pgfqpoint{3.985294in}{1.750000in}}{\pgfqpoint{2.279412in}{2.004545in}}%
\pgfusepath{clip}%
\pgfsetbuttcap%
\pgfsetroundjoin%
\pgfsetlinewidth{0.458497pt}%
\definecolor{currentstroke}{rgb}{0.283187,0.125848,0.444960}%
\pgfsetstrokecolor{currentstroke}%
\pgfsetdash{}{0pt}%
\pgfpathmoveto{\pgfqpoint{5.092923in}{3.110099in}}%
\pgfpathlineto{\pgfqpoint{5.051892in}{3.085358in}}%
\pgfusepath{stroke}%
\end{pgfscope}%
\begin{pgfscope}%
\pgfpathrectangle{\pgfqpoint{3.985294in}{1.750000in}}{\pgfqpoint{2.279412in}{2.004545in}}%
\pgfusepath{clip}%
\pgfsetbuttcap%
\pgfsetroundjoin%
\pgfsetlinewidth{0.455475pt}%
\definecolor{currentstroke}{rgb}{0.283187,0.125848,0.444960}%
\pgfsetstrokecolor{currentstroke}%
\pgfsetdash{}{0pt}%
\pgfpathmoveto{\pgfqpoint{5.051892in}{3.085358in}}%
\pgfpathlineto{\pgfqpoint{5.014172in}{3.057967in}}%
\pgfusepath{stroke}%
\end{pgfscope}%
\begin{pgfscope}%
\pgfpathrectangle{\pgfqpoint{3.985294in}{1.750000in}}{\pgfqpoint{2.279412in}{2.004545in}}%
\pgfusepath{clip}%
\pgfsetbuttcap%
\pgfsetroundjoin%
\pgfsetlinewidth{0.533510pt}%
\definecolor{currentstroke}{rgb}{0.278012,0.180367,0.486697}%
\pgfsetstrokecolor{currentstroke}%
\pgfsetdash{}{0pt}%
\pgfpathmoveto{\pgfqpoint{5.014172in}{3.057967in}}%
\pgfpathlineto{\pgfqpoint{4.978555in}{3.027336in}}%
\pgfusepath{stroke}%
\end{pgfscope}%
\begin{pgfscope}%
\pgfpathrectangle{\pgfqpoint{3.985294in}{1.750000in}}{\pgfqpoint{2.279412in}{2.004545in}}%
\pgfusepath{clip}%
\pgfsetbuttcap%
\pgfsetroundjoin%
\pgfsetlinewidth{0.550351pt}%
\definecolor{currentstroke}{rgb}{0.275191,0.194905,0.496005}%
\pgfsetstrokecolor{currentstroke}%
\pgfsetdash{}{0pt}%
\pgfpathmoveto{\pgfqpoint{4.978555in}{3.027336in}}%
\pgfpathlineto{\pgfqpoint{4.945961in}{2.994082in}}%
\pgfusepath{stroke}%
\end{pgfscope}%
\begin{pgfscope}%
\pgfpathrectangle{\pgfqpoint{3.985294in}{1.750000in}}{\pgfqpoint{2.279412in}{2.004545in}}%
\pgfusepath{clip}%
\pgfsetbuttcap%
\pgfsetroundjoin%
\pgfsetlinewidth{0.575940pt}%
\definecolor{currentstroke}{rgb}{0.270595,0.214069,0.507052}%
\pgfsetstrokecolor{currentstroke}%
\pgfsetdash{}{0pt}%
\pgfpathmoveto{\pgfqpoint{4.945961in}{2.994082in}}%
\pgfpathlineto{\pgfqpoint{4.912919in}{2.961216in}}%
\pgfusepath{stroke}%
\end{pgfscope}%
\begin{pgfscope}%
\pgfpathrectangle{\pgfqpoint{3.985294in}{1.750000in}}{\pgfqpoint{2.279412in}{2.004545in}}%
\pgfusepath{clip}%
\pgfsetbuttcap%
\pgfsetroundjoin%
\pgfsetlinewidth{0.639095pt}%
\definecolor{currentstroke}{rgb}{0.257322,0.256130,0.526563}%
\pgfsetstrokecolor{currentstroke}%
\pgfsetdash{}{0pt}%
\pgfpathmoveto{\pgfqpoint{4.912919in}{2.961216in}}%
\pgfpathlineto{\pgfqpoint{4.878968in}{2.929124in}}%
\pgfusepath{stroke}%
\end{pgfscope}%
\begin{pgfscope}%
\pgfpathrectangle{\pgfqpoint{3.985294in}{1.750000in}}{\pgfqpoint{2.279412in}{2.004545in}}%
\pgfusepath{clip}%
\pgfsetbuttcap%
\pgfsetroundjoin%
\pgfsetlinewidth{0.324853pt}%
\definecolor{currentstroke}{rgb}{0.271305,0.019942,0.347269}%
\pgfsetstrokecolor{currentstroke}%
\pgfsetdash{}{0pt}%
\pgfpathmoveto{\pgfqpoint{5.843087in}{3.293554in}}%
\pgfpathlineto{\pgfqpoint{5.793295in}{3.293758in}}%
\pgfusepath{stroke}%
\end{pgfscope}%
\begin{pgfscope}%
\pgfpathrectangle{\pgfqpoint{3.985294in}{1.750000in}}{\pgfqpoint{2.279412in}{2.004545in}}%
\pgfusepath{clip}%
\pgfsetbuttcap%
\pgfsetroundjoin%
\pgfsetlinewidth{0.319188pt}%
\definecolor{currentstroke}{rgb}{0.269944,0.014625,0.341379}%
\pgfsetstrokecolor{currentstroke}%
\pgfsetdash{}{0pt}%
\pgfpathmoveto{\pgfqpoint{5.793295in}{3.293758in}}%
\pgfpathlineto{\pgfqpoint{5.743779in}{3.294449in}}%
\pgfusepath{stroke}%
\end{pgfscope}%
\begin{pgfscope}%
\pgfpathrectangle{\pgfqpoint{3.985294in}{1.750000in}}{\pgfqpoint{2.279412in}{2.004545in}}%
\pgfusepath{clip}%
\pgfsetbuttcap%
\pgfsetroundjoin%
\pgfsetlinewidth{0.318684pt}%
\definecolor{currentstroke}{rgb}{0.269944,0.014625,0.341379}%
\pgfsetstrokecolor{currentstroke}%
\pgfsetdash{}{0pt}%
\pgfpathmoveto{\pgfqpoint{5.743779in}{3.294449in}}%
\pgfpathlineto{\pgfqpoint{5.693982in}{3.294061in}}%
\pgfusepath{stroke}%
\end{pgfscope}%
\begin{pgfscope}%
\pgfpathrectangle{\pgfqpoint{3.985294in}{1.750000in}}{\pgfqpoint{2.279412in}{2.004545in}}%
\pgfusepath{clip}%
\pgfsetbuttcap%
\pgfsetroundjoin%
\pgfsetlinewidth{0.325898pt}%
\definecolor{currentstroke}{rgb}{0.271305,0.019942,0.347269}%
\pgfsetstrokecolor{currentstroke}%
\pgfsetdash{}{0pt}%
\pgfpathmoveto{\pgfqpoint{5.693982in}{3.294061in}}%
\pgfpathlineto{\pgfqpoint{5.643866in}{3.292589in}}%
\pgfusepath{stroke}%
\end{pgfscope}%
\begin{pgfscope}%
\pgfpathrectangle{\pgfqpoint{3.985294in}{1.750000in}}{\pgfqpoint{2.279412in}{2.004545in}}%
\pgfusepath{clip}%
\pgfsetbuttcap%
\pgfsetroundjoin%
\pgfsetlinewidth{0.334003pt}%
\definecolor{currentstroke}{rgb}{0.272594,0.025563,0.353093}%
\pgfsetstrokecolor{currentstroke}%
\pgfsetdash{}{0pt}%
\pgfpathmoveto{\pgfqpoint{5.643866in}{3.292589in}}%
\pgfpathlineto{\pgfqpoint{5.593736in}{3.291486in}}%
\pgfusepath{stroke}%
\end{pgfscope}%
\begin{pgfscope}%
\pgfpathrectangle{\pgfqpoint{3.985294in}{1.750000in}}{\pgfqpoint{2.279412in}{2.004545in}}%
\pgfusepath{clip}%
\pgfsetbuttcap%
\pgfsetroundjoin%
\pgfsetlinewidth{0.344863pt}%
\definecolor{currentstroke}{rgb}{0.274952,0.037752,0.364543}%
\pgfsetstrokecolor{currentstroke}%
\pgfsetdash{}{0pt}%
\pgfpathmoveto{\pgfqpoint{5.593736in}{3.291486in}}%
\pgfpathlineto{\pgfqpoint{5.543642in}{3.289555in}}%
\pgfusepath{stroke}%
\end{pgfscope}%
\begin{pgfscope}%
\pgfpathrectangle{\pgfqpoint{3.985294in}{1.750000in}}{\pgfqpoint{2.279412in}{2.004545in}}%
\pgfusepath{clip}%
\pgfsetbuttcap%
\pgfsetroundjoin%
\pgfsetlinewidth{0.335178pt}%
\definecolor{currentstroke}{rgb}{0.272594,0.025563,0.353093}%
\pgfsetstrokecolor{currentstroke}%
\pgfsetdash{}{0pt}%
\pgfpathmoveto{\pgfqpoint{5.432751in}{3.383768in}}%
\pgfpathlineto{\pgfqpoint{5.383582in}{3.375413in}}%
\pgfusepath{stroke}%
\end{pgfscope}%
\begin{pgfscope}%
\pgfpathrectangle{\pgfqpoint{3.985294in}{1.750000in}}{\pgfqpoint{2.279412in}{2.004545in}}%
\pgfusepath{clip}%
\pgfsetbuttcap%
\pgfsetroundjoin%
\pgfsetlinewidth{0.332966pt}%
\definecolor{currentstroke}{rgb}{0.272594,0.025563,0.353093}%
\pgfsetstrokecolor{currentstroke}%
\pgfsetdash{}{0pt}%
\pgfpathmoveto{\pgfqpoint{5.383582in}{3.375413in}}%
\pgfpathlineto{\pgfqpoint{5.334787in}{3.365383in}}%
\pgfusepath{stroke}%
\end{pgfscope}%
\begin{pgfscope}%
\pgfpathrectangle{\pgfqpoint{3.985294in}{1.750000in}}{\pgfqpoint{2.279412in}{2.004545in}}%
\pgfusepath{clip}%
\pgfsetbuttcap%
\pgfsetroundjoin%
\pgfsetlinewidth{0.342813pt}%
\definecolor{currentstroke}{rgb}{0.274952,0.037752,0.364543}%
\pgfsetstrokecolor{currentstroke}%
\pgfsetdash{}{0pt}%
\pgfpathmoveto{\pgfqpoint{5.334787in}{3.365383in}}%
\pgfpathlineto{\pgfqpoint{5.286092in}{3.354954in}}%
\pgfusepath{stroke}%
\end{pgfscope}%
\begin{pgfscope}%
\pgfpathrectangle{\pgfqpoint{3.985294in}{1.750000in}}{\pgfqpoint{2.279412in}{2.004545in}}%
\pgfusepath{clip}%
\pgfsetbuttcap%
\pgfsetroundjoin%
\pgfsetlinewidth{0.338855pt}%
\definecolor{currentstroke}{rgb}{0.273809,0.031497,0.358853}%
\pgfsetstrokecolor{currentstroke}%
\pgfsetdash{}{0pt}%
\pgfpathmoveto{\pgfqpoint{5.286092in}{3.354954in}}%
\pgfpathlineto{\pgfqpoint{5.237418in}{3.344479in}}%
\pgfusepath{stroke}%
\end{pgfscope}%
\begin{pgfscope}%
\pgfpathrectangle{\pgfqpoint{3.985294in}{1.750000in}}{\pgfqpoint{2.279412in}{2.004545in}}%
\pgfusepath{clip}%
\pgfsetbuttcap%
\pgfsetroundjoin%
\pgfsetlinewidth{0.337151pt}%
\definecolor{currentstroke}{rgb}{0.273809,0.031497,0.358853}%
\pgfsetstrokecolor{currentstroke}%
\pgfsetdash{}{0pt}%
\pgfpathmoveto{\pgfqpoint{5.237418in}{3.344479in}}%
\pgfpathlineto{\pgfqpoint{5.237418in}{3.344479in}}%
\pgfusepath{stroke}%
\end{pgfscope}%
\begin{pgfscope}%
\pgfpathrectangle{\pgfqpoint{3.985294in}{1.750000in}}{\pgfqpoint{2.279412in}{2.004545in}}%
\pgfusepath{clip}%
\pgfsetbuttcap%
\pgfsetroundjoin%
\pgfsetlinewidth{0.337151pt}%
\definecolor{currentstroke}{rgb}{0.273809,0.031497,0.358853}%
\pgfsetstrokecolor{currentstroke}%
\pgfsetdash{}{0pt}%
\pgfpathmoveto{\pgfqpoint{5.237418in}{3.344479in}}%
\pgfpathlineto{\pgfqpoint{5.198586in}{3.331051in}}%
\pgfusepath{stroke}%
\end{pgfscope}%
\begin{pgfscope}%
\pgfpathrectangle{\pgfqpoint{3.985294in}{1.750000in}}{\pgfqpoint{2.279412in}{2.004545in}}%
\pgfusepath{clip}%
\pgfsetbuttcap%
\pgfsetroundjoin%
\pgfsetlinewidth{0.339159pt}%
\definecolor{currentstroke}{rgb}{0.273809,0.031497,0.358853}%
\pgfsetstrokecolor{currentstroke}%
\pgfsetdash{}{0pt}%
\pgfpathmoveto{\pgfqpoint{5.198586in}{3.331051in}}%
\pgfpathlineto{\pgfqpoint{5.162225in}{3.318050in}}%
\pgfusepath{stroke}%
\end{pgfscope}%
\begin{pgfscope}%
\pgfpathrectangle{\pgfqpoint{3.985294in}{1.750000in}}{\pgfqpoint{2.279412in}{2.004545in}}%
\pgfusepath{clip}%
\pgfsetbuttcap%
\pgfsetroundjoin%
\pgfsetlinewidth{0.353578pt}%
\definecolor{currentstroke}{rgb}{0.276022,0.044167,0.370164}%
\pgfsetstrokecolor{currentstroke}%
\pgfsetdash{}{0pt}%
\pgfpathmoveto{\pgfqpoint{5.162225in}{3.318050in}}%
\pgfpathlineto{\pgfqpoint{5.162225in}{3.318050in}}%
\pgfusepath{stroke}%
\end{pgfscope}%
\begin{pgfscope}%
\pgfpathrectangle{\pgfqpoint{3.985294in}{1.750000in}}{\pgfqpoint{2.279412in}{2.004545in}}%
\pgfusepath{clip}%
\pgfsetbuttcap%
\pgfsetroundjoin%
\pgfsetlinewidth{0.353578pt}%
\definecolor{currentstroke}{rgb}{0.276022,0.044167,0.370164}%
\pgfsetstrokecolor{currentstroke}%
\pgfsetdash{}{0pt}%
\pgfpathmoveto{\pgfqpoint{5.162225in}{3.318050in}}%
\pgfpathlineto{\pgfqpoint{5.133090in}{3.304178in}}%
\pgfusepath{stroke}%
\end{pgfscope}%
\begin{pgfscope}%
\pgfpathrectangle{\pgfqpoint{3.985294in}{1.750000in}}{\pgfqpoint{2.279412in}{2.004545in}}%
\pgfusepath{clip}%
\pgfsetbuttcap%
\pgfsetroundjoin%
\pgfsetlinewidth{0.343928pt}%
\definecolor{currentstroke}{rgb}{0.274952,0.037752,0.364543}%
\pgfsetstrokecolor{currentstroke}%
\pgfsetdash{}{0pt}%
\pgfpathmoveto{\pgfqpoint{5.133090in}{3.304178in}}%
\pgfpathlineto{\pgfqpoint{5.111502in}{3.286419in}}%
\pgfusepath{stroke}%
\end{pgfscope}%
\begin{pgfscope}%
\pgfpathrectangle{\pgfqpoint{3.985294in}{1.750000in}}{\pgfqpoint{2.279412in}{2.004545in}}%
\pgfusepath{clip}%
\pgfsetbuttcap%
\pgfsetroundjoin%
\pgfsetlinewidth{0.350505pt}%
\definecolor{currentstroke}{rgb}{0.276022,0.044167,0.370164}%
\pgfsetstrokecolor{currentstroke}%
\pgfsetdash{}{0pt}%
\pgfpathmoveto{\pgfqpoint{5.111502in}{3.286419in}}%
\pgfpathlineto{\pgfqpoint{5.093469in}{3.267547in}}%
\pgfusepath{stroke}%
\end{pgfscope}%
\begin{pgfscope}%
\pgfpathrectangle{\pgfqpoint{3.985294in}{1.750000in}}{\pgfqpoint{2.279412in}{2.004545in}}%
\pgfusepath{clip}%
\pgfsetbuttcap%
\pgfsetroundjoin%
\pgfsetlinewidth{0.343954pt}%
\definecolor{currentstroke}{rgb}{0.274952,0.037752,0.364543}%
\pgfsetstrokecolor{currentstroke}%
\pgfsetdash{}{0pt}%
\pgfpathmoveto{\pgfqpoint{5.073708in}{2.165885in}}%
\pgfpathlineto{\pgfqpoint{5.054097in}{2.202004in}}%
\pgfusepath{stroke}%
\end{pgfscope}%
\begin{pgfscope}%
\pgfpathrectangle{\pgfqpoint{3.985294in}{1.750000in}}{\pgfqpoint{2.279412in}{2.004545in}}%
\pgfusepath{clip}%
\pgfsetbuttcap%
\pgfsetroundjoin%
\pgfsetlinewidth{0.351156pt}%
\definecolor{currentstroke}{rgb}{0.276022,0.044167,0.370164}%
\pgfsetstrokecolor{currentstroke}%
\pgfsetdash{}{0pt}%
\pgfpathmoveto{\pgfqpoint{5.054097in}{2.202004in}}%
\pgfpathlineto{\pgfqpoint{5.054097in}{2.202004in}}%
\pgfusepath{stroke}%
\end{pgfscope}%
\begin{pgfscope}%
\pgfpathrectangle{\pgfqpoint{3.985294in}{1.750000in}}{\pgfqpoint{2.279412in}{2.004545in}}%
\pgfusepath{clip}%
\pgfsetbuttcap%
\pgfsetroundjoin%
\pgfsetlinewidth{0.351156pt}%
\definecolor{currentstroke}{rgb}{0.276022,0.044167,0.370164}%
\pgfsetstrokecolor{currentstroke}%
\pgfsetdash{}{0pt}%
\pgfpathmoveto{\pgfqpoint{5.054097in}{2.202004in}}%
\pgfpathlineto{\pgfqpoint{5.039648in}{2.233470in}}%
\pgfusepath{stroke}%
\end{pgfscope}%
\begin{pgfscope}%
\pgfpathrectangle{\pgfqpoint{3.985294in}{1.750000in}}{\pgfqpoint{2.279412in}{2.004545in}}%
\pgfusepath{clip}%
\pgfsetbuttcap%
\pgfsetroundjoin%
\pgfsetlinewidth{0.379428pt}%
\definecolor{currentstroke}{rgb}{0.279566,0.067836,0.391917}%
\pgfsetstrokecolor{currentstroke}%
\pgfsetdash{}{0pt}%
\pgfpathmoveto{\pgfqpoint{5.039648in}{2.233470in}}%
\pgfpathlineto{\pgfqpoint{5.030303in}{2.266779in}}%
\pgfusepath{stroke}%
\end{pgfscope}%
\begin{pgfscope}%
\pgfpathrectangle{\pgfqpoint{3.985294in}{1.750000in}}{\pgfqpoint{2.279412in}{2.004545in}}%
\pgfusepath{clip}%
\pgfsetbuttcap%
\pgfsetroundjoin%
\pgfsetlinewidth{0.378993pt}%
\definecolor{currentstroke}{rgb}{0.279566,0.067836,0.391917}%
\pgfsetstrokecolor{currentstroke}%
\pgfsetdash{}{0pt}%
\pgfpathmoveto{\pgfqpoint{5.030303in}{2.266779in}}%
\pgfpathlineto{\pgfqpoint{5.030303in}{2.266779in}}%
\pgfusepath{stroke}%
\end{pgfscope}%
\begin{pgfscope}%
\pgfpathrectangle{\pgfqpoint{3.985294in}{1.750000in}}{\pgfqpoint{2.279412in}{2.004545in}}%
\pgfusepath{clip}%
\pgfsetbuttcap%
\pgfsetroundjoin%
\pgfsetlinewidth{0.378993pt}%
\definecolor{currentstroke}{rgb}{0.279566,0.067836,0.391917}%
\pgfsetstrokecolor{currentstroke}%
\pgfsetdash{}{0pt}%
\pgfpathmoveto{\pgfqpoint{5.030303in}{2.266779in}}%
\pgfpathlineto{\pgfqpoint{5.018946in}{2.293498in}}%
\pgfusepath{stroke}%
\end{pgfscope}%
\begin{pgfscope}%
\pgfpathrectangle{\pgfqpoint{3.985294in}{1.750000in}}{\pgfqpoint{2.279412in}{2.004545in}}%
\pgfusepath{clip}%
\pgfsetbuttcap%
\pgfsetroundjoin%
\pgfsetlinewidth{0.381900pt}%
\definecolor{currentstroke}{rgb}{0.279566,0.067836,0.391917}%
\pgfsetstrokecolor{currentstroke}%
\pgfsetdash{}{0pt}%
\pgfpathmoveto{\pgfqpoint{5.018946in}{2.293498in}}%
\pgfpathlineto{\pgfqpoint{5.006401in}{2.318016in}}%
\pgfusepath{stroke}%
\end{pgfscope}%
\begin{pgfscope}%
\pgfpathrectangle{\pgfqpoint{3.985294in}{1.750000in}}{\pgfqpoint{2.279412in}{2.004545in}}%
\pgfusepath{clip}%
\pgfsetbuttcap%
\pgfsetroundjoin%
\pgfsetlinewidth{0.352447pt}%
\definecolor{currentstroke}{rgb}{0.276022,0.044167,0.370164}%
\pgfsetstrokecolor{currentstroke}%
\pgfsetdash{}{0pt}%
\pgfpathmoveto{\pgfqpoint{5.006401in}{2.318016in}}%
\pgfpathlineto{\pgfqpoint{4.985837in}{2.357411in}}%
\pgfusepath{stroke}%
\end{pgfscope}%
\begin{pgfscope}%
\pgfpathrectangle{\pgfqpoint{3.985294in}{1.750000in}}{\pgfqpoint{2.279412in}{2.004545in}}%
\pgfusepath{clip}%
\pgfsetbuttcap%
\pgfsetroundjoin%
\pgfsetlinewidth{0.433526pt}%
\definecolor{currentstroke}{rgb}{0.283091,0.110553,0.431554}%
\pgfsetstrokecolor{currentstroke}%
\pgfsetdash{}{0pt}%
\pgfpathmoveto{\pgfqpoint{4.985837in}{2.357411in}}%
\pgfpathlineto{\pgfqpoint{4.960382in}{2.394914in}}%
\pgfusepath{stroke}%
\end{pgfscope}%
\begin{pgfscope}%
\pgfpathrectangle{\pgfqpoint{3.985294in}{1.750000in}}{\pgfqpoint{2.279412in}{2.004545in}}%
\pgfusepath{clip}%
\pgfsetbuttcap%
\pgfsetroundjoin%
\pgfsetlinewidth{0.549425pt}%
\definecolor{currentstroke}{rgb}{0.275191,0.194905,0.496005}%
\pgfsetstrokecolor{currentstroke}%
\pgfsetdash{}{0pt}%
\pgfpathmoveto{\pgfqpoint{4.960382in}{2.394914in}}%
\pgfpathlineto{\pgfqpoint{4.937827in}{2.428264in}}%
\pgfusepath{stroke}%
\end{pgfscope}%
\begin{pgfscope}%
\pgfpathrectangle{\pgfqpoint{3.985294in}{1.750000in}}{\pgfqpoint{2.279412in}{2.004545in}}%
\pgfusepath{clip}%
\pgfsetbuttcap%
\pgfsetroundjoin%
\pgfsetlinewidth{0.495031pt}%
\definecolor{currentstroke}{rgb}{0.281412,0.155834,0.469201}%
\pgfsetstrokecolor{currentstroke}%
\pgfsetdash{}{0pt}%
\pgfpathmoveto{\pgfqpoint{4.937827in}{2.428264in}}%
\pgfpathlineto{\pgfqpoint{4.937827in}{2.428264in}}%
\pgfusepath{stroke}%
\end{pgfscope}%
\begin{pgfscope}%
\pgfpathrectangle{\pgfqpoint{3.985294in}{1.750000in}}{\pgfqpoint{2.279412in}{2.004545in}}%
\pgfusepath{clip}%
\pgfsetbuttcap%
\pgfsetroundjoin%
\pgfsetlinewidth{0.495031pt}%
\definecolor{currentstroke}{rgb}{0.281412,0.155834,0.469201}%
\pgfsetstrokecolor{currentstroke}%
\pgfsetdash{}{0pt}%
\pgfpathmoveto{\pgfqpoint{4.937827in}{2.428264in}}%
\pgfpathlineto{\pgfqpoint{4.916933in}{2.457927in}}%
\pgfusepath{stroke}%
\end{pgfscope}%
\begin{pgfscope}%
\pgfpathrectangle{\pgfqpoint{3.985294in}{1.750000in}}{\pgfqpoint{2.279412in}{2.004545in}}%
\pgfusepath{clip}%
\pgfsetbuttcap%
\pgfsetroundjoin%
\pgfsetlinewidth{0.350239pt}%
\definecolor{currentstroke}{rgb}{0.276022,0.044167,0.370164}%
\pgfsetstrokecolor{currentstroke}%
\pgfsetdash{}{0pt}%
\pgfpathmoveto{\pgfqpoint{5.330168in}{2.165885in}}%
\pgfpathlineto{\pgfqpoint{5.281276in}{2.175659in}}%
\pgfusepath{stroke}%
\end{pgfscope}%
\begin{pgfscope}%
\pgfpathrectangle{\pgfqpoint{3.985294in}{1.750000in}}{\pgfqpoint{2.279412in}{2.004545in}}%
\pgfusepath{clip}%
\pgfsetbuttcap%
\pgfsetroundjoin%
\pgfsetlinewidth{0.360264pt}%
\definecolor{currentstroke}{rgb}{0.277018,0.050344,0.375715}%
\pgfsetstrokecolor{currentstroke}%
\pgfsetdash{}{0pt}%
\pgfpathmoveto{\pgfqpoint{5.281276in}{2.175659in}}%
\pgfpathlineto{\pgfqpoint{5.233176in}{2.186844in}}%
\pgfusepath{stroke}%
\end{pgfscope}%
\begin{pgfscope}%
\pgfpathrectangle{\pgfqpoint{3.985294in}{1.750000in}}{\pgfqpoint{2.279412in}{2.004545in}}%
\pgfusepath{clip}%
\pgfsetbuttcap%
\pgfsetroundjoin%
\pgfsetlinewidth{0.355372pt}%
\definecolor{currentstroke}{rgb}{0.276022,0.044167,0.370164}%
\pgfsetstrokecolor{currentstroke}%
\pgfsetdash{}{0pt}%
\pgfpathmoveto{\pgfqpoint{5.233176in}{2.186844in}}%
\pgfpathlineto{\pgfqpoint{5.187049in}{2.202726in}}%
\pgfusepath{stroke}%
\end{pgfscope}%
\begin{pgfscope}%
\pgfpathrectangle{\pgfqpoint{3.985294in}{1.750000in}}{\pgfqpoint{2.279412in}{2.004545in}}%
\pgfusepath{clip}%
\pgfsetbuttcap%
\pgfsetroundjoin%
\pgfsetlinewidth{0.343553pt}%
\definecolor{currentstroke}{rgb}{0.274952,0.037752,0.364543}%
\pgfsetstrokecolor{currentstroke}%
\pgfsetdash{}{0pt}%
\pgfpathmoveto{\pgfqpoint{5.187049in}{2.202726in}}%
\pgfpathlineto{\pgfqpoint{5.143920in}{2.224020in}}%
\pgfusepath{stroke}%
\end{pgfscope}%
\begin{pgfscope}%
\pgfpathrectangle{\pgfqpoint{3.985294in}{1.750000in}}{\pgfqpoint{2.279412in}{2.004545in}}%
\pgfusepath{clip}%
\pgfsetbuttcap%
\pgfsetroundjoin%
\pgfsetlinewidth{0.360667pt}%
\definecolor{currentstroke}{rgb}{0.277018,0.050344,0.375715}%
\pgfsetstrokecolor{currentstroke}%
\pgfsetdash{}{0pt}%
\pgfpathmoveto{\pgfqpoint{5.143920in}{2.224020in}}%
\pgfpathlineto{\pgfqpoint{5.101867in}{2.247452in}}%
\pgfusepath{stroke}%
\end{pgfscope}%
\begin{pgfscope}%
\pgfpathrectangle{\pgfqpoint{3.985294in}{1.750000in}}{\pgfqpoint{2.279412in}{2.004545in}}%
\pgfusepath{clip}%
\pgfsetbuttcap%
\pgfsetroundjoin%
\pgfsetlinewidth{0.359189pt}%
\definecolor{currentstroke}{rgb}{0.277018,0.050344,0.375715}%
\pgfsetstrokecolor{currentstroke}%
\pgfsetdash{}{0pt}%
\pgfpathmoveto{\pgfqpoint{5.101867in}{2.247452in}}%
\pgfpathlineto{\pgfqpoint{5.058263in}{2.267747in}}%
\pgfusepath{stroke}%
\end{pgfscope}%
\begin{pgfscope}%
\pgfpathrectangle{\pgfqpoint{3.985294in}{1.750000in}}{\pgfqpoint{2.279412in}{2.004545in}}%
\pgfusepath{clip}%
\pgfsetbuttcap%
\pgfsetroundjoin%
\pgfsetlinewidth{0.330966pt}%
\definecolor{currentstroke}{rgb}{0.272594,0.025563,0.353093}%
\pgfsetstrokecolor{currentstroke}%
\pgfsetdash{}{0pt}%
\pgfpathmoveto{\pgfqpoint{5.058263in}{2.267747in}}%
\pgfpathlineto{\pgfqpoint{5.058263in}{2.267747in}}%
\pgfusepath{stroke}%
\end{pgfscope}%
\begin{pgfscope}%
\pgfpathrectangle{\pgfqpoint{3.985294in}{1.750000in}}{\pgfqpoint{2.279412in}{2.004545in}}%
\pgfusepath{clip}%
\pgfsetbuttcap%
\pgfsetroundjoin%
\pgfsetlinewidth{0.330966pt}%
\definecolor{currentstroke}{rgb}{0.272594,0.025563,0.353093}%
\pgfsetstrokecolor{currentstroke}%
\pgfsetdash{}{0pt}%
\pgfpathmoveto{\pgfqpoint{5.058263in}{2.267747in}}%
\pgfpathlineto{\pgfqpoint{5.058263in}{2.267747in}}%
\pgfusepath{stroke}%
\end{pgfscope}%
\begin{pgfscope}%
\pgfpathrectangle{\pgfqpoint{3.985294in}{1.750000in}}{\pgfqpoint{2.279412in}{2.004545in}}%
\pgfusepath{clip}%
\pgfsetbuttcap%
\pgfsetroundjoin%
\pgfsetlinewidth{0.310811pt}%
\definecolor{currentstroke}{rgb}{0.268510,0.009605,0.335427}%
\pgfsetstrokecolor{currentstroke}%
\pgfsetdash{}{0pt}%
\pgfpathmoveto{\pgfqpoint{5.863514in}{2.253412in}}%
\pgfpathlineto{\pgfqpoint{5.826864in}{2.254532in}}%
\pgfusepath{stroke}%
\end{pgfscope}%
\begin{pgfscope}%
\pgfpathrectangle{\pgfqpoint{3.985294in}{1.750000in}}{\pgfqpoint{2.279412in}{2.004545in}}%
\pgfusepath{clip}%
\pgfsetbuttcap%
\pgfsetroundjoin%
\pgfsetlinewidth{0.317216pt}%
\definecolor{currentstroke}{rgb}{0.269944,0.014625,0.341379}%
\pgfsetstrokecolor{currentstroke}%
\pgfsetdash{}{0pt}%
\pgfpathmoveto{\pgfqpoint{5.826864in}{2.254532in}}%
\pgfpathlineto{\pgfqpoint{5.791795in}{2.256098in}}%
\pgfusepath{stroke}%
\end{pgfscope}%
\begin{pgfscope}%
\pgfpathrectangle{\pgfqpoint{3.985294in}{1.750000in}}{\pgfqpoint{2.279412in}{2.004545in}}%
\pgfusepath{clip}%
\pgfsetbuttcap%
\pgfsetroundjoin%
\pgfsetlinewidth{0.315873pt}%
\definecolor{currentstroke}{rgb}{0.269944,0.014625,0.341379}%
\pgfsetstrokecolor{currentstroke}%
\pgfsetdash{}{0pt}%
\pgfpathmoveto{\pgfqpoint{5.791795in}{2.256098in}}%
\pgfpathlineto{\pgfqpoint{5.791795in}{2.256098in}}%
\pgfusepath{stroke}%
\end{pgfscope}%
\begin{pgfscope}%
\pgfpathrectangle{\pgfqpoint{3.985294in}{1.750000in}}{\pgfqpoint{2.279412in}{2.004545in}}%
\pgfusepath{clip}%
\pgfsetbuttcap%
\pgfsetroundjoin%
\pgfsetlinewidth{0.315873pt}%
\definecolor{currentstroke}{rgb}{0.269944,0.014625,0.341379}%
\pgfsetstrokecolor{currentstroke}%
\pgfsetdash{}{0pt}%
\pgfpathmoveto{\pgfqpoint{5.791795in}{2.256098in}}%
\pgfpathlineto{\pgfqpoint{5.791795in}{2.256098in}}%
\pgfusepath{stroke}%
\end{pgfscope}%
\begin{pgfscope}%
\pgfpathrectangle{\pgfqpoint{3.985294in}{1.750000in}}{\pgfqpoint{2.279412in}{2.004545in}}%
\pgfusepath{clip}%
\pgfsetbuttcap%
\pgfsetroundjoin%
\pgfsetlinewidth{0.315873pt}%
\definecolor{currentstroke}{rgb}{0.269944,0.014625,0.341379}%
\pgfsetstrokecolor{currentstroke}%
\pgfsetdash{}{0pt}%
\pgfpathmoveto{\pgfqpoint{5.791795in}{2.256098in}}%
\pgfpathlineto{\pgfqpoint{5.758706in}{2.256774in}}%
\pgfusepath{stroke}%
\end{pgfscope}%
\begin{pgfscope}%
\pgfpathrectangle{\pgfqpoint{3.985294in}{1.750000in}}{\pgfqpoint{2.279412in}{2.004545in}}%
\pgfusepath{clip}%
\pgfsetbuttcap%
\pgfsetroundjoin%
\pgfsetlinewidth{0.327232pt}%
\definecolor{currentstroke}{rgb}{0.271305,0.019942,0.347269}%
\pgfsetstrokecolor{currentstroke}%
\pgfsetdash{}{0pt}%
\pgfpathmoveto{\pgfqpoint{5.758706in}{2.256774in}}%
\pgfpathlineto{\pgfqpoint{5.713424in}{2.258256in}}%
\pgfusepath{stroke}%
\end{pgfscope}%
\begin{pgfscope}%
\pgfpathrectangle{\pgfqpoint{3.985294in}{1.750000in}}{\pgfqpoint{2.279412in}{2.004545in}}%
\pgfusepath{clip}%
\pgfsetbuttcap%
\pgfsetroundjoin%
\pgfsetlinewidth{0.327319pt}%
\definecolor{currentstroke}{rgb}{0.271305,0.019942,0.347269}%
\pgfsetstrokecolor{currentstroke}%
\pgfsetdash{}{0pt}%
\pgfpathmoveto{\pgfqpoint{5.713424in}{2.258256in}}%
\pgfpathlineto{\pgfqpoint{5.663321in}{2.259869in}}%
\pgfusepath{stroke}%
\end{pgfscope}%
\begin{pgfscope}%
\pgfpathrectangle{\pgfqpoint{3.985294in}{1.750000in}}{\pgfqpoint{2.279412in}{2.004545in}}%
\pgfusepath{clip}%
\pgfsetbuttcap%
\pgfsetroundjoin%
\pgfsetlinewidth{0.324662pt}%
\definecolor{currentstroke}{rgb}{0.271305,0.019942,0.347269}%
\pgfsetstrokecolor{currentstroke}%
\pgfsetdash{}{0pt}%
\pgfpathmoveto{\pgfqpoint{5.663321in}{2.259869in}}%
\pgfpathlineto{\pgfqpoint{5.613236in}{2.261881in}}%
\pgfusepath{stroke}%
\end{pgfscope}%
\begin{pgfscope}%
\pgfpathrectangle{\pgfqpoint{3.985294in}{1.750000in}}{\pgfqpoint{2.279412in}{2.004545in}}%
\pgfusepath{clip}%
\pgfsetbuttcap%
\pgfsetroundjoin%
\pgfsetlinewidth{0.347178pt}%
\definecolor{currentstroke}{rgb}{0.274952,0.037752,0.364543}%
\pgfsetstrokecolor{currentstroke}%
\pgfsetdash{}{0pt}%
\pgfpathmoveto{\pgfqpoint{5.613236in}{2.261881in}}%
\pgfpathlineto{\pgfqpoint{5.563154in}{2.263901in}}%
\pgfusepath{stroke}%
\end{pgfscope}%
\begin{pgfscope}%
\pgfpathrectangle{\pgfqpoint{3.985294in}{1.750000in}}{\pgfqpoint{2.279412in}{2.004545in}}%
\pgfusepath{clip}%
\pgfsetbuttcap%
\pgfsetroundjoin%
\pgfsetlinewidth{0.355644pt}%
\definecolor{currentstroke}{rgb}{0.276022,0.044167,0.370164}%
\pgfsetstrokecolor{currentstroke}%
\pgfsetdash{}{0pt}%
\pgfpathmoveto{\pgfqpoint{5.563154in}{2.263901in}}%
\pgfpathlineto{\pgfqpoint{5.513099in}{2.266480in}}%
\pgfusepath{stroke}%
\end{pgfscope}%
\begin{pgfscope}%
\pgfpathrectangle{\pgfqpoint{3.985294in}{1.750000in}}{\pgfqpoint{2.279412in}{2.004545in}}%
\pgfusepath{clip}%
\pgfsetbuttcap%
\pgfsetroundjoin%
\pgfsetlinewidth{0.371428pt}%
\definecolor{currentstroke}{rgb}{0.278791,0.062145,0.386592}%
\pgfsetstrokecolor{currentstroke}%
\pgfsetdash{}{0pt}%
\pgfpathmoveto{\pgfqpoint{5.513099in}{2.266480in}}%
\pgfpathlineto{\pgfqpoint{5.463145in}{2.270277in}}%
\pgfusepath{stroke}%
\end{pgfscope}%
\begin{pgfscope}%
\pgfpathrectangle{\pgfqpoint{3.985294in}{1.750000in}}{\pgfqpoint{2.279412in}{2.004545in}}%
\pgfusepath{clip}%
\pgfsetbuttcap%
\pgfsetroundjoin%
\pgfsetlinewidth{0.382822pt}%
\definecolor{currentstroke}{rgb}{0.279566,0.067836,0.391917}%
\pgfsetstrokecolor{currentstroke}%
\pgfsetdash{}{0pt}%
\pgfpathmoveto{\pgfqpoint{5.463145in}{2.270277in}}%
\pgfpathlineto{\pgfqpoint{5.413268in}{2.274861in}}%
\pgfusepath{stroke}%
\end{pgfscope}%
\begin{pgfscope}%
\pgfpathrectangle{\pgfqpoint{3.985294in}{1.750000in}}{\pgfqpoint{2.279412in}{2.004545in}}%
\pgfusepath{clip}%
\pgfsetbuttcap%
\pgfsetroundjoin%
\pgfsetlinewidth{0.332036pt}%
\definecolor{currentstroke}{rgb}{0.272594,0.025563,0.353093}%
\pgfsetstrokecolor{currentstroke}%
\pgfsetdash{}{0pt}%
\pgfpathmoveto{\pgfqpoint{5.791795in}{2.526739in}}%
\pgfpathlineto{\pgfqpoint{5.741650in}{2.527059in}}%
\pgfusepath{stroke}%
\end{pgfscope}%
\begin{pgfscope}%
\pgfpathrectangle{\pgfqpoint{3.985294in}{1.750000in}}{\pgfqpoint{2.279412in}{2.004545in}}%
\pgfusepath{clip}%
\pgfsetbuttcap%
\pgfsetroundjoin%
\pgfsetlinewidth{0.330438pt}%
\definecolor{currentstroke}{rgb}{0.272594,0.025563,0.353093}%
\pgfsetstrokecolor{currentstroke}%
\pgfsetdash{}{0pt}%
\pgfpathmoveto{\pgfqpoint{5.741650in}{2.527059in}}%
\pgfpathlineto{\pgfqpoint{5.691503in}{2.527526in}}%
\pgfusepath{stroke}%
\end{pgfscope}%
\begin{pgfscope}%
\pgfpathrectangle{\pgfqpoint{3.985294in}{1.750000in}}{\pgfqpoint{2.279412in}{2.004545in}}%
\pgfusepath{clip}%
\pgfsetbuttcap%
\pgfsetroundjoin%
\pgfsetlinewidth{0.338217pt}%
\definecolor{currentstroke}{rgb}{0.273809,0.031497,0.358853}%
\pgfsetstrokecolor{currentstroke}%
\pgfsetdash{}{0pt}%
\pgfpathmoveto{\pgfqpoint{5.691503in}{2.527526in}}%
\pgfpathlineto{\pgfqpoint{5.641362in}{2.528308in}}%
\pgfusepath{stroke}%
\end{pgfscope}%
\begin{pgfscope}%
\pgfpathrectangle{\pgfqpoint{3.985294in}{1.750000in}}{\pgfqpoint{2.279412in}{2.004545in}}%
\pgfusepath{clip}%
\pgfsetbuttcap%
\pgfsetroundjoin%
\pgfsetlinewidth{0.365912pt}%
\definecolor{currentstroke}{rgb}{0.277941,0.056324,0.381191}%
\pgfsetstrokecolor{currentstroke}%
\pgfsetdash{}{0pt}%
\pgfpathmoveto{\pgfqpoint{5.641362in}{2.528308in}}%
\pgfpathlineto{\pgfqpoint{5.591229in}{2.529502in}}%
\pgfusepath{stroke}%
\end{pgfscope}%
\begin{pgfscope}%
\pgfpathrectangle{\pgfqpoint{3.985294in}{1.750000in}}{\pgfqpoint{2.279412in}{2.004545in}}%
\pgfusepath{clip}%
\pgfsetbuttcap%
\pgfsetroundjoin%
\pgfsetlinewidth{0.410704pt}%
\definecolor{currentstroke}{rgb}{0.281924,0.089666,0.412415}%
\pgfsetstrokecolor{currentstroke}%
\pgfsetdash{}{0pt}%
\pgfpathmoveto{\pgfqpoint{5.591229in}{2.529502in}}%
\pgfpathlineto{\pgfqpoint{5.541099in}{2.530796in}}%
\pgfusepath{stroke}%
\end{pgfscope}%
\begin{pgfscope}%
\pgfpathrectangle{\pgfqpoint{3.985294in}{1.750000in}}{\pgfqpoint{2.279412in}{2.004545in}}%
\pgfusepath{clip}%
\pgfsetbuttcap%
\pgfsetroundjoin%
\pgfsetlinewidth{0.460182pt}%
\definecolor{currentstroke}{rgb}{0.283072,0.130895,0.449241}%
\pgfsetstrokecolor{currentstroke}%
\pgfsetdash{}{0pt}%
\pgfpathmoveto{\pgfqpoint{5.541099in}{2.530796in}}%
\pgfpathlineto{\pgfqpoint{5.490971in}{2.532174in}}%
\pgfusepath{stroke}%
\end{pgfscope}%
\begin{pgfscope}%
\pgfpathrectangle{\pgfqpoint{3.985294in}{1.750000in}}{\pgfqpoint{2.279412in}{2.004545in}}%
\pgfusepath{clip}%
\pgfsetbuttcap%
\pgfsetroundjoin%
\pgfsetlinewidth{0.507923pt}%
\definecolor{currentstroke}{rgb}{0.280255,0.165693,0.476498}%
\pgfsetstrokecolor{currentstroke}%
\pgfsetdash{}{0pt}%
\pgfpathmoveto{\pgfqpoint{5.490971in}{2.532174in}}%
\pgfpathlineto{\pgfqpoint{5.440861in}{2.533937in}}%
\pgfusepath{stroke}%
\end{pgfscope}%
\begin{pgfscope}%
\pgfpathrectangle{\pgfqpoint{3.985294in}{1.750000in}}{\pgfqpoint{2.279412in}{2.004545in}}%
\pgfusepath{clip}%
\pgfsetbuttcap%
\pgfsetroundjoin%
\pgfsetlinewidth{0.563479pt}%
\definecolor{currentstroke}{rgb}{0.273006,0.204520,0.501721}%
\pgfsetstrokecolor{currentstroke}%
\pgfsetdash{}{0pt}%
\pgfpathmoveto{\pgfqpoint{5.440861in}{2.533937in}}%
\pgfpathlineto{\pgfqpoint{5.390789in}{2.536399in}}%
\pgfusepath{stroke}%
\end{pgfscope}%
\begin{pgfscope}%
\pgfpathrectangle{\pgfqpoint{3.985294in}{1.750000in}}{\pgfqpoint{2.279412in}{2.004545in}}%
\pgfusepath{clip}%
\pgfsetbuttcap%
\pgfsetroundjoin%
\pgfsetlinewidth{0.635505pt}%
\definecolor{currentstroke}{rgb}{0.258965,0.251537,0.524736}%
\pgfsetstrokecolor{currentstroke}%
\pgfsetdash{}{0pt}%
\pgfpathmoveto{\pgfqpoint{5.390789in}{2.536399in}}%
\pgfpathlineto{\pgfqpoint{5.340760in}{2.539476in}}%
\pgfusepath{stroke}%
\end{pgfscope}%
\begin{pgfscope}%
\pgfpathrectangle{\pgfqpoint{3.985294in}{1.750000in}}{\pgfqpoint{2.279412in}{2.004545in}}%
\pgfusepath{clip}%
\pgfsetbuttcap%
\pgfsetroundjoin%
\pgfsetlinewidth{0.684846pt}%
\definecolor{currentstroke}{rgb}{0.246811,0.283237,0.535941}%
\pgfsetstrokecolor{currentstroke}%
\pgfsetdash{}{0pt}%
\pgfpathmoveto{\pgfqpoint{5.340760in}{2.539476in}}%
\pgfpathlineto{\pgfqpoint{5.290807in}{2.543353in}}%
\pgfusepath{stroke}%
\end{pgfscope}%
\begin{pgfscope}%
\pgfpathrectangle{\pgfqpoint{3.985294in}{1.750000in}}{\pgfqpoint{2.279412in}{2.004545in}}%
\pgfusepath{clip}%
\pgfsetbuttcap%
\pgfsetroundjoin%
\pgfsetlinewidth{0.330662pt}%
\definecolor{currentstroke}{rgb}{0.272594,0.025563,0.353093}%
\pgfsetstrokecolor{currentstroke}%
\pgfsetdash{}{0pt}%
\pgfpathmoveto{\pgfqpoint{5.535335in}{3.338661in}}%
\pgfpathlineto{\pgfqpoint{5.485381in}{3.335670in}}%
\pgfusepath{stroke}%
\end{pgfscope}%
\begin{pgfscope}%
\pgfpathrectangle{\pgfqpoint{3.985294in}{1.750000in}}{\pgfqpoint{2.279412in}{2.004545in}}%
\pgfusepath{clip}%
\pgfsetbuttcap%
\pgfsetroundjoin%
\pgfsetlinewidth{0.346881pt}%
\definecolor{currentstroke}{rgb}{0.274952,0.037752,0.364543}%
\pgfsetstrokecolor{currentstroke}%
\pgfsetdash{}{0pt}%
\pgfpathmoveto{\pgfqpoint{5.485381in}{3.335670in}}%
\pgfpathlineto{\pgfqpoint{5.435679in}{3.330333in}}%
\pgfusepath{stroke}%
\end{pgfscope}%
\begin{pgfscope}%
\pgfpathrectangle{\pgfqpoint{3.985294in}{1.750000in}}{\pgfqpoint{2.279412in}{2.004545in}}%
\pgfusepath{clip}%
\pgfsetbuttcap%
\pgfsetroundjoin%
\pgfsetlinewidth{0.366078pt}%
\definecolor{currentstroke}{rgb}{0.277941,0.056324,0.381191}%
\pgfsetstrokecolor{currentstroke}%
\pgfsetdash{}{0pt}%
\pgfpathmoveto{\pgfqpoint{5.435679in}{3.330333in}}%
\pgfpathlineto{\pgfqpoint{5.386172in}{3.323584in}}%
\pgfusepath{stroke}%
\end{pgfscope}%
\begin{pgfscope}%
\pgfpathrectangle{\pgfqpoint{3.985294in}{1.750000in}}{\pgfqpoint{2.279412in}{2.004545in}}%
\pgfusepath{clip}%
\pgfsetbuttcap%
\pgfsetroundjoin%
\pgfsetlinewidth{0.343407pt}%
\definecolor{currentstroke}{rgb}{0.274952,0.037752,0.364543}%
\pgfsetstrokecolor{currentstroke}%
\pgfsetdash{}{0pt}%
\pgfpathmoveto{\pgfqpoint{5.386172in}{3.323584in}}%
\pgfpathlineto{\pgfqpoint{5.336580in}{3.317370in}}%
\pgfusepath{stroke}%
\end{pgfscope}%
\begin{pgfscope}%
\pgfpathrectangle{\pgfqpoint{3.985294in}{1.750000in}}{\pgfqpoint{2.279412in}{2.004545in}}%
\pgfusepath{clip}%
\pgfsetbuttcap%
\pgfsetroundjoin%
\pgfsetlinewidth{0.352975pt}%
\definecolor{currentstroke}{rgb}{0.276022,0.044167,0.370164}%
\pgfsetstrokecolor{currentstroke}%
\pgfsetdash{}{0pt}%
\pgfpathmoveto{\pgfqpoint{5.336580in}{3.317370in}}%
\pgfpathlineto{\pgfqpoint{5.287339in}{3.309388in}}%
\pgfusepath{stroke}%
\end{pgfscope}%
\begin{pgfscope}%
\pgfpathrectangle{\pgfqpoint{3.985294in}{1.750000in}}{\pgfqpoint{2.279412in}{2.004545in}}%
\pgfusepath{clip}%
\pgfsetbuttcap%
\pgfsetroundjoin%
\pgfsetlinewidth{0.347975pt}%
\definecolor{currentstroke}{rgb}{0.274952,0.037752,0.364543}%
\pgfsetstrokecolor{currentstroke}%
\pgfsetdash{}{0pt}%
\pgfpathmoveto{\pgfqpoint{5.287339in}{3.309388in}}%
\pgfpathlineto{\pgfqpoint{5.238308in}{3.301242in}}%
\pgfusepath{stroke}%
\end{pgfscope}%
\begin{pgfscope}%
\pgfpathrectangle{\pgfqpoint{3.985294in}{1.750000in}}{\pgfqpoint{2.279412in}{2.004545in}}%
\pgfusepath{clip}%
\pgfsetbuttcap%
\pgfsetroundjoin%
\pgfsetlinewidth{0.348587pt}%
\definecolor{currentstroke}{rgb}{0.274952,0.037752,0.364543}%
\pgfsetstrokecolor{currentstroke}%
\pgfsetdash{}{0pt}%
\pgfpathmoveto{\pgfqpoint{5.238308in}{3.301242in}}%
\pgfpathlineto{\pgfqpoint{5.189509in}{3.292665in}}%
\pgfusepath{stroke}%
\end{pgfscope}%
\begin{pgfscope}%
\pgfpathrectangle{\pgfqpoint{3.985294in}{1.750000in}}{\pgfqpoint{2.279412in}{2.004545in}}%
\pgfusepath{clip}%
\pgfsetbuttcap%
\pgfsetroundjoin%
\pgfsetlinewidth{0.388454pt}%
\definecolor{currentstroke}{rgb}{0.280267,0.073417,0.397163}%
\pgfsetstrokecolor{currentstroke}%
\pgfsetdash{}{0pt}%
\pgfpathmoveto{\pgfqpoint{5.428948in}{2.196133in}}%
\pgfpathlineto{\pgfqpoint{5.379412in}{2.202713in}}%
\pgfusepath{stroke}%
\end{pgfscope}%
\begin{pgfscope}%
\pgfpathrectangle{\pgfqpoint{3.985294in}{1.750000in}}{\pgfqpoint{2.279412in}{2.004545in}}%
\pgfusepath{clip}%
\pgfsetbuttcap%
\pgfsetroundjoin%
\pgfsetlinewidth{0.372057pt}%
\definecolor{currentstroke}{rgb}{0.278791,0.062145,0.386592}%
\pgfsetstrokecolor{currentstroke}%
\pgfsetdash{}{0pt}%
\pgfpathmoveto{\pgfqpoint{5.379412in}{2.202713in}}%
\pgfpathlineto{\pgfqpoint{5.330168in}{2.210991in}}%
\pgfusepath{stroke}%
\end{pgfscope}%
\begin{pgfscope}%
\pgfpathrectangle{\pgfqpoint{3.985294in}{1.750000in}}{\pgfqpoint{2.279412in}{2.004545in}}%
\pgfusepath{clip}%
\pgfsetbuttcap%
\pgfsetroundjoin%
\pgfsetlinewidth{0.367601pt}%
\definecolor{currentstroke}{rgb}{0.277941,0.056324,0.381191}%
\pgfsetstrokecolor{currentstroke}%
\pgfsetdash{}{0pt}%
\pgfpathmoveto{\pgfqpoint{5.330168in}{2.210991in}}%
\pgfpathlineto{\pgfqpoint{5.281062in}{2.219870in}}%
\pgfusepath{stroke}%
\end{pgfscope}%
\begin{pgfscope}%
\pgfpathrectangle{\pgfqpoint{3.985294in}{1.750000in}}{\pgfqpoint{2.279412in}{2.004545in}}%
\pgfusepath{clip}%
\pgfsetbuttcap%
\pgfsetroundjoin%
\pgfsetlinewidth{0.360678pt}%
\definecolor{currentstroke}{rgb}{0.277018,0.050344,0.375715}%
\pgfsetstrokecolor{currentstroke}%
\pgfsetdash{}{0pt}%
\pgfpathmoveto{\pgfqpoint{5.281062in}{2.219870in}}%
\pgfpathlineto{\pgfqpoint{5.231950in}{2.228606in}}%
\pgfusepath{stroke}%
\end{pgfscope}%
\begin{pgfscope}%
\pgfpathrectangle{\pgfqpoint{3.985294in}{1.750000in}}{\pgfqpoint{2.279412in}{2.004545in}}%
\pgfusepath{clip}%
\pgfsetbuttcap%
\pgfsetroundjoin%
\pgfsetlinewidth{0.339275pt}%
\definecolor{currentstroke}{rgb}{0.273809,0.031497,0.358853}%
\pgfsetstrokecolor{currentstroke}%
\pgfsetdash{}{0pt}%
\pgfpathmoveto{\pgfqpoint{5.231950in}{2.228606in}}%
\pgfpathlineto{\pgfqpoint{5.183345in}{2.238918in}}%
\pgfusepath{stroke}%
\end{pgfscope}%
\begin{pgfscope}%
\pgfpathrectangle{\pgfqpoint{3.985294in}{1.750000in}}{\pgfqpoint{2.279412in}{2.004545in}}%
\pgfusepath{clip}%
\pgfsetbuttcap%
\pgfsetroundjoin%
\pgfsetlinewidth{0.312863pt}%
\definecolor{currentstroke}{rgb}{0.268510,0.009605,0.335427}%
\pgfsetstrokecolor{currentstroke}%
\pgfsetdash{}{0pt}%
\pgfpathmoveto{\pgfqpoint{5.740503in}{3.158234in}}%
\pgfpathlineto{\pgfqpoint{5.690402in}{3.159422in}}%
\pgfusepath{stroke}%
\end{pgfscope}%
\begin{pgfscope}%
\pgfpathrectangle{\pgfqpoint{3.985294in}{1.750000in}}{\pgfqpoint{2.279412in}{2.004545in}}%
\pgfusepath{clip}%
\pgfsetbuttcap%
\pgfsetroundjoin%
\pgfsetlinewidth{0.333853pt}%
\definecolor{currentstroke}{rgb}{0.272594,0.025563,0.353093}%
\pgfsetstrokecolor{currentstroke}%
\pgfsetdash{}{0pt}%
\pgfpathmoveto{\pgfqpoint{5.690402in}{3.159422in}}%
\pgfpathlineto{\pgfqpoint{5.640270in}{3.158658in}}%
\pgfusepath{stroke}%
\end{pgfscope}%
\begin{pgfscope}%
\pgfpathrectangle{\pgfqpoint{3.985294in}{1.750000in}}{\pgfqpoint{2.279412in}{2.004545in}}%
\pgfusepath{clip}%
\pgfsetbuttcap%
\pgfsetroundjoin%
\pgfsetlinewidth{0.342365pt}%
\definecolor{currentstroke}{rgb}{0.274952,0.037752,0.364543}%
\pgfsetstrokecolor{currentstroke}%
\pgfsetdash{}{0pt}%
\pgfpathmoveto{\pgfqpoint{5.640270in}{3.158658in}}%
\pgfpathlineto{\pgfqpoint{5.590171in}{3.156867in}}%
\pgfusepath{stroke}%
\end{pgfscope}%
\begin{pgfscope}%
\pgfpathrectangle{\pgfqpoint{3.985294in}{1.750000in}}{\pgfqpoint{2.279412in}{2.004545in}}%
\pgfusepath{clip}%
\pgfsetbuttcap%
\pgfsetroundjoin%
\pgfsetlinewidth{0.355737pt}%
\definecolor{currentstroke}{rgb}{0.276022,0.044167,0.370164}%
\pgfsetstrokecolor{currentstroke}%
\pgfsetdash{}{0pt}%
\pgfpathmoveto{\pgfqpoint{5.590171in}{3.156867in}}%
\pgfpathlineto{\pgfqpoint{5.540085in}{3.154774in}}%
\pgfusepath{stroke}%
\end{pgfscope}%
\begin{pgfscope}%
\pgfpathrectangle{\pgfqpoint{3.985294in}{1.750000in}}{\pgfqpoint{2.279412in}{2.004545in}}%
\pgfusepath{clip}%
\pgfsetbuttcap%
\pgfsetroundjoin%
\pgfsetlinewidth{0.374007pt}%
\definecolor{currentstroke}{rgb}{0.278791,0.062145,0.386592}%
\pgfsetstrokecolor{currentstroke}%
\pgfsetdash{}{0pt}%
\pgfpathmoveto{\pgfqpoint{5.540085in}{3.154774in}}%
\pgfpathlineto{\pgfqpoint{5.490033in}{3.152052in}}%
\pgfusepath{stroke}%
\end{pgfscope}%
\begin{pgfscope}%
\pgfpathrectangle{\pgfqpoint{3.985294in}{1.750000in}}{\pgfqpoint{2.279412in}{2.004545in}}%
\pgfusepath{clip}%
\pgfsetbuttcap%
\pgfsetroundjoin%
\pgfsetlinewidth{0.388676pt}%
\definecolor{currentstroke}{rgb}{0.280267,0.073417,0.397163}%
\pgfsetstrokecolor{currentstroke}%
\pgfsetdash{}{0pt}%
\pgfpathmoveto{\pgfqpoint{5.490033in}{3.152052in}}%
\pgfpathlineto{\pgfqpoint{5.440082in}{3.148185in}}%
\pgfusepath{stroke}%
\end{pgfscope}%
\begin{pgfscope}%
\pgfpathrectangle{\pgfqpoint{3.985294in}{1.750000in}}{\pgfqpoint{2.279412in}{2.004545in}}%
\pgfusepath{clip}%
\pgfsetbuttcap%
\pgfsetroundjoin%
\pgfsetlinewidth{0.317890pt}%
\definecolor{currentstroke}{rgb}{0.269944,0.014625,0.341379}%
\pgfsetstrokecolor{currentstroke}%
\pgfsetdash{}{0pt}%
\pgfpathmoveto{\pgfqpoint{5.740503in}{3.248447in}}%
\pgfpathlineto{\pgfqpoint{5.690431in}{3.247947in}}%
\pgfusepath{stroke}%
\end{pgfscope}%
\begin{pgfscope}%
\pgfpathrectangle{\pgfqpoint{3.985294in}{1.750000in}}{\pgfqpoint{2.279412in}{2.004545in}}%
\pgfusepath{clip}%
\pgfsetbuttcap%
\pgfsetroundjoin%
\pgfsetlinewidth{0.323565pt}%
\definecolor{currentstroke}{rgb}{0.271305,0.019942,0.347269}%
\pgfsetstrokecolor{currentstroke}%
\pgfsetdash{}{0pt}%
\pgfpathmoveto{\pgfqpoint{5.690431in}{3.247947in}}%
\pgfpathlineto{\pgfqpoint{5.640310in}{3.248135in}}%
\pgfusepath{stroke}%
\end{pgfscope}%
\begin{pgfscope}%
\pgfpathrectangle{\pgfqpoint{3.985294in}{1.750000in}}{\pgfqpoint{2.279412in}{2.004545in}}%
\pgfusepath{clip}%
\pgfsetbuttcap%
\pgfsetroundjoin%
\pgfsetlinewidth{0.333218pt}%
\definecolor{currentstroke}{rgb}{0.272594,0.025563,0.353093}%
\pgfsetstrokecolor{currentstroke}%
\pgfsetdash{}{0pt}%
\pgfpathmoveto{\pgfqpoint{5.640310in}{3.248135in}}%
\pgfpathlineto{\pgfqpoint{5.590296in}{3.246041in}}%
\pgfusepath{stroke}%
\end{pgfscope}%
\begin{pgfscope}%
\pgfpathrectangle{\pgfqpoint{3.985294in}{1.750000in}}{\pgfqpoint{2.279412in}{2.004545in}}%
\pgfusepath{clip}%
\pgfsetbuttcap%
\pgfsetroundjoin%
\pgfsetlinewidth{0.340591pt}%
\definecolor{currentstroke}{rgb}{0.273809,0.031497,0.358853}%
\pgfsetstrokecolor{currentstroke}%
\pgfsetdash{}{0pt}%
\pgfpathmoveto{\pgfqpoint{5.590296in}{3.246041in}}%
\pgfpathlineto{\pgfqpoint{5.540221in}{3.244121in}}%
\pgfusepath{stroke}%
\end{pgfscope}%
\begin{pgfscope}%
\pgfpathrectangle{\pgfqpoint{3.985294in}{1.750000in}}{\pgfqpoint{2.279412in}{2.004545in}}%
\pgfusepath{clip}%
\pgfsetbuttcap%
\pgfsetroundjoin%
\pgfsetlinewidth{0.357469pt}%
\definecolor{currentstroke}{rgb}{0.277018,0.050344,0.375715}%
\pgfsetstrokecolor{currentstroke}%
\pgfsetdash{}{0pt}%
\pgfpathmoveto{\pgfqpoint{5.540221in}{3.244121in}}%
\pgfpathlineto{\pgfqpoint{5.490159in}{3.241943in}}%
\pgfusepath{stroke}%
\end{pgfscope}%
\begin{pgfscope}%
\pgfpathrectangle{\pgfqpoint{3.985294in}{1.750000in}}{\pgfqpoint{2.279412in}{2.004545in}}%
\pgfusepath{clip}%
\pgfsetbuttcap%
\pgfsetroundjoin%
\pgfsetlinewidth{0.362457pt}%
\definecolor{currentstroke}{rgb}{0.277018,0.050344,0.375715}%
\pgfsetstrokecolor{currentstroke}%
\pgfsetdash{}{0pt}%
\pgfpathmoveto{\pgfqpoint{5.490159in}{3.241943in}}%
\pgfpathlineto{\pgfqpoint{5.440193in}{3.238188in}}%
\pgfusepath{stroke}%
\end{pgfscope}%
\begin{pgfscope}%
\pgfpathrectangle{\pgfqpoint{3.985294in}{1.750000in}}{\pgfqpoint{2.279412in}{2.004545in}}%
\pgfusepath{clip}%
\pgfsetbuttcap%
\pgfsetroundjoin%
\pgfsetlinewidth{0.376976pt}%
\definecolor{currentstroke}{rgb}{0.279566,0.067836,0.391917}%
\pgfsetstrokecolor{currentstroke}%
\pgfsetdash{}{0pt}%
\pgfpathmoveto{\pgfqpoint{5.440193in}{3.238188in}}%
\pgfpathlineto{\pgfqpoint{5.390342in}{3.233429in}}%
\pgfusepath{stroke}%
\end{pgfscope}%
\begin{pgfscope}%
\pgfpathrectangle{\pgfqpoint{3.985294in}{1.750000in}}{\pgfqpoint{2.279412in}{2.004545in}}%
\pgfusepath{clip}%
\pgfsetbuttcap%
\pgfsetroundjoin%
\pgfsetlinewidth{0.388201pt}%
\definecolor{currentstroke}{rgb}{0.280267,0.073417,0.397163}%
\pgfsetstrokecolor{currentstroke}%
\pgfsetdash{}{0pt}%
\pgfpathmoveto{\pgfqpoint{5.390342in}{3.233429in}}%
\pgfpathlineto{\pgfqpoint{5.340654in}{3.227538in}}%
\pgfusepath{stroke}%
\end{pgfscope}%
\begin{pgfscope}%
\pgfpathrectangle{\pgfqpoint{3.985294in}{1.750000in}}{\pgfqpoint{2.279412in}{2.004545in}}%
\pgfusepath{clip}%
\pgfsetbuttcap%
\pgfsetroundjoin%
\pgfsetlinewidth{0.371796pt}%
\definecolor{currentstroke}{rgb}{0.278791,0.062145,0.386592}%
\pgfsetstrokecolor{currentstroke}%
\pgfsetdash{}{0pt}%
\pgfpathmoveto{\pgfqpoint{5.340654in}{3.227538in}}%
\pgfpathlineto{\pgfqpoint{5.291286in}{3.219966in}}%
\pgfusepath{stroke}%
\end{pgfscope}%
\begin{pgfscope}%
\pgfpathrectangle{\pgfqpoint{3.985294in}{1.750000in}}{\pgfqpoint{2.279412in}{2.004545in}}%
\pgfusepath{clip}%
\pgfsetbuttcap%
\pgfsetroundjoin%
\pgfsetlinewidth{0.388474pt}%
\definecolor{currentstroke}{rgb}{0.280267,0.073417,0.397163}%
\pgfsetstrokecolor{currentstroke}%
\pgfsetdash{}{0pt}%
\pgfpathmoveto{\pgfqpoint{5.291286in}{3.219966in}}%
\pgfpathlineto{\pgfqpoint{5.242332in}{3.210545in}}%
\pgfusepath{stroke}%
\end{pgfscope}%
\begin{pgfscope}%
\pgfpathrectangle{\pgfqpoint{3.985294in}{1.750000in}}{\pgfqpoint{2.279412in}{2.004545in}}%
\pgfusepath{clip}%
\pgfsetbuttcap%
\pgfsetroundjoin%
\pgfsetlinewidth{0.378960pt}%
\definecolor{currentstroke}{rgb}{0.279566,0.067836,0.391917}%
\pgfsetstrokecolor{currentstroke}%
\pgfsetdash{}{0pt}%
\pgfpathmoveto{\pgfqpoint{5.242332in}{3.210545in}}%
\pgfpathlineto{\pgfqpoint{5.194120in}{3.198644in}}%
\pgfusepath{stroke}%
\end{pgfscope}%
\begin{pgfscope}%
\pgfpathrectangle{\pgfqpoint{3.985294in}{1.750000in}}{\pgfqpoint{2.279412in}{2.004545in}}%
\pgfusepath{clip}%
\pgfsetbuttcap%
\pgfsetroundjoin%
\pgfsetlinewidth{0.388065pt}%
\definecolor{currentstroke}{rgb}{0.280267,0.073417,0.397163}%
\pgfsetstrokecolor{currentstroke}%
\pgfsetdash{}{0pt}%
\pgfpathmoveto{\pgfqpoint{5.194120in}{3.198644in}}%
\pgfpathlineto{\pgfqpoint{5.147100in}{3.183401in}}%
\pgfusepath{stroke}%
\end{pgfscope}%
\begin{pgfscope}%
\pgfpathrectangle{\pgfqpoint{3.985294in}{1.750000in}}{\pgfqpoint{2.279412in}{2.004545in}}%
\pgfusepath{clip}%
\pgfsetbuttcap%
\pgfsetroundjoin%
\pgfsetlinewidth{0.391463pt}%
\definecolor{currentstroke}{rgb}{0.280894,0.078907,0.402329}%
\pgfsetstrokecolor{currentstroke}%
\pgfsetdash{}{0pt}%
\pgfpathmoveto{\pgfqpoint{5.147100in}{3.183401in}}%
\pgfpathlineto{\pgfqpoint{5.101784in}{3.164690in}}%
\pgfusepath{stroke}%
\end{pgfscope}%
\begin{pgfscope}%
\pgfpathrectangle{\pgfqpoint{3.985294in}{1.750000in}}{\pgfqpoint{2.279412in}{2.004545in}}%
\pgfusepath{clip}%
\pgfsetbuttcap%
\pgfsetroundjoin%
\pgfsetlinewidth{0.417246pt}%
\definecolor{currentstroke}{rgb}{0.282327,0.094955,0.417331}%
\pgfsetstrokecolor{currentstroke}%
\pgfsetdash{}{0pt}%
\pgfpathmoveto{\pgfqpoint{5.101784in}{3.164690in}}%
\pgfpathlineto{\pgfqpoint{5.061341in}{3.139506in}}%
\pgfusepath{stroke}%
\end{pgfscope}%
\begin{pgfscope}%
\pgfpathrectangle{\pgfqpoint{3.985294in}{1.750000in}}{\pgfqpoint{2.279412in}{2.004545in}}%
\pgfusepath{clip}%
\pgfsetbuttcap%
\pgfsetroundjoin%
\pgfsetlinewidth{0.438138pt}%
\definecolor{currentstroke}{rgb}{0.283091,0.110553,0.431554}%
\pgfsetstrokecolor{currentstroke}%
\pgfsetdash{}{0pt}%
\pgfpathmoveto{\pgfqpoint{5.061341in}{3.139506in}}%
\pgfpathlineto{\pgfqpoint{5.027967in}{3.112795in}}%
\pgfusepath{stroke}%
\end{pgfscope}%
\begin{pgfscope}%
\pgfpathrectangle{\pgfqpoint{3.985294in}{1.750000in}}{\pgfqpoint{2.279412in}{2.004545in}}%
\pgfusepath{clip}%
\pgfsetbuttcap%
\pgfsetroundjoin%
\pgfsetlinewidth{0.477260pt}%
\definecolor{currentstroke}{rgb}{0.282623,0.140926,0.457517}%
\pgfsetstrokecolor{currentstroke}%
\pgfsetdash{}{0pt}%
\pgfpathmoveto{\pgfqpoint{5.027967in}{3.112795in}}%
\pgfpathlineto{\pgfqpoint{4.995511in}{3.079532in}}%
\pgfusepath{stroke}%
\end{pgfscope}%
\begin{pgfscope}%
\pgfpathrectangle{\pgfqpoint{3.985294in}{1.750000in}}{\pgfqpoint{2.279412in}{2.004545in}}%
\pgfusepath{clip}%
\pgfsetbuttcap%
\pgfsetroundjoin%
\pgfsetlinewidth{0.342298pt}%
\definecolor{currentstroke}{rgb}{0.274952,0.037752,0.364543}%
\pgfsetstrokecolor{currentstroke}%
\pgfsetdash{}{0pt}%
\pgfpathmoveto{\pgfqpoint{5.484043in}{3.293554in}}%
\pgfpathlineto{\pgfqpoint{5.434025in}{3.290362in}}%
\pgfusepath{stroke}%
\end{pgfscope}%
\begin{pgfscope}%
\pgfpathrectangle{\pgfqpoint{3.985294in}{1.750000in}}{\pgfqpoint{2.279412in}{2.004545in}}%
\pgfusepath{clip}%
\pgfsetbuttcap%
\pgfsetroundjoin%
\pgfsetlinewidth{0.362360pt}%
\definecolor{currentstroke}{rgb}{0.277018,0.050344,0.375715}%
\pgfsetstrokecolor{currentstroke}%
\pgfsetdash{}{0pt}%
\pgfpathmoveto{\pgfqpoint{5.434025in}{3.290362in}}%
\pgfpathlineto{\pgfqpoint{5.384162in}{3.285799in}}%
\pgfusepath{stroke}%
\end{pgfscope}%
\begin{pgfscope}%
\pgfpathrectangle{\pgfqpoint{3.985294in}{1.750000in}}{\pgfqpoint{2.279412in}{2.004545in}}%
\pgfusepath{clip}%
\pgfsetbuttcap%
\pgfsetroundjoin%
\pgfsetlinewidth{0.376321pt}%
\definecolor{currentstroke}{rgb}{0.278791,0.062145,0.386592}%
\pgfsetstrokecolor{currentstroke}%
\pgfsetdash{}{0pt}%
\pgfpathmoveto{\pgfqpoint{5.384162in}{3.285799in}}%
\pgfpathlineto{\pgfqpoint{5.334603in}{3.279401in}}%
\pgfusepath{stroke}%
\end{pgfscope}%
\begin{pgfscope}%
\pgfpathrectangle{\pgfqpoint{3.985294in}{1.750000in}}{\pgfqpoint{2.279412in}{2.004545in}}%
\pgfusepath{clip}%
\pgfsetbuttcap%
\pgfsetroundjoin%
\pgfsetlinewidth{0.368544pt}%
\definecolor{currentstroke}{rgb}{0.277941,0.056324,0.381191}%
\pgfsetstrokecolor{currentstroke}%
\pgfsetdash{}{0pt}%
\pgfpathmoveto{\pgfqpoint{5.334603in}{3.279401in}}%
\pgfpathlineto{\pgfqpoint{5.286361in}{3.269194in}}%
\pgfusepath{stroke}%
\end{pgfscope}%
\begin{pgfscope}%
\pgfpathrectangle{\pgfqpoint{3.985294in}{1.750000in}}{\pgfqpoint{2.279412in}{2.004545in}}%
\pgfusepath{clip}%
\pgfsetbuttcap%
\pgfsetroundjoin%
\pgfsetlinewidth{0.324107pt}%
\definecolor{currentstroke}{rgb}{0.271305,0.019942,0.347269}%
\pgfsetstrokecolor{currentstroke}%
\pgfsetdash{}{0pt}%
\pgfpathmoveto{\pgfqpoint{5.286361in}{3.269194in}}%
\pgfpathlineto{\pgfqpoint{5.239398in}{3.255012in}}%
\pgfusepath{stroke}%
\end{pgfscope}%
\begin{pgfscope}%
\pgfpathrectangle{\pgfqpoint{3.985294in}{1.750000in}}{\pgfqpoint{2.279412in}{2.004545in}}%
\pgfusepath{clip}%
\pgfsetbuttcap%
\pgfsetroundjoin%
\pgfsetlinewidth{0.337729pt}%
\definecolor{currentstroke}{rgb}{0.273809,0.031497,0.358853}%
\pgfsetstrokecolor{currentstroke}%
\pgfsetdash{}{0pt}%
\pgfpathmoveto{\pgfqpoint{5.239398in}{3.255012in}}%
\pgfpathlineto{\pgfqpoint{5.192174in}{3.240570in}}%
\pgfusepath{stroke}%
\end{pgfscope}%
\begin{pgfscope}%
\pgfpathrectangle{\pgfqpoint{3.985294in}{1.750000in}}{\pgfqpoint{2.279412in}{2.004545in}}%
\pgfusepath{clip}%
\pgfsetbuttcap%
\pgfsetroundjoin%
\pgfsetlinewidth{0.367150pt}%
\definecolor{currentstroke}{rgb}{0.277941,0.056324,0.381191}%
\pgfsetstrokecolor{currentstroke}%
\pgfsetdash{}{0pt}%
\pgfpathmoveto{\pgfqpoint{5.192174in}{3.240570in}}%
\pgfpathlineto{\pgfqpoint{5.144346in}{3.228396in}}%
\pgfusepath{stroke}%
\end{pgfscope}%
\begin{pgfscope}%
\pgfpathrectangle{\pgfqpoint{3.985294in}{1.750000in}}{\pgfqpoint{2.279412in}{2.004545in}}%
\pgfusepath{clip}%
\pgfsetbuttcap%
\pgfsetroundjoin%
\pgfsetlinewidth{0.366580pt}%
\definecolor{currentstroke}{rgb}{0.277941,0.056324,0.381191}%
\pgfsetstrokecolor{currentstroke}%
\pgfsetdash{}{0pt}%
\pgfpathmoveto{\pgfqpoint{5.144346in}{3.228396in}}%
\pgfpathlineto{\pgfqpoint{5.144346in}{3.228396in}}%
\pgfusepath{stroke}%
\end{pgfscope}%
\begin{pgfscope}%
\pgfpathrectangle{\pgfqpoint{3.985294in}{1.750000in}}{\pgfqpoint{2.279412in}{2.004545in}}%
\pgfusepath{clip}%
\pgfsetbuttcap%
\pgfsetroundjoin%
\pgfsetlinewidth{0.394626pt}%
\definecolor{currentstroke}{rgb}{0.280894,0.078907,0.402329}%
\pgfsetstrokecolor{currentstroke}%
\pgfsetdash{}{0pt}%
\pgfpathmoveto{\pgfqpoint{5.370113in}{2.258977in}}%
\pgfpathlineto{\pgfqpoint{5.320682in}{2.266343in}}%
\pgfusepath{stroke}%
\end{pgfscope}%
\begin{pgfscope}%
\pgfpathrectangle{\pgfqpoint{3.985294in}{1.750000in}}{\pgfqpoint{2.279412in}{2.004545in}}%
\pgfusepath{clip}%
\pgfsetbuttcap%
\pgfsetroundjoin%
\pgfsetlinewidth{0.374979pt}%
\definecolor{currentstroke}{rgb}{0.278791,0.062145,0.386592}%
\pgfsetstrokecolor{currentstroke}%
\pgfsetdash{}{0pt}%
\pgfpathmoveto{\pgfqpoint{5.320682in}{2.266343in}}%
\pgfpathlineto{\pgfqpoint{5.271643in}{2.275381in}}%
\pgfusepath{stroke}%
\end{pgfscope}%
\begin{pgfscope}%
\pgfpathrectangle{\pgfqpoint{3.985294in}{1.750000in}}{\pgfqpoint{2.279412in}{2.004545in}}%
\pgfusepath{clip}%
\pgfsetbuttcap%
\pgfsetroundjoin%
\pgfsetlinewidth{0.405384pt}%
\definecolor{currentstroke}{rgb}{0.281924,0.089666,0.412415}%
\pgfsetstrokecolor{currentstroke}%
\pgfsetdash{}{0pt}%
\pgfpathmoveto{\pgfqpoint{5.271643in}{2.275381in}}%
\pgfpathlineto{\pgfqpoint{5.223015in}{2.286008in}}%
\pgfusepath{stroke}%
\end{pgfscope}%
\begin{pgfscope}%
\pgfpathrectangle{\pgfqpoint{3.985294in}{1.750000in}}{\pgfqpoint{2.279412in}{2.004545in}}%
\pgfusepath{clip}%
\pgfsetbuttcap%
\pgfsetroundjoin%
\pgfsetlinewidth{0.395971pt}%
\definecolor{currentstroke}{rgb}{0.280894,0.078907,0.402329}%
\pgfsetstrokecolor{currentstroke}%
\pgfsetdash{}{0pt}%
\pgfpathmoveto{\pgfqpoint{5.223015in}{2.286008in}}%
\pgfpathlineto{\pgfqpoint{5.176292in}{2.301205in}}%
\pgfusepath{stroke}%
\end{pgfscope}%
\begin{pgfscope}%
\pgfpathrectangle{\pgfqpoint{3.985294in}{1.750000in}}{\pgfqpoint{2.279412in}{2.004545in}}%
\pgfusepath{clip}%
\pgfsetbuttcap%
\pgfsetroundjoin%
\pgfsetlinewidth{0.392427pt}%
\definecolor{currentstroke}{rgb}{0.280894,0.078907,0.402329}%
\pgfsetstrokecolor{currentstroke}%
\pgfsetdash{}{0pt}%
\pgfpathmoveto{\pgfqpoint{5.176292in}{2.301205in}}%
\pgfpathlineto{\pgfqpoint{5.130488in}{2.318857in}}%
\pgfusepath{stroke}%
\end{pgfscope}%
\begin{pgfscope}%
\pgfpathrectangle{\pgfqpoint{3.985294in}{1.750000in}}{\pgfqpoint{2.279412in}{2.004545in}}%
\pgfusepath{clip}%
\pgfsetbuttcap%
\pgfsetroundjoin%
\pgfsetlinewidth{0.421594pt}%
\definecolor{currentstroke}{rgb}{0.282656,0.100196,0.422160}%
\pgfsetstrokecolor{currentstroke}%
\pgfsetdash{}{0pt}%
\pgfpathmoveto{\pgfqpoint{5.130488in}{2.318857in}}%
\pgfpathlineto{\pgfqpoint{5.085967in}{2.338724in}}%
\pgfusepath{stroke}%
\end{pgfscope}%
\begin{pgfscope}%
\pgfpathrectangle{\pgfqpoint{3.985294in}{1.750000in}}{\pgfqpoint{2.279412in}{2.004545in}}%
\pgfusepath{clip}%
\pgfsetbuttcap%
\pgfsetroundjoin%
\pgfsetlinewidth{0.423166pt}%
\definecolor{currentstroke}{rgb}{0.282656,0.100196,0.422160}%
\pgfsetstrokecolor{currentstroke}%
\pgfsetdash{}{0pt}%
\pgfpathmoveto{\pgfqpoint{5.085967in}{2.338724in}}%
\pgfpathlineto{\pgfqpoint{5.043709in}{2.362327in}}%
\pgfusepath{stroke}%
\end{pgfscope}%
\begin{pgfscope}%
\pgfpathrectangle{\pgfqpoint{3.985294in}{1.750000in}}{\pgfqpoint{2.279412in}{2.004545in}}%
\pgfusepath{clip}%
\pgfsetbuttcap%
\pgfsetroundjoin%
\pgfsetlinewidth{0.496440pt}%
\definecolor{currentstroke}{rgb}{0.281412,0.155834,0.469201}%
\pgfsetstrokecolor{currentstroke}%
\pgfsetdash{}{0pt}%
\pgfpathmoveto{\pgfqpoint{5.043709in}{2.362327in}}%
\pgfpathlineto{\pgfqpoint{5.005819in}{2.390153in}}%
\pgfusepath{stroke}%
\end{pgfscope}%
\begin{pgfscope}%
\pgfpathrectangle{\pgfqpoint{3.985294in}{1.750000in}}{\pgfqpoint{2.279412in}{2.004545in}}%
\pgfusepath{clip}%
\pgfsetbuttcap%
\pgfsetroundjoin%
\pgfsetlinewidth{0.930796pt}%
\definecolor{currentstroke}{rgb}{0.180629,0.429975,0.557282}%
\pgfsetstrokecolor{currentstroke}%
\pgfsetdash{}{0pt}%
\pgfpathmoveto{\pgfqpoint{4.612081in}{2.977807in}}%
\pgfpathlineto{\pgfqpoint{4.657697in}{2.959846in}}%
\pgfusepath{stroke}%
\end{pgfscope}%
\begin{pgfscope}%
\pgfpathrectangle{\pgfqpoint{3.985294in}{1.750000in}}{\pgfqpoint{2.279412in}{2.004545in}}%
\pgfusepath{clip}%
\pgfsetbuttcap%
\pgfsetroundjoin%
\pgfsetlinewidth{0.805218pt}%
\definecolor{currentstroke}{rgb}{0.212395,0.359683,0.551710}%
\pgfsetstrokecolor{currentstroke}%
\pgfsetdash{}{0pt}%
\pgfpathmoveto{\pgfqpoint{4.657697in}{2.959846in}}%
\pgfpathlineto{\pgfqpoint{4.657697in}{2.959846in}}%
\pgfusepath{stroke}%
\end{pgfscope}%
\begin{pgfscope}%
\pgfpathrectangle{\pgfqpoint{3.985294in}{1.750000in}}{\pgfqpoint{2.279412in}{2.004545in}}%
\pgfusepath{clip}%
\pgfsetbuttcap%
\pgfsetroundjoin%
\pgfsetlinewidth{0.805218pt}%
\definecolor{currentstroke}{rgb}{0.212395,0.359683,0.551710}%
\pgfsetstrokecolor{currentstroke}%
\pgfsetdash{}{0pt}%
\pgfpathmoveto{\pgfqpoint{4.657697in}{2.959846in}}%
\pgfpathlineto{\pgfqpoint{4.686683in}{2.942966in}}%
\pgfusepath{stroke}%
\end{pgfscope}%
\begin{pgfscope}%
\pgfpathrectangle{\pgfqpoint{3.985294in}{1.750000in}}{\pgfqpoint{2.279412in}{2.004545in}}%
\pgfusepath{clip}%
\pgfsetbuttcap%
\pgfsetroundjoin%
\pgfsetlinewidth{0.899117pt}%
\definecolor{currentstroke}{rgb}{0.187231,0.414746,0.556547}%
\pgfsetstrokecolor{currentstroke}%
\pgfsetdash{}{0pt}%
\pgfpathmoveto{\pgfqpoint{4.686683in}{2.942966in}}%
\pgfpathlineto{\pgfqpoint{4.710448in}{2.922228in}}%
\pgfusepath{stroke}%
\end{pgfscope}%
\begin{pgfscope}%
\pgfpathrectangle{\pgfqpoint{3.985294in}{1.750000in}}{\pgfqpoint{2.279412in}{2.004545in}}%
\pgfusepath{clip}%
\pgfsetbuttcap%
\pgfsetroundjoin%
\pgfsetlinewidth{0.858781pt}%
\definecolor{currentstroke}{rgb}{0.197636,0.391528,0.554969}%
\pgfsetstrokecolor{currentstroke}%
\pgfsetdash{}{0pt}%
\pgfpathmoveto{\pgfqpoint{4.710448in}{2.922228in}}%
\pgfpathlineto{\pgfqpoint{4.739210in}{2.890053in}}%
\pgfusepath{stroke}%
\end{pgfscope}%
\begin{pgfscope}%
\pgfpathrectangle{\pgfqpoint{3.985294in}{1.750000in}}{\pgfqpoint{2.279412in}{2.004545in}}%
\pgfusepath{clip}%
\pgfsetbuttcap%
\pgfsetroundjoin%
\pgfsetlinewidth{0.704103pt}%
\definecolor{currentstroke}{rgb}{0.241237,0.296485,0.539709}%
\pgfsetstrokecolor{currentstroke}%
\pgfsetdash{}{0pt}%
\pgfpathmoveto{\pgfqpoint{4.739210in}{2.890053in}}%
\pgfpathlineto{\pgfqpoint{4.739210in}{2.890053in}}%
\pgfusepath{stroke}%
\end{pgfscope}%
\begin{pgfscope}%
\pgfpathrectangle{\pgfqpoint{3.985294in}{1.750000in}}{\pgfqpoint{2.279412in}{2.004545in}}%
\pgfusepath{clip}%
\pgfsetbuttcap%
\pgfsetroundjoin%
\pgfsetlinewidth{0.704103pt}%
\definecolor{currentstroke}{rgb}{0.241237,0.296485,0.539709}%
\pgfsetstrokecolor{currentstroke}%
\pgfsetdash{}{0pt}%
\pgfpathmoveto{\pgfqpoint{4.739210in}{2.890053in}}%
\pgfpathlineto{\pgfqpoint{4.746044in}{2.873189in}}%
\pgfusepath{stroke}%
\end{pgfscope}%
\begin{pgfscope}%
\pgfpathrectangle{\pgfqpoint{3.985294in}{1.750000in}}{\pgfqpoint{2.279412in}{2.004545in}}%
\pgfusepath{clip}%
\pgfsetbuttcap%
\pgfsetroundjoin%
\pgfsetlinewidth{0.770836pt}%
\definecolor{currentstroke}{rgb}{0.221989,0.339161,0.548752}%
\pgfsetstrokecolor{currentstroke}%
\pgfsetdash{}{0pt}%
\pgfpathmoveto{\pgfqpoint{4.746044in}{2.873189in}}%
\pgfpathlineto{\pgfqpoint{4.745555in}{2.856810in}}%
\pgfusepath{stroke}%
\end{pgfscope}%
\begin{pgfscope}%
\pgfpathrectangle{\pgfqpoint{3.985294in}{1.750000in}}{\pgfqpoint{2.279412in}{2.004545in}}%
\pgfusepath{clip}%
\pgfsetbuttcap%
\pgfsetroundjoin%
\pgfsetlinewidth{0.781214pt}%
\definecolor{currentstroke}{rgb}{0.220057,0.343307,0.549413}%
\pgfsetstrokecolor{currentstroke}%
\pgfsetdash{}{0pt}%
\pgfpathmoveto{\pgfqpoint{4.745555in}{2.856810in}}%
\pgfpathlineto{\pgfqpoint{4.739626in}{2.837916in}}%
\pgfusepath{stroke}%
\end{pgfscope}%
\begin{pgfscope}%
\pgfpathrectangle{\pgfqpoint{3.985294in}{1.750000in}}{\pgfqpoint{2.279412in}{2.004545in}}%
\pgfusepath{clip}%
\pgfsetroundcap%
\pgfsetroundjoin%
\pgfsetlinewidth{0.935027pt}%
\definecolor{currentstroke}{rgb}{0.179019,0.433756,0.557430}%
\pgfsetstrokecolor{currentstroke}%
\pgfsetdash{}{0pt}%
\pgfpathmoveto{\pgfqpoint{5.293965in}{2.793064in}}%
\pgfpathquadraticcurveto{\pgfqpoint{5.281432in}{2.792761in}}{\pgfqpoint{5.283360in}{2.792808in}}%
\pgfusepath{stroke}%
\end{pgfscope}%
\begin{pgfscope}%
\pgfpathrectangle{\pgfqpoint{3.985294in}{1.750000in}}{\pgfqpoint{2.279412in}{2.004545in}}%
\pgfusepath{clip}%
\pgfsetroundcap%
\pgfsetroundjoin%
\definecolor{currentfill}{rgb}{0.179019,0.433756,0.557430}%
\pgfsetfillcolor{currentfill}%
\pgfsetlinewidth{0.935027pt}%
\definecolor{currentstroke}{rgb}{0.179019,0.433756,0.557430}%
\pgfsetstrokecolor{currentstroke}%
\pgfsetdash{}{0pt}%
\pgfpathmoveto{\pgfqpoint{5.339572in}{2.766383in}}%
\pgfpathlineto{\pgfqpoint{5.283360in}{2.792808in}}%
\pgfpathlineto{\pgfqpoint{5.338226in}{2.821922in}}%
\pgfpathlineto{\pgfqpoint{5.339572in}{2.766383in}}%
\pgfpathlineto{\pgfqpoint{5.339572in}{2.766383in}}%
\pgfpathclose%
\pgfusepath{stroke,fill}%
\end{pgfscope}%
\begin{pgfscope}%
\pgfpathrectangle{\pgfqpoint{3.985294in}{1.750000in}}{\pgfqpoint{2.279412in}{2.004545in}}%
\pgfusepath{clip}%
\pgfsetroundcap%
\pgfsetroundjoin%
\pgfsetlinewidth{0.498873pt}%
\definecolor{currentstroke}{rgb}{0.281412,0.155834,0.469201}%
\pgfsetstrokecolor{currentstroke}%
\pgfsetdash{}{0pt}%
\pgfpathmoveto{\pgfqpoint{5.526876in}{2.842356in}}%
\pgfpathquadraticcurveto{\pgfqpoint{5.514340in}{2.842191in}}{\pgfqpoint{5.509520in}{2.842128in}}%
\pgfusepath{stroke}%
\end{pgfscope}%
\begin{pgfscope}%
\pgfpathrectangle{\pgfqpoint{3.985294in}{1.750000in}}{\pgfqpoint{2.279412in}{2.004545in}}%
\pgfusepath{clip}%
\pgfsetroundcap%
\pgfsetroundjoin%
\definecolor{currentfill}{rgb}{0.281412,0.155834,0.469201}%
\pgfsetfillcolor{currentfill}%
\pgfsetlinewidth{0.498873pt}%
\definecolor{currentstroke}{rgb}{0.281412,0.155834,0.469201}%
\pgfsetstrokecolor{currentstroke}%
\pgfsetdash{}{0pt}%
\pgfpathmoveto{\pgfqpoint{5.565437in}{2.815083in}}%
\pgfpathlineto{\pgfqpoint{5.509520in}{2.842128in}}%
\pgfpathlineto{\pgfqpoint{5.564706in}{2.870634in}}%
\pgfpathlineto{\pgfqpoint{5.565437in}{2.815083in}}%
\pgfpathlineto{\pgfqpoint{5.565437in}{2.815083in}}%
\pgfpathclose%
\pgfusepath{stroke,fill}%
\end{pgfscope}%
\begin{pgfscope}%
\pgfpathrectangle{\pgfqpoint{3.985294in}{1.750000in}}{\pgfqpoint{2.279412in}{2.004545in}}%
\pgfusepath{clip}%
\pgfsetroundcap%
\pgfsetroundjoin%
\pgfsetlinewidth{0.746195pt}%
\definecolor{currentstroke}{rgb}{0.229739,0.322361,0.545706}%
\pgfsetstrokecolor{currentstroke}%
\pgfsetdash{}{0pt}%
\pgfpathmoveto{\pgfqpoint{5.300121in}{2.924508in}}%
\pgfpathquadraticcurveto{\pgfqpoint{5.287625in}{2.923619in}}{\pgfqpoint{5.286644in}{2.923549in}}%
\pgfusepath{stroke}%
\end{pgfscope}%
\begin{pgfscope}%
\pgfpathrectangle{\pgfqpoint{3.985294in}{1.750000in}}{\pgfqpoint{2.279412in}{2.004545in}}%
\pgfusepath{clip}%
\pgfsetroundcap%
\pgfsetroundjoin%
\definecolor{currentfill}{rgb}{0.229739,0.322361,0.545706}%
\pgfsetfillcolor{currentfill}%
\pgfsetlinewidth{0.746195pt}%
\definecolor{currentstroke}{rgb}{0.229739,0.322361,0.545706}%
\pgfsetstrokecolor{currentstroke}%
\pgfsetdash{}{0pt}%
\pgfpathmoveto{\pgfqpoint{5.344032in}{2.899787in}}%
\pgfpathlineto{\pgfqpoint{5.286644in}{2.923549in}}%
\pgfpathlineto{\pgfqpoint{5.340087in}{2.955202in}}%
\pgfpathlineto{\pgfqpoint{5.344032in}{2.899787in}}%
\pgfpathlineto{\pgfqpoint{5.344032in}{2.899787in}}%
\pgfpathclose%
\pgfusepath{stroke,fill}%
\end{pgfscope}%
\begin{pgfscope}%
\pgfpathrectangle{\pgfqpoint{3.985294in}{1.750000in}}{\pgfqpoint{2.279412in}{2.004545in}}%
\pgfusepath{clip}%
\pgfsetroundcap%
\pgfsetroundjoin%
\pgfsetlinewidth{0.331335pt}%
\definecolor{currentstroke}{rgb}{0.272594,0.025563,0.353093}%
\pgfsetstrokecolor{currentstroke}%
\pgfsetdash{}{0pt}%
\pgfpathmoveto{\pgfqpoint{5.322332in}{2.048726in}}%
\pgfpathquadraticcurveto{\pgfqpoint{5.309914in}{2.049858in}}{\pgfqpoint{5.302601in}{2.050525in}}%
\pgfusepath{stroke}%
\end{pgfscope}%
\begin{pgfscope}%
\pgfpathrectangle{\pgfqpoint{3.985294in}{1.750000in}}{\pgfqpoint{2.279412in}{2.004545in}}%
\pgfusepath{clip}%
\pgfsetroundcap%
\pgfsetroundjoin%
\definecolor{currentfill}{rgb}{0.272594,0.025563,0.353093}%
\pgfsetfillcolor{currentfill}%
\pgfsetlinewidth{0.331335pt}%
\definecolor{currentstroke}{rgb}{0.272594,0.025563,0.353093}%
\pgfsetstrokecolor{currentstroke}%
\pgfsetdash{}{0pt}%
\pgfpathmoveto{\pgfqpoint{5.355405in}{2.017818in}}%
\pgfpathlineto{\pgfqpoint{5.302601in}{2.050525in}}%
\pgfpathlineto{\pgfqpoint{5.360449in}{2.073144in}}%
\pgfpathlineto{\pgfqpoint{5.355405in}{2.017818in}}%
\pgfpathlineto{\pgfqpoint{5.355405in}{2.017818in}}%
\pgfpathclose%
\pgfusepath{stroke,fill}%
\end{pgfscope}%
\begin{pgfscope}%
\pgfpathrectangle{\pgfqpoint{3.985294in}{1.750000in}}{\pgfqpoint{2.279412in}{2.004545in}}%
\pgfusepath{clip}%
\pgfsetroundcap%
\pgfsetroundjoin%
\pgfsetlinewidth{0.550522pt}%
\definecolor{currentstroke}{rgb}{0.275191,0.194905,0.496005}%
\pgfsetstrokecolor{currentstroke}%
\pgfsetdash{}{0pt}%
\pgfpathmoveto{\pgfqpoint{5.056514in}{2.393137in}}%
\pgfpathquadraticcurveto{\pgfqpoint{5.047015in}{2.399248in}}{\pgfqpoint{5.044679in}{2.400751in}}%
\pgfusepath{stroke}%
\end{pgfscope}%
\begin{pgfscope}%
\pgfpathrectangle{\pgfqpoint{3.985294in}{1.750000in}}{\pgfqpoint{2.279412in}{2.004545in}}%
\pgfusepath{clip}%
\pgfsetroundcap%
\pgfsetroundjoin%
\definecolor{currentfill}{rgb}{0.275191,0.194905,0.496005}%
\pgfsetfillcolor{currentfill}%
\pgfsetlinewidth{0.550522pt}%
\definecolor{currentstroke}{rgb}{0.275191,0.194905,0.496005}%
\pgfsetstrokecolor{currentstroke}%
\pgfsetdash{}{0pt}%
\pgfpathmoveto{\pgfqpoint{5.076372in}{2.347332in}}%
\pgfpathlineto{\pgfqpoint{5.044679in}{2.400751in}}%
\pgfpathlineto{\pgfqpoint{5.106430in}{2.394055in}}%
\pgfpathlineto{\pgfqpoint{5.076372in}{2.347332in}}%
\pgfpathlineto{\pgfqpoint{5.076372in}{2.347332in}}%
\pgfpathclose%
\pgfusepath{stroke,fill}%
\end{pgfscope}%
\begin{pgfscope}%
\pgfpathrectangle{\pgfqpoint{3.985294in}{1.750000in}}{\pgfqpoint{2.279412in}{2.004545in}}%
\pgfusepath{clip}%
\pgfsetroundcap%
\pgfsetroundjoin%
\pgfsetlinewidth{0.380754pt}%
\definecolor{currentstroke}{rgb}{0.279566,0.067836,0.391917}%
\pgfsetstrokecolor{currentstroke}%
\pgfsetdash{}{0pt}%
\pgfpathmoveto{\pgfqpoint{5.543776in}{2.353611in}}%
\pgfpathquadraticcurveto{\pgfqpoint{5.531257in}{2.354205in}}{\pgfqpoint{5.524623in}{2.354520in}}%
\pgfusepath{stroke}%
\end{pgfscope}%
\begin{pgfscope}%
\pgfpathrectangle{\pgfqpoint{3.985294in}{1.750000in}}{\pgfqpoint{2.279412in}{2.004545in}}%
\pgfusepath{clip}%
\pgfsetroundcap%
\pgfsetroundjoin%
\definecolor{currentfill}{rgb}{0.279566,0.067836,0.391917}%
\pgfsetfillcolor{currentfill}%
\pgfsetlinewidth{0.380754pt}%
\definecolor{currentstroke}{rgb}{0.279566,0.067836,0.391917}%
\pgfsetstrokecolor{currentstroke}%
\pgfsetdash{}{0pt}%
\pgfpathmoveto{\pgfqpoint{5.578799in}{2.324140in}}%
\pgfpathlineto{\pgfqpoint{5.524623in}{2.354520in}}%
\pgfpathlineto{\pgfqpoint{5.581433in}{2.379633in}}%
\pgfpathlineto{\pgfqpoint{5.578799in}{2.324140in}}%
\pgfpathlineto{\pgfqpoint{5.578799in}{2.324140in}}%
\pgfpathclose%
\pgfusepath{stroke,fill}%
\end{pgfscope}%
\begin{pgfscope}%
\pgfpathrectangle{\pgfqpoint{3.985294in}{1.750000in}}{\pgfqpoint{2.279412in}{2.004545in}}%
\pgfusepath{clip}%
\pgfsetroundcap%
\pgfsetroundjoin%
\pgfsetlinewidth{0.488200pt}%
\definecolor{currentstroke}{rgb}{0.281887,0.150881,0.465405}%
\pgfsetstrokecolor{currentstroke}%
\pgfsetdash{}{0pt}%
\pgfpathmoveto{\pgfqpoint{5.394017in}{2.410836in}}%
\pgfpathquadraticcurveto{\pgfqpoint{5.381548in}{2.411966in}}{\pgfqpoint{5.376601in}{2.412414in}}%
\pgfusepath{stroke}%
\end{pgfscope}%
\begin{pgfscope}%
\pgfpathrectangle{\pgfqpoint{3.985294in}{1.750000in}}{\pgfqpoint{2.279412in}{2.004545in}}%
\pgfusepath{clip}%
\pgfsetroundcap%
\pgfsetroundjoin%
\definecolor{currentfill}{rgb}{0.281887,0.150881,0.465405}%
\pgfsetfillcolor{currentfill}%
\pgfsetlinewidth{0.488200pt}%
\definecolor{currentstroke}{rgb}{0.281887,0.150881,0.465405}%
\pgfsetstrokecolor{currentstroke}%
\pgfsetdash{}{0pt}%
\pgfpathmoveto{\pgfqpoint{5.429424in}{2.379737in}}%
\pgfpathlineto{\pgfqpoint{5.376601in}{2.412414in}}%
\pgfpathlineto{\pgfqpoint{5.434436in}{2.435066in}}%
\pgfpathlineto{\pgfqpoint{5.429424in}{2.379737in}}%
\pgfpathlineto{\pgfqpoint{5.429424in}{2.379737in}}%
\pgfpathclose%
\pgfusepath{stroke,fill}%
\end{pgfscope}%
\begin{pgfscope}%
\pgfpathrectangle{\pgfqpoint{3.985294in}{1.750000in}}{\pgfqpoint{2.279412in}{2.004545in}}%
\pgfusepath{clip}%
\pgfsetroundcap%
\pgfsetroundjoin%
\pgfsetlinewidth{0.840385pt}%
\definecolor{currentstroke}{rgb}{0.203063,0.379716,0.553925}%
\pgfsetstrokecolor{currentstroke}%
\pgfsetdash{}{0pt}%
\pgfpathmoveto{\pgfqpoint{5.294802in}{2.589195in}}%
\pgfpathquadraticcurveto{\pgfqpoint{5.282301in}{2.590037in}}{\pgfqpoint{5.282771in}{2.590005in}}%
\pgfusepath{stroke}%
\end{pgfscope}%
\begin{pgfscope}%
\pgfpathrectangle{\pgfqpoint{3.985294in}{1.750000in}}{\pgfqpoint{2.279412in}{2.004545in}}%
\pgfusepath{clip}%
\pgfsetroundcap%
\pgfsetroundjoin%
\definecolor{currentfill}{rgb}{0.203063,0.379716,0.553925}%
\pgfsetfillcolor{currentfill}%
\pgfsetlinewidth{0.840385pt}%
\definecolor{currentstroke}{rgb}{0.203063,0.379716,0.553925}%
\pgfsetstrokecolor{currentstroke}%
\pgfsetdash{}{0pt}%
\pgfpathmoveto{\pgfqpoint{5.336335in}{2.558558in}}%
\pgfpathlineto{\pgfqpoint{5.282771in}{2.590005in}}%
\pgfpathlineto{\pgfqpoint{5.340068in}{2.613988in}}%
\pgfpathlineto{\pgfqpoint{5.336335in}{2.558558in}}%
\pgfpathlineto{\pgfqpoint{5.336335in}{2.558558in}}%
\pgfpathclose%
\pgfusepath{stroke,fill}%
\end{pgfscope}%
\begin{pgfscope}%
\pgfpathrectangle{\pgfqpoint{3.985294in}{1.750000in}}{\pgfqpoint{2.279412in}{2.004545in}}%
\pgfusepath{clip}%
\pgfsetroundcap%
\pgfsetroundjoin%
\pgfsetlinewidth{0.424067pt}%
\definecolor{currentstroke}{rgb}{0.282656,0.100196,0.422160}%
\pgfsetstrokecolor{currentstroke}%
\pgfsetdash{}{0pt}%
\pgfpathmoveto{\pgfqpoint{5.593571in}{2.617655in}}%
\pgfpathquadraticcurveto{\pgfqpoint{5.581035in}{2.617814in}}{\pgfqpoint{5.575059in}{2.617891in}}%
\pgfusepath{stroke}%
\end{pgfscope}%
\begin{pgfscope}%
\pgfpathrectangle{\pgfqpoint{3.985294in}{1.750000in}}{\pgfqpoint{2.279412in}{2.004545in}}%
\pgfusepath{clip}%
\pgfsetroundcap%
\pgfsetroundjoin%
\definecolor{currentfill}{rgb}{0.282656,0.100196,0.422160}%
\pgfsetfillcolor{currentfill}%
\pgfsetlinewidth{0.424067pt}%
\definecolor{currentstroke}{rgb}{0.282656,0.100196,0.422160}%
\pgfsetstrokecolor{currentstroke}%
\pgfsetdash{}{0pt}%
\pgfpathmoveto{\pgfqpoint{5.630256in}{2.589407in}}%
\pgfpathlineto{\pgfqpoint{5.575059in}{2.617891in}}%
\pgfpathlineto{\pgfqpoint{5.630964in}{2.644958in}}%
\pgfpathlineto{\pgfqpoint{5.630256in}{2.589407in}}%
\pgfpathlineto{\pgfqpoint{5.630256in}{2.589407in}}%
\pgfpathclose%
\pgfusepath{stroke,fill}%
\end{pgfscope}%
\begin{pgfscope}%
\pgfpathrectangle{\pgfqpoint{3.985294in}{1.750000in}}{\pgfqpoint{2.279412in}{2.004545in}}%
\pgfusepath{clip}%
\pgfsetroundcap%
\pgfsetroundjoin%
\pgfsetlinewidth{0.885654pt}%
\definecolor{currentstroke}{rgb}{0.190631,0.407061,0.556089}%
\pgfsetstrokecolor{currentstroke}%
\pgfsetdash{}{0pt}%
\pgfpathmoveto{\pgfqpoint{5.344859in}{2.706719in}}%
\pgfpathquadraticcurveto{\pgfqpoint{5.332321in}{2.706767in}}{\pgfqpoint{5.333485in}{2.706762in}}%
\pgfusepath{stroke}%
\end{pgfscope}%
\begin{pgfscope}%
\pgfpathrectangle{\pgfqpoint{3.985294in}{1.750000in}}{\pgfqpoint{2.279412in}{2.004545in}}%
\pgfusepath{clip}%
\pgfsetroundcap%
\pgfsetroundjoin%
\definecolor{currentfill}{rgb}{0.190631,0.407061,0.556089}%
\pgfsetfillcolor{currentfill}%
\pgfsetlinewidth{0.885654pt}%
\definecolor{currentstroke}{rgb}{0.190631,0.407061,0.556089}%
\pgfsetstrokecolor{currentstroke}%
\pgfsetdash{}{0pt}%
\pgfpathmoveto{\pgfqpoint{5.388935in}{2.678774in}}%
\pgfpathlineto{\pgfqpoint{5.333485in}{2.706762in}}%
\pgfpathlineto{\pgfqpoint{5.389145in}{2.734329in}}%
\pgfpathlineto{\pgfqpoint{5.388935in}{2.678774in}}%
\pgfpathlineto{\pgfqpoint{5.388935in}{2.678774in}}%
\pgfpathclose%
\pgfusepath{stroke,fill}%
\end{pgfscope}%
\begin{pgfscope}%
\pgfpathrectangle{\pgfqpoint{3.985294in}{1.750000in}}{\pgfqpoint{2.279412in}{2.004545in}}%
\pgfusepath{clip}%
\pgfsetroundcap%
\pgfsetroundjoin%
\pgfsetlinewidth{0.658390pt}%
\definecolor{currentstroke}{rgb}{0.252194,0.269783,0.531579}%
\pgfsetstrokecolor{currentstroke}%
\pgfsetdash{}{0pt}%
\pgfpathmoveto{\pgfqpoint{5.443051in}{2.750661in}}%
\pgfpathquadraticcurveto{\pgfqpoint{5.430513in}{2.750616in}}{\pgfqpoint{5.428160in}{2.750607in}}%
\pgfusepath{stroke}%
\end{pgfscope}%
\begin{pgfscope}%
\pgfpathrectangle{\pgfqpoint{3.985294in}{1.750000in}}{\pgfqpoint{2.279412in}{2.004545in}}%
\pgfusepath{clip}%
\pgfsetroundcap%
\pgfsetroundjoin%
\definecolor{currentfill}{rgb}{0.252194,0.269783,0.531579}%
\pgfsetfillcolor{currentfill}%
\pgfsetlinewidth{0.658390pt}%
\definecolor{currentstroke}{rgb}{0.252194,0.269783,0.531579}%
\pgfsetstrokecolor{currentstroke}%
\pgfsetdash{}{0pt}%
\pgfpathmoveto{\pgfqpoint{5.483816in}{2.723030in}}%
\pgfpathlineto{\pgfqpoint{5.428160in}{2.750607in}}%
\pgfpathlineto{\pgfqpoint{5.483615in}{2.778585in}}%
\pgfpathlineto{\pgfqpoint{5.483816in}{2.723030in}}%
\pgfpathlineto{\pgfqpoint{5.483816in}{2.723030in}}%
\pgfpathclose%
\pgfusepath{stroke,fill}%
\end{pgfscope}%
\begin{pgfscope}%
\pgfpathrectangle{\pgfqpoint{3.985294in}{1.750000in}}{\pgfqpoint{2.279412in}{2.004545in}}%
\pgfusepath{clip}%
\pgfsetroundcap%
\pgfsetroundjoin%
\pgfsetlinewidth{0.539557pt}%
\definecolor{currentstroke}{rgb}{0.277134,0.185228,0.489898}%
\pgfsetstrokecolor{currentstroke}%
\pgfsetdash{}{0pt}%
\pgfpathmoveto{\pgfqpoint{5.294152in}{3.042877in}}%
\pgfpathquadraticcurveto{\pgfqpoint{5.281776in}{3.041127in}}{\pgfqpoint{5.277664in}{3.040546in}}%
\pgfusepath{stroke}%
\end{pgfscope}%
\begin{pgfscope}%
\pgfpathrectangle{\pgfqpoint{3.985294in}{1.750000in}}{\pgfqpoint{2.279412in}{2.004545in}}%
\pgfusepath{clip}%
\pgfsetroundcap%
\pgfsetroundjoin%
\definecolor{currentfill}{rgb}{0.277134,0.185228,0.489898}%
\pgfsetfillcolor{currentfill}%
\pgfsetlinewidth{0.539557pt}%
\definecolor{currentstroke}{rgb}{0.277134,0.185228,0.489898}%
\pgfsetstrokecolor{currentstroke}%
\pgfsetdash{}{0pt}%
\pgfpathmoveto{\pgfqpoint{5.336560in}{3.020816in}}%
\pgfpathlineto{\pgfqpoint{5.277664in}{3.040546in}}%
\pgfpathlineto{\pgfqpoint{5.328786in}{3.075825in}}%
\pgfpathlineto{\pgfqpoint{5.336560in}{3.020816in}}%
\pgfpathlineto{\pgfqpoint{5.336560in}{3.020816in}}%
\pgfpathclose%
\pgfusepath{stroke,fill}%
\end{pgfscope}%
\begin{pgfscope}%
\pgfpathrectangle{\pgfqpoint{3.985294in}{1.750000in}}{\pgfqpoint{2.279412in}{2.004545in}}%
\pgfusepath{clip}%
\pgfsetroundcap%
\pgfsetroundjoin%
\pgfsetlinewidth{0.391256pt}%
\definecolor{currentstroke}{rgb}{0.280894,0.078907,0.402329}%
\pgfsetstrokecolor{currentstroke}%
\pgfsetdash{}{0pt}%
\pgfpathmoveto{\pgfqpoint{5.543859in}{3.108163in}}%
\pgfpathquadraticcurveto{\pgfqpoint{5.531343in}{3.107517in}}{\pgfqpoint{5.524873in}{3.107184in}}%
\pgfusepath{stroke}%
\end{pgfscope}%
\begin{pgfscope}%
\pgfpathrectangle{\pgfqpoint{3.985294in}{1.750000in}}{\pgfqpoint{2.279412in}{2.004545in}}%
\pgfusepath{clip}%
\pgfsetroundcap%
\pgfsetroundjoin%
\definecolor{currentfill}{rgb}{0.280894,0.078907,0.402329}%
\pgfsetfillcolor{currentfill}%
\pgfsetlinewidth{0.391256pt}%
\definecolor{currentstroke}{rgb}{0.280894,0.078907,0.402329}%
\pgfsetstrokecolor{currentstroke}%
\pgfsetdash{}{0pt}%
\pgfpathmoveto{\pgfqpoint{5.581785in}{3.082304in}}%
\pgfpathlineto{\pgfqpoint{5.524873in}{3.107184in}}%
\pgfpathlineto{\pgfqpoint{5.578924in}{3.137786in}}%
\pgfpathlineto{\pgfqpoint{5.581785in}{3.082304in}}%
\pgfpathlineto{\pgfqpoint{5.581785in}{3.082304in}}%
\pgfpathclose%
\pgfusepath{stroke,fill}%
\end{pgfscope}%
\begin{pgfscope}%
\pgfpathrectangle{\pgfqpoint{3.985294in}{1.750000in}}{\pgfqpoint{2.279412in}{2.004545in}}%
\pgfusepath{clip}%
\pgfsetroundcap%
\pgfsetroundjoin%
\pgfsetlinewidth{0.314650pt}%
\definecolor{currentstroke}{rgb}{0.268510,0.009605,0.335427}%
\pgfsetstrokecolor{currentstroke}%
\pgfsetdash{}{0pt}%
\pgfpathmoveto{\pgfqpoint{5.743331in}{3.425848in}}%
\pgfpathquadraticcurveto{\pgfqpoint{5.730951in}{3.424521in}}{\pgfqpoint{5.723411in}{3.423714in}}%
\pgfusepath{stroke}%
\end{pgfscope}%
\begin{pgfscope}%
\pgfpathrectangle{\pgfqpoint{3.985294in}{1.750000in}}{\pgfqpoint{2.279412in}{2.004545in}}%
\pgfusepath{clip}%
\pgfsetroundcap%
\pgfsetroundjoin%
\definecolor{currentfill}{rgb}{0.268510,0.009605,0.335427}%
\pgfsetfillcolor{currentfill}%
\pgfsetlinewidth{0.314650pt}%
\definecolor{currentstroke}{rgb}{0.268510,0.009605,0.335427}%
\pgfsetstrokecolor{currentstroke}%
\pgfsetdash{}{0pt}%
\pgfpathmoveto{\pgfqpoint{5.781609in}{3.402011in}}%
\pgfpathlineto{\pgfqpoint{5.723411in}{3.423714in}}%
\pgfpathlineto{\pgfqpoint{5.775692in}{3.457250in}}%
\pgfpathlineto{\pgfqpoint{5.781609in}{3.402011in}}%
\pgfpathlineto{\pgfqpoint{5.781609in}{3.402011in}}%
\pgfpathclose%
\pgfusepath{stroke,fill}%
\end{pgfscope}%
\begin{pgfscope}%
\pgfpathrectangle{\pgfqpoint{3.985294in}{1.750000in}}{\pgfqpoint{2.279412in}{2.004545in}}%
\pgfusepath{clip}%
\pgfsetroundcap%
\pgfsetroundjoin%
\pgfsetlinewidth{0.356956pt}%
\definecolor{currentstroke}{rgb}{0.277018,0.050344,0.375715}%
\pgfsetstrokecolor{currentstroke}%
\pgfsetdash{}{0pt}%
\pgfpathmoveto{\pgfqpoint{5.646539in}{2.434918in}}%
\pgfpathquadraticcurveto{\pgfqpoint{5.634006in}{2.435073in}}{\pgfqpoint{5.626995in}{2.435160in}}%
\pgfusepath{stroke}%
\end{pgfscope}%
\begin{pgfscope}%
\pgfpathrectangle{\pgfqpoint{3.985294in}{1.750000in}}{\pgfqpoint{2.279412in}{2.004545in}}%
\pgfusepath{clip}%
\pgfsetroundcap%
\pgfsetroundjoin%
\definecolor{currentfill}{rgb}{0.277018,0.050344,0.375715}%
\pgfsetfillcolor{currentfill}%
\pgfsetlinewidth{0.356956pt}%
\definecolor{currentstroke}{rgb}{0.277018,0.050344,0.375715}%
\pgfsetstrokecolor{currentstroke}%
\pgfsetdash{}{0pt}%
\pgfpathmoveto{\pgfqpoint{5.682202in}{2.406696in}}%
\pgfpathlineto{\pgfqpoint{5.626995in}{2.435160in}}%
\pgfpathlineto{\pgfqpoint{5.682891in}{2.462247in}}%
\pgfpathlineto{\pgfqpoint{5.682202in}{2.406696in}}%
\pgfpathlineto{\pgfqpoint{5.682202in}{2.406696in}}%
\pgfpathclose%
\pgfusepath{stroke,fill}%
\end{pgfscope}%
\begin{pgfscope}%
\pgfpathrectangle{\pgfqpoint{3.985294in}{1.750000in}}{\pgfqpoint{2.279412in}{2.004545in}}%
\pgfusepath{clip}%
\pgfsetroundcap%
\pgfsetroundjoin%
\pgfsetlinewidth{0.642925pt}%
\definecolor{currentstroke}{rgb}{0.257322,0.256130,0.526563}%
\pgfsetstrokecolor{currentstroke}%
\pgfsetdash{}{0pt}%
\pgfpathmoveto{\pgfqpoint{5.342087in}{2.496580in}}%
\pgfpathquadraticcurveto{\pgfqpoint{5.329617in}{2.497715in}}{\pgfqpoint{5.327052in}{2.497949in}}%
\pgfusepath{stroke}%
\end{pgfscope}%
\begin{pgfscope}%
\pgfpathrectangle{\pgfqpoint{3.985294in}{1.750000in}}{\pgfqpoint{2.279412in}{2.004545in}}%
\pgfusepath{clip}%
\pgfsetroundcap%
\pgfsetroundjoin%
\definecolor{currentfill}{rgb}{0.257322,0.256130,0.526563}%
\pgfsetfillcolor{currentfill}%
\pgfsetlinewidth{0.642925pt}%
\definecolor{currentstroke}{rgb}{0.257322,0.256130,0.526563}%
\pgfsetstrokecolor{currentstroke}%
\pgfsetdash{}{0pt}%
\pgfpathmoveto{\pgfqpoint{5.379860in}{2.465248in}}%
\pgfpathlineto{\pgfqpoint{5.327052in}{2.497949in}}%
\pgfpathlineto{\pgfqpoint{5.384897in}{2.520575in}}%
\pgfpathlineto{\pgfqpoint{5.379860in}{2.465248in}}%
\pgfpathlineto{\pgfqpoint{5.379860in}{2.465248in}}%
\pgfpathclose%
\pgfusepath{stroke,fill}%
\end{pgfscope}%
\begin{pgfscope}%
\pgfpathrectangle{\pgfqpoint{3.985294in}{1.750000in}}{\pgfqpoint{2.279412in}{2.004545in}}%
\pgfusepath{clip}%
\pgfsetroundcap%
\pgfsetroundjoin%
\pgfsetlinewidth{0.675207pt}%
\definecolor{currentstroke}{rgb}{0.248629,0.278775,0.534556}%
\pgfsetstrokecolor{currentstroke}%
\pgfsetdash{}{0pt}%
\pgfpathmoveto{\pgfqpoint{5.441966in}{2.663778in}}%
\pgfpathquadraticcurveto{\pgfqpoint{5.429429in}{2.663895in}}{\pgfqpoint{5.427337in}{2.663915in}}%
\pgfusepath{stroke}%
\end{pgfscope}%
\begin{pgfscope}%
\pgfpathrectangle{\pgfqpoint{3.985294in}{1.750000in}}{\pgfqpoint{2.279412in}{2.004545in}}%
\pgfusepath{clip}%
\pgfsetroundcap%
\pgfsetroundjoin%
\definecolor{currentfill}{rgb}{0.248629,0.278775,0.534556}%
\pgfsetfillcolor{currentfill}%
\pgfsetlinewidth{0.675207pt}%
\definecolor{currentstroke}{rgb}{0.248629,0.278775,0.534556}%
\pgfsetstrokecolor{currentstroke}%
\pgfsetdash{}{0pt}%
\pgfpathmoveto{\pgfqpoint{5.482630in}{2.635618in}}%
\pgfpathlineto{\pgfqpoint{5.427337in}{2.663915in}}%
\pgfpathlineto{\pgfqpoint{5.483150in}{2.691171in}}%
\pgfpathlineto{\pgfqpoint{5.482630in}{2.635618in}}%
\pgfpathlineto{\pgfqpoint{5.482630in}{2.635618in}}%
\pgfpathclose%
\pgfusepath{stroke,fill}%
\end{pgfscope}%
\begin{pgfscope}%
\pgfpathrectangle{\pgfqpoint{3.985294in}{1.750000in}}{\pgfqpoint{2.279412in}{2.004545in}}%
\pgfusepath{clip}%
\pgfsetroundcap%
\pgfsetroundjoin%
\pgfsetlinewidth{0.464085pt}%
\definecolor{currentstroke}{rgb}{0.283072,0.130895,0.449241}%
\pgfsetstrokecolor{currentstroke}%
\pgfsetdash{}{0pt}%
\pgfpathmoveto{\pgfqpoint{5.542235in}{2.887284in}}%
\pgfpathquadraticcurveto{\pgfqpoint{5.529700in}{2.887074in}}{\pgfqpoint{5.524343in}{2.886985in}}%
\pgfusepath{stroke}%
\end{pgfscope}%
\begin{pgfscope}%
\pgfpathrectangle{\pgfqpoint{3.985294in}{1.750000in}}{\pgfqpoint{2.279412in}{2.004545in}}%
\pgfusepath{clip}%
\pgfsetroundcap%
\pgfsetroundjoin%
\definecolor{currentfill}{rgb}{0.283072,0.130895,0.449241}%
\pgfsetfillcolor{currentfill}%
\pgfsetlinewidth{0.464085pt}%
\definecolor{currentstroke}{rgb}{0.283072,0.130895,0.449241}%
\pgfsetstrokecolor{currentstroke}%
\pgfsetdash{}{0pt}%
\pgfpathmoveto{\pgfqpoint{5.580355in}{2.860140in}}%
\pgfpathlineto{\pgfqpoint{5.524343in}{2.886985in}}%
\pgfpathlineto{\pgfqpoint{5.579426in}{2.915688in}}%
\pgfpathlineto{\pgfqpoint{5.580355in}{2.860140in}}%
\pgfpathlineto{\pgfqpoint{5.580355in}{2.860140in}}%
\pgfpathclose%
\pgfusepath{stroke,fill}%
\end{pgfscope}%
\begin{pgfscope}%
\pgfpathrectangle{\pgfqpoint{3.985294in}{1.750000in}}{\pgfqpoint{2.279412in}{2.004545in}}%
\pgfusepath{clip}%
\pgfsetroundcap%
\pgfsetroundjoin%
\pgfsetlinewidth{0.391561pt}%
\definecolor{currentstroke}{rgb}{0.280894,0.078907,0.402329}%
\pgfsetstrokecolor{currentstroke}%
\pgfsetdash{}{0pt}%
\pgfpathmoveto{\pgfqpoint{5.592421in}{2.974043in}}%
\pgfpathquadraticcurveto{\pgfqpoint{5.579887in}{2.973751in}}{\pgfqpoint{5.573410in}{2.973600in}}%
\pgfusepath{stroke}%
\end{pgfscope}%
\begin{pgfscope}%
\pgfpathrectangle{\pgfqpoint{3.985294in}{1.750000in}}{\pgfqpoint{2.279412in}{2.004545in}}%
\pgfusepath{clip}%
\pgfsetroundcap%
\pgfsetroundjoin%
\definecolor{currentfill}{rgb}{0.280894,0.078907,0.402329}%
\pgfsetfillcolor{currentfill}%
\pgfsetlinewidth{0.391561pt}%
\definecolor{currentstroke}{rgb}{0.280894,0.078907,0.402329}%
\pgfsetstrokecolor{currentstroke}%
\pgfsetdash{}{0pt}%
\pgfpathmoveto{\pgfqpoint{5.629598in}{2.947125in}}%
\pgfpathlineto{\pgfqpoint{5.573410in}{2.973600in}}%
\pgfpathlineto{\pgfqpoint{5.628302in}{3.002666in}}%
\pgfpathlineto{\pgfqpoint{5.629598in}{2.947125in}}%
\pgfpathlineto{\pgfqpoint{5.629598in}{2.947125in}}%
\pgfpathclose%
\pgfusepath{stroke,fill}%
\end{pgfscope}%
\begin{pgfscope}%
\pgfpathrectangle{\pgfqpoint{3.985294in}{1.750000in}}{\pgfqpoint{2.279412in}{2.004545in}}%
\pgfusepath{clip}%
\pgfsetroundcap%
\pgfsetroundjoin%
\pgfsetlinewidth{0.387876pt}%
\definecolor{currentstroke}{rgb}{0.280267,0.073417,0.397163}%
\pgfsetstrokecolor{currentstroke}%
\pgfsetdash{}{0pt}%
\pgfpathmoveto{\pgfqpoint{5.592552in}{3.016668in}}%
\pgfpathquadraticcurveto{\pgfqpoint{5.580024in}{3.016251in}}{\pgfqpoint{5.573492in}{3.016033in}}%
\pgfusepath{stroke}%
\end{pgfscope}%
\begin{pgfscope}%
\pgfpathrectangle{\pgfqpoint{3.985294in}{1.750000in}}{\pgfqpoint{2.279412in}{2.004545in}}%
\pgfusepath{clip}%
\pgfsetroundcap%
\pgfsetroundjoin%
\definecolor{currentfill}{rgb}{0.280267,0.073417,0.397163}%
\pgfsetfillcolor{currentfill}%
\pgfsetlinewidth{0.387876pt}%
\definecolor{currentstroke}{rgb}{0.280267,0.073417,0.397163}%
\pgfsetstrokecolor{currentstroke}%
\pgfsetdash{}{0pt}%
\pgfpathmoveto{\pgfqpoint{5.629942in}{2.990120in}}%
\pgfpathlineto{\pgfqpoint{5.573492in}{3.016033in}}%
\pgfpathlineto{\pgfqpoint{5.628092in}{3.045645in}}%
\pgfpathlineto{\pgfqpoint{5.629942in}{2.990120in}}%
\pgfpathlineto{\pgfqpoint{5.629942in}{2.990120in}}%
\pgfpathclose%
\pgfusepath{stroke,fill}%
\end{pgfscope}%
\begin{pgfscope}%
\pgfpathrectangle{\pgfqpoint{3.985294in}{1.750000in}}{\pgfqpoint{2.279412in}{2.004545in}}%
\pgfusepath{clip}%
\pgfsetroundcap%
\pgfsetroundjoin%
\pgfsetlinewidth{0.399984pt}%
\definecolor{currentstroke}{rgb}{0.281446,0.084320,0.407414}%
\pgfsetstrokecolor{currentstroke}%
\pgfsetdash{}{0pt}%
\pgfpathmoveto{\pgfqpoint{5.184457in}{3.145253in}}%
\pgfpathquadraticcurveto{\pgfqpoint{5.172736in}{3.141393in}}{\pgfqpoint{5.166893in}{3.139468in}}%
\pgfusepath{stroke}%
\end{pgfscope}%
\begin{pgfscope}%
\pgfpathrectangle{\pgfqpoint{3.985294in}{1.750000in}}{\pgfqpoint{2.279412in}{2.004545in}}%
\pgfusepath{clip}%
\pgfsetroundcap%
\pgfsetroundjoin%
\definecolor{currentfill}{rgb}{0.281446,0.084320,0.407414}%
\pgfsetfillcolor{currentfill}%
\pgfsetlinewidth{0.399984pt}%
\definecolor{currentstroke}{rgb}{0.281446,0.084320,0.407414}%
\pgfsetstrokecolor{currentstroke}%
\pgfsetdash{}{0pt}%
\pgfpathmoveto{\pgfqpoint{5.228350in}{3.130464in}}%
\pgfpathlineto{\pgfqpoint{5.166893in}{3.139468in}}%
\pgfpathlineto{\pgfqpoint{5.210970in}{3.183231in}}%
\pgfpathlineto{\pgfqpoint{5.228350in}{3.130464in}}%
\pgfpathlineto{\pgfqpoint{5.228350in}{3.130464in}}%
\pgfpathclose%
\pgfusepath{stroke,fill}%
\end{pgfscope}%
\begin{pgfscope}%
\pgfpathrectangle{\pgfqpoint{3.985294in}{1.750000in}}{\pgfqpoint{2.279412in}{2.004545in}}%
\pgfusepath{clip}%
\pgfsetroundcap%
\pgfsetroundjoin%
\pgfsetlinewidth{0.325898pt}%
\definecolor{currentstroke}{rgb}{0.271305,0.019942,0.347269}%
\pgfsetstrokecolor{currentstroke}%
\pgfsetdash{}{0pt}%
\pgfpathmoveto{\pgfqpoint{5.693982in}{3.294061in}}%
\pgfpathquadraticcurveto{\pgfqpoint{5.681453in}{3.293693in}}{\pgfqpoint{5.673964in}{3.293473in}}%
\pgfusepath{stroke}%
\end{pgfscope}%
\begin{pgfscope}%
\pgfpathrectangle{\pgfqpoint{3.985294in}{1.750000in}}{\pgfqpoint{2.279412in}{2.004545in}}%
\pgfusepath{clip}%
\pgfsetroundcap%
\pgfsetroundjoin%
\definecolor{currentfill}{rgb}{0.271305,0.019942,0.347269}%
\pgfsetfillcolor{currentfill}%
\pgfsetlinewidth{0.325898pt}%
\definecolor{currentstroke}{rgb}{0.271305,0.019942,0.347269}%
\pgfsetstrokecolor{currentstroke}%
\pgfsetdash{}{0pt}%
\pgfpathmoveto{\pgfqpoint{5.730311in}{3.267339in}}%
\pgfpathlineto{\pgfqpoint{5.673964in}{3.293473in}}%
\pgfpathlineto{\pgfqpoint{5.728679in}{3.322870in}}%
\pgfpathlineto{\pgfqpoint{5.730311in}{3.267339in}}%
\pgfpathlineto{\pgfqpoint{5.730311in}{3.267339in}}%
\pgfpathclose%
\pgfusepath{stroke,fill}%
\end{pgfscope}%
\begin{pgfscope}%
\pgfpathrectangle{\pgfqpoint{3.985294in}{1.750000in}}{\pgfqpoint{2.279412in}{2.004545in}}%
\pgfusepath{clip}%
\pgfsetroundcap%
\pgfsetroundjoin%
\pgfsetlinewidth{0.337151pt}%
\definecolor{currentstroke}{rgb}{0.273809,0.031497,0.358853}%
\pgfsetstrokecolor{currentstroke}%
\pgfsetdash{}{0pt}%
\pgfpathmoveto{\pgfqpoint{5.237418in}{3.344479in}}%
\pgfpathquadraticcurveto{\pgfqpoint{5.227710in}{3.341122in}}{\pgfqpoint{5.222931in}{3.339469in}}%
\pgfusepath{stroke}%
\end{pgfscope}%
\begin{pgfscope}%
\pgfpathrectangle{\pgfqpoint{3.985294in}{1.750000in}}{\pgfqpoint{2.279412in}{2.004545in}}%
\pgfusepath{clip}%
\pgfsetroundcap%
\pgfsetroundjoin%
\definecolor{currentfill}{rgb}{0.273809,0.031497,0.358853}%
\pgfsetfillcolor{currentfill}%
\pgfsetlinewidth{0.337151pt}%
\definecolor{currentstroke}{rgb}{0.273809,0.031497,0.358853}%
\pgfsetstrokecolor{currentstroke}%
\pgfsetdash{}{0pt}%
\pgfpathmoveto{\pgfqpoint{5.284514in}{3.331373in}}%
\pgfpathlineto{\pgfqpoint{5.222931in}{3.339469in}}%
\pgfpathlineto{\pgfqpoint{5.266358in}{3.383878in}}%
\pgfpathlineto{\pgfqpoint{5.284514in}{3.331373in}}%
\pgfpathlineto{\pgfqpoint{5.284514in}{3.331373in}}%
\pgfpathclose%
\pgfusepath{stroke,fill}%
\end{pgfscope}%
\begin{pgfscope}%
\pgfpathrectangle{\pgfqpoint{3.985294in}{1.750000in}}{\pgfqpoint{2.279412in}{2.004545in}}%
\pgfusepath{clip}%
\pgfsetroundcap%
\pgfsetroundjoin%
\pgfsetlinewidth{0.381900pt}%
\definecolor{currentstroke}{rgb}{0.279566,0.067836,0.391917}%
\pgfsetstrokecolor{currentstroke}%
\pgfsetdash{}{0pt}%
\pgfpathmoveto{\pgfqpoint{5.018946in}{2.293498in}}%
\pgfpathquadraticcurveto{\pgfqpoint{5.015810in}{2.299628in}}{\pgfqpoint{5.015365in}{2.300497in}}%
\pgfusepath{stroke}%
\end{pgfscope}%
\begin{pgfscope}%
\pgfpathrectangle{\pgfqpoint{3.985294in}{1.750000in}}{\pgfqpoint{2.279412in}{2.004545in}}%
\pgfusepath{clip}%
\pgfsetroundcap%
\pgfsetroundjoin%
\definecolor{currentfill}{rgb}{0.279566,0.067836,0.391917}%
\pgfsetfillcolor{currentfill}%
\pgfsetlinewidth{0.381900pt}%
\definecolor{currentstroke}{rgb}{0.279566,0.067836,0.391917}%
\pgfsetstrokecolor{currentstroke}%
\pgfsetdash{}{0pt}%
\pgfpathmoveto{\pgfqpoint{5.015944in}{2.238387in}}%
\pgfpathlineto{\pgfqpoint{5.015365in}{2.300497in}}%
\pgfpathlineto{\pgfqpoint{5.065400in}{2.263694in}}%
\pgfpathlineto{\pgfqpoint{5.015944in}{2.238387in}}%
\pgfpathlineto{\pgfqpoint{5.015944in}{2.238387in}}%
\pgfpathclose%
\pgfusepath{stroke,fill}%
\end{pgfscope}%
\begin{pgfscope}%
\pgfpathrectangle{\pgfqpoint{3.985294in}{1.750000in}}{\pgfqpoint{2.279412in}{2.004545in}}%
\pgfusepath{clip}%
\pgfsetroundcap%
\pgfsetroundjoin%
\pgfsetlinewidth{0.343553pt}%
\definecolor{currentstroke}{rgb}{0.274952,0.037752,0.364543}%
\pgfsetstrokecolor{currentstroke}%
\pgfsetdash{}{0pt}%
\pgfpathmoveto{\pgfqpoint{5.187049in}{2.202726in}}%
\pgfpathquadraticcurveto{\pgfqpoint{5.176267in}{2.208050in}}{\pgfqpoint{5.170250in}{2.211020in}}%
\pgfusepath{stroke}%
\end{pgfscope}%
\begin{pgfscope}%
\pgfpathrectangle{\pgfqpoint{3.985294in}{1.750000in}}{\pgfqpoint{2.279412in}{2.004545in}}%
\pgfusepath{clip}%
\pgfsetroundcap%
\pgfsetroundjoin%
\definecolor{currentfill}{rgb}{0.274952,0.037752,0.364543}%
\pgfsetfillcolor{currentfill}%
\pgfsetlinewidth{0.343553pt}%
\definecolor{currentstroke}{rgb}{0.274952,0.037752,0.364543}%
\pgfsetstrokecolor{currentstroke}%
\pgfsetdash{}{0pt}%
\pgfpathmoveto{\pgfqpoint{5.207767in}{2.161518in}}%
\pgfpathlineto{\pgfqpoint{5.170250in}{2.211020in}}%
\pgfpathlineto{\pgfqpoint{5.232362in}{2.211333in}}%
\pgfpathlineto{\pgfqpoint{5.207767in}{2.161518in}}%
\pgfpathlineto{\pgfqpoint{5.207767in}{2.161518in}}%
\pgfpathclose%
\pgfusepath{stroke,fill}%
\end{pgfscope}%
\begin{pgfscope}%
\pgfpathrectangle{\pgfqpoint{3.985294in}{1.750000in}}{\pgfqpoint{2.279412in}{2.004545in}}%
\pgfusepath{clip}%
\pgfsetroundcap%
\pgfsetroundjoin%
\pgfsetlinewidth{0.324662pt}%
\definecolor{currentstroke}{rgb}{0.271305,0.019942,0.347269}%
\pgfsetstrokecolor{currentstroke}%
\pgfsetdash{}{0pt}%
\pgfpathmoveto{\pgfqpoint{5.663321in}{2.259869in}}%
\pgfpathquadraticcurveto{\pgfqpoint{5.650800in}{2.260372in}}{\pgfqpoint{5.643297in}{2.260674in}}%
\pgfusepath{stroke}%
\end{pgfscope}%
\begin{pgfscope}%
\pgfpathrectangle{\pgfqpoint{3.985294in}{1.750000in}}{\pgfqpoint{2.279412in}{2.004545in}}%
\pgfusepath{clip}%
\pgfsetroundcap%
\pgfsetroundjoin%
\definecolor{currentfill}{rgb}{0.271305,0.019942,0.347269}%
\pgfsetfillcolor{currentfill}%
\pgfsetlinewidth{0.324662pt}%
\definecolor{currentstroke}{rgb}{0.271305,0.019942,0.347269}%
\pgfsetstrokecolor{currentstroke}%
\pgfsetdash{}{0pt}%
\pgfpathmoveto{\pgfqpoint{5.697693in}{2.230689in}}%
\pgfpathlineto{\pgfqpoint{5.643297in}{2.260674in}}%
\pgfpathlineto{\pgfqpoint{5.699923in}{2.286200in}}%
\pgfpathlineto{\pgfqpoint{5.697693in}{2.230689in}}%
\pgfpathlineto{\pgfqpoint{5.697693in}{2.230689in}}%
\pgfpathclose%
\pgfusepath{stroke,fill}%
\end{pgfscope}%
\begin{pgfscope}%
\pgfpathrectangle{\pgfqpoint{3.985294in}{1.750000in}}{\pgfqpoint{2.279412in}{2.004545in}}%
\pgfusepath{clip}%
\pgfsetroundcap%
\pgfsetroundjoin%
\pgfsetlinewidth{0.460182pt}%
\definecolor{currentstroke}{rgb}{0.283072,0.130895,0.449241}%
\pgfsetstrokecolor{currentstroke}%
\pgfsetdash{}{0pt}%
\pgfpathmoveto{\pgfqpoint{5.541099in}{2.530796in}}%
\pgfpathquadraticcurveto{\pgfqpoint{5.528567in}{2.531141in}}{\pgfqpoint{5.523152in}{2.531290in}}%
\pgfusepath{stroke}%
\end{pgfscope}%
\begin{pgfscope}%
\pgfpathrectangle{\pgfqpoint{3.985294in}{1.750000in}}{\pgfqpoint{2.279412in}{2.004545in}}%
\pgfusepath{clip}%
\pgfsetroundcap%
\pgfsetroundjoin%
\definecolor{currentfill}{rgb}{0.283072,0.130895,0.449241}%
\pgfsetfillcolor{currentfill}%
\pgfsetlinewidth{0.460182pt}%
\definecolor{currentstroke}{rgb}{0.283072,0.130895,0.449241}%
\pgfsetstrokecolor{currentstroke}%
\pgfsetdash{}{0pt}%
\pgfpathmoveto{\pgfqpoint{5.577923in}{2.501996in}}%
\pgfpathlineto{\pgfqpoint{5.523152in}{2.531290in}}%
\pgfpathlineto{\pgfqpoint{5.579449in}{2.557531in}}%
\pgfpathlineto{\pgfqpoint{5.577923in}{2.501996in}}%
\pgfpathlineto{\pgfqpoint{5.577923in}{2.501996in}}%
\pgfpathclose%
\pgfusepath{stroke,fill}%
\end{pgfscope}%
\begin{pgfscope}%
\pgfpathrectangle{\pgfqpoint{3.985294in}{1.750000in}}{\pgfqpoint{2.279412in}{2.004545in}}%
\pgfusepath{clip}%
\pgfsetroundcap%
\pgfsetroundjoin%
\pgfsetlinewidth{0.343407pt}%
\definecolor{currentstroke}{rgb}{0.274952,0.037752,0.364543}%
\pgfsetstrokecolor{currentstroke}%
\pgfsetdash{}{0pt}%
\pgfpathmoveto{\pgfqpoint{5.386172in}{3.323584in}}%
\pgfpathquadraticcurveto{\pgfqpoint{5.373774in}{3.322031in}}{\pgfqpoint{5.366648in}{3.321138in}}%
\pgfusepath{stroke}%
\end{pgfscope}%
\begin{pgfscope}%
\pgfpathrectangle{\pgfqpoint{3.985294in}{1.750000in}}{\pgfqpoint{2.279412in}{2.004545in}}%
\pgfusepath{clip}%
\pgfsetroundcap%
\pgfsetroundjoin%
\definecolor{currentfill}{rgb}{0.274952,0.037752,0.364543}%
\pgfsetfillcolor{currentfill}%
\pgfsetlinewidth{0.343407pt}%
\definecolor{currentstroke}{rgb}{0.274952,0.037752,0.364543}%
\pgfsetstrokecolor{currentstroke}%
\pgfsetdash{}{0pt}%
\pgfpathmoveto{\pgfqpoint{5.425226in}{3.300482in}}%
\pgfpathlineto{\pgfqpoint{5.366648in}{3.321138in}}%
\pgfpathlineto{\pgfqpoint{5.418319in}{3.355607in}}%
\pgfpathlineto{\pgfqpoint{5.425226in}{3.300482in}}%
\pgfpathlineto{\pgfqpoint{5.425226in}{3.300482in}}%
\pgfpathclose%
\pgfusepath{stroke,fill}%
\end{pgfscope}%
\begin{pgfscope}%
\pgfpathrectangle{\pgfqpoint{3.985294in}{1.750000in}}{\pgfqpoint{2.279412in}{2.004545in}}%
\pgfusepath{clip}%
\pgfsetroundcap%
\pgfsetroundjoin%
\pgfsetlinewidth{0.367601pt}%
\definecolor{currentstroke}{rgb}{0.277941,0.056324,0.381191}%
\pgfsetstrokecolor{currentstroke}%
\pgfsetdash{}{0pt}%
\pgfpathmoveto{\pgfqpoint{5.330168in}{2.210991in}}%
\pgfpathquadraticcurveto{\pgfqpoint{5.317891in}{2.213211in}}{\pgfqpoint{5.311211in}{2.214419in}}%
\pgfusepath{stroke}%
\end{pgfscope}%
\begin{pgfscope}%
\pgfpathrectangle{\pgfqpoint{3.985294in}{1.750000in}}{\pgfqpoint{2.279412in}{2.004545in}}%
\pgfusepath{clip}%
\pgfsetroundcap%
\pgfsetroundjoin%
\definecolor{currentfill}{rgb}{0.277941,0.056324,0.381191}%
\pgfsetfillcolor{currentfill}%
\pgfsetlinewidth{0.367601pt}%
\definecolor{currentstroke}{rgb}{0.277941,0.056324,0.381191}%
\pgfsetstrokecolor{currentstroke}%
\pgfsetdash{}{0pt}%
\pgfpathmoveto{\pgfqpoint{5.360938in}{2.177200in}}%
\pgfpathlineto{\pgfqpoint{5.311211in}{2.214419in}}%
\pgfpathlineto{\pgfqpoint{5.370822in}{2.231869in}}%
\pgfpathlineto{\pgfqpoint{5.360938in}{2.177200in}}%
\pgfpathlineto{\pgfqpoint{5.360938in}{2.177200in}}%
\pgfpathclose%
\pgfusepath{stroke,fill}%
\end{pgfscope}%
\begin{pgfscope}%
\pgfpathrectangle{\pgfqpoint{3.985294in}{1.750000in}}{\pgfqpoint{2.279412in}{2.004545in}}%
\pgfusepath{clip}%
\pgfsetroundcap%
\pgfsetroundjoin%
\pgfsetlinewidth{0.355737pt}%
\definecolor{currentstroke}{rgb}{0.276022,0.044167,0.370164}%
\pgfsetstrokecolor{currentstroke}%
\pgfsetdash{}{0pt}%
\pgfpathmoveto{\pgfqpoint{5.590171in}{3.156867in}}%
\pgfpathquadraticcurveto{\pgfqpoint{5.577649in}{3.156343in}}{\pgfqpoint{5.570626in}{3.156050in}}%
\pgfusepath{stroke}%
\end{pgfscope}%
\begin{pgfscope}%
\pgfpathrectangle{\pgfqpoint{3.985294in}{1.750000in}}{\pgfqpoint{2.279412in}{2.004545in}}%
\pgfusepath{clip}%
\pgfsetroundcap%
\pgfsetroundjoin%
\definecolor{currentfill}{rgb}{0.276022,0.044167,0.370164}%
\pgfsetfillcolor{currentfill}%
\pgfsetlinewidth{0.355737pt}%
\definecolor{currentstroke}{rgb}{0.276022,0.044167,0.370164}%
\pgfsetstrokecolor{currentstroke}%
\pgfsetdash{}{0pt}%
\pgfpathmoveto{\pgfqpoint{5.627293in}{3.130616in}}%
\pgfpathlineto{\pgfqpoint{5.570626in}{3.156050in}}%
\pgfpathlineto{\pgfqpoint{5.624974in}{3.186123in}}%
\pgfpathlineto{\pgfqpoint{5.627293in}{3.130616in}}%
\pgfpathlineto{\pgfqpoint{5.627293in}{3.130616in}}%
\pgfpathclose%
\pgfusepath{stroke,fill}%
\end{pgfscope}%
\begin{pgfscope}%
\pgfpathrectangle{\pgfqpoint{3.985294in}{1.750000in}}{\pgfqpoint{2.279412in}{2.004545in}}%
\pgfusepath{clip}%
\pgfsetroundcap%
\pgfsetroundjoin%
\pgfsetlinewidth{0.388474pt}%
\definecolor{currentstroke}{rgb}{0.280267,0.073417,0.397163}%
\pgfsetstrokecolor{currentstroke}%
\pgfsetdash{}{0pt}%
\pgfpathmoveto{\pgfqpoint{5.291286in}{3.219966in}}%
\pgfpathquadraticcurveto{\pgfqpoint{5.279047in}{3.217611in}}{\pgfqpoint{5.272710in}{3.216391in}}%
\pgfusepath{stroke}%
\end{pgfscope}%
\begin{pgfscope}%
\pgfpathrectangle{\pgfqpoint{3.985294in}{1.750000in}}{\pgfqpoint{2.279412in}{2.004545in}}%
\pgfusepath{clip}%
\pgfsetroundcap%
\pgfsetroundjoin%
\definecolor{currentfill}{rgb}{0.280267,0.073417,0.397163}%
\pgfsetfillcolor{currentfill}%
\pgfsetlinewidth{0.388474pt}%
\definecolor{currentstroke}{rgb}{0.280267,0.073417,0.397163}%
\pgfsetstrokecolor{currentstroke}%
\pgfsetdash{}{0pt}%
\pgfpathmoveto{\pgfqpoint{5.332514in}{3.199613in}}%
\pgfpathlineto{\pgfqpoint{5.272710in}{3.216391in}}%
\pgfpathlineto{\pgfqpoint{5.322015in}{3.254168in}}%
\pgfpathlineto{\pgfqpoint{5.332514in}{3.199613in}}%
\pgfpathlineto{\pgfqpoint{5.332514in}{3.199613in}}%
\pgfpathclose%
\pgfusepath{stroke,fill}%
\end{pgfscope}%
\begin{pgfscope}%
\pgfpathrectangle{\pgfqpoint{3.985294in}{1.750000in}}{\pgfqpoint{2.279412in}{2.004545in}}%
\pgfusepath{clip}%
\pgfsetroundcap%
\pgfsetroundjoin%
\pgfsetlinewidth{0.368544pt}%
\definecolor{currentstroke}{rgb}{0.277941,0.056324,0.381191}%
\pgfsetstrokecolor{currentstroke}%
\pgfsetdash{}{0pt}%
\pgfpathmoveto{\pgfqpoint{5.334603in}{3.279401in}}%
\pgfpathquadraticcurveto{\pgfqpoint{5.322543in}{3.276850in}}{\pgfqpoint{5.316060in}{3.275478in}}%
\pgfusepath{stroke}%
\end{pgfscope}%
\begin{pgfscope}%
\pgfpathrectangle{\pgfqpoint{3.985294in}{1.750000in}}{\pgfqpoint{2.279412in}{2.004545in}}%
\pgfusepath{clip}%
\pgfsetroundcap%
\pgfsetroundjoin%
\definecolor{currentfill}{rgb}{0.277941,0.056324,0.381191}%
\pgfsetfillcolor{currentfill}%
\pgfsetlinewidth{0.368544pt}%
\definecolor{currentstroke}{rgb}{0.277941,0.056324,0.381191}%
\pgfsetstrokecolor{currentstroke}%
\pgfsetdash{}{0pt}%
\pgfpathmoveto{\pgfqpoint{5.376162in}{3.259802in}}%
\pgfpathlineto{\pgfqpoint{5.316060in}{3.275478in}}%
\pgfpathlineto{\pgfqpoint{5.364662in}{3.314154in}}%
\pgfpathlineto{\pgfqpoint{5.376162in}{3.259802in}}%
\pgfpathlineto{\pgfqpoint{5.376162in}{3.259802in}}%
\pgfpathclose%
\pgfusepath{stroke,fill}%
\end{pgfscope}%
\begin{pgfscope}%
\pgfpathrectangle{\pgfqpoint{3.985294in}{1.750000in}}{\pgfqpoint{2.279412in}{2.004545in}}%
\pgfusepath{clip}%
\pgfsetroundcap%
\pgfsetroundjoin%
\pgfsetlinewidth{0.392427pt}%
\definecolor{currentstroke}{rgb}{0.280894,0.078907,0.402329}%
\pgfsetstrokecolor{currentstroke}%
\pgfsetdash{}{0pt}%
\pgfpathmoveto{\pgfqpoint{5.176292in}{2.301205in}}%
\pgfpathquadraticcurveto{\pgfqpoint{5.164841in}{2.305618in}}{\pgfqpoint{5.159055in}{2.307848in}}%
\pgfusepath{stroke}%
\end{pgfscope}%
\begin{pgfscope}%
\pgfpathrectangle{\pgfqpoint{3.985294in}{1.750000in}}{\pgfqpoint{2.279412in}{2.004545in}}%
\pgfusepath{clip}%
\pgfsetroundcap%
\pgfsetroundjoin%
\definecolor{currentfill}{rgb}{0.280894,0.078907,0.402329}%
\pgfsetfillcolor{currentfill}%
\pgfsetlinewidth{0.392427pt}%
\definecolor{currentstroke}{rgb}{0.280894,0.078907,0.402329}%
\pgfsetstrokecolor{currentstroke}%
\pgfsetdash{}{0pt}%
\pgfpathmoveto{\pgfqpoint{5.200905in}{2.261950in}}%
\pgfpathlineto{\pgfqpoint{5.159055in}{2.307848in}}%
\pgfpathlineto{\pgfqpoint{5.220883in}{2.313789in}}%
\pgfpathlineto{\pgfqpoint{5.200905in}{2.261950in}}%
\pgfpathlineto{\pgfqpoint{5.200905in}{2.261950in}}%
\pgfpathclose%
\pgfusepath{stroke,fill}%
\end{pgfscope}%
\begin{pgfscope}%
\pgfpathrectangle{\pgfqpoint{3.985294in}{1.750000in}}{\pgfqpoint{2.279412in}{2.004545in}}%
\pgfusepath{clip}%
\pgfsetroundcap%
\pgfsetroundjoin%
\pgfsetlinewidth{0.858781pt}%
\definecolor{currentstroke}{rgb}{0.197636,0.391528,0.554969}%
\pgfsetstrokecolor{currentstroke}%
\pgfsetdash{}{0pt}%
\pgfpathmoveto{\pgfqpoint{4.710448in}{2.922228in}}%
\pgfpathquadraticcurveto{\pgfqpoint{4.717638in}{2.914184in}}{\pgfqpoint{4.715974in}{2.916045in}}%
\pgfusepath{stroke}%
\end{pgfscope}%
\begin{pgfscope}%
\pgfpathrectangle{\pgfqpoint{3.985294in}{1.750000in}}{\pgfqpoint{2.279412in}{2.004545in}}%
\pgfusepath{clip}%
\pgfsetroundcap%
\pgfsetroundjoin%
\definecolor{currentfill}{rgb}{0.197636,0.391528,0.554969}%
\pgfsetfillcolor{currentfill}%
\pgfsetlinewidth{0.858781pt}%
\definecolor{currentstroke}{rgb}{0.197636,0.391528,0.554969}%
\pgfsetstrokecolor{currentstroke}%
\pgfsetdash{}{0pt}%
\pgfpathmoveto{\pgfqpoint{4.699658in}{2.975977in}}%
\pgfpathlineto{\pgfqpoint{4.715974in}{2.916045in}}%
\pgfpathlineto{\pgfqpoint{4.658239in}{2.938951in}}%
\pgfpathlineto{\pgfqpoint{4.699658in}{2.975977in}}%
\pgfpathlineto{\pgfqpoint{4.699658in}{2.975977in}}%
\pgfpathclose%
\pgfusepath{stroke,fill}%
\end{pgfscope}%
\begin{pgfscope}%
\pgfpathrectangle{\pgfqpoint{3.985294in}{1.750000in}}{\pgfqpoint{2.279412in}{2.004545in}}%
\pgfusepath{clip}%
\pgfsetbuttcap%
\pgfsetroundjoin%
\pgfsetlinewidth{1.505625pt}%
\definecolor{currentstroke}{rgb}{0.000000,0.000000,0.000000}%
\pgfsetstrokecolor{currentstroke}%
\pgfsetdash{}{0pt}%
\pgfpathmoveto{\pgfqpoint{4.778678in}{2.089441in}}%
\pgfpathlineto{\pgfqpoint{4.778678in}{3.415105in}}%
\pgfusepath{stroke}%
\end{pgfscope}%
\begin{pgfscope}%
\pgfpathrectangle{\pgfqpoint{3.985294in}{1.750000in}}{\pgfqpoint{2.279412in}{2.004545in}}%
\pgfusepath{clip}%
\pgfsetbuttcap%
\pgfsetroundjoin%
\pgfsetlinewidth{1.505625pt}%
\definecolor{currentstroke}{rgb}{0.000000,0.000000,0.000000}%
\pgfsetstrokecolor{currentstroke}%
\pgfsetdash{}{0pt}%
\pgfpathmoveto{\pgfqpoint{5.927204in}{2.089441in}}%
\pgfpathlineto{\pgfqpoint{5.927204in}{3.415105in}}%
\pgfusepath{stroke}%
\end{pgfscope}%
\begin{pgfscope}%
\pgfsetrectcap%
\pgfsetmiterjoin%
\pgfsetlinewidth{0.803000pt}%
\definecolor{currentstroke}{rgb}{0.000000,0.000000,0.000000}%
\pgfsetstrokecolor{currentstroke}%
\pgfsetdash{}{0pt}%
\pgfpathmoveto{\pgfqpoint{3.985294in}{1.750000in}}%
\pgfpathlineto{\pgfqpoint{3.985294in}{3.754545in}}%
\pgfusepath{stroke}%
\end{pgfscope}%
\begin{pgfscope}%
\pgfsetrectcap%
\pgfsetmiterjoin%
\pgfsetlinewidth{0.803000pt}%
\definecolor{currentstroke}{rgb}{0.000000,0.000000,0.000000}%
\pgfsetstrokecolor{currentstroke}%
\pgfsetdash{}{0pt}%
\pgfpathmoveto{\pgfqpoint{6.264706in}{1.750000in}}%
\pgfpathlineto{\pgfqpoint{6.264706in}{3.754545in}}%
\pgfusepath{stroke}%
\end{pgfscope}%
\begin{pgfscope}%
\pgfsetrectcap%
\pgfsetmiterjoin%
\pgfsetlinewidth{0.803000pt}%
\definecolor{currentstroke}{rgb}{0.000000,0.000000,0.000000}%
\pgfsetstrokecolor{currentstroke}%
\pgfsetdash{}{0pt}%
\pgfpathmoveto{\pgfqpoint{3.985294in}{1.750000in}}%
\pgfpathlineto{\pgfqpoint{6.264706in}{1.750000in}}%
\pgfusepath{stroke}%
\end{pgfscope}%
\begin{pgfscope}%
\pgfsetrectcap%
\pgfsetmiterjoin%
\pgfsetlinewidth{0.803000pt}%
\definecolor{currentstroke}{rgb}{0.000000,0.000000,0.000000}%
\pgfsetstrokecolor{currentstroke}%
\pgfsetdash{}{0pt}%
\pgfpathmoveto{\pgfqpoint{3.985294in}{3.754545in}}%
\pgfpathlineto{\pgfqpoint{6.264706in}{3.754545in}}%
\pgfusepath{stroke}%
\end{pgfscope}%
\begin{pgfscope}%
\definecolor{textcolor}{rgb}{0.000000,0.000000,0.000000}%
\pgfsetstrokecolor{textcolor}%
\pgfsetfillcolor{textcolor}%
\pgftext[x=5.125000in,y=3.837879in,,base]{\color{textcolor}\sffamily\fontsize{12.000000}{14.400000}\selectfont e)}%
\end{pgfscope}%
\begin{pgfscope}%
\pgfsetbuttcap%
\pgfsetmiterjoin%
\definecolor{currentfill}{rgb}{1.000000,1.000000,1.000000}%
\pgfsetfillcolor{currentfill}%
\pgfsetlinewidth{0.000000pt}%
\definecolor{currentstroke}{rgb}{0.000000,0.000000,0.000000}%
\pgfsetstrokecolor{currentstroke}%
\pgfsetstrokeopacity{0.000000}%
\pgfsetdash{}{0pt}%
\pgfpathmoveto{\pgfqpoint{6.720588in}{1.750000in}}%
\pgfpathlineto{\pgfqpoint{9.000000in}{1.750000in}}%
\pgfpathlineto{\pgfqpoint{9.000000in}{3.754545in}}%
\pgfpathlineto{\pgfqpoint{6.720588in}{3.754545in}}%
\pgfpathlineto{\pgfqpoint{6.720588in}{1.750000in}}%
\pgfpathclose%
\pgfusepath{fill}%
\end{pgfscope}%
\begin{pgfscope}%
\pgfpathrectangle{\pgfqpoint{6.720588in}{1.750000in}}{\pgfqpoint{2.279412in}{2.004545in}}%
\pgfusepath{clip}%
\pgfsys@transformcm{2.291667}{0.000000}{0.000000}{2.013889}{6.720588in}{1.750000in}%
\pgftext[left,bottom]{\includegraphics[interpolate=false,width=1.000000in,height=1.000000in]{q_series-img5.png}}%
\end{pgfscope}%
\begin{pgfscope}%
\pgfsetbuttcap%
\pgfsetroundjoin%
\definecolor{currentfill}{rgb}{0.000000,0.000000,0.000000}%
\pgfsetfillcolor{currentfill}%
\pgfsetlinewidth{0.803000pt}%
\definecolor{currentstroke}{rgb}{0.000000,0.000000,0.000000}%
\pgfsetstrokecolor{currentstroke}%
\pgfsetdash{}{0pt}%
\pgfsys@defobject{currentmarker}{\pgfqpoint{0.000000in}{-0.048611in}}{\pgfqpoint{0.000000in}{0.000000in}}{%
\pgfpathmoveto{\pgfqpoint{0.000000in}{0.000000in}}%
\pgfpathlineto{\pgfqpoint{0.000000in}{-0.048611in}}%
\pgfusepath{stroke,fill}%
}%
\begin{pgfscope}%
\pgfsys@transformshift{7.131130in}{1.750000in}%
\pgfsys@useobject{currentmarker}{}%
\end{pgfscope}%
\end{pgfscope}%
\begin{pgfscope}%
\definecolor{textcolor}{rgb}{0.000000,0.000000,0.000000}%
\pgfsetstrokecolor{textcolor}%
\pgfsetfillcolor{textcolor}%
\pgftext[x=7.131130in,y=1.652778in,,top]{\color{textcolor}\sffamily\fontsize{10.000000}{12.000000}\selectfont \(\displaystyle {\ensuremath{-}10}\)}%
\end{pgfscope}%
\begin{pgfscope}%
\pgfsetbuttcap%
\pgfsetroundjoin%
\definecolor{currentfill}{rgb}{0.000000,0.000000,0.000000}%
\pgfsetfillcolor{currentfill}%
\pgfsetlinewidth{0.803000pt}%
\definecolor{currentstroke}{rgb}{0.000000,0.000000,0.000000}%
\pgfsetstrokecolor{currentstroke}%
\pgfsetdash{}{0pt}%
\pgfsys@defobject{currentmarker}{\pgfqpoint{0.000000in}{-0.048611in}}{\pgfqpoint{0.000000in}{0.000000in}}{%
\pgfpathmoveto{\pgfqpoint{0.000000in}{0.000000in}}%
\pgfpathlineto{\pgfqpoint{0.000000in}{-0.048611in}}%
\pgfusepath{stroke,fill}%
}%
\begin{pgfscope}%
\pgfsys@transformshift{7.609683in}{1.750000in}%
\pgfsys@useobject{currentmarker}{}%
\end{pgfscope}%
\end{pgfscope}%
\begin{pgfscope}%
\definecolor{textcolor}{rgb}{0.000000,0.000000,0.000000}%
\pgfsetstrokecolor{textcolor}%
\pgfsetfillcolor{textcolor}%
\pgftext[x=7.609683in,y=1.652778in,,top]{\color{textcolor}\sffamily\fontsize{10.000000}{12.000000}\selectfont \(\displaystyle {\ensuremath{-}5}\)}%
\end{pgfscope}%
\begin{pgfscope}%
\pgfsetbuttcap%
\pgfsetroundjoin%
\definecolor{currentfill}{rgb}{0.000000,0.000000,0.000000}%
\pgfsetfillcolor{currentfill}%
\pgfsetlinewidth{0.803000pt}%
\definecolor{currentstroke}{rgb}{0.000000,0.000000,0.000000}%
\pgfsetstrokecolor{currentstroke}%
\pgfsetdash{}{0pt}%
\pgfsys@defobject{currentmarker}{\pgfqpoint{0.000000in}{-0.048611in}}{\pgfqpoint{0.000000in}{0.000000in}}{%
\pgfpathmoveto{\pgfqpoint{0.000000in}{0.000000in}}%
\pgfpathlineto{\pgfqpoint{0.000000in}{-0.048611in}}%
\pgfusepath{stroke,fill}%
}%
\begin{pgfscope}%
\pgfsys@transformshift{8.088235in}{1.750000in}%
\pgfsys@useobject{currentmarker}{}%
\end{pgfscope}%
\end{pgfscope}%
\begin{pgfscope}%
\definecolor{textcolor}{rgb}{0.000000,0.000000,0.000000}%
\pgfsetstrokecolor{textcolor}%
\pgfsetfillcolor{textcolor}%
\pgftext[x=8.088235in,y=1.652778in,,top]{\color{textcolor}\sffamily\fontsize{10.000000}{12.000000}\selectfont \(\displaystyle {0}\)}%
\end{pgfscope}%
\begin{pgfscope}%
\pgfsetbuttcap%
\pgfsetroundjoin%
\definecolor{currentfill}{rgb}{0.000000,0.000000,0.000000}%
\pgfsetfillcolor{currentfill}%
\pgfsetlinewidth{0.803000pt}%
\definecolor{currentstroke}{rgb}{0.000000,0.000000,0.000000}%
\pgfsetstrokecolor{currentstroke}%
\pgfsetdash{}{0pt}%
\pgfsys@defobject{currentmarker}{\pgfqpoint{0.000000in}{-0.048611in}}{\pgfqpoint{0.000000in}{0.000000in}}{%
\pgfpathmoveto{\pgfqpoint{0.000000in}{0.000000in}}%
\pgfpathlineto{\pgfqpoint{0.000000in}{-0.048611in}}%
\pgfusepath{stroke,fill}%
}%
\begin{pgfscope}%
\pgfsys@transformshift{8.566788in}{1.750000in}%
\pgfsys@useobject{currentmarker}{}%
\end{pgfscope}%
\end{pgfscope}%
\begin{pgfscope}%
\definecolor{textcolor}{rgb}{0.000000,0.000000,0.000000}%
\pgfsetstrokecolor{textcolor}%
\pgfsetfillcolor{textcolor}%
\pgftext[x=8.566788in,y=1.652778in,,top]{\color{textcolor}\sffamily\fontsize{10.000000}{12.000000}\selectfont \(\displaystyle {5}\)}%
\end{pgfscope}%
\begin{pgfscope}%
\definecolor{textcolor}{rgb}{0.000000,0.000000,0.000000}%
\pgfsetstrokecolor{textcolor}%
\pgfsetfillcolor{textcolor}%
\pgftext[x=7.860294in,y=1.473766in,,top]{\color{textcolor}\sffamily\fontsize{10.000000}{12.000000}\selectfont \(\displaystyle \zeta \, \mathrm{[\mu m]}\)}%
\end{pgfscope}%
\begin{pgfscope}%
\pgfsetbuttcap%
\pgfsetroundjoin%
\definecolor{currentfill}{rgb}{0.000000,0.000000,0.000000}%
\pgfsetfillcolor{currentfill}%
\pgfsetlinewidth{0.803000pt}%
\definecolor{currentstroke}{rgb}{0.000000,0.000000,0.000000}%
\pgfsetstrokecolor{currentstroke}%
\pgfsetdash{}{0pt}%
\pgfsys@defobject{currentmarker}{\pgfqpoint{-0.048611in}{0.000000in}}{\pgfqpoint{-0.000000in}{0.000000in}}{%
\pgfpathmoveto{\pgfqpoint{-0.000000in}{0.000000in}}%
\pgfpathlineto{\pgfqpoint{-0.048611in}{0.000000in}}%
\pgfusepath{stroke,fill}%
}%
\begin{pgfscope}%
\pgfsys@transformshift{6.720588in}{1.758025in}%
\pgfsys@useobject{currentmarker}{}%
\end{pgfscope}%
\end{pgfscope}%
\begin{pgfscope}%
\pgfsetbuttcap%
\pgfsetroundjoin%
\definecolor{currentfill}{rgb}{0.000000,0.000000,0.000000}%
\pgfsetfillcolor{currentfill}%
\pgfsetlinewidth{0.803000pt}%
\definecolor{currentstroke}{rgb}{0.000000,0.000000,0.000000}%
\pgfsetstrokecolor{currentstroke}%
\pgfsetdash{}{0pt}%
\pgfsys@defobject{currentmarker}{\pgfqpoint{-0.048611in}{0.000000in}}{\pgfqpoint{-0.000000in}{0.000000in}}{%
\pgfpathmoveto{\pgfqpoint{-0.000000in}{0.000000in}}%
\pgfpathlineto{\pgfqpoint{-0.048611in}{0.000000in}}%
\pgfusepath{stroke,fill}%
}%
\begin{pgfscope}%
\pgfsys@transformshift{6.720588in}{2.089441in}%
\pgfsys@useobject{currentmarker}{}%
\end{pgfscope}%
\end{pgfscope}%
\begin{pgfscope}%
\pgfsetbuttcap%
\pgfsetroundjoin%
\definecolor{currentfill}{rgb}{0.000000,0.000000,0.000000}%
\pgfsetfillcolor{currentfill}%
\pgfsetlinewidth{0.803000pt}%
\definecolor{currentstroke}{rgb}{0.000000,0.000000,0.000000}%
\pgfsetstrokecolor{currentstroke}%
\pgfsetdash{}{0pt}%
\pgfsys@defobject{currentmarker}{\pgfqpoint{-0.048611in}{0.000000in}}{\pgfqpoint{-0.000000in}{0.000000in}}{%
\pgfpathmoveto{\pgfqpoint{-0.000000in}{0.000000in}}%
\pgfpathlineto{\pgfqpoint{-0.048611in}{0.000000in}}%
\pgfusepath{stroke,fill}%
}%
\begin{pgfscope}%
\pgfsys@transformshift{6.720588in}{2.420857in}%
\pgfsys@useobject{currentmarker}{}%
\end{pgfscope}%
\end{pgfscope}%
\begin{pgfscope}%
\pgfsetbuttcap%
\pgfsetroundjoin%
\definecolor{currentfill}{rgb}{0.000000,0.000000,0.000000}%
\pgfsetfillcolor{currentfill}%
\pgfsetlinewidth{0.803000pt}%
\definecolor{currentstroke}{rgb}{0.000000,0.000000,0.000000}%
\pgfsetstrokecolor{currentstroke}%
\pgfsetdash{}{0pt}%
\pgfsys@defobject{currentmarker}{\pgfqpoint{-0.048611in}{0.000000in}}{\pgfqpoint{-0.000000in}{0.000000in}}{%
\pgfpathmoveto{\pgfqpoint{-0.000000in}{0.000000in}}%
\pgfpathlineto{\pgfqpoint{-0.048611in}{0.000000in}}%
\pgfusepath{stroke,fill}%
}%
\begin{pgfscope}%
\pgfsys@transformshift{6.720588in}{2.752273in}%
\pgfsys@useobject{currentmarker}{}%
\end{pgfscope}%
\end{pgfscope}%
\begin{pgfscope}%
\pgfsetbuttcap%
\pgfsetroundjoin%
\definecolor{currentfill}{rgb}{0.000000,0.000000,0.000000}%
\pgfsetfillcolor{currentfill}%
\pgfsetlinewidth{0.803000pt}%
\definecolor{currentstroke}{rgb}{0.000000,0.000000,0.000000}%
\pgfsetstrokecolor{currentstroke}%
\pgfsetdash{}{0pt}%
\pgfsys@defobject{currentmarker}{\pgfqpoint{-0.048611in}{0.000000in}}{\pgfqpoint{-0.000000in}{0.000000in}}{%
\pgfpathmoveto{\pgfqpoint{-0.000000in}{0.000000in}}%
\pgfpathlineto{\pgfqpoint{-0.048611in}{0.000000in}}%
\pgfusepath{stroke,fill}%
}%
\begin{pgfscope}%
\pgfsys@transformshift{6.720588in}{3.083689in}%
\pgfsys@useobject{currentmarker}{}%
\end{pgfscope}%
\end{pgfscope}%
\begin{pgfscope}%
\pgfsetbuttcap%
\pgfsetroundjoin%
\definecolor{currentfill}{rgb}{0.000000,0.000000,0.000000}%
\pgfsetfillcolor{currentfill}%
\pgfsetlinewidth{0.803000pt}%
\definecolor{currentstroke}{rgb}{0.000000,0.000000,0.000000}%
\pgfsetstrokecolor{currentstroke}%
\pgfsetdash{}{0pt}%
\pgfsys@defobject{currentmarker}{\pgfqpoint{-0.048611in}{0.000000in}}{\pgfqpoint{-0.000000in}{0.000000in}}{%
\pgfpathmoveto{\pgfqpoint{-0.000000in}{0.000000in}}%
\pgfpathlineto{\pgfqpoint{-0.048611in}{0.000000in}}%
\pgfusepath{stroke,fill}%
}%
\begin{pgfscope}%
\pgfsys@transformshift{6.720588in}{3.415105in}%
\pgfsys@useobject{currentmarker}{}%
\end{pgfscope}%
\end{pgfscope}%
\begin{pgfscope}%
\pgfsetbuttcap%
\pgfsetroundjoin%
\definecolor{currentfill}{rgb}{0.000000,0.000000,0.000000}%
\pgfsetfillcolor{currentfill}%
\pgfsetlinewidth{0.803000pt}%
\definecolor{currentstroke}{rgb}{0.000000,0.000000,0.000000}%
\pgfsetstrokecolor{currentstroke}%
\pgfsetdash{}{0pt}%
\pgfsys@defobject{currentmarker}{\pgfqpoint{-0.048611in}{0.000000in}}{\pgfqpoint{-0.000000in}{0.000000in}}{%
\pgfpathmoveto{\pgfqpoint{-0.000000in}{0.000000in}}%
\pgfpathlineto{\pgfqpoint{-0.048611in}{0.000000in}}%
\pgfusepath{stroke,fill}%
}%
\begin{pgfscope}%
\pgfsys@transformshift{6.720588in}{3.746521in}%
\pgfsys@useobject{currentmarker}{}%
\end{pgfscope}%
\end{pgfscope}%
\begin{pgfscope}%
\definecolor{textcolor}{rgb}{0.000000,0.000000,0.000000}%
\pgfsetstrokecolor{textcolor}%
\pgfsetfillcolor{textcolor}%
\pgftext[x=6.665033in,y=2.752273in,,bottom,rotate=90.000000]{\color{textcolor}\sffamily\fontsize{10.000000}{12.000000}\selectfont \(\displaystyle z \, \mathrm{[\mu m]}\)}%
\end{pgfscope}%
\begin{pgfscope}%
\pgfpathrectangle{\pgfqpoint{6.720588in}{1.750000in}}{\pgfqpoint{2.279412in}{2.004545in}}%
\pgfusepath{clip}%
\pgfsetbuttcap%
\pgfsetroundjoin%
\pgfsetlinewidth{0.307977pt}%
\definecolor{currentstroke}{rgb}{0.267004,0.004874,0.329415}%
\pgfsetstrokecolor{currentstroke}%
\pgfsetdash{}{0pt}%
\pgfpathmoveto{\pgfqpoint{8.424505in}{1.850137in}}%
\pgfpathlineto{\pgfqpoint{8.424505in}{1.850137in}}%
\pgfusepath{stroke}%
\end{pgfscope}%
\begin{pgfscope}%
\pgfpathrectangle{\pgfqpoint{6.720588in}{1.750000in}}{\pgfqpoint{2.279412in}{2.004545in}}%
\pgfusepath{clip}%
\pgfsetbuttcap%
\pgfsetroundjoin%
\pgfsetlinewidth{0.307977pt}%
\definecolor{currentstroke}{rgb}{0.267004,0.004874,0.329415}%
\pgfsetstrokecolor{currentstroke}%
\pgfsetdash{}{0pt}%
\pgfpathmoveto{\pgfqpoint{8.424505in}{1.850137in}}%
\pgfpathlineto{\pgfqpoint{8.424505in}{1.850137in}}%
\pgfusepath{stroke}%
\end{pgfscope}%
\begin{pgfscope}%
\pgfpathrectangle{\pgfqpoint{6.720588in}{1.750000in}}{\pgfqpoint{2.279412in}{2.004545in}}%
\pgfusepath{clip}%
\pgfsetbuttcap%
\pgfsetroundjoin%
\pgfsetlinewidth{0.307977pt}%
\definecolor{currentstroke}{rgb}{0.267004,0.004874,0.329415}%
\pgfsetstrokecolor{currentstroke}%
\pgfsetdash{}{0pt}%
\pgfpathmoveto{\pgfqpoint{8.424505in}{1.850137in}}%
\pgfpathlineto{\pgfqpoint{8.427741in}{1.850465in}}%
\pgfusepath{stroke}%
\end{pgfscope}%
\begin{pgfscope}%
\pgfpathrectangle{\pgfqpoint{6.720588in}{1.750000in}}{\pgfqpoint{2.279412in}{2.004545in}}%
\pgfusepath{clip}%
\pgfsetbuttcap%
\pgfsetroundjoin%
\pgfsetlinewidth{0.308712pt}%
\definecolor{currentstroke}{rgb}{0.268510,0.009605,0.335427}%
\pgfsetstrokecolor{currentstroke}%
\pgfsetdash{}{0pt}%
\pgfpathmoveto{\pgfqpoint{8.427741in}{1.850465in}}%
\pgfpathlineto{\pgfqpoint{8.429201in}{1.850717in}}%
\pgfusepath{stroke}%
\end{pgfscope}%
\begin{pgfscope}%
\pgfpathrectangle{\pgfqpoint{6.720588in}{1.750000in}}{\pgfqpoint{2.279412in}{2.004545in}}%
\pgfusepath{clip}%
\pgfsetbuttcap%
\pgfsetroundjoin%
\pgfsetlinewidth{0.309076pt}%
\definecolor{currentstroke}{rgb}{0.268510,0.009605,0.335427}%
\pgfsetstrokecolor{currentstroke}%
\pgfsetdash{}{0pt}%
\pgfpathmoveto{\pgfqpoint{8.429201in}{1.850717in}}%
\pgfpathlineto{\pgfqpoint{8.429567in}{1.851087in}}%
\pgfusepath{stroke}%
\end{pgfscope}%
\begin{pgfscope}%
\pgfpathrectangle{\pgfqpoint{6.720588in}{1.750000in}}{\pgfqpoint{2.279412in}{2.004545in}}%
\pgfusepath{clip}%
\pgfsetbuttcap%
\pgfsetroundjoin%
\pgfsetlinewidth{0.309205pt}%
\definecolor{currentstroke}{rgb}{0.268510,0.009605,0.335427}%
\pgfsetstrokecolor{currentstroke}%
\pgfsetdash{}{0pt}%
\pgfpathmoveto{\pgfqpoint{8.429567in}{1.851087in}}%
\pgfpathlineto{\pgfqpoint{8.428880in}{1.851565in}}%
\pgfusepath{stroke}%
\end{pgfscope}%
\begin{pgfscope}%
\pgfpathrectangle{\pgfqpoint{6.720588in}{1.750000in}}{\pgfqpoint{2.279412in}{2.004545in}}%
\pgfusepath{clip}%
\pgfsetbuttcap%
\pgfsetroundjoin%
\pgfsetlinewidth{0.309075pt}%
\definecolor{currentstroke}{rgb}{0.268510,0.009605,0.335427}%
\pgfsetstrokecolor{currentstroke}%
\pgfsetdash{}{0pt}%
\pgfpathmoveto{\pgfqpoint{8.428880in}{1.851565in}}%
\pgfpathlineto{\pgfqpoint{8.428880in}{1.851565in}}%
\pgfusepath{stroke}%
\end{pgfscope}%
\begin{pgfscope}%
\pgfpathrectangle{\pgfqpoint{6.720588in}{1.750000in}}{\pgfqpoint{2.279412in}{2.004545in}}%
\pgfusepath{clip}%
\pgfsetbuttcap%
\pgfsetroundjoin%
\pgfsetlinewidth{0.309075pt}%
\definecolor{currentstroke}{rgb}{0.268510,0.009605,0.335427}%
\pgfsetstrokecolor{currentstroke}%
\pgfsetdash{}{0pt}%
\pgfpathmoveto{\pgfqpoint{8.428880in}{1.851565in}}%
\pgfpathlineto{\pgfqpoint{8.427815in}{1.851957in}}%
\pgfusepath{stroke}%
\end{pgfscope}%
\begin{pgfscope}%
\pgfpathrectangle{\pgfqpoint{6.720588in}{1.750000in}}{\pgfqpoint{2.279412in}{2.004545in}}%
\pgfusepath{clip}%
\pgfsetbuttcap%
\pgfsetroundjoin%
\pgfsetlinewidth{0.308815pt}%
\definecolor{currentstroke}{rgb}{0.268510,0.009605,0.335427}%
\pgfsetstrokecolor{currentstroke}%
\pgfsetdash{}{0pt}%
\pgfpathmoveto{\pgfqpoint{8.427815in}{1.851957in}}%
\pgfpathlineto{\pgfqpoint{8.427815in}{1.851957in}}%
\pgfusepath{stroke}%
\end{pgfscope}%
\begin{pgfscope}%
\pgfpathrectangle{\pgfqpoint{6.720588in}{1.750000in}}{\pgfqpoint{2.279412in}{2.004545in}}%
\pgfusepath{clip}%
\pgfsetbuttcap%
\pgfsetroundjoin%
\pgfsetlinewidth{0.308815pt}%
\definecolor{currentstroke}{rgb}{0.268510,0.009605,0.335427}%
\pgfsetstrokecolor{currentstroke}%
\pgfsetdash{}{0pt}%
\pgfpathmoveto{\pgfqpoint{8.427815in}{1.851957in}}%
\pgfpathlineto{\pgfqpoint{8.427934in}{1.852157in}}%
\pgfusepath{stroke}%
\end{pgfscope}%
\begin{pgfscope}%
\pgfpathrectangle{\pgfqpoint{6.720588in}{1.750000in}}{\pgfqpoint{2.279412in}{2.004545in}}%
\pgfusepath{clip}%
\pgfsetbuttcap%
\pgfsetroundjoin%
\pgfsetlinewidth{0.308860pt}%
\definecolor{currentstroke}{rgb}{0.268510,0.009605,0.335427}%
\pgfsetstrokecolor{currentstroke}%
\pgfsetdash{}{0pt}%
\pgfpathmoveto{\pgfqpoint{8.427934in}{1.852157in}}%
\pgfpathlineto{\pgfqpoint{8.428148in}{1.852306in}}%
\pgfusepath{stroke}%
\end{pgfscope}%
\begin{pgfscope}%
\pgfpathrectangle{\pgfqpoint{6.720588in}{1.750000in}}{\pgfqpoint{2.279412in}{2.004545in}}%
\pgfusepath{clip}%
\pgfsetbuttcap%
\pgfsetroundjoin%
\pgfsetlinewidth{0.308931pt}%
\definecolor{currentstroke}{rgb}{0.268510,0.009605,0.335427}%
\pgfsetstrokecolor{currentstroke}%
\pgfsetdash{}{0pt}%
\pgfpathmoveto{\pgfqpoint{8.428148in}{1.852306in}}%
\pgfpathlineto{\pgfqpoint{8.428206in}{1.852432in}}%
\pgfusepath{stroke}%
\end{pgfscope}%
\begin{pgfscope}%
\pgfpathrectangle{\pgfqpoint{6.720588in}{1.750000in}}{\pgfqpoint{2.279412in}{2.004545in}}%
\pgfusepath{clip}%
\pgfsetbuttcap%
\pgfsetroundjoin%
\pgfsetlinewidth{0.308957pt}%
\definecolor{currentstroke}{rgb}{0.268510,0.009605,0.335427}%
\pgfsetstrokecolor{currentstroke}%
\pgfsetdash{}{0pt}%
\pgfpathmoveto{\pgfqpoint{8.428206in}{1.852432in}}%
\pgfpathlineto{\pgfqpoint{8.428009in}{1.852550in}}%
\pgfusepath{stroke}%
\end{pgfscope}%
\begin{pgfscope}%
\pgfpathrectangle{\pgfqpoint{6.720588in}{1.750000in}}{\pgfqpoint{2.279412in}{2.004545in}}%
\pgfusepath{clip}%
\pgfsetbuttcap%
\pgfsetroundjoin%
\pgfsetlinewidth{0.308907pt}%
\definecolor{currentstroke}{rgb}{0.268510,0.009605,0.335427}%
\pgfsetstrokecolor{currentstroke}%
\pgfsetdash{}{0pt}%
\pgfpathmoveto{\pgfqpoint{8.428009in}{1.852550in}}%
\pgfpathlineto{\pgfqpoint{8.427671in}{1.852651in}}%
\pgfusepath{stroke}%
\end{pgfscope}%
\begin{pgfscope}%
\pgfpathrectangle{\pgfqpoint{6.720588in}{1.750000in}}{\pgfqpoint{2.279412in}{2.004545in}}%
\pgfusepath{clip}%
\pgfsetbuttcap%
\pgfsetroundjoin%
\pgfsetlinewidth{0.308812pt}%
\definecolor{currentstroke}{rgb}{0.268510,0.009605,0.335427}%
\pgfsetstrokecolor{currentstroke}%
\pgfsetdash{}{0pt}%
\pgfpathmoveto{\pgfqpoint{8.427671in}{1.852651in}}%
\pgfpathlineto{\pgfqpoint{8.427529in}{1.852721in}}%
\pgfusepath{stroke}%
\end{pgfscope}%
\begin{pgfscope}%
\pgfpathrectangle{\pgfqpoint{6.720588in}{1.750000in}}{\pgfqpoint{2.279412in}{2.004545in}}%
\pgfusepath{clip}%
\pgfsetbuttcap%
\pgfsetroundjoin%
\pgfsetlinewidth{0.308773pt}%
\definecolor{currentstroke}{rgb}{0.268510,0.009605,0.335427}%
\pgfsetstrokecolor{currentstroke}%
\pgfsetdash{}{0pt}%
\pgfpathmoveto{\pgfqpoint{8.427529in}{1.852721in}}%
\pgfpathlineto{\pgfqpoint{8.427746in}{1.852756in}}%
\pgfusepath{stroke}%
\end{pgfscope}%
\begin{pgfscope}%
\pgfpathrectangle{\pgfqpoint{6.720588in}{1.750000in}}{\pgfqpoint{2.279412in}{2.004545in}}%
\pgfusepath{clip}%
\pgfsetbuttcap%
\pgfsetroundjoin%
\pgfsetlinewidth{0.308840pt}%
\definecolor{currentstroke}{rgb}{0.268510,0.009605,0.335427}%
\pgfsetstrokecolor{currentstroke}%
\pgfsetdash{}{0pt}%
\pgfpathmoveto{\pgfqpoint{8.427746in}{1.852756in}}%
\pgfpathlineto{\pgfqpoint{8.428060in}{1.852776in}}%
\pgfusepath{stroke}%
\end{pgfscope}%
\begin{pgfscope}%
\pgfpathrectangle{\pgfqpoint{6.720588in}{1.750000in}}{\pgfqpoint{2.279412in}{2.004545in}}%
\pgfusepath{clip}%
\pgfsetbuttcap%
\pgfsetroundjoin%
\pgfsetlinewidth{0.308936pt}%
\definecolor{currentstroke}{rgb}{0.268510,0.009605,0.335427}%
\pgfsetstrokecolor{currentstroke}%
\pgfsetdash{}{0pt}%
\pgfpathmoveto{\pgfqpoint{8.428060in}{1.852776in}}%
\pgfpathlineto{\pgfqpoint{8.428169in}{1.852802in}}%
\pgfusepath{stroke}%
\end{pgfscope}%
\begin{pgfscope}%
\pgfpathrectangle{\pgfqpoint{6.720588in}{1.750000in}}{\pgfqpoint{2.279412in}{2.004545in}}%
\pgfusepath{clip}%
\pgfsetbuttcap%
\pgfsetroundjoin%
\pgfsetlinewidth{0.308971pt}%
\definecolor{currentstroke}{rgb}{0.268510,0.009605,0.335427}%
\pgfsetstrokecolor{currentstroke}%
\pgfsetdash{}{0pt}%
\pgfpathmoveto{\pgfqpoint{8.428169in}{1.852802in}}%
\pgfpathlineto{\pgfqpoint{8.427956in}{1.852838in}}%
\pgfusepath{stroke}%
\end{pgfscope}%
\begin{pgfscope}%
\pgfpathrectangle{\pgfqpoint{6.720588in}{1.750000in}}{\pgfqpoint{2.279412in}{2.004545in}}%
\pgfusepath{clip}%
\pgfsetbuttcap%
\pgfsetroundjoin%
\pgfsetlinewidth{0.308909pt}%
\definecolor{currentstroke}{rgb}{0.268510,0.009605,0.335427}%
\pgfsetstrokecolor{currentstroke}%
\pgfsetdash{}{0pt}%
\pgfpathmoveto{\pgfqpoint{8.427956in}{1.852838in}}%
\pgfpathlineto{\pgfqpoint{8.427547in}{1.852879in}}%
\pgfusepath{stroke}%
\end{pgfscope}%
\begin{pgfscope}%
\pgfpathrectangle{\pgfqpoint{6.720588in}{1.750000in}}{\pgfqpoint{2.279412in}{2.004545in}}%
\pgfusepath{clip}%
\pgfsetbuttcap%
\pgfsetroundjoin%
\pgfsetlinewidth{0.308786pt}%
\definecolor{currentstroke}{rgb}{0.268510,0.009605,0.335427}%
\pgfsetstrokecolor{currentstroke}%
\pgfsetdash{}{0pt}%
\pgfpathmoveto{\pgfqpoint{8.427547in}{1.852879in}}%
\pgfpathlineto{\pgfqpoint{8.427381in}{1.852900in}}%
\pgfusepath{stroke}%
\end{pgfscope}%
\begin{pgfscope}%
\pgfpathrectangle{\pgfqpoint{6.720588in}{1.750000in}}{\pgfqpoint{2.279412in}{2.004545in}}%
\pgfusepath{clip}%
\pgfsetbuttcap%
\pgfsetroundjoin%
\pgfsetlinewidth{0.308736pt}%
\definecolor{currentstroke}{rgb}{0.268510,0.009605,0.335427}%
\pgfsetstrokecolor{currentstroke}%
\pgfsetdash{}{0pt}%
\pgfpathmoveto{\pgfqpoint{8.427381in}{1.852900in}}%
\pgfpathlineto{\pgfqpoint{8.427680in}{1.852894in}}%
\pgfusepath{stroke}%
\end{pgfscope}%
\begin{pgfscope}%
\pgfpathrectangle{\pgfqpoint{6.720588in}{1.750000in}}{\pgfqpoint{2.279412in}{2.004545in}}%
\pgfusepath{clip}%
\pgfsetbuttcap%
\pgfsetroundjoin%
\pgfsetlinewidth{0.308827pt}%
\definecolor{currentstroke}{rgb}{0.268510,0.009605,0.335427}%
\pgfsetstrokecolor{currentstroke}%
\pgfsetdash{}{0pt}%
\pgfpathmoveto{\pgfqpoint{8.427680in}{1.852894in}}%
\pgfpathlineto{\pgfqpoint{8.428082in}{1.852882in}}%
\pgfusepath{stroke}%
\end{pgfscope}%
\begin{pgfscope}%
\pgfpathrectangle{\pgfqpoint{6.720588in}{1.750000in}}{\pgfqpoint{2.279412in}{2.004545in}}%
\pgfusepath{clip}%
\pgfsetbuttcap%
\pgfsetroundjoin%
\pgfsetlinewidth{0.308950pt}%
\definecolor{currentstroke}{rgb}{0.268510,0.009605,0.335427}%
\pgfsetstrokecolor{currentstroke}%
\pgfsetdash{}{0pt}%
\pgfpathmoveto{\pgfqpoint{8.428082in}{1.852882in}}%
\pgfpathlineto{\pgfqpoint{8.428225in}{1.852884in}}%
\pgfusepath{stroke}%
\end{pgfscope}%
\begin{pgfscope}%
\pgfpathrectangle{\pgfqpoint{6.720588in}{1.750000in}}{\pgfqpoint{2.279412in}{2.004545in}}%
\pgfusepath{clip}%
\pgfsetbuttcap%
\pgfsetroundjoin%
\pgfsetlinewidth{0.308994pt}%
\definecolor{currentstroke}{rgb}{0.268510,0.009605,0.335427}%
\pgfsetstrokecolor{currentstroke}%
\pgfsetdash{}{0pt}%
\pgfpathmoveto{\pgfqpoint{8.428225in}{1.852884in}}%
\pgfpathlineto{\pgfqpoint{8.427969in}{1.852904in}}%
\pgfusepath{stroke}%
\end{pgfscope}%
\begin{pgfscope}%
\pgfpathrectangle{\pgfqpoint{6.720588in}{1.750000in}}{\pgfqpoint{2.279412in}{2.004545in}}%
\pgfusepath{clip}%
\pgfsetbuttcap%
\pgfsetroundjoin%
\pgfsetlinewidth{0.308917pt}%
\definecolor{currentstroke}{rgb}{0.268510,0.009605,0.335427}%
\pgfsetstrokecolor{currentstroke}%
\pgfsetdash{}{0pt}%
\pgfpathmoveto{\pgfqpoint{8.427969in}{1.852904in}}%
\pgfpathlineto{\pgfqpoint{8.427447in}{1.852935in}}%
\pgfusepath{stroke}%
\end{pgfscope}%
\begin{pgfscope}%
\pgfpathrectangle{\pgfqpoint{6.720588in}{1.750000in}}{\pgfqpoint{2.279412in}{2.004545in}}%
\pgfusepath{clip}%
\pgfsetbuttcap%
\pgfsetroundjoin%
\pgfsetlinewidth{0.308758pt}%
\definecolor{currentstroke}{rgb}{0.268510,0.009605,0.335427}%
\pgfsetstrokecolor{currentstroke}%
\pgfsetdash{}{0pt}%
\pgfpathmoveto{\pgfqpoint{8.427447in}{1.852935in}}%
\pgfpathlineto{\pgfqpoint{8.427236in}{1.852946in}}%
\pgfusepath{stroke}%
\end{pgfscope}%
\begin{pgfscope}%
\pgfpathrectangle{\pgfqpoint{6.720588in}{1.750000in}}{\pgfqpoint{2.279412in}{2.004545in}}%
\pgfusepath{clip}%
\pgfsetbuttcap%
\pgfsetroundjoin%
\pgfsetlinewidth{0.308693pt}%
\definecolor{currentstroke}{rgb}{0.268510,0.009605,0.335427}%
\pgfsetstrokecolor{currentstroke}%
\pgfsetdash{}{0pt}%
\pgfpathmoveto{\pgfqpoint{8.427236in}{1.852946in}}%
\pgfpathlineto{\pgfqpoint{8.427632in}{1.852927in}}%
\pgfusepath{stroke}%
\end{pgfscope}%
\begin{pgfscope}%
\pgfpathrectangle{\pgfqpoint{6.720588in}{1.750000in}}{\pgfqpoint{2.279412in}{2.004545in}}%
\pgfusepath{clip}%
\pgfsetbuttcap%
\pgfsetroundjoin%
\pgfsetlinewidth{0.308815pt}%
\definecolor{currentstroke}{rgb}{0.268510,0.009605,0.335427}%
\pgfsetstrokecolor{currentstroke}%
\pgfsetdash{}{0pt}%
\pgfpathmoveto{\pgfqpoint{8.427632in}{1.852927in}}%
\pgfpathlineto{\pgfqpoint{8.428136in}{1.852904in}}%
\pgfusepath{stroke}%
\end{pgfscope}%
\begin{pgfscope}%
\pgfpathrectangle{\pgfqpoint{6.720588in}{1.750000in}}{\pgfqpoint{2.279412in}{2.004545in}}%
\pgfusepath{clip}%
\pgfsetbuttcap%
\pgfsetroundjoin%
\pgfsetlinewidth{0.308968pt}%
\definecolor{currentstroke}{rgb}{0.268510,0.009605,0.335427}%
\pgfsetstrokecolor{currentstroke}%
\pgfsetdash{}{0pt}%
\pgfpathmoveto{\pgfqpoint{8.428136in}{1.852904in}}%
\pgfpathlineto{\pgfqpoint{8.428311in}{1.852899in}}%
\pgfusepath{stroke}%
\end{pgfscope}%
\begin{pgfscope}%
\pgfpathrectangle{\pgfqpoint{6.720588in}{1.750000in}}{\pgfqpoint{2.279412in}{2.004545in}}%
\pgfusepath{clip}%
\pgfsetbuttcap%
\pgfsetroundjoin%
\pgfsetlinewidth{0.309021pt}%
\definecolor{currentstroke}{rgb}{0.268510,0.009605,0.335427}%
\pgfsetstrokecolor{currentstroke}%
\pgfsetdash{}{0pt}%
\pgfpathmoveto{\pgfqpoint{8.428311in}{1.852899in}}%
\pgfpathlineto{\pgfqpoint{8.428002in}{1.852918in}}%
\pgfusepath{stroke}%
\end{pgfscope}%
\begin{pgfscope}%
\pgfpathrectangle{\pgfqpoint{6.720588in}{1.750000in}}{\pgfqpoint{2.279412in}{2.004545in}}%
\pgfusepath{clip}%
\pgfsetbuttcap%
\pgfsetroundjoin%
\pgfsetlinewidth{0.308928pt}%
\definecolor{currentstroke}{rgb}{0.268510,0.009605,0.335427}%
\pgfsetstrokecolor{currentstroke}%
\pgfsetdash{}{0pt}%
\pgfpathmoveto{\pgfqpoint{8.428002in}{1.852918in}}%
\pgfpathlineto{\pgfqpoint{8.427333in}{1.852953in}}%
\pgfusepath{stroke}%
\end{pgfscope}%
\begin{pgfscope}%
\pgfpathrectangle{\pgfqpoint{6.720588in}{1.750000in}}{\pgfqpoint{2.279412in}{2.004545in}}%
\pgfusepath{clip}%
\pgfsetbuttcap%
\pgfsetroundjoin%
\pgfsetlinewidth{0.308724pt}%
\definecolor{currentstroke}{rgb}{0.268510,0.009605,0.335427}%
\pgfsetstrokecolor{currentstroke}%
\pgfsetdash{}{0pt}%
\pgfpathmoveto{\pgfqpoint{8.427333in}{1.852953in}}%
\pgfpathlineto{\pgfqpoint{8.427333in}{1.852953in}}%
\pgfusepath{stroke}%
\end{pgfscope}%
\begin{pgfscope}%
\pgfpathrectangle{\pgfqpoint{6.720588in}{1.750000in}}{\pgfqpoint{2.279412in}{2.004545in}}%
\pgfusepath{clip}%
\pgfsetbuttcap%
\pgfsetroundjoin%
\pgfsetlinewidth{0.308724pt}%
\definecolor{currentstroke}{rgb}{0.268510,0.009605,0.335427}%
\pgfsetstrokecolor{currentstroke}%
\pgfsetdash{}{0pt}%
\pgfpathmoveto{\pgfqpoint{8.427333in}{1.852953in}}%
\pgfpathlineto{\pgfqpoint{8.427779in}{1.852929in}}%
\pgfusepath{stroke}%
\end{pgfscope}%
\begin{pgfscope}%
\pgfpathrectangle{\pgfqpoint{6.720588in}{1.750000in}}{\pgfqpoint{2.279412in}{2.004545in}}%
\pgfusepath{clip}%
\pgfsetbuttcap%
\pgfsetroundjoin%
\pgfsetlinewidth{0.308860pt}%
\definecolor{currentstroke}{rgb}{0.268510,0.009605,0.335427}%
\pgfsetstrokecolor{currentstroke}%
\pgfsetdash{}{0pt}%
\pgfpathmoveto{\pgfqpoint{8.427779in}{1.852929in}}%
\pgfpathlineto{\pgfqpoint{8.428196in}{1.852909in}}%
\pgfusepath{stroke}%
\end{pgfscope}%
\begin{pgfscope}%
\pgfpathrectangle{\pgfqpoint{6.720588in}{1.750000in}}{\pgfqpoint{2.279412in}{2.004545in}}%
\pgfusepath{clip}%
\pgfsetbuttcap%
\pgfsetroundjoin%
\pgfsetlinewidth{0.308987pt}%
\definecolor{currentstroke}{rgb}{0.268510,0.009605,0.335427}%
\pgfsetstrokecolor{currentstroke}%
\pgfsetdash{}{0pt}%
\pgfpathmoveto{\pgfqpoint{8.428196in}{1.852909in}}%
\pgfpathlineto{\pgfqpoint{8.428248in}{1.852909in}}%
\pgfusepath{stroke}%
\end{pgfscope}%
\begin{pgfscope}%
\pgfpathrectangle{\pgfqpoint{6.720588in}{1.750000in}}{\pgfqpoint{2.279412in}{2.004545in}}%
\pgfusepath{clip}%
\pgfsetbuttcap%
\pgfsetroundjoin%
\pgfsetlinewidth{0.309003pt}%
\definecolor{currentstroke}{rgb}{0.268510,0.009605,0.335427}%
\pgfsetstrokecolor{currentstroke}%
\pgfsetdash{}{0pt}%
\pgfpathmoveto{\pgfqpoint{8.428248in}{1.852909in}}%
\pgfpathlineto{\pgfqpoint{8.427840in}{1.852931in}}%
\pgfusepath{stroke}%
\end{pgfscope}%
\begin{pgfscope}%
\pgfpathrectangle{\pgfqpoint{6.720588in}{1.750000in}}{\pgfqpoint{2.279412in}{2.004545in}}%
\pgfusepath{clip}%
\pgfsetbuttcap%
\pgfsetroundjoin%
\pgfsetlinewidth{0.308879pt}%
\definecolor{currentstroke}{rgb}{0.268510,0.009605,0.335427}%
\pgfsetstrokecolor{currentstroke}%
\pgfsetdash{}{0pt}%
\pgfpathmoveto{\pgfqpoint{8.427840in}{1.852931in}}%
\pgfpathlineto{\pgfqpoint{8.427234in}{1.852960in}}%
\pgfusepath{stroke}%
\end{pgfscope}%
\begin{pgfscope}%
\pgfpathrectangle{\pgfqpoint{6.720588in}{1.750000in}}{\pgfqpoint{2.279412in}{2.004545in}}%
\pgfusepath{clip}%
\pgfsetbuttcap%
\pgfsetroundjoin%
\pgfsetlinewidth{0.308693pt}%
\definecolor{currentstroke}{rgb}{0.268510,0.009605,0.335427}%
\pgfsetstrokecolor{currentstroke}%
\pgfsetdash{}{0pt}%
\pgfpathmoveto{\pgfqpoint{8.427234in}{1.852960in}}%
\pgfpathlineto{\pgfqpoint{8.427193in}{1.852958in}}%
\pgfusepath{stroke}%
\end{pgfscope}%
\begin{pgfscope}%
\pgfpathrectangle{\pgfqpoint{6.720588in}{1.750000in}}{\pgfqpoint{2.279412in}{2.004545in}}%
\pgfusepath{clip}%
\pgfsetbuttcap%
\pgfsetroundjoin%
\pgfsetlinewidth{0.308681pt}%
\definecolor{currentstroke}{rgb}{0.268510,0.009605,0.335427}%
\pgfsetstrokecolor{currentstroke}%
\pgfsetdash{}{0pt}%
\pgfpathmoveto{\pgfqpoint{8.427193in}{1.852958in}}%
\pgfpathlineto{\pgfqpoint{8.427769in}{1.852928in}}%
\pgfusepath{stroke}%
\end{pgfscope}%
\begin{pgfscope}%
\pgfpathrectangle{\pgfqpoint{6.720588in}{1.750000in}}{\pgfqpoint{2.279412in}{2.004545in}}%
\pgfusepath{clip}%
\pgfsetbuttcap%
\pgfsetroundjoin%
\pgfsetlinewidth{0.308857pt}%
\definecolor{currentstroke}{rgb}{0.268510,0.009605,0.335427}%
\pgfsetstrokecolor{currentstroke}%
\pgfsetdash{}{0pt}%
\pgfpathmoveto{\pgfqpoint{8.427769in}{1.852928in}}%
\pgfpathlineto{\pgfqpoint{8.428277in}{1.852904in}}%
\pgfusepath{stroke}%
\end{pgfscope}%
\begin{pgfscope}%
\pgfpathrectangle{\pgfqpoint{6.720588in}{1.750000in}}{\pgfqpoint{2.279412in}{2.004545in}}%
\pgfusepath{clip}%
\pgfsetbuttcap%
\pgfsetroundjoin%
\pgfsetlinewidth{0.309011pt}%
\definecolor{currentstroke}{rgb}{0.268510,0.009605,0.335427}%
\pgfsetstrokecolor{currentstroke}%
\pgfsetdash{}{0pt}%
\pgfpathmoveto{\pgfqpoint{8.428277in}{1.852904in}}%
\pgfpathlineto{\pgfqpoint{8.428342in}{1.852904in}}%
\pgfusepath{stroke}%
\end{pgfscope}%
\begin{pgfscope}%
\pgfpathrectangle{\pgfqpoint{6.720588in}{1.750000in}}{\pgfqpoint{2.279412in}{2.004545in}}%
\pgfusepath{clip}%
\pgfsetbuttcap%
\pgfsetroundjoin%
\pgfsetlinewidth{0.309031pt}%
\definecolor{currentstroke}{rgb}{0.268510,0.009605,0.335427}%
\pgfsetstrokecolor{currentstroke}%
\pgfsetdash{}{0pt}%
\pgfpathmoveto{\pgfqpoint{8.428342in}{1.852904in}}%
\pgfpathlineto{\pgfqpoint{8.427847in}{1.852931in}}%
\pgfusepath{stroke}%
\end{pgfscope}%
\begin{pgfscope}%
\pgfpathrectangle{\pgfqpoint{6.720588in}{1.750000in}}{\pgfqpoint{2.279412in}{2.004545in}}%
\pgfusepath{clip}%
\pgfsetbuttcap%
\pgfsetroundjoin%
\pgfsetlinewidth{0.308881pt}%
\definecolor{currentstroke}{rgb}{0.268510,0.009605,0.335427}%
\pgfsetstrokecolor{currentstroke}%
\pgfsetdash{}{0pt}%
\pgfpathmoveto{\pgfqpoint{8.427847in}{1.852931in}}%
\pgfpathlineto{\pgfqpoint{8.427064in}{1.852968in}}%
\pgfusepath{stroke}%
\end{pgfscope}%
\begin{pgfscope}%
\pgfpathrectangle{\pgfqpoint{6.720588in}{1.750000in}}{\pgfqpoint{2.279412in}{2.004545in}}%
\pgfusepath{clip}%
\pgfsetbuttcap%
\pgfsetroundjoin%
\pgfsetlinewidth{0.308641pt}%
\definecolor{currentstroke}{rgb}{0.268510,0.009605,0.335427}%
\pgfsetstrokecolor{currentstroke}%
\pgfsetdash{}{0pt}%
\pgfpathmoveto{\pgfqpoint{8.427064in}{1.852968in}}%
\pgfpathlineto{\pgfqpoint{8.427064in}{1.852968in}}%
\pgfusepath{stroke}%
\end{pgfscope}%
\begin{pgfscope}%
\pgfpathrectangle{\pgfqpoint{6.720588in}{1.750000in}}{\pgfqpoint{2.279412in}{2.004545in}}%
\pgfusepath{clip}%
\pgfsetbuttcap%
\pgfsetroundjoin%
\pgfsetlinewidth{0.308641pt}%
\definecolor{currentstroke}{rgb}{0.268510,0.009605,0.335427}%
\pgfsetstrokecolor{currentstroke}%
\pgfsetdash{}{0pt}%
\pgfpathmoveto{\pgfqpoint{8.427064in}{1.852968in}}%
\pgfpathlineto{\pgfqpoint{8.427782in}{1.852930in}}%
\pgfusepath{stroke}%
\end{pgfscope}%
\begin{pgfscope}%
\pgfpathrectangle{\pgfqpoint{6.720588in}{1.750000in}}{\pgfqpoint{2.279412in}{2.004545in}}%
\pgfusepath{clip}%
\pgfsetbuttcap%
\pgfsetroundjoin%
\pgfsetlinewidth{0.308861pt}%
\definecolor{currentstroke}{rgb}{0.268510,0.009605,0.335427}%
\pgfsetstrokecolor{currentstroke}%
\pgfsetdash{}{0pt}%
\pgfpathmoveto{\pgfqpoint{8.427782in}{1.852930in}}%
\pgfpathlineto{\pgfqpoint{8.428360in}{1.852902in}}%
\pgfusepath{stroke}%
\end{pgfscope}%
\begin{pgfscope}%
\pgfpathrectangle{\pgfqpoint{6.720588in}{1.750000in}}{\pgfqpoint{2.279412in}{2.004545in}}%
\pgfusepath{clip}%
\pgfsetbuttcap%
\pgfsetroundjoin%
\pgfsetlinewidth{0.309037pt}%
\definecolor{currentstroke}{rgb}{0.268510,0.009605,0.335427}%
\pgfsetstrokecolor{currentstroke}%
\pgfsetdash{}{0pt}%
\pgfpathmoveto{\pgfqpoint{8.428360in}{1.852902in}}%
\pgfpathlineto{\pgfqpoint{8.428418in}{1.852902in}}%
\pgfusepath{stroke}%
\end{pgfscope}%
\begin{pgfscope}%
\pgfpathrectangle{\pgfqpoint{6.720588in}{1.750000in}}{\pgfqpoint{2.279412in}{2.004545in}}%
\pgfusepath{clip}%
\pgfsetbuttcap%
\pgfsetroundjoin%
\pgfsetlinewidth{0.309055pt}%
\definecolor{currentstroke}{rgb}{0.268510,0.009605,0.335427}%
\pgfsetstrokecolor{currentstroke}%
\pgfsetdash{}{0pt}%
\pgfpathmoveto{\pgfqpoint{8.428418in}{1.852902in}}%
\pgfpathlineto{\pgfqpoint{8.427824in}{1.852934in}}%
\pgfusepath{stroke}%
\end{pgfscope}%
\begin{pgfscope}%
\pgfpathrectangle{\pgfqpoint{6.720588in}{1.750000in}}{\pgfqpoint{2.279412in}{2.004545in}}%
\pgfusepath{clip}%
\pgfsetbuttcap%
\pgfsetroundjoin%
\pgfsetlinewidth{0.308874pt}%
\definecolor{currentstroke}{rgb}{0.268510,0.009605,0.335427}%
\pgfsetstrokecolor{currentstroke}%
\pgfsetdash{}{0pt}%
\pgfpathmoveto{\pgfqpoint{8.427824in}{1.852934in}}%
\pgfpathlineto{\pgfqpoint{8.427824in}{1.852934in}}%
\pgfusepath{stroke}%
\end{pgfscope}%
\begin{pgfscope}%
\pgfpathrectangle{\pgfqpoint{6.720588in}{1.750000in}}{\pgfqpoint{2.279412in}{2.004545in}}%
\pgfusepath{clip}%
\pgfsetbuttcap%
\pgfsetroundjoin%
\pgfsetlinewidth{0.308874pt}%
\definecolor{currentstroke}{rgb}{0.268510,0.009605,0.335427}%
\pgfsetstrokecolor{currentstroke}%
\pgfsetdash{}{0pt}%
\pgfpathmoveto{\pgfqpoint{8.427824in}{1.852934in}}%
\pgfpathlineto{\pgfqpoint{8.427705in}{1.852939in}}%
\pgfusepath{stroke}%
\end{pgfscope}%
\begin{pgfscope}%
\pgfpathrectangle{\pgfqpoint{6.720588in}{1.750000in}}{\pgfqpoint{2.279412in}{2.004545in}}%
\pgfusepath{clip}%
\pgfsetbuttcap%
\pgfsetroundjoin%
\pgfsetlinewidth{0.308838pt}%
\definecolor{currentstroke}{rgb}{0.268510,0.009605,0.335427}%
\pgfsetstrokecolor{currentstroke}%
\pgfsetdash{}{0pt}%
\pgfpathmoveto{\pgfqpoint{8.427705in}{1.852939in}}%
\pgfpathlineto{\pgfqpoint{8.427688in}{1.852939in}}%
\pgfusepath{stroke}%
\end{pgfscope}%
\begin{pgfscope}%
\pgfpathrectangle{\pgfqpoint{6.720588in}{1.750000in}}{\pgfqpoint{2.279412in}{2.004545in}}%
\pgfusepath{clip}%
\pgfsetbuttcap%
\pgfsetroundjoin%
\pgfsetlinewidth{0.308832pt}%
\definecolor{currentstroke}{rgb}{0.268510,0.009605,0.335427}%
\pgfsetstrokecolor{currentstroke}%
\pgfsetdash{}{0pt}%
\pgfpathmoveto{\pgfqpoint{8.427688in}{1.852939in}}%
\pgfpathlineto{\pgfqpoint{8.427799in}{1.852932in}}%
\pgfusepath{stroke}%
\end{pgfscope}%
\begin{pgfscope}%
\pgfpathrectangle{\pgfqpoint{6.720588in}{1.750000in}}{\pgfqpoint{2.279412in}{2.004545in}}%
\pgfusepath{clip}%
\pgfsetbuttcap%
\pgfsetroundjoin%
\pgfsetlinewidth{0.308866pt}%
\definecolor{currentstroke}{rgb}{0.268510,0.009605,0.335427}%
\pgfsetstrokecolor{currentstroke}%
\pgfsetdash{}{0pt}%
\pgfpathmoveto{\pgfqpoint{8.427799in}{1.852932in}}%
\pgfpathlineto{\pgfqpoint{8.427927in}{1.852926in}}%
\pgfusepath{stroke}%
\end{pgfscope}%
\begin{pgfscope}%
\pgfpathrectangle{\pgfqpoint{6.720588in}{1.750000in}}{\pgfqpoint{2.279412in}{2.004545in}}%
\pgfusepath{clip}%
\pgfsetbuttcap%
\pgfsetroundjoin%
\pgfsetlinewidth{0.308905pt}%
\definecolor{currentstroke}{rgb}{0.268510,0.009605,0.335427}%
\pgfsetstrokecolor{currentstroke}%
\pgfsetdash{}{0pt}%
\pgfpathmoveto{\pgfqpoint{8.427927in}{1.852926in}}%
\pgfpathlineto{\pgfqpoint{8.427946in}{1.852926in}}%
\pgfusepath{stroke}%
\end{pgfscope}%
\begin{pgfscope}%
\pgfpathrectangle{\pgfqpoint{6.720588in}{1.750000in}}{\pgfqpoint{2.279412in}{2.004545in}}%
\pgfusepath{clip}%
\pgfsetbuttcap%
\pgfsetroundjoin%
\pgfsetlinewidth{0.308911pt}%
\definecolor{currentstroke}{rgb}{0.268510,0.009605,0.335427}%
\pgfsetstrokecolor{currentstroke}%
\pgfsetdash{}{0pt}%
\pgfpathmoveto{\pgfqpoint{8.427946in}{1.852926in}}%
\pgfpathlineto{\pgfqpoint{8.427826in}{1.852932in}}%
\pgfusepath{stroke}%
\end{pgfscope}%
\begin{pgfscope}%
\pgfpathrectangle{\pgfqpoint{6.720588in}{1.750000in}}{\pgfqpoint{2.279412in}{2.004545in}}%
\pgfusepath{clip}%
\pgfsetbuttcap%
\pgfsetroundjoin%
\pgfsetlinewidth{0.308875pt}%
\definecolor{currentstroke}{rgb}{0.268510,0.009605,0.335427}%
\pgfsetstrokecolor{currentstroke}%
\pgfsetdash{}{0pt}%
\pgfpathmoveto{\pgfqpoint{8.427826in}{1.852932in}}%
\pgfpathlineto{\pgfqpoint{8.427672in}{1.852939in}}%
\pgfusepath{stroke}%
\end{pgfscope}%
\begin{pgfscope}%
\pgfpathrectangle{\pgfqpoint{6.720588in}{1.750000in}}{\pgfqpoint{2.279412in}{2.004545in}}%
\pgfusepath{clip}%
\pgfsetbuttcap%
\pgfsetroundjoin%
\pgfsetlinewidth{0.308827pt}%
\definecolor{currentstroke}{rgb}{0.268510,0.009605,0.335427}%
\pgfsetstrokecolor{currentstroke}%
\pgfsetdash{}{0pt}%
\pgfpathmoveto{\pgfqpoint{8.427672in}{1.852939in}}%
\pgfpathlineto{\pgfqpoint{8.427651in}{1.852939in}}%
\pgfusepath{stroke}%
\end{pgfscope}%
\begin{pgfscope}%
\pgfpathrectangle{\pgfqpoint{6.720588in}{1.750000in}}{\pgfqpoint{2.279412in}{2.004545in}}%
\pgfusepath{clip}%
\pgfsetbuttcap%
\pgfsetroundjoin%
\pgfsetlinewidth{0.308821pt}%
\definecolor{currentstroke}{rgb}{0.268510,0.009605,0.335427}%
\pgfsetstrokecolor{currentstroke}%
\pgfsetdash{}{0pt}%
\pgfpathmoveto{\pgfqpoint{8.427651in}{1.852939in}}%
\pgfpathlineto{\pgfqpoint{8.427797in}{1.852931in}}%
\pgfusepath{stroke}%
\end{pgfscope}%
\begin{pgfscope}%
\pgfpathrectangle{\pgfqpoint{6.720588in}{1.750000in}}{\pgfqpoint{2.279412in}{2.004545in}}%
\pgfusepath{clip}%
\pgfsetbuttcap%
\pgfsetroundjoin%
\pgfsetlinewidth{0.308865pt}%
\definecolor{currentstroke}{rgb}{0.268510,0.009605,0.335427}%
\pgfsetstrokecolor{currentstroke}%
\pgfsetdash{}{0pt}%
\pgfpathmoveto{\pgfqpoint{8.427797in}{1.852931in}}%
\pgfpathlineto{\pgfqpoint{8.427959in}{1.852924in}}%
\pgfusepath{stroke}%
\end{pgfscope}%
\begin{pgfscope}%
\pgfpathrectangle{\pgfqpoint{6.720588in}{1.750000in}}{\pgfqpoint{2.279412in}{2.004545in}}%
\pgfusepath{clip}%
\pgfsetbuttcap%
\pgfsetroundjoin%
\pgfsetlinewidth{0.308915pt}%
\definecolor{currentstroke}{rgb}{0.268510,0.009605,0.335427}%
\pgfsetstrokecolor{currentstroke}%
\pgfsetdash{}{0pt}%
\pgfpathmoveto{\pgfqpoint{8.427959in}{1.852924in}}%
\pgfpathlineto{\pgfqpoint{8.427982in}{1.852923in}}%
\pgfusepath{stroke}%
\end{pgfscope}%
\begin{pgfscope}%
\pgfpathrectangle{\pgfqpoint{6.720588in}{1.750000in}}{\pgfqpoint{2.279412in}{2.004545in}}%
\pgfusepath{clip}%
\pgfsetbuttcap%
\pgfsetroundjoin%
\pgfsetlinewidth{0.308922pt}%
\definecolor{currentstroke}{rgb}{0.268510,0.009605,0.335427}%
\pgfsetstrokecolor{currentstroke}%
\pgfsetdash{}{0pt}%
\pgfpathmoveto{\pgfqpoint{8.427982in}{1.852923in}}%
\pgfpathlineto{\pgfqpoint{8.427829in}{1.852932in}}%
\pgfusepath{stroke}%
\end{pgfscope}%
\begin{pgfscope}%
\pgfpathrectangle{\pgfqpoint{6.720588in}{1.750000in}}{\pgfqpoint{2.279412in}{2.004545in}}%
\pgfusepath{clip}%
\pgfsetbuttcap%
\pgfsetroundjoin%
\pgfsetlinewidth{0.308875pt}%
\definecolor{currentstroke}{rgb}{0.268510,0.009605,0.335427}%
\pgfsetstrokecolor{currentstroke}%
\pgfsetdash{}{0pt}%
\pgfpathmoveto{\pgfqpoint{8.427829in}{1.852932in}}%
\pgfpathlineto{\pgfqpoint{8.427628in}{1.852941in}}%
\pgfusepath{stroke}%
\end{pgfscope}%
\begin{pgfscope}%
\pgfpathrectangle{\pgfqpoint{6.720588in}{1.750000in}}{\pgfqpoint{2.279412in}{2.004545in}}%
\pgfusepath{clip}%
\pgfsetbuttcap%
\pgfsetroundjoin%
\pgfsetlinewidth{0.308814pt}%
\definecolor{currentstroke}{rgb}{0.268510,0.009605,0.335427}%
\pgfsetstrokecolor{currentstroke}%
\pgfsetdash{}{0pt}%
\pgfpathmoveto{\pgfqpoint{8.427628in}{1.852941in}}%
\pgfpathlineto{\pgfqpoint{8.427603in}{1.852941in}}%
\pgfusepath{stroke}%
\end{pgfscope}%
\begin{pgfscope}%
\pgfpathrectangle{\pgfqpoint{6.720588in}{1.750000in}}{\pgfqpoint{2.279412in}{2.004545in}}%
\pgfusepath{clip}%
\pgfsetbuttcap%
\pgfsetroundjoin%
\pgfsetlinewidth{0.308806pt}%
\definecolor{currentstroke}{rgb}{0.268510,0.009605,0.335427}%
\pgfsetstrokecolor{currentstroke}%
\pgfsetdash{}{0pt}%
\pgfpathmoveto{\pgfqpoint{8.427603in}{1.852941in}}%
\pgfpathlineto{\pgfqpoint{8.427793in}{1.852931in}}%
\pgfusepath{stroke}%
\end{pgfscope}%
\begin{pgfscope}%
\pgfpathrectangle{\pgfqpoint{6.720588in}{1.750000in}}{\pgfqpoint{2.279412in}{2.004545in}}%
\pgfusepath{clip}%
\pgfsetbuttcap%
\pgfsetroundjoin%
\pgfsetlinewidth{0.308864pt}%
\definecolor{currentstroke}{rgb}{0.268510,0.009605,0.335427}%
\pgfsetstrokecolor{currentstroke}%
\pgfsetdash{}{0pt}%
\pgfpathmoveto{\pgfqpoint{8.427793in}{1.852931in}}%
\pgfpathlineto{\pgfqpoint{8.427999in}{1.852921in}}%
\pgfusepath{stroke}%
\end{pgfscope}%
\begin{pgfscope}%
\pgfpathrectangle{\pgfqpoint{6.720588in}{1.750000in}}{\pgfqpoint{2.279412in}{2.004545in}}%
\pgfusepath{clip}%
\pgfsetbuttcap%
\pgfsetroundjoin%
\pgfsetlinewidth{0.308927pt}%
\definecolor{currentstroke}{rgb}{0.268510,0.009605,0.335427}%
\pgfsetstrokecolor{currentstroke}%
\pgfsetdash{}{0pt}%
\pgfpathmoveto{\pgfqpoint{8.427999in}{1.852921in}}%
\pgfpathlineto{\pgfqpoint{8.428028in}{1.852921in}}%
\pgfusepath{stroke}%
\end{pgfscope}%
\begin{pgfscope}%
\pgfpathrectangle{\pgfqpoint{6.720588in}{1.750000in}}{\pgfqpoint{2.279412in}{2.004545in}}%
\pgfusepath{clip}%
\pgfsetbuttcap%
\pgfsetroundjoin%
\pgfsetlinewidth{0.308936pt}%
\definecolor{currentstroke}{rgb}{0.268510,0.009605,0.335427}%
\pgfsetstrokecolor{currentstroke}%
\pgfsetdash{}{0pt}%
\pgfpathmoveto{\pgfqpoint{8.428028in}{1.852921in}}%
\pgfpathlineto{\pgfqpoint{8.427832in}{1.852931in}}%
\pgfusepath{stroke}%
\end{pgfscope}%
\begin{pgfscope}%
\pgfpathrectangle{\pgfqpoint{6.720588in}{1.750000in}}{\pgfqpoint{2.279412in}{2.004545in}}%
\pgfusepath{clip}%
\pgfsetbuttcap%
\pgfsetroundjoin%
\pgfsetlinewidth{0.308876pt}%
\definecolor{currentstroke}{rgb}{0.268510,0.009605,0.335427}%
\pgfsetstrokecolor{currentstroke}%
\pgfsetdash{}{0pt}%
\pgfpathmoveto{\pgfqpoint{8.427832in}{1.852931in}}%
\pgfpathlineto{\pgfqpoint{8.427570in}{1.852944in}}%
\pgfusepath{stroke}%
\end{pgfscope}%
\begin{pgfscope}%
\pgfpathrectangle{\pgfqpoint{6.720588in}{1.750000in}}{\pgfqpoint{2.279412in}{2.004545in}}%
\pgfusepath{clip}%
\pgfsetbuttcap%
\pgfsetroundjoin%
\pgfsetlinewidth{0.308796pt}%
\definecolor{currentstroke}{rgb}{0.268510,0.009605,0.335427}%
\pgfsetstrokecolor{currentstroke}%
\pgfsetdash{}{0pt}%
\pgfpathmoveto{\pgfqpoint{8.427570in}{1.852944in}}%
\pgfpathlineto{\pgfqpoint{8.427540in}{1.852944in}}%
\pgfusepath{stroke}%
\end{pgfscope}%
\begin{pgfscope}%
\pgfpathrectangle{\pgfqpoint{6.720588in}{1.750000in}}{\pgfqpoint{2.279412in}{2.004545in}}%
\pgfusepath{clip}%
\pgfsetbuttcap%
\pgfsetroundjoin%
\pgfsetlinewidth{0.308787pt}%
\definecolor{currentstroke}{rgb}{0.268510,0.009605,0.335427}%
\pgfsetstrokecolor{currentstroke}%
\pgfsetdash{}{0pt}%
\pgfpathmoveto{\pgfqpoint{8.427540in}{1.852944in}}%
\pgfpathlineto{\pgfqpoint{8.427789in}{1.852931in}}%
\pgfusepath{stroke}%
\end{pgfscope}%
\begin{pgfscope}%
\pgfpathrectangle{\pgfqpoint{6.720588in}{1.750000in}}{\pgfqpoint{2.279412in}{2.004545in}}%
\pgfusepath{clip}%
\pgfsetbuttcap%
\pgfsetroundjoin%
\pgfsetlinewidth{0.308863pt}%
\definecolor{currentstroke}{rgb}{0.268510,0.009605,0.335427}%
\pgfsetstrokecolor{currentstroke}%
\pgfsetdash{}{0pt}%
\pgfpathmoveto{\pgfqpoint{8.427789in}{1.852931in}}%
\pgfpathlineto{\pgfqpoint{8.428049in}{1.852918in}}%
\pgfusepath{stroke}%
\end{pgfscope}%
\begin{pgfscope}%
\pgfpathrectangle{\pgfqpoint{6.720588in}{1.750000in}}{\pgfqpoint{2.279412in}{2.004545in}}%
\pgfusepath{clip}%
\pgfsetbuttcap%
\pgfsetroundjoin%
\pgfsetlinewidth{0.308942pt}%
\definecolor{currentstroke}{rgb}{0.268510,0.009605,0.335427}%
\pgfsetstrokecolor{currentstroke}%
\pgfsetdash{}{0pt}%
\pgfpathmoveto{\pgfqpoint{8.428049in}{1.852918in}}%
\pgfpathlineto{\pgfqpoint{8.428085in}{1.852918in}}%
\pgfusepath{stroke}%
\end{pgfscope}%
\begin{pgfscope}%
\pgfpathrectangle{\pgfqpoint{6.720588in}{1.750000in}}{\pgfqpoint{2.279412in}{2.004545in}}%
\pgfusepath{clip}%
\pgfsetbuttcap%
\pgfsetroundjoin%
\pgfsetlinewidth{0.308953pt}%
\definecolor{currentstroke}{rgb}{0.268510,0.009605,0.335427}%
\pgfsetstrokecolor{currentstroke}%
\pgfsetdash{}{0pt}%
\pgfpathmoveto{\pgfqpoint{8.428085in}{1.852918in}}%
\pgfpathlineto{\pgfqpoint{8.427835in}{1.852931in}}%
\pgfusepath{stroke}%
\end{pgfscope}%
\begin{pgfscope}%
\pgfpathrectangle{\pgfqpoint{6.720588in}{1.750000in}}{\pgfqpoint{2.279412in}{2.004545in}}%
\pgfusepath{clip}%
\pgfsetbuttcap%
\pgfsetroundjoin%
\pgfsetlinewidth{0.308877pt}%
\definecolor{currentstroke}{rgb}{0.268510,0.009605,0.335427}%
\pgfsetstrokecolor{currentstroke}%
\pgfsetdash{}{0pt}%
\pgfpathmoveto{\pgfqpoint{8.427835in}{1.852931in}}%
\pgfpathlineto{\pgfqpoint{8.427493in}{1.852948in}}%
\pgfusepath{stroke}%
\end{pgfscope}%
\begin{pgfscope}%
\pgfpathrectangle{\pgfqpoint{6.720588in}{1.750000in}}{\pgfqpoint{2.279412in}{2.004545in}}%
\pgfusepath{clip}%
\pgfsetbuttcap%
\pgfsetroundjoin%
\pgfsetlinewidth{0.308773pt}%
\definecolor{currentstroke}{rgb}{0.268510,0.009605,0.335427}%
\pgfsetstrokecolor{currentstroke}%
\pgfsetdash{}{0pt}%
\pgfpathmoveto{\pgfqpoint{8.427493in}{1.852948in}}%
\pgfpathlineto{\pgfqpoint{8.427457in}{1.852948in}}%
\pgfusepath{stroke}%
\end{pgfscope}%
\begin{pgfscope}%
\pgfpathrectangle{\pgfqpoint{6.720588in}{1.750000in}}{\pgfqpoint{2.279412in}{2.004545in}}%
\pgfusepath{clip}%
\pgfsetbuttcap%
\pgfsetroundjoin%
\pgfsetlinewidth{0.308762pt}%
\definecolor{currentstroke}{rgb}{0.268510,0.009605,0.335427}%
\pgfsetstrokecolor{currentstroke}%
\pgfsetdash{}{0pt}%
\pgfpathmoveto{\pgfqpoint{8.427457in}{1.852948in}}%
\pgfpathlineto{\pgfqpoint{8.427783in}{1.852930in}}%
\pgfusepath{stroke}%
\end{pgfscope}%
\begin{pgfscope}%
\pgfpathrectangle{\pgfqpoint{6.720588in}{1.750000in}}{\pgfqpoint{2.279412in}{2.004545in}}%
\pgfusepath{clip}%
\pgfsetbuttcap%
\pgfsetroundjoin%
\pgfsetlinewidth{0.308861pt}%
\definecolor{currentstroke}{rgb}{0.268510,0.009605,0.335427}%
\pgfsetstrokecolor{currentstroke}%
\pgfsetdash{}{0pt}%
\pgfpathmoveto{\pgfqpoint{8.427783in}{1.852930in}}%
\pgfpathlineto{\pgfqpoint{8.428110in}{1.852915in}}%
\pgfusepath{stroke}%
\end{pgfscope}%
\begin{pgfscope}%
\pgfpathrectangle{\pgfqpoint{6.720588in}{1.750000in}}{\pgfqpoint{2.279412in}{2.004545in}}%
\pgfusepath{clip}%
\pgfsetbuttcap%
\pgfsetroundjoin%
\pgfsetlinewidth{0.308961pt}%
\definecolor{currentstroke}{rgb}{0.268510,0.009605,0.335427}%
\pgfsetstrokecolor{currentstroke}%
\pgfsetdash{}{0pt}%
\pgfpathmoveto{\pgfqpoint{8.428110in}{1.852915in}}%
\pgfpathlineto{\pgfqpoint{8.428153in}{1.852914in}}%
\pgfusepath{stroke}%
\end{pgfscope}%
\begin{pgfscope}%
\pgfpathrectangle{\pgfqpoint{6.720588in}{1.750000in}}{\pgfqpoint{2.279412in}{2.004545in}}%
\pgfusepath{clip}%
\pgfsetbuttcap%
\pgfsetroundjoin%
\pgfsetlinewidth{0.308974pt}%
\definecolor{currentstroke}{rgb}{0.268510,0.009605,0.335427}%
\pgfsetstrokecolor{currentstroke}%
\pgfsetdash{}{0pt}%
\pgfpathmoveto{\pgfqpoint{8.428153in}{1.852914in}}%
\pgfpathlineto{\pgfqpoint{8.427839in}{1.852931in}}%
\pgfusepath{stroke}%
\end{pgfscope}%
\begin{pgfscope}%
\pgfpathrectangle{\pgfqpoint{6.720588in}{1.750000in}}{\pgfqpoint{2.279412in}{2.004545in}}%
\pgfusepath{clip}%
\pgfsetbuttcap%
\pgfsetroundjoin%
\pgfsetlinewidth{0.308879pt}%
\definecolor{currentstroke}{rgb}{0.268510,0.009605,0.335427}%
\pgfsetstrokecolor{currentstroke}%
\pgfsetdash{}{0pt}%
\pgfpathmoveto{\pgfqpoint{8.427839in}{1.852931in}}%
\pgfpathlineto{\pgfqpoint{8.427393in}{1.852953in}}%
\pgfusepath{stroke}%
\end{pgfscope}%
\begin{pgfscope}%
\pgfpathrectangle{\pgfqpoint{6.720588in}{1.750000in}}{\pgfqpoint{2.279412in}{2.004545in}}%
\pgfusepath{clip}%
\pgfsetbuttcap%
\pgfsetroundjoin%
\pgfsetlinewidth{0.308742pt}%
\definecolor{currentstroke}{rgb}{0.268510,0.009605,0.335427}%
\pgfsetstrokecolor{currentstroke}%
\pgfsetdash{}{0pt}%
\pgfpathmoveto{\pgfqpoint{8.427393in}{1.852953in}}%
\pgfpathlineto{\pgfqpoint{8.427351in}{1.852952in}}%
\pgfusepath{stroke}%
\end{pgfscope}%
\begin{pgfscope}%
\pgfpathrectangle{\pgfqpoint{6.720588in}{1.750000in}}{\pgfqpoint{2.279412in}{2.004545in}}%
\pgfusepath{clip}%
\pgfsetbuttcap%
\pgfsetroundjoin%
\pgfsetlinewidth{0.308729pt}%
\definecolor{currentstroke}{rgb}{0.268510,0.009605,0.335427}%
\pgfsetstrokecolor{currentstroke}%
\pgfsetdash{}{0pt}%
\pgfpathmoveto{\pgfqpoint{8.427351in}{1.852952in}}%
\pgfpathlineto{\pgfqpoint{8.427776in}{1.852930in}}%
\pgfusepath{stroke}%
\end{pgfscope}%
\begin{pgfscope}%
\pgfpathrectangle{\pgfqpoint{6.720588in}{1.750000in}}{\pgfqpoint{2.279412in}{2.004545in}}%
\pgfusepath{clip}%
\pgfsetbuttcap%
\pgfsetroundjoin%
\pgfsetlinewidth{0.308859pt}%
\definecolor{currentstroke}{rgb}{0.268510,0.009605,0.335427}%
\pgfsetstrokecolor{currentstroke}%
\pgfsetdash{}{0pt}%
\pgfpathmoveto{\pgfqpoint{8.427776in}{1.852930in}}%
\pgfpathlineto{\pgfqpoint{8.428180in}{1.852910in}}%
\pgfusepath{stroke}%
\end{pgfscope}%
\begin{pgfscope}%
\pgfpathrectangle{\pgfqpoint{6.720588in}{1.750000in}}{\pgfqpoint{2.279412in}{2.004545in}}%
\pgfusepath{clip}%
\pgfsetbuttcap%
\pgfsetroundjoin%
\pgfsetlinewidth{0.308982pt}%
\definecolor{currentstroke}{rgb}{0.268510,0.009605,0.335427}%
\pgfsetstrokecolor{currentstroke}%
\pgfsetdash{}{0pt}%
\pgfpathmoveto{\pgfqpoint{8.428180in}{1.852910in}}%
\pgfpathlineto{\pgfqpoint{8.428235in}{1.852910in}}%
\pgfusepath{stroke}%
\end{pgfscope}%
\begin{pgfscope}%
\pgfpathrectangle{\pgfqpoint{6.720588in}{1.750000in}}{\pgfqpoint{2.279412in}{2.004545in}}%
\pgfusepath{clip}%
\pgfsetbuttcap%
\pgfsetroundjoin%
\pgfsetlinewidth{0.308999pt}%
\definecolor{currentstroke}{rgb}{0.268510,0.009605,0.335427}%
\pgfsetstrokecolor{currentstroke}%
\pgfsetdash{}{0pt}%
\pgfpathmoveto{\pgfqpoint{8.428235in}{1.852910in}}%
\pgfpathlineto{\pgfqpoint{8.427845in}{1.852931in}}%
\pgfusepath{stroke}%
\end{pgfscope}%
\begin{pgfscope}%
\pgfpathrectangle{\pgfqpoint{6.720588in}{1.750000in}}{\pgfqpoint{2.279412in}{2.004545in}}%
\pgfusepath{clip}%
\pgfsetbuttcap%
\pgfsetroundjoin%
\pgfsetlinewidth{0.308880pt}%
\definecolor{currentstroke}{rgb}{0.268510,0.009605,0.335427}%
\pgfsetstrokecolor{currentstroke}%
\pgfsetdash{}{0pt}%
\pgfpathmoveto{\pgfqpoint{8.427845in}{1.852931in}}%
\pgfpathlineto{\pgfqpoint{8.427263in}{1.852959in}}%
\pgfusepath{stroke}%
\end{pgfscope}%
\begin{pgfscope}%
\pgfpathrectangle{\pgfqpoint{6.720588in}{1.750000in}}{\pgfqpoint{2.279412in}{2.004545in}}%
\pgfusepath{clip}%
\pgfsetbuttcap%
\pgfsetroundjoin%
\pgfsetlinewidth{0.308702pt}%
\definecolor{currentstroke}{rgb}{0.268510,0.009605,0.335427}%
\pgfsetstrokecolor{currentstroke}%
\pgfsetdash{}{0pt}%
\pgfpathmoveto{\pgfqpoint{8.427263in}{1.852959in}}%
\pgfpathlineto{\pgfqpoint{8.427216in}{1.852958in}}%
\pgfusepath{stroke}%
\end{pgfscope}%
\begin{pgfscope}%
\pgfpathrectangle{\pgfqpoint{6.720588in}{1.750000in}}{\pgfqpoint{2.279412in}{2.004545in}}%
\pgfusepath{clip}%
\pgfsetbuttcap%
\pgfsetroundjoin%
\pgfsetlinewidth{0.308688pt}%
\definecolor{currentstroke}{rgb}{0.268510,0.009605,0.335427}%
\pgfsetstrokecolor{currentstroke}%
\pgfsetdash{}{0pt}%
\pgfpathmoveto{\pgfqpoint{8.427216in}{1.852958in}}%
\pgfpathlineto{\pgfqpoint{8.427765in}{1.852929in}}%
\pgfusepath{stroke}%
\end{pgfscope}%
\begin{pgfscope}%
\pgfpathrectangle{\pgfqpoint{6.720588in}{1.750000in}}{\pgfqpoint{2.279412in}{2.004545in}}%
\pgfusepath{clip}%
\pgfsetbuttcap%
\pgfsetroundjoin%
\pgfsetlinewidth{0.308856pt}%
\definecolor{currentstroke}{rgb}{0.268510,0.009605,0.335427}%
\pgfsetstrokecolor{currentstroke}%
\pgfsetdash{}{0pt}%
\pgfpathmoveto{\pgfqpoint{8.427765in}{1.852929in}}%
\pgfpathlineto{\pgfqpoint{8.428259in}{1.852905in}}%
\pgfusepath{stroke}%
\end{pgfscope}%
\begin{pgfscope}%
\pgfpathrectangle{\pgfqpoint{6.720588in}{1.750000in}}{\pgfqpoint{2.279412in}{2.004545in}}%
\pgfusepath{clip}%
\pgfsetbuttcap%
\pgfsetroundjoin%
\pgfsetlinewidth{0.309006pt}%
\definecolor{currentstroke}{rgb}{0.268510,0.009605,0.335427}%
\pgfsetstrokecolor{currentstroke}%
\pgfsetdash{}{0pt}%
\pgfpathmoveto{\pgfqpoint{8.428259in}{1.852905in}}%
\pgfpathlineto{\pgfqpoint{8.428327in}{1.852905in}}%
\pgfusepath{stroke}%
\end{pgfscope}%
\begin{pgfscope}%
\pgfpathrectangle{\pgfqpoint{6.720588in}{1.750000in}}{\pgfqpoint{2.279412in}{2.004545in}}%
\pgfusepath{clip}%
\pgfsetbuttcap%
\pgfsetroundjoin%
\pgfsetlinewidth{0.309027pt}%
\definecolor{currentstroke}{rgb}{0.268510,0.009605,0.335427}%
\pgfsetstrokecolor{currentstroke}%
\pgfsetdash{}{0pt}%
\pgfpathmoveto{\pgfqpoint{8.428327in}{1.852905in}}%
\pgfpathlineto{\pgfqpoint{8.427852in}{1.852931in}}%
\pgfusepath{stroke}%
\end{pgfscope}%
\begin{pgfscope}%
\pgfpathrectangle{\pgfqpoint{6.720588in}{1.750000in}}{\pgfqpoint{2.279412in}{2.004545in}}%
\pgfusepath{clip}%
\pgfsetbuttcap%
\pgfsetroundjoin%
\pgfsetlinewidth{0.308883pt}%
\definecolor{currentstroke}{rgb}{0.268510,0.009605,0.335427}%
\pgfsetstrokecolor{currentstroke}%
\pgfsetdash{}{0pt}%
\pgfpathmoveto{\pgfqpoint{8.427852in}{1.852931in}}%
\pgfpathlineto{\pgfqpoint{8.427100in}{1.852967in}}%
\pgfusepath{stroke}%
\end{pgfscope}%
\begin{pgfscope}%
\pgfpathrectangle{\pgfqpoint{6.720588in}{1.750000in}}{\pgfqpoint{2.279412in}{2.004545in}}%
\pgfusepath{clip}%
\pgfsetbuttcap%
\pgfsetroundjoin%
\pgfsetlinewidth{0.308652pt}%
\definecolor{currentstroke}{rgb}{0.268510,0.009605,0.335427}%
\pgfsetstrokecolor{currentstroke}%
\pgfsetdash{}{0pt}%
\pgfpathmoveto{\pgfqpoint{8.427100in}{1.852967in}}%
\pgfpathlineto{\pgfqpoint{8.427100in}{1.852967in}}%
\pgfusepath{stroke}%
\end{pgfscope}%
\begin{pgfscope}%
\pgfpathrectangle{\pgfqpoint{6.720588in}{1.750000in}}{\pgfqpoint{2.279412in}{2.004545in}}%
\pgfusepath{clip}%
\pgfsetbuttcap%
\pgfsetroundjoin%
\pgfsetlinewidth{0.308652pt}%
\definecolor{currentstroke}{rgb}{0.268510,0.009605,0.335427}%
\pgfsetstrokecolor{currentstroke}%
\pgfsetdash{}{0pt}%
\pgfpathmoveto{\pgfqpoint{8.427100in}{1.852967in}}%
\pgfpathlineto{\pgfqpoint{8.427785in}{1.852930in}}%
\pgfusepath{stroke}%
\end{pgfscope}%
\begin{pgfscope}%
\pgfpathrectangle{\pgfqpoint{6.720588in}{1.750000in}}{\pgfqpoint{2.279412in}{2.004545in}}%
\pgfusepath{clip}%
\pgfsetbuttcap%
\pgfsetroundjoin%
\pgfsetlinewidth{0.308862pt}%
\definecolor{currentstroke}{rgb}{0.268510,0.009605,0.335427}%
\pgfsetstrokecolor{currentstroke}%
\pgfsetdash{}{0pt}%
\pgfpathmoveto{\pgfqpoint{8.427785in}{1.852930in}}%
\pgfpathlineto{\pgfqpoint{8.428342in}{1.852903in}}%
\pgfusepath{stroke}%
\end{pgfscope}%
\begin{pgfscope}%
\pgfpathrectangle{\pgfqpoint{6.720588in}{1.750000in}}{\pgfqpoint{2.279412in}{2.004545in}}%
\pgfusepath{clip}%
\pgfsetbuttcap%
\pgfsetroundjoin%
\pgfsetlinewidth{0.309031pt}%
\definecolor{currentstroke}{rgb}{0.268510,0.009605,0.335427}%
\pgfsetstrokecolor{currentstroke}%
\pgfsetdash{}{0pt}%
\pgfpathmoveto{\pgfqpoint{8.428342in}{1.852903in}}%
\pgfpathlineto{\pgfqpoint{8.428398in}{1.852903in}}%
\pgfusepath{stroke}%
\end{pgfscope}%
\begin{pgfscope}%
\pgfpathrectangle{\pgfqpoint{6.720588in}{1.750000in}}{\pgfqpoint{2.279412in}{2.004545in}}%
\pgfusepath{clip}%
\pgfsetbuttcap%
\pgfsetroundjoin%
\pgfsetlinewidth{0.309048pt}%
\definecolor{currentstroke}{rgb}{0.268510,0.009605,0.335427}%
\pgfsetstrokecolor{currentstroke}%
\pgfsetdash{}{0pt}%
\pgfpathmoveto{\pgfqpoint{8.428398in}{1.852903in}}%
\pgfpathlineto{\pgfqpoint{8.427822in}{1.852934in}}%
\pgfusepath{stroke}%
\end{pgfscope}%
\begin{pgfscope}%
\pgfpathrectangle{\pgfqpoint{6.720588in}{1.750000in}}{\pgfqpoint{2.279412in}{2.004545in}}%
\pgfusepath{clip}%
\pgfsetbuttcap%
\pgfsetroundjoin%
\pgfsetlinewidth{0.308873pt}%
\definecolor{currentstroke}{rgb}{0.268510,0.009605,0.335427}%
\pgfsetstrokecolor{currentstroke}%
\pgfsetdash{}{0pt}%
\pgfpathmoveto{\pgfqpoint{8.427822in}{1.852934in}}%
\pgfpathlineto{\pgfqpoint{8.427822in}{1.852934in}}%
\pgfusepath{stroke}%
\end{pgfscope}%
\begin{pgfscope}%
\pgfpathrectangle{\pgfqpoint{6.720588in}{1.750000in}}{\pgfqpoint{2.279412in}{2.004545in}}%
\pgfusepath{clip}%
\pgfsetbuttcap%
\pgfsetroundjoin%
\pgfsetlinewidth{0.308873pt}%
\definecolor{currentstroke}{rgb}{0.268510,0.009605,0.335427}%
\pgfsetstrokecolor{currentstroke}%
\pgfsetdash{}{0pt}%
\pgfpathmoveto{\pgfqpoint{8.427822in}{1.852934in}}%
\pgfpathlineto{\pgfqpoint{8.427709in}{1.852939in}}%
\pgfusepath{stroke}%
\end{pgfscope}%
\begin{pgfscope}%
\pgfpathrectangle{\pgfqpoint{6.720588in}{1.750000in}}{\pgfqpoint{2.279412in}{2.004545in}}%
\pgfusepath{clip}%
\pgfsetbuttcap%
\pgfsetroundjoin%
\pgfsetlinewidth{0.308839pt}%
\definecolor{currentstroke}{rgb}{0.268510,0.009605,0.335427}%
\pgfsetstrokecolor{currentstroke}%
\pgfsetdash{}{0pt}%
\pgfpathmoveto{\pgfqpoint{8.427709in}{1.852939in}}%
\pgfpathlineto{\pgfqpoint{8.427694in}{1.852938in}}%
\pgfusepath{stroke}%
\end{pgfscope}%
\begin{pgfscope}%
\pgfpathrectangle{\pgfqpoint{6.720588in}{1.750000in}}{\pgfqpoint{2.279412in}{2.004545in}}%
\pgfusepath{clip}%
\pgfsetbuttcap%
\pgfsetroundjoin%
\pgfsetlinewidth{0.308834pt}%
\definecolor{currentstroke}{rgb}{0.268510,0.009605,0.335427}%
\pgfsetstrokecolor{currentstroke}%
\pgfsetdash{}{0pt}%
\pgfpathmoveto{\pgfqpoint{8.427694in}{1.852938in}}%
\pgfpathlineto{\pgfqpoint{8.427801in}{1.852932in}}%
\pgfusepath{stroke}%
\end{pgfscope}%
\begin{pgfscope}%
\pgfpathrectangle{\pgfqpoint{6.720588in}{1.750000in}}{\pgfqpoint{2.279412in}{2.004545in}}%
\pgfusepath{clip}%
\pgfsetbuttcap%
\pgfsetroundjoin%
\pgfsetlinewidth{0.308867pt}%
\definecolor{currentstroke}{rgb}{0.268510,0.009605,0.335427}%
\pgfsetstrokecolor{currentstroke}%
\pgfsetdash{}{0pt}%
\pgfpathmoveto{\pgfqpoint{8.427801in}{1.852932in}}%
\pgfpathlineto{\pgfqpoint{8.427922in}{1.852926in}}%
\pgfusepath{stroke}%
\end{pgfscope}%
\begin{pgfscope}%
\pgfpathrectangle{\pgfqpoint{6.720588in}{1.750000in}}{\pgfqpoint{2.279412in}{2.004545in}}%
\pgfusepath{clip}%
\pgfsetbuttcap%
\pgfsetroundjoin%
\pgfsetlinewidth{0.308904pt}%
\definecolor{currentstroke}{rgb}{0.268510,0.009605,0.335427}%
\pgfsetstrokecolor{currentstroke}%
\pgfsetdash{}{0pt}%
\pgfpathmoveto{\pgfqpoint{8.427922in}{1.852926in}}%
\pgfpathlineto{\pgfqpoint{8.427939in}{1.852926in}}%
\pgfusepath{stroke}%
\end{pgfscope}%
\begin{pgfscope}%
\pgfpathrectangle{\pgfqpoint{6.720588in}{1.750000in}}{\pgfqpoint{2.279412in}{2.004545in}}%
\pgfusepath{clip}%
\pgfsetbuttcap%
\pgfsetroundjoin%
\pgfsetlinewidth{0.308909pt}%
\definecolor{currentstroke}{rgb}{0.268510,0.009605,0.335427}%
\pgfsetstrokecolor{currentstroke}%
\pgfsetdash{}{0pt}%
\pgfpathmoveto{\pgfqpoint{8.427939in}{1.852926in}}%
\pgfpathlineto{\pgfqpoint{8.427824in}{1.852932in}}%
\pgfusepath{stroke}%
\end{pgfscope}%
\begin{pgfscope}%
\pgfpathrectangle{\pgfqpoint{6.720588in}{1.750000in}}{\pgfqpoint{2.279412in}{2.004545in}}%
\pgfusepath{clip}%
\pgfsetbuttcap%
\pgfsetroundjoin%
\pgfsetlinewidth{0.308874pt}%
\definecolor{currentstroke}{rgb}{0.268510,0.009605,0.335427}%
\pgfsetstrokecolor{currentstroke}%
\pgfsetdash{}{0pt}%
\pgfpathmoveto{\pgfqpoint{8.427824in}{1.852932in}}%
\pgfpathlineto{\pgfqpoint{8.427678in}{1.852939in}}%
\pgfusepath{stroke}%
\end{pgfscope}%
\begin{pgfscope}%
\pgfpathrectangle{\pgfqpoint{6.720588in}{1.750000in}}{\pgfqpoint{2.279412in}{2.004545in}}%
\pgfusepath{clip}%
\pgfsetbuttcap%
\pgfsetroundjoin%
\pgfsetlinewidth{0.308829pt}%
\definecolor{currentstroke}{rgb}{0.268510,0.009605,0.335427}%
\pgfsetstrokecolor{currentstroke}%
\pgfsetdash{}{0pt}%
\pgfpathmoveto{\pgfqpoint{8.427678in}{1.852939in}}%
\pgfpathlineto{\pgfqpoint{8.427660in}{1.852939in}}%
\pgfusepath{stroke}%
\end{pgfscope}%
\begin{pgfscope}%
\pgfpathrectangle{\pgfqpoint{6.720588in}{1.750000in}}{\pgfqpoint{2.279412in}{2.004545in}}%
\pgfusepath{clip}%
\pgfsetbuttcap%
\pgfsetroundjoin%
\pgfsetlinewidth{0.308824pt}%
\definecolor{currentstroke}{rgb}{0.268510,0.009605,0.335427}%
\pgfsetstrokecolor{currentstroke}%
\pgfsetdash{}{0pt}%
\pgfpathmoveto{\pgfqpoint{8.427660in}{1.852939in}}%
\pgfpathlineto{\pgfqpoint{8.427799in}{1.852931in}}%
\pgfusepath{stroke}%
\end{pgfscope}%
\begin{pgfscope}%
\pgfpathrectangle{\pgfqpoint{6.720588in}{1.750000in}}{\pgfqpoint{2.279412in}{2.004545in}}%
\pgfusepath{clip}%
\pgfsetbuttcap%
\pgfsetroundjoin%
\pgfsetlinewidth{0.308866pt}%
\definecolor{currentstroke}{rgb}{0.268510,0.009605,0.335427}%
\pgfsetstrokecolor{currentstroke}%
\pgfsetdash{}{0pt}%
\pgfpathmoveto{\pgfqpoint{8.427799in}{1.852931in}}%
\pgfpathlineto{\pgfqpoint{8.427953in}{1.852924in}}%
\pgfusepath{stroke}%
\end{pgfscope}%
\begin{pgfscope}%
\pgfpathrectangle{\pgfqpoint{6.720588in}{1.750000in}}{\pgfqpoint{2.279412in}{2.004545in}}%
\pgfusepath{clip}%
\pgfsetbuttcap%
\pgfsetroundjoin%
\pgfsetlinewidth{0.308913pt}%
\definecolor{currentstroke}{rgb}{0.268510,0.009605,0.335427}%
\pgfsetstrokecolor{currentstroke}%
\pgfsetdash{}{0pt}%
\pgfpathmoveto{\pgfqpoint{8.427953in}{1.852924in}}%
\pgfpathlineto{\pgfqpoint{8.427974in}{1.852924in}}%
\pgfusepath{stroke}%
\end{pgfscope}%
\begin{pgfscope}%
\pgfpathrectangle{\pgfqpoint{6.720588in}{1.750000in}}{\pgfqpoint{2.279412in}{2.004545in}}%
\pgfusepath{clip}%
\pgfsetbuttcap%
\pgfsetroundjoin%
\pgfsetlinewidth{0.308920pt}%
\definecolor{currentstroke}{rgb}{0.268510,0.009605,0.335427}%
\pgfsetstrokecolor{currentstroke}%
\pgfsetdash{}{0pt}%
\pgfpathmoveto{\pgfqpoint{8.427974in}{1.852924in}}%
\pgfpathlineto{\pgfqpoint{8.427827in}{1.852932in}}%
\pgfusepath{stroke}%
\end{pgfscope}%
\begin{pgfscope}%
\pgfpathrectangle{\pgfqpoint{6.720588in}{1.750000in}}{\pgfqpoint{2.279412in}{2.004545in}}%
\pgfusepath{clip}%
\pgfsetbuttcap%
\pgfsetroundjoin%
\pgfsetlinewidth{0.308875pt}%
\definecolor{currentstroke}{rgb}{0.268510,0.009605,0.335427}%
\pgfsetstrokecolor{currentstroke}%
\pgfsetdash{}{0pt}%
\pgfpathmoveto{\pgfqpoint{8.427827in}{1.852932in}}%
\pgfpathlineto{\pgfqpoint{8.427636in}{1.852941in}}%
\pgfusepath{stroke}%
\end{pgfscope}%
\begin{pgfscope}%
\pgfpathrectangle{\pgfqpoint{6.720588in}{1.750000in}}{\pgfqpoint{2.279412in}{2.004545in}}%
\pgfusepath{clip}%
\pgfsetbuttcap%
\pgfsetroundjoin%
\pgfsetlinewidth{0.308816pt}%
\definecolor{currentstroke}{rgb}{0.268510,0.009605,0.335427}%
\pgfsetstrokecolor{currentstroke}%
\pgfsetdash{}{0pt}%
\pgfpathmoveto{\pgfqpoint{8.427636in}{1.852941in}}%
\pgfpathlineto{\pgfqpoint{8.427614in}{1.852941in}}%
\pgfusepath{stroke}%
\end{pgfscope}%
\begin{pgfscope}%
\pgfpathrectangle{\pgfqpoint{6.720588in}{1.750000in}}{\pgfqpoint{2.279412in}{2.004545in}}%
\pgfusepath{clip}%
\pgfsetbuttcap%
\pgfsetroundjoin%
\pgfsetlinewidth{0.308810pt}%
\definecolor{currentstroke}{rgb}{0.268510,0.009605,0.335427}%
\pgfsetstrokecolor{currentstroke}%
\pgfsetdash{}{0pt}%
\pgfpathmoveto{\pgfqpoint{8.427614in}{1.852941in}}%
\pgfpathlineto{\pgfqpoint{8.427796in}{1.852931in}}%
\pgfusepath{stroke}%
\end{pgfscope}%
\begin{pgfscope}%
\pgfpathrectangle{\pgfqpoint{6.720588in}{1.750000in}}{\pgfqpoint{2.279412in}{2.004545in}}%
\pgfusepath{clip}%
\pgfsetbuttcap%
\pgfsetroundjoin%
\pgfsetlinewidth{0.308865pt}%
\definecolor{currentstroke}{rgb}{0.268510,0.009605,0.335427}%
\pgfsetstrokecolor{currentstroke}%
\pgfsetdash{}{0pt}%
\pgfpathmoveto{\pgfqpoint{8.427796in}{1.852931in}}%
\pgfpathlineto{\pgfqpoint{8.427992in}{1.852922in}}%
\pgfusepath{stroke}%
\end{pgfscope}%
\begin{pgfscope}%
\pgfpathrectangle{\pgfqpoint{6.720588in}{1.750000in}}{\pgfqpoint{2.279412in}{2.004545in}}%
\pgfusepath{clip}%
\pgfsetbuttcap%
\pgfsetroundjoin%
\pgfsetlinewidth{0.308925pt}%
\definecolor{currentstroke}{rgb}{0.268510,0.009605,0.335427}%
\pgfsetstrokecolor{currentstroke}%
\pgfsetdash{}{0pt}%
\pgfpathmoveto{\pgfqpoint{8.427992in}{1.852922in}}%
\pgfpathlineto{\pgfqpoint{8.428018in}{1.852921in}}%
\pgfusepath{stroke}%
\end{pgfscope}%
\begin{pgfscope}%
\pgfpathrectangle{\pgfqpoint{6.720588in}{1.750000in}}{\pgfqpoint{2.279412in}{2.004545in}}%
\pgfusepath{clip}%
\pgfsetbuttcap%
\pgfsetroundjoin%
\pgfsetlinewidth{0.308933pt}%
\definecolor{currentstroke}{rgb}{0.268510,0.009605,0.335427}%
\pgfsetstrokecolor{currentstroke}%
\pgfsetdash{}{0pt}%
\pgfpathmoveto{\pgfqpoint{8.428018in}{1.852921in}}%
\pgfpathlineto{\pgfqpoint{8.427829in}{1.852932in}}%
\pgfusepath{stroke}%
\end{pgfscope}%
\begin{pgfscope}%
\pgfpathrectangle{\pgfqpoint{6.720588in}{1.750000in}}{\pgfqpoint{2.279412in}{2.004545in}}%
\pgfusepath{clip}%
\pgfsetbuttcap%
\pgfsetroundjoin%
\pgfsetlinewidth{0.308875pt}%
\definecolor{currentstroke}{rgb}{0.268510,0.009605,0.335427}%
\pgfsetstrokecolor{currentstroke}%
\pgfsetdash{}{0pt}%
\pgfpathmoveto{\pgfqpoint{8.427829in}{1.852932in}}%
\pgfpathlineto{\pgfqpoint{8.427581in}{1.852943in}}%
\pgfusepath{stroke}%
\end{pgfscope}%
\begin{pgfscope}%
\pgfpathrectangle{\pgfqpoint{6.720588in}{1.750000in}}{\pgfqpoint{2.279412in}{2.004545in}}%
\pgfusepath{clip}%
\pgfsetbuttcap%
\pgfsetroundjoin%
\pgfsetlinewidth{0.308800pt}%
\definecolor{currentstroke}{rgb}{0.268510,0.009605,0.335427}%
\pgfsetstrokecolor{currentstroke}%
\pgfsetdash{}{0pt}%
\pgfpathmoveto{\pgfqpoint{8.427581in}{1.852943in}}%
\pgfpathlineto{\pgfqpoint{8.427554in}{1.852943in}}%
\pgfusepath{stroke}%
\end{pgfscope}%
\begin{pgfscope}%
\pgfpathrectangle{\pgfqpoint{6.720588in}{1.750000in}}{\pgfqpoint{2.279412in}{2.004545in}}%
\pgfusepath{clip}%
\pgfsetbuttcap%
\pgfsetroundjoin%
\pgfsetlinewidth{0.308791pt}%
\definecolor{currentstroke}{rgb}{0.268510,0.009605,0.335427}%
\pgfsetstrokecolor{currentstroke}%
\pgfsetdash{}{0pt}%
\pgfpathmoveto{\pgfqpoint{8.427554in}{1.852943in}}%
\pgfpathlineto{\pgfqpoint{8.427793in}{1.852931in}}%
\pgfusepath{stroke}%
\end{pgfscope}%
\begin{pgfscope}%
\pgfpathrectangle{\pgfqpoint{6.720588in}{1.750000in}}{\pgfqpoint{2.279412in}{2.004545in}}%
\pgfusepath{clip}%
\pgfsetbuttcap%
\pgfsetroundjoin%
\pgfsetlinewidth{0.308864pt}%
\definecolor{currentstroke}{rgb}{0.268510,0.009605,0.335427}%
\pgfsetstrokecolor{currentstroke}%
\pgfsetdash{}{0pt}%
\pgfpathmoveto{\pgfqpoint{8.427793in}{1.852931in}}%
\pgfpathlineto{\pgfqpoint{8.428040in}{1.852919in}}%
\pgfusepath{stroke}%
\end{pgfscope}%
\begin{pgfscope}%
\pgfpathrectangle{\pgfqpoint{6.720588in}{1.750000in}}{\pgfqpoint{2.279412in}{2.004545in}}%
\pgfusepath{clip}%
\pgfsetbuttcap%
\pgfsetroundjoin%
\pgfsetlinewidth{0.308940pt}%
\definecolor{currentstroke}{rgb}{0.268510,0.009605,0.335427}%
\pgfsetstrokecolor{currentstroke}%
\pgfsetdash{}{0pt}%
\pgfpathmoveto{\pgfqpoint{8.428040in}{1.852919in}}%
\pgfpathlineto{\pgfqpoint{8.428072in}{1.852919in}}%
\pgfusepath{stroke}%
\end{pgfscope}%
\begin{pgfscope}%
\pgfpathrectangle{\pgfqpoint{6.720588in}{1.750000in}}{\pgfqpoint{2.279412in}{2.004545in}}%
\pgfusepath{clip}%
\pgfsetbuttcap%
\pgfsetroundjoin%
\pgfsetlinewidth{0.308949pt}%
\definecolor{currentstroke}{rgb}{0.268510,0.009605,0.335427}%
\pgfsetstrokecolor{currentstroke}%
\pgfsetdash{}{0pt}%
\pgfpathmoveto{\pgfqpoint{8.428072in}{1.852919in}}%
\pgfpathlineto{\pgfqpoint{8.427831in}{1.852932in}}%
\pgfusepath{stroke}%
\end{pgfscope}%
\begin{pgfscope}%
\pgfpathrectangle{\pgfqpoint{6.720588in}{1.750000in}}{\pgfqpoint{2.279412in}{2.004545in}}%
\pgfusepath{clip}%
\pgfsetbuttcap%
\pgfsetroundjoin%
\pgfsetlinewidth{0.308876pt}%
\definecolor{currentstroke}{rgb}{0.268510,0.009605,0.335427}%
\pgfsetstrokecolor{currentstroke}%
\pgfsetdash{}{0pt}%
\pgfpathmoveto{\pgfqpoint{8.427831in}{1.852932in}}%
\pgfpathlineto{\pgfqpoint{8.427507in}{1.852947in}}%
\pgfusepath{stroke}%
\end{pgfscope}%
\begin{pgfscope}%
\pgfpathrectangle{\pgfqpoint{6.720588in}{1.750000in}}{\pgfqpoint{2.279412in}{2.004545in}}%
\pgfusepath{clip}%
\pgfsetbuttcap%
\pgfsetroundjoin%
\pgfsetlinewidth{0.308777pt}%
\definecolor{currentstroke}{rgb}{0.268510,0.009605,0.335427}%
\pgfsetstrokecolor{currentstroke}%
\pgfsetdash{}{0pt}%
\pgfpathmoveto{\pgfqpoint{8.427507in}{1.852947in}}%
\pgfpathlineto{\pgfqpoint{8.427477in}{1.852947in}}%
\pgfusepath{stroke}%
\end{pgfscope}%
\begin{pgfscope}%
\pgfpathrectangle{\pgfqpoint{6.720588in}{1.750000in}}{\pgfqpoint{2.279412in}{2.004545in}}%
\pgfusepath{clip}%
\pgfsetbuttcap%
\pgfsetroundjoin%
\pgfsetlinewidth{0.308768pt}%
\definecolor{currentstroke}{rgb}{0.268510,0.009605,0.335427}%
\pgfsetstrokecolor{currentstroke}%
\pgfsetdash{}{0pt}%
\pgfpathmoveto{\pgfqpoint{8.427477in}{1.852947in}}%
\pgfpathlineto{\pgfqpoint{8.427788in}{1.852930in}}%
\pgfusepath{stroke}%
\end{pgfscope}%
\begin{pgfscope}%
\pgfpathrectangle{\pgfqpoint{6.720588in}{1.750000in}}{\pgfqpoint{2.279412in}{2.004545in}}%
\pgfusepath{clip}%
\pgfsetbuttcap%
\pgfsetroundjoin%
\pgfsetlinewidth{0.308863pt}%
\definecolor{currentstroke}{rgb}{0.268510,0.009605,0.335427}%
\pgfsetstrokecolor{currentstroke}%
\pgfsetdash{}{0pt}%
\pgfpathmoveto{\pgfqpoint{8.427788in}{1.852930in}}%
\pgfpathlineto{\pgfqpoint{8.428099in}{1.852915in}}%
\pgfusepath{stroke}%
\end{pgfscope}%
\begin{pgfscope}%
\pgfpathrectangle{\pgfqpoint{6.720588in}{1.750000in}}{\pgfqpoint{2.279412in}{2.004545in}}%
\pgfusepath{clip}%
\pgfsetbuttcap%
\pgfsetroundjoin%
\pgfsetlinewidth{0.308957pt}%
\definecolor{currentstroke}{rgb}{0.268510,0.009605,0.335427}%
\pgfsetstrokecolor{currentstroke}%
\pgfsetdash{}{0pt}%
\pgfpathmoveto{\pgfqpoint{8.428099in}{1.852915in}}%
\pgfpathlineto{\pgfqpoint{8.428138in}{1.852915in}}%
\pgfusepath{stroke}%
\end{pgfscope}%
\begin{pgfscope}%
\pgfpathrectangle{\pgfqpoint{6.720588in}{1.750000in}}{\pgfqpoint{2.279412in}{2.004545in}}%
\pgfusepath{clip}%
\pgfsetbuttcap%
\pgfsetroundjoin%
\pgfsetlinewidth{0.308969pt}%
\definecolor{currentstroke}{rgb}{0.268510,0.009605,0.335427}%
\pgfsetstrokecolor{currentstroke}%
\pgfsetdash{}{0pt}%
\pgfpathmoveto{\pgfqpoint{8.428138in}{1.852915in}}%
\pgfpathlineto{\pgfqpoint{8.427834in}{1.852932in}}%
\pgfusepath{stroke}%
\end{pgfscope}%
\begin{pgfscope}%
\pgfpathrectangle{\pgfqpoint{6.720588in}{1.750000in}}{\pgfqpoint{2.279412in}{2.004545in}}%
\pgfusepath{clip}%
\pgfsetbuttcap%
\pgfsetroundjoin%
\pgfsetlinewidth{0.308877pt}%
\definecolor{currentstroke}{rgb}{0.268510,0.009605,0.335427}%
\pgfsetstrokecolor{currentstroke}%
\pgfsetdash{}{0pt}%
\pgfpathmoveto{\pgfqpoint{8.427834in}{1.852932in}}%
\pgfpathlineto{\pgfqpoint{8.427411in}{1.852952in}}%
\pgfusepath{stroke}%
\end{pgfscope}%
\begin{pgfscope}%
\pgfpathrectangle{\pgfqpoint{6.720588in}{1.750000in}}{\pgfqpoint{2.279412in}{2.004545in}}%
\pgfusepath{clip}%
\pgfsetbuttcap%
\pgfsetroundjoin%
\pgfsetlinewidth{0.308748pt}%
\definecolor{currentstroke}{rgb}{0.268510,0.009605,0.335427}%
\pgfsetstrokecolor{currentstroke}%
\pgfsetdash{}{0pt}%
\pgfpathmoveto{\pgfqpoint{8.427411in}{1.852952in}}%
\pgfpathlineto{\pgfqpoint{8.427376in}{1.852951in}}%
\pgfusepath{stroke}%
\end{pgfscope}%
\begin{pgfscope}%
\pgfpathrectangle{\pgfqpoint{6.720588in}{1.750000in}}{\pgfqpoint{2.279412in}{2.004545in}}%
\pgfusepath{clip}%
\pgfsetbuttcap%
\pgfsetroundjoin%
\pgfsetlinewidth{0.308737pt}%
\definecolor{currentstroke}{rgb}{0.268510,0.009605,0.335427}%
\pgfsetstrokecolor{currentstroke}%
\pgfsetdash{}{0pt}%
\pgfpathmoveto{\pgfqpoint{8.427376in}{1.852951in}}%
\pgfpathlineto{\pgfqpoint{8.427782in}{1.852930in}}%
\pgfusepath{stroke}%
\end{pgfscope}%
\begin{pgfscope}%
\pgfpathrectangle{\pgfqpoint{6.720588in}{1.750000in}}{\pgfqpoint{2.279412in}{2.004545in}}%
\pgfusepath{clip}%
\pgfsetbuttcap%
\pgfsetroundjoin%
\pgfsetlinewidth{0.308861pt}%
\definecolor{currentstroke}{rgb}{0.268510,0.009605,0.335427}%
\pgfsetstrokecolor{currentstroke}%
\pgfsetdash{}{0pt}%
\pgfpathmoveto{\pgfqpoint{8.427782in}{1.852930in}}%
\pgfpathlineto{\pgfqpoint{8.428168in}{1.852911in}}%
\pgfusepath{stroke}%
\end{pgfscope}%
\begin{pgfscope}%
\pgfpathrectangle{\pgfqpoint{6.720588in}{1.750000in}}{\pgfqpoint{2.279412in}{2.004545in}}%
\pgfusepath{clip}%
\pgfsetbuttcap%
\pgfsetroundjoin%
\pgfsetlinewidth{0.308978pt}%
\definecolor{currentstroke}{rgb}{0.268510,0.009605,0.335427}%
\pgfsetstrokecolor{currentstroke}%
\pgfsetdash{}{0pt}%
\pgfpathmoveto{\pgfqpoint{8.428168in}{1.852911in}}%
\pgfpathlineto{\pgfqpoint{8.428216in}{1.852911in}}%
\pgfusepath{stroke}%
\end{pgfscope}%
\begin{pgfscope}%
\pgfpathrectangle{\pgfqpoint{6.720588in}{1.750000in}}{\pgfqpoint{2.279412in}{2.004545in}}%
\pgfusepath{clip}%
\pgfsetbuttcap%
\pgfsetroundjoin%
\pgfsetlinewidth{0.308993pt}%
\definecolor{currentstroke}{rgb}{0.268510,0.009605,0.335427}%
\pgfsetstrokecolor{currentstroke}%
\pgfsetdash{}{0pt}%
\pgfpathmoveto{\pgfqpoint{8.428216in}{1.852911in}}%
\pgfpathlineto{\pgfqpoint{8.427838in}{1.852931in}}%
\pgfusepath{stroke}%
\end{pgfscope}%
\begin{pgfscope}%
\pgfpathrectangle{\pgfqpoint{6.720588in}{1.750000in}}{\pgfqpoint{2.279412in}{2.004545in}}%
\pgfusepath{clip}%
\pgfsetbuttcap%
\pgfsetroundjoin%
\pgfsetlinewidth{0.308878pt}%
\definecolor{currentstroke}{rgb}{0.268510,0.009605,0.335427}%
\pgfsetstrokecolor{currentstroke}%
\pgfsetdash{}{0pt}%
\pgfpathmoveto{\pgfqpoint{8.427838in}{1.852931in}}%
\pgfpathlineto{\pgfqpoint{8.427287in}{1.852958in}}%
\pgfusepath{stroke}%
\end{pgfscope}%
\begin{pgfscope}%
\pgfpathrectangle{\pgfqpoint{6.720588in}{1.750000in}}{\pgfqpoint{2.279412in}{2.004545in}}%
\pgfusepath{clip}%
\pgfsetbuttcap%
\pgfsetroundjoin%
\pgfsetlinewidth{0.308710pt}%
\definecolor{currentstroke}{rgb}{0.268510,0.009605,0.335427}%
\pgfsetstrokecolor{currentstroke}%
\pgfsetdash{}{0pt}%
\pgfpathmoveto{\pgfqpoint{8.427287in}{1.852958in}}%
\pgfpathlineto{\pgfqpoint{8.427248in}{1.852956in}}%
\pgfusepath{stroke}%
\end{pgfscope}%
\begin{pgfscope}%
\pgfpathrectangle{\pgfqpoint{6.720588in}{1.750000in}}{\pgfqpoint{2.279412in}{2.004545in}}%
\pgfusepath{clip}%
\pgfsetbuttcap%
\pgfsetroundjoin%
\pgfsetlinewidth{0.308697pt}%
\definecolor{currentstroke}{rgb}{0.268510,0.009605,0.335427}%
\pgfsetstrokecolor{currentstroke}%
\pgfsetdash{}{0pt}%
\pgfpathmoveto{\pgfqpoint{8.427248in}{1.852956in}}%
\pgfpathlineto{\pgfqpoint{8.427774in}{1.852929in}}%
\pgfusepath{stroke}%
\end{pgfscope}%
\begin{pgfscope}%
\pgfpathrectangle{\pgfqpoint{6.720588in}{1.750000in}}{\pgfqpoint{2.279412in}{2.004545in}}%
\pgfusepath{clip}%
\pgfsetbuttcap%
\pgfsetroundjoin%
\pgfsetlinewidth{0.308858pt}%
\definecolor{currentstroke}{rgb}{0.268510,0.009605,0.335427}%
\pgfsetstrokecolor{currentstroke}%
\pgfsetdash{}{0pt}%
\pgfpathmoveto{\pgfqpoint{8.427774in}{1.852929in}}%
\pgfpathlineto{\pgfqpoint{8.428247in}{1.852906in}}%
\pgfusepath{stroke}%
\end{pgfscope}%
\begin{pgfscope}%
\pgfpathrectangle{\pgfqpoint{6.720588in}{1.750000in}}{\pgfqpoint{2.279412in}{2.004545in}}%
\pgfusepath{clip}%
\pgfsetbuttcap%
\pgfsetroundjoin%
\pgfsetlinewidth{0.309002pt}%
\definecolor{currentstroke}{rgb}{0.268510,0.009605,0.335427}%
\pgfsetstrokecolor{currentstroke}%
\pgfsetdash{}{0pt}%
\pgfpathmoveto{\pgfqpoint{8.428247in}{1.852906in}}%
\pgfpathlineto{\pgfqpoint{8.428307in}{1.852906in}}%
\pgfusepath{stroke}%
\end{pgfscope}%
\begin{pgfscope}%
\pgfpathrectangle{\pgfqpoint{6.720588in}{1.750000in}}{\pgfqpoint{2.279412in}{2.004545in}}%
\pgfusepath{clip}%
\pgfsetbuttcap%
\pgfsetroundjoin%
\pgfsetlinewidth{0.309021pt}%
\definecolor{currentstroke}{rgb}{0.268510,0.009605,0.335427}%
\pgfsetstrokecolor{currentstroke}%
\pgfsetdash{}{0pt}%
\pgfpathmoveto{\pgfqpoint{8.428307in}{1.852906in}}%
\pgfpathlineto{\pgfqpoint{8.427844in}{1.852931in}}%
\pgfusepath{stroke}%
\end{pgfscope}%
\begin{pgfscope}%
\pgfpathrectangle{\pgfqpoint{6.720588in}{1.750000in}}{\pgfqpoint{2.279412in}{2.004545in}}%
\pgfusepath{clip}%
\pgfsetbuttcap%
\pgfsetroundjoin%
\pgfsetlinewidth{0.308880pt}%
\definecolor{currentstroke}{rgb}{0.268510,0.009605,0.335427}%
\pgfsetstrokecolor{currentstroke}%
\pgfsetdash{}{0pt}%
\pgfpathmoveto{\pgfqpoint{8.427844in}{1.852931in}}%
\pgfpathlineto{\pgfqpoint{8.427129in}{1.852965in}}%
\pgfusepath{stroke}%
\end{pgfscope}%
\begin{pgfscope}%
\pgfpathrectangle{\pgfqpoint{6.720588in}{1.750000in}}{\pgfqpoint{2.279412in}{2.004545in}}%
\pgfusepath{clip}%
\pgfsetbuttcap%
\pgfsetroundjoin%
\pgfsetlinewidth{0.308661pt}%
\definecolor{currentstroke}{rgb}{0.268510,0.009605,0.335427}%
\pgfsetstrokecolor{currentstroke}%
\pgfsetdash{}{0pt}%
\pgfpathmoveto{\pgfqpoint{8.427129in}{1.852965in}}%
\pgfpathlineto{\pgfqpoint{8.427129in}{1.852965in}}%
\pgfusepath{stroke}%
\end{pgfscope}%
\begin{pgfscope}%
\pgfpathrectangle{\pgfqpoint{6.720588in}{1.750000in}}{\pgfqpoint{2.279412in}{2.004545in}}%
\pgfusepath{clip}%
\pgfsetbuttcap%
\pgfsetroundjoin%
\pgfsetlinewidth{0.308661pt}%
\definecolor{currentstroke}{rgb}{0.268510,0.009605,0.335427}%
\pgfsetstrokecolor{currentstroke}%
\pgfsetdash{}{0pt}%
\pgfpathmoveto{\pgfqpoint{8.427129in}{1.852965in}}%
\pgfpathlineto{\pgfqpoint{8.427789in}{1.852930in}}%
\pgfusepath{stroke}%
\end{pgfscope}%
\begin{pgfscope}%
\pgfpathrectangle{\pgfqpoint{6.720588in}{1.750000in}}{\pgfqpoint{2.279412in}{2.004545in}}%
\pgfusepath{clip}%
\pgfsetbuttcap%
\pgfsetroundjoin%
\pgfsetlinewidth{0.308863pt}%
\definecolor{currentstroke}{rgb}{0.268510,0.009605,0.335427}%
\pgfsetstrokecolor{currentstroke}%
\pgfsetdash{}{0pt}%
\pgfpathmoveto{\pgfqpoint{8.427789in}{1.852930in}}%
\pgfpathlineto{\pgfqpoint{8.428329in}{1.852904in}}%
\pgfusepath{stroke}%
\end{pgfscope}%
\begin{pgfscope}%
\pgfpathrectangle{\pgfqpoint{6.720588in}{1.750000in}}{\pgfqpoint{2.279412in}{2.004545in}}%
\pgfusepath{clip}%
\pgfsetbuttcap%
\pgfsetroundjoin%
\pgfsetlinewidth{0.309027pt}%
\definecolor{currentstroke}{rgb}{0.268510,0.009605,0.335427}%
\pgfsetstrokecolor{currentstroke}%
\pgfsetdash{}{0pt}%
\pgfpathmoveto{\pgfqpoint{8.428329in}{1.852904in}}%
\pgfpathlineto{\pgfqpoint{8.428381in}{1.852904in}}%
\pgfusepath{stroke}%
\end{pgfscope}%
\begin{pgfscope}%
\pgfpathrectangle{\pgfqpoint{6.720588in}{1.750000in}}{\pgfqpoint{2.279412in}{2.004545in}}%
\pgfusepath{clip}%
\pgfsetbuttcap%
\pgfsetroundjoin%
\pgfsetlinewidth{0.309043pt}%
\definecolor{currentstroke}{rgb}{0.268510,0.009605,0.335427}%
\pgfsetstrokecolor{currentstroke}%
\pgfsetdash{}{0pt}%
\pgfpathmoveto{\pgfqpoint{8.428381in}{1.852904in}}%
\pgfpathlineto{\pgfqpoint{8.427819in}{1.852934in}}%
\pgfusepath{stroke}%
\end{pgfscope}%
\begin{pgfscope}%
\pgfpathrectangle{\pgfqpoint{6.720588in}{1.750000in}}{\pgfqpoint{2.279412in}{2.004545in}}%
\pgfusepath{clip}%
\pgfsetbuttcap%
\pgfsetroundjoin%
\pgfsetlinewidth{0.308872pt}%
\definecolor{currentstroke}{rgb}{0.268510,0.009605,0.335427}%
\pgfsetstrokecolor{currentstroke}%
\pgfsetdash{}{0pt}%
\pgfpathmoveto{\pgfqpoint{8.427819in}{1.852934in}}%
\pgfpathlineto{\pgfqpoint{8.427819in}{1.852934in}}%
\pgfusepath{stroke}%
\end{pgfscope}%
\begin{pgfscope}%
\pgfpathrectangle{\pgfqpoint{6.720588in}{1.750000in}}{\pgfqpoint{2.279412in}{2.004545in}}%
\pgfusepath{clip}%
\pgfsetbuttcap%
\pgfsetroundjoin%
\pgfsetlinewidth{0.308872pt}%
\definecolor{currentstroke}{rgb}{0.268510,0.009605,0.335427}%
\pgfsetstrokecolor{currentstroke}%
\pgfsetdash{}{0pt}%
\pgfpathmoveto{\pgfqpoint{8.427819in}{1.852934in}}%
\pgfpathlineto{\pgfqpoint{8.427713in}{1.852938in}}%
\pgfusepath{stroke}%
\end{pgfscope}%
\begin{pgfscope}%
\pgfpathrectangle{\pgfqpoint{6.720588in}{1.750000in}}{\pgfqpoint{2.279412in}{2.004545in}}%
\pgfusepath{clip}%
\pgfsetbuttcap%
\pgfsetroundjoin%
\pgfsetlinewidth{0.308840pt}%
\definecolor{currentstroke}{rgb}{0.268510,0.009605,0.335427}%
\pgfsetstrokecolor{currentstroke}%
\pgfsetdash{}{0pt}%
\pgfpathmoveto{\pgfqpoint{8.427713in}{1.852938in}}%
\pgfpathlineto{\pgfqpoint{8.427702in}{1.852938in}}%
\pgfusepath{stroke}%
\end{pgfscope}%
\begin{pgfscope}%
\pgfpathrectangle{\pgfqpoint{6.720588in}{1.750000in}}{\pgfqpoint{2.279412in}{2.004545in}}%
\pgfusepath{clip}%
\pgfsetbuttcap%
\pgfsetroundjoin%
\pgfsetlinewidth{0.308837pt}%
\definecolor{currentstroke}{rgb}{0.268510,0.009605,0.335427}%
\pgfsetstrokecolor{currentstroke}%
\pgfsetdash{}{0pt}%
\pgfpathmoveto{\pgfqpoint{8.427702in}{1.852938in}}%
\pgfpathlineto{\pgfqpoint{8.427805in}{1.852932in}}%
\pgfusepath{stroke}%
\end{pgfscope}%
\begin{pgfscope}%
\pgfpathrectangle{\pgfqpoint{6.720588in}{1.750000in}}{\pgfqpoint{2.279412in}{2.004545in}}%
\pgfusepath{clip}%
\pgfsetbuttcap%
\pgfsetroundjoin%
\pgfsetlinewidth{0.308868pt}%
\definecolor{currentstroke}{rgb}{0.268510,0.009605,0.335427}%
\pgfsetstrokecolor{currentstroke}%
\pgfsetdash{}{0pt}%
\pgfpathmoveto{\pgfqpoint{8.427805in}{1.852932in}}%
\pgfpathlineto{\pgfqpoint{8.427918in}{1.852927in}}%
\pgfusepath{stroke}%
\end{pgfscope}%
\begin{pgfscope}%
\pgfpathrectangle{\pgfqpoint{6.720588in}{1.750000in}}{\pgfqpoint{2.279412in}{2.004545in}}%
\pgfusepath{clip}%
\pgfsetbuttcap%
\pgfsetroundjoin%
\pgfsetlinewidth{0.308903pt}%
\definecolor{currentstroke}{rgb}{0.268510,0.009605,0.335427}%
\pgfsetstrokecolor{currentstroke}%
\pgfsetdash{}{0pt}%
\pgfpathmoveto{\pgfqpoint{8.427918in}{1.852927in}}%
\pgfpathlineto{\pgfqpoint{8.427931in}{1.852926in}}%
\pgfusepath{stroke}%
\end{pgfscope}%
\begin{pgfscope}%
\pgfpathrectangle{\pgfqpoint{6.720588in}{1.750000in}}{\pgfqpoint{2.279412in}{2.004545in}}%
\pgfusepath{clip}%
\pgfsetbuttcap%
\pgfsetroundjoin%
\pgfsetlinewidth{0.308906pt}%
\definecolor{currentstroke}{rgb}{0.268510,0.009605,0.335427}%
\pgfsetstrokecolor{currentstroke}%
\pgfsetdash{}{0pt}%
\pgfpathmoveto{\pgfqpoint{8.427931in}{1.852926in}}%
\pgfpathlineto{\pgfqpoint{8.427820in}{1.852932in}}%
\pgfusepath{stroke}%
\end{pgfscope}%
\begin{pgfscope}%
\pgfpathrectangle{\pgfqpoint{6.720588in}{1.750000in}}{\pgfqpoint{2.279412in}{2.004545in}}%
\pgfusepath{clip}%
\pgfsetbuttcap%
\pgfsetroundjoin%
\pgfsetlinewidth{0.308873pt}%
\definecolor{currentstroke}{rgb}{0.268510,0.009605,0.335427}%
\pgfsetstrokecolor{currentstroke}%
\pgfsetdash{}{0pt}%
\pgfpathmoveto{\pgfqpoint{8.427820in}{1.852932in}}%
\pgfpathlineto{\pgfqpoint{8.427683in}{1.852939in}}%
\pgfusepath{stroke}%
\end{pgfscope}%
\begin{pgfscope}%
\pgfpathrectangle{\pgfqpoint{6.720588in}{1.750000in}}{\pgfqpoint{2.279412in}{2.004545in}}%
\pgfusepath{clip}%
\pgfsetbuttcap%
\pgfsetroundjoin%
\pgfsetlinewidth{0.308831pt}%
\definecolor{currentstroke}{rgb}{0.268510,0.009605,0.335427}%
\pgfsetstrokecolor{currentstroke}%
\pgfsetdash{}{0pt}%
\pgfpathmoveto{\pgfqpoint{8.427683in}{1.852939in}}%
\pgfpathlineto{\pgfqpoint{8.427670in}{1.852938in}}%
\pgfusepath{stroke}%
\end{pgfscope}%
\begin{pgfscope}%
\pgfpathrectangle{\pgfqpoint{6.720588in}{1.750000in}}{\pgfqpoint{2.279412in}{2.004545in}}%
\pgfusepath{clip}%
\pgfsetbuttcap%
\pgfsetroundjoin%
\pgfsetlinewidth{0.308827pt}%
\definecolor{currentstroke}{rgb}{0.268510,0.009605,0.335427}%
\pgfsetstrokecolor{currentstroke}%
\pgfsetdash{}{0pt}%
\pgfpathmoveto{\pgfqpoint{8.427670in}{1.852938in}}%
\pgfpathlineto{\pgfqpoint{8.427804in}{1.852931in}}%
\pgfusepath{stroke}%
\end{pgfscope}%
\begin{pgfscope}%
\pgfpathrectangle{\pgfqpoint{6.720588in}{1.750000in}}{\pgfqpoint{2.279412in}{2.004545in}}%
\pgfusepath{clip}%
\pgfsetbuttcap%
\pgfsetroundjoin%
\pgfsetlinewidth{0.308868pt}%
\definecolor{currentstroke}{rgb}{0.268510,0.009605,0.335427}%
\pgfsetstrokecolor{currentstroke}%
\pgfsetdash{}{0pt}%
\pgfpathmoveto{\pgfqpoint{8.427804in}{1.852931in}}%
\pgfpathlineto{\pgfqpoint{8.427948in}{1.852924in}}%
\pgfusepath{stroke}%
\end{pgfscope}%
\begin{pgfscope}%
\pgfpathrectangle{\pgfqpoint{6.720588in}{1.750000in}}{\pgfqpoint{2.279412in}{2.004545in}}%
\pgfusepath{clip}%
\pgfsetbuttcap%
\pgfsetroundjoin%
\pgfsetlinewidth{0.308912pt}%
\definecolor{currentstroke}{rgb}{0.268510,0.009605,0.335427}%
\pgfsetstrokecolor{currentstroke}%
\pgfsetdash{}{0pt}%
\pgfpathmoveto{\pgfqpoint{8.427948in}{1.852924in}}%
\pgfpathlineto{\pgfqpoint{8.427964in}{1.852924in}}%
\pgfusepath{stroke}%
\end{pgfscope}%
\begin{pgfscope}%
\pgfpathrectangle{\pgfqpoint{6.720588in}{1.750000in}}{\pgfqpoint{2.279412in}{2.004545in}}%
\pgfusepath{clip}%
\pgfsetbuttcap%
\pgfsetroundjoin%
\pgfsetlinewidth{0.308916pt}%
\definecolor{currentstroke}{rgb}{0.268510,0.009605,0.335427}%
\pgfsetstrokecolor{currentstroke}%
\pgfsetdash{}{0pt}%
\pgfpathmoveto{\pgfqpoint{8.427964in}{1.852924in}}%
\pgfpathlineto{\pgfqpoint{8.427821in}{1.852932in}}%
\pgfusepath{stroke}%
\end{pgfscope}%
\begin{pgfscope}%
\pgfpathrectangle{\pgfqpoint{6.720588in}{1.750000in}}{\pgfqpoint{2.279412in}{2.004545in}}%
\pgfusepath{clip}%
\pgfsetbuttcap%
\pgfsetroundjoin%
\pgfsetlinewidth{0.308873pt}%
\definecolor{currentstroke}{rgb}{0.268510,0.009605,0.335427}%
\pgfsetstrokecolor{currentstroke}%
\pgfsetdash{}{0pt}%
\pgfpathmoveto{\pgfqpoint{8.427821in}{1.852932in}}%
\pgfpathlineto{\pgfqpoint{8.427643in}{1.852940in}}%
\pgfusepath{stroke}%
\end{pgfscope}%
\begin{pgfscope}%
\pgfpathrectangle{\pgfqpoint{6.720588in}{1.750000in}}{\pgfqpoint{2.279412in}{2.004545in}}%
\pgfusepath{clip}%
\pgfsetbuttcap%
\pgfsetroundjoin%
\pgfsetlinewidth{0.308819pt}%
\definecolor{currentstroke}{rgb}{0.268510,0.009605,0.335427}%
\pgfsetstrokecolor{currentstroke}%
\pgfsetdash{}{0pt}%
\pgfpathmoveto{\pgfqpoint{8.427643in}{1.852940in}}%
\pgfpathlineto{\pgfqpoint{8.427628in}{1.852940in}}%
\pgfusepath{stroke}%
\end{pgfscope}%
\begin{pgfscope}%
\pgfpathrectangle{\pgfqpoint{6.720588in}{1.750000in}}{\pgfqpoint{2.279412in}{2.004545in}}%
\pgfusepath{clip}%
\pgfsetbuttcap%
\pgfsetroundjoin%
\pgfsetlinewidth{0.308814pt}%
\definecolor{currentstroke}{rgb}{0.268510,0.009605,0.335427}%
\pgfsetstrokecolor{currentstroke}%
\pgfsetdash{}{0pt}%
\pgfpathmoveto{\pgfqpoint{8.427628in}{1.852940in}}%
\pgfpathlineto{\pgfqpoint{8.427803in}{1.852931in}}%
\pgfusepath{stroke}%
\end{pgfscope}%
\begin{pgfscope}%
\pgfpathrectangle{\pgfqpoint{6.720588in}{1.750000in}}{\pgfqpoint{2.279412in}{2.004545in}}%
\pgfusepath{clip}%
\pgfsetbuttcap%
\pgfsetroundjoin%
\pgfsetlinewidth{0.308867pt}%
\definecolor{currentstroke}{rgb}{0.268510,0.009605,0.335427}%
\pgfsetstrokecolor{currentstroke}%
\pgfsetdash{}{0pt}%
\pgfpathmoveto{\pgfqpoint{8.427803in}{1.852931in}}%
\pgfpathlineto{\pgfqpoint{8.427986in}{1.852922in}}%
\pgfusepath{stroke}%
\end{pgfscope}%
\begin{pgfscope}%
\pgfpathrectangle{\pgfqpoint{6.720588in}{1.750000in}}{\pgfqpoint{2.279412in}{2.004545in}}%
\pgfusepath{clip}%
\pgfsetbuttcap%
\pgfsetroundjoin%
\pgfsetlinewidth{0.308923pt}%
\definecolor{currentstroke}{rgb}{0.268510,0.009605,0.335427}%
\pgfsetstrokecolor{currentstroke}%
\pgfsetdash{}{0pt}%
\pgfpathmoveto{\pgfqpoint{8.427986in}{1.852922in}}%
\pgfpathlineto{\pgfqpoint{8.428005in}{1.852922in}}%
\pgfusepath{stroke}%
\end{pgfscope}%
\begin{pgfscope}%
\pgfpathrectangle{\pgfqpoint{6.720588in}{1.750000in}}{\pgfqpoint{2.279412in}{2.004545in}}%
\pgfusepath{clip}%
\pgfsetbuttcap%
\pgfsetroundjoin%
\pgfsetlinewidth{0.308929pt}%
\definecolor{currentstroke}{rgb}{0.268510,0.009605,0.335427}%
\pgfsetstrokecolor{currentstroke}%
\pgfsetdash{}{0pt}%
\pgfpathmoveto{\pgfqpoint{8.428005in}{1.852922in}}%
\pgfpathlineto{\pgfqpoint{8.427821in}{1.852932in}}%
\pgfusepath{stroke}%
\end{pgfscope}%
\begin{pgfscope}%
\pgfpathrectangle{\pgfqpoint{6.720588in}{1.750000in}}{\pgfqpoint{2.279412in}{2.004545in}}%
\pgfusepath{clip}%
\pgfsetbuttcap%
\pgfsetroundjoin%
\pgfsetlinewidth{0.308873pt}%
\definecolor{currentstroke}{rgb}{0.268510,0.009605,0.335427}%
\pgfsetstrokecolor{currentstroke}%
\pgfsetdash{}{0pt}%
\pgfpathmoveto{\pgfqpoint{8.427821in}{1.852932in}}%
\pgfpathlineto{\pgfqpoint{8.427589in}{1.852943in}}%
\pgfusepath{stroke}%
\end{pgfscope}%
\begin{pgfscope}%
\pgfpathrectangle{\pgfqpoint{6.720588in}{1.750000in}}{\pgfqpoint{2.279412in}{2.004545in}}%
\pgfusepath{clip}%
\pgfsetbuttcap%
\pgfsetroundjoin%
\pgfsetlinewidth{0.308802pt}%
\definecolor{currentstroke}{rgb}{0.268510,0.009605,0.335427}%
\pgfsetstrokecolor{currentstroke}%
\pgfsetdash{}{0pt}%
\pgfpathmoveto{\pgfqpoint{8.427589in}{1.852943in}}%
\pgfpathlineto{\pgfqpoint{8.427572in}{1.852942in}}%
\pgfusepath{stroke}%
\end{pgfscope}%
\begin{pgfscope}%
\pgfpathrectangle{\pgfqpoint{6.720588in}{1.750000in}}{\pgfqpoint{2.279412in}{2.004545in}}%
\pgfusepath{clip}%
\pgfsetbuttcap%
\pgfsetroundjoin%
\pgfsetlinewidth{0.308797pt}%
\definecolor{currentstroke}{rgb}{0.268510,0.009605,0.335427}%
\pgfsetstrokecolor{currentstroke}%
\pgfsetdash{}{0pt}%
\pgfpathmoveto{\pgfqpoint{8.427572in}{1.852942in}}%
\pgfpathlineto{\pgfqpoint{8.427801in}{1.852930in}}%
\pgfusepath{stroke}%
\end{pgfscope}%
\begin{pgfscope}%
\pgfpathrectangle{\pgfqpoint{6.720588in}{1.750000in}}{\pgfqpoint{2.279412in}{2.004545in}}%
\pgfusepath{clip}%
\pgfsetbuttcap%
\pgfsetroundjoin%
\pgfsetlinewidth{0.308867pt}%
\definecolor{currentstroke}{rgb}{0.268510,0.009605,0.335427}%
\pgfsetstrokecolor{currentstroke}%
\pgfsetdash{}{0pt}%
\pgfpathmoveto{\pgfqpoint{8.427801in}{1.852930in}}%
\pgfpathlineto{\pgfqpoint{8.428033in}{1.852919in}}%
\pgfusepath{stroke}%
\end{pgfscope}%
\begin{pgfscope}%
\pgfpathrectangle{\pgfqpoint{6.720588in}{1.750000in}}{\pgfqpoint{2.279412in}{2.004545in}}%
\pgfusepath{clip}%
\pgfsetbuttcap%
\pgfsetroundjoin%
\pgfsetlinewidth{0.308937pt}%
\definecolor{currentstroke}{rgb}{0.268510,0.009605,0.335427}%
\pgfsetstrokecolor{currentstroke}%
\pgfsetdash{}{0pt}%
\pgfpathmoveto{\pgfqpoint{8.428033in}{1.852919in}}%
\pgfpathlineto{\pgfqpoint{8.428056in}{1.852920in}}%
\pgfusepath{stroke}%
\end{pgfscope}%
\begin{pgfscope}%
\pgfpathrectangle{\pgfqpoint{6.720588in}{1.750000in}}{\pgfqpoint{2.279412in}{2.004545in}}%
\pgfusepath{clip}%
\pgfsetbuttcap%
\pgfsetroundjoin%
\pgfsetlinewidth{0.308944pt}%
\definecolor{currentstroke}{rgb}{0.268510,0.009605,0.335427}%
\pgfsetstrokecolor{currentstroke}%
\pgfsetdash{}{0pt}%
\pgfpathmoveto{\pgfqpoint{8.428056in}{1.852920in}}%
\pgfpathlineto{\pgfqpoint{8.427822in}{1.852932in}}%
\pgfusepath{stroke}%
\end{pgfscope}%
\begin{pgfscope}%
\pgfpathrectangle{\pgfqpoint{6.720588in}{1.750000in}}{\pgfqpoint{2.279412in}{2.004545in}}%
\pgfusepath{clip}%
\pgfsetbuttcap%
\pgfsetroundjoin%
\pgfsetlinewidth{0.308873pt}%
\definecolor{currentstroke}{rgb}{0.268510,0.009605,0.335427}%
\pgfsetstrokecolor{currentstroke}%
\pgfsetdash{}{0pt}%
\pgfpathmoveto{\pgfqpoint{8.427822in}{1.852932in}}%
\pgfpathlineto{\pgfqpoint{8.427519in}{1.852946in}}%
\pgfusepath{stroke}%
\end{pgfscope}%
\begin{pgfscope}%
\pgfpathrectangle{\pgfqpoint{6.720588in}{1.750000in}}{\pgfqpoint{2.279412in}{2.004545in}}%
\pgfusepath{clip}%
\pgfsetbuttcap%
\pgfsetroundjoin%
\pgfsetlinewidth{0.308781pt}%
\definecolor{currentstroke}{rgb}{0.268510,0.009605,0.335427}%
\pgfsetstrokecolor{currentstroke}%
\pgfsetdash{}{0pt}%
\pgfpathmoveto{\pgfqpoint{8.427519in}{1.852946in}}%
\pgfpathlineto{\pgfqpoint{8.427500in}{1.852946in}}%
\pgfusepath{stroke}%
\end{pgfscope}%
\begin{pgfscope}%
\pgfpathrectangle{\pgfqpoint{6.720588in}{1.750000in}}{\pgfqpoint{2.279412in}{2.004545in}}%
\pgfusepath{clip}%
\pgfsetbuttcap%
\pgfsetroundjoin%
\pgfsetlinewidth{0.308775pt}%
\definecolor{currentstroke}{rgb}{0.268510,0.009605,0.335427}%
\pgfsetstrokecolor{currentstroke}%
\pgfsetdash{}{0pt}%
\pgfpathmoveto{\pgfqpoint{8.427500in}{1.852946in}}%
\pgfpathlineto{\pgfqpoint{8.427799in}{1.852930in}}%
\pgfusepath{stroke}%
\end{pgfscope}%
\begin{pgfscope}%
\pgfpathrectangle{\pgfqpoint{6.720588in}{1.750000in}}{\pgfqpoint{2.279412in}{2.004545in}}%
\pgfusepath{clip}%
\pgfsetbuttcap%
\pgfsetroundjoin%
\pgfsetlinewidth{0.308866pt}%
\definecolor{currentstroke}{rgb}{0.268510,0.009605,0.335427}%
\pgfsetstrokecolor{currentstroke}%
\pgfsetdash{}{0pt}%
\pgfpathmoveto{\pgfqpoint{8.427799in}{1.852930in}}%
\pgfpathlineto{\pgfqpoint{8.428091in}{1.852916in}}%
\pgfusepath{stroke}%
\end{pgfscope}%
\begin{pgfscope}%
\pgfpathrectangle{\pgfqpoint{6.720588in}{1.750000in}}{\pgfqpoint{2.279412in}{2.004545in}}%
\pgfusepath{clip}%
\pgfsetbuttcap%
\pgfsetroundjoin%
\pgfsetlinewidth{0.308955pt}%
\definecolor{currentstroke}{rgb}{0.268510,0.009605,0.335427}%
\pgfsetstrokecolor{currentstroke}%
\pgfsetdash{}{0pt}%
\pgfpathmoveto{\pgfqpoint{8.428091in}{1.852916in}}%
\pgfpathlineto{\pgfqpoint{8.428119in}{1.852916in}}%
\pgfusepath{stroke}%
\end{pgfscope}%
\begin{pgfscope}%
\pgfpathrectangle{\pgfqpoint{6.720588in}{1.750000in}}{\pgfqpoint{2.279412in}{2.004545in}}%
\pgfusepath{clip}%
\pgfsetbuttcap%
\pgfsetroundjoin%
\pgfsetlinewidth{0.308964pt}%
\definecolor{currentstroke}{rgb}{0.268510,0.009605,0.335427}%
\pgfsetstrokecolor{currentstroke}%
\pgfsetdash{}{0pt}%
\pgfpathmoveto{\pgfqpoint{8.428119in}{1.852916in}}%
\pgfpathlineto{\pgfqpoint{8.427822in}{1.852932in}}%
\pgfusepath{stroke}%
\end{pgfscope}%
\begin{pgfscope}%
\pgfpathrectangle{\pgfqpoint{6.720588in}{1.750000in}}{\pgfqpoint{2.279412in}{2.004545in}}%
\pgfusepath{clip}%
\pgfsetbuttcap%
\pgfsetroundjoin%
\pgfsetlinewidth{0.308873pt}%
\definecolor{currentstroke}{rgb}{0.268510,0.009605,0.335427}%
\pgfsetstrokecolor{currentstroke}%
\pgfsetdash{}{0pt}%
\pgfpathmoveto{\pgfqpoint{8.427822in}{1.852932in}}%
\pgfpathlineto{\pgfqpoint{8.427427in}{1.852951in}}%
\pgfusepath{stroke}%
\end{pgfscope}%
\begin{pgfscope}%
\pgfpathrectangle{\pgfqpoint{6.720588in}{1.750000in}}{\pgfqpoint{2.279412in}{2.004545in}}%
\pgfusepath{clip}%
\pgfsetbuttcap%
\pgfsetroundjoin%
\pgfsetlinewidth{0.308752pt}%
\definecolor{currentstroke}{rgb}{0.268510,0.009605,0.335427}%
\pgfsetstrokecolor{currentstroke}%
\pgfsetdash{}{0pt}%
\pgfpathmoveto{\pgfqpoint{8.427427in}{1.852951in}}%
\pgfpathlineto{\pgfqpoint{8.427407in}{1.852950in}}%
\pgfusepath{stroke}%
\end{pgfscope}%
\begin{pgfscope}%
\pgfpathrectangle{\pgfqpoint{6.720588in}{1.750000in}}{\pgfqpoint{2.279412in}{2.004545in}}%
\pgfusepath{clip}%
\pgfsetbuttcap%
\pgfsetroundjoin%
\pgfsetlinewidth{0.308746pt}%
\definecolor{currentstroke}{rgb}{0.268510,0.009605,0.335427}%
\pgfsetstrokecolor{currentstroke}%
\pgfsetdash{}{0pt}%
\pgfpathmoveto{\pgfqpoint{8.427407in}{1.852950in}}%
\pgfpathlineto{\pgfqpoint{8.427796in}{1.852929in}}%
\pgfusepath{stroke}%
\end{pgfscope}%
\begin{pgfscope}%
\pgfpathrectangle{\pgfqpoint{6.720588in}{1.750000in}}{\pgfqpoint{2.279412in}{2.004545in}}%
\pgfusepath{clip}%
\pgfsetbuttcap%
\pgfsetroundjoin%
\pgfsetlinewidth{0.308865pt}%
\definecolor{currentstroke}{rgb}{0.268510,0.009605,0.335427}%
\pgfsetstrokecolor{currentstroke}%
\pgfsetdash{}{0pt}%
\pgfpathmoveto{\pgfqpoint{8.427796in}{1.852929in}}%
\pgfpathlineto{\pgfqpoint{8.428159in}{1.852912in}}%
\pgfusepath{stroke}%
\end{pgfscope}%
\begin{pgfscope}%
\pgfpathrectangle{\pgfqpoint{6.720588in}{1.750000in}}{\pgfqpoint{2.279412in}{2.004545in}}%
\pgfusepath{clip}%
\pgfsetbuttcap%
\pgfsetroundjoin%
\pgfsetlinewidth{0.308976pt}%
\definecolor{currentstroke}{rgb}{0.268510,0.009605,0.335427}%
\pgfsetstrokecolor{currentstroke}%
\pgfsetdash{}{0pt}%
\pgfpathmoveto{\pgfqpoint{8.428159in}{1.852912in}}%
\pgfpathlineto{\pgfqpoint{8.428193in}{1.852912in}}%
\pgfusepath{stroke}%
\end{pgfscope}%
\begin{pgfscope}%
\pgfpathrectangle{\pgfqpoint{6.720588in}{1.750000in}}{\pgfqpoint{2.279412in}{2.004545in}}%
\pgfusepath{clip}%
\pgfsetbuttcap%
\pgfsetroundjoin%
\pgfsetlinewidth{0.308986pt}%
\definecolor{currentstroke}{rgb}{0.268510,0.009605,0.335427}%
\pgfsetstrokecolor{currentstroke}%
\pgfsetdash{}{0pt}%
\pgfpathmoveto{\pgfqpoint{8.428193in}{1.852912in}}%
\pgfpathlineto{\pgfqpoint{8.427822in}{1.852932in}}%
\pgfusepath{stroke}%
\end{pgfscope}%
\begin{pgfscope}%
\pgfpathrectangle{\pgfqpoint{6.720588in}{1.750000in}}{\pgfqpoint{2.279412in}{2.004545in}}%
\pgfusepath{clip}%
\pgfsetbuttcap%
\pgfsetroundjoin%
\pgfsetlinewidth{0.308873pt}%
\definecolor{currentstroke}{rgb}{0.268510,0.009605,0.335427}%
\pgfsetstrokecolor{currentstroke}%
\pgfsetdash{}{0pt}%
\pgfpathmoveto{\pgfqpoint{8.427822in}{1.852932in}}%
\pgfpathlineto{\pgfqpoint{8.427307in}{1.852957in}}%
\pgfusepath{stroke}%
\end{pgfscope}%
\begin{pgfscope}%
\pgfpathrectangle{\pgfqpoint{6.720588in}{1.750000in}}{\pgfqpoint{2.279412in}{2.004545in}}%
\pgfusepath{clip}%
\pgfsetbuttcap%
\pgfsetroundjoin%
\pgfsetlinewidth{0.308716pt}%
\definecolor{currentstroke}{rgb}{0.268510,0.009605,0.335427}%
\pgfsetstrokecolor{currentstroke}%
\pgfsetdash{}{0pt}%
\pgfpathmoveto{\pgfqpoint{8.427307in}{1.852957in}}%
\pgfpathlineto{\pgfqpoint{8.427287in}{1.852954in}}%
\pgfusepath{stroke}%
\end{pgfscope}%
\begin{pgfscope}%
\pgfpathrectangle{\pgfqpoint{6.720588in}{1.750000in}}{\pgfqpoint{2.279412in}{2.004545in}}%
\pgfusepath{clip}%
\pgfsetbuttcap%
\pgfsetroundjoin%
\pgfsetlinewidth{0.308710pt}%
\definecolor{currentstroke}{rgb}{0.268510,0.009605,0.335427}%
\pgfsetstrokecolor{currentstroke}%
\pgfsetdash{}{0pt}%
\pgfpathmoveto{\pgfqpoint{8.427287in}{1.852954in}}%
\pgfpathlineto{\pgfqpoint{8.427791in}{1.852928in}}%
\pgfusepath{stroke}%
\end{pgfscope}%
\begin{pgfscope}%
\pgfpathrectangle{\pgfqpoint{6.720588in}{1.750000in}}{\pgfqpoint{2.279412in}{2.004545in}}%
\pgfusepath{clip}%
\pgfsetbuttcap%
\pgfsetroundjoin%
\pgfsetlinewidth{0.308864pt}%
\definecolor{currentstroke}{rgb}{0.268510,0.009605,0.335427}%
\pgfsetstrokecolor{currentstroke}%
\pgfsetdash{}{0pt}%
\pgfpathmoveto{\pgfqpoint{8.427791in}{1.852928in}}%
\pgfpathlineto{\pgfqpoint{8.428237in}{1.852907in}}%
\pgfusepath{stroke}%
\end{pgfscope}%
\begin{pgfscope}%
\pgfpathrectangle{\pgfqpoint{6.720588in}{1.750000in}}{\pgfqpoint{2.279412in}{2.004545in}}%
\pgfusepath{clip}%
\pgfsetbuttcap%
\pgfsetroundjoin%
\pgfsetlinewidth{0.308999pt}%
\definecolor{currentstroke}{rgb}{0.268510,0.009605,0.335427}%
\pgfsetstrokecolor{currentstroke}%
\pgfsetdash{}{0pt}%
\pgfpathmoveto{\pgfqpoint{8.428237in}{1.852907in}}%
\pgfpathlineto{\pgfqpoint{8.428281in}{1.852907in}}%
\pgfusepath{stroke}%
\end{pgfscope}%
\begin{pgfscope}%
\pgfpathrectangle{\pgfqpoint{6.720588in}{1.750000in}}{\pgfqpoint{2.279412in}{2.004545in}}%
\pgfusepath{clip}%
\pgfsetbuttcap%
\pgfsetroundjoin%
\pgfsetlinewidth{0.309013pt}%
\definecolor{currentstroke}{rgb}{0.268510,0.009605,0.335427}%
\pgfsetstrokecolor{currentstroke}%
\pgfsetdash{}{0pt}%
\pgfpathmoveto{\pgfqpoint{8.428281in}{1.852907in}}%
\pgfpathlineto{\pgfqpoint{8.427824in}{1.852932in}}%
\pgfusepath{stroke}%
\end{pgfscope}%
\begin{pgfscope}%
\pgfpathrectangle{\pgfqpoint{6.720588in}{1.750000in}}{\pgfqpoint{2.279412in}{2.004545in}}%
\pgfusepath{clip}%
\pgfsetbuttcap%
\pgfsetroundjoin%
\pgfsetlinewidth{0.308874pt}%
\definecolor{currentstroke}{rgb}{0.268510,0.009605,0.335427}%
\pgfsetstrokecolor{currentstroke}%
\pgfsetdash{}{0pt}%
\pgfpathmoveto{\pgfqpoint{8.427824in}{1.852932in}}%
\pgfpathlineto{\pgfqpoint{8.427153in}{1.852964in}}%
\pgfusepath{stroke}%
\end{pgfscope}%
\begin{pgfscope}%
\pgfpathrectangle{\pgfqpoint{6.720588in}{1.750000in}}{\pgfqpoint{2.279412in}{2.004545in}}%
\pgfusepath{clip}%
\pgfsetbuttcap%
\pgfsetroundjoin%
\pgfsetlinewidth{0.308669pt}%
\definecolor{currentstroke}{rgb}{0.268510,0.009605,0.335427}%
\pgfsetstrokecolor{currentstroke}%
\pgfsetdash{}{0pt}%
\pgfpathmoveto{\pgfqpoint{8.427153in}{1.852964in}}%
\pgfpathlineto{\pgfqpoint{8.427153in}{1.852964in}}%
\pgfusepath{stroke}%
\end{pgfscope}%
\begin{pgfscope}%
\pgfpathrectangle{\pgfqpoint{6.720588in}{1.750000in}}{\pgfqpoint{2.279412in}{2.004545in}}%
\pgfusepath{clip}%
\pgfsetbuttcap%
\pgfsetroundjoin%
\pgfsetlinewidth{0.308669pt}%
\definecolor{currentstroke}{rgb}{0.268510,0.009605,0.335427}%
\pgfsetstrokecolor{currentstroke}%
\pgfsetdash{}{0pt}%
\pgfpathmoveto{\pgfqpoint{8.427153in}{1.852964in}}%
\pgfpathlineto{\pgfqpoint{8.427794in}{1.852930in}}%
\pgfusepath{stroke}%
\end{pgfscope}%
\begin{pgfscope}%
\pgfpathrectangle{\pgfqpoint{6.720588in}{1.750000in}}{\pgfqpoint{2.279412in}{2.004545in}}%
\pgfusepath{clip}%
\pgfsetbuttcap%
\pgfsetroundjoin%
\pgfsetlinewidth{0.308864pt}%
\definecolor{currentstroke}{rgb}{0.268510,0.009605,0.335427}%
\pgfsetstrokecolor{currentstroke}%
\pgfsetdash{}{0pt}%
\pgfpathmoveto{\pgfqpoint{8.427794in}{1.852930in}}%
\pgfpathlineto{\pgfqpoint{8.428320in}{1.852905in}}%
\pgfusepath{stroke}%
\end{pgfscope}%
\begin{pgfscope}%
\pgfpathrectangle{\pgfqpoint{6.720588in}{1.750000in}}{\pgfqpoint{2.279412in}{2.004545in}}%
\pgfusepath{clip}%
\pgfsetbuttcap%
\pgfsetroundjoin%
\pgfsetlinewidth{0.309024pt}%
\definecolor{currentstroke}{rgb}{0.268510,0.009605,0.335427}%
\pgfsetstrokecolor{currentstroke}%
\pgfsetdash{}{0pt}%
\pgfpathmoveto{\pgfqpoint{8.428320in}{1.852905in}}%
\pgfpathlineto{\pgfqpoint{8.428366in}{1.852905in}}%
\pgfusepath{stroke}%
\end{pgfscope}%
\begin{pgfscope}%
\pgfpathrectangle{\pgfqpoint{6.720588in}{1.750000in}}{\pgfqpoint{2.279412in}{2.004545in}}%
\pgfusepath{clip}%
\pgfsetbuttcap%
\pgfsetroundjoin%
\pgfsetlinewidth{0.309039pt}%
\definecolor{currentstroke}{rgb}{0.268510,0.009605,0.335427}%
\pgfsetstrokecolor{currentstroke}%
\pgfsetdash{}{0pt}%
\pgfpathmoveto{\pgfqpoint{8.428366in}{1.852905in}}%
\pgfpathlineto{\pgfqpoint{8.427814in}{1.852934in}}%
\pgfusepath{stroke}%
\end{pgfscope}%
\begin{pgfscope}%
\pgfpathrectangle{\pgfqpoint{6.720588in}{1.750000in}}{\pgfqpoint{2.279412in}{2.004545in}}%
\pgfusepath{clip}%
\pgfsetbuttcap%
\pgfsetroundjoin%
\pgfsetlinewidth{0.308871pt}%
\definecolor{currentstroke}{rgb}{0.268510,0.009605,0.335427}%
\pgfsetstrokecolor{currentstroke}%
\pgfsetdash{}{0pt}%
\pgfpathmoveto{\pgfqpoint{8.427814in}{1.852934in}}%
\pgfpathlineto{\pgfqpoint{8.426967in}{1.852974in}}%
\pgfusepath{stroke}%
\end{pgfscope}%
\begin{pgfscope}%
\pgfpathrectangle{\pgfqpoint{6.720588in}{1.750000in}}{\pgfqpoint{2.279412in}{2.004545in}}%
\pgfusepath{clip}%
\pgfsetbuttcap%
\pgfsetroundjoin%
\pgfsetlinewidth{0.308611pt}%
\definecolor{currentstroke}{rgb}{0.268510,0.009605,0.335427}%
\pgfsetstrokecolor{currentstroke}%
\pgfsetdash{}{0pt}%
\pgfpathmoveto{\pgfqpoint{8.426967in}{1.852974in}}%
\pgfpathlineto{\pgfqpoint{8.426967in}{1.852974in}}%
\pgfusepath{stroke}%
\end{pgfscope}%
\begin{pgfscope}%
\pgfpathrectangle{\pgfqpoint{6.720588in}{1.750000in}}{\pgfqpoint{2.279412in}{2.004545in}}%
\pgfusepath{clip}%
\pgfsetbuttcap%
\pgfsetroundjoin%
\pgfsetlinewidth{0.308611pt}%
\definecolor{currentstroke}{rgb}{0.268510,0.009605,0.335427}%
\pgfsetstrokecolor{currentstroke}%
\pgfsetdash{}{0pt}%
\pgfpathmoveto{\pgfqpoint{8.426967in}{1.852974in}}%
\pgfpathlineto{\pgfqpoint{8.427777in}{1.852931in}}%
\pgfusepath{stroke}%
\end{pgfscope}%
\begin{pgfscope}%
\pgfpathrectangle{\pgfqpoint{6.720588in}{1.750000in}}{\pgfqpoint{2.279412in}{2.004545in}}%
\pgfusepath{clip}%
\pgfsetbuttcap%
\pgfsetroundjoin%
\pgfsetlinewidth{0.308859pt}%
\definecolor{currentstroke}{rgb}{0.268510,0.009605,0.335427}%
\pgfsetstrokecolor{currentstroke}%
\pgfsetdash{}{0pt}%
\pgfpathmoveto{\pgfqpoint{8.427777in}{1.852931in}}%
\pgfpathlineto{\pgfqpoint{8.428404in}{1.852900in}}%
\pgfusepath{stroke}%
\end{pgfscope}%
\begin{pgfscope}%
\pgfpathrectangle{\pgfqpoint{6.720588in}{1.750000in}}{\pgfqpoint{2.279412in}{2.004545in}}%
\pgfusepath{clip}%
\pgfsetbuttcap%
\pgfsetroundjoin%
\pgfsetlinewidth{0.309050pt}%
\definecolor{currentstroke}{rgb}{0.268510,0.009605,0.335427}%
\pgfsetstrokecolor{currentstroke}%
\pgfsetdash{}{0pt}%
\pgfpathmoveto{\pgfqpoint{8.428404in}{1.852900in}}%
\pgfpathlineto{\pgfqpoint{8.428470in}{1.852900in}}%
\pgfusepath{stroke}%
\end{pgfscope}%
\begin{pgfscope}%
\pgfpathrectangle{\pgfqpoint{6.720588in}{1.750000in}}{\pgfqpoint{2.279412in}{2.004545in}}%
\pgfusepath{clip}%
\pgfsetbuttcap%
\pgfsetroundjoin%
\pgfsetlinewidth{0.309070pt}%
\definecolor{currentstroke}{rgb}{0.268510,0.009605,0.335427}%
\pgfsetstrokecolor{currentstroke}%
\pgfsetdash{}{0pt}%
\pgfpathmoveto{\pgfqpoint{8.428470in}{1.852900in}}%
\pgfpathlineto{\pgfqpoint{8.427826in}{1.852934in}}%
\pgfusepath{stroke}%
\end{pgfscope}%
\begin{pgfscope}%
\pgfpathrectangle{\pgfqpoint{6.720588in}{1.750000in}}{\pgfqpoint{2.279412in}{2.004545in}}%
\pgfusepath{clip}%
\pgfsetbuttcap%
\pgfsetroundjoin%
\pgfsetlinewidth{0.308874pt}%
\definecolor{currentstroke}{rgb}{0.268510,0.009605,0.335427}%
\pgfsetstrokecolor{currentstroke}%
\pgfsetdash{}{0pt}%
\pgfpathmoveto{\pgfqpoint{8.427826in}{1.852934in}}%
\pgfpathlineto{\pgfqpoint{8.427826in}{1.852934in}}%
\pgfusepath{stroke}%
\end{pgfscope}%
\begin{pgfscope}%
\pgfpathrectangle{\pgfqpoint{6.720588in}{1.750000in}}{\pgfqpoint{2.279412in}{2.004545in}}%
\pgfusepath{clip}%
\pgfsetbuttcap%
\pgfsetroundjoin%
\pgfsetlinewidth{0.308874pt}%
\definecolor{currentstroke}{rgb}{0.268510,0.009605,0.335427}%
\pgfsetstrokecolor{currentstroke}%
\pgfsetdash{}{0pt}%
\pgfpathmoveto{\pgfqpoint{8.427826in}{1.852934in}}%
\pgfpathlineto{\pgfqpoint{8.427693in}{1.852940in}}%
\pgfusepath{stroke}%
\end{pgfscope}%
\begin{pgfscope}%
\pgfpathrectangle{\pgfqpoint{6.720588in}{1.750000in}}{\pgfqpoint{2.279412in}{2.004545in}}%
\pgfusepath{clip}%
\pgfsetbuttcap%
\pgfsetroundjoin%
\pgfsetlinewidth{0.308834pt}%
\definecolor{currentstroke}{rgb}{0.268510,0.009605,0.335427}%
\pgfsetstrokecolor{currentstroke}%
\pgfsetdash{}{0pt}%
\pgfpathmoveto{\pgfqpoint{8.427693in}{1.852940in}}%
\pgfpathlineto{\pgfqpoint{8.427673in}{1.852940in}}%
\pgfusepath{stroke}%
\end{pgfscope}%
\begin{pgfscope}%
\pgfpathrectangle{\pgfqpoint{6.720588in}{1.750000in}}{\pgfqpoint{2.279412in}{2.004545in}}%
\pgfusepath{clip}%
\pgfsetbuttcap%
\pgfsetroundjoin%
\pgfsetlinewidth{0.308828pt}%
\definecolor{currentstroke}{rgb}{0.268510,0.009605,0.335427}%
\pgfsetstrokecolor{currentstroke}%
\pgfsetdash{}{0pt}%
\pgfpathmoveto{\pgfqpoint{8.427673in}{1.852940in}}%
\pgfpathlineto{\pgfqpoint{8.427797in}{1.852933in}}%
\pgfusepath{stroke}%
\end{pgfscope}%
\begin{pgfscope}%
\pgfpathrectangle{\pgfqpoint{6.720588in}{1.750000in}}{\pgfqpoint{2.279412in}{2.004545in}}%
\pgfusepath{clip}%
\pgfsetbuttcap%
\pgfsetroundjoin%
\pgfsetlinewidth{0.308866pt}%
\definecolor{currentstroke}{rgb}{0.268510,0.009605,0.335427}%
\pgfsetstrokecolor{currentstroke}%
\pgfsetdash{}{0pt}%
\pgfpathmoveto{\pgfqpoint{8.427797in}{1.852933in}}%
\pgfpathlineto{\pgfqpoint{8.427938in}{1.852926in}}%
\pgfusepath{stroke}%
\end{pgfscope}%
\begin{pgfscope}%
\pgfpathrectangle{\pgfqpoint{6.720588in}{1.750000in}}{\pgfqpoint{2.279412in}{2.004545in}}%
\pgfusepath{clip}%
\pgfsetbuttcap%
\pgfsetroundjoin%
\pgfsetlinewidth{0.308909pt}%
\definecolor{currentstroke}{rgb}{0.268510,0.009605,0.335427}%
\pgfsetstrokecolor{currentstroke}%
\pgfsetdash{}{0pt}%
\pgfpathmoveto{\pgfqpoint{8.427938in}{1.852926in}}%
\pgfpathlineto{\pgfqpoint{8.427960in}{1.852925in}}%
\pgfusepath{stroke}%
\end{pgfscope}%
\begin{pgfscope}%
\pgfpathrectangle{\pgfqpoint{6.720588in}{1.750000in}}{\pgfqpoint{2.279412in}{2.004545in}}%
\pgfusepath{clip}%
\pgfsetbuttcap%
\pgfsetroundjoin%
\pgfsetlinewidth{0.308915pt}%
\definecolor{currentstroke}{rgb}{0.268510,0.009605,0.335427}%
\pgfsetstrokecolor{currentstroke}%
\pgfsetdash{}{0pt}%
\pgfpathmoveto{\pgfqpoint{8.427960in}{1.852925in}}%
\pgfpathlineto{\pgfqpoint{8.427829in}{1.852932in}}%
\pgfusepath{stroke}%
\end{pgfscope}%
\begin{pgfscope}%
\pgfpathrectangle{\pgfqpoint{6.720588in}{1.750000in}}{\pgfqpoint{2.279412in}{2.004545in}}%
\pgfusepath{clip}%
\pgfsetbuttcap%
\pgfsetroundjoin%
\pgfsetlinewidth{0.308875pt}%
\definecolor{currentstroke}{rgb}{0.268510,0.009605,0.335427}%
\pgfsetstrokecolor{currentstroke}%
\pgfsetdash{}{0pt}%
\pgfpathmoveto{\pgfqpoint{8.427829in}{1.852932in}}%
\pgfpathlineto{\pgfqpoint{8.427656in}{1.852940in}}%
\pgfusepath{stroke}%
\end{pgfscope}%
\begin{pgfscope}%
\pgfpathrectangle{\pgfqpoint{6.720588in}{1.750000in}}{\pgfqpoint{2.279412in}{2.004545in}}%
\pgfusepath{clip}%
\pgfsetbuttcap%
\pgfsetroundjoin%
\pgfsetlinewidth{0.308823pt}%
\definecolor{currentstroke}{rgb}{0.268510,0.009605,0.335427}%
\pgfsetstrokecolor{currentstroke}%
\pgfsetdash{}{0pt}%
\pgfpathmoveto{\pgfqpoint{8.427656in}{1.852940in}}%
\pgfpathlineto{\pgfqpoint{8.427632in}{1.852940in}}%
\pgfusepath{stroke}%
\end{pgfscope}%
\begin{pgfscope}%
\pgfpathrectangle{\pgfqpoint{6.720588in}{1.750000in}}{\pgfqpoint{2.279412in}{2.004545in}}%
\pgfusepath{clip}%
\pgfsetbuttcap%
\pgfsetroundjoin%
\pgfsetlinewidth{0.308815pt}%
\definecolor{currentstroke}{rgb}{0.268510,0.009605,0.335427}%
\pgfsetstrokecolor{currentstroke}%
\pgfsetdash{}{0pt}%
\pgfpathmoveto{\pgfqpoint{8.427632in}{1.852940in}}%
\pgfpathlineto{\pgfqpoint{8.427794in}{1.852932in}}%
\pgfusepath{stroke}%
\end{pgfscope}%
\begin{pgfscope}%
\pgfpathrectangle{\pgfqpoint{6.720588in}{1.750000in}}{\pgfqpoint{2.279412in}{2.004545in}}%
\pgfusepath{clip}%
\pgfsetbuttcap%
\pgfsetroundjoin%
\pgfsetlinewidth{0.308865pt}%
\definecolor{currentstroke}{rgb}{0.268510,0.009605,0.335427}%
\pgfsetstrokecolor{currentstroke}%
\pgfsetdash{}{0pt}%
\pgfpathmoveto{\pgfqpoint{8.427794in}{1.852932in}}%
\pgfpathlineto{\pgfqpoint{8.427974in}{1.852923in}}%
\pgfusepath{stroke}%
\end{pgfscope}%
\begin{pgfscope}%
\pgfpathrectangle{\pgfqpoint{6.720588in}{1.750000in}}{\pgfqpoint{2.279412in}{2.004545in}}%
\pgfusepath{clip}%
\pgfsetbuttcap%
\pgfsetroundjoin%
\pgfsetlinewidth{0.308919pt}%
\definecolor{currentstroke}{rgb}{0.268510,0.009605,0.335427}%
\pgfsetstrokecolor{currentstroke}%
\pgfsetdash{}{0pt}%
\pgfpathmoveto{\pgfqpoint{8.427974in}{1.852923in}}%
\pgfpathlineto{\pgfqpoint{8.428001in}{1.852922in}}%
\pgfusepath{stroke}%
\end{pgfscope}%
\begin{pgfscope}%
\pgfpathrectangle{\pgfqpoint{6.720588in}{1.750000in}}{\pgfqpoint{2.279412in}{2.004545in}}%
\pgfusepath{clip}%
\pgfsetbuttcap%
\pgfsetroundjoin%
\pgfsetlinewidth{0.308928pt}%
\definecolor{currentstroke}{rgb}{0.268510,0.009605,0.335427}%
\pgfsetstrokecolor{currentstroke}%
\pgfsetdash{}{0pt}%
\pgfpathmoveto{\pgfqpoint{8.428001in}{1.852922in}}%
\pgfpathlineto{\pgfqpoint{8.427832in}{1.852932in}}%
\pgfusepath{stroke}%
\end{pgfscope}%
\begin{pgfscope}%
\pgfpathrectangle{\pgfqpoint{6.720588in}{1.750000in}}{\pgfqpoint{2.279412in}{2.004545in}}%
\pgfusepath{clip}%
\pgfsetbuttcap%
\pgfsetroundjoin%
\pgfsetlinewidth{0.308876pt}%
\definecolor{currentstroke}{rgb}{0.268510,0.009605,0.335427}%
\pgfsetstrokecolor{currentstroke}%
\pgfsetdash{}{0pt}%
\pgfpathmoveto{\pgfqpoint{8.427832in}{1.852932in}}%
\pgfpathlineto{\pgfqpoint{8.427607in}{1.852942in}}%
\pgfusepath{stroke}%
\end{pgfscope}%
\begin{pgfscope}%
\pgfpathrectangle{\pgfqpoint{6.720588in}{1.750000in}}{\pgfqpoint{2.279412in}{2.004545in}}%
\pgfusepath{clip}%
\pgfsetbuttcap%
\pgfsetroundjoin%
\pgfsetlinewidth{0.308808pt}%
\definecolor{currentstroke}{rgb}{0.268510,0.009605,0.335427}%
\pgfsetstrokecolor{currentstroke}%
\pgfsetdash{}{0pt}%
\pgfpathmoveto{\pgfqpoint{8.427607in}{1.852942in}}%
\pgfpathlineto{\pgfqpoint{8.427578in}{1.852942in}}%
\pgfusepath{stroke}%
\end{pgfscope}%
\begin{pgfscope}%
\pgfpathrectangle{\pgfqpoint{6.720588in}{1.750000in}}{\pgfqpoint{2.279412in}{2.004545in}}%
\pgfusepath{clip}%
\pgfsetbuttcap%
\pgfsetroundjoin%
\pgfsetlinewidth{0.308799pt}%
\definecolor{currentstroke}{rgb}{0.268510,0.009605,0.335427}%
\pgfsetstrokecolor{currentstroke}%
\pgfsetdash{}{0pt}%
\pgfpathmoveto{\pgfqpoint{8.427578in}{1.852942in}}%
\pgfpathlineto{\pgfqpoint{8.427790in}{1.852931in}}%
\pgfusepath{stroke}%
\end{pgfscope}%
\begin{pgfscope}%
\pgfpathrectangle{\pgfqpoint{6.720588in}{1.750000in}}{\pgfqpoint{2.279412in}{2.004545in}}%
\pgfusepath{clip}%
\pgfsetbuttcap%
\pgfsetroundjoin%
\pgfsetlinewidth{0.308863pt}%
\definecolor{currentstroke}{rgb}{0.268510,0.009605,0.335427}%
\pgfsetstrokecolor{currentstroke}%
\pgfsetdash{}{0pt}%
\pgfpathmoveto{\pgfqpoint{8.427790in}{1.852931in}}%
\pgfpathlineto{\pgfqpoint{8.428018in}{1.852920in}}%
\pgfusepath{stroke}%
\end{pgfscope}%
\begin{pgfscope}%
\pgfpathrectangle{\pgfqpoint{6.720588in}{1.750000in}}{\pgfqpoint{2.279412in}{2.004545in}}%
\pgfusepath{clip}%
\pgfsetbuttcap%
\pgfsetroundjoin%
\pgfsetlinewidth{0.308933pt}%
\definecolor{currentstroke}{rgb}{0.268510,0.009605,0.335427}%
\pgfsetstrokecolor{currentstroke}%
\pgfsetdash{}{0pt}%
\pgfpathmoveto{\pgfqpoint{8.428018in}{1.852920in}}%
\pgfpathlineto{\pgfqpoint{8.428051in}{1.852920in}}%
\pgfusepath{stroke}%
\end{pgfscope}%
\begin{pgfscope}%
\pgfpathrectangle{\pgfqpoint{6.720588in}{1.750000in}}{\pgfqpoint{2.279412in}{2.004545in}}%
\pgfusepath{clip}%
\pgfsetbuttcap%
\pgfsetroundjoin%
\pgfsetlinewidth{0.308943pt}%
\definecolor{currentstroke}{rgb}{0.268510,0.009605,0.335427}%
\pgfsetstrokecolor{currentstroke}%
\pgfsetdash{}{0pt}%
\pgfpathmoveto{\pgfqpoint{8.428051in}{1.852920in}}%
\pgfpathlineto{\pgfqpoint{8.427836in}{1.852931in}}%
\pgfusepath{stroke}%
\end{pgfscope}%
\begin{pgfscope}%
\pgfpathrectangle{\pgfqpoint{6.720588in}{1.750000in}}{\pgfqpoint{2.279412in}{2.004545in}}%
\pgfusepath{clip}%
\pgfsetbuttcap%
\pgfsetroundjoin%
\pgfsetlinewidth{0.308877pt}%
\definecolor{currentstroke}{rgb}{0.268510,0.009605,0.335427}%
\pgfsetstrokecolor{currentstroke}%
\pgfsetdash{}{0pt}%
\pgfpathmoveto{\pgfqpoint{8.427836in}{1.852931in}}%
\pgfpathlineto{\pgfqpoint{8.427542in}{1.852945in}}%
\pgfusepath{stroke}%
\end{pgfscope}%
\begin{pgfscope}%
\pgfpathrectangle{\pgfqpoint{6.720588in}{1.750000in}}{\pgfqpoint{2.279412in}{2.004545in}}%
\pgfusepath{clip}%
\pgfsetbuttcap%
\pgfsetroundjoin%
\pgfsetlinewidth{0.308788pt}%
\definecolor{currentstroke}{rgb}{0.268510,0.009605,0.335427}%
\pgfsetstrokecolor{currentstroke}%
\pgfsetdash{}{0pt}%
\pgfpathmoveto{\pgfqpoint{8.427542in}{1.852945in}}%
\pgfpathlineto{\pgfqpoint{8.427507in}{1.852945in}}%
\pgfusepath{stroke}%
\end{pgfscope}%
\begin{pgfscope}%
\pgfpathrectangle{\pgfqpoint{6.720588in}{1.750000in}}{\pgfqpoint{2.279412in}{2.004545in}}%
\pgfusepath{clip}%
\pgfsetbuttcap%
\pgfsetroundjoin%
\pgfsetlinewidth{0.308777pt}%
\definecolor{currentstroke}{rgb}{0.268510,0.009605,0.335427}%
\pgfsetstrokecolor{currentstroke}%
\pgfsetdash{}{0pt}%
\pgfpathmoveto{\pgfqpoint{8.427507in}{1.852945in}}%
\pgfpathlineto{\pgfqpoint{8.427784in}{1.852931in}}%
\pgfusepath{stroke}%
\end{pgfscope}%
\begin{pgfscope}%
\pgfpathrectangle{\pgfqpoint{6.720588in}{1.750000in}}{\pgfqpoint{2.279412in}{2.004545in}}%
\pgfusepath{clip}%
\pgfsetbuttcap%
\pgfsetroundjoin%
\pgfsetlinewidth{0.308862pt}%
\definecolor{currentstroke}{rgb}{0.268510,0.009605,0.335427}%
\pgfsetstrokecolor{currentstroke}%
\pgfsetdash{}{0pt}%
\pgfpathmoveto{\pgfqpoint{8.427784in}{1.852931in}}%
\pgfpathlineto{\pgfqpoint{8.428071in}{1.852917in}}%
\pgfusepath{stroke}%
\end{pgfscope}%
\begin{pgfscope}%
\pgfpathrectangle{\pgfqpoint{6.720588in}{1.750000in}}{\pgfqpoint{2.279412in}{2.004545in}}%
\pgfusepath{clip}%
\pgfsetbuttcap%
\pgfsetroundjoin%
\pgfsetlinewidth{0.308949pt}%
\definecolor{currentstroke}{rgb}{0.268510,0.009605,0.335427}%
\pgfsetstrokecolor{currentstroke}%
\pgfsetdash{}{0pt}%
\pgfpathmoveto{\pgfqpoint{8.428071in}{1.852917in}}%
\pgfpathlineto{\pgfqpoint{8.428112in}{1.852917in}}%
\pgfusepath{stroke}%
\end{pgfscope}%
\begin{pgfscope}%
\pgfpathrectangle{\pgfqpoint{6.720588in}{1.750000in}}{\pgfqpoint{2.279412in}{2.004545in}}%
\pgfusepath{clip}%
\pgfsetbuttcap%
\pgfsetroundjoin%
\pgfsetlinewidth{0.308962pt}%
\definecolor{currentstroke}{rgb}{0.268510,0.009605,0.335427}%
\pgfsetstrokecolor{currentstroke}%
\pgfsetdash{}{0pt}%
\pgfpathmoveto{\pgfqpoint{8.428112in}{1.852917in}}%
\pgfpathlineto{\pgfqpoint{8.427840in}{1.852931in}}%
\pgfusepath{stroke}%
\end{pgfscope}%
\begin{pgfscope}%
\pgfpathrectangle{\pgfqpoint{6.720588in}{1.750000in}}{\pgfqpoint{2.279412in}{2.004545in}}%
\pgfusepath{clip}%
\pgfsetbuttcap%
\pgfsetroundjoin%
\pgfsetlinewidth{0.308879pt}%
\definecolor{currentstroke}{rgb}{0.268510,0.009605,0.335427}%
\pgfsetstrokecolor{currentstroke}%
\pgfsetdash{}{0pt}%
\pgfpathmoveto{\pgfqpoint{8.427840in}{1.852931in}}%
\pgfpathlineto{\pgfqpoint{8.427457in}{1.852950in}}%
\pgfusepath{stroke}%
\end{pgfscope}%
\begin{pgfscope}%
\pgfpathrectangle{\pgfqpoint{6.720588in}{1.750000in}}{\pgfqpoint{2.279412in}{2.004545in}}%
\pgfusepath{clip}%
\pgfsetbuttcap%
\pgfsetroundjoin%
\pgfsetlinewidth{0.308762pt}%
\definecolor{currentstroke}{rgb}{0.268510,0.009605,0.335427}%
\pgfsetstrokecolor{currentstroke}%
\pgfsetdash{}{0pt}%
\pgfpathmoveto{\pgfqpoint{8.427457in}{1.852950in}}%
\pgfpathlineto{\pgfqpoint{8.427415in}{1.852949in}}%
\pgfusepath{stroke}%
\end{pgfscope}%
\begin{pgfscope}%
\pgfpathrectangle{\pgfqpoint{6.720588in}{1.750000in}}{\pgfqpoint{2.279412in}{2.004545in}}%
\pgfusepath{clip}%
\pgfsetbuttcap%
\pgfsetroundjoin%
\pgfsetlinewidth{0.308749pt}%
\definecolor{currentstroke}{rgb}{0.268510,0.009605,0.335427}%
\pgfsetstrokecolor{currentstroke}%
\pgfsetdash{}{0pt}%
\pgfpathmoveto{\pgfqpoint{8.427415in}{1.852949in}}%
\pgfpathlineto{\pgfqpoint{8.427777in}{1.852930in}}%
\pgfusepath{stroke}%
\end{pgfscope}%
\begin{pgfscope}%
\pgfpathrectangle{\pgfqpoint{6.720588in}{1.750000in}}{\pgfqpoint{2.279412in}{2.004545in}}%
\pgfusepath{clip}%
\pgfsetbuttcap%
\pgfsetroundjoin%
\pgfsetlinewidth{0.308859pt}%
\definecolor{currentstroke}{rgb}{0.268510,0.009605,0.335427}%
\pgfsetstrokecolor{currentstroke}%
\pgfsetdash{}{0pt}%
\pgfpathmoveto{\pgfqpoint{8.427777in}{1.852930in}}%
\pgfpathlineto{\pgfqpoint{8.428136in}{1.852913in}}%
\pgfusepath{stroke}%
\end{pgfscope}%
\begin{pgfscope}%
\pgfpathrectangle{\pgfqpoint{6.720588in}{1.750000in}}{\pgfqpoint{2.279412in}{2.004545in}}%
\pgfusepath{clip}%
\pgfsetbuttcap%
\pgfsetroundjoin%
\pgfsetlinewidth{0.308969pt}%
\definecolor{currentstroke}{rgb}{0.268510,0.009605,0.335427}%
\pgfsetstrokecolor{currentstroke}%
\pgfsetdash{}{0pt}%
\pgfpathmoveto{\pgfqpoint{8.428136in}{1.852913in}}%
\pgfpathlineto{\pgfqpoint{8.428186in}{1.852913in}}%
\pgfusepath{stroke}%
\end{pgfscope}%
\begin{pgfscope}%
\pgfpathrectangle{\pgfqpoint{6.720588in}{1.750000in}}{\pgfqpoint{2.279412in}{2.004545in}}%
\pgfusepath{clip}%
\pgfsetbuttcap%
\pgfsetroundjoin%
\pgfsetlinewidth{0.308984pt}%
\definecolor{currentstroke}{rgb}{0.268510,0.009605,0.335427}%
\pgfsetstrokecolor{currentstroke}%
\pgfsetdash{}{0pt}%
\pgfpathmoveto{\pgfqpoint{8.428186in}{1.852913in}}%
\pgfpathlineto{\pgfqpoint{8.427845in}{1.852931in}}%
\pgfusepath{stroke}%
\end{pgfscope}%
\begin{pgfscope}%
\pgfpathrectangle{\pgfqpoint{6.720588in}{1.750000in}}{\pgfqpoint{2.279412in}{2.004545in}}%
\pgfusepath{clip}%
\pgfsetbuttcap%
\pgfsetroundjoin%
\pgfsetlinewidth{0.308880pt}%
\definecolor{currentstroke}{rgb}{0.268510,0.009605,0.335427}%
\pgfsetstrokecolor{currentstroke}%
\pgfsetdash{}{0pt}%
\pgfpathmoveto{\pgfqpoint{8.427845in}{1.852931in}}%
\pgfpathlineto{\pgfqpoint{8.427346in}{1.852955in}}%
\pgfusepath{stroke}%
\end{pgfscope}%
\begin{pgfscope}%
\pgfpathrectangle{\pgfqpoint{6.720588in}{1.750000in}}{\pgfqpoint{2.279412in}{2.004545in}}%
\pgfusepath{clip}%
\pgfsetbuttcap%
\pgfsetroundjoin%
\pgfsetlinewidth{0.308728pt}%
\definecolor{currentstroke}{rgb}{0.268510,0.009605,0.335427}%
\pgfsetstrokecolor{currentstroke}%
\pgfsetdash{}{0pt}%
\pgfpathmoveto{\pgfqpoint{8.427346in}{1.852955in}}%
\pgfpathlineto{\pgfqpoint{8.427297in}{1.852954in}}%
\pgfusepath{stroke}%
\end{pgfscope}%
\begin{pgfscope}%
\pgfpathrectangle{\pgfqpoint{6.720588in}{1.750000in}}{\pgfqpoint{2.279412in}{2.004545in}}%
\pgfusepath{clip}%
\pgfsetbuttcap%
\pgfsetroundjoin%
\pgfsetlinewidth{0.308713pt}%
\definecolor{currentstroke}{rgb}{0.268510,0.009605,0.335427}%
\pgfsetstrokecolor{currentstroke}%
\pgfsetdash{}{0pt}%
\pgfpathmoveto{\pgfqpoint{8.427297in}{1.852954in}}%
\pgfpathlineto{\pgfqpoint{8.427768in}{1.852929in}}%
\pgfusepath{stroke}%
\end{pgfscope}%
\begin{pgfscope}%
\pgfpathrectangle{\pgfqpoint{6.720588in}{1.750000in}}{\pgfqpoint{2.279412in}{2.004545in}}%
\pgfusepath{clip}%
\pgfsetbuttcap%
\pgfsetroundjoin%
\pgfsetlinewidth{0.308856pt}%
\definecolor{currentstroke}{rgb}{0.268510,0.009605,0.335427}%
\pgfsetstrokecolor{currentstroke}%
\pgfsetdash{}{0pt}%
\pgfpathmoveto{\pgfqpoint{8.427768in}{1.852929in}}%
\pgfpathlineto{\pgfqpoint{8.428210in}{1.852909in}}%
\pgfusepath{stroke}%
\end{pgfscope}%
\begin{pgfscope}%
\pgfpathrectangle{\pgfqpoint{6.720588in}{1.750000in}}{\pgfqpoint{2.279412in}{2.004545in}}%
\pgfusepath{clip}%
\pgfsetbuttcap%
\pgfsetroundjoin%
\pgfsetlinewidth{0.308991pt}%
\definecolor{currentstroke}{rgb}{0.268510,0.009605,0.335427}%
\pgfsetstrokecolor{currentstroke}%
\pgfsetdash{}{0pt}%
\pgfpathmoveto{\pgfqpoint{8.428210in}{1.852909in}}%
\pgfpathlineto{\pgfqpoint{8.428273in}{1.852908in}}%
\pgfusepath{stroke}%
\end{pgfscope}%
\begin{pgfscope}%
\pgfpathrectangle{\pgfqpoint{6.720588in}{1.750000in}}{\pgfqpoint{2.279412in}{2.004545in}}%
\pgfusepath{clip}%
\pgfsetbuttcap%
\pgfsetroundjoin%
\pgfsetlinewidth{0.309010pt}%
\definecolor{currentstroke}{rgb}{0.268510,0.009605,0.335427}%
\pgfsetstrokecolor{currentstroke}%
\pgfsetdash{}{0pt}%
\pgfpathmoveto{\pgfqpoint{8.428273in}{1.852908in}}%
\pgfpathlineto{\pgfqpoint{8.427852in}{1.852931in}}%
\pgfusepath{stroke}%
\end{pgfscope}%
\begin{pgfscope}%
\pgfpathrectangle{\pgfqpoint{6.720588in}{1.750000in}}{\pgfqpoint{2.279412in}{2.004545in}}%
\pgfusepath{clip}%
\pgfsetbuttcap%
\pgfsetroundjoin%
\pgfsetlinewidth{0.308882pt}%
\definecolor{currentstroke}{rgb}{0.268510,0.009605,0.335427}%
\pgfsetstrokecolor{currentstroke}%
\pgfsetdash{}{0pt}%
\pgfpathmoveto{\pgfqpoint{8.427852in}{1.852931in}}%
\pgfpathlineto{\pgfqpoint{8.427204in}{1.852962in}}%
\pgfusepath{stroke}%
\end{pgfscope}%
\begin{pgfscope}%
\pgfpathrectangle{\pgfqpoint{6.720588in}{1.750000in}}{\pgfqpoint{2.279412in}{2.004545in}}%
\pgfusepath{clip}%
\pgfsetbuttcap%
\pgfsetroundjoin%
\pgfsetlinewidth{0.308684pt}%
\definecolor{currentstroke}{rgb}{0.268510,0.009605,0.335427}%
\pgfsetstrokecolor{currentstroke}%
\pgfsetdash{}{0pt}%
\pgfpathmoveto{\pgfqpoint{8.427204in}{1.852962in}}%
\pgfpathlineto{\pgfqpoint{8.427204in}{1.852962in}}%
\pgfusepath{stroke}%
\end{pgfscope}%
\begin{pgfscope}%
\pgfpathrectangle{\pgfqpoint{6.720588in}{1.750000in}}{\pgfqpoint{2.279412in}{2.004545in}}%
\pgfusepath{clip}%
\pgfsetbuttcap%
\pgfsetroundjoin%
\pgfsetlinewidth{0.308684pt}%
\definecolor{currentstroke}{rgb}{0.268510,0.009605,0.335427}%
\pgfsetstrokecolor{currentstroke}%
\pgfsetdash{}{0pt}%
\pgfpathmoveto{\pgfqpoint{8.427204in}{1.852962in}}%
\pgfpathlineto{\pgfqpoint{8.427793in}{1.852930in}}%
\pgfusepath{stroke}%
\end{pgfscope}%
\begin{pgfscope}%
\pgfpathrectangle{\pgfqpoint{6.720588in}{1.750000in}}{\pgfqpoint{2.279412in}{2.004545in}}%
\pgfusepath{clip}%
\pgfsetbuttcap%
\pgfsetroundjoin%
\pgfsetlinewidth{0.308864pt}%
\definecolor{currentstroke}{rgb}{0.268510,0.009605,0.335427}%
\pgfsetstrokecolor{currentstroke}%
\pgfsetdash{}{0pt}%
\pgfpathmoveto{\pgfqpoint{8.427793in}{1.852930in}}%
\pgfpathlineto{\pgfqpoint{8.428290in}{1.852906in}}%
\pgfusepath{stroke}%
\end{pgfscope}%
\begin{pgfscope}%
\pgfpathrectangle{\pgfqpoint{6.720588in}{1.750000in}}{\pgfqpoint{2.279412in}{2.004545in}}%
\pgfusepath{clip}%
\pgfsetbuttcap%
\pgfsetroundjoin%
\pgfsetlinewidth{0.309015pt}%
\definecolor{currentstroke}{rgb}{0.268510,0.009605,0.335427}%
\pgfsetstrokecolor{currentstroke}%
\pgfsetdash{}{0pt}%
\pgfpathmoveto{\pgfqpoint{8.428290in}{1.852906in}}%
\pgfpathlineto{\pgfqpoint{8.428335in}{1.852906in}}%
\pgfusepath{stroke}%
\end{pgfscope}%
\begin{pgfscope}%
\pgfpathrectangle{\pgfqpoint{6.720588in}{1.750000in}}{\pgfqpoint{2.279412in}{2.004545in}}%
\pgfusepath{clip}%
\pgfsetbuttcap%
\pgfsetroundjoin%
\pgfsetlinewidth{0.309029pt}%
\definecolor{currentstroke}{rgb}{0.268510,0.009605,0.335427}%
\pgfsetstrokecolor{currentstroke}%
\pgfsetdash{}{0pt}%
\pgfpathmoveto{\pgfqpoint{8.428335in}{1.852906in}}%
\pgfpathlineto{\pgfqpoint{8.427816in}{1.852934in}}%
\pgfusepath{stroke}%
\end{pgfscope}%
\begin{pgfscope}%
\pgfpathrectangle{\pgfqpoint{6.720588in}{1.750000in}}{\pgfqpoint{2.279412in}{2.004545in}}%
\pgfusepath{clip}%
\pgfsetbuttcap%
\pgfsetroundjoin%
\pgfsetlinewidth{0.308872pt}%
\definecolor{currentstroke}{rgb}{0.268510,0.009605,0.335427}%
\pgfsetstrokecolor{currentstroke}%
\pgfsetdash{}{0pt}%
\pgfpathmoveto{\pgfqpoint{8.427816in}{1.852934in}}%
\pgfpathlineto{\pgfqpoint{8.427037in}{1.852971in}}%
\pgfusepath{stroke}%
\end{pgfscope}%
\begin{pgfscope}%
\pgfpathrectangle{\pgfqpoint{6.720588in}{1.750000in}}{\pgfqpoint{2.279412in}{2.004545in}}%
\pgfusepath{clip}%
\pgfsetbuttcap%
\pgfsetroundjoin%
\pgfsetlinewidth{0.308633pt}%
\definecolor{currentstroke}{rgb}{0.268510,0.009605,0.335427}%
\pgfsetstrokecolor{currentstroke}%
\pgfsetdash{}{0pt}%
\pgfpathmoveto{\pgfqpoint{8.427037in}{1.852971in}}%
\pgfpathlineto{\pgfqpoint{8.427037in}{1.852971in}}%
\pgfusepath{stroke}%
\end{pgfscope}%
\begin{pgfscope}%
\pgfpathrectangle{\pgfqpoint{6.720588in}{1.750000in}}{\pgfqpoint{2.279412in}{2.004545in}}%
\pgfusepath{clip}%
\pgfsetbuttcap%
\pgfsetroundjoin%
\pgfsetlinewidth{0.308633pt}%
\definecolor{currentstroke}{rgb}{0.268510,0.009605,0.335427}%
\pgfsetstrokecolor{currentstroke}%
\pgfsetdash{}{0pt}%
\pgfpathmoveto{\pgfqpoint{8.427037in}{1.852971in}}%
\pgfpathlineto{\pgfqpoint{8.427784in}{1.852931in}}%
\pgfusepath{stroke}%
\end{pgfscope}%
\begin{pgfscope}%
\pgfpathrectangle{\pgfqpoint{6.720588in}{1.750000in}}{\pgfqpoint{2.279412in}{2.004545in}}%
\pgfusepath{clip}%
\pgfsetbuttcap%
\pgfsetroundjoin%
\pgfsetlinewidth{0.308861pt}%
\definecolor{currentstroke}{rgb}{0.268510,0.009605,0.335427}%
\pgfsetstrokecolor{currentstroke}%
\pgfsetdash{}{0pt}%
\pgfpathmoveto{\pgfqpoint{8.427784in}{1.852931in}}%
\pgfpathlineto{\pgfqpoint{8.428375in}{1.852902in}}%
\pgfusepath{stroke}%
\end{pgfscope}%
\begin{pgfscope}%
\pgfpathrectangle{\pgfqpoint{6.720588in}{1.750000in}}{\pgfqpoint{2.279412in}{2.004545in}}%
\pgfusepath{clip}%
\pgfsetbuttcap%
\pgfsetroundjoin%
\pgfsetlinewidth{0.309041pt}%
\definecolor{currentstroke}{rgb}{0.268510,0.009605,0.335427}%
\pgfsetstrokecolor{currentstroke}%
\pgfsetdash{}{0pt}%
\pgfpathmoveto{\pgfqpoint{8.428375in}{1.852902in}}%
\pgfpathlineto{\pgfqpoint{8.428433in}{1.852902in}}%
\pgfusepath{stroke}%
\end{pgfscope}%
\begin{pgfscope}%
\pgfpathrectangle{\pgfqpoint{6.720588in}{1.750000in}}{\pgfqpoint{2.279412in}{2.004545in}}%
\pgfusepath{clip}%
\pgfsetbuttcap%
\pgfsetroundjoin%
\pgfsetlinewidth{0.309059pt}%
\definecolor{currentstroke}{rgb}{0.268510,0.009605,0.335427}%
\pgfsetstrokecolor{currentstroke}%
\pgfsetdash{}{0pt}%
\pgfpathmoveto{\pgfqpoint{8.428433in}{1.852902in}}%
\pgfpathlineto{\pgfqpoint{8.427820in}{1.852934in}}%
\pgfusepath{stroke}%
\end{pgfscope}%
\begin{pgfscope}%
\pgfpathrectangle{\pgfqpoint{6.720588in}{1.750000in}}{\pgfqpoint{2.279412in}{2.004545in}}%
\pgfusepath{clip}%
\pgfsetbuttcap%
\pgfsetroundjoin%
\pgfsetlinewidth{0.308873pt}%
\definecolor{currentstroke}{rgb}{0.268510,0.009605,0.335427}%
\pgfsetstrokecolor{currentstroke}%
\pgfsetdash{}{0pt}%
\pgfpathmoveto{\pgfqpoint{8.427820in}{1.852934in}}%
\pgfpathlineto{\pgfqpoint{8.427820in}{1.852934in}}%
\pgfusepath{stroke}%
\end{pgfscope}%
\begin{pgfscope}%
\pgfpathrectangle{\pgfqpoint{6.720588in}{1.750000in}}{\pgfqpoint{2.279412in}{2.004545in}}%
\pgfusepath{clip}%
\pgfsetbuttcap%
\pgfsetroundjoin%
\pgfsetlinewidth{0.308873pt}%
\definecolor{currentstroke}{rgb}{0.268510,0.009605,0.335427}%
\pgfsetstrokecolor{currentstroke}%
\pgfsetdash{}{0pt}%
\pgfpathmoveto{\pgfqpoint{8.427820in}{1.852934in}}%
\pgfpathlineto{\pgfqpoint{8.427701in}{1.852939in}}%
\pgfusepath{stroke}%
\end{pgfscope}%
\begin{pgfscope}%
\pgfpathrectangle{\pgfqpoint{6.720588in}{1.750000in}}{\pgfqpoint{2.279412in}{2.004545in}}%
\pgfusepath{clip}%
\pgfsetbuttcap%
\pgfsetroundjoin%
\pgfsetlinewidth{0.308837pt}%
\definecolor{currentstroke}{rgb}{0.268510,0.009605,0.335427}%
\pgfsetstrokecolor{currentstroke}%
\pgfsetdash{}{0pt}%
\pgfpathmoveto{\pgfqpoint{8.427701in}{1.852939in}}%
\pgfpathlineto{\pgfqpoint{8.427688in}{1.852939in}}%
\pgfusepath{stroke}%
\end{pgfscope}%
\begin{pgfscope}%
\pgfpathrectangle{\pgfqpoint{6.720588in}{1.750000in}}{\pgfqpoint{2.279412in}{2.004545in}}%
\pgfusepath{clip}%
\pgfsetbuttcap%
\pgfsetroundjoin%
\pgfsetlinewidth{0.308833pt}%
\definecolor{currentstroke}{rgb}{0.268510,0.009605,0.335427}%
\pgfsetstrokecolor{currentstroke}%
\pgfsetdash{}{0pt}%
\pgfpathmoveto{\pgfqpoint{8.427688in}{1.852939in}}%
\pgfpathlineto{\pgfqpoint{8.427803in}{1.852932in}}%
\pgfusepath{stroke}%
\end{pgfscope}%
\begin{pgfscope}%
\pgfpathrectangle{\pgfqpoint{6.720588in}{1.750000in}}{\pgfqpoint{2.279412in}{2.004545in}}%
\pgfusepath{clip}%
\pgfsetbuttcap%
\pgfsetroundjoin%
\pgfsetlinewidth{0.308867pt}%
\definecolor{currentstroke}{rgb}{0.268510,0.009605,0.335427}%
\pgfsetstrokecolor{currentstroke}%
\pgfsetdash{}{0pt}%
\pgfpathmoveto{\pgfqpoint{8.427803in}{1.852932in}}%
\pgfpathlineto{\pgfqpoint{8.427930in}{1.852926in}}%
\pgfusepath{stroke}%
\end{pgfscope}%
\begin{pgfscope}%
\pgfpathrectangle{\pgfqpoint{6.720588in}{1.750000in}}{\pgfqpoint{2.279412in}{2.004545in}}%
\pgfusepath{clip}%
\pgfsetbuttcap%
\pgfsetroundjoin%
\pgfsetlinewidth{0.308906pt}%
\definecolor{currentstroke}{rgb}{0.268510,0.009605,0.335427}%
\pgfsetstrokecolor{currentstroke}%
\pgfsetdash{}{0pt}%
\pgfpathmoveto{\pgfqpoint{8.427930in}{1.852926in}}%
\pgfpathlineto{\pgfqpoint{8.427945in}{1.852926in}}%
\pgfusepath{stroke}%
\end{pgfscope}%
\begin{pgfscope}%
\pgfpathrectangle{\pgfqpoint{6.720588in}{1.750000in}}{\pgfqpoint{2.279412in}{2.004545in}}%
\pgfusepath{clip}%
\pgfsetbuttcap%
\pgfsetroundjoin%
\pgfsetlinewidth{0.308911pt}%
\definecolor{currentstroke}{rgb}{0.268510,0.009605,0.335427}%
\pgfsetstrokecolor{currentstroke}%
\pgfsetdash{}{0pt}%
\pgfpathmoveto{\pgfqpoint{8.427945in}{1.852926in}}%
\pgfpathlineto{\pgfqpoint{8.427822in}{1.852932in}}%
\pgfusepath{stroke}%
\end{pgfscope}%
\begin{pgfscope}%
\pgfpathrectangle{\pgfqpoint{6.720588in}{1.750000in}}{\pgfqpoint{2.279412in}{2.004545in}}%
\pgfusepath{clip}%
\pgfsetbuttcap%
\pgfsetroundjoin%
\pgfsetlinewidth{0.308873pt}%
\definecolor{currentstroke}{rgb}{0.268510,0.009605,0.335427}%
\pgfsetstrokecolor{currentstroke}%
\pgfsetdash{}{0pt}%
\pgfpathmoveto{\pgfqpoint{8.427822in}{1.852932in}}%
\pgfpathlineto{\pgfqpoint{8.427667in}{1.852940in}}%
\pgfusepath{stroke}%
\end{pgfscope}%
\begin{pgfscope}%
\pgfpathrectangle{\pgfqpoint{6.720588in}{1.750000in}}{\pgfqpoint{2.279412in}{2.004545in}}%
\pgfusepath{clip}%
\pgfsetbuttcap%
\pgfsetroundjoin%
\pgfsetlinewidth{0.308826pt}%
\definecolor{currentstroke}{rgb}{0.268510,0.009605,0.335427}%
\pgfsetstrokecolor{currentstroke}%
\pgfsetdash{}{0pt}%
\pgfpathmoveto{\pgfqpoint{8.427667in}{1.852940in}}%
\pgfpathlineto{\pgfqpoint{8.427652in}{1.852939in}}%
\pgfusepath{stroke}%
\end{pgfscope}%
\begin{pgfscope}%
\pgfpathrectangle{\pgfqpoint{6.720588in}{1.750000in}}{\pgfqpoint{2.279412in}{2.004545in}}%
\pgfusepath{clip}%
\pgfsetbuttcap%
\pgfsetroundjoin%
\pgfsetlinewidth{0.308821pt}%
\definecolor{currentstroke}{rgb}{0.268510,0.009605,0.335427}%
\pgfsetstrokecolor{currentstroke}%
\pgfsetdash{}{0pt}%
\pgfpathmoveto{\pgfqpoint{8.427652in}{1.852939in}}%
\pgfpathlineto{\pgfqpoint{8.427802in}{1.852931in}}%
\pgfusepath{stroke}%
\end{pgfscope}%
\begin{pgfscope}%
\pgfpathrectangle{\pgfqpoint{6.720588in}{1.750000in}}{\pgfqpoint{2.279412in}{2.004545in}}%
\pgfusepath{clip}%
\pgfsetbuttcap%
\pgfsetroundjoin%
\pgfsetlinewidth{0.308867pt}%
\definecolor{currentstroke}{rgb}{0.268510,0.009605,0.335427}%
\pgfsetstrokecolor{currentstroke}%
\pgfsetdash{}{0pt}%
\pgfpathmoveto{\pgfqpoint{8.427802in}{1.852931in}}%
\pgfpathlineto{\pgfqpoint{8.427963in}{1.852924in}}%
\pgfusepath{stroke}%
\end{pgfscope}%
\begin{pgfscope}%
\pgfpathrectangle{\pgfqpoint{6.720588in}{1.750000in}}{\pgfqpoint{2.279412in}{2.004545in}}%
\pgfusepath{clip}%
\pgfsetbuttcap%
\pgfsetroundjoin%
\pgfsetlinewidth{0.308916pt}%
\definecolor{currentstroke}{rgb}{0.268510,0.009605,0.335427}%
\pgfsetstrokecolor{currentstroke}%
\pgfsetdash{}{0pt}%
\pgfpathmoveto{\pgfqpoint{8.427963in}{1.852924in}}%
\pgfpathlineto{\pgfqpoint{8.427982in}{1.852923in}}%
\pgfusepath{stroke}%
\end{pgfscope}%
\begin{pgfscope}%
\pgfpathrectangle{\pgfqpoint{6.720588in}{1.750000in}}{\pgfqpoint{2.279412in}{2.004545in}}%
\pgfusepath{clip}%
\pgfsetbuttcap%
\pgfsetroundjoin%
\pgfsetlinewidth{0.308922pt}%
\definecolor{currentstroke}{rgb}{0.268510,0.009605,0.335427}%
\pgfsetstrokecolor{currentstroke}%
\pgfsetdash{}{0pt}%
\pgfpathmoveto{\pgfqpoint{8.427982in}{1.852923in}}%
\pgfpathlineto{\pgfqpoint{8.427823in}{1.852932in}}%
\pgfusepath{stroke}%
\end{pgfscope}%
\begin{pgfscope}%
\pgfpathrectangle{\pgfqpoint{6.720588in}{1.750000in}}{\pgfqpoint{2.279412in}{2.004545in}}%
\pgfusepath{clip}%
\pgfsetbuttcap%
\pgfsetroundjoin%
\pgfsetlinewidth{0.308874pt}%
\definecolor{currentstroke}{rgb}{0.268510,0.009605,0.335427}%
\pgfsetstrokecolor{currentstroke}%
\pgfsetdash{}{0pt}%
\pgfpathmoveto{\pgfqpoint{8.427823in}{1.852932in}}%
\pgfpathlineto{\pgfqpoint{8.427622in}{1.852941in}}%
\pgfusepath{stroke}%
\end{pgfscope}%
\begin{pgfscope}%
\pgfpathrectangle{\pgfqpoint{6.720588in}{1.750000in}}{\pgfqpoint{2.279412in}{2.004545in}}%
\pgfusepath{clip}%
\pgfsetbuttcap%
\pgfsetroundjoin%
\pgfsetlinewidth{0.308812pt}%
\definecolor{currentstroke}{rgb}{0.268510,0.009605,0.335427}%
\pgfsetstrokecolor{currentstroke}%
\pgfsetdash{}{0pt}%
\pgfpathmoveto{\pgfqpoint{8.427622in}{1.852941in}}%
\pgfpathlineto{\pgfqpoint{8.427604in}{1.852941in}}%
\pgfusepath{stroke}%
\end{pgfscope}%
\begin{pgfscope}%
\pgfpathrectangle{\pgfqpoint{6.720588in}{1.750000in}}{\pgfqpoint{2.279412in}{2.004545in}}%
\pgfusepath{clip}%
\pgfsetbuttcap%
\pgfsetroundjoin%
\pgfsetlinewidth{0.308807pt}%
\definecolor{currentstroke}{rgb}{0.268510,0.009605,0.335427}%
\pgfsetstrokecolor{currentstroke}%
\pgfsetdash{}{0pt}%
\pgfpathmoveto{\pgfqpoint{8.427604in}{1.852941in}}%
\pgfpathlineto{\pgfqpoint{8.427800in}{1.852931in}}%
\pgfusepath{stroke}%
\end{pgfscope}%
\begin{pgfscope}%
\pgfpathrectangle{\pgfqpoint{6.720588in}{1.750000in}}{\pgfqpoint{2.279412in}{2.004545in}}%
\pgfusepath{clip}%
\pgfsetbuttcap%
\pgfsetroundjoin%
\pgfsetlinewidth{0.308866pt}%
\definecolor{currentstroke}{rgb}{0.268510,0.009605,0.335427}%
\pgfsetstrokecolor{currentstroke}%
\pgfsetdash{}{0pt}%
\pgfpathmoveto{\pgfqpoint{8.427800in}{1.852931in}}%
\pgfpathlineto{\pgfqpoint{8.428005in}{1.852921in}}%
\pgfusepath{stroke}%
\end{pgfscope}%
\begin{pgfscope}%
\pgfpathrectangle{\pgfqpoint{6.720588in}{1.750000in}}{\pgfqpoint{2.279412in}{2.004545in}}%
\pgfusepath{clip}%
\pgfsetbuttcap%
\pgfsetroundjoin%
\pgfsetlinewidth{0.308929pt}%
\definecolor{currentstroke}{rgb}{0.268510,0.009605,0.335427}%
\pgfsetstrokecolor{currentstroke}%
\pgfsetdash{}{0pt}%
\pgfpathmoveto{\pgfqpoint{8.428005in}{1.852921in}}%
\pgfpathlineto{\pgfqpoint{8.428027in}{1.852921in}}%
\pgfusepath{stroke}%
\end{pgfscope}%
\begin{pgfscope}%
\pgfpathrectangle{\pgfqpoint{6.720588in}{1.750000in}}{\pgfqpoint{2.279412in}{2.004545in}}%
\pgfusepath{clip}%
\pgfsetbuttcap%
\pgfsetroundjoin%
\pgfsetlinewidth{0.308936pt}%
\definecolor{currentstroke}{rgb}{0.268510,0.009605,0.335427}%
\pgfsetstrokecolor{currentstroke}%
\pgfsetdash{}{0pt}%
\pgfpathmoveto{\pgfqpoint{8.428027in}{1.852921in}}%
\pgfpathlineto{\pgfqpoint{8.427824in}{1.852932in}}%
\pgfusepath{stroke}%
\end{pgfscope}%
\begin{pgfscope}%
\pgfpathrectangle{\pgfqpoint{6.720588in}{1.750000in}}{\pgfqpoint{2.279412in}{2.004545in}}%
\pgfusepath{clip}%
\pgfsetbuttcap%
\pgfsetroundjoin%
\pgfsetlinewidth{0.308874pt}%
\definecolor{currentstroke}{rgb}{0.268510,0.009605,0.335427}%
\pgfsetstrokecolor{currentstroke}%
\pgfsetdash{}{0pt}%
\pgfpathmoveto{\pgfqpoint{8.427824in}{1.852932in}}%
\pgfpathlineto{\pgfqpoint{8.427562in}{1.852944in}}%
\pgfusepath{stroke}%
\end{pgfscope}%
\begin{pgfscope}%
\pgfpathrectangle{\pgfqpoint{6.720588in}{1.750000in}}{\pgfqpoint{2.279412in}{2.004545in}}%
\pgfusepath{clip}%
\pgfsetbuttcap%
\pgfsetroundjoin%
\pgfsetlinewidth{0.308794pt}%
\definecolor{currentstroke}{rgb}{0.268510,0.009605,0.335427}%
\pgfsetstrokecolor{currentstroke}%
\pgfsetdash{}{0pt}%
\pgfpathmoveto{\pgfqpoint{8.427562in}{1.852944in}}%
\pgfpathlineto{\pgfqpoint{8.427541in}{1.852944in}}%
\pgfusepath{stroke}%
\end{pgfscope}%
\begin{pgfscope}%
\pgfpathrectangle{\pgfqpoint{6.720588in}{1.750000in}}{\pgfqpoint{2.279412in}{2.004545in}}%
\pgfusepath{clip}%
\pgfsetbuttcap%
\pgfsetroundjoin%
\pgfsetlinewidth{0.308787pt}%
\definecolor{currentstroke}{rgb}{0.268510,0.009605,0.335427}%
\pgfsetstrokecolor{currentstroke}%
\pgfsetdash{}{0pt}%
\pgfpathmoveto{\pgfqpoint{8.427541in}{1.852944in}}%
\pgfpathlineto{\pgfqpoint{8.427797in}{1.852930in}}%
\pgfusepath{stroke}%
\end{pgfscope}%
\begin{pgfscope}%
\pgfpathrectangle{\pgfqpoint{6.720588in}{1.750000in}}{\pgfqpoint{2.279412in}{2.004545in}}%
\pgfusepath{clip}%
\pgfsetbuttcap%
\pgfsetroundjoin%
\pgfsetlinewidth{0.308866pt}%
\definecolor{currentstroke}{rgb}{0.268510,0.009605,0.335427}%
\pgfsetstrokecolor{currentstroke}%
\pgfsetdash{}{0pt}%
\pgfpathmoveto{\pgfqpoint{8.427797in}{1.852930in}}%
\pgfpathlineto{\pgfqpoint{8.428056in}{1.852918in}}%
\pgfusepath{stroke}%
\end{pgfscope}%
\begin{pgfscope}%
\pgfpathrectangle{\pgfqpoint{6.720588in}{1.750000in}}{\pgfqpoint{2.279412in}{2.004545in}}%
\pgfusepath{clip}%
\pgfsetbuttcap%
\pgfsetroundjoin%
\pgfsetlinewidth{0.308944pt}%
\definecolor{currentstroke}{rgb}{0.268510,0.009605,0.335427}%
\pgfsetstrokecolor{currentstroke}%
\pgfsetdash{}{0pt}%
\pgfpathmoveto{\pgfqpoint{8.428056in}{1.852918in}}%
\pgfpathlineto{\pgfqpoint{8.428083in}{1.852918in}}%
\pgfusepath{stroke}%
\end{pgfscope}%
\begin{pgfscope}%
\pgfpathrectangle{\pgfqpoint{6.720588in}{1.750000in}}{\pgfqpoint{2.279412in}{2.004545in}}%
\pgfusepath{clip}%
\pgfsetbuttcap%
\pgfsetroundjoin%
\pgfsetlinewidth{1.518339pt}%
\definecolor{currentstroke}{rgb}{0.252899,0.742211,0.448284}%
\pgfsetstrokecolor{currentstroke}%
\pgfsetdash{}{0pt}%
\pgfpathmoveto{\pgfqpoint{6.988332in}{2.752273in}}%
\pgfpathlineto{\pgfqpoint{7.038436in}{2.750428in}}%
\pgfusepath{stroke}%
\end{pgfscope}%
\begin{pgfscope}%
\pgfpathrectangle{\pgfqpoint{6.720588in}{1.750000in}}{\pgfqpoint{2.279412in}{2.004545in}}%
\pgfusepath{clip}%
\pgfsetbuttcap%
\pgfsetroundjoin%
\pgfsetlinewidth{1.479753pt}%
\definecolor{currentstroke}{rgb}{0.214000,0.722114,0.469588}%
\pgfsetstrokecolor{currentstroke}%
\pgfsetdash{}{0pt}%
\pgfpathmoveto{\pgfqpoint{7.038436in}{2.750428in}}%
\pgfpathlineto{\pgfqpoint{7.088558in}{2.749019in}}%
\pgfusepath{stroke}%
\end{pgfscope}%
\begin{pgfscope}%
\pgfpathrectangle{\pgfqpoint{6.720588in}{1.750000in}}{\pgfqpoint{2.279412in}{2.004545in}}%
\pgfusepath{clip}%
\pgfsetbuttcap%
\pgfsetroundjoin%
\pgfsetlinewidth{1.347037pt}%
\definecolor{currentstroke}{rgb}{0.132268,0.655014,0.519661}%
\pgfsetstrokecolor{currentstroke}%
\pgfsetdash{}{0pt}%
\pgfpathmoveto{\pgfqpoint{7.088558in}{2.749019in}}%
\pgfpathlineto{\pgfqpoint{7.138683in}{2.747964in}}%
\pgfusepath{stroke}%
\end{pgfscope}%
\begin{pgfscope}%
\pgfpathrectangle{\pgfqpoint{6.720588in}{1.750000in}}{\pgfqpoint{2.279412in}{2.004545in}}%
\pgfusepath{clip}%
\pgfsetbuttcap%
\pgfsetroundjoin%
\pgfsetlinewidth{1.259718pt}%
\definecolor{currentstroke}{rgb}{0.119512,0.607464,0.540218}%
\pgfsetstrokecolor{currentstroke}%
\pgfsetdash{}{0pt}%
\pgfpathmoveto{\pgfqpoint{7.138683in}{2.747964in}}%
\pgfpathlineto{\pgfqpoint{7.188732in}{2.746085in}}%
\pgfusepath{stroke}%
\end{pgfscope}%
\begin{pgfscope}%
\pgfpathrectangle{\pgfqpoint{6.720588in}{1.750000in}}{\pgfqpoint{2.279412in}{2.004545in}}%
\pgfusepath{clip}%
\pgfsetbuttcap%
\pgfsetroundjoin%
\pgfsetlinewidth{1.173886pt}%
\definecolor{currentstroke}{rgb}{0.128729,0.563265,0.551229}%
\pgfsetstrokecolor{currentstroke}%
\pgfsetdash{}{0pt}%
\pgfpathmoveto{\pgfqpoint{7.188732in}{2.746085in}}%
\pgfpathlineto{\pgfqpoint{7.238779in}{2.743939in}}%
\pgfusepath{stroke}%
\end{pgfscope}%
\begin{pgfscope}%
\pgfpathrectangle{\pgfqpoint{6.720588in}{1.750000in}}{\pgfqpoint{2.279412in}{2.004545in}}%
\pgfusepath{clip}%
\pgfsetbuttcap%
\pgfsetroundjoin%
\pgfsetlinewidth{0.892541pt}%
\definecolor{currentstroke}{rgb}{0.188923,0.410910,0.556326}%
\pgfsetstrokecolor{currentstroke}%
\pgfsetdash{}{0pt}%
\pgfpathmoveto{\pgfqpoint{7.238779in}{2.743939in}}%
\pgfpathlineto{\pgfqpoint{7.288886in}{2.742650in}}%
\pgfusepath{stroke}%
\end{pgfscope}%
\begin{pgfscope}%
\pgfpathrectangle{\pgfqpoint{6.720588in}{1.750000in}}{\pgfqpoint{2.279412in}{2.004545in}}%
\pgfusepath{clip}%
\pgfsetbuttcap%
\pgfsetroundjoin%
\pgfsetlinewidth{0.770040pt}%
\definecolor{currentstroke}{rgb}{0.221989,0.339161,0.548752}%
\pgfsetstrokecolor{currentstroke}%
\pgfsetdash{}{0pt}%
\pgfpathmoveto{\pgfqpoint{7.288886in}{2.742650in}}%
\pgfpathlineto{\pgfqpoint{7.338863in}{2.740677in}}%
\pgfusepath{stroke}%
\end{pgfscope}%
\begin{pgfscope}%
\pgfpathrectangle{\pgfqpoint{6.720588in}{1.750000in}}{\pgfqpoint{2.279412in}{2.004545in}}%
\pgfusepath{clip}%
\pgfsetbuttcap%
\pgfsetroundjoin%
\pgfsetlinewidth{0.639142pt}%
\definecolor{currentstroke}{rgb}{0.257322,0.256130,0.526563}%
\pgfsetstrokecolor{currentstroke}%
\pgfsetdash{}{0pt}%
\pgfpathmoveto{\pgfqpoint{7.338863in}{2.740677in}}%
\pgfpathlineto{\pgfqpoint{7.388685in}{2.738515in}}%
\pgfusepath{stroke}%
\end{pgfscope}%
\begin{pgfscope}%
\pgfpathrectangle{\pgfqpoint{6.720588in}{1.750000in}}{\pgfqpoint{2.279412in}{2.004545in}}%
\pgfusepath{clip}%
\pgfsetbuttcap%
\pgfsetroundjoin%
\pgfsetlinewidth{0.555421pt}%
\definecolor{currentstroke}{rgb}{0.274128,0.199721,0.498911}%
\pgfsetstrokecolor{currentstroke}%
\pgfsetdash{}{0pt}%
\pgfpathmoveto{\pgfqpoint{7.388685in}{2.738515in}}%
\pgfpathlineto{\pgfqpoint{7.388685in}{2.738515in}}%
\pgfusepath{stroke}%
\end{pgfscope}%
\begin{pgfscope}%
\pgfpathrectangle{\pgfqpoint{6.720588in}{1.750000in}}{\pgfqpoint{2.279412in}{2.004545in}}%
\pgfusepath{clip}%
\pgfsetbuttcap%
\pgfsetroundjoin%
\pgfsetlinewidth{0.555421pt}%
\definecolor{currentstroke}{rgb}{0.274128,0.199721,0.498911}%
\pgfsetstrokecolor{currentstroke}%
\pgfsetdash{}{0pt}%
\pgfpathmoveto{\pgfqpoint{7.388685in}{2.738515in}}%
\pgfpathlineto{\pgfqpoint{7.413378in}{2.735121in}}%
\pgfusepath{stroke}%
\end{pgfscope}%
\begin{pgfscope}%
\pgfpathrectangle{\pgfqpoint{6.720588in}{1.750000in}}{\pgfqpoint{2.279412in}{2.004545in}}%
\pgfusepath{clip}%
\pgfsetbuttcap%
\pgfsetroundjoin%
\pgfsetlinewidth{0.384979pt}%
\definecolor{currentstroke}{rgb}{0.280267,0.073417,0.397163}%
\pgfsetstrokecolor{currentstroke}%
\pgfsetdash{}{0pt}%
\pgfpathmoveto{\pgfqpoint{7.413378in}{2.735121in}}%
\pgfpathlineto{\pgfqpoint{7.413378in}{2.735121in}}%
\pgfusepath{stroke}%
\end{pgfscope}%
\begin{pgfscope}%
\pgfpathrectangle{\pgfqpoint{6.720588in}{1.750000in}}{\pgfqpoint{2.279412in}{2.004545in}}%
\pgfusepath{clip}%
\pgfsetbuttcap%
\pgfsetroundjoin%
\pgfsetlinewidth{0.384979pt}%
\definecolor{currentstroke}{rgb}{0.280267,0.073417,0.397163}%
\pgfsetstrokecolor{currentstroke}%
\pgfsetdash{}{0pt}%
\pgfpathmoveto{\pgfqpoint{7.413378in}{2.735121in}}%
\pgfpathlineto{\pgfqpoint{7.428143in}{2.732588in}}%
\pgfusepath{stroke}%
\end{pgfscope}%
\begin{pgfscope}%
\pgfpathrectangle{\pgfqpoint{6.720588in}{1.750000in}}{\pgfqpoint{2.279412in}{2.004545in}}%
\pgfusepath{clip}%
\pgfsetbuttcap%
\pgfsetroundjoin%
\pgfsetlinewidth{0.357020pt}%
\definecolor{currentstroke}{rgb}{0.277018,0.050344,0.375715}%
\pgfsetstrokecolor{currentstroke}%
\pgfsetdash{}{0pt}%
\pgfpathmoveto{\pgfqpoint{7.428143in}{2.732588in}}%
\pgfpathlineto{\pgfqpoint{7.428143in}{2.732588in}}%
\pgfusepath{stroke}%
\end{pgfscope}%
\begin{pgfscope}%
\pgfpathrectangle{\pgfqpoint{6.720588in}{1.750000in}}{\pgfqpoint{2.279412in}{2.004545in}}%
\pgfusepath{clip}%
\pgfsetbuttcap%
\pgfsetroundjoin%
\pgfsetlinewidth{0.357020pt}%
\definecolor{currentstroke}{rgb}{0.277018,0.050344,0.375715}%
\pgfsetstrokecolor{currentstroke}%
\pgfsetdash{}{0pt}%
\pgfpathmoveto{\pgfqpoint{7.428143in}{2.732588in}}%
\pgfpathlineto{\pgfqpoint{7.428143in}{2.732588in}}%
\pgfusepath{stroke}%
\end{pgfscope}%
\begin{pgfscope}%
\pgfpathrectangle{\pgfqpoint{6.720588in}{1.750000in}}{\pgfqpoint{2.279412in}{2.004545in}}%
\pgfusepath{clip}%
\pgfsetbuttcap%
\pgfsetroundjoin%
\pgfsetlinewidth{0.357020pt}%
\definecolor{currentstroke}{rgb}{0.277018,0.050344,0.375715}%
\pgfsetstrokecolor{currentstroke}%
\pgfsetdash{}{0pt}%
\pgfpathmoveto{\pgfqpoint{7.428143in}{2.732588in}}%
\pgfpathlineto{\pgfqpoint{7.433929in}{2.731115in}}%
\pgfusepath{stroke}%
\end{pgfscope}%
\begin{pgfscope}%
\pgfpathrectangle{\pgfqpoint{6.720588in}{1.750000in}}{\pgfqpoint{2.279412in}{2.004545in}}%
\pgfusepath{clip}%
\pgfsetbuttcap%
\pgfsetroundjoin%
\pgfsetlinewidth{0.371726pt}%
\definecolor{currentstroke}{rgb}{0.278791,0.062145,0.386592}%
\pgfsetstrokecolor{currentstroke}%
\pgfsetdash{}{0pt}%
\pgfpathmoveto{\pgfqpoint{7.433929in}{2.731115in}}%
\pgfpathlineto{\pgfqpoint{7.436915in}{2.730025in}}%
\pgfusepath{stroke}%
\end{pgfscope}%
\begin{pgfscope}%
\pgfpathrectangle{\pgfqpoint{6.720588in}{1.750000in}}{\pgfqpoint{2.279412in}{2.004545in}}%
\pgfusepath{clip}%
\pgfsetbuttcap%
\pgfsetroundjoin%
\pgfsetlinewidth{0.391686pt}%
\definecolor{currentstroke}{rgb}{0.280894,0.078907,0.402329}%
\pgfsetstrokecolor{currentstroke}%
\pgfsetdash{}{0pt}%
\pgfpathmoveto{\pgfqpoint{7.436915in}{2.730025in}}%
\pgfpathlineto{\pgfqpoint{7.438476in}{2.729119in}}%
\pgfusepath{stroke}%
\end{pgfscope}%
\begin{pgfscope}%
\pgfpathrectangle{\pgfqpoint{6.720588in}{1.750000in}}{\pgfqpoint{2.279412in}{2.004545in}}%
\pgfusepath{clip}%
\pgfsetbuttcap%
\pgfsetroundjoin%
\pgfsetlinewidth{0.401071pt}%
\definecolor{currentstroke}{rgb}{0.281446,0.084320,0.407414}%
\pgfsetstrokecolor{currentstroke}%
\pgfsetdash{}{0pt}%
\pgfpathmoveto{\pgfqpoint{7.438476in}{2.729119in}}%
\pgfpathlineto{\pgfqpoint{7.439041in}{2.728181in}}%
\pgfusepath{stroke}%
\end{pgfscope}%
\begin{pgfscope}%
\pgfpathrectangle{\pgfqpoint{6.720588in}{1.750000in}}{\pgfqpoint{2.279412in}{2.004545in}}%
\pgfusepath{clip}%
\pgfsetbuttcap%
\pgfsetroundjoin%
\pgfsetlinewidth{0.404063pt}%
\definecolor{currentstroke}{rgb}{0.281924,0.089666,0.412415}%
\pgfsetstrokecolor{currentstroke}%
\pgfsetdash{}{0pt}%
\pgfpathmoveto{\pgfqpoint{7.439041in}{2.728181in}}%
\pgfpathlineto{\pgfqpoint{7.438567in}{2.727432in}}%
\pgfusepath{stroke}%
\end{pgfscope}%
\begin{pgfscope}%
\pgfpathrectangle{\pgfqpoint{6.720588in}{1.750000in}}{\pgfqpoint{2.279412in}{2.004545in}}%
\pgfusepath{clip}%
\pgfsetbuttcap%
\pgfsetroundjoin%
\pgfsetlinewidth{0.401565pt}%
\definecolor{currentstroke}{rgb}{0.281446,0.084320,0.407414}%
\pgfsetstrokecolor{currentstroke}%
\pgfsetdash{}{0pt}%
\pgfpathmoveto{\pgfqpoint{7.438567in}{2.727432in}}%
\pgfpathlineto{\pgfqpoint{7.438567in}{2.727432in}}%
\pgfusepath{stroke}%
\end{pgfscope}%
\begin{pgfscope}%
\pgfpathrectangle{\pgfqpoint{6.720588in}{1.750000in}}{\pgfqpoint{2.279412in}{2.004545in}}%
\pgfusepath{clip}%
\pgfsetbuttcap%
\pgfsetroundjoin%
\pgfsetlinewidth{0.401565pt}%
\definecolor{currentstroke}{rgb}{0.281446,0.084320,0.407414}%
\pgfsetstrokecolor{currentstroke}%
\pgfsetdash{}{0pt}%
\pgfpathmoveto{\pgfqpoint{7.438567in}{2.727432in}}%
\pgfpathlineto{\pgfqpoint{7.437976in}{2.727028in}}%
\pgfusepath{stroke}%
\end{pgfscope}%
\begin{pgfscope}%
\pgfpathrectangle{\pgfqpoint{6.720588in}{1.750000in}}{\pgfqpoint{2.279412in}{2.004545in}}%
\pgfusepath{clip}%
\pgfsetbuttcap%
\pgfsetroundjoin%
\pgfsetlinewidth{0.398731pt}%
\definecolor{currentstroke}{rgb}{0.281446,0.084320,0.407414}%
\pgfsetstrokecolor{currentstroke}%
\pgfsetdash{}{0pt}%
\pgfpathmoveto{\pgfqpoint{7.437976in}{2.727028in}}%
\pgfpathlineto{\pgfqpoint{7.437976in}{2.727028in}}%
\pgfusepath{stroke}%
\end{pgfscope}%
\begin{pgfscope}%
\pgfpathrectangle{\pgfqpoint{6.720588in}{1.750000in}}{\pgfqpoint{2.279412in}{2.004545in}}%
\pgfusepath{clip}%
\pgfsetbuttcap%
\pgfsetroundjoin%
\pgfsetlinewidth{0.398731pt}%
\definecolor{currentstroke}{rgb}{0.281446,0.084320,0.407414}%
\pgfsetstrokecolor{currentstroke}%
\pgfsetdash{}{0pt}%
\pgfpathmoveto{\pgfqpoint{7.437976in}{2.727028in}}%
\pgfpathlineto{\pgfqpoint{7.438100in}{2.726845in}}%
\pgfusepath{stroke}%
\end{pgfscope}%
\begin{pgfscope}%
\pgfpathrectangle{\pgfqpoint{6.720588in}{1.750000in}}{\pgfqpoint{2.279412in}{2.004545in}}%
\pgfusepath{clip}%
\pgfsetbuttcap%
\pgfsetroundjoin%
\pgfsetlinewidth{0.399359pt}%
\definecolor{currentstroke}{rgb}{0.281446,0.084320,0.407414}%
\pgfsetstrokecolor{currentstroke}%
\pgfsetdash{}{0pt}%
\pgfpathmoveto{\pgfqpoint{7.438100in}{2.726845in}}%
\pgfpathlineto{\pgfqpoint{7.438246in}{2.726740in}}%
\pgfusepath{stroke}%
\end{pgfscope}%
\begin{pgfscope}%
\pgfpathrectangle{\pgfqpoint{6.720588in}{1.750000in}}{\pgfqpoint{2.279412in}{2.004545in}}%
\pgfusepath{clip}%
\pgfsetbuttcap%
\pgfsetroundjoin%
\pgfsetlinewidth{0.400059pt}%
\definecolor{currentstroke}{rgb}{0.281446,0.084320,0.407414}%
\pgfsetstrokecolor{currentstroke}%
\pgfsetdash{}{0pt}%
\pgfpathmoveto{\pgfqpoint{7.438246in}{2.726740in}}%
\pgfpathlineto{\pgfqpoint{7.438301in}{2.726677in}}%
\pgfusepath{stroke}%
\end{pgfscope}%
\begin{pgfscope}%
\pgfpathrectangle{\pgfqpoint{6.720588in}{1.750000in}}{\pgfqpoint{2.279412in}{2.004545in}}%
\pgfusepath{clip}%
\pgfsetbuttcap%
\pgfsetroundjoin%
\pgfsetlinewidth{0.400318pt}%
\definecolor{currentstroke}{rgb}{0.281446,0.084320,0.407414}%
\pgfsetstrokecolor{currentstroke}%
\pgfsetdash{}{0pt}%
\pgfpathmoveto{\pgfqpoint{7.438301in}{2.726677in}}%
\pgfpathlineto{\pgfqpoint{7.438224in}{2.726634in}}%
\pgfusepath{stroke}%
\end{pgfscope}%
\begin{pgfscope}%
\pgfpathrectangle{\pgfqpoint{6.720588in}{1.750000in}}{\pgfqpoint{2.279412in}{2.004545in}}%
\pgfusepath{clip}%
\pgfsetbuttcap%
\pgfsetroundjoin%
\pgfsetlinewidth{0.399967pt}%
\definecolor{currentstroke}{rgb}{0.281446,0.084320,0.407414}%
\pgfsetstrokecolor{currentstroke}%
\pgfsetdash{}{0pt}%
\pgfpathmoveto{\pgfqpoint{7.438224in}{2.726634in}}%
\pgfpathlineto{\pgfqpoint{7.438090in}{2.726605in}}%
\pgfusepath{stroke}%
\end{pgfscope}%
\begin{pgfscope}%
\pgfpathrectangle{\pgfqpoint{6.720588in}{1.750000in}}{\pgfqpoint{2.279412in}{2.004545in}}%
\pgfusepath{clip}%
\pgfsetbuttcap%
\pgfsetroundjoin%
\pgfsetlinewidth{0.399354pt}%
\definecolor{currentstroke}{rgb}{0.281446,0.084320,0.407414}%
\pgfsetstrokecolor{currentstroke}%
\pgfsetdash{}{0pt}%
\pgfpathmoveto{\pgfqpoint{7.438090in}{2.726605in}}%
\pgfpathlineto{\pgfqpoint{7.438041in}{2.726590in}}%
\pgfusepath{stroke}%
\end{pgfscope}%
\begin{pgfscope}%
\pgfpathrectangle{\pgfqpoint{6.720588in}{1.750000in}}{\pgfqpoint{2.279412in}{2.004545in}}%
\pgfusepath{clip}%
\pgfsetbuttcap%
\pgfsetroundjoin%
\pgfsetlinewidth{0.399133pt}%
\definecolor{currentstroke}{rgb}{0.281446,0.084320,0.407414}%
\pgfsetstrokecolor{currentstroke}%
\pgfsetdash{}{0pt}%
\pgfpathmoveto{\pgfqpoint{7.438041in}{2.726590in}}%
\pgfpathlineto{\pgfqpoint{7.438137in}{2.726586in}}%
\pgfusepath{stroke}%
\end{pgfscope}%
\begin{pgfscope}%
\pgfpathrectangle{\pgfqpoint{6.720588in}{1.750000in}}{\pgfqpoint{2.279412in}{2.004545in}}%
\pgfusepath{clip}%
\pgfsetbuttcap%
\pgfsetroundjoin%
\pgfsetlinewidth{0.399574pt}%
\definecolor{currentstroke}{rgb}{0.281446,0.084320,0.407414}%
\pgfsetstrokecolor{currentstroke}%
\pgfsetdash{}{0pt}%
\pgfpathmoveto{\pgfqpoint{7.438137in}{2.726586in}}%
\pgfpathlineto{\pgfqpoint{7.438280in}{2.726588in}}%
\pgfusepath{stroke}%
\end{pgfscope}%
\begin{pgfscope}%
\pgfpathrectangle{\pgfqpoint{6.720588in}{1.750000in}}{\pgfqpoint{2.279412in}{2.004545in}}%
\pgfusepath{clip}%
\pgfsetbuttcap%
\pgfsetroundjoin%
\pgfsetlinewidth{0.400233pt}%
\definecolor{currentstroke}{rgb}{0.281446,0.084320,0.407414}%
\pgfsetstrokecolor{currentstroke}%
\pgfsetdash{}{0pt}%
\pgfpathmoveto{\pgfqpoint{7.438280in}{2.726588in}}%
\pgfpathlineto{\pgfqpoint{7.438335in}{2.726588in}}%
\pgfusepath{stroke}%
\end{pgfscope}%
\begin{pgfscope}%
\pgfpathrectangle{\pgfqpoint{6.720588in}{1.750000in}}{\pgfqpoint{2.279412in}{2.004545in}}%
\pgfusepath{clip}%
\pgfsetbuttcap%
\pgfsetroundjoin%
\pgfsetlinewidth{0.400482pt}%
\definecolor{currentstroke}{rgb}{0.281446,0.084320,0.407414}%
\pgfsetstrokecolor{currentstroke}%
\pgfsetdash{}{0pt}%
\pgfpathmoveto{\pgfqpoint{7.438335in}{2.726588in}}%
\pgfpathlineto{\pgfqpoint{7.438247in}{2.726583in}}%
\pgfusepath{stroke}%
\end{pgfscope}%
\begin{pgfscope}%
\pgfpathrectangle{\pgfqpoint{6.720588in}{1.750000in}}{\pgfqpoint{2.279412in}{2.004545in}}%
\pgfusepath{clip}%
\pgfsetbuttcap%
\pgfsetroundjoin%
\pgfsetlinewidth{0.400080pt}%
\definecolor{currentstroke}{rgb}{0.281446,0.084320,0.407414}%
\pgfsetstrokecolor{currentstroke}%
\pgfsetdash{}{0pt}%
\pgfpathmoveto{\pgfqpoint{7.438247in}{2.726583in}}%
\pgfpathlineto{\pgfqpoint{7.438091in}{2.726575in}}%
\pgfusepath{stroke}%
\end{pgfscope}%
\begin{pgfscope}%
\pgfpathrectangle{\pgfqpoint{6.720588in}{1.750000in}}{\pgfqpoint{2.279412in}{2.004545in}}%
\pgfusepath{clip}%
\pgfsetbuttcap%
\pgfsetroundjoin%
\pgfsetlinewidth{0.399364pt}%
\definecolor{currentstroke}{rgb}{0.281446,0.084320,0.407414}%
\pgfsetstrokecolor{currentstroke}%
\pgfsetdash{}{0pt}%
\pgfpathmoveto{\pgfqpoint{7.438091in}{2.726575in}}%
\pgfpathlineto{\pgfqpoint{7.438025in}{2.726571in}}%
\pgfusepath{stroke}%
\end{pgfscope}%
\begin{pgfscope}%
\pgfpathrectangle{\pgfqpoint{6.720588in}{1.750000in}}{\pgfqpoint{2.279412in}{2.004545in}}%
\pgfusepath{clip}%
\pgfsetbuttcap%
\pgfsetroundjoin%
\pgfsetlinewidth{0.399060pt}%
\definecolor{currentstroke}{rgb}{0.281446,0.084320,0.407414}%
\pgfsetstrokecolor{currentstroke}%
\pgfsetdash{}{0pt}%
\pgfpathmoveto{\pgfqpoint{7.438025in}{2.726571in}}%
\pgfpathlineto{\pgfqpoint{7.438124in}{2.726575in}}%
\pgfusepath{stroke}%
\end{pgfscope}%
\begin{pgfscope}%
\pgfpathrectangle{\pgfqpoint{6.720588in}{1.750000in}}{\pgfqpoint{2.279412in}{2.004545in}}%
\pgfusepath{clip}%
\pgfsetbuttcap%
\pgfsetroundjoin%
\pgfsetlinewidth{0.399517pt}%
\definecolor{currentstroke}{rgb}{0.281446,0.084320,0.407414}%
\pgfsetstrokecolor{currentstroke}%
\pgfsetdash{}{0pt}%
\pgfpathmoveto{\pgfqpoint{7.438124in}{2.726575in}}%
\pgfpathlineto{\pgfqpoint{7.438285in}{2.726582in}}%
\pgfusepath{stroke}%
\end{pgfscope}%
\begin{pgfscope}%
\pgfpathrectangle{\pgfqpoint{6.720588in}{1.750000in}}{\pgfqpoint{2.279412in}{2.004545in}}%
\pgfusepath{clip}%
\pgfsetbuttcap%
\pgfsetroundjoin%
\pgfsetlinewidth{0.400256pt}%
\definecolor{currentstroke}{rgb}{0.281446,0.084320,0.407414}%
\pgfsetstrokecolor{currentstroke}%
\pgfsetdash{}{0pt}%
\pgfpathmoveto{\pgfqpoint{7.438285in}{2.726582in}}%
\pgfpathlineto{\pgfqpoint{7.438353in}{2.726585in}}%
\pgfusepath{stroke}%
\end{pgfscope}%
\begin{pgfscope}%
\pgfpathrectangle{\pgfqpoint{6.720588in}{1.750000in}}{\pgfqpoint{2.279412in}{2.004545in}}%
\pgfusepath{clip}%
\pgfsetbuttcap%
\pgfsetroundjoin%
\pgfsetlinewidth{0.400567pt}%
\definecolor{currentstroke}{rgb}{0.281446,0.084320,0.407414}%
\pgfsetstrokecolor{currentstroke}%
\pgfsetdash{}{0pt}%
\pgfpathmoveto{\pgfqpoint{7.438353in}{2.726585in}}%
\pgfpathlineto{\pgfqpoint{7.438262in}{2.726581in}}%
\pgfusepath{stroke}%
\end{pgfscope}%
\begin{pgfscope}%
\pgfpathrectangle{\pgfqpoint{6.720588in}{1.750000in}}{\pgfqpoint{2.279412in}{2.004545in}}%
\pgfusepath{clip}%
\pgfsetbuttcap%
\pgfsetroundjoin%
\pgfsetlinewidth{0.400152pt}%
\definecolor{currentstroke}{rgb}{0.281446,0.084320,0.407414}%
\pgfsetstrokecolor{currentstroke}%
\pgfsetdash{}{0pt}%
\pgfpathmoveto{\pgfqpoint{7.438262in}{2.726581in}}%
\pgfpathlineto{\pgfqpoint{7.438087in}{2.726574in}}%
\pgfusepath{stroke}%
\end{pgfscope}%
\begin{pgfscope}%
\pgfpathrectangle{\pgfqpoint{6.720588in}{1.750000in}}{\pgfqpoint{2.279412in}{2.004545in}}%
\pgfusepath{clip}%
\pgfsetbuttcap%
\pgfsetroundjoin%
\pgfsetlinewidth{0.399345pt}%
\definecolor{currentstroke}{rgb}{0.281446,0.084320,0.407414}%
\pgfsetstrokecolor{currentstroke}%
\pgfsetdash{}{0pt}%
\pgfpathmoveto{\pgfqpoint{7.438087in}{2.726574in}}%
\pgfpathlineto{\pgfqpoint{7.438002in}{2.726570in}}%
\pgfusepath{stroke}%
\end{pgfscope}%
\begin{pgfscope}%
\pgfpathrectangle{\pgfqpoint{6.720588in}{1.750000in}}{\pgfqpoint{2.279412in}{2.004545in}}%
\pgfusepath{clip}%
\pgfsetbuttcap%
\pgfsetroundjoin%
\pgfsetlinewidth{0.398959pt}%
\definecolor{currentstroke}{rgb}{0.281446,0.084320,0.407414}%
\pgfsetstrokecolor{currentstroke}%
\pgfsetdash{}{0pt}%
\pgfpathmoveto{\pgfqpoint{7.438002in}{2.726570in}}%
\pgfpathlineto{\pgfqpoint{7.438106in}{2.726574in}}%
\pgfusepath{stroke}%
\end{pgfscope}%
\begin{pgfscope}%
\pgfpathrectangle{\pgfqpoint{6.720588in}{1.750000in}}{\pgfqpoint{2.279412in}{2.004545in}}%
\pgfusepath{clip}%
\pgfsetbuttcap%
\pgfsetroundjoin%
\pgfsetlinewidth{0.399436pt}%
\definecolor{currentstroke}{rgb}{0.281446,0.084320,0.407414}%
\pgfsetstrokecolor{currentstroke}%
\pgfsetdash{}{0pt}%
\pgfpathmoveto{\pgfqpoint{7.438106in}{2.726574in}}%
\pgfpathlineto{\pgfqpoint{7.438288in}{2.726581in}}%
\pgfusepath{stroke}%
\end{pgfscope}%
\begin{pgfscope}%
\pgfpathrectangle{\pgfqpoint{6.720588in}{1.750000in}}{\pgfqpoint{2.279412in}{2.004545in}}%
\pgfusepath{clip}%
\pgfsetbuttcap%
\pgfsetroundjoin%
\pgfsetlinewidth{0.400269pt}%
\definecolor{currentstroke}{rgb}{0.281446,0.084320,0.407414}%
\pgfsetstrokecolor{currentstroke}%
\pgfsetdash{}{0pt}%
\pgfpathmoveto{\pgfqpoint{7.438288in}{2.726581in}}%
\pgfpathlineto{\pgfqpoint{7.438372in}{2.726585in}}%
\pgfusepath{stroke}%
\end{pgfscope}%
\begin{pgfscope}%
\pgfpathrectangle{\pgfqpoint{6.720588in}{1.750000in}}{\pgfqpoint{2.279412in}{2.004545in}}%
\pgfusepath{clip}%
\pgfsetbuttcap%
\pgfsetroundjoin%
\pgfsetlinewidth{0.400655pt}%
\definecolor{currentstroke}{rgb}{0.281446,0.084320,0.407414}%
\pgfsetstrokecolor{currentstroke}%
\pgfsetdash{}{0pt}%
\pgfpathmoveto{\pgfqpoint{7.438372in}{2.726585in}}%
\pgfpathlineto{\pgfqpoint{7.438280in}{2.726582in}}%
\pgfusepath{stroke}%
\end{pgfscope}%
\begin{pgfscope}%
\pgfpathrectangle{\pgfqpoint{6.720588in}{1.750000in}}{\pgfqpoint{2.279412in}{2.004545in}}%
\pgfusepath{clip}%
\pgfsetbuttcap%
\pgfsetroundjoin%
\pgfsetlinewidth{0.400233pt}%
\definecolor{currentstroke}{rgb}{0.281446,0.084320,0.407414}%
\pgfsetstrokecolor{currentstroke}%
\pgfsetdash{}{0pt}%
\pgfpathmoveto{\pgfqpoint{7.438280in}{2.726582in}}%
\pgfpathlineto{\pgfqpoint{7.438083in}{2.726574in}}%
\pgfusepath{stroke}%
\end{pgfscope}%
\begin{pgfscope}%
\pgfpathrectangle{\pgfqpoint{6.720588in}{1.750000in}}{\pgfqpoint{2.279412in}{2.004545in}}%
\pgfusepath{clip}%
\pgfsetbuttcap%
\pgfsetroundjoin%
\pgfsetlinewidth{0.399330pt}%
\definecolor{currentstroke}{rgb}{0.281446,0.084320,0.407414}%
\pgfsetstrokecolor{currentstroke}%
\pgfsetdash{}{0pt}%
\pgfpathmoveto{\pgfqpoint{7.438083in}{2.726574in}}%
\pgfpathlineto{\pgfqpoint{7.437978in}{2.726569in}}%
\pgfusepath{stroke}%
\end{pgfscope}%
\begin{pgfscope}%
\pgfpathrectangle{\pgfqpoint{6.720588in}{1.750000in}}{\pgfqpoint{2.279412in}{2.004545in}}%
\pgfusepath{clip}%
\pgfsetbuttcap%
\pgfsetroundjoin%
\pgfsetlinewidth{0.398847pt}%
\definecolor{currentstroke}{rgb}{0.281446,0.084320,0.407414}%
\pgfsetstrokecolor{currentstroke}%
\pgfsetdash{}{0pt}%
\pgfpathmoveto{\pgfqpoint{7.437978in}{2.726569in}}%
\pgfpathlineto{\pgfqpoint{7.438086in}{2.726573in}}%
\pgfusepath{stroke}%
\end{pgfscope}%
\begin{pgfscope}%
\pgfpathrectangle{\pgfqpoint{6.720588in}{1.750000in}}{\pgfqpoint{2.279412in}{2.004545in}}%
\pgfusepath{clip}%
\pgfsetbuttcap%
\pgfsetroundjoin%
\pgfsetlinewidth{0.399340pt}%
\definecolor{currentstroke}{rgb}{0.281446,0.084320,0.407414}%
\pgfsetstrokecolor{currentstroke}%
\pgfsetdash{}{0pt}%
\pgfpathmoveto{\pgfqpoint{7.438086in}{2.726573in}}%
\pgfpathlineto{\pgfqpoint{7.438290in}{2.726581in}}%
\pgfusepath{stroke}%
\end{pgfscope}%
\begin{pgfscope}%
\pgfpathrectangle{\pgfqpoint{6.720588in}{1.750000in}}{\pgfqpoint{2.279412in}{2.004545in}}%
\pgfusepath{clip}%
\pgfsetbuttcap%
\pgfsetroundjoin%
\pgfsetlinewidth{0.400276pt}%
\definecolor{currentstroke}{rgb}{0.281446,0.084320,0.407414}%
\pgfsetstrokecolor{currentstroke}%
\pgfsetdash{}{0pt}%
\pgfpathmoveto{\pgfqpoint{7.438290in}{2.726581in}}%
\pgfpathlineto{\pgfqpoint{7.438393in}{2.726586in}}%
\pgfusepath{stroke}%
\end{pgfscope}%
\begin{pgfscope}%
\pgfpathrectangle{\pgfqpoint{6.720588in}{1.750000in}}{\pgfqpoint{2.279412in}{2.004545in}}%
\pgfusepath{clip}%
\pgfsetbuttcap%
\pgfsetroundjoin%
\pgfsetlinewidth{0.400750pt}%
\definecolor{currentstroke}{rgb}{0.281446,0.084320,0.407414}%
\pgfsetstrokecolor{currentstroke}%
\pgfsetdash{}{0pt}%
\pgfpathmoveto{\pgfqpoint{7.438393in}{2.726586in}}%
\pgfpathlineto{\pgfqpoint{7.438301in}{2.726583in}}%
\pgfusepath{stroke}%
\end{pgfscope}%
\begin{pgfscope}%
\pgfpathrectangle{\pgfqpoint{6.720588in}{1.750000in}}{\pgfqpoint{2.279412in}{2.004545in}}%
\pgfusepath{clip}%
\pgfsetbuttcap%
\pgfsetroundjoin%
\pgfsetlinewidth{0.400327pt}%
\definecolor{currentstroke}{rgb}{0.281446,0.084320,0.407414}%
\pgfsetstrokecolor{currentstroke}%
\pgfsetdash{}{0pt}%
\pgfpathmoveto{\pgfqpoint{7.438301in}{2.726583in}}%
\pgfpathlineto{\pgfqpoint{7.438081in}{2.726573in}}%
\pgfusepath{stroke}%
\end{pgfscope}%
\begin{pgfscope}%
\pgfpathrectangle{\pgfqpoint{6.720588in}{1.750000in}}{\pgfqpoint{2.279412in}{2.004545in}}%
\pgfusepath{clip}%
\pgfsetbuttcap%
\pgfsetroundjoin%
\pgfsetlinewidth{0.399320pt}%
\definecolor{currentstroke}{rgb}{0.281446,0.084320,0.407414}%
\pgfsetstrokecolor{currentstroke}%
\pgfsetdash{}{0pt}%
\pgfpathmoveto{\pgfqpoint{7.438081in}{2.726573in}}%
\pgfpathlineto{\pgfqpoint{7.437951in}{2.726567in}}%
\pgfusepath{stroke}%
\end{pgfscope}%
\begin{pgfscope}%
\pgfpathrectangle{\pgfqpoint{6.720588in}{1.750000in}}{\pgfqpoint{2.279412in}{2.004545in}}%
\pgfusepath{clip}%
\pgfsetbuttcap%
\pgfsetroundjoin%
\pgfsetlinewidth{0.398724pt}%
\definecolor{currentstroke}{rgb}{0.281446,0.084320,0.407414}%
\pgfsetstrokecolor{currentstroke}%
\pgfsetdash{}{0pt}%
\pgfpathmoveto{\pgfqpoint{7.437951in}{2.726567in}}%
\pgfpathlineto{\pgfqpoint{7.438061in}{2.726572in}}%
\pgfusepath{stroke}%
\end{pgfscope}%
\begin{pgfscope}%
\pgfpathrectangle{\pgfqpoint{6.720588in}{1.750000in}}{\pgfqpoint{2.279412in}{2.004545in}}%
\pgfusepath{clip}%
\pgfsetbuttcap%
\pgfsetroundjoin%
\pgfsetlinewidth{0.399227pt}%
\definecolor{currentstroke}{rgb}{0.281446,0.084320,0.407414}%
\pgfsetstrokecolor{currentstroke}%
\pgfsetdash{}{0pt}%
\pgfpathmoveto{\pgfqpoint{7.438061in}{2.726572in}}%
\pgfpathlineto{\pgfqpoint{7.438290in}{2.726581in}}%
\pgfusepath{stroke}%
\end{pgfscope}%
\begin{pgfscope}%
\pgfpathrectangle{\pgfqpoint{6.720588in}{1.750000in}}{\pgfqpoint{2.279412in}{2.004545in}}%
\pgfusepath{clip}%
\pgfsetbuttcap%
\pgfsetroundjoin%
\pgfsetlinewidth{0.400277pt}%
\definecolor{currentstroke}{rgb}{0.281446,0.084320,0.407414}%
\pgfsetstrokecolor{currentstroke}%
\pgfsetdash{}{0pt}%
\pgfpathmoveto{\pgfqpoint{7.438290in}{2.726581in}}%
\pgfpathlineto{\pgfqpoint{7.438414in}{2.726587in}}%
\pgfusepath{stroke}%
\end{pgfscope}%
\begin{pgfscope}%
\pgfpathrectangle{\pgfqpoint{6.720588in}{1.750000in}}{\pgfqpoint{2.279412in}{2.004545in}}%
\pgfusepath{clip}%
\pgfsetbuttcap%
\pgfsetroundjoin%
\pgfsetlinewidth{0.400850pt}%
\definecolor{currentstroke}{rgb}{0.281446,0.084320,0.407414}%
\pgfsetstrokecolor{currentstroke}%
\pgfsetdash{}{0pt}%
\pgfpathmoveto{\pgfqpoint{7.438414in}{2.726587in}}%
\pgfpathlineto{\pgfqpoint{7.438324in}{2.726584in}}%
\pgfusepath{stroke}%
\end{pgfscope}%
\begin{pgfscope}%
\pgfpathrectangle{\pgfqpoint{6.720588in}{1.750000in}}{\pgfqpoint{2.279412in}{2.004545in}}%
\pgfusepath{clip}%
\pgfsetbuttcap%
\pgfsetroundjoin%
\pgfsetlinewidth{0.400434pt}%
\definecolor{currentstroke}{rgb}{0.281446,0.084320,0.407414}%
\pgfsetstrokecolor{currentstroke}%
\pgfsetdash{}{0pt}%
\pgfpathmoveto{\pgfqpoint{7.438324in}{2.726584in}}%
\pgfpathlineto{\pgfqpoint{7.438081in}{2.726574in}}%
\pgfusepath{stroke}%
\end{pgfscope}%
\begin{pgfscope}%
\pgfpathrectangle{\pgfqpoint{6.720588in}{1.750000in}}{\pgfqpoint{2.279412in}{2.004545in}}%
\pgfusepath{clip}%
\pgfsetbuttcap%
\pgfsetroundjoin%
\pgfsetlinewidth{0.399318pt}%
\definecolor{currentstroke}{rgb}{0.281446,0.084320,0.407414}%
\pgfsetstrokecolor{currentstroke}%
\pgfsetdash{}{0pt}%
\pgfpathmoveto{\pgfqpoint{7.438081in}{2.726574in}}%
\pgfpathlineto{\pgfqpoint{7.437922in}{2.726566in}}%
\pgfusepath{stroke}%
\end{pgfscope}%
\begin{pgfscope}%
\pgfpathrectangle{\pgfqpoint{6.720588in}{1.750000in}}{\pgfqpoint{2.279412in}{2.004545in}}%
\pgfusepath{clip}%
\pgfsetbuttcap%
\pgfsetroundjoin%
\pgfsetlinewidth{0.398589pt}%
\definecolor{currentstroke}{rgb}{0.281446,0.084320,0.407414}%
\pgfsetstrokecolor{currentstroke}%
\pgfsetdash{}{0pt}%
\pgfpathmoveto{\pgfqpoint{7.437922in}{2.726566in}}%
\pgfpathlineto{\pgfqpoint{7.438032in}{2.726570in}}%
\pgfusepath{stroke}%
\end{pgfscope}%
\begin{pgfscope}%
\pgfpathrectangle{\pgfqpoint{6.720588in}{1.750000in}}{\pgfqpoint{2.279412in}{2.004545in}}%
\pgfusepath{clip}%
\pgfsetbuttcap%
\pgfsetroundjoin%
\pgfsetlinewidth{0.399096pt}%
\definecolor{currentstroke}{rgb}{0.281446,0.084320,0.407414}%
\pgfsetstrokecolor{currentstroke}%
\pgfsetdash{}{0pt}%
\pgfpathmoveto{\pgfqpoint{7.438032in}{2.726570in}}%
\pgfpathlineto{\pgfqpoint{7.438288in}{2.726581in}}%
\pgfusepath{stroke}%
\end{pgfscope}%
\begin{pgfscope}%
\pgfpathrectangle{\pgfqpoint{6.720588in}{1.750000in}}{\pgfqpoint{2.279412in}{2.004545in}}%
\pgfusepath{clip}%
\pgfsetbuttcap%
\pgfsetroundjoin%
\pgfsetlinewidth{0.400268pt}%
\definecolor{currentstroke}{rgb}{0.281446,0.084320,0.407414}%
\pgfsetstrokecolor{currentstroke}%
\pgfsetdash{}{0pt}%
\pgfpathmoveto{\pgfqpoint{7.438288in}{2.726581in}}%
\pgfpathlineto{\pgfqpoint{7.438437in}{2.726588in}}%
\pgfusepath{stroke}%
\end{pgfscope}%
\begin{pgfscope}%
\pgfpathrectangle{\pgfqpoint{6.720588in}{1.750000in}}{\pgfqpoint{2.279412in}{2.004545in}}%
\pgfusepath{clip}%
\pgfsetbuttcap%
\pgfsetroundjoin%
\pgfsetlinewidth{0.400956pt}%
\definecolor{currentstroke}{rgb}{0.281446,0.084320,0.407414}%
\pgfsetstrokecolor{currentstroke}%
\pgfsetdash{}{0pt}%
\pgfpathmoveto{\pgfqpoint{7.438437in}{2.726588in}}%
\pgfpathlineto{\pgfqpoint{7.438351in}{2.726585in}}%
\pgfusepath{stroke}%
\end{pgfscope}%
\begin{pgfscope}%
\pgfpathrectangle{\pgfqpoint{6.720588in}{1.750000in}}{\pgfqpoint{2.279412in}{2.004545in}}%
\pgfusepath{clip}%
\pgfsetbuttcap%
\pgfsetroundjoin%
\pgfsetlinewidth{0.400557pt}%
\definecolor{currentstroke}{rgb}{0.281446,0.084320,0.407414}%
\pgfsetstrokecolor{currentstroke}%
\pgfsetdash{}{0pt}%
\pgfpathmoveto{\pgfqpoint{7.438351in}{2.726585in}}%
\pgfpathlineto{\pgfqpoint{7.438082in}{2.726574in}}%
\pgfusepath{stroke}%
\end{pgfscope}%
\begin{pgfscope}%
\pgfpathrectangle{\pgfqpoint{6.720588in}{1.750000in}}{\pgfqpoint{2.279412in}{2.004545in}}%
\pgfusepath{clip}%
\pgfsetbuttcap%
\pgfsetroundjoin%
\pgfsetlinewidth{0.399326pt}%
\definecolor{currentstroke}{rgb}{0.281446,0.084320,0.407414}%
\pgfsetstrokecolor{currentstroke}%
\pgfsetdash{}{0pt}%
\pgfpathmoveto{\pgfqpoint{7.438082in}{2.726574in}}%
\pgfpathlineto{\pgfqpoint{7.437890in}{2.726565in}}%
\pgfusepath{stroke}%
\end{pgfscope}%
\begin{pgfscope}%
\pgfpathrectangle{\pgfqpoint{6.720588in}{1.750000in}}{\pgfqpoint{2.279412in}{2.004545in}}%
\pgfusepath{clip}%
\pgfsetbuttcap%
\pgfsetroundjoin%
\pgfsetlinewidth{0.398443pt}%
\definecolor{currentstroke}{rgb}{0.281446,0.084320,0.407414}%
\pgfsetstrokecolor{currentstroke}%
\pgfsetdash{}{0pt}%
\pgfpathmoveto{\pgfqpoint{7.437890in}{2.726565in}}%
\pgfpathlineto{\pgfqpoint{7.437999in}{2.726569in}}%
\pgfusepath{stroke}%
\end{pgfscope}%
\begin{pgfscope}%
\pgfpathrectangle{\pgfqpoint{6.720588in}{1.750000in}}{\pgfqpoint{2.279412in}{2.004545in}}%
\pgfusepath{clip}%
\pgfsetbuttcap%
\pgfsetroundjoin%
\pgfsetlinewidth{0.398942pt}%
\definecolor{currentstroke}{rgb}{0.281446,0.084320,0.407414}%
\pgfsetstrokecolor{currentstroke}%
\pgfsetdash{}{0pt}%
\pgfpathmoveto{\pgfqpoint{7.437999in}{2.726569in}}%
\pgfpathlineto{\pgfqpoint{7.438284in}{2.726581in}}%
\pgfusepath{stroke}%
\end{pgfscope}%
\begin{pgfscope}%
\pgfpathrectangle{\pgfqpoint{6.720588in}{1.750000in}}{\pgfqpoint{2.279412in}{2.004545in}}%
\pgfusepath{clip}%
\pgfsetbuttcap%
\pgfsetroundjoin%
\pgfsetlinewidth{0.400249pt}%
\definecolor{currentstroke}{rgb}{0.281446,0.084320,0.407414}%
\pgfsetstrokecolor{currentstroke}%
\pgfsetdash{}{0pt}%
\pgfpathmoveto{\pgfqpoint{7.438284in}{2.726581in}}%
\pgfpathlineto{\pgfqpoint{7.438461in}{2.726589in}}%
\pgfusepath{stroke}%
\end{pgfscope}%
\begin{pgfscope}%
\pgfpathrectangle{\pgfqpoint{6.720588in}{1.750000in}}{\pgfqpoint{2.279412in}{2.004545in}}%
\pgfusepath{clip}%
\pgfsetbuttcap%
\pgfsetroundjoin%
\pgfsetlinewidth{0.401066pt}%
\definecolor{currentstroke}{rgb}{0.281446,0.084320,0.407414}%
\pgfsetstrokecolor{currentstroke}%
\pgfsetdash{}{0pt}%
\pgfpathmoveto{\pgfqpoint{7.438461in}{2.726589in}}%
\pgfpathlineto{\pgfqpoint{7.438381in}{2.726586in}}%
\pgfusepath{stroke}%
\end{pgfscope}%
\begin{pgfscope}%
\pgfpathrectangle{\pgfqpoint{6.720588in}{1.750000in}}{\pgfqpoint{2.279412in}{2.004545in}}%
\pgfusepath{clip}%
\pgfsetbuttcap%
\pgfsetroundjoin%
\pgfsetlinewidth{0.400696pt}%
\definecolor{currentstroke}{rgb}{0.281446,0.084320,0.407414}%
\pgfsetstrokecolor{currentstroke}%
\pgfsetdash{}{0pt}%
\pgfpathmoveto{\pgfqpoint{7.438381in}{2.726586in}}%
\pgfpathlineto{\pgfqpoint{7.438087in}{2.726574in}}%
\pgfusepath{stroke}%
\end{pgfscope}%
\begin{pgfscope}%
\pgfpathrectangle{\pgfqpoint{6.720588in}{1.750000in}}{\pgfqpoint{2.279412in}{2.004545in}}%
\pgfusepath{clip}%
\pgfsetbuttcap%
\pgfsetroundjoin%
\pgfsetlinewidth{0.399346pt}%
\definecolor{currentstroke}{rgb}{0.281446,0.084320,0.407414}%
\pgfsetstrokecolor{currentstroke}%
\pgfsetdash{}{0pt}%
\pgfpathmoveto{\pgfqpoint{7.438087in}{2.726574in}}%
\pgfpathlineto{\pgfqpoint{7.437855in}{2.726563in}}%
\pgfusepath{stroke}%
\end{pgfscope}%
\begin{pgfscope}%
\pgfpathrectangle{\pgfqpoint{6.720588in}{1.750000in}}{\pgfqpoint{2.279412in}{2.004545in}}%
\pgfusepath{clip}%
\pgfsetbuttcap%
\pgfsetroundjoin%
\pgfsetlinewidth{0.398285pt}%
\definecolor{currentstroke}{rgb}{0.281446,0.084320,0.407414}%
\pgfsetstrokecolor{currentstroke}%
\pgfsetdash{}{0pt}%
\pgfpathmoveto{\pgfqpoint{7.437855in}{2.726563in}}%
\pgfpathlineto{\pgfqpoint{7.437960in}{2.726567in}}%
\pgfusepath{stroke}%
\end{pgfscope}%
\begin{pgfscope}%
\pgfpathrectangle{\pgfqpoint{6.720588in}{1.750000in}}{\pgfqpoint{2.279412in}{2.004545in}}%
\pgfusepath{clip}%
\pgfsetbuttcap%
\pgfsetroundjoin%
\pgfsetlinewidth{0.398764pt}%
\definecolor{currentstroke}{rgb}{0.281446,0.084320,0.407414}%
\pgfsetstrokecolor{currentstroke}%
\pgfsetdash{}{0pt}%
\pgfpathmoveto{\pgfqpoint{7.437960in}{2.726567in}}%
\pgfpathlineto{\pgfqpoint{7.438276in}{2.726580in}}%
\pgfusepath{stroke}%
\end{pgfscope}%
\begin{pgfscope}%
\pgfpathrectangle{\pgfqpoint{6.720588in}{1.750000in}}{\pgfqpoint{2.279412in}{2.004545in}}%
\pgfusepath{clip}%
\pgfsetbuttcap%
\pgfsetroundjoin%
\pgfsetlinewidth{0.400215pt}%
\definecolor{currentstroke}{rgb}{0.281446,0.084320,0.407414}%
\pgfsetstrokecolor{currentstroke}%
\pgfsetdash{}{0pt}%
\pgfpathmoveto{\pgfqpoint{7.438276in}{2.726580in}}%
\pgfpathlineto{\pgfqpoint{7.438486in}{2.726590in}}%
\pgfusepath{stroke}%
\end{pgfscope}%
\begin{pgfscope}%
\pgfpathrectangle{\pgfqpoint{6.720588in}{1.750000in}}{\pgfqpoint{2.279412in}{2.004545in}}%
\pgfusepath{clip}%
\pgfsetbuttcap%
\pgfsetroundjoin%
\pgfsetlinewidth{0.401178pt}%
\definecolor{currentstroke}{rgb}{0.281446,0.084320,0.407414}%
\pgfsetstrokecolor{currentstroke}%
\pgfsetdash{}{0pt}%
\pgfpathmoveto{\pgfqpoint{7.438486in}{2.726590in}}%
\pgfpathlineto{\pgfqpoint{7.438415in}{2.726588in}}%
\pgfusepath{stroke}%
\end{pgfscope}%
\begin{pgfscope}%
\pgfpathrectangle{\pgfqpoint{6.720588in}{1.750000in}}{\pgfqpoint{2.279412in}{2.004545in}}%
\pgfusepath{clip}%
\pgfsetbuttcap%
\pgfsetroundjoin%
\pgfsetlinewidth{0.400850pt}%
\definecolor{currentstroke}{rgb}{0.281446,0.084320,0.407414}%
\pgfsetstrokecolor{currentstroke}%
\pgfsetdash{}{0pt}%
\pgfpathmoveto{\pgfqpoint{7.438415in}{2.726588in}}%
\pgfpathlineto{\pgfqpoint{7.438095in}{2.726574in}}%
\pgfusepath{stroke}%
\end{pgfscope}%
\begin{pgfscope}%
\pgfpathrectangle{\pgfqpoint{6.720588in}{1.750000in}}{\pgfqpoint{2.279412in}{2.004545in}}%
\pgfusepath{clip}%
\pgfsetbuttcap%
\pgfsetroundjoin%
\pgfsetlinewidth{0.399383pt}%
\definecolor{currentstroke}{rgb}{0.281446,0.084320,0.407414}%
\pgfsetstrokecolor{currentstroke}%
\pgfsetdash{}{0pt}%
\pgfpathmoveto{\pgfqpoint{7.438095in}{2.726574in}}%
\pgfpathlineto{\pgfqpoint{7.437819in}{2.726562in}}%
\pgfusepath{stroke}%
\end{pgfscope}%
\begin{pgfscope}%
\pgfpathrectangle{\pgfqpoint{6.720588in}{1.750000in}}{\pgfqpoint{2.279412in}{2.004545in}}%
\pgfusepath{clip}%
\pgfsetbuttcap%
\pgfsetroundjoin%
\pgfsetlinewidth{0.398119pt}%
\definecolor{currentstroke}{rgb}{0.281446,0.084320,0.407414}%
\pgfsetstrokecolor{currentstroke}%
\pgfsetdash{}{0pt}%
\pgfpathmoveto{\pgfqpoint{7.437819in}{2.726562in}}%
\pgfpathlineto{\pgfqpoint{7.437915in}{2.726565in}}%
\pgfusepath{stroke}%
\end{pgfscope}%
\begin{pgfscope}%
\pgfpathrectangle{\pgfqpoint{6.720588in}{1.750000in}}{\pgfqpoint{2.279412in}{2.004545in}}%
\pgfusepath{clip}%
\pgfsetbuttcap%
\pgfsetroundjoin%
\pgfsetlinewidth{0.398559pt}%
\definecolor{currentstroke}{rgb}{0.281446,0.084320,0.407414}%
\pgfsetstrokecolor{currentstroke}%
\pgfsetdash{}{0pt}%
\pgfpathmoveto{\pgfqpoint{7.437915in}{2.726565in}}%
\pgfpathlineto{\pgfqpoint{7.438265in}{2.726580in}}%
\pgfusepath{stroke}%
\end{pgfscope}%
\begin{pgfscope}%
\pgfpathrectangle{\pgfqpoint{6.720588in}{1.750000in}}{\pgfqpoint{2.279412in}{2.004545in}}%
\pgfusepath{clip}%
\pgfsetbuttcap%
\pgfsetroundjoin%
\pgfsetlinewidth{0.400164pt}%
\definecolor{currentstroke}{rgb}{0.281446,0.084320,0.407414}%
\pgfsetstrokecolor{currentstroke}%
\pgfsetdash{}{0pt}%
\pgfpathmoveto{\pgfqpoint{7.438265in}{2.726580in}}%
\pgfpathlineto{\pgfqpoint{7.438510in}{2.726591in}}%
\pgfusepath{stroke}%
\end{pgfscope}%
\begin{pgfscope}%
\pgfpathrectangle{\pgfqpoint{6.720588in}{1.750000in}}{\pgfqpoint{2.279412in}{2.004545in}}%
\pgfusepath{clip}%
\pgfsetbuttcap%
\pgfsetroundjoin%
\pgfsetlinewidth{0.401291pt}%
\definecolor{currentstroke}{rgb}{0.281446,0.084320,0.407414}%
\pgfsetstrokecolor{currentstroke}%
\pgfsetdash{}{0pt}%
\pgfpathmoveto{\pgfqpoint{7.438510in}{2.726591in}}%
\pgfpathlineto{\pgfqpoint{7.438452in}{2.726589in}}%
\pgfusepath{stroke}%
\end{pgfscope}%
\begin{pgfscope}%
\pgfpathrectangle{\pgfqpoint{6.720588in}{1.750000in}}{\pgfqpoint{2.279412in}{2.004545in}}%
\pgfusepath{clip}%
\pgfsetbuttcap%
\pgfsetroundjoin%
\pgfsetlinewidth{0.401022pt}%
\definecolor{currentstroke}{rgb}{0.281446,0.084320,0.407414}%
\pgfsetstrokecolor{currentstroke}%
\pgfsetdash{}{0pt}%
\pgfpathmoveto{\pgfqpoint{7.438452in}{2.726589in}}%
\pgfpathlineto{\pgfqpoint{7.438107in}{2.726575in}}%
\pgfusepath{stroke}%
\end{pgfscope}%
\begin{pgfscope}%
\pgfpathrectangle{\pgfqpoint{6.720588in}{1.750000in}}{\pgfqpoint{2.279412in}{2.004545in}}%
\pgfusepath{clip}%
\pgfsetbuttcap%
\pgfsetroundjoin%
\pgfsetlinewidth{0.399439pt}%
\definecolor{currentstroke}{rgb}{0.281446,0.084320,0.407414}%
\pgfsetstrokecolor{currentstroke}%
\pgfsetdash{}{0pt}%
\pgfpathmoveto{\pgfqpoint{7.438107in}{2.726575in}}%
\pgfpathlineto{\pgfqpoint{7.437781in}{2.726560in}}%
\pgfusepath{stroke}%
\end{pgfscope}%
\begin{pgfscope}%
\pgfpathrectangle{\pgfqpoint{6.720588in}{1.750000in}}{\pgfqpoint{2.279412in}{2.004545in}}%
\pgfusepath{clip}%
\pgfsetbuttcap%
\pgfsetroundjoin%
\pgfsetlinewidth{0.397944pt}%
\definecolor{currentstroke}{rgb}{0.281446,0.084320,0.407414}%
\pgfsetstrokecolor{currentstroke}%
\pgfsetdash{}{0pt}%
\pgfpathmoveto{\pgfqpoint{7.437781in}{2.726560in}}%
\pgfpathlineto{\pgfqpoint{7.437863in}{2.726563in}}%
\pgfusepath{stroke}%
\end{pgfscope}%
\begin{pgfscope}%
\pgfpathrectangle{\pgfqpoint{6.720588in}{1.750000in}}{\pgfqpoint{2.279412in}{2.004545in}}%
\pgfusepath{clip}%
\pgfsetbuttcap%
\pgfsetroundjoin%
\pgfsetlinewidth{0.398322pt}%
\definecolor{currentstroke}{rgb}{0.281446,0.084320,0.407414}%
\pgfsetstrokecolor{currentstroke}%
\pgfsetdash{}{0pt}%
\pgfpathmoveto{\pgfqpoint{7.437863in}{2.726563in}}%
\pgfpathlineto{\pgfqpoint{7.438249in}{2.726579in}}%
\pgfusepath{stroke}%
\end{pgfscope}%
\begin{pgfscope}%
\pgfpathrectangle{\pgfqpoint{6.720588in}{1.750000in}}{\pgfqpoint{2.279412in}{2.004545in}}%
\pgfusepath{clip}%
\pgfsetbuttcap%
\pgfsetroundjoin%
\pgfsetlinewidth{0.400091pt}%
\definecolor{currentstroke}{rgb}{0.281446,0.084320,0.407414}%
\pgfsetstrokecolor{currentstroke}%
\pgfsetdash{}{0pt}%
\pgfpathmoveto{\pgfqpoint{7.438249in}{2.726579in}}%
\pgfpathlineto{\pgfqpoint{7.438534in}{2.726592in}}%
\pgfusepath{stroke}%
\end{pgfscope}%
\begin{pgfscope}%
\pgfpathrectangle{\pgfqpoint{6.720588in}{1.750000in}}{\pgfqpoint{2.279412in}{2.004545in}}%
\pgfusepath{clip}%
\pgfsetbuttcap%
\pgfsetroundjoin%
\pgfsetlinewidth{0.401402pt}%
\definecolor{currentstroke}{rgb}{0.281446,0.084320,0.407414}%
\pgfsetstrokecolor{currentstroke}%
\pgfsetdash{}{0pt}%
\pgfpathmoveto{\pgfqpoint{7.438534in}{2.726592in}}%
\pgfpathlineto{\pgfqpoint{7.438493in}{2.726591in}}%
\pgfusepath{stroke}%
\end{pgfscope}%
\begin{pgfscope}%
\pgfpathrectangle{\pgfqpoint{6.720588in}{1.750000in}}{\pgfqpoint{2.279412in}{2.004545in}}%
\pgfusepath{clip}%
\pgfsetbuttcap%
\pgfsetroundjoin%
\pgfsetlinewidth{0.401210pt}%
\definecolor{currentstroke}{rgb}{0.281446,0.084320,0.407414}%
\pgfsetstrokecolor{currentstroke}%
\pgfsetdash{}{0pt}%
\pgfpathmoveto{\pgfqpoint{7.438493in}{2.726591in}}%
\pgfpathlineto{\pgfqpoint{7.438125in}{2.726576in}}%
\pgfusepath{stroke}%
\end{pgfscope}%
\begin{pgfscope}%
\pgfpathrectangle{\pgfqpoint{6.720588in}{1.750000in}}{\pgfqpoint{2.279412in}{2.004545in}}%
\pgfusepath{clip}%
\pgfsetbuttcap%
\pgfsetroundjoin%
\pgfsetlinewidth{0.399519pt}%
\definecolor{currentstroke}{rgb}{0.281446,0.084320,0.407414}%
\pgfsetstrokecolor{currentstroke}%
\pgfsetdash{}{0pt}%
\pgfpathmoveto{\pgfqpoint{7.438125in}{2.726576in}}%
\pgfpathlineto{\pgfqpoint{7.437742in}{2.726559in}}%
\pgfusepath{stroke}%
\end{pgfscope}%
\begin{pgfscope}%
\pgfpathrectangle{\pgfqpoint{6.720588in}{1.750000in}}{\pgfqpoint{2.279412in}{2.004545in}}%
\pgfusepath{clip}%
\pgfsetbuttcap%
\pgfsetroundjoin%
\pgfsetlinewidth{0.397766pt}%
\definecolor{currentstroke}{rgb}{0.281446,0.084320,0.407414}%
\pgfsetstrokecolor{currentstroke}%
\pgfsetdash{}{0pt}%
\pgfpathmoveto{\pgfqpoint{7.437742in}{2.726559in}}%
\pgfpathlineto{\pgfqpoint{7.437805in}{2.726560in}}%
\pgfusepath{stroke}%
\end{pgfscope}%
\begin{pgfscope}%
\pgfpathrectangle{\pgfqpoint{6.720588in}{1.750000in}}{\pgfqpoint{2.279412in}{2.004545in}}%
\pgfusepath{clip}%
\pgfsetbuttcap%
\pgfsetroundjoin%
\pgfsetlinewidth{0.398053pt}%
\definecolor{currentstroke}{rgb}{0.281446,0.084320,0.407414}%
\pgfsetstrokecolor{currentstroke}%
\pgfsetdash{}{0pt}%
\pgfpathmoveto{\pgfqpoint{7.437805in}{2.726560in}}%
\pgfpathlineto{\pgfqpoint{7.438228in}{2.726578in}}%
\pgfusepath{stroke}%
\end{pgfscope}%
\begin{pgfscope}%
\pgfpathrectangle{\pgfqpoint{6.720588in}{1.750000in}}{\pgfqpoint{2.279412in}{2.004545in}}%
\pgfusepath{clip}%
\pgfsetbuttcap%
\pgfsetroundjoin%
\pgfsetlinewidth{0.399992pt}%
\definecolor{currentstroke}{rgb}{0.281446,0.084320,0.407414}%
\pgfsetstrokecolor{currentstroke}%
\pgfsetdash{}{0pt}%
\pgfpathmoveto{\pgfqpoint{7.438228in}{2.726578in}}%
\pgfpathlineto{\pgfqpoint{7.438557in}{2.726592in}}%
\pgfusepath{stroke}%
\end{pgfscope}%
\begin{pgfscope}%
\pgfpathrectangle{\pgfqpoint{6.720588in}{1.750000in}}{\pgfqpoint{2.279412in}{2.004545in}}%
\pgfusepath{clip}%
\pgfsetbuttcap%
\pgfsetroundjoin%
\pgfsetlinewidth{0.401507pt}%
\definecolor{currentstroke}{rgb}{0.281446,0.084320,0.407414}%
\pgfsetstrokecolor{currentstroke}%
\pgfsetdash{}{0pt}%
\pgfpathmoveto{\pgfqpoint{7.438557in}{2.726592in}}%
\pgfpathlineto{\pgfqpoint{7.438537in}{2.726593in}}%
\pgfusepath{stroke}%
\end{pgfscope}%
\begin{pgfscope}%
\pgfpathrectangle{\pgfqpoint{6.720588in}{1.750000in}}{\pgfqpoint{2.279412in}{2.004545in}}%
\pgfusepath{clip}%
\pgfsetbuttcap%
\pgfsetroundjoin%
\pgfsetlinewidth{0.401413pt}%
\definecolor{currentstroke}{rgb}{0.281446,0.084320,0.407414}%
\pgfsetstrokecolor{currentstroke}%
\pgfsetdash{}{0pt}%
\pgfpathmoveto{\pgfqpoint{7.438537in}{2.726593in}}%
\pgfpathlineto{\pgfqpoint{7.438148in}{2.726577in}}%
\pgfusepath{stroke}%
\end{pgfscope}%
\begin{pgfscope}%
\pgfpathrectangle{\pgfqpoint{6.720588in}{1.750000in}}{\pgfqpoint{2.279412in}{2.004545in}}%
\pgfusepath{clip}%
\pgfsetbuttcap%
\pgfsetroundjoin%
\pgfsetlinewidth{0.399626pt}%
\definecolor{currentstroke}{rgb}{0.281446,0.084320,0.407414}%
\pgfsetstrokecolor{currentstroke}%
\pgfsetdash{}{0pt}%
\pgfpathmoveto{\pgfqpoint{7.438148in}{2.726577in}}%
\pgfpathlineto{\pgfqpoint{7.437703in}{2.726557in}}%
\pgfusepath{stroke}%
\end{pgfscope}%
\begin{pgfscope}%
\pgfpathrectangle{\pgfqpoint{6.720588in}{1.750000in}}{\pgfqpoint{2.279412in}{2.004545in}}%
\pgfusepath{clip}%
\pgfsetbuttcap%
\pgfsetroundjoin%
\pgfsetlinewidth{0.397589pt}%
\definecolor{currentstroke}{rgb}{0.281446,0.084320,0.407414}%
\pgfsetstrokecolor{currentstroke}%
\pgfsetdash{}{0pt}%
\pgfpathmoveto{\pgfqpoint{7.437703in}{2.726557in}}%
\pgfpathlineto{\pgfqpoint{7.437738in}{2.726557in}}%
\pgfusepath{stroke}%
\end{pgfscope}%
\begin{pgfscope}%
\pgfpathrectangle{\pgfqpoint{6.720588in}{1.750000in}}{\pgfqpoint{2.279412in}{2.004545in}}%
\pgfusepath{clip}%
\pgfsetbuttcap%
\pgfsetroundjoin%
\pgfsetlinewidth{0.397748pt}%
\definecolor{currentstroke}{rgb}{0.281446,0.084320,0.407414}%
\pgfsetstrokecolor{currentstroke}%
\pgfsetdash{}{0pt}%
\pgfpathmoveto{\pgfqpoint{7.437738in}{2.726557in}}%
\pgfpathlineto{\pgfqpoint{7.438199in}{2.726576in}}%
\pgfusepath{stroke}%
\end{pgfscope}%
\begin{pgfscope}%
\pgfpathrectangle{\pgfqpoint{6.720588in}{1.750000in}}{\pgfqpoint{2.279412in}{2.004545in}}%
\pgfusepath{clip}%
\pgfsetbuttcap%
\pgfsetroundjoin%
\pgfsetlinewidth{0.399862pt}%
\definecolor{currentstroke}{rgb}{0.281446,0.084320,0.407414}%
\pgfsetstrokecolor{currentstroke}%
\pgfsetdash{}{0pt}%
\pgfpathmoveto{\pgfqpoint{7.438199in}{2.726576in}}%
\pgfpathlineto{\pgfqpoint{7.438578in}{2.726593in}}%
\pgfusepath{stroke}%
\end{pgfscope}%
\begin{pgfscope}%
\pgfpathrectangle{\pgfqpoint{6.720588in}{1.750000in}}{\pgfqpoint{2.279412in}{2.004545in}}%
\pgfusepath{clip}%
\pgfsetbuttcap%
\pgfsetroundjoin%
\pgfsetlinewidth{0.401601pt}%
\definecolor{currentstroke}{rgb}{0.281446,0.084320,0.407414}%
\pgfsetstrokecolor{currentstroke}%
\pgfsetdash{}{0pt}%
\pgfpathmoveto{\pgfqpoint{7.438578in}{2.726593in}}%
\pgfpathlineto{\pgfqpoint{7.438584in}{2.726594in}}%
\pgfusepath{stroke}%
\end{pgfscope}%
\begin{pgfscope}%
\pgfpathrectangle{\pgfqpoint{6.720588in}{1.750000in}}{\pgfqpoint{2.279412in}{2.004545in}}%
\pgfusepath{clip}%
\pgfsetbuttcap%
\pgfsetroundjoin%
\pgfsetlinewidth{0.401628pt}%
\definecolor{currentstroke}{rgb}{0.281446,0.084320,0.407414}%
\pgfsetstrokecolor{currentstroke}%
\pgfsetdash{}{0pt}%
\pgfpathmoveto{\pgfqpoint{7.438584in}{2.726594in}}%
\pgfpathlineto{\pgfqpoint{7.438178in}{2.726578in}}%
\pgfusepath{stroke}%
\end{pgfscope}%
\begin{pgfscope}%
\pgfpathrectangle{\pgfqpoint{6.720588in}{1.750000in}}{\pgfqpoint{2.279412in}{2.004545in}}%
\pgfusepath{clip}%
\pgfsetbuttcap%
\pgfsetroundjoin%
\pgfsetlinewidth{0.399765pt}%
\definecolor{currentstroke}{rgb}{0.281446,0.084320,0.407414}%
\pgfsetstrokecolor{currentstroke}%
\pgfsetdash{}{0pt}%
\pgfpathmoveto{\pgfqpoint{7.438178in}{2.726578in}}%
\pgfpathlineto{\pgfqpoint{7.437666in}{2.726556in}}%
\pgfusepath{stroke}%
\end{pgfscope}%
\begin{pgfscope}%
\pgfpathrectangle{\pgfqpoint{6.720588in}{1.750000in}}{\pgfqpoint{2.279412in}{2.004545in}}%
\pgfusepath{clip}%
\pgfsetbuttcap%
\pgfsetroundjoin%
\pgfsetlinewidth{0.397420pt}%
\definecolor{currentstroke}{rgb}{0.281446,0.084320,0.407414}%
\pgfsetstrokecolor{currentstroke}%
\pgfsetdash{}{0pt}%
\pgfpathmoveto{\pgfqpoint{7.437666in}{2.726556in}}%
\pgfpathlineto{\pgfqpoint{7.437663in}{2.726554in}}%
\pgfusepath{stroke}%
\end{pgfscope}%
\begin{pgfscope}%
\pgfpathrectangle{\pgfqpoint{6.720588in}{1.750000in}}{\pgfqpoint{2.279412in}{2.004545in}}%
\pgfusepath{clip}%
\pgfsetbuttcap%
\pgfsetroundjoin%
\pgfsetlinewidth{0.397406pt}%
\definecolor{currentstroke}{rgb}{0.281446,0.084320,0.407414}%
\pgfsetstrokecolor{currentstroke}%
\pgfsetdash{}{0pt}%
\pgfpathmoveto{\pgfqpoint{7.437663in}{2.726554in}}%
\pgfpathlineto{\pgfqpoint{7.438163in}{2.726574in}}%
\pgfusepath{stroke}%
\end{pgfscope}%
\begin{pgfscope}%
\pgfpathrectangle{\pgfqpoint{6.720588in}{1.750000in}}{\pgfqpoint{2.279412in}{2.004545in}}%
\pgfusepath{clip}%
\pgfsetbuttcap%
\pgfsetroundjoin%
\pgfsetlinewidth{0.399694pt}%
\definecolor{currentstroke}{rgb}{0.281446,0.084320,0.407414}%
\pgfsetstrokecolor{currentstroke}%
\pgfsetdash{}{0pt}%
\pgfpathmoveto{\pgfqpoint{7.438163in}{2.726574in}}%
\pgfpathlineto{\pgfqpoint{7.438595in}{2.726593in}}%
\pgfusepath{stroke}%
\end{pgfscope}%
\begin{pgfscope}%
\pgfpathrectangle{\pgfqpoint{6.720588in}{1.750000in}}{\pgfqpoint{2.279412in}{2.004545in}}%
\pgfusepath{clip}%
\pgfsetbuttcap%
\pgfsetroundjoin%
\pgfsetlinewidth{0.401679pt}%
\definecolor{currentstroke}{rgb}{0.281446,0.084320,0.407414}%
\pgfsetstrokecolor{currentstroke}%
\pgfsetdash{}{0pt}%
\pgfpathmoveto{\pgfqpoint{7.438595in}{2.726593in}}%
\pgfpathlineto{\pgfqpoint{7.438633in}{2.726596in}}%
\pgfusepath{stroke}%
\end{pgfscope}%
\begin{pgfscope}%
\pgfpathrectangle{\pgfqpoint{6.720588in}{1.750000in}}{\pgfqpoint{2.279412in}{2.004545in}}%
\pgfusepath{clip}%
\pgfsetbuttcap%
\pgfsetroundjoin%
\pgfsetlinewidth{0.401854pt}%
\definecolor{currentstroke}{rgb}{0.281446,0.084320,0.407414}%
\pgfsetstrokecolor{currentstroke}%
\pgfsetdash{}{0pt}%
\pgfpathmoveto{\pgfqpoint{7.438633in}{2.726596in}}%
\pgfpathlineto{\pgfqpoint{7.438216in}{2.726580in}}%
\pgfusepath{stroke}%
\end{pgfscope}%
\begin{pgfscope}%
\pgfpathrectangle{\pgfqpoint{6.720588in}{1.750000in}}{\pgfqpoint{2.279412in}{2.004545in}}%
\pgfusepath{clip}%
\pgfsetbuttcap%
\pgfsetroundjoin%
\pgfsetlinewidth{0.399940pt}%
\definecolor{currentstroke}{rgb}{0.281446,0.084320,0.407414}%
\pgfsetstrokecolor{currentstroke}%
\pgfsetdash{}{0pt}%
\pgfpathmoveto{\pgfqpoint{7.438216in}{2.726580in}}%
\pgfpathlineto{\pgfqpoint{7.437634in}{2.726554in}}%
\pgfusepath{stroke}%
\end{pgfscope}%
\begin{pgfscope}%
\pgfpathrectangle{\pgfqpoint{6.720588in}{1.750000in}}{\pgfqpoint{2.279412in}{2.004545in}}%
\pgfusepath{clip}%
\pgfsetbuttcap%
\pgfsetroundjoin%
\pgfsetlinewidth{0.397271pt}%
\definecolor{currentstroke}{rgb}{0.281446,0.084320,0.407414}%
\pgfsetstrokecolor{currentstroke}%
\pgfsetdash{}{0pt}%
\pgfpathmoveto{\pgfqpoint{7.437634in}{2.726554in}}%
\pgfpathlineto{\pgfqpoint{7.437580in}{2.726550in}}%
\pgfusepath{stroke}%
\end{pgfscope}%
\begin{pgfscope}%
\pgfpathrectangle{\pgfqpoint{6.720588in}{1.750000in}}{\pgfqpoint{2.279412in}{2.004545in}}%
\pgfusepath{clip}%
\pgfsetbuttcap%
\pgfsetroundjoin%
\pgfsetlinewidth{0.397029pt}%
\definecolor{currentstroke}{rgb}{0.280894,0.078907,0.402329}%
\pgfsetstrokecolor{currentstroke}%
\pgfsetdash{}{0pt}%
\pgfpathmoveto{\pgfqpoint{7.437580in}{2.726550in}}%
\pgfpathlineto{\pgfqpoint{7.438117in}{2.726572in}}%
\pgfusepath{stroke}%
\end{pgfscope}%
\begin{pgfscope}%
\pgfpathrectangle{\pgfqpoint{6.720588in}{1.750000in}}{\pgfqpoint{2.279412in}{2.004545in}}%
\pgfusepath{clip}%
\pgfsetbuttcap%
\pgfsetroundjoin%
\pgfsetlinewidth{0.399484pt}%
\definecolor{currentstroke}{rgb}{0.281446,0.084320,0.407414}%
\pgfsetstrokecolor{currentstroke}%
\pgfsetdash{}{0pt}%
\pgfpathmoveto{\pgfqpoint{7.438117in}{2.726572in}}%
\pgfpathlineto{\pgfqpoint{7.438607in}{2.726594in}}%
\pgfusepath{stroke}%
\end{pgfscope}%
\begin{pgfscope}%
\pgfpathrectangle{\pgfqpoint{6.720588in}{1.750000in}}{\pgfqpoint{2.279412in}{2.004545in}}%
\pgfusepath{clip}%
\pgfsetbuttcap%
\pgfsetroundjoin%
\pgfsetlinewidth{0.401735pt}%
\definecolor{currentstroke}{rgb}{0.281446,0.084320,0.407414}%
\pgfsetstrokecolor{currentstroke}%
\pgfsetdash{}{0pt}%
\pgfpathmoveto{\pgfqpoint{7.438607in}{2.726594in}}%
\pgfpathlineto{\pgfqpoint{7.438682in}{2.726598in}}%
\pgfusepath{stroke}%
\end{pgfscope}%
\begin{pgfscope}%
\pgfpathrectangle{\pgfqpoint{6.720588in}{1.750000in}}{\pgfqpoint{2.279412in}{2.004545in}}%
\pgfusepath{clip}%
\pgfsetbuttcap%
\pgfsetroundjoin%
\pgfsetlinewidth{0.402083pt}%
\definecolor{currentstroke}{rgb}{0.281446,0.084320,0.407414}%
\pgfsetstrokecolor{currentstroke}%
\pgfsetdash{}{0pt}%
\pgfpathmoveto{\pgfqpoint{7.438682in}{2.726598in}}%
\pgfpathlineto{\pgfqpoint{7.438263in}{2.726582in}}%
\pgfusepath{stroke}%
\end{pgfscope}%
\begin{pgfscope}%
\pgfpathrectangle{\pgfqpoint{6.720588in}{1.750000in}}{\pgfqpoint{2.279412in}{2.004545in}}%
\pgfusepath{clip}%
\pgfsetbuttcap%
\pgfsetroundjoin%
\pgfsetlinewidth{0.400154pt}%
\definecolor{currentstroke}{rgb}{0.281446,0.084320,0.407414}%
\pgfsetstrokecolor{currentstroke}%
\pgfsetdash{}{0pt}%
\pgfpathmoveto{\pgfqpoint{7.438263in}{2.726582in}}%
\pgfpathlineto{\pgfqpoint{7.437608in}{2.726553in}}%
\pgfusepath{stroke}%
\end{pgfscope}%
\begin{pgfscope}%
\pgfpathrectangle{\pgfqpoint{6.720588in}{1.750000in}}{\pgfqpoint{2.279412in}{2.004545in}}%
\pgfusepath{clip}%
\pgfsetbuttcap%
\pgfsetroundjoin%
\pgfsetlinewidth{0.397152pt}%
\definecolor{currentstroke}{rgb}{0.280894,0.078907,0.402329}%
\pgfsetstrokecolor{currentstroke}%
\pgfsetdash{}{0pt}%
\pgfpathmoveto{\pgfqpoint{7.437608in}{2.726553in}}%
\pgfpathlineto{\pgfqpoint{7.437491in}{2.726546in}}%
\pgfusepath{stroke}%
\end{pgfscope}%
\begin{pgfscope}%
\pgfpathrectangle{\pgfqpoint{6.720588in}{1.750000in}}{\pgfqpoint{2.279412in}{2.004545in}}%
\pgfusepath{clip}%
\pgfsetbuttcap%
\pgfsetroundjoin%
\pgfsetlinewidth{0.396618pt}%
\definecolor{currentstroke}{rgb}{0.280894,0.078907,0.402329}%
\pgfsetstrokecolor{currentstroke}%
\pgfsetdash{}{0pt}%
\pgfpathmoveto{\pgfqpoint{7.437491in}{2.726546in}}%
\pgfpathlineto{\pgfqpoint{7.438060in}{2.726570in}}%
\pgfusepath{stroke}%
\end{pgfscope}%
\begin{pgfscope}%
\pgfpathrectangle{\pgfqpoint{6.720588in}{1.750000in}}{\pgfqpoint{2.279412in}{2.004545in}}%
\pgfusepath{clip}%
\pgfsetbuttcap%
\pgfsetroundjoin%
\pgfsetlinewidth{0.399224pt}%
\definecolor{currentstroke}{rgb}{0.281446,0.084320,0.407414}%
\pgfsetstrokecolor{currentstroke}%
\pgfsetdash{}{0pt}%
\pgfpathmoveto{\pgfqpoint{7.438060in}{2.726570in}}%
\pgfpathlineto{\pgfqpoint{7.438613in}{2.726594in}}%
\pgfusepath{stroke}%
\end{pgfscope}%
\begin{pgfscope}%
\pgfpathrectangle{\pgfqpoint{6.720588in}{1.750000in}}{\pgfqpoint{2.279412in}{2.004545in}}%
\pgfusepath{clip}%
\pgfsetbuttcap%
\pgfsetroundjoin%
\pgfsetlinewidth{0.401762pt}%
\definecolor{currentstroke}{rgb}{0.281446,0.084320,0.407414}%
\pgfsetstrokecolor{currentstroke}%
\pgfsetdash{}{0pt}%
\pgfpathmoveto{\pgfqpoint{7.438613in}{2.726594in}}%
\pgfpathlineto{\pgfqpoint{7.438732in}{2.726600in}}%
\pgfusepath{stroke}%
\end{pgfscope}%
\begin{pgfscope}%
\pgfpathrectangle{\pgfqpoint{6.720588in}{1.750000in}}{\pgfqpoint{2.279412in}{2.004545in}}%
\pgfusepath{clip}%
\pgfsetbuttcap%
\pgfsetroundjoin%
\pgfsetlinewidth{0.402311pt}%
\definecolor{currentstroke}{rgb}{0.281446,0.084320,0.407414}%
\pgfsetstrokecolor{currentstroke}%
\pgfsetdash{}{0pt}%
\pgfpathmoveto{\pgfqpoint{7.438732in}{2.726600in}}%
\pgfpathlineto{\pgfqpoint{7.438318in}{2.726584in}}%
\pgfusepath{stroke}%
\end{pgfscope}%
\begin{pgfscope}%
\pgfpathrectangle{\pgfqpoint{6.720588in}{1.750000in}}{\pgfqpoint{2.279412in}{2.004545in}}%
\pgfusepath{clip}%
\pgfsetbuttcap%
\pgfsetroundjoin%
\pgfsetlinewidth{0.400407pt}%
\definecolor{currentstroke}{rgb}{0.281446,0.084320,0.407414}%
\pgfsetstrokecolor{currentstroke}%
\pgfsetdash{}{0pt}%
\pgfpathmoveto{\pgfqpoint{7.438318in}{2.726584in}}%
\pgfpathlineto{\pgfqpoint{7.437592in}{2.726553in}}%
\pgfusepath{stroke}%
\end{pgfscope}%
\begin{pgfscope}%
\pgfpathrectangle{\pgfqpoint{6.720588in}{1.750000in}}{\pgfqpoint{2.279412in}{2.004545in}}%
\pgfusepath{clip}%
\pgfsetbuttcap%
\pgfsetroundjoin%
\pgfsetlinewidth{0.397080pt}%
\definecolor{currentstroke}{rgb}{0.280894,0.078907,0.402329}%
\pgfsetstrokecolor{currentstroke}%
\pgfsetdash{}{0pt}%
\pgfpathmoveto{\pgfqpoint{7.437592in}{2.726553in}}%
\pgfpathlineto{\pgfqpoint{7.437395in}{2.726542in}}%
\pgfusepath{stroke}%
\end{pgfscope}%
\begin{pgfscope}%
\pgfpathrectangle{\pgfqpoint{6.720588in}{1.750000in}}{\pgfqpoint{2.279412in}{2.004545in}}%
\pgfusepath{clip}%
\pgfsetbuttcap%
\pgfsetroundjoin%
\pgfsetlinewidth{0.396182pt}%
\definecolor{currentstroke}{rgb}{0.280894,0.078907,0.402329}%
\pgfsetstrokecolor{currentstroke}%
\pgfsetdash{}{0pt}%
\pgfpathmoveto{\pgfqpoint{7.437395in}{2.726542in}}%
\pgfpathlineto{\pgfqpoint{7.437991in}{2.726566in}}%
\pgfusepath{stroke}%
\end{pgfscope}%
\begin{pgfscope}%
\pgfpathrectangle{\pgfqpoint{6.720588in}{1.750000in}}{\pgfqpoint{2.279412in}{2.004545in}}%
\pgfusepath{clip}%
\pgfsetbuttcap%
\pgfsetroundjoin%
\pgfsetlinewidth{0.398909pt}%
\definecolor{currentstroke}{rgb}{0.281446,0.084320,0.407414}%
\pgfsetstrokecolor{currentstroke}%
\pgfsetdash{}{0pt}%
\pgfpathmoveto{\pgfqpoint{7.437991in}{2.726566in}}%
\pgfpathlineto{\pgfqpoint{7.438610in}{2.726593in}}%
\pgfusepath{stroke}%
\end{pgfscope}%
\begin{pgfscope}%
\pgfpathrectangle{\pgfqpoint{6.720588in}{1.750000in}}{\pgfqpoint{2.279412in}{2.004545in}}%
\pgfusepath{clip}%
\pgfsetbuttcap%
\pgfsetroundjoin%
\pgfsetlinewidth{0.401750pt}%
\definecolor{currentstroke}{rgb}{0.281446,0.084320,0.407414}%
\pgfsetstrokecolor{currentstroke}%
\pgfsetdash{}{0pt}%
\pgfpathmoveto{\pgfqpoint{7.438610in}{2.726593in}}%
\pgfpathlineto{\pgfqpoint{7.438780in}{2.726602in}}%
\pgfusepath{stroke}%
\end{pgfscope}%
\begin{pgfscope}%
\pgfpathrectangle{\pgfqpoint{6.720588in}{1.750000in}}{\pgfqpoint{2.279412in}{2.004545in}}%
\pgfusepath{clip}%
\pgfsetbuttcap%
\pgfsetroundjoin%
\pgfsetlinewidth{0.402531pt}%
\definecolor{currentstroke}{rgb}{0.281446,0.084320,0.407414}%
\pgfsetstrokecolor{currentstroke}%
\pgfsetdash{}{0pt}%
\pgfpathmoveto{\pgfqpoint{7.438780in}{2.726602in}}%
\pgfpathlineto{\pgfqpoint{7.438381in}{2.726587in}}%
\pgfusepath{stroke}%
\end{pgfscope}%
\begin{pgfscope}%
\pgfpathrectangle{\pgfqpoint{6.720588in}{1.750000in}}{\pgfqpoint{2.279412in}{2.004545in}}%
\pgfusepath{clip}%
\pgfsetbuttcap%
\pgfsetroundjoin%
\pgfsetlinewidth{0.400697pt}%
\definecolor{currentstroke}{rgb}{0.281446,0.084320,0.407414}%
\pgfsetstrokecolor{currentstroke}%
\pgfsetdash{}{0pt}%
\pgfpathmoveto{\pgfqpoint{7.438381in}{2.726587in}}%
\pgfpathlineto{\pgfqpoint{7.437590in}{2.726553in}}%
\pgfusepath{stroke}%
\end{pgfscope}%
\begin{pgfscope}%
\pgfpathrectangle{\pgfqpoint{6.720588in}{1.750000in}}{\pgfqpoint{2.279412in}{2.004545in}}%
\pgfusepath{clip}%
\pgfsetbuttcap%
\pgfsetroundjoin%
\pgfsetlinewidth{0.397071pt}%
\definecolor{currentstroke}{rgb}{0.280894,0.078907,0.402329}%
\pgfsetstrokecolor{currentstroke}%
\pgfsetdash{}{0pt}%
\pgfpathmoveto{\pgfqpoint{7.437590in}{2.726553in}}%
\pgfpathlineto{\pgfqpoint{7.437590in}{2.726553in}}%
\pgfusepath{stroke}%
\end{pgfscope}%
\begin{pgfscope}%
\pgfpathrectangle{\pgfqpoint{6.720588in}{1.750000in}}{\pgfqpoint{2.279412in}{2.004545in}}%
\pgfusepath{clip}%
\pgfsetbuttcap%
\pgfsetroundjoin%
\pgfsetlinewidth{0.397071pt}%
\definecolor{currentstroke}{rgb}{0.280894,0.078907,0.402329}%
\pgfsetstrokecolor{currentstroke}%
\pgfsetdash{}{0pt}%
\pgfpathmoveto{\pgfqpoint{7.437590in}{2.726553in}}%
\pgfpathlineto{\pgfqpoint{7.438146in}{2.726575in}}%
\pgfusepath{stroke}%
\end{pgfscope}%
\begin{pgfscope}%
\pgfpathrectangle{\pgfqpoint{6.720588in}{1.750000in}}{\pgfqpoint{2.279412in}{2.004545in}}%
\pgfusepath{clip}%
\pgfsetbuttcap%
\pgfsetroundjoin%
\pgfsetlinewidth{0.399618pt}%
\definecolor{currentstroke}{rgb}{0.281446,0.084320,0.407414}%
\pgfsetstrokecolor{currentstroke}%
\pgfsetdash{}{0pt}%
\pgfpathmoveto{\pgfqpoint{7.438146in}{2.726575in}}%
\pgfpathlineto{\pgfqpoint{7.438626in}{2.726595in}}%
\pgfusepath{stroke}%
\end{pgfscope}%
\begin{pgfscope}%
\pgfpathrectangle{\pgfqpoint{6.720588in}{1.750000in}}{\pgfqpoint{2.279412in}{2.004545in}}%
\pgfusepath{clip}%
\pgfsetbuttcap%
\pgfsetroundjoin%
\pgfsetlinewidth{0.401822pt}%
\definecolor{currentstroke}{rgb}{0.281446,0.084320,0.407414}%
\pgfsetstrokecolor{currentstroke}%
\pgfsetdash{}{0pt}%
\pgfpathmoveto{\pgfqpoint{7.438626in}{2.726595in}}%
\pgfpathlineto{\pgfqpoint{7.438678in}{2.726599in}}%
\pgfusepath{stroke}%
\end{pgfscope}%
\begin{pgfscope}%
\pgfpathrectangle{\pgfqpoint{6.720588in}{1.750000in}}{\pgfqpoint{2.279412in}{2.004545in}}%
\pgfusepath{clip}%
\pgfsetbuttcap%
\pgfsetroundjoin%
\pgfsetlinewidth{0.402061pt}%
\definecolor{currentstroke}{rgb}{0.281446,0.084320,0.407414}%
\pgfsetstrokecolor{currentstroke}%
\pgfsetdash{}{0pt}%
\pgfpathmoveto{\pgfqpoint{7.438678in}{2.726599in}}%
\pgfpathlineto{\pgfqpoint{7.438232in}{2.726581in}}%
\pgfusepath{stroke}%
\end{pgfscope}%
\begin{pgfscope}%
\pgfpathrectangle{\pgfqpoint{6.720588in}{1.750000in}}{\pgfqpoint{2.279412in}{2.004545in}}%
\pgfusepath{clip}%
\pgfsetbuttcap%
\pgfsetroundjoin%
\pgfsetlinewidth{0.400010pt}%
\definecolor{currentstroke}{rgb}{0.281446,0.084320,0.407414}%
\pgfsetstrokecolor{currentstroke}%
\pgfsetdash{}{0pt}%
\pgfpathmoveto{\pgfqpoint{7.438232in}{2.726581in}}%
\pgfpathlineto{\pgfqpoint{7.437575in}{2.726552in}}%
\pgfusepath{stroke}%
\end{pgfscope}%
\begin{pgfscope}%
\pgfpathrectangle{\pgfqpoint{6.720588in}{1.750000in}}{\pgfqpoint{2.279412in}{2.004545in}}%
\pgfusepath{clip}%
\pgfsetbuttcap%
\pgfsetroundjoin%
\pgfsetlinewidth{0.397003pt}%
\definecolor{currentstroke}{rgb}{0.280894,0.078907,0.402329}%
\pgfsetstrokecolor{currentstroke}%
\pgfsetdash{}{0pt}%
\pgfpathmoveto{\pgfqpoint{7.437575in}{2.726552in}}%
\pgfpathlineto{\pgfqpoint{7.437499in}{2.726547in}}%
\pgfusepath{stroke}%
\end{pgfscope}%
\begin{pgfscope}%
\pgfpathrectangle{\pgfqpoint{6.720588in}{1.750000in}}{\pgfqpoint{2.279412in}{2.004545in}}%
\pgfusepath{clip}%
\pgfsetbuttcap%
\pgfsetroundjoin%
\pgfsetlinewidth{0.396656pt}%
\definecolor{currentstroke}{rgb}{0.280894,0.078907,0.402329}%
\pgfsetstrokecolor{currentstroke}%
\pgfsetdash{}{0pt}%
\pgfpathmoveto{\pgfqpoint{7.437499in}{2.726547in}}%
\pgfpathlineto{\pgfqpoint{7.438092in}{2.726571in}}%
\pgfusepath{stroke}%
\end{pgfscope}%
\begin{pgfscope}%
\pgfpathrectangle{\pgfqpoint{6.720588in}{1.750000in}}{\pgfqpoint{2.279412in}{2.004545in}}%
\pgfusepath{clip}%
\pgfsetbuttcap%
\pgfsetroundjoin%
\pgfsetlinewidth{0.399370pt}%
\definecolor{currentstroke}{rgb}{0.281446,0.084320,0.407414}%
\pgfsetstrokecolor{currentstroke}%
\pgfsetdash{}{0pt}%
\pgfpathmoveto{\pgfqpoint{7.438092in}{2.726571in}}%
\pgfpathlineto{\pgfqpoint{7.438634in}{2.726595in}}%
\pgfusepath{stroke}%
\end{pgfscope}%
\begin{pgfscope}%
\pgfpathrectangle{\pgfqpoint{6.720588in}{1.750000in}}{\pgfqpoint{2.279412in}{2.004545in}}%
\pgfusepath{clip}%
\pgfsetbuttcap%
\pgfsetroundjoin%
\pgfsetlinewidth{0.401860pt}%
\definecolor{currentstroke}{rgb}{0.281446,0.084320,0.407414}%
\pgfsetstrokecolor{currentstroke}%
\pgfsetdash{}{0pt}%
\pgfpathmoveto{\pgfqpoint{7.438634in}{2.726595in}}%
\pgfpathlineto{\pgfqpoint{7.438729in}{2.726600in}}%
\pgfusepath{stroke}%
\end{pgfscope}%
\begin{pgfscope}%
\pgfpathrectangle{\pgfqpoint{6.720588in}{1.750000in}}{\pgfqpoint{2.279412in}{2.004545in}}%
\pgfusepath{clip}%
\pgfsetbuttcap%
\pgfsetroundjoin%
\pgfsetlinewidth{0.402297pt}%
\definecolor{currentstroke}{rgb}{0.281446,0.084320,0.407414}%
\pgfsetstrokecolor{currentstroke}%
\pgfsetdash{}{0pt}%
\pgfpathmoveto{\pgfqpoint{7.438729in}{2.726600in}}%
\pgfpathlineto{\pgfqpoint{7.438286in}{2.726583in}}%
\pgfusepath{stroke}%
\end{pgfscope}%
\begin{pgfscope}%
\pgfpathrectangle{\pgfqpoint{6.720588in}{1.750000in}}{\pgfqpoint{2.279412in}{2.004545in}}%
\pgfusepath{clip}%
\pgfsetbuttcap%
\pgfsetroundjoin%
\pgfsetlinewidth{0.400260pt}%
\definecolor{currentstroke}{rgb}{0.281446,0.084320,0.407414}%
\pgfsetstrokecolor{currentstroke}%
\pgfsetdash{}{0pt}%
\pgfpathmoveto{\pgfqpoint{7.438286in}{2.726583in}}%
\pgfpathlineto{\pgfqpoint{7.437555in}{2.726551in}}%
\pgfusepath{stroke}%
\end{pgfscope}%
\begin{pgfscope}%
\pgfpathrectangle{\pgfqpoint{6.720588in}{1.750000in}}{\pgfqpoint{2.279412in}{2.004545in}}%
\pgfusepath{clip}%
\pgfsetbuttcap%
\pgfsetroundjoin%
\pgfsetlinewidth{0.396910pt}%
\definecolor{currentstroke}{rgb}{0.280894,0.078907,0.402329}%
\pgfsetstrokecolor{currentstroke}%
\pgfsetdash{}{0pt}%
\pgfpathmoveto{\pgfqpoint{7.437555in}{2.726551in}}%
\pgfpathlineto{\pgfqpoint{7.437401in}{2.726542in}}%
\pgfusepath{stroke}%
\end{pgfscope}%
\begin{pgfscope}%
\pgfpathrectangle{\pgfqpoint{6.720588in}{1.750000in}}{\pgfqpoint{2.279412in}{2.004545in}}%
\pgfusepath{clip}%
\pgfsetbuttcap%
\pgfsetroundjoin%
\pgfsetlinewidth{0.396210pt}%
\definecolor{currentstroke}{rgb}{0.280894,0.078907,0.402329}%
\pgfsetstrokecolor{currentstroke}%
\pgfsetdash{}{0pt}%
\pgfpathmoveto{\pgfqpoint{7.437401in}{2.726542in}}%
\pgfpathlineto{\pgfqpoint{7.438025in}{2.726568in}}%
\pgfusepath{stroke}%
\end{pgfscope}%
\begin{pgfscope}%
\pgfpathrectangle{\pgfqpoint{6.720588in}{1.750000in}}{\pgfqpoint{2.279412in}{2.004545in}}%
\pgfusepath{clip}%
\pgfsetbuttcap%
\pgfsetroundjoin%
\pgfsetlinewidth{0.399064pt}%
\definecolor{currentstroke}{rgb}{0.281446,0.084320,0.407414}%
\pgfsetstrokecolor{currentstroke}%
\pgfsetdash{}{0pt}%
\pgfpathmoveto{\pgfqpoint{7.438025in}{2.726568in}}%
\pgfpathlineto{\pgfqpoint{7.438634in}{2.726594in}}%
\pgfusepath{stroke}%
\end{pgfscope}%
\begin{pgfscope}%
\pgfpathrectangle{\pgfqpoint{6.720588in}{1.750000in}}{\pgfqpoint{2.279412in}{2.004545in}}%
\pgfusepath{clip}%
\pgfsetbuttcap%
\pgfsetroundjoin%
\pgfsetlinewidth{0.401859pt}%
\definecolor{currentstroke}{rgb}{0.281446,0.084320,0.407414}%
\pgfsetstrokecolor{currentstroke}%
\pgfsetdash{}{0pt}%
\pgfpathmoveto{\pgfqpoint{7.438634in}{2.726594in}}%
\pgfpathlineto{\pgfqpoint{7.438778in}{2.726602in}}%
\pgfusepath{stroke}%
\end{pgfscope}%
\begin{pgfscope}%
\pgfpathrectangle{\pgfqpoint{6.720588in}{1.750000in}}{\pgfqpoint{2.279412in}{2.004545in}}%
\pgfusepath{clip}%
\pgfsetbuttcap%
\pgfsetroundjoin%
\pgfsetlinewidth{0.402524pt}%
\definecolor{currentstroke}{rgb}{0.281446,0.084320,0.407414}%
\pgfsetstrokecolor{currentstroke}%
\pgfsetdash{}{0pt}%
\pgfpathmoveto{\pgfqpoint{7.438778in}{2.726602in}}%
\pgfpathlineto{\pgfqpoint{7.438349in}{2.726585in}}%
\pgfusepath{stroke}%
\end{pgfscope}%
\begin{pgfscope}%
\pgfpathrectangle{\pgfqpoint{6.720588in}{1.750000in}}{\pgfqpoint{2.279412in}{2.004545in}}%
\pgfusepath{clip}%
\pgfsetbuttcap%
\pgfsetroundjoin%
\pgfsetlinewidth{0.400550pt}%
\definecolor{currentstroke}{rgb}{0.281446,0.084320,0.407414}%
\pgfsetstrokecolor{currentstroke}%
\pgfsetdash{}{0pt}%
\pgfpathmoveto{\pgfqpoint{7.438349in}{2.726585in}}%
\pgfpathlineto{\pgfqpoint{7.437548in}{2.726551in}}%
\pgfusepath{stroke}%
\end{pgfscope}%
\begin{pgfscope}%
\pgfpathrectangle{\pgfqpoint{6.720588in}{1.750000in}}{\pgfqpoint{2.279412in}{2.004545in}}%
\pgfusepath{clip}%
\pgfsetbuttcap%
\pgfsetroundjoin%
\pgfsetlinewidth{0.396880pt}%
\definecolor{currentstroke}{rgb}{0.280894,0.078907,0.402329}%
\pgfsetstrokecolor{currentstroke}%
\pgfsetdash{}{0pt}%
\pgfpathmoveto{\pgfqpoint{7.437548in}{2.726551in}}%
\pgfpathlineto{\pgfqpoint{7.437548in}{2.726551in}}%
\pgfusepath{stroke}%
\end{pgfscope}%
\begin{pgfscope}%
\pgfpathrectangle{\pgfqpoint{6.720588in}{1.750000in}}{\pgfqpoint{2.279412in}{2.004545in}}%
\pgfusepath{clip}%
\pgfsetbuttcap%
\pgfsetroundjoin%
\pgfsetlinewidth{0.396880pt}%
\definecolor{currentstroke}{rgb}{0.280894,0.078907,0.402329}%
\pgfsetstrokecolor{currentstroke}%
\pgfsetdash{}{0pt}%
\pgfpathmoveto{\pgfqpoint{7.437548in}{2.726551in}}%
\pgfpathlineto{\pgfqpoint{7.438143in}{2.726575in}}%
\pgfusepath{stroke}%
\end{pgfscope}%
\begin{pgfscope}%
\pgfpathrectangle{\pgfqpoint{6.720588in}{1.750000in}}{\pgfqpoint{2.279412in}{2.004545in}}%
\pgfusepath{clip}%
\pgfsetbuttcap%
\pgfsetroundjoin%
\pgfsetlinewidth{0.399605pt}%
\definecolor{currentstroke}{rgb}{0.281446,0.084320,0.407414}%
\pgfsetstrokecolor{currentstroke}%
\pgfsetdash{}{0pt}%
\pgfpathmoveto{\pgfqpoint{7.438143in}{2.726575in}}%
\pgfpathlineto{\pgfqpoint{7.438648in}{2.726596in}}%
\pgfusepath{stroke}%
\end{pgfscope}%
\begin{pgfscope}%
\pgfpathrectangle{\pgfqpoint{6.720588in}{1.750000in}}{\pgfqpoint{2.279412in}{2.004545in}}%
\pgfusepath{clip}%
\pgfsetbuttcap%
\pgfsetroundjoin%
\pgfsetlinewidth{0.401924pt}%
\definecolor{currentstroke}{rgb}{0.281446,0.084320,0.407414}%
\pgfsetstrokecolor{currentstroke}%
\pgfsetdash{}{0pt}%
\pgfpathmoveto{\pgfqpoint{7.438648in}{2.726596in}}%
\pgfpathlineto{\pgfqpoint{7.438702in}{2.726600in}}%
\pgfusepath{stroke}%
\end{pgfscope}%
\begin{pgfscope}%
\pgfpathrectangle{\pgfqpoint{6.720588in}{1.750000in}}{\pgfqpoint{2.279412in}{2.004545in}}%
\pgfusepath{clip}%
\pgfsetbuttcap%
\pgfsetroundjoin%
\pgfsetlinewidth{0.402175pt}%
\definecolor{currentstroke}{rgb}{0.281446,0.084320,0.407414}%
\pgfsetstrokecolor{currentstroke}%
\pgfsetdash{}{0pt}%
\pgfpathmoveto{\pgfqpoint{7.438702in}{2.726600in}}%
\pgfpathlineto{\pgfqpoint{7.438233in}{2.726581in}}%
\pgfusepath{stroke}%
\end{pgfscope}%
\begin{pgfscope}%
\pgfpathrectangle{\pgfqpoint{6.720588in}{1.750000in}}{\pgfqpoint{2.279412in}{2.004545in}}%
\pgfusepath{clip}%
\pgfsetbuttcap%
\pgfsetroundjoin%
\pgfsetlinewidth{0.400017pt}%
\definecolor{currentstroke}{rgb}{0.281446,0.084320,0.407414}%
\pgfsetstrokecolor{currentstroke}%
\pgfsetdash{}{0pt}%
\pgfpathmoveto{\pgfqpoint{7.438233in}{2.726581in}}%
\pgfpathlineto{\pgfqpoint{7.437532in}{2.726550in}}%
\pgfusepath{stroke}%
\end{pgfscope}%
\begin{pgfscope}%
\pgfpathrectangle{\pgfqpoint{6.720588in}{1.750000in}}{\pgfqpoint{2.279412in}{2.004545in}}%
\pgfusepath{clip}%
\pgfsetbuttcap%
\pgfsetroundjoin%
\pgfsetlinewidth{0.396808pt}%
\definecolor{currentstroke}{rgb}{0.280894,0.078907,0.402329}%
\pgfsetstrokecolor{currentstroke}%
\pgfsetdash{}{0pt}%
\pgfpathmoveto{\pgfqpoint{7.437532in}{2.726550in}}%
\pgfpathlineto{\pgfqpoint{7.437452in}{2.726545in}}%
\pgfusepath{stroke}%
\end{pgfscope}%
\begin{pgfscope}%
\pgfpathrectangle{\pgfqpoint{6.720588in}{1.750000in}}{\pgfqpoint{2.279412in}{2.004545in}}%
\pgfusepath{clip}%
\pgfsetbuttcap%
\pgfsetroundjoin%
\pgfsetlinewidth{0.396441pt}%
\definecolor{currentstroke}{rgb}{0.280894,0.078907,0.402329}%
\pgfsetstrokecolor{currentstroke}%
\pgfsetdash{}{0pt}%
\pgfpathmoveto{\pgfqpoint{7.437452in}{2.726545in}}%
\pgfpathlineto{\pgfqpoint{7.438084in}{2.726570in}}%
\pgfusepath{stroke}%
\end{pgfscope}%
\begin{pgfscope}%
\pgfpathrectangle{\pgfqpoint{6.720588in}{1.750000in}}{\pgfqpoint{2.279412in}{2.004545in}}%
\pgfusepath{clip}%
\pgfsetbuttcap%
\pgfsetroundjoin%
\pgfsetlinewidth{0.399335pt}%
\definecolor{currentstroke}{rgb}{0.281446,0.084320,0.407414}%
\pgfsetstrokecolor{currentstroke}%
\pgfsetdash{}{0pt}%
\pgfpathmoveto{\pgfqpoint{7.438084in}{2.726570in}}%
\pgfpathlineto{\pgfqpoint{7.438654in}{2.726595in}}%
\pgfusepath{stroke}%
\end{pgfscope}%
\begin{pgfscope}%
\pgfpathrectangle{\pgfqpoint{6.720588in}{1.750000in}}{\pgfqpoint{2.279412in}{2.004545in}}%
\pgfusepath{clip}%
\pgfsetbuttcap%
\pgfsetroundjoin%
\pgfsetlinewidth{0.401951pt}%
\definecolor{currentstroke}{rgb}{0.281446,0.084320,0.407414}%
\pgfsetstrokecolor{currentstroke}%
\pgfsetdash{}{0pt}%
\pgfpathmoveto{\pgfqpoint{7.438654in}{2.726595in}}%
\pgfpathlineto{\pgfqpoint{7.438754in}{2.726601in}}%
\pgfusepath{stroke}%
\end{pgfscope}%
\begin{pgfscope}%
\pgfpathrectangle{\pgfqpoint{6.720588in}{1.750000in}}{\pgfqpoint{2.279412in}{2.004545in}}%
\pgfusepath{clip}%
\pgfsetbuttcap%
\pgfsetroundjoin%
\pgfsetlinewidth{0.402415pt}%
\definecolor{currentstroke}{rgb}{0.281446,0.084320,0.407414}%
\pgfsetstrokecolor{currentstroke}%
\pgfsetdash{}{0pt}%
\pgfpathmoveto{\pgfqpoint{7.438754in}{2.726601in}}%
\pgfpathlineto{\pgfqpoint{7.438292in}{2.726583in}}%
\pgfusepath{stroke}%
\end{pgfscope}%
\begin{pgfscope}%
\pgfpathrectangle{\pgfqpoint{6.720588in}{1.750000in}}{\pgfqpoint{2.279412in}{2.004545in}}%
\pgfusepath{clip}%
\pgfsetbuttcap%
\pgfsetroundjoin%
\pgfsetlinewidth{0.400288pt}%
\definecolor{currentstroke}{rgb}{0.281446,0.084320,0.407414}%
\pgfsetstrokecolor{currentstroke}%
\pgfsetdash{}{0pt}%
\pgfpathmoveto{\pgfqpoint{7.438292in}{2.726583in}}%
\pgfpathlineto{\pgfqpoint{7.437515in}{2.726550in}}%
\pgfusepath{stroke}%
\end{pgfscope}%
\begin{pgfscope}%
\pgfpathrectangle{\pgfqpoint{6.720588in}{1.750000in}}{\pgfqpoint{2.279412in}{2.004545in}}%
\pgfusepath{clip}%
\pgfsetbuttcap%
\pgfsetroundjoin%
\pgfsetlinewidth{0.396730pt}%
\definecolor{currentstroke}{rgb}{0.280894,0.078907,0.402329}%
\pgfsetstrokecolor{currentstroke}%
\pgfsetdash{}{0pt}%
\pgfpathmoveto{\pgfqpoint{7.437515in}{2.726550in}}%
\pgfpathlineto{\pgfqpoint{7.437349in}{2.726540in}}%
\pgfusepath{stroke}%
\end{pgfscope}%
\begin{pgfscope}%
\pgfpathrectangle{\pgfqpoint{6.720588in}{1.750000in}}{\pgfqpoint{2.279412in}{2.004545in}}%
\pgfusepath{clip}%
\pgfsetbuttcap%
\pgfsetroundjoin%
\pgfsetlinewidth{0.395975pt}%
\definecolor{currentstroke}{rgb}{0.280894,0.078907,0.402329}%
\pgfsetstrokecolor{currentstroke}%
\pgfsetdash{}{0pt}%
\pgfpathmoveto{\pgfqpoint{7.437349in}{2.726540in}}%
\pgfpathlineto{\pgfqpoint{7.438012in}{2.726567in}}%
\pgfusepath{stroke}%
\end{pgfscope}%
\begin{pgfscope}%
\pgfpathrectangle{\pgfqpoint{6.720588in}{1.750000in}}{\pgfqpoint{2.279412in}{2.004545in}}%
\pgfusepath{clip}%
\pgfsetbuttcap%
\pgfsetroundjoin%
\pgfsetlinewidth{0.399004pt}%
\definecolor{currentstroke}{rgb}{0.281446,0.084320,0.407414}%
\pgfsetstrokecolor{currentstroke}%
\pgfsetdash{}{0pt}%
\pgfpathmoveto{\pgfqpoint{7.438012in}{2.726567in}}%
\pgfpathlineto{\pgfqpoint{7.438650in}{2.726595in}}%
\pgfusepath{stroke}%
\end{pgfscope}%
\begin{pgfscope}%
\pgfpathrectangle{\pgfqpoint{6.720588in}{1.750000in}}{\pgfqpoint{2.279412in}{2.004545in}}%
\pgfusepath{clip}%
\pgfsetbuttcap%
\pgfsetroundjoin%
\pgfsetlinewidth{0.401936pt}%
\definecolor{currentstroke}{rgb}{0.281446,0.084320,0.407414}%
\pgfsetstrokecolor{currentstroke}%
\pgfsetdash{}{0pt}%
\pgfpathmoveto{\pgfqpoint{7.438650in}{2.726595in}}%
\pgfpathlineto{\pgfqpoint{7.438804in}{2.726603in}}%
\pgfusepath{stroke}%
\end{pgfscope}%
\begin{pgfscope}%
\pgfpathrectangle{\pgfqpoint{6.720588in}{1.750000in}}{\pgfqpoint{2.279412in}{2.004545in}}%
\pgfusepath{clip}%
\pgfsetbuttcap%
\pgfsetroundjoin%
\pgfsetlinewidth{0.402642pt}%
\definecolor{currentstroke}{rgb}{0.281446,0.084320,0.407414}%
\pgfsetstrokecolor{currentstroke}%
\pgfsetdash{}{0pt}%
\pgfpathmoveto{\pgfqpoint{7.438804in}{2.726603in}}%
\pgfpathlineto{\pgfqpoint{7.438360in}{2.726586in}}%
\pgfusepath{stroke}%
\end{pgfscope}%
\begin{pgfscope}%
\pgfpathrectangle{\pgfqpoint{6.720588in}{1.750000in}}{\pgfqpoint{2.279412in}{2.004545in}}%
\pgfusepath{clip}%
\pgfsetbuttcap%
\pgfsetroundjoin%
\pgfsetlinewidth{0.400601pt}%
\definecolor{currentstroke}{rgb}{0.281446,0.084320,0.407414}%
\pgfsetstrokecolor{currentstroke}%
\pgfsetdash{}{0pt}%
\pgfpathmoveto{\pgfqpoint{7.438360in}{2.726586in}}%
\pgfpathlineto{\pgfqpoint{7.437514in}{2.726550in}}%
\pgfusepath{stroke}%
\end{pgfscope}%
\begin{pgfscope}%
\pgfpathrectangle{\pgfqpoint{6.720588in}{1.750000in}}{\pgfqpoint{2.279412in}{2.004545in}}%
\pgfusepath{clip}%
\pgfsetbuttcap%
\pgfsetroundjoin%
\pgfsetlinewidth{0.396724pt}%
\definecolor{currentstroke}{rgb}{0.280894,0.078907,0.402329}%
\pgfsetstrokecolor{currentstroke}%
\pgfsetdash{}{0pt}%
\pgfpathmoveto{\pgfqpoint{7.437514in}{2.726550in}}%
\pgfpathlineto{\pgfqpoint{7.437514in}{2.726550in}}%
\pgfusepath{stroke}%
\end{pgfscope}%
\begin{pgfscope}%
\pgfpathrectangle{\pgfqpoint{6.720588in}{1.750000in}}{\pgfqpoint{2.279412in}{2.004545in}}%
\pgfusepath{clip}%
\pgfsetbuttcap%
\pgfsetroundjoin%
\pgfsetlinewidth{0.396724pt}%
\definecolor{currentstroke}{rgb}{0.280894,0.078907,0.402329}%
\pgfsetstrokecolor{currentstroke}%
\pgfsetdash{}{0pt}%
\pgfpathmoveto{\pgfqpoint{7.437514in}{2.726550in}}%
\pgfpathlineto{\pgfqpoint{7.438139in}{2.726574in}}%
\pgfusepath{stroke}%
\end{pgfscope}%
\begin{pgfscope}%
\pgfpathrectangle{\pgfqpoint{6.720588in}{1.750000in}}{\pgfqpoint{2.279412in}{2.004545in}}%
\pgfusepath{clip}%
\pgfsetbuttcap%
\pgfsetroundjoin%
\pgfsetlinewidth{0.399583pt}%
\definecolor{currentstroke}{rgb}{0.281446,0.084320,0.407414}%
\pgfsetstrokecolor{currentstroke}%
\pgfsetdash{}{0pt}%
\pgfpathmoveto{\pgfqpoint{7.438139in}{2.726574in}}%
\pgfpathlineto{\pgfqpoint{7.438663in}{2.726597in}}%
\pgfusepath{stroke}%
\end{pgfscope}%
\begin{pgfscope}%
\pgfpathrectangle{\pgfqpoint{6.720588in}{1.750000in}}{\pgfqpoint{2.279412in}{2.004545in}}%
\pgfusepath{clip}%
\pgfsetbuttcap%
\pgfsetroundjoin%
\pgfsetlinewidth{0.401995pt}%
\definecolor{currentstroke}{rgb}{0.281446,0.084320,0.407414}%
\pgfsetstrokecolor{currentstroke}%
\pgfsetdash{}{0pt}%
\pgfpathmoveto{\pgfqpoint{7.438663in}{2.726597in}}%
\pgfpathlineto{\pgfqpoint{7.438722in}{2.726601in}}%
\pgfusepath{stroke}%
\end{pgfscope}%
\begin{pgfscope}%
\pgfpathrectangle{\pgfqpoint{6.720588in}{1.750000in}}{\pgfqpoint{2.279412in}{2.004545in}}%
\pgfusepath{clip}%
\pgfsetbuttcap%
\pgfsetroundjoin%
\pgfsetlinewidth{0.402265pt}%
\definecolor{currentstroke}{rgb}{0.281446,0.084320,0.407414}%
\pgfsetstrokecolor{currentstroke}%
\pgfsetdash{}{0pt}%
\pgfpathmoveto{\pgfqpoint{7.438722in}{2.726601in}}%
\pgfpathlineto{\pgfqpoint{7.438237in}{2.726581in}}%
\pgfusepath{stroke}%
\end{pgfscope}%
\begin{pgfscope}%
\pgfpathrectangle{\pgfqpoint{6.720588in}{1.750000in}}{\pgfqpoint{2.279412in}{2.004545in}}%
\pgfusepath{clip}%
\pgfsetbuttcap%
\pgfsetroundjoin%
\pgfsetlinewidth{0.400034pt}%
\definecolor{currentstroke}{rgb}{0.281446,0.084320,0.407414}%
\pgfsetstrokecolor{currentstroke}%
\pgfsetdash{}{0pt}%
\pgfpathmoveto{\pgfqpoint{7.438237in}{2.726581in}}%
\pgfpathlineto{\pgfqpoint{7.437501in}{2.726549in}}%
\pgfusepath{stroke}%
\end{pgfscope}%
\begin{pgfscope}%
\pgfpathrectangle{\pgfqpoint{6.720588in}{1.750000in}}{\pgfqpoint{2.279412in}{2.004545in}}%
\pgfusepath{clip}%
\pgfsetbuttcap%
\pgfsetroundjoin%
\pgfsetlinewidth{0.396667pt}%
\definecolor{currentstroke}{rgb}{0.280894,0.078907,0.402329}%
\pgfsetstrokecolor{currentstroke}%
\pgfsetdash{}{0pt}%
\pgfpathmoveto{\pgfqpoint{7.437501in}{2.726549in}}%
\pgfpathlineto{\pgfqpoint{7.437414in}{2.726543in}}%
\pgfusepath{stroke}%
\end{pgfscope}%
\begin{pgfscope}%
\pgfpathrectangle{\pgfqpoint{6.720588in}{1.750000in}}{\pgfqpoint{2.279412in}{2.004545in}}%
\pgfusepath{clip}%
\pgfsetbuttcap%
\pgfsetroundjoin%
\pgfsetlinewidth{0.396269pt}%
\definecolor{currentstroke}{rgb}{0.280894,0.078907,0.402329}%
\pgfsetstrokecolor{currentstroke}%
\pgfsetdash{}{0pt}%
\pgfpathmoveto{\pgfqpoint{7.437414in}{2.726543in}}%
\pgfpathlineto{\pgfqpoint{7.438075in}{2.726570in}}%
\pgfusepath{stroke}%
\end{pgfscope}%
\begin{pgfscope}%
\pgfpathrectangle{\pgfqpoint{6.720588in}{1.750000in}}{\pgfqpoint{2.279412in}{2.004545in}}%
\pgfusepath{clip}%
\pgfsetbuttcap%
\pgfsetroundjoin%
\pgfsetlinewidth{0.399294pt}%
\definecolor{currentstroke}{rgb}{0.281446,0.084320,0.407414}%
\pgfsetstrokecolor{currentstroke}%
\pgfsetdash{}{0pt}%
\pgfpathmoveto{\pgfqpoint{7.438075in}{2.726570in}}%
\pgfpathlineto{\pgfqpoint{7.438667in}{2.726596in}}%
\pgfusepath{stroke}%
\end{pgfscope}%
\begin{pgfscope}%
\pgfpathrectangle{\pgfqpoint{6.720588in}{1.750000in}}{\pgfqpoint{2.279412in}{2.004545in}}%
\pgfusepath{clip}%
\pgfsetbuttcap%
\pgfsetroundjoin%
\pgfsetlinewidth{0.402012pt}%
\definecolor{currentstroke}{rgb}{0.281446,0.084320,0.407414}%
\pgfsetstrokecolor{currentstroke}%
\pgfsetdash{}{0pt}%
\pgfpathmoveto{\pgfqpoint{7.438667in}{2.726596in}}%
\pgfpathlineto{\pgfqpoint{7.438774in}{2.726602in}}%
\pgfusepath{stroke}%
\end{pgfscope}%
\begin{pgfscope}%
\pgfpathrectangle{\pgfqpoint{6.720588in}{1.750000in}}{\pgfqpoint{2.279412in}{2.004545in}}%
\pgfusepath{clip}%
\pgfsetbuttcap%
\pgfsetroundjoin%
\pgfsetlinewidth{0.402505pt}%
\definecolor{currentstroke}{rgb}{0.281446,0.084320,0.407414}%
\pgfsetstrokecolor{currentstroke}%
\pgfsetdash{}{0pt}%
\pgfpathmoveto{\pgfqpoint{7.438774in}{2.726602in}}%
\pgfpathlineto{\pgfqpoint{7.438300in}{2.726583in}}%
\pgfusepath{stroke}%
\end{pgfscope}%
\begin{pgfscope}%
\pgfpathrectangle{\pgfqpoint{6.720588in}{1.750000in}}{\pgfqpoint{2.279412in}{2.004545in}}%
\pgfusepath{clip}%
\pgfsetbuttcap%
\pgfsetroundjoin%
\pgfsetlinewidth{0.400323pt}%
\definecolor{currentstroke}{rgb}{0.281446,0.084320,0.407414}%
\pgfsetstrokecolor{currentstroke}%
\pgfsetdash{}{0pt}%
\pgfpathmoveto{\pgfqpoint{7.438300in}{2.726583in}}%
\pgfpathlineto{\pgfqpoint{7.437488in}{2.726548in}}%
\pgfusepath{stroke}%
\end{pgfscope}%
\begin{pgfscope}%
\pgfpathrectangle{\pgfqpoint{6.720588in}{1.750000in}}{\pgfqpoint{2.279412in}{2.004545in}}%
\pgfusepath{clip}%
\pgfsetbuttcap%
\pgfsetroundjoin%
\pgfsetlinewidth{0.396605pt}%
\definecolor{currentstroke}{rgb}{0.280894,0.078907,0.402329}%
\pgfsetstrokecolor{currentstroke}%
\pgfsetdash{}{0pt}%
\pgfpathmoveto{\pgfqpoint{7.437488in}{2.726548in}}%
\pgfpathlineto{\pgfqpoint{7.437488in}{2.726548in}}%
\pgfusepath{stroke}%
\end{pgfscope}%
\begin{pgfscope}%
\pgfpathrectangle{\pgfqpoint{6.720588in}{1.750000in}}{\pgfqpoint{2.279412in}{2.004545in}}%
\pgfusepath{clip}%
\pgfsetbuttcap%
\pgfsetroundjoin%
\pgfsetlinewidth{0.396605pt}%
\definecolor{currentstroke}{rgb}{0.280894,0.078907,0.402329}%
\pgfsetstrokecolor{currentstroke}%
\pgfsetdash{}{0pt}%
\pgfpathmoveto{\pgfqpoint{7.437488in}{2.726548in}}%
\pgfpathlineto{\pgfqpoint{7.438139in}{2.726574in}}%
\pgfusepath{stroke}%
\end{pgfscope}%
\begin{pgfscope}%
\pgfpathrectangle{\pgfqpoint{6.720588in}{1.750000in}}{\pgfqpoint{2.279412in}{2.004545in}}%
\pgfusepath{clip}%
\pgfsetbuttcap%
\pgfsetroundjoin%
\pgfsetlinewidth{0.399584pt}%
\definecolor{currentstroke}{rgb}{0.281446,0.084320,0.407414}%
\pgfsetstrokecolor{currentstroke}%
\pgfsetdash{}{0pt}%
\pgfpathmoveto{\pgfqpoint{7.438139in}{2.726574in}}%
\pgfpathlineto{\pgfqpoint{7.438678in}{2.726597in}}%
\pgfusepath{stroke}%
\end{pgfscope}%
\begin{pgfscope}%
\pgfpathrectangle{\pgfqpoint{6.720588in}{1.750000in}}{\pgfqpoint{2.279412in}{2.004545in}}%
\pgfusepath{clip}%
\pgfsetbuttcap%
\pgfsetroundjoin%
\pgfsetlinewidth{0.402063pt}%
\definecolor{currentstroke}{rgb}{0.281446,0.084320,0.407414}%
\pgfsetstrokecolor{currentstroke}%
\pgfsetdash{}{0pt}%
\pgfpathmoveto{\pgfqpoint{7.438678in}{2.726597in}}%
\pgfpathlineto{\pgfqpoint{7.438737in}{2.726601in}}%
\pgfusepath{stroke}%
\end{pgfscope}%
\begin{pgfscope}%
\pgfpathrectangle{\pgfqpoint{6.720588in}{1.750000in}}{\pgfqpoint{2.279412in}{2.004545in}}%
\pgfusepath{clip}%
\pgfsetbuttcap%
\pgfsetroundjoin%
\pgfsetlinewidth{0.402333pt}%
\definecolor{currentstroke}{rgb}{0.281446,0.084320,0.407414}%
\pgfsetstrokecolor{currentstroke}%
\pgfsetdash{}{0pt}%
\pgfpathmoveto{\pgfqpoint{7.438737in}{2.726601in}}%
\pgfpathlineto{\pgfqpoint{7.438235in}{2.726581in}}%
\pgfusepath{stroke}%
\end{pgfscope}%
\begin{pgfscope}%
\pgfpathrectangle{\pgfqpoint{6.720588in}{1.750000in}}{\pgfqpoint{2.279412in}{2.004545in}}%
\pgfusepath{clip}%
\pgfsetbuttcap%
\pgfsetroundjoin%
\pgfsetlinewidth{0.400027pt}%
\definecolor{currentstroke}{rgb}{0.281446,0.084320,0.407414}%
\pgfsetstrokecolor{currentstroke}%
\pgfsetdash{}{0pt}%
\pgfpathmoveto{\pgfqpoint{7.438235in}{2.726581in}}%
\pgfpathlineto{\pgfqpoint{7.437472in}{2.726548in}}%
\pgfusepath{stroke}%
\end{pgfscope}%
\begin{pgfscope}%
\pgfpathrectangle{\pgfqpoint{6.720588in}{1.750000in}}{\pgfqpoint{2.279412in}{2.004545in}}%
\pgfusepath{clip}%
\pgfsetbuttcap%
\pgfsetroundjoin%
\pgfsetlinewidth{0.396531pt}%
\definecolor{currentstroke}{rgb}{0.280894,0.078907,0.402329}%
\pgfsetstrokecolor{currentstroke}%
\pgfsetdash{}{0pt}%
\pgfpathmoveto{\pgfqpoint{7.437472in}{2.726548in}}%
\pgfpathlineto{\pgfqpoint{7.437385in}{2.726542in}}%
\pgfusepath{stroke}%
\end{pgfscope}%
\begin{pgfscope}%
\pgfpathrectangle{\pgfqpoint{6.720588in}{1.750000in}}{\pgfqpoint{2.279412in}{2.004545in}}%
\pgfusepath{clip}%
\pgfsetbuttcap%
\pgfsetroundjoin%
\pgfsetlinewidth{0.396135pt}%
\definecolor{currentstroke}{rgb}{0.280894,0.078907,0.402329}%
\pgfsetstrokecolor{currentstroke}%
\pgfsetdash{}{0pt}%
\pgfpathmoveto{\pgfqpoint{7.437385in}{2.726542in}}%
\pgfpathlineto{\pgfqpoint{7.438073in}{2.726570in}}%
\pgfusepath{stroke}%
\end{pgfscope}%
\begin{pgfscope}%
\pgfpathrectangle{\pgfqpoint{6.720588in}{1.750000in}}{\pgfqpoint{2.279412in}{2.004545in}}%
\pgfusepath{clip}%
\pgfsetbuttcap%
\pgfsetroundjoin%
\pgfsetlinewidth{0.399281pt}%
\definecolor{currentstroke}{rgb}{0.281446,0.084320,0.407414}%
\pgfsetstrokecolor{currentstroke}%
\pgfsetdash{}{0pt}%
\pgfpathmoveto{\pgfqpoint{7.438073in}{2.726570in}}%
\pgfpathlineto{\pgfqpoint{7.438680in}{2.726596in}}%
\pgfusepath{stroke}%
\end{pgfscope}%
\begin{pgfscope}%
\pgfpathrectangle{\pgfqpoint{6.720588in}{1.750000in}}{\pgfqpoint{2.279412in}{2.004545in}}%
\pgfusepath{clip}%
\pgfsetbuttcap%
\pgfsetroundjoin%
\pgfsetlinewidth{0.402072pt}%
\definecolor{currentstroke}{rgb}{0.281446,0.084320,0.407414}%
\pgfsetstrokecolor{currentstroke}%
\pgfsetdash{}{0pt}%
\pgfpathmoveto{\pgfqpoint{7.438680in}{2.726596in}}%
\pgfpathlineto{\pgfqpoint{7.438789in}{2.726603in}}%
\pgfusepath{stroke}%
\end{pgfscope}%
\begin{pgfscope}%
\pgfpathrectangle{\pgfqpoint{6.720588in}{1.750000in}}{\pgfqpoint{2.279412in}{2.004545in}}%
\pgfusepath{clip}%
\pgfsetbuttcap%
\pgfsetroundjoin%
\pgfsetlinewidth{0.402574pt}%
\definecolor{currentstroke}{rgb}{0.281446,0.084320,0.407414}%
\pgfsetstrokecolor{currentstroke}%
\pgfsetdash{}{0pt}%
\pgfpathmoveto{\pgfqpoint{7.438789in}{2.726603in}}%
\pgfpathlineto{\pgfqpoint{7.438302in}{2.726584in}}%
\pgfusepath{stroke}%
\end{pgfscope}%
\begin{pgfscope}%
\pgfpathrectangle{\pgfqpoint{6.720588in}{1.750000in}}{\pgfqpoint{2.279412in}{2.004545in}}%
\pgfusepath{clip}%
\pgfsetbuttcap%
\pgfsetroundjoin%
\pgfsetlinewidth{0.400332pt}%
\definecolor{currentstroke}{rgb}{0.281446,0.084320,0.407414}%
\pgfsetstrokecolor{currentstroke}%
\pgfsetdash{}{0pt}%
\pgfpathmoveto{\pgfqpoint{7.438302in}{2.726584in}}%
\pgfpathlineto{\pgfqpoint{7.437461in}{2.726547in}}%
\pgfusepath{stroke}%
\end{pgfscope}%
\begin{pgfscope}%
\pgfpathrectangle{\pgfqpoint{6.720588in}{1.750000in}}{\pgfqpoint{2.279412in}{2.004545in}}%
\pgfusepath{clip}%
\pgfsetbuttcap%
\pgfsetroundjoin%
\pgfsetlinewidth{0.396481pt}%
\definecolor{currentstroke}{rgb}{0.280894,0.078907,0.402329}%
\pgfsetstrokecolor{currentstroke}%
\pgfsetdash{}{0pt}%
\pgfpathmoveto{\pgfqpoint{7.437461in}{2.726547in}}%
\pgfpathlineto{\pgfqpoint{7.437461in}{2.726547in}}%
\pgfusepath{stroke}%
\end{pgfscope}%
\begin{pgfscope}%
\pgfpathrectangle{\pgfqpoint{6.720588in}{1.750000in}}{\pgfqpoint{2.279412in}{2.004545in}}%
\pgfusepath{clip}%
\pgfsetbuttcap%
\pgfsetroundjoin%
\pgfsetlinewidth{0.396481pt}%
\definecolor{currentstroke}{rgb}{0.280894,0.078907,0.402329}%
\pgfsetstrokecolor{currentstroke}%
\pgfsetdash{}{0pt}%
\pgfpathmoveto{\pgfqpoint{7.437461in}{2.726547in}}%
\pgfpathlineto{\pgfqpoint{7.438135in}{2.726574in}}%
\pgfusepath{stroke}%
\end{pgfscope}%
\begin{pgfscope}%
\pgfpathrectangle{\pgfqpoint{6.720588in}{1.750000in}}{\pgfqpoint{2.279412in}{2.004545in}}%
\pgfusepath{clip}%
\pgfsetbuttcap%
\pgfsetroundjoin%
\pgfsetlinewidth{0.399568pt}%
\definecolor{currentstroke}{rgb}{0.281446,0.084320,0.407414}%
\pgfsetstrokecolor{currentstroke}%
\pgfsetdash{}{0pt}%
\pgfpathmoveto{\pgfqpoint{7.438135in}{2.726574in}}%
\pgfpathlineto{\pgfqpoint{7.438690in}{2.726598in}}%
\pgfusepath{stroke}%
\end{pgfscope}%
\begin{pgfscope}%
\pgfpathrectangle{\pgfqpoint{6.720588in}{1.750000in}}{\pgfqpoint{2.279412in}{2.004545in}}%
\pgfusepath{clip}%
\pgfsetbuttcap%
\pgfsetroundjoin%
\pgfsetlinewidth{0.402118pt}%
\definecolor{currentstroke}{rgb}{0.281446,0.084320,0.407414}%
\pgfsetstrokecolor{currentstroke}%
\pgfsetdash{}{0pt}%
\pgfpathmoveto{\pgfqpoint{7.438690in}{2.726598in}}%
\pgfpathlineto{\pgfqpoint{7.438751in}{2.726602in}}%
\pgfusepath{stroke}%
\end{pgfscope}%
\begin{pgfscope}%
\pgfpathrectangle{\pgfqpoint{6.720588in}{1.750000in}}{\pgfqpoint{2.279412in}{2.004545in}}%
\pgfusepath{clip}%
\pgfsetbuttcap%
\pgfsetroundjoin%
\pgfsetlinewidth{0.402400pt}%
\definecolor{currentstroke}{rgb}{0.281446,0.084320,0.407414}%
\pgfsetstrokecolor{currentstroke}%
\pgfsetdash{}{0pt}%
\pgfpathmoveto{\pgfqpoint{7.438751in}{2.726602in}}%
\pgfpathlineto{\pgfqpoint{7.438238in}{2.726581in}}%
\pgfusepath{stroke}%
\end{pgfscope}%
\begin{pgfscope}%
\pgfpathrectangle{\pgfqpoint{6.720588in}{1.750000in}}{\pgfqpoint{2.279412in}{2.004545in}}%
\pgfusepath{clip}%
\pgfsetbuttcap%
\pgfsetroundjoin%
\pgfsetlinewidth{0.400040pt}%
\definecolor{currentstroke}{rgb}{0.281446,0.084320,0.407414}%
\pgfsetstrokecolor{currentstroke}%
\pgfsetdash{}{0pt}%
\pgfpathmoveto{\pgfqpoint{7.438238in}{2.726581in}}%
\pgfpathlineto{\pgfqpoint{7.437447in}{2.726547in}}%
\pgfusepath{stroke}%
\end{pgfscope}%
\begin{pgfscope}%
\pgfpathrectangle{\pgfqpoint{6.720588in}{1.750000in}}{\pgfqpoint{2.279412in}{2.004545in}}%
\pgfusepath{clip}%
\pgfsetbuttcap%
\pgfsetroundjoin%
\pgfsetlinewidth{0.396419pt}%
\definecolor{currentstroke}{rgb}{0.280894,0.078907,0.402329}%
\pgfsetstrokecolor{currentstroke}%
\pgfsetdash{}{0pt}%
\pgfpathmoveto{\pgfqpoint{7.437447in}{2.726547in}}%
\pgfpathlineto{\pgfqpoint{7.437355in}{2.726540in}}%
\pgfusepath{stroke}%
\end{pgfscope}%
\begin{pgfscope}%
\pgfpathrectangle{\pgfqpoint{6.720588in}{1.750000in}}{\pgfqpoint{2.279412in}{2.004545in}}%
\pgfusepath{clip}%
\pgfsetbuttcap%
\pgfsetroundjoin%
\pgfsetlinewidth{0.395999pt}%
\definecolor{currentstroke}{rgb}{0.280894,0.078907,0.402329}%
\pgfsetstrokecolor{currentstroke}%
\pgfsetdash{}{0pt}%
\pgfpathmoveto{\pgfqpoint{7.437355in}{2.726540in}}%
\pgfpathlineto{\pgfqpoint{7.438066in}{2.726569in}}%
\pgfusepath{stroke}%
\end{pgfscope}%
\begin{pgfscope}%
\pgfpathrectangle{\pgfqpoint{6.720588in}{1.750000in}}{\pgfqpoint{2.279412in}{2.004545in}}%
\pgfusepath{clip}%
\pgfsetbuttcap%
\pgfsetroundjoin%
\pgfsetlinewidth{0.399249pt}%
\definecolor{currentstroke}{rgb}{0.281446,0.084320,0.407414}%
\pgfsetstrokecolor{currentstroke}%
\pgfsetdash{}{0pt}%
\pgfpathmoveto{\pgfqpoint{7.438066in}{2.726569in}}%
\pgfpathlineto{\pgfqpoint{7.438690in}{2.726596in}}%
\pgfusepath{stroke}%
\end{pgfscope}%
\begin{pgfscope}%
\pgfpathrectangle{\pgfqpoint{6.720588in}{1.750000in}}{\pgfqpoint{2.279412in}{2.004545in}}%
\pgfusepath{clip}%
\pgfsetbuttcap%
\pgfsetroundjoin%
\pgfsetlinewidth{0.402117pt}%
\definecolor{currentstroke}{rgb}{0.281446,0.084320,0.407414}%
\pgfsetstrokecolor{currentstroke}%
\pgfsetdash{}{0pt}%
\pgfpathmoveto{\pgfqpoint{7.438690in}{2.726596in}}%
\pgfpathlineto{\pgfqpoint{7.438804in}{2.726603in}}%
\pgfusepath{stroke}%
\end{pgfscope}%
\begin{pgfscope}%
\pgfpathrectangle{\pgfqpoint{6.720588in}{1.750000in}}{\pgfqpoint{2.279412in}{2.004545in}}%
\pgfusepath{clip}%
\pgfsetbuttcap%
\pgfsetroundjoin%
\pgfsetlinewidth{0.402642pt}%
\definecolor{currentstroke}{rgb}{0.281446,0.084320,0.407414}%
\pgfsetstrokecolor{currentstroke}%
\pgfsetdash{}{0pt}%
\pgfpathmoveto{\pgfqpoint{7.438804in}{2.726603in}}%
\pgfpathlineto{\pgfqpoint{7.438308in}{2.726584in}}%
\pgfusepath{stroke}%
\end{pgfscope}%
\begin{pgfscope}%
\pgfpathrectangle{\pgfqpoint{6.720588in}{1.750000in}}{\pgfqpoint{2.279412in}{2.004545in}}%
\pgfusepath{clip}%
\pgfsetbuttcap%
\pgfsetroundjoin%
\pgfsetlinewidth{0.400359pt}%
\definecolor{currentstroke}{rgb}{0.281446,0.084320,0.407414}%
\pgfsetstrokecolor{currentstroke}%
\pgfsetdash{}{0pt}%
\pgfpathmoveto{\pgfqpoint{7.438308in}{2.726584in}}%
\pgfpathlineto{\pgfqpoint{7.437440in}{2.726546in}}%
\pgfusepath{stroke}%
\end{pgfscope}%
\begin{pgfscope}%
\pgfpathrectangle{\pgfqpoint{6.720588in}{1.750000in}}{\pgfqpoint{2.279412in}{2.004545in}}%
\pgfusepath{clip}%
\pgfsetbuttcap%
\pgfsetroundjoin%
\pgfsetlinewidth{0.396385pt}%
\definecolor{currentstroke}{rgb}{0.280894,0.078907,0.402329}%
\pgfsetstrokecolor{currentstroke}%
\pgfsetdash{}{0pt}%
\pgfpathmoveto{\pgfqpoint{7.437440in}{2.726546in}}%
\pgfpathlineto{\pgfqpoint{7.437440in}{2.726546in}}%
\pgfusepath{stroke}%
\end{pgfscope}%
\begin{pgfscope}%
\pgfpathrectangle{\pgfqpoint{6.720588in}{1.750000in}}{\pgfqpoint{2.279412in}{2.004545in}}%
\pgfusepath{clip}%
\pgfsetbuttcap%
\pgfsetroundjoin%
\pgfsetlinewidth{0.396385pt}%
\definecolor{currentstroke}{rgb}{0.280894,0.078907,0.402329}%
\pgfsetstrokecolor{currentstroke}%
\pgfsetdash{}{0pt}%
\pgfpathmoveto{\pgfqpoint{7.437440in}{2.726546in}}%
\pgfpathlineto{\pgfqpoint{7.438132in}{2.726574in}}%
\pgfusepath{stroke}%
\end{pgfscope}%
\begin{pgfscope}%
\pgfpathrectangle{\pgfqpoint{6.720588in}{1.750000in}}{\pgfqpoint{2.279412in}{2.004545in}}%
\pgfusepath{clip}%
\pgfsetbuttcap%
\pgfsetroundjoin%
\pgfsetlinewidth{0.399554pt}%
\definecolor{currentstroke}{rgb}{0.281446,0.084320,0.407414}%
\pgfsetstrokecolor{currentstroke}%
\pgfsetdash{}{0pt}%
\pgfpathmoveto{\pgfqpoint{7.438132in}{2.726574in}}%
\pgfpathlineto{\pgfqpoint{7.438699in}{2.726598in}}%
\pgfusepath{stroke}%
\end{pgfscope}%
\begin{pgfscope}%
\pgfpathrectangle{\pgfqpoint{6.720588in}{1.750000in}}{\pgfqpoint{2.279412in}{2.004545in}}%
\pgfusepath{clip}%
\pgfsetbuttcap%
\pgfsetroundjoin%
\pgfsetlinewidth{0.402158pt}%
\definecolor{currentstroke}{rgb}{0.281446,0.084320,0.407414}%
\pgfsetstrokecolor{currentstroke}%
\pgfsetdash{}{0pt}%
\pgfpathmoveto{\pgfqpoint{7.438699in}{2.726598in}}%
\pgfpathlineto{\pgfqpoint{7.438763in}{2.726602in}}%
\pgfusepath{stroke}%
\end{pgfscope}%
\begin{pgfscope}%
\pgfpathrectangle{\pgfqpoint{6.720588in}{1.750000in}}{\pgfqpoint{2.279412in}{2.004545in}}%
\pgfusepath{clip}%
\pgfsetbuttcap%
\pgfsetroundjoin%
\pgfsetlinewidth{0.402452pt}%
\definecolor{currentstroke}{rgb}{0.281446,0.084320,0.407414}%
\pgfsetstrokecolor{currentstroke}%
\pgfsetdash{}{0pt}%
\pgfpathmoveto{\pgfqpoint{7.438763in}{2.726602in}}%
\pgfpathlineto{\pgfqpoint{7.438241in}{2.726581in}}%
\pgfusepath{stroke}%
\end{pgfscope}%
\begin{pgfscope}%
\pgfpathrectangle{\pgfqpoint{6.720588in}{1.750000in}}{\pgfqpoint{2.279412in}{2.004545in}}%
\pgfusepath{clip}%
\pgfsetbuttcap%
\pgfsetroundjoin%
\pgfsetlinewidth{0.400051pt}%
\definecolor{currentstroke}{rgb}{0.281446,0.084320,0.407414}%
\pgfsetstrokecolor{currentstroke}%
\pgfsetdash{}{0pt}%
\pgfpathmoveto{\pgfqpoint{7.438241in}{2.726581in}}%
\pgfpathlineto{\pgfqpoint{7.437429in}{2.726546in}}%
\pgfusepath{stroke}%
\end{pgfscope}%
\begin{pgfscope}%
\pgfpathrectangle{\pgfqpoint{6.720588in}{1.750000in}}{\pgfqpoint{2.279412in}{2.004545in}}%
\pgfusepath{clip}%
\pgfsetbuttcap%
\pgfsetroundjoin%
\pgfsetlinewidth{0.396334pt}%
\definecolor{currentstroke}{rgb}{0.280894,0.078907,0.402329}%
\pgfsetstrokecolor{currentstroke}%
\pgfsetdash{}{0pt}%
\pgfpathmoveto{\pgfqpoint{7.437429in}{2.726546in}}%
\pgfpathlineto{\pgfqpoint{7.437429in}{2.726546in}}%
\pgfusepath{stroke}%
\end{pgfscope}%
\begin{pgfscope}%
\pgfpathrectangle{\pgfqpoint{6.720588in}{1.750000in}}{\pgfqpoint{2.279412in}{2.004545in}}%
\pgfusepath{clip}%
\pgfsetbuttcap%
\pgfsetroundjoin%
\pgfsetlinewidth{0.396334pt}%
\definecolor{currentstroke}{rgb}{0.280894,0.078907,0.402329}%
\pgfsetstrokecolor{currentstroke}%
\pgfsetdash{}{0pt}%
\pgfpathmoveto{\pgfqpoint{7.437429in}{2.726546in}}%
\pgfpathlineto{\pgfqpoint{7.438134in}{2.726574in}}%
\pgfusepath{stroke}%
\end{pgfscope}%
\begin{pgfscope}%
\pgfpathrectangle{\pgfqpoint{6.720588in}{1.750000in}}{\pgfqpoint{2.279412in}{2.004545in}}%
\pgfusepath{clip}%
\pgfsetbuttcap%
\pgfsetroundjoin%
\pgfsetlinewidth{0.399564pt}%
\definecolor{currentstroke}{rgb}{0.281446,0.084320,0.407414}%
\pgfsetstrokecolor{currentstroke}%
\pgfsetdash{}{0pt}%
\pgfpathmoveto{\pgfqpoint{7.438134in}{2.726574in}}%
\pgfpathlineto{\pgfqpoint{7.438706in}{2.726598in}}%
\pgfusepath{stroke}%
\end{pgfscope}%
\begin{pgfscope}%
\pgfpathrectangle{\pgfqpoint{6.720588in}{1.750000in}}{\pgfqpoint{2.279412in}{2.004545in}}%
\pgfusepath{clip}%
\pgfsetbuttcap%
\pgfsetroundjoin%
\pgfsetlinewidth{0.402193pt}%
\definecolor{currentstroke}{rgb}{0.281446,0.084320,0.407414}%
\pgfsetstrokecolor{currentstroke}%
\pgfsetdash{}{0pt}%
\pgfpathmoveto{\pgfqpoint{7.438706in}{2.726598in}}%
\pgfpathlineto{\pgfqpoint{7.438768in}{2.726603in}}%
\pgfusepath{stroke}%
\end{pgfscope}%
\begin{pgfscope}%
\pgfpathrectangle{\pgfqpoint{6.720588in}{1.750000in}}{\pgfqpoint{2.279412in}{2.004545in}}%
\pgfusepath{clip}%
\pgfsetbuttcap%
\pgfsetroundjoin%
\pgfsetlinewidth{0.313354pt}%
\definecolor{currentstroke}{rgb}{0.268510,0.009605,0.335427}%
\pgfsetstrokecolor{currentstroke}%
\pgfsetdash{}{0pt}%
\pgfpathmoveto{\pgfqpoint{8.680964in}{2.571846in}}%
\pgfpathlineto{\pgfqpoint{8.631406in}{2.571241in}}%
\pgfusepath{stroke}%
\end{pgfscope}%
\begin{pgfscope}%
\pgfpathrectangle{\pgfqpoint{6.720588in}{1.750000in}}{\pgfqpoint{2.279412in}{2.004545in}}%
\pgfusepath{clip}%
\pgfsetbuttcap%
\pgfsetroundjoin%
\pgfsetlinewidth{0.313952pt}%
\definecolor{currentstroke}{rgb}{0.268510,0.009605,0.335427}%
\pgfsetstrokecolor{currentstroke}%
\pgfsetdash{}{0pt}%
\pgfpathmoveto{\pgfqpoint{8.631406in}{2.571241in}}%
\pgfpathlineto{\pgfqpoint{8.582044in}{2.571025in}}%
\pgfusepath{stroke}%
\end{pgfscope}%
\begin{pgfscope}%
\pgfpathrectangle{\pgfqpoint{6.720588in}{1.750000in}}{\pgfqpoint{2.279412in}{2.004545in}}%
\pgfusepath{clip}%
\pgfsetbuttcap%
\pgfsetroundjoin%
\pgfsetlinewidth{0.311884pt}%
\definecolor{currentstroke}{rgb}{0.268510,0.009605,0.335427}%
\pgfsetstrokecolor{currentstroke}%
\pgfsetdash{}{0pt}%
\pgfpathmoveto{\pgfqpoint{8.582044in}{2.571025in}}%
\pgfpathlineto{\pgfqpoint{8.531910in}{2.570383in}}%
\pgfusepath{stroke}%
\end{pgfscope}%
\begin{pgfscope}%
\pgfpathrectangle{\pgfqpoint{6.720588in}{1.750000in}}{\pgfqpoint{2.279412in}{2.004545in}}%
\pgfusepath{clip}%
\pgfsetbuttcap%
\pgfsetroundjoin%
\pgfsetlinewidth{0.324455pt}%
\definecolor{currentstroke}{rgb}{0.271305,0.019942,0.347269}%
\pgfsetstrokecolor{currentstroke}%
\pgfsetdash{}{0pt}%
\pgfpathmoveto{\pgfqpoint{8.531910in}{2.570383in}}%
\pgfpathlineto{\pgfqpoint{8.481767in}{2.570698in}}%
\pgfusepath{stroke}%
\end{pgfscope}%
\begin{pgfscope}%
\pgfpathrectangle{\pgfqpoint{6.720588in}{1.750000in}}{\pgfqpoint{2.279412in}{2.004545in}}%
\pgfusepath{clip}%
\pgfsetbuttcap%
\pgfsetroundjoin%
\pgfsetlinewidth{0.328289pt}%
\definecolor{currentstroke}{rgb}{0.271305,0.019942,0.347269}%
\pgfsetstrokecolor{currentstroke}%
\pgfsetdash{}{0pt}%
\pgfpathmoveto{\pgfqpoint{8.481767in}{2.570698in}}%
\pgfpathlineto{\pgfqpoint{8.431627in}{2.571609in}}%
\pgfusepath{stroke}%
\end{pgfscope}%
\begin{pgfscope}%
\pgfpathrectangle{\pgfqpoint{6.720588in}{1.750000in}}{\pgfqpoint{2.279412in}{2.004545in}}%
\pgfusepath{clip}%
\pgfsetbuttcap%
\pgfsetroundjoin%
\pgfsetlinewidth{0.330659pt}%
\definecolor{currentstroke}{rgb}{0.272594,0.025563,0.353093}%
\pgfsetstrokecolor{currentstroke}%
\pgfsetdash{}{0pt}%
\pgfpathmoveto{\pgfqpoint{8.431627in}{2.571609in}}%
\pgfpathlineto{\pgfqpoint{8.381491in}{2.572680in}}%
\pgfusepath{stroke}%
\end{pgfscope}%
\begin{pgfscope}%
\pgfpathrectangle{\pgfqpoint{6.720588in}{1.750000in}}{\pgfqpoint{2.279412in}{2.004545in}}%
\pgfusepath{clip}%
\pgfsetbuttcap%
\pgfsetroundjoin%
\pgfsetlinewidth{0.365975pt}%
\definecolor{currentstroke}{rgb}{0.277941,0.056324,0.381191}%
\pgfsetstrokecolor{currentstroke}%
\pgfsetdash{}{0pt}%
\pgfpathmoveto{\pgfqpoint{8.381491in}{2.572680in}}%
\pgfpathlineto{\pgfqpoint{8.331350in}{2.573428in}}%
\pgfusepath{stroke}%
\end{pgfscope}%
\begin{pgfscope}%
\pgfpathrectangle{\pgfqpoint{6.720588in}{1.750000in}}{\pgfqpoint{2.279412in}{2.004545in}}%
\pgfusepath{clip}%
\pgfsetbuttcap%
\pgfsetroundjoin%
\pgfsetlinewidth{0.402346pt}%
\definecolor{currentstroke}{rgb}{0.281446,0.084320,0.407414}%
\pgfsetstrokecolor{currentstroke}%
\pgfsetdash{}{0pt}%
\pgfpathmoveto{\pgfqpoint{8.331350in}{2.573428in}}%
\pgfpathlineto{\pgfqpoint{8.281207in}{2.574152in}}%
\pgfusepath{stroke}%
\end{pgfscope}%
\begin{pgfscope}%
\pgfpathrectangle{\pgfqpoint{6.720588in}{1.750000in}}{\pgfqpoint{2.279412in}{2.004545in}}%
\pgfusepath{clip}%
\pgfsetbuttcap%
\pgfsetroundjoin%
\pgfsetlinewidth{0.437148pt}%
\definecolor{currentstroke}{rgb}{0.283091,0.110553,0.431554}%
\pgfsetstrokecolor{currentstroke}%
\pgfsetdash{}{0pt}%
\pgfpathmoveto{\pgfqpoint{8.281207in}{2.574152in}}%
\pgfpathlineto{\pgfqpoint{8.231072in}{2.575248in}}%
\pgfusepath{stroke}%
\end{pgfscope}%
\begin{pgfscope}%
\pgfpathrectangle{\pgfqpoint{6.720588in}{1.750000in}}{\pgfqpoint{2.279412in}{2.004545in}}%
\pgfusepath{clip}%
\pgfsetbuttcap%
\pgfsetroundjoin%
\pgfsetlinewidth{0.483535pt}%
\definecolor{currentstroke}{rgb}{0.282290,0.145912,0.461510}%
\pgfsetstrokecolor{currentstroke}%
\pgfsetdash{}{0pt}%
\pgfpathmoveto{\pgfqpoint{8.231072in}{2.575248in}}%
\pgfpathlineto{\pgfqpoint{8.180937in}{2.576382in}}%
\pgfusepath{stroke}%
\end{pgfscope}%
\begin{pgfscope}%
\pgfpathrectangle{\pgfqpoint{6.720588in}{1.750000in}}{\pgfqpoint{2.279412in}{2.004545in}}%
\pgfusepath{clip}%
\pgfsetbuttcap%
\pgfsetroundjoin%
\pgfsetlinewidth{0.560767pt}%
\definecolor{currentstroke}{rgb}{0.274128,0.199721,0.498911}%
\pgfsetstrokecolor{currentstroke}%
\pgfsetdash{}{0pt}%
\pgfpathmoveto{\pgfqpoint{8.180937in}{2.576382in}}%
\pgfpathlineto{\pgfqpoint{8.130811in}{2.577797in}}%
\pgfusepath{stroke}%
\end{pgfscope}%
\begin{pgfscope}%
\pgfpathrectangle{\pgfqpoint{6.720588in}{1.750000in}}{\pgfqpoint{2.279412in}{2.004545in}}%
\pgfusepath{clip}%
\pgfsetbuttcap%
\pgfsetroundjoin%
\pgfsetlinewidth{0.641564pt}%
\definecolor{currentstroke}{rgb}{0.257322,0.256130,0.526563}%
\pgfsetstrokecolor{currentstroke}%
\pgfsetdash{}{0pt}%
\pgfpathmoveto{\pgfqpoint{8.130811in}{2.577797in}}%
\pgfpathlineto{\pgfqpoint{8.080706in}{2.579677in}}%
\pgfusepath{stroke}%
\end{pgfscope}%
\begin{pgfscope}%
\pgfpathrectangle{\pgfqpoint{6.720588in}{1.750000in}}{\pgfqpoint{2.279412in}{2.004545in}}%
\pgfusepath{clip}%
\pgfsetbuttcap%
\pgfsetroundjoin%
\pgfsetlinewidth{0.713054pt}%
\definecolor{currentstroke}{rgb}{0.237441,0.305202,0.541921}%
\pgfsetstrokecolor{currentstroke}%
\pgfsetdash{}{0pt}%
\pgfpathmoveto{\pgfqpoint{8.080706in}{2.579677in}}%
\pgfpathlineto{\pgfqpoint{8.030631in}{2.582090in}}%
\pgfusepath{stroke}%
\end{pgfscope}%
\begin{pgfscope}%
\pgfpathrectangle{\pgfqpoint{6.720588in}{1.750000in}}{\pgfqpoint{2.279412in}{2.004545in}}%
\pgfusepath{clip}%
\pgfsetbuttcap%
\pgfsetroundjoin%
\pgfsetlinewidth{0.769418pt}%
\definecolor{currentstroke}{rgb}{0.221989,0.339161,0.548752}%
\pgfsetstrokecolor{currentstroke}%
\pgfsetdash{}{0pt}%
\pgfpathmoveto{\pgfqpoint{8.030631in}{2.582090in}}%
\pgfpathlineto{\pgfqpoint{7.980590in}{2.585003in}}%
\pgfusepath{stroke}%
\end{pgfscope}%
\begin{pgfscope}%
\pgfpathrectangle{\pgfqpoint{6.720588in}{1.750000in}}{\pgfqpoint{2.279412in}{2.004545in}}%
\pgfusepath{clip}%
\pgfsetbuttcap%
\pgfsetroundjoin%
\pgfsetlinewidth{0.808668pt}%
\definecolor{currentstroke}{rgb}{0.212395,0.359683,0.551710}%
\pgfsetstrokecolor{currentstroke}%
\pgfsetdash{}{0pt}%
\pgfpathmoveto{\pgfqpoint{7.980590in}{2.585003in}}%
\pgfpathlineto{\pgfqpoint{7.930610in}{2.588604in}}%
\pgfusepath{stroke}%
\end{pgfscope}%
\begin{pgfscope}%
\pgfpathrectangle{\pgfqpoint{6.720588in}{1.750000in}}{\pgfqpoint{2.279412in}{2.004545in}}%
\pgfusepath{clip}%
\pgfsetbuttcap%
\pgfsetroundjoin%
\pgfsetlinewidth{0.790437pt}%
\definecolor{currentstroke}{rgb}{0.216210,0.351535,0.550627}%
\pgfsetstrokecolor{currentstroke}%
\pgfsetdash{}{0pt}%
\pgfpathmoveto{\pgfqpoint{7.930610in}{2.588604in}}%
\pgfpathlineto{\pgfqpoint{7.880736in}{2.593196in}}%
\pgfusepath{stroke}%
\end{pgfscope}%
\begin{pgfscope}%
\pgfpathrectangle{\pgfqpoint{6.720588in}{1.750000in}}{\pgfqpoint{2.279412in}{2.004545in}}%
\pgfusepath{clip}%
\pgfsetbuttcap%
\pgfsetroundjoin%
\pgfsetlinewidth{0.705961pt}%
\definecolor{currentstroke}{rgb}{0.239346,0.300855,0.540844}%
\pgfsetstrokecolor{currentstroke}%
\pgfsetdash{}{0pt}%
\pgfpathmoveto{\pgfqpoint{7.880736in}{2.593196in}}%
\pgfpathlineto{\pgfqpoint{7.831002in}{2.598834in}}%
\pgfusepath{stroke}%
\end{pgfscope}%
\begin{pgfscope}%
\pgfpathrectangle{\pgfqpoint{6.720588in}{1.750000in}}{\pgfqpoint{2.279412in}{2.004545in}}%
\pgfusepath{clip}%
\pgfsetbuttcap%
\pgfsetroundjoin%
\pgfsetlinewidth{0.789398pt}%
\definecolor{currentstroke}{rgb}{0.216210,0.351535,0.550627}%
\pgfsetstrokecolor{currentstroke}%
\pgfsetdash{}{0pt}%
\pgfpathmoveto{\pgfqpoint{7.831002in}{2.598834in}}%
\pgfpathlineto{\pgfqpoint{7.781490in}{2.605785in}}%
\pgfusepath{stroke}%
\end{pgfscope}%
\begin{pgfscope}%
\pgfpathrectangle{\pgfqpoint{6.720588in}{1.750000in}}{\pgfqpoint{2.279412in}{2.004545in}}%
\pgfusepath{clip}%
\pgfsetbuttcap%
\pgfsetroundjoin%
\pgfsetlinewidth{0.698479pt}%
\definecolor{currentstroke}{rgb}{0.243113,0.292092,0.538516}%
\pgfsetstrokecolor{currentstroke}%
\pgfsetdash{}{0pt}%
\pgfpathmoveto{\pgfqpoint{7.781490in}{2.605785in}}%
\pgfpathlineto{\pgfqpoint{7.732249in}{2.614108in}}%
\pgfusepath{stroke}%
\end{pgfscope}%
\begin{pgfscope}%
\pgfpathrectangle{\pgfqpoint{6.720588in}{1.750000in}}{\pgfqpoint{2.279412in}{2.004545in}}%
\pgfusepath{clip}%
\pgfsetbuttcap%
\pgfsetroundjoin%
\pgfsetlinewidth{0.699337pt}%
\definecolor{currentstroke}{rgb}{0.241237,0.296485,0.539709}%
\pgfsetstrokecolor{currentstroke}%
\pgfsetdash{}{0pt}%
\pgfpathmoveto{\pgfqpoint{7.732249in}{2.614108in}}%
\pgfpathlineto{\pgfqpoint{7.683667in}{2.624785in}}%
\pgfusepath{stroke}%
\end{pgfscope}%
\begin{pgfscope}%
\pgfpathrectangle{\pgfqpoint{6.720588in}{1.750000in}}{\pgfqpoint{2.279412in}{2.004545in}}%
\pgfusepath{clip}%
\pgfsetbuttcap%
\pgfsetroundjoin%
\pgfsetlinewidth{0.604549pt}%
\definecolor{currentstroke}{rgb}{0.265145,0.232956,0.516599}%
\pgfsetstrokecolor{currentstroke}%
\pgfsetdash{}{0pt}%
\pgfpathmoveto{\pgfqpoint{7.683667in}{2.624785in}}%
\pgfpathlineto{\pgfqpoint{7.635592in}{2.637175in}}%
\pgfusepath{stroke}%
\end{pgfscope}%
\begin{pgfscope}%
\pgfpathrectangle{\pgfqpoint{6.720588in}{1.750000in}}{\pgfqpoint{2.279412in}{2.004545in}}%
\pgfusepath{clip}%
\pgfsetbuttcap%
\pgfsetroundjoin%
\pgfsetlinewidth{0.600022pt}%
\definecolor{currentstroke}{rgb}{0.266580,0.228262,0.514349}%
\pgfsetstrokecolor{currentstroke}%
\pgfsetdash{}{0pt}%
\pgfpathmoveto{\pgfqpoint{7.635592in}{2.637175in}}%
\pgfpathlineto{\pgfqpoint{7.587496in}{2.649539in}}%
\pgfusepath{stroke}%
\end{pgfscope}%
\begin{pgfscope}%
\pgfpathrectangle{\pgfqpoint{6.720588in}{1.750000in}}{\pgfqpoint{2.279412in}{2.004545in}}%
\pgfusepath{clip}%
\pgfsetbuttcap%
\pgfsetroundjoin%
\pgfsetlinewidth{0.579703pt}%
\definecolor{currentstroke}{rgb}{0.270595,0.214069,0.507052}%
\pgfsetstrokecolor{currentstroke}%
\pgfsetdash{}{0pt}%
\pgfpathmoveto{\pgfqpoint{7.587496in}{2.649539in}}%
\pgfpathlineto{\pgfqpoint{7.539751in}{2.662622in}}%
\pgfusepath{stroke}%
\end{pgfscope}%
\begin{pgfscope}%
\pgfpathrectangle{\pgfqpoint{6.720588in}{1.750000in}}{\pgfqpoint{2.279412in}{2.004545in}}%
\pgfusepath{clip}%
\pgfsetbuttcap%
\pgfsetroundjoin%
\pgfsetlinewidth{0.525085pt}%
\definecolor{currentstroke}{rgb}{0.278826,0.175490,0.483397}%
\pgfsetstrokecolor{currentstroke}%
\pgfsetdash{}{0pt}%
\pgfpathmoveto{\pgfqpoint{7.539751in}{2.662622in}}%
\pgfpathlineto{\pgfqpoint{7.539751in}{2.662622in}}%
\pgfusepath{stroke}%
\end{pgfscope}%
\begin{pgfscope}%
\pgfpathrectangle{\pgfqpoint{6.720588in}{1.750000in}}{\pgfqpoint{2.279412in}{2.004545in}}%
\pgfusepath{clip}%
\pgfsetbuttcap%
\pgfsetroundjoin%
\pgfsetlinewidth{0.525085pt}%
\definecolor{currentstroke}{rgb}{0.278826,0.175490,0.483397}%
\pgfsetstrokecolor{currentstroke}%
\pgfsetdash{}{0pt}%
\pgfpathmoveto{\pgfqpoint{7.539751in}{2.662622in}}%
\pgfpathlineto{\pgfqpoint{7.509931in}{2.673464in}}%
\pgfusepath{stroke}%
\end{pgfscope}%
\begin{pgfscope}%
\pgfpathrectangle{\pgfqpoint{6.720588in}{1.750000in}}{\pgfqpoint{2.279412in}{2.004545in}}%
\pgfusepath{clip}%
\pgfsetbuttcap%
\pgfsetroundjoin%
\pgfsetlinewidth{0.323323pt}%
\definecolor{currentstroke}{rgb}{0.271305,0.019942,0.347269}%
\pgfsetstrokecolor{currentstroke}%
\pgfsetdash{}{0pt}%
\pgfpathmoveto{\pgfqpoint{8.680964in}{2.616952in}}%
\pgfpathlineto{\pgfqpoint{8.632153in}{2.617371in}}%
\pgfusepath{stroke}%
\end{pgfscope}%
\begin{pgfscope}%
\pgfpathrectangle{\pgfqpoint{6.720588in}{1.750000in}}{\pgfqpoint{2.279412in}{2.004545in}}%
\pgfusepath{clip}%
\pgfsetbuttcap%
\pgfsetroundjoin%
\pgfsetlinewidth{0.316599pt}%
\definecolor{currentstroke}{rgb}{0.269944,0.014625,0.341379}%
\pgfsetstrokecolor{currentstroke}%
\pgfsetdash{}{0pt}%
\pgfpathmoveto{\pgfqpoint{8.632153in}{2.617371in}}%
\pgfpathlineto{\pgfqpoint{8.583924in}{2.617167in}}%
\pgfusepath{stroke}%
\end{pgfscope}%
\begin{pgfscope}%
\pgfpathrectangle{\pgfqpoint{6.720588in}{1.750000in}}{\pgfqpoint{2.279412in}{2.004545in}}%
\pgfusepath{clip}%
\pgfsetbuttcap%
\pgfsetroundjoin%
\pgfsetlinewidth{0.310289pt}%
\definecolor{currentstroke}{rgb}{0.268510,0.009605,0.335427}%
\pgfsetstrokecolor{currentstroke}%
\pgfsetdash{}{0pt}%
\pgfpathmoveto{\pgfqpoint{8.583924in}{2.617167in}}%
\pgfpathlineto{\pgfqpoint{8.533840in}{2.615187in}}%
\pgfusepath{stroke}%
\end{pgfscope}%
\begin{pgfscope}%
\pgfpathrectangle{\pgfqpoint{6.720588in}{1.750000in}}{\pgfqpoint{2.279412in}{2.004545in}}%
\pgfusepath{clip}%
\pgfsetbuttcap%
\pgfsetroundjoin%
\pgfsetlinewidth{0.327896pt}%
\definecolor{currentstroke}{rgb}{0.271305,0.019942,0.347269}%
\pgfsetstrokecolor{currentstroke}%
\pgfsetdash{}{0pt}%
\pgfpathmoveto{\pgfqpoint{8.533840in}{2.615187in}}%
\pgfpathlineto{\pgfqpoint{8.483728in}{2.613825in}}%
\pgfusepath{stroke}%
\end{pgfscope}%
\begin{pgfscope}%
\pgfpathrectangle{\pgfqpoint{6.720588in}{1.750000in}}{\pgfqpoint{2.279412in}{2.004545in}}%
\pgfusepath{clip}%
\pgfsetbuttcap%
\pgfsetroundjoin%
\pgfsetlinewidth{0.331547pt}%
\definecolor{currentstroke}{rgb}{0.272594,0.025563,0.353093}%
\pgfsetstrokecolor{currentstroke}%
\pgfsetdash{}{0pt}%
\pgfpathmoveto{\pgfqpoint{8.483728in}{2.613825in}}%
\pgfpathlineto{\pgfqpoint{8.433590in}{2.613646in}}%
\pgfusepath{stroke}%
\end{pgfscope}%
\begin{pgfscope}%
\pgfpathrectangle{\pgfqpoint{6.720588in}{1.750000in}}{\pgfqpoint{2.279412in}{2.004545in}}%
\pgfusepath{clip}%
\pgfsetbuttcap%
\pgfsetroundjoin%
\pgfsetlinewidth{0.338691pt}%
\definecolor{currentstroke}{rgb}{0.273809,0.031497,0.358853}%
\pgfsetstrokecolor{currentstroke}%
\pgfsetdash{}{0pt}%
\pgfpathmoveto{\pgfqpoint{8.433590in}{2.613646in}}%
\pgfpathlineto{\pgfqpoint{8.383443in}{2.613745in}}%
\pgfusepath{stroke}%
\end{pgfscope}%
\begin{pgfscope}%
\pgfpathrectangle{\pgfqpoint{6.720588in}{1.750000in}}{\pgfqpoint{2.279412in}{2.004545in}}%
\pgfusepath{clip}%
\pgfsetbuttcap%
\pgfsetroundjoin%
\pgfsetlinewidth{0.369827pt}%
\definecolor{currentstroke}{rgb}{0.278791,0.062145,0.386592}%
\pgfsetstrokecolor{currentstroke}%
\pgfsetdash{}{0pt}%
\pgfpathmoveto{\pgfqpoint{8.383443in}{2.613745in}}%
\pgfpathlineto{\pgfqpoint{8.333295in}{2.614120in}}%
\pgfusepath{stroke}%
\end{pgfscope}%
\begin{pgfscope}%
\pgfpathrectangle{\pgfqpoint{6.720588in}{1.750000in}}{\pgfqpoint{2.279412in}{2.004545in}}%
\pgfusepath{clip}%
\pgfsetbuttcap%
\pgfsetroundjoin%
\pgfsetlinewidth{0.397102pt}%
\definecolor{currentstroke}{rgb}{0.280894,0.078907,0.402329}%
\pgfsetstrokecolor{currentstroke}%
\pgfsetdash{}{0pt}%
\pgfpathmoveto{\pgfqpoint{8.333295in}{2.614120in}}%
\pgfpathlineto{\pgfqpoint{8.283150in}{2.614828in}}%
\pgfusepath{stroke}%
\end{pgfscope}%
\begin{pgfscope}%
\pgfpathrectangle{\pgfqpoint{6.720588in}{1.750000in}}{\pgfqpoint{2.279412in}{2.004545in}}%
\pgfusepath{clip}%
\pgfsetbuttcap%
\pgfsetroundjoin%
\pgfsetlinewidth{0.434315pt}%
\definecolor{currentstroke}{rgb}{0.283091,0.110553,0.431554}%
\pgfsetstrokecolor{currentstroke}%
\pgfsetdash{}{0pt}%
\pgfpathmoveto{\pgfqpoint{8.283150in}{2.614828in}}%
\pgfpathlineto{\pgfqpoint{8.233006in}{2.615557in}}%
\pgfusepath{stroke}%
\end{pgfscope}%
\begin{pgfscope}%
\pgfpathrectangle{\pgfqpoint{6.720588in}{1.750000in}}{\pgfqpoint{2.279412in}{2.004545in}}%
\pgfusepath{clip}%
\pgfsetbuttcap%
\pgfsetroundjoin%
\pgfsetlinewidth{0.505739pt}%
\definecolor{currentstroke}{rgb}{0.280868,0.160771,0.472899}%
\pgfsetstrokecolor{currentstroke}%
\pgfsetdash{}{0pt}%
\pgfpathmoveto{\pgfqpoint{8.233006in}{2.615557in}}%
\pgfpathlineto{\pgfqpoint{8.182864in}{2.616454in}}%
\pgfusepath{stroke}%
\end{pgfscope}%
\begin{pgfscope}%
\pgfpathrectangle{\pgfqpoint{6.720588in}{1.750000in}}{\pgfqpoint{2.279412in}{2.004545in}}%
\pgfusepath{clip}%
\pgfsetbuttcap%
\pgfsetroundjoin%
\pgfsetlinewidth{0.580408pt}%
\definecolor{currentstroke}{rgb}{0.270595,0.214069,0.507052}%
\pgfsetstrokecolor{currentstroke}%
\pgfsetdash{}{0pt}%
\pgfpathmoveto{\pgfqpoint{8.182864in}{2.616454in}}%
\pgfpathlineto{\pgfqpoint{8.132725in}{2.617453in}}%
\pgfusepath{stroke}%
\end{pgfscope}%
\begin{pgfscope}%
\pgfpathrectangle{\pgfqpoint{6.720588in}{1.750000in}}{\pgfqpoint{2.279412in}{2.004545in}}%
\pgfusepath{clip}%
\pgfsetbuttcap%
\pgfsetroundjoin%
\pgfsetlinewidth{0.664066pt}%
\definecolor{currentstroke}{rgb}{0.252194,0.269783,0.531579}%
\pgfsetstrokecolor{currentstroke}%
\pgfsetdash{}{0pt}%
\pgfpathmoveto{\pgfqpoint{8.132725in}{2.617453in}}%
\pgfpathlineto{\pgfqpoint{8.082595in}{2.618724in}}%
\pgfusepath{stroke}%
\end{pgfscope}%
\begin{pgfscope}%
\pgfpathrectangle{\pgfqpoint{6.720588in}{1.750000in}}{\pgfqpoint{2.279412in}{2.004545in}}%
\pgfusepath{clip}%
\pgfsetbuttcap%
\pgfsetroundjoin%
\pgfsetlinewidth{0.735743pt}%
\definecolor{currentstroke}{rgb}{0.231674,0.318106,0.544834}%
\pgfsetstrokecolor{currentstroke}%
\pgfsetdash{}{0pt}%
\pgfpathmoveto{\pgfqpoint{8.082595in}{2.618724in}}%
\pgfpathlineto{\pgfqpoint{8.032480in}{2.620406in}}%
\pgfusepath{stroke}%
\end{pgfscope}%
\begin{pgfscope}%
\pgfpathrectangle{\pgfqpoint{6.720588in}{1.750000in}}{\pgfqpoint{2.279412in}{2.004545in}}%
\pgfusepath{clip}%
\pgfsetbuttcap%
\pgfsetroundjoin%
\pgfsetlinewidth{0.797929pt}%
\definecolor{currentstroke}{rgb}{0.214298,0.355619,0.551184}%
\pgfsetstrokecolor{currentstroke}%
\pgfsetdash{}{0pt}%
\pgfpathmoveto{\pgfqpoint{8.032480in}{2.620406in}}%
\pgfpathlineto{\pgfqpoint{7.982386in}{2.622504in}}%
\pgfusepath{stroke}%
\end{pgfscope}%
\begin{pgfscope}%
\pgfpathrectangle{\pgfqpoint{6.720588in}{1.750000in}}{\pgfqpoint{2.279412in}{2.004545in}}%
\pgfusepath{clip}%
\pgfsetbuttcap%
\pgfsetroundjoin%
\pgfsetlinewidth{0.824844pt}%
\definecolor{currentstroke}{rgb}{0.206756,0.371758,0.553117}%
\pgfsetstrokecolor{currentstroke}%
\pgfsetdash{}{0pt}%
\pgfpathmoveto{\pgfqpoint{7.982386in}{2.622504in}}%
\pgfpathlineto{\pgfqpoint{7.932327in}{2.625139in}}%
\pgfusepath{stroke}%
\end{pgfscope}%
\begin{pgfscope}%
\pgfpathrectangle{\pgfqpoint{6.720588in}{1.750000in}}{\pgfqpoint{2.279412in}{2.004545in}}%
\pgfusepath{clip}%
\pgfsetbuttcap%
\pgfsetroundjoin%
\pgfsetlinewidth{0.850054pt}%
\definecolor{currentstroke}{rgb}{0.199430,0.387607,0.554642}%
\pgfsetstrokecolor{currentstroke}%
\pgfsetdash{}{0pt}%
\pgfpathmoveto{\pgfqpoint{7.932327in}{2.625139in}}%
\pgfpathlineto{\pgfqpoint{7.882308in}{2.628278in}}%
\pgfusepath{stroke}%
\end{pgfscope}%
\begin{pgfscope}%
\pgfpathrectangle{\pgfqpoint{6.720588in}{1.750000in}}{\pgfqpoint{2.279412in}{2.004545in}}%
\pgfusepath{clip}%
\pgfsetbuttcap%
\pgfsetroundjoin%
\pgfsetlinewidth{1.461559pt}%
\definecolor{currentstroke}{rgb}{0.202219,0.715272,0.476084}%
\pgfsetstrokecolor{currentstroke}%
\pgfsetdash{}{0pt}%
\pgfpathmoveto{\pgfqpoint{7.039624in}{2.707166in}}%
\pgfpathlineto{\pgfqpoint{7.089741in}{2.708618in}}%
\pgfusepath{stroke}%
\end{pgfscope}%
\begin{pgfscope}%
\pgfpathrectangle{\pgfqpoint{6.720588in}{1.750000in}}{\pgfqpoint{2.279412in}{2.004545in}}%
\pgfusepath{clip}%
\pgfsetbuttcap%
\pgfsetroundjoin%
\pgfsetlinewidth{1.335285pt}%
\definecolor{currentstroke}{rgb}{0.128087,0.647749,0.523491}%
\pgfsetstrokecolor{currentstroke}%
\pgfsetdash{}{0pt}%
\pgfpathmoveto{\pgfqpoint{7.089741in}{2.708618in}}%
\pgfpathlineto{\pgfqpoint{7.139868in}{2.709630in}}%
\pgfusepath{stroke}%
\end{pgfscope}%
\begin{pgfscope}%
\pgfpathrectangle{\pgfqpoint{6.720588in}{1.750000in}}{\pgfqpoint{2.279412in}{2.004545in}}%
\pgfusepath{clip}%
\pgfsetbuttcap%
\pgfsetroundjoin%
\pgfsetlinewidth{1.199924pt}%
\definecolor{currentstroke}{rgb}{0.125394,0.574318,0.549086}%
\pgfsetstrokecolor{currentstroke}%
\pgfsetdash{}{0pt}%
\pgfpathmoveto{\pgfqpoint{7.139868in}{2.709630in}}%
\pgfpathlineto{\pgfqpoint{7.190001in}{2.710582in}}%
\pgfusepath{stroke}%
\end{pgfscope}%
\begin{pgfscope}%
\pgfpathrectangle{\pgfqpoint{6.720588in}{1.750000in}}{\pgfqpoint{2.279412in}{2.004545in}}%
\pgfusepath{clip}%
\pgfsetbuttcap%
\pgfsetroundjoin%
\pgfsetlinewidth{1.074970pt}%
\definecolor{currentstroke}{rgb}{0.149039,0.508051,0.557250}%
\pgfsetstrokecolor{currentstroke}%
\pgfsetdash{}{0pt}%
\pgfpathmoveto{\pgfqpoint{7.190001in}{2.710582in}}%
\pgfpathlineto{\pgfqpoint{7.240118in}{2.712015in}}%
\pgfusepath{stroke}%
\end{pgfscope}%
\begin{pgfscope}%
\pgfpathrectangle{\pgfqpoint{6.720588in}{1.750000in}}{\pgfqpoint{2.279412in}{2.004545in}}%
\pgfusepath{clip}%
\pgfsetbuttcap%
\pgfsetroundjoin%
\pgfsetlinewidth{0.925003pt}%
\definecolor{currentstroke}{rgb}{0.182256,0.426184,0.557120}%
\pgfsetstrokecolor{currentstroke}%
\pgfsetdash{}{0pt}%
\pgfpathmoveto{\pgfqpoint{7.240118in}{2.712015in}}%
\pgfpathlineto{\pgfqpoint{7.290247in}{2.713076in}}%
\pgfusepath{stroke}%
\end{pgfscope}%
\begin{pgfscope}%
\pgfpathrectangle{\pgfqpoint{6.720588in}{1.750000in}}{\pgfqpoint{2.279412in}{2.004545in}}%
\pgfusepath{clip}%
\pgfsetbuttcap%
\pgfsetroundjoin%
\pgfsetlinewidth{0.810865pt}%
\definecolor{currentstroke}{rgb}{0.210503,0.363727,0.552206}%
\pgfsetstrokecolor{currentstroke}%
\pgfsetdash{}{0pt}%
\pgfpathmoveto{\pgfqpoint{7.290247in}{2.713076in}}%
\pgfpathlineto{\pgfqpoint{7.340329in}{2.714905in}}%
\pgfusepath{stroke}%
\end{pgfscope}%
\begin{pgfscope}%
\pgfpathrectangle{\pgfqpoint{6.720588in}{1.750000in}}{\pgfqpoint{2.279412in}{2.004545in}}%
\pgfusepath{clip}%
\pgfsetbuttcap%
\pgfsetroundjoin%
\pgfsetlinewidth{0.576628pt}%
\definecolor{currentstroke}{rgb}{0.270595,0.214069,0.507052}%
\pgfsetstrokecolor{currentstroke}%
\pgfsetdash{}{0pt}%
\pgfpathmoveto{\pgfqpoint{7.340329in}{2.714905in}}%
\pgfpathlineto{\pgfqpoint{7.390398in}{2.717005in}}%
\pgfusepath{stroke}%
\end{pgfscope}%
\begin{pgfscope}%
\pgfpathrectangle{\pgfqpoint{6.720588in}{1.750000in}}{\pgfqpoint{2.279412in}{2.004545in}}%
\pgfusepath{clip}%
\pgfsetbuttcap%
\pgfsetroundjoin%
\pgfsetlinewidth{0.521001pt}%
\definecolor{currentstroke}{rgb}{0.278826,0.175490,0.483397}%
\pgfsetstrokecolor{currentstroke}%
\pgfsetdash{}{0pt}%
\pgfpathmoveto{\pgfqpoint{7.390398in}{2.717005in}}%
\pgfpathlineto{\pgfqpoint{7.390398in}{2.717005in}}%
\pgfusepath{stroke}%
\end{pgfscope}%
\begin{pgfscope}%
\pgfpathrectangle{\pgfqpoint{6.720588in}{1.750000in}}{\pgfqpoint{2.279412in}{2.004545in}}%
\pgfusepath{clip}%
\pgfsetbuttcap%
\pgfsetroundjoin%
\pgfsetlinewidth{0.521001pt}%
\definecolor{currentstroke}{rgb}{0.278826,0.175490,0.483397}%
\pgfsetstrokecolor{currentstroke}%
\pgfsetdash{}{0pt}%
\pgfpathmoveto{\pgfqpoint{7.390398in}{2.717005in}}%
\pgfpathlineto{\pgfqpoint{7.413582in}{2.717406in}}%
\pgfusepath{stroke}%
\end{pgfscope}%
\begin{pgfscope}%
\pgfpathrectangle{\pgfqpoint{6.720588in}{1.750000in}}{\pgfqpoint{2.279412in}{2.004545in}}%
\pgfusepath{clip}%
\pgfsetbuttcap%
\pgfsetroundjoin%
\pgfsetlinewidth{0.398459pt}%
\definecolor{currentstroke}{rgb}{0.281446,0.084320,0.407414}%
\pgfsetstrokecolor{currentstroke}%
\pgfsetdash{}{0pt}%
\pgfpathmoveto{\pgfqpoint{7.413582in}{2.717406in}}%
\pgfpathlineto{\pgfqpoint{7.413582in}{2.717406in}}%
\pgfusepath{stroke}%
\end{pgfscope}%
\begin{pgfscope}%
\pgfpathrectangle{\pgfqpoint{6.720588in}{1.750000in}}{\pgfqpoint{2.279412in}{2.004545in}}%
\pgfusepath{clip}%
\pgfsetbuttcap%
\pgfsetroundjoin%
\pgfsetlinewidth{0.313049pt}%
\definecolor{currentstroke}{rgb}{0.268510,0.009605,0.335427}%
\pgfsetstrokecolor{currentstroke}%
\pgfsetdash{}{0pt}%
\pgfpathmoveto{\pgfqpoint{8.542422in}{2.070183in}}%
\pgfpathlineto{\pgfqpoint{8.497859in}{2.070337in}}%
\pgfusepath{stroke}%
\end{pgfscope}%
\begin{pgfscope}%
\pgfpathrectangle{\pgfqpoint{6.720588in}{1.750000in}}{\pgfqpoint{2.279412in}{2.004545in}}%
\pgfusepath{clip}%
\pgfsetbuttcap%
\pgfsetroundjoin%
\pgfsetlinewidth{0.314302pt}%
\definecolor{currentstroke}{rgb}{0.268510,0.009605,0.335427}%
\pgfsetstrokecolor{currentstroke}%
\pgfsetdash{}{0pt}%
\pgfpathmoveto{\pgfqpoint{8.497859in}{2.070337in}}%
\pgfpathlineto{\pgfqpoint{8.453315in}{2.071368in}}%
\pgfusepath{stroke}%
\end{pgfscope}%
\begin{pgfscope}%
\pgfpathrectangle{\pgfqpoint{6.720588in}{1.750000in}}{\pgfqpoint{2.279412in}{2.004545in}}%
\pgfusepath{clip}%
\pgfsetbuttcap%
\pgfsetroundjoin%
\pgfsetlinewidth{0.315544pt}%
\definecolor{currentstroke}{rgb}{0.269944,0.014625,0.341379}%
\pgfsetstrokecolor{currentstroke}%
\pgfsetdash{}{0pt}%
\pgfpathmoveto{\pgfqpoint{8.453315in}{2.071368in}}%
\pgfpathlineto{\pgfqpoint{8.416568in}{2.072518in}}%
\pgfusepath{stroke}%
\end{pgfscope}%
\begin{pgfscope}%
\pgfpathrectangle{\pgfqpoint{6.720588in}{1.750000in}}{\pgfqpoint{2.279412in}{2.004545in}}%
\pgfusepath{clip}%
\pgfsetbuttcap%
\pgfsetroundjoin%
\pgfsetlinewidth{0.315222pt}%
\definecolor{currentstroke}{rgb}{0.269944,0.014625,0.341379}%
\pgfsetstrokecolor{currentstroke}%
\pgfsetdash{}{0pt}%
\pgfpathmoveto{\pgfqpoint{8.416568in}{2.072518in}}%
\pgfpathlineto{\pgfqpoint{8.373213in}{2.075671in}}%
\pgfusepath{stroke}%
\end{pgfscope}%
\begin{pgfscope}%
\pgfpathrectangle{\pgfqpoint{6.720588in}{1.750000in}}{\pgfqpoint{2.279412in}{2.004545in}}%
\pgfusepath{clip}%
\pgfsetbuttcap%
\pgfsetroundjoin%
\pgfsetlinewidth{0.323073pt}%
\definecolor{currentstroke}{rgb}{0.271305,0.019942,0.347269}%
\pgfsetstrokecolor{currentstroke}%
\pgfsetdash{}{0pt}%
\pgfpathmoveto{\pgfqpoint{8.373213in}{2.075671in}}%
\pgfpathlineto{\pgfqpoint{8.323257in}{2.078301in}}%
\pgfusepath{stroke}%
\end{pgfscope}%
\begin{pgfscope}%
\pgfpathrectangle{\pgfqpoint{6.720588in}{1.750000in}}{\pgfqpoint{2.279412in}{2.004545in}}%
\pgfusepath{clip}%
\pgfsetbuttcap%
\pgfsetroundjoin%
\pgfsetlinewidth{0.340275pt}%
\definecolor{currentstroke}{rgb}{0.273809,0.031497,0.358853}%
\pgfsetstrokecolor{currentstroke}%
\pgfsetdash{}{0pt}%
\pgfpathmoveto{\pgfqpoint{8.323257in}{2.078301in}}%
\pgfpathlineto{\pgfqpoint{8.273204in}{2.079348in}}%
\pgfusepath{stroke}%
\end{pgfscope}%
\begin{pgfscope}%
\pgfpathrectangle{\pgfqpoint{6.720588in}{1.750000in}}{\pgfqpoint{2.279412in}{2.004545in}}%
\pgfusepath{clip}%
\pgfsetbuttcap%
\pgfsetroundjoin%
\pgfsetlinewidth{0.336925pt}%
\definecolor{currentstroke}{rgb}{0.273809,0.031497,0.358853}%
\pgfsetstrokecolor{currentstroke}%
\pgfsetdash{}{0pt}%
\pgfpathmoveto{\pgfqpoint{8.273204in}{2.079348in}}%
\pgfpathlineto{\pgfqpoint{8.223189in}{2.081738in}}%
\pgfusepath{stroke}%
\end{pgfscope}%
\begin{pgfscope}%
\pgfpathrectangle{\pgfqpoint{6.720588in}{1.750000in}}{\pgfqpoint{2.279412in}{2.004545in}}%
\pgfusepath{clip}%
\pgfsetbuttcap%
\pgfsetroundjoin%
\pgfsetlinewidth{0.340709pt}%
\definecolor{currentstroke}{rgb}{0.273809,0.031497,0.358853}%
\pgfsetstrokecolor{currentstroke}%
\pgfsetdash{}{0pt}%
\pgfpathmoveto{\pgfqpoint{8.223189in}{2.081738in}}%
\pgfpathlineto{\pgfqpoint{8.173304in}{2.085354in}}%
\pgfusepath{stroke}%
\end{pgfscope}%
\begin{pgfscope}%
\pgfpathrectangle{\pgfqpoint{6.720588in}{1.750000in}}{\pgfqpoint{2.279412in}{2.004545in}}%
\pgfusepath{clip}%
\pgfsetbuttcap%
\pgfsetroundjoin%
\pgfsetlinewidth{0.357263pt}%
\definecolor{currentstroke}{rgb}{0.277018,0.050344,0.375715}%
\pgfsetstrokecolor{currentstroke}%
\pgfsetdash{}{0pt}%
\pgfpathmoveto{\pgfqpoint{8.173304in}{2.085354in}}%
\pgfpathlineto{\pgfqpoint{8.123648in}{2.090659in}}%
\pgfusepath{stroke}%
\end{pgfscope}%
\begin{pgfscope}%
\pgfpathrectangle{\pgfqpoint{6.720588in}{1.750000in}}{\pgfqpoint{2.279412in}{2.004545in}}%
\pgfusepath{clip}%
\pgfsetbuttcap%
\pgfsetroundjoin%
\pgfsetlinewidth{0.348697pt}%
\definecolor{currentstroke}{rgb}{0.274952,0.037752,0.364543}%
\pgfsetstrokecolor{currentstroke}%
\pgfsetdash{}{0pt}%
\pgfpathmoveto{\pgfqpoint{8.123648in}{2.090659in}}%
\pgfpathlineto{\pgfqpoint{8.074111in}{2.097480in}}%
\pgfusepath{stroke}%
\end{pgfscope}%
\begin{pgfscope}%
\pgfpathrectangle{\pgfqpoint{6.720588in}{1.750000in}}{\pgfqpoint{2.279412in}{2.004545in}}%
\pgfusepath{clip}%
\pgfsetbuttcap%
\pgfsetroundjoin%
\pgfsetlinewidth{0.322248pt}%
\definecolor{currentstroke}{rgb}{0.271305,0.019942,0.347269}%
\pgfsetstrokecolor{currentstroke}%
\pgfsetdash{}{0pt}%
\pgfpathmoveto{\pgfqpoint{8.629673in}{2.707166in}}%
\pgfpathlineto{\pgfqpoint{8.629673in}{2.707166in}}%
\pgfusepath{stroke}%
\end{pgfscope}%
\begin{pgfscope}%
\pgfpathrectangle{\pgfqpoint{6.720588in}{1.750000in}}{\pgfqpoint{2.279412in}{2.004545in}}%
\pgfusepath{clip}%
\pgfsetbuttcap%
\pgfsetroundjoin%
\pgfsetlinewidth{0.322248pt}%
\definecolor{currentstroke}{rgb}{0.271305,0.019942,0.347269}%
\pgfsetstrokecolor{currentstroke}%
\pgfsetdash{}{0pt}%
\pgfpathmoveto{\pgfqpoint{8.629673in}{2.707166in}}%
\pgfpathlineto{\pgfqpoint{8.594412in}{2.707364in}}%
\pgfusepath{stroke}%
\end{pgfscope}%
\begin{pgfscope}%
\pgfpathrectangle{\pgfqpoint{6.720588in}{1.750000in}}{\pgfqpoint{2.279412in}{2.004545in}}%
\pgfusepath{clip}%
\pgfsetbuttcap%
\pgfsetroundjoin%
\pgfsetlinewidth{0.315638pt}%
\definecolor{currentstroke}{rgb}{0.269944,0.014625,0.341379}%
\pgfsetstrokecolor{currentstroke}%
\pgfsetdash{}{0pt}%
\pgfpathmoveto{\pgfqpoint{8.594412in}{2.707364in}}%
\pgfpathlineto{\pgfqpoint{8.558743in}{2.707128in}}%
\pgfusepath{stroke}%
\end{pgfscope}%
\begin{pgfscope}%
\pgfpathrectangle{\pgfqpoint{6.720588in}{1.750000in}}{\pgfqpoint{2.279412in}{2.004545in}}%
\pgfusepath{clip}%
\pgfsetbuttcap%
\pgfsetroundjoin%
\pgfsetlinewidth{0.309852pt}%
\definecolor{currentstroke}{rgb}{0.268510,0.009605,0.335427}%
\pgfsetstrokecolor{currentstroke}%
\pgfsetdash{}{0pt}%
\pgfpathmoveto{\pgfqpoint{8.558743in}{2.707128in}}%
\pgfpathlineto{\pgfqpoint{8.509009in}{2.705296in}}%
\pgfusepath{stroke}%
\end{pgfscope}%
\begin{pgfscope}%
\pgfpathrectangle{\pgfqpoint{6.720588in}{1.750000in}}{\pgfqpoint{2.279412in}{2.004545in}}%
\pgfusepath{clip}%
\pgfsetbuttcap%
\pgfsetroundjoin%
\pgfsetlinewidth{0.323152pt}%
\definecolor{currentstroke}{rgb}{0.271305,0.019942,0.347269}%
\pgfsetstrokecolor{currentstroke}%
\pgfsetdash{}{0pt}%
\pgfpathmoveto{\pgfqpoint{8.509009in}{2.705296in}}%
\pgfpathlineto{\pgfqpoint{8.458870in}{2.704764in}}%
\pgfusepath{stroke}%
\end{pgfscope}%
\begin{pgfscope}%
\pgfpathrectangle{\pgfqpoint{6.720588in}{1.750000in}}{\pgfqpoint{2.279412in}{2.004545in}}%
\pgfusepath{clip}%
\pgfsetbuttcap%
\pgfsetroundjoin%
\pgfsetlinewidth{0.336603pt}%
\definecolor{currentstroke}{rgb}{0.273809,0.031497,0.358853}%
\pgfsetstrokecolor{currentstroke}%
\pgfsetdash{}{0pt}%
\pgfpathmoveto{\pgfqpoint{8.458870in}{2.704764in}}%
\pgfpathlineto{\pgfqpoint{8.408725in}{2.704451in}}%
\pgfusepath{stroke}%
\end{pgfscope}%
\begin{pgfscope}%
\pgfpathrectangle{\pgfqpoint{6.720588in}{1.750000in}}{\pgfqpoint{2.279412in}{2.004545in}}%
\pgfusepath{clip}%
\pgfsetbuttcap%
\pgfsetroundjoin%
\pgfsetlinewidth{0.355007pt}%
\definecolor{currentstroke}{rgb}{0.276022,0.044167,0.370164}%
\pgfsetstrokecolor{currentstroke}%
\pgfsetdash{}{0pt}%
\pgfpathmoveto{\pgfqpoint{8.408725in}{2.704451in}}%
\pgfpathlineto{\pgfqpoint{8.358576in}{2.704417in}}%
\pgfusepath{stroke}%
\end{pgfscope}%
\begin{pgfscope}%
\pgfpathrectangle{\pgfqpoint{6.720588in}{1.750000in}}{\pgfqpoint{2.279412in}{2.004545in}}%
\pgfusepath{clip}%
\pgfsetbuttcap%
\pgfsetroundjoin%
\pgfsetlinewidth{0.375013pt}%
\definecolor{currentstroke}{rgb}{0.278791,0.062145,0.386592}%
\pgfsetstrokecolor{currentstroke}%
\pgfsetdash{}{0pt}%
\pgfpathmoveto{\pgfqpoint{8.358576in}{2.704417in}}%
\pgfpathlineto{\pgfqpoint{8.308425in}{2.704315in}}%
\pgfusepath{stroke}%
\end{pgfscope}%
\begin{pgfscope}%
\pgfpathrectangle{\pgfqpoint{6.720588in}{1.750000in}}{\pgfqpoint{2.279412in}{2.004545in}}%
\pgfusepath{clip}%
\pgfsetbuttcap%
\pgfsetroundjoin%
\pgfsetlinewidth{0.420560pt}%
\definecolor{currentstroke}{rgb}{0.282656,0.100196,0.422160}%
\pgfsetstrokecolor{currentstroke}%
\pgfsetdash{}{0pt}%
\pgfpathmoveto{\pgfqpoint{8.308425in}{2.704315in}}%
\pgfpathlineto{\pgfqpoint{8.258276in}{2.703946in}}%
\pgfusepath{stroke}%
\end{pgfscope}%
\begin{pgfscope}%
\pgfpathrectangle{\pgfqpoint{6.720588in}{1.750000in}}{\pgfqpoint{2.279412in}{2.004545in}}%
\pgfusepath{clip}%
\pgfsetbuttcap%
\pgfsetroundjoin%
\pgfsetlinewidth{0.479605pt}%
\definecolor{currentstroke}{rgb}{0.282290,0.145912,0.461510}%
\pgfsetstrokecolor{currentstroke}%
\pgfsetdash{}{0pt}%
\pgfpathmoveto{\pgfqpoint{8.258276in}{2.703946in}}%
\pgfpathlineto{\pgfqpoint{8.208125in}{2.703742in}}%
\pgfusepath{stroke}%
\end{pgfscope}%
\begin{pgfscope}%
\pgfpathrectangle{\pgfqpoint{6.720588in}{1.750000in}}{\pgfqpoint{2.279412in}{2.004545in}}%
\pgfusepath{clip}%
\pgfsetbuttcap%
\pgfsetroundjoin%
\pgfsetlinewidth{0.551161pt}%
\definecolor{currentstroke}{rgb}{0.275191,0.194905,0.496005}%
\pgfsetstrokecolor{currentstroke}%
\pgfsetdash{}{0pt}%
\pgfpathmoveto{\pgfqpoint{8.208125in}{2.703742in}}%
\pgfpathlineto{\pgfqpoint{8.157973in}{2.703639in}}%
\pgfusepath{stroke}%
\end{pgfscope}%
\begin{pgfscope}%
\pgfpathrectangle{\pgfqpoint{6.720588in}{1.750000in}}{\pgfqpoint{2.279412in}{2.004545in}}%
\pgfusepath{clip}%
\pgfsetbuttcap%
\pgfsetroundjoin%
\pgfsetlinewidth{0.652734pt}%
\definecolor{currentstroke}{rgb}{0.253935,0.265254,0.529983}%
\pgfsetstrokecolor{currentstroke}%
\pgfsetdash{}{0pt}%
\pgfpathmoveto{\pgfqpoint{8.157973in}{2.703639in}}%
\pgfpathlineto{\pgfqpoint{8.107822in}{2.703477in}}%
\pgfusepath{stroke}%
\end{pgfscope}%
\begin{pgfscope}%
\pgfpathrectangle{\pgfqpoint{6.720588in}{1.750000in}}{\pgfqpoint{2.279412in}{2.004545in}}%
\pgfusepath{clip}%
\pgfsetbuttcap%
\pgfsetroundjoin%
\pgfsetlinewidth{0.755468pt}%
\definecolor{currentstroke}{rgb}{0.225863,0.330805,0.547314}%
\pgfsetstrokecolor{currentstroke}%
\pgfsetdash{}{0pt}%
\pgfpathmoveto{\pgfqpoint{8.107822in}{2.703477in}}%
\pgfpathlineto{\pgfqpoint{8.057670in}{2.703430in}}%
\pgfusepath{stroke}%
\end{pgfscope}%
\begin{pgfscope}%
\pgfpathrectangle{\pgfqpoint{6.720588in}{1.750000in}}{\pgfqpoint{2.279412in}{2.004545in}}%
\pgfusepath{clip}%
\pgfsetbuttcap%
\pgfsetroundjoin%
\pgfsetlinewidth{0.797896pt}%
\definecolor{currentstroke}{rgb}{0.214298,0.355619,0.551184}%
\pgfsetstrokecolor{currentstroke}%
\pgfsetdash{}{0pt}%
\pgfpathmoveto{\pgfqpoint{8.057670in}{2.703430in}}%
\pgfpathlineto{\pgfqpoint{8.007519in}{2.703538in}}%
\pgfusepath{stroke}%
\end{pgfscope}%
\begin{pgfscope}%
\pgfpathrectangle{\pgfqpoint{6.720588in}{1.750000in}}{\pgfqpoint{2.279412in}{2.004545in}}%
\pgfusepath{clip}%
\pgfsetbuttcap%
\pgfsetroundjoin%
\pgfsetlinewidth{0.862616pt}%
\definecolor{currentstroke}{rgb}{0.197636,0.391528,0.554969}%
\pgfsetstrokecolor{currentstroke}%
\pgfsetdash{}{0pt}%
\pgfpathmoveto{\pgfqpoint{8.007519in}{2.703538in}}%
\pgfpathlineto{\pgfqpoint{7.957368in}{2.703680in}}%
\pgfusepath{stroke}%
\end{pgfscope}%
\begin{pgfscope}%
\pgfpathrectangle{\pgfqpoint{6.720588in}{1.750000in}}{\pgfqpoint{2.279412in}{2.004545in}}%
\pgfusepath{clip}%
\pgfsetbuttcap%
\pgfsetroundjoin%
\pgfsetlinewidth{0.860686pt}%
\definecolor{currentstroke}{rgb}{0.197636,0.391528,0.554969}%
\pgfsetstrokecolor{currentstroke}%
\pgfsetdash{}{0pt}%
\pgfpathmoveto{\pgfqpoint{7.957368in}{2.703680in}}%
\pgfpathlineto{\pgfqpoint{7.907218in}{2.703940in}}%
\pgfusepath{stroke}%
\end{pgfscope}%
\begin{pgfscope}%
\pgfpathrectangle{\pgfqpoint{6.720588in}{1.750000in}}{\pgfqpoint{2.279412in}{2.004545in}}%
\pgfusepath{clip}%
\pgfsetbuttcap%
\pgfsetroundjoin%
\pgfsetlinewidth{0.864542pt}%
\definecolor{currentstroke}{rgb}{0.195860,0.395433,0.555276}%
\pgfsetstrokecolor{currentstroke}%
\pgfsetdash{}{0pt}%
\pgfpathmoveto{\pgfqpoint{7.907218in}{2.703940in}}%
\pgfpathlineto{\pgfqpoint{7.857068in}{2.704364in}}%
\pgfusepath{stroke}%
\end{pgfscope}%
\begin{pgfscope}%
\pgfpathrectangle{\pgfqpoint{6.720588in}{1.750000in}}{\pgfqpoint{2.279412in}{2.004545in}}%
\pgfusepath{clip}%
\pgfsetbuttcap%
\pgfsetroundjoin%
\pgfsetlinewidth{0.788148pt}%
\definecolor{currentstroke}{rgb}{0.218130,0.347432,0.550038}%
\pgfsetstrokecolor{currentstroke}%
\pgfsetdash{}{0pt}%
\pgfpathmoveto{\pgfqpoint{7.857068in}{2.704364in}}%
\pgfpathlineto{\pgfqpoint{7.806925in}{2.705080in}}%
\pgfusepath{stroke}%
\end{pgfscope}%
\begin{pgfscope}%
\pgfpathrectangle{\pgfqpoint{6.720588in}{1.750000in}}{\pgfqpoint{2.279412in}{2.004545in}}%
\pgfusepath{clip}%
\pgfsetbuttcap%
\pgfsetroundjoin%
\pgfsetlinewidth{0.800799pt}%
\definecolor{currentstroke}{rgb}{0.214298,0.355619,0.551184}%
\pgfsetstrokecolor{currentstroke}%
\pgfsetdash{}{0pt}%
\pgfpathmoveto{\pgfqpoint{7.806925in}{2.705080in}}%
\pgfpathlineto{\pgfqpoint{7.756788in}{2.706070in}}%
\pgfusepath{stroke}%
\end{pgfscope}%
\begin{pgfscope}%
\pgfpathrectangle{\pgfqpoint{6.720588in}{1.750000in}}{\pgfqpoint{2.279412in}{2.004545in}}%
\pgfusepath{clip}%
\pgfsetbuttcap%
\pgfsetroundjoin%
\pgfsetlinewidth{0.724227pt}%
\definecolor{currentstroke}{rgb}{0.235526,0.309527,0.542944}%
\pgfsetstrokecolor{currentstroke}%
\pgfsetdash{}{0pt}%
\pgfpathmoveto{\pgfqpoint{7.756788in}{2.706070in}}%
\pgfpathlineto{\pgfqpoint{7.706660in}{2.707315in}}%
\pgfusepath{stroke}%
\end{pgfscope}%
\begin{pgfscope}%
\pgfpathrectangle{\pgfqpoint{6.720588in}{1.750000in}}{\pgfqpoint{2.279412in}{2.004545in}}%
\pgfusepath{clip}%
\pgfsetbuttcap%
\pgfsetroundjoin%
\pgfsetlinewidth{0.648490pt}%
\definecolor{currentstroke}{rgb}{0.255645,0.260703,0.528312}%
\pgfsetstrokecolor{currentstroke}%
\pgfsetdash{}{0pt}%
\pgfpathmoveto{\pgfqpoint{7.706660in}{2.707315in}}%
\pgfpathlineto{\pgfqpoint{7.656547in}{2.708992in}}%
\pgfusepath{stroke}%
\end{pgfscope}%
\begin{pgfscope}%
\pgfpathrectangle{\pgfqpoint{6.720588in}{1.750000in}}{\pgfqpoint{2.279412in}{2.004545in}}%
\pgfusepath{clip}%
\pgfsetbuttcap%
\pgfsetroundjoin%
\pgfsetlinewidth{0.578755pt}%
\definecolor{currentstroke}{rgb}{0.270595,0.214069,0.507052}%
\pgfsetstrokecolor{currentstroke}%
\pgfsetdash{}{0pt}%
\pgfpathmoveto{\pgfqpoint{7.656547in}{2.708992in}}%
\pgfpathlineto{\pgfqpoint{7.606452in}{2.711047in}}%
\pgfusepath{stroke}%
\end{pgfscope}%
\begin{pgfscope}%
\pgfpathrectangle{\pgfqpoint{6.720588in}{1.750000in}}{\pgfqpoint{2.279412in}{2.004545in}}%
\pgfusepath{clip}%
\pgfsetbuttcap%
\pgfsetroundjoin%
\pgfsetlinewidth{0.577729pt}%
\definecolor{currentstroke}{rgb}{0.270595,0.214069,0.507052}%
\pgfsetstrokecolor{currentstroke}%
\pgfsetdash{}{0pt}%
\pgfpathmoveto{\pgfqpoint{7.606452in}{2.711047in}}%
\pgfpathlineto{\pgfqpoint{7.556364in}{2.713182in}}%
\pgfusepath{stroke}%
\end{pgfscope}%
\begin{pgfscope}%
\pgfpathrectangle{\pgfqpoint{6.720588in}{1.750000in}}{\pgfqpoint{2.279412in}{2.004545in}}%
\pgfusepath{clip}%
\pgfsetbuttcap%
\pgfsetroundjoin%
\pgfsetlinewidth{0.539850pt}%
\definecolor{currentstroke}{rgb}{0.277134,0.185228,0.489898}%
\pgfsetstrokecolor{currentstroke}%
\pgfsetdash{}{0pt}%
\pgfpathmoveto{\pgfqpoint{7.556364in}{2.713182in}}%
\pgfpathlineto{\pgfqpoint{7.506310in}{2.715701in}}%
\pgfusepath{stroke}%
\end{pgfscope}%
\begin{pgfscope}%
\pgfpathrectangle{\pgfqpoint{6.720588in}{1.750000in}}{\pgfqpoint{2.279412in}{2.004545in}}%
\pgfusepath{clip}%
\pgfsetbuttcap%
\pgfsetroundjoin%
\pgfsetlinewidth{0.316636pt}%
\definecolor{currentstroke}{rgb}{0.269944,0.014625,0.341379}%
\pgfsetstrokecolor{currentstroke}%
\pgfsetdash{}{0pt}%
\pgfpathmoveto{\pgfqpoint{8.629673in}{2.752273in}}%
\pgfpathlineto{\pgfqpoint{8.579524in}{2.751819in}}%
\pgfusepath{stroke}%
\end{pgfscope}%
\begin{pgfscope}%
\pgfpathrectangle{\pgfqpoint{6.720588in}{1.750000in}}{\pgfqpoint{2.279412in}{2.004545in}}%
\pgfusepath{clip}%
\pgfsetbuttcap%
\pgfsetroundjoin%
\pgfsetlinewidth{0.313765pt}%
\definecolor{currentstroke}{rgb}{0.268510,0.009605,0.335427}%
\pgfsetstrokecolor{currentstroke}%
\pgfsetdash{}{0pt}%
\pgfpathmoveto{\pgfqpoint{8.579524in}{2.751819in}}%
\pgfpathlineto{\pgfqpoint{8.529432in}{2.751245in}}%
\pgfusepath{stroke}%
\end{pgfscope}%
\begin{pgfscope}%
\pgfpathrectangle{\pgfqpoint{6.720588in}{1.750000in}}{\pgfqpoint{2.279412in}{2.004545in}}%
\pgfusepath{clip}%
\pgfsetbuttcap%
\pgfsetroundjoin%
\pgfsetlinewidth{0.320566pt}%
\definecolor{currentstroke}{rgb}{0.269944,0.014625,0.341379}%
\pgfsetstrokecolor{currentstroke}%
\pgfsetdash{}{0pt}%
\pgfpathmoveto{\pgfqpoint{8.529432in}{2.751245in}}%
\pgfpathlineto{\pgfqpoint{8.479338in}{2.750894in}}%
\pgfusepath{stroke}%
\end{pgfscope}%
\begin{pgfscope}%
\pgfpathrectangle{\pgfqpoint{6.720588in}{1.750000in}}{\pgfqpoint{2.279412in}{2.004545in}}%
\pgfusepath{clip}%
\pgfsetbuttcap%
\pgfsetroundjoin%
\pgfsetlinewidth{0.329863pt}%
\definecolor{currentstroke}{rgb}{0.272594,0.025563,0.353093}%
\pgfsetstrokecolor{currentstroke}%
\pgfsetdash{}{0pt}%
\pgfpathmoveto{\pgfqpoint{8.479338in}{2.750894in}}%
\pgfpathlineto{\pgfqpoint{8.429195in}{2.750316in}}%
\pgfusepath{stroke}%
\end{pgfscope}%
\begin{pgfscope}%
\pgfpathrectangle{\pgfqpoint{6.720588in}{1.750000in}}{\pgfqpoint{2.279412in}{2.004545in}}%
\pgfusepath{clip}%
\pgfsetbuttcap%
\pgfsetroundjoin%
\pgfsetlinewidth{0.338714pt}%
\definecolor{currentstroke}{rgb}{0.273809,0.031497,0.358853}%
\pgfsetstrokecolor{currentstroke}%
\pgfsetdash{}{0pt}%
\pgfpathmoveto{\pgfqpoint{8.429195in}{2.750316in}}%
\pgfpathlineto{\pgfqpoint{8.379051in}{2.749677in}}%
\pgfusepath{stroke}%
\end{pgfscope}%
\begin{pgfscope}%
\pgfpathrectangle{\pgfqpoint{6.720588in}{1.750000in}}{\pgfqpoint{2.279412in}{2.004545in}}%
\pgfusepath{clip}%
\pgfsetbuttcap%
\pgfsetroundjoin%
\pgfsetlinewidth{0.361450pt}%
\definecolor{currentstroke}{rgb}{0.277018,0.050344,0.375715}%
\pgfsetstrokecolor{currentstroke}%
\pgfsetdash{}{0pt}%
\pgfpathmoveto{\pgfqpoint{8.379051in}{2.749677in}}%
\pgfpathlineto{\pgfqpoint{8.328900in}{2.749490in}}%
\pgfusepath{stroke}%
\end{pgfscope}%
\begin{pgfscope}%
\pgfpathrectangle{\pgfqpoint{6.720588in}{1.750000in}}{\pgfqpoint{2.279412in}{2.004545in}}%
\pgfusepath{clip}%
\pgfsetbuttcap%
\pgfsetroundjoin%
\pgfsetlinewidth{0.405281pt}%
\definecolor{currentstroke}{rgb}{0.281924,0.089666,0.412415}%
\pgfsetstrokecolor{currentstroke}%
\pgfsetdash{}{0pt}%
\pgfpathmoveto{\pgfqpoint{8.328900in}{2.749490in}}%
\pgfpathlineto{\pgfqpoint{8.278748in}{2.749340in}}%
\pgfusepath{stroke}%
\end{pgfscope}%
\begin{pgfscope}%
\pgfpathrectangle{\pgfqpoint{6.720588in}{1.750000in}}{\pgfqpoint{2.279412in}{2.004545in}}%
\pgfusepath{clip}%
\pgfsetbuttcap%
\pgfsetroundjoin%
\pgfsetlinewidth{0.445045pt}%
\definecolor{currentstroke}{rgb}{0.283197,0.115680,0.436115}%
\pgfsetstrokecolor{currentstroke}%
\pgfsetdash{}{0pt}%
\pgfpathmoveto{\pgfqpoint{8.278748in}{2.749340in}}%
\pgfpathlineto{\pgfqpoint{8.228597in}{2.749154in}}%
\pgfusepath{stroke}%
\end{pgfscope}%
\begin{pgfscope}%
\pgfpathrectangle{\pgfqpoint{6.720588in}{1.750000in}}{\pgfqpoint{2.279412in}{2.004545in}}%
\pgfusepath{clip}%
\pgfsetbuttcap%
\pgfsetroundjoin%
\pgfsetlinewidth{0.510426pt}%
\definecolor{currentstroke}{rgb}{0.280255,0.165693,0.476498}%
\pgfsetstrokecolor{currentstroke}%
\pgfsetdash{}{0pt}%
\pgfpathmoveto{\pgfqpoint{8.228597in}{2.749154in}}%
\pgfpathlineto{\pgfqpoint{8.178447in}{2.748821in}}%
\pgfusepath{stroke}%
\end{pgfscope}%
\begin{pgfscope}%
\pgfpathrectangle{\pgfqpoint{6.720588in}{1.750000in}}{\pgfqpoint{2.279412in}{2.004545in}}%
\pgfusepath{clip}%
\pgfsetbuttcap%
\pgfsetroundjoin%
\pgfsetlinewidth{0.579044pt}%
\definecolor{currentstroke}{rgb}{0.270595,0.214069,0.507052}%
\pgfsetstrokecolor{currentstroke}%
\pgfsetdash{}{0pt}%
\pgfpathmoveto{\pgfqpoint{8.178447in}{2.748821in}}%
\pgfpathlineto{\pgfqpoint{8.128298in}{2.748388in}}%
\pgfusepath{stroke}%
\end{pgfscope}%
\begin{pgfscope}%
\pgfpathrectangle{\pgfqpoint{6.720588in}{1.750000in}}{\pgfqpoint{2.279412in}{2.004545in}}%
\pgfusepath{clip}%
\pgfsetbuttcap%
\pgfsetroundjoin%
\pgfsetlinewidth{0.702240pt}%
\definecolor{currentstroke}{rgb}{0.241237,0.296485,0.539709}%
\pgfsetstrokecolor{currentstroke}%
\pgfsetdash{}{0pt}%
\pgfpathmoveto{\pgfqpoint{8.128298in}{2.748388in}}%
\pgfpathlineto{\pgfqpoint{8.078148in}{2.748026in}}%
\pgfusepath{stroke}%
\end{pgfscope}%
\begin{pgfscope}%
\pgfpathrectangle{\pgfqpoint{6.720588in}{1.750000in}}{\pgfqpoint{2.279412in}{2.004545in}}%
\pgfusepath{clip}%
\pgfsetbuttcap%
\pgfsetroundjoin%
\pgfsetlinewidth{0.778730pt}%
\definecolor{currentstroke}{rgb}{0.220057,0.343307,0.549413}%
\pgfsetstrokecolor{currentstroke}%
\pgfsetdash{}{0pt}%
\pgfpathmoveto{\pgfqpoint{8.078148in}{2.748026in}}%
\pgfpathlineto{\pgfqpoint{8.027999in}{2.747558in}}%
\pgfusepath{stroke}%
\end{pgfscope}%
\begin{pgfscope}%
\pgfpathrectangle{\pgfqpoint{6.720588in}{1.750000in}}{\pgfqpoint{2.279412in}{2.004545in}}%
\pgfusepath{clip}%
\pgfsetbuttcap%
\pgfsetroundjoin%
\pgfsetlinewidth{0.816439pt}%
\definecolor{currentstroke}{rgb}{0.208623,0.367752,0.552675}%
\pgfsetstrokecolor{currentstroke}%
\pgfsetdash{}{0pt}%
\pgfpathmoveto{\pgfqpoint{8.027999in}{2.747558in}}%
\pgfpathlineto{\pgfqpoint{7.977852in}{2.746940in}}%
\pgfusepath{stroke}%
\end{pgfscope}%
\begin{pgfscope}%
\pgfpathrectangle{\pgfqpoint{6.720588in}{1.750000in}}{\pgfqpoint{2.279412in}{2.004545in}}%
\pgfusepath{clip}%
\pgfsetbuttcap%
\pgfsetroundjoin%
\pgfsetlinewidth{0.875187pt}%
\definecolor{currentstroke}{rgb}{0.194100,0.399323,0.555565}%
\pgfsetstrokecolor{currentstroke}%
\pgfsetdash{}{0pt}%
\pgfpathmoveto{\pgfqpoint{7.977852in}{2.746940in}}%
\pgfpathlineto{\pgfqpoint{7.927706in}{2.746325in}}%
\pgfusepath{stroke}%
\end{pgfscope}%
\begin{pgfscope}%
\pgfpathrectangle{\pgfqpoint{6.720588in}{1.750000in}}{\pgfqpoint{2.279412in}{2.004545in}}%
\pgfusepath{clip}%
\pgfsetbuttcap%
\pgfsetroundjoin%
\pgfsetlinewidth{0.864321pt}%
\definecolor{currentstroke}{rgb}{0.195860,0.395433,0.555276}%
\pgfsetstrokecolor{currentstroke}%
\pgfsetdash{}{0pt}%
\pgfpathmoveto{\pgfqpoint{7.927706in}{2.746325in}}%
\pgfpathlineto{\pgfqpoint{7.877560in}{2.745705in}}%
\pgfusepath{stroke}%
\end{pgfscope}%
\begin{pgfscope}%
\pgfpathrectangle{\pgfqpoint{6.720588in}{1.750000in}}{\pgfqpoint{2.279412in}{2.004545in}}%
\pgfusepath{clip}%
\pgfsetbuttcap%
\pgfsetroundjoin%
\pgfsetlinewidth{0.824265pt}%
\definecolor{currentstroke}{rgb}{0.206756,0.371758,0.553117}%
\pgfsetstrokecolor{currentstroke}%
\pgfsetdash{}{0pt}%
\pgfpathmoveto{\pgfqpoint{7.877560in}{2.745705in}}%
\pgfpathlineto{\pgfqpoint{7.827416in}{2.745033in}}%
\pgfusepath{stroke}%
\end{pgfscope}%
\begin{pgfscope}%
\pgfpathrectangle{\pgfqpoint{6.720588in}{1.750000in}}{\pgfqpoint{2.279412in}{2.004545in}}%
\pgfusepath{clip}%
\pgfsetbuttcap%
\pgfsetroundjoin%
\pgfsetlinewidth{0.809475pt}%
\definecolor{currentstroke}{rgb}{0.210503,0.363727,0.552206}%
\pgfsetstrokecolor{currentstroke}%
\pgfsetdash{}{0pt}%
\pgfpathmoveto{\pgfqpoint{7.827416in}{2.745033in}}%
\pgfpathlineto{\pgfqpoint{7.777272in}{2.744276in}}%
\pgfusepath{stroke}%
\end{pgfscope}%
\begin{pgfscope}%
\pgfpathrectangle{\pgfqpoint{6.720588in}{1.750000in}}{\pgfqpoint{2.279412in}{2.004545in}}%
\pgfusepath{clip}%
\pgfsetbuttcap%
\pgfsetroundjoin%
\pgfsetlinewidth{0.723136pt}%
\definecolor{currentstroke}{rgb}{0.235526,0.309527,0.542944}%
\pgfsetstrokecolor{currentstroke}%
\pgfsetdash{}{0pt}%
\pgfpathmoveto{\pgfqpoint{7.777272in}{2.744276in}}%
\pgfpathlineto{\pgfqpoint{7.727151in}{2.743271in}}%
\pgfusepath{stroke}%
\end{pgfscope}%
\begin{pgfscope}%
\pgfpathrectangle{\pgfqpoint{6.720588in}{1.750000in}}{\pgfqpoint{2.279412in}{2.004545in}}%
\pgfusepath{clip}%
\pgfsetbuttcap%
\pgfsetroundjoin%
\pgfsetlinewidth{0.659719pt}%
\definecolor{currentstroke}{rgb}{0.252194,0.269783,0.531579}%
\pgfsetstrokecolor{currentstroke}%
\pgfsetdash{}{0pt}%
\pgfpathmoveto{\pgfqpoint{7.727151in}{2.743271in}}%
\pgfpathlineto{\pgfqpoint{7.677059in}{2.741798in}}%
\pgfusepath{stroke}%
\end{pgfscope}%
\begin{pgfscope}%
\pgfpathrectangle{\pgfqpoint{6.720588in}{1.750000in}}{\pgfqpoint{2.279412in}{2.004545in}}%
\pgfusepath{clip}%
\pgfsetbuttcap%
\pgfsetroundjoin%
\pgfsetlinewidth{0.607084pt}%
\definecolor{currentstroke}{rgb}{0.265145,0.232956,0.516599}%
\pgfsetstrokecolor{currentstroke}%
\pgfsetdash{}{0pt}%
\pgfpathmoveto{\pgfqpoint{7.677059in}{2.741798in}}%
\pgfpathlineto{\pgfqpoint{7.626977in}{2.740020in}}%
\pgfusepath{stroke}%
\end{pgfscope}%
\begin{pgfscope}%
\pgfpathrectangle{\pgfqpoint{6.720588in}{1.750000in}}{\pgfqpoint{2.279412in}{2.004545in}}%
\pgfusepath{clip}%
\pgfsetbuttcap%
\pgfsetroundjoin%
\pgfsetlinewidth{0.575731pt}%
\definecolor{currentstroke}{rgb}{0.270595,0.214069,0.507052}%
\pgfsetstrokecolor{currentstroke}%
\pgfsetdash{}{0pt}%
\pgfpathmoveto{\pgfqpoint{7.626977in}{2.740020in}}%
\pgfpathlineto{\pgfqpoint{7.576907in}{2.738136in}}%
\pgfusepath{stroke}%
\end{pgfscope}%
\begin{pgfscope}%
\pgfpathrectangle{\pgfqpoint{6.720588in}{1.750000in}}{\pgfqpoint{2.279412in}{2.004545in}}%
\pgfusepath{clip}%
\pgfsetbuttcap%
\pgfsetroundjoin%
\pgfsetlinewidth{0.572076pt}%
\definecolor{currentstroke}{rgb}{0.271828,0.209303,0.504434}%
\pgfsetstrokecolor{currentstroke}%
\pgfsetdash{}{0pt}%
\pgfpathmoveto{\pgfqpoint{7.576907in}{2.738136in}}%
\pgfpathlineto{\pgfqpoint{7.526850in}{2.736300in}}%
\pgfusepath{stroke}%
\end{pgfscope}%
\begin{pgfscope}%
\pgfpathrectangle{\pgfqpoint{6.720588in}{1.750000in}}{\pgfqpoint{2.279412in}{2.004545in}}%
\pgfusepath{clip}%
\pgfsetbuttcap%
\pgfsetroundjoin%
\pgfsetlinewidth{0.507861pt}%
\definecolor{currentstroke}{rgb}{0.280255,0.165693,0.476498}%
\pgfsetstrokecolor{currentstroke}%
\pgfsetdash{}{0pt}%
\pgfpathmoveto{\pgfqpoint{7.526850in}{2.736300in}}%
\pgfpathlineto{\pgfqpoint{7.476910in}{2.735042in}}%
\pgfusepath{stroke}%
\end{pgfscope}%
\begin{pgfscope}%
\pgfpathrectangle{\pgfqpoint{6.720588in}{1.750000in}}{\pgfqpoint{2.279412in}{2.004545in}}%
\pgfusepath{clip}%
\pgfsetbuttcap%
\pgfsetroundjoin%
\pgfsetlinewidth{0.404150pt}%
\definecolor{currentstroke}{rgb}{0.281924,0.089666,0.412415}%
\pgfsetstrokecolor{currentstroke}%
\pgfsetdash{}{0pt}%
\pgfpathmoveto{\pgfqpoint{7.476910in}{2.735042in}}%
\pgfpathlineto{\pgfqpoint{7.476910in}{2.735042in}}%
\pgfusepath{stroke}%
\end{pgfscope}%
\begin{pgfscope}%
\pgfpathrectangle{\pgfqpoint{6.720588in}{1.750000in}}{\pgfqpoint{2.279412in}{2.004545in}}%
\pgfusepath{clip}%
\pgfsetbuttcap%
\pgfsetroundjoin%
\pgfsetlinewidth{0.304353pt}%
\definecolor{currentstroke}{rgb}{0.267004,0.004874,0.329415}%
\pgfsetstrokecolor{currentstroke}%
\pgfsetdash{}{0pt}%
\pgfpathmoveto{\pgfqpoint{8.629673in}{2.932700in}}%
\pgfpathlineto{\pgfqpoint{8.585252in}{2.935656in}}%
\pgfusepath{stroke}%
\end{pgfscope}%
\begin{pgfscope}%
\pgfpathrectangle{\pgfqpoint{6.720588in}{1.750000in}}{\pgfqpoint{2.279412in}{2.004545in}}%
\pgfusepath{clip}%
\pgfsetbuttcap%
\pgfsetroundjoin%
\pgfsetlinewidth{0.326856pt}%
\definecolor{currentstroke}{rgb}{0.271305,0.019942,0.347269}%
\pgfsetstrokecolor{currentstroke}%
\pgfsetdash{}{0pt}%
\pgfpathmoveto{\pgfqpoint{8.585252in}{2.935656in}}%
\pgfpathlineto{\pgfqpoint{8.540726in}{2.936257in}}%
\pgfusepath{stroke}%
\end{pgfscope}%
\begin{pgfscope}%
\pgfpathrectangle{\pgfqpoint{6.720588in}{1.750000in}}{\pgfqpoint{2.279412in}{2.004545in}}%
\pgfusepath{clip}%
\pgfsetbuttcap%
\pgfsetroundjoin%
\pgfsetlinewidth{0.320589pt}%
\definecolor{currentstroke}{rgb}{0.269944,0.014625,0.341379}%
\pgfsetstrokecolor{currentstroke}%
\pgfsetdash{}{0pt}%
\pgfpathmoveto{\pgfqpoint{8.540726in}{2.936257in}}%
\pgfpathlineto{\pgfqpoint{8.490579in}{2.936092in}}%
\pgfusepath{stroke}%
\end{pgfscope}%
\begin{pgfscope}%
\pgfpathrectangle{\pgfqpoint{6.720588in}{1.750000in}}{\pgfqpoint{2.279412in}{2.004545in}}%
\pgfusepath{clip}%
\pgfsetbuttcap%
\pgfsetroundjoin%
\pgfsetlinewidth{0.323365pt}%
\definecolor{currentstroke}{rgb}{0.271305,0.019942,0.347269}%
\pgfsetstrokecolor{currentstroke}%
\pgfsetdash{}{0pt}%
\pgfpathmoveto{\pgfqpoint{8.490579in}{2.936092in}}%
\pgfpathlineto{\pgfqpoint{8.440436in}{2.935604in}}%
\pgfusepath{stroke}%
\end{pgfscope}%
\begin{pgfscope}%
\pgfpathrectangle{\pgfqpoint{6.720588in}{1.750000in}}{\pgfqpoint{2.279412in}{2.004545in}}%
\pgfusepath{clip}%
\pgfsetbuttcap%
\pgfsetroundjoin%
\pgfsetlinewidth{0.336142pt}%
\definecolor{currentstroke}{rgb}{0.273809,0.031497,0.358853}%
\pgfsetstrokecolor{currentstroke}%
\pgfsetdash{}{0pt}%
\pgfpathmoveto{\pgfqpoint{8.440436in}{2.935604in}}%
\pgfpathlineto{\pgfqpoint{8.390295in}{2.935002in}}%
\pgfusepath{stroke}%
\end{pgfscope}%
\begin{pgfscope}%
\pgfpathrectangle{\pgfqpoint{6.720588in}{1.750000in}}{\pgfqpoint{2.279412in}{2.004545in}}%
\pgfusepath{clip}%
\pgfsetbuttcap%
\pgfsetroundjoin%
\pgfsetlinewidth{0.350147pt}%
\definecolor{currentstroke}{rgb}{0.276022,0.044167,0.370164}%
\pgfsetstrokecolor{currentstroke}%
\pgfsetdash{}{0pt}%
\pgfpathmoveto{\pgfqpoint{8.390295in}{2.935002in}}%
\pgfpathlineto{\pgfqpoint{8.340149in}{2.934580in}}%
\pgfusepath{stroke}%
\end{pgfscope}%
\begin{pgfscope}%
\pgfpathrectangle{\pgfqpoint{6.720588in}{1.750000in}}{\pgfqpoint{2.279412in}{2.004545in}}%
\pgfusepath{clip}%
\pgfsetbuttcap%
\pgfsetroundjoin%
\pgfsetlinewidth{0.391805pt}%
\definecolor{currentstroke}{rgb}{0.280894,0.078907,0.402329}%
\pgfsetstrokecolor{currentstroke}%
\pgfsetdash{}{0pt}%
\pgfpathmoveto{\pgfqpoint{8.340149in}{2.934580in}}%
\pgfpathlineto{\pgfqpoint{8.290005in}{2.933873in}}%
\pgfusepath{stroke}%
\end{pgfscope}%
\begin{pgfscope}%
\pgfpathrectangle{\pgfqpoint{6.720588in}{1.750000in}}{\pgfqpoint{2.279412in}{2.004545in}}%
\pgfusepath{clip}%
\pgfsetbuttcap%
\pgfsetroundjoin%
\pgfsetlinewidth{0.428084pt}%
\definecolor{currentstroke}{rgb}{0.282910,0.105393,0.426902}%
\pgfsetstrokecolor{currentstroke}%
\pgfsetdash{}{0pt}%
\pgfpathmoveto{\pgfqpoint{8.290005in}{2.933873in}}%
\pgfpathlineto{\pgfqpoint{8.239862in}{2.933128in}}%
\pgfusepath{stroke}%
\end{pgfscope}%
\begin{pgfscope}%
\pgfpathrectangle{\pgfqpoint{6.720588in}{1.750000in}}{\pgfqpoint{2.279412in}{2.004545in}}%
\pgfusepath{clip}%
\pgfsetbuttcap%
\pgfsetroundjoin%
\pgfsetlinewidth{0.454722pt}%
\definecolor{currentstroke}{rgb}{0.283187,0.125848,0.444960}%
\pgfsetstrokecolor{currentstroke}%
\pgfsetdash{}{0pt}%
\pgfpathmoveto{\pgfqpoint{8.239862in}{2.933128in}}%
\pgfpathlineto{\pgfqpoint{8.189721in}{2.932293in}}%
\pgfusepath{stroke}%
\end{pgfscope}%
\begin{pgfscope}%
\pgfpathrectangle{\pgfqpoint{6.720588in}{1.750000in}}{\pgfqpoint{2.279412in}{2.004545in}}%
\pgfusepath{clip}%
\pgfsetbuttcap%
\pgfsetroundjoin%
\pgfsetlinewidth{0.513485pt}%
\definecolor{currentstroke}{rgb}{0.280255,0.165693,0.476498}%
\pgfsetstrokecolor{currentstroke}%
\pgfsetdash{}{0pt}%
\pgfpathmoveto{\pgfqpoint{8.189721in}{2.932293in}}%
\pgfpathlineto{\pgfqpoint{8.139605in}{2.930678in}}%
\pgfusepath{stroke}%
\end{pgfscope}%
\begin{pgfscope}%
\pgfpathrectangle{\pgfqpoint{6.720588in}{1.750000in}}{\pgfqpoint{2.279412in}{2.004545in}}%
\pgfusepath{clip}%
\pgfsetbuttcap%
\pgfsetroundjoin%
\pgfsetlinewidth{0.568939pt}%
\definecolor{currentstroke}{rgb}{0.271828,0.209303,0.504434}%
\pgfsetstrokecolor{currentstroke}%
\pgfsetdash{}{0pt}%
\pgfpathmoveto{\pgfqpoint{8.139605in}{2.930678in}}%
\pgfpathlineto{\pgfqpoint{8.089510in}{2.928600in}}%
\pgfusepath{stroke}%
\end{pgfscope}%
\begin{pgfscope}%
\pgfpathrectangle{\pgfqpoint{6.720588in}{1.750000in}}{\pgfqpoint{2.279412in}{2.004545in}}%
\pgfusepath{clip}%
\pgfsetbuttcap%
\pgfsetroundjoin%
\pgfsetlinewidth{0.638158pt}%
\definecolor{currentstroke}{rgb}{0.257322,0.256130,0.526563}%
\pgfsetstrokecolor{currentstroke}%
\pgfsetdash{}{0pt}%
\pgfpathmoveto{\pgfqpoint{8.089510in}{2.928600in}}%
\pgfpathlineto{\pgfqpoint{8.039456in}{2.925902in}}%
\pgfusepath{stroke}%
\end{pgfscope}%
\begin{pgfscope}%
\pgfpathrectangle{\pgfqpoint{6.720588in}{1.750000in}}{\pgfqpoint{2.279412in}{2.004545in}}%
\pgfusepath{clip}%
\pgfsetbuttcap%
\pgfsetroundjoin%
\pgfsetlinewidth{0.683123pt}%
\definecolor{currentstroke}{rgb}{0.246811,0.283237,0.535941}%
\pgfsetstrokecolor{currentstroke}%
\pgfsetdash{}{0pt}%
\pgfpathmoveto{\pgfqpoint{8.039456in}{2.925902in}}%
\pgfpathlineto{\pgfqpoint{7.989456in}{2.922487in}}%
\pgfusepath{stroke}%
\end{pgfscope}%
\begin{pgfscope}%
\pgfpathrectangle{\pgfqpoint{6.720588in}{1.750000in}}{\pgfqpoint{2.279412in}{2.004545in}}%
\pgfusepath{clip}%
\pgfsetbuttcap%
\pgfsetroundjoin%
\pgfsetlinewidth{0.704879pt}%
\definecolor{currentstroke}{rgb}{0.241237,0.296485,0.539709}%
\pgfsetstrokecolor{currentstroke}%
\pgfsetdash{}{0pt}%
\pgfpathmoveto{\pgfqpoint{7.989456in}{2.922487in}}%
\pgfpathlineto{\pgfqpoint{7.939519in}{2.918458in}}%
\pgfusepath{stroke}%
\end{pgfscope}%
\begin{pgfscope}%
\pgfpathrectangle{\pgfqpoint{6.720588in}{1.750000in}}{\pgfqpoint{2.279412in}{2.004545in}}%
\pgfusepath{clip}%
\pgfsetbuttcap%
\pgfsetroundjoin%
\pgfsetlinewidth{0.723325pt}%
\definecolor{currentstroke}{rgb}{0.235526,0.309527,0.542944}%
\pgfsetstrokecolor{currentstroke}%
\pgfsetdash{}{0pt}%
\pgfpathmoveto{\pgfqpoint{7.939519in}{2.918458in}}%
\pgfpathlineto{\pgfqpoint{7.889719in}{2.913297in}}%
\pgfusepath{stroke}%
\end{pgfscope}%
\begin{pgfscope}%
\pgfpathrectangle{\pgfqpoint{6.720588in}{1.750000in}}{\pgfqpoint{2.279412in}{2.004545in}}%
\pgfusepath{clip}%
\pgfsetbuttcap%
\pgfsetroundjoin%
\pgfsetlinewidth{0.760612pt}%
\definecolor{currentstroke}{rgb}{0.225863,0.330805,0.547314}%
\pgfsetstrokecolor{currentstroke}%
\pgfsetdash{}{0pt}%
\pgfpathmoveto{\pgfqpoint{7.889719in}{2.913297in}}%
\pgfpathlineto{\pgfqpoint{7.840122in}{2.906797in}}%
\pgfusepath{stroke}%
\end{pgfscope}%
\begin{pgfscope}%
\pgfpathrectangle{\pgfqpoint{6.720588in}{1.750000in}}{\pgfqpoint{2.279412in}{2.004545in}}%
\pgfusepath{clip}%
\pgfsetbuttcap%
\pgfsetroundjoin%
\pgfsetlinewidth{0.709423pt}%
\definecolor{currentstroke}{rgb}{0.239346,0.300855,0.540844}%
\pgfsetstrokecolor{currentstroke}%
\pgfsetdash{}{0pt}%
\pgfpathmoveto{\pgfqpoint{7.840122in}{2.906797in}}%
\pgfpathlineto{\pgfqpoint{7.790797in}{2.898879in}}%
\pgfusepath{stroke}%
\end{pgfscope}%
\begin{pgfscope}%
\pgfpathrectangle{\pgfqpoint{6.720588in}{1.750000in}}{\pgfqpoint{2.279412in}{2.004545in}}%
\pgfusepath{clip}%
\pgfsetbuttcap%
\pgfsetroundjoin%
\pgfsetlinewidth{0.639896pt}%
\definecolor{currentstroke}{rgb}{0.257322,0.256130,0.526563}%
\pgfsetstrokecolor{currentstroke}%
\pgfsetdash{}{0pt}%
\pgfpathmoveto{\pgfqpoint{7.790797in}{2.898879in}}%
\pgfpathlineto{\pgfqpoint{7.741987in}{2.888870in}}%
\pgfusepath{stroke}%
\end{pgfscope}%
\begin{pgfscope}%
\pgfpathrectangle{\pgfqpoint{6.720588in}{1.750000in}}{\pgfqpoint{2.279412in}{2.004545in}}%
\pgfusepath{clip}%
\pgfsetbuttcap%
\pgfsetroundjoin%
\pgfsetlinewidth{0.635352pt}%
\definecolor{currentstroke}{rgb}{0.258965,0.251537,0.524736}%
\pgfsetstrokecolor{currentstroke}%
\pgfsetdash{}{0pt}%
\pgfpathmoveto{\pgfqpoint{7.741987in}{2.888870in}}%
\pgfpathlineto{\pgfqpoint{7.693908in}{2.876439in}}%
\pgfusepath{stroke}%
\end{pgfscope}%
\begin{pgfscope}%
\pgfpathrectangle{\pgfqpoint{6.720588in}{1.750000in}}{\pgfqpoint{2.279412in}{2.004545in}}%
\pgfusepath{clip}%
\pgfsetbuttcap%
\pgfsetroundjoin%
\pgfsetlinewidth{0.615091pt}%
\definecolor{currentstroke}{rgb}{0.263663,0.237631,0.518762}%
\pgfsetstrokecolor{currentstroke}%
\pgfsetdash{}{0pt}%
\pgfpathmoveto{\pgfqpoint{7.693908in}{2.876439in}}%
\pgfpathlineto{\pgfqpoint{7.646835in}{2.861406in}}%
\pgfusepath{stroke}%
\end{pgfscope}%
\begin{pgfscope}%
\pgfpathrectangle{\pgfqpoint{6.720588in}{1.750000in}}{\pgfqpoint{2.279412in}{2.004545in}}%
\pgfusepath{clip}%
\pgfsetbuttcap%
\pgfsetroundjoin%
\pgfsetlinewidth{0.589973pt}%
\definecolor{currentstroke}{rgb}{0.267968,0.223549,0.512008}%
\pgfsetstrokecolor{currentstroke}%
\pgfsetdash{}{0pt}%
\pgfpathmoveto{\pgfqpoint{7.646835in}{2.861406in}}%
\pgfpathlineto{\pgfqpoint{7.600691in}{2.844281in}}%
\pgfusepath{stroke}%
\end{pgfscope}%
\begin{pgfscope}%
\pgfpathrectangle{\pgfqpoint{6.720588in}{1.750000in}}{\pgfqpoint{2.279412in}{2.004545in}}%
\pgfusepath{clip}%
\pgfsetbuttcap%
\pgfsetroundjoin%
\pgfsetlinewidth{0.576546pt}%
\definecolor{currentstroke}{rgb}{0.270595,0.214069,0.507052}%
\pgfsetstrokecolor{currentstroke}%
\pgfsetdash{}{0pt}%
\pgfpathmoveto{\pgfqpoint{7.600691in}{2.844281in}}%
\pgfpathlineto{\pgfqpoint{7.554936in}{2.826423in}}%
\pgfusepath{stroke}%
\end{pgfscope}%
\begin{pgfscope}%
\pgfpathrectangle{\pgfqpoint{6.720588in}{1.750000in}}{\pgfqpoint{2.279412in}{2.004545in}}%
\pgfusepath{clip}%
\pgfsetbuttcap%
\pgfsetroundjoin%
\pgfsetlinewidth{0.568777pt}%
\definecolor{currentstroke}{rgb}{0.271828,0.209303,0.504434}%
\pgfsetstrokecolor{currentstroke}%
\pgfsetdash{}{0pt}%
\pgfpathmoveto{\pgfqpoint{7.554936in}{2.826423in}}%
\pgfpathlineto{\pgfqpoint{7.511409in}{2.805201in}}%
\pgfusepath{stroke}%
\end{pgfscope}%
\begin{pgfscope}%
\pgfpathrectangle{\pgfqpoint{6.720588in}{1.750000in}}{\pgfqpoint{2.279412in}{2.004545in}}%
\pgfusepath{clip}%
\pgfsetbuttcap%
\pgfsetroundjoin%
\pgfsetlinewidth{0.518621pt}%
\definecolor{currentstroke}{rgb}{0.279574,0.170599,0.479997}%
\pgfsetstrokecolor{currentstroke}%
\pgfsetdash{}{0pt}%
\pgfpathmoveto{\pgfqpoint{7.511409in}{2.805201in}}%
\pgfpathlineto{\pgfqpoint{7.511409in}{2.805201in}}%
\pgfusepath{stroke}%
\end{pgfscope}%
\begin{pgfscope}%
\pgfpathrectangle{\pgfqpoint{6.720588in}{1.750000in}}{\pgfqpoint{2.279412in}{2.004545in}}%
\pgfusepath{clip}%
\pgfsetbuttcap%
\pgfsetroundjoin%
\pgfsetlinewidth{0.518621pt}%
\definecolor{currentstroke}{rgb}{0.279574,0.170599,0.479997}%
\pgfsetstrokecolor{currentstroke}%
\pgfsetdash{}{0pt}%
\pgfpathmoveto{\pgfqpoint{7.511409in}{2.805201in}}%
\pgfpathlineto{\pgfqpoint{7.487222in}{2.787890in}}%
\pgfusepath{stroke}%
\end{pgfscope}%
\begin{pgfscope}%
\pgfpathrectangle{\pgfqpoint{6.720588in}{1.750000in}}{\pgfqpoint{2.279412in}{2.004545in}}%
\pgfusepath{clip}%
\pgfsetbuttcap%
\pgfsetroundjoin%
\pgfsetlinewidth{0.489911pt}%
\definecolor{currentstroke}{rgb}{0.281887,0.150881,0.465405}%
\pgfsetstrokecolor{currentstroke}%
\pgfsetdash{}{0pt}%
\pgfpathmoveto{\pgfqpoint{7.487222in}{2.787890in}}%
\pgfpathlineto{\pgfqpoint{7.487222in}{2.787890in}}%
\pgfusepath{stroke}%
\end{pgfscope}%
\begin{pgfscope}%
\pgfpathrectangle{\pgfqpoint{6.720588in}{1.750000in}}{\pgfqpoint{2.279412in}{2.004545in}}%
\pgfusepath{clip}%
\pgfsetbuttcap%
\pgfsetroundjoin%
\pgfsetlinewidth{0.321701pt}%
\definecolor{currentstroke}{rgb}{0.269944,0.014625,0.341379}%
\pgfsetstrokecolor{currentstroke}%
\pgfsetdash{}{0pt}%
\pgfpathmoveto{\pgfqpoint{7.706418in}{2.120778in}}%
\pgfpathlineto{\pgfqpoint{7.732022in}{2.154751in}}%
\pgfusepath{stroke}%
\end{pgfscope}%
\begin{pgfscope}%
\pgfpathrectangle{\pgfqpoint{6.720588in}{1.750000in}}{\pgfqpoint{2.279412in}{2.004545in}}%
\pgfusepath{clip}%
\pgfsetbuttcap%
\pgfsetroundjoin%
\pgfsetlinewidth{0.349871pt}%
\definecolor{currentstroke}{rgb}{0.276022,0.044167,0.370164}%
\pgfsetstrokecolor{currentstroke}%
\pgfsetdash{}{0pt}%
\pgfpathmoveto{\pgfqpoint{7.732022in}{2.154751in}}%
\pgfpathlineto{\pgfqpoint{7.732022in}{2.154751in}}%
\pgfusepath{stroke}%
\end{pgfscope}%
\begin{pgfscope}%
\pgfpathrectangle{\pgfqpoint{6.720588in}{1.750000in}}{\pgfqpoint{2.279412in}{2.004545in}}%
\pgfusepath{clip}%
\pgfsetbuttcap%
\pgfsetroundjoin%
\pgfsetlinewidth{0.349871pt}%
\definecolor{currentstroke}{rgb}{0.276022,0.044167,0.370164}%
\pgfsetstrokecolor{currentstroke}%
\pgfsetdash{}{0pt}%
\pgfpathmoveto{\pgfqpoint{7.732022in}{2.154751in}}%
\pgfpathlineto{\pgfqpoint{7.732022in}{2.154751in}}%
\pgfusepath{stroke}%
\end{pgfscope}%
\begin{pgfscope}%
\pgfpathrectangle{\pgfqpoint{6.720588in}{1.750000in}}{\pgfqpoint{2.279412in}{2.004545in}}%
\pgfusepath{clip}%
\pgfsetbuttcap%
\pgfsetroundjoin%
\pgfsetlinewidth{0.349871pt}%
\definecolor{currentstroke}{rgb}{0.276022,0.044167,0.370164}%
\pgfsetstrokecolor{currentstroke}%
\pgfsetdash{}{0pt}%
\pgfpathmoveto{\pgfqpoint{7.732022in}{2.154751in}}%
\pgfpathlineto{\pgfqpoint{7.736160in}{2.166925in}}%
\pgfusepath{stroke}%
\end{pgfscope}%
\begin{pgfscope}%
\pgfpathrectangle{\pgfqpoint{6.720588in}{1.750000in}}{\pgfqpoint{2.279412in}{2.004545in}}%
\pgfusepath{clip}%
\pgfsetbuttcap%
\pgfsetroundjoin%
\pgfsetlinewidth{0.353565pt}%
\definecolor{currentstroke}{rgb}{0.276022,0.044167,0.370164}%
\pgfsetstrokecolor{currentstroke}%
\pgfsetdash{}{0pt}%
\pgfpathmoveto{\pgfqpoint{7.736160in}{2.166925in}}%
\pgfpathlineto{\pgfqpoint{7.734116in}{2.179903in}}%
\pgfusepath{stroke}%
\end{pgfscope}%
\begin{pgfscope}%
\pgfpathrectangle{\pgfqpoint{6.720588in}{1.750000in}}{\pgfqpoint{2.279412in}{2.004545in}}%
\pgfusepath{clip}%
\pgfsetbuttcap%
\pgfsetroundjoin%
\pgfsetlinewidth{0.349238pt}%
\definecolor{currentstroke}{rgb}{0.276022,0.044167,0.370164}%
\pgfsetstrokecolor{currentstroke}%
\pgfsetdash{}{0pt}%
\pgfpathmoveto{\pgfqpoint{7.734116in}{2.179903in}}%
\pgfpathlineto{\pgfqpoint{7.734666in}{2.193318in}}%
\pgfusepath{stroke}%
\end{pgfscope}%
\begin{pgfscope}%
\pgfpathrectangle{\pgfqpoint{6.720588in}{1.750000in}}{\pgfqpoint{2.279412in}{2.004545in}}%
\pgfusepath{clip}%
\pgfsetbuttcap%
\pgfsetroundjoin%
\pgfsetlinewidth{0.334176pt}%
\definecolor{currentstroke}{rgb}{0.272594,0.025563,0.353093}%
\pgfsetstrokecolor{currentstroke}%
\pgfsetdash{}{0pt}%
\pgfpathmoveto{\pgfqpoint{7.734666in}{2.193318in}}%
\pgfpathlineto{\pgfqpoint{7.741644in}{2.217855in}}%
\pgfusepath{stroke}%
\end{pgfscope}%
\begin{pgfscope}%
\pgfpathrectangle{\pgfqpoint{6.720588in}{1.750000in}}{\pgfqpoint{2.279412in}{2.004545in}}%
\pgfusepath{clip}%
\pgfsetbuttcap%
\pgfsetroundjoin%
\pgfsetlinewidth{0.327938pt}%
\definecolor{currentstroke}{rgb}{0.271305,0.019942,0.347269}%
\pgfsetstrokecolor{currentstroke}%
\pgfsetdash{}{0pt}%
\pgfpathmoveto{\pgfqpoint{7.741644in}{2.217855in}}%
\pgfpathlineto{\pgfqpoint{7.741644in}{2.217855in}}%
\pgfusepath{stroke}%
\end{pgfscope}%
\begin{pgfscope}%
\pgfpathrectangle{\pgfqpoint{6.720588in}{1.750000in}}{\pgfqpoint{2.279412in}{2.004545in}}%
\pgfusepath{clip}%
\pgfsetbuttcap%
\pgfsetroundjoin%
\pgfsetlinewidth{0.327938pt}%
\definecolor{currentstroke}{rgb}{0.271305,0.019942,0.347269}%
\pgfsetstrokecolor{currentstroke}%
\pgfsetdash{}{0pt}%
\pgfpathmoveto{\pgfqpoint{7.741644in}{2.217855in}}%
\pgfpathlineto{\pgfqpoint{7.741644in}{2.217855in}}%
\pgfusepath{stroke}%
\end{pgfscope}%
\begin{pgfscope}%
\pgfpathrectangle{\pgfqpoint{6.720588in}{1.750000in}}{\pgfqpoint{2.279412in}{2.004545in}}%
\pgfusepath{clip}%
\pgfsetbuttcap%
\pgfsetroundjoin%
\pgfsetlinewidth{0.327938pt}%
\definecolor{currentstroke}{rgb}{0.271305,0.019942,0.347269}%
\pgfsetstrokecolor{currentstroke}%
\pgfsetdash{}{0pt}%
\pgfpathmoveto{\pgfqpoint{7.741644in}{2.217855in}}%
\pgfpathlineto{\pgfqpoint{7.741498in}{2.231579in}}%
\pgfusepath{stroke}%
\end{pgfscope}%
\begin{pgfscope}%
\pgfpathrectangle{\pgfqpoint{6.720588in}{1.750000in}}{\pgfqpoint{2.279412in}{2.004545in}}%
\pgfusepath{clip}%
\pgfsetbuttcap%
\pgfsetroundjoin%
\pgfsetlinewidth{0.347164pt}%
\definecolor{currentstroke}{rgb}{0.274952,0.037752,0.364543}%
\pgfsetstrokecolor{currentstroke}%
\pgfsetdash{}{0pt}%
\pgfpathmoveto{\pgfqpoint{7.741498in}{2.231579in}}%
\pgfpathlineto{\pgfqpoint{7.737884in}{2.244851in}}%
\pgfusepath{stroke}%
\end{pgfscope}%
\begin{pgfscope}%
\pgfpathrectangle{\pgfqpoint{6.720588in}{1.750000in}}{\pgfqpoint{2.279412in}{2.004545in}}%
\pgfusepath{clip}%
\pgfsetbuttcap%
\pgfsetroundjoin%
\pgfsetlinewidth{0.357764pt}%
\definecolor{currentstroke}{rgb}{0.277018,0.050344,0.375715}%
\pgfsetstrokecolor{currentstroke}%
\pgfsetdash{}{0pt}%
\pgfpathmoveto{\pgfqpoint{7.737884in}{2.244851in}}%
\pgfpathlineto{\pgfqpoint{7.731615in}{2.286213in}}%
\pgfusepath{stroke}%
\end{pgfscope}%
\begin{pgfscope}%
\pgfpathrectangle{\pgfqpoint{6.720588in}{1.750000in}}{\pgfqpoint{2.279412in}{2.004545in}}%
\pgfusepath{clip}%
\pgfsetbuttcap%
\pgfsetroundjoin%
\pgfsetlinewidth{0.357719pt}%
\definecolor{currentstroke}{rgb}{0.277018,0.050344,0.375715}%
\pgfsetstrokecolor{currentstroke}%
\pgfsetdash{}{0pt}%
\pgfpathmoveto{\pgfqpoint{7.731615in}{2.286213in}}%
\pgfpathlineto{\pgfqpoint{7.731615in}{2.286213in}}%
\pgfusepath{stroke}%
\end{pgfscope}%
\begin{pgfscope}%
\pgfpathrectangle{\pgfqpoint{6.720588in}{1.750000in}}{\pgfqpoint{2.279412in}{2.004545in}}%
\pgfusepath{clip}%
\pgfsetbuttcap%
\pgfsetroundjoin%
\pgfsetlinewidth{0.357719pt}%
\definecolor{currentstroke}{rgb}{0.277018,0.050344,0.375715}%
\pgfsetstrokecolor{currentstroke}%
\pgfsetdash{}{0pt}%
\pgfpathmoveto{\pgfqpoint{7.731615in}{2.286213in}}%
\pgfpathlineto{\pgfqpoint{7.720347in}{2.319284in}}%
\pgfusepath{stroke}%
\end{pgfscope}%
\begin{pgfscope}%
\pgfpathrectangle{\pgfqpoint{6.720588in}{1.750000in}}{\pgfqpoint{2.279412in}{2.004545in}}%
\pgfusepath{clip}%
\pgfsetbuttcap%
\pgfsetroundjoin%
\pgfsetlinewidth{0.433096pt}%
\definecolor{currentstroke}{rgb}{0.283091,0.110553,0.431554}%
\pgfsetstrokecolor{currentstroke}%
\pgfsetdash{}{0pt}%
\pgfpathmoveto{\pgfqpoint{7.720347in}{2.319284in}}%
\pgfpathlineto{\pgfqpoint{7.707634in}{2.348865in}}%
\pgfusepath{stroke}%
\end{pgfscope}%
\begin{pgfscope}%
\pgfpathrectangle{\pgfqpoint{6.720588in}{1.750000in}}{\pgfqpoint{2.279412in}{2.004545in}}%
\pgfusepath{clip}%
\pgfsetbuttcap%
\pgfsetroundjoin%
\pgfsetlinewidth{0.429444pt}%
\definecolor{currentstroke}{rgb}{0.282910,0.105393,0.426902}%
\pgfsetstrokecolor{currentstroke}%
\pgfsetdash{}{0pt}%
\pgfpathmoveto{\pgfqpoint{7.707634in}{2.348865in}}%
\pgfpathlineto{\pgfqpoint{7.688073in}{2.388487in}}%
\pgfusepath{stroke}%
\end{pgfscope}%
\begin{pgfscope}%
\pgfpathrectangle{\pgfqpoint{6.720588in}{1.750000in}}{\pgfqpoint{2.279412in}{2.004545in}}%
\pgfusepath{clip}%
\pgfsetbuttcap%
\pgfsetroundjoin%
\pgfsetlinewidth{0.481633pt}%
\definecolor{currentstroke}{rgb}{0.282290,0.145912,0.461510}%
\pgfsetstrokecolor{currentstroke}%
\pgfsetdash{}{0pt}%
\pgfpathmoveto{\pgfqpoint{7.688073in}{2.388487in}}%
\pgfpathlineto{\pgfqpoint{7.667373in}{2.427962in}}%
\pgfusepath{stroke}%
\end{pgfscope}%
\begin{pgfscope}%
\pgfpathrectangle{\pgfqpoint{6.720588in}{1.750000in}}{\pgfqpoint{2.279412in}{2.004545in}}%
\pgfusepath{clip}%
\pgfsetbuttcap%
\pgfsetroundjoin%
\pgfsetlinewidth{0.511164pt}%
\definecolor{currentstroke}{rgb}{0.280255,0.165693,0.476498}%
\pgfsetstrokecolor{currentstroke}%
\pgfsetdash{}{0pt}%
\pgfpathmoveto{\pgfqpoint{7.667373in}{2.427962in}}%
\pgfpathlineto{\pgfqpoint{7.642207in}{2.465707in}}%
\pgfusepath{stroke}%
\end{pgfscope}%
\begin{pgfscope}%
\pgfpathrectangle{\pgfqpoint{6.720588in}{1.750000in}}{\pgfqpoint{2.279412in}{2.004545in}}%
\pgfusepath{clip}%
\pgfsetbuttcap%
\pgfsetroundjoin%
\pgfsetlinewidth{0.616603pt}%
\definecolor{currentstroke}{rgb}{0.263663,0.237631,0.518762}%
\pgfsetstrokecolor{currentstroke}%
\pgfsetdash{}{0pt}%
\pgfpathmoveto{\pgfqpoint{7.642207in}{2.465707in}}%
\pgfpathlineto{\pgfqpoint{7.614331in}{2.502208in}}%
\pgfusepath{stroke}%
\end{pgfscope}%
\begin{pgfscope}%
\pgfpathrectangle{\pgfqpoint{6.720588in}{1.750000in}}{\pgfqpoint{2.279412in}{2.004545in}}%
\pgfusepath{clip}%
\pgfsetbuttcap%
\pgfsetroundjoin%
\pgfsetlinewidth{0.632848pt}%
\definecolor{currentstroke}{rgb}{0.258965,0.251537,0.524736}%
\pgfsetstrokecolor{currentstroke}%
\pgfsetdash{}{0pt}%
\pgfpathmoveto{\pgfqpoint{7.614331in}{2.502208in}}%
\pgfpathlineto{\pgfqpoint{7.585263in}{2.538042in}}%
\pgfusepath{stroke}%
\end{pgfscope}%
\begin{pgfscope}%
\pgfpathrectangle{\pgfqpoint{6.720588in}{1.750000in}}{\pgfqpoint{2.279412in}{2.004545in}}%
\pgfusepath{clip}%
\pgfsetbuttcap%
\pgfsetroundjoin%
\pgfsetlinewidth{0.637716pt}%
\definecolor{currentstroke}{rgb}{0.257322,0.256130,0.526563}%
\pgfsetstrokecolor{currentstroke}%
\pgfsetdash{}{0pt}%
\pgfpathmoveto{\pgfqpoint{7.585263in}{2.538042in}}%
\pgfpathlineto{\pgfqpoint{7.555448in}{2.573221in}}%
\pgfusepath{stroke}%
\end{pgfscope}%
\begin{pgfscope}%
\pgfpathrectangle{\pgfqpoint{6.720588in}{1.750000in}}{\pgfqpoint{2.279412in}{2.004545in}}%
\pgfusepath{clip}%
\pgfsetbuttcap%
\pgfsetroundjoin%
\pgfsetlinewidth{0.624247pt}%
\definecolor{currentstroke}{rgb}{0.260571,0.246922,0.522828}%
\pgfsetstrokecolor{currentstroke}%
\pgfsetdash{}{0pt}%
\pgfpathmoveto{\pgfqpoint{7.555448in}{2.573221in}}%
\pgfpathlineto{\pgfqpoint{7.524440in}{2.607475in}}%
\pgfusepath{stroke}%
\end{pgfscope}%
\begin{pgfscope}%
\pgfpathrectangle{\pgfqpoint{6.720588in}{1.750000in}}{\pgfqpoint{2.279412in}{2.004545in}}%
\pgfusepath{clip}%
\pgfsetbuttcap%
\pgfsetroundjoin%
\pgfsetlinewidth{0.315579pt}%
\definecolor{currentstroke}{rgb}{0.269944,0.014625,0.341379}%
\pgfsetstrokecolor{currentstroke}%
\pgfsetdash{}{0pt}%
\pgfpathmoveto{\pgfqpoint{8.416169in}{2.099273in}}%
\pgfpathlineto{\pgfqpoint{8.366169in}{2.101967in}}%
\pgfusepath{stroke}%
\end{pgfscope}%
\begin{pgfscope}%
\pgfpathrectangle{\pgfqpoint{6.720588in}{1.750000in}}{\pgfqpoint{2.279412in}{2.004545in}}%
\pgfusepath{clip}%
\pgfsetbuttcap%
\pgfsetroundjoin%
\pgfsetlinewidth{0.321765pt}%
\definecolor{currentstroke}{rgb}{0.271305,0.019942,0.347269}%
\pgfsetstrokecolor{currentstroke}%
\pgfsetdash{}{0pt}%
\pgfpathmoveto{\pgfqpoint{8.366169in}{2.101967in}}%
\pgfpathlineto{\pgfqpoint{8.316094in}{2.103828in}}%
\pgfusepath{stroke}%
\end{pgfscope}%
\begin{pgfscope}%
\pgfpathrectangle{\pgfqpoint{6.720588in}{1.750000in}}{\pgfqpoint{2.279412in}{2.004545in}}%
\pgfusepath{clip}%
\pgfsetbuttcap%
\pgfsetroundjoin%
\pgfsetlinewidth{0.327453pt}%
\definecolor{currentstroke}{rgb}{0.271305,0.019942,0.347269}%
\pgfsetstrokecolor{currentstroke}%
\pgfsetdash{}{0pt}%
\pgfpathmoveto{\pgfqpoint{8.316094in}{2.103828in}}%
\pgfpathlineto{\pgfqpoint{8.266084in}{2.105953in}}%
\pgfusepath{stroke}%
\end{pgfscope}%
\begin{pgfscope}%
\pgfpathrectangle{\pgfqpoint{6.720588in}{1.750000in}}{\pgfqpoint{2.279412in}{2.004545in}}%
\pgfusepath{clip}%
\pgfsetbuttcap%
\pgfsetroundjoin%
\pgfsetlinewidth{0.328680pt}%
\definecolor{currentstroke}{rgb}{0.272594,0.025563,0.353093}%
\pgfsetstrokecolor{currentstroke}%
\pgfsetdash{}{0pt}%
\pgfpathmoveto{\pgfqpoint{8.266084in}{2.105953in}}%
\pgfpathlineto{\pgfqpoint{8.216296in}{2.110661in}}%
\pgfusepath{stroke}%
\end{pgfscope}%
\begin{pgfscope}%
\pgfpathrectangle{\pgfqpoint{6.720588in}{1.750000in}}{\pgfqpoint{2.279412in}{2.004545in}}%
\pgfusepath{clip}%
\pgfsetbuttcap%
\pgfsetroundjoin%
\pgfsetlinewidth{0.342919pt}%
\definecolor{currentstroke}{rgb}{0.274952,0.037752,0.364543}%
\pgfsetstrokecolor{currentstroke}%
\pgfsetdash{}{0pt}%
\pgfpathmoveto{\pgfqpoint{8.216296in}{2.110661in}}%
\pgfpathlineto{\pgfqpoint{8.166482in}{2.115449in}}%
\pgfusepath{stroke}%
\end{pgfscope}%
\begin{pgfscope}%
\pgfpathrectangle{\pgfqpoint{6.720588in}{1.750000in}}{\pgfqpoint{2.279412in}{2.004545in}}%
\pgfusepath{clip}%
\pgfsetbuttcap%
\pgfsetroundjoin%
\pgfsetlinewidth{0.352025pt}%
\definecolor{currentstroke}{rgb}{0.276022,0.044167,0.370164}%
\pgfsetstrokecolor{currentstroke}%
\pgfsetdash{}{0pt}%
\pgfpathmoveto{\pgfqpoint{8.166482in}{2.115449in}}%
\pgfpathlineto{\pgfqpoint{8.116754in}{2.120778in}}%
\pgfusepath{stroke}%
\end{pgfscope}%
\begin{pgfscope}%
\pgfpathrectangle{\pgfqpoint{6.720588in}{1.750000in}}{\pgfqpoint{2.279412in}{2.004545in}}%
\pgfusepath{clip}%
\pgfsetbuttcap%
\pgfsetroundjoin%
\pgfsetlinewidth{0.355761pt}%
\definecolor{currentstroke}{rgb}{0.276022,0.044167,0.370164}%
\pgfsetstrokecolor{currentstroke}%
\pgfsetdash{}{0pt}%
\pgfpathmoveto{\pgfqpoint{8.116754in}{2.120778in}}%
\pgfpathlineto{\pgfqpoint{8.067271in}{2.127716in}}%
\pgfusepath{stroke}%
\end{pgfscope}%
\begin{pgfscope}%
\pgfpathrectangle{\pgfqpoint{6.720588in}{1.750000in}}{\pgfqpoint{2.279412in}{2.004545in}}%
\pgfusepath{clip}%
\pgfsetbuttcap%
\pgfsetroundjoin%
\pgfsetlinewidth{0.341344pt}%
\definecolor{currentstroke}{rgb}{0.273809,0.031497,0.358853}%
\pgfsetstrokecolor{currentstroke}%
\pgfsetdash{}{0pt}%
\pgfpathmoveto{\pgfqpoint{8.067271in}{2.127716in}}%
\pgfpathlineto{\pgfqpoint{8.017882in}{2.135171in}}%
\pgfusepath{stroke}%
\end{pgfscope}%
\begin{pgfscope}%
\pgfpathrectangle{\pgfqpoint{6.720588in}{1.750000in}}{\pgfqpoint{2.279412in}{2.004545in}}%
\pgfusepath{clip}%
\pgfsetbuttcap%
\pgfsetroundjoin%
\pgfsetlinewidth{0.347336pt}%
\definecolor{currentstroke}{rgb}{0.274952,0.037752,0.364543}%
\pgfsetstrokecolor{currentstroke}%
\pgfsetdash{}{0pt}%
\pgfpathmoveto{\pgfqpoint{8.017882in}{2.135171in}}%
\pgfpathlineto{\pgfqpoint{7.968588in}{2.142931in}}%
\pgfusepath{stroke}%
\end{pgfscope}%
\begin{pgfscope}%
\pgfpathrectangle{\pgfqpoint{6.720588in}{1.750000in}}{\pgfqpoint{2.279412in}{2.004545in}}%
\pgfusepath{clip}%
\pgfsetbuttcap%
\pgfsetroundjoin%
\pgfsetlinewidth{0.375670pt}%
\definecolor{currentstroke}{rgb}{0.278791,0.062145,0.386592}%
\pgfsetstrokecolor{currentstroke}%
\pgfsetdash{}{0pt}%
\pgfpathmoveto{\pgfqpoint{7.968588in}{2.142931in}}%
\pgfpathlineto{\pgfqpoint{7.920492in}{2.154223in}}%
\pgfusepath{stroke}%
\end{pgfscope}%
\begin{pgfscope}%
\pgfpathrectangle{\pgfqpoint{6.720588in}{1.750000in}}{\pgfqpoint{2.279412in}{2.004545in}}%
\pgfusepath{clip}%
\pgfsetbuttcap%
\pgfsetroundjoin%
\pgfsetlinewidth{0.367823pt}%
\definecolor{currentstroke}{rgb}{0.277941,0.056324,0.381191}%
\pgfsetstrokecolor{currentstroke}%
\pgfsetdash{}{0pt}%
\pgfpathmoveto{\pgfqpoint{7.920492in}{2.154223in}}%
\pgfpathlineto{\pgfqpoint{7.874713in}{2.171355in}}%
\pgfusepath{stroke}%
\end{pgfscope}%
\begin{pgfscope}%
\pgfpathrectangle{\pgfqpoint{6.720588in}{1.750000in}}{\pgfqpoint{2.279412in}{2.004545in}}%
\pgfusepath{clip}%
\pgfsetbuttcap%
\pgfsetroundjoin%
\pgfsetlinewidth{0.343879pt}%
\definecolor{currentstroke}{rgb}{0.274952,0.037752,0.364543}%
\pgfsetstrokecolor{currentstroke}%
\pgfsetdash{}{0pt}%
\pgfpathmoveto{\pgfqpoint{7.874713in}{2.171355in}}%
\pgfpathlineto{\pgfqpoint{7.830509in}{2.190858in}}%
\pgfusepath{stroke}%
\end{pgfscope}%
\begin{pgfscope}%
\pgfpathrectangle{\pgfqpoint{6.720588in}{1.750000in}}{\pgfqpoint{2.279412in}{2.004545in}}%
\pgfusepath{clip}%
\pgfsetbuttcap%
\pgfsetroundjoin%
\pgfsetlinewidth{0.341083pt}%
\definecolor{currentstroke}{rgb}{0.273809,0.031497,0.358853}%
\pgfsetstrokecolor{currentstroke}%
\pgfsetdash{}{0pt}%
\pgfpathmoveto{\pgfqpoint{7.830509in}{2.190858in}}%
\pgfpathlineto{\pgfqpoint{7.789675in}{2.214232in}}%
\pgfusepath{stroke}%
\end{pgfscope}%
\begin{pgfscope}%
\pgfpathrectangle{\pgfqpoint{6.720588in}{1.750000in}}{\pgfqpoint{2.279412in}{2.004545in}}%
\pgfusepath{clip}%
\pgfsetbuttcap%
\pgfsetroundjoin%
\pgfsetlinewidth{0.318258pt}%
\definecolor{currentstroke}{rgb}{0.269944,0.014625,0.341379}%
\pgfsetstrokecolor{currentstroke}%
\pgfsetdash{}{0pt}%
\pgfpathmoveto{\pgfqpoint{8.578381in}{2.165885in}}%
\pgfpathlineto{\pgfqpoint{8.528597in}{2.166381in}}%
\pgfusepath{stroke}%
\end{pgfscope}%
\begin{pgfscope}%
\pgfpathrectangle{\pgfqpoint{6.720588in}{1.750000in}}{\pgfqpoint{2.279412in}{2.004545in}}%
\pgfusepath{clip}%
\pgfsetbuttcap%
\pgfsetroundjoin%
\pgfsetlinewidth{0.318938pt}%
\definecolor{currentstroke}{rgb}{0.269944,0.014625,0.341379}%
\pgfsetstrokecolor{currentstroke}%
\pgfsetdash{}{0pt}%
\pgfpathmoveto{\pgfqpoint{8.528597in}{2.166381in}}%
\pgfpathlineto{\pgfqpoint{8.478866in}{2.165984in}}%
\pgfusepath{stroke}%
\end{pgfscope}%
\begin{pgfscope}%
\pgfpathrectangle{\pgfqpoint{6.720588in}{1.750000in}}{\pgfqpoint{2.279412in}{2.004545in}}%
\pgfusepath{clip}%
\pgfsetbuttcap%
\pgfsetroundjoin%
\pgfsetlinewidth{0.322322pt}%
\definecolor{currentstroke}{rgb}{0.271305,0.019942,0.347269}%
\pgfsetstrokecolor{currentstroke}%
\pgfsetdash{}{0pt}%
\pgfpathmoveto{\pgfqpoint{8.478866in}{2.165984in}}%
\pgfpathlineto{\pgfqpoint{8.428739in}{2.166391in}}%
\pgfusepath{stroke}%
\end{pgfscope}%
\begin{pgfscope}%
\pgfpathrectangle{\pgfqpoint{6.720588in}{1.750000in}}{\pgfqpoint{2.279412in}{2.004545in}}%
\pgfusepath{clip}%
\pgfsetbuttcap%
\pgfsetroundjoin%
\pgfsetlinewidth{0.318374pt}%
\definecolor{currentstroke}{rgb}{0.269944,0.014625,0.341379}%
\pgfsetstrokecolor{currentstroke}%
\pgfsetdash{}{0pt}%
\pgfpathmoveto{\pgfqpoint{8.428739in}{2.166391in}}%
\pgfpathlineto{\pgfqpoint{8.378612in}{2.166605in}}%
\pgfusepath{stroke}%
\end{pgfscope}%
\begin{pgfscope}%
\pgfpathrectangle{\pgfqpoint{6.720588in}{1.750000in}}{\pgfqpoint{2.279412in}{2.004545in}}%
\pgfusepath{clip}%
\pgfsetbuttcap%
\pgfsetroundjoin%
\pgfsetlinewidth{0.334097pt}%
\definecolor{currentstroke}{rgb}{0.272594,0.025563,0.353093}%
\pgfsetstrokecolor{currentstroke}%
\pgfsetdash{}{0pt}%
\pgfpathmoveto{\pgfqpoint{8.378612in}{2.166605in}}%
\pgfpathlineto{\pgfqpoint{8.328530in}{2.168451in}}%
\pgfusepath{stroke}%
\end{pgfscope}%
\begin{pgfscope}%
\pgfpathrectangle{\pgfqpoint{6.720588in}{1.750000in}}{\pgfqpoint{2.279412in}{2.004545in}}%
\pgfusepath{clip}%
\pgfsetbuttcap%
\pgfsetroundjoin%
\pgfsetlinewidth{0.339042pt}%
\definecolor{currentstroke}{rgb}{0.273809,0.031497,0.358853}%
\pgfsetstrokecolor{currentstroke}%
\pgfsetdash{}{0pt}%
\pgfpathmoveto{\pgfqpoint{8.328530in}{2.168451in}}%
\pgfpathlineto{\pgfqpoint{8.278494in}{2.171041in}}%
\pgfusepath{stroke}%
\end{pgfscope}%
\begin{pgfscope}%
\pgfpathrectangle{\pgfqpoint{6.720588in}{1.750000in}}{\pgfqpoint{2.279412in}{2.004545in}}%
\pgfusepath{clip}%
\pgfsetbuttcap%
\pgfsetroundjoin%
\pgfsetlinewidth{0.333123pt}%
\definecolor{currentstroke}{rgb}{0.272594,0.025563,0.353093}%
\pgfsetstrokecolor{currentstroke}%
\pgfsetdash{}{0pt}%
\pgfpathmoveto{\pgfqpoint{8.278494in}{2.171041in}}%
\pgfpathlineto{\pgfqpoint{8.228555in}{2.174441in}}%
\pgfusepath{stroke}%
\end{pgfscope}%
\begin{pgfscope}%
\pgfpathrectangle{\pgfqpoint{6.720588in}{1.750000in}}{\pgfqpoint{2.279412in}{2.004545in}}%
\pgfusepath{clip}%
\pgfsetbuttcap%
\pgfsetroundjoin%
\pgfsetlinewidth{0.336550pt}%
\definecolor{currentstroke}{rgb}{0.273809,0.031497,0.358853}%
\pgfsetstrokecolor{currentstroke}%
\pgfsetdash{}{0pt}%
\pgfpathmoveto{\pgfqpoint{8.228555in}{2.174441in}}%
\pgfpathlineto{\pgfqpoint{8.178698in}{2.178853in}}%
\pgfusepath{stroke}%
\end{pgfscope}%
\begin{pgfscope}%
\pgfpathrectangle{\pgfqpoint{6.720588in}{1.750000in}}{\pgfqpoint{2.279412in}{2.004545in}}%
\pgfusepath{clip}%
\pgfsetbuttcap%
\pgfsetroundjoin%
\pgfsetlinewidth{0.381913pt}%
\definecolor{currentstroke}{rgb}{0.279566,0.067836,0.391917}%
\pgfsetstrokecolor{currentstroke}%
\pgfsetdash{}{0pt}%
\pgfpathmoveto{\pgfqpoint{8.178698in}{2.178853in}}%
\pgfpathlineto{\pgfqpoint{8.128878in}{2.183698in}}%
\pgfusepath{stroke}%
\end{pgfscope}%
\begin{pgfscope}%
\pgfpathrectangle{\pgfqpoint{6.720588in}{1.750000in}}{\pgfqpoint{2.279412in}{2.004545in}}%
\pgfusepath{clip}%
\pgfsetbuttcap%
\pgfsetroundjoin%
\pgfsetlinewidth{0.360529pt}%
\definecolor{currentstroke}{rgb}{0.277018,0.050344,0.375715}%
\pgfsetstrokecolor{currentstroke}%
\pgfsetdash{}{0pt}%
\pgfpathmoveto{\pgfqpoint{8.128878in}{2.183698in}}%
\pgfpathlineto{\pgfqpoint{8.079210in}{2.189552in}}%
\pgfusepath{stroke}%
\end{pgfscope}%
\begin{pgfscope}%
\pgfpathrectangle{\pgfqpoint{6.720588in}{1.750000in}}{\pgfqpoint{2.279412in}{2.004545in}}%
\pgfusepath{clip}%
\pgfsetbuttcap%
\pgfsetroundjoin%
\pgfsetlinewidth{0.358792pt}%
\definecolor{currentstroke}{rgb}{0.277018,0.050344,0.375715}%
\pgfsetstrokecolor{currentstroke}%
\pgfsetdash{}{0pt}%
\pgfpathmoveto{\pgfqpoint{8.079210in}{2.189552in}}%
\pgfpathlineto{\pgfqpoint{8.029614in}{2.195973in}}%
\pgfusepath{stroke}%
\end{pgfscope}%
\begin{pgfscope}%
\pgfpathrectangle{\pgfqpoint{6.720588in}{1.750000in}}{\pgfqpoint{2.279412in}{2.004545in}}%
\pgfusepath{clip}%
\pgfsetbuttcap%
\pgfsetroundjoin%
\pgfsetlinewidth{0.333610pt}%
\definecolor{currentstroke}{rgb}{0.272594,0.025563,0.353093}%
\pgfsetstrokecolor{currentstroke}%
\pgfsetdash{}{0pt}%
\pgfpathmoveto{\pgfqpoint{8.029614in}{2.195973in}}%
\pgfpathlineto{\pgfqpoint{7.980676in}{2.204999in}}%
\pgfusepath{stroke}%
\end{pgfscope}%
\begin{pgfscope}%
\pgfpathrectangle{\pgfqpoint{6.720588in}{1.750000in}}{\pgfqpoint{2.279412in}{2.004545in}}%
\pgfusepath{clip}%
\pgfsetbuttcap%
\pgfsetroundjoin%
\pgfsetlinewidth{0.370199pt}%
\definecolor{currentstroke}{rgb}{0.278791,0.062145,0.386592}%
\pgfsetstrokecolor{currentstroke}%
\pgfsetdash{}{0pt}%
\pgfpathmoveto{\pgfqpoint{7.980676in}{2.204999in}}%
\pgfpathlineto{\pgfqpoint{7.933072in}{2.218599in}}%
\pgfusepath{stroke}%
\end{pgfscope}%
\begin{pgfscope}%
\pgfpathrectangle{\pgfqpoint{6.720588in}{1.750000in}}{\pgfqpoint{2.279412in}{2.004545in}}%
\pgfusepath{clip}%
\pgfsetbuttcap%
\pgfsetroundjoin%
\pgfsetlinewidth{0.387681pt}%
\definecolor{currentstroke}{rgb}{0.280267,0.073417,0.397163}%
\pgfsetstrokecolor{currentstroke}%
\pgfsetdash{}{0pt}%
\pgfpathmoveto{\pgfqpoint{7.933072in}{2.218599in}}%
\pgfpathlineto{\pgfqpoint{7.886822in}{2.235338in}}%
\pgfusepath{stroke}%
\end{pgfscope}%
\begin{pgfscope}%
\pgfpathrectangle{\pgfqpoint{6.720588in}{1.750000in}}{\pgfqpoint{2.279412in}{2.004545in}}%
\pgfusepath{clip}%
\pgfsetbuttcap%
\pgfsetroundjoin%
\pgfsetlinewidth{0.362707pt}%
\definecolor{currentstroke}{rgb}{0.277018,0.050344,0.375715}%
\pgfsetstrokecolor{currentstroke}%
\pgfsetdash{}{0pt}%
\pgfpathmoveto{\pgfqpoint{7.886822in}{2.235338in}}%
\pgfpathlineto{\pgfqpoint{7.842129in}{2.255235in}}%
\pgfusepath{stroke}%
\end{pgfscope}%
\begin{pgfscope}%
\pgfpathrectangle{\pgfqpoint{6.720588in}{1.750000in}}{\pgfqpoint{2.279412in}{2.004545in}}%
\pgfusepath{clip}%
\pgfsetbuttcap%
\pgfsetroundjoin%
\pgfsetlinewidth{0.367452pt}%
\definecolor{currentstroke}{rgb}{0.277941,0.056324,0.381191}%
\pgfsetstrokecolor{currentstroke}%
\pgfsetdash{}{0pt}%
\pgfpathmoveto{\pgfqpoint{7.842129in}{2.255235in}}%
\pgfpathlineto{\pgfqpoint{7.799587in}{2.278375in}}%
\pgfusepath{stroke}%
\end{pgfscope}%
\begin{pgfscope}%
\pgfpathrectangle{\pgfqpoint{6.720588in}{1.750000in}}{\pgfqpoint{2.279412in}{2.004545in}}%
\pgfusepath{clip}%
\pgfsetbuttcap%
\pgfsetroundjoin%
\pgfsetlinewidth{0.389010pt}%
\definecolor{currentstroke}{rgb}{0.280267,0.073417,0.397163}%
\pgfsetstrokecolor{currentstroke}%
\pgfsetdash{}{0pt}%
\pgfpathmoveto{\pgfqpoint{7.799587in}{2.278375in}}%
\pgfpathlineto{\pgfqpoint{7.799587in}{2.278375in}}%
\pgfusepath{stroke}%
\end{pgfscope}%
\begin{pgfscope}%
\pgfpathrectangle{\pgfqpoint{6.720588in}{1.750000in}}{\pgfqpoint{2.279412in}{2.004545in}}%
\pgfusepath{clip}%
\pgfsetbuttcap%
\pgfsetroundjoin%
\pgfsetlinewidth{0.389010pt}%
\definecolor{currentstroke}{rgb}{0.280267,0.073417,0.397163}%
\pgfsetstrokecolor{currentstroke}%
\pgfsetdash{}{0pt}%
\pgfpathmoveto{\pgfqpoint{7.799587in}{2.278375in}}%
\pgfpathlineto{\pgfqpoint{7.768842in}{2.302756in}}%
\pgfusepath{stroke}%
\end{pgfscope}%
\begin{pgfscope}%
\pgfpathrectangle{\pgfqpoint{6.720588in}{1.750000in}}{\pgfqpoint{2.279412in}{2.004545in}}%
\pgfusepath{clip}%
\pgfsetbuttcap%
\pgfsetroundjoin%
\pgfsetlinewidth{0.366362pt}%
\definecolor{currentstroke}{rgb}{0.277941,0.056324,0.381191}%
\pgfsetstrokecolor{currentstroke}%
\pgfsetdash{}{0pt}%
\pgfpathmoveto{\pgfqpoint{7.768842in}{2.302756in}}%
\pgfpathlineto{\pgfqpoint{7.745619in}{2.331598in}}%
\pgfusepath{stroke}%
\end{pgfscope}%
\begin{pgfscope}%
\pgfpathrectangle{\pgfqpoint{6.720588in}{1.750000in}}{\pgfqpoint{2.279412in}{2.004545in}}%
\pgfusepath{clip}%
\pgfsetbuttcap%
\pgfsetroundjoin%
\pgfsetlinewidth{0.320664pt}%
\definecolor{currentstroke}{rgb}{0.269944,0.014625,0.341379}%
\pgfsetstrokecolor{currentstroke}%
\pgfsetdash{}{0pt}%
\pgfpathmoveto{\pgfqpoint{8.578381in}{2.256098in}}%
\pgfpathlineto{\pgfqpoint{8.528258in}{2.256011in}}%
\pgfusepath{stroke}%
\end{pgfscope}%
\begin{pgfscope}%
\pgfpathrectangle{\pgfqpoint{6.720588in}{1.750000in}}{\pgfqpoint{2.279412in}{2.004545in}}%
\pgfusepath{clip}%
\pgfsetbuttcap%
\pgfsetroundjoin%
\pgfsetlinewidth{0.313446pt}%
\definecolor{currentstroke}{rgb}{0.268510,0.009605,0.335427}%
\pgfsetstrokecolor{currentstroke}%
\pgfsetdash{}{0pt}%
\pgfpathmoveto{\pgfqpoint{8.528258in}{2.256011in}}%
\pgfpathlineto{\pgfqpoint{8.478220in}{2.256744in}}%
\pgfusepath{stroke}%
\end{pgfscope}%
\begin{pgfscope}%
\pgfpathrectangle{\pgfqpoint{6.720588in}{1.750000in}}{\pgfqpoint{2.279412in}{2.004545in}}%
\pgfusepath{clip}%
\pgfsetbuttcap%
\pgfsetroundjoin%
\pgfsetlinewidth{0.319803pt}%
\definecolor{currentstroke}{rgb}{0.269944,0.014625,0.341379}%
\pgfsetstrokecolor{currentstroke}%
\pgfsetdash{}{0pt}%
\pgfpathmoveto{\pgfqpoint{8.478220in}{2.256744in}}%
\pgfpathlineto{\pgfqpoint{8.428183in}{2.258719in}}%
\pgfusepath{stroke}%
\end{pgfscope}%
\begin{pgfscope}%
\pgfpathrectangle{\pgfqpoint{6.720588in}{1.750000in}}{\pgfqpoint{2.279412in}{2.004545in}}%
\pgfusepath{clip}%
\pgfsetbuttcap%
\pgfsetroundjoin%
\pgfsetlinewidth{0.323399pt}%
\definecolor{currentstroke}{rgb}{0.271305,0.019942,0.347269}%
\pgfsetstrokecolor{currentstroke}%
\pgfsetdash{}{0pt}%
\pgfpathmoveto{\pgfqpoint{8.428183in}{2.258719in}}%
\pgfpathlineto{\pgfqpoint{8.378055in}{2.259865in}}%
\pgfusepath{stroke}%
\end{pgfscope}%
\begin{pgfscope}%
\pgfpathrectangle{\pgfqpoint{6.720588in}{1.750000in}}{\pgfqpoint{2.279412in}{2.004545in}}%
\pgfusepath{clip}%
\pgfsetbuttcap%
\pgfsetroundjoin%
\pgfsetlinewidth{0.346094pt}%
\definecolor{currentstroke}{rgb}{0.274952,0.037752,0.364543}%
\pgfsetstrokecolor{currentstroke}%
\pgfsetdash{}{0pt}%
\pgfpathmoveto{\pgfqpoint{8.378055in}{2.259865in}}%
\pgfpathlineto{\pgfqpoint{8.327921in}{2.260935in}}%
\pgfusepath{stroke}%
\end{pgfscope}%
\begin{pgfscope}%
\pgfpathrectangle{\pgfqpoint{6.720588in}{1.750000in}}{\pgfqpoint{2.279412in}{2.004545in}}%
\pgfusepath{clip}%
\pgfsetbuttcap%
\pgfsetroundjoin%
\pgfsetlinewidth{0.347820pt}%
\definecolor{currentstroke}{rgb}{0.274952,0.037752,0.364543}%
\pgfsetstrokecolor{currentstroke}%
\pgfsetdash{}{0pt}%
\pgfpathmoveto{\pgfqpoint{8.327921in}{2.260935in}}%
\pgfpathlineto{\pgfqpoint{8.277806in}{2.262447in}}%
\pgfusepath{stroke}%
\end{pgfscope}%
\begin{pgfscope}%
\pgfpathrectangle{\pgfqpoint{6.720588in}{1.750000in}}{\pgfqpoint{2.279412in}{2.004545in}}%
\pgfusepath{clip}%
\pgfsetbuttcap%
\pgfsetroundjoin%
\pgfsetlinewidth{0.361782pt}%
\definecolor{currentstroke}{rgb}{0.277018,0.050344,0.375715}%
\pgfsetstrokecolor{currentstroke}%
\pgfsetdash{}{0pt}%
\pgfpathmoveto{\pgfqpoint{8.277806in}{2.262447in}}%
\pgfpathlineto{\pgfqpoint{8.227721in}{2.264609in}}%
\pgfusepath{stroke}%
\end{pgfscope}%
\begin{pgfscope}%
\pgfpathrectangle{\pgfqpoint{6.720588in}{1.750000in}}{\pgfqpoint{2.279412in}{2.004545in}}%
\pgfusepath{clip}%
\pgfsetbuttcap%
\pgfsetroundjoin%
\pgfsetlinewidth{0.375632pt}%
\definecolor{currentstroke}{rgb}{0.278791,0.062145,0.386592}%
\pgfsetstrokecolor{currentstroke}%
\pgfsetdash{}{0pt}%
\pgfpathmoveto{\pgfqpoint{8.227721in}{2.264609in}}%
\pgfpathlineto{\pgfqpoint{8.177703in}{2.267714in}}%
\pgfusepath{stroke}%
\end{pgfscope}%
\begin{pgfscope}%
\pgfpathrectangle{\pgfqpoint{6.720588in}{1.750000in}}{\pgfqpoint{2.279412in}{2.004545in}}%
\pgfusepath{clip}%
\pgfsetbuttcap%
\pgfsetroundjoin%
\pgfsetlinewidth{0.387996pt}%
\definecolor{currentstroke}{rgb}{0.280267,0.073417,0.397163}%
\pgfsetstrokecolor{currentstroke}%
\pgfsetdash{}{0pt}%
\pgfpathmoveto{\pgfqpoint{8.177703in}{2.267714in}}%
\pgfpathlineto{\pgfqpoint{8.127785in}{2.271897in}}%
\pgfusepath{stroke}%
\end{pgfscope}%
\begin{pgfscope}%
\pgfpathrectangle{\pgfqpoint{6.720588in}{1.750000in}}{\pgfqpoint{2.279412in}{2.004545in}}%
\pgfusepath{clip}%
\pgfsetbuttcap%
\pgfsetroundjoin%
\pgfsetlinewidth{0.387416pt}%
\definecolor{currentstroke}{rgb}{0.280267,0.073417,0.397163}%
\pgfsetstrokecolor{currentstroke}%
\pgfsetdash{}{0pt}%
\pgfpathmoveto{\pgfqpoint{8.127785in}{2.271897in}}%
\pgfpathlineto{\pgfqpoint{8.077982in}{2.277061in}}%
\pgfusepath{stroke}%
\end{pgfscope}%
\begin{pgfscope}%
\pgfpathrectangle{\pgfqpoint{6.720588in}{1.750000in}}{\pgfqpoint{2.279412in}{2.004545in}}%
\pgfusepath{clip}%
\pgfsetbuttcap%
\pgfsetroundjoin%
\pgfsetlinewidth{0.378658pt}%
\definecolor{currentstroke}{rgb}{0.279566,0.067836,0.391917}%
\pgfsetstrokecolor{currentstroke}%
\pgfsetdash{}{0pt}%
\pgfpathmoveto{\pgfqpoint{8.077982in}{2.277061in}}%
\pgfpathlineto{\pgfqpoint{8.028400in}{2.283585in}}%
\pgfusepath{stroke}%
\end{pgfscope}%
\begin{pgfscope}%
\pgfpathrectangle{\pgfqpoint{6.720588in}{1.750000in}}{\pgfqpoint{2.279412in}{2.004545in}}%
\pgfusepath{clip}%
\pgfsetbuttcap%
\pgfsetroundjoin%
\pgfsetlinewidth{0.407103pt}%
\definecolor{currentstroke}{rgb}{0.281924,0.089666,0.412415}%
\pgfsetstrokecolor{currentstroke}%
\pgfsetdash{}{0pt}%
\pgfpathmoveto{\pgfqpoint{8.028400in}{2.283585in}}%
\pgfpathlineto{\pgfqpoint{7.979317in}{2.292216in}}%
\pgfusepath{stroke}%
\end{pgfscope}%
\begin{pgfscope}%
\pgfpathrectangle{\pgfqpoint{6.720588in}{1.750000in}}{\pgfqpoint{2.279412in}{2.004545in}}%
\pgfusepath{clip}%
\pgfsetbuttcap%
\pgfsetroundjoin%
\pgfsetlinewidth{0.401047pt}%
\definecolor{currentstroke}{rgb}{0.281446,0.084320,0.407414}%
\pgfsetstrokecolor{currentstroke}%
\pgfsetdash{}{0pt}%
\pgfpathmoveto{\pgfqpoint{7.979317in}{2.292216in}}%
\pgfpathlineto{\pgfqpoint{7.930655in}{2.302591in}}%
\pgfusepath{stroke}%
\end{pgfscope}%
\begin{pgfscope}%
\pgfpathrectangle{\pgfqpoint{6.720588in}{1.750000in}}{\pgfqpoint{2.279412in}{2.004545in}}%
\pgfusepath{clip}%
\pgfsetbuttcap%
\pgfsetroundjoin%
\pgfsetlinewidth{0.396294pt}%
\definecolor{currentstroke}{rgb}{0.280894,0.078907,0.402329}%
\pgfsetstrokecolor{currentstroke}%
\pgfsetdash{}{0pt}%
\pgfpathmoveto{\pgfqpoint{7.930655in}{2.302591in}}%
\pgfpathlineto{\pgfqpoint{7.883161in}{2.316117in}}%
\pgfusepath{stroke}%
\end{pgfscope}%
\begin{pgfscope}%
\pgfpathrectangle{\pgfqpoint{6.720588in}{1.750000in}}{\pgfqpoint{2.279412in}{2.004545in}}%
\pgfusepath{clip}%
\pgfsetbuttcap%
\pgfsetroundjoin%
\pgfsetlinewidth{0.436377pt}%
\definecolor{currentstroke}{rgb}{0.283091,0.110553,0.431554}%
\pgfsetstrokecolor{currentstroke}%
\pgfsetdash{}{0pt}%
\pgfpathmoveto{\pgfqpoint{7.883161in}{2.316117in}}%
\pgfpathlineto{\pgfqpoint{7.837756in}{2.334681in}}%
\pgfusepath{stroke}%
\end{pgfscope}%
\begin{pgfscope}%
\pgfpathrectangle{\pgfqpoint{6.720588in}{1.750000in}}{\pgfqpoint{2.279412in}{2.004545in}}%
\pgfusepath{clip}%
\pgfsetbuttcap%
\pgfsetroundjoin%
\pgfsetlinewidth{0.457848pt}%
\definecolor{currentstroke}{rgb}{0.283187,0.125848,0.444960}%
\pgfsetstrokecolor{currentstroke}%
\pgfsetdash{}{0pt}%
\pgfpathmoveto{\pgfqpoint{7.837756in}{2.334681in}}%
\pgfpathlineto{\pgfqpoint{7.793858in}{2.355996in}}%
\pgfusepath{stroke}%
\end{pgfscope}%
\begin{pgfscope}%
\pgfpathrectangle{\pgfqpoint{6.720588in}{1.750000in}}{\pgfqpoint{2.279412in}{2.004545in}}%
\pgfusepath{clip}%
\pgfsetbuttcap%
\pgfsetroundjoin%
\pgfsetlinewidth{0.414996pt}%
\definecolor{currentstroke}{rgb}{0.282327,0.094955,0.417331}%
\pgfsetstrokecolor{currentstroke}%
\pgfsetdash{}{0pt}%
\pgfpathmoveto{\pgfqpoint{7.793858in}{2.355996in}}%
\pgfpathlineto{\pgfqpoint{7.753501in}{2.381562in}}%
\pgfusepath{stroke}%
\end{pgfscope}%
\begin{pgfscope}%
\pgfpathrectangle{\pgfqpoint{6.720588in}{1.750000in}}{\pgfqpoint{2.279412in}{2.004545in}}%
\pgfusepath{clip}%
\pgfsetbuttcap%
\pgfsetroundjoin%
\pgfsetlinewidth{0.317489pt}%
\definecolor{currentstroke}{rgb}{0.269944,0.014625,0.341379}%
\pgfsetstrokecolor{currentstroke}%
\pgfsetdash{}{0pt}%
\pgfpathmoveto{\pgfqpoint{8.644093in}{2.297971in}}%
\pgfpathlineto{\pgfqpoint{8.597076in}{2.300014in}}%
\pgfusepath{stroke}%
\end{pgfscope}%
\begin{pgfscope}%
\pgfpathrectangle{\pgfqpoint{6.720588in}{1.750000in}}{\pgfqpoint{2.279412in}{2.004545in}}%
\pgfusepath{clip}%
\pgfsetbuttcap%
\pgfsetroundjoin%
\pgfsetlinewidth{0.307698pt}%
\definecolor{currentstroke}{rgb}{0.267004,0.004874,0.329415}%
\pgfsetstrokecolor{currentstroke}%
\pgfsetdash{}{0pt}%
\pgfpathmoveto{\pgfqpoint{8.597076in}{2.300014in}}%
\pgfpathlineto{\pgfqpoint{8.578381in}{2.301205in}}%
\pgfusepath{stroke}%
\end{pgfscope}%
\begin{pgfscope}%
\pgfpathrectangle{\pgfqpoint{6.720588in}{1.750000in}}{\pgfqpoint{2.279412in}{2.004545in}}%
\pgfusepath{clip}%
\pgfsetbuttcap%
\pgfsetroundjoin%
\pgfsetlinewidth{0.310594pt}%
\definecolor{currentstroke}{rgb}{0.268510,0.009605,0.335427}%
\pgfsetstrokecolor{currentstroke}%
\pgfsetdash{}{0pt}%
\pgfpathmoveto{\pgfqpoint{8.578381in}{2.301205in}}%
\pgfpathlineto{\pgfqpoint{8.578381in}{2.301205in}}%
\pgfusepath{stroke}%
\end{pgfscope}%
\begin{pgfscope}%
\pgfpathrectangle{\pgfqpoint{6.720588in}{1.750000in}}{\pgfqpoint{2.279412in}{2.004545in}}%
\pgfusepath{clip}%
\pgfsetbuttcap%
\pgfsetroundjoin%
\pgfsetlinewidth{0.310594pt}%
\definecolor{currentstroke}{rgb}{0.268510,0.009605,0.335427}%
\pgfsetstrokecolor{currentstroke}%
\pgfsetdash{}{0pt}%
\pgfpathmoveto{\pgfqpoint{8.578381in}{2.301205in}}%
\pgfpathlineto{\pgfqpoint{8.528460in}{2.303109in}}%
\pgfusepath{stroke}%
\end{pgfscope}%
\begin{pgfscope}%
\pgfpathrectangle{\pgfqpoint{6.720588in}{1.750000in}}{\pgfqpoint{2.279412in}{2.004545in}}%
\pgfusepath{clip}%
\pgfsetbuttcap%
\pgfsetroundjoin%
\pgfsetlinewidth{0.317422pt}%
\definecolor{currentstroke}{rgb}{0.269944,0.014625,0.341379}%
\pgfsetstrokecolor{currentstroke}%
\pgfsetdash{}{0pt}%
\pgfpathmoveto{\pgfqpoint{8.528460in}{2.303109in}}%
\pgfpathlineto{\pgfqpoint{8.478379in}{2.302871in}}%
\pgfusepath{stroke}%
\end{pgfscope}%
\begin{pgfscope}%
\pgfpathrectangle{\pgfqpoint{6.720588in}{1.750000in}}{\pgfqpoint{2.279412in}{2.004545in}}%
\pgfusepath{clip}%
\pgfsetbuttcap%
\pgfsetroundjoin%
\pgfsetlinewidth{0.318434pt}%
\definecolor{currentstroke}{rgb}{0.269944,0.014625,0.341379}%
\pgfsetstrokecolor{currentstroke}%
\pgfsetdash{}{0pt}%
\pgfpathmoveto{\pgfqpoint{8.478379in}{2.302871in}}%
\pgfpathlineto{\pgfqpoint{8.428286in}{2.302733in}}%
\pgfusepath{stroke}%
\end{pgfscope}%
\begin{pgfscope}%
\pgfpathrectangle{\pgfqpoint{6.720588in}{1.750000in}}{\pgfqpoint{2.279412in}{2.004545in}}%
\pgfusepath{clip}%
\pgfsetbuttcap%
\pgfsetroundjoin%
\pgfsetlinewidth{0.328331pt}%
\definecolor{currentstroke}{rgb}{0.271305,0.019942,0.347269}%
\pgfsetstrokecolor{currentstroke}%
\pgfsetdash{}{0pt}%
\pgfpathmoveto{\pgfqpoint{8.428286in}{2.302733in}}%
\pgfpathlineto{\pgfqpoint{8.378233in}{2.304304in}}%
\pgfusepath{stroke}%
\end{pgfscope}%
\begin{pgfscope}%
\pgfpathrectangle{\pgfqpoint{6.720588in}{1.750000in}}{\pgfqpoint{2.279412in}{2.004545in}}%
\pgfusepath{clip}%
\pgfsetbuttcap%
\pgfsetroundjoin%
\pgfsetlinewidth{0.327262pt}%
\definecolor{currentstroke}{rgb}{0.271305,0.019942,0.347269}%
\pgfsetstrokecolor{currentstroke}%
\pgfsetdash{}{0pt}%
\pgfpathmoveto{\pgfqpoint{8.378233in}{2.304304in}}%
\pgfpathlineto{\pgfqpoint{8.328176in}{2.306523in}}%
\pgfusepath{stroke}%
\end{pgfscope}%
\begin{pgfscope}%
\pgfpathrectangle{\pgfqpoint{6.720588in}{1.750000in}}{\pgfqpoint{2.279412in}{2.004545in}}%
\pgfusepath{clip}%
\pgfsetbuttcap%
\pgfsetroundjoin%
\pgfsetlinewidth{0.349177pt}%
\definecolor{currentstroke}{rgb}{0.276022,0.044167,0.370164}%
\pgfsetstrokecolor{currentstroke}%
\pgfsetdash{}{0pt}%
\pgfpathmoveto{\pgfqpoint{8.328176in}{2.306523in}}%
\pgfpathlineto{\pgfqpoint{8.278095in}{2.308758in}}%
\pgfusepath{stroke}%
\end{pgfscope}%
\begin{pgfscope}%
\pgfpathrectangle{\pgfqpoint{6.720588in}{1.750000in}}{\pgfqpoint{2.279412in}{2.004545in}}%
\pgfusepath{clip}%
\pgfsetbuttcap%
\pgfsetroundjoin%
\pgfsetlinewidth{0.365210pt}%
\definecolor{currentstroke}{rgb}{0.277941,0.056324,0.381191}%
\pgfsetstrokecolor{currentstroke}%
\pgfsetdash{}{0pt}%
\pgfpathmoveto{\pgfqpoint{8.278095in}{2.308758in}}%
\pgfpathlineto{\pgfqpoint{8.228008in}{2.310930in}}%
\pgfusepath{stroke}%
\end{pgfscope}%
\begin{pgfscope}%
\pgfpathrectangle{\pgfqpoint{6.720588in}{1.750000in}}{\pgfqpoint{2.279412in}{2.004545in}}%
\pgfusepath{clip}%
\pgfsetbuttcap%
\pgfsetroundjoin%
\pgfsetlinewidth{0.390395pt}%
\definecolor{currentstroke}{rgb}{0.280894,0.078907,0.402329}%
\pgfsetstrokecolor{currentstroke}%
\pgfsetdash{}{0pt}%
\pgfpathmoveto{\pgfqpoint{8.228008in}{2.310930in}}%
\pgfpathlineto{\pgfqpoint{8.177945in}{2.313501in}}%
\pgfusepath{stroke}%
\end{pgfscope}%
\begin{pgfscope}%
\pgfpathrectangle{\pgfqpoint{6.720588in}{1.750000in}}{\pgfqpoint{2.279412in}{2.004545in}}%
\pgfusepath{clip}%
\pgfsetbuttcap%
\pgfsetroundjoin%
\pgfsetlinewidth{0.391466pt}%
\definecolor{currentstroke}{rgb}{0.280894,0.078907,0.402329}%
\pgfsetstrokecolor{currentstroke}%
\pgfsetdash{}{0pt}%
\pgfpathmoveto{\pgfqpoint{8.177945in}{2.313501in}}%
\pgfpathlineto{\pgfqpoint{8.127971in}{2.317127in}}%
\pgfusepath{stroke}%
\end{pgfscope}%
\begin{pgfscope}%
\pgfpathrectangle{\pgfqpoint{6.720588in}{1.750000in}}{\pgfqpoint{2.279412in}{2.004545in}}%
\pgfusepath{clip}%
\pgfsetbuttcap%
\pgfsetroundjoin%
\pgfsetlinewidth{0.404083pt}%
\definecolor{currentstroke}{rgb}{0.281924,0.089666,0.412415}%
\pgfsetstrokecolor{currentstroke}%
\pgfsetdash{}{0pt}%
\pgfpathmoveto{\pgfqpoint{8.127971in}{2.317127in}}%
\pgfpathlineto{\pgfqpoint{8.078093in}{2.321677in}}%
\pgfusepath{stroke}%
\end{pgfscope}%
\begin{pgfscope}%
\pgfpathrectangle{\pgfqpoint{6.720588in}{1.750000in}}{\pgfqpoint{2.279412in}{2.004545in}}%
\pgfusepath{clip}%
\pgfsetbuttcap%
\pgfsetroundjoin%
\pgfsetlinewidth{0.428940pt}%
\definecolor{currentstroke}{rgb}{0.282910,0.105393,0.426902}%
\pgfsetstrokecolor{currentstroke}%
\pgfsetdash{}{0pt}%
\pgfpathmoveto{\pgfqpoint{8.078093in}{2.321677in}}%
\pgfpathlineto{\pgfqpoint{8.028433in}{2.327605in}}%
\pgfusepath{stroke}%
\end{pgfscope}%
\begin{pgfscope}%
\pgfpathrectangle{\pgfqpoint{6.720588in}{1.750000in}}{\pgfqpoint{2.279412in}{2.004545in}}%
\pgfusepath{clip}%
\pgfsetbuttcap%
\pgfsetroundjoin%
\pgfsetlinewidth{0.422048pt}%
\definecolor{currentstroke}{rgb}{0.282656,0.100196,0.422160}%
\pgfsetstrokecolor{currentstroke}%
\pgfsetdash{}{0pt}%
\pgfpathmoveto{\pgfqpoint{8.028433in}{2.327605in}}%
\pgfpathlineto{\pgfqpoint{7.979116in}{2.335518in}}%
\pgfusepath{stroke}%
\end{pgfscope}%
\begin{pgfscope}%
\pgfpathrectangle{\pgfqpoint{6.720588in}{1.750000in}}{\pgfqpoint{2.279412in}{2.004545in}}%
\pgfusepath{clip}%
\pgfsetbuttcap%
\pgfsetroundjoin%
\pgfsetlinewidth{0.465010pt}%
\definecolor{currentstroke}{rgb}{0.283072,0.130895,0.449241}%
\pgfsetstrokecolor{currentstroke}%
\pgfsetdash{}{0pt}%
\pgfpathmoveto{\pgfqpoint{7.979116in}{2.335518in}}%
\pgfpathlineto{\pgfqpoint{7.930280in}{2.345433in}}%
\pgfusepath{stroke}%
\end{pgfscope}%
\begin{pgfscope}%
\pgfpathrectangle{\pgfqpoint{6.720588in}{1.750000in}}{\pgfqpoint{2.279412in}{2.004545in}}%
\pgfusepath{clip}%
\pgfsetbuttcap%
\pgfsetroundjoin%
\pgfsetlinewidth{0.326887pt}%
\definecolor{currentstroke}{rgb}{0.271305,0.019942,0.347269}%
\pgfsetstrokecolor{currentstroke}%
\pgfsetdash{}{0pt}%
\pgfpathmoveto{\pgfqpoint{8.578381in}{2.346312in}}%
\pgfpathlineto{\pgfqpoint{8.528247in}{2.346712in}}%
\pgfusepath{stroke}%
\end{pgfscope}%
\begin{pgfscope}%
\pgfpathrectangle{\pgfqpoint{6.720588in}{1.750000in}}{\pgfqpoint{2.279412in}{2.004545in}}%
\pgfusepath{clip}%
\pgfsetbuttcap%
\pgfsetroundjoin%
\pgfsetlinewidth{0.317416pt}%
\definecolor{currentstroke}{rgb}{0.269944,0.014625,0.341379}%
\pgfsetstrokecolor{currentstroke}%
\pgfsetdash{}{0pt}%
\pgfpathmoveto{\pgfqpoint{8.528247in}{2.346712in}}%
\pgfpathlineto{\pgfqpoint{8.478148in}{2.347563in}}%
\pgfusepath{stroke}%
\end{pgfscope}%
\begin{pgfscope}%
\pgfpathrectangle{\pgfqpoint{6.720588in}{1.750000in}}{\pgfqpoint{2.279412in}{2.004545in}}%
\pgfusepath{clip}%
\pgfsetbuttcap%
\pgfsetroundjoin%
\pgfsetlinewidth{0.321282pt}%
\definecolor{currentstroke}{rgb}{0.269944,0.014625,0.341379}%
\pgfsetstrokecolor{currentstroke}%
\pgfsetdash{}{0pt}%
\pgfpathmoveto{\pgfqpoint{8.478148in}{2.347563in}}%
\pgfpathlineto{\pgfqpoint{8.428047in}{2.348745in}}%
\pgfusepath{stroke}%
\end{pgfscope}%
\begin{pgfscope}%
\pgfpathrectangle{\pgfqpoint{6.720588in}{1.750000in}}{\pgfqpoint{2.279412in}{2.004545in}}%
\pgfusepath{clip}%
\pgfsetbuttcap%
\pgfsetroundjoin%
\pgfsetlinewidth{0.325669pt}%
\definecolor{currentstroke}{rgb}{0.271305,0.019942,0.347269}%
\pgfsetstrokecolor{currentstroke}%
\pgfsetdash{}{0pt}%
\pgfpathmoveto{\pgfqpoint{8.428047in}{2.348745in}}%
\pgfpathlineto{\pgfqpoint{8.377940in}{2.350398in}}%
\pgfusepath{stroke}%
\end{pgfscope}%
\begin{pgfscope}%
\pgfpathrectangle{\pgfqpoint{6.720588in}{1.750000in}}{\pgfqpoint{2.279412in}{2.004545in}}%
\pgfusepath{clip}%
\pgfsetbuttcap%
\pgfsetroundjoin%
\pgfsetlinewidth{0.341560pt}%
\definecolor{currentstroke}{rgb}{0.273809,0.031497,0.358853}%
\pgfsetstrokecolor{currentstroke}%
\pgfsetdash{}{0pt}%
\pgfpathmoveto{\pgfqpoint{8.377940in}{2.350398in}}%
\pgfpathlineto{\pgfqpoint{8.327857in}{2.352667in}}%
\pgfusepath{stroke}%
\end{pgfscope}%
\begin{pgfscope}%
\pgfpathrectangle{\pgfqpoint{6.720588in}{1.750000in}}{\pgfqpoint{2.279412in}{2.004545in}}%
\pgfusepath{clip}%
\pgfsetbuttcap%
\pgfsetroundjoin%
\pgfsetlinewidth{0.354989pt}%
\definecolor{currentstroke}{rgb}{0.276022,0.044167,0.370164}%
\pgfsetstrokecolor{currentstroke}%
\pgfsetdash{}{0pt}%
\pgfpathmoveto{\pgfqpoint{8.327857in}{2.352667in}}%
\pgfpathlineto{\pgfqpoint{8.277773in}{2.354904in}}%
\pgfusepath{stroke}%
\end{pgfscope}%
\begin{pgfscope}%
\pgfpathrectangle{\pgfqpoint{6.720588in}{1.750000in}}{\pgfqpoint{2.279412in}{2.004545in}}%
\pgfusepath{clip}%
\pgfsetbuttcap%
\pgfsetroundjoin%
\pgfsetlinewidth{0.378279pt}%
\definecolor{currentstroke}{rgb}{0.279566,0.067836,0.391917}%
\pgfsetstrokecolor{currentstroke}%
\pgfsetdash{}{0pt}%
\pgfpathmoveto{\pgfqpoint{8.277773in}{2.354904in}}%
\pgfpathlineto{\pgfqpoint{8.227692in}{2.357205in}}%
\pgfusepath{stroke}%
\end{pgfscope}%
\begin{pgfscope}%
\pgfpathrectangle{\pgfqpoint{6.720588in}{1.750000in}}{\pgfqpoint{2.279412in}{2.004545in}}%
\pgfusepath{clip}%
\pgfsetbuttcap%
\pgfsetroundjoin%
\pgfsetlinewidth{0.407372pt}%
\definecolor{currentstroke}{rgb}{0.281924,0.089666,0.412415}%
\pgfsetstrokecolor{currentstroke}%
\pgfsetdash{}{0pt}%
\pgfpathmoveto{\pgfqpoint{8.227692in}{2.357205in}}%
\pgfpathlineto{\pgfqpoint{8.177635in}{2.359898in}}%
\pgfusepath{stroke}%
\end{pgfscope}%
\begin{pgfscope}%
\pgfpathrectangle{\pgfqpoint{6.720588in}{1.750000in}}{\pgfqpoint{2.279412in}{2.004545in}}%
\pgfusepath{clip}%
\pgfsetbuttcap%
\pgfsetroundjoin%
\pgfsetlinewidth{0.401709pt}%
\definecolor{currentstroke}{rgb}{0.281446,0.084320,0.407414}%
\pgfsetstrokecolor{currentstroke}%
\pgfsetdash{}{0pt}%
\pgfpathmoveto{\pgfqpoint{8.177635in}{2.359898in}}%
\pgfpathlineto{\pgfqpoint{8.127604in}{2.362952in}}%
\pgfusepath{stroke}%
\end{pgfscope}%
\begin{pgfscope}%
\pgfpathrectangle{\pgfqpoint{6.720588in}{1.750000in}}{\pgfqpoint{2.279412in}{2.004545in}}%
\pgfusepath{clip}%
\pgfsetbuttcap%
\pgfsetroundjoin%
\pgfsetlinewidth{0.423714pt}%
\definecolor{currentstroke}{rgb}{0.282656,0.100196,0.422160}%
\pgfsetstrokecolor{currentstroke}%
\pgfsetdash{}{0pt}%
\pgfpathmoveto{\pgfqpoint{8.127604in}{2.362952in}}%
\pgfpathlineto{\pgfqpoint{8.077664in}{2.366880in}}%
\pgfusepath{stroke}%
\end{pgfscope}%
\begin{pgfscope}%
\pgfpathrectangle{\pgfqpoint{6.720588in}{1.750000in}}{\pgfqpoint{2.279412in}{2.004545in}}%
\pgfusepath{clip}%
\pgfsetbuttcap%
\pgfsetroundjoin%
\pgfsetlinewidth{0.446275pt}%
\definecolor{currentstroke}{rgb}{0.283229,0.120777,0.440584}%
\pgfsetstrokecolor{currentstroke}%
\pgfsetdash{}{0pt}%
\pgfpathmoveto{\pgfqpoint{8.077664in}{2.366880in}}%
\pgfpathlineto{\pgfqpoint{8.027931in}{2.372477in}}%
\pgfusepath{stroke}%
\end{pgfscope}%
\begin{pgfscope}%
\pgfpathrectangle{\pgfqpoint{6.720588in}{1.750000in}}{\pgfqpoint{2.279412in}{2.004545in}}%
\pgfusepath{clip}%
\pgfsetbuttcap%
\pgfsetroundjoin%
\pgfsetlinewidth{0.483752pt}%
\definecolor{currentstroke}{rgb}{0.282290,0.145912,0.461510}%
\pgfsetstrokecolor{currentstroke}%
\pgfsetdash{}{0pt}%
\pgfpathmoveto{\pgfqpoint{8.027931in}{2.372477in}}%
\pgfpathlineto{\pgfqpoint{7.978448in}{2.379606in}}%
\pgfusepath{stroke}%
\end{pgfscope}%
\begin{pgfscope}%
\pgfpathrectangle{\pgfqpoint{6.720588in}{1.750000in}}{\pgfqpoint{2.279412in}{2.004545in}}%
\pgfusepath{clip}%
\pgfsetbuttcap%
\pgfsetroundjoin%
\pgfsetlinewidth{0.478967pt}%
\definecolor{currentstroke}{rgb}{0.282623,0.140926,0.457517}%
\pgfsetstrokecolor{currentstroke}%
\pgfsetdash{}{0pt}%
\pgfpathmoveto{\pgfqpoint{7.978448in}{2.379606in}}%
\pgfpathlineto{\pgfqpoint{7.929365in}{2.388569in}}%
\pgfusepath{stroke}%
\end{pgfscope}%
\begin{pgfscope}%
\pgfpathrectangle{\pgfqpoint{6.720588in}{1.750000in}}{\pgfqpoint{2.279412in}{2.004545in}}%
\pgfusepath{clip}%
\pgfsetbuttcap%
\pgfsetroundjoin%
\pgfsetlinewidth{0.497443pt}%
\definecolor{currentstroke}{rgb}{0.281412,0.155834,0.469201}%
\pgfsetstrokecolor{currentstroke}%
\pgfsetdash{}{0pt}%
\pgfpathmoveto{\pgfqpoint{7.929365in}{2.388569in}}%
\pgfpathlineto{\pgfqpoint{7.880893in}{2.399809in}}%
\pgfusepath{stroke}%
\end{pgfscope}%
\begin{pgfscope}%
\pgfpathrectangle{\pgfqpoint{6.720588in}{1.750000in}}{\pgfqpoint{2.279412in}{2.004545in}}%
\pgfusepath{clip}%
\pgfsetbuttcap%
\pgfsetroundjoin%
\pgfsetlinewidth{0.497071pt}%
\definecolor{currentstroke}{rgb}{0.281412,0.155834,0.469201}%
\pgfsetstrokecolor{currentstroke}%
\pgfsetdash{}{0pt}%
\pgfpathmoveto{\pgfqpoint{7.880893in}{2.399809in}}%
\pgfpathlineto{\pgfqpoint{7.833159in}{2.413268in}}%
\pgfusepath{stroke}%
\end{pgfscope}%
\begin{pgfscope}%
\pgfpathrectangle{\pgfqpoint{6.720588in}{1.750000in}}{\pgfqpoint{2.279412in}{2.004545in}}%
\pgfusepath{clip}%
\pgfsetbuttcap%
\pgfsetroundjoin%
\pgfsetlinewidth{0.535976pt}%
\definecolor{currentstroke}{rgb}{0.277134,0.185228,0.489898}%
\pgfsetstrokecolor{currentstroke}%
\pgfsetdash{}{0pt}%
\pgfpathmoveto{\pgfqpoint{7.833159in}{2.413268in}}%
\pgfpathlineto{\pgfqpoint{7.786605in}{2.429438in}}%
\pgfusepath{stroke}%
\end{pgfscope}%
\begin{pgfscope}%
\pgfpathrectangle{\pgfqpoint{6.720588in}{1.750000in}}{\pgfqpoint{2.279412in}{2.004545in}}%
\pgfusepath{clip}%
\pgfsetbuttcap%
\pgfsetroundjoin%
\pgfsetlinewidth{0.525679pt}%
\definecolor{currentstroke}{rgb}{0.278826,0.175490,0.483397}%
\pgfsetstrokecolor{currentstroke}%
\pgfsetdash{}{0pt}%
\pgfpathmoveto{\pgfqpoint{7.786605in}{2.429438in}}%
\pgfpathlineto{\pgfqpoint{7.742225in}{2.449625in}}%
\pgfusepath{stroke}%
\end{pgfscope}%
\begin{pgfscope}%
\pgfpathrectangle{\pgfqpoint{6.720588in}{1.750000in}}{\pgfqpoint{2.279412in}{2.004545in}}%
\pgfusepath{clip}%
\pgfsetbuttcap%
\pgfsetroundjoin%
\pgfsetlinewidth{0.520914pt}%
\definecolor{currentstroke}{rgb}{0.278826,0.175490,0.483397}%
\pgfsetstrokecolor{currentstroke}%
\pgfsetdash{}{0pt}%
\pgfpathmoveto{\pgfqpoint{7.742225in}{2.449625in}}%
\pgfpathlineto{\pgfqpoint{7.700602in}{2.474060in}}%
\pgfusepath{stroke}%
\end{pgfscope}%
\begin{pgfscope}%
\pgfpathrectangle{\pgfqpoint{6.720588in}{1.750000in}}{\pgfqpoint{2.279412in}{2.004545in}}%
\pgfusepath{clip}%
\pgfsetbuttcap%
\pgfsetroundjoin%
\pgfsetlinewidth{0.320176pt}%
\definecolor{currentstroke}{rgb}{0.269944,0.014625,0.341379}%
\pgfsetstrokecolor{currentstroke}%
\pgfsetdash{}{0pt}%
\pgfpathmoveto{\pgfqpoint{8.578381in}{2.436525in}}%
\pgfpathlineto{\pgfqpoint{8.528253in}{2.437713in}}%
\pgfusepath{stroke}%
\end{pgfscope}%
\begin{pgfscope}%
\pgfpathrectangle{\pgfqpoint{6.720588in}{1.750000in}}{\pgfqpoint{2.279412in}{2.004545in}}%
\pgfusepath{clip}%
\pgfsetbuttcap%
\pgfsetroundjoin%
\pgfsetlinewidth{0.319231pt}%
\definecolor{currentstroke}{rgb}{0.269944,0.014625,0.341379}%
\pgfsetstrokecolor{currentstroke}%
\pgfsetdash{}{0pt}%
\pgfpathmoveto{\pgfqpoint{8.528253in}{2.437713in}}%
\pgfpathlineto{\pgfqpoint{8.478124in}{2.438092in}}%
\pgfusepath{stroke}%
\end{pgfscope}%
\begin{pgfscope}%
\pgfpathrectangle{\pgfqpoint{6.720588in}{1.750000in}}{\pgfqpoint{2.279412in}{2.004545in}}%
\pgfusepath{clip}%
\pgfsetbuttcap%
\pgfsetroundjoin%
\pgfsetlinewidth{0.316919pt}%
\definecolor{currentstroke}{rgb}{0.269944,0.014625,0.341379}%
\pgfsetstrokecolor{currentstroke}%
\pgfsetdash{}{0pt}%
\pgfpathmoveto{\pgfqpoint{8.478124in}{2.438092in}}%
\pgfpathlineto{\pgfqpoint{8.427994in}{2.438320in}}%
\pgfusepath{stroke}%
\end{pgfscope}%
\begin{pgfscope}%
\pgfpathrectangle{\pgfqpoint{6.720588in}{1.750000in}}{\pgfqpoint{2.279412in}{2.004545in}}%
\pgfusepath{clip}%
\pgfsetbuttcap%
\pgfsetroundjoin%
\pgfsetlinewidth{0.337112pt}%
\definecolor{currentstroke}{rgb}{0.273809,0.031497,0.358853}%
\pgfsetstrokecolor{currentstroke}%
\pgfsetdash{}{0pt}%
\pgfpathmoveto{\pgfqpoint{8.427994in}{2.438320in}}%
\pgfpathlineto{\pgfqpoint{8.377858in}{2.439237in}}%
\pgfusepath{stroke}%
\end{pgfscope}%
\begin{pgfscope}%
\pgfpathrectangle{\pgfqpoint{6.720588in}{1.750000in}}{\pgfqpoint{2.279412in}{2.004545in}}%
\pgfusepath{clip}%
\pgfsetbuttcap%
\pgfsetroundjoin%
\pgfsetlinewidth{0.362245pt}%
\definecolor{currentstroke}{rgb}{0.277018,0.050344,0.375715}%
\pgfsetstrokecolor{currentstroke}%
\pgfsetdash{}{0pt}%
\pgfpathmoveto{\pgfqpoint{8.377858in}{2.439237in}}%
\pgfpathlineto{\pgfqpoint{8.327729in}{2.440522in}}%
\pgfusepath{stroke}%
\end{pgfscope}%
\begin{pgfscope}%
\pgfpathrectangle{\pgfqpoint{6.720588in}{1.750000in}}{\pgfqpoint{2.279412in}{2.004545in}}%
\pgfusepath{clip}%
\pgfsetbuttcap%
\pgfsetroundjoin%
\pgfsetlinewidth{0.377348pt}%
\definecolor{currentstroke}{rgb}{0.279566,0.067836,0.391917}%
\pgfsetstrokecolor{currentstroke}%
\pgfsetdash{}{0pt}%
\pgfpathmoveto{\pgfqpoint{8.327729in}{2.440522in}}%
\pgfpathlineto{\pgfqpoint{8.277609in}{2.442020in}}%
\pgfusepath{stroke}%
\end{pgfscope}%
\begin{pgfscope}%
\pgfpathrectangle{\pgfqpoint{6.720588in}{1.750000in}}{\pgfqpoint{2.279412in}{2.004545in}}%
\pgfusepath{clip}%
\pgfsetbuttcap%
\pgfsetroundjoin%
\pgfsetlinewidth{0.411846pt}%
\definecolor{currentstroke}{rgb}{0.282327,0.094955,0.417331}%
\pgfsetstrokecolor{currentstroke}%
\pgfsetdash{}{0pt}%
\pgfpathmoveto{\pgfqpoint{8.277609in}{2.442020in}}%
\pgfpathlineto{\pgfqpoint{8.227500in}{2.443846in}}%
\pgfusepath{stroke}%
\end{pgfscope}%
\begin{pgfscope}%
\pgfpathrectangle{\pgfqpoint{6.720588in}{1.750000in}}{\pgfqpoint{2.279412in}{2.004545in}}%
\pgfusepath{clip}%
\pgfsetbuttcap%
\pgfsetroundjoin%
\pgfsetlinewidth{0.447010pt}%
\definecolor{currentstroke}{rgb}{0.283229,0.120777,0.440584}%
\pgfsetstrokecolor{currentstroke}%
\pgfsetdash{}{0pt}%
\pgfpathmoveto{\pgfqpoint{8.227500in}{2.443846in}}%
\pgfpathlineto{\pgfqpoint{8.177421in}{2.446168in}}%
\pgfusepath{stroke}%
\end{pgfscope}%
\begin{pgfscope}%
\pgfpathrectangle{\pgfqpoint{6.720588in}{1.750000in}}{\pgfqpoint{2.279412in}{2.004545in}}%
\pgfusepath{clip}%
\pgfsetbuttcap%
\pgfsetroundjoin%
\pgfsetlinewidth{0.478393pt}%
\definecolor{currentstroke}{rgb}{0.282623,0.140926,0.457517}%
\pgfsetstrokecolor{currentstroke}%
\pgfsetdash{}{0pt}%
\pgfpathmoveto{\pgfqpoint{8.177421in}{2.446168in}}%
\pgfpathlineto{\pgfqpoint{8.127384in}{2.449125in}}%
\pgfusepath{stroke}%
\end{pgfscope}%
\begin{pgfscope}%
\pgfpathrectangle{\pgfqpoint{6.720588in}{1.750000in}}{\pgfqpoint{2.279412in}{2.004545in}}%
\pgfusepath{clip}%
\pgfsetbuttcap%
\pgfsetroundjoin%
\pgfsetlinewidth{0.527092pt}%
\definecolor{currentstroke}{rgb}{0.278826,0.175490,0.483397}%
\pgfsetstrokecolor{currentstroke}%
\pgfsetdash{}{0pt}%
\pgfpathmoveto{\pgfqpoint{8.127384in}{2.449125in}}%
\pgfpathlineto{\pgfqpoint{8.077389in}{2.452588in}}%
\pgfusepath{stroke}%
\end{pgfscope}%
\begin{pgfscope}%
\pgfpathrectangle{\pgfqpoint{6.720588in}{1.750000in}}{\pgfqpoint{2.279412in}{2.004545in}}%
\pgfusepath{clip}%
\pgfsetbuttcap%
\pgfsetroundjoin%
\pgfsetlinewidth{0.528569pt}%
\definecolor{currentstroke}{rgb}{0.278012,0.180367,0.486697}%
\pgfsetstrokecolor{currentstroke}%
\pgfsetdash{}{0pt}%
\pgfpathmoveto{\pgfqpoint{8.077389in}{2.452588in}}%
\pgfpathlineto{\pgfqpoint{8.027477in}{2.456862in}}%
\pgfusepath{stroke}%
\end{pgfscope}%
\begin{pgfscope}%
\pgfpathrectangle{\pgfqpoint{6.720588in}{1.750000in}}{\pgfqpoint{2.279412in}{2.004545in}}%
\pgfusepath{clip}%
\pgfsetbuttcap%
\pgfsetroundjoin%
\pgfsetlinewidth{0.573626pt}%
\definecolor{currentstroke}{rgb}{0.271828,0.209303,0.504434}%
\pgfsetstrokecolor{currentstroke}%
\pgfsetdash{}{0pt}%
\pgfpathmoveto{\pgfqpoint{8.027477in}{2.456862in}}%
\pgfpathlineto{\pgfqpoint{7.977688in}{2.462124in}}%
\pgfusepath{stroke}%
\end{pgfscope}%
\begin{pgfscope}%
\pgfpathrectangle{\pgfqpoint{6.720588in}{1.750000in}}{\pgfqpoint{2.279412in}{2.004545in}}%
\pgfusepath{clip}%
\pgfsetbuttcap%
\pgfsetroundjoin%
\pgfsetlinewidth{0.604011pt}%
\definecolor{currentstroke}{rgb}{0.265145,0.232956,0.516599}%
\pgfsetstrokecolor{currentstroke}%
\pgfsetdash{}{0pt}%
\pgfpathmoveto{\pgfqpoint{7.977688in}{2.462124in}}%
\pgfpathlineto{\pgfqpoint{7.928099in}{2.468657in}}%
\pgfusepath{stroke}%
\end{pgfscope}%
\begin{pgfscope}%
\pgfpathrectangle{\pgfqpoint{6.720588in}{1.750000in}}{\pgfqpoint{2.279412in}{2.004545in}}%
\pgfusepath{clip}%
\pgfsetbuttcap%
\pgfsetroundjoin%
\pgfsetlinewidth{0.638196pt}%
\definecolor{currentstroke}{rgb}{0.257322,0.256130,0.526563}%
\pgfsetstrokecolor{currentstroke}%
\pgfsetdash{}{0pt}%
\pgfpathmoveto{\pgfqpoint{7.928099in}{2.468657in}}%
\pgfpathlineto{\pgfqpoint{7.878928in}{2.477224in}}%
\pgfusepath{stroke}%
\end{pgfscope}%
\begin{pgfscope}%
\pgfpathrectangle{\pgfqpoint{6.720588in}{1.750000in}}{\pgfqpoint{2.279412in}{2.004545in}}%
\pgfusepath{clip}%
\pgfsetbuttcap%
\pgfsetroundjoin%
\pgfsetlinewidth{0.597134pt}%
\definecolor{currentstroke}{rgb}{0.266580,0.228262,0.514349}%
\pgfsetstrokecolor{currentstroke}%
\pgfsetdash{}{0pt}%
\pgfpathmoveto{\pgfqpoint{7.878928in}{2.477224in}}%
\pgfpathlineto{\pgfqpoint{7.830319in}{2.488009in}}%
\pgfusepath{stroke}%
\end{pgfscope}%
\begin{pgfscope}%
\pgfpathrectangle{\pgfqpoint{6.720588in}{1.750000in}}{\pgfqpoint{2.279412in}{2.004545in}}%
\pgfusepath{clip}%
\pgfsetbuttcap%
\pgfsetroundjoin%
\pgfsetlinewidth{0.634633pt}%
\definecolor{currentstroke}{rgb}{0.258965,0.251537,0.524736}%
\pgfsetstrokecolor{currentstroke}%
\pgfsetdash{}{0pt}%
\pgfpathmoveto{\pgfqpoint{7.830319in}{2.488009in}}%
\pgfpathlineto{\pgfqpoint{7.782498in}{2.501160in}}%
\pgfusepath{stroke}%
\end{pgfscope}%
\begin{pgfscope}%
\pgfpathrectangle{\pgfqpoint{6.720588in}{1.750000in}}{\pgfqpoint{2.279412in}{2.004545in}}%
\pgfusepath{clip}%
\pgfsetbuttcap%
\pgfsetroundjoin%
\pgfsetlinewidth{0.598576pt}%
\definecolor{currentstroke}{rgb}{0.266580,0.228262,0.514349}%
\pgfsetstrokecolor{currentstroke}%
\pgfsetdash{}{0pt}%
\pgfpathmoveto{\pgfqpoint{7.782498in}{2.501160in}}%
\pgfpathlineto{\pgfqpoint{7.735642in}{2.516782in}}%
\pgfusepath{stroke}%
\end{pgfscope}%
\begin{pgfscope}%
\pgfpathrectangle{\pgfqpoint{6.720588in}{1.750000in}}{\pgfqpoint{2.279412in}{2.004545in}}%
\pgfusepath{clip}%
\pgfsetbuttcap%
\pgfsetroundjoin%
\pgfsetlinewidth{0.617755pt}%
\definecolor{currentstroke}{rgb}{0.262138,0.242286,0.520837}%
\pgfsetstrokecolor{currentstroke}%
\pgfsetdash{}{0pt}%
\pgfpathmoveto{\pgfqpoint{7.735642in}{2.516782in}}%
\pgfpathlineto{\pgfqpoint{7.690809in}{2.536264in}}%
\pgfusepath{stroke}%
\end{pgfscope}%
\begin{pgfscope}%
\pgfpathrectangle{\pgfqpoint{6.720588in}{1.750000in}}{\pgfqpoint{2.279412in}{2.004545in}}%
\pgfusepath{clip}%
\pgfsetbuttcap%
\pgfsetroundjoin%
\pgfsetlinewidth{0.553029pt}%
\definecolor{currentstroke}{rgb}{0.275191,0.194905,0.496005}%
\pgfsetstrokecolor{currentstroke}%
\pgfsetdash{}{0pt}%
\pgfpathmoveto{\pgfqpoint{7.690809in}{2.536264in}}%
\pgfpathlineto{\pgfqpoint{7.647938in}{2.558945in}}%
\pgfusepath{stroke}%
\end{pgfscope}%
\begin{pgfscope}%
\pgfpathrectangle{\pgfqpoint{6.720588in}{1.750000in}}{\pgfqpoint{2.279412in}{2.004545in}}%
\pgfusepath{clip}%
\pgfsetbuttcap%
\pgfsetroundjoin%
\pgfsetlinewidth{0.621867pt}%
\definecolor{currentstroke}{rgb}{0.262138,0.242286,0.520837}%
\pgfsetstrokecolor{currentstroke}%
\pgfsetdash{}{0pt}%
\pgfpathmoveto{\pgfqpoint{7.647938in}{2.558945in}}%
\pgfpathlineto{\pgfqpoint{7.606037in}{2.582968in}}%
\pgfusepath{stroke}%
\end{pgfscope}%
\begin{pgfscope}%
\pgfpathrectangle{\pgfqpoint{6.720588in}{1.750000in}}{\pgfqpoint{2.279412in}{2.004545in}}%
\pgfusepath{clip}%
\pgfsetbuttcap%
\pgfsetroundjoin%
\pgfsetlinewidth{0.618773pt}%
\definecolor{currentstroke}{rgb}{0.262138,0.242286,0.520837}%
\pgfsetstrokecolor{currentstroke}%
\pgfsetdash{}{0pt}%
\pgfpathmoveto{\pgfqpoint{7.606037in}{2.582968in}}%
\pgfpathlineto{\pgfqpoint{7.565372in}{2.608516in}}%
\pgfusepath{stroke}%
\end{pgfscope}%
\begin{pgfscope}%
\pgfpathrectangle{\pgfqpoint{6.720588in}{1.750000in}}{\pgfqpoint{2.279412in}{2.004545in}}%
\pgfusepath{clip}%
\pgfsetbuttcap%
\pgfsetroundjoin%
\pgfsetlinewidth{0.313257pt}%
\definecolor{currentstroke}{rgb}{0.268510,0.009605,0.335427}%
\pgfsetstrokecolor{currentstroke}%
\pgfsetdash{}{0pt}%
\pgfpathmoveto{\pgfqpoint{8.603141in}{2.528385in}}%
\pgfpathlineto{\pgfqpoint{8.578381in}{2.526739in}}%
\pgfusepath{stroke}%
\end{pgfscope}%
\begin{pgfscope}%
\pgfpathrectangle{\pgfqpoint{6.720588in}{1.750000in}}{\pgfqpoint{2.279412in}{2.004545in}}%
\pgfusepath{clip}%
\pgfsetbuttcap%
\pgfsetroundjoin%
\pgfsetlinewidth{0.305613pt}%
\definecolor{currentstroke}{rgb}{0.267004,0.004874,0.329415}%
\pgfsetstrokecolor{currentstroke}%
\pgfsetdash{}{0pt}%
\pgfpathmoveto{\pgfqpoint{8.578381in}{2.526739in}}%
\pgfpathlineto{\pgfqpoint{8.578381in}{2.526739in}}%
\pgfusepath{stroke}%
\end{pgfscope}%
\begin{pgfscope}%
\pgfpathrectangle{\pgfqpoint{6.720588in}{1.750000in}}{\pgfqpoint{2.279412in}{2.004545in}}%
\pgfusepath{clip}%
\pgfsetbuttcap%
\pgfsetroundjoin%
\pgfsetlinewidth{0.305613pt}%
\definecolor{currentstroke}{rgb}{0.267004,0.004874,0.329415}%
\pgfsetstrokecolor{currentstroke}%
\pgfsetdash{}{0pt}%
\pgfpathmoveto{\pgfqpoint{8.578381in}{2.526739in}}%
\pgfpathlineto{\pgfqpoint{8.531360in}{2.524329in}}%
\pgfusepath{stroke}%
\end{pgfscope}%
\begin{pgfscope}%
\pgfpathrectangle{\pgfqpoint{6.720588in}{1.750000in}}{\pgfqpoint{2.279412in}{2.004545in}}%
\pgfusepath{clip}%
\pgfsetbuttcap%
\pgfsetroundjoin%
\pgfsetlinewidth{0.329241pt}%
\definecolor{currentstroke}{rgb}{0.272594,0.025563,0.353093}%
\pgfsetstrokecolor{currentstroke}%
\pgfsetdash{}{0pt}%
\pgfpathmoveto{\pgfqpoint{8.531360in}{2.524329in}}%
\pgfpathlineto{\pgfqpoint{8.481212in}{2.524722in}}%
\pgfusepath{stroke}%
\end{pgfscope}%
\begin{pgfscope}%
\pgfpathrectangle{\pgfqpoint{6.720588in}{1.750000in}}{\pgfqpoint{2.279412in}{2.004545in}}%
\pgfusepath{clip}%
\pgfsetbuttcap%
\pgfsetroundjoin%
\pgfsetlinewidth{0.328978pt}%
\definecolor{currentstroke}{rgb}{0.272594,0.025563,0.353093}%
\pgfsetstrokecolor{currentstroke}%
\pgfsetdash{}{0pt}%
\pgfpathmoveto{\pgfqpoint{8.481212in}{2.524722in}}%
\pgfpathlineto{\pgfqpoint{8.431071in}{2.525293in}}%
\pgfusepath{stroke}%
\end{pgfscope}%
\begin{pgfscope}%
\pgfpathrectangle{\pgfqpoint{6.720588in}{1.750000in}}{\pgfqpoint{2.279412in}{2.004545in}}%
\pgfusepath{clip}%
\pgfsetbuttcap%
\pgfsetroundjoin%
\pgfsetlinewidth{0.339910pt}%
\definecolor{currentstroke}{rgb}{0.273809,0.031497,0.358853}%
\pgfsetstrokecolor{currentstroke}%
\pgfsetdash{}{0pt}%
\pgfpathmoveto{\pgfqpoint{8.431071in}{2.525293in}}%
\pgfpathlineto{\pgfqpoint{8.380929in}{2.525844in}}%
\pgfusepath{stroke}%
\end{pgfscope}%
\begin{pgfscope}%
\pgfpathrectangle{\pgfqpoint{6.720588in}{1.750000in}}{\pgfqpoint{2.279412in}{2.004545in}}%
\pgfusepath{clip}%
\pgfsetbuttcap%
\pgfsetroundjoin%
\pgfsetlinewidth{0.361629pt}%
\definecolor{currentstroke}{rgb}{0.277018,0.050344,0.375715}%
\pgfsetstrokecolor{currentstroke}%
\pgfsetdash{}{0pt}%
\pgfpathmoveto{\pgfqpoint{8.380929in}{2.525844in}}%
\pgfpathlineto{\pgfqpoint{8.330785in}{2.526440in}}%
\pgfusepath{stroke}%
\end{pgfscope}%
\begin{pgfscope}%
\pgfpathrectangle{\pgfqpoint{6.720588in}{1.750000in}}{\pgfqpoint{2.279412in}{2.004545in}}%
\pgfusepath{clip}%
\pgfsetbuttcap%
\pgfsetroundjoin%
\pgfsetlinewidth{0.387158pt}%
\definecolor{currentstroke}{rgb}{0.280267,0.073417,0.397163}%
\pgfsetstrokecolor{currentstroke}%
\pgfsetdash{}{0pt}%
\pgfpathmoveto{\pgfqpoint{8.330785in}{2.526440in}}%
\pgfpathlineto{\pgfqpoint{8.280651in}{2.527596in}}%
\pgfusepath{stroke}%
\end{pgfscope}%
\begin{pgfscope}%
\pgfpathrectangle{\pgfqpoint{6.720588in}{1.750000in}}{\pgfqpoint{2.279412in}{2.004545in}}%
\pgfusepath{clip}%
\pgfsetbuttcap%
\pgfsetroundjoin%
\pgfsetlinewidth{0.424953pt}%
\definecolor{currentstroke}{rgb}{0.282910,0.105393,0.426902}%
\pgfsetstrokecolor{currentstroke}%
\pgfsetdash{}{0pt}%
\pgfpathmoveto{\pgfqpoint{8.280651in}{2.527596in}}%
\pgfpathlineto{\pgfqpoint{8.230525in}{2.528953in}}%
\pgfusepath{stroke}%
\end{pgfscope}%
\begin{pgfscope}%
\pgfpathrectangle{\pgfqpoint{6.720588in}{1.750000in}}{\pgfqpoint{2.279412in}{2.004545in}}%
\pgfusepath{clip}%
\pgfsetbuttcap%
\pgfsetroundjoin%
\pgfsetlinewidth{0.468377pt}%
\definecolor{currentstroke}{rgb}{0.282884,0.135920,0.453427}%
\pgfsetstrokecolor{currentstroke}%
\pgfsetdash{}{0pt}%
\pgfpathmoveto{\pgfqpoint{8.230525in}{2.528953in}}%
\pgfpathlineto{\pgfqpoint{8.180405in}{2.530476in}}%
\pgfusepath{stroke}%
\end{pgfscope}%
\begin{pgfscope}%
\pgfpathrectangle{\pgfqpoint{6.720588in}{1.750000in}}{\pgfqpoint{2.279412in}{2.004545in}}%
\pgfusepath{clip}%
\pgfsetbuttcap%
\pgfsetroundjoin%
\pgfsetlinewidth{0.532652pt}%
\definecolor{currentstroke}{rgb}{0.278012,0.180367,0.486697}%
\pgfsetstrokecolor{currentstroke}%
\pgfsetdash{}{0pt}%
\pgfpathmoveto{\pgfqpoint{8.180405in}{2.530476in}}%
\pgfpathlineto{\pgfqpoint{8.130296in}{2.532280in}}%
\pgfusepath{stroke}%
\end{pgfscope}%
\begin{pgfscope}%
\pgfpathrectangle{\pgfqpoint{6.720588in}{1.750000in}}{\pgfqpoint{2.279412in}{2.004545in}}%
\pgfusepath{clip}%
\pgfsetbuttcap%
\pgfsetroundjoin%
\pgfsetlinewidth{0.616629pt}%
\definecolor{currentstroke}{rgb}{0.263663,0.237631,0.518762}%
\pgfsetstrokecolor{currentstroke}%
\pgfsetdash{}{0pt}%
\pgfpathmoveto{\pgfqpoint{8.130296in}{2.532280in}}%
\pgfpathlineto{\pgfqpoint{8.080220in}{2.534680in}}%
\pgfusepath{stroke}%
\end{pgfscope}%
\begin{pgfscope}%
\pgfpathrectangle{\pgfqpoint{6.720588in}{1.750000in}}{\pgfqpoint{2.279412in}{2.004545in}}%
\pgfusepath{clip}%
\pgfsetbuttcap%
\pgfsetroundjoin%
\pgfsetlinewidth{0.661227pt}%
\definecolor{currentstroke}{rgb}{0.252194,0.269783,0.531579}%
\pgfsetstrokecolor{currentstroke}%
\pgfsetdash{}{0pt}%
\pgfpathmoveto{\pgfqpoint{8.080220in}{2.534680in}}%
\pgfpathlineto{\pgfqpoint{8.030201in}{2.537868in}}%
\pgfusepath{stroke}%
\end{pgfscope}%
\begin{pgfscope}%
\pgfpathrectangle{\pgfqpoint{6.720588in}{1.750000in}}{\pgfqpoint{2.279412in}{2.004545in}}%
\pgfusepath{clip}%
\pgfsetbuttcap%
\pgfsetroundjoin%
\pgfsetlinewidth{0.682459pt}%
\definecolor{currentstroke}{rgb}{0.246811,0.283237,0.535941}%
\pgfsetstrokecolor{currentstroke}%
\pgfsetdash{}{0pt}%
\pgfpathmoveto{\pgfqpoint{8.030201in}{2.537868in}}%
\pgfpathlineto{\pgfqpoint{7.980248in}{2.541773in}}%
\pgfusepath{stroke}%
\end{pgfscope}%
\begin{pgfscope}%
\pgfpathrectangle{\pgfqpoint{6.720588in}{1.750000in}}{\pgfqpoint{2.279412in}{2.004545in}}%
\pgfusepath{clip}%
\pgfsetbuttcap%
\pgfsetroundjoin%
\pgfsetlinewidth{0.733285pt}%
\definecolor{currentstroke}{rgb}{0.233603,0.313828,0.543914}%
\pgfsetstrokecolor{currentstroke}%
\pgfsetdash{}{0pt}%
\pgfpathmoveto{\pgfqpoint{7.980248in}{2.541773in}}%
\pgfpathlineto{\pgfqpoint{7.930398in}{2.546568in}}%
\pgfusepath{stroke}%
\end{pgfscope}%
\begin{pgfscope}%
\pgfpathrectangle{\pgfqpoint{6.720588in}{1.750000in}}{\pgfqpoint{2.279412in}{2.004545in}}%
\pgfusepath{clip}%
\pgfsetbuttcap%
\pgfsetroundjoin%
\pgfsetlinewidth{0.316983pt}%
\definecolor{currentstroke}{rgb}{0.269944,0.014625,0.341379}%
\pgfsetstrokecolor{currentstroke}%
\pgfsetdash{}{0pt}%
\pgfpathmoveto{\pgfqpoint{8.578381in}{2.977807in}}%
\pgfpathlineto{\pgfqpoint{8.528237in}{2.977961in}}%
\pgfusepath{stroke}%
\end{pgfscope}%
\begin{pgfscope}%
\pgfpathrectangle{\pgfqpoint{6.720588in}{1.750000in}}{\pgfqpoint{2.279412in}{2.004545in}}%
\pgfusepath{clip}%
\pgfsetbuttcap%
\pgfsetroundjoin%
\pgfsetlinewidth{0.323491pt}%
\definecolor{currentstroke}{rgb}{0.271305,0.019942,0.347269}%
\pgfsetstrokecolor{currentstroke}%
\pgfsetdash{}{0pt}%
\pgfpathmoveto{\pgfqpoint{8.528237in}{2.977961in}}%
\pgfpathlineto{\pgfqpoint{8.478092in}{2.978522in}}%
\pgfusepath{stroke}%
\end{pgfscope}%
\begin{pgfscope}%
\pgfpathrectangle{\pgfqpoint{6.720588in}{1.750000in}}{\pgfqpoint{2.279412in}{2.004545in}}%
\pgfusepath{clip}%
\pgfsetbuttcap%
\pgfsetroundjoin%
\pgfsetlinewidth{0.328514pt}%
\definecolor{currentstroke}{rgb}{0.271305,0.019942,0.347269}%
\pgfsetstrokecolor{currentstroke}%
\pgfsetdash{}{0pt}%
\pgfpathmoveto{\pgfqpoint{8.478092in}{2.978522in}}%
\pgfpathlineto{\pgfqpoint{8.427947in}{2.978272in}}%
\pgfusepath{stroke}%
\end{pgfscope}%
\begin{pgfscope}%
\pgfpathrectangle{\pgfqpoint{6.720588in}{1.750000in}}{\pgfqpoint{2.279412in}{2.004545in}}%
\pgfusepath{clip}%
\pgfsetbuttcap%
\pgfsetroundjoin%
\pgfsetlinewidth{0.342569pt}%
\definecolor{currentstroke}{rgb}{0.274952,0.037752,0.364543}%
\pgfsetstrokecolor{currentstroke}%
\pgfsetdash{}{0pt}%
\pgfpathmoveto{\pgfqpoint{8.427947in}{2.978272in}}%
\pgfpathlineto{\pgfqpoint{8.377804in}{2.977486in}}%
\pgfusepath{stroke}%
\end{pgfscope}%
\begin{pgfscope}%
\pgfpathrectangle{\pgfqpoint{6.720588in}{1.750000in}}{\pgfqpoint{2.279412in}{2.004545in}}%
\pgfusepath{clip}%
\pgfsetbuttcap%
\pgfsetroundjoin%
\pgfsetlinewidth{0.357382pt}%
\definecolor{currentstroke}{rgb}{0.277018,0.050344,0.375715}%
\pgfsetstrokecolor{currentstroke}%
\pgfsetdash{}{0pt}%
\pgfpathmoveto{\pgfqpoint{8.377804in}{2.977486in}}%
\pgfpathlineto{\pgfqpoint{8.327659in}{2.976789in}}%
\pgfusepath{stroke}%
\end{pgfscope}%
\begin{pgfscope}%
\pgfpathrectangle{\pgfqpoint{6.720588in}{1.750000in}}{\pgfqpoint{2.279412in}{2.004545in}}%
\pgfusepath{clip}%
\pgfsetbuttcap%
\pgfsetroundjoin%
\pgfsetlinewidth{0.386344pt}%
\definecolor{currentstroke}{rgb}{0.280267,0.073417,0.397163}%
\pgfsetstrokecolor{currentstroke}%
\pgfsetdash{}{0pt}%
\pgfpathmoveto{\pgfqpoint{8.327659in}{2.976789in}}%
\pgfpathlineto{\pgfqpoint{8.277514in}{2.976168in}}%
\pgfusepath{stroke}%
\end{pgfscope}%
\begin{pgfscope}%
\pgfpathrectangle{\pgfqpoint{6.720588in}{1.750000in}}{\pgfqpoint{2.279412in}{2.004545in}}%
\pgfusepath{clip}%
\pgfsetbuttcap%
\pgfsetroundjoin%
\pgfsetlinewidth{0.426921pt}%
\definecolor{currentstroke}{rgb}{0.282910,0.105393,0.426902}%
\pgfsetstrokecolor{currentstroke}%
\pgfsetdash{}{0pt}%
\pgfpathmoveto{\pgfqpoint{8.277514in}{2.976168in}}%
\pgfpathlineto{\pgfqpoint{8.227376in}{2.975258in}}%
\pgfusepath{stroke}%
\end{pgfscope}%
\begin{pgfscope}%
\pgfpathrectangle{\pgfqpoint{6.720588in}{1.750000in}}{\pgfqpoint{2.279412in}{2.004545in}}%
\pgfusepath{clip}%
\pgfsetbuttcap%
\pgfsetroundjoin%
\pgfsetlinewidth{0.453322pt}%
\definecolor{currentstroke}{rgb}{0.283187,0.125848,0.444960}%
\pgfsetstrokecolor{currentstroke}%
\pgfsetdash{}{0pt}%
\pgfpathmoveto{\pgfqpoint{8.227376in}{2.975258in}}%
\pgfpathlineto{\pgfqpoint{8.177249in}{2.973879in}}%
\pgfusepath{stroke}%
\end{pgfscope}%
\begin{pgfscope}%
\pgfpathrectangle{\pgfqpoint{6.720588in}{1.750000in}}{\pgfqpoint{2.279412in}{2.004545in}}%
\pgfusepath{clip}%
\pgfsetbuttcap%
\pgfsetroundjoin%
\pgfsetlinewidth{0.505999pt}%
\definecolor{currentstroke}{rgb}{0.280868,0.160771,0.472899}%
\pgfsetstrokecolor{currentstroke}%
\pgfsetdash{}{0pt}%
\pgfpathmoveto{\pgfqpoint{8.177249in}{2.973879in}}%
\pgfpathlineto{\pgfqpoint{8.127140in}{2.972089in}}%
\pgfusepath{stroke}%
\end{pgfscope}%
\begin{pgfscope}%
\pgfpathrectangle{\pgfqpoint{6.720588in}{1.750000in}}{\pgfqpoint{2.279412in}{2.004545in}}%
\pgfusepath{clip}%
\pgfsetbuttcap%
\pgfsetroundjoin%
\pgfsetlinewidth{0.553892pt}%
\definecolor{currentstroke}{rgb}{0.275191,0.194905,0.496005}%
\pgfsetstrokecolor{currentstroke}%
\pgfsetdash{}{0pt}%
\pgfpathmoveto{\pgfqpoint{8.127140in}{2.972089in}}%
\pgfpathlineto{\pgfqpoint{8.077063in}{2.969713in}}%
\pgfusepath{stroke}%
\end{pgfscope}%
\begin{pgfscope}%
\pgfpathrectangle{\pgfqpoint{6.720588in}{1.750000in}}{\pgfqpoint{2.279412in}{2.004545in}}%
\pgfusepath{clip}%
\pgfsetbuttcap%
\pgfsetroundjoin%
\pgfsetlinewidth{0.581712pt}%
\definecolor{currentstroke}{rgb}{0.270595,0.214069,0.507052}%
\pgfsetstrokecolor{currentstroke}%
\pgfsetdash{}{0pt}%
\pgfpathmoveto{\pgfqpoint{8.077063in}{2.969713in}}%
\pgfpathlineto{\pgfqpoint{8.027033in}{2.966670in}}%
\pgfusepath{stroke}%
\end{pgfscope}%
\begin{pgfscope}%
\pgfpathrectangle{\pgfqpoint{6.720588in}{1.750000in}}{\pgfqpoint{2.279412in}{2.004545in}}%
\pgfusepath{clip}%
\pgfsetbuttcap%
\pgfsetroundjoin%
\pgfsetlinewidth{0.604344pt}%
\definecolor{currentstroke}{rgb}{0.265145,0.232956,0.516599}%
\pgfsetstrokecolor{currentstroke}%
\pgfsetdash{}{0pt}%
\pgfpathmoveto{\pgfqpoint{8.027033in}{2.966670in}}%
\pgfpathlineto{\pgfqpoint{7.977073in}{2.962849in}}%
\pgfusepath{stroke}%
\end{pgfscope}%
\begin{pgfscope}%
\pgfpathrectangle{\pgfqpoint{6.720588in}{1.750000in}}{\pgfqpoint{2.279412in}{2.004545in}}%
\pgfusepath{clip}%
\pgfsetbuttcap%
\pgfsetroundjoin%
\pgfsetlinewidth{0.644747pt}%
\definecolor{currentstroke}{rgb}{0.255645,0.260703,0.528312}%
\pgfsetstrokecolor{currentstroke}%
\pgfsetdash{}{0pt}%
\pgfpathmoveto{\pgfqpoint{7.977073in}{2.962849in}}%
\pgfpathlineto{\pgfqpoint{7.927226in}{2.958026in}}%
\pgfusepath{stroke}%
\end{pgfscope}%
\begin{pgfscope}%
\pgfpathrectangle{\pgfqpoint{6.720588in}{1.750000in}}{\pgfqpoint{2.279412in}{2.004545in}}%
\pgfusepath{clip}%
\pgfsetbuttcap%
\pgfsetroundjoin%
\pgfsetlinewidth{0.665906pt}%
\definecolor{currentstroke}{rgb}{0.250425,0.274290,0.533103}%
\pgfsetstrokecolor{currentstroke}%
\pgfsetdash{}{0pt}%
\pgfpathmoveto{\pgfqpoint{7.927226in}{2.958026in}}%
\pgfpathlineto{\pgfqpoint{7.877572in}{2.951888in}}%
\pgfusepath{stroke}%
\end{pgfscope}%
\begin{pgfscope}%
\pgfpathrectangle{\pgfqpoint{6.720588in}{1.750000in}}{\pgfqpoint{2.279412in}{2.004545in}}%
\pgfusepath{clip}%
\pgfsetbuttcap%
\pgfsetroundjoin%
\pgfsetlinewidth{0.689042pt}%
\definecolor{currentstroke}{rgb}{0.244972,0.287675,0.537260}%
\pgfsetstrokecolor{currentstroke}%
\pgfsetdash{}{0pt}%
\pgfpathmoveto{\pgfqpoint{7.877572in}{2.951888in}}%
\pgfpathlineto{\pgfqpoint{7.828291in}{2.943832in}}%
\pgfusepath{stroke}%
\end{pgfscope}%
\begin{pgfscope}%
\pgfpathrectangle{\pgfqpoint{6.720588in}{1.750000in}}{\pgfqpoint{2.279412in}{2.004545in}}%
\pgfusepath{clip}%
\pgfsetbuttcap%
\pgfsetroundjoin%
\pgfsetlinewidth{0.642579pt}%
\definecolor{currentstroke}{rgb}{0.257322,0.256130,0.526563}%
\pgfsetstrokecolor{currentstroke}%
\pgfsetdash{}{0pt}%
\pgfpathmoveto{\pgfqpoint{7.828291in}{2.943832in}}%
\pgfpathlineto{\pgfqpoint{7.779592in}{2.933415in}}%
\pgfusepath{stroke}%
\end{pgfscope}%
\begin{pgfscope}%
\pgfpathrectangle{\pgfqpoint{6.720588in}{1.750000in}}{\pgfqpoint{2.279412in}{2.004545in}}%
\pgfusepath{clip}%
\pgfsetbuttcap%
\pgfsetroundjoin%
\pgfsetlinewidth{0.647166pt}%
\definecolor{currentstroke}{rgb}{0.255645,0.260703,0.528312}%
\pgfsetstrokecolor{currentstroke}%
\pgfsetdash{}{0pt}%
\pgfpathmoveto{\pgfqpoint{7.779592in}{2.933415in}}%
\pgfpathlineto{\pgfqpoint{7.731751in}{2.920322in}}%
\pgfusepath{stroke}%
\end{pgfscope}%
\begin{pgfscope}%
\pgfpathrectangle{\pgfqpoint{6.720588in}{1.750000in}}{\pgfqpoint{2.279412in}{2.004545in}}%
\pgfusepath{clip}%
\pgfsetbuttcap%
\pgfsetroundjoin%
\pgfsetlinewidth{0.319448pt}%
\definecolor{currentstroke}{rgb}{0.269944,0.014625,0.341379}%
\pgfsetstrokecolor{currentstroke}%
\pgfsetdash{}{0pt}%
\pgfpathmoveto{\pgfqpoint{8.578381in}{3.022913in}}%
\pgfpathlineto{\pgfqpoint{8.528259in}{3.022610in}}%
\pgfusepath{stroke}%
\end{pgfscope}%
\begin{pgfscope}%
\pgfpathrectangle{\pgfqpoint{6.720588in}{1.750000in}}{\pgfqpoint{2.279412in}{2.004545in}}%
\pgfusepath{clip}%
\pgfsetbuttcap%
\pgfsetroundjoin%
\pgfsetlinewidth{0.319735pt}%
\definecolor{currentstroke}{rgb}{0.269944,0.014625,0.341379}%
\pgfsetstrokecolor{currentstroke}%
\pgfsetdash{}{0pt}%
\pgfpathmoveto{\pgfqpoint{8.528259in}{3.022610in}}%
\pgfpathlineto{\pgfqpoint{8.478147in}{3.022519in}}%
\pgfusepath{stroke}%
\end{pgfscope}%
\begin{pgfscope}%
\pgfpathrectangle{\pgfqpoint{6.720588in}{1.750000in}}{\pgfqpoint{2.279412in}{2.004545in}}%
\pgfusepath{clip}%
\pgfsetbuttcap%
\pgfsetroundjoin%
\pgfsetlinewidth{0.325995pt}%
\definecolor{currentstroke}{rgb}{0.271305,0.019942,0.347269}%
\pgfsetstrokecolor{currentstroke}%
\pgfsetdash{}{0pt}%
\pgfpathmoveto{\pgfqpoint{8.478147in}{3.022519in}}%
\pgfpathlineto{\pgfqpoint{8.428029in}{3.022655in}}%
\pgfusepath{stroke}%
\end{pgfscope}%
\begin{pgfscope}%
\pgfpathrectangle{\pgfqpoint{6.720588in}{1.750000in}}{\pgfqpoint{2.279412in}{2.004545in}}%
\pgfusepath{clip}%
\pgfsetbuttcap%
\pgfsetroundjoin%
\pgfsetlinewidth{0.339974pt}%
\definecolor{currentstroke}{rgb}{0.273809,0.031497,0.358853}%
\pgfsetstrokecolor{currentstroke}%
\pgfsetdash{}{0pt}%
\pgfpathmoveto{\pgfqpoint{8.428029in}{3.022655in}}%
\pgfpathlineto{\pgfqpoint{8.377899in}{3.021682in}}%
\pgfusepath{stroke}%
\end{pgfscope}%
\begin{pgfscope}%
\pgfpathrectangle{\pgfqpoint{6.720588in}{1.750000in}}{\pgfqpoint{2.279412in}{2.004545in}}%
\pgfusepath{clip}%
\pgfsetbuttcap%
\pgfsetroundjoin%
\pgfsetlinewidth{0.356077pt}%
\definecolor{currentstroke}{rgb}{0.277018,0.050344,0.375715}%
\pgfsetstrokecolor{currentstroke}%
\pgfsetdash{}{0pt}%
\pgfpathmoveto{\pgfqpoint{8.377899in}{3.021682in}}%
\pgfpathlineto{\pgfqpoint{8.327776in}{3.020253in}}%
\pgfusepath{stroke}%
\end{pgfscope}%
\begin{pgfscope}%
\pgfpathrectangle{\pgfqpoint{6.720588in}{1.750000in}}{\pgfqpoint{2.279412in}{2.004545in}}%
\pgfusepath{clip}%
\pgfsetbuttcap%
\pgfsetroundjoin%
\pgfsetlinewidth{0.382162pt}%
\definecolor{currentstroke}{rgb}{0.279566,0.067836,0.391917}%
\pgfsetstrokecolor{currentstroke}%
\pgfsetdash{}{0pt}%
\pgfpathmoveto{\pgfqpoint{8.327776in}{3.020253in}}%
\pgfpathlineto{\pgfqpoint{8.277653in}{3.018833in}}%
\pgfusepath{stroke}%
\end{pgfscope}%
\begin{pgfscope}%
\pgfpathrectangle{\pgfqpoint{6.720588in}{1.750000in}}{\pgfqpoint{2.279412in}{2.004545in}}%
\pgfusepath{clip}%
\pgfsetbuttcap%
\pgfsetroundjoin%
\pgfsetlinewidth{0.410948pt}%
\definecolor{currentstroke}{rgb}{0.282327,0.094955,0.417331}%
\pgfsetstrokecolor{currentstroke}%
\pgfsetdash{}{0pt}%
\pgfpathmoveto{\pgfqpoint{8.277653in}{3.018833in}}%
\pgfpathlineto{\pgfqpoint{8.227549in}{3.016965in}}%
\pgfusepath{stroke}%
\end{pgfscope}%
\begin{pgfscope}%
\pgfpathrectangle{\pgfqpoint{6.720588in}{1.750000in}}{\pgfqpoint{2.279412in}{2.004545in}}%
\pgfusepath{clip}%
\pgfsetbuttcap%
\pgfsetroundjoin%
\pgfsetlinewidth{0.436133pt}%
\definecolor{currentstroke}{rgb}{0.283091,0.110553,0.431554}%
\pgfsetstrokecolor{currentstroke}%
\pgfsetdash{}{0pt}%
\pgfpathmoveto{\pgfqpoint{8.227549in}{3.016965in}}%
\pgfpathlineto{\pgfqpoint{8.177465in}{3.014700in}}%
\pgfusepath{stroke}%
\end{pgfscope}%
\begin{pgfscope}%
\pgfpathrectangle{\pgfqpoint{6.720588in}{1.750000in}}{\pgfqpoint{2.279412in}{2.004545in}}%
\pgfusepath{clip}%
\pgfsetbuttcap%
\pgfsetroundjoin%
\pgfsetlinewidth{0.470256pt}%
\definecolor{currentstroke}{rgb}{0.282884,0.135920,0.453427}%
\pgfsetstrokecolor{currentstroke}%
\pgfsetdash{}{0pt}%
\pgfpathmoveto{\pgfqpoint{8.177465in}{3.014700in}}%
\pgfpathlineto{\pgfqpoint{8.127402in}{3.012092in}}%
\pgfusepath{stroke}%
\end{pgfscope}%
\begin{pgfscope}%
\pgfpathrectangle{\pgfqpoint{6.720588in}{1.750000in}}{\pgfqpoint{2.279412in}{2.004545in}}%
\pgfusepath{clip}%
\pgfsetbuttcap%
\pgfsetroundjoin%
\pgfsetlinewidth{0.514603pt}%
\definecolor{currentstroke}{rgb}{0.279574,0.170599,0.479997}%
\pgfsetstrokecolor{currentstroke}%
\pgfsetdash{}{0pt}%
\pgfpathmoveto{\pgfqpoint{8.127402in}{3.012092in}}%
\pgfpathlineto{\pgfqpoint{8.077384in}{3.008911in}}%
\pgfusepath{stroke}%
\end{pgfscope}%
\begin{pgfscope}%
\pgfpathrectangle{\pgfqpoint{6.720588in}{1.750000in}}{\pgfqpoint{2.279412in}{2.004545in}}%
\pgfusepath{clip}%
\pgfsetbuttcap%
\pgfsetroundjoin%
\pgfsetlinewidth{0.542206pt}%
\definecolor{currentstroke}{rgb}{0.276194,0.190074,0.493001}%
\pgfsetstrokecolor{currentstroke}%
\pgfsetdash{}{0pt}%
\pgfpathmoveto{\pgfqpoint{8.077384in}{3.008911in}}%
\pgfpathlineto{\pgfqpoint{8.027436in}{3.004967in}}%
\pgfusepath{stroke}%
\end{pgfscope}%
\begin{pgfscope}%
\pgfpathrectangle{\pgfqpoint{6.720588in}{1.750000in}}{\pgfqpoint{2.279412in}{2.004545in}}%
\pgfusepath{clip}%
\pgfsetbuttcap%
\pgfsetroundjoin%
\pgfsetlinewidth{0.310955pt}%
\definecolor{currentstroke}{rgb}{0.268510,0.009605,0.335427}%
\pgfsetstrokecolor{currentstroke}%
\pgfsetdash{}{0pt}%
\pgfpathmoveto{\pgfqpoint{8.628398in}{3.070945in}}%
\pgfpathlineto{\pgfqpoint{8.578381in}{3.068020in}}%
\pgfusepath{stroke}%
\end{pgfscope}%
\begin{pgfscope}%
\pgfpathrectangle{\pgfqpoint{6.720588in}{1.750000in}}{\pgfqpoint{2.279412in}{2.004545in}}%
\pgfusepath{clip}%
\pgfsetbuttcap%
\pgfsetroundjoin%
\pgfsetlinewidth{0.316483pt}%
\definecolor{currentstroke}{rgb}{0.269944,0.014625,0.341379}%
\pgfsetstrokecolor{currentstroke}%
\pgfsetdash{}{0pt}%
\pgfpathmoveto{\pgfqpoint{8.578381in}{3.068020in}}%
\pgfpathlineto{\pgfqpoint{8.528282in}{3.066906in}}%
\pgfusepath{stroke}%
\end{pgfscope}%
\begin{pgfscope}%
\pgfpathrectangle{\pgfqpoint{6.720588in}{1.750000in}}{\pgfqpoint{2.279412in}{2.004545in}}%
\pgfusepath{clip}%
\pgfsetbuttcap%
\pgfsetroundjoin%
\pgfsetlinewidth{0.318268pt}%
\definecolor{currentstroke}{rgb}{0.269944,0.014625,0.341379}%
\pgfsetstrokecolor{currentstroke}%
\pgfsetdash{}{0pt}%
\pgfpathmoveto{\pgfqpoint{8.528282in}{3.066906in}}%
\pgfpathlineto{\pgfqpoint{8.478189in}{3.065649in}}%
\pgfusepath{stroke}%
\end{pgfscope}%
\begin{pgfscope}%
\pgfpathrectangle{\pgfqpoint{6.720588in}{1.750000in}}{\pgfqpoint{2.279412in}{2.004545in}}%
\pgfusepath{clip}%
\pgfsetbuttcap%
\pgfsetroundjoin%
\pgfsetlinewidth{0.324323pt}%
\definecolor{currentstroke}{rgb}{0.271305,0.019942,0.347269}%
\pgfsetstrokecolor{currentstroke}%
\pgfsetdash{}{0pt}%
\pgfpathmoveto{\pgfqpoint{8.478189in}{3.065649in}}%
\pgfpathlineto{\pgfqpoint{8.428086in}{3.064374in}}%
\pgfusepath{stroke}%
\end{pgfscope}%
\begin{pgfscope}%
\pgfpathrectangle{\pgfqpoint{6.720588in}{1.750000in}}{\pgfqpoint{2.279412in}{2.004545in}}%
\pgfusepath{clip}%
\pgfsetbuttcap%
\pgfsetroundjoin%
\pgfsetlinewidth{0.335629pt}%
\definecolor{currentstroke}{rgb}{0.273809,0.031497,0.358853}%
\pgfsetstrokecolor{currentstroke}%
\pgfsetdash{}{0pt}%
\pgfpathmoveto{\pgfqpoint{8.428086in}{3.064374in}}%
\pgfpathlineto{\pgfqpoint{8.377952in}{3.063562in}}%
\pgfusepath{stroke}%
\end{pgfscope}%
\begin{pgfscope}%
\pgfpathrectangle{\pgfqpoint{6.720588in}{1.750000in}}{\pgfqpoint{2.279412in}{2.004545in}}%
\pgfusepath{clip}%
\pgfsetbuttcap%
\pgfsetroundjoin%
\pgfsetlinewidth{0.357610pt}%
\definecolor{currentstroke}{rgb}{0.277018,0.050344,0.375715}%
\pgfsetstrokecolor{currentstroke}%
\pgfsetdash{}{0pt}%
\pgfpathmoveto{\pgfqpoint{8.377952in}{3.063562in}}%
\pgfpathlineto{\pgfqpoint{8.327823in}{3.062358in}}%
\pgfusepath{stroke}%
\end{pgfscope}%
\begin{pgfscope}%
\pgfpathrectangle{\pgfqpoint{6.720588in}{1.750000in}}{\pgfqpoint{2.279412in}{2.004545in}}%
\pgfusepath{clip}%
\pgfsetbuttcap%
\pgfsetroundjoin%
\pgfsetlinewidth{0.369196pt}%
\definecolor{currentstroke}{rgb}{0.277941,0.056324,0.381191}%
\pgfsetstrokecolor{currentstroke}%
\pgfsetdash{}{0pt}%
\pgfpathmoveto{\pgfqpoint{8.327823in}{3.062358in}}%
\pgfpathlineto{\pgfqpoint{8.277702in}{3.060986in}}%
\pgfusepath{stroke}%
\end{pgfscope}%
\begin{pgfscope}%
\pgfpathrectangle{\pgfqpoint{6.720588in}{1.750000in}}{\pgfqpoint{2.279412in}{2.004545in}}%
\pgfusepath{clip}%
\pgfsetbuttcap%
\pgfsetroundjoin%
\pgfsetlinewidth{0.400435pt}%
\definecolor{currentstroke}{rgb}{0.281446,0.084320,0.407414}%
\pgfsetstrokecolor{currentstroke}%
\pgfsetdash{}{0pt}%
\pgfpathmoveto{\pgfqpoint{8.277702in}{3.060986in}}%
\pgfpathlineto{\pgfqpoint{8.227599in}{3.059085in}}%
\pgfusepath{stroke}%
\end{pgfscope}%
\begin{pgfscope}%
\pgfpathrectangle{\pgfqpoint{6.720588in}{1.750000in}}{\pgfqpoint{2.279412in}{2.004545in}}%
\pgfusepath{clip}%
\pgfsetbuttcap%
\pgfsetroundjoin%
\pgfsetlinewidth{0.417259pt}%
\definecolor{currentstroke}{rgb}{0.282327,0.094955,0.417331}%
\pgfsetstrokecolor{currentstroke}%
\pgfsetdash{}{0pt}%
\pgfpathmoveto{\pgfqpoint{8.227599in}{3.059085in}}%
\pgfpathlineto{\pgfqpoint{8.177510in}{3.056922in}}%
\pgfusepath{stroke}%
\end{pgfscope}%
\begin{pgfscope}%
\pgfpathrectangle{\pgfqpoint{6.720588in}{1.750000in}}{\pgfqpoint{2.279412in}{2.004545in}}%
\pgfusepath{clip}%
\pgfsetbuttcap%
\pgfsetroundjoin%
\pgfsetlinewidth{0.455859pt}%
\definecolor{currentstroke}{rgb}{0.283187,0.125848,0.444960}%
\pgfsetstrokecolor{currentstroke}%
\pgfsetdash{}{0pt}%
\pgfpathmoveto{\pgfqpoint{8.177510in}{3.056922in}}%
\pgfpathlineto{\pgfqpoint{8.127456in}{3.054197in}}%
\pgfusepath{stroke}%
\end{pgfscope}%
\begin{pgfscope}%
\pgfpathrectangle{\pgfqpoint{6.720588in}{1.750000in}}{\pgfqpoint{2.279412in}{2.004545in}}%
\pgfusepath{clip}%
\pgfsetbuttcap%
\pgfsetroundjoin%
\pgfsetlinewidth{0.476041pt}%
\definecolor{currentstroke}{rgb}{0.282623,0.140926,0.457517}%
\pgfsetstrokecolor{currentstroke}%
\pgfsetdash{}{0pt}%
\pgfpathmoveto{\pgfqpoint{8.127456in}{3.054197in}}%
\pgfpathlineto{\pgfqpoint{8.077469in}{3.050679in}}%
\pgfusepath{stroke}%
\end{pgfscope}%
\begin{pgfscope}%
\pgfpathrectangle{\pgfqpoint{6.720588in}{1.750000in}}{\pgfqpoint{2.279412in}{2.004545in}}%
\pgfusepath{clip}%
\pgfsetbuttcap%
\pgfsetroundjoin%
\pgfsetlinewidth{0.505525pt}%
\definecolor{currentstroke}{rgb}{0.280868,0.160771,0.472899}%
\pgfsetstrokecolor{currentstroke}%
\pgfsetdash{}{0pt}%
\pgfpathmoveto{\pgfqpoint{8.077469in}{3.050679in}}%
\pgfpathlineto{\pgfqpoint{8.027599in}{3.046057in}}%
\pgfusepath{stroke}%
\end{pgfscope}%
\begin{pgfscope}%
\pgfpathrectangle{\pgfqpoint{6.720588in}{1.750000in}}{\pgfqpoint{2.279412in}{2.004545in}}%
\pgfusepath{clip}%
\pgfsetbuttcap%
\pgfsetroundjoin%
\pgfsetlinewidth{0.521339pt}%
\definecolor{currentstroke}{rgb}{0.278826,0.175490,0.483397}%
\pgfsetstrokecolor{currentstroke}%
\pgfsetdash{}{0pt}%
\pgfpathmoveto{\pgfqpoint{8.027599in}{3.046057in}}%
\pgfpathlineto{\pgfqpoint{7.977881in}{3.040307in}}%
\pgfusepath{stroke}%
\end{pgfscope}%
\begin{pgfscope}%
\pgfpathrectangle{\pgfqpoint{6.720588in}{1.750000in}}{\pgfqpoint{2.279412in}{2.004545in}}%
\pgfusepath{clip}%
\pgfsetbuttcap%
\pgfsetroundjoin%
\pgfsetlinewidth{0.552460pt}%
\definecolor{currentstroke}{rgb}{0.275191,0.194905,0.496005}%
\pgfsetstrokecolor{currentstroke}%
\pgfsetdash{}{0pt}%
\pgfpathmoveto{\pgfqpoint{7.977881in}{3.040307in}}%
\pgfpathlineto{\pgfqpoint{7.928502in}{3.032704in}}%
\pgfusepath{stroke}%
\end{pgfscope}%
\begin{pgfscope}%
\pgfpathrectangle{\pgfqpoint{6.720588in}{1.750000in}}{\pgfqpoint{2.279412in}{2.004545in}}%
\pgfusepath{clip}%
\pgfsetbuttcap%
\pgfsetroundjoin%
\pgfsetlinewidth{0.499045pt}%
\definecolor{currentstroke}{rgb}{0.281412,0.155834,0.469201}%
\pgfsetstrokecolor{currentstroke}%
\pgfsetdash{}{0pt}%
\pgfpathmoveto{\pgfqpoint{7.928502in}{3.032704in}}%
\pgfpathlineto{\pgfqpoint{7.879706in}{3.022646in}}%
\pgfusepath{stroke}%
\end{pgfscope}%
\begin{pgfscope}%
\pgfpathrectangle{\pgfqpoint{6.720588in}{1.750000in}}{\pgfqpoint{2.279412in}{2.004545in}}%
\pgfusepath{clip}%
\pgfsetbuttcap%
\pgfsetroundjoin%
\pgfsetlinewidth{0.519017pt}%
\definecolor{currentstroke}{rgb}{0.279574,0.170599,0.479997}%
\pgfsetstrokecolor{currentstroke}%
\pgfsetdash{}{0pt}%
\pgfpathmoveto{\pgfqpoint{7.879706in}{3.022646in}}%
\pgfpathlineto{\pgfqpoint{7.831568in}{3.010370in}}%
\pgfusepath{stroke}%
\end{pgfscope}%
\begin{pgfscope}%
\pgfpathrectangle{\pgfqpoint{6.720588in}{1.750000in}}{\pgfqpoint{2.279412in}{2.004545in}}%
\pgfusepath{clip}%
\pgfsetbuttcap%
\pgfsetroundjoin%
\pgfsetlinewidth{0.552223pt}%
\definecolor{currentstroke}{rgb}{0.275191,0.194905,0.496005}%
\pgfsetstrokecolor{currentstroke}%
\pgfsetdash{}{0pt}%
\pgfpathmoveto{\pgfqpoint{7.831568in}{3.010370in}}%
\pgfpathlineto{\pgfqpoint{7.784248in}{2.995887in}}%
\pgfusepath{stroke}%
\end{pgfscope}%
\begin{pgfscope}%
\pgfpathrectangle{\pgfqpoint{6.720588in}{1.750000in}}{\pgfqpoint{2.279412in}{2.004545in}}%
\pgfusepath{clip}%
\pgfsetbuttcap%
\pgfsetroundjoin%
\pgfsetlinewidth{0.590620pt}%
\definecolor{currentstroke}{rgb}{0.267968,0.223549,0.512008}%
\pgfsetstrokecolor{currentstroke}%
\pgfsetdash{}{0pt}%
\pgfpathmoveto{\pgfqpoint{7.784248in}{2.995887in}}%
\pgfpathlineto{\pgfqpoint{7.737993in}{2.979057in}}%
\pgfusepath{stroke}%
\end{pgfscope}%
\begin{pgfscope}%
\pgfpathrectangle{\pgfqpoint{6.720588in}{1.750000in}}{\pgfqpoint{2.279412in}{2.004545in}}%
\pgfusepath{clip}%
\pgfsetbuttcap%
\pgfsetroundjoin%
\pgfsetlinewidth{0.571470pt}%
\definecolor{currentstroke}{rgb}{0.271828,0.209303,0.504434}%
\pgfsetstrokecolor{currentstroke}%
\pgfsetdash{}{0pt}%
\pgfpathmoveto{\pgfqpoint{7.737993in}{2.979057in}}%
\pgfpathlineto{\pgfqpoint{7.693863in}{2.958487in}}%
\pgfusepath{stroke}%
\end{pgfscope}%
\begin{pgfscope}%
\pgfpathrectangle{\pgfqpoint{6.720588in}{1.750000in}}{\pgfqpoint{2.279412in}{2.004545in}}%
\pgfusepath{clip}%
\pgfsetbuttcap%
\pgfsetroundjoin%
\pgfsetlinewidth{0.571503pt}%
\definecolor{currentstroke}{rgb}{0.271828,0.209303,0.504434}%
\pgfsetstrokecolor{currentstroke}%
\pgfsetdash{}{0pt}%
\pgfpathmoveto{\pgfqpoint{7.693863in}{2.958487in}}%
\pgfpathlineto{\pgfqpoint{7.652383in}{2.934006in}}%
\pgfusepath{stroke}%
\end{pgfscope}%
\begin{pgfscope}%
\pgfpathrectangle{\pgfqpoint{6.720588in}{1.750000in}}{\pgfqpoint{2.279412in}{2.004545in}}%
\pgfusepath{clip}%
\pgfsetbuttcap%
\pgfsetroundjoin%
\pgfsetlinewidth{0.592946pt}%
\definecolor{currentstroke}{rgb}{0.267968,0.223549,0.512008}%
\pgfsetstrokecolor{currentstroke}%
\pgfsetdash{}{0pt}%
\pgfpathmoveto{\pgfqpoint{7.652383in}{2.934006in}}%
\pgfpathlineto{\pgfqpoint{7.612978in}{2.906957in}}%
\pgfusepath{stroke}%
\end{pgfscope}%
\begin{pgfscope}%
\pgfpathrectangle{\pgfqpoint{6.720588in}{1.750000in}}{\pgfqpoint{2.279412in}{2.004545in}}%
\pgfusepath{clip}%
\pgfsetbuttcap%
\pgfsetroundjoin%
\pgfsetlinewidth{0.621067pt}%
\definecolor{currentstroke}{rgb}{0.262138,0.242286,0.520837}%
\pgfsetstrokecolor{currentstroke}%
\pgfsetdash{}{0pt}%
\pgfpathmoveto{\pgfqpoint{7.612978in}{2.906957in}}%
\pgfpathlineto{\pgfqpoint{7.573824in}{2.879668in}}%
\pgfusepath{stroke}%
\end{pgfscope}%
\begin{pgfscope}%
\pgfpathrectangle{\pgfqpoint{6.720588in}{1.750000in}}{\pgfqpoint{2.279412in}{2.004545in}}%
\pgfusepath{clip}%
\pgfsetbuttcap%
\pgfsetroundjoin%
\pgfsetlinewidth{0.319043pt}%
\definecolor{currentstroke}{rgb}{0.269944,0.014625,0.341379}%
\pgfsetstrokecolor{currentstroke}%
\pgfsetdash{}{0pt}%
\pgfpathmoveto{\pgfqpoint{8.578381in}{3.293554in}}%
\pgfpathlineto{\pgfqpoint{8.578381in}{3.293554in}}%
\pgfusepath{stroke}%
\end{pgfscope}%
\begin{pgfscope}%
\pgfpathrectangle{\pgfqpoint{6.720588in}{1.750000in}}{\pgfqpoint{2.279412in}{2.004545in}}%
\pgfusepath{clip}%
\pgfsetbuttcap%
\pgfsetroundjoin%
\pgfsetlinewidth{0.319043pt}%
\definecolor{currentstroke}{rgb}{0.269944,0.014625,0.341379}%
\pgfsetstrokecolor{currentstroke}%
\pgfsetdash{}{0pt}%
\pgfpathmoveto{\pgfqpoint{8.578381in}{3.293554in}}%
\pgfpathlineto{\pgfqpoint{8.578381in}{3.293554in}}%
\pgfusepath{stroke}%
\end{pgfscope}%
\begin{pgfscope}%
\pgfpathrectangle{\pgfqpoint{6.720588in}{1.750000in}}{\pgfqpoint{2.279412in}{2.004545in}}%
\pgfusepath{clip}%
\pgfsetbuttcap%
\pgfsetroundjoin%
\pgfsetlinewidth{0.319043pt}%
\definecolor{currentstroke}{rgb}{0.269944,0.014625,0.341379}%
\pgfsetstrokecolor{currentstroke}%
\pgfsetdash{}{0pt}%
\pgfpathmoveto{\pgfqpoint{8.578381in}{3.293554in}}%
\pgfpathlineto{\pgfqpoint{8.553879in}{3.293774in}}%
\pgfusepath{stroke}%
\end{pgfscope}%
\begin{pgfscope}%
\pgfpathrectangle{\pgfqpoint{6.720588in}{1.750000in}}{\pgfqpoint{2.279412in}{2.004545in}}%
\pgfusepath{clip}%
\pgfsetbuttcap%
\pgfsetroundjoin%
\pgfsetlinewidth{0.316627pt}%
\definecolor{currentstroke}{rgb}{0.269944,0.014625,0.341379}%
\pgfsetstrokecolor{currentstroke}%
\pgfsetdash{}{0pt}%
\pgfpathmoveto{\pgfqpoint{8.553879in}{3.293774in}}%
\pgfpathlineto{\pgfqpoint{8.505662in}{3.294519in}}%
\pgfusepath{stroke}%
\end{pgfscope}%
\begin{pgfscope}%
\pgfpathrectangle{\pgfqpoint{6.720588in}{1.750000in}}{\pgfqpoint{2.279412in}{2.004545in}}%
\pgfusepath{clip}%
\pgfsetbuttcap%
\pgfsetroundjoin%
\pgfsetlinewidth{0.313283pt}%
\definecolor{currentstroke}{rgb}{0.268510,0.009605,0.335427}%
\pgfsetstrokecolor{currentstroke}%
\pgfsetdash{}{0pt}%
\pgfpathmoveto{\pgfqpoint{8.505662in}{3.294519in}}%
\pgfpathlineto{\pgfqpoint{8.455632in}{3.295239in}}%
\pgfusepath{stroke}%
\end{pgfscope}%
\begin{pgfscope}%
\pgfpathrectangle{\pgfqpoint{6.720588in}{1.750000in}}{\pgfqpoint{2.279412in}{2.004545in}}%
\pgfusepath{clip}%
\pgfsetbuttcap%
\pgfsetroundjoin%
\pgfsetlinewidth{0.319466pt}%
\definecolor{currentstroke}{rgb}{0.269944,0.014625,0.341379}%
\pgfsetstrokecolor{currentstroke}%
\pgfsetdash{}{0pt}%
\pgfpathmoveto{\pgfqpoint{8.455632in}{3.295239in}}%
\pgfpathlineto{\pgfqpoint{8.405522in}{3.294577in}}%
\pgfusepath{stroke}%
\end{pgfscope}%
\begin{pgfscope}%
\pgfpathrectangle{\pgfqpoint{6.720588in}{1.750000in}}{\pgfqpoint{2.279412in}{2.004545in}}%
\pgfusepath{clip}%
\pgfsetbuttcap%
\pgfsetroundjoin%
\pgfsetlinewidth{0.326990pt}%
\definecolor{currentstroke}{rgb}{0.271305,0.019942,0.347269}%
\pgfsetstrokecolor{currentstroke}%
\pgfsetdash{}{0pt}%
\pgfpathmoveto{\pgfqpoint{8.405522in}{3.294577in}}%
\pgfpathlineto{\pgfqpoint{8.355421in}{3.293842in}}%
\pgfusepath{stroke}%
\end{pgfscope}%
\begin{pgfscope}%
\pgfpathrectangle{\pgfqpoint{6.720588in}{1.750000in}}{\pgfqpoint{2.279412in}{2.004545in}}%
\pgfusepath{clip}%
\pgfsetbuttcap%
\pgfsetroundjoin%
\pgfsetlinewidth{0.330452pt}%
\definecolor{currentstroke}{rgb}{0.272594,0.025563,0.353093}%
\pgfsetstrokecolor{currentstroke}%
\pgfsetdash{}{0pt}%
\pgfpathmoveto{\pgfqpoint{8.355421in}{3.293842in}}%
\pgfpathlineto{\pgfqpoint{8.305300in}{3.293214in}}%
\pgfusepath{stroke}%
\end{pgfscope}%
\begin{pgfscope}%
\pgfpathrectangle{\pgfqpoint{6.720588in}{1.750000in}}{\pgfqpoint{2.279412in}{2.004545in}}%
\pgfusepath{clip}%
\pgfsetbuttcap%
\pgfsetroundjoin%
\pgfsetlinewidth{0.340097pt}%
\definecolor{currentstroke}{rgb}{0.273809,0.031497,0.358853}%
\pgfsetstrokecolor{currentstroke}%
\pgfsetdash{}{0pt}%
\pgfpathmoveto{\pgfqpoint{8.305300in}{3.293214in}}%
\pgfpathlineto{\pgfqpoint{8.255176in}{3.292437in}}%
\pgfusepath{stroke}%
\end{pgfscope}%
\begin{pgfscope}%
\pgfpathrectangle{\pgfqpoint{6.720588in}{1.750000in}}{\pgfqpoint{2.279412in}{2.004545in}}%
\pgfusepath{clip}%
\pgfsetbuttcap%
\pgfsetroundjoin%
\pgfsetlinewidth{0.337415pt}%
\definecolor{currentstroke}{rgb}{0.273809,0.031497,0.358853}%
\pgfsetstrokecolor{currentstroke}%
\pgfsetdash{}{0pt}%
\pgfpathmoveto{\pgfqpoint{8.255176in}{3.292437in}}%
\pgfpathlineto{\pgfqpoint{8.205090in}{3.290756in}}%
\pgfusepath{stroke}%
\end{pgfscope}%
\begin{pgfscope}%
\pgfpathrectangle{\pgfqpoint{6.720588in}{1.750000in}}{\pgfqpoint{2.279412in}{2.004545in}}%
\pgfusepath{clip}%
\pgfsetbuttcap%
\pgfsetroundjoin%
\pgfsetlinewidth{0.353681pt}%
\definecolor{currentstroke}{rgb}{0.276022,0.044167,0.370164}%
\pgfsetstrokecolor{currentstroke}%
\pgfsetdash{}{0pt}%
\pgfpathmoveto{\pgfqpoint{8.205090in}{3.290756in}}%
\pgfpathlineto{\pgfqpoint{8.155098in}{3.287285in}}%
\pgfusepath{stroke}%
\end{pgfscope}%
\begin{pgfscope}%
\pgfpathrectangle{\pgfqpoint{6.720588in}{1.750000in}}{\pgfqpoint{2.279412in}{2.004545in}}%
\pgfusepath{clip}%
\pgfsetbuttcap%
\pgfsetroundjoin%
\pgfsetlinewidth{0.369912pt}%
\definecolor{currentstroke}{rgb}{0.278791,0.062145,0.386592}%
\pgfsetstrokecolor{currentstroke}%
\pgfsetdash{}{0pt}%
\pgfpathmoveto{\pgfqpoint{8.155098in}{3.287285in}}%
\pgfpathlineto{\pgfqpoint{8.105187in}{3.283035in}}%
\pgfusepath{stroke}%
\end{pgfscope}%
\begin{pgfscope}%
\pgfpathrectangle{\pgfqpoint{6.720588in}{1.750000in}}{\pgfqpoint{2.279412in}{2.004545in}}%
\pgfusepath{clip}%
\pgfsetbuttcap%
\pgfsetroundjoin%
\pgfsetlinewidth{0.322849pt}%
\definecolor{currentstroke}{rgb}{0.271305,0.019942,0.347269}%
\pgfsetstrokecolor{currentstroke}%
\pgfsetdash{}{0pt}%
\pgfpathmoveto{\pgfqpoint{8.219337in}{3.383768in}}%
\pgfpathlineto{\pgfqpoint{8.169363in}{3.381139in}}%
\pgfusepath{stroke}%
\end{pgfscope}%
\begin{pgfscope}%
\pgfpathrectangle{\pgfqpoint{6.720588in}{1.750000in}}{\pgfqpoint{2.279412in}{2.004545in}}%
\pgfusepath{clip}%
\pgfsetbuttcap%
\pgfsetroundjoin%
\pgfsetlinewidth{0.343150pt}%
\definecolor{currentstroke}{rgb}{0.274952,0.037752,0.364543}%
\pgfsetstrokecolor{currentstroke}%
\pgfsetdash{}{0pt}%
\pgfpathmoveto{\pgfqpoint{8.169363in}{3.381139in}}%
\pgfpathlineto{\pgfqpoint{8.119603in}{3.375816in}}%
\pgfusepath{stroke}%
\end{pgfscope}%
\begin{pgfscope}%
\pgfpathrectangle{\pgfqpoint{6.720588in}{1.750000in}}{\pgfqpoint{2.279412in}{2.004545in}}%
\pgfusepath{clip}%
\pgfsetbuttcap%
\pgfsetroundjoin%
\pgfsetlinewidth{0.348245pt}%
\definecolor{currentstroke}{rgb}{0.274952,0.037752,0.364543}%
\pgfsetstrokecolor{currentstroke}%
\pgfsetdash{}{0pt}%
\pgfpathmoveto{\pgfqpoint{8.119603in}{3.375816in}}%
\pgfpathlineto{\pgfqpoint{8.069909in}{3.370104in}}%
\pgfusepath{stroke}%
\end{pgfscope}%
\begin{pgfscope}%
\pgfpathrectangle{\pgfqpoint{6.720588in}{1.750000in}}{\pgfqpoint{2.279412in}{2.004545in}}%
\pgfusepath{clip}%
\pgfsetbuttcap%
\pgfsetroundjoin%
\pgfsetlinewidth{0.331492pt}%
\definecolor{currentstroke}{rgb}{0.272594,0.025563,0.353093}%
\pgfsetstrokecolor{currentstroke}%
\pgfsetdash{}{0pt}%
\pgfpathmoveto{\pgfqpoint{8.069909in}{3.370104in}}%
\pgfpathlineto{\pgfqpoint{8.020228in}{3.364249in}}%
\pgfusepath{stroke}%
\end{pgfscope}%
\begin{pgfscope}%
\pgfpathrectangle{\pgfqpoint{6.720588in}{1.750000in}}{\pgfqpoint{2.279412in}{2.004545in}}%
\pgfusepath{clip}%
\pgfsetbuttcap%
\pgfsetroundjoin%
\pgfsetlinewidth{0.350555pt}%
\definecolor{currentstroke}{rgb}{0.276022,0.044167,0.370164}%
\pgfsetstrokecolor{currentstroke}%
\pgfsetdash{}{0pt}%
\pgfpathmoveto{\pgfqpoint{8.020228in}{3.364249in}}%
\pgfpathlineto{\pgfqpoint{7.970734in}{3.357545in}}%
\pgfusepath{stroke}%
\end{pgfscope}%
\begin{pgfscope}%
\pgfpathrectangle{\pgfqpoint{6.720588in}{1.750000in}}{\pgfqpoint{2.279412in}{2.004545in}}%
\pgfusepath{clip}%
\pgfsetbuttcap%
\pgfsetroundjoin%
\pgfsetlinewidth{0.325921pt}%
\definecolor{currentstroke}{rgb}{0.271305,0.019942,0.347269}%
\pgfsetstrokecolor{currentstroke}%
\pgfsetdash{}{0pt}%
\pgfpathmoveto{\pgfqpoint{8.527089in}{2.391418in}}%
\pgfpathlineto{\pgfqpoint{8.476957in}{2.390646in}}%
\pgfusepath{stroke}%
\end{pgfscope}%
\begin{pgfscope}%
\pgfpathrectangle{\pgfqpoint{6.720588in}{1.750000in}}{\pgfqpoint{2.279412in}{2.004545in}}%
\pgfusepath{clip}%
\pgfsetbuttcap%
\pgfsetroundjoin%
\pgfsetlinewidth{0.325775pt}%
\definecolor{currentstroke}{rgb}{0.271305,0.019942,0.347269}%
\pgfsetstrokecolor{currentstroke}%
\pgfsetdash{}{0pt}%
\pgfpathmoveto{\pgfqpoint{8.476957in}{2.390646in}}%
\pgfpathlineto{\pgfqpoint{8.426825in}{2.390582in}}%
\pgfusepath{stroke}%
\end{pgfscope}%
\begin{pgfscope}%
\pgfpathrectangle{\pgfqpoint{6.720588in}{1.750000in}}{\pgfqpoint{2.279412in}{2.004545in}}%
\pgfusepath{clip}%
\pgfsetbuttcap%
\pgfsetroundjoin%
\pgfsetlinewidth{0.336873pt}%
\definecolor{currentstroke}{rgb}{0.273809,0.031497,0.358853}%
\pgfsetstrokecolor{currentstroke}%
\pgfsetdash{}{0pt}%
\pgfpathmoveto{\pgfqpoint{8.426825in}{2.390582in}}%
\pgfpathlineto{\pgfqpoint{8.376698in}{2.391749in}}%
\pgfusepath{stroke}%
\end{pgfscope}%
\begin{pgfscope}%
\pgfpathrectangle{\pgfqpoint{6.720588in}{1.750000in}}{\pgfqpoint{2.279412in}{2.004545in}}%
\pgfusepath{clip}%
\pgfsetbuttcap%
\pgfsetroundjoin%
\pgfsetlinewidth{0.344878pt}%
\definecolor{currentstroke}{rgb}{0.274952,0.037752,0.364543}%
\pgfsetstrokecolor{currentstroke}%
\pgfsetdash{}{0pt}%
\pgfpathmoveto{\pgfqpoint{8.376698in}{2.391749in}}%
\pgfpathlineto{\pgfqpoint{8.326573in}{2.393104in}}%
\pgfusepath{stroke}%
\end{pgfscope}%
\begin{pgfscope}%
\pgfpathrectangle{\pgfqpoint{6.720588in}{1.750000in}}{\pgfqpoint{2.279412in}{2.004545in}}%
\pgfusepath{clip}%
\pgfsetbuttcap%
\pgfsetroundjoin%
\pgfsetlinewidth{0.363655pt}%
\definecolor{currentstroke}{rgb}{0.277941,0.056324,0.381191}%
\pgfsetstrokecolor{currentstroke}%
\pgfsetdash{}{0pt}%
\pgfpathmoveto{\pgfqpoint{8.326573in}{2.393104in}}%
\pgfpathlineto{\pgfqpoint{8.276469in}{2.394931in}}%
\pgfusepath{stroke}%
\end{pgfscope}%
\begin{pgfscope}%
\pgfpathrectangle{\pgfqpoint{6.720588in}{1.750000in}}{\pgfqpoint{2.279412in}{2.004545in}}%
\pgfusepath{clip}%
\pgfsetbuttcap%
\pgfsetroundjoin%
\pgfsetlinewidth{0.394982pt}%
\definecolor{currentstroke}{rgb}{0.280894,0.078907,0.402329}%
\pgfsetstrokecolor{currentstroke}%
\pgfsetdash{}{0pt}%
\pgfpathmoveto{\pgfqpoint{8.276469in}{2.394931in}}%
\pgfpathlineto{\pgfqpoint{8.226393in}{2.397354in}}%
\pgfusepath{stroke}%
\end{pgfscope}%
\begin{pgfscope}%
\pgfpathrectangle{\pgfqpoint{6.720588in}{1.750000in}}{\pgfqpoint{2.279412in}{2.004545in}}%
\pgfusepath{clip}%
\pgfsetbuttcap%
\pgfsetroundjoin%
\pgfsetlinewidth{0.413690pt}%
\definecolor{currentstroke}{rgb}{0.282327,0.094955,0.417331}%
\pgfsetstrokecolor{currentstroke}%
\pgfsetdash{}{0pt}%
\pgfpathmoveto{\pgfqpoint{8.226393in}{2.397354in}}%
\pgfpathlineto{\pgfqpoint{8.176313in}{2.399707in}}%
\pgfusepath{stroke}%
\end{pgfscope}%
\begin{pgfscope}%
\pgfpathrectangle{\pgfqpoint{6.720588in}{1.750000in}}{\pgfqpoint{2.279412in}{2.004545in}}%
\pgfusepath{clip}%
\pgfsetbuttcap%
\pgfsetroundjoin%
\pgfsetlinewidth{0.467673pt}%
\definecolor{currentstroke}{rgb}{0.282884,0.135920,0.453427}%
\pgfsetstrokecolor{currentstroke}%
\pgfsetdash{}{0pt}%
\pgfpathmoveto{\pgfqpoint{8.176313in}{2.399707in}}%
\pgfpathlineto{\pgfqpoint{8.126288in}{2.402755in}}%
\pgfusepath{stroke}%
\end{pgfscope}%
\begin{pgfscope}%
\pgfpathrectangle{\pgfqpoint{6.720588in}{1.750000in}}{\pgfqpoint{2.279412in}{2.004545in}}%
\pgfusepath{clip}%
\pgfsetbuttcap%
\pgfsetroundjoin%
\pgfsetlinewidth{0.473650pt}%
\definecolor{currentstroke}{rgb}{0.282623,0.140926,0.457517}%
\pgfsetstrokecolor{currentstroke}%
\pgfsetdash{}{0pt}%
\pgfpathmoveto{\pgfqpoint{8.126288in}{2.402755in}}%
\pgfpathlineto{\pgfqpoint{8.076343in}{2.406739in}}%
\pgfusepath{stroke}%
\end{pgfscope}%
\begin{pgfscope}%
\pgfpathrectangle{\pgfqpoint{6.720588in}{1.750000in}}{\pgfqpoint{2.279412in}{2.004545in}}%
\pgfusepath{clip}%
\pgfsetbuttcap%
\pgfsetroundjoin%
\pgfsetlinewidth{0.324475pt}%
\definecolor{currentstroke}{rgb}{0.271305,0.019942,0.347269}%
\pgfsetstrokecolor{currentstroke}%
\pgfsetdash{}{0pt}%
\pgfpathmoveto{\pgfqpoint{8.527089in}{2.662059in}}%
\pgfpathlineto{\pgfqpoint{8.476965in}{2.662153in}}%
\pgfusepath{stroke}%
\end{pgfscope}%
\begin{pgfscope}%
\pgfpathrectangle{\pgfqpoint{6.720588in}{1.750000in}}{\pgfqpoint{2.279412in}{2.004545in}}%
\pgfusepath{clip}%
\pgfsetbuttcap%
\pgfsetroundjoin%
\pgfsetlinewidth{0.335442pt}%
\definecolor{currentstroke}{rgb}{0.273809,0.031497,0.358853}%
\pgfsetstrokecolor{currentstroke}%
\pgfsetdash{}{0pt}%
\pgfpathmoveto{\pgfqpoint{8.476965in}{2.662153in}}%
\pgfpathlineto{\pgfqpoint{8.426830in}{2.662286in}}%
\pgfusepath{stroke}%
\end{pgfscope}%
\begin{pgfscope}%
\pgfpathrectangle{\pgfqpoint{6.720588in}{1.750000in}}{\pgfqpoint{2.279412in}{2.004545in}}%
\pgfusepath{clip}%
\pgfsetbuttcap%
\pgfsetroundjoin%
\pgfsetlinewidth{0.341138pt}%
\definecolor{currentstroke}{rgb}{0.273809,0.031497,0.358853}%
\pgfsetstrokecolor{currentstroke}%
\pgfsetdash{}{0pt}%
\pgfpathmoveto{\pgfqpoint{8.426830in}{2.662286in}}%
\pgfpathlineto{\pgfqpoint{8.376679in}{2.662493in}}%
\pgfusepath{stroke}%
\end{pgfscope}%
\begin{pgfscope}%
\pgfpathrectangle{\pgfqpoint{6.720588in}{1.750000in}}{\pgfqpoint{2.279412in}{2.004545in}}%
\pgfusepath{clip}%
\pgfsetbuttcap%
\pgfsetroundjoin%
\pgfsetlinewidth{0.375301pt}%
\definecolor{currentstroke}{rgb}{0.278791,0.062145,0.386592}%
\pgfsetstrokecolor{currentstroke}%
\pgfsetdash{}{0pt}%
\pgfpathmoveto{\pgfqpoint{8.376679in}{2.662493in}}%
\pgfpathlineto{\pgfqpoint{8.326529in}{2.662425in}}%
\pgfusepath{stroke}%
\end{pgfscope}%
\begin{pgfscope}%
\pgfpathrectangle{\pgfqpoint{6.720588in}{1.750000in}}{\pgfqpoint{2.279412in}{2.004545in}}%
\pgfusepath{clip}%
\pgfsetbuttcap%
\pgfsetroundjoin%
\pgfsetlinewidth{0.400809pt}%
\definecolor{currentstroke}{rgb}{0.281446,0.084320,0.407414}%
\pgfsetstrokecolor{currentstroke}%
\pgfsetdash{}{0pt}%
\pgfpathmoveto{\pgfqpoint{8.326529in}{2.662425in}}%
\pgfpathlineto{\pgfqpoint{8.276379in}{2.662327in}}%
\pgfusepath{stroke}%
\end{pgfscope}%
\begin{pgfscope}%
\pgfpathrectangle{\pgfqpoint{6.720588in}{1.750000in}}{\pgfqpoint{2.279412in}{2.004545in}}%
\pgfusepath{clip}%
\pgfsetbuttcap%
\pgfsetroundjoin%
\pgfsetlinewidth{0.454331pt}%
\definecolor{currentstroke}{rgb}{0.283187,0.125848,0.444960}%
\pgfsetstrokecolor{currentstroke}%
\pgfsetdash{}{0pt}%
\pgfpathmoveto{\pgfqpoint{8.276379in}{2.662327in}}%
\pgfpathlineto{\pgfqpoint{8.226229in}{2.662453in}}%
\pgfusepath{stroke}%
\end{pgfscope}%
\begin{pgfscope}%
\pgfpathrectangle{\pgfqpoint{6.720588in}{1.750000in}}{\pgfqpoint{2.279412in}{2.004545in}}%
\pgfusepath{clip}%
\pgfsetbuttcap%
\pgfsetroundjoin%
\pgfsetlinewidth{0.509482pt}%
\definecolor{currentstroke}{rgb}{0.280255,0.165693,0.476498}%
\pgfsetstrokecolor{currentstroke}%
\pgfsetdash{}{0pt}%
\pgfpathmoveto{\pgfqpoint{8.226229in}{2.662453in}}%
\pgfpathlineto{\pgfqpoint{8.176079in}{2.662766in}}%
\pgfusepath{stroke}%
\end{pgfscope}%
\begin{pgfscope}%
\pgfpathrectangle{\pgfqpoint{6.720588in}{1.750000in}}{\pgfqpoint{2.279412in}{2.004545in}}%
\pgfusepath{clip}%
\pgfsetbuttcap%
\pgfsetroundjoin%
\pgfsetlinewidth{0.618993pt}%
\definecolor{currentstroke}{rgb}{0.262138,0.242286,0.520837}%
\pgfsetstrokecolor{currentstroke}%
\pgfsetdash{}{0pt}%
\pgfpathmoveto{\pgfqpoint{8.176079in}{2.662766in}}%
\pgfpathlineto{\pgfqpoint{8.125929in}{2.663081in}}%
\pgfusepath{stroke}%
\end{pgfscope}%
\begin{pgfscope}%
\pgfpathrectangle{\pgfqpoint{6.720588in}{1.750000in}}{\pgfqpoint{2.279412in}{2.004545in}}%
\pgfusepath{clip}%
\pgfsetbuttcap%
\pgfsetroundjoin%
\pgfsetlinewidth{0.704088pt}%
\definecolor{currentstroke}{rgb}{0.241237,0.296485,0.539709}%
\pgfsetstrokecolor{currentstroke}%
\pgfsetdash{}{0pt}%
\pgfpathmoveto{\pgfqpoint{8.125929in}{2.663081in}}%
\pgfpathlineto{\pgfqpoint{8.075780in}{2.663500in}}%
\pgfusepath{stroke}%
\end{pgfscope}%
\begin{pgfscope}%
\pgfpathrectangle{\pgfqpoint{6.720588in}{1.750000in}}{\pgfqpoint{2.279412in}{2.004545in}}%
\pgfusepath{clip}%
\pgfsetbuttcap%
\pgfsetroundjoin%
\pgfsetlinewidth{0.773698pt}%
\definecolor{currentstroke}{rgb}{0.221989,0.339161,0.548752}%
\pgfsetstrokecolor{currentstroke}%
\pgfsetdash{}{0pt}%
\pgfpathmoveto{\pgfqpoint{8.075780in}{2.663500in}}%
\pgfpathlineto{\pgfqpoint{8.025635in}{2.664149in}}%
\pgfusepath{stroke}%
\end{pgfscope}%
\begin{pgfscope}%
\pgfpathrectangle{\pgfqpoint{6.720588in}{1.750000in}}{\pgfqpoint{2.279412in}{2.004545in}}%
\pgfusepath{clip}%
\pgfsetbuttcap%
\pgfsetroundjoin%
\pgfsetlinewidth{0.853058pt}%
\definecolor{currentstroke}{rgb}{0.199430,0.387607,0.554642}%
\pgfsetstrokecolor{currentstroke}%
\pgfsetdash{}{0pt}%
\pgfpathmoveto{\pgfqpoint{8.025635in}{2.664149in}}%
\pgfpathlineto{\pgfqpoint{7.975493in}{2.664992in}}%
\pgfusepath{stroke}%
\end{pgfscope}%
\begin{pgfscope}%
\pgfpathrectangle{\pgfqpoint{6.720588in}{1.750000in}}{\pgfqpoint{2.279412in}{2.004545in}}%
\pgfusepath{clip}%
\pgfsetbuttcap%
\pgfsetroundjoin%
\pgfsetlinewidth{0.837725pt}%
\definecolor{currentstroke}{rgb}{0.203063,0.379716,0.553925}%
\pgfsetstrokecolor{currentstroke}%
\pgfsetdash{}{0pt}%
\pgfpathmoveto{\pgfqpoint{7.975493in}{2.664992in}}%
\pgfpathlineto{\pgfqpoint{7.925359in}{2.666064in}}%
\pgfusepath{stroke}%
\end{pgfscope}%
\begin{pgfscope}%
\pgfpathrectangle{\pgfqpoint{6.720588in}{1.750000in}}{\pgfqpoint{2.279412in}{2.004545in}}%
\pgfusepath{clip}%
\pgfsetbuttcap%
\pgfsetroundjoin%
\pgfsetlinewidth{0.816669pt}%
\definecolor{currentstroke}{rgb}{0.208623,0.367752,0.552675}%
\pgfsetstrokecolor{currentstroke}%
\pgfsetdash{}{0pt}%
\pgfpathmoveto{\pgfqpoint{7.925359in}{2.666064in}}%
\pgfpathlineto{\pgfqpoint{7.875240in}{2.667514in}}%
\pgfusepath{stroke}%
\end{pgfscope}%
\begin{pgfscope}%
\pgfpathrectangle{\pgfqpoint{6.720588in}{1.750000in}}{\pgfqpoint{2.279412in}{2.004545in}}%
\pgfusepath{clip}%
\pgfsetbuttcap%
\pgfsetroundjoin%
\pgfsetlinewidth{0.794694pt}%
\definecolor{currentstroke}{rgb}{0.216210,0.351535,0.550627}%
\pgfsetstrokecolor{currentstroke}%
\pgfsetdash{}{0pt}%
\pgfpathmoveto{\pgfqpoint{7.875240in}{2.667514in}}%
\pgfpathlineto{\pgfqpoint{7.825144in}{2.669484in}}%
\pgfusepath{stroke}%
\end{pgfscope}%
\begin{pgfscope}%
\pgfpathrectangle{\pgfqpoint{6.720588in}{1.750000in}}{\pgfqpoint{2.279412in}{2.004545in}}%
\pgfusepath{clip}%
\pgfsetbuttcap%
\pgfsetroundjoin%
\pgfsetlinewidth{0.745747pt}%
\definecolor{currentstroke}{rgb}{0.229739,0.322361,0.545706}%
\pgfsetstrokecolor{currentstroke}%
\pgfsetdash{}{0pt}%
\pgfpathmoveto{\pgfqpoint{7.825144in}{2.669484in}}%
\pgfpathlineto{\pgfqpoint{7.775072in}{2.671898in}}%
\pgfusepath{stroke}%
\end{pgfscope}%
\begin{pgfscope}%
\pgfpathrectangle{\pgfqpoint{6.720588in}{1.750000in}}{\pgfqpoint{2.279412in}{2.004545in}}%
\pgfusepath{clip}%
\pgfsetbuttcap%
\pgfsetroundjoin%
\pgfsetlinewidth{0.712348pt}%
\definecolor{currentstroke}{rgb}{0.239346,0.300855,0.540844}%
\pgfsetstrokecolor{currentstroke}%
\pgfsetdash{}{0pt}%
\pgfpathmoveto{\pgfqpoint{7.775072in}{2.671898in}}%
\pgfpathlineto{\pgfqpoint{7.725088in}{2.675305in}}%
\pgfusepath{stroke}%
\end{pgfscope}%
\begin{pgfscope}%
\pgfpathrectangle{\pgfqpoint{6.720588in}{1.750000in}}{\pgfqpoint{2.279412in}{2.004545in}}%
\pgfusepath{clip}%
\pgfsetbuttcap%
\pgfsetroundjoin%
\pgfsetlinewidth{0.611992pt}%
\definecolor{currentstroke}{rgb}{0.263663,0.237631,0.518762}%
\pgfsetstrokecolor{currentstroke}%
\pgfsetdash{}{0pt}%
\pgfpathmoveto{\pgfqpoint{7.725088in}{2.675305in}}%
\pgfpathlineto{\pgfqpoint{7.675161in}{2.679257in}}%
\pgfusepath{stroke}%
\end{pgfscope}%
\begin{pgfscope}%
\pgfpathrectangle{\pgfqpoint{6.720588in}{1.750000in}}{\pgfqpoint{2.279412in}{2.004545in}}%
\pgfusepath{clip}%
\pgfsetbuttcap%
\pgfsetroundjoin%
\pgfsetlinewidth{0.318710pt}%
\definecolor{currentstroke}{rgb}{0.269944,0.014625,0.341379}%
\pgfsetstrokecolor{currentstroke}%
\pgfsetdash{}{0pt}%
\pgfpathmoveto{\pgfqpoint{8.527089in}{2.797380in}}%
\pgfpathlineto{\pgfqpoint{8.477045in}{2.797720in}}%
\pgfusepath{stroke}%
\end{pgfscope}%
\begin{pgfscope}%
\pgfpathrectangle{\pgfqpoint{6.720588in}{1.750000in}}{\pgfqpoint{2.279412in}{2.004545in}}%
\pgfusepath{clip}%
\pgfsetbuttcap%
\pgfsetroundjoin%
\pgfsetlinewidth{0.328028pt}%
\definecolor{currentstroke}{rgb}{0.271305,0.019942,0.347269}%
\pgfsetstrokecolor{currentstroke}%
\pgfsetdash{}{0pt}%
\pgfpathmoveto{\pgfqpoint{8.477045in}{2.797720in}}%
\pgfpathlineto{\pgfqpoint{8.426900in}{2.797296in}}%
\pgfusepath{stroke}%
\end{pgfscope}%
\begin{pgfscope}%
\pgfpathrectangle{\pgfqpoint{6.720588in}{1.750000in}}{\pgfqpoint{2.279412in}{2.004545in}}%
\pgfusepath{clip}%
\pgfsetbuttcap%
\pgfsetroundjoin%
\pgfsetlinewidth{0.344465pt}%
\definecolor{currentstroke}{rgb}{0.274952,0.037752,0.364543}%
\pgfsetstrokecolor{currentstroke}%
\pgfsetdash{}{0pt}%
\pgfpathmoveto{\pgfqpoint{8.426900in}{2.797296in}}%
\pgfpathlineto{\pgfqpoint{8.376752in}{2.796877in}}%
\pgfusepath{stroke}%
\end{pgfscope}%
\begin{pgfscope}%
\pgfpathrectangle{\pgfqpoint{6.720588in}{1.750000in}}{\pgfqpoint{2.279412in}{2.004545in}}%
\pgfusepath{clip}%
\pgfsetbuttcap%
\pgfsetroundjoin%
\pgfsetlinewidth{0.368336pt}%
\definecolor{currentstroke}{rgb}{0.277941,0.056324,0.381191}%
\pgfsetstrokecolor{currentstroke}%
\pgfsetdash{}{0pt}%
\pgfpathmoveto{\pgfqpoint{8.376752in}{2.796877in}}%
\pgfpathlineto{\pgfqpoint{8.326602in}{2.796697in}}%
\pgfusepath{stroke}%
\end{pgfscope}%
\begin{pgfscope}%
\pgfpathrectangle{\pgfqpoint{6.720588in}{1.750000in}}{\pgfqpoint{2.279412in}{2.004545in}}%
\pgfusepath{clip}%
\pgfsetbuttcap%
\pgfsetroundjoin%
\pgfsetlinewidth{0.406127pt}%
\definecolor{currentstroke}{rgb}{0.281924,0.089666,0.412415}%
\pgfsetstrokecolor{currentstroke}%
\pgfsetdash{}{0pt}%
\pgfpathmoveto{\pgfqpoint{8.326602in}{2.796697in}}%
\pgfpathlineto{\pgfqpoint{8.276450in}{2.796600in}}%
\pgfusepath{stroke}%
\end{pgfscope}%
\begin{pgfscope}%
\pgfpathrectangle{\pgfqpoint{6.720588in}{1.750000in}}{\pgfqpoint{2.279412in}{2.004545in}}%
\pgfusepath{clip}%
\pgfsetbuttcap%
\pgfsetroundjoin%
\pgfsetlinewidth{0.447116pt}%
\definecolor{currentstroke}{rgb}{0.283229,0.120777,0.440584}%
\pgfsetstrokecolor{currentstroke}%
\pgfsetdash{}{0pt}%
\pgfpathmoveto{\pgfqpoint{8.276450in}{2.796600in}}%
\pgfpathlineto{\pgfqpoint{8.226300in}{2.796385in}}%
\pgfusepath{stroke}%
\end{pgfscope}%
\begin{pgfscope}%
\pgfpathrectangle{\pgfqpoint{6.720588in}{1.750000in}}{\pgfqpoint{2.279412in}{2.004545in}}%
\pgfusepath{clip}%
\pgfsetbuttcap%
\pgfsetroundjoin%
\pgfsetlinewidth{0.497365pt}%
\definecolor{currentstroke}{rgb}{0.281412,0.155834,0.469201}%
\pgfsetstrokecolor{currentstroke}%
\pgfsetdash{}{0pt}%
\pgfpathmoveto{\pgfqpoint{8.226300in}{2.796385in}}%
\pgfpathlineto{\pgfqpoint{8.176150in}{2.796037in}}%
\pgfusepath{stroke}%
\end{pgfscope}%
\begin{pgfscope}%
\pgfpathrectangle{\pgfqpoint{6.720588in}{1.750000in}}{\pgfqpoint{2.279412in}{2.004545in}}%
\pgfusepath{clip}%
\pgfsetbuttcap%
\pgfsetroundjoin%
\pgfsetlinewidth{0.586111pt}%
\definecolor{currentstroke}{rgb}{0.269308,0.218818,0.509577}%
\pgfsetstrokecolor{currentstroke}%
\pgfsetdash{}{0pt}%
\pgfpathmoveto{\pgfqpoint{8.176150in}{2.796037in}}%
\pgfpathlineto{\pgfqpoint{8.125999in}{2.795686in}}%
\pgfusepath{stroke}%
\end{pgfscope}%
\begin{pgfscope}%
\pgfpathrectangle{\pgfqpoint{6.720588in}{1.750000in}}{\pgfqpoint{2.279412in}{2.004545in}}%
\pgfusepath{clip}%
\pgfsetbuttcap%
\pgfsetroundjoin%
\pgfsetlinewidth{0.704206pt}%
\definecolor{currentstroke}{rgb}{0.241237,0.296485,0.539709}%
\pgfsetstrokecolor{currentstroke}%
\pgfsetdash{}{0pt}%
\pgfpathmoveto{\pgfqpoint{8.125999in}{2.795686in}}%
\pgfpathlineto{\pgfqpoint{8.075851in}{2.795203in}}%
\pgfusepath{stroke}%
\end{pgfscope}%
\begin{pgfscope}%
\pgfpathrectangle{\pgfqpoint{6.720588in}{1.750000in}}{\pgfqpoint{2.279412in}{2.004545in}}%
\pgfusepath{clip}%
\pgfsetbuttcap%
\pgfsetroundjoin%
\pgfsetlinewidth{0.778683pt}%
\definecolor{currentstroke}{rgb}{0.220057,0.343307,0.549413}%
\pgfsetstrokecolor{currentstroke}%
\pgfsetdash{}{0pt}%
\pgfpathmoveto{\pgfqpoint{8.075851in}{2.795203in}}%
\pgfpathlineto{\pgfqpoint{8.025707in}{2.794468in}}%
\pgfusepath{stroke}%
\end{pgfscope}%
\begin{pgfscope}%
\pgfpathrectangle{\pgfqpoint{6.720588in}{1.750000in}}{\pgfqpoint{2.279412in}{2.004545in}}%
\pgfusepath{clip}%
\pgfsetbuttcap%
\pgfsetroundjoin%
\pgfsetlinewidth{0.782071pt}%
\definecolor{currentstroke}{rgb}{0.218130,0.347432,0.550038}%
\pgfsetstrokecolor{currentstroke}%
\pgfsetdash{}{0pt}%
\pgfpathmoveto{\pgfqpoint{8.025707in}{2.794468in}}%
\pgfpathlineto{\pgfqpoint{7.975566in}{2.793535in}}%
\pgfusepath{stroke}%
\end{pgfscope}%
\begin{pgfscope}%
\pgfpathrectangle{\pgfqpoint{6.720588in}{1.750000in}}{\pgfqpoint{2.279412in}{2.004545in}}%
\pgfusepath{clip}%
\pgfsetbuttcap%
\pgfsetroundjoin%
\pgfsetlinewidth{0.840337pt}%
\definecolor{currentstroke}{rgb}{0.203063,0.379716,0.553925}%
\pgfsetstrokecolor{currentstroke}%
\pgfsetdash{}{0pt}%
\pgfpathmoveto{\pgfqpoint{7.975566in}{2.793535in}}%
\pgfpathlineto{\pgfqpoint{7.925433in}{2.792353in}}%
\pgfusepath{stroke}%
\end{pgfscope}%
\begin{pgfscope}%
\pgfpathrectangle{\pgfqpoint{6.720588in}{1.750000in}}{\pgfqpoint{2.279412in}{2.004545in}}%
\pgfusepath{clip}%
\pgfsetbuttcap%
\pgfsetroundjoin%
\pgfsetlinewidth{0.821684pt}%
\definecolor{currentstroke}{rgb}{0.208623,0.367752,0.552675}%
\pgfsetstrokecolor{currentstroke}%
\pgfsetdash{}{0pt}%
\pgfpathmoveto{\pgfqpoint{7.925433in}{2.792353in}}%
\pgfpathlineto{\pgfqpoint{7.875315in}{2.790743in}}%
\pgfusepath{stroke}%
\end{pgfscope}%
\begin{pgfscope}%
\pgfpathrectangle{\pgfqpoint{6.720588in}{1.750000in}}{\pgfqpoint{2.279412in}{2.004545in}}%
\pgfusepath{clip}%
\pgfsetbuttcap%
\pgfsetroundjoin%
\pgfsetlinewidth{0.821935pt}%
\definecolor{currentstroke}{rgb}{0.208623,0.367752,0.552675}%
\pgfsetstrokecolor{currentstroke}%
\pgfsetdash{}{0pt}%
\pgfpathmoveto{\pgfqpoint{7.875315in}{2.790743in}}%
\pgfpathlineto{\pgfqpoint{7.825229in}{2.788532in}}%
\pgfusepath{stroke}%
\end{pgfscope}%
\begin{pgfscope}%
\pgfpathrectangle{\pgfqpoint{6.720588in}{1.750000in}}{\pgfqpoint{2.279412in}{2.004545in}}%
\pgfusepath{clip}%
\pgfsetbuttcap%
\pgfsetroundjoin%
\pgfsetlinewidth{0.825559pt}%
\definecolor{currentstroke}{rgb}{0.206756,0.371758,0.553117}%
\pgfsetstrokecolor{currentstroke}%
\pgfsetdash{}{0pt}%
\pgfpathmoveto{\pgfqpoint{7.825229in}{2.788532in}}%
\pgfpathlineto{\pgfqpoint{7.775173in}{2.785852in}}%
\pgfusepath{stroke}%
\end{pgfscope}%
\begin{pgfscope}%
\pgfpathrectangle{\pgfqpoint{6.720588in}{1.750000in}}{\pgfqpoint{2.279412in}{2.004545in}}%
\pgfusepath{clip}%
\pgfsetbuttcap%
\pgfsetroundjoin%
\pgfsetlinewidth{0.705534pt}%
\definecolor{currentstroke}{rgb}{0.241237,0.296485,0.539709}%
\pgfsetstrokecolor{currentstroke}%
\pgfsetdash{}{0pt}%
\pgfpathmoveto{\pgfqpoint{7.775173in}{2.785852in}}%
\pgfpathlineto{\pgfqpoint{7.725155in}{2.782712in}}%
\pgfusepath{stroke}%
\end{pgfscope}%
\begin{pgfscope}%
\pgfpathrectangle{\pgfqpoint{6.720588in}{1.750000in}}{\pgfqpoint{2.279412in}{2.004545in}}%
\pgfusepath{clip}%
\pgfsetbuttcap%
\pgfsetroundjoin%
\pgfsetlinewidth{0.679029pt}%
\definecolor{currentstroke}{rgb}{0.246811,0.283237,0.535941}%
\pgfsetstrokecolor{currentstroke}%
\pgfsetdash{}{0pt}%
\pgfpathmoveto{\pgfqpoint{7.725155in}{2.782712in}}%
\pgfpathlineto{\pgfqpoint{7.675233in}{2.778623in}}%
\pgfusepath{stroke}%
\end{pgfscope}%
\begin{pgfscope}%
\pgfpathrectangle{\pgfqpoint{6.720588in}{1.750000in}}{\pgfqpoint{2.279412in}{2.004545in}}%
\pgfusepath{clip}%
\pgfsetbuttcap%
\pgfsetroundjoin%
\pgfsetlinewidth{0.321447pt}%
\definecolor{currentstroke}{rgb}{0.269944,0.014625,0.341379}%
\pgfsetstrokecolor{currentstroke}%
\pgfsetdash{}{0pt}%
\pgfpathmoveto{\pgfqpoint{8.527089in}{2.842486in}}%
\pgfpathlineto{\pgfqpoint{8.476960in}{2.842157in}}%
\pgfusepath{stroke}%
\end{pgfscope}%
\begin{pgfscope}%
\pgfpathrectangle{\pgfqpoint{6.720588in}{1.750000in}}{\pgfqpoint{2.279412in}{2.004545in}}%
\pgfusepath{clip}%
\pgfsetbuttcap%
\pgfsetroundjoin%
\pgfsetlinewidth{0.336164pt}%
\definecolor{currentstroke}{rgb}{0.273809,0.031497,0.358853}%
\pgfsetstrokecolor{currentstroke}%
\pgfsetdash{}{0pt}%
\pgfpathmoveto{\pgfqpoint{8.476960in}{2.842157in}}%
\pgfpathlineto{\pgfqpoint{8.426825in}{2.841118in}}%
\pgfusepath{stroke}%
\end{pgfscope}%
\begin{pgfscope}%
\pgfpathrectangle{\pgfqpoint{6.720588in}{1.750000in}}{\pgfqpoint{2.279412in}{2.004545in}}%
\pgfusepath{clip}%
\pgfsetbuttcap%
\pgfsetroundjoin%
\pgfsetlinewidth{0.338366pt}%
\definecolor{currentstroke}{rgb}{0.273809,0.031497,0.358853}%
\pgfsetstrokecolor{currentstroke}%
\pgfsetdash{}{0pt}%
\pgfpathmoveto{\pgfqpoint{8.426825in}{2.841118in}}%
\pgfpathlineto{\pgfqpoint{8.376682in}{2.840682in}}%
\pgfusepath{stroke}%
\end{pgfscope}%
\begin{pgfscope}%
\pgfpathrectangle{\pgfqpoint{6.720588in}{1.750000in}}{\pgfqpoint{2.279412in}{2.004545in}}%
\pgfusepath{clip}%
\pgfsetbuttcap%
\pgfsetroundjoin%
\pgfsetlinewidth{0.364067pt}%
\definecolor{currentstroke}{rgb}{0.277941,0.056324,0.381191}%
\pgfsetstrokecolor{currentstroke}%
\pgfsetdash{}{0pt}%
\pgfpathmoveto{\pgfqpoint{8.376682in}{2.840682in}}%
\pgfpathlineto{\pgfqpoint{8.326533in}{2.840581in}}%
\pgfusepath{stroke}%
\end{pgfscope}%
\begin{pgfscope}%
\pgfpathrectangle{\pgfqpoint{6.720588in}{1.750000in}}{\pgfqpoint{2.279412in}{2.004545in}}%
\pgfusepath{clip}%
\pgfsetbuttcap%
\pgfsetroundjoin%
\pgfsetlinewidth{0.406574pt}%
\definecolor{currentstroke}{rgb}{0.281924,0.089666,0.412415}%
\pgfsetstrokecolor{currentstroke}%
\pgfsetdash{}{0pt}%
\pgfpathmoveto{\pgfqpoint{8.326533in}{2.840581in}}%
\pgfpathlineto{\pgfqpoint{8.276385in}{2.840134in}}%
\pgfusepath{stroke}%
\end{pgfscope}%
\begin{pgfscope}%
\pgfpathrectangle{\pgfqpoint{6.720588in}{1.750000in}}{\pgfqpoint{2.279412in}{2.004545in}}%
\pgfusepath{clip}%
\pgfsetbuttcap%
\pgfsetroundjoin%
\pgfsetlinewidth{0.445556pt}%
\definecolor{currentstroke}{rgb}{0.283229,0.120777,0.440584}%
\pgfsetstrokecolor{currentstroke}%
\pgfsetdash{}{0pt}%
\pgfpathmoveto{\pgfqpoint{8.276385in}{2.840134in}}%
\pgfpathlineto{\pgfqpoint{8.226236in}{2.839656in}}%
\pgfusepath{stroke}%
\end{pgfscope}%
\begin{pgfscope}%
\pgfpathrectangle{\pgfqpoint{6.720588in}{1.750000in}}{\pgfqpoint{2.279412in}{2.004545in}}%
\pgfusepath{clip}%
\pgfsetbuttcap%
\pgfsetroundjoin%
\pgfsetlinewidth{0.505432pt}%
\definecolor{currentstroke}{rgb}{0.280868,0.160771,0.472899}%
\pgfsetstrokecolor{currentstroke}%
\pgfsetdash{}{0pt}%
\pgfpathmoveto{\pgfqpoint{8.226236in}{2.839656in}}%
\pgfpathlineto{\pgfqpoint{8.176086in}{2.839254in}}%
\pgfusepath{stroke}%
\end{pgfscope}%
\begin{pgfscope}%
\pgfpathrectangle{\pgfqpoint{6.720588in}{1.750000in}}{\pgfqpoint{2.279412in}{2.004545in}}%
\pgfusepath{clip}%
\pgfsetbuttcap%
\pgfsetroundjoin%
\pgfsetlinewidth{0.590069pt}%
\definecolor{currentstroke}{rgb}{0.267968,0.223549,0.512008}%
\pgfsetstrokecolor{currentstroke}%
\pgfsetdash{}{0pt}%
\pgfpathmoveto{\pgfqpoint{8.176086in}{2.839254in}}%
\pgfpathlineto{\pgfqpoint{8.125942in}{2.838592in}}%
\pgfusepath{stroke}%
\end{pgfscope}%
\begin{pgfscope}%
\pgfpathrectangle{\pgfqpoint{6.720588in}{1.750000in}}{\pgfqpoint{2.279412in}{2.004545in}}%
\pgfusepath{clip}%
\pgfsetbuttcap%
\pgfsetroundjoin%
\pgfsetlinewidth{0.672618pt}%
\definecolor{currentstroke}{rgb}{0.248629,0.278775,0.534556}%
\pgfsetstrokecolor{currentstroke}%
\pgfsetdash{}{0pt}%
\pgfpathmoveto{\pgfqpoint{8.125942in}{2.838592in}}%
\pgfpathlineto{\pgfqpoint{8.075803in}{2.837568in}}%
\pgfusepath{stroke}%
\end{pgfscope}%
\begin{pgfscope}%
\pgfpathrectangle{\pgfqpoint{6.720588in}{1.750000in}}{\pgfqpoint{2.279412in}{2.004545in}}%
\pgfusepath{clip}%
\pgfsetbuttcap%
\pgfsetroundjoin%
\pgfsetlinewidth{0.712869pt}%
\definecolor{currentstroke}{rgb}{0.237441,0.305202,0.541921}%
\pgfsetstrokecolor{currentstroke}%
\pgfsetdash{}{0pt}%
\pgfpathmoveto{\pgfqpoint{8.075803in}{2.837568in}}%
\pgfpathlineto{\pgfqpoint{8.025671in}{2.836357in}}%
\pgfusepath{stroke}%
\end{pgfscope}%
\begin{pgfscope}%
\pgfpathrectangle{\pgfqpoint{6.720588in}{1.750000in}}{\pgfqpoint{2.279412in}{2.004545in}}%
\pgfusepath{clip}%
\pgfsetbuttcap%
\pgfsetroundjoin%
\pgfsetlinewidth{0.771943pt}%
\definecolor{currentstroke}{rgb}{0.221989,0.339161,0.548752}%
\pgfsetstrokecolor{currentstroke}%
\pgfsetdash{}{0pt}%
\pgfpathmoveto{\pgfqpoint{8.025671in}{2.836357in}}%
\pgfpathlineto{\pgfqpoint{7.975551in}{2.834817in}}%
\pgfusepath{stroke}%
\end{pgfscope}%
\begin{pgfscope}%
\pgfpathrectangle{\pgfqpoint{6.720588in}{1.750000in}}{\pgfqpoint{2.279412in}{2.004545in}}%
\pgfusepath{clip}%
\pgfsetbuttcap%
\pgfsetroundjoin%
\pgfsetlinewidth{0.806332pt}%
\definecolor{currentstroke}{rgb}{0.212395,0.359683,0.551710}%
\pgfsetstrokecolor{currentstroke}%
\pgfsetdash{}{0pt}%
\pgfpathmoveto{\pgfqpoint{7.975551in}{2.834817in}}%
\pgfpathlineto{\pgfqpoint{7.925453in}{2.832809in}}%
\pgfusepath{stroke}%
\end{pgfscope}%
\begin{pgfscope}%
\pgfpathrectangle{\pgfqpoint{6.720588in}{1.750000in}}{\pgfqpoint{2.279412in}{2.004545in}}%
\pgfusepath{clip}%
\pgfsetbuttcap%
\pgfsetroundjoin%
\pgfsetlinewidth{0.802518pt}%
\definecolor{currentstroke}{rgb}{0.212395,0.359683,0.551710}%
\pgfsetstrokecolor{currentstroke}%
\pgfsetdash{}{0pt}%
\pgfpathmoveto{\pgfqpoint{7.925453in}{2.832809in}}%
\pgfpathlineto{\pgfqpoint{7.875390in}{2.830212in}}%
\pgfusepath{stroke}%
\end{pgfscope}%
\begin{pgfscope}%
\pgfpathrectangle{\pgfqpoint{6.720588in}{1.750000in}}{\pgfqpoint{2.279412in}{2.004545in}}%
\pgfusepath{clip}%
\pgfsetbuttcap%
\pgfsetroundjoin%
\pgfsetlinewidth{0.720044pt}%
\definecolor{currentstroke}{rgb}{0.235526,0.309527,0.542944}%
\pgfsetstrokecolor{currentstroke}%
\pgfsetdash{}{0pt}%
\pgfpathmoveto{\pgfqpoint{7.875390in}{2.830212in}}%
\pgfpathlineto{\pgfqpoint{7.825377in}{2.826968in}}%
\pgfusepath{stroke}%
\end{pgfscope}%
\begin{pgfscope}%
\pgfpathrectangle{\pgfqpoint{6.720588in}{1.750000in}}{\pgfqpoint{2.279412in}{2.004545in}}%
\pgfusepath{clip}%
\pgfsetbuttcap%
\pgfsetroundjoin%
\pgfsetlinewidth{0.767860pt}%
\definecolor{currentstroke}{rgb}{0.221989,0.339161,0.548752}%
\pgfsetstrokecolor{currentstroke}%
\pgfsetdash{}{0pt}%
\pgfpathmoveto{\pgfqpoint{7.825377in}{2.826968in}}%
\pgfpathlineto{\pgfqpoint{7.775427in}{2.823065in}}%
\pgfusepath{stroke}%
\end{pgfscope}%
\begin{pgfscope}%
\pgfpathrectangle{\pgfqpoint{6.720588in}{1.750000in}}{\pgfqpoint{2.279412in}{2.004545in}}%
\pgfusepath{clip}%
\pgfsetbuttcap%
\pgfsetroundjoin%
\pgfsetlinewidth{0.311866pt}%
\definecolor{currentstroke}{rgb}{0.268510,0.009605,0.335427}%
\pgfsetstrokecolor{currentstroke}%
\pgfsetdash{}{0pt}%
\pgfpathmoveto{\pgfqpoint{8.527089in}{3.158234in}}%
\pgfpathlineto{\pgfqpoint{8.477399in}{3.156007in}}%
\pgfusepath{stroke}%
\end{pgfscope}%
\begin{pgfscope}%
\pgfpathrectangle{\pgfqpoint{6.720588in}{1.750000in}}{\pgfqpoint{2.279412in}{2.004545in}}%
\pgfusepath{clip}%
\pgfsetbuttcap%
\pgfsetroundjoin%
\pgfsetlinewidth{0.317930pt}%
\definecolor{currentstroke}{rgb}{0.269944,0.014625,0.341379}%
\pgfsetstrokecolor{currentstroke}%
\pgfsetdash{}{0pt}%
\pgfpathmoveto{\pgfqpoint{8.477399in}{3.156007in}}%
\pgfpathlineto{\pgfqpoint{8.427300in}{3.155065in}}%
\pgfusepath{stroke}%
\end{pgfscope}%
\begin{pgfscope}%
\pgfpathrectangle{\pgfqpoint{6.720588in}{1.750000in}}{\pgfqpoint{2.279412in}{2.004545in}}%
\pgfusepath{clip}%
\pgfsetbuttcap%
\pgfsetroundjoin%
\pgfsetlinewidth{0.330865pt}%
\definecolor{currentstroke}{rgb}{0.272594,0.025563,0.353093}%
\pgfsetstrokecolor{currentstroke}%
\pgfsetdash{}{0pt}%
\pgfpathmoveto{\pgfqpoint{8.427300in}{3.155065in}}%
\pgfpathlineto{\pgfqpoint{8.377161in}{3.154251in}}%
\pgfusepath{stroke}%
\end{pgfscope}%
\begin{pgfscope}%
\pgfpathrectangle{\pgfqpoint{6.720588in}{1.750000in}}{\pgfqpoint{2.279412in}{2.004545in}}%
\pgfusepath{clip}%
\pgfsetbuttcap%
\pgfsetroundjoin%
\pgfsetlinewidth{0.357139pt}%
\definecolor{currentstroke}{rgb}{0.277018,0.050344,0.375715}%
\pgfsetstrokecolor{currentstroke}%
\pgfsetdash{}{0pt}%
\pgfpathmoveto{\pgfqpoint{8.377161in}{3.154251in}}%
\pgfpathlineto{\pgfqpoint{8.327069in}{3.152319in}}%
\pgfusepath{stroke}%
\end{pgfscope}%
\begin{pgfscope}%
\pgfpathrectangle{\pgfqpoint{6.720588in}{1.750000in}}{\pgfqpoint{2.279412in}{2.004545in}}%
\pgfusepath{clip}%
\pgfsetbuttcap%
\pgfsetroundjoin%
\pgfsetlinewidth{0.368325pt}%
\definecolor{currentstroke}{rgb}{0.277941,0.056324,0.381191}%
\pgfsetstrokecolor{currentstroke}%
\pgfsetdash{}{0pt}%
\pgfpathmoveto{\pgfqpoint{8.327069in}{3.152319in}}%
\pgfpathlineto{\pgfqpoint{8.277011in}{3.149689in}}%
\pgfusepath{stroke}%
\end{pgfscope}%
\begin{pgfscope}%
\pgfpathrectangle{\pgfqpoint{6.720588in}{1.750000in}}{\pgfqpoint{2.279412in}{2.004545in}}%
\pgfusepath{clip}%
\pgfsetbuttcap%
\pgfsetroundjoin%
\pgfsetlinewidth{0.371214pt}%
\definecolor{currentstroke}{rgb}{0.278791,0.062145,0.386592}%
\pgfsetstrokecolor{currentstroke}%
\pgfsetdash{}{0pt}%
\pgfpathmoveto{\pgfqpoint{8.277011in}{3.149689in}}%
\pgfpathlineto{\pgfqpoint{8.226956in}{3.146991in}}%
\pgfusepath{stroke}%
\end{pgfscope}%
\begin{pgfscope}%
\pgfpathrectangle{\pgfqpoint{6.720588in}{1.750000in}}{\pgfqpoint{2.279412in}{2.004545in}}%
\pgfusepath{clip}%
\pgfsetbuttcap%
\pgfsetroundjoin%
\pgfsetlinewidth{0.390003pt}%
\definecolor{currentstroke}{rgb}{0.280267,0.073417,0.397163}%
\pgfsetstrokecolor{currentstroke}%
\pgfsetdash{}{0pt}%
\pgfpathmoveto{\pgfqpoint{8.226956in}{3.146991in}}%
\pgfpathlineto{\pgfqpoint{8.176917in}{3.144022in}}%
\pgfusepath{stroke}%
\end{pgfscope}%
\begin{pgfscope}%
\pgfpathrectangle{\pgfqpoint{6.720588in}{1.750000in}}{\pgfqpoint{2.279412in}{2.004545in}}%
\pgfusepath{clip}%
\pgfsetbuttcap%
\pgfsetroundjoin%
\pgfsetlinewidth{0.403495pt}%
\definecolor{currentstroke}{rgb}{0.281446,0.084320,0.407414}%
\pgfsetstrokecolor{currentstroke}%
\pgfsetdash{}{0pt}%
\pgfpathmoveto{\pgfqpoint{8.176917in}{3.144022in}}%
\pgfpathlineto{\pgfqpoint{8.126944in}{3.140397in}}%
\pgfusepath{stroke}%
\end{pgfscope}%
\begin{pgfscope}%
\pgfpathrectangle{\pgfqpoint{6.720588in}{1.750000in}}{\pgfqpoint{2.279412in}{2.004545in}}%
\pgfusepath{clip}%
\pgfsetbuttcap%
\pgfsetroundjoin%
\pgfsetlinewidth{0.423719pt}%
\definecolor{currentstroke}{rgb}{0.282656,0.100196,0.422160}%
\pgfsetstrokecolor{currentstroke}%
\pgfsetdash{}{0pt}%
\pgfpathmoveto{\pgfqpoint{8.126944in}{3.140397in}}%
\pgfpathlineto{\pgfqpoint{8.077103in}{3.135596in}}%
\pgfusepath{stroke}%
\end{pgfscope}%
\begin{pgfscope}%
\pgfpathrectangle{\pgfqpoint{6.720588in}{1.750000in}}{\pgfqpoint{2.279412in}{2.004545in}}%
\pgfusepath{clip}%
\pgfsetbuttcap%
\pgfsetroundjoin%
\pgfsetlinewidth{0.435522pt}%
\definecolor{currentstroke}{rgb}{0.283091,0.110553,0.431554}%
\pgfsetstrokecolor{currentstroke}%
\pgfsetdash{}{0pt}%
\pgfpathmoveto{\pgfqpoint{8.077103in}{3.135596in}}%
\pgfpathlineto{\pgfqpoint{8.027516in}{3.129098in}}%
\pgfusepath{stroke}%
\end{pgfscope}%
\begin{pgfscope}%
\pgfpathrectangle{\pgfqpoint{6.720588in}{1.750000in}}{\pgfqpoint{2.279412in}{2.004545in}}%
\pgfusepath{clip}%
\pgfsetbuttcap%
\pgfsetroundjoin%
\pgfsetlinewidth{0.445779pt}%
\definecolor{currentstroke}{rgb}{0.283229,0.120777,0.440584}%
\pgfsetstrokecolor{currentstroke}%
\pgfsetdash{}{0pt}%
\pgfpathmoveto{\pgfqpoint{8.027516in}{3.129098in}}%
\pgfpathlineto{\pgfqpoint{7.978157in}{3.121308in}}%
\pgfusepath{stroke}%
\end{pgfscope}%
\begin{pgfscope}%
\pgfpathrectangle{\pgfqpoint{6.720588in}{1.750000in}}{\pgfqpoint{2.279412in}{2.004545in}}%
\pgfusepath{clip}%
\pgfsetbuttcap%
\pgfsetroundjoin%
\pgfsetlinewidth{0.449660pt}%
\definecolor{currentstroke}{rgb}{0.283229,0.120777,0.440584}%
\pgfsetstrokecolor{currentstroke}%
\pgfsetdash{}{0pt}%
\pgfpathmoveto{\pgfqpoint{7.978157in}{3.121308in}}%
\pgfpathlineto{\pgfqpoint{7.929017in}{3.112535in}}%
\pgfusepath{stroke}%
\end{pgfscope}%
\begin{pgfscope}%
\pgfpathrectangle{\pgfqpoint{6.720588in}{1.750000in}}{\pgfqpoint{2.279412in}{2.004545in}}%
\pgfusepath{clip}%
\pgfsetbuttcap%
\pgfsetroundjoin%
\pgfsetlinewidth{0.454368pt}%
\definecolor{currentstroke}{rgb}{0.283187,0.125848,0.444960}%
\pgfsetstrokecolor{currentstroke}%
\pgfsetdash{}{0pt}%
\pgfpathmoveto{\pgfqpoint{7.929017in}{3.112535in}}%
\pgfpathlineto{\pgfqpoint{7.880508in}{3.101484in}}%
\pgfusepath{stroke}%
\end{pgfscope}%
\begin{pgfscope}%
\pgfpathrectangle{\pgfqpoint{6.720588in}{1.750000in}}{\pgfqpoint{2.279412in}{2.004545in}}%
\pgfusepath{clip}%
\pgfsetbuttcap%
\pgfsetroundjoin%
\pgfsetlinewidth{0.481662pt}%
\definecolor{currentstroke}{rgb}{0.282290,0.145912,0.461510}%
\pgfsetstrokecolor{currentstroke}%
\pgfsetdash{}{0pt}%
\pgfpathmoveto{\pgfqpoint{7.880508in}{3.101484in}}%
\pgfpathlineto{\pgfqpoint{7.832912in}{3.087784in}}%
\pgfusepath{stroke}%
\end{pgfscope}%
\begin{pgfscope}%
\pgfpathrectangle{\pgfqpoint{6.720588in}{1.750000in}}{\pgfqpoint{2.279412in}{2.004545in}}%
\pgfusepath{clip}%
\pgfsetbuttcap%
\pgfsetroundjoin%
\pgfsetlinewidth{0.480220pt}%
\definecolor{currentstroke}{rgb}{0.282290,0.145912,0.461510}%
\pgfsetstrokecolor{currentstroke}%
\pgfsetdash{}{0pt}%
\pgfpathmoveto{\pgfqpoint{7.832912in}{3.087784in}}%
\pgfpathlineto{\pgfqpoint{7.787938in}{3.068903in}}%
\pgfusepath{stroke}%
\end{pgfscope}%
\begin{pgfscope}%
\pgfpathrectangle{\pgfqpoint{6.720588in}{1.750000in}}{\pgfqpoint{2.279412in}{2.004545in}}%
\pgfusepath{clip}%
\pgfsetbuttcap%
\pgfsetroundjoin%
\pgfsetlinewidth{0.434898pt}%
\definecolor{currentstroke}{rgb}{0.283091,0.110553,0.431554}%
\pgfsetstrokecolor{currentstroke}%
\pgfsetdash{}{0pt}%
\pgfpathmoveto{\pgfqpoint{7.787938in}{3.068903in}}%
\pgfpathlineto{\pgfqpoint{7.746884in}{3.045114in}}%
\pgfusepath{stroke}%
\end{pgfscope}%
\begin{pgfscope}%
\pgfpathrectangle{\pgfqpoint{6.720588in}{1.750000in}}{\pgfqpoint{2.279412in}{2.004545in}}%
\pgfusepath{clip}%
\pgfsetbuttcap%
\pgfsetroundjoin%
\pgfsetlinewidth{0.501558pt}%
\definecolor{currentstroke}{rgb}{0.280868,0.160771,0.472899}%
\pgfsetstrokecolor{currentstroke}%
\pgfsetdash{}{0pt}%
\pgfpathmoveto{\pgfqpoint{7.746884in}{3.045114in}}%
\pgfpathlineto{\pgfqpoint{7.709053in}{3.016577in}}%
\pgfusepath{stroke}%
\end{pgfscope}%
\begin{pgfscope}%
\pgfpathrectangle{\pgfqpoint{6.720588in}{1.750000in}}{\pgfqpoint{2.279412in}{2.004545in}}%
\pgfusepath{clip}%
\pgfsetbuttcap%
\pgfsetroundjoin%
\pgfsetlinewidth{0.532767pt}%
\definecolor{currentstroke}{rgb}{0.278012,0.180367,0.486697}%
\pgfsetstrokecolor{currentstroke}%
\pgfsetdash{}{0pt}%
\pgfpathmoveto{\pgfqpoint{7.709053in}{3.016577in}}%
\pgfpathlineto{\pgfqpoint{7.672799in}{2.986383in}}%
\pgfusepath{stroke}%
\end{pgfscope}%
\begin{pgfscope}%
\pgfpathrectangle{\pgfqpoint{6.720588in}{1.750000in}}{\pgfqpoint{2.279412in}{2.004545in}}%
\pgfusepath{clip}%
\pgfsetbuttcap%
\pgfsetroundjoin%
\pgfsetlinewidth{0.558586pt}%
\definecolor{currentstroke}{rgb}{0.274128,0.199721,0.498911}%
\pgfsetstrokecolor{currentstroke}%
\pgfsetdash{}{0pt}%
\pgfpathmoveto{\pgfqpoint{7.672799in}{2.986383in}}%
\pgfpathlineto{\pgfqpoint{7.636854in}{2.955803in}}%
\pgfusepath{stroke}%
\end{pgfscope}%
\begin{pgfscope}%
\pgfpathrectangle{\pgfqpoint{6.720588in}{1.750000in}}{\pgfqpoint{2.279412in}{2.004545in}}%
\pgfusepath{clip}%
\pgfsetbuttcap%
\pgfsetroundjoin%
\pgfsetlinewidth{0.600432pt}%
\definecolor{currentstroke}{rgb}{0.266580,0.228262,0.514349}%
\pgfsetstrokecolor{currentstroke}%
\pgfsetdash{}{0pt}%
\pgfpathmoveto{\pgfqpoint{7.636854in}{2.955803in}}%
\pgfpathlineto{\pgfqpoint{7.601403in}{2.924697in}}%
\pgfusepath{stroke}%
\end{pgfscope}%
\begin{pgfscope}%
\pgfpathrectangle{\pgfqpoint{6.720588in}{1.750000in}}{\pgfqpoint{2.279412in}{2.004545in}}%
\pgfusepath{clip}%
\pgfsetbuttcap%
\pgfsetroundjoin%
\pgfsetlinewidth{0.326548pt}%
\definecolor{currentstroke}{rgb}{0.271305,0.019942,0.347269}%
\pgfsetstrokecolor{currentstroke}%
\pgfsetdash{}{0pt}%
\pgfpathmoveto{\pgfqpoint{8.517200in}{2.194357in}}%
\pgfpathlineto{\pgfqpoint{8.467060in}{2.194781in}}%
\pgfusepath{stroke}%
\end{pgfscope}%
\begin{pgfscope}%
\pgfpathrectangle{\pgfqpoint{6.720588in}{1.750000in}}{\pgfqpoint{2.279412in}{2.004545in}}%
\pgfusepath{clip}%
\pgfsetbuttcap%
\pgfsetroundjoin%
\pgfsetlinewidth{0.324652pt}%
\definecolor{currentstroke}{rgb}{0.271305,0.019942,0.347269}%
\pgfsetstrokecolor{currentstroke}%
\pgfsetdash{}{0pt}%
\pgfpathmoveto{\pgfqpoint{8.467060in}{2.194781in}}%
\pgfpathlineto{\pgfqpoint{8.416915in}{2.194723in}}%
\pgfusepath{stroke}%
\end{pgfscope}%
\begin{pgfscope}%
\pgfpathrectangle{\pgfqpoint{6.720588in}{1.750000in}}{\pgfqpoint{2.279412in}{2.004545in}}%
\pgfusepath{clip}%
\pgfsetbuttcap%
\pgfsetroundjoin%
\pgfsetlinewidth{0.320957pt}%
\definecolor{currentstroke}{rgb}{0.269944,0.014625,0.341379}%
\pgfsetstrokecolor{currentstroke}%
\pgfsetdash{}{0pt}%
\pgfpathmoveto{\pgfqpoint{8.416915in}{2.194723in}}%
\pgfpathlineto{\pgfqpoint{8.366797in}{2.195851in}}%
\pgfusepath{stroke}%
\end{pgfscope}%
\begin{pgfscope}%
\pgfpathrectangle{\pgfqpoint{6.720588in}{1.750000in}}{\pgfqpoint{2.279412in}{2.004545in}}%
\pgfusepath{clip}%
\pgfsetbuttcap%
\pgfsetroundjoin%
\pgfsetlinewidth{0.333149pt}%
\definecolor{currentstroke}{rgb}{0.272594,0.025563,0.353093}%
\pgfsetstrokecolor{currentstroke}%
\pgfsetdash{}{0pt}%
\pgfpathmoveto{\pgfqpoint{8.366797in}{2.195851in}}%
\pgfpathlineto{\pgfqpoint{8.316720in}{2.198208in}}%
\pgfusepath{stroke}%
\end{pgfscope}%
\begin{pgfscope}%
\pgfpathrectangle{\pgfqpoint{6.720588in}{1.750000in}}{\pgfqpoint{2.279412in}{2.004545in}}%
\pgfusepath{clip}%
\pgfsetbuttcap%
\pgfsetroundjoin%
\pgfsetlinewidth{0.344053pt}%
\definecolor{currentstroke}{rgb}{0.274952,0.037752,0.364543}%
\pgfsetstrokecolor{currentstroke}%
\pgfsetdash{}{0pt}%
\pgfpathmoveto{\pgfqpoint{8.316720in}{2.198208in}}%
\pgfpathlineto{\pgfqpoint{8.266635in}{2.200436in}}%
\pgfusepath{stroke}%
\end{pgfscope}%
\begin{pgfscope}%
\pgfpathrectangle{\pgfqpoint{6.720588in}{1.750000in}}{\pgfqpoint{2.279412in}{2.004545in}}%
\pgfusepath{clip}%
\pgfsetbuttcap%
\pgfsetroundjoin%
\pgfsetlinewidth{0.345284pt}%
\definecolor{currentstroke}{rgb}{0.274952,0.037752,0.364543}%
\pgfsetstrokecolor{currentstroke}%
\pgfsetdash{}{0pt}%
\pgfpathmoveto{\pgfqpoint{8.266635in}{2.200436in}}%
\pgfpathlineto{\pgfqpoint{8.216566in}{2.202832in}}%
\pgfusepath{stroke}%
\end{pgfscope}%
\begin{pgfscope}%
\pgfpathrectangle{\pgfqpoint{6.720588in}{1.750000in}}{\pgfqpoint{2.279412in}{2.004545in}}%
\pgfusepath{clip}%
\pgfsetbuttcap%
\pgfsetroundjoin%
\pgfsetlinewidth{0.357384pt}%
\definecolor{currentstroke}{rgb}{0.277018,0.050344,0.375715}%
\pgfsetstrokecolor{currentstroke}%
\pgfsetdash{}{0pt}%
\pgfpathmoveto{\pgfqpoint{8.216566in}{2.202832in}}%
\pgfpathlineto{\pgfqpoint{8.166555in}{2.206059in}}%
\pgfusepath{stroke}%
\end{pgfscope}%
\begin{pgfscope}%
\pgfpathrectangle{\pgfqpoint{6.720588in}{1.750000in}}{\pgfqpoint{2.279412in}{2.004545in}}%
\pgfusepath{clip}%
\pgfsetbuttcap%
\pgfsetroundjoin%
\pgfsetlinewidth{0.376231pt}%
\definecolor{currentstroke}{rgb}{0.278791,0.062145,0.386592}%
\pgfsetstrokecolor{currentstroke}%
\pgfsetdash{}{0pt}%
\pgfpathmoveto{\pgfqpoint{8.166555in}{2.206059in}}%
\pgfpathlineto{\pgfqpoint{8.116754in}{2.210991in}}%
\pgfusepath{stroke}%
\end{pgfscope}%
\begin{pgfscope}%
\pgfpathrectangle{\pgfqpoint{6.720588in}{1.750000in}}{\pgfqpoint{2.279412in}{2.004545in}}%
\pgfusepath{clip}%
\pgfsetbuttcap%
\pgfsetroundjoin%
\pgfsetlinewidth{0.328570pt}%
\definecolor{currentstroke}{rgb}{0.271305,0.019942,0.347269}%
\pgfsetstrokecolor{currentstroke}%
\pgfsetdash{}{0pt}%
\pgfpathmoveto{\pgfqpoint{8.475797in}{2.481632in}}%
\pgfpathlineto{\pgfqpoint{8.425651in}{2.482136in}}%
\pgfusepath{stroke}%
\end{pgfscope}%
\begin{pgfscope}%
\pgfpathrectangle{\pgfqpoint{6.720588in}{1.750000in}}{\pgfqpoint{2.279412in}{2.004545in}}%
\pgfusepath{clip}%
\pgfsetbuttcap%
\pgfsetroundjoin%
\pgfsetlinewidth{0.338940pt}%
\definecolor{currentstroke}{rgb}{0.273809,0.031497,0.358853}%
\pgfsetstrokecolor{currentstroke}%
\pgfsetdash{}{0pt}%
\pgfpathmoveto{\pgfqpoint{8.425651in}{2.482136in}}%
\pgfpathlineto{\pgfqpoint{8.375509in}{2.482847in}}%
\pgfusepath{stroke}%
\end{pgfscope}%
\begin{pgfscope}%
\pgfpathrectangle{\pgfqpoint{6.720588in}{1.750000in}}{\pgfqpoint{2.279412in}{2.004545in}}%
\pgfusepath{clip}%
\pgfsetbuttcap%
\pgfsetroundjoin%
\pgfsetlinewidth{0.357824pt}%
\definecolor{currentstroke}{rgb}{0.277018,0.050344,0.375715}%
\pgfsetstrokecolor{currentstroke}%
\pgfsetdash{}{0pt}%
\pgfpathmoveto{\pgfqpoint{8.375509in}{2.482847in}}%
\pgfpathlineto{\pgfqpoint{8.325374in}{2.483907in}}%
\pgfusepath{stroke}%
\end{pgfscope}%
\begin{pgfscope}%
\pgfpathrectangle{\pgfqpoint{6.720588in}{1.750000in}}{\pgfqpoint{2.279412in}{2.004545in}}%
\pgfusepath{clip}%
\pgfsetbuttcap%
\pgfsetroundjoin%
\pgfsetlinewidth{0.382933pt}%
\definecolor{currentstroke}{rgb}{0.279566,0.067836,0.391917}%
\pgfsetstrokecolor{currentstroke}%
\pgfsetdash{}{0pt}%
\pgfpathmoveto{\pgfqpoint{8.325374in}{2.483907in}}%
\pgfpathlineto{\pgfqpoint{8.275242in}{2.485106in}}%
\pgfusepath{stroke}%
\end{pgfscope}%
\begin{pgfscope}%
\pgfpathrectangle{\pgfqpoint{6.720588in}{1.750000in}}{\pgfqpoint{2.279412in}{2.004545in}}%
\pgfusepath{clip}%
\pgfsetbuttcap%
\pgfsetroundjoin%
\pgfsetlinewidth{0.423402pt}%
\definecolor{currentstroke}{rgb}{0.282656,0.100196,0.422160}%
\pgfsetstrokecolor{currentstroke}%
\pgfsetdash{}{0pt}%
\pgfpathmoveto{\pgfqpoint{8.275242in}{2.485106in}}%
\pgfpathlineto{\pgfqpoint{8.225121in}{2.486637in}}%
\pgfusepath{stroke}%
\end{pgfscope}%
\begin{pgfscope}%
\pgfpathrectangle{\pgfqpoint{6.720588in}{1.750000in}}{\pgfqpoint{2.279412in}{2.004545in}}%
\pgfusepath{clip}%
\pgfsetbuttcap%
\pgfsetroundjoin%
\pgfsetlinewidth{0.460496pt}%
\definecolor{currentstroke}{rgb}{0.283072,0.130895,0.449241}%
\pgfsetstrokecolor{currentstroke}%
\pgfsetdash{}{0pt}%
\pgfpathmoveto{\pgfqpoint{8.225121in}{2.486637in}}%
\pgfpathlineto{\pgfqpoint{8.175024in}{2.488671in}}%
\pgfusepath{stroke}%
\end{pgfscope}%
\begin{pgfscope}%
\pgfpathrectangle{\pgfqpoint{6.720588in}{1.750000in}}{\pgfqpoint{2.279412in}{2.004545in}}%
\pgfusepath{clip}%
\pgfsetbuttcap%
\pgfsetroundjoin%
\pgfsetlinewidth{0.518177pt}%
\definecolor{currentstroke}{rgb}{0.279574,0.170599,0.479997}%
\pgfsetstrokecolor{currentstroke}%
\pgfsetdash{}{0pt}%
\pgfpathmoveto{\pgfqpoint{8.175024in}{2.488671in}}%
\pgfpathlineto{\pgfqpoint{8.124950in}{2.491121in}}%
\pgfusepath{stroke}%
\end{pgfscope}%
\begin{pgfscope}%
\pgfpathrectangle{\pgfqpoint{6.720588in}{1.750000in}}{\pgfqpoint{2.279412in}{2.004545in}}%
\pgfusepath{clip}%
\pgfsetbuttcap%
\pgfsetroundjoin%
\pgfsetlinewidth{0.560406pt}%
\definecolor{currentstroke}{rgb}{0.274128,0.199721,0.498911}%
\pgfsetstrokecolor{currentstroke}%
\pgfsetdash{}{0pt}%
\pgfpathmoveto{\pgfqpoint{8.124950in}{2.491121in}}%
\pgfpathlineto{\pgfqpoint{8.074908in}{2.494013in}}%
\pgfusepath{stroke}%
\end{pgfscope}%
\begin{pgfscope}%
\pgfpathrectangle{\pgfqpoint{6.720588in}{1.750000in}}{\pgfqpoint{2.279412in}{2.004545in}}%
\pgfusepath{clip}%
\pgfsetbuttcap%
\pgfsetroundjoin%
\pgfsetlinewidth{0.601973pt}%
\definecolor{currentstroke}{rgb}{0.266580,0.228262,0.514349}%
\pgfsetstrokecolor{currentstroke}%
\pgfsetdash{}{0pt}%
\pgfpathmoveto{\pgfqpoint{8.074908in}{2.494013in}}%
\pgfpathlineto{\pgfqpoint{8.024944in}{2.497793in}}%
\pgfusepath{stroke}%
\end{pgfscope}%
\begin{pgfscope}%
\pgfpathrectangle{\pgfqpoint{6.720588in}{1.750000in}}{\pgfqpoint{2.279412in}{2.004545in}}%
\pgfusepath{clip}%
\pgfsetbuttcap%
\pgfsetroundjoin%
\pgfsetlinewidth{0.322913pt}%
\definecolor{currentstroke}{rgb}{0.271305,0.019942,0.347269}%
\pgfsetstrokecolor{currentstroke}%
\pgfsetdash{}{0pt}%
\pgfpathmoveto{\pgfqpoint{8.475797in}{2.887593in}}%
\pgfpathlineto{\pgfqpoint{8.425653in}{2.887643in}}%
\pgfusepath{stroke}%
\end{pgfscope}%
\begin{pgfscope}%
\pgfpathrectangle{\pgfqpoint{6.720588in}{1.750000in}}{\pgfqpoint{2.279412in}{2.004545in}}%
\pgfusepath{clip}%
\pgfsetbuttcap%
\pgfsetroundjoin%
\pgfsetlinewidth{0.338614pt}%
\definecolor{currentstroke}{rgb}{0.273809,0.031497,0.358853}%
\pgfsetstrokecolor{currentstroke}%
\pgfsetdash{}{0pt}%
\pgfpathmoveto{\pgfqpoint{8.425653in}{2.887643in}}%
\pgfpathlineto{\pgfqpoint{8.375501in}{2.887738in}}%
\pgfusepath{stroke}%
\end{pgfscope}%
\begin{pgfscope}%
\pgfpathrectangle{\pgfqpoint{6.720588in}{1.750000in}}{\pgfqpoint{2.279412in}{2.004545in}}%
\pgfusepath{clip}%
\pgfsetbuttcap%
\pgfsetroundjoin%
\pgfsetlinewidth{0.372451pt}%
\definecolor{currentstroke}{rgb}{0.278791,0.062145,0.386592}%
\pgfsetstrokecolor{currentstroke}%
\pgfsetdash{}{0pt}%
\pgfpathmoveto{\pgfqpoint{8.375501in}{2.887738in}}%
\pgfpathlineto{\pgfqpoint{8.325352in}{2.887515in}}%
\pgfusepath{stroke}%
\end{pgfscope}%
\begin{pgfscope}%
\pgfpathrectangle{\pgfqpoint{6.720588in}{1.750000in}}{\pgfqpoint{2.279412in}{2.004545in}}%
\pgfusepath{clip}%
\pgfsetbuttcap%
\pgfsetroundjoin%
\pgfsetlinewidth{0.399119pt}%
\definecolor{currentstroke}{rgb}{0.281446,0.084320,0.407414}%
\pgfsetstrokecolor{currentstroke}%
\pgfsetdash{}{0pt}%
\pgfpathmoveto{\pgfqpoint{8.325352in}{2.887515in}}%
\pgfpathlineto{\pgfqpoint{8.275204in}{2.886985in}}%
\pgfusepath{stroke}%
\end{pgfscope}%
\begin{pgfscope}%
\pgfpathrectangle{\pgfqpoint{6.720588in}{1.750000in}}{\pgfqpoint{2.279412in}{2.004545in}}%
\pgfusepath{clip}%
\pgfsetbuttcap%
\pgfsetroundjoin%
\pgfsetlinewidth{0.432833pt}%
\definecolor{currentstroke}{rgb}{0.283091,0.110553,0.431554}%
\pgfsetstrokecolor{currentstroke}%
\pgfsetdash{}{0pt}%
\pgfpathmoveto{\pgfqpoint{8.275204in}{2.886985in}}%
\pgfpathlineto{\pgfqpoint{8.225059in}{2.886299in}}%
\pgfusepath{stroke}%
\end{pgfscope}%
\begin{pgfscope}%
\pgfpathrectangle{\pgfqpoint{6.720588in}{1.750000in}}{\pgfqpoint{2.279412in}{2.004545in}}%
\pgfusepath{clip}%
\pgfsetbuttcap%
\pgfsetroundjoin%
\pgfsetlinewidth{0.500666pt}%
\definecolor{currentstroke}{rgb}{0.280868,0.160771,0.472899}%
\pgfsetstrokecolor{currentstroke}%
\pgfsetdash{}{0pt}%
\pgfpathmoveto{\pgfqpoint{8.225059in}{2.886299in}}%
\pgfpathlineto{\pgfqpoint{8.174919in}{2.885323in}}%
\pgfusepath{stroke}%
\end{pgfscope}%
\begin{pgfscope}%
\pgfpathrectangle{\pgfqpoint{6.720588in}{1.750000in}}{\pgfqpoint{2.279412in}{2.004545in}}%
\pgfusepath{clip}%
\pgfsetbuttcap%
\pgfsetroundjoin%
\pgfsetlinewidth{0.559617pt}%
\definecolor{currentstroke}{rgb}{0.274128,0.199721,0.498911}%
\pgfsetstrokecolor{currentstroke}%
\pgfsetdash{}{0pt}%
\pgfpathmoveto{\pgfqpoint{8.174919in}{2.885323in}}%
\pgfpathlineto{\pgfqpoint{8.124790in}{2.884040in}}%
\pgfusepath{stroke}%
\end{pgfscope}%
\begin{pgfscope}%
\pgfpathrectangle{\pgfqpoint{6.720588in}{1.750000in}}{\pgfqpoint{2.279412in}{2.004545in}}%
\pgfusepath{clip}%
\pgfsetbuttcap%
\pgfsetroundjoin%
\pgfsetlinewidth{0.611546pt}%
\definecolor{currentstroke}{rgb}{0.263663,0.237631,0.518762}%
\pgfsetstrokecolor{currentstroke}%
\pgfsetdash{}{0pt}%
\pgfpathmoveto{\pgfqpoint{8.124790in}{2.884040in}}%
\pgfpathlineto{\pgfqpoint{8.074677in}{2.882316in}}%
\pgfusepath{stroke}%
\end{pgfscope}%
\begin{pgfscope}%
\pgfpathrectangle{\pgfqpoint{6.720588in}{1.750000in}}{\pgfqpoint{2.279412in}{2.004545in}}%
\pgfusepath{clip}%
\pgfsetbuttcap%
\pgfsetroundjoin%
\pgfsetlinewidth{0.716214pt}%
\definecolor{currentstroke}{rgb}{0.237441,0.305202,0.541921}%
\pgfsetstrokecolor{currentstroke}%
\pgfsetdash{}{0pt}%
\pgfpathmoveto{\pgfqpoint{8.074677in}{2.882316in}}%
\pgfpathlineto{\pgfqpoint{8.024588in}{2.880134in}}%
\pgfusepath{stroke}%
\end{pgfscope}%
\begin{pgfscope}%
\pgfpathrectangle{\pgfqpoint{6.720588in}{1.750000in}}{\pgfqpoint{2.279412in}{2.004545in}}%
\pgfusepath{clip}%
\pgfsetbuttcap%
\pgfsetroundjoin%
\pgfsetlinewidth{0.724459pt}%
\definecolor{currentstroke}{rgb}{0.235526,0.309527,0.542944}%
\pgfsetstrokecolor{currentstroke}%
\pgfsetdash{}{0pt}%
\pgfpathmoveto{\pgfqpoint{8.024588in}{2.880134in}}%
\pgfpathlineto{\pgfqpoint{7.974529in}{2.877471in}}%
\pgfusepath{stroke}%
\end{pgfscope}%
\begin{pgfscope}%
\pgfpathrectangle{\pgfqpoint{6.720588in}{1.750000in}}{\pgfqpoint{2.279412in}{2.004545in}}%
\pgfusepath{clip}%
\pgfsetbuttcap%
\pgfsetroundjoin%
\pgfsetlinewidth{0.757878pt}%
\definecolor{currentstroke}{rgb}{0.225863,0.330805,0.547314}%
\pgfsetstrokecolor{currentstroke}%
\pgfsetdash{}{0pt}%
\pgfpathmoveto{\pgfqpoint{7.974529in}{2.877471in}}%
\pgfpathlineto{\pgfqpoint{7.924536in}{2.874012in}}%
\pgfusepath{stroke}%
\end{pgfscope}%
\begin{pgfscope}%
\pgfpathrectangle{\pgfqpoint{6.720588in}{1.750000in}}{\pgfqpoint{2.279412in}{2.004545in}}%
\pgfusepath{clip}%
\pgfsetbuttcap%
\pgfsetroundjoin%
\pgfsetlinewidth{0.320612pt}%
\definecolor{currentstroke}{rgb}{0.269944,0.014625,0.341379}%
\pgfsetstrokecolor{currentstroke}%
\pgfsetdash{}{0pt}%
\pgfpathmoveto{\pgfqpoint{8.475797in}{3.113127in}}%
\pgfpathlineto{\pgfqpoint{8.425743in}{3.110506in}}%
\pgfusepath{stroke}%
\end{pgfscope}%
\begin{pgfscope}%
\pgfpathrectangle{\pgfqpoint{6.720588in}{1.750000in}}{\pgfqpoint{2.279412in}{2.004545in}}%
\pgfusepath{clip}%
\pgfsetbuttcap%
\pgfsetroundjoin%
\pgfsetlinewidth{0.327890pt}%
\definecolor{currentstroke}{rgb}{0.271305,0.019942,0.347269}%
\pgfsetstrokecolor{currentstroke}%
\pgfsetdash{}{0pt}%
\pgfpathmoveto{\pgfqpoint{8.425743in}{3.110506in}}%
\pgfpathlineto{\pgfqpoint{8.375663in}{3.108799in}}%
\pgfusepath{stroke}%
\end{pgfscope}%
\begin{pgfscope}%
\pgfpathrectangle{\pgfqpoint{6.720588in}{1.750000in}}{\pgfqpoint{2.279412in}{2.004545in}}%
\pgfusepath{clip}%
\pgfsetbuttcap%
\pgfsetroundjoin%
\pgfsetlinewidth{0.346370pt}%
\definecolor{currentstroke}{rgb}{0.274952,0.037752,0.364543}%
\pgfsetstrokecolor{currentstroke}%
\pgfsetdash{}{0pt}%
\pgfpathmoveto{\pgfqpoint{8.375663in}{3.108799in}}%
\pgfpathlineto{\pgfqpoint{8.325537in}{3.107828in}}%
\pgfusepath{stroke}%
\end{pgfscope}%
\begin{pgfscope}%
\pgfpathrectangle{\pgfqpoint{6.720588in}{1.750000in}}{\pgfqpoint{2.279412in}{2.004545in}}%
\pgfusepath{clip}%
\pgfsetbuttcap%
\pgfsetroundjoin%
\pgfsetlinewidth{0.367247pt}%
\definecolor{currentstroke}{rgb}{0.277941,0.056324,0.381191}%
\pgfsetstrokecolor{currentstroke}%
\pgfsetdash{}{0pt}%
\pgfpathmoveto{\pgfqpoint{8.325537in}{3.107828in}}%
\pgfpathlineto{\pgfqpoint{8.275436in}{3.105890in}}%
\pgfusepath{stroke}%
\end{pgfscope}%
\begin{pgfscope}%
\pgfpathrectangle{\pgfqpoint{6.720588in}{1.750000in}}{\pgfqpoint{2.279412in}{2.004545in}}%
\pgfusepath{clip}%
\pgfsetbuttcap%
\pgfsetroundjoin%
\pgfsetlinewidth{0.389827pt}%
\definecolor{currentstroke}{rgb}{0.280267,0.073417,0.397163}%
\pgfsetstrokecolor{currentstroke}%
\pgfsetdash{}{0pt}%
\pgfpathmoveto{\pgfqpoint{8.275436in}{3.105890in}}%
\pgfpathlineto{\pgfqpoint{8.225356in}{3.103569in}}%
\pgfusepath{stroke}%
\end{pgfscope}%
\begin{pgfscope}%
\pgfpathrectangle{\pgfqpoint{6.720588in}{1.750000in}}{\pgfqpoint{2.279412in}{2.004545in}}%
\pgfusepath{clip}%
\pgfsetbuttcap%
\pgfsetroundjoin%
\pgfsetlinewidth{0.396960pt}%
\definecolor{currentstroke}{rgb}{0.280894,0.078907,0.402329}%
\pgfsetstrokecolor{currentstroke}%
\pgfsetdash{}{0pt}%
\pgfpathmoveto{\pgfqpoint{8.225356in}{3.103569in}}%
\pgfpathlineto{\pgfqpoint{8.175303in}{3.100813in}}%
\pgfusepath{stroke}%
\end{pgfscope}%
\begin{pgfscope}%
\pgfpathrectangle{\pgfqpoint{6.720588in}{1.750000in}}{\pgfqpoint{2.279412in}{2.004545in}}%
\pgfusepath{clip}%
\pgfsetbuttcap%
\pgfsetroundjoin%
\pgfsetlinewidth{0.407297pt}%
\definecolor{currentstroke}{rgb}{0.281924,0.089666,0.412415}%
\pgfsetstrokecolor{currentstroke}%
\pgfsetdash{}{0pt}%
\pgfpathmoveto{\pgfqpoint{8.175303in}{3.100813in}}%
\pgfpathlineto{\pgfqpoint{8.125284in}{3.097654in}}%
\pgfusepath{stroke}%
\end{pgfscope}%
\begin{pgfscope}%
\pgfpathrectangle{\pgfqpoint{6.720588in}{1.750000in}}{\pgfqpoint{2.279412in}{2.004545in}}%
\pgfusepath{clip}%
\pgfsetbuttcap%
\pgfsetroundjoin%
\pgfsetlinewidth{0.332934pt}%
\definecolor{currentstroke}{rgb}{0.272594,0.025563,0.353093}%
\pgfsetstrokecolor{currentstroke}%
\pgfsetdash{}{0pt}%
\pgfpathmoveto{\pgfqpoint{7.757710in}{3.293554in}}%
\pgfpathlineto{\pgfqpoint{7.745515in}{3.260113in}}%
\pgfusepath{stroke}%
\end{pgfscope}%
\begin{pgfscope}%
\pgfpathrectangle{\pgfqpoint{6.720588in}{1.750000in}}{\pgfqpoint{2.279412in}{2.004545in}}%
\pgfusepath{clip}%
\pgfsetbuttcap%
\pgfsetroundjoin%
\pgfsetlinewidth{0.334022pt}%
\definecolor{currentstroke}{rgb}{0.272594,0.025563,0.353093}%
\pgfsetstrokecolor{currentstroke}%
\pgfsetdash{}{0pt}%
\pgfpathmoveto{\pgfqpoint{7.745515in}{3.260113in}}%
\pgfpathlineto{\pgfqpoint{7.729520in}{3.226837in}}%
\pgfusepath{stroke}%
\end{pgfscope}%
\begin{pgfscope}%
\pgfpathrectangle{\pgfqpoint{6.720588in}{1.750000in}}{\pgfqpoint{2.279412in}{2.004545in}}%
\pgfusepath{clip}%
\pgfsetbuttcap%
\pgfsetroundjoin%
\pgfsetlinewidth{0.369979pt}%
\definecolor{currentstroke}{rgb}{0.278791,0.062145,0.386592}%
\pgfsetstrokecolor{currentstroke}%
\pgfsetdash{}{0pt}%
\pgfpathmoveto{\pgfqpoint{7.729520in}{3.226837in}}%
\pgfpathlineto{\pgfqpoint{7.729520in}{3.226837in}}%
\pgfusepath{stroke}%
\end{pgfscope}%
\begin{pgfscope}%
\pgfpathrectangle{\pgfqpoint{6.720588in}{1.750000in}}{\pgfqpoint{2.279412in}{2.004545in}}%
\pgfusepath{clip}%
\pgfsetbuttcap%
\pgfsetroundjoin%
\pgfsetlinewidth{0.369979pt}%
\definecolor{currentstroke}{rgb}{0.278791,0.062145,0.386592}%
\pgfsetstrokecolor{currentstroke}%
\pgfsetdash{}{0pt}%
\pgfpathmoveto{\pgfqpoint{7.729520in}{3.226837in}}%
\pgfpathlineto{\pgfqpoint{7.729520in}{3.226837in}}%
\pgfusepath{stroke}%
\end{pgfscope}%
\begin{pgfscope}%
\pgfpathrectangle{\pgfqpoint{6.720588in}{1.750000in}}{\pgfqpoint{2.279412in}{2.004545in}}%
\pgfusepath{clip}%
\pgfsetbuttcap%
\pgfsetroundjoin%
\pgfsetlinewidth{0.369979pt}%
\definecolor{currentstroke}{rgb}{0.278791,0.062145,0.386592}%
\pgfsetstrokecolor{currentstroke}%
\pgfsetdash{}{0pt}%
\pgfpathmoveto{\pgfqpoint{7.729520in}{3.226837in}}%
\pgfpathlineto{\pgfqpoint{7.725373in}{3.204051in}}%
\pgfusepath{stroke}%
\end{pgfscope}%
\begin{pgfscope}%
\pgfpathrectangle{\pgfqpoint{6.720588in}{1.750000in}}{\pgfqpoint{2.279412in}{2.004545in}}%
\pgfusepath{clip}%
\pgfsetbuttcap%
\pgfsetroundjoin%
\pgfsetlinewidth{0.384612pt}%
\definecolor{currentstroke}{rgb}{0.280267,0.073417,0.397163}%
\pgfsetstrokecolor{currentstroke}%
\pgfsetdash{}{0pt}%
\pgfpathmoveto{\pgfqpoint{7.725373in}{3.204051in}}%
\pgfpathlineto{\pgfqpoint{7.725966in}{3.182342in}}%
\pgfusepath{stroke}%
\end{pgfscope}%
\begin{pgfscope}%
\pgfpathrectangle{\pgfqpoint{6.720588in}{1.750000in}}{\pgfqpoint{2.279412in}{2.004545in}}%
\pgfusepath{clip}%
\pgfsetbuttcap%
\pgfsetroundjoin%
\pgfsetlinewidth{0.387732pt}%
\definecolor{currentstroke}{rgb}{0.280267,0.073417,0.397163}%
\pgfsetstrokecolor{currentstroke}%
\pgfsetdash{}{0pt}%
\pgfpathmoveto{\pgfqpoint{7.725966in}{3.182342in}}%
\pgfpathlineto{\pgfqpoint{7.725966in}{3.182342in}}%
\pgfusepath{stroke}%
\end{pgfscope}%
\begin{pgfscope}%
\pgfpathrectangle{\pgfqpoint{6.720588in}{1.750000in}}{\pgfqpoint{2.279412in}{2.004545in}}%
\pgfusepath{clip}%
\pgfsetbuttcap%
\pgfsetroundjoin%
\pgfsetlinewidth{0.387732pt}%
\definecolor{currentstroke}{rgb}{0.280267,0.073417,0.397163}%
\pgfsetstrokecolor{currentstroke}%
\pgfsetdash{}{0pt}%
\pgfpathmoveto{\pgfqpoint{7.725966in}{3.182342in}}%
\pgfpathlineto{\pgfqpoint{7.722673in}{3.156996in}}%
\pgfusepath{stroke}%
\end{pgfscope}%
\begin{pgfscope}%
\pgfpathrectangle{\pgfqpoint{6.720588in}{1.750000in}}{\pgfqpoint{2.279412in}{2.004545in}}%
\pgfusepath{clip}%
\pgfsetbuttcap%
\pgfsetroundjoin%
\pgfsetlinewidth{0.396936pt}%
\definecolor{currentstroke}{rgb}{0.280894,0.078907,0.402329}%
\pgfsetstrokecolor{currentstroke}%
\pgfsetdash{}{0pt}%
\pgfpathmoveto{\pgfqpoint{7.722673in}{3.156996in}}%
\pgfpathlineto{\pgfqpoint{7.716390in}{3.132258in}}%
\pgfusepath{stroke}%
\end{pgfscope}%
\begin{pgfscope}%
\pgfpathrectangle{\pgfqpoint{6.720588in}{1.750000in}}{\pgfqpoint{2.279412in}{2.004545in}}%
\pgfusepath{clip}%
\pgfsetbuttcap%
\pgfsetroundjoin%
\pgfsetlinewidth{0.381338pt}%
\definecolor{currentstroke}{rgb}{0.279566,0.067836,0.391917}%
\pgfsetstrokecolor{currentstroke}%
\pgfsetdash{}{0pt}%
\pgfpathmoveto{\pgfqpoint{7.716390in}{3.132258in}}%
\pgfpathlineto{\pgfqpoint{7.709632in}{3.106824in}}%
\pgfusepath{stroke}%
\end{pgfscope}%
\begin{pgfscope}%
\pgfpathrectangle{\pgfqpoint{6.720588in}{1.750000in}}{\pgfqpoint{2.279412in}{2.004545in}}%
\pgfusepath{clip}%
\pgfsetbuttcap%
\pgfsetroundjoin%
\pgfsetlinewidth{1.386205pt}%
\definecolor{currentstroke}{rgb}{0.150148,0.676631,0.506589}%
\pgfsetstrokecolor{currentstroke}%
\pgfsetdash{}{0pt}%
\pgfpathmoveto{\pgfqpoint{7.095839in}{2.815869in}}%
\pgfpathlineto{\pgfqpoint{7.145534in}{2.809966in}}%
\pgfusepath{stroke}%
\end{pgfscope}%
\begin{pgfscope}%
\pgfpathrectangle{\pgfqpoint{6.720588in}{1.750000in}}{\pgfqpoint{2.279412in}{2.004545in}}%
\pgfusepath{clip}%
\pgfsetbuttcap%
\pgfsetroundjoin%
\pgfsetlinewidth{1.211341pt}%
\definecolor{currentstroke}{rgb}{0.123463,0.581687,0.547445}%
\pgfsetstrokecolor{currentstroke}%
\pgfsetdash{}{0pt}%
\pgfpathmoveto{\pgfqpoint{7.145534in}{2.809966in}}%
\pgfpathlineto{\pgfqpoint{7.195216in}{2.803993in}}%
\pgfusepath{stroke}%
\end{pgfscope}%
\begin{pgfscope}%
\pgfpathrectangle{\pgfqpoint{6.720588in}{1.750000in}}{\pgfqpoint{2.279412in}{2.004545in}}%
\pgfusepath{clip}%
\pgfsetbuttcap%
\pgfsetroundjoin%
\pgfsetlinewidth{1.052322pt}%
\definecolor{currentstroke}{rgb}{0.153364,0.497000,0.557724}%
\pgfsetstrokecolor{currentstroke}%
\pgfsetdash{}{0pt}%
\pgfpathmoveto{\pgfqpoint{7.195216in}{2.803993in}}%
\pgfpathlineto{\pgfqpoint{7.244791in}{2.797380in}}%
\pgfusepath{stroke}%
\end{pgfscope}%
\begin{pgfscope}%
\pgfpathrectangle{\pgfqpoint{6.720588in}{1.750000in}}{\pgfqpoint{2.279412in}{2.004545in}}%
\pgfusepath{clip}%
\pgfsetbuttcap%
\pgfsetroundjoin%
\pgfsetlinewidth{0.891593pt}%
\definecolor{currentstroke}{rgb}{0.188923,0.410910,0.556326}%
\pgfsetstrokecolor{currentstroke}%
\pgfsetdash{}{0pt}%
\pgfpathmoveto{\pgfqpoint{7.244791in}{2.797380in}}%
\pgfpathlineto{\pgfqpoint{7.294260in}{2.790198in}}%
\pgfusepath{stroke}%
\end{pgfscope}%
\begin{pgfscope}%
\pgfpathrectangle{\pgfqpoint{6.720588in}{1.750000in}}{\pgfqpoint{2.279412in}{2.004545in}}%
\pgfusepath{clip}%
\pgfsetbuttcap%
\pgfsetroundjoin%
\pgfsetlinewidth{0.797037pt}%
\definecolor{currentstroke}{rgb}{0.214298,0.355619,0.551184}%
\pgfsetstrokecolor{currentstroke}%
\pgfsetdash{}{0pt}%
\pgfpathmoveto{\pgfqpoint{7.294260in}{2.790198in}}%
\pgfpathlineto{\pgfqpoint{7.343196in}{2.781175in}}%
\pgfusepath{stroke}%
\end{pgfscope}%
\begin{pgfscope}%
\pgfpathrectangle{\pgfqpoint{6.720588in}{1.750000in}}{\pgfqpoint{2.279412in}{2.004545in}}%
\pgfusepath{clip}%
\pgfsetbuttcap%
\pgfsetroundjoin%
\pgfsetlinewidth{0.331499pt}%
\definecolor{currentstroke}{rgb}{0.272594,0.025563,0.353093}%
\pgfsetstrokecolor{currentstroke}%
\pgfsetdash{}{0pt}%
\pgfpathmoveto{\pgfqpoint{8.424505in}{3.203341in}}%
\pgfpathlineto{\pgfqpoint{8.374357in}{3.203018in}}%
\pgfusepath{stroke}%
\end{pgfscope}%
\begin{pgfscope}%
\pgfpathrectangle{\pgfqpoint{6.720588in}{1.750000in}}{\pgfqpoint{2.279412in}{2.004545in}}%
\pgfusepath{clip}%
\pgfsetbuttcap%
\pgfsetroundjoin%
\pgfsetlinewidth{0.338183pt}%
\definecolor{currentstroke}{rgb}{0.273809,0.031497,0.358853}%
\pgfsetstrokecolor{currentstroke}%
\pgfsetdash{}{0pt}%
\pgfpathmoveto{\pgfqpoint{8.374357in}{3.203018in}}%
\pgfpathlineto{\pgfqpoint{8.324239in}{3.202155in}}%
\pgfusepath{stroke}%
\end{pgfscope}%
\begin{pgfscope}%
\pgfpathrectangle{\pgfqpoint{6.720588in}{1.750000in}}{\pgfqpoint{2.279412in}{2.004545in}}%
\pgfusepath{clip}%
\pgfsetbuttcap%
\pgfsetroundjoin%
\pgfsetlinewidth{0.343065pt}%
\definecolor{currentstroke}{rgb}{0.274952,0.037752,0.364543}%
\pgfsetstrokecolor{currentstroke}%
\pgfsetdash{}{0pt}%
\pgfpathmoveto{\pgfqpoint{8.324239in}{3.202155in}}%
\pgfpathlineto{\pgfqpoint{8.274143in}{3.200663in}}%
\pgfusepath{stroke}%
\end{pgfscope}%
\begin{pgfscope}%
\pgfpathrectangle{\pgfqpoint{6.720588in}{1.750000in}}{\pgfqpoint{2.279412in}{2.004545in}}%
\pgfusepath{clip}%
\pgfsetbuttcap%
\pgfsetroundjoin%
\pgfsetlinewidth{0.364458pt}%
\definecolor{currentstroke}{rgb}{0.277941,0.056324,0.381191}%
\pgfsetstrokecolor{currentstroke}%
\pgfsetdash{}{0pt}%
\pgfpathmoveto{\pgfqpoint{8.274143in}{3.200663in}}%
\pgfpathlineto{\pgfqpoint{8.224090in}{3.198125in}}%
\pgfusepath{stroke}%
\end{pgfscope}%
\begin{pgfscope}%
\pgfpathrectangle{\pgfqpoint{6.720588in}{1.750000in}}{\pgfqpoint{2.279412in}{2.004545in}}%
\pgfusepath{clip}%
\pgfsetbuttcap%
\pgfsetroundjoin%
\pgfsetlinewidth{0.385535pt}%
\definecolor{currentstroke}{rgb}{0.280267,0.073417,0.397163}%
\pgfsetstrokecolor{currentstroke}%
\pgfsetdash{}{0pt}%
\pgfpathmoveto{\pgfqpoint{8.224090in}{3.198125in}}%
\pgfpathlineto{\pgfqpoint{8.174114in}{3.194478in}}%
\pgfusepath{stroke}%
\end{pgfscope}%
\begin{pgfscope}%
\pgfpathrectangle{\pgfqpoint{6.720588in}{1.750000in}}{\pgfqpoint{2.279412in}{2.004545in}}%
\pgfusepath{clip}%
\pgfsetbuttcap%
\pgfsetroundjoin%
\pgfsetlinewidth{0.392484pt}%
\definecolor{currentstroke}{rgb}{0.280894,0.078907,0.402329}%
\pgfsetstrokecolor{currentstroke}%
\pgfsetdash{}{0pt}%
\pgfpathmoveto{\pgfqpoint{8.174114in}{3.194478in}}%
\pgfpathlineto{\pgfqpoint{8.124166in}{3.190571in}}%
\pgfusepath{stroke}%
\end{pgfscope}%
\begin{pgfscope}%
\pgfpathrectangle{\pgfqpoint{6.720588in}{1.750000in}}{\pgfqpoint{2.279412in}{2.004545in}}%
\pgfusepath{clip}%
\pgfsetbuttcap%
\pgfsetroundjoin%
\pgfsetlinewidth{0.401018pt}%
\definecolor{currentstroke}{rgb}{0.281446,0.084320,0.407414}%
\pgfsetstrokecolor{currentstroke}%
\pgfsetdash{}{0pt}%
\pgfpathmoveto{\pgfqpoint{8.124166in}{3.190571in}}%
\pgfpathlineto{\pgfqpoint{8.074383in}{3.185411in}}%
\pgfusepath{stroke}%
\end{pgfscope}%
\begin{pgfscope}%
\pgfpathrectangle{\pgfqpoint{6.720588in}{1.750000in}}{\pgfqpoint{2.279412in}{2.004545in}}%
\pgfusepath{clip}%
\pgfsetbuttcap%
\pgfsetroundjoin%
\pgfsetlinewidth{0.426624pt}%
\definecolor{currentstroke}{rgb}{0.282910,0.105393,0.426902}%
\pgfsetstrokecolor{currentstroke}%
\pgfsetdash{}{0pt}%
\pgfpathmoveto{\pgfqpoint{8.074383in}{3.185411in}}%
\pgfpathlineto{\pgfqpoint{8.025005in}{3.177854in}}%
\pgfusepath{stroke}%
\end{pgfscope}%
\begin{pgfscope}%
\pgfpathrectangle{\pgfqpoint{6.720588in}{1.750000in}}{\pgfqpoint{2.279412in}{2.004545in}}%
\pgfusepath{clip}%
\pgfsetbuttcap%
\pgfsetroundjoin%
\pgfsetlinewidth{0.391750pt}%
\definecolor{currentstroke}{rgb}{0.280894,0.078907,0.402329}%
\pgfsetstrokecolor{currentstroke}%
\pgfsetdash{}{0pt}%
\pgfpathmoveto{\pgfqpoint{8.025005in}{3.177854in}}%
\pgfpathlineto{\pgfqpoint{7.976104in}{3.168182in}}%
\pgfusepath{stroke}%
\end{pgfscope}%
\begin{pgfscope}%
\pgfpathrectangle{\pgfqpoint{6.720588in}{1.750000in}}{\pgfqpoint{2.279412in}{2.004545in}}%
\pgfusepath{clip}%
\pgfsetbuttcap%
\pgfsetroundjoin%
\pgfsetlinewidth{0.418645pt}%
\definecolor{currentstroke}{rgb}{0.282656,0.100196,0.422160}%
\pgfsetstrokecolor{currentstroke}%
\pgfsetdash{}{0pt}%
\pgfpathmoveto{\pgfqpoint{7.976104in}{3.168182in}}%
\pgfpathlineto{\pgfqpoint{7.927598in}{3.157025in}}%
\pgfusepath{stroke}%
\end{pgfscope}%
\begin{pgfscope}%
\pgfpathrectangle{\pgfqpoint{6.720588in}{1.750000in}}{\pgfqpoint{2.279412in}{2.004545in}}%
\pgfusepath{clip}%
\pgfsetbuttcap%
\pgfsetroundjoin%
\pgfsetlinewidth{0.426612pt}%
\definecolor{currentstroke}{rgb}{0.282910,0.105393,0.426902}%
\pgfsetstrokecolor{currentstroke}%
\pgfsetdash{}{0pt}%
\pgfpathmoveto{\pgfqpoint{7.927598in}{3.157025in}}%
\pgfpathlineto{\pgfqpoint{7.879743in}{3.144003in}}%
\pgfusepath{stroke}%
\end{pgfscope}%
\begin{pgfscope}%
\pgfpathrectangle{\pgfqpoint{6.720588in}{1.750000in}}{\pgfqpoint{2.279412in}{2.004545in}}%
\pgfusepath{clip}%
\pgfsetbuttcap%
\pgfsetroundjoin%
\pgfsetlinewidth{0.418073pt}%
\definecolor{currentstroke}{rgb}{0.282656,0.100196,0.422160}%
\pgfsetstrokecolor{currentstroke}%
\pgfsetdash{}{0pt}%
\pgfpathmoveto{\pgfqpoint{7.879743in}{3.144003in}}%
\pgfpathlineto{\pgfqpoint{7.834529in}{3.125564in}}%
\pgfusepath{stroke}%
\end{pgfscope}%
\begin{pgfscope}%
\pgfpathrectangle{\pgfqpoint{6.720588in}{1.750000in}}{\pgfqpoint{2.279412in}{2.004545in}}%
\pgfusepath{clip}%
\pgfsetbuttcap%
\pgfsetroundjoin%
\pgfsetlinewidth{0.424347pt}%
\definecolor{currentstroke}{rgb}{0.282656,0.100196,0.422160}%
\pgfsetstrokecolor{currentstroke}%
\pgfsetdash{}{0pt}%
\pgfpathmoveto{\pgfqpoint{7.834529in}{3.125564in}}%
\pgfpathlineto{\pgfqpoint{7.793358in}{3.103801in}}%
\pgfusepath{stroke}%
\end{pgfscope}%
\begin{pgfscope}%
\pgfpathrectangle{\pgfqpoint{6.720588in}{1.750000in}}{\pgfqpoint{2.279412in}{2.004545in}}%
\pgfusepath{clip}%
\pgfsetbuttcap%
\pgfsetroundjoin%
\pgfsetlinewidth{0.461936pt}%
\definecolor{currentstroke}{rgb}{0.283072,0.130895,0.449241}%
\pgfsetstrokecolor{currentstroke}%
\pgfsetdash{}{0pt}%
\pgfpathmoveto{\pgfqpoint{7.793358in}{3.103801in}}%
\pgfpathlineto{\pgfqpoint{7.753644in}{3.077269in}}%
\pgfusepath{stroke}%
\end{pgfscope}%
\begin{pgfscope}%
\pgfpathrectangle{\pgfqpoint{6.720588in}{1.750000in}}{\pgfqpoint{2.279412in}{2.004545in}}%
\pgfusepath{clip}%
\pgfsetbuttcap%
\pgfsetroundjoin%
\pgfsetlinewidth{0.481744pt}%
\definecolor{currentstroke}{rgb}{0.282290,0.145912,0.461510}%
\pgfsetstrokecolor{currentstroke}%
\pgfsetdash{}{0pt}%
\pgfpathmoveto{\pgfqpoint{7.753644in}{3.077269in}}%
\pgfpathlineto{\pgfqpoint{7.716144in}{3.048424in}}%
\pgfusepath{stroke}%
\end{pgfscope}%
\begin{pgfscope}%
\pgfpathrectangle{\pgfqpoint{6.720588in}{1.750000in}}{\pgfqpoint{2.279412in}{2.004545in}}%
\pgfusepath{clip}%
\pgfsetbuttcap%
\pgfsetroundjoin%
\pgfsetlinewidth{0.323178pt}%
\definecolor{currentstroke}{rgb}{0.271305,0.019942,0.347269}%
\pgfsetstrokecolor{currentstroke}%
\pgfsetdash{}{0pt}%
\pgfpathmoveto{\pgfqpoint{8.424505in}{3.248447in}}%
\pgfpathlineto{\pgfqpoint{8.374394in}{3.246823in}}%
\pgfusepath{stroke}%
\end{pgfscope}%
\begin{pgfscope}%
\pgfpathrectangle{\pgfqpoint{6.720588in}{1.750000in}}{\pgfqpoint{2.279412in}{2.004545in}}%
\pgfusepath{clip}%
\pgfsetbuttcap%
\pgfsetroundjoin%
\pgfsetlinewidth{0.338091pt}%
\definecolor{currentstroke}{rgb}{0.273809,0.031497,0.358853}%
\pgfsetstrokecolor{currentstroke}%
\pgfsetdash{}{0pt}%
\pgfpathmoveto{\pgfqpoint{8.374394in}{3.246823in}}%
\pgfpathlineto{\pgfqpoint{8.324263in}{3.245859in}}%
\pgfusepath{stroke}%
\end{pgfscope}%
\begin{pgfscope}%
\pgfpathrectangle{\pgfqpoint{6.720588in}{1.750000in}}{\pgfqpoint{2.279412in}{2.004545in}}%
\pgfusepath{clip}%
\pgfsetbuttcap%
\pgfsetroundjoin%
\pgfsetlinewidth{0.350127pt}%
\definecolor{currentstroke}{rgb}{0.276022,0.044167,0.370164}%
\pgfsetstrokecolor{currentstroke}%
\pgfsetdash{}{0pt}%
\pgfpathmoveto{\pgfqpoint{8.324263in}{3.245859in}}%
\pgfpathlineto{\pgfqpoint{8.274146in}{3.244524in}}%
\pgfusepath{stroke}%
\end{pgfscope}%
\begin{pgfscope}%
\pgfpathrectangle{\pgfqpoint{6.720588in}{1.750000in}}{\pgfqpoint{2.279412in}{2.004545in}}%
\pgfusepath{clip}%
\pgfsetbuttcap%
\pgfsetroundjoin%
\pgfsetlinewidth{0.351620pt}%
\definecolor{currentstroke}{rgb}{0.276022,0.044167,0.370164}%
\pgfsetstrokecolor{currentstroke}%
\pgfsetdash{}{0pt}%
\pgfpathmoveto{\pgfqpoint{8.274146in}{3.244524in}}%
\pgfpathlineto{\pgfqpoint{8.224087in}{3.242016in}}%
\pgfusepath{stroke}%
\end{pgfscope}%
\begin{pgfscope}%
\pgfpathrectangle{\pgfqpoint{6.720588in}{1.750000in}}{\pgfqpoint{2.279412in}{2.004545in}}%
\pgfusepath{clip}%
\pgfsetbuttcap%
\pgfsetroundjoin%
\pgfsetlinewidth{0.363601pt}%
\definecolor{currentstroke}{rgb}{0.277941,0.056324,0.381191}%
\pgfsetstrokecolor{currentstroke}%
\pgfsetdash{}{0pt}%
\pgfpathmoveto{\pgfqpoint{8.224087in}{3.242016in}}%
\pgfpathlineto{\pgfqpoint{8.174134in}{3.238380in}}%
\pgfusepath{stroke}%
\end{pgfscope}%
\begin{pgfscope}%
\pgfpathrectangle{\pgfqpoint{6.720588in}{1.750000in}}{\pgfqpoint{2.279412in}{2.004545in}}%
\pgfusepath{clip}%
\pgfsetbuttcap%
\pgfsetroundjoin%
\pgfsetlinewidth{0.380897pt}%
\definecolor{currentstroke}{rgb}{0.279566,0.067836,0.391917}%
\pgfsetstrokecolor{currentstroke}%
\pgfsetdash{}{0pt}%
\pgfpathmoveto{\pgfqpoint{8.174134in}{3.238380in}}%
\pgfpathlineto{\pgfqpoint{8.124357in}{3.233312in}}%
\pgfusepath{stroke}%
\end{pgfscope}%
\begin{pgfscope}%
\pgfpathrectangle{\pgfqpoint{6.720588in}{1.750000in}}{\pgfqpoint{2.279412in}{2.004545in}}%
\pgfusepath{clip}%
\pgfsetbuttcap%
\pgfsetroundjoin%
\pgfsetlinewidth{0.379900pt}%
\definecolor{currentstroke}{rgb}{0.279566,0.067836,0.391917}%
\pgfsetstrokecolor{currentstroke}%
\pgfsetdash{}{0pt}%
\pgfpathmoveto{\pgfqpoint{8.065462in}{3.248447in}}%
\pgfpathlineto{\pgfqpoint{8.016077in}{3.240843in}}%
\pgfusepath{stroke}%
\end{pgfscope}%
\begin{pgfscope}%
\pgfpathrectangle{\pgfqpoint{6.720588in}{1.750000in}}{\pgfqpoint{2.279412in}{2.004545in}}%
\pgfusepath{clip}%
\pgfsetbuttcap%
\pgfsetroundjoin%
\pgfsetlinewidth{0.397009pt}%
\definecolor{currentstroke}{rgb}{0.280894,0.078907,0.402329}%
\pgfsetstrokecolor{currentstroke}%
\pgfsetdash{}{0pt}%
\pgfpathmoveto{\pgfqpoint{8.016077in}{3.240843in}}%
\pgfpathlineto{\pgfqpoint{7.967874in}{3.229285in}}%
\pgfusepath{stroke}%
\end{pgfscope}%
\begin{pgfscope}%
\pgfpathrectangle{\pgfqpoint{6.720588in}{1.750000in}}{\pgfqpoint{2.279412in}{2.004545in}}%
\pgfusepath{clip}%
\pgfsetbuttcap%
\pgfsetroundjoin%
\pgfsetlinewidth{0.389370pt}%
\definecolor{currentstroke}{rgb}{0.280267,0.073417,0.397163}%
\pgfsetstrokecolor{currentstroke}%
\pgfsetdash{}{0pt}%
\pgfpathmoveto{\pgfqpoint{7.967874in}{3.229285in}}%
\pgfpathlineto{\pgfqpoint{7.920662in}{3.214610in}}%
\pgfusepath{stroke}%
\end{pgfscope}%
\begin{pgfscope}%
\pgfpathrectangle{\pgfqpoint{6.720588in}{1.750000in}}{\pgfqpoint{2.279412in}{2.004545in}}%
\pgfusepath{clip}%
\pgfsetbuttcap%
\pgfsetroundjoin%
\pgfsetlinewidth{0.390290pt}%
\definecolor{currentstroke}{rgb}{0.280267,0.073417,0.397163}%
\pgfsetstrokecolor{currentstroke}%
\pgfsetdash{}{0pt}%
\pgfpathmoveto{\pgfqpoint{7.920662in}{3.214610in}}%
\pgfpathlineto{\pgfqpoint{7.873778in}{3.199282in}}%
\pgfusepath{stroke}%
\end{pgfscope}%
\begin{pgfscope}%
\pgfpathrectangle{\pgfqpoint{6.720588in}{1.750000in}}{\pgfqpoint{2.279412in}{2.004545in}}%
\pgfusepath{clip}%
\pgfsetbuttcap%
\pgfsetroundjoin%
\pgfsetlinewidth{0.376689pt}%
\definecolor{currentstroke}{rgb}{0.279566,0.067836,0.391917}%
\pgfsetstrokecolor{currentstroke}%
\pgfsetdash{}{0pt}%
\pgfpathmoveto{\pgfqpoint{7.873778in}{3.199282in}}%
\pgfpathlineto{\pgfqpoint{7.829303in}{3.179880in}}%
\pgfusepath{stroke}%
\end{pgfscope}%
\begin{pgfscope}%
\pgfpathrectangle{\pgfqpoint{6.720588in}{1.750000in}}{\pgfqpoint{2.279412in}{2.004545in}}%
\pgfusepath{clip}%
\pgfsetbuttcap%
\pgfsetroundjoin%
\pgfsetlinewidth{0.372502pt}%
\definecolor{currentstroke}{rgb}{0.278791,0.062145,0.386592}%
\pgfsetstrokecolor{currentstroke}%
\pgfsetdash{}{0pt}%
\pgfpathmoveto{\pgfqpoint{7.829303in}{3.179880in}}%
\pgfpathlineto{\pgfqpoint{7.788752in}{3.155199in}}%
\pgfusepath{stroke}%
\end{pgfscope}%
\begin{pgfscope}%
\pgfpathrectangle{\pgfqpoint{6.720588in}{1.750000in}}{\pgfqpoint{2.279412in}{2.004545in}}%
\pgfusepath{clip}%
\pgfsetbuttcap%
\pgfsetroundjoin%
\pgfsetlinewidth{0.395876pt}%
\definecolor{currentstroke}{rgb}{0.280894,0.078907,0.402329}%
\pgfsetstrokecolor{currentstroke}%
\pgfsetdash{}{0pt}%
\pgfpathmoveto{\pgfqpoint{7.788752in}{3.155199in}}%
\pgfpathlineto{\pgfqpoint{7.751101in}{3.128339in}}%
\pgfusepath{stroke}%
\end{pgfscope}%
\begin{pgfscope}%
\pgfpathrectangle{\pgfqpoint{6.720588in}{1.750000in}}{\pgfqpoint{2.279412in}{2.004545in}}%
\pgfusepath{clip}%
\pgfsetbuttcap%
\pgfsetroundjoin%
\pgfsetlinewidth{0.392835pt}%
\definecolor{currentstroke}{rgb}{0.280894,0.078907,0.402329}%
\pgfsetstrokecolor{currentstroke}%
\pgfsetdash{}{0pt}%
\pgfpathmoveto{\pgfqpoint{7.751101in}{3.128339in}}%
\pgfpathlineto{\pgfqpoint{7.751101in}{3.128339in}}%
\pgfusepath{stroke}%
\end{pgfscope}%
\begin{pgfscope}%
\pgfpathrectangle{\pgfqpoint{6.720588in}{1.750000in}}{\pgfqpoint{2.279412in}{2.004545in}}%
\pgfusepath{clip}%
\pgfsetbuttcap%
\pgfsetroundjoin%
\pgfsetlinewidth{0.339876pt}%
\definecolor{currentstroke}{rgb}{0.273809,0.031497,0.358853}%
\pgfsetstrokecolor{currentstroke}%
\pgfsetdash{}{0pt}%
\pgfpathmoveto{\pgfqpoint{8.005012in}{3.282031in}}%
\pgfpathlineto{\pgfqpoint{7.955703in}{3.274872in}}%
\pgfusepath{stroke}%
\end{pgfscope}%
\begin{pgfscope}%
\pgfpathrectangle{\pgfqpoint{6.720588in}{1.750000in}}{\pgfqpoint{2.279412in}{2.004545in}}%
\pgfusepath{clip}%
\pgfsetbuttcap%
\pgfsetroundjoin%
\pgfsetlinewidth{0.358083pt}%
\definecolor{currentstroke}{rgb}{0.277018,0.050344,0.375715}%
\pgfsetstrokecolor{currentstroke}%
\pgfsetdash{}{0pt}%
\pgfpathmoveto{\pgfqpoint{7.955703in}{3.274872in}}%
\pgfpathlineto{\pgfqpoint{7.907647in}{3.262840in}}%
\pgfusepath{stroke}%
\end{pgfscope}%
\begin{pgfscope}%
\pgfpathrectangle{\pgfqpoint{6.720588in}{1.750000in}}{\pgfqpoint{2.279412in}{2.004545in}}%
\pgfusepath{clip}%
\pgfsetbuttcap%
\pgfsetroundjoin%
\pgfsetlinewidth{0.359353pt}%
\definecolor{currentstroke}{rgb}{0.277018,0.050344,0.375715}%
\pgfsetstrokecolor{currentstroke}%
\pgfsetdash{}{0pt}%
\pgfpathmoveto{\pgfqpoint{7.907647in}{3.262840in}}%
\pgfpathlineto{\pgfqpoint{7.860294in}{3.248447in}}%
\pgfusepath{stroke}%
\end{pgfscope}%
\begin{pgfscope}%
\pgfpathrectangle{\pgfqpoint{6.720588in}{1.750000in}}{\pgfqpoint{2.279412in}{2.004545in}}%
\pgfusepath{clip}%
\pgfsetbuttcap%
\pgfsetroundjoin%
\pgfsetlinewidth{0.379629pt}%
\definecolor{currentstroke}{rgb}{0.279566,0.067836,0.391917}%
\pgfsetstrokecolor{currentstroke}%
\pgfsetdash{}{0pt}%
\pgfpathmoveto{\pgfqpoint{7.860294in}{3.248447in}}%
\pgfpathlineto{\pgfqpoint{7.860294in}{3.248447in}}%
\pgfusepath{stroke}%
\end{pgfscope}%
\begin{pgfscope}%
\pgfpathrectangle{\pgfqpoint{6.720588in}{1.750000in}}{\pgfqpoint{2.279412in}{2.004545in}}%
\pgfusepath{clip}%
\pgfsetbuttcap%
\pgfsetroundjoin%
\pgfsetlinewidth{0.379629pt}%
\definecolor{currentstroke}{rgb}{0.279566,0.067836,0.391917}%
\pgfsetstrokecolor{currentstroke}%
\pgfsetdash{}{0pt}%
\pgfpathmoveto{\pgfqpoint{7.860294in}{3.248447in}}%
\pgfpathlineto{\pgfqpoint{7.829523in}{3.233368in}}%
\pgfusepath{stroke}%
\end{pgfscope}%
\begin{pgfscope}%
\pgfpathrectangle{\pgfqpoint{6.720588in}{1.750000in}}{\pgfqpoint{2.279412in}{2.004545in}}%
\pgfusepath{clip}%
\pgfsetbuttcap%
\pgfsetroundjoin%
\pgfsetlinewidth{0.367284pt}%
\definecolor{currentstroke}{rgb}{0.277941,0.056324,0.381191}%
\pgfsetstrokecolor{currentstroke}%
\pgfsetdash{}{0pt}%
\pgfpathmoveto{\pgfqpoint{7.829523in}{3.233368in}}%
\pgfpathlineto{\pgfqpoint{7.805012in}{3.215602in}}%
\pgfusepath{stroke}%
\end{pgfscope}%
\begin{pgfscope}%
\pgfpathrectangle{\pgfqpoint{6.720588in}{1.750000in}}{\pgfqpoint{2.279412in}{2.004545in}}%
\pgfusepath{clip}%
\pgfsetbuttcap%
\pgfsetroundjoin%
\pgfsetlinewidth{0.367493pt}%
\definecolor{currentstroke}{rgb}{0.277941,0.056324,0.381191}%
\pgfsetstrokecolor{currentstroke}%
\pgfsetdash{}{0pt}%
\pgfpathmoveto{\pgfqpoint{7.805012in}{3.215602in}}%
\pgfpathlineto{\pgfqpoint{7.776758in}{3.187482in}}%
\pgfusepath{stroke}%
\end{pgfscope}%
\begin{pgfscope}%
\pgfpathrectangle{\pgfqpoint{6.720588in}{1.750000in}}{\pgfqpoint{2.279412in}{2.004545in}}%
\pgfusepath{clip}%
\pgfsetbuttcap%
\pgfsetroundjoin%
\pgfsetlinewidth{0.420282pt}%
\definecolor{currentstroke}{rgb}{0.282656,0.100196,0.422160}%
\pgfsetstrokecolor{currentstroke}%
\pgfsetdash{}{0pt}%
\pgfpathmoveto{\pgfqpoint{7.776758in}{3.187482in}}%
\pgfpathlineto{\pgfqpoint{7.751533in}{3.150574in}}%
\pgfusepath{stroke}%
\end{pgfscope}%
\begin{pgfscope}%
\pgfpathrectangle{\pgfqpoint{6.720588in}{1.750000in}}{\pgfqpoint{2.279412in}{2.004545in}}%
\pgfusepath{clip}%
\pgfsetbuttcap%
\pgfsetroundjoin%
\pgfsetlinewidth{0.417665pt}%
\definecolor{currentstroke}{rgb}{0.282327,0.094955,0.417331}%
\pgfsetstrokecolor{currentstroke}%
\pgfsetdash{}{0pt}%
\pgfpathmoveto{\pgfqpoint{7.751533in}{3.150574in}}%
\pgfpathlineto{\pgfqpoint{7.751533in}{3.150574in}}%
\pgfusepath{stroke}%
\end{pgfscope}%
\begin{pgfscope}%
\pgfpathrectangle{\pgfqpoint{6.720588in}{1.750000in}}{\pgfqpoint{2.279412in}{2.004545in}}%
\pgfusepath{clip}%
\pgfsetbuttcap%
\pgfsetroundjoin%
\pgfsetlinewidth{0.936083pt}%
\definecolor{currentstroke}{rgb}{0.179019,0.433756,0.557430}%
\pgfsetstrokecolor{currentstroke}%
\pgfsetdash{}{0pt}%
\pgfpathmoveto{\pgfqpoint{7.296083in}{2.887593in}}%
\pgfpathlineto{\pgfqpoint{7.343122in}{2.872504in}}%
\pgfusepath{stroke}%
\end{pgfscope}%
\begin{pgfscope}%
\pgfpathrectangle{\pgfqpoint{6.720588in}{1.750000in}}{\pgfqpoint{2.279412in}{2.004545in}}%
\pgfusepath{clip}%
\pgfsetbuttcap%
\pgfsetroundjoin%
\pgfsetlinewidth{0.864030pt}%
\definecolor{currentstroke}{rgb}{0.195860,0.395433,0.555276}%
\pgfsetstrokecolor{currentstroke}%
\pgfsetdash{}{0pt}%
\pgfpathmoveto{\pgfqpoint{7.343122in}{2.872504in}}%
\pgfpathlineto{\pgfqpoint{7.388368in}{2.853695in}}%
\pgfusepath{stroke}%
\end{pgfscope}%
\begin{pgfscope}%
\pgfpathrectangle{\pgfqpoint{6.720588in}{1.750000in}}{\pgfqpoint{2.279412in}{2.004545in}}%
\pgfusepath{clip}%
\pgfsetbuttcap%
\pgfsetroundjoin%
\pgfsetlinewidth{0.664473pt}%
\definecolor{currentstroke}{rgb}{0.252194,0.269783,0.531579}%
\pgfsetstrokecolor{currentstroke}%
\pgfsetdash{}{0pt}%
\pgfpathmoveto{\pgfqpoint{7.388368in}{2.853695in}}%
\pgfpathlineto{\pgfqpoint{7.431229in}{2.831423in}}%
\pgfusepath{stroke}%
\end{pgfscope}%
\begin{pgfscope}%
\pgfpathrectangle{\pgfqpoint{6.720588in}{1.750000in}}{\pgfqpoint{2.279412in}{2.004545in}}%
\pgfusepath{clip}%
\pgfsetbuttcap%
\pgfsetroundjoin%
\pgfsetlinewidth{0.511082pt}%
\definecolor{currentstroke}{rgb}{0.280255,0.165693,0.476498}%
\pgfsetstrokecolor{currentstroke}%
\pgfsetdash{}{0pt}%
\pgfpathmoveto{\pgfqpoint{7.431229in}{2.831423in}}%
\pgfpathlineto{\pgfqpoint{7.431229in}{2.831423in}}%
\pgfusepath{stroke}%
\end{pgfscope}%
\begin{pgfscope}%
\pgfpathrectangle{\pgfqpoint{6.720588in}{1.750000in}}{\pgfqpoint{2.279412in}{2.004545in}}%
\pgfusepath{clip}%
\pgfsetbuttcap%
\pgfsetroundjoin%
\pgfsetlinewidth{0.511082pt}%
\definecolor{currentstroke}{rgb}{0.280255,0.165693,0.476498}%
\pgfsetstrokecolor{currentstroke}%
\pgfsetdash{}{0pt}%
\pgfpathmoveto{\pgfqpoint{7.431229in}{2.831423in}}%
\pgfpathlineto{\pgfqpoint{7.445235in}{2.819871in}}%
\pgfusepath{stroke}%
\end{pgfscope}%
\begin{pgfscope}%
\pgfpathrectangle{\pgfqpoint{6.720588in}{1.750000in}}{\pgfqpoint{2.279412in}{2.004545in}}%
\pgfusepath{clip}%
\pgfsetbuttcap%
\pgfsetroundjoin%
\pgfsetlinewidth{0.523688pt}%
\definecolor{currentstroke}{rgb}{0.278826,0.175490,0.483397}%
\pgfsetstrokecolor{currentstroke}%
\pgfsetdash{}{0pt}%
\pgfpathmoveto{\pgfqpoint{7.445235in}{2.819871in}}%
\pgfpathlineto{\pgfqpoint{7.455090in}{2.803297in}}%
\pgfusepath{stroke}%
\end{pgfscope}%
\begin{pgfscope}%
\pgfpathrectangle{\pgfqpoint{6.720588in}{1.750000in}}{\pgfqpoint{2.279412in}{2.004545in}}%
\pgfusepath{clip}%
\pgfsetbuttcap%
\pgfsetroundjoin%
\pgfsetlinewidth{0.490055pt}%
\definecolor{currentstroke}{rgb}{0.281887,0.150881,0.465405}%
\pgfsetstrokecolor{currentstroke}%
\pgfsetdash{}{0pt}%
\pgfpathmoveto{\pgfqpoint{7.455090in}{2.803297in}}%
\pgfpathlineto{\pgfqpoint{7.455090in}{2.803297in}}%
\pgfusepath{stroke}%
\end{pgfscope}%
\begin{pgfscope}%
\pgfpathrectangle{\pgfqpoint{6.720588in}{1.750000in}}{\pgfqpoint{2.279412in}{2.004545in}}%
\pgfusepath{clip}%
\pgfsetbuttcap%
\pgfsetroundjoin%
\pgfsetlinewidth{0.490055pt}%
\definecolor{currentstroke}{rgb}{0.281887,0.150881,0.465405}%
\pgfsetstrokecolor{currentstroke}%
\pgfsetdash{}{0pt}%
\pgfpathmoveto{\pgfqpoint{7.455090in}{2.803297in}}%
\pgfpathlineto{\pgfqpoint{7.458467in}{2.789544in}}%
\pgfusepath{stroke}%
\end{pgfscope}%
\begin{pgfscope}%
\pgfpathrectangle{\pgfqpoint{6.720588in}{1.750000in}}{\pgfqpoint{2.279412in}{2.004545in}}%
\pgfusepath{clip}%
\pgfsetbuttcap%
\pgfsetroundjoin%
\pgfsetlinewidth{0.479145pt}%
\definecolor{currentstroke}{rgb}{0.282623,0.140926,0.457517}%
\pgfsetstrokecolor{currentstroke}%
\pgfsetdash{}{0pt}%
\pgfpathmoveto{\pgfqpoint{7.458467in}{2.789544in}}%
\pgfpathlineto{\pgfqpoint{7.458798in}{2.775868in}}%
\pgfusepath{stroke}%
\end{pgfscope}%
\begin{pgfscope}%
\pgfpathrectangle{\pgfqpoint{6.720588in}{1.750000in}}{\pgfqpoint{2.279412in}{2.004545in}}%
\pgfusepath{clip}%
\pgfsetbuttcap%
\pgfsetroundjoin%
\pgfsetlinewidth{0.469090pt}%
\definecolor{currentstroke}{rgb}{0.282884,0.135920,0.453427}%
\pgfsetstrokecolor{currentstroke}%
\pgfsetdash{}{0pt}%
\pgfpathmoveto{\pgfqpoint{7.458798in}{2.775868in}}%
\pgfpathlineto{\pgfqpoint{7.458798in}{2.775868in}}%
\pgfusepath{stroke}%
\end{pgfscope}%
\begin{pgfscope}%
\pgfpathrectangle{\pgfqpoint{6.720588in}{1.750000in}}{\pgfqpoint{2.279412in}{2.004545in}}%
\pgfusepath{clip}%
\pgfsetbuttcap%
\pgfsetroundjoin%
\pgfsetlinewidth{0.874369pt}%
\definecolor{currentstroke}{rgb}{0.194100,0.399323,0.555565}%
\pgfsetstrokecolor{currentstroke}%
\pgfsetdash{}{0pt}%
\pgfpathmoveto{\pgfqpoint{7.299860in}{2.558216in}}%
\pgfpathlineto{\pgfqpoint{7.347375in}{2.571846in}}%
\pgfusepath{stroke}%
\end{pgfscope}%
\begin{pgfscope}%
\pgfpathrectangle{\pgfqpoint{6.720588in}{1.750000in}}{\pgfqpoint{2.279412in}{2.004545in}}%
\pgfusepath{clip}%
\pgfsetbuttcap%
\pgfsetroundjoin%
\pgfsetlinewidth{0.784771pt}%
\definecolor{currentstroke}{rgb}{0.218130,0.347432,0.550038}%
\pgfsetstrokecolor{currentstroke}%
\pgfsetdash{}{0pt}%
\pgfpathmoveto{\pgfqpoint{7.347375in}{2.571846in}}%
\pgfpathlineto{\pgfqpoint{7.392362in}{2.590386in}}%
\pgfusepath{stroke}%
\end{pgfscope}%
\begin{pgfscope}%
\pgfpathrectangle{\pgfqpoint{6.720588in}{1.750000in}}{\pgfqpoint{2.279412in}{2.004545in}}%
\pgfusepath{clip}%
\pgfsetbuttcap%
\pgfsetroundjoin%
\pgfsetlinewidth{0.704424pt}%
\definecolor{currentstroke}{rgb}{0.241237,0.296485,0.539709}%
\pgfsetstrokecolor{currentstroke}%
\pgfsetdash{}{0pt}%
\pgfpathmoveto{\pgfqpoint{7.392362in}{2.590386in}}%
\pgfpathlineto{\pgfqpoint{7.392362in}{2.590386in}}%
\pgfusepath{stroke}%
\end{pgfscope}%
\begin{pgfscope}%
\pgfpathrectangle{\pgfqpoint{6.720588in}{1.750000in}}{\pgfqpoint{2.279412in}{2.004545in}}%
\pgfusepath{clip}%
\pgfsetbuttcap%
\pgfsetroundjoin%
\pgfsetlinewidth{0.704424pt}%
\definecolor{currentstroke}{rgb}{0.241237,0.296485,0.539709}%
\pgfsetstrokecolor{currentstroke}%
\pgfsetdash{}{0pt}%
\pgfpathmoveto{\pgfqpoint{7.392362in}{2.590386in}}%
\pgfpathlineto{\pgfqpoint{7.414502in}{2.605198in}}%
\pgfusepath{stroke}%
\end{pgfscope}%
\begin{pgfscope}%
\pgfpathrectangle{\pgfqpoint{6.720588in}{1.750000in}}{\pgfqpoint{2.279412in}{2.004545in}}%
\pgfusepath{clip}%
\pgfsetbuttcap%
\pgfsetroundjoin%
\pgfsetlinewidth{0.667022pt}%
\definecolor{currentstroke}{rgb}{0.250425,0.274290,0.533103}%
\pgfsetstrokecolor{currentstroke}%
\pgfsetdash{}{0pt}%
\pgfpathmoveto{\pgfqpoint{7.414502in}{2.605198in}}%
\pgfpathlineto{\pgfqpoint{7.439143in}{2.625248in}}%
\pgfusepath{stroke}%
\end{pgfscope}%
\begin{pgfscope}%
\pgfpathrectangle{\pgfqpoint{6.720588in}{1.750000in}}{\pgfqpoint{2.279412in}{2.004545in}}%
\pgfusepath{clip}%
\pgfsetbuttcap%
\pgfsetroundjoin%
\pgfsetlinewidth{0.631666pt}%
\definecolor{currentstroke}{rgb}{0.258965,0.251537,0.524736}%
\pgfsetstrokecolor{currentstroke}%
\pgfsetdash{}{0pt}%
\pgfpathmoveto{\pgfqpoint{7.439143in}{2.625248in}}%
\pgfpathlineto{\pgfqpoint{7.439143in}{2.625248in}}%
\pgfusepath{stroke}%
\end{pgfscope}%
\begin{pgfscope}%
\pgfpathrectangle{\pgfqpoint{6.720588in}{1.750000in}}{\pgfqpoint{2.279412in}{2.004545in}}%
\pgfusepath{clip}%
\pgfsetbuttcap%
\pgfsetroundjoin%
\pgfsetlinewidth{0.631666pt}%
\definecolor{currentstroke}{rgb}{0.258965,0.251537,0.524736}%
\pgfsetstrokecolor{currentstroke}%
\pgfsetdash{}{0pt}%
\pgfpathmoveto{\pgfqpoint{7.439143in}{2.625248in}}%
\pgfpathlineto{\pgfqpoint{7.453590in}{2.641912in}}%
\pgfusepath{stroke}%
\end{pgfscope}%
\begin{pgfscope}%
\pgfpathrectangle{\pgfqpoint{6.720588in}{1.750000in}}{\pgfqpoint{2.279412in}{2.004545in}}%
\pgfusepath{clip}%
\pgfsetbuttcap%
\pgfsetroundjoin%
\pgfsetlinewidth{0.600587pt}%
\definecolor{currentstroke}{rgb}{0.266580,0.228262,0.514349}%
\pgfsetstrokecolor{currentstroke}%
\pgfsetdash{}{0pt}%
\pgfpathmoveto{\pgfqpoint{7.453590in}{2.641912in}}%
\pgfpathlineto{\pgfqpoint{7.453590in}{2.641912in}}%
\pgfusepath{stroke}%
\end{pgfscope}%
\begin{pgfscope}%
\pgfpathrectangle{\pgfqpoint{6.720588in}{1.750000in}}{\pgfqpoint{2.279412in}{2.004545in}}%
\pgfusepath{clip}%
\pgfsetbuttcap%
\pgfsetroundjoin%
\pgfsetlinewidth{0.600587pt}%
\definecolor{currentstroke}{rgb}{0.266580,0.228262,0.514349}%
\pgfsetstrokecolor{currentstroke}%
\pgfsetdash{}{0pt}%
\pgfpathmoveto{\pgfqpoint{7.453590in}{2.641912in}}%
\pgfpathlineto{\pgfqpoint{7.457390in}{2.659299in}}%
\pgfusepath{stroke}%
\end{pgfscope}%
\begin{pgfscope}%
\pgfpathrectangle{\pgfqpoint{6.720588in}{1.750000in}}{\pgfqpoint{2.279412in}{2.004545in}}%
\pgfusepath{clip}%
\pgfsetbuttcap%
\pgfsetroundjoin%
\pgfsetlinewidth{0.515682pt}%
\definecolor{currentstroke}{rgb}{0.279574,0.170599,0.479997}%
\pgfsetstrokecolor{currentstroke}%
\pgfsetdash{}{0pt}%
\pgfpathmoveto{\pgfqpoint{7.457390in}{2.659299in}}%
\pgfpathlineto{\pgfqpoint{7.452367in}{2.676387in}}%
\pgfusepath{stroke}%
\end{pgfscope}%
\begin{pgfscope}%
\pgfpathrectangle{\pgfqpoint{6.720588in}{1.750000in}}{\pgfqpoint{2.279412in}{2.004545in}}%
\pgfusepath{clip}%
\pgfsetbuttcap%
\pgfsetroundjoin%
\pgfsetlinewidth{0.621999pt}%
\definecolor{currentstroke}{rgb}{0.262138,0.242286,0.520837}%
\pgfsetstrokecolor{currentstroke}%
\pgfsetdash{}{0pt}%
\pgfpathmoveto{\pgfqpoint{7.398667in}{2.436525in}}%
\pgfpathlineto{\pgfqpoint{7.442987in}{2.456176in}}%
\pgfusepath{stroke}%
\end{pgfscope}%
\begin{pgfscope}%
\pgfpathrectangle{\pgfqpoint{6.720588in}{1.750000in}}{\pgfqpoint{2.279412in}{2.004545in}}%
\pgfusepath{clip}%
\pgfsetbuttcap%
\pgfsetroundjoin%
\pgfsetlinewidth{0.628394pt}%
\definecolor{currentstroke}{rgb}{0.260571,0.246922,0.522828}%
\pgfsetstrokecolor{currentstroke}%
\pgfsetdash{}{0pt}%
\pgfpathmoveto{\pgfqpoint{7.442987in}{2.456176in}}%
\pgfpathlineto{\pgfqpoint{7.480582in}{2.483551in}}%
\pgfusepath{stroke}%
\end{pgfscope}%
\begin{pgfscope}%
\pgfpathrectangle{\pgfqpoint{6.720588in}{1.750000in}}{\pgfqpoint{2.279412in}{2.004545in}}%
\pgfusepath{clip}%
\pgfsetbuttcap%
\pgfsetroundjoin%
\pgfsetlinewidth{0.574882pt}%
\definecolor{currentstroke}{rgb}{0.271828,0.209303,0.504434}%
\pgfsetstrokecolor{currentstroke}%
\pgfsetdash{}{0pt}%
\pgfpathmoveto{\pgfqpoint{7.480582in}{2.483551in}}%
\pgfpathlineto{\pgfqpoint{7.480582in}{2.483551in}}%
\pgfusepath{stroke}%
\end{pgfscope}%
\begin{pgfscope}%
\pgfpathrectangle{\pgfqpoint{6.720588in}{1.750000in}}{\pgfqpoint{2.279412in}{2.004545in}}%
\pgfusepath{clip}%
\pgfsetbuttcap%
\pgfsetroundjoin%
\pgfsetlinewidth{0.574882pt}%
\definecolor{currentstroke}{rgb}{0.271828,0.209303,0.504434}%
\pgfsetstrokecolor{currentstroke}%
\pgfsetdash{}{0pt}%
\pgfpathmoveto{\pgfqpoint{7.480582in}{2.483551in}}%
\pgfpathlineto{\pgfqpoint{7.498690in}{2.506417in}}%
\pgfusepath{stroke}%
\end{pgfscope}%
\begin{pgfscope}%
\pgfpathrectangle{\pgfqpoint{6.720588in}{1.750000in}}{\pgfqpoint{2.279412in}{2.004545in}}%
\pgfusepath{clip}%
\pgfsetbuttcap%
\pgfsetroundjoin%
\pgfsetlinewidth{0.667721pt}%
\definecolor{currentstroke}{rgb}{0.250425,0.274290,0.533103}%
\pgfsetstrokecolor{currentstroke}%
\pgfsetdash{}{0pt}%
\pgfpathmoveto{\pgfqpoint{7.498690in}{2.506417in}}%
\pgfpathlineto{\pgfqpoint{7.507139in}{2.530272in}}%
\pgfusepath{stroke}%
\end{pgfscope}%
\begin{pgfscope}%
\pgfpathrectangle{\pgfqpoint{6.720588in}{1.750000in}}{\pgfqpoint{2.279412in}{2.004545in}}%
\pgfusepath{clip}%
\pgfsetbuttcap%
\pgfsetroundjoin%
\pgfsetlinewidth{0.612854pt}%
\definecolor{currentstroke}{rgb}{0.263663,0.237631,0.518762}%
\pgfsetstrokecolor{currentstroke}%
\pgfsetdash{}{0pt}%
\pgfpathmoveto{\pgfqpoint{7.507139in}{2.530272in}}%
\pgfpathlineto{\pgfqpoint{7.510855in}{2.560200in}}%
\pgfusepath{stroke}%
\end{pgfscope}%
\begin{pgfscope}%
\pgfpathrectangle{\pgfqpoint{6.720588in}{1.750000in}}{\pgfqpoint{2.279412in}{2.004545in}}%
\pgfusepath{clip}%
\pgfsetbuttcap%
\pgfsetroundjoin%
\pgfsetlinewidth{0.626117pt}%
\definecolor{currentstroke}{rgb}{0.260571,0.246922,0.522828}%
\pgfsetstrokecolor{currentstroke}%
\pgfsetdash{}{0pt}%
\pgfpathmoveto{\pgfqpoint{7.510855in}{2.560200in}}%
\pgfpathlineto{\pgfqpoint{7.504717in}{2.590762in}}%
\pgfusepath{stroke}%
\end{pgfscope}%
\begin{pgfscope}%
\pgfpathrectangle{\pgfqpoint{6.720588in}{1.750000in}}{\pgfqpoint{2.279412in}{2.004545in}}%
\pgfusepath{clip}%
\pgfsetbuttcap%
\pgfsetroundjoin%
\pgfsetlinewidth{0.570805pt}%
\definecolor{currentstroke}{rgb}{0.271828,0.209303,0.504434}%
\pgfsetstrokecolor{currentstroke}%
\pgfsetdash{}{0pt}%
\pgfpathmoveto{\pgfqpoint{7.405053in}{2.372337in}}%
\pgfpathlineto{\pgfqpoint{7.449959in}{2.391418in}}%
\pgfusepath{stroke}%
\end{pgfscope}%
\begin{pgfscope}%
\pgfpathrectangle{\pgfqpoint{6.720588in}{1.750000in}}{\pgfqpoint{2.279412in}{2.004545in}}%
\pgfusepath{clip}%
\pgfsetbuttcap%
\pgfsetroundjoin%
\pgfsetlinewidth{0.638772pt}%
\definecolor{currentstroke}{rgb}{0.257322,0.256130,0.526563}%
\pgfsetstrokecolor{currentstroke}%
\pgfsetdash{}{0pt}%
\pgfpathmoveto{\pgfqpoint{7.449959in}{2.391418in}}%
\pgfpathlineto{\pgfqpoint{7.492290in}{2.414568in}}%
\pgfusepath{stroke}%
\end{pgfscope}%
\begin{pgfscope}%
\pgfpathrectangle{\pgfqpoint{6.720588in}{1.750000in}}{\pgfqpoint{2.279412in}{2.004545in}}%
\pgfusepath{clip}%
\pgfsetbuttcap%
\pgfsetroundjoin%
\pgfsetlinewidth{0.551090pt}%
\definecolor{currentstroke}{rgb}{0.275191,0.194905,0.496005}%
\pgfsetstrokecolor{currentstroke}%
\pgfsetdash{}{0pt}%
\pgfpathmoveto{\pgfqpoint{7.492290in}{2.414568in}}%
\pgfpathlineto{\pgfqpoint{7.492290in}{2.414568in}}%
\pgfusepath{stroke}%
\end{pgfscope}%
\begin{pgfscope}%
\pgfpathrectangle{\pgfqpoint{6.720588in}{1.750000in}}{\pgfqpoint{2.279412in}{2.004545in}}%
\pgfusepath{clip}%
\pgfsetbuttcap%
\pgfsetroundjoin%
\pgfsetlinewidth{0.551090pt}%
\definecolor{currentstroke}{rgb}{0.275191,0.194905,0.496005}%
\pgfsetstrokecolor{currentstroke}%
\pgfsetdash{}{0pt}%
\pgfpathmoveto{\pgfqpoint{7.492290in}{2.414568in}}%
\pgfpathlineto{\pgfqpoint{7.522067in}{2.435565in}}%
\pgfusepath{stroke}%
\end{pgfscope}%
\begin{pgfscope}%
\pgfpathrectangle{\pgfqpoint{6.720588in}{1.750000in}}{\pgfqpoint{2.279412in}{2.004545in}}%
\pgfusepath{clip}%
\pgfsetbuttcap%
\pgfsetroundjoin%
\pgfsetlinewidth{0.494696pt}%
\definecolor{currentstroke}{rgb}{0.281412,0.155834,0.469201}%
\pgfsetstrokecolor{currentstroke}%
\pgfsetdash{}{0pt}%
\pgfpathmoveto{\pgfqpoint{7.522067in}{2.435565in}}%
\pgfpathlineto{\pgfqpoint{7.522067in}{2.435565in}}%
\pgfusepath{stroke}%
\end{pgfscope}%
\begin{pgfscope}%
\pgfpathrectangle{\pgfqpoint{6.720588in}{1.750000in}}{\pgfqpoint{2.279412in}{2.004545in}}%
\pgfusepath{clip}%
\pgfsetbuttcap%
\pgfsetroundjoin%
\pgfsetlinewidth{0.494696pt}%
\definecolor{currentstroke}{rgb}{0.281412,0.155834,0.469201}%
\pgfsetstrokecolor{currentstroke}%
\pgfsetdash{}{0pt}%
\pgfpathmoveto{\pgfqpoint{7.522067in}{2.435565in}}%
\pgfpathlineto{\pgfqpoint{7.539731in}{2.455824in}}%
\pgfusepath{stroke}%
\end{pgfscope}%
\begin{pgfscope}%
\pgfpathrectangle{\pgfqpoint{6.720588in}{1.750000in}}{\pgfqpoint{2.279412in}{2.004545in}}%
\pgfusepath{clip}%
\pgfsetbuttcap%
\pgfsetroundjoin%
\pgfsetlinewidth{0.555094pt}%
\definecolor{currentstroke}{rgb}{0.274128,0.199721,0.498911}%
\pgfsetstrokecolor{currentstroke}%
\pgfsetdash{}{0pt}%
\pgfpathmoveto{\pgfqpoint{7.539731in}{2.455824in}}%
\pgfpathlineto{\pgfqpoint{7.548420in}{2.479495in}}%
\pgfusepath{stroke}%
\end{pgfscope}%
\begin{pgfscope}%
\pgfpathrectangle{\pgfqpoint{6.720588in}{1.750000in}}{\pgfqpoint{2.279412in}{2.004545in}}%
\pgfusepath{clip}%
\pgfsetbuttcap%
\pgfsetroundjoin%
\pgfsetlinewidth{0.630657pt}%
\definecolor{currentstroke}{rgb}{0.258965,0.251537,0.524736}%
\pgfsetstrokecolor{currentstroke}%
\pgfsetdash{}{0pt}%
\pgfpathmoveto{\pgfqpoint{7.548420in}{2.479495in}}%
\pgfpathlineto{\pgfqpoint{7.549195in}{2.502568in}}%
\pgfusepath{stroke}%
\end{pgfscope}%
\begin{pgfscope}%
\pgfpathrectangle{\pgfqpoint{6.720588in}{1.750000in}}{\pgfqpoint{2.279412in}{2.004545in}}%
\pgfusepath{clip}%
\pgfsetbuttcap%
\pgfsetroundjoin%
\pgfsetlinewidth{0.637273pt}%
\definecolor{currentstroke}{rgb}{0.258965,0.251537,0.524736}%
\pgfsetstrokecolor{currentstroke}%
\pgfsetdash{}{0pt}%
\pgfpathmoveto{\pgfqpoint{7.549195in}{2.502568in}}%
\pgfpathlineto{\pgfqpoint{7.541882in}{2.533203in}}%
\pgfusepath{stroke}%
\end{pgfscope}%
\begin{pgfscope}%
\pgfpathrectangle{\pgfqpoint{6.720588in}{1.750000in}}{\pgfqpoint{2.279412in}{2.004545in}}%
\pgfusepath{clip}%
\pgfsetbuttcap%
\pgfsetroundjoin%
\pgfsetlinewidth{0.556971pt}%
\definecolor{currentstroke}{rgb}{0.274128,0.199721,0.498911}%
\pgfsetstrokecolor{currentstroke}%
\pgfsetdash{}{0pt}%
\pgfpathmoveto{\pgfqpoint{7.470828in}{3.042106in}}%
\pgfpathlineto{\pgfqpoint{7.501251in}{3.022913in}}%
\pgfusepath{stroke}%
\end{pgfscope}%
\begin{pgfscope}%
\pgfpathrectangle{\pgfqpoint{6.720588in}{1.750000in}}{\pgfqpoint{2.279412in}{2.004545in}}%
\pgfusepath{clip}%
\pgfsetbuttcap%
\pgfsetroundjoin%
\pgfsetlinewidth{0.625852pt}%
\definecolor{currentstroke}{rgb}{0.260571,0.246922,0.522828}%
\pgfsetstrokecolor{currentstroke}%
\pgfsetdash{}{0pt}%
\pgfpathmoveto{\pgfqpoint{7.501251in}{3.022913in}}%
\pgfpathlineto{\pgfqpoint{7.501251in}{3.022913in}}%
\pgfusepath{stroke}%
\end{pgfscope}%
\begin{pgfscope}%
\pgfpathrectangle{\pgfqpoint{6.720588in}{1.750000in}}{\pgfqpoint{2.279412in}{2.004545in}}%
\pgfusepath{clip}%
\pgfsetbuttcap%
\pgfsetroundjoin%
\pgfsetlinewidth{0.625852pt}%
\definecolor{currentstroke}{rgb}{0.260571,0.246922,0.522828}%
\pgfsetstrokecolor{currentstroke}%
\pgfsetdash{}{0pt}%
\pgfpathmoveto{\pgfqpoint{7.501251in}{3.022913in}}%
\pgfpathlineto{\pgfqpoint{7.501251in}{3.022913in}}%
\pgfusepath{stroke}%
\end{pgfscope}%
\begin{pgfscope}%
\pgfpathrectangle{\pgfqpoint{6.720588in}{1.750000in}}{\pgfqpoint{2.279412in}{2.004545in}}%
\pgfusepath{clip}%
\pgfsetbuttcap%
\pgfsetroundjoin%
\pgfsetlinewidth{0.625852pt}%
\definecolor{currentstroke}{rgb}{0.260571,0.246922,0.522828}%
\pgfsetstrokecolor{currentstroke}%
\pgfsetdash{}{0pt}%
\pgfpathmoveto{\pgfqpoint{7.501251in}{3.022913in}}%
\pgfpathlineto{\pgfqpoint{7.522373in}{3.000319in}}%
\pgfusepath{stroke}%
\end{pgfscope}%
\begin{pgfscope}%
\pgfpathrectangle{\pgfqpoint{6.720588in}{1.750000in}}{\pgfqpoint{2.279412in}{2.004545in}}%
\pgfusepath{clip}%
\pgfsetbuttcap%
\pgfsetroundjoin%
\pgfsetlinewidth{0.596013pt}%
\definecolor{currentstroke}{rgb}{0.267968,0.223549,0.512008}%
\pgfsetstrokecolor{currentstroke}%
\pgfsetdash{}{0pt}%
\pgfpathmoveto{\pgfqpoint{7.522373in}{3.000319in}}%
\pgfpathlineto{\pgfqpoint{7.522373in}{3.000319in}}%
\pgfusepath{stroke}%
\end{pgfscope}%
\begin{pgfscope}%
\pgfpathrectangle{\pgfqpoint{6.720588in}{1.750000in}}{\pgfqpoint{2.279412in}{2.004545in}}%
\pgfusepath{clip}%
\pgfsetbuttcap%
\pgfsetroundjoin%
\pgfsetlinewidth{0.596013pt}%
\definecolor{currentstroke}{rgb}{0.267968,0.223549,0.512008}%
\pgfsetstrokecolor{currentstroke}%
\pgfsetdash{}{0pt}%
\pgfpathmoveto{\pgfqpoint{7.522373in}{3.000319in}}%
\pgfpathlineto{\pgfqpoint{7.533440in}{2.980310in}}%
\pgfusepath{stroke}%
\end{pgfscope}%
\begin{pgfscope}%
\pgfpathrectangle{\pgfqpoint{6.720588in}{1.750000in}}{\pgfqpoint{2.279412in}{2.004545in}}%
\pgfusepath{clip}%
\pgfsetbuttcap%
\pgfsetroundjoin%
\pgfsetlinewidth{0.613832pt}%
\definecolor{currentstroke}{rgb}{0.263663,0.237631,0.518762}%
\pgfsetstrokecolor{currentstroke}%
\pgfsetdash{}{0pt}%
\pgfpathmoveto{\pgfqpoint{7.533440in}{2.980310in}}%
\pgfpathlineto{\pgfqpoint{7.537833in}{2.958789in}}%
\pgfusepath{stroke}%
\end{pgfscope}%
\begin{pgfscope}%
\pgfpathrectangle{\pgfqpoint{6.720588in}{1.750000in}}{\pgfqpoint{2.279412in}{2.004545in}}%
\pgfusepath{clip}%
\pgfsetbuttcap%
\pgfsetroundjoin%
\pgfsetlinewidth{0.638816pt}%
\definecolor{currentstroke}{rgb}{0.257322,0.256130,0.526563}%
\pgfsetstrokecolor{currentstroke}%
\pgfsetdash{}{0pt}%
\pgfpathmoveto{\pgfqpoint{7.537833in}{2.958789in}}%
\pgfpathlineto{\pgfqpoint{7.535814in}{2.936450in}}%
\pgfusepath{stroke}%
\end{pgfscope}%
\begin{pgfscope}%
\pgfpathrectangle{\pgfqpoint{6.720588in}{1.750000in}}{\pgfqpoint{2.279412in}{2.004545in}}%
\pgfusepath{clip}%
\pgfsetbuttcap%
\pgfsetroundjoin%
\pgfsetlinewidth{0.641897pt}%
\definecolor{currentstroke}{rgb}{0.257322,0.256130,0.526563}%
\pgfsetstrokecolor{currentstroke}%
\pgfsetdash{}{0pt}%
\pgfpathmoveto{\pgfqpoint{7.535814in}{2.936450in}}%
\pgfpathlineto{\pgfqpoint{7.531354in}{2.910832in}}%
\pgfusepath{stroke}%
\end{pgfscope}%
\begin{pgfscope}%
\pgfpathrectangle{\pgfqpoint{6.720588in}{1.750000in}}{\pgfqpoint{2.279412in}{2.004545in}}%
\pgfusepath{clip}%
\pgfsetbuttcap%
\pgfsetroundjoin%
\pgfsetlinewidth{0.587531pt}%
\definecolor{currentstroke}{rgb}{0.269308,0.218818,0.509577}%
\pgfsetstrokecolor{currentstroke}%
\pgfsetdash{}{0pt}%
\pgfpathmoveto{\pgfqpoint{7.531354in}{2.910832in}}%
\pgfpathlineto{\pgfqpoint{7.531354in}{2.910832in}}%
\pgfusepath{stroke}%
\end{pgfscope}%
\begin{pgfscope}%
\pgfpathrectangle{\pgfqpoint{6.720588in}{1.750000in}}{\pgfqpoint{2.279412in}{2.004545in}}%
\pgfusepath{clip}%
\pgfsetbuttcap%
\pgfsetroundjoin%
\pgfsetlinewidth{0.587531pt}%
\definecolor{currentstroke}{rgb}{0.269308,0.218818,0.509577}%
\pgfsetstrokecolor{currentstroke}%
\pgfsetdash{}{0pt}%
\pgfpathmoveto{\pgfqpoint{7.531354in}{2.910832in}}%
\pgfpathlineto{\pgfqpoint{7.521548in}{2.884682in}}%
\pgfusepath{stroke}%
\end{pgfscope}%
\begin{pgfscope}%
\pgfpathrectangle{\pgfqpoint{6.720588in}{1.750000in}}{\pgfqpoint{2.279412in}{2.004545in}}%
\pgfusepath{clip}%
\pgfsetbuttcap%
\pgfsetroundjoin%
\pgfsetlinewidth{0.578335pt}%
\definecolor{currentstroke}{rgb}{0.270595,0.214069,0.507052}%
\pgfsetstrokecolor{currentstroke}%
\pgfsetdash{}{0pt}%
\pgfpathmoveto{\pgfqpoint{7.521548in}{2.884682in}}%
\pgfpathlineto{\pgfqpoint{7.510363in}{2.860787in}}%
\pgfusepath{stroke}%
\end{pgfscope}%
\begin{pgfscope}%
\pgfpathrectangle{\pgfqpoint{6.720588in}{1.750000in}}{\pgfqpoint{2.279412in}{2.004545in}}%
\pgfusepath{clip}%
\pgfsetbuttcap%
\pgfsetroundjoin%
\pgfsetlinewidth{0.530358pt}%
\definecolor{currentstroke}{rgb}{0.278012,0.180367,0.486697}%
\pgfsetstrokecolor{currentstroke}%
\pgfsetdash{}{0pt}%
\pgfpathmoveto{\pgfqpoint{7.510363in}{2.860787in}}%
\pgfpathlineto{\pgfqpoint{7.487307in}{2.823570in}}%
\pgfusepath{stroke}%
\end{pgfscope}%
\begin{pgfscope}%
\pgfpathrectangle{\pgfqpoint{6.720588in}{1.750000in}}{\pgfqpoint{2.279412in}{2.004545in}}%
\pgfusepath{clip}%
\pgfsetbuttcap%
\pgfsetroundjoin%
\pgfsetlinewidth{0.525488pt}%
\definecolor{currentstroke}{rgb}{0.278826,0.175490,0.483397}%
\pgfsetstrokecolor{currentstroke}%
\pgfsetdash{}{0pt}%
\pgfpathmoveto{\pgfqpoint{7.487307in}{2.823570in}}%
\pgfpathlineto{\pgfqpoint{7.487307in}{2.823570in}}%
\pgfusepath{stroke}%
\end{pgfscope}%
\begin{pgfscope}%
\pgfpathrectangle{\pgfqpoint{6.720588in}{1.750000in}}{\pgfqpoint{2.279412in}{2.004545in}}%
\pgfusepath{clip}%
\pgfsetroundcap%
\pgfsetroundjoin%
\pgfsetlinewidth{0.308875pt}%
\definecolor{currentstroke}{rgb}{0.268510,0.009605,0.335427}%
\pgfsetstrokecolor{currentstroke}%
\pgfsetdash{}{0pt}%
\pgfpathmoveto{\pgfqpoint{8.427829in}{1.852932in}}%
\pgfpathquadraticcurveto{\pgfqpoint{8.427767in}{1.852935in}}{\pgfqpoint{8.432478in}{1.852709in}}%
\pgfusepath{stroke}%
\end{pgfscope}%
\begin{pgfscope}%
\pgfpathrectangle{\pgfqpoint{6.720588in}{1.750000in}}{\pgfqpoint{2.279412in}{2.004545in}}%
\pgfusepath{clip}%
\pgfsetroundcap%
\pgfsetroundjoin%
\definecolor{currentfill}{rgb}{0.268510,0.009605,0.335427}%
\pgfsetfillcolor{currentfill}%
\pgfsetlinewidth{0.308875pt}%
\definecolor{currentstroke}{rgb}{0.268510,0.009605,0.335427}%
\pgfsetstrokecolor{currentstroke}%
\pgfsetdash{}{0pt}%
\pgfpathmoveto{\pgfqpoint{8.486644in}{1.822312in}}%
\pgfpathlineto{\pgfqpoint{8.432478in}{1.852709in}}%
\pgfpathlineto{\pgfqpoint{8.489296in}{1.877804in}}%
\pgfpathlineto{\pgfqpoint{8.486644in}{1.822312in}}%
\pgfpathlineto{\pgfqpoint{8.486644in}{1.822312in}}%
\pgfpathclose%
\pgfusepath{stroke,fill}%
\end{pgfscope}%
\begin{pgfscope}%
\pgfpathrectangle{\pgfqpoint{6.720588in}{1.750000in}}{\pgfqpoint{2.279412in}{2.004545in}}%
\pgfusepath{clip}%
\pgfsetroundcap%
\pgfsetroundjoin%
\pgfsetlinewidth{0.892541pt}%
\definecolor{currentstroke}{rgb}{0.188923,0.410910,0.556326}%
\pgfsetstrokecolor{currentstroke}%
\pgfsetdash{}{0pt}%
\pgfpathmoveto{\pgfqpoint{7.238779in}{2.743939in}}%
\pgfpathquadraticcurveto{\pgfqpoint{7.251306in}{2.743617in}}{\pgfqpoint{7.250029in}{2.743649in}}%
\pgfusepath{stroke}%
\end{pgfscope}%
\begin{pgfscope}%
\pgfpathrectangle{\pgfqpoint{6.720588in}{1.750000in}}{\pgfqpoint{2.279412in}{2.004545in}}%
\pgfusepath{clip}%
\pgfsetroundcap%
\pgfsetroundjoin%
\definecolor{currentfill}{rgb}{0.188923,0.410910,0.556326}%
\pgfsetfillcolor{currentfill}%
\pgfsetlinewidth{0.892541pt}%
\definecolor{currentstroke}{rgb}{0.188923,0.410910,0.556326}%
\pgfsetstrokecolor{currentstroke}%
\pgfsetdash{}{0pt}%
\pgfpathmoveto{\pgfqpoint{7.195206in}{2.772846in}}%
\pgfpathlineto{\pgfqpoint{7.250029in}{2.743649in}}%
\pgfpathlineto{\pgfqpoint{7.193778in}{2.717309in}}%
\pgfpathlineto{\pgfqpoint{7.195206in}{2.772846in}}%
\pgfpathlineto{\pgfqpoint{7.195206in}{2.772846in}}%
\pgfpathclose%
\pgfusepath{stroke,fill}%
\end{pgfscope}%
\begin{pgfscope}%
\pgfpathrectangle{\pgfqpoint{6.720588in}{1.750000in}}{\pgfqpoint{2.279412in}{2.004545in}}%
\pgfusepath{clip}%
\pgfsetroundcap%
\pgfsetroundjoin%
\pgfsetlinewidth{0.713054pt}%
\definecolor{currentstroke}{rgb}{0.237441,0.305202,0.541921}%
\pgfsetstrokecolor{currentstroke}%
\pgfsetdash{}{0pt}%
\pgfpathmoveto{\pgfqpoint{8.080706in}{2.579677in}}%
\pgfpathquadraticcurveto{\pgfqpoint{8.068187in}{2.580280in}}{\pgfqpoint{8.066687in}{2.580352in}}%
\pgfusepath{stroke}%
\end{pgfscope}%
\begin{pgfscope}%
\pgfpathrectangle{\pgfqpoint{6.720588in}{1.750000in}}{\pgfqpoint{2.279412in}{2.004545in}}%
\pgfusepath{clip}%
\pgfsetroundcap%
\pgfsetroundjoin%
\definecolor{currentfill}{rgb}{0.237441,0.305202,0.541921}%
\pgfsetfillcolor{currentfill}%
\pgfsetlinewidth{0.713054pt}%
\definecolor{currentstroke}{rgb}{0.237441,0.305202,0.541921}%
\pgfsetstrokecolor{currentstroke}%
\pgfsetdash{}{0pt}%
\pgfpathmoveto{\pgfqpoint{8.120841in}{2.549933in}}%
\pgfpathlineto{\pgfqpoint{8.066687in}{2.580352in}}%
\pgfpathlineto{\pgfqpoint{8.123515in}{2.605424in}}%
\pgfpathlineto{\pgfqpoint{8.120841in}{2.549933in}}%
\pgfpathlineto{\pgfqpoint{8.120841in}{2.549933in}}%
\pgfpathclose%
\pgfusepath{stroke,fill}%
\end{pgfscope}%
\begin{pgfscope}%
\pgfpathrectangle{\pgfqpoint{6.720588in}{1.750000in}}{\pgfqpoint{2.279412in}{2.004545in}}%
\pgfusepath{clip}%
\pgfsetroundcap%
\pgfsetroundjoin%
\pgfsetlinewidth{0.434315pt}%
\definecolor{currentstroke}{rgb}{0.283091,0.110553,0.431554}%
\pgfsetstrokecolor{currentstroke}%
\pgfsetdash{}{0pt}%
\pgfpathmoveto{\pgfqpoint{8.283150in}{2.614828in}}%
\pgfpathquadraticcurveto{\pgfqpoint{8.270614in}{2.615010in}}{\pgfqpoint{8.264796in}{2.615095in}}%
\pgfusepath{stroke}%
\end{pgfscope}%
\begin{pgfscope}%
\pgfpathrectangle{\pgfqpoint{6.720588in}{1.750000in}}{\pgfqpoint{2.279412in}{2.004545in}}%
\pgfusepath{clip}%
\pgfsetroundcap%
\pgfsetroundjoin%
\definecolor{currentfill}{rgb}{0.283091,0.110553,0.431554}%
\pgfsetfillcolor{currentfill}%
\pgfsetlinewidth{0.434315pt}%
\definecolor{currentstroke}{rgb}{0.283091,0.110553,0.431554}%
\pgfsetstrokecolor{currentstroke}%
\pgfsetdash{}{0pt}%
\pgfpathmoveto{\pgfqpoint{8.319942in}{2.586513in}}%
\pgfpathlineto{\pgfqpoint{8.264796in}{2.615095in}}%
\pgfpathlineto{\pgfqpoint{8.320750in}{2.642062in}}%
\pgfpathlineto{\pgfqpoint{8.319942in}{2.586513in}}%
\pgfpathlineto{\pgfqpoint{8.319942in}{2.586513in}}%
\pgfpathclose%
\pgfusepath{stroke,fill}%
\end{pgfscope}%
\begin{pgfscope}%
\pgfpathrectangle{\pgfqpoint{6.720588in}{1.750000in}}{\pgfqpoint{2.279412in}{2.004545in}}%
\pgfusepath{clip}%
\pgfsetroundcap%
\pgfsetroundjoin%
\pgfsetlinewidth{1.074970pt}%
\definecolor{currentstroke}{rgb}{0.149039,0.508051,0.557250}%
\pgfsetstrokecolor{currentstroke}%
\pgfsetdash{}{0pt}%
\pgfpathmoveto{\pgfqpoint{7.190001in}{2.710582in}}%
\pgfpathquadraticcurveto{\pgfqpoint{7.202530in}{2.710940in}}{\pgfqpoint{7.198436in}{2.710823in}}%
\pgfusepath{stroke}%
\end{pgfscope}%
\begin{pgfscope}%
\pgfpathrectangle{\pgfqpoint{6.720588in}{1.750000in}}{\pgfqpoint{2.279412in}{2.004545in}}%
\pgfusepath{clip}%
\pgfsetroundcap%
\pgfsetroundjoin%
\definecolor{currentfill}{rgb}{0.149039,0.508051,0.557250}%
\pgfsetfillcolor{currentfill}%
\pgfsetlinewidth{1.074970pt}%
\definecolor{currentstroke}{rgb}{0.149039,0.508051,0.557250}%
\pgfsetstrokecolor{currentstroke}%
\pgfsetdash{}{0pt}%
\pgfpathmoveto{\pgfqpoint{7.142109in}{2.737001in}}%
\pgfpathlineto{\pgfqpoint{7.198436in}{2.710823in}}%
\pgfpathlineto{\pgfqpoint{7.143697in}{2.681469in}}%
\pgfpathlineto{\pgfqpoint{7.142109in}{2.737001in}}%
\pgfpathlineto{\pgfqpoint{7.142109in}{2.737001in}}%
\pgfpathclose%
\pgfusepath{stroke,fill}%
\end{pgfscope}%
\begin{pgfscope}%
\pgfpathrectangle{\pgfqpoint{6.720588in}{1.750000in}}{\pgfqpoint{2.279412in}{2.004545in}}%
\pgfusepath{clip}%
\pgfsetroundcap%
\pgfsetroundjoin%
\pgfsetlinewidth{0.340275pt}%
\definecolor{currentstroke}{rgb}{0.273809,0.031497,0.358853}%
\pgfsetstrokecolor{currentstroke}%
\pgfsetdash{}{0pt}%
\pgfpathmoveto{\pgfqpoint{8.323257in}{2.078301in}}%
\pgfpathquadraticcurveto{\pgfqpoint{8.310743in}{2.078562in}}{\pgfqpoint{8.303493in}{2.078714in}}%
\pgfusepath{stroke}%
\end{pgfscope}%
\begin{pgfscope}%
\pgfpathrectangle{\pgfqpoint{6.720588in}{1.750000in}}{\pgfqpoint{2.279412in}{2.004545in}}%
\pgfusepath{clip}%
\pgfsetroundcap%
\pgfsetroundjoin%
\definecolor{currentfill}{rgb}{0.273809,0.031497,0.358853}%
\pgfsetfillcolor{currentfill}%
\pgfsetlinewidth{0.340275pt}%
\definecolor{currentstroke}{rgb}{0.273809,0.031497,0.358853}%
\pgfsetstrokecolor{currentstroke}%
\pgfsetdash{}{0pt}%
\pgfpathmoveto{\pgfqpoint{8.358456in}{2.049780in}}%
\pgfpathlineto{\pgfqpoint{8.303493in}{2.078714in}}%
\pgfpathlineto{\pgfqpoint{8.359618in}{2.105324in}}%
\pgfpathlineto{\pgfqpoint{8.358456in}{2.049780in}}%
\pgfpathlineto{\pgfqpoint{8.358456in}{2.049780in}}%
\pgfpathclose%
\pgfusepath{stroke,fill}%
\end{pgfscope}%
\begin{pgfscope}%
\pgfpathrectangle{\pgfqpoint{6.720588in}{1.750000in}}{\pgfqpoint{2.279412in}{2.004545in}}%
\pgfusepath{clip}%
\pgfsetroundcap%
\pgfsetroundjoin%
\pgfsetlinewidth{0.755468pt}%
\definecolor{currentstroke}{rgb}{0.225863,0.330805,0.547314}%
\pgfsetstrokecolor{currentstroke}%
\pgfsetdash{}{0pt}%
\pgfpathmoveto{\pgfqpoint{8.107822in}{2.703477in}}%
\pgfpathquadraticcurveto{\pgfqpoint{8.095284in}{2.703465in}}{\pgfqpoint{8.094433in}{2.703464in}}%
\pgfusepath{stroke}%
\end{pgfscope}%
\begin{pgfscope}%
\pgfpathrectangle{\pgfqpoint{6.720588in}{1.750000in}}{\pgfqpoint{2.279412in}{2.004545in}}%
\pgfusepath{clip}%
\pgfsetroundcap%
\pgfsetroundjoin%
\definecolor{currentfill}{rgb}{0.225863,0.330805,0.547314}%
\pgfsetfillcolor{currentfill}%
\pgfsetlinewidth{0.755468pt}%
\definecolor{currentstroke}{rgb}{0.225863,0.330805,0.547314}%
\pgfsetstrokecolor{currentstroke}%
\pgfsetdash{}{0pt}%
\pgfpathmoveto{\pgfqpoint{8.150015in}{2.675739in}}%
\pgfpathlineto{\pgfqpoint{8.094433in}{2.703464in}}%
\pgfpathlineto{\pgfqpoint{8.149963in}{2.731294in}}%
\pgfpathlineto{\pgfqpoint{8.150015in}{2.675739in}}%
\pgfpathlineto{\pgfqpoint{8.150015in}{2.675739in}}%
\pgfpathclose%
\pgfusepath{stroke,fill}%
\end{pgfscope}%
\begin{pgfscope}%
\pgfpathrectangle{\pgfqpoint{6.720588in}{1.750000in}}{\pgfqpoint{2.279412in}{2.004545in}}%
\pgfusepath{clip}%
\pgfsetroundcap%
\pgfsetroundjoin%
\pgfsetlinewidth{0.778730pt}%
\definecolor{currentstroke}{rgb}{0.220057,0.343307,0.549413}%
\pgfsetstrokecolor{currentstroke}%
\pgfsetdash{}{0pt}%
\pgfpathmoveto{\pgfqpoint{8.078148in}{2.748026in}}%
\pgfpathquadraticcurveto{\pgfqpoint{8.065611in}{2.747909in}}{\pgfqpoint{8.065120in}{2.747904in}}%
\pgfusepath{stroke}%
\end{pgfscope}%
\begin{pgfscope}%
\pgfpathrectangle{\pgfqpoint{6.720588in}{1.750000in}}{\pgfqpoint{2.279412in}{2.004545in}}%
\pgfusepath{clip}%
\pgfsetroundcap%
\pgfsetroundjoin%
\definecolor{currentfill}{rgb}{0.220057,0.343307,0.549413}%
\pgfsetfillcolor{currentfill}%
\pgfsetlinewidth{0.778730pt}%
\definecolor{currentstroke}{rgb}{0.220057,0.343307,0.549413}%
\pgfsetstrokecolor{currentstroke}%
\pgfsetdash{}{0pt}%
\pgfpathmoveto{\pgfqpoint{8.120933in}{2.720647in}}%
\pgfpathlineto{\pgfqpoint{8.065120in}{2.747904in}}%
\pgfpathlineto{\pgfqpoint{8.120414in}{2.776200in}}%
\pgfpathlineto{\pgfqpoint{8.120933in}{2.720647in}}%
\pgfpathlineto{\pgfqpoint{8.120933in}{2.720647in}}%
\pgfpathclose%
\pgfusepath{stroke,fill}%
\end{pgfscope}%
\begin{pgfscope}%
\pgfpathrectangle{\pgfqpoint{6.720588in}{1.750000in}}{\pgfqpoint{2.279412in}{2.004545in}}%
\pgfusepath{clip}%
\pgfsetroundcap%
\pgfsetroundjoin%
\pgfsetlinewidth{0.704879pt}%
\definecolor{currentstroke}{rgb}{0.241237,0.296485,0.539709}%
\pgfsetstrokecolor{currentstroke}%
\pgfsetdash{}{0pt}%
\pgfpathmoveto{\pgfqpoint{7.989456in}{2.922487in}}%
\pgfpathquadraticcurveto{\pgfqpoint{7.976972in}{2.921480in}}{\pgfqpoint{7.975357in}{2.921350in}}%
\pgfusepath{stroke}%
\end{pgfscope}%
\begin{pgfscope}%
\pgfpathrectangle{\pgfqpoint{6.720588in}{1.750000in}}{\pgfqpoint{2.279412in}{2.004545in}}%
\pgfusepath{clip}%
\pgfsetroundcap%
\pgfsetroundjoin%
\definecolor{currentfill}{rgb}{0.241237,0.296485,0.539709}%
\pgfsetfillcolor{currentfill}%
\pgfsetlinewidth{0.704879pt}%
\definecolor{currentstroke}{rgb}{0.241237,0.296485,0.539709}%
\pgfsetstrokecolor{currentstroke}%
\pgfsetdash{}{0pt}%
\pgfpathmoveto{\pgfqpoint{8.032966in}{2.898129in}}%
\pgfpathlineto{\pgfqpoint{7.975357in}{2.921350in}}%
\pgfpathlineto{\pgfqpoint{8.028499in}{2.953505in}}%
\pgfpathlineto{\pgfqpoint{8.032966in}{2.898129in}}%
\pgfpathlineto{\pgfqpoint{8.032966in}{2.898129in}}%
\pgfpathclose%
\pgfusepath{stroke,fill}%
\end{pgfscope}%
\begin{pgfscope}%
\pgfpathrectangle{\pgfqpoint{6.720588in}{1.750000in}}{\pgfqpoint{2.279412in}{2.004545in}}%
\pgfusepath{clip}%
\pgfsetroundcap%
\pgfsetroundjoin%
\pgfsetlinewidth{0.429444pt}%
\definecolor{currentstroke}{rgb}{0.282910,0.105393,0.426902}%
\pgfsetstrokecolor{currentstroke}%
\pgfsetdash{}{0pt}%
\pgfpathmoveto{\pgfqpoint{7.707634in}{2.348865in}}%
\pgfpathquadraticcurveto{\pgfqpoint{7.702744in}{2.358770in}}{\pgfqpoint{7.700795in}{2.362719in}}%
\pgfusepath{stroke}%
\end{pgfscope}%
\begin{pgfscope}%
\pgfpathrectangle{\pgfqpoint{6.720588in}{1.750000in}}{\pgfqpoint{2.279412in}{2.004545in}}%
\pgfusepath{clip}%
\pgfsetroundcap%
\pgfsetroundjoin%
\definecolor{currentfill}{rgb}{0.282910,0.105393,0.426902}%
\pgfsetfillcolor{currentfill}%
\pgfsetlinewidth{0.429444pt}%
\definecolor{currentstroke}{rgb}{0.282910,0.105393,0.426902}%
\pgfsetstrokecolor{currentstroke}%
\pgfsetdash{}{0pt}%
\pgfpathmoveto{\pgfqpoint{7.700480in}{2.300606in}}%
\pgfpathlineto{\pgfqpoint{7.700795in}{2.362719in}}%
\pgfpathlineto{\pgfqpoint{7.750296in}{2.325200in}}%
\pgfpathlineto{\pgfqpoint{7.700480in}{2.300606in}}%
\pgfpathlineto{\pgfqpoint{7.700480in}{2.300606in}}%
\pgfpathclose%
\pgfusepath{stroke,fill}%
\end{pgfscope}%
\begin{pgfscope}%
\pgfpathrectangle{\pgfqpoint{6.720588in}{1.750000in}}{\pgfqpoint{2.279412in}{2.004545in}}%
\pgfusepath{clip}%
\pgfsetroundcap%
\pgfsetroundjoin%
\pgfsetlinewidth{0.341344pt}%
\definecolor{currentstroke}{rgb}{0.273809,0.031497,0.358853}%
\pgfsetstrokecolor{currentstroke}%
\pgfsetdash{}{0pt}%
\pgfpathmoveto{\pgfqpoint{8.067271in}{2.127716in}}%
\pgfpathquadraticcurveto{\pgfqpoint{8.054924in}{2.129580in}}{\pgfqpoint{8.047798in}{2.130655in}}%
\pgfusepath{stroke}%
\end{pgfscope}%
\begin{pgfscope}%
\pgfpathrectangle{\pgfqpoint{6.720588in}{1.750000in}}{\pgfqpoint{2.279412in}{2.004545in}}%
\pgfusepath{clip}%
\pgfsetroundcap%
\pgfsetroundjoin%
\definecolor{currentfill}{rgb}{0.273809,0.031497,0.358853}%
\pgfsetfillcolor{currentfill}%
\pgfsetlinewidth{0.341344pt}%
\definecolor{currentstroke}{rgb}{0.273809,0.031497,0.358853}%
\pgfsetstrokecolor{currentstroke}%
\pgfsetdash{}{0pt}%
\pgfpathmoveto{\pgfqpoint{8.098585in}{2.094897in}}%
\pgfpathlineto{\pgfqpoint{8.047798in}{2.130655in}}%
\pgfpathlineto{\pgfqpoint{8.106877in}{2.149830in}}%
\pgfpathlineto{\pgfqpoint{8.098585in}{2.094897in}}%
\pgfpathlineto{\pgfqpoint{8.098585in}{2.094897in}}%
\pgfpathclose%
\pgfusepath{stroke,fill}%
\end{pgfscope}%
\begin{pgfscope}%
\pgfpathrectangle{\pgfqpoint{6.720588in}{1.750000in}}{\pgfqpoint{2.279412in}{2.004545in}}%
\pgfusepath{clip}%
\pgfsetroundcap%
\pgfsetroundjoin%
\pgfsetlinewidth{0.358792pt}%
\definecolor{currentstroke}{rgb}{0.277018,0.050344,0.375715}%
\pgfsetstrokecolor{currentstroke}%
\pgfsetdash{}{0pt}%
\pgfpathmoveto{\pgfqpoint{8.079210in}{2.189552in}}%
\pgfpathquadraticcurveto{\pgfqpoint{8.066811in}{2.191157in}}{\pgfqpoint{8.059916in}{2.192050in}}%
\pgfusepath{stroke}%
\end{pgfscope}%
\begin{pgfscope}%
\pgfpathrectangle{\pgfqpoint{6.720588in}{1.750000in}}{\pgfqpoint{2.279412in}{2.004545in}}%
\pgfusepath{clip}%
\pgfsetroundcap%
\pgfsetroundjoin%
\definecolor{currentfill}{rgb}{0.277018,0.050344,0.375715}%
\pgfsetfillcolor{currentfill}%
\pgfsetlinewidth{0.358792pt}%
\definecolor{currentstroke}{rgb}{0.277018,0.050344,0.375715}%
\pgfsetstrokecolor{currentstroke}%
\pgfsetdash{}{0pt}%
\pgfpathmoveto{\pgfqpoint{8.111445in}{2.157369in}}%
\pgfpathlineto{\pgfqpoint{8.059916in}{2.192050in}}%
\pgfpathlineto{\pgfqpoint{8.118578in}{2.212464in}}%
\pgfpathlineto{\pgfqpoint{8.111445in}{2.157369in}}%
\pgfpathlineto{\pgfqpoint{8.111445in}{2.157369in}}%
\pgfpathclose%
\pgfusepath{stroke,fill}%
\end{pgfscope}%
\begin{pgfscope}%
\pgfpathrectangle{\pgfqpoint{6.720588in}{1.750000in}}{\pgfqpoint{2.279412in}{2.004545in}}%
\pgfusepath{clip}%
\pgfsetroundcap%
\pgfsetroundjoin%
\pgfsetlinewidth{0.387416pt}%
\definecolor{currentstroke}{rgb}{0.280267,0.073417,0.397163}%
\pgfsetstrokecolor{currentstroke}%
\pgfsetdash{}{0pt}%
\pgfpathmoveto{\pgfqpoint{8.127785in}{2.271897in}}%
\pgfpathquadraticcurveto{\pgfqpoint{8.115334in}{2.273188in}}{\pgfqpoint{8.108845in}{2.273861in}}%
\pgfusepath{stroke}%
\end{pgfscope}%
\begin{pgfscope}%
\pgfpathrectangle{\pgfqpoint{6.720588in}{1.750000in}}{\pgfqpoint{2.279412in}{2.004545in}}%
\pgfusepath{clip}%
\pgfsetroundcap%
\pgfsetroundjoin%
\definecolor{currentfill}{rgb}{0.280267,0.073417,0.397163}%
\pgfsetfillcolor{currentfill}%
\pgfsetlinewidth{0.387416pt}%
\definecolor{currentstroke}{rgb}{0.280267,0.073417,0.397163}%
\pgfsetstrokecolor{currentstroke}%
\pgfsetdash{}{0pt}%
\pgfpathmoveto{\pgfqpoint{8.161239in}{2.240501in}}%
\pgfpathlineto{\pgfqpoint{8.108845in}{2.273861in}}%
\pgfpathlineto{\pgfqpoint{8.166969in}{2.295760in}}%
\pgfpathlineto{\pgfqpoint{8.161239in}{2.240501in}}%
\pgfpathlineto{\pgfqpoint{8.161239in}{2.240501in}}%
\pgfpathclose%
\pgfusepath{stroke,fill}%
\end{pgfscope}%
\begin{pgfscope}%
\pgfpathrectangle{\pgfqpoint{6.720588in}{1.750000in}}{\pgfqpoint{2.279412in}{2.004545in}}%
\pgfusepath{clip}%
\pgfsetroundcap%
\pgfsetroundjoin%
\pgfsetlinewidth{0.349177pt}%
\definecolor{currentstroke}{rgb}{0.276022,0.044167,0.370164}%
\pgfsetstrokecolor{currentstroke}%
\pgfsetdash{}{0pt}%
\pgfpathmoveto{\pgfqpoint{8.328176in}{2.306523in}}%
\pgfpathquadraticcurveto{\pgfqpoint{8.315656in}{2.307082in}}{\pgfqpoint{8.308532in}{2.307400in}}%
\pgfusepath{stroke}%
\end{pgfscope}%
\begin{pgfscope}%
\pgfpathrectangle{\pgfqpoint{6.720588in}{1.750000in}}{\pgfqpoint{2.279412in}{2.004545in}}%
\pgfusepath{clip}%
\pgfsetroundcap%
\pgfsetroundjoin%
\definecolor{currentfill}{rgb}{0.276022,0.044167,0.370164}%
\pgfsetfillcolor{currentfill}%
\pgfsetlinewidth{0.349177pt}%
\definecolor{currentstroke}{rgb}{0.276022,0.044167,0.370164}%
\pgfsetstrokecolor{currentstroke}%
\pgfsetdash{}{0pt}%
\pgfpathmoveto{\pgfqpoint{8.362794in}{2.277173in}}%
\pgfpathlineto{\pgfqpoint{8.308532in}{2.307400in}}%
\pgfpathlineto{\pgfqpoint{8.365271in}{2.332674in}}%
\pgfpathlineto{\pgfqpoint{8.362794in}{2.277173in}}%
\pgfpathlineto{\pgfqpoint{8.362794in}{2.277173in}}%
\pgfpathclose%
\pgfusepath{stroke,fill}%
\end{pgfscope}%
\begin{pgfscope}%
\pgfpathrectangle{\pgfqpoint{6.720588in}{1.750000in}}{\pgfqpoint{2.279412in}{2.004545in}}%
\pgfusepath{clip}%
\pgfsetroundcap%
\pgfsetroundjoin%
\pgfsetlinewidth{0.423714pt}%
\definecolor{currentstroke}{rgb}{0.282656,0.100196,0.422160}%
\pgfsetstrokecolor{currentstroke}%
\pgfsetdash{}{0pt}%
\pgfpathmoveto{\pgfqpoint{8.127604in}{2.362952in}}%
\pgfpathquadraticcurveto{\pgfqpoint{8.115119in}{2.363934in}}{\pgfqpoint{8.109169in}{2.364402in}}%
\pgfusepath{stroke}%
\end{pgfscope}%
\begin{pgfscope}%
\pgfpathrectangle{\pgfqpoint{6.720588in}{1.750000in}}{\pgfqpoint{2.279412in}{2.004545in}}%
\pgfusepath{clip}%
\pgfsetroundcap%
\pgfsetroundjoin%
\definecolor{currentfill}{rgb}{0.282656,0.100196,0.422160}%
\pgfsetfillcolor{currentfill}%
\pgfsetlinewidth{0.423714pt}%
\definecolor{currentstroke}{rgb}{0.282656,0.100196,0.422160}%
\pgfsetstrokecolor{currentstroke}%
\pgfsetdash{}{0pt}%
\pgfpathmoveto{\pgfqpoint{8.162375in}{2.332353in}}%
\pgfpathlineto{\pgfqpoint{8.109169in}{2.364402in}}%
\pgfpathlineto{\pgfqpoint{8.166732in}{2.387737in}}%
\pgfpathlineto{\pgfqpoint{8.162375in}{2.332353in}}%
\pgfpathlineto{\pgfqpoint{8.162375in}{2.332353in}}%
\pgfpathclose%
\pgfusepath{stroke,fill}%
\end{pgfscope}%
\begin{pgfscope}%
\pgfpathrectangle{\pgfqpoint{6.720588in}{1.750000in}}{\pgfqpoint{2.279412in}{2.004545in}}%
\pgfusepath{clip}%
\pgfsetroundcap%
\pgfsetroundjoin%
\pgfsetlinewidth{0.604011pt}%
\definecolor{currentstroke}{rgb}{0.265145,0.232956,0.516599}%
\pgfsetstrokecolor{currentstroke}%
\pgfsetdash{}{0pt}%
\pgfpathmoveto{\pgfqpoint{7.977688in}{2.462124in}}%
\pgfpathquadraticcurveto{\pgfqpoint{7.965290in}{2.463757in}}{\pgfqpoint{7.962157in}{2.464170in}}%
\pgfusepath{stroke}%
\end{pgfscope}%
\begin{pgfscope}%
\pgfpathrectangle{\pgfqpoint{6.720588in}{1.750000in}}{\pgfqpoint{2.279412in}{2.004545in}}%
\pgfusepath{clip}%
\pgfsetroundcap%
\pgfsetroundjoin%
\definecolor{currentfill}{rgb}{0.265145,0.232956,0.516599}%
\pgfsetfillcolor{currentfill}%
\pgfsetlinewidth{0.604011pt}%
\definecolor{currentstroke}{rgb}{0.265145,0.232956,0.516599}%
\pgfsetstrokecolor{currentstroke}%
\pgfsetdash{}{0pt}%
\pgfpathmoveto{\pgfqpoint{8.013609in}{2.429374in}}%
\pgfpathlineto{\pgfqpoint{7.962157in}{2.464170in}}%
\pgfpathlineto{\pgfqpoint{8.020865in}{2.484454in}}%
\pgfpathlineto{\pgfqpoint{8.013609in}{2.429374in}}%
\pgfpathlineto{\pgfqpoint{8.013609in}{2.429374in}}%
\pgfpathclose%
\pgfusepath{stroke,fill}%
\end{pgfscope}%
\begin{pgfscope}%
\pgfpathrectangle{\pgfqpoint{6.720588in}{1.750000in}}{\pgfqpoint{2.279412in}{2.004545in}}%
\pgfusepath{clip}%
\pgfsetroundcap%
\pgfsetroundjoin%
\pgfsetlinewidth{0.424953pt}%
\definecolor{currentstroke}{rgb}{0.282910,0.105393,0.426902}%
\pgfsetstrokecolor{currentstroke}%
\pgfsetdash{}{0pt}%
\pgfpathmoveto{\pgfqpoint{8.280651in}{2.527596in}}%
\pgfpathquadraticcurveto{\pgfqpoint{8.268119in}{2.527935in}}{\pgfqpoint{8.262160in}{2.528097in}}%
\pgfusepath{stroke}%
\end{pgfscope}%
\begin{pgfscope}%
\pgfpathrectangle{\pgfqpoint{6.720588in}{1.750000in}}{\pgfqpoint{2.279412in}{2.004545in}}%
\pgfusepath{clip}%
\pgfsetroundcap%
\pgfsetroundjoin%
\definecolor{currentfill}{rgb}{0.282910,0.105393,0.426902}%
\pgfsetfillcolor{currentfill}%
\pgfsetlinewidth{0.424953pt}%
\definecolor{currentstroke}{rgb}{0.282910,0.105393,0.426902}%
\pgfsetstrokecolor{currentstroke}%
\pgfsetdash{}{0pt}%
\pgfpathmoveto{\pgfqpoint{8.316943in}{2.498826in}}%
\pgfpathlineto{\pgfqpoint{8.262160in}{2.528097in}}%
\pgfpathlineto{\pgfqpoint{8.318446in}{2.554361in}}%
\pgfpathlineto{\pgfqpoint{8.316943in}{2.498826in}}%
\pgfpathlineto{\pgfqpoint{8.316943in}{2.498826in}}%
\pgfpathclose%
\pgfusepath{stroke,fill}%
\end{pgfscope}%
\begin{pgfscope}%
\pgfpathrectangle{\pgfqpoint{6.720588in}{1.750000in}}{\pgfqpoint{2.279412in}{2.004545in}}%
\pgfusepath{clip}%
\pgfsetroundcap%
\pgfsetroundjoin%
\pgfsetlinewidth{0.505999pt}%
\definecolor{currentstroke}{rgb}{0.280868,0.160771,0.472899}%
\pgfsetstrokecolor{currentstroke}%
\pgfsetdash{}{0pt}%
\pgfpathmoveto{\pgfqpoint{8.177249in}{2.973879in}}%
\pgfpathquadraticcurveto{\pgfqpoint{8.164722in}{2.973432in}}{\pgfqpoint{8.160018in}{2.973263in}}%
\pgfusepath{stroke}%
\end{pgfscope}%
\begin{pgfscope}%
\pgfpathrectangle{\pgfqpoint{6.720588in}{1.750000in}}{\pgfqpoint{2.279412in}{2.004545in}}%
\pgfusepath{clip}%
\pgfsetroundcap%
\pgfsetroundjoin%
\definecolor{currentfill}{rgb}{0.280868,0.160771,0.472899}%
\pgfsetfillcolor{currentfill}%
\pgfsetlinewidth{0.505999pt}%
\definecolor{currentstroke}{rgb}{0.280868,0.160771,0.472899}%
\pgfsetstrokecolor{currentstroke}%
\pgfsetdash{}{0pt}%
\pgfpathmoveto{\pgfqpoint{8.216529in}{2.947487in}}%
\pgfpathlineto{\pgfqpoint{8.160018in}{2.973263in}}%
\pgfpathlineto{\pgfqpoint{8.214546in}{3.003007in}}%
\pgfpathlineto{\pgfqpoint{8.216529in}{2.947487in}}%
\pgfpathlineto{\pgfqpoint{8.216529in}{2.947487in}}%
\pgfpathclose%
\pgfusepath{stroke,fill}%
\end{pgfscope}%
\begin{pgfscope}%
\pgfpathrectangle{\pgfqpoint{6.720588in}{1.750000in}}{\pgfqpoint{2.279412in}{2.004545in}}%
\pgfusepath{clip}%
\pgfsetroundcap%
\pgfsetroundjoin%
\pgfsetlinewidth{0.382162pt}%
\definecolor{currentstroke}{rgb}{0.279566,0.067836,0.391917}%
\pgfsetstrokecolor{currentstroke}%
\pgfsetdash{}{0pt}%
\pgfpathmoveto{\pgfqpoint{8.327776in}{3.020253in}}%
\pgfpathquadraticcurveto{\pgfqpoint{8.315245in}{3.019898in}}{\pgfqpoint{8.308624in}{3.019711in}}%
\pgfusepath{stroke}%
\end{pgfscope}%
\begin{pgfscope}%
\pgfpathrectangle{\pgfqpoint{6.720588in}{1.750000in}}{\pgfqpoint{2.279412in}{2.004545in}}%
\pgfusepath{clip}%
\pgfsetroundcap%
\pgfsetroundjoin%
\definecolor{currentfill}{rgb}{0.279566,0.067836,0.391917}%
\pgfsetfillcolor{currentfill}%
\pgfsetlinewidth{0.382162pt}%
\definecolor{currentstroke}{rgb}{0.279566,0.067836,0.391917}%
\pgfsetstrokecolor{currentstroke}%
\pgfsetdash{}{0pt}%
\pgfpathmoveto{\pgfqpoint{8.364944in}{2.993518in}}%
\pgfpathlineto{\pgfqpoint{8.308624in}{3.019711in}}%
\pgfpathlineto{\pgfqpoint{8.363371in}{3.049051in}}%
\pgfpathlineto{\pgfqpoint{8.364944in}{2.993518in}}%
\pgfpathlineto{\pgfqpoint{8.364944in}{2.993518in}}%
\pgfpathclose%
\pgfusepath{stroke,fill}%
\end{pgfscope}%
\begin{pgfscope}%
\pgfpathrectangle{\pgfqpoint{6.720588in}{1.750000in}}{\pgfqpoint{2.279412in}{2.004545in}}%
\pgfusepath{clip}%
\pgfsetroundcap%
\pgfsetroundjoin%
\pgfsetlinewidth{0.521339pt}%
\definecolor{currentstroke}{rgb}{0.278826,0.175490,0.483397}%
\pgfsetstrokecolor{currentstroke}%
\pgfsetdash{}{0pt}%
\pgfpathmoveto{\pgfqpoint{8.027599in}{3.046057in}}%
\pgfpathquadraticcurveto{\pgfqpoint{8.015169in}{3.044620in}}{\pgfqpoint{8.010752in}{3.044109in}}%
\pgfusepath{stroke}%
\end{pgfscope}%
\begin{pgfscope}%
\pgfpathrectangle{\pgfqpoint{6.720588in}{1.750000in}}{\pgfqpoint{2.279412in}{2.004545in}}%
\pgfusepath{clip}%
\pgfsetroundcap%
\pgfsetroundjoin%
\definecolor{currentfill}{rgb}{0.278826,0.175490,0.483397}%
\pgfsetfillcolor{currentfill}%
\pgfsetlinewidth{0.521339pt}%
\definecolor{currentstroke}{rgb}{0.278826,0.175490,0.483397}%
\pgfsetstrokecolor{currentstroke}%
\pgfsetdash{}{0pt}%
\pgfpathmoveto{\pgfqpoint{8.069131in}{3.022898in}}%
\pgfpathlineto{\pgfqpoint{8.010752in}{3.044109in}}%
\pgfpathlineto{\pgfqpoint{8.062748in}{3.078086in}}%
\pgfpathlineto{\pgfqpoint{8.069131in}{3.022898in}}%
\pgfpathlineto{\pgfqpoint{8.069131in}{3.022898in}}%
\pgfpathclose%
\pgfusepath{stroke,fill}%
\end{pgfscope}%
\begin{pgfscope}%
\pgfpathrectangle{\pgfqpoint{6.720588in}{1.750000in}}{\pgfqpoint{2.279412in}{2.004545in}}%
\pgfusepath{clip}%
\pgfsetroundcap%
\pgfsetroundjoin%
\pgfsetlinewidth{0.330452pt}%
\definecolor{currentstroke}{rgb}{0.272594,0.025563,0.353093}%
\pgfsetstrokecolor{currentstroke}%
\pgfsetdash{}{0pt}%
\pgfpathmoveto{\pgfqpoint{8.355421in}{3.293842in}}%
\pgfpathquadraticcurveto{\pgfqpoint{8.342890in}{3.293685in}}{\pgfqpoint{8.335472in}{3.293592in}}%
\pgfusepath{stroke}%
\end{pgfscope}%
\begin{pgfscope}%
\pgfpathrectangle{\pgfqpoint{6.720588in}{1.750000in}}{\pgfqpoint{2.279412in}{2.004545in}}%
\pgfusepath{clip}%
\pgfsetroundcap%
\pgfsetroundjoin%
\definecolor{currentfill}{rgb}{0.272594,0.025563,0.353093}%
\pgfsetfillcolor{currentfill}%
\pgfsetlinewidth{0.330452pt}%
\definecolor{currentstroke}{rgb}{0.272594,0.025563,0.353093}%
\pgfsetstrokecolor{currentstroke}%
\pgfsetdash{}{0pt}%
\pgfpathmoveto{\pgfqpoint{8.391371in}{3.266512in}}%
\pgfpathlineto{\pgfqpoint{8.335472in}{3.293592in}}%
\pgfpathlineto{\pgfqpoint{8.390675in}{3.322063in}}%
\pgfpathlineto{\pgfqpoint{8.391371in}{3.266512in}}%
\pgfpathlineto{\pgfqpoint{8.391371in}{3.266512in}}%
\pgfpathclose%
\pgfusepath{stroke,fill}%
\end{pgfscope}%
\begin{pgfscope}%
\pgfpathrectangle{\pgfqpoint{6.720588in}{1.750000in}}{\pgfqpoint{2.279412in}{2.004545in}}%
\pgfusepath{clip}%
\pgfsetroundcap%
\pgfsetroundjoin%
\pgfsetlinewidth{0.348245pt}%
\definecolor{currentstroke}{rgb}{0.274952,0.037752,0.364543}%
\pgfsetstrokecolor{currentstroke}%
\pgfsetdash{}{0pt}%
\pgfpathmoveto{\pgfqpoint{8.119603in}{3.375816in}}%
\pgfpathquadraticcurveto{\pgfqpoint{8.107179in}{3.374388in}}{\pgfqpoint{8.100108in}{3.373575in}}%
\pgfusepath{stroke}%
\end{pgfscope}%
\begin{pgfscope}%
\pgfpathrectangle{\pgfqpoint{6.720588in}{1.750000in}}{\pgfqpoint{2.279412in}{2.004545in}}%
\pgfusepath{clip}%
\pgfsetroundcap%
\pgfsetroundjoin%
\definecolor{currentfill}{rgb}{0.274952,0.037752,0.364543}%
\pgfsetfillcolor{currentfill}%
\pgfsetlinewidth{0.348245pt}%
\definecolor{currentstroke}{rgb}{0.274952,0.037752,0.364543}%
\pgfsetstrokecolor{currentstroke}%
\pgfsetdash{}{0pt}%
\pgfpathmoveto{\pgfqpoint{8.158472in}{3.352323in}}%
\pgfpathlineto{\pgfqpoint{8.100108in}{3.373575in}}%
\pgfpathlineto{\pgfqpoint{8.152128in}{3.407515in}}%
\pgfpathlineto{\pgfqpoint{8.158472in}{3.352323in}}%
\pgfpathlineto{\pgfqpoint{8.158472in}{3.352323in}}%
\pgfpathclose%
\pgfusepath{stroke,fill}%
\end{pgfscope}%
\begin{pgfscope}%
\pgfpathrectangle{\pgfqpoint{6.720588in}{1.750000in}}{\pgfqpoint{2.279412in}{2.004545in}}%
\pgfusepath{clip}%
\pgfsetroundcap%
\pgfsetroundjoin%
\pgfsetlinewidth{0.363655pt}%
\definecolor{currentstroke}{rgb}{0.277941,0.056324,0.381191}%
\pgfsetstrokecolor{currentstroke}%
\pgfsetdash{}{0pt}%
\pgfpathmoveto{\pgfqpoint{8.326573in}{2.393104in}}%
\pgfpathquadraticcurveto{\pgfqpoint{8.314047in}{2.393561in}}{\pgfqpoint{8.307143in}{2.393812in}}%
\pgfusepath{stroke}%
\end{pgfscope}%
\begin{pgfscope}%
\pgfpathrectangle{\pgfqpoint{6.720588in}{1.750000in}}{\pgfqpoint{2.279412in}{2.004545in}}%
\pgfusepath{clip}%
\pgfsetroundcap%
\pgfsetroundjoin%
\definecolor{currentfill}{rgb}{0.277941,0.056324,0.381191}%
\pgfsetfillcolor{currentfill}%
\pgfsetlinewidth{0.363655pt}%
\definecolor{currentstroke}{rgb}{0.277941,0.056324,0.381191}%
\pgfsetstrokecolor{currentstroke}%
\pgfsetdash{}{0pt}%
\pgfpathmoveto{\pgfqpoint{8.361649in}{2.364028in}}%
\pgfpathlineto{\pgfqpoint{8.307143in}{2.393812in}}%
\pgfpathlineto{\pgfqpoint{8.363674in}{2.419546in}}%
\pgfpathlineto{\pgfqpoint{8.361649in}{2.364028in}}%
\pgfpathlineto{\pgfqpoint{8.361649in}{2.364028in}}%
\pgfpathclose%
\pgfusepath{stroke,fill}%
\end{pgfscope}%
\begin{pgfscope}%
\pgfpathrectangle{\pgfqpoint{6.720588in}{1.750000in}}{\pgfqpoint{2.279412in}{2.004545in}}%
\pgfusepath{clip}%
\pgfsetroundcap%
\pgfsetroundjoin%
\pgfsetlinewidth{0.704088pt}%
\definecolor{currentstroke}{rgb}{0.241237,0.296485,0.539709}%
\pgfsetstrokecolor{currentstroke}%
\pgfsetdash{}{0pt}%
\pgfpathmoveto{\pgfqpoint{8.125929in}{2.663081in}}%
\pgfpathquadraticcurveto{\pgfqpoint{8.113392in}{2.663186in}}{\pgfqpoint{8.111746in}{2.663200in}}%
\pgfusepath{stroke}%
\end{pgfscope}%
\begin{pgfscope}%
\pgfpathrectangle{\pgfqpoint{6.720588in}{1.750000in}}{\pgfqpoint{2.279412in}{2.004545in}}%
\pgfusepath{clip}%
\pgfsetroundcap%
\pgfsetroundjoin%
\definecolor{currentfill}{rgb}{0.241237,0.296485,0.539709}%
\pgfsetfillcolor{currentfill}%
\pgfsetlinewidth{0.704088pt}%
\definecolor{currentstroke}{rgb}{0.241237,0.296485,0.539709}%
\pgfsetstrokecolor{currentstroke}%
\pgfsetdash{}{0pt}%
\pgfpathmoveto{\pgfqpoint{8.167068in}{2.634959in}}%
\pgfpathlineto{\pgfqpoint{8.111746in}{2.663200in}}%
\pgfpathlineto{\pgfqpoint{8.167532in}{2.690512in}}%
\pgfpathlineto{\pgfqpoint{8.167068in}{2.634959in}}%
\pgfpathlineto{\pgfqpoint{8.167068in}{2.634959in}}%
\pgfpathclose%
\pgfusepath{stroke,fill}%
\end{pgfscope}%
\begin{pgfscope}%
\pgfpathrectangle{\pgfqpoint{6.720588in}{1.750000in}}{\pgfqpoint{2.279412in}{2.004545in}}%
\pgfusepath{clip}%
\pgfsetroundcap%
\pgfsetroundjoin%
\pgfsetlinewidth{0.704206pt}%
\definecolor{currentstroke}{rgb}{0.241237,0.296485,0.539709}%
\pgfsetstrokecolor{currentstroke}%
\pgfsetdash{}{0pt}%
\pgfpathmoveto{\pgfqpoint{8.125999in}{2.795686in}}%
\pgfpathquadraticcurveto{\pgfqpoint{8.113462in}{2.795565in}}{\pgfqpoint{8.111819in}{2.795550in}}%
\pgfusepath{stroke}%
\end{pgfscope}%
\begin{pgfscope}%
\pgfpathrectangle{\pgfqpoint{6.720588in}{1.750000in}}{\pgfqpoint{2.279412in}{2.004545in}}%
\pgfusepath{clip}%
\pgfsetroundcap%
\pgfsetroundjoin%
\definecolor{currentfill}{rgb}{0.241237,0.296485,0.539709}%
\pgfsetfillcolor{currentfill}%
\pgfsetlinewidth{0.704206pt}%
\definecolor{currentstroke}{rgb}{0.241237,0.296485,0.539709}%
\pgfsetstrokecolor{currentstroke}%
\pgfsetdash{}{0pt}%
\pgfpathmoveto{\pgfqpoint{8.167639in}{2.768308in}}%
\pgfpathlineto{\pgfqpoint{8.111819in}{2.795550in}}%
\pgfpathlineto{\pgfqpoint{8.167104in}{2.823861in}}%
\pgfpathlineto{\pgfqpoint{8.167639in}{2.768308in}}%
\pgfpathlineto{\pgfqpoint{8.167639in}{2.768308in}}%
\pgfpathclose%
\pgfusepath{stroke,fill}%
\end{pgfscope}%
\begin{pgfscope}%
\pgfpathrectangle{\pgfqpoint{6.720588in}{1.750000in}}{\pgfqpoint{2.279412in}{2.004545in}}%
\pgfusepath{clip}%
\pgfsetroundcap%
\pgfsetroundjoin%
\pgfsetlinewidth{0.590069pt}%
\definecolor{currentstroke}{rgb}{0.267968,0.223549,0.512008}%
\pgfsetstrokecolor{currentstroke}%
\pgfsetdash{}{0pt}%
\pgfpathmoveto{\pgfqpoint{8.176086in}{2.839254in}}%
\pgfpathquadraticcurveto{\pgfqpoint{8.163550in}{2.839089in}}{\pgfqpoint{8.160142in}{2.839044in}}%
\pgfusepath{stroke}%
\end{pgfscope}%
\begin{pgfscope}%
\pgfpathrectangle{\pgfqpoint{6.720588in}{1.750000in}}{\pgfqpoint{2.279412in}{2.004545in}}%
\pgfusepath{clip}%
\pgfsetroundcap%
\pgfsetroundjoin%
\definecolor{currentfill}{rgb}{0.267968,0.223549,0.512008}%
\pgfsetfillcolor{currentfill}%
\pgfsetlinewidth{0.590069pt}%
\definecolor{currentstroke}{rgb}{0.267968,0.223549,0.512008}%
\pgfsetstrokecolor{currentstroke}%
\pgfsetdash{}{0pt}%
\pgfpathmoveto{\pgfqpoint{8.216059in}{2.812002in}}%
\pgfpathlineto{\pgfqpoint{8.160142in}{2.839044in}}%
\pgfpathlineto{\pgfqpoint{8.215325in}{2.867553in}}%
\pgfpathlineto{\pgfqpoint{8.216059in}{2.812002in}}%
\pgfpathlineto{\pgfqpoint{8.216059in}{2.812002in}}%
\pgfpathclose%
\pgfusepath{stroke,fill}%
\end{pgfscope}%
\begin{pgfscope}%
\pgfpathrectangle{\pgfqpoint{6.720588in}{1.750000in}}{\pgfqpoint{2.279412in}{2.004545in}}%
\pgfusepath{clip}%
\pgfsetroundcap%
\pgfsetroundjoin%
\pgfsetlinewidth{0.454368pt}%
\definecolor{currentstroke}{rgb}{0.283187,0.125848,0.444960}%
\pgfsetstrokecolor{currentstroke}%
\pgfsetdash{}{0pt}%
\pgfpathmoveto{\pgfqpoint{7.929017in}{3.112535in}}%
\pgfpathquadraticcurveto{\pgfqpoint{7.916890in}{3.109772in}}{\pgfqpoint{7.911616in}{3.108571in}}%
\pgfusepath{stroke}%
\end{pgfscope}%
\begin{pgfscope}%
\pgfpathrectangle{\pgfqpoint{6.720588in}{1.750000in}}{\pgfqpoint{2.279412in}{2.004545in}}%
\pgfusepath{clip}%
\pgfsetroundcap%
\pgfsetroundjoin%
\definecolor{currentfill}{rgb}{0.283187,0.125848,0.444960}%
\pgfsetfillcolor{currentfill}%
\pgfsetlinewidth{0.454368pt}%
\definecolor{currentstroke}{rgb}{0.283187,0.125848,0.444960}%
\pgfsetstrokecolor{currentstroke}%
\pgfsetdash{}{0pt}%
\pgfpathmoveto{\pgfqpoint{7.971954in}{3.093828in}}%
\pgfpathlineto{\pgfqpoint{7.911616in}{3.108571in}}%
\pgfpathlineto{\pgfqpoint{7.959613in}{3.147995in}}%
\pgfpathlineto{\pgfqpoint{7.971954in}{3.093828in}}%
\pgfpathlineto{\pgfqpoint{7.971954in}{3.093828in}}%
\pgfpathclose%
\pgfusepath{stroke,fill}%
\end{pgfscope}%
\begin{pgfscope}%
\pgfpathrectangle{\pgfqpoint{6.720588in}{1.750000in}}{\pgfqpoint{2.279412in}{2.004545in}}%
\pgfusepath{clip}%
\pgfsetroundcap%
\pgfsetroundjoin%
\pgfsetlinewidth{0.344053pt}%
\definecolor{currentstroke}{rgb}{0.274952,0.037752,0.364543}%
\pgfsetstrokecolor{currentstroke}%
\pgfsetdash{}{0pt}%
\pgfpathmoveto{\pgfqpoint{8.316720in}{2.198208in}}%
\pgfpathquadraticcurveto{\pgfqpoint{8.304198in}{2.198765in}}{\pgfqpoint{8.296995in}{2.199085in}}%
\pgfusepath{stroke}%
\end{pgfscope}%
\begin{pgfscope}%
\pgfpathrectangle{\pgfqpoint{6.720588in}{1.750000in}}{\pgfqpoint{2.279412in}{2.004545in}}%
\pgfusepath{clip}%
\pgfsetroundcap%
\pgfsetroundjoin%
\definecolor{currentfill}{rgb}{0.274952,0.037752,0.364543}%
\pgfsetfillcolor{currentfill}%
\pgfsetlinewidth{0.344053pt}%
\definecolor{currentstroke}{rgb}{0.274952,0.037752,0.364543}%
\pgfsetstrokecolor{currentstroke}%
\pgfsetdash{}{0pt}%
\pgfpathmoveto{\pgfqpoint{8.351261in}{2.168866in}}%
\pgfpathlineto{\pgfqpoint{8.296995in}{2.199085in}}%
\pgfpathlineto{\pgfqpoint{8.353730in}{2.224367in}}%
\pgfpathlineto{\pgfqpoint{8.351261in}{2.168866in}}%
\pgfpathlineto{\pgfqpoint{8.351261in}{2.168866in}}%
\pgfpathclose%
\pgfusepath{stroke,fill}%
\end{pgfscope}%
\begin{pgfscope}%
\pgfpathrectangle{\pgfqpoint{6.720588in}{1.750000in}}{\pgfqpoint{2.279412in}{2.004545in}}%
\pgfusepath{clip}%
\pgfsetroundcap%
\pgfsetroundjoin%
\pgfsetlinewidth{0.423402pt}%
\definecolor{currentstroke}{rgb}{0.282656,0.100196,0.422160}%
\pgfsetstrokecolor{currentstroke}%
\pgfsetdash{}{0pt}%
\pgfpathmoveto{\pgfqpoint{8.275242in}{2.485106in}}%
\pgfpathquadraticcurveto{\pgfqpoint{8.262711in}{2.485489in}}{\pgfqpoint{8.256728in}{2.485671in}}%
\pgfusepath{stroke}%
\end{pgfscope}%
\begin{pgfscope}%
\pgfpathrectangle{\pgfqpoint{6.720588in}{1.750000in}}{\pgfqpoint{2.279412in}{2.004545in}}%
\pgfusepath{clip}%
\pgfsetroundcap%
\pgfsetroundjoin%
\definecolor{currentfill}{rgb}{0.282656,0.100196,0.422160}%
\pgfsetfillcolor{currentfill}%
\pgfsetlinewidth{0.423402pt}%
\definecolor{currentstroke}{rgb}{0.282656,0.100196,0.422160}%
\pgfsetstrokecolor{currentstroke}%
\pgfsetdash{}{0pt}%
\pgfpathmoveto{\pgfqpoint{8.311410in}{2.456211in}}%
\pgfpathlineto{\pgfqpoint{8.256728in}{2.485671in}}%
\pgfpathlineto{\pgfqpoint{8.313106in}{2.511741in}}%
\pgfpathlineto{\pgfqpoint{8.311410in}{2.456211in}}%
\pgfpathlineto{\pgfqpoint{8.311410in}{2.456211in}}%
\pgfpathclose%
\pgfusepath{stroke,fill}%
\end{pgfscope}%
\begin{pgfscope}%
\pgfpathrectangle{\pgfqpoint{6.720588in}{1.750000in}}{\pgfqpoint{2.279412in}{2.004545in}}%
\pgfusepath{clip}%
\pgfsetroundcap%
\pgfsetroundjoin%
\pgfsetlinewidth{0.500666pt}%
\definecolor{currentstroke}{rgb}{0.280868,0.160771,0.472899}%
\pgfsetstrokecolor{currentstroke}%
\pgfsetdash{}{0pt}%
\pgfpathmoveto{\pgfqpoint{8.225059in}{2.886299in}}%
\pgfpathquadraticcurveto{\pgfqpoint{8.212524in}{2.886055in}}{\pgfqpoint{8.207733in}{2.885961in}}%
\pgfusepath{stroke}%
\end{pgfscope}%
\begin{pgfscope}%
\pgfpathrectangle{\pgfqpoint{6.720588in}{1.750000in}}{\pgfqpoint{2.279412in}{2.004545in}}%
\pgfusepath{clip}%
\pgfsetroundcap%
\pgfsetroundjoin%
\definecolor{currentfill}{rgb}{0.280868,0.160771,0.472899}%
\pgfsetfillcolor{currentfill}%
\pgfsetlinewidth{0.500666pt}%
\definecolor{currentstroke}{rgb}{0.280868,0.160771,0.472899}%
\pgfsetstrokecolor{currentstroke}%
\pgfsetdash{}{0pt}%
\pgfpathmoveto{\pgfqpoint{8.263818in}{2.859270in}}%
\pgfpathlineto{\pgfqpoint{8.207733in}{2.885961in}}%
\pgfpathlineto{\pgfqpoint{8.262737in}{2.914815in}}%
\pgfpathlineto{\pgfqpoint{8.263818in}{2.859270in}}%
\pgfpathlineto{\pgfqpoint{8.263818in}{2.859270in}}%
\pgfpathclose%
\pgfusepath{stroke,fill}%
\end{pgfscope}%
\begin{pgfscope}%
\pgfpathrectangle{\pgfqpoint{6.720588in}{1.750000in}}{\pgfqpoint{2.279412in}{2.004545in}}%
\pgfusepath{clip}%
\pgfsetroundcap%
\pgfsetroundjoin%
\pgfsetlinewidth{0.367247pt}%
\definecolor{currentstroke}{rgb}{0.277941,0.056324,0.381191}%
\pgfsetstrokecolor{currentstroke}%
\pgfsetdash{}{0pt}%
\pgfpathmoveto{\pgfqpoint{8.325537in}{3.107828in}}%
\pgfpathquadraticcurveto{\pgfqpoint{8.313011in}{3.107344in}}{\pgfqpoint{8.306163in}{3.107079in}}%
\pgfusepath{stroke}%
\end{pgfscope}%
\begin{pgfscope}%
\pgfpathrectangle{\pgfqpoint{6.720588in}{1.750000in}}{\pgfqpoint{2.279412in}{2.004545in}}%
\pgfusepath{clip}%
\pgfsetroundcap%
\pgfsetroundjoin%
\definecolor{currentfill}{rgb}{0.277941,0.056324,0.381191}%
\pgfsetfillcolor{currentfill}%
\pgfsetlinewidth{0.367247pt}%
\definecolor{currentstroke}{rgb}{0.277941,0.056324,0.381191}%
\pgfsetstrokecolor{currentstroke}%
\pgfsetdash{}{0pt}%
\pgfpathmoveto{\pgfqpoint{8.362751in}{3.081469in}}%
\pgfpathlineto{\pgfqpoint{8.306163in}{3.107079in}}%
\pgfpathlineto{\pgfqpoint{8.360603in}{3.136983in}}%
\pgfpathlineto{\pgfqpoint{8.362751in}{3.081469in}}%
\pgfpathlineto{\pgfqpoint{8.362751in}{3.081469in}}%
\pgfpathclose%
\pgfusepath{stroke,fill}%
\end{pgfscope}%
\begin{pgfscope}%
\pgfpathrectangle{\pgfqpoint{6.720588in}{1.750000in}}{\pgfqpoint{2.279412in}{2.004545in}}%
\pgfusepath{clip}%
\pgfsetroundcap%
\pgfsetroundjoin%
\pgfsetlinewidth{0.384612pt}%
\definecolor{currentstroke}{rgb}{0.280267,0.073417,0.397163}%
\pgfsetstrokecolor{currentstroke}%
\pgfsetdash{}{0pt}%
\pgfpathmoveto{\pgfqpoint{7.725373in}{3.204051in}}%
\pgfpathquadraticcurveto{\pgfqpoint{7.725521in}{3.198624in}}{\pgfqpoint{7.725507in}{3.199144in}}%
\pgfusepath{stroke}%
\end{pgfscope}%
\begin{pgfscope}%
\pgfpathrectangle{\pgfqpoint{6.720588in}{1.750000in}}{\pgfqpoint{2.279412in}{2.004545in}}%
\pgfusepath{clip}%
\pgfsetroundcap%
\pgfsetroundjoin%
\definecolor{currentfill}{rgb}{0.280267,0.073417,0.397163}%
\pgfsetfillcolor{currentfill}%
\pgfsetlinewidth{0.384612pt}%
\definecolor{currentstroke}{rgb}{0.280267,0.073417,0.397163}%
\pgfsetstrokecolor{currentstroke}%
\pgfsetdash{}{0pt}%
\pgfpathmoveto{\pgfqpoint{7.751756in}{3.255438in}}%
\pgfpathlineto{\pgfqpoint{7.725507in}{3.199144in}}%
\pgfpathlineto{\pgfqpoint{7.696221in}{3.253920in}}%
\pgfpathlineto{\pgfqpoint{7.751756in}{3.255438in}}%
\pgfpathlineto{\pgfqpoint{7.751756in}{3.255438in}}%
\pgfpathclose%
\pgfusepath{stroke,fill}%
\end{pgfscope}%
\begin{pgfscope}%
\pgfpathrectangle{\pgfqpoint{6.720588in}{1.750000in}}{\pgfqpoint{2.279412in}{2.004545in}}%
\pgfusepath{clip}%
\pgfsetroundcap%
\pgfsetroundjoin%
\pgfsetlinewidth{1.052322pt}%
\definecolor{currentstroke}{rgb}{0.153364,0.497000,0.557724}%
\pgfsetstrokecolor{currentstroke}%
\pgfsetdash{}{0pt}%
\pgfpathmoveto{\pgfqpoint{7.195216in}{2.803993in}}%
\pgfpathquadraticcurveto{\pgfqpoint{7.207610in}{2.802339in}}{\pgfqpoint{7.203867in}{2.802839in}}%
\pgfusepath{stroke}%
\end{pgfscope}%
\begin{pgfscope}%
\pgfpathrectangle{\pgfqpoint{6.720588in}{1.750000in}}{\pgfqpoint{2.279412in}{2.004545in}}%
\pgfusepath{clip}%
\pgfsetroundcap%
\pgfsetroundjoin%
\definecolor{currentfill}{rgb}{0.153364,0.497000,0.557724}%
\pgfsetfillcolor{currentfill}%
\pgfsetlinewidth{1.052322pt}%
\definecolor{currentstroke}{rgb}{0.153364,0.497000,0.557724}%
\pgfsetstrokecolor{currentstroke}%
\pgfsetdash{}{0pt}%
\pgfpathmoveto{\pgfqpoint{7.152472in}{2.837719in}}%
\pgfpathlineto{\pgfqpoint{7.203867in}{2.802839in}}%
\pgfpathlineto{\pgfqpoint{7.145126in}{2.782651in}}%
\pgfpathlineto{\pgfqpoint{7.152472in}{2.837719in}}%
\pgfpathlineto{\pgfqpoint{7.152472in}{2.837719in}}%
\pgfpathclose%
\pgfusepath{stroke,fill}%
\end{pgfscope}%
\begin{pgfscope}%
\pgfpathrectangle{\pgfqpoint{6.720588in}{1.750000in}}{\pgfqpoint{2.279412in}{2.004545in}}%
\pgfusepath{clip}%
\pgfsetroundcap%
\pgfsetroundjoin%
\pgfsetlinewidth{0.391750pt}%
\definecolor{currentstroke}{rgb}{0.280894,0.078907,0.402329}%
\pgfsetstrokecolor{currentstroke}%
\pgfsetdash{}{0pt}%
\pgfpathmoveto{\pgfqpoint{8.025005in}{3.177854in}}%
\pgfpathquadraticcurveto{\pgfqpoint{8.012780in}{3.175436in}}{\pgfqpoint{8.006500in}{3.174194in}}%
\pgfusepath{stroke}%
\end{pgfscope}%
\begin{pgfscope}%
\pgfpathrectangle{\pgfqpoint{6.720588in}{1.750000in}}{\pgfqpoint{2.279412in}{2.004545in}}%
\pgfusepath{clip}%
\pgfsetroundcap%
\pgfsetroundjoin%
\definecolor{currentfill}{rgb}{0.280894,0.078907,0.402329}%
\pgfsetfillcolor{currentfill}%
\pgfsetlinewidth{0.391750pt}%
\definecolor{currentstroke}{rgb}{0.280894,0.078907,0.402329}%
\pgfsetstrokecolor{currentstroke}%
\pgfsetdash{}{0pt}%
\pgfpathmoveto{\pgfqpoint{8.066389in}{3.157723in}}%
\pgfpathlineto{\pgfqpoint{8.006500in}{3.174194in}}%
\pgfpathlineto{\pgfqpoint{8.055610in}{3.212222in}}%
\pgfpathlineto{\pgfqpoint{8.066389in}{3.157723in}}%
\pgfpathlineto{\pgfqpoint{8.066389in}{3.157723in}}%
\pgfpathclose%
\pgfusepath{stroke,fill}%
\end{pgfscope}%
\begin{pgfscope}%
\pgfpathrectangle{\pgfqpoint{6.720588in}{1.750000in}}{\pgfqpoint{2.279412in}{2.004545in}}%
\pgfusepath{clip}%
\pgfsetroundcap%
\pgfsetroundjoin%
\pgfsetlinewidth{0.351620pt}%
\definecolor{currentstroke}{rgb}{0.276022,0.044167,0.370164}%
\pgfsetstrokecolor{currentstroke}%
\pgfsetdash{}{0pt}%
\pgfpathmoveto{\pgfqpoint{8.274146in}{3.244524in}}%
\pgfpathquadraticcurveto{\pgfqpoint{8.261631in}{3.243897in}}{\pgfqpoint{8.254549in}{3.243542in}}%
\pgfusepath{stroke}%
\end{pgfscope}%
\begin{pgfscope}%
\pgfpathrectangle{\pgfqpoint{6.720588in}{1.750000in}}{\pgfqpoint{2.279412in}{2.004545in}}%
\pgfusepath{clip}%
\pgfsetroundcap%
\pgfsetroundjoin%
\definecolor{currentfill}{rgb}{0.276022,0.044167,0.370164}%
\pgfsetfillcolor{currentfill}%
\pgfsetlinewidth{0.351620pt}%
\definecolor{currentstroke}{rgb}{0.276022,0.044167,0.370164}%
\pgfsetstrokecolor{currentstroke}%
\pgfsetdash{}{0pt}%
\pgfpathmoveto{\pgfqpoint{8.311426in}{3.218580in}}%
\pgfpathlineto{\pgfqpoint{8.254549in}{3.243542in}}%
\pgfpathlineto{\pgfqpoint{8.308645in}{3.274066in}}%
\pgfpathlineto{\pgfqpoint{8.311426in}{3.218580in}}%
\pgfpathlineto{\pgfqpoint{8.311426in}{3.218580in}}%
\pgfpathclose%
\pgfusepath{stroke,fill}%
\end{pgfscope}%
\begin{pgfscope}%
\pgfpathrectangle{\pgfqpoint{6.720588in}{1.750000in}}{\pgfqpoint{2.279412in}{2.004545in}}%
\pgfusepath{clip}%
\pgfsetroundcap%
\pgfsetroundjoin%
\pgfsetlinewidth{0.390290pt}%
\definecolor{currentstroke}{rgb}{0.280267,0.073417,0.397163}%
\pgfsetstrokecolor{currentstroke}%
\pgfsetdash{}{0pt}%
\pgfpathmoveto{\pgfqpoint{7.920662in}{3.214610in}}%
\pgfpathquadraticcurveto{\pgfqpoint{7.908941in}{3.210778in}}{\pgfqpoint{7.902959in}{3.208823in}}%
\pgfusepath{stroke}%
\end{pgfscope}%
\begin{pgfscope}%
\pgfpathrectangle{\pgfqpoint{6.720588in}{1.750000in}}{\pgfqpoint{2.279412in}{2.004545in}}%
\pgfusepath{clip}%
\pgfsetroundcap%
\pgfsetroundjoin%
\definecolor{currentfill}{rgb}{0.280267,0.073417,0.397163}%
\pgfsetfillcolor{currentfill}%
\pgfsetlinewidth{0.390290pt}%
\definecolor{currentstroke}{rgb}{0.280267,0.073417,0.397163}%
\pgfsetstrokecolor{currentstroke}%
\pgfsetdash{}{0pt}%
\pgfpathmoveto{\pgfqpoint{7.964396in}{3.199685in}}%
\pgfpathlineto{\pgfqpoint{7.902959in}{3.208823in}}%
\pgfpathlineto{\pgfqpoint{7.947132in}{3.252490in}}%
\pgfpathlineto{\pgfqpoint{7.964396in}{3.199685in}}%
\pgfpathlineto{\pgfqpoint{7.964396in}{3.199685in}}%
\pgfpathclose%
\pgfusepath{stroke,fill}%
\end{pgfscope}%
\begin{pgfscope}%
\pgfpathrectangle{\pgfqpoint{6.720588in}{1.750000in}}{\pgfqpoint{2.279412in}{2.004545in}}%
\pgfusepath{clip}%
\pgfsetroundcap%
\pgfsetroundjoin%
\pgfsetlinewidth{0.367284pt}%
\definecolor{currentstroke}{rgb}{0.277941,0.056324,0.381191}%
\pgfsetstrokecolor{currentstroke}%
\pgfsetdash{}{0pt}%
\pgfpathmoveto{\pgfqpoint{7.829523in}{3.233368in}}%
\pgfpathquadraticcurveto{\pgfqpoint{7.823395in}{3.228927in}}{\pgfqpoint{7.821868in}{3.227820in}}%
\pgfusepath{stroke}%
\end{pgfscope}%
\begin{pgfscope}%
\pgfpathrectangle{\pgfqpoint{6.720588in}{1.750000in}}{\pgfqpoint{2.279412in}{2.004545in}}%
\pgfusepath{clip}%
\pgfsetroundcap%
\pgfsetroundjoin%
\definecolor{currentfill}{rgb}{0.277941,0.056324,0.381191}%
\pgfsetfillcolor{currentfill}%
\pgfsetlinewidth{0.367284pt}%
\definecolor{currentstroke}{rgb}{0.277941,0.056324,0.381191}%
\pgfsetstrokecolor{currentstroke}%
\pgfsetdash{}{0pt}%
\pgfpathmoveto{\pgfqpoint{7.883152in}{3.237934in}}%
\pgfpathlineto{\pgfqpoint{7.821868in}{3.227820in}}%
\pgfpathlineto{\pgfqpoint{7.850547in}{3.282916in}}%
\pgfpathlineto{\pgfqpoint{7.883152in}{3.237934in}}%
\pgfpathlineto{\pgfqpoint{7.883152in}{3.237934in}}%
\pgfpathclose%
\pgfusepath{stroke,fill}%
\end{pgfscope}%
\begin{pgfscope}%
\pgfpathrectangle{\pgfqpoint{6.720588in}{1.750000in}}{\pgfqpoint{2.279412in}{2.004545in}}%
\pgfusepath{clip}%
\pgfsetroundcap%
\pgfsetroundjoin%
\pgfsetlinewidth{0.664473pt}%
\definecolor{currentstroke}{rgb}{0.252194,0.269783,0.531579}%
\pgfsetstrokecolor{currentstroke}%
\pgfsetdash{}{0pt}%
\pgfpathmoveto{\pgfqpoint{7.388368in}{2.853695in}}%
\pgfpathquadraticcurveto{\pgfqpoint{7.399083in}{2.848127in}}{\pgfqpoint{7.400677in}{2.847299in}}%
\pgfusepath{stroke}%
\end{pgfscope}%
\begin{pgfscope}%
\pgfpathrectangle{\pgfqpoint{6.720588in}{1.750000in}}{\pgfqpoint{2.279412in}{2.004545in}}%
\pgfusepath{clip}%
\pgfsetroundcap%
\pgfsetroundjoin%
\definecolor{currentfill}{rgb}{0.252194,0.269783,0.531579}%
\pgfsetfillcolor{currentfill}%
\pgfsetlinewidth{0.664473pt}%
\definecolor{currentstroke}{rgb}{0.252194,0.269783,0.531579}%
\pgfsetstrokecolor{currentstroke}%
\pgfsetdash{}{0pt}%
\pgfpathmoveto{\pgfqpoint{7.364187in}{2.897564in}}%
\pgfpathlineto{\pgfqpoint{7.400677in}{2.847299in}}%
\pgfpathlineto{\pgfqpoint{7.338571in}{2.848267in}}%
\pgfpathlineto{\pgfqpoint{7.364187in}{2.897564in}}%
\pgfpathlineto{\pgfqpoint{7.364187in}{2.897564in}}%
\pgfpathclose%
\pgfusepath{stroke,fill}%
\end{pgfscope}%
\begin{pgfscope}%
\pgfpathrectangle{\pgfqpoint{6.720588in}{1.750000in}}{\pgfqpoint{2.279412in}{2.004545in}}%
\pgfusepath{clip}%
\pgfsetroundcap%
\pgfsetroundjoin%
\pgfsetlinewidth{0.667022pt}%
\definecolor{currentstroke}{rgb}{0.250425,0.274290,0.533103}%
\pgfsetstrokecolor{currentstroke}%
\pgfsetdash{}{0pt}%
\pgfpathmoveto{\pgfqpoint{7.414502in}{2.605198in}}%
\pgfpathquadraticcurveto{\pgfqpoint{7.420662in}{2.610211in}}{\pgfqpoint{7.418818in}{2.608710in}}%
\pgfusepath{stroke}%
\end{pgfscope}%
\begin{pgfscope}%
\pgfpathrectangle{\pgfqpoint{6.720588in}{1.750000in}}{\pgfqpoint{2.279412in}{2.004545in}}%
\pgfusepath{clip}%
\pgfsetroundcap%
\pgfsetroundjoin%
\definecolor{currentfill}{rgb}{0.250425,0.274290,0.533103}%
\pgfsetfillcolor{currentfill}%
\pgfsetlinewidth{0.667022pt}%
\definecolor{currentstroke}{rgb}{0.250425,0.274290,0.533103}%
\pgfsetstrokecolor{currentstroke}%
\pgfsetdash{}{0pt}%
\pgfpathmoveto{\pgfqpoint{7.358194in}{2.595194in}}%
\pgfpathlineto{\pgfqpoint{7.418818in}{2.608710in}}%
\pgfpathlineto{\pgfqpoint{7.393257in}{2.552101in}}%
\pgfpathlineto{\pgfqpoint{7.358194in}{2.595194in}}%
\pgfpathlineto{\pgfqpoint{7.358194in}{2.595194in}}%
\pgfpathclose%
\pgfusepath{stroke,fill}%
\end{pgfscope}%
\begin{pgfscope}%
\pgfpathrectangle{\pgfqpoint{6.720588in}{1.750000in}}{\pgfqpoint{2.279412in}{2.004545in}}%
\pgfusepath{clip}%
\pgfsetroundcap%
\pgfsetroundjoin%
\pgfsetlinewidth{0.667721pt}%
\definecolor{currentstroke}{rgb}{0.250425,0.274290,0.533103}%
\pgfsetstrokecolor{currentstroke}%
\pgfsetdash{}{0pt}%
\pgfpathmoveto{\pgfqpoint{7.498690in}{2.506417in}}%
\pgfpathquadraticcurveto{\pgfqpoint{7.500803in}{2.512381in}}{\pgfqpoint{7.499466in}{2.508608in}}%
\pgfusepath{stroke}%
\end{pgfscope}%
\begin{pgfscope}%
\pgfpathrectangle{\pgfqpoint{6.720588in}{1.750000in}}{\pgfqpoint{2.279412in}{2.004545in}}%
\pgfusepath{clip}%
\pgfsetroundcap%
\pgfsetroundjoin%
\definecolor{currentfill}{rgb}{0.250425,0.274290,0.533103}%
\pgfsetfillcolor{currentfill}%
\pgfsetlinewidth{0.667721pt}%
\definecolor{currentstroke}{rgb}{0.250425,0.274290,0.533103}%
\pgfsetstrokecolor{currentstroke}%
\pgfsetdash{}{0pt}%
\pgfpathmoveto{\pgfqpoint{7.454734in}{2.465513in}}%
\pgfpathlineto{\pgfqpoint{7.499466in}{2.508608in}}%
\pgfpathlineto{\pgfqpoint{7.507102in}{2.446966in}}%
\pgfpathlineto{\pgfqpoint{7.454734in}{2.465513in}}%
\pgfpathlineto{\pgfqpoint{7.454734in}{2.465513in}}%
\pgfpathclose%
\pgfusepath{stroke,fill}%
\end{pgfscope}%
\begin{pgfscope}%
\pgfpathrectangle{\pgfqpoint{6.720588in}{1.750000in}}{\pgfqpoint{2.279412in}{2.004545in}}%
\pgfusepath{clip}%
\pgfsetroundcap%
\pgfsetroundjoin%
\pgfsetlinewidth{0.494696pt}%
\definecolor{currentstroke}{rgb}{0.281412,0.155834,0.469201}%
\pgfsetstrokecolor{currentstroke}%
\pgfsetdash{}{0pt}%
\pgfpathmoveto{\pgfqpoint{7.522067in}{2.435565in}}%
\pgfpathquadraticcurveto{\pgfqpoint{7.526483in}{2.440630in}}{\pgfqpoint{7.525869in}{2.439926in}}%
\pgfusepath{stroke}%
\end{pgfscope}%
\begin{pgfscope}%
\pgfpathrectangle{\pgfqpoint{6.720588in}{1.750000in}}{\pgfqpoint{2.279412in}{2.004545in}}%
\pgfusepath{clip}%
\pgfsetroundcap%
\pgfsetroundjoin%
\definecolor{currentfill}{rgb}{0.281412,0.155834,0.469201}%
\pgfsetfillcolor{currentfill}%
\pgfsetlinewidth{0.494696pt}%
\definecolor{currentstroke}{rgb}{0.281412,0.155834,0.469201}%
\pgfsetstrokecolor{currentstroke}%
\pgfsetdash{}{0pt}%
\pgfpathmoveto{\pgfqpoint{7.468422in}{2.416307in}}%
\pgfpathlineto{\pgfqpoint{7.525869in}{2.439926in}}%
\pgfpathlineto{\pgfqpoint{7.510297in}{2.379797in}}%
\pgfpathlineto{\pgfqpoint{7.468422in}{2.416307in}}%
\pgfpathlineto{\pgfqpoint{7.468422in}{2.416307in}}%
\pgfpathclose%
\pgfusepath{stroke,fill}%
\end{pgfscope}%
\begin{pgfscope}%
\pgfpathrectangle{\pgfqpoint{6.720588in}{1.750000in}}{\pgfqpoint{2.279412in}{2.004545in}}%
\pgfusepath{clip}%
\pgfsetroundcap%
\pgfsetroundjoin%
\pgfsetlinewidth{0.641897pt}%
\definecolor{currentstroke}{rgb}{0.257322,0.256130,0.526563}%
\pgfsetstrokecolor{currentstroke}%
\pgfsetdash{}{0pt}%
\pgfpathmoveto{\pgfqpoint{7.535814in}{2.936450in}}%
\pgfpathquadraticcurveto{\pgfqpoint{7.534699in}{2.930046in}}{\pgfqpoint{7.535287in}{2.933424in}}%
\pgfusepath{stroke}%
\end{pgfscope}%
\begin{pgfscope}%
\pgfpathrectangle{\pgfqpoint{6.720588in}{1.750000in}}{\pgfqpoint{2.279412in}{2.004545in}}%
\pgfusepath{clip}%
\pgfsetroundcap%
\pgfsetroundjoin%
\definecolor{currentfill}{rgb}{0.257322,0.256130,0.526563}%
\pgfsetfillcolor{currentfill}%
\pgfsetlinewidth{0.641897pt}%
\definecolor{currentstroke}{rgb}{0.257322,0.256130,0.526563}%
\pgfsetstrokecolor{currentstroke}%
\pgfsetdash{}{0pt}%
\pgfpathmoveto{\pgfqpoint{7.572182in}{2.983392in}}%
\pgfpathlineto{\pgfqpoint{7.535287in}{2.933424in}}%
\pgfpathlineto{\pgfqpoint{7.517450in}{2.992921in}}%
\pgfpathlineto{\pgfqpoint{7.572182in}{2.983392in}}%
\pgfpathlineto{\pgfqpoint{7.572182in}{2.983392in}}%
\pgfpathclose%
\pgfusepath{stroke,fill}%
\end{pgfscope}%
\begin{pgfscope}%
\pgfpathrectangle{\pgfqpoint{6.720588in}{1.750000in}}{\pgfqpoint{2.279412in}{2.004545in}}%
\pgfusepath{clip}%
\pgfsetbuttcap%
\pgfsetroundjoin%
\pgfsetlinewidth{1.505625pt}%
\definecolor{currentstroke}{rgb}{0.000000,0.000000,0.000000}%
\pgfsetstrokecolor{currentstroke}%
\pgfsetdash{}{0pt}%
\pgfpathmoveto{\pgfqpoint{7.513972in}{2.089441in}}%
\pgfpathlineto{\pgfqpoint{7.513972in}{3.415105in}}%
\pgfusepath{stroke}%
\end{pgfscope}%
\begin{pgfscope}%
\pgfpathrectangle{\pgfqpoint{6.720588in}{1.750000in}}{\pgfqpoint{2.279412in}{2.004545in}}%
\pgfusepath{clip}%
\pgfsetbuttcap%
\pgfsetroundjoin%
\pgfsetlinewidth{1.505625pt}%
\definecolor{currentstroke}{rgb}{0.000000,0.000000,0.000000}%
\pgfsetstrokecolor{currentstroke}%
\pgfsetdash{}{0pt}%
\pgfpathmoveto{\pgfqpoint{8.662499in}{2.089441in}}%
\pgfpathlineto{\pgfqpoint{8.662499in}{3.415105in}}%
\pgfusepath{stroke}%
\end{pgfscope}%
\begin{pgfscope}%
\pgfsetrectcap%
\pgfsetmiterjoin%
\pgfsetlinewidth{0.803000pt}%
\definecolor{currentstroke}{rgb}{0.000000,0.000000,0.000000}%
\pgfsetstrokecolor{currentstroke}%
\pgfsetdash{}{0pt}%
\pgfpathmoveto{\pgfqpoint{6.720588in}{1.750000in}}%
\pgfpathlineto{\pgfqpoint{6.720588in}{3.754545in}}%
\pgfusepath{stroke}%
\end{pgfscope}%
\begin{pgfscope}%
\pgfsetrectcap%
\pgfsetmiterjoin%
\pgfsetlinewidth{0.803000pt}%
\definecolor{currentstroke}{rgb}{0.000000,0.000000,0.000000}%
\pgfsetstrokecolor{currentstroke}%
\pgfsetdash{}{0pt}%
\pgfpathmoveto{\pgfqpoint{9.000000in}{1.750000in}}%
\pgfpathlineto{\pgfqpoint{9.000000in}{3.754545in}}%
\pgfusepath{stroke}%
\end{pgfscope}%
\begin{pgfscope}%
\pgfsetrectcap%
\pgfsetmiterjoin%
\pgfsetlinewidth{0.803000pt}%
\definecolor{currentstroke}{rgb}{0.000000,0.000000,0.000000}%
\pgfsetstrokecolor{currentstroke}%
\pgfsetdash{}{0pt}%
\pgfpathmoveto{\pgfqpoint{6.720588in}{1.750000in}}%
\pgfpathlineto{\pgfqpoint{9.000000in}{1.750000in}}%
\pgfusepath{stroke}%
\end{pgfscope}%
\begin{pgfscope}%
\pgfsetrectcap%
\pgfsetmiterjoin%
\pgfsetlinewidth{0.803000pt}%
\definecolor{currentstroke}{rgb}{0.000000,0.000000,0.000000}%
\pgfsetstrokecolor{currentstroke}%
\pgfsetdash{}{0pt}%
\pgfpathmoveto{\pgfqpoint{6.720588in}{3.754545in}}%
\pgfpathlineto{\pgfqpoint{9.000000in}{3.754545in}}%
\pgfusepath{stroke}%
\end{pgfscope}%
\begin{pgfscope}%
\definecolor{textcolor}{rgb}{0.000000,0.000000,0.000000}%
\pgfsetstrokecolor{textcolor}%
\pgfsetfillcolor{textcolor}%
\pgftext[x=7.860294in,y=3.837879in,,base]{\color{textcolor}\sffamily\fontsize{12.000000}{14.400000}\selectfont f)}%
\end{pgfscope}%
\begin{pgfscope}%
\pgfsetbuttcap%
\pgfsetmiterjoin%
\definecolor{currentfill}{rgb}{1.000000,1.000000,1.000000}%
\pgfsetfillcolor{currentfill}%
\pgfsetlinewidth{0.000000pt}%
\definecolor{currentstroke}{rgb}{0.000000,0.000000,0.000000}%
\pgfsetstrokecolor{currentstroke}%
\pgfsetstrokeopacity{0.000000}%
\pgfsetdash{}{0pt}%
\pgfpathmoveto{\pgfqpoint{3.000000in}{0.700000in}}%
\pgfpathlineto{\pgfqpoint{5.000000in}{0.700000in}}%
\pgfpathlineto{\pgfqpoint{5.000000in}{1.050000in}}%
\pgfpathlineto{\pgfqpoint{3.000000in}{1.050000in}}%
\pgfpathlineto{\pgfqpoint{3.000000in}{0.700000in}}%
\pgfpathclose%
\pgfusepath{fill}%
\end{pgfscope}%
\begin{pgfscope}%
\pgfpathrectangle{\pgfqpoint{3.000000in}{0.700000in}}{\pgfqpoint{2.000000in}{0.350000in}}%
\pgfusepath{clip}%
\pgfsetbuttcap%
\pgfsetmiterjoin%
\definecolor{currentfill}{rgb}{1.000000,1.000000,1.000000}%
\pgfsetfillcolor{currentfill}%
\pgfsetlinewidth{0.010037pt}%
\definecolor{currentstroke}{rgb}{1.000000,1.000000,1.000000}%
\pgfsetstrokecolor{currentstroke}%
\pgfsetdash{}{0pt}%
\pgfusepath{stroke,fill}%
\end{pgfscope}%
\begin{pgfscope}%
\pgfsys@transformshift{3.000000in}{0.694444in}%
\pgftext[left,bottom]{\includegraphics[interpolate=true,width=2.000000in,height=0.361111in]{q_series-img6.png}}%
\end{pgfscope}%
\begin{pgfscope}%
\pgfsetbuttcap%
\pgfsetroundjoin%
\definecolor{currentfill}{rgb}{0.000000,0.000000,0.000000}%
\pgfsetfillcolor{currentfill}%
\pgfsetlinewidth{0.803000pt}%
\definecolor{currentstroke}{rgb}{0.000000,0.000000,0.000000}%
\pgfsetstrokecolor{currentstroke}%
\pgfsetdash{}{0pt}%
\pgfsys@defobject{currentmarker}{\pgfqpoint{0.000000in}{-0.048611in}}{\pgfqpoint{0.000000in}{0.000000in}}{%
\pgfpathmoveto{\pgfqpoint{0.000000in}{0.000000in}}%
\pgfpathlineto{\pgfqpoint{0.000000in}{-0.048611in}}%
\pgfusepath{stroke,fill}%
}%
\begin{pgfscope}%
\pgfsys@transformshift{3.000000in}{0.700000in}%
\pgfsys@useobject{currentmarker}{}%
\end{pgfscope}%
\end{pgfscope}%
\begin{pgfscope}%
\definecolor{textcolor}{rgb}{0.000000,0.000000,0.000000}%
\pgfsetstrokecolor{textcolor}%
\pgfsetfillcolor{textcolor}%
\pgftext[x=3.000000in,y=0.602778in,,top]{\color{textcolor}\sffamily\fontsize{10.000000}{12.000000}\selectfont \(\displaystyle {10^{-1}}\)}%
\end{pgfscope}%
\begin{pgfscope}%
\pgfsetbuttcap%
\pgfsetroundjoin%
\definecolor{currentfill}{rgb}{0.000000,0.000000,0.000000}%
\pgfsetfillcolor{currentfill}%
\pgfsetlinewidth{0.803000pt}%
\definecolor{currentstroke}{rgb}{0.000000,0.000000,0.000000}%
\pgfsetstrokecolor{currentstroke}%
\pgfsetdash{}{0pt}%
\pgfsys@defobject{currentmarker}{\pgfqpoint{0.000000in}{-0.048611in}}{\pgfqpoint{0.000000in}{0.000000in}}{%
\pgfpathmoveto{\pgfqpoint{0.000000in}{0.000000in}}%
\pgfpathlineto{\pgfqpoint{0.000000in}{-0.048611in}}%
\pgfusepath{stroke,fill}%
}%
\begin{pgfscope}%
\pgfsys@transformshift{4.000000in}{0.700000in}%
\pgfsys@useobject{currentmarker}{}%
\end{pgfscope}%
\end{pgfscope}%
\begin{pgfscope}%
\definecolor{textcolor}{rgb}{0.000000,0.000000,0.000000}%
\pgfsetstrokecolor{textcolor}%
\pgfsetfillcolor{textcolor}%
\pgftext[x=4.000000in,y=0.602778in,,top]{\color{textcolor}\sffamily\fontsize{10.000000}{12.000000}\selectfont \(\displaystyle {10^{0}}\)}%
\end{pgfscope}%
\begin{pgfscope}%
\pgfsetbuttcap%
\pgfsetroundjoin%
\definecolor{currentfill}{rgb}{0.000000,0.000000,0.000000}%
\pgfsetfillcolor{currentfill}%
\pgfsetlinewidth{0.803000pt}%
\definecolor{currentstroke}{rgb}{0.000000,0.000000,0.000000}%
\pgfsetstrokecolor{currentstroke}%
\pgfsetdash{}{0pt}%
\pgfsys@defobject{currentmarker}{\pgfqpoint{0.000000in}{-0.048611in}}{\pgfqpoint{0.000000in}{0.000000in}}{%
\pgfpathmoveto{\pgfqpoint{0.000000in}{0.000000in}}%
\pgfpathlineto{\pgfqpoint{0.000000in}{-0.048611in}}%
\pgfusepath{stroke,fill}%
}%
\begin{pgfscope}%
\pgfsys@transformshift{5.000000in}{0.700000in}%
\pgfsys@useobject{currentmarker}{}%
\end{pgfscope}%
\end{pgfscope}%
\begin{pgfscope}%
\definecolor{textcolor}{rgb}{0.000000,0.000000,0.000000}%
\pgfsetstrokecolor{textcolor}%
\pgfsetfillcolor{textcolor}%
\pgftext[x=5.000000in,y=0.602778in,,top]{\color{textcolor}\sffamily\fontsize{10.000000}{12.000000}\selectfont \(\displaystyle {10^{1}}\)}%
\end{pgfscope}%
\begin{pgfscope}%
\pgfsetbuttcap%
\pgfsetroundjoin%
\definecolor{currentfill}{rgb}{0.000000,0.000000,0.000000}%
\pgfsetfillcolor{currentfill}%
\pgfsetlinewidth{0.602250pt}%
\definecolor{currentstroke}{rgb}{0.000000,0.000000,0.000000}%
\pgfsetstrokecolor{currentstroke}%
\pgfsetdash{}{0pt}%
\pgfsys@defobject{currentmarker}{\pgfqpoint{0.000000in}{-0.027778in}}{\pgfqpoint{0.000000in}{0.000000in}}{%
\pgfpathmoveto{\pgfqpoint{0.000000in}{0.000000in}}%
\pgfpathlineto{\pgfqpoint{0.000000in}{-0.027778in}}%
\pgfusepath{stroke,fill}%
}%
\begin{pgfscope}%
\pgfsys@transformshift{3.301030in}{0.700000in}%
\pgfsys@useobject{currentmarker}{}%
\end{pgfscope}%
\end{pgfscope}%
\begin{pgfscope}%
\pgfsetbuttcap%
\pgfsetroundjoin%
\definecolor{currentfill}{rgb}{0.000000,0.000000,0.000000}%
\pgfsetfillcolor{currentfill}%
\pgfsetlinewidth{0.602250pt}%
\definecolor{currentstroke}{rgb}{0.000000,0.000000,0.000000}%
\pgfsetstrokecolor{currentstroke}%
\pgfsetdash{}{0pt}%
\pgfsys@defobject{currentmarker}{\pgfqpoint{0.000000in}{-0.027778in}}{\pgfqpoint{0.000000in}{0.000000in}}{%
\pgfpathmoveto{\pgfqpoint{0.000000in}{0.000000in}}%
\pgfpathlineto{\pgfqpoint{0.000000in}{-0.027778in}}%
\pgfusepath{stroke,fill}%
}%
\begin{pgfscope}%
\pgfsys@transformshift{3.477121in}{0.700000in}%
\pgfsys@useobject{currentmarker}{}%
\end{pgfscope}%
\end{pgfscope}%
\begin{pgfscope}%
\pgfsetbuttcap%
\pgfsetroundjoin%
\definecolor{currentfill}{rgb}{0.000000,0.000000,0.000000}%
\pgfsetfillcolor{currentfill}%
\pgfsetlinewidth{0.602250pt}%
\definecolor{currentstroke}{rgb}{0.000000,0.000000,0.000000}%
\pgfsetstrokecolor{currentstroke}%
\pgfsetdash{}{0pt}%
\pgfsys@defobject{currentmarker}{\pgfqpoint{0.000000in}{-0.027778in}}{\pgfqpoint{0.000000in}{0.000000in}}{%
\pgfpathmoveto{\pgfqpoint{0.000000in}{0.000000in}}%
\pgfpathlineto{\pgfqpoint{0.000000in}{-0.027778in}}%
\pgfusepath{stroke,fill}%
}%
\begin{pgfscope}%
\pgfsys@transformshift{3.602060in}{0.700000in}%
\pgfsys@useobject{currentmarker}{}%
\end{pgfscope}%
\end{pgfscope}%
\begin{pgfscope}%
\pgfsetbuttcap%
\pgfsetroundjoin%
\definecolor{currentfill}{rgb}{0.000000,0.000000,0.000000}%
\pgfsetfillcolor{currentfill}%
\pgfsetlinewidth{0.602250pt}%
\definecolor{currentstroke}{rgb}{0.000000,0.000000,0.000000}%
\pgfsetstrokecolor{currentstroke}%
\pgfsetdash{}{0pt}%
\pgfsys@defobject{currentmarker}{\pgfqpoint{0.000000in}{-0.027778in}}{\pgfqpoint{0.000000in}{0.000000in}}{%
\pgfpathmoveto{\pgfqpoint{0.000000in}{0.000000in}}%
\pgfpathlineto{\pgfqpoint{0.000000in}{-0.027778in}}%
\pgfusepath{stroke,fill}%
}%
\begin{pgfscope}%
\pgfsys@transformshift{3.698970in}{0.700000in}%
\pgfsys@useobject{currentmarker}{}%
\end{pgfscope}%
\end{pgfscope}%
\begin{pgfscope}%
\pgfsetbuttcap%
\pgfsetroundjoin%
\definecolor{currentfill}{rgb}{0.000000,0.000000,0.000000}%
\pgfsetfillcolor{currentfill}%
\pgfsetlinewidth{0.602250pt}%
\definecolor{currentstroke}{rgb}{0.000000,0.000000,0.000000}%
\pgfsetstrokecolor{currentstroke}%
\pgfsetdash{}{0pt}%
\pgfsys@defobject{currentmarker}{\pgfqpoint{0.000000in}{-0.027778in}}{\pgfqpoint{0.000000in}{0.000000in}}{%
\pgfpathmoveto{\pgfqpoint{0.000000in}{0.000000in}}%
\pgfpathlineto{\pgfqpoint{0.000000in}{-0.027778in}}%
\pgfusepath{stroke,fill}%
}%
\begin{pgfscope}%
\pgfsys@transformshift{3.778151in}{0.700000in}%
\pgfsys@useobject{currentmarker}{}%
\end{pgfscope}%
\end{pgfscope}%
\begin{pgfscope}%
\pgfsetbuttcap%
\pgfsetroundjoin%
\definecolor{currentfill}{rgb}{0.000000,0.000000,0.000000}%
\pgfsetfillcolor{currentfill}%
\pgfsetlinewidth{0.602250pt}%
\definecolor{currentstroke}{rgb}{0.000000,0.000000,0.000000}%
\pgfsetstrokecolor{currentstroke}%
\pgfsetdash{}{0pt}%
\pgfsys@defobject{currentmarker}{\pgfqpoint{0.000000in}{-0.027778in}}{\pgfqpoint{0.000000in}{0.000000in}}{%
\pgfpathmoveto{\pgfqpoint{0.000000in}{0.000000in}}%
\pgfpathlineto{\pgfqpoint{0.000000in}{-0.027778in}}%
\pgfusepath{stroke,fill}%
}%
\begin{pgfscope}%
\pgfsys@transformshift{3.845098in}{0.700000in}%
\pgfsys@useobject{currentmarker}{}%
\end{pgfscope}%
\end{pgfscope}%
\begin{pgfscope}%
\pgfsetbuttcap%
\pgfsetroundjoin%
\definecolor{currentfill}{rgb}{0.000000,0.000000,0.000000}%
\pgfsetfillcolor{currentfill}%
\pgfsetlinewidth{0.602250pt}%
\definecolor{currentstroke}{rgb}{0.000000,0.000000,0.000000}%
\pgfsetstrokecolor{currentstroke}%
\pgfsetdash{}{0pt}%
\pgfsys@defobject{currentmarker}{\pgfqpoint{0.000000in}{-0.027778in}}{\pgfqpoint{0.000000in}{0.000000in}}{%
\pgfpathmoveto{\pgfqpoint{0.000000in}{0.000000in}}%
\pgfpathlineto{\pgfqpoint{0.000000in}{-0.027778in}}%
\pgfusepath{stroke,fill}%
}%
\begin{pgfscope}%
\pgfsys@transformshift{3.903090in}{0.700000in}%
\pgfsys@useobject{currentmarker}{}%
\end{pgfscope}%
\end{pgfscope}%
\begin{pgfscope}%
\pgfsetbuttcap%
\pgfsetroundjoin%
\definecolor{currentfill}{rgb}{0.000000,0.000000,0.000000}%
\pgfsetfillcolor{currentfill}%
\pgfsetlinewidth{0.602250pt}%
\definecolor{currentstroke}{rgb}{0.000000,0.000000,0.000000}%
\pgfsetstrokecolor{currentstroke}%
\pgfsetdash{}{0pt}%
\pgfsys@defobject{currentmarker}{\pgfqpoint{0.000000in}{-0.027778in}}{\pgfqpoint{0.000000in}{0.000000in}}{%
\pgfpathmoveto{\pgfqpoint{0.000000in}{0.000000in}}%
\pgfpathlineto{\pgfqpoint{0.000000in}{-0.027778in}}%
\pgfusepath{stroke,fill}%
}%
\begin{pgfscope}%
\pgfsys@transformshift{3.954243in}{0.700000in}%
\pgfsys@useobject{currentmarker}{}%
\end{pgfscope}%
\end{pgfscope}%
\begin{pgfscope}%
\pgfsetbuttcap%
\pgfsetroundjoin%
\definecolor{currentfill}{rgb}{0.000000,0.000000,0.000000}%
\pgfsetfillcolor{currentfill}%
\pgfsetlinewidth{0.602250pt}%
\definecolor{currentstroke}{rgb}{0.000000,0.000000,0.000000}%
\pgfsetstrokecolor{currentstroke}%
\pgfsetdash{}{0pt}%
\pgfsys@defobject{currentmarker}{\pgfqpoint{0.000000in}{-0.027778in}}{\pgfqpoint{0.000000in}{0.000000in}}{%
\pgfpathmoveto{\pgfqpoint{0.000000in}{0.000000in}}%
\pgfpathlineto{\pgfqpoint{0.000000in}{-0.027778in}}%
\pgfusepath{stroke,fill}%
}%
\begin{pgfscope}%
\pgfsys@transformshift{4.301030in}{0.700000in}%
\pgfsys@useobject{currentmarker}{}%
\end{pgfscope}%
\end{pgfscope}%
\begin{pgfscope}%
\pgfsetbuttcap%
\pgfsetroundjoin%
\definecolor{currentfill}{rgb}{0.000000,0.000000,0.000000}%
\pgfsetfillcolor{currentfill}%
\pgfsetlinewidth{0.602250pt}%
\definecolor{currentstroke}{rgb}{0.000000,0.000000,0.000000}%
\pgfsetstrokecolor{currentstroke}%
\pgfsetdash{}{0pt}%
\pgfsys@defobject{currentmarker}{\pgfqpoint{0.000000in}{-0.027778in}}{\pgfqpoint{0.000000in}{0.000000in}}{%
\pgfpathmoveto{\pgfqpoint{0.000000in}{0.000000in}}%
\pgfpathlineto{\pgfqpoint{0.000000in}{-0.027778in}}%
\pgfusepath{stroke,fill}%
}%
\begin{pgfscope}%
\pgfsys@transformshift{4.477121in}{0.700000in}%
\pgfsys@useobject{currentmarker}{}%
\end{pgfscope}%
\end{pgfscope}%
\begin{pgfscope}%
\pgfsetbuttcap%
\pgfsetroundjoin%
\definecolor{currentfill}{rgb}{0.000000,0.000000,0.000000}%
\pgfsetfillcolor{currentfill}%
\pgfsetlinewidth{0.602250pt}%
\definecolor{currentstroke}{rgb}{0.000000,0.000000,0.000000}%
\pgfsetstrokecolor{currentstroke}%
\pgfsetdash{}{0pt}%
\pgfsys@defobject{currentmarker}{\pgfqpoint{0.000000in}{-0.027778in}}{\pgfqpoint{0.000000in}{0.000000in}}{%
\pgfpathmoveto{\pgfqpoint{0.000000in}{0.000000in}}%
\pgfpathlineto{\pgfqpoint{0.000000in}{-0.027778in}}%
\pgfusepath{stroke,fill}%
}%
\begin{pgfscope}%
\pgfsys@transformshift{4.602060in}{0.700000in}%
\pgfsys@useobject{currentmarker}{}%
\end{pgfscope}%
\end{pgfscope}%
\begin{pgfscope}%
\pgfsetbuttcap%
\pgfsetroundjoin%
\definecolor{currentfill}{rgb}{0.000000,0.000000,0.000000}%
\pgfsetfillcolor{currentfill}%
\pgfsetlinewidth{0.602250pt}%
\definecolor{currentstroke}{rgb}{0.000000,0.000000,0.000000}%
\pgfsetstrokecolor{currentstroke}%
\pgfsetdash{}{0pt}%
\pgfsys@defobject{currentmarker}{\pgfqpoint{0.000000in}{-0.027778in}}{\pgfqpoint{0.000000in}{0.000000in}}{%
\pgfpathmoveto{\pgfqpoint{0.000000in}{0.000000in}}%
\pgfpathlineto{\pgfqpoint{0.000000in}{-0.027778in}}%
\pgfusepath{stroke,fill}%
}%
\begin{pgfscope}%
\pgfsys@transformshift{4.698970in}{0.700000in}%
\pgfsys@useobject{currentmarker}{}%
\end{pgfscope}%
\end{pgfscope}%
\begin{pgfscope}%
\pgfsetbuttcap%
\pgfsetroundjoin%
\definecolor{currentfill}{rgb}{0.000000,0.000000,0.000000}%
\pgfsetfillcolor{currentfill}%
\pgfsetlinewidth{0.602250pt}%
\definecolor{currentstroke}{rgb}{0.000000,0.000000,0.000000}%
\pgfsetstrokecolor{currentstroke}%
\pgfsetdash{}{0pt}%
\pgfsys@defobject{currentmarker}{\pgfqpoint{0.000000in}{-0.027778in}}{\pgfqpoint{0.000000in}{0.000000in}}{%
\pgfpathmoveto{\pgfqpoint{0.000000in}{0.000000in}}%
\pgfpathlineto{\pgfqpoint{0.000000in}{-0.027778in}}%
\pgfusepath{stroke,fill}%
}%
\begin{pgfscope}%
\pgfsys@transformshift{4.778151in}{0.700000in}%
\pgfsys@useobject{currentmarker}{}%
\end{pgfscope}%
\end{pgfscope}%
\begin{pgfscope}%
\pgfsetbuttcap%
\pgfsetroundjoin%
\definecolor{currentfill}{rgb}{0.000000,0.000000,0.000000}%
\pgfsetfillcolor{currentfill}%
\pgfsetlinewidth{0.602250pt}%
\definecolor{currentstroke}{rgb}{0.000000,0.000000,0.000000}%
\pgfsetstrokecolor{currentstroke}%
\pgfsetdash{}{0pt}%
\pgfsys@defobject{currentmarker}{\pgfqpoint{0.000000in}{-0.027778in}}{\pgfqpoint{0.000000in}{0.000000in}}{%
\pgfpathmoveto{\pgfqpoint{0.000000in}{0.000000in}}%
\pgfpathlineto{\pgfqpoint{0.000000in}{-0.027778in}}%
\pgfusepath{stroke,fill}%
}%
\begin{pgfscope}%
\pgfsys@transformshift{4.845098in}{0.700000in}%
\pgfsys@useobject{currentmarker}{}%
\end{pgfscope}%
\end{pgfscope}%
\begin{pgfscope}%
\pgfsetbuttcap%
\pgfsetroundjoin%
\definecolor{currentfill}{rgb}{0.000000,0.000000,0.000000}%
\pgfsetfillcolor{currentfill}%
\pgfsetlinewidth{0.602250pt}%
\definecolor{currentstroke}{rgb}{0.000000,0.000000,0.000000}%
\pgfsetstrokecolor{currentstroke}%
\pgfsetdash{}{0pt}%
\pgfsys@defobject{currentmarker}{\pgfqpoint{0.000000in}{-0.027778in}}{\pgfqpoint{0.000000in}{0.000000in}}{%
\pgfpathmoveto{\pgfqpoint{0.000000in}{0.000000in}}%
\pgfpathlineto{\pgfqpoint{0.000000in}{-0.027778in}}%
\pgfusepath{stroke,fill}%
}%
\begin{pgfscope}%
\pgfsys@transformshift{4.903090in}{0.700000in}%
\pgfsys@useobject{currentmarker}{}%
\end{pgfscope}%
\end{pgfscope}%
\begin{pgfscope}%
\pgfsetbuttcap%
\pgfsetroundjoin%
\definecolor{currentfill}{rgb}{0.000000,0.000000,0.000000}%
\pgfsetfillcolor{currentfill}%
\pgfsetlinewidth{0.602250pt}%
\definecolor{currentstroke}{rgb}{0.000000,0.000000,0.000000}%
\pgfsetstrokecolor{currentstroke}%
\pgfsetdash{}{0pt}%
\pgfsys@defobject{currentmarker}{\pgfqpoint{0.000000in}{-0.027778in}}{\pgfqpoint{0.000000in}{0.000000in}}{%
\pgfpathmoveto{\pgfqpoint{0.000000in}{0.000000in}}%
\pgfpathlineto{\pgfqpoint{0.000000in}{-0.027778in}}%
\pgfusepath{stroke,fill}%
}%
\begin{pgfscope}%
\pgfsys@transformshift{4.954243in}{0.700000in}%
\pgfsys@useobject{currentmarker}{}%
\end{pgfscope}%
\end{pgfscope}%
\begin{pgfscope}%
\definecolor{textcolor}{rgb}{0.000000,0.000000,0.000000}%
\pgfsetstrokecolor{textcolor}%
\pgfsetfillcolor{textcolor}%
\pgftext[x=4.000000in,y=0.423766in,,top]{\color{textcolor}\sffamily\fontsize{10.000000}{12.000000}\selectfont \(\displaystyle dQ/dy/dz \, \mathrm{[pC/\mu m^2]}\)}%
\end{pgfscope}%
\begin{pgfscope}%
\pgfsetrectcap%
\pgfsetmiterjoin%
\pgfsetlinewidth{0.803000pt}%
\definecolor{currentstroke}{rgb}{0.000000,0.000000,0.000000}%
\pgfsetstrokecolor{currentstroke}%
\pgfsetdash{}{0pt}%
\pgfpathmoveto{\pgfqpoint{3.000000in}{0.700000in}}%
\pgfpathlineto{\pgfqpoint{3.000000in}{0.875000in}}%
\pgfpathlineto{\pgfqpoint{3.000000in}{1.050000in}}%
\pgfpathlineto{\pgfqpoint{5.000000in}{1.050000in}}%
\pgfpathlineto{\pgfqpoint{5.000000in}{0.875000in}}%
\pgfpathlineto{\pgfqpoint{5.000000in}{0.700000in}}%
\pgfpathlineto{\pgfqpoint{3.000000in}{0.700000in}}%
\pgfpathclose%
\pgfusepath{stroke}%
\end{pgfscope}%
\begin{pgfscope}%
\pgfsetbuttcap%
\pgfsetmiterjoin%
\definecolor{currentfill}{rgb}{1.000000,1.000000,1.000000}%
\pgfsetfillcolor{currentfill}%
\pgfsetlinewidth{0.000000pt}%
\definecolor{currentstroke}{rgb}{0.000000,0.000000,0.000000}%
\pgfsetstrokecolor{currentstroke}%
\pgfsetstrokeopacity{0.000000}%
\pgfsetdash{}{0pt}%
\pgfpathmoveto{\pgfqpoint{5.500000in}{0.700000in}}%
\pgfpathlineto{\pgfqpoint{7.500000in}{0.700000in}}%
\pgfpathlineto{\pgfqpoint{7.500000in}{1.050000in}}%
\pgfpathlineto{\pgfqpoint{5.500000in}{1.050000in}}%
\pgfpathlineto{\pgfqpoint{5.500000in}{0.700000in}}%
\pgfpathclose%
\pgfusepath{fill}%
\end{pgfscope}%
\begin{pgfscope}%
\pgfpathrectangle{\pgfqpoint{5.500000in}{0.700000in}}{\pgfqpoint{2.000000in}{0.350000in}}%
\pgfusepath{clip}%
\pgfsetbuttcap%
\pgfsetmiterjoin%
\definecolor{currentfill}{rgb}{1.000000,1.000000,1.000000}%
\pgfsetfillcolor{currentfill}%
\pgfsetlinewidth{0.010037pt}%
\definecolor{currentstroke}{rgb}{1.000000,1.000000,1.000000}%
\pgfsetstrokecolor{currentstroke}%
\pgfsetdash{}{0pt}%
\pgfusepath{stroke,fill}%
\end{pgfscope}%
\begin{pgfscope}%
\pgfsys@transformshift{5.500000in}{0.694444in}%
\pgftext[left,bottom]{\includegraphics[interpolate=true,width=2.000000in,height=0.361111in]{q_series-img7.png}}%
\end{pgfscope}%
\begin{pgfscope}%
\pgfsetbuttcap%
\pgfsetroundjoin%
\definecolor{currentfill}{rgb}{0.000000,0.000000,0.000000}%
\pgfsetfillcolor{currentfill}%
\pgfsetlinewidth{0.803000pt}%
\definecolor{currentstroke}{rgb}{0.000000,0.000000,0.000000}%
\pgfsetstrokecolor{currentstroke}%
\pgfsetdash{}{0pt}%
\pgfsys@defobject{currentmarker}{\pgfqpoint{0.000000in}{-0.048611in}}{\pgfqpoint{0.000000in}{0.000000in}}{%
\pgfpathmoveto{\pgfqpoint{0.000000in}{0.000000in}}%
\pgfpathlineto{\pgfqpoint{0.000000in}{-0.048611in}}%
\pgfusepath{stroke,fill}%
}%
\begin{pgfscope}%
\pgfsys@transformshift{5.500000in}{0.700000in}%
\pgfsys@useobject{currentmarker}{}%
\end{pgfscope}%
\end{pgfscope}%
\begin{pgfscope}%
\definecolor{textcolor}{rgb}{0.000000,0.000000,0.000000}%
\pgfsetstrokecolor{textcolor}%
\pgfsetfillcolor{textcolor}%
\pgftext[x=5.500000in,y=0.602778in,,top]{\color{textcolor}\sffamily\fontsize{10.000000}{12.000000}\selectfont \(\displaystyle {0}\)}%
\end{pgfscope}%
\begin{pgfscope}%
\pgfsetbuttcap%
\pgfsetroundjoin%
\definecolor{currentfill}{rgb}{0.000000,0.000000,0.000000}%
\pgfsetfillcolor{currentfill}%
\pgfsetlinewidth{0.803000pt}%
\definecolor{currentstroke}{rgb}{0.000000,0.000000,0.000000}%
\pgfsetstrokecolor{currentstroke}%
\pgfsetdash{}{0pt}%
\pgfsys@defobject{currentmarker}{\pgfqpoint{0.000000in}{-0.048611in}}{\pgfqpoint{0.000000in}{0.000000in}}{%
\pgfpathmoveto{\pgfqpoint{0.000000in}{0.000000in}}%
\pgfpathlineto{\pgfqpoint{0.000000in}{-0.048611in}}%
\pgfusepath{stroke,fill}%
}%
\begin{pgfscope}%
\pgfsys@transformshift{6.071429in}{0.700000in}%
\pgfsys@useobject{currentmarker}{}%
\end{pgfscope}%
\end{pgfscope}%
\begin{pgfscope}%
\definecolor{textcolor}{rgb}{0.000000,0.000000,0.000000}%
\pgfsetstrokecolor{textcolor}%
\pgfsetfillcolor{textcolor}%
\pgftext[x=6.071429in,y=0.602778in,,top]{\color{textcolor}\sffamily\fontsize{10.000000}{12.000000}\selectfont \(\displaystyle {200}\)}%
\end{pgfscope}%
\begin{pgfscope}%
\pgfsetbuttcap%
\pgfsetroundjoin%
\definecolor{currentfill}{rgb}{0.000000,0.000000,0.000000}%
\pgfsetfillcolor{currentfill}%
\pgfsetlinewidth{0.803000pt}%
\definecolor{currentstroke}{rgb}{0.000000,0.000000,0.000000}%
\pgfsetstrokecolor{currentstroke}%
\pgfsetdash{}{0pt}%
\pgfsys@defobject{currentmarker}{\pgfqpoint{0.000000in}{-0.048611in}}{\pgfqpoint{0.000000in}{0.000000in}}{%
\pgfpathmoveto{\pgfqpoint{0.000000in}{0.000000in}}%
\pgfpathlineto{\pgfqpoint{0.000000in}{-0.048611in}}%
\pgfusepath{stroke,fill}%
}%
\begin{pgfscope}%
\pgfsys@transformshift{6.642857in}{0.700000in}%
\pgfsys@useobject{currentmarker}{}%
\end{pgfscope}%
\end{pgfscope}%
\begin{pgfscope}%
\definecolor{textcolor}{rgb}{0.000000,0.000000,0.000000}%
\pgfsetstrokecolor{textcolor}%
\pgfsetfillcolor{textcolor}%
\pgftext[x=6.642857in,y=0.602778in,,top]{\color{textcolor}\sffamily\fontsize{10.000000}{12.000000}\selectfont \(\displaystyle {400}\)}%
\end{pgfscope}%
\begin{pgfscope}%
\pgfsetbuttcap%
\pgfsetroundjoin%
\definecolor{currentfill}{rgb}{0.000000,0.000000,0.000000}%
\pgfsetfillcolor{currentfill}%
\pgfsetlinewidth{0.803000pt}%
\definecolor{currentstroke}{rgb}{0.000000,0.000000,0.000000}%
\pgfsetstrokecolor{currentstroke}%
\pgfsetdash{}{0pt}%
\pgfsys@defobject{currentmarker}{\pgfqpoint{0.000000in}{-0.048611in}}{\pgfqpoint{0.000000in}{0.000000in}}{%
\pgfpathmoveto{\pgfqpoint{0.000000in}{0.000000in}}%
\pgfpathlineto{\pgfqpoint{0.000000in}{-0.048611in}}%
\pgfusepath{stroke,fill}%
}%
\begin{pgfscope}%
\pgfsys@transformshift{7.214286in}{0.700000in}%
\pgfsys@useobject{currentmarker}{}%
\end{pgfscope}%
\end{pgfscope}%
\begin{pgfscope}%
\definecolor{textcolor}{rgb}{0.000000,0.000000,0.000000}%
\pgfsetstrokecolor{textcolor}%
\pgfsetfillcolor{textcolor}%
\pgftext[x=7.214286in,y=0.602778in,,top]{\color{textcolor}\sffamily\fontsize{10.000000}{12.000000}\selectfont \(\displaystyle {600}\)}%
\end{pgfscope}%
\begin{pgfscope}%
\definecolor{textcolor}{rgb}{0.000000,0.000000,0.000000}%
\pgfsetstrokecolor{textcolor}%
\pgfsetfillcolor{textcolor}%
\pgftext[x=6.500000in,y=0.423766in,,top]{\color{textcolor}\sffamily\fontsize{10.000000}{12.000000}\selectfont \(\displaystyle F_{Lorentz} \, \mathrm{[mN]}\)}%
\end{pgfscope}%
\begin{pgfscope}%
\pgfsetrectcap%
\pgfsetmiterjoin%
\pgfsetlinewidth{0.803000pt}%
\definecolor{currentstroke}{rgb}{0.000000,0.000000,0.000000}%
\pgfsetstrokecolor{currentstroke}%
\pgfsetdash{}{0pt}%
\pgfpathmoveto{\pgfqpoint{5.500000in}{0.700000in}}%
\pgfpathlineto{\pgfqpoint{5.500000in}{0.875000in}}%
\pgfpathlineto{\pgfqpoint{5.500000in}{1.050000in}}%
\pgfpathlineto{\pgfqpoint{7.500000in}{1.050000in}}%
\pgfpathlineto{\pgfqpoint{7.500000in}{0.875000in}}%
\pgfpathlineto{\pgfqpoint{7.500000in}{0.700000in}}%
\pgfpathlineto{\pgfqpoint{5.500000in}{0.700000in}}%
\pgfpathclose%
\pgfusepath{stroke}%
\end{pgfscope}%
\end{pgfpicture}%
\makeatother%
\endgroup%

	\caption{Time series of a charge density histogram of the driver electrons integrated over the $x$-axis in Log scale. The acting Lorentz Force is drawn on top as vector lines. Vertical lines are drawn to make changes in length of the bunch better visible. 
	The histogram is plotted in the $\zeta$-$z$ plane, with $\zeta$ being a co-moving coordinate axis. The distance $y$ traversed in the plasma is given.
	\textbf{a)} ($y=$ \qty{0.04}{mm}) When entering the plasma. Still a Gaussian distribution, weak focus forces at the backside.
	\textbf{b)} ($y=$ \qty{0.36}{mm}) Formation of the tail from focusing forces. Decelerating forces grow.
	\textbf{c)} ($y=$ \qty{0.76}{mm}) First wing spreads from the tail. Strong decelerating forces on the backside.
	\textbf{d)} ($y=$ \qty{1.08}{mm}) More wings emerge and spread. Still strong decelerating forces.
	\textbf{e)} ($y=$ \qty{2.90}{mm}) Shortly before bunch breakup with visible elongation of the bunch. Only weak forces remain.
	\textbf{f)} ($y=$ \qty{3.38}{mm}) Bunch after breakup. The fallen back part gets accelerated from the backside of the first cavity.}
	\label{fig:q_series}
\end{figure}

A log scale is chosen to make the borders with low density visible. Additionally, the Lorentz force is layered over the histogram to visualize the cause of the transformation of the driver. It is retrieved from the $\vec{E}$- and $\vec{B}$-field, that every macroparticle stores.
The window is then separated into bins and for every bin the mean force on the macroparticles was calculated and plotted as a force field, with the color of the lines quantifying the absolute force. 
This is similar to the perspective, that there is one macroparticle per bin and the force acting on this particle is plotted.

At start, the spatial distribution still follows a 2D Gaussian distribution, as seen in \autoref{chap:init}. After the driver propagated for \qty{0.3}{mm} through the plasma, the first cavities start to arise. The focusing Lorentz force of these cavities forms a tail at the end of the driver while the cavities are still in the linear regime.
Only small forces act on the front of the beam, so that the front part can diverge freely. In the center and back of the driving bunch are great forces, pushing the particles back and simultaneously centering them, resulting in the creation of a tail.
These forces result from the formation of the first cavity behind the driver. As can be seen in \autoref{fig:cavity}b, the nearest cavity forms directly behind the center of the driver so the backside already experiences the decelerating and focusing fields of the cavity.
\begin{figure}
	\centering
	%% Creator: Matplotlib, PGF backend
%%
%% To include the figure in your LaTeX document, write
%%   \input{<filename>.pgf}
%%
%% Make sure the required packages are loaded in your preamble
%%   \usepackage{pgf}
%%
%% Also ensure that all the required font packages are loaded; for instance,
%% the lmodern package is sometimes necessary when using math font.
%%   \usepackage{lmodern}
%%
%% Figures using additional raster images can only be included by \input if
%% they are in the same directory as the main LaTeX file. For loading figures
%% from other directories you can use the `import` package
%%   \usepackage{import}
%%
%% and then include the figures with
%%   \import{<path to file>}{<filename>.pgf}
%%
%% Matplotlib used the following preamble
%%
\begingroup%
\makeatletter%
\begin{pgfpicture}%
\pgfpathrectangle{\pgfpointorigin}{\pgfqpoint{6.400000in}{5.500000in}}%
\pgfusepath{use as bounding box, clip}%
\begin{pgfscope}%
\pgfsetbuttcap%
\pgfsetmiterjoin%
\pgfsetlinewidth{0.000000pt}%
\definecolor{currentstroke}{rgb}{1.000000,1.000000,1.000000}%
\pgfsetstrokecolor{currentstroke}%
\pgfsetstrokeopacity{0.000000}%
\pgfsetdash{}{0pt}%
\pgfpathmoveto{\pgfqpoint{0.000000in}{0.000000in}}%
\pgfpathlineto{\pgfqpoint{6.400000in}{0.000000in}}%
\pgfpathlineto{\pgfqpoint{6.400000in}{5.500000in}}%
\pgfpathlineto{\pgfqpoint{0.000000in}{5.500000in}}%
\pgfpathlineto{\pgfqpoint{0.000000in}{0.000000in}}%
\pgfpathclose%
\pgfusepath{}%
\end{pgfscope}%
\begin{pgfscope}%
\pgfsetbuttcap%
\pgfsetmiterjoin%
\definecolor{currentfill}{rgb}{1.000000,1.000000,1.000000}%
\pgfsetfillcolor{currentfill}%
\pgfsetlinewidth{0.000000pt}%
\definecolor{currentstroke}{rgb}{0.000000,0.000000,0.000000}%
\pgfsetstrokecolor{currentstroke}%
\pgfsetstrokeopacity{0.000000}%
\pgfsetdash{}{0pt}%
\pgfpathmoveto{\pgfqpoint{0.800000in}{2.994444in}}%
\pgfpathlineto{\pgfqpoint{2.809302in}{2.994444in}}%
\pgfpathlineto{\pgfqpoint{2.809302in}{4.840000in}}%
\pgfpathlineto{\pgfqpoint{0.800000in}{4.840000in}}%
\pgfpathlineto{\pgfqpoint{0.800000in}{2.994444in}}%
\pgfpathclose%
\pgfusepath{fill}%
\end{pgfscope}%
\begin{pgfscope}%
\pgfpathrectangle{\pgfqpoint{0.800000in}{2.994444in}}{\pgfqpoint{2.009302in}{1.845556in}}%
\pgfusepath{clip}%
\pgfsys@transformcm{2.013889}{0.000000}{0.000000}{1.847222}{0.800000in}{2.994444in}%
\pgftext[left,bottom]{\includegraphics[interpolate=false,width=1.000000in,height=1.000000in]{cavity-img0.png}}%
\end{pgfscope}%
\begin{pgfscope}%
\pgfpathrectangle{\pgfqpoint{0.800000in}{2.994444in}}{\pgfqpoint{2.009302in}{1.845556in}}%
\pgfusepath{clip}%
\pgfsys@transformcm{2.013889}{0.000000}{0.000000}{1.847222}{0.800000in}{2.994444in}%
\pgftext[left,bottom]{\includegraphics[interpolate=false,width=1.000000in,height=1.000000in]{cavity-img1.png}}%
\end{pgfscope}%
\begin{pgfscope}%
\pgfsetbuttcap%
\pgfsetroundjoin%
\definecolor{currentfill}{rgb}{0.000000,0.000000,0.000000}%
\pgfsetfillcolor{currentfill}%
\pgfsetlinewidth{0.803000pt}%
\definecolor{currentstroke}{rgb}{0.000000,0.000000,0.000000}%
\pgfsetstrokecolor{currentstroke}%
\pgfsetdash{}{0pt}%
\pgfsys@defobject{currentmarker}{\pgfqpoint{0.000000in}{-0.048611in}}{\pgfqpoint{0.000000in}{0.000000in}}{%
\pgfpathmoveto{\pgfqpoint{0.000000in}{0.000000in}}%
\pgfpathlineto{\pgfqpoint{0.000000in}{-0.048611in}}%
\pgfusepath{stroke,fill}%
}%
\begin{pgfscope}%
\pgfsys@transformshift{1.023035in}{2.994444in}%
\pgfsys@useobject{currentmarker}{}%
\end{pgfscope}%
\end{pgfscope}%
\begin{pgfscope}%
\pgfsetbuttcap%
\pgfsetroundjoin%
\definecolor{currentfill}{rgb}{0.000000,0.000000,0.000000}%
\pgfsetfillcolor{currentfill}%
\pgfsetlinewidth{0.803000pt}%
\definecolor{currentstroke}{rgb}{0.000000,0.000000,0.000000}%
\pgfsetstrokecolor{currentstroke}%
\pgfsetdash{}{0pt}%
\pgfsys@defobject{currentmarker}{\pgfqpoint{0.000000in}{-0.048611in}}{\pgfqpoint{0.000000in}{0.000000in}}{%
\pgfpathmoveto{\pgfqpoint{0.000000in}{0.000000in}}%
\pgfpathlineto{\pgfqpoint{0.000000in}{-0.048611in}}%
\pgfusepath{stroke,fill}%
}%
\begin{pgfscope}%
\pgfsys@transformshift{1.670988in}{2.994444in}%
\pgfsys@useobject{currentmarker}{}%
\end{pgfscope}%
\end{pgfscope}%
\begin{pgfscope}%
\pgfsetbuttcap%
\pgfsetroundjoin%
\definecolor{currentfill}{rgb}{0.000000,0.000000,0.000000}%
\pgfsetfillcolor{currentfill}%
\pgfsetlinewidth{0.803000pt}%
\definecolor{currentstroke}{rgb}{0.000000,0.000000,0.000000}%
\pgfsetstrokecolor{currentstroke}%
\pgfsetdash{}{0pt}%
\pgfsys@defobject{currentmarker}{\pgfqpoint{0.000000in}{-0.048611in}}{\pgfqpoint{0.000000in}{0.000000in}}{%
\pgfpathmoveto{\pgfqpoint{0.000000in}{0.000000in}}%
\pgfpathlineto{\pgfqpoint{0.000000in}{-0.048611in}}%
\pgfusepath{stroke,fill}%
}%
\begin{pgfscope}%
\pgfsys@transformshift{2.318941in}{2.994444in}%
\pgfsys@useobject{currentmarker}{}%
\end{pgfscope}%
\end{pgfscope}%
\begin{pgfscope}%
\definecolor{textcolor}{rgb}{0.000000,0.000000,0.000000}%
\pgfsetstrokecolor{textcolor}%
\pgfsetfillcolor{textcolor}%
\pgftext[x=1.804651in,y=2.938889in,,top]{\color{textcolor}\sffamily\fontsize{10.000000}{12.000000}\selectfont \(\displaystyle \zeta \, \mathrm{[\mu m]}\)}%
\end{pgfscope}%
\begin{pgfscope}%
\pgfsetbuttcap%
\pgfsetroundjoin%
\definecolor{currentfill}{rgb}{0.000000,0.000000,0.000000}%
\pgfsetfillcolor{currentfill}%
\pgfsetlinewidth{0.803000pt}%
\definecolor{currentstroke}{rgb}{0.000000,0.000000,0.000000}%
\pgfsetstrokecolor{currentstroke}%
\pgfsetdash{}{0pt}%
\pgfsys@defobject{currentmarker}{\pgfqpoint{-0.048611in}{0.000000in}}{\pgfqpoint{-0.000000in}{0.000000in}}{%
\pgfpathmoveto{\pgfqpoint{-0.000000in}{0.000000in}}%
\pgfpathlineto{\pgfqpoint{-0.048611in}{0.000000in}}%
\pgfusepath{stroke,fill}%
}%
\begin{pgfscope}%
\pgfsys@transformshift{0.800000in}{3.101950in}%
\pgfsys@useobject{currentmarker}{}%
\end{pgfscope}%
\end{pgfscope}%
\begin{pgfscope}%
\definecolor{textcolor}{rgb}{0.000000,0.000000,0.000000}%
\pgfsetstrokecolor{textcolor}%
\pgfsetfillcolor{textcolor}%
\pgftext[x=0.455863in, y=3.053725in, left, base]{\color{textcolor}\sffamily\fontsize{10.000000}{12.000000}\selectfont \(\displaystyle {\ensuremath{-}20}\)}%
\end{pgfscope}%
\begin{pgfscope}%
\pgfsetbuttcap%
\pgfsetroundjoin%
\definecolor{currentfill}{rgb}{0.000000,0.000000,0.000000}%
\pgfsetfillcolor{currentfill}%
\pgfsetlinewidth{0.803000pt}%
\definecolor{currentstroke}{rgb}{0.000000,0.000000,0.000000}%
\pgfsetstrokecolor{currentstroke}%
\pgfsetdash{}{0pt}%
\pgfsys@defobject{currentmarker}{\pgfqpoint{-0.048611in}{0.000000in}}{\pgfqpoint{-0.000000in}{0.000000in}}{%
\pgfpathmoveto{\pgfqpoint{-0.000000in}{0.000000in}}%
\pgfpathlineto{\pgfqpoint{-0.048611in}{0.000000in}}%
\pgfusepath{stroke,fill}%
}%
\begin{pgfscope}%
\pgfsys@transformshift{0.800000in}{3.509586in}%
\pgfsys@useobject{currentmarker}{}%
\end{pgfscope}%
\end{pgfscope}%
\begin{pgfscope}%
\definecolor{textcolor}{rgb}{0.000000,0.000000,0.000000}%
\pgfsetstrokecolor{textcolor}%
\pgfsetfillcolor{textcolor}%
\pgftext[x=0.455863in, y=3.461361in, left, base]{\color{textcolor}\sffamily\fontsize{10.000000}{12.000000}\selectfont \(\displaystyle {\ensuremath{-}10}\)}%
\end{pgfscope}%
\begin{pgfscope}%
\pgfsetbuttcap%
\pgfsetroundjoin%
\definecolor{currentfill}{rgb}{0.000000,0.000000,0.000000}%
\pgfsetfillcolor{currentfill}%
\pgfsetlinewidth{0.803000pt}%
\definecolor{currentstroke}{rgb}{0.000000,0.000000,0.000000}%
\pgfsetstrokecolor{currentstroke}%
\pgfsetdash{}{0pt}%
\pgfsys@defobject{currentmarker}{\pgfqpoint{-0.048611in}{0.000000in}}{\pgfqpoint{-0.000000in}{0.000000in}}{%
\pgfpathmoveto{\pgfqpoint{-0.000000in}{0.000000in}}%
\pgfpathlineto{\pgfqpoint{-0.048611in}{0.000000in}}%
\pgfusepath{stroke,fill}%
}%
\begin{pgfscope}%
\pgfsys@transformshift{0.800000in}{3.917222in}%
\pgfsys@useobject{currentmarker}{}%
\end{pgfscope}%
\end{pgfscope}%
\begin{pgfscope}%
\definecolor{textcolor}{rgb}{0.000000,0.000000,0.000000}%
\pgfsetstrokecolor{textcolor}%
\pgfsetfillcolor{textcolor}%
\pgftext[x=0.633333in, y=3.868997in, left, base]{\color{textcolor}\sffamily\fontsize{10.000000}{12.000000}\selectfont \(\displaystyle {0}\)}%
\end{pgfscope}%
\begin{pgfscope}%
\pgfsetbuttcap%
\pgfsetroundjoin%
\definecolor{currentfill}{rgb}{0.000000,0.000000,0.000000}%
\pgfsetfillcolor{currentfill}%
\pgfsetlinewidth{0.803000pt}%
\definecolor{currentstroke}{rgb}{0.000000,0.000000,0.000000}%
\pgfsetstrokecolor{currentstroke}%
\pgfsetdash{}{0pt}%
\pgfsys@defobject{currentmarker}{\pgfqpoint{-0.048611in}{0.000000in}}{\pgfqpoint{-0.000000in}{0.000000in}}{%
\pgfpathmoveto{\pgfqpoint{-0.000000in}{0.000000in}}%
\pgfpathlineto{\pgfqpoint{-0.048611in}{0.000000in}}%
\pgfusepath{stroke,fill}%
}%
\begin{pgfscope}%
\pgfsys@transformshift{0.800000in}{4.324858in}%
\pgfsys@useobject{currentmarker}{}%
\end{pgfscope}%
\end{pgfscope}%
\begin{pgfscope}%
\definecolor{textcolor}{rgb}{0.000000,0.000000,0.000000}%
\pgfsetstrokecolor{textcolor}%
\pgfsetfillcolor{textcolor}%
\pgftext[x=0.563888in, y=4.276633in, left, base]{\color{textcolor}\sffamily\fontsize{10.000000}{12.000000}\selectfont \(\displaystyle {10}\)}%
\end{pgfscope}%
\begin{pgfscope}%
\pgfsetbuttcap%
\pgfsetroundjoin%
\definecolor{currentfill}{rgb}{0.000000,0.000000,0.000000}%
\pgfsetfillcolor{currentfill}%
\pgfsetlinewidth{0.803000pt}%
\definecolor{currentstroke}{rgb}{0.000000,0.000000,0.000000}%
\pgfsetstrokecolor{currentstroke}%
\pgfsetdash{}{0pt}%
\pgfsys@defobject{currentmarker}{\pgfqpoint{-0.048611in}{0.000000in}}{\pgfqpoint{-0.000000in}{0.000000in}}{%
\pgfpathmoveto{\pgfqpoint{-0.000000in}{0.000000in}}%
\pgfpathlineto{\pgfqpoint{-0.048611in}{0.000000in}}%
\pgfusepath{stroke,fill}%
}%
\begin{pgfscope}%
\pgfsys@transformshift{0.800000in}{4.732494in}%
\pgfsys@useobject{currentmarker}{}%
\end{pgfscope}%
\end{pgfscope}%
\begin{pgfscope}%
\definecolor{textcolor}{rgb}{0.000000,0.000000,0.000000}%
\pgfsetstrokecolor{textcolor}%
\pgfsetfillcolor{textcolor}%
\pgftext[x=0.563888in, y=4.684269in, left, base]{\color{textcolor}\sffamily\fontsize{10.000000}{12.000000}\selectfont \(\displaystyle {20}\)}%
\end{pgfscope}%
\begin{pgfscope}%
\definecolor{textcolor}{rgb}{0.000000,0.000000,0.000000}%
\pgfsetstrokecolor{textcolor}%
\pgfsetfillcolor{textcolor}%
\pgftext[x=0.400308in,y=3.917222in,,bottom,rotate=90.000000]{\color{textcolor}\sffamily\fontsize{10.000000}{12.000000}\selectfont \(\displaystyle z \, \mathrm{[\mu m]}\)}%
\end{pgfscope}%
\begin{pgfscope}%
\pgfsetrectcap%
\pgfsetmiterjoin%
\pgfsetlinewidth{0.803000pt}%
\definecolor{currentstroke}{rgb}{0.000000,0.000000,0.000000}%
\pgfsetstrokecolor{currentstroke}%
\pgfsetdash{}{0pt}%
\pgfpathmoveto{\pgfqpoint{0.800000in}{2.994444in}}%
\pgfpathlineto{\pgfqpoint{0.800000in}{4.840000in}}%
\pgfusepath{stroke}%
\end{pgfscope}%
\begin{pgfscope}%
\pgfsetrectcap%
\pgfsetmiterjoin%
\pgfsetlinewidth{0.803000pt}%
\definecolor{currentstroke}{rgb}{0.000000,0.000000,0.000000}%
\pgfsetstrokecolor{currentstroke}%
\pgfsetdash{}{0pt}%
\pgfpathmoveto{\pgfqpoint{2.809302in}{2.994444in}}%
\pgfpathlineto{\pgfqpoint{2.809302in}{4.840000in}}%
\pgfusepath{stroke}%
\end{pgfscope}%
\begin{pgfscope}%
\pgfsetrectcap%
\pgfsetmiterjoin%
\pgfsetlinewidth{0.803000pt}%
\definecolor{currentstroke}{rgb}{0.000000,0.000000,0.000000}%
\pgfsetstrokecolor{currentstroke}%
\pgfsetdash{}{0pt}%
\pgfpathmoveto{\pgfqpoint{0.800000in}{2.994444in}}%
\pgfpathlineto{\pgfqpoint{2.809302in}{2.994444in}}%
\pgfusepath{stroke}%
\end{pgfscope}%
\begin{pgfscope}%
\pgfsetrectcap%
\pgfsetmiterjoin%
\pgfsetlinewidth{0.803000pt}%
\definecolor{currentstroke}{rgb}{0.000000,0.000000,0.000000}%
\pgfsetstrokecolor{currentstroke}%
\pgfsetdash{}{0pt}%
\pgfpathmoveto{\pgfqpoint{0.800000in}{4.840000in}}%
\pgfpathlineto{\pgfqpoint{2.809302in}{4.840000in}}%
\pgfusepath{stroke}%
\end{pgfscope}%
\begin{pgfscope}%
\definecolor{textcolor}{rgb}{0.000000,0.000000,0.000000}%
\pgfsetstrokecolor{textcolor}%
\pgfsetfillcolor{textcolor}%
\pgftext[x=1.804651in,y=4.923333in,,base]{\color{textcolor}\sffamily\fontsize{12.000000}{14.400000}\selectfont a)}%
\end{pgfscope}%
\begin{pgfscope}%
\pgfsetbuttcap%
\pgfsetmiterjoin%
\definecolor{currentfill}{rgb}{1.000000,1.000000,1.000000}%
\pgfsetfillcolor{currentfill}%
\pgfsetlinewidth{0.000000pt}%
\definecolor{currentstroke}{rgb}{0.000000,0.000000,0.000000}%
\pgfsetstrokecolor{currentstroke}%
\pgfsetstrokeopacity{0.000000}%
\pgfsetdash{}{0pt}%
\pgfpathmoveto{\pgfqpoint{3.110698in}{2.994444in}}%
\pgfpathlineto{\pgfqpoint{5.120000in}{2.994444in}}%
\pgfpathlineto{\pgfqpoint{5.120000in}{4.840000in}}%
\pgfpathlineto{\pgfqpoint{3.110698in}{4.840000in}}%
\pgfpathlineto{\pgfqpoint{3.110698in}{2.994444in}}%
\pgfpathclose%
\pgfusepath{fill}%
\end{pgfscope}%
\begin{pgfscope}%
\pgfpathrectangle{\pgfqpoint{3.110698in}{2.994444in}}{\pgfqpoint{2.009302in}{1.845556in}}%
\pgfusepath{clip}%
\pgfsys@transformcm{2.013889}{0.000000}{0.000000}{1.847222}{3.110698in}{2.994444in}%
\pgftext[left,bottom]{\includegraphics[interpolate=false,width=1.000000in,height=1.000000in]{cavity-img2.png}}%
\end{pgfscope}%
\begin{pgfscope}%
\pgfpathrectangle{\pgfqpoint{3.110698in}{2.994444in}}{\pgfqpoint{2.009302in}{1.845556in}}%
\pgfusepath{clip}%
\pgfsys@transformcm{2.013889}{0.000000}{0.000000}{1.847222}{3.110698in}{2.994444in}%
\pgftext[left,bottom]{\includegraphics[interpolate=false,width=1.000000in,height=1.000000in]{cavity-img3.png}}%
\end{pgfscope}%
\begin{pgfscope}%
\pgfsetbuttcap%
\pgfsetroundjoin%
\definecolor{currentfill}{rgb}{0.000000,0.000000,0.000000}%
\pgfsetfillcolor{currentfill}%
\pgfsetlinewidth{0.803000pt}%
\definecolor{currentstroke}{rgb}{0.000000,0.000000,0.000000}%
\pgfsetstrokecolor{currentstroke}%
\pgfsetdash{}{0pt}%
\pgfsys@defobject{currentmarker}{\pgfqpoint{0.000000in}{-0.048611in}}{\pgfqpoint{0.000000in}{0.000000in}}{%
\pgfpathmoveto{\pgfqpoint{0.000000in}{0.000000in}}%
\pgfpathlineto{\pgfqpoint{0.000000in}{-0.048611in}}%
\pgfusepath{stroke,fill}%
}%
\begin{pgfscope}%
\pgfsys@transformshift{3.333733in}{2.994444in}%
\pgfsys@useobject{currentmarker}{}%
\end{pgfscope}%
\end{pgfscope}%
\begin{pgfscope}%
\pgfsetbuttcap%
\pgfsetroundjoin%
\definecolor{currentfill}{rgb}{0.000000,0.000000,0.000000}%
\pgfsetfillcolor{currentfill}%
\pgfsetlinewidth{0.803000pt}%
\definecolor{currentstroke}{rgb}{0.000000,0.000000,0.000000}%
\pgfsetstrokecolor{currentstroke}%
\pgfsetdash{}{0pt}%
\pgfsys@defobject{currentmarker}{\pgfqpoint{0.000000in}{-0.048611in}}{\pgfqpoint{0.000000in}{0.000000in}}{%
\pgfpathmoveto{\pgfqpoint{0.000000in}{0.000000in}}%
\pgfpathlineto{\pgfqpoint{0.000000in}{-0.048611in}}%
\pgfusepath{stroke,fill}%
}%
\begin{pgfscope}%
\pgfsys@transformshift{3.981686in}{2.994444in}%
\pgfsys@useobject{currentmarker}{}%
\end{pgfscope}%
\end{pgfscope}%
\begin{pgfscope}%
\pgfsetbuttcap%
\pgfsetroundjoin%
\definecolor{currentfill}{rgb}{0.000000,0.000000,0.000000}%
\pgfsetfillcolor{currentfill}%
\pgfsetlinewidth{0.803000pt}%
\definecolor{currentstroke}{rgb}{0.000000,0.000000,0.000000}%
\pgfsetstrokecolor{currentstroke}%
\pgfsetdash{}{0pt}%
\pgfsys@defobject{currentmarker}{\pgfqpoint{0.000000in}{-0.048611in}}{\pgfqpoint{0.000000in}{0.000000in}}{%
\pgfpathmoveto{\pgfqpoint{0.000000in}{0.000000in}}%
\pgfpathlineto{\pgfqpoint{0.000000in}{-0.048611in}}%
\pgfusepath{stroke,fill}%
}%
\begin{pgfscope}%
\pgfsys@transformshift{4.629639in}{2.994444in}%
\pgfsys@useobject{currentmarker}{}%
\end{pgfscope}%
\end{pgfscope}%
\begin{pgfscope}%
\definecolor{textcolor}{rgb}{0.000000,0.000000,0.000000}%
\pgfsetstrokecolor{textcolor}%
\pgfsetfillcolor{textcolor}%
\pgftext[x=4.115349in,y=2.938889in,,top]{\color{textcolor}\sffamily\fontsize{10.000000}{12.000000}\selectfont \(\displaystyle \zeta \, \mathrm{[\mu m]}\)}%
\end{pgfscope}%
\begin{pgfscope}%
\pgfsetbuttcap%
\pgfsetroundjoin%
\definecolor{currentfill}{rgb}{0.000000,0.000000,0.000000}%
\pgfsetfillcolor{currentfill}%
\pgfsetlinewidth{0.803000pt}%
\definecolor{currentstroke}{rgb}{0.000000,0.000000,0.000000}%
\pgfsetstrokecolor{currentstroke}%
\pgfsetdash{}{0pt}%
\pgfsys@defobject{currentmarker}{\pgfqpoint{-0.048611in}{0.000000in}}{\pgfqpoint{-0.000000in}{0.000000in}}{%
\pgfpathmoveto{\pgfqpoint{-0.000000in}{0.000000in}}%
\pgfpathlineto{\pgfqpoint{-0.048611in}{0.000000in}}%
\pgfusepath{stroke,fill}%
}%
\begin{pgfscope}%
\pgfsys@transformshift{3.110698in}{3.101950in}%
\pgfsys@useobject{currentmarker}{}%
\end{pgfscope}%
\end{pgfscope}%
\begin{pgfscope}%
\pgfsetbuttcap%
\pgfsetroundjoin%
\definecolor{currentfill}{rgb}{0.000000,0.000000,0.000000}%
\pgfsetfillcolor{currentfill}%
\pgfsetlinewidth{0.803000pt}%
\definecolor{currentstroke}{rgb}{0.000000,0.000000,0.000000}%
\pgfsetstrokecolor{currentstroke}%
\pgfsetdash{}{0pt}%
\pgfsys@defobject{currentmarker}{\pgfqpoint{-0.048611in}{0.000000in}}{\pgfqpoint{-0.000000in}{0.000000in}}{%
\pgfpathmoveto{\pgfqpoint{-0.000000in}{0.000000in}}%
\pgfpathlineto{\pgfqpoint{-0.048611in}{0.000000in}}%
\pgfusepath{stroke,fill}%
}%
\begin{pgfscope}%
\pgfsys@transformshift{3.110698in}{3.509586in}%
\pgfsys@useobject{currentmarker}{}%
\end{pgfscope}%
\end{pgfscope}%
\begin{pgfscope}%
\pgfsetbuttcap%
\pgfsetroundjoin%
\definecolor{currentfill}{rgb}{0.000000,0.000000,0.000000}%
\pgfsetfillcolor{currentfill}%
\pgfsetlinewidth{0.803000pt}%
\definecolor{currentstroke}{rgb}{0.000000,0.000000,0.000000}%
\pgfsetstrokecolor{currentstroke}%
\pgfsetdash{}{0pt}%
\pgfsys@defobject{currentmarker}{\pgfqpoint{-0.048611in}{0.000000in}}{\pgfqpoint{-0.000000in}{0.000000in}}{%
\pgfpathmoveto{\pgfqpoint{-0.000000in}{0.000000in}}%
\pgfpathlineto{\pgfqpoint{-0.048611in}{0.000000in}}%
\pgfusepath{stroke,fill}%
}%
\begin{pgfscope}%
\pgfsys@transformshift{3.110698in}{3.917222in}%
\pgfsys@useobject{currentmarker}{}%
\end{pgfscope}%
\end{pgfscope}%
\begin{pgfscope}%
\pgfsetbuttcap%
\pgfsetroundjoin%
\definecolor{currentfill}{rgb}{0.000000,0.000000,0.000000}%
\pgfsetfillcolor{currentfill}%
\pgfsetlinewidth{0.803000pt}%
\definecolor{currentstroke}{rgb}{0.000000,0.000000,0.000000}%
\pgfsetstrokecolor{currentstroke}%
\pgfsetdash{}{0pt}%
\pgfsys@defobject{currentmarker}{\pgfqpoint{-0.048611in}{0.000000in}}{\pgfqpoint{-0.000000in}{0.000000in}}{%
\pgfpathmoveto{\pgfqpoint{-0.000000in}{0.000000in}}%
\pgfpathlineto{\pgfqpoint{-0.048611in}{0.000000in}}%
\pgfusepath{stroke,fill}%
}%
\begin{pgfscope}%
\pgfsys@transformshift{3.110698in}{4.324858in}%
\pgfsys@useobject{currentmarker}{}%
\end{pgfscope}%
\end{pgfscope}%
\begin{pgfscope}%
\pgfsetbuttcap%
\pgfsetroundjoin%
\definecolor{currentfill}{rgb}{0.000000,0.000000,0.000000}%
\pgfsetfillcolor{currentfill}%
\pgfsetlinewidth{0.803000pt}%
\definecolor{currentstroke}{rgb}{0.000000,0.000000,0.000000}%
\pgfsetstrokecolor{currentstroke}%
\pgfsetdash{}{0pt}%
\pgfsys@defobject{currentmarker}{\pgfqpoint{-0.048611in}{0.000000in}}{\pgfqpoint{-0.000000in}{0.000000in}}{%
\pgfpathmoveto{\pgfqpoint{-0.000000in}{0.000000in}}%
\pgfpathlineto{\pgfqpoint{-0.048611in}{0.000000in}}%
\pgfusepath{stroke,fill}%
}%
\begin{pgfscope}%
\pgfsys@transformshift{3.110698in}{4.732494in}%
\pgfsys@useobject{currentmarker}{}%
\end{pgfscope}%
\end{pgfscope}%
\begin{pgfscope}%
\definecolor{textcolor}{rgb}{0.000000,0.000000,0.000000}%
\pgfsetstrokecolor{textcolor}%
\pgfsetfillcolor{textcolor}%
\pgftext[x=3.055142in,y=3.917222in,,bottom,rotate=90.000000]{\color{textcolor}\sffamily\fontsize{10.000000}{12.000000}\selectfont \(\displaystyle z \, \mathrm{[\mu m]}\)}%
\end{pgfscope}%
\begin{pgfscope}%
\pgfsetrectcap%
\pgfsetmiterjoin%
\pgfsetlinewidth{0.803000pt}%
\definecolor{currentstroke}{rgb}{0.000000,0.000000,0.000000}%
\pgfsetstrokecolor{currentstroke}%
\pgfsetdash{}{0pt}%
\pgfpathmoveto{\pgfqpoint{3.110698in}{2.994444in}}%
\pgfpathlineto{\pgfqpoint{3.110698in}{4.840000in}}%
\pgfusepath{stroke}%
\end{pgfscope}%
\begin{pgfscope}%
\pgfsetrectcap%
\pgfsetmiterjoin%
\pgfsetlinewidth{0.803000pt}%
\definecolor{currentstroke}{rgb}{0.000000,0.000000,0.000000}%
\pgfsetstrokecolor{currentstroke}%
\pgfsetdash{}{0pt}%
\pgfpathmoveto{\pgfqpoint{5.120000in}{2.994444in}}%
\pgfpathlineto{\pgfqpoint{5.120000in}{4.840000in}}%
\pgfusepath{stroke}%
\end{pgfscope}%
\begin{pgfscope}%
\pgfsetrectcap%
\pgfsetmiterjoin%
\pgfsetlinewidth{0.803000pt}%
\definecolor{currentstroke}{rgb}{0.000000,0.000000,0.000000}%
\pgfsetstrokecolor{currentstroke}%
\pgfsetdash{}{0pt}%
\pgfpathmoveto{\pgfqpoint{3.110698in}{2.994444in}}%
\pgfpathlineto{\pgfqpoint{5.120000in}{2.994444in}}%
\pgfusepath{stroke}%
\end{pgfscope}%
\begin{pgfscope}%
\pgfsetrectcap%
\pgfsetmiterjoin%
\pgfsetlinewidth{0.803000pt}%
\definecolor{currentstroke}{rgb}{0.000000,0.000000,0.000000}%
\pgfsetstrokecolor{currentstroke}%
\pgfsetdash{}{0pt}%
\pgfpathmoveto{\pgfqpoint{3.110698in}{4.840000in}}%
\pgfpathlineto{\pgfqpoint{5.120000in}{4.840000in}}%
\pgfusepath{stroke}%
\end{pgfscope}%
\begin{pgfscope}%
\definecolor{textcolor}{rgb}{0.000000,0.000000,0.000000}%
\pgfsetstrokecolor{textcolor}%
\pgfsetfillcolor{textcolor}%
\pgftext[x=4.115349in,y=4.923333in,,base]{\color{textcolor}\sffamily\fontsize{12.000000}{14.400000}\selectfont b)}%
\end{pgfscope}%
\begin{pgfscope}%
\pgfsetbuttcap%
\pgfsetmiterjoin%
\definecolor{currentfill}{rgb}{1.000000,1.000000,1.000000}%
\pgfsetfillcolor{currentfill}%
\pgfsetlinewidth{0.000000pt}%
\definecolor{currentstroke}{rgb}{0.000000,0.000000,0.000000}%
\pgfsetstrokecolor{currentstroke}%
\pgfsetstrokeopacity{0.000000}%
\pgfsetdash{}{0pt}%
\pgfpathmoveto{\pgfqpoint{0.800000in}{0.687500in}}%
\pgfpathlineto{\pgfqpoint{2.809302in}{0.687500in}}%
\pgfpathlineto{\pgfqpoint{2.809302in}{2.533056in}}%
\pgfpathlineto{\pgfqpoint{0.800000in}{2.533056in}}%
\pgfpathlineto{\pgfqpoint{0.800000in}{0.687500in}}%
\pgfpathclose%
\pgfusepath{fill}%
\end{pgfscope}%
\begin{pgfscope}%
\pgfpathrectangle{\pgfqpoint{0.800000in}{0.687500in}}{\pgfqpoint{2.009302in}{1.845556in}}%
\pgfusepath{clip}%
\pgfsys@transformcm{2.013889}{0.000000}{0.000000}{1.847222}{0.800000in}{0.687500in}%
\pgftext[left,bottom]{\includegraphics[interpolate=false,width=1.000000in,height=1.000000in]{cavity-img4.png}}%
\end{pgfscope}%
\begin{pgfscope}%
\pgfpathrectangle{\pgfqpoint{0.800000in}{0.687500in}}{\pgfqpoint{2.009302in}{1.845556in}}%
\pgfusepath{clip}%
\pgfsys@transformcm{2.013889}{0.000000}{0.000000}{1.847222}{0.800000in}{0.687500in}%
\pgftext[left,bottom]{\includegraphics[interpolate=false,width=1.000000in,height=1.000000in]{cavity-img5.png}}%
\end{pgfscope}%
\begin{pgfscope}%
\pgfsetbuttcap%
\pgfsetroundjoin%
\definecolor{currentfill}{rgb}{0.000000,0.000000,0.000000}%
\pgfsetfillcolor{currentfill}%
\pgfsetlinewidth{0.803000pt}%
\definecolor{currentstroke}{rgb}{0.000000,0.000000,0.000000}%
\pgfsetstrokecolor{currentstroke}%
\pgfsetdash{}{0pt}%
\pgfsys@defobject{currentmarker}{\pgfqpoint{0.000000in}{-0.048611in}}{\pgfqpoint{0.000000in}{0.000000in}}{%
\pgfpathmoveto{\pgfqpoint{0.000000in}{0.000000in}}%
\pgfpathlineto{\pgfqpoint{0.000000in}{-0.048611in}}%
\pgfusepath{stroke,fill}%
}%
\begin{pgfscope}%
\pgfsys@transformshift{1.023035in}{0.687500in}%
\pgfsys@useobject{currentmarker}{}%
\end{pgfscope}%
\end{pgfscope}%
\begin{pgfscope}%
\definecolor{textcolor}{rgb}{0.000000,0.000000,0.000000}%
\pgfsetstrokecolor{textcolor}%
\pgfsetfillcolor{textcolor}%
\pgftext[x=1.023035in,y=0.590278in,,top]{\color{textcolor}\sffamily\fontsize{10.000000}{12.000000}\selectfont \(\displaystyle {\ensuremath{-}40}\)}%
\end{pgfscope}%
\begin{pgfscope}%
\pgfsetbuttcap%
\pgfsetroundjoin%
\definecolor{currentfill}{rgb}{0.000000,0.000000,0.000000}%
\pgfsetfillcolor{currentfill}%
\pgfsetlinewidth{0.803000pt}%
\definecolor{currentstroke}{rgb}{0.000000,0.000000,0.000000}%
\pgfsetstrokecolor{currentstroke}%
\pgfsetdash{}{0pt}%
\pgfsys@defobject{currentmarker}{\pgfqpoint{0.000000in}{-0.048611in}}{\pgfqpoint{0.000000in}{0.000000in}}{%
\pgfpathmoveto{\pgfqpoint{0.000000in}{0.000000in}}%
\pgfpathlineto{\pgfqpoint{0.000000in}{-0.048611in}}%
\pgfusepath{stroke,fill}%
}%
\begin{pgfscope}%
\pgfsys@transformshift{1.670988in}{0.687500in}%
\pgfsys@useobject{currentmarker}{}%
\end{pgfscope}%
\end{pgfscope}%
\begin{pgfscope}%
\definecolor{textcolor}{rgb}{0.000000,0.000000,0.000000}%
\pgfsetstrokecolor{textcolor}%
\pgfsetfillcolor{textcolor}%
\pgftext[x=1.670988in,y=0.590278in,,top]{\color{textcolor}\sffamily\fontsize{10.000000}{12.000000}\selectfont \(\displaystyle {\ensuremath{-}20}\)}%
\end{pgfscope}%
\begin{pgfscope}%
\pgfsetbuttcap%
\pgfsetroundjoin%
\definecolor{currentfill}{rgb}{0.000000,0.000000,0.000000}%
\pgfsetfillcolor{currentfill}%
\pgfsetlinewidth{0.803000pt}%
\definecolor{currentstroke}{rgb}{0.000000,0.000000,0.000000}%
\pgfsetstrokecolor{currentstroke}%
\pgfsetdash{}{0pt}%
\pgfsys@defobject{currentmarker}{\pgfqpoint{0.000000in}{-0.048611in}}{\pgfqpoint{0.000000in}{0.000000in}}{%
\pgfpathmoveto{\pgfqpoint{0.000000in}{0.000000in}}%
\pgfpathlineto{\pgfqpoint{0.000000in}{-0.048611in}}%
\pgfusepath{stroke,fill}%
}%
\begin{pgfscope}%
\pgfsys@transformshift{2.318941in}{0.687500in}%
\pgfsys@useobject{currentmarker}{}%
\end{pgfscope}%
\end{pgfscope}%
\begin{pgfscope}%
\definecolor{textcolor}{rgb}{0.000000,0.000000,0.000000}%
\pgfsetstrokecolor{textcolor}%
\pgfsetfillcolor{textcolor}%
\pgftext[x=2.318941in,y=0.590278in,,top]{\color{textcolor}\sffamily\fontsize{10.000000}{12.000000}\selectfont \(\displaystyle {0}\)}%
\end{pgfscope}%
\begin{pgfscope}%
\definecolor{textcolor}{rgb}{0.000000,0.000000,0.000000}%
\pgfsetstrokecolor{textcolor}%
\pgfsetfillcolor{textcolor}%
\pgftext[x=1.804651in,y=0.411266in,,top]{\color{textcolor}\sffamily\fontsize{10.000000}{12.000000}\selectfont \(\displaystyle \zeta \, \mathrm{[\mu m]}\)}%
\end{pgfscope}%
\begin{pgfscope}%
\pgfsetbuttcap%
\pgfsetroundjoin%
\definecolor{currentfill}{rgb}{0.000000,0.000000,0.000000}%
\pgfsetfillcolor{currentfill}%
\pgfsetlinewidth{0.803000pt}%
\definecolor{currentstroke}{rgb}{0.000000,0.000000,0.000000}%
\pgfsetstrokecolor{currentstroke}%
\pgfsetdash{}{0pt}%
\pgfsys@defobject{currentmarker}{\pgfqpoint{-0.048611in}{0.000000in}}{\pgfqpoint{-0.000000in}{0.000000in}}{%
\pgfpathmoveto{\pgfqpoint{-0.000000in}{0.000000in}}%
\pgfpathlineto{\pgfqpoint{-0.048611in}{0.000000in}}%
\pgfusepath{stroke,fill}%
}%
\begin{pgfscope}%
\pgfsys@transformshift{0.800000in}{0.795006in}%
\pgfsys@useobject{currentmarker}{}%
\end{pgfscope}%
\end{pgfscope}%
\begin{pgfscope}%
\definecolor{textcolor}{rgb}{0.000000,0.000000,0.000000}%
\pgfsetstrokecolor{textcolor}%
\pgfsetfillcolor{textcolor}%
\pgftext[x=0.455863in, y=0.746781in, left, base]{\color{textcolor}\sffamily\fontsize{10.000000}{12.000000}\selectfont \(\displaystyle {\ensuremath{-}20}\)}%
\end{pgfscope}%
\begin{pgfscope}%
\pgfsetbuttcap%
\pgfsetroundjoin%
\definecolor{currentfill}{rgb}{0.000000,0.000000,0.000000}%
\pgfsetfillcolor{currentfill}%
\pgfsetlinewidth{0.803000pt}%
\definecolor{currentstroke}{rgb}{0.000000,0.000000,0.000000}%
\pgfsetstrokecolor{currentstroke}%
\pgfsetdash{}{0pt}%
\pgfsys@defobject{currentmarker}{\pgfqpoint{-0.048611in}{0.000000in}}{\pgfqpoint{-0.000000in}{0.000000in}}{%
\pgfpathmoveto{\pgfqpoint{-0.000000in}{0.000000in}}%
\pgfpathlineto{\pgfqpoint{-0.048611in}{0.000000in}}%
\pgfusepath{stroke,fill}%
}%
\begin{pgfscope}%
\pgfsys@transformshift{0.800000in}{1.202642in}%
\pgfsys@useobject{currentmarker}{}%
\end{pgfscope}%
\end{pgfscope}%
\begin{pgfscope}%
\definecolor{textcolor}{rgb}{0.000000,0.000000,0.000000}%
\pgfsetstrokecolor{textcolor}%
\pgfsetfillcolor{textcolor}%
\pgftext[x=0.455863in, y=1.154417in, left, base]{\color{textcolor}\sffamily\fontsize{10.000000}{12.000000}\selectfont \(\displaystyle {\ensuremath{-}10}\)}%
\end{pgfscope}%
\begin{pgfscope}%
\pgfsetbuttcap%
\pgfsetroundjoin%
\definecolor{currentfill}{rgb}{0.000000,0.000000,0.000000}%
\pgfsetfillcolor{currentfill}%
\pgfsetlinewidth{0.803000pt}%
\definecolor{currentstroke}{rgb}{0.000000,0.000000,0.000000}%
\pgfsetstrokecolor{currentstroke}%
\pgfsetdash{}{0pt}%
\pgfsys@defobject{currentmarker}{\pgfqpoint{-0.048611in}{0.000000in}}{\pgfqpoint{-0.000000in}{0.000000in}}{%
\pgfpathmoveto{\pgfqpoint{-0.000000in}{0.000000in}}%
\pgfpathlineto{\pgfqpoint{-0.048611in}{0.000000in}}%
\pgfusepath{stroke,fill}%
}%
\begin{pgfscope}%
\pgfsys@transformshift{0.800000in}{1.610278in}%
\pgfsys@useobject{currentmarker}{}%
\end{pgfscope}%
\end{pgfscope}%
\begin{pgfscope}%
\definecolor{textcolor}{rgb}{0.000000,0.000000,0.000000}%
\pgfsetstrokecolor{textcolor}%
\pgfsetfillcolor{textcolor}%
\pgftext[x=0.633333in, y=1.562053in, left, base]{\color{textcolor}\sffamily\fontsize{10.000000}{12.000000}\selectfont \(\displaystyle {0}\)}%
\end{pgfscope}%
\begin{pgfscope}%
\pgfsetbuttcap%
\pgfsetroundjoin%
\definecolor{currentfill}{rgb}{0.000000,0.000000,0.000000}%
\pgfsetfillcolor{currentfill}%
\pgfsetlinewidth{0.803000pt}%
\definecolor{currentstroke}{rgb}{0.000000,0.000000,0.000000}%
\pgfsetstrokecolor{currentstroke}%
\pgfsetdash{}{0pt}%
\pgfsys@defobject{currentmarker}{\pgfqpoint{-0.048611in}{0.000000in}}{\pgfqpoint{-0.000000in}{0.000000in}}{%
\pgfpathmoveto{\pgfqpoint{-0.000000in}{0.000000in}}%
\pgfpathlineto{\pgfqpoint{-0.048611in}{0.000000in}}%
\pgfusepath{stroke,fill}%
}%
\begin{pgfscope}%
\pgfsys@transformshift{0.800000in}{2.017914in}%
\pgfsys@useobject{currentmarker}{}%
\end{pgfscope}%
\end{pgfscope}%
\begin{pgfscope}%
\definecolor{textcolor}{rgb}{0.000000,0.000000,0.000000}%
\pgfsetstrokecolor{textcolor}%
\pgfsetfillcolor{textcolor}%
\pgftext[x=0.563888in, y=1.969688in, left, base]{\color{textcolor}\sffamily\fontsize{10.000000}{12.000000}\selectfont \(\displaystyle {10}\)}%
\end{pgfscope}%
\begin{pgfscope}%
\pgfsetbuttcap%
\pgfsetroundjoin%
\definecolor{currentfill}{rgb}{0.000000,0.000000,0.000000}%
\pgfsetfillcolor{currentfill}%
\pgfsetlinewidth{0.803000pt}%
\definecolor{currentstroke}{rgb}{0.000000,0.000000,0.000000}%
\pgfsetstrokecolor{currentstroke}%
\pgfsetdash{}{0pt}%
\pgfsys@defobject{currentmarker}{\pgfqpoint{-0.048611in}{0.000000in}}{\pgfqpoint{-0.000000in}{0.000000in}}{%
\pgfpathmoveto{\pgfqpoint{-0.000000in}{0.000000in}}%
\pgfpathlineto{\pgfqpoint{-0.048611in}{0.000000in}}%
\pgfusepath{stroke,fill}%
}%
\begin{pgfscope}%
\pgfsys@transformshift{0.800000in}{2.425550in}%
\pgfsys@useobject{currentmarker}{}%
\end{pgfscope}%
\end{pgfscope}%
\begin{pgfscope}%
\definecolor{textcolor}{rgb}{0.000000,0.000000,0.000000}%
\pgfsetstrokecolor{textcolor}%
\pgfsetfillcolor{textcolor}%
\pgftext[x=0.563888in, y=2.377324in, left, base]{\color{textcolor}\sffamily\fontsize{10.000000}{12.000000}\selectfont \(\displaystyle {20}\)}%
\end{pgfscope}%
\begin{pgfscope}%
\definecolor{textcolor}{rgb}{0.000000,0.000000,0.000000}%
\pgfsetstrokecolor{textcolor}%
\pgfsetfillcolor{textcolor}%
\pgftext[x=0.400308in,y=1.610278in,,bottom,rotate=90.000000]{\color{textcolor}\sffamily\fontsize{10.000000}{12.000000}\selectfont \(\displaystyle z \, \mathrm{[\mu m]}\)}%
\end{pgfscope}%
\begin{pgfscope}%
\pgfsetrectcap%
\pgfsetmiterjoin%
\pgfsetlinewidth{0.803000pt}%
\definecolor{currentstroke}{rgb}{0.000000,0.000000,0.000000}%
\pgfsetstrokecolor{currentstroke}%
\pgfsetdash{}{0pt}%
\pgfpathmoveto{\pgfqpoint{0.800000in}{0.687500in}}%
\pgfpathlineto{\pgfqpoint{0.800000in}{2.533056in}}%
\pgfusepath{stroke}%
\end{pgfscope}%
\begin{pgfscope}%
\pgfsetrectcap%
\pgfsetmiterjoin%
\pgfsetlinewidth{0.803000pt}%
\definecolor{currentstroke}{rgb}{0.000000,0.000000,0.000000}%
\pgfsetstrokecolor{currentstroke}%
\pgfsetdash{}{0pt}%
\pgfpathmoveto{\pgfqpoint{2.809302in}{0.687500in}}%
\pgfpathlineto{\pgfqpoint{2.809302in}{2.533056in}}%
\pgfusepath{stroke}%
\end{pgfscope}%
\begin{pgfscope}%
\pgfsetrectcap%
\pgfsetmiterjoin%
\pgfsetlinewidth{0.803000pt}%
\definecolor{currentstroke}{rgb}{0.000000,0.000000,0.000000}%
\pgfsetstrokecolor{currentstroke}%
\pgfsetdash{}{0pt}%
\pgfpathmoveto{\pgfqpoint{0.800000in}{0.687500in}}%
\pgfpathlineto{\pgfqpoint{2.809302in}{0.687500in}}%
\pgfusepath{stroke}%
\end{pgfscope}%
\begin{pgfscope}%
\pgfsetrectcap%
\pgfsetmiterjoin%
\pgfsetlinewidth{0.803000pt}%
\definecolor{currentstroke}{rgb}{0.000000,0.000000,0.000000}%
\pgfsetstrokecolor{currentstroke}%
\pgfsetdash{}{0pt}%
\pgfpathmoveto{\pgfqpoint{0.800000in}{2.533056in}}%
\pgfpathlineto{\pgfqpoint{2.809302in}{2.533056in}}%
\pgfusepath{stroke}%
\end{pgfscope}%
\begin{pgfscope}%
\definecolor{textcolor}{rgb}{0.000000,0.000000,0.000000}%
\pgfsetstrokecolor{textcolor}%
\pgfsetfillcolor{textcolor}%
\pgftext[x=1.804651in,y=2.616389in,,base]{\color{textcolor}\sffamily\fontsize{12.000000}{14.400000}\selectfont c)}%
\end{pgfscope}%
\begin{pgfscope}%
\pgfsetbuttcap%
\pgfsetmiterjoin%
\definecolor{currentfill}{rgb}{1.000000,1.000000,1.000000}%
\pgfsetfillcolor{currentfill}%
\pgfsetlinewidth{0.000000pt}%
\definecolor{currentstroke}{rgb}{0.000000,0.000000,0.000000}%
\pgfsetstrokecolor{currentstroke}%
\pgfsetstrokeopacity{0.000000}%
\pgfsetdash{}{0pt}%
\pgfpathmoveto{\pgfqpoint{3.110698in}{0.687500in}}%
\pgfpathlineto{\pgfqpoint{5.120000in}{0.687500in}}%
\pgfpathlineto{\pgfqpoint{5.120000in}{2.533056in}}%
\pgfpathlineto{\pgfqpoint{3.110698in}{2.533056in}}%
\pgfpathlineto{\pgfqpoint{3.110698in}{0.687500in}}%
\pgfpathclose%
\pgfusepath{fill}%
\end{pgfscope}%
\begin{pgfscope}%
\pgfpathrectangle{\pgfqpoint{3.110698in}{0.687500in}}{\pgfqpoint{2.009302in}{1.845556in}}%
\pgfusepath{clip}%
\pgfsys@transformcm{2.013889}{0.000000}{0.000000}{1.847222}{3.110698in}{0.687500in}%
\pgftext[left,bottom]{\includegraphics[interpolate=false,width=1.000000in,height=1.000000in]{cavity-img6.png}}%
\end{pgfscope}%
\begin{pgfscope}%
\pgfpathrectangle{\pgfqpoint{3.110698in}{0.687500in}}{\pgfqpoint{2.009302in}{1.845556in}}%
\pgfusepath{clip}%
\pgfsys@transformcm{2.013889}{0.000000}{0.000000}{1.847222}{3.110698in}{0.687500in}%
\pgftext[left,bottom]{\includegraphics[interpolate=false,width=1.000000in,height=1.000000in]{cavity-img7.png}}%
\end{pgfscope}%
\begin{pgfscope}%
\pgfsetbuttcap%
\pgfsetroundjoin%
\definecolor{currentfill}{rgb}{0.000000,0.000000,0.000000}%
\pgfsetfillcolor{currentfill}%
\pgfsetlinewidth{0.803000pt}%
\definecolor{currentstroke}{rgb}{0.000000,0.000000,0.000000}%
\pgfsetstrokecolor{currentstroke}%
\pgfsetdash{}{0pt}%
\pgfsys@defobject{currentmarker}{\pgfqpoint{0.000000in}{-0.048611in}}{\pgfqpoint{0.000000in}{0.000000in}}{%
\pgfpathmoveto{\pgfqpoint{0.000000in}{0.000000in}}%
\pgfpathlineto{\pgfqpoint{0.000000in}{-0.048611in}}%
\pgfusepath{stroke,fill}%
}%
\begin{pgfscope}%
\pgfsys@transformshift{3.333733in}{0.687500in}%
\pgfsys@useobject{currentmarker}{}%
\end{pgfscope}%
\end{pgfscope}%
\begin{pgfscope}%
\definecolor{textcolor}{rgb}{0.000000,0.000000,0.000000}%
\pgfsetstrokecolor{textcolor}%
\pgfsetfillcolor{textcolor}%
\pgftext[x=3.333733in,y=0.590278in,,top]{\color{textcolor}\sffamily\fontsize{10.000000}{12.000000}\selectfont \(\displaystyle {\ensuremath{-}40}\)}%
\end{pgfscope}%
\begin{pgfscope}%
\pgfsetbuttcap%
\pgfsetroundjoin%
\definecolor{currentfill}{rgb}{0.000000,0.000000,0.000000}%
\pgfsetfillcolor{currentfill}%
\pgfsetlinewidth{0.803000pt}%
\definecolor{currentstroke}{rgb}{0.000000,0.000000,0.000000}%
\pgfsetstrokecolor{currentstroke}%
\pgfsetdash{}{0pt}%
\pgfsys@defobject{currentmarker}{\pgfqpoint{0.000000in}{-0.048611in}}{\pgfqpoint{0.000000in}{0.000000in}}{%
\pgfpathmoveto{\pgfqpoint{0.000000in}{0.000000in}}%
\pgfpathlineto{\pgfqpoint{0.000000in}{-0.048611in}}%
\pgfusepath{stroke,fill}%
}%
\begin{pgfscope}%
\pgfsys@transformshift{3.981686in}{0.687500in}%
\pgfsys@useobject{currentmarker}{}%
\end{pgfscope}%
\end{pgfscope}%
\begin{pgfscope}%
\definecolor{textcolor}{rgb}{0.000000,0.000000,0.000000}%
\pgfsetstrokecolor{textcolor}%
\pgfsetfillcolor{textcolor}%
\pgftext[x=3.981686in,y=0.590278in,,top]{\color{textcolor}\sffamily\fontsize{10.000000}{12.000000}\selectfont \(\displaystyle {\ensuremath{-}20}\)}%
\end{pgfscope}%
\begin{pgfscope}%
\pgfsetbuttcap%
\pgfsetroundjoin%
\definecolor{currentfill}{rgb}{0.000000,0.000000,0.000000}%
\pgfsetfillcolor{currentfill}%
\pgfsetlinewidth{0.803000pt}%
\definecolor{currentstroke}{rgb}{0.000000,0.000000,0.000000}%
\pgfsetstrokecolor{currentstroke}%
\pgfsetdash{}{0pt}%
\pgfsys@defobject{currentmarker}{\pgfqpoint{0.000000in}{-0.048611in}}{\pgfqpoint{0.000000in}{0.000000in}}{%
\pgfpathmoveto{\pgfqpoint{0.000000in}{0.000000in}}%
\pgfpathlineto{\pgfqpoint{0.000000in}{-0.048611in}}%
\pgfusepath{stroke,fill}%
}%
\begin{pgfscope}%
\pgfsys@transformshift{4.629639in}{0.687500in}%
\pgfsys@useobject{currentmarker}{}%
\end{pgfscope}%
\end{pgfscope}%
\begin{pgfscope}%
\definecolor{textcolor}{rgb}{0.000000,0.000000,0.000000}%
\pgfsetstrokecolor{textcolor}%
\pgfsetfillcolor{textcolor}%
\pgftext[x=4.629639in,y=0.590278in,,top]{\color{textcolor}\sffamily\fontsize{10.000000}{12.000000}\selectfont \(\displaystyle {0}\)}%
\end{pgfscope}%
\begin{pgfscope}%
\definecolor{textcolor}{rgb}{0.000000,0.000000,0.000000}%
\pgfsetstrokecolor{textcolor}%
\pgfsetfillcolor{textcolor}%
\pgftext[x=4.115349in,y=0.411266in,,top]{\color{textcolor}\sffamily\fontsize{10.000000}{12.000000}\selectfont \(\displaystyle \zeta \, \mathrm{[\mu m]}\)}%
\end{pgfscope}%
\begin{pgfscope}%
\pgfsetbuttcap%
\pgfsetroundjoin%
\definecolor{currentfill}{rgb}{0.000000,0.000000,0.000000}%
\pgfsetfillcolor{currentfill}%
\pgfsetlinewidth{0.803000pt}%
\definecolor{currentstroke}{rgb}{0.000000,0.000000,0.000000}%
\pgfsetstrokecolor{currentstroke}%
\pgfsetdash{}{0pt}%
\pgfsys@defobject{currentmarker}{\pgfqpoint{-0.048611in}{0.000000in}}{\pgfqpoint{-0.000000in}{0.000000in}}{%
\pgfpathmoveto{\pgfqpoint{-0.000000in}{0.000000in}}%
\pgfpathlineto{\pgfqpoint{-0.048611in}{0.000000in}}%
\pgfusepath{stroke,fill}%
}%
\begin{pgfscope}%
\pgfsys@transformshift{3.110698in}{0.795006in}%
\pgfsys@useobject{currentmarker}{}%
\end{pgfscope}%
\end{pgfscope}%
\begin{pgfscope}%
\pgfsetbuttcap%
\pgfsetroundjoin%
\definecolor{currentfill}{rgb}{0.000000,0.000000,0.000000}%
\pgfsetfillcolor{currentfill}%
\pgfsetlinewidth{0.803000pt}%
\definecolor{currentstroke}{rgb}{0.000000,0.000000,0.000000}%
\pgfsetstrokecolor{currentstroke}%
\pgfsetdash{}{0pt}%
\pgfsys@defobject{currentmarker}{\pgfqpoint{-0.048611in}{0.000000in}}{\pgfqpoint{-0.000000in}{0.000000in}}{%
\pgfpathmoveto{\pgfqpoint{-0.000000in}{0.000000in}}%
\pgfpathlineto{\pgfqpoint{-0.048611in}{0.000000in}}%
\pgfusepath{stroke,fill}%
}%
\begin{pgfscope}%
\pgfsys@transformshift{3.110698in}{1.202642in}%
\pgfsys@useobject{currentmarker}{}%
\end{pgfscope}%
\end{pgfscope}%
\begin{pgfscope}%
\pgfsetbuttcap%
\pgfsetroundjoin%
\definecolor{currentfill}{rgb}{0.000000,0.000000,0.000000}%
\pgfsetfillcolor{currentfill}%
\pgfsetlinewidth{0.803000pt}%
\definecolor{currentstroke}{rgb}{0.000000,0.000000,0.000000}%
\pgfsetstrokecolor{currentstroke}%
\pgfsetdash{}{0pt}%
\pgfsys@defobject{currentmarker}{\pgfqpoint{-0.048611in}{0.000000in}}{\pgfqpoint{-0.000000in}{0.000000in}}{%
\pgfpathmoveto{\pgfqpoint{-0.000000in}{0.000000in}}%
\pgfpathlineto{\pgfqpoint{-0.048611in}{0.000000in}}%
\pgfusepath{stroke,fill}%
}%
\begin{pgfscope}%
\pgfsys@transformshift{3.110698in}{1.610278in}%
\pgfsys@useobject{currentmarker}{}%
\end{pgfscope}%
\end{pgfscope}%
\begin{pgfscope}%
\pgfsetbuttcap%
\pgfsetroundjoin%
\definecolor{currentfill}{rgb}{0.000000,0.000000,0.000000}%
\pgfsetfillcolor{currentfill}%
\pgfsetlinewidth{0.803000pt}%
\definecolor{currentstroke}{rgb}{0.000000,0.000000,0.000000}%
\pgfsetstrokecolor{currentstroke}%
\pgfsetdash{}{0pt}%
\pgfsys@defobject{currentmarker}{\pgfqpoint{-0.048611in}{0.000000in}}{\pgfqpoint{-0.000000in}{0.000000in}}{%
\pgfpathmoveto{\pgfqpoint{-0.000000in}{0.000000in}}%
\pgfpathlineto{\pgfqpoint{-0.048611in}{0.000000in}}%
\pgfusepath{stroke,fill}%
}%
\begin{pgfscope}%
\pgfsys@transformshift{3.110698in}{2.017914in}%
\pgfsys@useobject{currentmarker}{}%
\end{pgfscope}%
\end{pgfscope}%
\begin{pgfscope}%
\pgfsetbuttcap%
\pgfsetroundjoin%
\definecolor{currentfill}{rgb}{0.000000,0.000000,0.000000}%
\pgfsetfillcolor{currentfill}%
\pgfsetlinewidth{0.803000pt}%
\definecolor{currentstroke}{rgb}{0.000000,0.000000,0.000000}%
\pgfsetstrokecolor{currentstroke}%
\pgfsetdash{}{0pt}%
\pgfsys@defobject{currentmarker}{\pgfqpoint{-0.048611in}{0.000000in}}{\pgfqpoint{-0.000000in}{0.000000in}}{%
\pgfpathmoveto{\pgfqpoint{-0.000000in}{0.000000in}}%
\pgfpathlineto{\pgfqpoint{-0.048611in}{0.000000in}}%
\pgfusepath{stroke,fill}%
}%
\begin{pgfscope}%
\pgfsys@transformshift{3.110698in}{2.425550in}%
\pgfsys@useobject{currentmarker}{}%
\end{pgfscope}%
\end{pgfscope}%
\begin{pgfscope}%
\definecolor{textcolor}{rgb}{0.000000,0.000000,0.000000}%
\pgfsetstrokecolor{textcolor}%
\pgfsetfillcolor{textcolor}%
\pgftext[x=3.055142in,y=1.610278in,,bottom,rotate=90.000000]{\color{textcolor}\sffamily\fontsize{10.000000}{12.000000}\selectfont \(\displaystyle z \, \mathrm{[\mu m]}\)}%
\end{pgfscope}%
\begin{pgfscope}%
\pgfsetrectcap%
\pgfsetmiterjoin%
\pgfsetlinewidth{0.803000pt}%
\definecolor{currentstroke}{rgb}{0.000000,0.000000,0.000000}%
\pgfsetstrokecolor{currentstroke}%
\pgfsetdash{}{0pt}%
\pgfpathmoveto{\pgfqpoint{3.110698in}{0.687500in}}%
\pgfpathlineto{\pgfqpoint{3.110698in}{2.533056in}}%
\pgfusepath{stroke}%
\end{pgfscope}%
\begin{pgfscope}%
\pgfsetrectcap%
\pgfsetmiterjoin%
\pgfsetlinewidth{0.803000pt}%
\definecolor{currentstroke}{rgb}{0.000000,0.000000,0.000000}%
\pgfsetstrokecolor{currentstroke}%
\pgfsetdash{}{0pt}%
\pgfpathmoveto{\pgfqpoint{5.120000in}{0.687500in}}%
\pgfpathlineto{\pgfqpoint{5.120000in}{2.533056in}}%
\pgfusepath{stroke}%
\end{pgfscope}%
\begin{pgfscope}%
\pgfsetrectcap%
\pgfsetmiterjoin%
\pgfsetlinewidth{0.803000pt}%
\definecolor{currentstroke}{rgb}{0.000000,0.000000,0.000000}%
\pgfsetstrokecolor{currentstroke}%
\pgfsetdash{}{0pt}%
\pgfpathmoveto{\pgfqpoint{3.110698in}{0.687500in}}%
\pgfpathlineto{\pgfqpoint{5.120000in}{0.687500in}}%
\pgfusepath{stroke}%
\end{pgfscope}%
\begin{pgfscope}%
\pgfsetrectcap%
\pgfsetmiterjoin%
\pgfsetlinewidth{0.803000pt}%
\definecolor{currentstroke}{rgb}{0.000000,0.000000,0.000000}%
\pgfsetstrokecolor{currentstroke}%
\pgfsetdash{}{0pt}%
\pgfpathmoveto{\pgfqpoint{3.110698in}{2.533056in}}%
\pgfpathlineto{\pgfqpoint{5.120000in}{2.533056in}}%
\pgfusepath{stroke}%
\end{pgfscope}%
\begin{pgfscope}%
\definecolor{textcolor}{rgb}{0.000000,0.000000,0.000000}%
\pgfsetstrokecolor{textcolor}%
\pgfsetfillcolor{textcolor}%
\pgftext[x=4.115349in,y=2.616389in,,base]{\color{textcolor}\sffamily\fontsize{12.000000}{14.400000}\selectfont d)}%
\end{pgfscope}%
\begin{pgfscope}%
\pgfsetbuttcap%
\pgfsetmiterjoin%
\definecolor{currentfill}{rgb}{1.000000,1.000000,1.000000}%
\pgfsetfillcolor{currentfill}%
\pgfsetlinewidth{0.000000pt}%
\definecolor{currentstroke}{rgb}{0.000000,0.000000,0.000000}%
\pgfsetstrokecolor{currentstroke}%
\pgfsetstrokeopacity{0.000000}%
\pgfsetdash{}{0pt}%
\pgfpathmoveto{\pgfqpoint{5.312000in}{0.825000in}}%
\pgfpathlineto{\pgfqpoint{5.504000in}{0.825000in}}%
\pgfpathlineto{\pgfqpoint{5.504000in}{2.475000in}}%
\pgfpathlineto{\pgfqpoint{5.312000in}{2.475000in}}%
\pgfpathlineto{\pgfqpoint{5.312000in}{0.825000in}}%
\pgfpathclose%
\pgfusepath{fill}%
\end{pgfscope}%
\begin{pgfscope}%
\pgfpathrectangle{\pgfqpoint{5.312000in}{0.825000in}}{\pgfqpoint{0.192000in}{1.650000in}}%
\pgfusepath{clip}%
\pgfsetbuttcap%
\pgfsetmiterjoin%
\definecolor{currentfill}{rgb}{1.000000,1.000000,1.000000}%
\pgfsetfillcolor{currentfill}%
\pgfsetlinewidth{0.010037pt}%
\definecolor{currentstroke}{rgb}{1.000000,1.000000,1.000000}%
\pgfsetstrokecolor{currentstroke}%
\pgfsetdash{}{0pt}%
\pgfusepath{stroke,fill}%
\end{pgfscope}%
\begin{pgfscope}%
\pgfsys@transformshift{5.305556in}{0.819444in}%
\pgftext[left,bottom]{\includegraphics[interpolate=true,width=0.194444in,height=1.652778in]{cavity-img8.png}}%
\end{pgfscope}%
\begin{pgfscope}%
\pgfsetbuttcap%
\pgfsetroundjoin%
\definecolor{currentfill}{rgb}{0.000000,0.000000,0.000000}%
\pgfsetfillcolor{currentfill}%
\pgfsetlinewidth{0.803000pt}%
\definecolor{currentstroke}{rgb}{0.000000,0.000000,0.000000}%
\pgfsetstrokecolor{currentstroke}%
\pgfsetdash{}{0pt}%
\pgfsys@defobject{currentmarker}{\pgfqpoint{0.000000in}{0.000000in}}{\pgfqpoint{0.048611in}{0.000000in}}{%
\pgfpathmoveto{\pgfqpoint{0.000000in}{0.000000in}}%
\pgfpathlineto{\pgfqpoint{0.048611in}{0.000000in}}%
\pgfusepath{stroke,fill}%
}%
\begin{pgfscope}%
\pgfsys@transformshift{5.504000in}{0.825000in}%
\pgfsys@useobject{currentmarker}{}%
\end{pgfscope}%
\end{pgfscope}%
\begin{pgfscope}%
\definecolor{textcolor}{rgb}{0.000000,0.000000,0.000000}%
\pgfsetstrokecolor{textcolor}%
\pgfsetfillcolor{textcolor}%
\pgftext[x=5.601222in, y=0.776775in, left, base]{\color{textcolor}\sffamily\fontsize{10.000000}{12.000000}\selectfont \(\displaystyle {10^{-3}}\)}%
\end{pgfscope}%
\begin{pgfscope}%
\pgfsetbuttcap%
\pgfsetroundjoin%
\definecolor{currentfill}{rgb}{0.000000,0.000000,0.000000}%
\pgfsetfillcolor{currentfill}%
\pgfsetlinewidth{0.803000pt}%
\definecolor{currentstroke}{rgb}{0.000000,0.000000,0.000000}%
\pgfsetstrokecolor{currentstroke}%
\pgfsetdash{}{0pt}%
\pgfsys@defobject{currentmarker}{\pgfqpoint{0.000000in}{0.000000in}}{\pgfqpoint{0.048611in}{0.000000in}}{%
\pgfpathmoveto{\pgfqpoint{0.000000in}{0.000000in}}%
\pgfpathlineto{\pgfqpoint{0.048611in}{0.000000in}}%
\pgfusepath{stroke,fill}%
}%
\begin{pgfscope}%
\pgfsys@transformshift{5.504000in}{1.447521in}%
\pgfsys@useobject{currentmarker}{}%
\end{pgfscope}%
\end{pgfscope}%
\begin{pgfscope}%
\definecolor{textcolor}{rgb}{0.000000,0.000000,0.000000}%
\pgfsetstrokecolor{textcolor}%
\pgfsetfillcolor{textcolor}%
\pgftext[x=5.601222in, y=1.399295in, left, base]{\color{textcolor}\sffamily\fontsize{10.000000}{12.000000}\selectfont \(\displaystyle {10^{-1}}\)}%
\end{pgfscope}%
\begin{pgfscope}%
\pgfsetbuttcap%
\pgfsetroundjoin%
\definecolor{currentfill}{rgb}{0.000000,0.000000,0.000000}%
\pgfsetfillcolor{currentfill}%
\pgfsetlinewidth{0.803000pt}%
\definecolor{currentstroke}{rgb}{0.000000,0.000000,0.000000}%
\pgfsetstrokecolor{currentstroke}%
\pgfsetdash{}{0pt}%
\pgfsys@defobject{currentmarker}{\pgfqpoint{0.000000in}{0.000000in}}{\pgfqpoint{0.048611in}{0.000000in}}{%
\pgfpathmoveto{\pgfqpoint{0.000000in}{0.000000in}}%
\pgfpathlineto{\pgfqpoint{0.048611in}{0.000000in}}%
\pgfusepath{stroke,fill}%
}%
\begin{pgfscope}%
\pgfsys@transformshift{5.504000in}{2.070041in}%
\pgfsys@useobject{currentmarker}{}%
\end{pgfscope}%
\end{pgfscope}%
\begin{pgfscope}%
\definecolor{textcolor}{rgb}{0.000000,0.000000,0.000000}%
\pgfsetstrokecolor{textcolor}%
\pgfsetfillcolor{textcolor}%
\pgftext[x=5.601222in, y=2.021816in, left, base]{\color{textcolor}\sffamily\fontsize{10.000000}{12.000000}\selectfont \(\displaystyle {10^{1}}\)}%
\end{pgfscope}%
\begin{pgfscope}%
\definecolor{textcolor}{rgb}{0.000000,0.000000,0.000000}%
\pgfsetstrokecolor{textcolor}%
\pgfsetfillcolor{textcolor}%
\pgftext[x=5.944780in,y=1.650000in,,top,rotate=90.000000]{\color{textcolor}\sffamily\fontsize{10.000000}{12.000000}\selectfont \(\displaystyle -dQ_{e^-}/dx/dy/dz \, \mathrm{[pC/\mu m^3]}\)}%
\end{pgfscope}%
\begin{pgfscope}%
\pgfsetrectcap%
\pgfsetmiterjoin%
\pgfsetlinewidth{0.803000pt}%
\definecolor{currentstroke}{rgb}{0.000000,0.000000,0.000000}%
\pgfsetstrokecolor{currentstroke}%
\pgfsetdash{}{0pt}%
\pgfpathmoveto{\pgfqpoint{5.312000in}{0.825000in}}%
\pgfpathlineto{\pgfqpoint{5.408000in}{0.825000in}}%
\pgfpathlineto{\pgfqpoint{5.504000in}{0.825000in}}%
\pgfpathlineto{\pgfqpoint{5.504000in}{2.475000in}}%
\pgfpathlineto{\pgfqpoint{5.408000in}{2.475000in}}%
\pgfpathlineto{\pgfqpoint{5.312000in}{2.475000in}}%
\pgfpathlineto{\pgfqpoint{5.312000in}{0.825000in}}%
\pgfpathclose%
\pgfusepath{stroke}%
\end{pgfscope}%
\begin{pgfscope}%
\pgfsetbuttcap%
\pgfsetmiterjoin%
\definecolor{currentfill}{rgb}{1.000000,1.000000,1.000000}%
\pgfsetfillcolor{currentfill}%
\pgfsetlinewidth{0.000000pt}%
\definecolor{currentstroke}{rgb}{0.000000,0.000000,0.000000}%
\pgfsetstrokecolor{currentstroke}%
\pgfsetstrokeopacity{0.000000}%
\pgfsetdash{}{0pt}%
\pgfpathmoveto{\pgfqpoint{5.312000in}{3.080000in}}%
\pgfpathlineto{\pgfqpoint{5.504000in}{3.080000in}}%
\pgfpathlineto{\pgfqpoint{5.504000in}{4.730000in}}%
\pgfpathlineto{\pgfqpoint{5.312000in}{4.730000in}}%
\pgfpathlineto{\pgfqpoint{5.312000in}{3.080000in}}%
\pgfpathclose%
\pgfusepath{fill}%
\end{pgfscope}%
\begin{pgfscope}%
\pgfpathrectangle{\pgfqpoint{5.312000in}{3.080000in}}{\pgfqpoint{0.192000in}{1.650000in}}%
\pgfusepath{clip}%
\pgfsetbuttcap%
\pgfsetmiterjoin%
\definecolor{currentfill}{rgb}{1.000000,1.000000,1.000000}%
\pgfsetfillcolor{currentfill}%
\pgfsetlinewidth{0.010037pt}%
\definecolor{currentstroke}{rgb}{1.000000,1.000000,1.000000}%
\pgfsetstrokecolor{currentstroke}%
\pgfsetdash{}{0pt}%
\pgfusepath{stroke,fill}%
\end{pgfscope}%
\begin{pgfscope}%
\pgfsys@transformshift{5.305556in}{3.083333in}%
\pgftext[left,bottom]{\includegraphics[interpolate=true,width=0.194444in,height=1.652778in]{cavity-img9.png}}%
\end{pgfscope}%
\begin{pgfscope}%
\pgfsetbuttcap%
\pgfsetroundjoin%
\definecolor{currentfill}{rgb}{0.000000,0.000000,0.000000}%
\pgfsetfillcolor{currentfill}%
\pgfsetlinewidth{0.803000pt}%
\definecolor{currentstroke}{rgb}{0.000000,0.000000,0.000000}%
\pgfsetstrokecolor{currentstroke}%
\pgfsetdash{}{0pt}%
\pgfsys@defobject{currentmarker}{\pgfqpoint{0.000000in}{0.000000in}}{\pgfqpoint{0.048611in}{0.000000in}}{%
\pgfpathmoveto{\pgfqpoint{0.000000in}{0.000000in}}%
\pgfpathlineto{\pgfqpoint{0.048611in}{0.000000in}}%
\pgfusepath{stroke,fill}%
}%
\begin{pgfscope}%
\pgfsys@transformshift{5.504000in}{3.080000in}%
\pgfsys@useobject{currentmarker}{}%
\end{pgfscope}%
\end{pgfscope}%
\begin{pgfscope}%
\definecolor{textcolor}{rgb}{0.000000,0.000000,0.000000}%
\pgfsetstrokecolor{textcolor}%
\pgfsetfillcolor{textcolor}%
\pgftext[x=5.601222in, y=3.031775in, left, base]{\color{textcolor}\sffamily\fontsize{10.000000}{12.000000}\selectfont \(\displaystyle {10^{-3}}\)}%
\end{pgfscope}%
\begin{pgfscope}%
\pgfsetbuttcap%
\pgfsetroundjoin%
\definecolor{currentfill}{rgb}{0.000000,0.000000,0.000000}%
\pgfsetfillcolor{currentfill}%
\pgfsetlinewidth{0.803000pt}%
\definecolor{currentstroke}{rgb}{0.000000,0.000000,0.000000}%
\pgfsetstrokecolor{currentstroke}%
\pgfsetdash{}{0pt}%
\pgfsys@defobject{currentmarker}{\pgfqpoint{0.000000in}{0.000000in}}{\pgfqpoint{0.048611in}{0.000000in}}{%
\pgfpathmoveto{\pgfqpoint{0.000000in}{0.000000in}}%
\pgfpathlineto{\pgfqpoint{0.048611in}{0.000000in}}%
\pgfusepath{stroke,fill}%
}%
\begin{pgfscope}%
\pgfsys@transformshift{5.504000in}{3.492500in}%
\pgfsys@useobject{currentmarker}{}%
\end{pgfscope}%
\end{pgfscope}%
\begin{pgfscope}%
\definecolor{textcolor}{rgb}{0.000000,0.000000,0.000000}%
\pgfsetstrokecolor{textcolor}%
\pgfsetfillcolor{textcolor}%
\pgftext[x=5.601222in, y=3.444275in, left, base]{\color{textcolor}\sffamily\fontsize{10.000000}{12.000000}\selectfont \(\displaystyle {10^{-2}}\)}%
\end{pgfscope}%
\begin{pgfscope}%
\pgfsetbuttcap%
\pgfsetroundjoin%
\definecolor{currentfill}{rgb}{0.000000,0.000000,0.000000}%
\pgfsetfillcolor{currentfill}%
\pgfsetlinewidth{0.803000pt}%
\definecolor{currentstroke}{rgb}{0.000000,0.000000,0.000000}%
\pgfsetstrokecolor{currentstroke}%
\pgfsetdash{}{0pt}%
\pgfsys@defobject{currentmarker}{\pgfqpoint{0.000000in}{0.000000in}}{\pgfqpoint{0.048611in}{0.000000in}}{%
\pgfpathmoveto{\pgfqpoint{0.000000in}{0.000000in}}%
\pgfpathlineto{\pgfqpoint{0.048611in}{0.000000in}}%
\pgfusepath{stroke,fill}%
}%
\begin{pgfscope}%
\pgfsys@transformshift{5.504000in}{3.905000in}%
\pgfsys@useobject{currentmarker}{}%
\end{pgfscope}%
\end{pgfscope}%
\begin{pgfscope}%
\definecolor{textcolor}{rgb}{0.000000,0.000000,0.000000}%
\pgfsetstrokecolor{textcolor}%
\pgfsetfillcolor{textcolor}%
\pgftext[x=5.601222in, y=3.856775in, left, base]{\color{textcolor}\sffamily\fontsize{10.000000}{12.000000}\selectfont \(\displaystyle {10^{-1}}\)}%
\end{pgfscope}%
\begin{pgfscope}%
\pgfsetbuttcap%
\pgfsetroundjoin%
\definecolor{currentfill}{rgb}{0.000000,0.000000,0.000000}%
\pgfsetfillcolor{currentfill}%
\pgfsetlinewidth{0.803000pt}%
\definecolor{currentstroke}{rgb}{0.000000,0.000000,0.000000}%
\pgfsetstrokecolor{currentstroke}%
\pgfsetdash{}{0pt}%
\pgfsys@defobject{currentmarker}{\pgfqpoint{0.000000in}{0.000000in}}{\pgfqpoint{0.048611in}{0.000000in}}{%
\pgfpathmoveto{\pgfqpoint{0.000000in}{0.000000in}}%
\pgfpathlineto{\pgfqpoint{0.048611in}{0.000000in}}%
\pgfusepath{stroke,fill}%
}%
\begin{pgfscope}%
\pgfsys@transformshift{5.504000in}{4.317500in}%
\pgfsys@useobject{currentmarker}{}%
\end{pgfscope}%
\end{pgfscope}%
\begin{pgfscope}%
\definecolor{textcolor}{rgb}{0.000000,0.000000,0.000000}%
\pgfsetstrokecolor{textcolor}%
\pgfsetfillcolor{textcolor}%
\pgftext[x=5.601222in, y=4.269275in, left, base]{\color{textcolor}\sffamily\fontsize{10.000000}{12.000000}\selectfont \(\displaystyle {10^{0}}\)}%
\end{pgfscope}%
\begin{pgfscope}%
\pgfsetbuttcap%
\pgfsetroundjoin%
\definecolor{currentfill}{rgb}{0.000000,0.000000,0.000000}%
\pgfsetfillcolor{currentfill}%
\pgfsetlinewidth{0.803000pt}%
\definecolor{currentstroke}{rgb}{0.000000,0.000000,0.000000}%
\pgfsetstrokecolor{currentstroke}%
\pgfsetdash{}{0pt}%
\pgfsys@defobject{currentmarker}{\pgfqpoint{0.000000in}{0.000000in}}{\pgfqpoint{0.048611in}{0.000000in}}{%
\pgfpathmoveto{\pgfqpoint{0.000000in}{0.000000in}}%
\pgfpathlineto{\pgfqpoint{0.048611in}{0.000000in}}%
\pgfusepath{stroke,fill}%
}%
\begin{pgfscope}%
\pgfsys@transformshift{5.504000in}{4.730000in}%
\pgfsys@useobject{currentmarker}{}%
\end{pgfscope}%
\end{pgfscope}%
\begin{pgfscope}%
\definecolor{textcolor}{rgb}{0.000000,0.000000,0.000000}%
\pgfsetstrokecolor{textcolor}%
\pgfsetfillcolor{textcolor}%
\pgftext[x=5.601222in, y=4.681775in, left, base]{\color{textcolor}\sffamily\fontsize{10.000000}{12.000000}\selectfont \(\displaystyle {10^{1}}\)}%
\end{pgfscope}%
\begin{pgfscope}%
\pgfsetbuttcap%
\pgfsetroundjoin%
\definecolor{currentfill}{rgb}{0.000000,0.000000,0.000000}%
\pgfsetfillcolor{currentfill}%
\pgfsetlinewidth{0.602250pt}%
\definecolor{currentstroke}{rgb}{0.000000,0.000000,0.000000}%
\pgfsetstrokecolor{currentstroke}%
\pgfsetdash{}{0pt}%
\pgfsys@defobject{currentmarker}{\pgfqpoint{0.000000in}{0.000000in}}{\pgfqpoint{0.027778in}{0.000000in}}{%
\pgfpathmoveto{\pgfqpoint{0.000000in}{0.000000in}}%
\pgfpathlineto{\pgfqpoint{0.027778in}{0.000000in}}%
\pgfusepath{stroke,fill}%
}%
\begin{pgfscope}%
\pgfsys@transformshift{5.504000in}{3.204175in}%
\pgfsys@useobject{currentmarker}{}%
\end{pgfscope}%
\end{pgfscope}%
\begin{pgfscope}%
\pgfsetbuttcap%
\pgfsetroundjoin%
\definecolor{currentfill}{rgb}{0.000000,0.000000,0.000000}%
\pgfsetfillcolor{currentfill}%
\pgfsetlinewidth{0.602250pt}%
\definecolor{currentstroke}{rgb}{0.000000,0.000000,0.000000}%
\pgfsetstrokecolor{currentstroke}%
\pgfsetdash{}{0pt}%
\pgfsys@defobject{currentmarker}{\pgfqpoint{0.000000in}{0.000000in}}{\pgfqpoint{0.027778in}{0.000000in}}{%
\pgfpathmoveto{\pgfqpoint{0.000000in}{0.000000in}}%
\pgfpathlineto{\pgfqpoint{0.027778in}{0.000000in}}%
\pgfusepath{stroke,fill}%
}%
\begin{pgfscope}%
\pgfsys@transformshift{5.504000in}{3.276813in}%
\pgfsys@useobject{currentmarker}{}%
\end{pgfscope}%
\end{pgfscope}%
\begin{pgfscope}%
\pgfsetbuttcap%
\pgfsetroundjoin%
\definecolor{currentfill}{rgb}{0.000000,0.000000,0.000000}%
\pgfsetfillcolor{currentfill}%
\pgfsetlinewidth{0.602250pt}%
\definecolor{currentstroke}{rgb}{0.000000,0.000000,0.000000}%
\pgfsetstrokecolor{currentstroke}%
\pgfsetdash{}{0pt}%
\pgfsys@defobject{currentmarker}{\pgfqpoint{0.000000in}{0.000000in}}{\pgfqpoint{0.027778in}{0.000000in}}{%
\pgfpathmoveto{\pgfqpoint{0.000000in}{0.000000in}}%
\pgfpathlineto{\pgfqpoint{0.027778in}{0.000000in}}%
\pgfusepath{stroke,fill}%
}%
\begin{pgfscope}%
\pgfsys@transformshift{5.504000in}{3.328350in}%
\pgfsys@useobject{currentmarker}{}%
\end{pgfscope}%
\end{pgfscope}%
\begin{pgfscope}%
\pgfsetbuttcap%
\pgfsetroundjoin%
\definecolor{currentfill}{rgb}{0.000000,0.000000,0.000000}%
\pgfsetfillcolor{currentfill}%
\pgfsetlinewidth{0.602250pt}%
\definecolor{currentstroke}{rgb}{0.000000,0.000000,0.000000}%
\pgfsetstrokecolor{currentstroke}%
\pgfsetdash{}{0pt}%
\pgfsys@defobject{currentmarker}{\pgfqpoint{0.000000in}{0.000000in}}{\pgfqpoint{0.027778in}{0.000000in}}{%
\pgfpathmoveto{\pgfqpoint{0.000000in}{0.000000in}}%
\pgfpathlineto{\pgfqpoint{0.027778in}{0.000000in}}%
\pgfusepath{stroke,fill}%
}%
\begin{pgfscope}%
\pgfsys@transformshift{5.504000in}{3.368325in}%
\pgfsys@useobject{currentmarker}{}%
\end{pgfscope}%
\end{pgfscope}%
\begin{pgfscope}%
\pgfsetbuttcap%
\pgfsetroundjoin%
\definecolor{currentfill}{rgb}{0.000000,0.000000,0.000000}%
\pgfsetfillcolor{currentfill}%
\pgfsetlinewidth{0.602250pt}%
\definecolor{currentstroke}{rgb}{0.000000,0.000000,0.000000}%
\pgfsetstrokecolor{currentstroke}%
\pgfsetdash{}{0pt}%
\pgfsys@defobject{currentmarker}{\pgfqpoint{0.000000in}{0.000000in}}{\pgfqpoint{0.027778in}{0.000000in}}{%
\pgfpathmoveto{\pgfqpoint{0.000000in}{0.000000in}}%
\pgfpathlineto{\pgfqpoint{0.027778in}{0.000000in}}%
\pgfusepath{stroke,fill}%
}%
\begin{pgfscope}%
\pgfsys@transformshift{5.504000in}{3.400987in}%
\pgfsys@useobject{currentmarker}{}%
\end{pgfscope}%
\end{pgfscope}%
\begin{pgfscope}%
\pgfsetbuttcap%
\pgfsetroundjoin%
\definecolor{currentfill}{rgb}{0.000000,0.000000,0.000000}%
\pgfsetfillcolor{currentfill}%
\pgfsetlinewidth{0.602250pt}%
\definecolor{currentstroke}{rgb}{0.000000,0.000000,0.000000}%
\pgfsetstrokecolor{currentstroke}%
\pgfsetdash{}{0pt}%
\pgfsys@defobject{currentmarker}{\pgfqpoint{0.000000in}{0.000000in}}{\pgfqpoint{0.027778in}{0.000000in}}{%
\pgfpathmoveto{\pgfqpoint{0.000000in}{0.000000in}}%
\pgfpathlineto{\pgfqpoint{0.027778in}{0.000000in}}%
\pgfusepath{stroke,fill}%
}%
\begin{pgfscope}%
\pgfsys@transformshift{5.504000in}{3.428603in}%
\pgfsys@useobject{currentmarker}{}%
\end{pgfscope}%
\end{pgfscope}%
\begin{pgfscope}%
\pgfsetbuttcap%
\pgfsetroundjoin%
\definecolor{currentfill}{rgb}{0.000000,0.000000,0.000000}%
\pgfsetfillcolor{currentfill}%
\pgfsetlinewidth{0.602250pt}%
\definecolor{currentstroke}{rgb}{0.000000,0.000000,0.000000}%
\pgfsetstrokecolor{currentstroke}%
\pgfsetdash{}{0pt}%
\pgfsys@defobject{currentmarker}{\pgfqpoint{0.000000in}{0.000000in}}{\pgfqpoint{0.027778in}{0.000000in}}{%
\pgfpathmoveto{\pgfqpoint{0.000000in}{0.000000in}}%
\pgfpathlineto{\pgfqpoint{0.027778in}{0.000000in}}%
\pgfusepath{stroke,fill}%
}%
\begin{pgfscope}%
\pgfsys@transformshift{5.504000in}{3.452525in}%
\pgfsys@useobject{currentmarker}{}%
\end{pgfscope}%
\end{pgfscope}%
\begin{pgfscope}%
\pgfsetbuttcap%
\pgfsetroundjoin%
\definecolor{currentfill}{rgb}{0.000000,0.000000,0.000000}%
\pgfsetfillcolor{currentfill}%
\pgfsetlinewidth{0.602250pt}%
\definecolor{currentstroke}{rgb}{0.000000,0.000000,0.000000}%
\pgfsetstrokecolor{currentstroke}%
\pgfsetdash{}{0pt}%
\pgfsys@defobject{currentmarker}{\pgfqpoint{0.000000in}{0.000000in}}{\pgfqpoint{0.027778in}{0.000000in}}{%
\pgfpathmoveto{\pgfqpoint{0.000000in}{0.000000in}}%
\pgfpathlineto{\pgfqpoint{0.027778in}{0.000000in}}%
\pgfusepath{stroke,fill}%
}%
\begin{pgfscope}%
\pgfsys@transformshift{5.504000in}{3.473625in}%
\pgfsys@useobject{currentmarker}{}%
\end{pgfscope}%
\end{pgfscope}%
\begin{pgfscope}%
\pgfsetbuttcap%
\pgfsetroundjoin%
\definecolor{currentfill}{rgb}{0.000000,0.000000,0.000000}%
\pgfsetfillcolor{currentfill}%
\pgfsetlinewidth{0.602250pt}%
\definecolor{currentstroke}{rgb}{0.000000,0.000000,0.000000}%
\pgfsetstrokecolor{currentstroke}%
\pgfsetdash{}{0pt}%
\pgfsys@defobject{currentmarker}{\pgfqpoint{0.000000in}{0.000000in}}{\pgfqpoint{0.027778in}{0.000000in}}{%
\pgfpathmoveto{\pgfqpoint{0.000000in}{0.000000in}}%
\pgfpathlineto{\pgfqpoint{0.027778in}{0.000000in}}%
\pgfusepath{stroke,fill}%
}%
\begin{pgfscope}%
\pgfsys@transformshift{5.504000in}{3.616675in}%
\pgfsys@useobject{currentmarker}{}%
\end{pgfscope}%
\end{pgfscope}%
\begin{pgfscope}%
\pgfsetbuttcap%
\pgfsetroundjoin%
\definecolor{currentfill}{rgb}{0.000000,0.000000,0.000000}%
\pgfsetfillcolor{currentfill}%
\pgfsetlinewidth{0.602250pt}%
\definecolor{currentstroke}{rgb}{0.000000,0.000000,0.000000}%
\pgfsetstrokecolor{currentstroke}%
\pgfsetdash{}{0pt}%
\pgfsys@defobject{currentmarker}{\pgfqpoint{0.000000in}{0.000000in}}{\pgfqpoint{0.027778in}{0.000000in}}{%
\pgfpathmoveto{\pgfqpoint{0.000000in}{0.000000in}}%
\pgfpathlineto{\pgfqpoint{0.027778in}{0.000000in}}%
\pgfusepath{stroke,fill}%
}%
\begin{pgfscope}%
\pgfsys@transformshift{5.504000in}{3.689313in}%
\pgfsys@useobject{currentmarker}{}%
\end{pgfscope}%
\end{pgfscope}%
\begin{pgfscope}%
\pgfsetbuttcap%
\pgfsetroundjoin%
\definecolor{currentfill}{rgb}{0.000000,0.000000,0.000000}%
\pgfsetfillcolor{currentfill}%
\pgfsetlinewidth{0.602250pt}%
\definecolor{currentstroke}{rgb}{0.000000,0.000000,0.000000}%
\pgfsetstrokecolor{currentstroke}%
\pgfsetdash{}{0pt}%
\pgfsys@defobject{currentmarker}{\pgfqpoint{0.000000in}{0.000000in}}{\pgfqpoint{0.027778in}{0.000000in}}{%
\pgfpathmoveto{\pgfqpoint{0.000000in}{0.000000in}}%
\pgfpathlineto{\pgfqpoint{0.027778in}{0.000000in}}%
\pgfusepath{stroke,fill}%
}%
\begin{pgfscope}%
\pgfsys@transformshift{5.504000in}{3.740850in}%
\pgfsys@useobject{currentmarker}{}%
\end{pgfscope}%
\end{pgfscope}%
\begin{pgfscope}%
\pgfsetbuttcap%
\pgfsetroundjoin%
\definecolor{currentfill}{rgb}{0.000000,0.000000,0.000000}%
\pgfsetfillcolor{currentfill}%
\pgfsetlinewidth{0.602250pt}%
\definecolor{currentstroke}{rgb}{0.000000,0.000000,0.000000}%
\pgfsetstrokecolor{currentstroke}%
\pgfsetdash{}{0pt}%
\pgfsys@defobject{currentmarker}{\pgfqpoint{0.000000in}{0.000000in}}{\pgfqpoint{0.027778in}{0.000000in}}{%
\pgfpathmoveto{\pgfqpoint{0.000000in}{0.000000in}}%
\pgfpathlineto{\pgfqpoint{0.027778in}{0.000000in}}%
\pgfusepath{stroke,fill}%
}%
\begin{pgfscope}%
\pgfsys@transformshift{5.504000in}{3.780825in}%
\pgfsys@useobject{currentmarker}{}%
\end{pgfscope}%
\end{pgfscope}%
\begin{pgfscope}%
\pgfsetbuttcap%
\pgfsetroundjoin%
\definecolor{currentfill}{rgb}{0.000000,0.000000,0.000000}%
\pgfsetfillcolor{currentfill}%
\pgfsetlinewidth{0.602250pt}%
\definecolor{currentstroke}{rgb}{0.000000,0.000000,0.000000}%
\pgfsetstrokecolor{currentstroke}%
\pgfsetdash{}{0pt}%
\pgfsys@defobject{currentmarker}{\pgfqpoint{0.000000in}{0.000000in}}{\pgfqpoint{0.027778in}{0.000000in}}{%
\pgfpathmoveto{\pgfqpoint{0.000000in}{0.000000in}}%
\pgfpathlineto{\pgfqpoint{0.027778in}{0.000000in}}%
\pgfusepath{stroke,fill}%
}%
\begin{pgfscope}%
\pgfsys@transformshift{5.504000in}{3.813487in}%
\pgfsys@useobject{currentmarker}{}%
\end{pgfscope}%
\end{pgfscope}%
\begin{pgfscope}%
\pgfsetbuttcap%
\pgfsetroundjoin%
\definecolor{currentfill}{rgb}{0.000000,0.000000,0.000000}%
\pgfsetfillcolor{currentfill}%
\pgfsetlinewidth{0.602250pt}%
\definecolor{currentstroke}{rgb}{0.000000,0.000000,0.000000}%
\pgfsetstrokecolor{currentstroke}%
\pgfsetdash{}{0pt}%
\pgfsys@defobject{currentmarker}{\pgfqpoint{0.000000in}{0.000000in}}{\pgfqpoint{0.027778in}{0.000000in}}{%
\pgfpathmoveto{\pgfqpoint{0.000000in}{0.000000in}}%
\pgfpathlineto{\pgfqpoint{0.027778in}{0.000000in}}%
\pgfusepath{stroke,fill}%
}%
\begin{pgfscope}%
\pgfsys@transformshift{5.504000in}{3.841103in}%
\pgfsys@useobject{currentmarker}{}%
\end{pgfscope}%
\end{pgfscope}%
\begin{pgfscope}%
\pgfsetbuttcap%
\pgfsetroundjoin%
\definecolor{currentfill}{rgb}{0.000000,0.000000,0.000000}%
\pgfsetfillcolor{currentfill}%
\pgfsetlinewidth{0.602250pt}%
\definecolor{currentstroke}{rgb}{0.000000,0.000000,0.000000}%
\pgfsetstrokecolor{currentstroke}%
\pgfsetdash{}{0pt}%
\pgfsys@defobject{currentmarker}{\pgfqpoint{0.000000in}{0.000000in}}{\pgfqpoint{0.027778in}{0.000000in}}{%
\pgfpathmoveto{\pgfqpoint{0.000000in}{0.000000in}}%
\pgfpathlineto{\pgfqpoint{0.027778in}{0.000000in}}%
\pgfusepath{stroke,fill}%
}%
\begin{pgfscope}%
\pgfsys@transformshift{5.504000in}{3.865025in}%
\pgfsys@useobject{currentmarker}{}%
\end{pgfscope}%
\end{pgfscope}%
\begin{pgfscope}%
\pgfsetbuttcap%
\pgfsetroundjoin%
\definecolor{currentfill}{rgb}{0.000000,0.000000,0.000000}%
\pgfsetfillcolor{currentfill}%
\pgfsetlinewidth{0.602250pt}%
\definecolor{currentstroke}{rgb}{0.000000,0.000000,0.000000}%
\pgfsetstrokecolor{currentstroke}%
\pgfsetdash{}{0pt}%
\pgfsys@defobject{currentmarker}{\pgfqpoint{0.000000in}{0.000000in}}{\pgfqpoint{0.027778in}{0.000000in}}{%
\pgfpathmoveto{\pgfqpoint{0.000000in}{0.000000in}}%
\pgfpathlineto{\pgfqpoint{0.027778in}{0.000000in}}%
\pgfusepath{stroke,fill}%
}%
\begin{pgfscope}%
\pgfsys@transformshift{5.504000in}{3.886125in}%
\pgfsys@useobject{currentmarker}{}%
\end{pgfscope}%
\end{pgfscope}%
\begin{pgfscope}%
\pgfsetbuttcap%
\pgfsetroundjoin%
\definecolor{currentfill}{rgb}{0.000000,0.000000,0.000000}%
\pgfsetfillcolor{currentfill}%
\pgfsetlinewidth{0.602250pt}%
\definecolor{currentstroke}{rgb}{0.000000,0.000000,0.000000}%
\pgfsetstrokecolor{currentstroke}%
\pgfsetdash{}{0pt}%
\pgfsys@defobject{currentmarker}{\pgfqpoint{0.000000in}{0.000000in}}{\pgfqpoint{0.027778in}{0.000000in}}{%
\pgfpathmoveto{\pgfqpoint{0.000000in}{0.000000in}}%
\pgfpathlineto{\pgfqpoint{0.027778in}{0.000000in}}%
\pgfusepath{stroke,fill}%
}%
\begin{pgfscope}%
\pgfsys@transformshift{5.504000in}{4.029175in}%
\pgfsys@useobject{currentmarker}{}%
\end{pgfscope}%
\end{pgfscope}%
\begin{pgfscope}%
\pgfsetbuttcap%
\pgfsetroundjoin%
\definecolor{currentfill}{rgb}{0.000000,0.000000,0.000000}%
\pgfsetfillcolor{currentfill}%
\pgfsetlinewidth{0.602250pt}%
\definecolor{currentstroke}{rgb}{0.000000,0.000000,0.000000}%
\pgfsetstrokecolor{currentstroke}%
\pgfsetdash{}{0pt}%
\pgfsys@defobject{currentmarker}{\pgfqpoint{0.000000in}{0.000000in}}{\pgfqpoint{0.027778in}{0.000000in}}{%
\pgfpathmoveto{\pgfqpoint{0.000000in}{0.000000in}}%
\pgfpathlineto{\pgfqpoint{0.027778in}{0.000000in}}%
\pgfusepath{stroke,fill}%
}%
\begin{pgfscope}%
\pgfsys@transformshift{5.504000in}{4.101813in}%
\pgfsys@useobject{currentmarker}{}%
\end{pgfscope}%
\end{pgfscope}%
\begin{pgfscope}%
\pgfsetbuttcap%
\pgfsetroundjoin%
\definecolor{currentfill}{rgb}{0.000000,0.000000,0.000000}%
\pgfsetfillcolor{currentfill}%
\pgfsetlinewidth{0.602250pt}%
\definecolor{currentstroke}{rgb}{0.000000,0.000000,0.000000}%
\pgfsetstrokecolor{currentstroke}%
\pgfsetdash{}{0pt}%
\pgfsys@defobject{currentmarker}{\pgfqpoint{0.000000in}{0.000000in}}{\pgfqpoint{0.027778in}{0.000000in}}{%
\pgfpathmoveto{\pgfqpoint{0.000000in}{0.000000in}}%
\pgfpathlineto{\pgfqpoint{0.027778in}{0.000000in}}%
\pgfusepath{stroke,fill}%
}%
\begin{pgfscope}%
\pgfsys@transformshift{5.504000in}{4.153350in}%
\pgfsys@useobject{currentmarker}{}%
\end{pgfscope}%
\end{pgfscope}%
\begin{pgfscope}%
\pgfsetbuttcap%
\pgfsetroundjoin%
\definecolor{currentfill}{rgb}{0.000000,0.000000,0.000000}%
\pgfsetfillcolor{currentfill}%
\pgfsetlinewidth{0.602250pt}%
\definecolor{currentstroke}{rgb}{0.000000,0.000000,0.000000}%
\pgfsetstrokecolor{currentstroke}%
\pgfsetdash{}{0pt}%
\pgfsys@defobject{currentmarker}{\pgfqpoint{0.000000in}{0.000000in}}{\pgfqpoint{0.027778in}{0.000000in}}{%
\pgfpathmoveto{\pgfqpoint{0.000000in}{0.000000in}}%
\pgfpathlineto{\pgfqpoint{0.027778in}{0.000000in}}%
\pgfusepath{stroke,fill}%
}%
\begin{pgfscope}%
\pgfsys@transformshift{5.504000in}{4.193325in}%
\pgfsys@useobject{currentmarker}{}%
\end{pgfscope}%
\end{pgfscope}%
\begin{pgfscope}%
\pgfsetbuttcap%
\pgfsetroundjoin%
\definecolor{currentfill}{rgb}{0.000000,0.000000,0.000000}%
\pgfsetfillcolor{currentfill}%
\pgfsetlinewidth{0.602250pt}%
\definecolor{currentstroke}{rgb}{0.000000,0.000000,0.000000}%
\pgfsetstrokecolor{currentstroke}%
\pgfsetdash{}{0pt}%
\pgfsys@defobject{currentmarker}{\pgfqpoint{0.000000in}{0.000000in}}{\pgfqpoint{0.027778in}{0.000000in}}{%
\pgfpathmoveto{\pgfqpoint{0.000000in}{0.000000in}}%
\pgfpathlineto{\pgfqpoint{0.027778in}{0.000000in}}%
\pgfusepath{stroke,fill}%
}%
\begin{pgfscope}%
\pgfsys@transformshift{5.504000in}{4.225987in}%
\pgfsys@useobject{currentmarker}{}%
\end{pgfscope}%
\end{pgfscope}%
\begin{pgfscope}%
\pgfsetbuttcap%
\pgfsetroundjoin%
\definecolor{currentfill}{rgb}{0.000000,0.000000,0.000000}%
\pgfsetfillcolor{currentfill}%
\pgfsetlinewidth{0.602250pt}%
\definecolor{currentstroke}{rgb}{0.000000,0.000000,0.000000}%
\pgfsetstrokecolor{currentstroke}%
\pgfsetdash{}{0pt}%
\pgfsys@defobject{currentmarker}{\pgfqpoint{0.000000in}{0.000000in}}{\pgfqpoint{0.027778in}{0.000000in}}{%
\pgfpathmoveto{\pgfqpoint{0.000000in}{0.000000in}}%
\pgfpathlineto{\pgfqpoint{0.027778in}{0.000000in}}%
\pgfusepath{stroke,fill}%
}%
\begin{pgfscope}%
\pgfsys@transformshift{5.504000in}{4.253603in}%
\pgfsys@useobject{currentmarker}{}%
\end{pgfscope}%
\end{pgfscope}%
\begin{pgfscope}%
\pgfsetbuttcap%
\pgfsetroundjoin%
\definecolor{currentfill}{rgb}{0.000000,0.000000,0.000000}%
\pgfsetfillcolor{currentfill}%
\pgfsetlinewidth{0.602250pt}%
\definecolor{currentstroke}{rgb}{0.000000,0.000000,0.000000}%
\pgfsetstrokecolor{currentstroke}%
\pgfsetdash{}{0pt}%
\pgfsys@defobject{currentmarker}{\pgfqpoint{0.000000in}{0.000000in}}{\pgfqpoint{0.027778in}{0.000000in}}{%
\pgfpathmoveto{\pgfqpoint{0.000000in}{0.000000in}}%
\pgfpathlineto{\pgfqpoint{0.027778in}{0.000000in}}%
\pgfusepath{stroke,fill}%
}%
\begin{pgfscope}%
\pgfsys@transformshift{5.504000in}{4.277525in}%
\pgfsys@useobject{currentmarker}{}%
\end{pgfscope}%
\end{pgfscope}%
\begin{pgfscope}%
\pgfsetbuttcap%
\pgfsetroundjoin%
\definecolor{currentfill}{rgb}{0.000000,0.000000,0.000000}%
\pgfsetfillcolor{currentfill}%
\pgfsetlinewidth{0.602250pt}%
\definecolor{currentstroke}{rgb}{0.000000,0.000000,0.000000}%
\pgfsetstrokecolor{currentstroke}%
\pgfsetdash{}{0pt}%
\pgfsys@defobject{currentmarker}{\pgfqpoint{0.000000in}{0.000000in}}{\pgfqpoint{0.027778in}{0.000000in}}{%
\pgfpathmoveto{\pgfqpoint{0.000000in}{0.000000in}}%
\pgfpathlineto{\pgfqpoint{0.027778in}{0.000000in}}%
\pgfusepath{stroke,fill}%
}%
\begin{pgfscope}%
\pgfsys@transformshift{5.504000in}{4.298625in}%
\pgfsys@useobject{currentmarker}{}%
\end{pgfscope}%
\end{pgfscope}%
\begin{pgfscope}%
\pgfsetbuttcap%
\pgfsetroundjoin%
\definecolor{currentfill}{rgb}{0.000000,0.000000,0.000000}%
\pgfsetfillcolor{currentfill}%
\pgfsetlinewidth{0.602250pt}%
\definecolor{currentstroke}{rgb}{0.000000,0.000000,0.000000}%
\pgfsetstrokecolor{currentstroke}%
\pgfsetdash{}{0pt}%
\pgfsys@defobject{currentmarker}{\pgfqpoint{0.000000in}{0.000000in}}{\pgfqpoint{0.027778in}{0.000000in}}{%
\pgfpathmoveto{\pgfqpoint{0.000000in}{0.000000in}}%
\pgfpathlineto{\pgfqpoint{0.027778in}{0.000000in}}%
\pgfusepath{stroke,fill}%
}%
\begin{pgfscope}%
\pgfsys@transformshift{5.504000in}{4.441675in}%
\pgfsys@useobject{currentmarker}{}%
\end{pgfscope}%
\end{pgfscope}%
\begin{pgfscope}%
\pgfsetbuttcap%
\pgfsetroundjoin%
\definecolor{currentfill}{rgb}{0.000000,0.000000,0.000000}%
\pgfsetfillcolor{currentfill}%
\pgfsetlinewidth{0.602250pt}%
\definecolor{currentstroke}{rgb}{0.000000,0.000000,0.000000}%
\pgfsetstrokecolor{currentstroke}%
\pgfsetdash{}{0pt}%
\pgfsys@defobject{currentmarker}{\pgfqpoint{0.000000in}{0.000000in}}{\pgfqpoint{0.027778in}{0.000000in}}{%
\pgfpathmoveto{\pgfqpoint{0.000000in}{0.000000in}}%
\pgfpathlineto{\pgfqpoint{0.027778in}{0.000000in}}%
\pgfusepath{stroke,fill}%
}%
\begin{pgfscope}%
\pgfsys@transformshift{5.504000in}{4.514313in}%
\pgfsys@useobject{currentmarker}{}%
\end{pgfscope}%
\end{pgfscope}%
\begin{pgfscope}%
\pgfsetbuttcap%
\pgfsetroundjoin%
\definecolor{currentfill}{rgb}{0.000000,0.000000,0.000000}%
\pgfsetfillcolor{currentfill}%
\pgfsetlinewidth{0.602250pt}%
\definecolor{currentstroke}{rgb}{0.000000,0.000000,0.000000}%
\pgfsetstrokecolor{currentstroke}%
\pgfsetdash{}{0pt}%
\pgfsys@defobject{currentmarker}{\pgfqpoint{0.000000in}{0.000000in}}{\pgfqpoint{0.027778in}{0.000000in}}{%
\pgfpathmoveto{\pgfqpoint{0.000000in}{0.000000in}}%
\pgfpathlineto{\pgfqpoint{0.027778in}{0.000000in}}%
\pgfusepath{stroke,fill}%
}%
\begin{pgfscope}%
\pgfsys@transformshift{5.504000in}{4.565850in}%
\pgfsys@useobject{currentmarker}{}%
\end{pgfscope}%
\end{pgfscope}%
\begin{pgfscope}%
\pgfsetbuttcap%
\pgfsetroundjoin%
\definecolor{currentfill}{rgb}{0.000000,0.000000,0.000000}%
\pgfsetfillcolor{currentfill}%
\pgfsetlinewidth{0.602250pt}%
\definecolor{currentstroke}{rgb}{0.000000,0.000000,0.000000}%
\pgfsetstrokecolor{currentstroke}%
\pgfsetdash{}{0pt}%
\pgfsys@defobject{currentmarker}{\pgfqpoint{0.000000in}{0.000000in}}{\pgfqpoint{0.027778in}{0.000000in}}{%
\pgfpathmoveto{\pgfqpoint{0.000000in}{0.000000in}}%
\pgfpathlineto{\pgfqpoint{0.027778in}{0.000000in}}%
\pgfusepath{stroke,fill}%
}%
\begin{pgfscope}%
\pgfsys@transformshift{5.504000in}{4.605825in}%
\pgfsys@useobject{currentmarker}{}%
\end{pgfscope}%
\end{pgfscope}%
\begin{pgfscope}%
\pgfsetbuttcap%
\pgfsetroundjoin%
\definecolor{currentfill}{rgb}{0.000000,0.000000,0.000000}%
\pgfsetfillcolor{currentfill}%
\pgfsetlinewidth{0.602250pt}%
\definecolor{currentstroke}{rgb}{0.000000,0.000000,0.000000}%
\pgfsetstrokecolor{currentstroke}%
\pgfsetdash{}{0pt}%
\pgfsys@defobject{currentmarker}{\pgfqpoint{0.000000in}{0.000000in}}{\pgfqpoint{0.027778in}{0.000000in}}{%
\pgfpathmoveto{\pgfqpoint{0.000000in}{0.000000in}}%
\pgfpathlineto{\pgfqpoint{0.027778in}{0.000000in}}%
\pgfusepath{stroke,fill}%
}%
\begin{pgfscope}%
\pgfsys@transformshift{5.504000in}{4.638487in}%
\pgfsys@useobject{currentmarker}{}%
\end{pgfscope}%
\end{pgfscope}%
\begin{pgfscope}%
\pgfsetbuttcap%
\pgfsetroundjoin%
\definecolor{currentfill}{rgb}{0.000000,0.000000,0.000000}%
\pgfsetfillcolor{currentfill}%
\pgfsetlinewidth{0.602250pt}%
\definecolor{currentstroke}{rgb}{0.000000,0.000000,0.000000}%
\pgfsetstrokecolor{currentstroke}%
\pgfsetdash{}{0pt}%
\pgfsys@defobject{currentmarker}{\pgfqpoint{0.000000in}{0.000000in}}{\pgfqpoint{0.027778in}{0.000000in}}{%
\pgfpathmoveto{\pgfqpoint{0.000000in}{0.000000in}}%
\pgfpathlineto{\pgfqpoint{0.027778in}{0.000000in}}%
\pgfusepath{stroke,fill}%
}%
\begin{pgfscope}%
\pgfsys@transformshift{5.504000in}{4.666103in}%
\pgfsys@useobject{currentmarker}{}%
\end{pgfscope}%
\end{pgfscope}%
\begin{pgfscope}%
\pgfsetbuttcap%
\pgfsetroundjoin%
\definecolor{currentfill}{rgb}{0.000000,0.000000,0.000000}%
\pgfsetfillcolor{currentfill}%
\pgfsetlinewidth{0.602250pt}%
\definecolor{currentstroke}{rgb}{0.000000,0.000000,0.000000}%
\pgfsetstrokecolor{currentstroke}%
\pgfsetdash{}{0pt}%
\pgfsys@defobject{currentmarker}{\pgfqpoint{0.000000in}{0.000000in}}{\pgfqpoint{0.027778in}{0.000000in}}{%
\pgfpathmoveto{\pgfqpoint{0.000000in}{0.000000in}}%
\pgfpathlineto{\pgfqpoint{0.027778in}{0.000000in}}%
\pgfusepath{stroke,fill}%
}%
\begin{pgfscope}%
\pgfsys@transformshift{5.504000in}{4.690025in}%
\pgfsys@useobject{currentmarker}{}%
\end{pgfscope}%
\end{pgfscope}%
\begin{pgfscope}%
\pgfsetbuttcap%
\pgfsetroundjoin%
\definecolor{currentfill}{rgb}{0.000000,0.000000,0.000000}%
\pgfsetfillcolor{currentfill}%
\pgfsetlinewidth{0.602250pt}%
\definecolor{currentstroke}{rgb}{0.000000,0.000000,0.000000}%
\pgfsetstrokecolor{currentstroke}%
\pgfsetdash{}{0pt}%
\pgfsys@defobject{currentmarker}{\pgfqpoint{0.000000in}{0.000000in}}{\pgfqpoint{0.027778in}{0.000000in}}{%
\pgfpathmoveto{\pgfqpoint{0.000000in}{0.000000in}}%
\pgfpathlineto{\pgfqpoint{0.027778in}{0.000000in}}%
\pgfusepath{stroke,fill}%
}%
\begin{pgfscope}%
\pgfsys@transformshift{5.504000in}{4.711125in}%
\pgfsys@useobject{currentmarker}{}%
\end{pgfscope}%
\end{pgfscope}%
\begin{pgfscope}%
\definecolor{textcolor}{rgb}{0.000000,0.000000,0.000000}%
\pgfsetstrokecolor{textcolor}%
\pgfsetfillcolor{textcolor}%
\pgftext[x=5.944780in,y=3.905000in,,top,rotate=90.000000]{\color{textcolor}\sffamily\fontsize{10.000000}{12.000000}\selectfont \(\displaystyle -dQ_{driver}/dx/dy/dz \, \mathrm{[pC/\mu m^3]}\)}%
\end{pgfscope}%
\begin{pgfscope}%
\pgfsetrectcap%
\pgfsetmiterjoin%
\pgfsetlinewidth{0.803000pt}%
\definecolor{currentstroke}{rgb}{0.000000,0.000000,0.000000}%
\pgfsetstrokecolor{currentstroke}%
\pgfsetdash{}{0pt}%
\pgfpathmoveto{\pgfqpoint{5.312000in}{3.080000in}}%
\pgfpathlineto{\pgfqpoint{5.408000in}{3.080000in}}%
\pgfpathlineto{\pgfqpoint{5.504000in}{3.080000in}}%
\pgfpathlineto{\pgfqpoint{5.504000in}{4.730000in}}%
\pgfpathlineto{\pgfqpoint{5.408000in}{4.730000in}}%
\pgfpathlineto{\pgfqpoint{5.312000in}{4.730000in}}%
\pgfpathlineto{\pgfqpoint{5.312000in}{3.080000in}}%
\pgfpathclose%
\pgfusepath{stroke}%
\end{pgfscope}%
\end{pgfpicture}%
\makeatother%
\endgroup%

	\caption{Time series of the charge density of the plasma and the driver at $x=0$ over co-moving $\zeta$ and $z$.
	\textbf{a)} ($y=$ \qty{0.04}{mm}) After entering plasma. The vacuum can still be seen on the left border. Weak cavities can already be seen.
	\textbf{b)} ($y=$ \qty{0.36}{mm}) Cavities forming together with the tail of the driver.
	\textbf{c)} ($y=$ \qty{0.68}{mm}) Blowout regime during spread of the wings on the backside of the driver.
	\textbf{d)} ($y=$ \qty{2.27}{mm}) Wakefield returns to linear regime before bunch breakup.}
	\label{fig:cavity}
\end{figure}
Comparing the length of the beam over time (see vertical lines in \autoref{fig:q_series}) shows that the tail is not an elongation of the driver backwards but the backside experiencing focusing forces, which are narrowing the backside.
Cavities in blowout regime form with the same width as the tail of the driver and widen when the tail spreads apart.

The focus forces at the back of the driver cause the electrons to overshoot. This results in a widening of the tail and the formation of wing-like structures. The cavities also start to widen and form the strongest electric fields of the whole \gls{pwfa} stage during this process.
The formed wings are diverging while new wings form behind them by particles which oscillate as they got pulled back by the focusing force, causing them to overshoot again. This forms a chain of smaller wings, all spreading with time and broadening the tail further.
In this stage, the strength of the electric fields in the cavities already decreases while the electron density in the wakefield rises.

The backside of the driver spreads further as particles from the wings leave the cavities and therefore the impact of the focusing fields. This causes the backside to grow in transverse direction together with the front of the bunch. 
Meanwhile the $\vec{E}$-fields of the wakefield reduce in strength and the cavities are flooded by electrons again. The blowout regime goes over into the linear regime, as seen in \autoref{fig:cavity}d.
Still there are strong forces acting on the center of the bunch, causing it to lose energy. The discussion of this energy loss is continued in \autoref{chap:E_shift}, for now we only look at the distribution of these particles.

In \autoref{fig:force_time} the strength and position change of the longitudinal Lorentz force can be seen over the traversed distance $y$. It shows how the force pushing the driver back first builds up and then loses its strength after the maximum at \qty{1}{\mm}.
\begin{figure}
	\centering
	%% Creator: Matplotlib, PGF backend
%%
%% To include the figure in your LaTeX document, write
%%   \input{<filename>.pgf}
%%
%% Make sure the required packages are loaded in your preamble
%%   \usepackage{pgf}
%%
%% Also ensure that all the required font packages are loaded; for instance,
%% the lmodern package is sometimes necessary when using math font.
%%   \usepackage{lmodern}
%%
%% Figures using additional raster images can only be included by \input if
%% they are in the same directory as the main LaTeX file. For loading figures
%% from other directories you can use the `import` package
%%   \usepackage{import}
%%
%% and then include the figures with
%%   \import{<path to file>}{<filename>.pgf}
%%
%% Matplotlib used the following preamble
%%
\begingroup%
\makeatletter%
\begin{pgfpicture}%
\pgfpathrectangle{\pgfpointorigin}{\pgfqpoint{6.000000in}{4.000000in}}%
\pgfusepath{use as bounding box, clip}%
\begin{pgfscope}%
\pgfsetbuttcap%
\pgfsetmiterjoin%
\pgfsetlinewidth{0.000000pt}%
\definecolor{currentstroke}{rgb}{1.000000,1.000000,1.000000}%
\pgfsetstrokecolor{currentstroke}%
\pgfsetstrokeopacity{0.000000}%
\pgfsetdash{}{0pt}%
\pgfpathmoveto{\pgfqpoint{0.000000in}{0.000000in}}%
\pgfpathlineto{\pgfqpoint{6.000000in}{0.000000in}}%
\pgfpathlineto{\pgfqpoint{6.000000in}{4.000000in}}%
\pgfpathlineto{\pgfqpoint{0.000000in}{4.000000in}}%
\pgfpathlineto{\pgfqpoint{0.000000in}{0.000000in}}%
\pgfpathclose%
\pgfusepath{}%
\end{pgfscope}%
\begin{pgfscope}%
\pgfsetbuttcap%
\pgfsetmiterjoin%
\definecolor{currentfill}{rgb}{1.000000,1.000000,1.000000}%
\pgfsetfillcolor{currentfill}%
\pgfsetlinewidth{0.000000pt}%
\definecolor{currentstroke}{rgb}{0.000000,0.000000,0.000000}%
\pgfsetstrokecolor{currentstroke}%
\pgfsetstrokeopacity{0.000000}%
\pgfsetdash{}{0pt}%
\pgfpathmoveto{\pgfqpoint{0.750000in}{0.500000in}}%
\pgfpathlineto{\pgfqpoint{4.470000in}{0.500000in}}%
\pgfpathlineto{\pgfqpoint{4.470000in}{3.520000in}}%
\pgfpathlineto{\pgfqpoint{0.750000in}{3.520000in}}%
\pgfpathlineto{\pgfqpoint{0.750000in}{0.500000in}}%
\pgfpathclose%
\pgfusepath{fill}%
\end{pgfscope}%
\begin{pgfscope}%
\pgfpathrectangle{\pgfqpoint{0.750000in}{0.500000in}}{\pgfqpoint{3.720000in}{3.020000in}}%
\pgfusepath{clip}%
\pgfsys@transformcm{3.722222}{0.000000}{0.000000}{3.027778}{0.750000in}{0.500000in}%
\pgftext[left,bottom]{\includegraphics[interpolate=false,width=1.000000in,height=1.000000in]{force_time-img0.png}}%
\end{pgfscope}%
\begin{pgfscope}%
\pgfsetbuttcap%
\pgfsetroundjoin%
\definecolor{currentfill}{rgb}{0.000000,0.000000,0.000000}%
\pgfsetfillcolor{currentfill}%
\pgfsetlinewidth{0.803000pt}%
\definecolor{currentstroke}{rgb}{0.000000,0.000000,0.000000}%
\pgfsetstrokecolor{currentstroke}%
\pgfsetdash{}{0pt}%
\pgfsys@defobject{currentmarker}{\pgfqpoint{0.000000in}{-0.048611in}}{\pgfqpoint{0.000000in}{0.000000in}}{%
\pgfpathmoveto{\pgfqpoint{0.000000in}{0.000000in}}%
\pgfpathlineto{\pgfqpoint{0.000000in}{-0.048611in}}%
\pgfusepath{stroke,fill}%
}%
\begin{pgfscope}%
\pgfsys@transformshift{1.224754in}{0.500000in}%
\pgfsys@useobject{currentmarker}{}%
\end{pgfscope}%
\end{pgfscope}%
\begin{pgfscope}%
\definecolor{textcolor}{rgb}{0.000000,0.000000,0.000000}%
\pgfsetstrokecolor{textcolor}%
\pgfsetfillcolor{textcolor}%
\pgftext[x=1.224754in,y=0.402778in,,top]{\color{textcolor}\sffamily\fontsize{10.000000}{12.000000}\selectfont \(\displaystyle {\ensuremath{-}15}\)}%
\end{pgfscope}%
\begin{pgfscope}%
\pgfsetbuttcap%
\pgfsetroundjoin%
\definecolor{currentfill}{rgb}{0.000000,0.000000,0.000000}%
\pgfsetfillcolor{currentfill}%
\pgfsetlinewidth{0.803000pt}%
\definecolor{currentstroke}{rgb}{0.000000,0.000000,0.000000}%
\pgfsetstrokecolor{currentstroke}%
\pgfsetdash{}{0pt}%
\pgfsys@defobject{currentmarker}{\pgfqpoint{0.000000in}{-0.048611in}}{\pgfqpoint{0.000000in}{0.000000in}}{%
\pgfpathmoveto{\pgfqpoint{0.000000in}{0.000000in}}%
\pgfpathlineto{\pgfqpoint{0.000000in}{-0.048611in}}%
\pgfusepath{stroke,fill}%
}%
\begin{pgfscope}%
\pgfsys@transformshift{1.810503in}{0.500000in}%
\pgfsys@useobject{currentmarker}{}%
\end{pgfscope}%
\end{pgfscope}%
\begin{pgfscope}%
\definecolor{textcolor}{rgb}{0.000000,0.000000,0.000000}%
\pgfsetstrokecolor{textcolor}%
\pgfsetfillcolor{textcolor}%
\pgftext[x=1.810503in,y=0.402778in,,top]{\color{textcolor}\sffamily\fontsize{10.000000}{12.000000}\selectfont \(\displaystyle {\ensuremath{-}10}\)}%
\end{pgfscope}%
\begin{pgfscope}%
\pgfsetbuttcap%
\pgfsetroundjoin%
\definecolor{currentfill}{rgb}{0.000000,0.000000,0.000000}%
\pgfsetfillcolor{currentfill}%
\pgfsetlinewidth{0.803000pt}%
\definecolor{currentstroke}{rgb}{0.000000,0.000000,0.000000}%
\pgfsetstrokecolor{currentstroke}%
\pgfsetdash{}{0pt}%
\pgfsys@defobject{currentmarker}{\pgfqpoint{0.000000in}{-0.048611in}}{\pgfqpoint{0.000000in}{0.000000in}}{%
\pgfpathmoveto{\pgfqpoint{0.000000in}{0.000000in}}%
\pgfpathlineto{\pgfqpoint{0.000000in}{-0.048611in}}%
\pgfusepath{stroke,fill}%
}%
\begin{pgfscope}%
\pgfsys@transformshift{2.396251in}{0.500000in}%
\pgfsys@useobject{currentmarker}{}%
\end{pgfscope}%
\end{pgfscope}%
\begin{pgfscope}%
\definecolor{textcolor}{rgb}{0.000000,0.000000,0.000000}%
\pgfsetstrokecolor{textcolor}%
\pgfsetfillcolor{textcolor}%
\pgftext[x=2.396251in,y=0.402778in,,top]{\color{textcolor}\sffamily\fontsize{10.000000}{12.000000}\selectfont \(\displaystyle {\ensuremath{-}5}\)}%
\end{pgfscope}%
\begin{pgfscope}%
\pgfsetbuttcap%
\pgfsetroundjoin%
\definecolor{currentfill}{rgb}{0.000000,0.000000,0.000000}%
\pgfsetfillcolor{currentfill}%
\pgfsetlinewidth{0.803000pt}%
\definecolor{currentstroke}{rgb}{0.000000,0.000000,0.000000}%
\pgfsetstrokecolor{currentstroke}%
\pgfsetdash{}{0pt}%
\pgfsys@defobject{currentmarker}{\pgfqpoint{0.000000in}{-0.048611in}}{\pgfqpoint{0.000000in}{0.000000in}}{%
\pgfpathmoveto{\pgfqpoint{0.000000in}{0.000000in}}%
\pgfpathlineto{\pgfqpoint{0.000000in}{-0.048611in}}%
\pgfusepath{stroke,fill}%
}%
\begin{pgfscope}%
\pgfsys@transformshift{2.982000in}{0.500000in}%
\pgfsys@useobject{currentmarker}{}%
\end{pgfscope}%
\end{pgfscope}%
\begin{pgfscope}%
\definecolor{textcolor}{rgb}{0.000000,0.000000,0.000000}%
\pgfsetstrokecolor{textcolor}%
\pgfsetfillcolor{textcolor}%
\pgftext[x=2.982000in,y=0.402778in,,top]{\color{textcolor}\sffamily\fontsize{10.000000}{12.000000}\selectfont \(\displaystyle {0}\)}%
\end{pgfscope}%
\begin{pgfscope}%
\pgfsetbuttcap%
\pgfsetroundjoin%
\definecolor{currentfill}{rgb}{0.000000,0.000000,0.000000}%
\pgfsetfillcolor{currentfill}%
\pgfsetlinewidth{0.803000pt}%
\definecolor{currentstroke}{rgb}{0.000000,0.000000,0.000000}%
\pgfsetstrokecolor{currentstroke}%
\pgfsetdash{}{0pt}%
\pgfsys@defobject{currentmarker}{\pgfqpoint{0.000000in}{-0.048611in}}{\pgfqpoint{0.000000in}{0.000000in}}{%
\pgfpathmoveto{\pgfqpoint{0.000000in}{0.000000in}}%
\pgfpathlineto{\pgfqpoint{0.000000in}{-0.048611in}}%
\pgfusepath{stroke,fill}%
}%
\begin{pgfscope}%
\pgfsys@transformshift{3.567749in}{0.500000in}%
\pgfsys@useobject{currentmarker}{}%
\end{pgfscope}%
\end{pgfscope}%
\begin{pgfscope}%
\definecolor{textcolor}{rgb}{0.000000,0.000000,0.000000}%
\pgfsetstrokecolor{textcolor}%
\pgfsetfillcolor{textcolor}%
\pgftext[x=3.567749in,y=0.402778in,,top]{\color{textcolor}\sffamily\fontsize{10.000000}{12.000000}\selectfont \(\displaystyle {5}\)}%
\end{pgfscope}%
\begin{pgfscope}%
\pgfsetbuttcap%
\pgfsetroundjoin%
\definecolor{currentfill}{rgb}{0.000000,0.000000,0.000000}%
\pgfsetfillcolor{currentfill}%
\pgfsetlinewidth{0.803000pt}%
\definecolor{currentstroke}{rgb}{0.000000,0.000000,0.000000}%
\pgfsetstrokecolor{currentstroke}%
\pgfsetdash{}{0pt}%
\pgfsys@defobject{currentmarker}{\pgfqpoint{0.000000in}{-0.048611in}}{\pgfqpoint{0.000000in}{0.000000in}}{%
\pgfpathmoveto{\pgfqpoint{0.000000in}{0.000000in}}%
\pgfpathlineto{\pgfqpoint{0.000000in}{-0.048611in}}%
\pgfusepath{stroke,fill}%
}%
\begin{pgfscope}%
\pgfsys@transformshift{4.153497in}{0.500000in}%
\pgfsys@useobject{currentmarker}{}%
\end{pgfscope}%
\end{pgfscope}%
\begin{pgfscope}%
\definecolor{textcolor}{rgb}{0.000000,0.000000,0.000000}%
\pgfsetstrokecolor{textcolor}%
\pgfsetfillcolor{textcolor}%
\pgftext[x=4.153497in,y=0.402778in,,top]{\color{textcolor}\sffamily\fontsize{10.000000}{12.000000}\selectfont \(\displaystyle {10}\)}%
\end{pgfscope}%
\begin{pgfscope}%
\definecolor{textcolor}{rgb}{0.000000,0.000000,0.000000}%
\pgfsetstrokecolor{textcolor}%
\pgfsetfillcolor{textcolor}%
\pgftext[x=2.610000in,y=0.223766in,,top]{\color{textcolor}\sffamily\fontsize{10.000000}{12.000000}\selectfont \(\displaystyle \zeta \, \mathrm{[\mu m]}\)}%
\end{pgfscope}%
\begin{pgfscope}%
\pgfsetbuttcap%
\pgfsetroundjoin%
\definecolor{currentfill}{rgb}{0.000000,0.000000,0.000000}%
\pgfsetfillcolor{currentfill}%
\pgfsetlinewidth{0.803000pt}%
\definecolor{currentstroke}{rgb}{0.000000,0.000000,0.000000}%
\pgfsetstrokecolor{currentstroke}%
\pgfsetdash{}{0pt}%
\pgfsys@defobject{currentmarker}{\pgfqpoint{-0.048611in}{0.000000in}}{\pgfqpoint{-0.000000in}{0.000000in}}{%
\pgfpathmoveto{\pgfqpoint{-0.000000in}{0.000000in}}%
\pgfpathlineto{\pgfqpoint{-0.048611in}{0.000000in}}%
\pgfusepath{stroke,fill}%
}%
\begin{pgfscope}%
\pgfsys@transformshift{0.750000in}{0.574716in}%
\pgfsys@useobject{currentmarker}{}%
\end{pgfscope}%
\end{pgfscope}%
\begin{pgfscope}%
\definecolor{textcolor}{rgb}{0.000000,0.000000,0.000000}%
\pgfsetstrokecolor{textcolor}%
\pgfsetfillcolor{textcolor}%
\pgftext[x=0.583333in, y=0.526490in, left, base]{\color{textcolor}\sffamily\fontsize{10.000000}{12.000000}\selectfont \(\displaystyle {0}\)}%
\end{pgfscope}%
\begin{pgfscope}%
\pgfsetbuttcap%
\pgfsetroundjoin%
\definecolor{currentfill}{rgb}{0.000000,0.000000,0.000000}%
\pgfsetfillcolor{currentfill}%
\pgfsetlinewidth{0.803000pt}%
\definecolor{currentstroke}{rgb}{0.000000,0.000000,0.000000}%
\pgfsetstrokecolor{currentstroke}%
\pgfsetdash{}{0pt}%
\pgfsys@defobject{currentmarker}{\pgfqpoint{-0.048611in}{0.000000in}}{\pgfqpoint{-0.000000in}{0.000000in}}{%
\pgfpathmoveto{\pgfqpoint{-0.000000in}{0.000000in}}%
\pgfpathlineto{\pgfqpoint{-0.048611in}{0.000000in}}%
\pgfusepath{stroke,fill}%
}%
\begin{pgfscope}%
\pgfsys@transformshift{0.750000in}{1.049430in}%
\pgfsys@useobject{currentmarker}{}%
\end{pgfscope}%
\end{pgfscope}%
\begin{pgfscope}%
\definecolor{textcolor}{rgb}{0.000000,0.000000,0.000000}%
\pgfsetstrokecolor{textcolor}%
\pgfsetfillcolor{textcolor}%
\pgftext[x=0.583333in, y=1.001205in, left, base]{\color{textcolor}\sffamily\fontsize{10.000000}{12.000000}\selectfont \(\displaystyle {1}\)}%
\end{pgfscope}%
\begin{pgfscope}%
\pgfsetbuttcap%
\pgfsetroundjoin%
\definecolor{currentfill}{rgb}{0.000000,0.000000,0.000000}%
\pgfsetfillcolor{currentfill}%
\pgfsetlinewidth{0.803000pt}%
\definecolor{currentstroke}{rgb}{0.000000,0.000000,0.000000}%
\pgfsetstrokecolor{currentstroke}%
\pgfsetdash{}{0pt}%
\pgfsys@defobject{currentmarker}{\pgfqpoint{-0.048611in}{0.000000in}}{\pgfqpoint{-0.000000in}{0.000000in}}{%
\pgfpathmoveto{\pgfqpoint{-0.000000in}{0.000000in}}%
\pgfpathlineto{\pgfqpoint{-0.048611in}{0.000000in}}%
\pgfusepath{stroke,fill}%
}%
\begin{pgfscope}%
\pgfsys@transformshift{0.750000in}{1.524145in}%
\pgfsys@useobject{currentmarker}{}%
\end{pgfscope}%
\end{pgfscope}%
\begin{pgfscope}%
\definecolor{textcolor}{rgb}{0.000000,0.000000,0.000000}%
\pgfsetstrokecolor{textcolor}%
\pgfsetfillcolor{textcolor}%
\pgftext[x=0.583333in, y=1.475920in, left, base]{\color{textcolor}\sffamily\fontsize{10.000000}{12.000000}\selectfont \(\displaystyle {2}\)}%
\end{pgfscope}%
\begin{pgfscope}%
\pgfsetbuttcap%
\pgfsetroundjoin%
\definecolor{currentfill}{rgb}{0.000000,0.000000,0.000000}%
\pgfsetfillcolor{currentfill}%
\pgfsetlinewidth{0.803000pt}%
\definecolor{currentstroke}{rgb}{0.000000,0.000000,0.000000}%
\pgfsetstrokecolor{currentstroke}%
\pgfsetdash{}{0pt}%
\pgfsys@defobject{currentmarker}{\pgfqpoint{-0.048611in}{0.000000in}}{\pgfqpoint{-0.000000in}{0.000000in}}{%
\pgfpathmoveto{\pgfqpoint{-0.000000in}{0.000000in}}%
\pgfpathlineto{\pgfqpoint{-0.048611in}{0.000000in}}%
\pgfusepath{stroke,fill}%
}%
\begin{pgfscope}%
\pgfsys@transformshift{0.750000in}{1.998860in}%
\pgfsys@useobject{currentmarker}{}%
\end{pgfscope}%
\end{pgfscope}%
\begin{pgfscope}%
\definecolor{textcolor}{rgb}{0.000000,0.000000,0.000000}%
\pgfsetstrokecolor{textcolor}%
\pgfsetfillcolor{textcolor}%
\pgftext[x=0.583333in, y=1.950635in, left, base]{\color{textcolor}\sffamily\fontsize{10.000000}{12.000000}\selectfont \(\displaystyle {3}\)}%
\end{pgfscope}%
\begin{pgfscope}%
\pgfsetbuttcap%
\pgfsetroundjoin%
\definecolor{currentfill}{rgb}{0.000000,0.000000,0.000000}%
\pgfsetfillcolor{currentfill}%
\pgfsetlinewidth{0.803000pt}%
\definecolor{currentstroke}{rgb}{0.000000,0.000000,0.000000}%
\pgfsetstrokecolor{currentstroke}%
\pgfsetdash{}{0pt}%
\pgfsys@defobject{currentmarker}{\pgfqpoint{-0.048611in}{0.000000in}}{\pgfqpoint{-0.000000in}{0.000000in}}{%
\pgfpathmoveto{\pgfqpoint{-0.000000in}{0.000000in}}%
\pgfpathlineto{\pgfqpoint{-0.048611in}{0.000000in}}%
\pgfusepath{stroke,fill}%
}%
\begin{pgfscope}%
\pgfsys@transformshift{0.750000in}{2.473575in}%
\pgfsys@useobject{currentmarker}{}%
\end{pgfscope}%
\end{pgfscope}%
\begin{pgfscope}%
\definecolor{textcolor}{rgb}{0.000000,0.000000,0.000000}%
\pgfsetstrokecolor{textcolor}%
\pgfsetfillcolor{textcolor}%
\pgftext[x=0.583333in, y=2.425350in, left, base]{\color{textcolor}\sffamily\fontsize{10.000000}{12.000000}\selectfont \(\displaystyle {4}\)}%
\end{pgfscope}%
\begin{pgfscope}%
\pgfsetbuttcap%
\pgfsetroundjoin%
\definecolor{currentfill}{rgb}{0.000000,0.000000,0.000000}%
\pgfsetfillcolor{currentfill}%
\pgfsetlinewidth{0.803000pt}%
\definecolor{currentstroke}{rgb}{0.000000,0.000000,0.000000}%
\pgfsetstrokecolor{currentstroke}%
\pgfsetdash{}{0pt}%
\pgfsys@defobject{currentmarker}{\pgfqpoint{-0.048611in}{0.000000in}}{\pgfqpoint{-0.000000in}{0.000000in}}{%
\pgfpathmoveto{\pgfqpoint{-0.000000in}{0.000000in}}%
\pgfpathlineto{\pgfqpoint{-0.048611in}{0.000000in}}%
\pgfusepath{stroke,fill}%
}%
\begin{pgfscope}%
\pgfsys@transformshift{0.750000in}{2.948290in}%
\pgfsys@useobject{currentmarker}{}%
\end{pgfscope}%
\end{pgfscope}%
\begin{pgfscope}%
\definecolor{textcolor}{rgb}{0.000000,0.000000,0.000000}%
\pgfsetstrokecolor{textcolor}%
\pgfsetfillcolor{textcolor}%
\pgftext[x=0.583333in, y=2.900065in, left, base]{\color{textcolor}\sffamily\fontsize{10.000000}{12.000000}\selectfont \(\displaystyle {5}\)}%
\end{pgfscope}%
\begin{pgfscope}%
\pgfsetbuttcap%
\pgfsetroundjoin%
\definecolor{currentfill}{rgb}{0.000000,0.000000,0.000000}%
\pgfsetfillcolor{currentfill}%
\pgfsetlinewidth{0.803000pt}%
\definecolor{currentstroke}{rgb}{0.000000,0.000000,0.000000}%
\pgfsetstrokecolor{currentstroke}%
\pgfsetdash{}{0pt}%
\pgfsys@defobject{currentmarker}{\pgfqpoint{-0.048611in}{0.000000in}}{\pgfqpoint{-0.000000in}{0.000000in}}{%
\pgfpathmoveto{\pgfqpoint{-0.000000in}{0.000000in}}%
\pgfpathlineto{\pgfqpoint{-0.048611in}{0.000000in}}%
\pgfusepath{stroke,fill}%
}%
\begin{pgfscope}%
\pgfsys@transformshift{0.750000in}{3.423005in}%
\pgfsys@useobject{currentmarker}{}%
\end{pgfscope}%
\end{pgfscope}%
\begin{pgfscope}%
\definecolor{textcolor}{rgb}{0.000000,0.000000,0.000000}%
\pgfsetstrokecolor{textcolor}%
\pgfsetfillcolor{textcolor}%
\pgftext[x=0.583333in, y=3.374780in, left, base]{\color{textcolor}\sffamily\fontsize{10.000000}{12.000000}\selectfont \(\displaystyle {6}\)}%
\end{pgfscope}%
\begin{pgfscope}%
\definecolor{textcolor}{rgb}{0.000000,0.000000,0.000000}%
\pgfsetstrokecolor{textcolor}%
\pgfsetfillcolor{textcolor}%
\pgftext[x=0.527778in,y=2.010000in,,bottom,rotate=90.000000]{\color{textcolor}\sffamily\fontsize{10.000000}{12.000000}\selectfont \(\displaystyle y \, \mathrm{[mm]}\)}%
\end{pgfscope}%
\begin{pgfscope}%
\pgfsetrectcap%
\pgfsetmiterjoin%
\pgfsetlinewidth{0.803000pt}%
\definecolor{currentstroke}{rgb}{0.000000,0.000000,0.000000}%
\pgfsetstrokecolor{currentstroke}%
\pgfsetdash{}{0pt}%
\pgfpathmoveto{\pgfqpoint{0.750000in}{0.500000in}}%
\pgfpathlineto{\pgfqpoint{0.750000in}{3.520000in}}%
\pgfusepath{stroke}%
\end{pgfscope}%
\begin{pgfscope}%
\pgfsetrectcap%
\pgfsetmiterjoin%
\pgfsetlinewidth{0.803000pt}%
\definecolor{currentstroke}{rgb}{0.000000,0.000000,0.000000}%
\pgfsetstrokecolor{currentstroke}%
\pgfsetdash{}{0pt}%
\pgfpathmoveto{\pgfqpoint{4.470000in}{0.500000in}}%
\pgfpathlineto{\pgfqpoint{4.470000in}{3.520000in}}%
\pgfusepath{stroke}%
\end{pgfscope}%
\begin{pgfscope}%
\pgfsetrectcap%
\pgfsetmiterjoin%
\pgfsetlinewidth{0.803000pt}%
\definecolor{currentstroke}{rgb}{0.000000,0.000000,0.000000}%
\pgfsetstrokecolor{currentstroke}%
\pgfsetdash{}{0pt}%
\pgfpathmoveto{\pgfqpoint{0.750000in}{0.500000in}}%
\pgfpathlineto{\pgfqpoint{4.470000in}{0.500000in}}%
\pgfusepath{stroke}%
\end{pgfscope}%
\begin{pgfscope}%
\pgfsetrectcap%
\pgfsetmiterjoin%
\pgfsetlinewidth{0.803000pt}%
\definecolor{currentstroke}{rgb}{0.000000,0.000000,0.000000}%
\pgfsetstrokecolor{currentstroke}%
\pgfsetdash{}{0pt}%
\pgfpathmoveto{\pgfqpoint{0.750000in}{3.520000in}}%
\pgfpathlineto{\pgfqpoint{4.470000in}{3.520000in}}%
\pgfusepath{stroke}%
\end{pgfscope}%
\begin{pgfscope}%
\pgfsetbuttcap%
\pgfsetmiterjoin%
\definecolor{currentfill}{rgb}{1.000000,1.000000,1.000000}%
\pgfsetfillcolor{currentfill}%
\pgfsetlinewidth{0.000000pt}%
\definecolor{currentstroke}{rgb}{0.000000,0.000000,0.000000}%
\pgfsetstrokecolor{currentstroke}%
\pgfsetstrokeopacity{0.000000}%
\pgfsetdash{}{0pt}%
\pgfpathmoveto{\pgfqpoint{4.702500in}{0.500000in}}%
\pgfpathlineto{\pgfqpoint{4.853500in}{0.500000in}}%
\pgfpathlineto{\pgfqpoint{4.853500in}{3.520000in}}%
\pgfpathlineto{\pgfqpoint{4.702500in}{3.520000in}}%
\pgfpathlineto{\pgfqpoint{4.702500in}{0.500000in}}%
\pgfpathclose%
\pgfusepath{fill}%
\end{pgfscope}%
\begin{pgfscope}%
\pgfpathrectangle{\pgfqpoint{4.702500in}{0.500000in}}{\pgfqpoint{0.151000in}{3.020000in}}%
\pgfusepath{clip}%
\pgfsetbuttcap%
\pgfsetmiterjoin%
\definecolor{currentfill}{rgb}{1.000000,1.000000,1.000000}%
\pgfsetfillcolor{currentfill}%
\pgfsetlinewidth{0.010037pt}%
\definecolor{currentstroke}{rgb}{1.000000,1.000000,1.000000}%
\pgfsetstrokecolor{currentstroke}%
\pgfsetdash{}{0pt}%
\pgfusepath{stroke,fill}%
\end{pgfscope}%
\begin{pgfscope}%
\pgfsys@transformshift{4.708333in}{0.500000in}%
\pgftext[left,bottom]{\includegraphics[interpolate=true,width=0.138889in,height=3.013889in]{force_time-img1.png}}%
\end{pgfscope}%
\begin{pgfscope}%
\pgfsetbuttcap%
\pgfsetroundjoin%
\definecolor{currentfill}{rgb}{0.000000,0.000000,0.000000}%
\pgfsetfillcolor{currentfill}%
\pgfsetlinewidth{0.803000pt}%
\definecolor{currentstroke}{rgb}{0.000000,0.000000,0.000000}%
\pgfsetstrokecolor{currentstroke}%
\pgfsetdash{}{0pt}%
\pgfsys@defobject{currentmarker}{\pgfqpoint{0.000000in}{0.000000in}}{\pgfqpoint{0.048611in}{0.000000in}}{%
\pgfpathmoveto{\pgfqpoint{0.000000in}{0.000000in}}%
\pgfpathlineto{\pgfqpoint{0.048611in}{0.000000in}}%
\pgfusepath{stroke,fill}%
}%
\begin{pgfscope}%
\pgfsys@transformshift{4.853500in}{0.667778in}%
\pgfsys@useobject{currentmarker}{}%
\end{pgfscope}%
\end{pgfscope}%
\begin{pgfscope}%
\definecolor{textcolor}{rgb}{0.000000,0.000000,0.000000}%
\pgfsetstrokecolor{textcolor}%
\pgfsetfillcolor{textcolor}%
\pgftext[x=4.950722in, y=0.619553in, left, base]{\color{textcolor}\sffamily\fontsize{10.000000}{12.000000}\selectfont \(\displaystyle {\ensuremath{-}800}\)}%
\end{pgfscope}%
\begin{pgfscope}%
\pgfsetbuttcap%
\pgfsetroundjoin%
\definecolor{currentfill}{rgb}{0.000000,0.000000,0.000000}%
\pgfsetfillcolor{currentfill}%
\pgfsetlinewidth{0.803000pt}%
\definecolor{currentstroke}{rgb}{0.000000,0.000000,0.000000}%
\pgfsetstrokecolor{currentstroke}%
\pgfsetdash{}{0pt}%
\pgfsys@defobject{currentmarker}{\pgfqpoint{0.000000in}{0.000000in}}{\pgfqpoint{0.048611in}{0.000000in}}{%
\pgfpathmoveto{\pgfqpoint{0.000000in}{0.000000in}}%
\pgfpathlineto{\pgfqpoint{0.048611in}{0.000000in}}%
\pgfusepath{stroke,fill}%
}%
\begin{pgfscope}%
\pgfsys@transformshift{4.853500in}{1.003333in}%
\pgfsys@useobject{currentmarker}{}%
\end{pgfscope}%
\end{pgfscope}%
\begin{pgfscope}%
\definecolor{textcolor}{rgb}{0.000000,0.000000,0.000000}%
\pgfsetstrokecolor{textcolor}%
\pgfsetfillcolor{textcolor}%
\pgftext[x=4.950722in, y=0.955108in, left, base]{\color{textcolor}\sffamily\fontsize{10.000000}{12.000000}\selectfont \(\displaystyle {\ensuremath{-}600}\)}%
\end{pgfscope}%
\begin{pgfscope}%
\pgfsetbuttcap%
\pgfsetroundjoin%
\definecolor{currentfill}{rgb}{0.000000,0.000000,0.000000}%
\pgfsetfillcolor{currentfill}%
\pgfsetlinewidth{0.803000pt}%
\definecolor{currentstroke}{rgb}{0.000000,0.000000,0.000000}%
\pgfsetstrokecolor{currentstroke}%
\pgfsetdash{}{0pt}%
\pgfsys@defobject{currentmarker}{\pgfqpoint{0.000000in}{0.000000in}}{\pgfqpoint{0.048611in}{0.000000in}}{%
\pgfpathmoveto{\pgfqpoint{0.000000in}{0.000000in}}%
\pgfpathlineto{\pgfqpoint{0.048611in}{0.000000in}}%
\pgfusepath{stroke,fill}%
}%
\begin{pgfscope}%
\pgfsys@transformshift{4.853500in}{1.338889in}%
\pgfsys@useobject{currentmarker}{}%
\end{pgfscope}%
\end{pgfscope}%
\begin{pgfscope}%
\definecolor{textcolor}{rgb}{0.000000,0.000000,0.000000}%
\pgfsetstrokecolor{textcolor}%
\pgfsetfillcolor{textcolor}%
\pgftext[x=4.950722in, y=1.290664in, left, base]{\color{textcolor}\sffamily\fontsize{10.000000}{12.000000}\selectfont \(\displaystyle {\ensuremath{-}400}\)}%
\end{pgfscope}%
\begin{pgfscope}%
\pgfsetbuttcap%
\pgfsetroundjoin%
\definecolor{currentfill}{rgb}{0.000000,0.000000,0.000000}%
\pgfsetfillcolor{currentfill}%
\pgfsetlinewidth{0.803000pt}%
\definecolor{currentstroke}{rgb}{0.000000,0.000000,0.000000}%
\pgfsetstrokecolor{currentstroke}%
\pgfsetdash{}{0pt}%
\pgfsys@defobject{currentmarker}{\pgfqpoint{0.000000in}{0.000000in}}{\pgfqpoint{0.048611in}{0.000000in}}{%
\pgfpathmoveto{\pgfqpoint{0.000000in}{0.000000in}}%
\pgfpathlineto{\pgfqpoint{0.048611in}{0.000000in}}%
\pgfusepath{stroke,fill}%
}%
\begin{pgfscope}%
\pgfsys@transformshift{4.853500in}{1.674444in}%
\pgfsys@useobject{currentmarker}{}%
\end{pgfscope}%
\end{pgfscope}%
\begin{pgfscope}%
\definecolor{textcolor}{rgb}{0.000000,0.000000,0.000000}%
\pgfsetstrokecolor{textcolor}%
\pgfsetfillcolor{textcolor}%
\pgftext[x=4.950722in, y=1.626219in, left, base]{\color{textcolor}\sffamily\fontsize{10.000000}{12.000000}\selectfont \(\displaystyle {\ensuremath{-}200}\)}%
\end{pgfscope}%
\begin{pgfscope}%
\pgfsetbuttcap%
\pgfsetroundjoin%
\definecolor{currentfill}{rgb}{0.000000,0.000000,0.000000}%
\pgfsetfillcolor{currentfill}%
\pgfsetlinewidth{0.803000pt}%
\definecolor{currentstroke}{rgb}{0.000000,0.000000,0.000000}%
\pgfsetstrokecolor{currentstroke}%
\pgfsetdash{}{0pt}%
\pgfsys@defobject{currentmarker}{\pgfqpoint{0.000000in}{0.000000in}}{\pgfqpoint{0.048611in}{0.000000in}}{%
\pgfpathmoveto{\pgfqpoint{0.000000in}{0.000000in}}%
\pgfpathlineto{\pgfqpoint{0.048611in}{0.000000in}}%
\pgfusepath{stroke,fill}%
}%
\begin{pgfscope}%
\pgfsys@transformshift{4.853500in}{2.010000in}%
\pgfsys@useobject{currentmarker}{}%
\end{pgfscope}%
\end{pgfscope}%
\begin{pgfscope}%
\definecolor{textcolor}{rgb}{0.000000,0.000000,0.000000}%
\pgfsetstrokecolor{textcolor}%
\pgfsetfillcolor{textcolor}%
\pgftext[x=4.950722in, y=1.961775in, left, base]{\color{textcolor}\sffamily\fontsize{10.000000}{12.000000}\selectfont \(\displaystyle {0}\)}%
\end{pgfscope}%
\begin{pgfscope}%
\pgfsetbuttcap%
\pgfsetroundjoin%
\definecolor{currentfill}{rgb}{0.000000,0.000000,0.000000}%
\pgfsetfillcolor{currentfill}%
\pgfsetlinewidth{0.803000pt}%
\definecolor{currentstroke}{rgb}{0.000000,0.000000,0.000000}%
\pgfsetstrokecolor{currentstroke}%
\pgfsetdash{}{0pt}%
\pgfsys@defobject{currentmarker}{\pgfqpoint{0.000000in}{0.000000in}}{\pgfqpoint{0.048611in}{0.000000in}}{%
\pgfpathmoveto{\pgfqpoint{0.000000in}{0.000000in}}%
\pgfpathlineto{\pgfqpoint{0.048611in}{0.000000in}}%
\pgfusepath{stroke,fill}%
}%
\begin{pgfscope}%
\pgfsys@transformshift{4.853500in}{2.345556in}%
\pgfsys@useobject{currentmarker}{}%
\end{pgfscope}%
\end{pgfscope}%
\begin{pgfscope}%
\definecolor{textcolor}{rgb}{0.000000,0.000000,0.000000}%
\pgfsetstrokecolor{textcolor}%
\pgfsetfillcolor{textcolor}%
\pgftext[x=4.950722in, y=2.297330in, left, base]{\color{textcolor}\sffamily\fontsize{10.000000}{12.000000}\selectfont \(\displaystyle {200}\)}%
\end{pgfscope}%
\begin{pgfscope}%
\pgfsetbuttcap%
\pgfsetroundjoin%
\definecolor{currentfill}{rgb}{0.000000,0.000000,0.000000}%
\pgfsetfillcolor{currentfill}%
\pgfsetlinewidth{0.803000pt}%
\definecolor{currentstroke}{rgb}{0.000000,0.000000,0.000000}%
\pgfsetstrokecolor{currentstroke}%
\pgfsetdash{}{0pt}%
\pgfsys@defobject{currentmarker}{\pgfqpoint{0.000000in}{0.000000in}}{\pgfqpoint{0.048611in}{0.000000in}}{%
\pgfpathmoveto{\pgfqpoint{0.000000in}{0.000000in}}%
\pgfpathlineto{\pgfqpoint{0.048611in}{0.000000in}}%
\pgfusepath{stroke,fill}%
}%
\begin{pgfscope}%
\pgfsys@transformshift{4.853500in}{2.681111in}%
\pgfsys@useobject{currentmarker}{}%
\end{pgfscope}%
\end{pgfscope}%
\begin{pgfscope}%
\definecolor{textcolor}{rgb}{0.000000,0.000000,0.000000}%
\pgfsetstrokecolor{textcolor}%
\pgfsetfillcolor{textcolor}%
\pgftext[x=4.950722in, y=2.632886in, left, base]{\color{textcolor}\sffamily\fontsize{10.000000}{12.000000}\selectfont \(\displaystyle {400}\)}%
\end{pgfscope}%
\begin{pgfscope}%
\pgfsetbuttcap%
\pgfsetroundjoin%
\definecolor{currentfill}{rgb}{0.000000,0.000000,0.000000}%
\pgfsetfillcolor{currentfill}%
\pgfsetlinewidth{0.803000pt}%
\definecolor{currentstroke}{rgb}{0.000000,0.000000,0.000000}%
\pgfsetstrokecolor{currentstroke}%
\pgfsetdash{}{0pt}%
\pgfsys@defobject{currentmarker}{\pgfqpoint{0.000000in}{0.000000in}}{\pgfqpoint{0.048611in}{0.000000in}}{%
\pgfpathmoveto{\pgfqpoint{0.000000in}{0.000000in}}%
\pgfpathlineto{\pgfqpoint{0.048611in}{0.000000in}}%
\pgfusepath{stroke,fill}%
}%
\begin{pgfscope}%
\pgfsys@transformshift{4.853500in}{3.016667in}%
\pgfsys@useobject{currentmarker}{}%
\end{pgfscope}%
\end{pgfscope}%
\begin{pgfscope}%
\definecolor{textcolor}{rgb}{0.000000,0.000000,0.000000}%
\pgfsetstrokecolor{textcolor}%
\pgfsetfillcolor{textcolor}%
\pgftext[x=4.950722in, y=2.968441in, left, base]{\color{textcolor}\sffamily\fontsize{10.000000}{12.000000}\selectfont \(\displaystyle {600}\)}%
\end{pgfscope}%
\begin{pgfscope}%
\pgfsetbuttcap%
\pgfsetroundjoin%
\definecolor{currentfill}{rgb}{0.000000,0.000000,0.000000}%
\pgfsetfillcolor{currentfill}%
\pgfsetlinewidth{0.803000pt}%
\definecolor{currentstroke}{rgb}{0.000000,0.000000,0.000000}%
\pgfsetstrokecolor{currentstroke}%
\pgfsetdash{}{0pt}%
\pgfsys@defobject{currentmarker}{\pgfqpoint{0.000000in}{0.000000in}}{\pgfqpoint{0.048611in}{0.000000in}}{%
\pgfpathmoveto{\pgfqpoint{0.000000in}{0.000000in}}%
\pgfpathlineto{\pgfqpoint{0.048611in}{0.000000in}}%
\pgfusepath{stroke,fill}%
}%
\begin{pgfscope}%
\pgfsys@transformshift{4.853500in}{3.352222in}%
\pgfsys@useobject{currentmarker}{}%
\end{pgfscope}%
\end{pgfscope}%
\begin{pgfscope}%
\definecolor{textcolor}{rgb}{0.000000,0.000000,0.000000}%
\pgfsetstrokecolor{textcolor}%
\pgfsetfillcolor{textcolor}%
\pgftext[x=4.950722in, y=3.303997in, left, base]{\color{textcolor}\sffamily\fontsize{10.000000}{12.000000}\selectfont \(\displaystyle {800}\)}%
\end{pgfscope}%
\begin{pgfscope}%
\definecolor{textcolor}{rgb}{0.000000,0.000000,0.000000}%
\pgfsetstrokecolor{textcolor}%
\pgfsetfillcolor{textcolor}%
\pgftext[x=5.322637in,y=2.010000in,,top,rotate=90.000000]{\color{textcolor}\sffamily\fontsize{10.000000}{12.000000}\selectfont \(\displaystyle F_{\parallel} \, \mathrm{[\mu N]}\)}%
\end{pgfscope}%
\begin{pgfscope}%
\pgfsetrectcap%
\pgfsetmiterjoin%
\pgfsetlinewidth{0.803000pt}%
\definecolor{currentstroke}{rgb}{0.000000,0.000000,0.000000}%
\pgfsetstrokecolor{currentstroke}%
\pgfsetdash{}{0pt}%
\pgfpathmoveto{\pgfqpoint{4.702500in}{0.500000in}}%
\pgfpathlineto{\pgfqpoint{4.778000in}{0.500000in}}%
\pgfpathlineto{\pgfqpoint{4.853500in}{0.500000in}}%
\pgfpathlineto{\pgfqpoint{4.853500in}{3.520000in}}%
\pgfpathlineto{\pgfqpoint{4.778000in}{3.520000in}}%
\pgfpathlineto{\pgfqpoint{4.702500in}{3.520000in}}%
\pgfpathlineto{\pgfqpoint{4.702500in}{0.500000in}}%
\pgfpathclose%
\pgfusepath{stroke}%
\end{pgfscope}%
\end{pgfpicture}%
\makeatother%
\endgroup%

	\caption{Histogram of the longitudinal part of the Lorentz force. The force is sampled over the co-moving $\zeta$-axis at a slice in the middle in $z$-direction for every 2000 timesteps. This slice here is plotted over $y$, the distance of the driver from the start of the plasma, showing how it changes in position and strength.}
	\label{fig:force_time}
\end{figure}
A forward pushing force on the backside of the driver can also be seen, as the Gaussian distribution of the driver reaches so far back, that some particles are positioned in the accelerating part of the first cavity.
Notable is the situation around $y=$ \qty{3}{mm}, where the bunch collapses and particles start to fall back rapidly, so they get accelerated again in the back of the first cavity. These particles stem mostly from the middle of the driver,
where the strongest backwards-pushing forces acted before and caused them to drain their energy. 

Here the bunch breakup has no big effect, as the cavities already resumed to the linear regime and only a small part of the driver did fall back. Only small forces remain to act on the driver, so it slowly diverges to the transverse sides.

\subsection{Particle tracking}
To support the claims made about the particle movement, particle tracking was used. In \autoref{fig:y_hist_time}a the particle distribution, shortly after entering the plasma, can be seen.
\begin{figure}
	\centering
	%% Creator: Matplotlib, PGF backend
%%
%% To include the figure in your LaTeX document, write
%%   \input{<filename>.pgf}
%%
%% Make sure the required packages are loaded in your preamble
%%   \usepackage{pgf}
%%
%% Also ensure that all the required font packages are loaded; for instance,
%% the lmodern package is sometimes necessary when using math font.
%%   \usepackage{lmodern}
%%
%% Figures using additional raster images can only be included by \input if
%% they are in the same directory as the main LaTeX file. For loading figures
%% from other directories you can use the `import` package
%%   \usepackage{import}
%%
%% and then include the figures with
%%   \import{<path to file>}{<filename>.pgf}
%%
%% Matplotlib used the following preamble
%%
\begingroup%
\makeatletter%
\begin{pgfpicture}%
\pgfpathrectangle{\pgfpointorigin}{\pgfqpoint{6.000000in}{4.000000in}}%
\pgfusepath{use as bounding box, clip}%
\begin{pgfscope}%
\pgfsetbuttcap%
\pgfsetmiterjoin%
\pgfsetlinewidth{0.000000pt}%
\definecolor{currentstroke}{rgb}{1.000000,1.000000,1.000000}%
\pgfsetstrokecolor{currentstroke}%
\pgfsetstrokeopacity{0.000000}%
\pgfsetdash{}{0pt}%
\pgfpathmoveto{\pgfqpoint{0.000000in}{0.000000in}}%
\pgfpathlineto{\pgfqpoint{6.000000in}{0.000000in}}%
\pgfpathlineto{\pgfqpoint{6.000000in}{4.000000in}}%
\pgfpathlineto{\pgfqpoint{0.000000in}{4.000000in}}%
\pgfpathlineto{\pgfqpoint{0.000000in}{0.000000in}}%
\pgfpathclose%
\pgfusepath{}%
\end{pgfscope}%
\begin{pgfscope}%
\pgfsetbuttcap%
\pgfsetmiterjoin%
\definecolor{currentfill}{rgb}{1.000000,1.000000,1.000000}%
\pgfsetfillcolor{currentfill}%
\pgfsetlinewidth{0.000000pt}%
\definecolor{currentstroke}{rgb}{0.000000,0.000000,0.000000}%
\pgfsetstrokecolor{currentstroke}%
\pgfsetstrokeopacity{0.000000}%
\pgfsetdash{}{0pt}%
\pgfpathmoveto{\pgfqpoint{0.750000in}{1.406000in}}%
\pgfpathlineto{\pgfqpoint{5.400000in}{1.406000in}}%
\pgfpathlineto{\pgfqpoint{5.400000in}{3.520000in}}%
\pgfpathlineto{\pgfqpoint{0.750000in}{3.520000in}}%
\pgfpathlineto{\pgfqpoint{0.750000in}{1.406000in}}%
\pgfpathclose%
\pgfusepath{fill}%
\end{pgfscope}%
\begin{pgfscope}%
\pgfpathrectangle{\pgfqpoint{0.750000in}{1.406000in}}{\pgfqpoint{4.650000in}{2.114000in}}%
\pgfusepath{clip}%
\pgfsys@transformcm{4.652778}{0.000000}{0.000000}{2.125000}{0.750000in}{1.406000in}%
\pgftext[left,bottom]{\includegraphics[interpolate=false,width=1.000000in,height=1.000000in]{y_hist_time-img0.png}}%
\end{pgfscope}%
\begin{pgfscope}%
\pgfsetbuttcap%
\pgfsetroundjoin%
\definecolor{currentfill}{rgb}{0.000000,0.000000,0.000000}%
\pgfsetfillcolor{currentfill}%
\pgfsetlinewidth{0.803000pt}%
\definecolor{currentstroke}{rgb}{0.000000,0.000000,0.000000}%
\pgfsetstrokecolor{currentstroke}%
\pgfsetdash{}{0pt}%
\pgfsys@defobject{currentmarker}{\pgfqpoint{0.000000in}{-0.048611in}}{\pgfqpoint{0.000000in}{0.000000in}}{%
\pgfpathmoveto{\pgfqpoint{0.000000in}{0.000000in}}%
\pgfpathlineto{\pgfqpoint{0.000000in}{-0.048611in}}%
\pgfusepath{stroke,fill}%
}%
\begin{pgfscope}%
\pgfsys@transformshift{1.343443in}{1.406000in}%
\pgfsys@useobject{currentmarker}{}%
\end{pgfscope}%
\end{pgfscope}%
\begin{pgfscope}%
\definecolor{textcolor}{rgb}{0.000000,0.000000,0.000000}%
\pgfsetstrokecolor{textcolor}%
\pgfsetfillcolor{textcolor}%
\pgftext[x=1.343443in,y=1.308778in,,top]{\color{textcolor}\sffamily\fontsize{10.000000}{12.000000}\selectfont \(\displaystyle {\ensuremath{-}15}\)}%
\end{pgfscope}%
\begin{pgfscope}%
\pgfsetbuttcap%
\pgfsetroundjoin%
\definecolor{currentfill}{rgb}{0.000000,0.000000,0.000000}%
\pgfsetfillcolor{currentfill}%
\pgfsetlinewidth{0.803000pt}%
\definecolor{currentstroke}{rgb}{0.000000,0.000000,0.000000}%
\pgfsetstrokecolor{currentstroke}%
\pgfsetdash{}{0pt}%
\pgfsys@defobject{currentmarker}{\pgfqpoint{0.000000in}{-0.048611in}}{\pgfqpoint{0.000000in}{0.000000in}}{%
\pgfpathmoveto{\pgfqpoint{0.000000in}{0.000000in}}%
\pgfpathlineto{\pgfqpoint{0.000000in}{-0.048611in}}%
\pgfusepath{stroke,fill}%
}%
\begin{pgfscope}%
\pgfsys@transformshift{2.075629in}{1.406000in}%
\pgfsys@useobject{currentmarker}{}%
\end{pgfscope}%
\end{pgfscope}%
\begin{pgfscope}%
\definecolor{textcolor}{rgb}{0.000000,0.000000,0.000000}%
\pgfsetstrokecolor{textcolor}%
\pgfsetfillcolor{textcolor}%
\pgftext[x=2.075629in,y=1.308778in,,top]{\color{textcolor}\sffamily\fontsize{10.000000}{12.000000}\selectfont \(\displaystyle {\ensuremath{-}10}\)}%
\end{pgfscope}%
\begin{pgfscope}%
\pgfsetbuttcap%
\pgfsetroundjoin%
\definecolor{currentfill}{rgb}{0.000000,0.000000,0.000000}%
\pgfsetfillcolor{currentfill}%
\pgfsetlinewidth{0.803000pt}%
\definecolor{currentstroke}{rgb}{0.000000,0.000000,0.000000}%
\pgfsetstrokecolor{currentstroke}%
\pgfsetdash{}{0pt}%
\pgfsys@defobject{currentmarker}{\pgfqpoint{0.000000in}{-0.048611in}}{\pgfqpoint{0.000000in}{0.000000in}}{%
\pgfpathmoveto{\pgfqpoint{0.000000in}{0.000000in}}%
\pgfpathlineto{\pgfqpoint{0.000000in}{-0.048611in}}%
\pgfusepath{stroke,fill}%
}%
\begin{pgfscope}%
\pgfsys@transformshift{2.807814in}{1.406000in}%
\pgfsys@useobject{currentmarker}{}%
\end{pgfscope}%
\end{pgfscope}%
\begin{pgfscope}%
\definecolor{textcolor}{rgb}{0.000000,0.000000,0.000000}%
\pgfsetstrokecolor{textcolor}%
\pgfsetfillcolor{textcolor}%
\pgftext[x=2.807814in,y=1.308778in,,top]{\color{textcolor}\sffamily\fontsize{10.000000}{12.000000}\selectfont \(\displaystyle {\ensuremath{-}5}\)}%
\end{pgfscope}%
\begin{pgfscope}%
\pgfsetbuttcap%
\pgfsetroundjoin%
\definecolor{currentfill}{rgb}{0.000000,0.000000,0.000000}%
\pgfsetfillcolor{currentfill}%
\pgfsetlinewidth{0.803000pt}%
\definecolor{currentstroke}{rgb}{0.000000,0.000000,0.000000}%
\pgfsetstrokecolor{currentstroke}%
\pgfsetdash{}{0pt}%
\pgfsys@defobject{currentmarker}{\pgfqpoint{0.000000in}{-0.048611in}}{\pgfqpoint{0.000000in}{0.000000in}}{%
\pgfpathmoveto{\pgfqpoint{0.000000in}{0.000000in}}%
\pgfpathlineto{\pgfqpoint{0.000000in}{-0.048611in}}%
\pgfusepath{stroke,fill}%
}%
\begin{pgfscope}%
\pgfsys@transformshift{3.540000in}{1.406000in}%
\pgfsys@useobject{currentmarker}{}%
\end{pgfscope}%
\end{pgfscope}%
\begin{pgfscope}%
\definecolor{textcolor}{rgb}{0.000000,0.000000,0.000000}%
\pgfsetstrokecolor{textcolor}%
\pgfsetfillcolor{textcolor}%
\pgftext[x=3.540000in,y=1.308778in,,top]{\color{textcolor}\sffamily\fontsize{10.000000}{12.000000}\selectfont \(\displaystyle {0}\)}%
\end{pgfscope}%
\begin{pgfscope}%
\pgfsetbuttcap%
\pgfsetroundjoin%
\definecolor{currentfill}{rgb}{0.000000,0.000000,0.000000}%
\pgfsetfillcolor{currentfill}%
\pgfsetlinewidth{0.803000pt}%
\definecolor{currentstroke}{rgb}{0.000000,0.000000,0.000000}%
\pgfsetstrokecolor{currentstroke}%
\pgfsetdash{}{0pt}%
\pgfsys@defobject{currentmarker}{\pgfqpoint{0.000000in}{-0.048611in}}{\pgfqpoint{0.000000in}{0.000000in}}{%
\pgfpathmoveto{\pgfqpoint{0.000000in}{0.000000in}}%
\pgfpathlineto{\pgfqpoint{0.000000in}{-0.048611in}}%
\pgfusepath{stroke,fill}%
}%
\begin{pgfscope}%
\pgfsys@transformshift{4.272186in}{1.406000in}%
\pgfsys@useobject{currentmarker}{}%
\end{pgfscope}%
\end{pgfscope}%
\begin{pgfscope}%
\definecolor{textcolor}{rgb}{0.000000,0.000000,0.000000}%
\pgfsetstrokecolor{textcolor}%
\pgfsetfillcolor{textcolor}%
\pgftext[x=4.272186in,y=1.308778in,,top]{\color{textcolor}\sffamily\fontsize{10.000000}{12.000000}\selectfont \(\displaystyle {5}\)}%
\end{pgfscope}%
\begin{pgfscope}%
\pgfsetbuttcap%
\pgfsetroundjoin%
\definecolor{currentfill}{rgb}{0.000000,0.000000,0.000000}%
\pgfsetfillcolor{currentfill}%
\pgfsetlinewidth{0.803000pt}%
\definecolor{currentstroke}{rgb}{0.000000,0.000000,0.000000}%
\pgfsetstrokecolor{currentstroke}%
\pgfsetdash{}{0pt}%
\pgfsys@defobject{currentmarker}{\pgfqpoint{0.000000in}{-0.048611in}}{\pgfqpoint{0.000000in}{0.000000in}}{%
\pgfpathmoveto{\pgfqpoint{0.000000in}{0.000000in}}%
\pgfpathlineto{\pgfqpoint{0.000000in}{-0.048611in}}%
\pgfusepath{stroke,fill}%
}%
\begin{pgfscope}%
\pgfsys@transformshift{5.004371in}{1.406000in}%
\pgfsys@useobject{currentmarker}{}%
\end{pgfscope}%
\end{pgfscope}%
\begin{pgfscope}%
\definecolor{textcolor}{rgb}{0.000000,0.000000,0.000000}%
\pgfsetstrokecolor{textcolor}%
\pgfsetfillcolor{textcolor}%
\pgftext[x=5.004371in,y=1.308778in,,top]{\color{textcolor}\sffamily\fontsize{10.000000}{12.000000}\selectfont \(\displaystyle {10}\)}%
\end{pgfscope}%
\begin{pgfscope}%
\definecolor{textcolor}{rgb}{0.000000,0.000000,0.000000}%
\pgfsetstrokecolor{textcolor}%
\pgfsetfillcolor{textcolor}%
\pgftext[x=3.075000in,y=1.129766in,,top]{\color{textcolor}\sffamily\fontsize{10.000000}{12.000000}\selectfont \(\displaystyle \zeta \, \mathrm{[\mu m]}\)}%
\end{pgfscope}%
\begin{pgfscope}%
\pgfsetbuttcap%
\pgfsetroundjoin%
\definecolor{currentfill}{rgb}{0.000000,0.000000,0.000000}%
\pgfsetfillcolor{currentfill}%
\pgfsetlinewidth{0.803000pt}%
\definecolor{currentstroke}{rgb}{0.000000,0.000000,0.000000}%
\pgfsetstrokecolor{currentstroke}%
\pgfsetdash{}{0pt}%
\pgfsys@defobject{currentmarker}{\pgfqpoint{-0.048611in}{0.000000in}}{\pgfqpoint{-0.000000in}{0.000000in}}{%
\pgfpathmoveto{\pgfqpoint{-0.000000in}{0.000000in}}%
\pgfpathlineto{\pgfqpoint{-0.048611in}{0.000000in}}%
\pgfusepath{stroke,fill}%
}%
\begin{pgfscope}%
\pgfsys@transformshift{0.750000in}{1.458301in}%
\pgfsys@useobject{currentmarker}{}%
\end{pgfscope}%
\end{pgfscope}%
\begin{pgfscope}%
\definecolor{textcolor}{rgb}{0.000000,0.000000,0.000000}%
\pgfsetstrokecolor{textcolor}%
\pgfsetfillcolor{textcolor}%
\pgftext[x=0.583333in, y=1.410076in, left, base]{\color{textcolor}\sffamily\fontsize{10.000000}{12.000000}\selectfont \(\displaystyle {0}\)}%
\end{pgfscope}%
\begin{pgfscope}%
\pgfsetbuttcap%
\pgfsetroundjoin%
\definecolor{currentfill}{rgb}{0.000000,0.000000,0.000000}%
\pgfsetfillcolor{currentfill}%
\pgfsetlinewidth{0.803000pt}%
\definecolor{currentstroke}{rgb}{0.000000,0.000000,0.000000}%
\pgfsetstrokecolor{currentstroke}%
\pgfsetdash{}{0pt}%
\pgfsys@defobject{currentmarker}{\pgfqpoint{-0.048611in}{0.000000in}}{\pgfqpoint{-0.000000in}{0.000000in}}{%
\pgfpathmoveto{\pgfqpoint{-0.000000in}{0.000000in}}%
\pgfpathlineto{\pgfqpoint{-0.048611in}{0.000000in}}%
\pgfusepath{stroke,fill}%
}%
\begin{pgfscope}%
\pgfsys@transformshift{0.750000in}{1.790601in}%
\pgfsys@useobject{currentmarker}{}%
\end{pgfscope}%
\end{pgfscope}%
\begin{pgfscope}%
\definecolor{textcolor}{rgb}{0.000000,0.000000,0.000000}%
\pgfsetstrokecolor{textcolor}%
\pgfsetfillcolor{textcolor}%
\pgftext[x=0.583333in, y=1.742376in, left, base]{\color{textcolor}\sffamily\fontsize{10.000000}{12.000000}\selectfont \(\displaystyle {1}\)}%
\end{pgfscope}%
\begin{pgfscope}%
\pgfsetbuttcap%
\pgfsetroundjoin%
\definecolor{currentfill}{rgb}{0.000000,0.000000,0.000000}%
\pgfsetfillcolor{currentfill}%
\pgfsetlinewidth{0.803000pt}%
\definecolor{currentstroke}{rgb}{0.000000,0.000000,0.000000}%
\pgfsetstrokecolor{currentstroke}%
\pgfsetdash{}{0pt}%
\pgfsys@defobject{currentmarker}{\pgfqpoint{-0.048611in}{0.000000in}}{\pgfqpoint{-0.000000in}{0.000000in}}{%
\pgfpathmoveto{\pgfqpoint{-0.000000in}{0.000000in}}%
\pgfpathlineto{\pgfqpoint{-0.048611in}{0.000000in}}%
\pgfusepath{stroke,fill}%
}%
\begin{pgfscope}%
\pgfsys@transformshift{0.750000in}{2.122902in}%
\pgfsys@useobject{currentmarker}{}%
\end{pgfscope}%
\end{pgfscope}%
\begin{pgfscope}%
\definecolor{textcolor}{rgb}{0.000000,0.000000,0.000000}%
\pgfsetstrokecolor{textcolor}%
\pgfsetfillcolor{textcolor}%
\pgftext[x=0.583333in, y=2.074677in, left, base]{\color{textcolor}\sffamily\fontsize{10.000000}{12.000000}\selectfont \(\displaystyle {2}\)}%
\end{pgfscope}%
\begin{pgfscope}%
\pgfsetbuttcap%
\pgfsetroundjoin%
\definecolor{currentfill}{rgb}{0.000000,0.000000,0.000000}%
\pgfsetfillcolor{currentfill}%
\pgfsetlinewidth{0.803000pt}%
\definecolor{currentstroke}{rgb}{0.000000,0.000000,0.000000}%
\pgfsetstrokecolor{currentstroke}%
\pgfsetdash{}{0pt}%
\pgfsys@defobject{currentmarker}{\pgfqpoint{-0.048611in}{0.000000in}}{\pgfqpoint{-0.000000in}{0.000000in}}{%
\pgfpathmoveto{\pgfqpoint{-0.000000in}{0.000000in}}%
\pgfpathlineto{\pgfqpoint{-0.048611in}{0.000000in}}%
\pgfusepath{stroke,fill}%
}%
\begin{pgfscope}%
\pgfsys@transformshift{0.750000in}{2.455202in}%
\pgfsys@useobject{currentmarker}{}%
\end{pgfscope}%
\end{pgfscope}%
\begin{pgfscope}%
\definecolor{textcolor}{rgb}{0.000000,0.000000,0.000000}%
\pgfsetstrokecolor{textcolor}%
\pgfsetfillcolor{textcolor}%
\pgftext[x=0.583333in, y=2.406977in, left, base]{\color{textcolor}\sffamily\fontsize{10.000000}{12.000000}\selectfont \(\displaystyle {3}\)}%
\end{pgfscope}%
\begin{pgfscope}%
\pgfsetbuttcap%
\pgfsetroundjoin%
\definecolor{currentfill}{rgb}{0.000000,0.000000,0.000000}%
\pgfsetfillcolor{currentfill}%
\pgfsetlinewidth{0.803000pt}%
\definecolor{currentstroke}{rgb}{0.000000,0.000000,0.000000}%
\pgfsetstrokecolor{currentstroke}%
\pgfsetdash{}{0pt}%
\pgfsys@defobject{currentmarker}{\pgfqpoint{-0.048611in}{0.000000in}}{\pgfqpoint{-0.000000in}{0.000000in}}{%
\pgfpathmoveto{\pgfqpoint{-0.000000in}{0.000000in}}%
\pgfpathlineto{\pgfqpoint{-0.048611in}{0.000000in}}%
\pgfusepath{stroke,fill}%
}%
\begin{pgfscope}%
\pgfsys@transformshift{0.750000in}{2.787503in}%
\pgfsys@useobject{currentmarker}{}%
\end{pgfscope}%
\end{pgfscope}%
\begin{pgfscope}%
\definecolor{textcolor}{rgb}{0.000000,0.000000,0.000000}%
\pgfsetstrokecolor{textcolor}%
\pgfsetfillcolor{textcolor}%
\pgftext[x=0.583333in, y=2.739277in, left, base]{\color{textcolor}\sffamily\fontsize{10.000000}{12.000000}\selectfont \(\displaystyle {4}\)}%
\end{pgfscope}%
\begin{pgfscope}%
\pgfsetbuttcap%
\pgfsetroundjoin%
\definecolor{currentfill}{rgb}{0.000000,0.000000,0.000000}%
\pgfsetfillcolor{currentfill}%
\pgfsetlinewidth{0.803000pt}%
\definecolor{currentstroke}{rgb}{0.000000,0.000000,0.000000}%
\pgfsetstrokecolor{currentstroke}%
\pgfsetdash{}{0pt}%
\pgfsys@defobject{currentmarker}{\pgfqpoint{-0.048611in}{0.000000in}}{\pgfqpoint{-0.000000in}{0.000000in}}{%
\pgfpathmoveto{\pgfqpoint{-0.000000in}{0.000000in}}%
\pgfpathlineto{\pgfqpoint{-0.048611in}{0.000000in}}%
\pgfusepath{stroke,fill}%
}%
\begin{pgfscope}%
\pgfsys@transformshift{0.750000in}{3.119803in}%
\pgfsys@useobject{currentmarker}{}%
\end{pgfscope}%
\end{pgfscope}%
\begin{pgfscope}%
\definecolor{textcolor}{rgb}{0.000000,0.000000,0.000000}%
\pgfsetstrokecolor{textcolor}%
\pgfsetfillcolor{textcolor}%
\pgftext[x=0.583333in, y=3.071578in, left, base]{\color{textcolor}\sffamily\fontsize{10.000000}{12.000000}\selectfont \(\displaystyle {5}\)}%
\end{pgfscope}%
\begin{pgfscope}%
\pgfsetbuttcap%
\pgfsetroundjoin%
\definecolor{currentfill}{rgb}{0.000000,0.000000,0.000000}%
\pgfsetfillcolor{currentfill}%
\pgfsetlinewidth{0.803000pt}%
\definecolor{currentstroke}{rgb}{0.000000,0.000000,0.000000}%
\pgfsetstrokecolor{currentstroke}%
\pgfsetdash{}{0pt}%
\pgfsys@defobject{currentmarker}{\pgfqpoint{-0.048611in}{0.000000in}}{\pgfqpoint{-0.000000in}{0.000000in}}{%
\pgfpathmoveto{\pgfqpoint{-0.000000in}{0.000000in}}%
\pgfpathlineto{\pgfqpoint{-0.048611in}{0.000000in}}%
\pgfusepath{stroke,fill}%
}%
\begin{pgfscope}%
\pgfsys@transformshift{0.750000in}{3.452104in}%
\pgfsys@useobject{currentmarker}{}%
\end{pgfscope}%
\end{pgfscope}%
\begin{pgfscope}%
\definecolor{textcolor}{rgb}{0.000000,0.000000,0.000000}%
\pgfsetstrokecolor{textcolor}%
\pgfsetfillcolor{textcolor}%
\pgftext[x=0.583333in, y=3.403878in, left, base]{\color{textcolor}\sffamily\fontsize{10.000000}{12.000000}\selectfont \(\displaystyle {6}\)}%
\end{pgfscope}%
\begin{pgfscope}%
\definecolor{textcolor}{rgb}{0.000000,0.000000,0.000000}%
\pgfsetstrokecolor{textcolor}%
\pgfsetfillcolor{textcolor}%
\pgftext[x=0.527778in,y=2.463000in,,bottom,rotate=90.000000]{\color{textcolor}\sffamily\fontsize{10.000000}{12.000000}\selectfont \(\displaystyle y \, \mathrm{[mm]}\)}%
\end{pgfscope}%
\begin{pgfscope}%
\pgfsetrectcap%
\pgfsetmiterjoin%
\pgfsetlinewidth{0.803000pt}%
\definecolor{currentstroke}{rgb}{0.000000,0.000000,0.000000}%
\pgfsetstrokecolor{currentstroke}%
\pgfsetdash{}{0pt}%
\pgfpathmoveto{\pgfqpoint{0.750000in}{1.406000in}}%
\pgfpathlineto{\pgfqpoint{0.750000in}{3.520000in}}%
\pgfusepath{stroke}%
\end{pgfscope}%
\begin{pgfscope}%
\pgfsetrectcap%
\pgfsetmiterjoin%
\pgfsetlinewidth{0.803000pt}%
\definecolor{currentstroke}{rgb}{0.000000,0.000000,0.000000}%
\pgfsetstrokecolor{currentstroke}%
\pgfsetdash{}{0pt}%
\pgfpathmoveto{\pgfqpoint{5.400000in}{1.406000in}}%
\pgfpathlineto{\pgfqpoint{5.400000in}{3.520000in}}%
\pgfusepath{stroke}%
\end{pgfscope}%
\begin{pgfscope}%
\pgfsetrectcap%
\pgfsetmiterjoin%
\pgfsetlinewidth{0.803000pt}%
\definecolor{currentstroke}{rgb}{0.000000,0.000000,0.000000}%
\pgfsetstrokecolor{currentstroke}%
\pgfsetdash{}{0pt}%
\pgfpathmoveto{\pgfqpoint{0.750000in}{1.406000in}}%
\pgfpathlineto{\pgfqpoint{5.400000in}{1.406000in}}%
\pgfusepath{stroke}%
\end{pgfscope}%
\begin{pgfscope}%
\pgfsetrectcap%
\pgfsetmiterjoin%
\pgfsetlinewidth{0.803000pt}%
\definecolor{currentstroke}{rgb}{0.000000,0.000000,0.000000}%
\pgfsetstrokecolor{currentstroke}%
\pgfsetdash{}{0pt}%
\pgfpathmoveto{\pgfqpoint{0.750000in}{3.520000in}}%
\pgfpathlineto{\pgfqpoint{5.400000in}{3.520000in}}%
\pgfusepath{stroke}%
\end{pgfscope}%
\begin{pgfscope}%
\pgfsetbuttcap%
\pgfsetmiterjoin%
\definecolor{currentfill}{rgb}{1.000000,1.000000,1.000000}%
\pgfsetfillcolor{currentfill}%
\pgfsetlinewidth{0.000000pt}%
\definecolor{currentstroke}{rgb}{0.000000,0.000000,0.000000}%
\pgfsetstrokecolor{currentstroke}%
\pgfsetstrokeopacity{0.000000}%
\pgfsetdash{}{0pt}%
\pgfpathmoveto{\pgfqpoint{0.750000in}{0.720500in}}%
\pgfpathlineto{\pgfqpoint{5.400000in}{0.720500in}}%
\pgfpathlineto{\pgfqpoint{5.400000in}{0.953000in}}%
\pgfpathlineto{\pgfqpoint{0.750000in}{0.953000in}}%
\pgfpathlineto{\pgfqpoint{0.750000in}{0.720500in}}%
\pgfpathclose%
\pgfusepath{fill}%
\end{pgfscope}%
\begin{pgfscope}%
\pgfpathrectangle{\pgfqpoint{0.750000in}{0.720500in}}{\pgfqpoint{4.650000in}{0.232500in}}%
\pgfusepath{clip}%
\pgfsetbuttcap%
\pgfsetmiterjoin%
\definecolor{currentfill}{rgb}{1.000000,1.000000,1.000000}%
\pgfsetfillcolor{currentfill}%
\pgfsetlinewidth{0.010037pt}%
\definecolor{currentstroke}{rgb}{1.000000,1.000000,1.000000}%
\pgfsetstrokecolor{currentstroke}%
\pgfsetdash{}{0pt}%
\pgfusepath{stroke,fill}%
\end{pgfscope}%
\begin{pgfscope}%
\pgfsys@transformshift{0.750000in}{0.722222in}%
\pgftext[left,bottom]{\includegraphics[interpolate=true,width=4.652778in,height=0.236111in]{y_hist_time-img1.png}}%
\end{pgfscope}%
\begin{pgfscope}%
\pgfsetbuttcap%
\pgfsetroundjoin%
\definecolor{currentfill}{rgb}{0.000000,0.000000,0.000000}%
\pgfsetfillcolor{currentfill}%
\pgfsetlinewidth{0.803000pt}%
\definecolor{currentstroke}{rgb}{0.000000,0.000000,0.000000}%
\pgfsetstrokecolor{currentstroke}%
\pgfsetdash{}{0pt}%
\pgfsys@defobject{currentmarker}{\pgfqpoint{0.000000in}{-0.048611in}}{\pgfqpoint{0.000000in}{0.000000in}}{%
\pgfpathmoveto{\pgfqpoint{0.000000in}{0.000000in}}%
\pgfpathlineto{\pgfqpoint{0.000000in}{-0.048611in}}%
\pgfusepath{stroke,fill}%
}%
\begin{pgfscope}%
\pgfsys@transformshift{1.343443in}{0.720500in}%
\pgfsys@useobject{currentmarker}{}%
\end{pgfscope}%
\end{pgfscope}%
\begin{pgfscope}%
\definecolor{textcolor}{rgb}{0.000000,0.000000,0.000000}%
\pgfsetstrokecolor{textcolor}%
\pgfsetfillcolor{textcolor}%
\pgftext[x=1.343443in,y=0.623278in,,top]{\color{textcolor}\sffamily\fontsize{10.000000}{12.000000}\selectfont \(\displaystyle {\ensuremath{-}15}\)}%
\end{pgfscope}%
\begin{pgfscope}%
\pgfsetbuttcap%
\pgfsetroundjoin%
\definecolor{currentfill}{rgb}{0.000000,0.000000,0.000000}%
\pgfsetfillcolor{currentfill}%
\pgfsetlinewidth{0.803000pt}%
\definecolor{currentstroke}{rgb}{0.000000,0.000000,0.000000}%
\pgfsetstrokecolor{currentstroke}%
\pgfsetdash{}{0pt}%
\pgfsys@defobject{currentmarker}{\pgfqpoint{0.000000in}{-0.048611in}}{\pgfqpoint{0.000000in}{0.000000in}}{%
\pgfpathmoveto{\pgfqpoint{0.000000in}{0.000000in}}%
\pgfpathlineto{\pgfqpoint{0.000000in}{-0.048611in}}%
\pgfusepath{stroke,fill}%
}%
\begin{pgfscope}%
\pgfsys@transformshift{2.075629in}{0.720500in}%
\pgfsys@useobject{currentmarker}{}%
\end{pgfscope}%
\end{pgfscope}%
\begin{pgfscope}%
\definecolor{textcolor}{rgb}{0.000000,0.000000,0.000000}%
\pgfsetstrokecolor{textcolor}%
\pgfsetfillcolor{textcolor}%
\pgftext[x=2.075629in,y=0.623278in,,top]{\color{textcolor}\sffamily\fontsize{10.000000}{12.000000}\selectfont \(\displaystyle {\ensuremath{-}10}\)}%
\end{pgfscope}%
\begin{pgfscope}%
\pgfsetbuttcap%
\pgfsetroundjoin%
\definecolor{currentfill}{rgb}{0.000000,0.000000,0.000000}%
\pgfsetfillcolor{currentfill}%
\pgfsetlinewidth{0.803000pt}%
\definecolor{currentstroke}{rgb}{0.000000,0.000000,0.000000}%
\pgfsetstrokecolor{currentstroke}%
\pgfsetdash{}{0pt}%
\pgfsys@defobject{currentmarker}{\pgfqpoint{0.000000in}{-0.048611in}}{\pgfqpoint{0.000000in}{0.000000in}}{%
\pgfpathmoveto{\pgfqpoint{0.000000in}{0.000000in}}%
\pgfpathlineto{\pgfqpoint{0.000000in}{-0.048611in}}%
\pgfusepath{stroke,fill}%
}%
\begin{pgfscope}%
\pgfsys@transformshift{2.807814in}{0.720500in}%
\pgfsys@useobject{currentmarker}{}%
\end{pgfscope}%
\end{pgfscope}%
\begin{pgfscope}%
\definecolor{textcolor}{rgb}{0.000000,0.000000,0.000000}%
\pgfsetstrokecolor{textcolor}%
\pgfsetfillcolor{textcolor}%
\pgftext[x=2.807814in,y=0.623278in,,top]{\color{textcolor}\sffamily\fontsize{10.000000}{12.000000}\selectfont \(\displaystyle {\ensuremath{-}5}\)}%
\end{pgfscope}%
\begin{pgfscope}%
\pgfsetbuttcap%
\pgfsetroundjoin%
\definecolor{currentfill}{rgb}{0.000000,0.000000,0.000000}%
\pgfsetfillcolor{currentfill}%
\pgfsetlinewidth{0.803000pt}%
\definecolor{currentstroke}{rgb}{0.000000,0.000000,0.000000}%
\pgfsetstrokecolor{currentstroke}%
\pgfsetdash{}{0pt}%
\pgfsys@defobject{currentmarker}{\pgfqpoint{0.000000in}{-0.048611in}}{\pgfqpoint{0.000000in}{0.000000in}}{%
\pgfpathmoveto{\pgfqpoint{0.000000in}{0.000000in}}%
\pgfpathlineto{\pgfqpoint{0.000000in}{-0.048611in}}%
\pgfusepath{stroke,fill}%
}%
\begin{pgfscope}%
\pgfsys@transformshift{3.540000in}{0.720500in}%
\pgfsys@useobject{currentmarker}{}%
\end{pgfscope}%
\end{pgfscope}%
\begin{pgfscope}%
\definecolor{textcolor}{rgb}{0.000000,0.000000,0.000000}%
\pgfsetstrokecolor{textcolor}%
\pgfsetfillcolor{textcolor}%
\pgftext[x=3.540000in,y=0.623278in,,top]{\color{textcolor}\sffamily\fontsize{10.000000}{12.000000}\selectfont \(\displaystyle {0}\)}%
\end{pgfscope}%
\begin{pgfscope}%
\pgfsetbuttcap%
\pgfsetroundjoin%
\definecolor{currentfill}{rgb}{0.000000,0.000000,0.000000}%
\pgfsetfillcolor{currentfill}%
\pgfsetlinewidth{0.803000pt}%
\definecolor{currentstroke}{rgb}{0.000000,0.000000,0.000000}%
\pgfsetstrokecolor{currentstroke}%
\pgfsetdash{}{0pt}%
\pgfsys@defobject{currentmarker}{\pgfqpoint{0.000000in}{-0.048611in}}{\pgfqpoint{0.000000in}{0.000000in}}{%
\pgfpathmoveto{\pgfqpoint{0.000000in}{0.000000in}}%
\pgfpathlineto{\pgfqpoint{0.000000in}{-0.048611in}}%
\pgfusepath{stroke,fill}%
}%
\begin{pgfscope}%
\pgfsys@transformshift{4.272186in}{0.720500in}%
\pgfsys@useobject{currentmarker}{}%
\end{pgfscope}%
\end{pgfscope}%
\begin{pgfscope}%
\definecolor{textcolor}{rgb}{0.000000,0.000000,0.000000}%
\pgfsetstrokecolor{textcolor}%
\pgfsetfillcolor{textcolor}%
\pgftext[x=4.272186in,y=0.623278in,,top]{\color{textcolor}\sffamily\fontsize{10.000000}{12.000000}\selectfont \(\displaystyle {5}\)}%
\end{pgfscope}%
\begin{pgfscope}%
\pgfsetbuttcap%
\pgfsetroundjoin%
\definecolor{currentfill}{rgb}{0.000000,0.000000,0.000000}%
\pgfsetfillcolor{currentfill}%
\pgfsetlinewidth{0.803000pt}%
\definecolor{currentstroke}{rgb}{0.000000,0.000000,0.000000}%
\pgfsetstrokecolor{currentstroke}%
\pgfsetdash{}{0pt}%
\pgfsys@defobject{currentmarker}{\pgfqpoint{0.000000in}{-0.048611in}}{\pgfqpoint{0.000000in}{0.000000in}}{%
\pgfpathmoveto{\pgfqpoint{0.000000in}{0.000000in}}%
\pgfpathlineto{\pgfqpoint{0.000000in}{-0.048611in}}%
\pgfusepath{stroke,fill}%
}%
\begin{pgfscope}%
\pgfsys@transformshift{5.004371in}{0.720500in}%
\pgfsys@useobject{currentmarker}{}%
\end{pgfscope}%
\end{pgfscope}%
\begin{pgfscope}%
\definecolor{textcolor}{rgb}{0.000000,0.000000,0.000000}%
\pgfsetstrokecolor{textcolor}%
\pgfsetfillcolor{textcolor}%
\pgftext[x=5.004371in,y=0.623278in,,top]{\color{textcolor}\sffamily\fontsize{10.000000}{12.000000}\selectfont \(\displaystyle {10}\)}%
\end{pgfscope}%
\begin{pgfscope}%
\definecolor{textcolor}{rgb}{0.000000,0.000000,0.000000}%
\pgfsetstrokecolor{textcolor}%
\pgfsetfillcolor{textcolor}%
\pgftext[x=3.075000in,y=0.444266in,,top]{\color{textcolor}\sffamily\fontsize{10.000000}{12.000000}\selectfont \(\displaystyle \zeta_{end} \, \mathrm{[\mu m]}\)}%
\end{pgfscope}%
\begin{pgfscope}%
\pgfsetrectcap%
\pgfsetmiterjoin%
\pgfsetlinewidth{0.803000pt}%
\definecolor{currentstroke}{rgb}{0.000000,0.000000,0.000000}%
\pgfsetstrokecolor{currentstroke}%
\pgfsetdash{}{0pt}%
\pgfpathmoveto{\pgfqpoint{0.750000in}{0.720500in}}%
\pgfpathlineto{\pgfqpoint{0.750000in}{0.836750in}}%
\pgfpathlineto{\pgfqpoint{0.750000in}{0.953000in}}%
\pgfpathlineto{\pgfqpoint{5.400000in}{0.953000in}}%
\pgfpathlineto{\pgfqpoint{5.400000in}{0.836750in}}%
\pgfpathlineto{\pgfqpoint{5.400000in}{0.720500in}}%
\pgfpathlineto{\pgfqpoint{0.750000in}{0.720500in}}%
\pgfpathclose%
\pgfusepath{stroke}%
\end{pgfscope}%
\end{pgfpicture}%
\makeatother%
\endgroup%

	\caption{Position of the particles in bins after bunch breakup. Each bin shows the minimal $\zeta$-position (co-moving position) of the particles in it to visualize how it will move over time.
	\textbf{a)} 2D distribution of the bins for $y=$ \qty{0.60}{\mm}. The blue particles in the middle will fall back to the backside of the driver with time.
	\textbf{b)} The bins centered at $z=0$ are sampled every 2000 timesteps and plotted over the traversed distance $y$ in the plasma. Around $y=$ \qty{3}{mm} the particles in the middle falls behind.}
	\label{fig:y_hist_time}
\end{figure}
The colors show the co-moving $\zeta$-position of the respective particles after the bunch breakup. Driver electrons which fall behind during the bunch breakup originate from a small region right behind the center of the bunch. 
As seen in \autoref{fig:q_series}, this is the region where the strongest Lorentz force acted.
The $\zeta$ position change of other particles is neglectable small, as only weak forces act.

Additionally in \autoref{fig:y_hist_time}b the middle slice of this distribution was captured every 2000 timesteps and plotted over the traversed distance $y$. The fallback of the particles in the middle during bunch breakup can clearly be seen.

The same tracking was done for the $z$-direction, seen in the time series in \autoref{fig:z_hist_0}.
\begin{figure}
	\centering
	\input{../images/z_hist_0.pgf}
	\caption{$z$-positions of the particles before entering the plasma plotted over their current position ($y=$ \qty{1.31}{mm}). Each bin represents the mean $z$-value at a timestep. In the back of the driver, 
	particles from below and above $z=0$ alternate. }
	\label{fig:z_hist_0}
\end{figure}
As the movement in z-direction is symmetric, this results in bins with equal number of particles from the top and bottom half to be displayed as yellowish white, thus the transition of particles from top to bottom and reversed can not be seen. Still, the effect is clear as the wings can clearly be seen at the end of the driver.
The alternating colors can be explained by the oscillation of the electrons, as they get pulled back to the center and form smaller wings. No such motion is observed in the front part in the driver.


\subsection{Parameter comparison}
The transformation of the driver beam shape and subsequent changes of quality of the wakefield were compared for different initial parameters. To quantify the wakefield quality, the maximally gained energy 
for a potential witness beam is considered. In \autoref{fig:E_y_hist} the charge distribution and longituduinal direction $E_{\parallel}$ (accelerating and decelerating) of the electric field of the plasma electrons are compared over time. 
\begin{figure}
	\centering
	%% Creator: Matplotlib, PGF backend
%%
%% To include the figure in your LaTeX document, write
%%   \input{<filename>.pgf}
%%
%% Make sure the required packages are loaded in your preamble
%%   \usepackage{pgf}
%%
%% Also ensure that all the required font packages are loaded; for instance,
%% the lmodern package is sometimes necessary when using math font.
%%   \usepackage{lmodern}
%%
%% Figures using additional raster images can only be included by \input if
%% they are in the same directory as the main LaTeX file. For loading figures
%% from other directories you can use the `import` package
%%   \usepackage{import}
%%
%% and then include the figures with
%%   \import{<path to file>}{<filename>.pgf}
%%
%% Matplotlib used the following preamble
%%
\begingroup%
\makeatletter%
\begin{pgfpicture}%
\pgfpathrectangle{\pgfpointorigin}{\pgfqpoint{6.400000in}{7.000000in}}%
\pgfusepath{use as bounding box, clip}%
\begin{pgfscope}%
\pgfsetbuttcap%
\pgfsetmiterjoin%
\pgfsetlinewidth{0.000000pt}%
\definecolor{currentstroke}{rgb}{1.000000,1.000000,1.000000}%
\pgfsetstrokecolor{currentstroke}%
\pgfsetstrokeopacity{0.000000}%
\pgfsetdash{}{0pt}%
\pgfpathmoveto{\pgfqpoint{0.000000in}{0.000000in}}%
\pgfpathlineto{\pgfqpoint{6.400000in}{0.000000in}}%
\pgfpathlineto{\pgfqpoint{6.400000in}{7.000000in}}%
\pgfpathlineto{\pgfqpoint{0.000000in}{7.000000in}}%
\pgfpathlineto{\pgfqpoint{0.000000in}{0.000000in}}%
\pgfpathclose%
\pgfusepath{}%
\end{pgfscope}%
\begin{pgfscope}%
\pgfsetbuttcap%
\pgfsetmiterjoin%
\definecolor{currentfill}{rgb}{1.000000,1.000000,1.000000}%
\pgfsetfillcolor{currentfill}%
\pgfsetlinewidth{0.000000pt}%
\definecolor{currentstroke}{rgb}{0.000000,0.000000,0.000000}%
\pgfsetstrokecolor{currentstroke}%
\pgfsetstrokeopacity{0.000000}%
\pgfsetdash{}{0pt}%
\pgfpathmoveto{\pgfqpoint{0.800000in}{3.862174in}}%
\pgfpathlineto{\pgfqpoint{4.672000in}{3.862174in}}%
\pgfpathlineto{\pgfqpoint{4.672000in}{6.160000in}}%
\pgfpathlineto{\pgfqpoint{0.800000in}{6.160000in}}%
\pgfpathlineto{\pgfqpoint{0.800000in}{3.862174in}}%
\pgfpathclose%
\pgfusepath{fill}%
\end{pgfscope}%
\begin{pgfscope}%
\pgfpathrectangle{\pgfqpoint{0.800000in}{3.862174in}}{\pgfqpoint{3.872000in}{2.297826in}}%
\pgfusepath{clip}%
\pgfsys@transformcm{3.875000}{0.000000}{0.000000}{2.305556}{0.800000in}{3.862174in}%
\pgftext[left,bottom]{\includegraphics[interpolate=false,width=1.000000in,height=1.000000in]{E_y_hist-img0.png}}%
\end{pgfscope}%
\begin{pgfscope}%
\pgfsetbuttcap%
\pgfsetroundjoin%
\definecolor{currentfill}{rgb}{0.000000,0.000000,0.000000}%
\pgfsetfillcolor{currentfill}%
\pgfsetlinewidth{0.803000pt}%
\definecolor{currentstroke}{rgb}{0.000000,0.000000,0.000000}%
\pgfsetstrokecolor{currentstroke}%
\pgfsetdash{}{0pt}%
\pgfsys@defobject{currentmarker}{\pgfqpoint{0.000000in}{-0.048611in}}{\pgfqpoint{0.000000in}{0.000000in}}{%
\pgfpathmoveto{\pgfqpoint{0.000000in}{0.000000in}}%
\pgfpathlineto{\pgfqpoint{0.000000in}{-0.048611in}}%
\pgfusepath{stroke,fill}%
}%
\begin{pgfscope}%
\pgfsys@transformshift{1.384464in}{3.862174in}%
\pgfsys@useobject{currentmarker}{}%
\end{pgfscope}%
\end{pgfscope}%
\begin{pgfscope}%
\definecolor{textcolor}{rgb}{0.000000,0.000000,0.000000}%
\pgfsetstrokecolor{textcolor}%
\pgfsetfillcolor{textcolor}%
\pgftext[x=1.384464in,y=3.764952in,,top]{\color{textcolor}\sffamily\fontsize{10.000000}{12.000000}\selectfont \(\displaystyle {\ensuremath{-}30}\)}%
\end{pgfscope}%
\begin{pgfscope}%
\pgfsetbuttcap%
\pgfsetroundjoin%
\definecolor{currentfill}{rgb}{0.000000,0.000000,0.000000}%
\pgfsetfillcolor{currentfill}%
\pgfsetlinewidth{0.803000pt}%
\definecolor{currentstroke}{rgb}{0.000000,0.000000,0.000000}%
\pgfsetstrokecolor{currentstroke}%
\pgfsetdash{}{0pt}%
\pgfsys@defobject{currentmarker}{\pgfqpoint{0.000000in}{-0.048611in}}{\pgfqpoint{0.000000in}{0.000000in}}{%
\pgfpathmoveto{\pgfqpoint{0.000000in}{0.000000in}}%
\pgfpathlineto{\pgfqpoint{0.000000in}{-0.048611in}}%
\pgfusepath{stroke,fill}%
}%
\begin{pgfscope}%
\pgfsys@transformshift{2.112831in}{3.862174in}%
\pgfsys@useobject{currentmarker}{}%
\end{pgfscope}%
\end{pgfscope}%
\begin{pgfscope}%
\definecolor{textcolor}{rgb}{0.000000,0.000000,0.000000}%
\pgfsetstrokecolor{textcolor}%
\pgfsetfillcolor{textcolor}%
\pgftext[x=2.112831in,y=3.764952in,,top]{\color{textcolor}\sffamily\fontsize{10.000000}{12.000000}\selectfont \(\displaystyle {\ensuremath{-}20}\)}%
\end{pgfscope}%
\begin{pgfscope}%
\pgfsetbuttcap%
\pgfsetroundjoin%
\definecolor{currentfill}{rgb}{0.000000,0.000000,0.000000}%
\pgfsetfillcolor{currentfill}%
\pgfsetlinewidth{0.803000pt}%
\definecolor{currentstroke}{rgb}{0.000000,0.000000,0.000000}%
\pgfsetstrokecolor{currentstroke}%
\pgfsetdash{}{0pt}%
\pgfsys@defobject{currentmarker}{\pgfqpoint{0.000000in}{-0.048611in}}{\pgfqpoint{0.000000in}{0.000000in}}{%
\pgfpathmoveto{\pgfqpoint{0.000000in}{0.000000in}}%
\pgfpathlineto{\pgfqpoint{0.000000in}{-0.048611in}}%
\pgfusepath{stroke,fill}%
}%
\begin{pgfscope}%
\pgfsys@transformshift{2.841198in}{3.862174in}%
\pgfsys@useobject{currentmarker}{}%
\end{pgfscope}%
\end{pgfscope}%
\begin{pgfscope}%
\definecolor{textcolor}{rgb}{0.000000,0.000000,0.000000}%
\pgfsetstrokecolor{textcolor}%
\pgfsetfillcolor{textcolor}%
\pgftext[x=2.841198in,y=3.764952in,,top]{\color{textcolor}\sffamily\fontsize{10.000000}{12.000000}\selectfont \(\displaystyle {\ensuremath{-}10}\)}%
\end{pgfscope}%
\begin{pgfscope}%
\pgfsetbuttcap%
\pgfsetroundjoin%
\definecolor{currentfill}{rgb}{0.000000,0.000000,0.000000}%
\pgfsetfillcolor{currentfill}%
\pgfsetlinewidth{0.803000pt}%
\definecolor{currentstroke}{rgb}{0.000000,0.000000,0.000000}%
\pgfsetstrokecolor{currentstroke}%
\pgfsetdash{}{0pt}%
\pgfsys@defobject{currentmarker}{\pgfqpoint{0.000000in}{-0.048611in}}{\pgfqpoint{0.000000in}{0.000000in}}{%
\pgfpathmoveto{\pgfqpoint{0.000000in}{0.000000in}}%
\pgfpathlineto{\pgfqpoint{0.000000in}{-0.048611in}}%
\pgfusepath{stroke,fill}%
}%
\begin{pgfscope}%
\pgfsys@transformshift{3.569565in}{3.862174in}%
\pgfsys@useobject{currentmarker}{}%
\end{pgfscope}%
\end{pgfscope}%
\begin{pgfscope}%
\definecolor{textcolor}{rgb}{0.000000,0.000000,0.000000}%
\pgfsetstrokecolor{textcolor}%
\pgfsetfillcolor{textcolor}%
\pgftext[x=3.569565in,y=3.764952in,,top]{\color{textcolor}\sffamily\fontsize{10.000000}{12.000000}\selectfont \(\displaystyle {0}\)}%
\end{pgfscope}%
\begin{pgfscope}%
\pgfsetbuttcap%
\pgfsetroundjoin%
\definecolor{currentfill}{rgb}{0.000000,0.000000,0.000000}%
\pgfsetfillcolor{currentfill}%
\pgfsetlinewidth{0.803000pt}%
\definecolor{currentstroke}{rgb}{0.000000,0.000000,0.000000}%
\pgfsetstrokecolor{currentstroke}%
\pgfsetdash{}{0pt}%
\pgfsys@defobject{currentmarker}{\pgfqpoint{0.000000in}{-0.048611in}}{\pgfqpoint{0.000000in}{0.000000in}}{%
\pgfpathmoveto{\pgfqpoint{0.000000in}{0.000000in}}%
\pgfpathlineto{\pgfqpoint{0.000000in}{-0.048611in}}%
\pgfusepath{stroke,fill}%
}%
\begin{pgfscope}%
\pgfsys@transformshift{4.297932in}{3.862174in}%
\pgfsys@useobject{currentmarker}{}%
\end{pgfscope}%
\end{pgfscope}%
\begin{pgfscope}%
\definecolor{textcolor}{rgb}{0.000000,0.000000,0.000000}%
\pgfsetstrokecolor{textcolor}%
\pgfsetfillcolor{textcolor}%
\pgftext[x=4.297932in,y=3.764952in,,top]{\color{textcolor}\sffamily\fontsize{10.000000}{12.000000}\selectfont \(\displaystyle {10}\)}%
\end{pgfscope}%
\begin{pgfscope}%
\definecolor{textcolor}{rgb}{0.000000,0.000000,0.000000}%
\pgfsetstrokecolor{textcolor}%
\pgfsetfillcolor{textcolor}%
\pgftext[x=2.736000in,y=3.585939in,,top]{\color{textcolor}\sffamily\fontsize{10.000000}{12.000000}\selectfont \(\displaystyle \zeta \, \mathrm{[\mu m]}\)}%
\end{pgfscope}%
\begin{pgfscope}%
\pgfsetbuttcap%
\pgfsetroundjoin%
\definecolor{currentfill}{rgb}{0.000000,0.000000,0.000000}%
\pgfsetfillcolor{currentfill}%
\pgfsetlinewidth{0.803000pt}%
\definecolor{currentstroke}{rgb}{0.000000,0.000000,0.000000}%
\pgfsetstrokecolor{currentstroke}%
\pgfsetdash{}{0pt}%
\pgfsys@defobject{currentmarker}{\pgfqpoint{-0.048611in}{0.000000in}}{\pgfqpoint{-0.000000in}{0.000000in}}{%
\pgfpathmoveto{\pgfqpoint{-0.000000in}{0.000000in}}%
\pgfpathlineto{\pgfqpoint{-0.048611in}{0.000000in}}%
\pgfusepath{stroke,fill}%
}%
\begin{pgfscope}%
\pgfsys@transformshift{0.800000in}{3.933384in}%
\pgfsys@useobject{currentmarker}{}%
\end{pgfscope}%
\end{pgfscope}%
\begin{pgfscope}%
\definecolor{textcolor}{rgb}{0.000000,0.000000,0.000000}%
\pgfsetstrokecolor{textcolor}%
\pgfsetfillcolor{textcolor}%
\pgftext[x=0.633333in, y=3.885159in, left, base]{\color{textcolor}\sffamily\fontsize{10.000000}{12.000000}\selectfont \(\displaystyle {0}\)}%
\end{pgfscope}%
\begin{pgfscope}%
\pgfsetbuttcap%
\pgfsetroundjoin%
\definecolor{currentfill}{rgb}{0.000000,0.000000,0.000000}%
\pgfsetfillcolor{currentfill}%
\pgfsetlinewidth{0.803000pt}%
\definecolor{currentstroke}{rgb}{0.000000,0.000000,0.000000}%
\pgfsetstrokecolor{currentstroke}%
\pgfsetdash{}{0pt}%
\pgfsys@defobject{currentmarker}{\pgfqpoint{-0.048611in}{0.000000in}}{\pgfqpoint{-0.000000in}{0.000000in}}{%
\pgfpathmoveto{\pgfqpoint{-0.000000in}{0.000000in}}%
\pgfpathlineto{\pgfqpoint{-0.048611in}{0.000000in}}%
\pgfusepath{stroke,fill}%
}%
\begin{pgfscope}%
\pgfsys@transformshift{0.800000in}{4.294580in}%
\pgfsys@useobject{currentmarker}{}%
\end{pgfscope}%
\end{pgfscope}%
\begin{pgfscope}%
\definecolor{textcolor}{rgb}{0.000000,0.000000,0.000000}%
\pgfsetstrokecolor{textcolor}%
\pgfsetfillcolor{textcolor}%
\pgftext[x=0.633333in, y=4.246355in, left, base]{\color{textcolor}\sffamily\fontsize{10.000000}{12.000000}\selectfont \(\displaystyle {1}\)}%
\end{pgfscope}%
\begin{pgfscope}%
\pgfsetbuttcap%
\pgfsetroundjoin%
\definecolor{currentfill}{rgb}{0.000000,0.000000,0.000000}%
\pgfsetfillcolor{currentfill}%
\pgfsetlinewidth{0.803000pt}%
\definecolor{currentstroke}{rgb}{0.000000,0.000000,0.000000}%
\pgfsetstrokecolor{currentstroke}%
\pgfsetdash{}{0pt}%
\pgfsys@defobject{currentmarker}{\pgfqpoint{-0.048611in}{0.000000in}}{\pgfqpoint{-0.000000in}{0.000000in}}{%
\pgfpathmoveto{\pgfqpoint{-0.000000in}{0.000000in}}%
\pgfpathlineto{\pgfqpoint{-0.048611in}{0.000000in}}%
\pgfusepath{stroke,fill}%
}%
\begin{pgfscope}%
\pgfsys@transformshift{0.800000in}{4.655776in}%
\pgfsys@useobject{currentmarker}{}%
\end{pgfscope}%
\end{pgfscope}%
\begin{pgfscope}%
\definecolor{textcolor}{rgb}{0.000000,0.000000,0.000000}%
\pgfsetstrokecolor{textcolor}%
\pgfsetfillcolor{textcolor}%
\pgftext[x=0.633333in, y=4.607551in, left, base]{\color{textcolor}\sffamily\fontsize{10.000000}{12.000000}\selectfont \(\displaystyle {2}\)}%
\end{pgfscope}%
\begin{pgfscope}%
\pgfsetbuttcap%
\pgfsetroundjoin%
\definecolor{currentfill}{rgb}{0.000000,0.000000,0.000000}%
\pgfsetfillcolor{currentfill}%
\pgfsetlinewidth{0.803000pt}%
\definecolor{currentstroke}{rgb}{0.000000,0.000000,0.000000}%
\pgfsetstrokecolor{currentstroke}%
\pgfsetdash{}{0pt}%
\pgfsys@defobject{currentmarker}{\pgfqpoint{-0.048611in}{0.000000in}}{\pgfqpoint{-0.000000in}{0.000000in}}{%
\pgfpathmoveto{\pgfqpoint{-0.000000in}{0.000000in}}%
\pgfpathlineto{\pgfqpoint{-0.048611in}{0.000000in}}%
\pgfusepath{stroke,fill}%
}%
\begin{pgfscope}%
\pgfsys@transformshift{0.800000in}{5.016973in}%
\pgfsys@useobject{currentmarker}{}%
\end{pgfscope}%
\end{pgfscope}%
\begin{pgfscope}%
\definecolor{textcolor}{rgb}{0.000000,0.000000,0.000000}%
\pgfsetstrokecolor{textcolor}%
\pgfsetfillcolor{textcolor}%
\pgftext[x=0.633333in, y=4.968747in, left, base]{\color{textcolor}\sffamily\fontsize{10.000000}{12.000000}\selectfont \(\displaystyle {3}\)}%
\end{pgfscope}%
\begin{pgfscope}%
\pgfsetbuttcap%
\pgfsetroundjoin%
\definecolor{currentfill}{rgb}{0.000000,0.000000,0.000000}%
\pgfsetfillcolor{currentfill}%
\pgfsetlinewidth{0.803000pt}%
\definecolor{currentstroke}{rgb}{0.000000,0.000000,0.000000}%
\pgfsetstrokecolor{currentstroke}%
\pgfsetdash{}{0pt}%
\pgfsys@defobject{currentmarker}{\pgfqpoint{-0.048611in}{0.000000in}}{\pgfqpoint{-0.000000in}{0.000000in}}{%
\pgfpathmoveto{\pgfqpoint{-0.000000in}{0.000000in}}%
\pgfpathlineto{\pgfqpoint{-0.048611in}{0.000000in}}%
\pgfusepath{stroke,fill}%
}%
\begin{pgfscope}%
\pgfsys@transformshift{0.800000in}{5.378169in}%
\pgfsys@useobject{currentmarker}{}%
\end{pgfscope}%
\end{pgfscope}%
\begin{pgfscope}%
\definecolor{textcolor}{rgb}{0.000000,0.000000,0.000000}%
\pgfsetstrokecolor{textcolor}%
\pgfsetfillcolor{textcolor}%
\pgftext[x=0.633333in, y=5.329943in, left, base]{\color{textcolor}\sffamily\fontsize{10.000000}{12.000000}\selectfont \(\displaystyle {4}\)}%
\end{pgfscope}%
\begin{pgfscope}%
\pgfsetbuttcap%
\pgfsetroundjoin%
\definecolor{currentfill}{rgb}{0.000000,0.000000,0.000000}%
\pgfsetfillcolor{currentfill}%
\pgfsetlinewidth{0.803000pt}%
\definecolor{currentstroke}{rgb}{0.000000,0.000000,0.000000}%
\pgfsetstrokecolor{currentstroke}%
\pgfsetdash{}{0pt}%
\pgfsys@defobject{currentmarker}{\pgfqpoint{-0.048611in}{0.000000in}}{\pgfqpoint{-0.000000in}{0.000000in}}{%
\pgfpathmoveto{\pgfqpoint{-0.000000in}{0.000000in}}%
\pgfpathlineto{\pgfqpoint{-0.048611in}{0.000000in}}%
\pgfusepath{stroke,fill}%
}%
\begin{pgfscope}%
\pgfsys@transformshift{0.800000in}{5.739365in}%
\pgfsys@useobject{currentmarker}{}%
\end{pgfscope}%
\end{pgfscope}%
\begin{pgfscope}%
\definecolor{textcolor}{rgb}{0.000000,0.000000,0.000000}%
\pgfsetstrokecolor{textcolor}%
\pgfsetfillcolor{textcolor}%
\pgftext[x=0.633333in, y=5.691140in, left, base]{\color{textcolor}\sffamily\fontsize{10.000000}{12.000000}\selectfont \(\displaystyle {5}\)}%
\end{pgfscope}%
\begin{pgfscope}%
\pgfsetbuttcap%
\pgfsetroundjoin%
\definecolor{currentfill}{rgb}{0.000000,0.000000,0.000000}%
\pgfsetfillcolor{currentfill}%
\pgfsetlinewidth{0.803000pt}%
\definecolor{currentstroke}{rgb}{0.000000,0.000000,0.000000}%
\pgfsetstrokecolor{currentstroke}%
\pgfsetdash{}{0pt}%
\pgfsys@defobject{currentmarker}{\pgfqpoint{-0.048611in}{0.000000in}}{\pgfqpoint{-0.000000in}{0.000000in}}{%
\pgfpathmoveto{\pgfqpoint{-0.000000in}{0.000000in}}%
\pgfpathlineto{\pgfqpoint{-0.048611in}{0.000000in}}%
\pgfusepath{stroke,fill}%
}%
\begin{pgfscope}%
\pgfsys@transformshift{0.800000in}{6.100561in}%
\pgfsys@useobject{currentmarker}{}%
\end{pgfscope}%
\end{pgfscope}%
\begin{pgfscope}%
\definecolor{textcolor}{rgb}{0.000000,0.000000,0.000000}%
\pgfsetstrokecolor{textcolor}%
\pgfsetfillcolor{textcolor}%
\pgftext[x=0.633333in, y=6.052336in, left, base]{\color{textcolor}\sffamily\fontsize{10.000000}{12.000000}\selectfont \(\displaystyle {6}\)}%
\end{pgfscope}%
\begin{pgfscope}%
\definecolor{textcolor}{rgb}{0.000000,0.000000,0.000000}%
\pgfsetstrokecolor{textcolor}%
\pgfsetfillcolor{textcolor}%
\pgftext[x=0.577778in,y=5.011087in,,bottom,rotate=90.000000]{\color{textcolor}\sffamily\fontsize{10.000000}{12.000000}\selectfont \(\displaystyle y \, \mathrm{[mm]}\)}%
\end{pgfscope}%
\begin{pgfscope}%
\pgfsetrectcap%
\pgfsetmiterjoin%
\pgfsetlinewidth{0.803000pt}%
\definecolor{currentstroke}{rgb}{0.000000,0.000000,0.000000}%
\pgfsetstrokecolor{currentstroke}%
\pgfsetdash{}{0pt}%
\pgfpathmoveto{\pgfqpoint{0.800000in}{3.862174in}}%
\pgfpathlineto{\pgfqpoint{0.800000in}{6.160000in}}%
\pgfusepath{stroke}%
\end{pgfscope}%
\begin{pgfscope}%
\pgfsetrectcap%
\pgfsetmiterjoin%
\pgfsetlinewidth{0.803000pt}%
\definecolor{currentstroke}{rgb}{0.000000,0.000000,0.000000}%
\pgfsetstrokecolor{currentstroke}%
\pgfsetdash{}{0pt}%
\pgfpathmoveto{\pgfqpoint{4.672000in}{3.862174in}}%
\pgfpathlineto{\pgfqpoint{4.672000in}{6.160000in}}%
\pgfusepath{stroke}%
\end{pgfscope}%
\begin{pgfscope}%
\pgfsetrectcap%
\pgfsetmiterjoin%
\pgfsetlinewidth{0.803000pt}%
\definecolor{currentstroke}{rgb}{0.000000,0.000000,0.000000}%
\pgfsetstrokecolor{currentstroke}%
\pgfsetdash{}{0pt}%
\pgfpathmoveto{\pgfqpoint{0.800000in}{3.862174in}}%
\pgfpathlineto{\pgfqpoint{4.672000in}{3.862174in}}%
\pgfusepath{stroke}%
\end{pgfscope}%
\begin{pgfscope}%
\pgfsetrectcap%
\pgfsetmiterjoin%
\pgfsetlinewidth{0.803000pt}%
\definecolor{currentstroke}{rgb}{0.000000,0.000000,0.000000}%
\pgfsetstrokecolor{currentstroke}%
\pgfsetdash{}{0pt}%
\pgfpathmoveto{\pgfqpoint{0.800000in}{6.160000in}}%
\pgfpathlineto{\pgfqpoint{4.672000in}{6.160000in}}%
\pgfusepath{stroke}%
\end{pgfscope}%
\begin{pgfscope}%
\definecolor{textcolor}{rgb}{0.000000,0.000000,0.000000}%
\pgfsetstrokecolor{textcolor}%
\pgfsetfillcolor{textcolor}%
\pgftext[x=2.736000in,y=6.243333in,,base]{\color{textcolor}\sffamily\fontsize{12.000000}{14.400000}\selectfont a)}%
\end{pgfscope}%
\begin{pgfscope}%
\pgfsetbuttcap%
\pgfsetmiterjoin%
\definecolor{currentfill}{rgb}{1.000000,1.000000,1.000000}%
\pgfsetfillcolor{currentfill}%
\pgfsetlinewidth{0.000000pt}%
\definecolor{currentstroke}{rgb}{0.000000,0.000000,0.000000}%
\pgfsetstrokecolor{currentstroke}%
\pgfsetstrokeopacity{0.000000}%
\pgfsetdash{}{0pt}%
\pgfpathmoveto{\pgfqpoint{0.800000in}{0.875000in}}%
\pgfpathlineto{\pgfqpoint{4.672000in}{0.875000in}}%
\pgfpathlineto{\pgfqpoint{4.672000in}{3.172826in}}%
\pgfpathlineto{\pgfqpoint{0.800000in}{3.172826in}}%
\pgfpathlineto{\pgfqpoint{0.800000in}{0.875000in}}%
\pgfpathclose%
\pgfusepath{fill}%
\end{pgfscope}%
\begin{pgfscope}%
\pgfpathrectangle{\pgfqpoint{0.800000in}{0.875000in}}{\pgfqpoint{3.872000in}{2.297826in}}%
\pgfusepath{clip}%
\pgfsys@transformcm{3.875000}{0.000000}{0.000000}{2.305556}{0.800000in}{0.875000in}%
\pgftext[left,bottom]{\includegraphics[interpolate=false,width=1.000000in,height=1.000000in]{E_y_hist-img1.png}}%
\end{pgfscope}%
\begin{pgfscope}%
\pgfsetbuttcap%
\pgfsetroundjoin%
\definecolor{currentfill}{rgb}{0.000000,0.000000,0.000000}%
\pgfsetfillcolor{currentfill}%
\pgfsetlinewidth{0.803000pt}%
\definecolor{currentstroke}{rgb}{0.000000,0.000000,0.000000}%
\pgfsetstrokecolor{currentstroke}%
\pgfsetdash{}{0pt}%
\pgfsys@defobject{currentmarker}{\pgfqpoint{0.000000in}{-0.048611in}}{\pgfqpoint{0.000000in}{0.000000in}}{%
\pgfpathmoveto{\pgfqpoint{0.000000in}{0.000000in}}%
\pgfpathlineto{\pgfqpoint{0.000000in}{-0.048611in}}%
\pgfusepath{stroke,fill}%
}%
\begin{pgfscope}%
\pgfsys@transformshift{1.384464in}{0.875000in}%
\pgfsys@useobject{currentmarker}{}%
\end{pgfscope}%
\end{pgfscope}%
\begin{pgfscope}%
\definecolor{textcolor}{rgb}{0.000000,0.000000,0.000000}%
\pgfsetstrokecolor{textcolor}%
\pgfsetfillcolor{textcolor}%
\pgftext[x=1.384464in,y=0.777778in,,top]{\color{textcolor}\sffamily\fontsize{10.000000}{12.000000}\selectfont \(\displaystyle {\ensuremath{-}30}\)}%
\end{pgfscope}%
\begin{pgfscope}%
\pgfsetbuttcap%
\pgfsetroundjoin%
\definecolor{currentfill}{rgb}{0.000000,0.000000,0.000000}%
\pgfsetfillcolor{currentfill}%
\pgfsetlinewidth{0.803000pt}%
\definecolor{currentstroke}{rgb}{0.000000,0.000000,0.000000}%
\pgfsetstrokecolor{currentstroke}%
\pgfsetdash{}{0pt}%
\pgfsys@defobject{currentmarker}{\pgfqpoint{0.000000in}{-0.048611in}}{\pgfqpoint{0.000000in}{0.000000in}}{%
\pgfpathmoveto{\pgfqpoint{0.000000in}{0.000000in}}%
\pgfpathlineto{\pgfqpoint{0.000000in}{-0.048611in}}%
\pgfusepath{stroke,fill}%
}%
\begin{pgfscope}%
\pgfsys@transformshift{2.112831in}{0.875000in}%
\pgfsys@useobject{currentmarker}{}%
\end{pgfscope}%
\end{pgfscope}%
\begin{pgfscope}%
\definecolor{textcolor}{rgb}{0.000000,0.000000,0.000000}%
\pgfsetstrokecolor{textcolor}%
\pgfsetfillcolor{textcolor}%
\pgftext[x=2.112831in,y=0.777778in,,top]{\color{textcolor}\sffamily\fontsize{10.000000}{12.000000}\selectfont \(\displaystyle {\ensuremath{-}20}\)}%
\end{pgfscope}%
\begin{pgfscope}%
\pgfsetbuttcap%
\pgfsetroundjoin%
\definecolor{currentfill}{rgb}{0.000000,0.000000,0.000000}%
\pgfsetfillcolor{currentfill}%
\pgfsetlinewidth{0.803000pt}%
\definecolor{currentstroke}{rgb}{0.000000,0.000000,0.000000}%
\pgfsetstrokecolor{currentstroke}%
\pgfsetdash{}{0pt}%
\pgfsys@defobject{currentmarker}{\pgfqpoint{0.000000in}{-0.048611in}}{\pgfqpoint{0.000000in}{0.000000in}}{%
\pgfpathmoveto{\pgfqpoint{0.000000in}{0.000000in}}%
\pgfpathlineto{\pgfqpoint{0.000000in}{-0.048611in}}%
\pgfusepath{stroke,fill}%
}%
\begin{pgfscope}%
\pgfsys@transformshift{2.841198in}{0.875000in}%
\pgfsys@useobject{currentmarker}{}%
\end{pgfscope}%
\end{pgfscope}%
\begin{pgfscope}%
\definecolor{textcolor}{rgb}{0.000000,0.000000,0.000000}%
\pgfsetstrokecolor{textcolor}%
\pgfsetfillcolor{textcolor}%
\pgftext[x=2.841198in,y=0.777778in,,top]{\color{textcolor}\sffamily\fontsize{10.000000}{12.000000}\selectfont \(\displaystyle {\ensuremath{-}10}\)}%
\end{pgfscope}%
\begin{pgfscope}%
\pgfsetbuttcap%
\pgfsetroundjoin%
\definecolor{currentfill}{rgb}{0.000000,0.000000,0.000000}%
\pgfsetfillcolor{currentfill}%
\pgfsetlinewidth{0.803000pt}%
\definecolor{currentstroke}{rgb}{0.000000,0.000000,0.000000}%
\pgfsetstrokecolor{currentstroke}%
\pgfsetdash{}{0pt}%
\pgfsys@defobject{currentmarker}{\pgfqpoint{0.000000in}{-0.048611in}}{\pgfqpoint{0.000000in}{0.000000in}}{%
\pgfpathmoveto{\pgfqpoint{0.000000in}{0.000000in}}%
\pgfpathlineto{\pgfqpoint{0.000000in}{-0.048611in}}%
\pgfusepath{stroke,fill}%
}%
\begin{pgfscope}%
\pgfsys@transformshift{3.569565in}{0.875000in}%
\pgfsys@useobject{currentmarker}{}%
\end{pgfscope}%
\end{pgfscope}%
\begin{pgfscope}%
\definecolor{textcolor}{rgb}{0.000000,0.000000,0.000000}%
\pgfsetstrokecolor{textcolor}%
\pgfsetfillcolor{textcolor}%
\pgftext[x=3.569565in,y=0.777778in,,top]{\color{textcolor}\sffamily\fontsize{10.000000}{12.000000}\selectfont \(\displaystyle {0}\)}%
\end{pgfscope}%
\begin{pgfscope}%
\pgfsetbuttcap%
\pgfsetroundjoin%
\definecolor{currentfill}{rgb}{0.000000,0.000000,0.000000}%
\pgfsetfillcolor{currentfill}%
\pgfsetlinewidth{0.803000pt}%
\definecolor{currentstroke}{rgb}{0.000000,0.000000,0.000000}%
\pgfsetstrokecolor{currentstroke}%
\pgfsetdash{}{0pt}%
\pgfsys@defobject{currentmarker}{\pgfqpoint{0.000000in}{-0.048611in}}{\pgfqpoint{0.000000in}{0.000000in}}{%
\pgfpathmoveto{\pgfqpoint{0.000000in}{0.000000in}}%
\pgfpathlineto{\pgfqpoint{0.000000in}{-0.048611in}}%
\pgfusepath{stroke,fill}%
}%
\begin{pgfscope}%
\pgfsys@transformshift{4.297932in}{0.875000in}%
\pgfsys@useobject{currentmarker}{}%
\end{pgfscope}%
\end{pgfscope}%
\begin{pgfscope}%
\definecolor{textcolor}{rgb}{0.000000,0.000000,0.000000}%
\pgfsetstrokecolor{textcolor}%
\pgfsetfillcolor{textcolor}%
\pgftext[x=4.297932in,y=0.777778in,,top]{\color{textcolor}\sffamily\fontsize{10.000000}{12.000000}\selectfont \(\displaystyle {10}\)}%
\end{pgfscope}%
\begin{pgfscope}%
\definecolor{textcolor}{rgb}{0.000000,0.000000,0.000000}%
\pgfsetstrokecolor{textcolor}%
\pgfsetfillcolor{textcolor}%
\pgftext[x=2.736000in,y=0.598766in,,top]{\color{textcolor}\sffamily\fontsize{10.000000}{12.000000}\selectfont \(\displaystyle \zeta \, \mathrm{[\mu m]}\)}%
\end{pgfscope}%
\begin{pgfscope}%
\pgfsetbuttcap%
\pgfsetroundjoin%
\definecolor{currentfill}{rgb}{0.000000,0.000000,0.000000}%
\pgfsetfillcolor{currentfill}%
\pgfsetlinewidth{0.803000pt}%
\definecolor{currentstroke}{rgb}{0.000000,0.000000,0.000000}%
\pgfsetstrokecolor{currentstroke}%
\pgfsetdash{}{0pt}%
\pgfsys@defobject{currentmarker}{\pgfqpoint{-0.048611in}{0.000000in}}{\pgfqpoint{-0.000000in}{0.000000in}}{%
\pgfpathmoveto{\pgfqpoint{-0.000000in}{0.000000in}}%
\pgfpathlineto{\pgfqpoint{-0.048611in}{0.000000in}}%
\pgfusepath{stroke,fill}%
}%
\begin{pgfscope}%
\pgfsys@transformshift{0.800000in}{0.946210in}%
\pgfsys@useobject{currentmarker}{}%
\end{pgfscope}%
\end{pgfscope}%
\begin{pgfscope}%
\definecolor{textcolor}{rgb}{0.000000,0.000000,0.000000}%
\pgfsetstrokecolor{textcolor}%
\pgfsetfillcolor{textcolor}%
\pgftext[x=0.633333in, y=0.897985in, left, base]{\color{textcolor}\sffamily\fontsize{10.000000}{12.000000}\selectfont \(\displaystyle {0}\)}%
\end{pgfscope}%
\begin{pgfscope}%
\pgfsetbuttcap%
\pgfsetroundjoin%
\definecolor{currentfill}{rgb}{0.000000,0.000000,0.000000}%
\pgfsetfillcolor{currentfill}%
\pgfsetlinewidth{0.803000pt}%
\definecolor{currentstroke}{rgb}{0.000000,0.000000,0.000000}%
\pgfsetstrokecolor{currentstroke}%
\pgfsetdash{}{0pt}%
\pgfsys@defobject{currentmarker}{\pgfqpoint{-0.048611in}{0.000000in}}{\pgfqpoint{-0.000000in}{0.000000in}}{%
\pgfpathmoveto{\pgfqpoint{-0.000000in}{0.000000in}}%
\pgfpathlineto{\pgfqpoint{-0.048611in}{0.000000in}}%
\pgfusepath{stroke,fill}%
}%
\begin{pgfscope}%
\pgfsys@transformshift{0.800000in}{1.307406in}%
\pgfsys@useobject{currentmarker}{}%
\end{pgfscope}%
\end{pgfscope}%
\begin{pgfscope}%
\definecolor{textcolor}{rgb}{0.000000,0.000000,0.000000}%
\pgfsetstrokecolor{textcolor}%
\pgfsetfillcolor{textcolor}%
\pgftext[x=0.633333in, y=1.259181in, left, base]{\color{textcolor}\sffamily\fontsize{10.000000}{12.000000}\selectfont \(\displaystyle {1}\)}%
\end{pgfscope}%
\begin{pgfscope}%
\pgfsetbuttcap%
\pgfsetroundjoin%
\definecolor{currentfill}{rgb}{0.000000,0.000000,0.000000}%
\pgfsetfillcolor{currentfill}%
\pgfsetlinewidth{0.803000pt}%
\definecolor{currentstroke}{rgb}{0.000000,0.000000,0.000000}%
\pgfsetstrokecolor{currentstroke}%
\pgfsetdash{}{0pt}%
\pgfsys@defobject{currentmarker}{\pgfqpoint{-0.048611in}{0.000000in}}{\pgfqpoint{-0.000000in}{0.000000in}}{%
\pgfpathmoveto{\pgfqpoint{-0.000000in}{0.000000in}}%
\pgfpathlineto{\pgfqpoint{-0.048611in}{0.000000in}}%
\pgfusepath{stroke,fill}%
}%
\begin{pgfscope}%
\pgfsys@transformshift{0.800000in}{1.668602in}%
\pgfsys@useobject{currentmarker}{}%
\end{pgfscope}%
\end{pgfscope}%
\begin{pgfscope}%
\definecolor{textcolor}{rgb}{0.000000,0.000000,0.000000}%
\pgfsetstrokecolor{textcolor}%
\pgfsetfillcolor{textcolor}%
\pgftext[x=0.633333in, y=1.620377in, left, base]{\color{textcolor}\sffamily\fontsize{10.000000}{12.000000}\selectfont \(\displaystyle {2}\)}%
\end{pgfscope}%
\begin{pgfscope}%
\pgfsetbuttcap%
\pgfsetroundjoin%
\definecolor{currentfill}{rgb}{0.000000,0.000000,0.000000}%
\pgfsetfillcolor{currentfill}%
\pgfsetlinewidth{0.803000pt}%
\definecolor{currentstroke}{rgb}{0.000000,0.000000,0.000000}%
\pgfsetstrokecolor{currentstroke}%
\pgfsetdash{}{0pt}%
\pgfsys@defobject{currentmarker}{\pgfqpoint{-0.048611in}{0.000000in}}{\pgfqpoint{-0.000000in}{0.000000in}}{%
\pgfpathmoveto{\pgfqpoint{-0.000000in}{0.000000in}}%
\pgfpathlineto{\pgfqpoint{-0.048611in}{0.000000in}}%
\pgfusepath{stroke,fill}%
}%
\begin{pgfscope}%
\pgfsys@transformshift{0.800000in}{2.029799in}%
\pgfsys@useobject{currentmarker}{}%
\end{pgfscope}%
\end{pgfscope}%
\begin{pgfscope}%
\definecolor{textcolor}{rgb}{0.000000,0.000000,0.000000}%
\pgfsetstrokecolor{textcolor}%
\pgfsetfillcolor{textcolor}%
\pgftext[x=0.633333in, y=1.981573in, left, base]{\color{textcolor}\sffamily\fontsize{10.000000}{12.000000}\selectfont \(\displaystyle {3}\)}%
\end{pgfscope}%
\begin{pgfscope}%
\pgfsetbuttcap%
\pgfsetroundjoin%
\definecolor{currentfill}{rgb}{0.000000,0.000000,0.000000}%
\pgfsetfillcolor{currentfill}%
\pgfsetlinewidth{0.803000pt}%
\definecolor{currentstroke}{rgb}{0.000000,0.000000,0.000000}%
\pgfsetstrokecolor{currentstroke}%
\pgfsetdash{}{0pt}%
\pgfsys@defobject{currentmarker}{\pgfqpoint{-0.048611in}{0.000000in}}{\pgfqpoint{-0.000000in}{0.000000in}}{%
\pgfpathmoveto{\pgfqpoint{-0.000000in}{0.000000in}}%
\pgfpathlineto{\pgfqpoint{-0.048611in}{0.000000in}}%
\pgfusepath{stroke,fill}%
}%
\begin{pgfscope}%
\pgfsys@transformshift{0.800000in}{2.390995in}%
\pgfsys@useobject{currentmarker}{}%
\end{pgfscope}%
\end{pgfscope}%
\begin{pgfscope}%
\definecolor{textcolor}{rgb}{0.000000,0.000000,0.000000}%
\pgfsetstrokecolor{textcolor}%
\pgfsetfillcolor{textcolor}%
\pgftext[x=0.633333in, y=2.342769in, left, base]{\color{textcolor}\sffamily\fontsize{10.000000}{12.000000}\selectfont \(\displaystyle {4}\)}%
\end{pgfscope}%
\begin{pgfscope}%
\pgfsetbuttcap%
\pgfsetroundjoin%
\definecolor{currentfill}{rgb}{0.000000,0.000000,0.000000}%
\pgfsetfillcolor{currentfill}%
\pgfsetlinewidth{0.803000pt}%
\definecolor{currentstroke}{rgb}{0.000000,0.000000,0.000000}%
\pgfsetstrokecolor{currentstroke}%
\pgfsetdash{}{0pt}%
\pgfsys@defobject{currentmarker}{\pgfqpoint{-0.048611in}{0.000000in}}{\pgfqpoint{-0.000000in}{0.000000in}}{%
\pgfpathmoveto{\pgfqpoint{-0.000000in}{0.000000in}}%
\pgfpathlineto{\pgfqpoint{-0.048611in}{0.000000in}}%
\pgfusepath{stroke,fill}%
}%
\begin{pgfscope}%
\pgfsys@transformshift{0.800000in}{2.752191in}%
\pgfsys@useobject{currentmarker}{}%
\end{pgfscope}%
\end{pgfscope}%
\begin{pgfscope}%
\definecolor{textcolor}{rgb}{0.000000,0.000000,0.000000}%
\pgfsetstrokecolor{textcolor}%
\pgfsetfillcolor{textcolor}%
\pgftext[x=0.633333in, y=2.703966in, left, base]{\color{textcolor}\sffamily\fontsize{10.000000}{12.000000}\selectfont \(\displaystyle {5}\)}%
\end{pgfscope}%
\begin{pgfscope}%
\pgfsetbuttcap%
\pgfsetroundjoin%
\definecolor{currentfill}{rgb}{0.000000,0.000000,0.000000}%
\pgfsetfillcolor{currentfill}%
\pgfsetlinewidth{0.803000pt}%
\definecolor{currentstroke}{rgb}{0.000000,0.000000,0.000000}%
\pgfsetstrokecolor{currentstroke}%
\pgfsetdash{}{0pt}%
\pgfsys@defobject{currentmarker}{\pgfqpoint{-0.048611in}{0.000000in}}{\pgfqpoint{-0.000000in}{0.000000in}}{%
\pgfpathmoveto{\pgfqpoint{-0.000000in}{0.000000in}}%
\pgfpathlineto{\pgfqpoint{-0.048611in}{0.000000in}}%
\pgfusepath{stroke,fill}%
}%
\begin{pgfscope}%
\pgfsys@transformshift{0.800000in}{3.113387in}%
\pgfsys@useobject{currentmarker}{}%
\end{pgfscope}%
\end{pgfscope}%
\begin{pgfscope}%
\definecolor{textcolor}{rgb}{0.000000,0.000000,0.000000}%
\pgfsetstrokecolor{textcolor}%
\pgfsetfillcolor{textcolor}%
\pgftext[x=0.633333in, y=3.065162in, left, base]{\color{textcolor}\sffamily\fontsize{10.000000}{12.000000}\selectfont \(\displaystyle {6}\)}%
\end{pgfscope}%
\begin{pgfscope}%
\definecolor{textcolor}{rgb}{0.000000,0.000000,0.000000}%
\pgfsetstrokecolor{textcolor}%
\pgfsetfillcolor{textcolor}%
\pgftext[x=0.577778in,y=2.023913in,,bottom,rotate=90.000000]{\color{textcolor}\sffamily\fontsize{10.000000}{12.000000}\selectfont \(\displaystyle y \, \mathrm{[mm]}\)}%
\end{pgfscope}%
\begin{pgfscope}%
\pgfsetrectcap%
\pgfsetmiterjoin%
\pgfsetlinewidth{0.803000pt}%
\definecolor{currentstroke}{rgb}{0.000000,0.000000,0.000000}%
\pgfsetstrokecolor{currentstroke}%
\pgfsetdash{}{0pt}%
\pgfpathmoveto{\pgfqpoint{0.800000in}{0.875000in}}%
\pgfpathlineto{\pgfqpoint{0.800000in}{3.172826in}}%
\pgfusepath{stroke}%
\end{pgfscope}%
\begin{pgfscope}%
\pgfsetrectcap%
\pgfsetmiterjoin%
\pgfsetlinewidth{0.803000pt}%
\definecolor{currentstroke}{rgb}{0.000000,0.000000,0.000000}%
\pgfsetstrokecolor{currentstroke}%
\pgfsetdash{}{0pt}%
\pgfpathmoveto{\pgfqpoint{4.672000in}{0.875000in}}%
\pgfpathlineto{\pgfqpoint{4.672000in}{3.172826in}}%
\pgfusepath{stroke}%
\end{pgfscope}%
\begin{pgfscope}%
\pgfsetrectcap%
\pgfsetmiterjoin%
\pgfsetlinewidth{0.803000pt}%
\definecolor{currentstroke}{rgb}{0.000000,0.000000,0.000000}%
\pgfsetstrokecolor{currentstroke}%
\pgfsetdash{}{0pt}%
\pgfpathmoveto{\pgfqpoint{0.800000in}{0.875000in}}%
\pgfpathlineto{\pgfqpoint{4.672000in}{0.875000in}}%
\pgfusepath{stroke}%
\end{pgfscope}%
\begin{pgfscope}%
\pgfsetrectcap%
\pgfsetmiterjoin%
\pgfsetlinewidth{0.803000pt}%
\definecolor{currentstroke}{rgb}{0.000000,0.000000,0.000000}%
\pgfsetstrokecolor{currentstroke}%
\pgfsetdash{}{0pt}%
\pgfpathmoveto{\pgfqpoint{0.800000in}{3.172826in}}%
\pgfpathlineto{\pgfqpoint{4.672000in}{3.172826in}}%
\pgfusepath{stroke}%
\end{pgfscope}%
\begin{pgfscope}%
\pgfsetbuttcap%
\pgfsetmiterjoin%
\definecolor{currentfill}{rgb}{1.000000,1.000000,1.000000}%
\pgfsetfillcolor{currentfill}%
\pgfsetlinewidth{0.000000pt}%
\definecolor{currentstroke}{rgb}{0.000000,0.000000,0.000000}%
\pgfsetstrokecolor{currentstroke}%
\pgfsetstrokeopacity{0.000000}%
\pgfsetdash{}{0pt}%
\pgfpathmoveto{\pgfqpoint{5.440000in}{3.780000in}}%
\pgfpathlineto{\pgfqpoint{5.632000in}{3.780000in}}%
\pgfpathlineto{\pgfqpoint{5.632000in}{6.160000in}}%
\pgfpathlineto{\pgfqpoint{5.440000in}{6.160000in}}%
\pgfpathlineto{\pgfqpoint{5.440000in}{3.780000in}}%
\pgfpathclose%
\pgfusepath{fill}%
\end{pgfscope}%
\begin{pgfscope}%
\pgfpathrectangle{\pgfqpoint{5.440000in}{3.780000in}}{\pgfqpoint{0.192000in}{2.380000in}}%
\pgfusepath{clip}%
\pgfsetbuttcap%
\pgfsetmiterjoin%
\definecolor{currentfill}{rgb}{1.000000,1.000000,1.000000}%
\pgfsetfillcolor{currentfill}%
\pgfsetlinewidth{0.010037pt}%
\definecolor{currentstroke}{rgb}{1.000000,1.000000,1.000000}%
\pgfsetstrokecolor{currentstroke}%
\pgfsetdash{}{0pt}%
\pgfusepath{stroke,fill}%
\end{pgfscope}%
\begin{pgfscope}%
\pgfsys@transformshift{5.444444in}{3.777778in}%
\pgftext[left,bottom]{\includegraphics[interpolate=true,width=0.194444in,height=2.388889in]{E_y_hist-img2.png}}%
\end{pgfscope}%
\begin{pgfscope}%
\pgfsetbuttcap%
\pgfsetroundjoin%
\definecolor{currentfill}{rgb}{0.000000,0.000000,0.000000}%
\pgfsetfillcolor{currentfill}%
\pgfsetlinewidth{0.803000pt}%
\definecolor{currentstroke}{rgb}{0.000000,0.000000,0.000000}%
\pgfsetstrokecolor{currentstroke}%
\pgfsetdash{}{0pt}%
\pgfsys@defobject{currentmarker}{\pgfqpoint{0.000000in}{0.000000in}}{\pgfqpoint{0.048611in}{0.000000in}}{%
\pgfpathmoveto{\pgfqpoint{0.000000in}{0.000000in}}%
\pgfpathlineto{\pgfqpoint{0.048611in}{0.000000in}}%
\pgfusepath{stroke,fill}%
}%
\begin{pgfscope}%
\pgfsys@transformshift{5.632000in}{3.780000in}%
\pgfsys@useobject{currentmarker}{}%
\end{pgfscope}%
\end{pgfscope}%
\begin{pgfscope}%
\definecolor{textcolor}{rgb}{0.000000,0.000000,0.000000}%
\pgfsetstrokecolor{textcolor}%
\pgfsetfillcolor{textcolor}%
\pgftext[x=5.729222in, y=3.731775in, left, base]{\color{textcolor}\sffamily\fontsize{10.000000}{12.000000}\selectfont \(\displaystyle {10^{-3}}\)}%
\end{pgfscope}%
\begin{pgfscope}%
\pgfsetbuttcap%
\pgfsetroundjoin%
\definecolor{currentfill}{rgb}{0.000000,0.000000,0.000000}%
\pgfsetfillcolor{currentfill}%
\pgfsetlinewidth{0.803000pt}%
\definecolor{currentstroke}{rgb}{0.000000,0.000000,0.000000}%
\pgfsetstrokecolor{currentstroke}%
\pgfsetdash{}{0pt}%
\pgfsys@defobject{currentmarker}{\pgfqpoint{0.000000in}{0.000000in}}{\pgfqpoint{0.048611in}{0.000000in}}{%
\pgfpathmoveto{\pgfqpoint{0.000000in}{0.000000in}}%
\pgfpathlineto{\pgfqpoint{0.048611in}{0.000000in}}%
\pgfusepath{stroke,fill}%
}%
\begin{pgfscope}%
\pgfsys@transformshift{5.632000in}{4.228969in}%
\pgfsys@useobject{currentmarker}{}%
\end{pgfscope}%
\end{pgfscope}%
\begin{pgfscope}%
\definecolor{textcolor}{rgb}{0.000000,0.000000,0.000000}%
\pgfsetstrokecolor{textcolor}%
\pgfsetfillcolor{textcolor}%
\pgftext[x=5.729222in, y=4.180744in, left, base]{\color{textcolor}\sffamily\fontsize{10.000000}{12.000000}\selectfont \(\displaystyle {10^{-2}}\)}%
\end{pgfscope}%
\begin{pgfscope}%
\pgfsetbuttcap%
\pgfsetroundjoin%
\definecolor{currentfill}{rgb}{0.000000,0.000000,0.000000}%
\pgfsetfillcolor{currentfill}%
\pgfsetlinewidth{0.803000pt}%
\definecolor{currentstroke}{rgb}{0.000000,0.000000,0.000000}%
\pgfsetstrokecolor{currentstroke}%
\pgfsetdash{}{0pt}%
\pgfsys@defobject{currentmarker}{\pgfqpoint{0.000000in}{0.000000in}}{\pgfqpoint{0.048611in}{0.000000in}}{%
\pgfpathmoveto{\pgfqpoint{0.000000in}{0.000000in}}%
\pgfpathlineto{\pgfqpoint{0.048611in}{0.000000in}}%
\pgfusepath{stroke,fill}%
}%
\begin{pgfscope}%
\pgfsys@transformshift{5.632000in}{4.677939in}%
\pgfsys@useobject{currentmarker}{}%
\end{pgfscope}%
\end{pgfscope}%
\begin{pgfscope}%
\definecolor{textcolor}{rgb}{0.000000,0.000000,0.000000}%
\pgfsetstrokecolor{textcolor}%
\pgfsetfillcolor{textcolor}%
\pgftext[x=5.729222in, y=4.629713in, left, base]{\color{textcolor}\sffamily\fontsize{10.000000}{12.000000}\selectfont \(\displaystyle {10^{-1}}\)}%
\end{pgfscope}%
\begin{pgfscope}%
\pgfsetbuttcap%
\pgfsetroundjoin%
\definecolor{currentfill}{rgb}{0.000000,0.000000,0.000000}%
\pgfsetfillcolor{currentfill}%
\pgfsetlinewidth{0.803000pt}%
\definecolor{currentstroke}{rgb}{0.000000,0.000000,0.000000}%
\pgfsetstrokecolor{currentstroke}%
\pgfsetdash{}{0pt}%
\pgfsys@defobject{currentmarker}{\pgfqpoint{0.000000in}{0.000000in}}{\pgfqpoint{0.048611in}{0.000000in}}{%
\pgfpathmoveto{\pgfqpoint{0.000000in}{0.000000in}}%
\pgfpathlineto{\pgfqpoint{0.048611in}{0.000000in}}%
\pgfusepath{stroke,fill}%
}%
\begin{pgfscope}%
\pgfsys@transformshift{5.632000in}{5.126908in}%
\pgfsys@useobject{currentmarker}{}%
\end{pgfscope}%
\end{pgfscope}%
\begin{pgfscope}%
\definecolor{textcolor}{rgb}{0.000000,0.000000,0.000000}%
\pgfsetstrokecolor{textcolor}%
\pgfsetfillcolor{textcolor}%
\pgftext[x=5.729222in, y=5.078683in, left, base]{\color{textcolor}\sffamily\fontsize{10.000000}{12.000000}\selectfont \(\displaystyle {10^{0}}\)}%
\end{pgfscope}%
\begin{pgfscope}%
\pgfsetbuttcap%
\pgfsetroundjoin%
\definecolor{currentfill}{rgb}{0.000000,0.000000,0.000000}%
\pgfsetfillcolor{currentfill}%
\pgfsetlinewidth{0.803000pt}%
\definecolor{currentstroke}{rgb}{0.000000,0.000000,0.000000}%
\pgfsetstrokecolor{currentstroke}%
\pgfsetdash{}{0pt}%
\pgfsys@defobject{currentmarker}{\pgfqpoint{0.000000in}{0.000000in}}{\pgfqpoint{0.048611in}{0.000000in}}{%
\pgfpathmoveto{\pgfqpoint{0.000000in}{0.000000in}}%
\pgfpathlineto{\pgfqpoint{0.048611in}{0.000000in}}%
\pgfusepath{stroke,fill}%
}%
\begin{pgfscope}%
\pgfsys@transformshift{5.632000in}{5.575877in}%
\pgfsys@useobject{currentmarker}{}%
\end{pgfscope}%
\end{pgfscope}%
\begin{pgfscope}%
\definecolor{textcolor}{rgb}{0.000000,0.000000,0.000000}%
\pgfsetstrokecolor{textcolor}%
\pgfsetfillcolor{textcolor}%
\pgftext[x=5.729222in, y=5.527652in, left, base]{\color{textcolor}\sffamily\fontsize{10.000000}{12.000000}\selectfont \(\displaystyle {10^{1}}\)}%
\end{pgfscope}%
\begin{pgfscope}%
\pgfsetbuttcap%
\pgfsetroundjoin%
\definecolor{currentfill}{rgb}{0.000000,0.000000,0.000000}%
\pgfsetfillcolor{currentfill}%
\pgfsetlinewidth{0.803000pt}%
\definecolor{currentstroke}{rgb}{0.000000,0.000000,0.000000}%
\pgfsetstrokecolor{currentstroke}%
\pgfsetdash{}{0pt}%
\pgfsys@defobject{currentmarker}{\pgfqpoint{0.000000in}{0.000000in}}{\pgfqpoint{0.048611in}{0.000000in}}{%
\pgfpathmoveto{\pgfqpoint{0.000000in}{0.000000in}}%
\pgfpathlineto{\pgfqpoint{0.048611in}{0.000000in}}%
\pgfusepath{stroke,fill}%
}%
\begin{pgfscope}%
\pgfsys@transformshift{5.632000in}{6.024847in}%
\pgfsys@useobject{currentmarker}{}%
\end{pgfscope}%
\end{pgfscope}%
\begin{pgfscope}%
\definecolor{textcolor}{rgb}{0.000000,0.000000,0.000000}%
\pgfsetstrokecolor{textcolor}%
\pgfsetfillcolor{textcolor}%
\pgftext[x=5.729222in, y=5.976621in, left, base]{\color{textcolor}\sffamily\fontsize{10.000000}{12.000000}\selectfont \(\displaystyle {10^{2}}\)}%
\end{pgfscope}%
\begin{pgfscope}%
\pgfsetbuttcap%
\pgfsetroundjoin%
\definecolor{currentfill}{rgb}{0.000000,0.000000,0.000000}%
\pgfsetfillcolor{currentfill}%
\pgfsetlinewidth{0.602250pt}%
\definecolor{currentstroke}{rgb}{0.000000,0.000000,0.000000}%
\pgfsetstrokecolor{currentstroke}%
\pgfsetdash{}{0pt}%
\pgfsys@defobject{currentmarker}{\pgfqpoint{0.000000in}{0.000000in}}{\pgfqpoint{0.027778in}{0.000000in}}{%
\pgfpathmoveto{\pgfqpoint{0.000000in}{0.000000in}}%
\pgfpathlineto{\pgfqpoint{0.027778in}{0.000000in}}%
\pgfusepath{stroke,fill}%
}%
\begin{pgfscope}%
\pgfsys@transformshift{5.632000in}{3.915153in}%
\pgfsys@useobject{currentmarker}{}%
\end{pgfscope}%
\end{pgfscope}%
\begin{pgfscope}%
\pgfsetbuttcap%
\pgfsetroundjoin%
\definecolor{currentfill}{rgb}{0.000000,0.000000,0.000000}%
\pgfsetfillcolor{currentfill}%
\pgfsetlinewidth{0.602250pt}%
\definecolor{currentstroke}{rgb}{0.000000,0.000000,0.000000}%
\pgfsetstrokecolor{currentstroke}%
\pgfsetdash{}{0pt}%
\pgfsys@defobject{currentmarker}{\pgfqpoint{0.000000in}{0.000000in}}{\pgfqpoint{0.027778in}{0.000000in}}{%
\pgfpathmoveto{\pgfqpoint{0.000000in}{0.000000in}}%
\pgfpathlineto{\pgfqpoint{0.027778in}{0.000000in}}%
\pgfusepath{stroke,fill}%
}%
\begin{pgfscope}%
\pgfsys@transformshift{5.632000in}{3.994213in}%
\pgfsys@useobject{currentmarker}{}%
\end{pgfscope}%
\end{pgfscope}%
\begin{pgfscope}%
\pgfsetbuttcap%
\pgfsetroundjoin%
\definecolor{currentfill}{rgb}{0.000000,0.000000,0.000000}%
\pgfsetfillcolor{currentfill}%
\pgfsetlinewidth{0.602250pt}%
\definecolor{currentstroke}{rgb}{0.000000,0.000000,0.000000}%
\pgfsetstrokecolor{currentstroke}%
\pgfsetdash{}{0pt}%
\pgfsys@defobject{currentmarker}{\pgfqpoint{0.000000in}{0.000000in}}{\pgfqpoint{0.027778in}{0.000000in}}{%
\pgfpathmoveto{\pgfqpoint{0.000000in}{0.000000in}}%
\pgfpathlineto{\pgfqpoint{0.027778in}{0.000000in}}%
\pgfusepath{stroke,fill}%
}%
\begin{pgfscope}%
\pgfsys@transformshift{5.632000in}{4.050306in}%
\pgfsys@useobject{currentmarker}{}%
\end{pgfscope}%
\end{pgfscope}%
\begin{pgfscope}%
\pgfsetbuttcap%
\pgfsetroundjoin%
\definecolor{currentfill}{rgb}{0.000000,0.000000,0.000000}%
\pgfsetfillcolor{currentfill}%
\pgfsetlinewidth{0.602250pt}%
\definecolor{currentstroke}{rgb}{0.000000,0.000000,0.000000}%
\pgfsetstrokecolor{currentstroke}%
\pgfsetdash{}{0pt}%
\pgfsys@defobject{currentmarker}{\pgfqpoint{0.000000in}{0.000000in}}{\pgfqpoint{0.027778in}{0.000000in}}{%
\pgfpathmoveto{\pgfqpoint{0.000000in}{0.000000in}}%
\pgfpathlineto{\pgfqpoint{0.027778in}{0.000000in}}%
\pgfusepath{stroke,fill}%
}%
\begin{pgfscope}%
\pgfsys@transformshift{5.632000in}{4.093816in}%
\pgfsys@useobject{currentmarker}{}%
\end{pgfscope}%
\end{pgfscope}%
\begin{pgfscope}%
\pgfsetbuttcap%
\pgfsetroundjoin%
\definecolor{currentfill}{rgb}{0.000000,0.000000,0.000000}%
\pgfsetfillcolor{currentfill}%
\pgfsetlinewidth{0.602250pt}%
\definecolor{currentstroke}{rgb}{0.000000,0.000000,0.000000}%
\pgfsetstrokecolor{currentstroke}%
\pgfsetdash{}{0pt}%
\pgfsys@defobject{currentmarker}{\pgfqpoint{0.000000in}{0.000000in}}{\pgfqpoint{0.027778in}{0.000000in}}{%
\pgfpathmoveto{\pgfqpoint{0.000000in}{0.000000in}}%
\pgfpathlineto{\pgfqpoint{0.027778in}{0.000000in}}%
\pgfusepath{stroke,fill}%
}%
\begin{pgfscope}%
\pgfsys@transformshift{5.632000in}{4.129366in}%
\pgfsys@useobject{currentmarker}{}%
\end{pgfscope}%
\end{pgfscope}%
\begin{pgfscope}%
\pgfsetbuttcap%
\pgfsetroundjoin%
\definecolor{currentfill}{rgb}{0.000000,0.000000,0.000000}%
\pgfsetfillcolor{currentfill}%
\pgfsetlinewidth{0.602250pt}%
\definecolor{currentstroke}{rgb}{0.000000,0.000000,0.000000}%
\pgfsetstrokecolor{currentstroke}%
\pgfsetdash{}{0pt}%
\pgfsys@defobject{currentmarker}{\pgfqpoint{0.000000in}{0.000000in}}{\pgfqpoint{0.027778in}{0.000000in}}{%
\pgfpathmoveto{\pgfqpoint{0.000000in}{0.000000in}}%
\pgfpathlineto{\pgfqpoint{0.027778in}{0.000000in}}%
\pgfusepath{stroke,fill}%
}%
\begin{pgfscope}%
\pgfsys@transformshift{5.632000in}{4.159423in}%
\pgfsys@useobject{currentmarker}{}%
\end{pgfscope}%
\end{pgfscope}%
\begin{pgfscope}%
\pgfsetbuttcap%
\pgfsetroundjoin%
\definecolor{currentfill}{rgb}{0.000000,0.000000,0.000000}%
\pgfsetfillcolor{currentfill}%
\pgfsetlinewidth{0.602250pt}%
\definecolor{currentstroke}{rgb}{0.000000,0.000000,0.000000}%
\pgfsetstrokecolor{currentstroke}%
\pgfsetdash{}{0pt}%
\pgfsys@defobject{currentmarker}{\pgfqpoint{0.000000in}{0.000000in}}{\pgfqpoint{0.027778in}{0.000000in}}{%
\pgfpathmoveto{\pgfqpoint{0.000000in}{0.000000in}}%
\pgfpathlineto{\pgfqpoint{0.027778in}{0.000000in}}%
\pgfusepath{stroke,fill}%
}%
\begin{pgfscope}%
\pgfsys@transformshift{5.632000in}{4.185460in}%
\pgfsys@useobject{currentmarker}{}%
\end{pgfscope}%
\end{pgfscope}%
\begin{pgfscope}%
\pgfsetbuttcap%
\pgfsetroundjoin%
\definecolor{currentfill}{rgb}{0.000000,0.000000,0.000000}%
\pgfsetfillcolor{currentfill}%
\pgfsetlinewidth{0.602250pt}%
\definecolor{currentstroke}{rgb}{0.000000,0.000000,0.000000}%
\pgfsetstrokecolor{currentstroke}%
\pgfsetdash{}{0pt}%
\pgfsys@defobject{currentmarker}{\pgfqpoint{0.000000in}{0.000000in}}{\pgfqpoint{0.027778in}{0.000000in}}{%
\pgfpathmoveto{\pgfqpoint{0.000000in}{0.000000in}}%
\pgfpathlineto{\pgfqpoint{0.027778in}{0.000000in}}%
\pgfusepath{stroke,fill}%
}%
\begin{pgfscope}%
\pgfsys@transformshift{5.632000in}{4.208426in}%
\pgfsys@useobject{currentmarker}{}%
\end{pgfscope}%
\end{pgfscope}%
\begin{pgfscope}%
\pgfsetbuttcap%
\pgfsetroundjoin%
\definecolor{currentfill}{rgb}{0.000000,0.000000,0.000000}%
\pgfsetfillcolor{currentfill}%
\pgfsetlinewidth{0.602250pt}%
\definecolor{currentstroke}{rgb}{0.000000,0.000000,0.000000}%
\pgfsetstrokecolor{currentstroke}%
\pgfsetdash{}{0pt}%
\pgfsys@defobject{currentmarker}{\pgfqpoint{0.000000in}{0.000000in}}{\pgfqpoint{0.027778in}{0.000000in}}{%
\pgfpathmoveto{\pgfqpoint{0.000000in}{0.000000in}}%
\pgfpathlineto{\pgfqpoint{0.027778in}{0.000000in}}%
\pgfusepath{stroke,fill}%
}%
\begin{pgfscope}%
\pgfsys@transformshift{5.632000in}{4.364123in}%
\pgfsys@useobject{currentmarker}{}%
\end{pgfscope}%
\end{pgfscope}%
\begin{pgfscope}%
\pgfsetbuttcap%
\pgfsetroundjoin%
\definecolor{currentfill}{rgb}{0.000000,0.000000,0.000000}%
\pgfsetfillcolor{currentfill}%
\pgfsetlinewidth{0.602250pt}%
\definecolor{currentstroke}{rgb}{0.000000,0.000000,0.000000}%
\pgfsetstrokecolor{currentstroke}%
\pgfsetdash{}{0pt}%
\pgfsys@defobject{currentmarker}{\pgfqpoint{0.000000in}{0.000000in}}{\pgfqpoint{0.027778in}{0.000000in}}{%
\pgfpathmoveto{\pgfqpoint{0.000000in}{0.000000in}}%
\pgfpathlineto{\pgfqpoint{0.027778in}{0.000000in}}%
\pgfusepath{stroke,fill}%
}%
\begin{pgfscope}%
\pgfsys@transformshift{5.632000in}{4.443182in}%
\pgfsys@useobject{currentmarker}{}%
\end{pgfscope}%
\end{pgfscope}%
\begin{pgfscope}%
\pgfsetbuttcap%
\pgfsetroundjoin%
\definecolor{currentfill}{rgb}{0.000000,0.000000,0.000000}%
\pgfsetfillcolor{currentfill}%
\pgfsetlinewidth{0.602250pt}%
\definecolor{currentstroke}{rgb}{0.000000,0.000000,0.000000}%
\pgfsetstrokecolor{currentstroke}%
\pgfsetdash{}{0pt}%
\pgfsys@defobject{currentmarker}{\pgfqpoint{0.000000in}{0.000000in}}{\pgfqpoint{0.027778in}{0.000000in}}{%
\pgfpathmoveto{\pgfqpoint{0.000000in}{0.000000in}}%
\pgfpathlineto{\pgfqpoint{0.027778in}{0.000000in}}%
\pgfusepath{stroke,fill}%
}%
\begin{pgfscope}%
\pgfsys@transformshift{5.632000in}{4.499276in}%
\pgfsys@useobject{currentmarker}{}%
\end{pgfscope}%
\end{pgfscope}%
\begin{pgfscope}%
\pgfsetbuttcap%
\pgfsetroundjoin%
\definecolor{currentfill}{rgb}{0.000000,0.000000,0.000000}%
\pgfsetfillcolor{currentfill}%
\pgfsetlinewidth{0.602250pt}%
\definecolor{currentstroke}{rgb}{0.000000,0.000000,0.000000}%
\pgfsetstrokecolor{currentstroke}%
\pgfsetdash{}{0pt}%
\pgfsys@defobject{currentmarker}{\pgfqpoint{0.000000in}{0.000000in}}{\pgfqpoint{0.027778in}{0.000000in}}{%
\pgfpathmoveto{\pgfqpoint{0.000000in}{0.000000in}}%
\pgfpathlineto{\pgfqpoint{0.027778in}{0.000000in}}%
\pgfusepath{stroke,fill}%
}%
\begin{pgfscope}%
\pgfsys@transformshift{5.632000in}{4.542785in}%
\pgfsys@useobject{currentmarker}{}%
\end{pgfscope}%
\end{pgfscope}%
\begin{pgfscope}%
\pgfsetbuttcap%
\pgfsetroundjoin%
\definecolor{currentfill}{rgb}{0.000000,0.000000,0.000000}%
\pgfsetfillcolor{currentfill}%
\pgfsetlinewidth{0.602250pt}%
\definecolor{currentstroke}{rgb}{0.000000,0.000000,0.000000}%
\pgfsetstrokecolor{currentstroke}%
\pgfsetdash{}{0pt}%
\pgfsys@defobject{currentmarker}{\pgfqpoint{0.000000in}{0.000000in}}{\pgfqpoint{0.027778in}{0.000000in}}{%
\pgfpathmoveto{\pgfqpoint{0.000000in}{0.000000in}}%
\pgfpathlineto{\pgfqpoint{0.027778in}{0.000000in}}%
\pgfusepath{stroke,fill}%
}%
\begin{pgfscope}%
\pgfsys@transformshift{5.632000in}{4.578335in}%
\pgfsys@useobject{currentmarker}{}%
\end{pgfscope}%
\end{pgfscope}%
\begin{pgfscope}%
\pgfsetbuttcap%
\pgfsetroundjoin%
\definecolor{currentfill}{rgb}{0.000000,0.000000,0.000000}%
\pgfsetfillcolor{currentfill}%
\pgfsetlinewidth{0.602250pt}%
\definecolor{currentstroke}{rgb}{0.000000,0.000000,0.000000}%
\pgfsetstrokecolor{currentstroke}%
\pgfsetdash{}{0pt}%
\pgfsys@defobject{currentmarker}{\pgfqpoint{0.000000in}{0.000000in}}{\pgfqpoint{0.027778in}{0.000000in}}{%
\pgfpathmoveto{\pgfqpoint{0.000000in}{0.000000in}}%
\pgfpathlineto{\pgfqpoint{0.027778in}{0.000000in}}%
\pgfusepath{stroke,fill}%
}%
\begin{pgfscope}%
\pgfsys@transformshift{5.632000in}{4.608392in}%
\pgfsys@useobject{currentmarker}{}%
\end{pgfscope}%
\end{pgfscope}%
\begin{pgfscope}%
\pgfsetbuttcap%
\pgfsetroundjoin%
\definecolor{currentfill}{rgb}{0.000000,0.000000,0.000000}%
\pgfsetfillcolor{currentfill}%
\pgfsetlinewidth{0.602250pt}%
\definecolor{currentstroke}{rgb}{0.000000,0.000000,0.000000}%
\pgfsetstrokecolor{currentstroke}%
\pgfsetdash{}{0pt}%
\pgfsys@defobject{currentmarker}{\pgfqpoint{0.000000in}{0.000000in}}{\pgfqpoint{0.027778in}{0.000000in}}{%
\pgfpathmoveto{\pgfqpoint{0.000000in}{0.000000in}}%
\pgfpathlineto{\pgfqpoint{0.027778in}{0.000000in}}%
\pgfusepath{stroke,fill}%
}%
\begin{pgfscope}%
\pgfsys@transformshift{5.632000in}{4.634429in}%
\pgfsys@useobject{currentmarker}{}%
\end{pgfscope}%
\end{pgfscope}%
\begin{pgfscope}%
\pgfsetbuttcap%
\pgfsetroundjoin%
\definecolor{currentfill}{rgb}{0.000000,0.000000,0.000000}%
\pgfsetfillcolor{currentfill}%
\pgfsetlinewidth{0.602250pt}%
\definecolor{currentstroke}{rgb}{0.000000,0.000000,0.000000}%
\pgfsetstrokecolor{currentstroke}%
\pgfsetdash{}{0pt}%
\pgfsys@defobject{currentmarker}{\pgfqpoint{0.000000in}{0.000000in}}{\pgfqpoint{0.027778in}{0.000000in}}{%
\pgfpathmoveto{\pgfqpoint{0.000000in}{0.000000in}}%
\pgfpathlineto{\pgfqpoint{0.027778in}{0.000000in}}%
\pgfusepath{stroke,fill}%
}%
\begin{pgfscope}%
\pgfsys@transformshift{5.632000in}{4.657395in}%
\pgfsys@useobject{currentmarker}{}%
\end{pgfscope}%
\end{pgfscope}%
\begin{pgfscope}%
\pgfsetbuttcap%
\pgfsetroundjoin%
\definecolor{currentfill}{rgb}{0.000000,0.000000,0.000000}%
\pgfsetfillcolor{currentfill}%
\pgfsetlinewidth{0.602250pt}%
\definecolor{currentstroke}{rgb}{0.000000,0.000000,0.000000}%
\pgfsetstrokecolor{currentstroke}%
\pgfsetdash{}{0pt}%
\pgfsys@defobject{currentmarker}{\pgfqpoint{0.000000in}{0.000000in}}{\pgfqpoint{0.027778in}{0.000000in}}{%
\pgfpathmoveto{\pgfqpoint{0.000000in}{0.000000in}}%
\pgfpathlineto{\pgfqpoint{0.027778in}{0.000000in}}%
\pgfusepath{stroke,fill}%
}%
\begin{pgfscope}%
\pgfsys@transformshift{5.632000in}{4.813092in}%
\pgfsys@useobject{currentmarker}{}%
\end{pgfscope}%
\end{pgfscope}%
\begin{pgfscope}%
\pgfsetbuttcap%
\pgfsetroundjoin%
\definecolor{currentfill}{rgb}{0.000000,0.000000,0.000000}%
\pgfsetfillcolor{currentfill}%
\pgfsetlinewidth{0.602250pt}%
\definecolor{currentstroke}{rgb}{0.000000,0.000000,0.000000}%
\pgfsetstrokecolor{currentstroke}%
\pgfsetdash{}{0pt}%
\pgfsys@defobject{currentmarker}{\pgfqpoint{0.000000in}{0.000000in}}{\pgfqpoint{0.027778in}{0.000000in}}{%
\pgfpathmoveto{\pgfqpoint{0.000000in}{0.000000in}}%
\pgfpathlineto{\pgfqpoint{0.027778in}{0.000000in}}%
\pgfusepath{stroke,fill}%
}%
\begin{pgfscope}%
\pgfsys@transformshift{5.632000in}{4.892152in}%
\pgfsys@useobject{currentmarker}{}%
\end{pgfscope}%
\end{pgfscope}%
\begin{pgfscope}%
\pgfsetbuttcap%
\pgfsetroundjoin%
\definecolor{currentfill}{rgb}{0.000000,0.000000,0.000000}%
\pgfsetfillcolor{currentfill}%
\pgfsetlinewidth{0.602250pt}%
\definecolor{currentstroke}{rgb}{0.000000,0.000000,0.000000}%
\pgfsetstrokecolor{currentstroke}%
\pgfsetdash{}{0pt}%
\pgfsys@defobject{currentmarker}{\pgfqpoint{0.000000in}{0.000000in}}{\pgfqpoint{0.027778in}{0.000000in}}{%
\pgfpathmoveto{\pgfqpoint{0.000000in}{0.000000in}}%
\pgfpathlineto{\pgfqpoint{0.027778in}{0.000000in}}%
\pgfusepath{stroke,fill}%
}%
\begin{pgfscope}%
\pgfsys@transformshift{5.632000in}{4.948245in}%
\pgfsys@useobject{currentmarker}{}%
\end{pgfscope}%
\end{pgfscope}%
\begin{pgfscope}%
\pgfsetbuttcap%
\pgfsetroundjoin%
\definecolor{currentfill}{rgb}{0.000000,0.000000,0.000000}%
\pgfsetfillcolor{currentfill}%
\pgfsetlinewidth{0.602250pt}%
\definecolor{currentstroke}{rgb}{0.000000,0.000000,0.000000}%
\pgfsetstrokecolor{currentstroke}%
\pgfsetdash{}{0pt}%
\pgfsys@defobject{currentmarker}{\pgfqpoint{0.000000in}{0.000000in}}{\pgfqpoint{0.027778in}{0.000000in}}{%
\pgfpathmoveto{\pgfqpoint{0.000000in}{0.000000in}}%
\pgfpathlineto{\pgfqpoint{0.027778in}{0.000000in}}%
\pgfusepath{stroke,fill}%
}%
\begin{pgfscope}%
\pgfsys@transformshift{5.632000in}{4.991755in}%
\pgfsys@useobject{currentmarker}{}%
\end{pgfscope}%
\end{pgfscope}%
\begin{pgfscope}%
\pgfsetbuttcap%
\pgfsetroundjoin%
\definecolor{currentfill}{rgb}{0.000000,0.000000,0.000000}%
\pgfsetfillcolor{currentfill}%
\pgfsetlinewidth{0.602250pt}%
\definecolor{currentstroke}{rgb}{0.000000,0.000000,0.000000}%
\pgfsetstrokecolor{currentstroke}%
\pgfsetdash{}{0pt}%
\pgfsys@defobject{currentmarker}{\pgfqpoint{0.000000in}{0.000000in}}{\pgfqpoint{0.027778in}{0.000000in}}{%
\pgfpathmoveto{\pgfqpoint{0.000000in}{0.000000in}}%
\pgfpathlineto{\pgfqpoint{0.027778in}{0.000000in}}%
\pgfusepath{stroke,fill}%
}%
\begin{pgfscope}%
\pgfsys@transformshift{5.632000in}{5.027305in}%
\pgfsys@useobject{currentmarker}{}%
\end{pgfscope}%
\end{pgfscope}%
\begin{pgfscope}%
\pgfsetbuttcap%
\pgfsetroundjoin%
\definecolor{currentfill}{rgb}{0.000000,0.000000,0.000000}%
\pgfsetfillcolor{currentfill}%
\pgfsetlinewidth{0.602250pt}%
\definecolor{currentstroke}{rgb}{0.000000,0.000000,0.000000}%
\pgfsetstrokecolor{currentstroke}%
\pgfsetdash{}{0pt}%
\pgfsys@defobject{currentmarker}{\pgfqpoint{0.000000in}{0.000000in}}{\pgfqpoint{0.027778in}{0.000000in}}{%
\pgfpathmoveto{\pgfqpoint{0.000000in}{0.000000in}}%
\pgfpathlineto{\pgfqpoint{0.027778in}{0.000000in}}%
\pgfusepath{stroke,fill}%
}%
\begin{pgfscope}%
\pgfsys@transformshift{5.632000in}{5.057362in}%
\pgfsys@useobject{currentmarker}{}%
\end{pgfscope}%
\end{pgfscope}%
\begin{pgfscope}%
\pgfsetbuttcap%
\pgfsetroundjoin%
\definecolor{currentfill}{rgb}{0.000000,0.000000,0.000000}%
\pgfsetfillcolor{currentfill}%
\pgfsetlinewidth{0.602250pt}%
\definecolor{currentstroke}{rgb}{0.000000,0.000000,0.000000}%
\pgfsetstrokecolor{currentstroke}%
\pgfsetdash{}{0pt}%
\pgfsys@defobject{currentmarker}{\pgfqpoint{0.000000in}{0.000000in}}{\pgfqpoint{0.027778in}{0.000000in}}{%
\pgfpathmoveto{\pgfqpoint{0.000000in}{0.000000in}}%
\pgfpathlineto{\pgfqpoint{0.027778in}{0.000000in}}%
\pgfusepath{stroke,fill}%
}%
\begin{pgfscope}%
\pgfsys@transformshift{5.632000in}{5.083398in}%
\pgfsys@useobject{currentmarker}{}%
\end{pgfscope}%
\end{pgfscope}%
\begin{pgfscope}%
\pgfsetbuttcap%
\pgfsetroundjoin%
\definecolor{currentfill}{rgb}{0.000000,0.000000,0.000000}%
\pgfsetfillcolor{currentfill}%
\pgfsetlinewidth{0.602250pt}%
\definecolor{currentstroke}{rgb}{0.000000,0.000000,0.000000}%
\pgfsetstrokecolor{currentstroke}%
\pgfsetdash{}{0pt}%
\pgfsys@defobject{currentmarker}{\pgfqpoint{0.000000in}{0.000000in}}{\pgfqpoint{0.027778in}{0.000000in}}{%
\pgfpathmoveto{\pgfqpoint{0.000000in}{0.000000in}}%
\pgfpathlineto{\pgfqpoint{0.027778in}{0.000000in}}%
\pgfusepath{stroke,fill}%
}%
\begin{pgfscope}%
\pgfsys@transformshift{5.632000in}{5.106364in}%
\pgfsys@useobject{currentmarker}{}%
\end{pgfscope}%
\end{pgfscope}%
\begin{pgfscope}%
\pgfsetbuttcap%
\pgfsetroundjoin%
\definecolor{currentfill}{rgb}{0.000000,0.000000,0.000000}%
\pgfsetfillcolor{currentfill}%
\pgfsetlinewidth{0.602250pt}%
\definecolor{currentstroke}{rgb}{0.000000,0.000000,0.000000}%
\pgfsetstrokecolor{currentstroke}%
\pgfsetdash{}{0pt}%
\pgfsys@defobject{currentmarker}{\pgfqpoint{0.000000in}{0.000000in}}{\pgfqpoint{0.027778in}{0.000000in}}{%
\pgfpathmoveto{\pgfqpoint{0.000000in}{0.000000in}}%
\pgfpathlineto{\pgfqpoint{0.027778in}{0.000000in}}%
\pgfusepath{stroke,fill}%
}%
\begin{pgfscope}%
\pgfsys@transformshift{5.632000in}{5.262061in}%
\pgfsys@useobject{currentmarker}{}%
\end{pgfscope}%
\end{pgfscope}%
\begin{pgfscope}%
\pgfsetbuttcap%
\pgfsetroundjoin%
\definecolor{currentfill}{rgb}{0.000000,0.000000,0.000000}%
\pgfsetfillcolor{currentfill}%
\pgfsetlinewidth{0.602250pt}%
\definecolor{currentstroke}{rgb}{0.000000,0.000000,0.000000}%
\pgfsetstrokecolor{currentstroke}%
\pgfsetdash{}{0pt}%
\pgfsys@defobject{currentmarker}{\pgfqpoint{0.000000in}{0.000000in}}{\pgfqpoint{0.027778in}{0.000000in}}{%
\pgfpathmoveto{\pgfqpoint{0.000000in}{0.000000in}}%
\pgfpathlineto{\pgfqpoint{0.027778in}{0.000000in}}%
\pgfusepath{stroke,fill}%
}%
\begin{pgfscope}%
\pgfsys@transformshift{5.632000in}{5.341121in}%
\pgfsys@useobject{currentmarker}{}%
\end{pgfscope}%
\end{pgfscope}%
\begin{pgfscope}%
\pgfsetbuttcap%
\pgfsetroundjoin%
\definecolor{currentfill}{rgb}{0.000000,0.000000,0.000000}%
\pgfsetfillcolor{currentfill}%
\pgfsetlinewidth{0.602250pt}%
\definecolor{currentstroke}{rgb}{0.000000,0.000000,0.000000}%
\pgfsetstrokecolor{currentstroke}%
\pgfsetdash{}{0pt}%
\pgfsys@defobject{currentmarker}{\pgfqpoint{0.000000in}{0.000000in}}{\pgfqpoint{0.027778in}{0.000000in}}{%
\pgfpathmoveto{\pgfqpoint{0.000000in}{0.000000in}}%
\pgfpathlineto{\pgfqpoint{0.027778in}{0.000000in}}%
\pgfusepath{stroke,fill}%
}%
\begin{pgfscope}%
\pgfsys@transformshift{5.632000in}{5.397215in}%
\pgfsys@useobject{currentmarker}{}%
\end{pgfscope}%
\end{pgfscope}%
\begin{pgfscope}%
\pgfsetbuttcap%
\pgfsetroundjoin%
\definecolor{currentfill}{rgb}{0.000000,0.000000,0.000000}%
\pgfsetfillcolor{currentfill}%
\pgfsetlinewidth{0.602250pt}%
\definecolor{currentstroke}{rgb}{0.000000,0.000000,0.000000}%
\pgfsetstrokecolor{currentstroke}%
\pgfsetdash{}{0pt}%
\pgfsys@defobject{currentmarker}{\pgfqpoint{0.000000in}{0.000000in}}{\pgfqpoint{0.027778in}{0.000000in}}{%
\pgfpathmoveto{\pgfqpoint{0.000000in}{0.000000in}}%
\pgfpathlineto{\pgfqpoint{0.027778in}{0.000000in}}%
\pgfusepath{stroke,fill}%
}%
\begin{pgfscope}%
\pgfsys@transformshift{5.632000in}{5.440724in}%
\pgfsys@useobject{currentmarker}{}%
\end{pgfscope}%
\end{pgfscope}%
\begin{pgfscope}%
\pgfsetbuttcap%
\pgfsetroundjoin%
\definecolor{currentfill}{rgb}{0.000000,0.000000,0.000000}%
\pgfsetfillcolor{currentfill}%
\pgfsetlinewidth{0.602250pt}%
\definecolor{currentstroke}{rgb}{0.000000,0.000000,0.000000}%
\pgfsetstrokecolor{currentstroke}%
\pgfsetdash{}{0pt}%
\pgfsys@defobject{currentmarker}{\pgfqpoint{0.000000in}{0.000000in}}{\pgfqpoint{0.027778in}{0.000000in}}{%
\pgfpathmoveto{\pgfqpoint{0.000000in}{0.000000in}}%
\pgfpathlineto{\pgfqpoint{0.027778in}{0.000000in}}%
\pgfusepath{stroke,fill}%
}%
\begin{pgfscope}%
\pgfsys@transformshift{5.632000in}{5.476274in}%
\pgfsys@useobject{currentmarker}{}%
\end{pgfscope}%
\end{pgfscope}%
\begin{pgfscope}%
\pgfsetbuttcap%
\pgfsetroundjoin%
\definecolor{currentfill}{rgb}{0.000000,0.000000,0.000000}%
\pgfsetfillcolor{currentfill}%
\pgfsetlinewidth{0.602250pt}%
\definecolor{currentstroke}{rgb}{0.000000,0.000000,0.000000}%
\pgfsetstrokecolor{currentstroke}%
\pgfsetdash{}{0pt}%
\pgfsys@defobject{currentmarker}{\pgfqpoint{0.000000in}{0.000000in}}{\pgfqpoint{0.027778in}{0.000000in}}{%
\pgfpathmoveto{\pgfqpoint{0.000000in}{0.000000in}}%
\pgfpathlineto{\pgfqpoint{0.027778in}{0.000000in}}%
\pgfusepath{stroke,fill}%
}%
\begin{pgfscope}%
\pgfsys@transformshift{5.632000in}{5.506331in}%
\pgfsys@useobject{currentmarker}{}%
\end{pgfscope}%
\end{pgfscope}%
\begin{pgfscope}%
\pgfsetbuttcap%
\pgfsetroundjoin%
\definecolor{currentfill}{rgb}{0.000000,0.000000,0.000000}%
\pgfsetfillcolor{currentfill}%
\pgfsetlinewidth{0.602250pt}%
\definecolor{currentstroke}{rgb}{0.000000,0.000000,0.000000}%
\pgfsetstrokecolor{currentstroke}%
\pgfsetdash{}{0pt}%
\pgfsys@defobject{currentmarker}{\pgfqpoint{0.000000in}{0.000000in}}{\pgfqpoint{0.027778in}{0.000000in}}{%
\pgfpathmoveto{\pgfqpoint{0.000000in}{0.000000in}}%
\pgfpathlineto{\pgfqpoint{0.027778in}{0.000000in}}%
\pgfusepath{stroke,fill}%
}%
\begin{pgfscope}%
\pgfsys@transformshift{5.632000in}{5.532368in}%
\pgfsys@useobject{currentmarker}{}%
\end{pgfscope}%
\end{pgfscope}%
\begin{pgfscope}%
\pgfsetbuttcap%
\pgfsetroundjoin%
\definecolor{currentfill}{rgb}{0.000000,0.000000,0.000000}%
\pgfsetfillcolor{currentfill}%
\pgfsetlinewidth{0.602250pt}%
\definecolor{currentstroke}{rgb}{0.000000,0.000000,0.000000}%
\pgfsetstrokecolor{currentstroke}%
\pgfsetdash{}{0pt}%
\pgfsys@defobject{currentmarker}{\pgfqpoint{0.000000in}{0.000000in}}{\pgfqpoint{0.027778in}{0.000000in}}{%
\pgfpathmoveto{\pgfqpoint{0.000000in}{0.000000in}}%
\pgfpathlineto{\pgfqpoint{0.027778in}{0.000000in}}%
\pgfusepath{stroke,fill}%
}%
\begin{pgfscope}%
\pgfsys@transformshift{5.632000in}{5.555334in}%
\pgfsys@useobject{currentmarker}{}%
\end{pgfscope}%
\end{pgfscope}%
\begin{pgfscope}%
\pgfsetbuttcap%
\pgfsetroundjoin%
\definecolor{currentfill}{rgb}{0.000000,0.000000,0.000000}%
\pgfsetfillcolor{currentfill}%
\pgfsetlinewidth{0.602250pt}%
\definecolor{currentstroke}{rgb}{0.000000,0.000000,0.000000}%
\pgfsetstrokecolor{currentstroke}%
\pgfsetdash{}{0pt}%
\pgfsys@defobject{currentmarker}{\pgfqpoint{0.000000in}{0.000000in}}{\pgfqpoint{0.027778in}{0.000000in}}{%
\pgfpathmoveto{\pgfqpoint{0.000000in}{0.000000in}}%
\pgfpathlineto{\pgfqpoint{0.027778in}{0.000000in}}%
\pgfusepath{stroke,fill}%
}%
\begin{pgfscope}%
\pgfsys@transformshift{5.632000in}{5.711031in}%
\pgfsys@useobject{currentmarker}{}%
\end{pgfscope}%
\end{pgfscope}%
\begin{pgfscope}%
\pgfsetbuttcap%
\pgfsetroundjoin%
\definecolor{currentfill}{rgb}{0.000000,0.000000,0.000000}%
\pgfsetfillcolor{currentfill}%
\pgfsetlinewidth{0.602250pt}%
\definecolor{currentstroke}{rgb}{0.000000,0.000000,0.000000}%
\pgfsetstrokecolor{currentstroke}%
\pgfsetdash{}{0pt}%
\pgfsys@defobject{currentmarker}{\pgfqpoint{0.000000in}{0.000000in}}{\pgfqpoint{0.027778in}{0.000000in}}{%
\pgfpathmoveto{\pgfqpoint{0.000000in}{0.000000in}}%
\pgfpathlineto{\pgfqpoint{0.027778in}{0.000000in}}%
\pgfusepath{stroke,fill}%
}%
\begin{pgfscope}%
\pgfsys@transformshift{5.632000in}{5.790090in}%
\pgfsys@useobject{currentmarker}{}%
\end{pgfscope}%
\end{pgfscope}%
\begin{pgfscope}%
\pgfsetbuttcap%
\pgfsetroundjoin%
\definecolor{currentfill}{rgb}{0.000000,0.000000,0.000000}%
\pgfsetfillcolor{currentfill}%
\pgfsetlinewidth{0.602250pt}%
\definecolor{currentstroke}{rgb}{0.000000,0.000000,0.000000}%
\pgfsetstrokecolor{currentstroke}%
\pgfsetdash{}{0pt}%
\pgfsys@defobject{currentmarker}{\pgfqpoint{0.000000in}{0.000000in}}{\pgfqpoint{0.027778in}{0.000000in}}{%
\pgfpathmoveto{\pgfqpoint{0.000000in}{0.000000in}}%
\pgfpathlineto{\pgfqpoint{0.027778in}{0.000000in}}%
\pgfusepath{stroke,fill}%
}%
\begin{pgfscope}%
\pgfsys@transformshift{5.632000in}{5.846184in}%
\pgfsys@useobject{currentmarker}{}%
\end{pgfscope}%
\end{pgfscope}%
\begin{pgfscope}%
\pgfsetbuttcap%
\pgfsetroundjoin%
\definecolor{currentfill}{rgb}{0.000000,0.000000,0.000000}%
\pgfsetfillcolor{currentfill}%
\pgfsetlinewidth{0.602250pt}%
\definecolor{currentstroke}{rgb}{0.000000,0.000000,0.000000}%
\pgfsetstrokecolor{currentstroke}%
\pgfsetdash{}{0pt}%
\pgfsys@defobject{currentmarker}{\pgfqpoint{0.000000in}{0.000000in}}{\pgfqpoint{0.027778in}{0.000000in}}{%
\pgfpathmoveto{\pgfqpoint{0.000000in}{0.000000in}}%
\pgfpathlineto{\pgfqpoint{0.027778in}{0.000000in}}%
\pgfusepath{stroke,fill}%
}%
\begin{pgfscope}%
\pgfsys@transformshift{5.632000in}{5.889694in}%
\pgfsys@useobject{currentmarker}{}%
\end{pgfscope}%
\end{pgfscope}%
\begin{pgfscope}%
\pgfsetbuttcap%
\pgfsetroundjoin%
\definecolor{currentfill}{rgb}{0.000000,0.000000,0.000000}%
\pgfsetfillcolor{currentfill}%
\pgfsetlinewidth{0.602250pt}%
\definecolor{currentstroke}{rgb}{0.000000,0.000000,0.000000}%
\pgfsetstrokecolor{currentstroke}%
\pgfsetdash{}{0pt}%
\pgfsys@defobject{currentmarker}{\pgfqpoint{0.000000in}{0.000000in}}{\pgfqpoint{0.027778in}{0.000000in}}{%
\pgfpathmoveto{\pgfqpoint{0.000000in}{0.000000in}}%
\pgfpathlineto{\pgfqpoint{0.027778in}{0.000000in}}%
\pgfusepath{stroke,fill}%
}%
\begin{pgfscope}%
\pgfsys@transformshift{5.632000in}{5.925243in}%
\pgfsys@useobject{currentmarker}{}%
\end{pgfscope}%
\end{pgfscope}%
\begin{pgfscope}%
\pgfsetbuttcap%
\pgfsetroundjoin%
\definecolor{currentfill}{rgb}{0.000000,0.000000,0.000000}%
\pgfsetfillcolor{currentfill}%
\pgfsetlinewidth{0.602250pt}%
\definecolor{currentstroke}{rgb}{0.000000,0.000000,0.000000}%
\pgfsetstrokecolor{currentstroke}%
\pgfsetdash{}{0pt}%
\pgfsys@defobject{currentmarker}{\pgfqpoint{0.000000in}{0.000000in}}{\pgfqpoint{0.027778in}{0.000000in}}{%
\pgfpathmoveto{\pgfqpoint{0.000000in}{0.000000in}}%
\pgfpathlineto{\pgfqpoint{0.027778in}{0.000000in}}%
\pgfusepath{stroke,fill}%
}%
\begin{pgfscope}%
\pgfsys@transformshift{5.632000in}{5.955301in}%
\pgfsys@useobject{currentmarker}{}%
\end{pgfscope}%
\end{pgfscope}%
\begin{pgfscope}%
\pgfsetbuttcap%
\pgfsetroundjoin%
\definecolor{currentfill}{rgb}{0.000000,0.000000,0.000000}%
\pgfsetfillcolor{currentfill}%
\pgfsetlinewidth{0.602250pt}%
\definecolor{currentstroke}{rgb}{0.000000,0.000000,0.000000}%
\pgfsetstrokecolor{currentstroke}%
\pgfsetdash{}{0pt}%
\pgfsys@defobject{currentmarker}{\pgfqpoint{0.000000in}{0.000000in}}{\pgfqpoint{0.027778in}{0.000000in}}{%
\pgfpathmoveto{\pgfqpoint{0.000000in}{0.000000in}}%
\pgfpathlineto{\pgfqpoint{0.027778in}{0.000000in}}%
\pgfusepath{stroke,fill}%
}%
\begin{pgfscope}%
\pgfsys@transformshift{5.632000in}{5.981337in}%
\pgfsys@useobject{currentmarker}{}%
\end{pgfscope}%
\end{pgfscope}%
\begin{pgfscope}%
\pgfsetbuttcap%
\pgfsetroundjoin%
\definecolor{currentfill}{rgb}{0.000000,0.000000,0.000000}%
\pgfsetfillcolor{currentfill}%
\pgfsetlinewidth{0.602250pt}%
\definecolor{currentstroke}{rgb}{0.000000,0.000000,0.000000}%
\pgfsetstrokecolor{currentstroke}%
\pgfsetdash{}{0pt}%
\pgfsys@defobject{currentmarker}{\pgfqpoint{0.000000in}{0.000000in}}{\pgfqpoint{0.027778in}{0.000000in}}{%
\pgfpathmoveto{\pgfqpoint{0.000000in}{0.000000in}}%
\pgfpathlineto{\pgfqpoint{0.027778in}{0.000000in}}%
\pgfusepath{stroke,fill}%
}%
\begin{pgfscope}%
\pgfsys@transformshift{5.632000in}{6.004303in}%
\pgfsys@useobject{currentmarker}{}%
\end{pgfscope}%
\end{pgfscope}%
\begin{pgfscope}%
\pgfsetbuttcap%
\pgfsetroundjoin%
\definecolor{currentfill}{rgb}{0.000000,0.000000,0.000000}%
\pgfsetfillcolor{currentfill}%
\pgfsetlinewidth{0.602250pt}%
\definecolor{currentstroke}{rgb}{0.000000,0.000000,0.000000}%
\pgfsetstrokecolor{currentstroke}%
\pgfsetdash{}{0pt}%
\pgfsys@defobject{currentmarker}{\pgfqpoint{0.000000in}{0.000000in}}{\pgfqpoint{0.027778in}{0.000000in}}{%
\pgfpathmoveto{\pgfqpoint{0.000000in}{0.000000in}}%
\pgfpathlineto{\pgfqpoint{0.027778in}{0.000000in}}%
\pgfusepath{stroke,fill}%
}%
\begin{pgfscope}%
\pgfsys@transformshift{5.632000in}{6.160000in}%
\pgfsys@useobject{currentmarker}{}%
\end{pgfscope}%
\end{pgfscope}%
\begin{pgfscope}%
\definecolor{textcolor}{rgb}{0.000000,0.000000,0.000000}%
\pgfsetstrokecolor{textcolor}%
\pgfsetfillcolor{textcolor}%
\pgftext[x=6.072780in,y=4.970000in,,top,rotate=90.000000]{\color{textcolor}\sffamily\fontsize{10.000000}{12.000000}\selectfont \(\displaystyle -dQ_{e^-}/dx/dy/dz \, \mathrm{[pC/\mu m^3]}\)}%
\end{pgfscope}%
\begin{pgfscope}%
\pgfsetrectcap%
\pgfsetmiterjoin%
\pgfsetlinewidth{0.803000pt}%
\definecolor{currentstroke}{rgb}{0.000000,0.000000,0.000000}%
\pgfsetstrokecolor{currentstroke}%
\pgfsetdash{}{0pt}%
\pgfpathmoveto{\pgfqpoint{5.440000in}{3.780000in}}%
\pgfpathlineto{\pgfqpoint{5.536000in}{3.780000in}}%
\pgfpathlineto{\pgfqpoint{5.632000in}{3.780000in}}%
\pgfpathlineto{\pgfqpoint{5.632000in}{6.160000in}}%
\pgfpathlineto{\pgfqpoint{5.536000in}{6.160000in}}%
\pgfpathlineto{\pgfqpoint{5.440000in}{6.160000in}}%
\pgfpathlineto{\pgfqpoint{5.440000in}{3.780000in}}%
\pgfpathclose%
\pgfusepath{stroke}%
\end{pgfscope}%
\begin{pgfscope}%
\pgfsetbuttcap%
\pgfsetmiterjoin%
\definecolor{currentfill}{rgb}{1.000000,1.000000,1.000000}%
\pgfsetfillcolor{currentfill}%
\pgfsetlinewidth{0.000000pt}%
\definecolor{currentstroke}{rgb}{0.000000,0.000000,0.000000}%
\pgfsetstrokecolor{currentstroke}%
\pgfsetstrokeopacity{0.000000}%
\pgfsetdash{}{0pt}%
\pgfpathmoveto{\pgfqpoint{5.440000in}{0.840000in}}%
\pgfpathlineto{\pgfqpoint{5.632000in}{0.840000in}}%
\pgfpathlineto{\pgfqpoint{5.632000in}{3.220000in}}%
\pgfpathlineto{\pgfqpoint{5.440000in}{3.220000in}}%
\pgfpathlineto{\pgfqpoint{5.440000in}{0.840000in}}%
\pgfpathclose%
\pgfusepath{fill}%
\end{pgfscope}%
\begin{pgfscope}%
\pgfpathrectangle{\pgfqpoint{5.440000in}{0.840000in}}{\pgfqpoint{0.192000in}{2.380000in}}%
\pgfusepath{clip}%
\pgfsetbuttcap%
\pgfsetmiterjoin%
\definecolor{currentfill}{rgb}{1.000000,1.000000,1.000000}%
\pgfsetfillcolor{currentfill}%
\pgfsetlinewidth{0.010037pt}%
\definecolor{currentstroke}{rgb}{1.000000,1.000000,1.000000}%
\pgfsetstrokecolor{currentstroke}%
\pgfsetdash{}{0pt}%
\pgfusepath{stroke,fill}%
\end{pgfscope}%
\begin{pgfscope}%
\pgfsys@transformshift{5.444444in}{0.833333in}%
\pgftext[left,bottom]{\includegraphics[interpolate=true,width=0.194444in,height=2.388889in]{E_y_hist-img3.png}}%
\end{pgfscope}%
\begin{pgfscope}%
\pgfsetbuttcap%
\pgfsetroundjoin%
\definecolor{currentfill}{rgb}{0.000000,0.000000,0.000000}%
\pgfsetfillcolor{currentfill}%
\pgfsetlinewidth{0.803000pt}%
\definecolor{currentstroke}{rgb}{0.000000,0.000000,0.000000}%
\pgfsetstrokecolor{currentstroke}%
\pgfsetdash{}{0pt}%
\pgfsys@defobject{currentmarker}{\pgfqpoint{0.000000in}{0.000000in}}{\pgfqpoint{0.048611in}{0.000000in}}{%
\pgfpathmoveto{\pgfqpoint{0.000000in}{0.000000in}}%
\pgfpathlineto{\pgfqpoint{0.048611in}{0.000000in}}%
\pgfusepath{stroke,fill}%
}%
\begin{pgfscope}%
\pgfsys@transformshift{5.632000in}{0.840000in}%
\pgfsys@useobject{currentmarker}{}%
\end{pgfscope}%
\end{pgfscope}%
\begin{pgfscope}%
\definecolor{textcolor}{rgb}{0.000000,0.000000,0.000000}%
\pgfsetstrokecolor{textcolor}%
\pgfsetfillcolor{textcolor}%
\pgftext[x=5.729222in, y=0.791775in, left, base]{\color{textcolor}\sffamily\fontsize{10.000000}{12.000000}\selectfont \(\displaystyle {\ensuremath{-}300}\)}%
\end{pgfscope}%
\begin{pgfscope}%
\pgfsetbuttcap%
\pgfsetroundjoin%
\definecolor{currentfill}{rgb}{0.000000,0.000000,0.000000}%
\pgfsetfillcolor{currentfill}%
\pgfsetlinewidth{0.803000pt}%
\definecolor{currentstroke}{rgb}{0.000000,0.000000,0.000000}%
\pgfsetstrokecolor{currentstroke}%
\pgfsetdash{}{0pt}%
\pgfsys@defobject{currentmarker}{\pgfqpoint{0.000000in}{0.000000in}}{\pgfqpoint{0.048611in}{0.000000in}}{%
\pgfpathmoveto{\pgfqpoint{0.000000in}{0.000000in}}%
\pgfpathlineto{\pgfqpoint{0.048611in}{0.000000in}}%
\pgfusepath{stroke,fill}%
}%
\begin{pgfscope}%
\pgfsys@transformshift{5.632000in}{1.236667in}%
\pgfsys@useobject{currentmarker}{}%
\end{pgfscope}%
\end{pgfscope}%
\begin{pgfscope}%
\definecolor{textcolor}{rgb}{0.000000,0.000000,0.000000}%
\pgfsetstrokecolor{textcolor}%
\pgfsetfillcolor{textcolor}%
\pgftext[x=5.729222in, y=1.188441in, left, base]{\color{textcolor}\sffamily\fontsize{10.000000}{12.000000}\selectfont \(\displaystyle {\ensuremath{-}200}\)}%
\end{pgfscope}%
\begin{pgfscope}%
\pgfsetbuttcap%
\pgfsetroundjoin%
\definecolor{currentfill}{rgb}{0.000000,0.000000,0.000000}%
\pgfsetfillcolor{currentfill}%
\pgfsetlinewidth{0.803000pt}%
\definecolor{currentstroke}{rgb}{0.000000,0.000000,0.000000}%
\pgfsetstrokecolor{currentstroke}%
\pgfsetdash{}{0pt}%
\pgfsys@defobject{currentmarker}{\pgfqpoint{0.000000in}{0.000000in}}{\pgfqpoint{0.048611in}{0.000000in}}{%
\pgfpathmoveto{\pgfqpoint{0.000000in}{0.000000in}}%
\pgfpathlineto{\pgfqpoint{0.048611in}{0.000000in}}%
\pgfusepath{stroke,fill}%
}%
\begin{pgfscope}%
\pgfsys@transformshift{5.632000in}{1.633333in}%
\pgfsys@useobject{currentmarker}{}%
\end{pgfscope}%
\end{pgfscope}%
\begin{pgfscope}%
\definecolor{textcolor}{rgb}{0.000000,0.000000,0.000000}%
\pgfsetstrokecolor{textcolor}%
\pgfsetfillcolor{textcolor}%
\pgftext[x=5.729222in, y=1.585108in, left, base]{\color{textcolor}\sffamily\fontsize{10.000000}{12.000000}\selectfont \(\displaystyle {\ensuremath{-}100}\)}%
\end{pgfscope}%
\begin{pgfscope}%
\pgfsetbuttcap%
\pgfsetroundjoin%
\definecolor{currentfill}{rgb}{0.000000,0.000000,0.000000}%
\pgfsetfillcolor{currentfill}%
\pgfsetlinewidth{0.803000pt}%
\definecolor{currentstroke}{rgb}{0.000000,0.000000,0.000000}%
\pgfsetstrokecolor{currentstroke}%
\pgfsetdash{}{0pt}%
\pgfsys@defobject{currentmarker}{\pgfqpoint{0.000000in}{0.000000in}}{\pgfqpoint{0.048611in}{0.000000in}}{%
\pgfpathmoveto{\pgfqpoint{0.000000in}{0.000000in}}%
\pgfpathlineto{\pgfqpoint{0.048611in}{0.000000in}}%
\pgfusepath{stroke,fill}%
}%
\begin{pgfscope}%
\pgfsys@transformshift{5.632000in}{2.030000in}%
\pgfsys@useobject{currentmarker}{}%
\end{pgfscope}%
\end{pgfscope}%
\begin{pgfscope}%
\definecolor{textcolor}{rgb}{0.000000,0.000000,0.000000}%
\pgfsetstrokecolor{textcolor}%
\pgfsetfillcolor{textcolor}%
\pgftext[x=5.729222in, y=1.981775in, left, base]{\color{textcolor}\sffamily\fontsize{10.000000}{12.000000}\selectfont \(\displaystyle {0}\)}%
\end{pgfscope}%
\begin{pgfscope}%
\pgfsetbuttcap%
\pgfsetroundjoin%
\definecolor{currentfill}{rgb}{0.000000,0.000000,0.000000}%
\pgfsetfillcolor{currentfill}%
\pgfsetlinewidth{0.803000pt}%
\definecolor{currentstroke}{rgb}{0.000000,0.000000,0.000000}%
\pgfsetstrokecolor{currentstroke}%
\pgfsetdash{}{0pt}%
\pgfsys@defobject{currentmarker}{\pgfqpoint{0.000000in}{0.000000in}}{\pgfqpoint{0.048611in}{0.000000in}}{%
\pgfpathmoveto{\pgfqpoint{0.000000in}{0.000000in}}%
\pgfpathlineto{\pgfqpoint{0.048611in}{0.000000in}}%
\pgfusepath{stroke,fill}%
}%
\begin{pgfscope}%
\pgfsys@transformshift{5.632000in}{2.426667in}%
\pgfsys@useobject{currentmarker}{}%
\end{pgfscope}%
\end{pgfscope}%
\begin{pgfscope}%
\definecolor{textcolor}{rgb}{0.000000,0.000000,0.000000}%
\pgfsetstrokecolor{textcolor}%
\pgfsetfillcolor{textcolor}%
\pgftext[x=5.729222in, y=2.378441in, left, base]{\color{textcolor}\sffamily\fontsize{10.000000}{12.000000}\selectfont \(\displaystyle {100}\)}%
\end{pgfscope}%
\begin{pgfscope}%
\pgfsetbuttcap%
\pgfsetroundjoin%
\definecolor{currentfill}{rgb}{0.000000,0.000000,0.000000}%
\pgfsetfillcolor{currentfill}%
\pgfsetlinewidth{0.803000pt}%
\definecolor{currentstroke}{rgb}{0.000000,0.000000,0.000000}%
\pgfsetstrokecolor{currentstroke}%
\pgfsetdash{}{0pt}%
\pgfsys@defobject{currentmarker}{\pgfqpoint{0.000000in}{0.000000in}}{\pgfqpoint{0.048611in}{0.000000in}}{%
\pgfpathmoveto{\pgfqpoint{0.000000in}{0.000000in}}%
\pgfpathlineto{\pgfqpoint{0.048611in}{0.000000in}}%
\pgfusepath{stroke,fill}%
}%
\begin{pgfscope}%
\pgfsys@transformshift{5.632000in}{2.823333in}%
\pgfsys@useobject{currentmarker}{}%
\end{pgfscope}%
\end{pgfscope}%
\begin{pgfscope}%
\definecolor{textcolor}{rgb}{0.000000,0.000000,0.000000}%
\pgfsetstrokecolor{textcolor}%
\pgfsetfillcolor{textcolor}%
\pgftext[x=5.729222in, y=2.775108in, left, base]{\color{textcolor}\sffamily\fontsize{10.000000}{12.000000}\selectfont \(\displaystyle {200}\)}%
\end{pgfscope}%
\begin{pgfscope}%
\pgfsetbuttcap%
\pgfsetroundjoin%
\definecolor{currentfill}{rgb}{0.000000,0.000000,0.000000}%
\pgfsetfillcolor{currentfill}%
\pgfsetlinewidth{0.803000pt}%
\definecolor{currentstroke}{rgb}{0.000000,0.000000,0.000000}%
\pgfsetstrokecolor{currentstroke}%
\pgfsetdash{}{0pt}%
\pgfsys@defobject{currentmarker}{\pgfqpoint{0.000000in}{0.000000in}}{\pgfqpoint{0.048611in}{0.000000in}}{%
\pgfpathmoveto{\pgfqpoint{0.000000in}{0.000000in}}%
\pgfpathlineto{\pgfqpoint{0.048611in}{0.000000in}}%
\pgfusepath{stroke,fill}%
}%
\begin{pgfscope}%
\pgfsys@transformshift{5.632000in}{3.220000in}%
\pgfsys@useobject{currentmarker}{}%
\end{pgfscope}%
\end{pgfscope}%
\begin{pgfscope}%
\definecolor{textcolor}{rgb}{0.000000,0.000000,0.000000}%
\pgfsetstrokecolor{textcolor}%
\pgfsetfillcolor{textcolor}%
\pgftext[x=5.729222in, y=3.171775in, left, base]{\color{textcolor}\sffamily\fontsize{10.000000}{12.000000}\selectfont \(\displaystyle {300}\)}%
\end{pgfscope}%
\begin{pgfscope}%
\definecolor{textcolor}{rgb}{0.000000,0.000000,0.000000}%
\pgfsetstrokecolor{textcolor}%
\pgfsetfillcolor{textcolor}%
\pgftext[x=6.101137in,y=2.030000in,,top,rotate=90.000000]{\color{textcolor}\sffamily\fontsize{10.000000}{12.000000}\selectfont \(\displaystyle E_{\parallel} \, \mathrm{[GV/m]}\)}%
\end{pgfscope}%
\begin{pgfscope}%
\pgfsetrectcap%
\pgfsetmiterjoin%
\pgfsetlinewidth{0.803000pt}%
\definecolor{currentstroke}{rgb}{0.000000,0.000000,0.000000}%
\pgfsetstrokecolor{currentstroke}%
\pgfsetdash{}{0pt}%
\pgfpathmoveto{\pgfqpoint{5.440000in}{0.840000in}}%
\pgfpathlineto{\pgfqpoint{5.536000in}{0.840000in}}%
\pgfpathlineto{\pgfqpoint{5.632000in}{0.840000in}}%
\pgfpathlineto{\pgfqpoint{5.632000in}{3.220000in}}%
\pgfpathlineto{\pgfqpoint{5.536000in}{3.220000in}}%
\pgfpathlineto{\pgfqpoint{5.440000in}{3.220000in}}%
\pgfpathlineto{\pgfqpoint{5.440000in}{0.840000in}}%
\pgfpathclose%
\pgfusepath{stroke}%
\end{pgfscope}%
\begin{pgfscope}%
\pgfsetbuttcap%
\pgfsetroundjoin%
\definecolor{currentfill}{rgb}{0.000000,0.000000,0.000000}%
\pgfsetfillcolor{currentfill}%
\pgfsetlinewidth{0.803000pt}%
\definecolor{currentstroke}{rgb}{0.000000,0.000000,0.000000}%
\pgfsetstrokecolor{currentstroke}%
\pgfsetdash{}{0pt}%
\pgfsys@defobject{currentmarker}{\pgfqpoint{0.000000in}{0.000000in}}{\pgfqpoint{0.048611in}{0.000000in}}{%
\pgfpathmoveto{\pgfqpoint{0.000000in}{0.000000in}}%
\pgfpathlineto{\pgfqpoint{0.048611in}{0.000000in}}%
\pgfusepath{stroke,fill}%
}%
\begin{pgfscope}%
\pgfsys@transformshift{4.672000in}{1.175935in}%
\pgfsys@useobject{currentmarker}{}%
\end{pgfscope}%
\end{pgfscope}%
\begin{pgfscope}%
\definecolor{textcolor}{rgb}{0.000000,0.000000,0.000000}%
\pgfsetstrokecolor{textcolor}%
\pgfsetfillcolor{textcolor}%
\pgftext[x=4.769222in, y=1.127709in, left, base]{\color{textcolor}\sffamily\fontsize{10.000000}{12.000000}\selectfont \(\displaystyle {\ensuremath{-}600}\)}%
\end{pgfscope}%
\begin{pgfscope}%
\pgfsetbuttcap%
\pgfsetroundjoin%
\definecolor{currentfill}{rgb}{0.000000,0.000000,0.000000}%
\pgfsetfillcolor{currentfill}%
\pgfsetlinewidth{0.803000pt}%
\definecolor{currentstroke}{rgb}{0.000000,0.000000,0.000000}%
\pgfsetstrokecolor{currentstroke}%
\pgfsetdash{}{0pt}%
\pgfsys@defobject{currentmarker}{\pgfqpoint{0.000000in}{0.000000in}}{\pgfqpoint{0.048611in}{0.000000in}}{%
\pgfpathmoveto{\pgfqpoint{0.000000in}{0.000000in}}%
\pgfpathlineto{\pgfqpoint{0.048611in}{0.000000in}}%
\pgfusepath{stroke,fill}%
}%
\begin{pgfscope}%
\pgfsys@transformshift{4.672000in}{1.539982in}%
\pgfsys@useobject{currentmarker}{}%
\end{pgfscope}%
\end{pgfscope}%
\begin{pgfscope}%
\definecolor{textcolor}{rgb}{0.000000,0.000000,0.000000}%
\pgfsetstrokecolor{textcolor}%
\pgfsetfillcolor{textcolor}%
\pgftext[x=4.769222in, y=1.491756in, left, base]{\color{textcolor}\sffamily\fontsize{10.000000}{12.000000}\selectfont \(\displaystyle {\ensuremath{-}400}\)}%
\end{pgfscope}%
\begin{pgfscope}%
\pgfsetbuttcap%
\pgfsetroundjoin%
\definecolor{currentfill}{rgb}{0.000000,0.000000,0.000000}%
\pgfsetfillcolor{currentfill}%
\pgfsetlinewidth{0.803000pt}%
\definecolor{currentstroke}{rgb}{0.000000,0.000000,0.000000}%
\pgfsetstrokecolor{currentstroke}%
\pgfsetdash{}{0pt}%
\pgfsys@defobject{currentmarker}{\pgfqpoint{0.000000in}{0.000000in}}{\pgfqpoint{0.048611in}{0.000000in}}{%
\pgfpathmoveto{\pgfqpoint{0.000000in}{0.000000in}}%
\pgfpathlineto{\pgfqpoint{0.048611in}{0.000000in}}%
\pgfusepath{stroke,fill}%
}%
\begin{pgfscope}%
\pgfsys@transformshift{4.672000in}{1.904028in}%
\pgfsys@useobject{currentmarker}{}%
\end{pgfscope}%
\end{pgfscope}%
\begin{pgfscope}%
\definecolor{textcolor}{rgb}{0.000000,0.000000,0.000000}%
\pgfsetstrokecolor{textcolor}%
\pgfsetfillcolor{textcolor}%
\pgftext[x=4.769222in, y=1.855803in, left, base]{\color{textcolor}\sffamily\fontsize{10.000000}{12.000000}\selectfont \(\displaystyle {\ensuremath{-}200}\)}%
\end{pgfscope}%
\begin{pgfscope}%
\pgfsetbuttcap%
\pgfsetroundjoin%
\definecolor{currentfill}{rgb}{0.000000,0.000000,0.000000}%
\pgfsetfillcolor{currentfill}%
\pgfsetlinewidth{0.803000pt}%
\definecolor{currentstroke}{rgb}{0.000000,0.000000,0.000000}%
\pgfsetstrokecolor{currentstroke}%
\pgfsetdash{}{0pt}%
\pgfsys@defobject{currentmarker}{\pgfqpoint{0.000000in}{0.000000in}}{\pgfqpoint{0.048611in}{0.000000in}}{%
\pgfpathmoveto{\pgfqpoint{0.000000in}{0.000000in}}%
\pgfpathlineto{\pgfqpoint{0.048611in}{0.000000in}}%
\pgfusepath{stroke,fill}%
}%
\begin{pgfscope}%
\pgfsys@transformshift{4.672000in}{2.268075in}%
\pgfsys@useobject{currentmarker}{}%
\end{pgfscope}%
\end{pgfscope}%
\begin{pgfscope}%
\definecolor{textcolor}{rgb}{0.000000,0.000000,0.000000}%
\pgfsetstrokecolor{textcolor}%
\pgfsetfillcolor{textcolor}%
\pgftext[x=4.769222in, y=2.219850in, left, base]{\color{textcolor}\sffamily\fontsize{10.000000}{12.000000}\selectfont \(\displaystyle {0}\)}%
\end{pgfscope}%
\begin{pgfscope}%
\pgfsetbuttcap%
\pgfsetroundjoin%
\definecolor{currentfill}{rgb}{0.000000,0.000000,0.000000}%
\pgfsetfillcolor{currentfill}%
\pgfsetlinewidth{0.803000pt}%
\definecolor{currentstroke}{rgb}{0.000000,0.000000,0.000000}%
\pgfsetstrokecolor{currentstroke}%
\pgfsetdash{}{0pt}%
\pgfsys@defobject{currentmarker}{\pgfqpoint{0.000000in}{0.000000in}}{\pgfqpoint{0.048611in}{0.000000in}}{%
\pgfpathmoveto{\pgfqpoint{0.000000in}{0.000000in}}%
\pgfpathlineto{\pgfqpoint{0.048611in}{0.000000in}}%
\pgfusepath{stroke,fill}%
}%
\begin{pgfscope}%
\pgfsys@transformshift{4.672000in}{2.632122in}%
\pgfsys@useobject{currentmarker}{}%
\end{pgfscope}%
\end{pgfscope}%
\begin{pgfscope}%
\definecolor{textcolor}{rgb}{0.000000,0.000000,0.000000}%
\pgfsetstrokecolor{textcolor}%
\pgfsetfillcolor{textcolor}%
\pgftext[x=4.769222in, y=2.583897in, left, base]{\color{textcolor}\sffamily\fontsize{10.000000}{12.000000}\selectfont \(\displaystyle {200}\)}%
\end{pgfscope}%
\begin{pgfscope}%
\pgfsetbuttcap%
\pgfsetroundjoin%
\definecolor{currentfill}{rgb}{0.000000,0.000000,0.000000}%
\pgfsetfillcolor{currentfill}%
\pgfsetlinewidth{0.803000pt}%
\definecolor{currentstroke}{rgb}{0.000000,0.000000,0.000000}%
\pgfsetstrokecolor{currentstroke}%
\pgfsetdash{}{0pt}%
\pgfsys@defobject{currentmarker}{\pgfqpoint{0.000000in}{0.000000in}}{\pgfqpoint{0.048611in}{0.000000in}}{%
\pgfpathmoveto{\pgfqpoint{0.000000in}{0.000000in}}%
\pgfpathlineto{\pgfqpoint{0.048611in}{0.000000in}}%
\pgfusepath{stroke,fill}%
}%
\begin{pgfscope}%
\pgfsys@transformshift{4.672000in}{2.996169in}%
\pgfsys@useobject{currentmarker}{}%
\end{pgfscope}%
\end{pgfscope}%
\begin{pgfscope}%
\definecolor{textcolor}{rgb}{0.000000,0.000000,0.000000}%
\pgfsetstrokecolor{textcolor}%
\pgfsetfillcolor{textcolor}%
\pgftext[x=4.769222in, y=2.947944in, left, base]{\color{textcolor}\sffamily\fontsize{10.000000}{12.000000}\selectfont \(\displaystyle {400}\)}%
\end{pgfscope}%
\begin{pgfscope}%
\definecolor{textcolor}{rgb}{0.000000,0.000000,0.000000}%
\pgfsetstrokecolor{textcolor}%
\pgfsetfillcolor{textcolor}%
\pgftext[x=5.141137in,y=2.023913in,,top,rotate=90.000000]{\color{textcolor}\sffamily\fontsize{10.000000}{12.000000}\selectfont \(\displaystyle \mathrm{Energy \, gain \, [MeV]}\)}%
\end{pgfscope}%
\begin{pgfscope}%
\pgfpathrectangle{\pgfqpoint{0.800000in}{0.875000in}}{\pgfqpoint{3.872000in}{2.297826in}}%
\pgfusepath{clip}%
\pgfsetrectcap%
\pgfsetroundjoin%
\pgfsetlinewidth{1.505625pt}%
\definecolor{currentstroke}{rgb}{0.000000,0.501961,0.000000}%
\pgfsetstrokecolor{currentstroke}%
\pgfsetdash{}{0pt}%
\pgfpathmoveto{\pgfqpoint{0.803227in}{1.382157in}}%
\pgfpathlineto{\pgfqpoint{0.816133in}{1.341135in}}%
\pgfpathlineto{\pgfqpoint{0.822587in}{1.326930in}}%
\pgfpathlineto{\pgfqpoint{0.829040in}{1.300876in}}%
\pgfpathlineto{\pgfqpoint{0.848400in}{1.242808in}}%
\pgfpathlineto{\pgfqpoint{0.867760in}{1.182130in}}%
\pgfpathlineto{\pgfqpoint{0.887120in}{1.126556in}}%
\pgfpathlineto{\pgfqpoint{0.893573in}{1.101838in}}%
\pgfpathlineto{\pgfqpoint{0.906480in}{1.067136in}}%
\pgfpathlineto{\pgfqpoint{0.912933in}{1.044457in}}%
\pgfpathlineto{\pgfqpoint{0.932293in}{0.998673in}}%
\pgfpathlineto{\pgfqpoint{0.938747in}{0.985050in}}%
\pgfpathlineto{\pgfqpoint{0.945200in}{0.992602in}}%
\pgfpathlineto{\pgfqpoint{0.951653in}{0.979447in}}%
\pgfpathlineto{\pgfqpoint{0.958107in}{0.996877in}}%
\pgfpathlineto{\pgfqpoint{0.971013in}{1.118668in}}%
\pgfpathlineto{\pgfqpoint{0.983920in}{1.407201in}}%
\pgfpathlineto{\pgfqpoint{0.996827in}{1.763564in}}%
\pgfpathlineto{\pgfqpoint{1.009733in}{2.146156in}}%
\pgfpathlineto{\pgfqpoint{1.016187in}{2.261230in}}%
\pgfpathlineto{\pgfqpoint{1.029093in}{2.350363in}}%
\pgfpathlineto{\pgfqpoint{1.035547in}{2.377453in}}%
\pgfpathlineto{\pgfqpoint{1.048453in}{2.460084in}}%
\pgfpathlineto{\pgfqpoint{1.061360in}{2.533090in}}%
\pgfpathlineto{\pgfqpoint{1.067813in}{2.565864in}}%
\pgfpathlineto{\pgfqpoint{1.074267in}{2.586616in}}%
\pgfpathlineto{\pgfqpoint{1.100080in}{2.698973in}}%
\pgfpathlineto{\pgfqpoint{1.125893in}{2.779332in}}%
\pgfpathlineto{\pgfqpoint{1.132347in}{2.792759in}}%
\pgfpathlineto{\pgfqpoint{1.138800in}{2.812177in}}%
\pgfpathlineto{\pgfqpoint{1.158160in}{2.850069in}}%
\pgfpathlineto{\pgfqpoint{1.171067in}{2.875493in}}%
\pgfpathlineto{\pgfqpoint{1.177520in}{2.882281in}}%
\pgfpathlineto{\pgfqpoint{1.183973in}{2.898758in}}%
\pgfpathlineto{\pgfqpoint{1.190427in}{2.901650in}}%
\pgfpathlineto{\pgfqpoint{1.196880in}{2.917813in}}%
\pgfpathlineto{\pgfqpoint{1.203333in}{2.919162in}}%
\pgfpathlineto{\pgfqpoint{1.209787in}{2.936458in}}%
\pgfpathlineto{\pgfqpoint{1.222693in}{2.940553in}}%
\pgfpathlineto{\pgfqpoint{1.229147in}{2.954584in}}%
\pgfpathlineto{\pgfqpoint{1.235600in}{2.951215in}}%
\pgfpathlineto{\pgfqpoint{1.242053in}{2.971015in}}%
\pgfpathlineto{\pgfqpoint{1.248507in}{2.966735in}}%
\pgfpathlineto{\pgfqpoint{1.254960in}{2.973345in}}%
\pgfpathlineto{\pgfqpoint{1.261413in}{2.974801in}}%
\pgfpathlineto{\pgfqpoint{1.274320in}{2.983446in}}%
\pgfpathlineto{\pgfqpoint{1.280773in}{2.987134in}}%
\pgfpathlineto{\pgfqpoint{1.287227in}{2.983389in}}%
\pgfpathlineto{\pgfqpoint{1.293680in}{2.992037in}}%
\pgfpathlineto{\pgfqpoint{1.300133in}{2.990764in}}%
\pgfpathlineto{\pgfqpoint{1.306587in}{2.993849in}}%
\pgfpathlineto{\pgfqpoint{1.313040in}{2.987950in}}%
\pgfpathlineto{\pgfqpoint{1.319493in}{2.993446in}}%
\pgfpathlineto{\pgfqpoint{1.325947in}{2.990770in}}%
\pgfpathlineto{\pgfqpoint{1.332400in}{2.995200in}}%
\pgfpathlineto{\pgfqpoint{1.338853in}{2.987922in}}%
\pgfpathlineto{\pgfqpoint{1.345307in}{2.987959in}}%
\pgfpathlineto{\pgfqpoint{1.351760in}{2.986450in}}%
\pgfpathlineto{\pgfqpoint{1.358213in}{2.979074in}}%
\pgfpathlineto{\pgfqpoint{1.364667in}{2.979983in}}%
\pgfpathlineto{\pgfqpoint{1.371120in}{2.974367in}}%
\pgfpathlineto{\pgfqpoint{1.384027in}{2.967581in}}%
\pgfpathlineto{\pgfqpoint{1.390480in}{2.959957in}}%
\pgfpathlineto{\pgfqpoint{1.396933in}{2.960755in}}%
\pgfpathlineto{\pgfqpoint{1.403387in}{2.945449in}}%
\pgfpathlineto{\pgfqpoint{1.409840in}{2.945382in}}%
\pgfpathlineto{\pgfqpoint{1.416293in}{2.940428in}}%
\pgfpathlineto{\pgfqpoint{1.422747in}{2.930563in}}%
\pgfpathlineto{\pgfqpoint{1.429200in}{2.926116in}}%
\pgfpathlineto{\pgfqpoint{1.442107in}{2.907099in}}%
\pgfpathlineto{\pgfqpoint{1.448560in}{2.908677in}}%
\pgfpathlineto{\pgfqpoint{1.455013in}{2.892620in}}%
\pgfpathlineto{\pgfqpoint{1.461467in}{2.883353in}}%
\pgfpathlineto{\pgfqpoint{1.467920in}{2.876827in}}%
\pgfpathlineto{\pgfqpoint{1.474373in}{2.863034in}}%
\pgfpathlineto{\pgfqpoint{1.480827in}{2.858486in}}%
\pgfpathlineto{\pgfqpoint{1.487280in}{2.850665in}}%
\pgfpathlineto{\pgfqpoint{1.493733in}{2.832470in}}%
\pgfpathlineto{\pgfqpoint{1.500187in}{2.830764in}}%
\pgfpathlineto{\pgfqpoint{1.506640in}{2.814343in}}%
\pgfpathlineto{\pgfqpoint{1.513093in}{2.804031in}}%
\pgfpathlineto{\pgfqpoint{1.519547in}{2.797364in}}%
\pgfpathlineto{\pgfqpoint{1.526000in}{2.778943in}}%
\pgfpathlineto{\pgfqpoint{1.532453in}{2.772803in}}%
\pgfpathlineto{\pgfqpoint{1.538907in}{2.762196in}}%
\pgfpathlineto{\pgfqpoint{1.545360in}{2.747298in}}%
\pgfpathlineto{\pgfqpoint{1.551813in}{2.738094in}}%
\pgfpathlineto{\pgfqpoint{1.577627in}{2.685513in}}%
\pgfpathlineto{\pgfqpoint{1.584080in}{2.669872in}}%
\pgfpathlineto{\pgfqpoint{1.590533in}{2.667692in}}%
\pgfpathlineto{\pgfqpoint{1.596987in}{2.638733in}}%
\pgfpathlineto{\pgfqpoint{1.603440in}{2.635712in}}%
\pgfpathlineto{\pgfqpoint{1.609893in}{2.615346in}}%
\pgfpathlineto{\pgfqpoint{1.616347in}{2.607469in}}%
\pgfpathlineto{\pgfqpoint{1.642160in}{2.547419in}}%
\pgfpathlineto{\pgfqpoint{1.648613in}{2.525795in}}%
\pgfpathlineto{\pgfqpoint{1.655067in}{2.524607in}}%
\pgfpathlineto{\pgfqpoint{1.661520in}{2.498263in}}%
\pgfpathlineto{\pgfqpoint{1.667973in}{2.490615in}}%
\pgfpathlineto{\pgfqpoint{1.674427in}{2.471244in}}%
\pgfpathlineto{\pgfqpoint{1.687333in}{2.444729in}}%
\pgfpathlineto{\pgfqpoint{1.693787in}{2.423248in}}%
\pgfpathlineto{\pgfqpoint{1.700240in}{2.411431in}}%
\pgfpathlineto{\pgfqpoint{1.706693in}{2.394431in}}%
\pgfpathlineto{\pgfqpoint{1.713147in}{2.382773in}}%
\pgfpathlineto{\pgfqpoint{1.726053in}{2.346716in}}%
\pgfpathlineto{\pgfqpoint{1.732507in}{2.332901in}}%
\pgfpathlineto{\pgfqpoint{1.738960in}{2.310540in}}%
\pgfpathlineto{\pgfqpoint{1.745413in}{2.304460in}}%
\pgfpathlineto{\pgfqpoint{1.751867in}{2.277976in}}%
\pgfpathlineto{\pgfqpoint{1.764773in}{2.253305in}}%
\pgfpathlineto{\pgfqpoint{1.771227in}{2.229801in}}%
\pgfpathlineto{\pgfqpoint{1.777680in}{2.223159in}}%
\pgfpathlineto{\pgfqpoint{1.784133in}{2.197039in}}%
\pgfpathlineto{\pgfqpoint{1.790587in}{2.184346in}}%
\pgfpathlineto{\pgfqpoint{1.797040in}{2.164260in}}%
\pgfpathlineto{\pgfqpoint{1.803493in}{2.153278in}}%
\pgfpathlineto{\pgfqpoint{1.809947in}{2.125857in}}%
\pgfpathlineto{\pgfqpoint{1.822853in}{2.100458in}}%
\pgfpathlineto{\pgfqpoint{1.829307in}{2.072849in}}%
\pgfpathlineto{\pgfqpoint{1.835760in}{2.073052in}}%
\pgfpathlineto{\pgfqpoint{1.842213in}{2.041140in}}%
\pgfpathlineto{\pgfqpoint{1.848667in}{2.030616in}}%
\pgfpathlineto{\pgfqpoint{1.855120in}{2.014481in}}%
\pgfpathlineto{\pgfqpoint{1.861573in}{1.988842in}}%
\pgfpathlineto{\pgfqpoint{1.868027in}{1.981089in}}%
\pgfpathlineto{\pgfqpoint{1.874480in}{1.962208in}}%
\pgfpathlineto{\pgfqpoint{1.880933in}{1.937335in}}%
\pgfpathlineto{\pgfqpoint{1.887387in}{1.927366in}}%
\pgfpathlineto{\pgfqpoint{1.893840in}{1.904135in}}%
\pgfpathlineto{\pgfqpoint{1.913200in}{1.856987in}}%
\pgfpathlineto{\pgfqpoint{1.919653in}{1.832249in}}%
\pgfpathlineto{\pgfqpoint{1.926107in}{1.818037in}}%
\pgfpathlineto{\pgfqpoint{1.932560in}{1.796434in}}%
\pgfpathlineto{\pgfqpoint{1.939013in}{1.785617in}}%
\pgfpathlineto{\pgfqpoint{1.945467in}{1.757606in}}%
\pgfpathlineto{\pgfqpoint{1.951920in}{1.749105in}}%
\pgfpathlineto{\pgfqpoint{1.964827in}{1.704831in}}%
\pgfpathlineto{\pgfqpoint{1.971280in}{1.697257in}}%
\pgfpathlineto{\pgfqpoint{1.977733in}{1.671618in}}%
\pgfpathlineto{\pgfqpoint{1.984187in}{1.661782in}}%
\pgfpathlineto{\pgfqpoint{1.990640in}{1.629946in}}%
\pgfpathlineto{\pgfqpoint{1.997093in}{1.621942in}}%
\pgfpathlineto{\pgfqpoint{2.016453in}{1.561824in}}%
\pgfpathlineto{\pgfqpoint{2.022907in}{1.552512in}}%
\pgfpathlineto{\pgfqpoint{2.029360in}{1.525707in}}%
\pgfpathlineto{\pgfqpoint{2.035813in}{1.511995in}}%
\pgfpathlineto{\pgfqpoint{2.048720in}{1.471652in}}%
\pgfpathlineto{\pgfqpoint{2.055173in}{1.463612in}}%
\pgfpathlineto{\pgfqpoint{2.061627in}{1.434699in}}%
\pgfpathlineto{\pgfqpoint{2.068080in}{1.422656in}}%
\pgfpathlineto{\pgfqpoint{2.074533in}{1.398844in}}%
\pgfpathlineto{\pgfqpoint{2.080987in}{1.384139in}}%
\pgfpathlineto{\pgfqpoint{2.093893in}{1.347547in}}%
\pgfpathlineto{\pgfqpoint{2.100347in}{1.331838in}}%
\pgfpathlineto{\pgfqpoint{2.106800in}{1.307952in}}%
\pgfpathlineto{\pgfqpoint{2.113253in}{1.298828in}}%
\pgfpathlineto{\pgfqpoint{2.126160in}{1.255328in}}%
\pgfpathlineto{\pgfqpoint{2.139067in}{1.224178in}}%
\pgfpathlineto{\pgfqpoint{2.145520in}{1.197389in}}%
\pgfpathlineto{\pgfqpoint{2.151973in}{1.188734in}}%
\pgfpathlineto{\pgfqpoint{2.158427in}{1.170239in}}%
\pgfpathlineto{\pgfqpoint{2.164880in}{1.142337in}}%
\pgfpathlineto{\pgfqpoint{2.171333in}{1.145366in}}%
\pgfpathlineto{\pgfqpoint{2.177787in}{1.111125in}}%
\pgfpathlineto{\pgfqpoint{2.184240in}{1.095924in}}%
\pgfpathlineto{\pgfqpoint{2.190693in}{1.095926in}}%
\pgfpathlineto{\pgfqpoint{2.197147in}{1.058773in}}%
\pgfpathlineto{\pgfqpoint{2.203600in}{1.055658in}}%
\pgfpathlineto{\pgfqpoint{2.210053in}{1.043304in}}%
\pgfpathlineto{\pgfqpoint{2.216507in}{1.012098in}}%
\pgfpathlineto{\pgfqpoint{2.222960in}{1.026515in}}%
\pgfpathlineto{\pgfqpoint{2.229413in}{0.993980in}}%
\pgfpathlineto{\pgfqpoint{2.235867in}{0.997514in}}%
\pgfpathlineto{\pgfqpoint{2.248773in}{0.994892in}}%
\pgfpathlineto{\pgfqpoint{2.255227in}{1.020140in}}%
\pgfpathlineto{\pgfqpoint{2.261680in}{1.052427in}}%
\pgfpathlineto{\pgfqpoint{2.268133in}{1.106187in}}%
\pgfpathlineto{\pgfqpoint{2.274587in}{1.228961in}}%
\pgfpathlineto{\pgfqpoint{2.281040in}{1.389242in}}%
\pgfpathlineto{\pgfqpoint{2.293947in}{2.073834in}}%
\pgfpathlineto{\pgfqpoint{2.300400in}{2.411919in}}%
\pgfpathlineto{\pgfqpoint{2.306853in}{2.575766in}}%
\pgfpathlineto{\pgfqpoint{2.319760in}{2.611252in}}%
\pgfpathlineto{\pgfqpoint{2.332667in}{2.717237in}}%
\pgfpathlineto{\pgfqpoint{2.339120in}{2.724606in}}%
\pgfpathlineto{\pgfqpoint{2.352027in}{2.730677in}}%
\pgfpathlineto{\pgfqpoint{2.358480in}{2.760020in}}%
\pgfpathlineto{\pgfqpoint{2.364933in}{2.761133in}}%
\pgfpathlineto{\pgfqpoint{2.371387in}{2.792723in}}%
\pgfpathlineto{\pgfqpoint{2.377840in}{2.804371in}}%
\pgfpathlineto{\pgfqpoint{2.384293in}{2.819549in}}%
\pgfpathlineto{\pgfqpoint{2.390747in}{2.850408in}}%
\pgfpathlineto{\pgfqpoint{2.397200in}{2.851359in}}%
\pgfpathlineto{\pgfqpoint{2.410107in}{2.892828in}}%
\pgfpathlineto{\pgfqpoint{2.416560in}{2.898686in}}%
\pgfpathlineto{\pgfqpoint{2.423013in}{2.922360in}}%
\pgfpathlineto{\pgfqpoint{2.429467in}{2.928305in}}%
\pgfpathlineto{\pgfqpoint{2.435920in}{2.948051in}}%
\pgfpathlineto{\pgfqpoint{2.442373in}{2.947610in}}%
\pgfpathlineto{\pgfqpoint{2.448827in}{2.980423in}}%
\pgfpathlineto{\pgfqpoint{2.455280in}{2.967996in}}%
\pgfpathlineto{\pgfqpoint{2.461733in}{2.995631in}}%
\pgfpathlineto{\pgfqpoint{2.468187in}{2.991681in}}%
\pgfpathlineto{\pgfqpoint{2.481093in}{3.019744in}}%
\pgfpathlineto{\pgfqpoint{2.487547in}{3.007339in}}%
\pgfpathlineto{\pgfqpoint{2.494000in}{3.037839in}}%
\pgfpathlineto{\pgfqpoint{2.500453in}{3.026352in}}%
\pgfpathlineto{\pgfqpoint{2.513360in}{3.051207in}}%
\pgfpathlineto{\pgfqpoint{2.519813in}{3.038067in}}%
\pgfpathlineto{\pgfqpoint{2.532720in}{3.061262in}}%
\pgfpathlineto{\pgfqpoint{2.539173in}{3.044605in}}%
\pgfpathlineto{\pgfqpoint{2.545627in}{3.068194in}}%
\pgfpathlineto{\pgfqpoint{2.552080in}{3.057658in}}%
\pgfpathlineto{\pgfqpoint{2.558533in}{3.057776in}}%
\pgfpathlineto{\pgfqpoint{2.564987in}{3.061815in}}%
\pgfpathlineto{\pgfqpoint{2.571440in}{3.060509in}}%
\pgfpathlineto{\pgfqpoint{2.577893in}{3.052472in}}%
\pgfpathlineto{\pgfqpoint{2.584347in}{3.068379in}}%
\pgfpathlineto{\pgfqpoint{2.590800in}{3.042613in}}%
\pgfpathlineto{\pgfqpoint{2.597253in}{3.062541in}}%
\pgfpathlineto{\pgfqpoint{2.610160in}{3.040238in}}%
\pgfpathlineto{\pgfqpoint{2.616613in}{3.048458in}}%
\pgfpathlineto{\pgfqpoint{2.623067in}{3.031893in}}%
\pgfpathlineto{\pgfqpoint{2.635973in}{3.036314in}}%
\pgfpathlineto{\pgfqpoint{2.642427in}{3.014984in}}%
\pgfpathlineto{\pgfqpoint{2.648880in}{3.023787in}}%
\pgfpathlineto{\pgfqpoint{2.655333in}{3.010071in}}%
\pgfpathlineto{\pgfqpoint{2.661787in}{3.001337in}}%
\pgfpathlineto{\pgfqpoint{2.668240in}{2.999120in}}%
\pgfpathlineto{\pgfqpoint{2.694053in}{2.963977in}}%
\pgfpathlineto{\pgfqpoint{2.700507in}{2.960757in}}%
\pgfpathlineto{\pgfqpoint{2.706960in}{2.937890in}}%
\pgfpathlineto{\pgfqpoint{2.713413in}{2.950107in}}%
\pgfpathlineto{\pgfqpoint{2.719867in}{2.915094in}}%
\pgfpathlineto{\pgfqpoint{2.726320in}{2.925915in}}%
\pgfpathlineto{\pgfqpoint{2.732773in}{2.901570in}}%
\pgfpathlineto{\pgfqpoint{2.739227in}{2.899536in}}%
\pgfpathlineto{\pgfqpoint{2.745680in}{2.891577in}}%
\pgfpathlineto{\pgfqpoint{2.752133in}{2.874981in}}%
\pgfpathlineto{\pgfqpoint{2.758587in}{2.880439in}}%
\pgfpathlineto{\pgfqpoint{2.765040in}{2.861177in}}%
\pgfpathlineto{\pgfqpoint{2.771493in}{2.849579in}}%
\pgfpathlineto{\pgfqpoint{2.777947in}{2.851671in}}%
\pgfpathlineto{\pgfqpoint{2.784400in}{2.829422in}}%
\pgfpathlineto{\pgfqpoint{2.790853in}{2.826358in}}%
\pgfpathlineto{\pgfqpoint{2.797307in}{2.818482in}}%
\pgfpathlineto{\pgfqpoint{2.803760in}{2.797344in}}%
\pgfpathlineto{\pgfqpoint{2.810213in}{2.798185in}}%
\pgfpathlineto{\pgfqpoint{2.816667in}{2.779583in}}%
\pgfpathlineto{\pgfqpoint{2.823120in}{2.775301in}}%
\pgfpathlineto{\pgfqpoint{2.829573in}{2.758331in}}%
\pgfpathlineto{\pgfqpoint{2.836027in}{2.750072in}}%
\pgfpathlineto{\pgfqpoint{2.842480in}{2.736440in}}%
\pgfpathlineto{\pgfqpoint{2.848933in}{2.726738in}}%
\pgfpathlineto{\pgfqpoint{2.855387in}{2.709281in}}%
\pgfpathlineto{\pgfqpoint{2.868293in}{2.687141in}}%
\pgfpathlineto{\pgfqpoint{2.874747in}{2.668904in}}%
\pgfpathlineto{\pgfqpoint{2.881200in}{2.667858in}}%
\pgfpathlineto{\pgfqpoint{2.887653in}{2.636276in}}%
\pgfpathlineto{\pgfqpoint{2.894107in}{2.641094in}}%
\pgfpathlineto{\pgfqpoint{2.900560in}{2.612544in}}%
\pgfpathlineto{\pgfqpoint{2.907013in}{2.600720in}}%
\pgfpathlineto{\pgfqpoint{2.913467in}{2.598586in}}%
\pgfpathlineto{\pgfqpoint{2.919920in}{2.560430in}}%
\pgfpathlineto{\pgfqpoint{2.926373in}{2.571537in}}%
\pgfpathlineto{\pgfqpoint{2.932827in}{2.540306in}}%
\pgfpathlineto{\pgfqpoint{2.945733in}{2.516988in}}%
\pgfpathlineto{\pgfqpoint{2.952187in}{2.495350in}}%
\pgfpathlineto{\pgfqpoint{2.958640in}{2.481377in}}%
\pgfpathlineto{\pgfqpoint{2.965093in}{2.473538in}}%
\pgfpathlineto{\pgfqpoint{2.971547in}{2.436408in}}%
\pgfpathlineto{\pgfqpoint{2.978000in}{2.448588in}}%
\pgfpathlineto{\pgfqpoint{2.984453in}{2.415182in}}%
\pgfpathlineto{\pgfqpoint{2.990907in}{2.401704in}}%
\pgfpathlineto{\pgfqpoint{2.997360in}{2.396037in}}%
\pgfpathlineto{\pgfqpoint{3.003813in}{2.369553in}}%
\pgfpathlineto{\pgfqpoint{3.010267in}{2.352943in}}%
\pgfpathlineto{\pgfqpoint{3.016720in}{2.350027in}}%
\pgfpathlineto{\pgfqpoint{3.023173in}{2.313395in}}%
\pgfpathlineto{\pgfqpoint{3.029627in}{2.312515in}}%
\pgfpathlineto{\pgfqpoint{3.042533in}{2.272107in}}%
\pgfpathlineto{\pgfqpoint{3.048987in}{2.264665in}}%
\pgfpathlineto{\pgfqpoint{3.061893in}{2.226036in}}%
\pgfpathlineto{\pgfqpoint{3.068347in}{2.221584in}}%
\pgfpathlineto{\pgfqpoint{3.074800in}{2.191158in}}%
\pgfpathlineto{\pgfqpoint{3.081253in}{2.187541in}}%
\pgfpathlineto{\pgfqpoint{3.087707in}{2.173837in}}%
\pgfpathlineto{\pgfqpoint{3.094160in}{2.144031in}}%
\pgfpathlineto{\pgfqpoint{3.100613in}{2.154608in}}%
\pgfpathlineto{\pgfqpoint{3.107067in}{2.116726in}}%
\pgfpathlineto{\pgfqpoint{3.113520in}{2.119809in}}%
\pgfpathlineto{\pgfqpoint{3.119973in}{2.100129in}}%
\pgfpathlineto{\pgfqpoint{3.126427in}{2.091992in}}%
\pgfpathlineto{\pgfqpoint{3.132880in}{2.073500in}}%
\pgfpathlineto{\pgfqpoint{3.139333in}{2.066249in}}%
\pgfpathlineto{\pgfqpoint{3.145787in}{2.050776in}}%
\pgfpathlineto{\pgfqpoint{3.152240in}{2.044224in}}%
\pgfpathlineto{\pgfqpoint{3.158693in}{2.019613in}}%
\pgfpathlineto{\pgfqpoint{3.165147in}{2.024261in}}%
\pgfpathlineto{\pgfqpoint{3.178053in}{1.983403in}}%
\pgfpathlineto{\pgfqpoint{3.184507in}{1.985750in}}%
\pgfpathlineto{\pgfqpoint{3.190960in}{1.970558in}}%
\pgfpathlineto{\pgfqpoint{3.197413in}{1.945108in}}%
\pgfpathlineto{\pgfqpoint{3.203867in}{1.957067in}}%
\pgfpathlineto{\pgfqpoint{3.210320in}{1.914530in}}%
\pgfpathlineto{\pgfqpoint{3.216773in}{1.929227in}}%
\pgfpathlineto{\pgfqpoint{3.229680in}{1.885841in}}%
\pgfpathlineto{\pgfqpoint{3.236133in}{1.898896in}}%
\pgfpathlineto{\pgfqpoint{3.242587in}{1.853411in}}%
\pgfpathlineto{\pgfqpoint{3.249040in}{1.873895in}}%
\pgfpathlineto{\pgfqpoint{3.255493in}{1.835503in}}%
\pgfpathlineto{\pgfqpoint{3.261947in}{1.844054in}}%
\pgfpathlineto{\pgfqpoint{3.268400in}{1.818964in}}%
\pgfpathlineto{\pgfqpoint{3.274853in}{1.827147in}}%
\pgfpathlineto{\pgfqpoint{3.281307in}{1.789963in}}%
\pgfpathlineto{\pgfqpoint{3.287760in}{1.809146in}}%
\pgfpathlineto{\pgfqpoint{3.294213in}{1.766749in}}%
\pgfpathlineto{\pgfqpoint{3.300667in}{1.786582in}}%
\pgfpathlineto{\pgfqpoint{3.307120in}{1.752706in}}%
\pgfpathlineto{\pgfqpoint{3.313573in}{1.761444in}}%
\pgfpathlineto{\pgfqpoint{3.320027in}{1.741280in}}%
\pgfpathlineto{\pgfqpoint{3.326480in}{1.734224in}}%
\pgfpathlineto{\pgfqpoint{3.332933in}{1.729501in}}%
\pgfpathlineto{\pgfqpoint{3.339387in}{1.722592in}}%
\pgfpathlineto{\pgfqpoint{3.345840in}{1.704396in}}%
\pgfpathlineto{\pgfqpoint{3.352293in}{1.718201in}}%
\pgfpathlineto{\pgfqpoint{3.358747in}{1.682960in}}%
\pgfpathlineto{\pgfqpoint{3.365200in}{1.700659in}}%
\pgfpathlineto{\pgfqpoint{3.371653in}{1.681209in}}%
\pgfpathlineto{\pgfqpoint{3.378107in}{1.686188in}}%
\pgfpathlineto{\pgfqpoint{3.384560in}{1.676052in}}%
\pgfpathlineto{\pgfqpoint{3.391013in}{1.672614in}}%
\pgfpathlineto{\pgfqpoint{3.397467in}{1.664008in}}%
\pgfpathlineto{\pgfqpoint{3.403920in}{1.671878in}}%
\pgfpathlineto{\pgfqpoint{3.410373in}{1.648076in}}%
\pgfpathlineto{\pgfqpoint{3.416827in}{1.683023in}}%
\pgfpathlineto{\pgfqpoint{3.423280in}{1.642219in}}%
\pgfpathlineto{\pgfqpoint{3.429733in}{1.682580in}}%
\pgfpathlineto{\pgfqpoint{3.436187in}{1.640874in}}%
\pgfpathlineto{\pgfqpoint{3.442640in}{1.675819in}}%
\pgfpathlineto{\pgfqpoint{3.449093in}{1.650583in}}%
\pgfpathlineto{\pgfqpoint{3.455547in}{1.667555in}}%
\pgfpathlineto{\pgfqpoint{3.462000in}{1.671285in}}%
\pgfpathlineto{\pgfqpoint{3.468453in}{1.669906in}}%
\pgfpathlineto{\pgfqpoint{3.474907in}{1.680627in}}%
\pgfpathlineto{\pgfqpoint{3.481360in}{1.673358in}}%
\pgfpathlineto{\pgfqpoint{3.487813in}{1.699716in}}%
\pgfpathlineto{\pgfqpoint{3.494267in}{1.675120in}}%
\pgfpathlineto{\pgfqpoint{3.500720in}{1.707095in}}%
\pgfpathlineto{\pgfqpoint{3.507173in}{1.706374in}}%
\pgfpathlineto{\pgfqpoint{3.513627in}{1.716401in}}%
\pgfpathlineto{\pgfqpoint{3.520080in}{1.720120in}}%
\pgfpathlineto{\pgfqpoint{3.526533in}{1.741420in}}%
\pgfpathlineto{\pgfqpoint{3.532987in}{1.741260in}}%
\pgfpathlineto{\pgfqpoint{3.539440in}{1.756357in}}%
\pgfpathlineto{\pgfqpoint{3.545893in}{1.765208in}}%
\pgfpathlineto{\pgfqpoint{3.552347in}{1.781621in}}%
\pgfpathlineto{\pgfqpoint{3.558800in}{1.793727in}}%
\pgfpathlineto{\pgfqpoint{3.565253in}{1.811568in}}%
\pgfpathlineto{\pgfqpoint{3.597520in}{1.878134in}}%
\pgfpathlineto{\pgfqpoint{3.603973in}{1.895614in}}%
\pgfpathlineto{\pgfqpoint{3.610427in}{1.906786in}}%
\pgfpathlineto{\pgfqpoint{3.616880in}{1.926469in}}%
\pgfpathlineto{\pgfqpoint{3.623333in}{1.937749in}}%
\pgfpathlineto{\pgfqpoint{3.629787in}{1.958943in}}%
\pgfpathlineto{\pgfqpoint{3.636240in}{1.969502in}}%
\pgfpathlineto{\pgfqpoint{3.642693in}{1.989315in}}%
\pgfpathlineto{\pgfqpoint{3.649147in}{1.999082in}}%
\pgfpathlineto{\pgfqpoint{3.668507in}{2.034811in}}%
\pgfpathlineto{\pgfqpoint{3.674960in}{2.055153in}}%
\pgfpathlineto{\pgfqpoint{3.681413in}{2.068049in}}%
\pgfpathlineto{\pgfqpoint{3.687867in}{2.072848in}}%
\pgfpathlineto{\pgfqpoint{3.694320in}{2.094314in}}%
\pgfpathlineto{\pgfqpoint{3.700773in}{2.093280in}}%
\pgfpathlineto{\pgfqpoint{3.707227in}{2.119793in}}%
\pgfpathlineto{\pgfqpoint{3.713680in}{2.115257in}}%
\pgfpathlineto{\pgfqpoint{3.720133in}{2.138880in}}%
\pgfpathlineto{\pgfqpoint{3.726587in}{2.138709in}}%
\pgfpathlineto{\pgfqpoint{3.739493in}{2.156765in}}%
\pgfpathlineto{\pgfqpoint{3.745947in}{2.163644in}}%
\pgfpathlineto{\pgfqpoint{3.752400in}{2.173433in}}%
\pgfpathlineto{\pgfqpoint{3.758853in}{2.174962in}}%
\pgfpathlineto{\pgfqpoint{3.765307in}{2.185716in}}%
\pgfpathlineto{\pgfqpoint{3.771760in}{2.186833in}}%
\pgfpathlineto{\pgfqpoint{3.778213in}{2.197827in}}%
\pgfpathlineto{\pgfqpoint{3.784667in}{2.198384in}}%
\pgfpathlineto{\pgfqpoint{3.797573in}{2.212053in}}%
\pgfpathlineto{\pgfqpoint{3.804027in}{2.213476in}}%
\pgfpathlineto{\pgfqpoint{3.810480in}{2.217371in}}%
\pgfpathlineto{\pgfqpoint{3.816933in}{2.225287in}}%
\pgfpathlineto{\pgfqpoint{3.823387in}{2.223388in}}%
\pgfpathlineto{\pgfqpoint{3.829840in}{2.229711in}}%
\pgfpathlineto{\pgfqpoint{3.836293in}{2.229515in}}%
\pgfpathlineto{\pgfqpoint{3.842747in}{2.235132in}}%
\pgfpathlineto{\pgfqpoint{3.849200in}{2.235810in}}%
\pgfpathlineto{\pgfqpoint{3.862107in}{2.241664in}}%
\pgfpathlineto{\pgfqpoint{3.868560in}{2.241690in}}%
\pgfpathlineto{\pgfqpoint{3.875013in}{2.245884in}}%
\pgfpathlineto{\pgfqpoint{3.881467in}{2.245673in}}%
\pgfpathlineto{\pgfqpoint{3.887920in}{2.250011in}}%
\pgfpathlineto{\pgfqpoint{3.894373in}{2.248757in}}%
\pgfpathlineto{\pgfqpoint{3.900827in}{2.252379in}}%
\pgfpathlineto{\pgfqpoint{3.907280in}{2.251578in}}%
\pgfpathlineto{\pgfqpoint{3.913733in}{2.254342in}}%
\pgfpathlineto{\pgfqpoint{3.926640in}{2.255385in}}%
\pgfpathlineto{\pgfqpoint{3.933093in}{2.257463in}}%
\pgfpathlineto{\pgfqpoint{3.939547in}{2.255993in}}%
\pgfpathlineto{\pgfqpoint{3.946000in}{2.260851in}}%
\pgfpathlineto{\pgfqpoint{3.952453in}{2.257388in}}%
\pgfpathlineto{\pgfqpoint{3.958907in}{2.260640in}}%
\pgfpathlineto{\pgfqpoint{3.978267in}{2.260582in}}%
\pgfpathlineto{\pgfqpoint{3.984720in}{2.262042in}}%
\pgfpathlineto{\pgfqpoint{3.997627in}{2.262523in}}%
\pgfpathlineto{\pgfqpoint{4.036347in}{2.265809in}}%
\pgfpathlineto{\pgfqpoint{4.055707in}{2.265549in}}%
\pgfpathlineto{\pgfqpoint{4.062160in}{2.266962in}}%
\pgfpathlineto{\pgfqpoint{4.068613in}{2.265911in}}%
\pgfpathlineto{\pgfqpoint{4.075067in}{2.267574in}}%
\pgfpathlineto{\pgfqpoint{4.081520in}{2.266573in}}%
\pgfpathlineto{\pgfqpoint{4.100880in}{2.267306in}}%
\pgfpathlineto{\pgfqpoint{4.320293in}{2.268065in}}%
\pgfpathlineto{\pgfqpoint{4.668773in}{2.268073in}}%
\pgfpathlineto{\pgfqpoint{4.668773in}{2.268073in}}%
\pgfusepath{stroke}%
\end{pgfscope}%
\begin{pgfscope}%
\pgfsetrectcap%
\pgfsetmiterjoin%
\pgfsetlinewidth{0.803000pt}%
\definecolor{currentstroke}{rgb}{0.000000,0.000000,0.000000}%
\pgfsetstrokecolor{currentstroke}%
\pgfsetdash{}{0pt}%
\pgfpathmoveto{\pgfqpoint{0.800000in}{0.875000in}}%
\pgfpathlineto{\pgfqpoint{0.800000in}{3.172826in}}%
\pgfusepath{stroke}%
\end{pgfscope}%
\begin{pgfscope}%
\pgfsetrectcap%
\pgfsetmiterjoin%
\pgfsetlinewidth{0.803000pt}%
\definecolor{currentstroke}{rgb}{0.000000,0.000000,0.000000}%
\pgfsetstrokecolor{currentstroke}%
\pgfsetdash{}{0pt}%
\pgfpathmoveto{\pgfqpoint{4.672000in}{0.875000in}}%
\pgfpathlineto{\pgfqpoint{4.672000in}{3.172826in}}%
\pgfusepath{stroke}%
\end{pgfscope}%
\begin{pgfscope}%
\pgfsetrectcap%
\pgfsetmiterjoin%
\pgfsetlinewidth{0.803000pt}%
\definecolor{currentstroke}{rgb}{0.000000,0.000000,0.000000}%
\pgfsetstrokecolor{currentstroke}%
\pgfsetdash{}{0pt}%
\pgfpathmoveto{\pgfqpoint{0.800000in}{0.875000in}}%
\pgfpathlineto{\pgfqpoint{4.672000in}{0.875000in}}%
\pgfusepath{stroke}%
\end{pgfscope}%
\begin{pgfscope}%
\pgfsetrectcap%
\pgfsetmiterjoin%
\pgfsetlinewidth{0.803000pt}%
\definecolor{currentstroke}{rgb}{0.000000,0.000000,0.000000}%
\pgfsetstrokecolor{currentstroke}%
\pgfsetdash{}{0pt}%
\pgfpathmoveto{\pgfqpoint{0.800000in}{3.172826in}}%
\pgfpathlineto{\pgfqpoint{4.672000in}{3.172826in}}%
\pgfusepath{stroke}%
\end{pgfscope}%
\begin{pgfscope}%
\definecolor{textcolor}{rgb}{0.000000,0.000000,0.000000}%
\pgfsetstrokecolor{textcolor}%
\pgfsetfillcolor{textcolor}%
\pgftext[x=2.736000in,y=3.256159in,,base]{\color{textcolor}\sffamily\fontsize{12.000000}{14.400000}\selectfont b)}%
\end{pgfscope}%
\end{pgfpicture}%
\makeatother%
\endgroup%

	\caption{
	\textbf{a)} The charge distribution at the co-moving $\zeta$ axis with $x=z=0$ is plotted over the distance in the plasma $y$. Clearly visible is a change in the width of the cavities after $y=$ \qty{2}{\mm}.
	\textbf{b)} The electric field in co-moving $\zeta$ direction for all grid points at $x=z=0$ is plotted over $y$. As a green line, the gained energy for a theoretical witness beam is plotted over $\zeta$ too.}
	\label{fig:E_y_hist}
\end{figure}

In a), a change in width of the cavities can be observed, an effect that was measured in experiment \cite{Schoebel2022}. A witness beam which is phase locked with the driver, meaning it is constant in $\zeta$,
would therefore not get maximal energy gain when placed at a position where the accelerating field is maximal as it would later experience a decelerating force, when the cavities shrink again.
In b) this is visible, as the decelerating blue field moves to the right over the position of the strongest red accelerating field. Drawn in is also the expected energy gain for a potential witness beam.
This gain was calculated by integrating the Lorentz Force, created by the fields, over the traversed distance. The peak is not over the position with the highest field but at the position where the highest field is after the cavities shrank.
This maximum of gained energy can now be used to compare the wakefields for different drivers. Effects of the hypothetical witness beam on the wakefield, like beam loading \cite{Kirchen2021, Goetzfried2020}, are not considered in this discussion.

\paragraph*{Energy comparison}\hspace{0pt} \\
A comparison between different initial mean kinetic energies of the drivers with same divergence was made. We compare the three energies \qtylist{250; 300; 350}{\MeV} under the aspect how much more witness energy can be gained with higher initial bunch energy.
Only small qualitative differences exist in the wakefields between the three energies. Generally, the blowout regime can be achieved over a longer distance for higher energy drivers.
This results in slightly increased energy gains, as seen in \autoref{fig:gain_E}. The maximally gainable energies can be found in \autoref{tab:gain_E}.
\begin{table}[h]
\begin{center}
\begin{tabular}{|c|c|} 
	\hline
 	$E_{kin} \, \mathrm{[MeV]}$ & $E_{gain} \, \mathrm{[MeV]}$ \\ 
 	\hline
	$250$ & $439.7$ \\ 
 	$300$ & $502.6$ \\
	$350$ & $518.9$ \\
	\hline
\end{tabular}
\caption{Maximally possible energy gain for different initial kinetic energies.}\label{tab:gain_E}
\end{center}
\end{table}

\begin{figure}
	\centering
	%% Creator: Matplotlib, PGF backend
%%
%% To include the figure in your LaTeX document, write
%%   \input{<filename>.pgf}
%%
%% Make sure the required packages are loaded in your preamble
%%   \usepackage{pgf}
%%
%% Also ensure that all the required font packages are loaded; for instance,
%% the lmodern package is sometimes necessary when using math font.
%%   \usepackage{lmodern}
%%
%% Figures using additional raster images can only be included by \input if
%% they are in the same directory as the main LaTeX file. For loading figures
%% from other directories you can use the `import` package
%%   \usepackage{import}
%%
%% and then include the figures with
%%   \import{<path to file>}{<filename>.pgf}
%%
%% Matplotlib used the following preamble
%%
\begingroup%
\makeatletter%
\begin{pgfpicture}%
\pgfpathrectangle{\pgfpointorigin}{\pgfqpoint{6.000000in}{4.000000in}}%
\pgfusepath{use as bounding box, clip}%
\begin{pgfscope}%
\pgfsetbuttcap%
\pgfsetmiterjoin%
\pgfsetlinewidth{0.000000pt}%
\definecolor{currentstroke}{rgb}{1.000000,1.000000,1.000000}%
\pgfsetstrokecolor{currentstroke}%
\pgfsetstrokeopacity{0.000000}%
\pgfsetdash{}{0pt}%
\pgfpathmoveto{\pgfqpoint{0.000000in}{0.000000in}}%
\pgfpathlineto{\pgfqpoint{6.000000in}{0.000000in}}%
\pgfpathlineto{\pgfqpoint{6.000000in}{4.000000in}}%
\pgfpathlineto{\pgfqpoint{0.000000in}{4.000000in}}%
\pgfpathlineto{\pgfqpoint{0.000000in}{0.000000in}}%
\pgfpathclose%
\pgfusepath{}%
\end{pgfscope}%
\begin{pgfscope}%
\pgfsetbuttcap%
\pgfsetmiterjoin%
\definecolor{currentfill}{rgb}{1.000000,1.000000,1.000000}%
\pgfsetfillcolor{currentfill}%
\pgfsetlinewidth{0.000000pt}%
\definecolor{currentstroke}{rgb}{0.000000,0.000000,0.000000}%
\pgfsetstrokecolor{currentstroke}%
\pgfsetstrokeopacity{0.000000}%
\pgfsetdash{}{0pt}%
\pgfpathmoveto{\pgfqpoint{0.750000in}{0.500000in}}%
\pgfpathlineto{\pgfqpoint{5.400000in}{0.500000in}}%
\pgfpathlineto{\pgfqpoint{5.400000in}{3.520000in}}%
\pgfpathlineto{\pgfqpoint{0.750000in}{3.520000in}}%
\pgfpathlineto{\pgfqpoint{0.750000in}{0.500000in}}%
\pgfpathclose%
\pgfusepath{fill}%
\end{pgfscope}%
\begin{pgfscope}%
\pgfpathrectangle{\pgfqpoint{0.750000in}{0.500000in}}{\pgfqpoint{4.650000in}{3.020000in}}%
\pgfusepath{clip}%
\pgfsetrectcap%
\pgfsetroundjoin%
\pgfsetlinewidth{0.803000pt}%
\definecolor{currentstroke}{rgb}{0.690196,0.690196,0.690196}%
\pgfsetstrokecolor{currentstroke}%
\pgfsetdash{}{0pt}%
\pgfpathmoveto{\pgfqpoint{1.428358in}{0.500000in}}%
\pgfpathlineto{\pgfqpoint{1.428358in}{3.520000in}}%
\pgfusepath{stroke}%
\end{pgfscope}%
\begin{pgfscope}%
\pgfsetbuttcap%
\pgfsetroundjoin%
\definecolor{currentfill}{rgb}{0.000000,0.000000,0.000000}%
\pgfsetfillcolor{currentfill}%
\pgfsetlinewidth{0.803000pt}%
\definecolor{currentstroke}{rgb}{0.000000,0.000000,0.000000}%
\pgfsetstrokecolor{currentstroke}%
\pgfsetdash{}{0pt}%
\pgfsys@defobject{currentmarker}{\pgfqpoint{0.000000in}{-0.048611in}}{\pgfqpoint{0.000000in}{0.000000in}}{%
\pgfpathmoveto{\pgfqpoint{0.000000in}{0.000000in}}%
\pgfpathlineto{\pgfqpoint{0.000000in}{-0.048611in}}%
\pgfusepath{stroke,fill}%
}%
\begin{pgfscope}%
\pgfsys@transformshift{1.428358in}{0.500000in}%
\pgfsys@useobject{currentmarker}{}%
\end{pgfscope}%
\end{pgfscope}%
\begin{pgfscope}%
\definecolor{textcolor}{rgb}{0.000000,0.000000,0.000000}%
\pgfsetstrokecolor{textcolor}%
\pgfsetfillcolor{textcolor}%
\pgftext[x=1.428358in,y=0.402778in,,top]{\color{textcolor}\sffamily\fontsize{10.000000}{12.000000}\selectfont \(\displaystyle {\ensuremath{-}40}\)}%
\end{pgfscope}%
\begin{pgfscope}%
\pgfpathrectangle{\pgfqpoint{0.750000in}{0.500000in}}{\pgfqpoint{4.650000in}{3.020000in}}%
\pgfusepath{clip}%
\pgfsetrectcap%
\pgfsetroundjoin%
\pgfsetlinewidth{0.803000pt}%
\definecolor{currentstroke}{rgb}{0.690196,0.690196,0.690196}%
\pgfsetstrokecolor{currentstroke}%
\pgfsetdash{}{0pt}%
\pgfpathmoveto{\pgfqpoint{2.112760in}{0.500000in}}%
\pgfpathlineto{\pgfqpoint{2.112760in}{3.520000in}}%
\pgfusepath{stroke}%
\end{pgfscope}%
\begin{pgfscope}%
\pgfsetbuttcap%
\pgfsetroundjoin%
\definecolor{currentfill}{rgb}{0.000000,0.000000,0.000000}%
\pgfsetfillcolor{currentfill}%
\pgfsetlinewidth{0.803000pt}%
\definecolor{currentstroke}{rgb}{0.000000,0.000000,0.000000}%
\pgfsetstrokecolor{currentstroke}%
\pgfsetdash{}{0pt}%
\pgfsys@defobject{currentmarker}{\pgfqpoint{0.000000in}{-0.048611in}}{\pgfqpoint{0.000000in}{0.000000in}}{%
\pgfpathmoveto{\pgfqpoint{0.000000in}{0.000000in}}%
\pgfpathlineto{\pgfqpoint{0.000000in}{-0.048611in}}%
\pgfusepath{stroke,fill}%
}%
\begin{pgfscope}%
\pgfsys@transformshift{2.112760in}{0.500000in}%
\pgfsys@useobject{currentmarker}{}%
\end{pgfscope}%
\end{pgfscope}%
\begin{pgfscope}%
\definecolor{textcolor}{rgb}{0.000000,0.000000,0.000000}%
\pgfsetstrokecolor{textcolor}%
\pgfsetfillcolor{textcolor}%
\pgftext[x=2.112760in,y=0.402778in,,top]{\color{textcolor}\sffamily\fontsize{10.000000}{12.000000}\selectfont \(\displaystyle {\ensuremath{-}30}\)}%
\end{pgfscope}%
\begin{pgfscope}%
\pgfpathrectangle{\pgfqpoint{0.750000in}{0.500000in}}{\pgfqpoint{4.650000in}{3.020000in}}%
\pgfusepath{clip}%
\pgfsetrectcap%
\pgfsetroundjoin%
\pgfsetlinewidth{0.803000pt}%
\definecolor{currentstroke}{rgb}{0.690196,0.690196,0.690196}%
\pgfsetstrokecolor{currentstroke}%
\pgfsetdash{}{0pt}%
\pgfpathmoveto{\pgfqpoint{2.797161in}{0.500000in}}%
\pgfpathlineto{\pgfqpoint{2.797161in}{3.520000in}}%
\pgfusepath{stroke}%
\end{pgfscope}%
\begin{pgfscope}%
\pgfsetbuttcap%
\pgfsetroundjoin%
\definecolor{currentfill}{rgb}{0.000000,0.000000,0.000000}%
\pgfsetfillcolor{currentfill}%
\pgfsetlinewidth{0.803000pt}%
\definecolor{currentstroke}{rgb}{0.000000,0.000000,0.000000}%
\pgfsetstrokecolor{currentstroke}%
\pgfsetdash{}{0pt}%
\pgfsys@defobject{currentmarker}{\pgfqpoint{0.000000in}{-0.048611in}}{\pgfqpoint{0.000000in}{0.000000in}}{%
\pgfpathmoveto{\pgfqpoint{0.000000in}{0.000000in}}%
\pgfpathlineto{\pgfqpoint{0.000000in}{-0.048611in}}%
\pgfusepath{stroke,fill}%
}%
\begin{pgfscope}%
\pgfsys@transformshift{2.797161in}{0.500000in}%
\pgfsys@useobject{currentmarker}{}%
\end{pgfscope}%
\end{pgfscope}%
\begin{pgfscope}%
\definecolor{textcolor}{rgb}{0.000000,0.000000,0.000000}%
\pgfsetstrokecolor{textcolor}%
\pgfsetfillcolor{textcolor}%
\pgftext[x=2.797161in,y=0.402778in,,top]{\color{textcolor}\sffamily\fontsize{10.000000}{12.000000}\selectfont \(\displaystyle {\ensuremath{-}20}\)}%
\end{pgfscope}%
\begin{pgfscope}%
\pgfpathrectangle{\pgfqpoint{0.750000in}{0.500000in}}{\pgfqpoint{4.650000in}{3.020000in}}%
\pgfusepath{clip}%
\pgfsetrectcap%
\pgfsetroundjoin%
\pgfsetlinewidth{0.803000pt}%
\definecolor{currentstroke}{rgb}{0.690196,0.690196,0.690196}%
\pgfsetstrokecolor{currentstroke}%
\pgfsetdash{}{0pt}%
\pgfpathmoveto{\pgfqpoint{3.481563in}{0.500000in}}%
\pgfpathlineto{\pgfqpoint{3.481563in}{3.520000in}}%
\pgfusepath{stroke}%
\end{pgfscope}%
\begin{pgfscope}%
\pgfsetbuttcap%
\pgfsetroundjoin%
\definecolor{currentfill}{rgb}{0.000000,0.000000,0.000000}%
\pgfsetfillcolor{currentfill}%
\pgfsetlinewidth{0.803000pt}%
\definecolor{currentstroke}{rgb}{0.000000,0.000000,0.000000}%
\pgfsetstrokecolor{currentstroke}%
\pgfsetdash{}{0pt}%
\pgfsys@defobject{currentmarker}{\pgfqpoint{0.000000in}{-0.048611in}}{\pgfqpoint{0.000000in}{0.000000in}}{%
\pgfpathmoveto{\pgfqpoint{0.000000in}{0.000000in}}%
\pgfpathlineto{\pgfqpoint{0.000000in}{-0.048611in}}%
\pgfusepath{stroke,fill}%
}%
\begin{pgfscope}%
\pgfsys@transformshift{3.481563in}{0.500000in}%
\pgfsys@useobject{currentmarker}{}%
\end{pgfscope}%
\end{pgfscope}%
\begin{pgfscope}%
\definecolor{textcolor}{rgb}{0.000000,0.000000,0.000000}%
\pgfsetstrokecolor{textcolor}%
\pgfsetfillcolor{textcolor}%
\pgftext[x=3.481563in,y=0.402778in,,top]{\color{textcolor}\sffamily\fontsize{10.000000}{12.000000}\selectfont \(\displaystyle {\ensuremath{-}10}\)}%
\end{pgfscope}%
\begin{pgfscope}%
\pgfpathrectangle{\pgfqpoint{0.750000in}{0.500000in}}{\pgfqpoint{4.650000in}{3.020000in}}%
\pgfusepath{clip}%
\pgfsetrectcap%
\pgfsetroundjoin%
\pgfsetlinewidth{0.803000pt}%
\definecolor{currentstroke}{rgb}{0.690196,0.690196,0.690196}%
\pgfsetstrokecolor{currentstroke}%
\pgfsetdash{}{0pt}%
\pgfpathmoveto{\pgfqpoint{4.165964in}{0.500000in}}%
\pgfpathlineto{\pgfqpoint{4.165964in}{3.520000in}}%
\pgfusepath{stroke}%
\end{pgfscope}%
\begin{pgfscope}%
\pgfsetbuttcap%
\pgfsetroundjoin%
\definecolor{currentfill}{rgb}{0.000000,0.000000,0.000000}%
\pgfsetfillcolor{currentfill}%
\pgfsetlinewidth{0.803000pt}%
\definecolor{currentstroke}{rgb}{0.000000,0.000000,0.000000}%
\pgfsetstrokecolor{currentstroke}%
\pgfsetdash{}{0pt}%
\pgfsys@defobject{currentmarker}{\pgfqpoint{0.000000in}{-0.048611in}}{\pgfqpoint{0.000000in}{0.000000in}}{%
\pgfpathmoveto{\pgfqpoint{0.000000in}{0.000000in}}%
\pgfpathlineto{\pgfqpoint{0.000000in}{-0.048611in}}%
\pgfusepath{stroke,fill}%
}%
\begin{pgfscope}%
\pgfsys@transformshift{4.165964in}{0.500000in}%
\pgfsys@useobject{currentmarker}{}%
\end{pgfscope}%
\end{pgfscope}%
\begin{pgfscope}%
\definecolor{textcolor}{rgb}{0.000000,0.000000,0.000000}%
\pgfsetstrokecolor{textcolor}%
\pgfsetfillcolor{textcolor}%
\pgftext[x=4.165964in,y=0.402778in,,top]{\color{textcolor}\sffamily\fontsize{10.000000}{12.000000}\selectfont \(\displaystyle {0}\)}%
\end{pgfscope}%
\begin{pgfscope}%
\pgfpathrectangle{\pgfqpoint{0.750000in}{0.500000in}}{\pgfqpoint{4.650000in}{3.020000in}}%
\pgfusepath{clip}%
\pgfsetrectcap%
\pgfsetroundjoin%
\pgfsetlinewidth{0.803000pt}%
\definecolor{currentstroke}{rgb}{0.690196,0.690196,0.690196}%
\pgfsetstrokecolor{currentstroke}%
\pgfsetdash{}{0pt}%
\pgfpathmoveto{\pgfqpoint{4.850366in}{0.500000in}}%
\pgfpathlineto{\pgfqpoint{4.850366in}{3.520000in}}%
\pgfusepath{stroke}%
\end{pgfscope}%
\begin{pgfscope}%
\pgfsetbuttcap%
\pgfsetroundjoin%
\definecolor{currentfill}{rgb}{0.000000,0.000000,0.000000}%
\pgfsetfillcolor{currentfill}%
\pgfsetlinewidth{0.803000pt}%
\definecolor{currentstroke}{rgb}{0.000000,0.000000,0.000000}%
\pgfsetstrokecolor{currentstroke}%
\pgfsetdash{}{0pt}%
\pgfsys@defobject{currentmarker}{\pgfqpoint{0.000000in}{-0.048611in}}{\pgfqpoint{0.000000in}{0.000000in}}{%
\pgfpathmoveto{\pgfqpoint{0.000000in}{0.000000in}}%
\pgfpathlineto{\pgfqpoint{0.000000in}{-0.048611in}}%
\pgfusepath{stroke,fill}%
}%
\begin{pgfscope}%
\pgfsys@transformshift{4.850366in}{0.500000in}%
\pgfsys@useobject{currentmarker}{}%
\end{pgfscope}%
\end{pgfscope}%
\begin{pgfscope}%
\definecolor{textcolor}{rgb}{0.000000,0.000000,0.000000}%
\pgfsetstrokecolor{textcolor}%
\pgfsetfillcolor{textcolor}%
\pgftext[x=4.850366in,y=0.402778in,,top]{\color{textcolor}\sffamily\fontsize{10.000000}{12.000000}\selectfont \(\displaystyle {10}\)}%
\end{pgfscope}%
\begin{pgfscope}%
\definecolor{textcolor}{rgb}{0.000000,0.000000,0.000000}%
\pgfsetstrokecolor{textcolor}%
\pgfsetfillcolor{textcolor}%
\pgftext[x=3.075000in,y=0.223766in,,top]{\color{textcolor}\sffamily\fontsize{10.000000}{12.000000}\selectfont \(\displaystyle \zeta \, \mathrm{[\mu m]}\)}%
\end{pgfscope}%
\begin{pgfscope}%
\pgfpathrectangle{\pgfqpoint{0.750000in}{0.500000in}}{\pgfqpoint{4.650000in}{3.020000in}}%
\pgfusepath{clip}%
\pgfsetrectcap%
\pgfsetroundjoin%
\pgfsetlinewidth{0.803000pt}%
\definecolor{currentstroke}{rgb}{0.690196,0.690196,0.690196}%
\pgfsetstrokecolor{currentstroke}%
\pgfsetdash{}{0pt}%
\pgfpathmoveto{\pgfqpoint{0.750000in}{0.654181in}}%
\pgfpathlineto{\pgfqpoint{5.400000in}{0.654181in}}%
\pgfusepath{stroke}%
\end{pgfscope}%
\begin{pgfscope}%
\pgfsetbuttcap%
\pgfsetroundjoin%
\definecolor{currentfill}{rgb}{0.000000,0.000000,0.000000}%
\pgfsetfillcolor{currentfill}%
\pgfsetlinewidth{0.803000pt}%
\definecolor{currentstroke}{rgb}{0.000000,0.000000,0.000000}%
\pgfsetstrokecolor{currentstroke}%
\pgfsetdash{}{0pt}%
\pgfsys@defobject{currentmarker}{\pgfqpoint{-0.048611in}{0.000000in}}{\pgfqpoint{-0.000000in}{0.000000in}}{%
\pgfpathmoveto{\pgfqpoint{-0.000000in}{0.000000in}}%
\pgfpathlineto{\pgfqpoint{-0.048611in}{0.000000in}}%
\pgfusepath{stroke,fill}%
}%
\begin{pgfscope}%
\pgfsys@transformshift{0.750000in}{0.654181in}%
\pgfsys@useobject{currentmarker}{}%
\end{pgfscope}%
\end{pgfscope}%
\begin{pgfscope}%
\definecolor{textcolor}{rgb}{0.000000,0.000000,0.000000}%
\pgfsetstrokecolor{textcolor}%
\pgfsetfillcolor{textcolor}%
\pgftext[x=0.336419in, y=0.605956in, left, base]{\color{textcolor}\sffamily\fontsize{10.000000}{12.000000}\selectfont \(\displaystyle {\ensuremath{-}800}\)}%
\end{pgfscope}%
\begin{pgfscope}%
\pgfpathrectangle{\pgfqpoint{0.750000in}{0.500000in}}{\pgfqpoint{4.650000in}{3.020000in}}%
\pgfusepath{clip}%
\pgfsetrectcap%
\pgfsetroundjoin%
\pgfsetlinewidth{0.803000pt}%
\definecolor{currentstroke}{rgb}{0.690196,0.690196,0.690196}%
\pgfsetstrokecolor{currentstroke}%
\pgfsetdash{}{0pt}%
\pgfpathmoveto{\pgfqpoint{0.750000in}{1.067939in}}%
\pgfpathlineto{\pgfqpoint{5.400000in}{1.067939in}}%
\pgfusepath{stroke}%
\end{pgfscope}%
\begin{pgfscope}%
\pgfsetbuttcap%
\pgfsetroundjoin%
\definecolor{currentfill}{rgb}{0.000000,0.000000,0.000000}%
\pgfsetfillcolor{currentfill}%
\pgfsetlinewidth{0.803000pt}%
\definecolor{currentstroke}{rgb}{0.000000,0.000000,0.000000}%
\pgfsetstrokecolor{currentstroke}%
\pgfsetdash{}{0pt}%
\pgfsys@defobject{currentmarker}{\pgfqpoint{-0.048611in}{0.000000in}}{\pgfqpoint{-0.000000in}{0.000000in}}{%
\pgfpathmoveto{\pgfqpoint{-0.000000in}{0.000000in}}%
\pgfpathlineto{\pgfqpoint{-0.048611in}{0.000000in}}%
\pgfusepath{stroke,fill}%
}%
\begin{pgfscope}%
\pgfsys@transformshift{0.750000in}{1.067939in}%
\pgfsys@useobject{currentmarker}{}%
\end{pgfscope}%
\end{pgfscope}%
\begin{pgfscope}%
\definecolor{textcolor}{rgb}{0.000000,0.000000,0.000000}%
\pgfsetstrokecolor{textcolor}%
\pgfsetfillcolor{textcolor}%
\pgftext[x=0.336419in, y=1.019714in, left, base]{\color{textcolor}\sffamily\fontsize{10.000000}{12.000000}\selectfont \(\displaystyle {\ensuremath{-}600}\)}%
\end{pgfscope}%
\begin{pgfscope}%
\pgfpathrectangle{\pgfqpoint{0.750000in}{0.500000in}}{\pgfqpoint{4.650000in}{3.020000in}}%
\pgfusepath{clip}%
\pgfsetrectcap%
\pgfsetroundjoin%
\pgfsetlinewidth{0.803000pt}%
\definecolor{currentstroke}{rgb}{0.690196,0.690196,0.690196}%
\pgfsetstrokecolor{currentstroke}%
\pgfsetdash{}{0pt}%
\pgfpathmoveto{\pgfqpoint{0.750000in}{1.481697in}}%
\pgfpathlineto{\pgfqpoint{5.400000in}{1.481697in}}%
\pgfusepath{stroke}%
\end{pgfscope}%
\begin{pgfscope}%
\pgfsetbuttcap%
\pgfsetroundjoin%
\definecolor{currentfill}{rgb}{0.000000,0.000000,0.000000}%
\pgfsetfillcolor{currentfill}%
\pgfsetlinewidth{0.803000pt}%
\definecolor{currentstroke}{rgb}{0.000000,0.000000,0.000000}%
\pgfsetstrokecolor{currentstroke}%
\pgfsetdash{}{0pt}%
\pgfsys@defobject{currentmarker}{\pgfqpoint{-0.048611in}{0.000000in}}{\pgfqpoint{-0.000000in}{0.000000in}}{%
\pgfpathmoveto{\pgfqpoint{-0.000000in}{0.000000in}}%
\pgfpathlineto{\pgfqpoint{-0.048611in}{0.000000in}}%
\pgfusepath{stroke,fill}%
}%
\begin{pgfscope}%
\pgfsys@transformshift{0.750000in}{1.481697in}%
\pgfsys@useobject{currentmarker}{}%
\end{pgfscope}%
\end{pgfscope}%
\begin{pgfscope}%
\definecolor{textcolor}{rgb}{0.000000,0.000000,0.000000}%
\pgfsetstrokecolor{textcolor}%
\pgfsetfillcolor{textcolor}%
\pgftext[x=0.336419in, y=1.433472in, left, base]{\color{textcolor}\sffamily\fontsize{10.000000}{12.000000}\selectfont \(\displaystyle {\ensuremath{-}400}\)}%
\end{pgfscope}%
\begin{pgfscope}%
\pgfpathrectangle{\pgfqpoint{0.750000in}{0.500000in}}{\pgfqpoint{4.650000in}{3.020000in}}%
\pgfusepath{clip}%
\pgfsetrectcap%
\pgfsetroundjoin%
\pgfsetlinewidth{0.803000pt}%
\definecolor{currentstroke}{rgb}{0.690196,0.690196,0.690196}%
\pgfsetstrokecolor{currentstroke}%
\pgfsetdash{}{0pt}%
\pgfpathmoveto{\pgfqpoint{0.750000in}{1.895456in}}%
\pgfpathlineto{\pgfqpoint{5.400000in}{1.895456in}}%
\pgfusepath{stroke}%
\end{pgfscope}%
\begin{pgfscope}%
\pgfsetbuttcap%
\pgfsetroundjoin%
\definecolor{currentfill}{rgb}{0.000000,0.000000,0.000000}%
\pgfsetfillcolor{currentfill}%
\pgfsetlinewidth{0.803000pt}%
\definecolor{currentstroke}{rgb}{0.000000,0.000000,0.000000}%
\pgfsetstrokecolor{currentstroke}%
\pgfsetdash{}{0pt}%
\pgfsys@defobject{currentmarker}{\pgfqpoint{-0.048611in}{0.000000in}}{\pgfqpoint{-0.000000in}{0.000000in}}{%
\pgfpathmoveto{\pgfqpoint{-0.000000in}{0.000000in}}%
\pgfpathlineto{\pgfqpoint{-0.048611in}{0.000000in}}%
\pgfusepath{stroke,fill}%
}%
\begin{pgfscope}%
\pgfsys@transformshift{0.750000in}{1.895456in}%
\pgfsys@useobject{currentmarker}{}%
\end{pgfscope}%
\end{pgfscope}%
\begin{pgfscope}%
\definecolor{textcolor}{rgb}{0.000000,0.000000,0.000000}%
\pgfsetstrokecolor{textcolor}%
\pgfsetfillcolor{textcolor}%
\pgftext[x=0.336419in, y=1.847230in, left, base]{\color{textcolor}\sffamily\fontsize{10.000000}{12.000000}\selectfont \(\displaystyle {\ensuremath{-}200}\)}%
\end{pgfscope}%
\begin{pgfscope}%
\pgfpathrectangle{\pgfqpoint{0.750000in}{0.500000in}}{\pgfqpoint{4.650000in}{3.020000in}}%
\pgfusepath{clip}%
\pgfsetrectcap%
\pgfsetroundjoin%
\pgfsetlinewidth{0.803000pt}%
\definecolor{currentstroke}{rgb}{0.690196,0.690196,0.690196}%
\pgfsetstrokecolor{currentstroke}%
\pgfsetdash{}{0pt}%
\pgfpathmoveto{\pgfqpoint{0.750000in}{2.309214in}}%
\pgfpathlineto{\pgfqpoint{5.400000in}{2.309214in}}%
\pgfusepath{stroke}%
\end{pgfscope}%
\begin{pgfscope}%
\pgfsetbuttcap%
\pgfsetroundjoin%
\definecolor{currentfill}{rgb}{0.000000,0.000000,0.000000}%
\pgfsetfillcolor{currentfill}%
\pgfsetlinewidth{0.803000pt}%
\definecolor{currentstroke}{rgb}{0.000000,0.000000,0.000000}%
\pgfsetstrokecolor{currentstroke}%
\pgfsetdash{}{0pt}%
\pgfsys@defobject{currentmarker}{\pgfqpoint{-0.048611in}{0.000000in}}{\pgfqpoint{-0.000000in}{0.000000in}}{%
\pgfpathmoveto{\pgfqpoint{-0.000000in}{0.000000in}}%
\pgfpathlineto{\pgfqpoint{-0.048611in}{0.000000in}}%
\pgfusepath{stroke,fill}%
}%
\begin{pgfscope}%
\pgfsys@transformshift{0.750000in}{2.309214in}%
\pgfsys@useobject{currentmarker}{}%
\end{pgfscope}%
\end{pgfscope}%
\begin{pgfscope}%
\definecolor{textcolor}{rgb}{0.000000,0.000000,0.000000}%
\pgfsetstrokecolor{textcolor}%
\pgfsetfillcolor{textcolor}%
\pgftext[x=0.583333in, y=2.260989in, left, base]{\color{textcolor}\sffamily\fontsize{10.000000}{12.000000}\selectfont \(\displaystyle {0}\)}%
\end{pgfscope}%
\begin{pgfscope}%
\pgfpathrectangle{\pgfqpoint{0.750000in}{0.500000in}}{\pgfqpoint{4.650000in}{3.020000in}}%
\pgfusepath{clip}%
\pgfsetrectcap%
\pgfsetroundjoin%
\pgfsetlinewidth{0.803000pt}%
\definecolor{currentstroke}{rgb}{0.690196,0.690196,0.690196}%
\pgfsetstrokecolor{currentstroke}%
\pgfsetdash{}{0pt}%
\pgfpathmoveto{\pgfqpoint{0.750000in}{2.722972in}}%
\pgfpathlineto{\pgfqpoint{5.400000in}{2.722972in}}%
\pgfusepath{stroke}%
\end{pgfscope}%
\begin{pgfscope}%
\pgfsetbuttcap%
\pgfsetroundjoin%
\definecolor{currentfill}{rgb}{0.000000,0.000000,0.000000}%
\pgfsetfillcolor{currentfill}%
\pgfsetlinewidth{0.803000pt}%
\definecolor{currentstroke}{rgb}{0.000000,0.000000,0.000000}%
\pgfsetstrokecolor{currentstroke}%
\pgfsetdash{}{0pt}%
\pgfsys@defobject{currentmarker}{\pgfqpoint{-0.048611in}{0.000000in}}{\pgfqpoint{-0.000000in}{0.000000in}}{%
\pgfpathmoveto{\pgfqpoint{-0.000000in}{0.000000in}}%
\pgfpathlineto{\pgfqpoint{-0.048611in}{0.000000in}}%
\pgfusepath{stroke,fill}%
}%
\begin{pgfscope}%
\pgfsys@transformshift{0.750000in}{2.722972in}%
\pgfsys@useobject{currentmarker}{}%
\end{pgfscope}%
\end{pgfscope}%
\begin{pgfscope}%
\definecolor{textcolor}{rgb}{0.000000,0.000000,0.000000}%
\pgfsetstrokecolor{textcolor}%
\pgfsetfillcolor{textcolor}%
\pgftext[x=0.444444in, y=2.674747in, left, base]{\color{textcolor}\sffamily\fontsize{10.000000}{12.000000}\selectfont \(\displaystyle {200}\)}%
\end{pgfscope}%
\begin{pgfscope}%
\pgfpathrectangle{\pgfqpoint{0.750000in}{0.500000in}}{\pgfqpoint{4.650000in}{3.020000in}}%
\pgfusepath{clip}%
\pgfsetrectcap%
\pgfsetroundjoin%
\pgfsetlinewidth{0.803000pt}%
\definecolor{currentstroke}{rgb}{0.690196,0.690196,0.690196}%
\pgfsetstrokecolor{currentstroke}%
\pgfsetdash{}{0pt}%
\pgfpathmoveto{\pgfqpoint{0.750000in}{3.136730in}}%
\pgfpathlineto{\pgfqpoint{5.400000in}{3.136730in}}%
\pgfusepath{stroke}%
\end{pgfscope}%
\begin{pgfscope}%
\pgfsetbuttcap%
\pgfsetroundjoin%
\definecolor{currentfill}{rgb}{0.000000,0.000000,0.000000}%
\pgfsetfillcolor{currentfill}%
\pgfsetlinewidth{0.803000pt}%
\definecolor{currentstroke}{rgb}{0.000000,0.000000,0.000000}%
\pgfsetstrokecolor{currentstroke}%
\pgfsetdash{}{0pt}%
\pgfsys@defobject{currentmarker}{\pgfqpoint{-0.048611in}{0.000000in}}{\pgfqpoint{-0.000000in}{0.000000in}}{%
\pgfpathmoveto{\pgfqpoint{-0.000000in}{0.000000in}}%
\pgfpathlineto{\pgfqpoint{-0.048611in}{0.000000in}}%
\pgfusepath{stroke,fill}%
}%
\begin{pgfscope}%
\pgfsys@transformshift{0.750000in}{3.136730in}%
\pgfsys@useobject{currentmarker}{}%
\end{pgfscope}%
\end{pgfscope}%
\begin{pgfscope}%
\definecolor{textcolor}{rgb}{0.000000,0.000000,0.000000}%
\pgfsetstrokecolor{textcolor}%
\pgfsetfillcolor{textcolor}%
\pgftext[x=0.444444in, y=3.088505in, left, base]{\color{textcolor}\sffamily\fontsize{10.000000}{12.000000}\selectfont \(\displaystyle {400}\)}%
\end{pgfscope}%
\begin{pgfscope}%
\definecolor{textcolor}{rgb}{0.000000,0.000000,0.000000}%
\pgfsetstrokecolor{textcolor}%
\pgfsetfillcolor{textcolor}%
\pgftext[x=0.280863in,y=2.010000in,,bottom,rotate=90.000000]{\color{textcolor}\sffamily\fontsize{10.000000}{12.000000}\selectfont \(\displaystyle \mathrm{Energy \, gain \, [MeV]}\)}%
\end{pgfscope}%
\begin{pgfscope}%
\pgfpathrectangle{\pgfqpoint{0.750000in}{0.500000in}}{\pgfqpoint{4.650000in}{3.020000in}}%
\pgfusepath{clip}%
\pgfsetrectcap%
\pgfsetroundjoin%
\pgfsetlinewidth{1.505625pt}%
\definecolor{currentstroke}{rgb}{0.121569,0.466667,0.705882}%
\pgfsetstrokecolor{currentstroke}%
\pgfsetdash{}{0pt}%
\pgfpathmoveto{\pgfqpoint{1.556422in}{1.302322in}}%
\pgfpathlineto{\pgfqpoint{1.568549in}{1.255698in}}%
\pgfpathlineto{\pgfqpoint{1.574613in}{1.239553in}}%
\pgfpathlineto{\pgfqpoint{1.580677in}{1.209941in}}%
\pgfpathlineto{\pgfqpoint{1.598868in}{1.143944in}}%
\pgfpathlineto{\pgfqpoint{1.617060in}{1.074981in}}%
\pgfpathlineto{\pgfqpoint{1.635251in}{1.011818in}}%
\pgfpathlineto{\pgfqpoint{1.641315in}{0.983725in}}%
\pgfpathlineto{\pgfqpoint{1.653443in}{0.944284in}}%
\pgfpathlineto{\pgfqpoint{1.659506in}{0.918507in}}%
\pgfpathlineto{\pgfqpoint{1.677698in}{0.866472in}}%
\pgfpathlineto{\pgfqpoint{1.683762in}{0.850989in}}%
\pgfpathlineto{\pgfqpoint{1.689825in}{0.859572in}}%
\pgfpathlineto{\pgfqpoint{1.695889in}{0.844620in}}%
\pgfpathlineto{\pgfqpoint{1.701953in}{0.864431in}}%
\pgfpathlineto{\pgfqpoint{1.714081in}{1.002852in}}%
\pgfpathlineto{\pgfqpoint{1.726208in}{1.330785in}}%
\pgfpathlineto{\pgfqpoint{1.738336in}{1.735811in}}%
\pgfpathlineto{\pgfqpoint{1.750463in}{2.170647in}}%
\pgfpathlineto{\pgfqpoint{1.756527in}{2.301434in}}%
\pgfpathlineto{\pgfqpoint{1.768655in}{2.402738in}}%
\pgfpathlineto{\pgfqpoint{1.774718in}{2.433528in}}%
\pgfpathlineto{\pgfqpoint{1.786846in}{2.527442in}}%
\pgfpathlineto{\pgfqpoint{1.798974in}{2.610417in}}%
\pgfpathlineto{\pgfqpoint{1.805037in}{2.647666in}}%
\pgfpathlineto{\pgfqpoint{1.811101in}{2.671252in}}%
\pgfpathlineto{\pgfqpoint{1.835356in}{2.798952in}}%
\pgfpathlineto{\pgfqpoint{1.859612in}{2.890284in}}%
\pgfpathlineto{\pgfqpoint{1.865675in}{2.905544in}}%
\pgfpathlineto{\pgfqpoint{1.871739in}{2.927614in}}%
\pgfpathlineto{\pgfqpoint{1.889931in}{2.970680in}}%
\pgfpathlineto{\pgfqpoint{1.902058in}{2.999575in}}%
\pgfpathlineto{\pgfqpoint{1.908122in}{3.007291in}}%
\pgfpathlineto{\pgfqpoint{1.914186in}{3.026018in}}%
\pgfpathlineto{\pgfqpoint{1.920250in}{3.029305in}}%
\pgfpathlineto{\pgfqpoint{1.926313in}{3.047674in}}%
\pgfpathlineto{\pgfqpoint{1.932377in}{3.049208in}}%
\pgfpathlineto{\pgfqpoint{1.938441in}{3.068866in}}%
\pgfpathlineto{\pgfqpoint{1.944505in}{3.070359in}}%
\pgfpathlineto{\pgfqpoint{1.950569in}{3.073520in}}%
\pgfpathlineto{\pgfqpoint{1.956632in}{3.089467in}}%
\pgfpathlineto{\pgfqpoint{1.962696in}{3.085638in}}%
\pgfpathlineto{\pgfqpoint{1.968760in}{3.108142in}}%
\pgfpathlineto{\pgfqpoint{1.974824in}{3.103278in}}%
\pgfpathlineto{\pgfqpoint{1.980888in}{3.110790in}}%
\pgfpathlineto{\pgfqpoint{1.986951in}{3.112445in}}%
\pgfpathlineto{\pgfqpoint{1.999079in}{3.122270in}}%
\pgfpathlineto{\pgfqpoint{2.005143in}{3.126462in}}%
\pgfpathlineto{\pgfqpoint{2.011207in}{3.122206in}}%
\pgfpathlineto{\pgfqpoint{2.017270in}{3.132034in}}%
\pgfpathlineto{\pgfqpoint{2.023334in}{3.130587in}}%
\pgfpathlineto{\pgfqpoint{2.029398in}{3.134094in}}%
\pgfpathlineto{\pgfqpoint{2.035462in}{3.127390in}}%
\pgfpathlineto{\pgfqpoint{2.041526in}{3.133636in}}%
\pgfpathlineto{\pgfqpoint{2.047589in}{3.130594in}}%
\pgfpathlineto{\pgfqpoint{2.053653in}{3.135629in}}%
\pgfpathlineto{\pgfqpoint{2.059717in}{3.127357in}}%
\pgfpathlineto{\pgfqpoint{2.065781in}{3.127399in}}%
\pgfpathlineto{\pgfqpoint{2.071845in}{3.125684in}}%
\pgfpathlineto{\pgfqpoint{2.077908in}{3.117301in}}%
\pgfpathlineto{\pgfqpoint{2.083972in}{3.118334in}}%
\pgfpathlineto{\pgfqpoint{2.090036in}{3.111951in}}%
\pgfpathlineto{\pgfqpoint{2.102164in}{3.104239in}}%
\pgfpathlineto{\pgfqpoint{2.108227in}{3.095574in}}%
\pgfpathlineto{\pgfqpoint{2.114291in}{3.096481in}}%
\pgfpathlineto{\pgfqpoint{2.120355in}{3.079084in}}%
\pgfpathlineto{\pgfqpoint{2.126419in}{3.079009in}}%
\pgfpathlineto{\pgfqpoint{2.132483in}{3.073378in}}%
\pgfpathlineto{\pgfqpoint{2.138546in}{3.062166in}}%
\pgfpathlineto{\pgfqpoint{2.144610in}{3.057112in}}%
\pgfpathlineto{\pgfqpoint{2.156738in}{3.035498in}}%
\pgfpathlineto{\pgfqpoint{2.162802in}{3.037291in}}%
\pgfpathlineto{\pgfqpoint{2.168865in}{3.019041in}}%
\pgfpathlineto{\pgfqpoint{2.174929in}{3.008509in}}%
\pgfpathlineto{\pgfqpoint{2.180993in}{3.001092in}}%
\pgfpathlineto{\pgfqpoint{2.187057in}{2.985415in}}%
\pgfpathlineto{\pgfqpoint{2.193120in}{2.980246in}}%
\pgfpathlineto{\pgfqpoint{2.199184in}{2.971357in}}%
\pgfpathlineto{\pgfqpoint{2.205248in}{2.950678in}}%
\pgfpathlineto{\pgfqpoint{2.211312in}{2.948739in}}%
\pgfpathlineto{\pgfqpoint{2.217376in}{2.930076in}}%
\pgfpathlineto{\pgfqpoint{2.223439in}{2.918355in}}%
\pgfpathlineto{\pgfqpoint{2.229503in}{2.910778in}}%
\pgfpathlineto{\pgfqpoint{2.235567in}{2.889842in}}%
\pgfpathlineto{\pgfqpoint{2.241631in}{2.882863in}}%
\pgfpathlineto{\pgfqpoint{2.247695in}{2.870808in}}%
\pgfpathlineto{\pgfqpoint{2.253758in}{2.853875in}}%
\pgfpathlineto{\pgfqpoint{2.259822in}{2.843414in}}%
\pgfpathlineto{\pgfqpoint{2.284077in}{2.783653in}}%
\pgfpathlineto{\pgfqpoint{2.290141in}{2.765877in}}%
\pgfpathlineto{\pgfqpoint{2.296205in}{2.763399in}}%
\pgfpathlineto{\pgfqpoint{2.302269in}{2.730485in}}%
\pgfpathlineto{\pgfqpoint{2.308333in}{2.727052in}}%
\pgfpathlineto{\pgfqpoint{2.314396in}{2.703905in}}%
\pgfpathlineto{\pgfqpoint{2.320460in}{2.694953in}}%
\pgfpathlineto{\pgfqpoint{2.344715in}{2.626703in}}%
\pgfpathlineto{\pgfqpoint{2.350779in}{2.602126in}}%
\pgfpathlineto{\pgfqpoint{2.356843in}{2.600776in}}%
\pgfpathlineto{\pgfqpoint{2.362907in}{2.570834in}}%
\pgfpathlineto{\pgfqpoint{2.368971in}{2.562142in}}%
\pgfpathlineto{\pgfqpoint{2.375034in}{2.540126in}}%
\pgfpathlineto{\pgfqpoint{2.387162in}{2.509990in}}%
\pgfpathlineto{\pgfqpoint{2.393226in}{2.485576in}}%
\pgfpathlineto{\pgfqpoint{2.399290in}{2.472145in}}%
\pgfpathlineto{\pgfqpoint{2.405353in}{2.452824in}}%
\pgfpathlineto{\pgfqpoint{2.411417in}{2.439574in}}%
\pgfpathlineto{\pgfqpoint{2.423545in}{2.398593in}}%
\pgfpathlineto{\pgfqpoint{2.429609in}{2.382891in}}%
\pgfpathlineto{\pgfqpoint{2.435672in}{2.357477in}}%
\pgfpathlineto{\pgfqpoint{2.441736in}{2.350567in}}%
\pgfpathlineto{\pgfqpoint{2.447800in}{2.320466in}}%
\pgfpathlineto{\pgfqpoint{2.459928in}{2.292426in}}%
\pgfpathlineto{\pgfqpoint{2.465991in}{2.265713in}}%
\pgfpathlineto{\pgfqpoint{2.472055in}{2.258165in}}%
\pgfpathlineto{\pgfqpoint{2.478119in}{2.228477in}}%
\pgfpathlineto{\pgfqpoint{2.484183in}{2.214051in}}%
\pgfpathlineto{\pgfqpoint{2.490247in}{2.191222in}}%
\pgfpathlineto{\pgfqpoint{2.496310in}{2.178741in}}%
\pgfpathlineto{\pgfqpoint{2.502374in}{2.147576in}}%
\pgfpathlineto{\pgfqpoint{2.514502in}{2.118708in}}%
\pgfpathlineto{\pgfqpoint{2.520566in}{2.087329in}}%
\pgfpathlineto{\pgfqpoint{2.526629in}{2.087560in}}%
\pgfpathlineto{\pgfqpoint{2.532693in}{2.051290in}}%
\pgfpathlineto{\pgfqpoint{2.538757in}{2.039329in}}%
\pgfpathlineto{\pgfqpoint{2.544821in}{2.020991in}}%
\pgfpathlineto{\pgfqpoint{2.550885in}{1.991851in}}%
\pgfpathlineto{\pgfqpoint{2.556948in}{1.983040in}}%
\pgfpathlineto{\pgfqpoint{2.563012in}{1.961579in}}%
\pgfpathlineto{\pgfqpoint{2.569076in}{1.933310in}}%
\pgfpathlineto{\pgfqpoint{2.575140in}{1.921980in}}%
\pgfpathlineto{\pgfqpoint{2.581204in}{1.895577in}}%
\pgfpathlineto{\pgfqpoint{2.599395in}{1.841990in}}%
\pgfpathlineto{\pgfqpoint{2.605459in}{1.813874in}}%
\pgfpathlineto{\pgfqpoint{2.611523in}{1.797722in}}%
\pgfpathlineto{\pgfqpoint{2.617586in}{1.773169in}}%
\pgfpathlineto{\pgfqpoint{2.623650in}{1.760875in}}%
\pgfpathlineto{\pgfqpoint{2.629714in}{1.729039in}}%
\pgfpathlineto{\pgfqpoint{2.635778in}{1.719377in}}%
\pgfpathlineto{\pgfqpoint{2.647905in}{1.669057in}}%
\pgfpathlineto{\pgfqpoint{2.653969in}{1.660449in}}%
\pgfpathlineto{\pgfqpoint{2.660033in}{1.631309in}}%
\pgfpathlineto{\pgfqpoint{2.666097in}{1.620130in}}%
\pgfpathlineto{\pgfqpoint{2.672160in}{1.583946in}}%
\pgfpathlineto{\pgfqpoint{2.678224in}{1.574849in}}%
\pgfpathlineto{\pgfqpoint{2.696416in}{1.506522in}}%
\pgfpathlineto{\pgfqpoint{2.702479in}{1.495939in}}%
\pgfpathlineto{\pgfqpoint{2.708543in}{1.465474in}}%
\pgfpathlineto{\pgfqpoint{2.714607in}{1.449889in}}%
\pgfpathlineto{\pgfqpoint{2.726735in}{1.404038in}}%
\pgfpathlineto{\pgfqpoint{2.732798in}{1.394900in}}%
\pgfpathlineto{\pgfqpoint{2.738862in}{1.362038in}}%
\pgfpathlineto{\pgfqpoint{2.744926in}{1.348351in}}%
\pgfpathlineto{\pgfqpoint{2.750990in}{1.321288in}}%
\pgfpathlineto{\pgfqpoint{2.757054in}{1.304574in}}%
\pgfpathlineto{\pgfqpoint{2.769181in}{1.262986in}}%
\pgfpathlineto{\pgfqpoint{2.775245in}{1.245132in}}%
\pgfpathlineto{\pgfqpoint{2.781309in}{1.217984in}}%
\pgfpathlineto{\pgfqpoint{2.787373in}{1.207614in}}%
\pgfpathlineto{\pgfqpoint{2.799500in}{1.158174in}}%
\pgfpathlineto{\pgfqpoint{2.811628in}{1.122770in}}%
\pgfpathlineto{\pgfqpoint{2.817692in}{1.092323in}}%
\pgfpathlineto{\pgfqpoint{2.823755in}{1.082487in}}%
\pgfpathlineto{\pgfqpoint{2.829819in}{1.061466in}}%
\pgfpathlineto{\pgfqpoint{2.835883in}{1.029754in}}%
\pgfpathlineto{\pgfqpoint{2.841947in}{1.033196in}}%
\pgfpathlineto{\pgfqpoint{2.848011in}{0.994280in}}%
\pgfpathlineto{\pgfqpoint{2.854074in}{0.977003in}}%
\pgfpathlineto{\pgfqpoint{2.860138in}{0.977005in}}%
\pgfpathlineto{\pgfqpoint{2.866202in}{0.934778in}}%
\pgfpathlineto{\pgfqpoint{2.872266in}{0.931239in}}%
\pgfpathlineto{\pgfqpoint{2.878330in}{0.917198in}}%
\pgfpathlineto{\pgfqpoint{2.884393in}{0.881730in}}%
\pgfpathlineto{\pgfqpoint{2.890457in}{0.898116in}}%
\pgfpathlineto{\pgfqpoint{2.896521in}{0.861138in}}%
\pgfpathlineto{\pgfqpoint{2.902585in}{0.865154in}}%
\pgfpathlineto{\pgfqpoint{2.914712in}{0.862174in}}%
\pgfpathlineto{\pgfqpoint{2.920776in}{0.890871in}}%
\pgfpathlineto{\pgfqpoint{2.926840in}{0.927566in}}%
\pgfpathlineto{\pgfqpoint{2.932904in}{0.988667in}}%
\pgfpathlineto{\pgfqpoint{2.938968in}{1.128206in}}%
\pgfpathlineto{\pgfqpoint{2.945031in}{1.310374in}}%
\pgfpathlineto{\pgfqpoint{2.957159in}{2.088449in}}%
\pgfpathlineto{\pgfqpoint{2.963223in}{2.472700in}}%
\pgfpathlineto{\pgfqpoint{2.969287in}{2.658921in}}%
\pgfpathlineto{\pgfqpoint{2.981414in}{2.699252in}}%
\pgfpathlineto{\pgfqpoint{2.993542in}{2.819709in}}%
\pgfpathlineto{\pgfqpoint{2.999606in}{2.828085in}}%
\pgfpathlineto{\pgfqpoint{3.011733in}{2.834985in}}%
\pgfpathlineto{\pgfqpoint{3.017797in}{2.868335in}}%
\pgfpathlineto{\pgfqpoint{3.023861in}{2.869599in}}%
\pgfpathlineto{\pgfqpoint{3.029925in}{2.905504in}}%
\pgfpathlineto{\pgfqpoint{3.035988in}{2.918742in}}%
\pgfpathlineto{\pgfqpoint{3.042052in}{2.935993in}}%
\pgfpathlineto{\pgfqpoint{3.048116in}{2.971065in}}%
\pgfpathlineto{\pgfqpoint{3.054180in}{2.972146in}}%
\pgfpathlineto{\pgfqpoint{3.066307in}{3.019278in}}%
\pgfpathlineto{\pgfqpoint{3.072371in}{3.025936in}}%
\pgfpathlineto{\pgfqpoint{3.078435in}{3.052843in}}%
\pgfpathlineto{\pgfqpoint{3.084499in}{3.059600in}}%
\pgfpathlineto{\pgfqpoint{3.090562in}{3.082042in}}%
\pgfpathlineto{\pgfqpoint{3.096626in}{3.081541in}}%
\pgfpathlineto{\pgfqpoint{3.102690in}{3.118834in}}%
\pgfpathlineto{\pgfqpoint{3.108754in}{3.104711in}}%
\pgfpathlineto{\pgfqpoint{3.114818in}{3.136119in}}%
\pgfpathlineto{\pgfqpoint{3.120881in}{3.131630in}}%
\pgfpathlineto{\pgfqpoint{3.133009in}{3.163525in}}%
\pgfpathlineto{\pgfqpoint{3.139073in}{3.149425in}}%
\pgfpathlineto{\pgfqpoint{3.145137in}{3.184090in}}%
\pgfpathlineto{\pgfqpoint{3.151200in}{3.171035in}}%
\pgfpathlineto{\pgfqpoint{3.163328in}{3.199284in}}%
\pgfpathlineto{\pgfqpoint{3.169392in}{3.184350in}}%
\pgfpathlineto{\pgfqpoint{3.181519in}{3.210712in}}%
\pgfpathlineto{\pgfqpoint{3.187583in}{3.191780in}}%
\pgfpathlineto{\pgfqpoint{3.193647in}{3.218591in}}%
\pgfpathlineto{\pgfqpoint{3.199711in}{3.206615in}}%
\pgfpathlineto{\pgfqpoint{3.205775in}{3.206750in}}%
\pgfpathlineto{\pgfqpoint{3.211838in}{3.211341in}}%
\pgfpathlineto{\pgfqpoint{3.217902in}{3.209856in}}%
\pgfpathlineto{\pgfqpoint{3.223966in}{3.200721in}}%
\pgfpathlineto{\pgfqpoint{3.230030in}{3.218801in}}%
\pgfpathlineto{\pgfqpoint{3.236094in}{3.189516in}}%
\pgfpathlineto{\pgfqpoint{3.242157in}{3.212165in}}%
\pgfpathlineto{\pgfqpoint{3.254285in}{3.186817in}}%
\pgfpathlineto{\pgfqpoint{3.260349in}{3.196160in}}%
\pgfpathlineto{\pgfqpoint{3.266413in}{3.177333in}}%
\pgfpathlineto{\pgfqpoint{3.278540in}{3.182357in}}%
\pgfpathlineto{\pgfqpoint{3.284604in}{3.158115in}}%
\pgfpathlineto{\pgfqpoint{3.290668in}{3.168120in}}%
\pgfpathlineto{\pgfqpoint{3.296732in}{3.152531in}}%
\pgfpathlineto{\pgfqpoint{3.302795in}{3.142604in}}%
\pgfpathlineto{\pgfqpoint{3.308859in}{3.140085in}}%
\pgfpathlineto{\pgfqpoint{3.333114in}{3.100142in}}%
\pgfpathlineto{\pgfqpoint{3.339178in}{3.096483in}}%
\pgfpathlineto{\pgfqpoint{3.345242in}{3.070494in}}%
\pgfpathlineto{\pgfqpoint{3.351306in}{3.084379in}}%
\pgfpathlineto{\pgfqpoint{3.357370in}{3.044585in}}%
\pgfpathlineto{\pgfqpoint{3.363433in}{3.056883in}}%
\pgfpathlineto{\pgfqpoint{3.369497in}{3.029214in}}%
\pgfpathlineto{\pgfqpoint{3.375561in}{3.026902in}}%
\pgfpathlineto{\pgfqpoint{3.381625in}{3.017856in}}%
\pgfpathlineto{\pgfqpoint{3.387689in}{2.998994in}}%
\pgfpathlineto{\pgfqpoint{3.393752in}{3.005197in}}%
\pgfpathlineto{\pgfqpoint{3.399816in}{2.983304in}}%
\pgfpathlineto{\pgfqpoint{3.405880in}{2.970123in}}%
\pgfpathlineto{\pgfqpoint{3.411944in}{2.972501in}}%
\pgfpathlineto{\pgfqpoint{3.418008in}{2.947214in}}%
\pgfpathlineto{\pgfqpoint{3.424071in}{2.943731in}}%
\pgfpathlineto{\pgfqpoint{3.430135in}{2.934780in}}%
\pgfpathlineto{\pgfqpoint{3.436199in}{2.910756in}}%
\pgfpathlineto{\pgfqpoint{3.442263in}{2.911712in}}%
\pgfpathlineto{\pgfqpoint{3.448327in}{2.890569in}}%
\pgfpathlineto{\pgfqpoint{3.454390in}{2.885702in}}%
\pgfpathlineto{\pgfqpoint{3.460454in}{2.866415in}}%
\pgfpathlineto{\pgfqpoint{3.466518in}{2.857028in}}%
\pgfpathlineto{\pgfqpoint{3.472582in}{2.841535in}}%
\pgfpathlineto{\pgfqpoint{3.478646in}{2.830508in}}%
\pgfpathlineto{\pgfqpoint{3.484709in}{2.810667in}}%
\pgfpathlineto{\pgfqpoint{3.496837in}{2.785504in}}%
\pgfpathlineto{\pgfqpoint{3.502901in}{2.764777in}}%
\pgfpathlineto{\pgfqpoint{3.508965in}{2.763588in}}%
\pgfpathlineto{\pgfqpoint{3.515028in}{2.727694in}}%
\pgfpathlineto{\pgfqpoint{3.521092in}{2.733169in}}%
\pgfpathlineto{\pgfqpoint{3.527156in}{2.700720in}}%
\pgfpathlineto{\pgfqpoint{3.533220in}{2.687282in}}%
\pgfpathlineto{\pgfqpoint{3.539283in}{2.684856in}}%
\pgfpathlineto{\pgfqpoint{3.545347in}{2.641490in}}%
\pgfpathlineto{\pgfqpoint{3.551411in}{2.654114in}}%
\pgfpathlineto{\pgfqpoint{3.557475in}{2.618618in}}%
\pgfpathlineto{\pgfqpoint{3.569602in}{2.592117in}}%
\pgfpathlineto{\pgfqpoint{3.575666in}{2.567523in}}%
\pgfpathlineto{\pgfqpoint{3.581730in}{2.551643in}}%
\pgfpathlineto{\pgfqpoint{3.587794in}{2.542733in}}%
\pgfpathlineto{\pgfqpoint{3.593858in}{2.500533in}}%
\pgfpathlineto{\pgfqpoint{3.599921in}{2.514376in}}%
\pgfpathlineto{\pgfqpoint{3.605985in}{2.476409in}}%
\pgfpathlineto{\pgfqpoint{3.612049in}{2.461090in}}%
\pgfpathlineto{\pgfqpoint{3.618113in}{2.454649in}}%
\pgfpathlineto{\pgfqpoint{3.624177in}{2.424549in}}%
\pgfpathlineto{\pgfqpoint{3.630240in}{2.405670in}}%
\pgfpathlineto{\pgfqpoint{3.636304in}{2.402356in}}%
\pgfpathlineto{\pgfqpoint{3.642368in}{2.360722in}}%
\pgfpathlineto{\pgfqpoint{3.648432in}{2.359722in}}%
\pgfpathlineto{\pgfqpoint{3.660559in}{2.313796in}}%
\pgfpathlineto{\pgfqpoint{3.666623in}{2.305338in}}%
\pgfpathlineto{\pgfqpoint{3.678751in}{2.261434in}}%
\pgfpathlineto{\pgfqpoint{3.684815in}{2.256374in}}%
\pgfpathlineto{\pgfqpoint{3.690878in}{2.221794in}}%
\pgfpathlineto{\pgfqpoint{3.696942in}{2.217683in}}%
\pgfpathlineto{\pgfqpoint{3.703006in}{2.202107in}}%
\pgfpathlineto{\pgfqpoint{3.709070in}{2.168231in}}%
\pgfpathlineto{\pgfqpoint{3.715134in}{2.180252in}}%
\pgfpathlineto{\pgfqpoint{3.721197in}{2.137198in}}%
\pgfpathlineto{\pgfqpoint{3.727261in}{2.140701in}}%
\pgfpathlineto{\pgfqpoint{3.733325in}{2.118335in}}%
\pgfpathlineto{\pgfqpoint{3.739389in}{2.109086in}}%
\pgfpathlineto{\pgfqpoint{3.745453in}{2.088069in}}%
\pgfpathlineto{\pgfqpoint{3.751516in}{2.079828in}}%
\pgfpathlineto{\pgfqpoint{3.757580in}{2.062242in}}%
\pgfpathlineto{\pgfqpoint{3.763644in}{2.054795in}}%
\pgfpathlineto{\pgfqpoint{3.769708in}{2.026823in}}%
\pgfpathlineto{\pgfqpoint{3.775772in}{2.032106in}}%
\pgfpathlineto{\pgfqpoint{3.787899in}{1.985669in}}%
\pgfpathlineto{\pgfqpoint{3.793963in}{1.988337in}}%
\pgfpathlineto{\pgfqpoint{3.800027in}{1.971070in}}%
\pgfpathlineto{\pgfqpoint{3.806091in}{1.942145in}}%
\pgfpathlineto{\pgfqpoint{3.812154in}{1.955736in}}%
\pgfpathlineto{\pgfqpoint{3.818218in}{1.907391in}}%
\pgfpathlineto{\pgfqpoint{3.824282in}{1.924095in}}%
\pgfpathlineto{\pgfqpoint{3.836410in}{1.874785in}}%
\pgfpathlineto{\pgfqpoint{3.842473in}{1.889622in}}%
\pgfpathlineto{\pgfqpoint{3.848537in}{1.837926in}}%
\pgfpathlineto{\pgfqpoint{3.854601in}{1.861208in}}%
\pgfpathlineto{\pgfqpoint{3.860665in}{1.817572in}}%
\pgfpathlineto{\pgfqpoint{3.866729in}{1.827291in}}%
\pgfpathlineto{\pgfqpoint{3.872792in}{1.798776in}}%
\pgfpathlineto{\pgfqpoint{3.878856in}{1.808076in}}%
\pgfpathlineto{\pgfqpoint{3.884920in}{1.765814in}}%
\pgfpathlineto{\pgfqpoint{3.890984in}{1.787616in}}%
\pgfpathlineto{\pgfqpoint{3.897048in}{1.739430in}}%
\pgfpathlineto{\pgfqpoint{3.903111in}{1.761971in}}%
\pgfpathlineto{\pgfqpoint{3.909175in}{1.723470in}}%
\pgfpathlineto{\pgfqpoint{3.915239in}{1.733401in}}%
\pgfpathlineto{\pgfqpoint{3.921303in}{1.710484in}}%
\pgfpathlineto{\pgfqpoint{3.927367in}{1.702464in}}%
\pgfpathlineto{\pgfqpoint{3.933430in}{1.697096in}}%
\pgfpathlineto{\pgfqpoint{3.939494in}{1.689244in}}%
\pgfpathlineto{\pgfqpoint{3.945558in}{1.668563in}}%
\pgfpathlineto{\pgfqpoint{3.951622in}{1.684253in}}%
\pgfpathlineto{\pgfqpoint{3.957686in}{1.644200in}}%
\pgfpathlineto{\pgfqpoint{3.963749in}{1.664316in}}%
\pgfpathlineto{\pgfqpoint{3.969813in}{1.642209in}}%
\pgfpathlineto{\pgfqpoint{3.975877in}{1.647869in}}%
\pgfpathlineto{\pgfqpoint{3.981941in}{1.636348in}}%
\pgfpathlineto{\pgfqpoint{3.988004in}{1.632442in}}%
\pgfpathlineto{\pgfqpoint{3.994068in}{1.622659in}}%
\pgfpathlineto{\pgfqpoint{4.000132in}{1.631604in}}%
\pgfpathlineto{\pgfqpoint{4.006196in}{1.604552in}}%
\pgfpathlineto{\pgfqpoint{4.012260in}{1.644271in}}%
\pgfpathlineto{\pgfqpoint{4.018323in}{1.597896in}}%
\pgfpathlineto{\pgfqpoint{4.024387in}{1.643768in}}%
\pgfpathlineto{\pgfqpoint{4.030451in}{1.596367in}}%
\pgfpathlineto{\pgfqpoint{4.036515in}{1.636084in}}%
\pgfpathlineto{\pgfqpoint{4.042579in}{1.607402in}}%
\pgfpathlineto{\pgfqpoint{4.048642in}{1.626691in}}%
\pgfpathlineto{\pgfqpoint{4.054706in}{1.630931in}}%
\pgfpathlineto{\pgfqpoint{4.060770in}{1.629363in}}%
\pgfpathlineto{\pgfqpoint{4.066834in}{1.641549in}}%
\pgfpathlineto{\pgfqpoint{4.072898in}{1.633287in}}%
\pgfpathlineto{\pgfqpoint{4.078961in}{1.663244in}}%
\pgfpathlineto{\pgfqpoint{4.085025in}{1.635290in}}%
\pgfpathlineto{\pgfqpoint{4.091089in}{1.671631in}}%
\pgfpathlineto{\pgfqpoint{4.097153in}{1.670811in}}%
\pgfpathlineto{\pgfqpoint{4.103217in}{1.682208in}}%
\pgfpathlineto{\pgfqpoint{4.109280in}{1.686434in}}%
\pgfpathlineto{\pgfqpoint{4.115344in}{1.710643in}}%
\pgfpathlineto{\pgfqpoint{4.121408in}{1.710460in}}%
\pgfpathlineto{\pgfqpoint{4.127472in}{1.727620in}}%
\pgfpathlineto{\pgfqpoint{4.133536in}{1.737679in}}%
\pgfpathlineto{\pgfqpoint{4.151727in}{1.790370in}}%
\pgfpathlineto{\pgfqpoint{4.182046in}{1.866025in}}%
\pgfpathlineto{\pgfqpoint{4.188110in}{1.885892in}}%
\pgfpathlineto{\pgfqpoint{4.194174in}{1.898590in}}%
\pgfpathlineto{\pgfqpoint{4.200237in}{1.920960in}}%
\pgfpathlineto{\pgfqpoint{4.206301in}{1.933781in}}%
\pgfpathlineto{\pgfqpoint{4.212365in}{1.957869in}}%
\pgfpathlineto{\pgfqpoint{4.218429in}{1.969870in}}%
\pgfpathlineto{\pgfqpoint{4.224493in}{1.992389in}}%
\pgfpathlineto{\pgfqpoint{4.230556in}{2.003489in}}%
\pgfpathlineto{\pgfqpoint{4.248748in}{2.044097in}}%
\pgfpathlineto{\pgfqpoint{4.254812in}{2.067217in}}%
\pgfpathlineto{\pgfqpoint{4.260875in}{2.081874in}}%
\pgfpathlineto{\pgfqpoint{4.266939in}{2.087328in}}%
\pgfpathlineto{\pgfqpoint{4.273003in}{2.111725in}}%
\pgfpathlineto{\pgfqpoint{4.279067in}{2.110550in}}%
\pgfpathlineto{\pgfqpoint{4.285131in}{2.140683in}}%
\pgfpathlineto{\pgfqpoint{4.291194in}{2.135528in}}%
\pgfpathlineto{\pgfqpoint{4.297258in}{2.162377in}}%
\pgfpathlineto{\pgfqpoint{4.303322in}{2.162182in}}%
\pgfpathlineto{\pgfqpoint{4.315450in}{2.182704in}}%
\pgfpathlineto{\pgfqpoint{4.321513in}{2.190522in}}%
\pgfpathlineto{\pgfqpoint{4.327577in}{2.201649in}}%
\pgfpathlineto{\pgfqpoint{4.333641in}{2.203386in}}%
\pgfpathlineto{\pgfqpoint{4.339705in}{2.215608in}}%
\pgfpathlineto{\pgfqpoint{4.345769in}{2.216878in}}%
\pgfpathlineto{\pgfqpoint{4.351832in}{2.229373in}}%
\pgfpathlineto{\pgfqpoint{4.357896in}{2.230006in}}%
\pgfpathlineto{\pgfqpoint{4.370024in}{2.245541in}}%
\pgfpathlineto{\pgfqpoint{4.376088in}{2.247159in}}%
\pgfpathlineto{\pgfqpoint{4.382151in}{2.251586in}}%
\pgfpathlineto{\pgfqpoint{4.388215in}{2.260583in}}%
\pgfpathlineto{\pgfqpoint{4.394279in}{2.258425in}}%
\pgfpathlineto{\pgfqpoint{4.400343in}{2.265611in}}%
\pgfpathlineto{\pgfqpoint{4.406407in}{2.265388in}}%
\pgfpathlineto{\pgfqpoint{4.412470in}{2.271772in}}%
\pgfpathlineto{\pgfqpoint{4.418534in}{2.272543in}}%
\pgfpathlineto{\pgfqpoint{4.430662in}{2.279196in}}%
\pgfpathlineto{\pgfqpoint{4.436725in}{2.279225in}}%
\pgfpathlineto{\pgfqpoint{4.442789in}{2.283992in}}%
\pgfpathlineto{\pgfqpoint{4.448853in}{2.283752in}}%
\pgfpathlineto{\pgfqpoint{4.454917in}{2.288683in}}%
\pgfpathlineto{\pgfqpoint{4.460981in}{2.287258in}}%
\pgfpathlineto{\pgfqpoint{4.467044in}{2.291374in}}%
\pgfpathlineto{\pgfqpoint{4.473108in}{2.290464in}}%
\pgfpathlineto{\pgfqpoint{4.479172in}{2.293605in}}%
\pgfpathlineto{\pgfqpoint{4.491300in}{2.294791in}}%
\pgfpathlineto{\pgfqpoint{4.497363in}{2.297152in}}%
\pgfpathlineto{\pgfqpoint{4.503427in}{2.295482in}}%
\pgfpathlineto{\pgfqpoint{4.509491in}{2.301003in}}%
\pgfpathlineto{\pgfqpoint{4.515555in}{2.297067in}}%
\pgfpathlineto{\pgfqpoint{4.521619in}{2.300764in}}%
\pgfpathlineto{\pgfqpoint{4.539810in}{2.300698in}}%
\pgfpathlineto{\pgfqpoint{4.545874in}{2.302357in}}%
\pgfpathlineto{\pgfqpoint{4.558001in}{2.302904in}}%
\pgfpathlineto{\pgfqpoint{4.594384in}{2.306638in}}%
\pgfpathlineto{\pgfqpoint{4.612576in}{2.306342in}}%
\pgfpathlineto{\pgfqpoint{4.618639in}{2.307949in}}%
\pgfpathlineto{\pgfqpoint{4.624703in}{2.306754in}}%
\pgfpathlineto{\pgfqpoint{4.630767in}{2.308644in}}%
\pgfpathlineto{\pgfqpoint{4.636831in}{2.307506in}}%
\pgfpathlineto{\pgfqpoint{4.655022in}{2.308340in}}%
\pgfpathlineto{\pgfqpoint{4.836936in}{2.309171in}}%
\pgfpathlineto{\pgfqpoint{5.188636in}{2.309211in}}%
\pgfpathlineto{\pgfqpoint{5.188636in}{2.309211in}}%
\pgfusepath{stroke}%
\end{pgfscope}%
\begin{pgfscope}%
\pgfpathrectangle{\pgfqpoint{0.750000in}{0.500000in}}{\pgfqpoint{4.650000in}{3.020000in}}%
\pgfusepath{clip}%
\pgfsetrectcap%
\pgfsetroundjoin%
\pgfsetlinewidth{1.505625pt}%
\definecolor{currentstroke}{rgb}{1.000000,0.498039,0.054902}%
\pgfsetstrokecolor{currentstroke}%
\pgfsetdash{}{0pt}%
\pgfpathmoveto{\pgfqpoint{0.962824in}{3.144227in}}%
\pgfpathlineto{\pgfqpoint{0.974951in}{3.129625in}}%
\pgfpathlineto{\pgfqpoint{0.981015in}{3.124513in}}%
\pgfpathlineto{\pgfqpoint{0.987079in}{3.116411in}}%
\pgfpathlineto{\pgfqpoint{0.993143in}{3.102391in}}%
\pgfpathlineto{\pgfqpoint{0.999207in}{3.099162in}}%
\pgfpathlineto{\pgfqpoint{1.005270in}{3.082263in}}%
\pgfpathlineto{\pgfqpoint{1.017398in}{3.064799in}}%
\pgfpathlineto{\pgfqpoint{1.023462in}{3.050717in}}%
\pgfpathlineto{\pgfqpoint{1.029526in}{3.043765in}}%
\pgfpathlineto{\pgfqpoint{1.035589in}{3.032062in}}%
\pgfpathlineto{\pgfqpoint{1.053781in}{2.989359in}}%
\pgfpathlineto{\pgfqpoint{1.071972in}{2.954399in}}%
\pgfpathlineto{\pgfqpoint{1.078036in}{2.934434in}}%
\pgfpathlineto{\pgfqpoint{1.102291in}{2.876747in}}%
\pgfpathlineto{\pgfqpoint{1.108355in}{2.857702in}}%
\pgfpathlineto{\pgfqpoint{1.114419in}{2.844327in}}%
\pgfpathlineto{\pgfqpoint{1.150801in}{2.742635in}}%
\pgfpathlineto{\pgfqpoint{1.156865in}{2.727465in}}%
\pgfpathlineto{\pgfqpoint{1.162929in}{2.704342in}}%
\pgfpathlineto{\pgfqpoint{1.168993in}{2.690406in}}%
\pgfpathlineto{\pgfqpoint{1.199312in}{2.593844in}}%
\pgfpathlineto{\pgfqpoint{1.296333in}{2.267645in}}%
\pgfpathlineto{\pgfqpoint{1.302396in}{2.238759in}}%
\pgfpathlineto{\pgfqpoint{1.308460in}{2.222944in}}%
\pgfpathlineto{\pgfqpoint{1.314524in}{2.193440in}}%
\pgfpathlineto{\pgfqpoint{1.326652in}{2.154349in}}%
\pgfpathlineto{\pgfqpoint{1.332715in}{2.124509in}}%
\pgfpathlineto{\pgfqpoint{1.338779in}{2.112602in}}%
\pgfpathlineto{\pgfqpoint{1.344843in}{2.083043in}}%
\pgfpathlineto{\pgfqpoint{1.356971in}{2.038878in}}%
\pgfpathlineto{\pgfqpoint{1.363034in}{2.010919in}}%
\pgfpathlineto{\pgfqpoint{1.369098in}{1.995075in}}%
\pgfpathlineto{\pgfqpoint{1.381226in}{1.941988in}}%
\pgfpathlineto{\pgfqpoint{1.387290in}{1.924575in}}%
\pgfpathlineto{\pgfqpoint{1.411545in}{1.818915in}}%
\pgfpathlineto{\pgfqpoint{1.423672in}{1.779752in}}%
\pgfpathlineto{\pgfqpoint{1.429736in}{1.751390in}}%
\pgfpathlineto{\pgfqpoint{1.435800in}{1.734671in}}%
\pgfpathlineto{\pgfqpoint{1.441864in}{1.701937in}}%
\pgfpathlineto{\pgfqpoint{1.472183in}{1.581580in}}%
\pgfpathlineto{\pgfqpoint{1.490374in}{1.504428in}}%
\pgfpathlineto{\pgfqpoint{1.496438in}{1.483887in}}%
\pgfpathlineto{\pgfqpoint{1.514629in}{1.402300in}}%
\pgfpathlineto{\pgfqpoint{1.557076in}{1.227074in}}%
\pgfpathlineto{\pgfqpoint{1.569203in}{1.171541in}}%
\pgfpathlineto{\pgfqpoint{1.575267in}{1.157021in}}%
\pgfpathlineto{\pgfqpoint{1.581331in}{1.117417in}}%
\pgfpathlineto{\pgfqpoint{1.593459in}{1.072563in}}%
\pgfpathlineto{\pgfqpoint{1.599522in}{1.038437in}}%
\pgfpathlineto{\pgfqpoint{1.605586in}{1.022325in}}%
\pgfpathlineto{\pgfqpoint{1.611650in}{0.987921in}}%
\pgfpathlineto{\pgfqpoint{1.623778in}{0.936247in}}%
\pgfpathlineto{\pgfqpoint{1.629841in}{0.908140in}}%
\pgfpathlineto{\pgfqpoint{1.635905in}{0.891699in}}%
\pgfpathlineto{\pgfqpoint{1.648033in}{0.842047in}}%
\pgfpathlineto{\pgfqpoint{1.672288in}{0.745929in}}%
\pgfpathlineto{\pgfqpoint{1.678352in}{0.734929in}}%
\pgfpathlineto{\pgfqpoint{1.684416in}{0.719726in}}%
\pgfpathlineto{\pgfqpoint{1.696543in}{0.703880in}}%
\pgfpathlineto{\pgfqpoint{1.708671in}{0.742949in}}%
\pgfpathlineto{\pgfqpoint{1.714735in}{0.794044in}}%
\pgfpathlineto{\pgfqpoint{1.720798in}{0.882404in}}%
\pgfpathlineto{\pgfqpoint{1.726862in}{0.997343in}}%
\pgfpathlineto{\pgfqpoint{1.738990in}{1.385078in}}%
\pgfpathlineto{\pgfqpoint{1.751117in}{1.846769in}}%
\pgfpathlineto{\pgfqpoint{1.757181in}{1.992809in}}%
\pgfpathlineto{\pgfqpoint{1.763245in}{2.093673in}}%
\pgfpathlineto{\pgfqpoint{1.769309in}{2.159375in}}%
\pgfpathlineto{\pgfqpoint{1.775373in}{2.200404in}}%
\pgfpathlineto{\pgfqpoint{1.787500in}{2.305496in}}%
\pgfpathlineto{\pgfqpoint{1.805692in}{2.443799in}}%
\pgfpathlineto{\pgfqpoint{1.811755in}{2.471268in}}%
\pgfpathlineto{\pgfqpoint{1.817819in}{2.516958in}}%
\pgfpathlineto{\pgfqpoint{1.872393in}{2.807910in}}%
\pgfpathlineto{\pgfqpoint{1.878457in}{2.826546in}}%
\pgfpathlineto{\pgfqpoint{1.890585in}{2.884761in}}%
\pgfpathlineto{\pgfqpoint{1.920904in}{3.001252in}}%
\pgfpathlineto{\pgfqpoint{1.926968in}{3.016915in}}%
\pgfpathlineto{\pgfqpoint{1.933031in}{3.044619in}}%
\pgfpathlineto{\pgfqpoint{1.939095in}{3.054875in}}%
\pgfpathlineto{\pgfqpoint{1.945159in}{3.079941in}}%
\pgfpathlineto{\pgfqpoint{1.951223in}{3.089017in}}%
\pgfpathlineto{\pgfqpoint{1.957287in}{3.113977in}}%
\pgfpathlineto{\pgfqpoint{1.963350in}{3.116460in}}%
\pgfpathlineto{\pgfqpoint{1.969414in}{3.142472in}}%
\pgfpathlineto{\pgfqpoint{1.975478in}{3.145525in}}%
\pgfpathlineto{\pgfqpoint{1.981542in}{3.163735in}}%
\pgfpathlineto{\pgfqpoint{1.987606in}{3.175888in}}%
\pgfpathlineto{\pgfqpoint{1.993669in}{3.177956in}}%
\pgfpathlineto{\pgfqpoint{1.999733in}{3.200973in}}%
\pgfpathlineto{\pgfqpoint{2.005797in}{3.202827in}}%
\pgfpathlineto{\pgfqpoint{2.011861in}{3.214376in}}%
\pgfpathlineto{\pgfqpoint{2.023988in}{3.228541in}}%
\pgfpathlineto{\pgfqpoint{2.030052in}{3.233277in}}%
\pgfpathlineto{\pgfqpoint{2.042180in}{3.238758in}}%
\pgfpathlineto{\pgfqpoint{2.048243in}{3.247275in}}%
\pgfpathlineto{\pgfqpoint{2.054307in}{3.247479in}}%
\pgfpathlineto{\pgfqpoint{2.060371in}{3.249496in}}%
\pgfpathlineto{\pgfqpoint{2.066435in}{3.255194in}}%
\pgfpathlineto{\pgfqpoint{2.078562in}{3.251231in}}%
\pgfpathlineto{\pgfqpoint{2.084626in}{3.254289in}}%
\pgfpathlineto{\pgfqpoint{2.090690in}{3.243975in}}%
\pgfpathlineto{\pgfqpoint{2.096754in}{3.250105in}}%
\pgfpathlineto{\pgfqpoint{2.108881in}{3.239522in}}%
\pgfpathlineto{\pgfqpoint{2.114945in}{3.241236in}}%
\pgfpathlineto{\pgfqpoint{2.121009in}{3.228790in}}%
\pgfpathlineto{\pgfqpoint{2.127073in}{3.229798in}}%
\pgfpathlineto{\pgfqpoint{2.133137in}{3.218630in}}%
\pgfpathlineto{\pgfqpoint{2.139200in}{3.215000in}}%
\pgfpathlineto{\pgfqpoint{2.145264in}{3.200791in}}%
\pgfpathlineto{\pgfqpoint{2.151328in}{3.205434in}}%
\pgfpathlineto{\pgfqpoint{2.157392in}{3.184664in}}%
\pgfpathlineto{\pgfqpoint{2.163456in}{3.185511in}}%
\pgfpathlineto{\pgfqpoint{2.169519in}{3.164795in}}%
\pgfpathlineto{\pgfqpoint{2.175583in}{3.166899in}}%
\pgfpathlineto{\pgfqpoint{2.181647in}{3.148487in}}%
\pgfpathlineto{\pgfqpoint{2.187711in}{3.143817in}}%
\pgfpathlineto{\pgfqpoint{2.193775in}{3.126228in}}%
\pgfpathlineto{\pgfqpoint{2.199838in}{3.120524in}}%
\pgfpathlineto{\pgfqpoint{2.205902in}{3.109176in}}%
\pgfpathlineto{\pgfqpoint{2.211966in}{3.092824in}}%
\pgfpathlineto{\pgfqpoint{2.218030in}{3.084438in}}%
\pgfpathlineto{\pgfqpoint{2.224094in}{3.061407in}}%
\pgfpathlineto{\pgfqpoint{2.230157in}{3.057775in}}%
\pgfpathlineto{\pgfqpoint{2.236221in}{3.036012in}}%
\pgfpathlineto{\pgfqpoint{2.242285in}{3.029724in}}%
\pgfpathlineto{\pgfqpoint{2.248349in}{3.006209in}}%
\pgfpathlineto{\pgfqpoint{2.254413in}{3.002035in}}%
\pgfpathlineto{\pgfqpoint{2.260476in}{2.977249in}}%
\pgfpathlineto{\pgfqpoint{2.284732in}{2.918807in}}%
\pgfpathlineto{\pgfqpoint{2.302923in}{2.861547in}}%
\pgfpathlineto{\pgfqpoint{2.308987in}{2.850614in}}%
\pgfpathlineto{\pgfqpoint{2.333242in}{2.773022in}}%
\pgfpathlineto{\pgfqpoint{2.339306in}{2.757815in}}%
\pgfpathlineto{\pgfqpoint{2.345370in}{2.733407in}}%
\pgfpathlineto{\pgfqpoint{2.351433in}{2.720010in}}%
\pgfpathlineto{\pgfqpoint{2.363561in}{2.679883in}}%
\pgfpathlineto{\pgfqpoint{2.369625in}{2.662768in}}%
\pgfpathlineto{\pgfqpoint{2.381752in}{2.621941in}}%
\pgfpathlineto{\pgfqpoint{2.387816in}{2.603068in}}%
\pgfpathlineto{\pgfqpoint{2.393880in}{2.574821in}}%
\pgfpathlineto{\pgfqpoint{2.399944in}{2.564190in}}%
\pgfpathlineto{\pgfqpoint{2.406008in}{2.535198in}}%
\pgfpathlineto{\pgfqpoint{2.412071in}{2.519405in}}%
\pgfpathlineto{\pgfqpoint{2.430263in}{2.450569in}}%
\pgfpathlineto{\pgfqpoint{2.436327in}{2.437665in}}%
\pgfpathlineto{\pgfqpoint{2.442390in}{2.411957in}}%
\pgfpathlineto{\pgfqpoint{2.448454in}{2.392973in}}%
\pgfpathlineto{\pgfqpoint{2.454518in}{2.365801in}}%
\pgfpathlineto{\pgfqpoint{2.460582in}{2.352030in}}%
\pgfpathlineto{\pgfqpoint{2.466646in}{2.318969in}}%
\pgfpathlineto{\pgfqpoint{2.472709in}{2.308589in}}%
\pgfpathlineto{\pgfqpoint{2.490901in}{2.232901in}}%
\pgfpathlineto{\pgfqpoint{2.496964in}{2.214749in}}%
\pgfpathlineto{\pgfqpoint{2.521220in}{2.121352in}}%
\pgfpathlineto{\pgfqpoint{2.527283in}{2.103692in}}%
\pgfpathlineto{\pgfqpoint{2.533347in}{2.073290in}}%
\pgfpathlineto{\pgfqpoint{2.539411in}{2.058839in}}%
\pgfpathlineto{\pgfqpoint{2.545475in}{2.026825in}}%
\pgfpathlineto{\pgfqpoint{2.551539in}{2.012413in}}%
\pgfpathlineto{\pgfqpoint{2.557602in}{1.980591in}}%
\pgfpathlineto{\pgfqpoint{2.563666in}{1.963385in}}%
\pgfpathlineto{\pgfqpoint{2.569730in}{1.940683in}}%
\pgfpathlineto{\pgfqpoint{2.575794in}{1.910303in}}%
\pgfpathlineto{\pgfqpoint{2.593985in}{1.845922in}}%
\pgfpathlineto{\pgfqpoint{2.600049in}{1.812209in}}%
\pgfpathlineto{\pgfqpoint{2.606113in}{1.804330in}}%
\pgfpathlineto{\pgfqpoint{2.612177in}{1.768453in}}%
\pgfpathlineto{\pgfqpoint{2.642496in}{1.653806in}}%
\pgfpathlineto{\pgfqpoint{2.648559in}{1.624971in}}%
\pgfpathlineto{\pgfqpoint{2.654623in}{1.609634in}}%
\pgfpathlineto{\pgfqpoint{2.666751in}{1.552455in}}%
\pgfpathlineto{\pgfqpoint{2.672815in}{1.532265in}}%
\pgfpathlineto{\pgfqpoint{2.678878in}{1.505450in}}%
\pgfpathlineto{\pgfqpoint{2.684942in}{1.488233in}}%
\pgfpathlineto{\pgfqpoint{2.691006in}{1.457341in}}%
\pgfpathlineto{\pgfqpoint{2.703134in}{1.412899in}}%
\pgfpathlineto{\pgfqpoint{2.715261in}{1.360256in}}%
\pgfpathlineto{\pgfqpoint{2.721325in}{1.344282in}}%
\pgfpathlineto{\pgfqpoint{2.727389in}{1.305408in}}%
\pgfpathlineto{\pgfqpoint{2.733453in}{1.296666in}}%
\pgfpathlineto{\pgfqpoint{2.745580in}{1.239176in}}%
\pgfpathlineto{\pgfqpoint{2.751644in}{1.224509in}}%
\pgfpathlineto{\pgfqpoint{2.757708in}{1.189491in}}%
\pgfpathlineto{\pgfqpoint{2.763772in}{1.164278in}}%
\pgfpathlineto{\pgfqpoint{2.769835in}{1.150983in}}%
\pgfpathlineto{\pgfqpoint{2.775899in}{1.108886in}}%
\pgfpathlineto{\pgfqpoint{2.781963in}{1.107542in}}%
\pgfpathlineto{\pgfqpoint{2.788027in}{1.060493in}}%
\pgfpathlineto{\pgfqpoint{2.794091in}{1.054612in}}%
\pgfpathlineto{\pgfqpoint{2.806218in}{1.000995in}}%
\pgfpathlineto{\pgfqpoint{2.818346in}{0.953404in}}%
\pgfpathlineto{\pgfqpoint{2.824410in}{0.929553in}}%
\pgfpathlineto{\pgfqpoint{2.830473in}{0.913442in}}%
\pgfpathlineto{\pgfqpoint{2.836537in}{0.883572in}}%
\pgfpathlineto{\pgfqpoint{2.848665in}{0.843162in}}%
\pgfpathlineto{\pgfqpoint{2.854729in}{0.821272in}}%
\pgfpathlineto{\pgfqpoint{2.860792in}{0.805827in}}%
\pgfpathlineto{\pgfqpoint{2.866856in}{0.783372in}}%
\pgfpathlineto{\pgfqpoint{2.872920in}{0.753975in}}%
\pgfpathlineto{\pgfqpoint{2.878984in}{0.755118in}}%
\pgfpathlineto{\pgfqpoint{2.885048in}{0.723057in}}%
\pgfpathlineto{\pgfqpoint{2.891111in}{0.719377in}}%
\pgfpathlineto{\pgfqpoint{2.897175in}{0.700497in}}%
\pgfpathlineto{\pgfqpoint{2.903239in}{0.705375in}}%
\pgfpathlineto{\pgfqpoint{2.909303in}{0.692905in}}%
\pgfpathlineto{\pgfqpoint{2.921430in}{0.729150in}}%
\pgfpathlineto{\pgfqpoint{2.927494in}{0.767553in}}%
\pgfpathlineto{\pgfqpoint{2.933558in}{0.822413in}}%
\pgfpathlineto{\pgfqpoint{2.939622in}{0.930363in}}%
\pgfpathlineto{\pgfqpoint{2.945685in}{1.099913in}}%
\pgfpathlineto{\pgfqpoint{2.951749in}{1.387634in}}%
\pgfpathlineto{\pgfqpoint{2.963877in}{2.073363in}}%
\pgfpathlineto{\pgfqpoint{2.969941in}{2.227464in}}%
\pgfpathlineto{\pgfqpoint{2.976004in}{2.295912in}}%
\pgfpathlineto{\pgfqpoint{2.988132in}{2.388528in}}%
\pgfpathlineto{\pgfqpoint{2.994196in}{2.489260in}}%
\pgfpathlineto{\pgfqpoint{3.000260in}{2.522994in}}%
\pgfpathlineto{\pgfqpoint{3.006323in}{2.569044in}}%
\pgfpathlineto{\pgfqpoint{3.012387in}{2.577645in}}%
\pgfpathlineto{\pgfqpoint{3.018451in}{2.605078in}}%
\pgfpathlineto{\pgfqpoint{3.024515in}{2.659277in}}%
\pgfpathlineto{\pgfqpoint{3.030579in}{2.673096in}}%
\pgfpathlineto{\pgfqpoint{3.036642in}{2.732335in}}%
\pgfpathlineto{\pgfqpoint{3.048770in}{2.793109in}}%
\pgfpathlineto{\pgfqpoint{3.054834in}{2.842389in}}%
\pgfpathlineto{\pgfqpoint{3.060898in}{2.865869in}}%
\pgfpathlineto{\pgfqpoint{3.073025in}{2.941382in}}%
\pgfpathlineto{\pgfqpoint{3.079089in}{2.966585in}}%
\pgfpathlineto{\pgfqpoint{3.085153in}{3.001059in}}%
\pgfpathlineto{\pgfqpoint{3.121536in}{3.146697in}}%
\pgfpathlineto{\pgfqpoint{3.139727in}{3.210311in}}%
\pgfpathlineto{\pgfqpoint{3.145791in}{3.210233in}}%
\pgfpathlineto{\pgfqpoint{3.151855in}{3.251477in}}%
\pgfpathlineto{\pgfqpoint{3.157918in}{3.236065in}}%
\pgfpathlineto{\pgfqpoint{3.163982in}{3.279066in}}%
\pgfpathlineto{\pgfqpoint{3.170046in}{3.272939in}}%
\pgfpathlineto{\pgfqpoint{3.176110in}{3.286897in}}%
\pgfpathlineto{\pgfqpoint{3.182174in}{3.305472in}}%
\pgfpathlineto{\pgfqpoint{3.188237in}{3.303806in}}%
\pgfpathlineto{\pgfqpoint{3.200365in}{3.327145in}}%
\pgfpathlineto{\pgfqpoint{3.206429in}{3.324312in}}%
\pgfpathlineto{\pgfqpoint{3.212493in}{3.337714in}}%
\pgfpathlineto{\pgfqpoint{3.218556in}{3.339368in}}%
\pgfpathlineto{\pgfqpoint{3.224620in}{3.338413in}}%
\pgfpathlineto{\pgfqpoint{3.230684in}{3.348942in}}%
\pgfpathlineto{\pgfqpoint{3.236748in}{3.345371in}}%
\pgfpathlineto{\pgfqpoint{3.242812in}{3.348401in}}%
\pgfpathlineto{\pgfqpoint{3.254939in}{3.345914in}}%
\pgfpathlineto{\pgfqpoint{3.261003in}{3.343272in}}%
\pgfpathlineto{\pgfqpoint{3.267067in}{3.348208in}}%
\pgfpathlineto{\pgfqpoint{3.273131in}{3.331181in}}%
\pgfpathlineto{\pgfqpoint{3.279194in}{3.345545in}}%
\pgfpathlineto{\pgfqpoint{3.285258in}{3.322313in}}%
\pgfpathlineto{\pgfqpoint{3.291322in}{3.330210in}}%
\pgfpathlineto{\pgfqpoint{3.297386in}{3.316730in}}%
\pgfpathlineto{\pgfqpoint{3.303450in}{3.308998in}}%
\pgfpathlineto{\pgfqpoint{3.309513in}{3.309102in}}%
\pgfpathlineto{\pgfqpoint{3.315577in}{3.290422in}}%
\pgfpathlineto{\pgfqpoint{3.321641in}{3.293309in}}%
\pgfpathlineto{\pgfqpoint{3.333769in}{3.264881in}}%
\pgfpathlineto{\pgfqpoint{3.339832in}{3.260571in}}%
\pgfpathlineto{\pgfqpoint{3.351960in}{3.236716in}}%
\pgfpathlineto{\pgfqpoint{3.358024in}{3.220855in}}%
\pgfpathlineto{\pgfqpoint{3.364088in}{3.215820in}}%
\pgfpathlineto{\pgfqpoint{3.376215in}{3.185635in}}%
\pgfpathlineto{\pgfqpoint{3.388343in}{3.163524in}}%
\pgfpathlineto{\pgfqpoint{3.394406in}{3.142217in}}%
\pgfpathlineto{\pgfqpoint{3.400470in}{3.133272in}}%
\pgfpathlineto{\pgfqpoint{3.412598in}{3.099963in}}%
\pgfpathlineto{\pgfqpoint{3.418662in}{3.088329in}}%
\pgfpathlineto{\pgfqpoint{3.442917in}{3.020112in}}%
\pgfpathlineto{\pgfqpoint{3.448981in}{3.011308in}}%
\pgfpathlineto{\pgfqpoint{3.455044in}{2.983482in}}%
\pgfpathlineto{\pgfqpoint{3.467172in}{2.962415in}}%
\pgfpathlineto{\pgfqpoint{3.473236in}{2.930883in}}%
\pgfpathlineto{\pgfqpoint{3.479300in}{2.923681in}}%
\pgfpathlineto{\pgfqpoint{3.485363in}{2.901279in}}%
\pgfpathlineto{\pgfqpoint{3.491427in}{2.889925in}}%
\pgfpathlineto{\pgfqpoint{3.497491in}{2.862391in}}%
\pgfpathlineto{\pgfqpoint{3.503555in}{2.855874in}}%
\pgfpathlineto{\pgfqpoint{3.509619in}{2.828766in}}%
\pgfpathlineto{\pgfqpoint{3.515682in}{2.808794in}}%
\pgfpathlineto{\pgfqpoint{3.521746in}{2.797549in}}%
\pgfpathlineto{\pgfqpoint{3.527810in}{2.770004in}}%
\pgfpathlineto{\pgfqpoint{3.539938in}{2.736775in}}%
\pgfpathlineto{\pgfqpoint{3.552065in}{2.692794in}}%
\pgfpathlineto{\pgfqpoint{3.558129in}{2.684940in}}%
\pgfpathlineto{\pgfqpoint{3.564193in}{2.641258in}}%
\pgfpathlineto{\pgfqpoint{3.570257in}{2.654286in}}%
\pgfpathlineto{\pgfqpoint{3.576320in}{2.595436in}}%
\pgfpathlineto{\pgfqpoint{3.582384in}{2.610081in}}%
\pgfpathlineto{\pgfqpoint{3.588448in}{2.567525in}}%
\pgfpathlineto{\pgfqpoint{3.594512in}{2.551278in}}%
\pgfpathlineto{\pgfqpoint{3.600576in}{2.544838in}}%
\pgfpathlineto{\pgfqpoint{3.606639in}{2.508063in}}%
\pgfpathlineto{\pgfqpoint{3.612703in}{2.490326in}}%
\pgfpathlineto{\pgfqpoint{3.618767in}{2.479699in}}%
\pgfpathlineto{\pgfqpoint{3.624831in}{2.440695in}}%
\pgfpathlineto{\pgfqpoint{3.630895in}{2.439279in}}%
\pgfpathlineto{\pgfqpoint{3.636958in}{2.404458in}}%
\pgfpathlineto{\pgfqpoint{3.643022in}{2.391371in}}%
\pgfpathlineto{\pgfqpoint{3.649086in}{2.365473in}}%
\pgfpathlineto{\pgfqpoint{3.655150in}{2.347940in}}%
\pgfpathlineto{\pgfqpoint{3.661214in}{2.316567in}}%
\pgfpathlineto{\pgfqpoint{3.667277in}{2.320858in}}%
\pgfpathlineto{\pgfqpoint{3.673341in}{2.270382in}}%
\pgfpathlineto{\pgfqpoint{3.679405in}{2.269135in}}%
\pgfpathlineto{\pgfqpoint{3.685469in}{2.246266in}}%
\pgfpathlineto{\pgfqpoint{3.691533in}{2.211062in}}%
\pgfpathlineto{\pgfqpoint{3.697596in}{2.214490in}}%
\pgfpathlineto{\pgfqpoint{3.703660in}{2.170029in}}%
\pgfpathlineto{\pgfqpoint{3.709724in}{2.177267in}}%
\pgfpathlineto{\pgfqpoint{3.715788in}{2.119582in}}%
\pgfpathlineto{\pgfqpoint{3.721852in}{2.144978in}}%
\pgfpathlineto{\pgfqpoint{3.727915in}{2.082317in}}%
\pgfpathlineto{\pgfqpoint{3.733979in}{2.085916in}}%
\pgfpathlineto{\pgfqpoint{3.752171in}{2.024522in}}%
\pgfpathlineto{\pgfqpoint{3.758234in}{2.009659in}}%
\pgfpathlineto{\pgfqpoint{3.764298in}{1.989122in}}%
\pgfpathlineto{\pgfqpoint{3.770362in}{1.979545in}}%
\pgfpathlineto{\pgfqpoint{3.776426in}{1.954694in}}%
\pgfpathlineto{\pgfqpoint{3.782490in}{1.943298in}}%
\pgfpathlineto{\pgfqpoint{3.788553in}{1.925411in}}%
\pgfpathlineto{\pgfqpoint{3.794617in}{1.912459in}}%
\pgfpathlineto{\pgfqpoint{3.800681in}{1.893959in}}%
\pgfpathlineto{\pgfqpoint{3.806745in}{1.885678in}}%
\pgfpathlineto{\pgfqpoint{3.812809in}{1.856374in}}%
\pgfpathlineto{\pgfqpoint{3.818872in}{1.869441in}}%
\pgfpathlineto{\pgfqpoint{3.824936in}{1.829416in}}%
\pgfpathlineto{\pgfqpoint{3.831000in}{1.829075in}}%
\pgfpathlineto{\pgfqpoint{3.837064in}{1.808678in}}%
\pgfpathlineto{\pgfqpoint{3.849191in}{1.787518in}}%
\pgfpathlineto{\pgfqpoint{3.855255in}{1.767628in}}%
\pgfpathlineto{\pgfqpoint{3.861319in}{1.764159in}}%
\pgfpathlineto{\pgfqpoint{3.867383in}{1.736806in}}%
\pgfpathlineto{\pgfqpoint{3.873446in}{1.743570in}}%
\pgfpathlineto{\pgfqpoint{3.879510in}{1.716310in}}%
\pgfpathlineto{\pgfqpoint{3.891638in}{1.702670in}}%
\pgfpathlineto{\pgfqpoint{3.903765in}{1.670228in}}%
\pgfpathlineto{\pgfqpoint{3.909829in}{1.665897in}}%
\pgfpathlineto{\pgfqpoint{3.915893in}{1.665810in}}%
\pgfpathlineto{\pgfqpoint{3.921957in}{1.635807in}}%
\pgfpathlineto{\pgfqpoint{3.928021in}{1.658736in}}%
\pgfpathlineto{\pgfqpoint{3.934084in}{1.601527in}}%
\pgfpathlineto{\pgfqpoint{3.940148in}{1.661132in}}%
\pgfpathlineto{\pgfqpoint{3.946212in}{1.576457in}}%
\pgfpathlineto{\pgfqpoint{3.952276in}{1.640049in}}%
\pgfpathlineto{\pgfqpoint{3.958340in}{1.584968in}}%
\pgfpathlineto{\pgfqpoint{3.964403in}{1.600810in}}%
\pgfpathlineto{\pgfqpoint{3.970467in}{1.604427in}}%
\pgfpathlineto{\pgfqpoint{3.976531in}{1.570251in}}%
\pgfpathlineto{\pgfqpoint{3.982595in}{1.605013in}}%
\pgfpathlineto{\pgfqpoint{3.988659in}{1.573959in}}%
\pgfpathlineto{\pgfqpoint{3.994722in}{1.585104in}}%
\pgfpathlineto{\pgfqpoint{4.000786in}{1.574974in}}%
\pgfpathlineto{\pgfqpoint{4.006850in}{1.584031in}}%
\pgfpathlineto{\pgfqpoint{4.012914in}{1.574152in}}%
\pgfpathlineto{\pgfqpoint{4.018978in}{1.593163in}}%
\pgfpathlineto{\pgfqpoint{4.025041in}{1.573619in}}%
\pgfpathlineto{\pgfqpoint{4.031105in}{1.592689in}}%
\pgfpathlineto{\pgfqpoint{4.037169in}{1.585954in}}%
\pgfpathlineto{\pgfqpoint{4.043233in}{1.603003in}}%
\pgfpathlineto{\pgfqpoint{4.049297in}{1.587021in}}%
\pgfpathlineto{\pgfqpoint{4.055360in}{1.614989in}}%
\pgfpathlineto{\pgfqpoint{4.061424in}{1.613853in}}%
\pgfpathlineto{\pgfqpoint{4.067488in}{1.620021in}}%
\pgfpathlineto{\pgfqpoint{4.073552in}{1.634437in}}%
\pgfpathlineto{\pgfqpoint{4.079616in}{1.629179in}}%
\pgfpathlineto{\pgfqpoint{4.085679in}{1.654442in}}%
\pgfpathlineto{\pgfqpoint{4.091743in}{1.663078in}}%
\pgfpathlineto{\pgfqpoint{4.097807in}{1.677590in}}%
\pgfpathlineto{\pgfqpoint{4.103871in}{1.680599in}}%
\pgfpathlineto{\pgfqpoint{4.109935in}{1.713246in}}%
\pgfpathlineto{\pgfqpoint{4.115998in}{1.701775in}}%
\pgfpathlineto{\pgfqpoint{4.122062in}{1.738478in}}%
\pgfpathlineto{\pgfqpoint{4.128126in}{1.734761in}}%
\pgfpathlineto{\pgfqpoint{4.134190in}{1.758609in}}%
\pgfpathlineto{\pgfqpoint{4.140254in}{1.766003in}}%
\pgfpathlineto{\pgfqpoint{4.146317in}{1.794032in}}%
\pgfpathlineto{\pgfqpoint{4.152381in}{1.808889in}}%
\pgfpathlineto{\pgfqpoint{4.158445in}{1.814438in}}%
\pgfpathlineto{\pgfqpoint{4.164509in}{1.845767in}}%
\pgfpathlineto{\pgfqpoint{4.170573in}{1.855832in}}%
\pgfpathlineto{\pgfqpoint{4.176636in}{1.872913in}}%
\pgfpathlineto{\pgfqpoint{4.182700in}{1.897501in}}%
\pgfpathlineto{\pgfqpoint{4.188764in}{1.894899in}}%
\pgfpathlineto{\pgfqpoint{4.194828in}{1.939906in}}%
\pgfpathlineto{\pgfqpoint{4.200892in}{1.936573in}}%
\pgfpathlineto{\pgfqpoint{4.206955in}{1.964585in}}%
\pgfpathlineto{\pgfqpoint{4.219083in}{1.986024in}}%
\pgfpathlineto{\pgfqpoint{4.225147in}{2.011676in}}%
\pgfpathlineto{\pgfqpoint{4.231211in}{2.014210in}}%
\pgfpathlineto{\pgfqpoint{4.243338in}{2.054445in}}%
\pgfpathlineto{\pgfqpoint{4.249402in}{2.060832in}}%
\pgfpathlineto{\pgfqpoint{4.255466in}{2.082678in}}%
\pgfpathlineto{\pgfqpoint{4.261530in}{2.091143in}}%
\pgfpathlineto{\pgfqpoint{4.273657in}{2.116060in}}%
\pgfpathlineto{\pgfqpoint{4.279721in}{2.125813in}}%
\pgfpathlineto{\pgfqpoint{4.285785in}{2.140414in}}%
\pgfpathlineto{\pgfqpoint{4.291848in}{2.145990in}}%
\pgfpathlineto{\pgfqpoint{4.297912in}{2.154745in}}%
\pgfpathlineto{\pgfqpoint{4.310040in}{2.175384in}}%
\pgfpathlineto{\pgfqpoint{4.316104in}{2.185310in}}%
\pgfpathlineto{\pgfqpoint{4.322167in}{2.186342in}}%
\pgfpathlineto{\pgfqpoint{4.334295in}{2.208005in}}%
\pgfpathlineto{\pgfqpoint{4.340359in}{2.210945in}}%
\pgfpathlineto{\pgfqpoint{4.346423in}{2.224573in}}%
\pgfpathlineto{\pgfqpoint{4.352486in}{2.223043in}}%
\pgfpathlineto{\pgfqpoint{4.358550in}{2.234873in}}%
\pgfpathlineto{\pgfqpoint{4.364614in}{2.236955in}}%
\pgfpathlineto{\pgfqpoint{4.394933in}{2.261183in}}%
\pgfpathlineto{\pgfqpoint{4.400997in}{2.260280in}}%
\pgfpathlineto{\pgfqpoint{4.407061in}{2.268771in}}%
\pgfpathlineto{\pgfqpoint{4.413124in}{2.269361in}}%
\pgfpathlineto{\pgfqpoint{4.425252in}{2.276197in}}%
\pgfpathlineto{\pgfqpoint{4.431316in}{2.277055in}}%
\pgfpathlineto{\pgfqpoint{4.437380in}{2.280460in}}%
\pgfpathlineto{\pgfqpoint{4.485890in}{2.293673in}}%
\pgfpathlineto{\pgfqpoint{4.491954in}{2.292111in}}%
\pgfpathlineto{\pgfqpoint{4.498018in}{2.295694in}}%
\pgfpathlineto{\pgfqpoint{4.504081in}{2.296384in}}%
\pgfpathlineto{\pgfqpoint{4.510145in}{2.299164in}}%
\pgfpathlineto{\pgfqpoint{4.528337in}{2.300004in}}%
\pgfpathlineto{\pgfqpoint{4.534400in}{2.302157in}}%
\pgfpathlineto{\pgfqpoint{4.540464in}{2.301856in}}%
\pgfpathlineto{\pgfqpoint{4.546528in}{2.303180in}}%
\pgfpathlineto{\pgfqpoint{4.552592in}{2.302314in}}%
\pgfpathlineto{\pgfqpoint{4.558656in}{2.304071in}}%
\pgfpathlineto{\pgfqpoint{4.564719in}{2.303666in}}%
\pgfpathlineto{\pgfqpoint{4.576847in}{2.305419in}}%
\pgfpathlineto{\pgfqpoint{4.595038in}{2.305824in}}%
\pgfpathlineto{\pgfqpoint{4.595038in}{2.305824in}}%
\pgfusepath{stroke}%
\end{pgfscope}%
\begin{pgfscope}%
\pgfpathrectangle{\pgfqpoint{0.750000in}{0.500000in}}{\pgfqpoint{4.650000in}{3.020000in}}%
\pgfusepath{clip}%
\pgfsetrectcap%
\pgfsetroundjoin%
\pgfsetlinewidth{1.505625pt}%
\definecolor{currentstroke}{rgb}{0.172549,0.627451,0.172549}%
\pgfsetstrokecolor{currentstroke}%
\pgfsetdash{}{0pt}%
\pgfpathmoveto{\pgfqpoint{0.961364in}{3.184452in}}%
\pgfpathlineto{\pgfqpoint{0.967427in}{3.173400in}}%
\pgfpathlineto{\pgfqpoint{0.973491in}{3.169486in}}%
\pgfpathlineto{\pgfqpoint{0.985619in}{3.151899in}}%
\pgfpathlineto{\pgfqpoint{0.991683in}{3.143268in}}%
\pgfpathlineto{\pgfqpoint{1.015938in}{3.098812in}}%
\pgfpathlineto{\pgfqpoint{1.034129in}{3.067224in}}%
\pgfpathlineto{\pgfqpoint{1.040193in}{3.051654in}}%
\pgfpathlineto{\pgfqpoint{1.046257in}{3.041022in}}%
\pgfpathlineto{\pgfqpoint{1.064448in}{2.997179in}}%
\pgfpathlineto{\pgfqpoint{1.076576in}{2.969442in}}%
\pgfpathlineto{\pgfqpoint{1.088703in}{2.938903in}}%
\pgfpathlineto{\pgfqpoint{1.094767in}{2.927451in}}%
\pgfpathlineto{\pgfqpoint{1.100831in}{2.902130in}}%
\pgfpathlineto{\pgfqpoint{1.106895in}{2.891405in}}%
\pgfpathlineto{\pgfqpoint{1.112959in}{2.876316in}}%
\pgfpathlineto{\pgfqpoint{1.119022in}{2.852780in}}%
\pgfpathlineto{\pgfqpoint{1.125086in}{2.845728in}}%
\pgfpathlineto{\pgfqpoint{1.131150in}{2.823509in}}%
\pgfpathlineto{\pgfqpoint{1.137214in}{2.812435in}}%
\pgfpathlineto{\pgfqpoint{1.143278in}{2.786435in}}%
\pgfpathlineto{\pgfqpoint{1.149341in}{2.768123in}}%
\pgfpathlineto{\pgfqpoint{1.155405in}{2.755727in}}%
\pgfpathlineto{\pgfqpoint{1.161469in}{2.732431in}}%
\pgfpathlineto{\pgfqpoint{1.173597in}{2.696957in}}%
\pgfpathlineto{\pgfqpoint{1.185724in}{2.655156in}}%
\pgfpathlineto{\pgfqpoint{1.191788in}{2.639623in}}%
\pgfpathlineto{\pgfqpoint{1.197852in}{2.615949in}}%
\pgfpathlineto{\pgfqpoint{1.203916in}{2.600280in}}%
\pgfpathlineto{\pgfqpoint{1.216043in}{2.555014in}}%
\pgfpathlineto{\pgfqpoint{1.222107in}{2.539762in}}%
\pgfpathlineto{\pgfqpoint{1.234235in}{2.494739in}}%
\pgfpathlineto{\pgfqpoint{1.288809in}{2.303378in}}%
\pgfpathlineto{\pgfqpoint{1.307000in}{2.234042in}}%
\pgfpathlineto{\pgfqpoint{1.343383in}{2.095549in}}%
\pgfpathlineto{\pgfqpoint{1.349447in}{2.074894in}}%
\pgfpathlineto{\pgfqpoint{1.373702in}{1.976071in}}%
\pgfpathlineto{\pgfqpoint{1.397957in}{1.882779in}}%
\pgfpathlineto{\pgfqpoint{1.410085in}{1.825374in}}%
\pgfpathlineto{\pgfqpoint{1.416148in}{1.808468in}}%
\pgfpathlineto{\pgfqpoint{1.422212in}{1.777518in}}%
\pgfpathlineto{\pgfqpoint{1.434340in}{1.732863in}}%
\pgfpathlineto{\pgfqpoint{1.446467in}{1.680159in}}%
\pgfpathlineto{\pgfqpoint{1.458595in}{1.634287in}}%
\pgfpathlineto{\pgfqpoint{1.482850in}{1.525808in}}%
\pgfpathlineto{\pgfqpoint{1.494978in}{1.477731in}}%
\pgfpathlineto{\pgfqpoint{1.573807in}{1.128809in}}%
\pgfpathlineto{\pgfqpoint{1.579871in}{1.109786in}}%
\pgfpathlineto{\pgfqpoint{1.585935in}{1.074548in}}%
\pgfpathlineto{\pgfqpoint{1.591999in}{1.054618in}}%
\pgfpathlineto{\pgfqpoint{1.604126in}{0.996690in}}%
\pgfpathlineto{\pgfqpoint{1.610190in}{0.973696in}}%
\pgfpathlineto{\pgfqpoint{1.634445in}{0.859222in}}%
\pgfpathlineto{\pgfqpoint{1.646573in}{0.814059in}}%
\pgfpathlineto{\pgfqpoint{1.652637in}{0.782866in}}%
\pgfpathlineto{\pgfqpoint{1.658700in}{0.763541in}}%
\pgfpathlineto{\pgfqpoint{1.670828in}{0.714676in}}%
\pgfpathlineto{\pgfqpoint{1.682956in}{0.673576in}}%
\pgfpathlineto{\pgfqpoint{1.695083in}{0.651197in}}%
\pgfpathlineto{\pgfqpoint{1.701147in}{0.659736in}}%
\pgfpathlineto{\pgfqpoint{1.707211in}{0.671109in}}%
\pgfpathlineto{\pgfqpoint{1.713275in}{0.695806in}}%
\pgfpathlineto{\pgfqpoint{1.719338in}{0.752097in}}%
\pgfpathlineto{\pgfqpoint{1.725402in}{0.839798in}}%
\pgfpathlineto{\pgfqpoint{1.731466in}{0.955515in}}%
\pgfpathlineto{\pgfqpoint{1.737530in}{1.145475in}}%
\pgfpathlineto{\pgfqpoint{1.755721in}{1.890091in}}%
\pgfpathlineto{\pgfqpoint{1.761785in}{2.033943in}}%
\pgfpathlineto{\pgfqpoint{1.779976in}{2.190930in}}%
\pgfpathlineto{\pgfqpoint{1.786040in}{2.230899in}}%
\pgfpathlineto{\pgfqpoint{1.822423in}{2.559988in}}%
\pgfpathlineto{\pgfqpoint{1.876997in}{2.832699in}}%
\pgfpathlineto{\pgfqpoint{1.895188in}{2.910057in}}%
\pgfpathlineto{\pgfqpoint{1.919444in}{3.008912in}}%
\pgfpathlineto{\pgfqpoint{1.925507in}{3.027770in}}%
\pgfpathlineto{\pgfqpoint{1.931571in}{3.053036in}}%
\pgfpathlineto{\pgfqpoint{1.949763in}{3.103431in}}%
\pgfpathlineto{\pgfqpoint{1.955826in}{3.130524in}}%
\pgfpathlineto{\pgfqpoint{1.961890in}{3.137246in}}%
\pgfpathlineto{\pgfqpoint{1.967954in}{3.159592in}}%
\pgfpathlineto{\pgfqpoint{1.980082in}{3.182483in}}%
\pgfpathlineto{\pgfqpoint{1.986145in}{3.197807in}}%
\pgfpathlineto{\pgfqpoint{1.998273in}{3.220353in}}%
\pgfpathlineto{\pgfqpoint{2.004337in}{3.236083in}}%
\pgfpathlineto{\pgfqpoint{2.010401in}{3.235373in}}%
\pgfpathlineto{\pgfqpoint{2.016464in}{3.250962in}}%
\pgfpathlineto{\pgfqpoint{2.022528in}{3.259062in}}%
\pgfpathlineto{\pgfqpoint{2.028592in}{3.261498in}}%
\pgfpathlineto{\pgfqpoint{2.034656in}{3.272521in}}%
\pgfpathlineto{\pgfqpoint{2.040720in}{3.267252in}}%
\pgfpathlineto{\pgfqpoint{2.046783in}{3.277138in}}%
\pgfpathlineto{\pgfqpoint{2.052847in}{3.281736in}}%
\pgfpathlineto{\pgfqpoint{2.058911in}{3.281407in}}%
\pgfpathlineto{\pgfqpoint{2.064975in}{3.287346in}}%
\pgfpathlineto{\pgfqpoint{2.077102in}{3.285953in}}%
\pgfpathlineto{\pgfqpoint{2.083166in}{3.287536in}}%
\pgfpathlineto{\pgfqpoint{2.089230in}{3.280081in}}%
\pgfpathlineto{\pgfqpoint{2.095294in}{3.285532in}}%
\pgfpathlineto{\pgfqpoint{2.101358in}{3.279059in}}%
\pgfpathlineto{\pgfqpoint{2.119549in}{3.267596in}}%
\pgfpathlineto{\pgfqpoint{2.125613in}{3.259985in}}%
\pgfpathlineto{\pgfqpoint{2.131677in}{3.259724in}}%
\pgfpathlineto{\pgfqpoint{2.137740in}{3.243229in}}%
\pgfpathlineto{\pgfqpoint{2.143804in}{3.245156in}}%
\pgfpathlineto{\pgfqpoint{2.149868in}{3.231817in}}%
\pgfpathlineto{\pgfqpoint{2.155932in}{3.229239in}}%
\pgfpathlineto{\pgfqpoint{2.161996in}{3.209217in}}%
\pgfpathlineto{\pgfqpoint{2.168059in}{3.209873in}}%
\pgfpathlineto{\pgfqpoint{2.174123in}{3.192239in}}%
\pgfpathlineto{\pgfqpoint{2.180187in}{3.189625in}}%
\pgfpathlineto{\pgfqpoint{2.186251in}{3.172741in}}%
\pgfpathlineto{\pgfqpoint{2.192314in}{3.162649in}}%
\pgfpathlineto{\pgfqpoint{2.198378in}{3.156488in}}%
\pgfpathlineto{\pgfqpoint{2.228697in}{3.081743in}}%
\pgfpathlineto{\pgfqpoint{2.234761in}{3.078220in}}%
\pgfpathlineto{\pgfqpoint{2.240825in}{3.050719in}}%
\pgfpathlineto{\pgfqpoint{2.246889in}{3.043825in}}%
\pgfpathlineto{\pgfqpoint{2.252952in}{3.028765in}}%
\pgfpathlineto{\pgfqpoint{2.259016in}{3.007941in}}%
\pgfpathlineto{\pgfqpoint{2.277208in}{2.964309in}}%
\pgfpathlineto{\pgfqpoint{2.295399in}{2.904802in}}%
\pgfpathlineto{\pgfqpoint{2.301463in}{2.895262in}}%
\pgfpathlineto{\pgfqpoint{2.307527in}{2.871817in}}%
\pgfpathlineto{\pgfqpoint{2.319654in}{2.842557in}}%
\pgfpathlineto{\pgfqpoint{2.325718in}{2.814109in}}%
\pgfpathlineto{\pgfqpoint{2.331782in}{2.802876in}}%
\pgfpathlineto{\pgfqpoint{2.343909in}{2.757572in}}%
\pgfpathlineto{\pgfqpoint{2.349973in}{2.743988in}}%
\pgfpathlineto{\pgfqpoint{2.362101in}{2.704713in}}%
\pgfpathlineto{\pgfqpoint{2.368165in}{2.685898in}}%
\pgfpathlineto{\pgfqpoint{2.386356in}{2.619895in}}%
\pgfpathlineto{\pgfqpoint{2.404547in}{2.560925in}}%
\pgfpathlineto{\pgfqpoint{2.410611in}{2.533603in}}%
\pgfpathlineto{\pgfqpoint{2.416675in}{2.522088in}}%
\pgfpathlineto{\pgfqpoint{2.422739in}{2.486524in}}%
\pgfpathlineto{\pgfqpoint{2.428803in}{2.480290in}}%
\pgfpathlineto{\pgfqpoint{2.434866in}{2.444834in}}%
\pgfpathlineto{\pgfqpoint{2.440930in}{2.432691in}}%
\pgfpathlineto{\pgfqpoint{2.446994in}{2.401061in}}%
\pgfpathlineto{\pgfqpoint{2.453058in}{2.391982in}}%
\pgfpathlineto{\pgfqpoint{2.459122in}{2.355740in}}%
\pgfpathlineto{\pgfqpoint{2.465185in}{2.341697in}}%
\pgfpathlineto{\pgfqpoint{2.471249in}{2.313556in}}%
\pgfpathlineto{\pgfqpoint{2.477313in}{2.296772in}}%
\pgfpathlineto{\pgfqpoint{2.483377in}{2.274119in}}%
\pgfpathlineto{\pgfqpoint{2.489441in}{2.240704in}}%
\pgfpathlineto{\pgfqpoint{2.495504in}{2.226495in}}%
\pgfpathlineto{\pgfqpoint{2.525823in}{2.104983in}}%
\pgfpathlineto{\pgfqpoint{2.531887in}{2.088671in}}%
\pgfpathlineto{\pgfqpoint{2.537951in}{2.059236in}}%
\pgfpathlineto{\pgfqpoint{2.544015in}{2.038871in}}%
\pgfpathlineto{\pgfqpoint{2.556142in}{1.984732in}}%
\pgfpathlineto{\pgfqpoint{2.562206in}{1.972295in}}%
\pgfpathlineto{\pgfqpoint{2.568270in}{1.931322in}}%
\pgfpathlineto{\pgfqpoint{2.574334in}{1.924239in}}%
\pgfpathlineto{\pgfqpoint{2.580398in}{1.887270in}}%
\pgfpathlineto{\pgfqpoint{2.586461in}{1.870084in}}%
\pgfpathlineto{\pgfqpoint{2.622844in}{1.715647in}}%
\pgfpathlineto{\pgfqpoint{2.628908in}{1.704987in}}%
\pgfpathlineto{\pgfqpoint{2.634972in}{1.666655in}}%
\pgfpathlineto{\pgfqpoint{2.647099in}{1.630103in}}%
\pgfpathlineto{\pgfqpoint{2.653163in}{1.593781in}}%
\pgfpathlineto{\pgfqpoint{2.659227in}{1.575378in}}%
\pgfpathlineto{\pgfqpoint{2.665291in}{1.545165in}}%
\pgfpathlineto{\pgfqpoint{2.671354in}{1.524296in}}%
\pgfpathlineto{\pgfqpoint{2.677418in}{1.491196in}}%
\pgfpathlineto{\pgfqpoint{2.683482in}{1.481174in}}%
\pgfpathlineto{\pgfqpoint{2.689546in}{1.442039in}}%
\pgfpathlineto{\pgfqpoint{2.695610in}{1.428796in}}%
\pgfpathlineto{\pgfqpoint{2.701673in}{1.391731in}}%
\pgfpathlineto{\pgfqpoint{2.707737in}{1.377429in}}%
\pgfpathlineto{\pgfqpoint{2.713801in}{1.344673in}}%
\pgfpathlineto{\pgfqpoint{2.719865in}{1.324735in}}%
\pgfpathlineto{\pgfqpoint{2.725929in}{1.292631in}}%
\pgfpathlineto{\pgfqpoint{2.731992in}{1.276689in}}%
\pgfpathlineto{\pgfqpoint{2.738056in}{1.243485in}}%
\pgfpathlineto{\pgfqpoint{2.744120in}{1.224819in}}%
\pgfpathlineto{\pgfqpoint{2.750184in}{1.193371in}}%
\pgfpathlineto{\pgfqpoint{2.756248in}{1.173808in}}%
\pgfpathlineto{\pgfqpoint{2.762311in}{1.142871in}}%
\pgfpathlineto{\pgfqpoint{2.768375in}{1.126933in}}%
\pgfpathlineto{\pgfqpoint{2.774439in}{1.082198in}}%
\pgfpathlineto{\pgfqpoint{2.780503in}{1.082862in}}%
\pgfpathlineto{\pgfqpoint{2.786567in}{1.037291in}}%
\pgfpathlineto{\pgfqpoint{2.792630in}{1.019239in}}%
\pgfpathlineto{\pgfqpoint{2.798694in}{1.007241in}}%
\pgfpathlineto{\pgfqpoint{2.804758in}{0.957889in}}%
\pgfpathlineto{\pgfqpoint{2.810822in}{0.964355in}}%
\pgfpathlineto{\pgfqpoint{2.816886in}{0.908014in}}%
\pgfpathlineto{\pgfqpoint{2.822949in}{0.904583in}}%
\pgfpathlineto{\pgfqpoint{2.829013in}{0.873743in}}%
\pgfpathlineto{\pgfqpoint{2.853268in}{0.777893in}}%
\pgfpathlineto{\pgfqpoint{2.859332in}{0.772596in}}%
\pgfpathlineto{\pgfqpoint{2.865396in}{0.730692in}}%
\pgfpathlineto{\pgfqpoint{2.871460in}{0.721221in}}%
\pgfpathlineto{\pgfqpoint{2.883587in}{0.675833in}}%
\pgfpathlineto{\pgfqpoint{2.889651in}{0.670119in}}%
\pgfpathlineto{\pgfqpoint{2.895715in}{0.648924in}}%
\pgfpathlineto{\pgfqpoint{2.901779in}{0.646006in}}%
\pgfpathlineto{\pgfqpoint{2.907843in}{0.637273in}}%
\pgfpathlineto{\pgfqpoint{2.913906in}{0.639619in}}%
\pgfpathlineto{\pgfqpoint{2.919970in}{0.668765in}}%
\pgfpathlineto{\pgfqpoint{2.926034in}{0.682478in}}%
\pgfpathlineto{\pgfqpoint{2.932098in}{0.743147in}}%
\pgfpathlineto{\pgfqpoint{2.938162in}{0.820999in}}%
\pgfpathlineto{\pgfqpoint{2.944225in}{0.973406in}}%
\pgfpathlineto{\pgfqpoint{2.950289in}{1.196030in}}%
\pgfpathlineto{\pgfqpoint{2.962417in}{1.925834in}}%
\pgfpathlineto{\pgfqpoint{2.968481in}{2.155301in}}%
\pgfpathlineto{\pgfqpoint{2.974544in}{2.206267in}}%
\pgfpathlineto{\pgfqpoint{2.986672in}{2.382148in}}%
\pgfpathlineto{\pgfqpoint{2.992736in}{2.460479in}}%
\pgfpathlineto{\pgfqpoint{3.083693in}{2.995458in}}%
\pgfpathlineto{\pgfqpoint{3.095820in}{3.060147in}}%
\pgfpathlineto{\pgfqpoint{3.101884in}{3.078167in}}%
\pgfpathlineto{\pgfqpoint{3.107948in}{3.118624in}}%
\pgfpathlineto{\pgfqpoint{3.114012in}{3.130561in}}%
\pgfpathlineto{\pgfqpoint{3.126139in}{3.182913in}}%
\pgfpathlineto{\pgfqpoint{3.132203in}{3.198538in}}%
\pgfpathlineto{\pgfqpoint{3.144331in}{3.243319in}}%
\pgfpathlineto{\pgfqpoint{3.150394in}{3.249348in}}%
\pgfpathlineto{\pgfqpoint{3.156458in}{3.279292in}}%
\pgfpathlineto{\pgfqpoint{3.162522in}{3.290947in}}%
\pgfpathlineto{\pgfqpoint{3.168586in}{3.288956in}}%
\pgfpathlineto{\pgfqpoint{3.174650in}{3.325940in}}%
\pgfpathlineto{\pgfqpoint{3.180713in}{3.318109in}}%
\pgfpathlineto{\pgfqpoint{3.192841in}{3.350284in}}%
\pgfpathlineto{\pgfqpoint{3.198905in}{3.340868in}}%
\pgfpathlineto{\pgfqpoint{3.204969in}{3.367731in}}%
\pgfpathlineto{\pgfqpoint{3.211032in}{3.358819in}}%
\pgfpathlineto{\pgfqpoint{3.223160in}{3.376312in}}%
\pgfpathlineto{\pgfqpoint{3.229224in}{3.369855in}}%
\pgfpathlineto{\pgfqpoint{3.235288in}{3.382727in}}%
\pgfpathlineto{\pgfqpoint{3.247415in}{3.376308in}}%
\pgfpathlineto{\pgfqpoint{3.253479in}{3.382113in}}%
\pgfpathlineto{\pgfqpoint{3.259543in}{3.373574in}}%
\pgfpathlineto{\pgfqpoint{3.265607in}{3.380979in}}%
\pgfpathlineto{\pgfqpoint{3.271670in}{3.368902in}}%
\pgfpathlineto{\pgfqpoint{3.277734in}{3.365833in}}%
\pgfpathlineto{\pgfqpoint{3.283798in}{3.368797in}}%
\pgfpathlineto{\pgfqpoint{3.289862in}{3.352434in}}%
\pgfpathlineto{\pgfqpoint{3.295926in}{3.359961in}}%
\pgfpathlineto{\pgfqpoint{3.301989in}{3.340090in}}%
\pgfpathlineto{\pgfqpoint{3.308053in}{3.338271in}}%
\pgfpathlineto{\pgfqpoint{3.314117in}{3.333903in}}%
\pgfpathlineto{\pgfqpoint{3.320181in}{3.319481in}}%
\pgfpathlineto{\pgfqpoint{3.326245in}{3.313950in}}%
\pgfpathlineto{\pgfqpoint{3.332308in}{3.300925in}}%
\pgfpathlineto{\pgfqpoint{3.344436in}{3.282884in}}%
\pgfpathlineto{\pgfqpoint{3.350500in}{3.267062in}}%
\pgfpathlineto{\pgfqpoint{3.356564in}{3.260828in}}%
\pgfpathlineto{\pgfqpoint{3.362627in}{3.243185in}}%
\pgfpathlineto{\pgfqpoint{3.368691in}{3.237072in}}%
\pgfpathlineto{\pgfqpoint{3.374755in}{3.211560in}}%
\pgfpathlineto{\pgfqpoint{3.380819in}{3.216957in}}%
\pgfpathlineto{\pgfqpoint{3.386883in}{3.187678in}}%
\pgfpathlineto{\pgfqpoint{3.392946in}{3.178921in}}%
\pgfpathlineto{\pgfqpoint{3.399010in}{3.160471in}}%
\pgfpathlineto{\pgfqpoint{3.405074in}{3.152241in}}%
\pgfpathlineto{\pgfqpoint{3.411138in}{3.124889in}}%
\pgfpathlineto{\pgfqpoint{3.417202in}{3.126033in}}%
\pgfpathlineto{\pgfqpoint{3.423265in}{3.091398in}}%
\pgfpathlineto{\pgfqpoint{3.429329in}{3.092418in}}%
\pgfpathlineto{\pgfqpoint{3.435393in}{3.059865in}}%
\pgfpathlineto{\pgfqpoint{3.441457in}{3.057819in}}%
\pgfpathlineto{\pgfqpoint{3.453584in}{3.009182in}}%
\pgfpathlineto{\pgfqpoint{3.459648in}{3.008755in}}%
\pgfpathlineto{\pgfqpoint{3.465712in}{2.969967in}}%
\pgfpathlineto{\pgfqpoint{3.471776in}{2.970403in}}%
\pgfpathlineto{\pgfqpoint{3.477840in}{2.931626in}}%
\pgfpathlineto{\pgfqpoint{3.483903in}{2.939598in}}%
\pgfpathlineto{\pgfqpoint{3.489967in}{2.888302in}}%
\pgfpathlineto{\pgfqpoint{3.496031in}{2.906239in}}%
\pgfpathlineto{\pgfqpoint{3.502095in}{2.850712in}}%
\pgfpathlineto{\pgfqpoint{3.508159in}{2.862490in}}%
\pgfpathlineto{\pgfqpoint{3.514222in}{2.814364in}}%
\pgfpathlineto{\pgfqpoint{3.520286in}{2.818410in}}%
\pgfpathlineto{\pgfqpoint{3.526350in}{2.780723in}}%
\pgfpathlineto{\pgfqpoint{3.532414in}{2.770786in}}%
\pgfpathlineto{\pgfqpoint{3.550605in}{2.705166in}}%
\pgfpathlineto{\pgfqpoint{3.556669in}{2.690531in}}%
\pgfpathlineto{\pgfqpoint{3.568796in}{2.642117in}}%
\pgfpathlineto{\pgfqpoint{3.574860in}{2.620871in}}%
\pgfpathlineto{\pgfqpoint{3.580924in}{2.610150in}}%
\pgfpathlineto{\pgfqpoint{3.586988in}{2.566390in}}%
\pgfpathlineto{\pgfqpoint{3.593052in}{2.581926in}}%
\pgfpathlineto{\pgfqpoint{3.599115in}{2.518323in}}%
\pgfpathlineto{\pgfqpoint{3.605179in}{2.542199in}}%
\pgfpathlineto{\pgfqpoint{3.611243in}{2.473377in}}%
\pgfpathlineto{\pgfqpoint{3.617307in}{2.494749in}}%
\pgfpathlineto{\pgfqpoint{3.623371in}{2.441377in}}%
\pgfpathlineto{\pgfqpoint{3.629434in}{2.439282in}}%
\pgfpathlineto{\pgfqpoint{3.641562in}{2.380743in}}%
\pgfpathlineto{\pgfqpoint{3.647626in}{2.371267in}}%
\pgfpathlineto{\pgfqpoint{3.665817in}{2.297941in}}%
\pgfpathlineto{\pgfqpoint{3.671881in}{2.297041in}}%
\pgfpathlineto{\pgfqpoint{3.677945in}{2.239446in}}%
\pgfpathlineto{\pgfqpoint{3.684009in}{2.259866in}}%
\pgfpathlineto{\pgfqpoint{3.690072in}{2.200497in}}%
\pgfpathlineto{\pgfqpoint{3.696136in}{2.205181in}}%
\pgfpathlineto{\pgfqpoint{3.702200in}{2.170725in}}%
\pgfpathlineto{\pgfqpoint{3.714328in}{2.138450in}}%
\pgfpathlineto{\pgfqpoint{3.720391in}{2.101209in}}%
\pgfpathlineto{\pgfqpoint{3.726455in}{2.099456in}}%
\pgfpathlineto{\pgfqpoint{3.732519in}{2.066403in}}%
\pgfpathlineto{\pgfqpoint{3.738583in}{2.043729in}}%
\pgfpathlineto{\pgfqpoint{3.744647in}{2.048569in}}%
\pgfpathlineto{\pgfqpoint{3.750710in}{1.995495in}}%
\pgfpathlineto{\pgfqpoint{3.756774in}{1.993517in}}%
\pgfpathlineto{\pgfqpoint{3.762838in}{1.993565in}}%
\pgfpathlineto{\pgfqpoint{3.768902in}{1.925705in}}%
\pgfpathlineto{\pgfqpoint{3.774966in}{1.971135in}}%
\pgfpathlineto{\pgfqpoint{3.781029in}{1.900179in}}%
\pgfpathlineto{\pgfqpoint{3.787093in}{1.905280in}}%
\pgfpathlineto{\pgfqpoint{3.793157in}{1.888995in}}%
\pgfpathlineto{\pgfqpoint{3.799221in}{1.856938in}}%
\pgfpathlineto{\pgfqpoint{3.805285in}{1.865818in}}%
\pgfpathlineto{\pgfqpoint{3.811348in}{1.815389in}}%
\pgfpathlineto{\pgfqpoint{3.817412in}{1.840028in}}%
\pgfpathlineto{\pgfqpoint{3.823476in}{1.783140in}}%
\pgfpathlineto{\pgfqpoint{3.829540in}{1.807055in}}%
\pgfpathlineto{\pgfqpoint{3.835604in}{1.755076in}}%
\pgfpathlineto{\pgfqpoint{3.841667in}{1.760870in}}%
\pgfpathlineto{\pgfqpoint{3.847731in}{1.746040in}}%
\pgfpathlineto{\pgfqpoint{3.853795in}{1.726078in}}%
\pgfpathlineto{\pgfqpoint{3.859859in}{1.719422in}}%
\pgfpathlineto{\pgfqpoint{3.865923in}{1.704364in}}%
\pgfpathlineto{\pgfqpoint{3.871986in}{1.698371in}}%
\pgfpathlineto{\pgfqpoint{3.878050in}{1.670064in}}%
\pgfpathlineto{\pgfqpoint{3.884114in}{1.682106in}}%
\pgfpathlineto{\pgfqpoint{3.890178in}{1.646457in}}%
\pgfpathlineto{\pgfqpoint{3.896242in}{1.652474in}}%
\pgfpathlineto{\pgfqpoint{3.902305in}{1.632682in}}%
\pgfpathlineto{\pgfqpoint{3.908369in}{1.632100in}}%
\pgfpathlineto{\pgfqpoint{3.914433in}{1.619297in}}%
\pgfpathlineto{\pgfqpoint{3.920497in}{1.618996in}}%
\pgfpathlineto{\pgfqpoint{3.926561in}{1.601552in}}%
\pgfpathlineto{\pgfqpoint{3.932624in}{1.611770in}}%
\pgfpathlineto{\pgfqpoint{3.938688in}{1.578663in}}%
\pgfpathlineto{\pgfqpoint{3.944752in}{1.591296in}}%
\pgfpathlineto{\pgfqpoint{3.950816in}{1.583831in}}%
\pgfpathlineto{\pgfqpoint{3.956880in}{1.569048in}}%
\pgfpathlineto{\pgfqpoint{3.962943in}{1.587618in}}%
\pgfpathlineto{\pgfqpoint{3.969007in}{1.560179in}}%
\pgfpathlineto{\pgfqpoint{3.975071in}{1.577382in}}%
\pgfpathlineto{\pgfqpoint{3.981135in}{1.562772in}}%
\pgfpathlineto{\pgfqpoint{3.987198in}{1.567050in}}%
\pgfpathlineto{\pgfqpoint{3.993262in}{1.567126in}}%
\pgfpathlineto{\pgfqpoint{3.999326in}{1.572853in}}%
\pgfpathlineto{\pgfqpoint{4.005390in}{1.561010in}}%
\pgfpathlineto{\pgfqpoint{4.011454in}{1.596471in}}%
\pgfpathlineto{\pgfqpoint{4.017517in}{1.543004in}}%
\pgfpathlineto{\pgfqpoint{4.023581in}{1.605138in}}%
\pgfpathlineto{\pgfqpoint{4.029645in}{1.562172in}}%
\pgfpathlineto{\pgfqpoint{4.035709in}{1.611493in}}%
\pgfpathlineto{\pgfqpoint{4.041773in}{1.594445in}}%
\pgfpathlineto{\pgfqpoint{4.047836in}{1.604891in}}%
\pgfpathlineto{\pgfqpoint{4.053900in}{1.622590in}}%
\pgfpathlineto{\pgfqpoint{4.059964in}{1.627728in}}%
\pgfpathlineto{\pgfqpoint{4.066028in}{1.630629in}}%
\pgfpathlineto{\pgfqpoint{4.072092in}{1.640463in}}%
\pgfpathlineto{\pgfqpoint{4.078155in}{1.664120in}}%
\pgfpathlineto{\pgfqpoint{4.084219in}{1.662028in}}%
\pgfpathlineto{\pgfqpoint{4.090283in}{1.693412in}}%
\pgfpathlineto{\pgfqpoint{4.096347in}{1.683120in}}%
\pgfpathlineto{\pgfqpoint{4.102411in}{1.708237in}}%
\pgfpathlineto{\pgfqpoint{4.108474in}{1.725012in}}%
\pgfpathlineto{\pgfqpoint{4.120602in}{1.750450in}}%
\pgfpathlineto{\pgfqpoint{4.126666in}{1.769631in}}%
\pgfpathlineto{\pgfqpoint{4.132730in}{1.768987in}}%
\pgfpathlineto{\pgfqpoint{4.138793in}{1.808527in}}%
\pgfpathlineto{\pgfqpoint{4.144857in}{1.804464in}}%
\pgfpathlineto{\pgfqpoint{4.150921in}{1.835894in}}%
\pgfpathlineto{\pgfqpoint{4.156985in}{1.838385in}}%
\pgfpathlineto{\pgfqpoint{4.163049in}{1.864172in}}%
\pgfpathlineto{\pgfqpoint{4.175176in}{1.887452in}}%
\pgfpathlineto{\pgfqpoint{4.181240in}{1.913376in}}%
\pgfpathlineto{\pgfqpoint{4.187304in}{1.919401in}}%
\pgfpathlineto{\pgfqpoint{4.193368in}{1.946335in}}%
\pgfpathlineto{\pgfqpoint{4.199431in}{1.961002in}}%
\pgfpathlineto{\pgfqpoint{4.205495in}{1.967788in}}%
\pgfpathlineto{\pgfqpoint{4.211559in}{1.999271in}}%
\pgfpathlineto{\pgfqpoint{4.217623in}{1.997253in}}%
\pgfpathlineto{\pgfqpoint{4.223687in}{2.030052in}}%
\pgfpathlineto{\pgfqpoint{4.229750in}{2.024307in}}%
\pgfpathlineto{\pgfqpoint{4.235814in}{2.052443in}}%
\pgfpathlineto{\pgfqpoint{4.241878in}{2.056927in}}%
\pgfpathlineto{\pgfqpoint{4.247942in}{2.079285in}}%
\pgfpathlineto{\pgfqpoint{4.254006in}{2.083067in}}%
\pgfpathlineto{\pgfqpoint{4.260069in}{2.106899in}}%
\pgfpathlineto{\pgfqpoint{4.266133in}{2.104995in}}%
\pgfpathlineto{\pgfqpoint{4.272197in}{2.125864in}}%
\pgfpathlineto{\pgfqpoint{4.284325in}{2.139018in}}%
\pgfpathlineto{\pgfqpoint{4.290388in}{2.155243in}}%
\pgfpathlineto{\pgfqpoint{4.296452in}{2.155636in}}%
\pgfpathlineto{\pgfqpoint{4.302516in}{2.172246in}}%
\pgfpathlineto{\pgfqpoint{4.308580in}{2.179988in}}%
\pgfpathlineto{\pgfqpoint{4.314644in}{2.181328in}}%
\pgfpathlineto{\pgfqpoint{4.320707in}{2.199309in}}%
\pgfpathlineto{\pgfqpoint{4.326771in}{2.199908in}}%
\pgfpathlineto{\pgfqpoint{4.338899in}{2.217075in}}%
\pgfpathlineto{\pgfqpoint{4.344963in}{2.220924in}}%
\pgfpathlineto{\pgfqpoint{4.351026in}{2.227082in}}%
\pgfpathlineto{\pgfqpoint{4.357090in}{2.237229in}}%
\pgfpathlineto{\pgfqpoint{4.363154in}{2.234181in}}%
\pgfpathlineto{\pgfqpoint{4.369218in}{2.251555in}}%
\pgfpathlineto{\pgfqpoint{4.375282in}{2.243027in}}%
\pgfpathlineto{\pgfqpoint{4.381345in}{2.254822in}}%
\pgfpathlineto{\pgfqpoint{4.387409in}{2.255465in}}%
\pgfpathlineto{\pgfqpoint{4.393473in}{2.258763in}}%
\pgfpathlineto{\pgfqpoint{4.399537in}{2.264708in}}%
\pgfpathlineto{\pgfqpoint{4.405601in}{2.265384in}}%
\pgfpathlineto{\pgfqpoint{4.411664in}{2.272259in}}%
\pgfpathlineto{\pgfqpoint{4.417728in}{2.270484in}}%
\pgfpathlineto{\pgfqpoint{4.423792in}{2.277866in}}%
\pgfpathlineto{\pgfqpoint{4.429856in}{2.275693in}}%
\pgfpathlineto{\pgfqpoint{4.435919in}{2.281115in}}%
\pgfpathlineto{\pgfqpoint{4.441983in}{2.280756in}}%
\pgfpathlineto{\pgfqpoint{4.448047in}{2.286578in}}%
\pgfpathlineto{\pgfqpoint{4.454111in}{2.285083in}}%
\pgfpathlineto{\pgfqpoint{4.460175in}{2.289505in}}%
\pgfpathlineto{\pgfqpoint{4.466238in}{2.287238in}}%
\pgfpathlineto{\pgfqpoint{4.472302in}{2.292202in}}%
\pgfpathlineto{\pgfqpoint{4.478366in}{2.291813in}}%
\pgfpathlineto{\pgfqpoint{4.484430in}{2.293859in}}%
\pgfpathlineto{\pgfqpoint{4.490494in}{2.293623in}}%
\pgfpathlineto{\pgfqpoint{4.496557in}{2.296731in}}%
\pgfpathlineto{\pgfqpoint{4.502621in}{2.296186in}}%
\pgfpathlineto{\pgfqpoint{4.520813in}{2.299633in}}%
\pgfpathlineto{\pgfqpoint{4.545068in}{2.300838in}}%
\pgfpathlineto{\pgfqpoint{4.551132in}{2.303020in}}%
\pgfpathlineto{\pgfqpoint{4.557195in}{2.302283in}}%
\pgfpathlineto{\pgfqpoint{4.569323in}{2.303780in}}%
\pgfpathlineto{\pgfqpoint{4.593578in}{2.305658in}}%
\pgfpathlineto{\pgfqpoint{4.593578in}{2.305658in}}%
\pgfusepath{stroke}%
\end{pgfscope}%
\begin{pgfscope}%
\pgfsetrectcap%
\pgfsetmiterjoin%
\pgfsetlinewidth{0.803000pt}%
\definecolor{currentstroke}{rgb}{0.000000,0.000000,0.000000}%
\pgfsetstrokecolor{currentstroke}%
\pgfsetdash{}{0pt}%
\pgfpathmoveto{\pgfqpoint{0.750000in}{0.500000in}}%
\pgfpathlineto{\pgfqpoint{0.750000in}{3.520000in}}%
\pgfusepath{stroke}%
\end{pgfscope}%
\begin{pgfscope}%
\pgfsetrectcap%
\pgfsetmiterjoin%
\pgfsetlinewidth{0.803000pt}%
\definecolor{currentstroke}{rgb}{0.000000,0.000000,0.000000}%
\pgfsetstrokecolor{currentstroke}%
\pgfsetdash{}{0pt}%
\pgfpathmoveto{\pgfqpoint{5.400000in}{0.500000in}}%
\pgfpathlineto{\pgfqpoint{5.400000in}{3.520000in}}%
\pgfusepath{stroke}%
\end{pgfscope}%
\begin{pgfscope}%
\pgfsetrectcap%
\pgfsetmiterjoin%
\pgfsetlinewidth{0.803000pt}%
\definecolor{currentstroke}{rgb}{0.000000,0.000000,0.000000}%
\pgfsetstrokecolor{currentstroke}%
\pgfsetdash{}{0pt}%
\pgfpathmoveto{\pgfqpoint{0.750000in}{0.500000in}}%
\pgfpathlineto{\pgfqpoint{5.400000in}{0.500000in}}%
\pgfusepath{stroke}%
\end{pgfscope}%
\begin{pgfscope}%
\pgfsetrectcap%
\pgfsetmiterjoin%
\pgfsetlinewidth{0.803000pt}%
\definecolor{currentstroke}{rgb}{0.000000,0.000000,0.000000}%
\pgfsetstrokecolor{currentstroke}%
\pgfsetdash{}{0pt}%
\pgfpathmoveto{\pgfqpoint{0.750000in}{3.520000in}}%
\pgfpathlineto{\pgfqpoint{5.400000in}{3.520000in}}%
\pgfusepath{stroke}%
\end{pgfscope}%
\begin{pgfscope}%
\pgfsetbuttcap%
\pgfsetmiterjoin%
\definecolor{currentfill}{rgb}{1.000000,1.000000,1.000000}%
\pgfsetfillcolor{currentfill}%
\pgfsetfillopacity{0.800000}%
\pgfsetlinewidth{1.003750pt}%
\definecolor{currentstroke}{rgb}{0.800000,0.800000,0.800000}%
\pgfsetstrokecolor{currentstroke}%
\pgfsetstrokeopacity{0.800000}%
\pgfsetdash{}{0pt}%
\pgfpathmoveto{\pgfqpoint{4.331713in}{2.827871in}}%
\pgfpathlineto{\pgfqpoint{5.302778in}{2.827871in}}%
\pgfpathquadraticcurveto{\pgfqpoint{5.330556in}{2.827871in}}{\pgfqpoint{5.330556in}{2.855648in}}%
\pgfpathlineto{\pgfqpoint{5.330556in}{3.422778in}}%
\pgfpathquadraticcurveto{\pgfqpoint{5.330556in}{3.450556in}}{\pgfqpoint{5.302778in}{3.450556in}}%
\pgfpathlineto{\pgfqpoint{4.331713in}{3.450556in}}%
\pgfpathquadraticcurveto{\pgfqpoint{4.303935in}{3.450556in}}{\pgfqpoint{4.303935in}{3.422778in}}%
\pgfpathlineto{\pgfqpoint{4.303935in}{2.855648in}}%
\pgfpathquadraticcurveto{\pgfqpoint{4.303935in}{2.827871in}}{\pgfqpoint{4.331713in}{2.827871in}}%
\pgfpathlineto{\pgfqpoint{4.331713in}{2.827871in}}%
\pgfpathclose%
\pgfusepath{stroke,fill}%
\end{pgfscope}%
\begin{pgfscope}%
\pgfsetrectcap%
\pgfsetroundjoin%
\pgfsetlinewidth{1.505625pt}%
\definecolor{currentstroke}{rgb}{0.121569,0.466667,0.705882}%
\pgfsetstrokecolor{currentstroke}%
\pgfsetdash{}{0pt}%
\pgfpathmoveto{\pgfqpoint{4.359490in}{3.346389in}}%
\pgfpathlineto{\pgfqpoint{4.498379in}{3.346389in}}%
\pgfpathlineto{\pgfqpoint{4.637268in}{3.346389in}}%
\pgfusepath{stroke}%
\end{pgfscope}%
\begin{pgfscope}%
\definecolor{textcolor}{rgb}{0.000000,0.000000,0.000000}%
\pgfsetstrokecolor{textcolor}%
\pgfsetfillcolor{textcolor}%
\pgftext[x=4.748379in,y=3.297778in,left,base]{\color{textcolor}\sffamily\fontsize{10.000000}{12.000000}\selectfont \(\displaystyle 250 \, \mathrm{MeV}\)}%
\end{pgfscope}%
\begin{pgfscope}%
\pgfsetrectcap%
\pgfsetroundjoin%
\pgfsetlinewidth{1.505625pt}%
\definecolor{currentstroke}{rgb}{1.000000,0.498039,0.054902}%
\pgfsetstrokecolor{currentstroke}%
\pgfsetdash{}{0pt}%
\pgfpathmoveto{\pgfqpoint{4.359490in}{3.152716in}}%
\pgfpathlineto{\pgfqpoint{4.498379in}{3.152716in}}%
\pgfpathlineto{\pgfqpoint{4.637268in}{3.152716in}}%
\pgfusepath{stroke}%
\end{pgfscope}%
\begin{pgfscope}%
\definecolor{textcolor}{rgb}{0.000000,0.000000,0.000000}%
\pgfsetstrokecolor{textcolor}%
\pgfsetfillcolor{textcolor}%
\pgftext[x=4.748379in,y=3.104105in,left,base]{\color{textcolor}\sffamily\fontsize{10.000000}{12.000000}\selectfont \(\displaystyle 300 \, \mathrm{MeV}\)}%
\end{pgfscope}%
\begin{pgfscope}%
\pgfsetrectcap%
\pgfsetroundjoin%
\pgfsetlinewidth{1.505625pt}%
\definecolor{currentstroke}{rgb}{0.172549,0.627451,0.172549}%
\pgfsetstrokecolor{currentstroke}%
\pgfsetdash{}{0pt}%
\pgfpathmoveto{\pgfqpoint{4.359490in}{2.959043in}}%
\pgfpathlineto{\pgfqpoint{4.498379in}{2.959043in}}%
\pgfpathlineto{\pgfqpoint{4.637268in}{2.959043in}}%
\pgfusepath{stroke}%
\end{pgfscope}%
\begin{pgfscope}%
\definecolor{textcolor}{rgb}{0.000000,0.000000,0.000000}%
\pgfsetstrokecolor{textcolor}%
\pgfsetfillcolor{textcolor}%
\pgftext[x=4.748379in,y=2.910432in,left,base]{\color{textcolor}\sffamily\fontsize{10.000000}{12.000000}\selectfont \(\displaystyle 350 \, \mathrm{MeV}\)}%
\end{pgfscope}%
\end{pgfpicture}%
\makeatother%
\endgroup%

	\caption{Witness energy gain for three different initial kinetic energies of the driver.}
	\label{fig:gain_E}
\end{figure}

While the jump from \qtyrange{250}{300}{\MeV} results in \qty{63}{\MeV} higher gains, there is a diminishing return, as only \qty{16}{\MeV} more are reached when increasing from \qtyrange{300}{350}{\MeV}. Real experiments are restricted by the length of the plasma jet.
This results in an even smaller witness energy increase, as the main difference between the three \gls{pwfa} runs is the longer duration of the blowout regime for higher energy drivers. Therefore, there may be no need for higher energy drivers, as the advantages vanish.
The values make also apparent, that there are only small differences in the energy loss behind the driver (minimums at \qty{-3}{\um} in \autoref{fig:gain_E}). The driver with higher energy lives only longer because it has more energy to lose, not because it loses less per time.

\paragraph*{Divergence comparison}\label{para:div_comp}\hspace{0pt} \\
Next, the driver qualities for three different divergences after passing the metal foil were compared. The energy gains over $\zeta$ for a high divergence ($\sigma_{\theta}=$ \qty{8.7}{\mrad}), middle divergence ($\sigma_{\theta}=$ \qty{4.2}{\mrad}, as seen in experiments) 
and low divergence ($\sigma_{\theta}=$ \qty{1.7}{\mrad}) beam are shown in \autoref{fig:gain_div}.
\begin{figure}
	\centering
	%% Creator: Matplotlib, PGF backend
%%
%% To include the figure in your LaTeX document, write
%%   \input{<filename>.pgf}
%%
%% Make sure the required packages are loaded in your preamble
%%   \usepackage{pgf}
%%
%% Also ensure that all the required font packages are loaded; for instance,
%% the lmodern package is sometimes necessary when using math font.
%%   \usepackage{lmodern}
%%
%% Figures using additional raster images can only be included by \input if
%% they are in the same directory as the main LaTeX file. For loading figures
%% from other directories you can use the `import` package
%%   \usepackage{import}
%%
%% and then include the figures with
%%   \import{<path to file>}{<filename>.pgf}
%%
%% Matplotlib used the following preamble
%%
\begingroup%
\makeatletter%
\begin{pgfpicture}%
\pgfpathrectangle{\pgfpointorigin}{\pgfqpoint{6.000000in}{4.000000in}}%
\pgfusepath{use as bounding box, clip}%
\begin{pgfscope}%
\pgfsetbuttcap%
\pgfsetmiterjoin%
\pgfsetlinewidth{0.000000pt}%
\definecolor{currentstroke}{rgb}{1.000000,1.000000,1.000000}%
\pgfsetstrokecolor{currentstroke}%
\pgfsetstrokeopacity{0.000000}%
\pgfsetdash{}{0pt}%
\pgfpathmoveto{\pgfqpoint{0.000000in}{0.000000in}}%
\pgfpathlineto{\pgfqpoint{6.000000in}{0.000000in}}%
\pgfpathlineto{\pgfqpoint{6.000000in}{4.000000in}}%
\pgfpathlineto{\pgfqpoint{0.000000in}{4.000000in}}%
\pgfpathlineto{\pgfqpoint{0.000000in}{0.000000in}}%
\pgfpathclose%
\pgfusepath{}%
\end{pgfscope}%
\begin{pgfscope}%
\pgfsetbuttcap%
\pgfsetmiterjoin%
\definecolor{currentfill}{rgb}{1.000000,1.000000,1.000000}%
\pgfsetfillcolor{currentfill}%
\pgfsetlinewidth{0.000000pt}%
\definecolor{currentstroke}{rgb}{0.000000,0.000000,0.000000}%
\pgfsetstrokecolor{currentstroke}%
\pgfsetstrokeopacity{0.000000}%
\pgfsetdash{}{0pt}%
\pgfpathmoveto{\pgfqpoint{0.750000in}{0.500000in}}%
\pgfpathlineto{\pgfqpoint{5.400000in}{0.500000in}}%
\pgfpathlineto{\pgfqpoint{5.400000in}{3.520000in}}%
\pgfpathlineto{\pgfqpoint{0.750000in}{3.520000in}}%
\pgfpathlineto{\pgfqpoint{0.750000in}{0.500000in}}%
\pgfpathclose%
\pgfusepath{fill}%
\end{pgfscope}%
\begin{pgfscope}%
\pgfpathrectangle{\pgfqpoint{0.750000in}{0.500000in}}{\pgfqpoint{4.650000in}{3.020000in}}%
\pgfusepath{clip}%
\pgfsetrectcap%
\pgfsetroundjoin%
\pgfsetlinewidth{0.803000pt}%
\definecolor{currentstroke}{rgb}{0.690196,0.690196,0.690196}%
\pgfsetstrokecolor{currentstroke}%
\pgfsetdash{}{0pt}%
\pgfpathmoveto{\pgfqpoint{0.856876in}{0.500000in}}%
\pgfpathlineto{\pgfqpoint{0.856876in}{3.520000in}}%
\pgfusepath{stroke}%
\end{pgfscope}%
\begin{pgfscope}%
\pgfsetbuttcap%
\pgfsetroundjoin%
\definecolor{currentfill}{rgb}{0.000000,0.000000,0.000000}%
\pgfsetfillcolor{currentfill}%
\pgfsetlinewidth{0.803000pt}%
\definecolor{currentstroke}{rgb}{0.000000,0.000000,0.000000}%
\pgfsetstrokecolor{currentstroke}%
\pgfsetdash{}{0pt}%
\pgfsys@defobject{currentmarker}{\pgfqpoint{0.000000in}{-0.048611in}}{\pgfqpoint{0.000000in}{0.000000in}}{%
\pgfpathmoveto{\pgfqpoint{0.000000in}{0.000000in}}%
\pgfpathlineto{\pgfqpoint{0.000000in}{-0.048611in}}%
\pgfusepath{stroke,fill}%
}%
\begin{pgfscope}%
\pgfsys@transformshift{0.856876in}{0.500000in}%
\pgfsys@useobject{currentmarker}{}%
\end{pgfscope}%
\end{pgfscope}%
\begin{pgfscope}%
\definecolor{textcolor}{rgb}{0.000000,0.000000,0.000000}%
\pgfsetstrokecolor{textcolor}%
\pgfsetfillcolor{textcolor}%
\pgftext[x=0.856876in,y=0.402778in,,top]{\color{textcolor}\sffamily\fontsize{10.000000}{12.000000}\selectfont \(\displaystyle {\ensuremath{-}40}\)}%
\end{pgfscope}%
\begin{pgfscope}%
\pgfpathrectangle{\pgfqpoint{0.750000in}{0.500000in}}{\pgfqpoint{4.650000in}{3.020000in}}%
\pgfusepath{clip}%
\pgfsetrectcap%
\pgfsetroundjoin%
\pgfsetlinewidth{0.803000pt}%
\definecolor{currentstroke}{rgb}{0.690196,0.690196,0.690196}%
\pgfsetstrokecolor{currentstroke}%
\pgfsetdash{}{0pt}%
\pgfpathmoveto{\pgfqpoint{1.623979in}{0.500000in}}%
\pgfpathlineto{\pgfqpoint{1.623979in}{3.520000in}}%
\pgfusepath{stroke}%
\end{pgfscope}%
\begin{pgfscope}%
\pgfsetbuttcap%
\pgfsetroundjoin%
\definecolor{currentfill}{rgb}{0.000000,0.000000,0.000000}%
\pgfsetfillcolor{currentfill}%
\pgfsetlinewidth{0.803000pt}%
\definecolor{currentstroke}{rgb}{0.000000,0.000000,0.000000}%
\pgfsetstrokecolor{currentstroke}%
\pgfsetdash{}{0pt}%
\pgfsys@defobject{currentmarker}{\pgfqpoint{0.000000in}{-0.048611in}}{\pgfqpoint{0.000000in}{0.000000in}}{%
\pgfpathmoveto{\pgfqpoint{0.000000in}{0.000000in}}%
\pgfpathlineto{\pgfqpoint{0.000000in}{-0.048611in}}%
\pgfusepath{stroke,fill}%
}%
\begin{pgfscope}%
\pgfsys@transformshift{1.623979in}{0.500000in}%
\pgfsys@useobject{currentmarker}{}%
\end{pgfscope}%
\end{pgfscope}%
\begin{pgfscope}%
\definecolor{textcolor}{rgb}{0.000000,0.000000,0.000000}%
\pgfsetstrokecolor{textcolor}%
\pgfsetfillcolor{textcolor}%
\pgftext[x=1.623979in,y=0.402778in,,top]{\color{textcolor}\sffamily\fontsize{10.000000}{12.000000}\selectfont \(\displaystyle {\ensuremath{-}30}\)}%
\end{pgfscope}%
\begin{pgfscope}%
\pgfpathrectangle{\pgfqpoint{0.750000in}{0.500000in}}{\pgfqpoint{4.650000in}{3.020000in}}%
\pgfusepath{clip}%
\pgfsetrectcap%
\pgfsetroundjoin%
\pgfsetlinewidth{0.803000pt}%
\definecolor{currentstroke}{rgb}{0.690196,0.690196,0.690196}%
\pgfsetstrokecolor{currentstroke}%
\pgfsetdash{}{0pt}%
\pgfpathmoveto{\pgfqpoint{2.391082in}{0.500000in}}%
\pgfpathlineto{\pgfqpoint{2.391082in}{3.520000in}}%
\pgfusepath{stroke}%
\end{pgfscope}%
\begin{pgfscope}%
\pgfsetbuttcap%
\pgfsetroundjoin%
\definecolor{currentfill}{rgb}{0.000000,0.000000,0.000000}%
\pgfsetfillcolor{currentfill}%
\pgfsetlinewidth{0.803000pt}%
\definecolor{currentstroke}{rgb}{0.000000,0.000000,0.000000}%
\pgfsetstrokecolor{currentstroke}%
\pgfsetdash{}{0pt}%
\pgfsys@defobject{currentmarker}{\pgfqpoint{0.000000in}{-0.048611in}}{\pgfqpoint{0.000000in}{0.000000in}}{%
\pgfpathmoveto{\pgfqpoint{0.000000in}{0.000000in}}%
\pgfpathlineto{\pgfqpoint{0.000000in}{-0.048611in}}%
\pgfusepath{stroke,fill}%
}%
\begin{pgfscope}%
\pgfsys@transformshift{2.391082in}{0.500000in}%
\pgfsys@useobject{currentmarker}{}%
\end{pgfscope}%
\end{pgfscope}%
\begin{pgfscope}%
\definecolor{textcolor}{rgb}{0.000000,0.000000,0.000000}%
\pgfsetstrokecolor{textcolor}%
\pgfsetfillcolor{textcolor}%
\pgftext[x=2.391082in,y=0.402778in,,top]{\color{textcolor}\sffamily\fontsize{10.000000}{12.000000}\selectfont \(\displaystyle {\ensuremath{-}20}\)}%
\end{pgfscope}%
\begin{pgfscope}%
\pgfpathrectangle{\pgfqpoint{0.750000in}{0.500000in}}{\pgfqpoint{4.650000in}{3.020000in}}%
\pgfusepath{clip}%
\pgfsetrectcap%
\pgfsetroundjoin%
\pgfsetlinewidth{0.803000pt}%
\definecolor{currentstroke}{rgb}{0.690196,0.690196,0.690196}%
\pgfsetstrokecolor{currentstroke}%
\pgfsetdash{}{0pt}%
\pgfpathmoveto{\pgfqpoint{3.158184in}{0.500000in}}%
\pgfpathlineto{\pgfqpoint{3.158184in}{3.520000in}}%
\pgfusepath{stroke}%
\end{pgfscope}%
\begin{pgfscope}%
\pgfsetbuttcap%
\pgfsetroundjoin%
\definecolor{currentfill}{rgb}{0.000000,0.000000,0.000000}%
\pgfsetfillcolor{currentfill}%
\pgfsetlinewidth{0.803000pt}%
\definecolor{currentstroke}{rgb}{0.000000,0.000000,0.000000}%
\pgfsetstrokecolor{currentstroke}%
\pgfsetdash{}{0pt}%
\pgfsys@defobject{currentmarker}{\pgfqpoint{0.000000in}{-0.048611in}}{\pgfqpoint{0.000000in}{0.000000in}}{%
\pgfpathmoveto{\pgfqpoint{0.000000in}{0.000000in}}%
\pgfpathlineto{\pgfqpoint{0.000000in}{-0.048611in}}%
\pgfusepath{stroke,fill}%
}%
\begin{pgfscope}%
\pgfsys@transformshift{3.158184in}{0.500000in}%
\pgfsys@useobject{currentmarker}{}%
\end{pgfscope}%
\end{pgfscope}%
\begin{pgfscope}%
\definecolor{textcolor}{rgb}{0.000000,0.000000,0.000000}%
\pgfsetstrokecolor{textcolor}%
\pgfsetfillcolor{textcolor}%
\pgftext[x=3.158184in,y=0.402778in,,top]{\color{textcolor}\sffamily\fontsize{10.000000}{12.000000}\selectfont \(\displaystyle {\ensuremath{-}10}\)}%
\end{pgfscope}%
\begin{pgfscope}%
\pgfpathrectangle{\pgfqpoint{0.750000in}{0.500000in}}{\pgfqpoint{4.650000in}{3.020000in}}%
\pgfusepath{clip}%
\pgfsetrectcap%
\pgfsetroundjoin%
\pgfsetlinewidth{0.803000pt}%
\definecolor{currentstroke}{rgb}{0.690196,0.690196,0.690196}%
\pgfsetstrokecolor{currentstroke}%
\pgfsetdash{}{0pt}%
\pgfpathmoveto{\pgfqpoint{3.925287in}{0.500000in}}%
\pgfpathlineto{\pgfqpoint{3.925287in}{3.520000in}}%
\pgfusepath{stroke}%
\end{pgfscope}%
\begin{pgfscope}%
\pgfsetbuttcap%
\pgfsetroundjoin%
\definecolor{currentfill}{rgb}{0.000000,0.000000,0.000000}%
\pgfsetfillcolor{currentfill}%
\pgfsetlinewidth{0.803000pt}%
\definecolor{currentstroke}{rgb}{0.000000,0.000000,0.000000}%
\pgfsetstrokecolor{currentstroke}%
\pgfsetdash{}{0pt}%
\pgfsys@defobject{currentmarker}{\pgfqpoint{0.000000in}{-0.048611in}}{\pgfqpoint{0.000000in}{0.000000in}}{%
\pgfpathmoveto{\pgfqpoint{0.000000in}{0.000000in}}%
\pgfpathlineto{\pgfqpoint{0.000000in}{-0.048611in}}%
\pgfusepath{stroke,fill}%
}%
\begin{pgfscope}%
\pgfsys@transformshift{3.925287in}{0.500000in}%
\pgfsys@useobject{currentmarker}{}%
\end{pgfscope}%
\end{pgfscope}%
\begin{pgfscope}%
\definecolor{textcolor}{rgb}{0.000000,0.000000,0.000000}%
\pgfsetstrokecolor{textcolor}%
\pgfsetfillcolor{textcolor}%
\pgftext[x=3.925287in,y=0.402778in,,top]{\color{textcolor}\sffamily\fontsize{10.000000}{12.000000}\selectfont \(\displaystyle {0}\)}%
\end{pgfscope}%
\begin{pgfscope}%
\pgfpathrectangle{\pgfqpoint{0.750000in}{0.500000in}}{\pgfqpoint{4.650000in}{3.020000in}}%
\pgfusepath{clip}%
\pgfsetrectcap%
\pgfsetroundjoin%
\pgfsetlinewidth{0.803000pt}%
\definecolor{currentstroke}{rgb}{0.690196,0.690196,0.690196}%
\pgfsetstrokecolor{currentstroke}%
\pgfsetdash{}{0pt}%
\pgfpathmoveto{\pgfqpoint{4.692390in}{0.500000in}}%
\pgfpathlineto{\pgfqpoint{4.692390in}{3.520000in}}%
\pgfusepath{stroke}%
\end{pgfscope}%
\begin{pgfscope}%
\pgfsetbuttcap%
\pgfsetroundjoin%
\definecolor{currentfill}{rgb}{0.000000,0.000000,0.000000}%
\pgfsetfillcolor{currentfill}%
\pgfsetlinewidth{0.803000pt}%
\definecolor{currentstroke}{rgb}{0.000000,0.000000,0.000000}%
\pgfsetstrokecolor{currentstroke}%
\pgfsetdash{}{0pt}%
\pgfsys@defobject{currentmarker}{\pgfqpoint{0.000000in}{-0.048611in}}{\pgfqpoint{0.000000in}{0.000000in}}{%
\pgfpathmoveto{\pgfqpoint{0.000000in}{0.000000in}}%
\pgfpathlineto{\pgfqpoint{0.000000in}{-0.048611in}}%
\pgfusepath{stroke,fill}%
}%
\begin{pgfscope}%
\pgfsys@transformshift{4.692390in}{0.500000in}%
\pgfsys@useobject{currentmarker}{}%
\end{pgfscope}%
\end{pgfscope}%
\begin{pgfscope}%
\definecolor{textcolor}{rgb}{0.000000,0.000000,0.000000}%
\pgfsetstrokecolor{textcolor}%
\pgfsetfillcolor{textcolor}%
\pgftext[x=4.692390in,y=0.402778in,,top]{\color{textcolor}\sffamily\fontsize{10.000000}{12.000000}\selectfont \(\displaystyle {10}\)}%
\end{pgfscope}%
\begin{pgfscope}%
\definecolor{textcolor}{rgb}{0.000000,0.000000,0.000000}%
\pgfsetstrokecolor{textcolor}%
\pgfsetfillcolor{textcolor}%
\pgftext[x=3.075000in,y=0.223766in,,top]{\color{textcolor}\sffamily\fontsize{10.000000}{12.000000}\selectfont \(\displaystyle \zeta \, \mathrm{[\mu m]}\)}%
\end{pgfscope}%
\begin{pgfscope}%
\pgfpathrectangle{\pgfqpoint{0.750000in}{0.500000in}}{\pgfqpoint{4.650000in}{3.020000in}}%
\pgfusepath{clip}%
\pgfsetrectcap%
\pgfsetroundjoin%
\pgfsetlinewidth{0.803000pt}%
\definecolor{currentstroke}{rgb}{0.690196,0.690196,0.690196}%
\pgfsetstrokecolor{currentstroke}%
\pgfsetdash{}{0pt}%
\pgfpathmoveto{\pgfqpoint{0.750000in}{0.584911in}}%
\pgfpathlineto{\pgfqpoint{5.400000in}{0.584911in}}%
\pgfusepath{stroke}%
\end{pgfscope}%
\begin{pgfscope}%
\pgfsetbuttcap%
\pgfsetroundjoin%
\definecolor{currentfill}{rgb}{0.000000,0.000000,0.000000}%
\pgfsetfillcolor{currentfill}%
\pgfsetlinewidth{0.803000pt}%
\definecolor{currentstroke}{rgb}{0.000000,0.000000,0.000000}%
\pgfsetstrokecolor{currentstroke}%
\pgfsetdash{}{0pt}%
\pgfsys@defobject{currentmarker}{\pgfqpoint{-0.048611in}{0.000000in}}{\pgfqpoint{-0.000000in}{0.000000in}}{%
\pgfpathmoveto{\pgfqpoint{-0.000000in}{0.000000in}}%
\pgfpathlineto{\pgfqpoint{-0.048611in}{0.000000in}}%
\pgfusepath{stroke,fill}%
}%
\begin{pgfscope}%
\pgfsys@transformshift{0.750000in}{0.584911in}%
\pgfsys@useobject{currentmarker}{}%
\end{pgfscope}%
\end{pgfscope}%
\begin{pgfscope}%
\definecolor{textcolor}{rgb}{0.000000,0.000000,0.000000}%
\pgfsetstrokecolor{textcolor}%
\pgfsetfillcolor{textcolor}%
\pgftext[x=0.336419in, y=0.536685in, left, base]{\color{textcolor}\sffamily\fontsize{10.000000}{12.000000}\selectfont \(\displaystyle {\ensuremath{-}800}\)}%
\end{pgfscope}%
\begin{pgfscope}%
\pgfpathrectangle{\pgfqpoint{0.750000in}{0.500000in}}{\pgfqpoint{4.650000in}{3.020000in}}%
\pgfusepath{clip}%
\pgfsetrectcap%
\pgfsetroundjoin%
\pgfsetlinewidth{0.803000pt}%
\definecolor{currentstroke}{rgb}{0.690196,0.690196,0.690196}%
\pgfsetstrokecolor{currentstroke}%
\pgfsetdash{}{0pt}%
\pgfpathmoveto{\pgfqpoint{0.750000in}{0.925963in}}%
\pgfpathlineto{\pgfqpoint{5.400000in}{0.925963in}}%
\pgfusepath{stroke}%
\end{pgfscope}%
\begin{pgfscope}%
\pgfsetbuttcap%
\pgfsetroundjoin%
\definecolor{currentfill}{rgb}{0.000000,0.000000,0.000000}%
\pgfsetfillcolor{currentfill}%
\pgfsetlinewidth{0.803000pt}%
\definecolor{currentstroke}{rgb}{0.000000,0.000000,0.000000}%
\pgfsetstrokecolor{currentstroke}%
\pgfsetdash{}{0pt}%
\pgfsys@defobject{currentmarker}{\pgfqpoint{-0.048611in}{0.000000in}}{\pgfqpoint{-0.000000in}{0.000000in}}{%
\pgfpathmoveto{\pgfqpoint{-0.000000in}{0.000000in}}%
\pgfpathlineto{\pgfqpoint{-0.048611in}{0.000000in}}%
\pgfusepath{stroke,fill}%
}%
\begin{pgfscope}%
\pgfsys@transformshift{0.750000in}{0.925963in}%
\pgfsys@useobject{currentmarker}{}%
\end{pgfscope}%
\end{pgfscope}%
\begin{pgfscope}%
\definecolor{textcolor}{rgb}{0.000000,0.000000,0.000000}%
\pgfsetstrokecolor{textcolor}%
\pgfsetfillcolor{textcolor}%
\pgftext[x=0.336419in, y=0.877738in, left, base]{\color{textcolor}\sffamily\fontsize{10.000000}{12.000000}\selectfont \(\displaystyle {\ensuremath{-}600}\)}%
\end{pgfscope}%
\begin{pgfscope}%
\pgfpathrectangle{\pgfqpoint{0.750000in}{0.500000in}}{\pgfqpoint{4.650000in}{3.020000in}}%
\pgfusepath{clip}%
\pgfsetrectcap%
\pgfsetroundjoin%
\pgfsetlinewidth{0.803000pt}%
\definecolor{currentstroke}{rgb}{0.690196,0.690196,0.690196}%
\pgfsetstrokecolor{currentstroke}%
\pgfsetdash{}{0pt}%
\pgfpathmoveto{\pgfqpoint{0.750000in}{1.267016in}}%
\pgfpathlineto{\pgfqpoint{5.400000in}{1.267016in}}%
\pgfusepath{stroke}%
\end{pgfscope}%
\begin{pgfscope}%
\pgfsetbuttcap%
\pgfsetroundjoin%
\definecolor{currentfill}{rgb}{0.000000,0.000000,0.000000}%
\pgfsetfillcolor{currentfill}%
\pgfsetlinewidth{0.803000pt}%
\definecolor{currentstroke}{rgb}{0.000000,0.000000,0.000000}%
\pgfsetstrokecolor{currentstroke}%
\pgfsetdash{}{0pt}%
\pgfsys@defobject{currentmarker}{\pgfqpoint{-0.048611in}{0.000000in}}{\pgfqpoint{-0.000000in}{0.000000in}}{%
\pgfpathmoveto{\pgfqpoint{-0.000000in}{0.000000in}}%
\pgfpathlineto{\pgfqpoint{-0.048611in}{0.000000in}}%
\pgfusepath{stroke,fill}%
}%
\begin{pgfscope}%
\pgfsys@transformshift{0.750000in}{1.267016in}%
\pgfsys@useobject{currentmarker}{}%
\end{pgfscope}%
\end{pgfscope}%
\begin{pgfscope}%
\definecolor{textcolor}{rgb}{0.000000,0.000000,0.000000}%
\pgfsetstrokecolor{textcolor}%
\pgfsetfillcolor{textcolor}%
\pgftext[x=0.336419in, y=1.218791in, left, base]{\color{textcolor}\sffamily\fontsize{10.000000}{12.000000}\selectfont \(\displaystyle {\ensuremath{-}400}\)}%
\end{pgfscope}%
\begin{pgfscope}%
\pgfpathrectangle{\pgfqpoint{0.750000in}{0.500000in}}{\pgfqpoint{4.650000in}{3.020000in}}%
\pgfusepath{clip}%
\pgfsetrectcap%
\pgfsetroundjoin%
\pgfsetlinewidth{0.803000pt}%
\definecolor{currentstroke}{rgb}{0.690196,0.690196,0.690196}%
\pgfsetstrokecolor{currentstroke}%
\pgfsetdash{}{0pt}%
\pgfpathmoveto{\pgfqpoint{0.750000in}{1.608069in}}%
\pgfpathlineto{\pgfqpoint{5.400000in}{1.608069in}}%
\pgfusepath{stroke}%
\end{pgfscope}%
\begin{pgfscope}%
\pgfsetbuttcap%
\pgfsetroundjoin%
\definecolor{currentfill}{rgb}{0.000000,0.000000,0.000000}%
\pgfsetfillcolor{currentfill}%
\pgfsetlinewidth{0.803000pt}%
\definecolor{currentstroke}{rgb}{0.000000,0.000000,0.000000}%
\pgfsetstrokecolor{currentstroke}%
\pgfsetdash{}{0pt}%
\pgfsys@defobject{currentmarker}{\pgfqpoint{-0.048611in}{0.000000in}}{\pgfqpoint{-0.000000in}{0.000000in}}{%
\pgfpathmoveto{\pgfqpoint{-0.000000in}{0.000000in}}%
\pgfpathlineto{\pgfqpoint{-0.048611in}{0.000000in}}%
\pgfusepath{stroke,fill}%
}%
\begin{pgfscope}%
\pgfsys@transformshift{0.750000in}{1.608069in}%
\pgfsys@useobject{currentmarker}{}%
\end{pgfscope}%
\end{pgfscope}%
\begin{pgfscope}%
\definecolor{textcolor}{rgb}{0.000000,0.000000,0.000000}%
\pgfsetstrokecolor{textcolor}%
\pgfsetfillcolor{textcolor}%
\pgftext[x=0.336419in, y=1.559844in, left, base]{\color{textcolor}\sffamily\fontsize{10.000000}{12.000000}\selectfont \(\displaystyle {\ensuremath{-}200}\)}%
\end{pgfscope}%
\begin{pgfscope}%
\pgfpathrectangle{\pgfqpoint{0.750000in}{0.500000in}}{\pgfqpoint{4.650000in}{3.020000in}}%
\pgfusepath{clip}%
\pgfsetrectcap%
\pgfsetroundjoin%
\pgfsetlinewidth{0.803000pt}%
\definecolor{currentstroke}{rgb}{0.690196,0.690196,0.690196}%
\pgfsetstrokecolor{currentstroke}%
\pgfsetdash{}{0pt}%
\pgfpathmoveto{\pgfqpoint{0.750000in}{1.949122in}}%
\pgfpathlineto{\pgfqpoint{5.400000in}{1.949122in}}%
\pgfusepath{stroke}%
\end{pgfscope}%
\begin{pgfscope}%
\pgfsetbuttcap%
\pgfsetroundjoin%
\definecolor{currentfill}{rgb}{0.000000,0.000000,0.000000}%
\pgfsetfillcolor{currentfill}%
\pgfsetlinewidth{0.803000pt}%
\definecolor{currentstroke}{rgb}{0.000000,0.000000,0.000000}%
\pgfsetstrokecolor{currentstroke}%
\pgfsetdash{}{0pt}%
\pgfsys@defobject{currentmarker}{\pgfqpoint{-0.048611in}{0.000000in}}{\pgfqpoint{-0.000000in}{0.000000in}}{%
\pgfpathmoveto{\pgfqpoint{-0.000000in}{0.000000in}}%
\pgfpathlineto{\pgfqpoint{-0.048611in}{0.000000in}}%
\pgfusepath{stroke,fill}%
}%
\begin{pgfscope}%
\pgfsys@transformshift{0.750000in}{1.949122in}%
\pgfsys@useobject{currentmarker}{}%
\end{pgfscope}%
\end{pgfscope}%
\begin{pgfscope}%
\definecolor{textcolor}{rgb}{0.000000,0.000000,0.000000}%
\pgfsetstrokecolor{textcolor}%
\pgfsetfillcolor{textcolor}%
\pgftext[x=0.583333in, y=1.900896in, left, base]{\color{textcolor}\sffamily\fontsize{10.000000}{12.000000}\selectfont \(\displaystyle {0}\)}%
\end{pgfscope}%
\begin{pgfscope}%
\pgfpathrectangle{\pgfqpoint{0.750000in}{0.500000in}}{\pgfqpoint{4.650000in}{3.020000in}}%
\pgfusepath{clip}%
\pgfsetrectcap%
\pgfsetroundjoin%
\pgfsetlinewidth{0.803000pt}%
\definecolor{currentstroke}{rgb}{0.690196,0.690196,0.690196}%
\pgfsetstrokecolor{currentstroke}%
\pgfsetdash{}{0pt}%
\pgfpathmoveto{\pgfqpoint{0.750000in}{2.290174in}}%
\pgfpathlineto{\pgfqpoint{5.400000in}{2.290174in}}%
\pgfusepath{stroke}%
\end{pgfscope}%
\begin{pgfscope}%
\pgfsetbuttcap%
\pgfsetroundjoin%
\definecolor{currentfill}{rgb}{0.000000,0.000000,0.000000}%
\pgfsetfillcolor{currentfill}%
\pgfsetlinewidth{0.803000pt}%
\definecolor{currentstroke}{rgb}{0.000000,0.000000,0.000000}%
\pgfsetstrokecolor{currentstroke}%
\pgfsetdash{}{0pt}%
\pgfsys@defobject{currentmarker}{\pgfqpoint{-0.048611in}{0.000000in}}{\pgfqpoint{-0.000000in}{0.000000in}}{%
\pgfpathmoveto{\pgfqpoint{-0.000000in}{0.000000in}}%
\pgfpathlineto{\pgfqpoint{-0.048611in}{0.000000in}}%
\pgfusepath{stroke,fill}%
}%
\begin{pgfscope}%
\pgfsys@transformshift{0.750000in}{2.290174in}%
\pgfsys@useobject{currentmarker}{}%
\end{pgfscope}%
\end{pgfscope}%
\begin{pgfscope}%
\definecolor{textcolor}{rgb}{0.000000,0.000000,0.000000}%
\pgfsetstrokecolor{textcolor}%
\pgfsetfillcolor{textcolor}%
\pgftext[x=0.444444in, y=2.241949in, left, base]{\color{textcolor}\sffamily\fontsize{10.000000}{12.000000}\selectfont \(\displaystyle {200}\)}%
\end{pgfscope}%
\begin{pgfscope}%
\pgfpathrectangle{\pgfqpoint{0.750000in}{0.500000in}}{\pgfqpoint{4.650000in}{3.020000in}}%
\pgfusepath{clip}%
\pgfsetrectcap%
\pgfsetroundjoin%
\pgfsetlinewidth{0.803000pt}%
\definecolor{currentstroke}{rgb}{0.690196,0.690196,0.690196}%
\pgfsetstrokecolor{currentstroke}%
\pgfsetdash{}{0pt}%
\pgfpathmoveto{\pgfqpoint{0.750000in}{2.631227in}}%
\pgfpathlineto{\pgfqpoint{5.400000in}{2.631227in}}%
\pgfusepath{stroke}%
\end{pgfscope}%
\begin{pgfscope}%
\pgfsetbuttcap%
\pgfsetroundjoin%
\definecolor{currentfill}{rgb}{0.000000,0.000000,0.000000}%
\pgfsetfillcolor{currentfill}%
\pgfsetlinewidth{0.803000pt}%
\definecolor{currentstroke}{rgb}{0.000000,0.000000,0.000000}%
\pgfsetstrokecolor{currentstroke}%
\pgfsetdash{}{0pt}%
\pgfsys@defobject{currentmarker}{\pgfqpoint{-0.048611in}{0.000000in}}{\pgfqpoint{-0.000000in}{0.000000in}}{%
\pgfpathmoveto{\pgfqpoint{-0.000000in}{0.000000in}}%
\pgfpathlineto{\pgfqpoint{-0.048611in}{0.000000in}}%
\pgfusepath{stroke,fill}%
}%
\begin{pgfscope}%
\pgfsys@transformshift{0.750000in}{2.631227in}%
\pgfsys@useobject{currentmarker}{}%
\end{pgfscope}%
\end{pgfscope}%
\begin{pgfscope}%
\definecolor{textcolor}{rgb}{0.000000,0.000000,0.000000}%
\pgfsetstrokecolor{textcolor}%
\pgfsetfillcolor{textcolor}%
\pgftext[x=0.444444in, y=2.583002in, left, base]{\color{textcolor}\sffamily\fontsize{10.000000}{12.000000}\selectfont \(\displaystyle {400}\)}%
\end{pgfscope}%
\begin{pgfscope}%
\pgfpathrectangle{\pgfqpoint{0.750000in}{0.500000in}}{\pgfqpoint{4.650000in}{3.020000in}}%
\pgfusepath{clip}%
\pgfsetrectcap%
\pgfsetroundjoin%
\pgfsetlinewidth{0.803000pt}%
\definecolor{currentstroke}{rgb}{0.690196,0.690196,0.690196}%
\pgfsetstrokecolor{currentstroke}%
\pgfsetdash{}{0pt}%
\pgfpathmoveto{\pgfqpoint{0.750000in}{2.972280in}}%
\pgfpathlineto{\pgfqpoint{5.400000in}{2.972280in}}%
\pgfusepath{stroke}%
\end{pgfscope}%
\begin{pgfscope}%
\pgfsetbuttcap%
\pgfsetroundjoin%
\definecolor{currentfill}{rgb}{0.000000,0.000000,0.000000}%
\pgfsetfillcolor{currentfill}%
\pgfsetlinewidth{0.803000pt}%
\definecolor{currentstroke}{rgb}{0.000000,0.000000,0.000000}%
\pgfsetstrokecolor{currentstroke}%
\pgfsetdash{}{0pt}%
\pgfsys@defobject{currentmarker}{\pgfqpoint{-0.048611in}{0.000000in}}{\pgfqpoint{-0.000000in}{0.000000in}}{%
\pgfpathmoveto{\pgfqpoint{-0.000000in}{0.000000in}}%
\pgfpathlineto{\pgfqpoint{-0.048611in}{0.000000in}}%
\pgfusepath{stroke,fill}%
}%
\begin{pgfscope}%
\pgfsys@transformshift{0.750000in}{2.972280in}%
\pgfsys@useobject{currentmarker}{}%
\end{pgfscope}%
\end{pgfscope}%
\begin{pgfscope}%
\definecolor{textcolor}{rgb}{0.000000,0.000000,0.000000}%
\pgfsetstrokecolor{textcolor}%
\pgfsetfillcolor{textcolor}%
\pgftext[x=0.444444in, y=2.924055in, left, base]{\color{textcolor}\sffamily\fontsize{10.000000}{12.000000}\selectfont \(\displaystyle {600}\)}%
\end{pgfscope}%
\begin{pgfscope}%
\pgfpathrectangle{\pgfqpoint{0.750000in}{0.500000in}}{\pgfqpoint{4.650000in}{3.020000in}}%
\pgfusepath{clip}%
\pgfsetrectcap%
\pgfsetroundjoin%
\pgfsetlinewidth{0.803000pt}%
\definecolor{currentstroke}{rgb}{0.690196,0.690196,0.690196}%
\pgfsetstrokecolor{currentstroke}%
\pgfsetdash{}{0pt}%
\pgfpathmoveto{\pgfqpoint{0.750000in}{3.313333in}}%
\pgfpathlineto{\pgfqpoint{5.400000in}{3.313333in}}%
\pgfusepath{stroke}%
\end{pgfscope}%
\begin{pgfscope}%
\pgfsetbuttcap%
\pgfsetroundjoin%
\definecolor{currentfill}{rgb}{0.000000,0.000000,0.000000}%
\pgfsetfillcolor{currentfill}%
\pgfsetlinewidth{0.803000pt}%
\definecolor{currentstroke}{rgb}{0.000000,0.000000,0.000000}%
\pgfsetstrokecolor{currentstroke}%
\pgfsetdash{}{0pt}%
\pgfsys@defobject{currentmarker}{\pgfqpoint{-0.048611in}{0.000000in}}{\pgfqpoint{-0.000000in}{0.000000in}}{%
\pgfpathmoveto{\pgfqpoint{-0.000000in}{0.000000in}}%
\pgfpathlineto{\pgfqpoint{-0.048611in}{0.000000in}}%
\pgfusepath{stroke,fill}%
}%
\begin{pgfscope}%
\pgfsys@transformshift{0.750000in}{3.313333in}%
\pgfsys@useobject{currentmarker}{}%
\end{pgfscope}%
\end{pgfscope}%
\begin{pgfscope}%
\definecolor{textcolor}{rgb}{0.000000,0.000000,0.000000}%
\pgfsetstrokecolor{textcolor}%
\pgfsetfillcolor{textcolor}%
\pgftext[x=0.444444in, y=3.265107in, left, base]{\color{textcolor}\sffamily\fontsize{10.000000}{12.000000}\selectfont \(\displaystyle {800}\)}%
\end{pgfscope}%
\begin{pgfscope}%
\definecolor{textcolor}{rgb}{0.000000,0.000000,0.000000}%
\pgfsetstrokecolor{textcolor}%
\pgfsetfillcolor{textcolor}%
\pgftext[x=0.280863in,y=2.010000in,,bottom,rotate=90.000000]{\color{textcolor}\sffamily\fontsize{10.000000}{12.000000}\selectfont \(\displaystyle \mathrm{Energy \, gain \, [MeV]}\)}%
\end{pgfscope}%
\begin{pgfscope}%
\pgfpathrectangle{\pgfqpoint{0.750000in}{0.500000in}}{\pgfqpoint{4.650000in}{3.020000in}}%
\pgfusepath{clip}%
\pgfsetrectcap%
\pgfsetroundjoin%
\pgfsetlinewidth{1.505625pt}%
\definecolor{currentstroke}{rgb}{0.121569,0.466667,0.705882}%
\pgfsetstrokecolor{currentstroke}%
\pgfsetdash{}{0pt}%
\pgfpathmoveto{\pgfqpoint{0.961364in}{1.287819in}}%
\pgfpathlineto{\pgfqpoint{0.968160in}{1.265149in}}%
\pgfpathlineto{\pgfqpoint{0.981753in}{1.233661in}}%
\pgfpathlineto{\pgfqpoint{1.015736in}{1.136131in}}%
\pgfpathlineto{\pgfqpoint{1.022532in}{1.108035in}}%
\pgfpathlineto{\pgfqpoint{1.029329in}{1.096324in}}%
\pgfpathlineto{\pgfqpoint{1.036125in}{1.068785in}}%
\pgfpathlineto{\pgfqpoint{1.042922in}{1.053613in}}%
\pgfpathlineto{\pgfqpoint{1.049719in}{1.032215in}}%
\pgfpathlineto{\pgfqpoint{1.063312in}{0.982180in}}%
\pgfpathlineto{\pgfqpoint{1.090498in}{0.891940in}}%
\pgfpathlineto{\pgfqpoint{1.097294in}{0.859197in}}%
\pgfpathlineto{\pgfqpoint{1.110887in}{0.842696in}}%
\pgfpathlineto{\pgfqpoint{1.117684in}{0.882718in}}%
\pgfpathlineto{\pgfqpoint{1.124480in}{0.997031in}}%
\pgfpathlineto{\pgfqpoint{1.144870in}{1.796110in}}%
\pgfpathlineto{\pgfqpoint{1.151666in}{1.964769in}}%
\pgfpathlineto{\pgfqpoint{1.158463in}{2.091420in}}%
\pgfpathlineto{\pgfqpoint{1.165260in}{2.148147in}}%
\pgfpathlineto{\pgfqpoint{1.172056in}{2.183454in}}%
\pgfpathlineto{\pgfqpoint{1.178853in}{2.202076in}}%
\pgfpathlineto{\pgfqpoint{1.185649in}{2.209523in}}%
\pgfpathlineto{\pgfqpoint{1.199242in}{2.258674in}}%
\pgfpathlineto{\pgfqpoint{1.212835in}{2.340394in}}%
\pgfpathlineto{\pgfqpoint{1.226428in}{2.347207in}}%
\pgfpathlineto{\pgfqpoint{1.233225in}{2.338060in}}%
\pgfpathlineto{\pgfqpoint{1.240021in}{2.344112in}}%
\pgfpathlineto{\pgfqpoint{1.260411in}{2.424047in}}%
\pgfpathlineto{\pgfqpoint{1.267207in}{2.445206in}}%
\pgfpathlineto{\pgfqpoint{1.274004in}{2.442306in}}%
\pgfpathlineto{\pgfqpoint{1.280801in}{2.446931in}}%
\pgfpathlineto{\pgfqpoint{1.287597in}{2.446716in}}%
\pgfpathlineto{\pgfqpoint{1.294394in}{2.442244in}}%
\pgfpathlineto{\pgfqpoint{1.301190in}{2.451657in}}%
\pgfpathlineto{\pgfqpoint{1.307987in}{2.441832in}}%
\pgfpathlineto{\pgfqpoint{1.314783in}{2.439195in}}%
\pgfpathlineto{\pgfqpoint{1.321580in}{2.438668in}}%
\pgfpathlineto{\pgfqpoint{1.328376in}{2.426356in}}%
\pgfpathlineto{\pgfqpoint{1.335173in}{2.424633in}}%
\pgfpathlineto{\pgfqpoint{1.348766in}{2.411330in}}%
\pgfpathlineto{\pgfqpoint{1.362359in}{2.397495in}}%
\pgfpathlineto{\pgfqpoint{1.369155in}{2.386703in}}%
\pgfpathlineto{\pgfqpoint{1.375952in}{2.383818in}}%
\pgfpathlineto{\pgfqpoint{1.389545in}{2.370208in}}%
\pgfpathlineto{\pgfqpoint{1.396342in}{2.367811in}}%
\pgfpathlineto{\pgfqpoint{1.409935in}{2.351445in}}%
\pgfpathlineto{\pgfqpoint{1.430324in}{2.341758in}}%
\pgfpathlineto{\pgfqpoint{1.437121in}{2.334546in}}%
\pgfpathlineto{\pgfqpoint{1.457510in}{2.327015in}}%
\pgfpathlineto{\pgfqpoint{1.471103in}{2.317263in}}%
\pgfpathlineto{\pgfqpoint{1.477900in}{2.318029in}}%
\pgfpathlineto{\pgfqpoint{1.484696in}{2.314365in}}%
\pgfpathlineto{\pgfqpoint{1.491493in}{2.313216in}}%
\pgfpathlineto{\pgfqpoint{1.498289in}{2.309875in}}%
\pgfpathlineto{\pgfqpoint{1.511883in}{2.308209in}}%
\pgfpathlineto{\pgfqpoint{1.518679in}{2.309402in}}%
\pgfpathlineto{\pgfqpoint{1.525476in}{2.307619in}}%
\pgfpathlineto{\pgfqpoint{1.532272in}{2.303605in}}%
\pgfpathlineto{\pgfqpoint{1.539069in}{2.301839in}}%
\pgfpathlineto{\pgfqpoint{1.545865in}{2.304973in}}%
\pgfpathlineto{\pgfqpoint{1.552662in}{2.302190in}}%
\pgfpathlineto{\pgfqpoint{1.559458in}{2.301748in}}%
\pgfpathlineto{\pgfqpoint{1.566255in}{2.305296in}}%
\pgfpathlineto{\pgfqpoint{1.573051in}{2.296501in}}%
\pgfpathlineto{\pgfqpoint{1.579848in}{2.303802in}}%
\pgfpathlineto{\pgfqpoint{1.586644in}{2.300110in}}%
\pgfpathlineto{\pgfqpoint{1.593441in}{2.299332in}}%
\pgfpathlineto{\pgfqpoint{1.600237in}{2.301504in}}%
\pgfpathlineto{\pgfqpoint{1.607034in}{2.300368in}}%
\pgfpathlineto{\pgfqpoint{1.613830in}{2.296230in}}%
\pgfpathlineto{\pgfqpoint{1.620627in}{2.298147in}}%
\pgfpathlineto{\pgfqpoint{1.634220in}{2.294984in}}%
\pgfpathlineto{\pgfqpoint{1.641017in}{2.297731in}}%
\pgfpathlineto{\pgfqpoint{1.647813in}{2.293111in}}%
\pgfpathlineto{\pgfqpoint{1.654610in}{2.291287in}}%
\pgfpathlineto{\pgfqpoint{1.661406in}{2.291493in}}%
\pgfpathlineto{\pgfqpoint{1.668203in}{2.287435in}}%
\pgfpathlineto{\pgfqpoint{1.674999in}{2.291125in}}%
\pgfpathlineto{\pgfqpoint{1.688592in}{2.284196in}}%
\pgfpathlineto{\pgfqpoint{1.695389in}{2.284866in}}%
\pgfpathlineto{\pgfqpoint{1.702185in}{2.278325in}}%
\pgfpathlineto{\pgfqpoint{1.708982in}{2.274944in}}%
\pgfpathlineto{\pgfqpoint{1.715778in}{2.277662in}}%
\pgfpathlineto{\pgfqpoint{1.722575in}{2.267411in}}%
\pgfpathlineto{\pgfqpoint{1.729371in}{2.274909in}}%
\pgfpathlineto{\pgfqpoint{1.736168in}{2.264431in}}%
\pgfpathlineto{\pgfqpoint{1.742965in}{2.260917in}}%
\pgfpathlineto{\pgfqpoint{1.749761in}{2.259659in}}%
\pgfpathlineto{\pgfqpoint{1.756558in}{2.251574in}}%
\pgfpathlineto{\pgfqpoint{1.770151in}{2.248156in}}%
\pgfpathlineto{\pgfqpoint{1.776947in}{2.234546in}}%
\pgfpathlineto{\pgfqpoint{1.783744in}{2.238542in}}%
\pgfpathlineto{\pgfqpoint{1.797337in}{2.224840in}}%
\pgfpathlineto{\pgfqpoint{1.804133in}{2.223706in}}%
\pgfpathlineto{\pgfqpoint{1.810930in}{2.209869in}}%
\pgfpathlineto{\pgfqpoint{1.817726in}{2.210812in}}%
\pgfpathlineto{\pgfqpoint{1.824523in}{2.204178in}}%
\pgfpathlineto{\pgfqpoint{1.831319in}{2.187983in}}%
\pgfpathlineto{\pgfqpoint{1.838116in}{2.189806in}}%
\pgfpathlineto{\pgfqpoint{1.844912in}{2.180404in}}%
\pgfpathlineto{\pgfqpoint{1.851709in}{2.173687in}}%
\pgfpathlineto{\pgfqpoint{1.858506in}{2.171623in}}%
\pgfpathlineto{\pgfqpoint{1.865302in}{2.157771in}}%
\pgfpathlineto{\pgfqpoint{1.872099in}{2.152787in}}%
\pgfpathlineto{\pgfqpoint{1.878895in}{2.145267in}}%
\pgfpathlineto{\pgfqpoint{1.885692in}{2.132705in}}%
\pgfpathlineto{\pgfqpoint{1.892488in}{2.128321in}}%
\pgfpathlineto{\pgfqpoint{1.899285in}{2.118704in}}%
\pgfpathlineto{\pgfqpoint{1.906081in}{2.105604in}}%
\pgfpathlineto{\pgfqpoint{1.912878in}{2.104047in}}%
\pgfpathlineto{\pgfqpoint{1.919674in}{2.088748in}}%
\pgfpathlineto{\pgfqpoint{1.933267in}{2.074456in}}%
\pgfpathlineto{\pgfqpoint{1.940064in}{2.056136in}}%
\pgfpathlineto{\pgfqpoint{1.946860in}{2.053789in}}%
\pgfpathlineto{\pgfqpoint{1.953657in}{2.043827in}}%
\pgfpathlineto{\pgfqpoint{1.960453in}{2.022196in}}%
\pgfpathlineto{\pgfqpoint{1.967250in}{2.028573in}}%
\pgfpathlineto{\pgfqpoint{1.974047in}{2.001101in}}%
\pgfpathlineto{\pgfqpoint{1.980843in}{1.998545in}}%
\pgfpathlineto{\pgfqpoint{1.987640in}{1.992323in}}%
\pgfpathlineto{\pgfqpoint{1.994436in}{1.967557in}}%
\pgfpathlineto{\pgfqpoint{2.001233in}{1.965714in}}%
\pgfpathlineto{\pgfqpoint{2.035215in}{1.901958in}}%
\pgfpathlineto{\pgfqpoint{2.042012in}{1.892588in}}%
\pgfpathlineto{\pgfqpoint{2.069198in}{1.835145in}}%
\pgfpathlineto{\pgfqpoint{2.075994in}{1.828080in}}%
\pgfpathlineto{\pgfqpoint{2.082791in}{1.811161in}}%
\pgfpathlineto{\pgfqpoint{2.089588in}{1.798859in}}%
\pgfpathlineto{\pgfqpoint{2.096384in}{1.782865in}}%
\pgfpathlineto{\pgfqpoint{2.103181in}{1.773136in}}%
\pgfpathlineto{\pgfqpoint{2.109977in}{1.752828in}}%
\pgfpathlineto{\pgfqpoint{2.116774in}{1.748974in}}%
\pgfpathlineto{\pgfqpoint{2.123570in}{1.720679in}}%
\pgfpathlineto{\pgfqpoint{2.130367in}{1.724293in}}%
\pgfpathlineto{\pgfqpoint{2.137163in}{1.694109in}}%
\pgfpathlineto{\pgfqpoint{2.143960in}{1.685725in}}%
\pgfpathlineto{\pgfqpoint{2.150756in}{1.674004in}}%
\pgfpathlineto{\pgfqpoint{2.157553in}{1.649355in}}%
\pgfpathlineto{\pgfqpoint{2.164349in}{1.644361in}}%
\pgfpathlineto{\pgfqpoint{2.171146in}{1.621331in}}%
\pgfpathlineto{\pgfqpoint{2.177942in}{1.606012in}}%
\pgfpathlineto{\pgfqpoint{2.184739in}{1.599586in}}%
\pgfpathlineto{\pgfqpoint{2.191535in}{1.577560in}}%
\pgfpathlineto{\pgfqpoint{2.205129in}{1.554966in}}%
\pgfpathlineto{\pgfqpoint{2.211925in}{1.525319in}}%
\pgfpathlineto{\pgfqpoint{2.218722in}{1.518670in}}%
\pgfpathlineto{\pgfqpoint{2.225518in}{1.495289in}}%
\pgfpathlineto{\pgfqpoint{2.232315in}{1.489141in}}%
\pgfpathlineto{\pgfqpoint{2.239111in}{1.465390in}}%
\pgfpathlineto{\pgfqpoint{2.245908in}{1.455189in}}%
\pgfpathlineto{\pgfqpoint{2.286687in}{1.352927in}}%
\pgfpathlineto{\pgfqpoint{2.293483in}{1.340299in}}%
\pgfpathlineto{\pgfqpoint{2.300280in}{1.315907in}}%
\pgfpathlineto{\pgfqpoint{2.307076in}{1.305112in}}%
\pgfpathlineto{\pgfqpoint{2.313873in}{1.284239in}}%
\pgfpathlineto{\pgfqpoint{2.320670in}{1.270928in}}%
\pgfpathlineto{\pgfqpoint{2.327466in}{1.253570in}}%
\pgfpathlineto{\pgfqpoint{2.334263in}{1.228584in}}%
\pgfpathlineto{\pgfqpoint{2.341059in}{1.221761in}}%
\pgfpathlineto{\pgfqpoint{2.354652in}{1.179054in}}%
\pgfpathlineto{\pgfqpoint{2.361449in}{1.167102in}}%
\pgfpathlineto{\pgfqpoint{2.368245in}{1.140957in}}%
\pgfpathlineto{\pgfqpoint{2.388635in}{1.090610in}}%
\pgfpathlineto{\pgfqpoint{2.395431in}{1.069501in}}%
\pgfpathlineto{\pgfqpoint{2.402228in}{1.055707in}}%
\pgfpathlineto{\pgfqpoint{2.429414in}{0.972606in}}%
\pgfpathlineto{\pgfqpoint{2.436211in}{0.944197in}}%
\pgfpathlineto{\pgfqpoint{2.443007in}{0.947952in}}%
\pgfpathlineto{\pgfqpoint{2.449804in}{0.896916in}}%
\pgfpathlineto{\pgfqpoint{2.456600in}{0.894708in}}%
\pgfpathlineto{\pgfqpoint{2.463397in}{0.869642in}}%
\pgfpathlineto{\pgfqpoint{2.470193in}{0.834599in}}%
\pgfpathlineto{\pgfqpoint{2.476990in}{0.833379in}}%
\pgfpathlineto{\pgfqpoint{2.483786in}{0.779828in}}%
\pgfpathlineto{\pgfqpoint{2.490583in}{0.783679in}}%
\pgfpathlineto{\pgfqpoint{2.497379in}{0.736986in}}%
\pgfpathlineto{\pgfqpoint{2.504176in}{0.715852in}}%
\pgfpathlineto{\pgfqpoint{2.510972in}{0.705962in}}%
\pgfpathlineto{\pgfqpoint{2.517769in}{0.637273in}}%
\pgfpathlineto{\pgfqpoint{2.524565in}{0.749871in}}%
\pgfpathlineto{\pgfqpoint{2.531362in}{1.635585in}}%
\pgfpathlineto{\pgfqpoint{2.538158in}{2.912111in}}%
\pgfpathlineto{\pgfqpoint{2.544955in}{3.382727in}}%
\pgfpathlineto{\pgfqpoint{2.551752in}{3.287959in}}%
\pgfpathlineto{\pgfqpoint{2.558548in}{3.276488in}}%
\pgfpathlineto{\pgfqpoint{2.565345in}{3.273415in}}%
\pgfpathlineto{\pgfqpoint{2.572141in}{3.115233in}}%
\pgfpathlineto{\pgfqpoint{2.585734in}{3.008598in}}%
\pgfpathlineto{\pgfqpoint{2.592531in}{3.079013in}}%
\pgfpathlineto{\pgfqpoint{2.599327in}{3.043412in}}%
\pgfpathlineto{\pgfqpoint{2.612920in}{2.901405in}}%
\pgfpathlineto{\pgfqpoint{2.619717in}{2.876332in}}%
\pgfpathlineto{\pgfqpoint{2.626513in}{2.797947in}}%
\pgfpathlineto{\pgfqpoint{2.633310in}{2.799950in}}%
\pgfpathlineto{\pgfqpoint{2.640106in}{2.731812in}}%
\pgfpathlineto{\pgfqpoint{2.646903in}{2.721186in}}%
\pgfpathlineto{\pgfqpoint{2.653699in}{2.679541in}}%
\pgfpathlineto{\pgfqpoint{2.660496in}{2.666676in}}%
\pgfpathlineto{\pgfqpoint{2.667293in}{2.631752in}}%
\pgfpathlineto{\pgfqpoint{2.674089in}{2.611904in}}%
\pgfpathlineto{\pgfqpoint{2.680886in}{2.600481in}}%
\pgfpathlineto{\pgfqpoint{2.687682in}{2.568238in}}%
\pgfpathlineto{\pgfqpoint{2.694479in}{2.574064in}}%
\pgfpathlineto{\pgfqpoint{2.701275in}{2.539574in}}%
\pgfpathlineto{\pgfqpoint{2.708072in}{2.533496in}}%
\pgfpathlineto{\pgfqpoint{2.721665in}{2.506785in}}%
\pgfpathlineto{\pgfqpoint{2.728461in}{2.490726in}}%
\pgfpathlineto{\pgfqpoint{2.735258in}{2.498817in}}%
\pgfpathlineto{\pgfqpoint{2.742054in}{2.467600in}}%
\pgfpathlineto{\pgfqpoint{2.748851in}{2.496653in}}%
\pgfpathlineto{\pgfqpoint{2.755647in}{2.479152in}}%
\pgfpathlineto{\pgfqpoint{2.762444in}{2.501078in}}%
\pgfpathlineto{\pgfqpoint{2.769240in}{2.486570in}}%
\pgfpathlineto{\pgfqpoint{2.776037in}{2.509463in}}%
\pgfpathlineto{\pgfqpoint{2.782834in}{2.498411in}}%
\pgfpathlineto{\pgfqpoint{2.789630in}{2.518341in}}%
\pgfpathlineto{\pgfqpoint{2.796427in}{2.507619in}}%
\pgfpathlineto{\pgfqpoint{2.803223in}{2.525616in}}%
\pgfpathlineto{\pgfqpoint{2.810020in}{2.527634in}}%
\pgfpathlineto{\pgfqpoint{2.816816in}{2.512040in}}%
\pgfpathlineto{\pgfqpoint{2.823613in}{2.541109in}}%
\pgfpathlineto{\pgfqpoint{2.830409in}{2.526637in}}%
\pgfpathlineto{\pgfqpoint{2.837206in}{2.539336in}}%
\pgfpathlineto{\pgfqpoint{2.844002in}{2.528415in}}%
\pgfpathlineto{\pgfqpoint{2.850799in}{2.551898in}}%
\pgfpathlineto{\pgfqpoint{2.857595in}{2.537227in}}%
\pgfpathlineto{\pgfqpoint{2.864392in}{2.553222in}}%
\pgfpathlineto{\pgfqpoint{2.871188in}{2.536433in}}%
\pgfpathlineto{\pgfqpoint{2.877985in}{2.552166in}}%
\pgfpathlineto{\pgfqpoint{2.884781in}{2.549147in}}%
\pgfpathlineto{\pgfqpoint{2.898375in}{2.547014in}}%
\pgfpathlineto{\pgfqpoint{2.905171in}{2.561833in}}%
\pgfpathlineto{\pgfqpoint{2.911968in}{2.540895in}}%
\pgfpathlineto{\pgfqpoint{2.918764in}{2.560233in}}%
\pgfpathlineto{\pgfqpoint{2.925561in}{2.555450in}}%
\pgfpathlineto{\pgfqpoint{2.932357in}{2.555291in}}%
\pgfpathlineto{\pgfqpoint{2.939154in}{2.559710in}}%
\pgfpathlineto{\pgfqpoint{2.945950in}{2.546537in}}%
\pgfpathlineto{\pgfqpoint{2.952747in}{2.555078in}}%
\pgfpathlineto{\pgfqpoint{2.959543in}{2.544958in}}%
\pgfpathlineto{\pgfqpoint{2.966340in}{2.555136in}}%
\pgfpathlineto{\pgfqpoint{2.973136in}{2.536435in}}%
\pgfpathlineto{\pgfqpoint{2.979933in}{2.550583in}}%
\pgfpathlineto{\pgfqpoint{2.986729in}{2.533325in}}%
\pgfpathlineto{\pgfqpoint{2.993526in}{2.538502in}}%
\pgfpathlineto{\pgfqpoint{3.000322in}{2.537777in}}%
\pgfpathlineto{\pgfqpoint{3.007119in}{2.522568in}}%
\pgfpathlineto{\pgfqpoint{3.013916in}{2.539216in}}%
\pgfpathlineto{\pgfqpoint{3.020712in}{2.520433in}}%
\pgfpathlineto{\pgfqpoint{3.027509in}{2.517763in}}%
\pgfpathlineto{\pgfqpoint{3.034305in}{2.519961in}}%
\pgfpathlineto{\pgfqpoint{3.041102in}{2.496127in}}%
\pgfpathlineto{\pgfqpoint{3.047898in}{2.524649in}}%
\pgfpathlineto{\pgfqpoint{3.054695in}{2.493552in}}%
\pgfpathlineto{\pgfqpoint{3.061491in}{2.491599in}}%
\pgfpathlineto{\pgfqpoint{3.068288in}{2.492990in}}%
\pgfpathlineto{\pgfqpoint{3.075084in}{2.478590in}}%
\pgfpathlineto{\pgfqpoint{3.081881in}{2.467981in}}%
\pgfpathlineto{\pgfqpoint{3.088677in}{2.485683in}}%
\pgfpathlineto{\pgfqpoint{3.095474in}{2.445285in}}%
\pgfpathlineto{\pgfqpoint{3.102270in}{2.466691in}}%
\pgfpathlineto{\pgfqpoint{3.109067in}{2.448025in}}%
\pgfpathlineto{\pgfqpoint{3.115863in}{2.435727in}}%
\pgfpathlineto{\pgfqpoint{3.122660in}{2.439115in}}%
\pgfpathlineto{\pgfqpoint{3.129456in}{2.425457in}}%
\pgfpathlineto{\pgfqpoint{3.136253in}{2.417689in}}%
\pgfpathlineto{\pgfqpoint{3.143050in}{2.418084in}}%
\pgfpathlineto{\pgfqpoint{3.149846in}{2.395466in}}%
\pgfpathlineto{\pgfqpoint{3.156643in}{2.403397in}}%
\pgfpathlineto{\pgfqpoint{3.163439in}{2.386867in}}%
\pgfpathlineto{\pgfqpoint{3.170236in}{2.383105in}}%
\pgfpathlineto{\pgfqpoint{3.177032in}{2.359903in}}%
\pgfpathlineto{\pgfqpoint{3.183829in}{2.376222in}}%
\pgfpathlineto{\pgfqpoint{3.190625in}{2.337924in}}%
\pgfpathlineto{\pgfqpoint{3.197422in}{2.364547in}}%
\pgfpathlineto{\pgfqpoint{3.204218in}{2.319358in}}%
\pgfpathlineto{\pgfqpoint{3.211015in}{2.328159in}}%
\pgfpathlineto{\pgfqpoint{3.217811in}{2.309199in}}%
\pgfpathlineto{\pgfqpoint{3.224608in}{2.314723in}}%
\pgfpathlineto{\pgfqpoint{3.231404in}{2.273736in}}%
\pgfpathlineto{\pgfqpoint{3.238201in}{2.305424in}}%
\pgfpathlineto{\pgfqpoint{3.244997in}{2.256189in}}%
\pgfpathlineto{\pgfqpoint{3.251794in}{2.274152in}}%
\pgfpathlineto{\pgfqpoint{3.258591in}{2.257627in}}%
\pgfpathlineto{\pgfqpoint{3.265387in}{2.225650in}}%
\pgfpathlineto{\pgfqpoint{3.272184in}{2.260171in}}%
\pgfpathlineto{\pgfqpoint{3.278980in}{2.204356in}}%
\pgfpathlineto{\pgfqpoint{3.285777in}{2.225083in}}%
\pgfpathlineto{\pgfqpoint{3.292573in}{2.204409in}}%
\pgfpathlineto{\pgfqpoint{3.299370in}{2.190828in}}%
\pgfpathlineto{\pgfqpoint{3.306166in}{2.190869in}}%
\pgfpathlineto{\pgfqpoint{3.312963in}{2.172021in}}%
\pgfpathlineto{\pgfqpoint{3.319759in}{2.158684in}}%
\pgfpathlineto{\pgfqpoint{3.326556in}{2.164202in}}%
\pgfpathlineto{\pgfqpoint{3.333352in}{2.137304in}}%
\pgfpathlineto{\pgfqpoint{3.340149in}{2.138142in}}%
\pgfpathlineto{\pgfqpoint{3.353742in}{2.110230in}}%
\pgfpathlineto{\pgfqpoint{3.360538in}{2.096278in}}%
\pgfpathlineto{\pgfqpoint{3.367335in}{2.099904in}}%
\pgfpathlineto{\pgfqpoint{3.374132in}{2.063795in}}%
\pgfpathlineto{\pgfqpoint{3.380928in}{2.068751in}}%
\pgfpathlineto{\pgfqpoint{3.387725in}{2.053366in}}%
\pgfpathlineto{\pgfqpoint{3.394521in}{2.031334in}}%
\pgfpathlineto{\pgfqpoint{3.401318in}{2.033164in}}%
\pgfpathlineto{\pgfqpoint{3.408114in}{2.006331in}}%
\pgfpathlineto{\pgfqpoint{3.414911in}{2.001163in}}%
\pgfpathlineto{\pgfqpoint{3.421707in}{1.998795in}}%
\pgfpathlineto{\pgfqpoint{3.428504in}{1.949961in}}%
\pgfpathlineto{\pgfqpoint{3.435300in}{1.990255in}}%
\pgfpathlineto{\pgfqpoint{3.442097in}{1.923842in}}%
\pgfpathlineto{\pgfqpoint{3.448893in}{1.950477in}}%
\pgfpathlineto{\pgfqpoint{3.455690in}{1.904988in}}%
\pgfpathlineto{\pgfqpoint{3.462486in}{1.914924in}}%
\pgfpathlineto{\pgfqpoint{3.469283in}{1.884243in}}%
\pgfpathlineto{\pgfqpoint{3.476079in}{1.889256in}}%
\pgfpathlineto{\pgfqpoint{3.482876in}{1.856187in}}%
\pgfpathlineto{\pgfqpoint{3.496469in}{1.841581in}}%
\pgfpathlineto{\pgfqpoint{3.503266in}{1.800471in}}%
\pgfpathlineto{\pgfqpoint{3.510062in}{1.818833in}}%
\pgfpathlineto{\pgfqpoint{3.516859in}{1.773286in}}%
\pgfpathlineto{\pgfqpoint{3.523655in}{1.782929in}}%
\pgfpathlineto{\pgfqpoint{3.530452in}{1.744993in}}%
\pgfpathlineto{\pgfqpoint{3.537248in}{1.742512in}}%
\pgfpathlineto{\pgfqpoint{3.544045in}{1.724955in}}%
\pgfpathlineto{\pgfqpoint{3.550841in}{1.696840in}}%
\pgfpathlineto{\pgfqpoint{3.557638in}{1.702964in}}%
\pgfpathlineto{\pgfqpoint{3.564434in}{1.658422in}}%
\pgfpathlineto{\pgfqpoint{3.571231in}{1.659234in}}%
\pgfpathlineto{\pgfqpoint{3.578027in}{1.644341in}}%
\pgfpathlineto{\pgfqpoint{3.584824in}{1.614914in}}%
\pgfpathlineto{\pgfqpoint{3.591620in}{1.608840in}}%
\pgfpathlineto{\pgfqpoint{3.605214in}{1.573359in}}%
\pgfpathlineto{\pgfqpoint{3.612010in}{1.561466in}}%
\pgfpathlineto{\pgfqpoint{3.618807in}{1.543010in}}%
\pgfpathlineto{\pgfqpoint{3.625603in}{1.519466in}}%
\pgfpathlineto{\pgfqpoint{3.632400in}{1.520074in}}%
\pgfpathlineto{\pgfqpoint{3.639196in}{1.481614in}}%
\pgfpathlineto{\pgfqpoint{3.645993in}{1.493907in}}%
\pgfpathlineto{\pgfqpoint{3.652789in}{1.449245in}}%
\pgfpathlineto{\pgfqpoint{3.666382in}{1.446263in}}%
\pgfpathlineto{\pgfqpoint{3.673179in}{1.404452in}}%
\pgfpathlineto{\pgfqpoint{3.679975in}{1.406847in}}%
\pgfpathlineto{\pgfqpoint{3.686772in}{1.375989in}}%
\pgfpathlineto{\pgfqpoint{3.693568in}{1.387408in}}%
\pgfpathlineto{\pgfqpoint{3.700365in}{1.342021in}}%
\pgfpathlineto{\pgfqpoint{3.707161in}{1.358210in}}%
\pgfpathlineto{\pgfqpoint{3.713958in}{1.326227in}}%
\pgfpathlineto{\pgfqpoint{3.720755in}{1.322378in}}%
\pgfpathlineto{\pgfqpoint{3.734348in}{1.286459in}}%
\pgfpathlineto{\pgfqpoint{3.754737in}{1.257194in}}%
\pgfpathlineto{\pgfqpoint{3.761534in}{1.243556in}}%
\pgfpathlineto{\pgfqpoint{3.768330in}{1.234137in}}%
\pgfpathlineto{\pgfqpoint{3.775127in}{1.211236in}}%
\pgfpathlineto{\pgfqpoint{3.781923in}{1.223844in}}%
\pgfpathlineto{\pgfqpoint{3.788720in}{1.189465in}}%
\pgfpathlineto{\pgfqpoint{3.795516in}{1.189538in}}%
\pgfpathlineto{\pgfqpoint{3.802313in}{1.195467in}}%
\pgfpathlineto{\pgfqpoint{3.809109in}{1.154871in}}%
\pgfpathlineto{\pgfqpoint{3.815906in}{1.178731in}}%
\pgfpathlineto{\pgfqpoint{3.822702in}{1.145158in}}%
\pgfpathlineto{\pgfqpoint{3.829499in}{1.154212in}}%
\pgfpathlineto{\pgfqpoint{3.836296in}{1.156182in}}%
\pgfpathlineto{\pgfqpoint{3.843092in}{1.127476in}}%
\pgfpathlineto{\pgfqpoint{3.849889in}{1.140738in}}%
\pgfpathlineto{\pgfqpoint{3.856685in}{1.137461in}}%
\pgfpathlineto{\pgfqpoint{3.863482in}{1.108398in}}%
\pgfpathlineto{\pgfqpoint{3.870278in}{1.143249in}}%
\pgfpathlineto{\pgfqpoint{3.877075in}{1.105592in}}%
\pgfpathlineto{\pgfqpoint{3.883871in}{1.116533in}}%
\pgfpathlineto{\pgfqpoint{3.890668in}{1.122414in}}%
\pgfpathlineto{\pgfqpoint{3.897464in}{1.111769in}}%
\pgfpathlineto{\pgfqpoint{3.911057in}{1.123674in}}%
\pgfpathlineto{\pgfqpoint{3.917854in}{1.135580in}}%
\pgfpathlineto{\pgfqpoint{3.924650in}{1.109809in}}%
\pgfpathlineto{\pgfqpoint{3.931447in}{1.161594in}}%
\pgfpathlineto{\pgfqpoint{3.938243in}{1.110410in}}%
\pgfpathlineto{\pgfqpoint{3.945040in}{1.174609in}}%
\pgfpathlineto{\pgfqpoint{3.951837in}{1.152613in}}%
\pgfpathlineto{\pgfqpoint{3.958633in}{1.154264in}}%
\pgfpathlineto{\pgfqpoint{3.965430in}{1.192357in}}%
\pgfpathlineto{\pgfqpoint{3.972226in}{1.178739in}}%
\pgfpathlineto{\pgfqpoint{3.979023in}{1.203046in}}%
\pgfpathlineto{\pgfqpoint{3.985819in}{1.210543in}}%
\pgfpathlineto{\pgfqpoint{3.992616in}{1.235649in}}%
\pgfpathlineto{\pgfqpoint{3.999412in}{1.247205in}}%
\pgfpathlineto{\pgfqpoint{4.006209in}{1.274369in}}%
\pgfpathlineto{\pgfqpoint{4.013005in}{1.263415in}}%
\pgfpathlineto{\pgfqpoint{4.019802in}{1.308844in}}%
\pgfpathlineto{\pgfqpoint{4.026598in}{1.316837in}}%
\pgfpathlineto{\pgfqpoint{4.033395in}{1.337299in}}%
\pgfpathlineto{\pgfqpoint{4.040191in}{1.373683in}}%
\pgfpathlineto{\pgfqpoint{4.046988in}{1.374014in}}%
\pgfpathlineto{\pgfqpoint{4.053784in}{1.417476in}}%
\pgfpathlineto{\pgfqpoint{4.060581in}{1.420200in}}%
\pgfpathlineto{\pgfqpoint{4.067378in}{1.455806in}}%
\pgfpathlineto{\pgfqpoint{4.074174in}{1.467683in}}%
\pgfpathlineto{\pgfqpoint{4.080971in}{1.506582in}}%
\pgfpathlineto{\pgfqpoint{4.087767in}{1.515709in}}%
\pgfpathlineto{\pgfqpoint{4.094564in}{1.560414in}}%
\pgfpathlineto{\pgfqpoint{4.101360in}{1.556389in}}%
\pgfpathlineto{\pgfqpoint{4.108157in}{1.613693in}}%
\pgfpathlineto{\pgfqpoint{4.114953in}{1.604397in}}%
\pgfpathlineto{\pgfqpoint{4.121750in}{1.654413in}}%
\pgfpathlineto{\pgfqpoint{4.128546in}{1.651500in}}%
\pgfpathlineto{\pgfqpoint{4.135343in}{1.684103in}}%
\pgfpathlineto{\pgfqpoint{4.142139in}{1.700762in}}%
\pgfpathlineto{\pgfqpoint{4.148936in}{1.712918in}}%
\pgfpathlineto{\pgfqpoint{4.155732in}{1.737680in}}%
\pgfpathlineto{\pgfqpoint{4.162529in}{1.751932in}}%
\pgfpathlineto{\pgfqpoint{4.182919in}{1.806003in}}%
\pgfpathlineto{\pgfqpoint{4.189715in}{1.808612in}}%
\pgfpathlineto{\pgfqpoint{4.196512in}{1.824616in}}%
\pgfpathlineto{\pgfqpoint{4.203308in}{1.833085in}}%
\pgfpathlineto{\pgfqpoint{4.216901in}{1.853008in}}%
\pgfpathlineto{\pgfqpoint{4.230494in}{1.872638in}}%
\pgfpathlineto{\pgfqpoint{4.237291in}{1.874298in}}%
\pgfpathlineto{\pgfqpoint{4.244087in}{1.891535in}}%
\pgfpathlineto{\pgfqpoint{4.250884in}{1.892101in}}%
\pgfpathlineto{\pgfqpoint{4.257680in}{1.903320in}}%
\pgfpathlineto{\pgfqpoint{4.264477in}{1.902598in}}%
\pgfpathlineto{\pgfqpoint{4.271273in}{1.913340in}}%
\pgfpathlineto{\pgfqpoint{4.278070in}{1.911725in}}%
\pgfpathlineto{\pgfqpoint{4.284866in}{1.921040in}}%
\pgfpathlineto{\pgfqpoint{4.291663in}{1.918633in}}%
\pgfpathlineto{\pgfqpoint{4.298460in}{1.926020in}}%
\pgfpathlineto{\pgfqpoint{4.305256in}{1.924588in}}%
\pgfpathlineto{\pgfqpoint{4.312053in}{1.933137in}}%
\pgfpathlineto{\pgfqpoint{4.318849in}{1.931186in}}%
\pgfpathlineto{\pgfqpoint{4.325646in}{1.934823in}}%
\pgfpathlineto{\pgfqpoint{4.339239in}{1.935300in}}%
\pgfpathlineto{\pgfqpoint{4.346035in}{1.938848in}}%
\pgfpathlineto{\pgfqpoint{4.359628in}{1.939876in}}%
\pgfpathlineto{\pgfqpoint{4.393611in}{1.944253in}}%
\pgfpathlineto{\pgfqpoint{4.400407in}{1.943135in}}%
\pgfpathlineto{\pgfqpoint{4.427594in}{1.945521in}}%
\pgfpathlineto{\pgfqpoint{4.454780in}{1.945965in}}%
\pgfpathlineto{\pgfqpoint{4.536338in}{1.948811in}}%
\pgfpathlineto{\pgfqpoint{4.583914in}{1.948752in}}%
\pgfpathlineto{\pgfqpoint{4.964519in}{1.949122in}}%
\pgfpathlineto{\pgfqpoint{5.032485in}{1.949109in}}%
\pgfpathlineto{\pgfqpoint{5.032485in}{1.949109in}}%
\pgfusepath{stroke}%
\end{pgfscope}%
\begin{pgfscope}%
\pgfpathrectangle{\pgfqpoint{0.750000in}{0.500000in}}{\pgfqpoint{4.650000in}{3.020000in}}%
\pgfusepath{clip}%
\pgfsetrectcap%
\pgfsetroundjoin%
\pgfsetlinewidth{1.505625pt}%
\definecolor{currentstroke}{rgb}{1.000000,0.498039,0.054902}%
\pgfsetstrokecolor{currentstroke}%
\pgfsetdash{}{0pt}%
\pgfpathmoveto{\pgfqpoint{1.000415in}{1.119160in}}%
\pgfpathlineto{\pgfqpoint{1.014008in}{1.080729in}}%
\pgfpathlineto{\pgfqpoint{1.020804in}{1.067421in}}%
\pgfpathlineto{\pgfqpoint{1.027601in}{1.043013in}}%
\pgfpathlineto{\pgfqpoint{1.047991in}{0.988612in}}%
\pgfpathlineto{\pgfqpoint{1.068380in}{0.931768in}}%
\pgfpathlineto{\pgfqpoint{1.088770in}{0.879704in}}%
\pgfpathlineto{\pgfqpoint{1.095566in}{0.856547in}}%
\pgfpathlineto{\pgfqpoint{1.109159in}{0.824037in}}%
\pgfpathlineto{\pgfqpoint{1.115956in}{0.802790in}}%
\pgfpathlineto{\pgfqpoint{1.136345in}{0.759898in}}%
\pgfpathlineto{\pgfqpoint{1.143142in}{0.747135in}}%
\pgfpathlineto{\pgfqpoint{1.149939in}{0.754210in}}%
\pgfpathlineto{\pgfqpoint{1.156735in}{0.741886in}}%
\pgfpathlineto{\pgfqpoint{1.163532in}{0.758216in}}%
\pgfpathlineto{\pgfqpoint{1.177125in}{0.872313in}}%
\pgfpathlineto{\pgfqpoint{1.190718in}{1.142622in}}%
\pgfpathlineto{\pgfqpoint{1.204311in}{1.476477in}}%
\pgfpathlineto{\pgfqpoint{1.217904in}{1.834903in}}%
\pgfpathlineto{\pgfqpoint{1.224700in}{1.942709in}}%
\pgfpathlineto{\pgfqpoint{1.238293in}{2.026212in}}%
\pgfpathlineto{\pgfqpoint{1.245090in}{2.051591in}}%
\pgfpathlineto{\pgfqpoint{1.258683in}{2.129002in}}%
\pgfpathlineto{\pgfqpoint{1.272276in}{2.197397in}}%
\pgfpathlineto{\pgfqpoint{1.279073in}{2.228101in}}%
\pgfpathlineto{\pgfqpoint{1.285869in}{2.247542in}}%
\pgfpathlineto{\pgfqpoint{1.313055in}{2.352803in}}%
\pgfpathlineto{\pgfqpoint{1.340241in}{2.428086in}}%
\pgfpathlineto{\pgfqpoint{1.347038in}{2.440665in}}%
\pgfpathlineto{\pgfqpoint{1.353834in}{2.458857in}}%
\pgfpathlineto{\pgfqpoint{1.374224in}{2.494355in}}%
\pgfpathlineto{\pgfqpoint{1.387817in}{2.518173in}}%
\pgfpathlineto{\pgfqpoint{1.394614in}{2.524533in}}%
\pgfpathlineto{\pgfqpoint{1.401410in}{2.539969in}}%
\pgfpathlineto{\pgfqpoint{1.408207in}{2.542678in}}%
\pgfpathlineto{\pgfqpoint{1.415003in}{2.557820in}}%
\pgfpathlineto{\pgfqpoint{1.421800in}{2.559084in}}%
\pgfpathlineto{\pgfqpoint{1.428596in}{2.575288in}}%
\pgfpathlineto{\pgfqpoint{1.442189in}{2.579124in}}%
\pgfpathlineto{\pgfqpoint{1.448986in}{2.592269in}}%
\pgfpathlineto{\pgfqpoint{1.455782in}{2.589113in}}%
\pgfpathlineto{\pgfqpoint{1.462579in}{2.607662in}}%
\pgfpathlineto{\pgfqpoint{1.469375in}{2.603653in}}%
\pgfpathlineto{\pgfqpoint{1.476172in}{2.609845in}}%
\pgfpathlineto{\pgfqpoint{1.482968in}{2.611209in}}%
\pgfpathlineto{\pgfqpoint{1.496562in}{2.619308in}}%
\pgfpathlineto{\pgfqpoint{1.503358in}{2.622763in}}%
\pgfpathlineto{\pgfqpoint{1.510155in}{2.619255in}}%
\pgfpathlineto{\pgfqpoint{1.516951in}{2.627356in}}%
\pgfpathlineto{\pgfqpoint{1.523748in}{2.626164in}}%
\pgfpathlineto{\pgfqpoint{1.530544in}{2.629054in}}%
\pgfpathlineto{\pgfqpoint{1.537341in}{2.623528in}}%
\pgfpathlineto{\pgfqpoint{1.544137in}{2.628676in}}%
\pgfpathlineto{\pgfqpoint{1.550934in}{2.626169in}}%
\pgfpathlineto{\pgfqpoint{1.557730in}{2.630319in}}%
\pgfpathlineto{\pgfqpoint{1.564527in}{2.623501in}}%
\pgfpathlineto{\pgfqpoint{1.578120in}{2.622122in}}%
\pgfpathlineto{\pgfqpoint{1.584916in}{2.615212in}}%
\pgfpathlineto{\pgfqpoint{1.591713in}{2.616063in}}%
\pgfpathlineto{\pgfqpoint{1.598509in}{2.610802in}}%
\pgfpathlineto{\pgfqpoint{1.612103in}{2.604445in}}%
\pgfpathlineto{\pgfqpoint{1.618899in}{2.597303in}}%
\pgfpathlineto{\pgfqpoint{1.625696in}{2.598050in}}%
\pgfpathlineto{\pgfqpoint{1.632492in}{2.583711in}}%
\pgfpathlineto{\pgfqpoint{1.639289in}{2.583648in}}%
\pgfpathlineto{\pgfqpoint{1.646085in}{2.579007in}}%
\pgfpathlineto{\pgfqpoint{1.652882in}{2.569765in}}%
\pgfpathlineto{\pgfqpoint{1.659678in}{2.565599in}}%
\pgfpathlineto{\pgfqpoint{1.673271in}{2.547783in}}%
\pgfpathlineto{\pgfqpoint{1.680068in}{2.549261in}}%
\pgfpathlineto{\pgfqpoint{1.686864in}{2.534219in}}%
\pgfpathlineto{\pgfqpoint{1.693661in}{2.525537in}}%
\pgfpathlineto{\pgfqpoint{1.700457in}{2.519423in}}%
\pgfpathlineto{\pgfqpoint{1.707254in}{2.506501in}}%
\pgfpathlineto{\pgfqpoint{1.714050in}{2.502240in}}%
\pgfpathlineto{\pgfqpoint{1.720847in}{2.494914in}}%
\pgfpathlineto{\pgfqpoint{1.727643in}{2.477868in}}%
\pgfpathlineto{\pgfqpoint{1.734440in}{2.476269in}}%
\pgfpathlineto{\pgfqpoint{1.741237in}{2.460886in}}%
\pgfpathlineto{\pgfqpoint{1.748033in}{2.451225in}}%
\pgfpathlineto{\pgfqpoint{1.754830in}{2.444979in}}%
\pgfpathlineto{\pgfqpoint{1.761626in}{2.427722in}}%
\pgfpathlineto{\pgfqpoint{1.768423in}{2.421969in}}%
\pgfpathlineto{\pgfqpoint{1.775219in}{2.412032in}}%
\pgfpathlineto{\pgfqpoint{1.782016in}{2.398075in}}%
\pgfpathlineto{\pgfqpoint{1.788812in}{2.389453in}}%
\pgfpathlineto{\pgfqpoint{1.815998in}{2.340193in}}%
\pgfpathlineto{\pgfqpoint{1.822795in}{2.325540in}}%
\pgfpathlineto{\pgfqpoint{1.829591in}{2.323497in}}%
\pgfpathlineto{\pgfqpoint{1.836388in}{2.296367in}}%
\pgfpathlineto{\pgfqpoint{1.843184in}{2.293538in}}%
\pgfpathlineto{\pgfqpoint{1.849981in}{2.274458in}}%
\pgfpathlineto{\pgfqpoint{1.856778in}{2.267079in}}%
\pgfpathlineto{\pgfqpoint{1.883964in}{2.210822in}}%
\pgfpathlineto{\pgfqpoint{1.890760in}{2.190563in}}%
\pgfpathlineto{\pgfqpoint{1.897557in}{2.189450in}}%
\pgfpathlineto{\pgfqpoint{1.904353in}{2.164770in}}%
\pgfpathlineto{\pgfqpoint{1.911150in}{2.157605in}}%
\pgfpathlineto{\pgfqpoint{1.917946in}{2.139458in}}%
\pgfpathlineto{\pgfqpoint{1.931539in}{2.114617in}}%
\pgfpathlineto{\pgfqpoint{1.938336in}{2.094494in}}%
\pgfpathlineto{\pgfqpoint{1.945132in}{2.083422in}}%
\pgfpathlineto{\pgfqpoint{1.951929in}{2.067497in}}%
\pgfpathlineto{\pgfqpoint{1.958725in}{2.056575in}}%
\pgfpathlineto{\pgfqpoint{1.972319in}{2.022795in}}%
\pgfpathlineto{\pgfqpoint{1.979115in}{2.009852in}}%
\pgfpathlineto{\pgfqpoint{1.985912in}{1.988904in}}%
\pgfpathlineto{\pgfqpoint{1.992708in}{1.983208in}}%
\pgfpathlineto{\pgfqpoint{1.999505in}{1.958397in}}%
\pgfpathlineto{\pgfqpoint{2.013098in}{1.935284in}}%
\pgfpathlineto{\pgfqpoint{2.019894in}{1.913265in}}%
\pgfpathlineto{\pgfqpoint{2.026691in}{1.907043in}}%
\pgfpathlineto{\pgfqpoint{2.033487in}{1.882572in}}%
\pgfpathlineto{\pgfqpoint{2.040284in}{1.870681in}}%
\pgfpathlineto{\pgfqpoint{2.047080in}{1.851864in}}%
\pgfpathlineto{\pgfqpoint{2.053877in}{1.841576in}}%
\pgfpathlineto{\pgfqpoint{2.060673in}{1.815886in}}%
\pgfpathlineto{\pgfqpoint{2.074266in}{1.792091in}}%
\pgfpathlineto{\pgfqpoint{2.081063in}{1.766226in}}%
\pgfpathlineto{\pgfqpoint{2.087860in}{1.766417in}}%
\pgfpathlineto{\pgfqpoint{2.094656in}{1.736520in}}%
\pgfpathlineto{\pgfqpoint{2.101453in}{1.726661in}}%
\pgfpathlineto{\pgfqpoint{2.108249in}{1.711545in}}%
\pgfpathlineto{\pgfqpoint{2.115046in}{1.687526in}}%
\pgfpathlineto{\pgfqpoint{2.121842in}{1.680263in}}%
\pgfpathlineto{\pgfqpoint{2.128639in}{1.662573in}}%
\pgfpathlineto{\pgfqpoint{2.135435in}{1.639272in}}%
\pgfpathlineto{\pgfqpoint{2.142232in}{1.629932in}}%
\pgfpathlineto{\pgfqpoint{2.149028in}{1.608169in}}%
\pgfpathlineto{\pgfqpoint{2.169418in}{1.563999in}}%
\pgfpathlineto{\pgfqpoint{2.176214in}{1.540823in}}%
\pgfpathlineto{\pgfqpoint{2.183011in}{1.527509in}}%
\pgfpathlineto{\pgfqpoint{2.189807in}{1.507270in}}%
\pgfpathlineto{\pgfqpoint{2.196604in}{1.497136in}}%
\pgfpathlineto{\pgfqpoint{2.203401in}{1.470895in}}%
\pgfpathlineto{\pgfqpoint{2.210197in}{1.462931in}}%
\pgfpathlineto{\pgfqpoint{2.223790in}{1.421453in}}%
\pgfpathlineto{\pgfqpoint{2.230587in}{1.414358in}}%
\pgfpathlineto{\pgfqpoint{2.237383in}{1.390338in}}%
\pgfpathlineto{\pgfqpoint{2.244180in}{1.381123in}}%
\pgfpathlineto{\pgfqpoint{2.250976in}{1.351298in}}%
\pgfpathlineto{\pgfqpoint{2.257773in}{1.343799in}}%
\pgfpathlineto{\pgfqpoint{2.278162in}{1.287479in}}%
\pgfpathlineto{\pgfqpoint{2.284959in}{1.278755in}}%
\pgfpathlineto{\pgfqpoint{2.291755in}{1.253643in}}%
\pgfpathlineto{\pgfqpoint{2.298552in}{1.240797in}}%
\pgfpathlineto{\pgfqpoint{2.305348in}{1.223969in}}%
\pgfpathlineto{\pgfqpoint{2.312145in}{1.203003in}}%
\pgfpathlineto{\pgfqpoint{2.318942in}{1.195470in}}%
\pgfpathlineto{\pgfqpoint{2.325738in}{1.168383in}}%
\pgfpathlineto{\pgfqpoint{2.332535in}{1.157102in}}%
\pgfpathlineto{\pgfqpoint{2.339331in}{1.134793in}}%
\pgfpathlineto{\pgfqpoint{2.346128in}{1.121017in}}%
\pgfpathlineto{\pgfqpoint{2.359721in}{1.086736in}}%
\pgfpathlineto{\pgfqpoint{2.366517in}{1.072020in}}%
\pgfpathlineto{\pgfqpoint{2.373314in}{1.049642in}}%
\pgfpathlineto{\pgfqpoint{2.380110in}{1.041095in}}%
\pgfpathlineto{\pgfqpoint{2.393703in}{1.000342in}}%
\pgfpathlineto{\pgfqpoint{2.407296in}{0.971159in}}%
\pgfpathlineto{\pgfqpoint{2.414093in}{0.946062in}}%
\pgfpathlineto{\pgfqpoint{2.420889in}{0.937955in}}%
\pgfpathlineto{\pgfqpoint{2.427686in}{0.920628in}}%
\pgfpathlineto{\pgfqpoint{2.434483in}{0.894488in}}%
\pgfpathlineto{\pgfqpoint{2.441279in}{0.897326in}}%
\pgfpathlineto{\pgfqpoint{2.448076in}{0.865247in}}%
\pgfpathlineto{\pgfqpoint{2.454872in}{0.851007in}}%
\pgfpathlineto{\pgfqpoint{2.461669in}{0.851008in}}%
\pgfpathlineto{\pgfqpoint{2.468465in}{0.816202in}}%
\pgfpathlineto{\pgfqpoint{2.475262in}{0.813284in}}%
\pgfpathlineto{\pgfqpoint{2.482058in}{0.801710in}}%
\pgfpathlineto{\pgfqpoint{2.488855in}{0.772475in}}%
\pgfpathlineto{\pgfqpoint{2.495651in}{0.785982in}}%
\pgfpathlineto{\pgfqpoint{2.502448in}{0.755501in}}%
\pgfpathlineto{\pgfqpoint{2.509244in}{0.758812in}}%
\pgfpathlineto{\pgfqpoint{2.522837in}{0.756356in}}%
\pgfpathlineto{\pgfqpoint{2.529634in}{0.780009in}}%
\pgfpathlineto{\pgfqpoint{2.536430in}{0.810257in}}%
\pgfpathlineto{\pgfqpoint{2.543227in}{0.860621in}}%
\pgfpathlineto{\pgfqpoint{2.550024in}{0.975640in}}%
\pgfpathlineto{\pgfqpoint{2.556820in}{1.125797in}}%
\pgfpathlineto{\pgfqpoint{2.563617in}{1.413179in}}%
\pgfpathlineto{\pgfqpoint{2.577210in}{2.083880in}}%
\pgfpathlineto{\pgfqpoint{2.584006in}{2.237378in}}%
\pgfpathlineto{\pgfqpoint{2.597599in}{2.270623in}}%
\pgfpathlineto{\pgfqpoint{2.611192in}{2.369913in}}%
\pgfpathlineto{\pgfqpoint{2.617989in}{2.376817in}}%
\pgfpathlineto{\pgfqpoint{2.631582in}{2.382505in}}%
\pgfpathlineto{\pgfqpoint{2.638378in}{2.409994in}}%
\pgfpathlineto{\pgfqpoint{2.645175in}{2.411036in}}%
\pgfpathlineto{\pgfqpoint{2.651971in}{2.440632in}}%
\pgfpathlineto{\pgfqpoint{2.658768in}{2.451544in}}%
\pgfpathlineto{\pgfqpoint{2.665565in}{2.465763in}}%
\pgfpathlineto{\pgfqpoint{2.672361in}{2.494672in}}%
\pgfpathlineto{\pgfqpoint{2.679158in}{2.495563in}}%
\pgfpathlineto{\pgfqpoint{2.692751in}{2.534414in}}%
\pgfpathlineto{\pgfqpoint{2.699547in}{2.539902in}}%
\pgfpathlineto{\pgfqpoint{2.706344in}{2.562080in}}%
\pgfpathlineto{\pgfqpoint{2.713140in}{2.567650in}}%
\pgfpathlineto{\pgfqpoint{2.719937in}{2.586149in}}%
\pgfpathlineto{\pgfqpoint{2.726733in}{2.585736in}}%
\pgfpathlineto{\pgfqpoint{2.733530in}{2.616476in}}%
\pgfpathlineto{\pgfqpoint{2.740326in}{2.604834in}}%
\pgfpathlineto{\pgfqpoint{2.747123in}{2.630723in}}%
\pgfpathlineto{\pgfqpoint{2.753919in}{2.627023in}}%
\pgfpathlineto{\pgfqpoint{2.767512in}{2.653314in}}%
\pgfpathlineto{\pgfqpoint{2.774309in}{2.641692in}}%
\pgfpathlineto{\pgfqpoint{2.781106in}{2.670265in}}%
\pgfpathlineto{\pgfqpoint{2.787902in}{2.659504in}}%
\pgfpathlineto{\pgfqpoint{2.801495in}{2.682789in}}%
\pgfpathlineto{\pgfqpoint{2.808292in}{2.670479in}}%
\pgfpathlineto{\pgfqpoint{2.821885in}{2.692209in}}%
\pgfpathlineto{\pgfqpoint{2.828681in}{2.676604in}}%
\pgfpathlineto{\pgfqpoint{2.835478in}{2.698703in}}%
\pgfpathlineto{\pgfqpoint{2.842274in}{2.688832in}}%
\pgfpathlineto{\pgfqpoint{2.849071in}{2.688943in}}%
\pgfpathlineto{\pgfqpoint{2.855867in}{2.692727in}}%
\pgfpathlineto{\pgfqpoint{2.862664in}{2.691504in}}%
\pgfpathlineto{\pgfqpoint{2.869460in}{2.683974in}}%
\pgfpathlineto{\pgfqpoint{2.876257in}{2.698877in}}%
\pgfpathlineto{\pgfqpoint{2.883053in}{2.674737in}}%
\pgfpathlineto{\pgfqpoint{2.889850in}{2.693407in}}%
\pgfpathlineto{\pgfqpoint{2.903443in}{2.672512in}}%
\pgfpathlineto{\pgfqpoint{2.910240in}{2.680214in}}%
\pgfpathlineto{\pgfqpoint{2.917036in}{2.664695in}}%
\pgfpathlineto{\pgfqpoint{2.930629in}{2.668836in}}%
\pgfpathlineto{\pgfqpoint{2.937426in}{2.648854in}}%
\pgfpathlineto{\pgfqpoint{2.944222in}{2.657101in}}%
\pgfpathlineto{\pgfqpoint{2.951019in}{2.644251in}}%
\pgfpathlineto{\pgfqpoint{2.957815in}{2.636069in}}%
\pgfpathlineto{\pgfqpoint{2.964612in}{2.633992in}}%
\pgfpathlineto{\pgfqpoint{2.991798in}{2.601068in}}%
\pgfpathlineto{\pgfqpoint{2.998594in}{2.598052in}}%
\pgfpathlineto{\pgfqpoint{3.005391in}{2.576630in}}%
\pgfpathlineto{\pgfqpoint{3.012188in}{2.588075in}}%
\pgfpathlineto{\pgfqpoint{3.018984in}{2.555273in}}%
\pgfpathlineto{\pgfqpoint{3.025781in}{2.565411in}}%
\pgfpathlineto{\pgfqpoint{3.032577in}{2.542603in}}%
\pgfpathlineto{\pgfqpoint{3.039374in}{2.540698in}}%
\pgfpathlineto{\pgfqpoint{3.046170in}{2.533241in}}%
\pgfpathlineto{\pgfqpoint{3.052967in}{2.517694in}}%
\pgfpathlineto{\pgfqpoint{3.059763in}{2.522807in}}%
\pgfpathlineto{\pgfqpoint{3.066560in}{2.504761in}}%
\pgfpathlineto{\pgfqpoint{3.073356in}{2.493896in}}%
\pgfpathlineto{\pgfqpoint{3.080153in}{2.495856in}}%
\pgfpathlineto{\pgfqpoint{3.086949in}{2.475012in}}%
\pgfpathlineto{\pgfqpoint{3.093746in}{2.472142in}}%
\pgfpathlineto{\pgfqpoint{3.100542in}{2.464763in}}%
\pgfpathlineto{\pgfqpoint{3.107339in}{2.444961in}}%
\pgfpathlineto{\pgfqpoint{3.114135in}{2.445749in}}%
\pgfpathlineto{\pgfqpoint{3.120932in}{2.428322in}}%
\pgfpathlineto{\pgfqpoint{3.127729in}{2.424310in}}%
\pgfpathlineto{\pgfqpoint{3.134525in}{2.408411in}}%
\pgfpathlineto{\pgfqpoint{3.141322in}{2.400674in}}%
\pgfpathlineto{\pgfqpoint{3.148118in}{2.387903in}}%
\pgfpathlineto{\pgfqpoint{3.154915in}{2.378814in}}%
\pgfpathlineto{\pgfqpoint{3.161711in}{2.362460in}}%
\pgfpathlineto{\pgfqpoint{3.175304in}{2.341718in}}%
\pgfpathlineto{\pgfqpoint{3.182101in}{2.324633in}}%
\pgfpathlineto{\pgfqpoint{3.188897in}{2.323653in}}%
\pgfpathlineto{\pgfqpoint{3.195694in}{2.294066in}}%
\pgfpathlineto{\pgfqpoint{3.202490in}{2.298579in}}%
\pgfpathlineto{\pgfqpoint{3.209287in}{2.271833in}}%
\pgfpathlineto{\pgfqpoint{3.216083in}{2.260756in}}%
\pgfpathlineto{\pgfqpoint{3.222880in}{2.258756in}}%
\pgfpathlineto{\pgfqpoint{3.229676in}{2.223010in}}%
\pgfpathlineto{\pgfqpoint{3.236473in}{2.233416in}}%
\pgfpathlineto{\pgfqpoint{3.243270in}{2.204157in}}%
\pgfpathlineto{\pgfqpoint{3.256863in}{2.182313in}}%
\pgfpathlineto{\pgfqpoint{3.263659in}{2.162041in}}%
\pgfpathlineto{\pgfqpoint{3.270456in}{2.148951in}}%
\pgfpathlineto{\pgfqpoint{3.277252in}{2.141607in}}%
\pgfpathlineto{\pgfqpoint{3.284049in}{2.106822in}}%
\pgfpathlineto{\pgfqpoint{3.290845in}{2.118232in}}%
\pgfpathlineto{\pgfqpoint{3.297642in}{2.086937in}}%
\pgfpathlineto{\pgfqpoint{3.304438in}{2.074310in}}%
\pgfpathlineto{\pgfqpoint{3.311235in}{2.069001in}}%
\pgfpathlineto{\pgfqpoint{3.318031in}{2.044190in}}%
\pgfpathlineto{\pgfqpoint{3.324828in}{2.028629in}}%
\pgfpathlineto{\pgfqpoint{3.331624in}{2.025897in}}%
\pgfpathlineto{\pgfqpoint{3.338421in}{1.991579in}}%
\pgfpathlineto{\pgfqpoint{3.345217in}{1.990755in}}%
\pgfpathlineto{\pgfqpoint{3.358811in}{1.952899in}}%
\pgfpathlineto{\pgfqpoint{3.365607in}{1.945927in}}%
\pgfpathlineto{\pgfqpoint{3.379200in}{1.909738in}}%
\pgfpathlineto{\pgfqpoint{3.385997in}{1.905567in}}%
\pgfpathlineto{\pgfqpoint{3.392793in}{1.877063in}}%
\pgfpathlineto{\pgfqpoint{3.399590in}{1.873674in}}%
\pgfpathlineto{\pgfqpoint{3.406386in}{1.860836in}}%
\pgfpathlineto{\pgfqpoint{3.413183in}{1.832913in}}%
\pgfpathlineto{\pgfqpoint{3.419979in}{1.842821in}}%
\pgfpathlineto{\pgfqpoint{3.426776in}{1.807332in}}%
\pgfpathlineto{\pgfqpoint{3.433572in}{1.810220in}}%
\pgfpathlineto{\pgfqpoint{3.440369in}{1.791784in}}%
\pgfpathlineto{\pgfqpoint{3.447165in}{1.784160in}}%
\pgfpathlineto{\pgfqpoint{3.453962in}{1.766836in}}%
\pgfpathlineto{\pgfqpoint{3.460758in}{1.760044in}}%
\pgfpathlineto{\pgfqpoint{3.467555in}{1.745548in}}%
\pgfpathlineto{\pgfqpoint{3.474352in}{1.739409in}}%
\pgfpathlineto{\pgfqpoint{3.481148in}{1.716353in}}%
\pgfpathlineto{\pgfqpoint{3.487945in}{1.720707in}}%
\pgfpathlineto{\pgfqpoint{3.501538in}{1.682430in}}%
\pgfpathlineto{\pgfqpoint{3.508334in}{1.684629in}}%
\pgfpathlineto{\pgfqpoint{3.515131in}{1.670397in}}%
\pgfpathlineto{\pgfqpoint{3.521927in}{1.646554in}}%
\pgfpathlineto{\pgfqpoint{3.528724in}{1.657757in}}%
\pgfpathlineto{\pgfqpoint{3.535520in}{1.617907in}}%
\pgfpathlineto{\pgfqpoint{3.542317in}{1.631676in}}%
\pgfpathlineto{\pgfqpoint{3.555910in}{1.591031in}}%
\pgfpathlineto{\pgfqpoint{3.562706in}{1.603260in}}%
\pgfpathlineto{\pgfqpoint{3.569503in}{1.560649in}}%
\pgfpathlineto{\pgfqpoint{3.576299in}{1.579839in}}%
\pgfpathlineto{\pgfqpoint{3.583096in}{1.543871in}}%
\pgfpathlineto{\pgfqpoint{3.589893in}{1.551882in}}%
\pgfpathlineto{\pgfqpoint{3.596689in}{1.528378in}}%
\pgfpathlineto{\pgfqpoint{3.603486in}{1.536044in}}%
\pgfpathlineto{\pgfqpoint{3.610282in}{1.501208in}}%
\pgfpathlineto{\pgfqpoint{3.617079in}{1.519179in}}%
\pgfpathlineto{\pgfqpoint{3.623875in}{1.479460in}}%
\pgfpathlineto{\pgfqpoint{3.630672in}{1.498040in}}%
\pgfpathlineto{\pgfqpoint{3.637468in}{1.466304in}}%
\pgfpathlineto{\pgfqpoint{3.644265in}{1.474491in}}%
\pgfpathlineto{\pgfqpoint{3.651061in}{1.455600in}}%
\pgfpathlineto{\pgfqpoint{3.657858in}{1.448990in}}%
\pgfpathlineto{\pgfqpoint{3.664654in}{1.444565in}}%
\pgfpathlineto{\pgfqpoint{3.671451in}{1.438093in}}%
\pgfpathlineto{\pgfqpoint{3.678247in}{1.421046in}}%
\pgfpathlineto{\pgfqpoint{3.685044in}{1.433979in}}%
\pgfpathlineto{\pgfqpoint{3.691840in}{1.400964in}}%
\pgfpathlineto{\pgfqpoint{3.698637in}{1.417545in}}%
\pgfpathlineto{\pgfqpoint{3.705434in}{1.399323in}}%
\pgfpathlineto{\pgfqpoint{3.712230in}{1.403988in}}%
\pgfpathlineto{\pgfqpoint{3.719027in}{1.394492in}}%
\pgfpathlineto{\pgfqpoint{3.725823in}{1.391272in}}%
\pgfpathlineto{\pgfqpoint{3.732620in}{1.383208in}}%
\pgfpathlineto{\pgfqpoint{3.739416in}{1.390581in}}%
\pgfpathlineto{\pgfqpoint{3.746213in}{1.368283in}}%
\pgfpathlineto{\pgfqpoint{3.753009in}{1.401023in}}%
\pgfpathlineto{\pgfqpoint{3.759806in}{1.362796in}}%
\pgfpathlineto{\pgfqpoint{3.766602in}{1.400608in}}%
\pgfpathlineto{\pgfqpoint{3.773399in}{1.361536in}}%
\pgfpathlineto{\pgfqpoint{3.780195in}{1.394274in}}%
\pgfpathlineto{\pgfqpoint{3.786992in}{1.370632in}}%
\pgfpathlineto{\pgfqpoint{3.793788in}{1.386531in}}%
\pgfpathlineto{\pgfqpoint{3.800585in}{1.390027in}}%
\pgfpathlineto{\pgfqpoint{3.807381in}{1.388734in}}%
\pgfpathlineto{\pgfqpoint{3.814178in}{1.398778in}}%
\pgfpathlineto{\pgfqpoint{3.820975in}{1.391968in}}%
\pgfpathlineto{\pgfqpoint{3.827771in}{1.416662in}}%
\pgfpathlineto{\pgfqpoint{3.834568in}{1.393619in}}%
\pgfpathlineto{\pgfqpoint{3.841364in}{1.423574in}}%
\pgfpathlineto{\pgfqpoint{3.848161in}{1.422899in}}%
\pgfpathlineto{\pgfqpoint{3.854957in}{1.432293in}}%
\pgfpathlineto{\pgfqpoint{3.861754in}{1.435777in}}%
\pgfpathlineto{\pgfqpoint{3.868550in}{1.455732in}}%
\pgfpathlineto{\pgfqpoint{3.875347in}{1.455581in}}%
\pgfpathlineto{\pgfqpoint{3.882143in}{1.469725in}}%
\pgfpathlineto{\pgfqpoint{3.888940in}{1.478017in}}%
\pgfpathlineto{\pgfqpoint{3.895736in}{1.493393in}}%
\pgfpathlineto{\pgfqpoint{3.902533in}{1.504734in}}%
\pgfpathlineto{\pgfqpoint{3.909329in}{1.521449in}}%
\pgfpathlineto{\pgfqpoint{3.943312in}{1.583810in}}%
\pgfpathlineto{\pgfqpoint{3.950109in}{1.600186in}}%
\pgfpathlineto{\pgfqpoint{3.956905in}{1.610652in}}%
\pgfpathlineto{\pgfqpoint{3.963702in}{1.629092in}}%
\pgfpathlineto{\pgfqpoint{3.970498in}{1.639660in}}%
\pgfpathlineto{\pgfqpoint{3.977295in}{1.659515in}}%
\pgfpathlineto{\pgfqpoint{3.984091in}{1.669407in}}%
\pgfpathlineto{\pgfqpoint{3.990888in}{1.687969in}}%
\pgfpathlineto{\pgfqpoint{3.997684in}{1.697119in}}%
\pgfpathlineto{\pgfqpoint{4.018074in}{1.730591in}}%
\pgfpathlineto{\pgfqpoint{4.024870in}{1.749649in}}%
\pgfpathlineto{\pgfqpoint{4.031667in}{1.761730in}}%
\pgfpathlineto{\pgfqpoint{4.038463in}{1.766226in}}%
\pgfpathlineto{\pgfqpoint{4.045260in}{1.786335in}}%
\pgfpathlineto{\pgfqpoint{4.052057in}{1.785367in}}%
\pgfpathlineto{\pgfqpoint{4.058853in}{1.810205in}}%
\pgfpathlineto{\pgfqpoint{4.065650in}{1.805955in}}%
\pgfpathlineto{\pgfqpoint{4.072446in}{1.828087in}}%
\pgfpathlineto{\pgfqpoint{4.079243in}{1.827926in}}%
\pgfpathlineto{\pgfqpoint{4.092836in}{1.844842in}}%
\pgfpathlineto{\pgfqpoint{4.099632in}{1.851286in}}%
\pgfpathlineto{\pgfqpoint{4.106429in}{1.860458in}}%
\pgfpathlineto{\pgfqpoint{4.113225in}{1.861890in}}%
\pgfpathlineto{\pgfqpoint{4.120022in}{1.871964in}}%
\pgfpathlineto{\pgfqpoint{4.126818in}{1.873011in}}%
\pgfpathlineto{\pgfqpoint{4.133615in}{1.883310in}}%
\pgfpathlineto{\pgfqpoint{4.140411in}{1.883832in}}%
\pgfpathlineto{\pgfqpoint{4.154004in}{1.896638in}}%
\pgfpathlineto{\pgfqpoint{4.160801in}{1.897971in}}%
\pgfpathlineto{\pgfqpoint{4.167598in}{1.901620in}}%
\pgfpathlineto{\pgfqpoint{4.174394in}{1.909036in}}%
\pgfpathlineto{\pgfqpoint{4.181191in}{1.907257in}}%
\pgfpathlineto{\pgfqpoint{4.187987in}{1.913181in}}%
\pgfpathlineto{\pgfqpoint{4.194784in}{1.912997in}}%
\pgfpathlineto{\pgfqpoint{4.201580in}{1.918259in}}%
\pgfpathlineto{\pgfqpoint{4.208377in}{1.918895in}}%
\pgfpathlineto{\pgfqpoint{4.221970in}{1.924378in}}%
\pgfpathlineto{\pgfqpoint{4.228766in}{1.924403in}}%
\pgfpathlineto{\pgfqpoint{4.235563in}{1.928332in}}%
\pgfpathlineto{\pgfqpoint{4.242359in}{1.928134in}}%
\pgfpathlineto{\pgfqpoint{4.249156in}{1.932198in}}%
\pgfpathlineto{\pgfqpoint{4.255952in}{1.931024in}}%
\pgfpathlineto{\pgfqpoint{4.262749in}{1.934417in}}%
\pgfpathlineto{\pgfqpoint{4.269545in}{1.933666in}}%
\pgfpathlineto{\pgfqpoint{4.276342in}{1.936255in}}%
\pgfpathlineto{\pgfqpoint{4.289935in}{1.937233in}}%
\pgfpathlineto{\pgfqpoint{4.296732in}{1.939179in}}%
\pgfpathlineto{\pgfqpoint{4.303528in}{1.937803in}}%
\pgfpathlineto{\pgfqpoint{4.310325in}{1.942353in}}%
\pgfpathlineto{\pgfqpoint{4.317121in}{1.939109in}}%
\pgfpathlineto{\pgfqpoint{4.323918in}{1.942156in}}%
\pgfpathlineto{\pgfqpoint{4.344307in}{1.942102in}}%
\pgfpathlineto{\pgfqpoint{4.351104in}{1.943470in}}%
\pgfpathlineto{\pgfqpoint{4.364697in}{1.943920in}}%
\pgfpathlineto{\pgfqpoint{4.405476in}{1.946999in}}%
\pgfpathlineto{\pgfqpoint{4.425866in}{1.946755in}}%
\pgfpathlineto{\pgfqpoint{4.432662in}{1.948079in}}%
\pgfpathlineto{\pgfqpoint{4.439459in}{1.947094in}}%
\pgfpathlineto{\pgfqpoint{4.446255in}{1.948652in}}%
\pgfpathlineto{\pgfqpoint{4.453052in}{1.947714in}}%
\pgfpathlineto{\pgfqpoint{4.473441in}{1.948401in}}%
\pgfpathlineto{\pgfqpoint{4.711320in}{1.949118in}}%
\pgfpathlineto{\pgfqpoint{5.071536in}{1.949120in}}%
\pgfpathlineto{\pgfqpoint{5.071536in}{1.949120in}}%
\pgfusepath{stroke}%
\end{pgfscope}%
\begin{pgfscope}%
\pgfpathrectangle{\pgfqpoint{0.750000in}{0.500000in}}{\pgfqpoint{4.650000in}{3.020000in}}%
\pgfusepath{clip}%
\pgfsetrectcap%
\pgfsetroundjoin%
\pgfsetlinewidth{1.505625pt}%
\definecolor{currentstroke}{rgb}{0.172549,0.627451,0.172549}%
\pgfsetstrokecolor{currentstroke}%
\pgfsetdash{}{0pt}%
\pgfpathmoveto{\pgfqpoint{1.117515in}{1.663923in}}%
\pgfpathlineto{\pgfqpoint{1.124312in}{1.659769in}}%
\pgfpathlineto{\pgfqpoint{1.131108in}{1.653474in}}%
\pgfpathlineto{\pgfqpoint{1.137905in}{1.652963in}}%
\pgfpathlineto{\pgfqpoint{1.165091in}{1.635186in}}%
\pgfpathlineto{\pgfqpoint{1.171888in}{1.631331in}}%
\pgfpathlineto{\pgfqpoint{1.178684in}{1.632465in}}%
\pgfpathlineto{\pgfqpoint{1.185481in}{1.628317in}}%
\pgfpathlineto{\pgfqpoint{1.192277in}{1.627996in}}%
\pgfpathlineto{\pgfqpoint{1.199074in}{1.630805in}}%
\pgfpathlineto{\pgfqpoint{1.205870in}{1.617876in}}%
\pgfpathlineto{\pgfqpoint{1.212667in}{1.629660in}}%
\pgfpathlineto{\pgfqpoint{1.219463in}{1.619243in}}%
\pgfpathlineto{\pgfqpoint{1.226260in}{1.623047in}}%
\pgfpathlineto{\pgfqpoint{1.239853in}{1.624802in}}%
\pgfpathlineto{\pgfqpoint{1.253446in}{1.630930in}}%
\pgfpathlineto{\pgfqpoint{1.260242in}{1.627900in}}%
\pgfpathlineto{\pgfqpoint{1.267039in}{1.640969in}}%
\pgfpathlineto{\pgfqpoint{1.273835in}{1.634940in}}%
\pgfpathlineto{\pgfqpoint{1.280632in}{1.641248in}}%
\pgfpathlineto{\pgfqpoint{1.287429in}{1.652379in}}%
\pgfpathlineto{\pgfqpoint{1.294225in}{1.645951in}}%
\pgfpathlineto{\pgfqpoint{1.301022in}{1.666275in}}%
\pgfpathlineto{\pgfqpoint{1.307818in}{1.664887in}}%
\pgfpathlineto{\pgfqpoint{1.314615in}{1.672295in}}%
\pgfpathlineto{\pgfqpoint{1.321411in}{1.683830in}}%
\pgfpathlineto{\pgfqpoint{1.335004in}{1.700106in}}%
\pgfpathlineto{\pgfqpoint{1.348597in}{1.721739in}}%
\pgfpathlineto{\pgfqpoint{1.355394in}{1.729988in}}%
\pgfpathlineto{\pgfqpoint{1.362190in}{1.744472in}}%
\pgfpathlineto{\pgfqpoint{1.368987in}{1.754392in}}%
\pgfpathlineto{\pgfqpoint{1.382580in}{1.785459in}}%
\pgfpathlineto{\pgfqpoint{1.389376in}{1.791743in}}%
\pgfpathlineto{\pgfqpoint{1.396173in}{1.808553in}}%
\pgfpathlineto{\pgfqpoint{1.409766in}{1.832892in}}%
\pgfpathlineto{\pgfqpoint{1.423359in}{1.869425in}}%
\pgfpathlineto{\pgfqpoint{1.430156in}{1.875351in}}%
\pgfpathlineto{\pgfqpoint{1.436952in}{1.894608in}}%
\pgfpathlineto{\pgfqpoint{1.443749in}{1.899675in}}%
\pgfpathlineto{\pgfqpoint{1.450545in}{1.927053in}}%
\pgfpathlineto{\pgfqpoint{1.457342in}{1.932822in}}%
\pgfpathlineto{\pgfqpoint{1.464138in}{1.950482in}}%
\pgfpathlineto{\pgfqpoint{1.477731in}{1.973746in}}%
\pgfpathlineto{\pgfqpoint{1.511714in}{2.041670in}}%
\pgfpathlineto{\pgfqpoint{1.518511in}{2.043689in}}%
\pgfpathlineto{\pgfqpoint{1.525307in}{2.064262in}}%
\pgfpathlineto{\pgfqpoint{1.532104in}{2.070375in}}%
\pgfpathlineto{\pgfqpoint{1.545697in}{2.097482in}}%
\pgfpathlineto{\pgfqpoint{1.552493in}{2.104612in}}%
\pgfpathlineto{\pgfqpoint{1.559290in}{2.114984in}}%
\pgfpathlineto{\pgfqpoint{1.572883in}{2.130948in}}%
\pgfpathlineto{\pgfqpoint{1.579679in}{2.145990in}}%
\pgfpathlineto{\pgfqpoint{1.586476in}{2.149623in}}%
\pgfpathlineto{\pgfqpoint{1.600069in}{2.165753in}}%
\pgfpathlineto{\pgfqpoint{1.606865in}{2.170946in}}%
\pgfpathlineto{\pgfqpoint{1.613662in}{2.183395in}}%
\pgfpathlineto{\pgfqpoint{1.620458in}{2.183294in}}%
\pgfpathlineto{\pgfqpoint{1.627255in}{2.193995in}}%
\pgfpathlineto{\pgfqpoint{1.634052in}{2.199315in}}%
\pgfpathlineto{\pgfqpoint{1.640848in}{2.209568in}}%
\pgfpathlineto{\pgfqpoint{1.647645in}{2.212510in}}%
\pgfpathlineto{\pgfqpoint{1.654441in}{2.217325in}}%
\pgfpathlineto{\pgfqpoint{1.661238in}{2.215431in}}%
\pgfpathlineto{\pgfqpoint{1.668034in}{2.226929in}}%
\pgfpathlineto{\pgfqpoint{1.674831in}{2.226100in}}%
\pgfpathlineto{\pgfqpoint{1.681627in}{2.235689in}}%
\pgfpathlineto{\pgfqpoint{1.695220in}{2.234361in}}%
\pgfpathlineto{\pgfqpoint{1.702017in}{2.242287in}}%
\pgfpathlineto{\pgfqpoint{1.708813in}{2.238703in}}%
\pgfpathlineto{\pgfqpoint{1.715610in}{2.246468in}}%
\pgfpathlineto{\pgfqpoint{1.729203in}{2.244019in}}%
\pgfpathlineto{\pgfqpoint{1.735999in}{2.246374in}}%
\pgfpathlineto{\pgfqpoint{1.742796in}{2.250495in}}%
\pgfpathlineto{\pgfqpoint{1.749593in}{2.245330in}}%
\pgfpathlineto{\pgfqpoint{1.756389in}{2.248025in}}%
\pgfpathlineto{\pgfqpoint{1.763186in}{2.247180in}}%
\pgfpathlineto{\pgfqpoint{1.769982in}{2.248968in}}%
\pgfpathlineto{\pgfqpoint{1.783575in}{2.244085in}}%
\pgfpathlineto{\pgfqpoint{1.790372in}{2.245641in}}%
\pgfpathlineto{\pgfqpoint{1.797168in}{2.243145in}}%
\pgfpathlineto{\pgfqpoint{1.803965in}{2.243645in}}%
\pgfpathlineto{\pgfqpoint{1.817558in}{2.231801in}}%
\pgfpathlineto{\pgfqpoint{1.824354in}{2.233863in}}%
\pgfpathlineto{\pgfqpoint{1.837947in}{2.231481in}}%
\pgfpathlineto{\pgfqpoint{1.844744in}{2.224417in}}%
\pgfpathlineto{\pgfqpoint{1.851540in}{2.223338in}}%
\pgfpathlineto{\pgfqpoint{1.858337in}{2.220023in}}%
\pgfpathlineto{\pgfqpoint{1.871930in}{2.208804in}}%
\pgfpathlineto{\pgfqpoint{1.885523in}{2.201510in}}%
\pgfpathlineto{\pgfqpoint{1.892320in}{2.196720in}}%
\pgfpathlineto{\pgfqpoint{1.899116in}{2.196170in}}%
\pgfpathlineto{\pgfqpoint{1.905913in}{2.183793in}}%
\pgfpathlineto{\pgfqpoint{1.912709in}{2.188198in}}%
\pgfpathlineto{\pgfqpoint{1.919506in}{2.175942in}}%
\pgfpathlineto{\pgfqpoint{1.933099in}{2.170297in}}%
\pgfpathlineto{\pgfqpoint{1.939895in}{2.158445in}}%
\pgfpathlineto{\pgfqpoint{1.946692in}{2.161439in}}%
\pgfpathlineto{\pgfqpoint{1.960285in}{2.141975in}}%
\pgfpathlineto{\pgfqpoint{1.967081in}{2.141833in}}%
\pgfpathlineto{\pgfqpoint{1.973878in}{2.132728in}}%
\pgfpathlineto{\pgfqpoint{1.980675in}{2.127288in}}%
\pgfpathlineto{\pgfqpoint{1.987471in}{2.118734in}}%
\pgfpathlineto{\pgfqpoint{1.994268in}{2.115658in}}%
\pgfpathlineto{\pgfqpoint{2.001064in}{2.101427in}}%
\pgfpathlineto{\pgfqpoint{2.007861in}{2.099408in}}%
\pgfpathlineto{\pgfqpoint{2.014657in}{2.094710in}}%
\pgfpathlineto{\pgfqpoint{2.021454in}{2.083940in}}%
\pgfpathlineto{\pgfqpoint{2.028250in}{2.084471in}}%
\pgfpathlineto{\pgfqpoint{2.035047in}{2.067285in}}%
\pgfpathlineto{\pgfqpoint{2.041843in}{2.070513in}}%
\pgfpathlineto{\pgfqpoint{2.048640in}{2.053125in}}%
\pgfpathlineto{\pgfqpoint{2.055436in}{2.053259in}}%
\pgfpathlineto{\pgfqpoint{2.062233in}{2.037998in}}%
\pgfpathlineto{\pgfqpoint{2.069029in}{2.037400in}}%
\pgfpathlineto{\pgfqpoint{2.075826in}{2.031715in}}%
\pgfpathlineto{\pgfqpoint{2.082622in}{2.018832in}}%
\pgfpathlineto{\pgfqpoint{2.089419in}{2.013504in}}%
\pgfpathlineto{\pgfqpoint{2.096216in}{2.005471in}}%
\pgfpathlineto{\pgfqpoint{2.103012in}{1.991620in}}%
\pgfpathlineto{\pgfqpoint{2.109809in}{1.993403in}}%
\pgfpathlineto{\pgfqpoint{2.116605in}{1.972742in}}%
\pgfpathlineto{\pgfqpoint{2.123402in}{1.977437in}}%
\pgfpathlineto{\pgfqpoint{2.130198in}{1.963350in}}%
\pgfpathlineto{\pgfqpoint{2.136995in}{1.959593in}}%
\pgfpathlineto{\pgfqpoint{2.143791in}{1.954059in}}%
\pgfpathlineto{\pgfqpoint{2.150588in}{1.940630in}}%
\pgfpathlineto{\pgfqpoint{2.157384in}{1.936499in}}%
\pgfpathlineto{\pgfqpoint{2.164181in}{1.922975in}}%
\pgfpathlineto{\pgfqpoint{2.170977in}{1.922902in}}%
\pgfpathlineto{\pgfqpoint{2.177774in}{1.904678in}}%
\pgfpathlineto{\pgfqpoint{2.184570in}{1.903684in}}%
\pgfpathlineto{\pgfqpoint{2.191367in}{1.892251in}}%
\pgfpathlineto{\pgfqpoint{2.198163in}{1.883979in}}%
\pgfpathlineto{\pgfqpoint{2.204960in}{1.881508in}}%
\pgfpathlineto{\pgfqpoint{2.211757in}{1.868666in}}%
\pgfpathlineto{\pgfqpoint{2.218553in}{1.864456in}}%
\pgfpathlineto{\pgfqpoint{2.225350in}{1.854664in}}%
\pgfpathlineto{\pgfqpoint{2.232146in}{1.840144in}}%
\pgfpathlineto{\pgfqpoint{2.238943in}{1.848452in}}%
\pgfpathlineto{\pgfqpoint{2.245739in}{1.823536in}}%
\pgfpathlineto{\pgfqpoint{2.252536in}{1.827729in}}%
\pgfpathlineto{\pgfqpoint{2.259332in}{1.812951in}}%
\pgfpathlineto{\pgfqpoint{2.266129in}{1.811220in}}%
\pgfpathlineto{\pgfqpoint{2.272925in}{1.796195in}}%
\pgfpathlineto{\pgfqpoint{2.279722in}{1.796296in}}%
\pgfpathlineto{\pgfqpoint{2.286518in}{1.781052in}}%
\pgfpathlineto{\pgfqpoint{2.293315in}{1.778352in}}%
\pgfpathlineto{\pgfqpoint{2.300111in}{1.771792in}}%
\pgfpathlineto{\pgfqpoint{2.306908in}{1.760667in}}%
\pgfpathlineto{\pgfqpoint{2.313704in}{1.761100in}}%
\pgfpathlineto{\pgfqpoint{2.320501in}{1.744001in}}%
\pgfpathlineto{\pgfqpoint{2.327298in}{1.744390in}}%
\pgfpathlineto{\pgfqpoint{2.334094in}{1.728829in}}%
\pgfpathlineto{\pgfqpoint{2.340891in}{1.735790in}}%
\pgfpathlineto{\pgfqpoint{2.347687in}{1.717485in}}%
\pgfpathlineto{\pgfqpoint{2.354484in}{1.717128in}}%
\pgfpathlineto{\pgfqpoint{2.361280in}{1.714170in}}%
\pgfpathlineto{\pgfqpoint{2.368077in}{1.698816in}}%
\pgfpathlineto{\pgfqpoint{2.381670in}{1.691712in}}%
\pgfpathlineto{\pgfqpoint{2.395263in}{1.679691in}}%
\pgfpathlineto{\pgfqpoint{2.402059in}{1.679125in}}%
\pgfpathlineto{\pgfqpoint{2.408856in}{1.663484in}}%
\pgfpathlineto{\pgfqpoint{2.415652in}{1.667468in}}%
\pgfpathlineto{\pgfqpoint{2.429245in}{1.654760in}}%
\pgfpathlineto{\pgfqpoint{2.442839in}{1.645568in}}%
\pgfpathlineto{\pgfqpoint{2.449635in}{1.645424in}}%
\pgfpathlineto{\pgfqpoint{2.456432in}{1.636020in}}%
\pgfpathlineto{\pgfqpoint{2.463228in}{1.635767in}}%
\pgfpathlineto{\pgfqpoint{2.476821in}{1.630526in}}%
\pgfpathlineto{\pgfqpoint{2.490414in}{1.622077in}}%
\pgfpathlineto{\pgfqpoint{2.497211in}{1.625453in}}%
\pgfpathlineto{\pgfqpoint{2.504007in}{1.619658in}}%
\pgfpathlineto{\pgfqpoint{2.510804in}{1.627961in}}%
\pgfpathlineto{\pgfqpoint{2.517600in}{1.612887in}}%
\pgfpathlineto{\pgfqpoint{2.524397in}{1.624198in}}%
\pgfpathlineto{\pgfqpoint{2.531193in}{1.616371in}}%
\pgfpathlineto{\pgfqpoint{2.537990in}{1.622647in}}%
\pgfpathlineto{\pgfqpoint{2.544786in}{1.616463in}}%
\pgfpathlineto{\pgfqpoint{2.551583in}{1.626635in}}%
\pgfpathlineto{\pgfqpoint{2.558380in}{1.624768in}}%
\pgfpathlineto{\pgfqpoint{2.565176in}{1.624771in}}%
\pgfpathlineto{\pgfqpoint{2.571973in}{1.631957in}}%
\pgfpathlineto{\pgfqpoint{2.578769in}{1.628342in}}%
\pgfpathlineto{\pgfqpoint{2.585566in}{1.643760in}}%
\pgfpathlineto{\pgfqpoint{2.592362in}{1.636430in}}%
\pgfpathlineto{\pgfqpoint{2.599159in}{1.652871in}}%
\pgfpathlineto{\pgfqpoint{2.605955in}{1.654488in}}%
\pgfpathlineto{\pgfqpoint{2.612752in}{1.650495in}}%
\pgfpathlineto{\pgfqpoint{2.619548in}{1.674127in}}%
\pgfpathlineto{\pgfqpoint{2.626345in}{1.672027in}}%
\pgfpathlineto{\pgfqpoint{2.633141in}{1.688974in}}%
\pgfpathlineto{\pgfqpoint{2.639938in}{1.693477in}}%
\pgfpathlineto{\pgfqpoint{2.653531in}{1.709340in}}%
\pgfpathlineto{\pgfqpoint{2.660327in}{1.728021in}}%
\pgfpathlineto{\pgfqpoint{2.667124in}{1.729931in}}%
\pgfpathlineto{\pgfqpoint{2.673921in}{1.750217in}}%
\pgfpathlineto{\pgfqpoint{2.680717in}{1.762176in}}%
\pgfpathlineto{\pgfqpoint{2.687514in}{1.767346in}}%
\pgfpathlineto{\pgfqpoint{2.694310in}{1.795543in}}%
\pgfpathlineto{\pgfqpoint{2.701107in}{1.795000in}}%
\pgfpathlineto{\pgfqpoint{2.707903in}{1.818558in}}%
\pgfpathlineto{\pgfqpoint{2.714700in}{1.826275in}}%
\pgfpathlineto{\pgfqpoint{2.728293in}{1.857969in}}%
\pgfpathlineto{\pgfqpoint{2.755479in}{1.915777in}}%
\pgfpathlineto{\pgfqpoint{2.769072in}{1.947107in}}%
\pgfpathlineto{\pgfqpoint{2.775868in}{1.956146in}}%
\pgfpathlineto{\pgfqpoint{2.782665in}{1.982161in}}%
\pgfpathlineto{\pgfqpoint{2.789462in}{1.980760in}}%
\pgfpathlineto{\pgfqpoint{2.796258in}{2.011732in}}%
\pgfpathlineto{\pgfqpoint{2.803055in}{2.010730in}}%
\pgfpathlineto{\pgfqpoint{2.816648in}{2.044659in}}%
\pgfpathlineto{\pgfqpoint{2.830241in}{2.058062in}}%
\pgfpathlineto{\pgfqpoint{2.837037in}{2.086711in}}%
\pgfpathlineto{\pgfqpoint{2.843834in}{2.080103in}}%
\pgfpathlineto{\pgfqpoint{2.850630in}{2.108951in}}%
\pgfpathlineto{\pgfqpoint{2.857427in}{2.104683in}}%
\pgfpathlineto{\pgfqpoint{2.864223in}{2.121096in}}%
\pgfpathlineto{\pgfqpoint{2.871020in}{2.133407in}}%
\pgfpathlineto{\pgfqpoint{2.877816in}{2.132904in}}%
\pgfpathlineto{\pgfqpoint{2.884613in}{2.156703in}}%
\pgfpathlineto{\pgfqpoint{2.891409in}{2.148254in}}%
\pgfpathlineto{\pgfqpoint{2.898206in}{2.173897in}}%
\pgfpathlineto{\pgfqpoint{2.905003in}{2.167281in}}%
\pgfpathlineto{\pgfqpoint{2.911799in}{2.185518in}}%
\pgfpathlineto{\pgfqpoint{2.918596in}{2.182528in}}%
\pgfpathlineto{\pgfqpoint{2.925392in}{2.196342in}}%
\pgfpathlineto{\pgfqpoint{2.932189in}{2.197373in}}%
\pgfpathlineto{\pgfqpoint{2.952578in}{2.216846in}}%
\pgfpathlineto{\pgfqpoint{2.959375in}{2.220114in}}%
\pgfpathlineto{\pgfqpoint{2.966171in}{2.230193in}}%
\pgfpathlineto{\pgfqpoint{2.972968in}{2.221224in}}%
\pgfpathlineto{\pgfqpoint{2.979764in}{2.240019in}}%
\pgfpathlineto{\pgfqpoint{2.986561in}{2.231435in}}%
\pgfpathlineto{\pgfqpoint{2.993357in}{2.239096in}}%
\pgfpathlineto{\pgfqpoint{3.000154in}{2.241196in}}%
\pgfpathlineto{\pgfqpoint{3.006950in}{2.240056in}}%
\pgfpathlineto{\pgfqpoint{3.020544in}{2.247988in}}%
\pgfpathlineto{\pgfqpoint{3.027340in}{2.245716in}}%
\pgfpathlineto{\pgfqpoint{3.034137in}{2.248333in}}%
\pgfpathlineto{\pgfqpoint{3.040933in}{2.247452in}}%
\pgfpathlineto{\pgfqpoint{3.047730in}{2.250861in}}%
\pgfpathlineto{\pgfqpoint{3.061323in}{2.247883in}}%
\pgfpathlineto{\pgfqpoint{3.068119in}{2.248423in}}%
\pgfpathlineto{\pgfqpoint{3.074916in}{2.244876in}}%
\pgfpathlineto{\pgfqpoint{3.088509in}{2.245162in}}%
\pgfpathlineto{\pgfqpoint{3.095305in}{2.241468in}}%
\pgfpathlineto{\pgfqpoint{3.108898in}{2.240661in}}%
\pgfpathlineto{\pgfqpoint{3.115695in}{2.236828in}}%
\pgfpathlineto{\pgfqpoint{3.122491in}{2.229100in}}%
\pgfpathlineto{\pgfqpoint{3.129288in}{2.237146in}}%
\pgfpathlineto{\pgfqpoint{3.136084in}{2.223469in}}%
\pgfpathlineto{\pgfqpoint{3.149678in}{2.226162in}}%
\pgfpathlineto{\pgfqpoint{3.156474in}{2.209943in}}%
\pgfpathlineto{\pgfqpoint{3.163271in}{2.222297in}}%
\pgfpathlineto{\pgfqpoint{3.170067in}{2.199334in}}%
\pgfpathlineto{\pgfqpoint{3.176864in}{2.213566in}}%
\pgfpathlineto{\pgfqpoint{3.183660in}{2.200370in}}%
\pgfpathlineto{\pgfqpoint{3.197253in}{2.190730in}}%
\pgfpathlineto{\pgfqpoint{3.204050in}{2.190645in}}%
\pgfpathlineto{\pgfqpoint{3.210846in}{2.179073in}}%
\pgfpathlineto{\pgfqpoint{3.217643in}{2.180224in}}%
\pgfpathlineto{\pgfqpoint{3.224439in}{2.168226in}}%
\pgfpathlineto{\pgfqpoint{3.231236in}{2.169947in}}%
\pgfpathlineto{\pgfqpoint{3.238032in}{2.156468in}}%
\pgfpathlineto{\pgfqpoint{3.244829in}{2.161559in}}%
\pgfpathlineto{\pgfqpoint{3.251625in}{2.140944in}}%
\pgfpathlineto{\pgfqpoint{3.258422in}{2.147190in}}%
\pgfpathlineto{\pgfqpoint{3.272015in}{2.127341in}}%
\pgfpathlineto{\pgfqpoint{3.278812in}{2.131563in}}%
\pgfpathlineto{\pgfqpoint{3.285608in}{2.111952in}}%
\pgfpathlineto{\pgfqpoint{3.292405in}{2.115843in}}%
\pgfpathlineto{\pgfqpoint{3.299201in}{2.103299in}}%
\pgfpathlineto{\pgfqpoint{3.305998in}{2.099425in}}%
\pgfpathlineto{\pgfqpoint{3.319591in}{2.085711in}}%
\pgfpathlineto{\pgfqpoint{3.326387in}{2.076309in}}%
\pgfpathlineto{\pgfqpoint{3.333184in}{2.073782in}}%
\pgfpathlineto{\pgfqpoint{3.339980in}{2.059827in}}%
\pgfpathlineto{\pgfqpoint{3.346777in}{2.058633in}}%
\pgfpathlineto{\pgfqpoint{3.353573in}{2.054490in}}%
\pgfpathlineto{\pgfqpoint{3.360370in}{2.039142in}}%
\pgfpathlineto{\pgfqpoint{3.367166in}{2.040670in}}%
\pgfpathlineto{\pgfqpoint{3.380760in}{2.016151in}}%
\pgfpathlineto{\pgfqpoint{3.387556in}{2.020876in}}%
\pgfpathlineto{\pgfqpoint{3.394353in}{1.999020in}}%
\pgfpathlineto{\pgfqpoint{3.401149in}{2.000560in}}%
\pgfpathlineto{\pgfqpoint{3.407946in}{1.993689in}}%
\pgfpathlineto{\pgfqpoint{3.414742in}{1.981429in}}%
\pgfpathlineto{\pgfqpoint{3.421539in}{1.982727in}}%
\pgfpathlineto{\pgfqpoint{3.428335in}{1.962996in}}%
\pgfpathlineto{\pgfqpoint{3.435132in}{1.968893in}}%
\pgfpathlineto{\pgfqpoint{3.441928in}{1.952156in}}%
\pgfpathlineto{\pgfqpoint{3.448725in}{1.953024in}}%
\pgfpathlineto{\pgfqpoint{3.455521in}{1.931875in}}%
\pgfpathlineto{\pgfqpoint{3.462318in}{1.942092in}}%
\pgfpathlineto{\pgfqpoint{3.469114in}{1.920543in}}%
\pgfpathlineto{\pgfqpoint{3.475911in}{1.922498in}}%
\pgfpathlineto{\pgfqpoint{3.482707in}{1.916168in}}%
\pgfpathlineto{\pgfqpoint{3.489504in}{1.903093in}}%
\pgfpathlineto{\pgfqpoint{3.496301in}{1.906848in}}%
\pgfpathlineto{\pgfqpoint{3.503097in}{1.890555in}}%
\pgfpathlineto{\pgfqpoint{3.509894in}{1.886867in}}%
\pgfpathlineto{\pgfqpoint{3.516690in}{1.885380in}}%
\pgfpathlineto{\pgfqpoint{3.523487in}{1.872422in}}%
\pgfpathlineto{\pgfqpoint{3.530283in}{1.871601in}}%
\pgfpathlineto{\pgfqpoint{3.537080in}{1.860565in}}%
\pgfpathlineto{\pgfqpoint{3.543876in}{1.862461in}}%
\pgfpathlineto{\pgfqpoint{3.550673in}{1.844972in}}%
\pgfpathlineto{\pgfqpoint{3.557469in}{1.857372in}}%
\pgfpathlineto{\pgfqpoint{3.564266in}{1.826949in}}%
\pgfpathlineto{\pgfqpoint{3.571062in}{1.849054in}}%
\pgfpathlineto{\pgfqpoint{3.577859in}{1.823014in}}%
\pgfpathlineto{\pgfqpoint{3.584655in}{1.832334in}}%
\pgfpathlineto{\pgfqpoint{3.591452in}{1.815981in}}%
\pgfpathlineto{\pgfqpoint{3.598248in}{1.823405in}}%
\pgfpathlineto{\pgfqpoint{3.605045in}{1.807719in}}%
\pgfpathlineto{\pgfqpoint{3.611842in}{1.816472in}}%
\pgfpathlineto{\pgfqpoint{3.618638in}{1.799203in}}%
\pgfpathlineto{\pgfqpoint{3.625435in}{1.804839in}}%
\pgfpathlineto{\pgfqpoint{3.632231in}{1.797747in}}%
\pgfpathlineto{\pgfqpoint{3.639028in}{1.798855in}}%
\pgfpathlineto{\pgfqpoint{3.645824in}{1.787162in}}%
\pgfpathlineto{\pgfqpoint{3.652621in}{1.800045in}}%
\pgfpathlineto{\pgfqpoint{3.659417in}{1.780544in}}%
\pgfpathlineto{\pgfqpoint{3.666214in}{1.791085in}}%
\pgfpathlineto{\pgfqpoint{3.679807in}{1.776771in}}%
\pgfpathlineto{\pgfqpoint{3.686603in}{1.782154in}}%
\pgfpathlineto{\pgfqpoint{3.693400in}{1.780458in}}%
\pgfpathlineto{\pgfqpoint{3.700196in}{1.774844in}}%
\pgfpathlineto{\pgfqpoint{3.706993in}{1.777893in}}%
\pgfpathlineto{\pgfqpoint{3.713789in}{1.772670in}}%
\pgfpathlineto{\pgfqpoint{3.720586in}{1.780450in}}%
\pgfpathlineto{\pgfqpoint{3.727383in}{1.773941in}}%
\pgfpathlineto{\pgfqpoint{3.740976in}{1.776278in}}%
\pgfpathlineto{\pgfqpoint{3.754569in}{1.774920in}}%
\pgfpathlineto{\pgfqpoint{3.761365in}{1.782586in}}%
\pgfpathlineto{\pgfqpoint{3.768162in}{1.775349in}}%
\pgfpathlineto{\pgfqpoint{3.774958in}{1.785428in}}%
\pgfpathlineto{\pgfqpoint{3.781755in}{1.780949in}}%
\pgfpathlineto{\pgfqpoint{3.788551in}{1.786402in}}%
\pgfpathlineto{\pgfqpoint{3.802144in}{1.789590in}}%
\pgfpathlineto{\pgfqpoint{3.808941in}{1.787695in}}%
\pgfpathlineto{\pgfqpoint{3.815737in}{1.801463in}}%
\pgfpathlineto{\pgfqpoint{3.822534in}{1.790223in}}%
\pgfpathlineto{\pgfqpoint{3.829330in}{1.803118in}}%
\pgfpathlineto{\pgfqpoint{3.842924in}{1.805522in}}%
\pgfpathlineto{\pgfqpoint{3.849720in}{1.806487in}}%
\pgfpathlineto{\pgfqpoint{3.856517in}{1.814668in}}%
\pgfpathlineto{\pgfqpoint{3.863313in}{1.813819in}}%
\pgfpathlineto{\pgfqpoint{3.876906in}{1.824737in}}%
\pgfpathlineto{\pgfqpoint{3.883703in}{1.824045in}}%
\pgfpathlineto{\pgfqpoint{3.890499in}{1.829706in}}%
\pgfpathlineto{\pgfqpoint{3.897296in}{1.827845in}}%
\pgfpathlineto{\pgfqpoint{3.904092in}{1.840121in}}%
\pgfpathlineto{\pgfqpoint{3.910889in}{1.833356in}}%
\pgfpathlineto{\pgfqpoint{3.917685in}{1.846570in}}%
\pgfpathlineto{\pgfqpoint{3.924482in}{1.844360in}}%
\pgfpathlineto{\pgfqpoint{3.931278in}{1.852115in}}%
\pgfpathlineto{\pgfqpoint{3.938075in}{1.856510in}}%
\pgfpathlineto{\pgfqpoint{3.944871in}{1.852372in}}%
\pgfpathlineto{\pgfqpoint{3.951668in}{1.865033in}}%
\pgfpathlineto{\pgfqpoint{3.958465in}{1.860317in}}%
\pgfpathlineto{\pgfqpoint{3.965261in}{1.867028in}}%
\pgfpathlineto{\pgfqpoint{3.978854in}{1.871095in}}%
\pgfpathlineto{\pgfqpoint{3.985651in}{1.877374in}}%
\pgfpathlineto{\pgfqpoint{3.992447in}{1.877854in}}%
\pgfpathlineto{\pgfqpoint{3.999244in}{1.884603in}}%
\pgfpathlineto{\pgfqpoint{4.006040in}{1.883790in}}%
\pgfpathlineto{\pgfqpoint{4.012837in}{1.886366in}}%
\pgfpathlineto{\pgfqpoint{4.019633in}{1.891274in}}%
\pgfpathlineto{\pgfqpoint{4.026430in}{1.891502in}}%
\pgfpathlineto{\pgfqpoint{4.033226in}{1.893792in}}%
\pgfpathlineto{\pgfqpoint{4.040023in}{1.901042in}}%
\pgfpathlineto{\pgfqpoint{4.046819in}{1.898158in}}%
\pgfpathlineto{\pgfqpoint{4.053616in}{1.903933in}}%
\pgfpathlineto{\pgfqpoint{4.060412in}{1.904353in}}%
\pgfpathlineto{\pgfqpoint{4.094395in}{1.916315in}}%
\pgfpathlineto{\pgfqpoint{4.101192in}{1.914939in}}%
\pgfpathlineto{\pgfqpoint{4.107988in}{1.920157in}}%
\pgfpathlineto{\pgfqpoint{4.114785in}{1.918579in}}%
\pgfpathlineto{\pgfqpoint{4.128378in}{1.925083in}}%
\pgfpathlineto{\pgfqpoint{4.135174in}{1.923675in}}%
\pgfpathlineto{\pgfqpoint{4.141971in}{1.927442in}}%
\pgfpathlineto{\pgfqpoint{4.169157in}{1.929710in}}%
\pgfpathlineto{\pgfqpoint{4.175953in}{1.934009in}}%
\pgfpathlineto{\pgfqpoint{4.182750in}{1.931295in}}%
\pgfpathlineto{\pgfqpoint{4.189547in}{1.936025in}}%
\pgfpathlineto{\pgfqpoint{4.196343in}{1.934047in}}%
\pgfpathlineto{\pgfqpoint{4.203140in}{1.936490in}}%
\pgfpathlineto{\pgfqpoint{4.223529in}{1.937303in}}%
\pgfpathlineto{\pgfqpoint{4.230326in}{1.939604in}}%
\pgfpathlineto{\pgfqpoint{4.237122in}{1.937457in}}%
\pgfpathlineto{\pgfqpoint{4.243919in}{1.941723in}}%
\pgfpathlineto{\pgfqpoint{4.250715in}{1.939587in}}%
\pgfpathlineto{\pgfqpoint{4.257512in}{1.941790in}}%
\pgfpathlineto{\pgfqpoint{4.468204in}{1.948556in}}%
\pgfpathlineto{\pgfqpoint{4.706083in}{1.949112in}}%
\pgfpathlineto{\pgfqpoint{5.188636in}{1.949121in}}%
\pgfpathlineto{\pgfqpoint{5.188636in}{1.949121in}}%
\pgfusepath{stroke}%
\end{pgfscope}%
\begin{pgfscope}%
\pgfsetrectcap%
\pgfsetmiterjoin%
\pgfsetlinewidth{0.803000pt}%
\definecolor{currentstroke}{rgb}{0.000000,0.000000,0.000000}%
\pgfsetstrokecolor{currentstroke}%
\pgfsetdash{}{0pt}%
\pgfpathmoveto{\pgfqpoint{0.750000in}{0.500000in}}%
\pgfpathlineto{\pgfqpoint{0.750000in}{3.520000in}}%
\pgfusepath{stroke}%
\end{pgfscope}%
\begin{pgfscope}%
\pgfsetrectcap%
\pgfsetmiterjoin%
\pgfsetlinewidth{0.803000pt}%
\definecolor{currentstroke}{rgb}{0.000000,0.000000,0.000000}%
\pgfsetstrokecolor{currentstroke}%
\pgfsetdash{}{0pt}%
\pgfpathmoveto{\pgfqpoint{5.400000in}{0.500000in}}%
\pgfpathlineto{\pgfqpoint{5.400000in}{3.520000in}}%
\pgfusepath{stroke}%
\end{pgfscope}%
\begin{pgfscope}%
\pgfsetrectcap%
\pgfsetmiterjoin%
\pgfsetlinewidth{0.803000pt}%
\definecolor{currentstroke}{rgb}{0.000000,0.000000,0.000000}%
\pgfsetstrokecolor{currentstroke}%
\pgfsetdash{}{0pt}%
\pgfpathmoveto{\pgfqpoint{0.750000in}{0.500000in}}%
\pgfpathlineto{\pgfqpoint{5.400000in}{0.500000in}}%
\pgfusepath{stroke}%
\end{pgfscope}%
\begin{pgfscope}%
\pgfsetrectcap%
\pgfsetmiterjoin%
\pgfsetlinewidth{0.803000pt}%
\definecolor{currentstroke}{rgb}{0.000000,0.000000,0.000000}%
\pgfsetstrokecolor{currentstroke}%
\pgfsetdash{}{0pt}%
\pgfpathmoveto{\pgfqpoint{0.750000in}{3.520000in}}%
\pgfpathlineto{\pgfqpoint{5.400000in}{3.520000in}}%
\pgfusepath{stroke}%
\end{pgfscope}%
\begin{pgfscope}%
\pgfsetbuttcap%
\pgfsetmiterjoin%
\definecolor{currentfill}{rgb}{1.000000,1.000000,1.000000}%
\pgfsetfillcolor{currentfill}%
\pgfsetfillopacity{0.800000}%
\pgfsetlinewidth{1.003750pt}%
\definecolor{currentstroke}{rgb}{0.800000,0.800000,0.800000}%
\pgfsetstrokecolor{currentstroke}%
\pgfsetstrokeopacity{0.800000}%
\pgfsetdash{}{0pt}%
\pgfpathmoveto{\pgfqpoint{4.340971in}{2.827871in}}%
\pgfpathlineto{\pgfqpoint{5.302778in}{2.827871in}}%
\pgfpathquadraticcurveto{\pgfqpoint{5.330556in}{2.827871in}}{\pgfqpoint{5.330556in}{2.855648in}}%
\pgfpathlineto{\pgfqpoint{5.330556in}{3.422778in}}%
\pgfpathquadraticcurveto{\pgfqpoint{5.330556in}{3.450556in}}{\pgfqpoint{5.302778in}{3.450556in}}%
\pgfpathlineto{\pgfqpoint{4.340971in}{3.450556in}}%
\pgfpathquadraticcurveto{\pgfqpoint{4.313194in}{3.450556in}}{\pgfqpoint{4.313194in}{3.422778in}}%
\pgfpathlineto{\pgfqpoint{4.313194in}{2.855648in}}%
\pgfpathquadraticcurveto{\pgfqpoint{4.313194in}{2.827871in}}{\pgfqpoint{4.340971in}{2.827871in}}%
\pgfpathlineto{\pgfqpoint{4.340971in}{2.827871in}}%
\pgfpathclose%
\pgfusepath{stroke,fill}%
\end{pgfscope}%
\begin{pgfscope}%
\pgfsetrectcap%
\pgfsetroundjoin%
\pgfsetlinewidth{1.505625pt}%
\definecolor{currentstroke}{rgb}{0.121569,0.466667,0.705882}%
\pgfsetstrokecolor{currentstroke}%
\pgfsetdash{}{0pt}%
\pgfpathmoveto{\pgfqpoint{4.368749in}{3.346389in}}%
\pgfpathlineto{\pgfqpoint{4.507638in}{3.346389in}}%
\pgfpathlineto{\pgfqpoint{4.646527in}{3.346389in}}%
\pgfusepath{stroke}%
\end{pgfscope}%
\begin{pgfscope}%
\definecolor{textcolor}{rgb}{0.000000,0.000000,0.000000}%
\pgfsetstrokecolor{textcolor}%
\pgfsetfillcolor{textcolor}%
\pgftext[x=4.757638in,y=3.297778in,left,base]{\color{textcolor}\sffamily\fontsize{10.000000}{12.000000}\selectfont \(\displaystyle 1.7 \, \mathrm{mrad}\)}%
\end{pgfscope}%
\begin{pgfscope}%
\pgfsetrectcap%
\pgfsetroundjoin%
\pgfsetlinewidth{1.505625pt}%
\definecolor{currentstroke}{rgb}{1.000000,0.498039,0.054902}%
\pgfsetstrokecolor{currentstroke}%
\pgfsetdash{}{0pt}%
\pgfpathmoveto{\pgfqpoint{4.368749in}{3.152716in}}%
\pgfpathlineto{\pgfqpoint{4.507638in}{3.152716in}}%
\pgfpathlineto{\pgfqpoint{4.646527in}{3.152716in}}%
\pgfusepath{stroke}%
\end{pgfscope}%
\begin{pgfscope}%
\definecolor{textcolor}{rgb}{0.000000,0.000000,0.000000}%
\pgfsetstrokecolor{textcolor}%
\pgfsetfillcolor{textcolor}%
\pgftext[x=4.757638in,y=3.104105in,left,base]{\color{textcolor}\sffamily\fontsize{10.000000}{12.000000}\selectfont \(\displaystyle 4.2 \, \mathrm{mrad}\)}%
\end{pgfscope}%
\begin{pgfscope}%
\pgfsetrectcap%
\pgfsetroundjoin%
\pgfsetlinewidth{1.505625pt}%
\definecolor{currentstroke}{rgb}{0.172549,0.627451,0.172549}%
\pgfsetstrokecolor{currentstroke}%
\pgfsetdash{}{0pt}%
\pgfpathmoveto{\pgfqpoint{4.368749in}{2.959043in}}%
\pgfpathlineto{\pgfqpoint{4.507638in}{2.959043in}}%
\pgfpathlineto{\pgfqpoint{4.646527in}{2.959043in}}%
\pgfusepath{stroke}%
\end{pgfscope}%
\begin{pgfscope}%
\definecolor{textcolor}{rgb}{0.000000,0.000000,0.000000}%
\pgfsetstrokecolor{textcolor}%
\pgfsetfillcolor{textcolor}%
\pgftext[x=4.757638in,y=2.910432in,left,base]{\color{textcolor}\sffamily\fontsize{10.000000}{12.000000}\selectfont \(\displaystyle 8.7 \, \mathrm{mrad}\)}%
\end{pgfscope}%
\end{pgfpicture}%
\makeatother%
\endgroup%

	\caption{Witness energy gain for three different initial driver divergences.}
	\label{fig:gain_div}
\end{figure}
For all curves at least one maximum, like in the graphs before, can be seen per cavity. Additionally, there is a second much higher peak for the low divergence curve at the position, where the backside of the cavity was during blowout.
This peak results from the long-standing blowout with extreme fields and the following smaller blowout, caused by the remaining driver, as seen in the charge density graphs \autoref{fig:cavity_low}.

\begin{figure}
	\centering
	%% Creator: Matplotlib, PGF backend
%%
%% To include the figure in your LaTeX document, write
%%   \input{<filename>.pgf}
%%
%% Make sure the required packages are loaded in your preamble
%%   \usepackage{pgf}
%%
%% Also ensure that all the required font packages are loaded; for instance,
%% the lmodern package is sometimes necessary when using math font.
%%   \usepackage{lmodern}
%%
%% Figures using additional raster images can only be included by \input if
%% they are in the same directory as the main LaTeX file. For loading figures
%% from other directories you can use the `import` package
%%   \usepackage{import}
%%
%% and then include the figures with
%%   \import{<path to file>}{<filename>.pgf}
%%
%% Matplotlib used the following preamble
%%
\begingroup%
\makeatletter%
\begin{pgfpicture}%
\pgfpathrectangle{\pgfpointorigin}{\pgfqpoint{6.400000in}{6.000000in}}%
\pgfusepath{use as bounding box, clip}%
\begin{pgfscope}%
\pgfsetbuttcap%
\pgfsetmiterjoin%
\pgfsetlinewidth{0.000000pt}%
\definecolor{currentstroke}{rgb}{1.000000,1.000000,1.000000}%
\pgfsetstrokecolor{currentstroke}%
\pgfsetstrokeopacity{0.000000}%
\pgfsetdash{}{0pt}%
\pgfpathmoveto{\pgfqpoint{0.000000in}{0.000000in}}%
\pgfpathlineto{\pgfqpoint{6.400000in}{0.000000in}}%
\pgfpathlineto{\pgfqpoint{6.400000in}{6.000000in}}%
\pgfpathlineto{\pgfqpoint{0.000000in}{6.000000in}}%
\pgfpathlineto{\pgfqpoint{0.000000in}{0.000000in}}%
\pgfpathclose%
\pgfusepath{}%
\end{pgfscope}%
\begin{pgfscope}%
\pgfsetbuttcap%
\pgfsetmiterjoin%
\definecolor{currentfill}{rgb}{1.000000,1.000000,1.000000}%
\pgfsetfillcolor{currentfill}%
\pgfsetlinewidth{0.000000pt}%
\definecolor{currentstroke}{rgb}{0.000000,0.000000,0.000000}%
\pgfsetstrokecolor{currentstroke}%
\pgfsetstrokeopacity{0.000000}%
\pgfsetdash{}{0pt}%
\pgfpathmoveto{\pgfqpoint{0.800000in}{3.220909in}}%
\pgfpathlineto{\pgfqpoint{5.120000in}{3.220909in}}%
\pgfpathlineto{\pgfqpoint{5.120000in}{5.280000in}}%
\pgfpathlineto{\pgfqpoint{0.800000in}{5.280000in}}%
\pgfpathlineto{\pgfqpoint{0.800000in}{3.220909in}}%
\pgfpathclose%
\pgfusepath{fill}%
\end{pgfscope}%
\begin{pgfscope}%
\pgfpathrectangle{\pgfqpoint{0.800000in}{3.220909in}}{\pgfqpoint{4.320000in}{2.059091in}}%
\pgfusepath{clip}%
\pgfsys@transformcm{4.333333}{0.000000}{0.000000}{2.069444}{0.800000in}{3.220909in}%
\pgftext[left,bottom]{\includegraphics[interpolate=false,width=1.000000in,height=1.000000in]{cavity_low-img0.png}}%
\end{pgfscope}%
\begin{pgfscope}%
\pgfpathrectangle{\pgfqpoint{0.800000in}{3.220909in}}{\pgfqpoint{4.320000in}{2.059091in}}%
\pgfusepath{clip}%
\pgfsys@transformcm{4.333333}{0.000000}{0.000000}{2.069444}{0.800000in}{3.220909in}%
\pgftext[left,bottom]{\includegraphics[interpolate=false,width=1.000000in,height=1.000000in]{cavity_low-img1.png}}%
\end{pgfscope}%
\begin{pgfscope}%
\pgfsetbuttcap%
\pgfsetroundjoin%
\definecolor{currentfill}{rgb}{0.000000,0.000000,0.000000}%
\pgfsetfillcolor{currentfill}%
\pgfsetlinewidth{0.803000pt}%
\definecolor{currentstroke}{rgb}{0.000000,0.000000,0.000000}%
\pgfsetstrokecolor{currentstroke}%
\pgfsetdash{}{0pt}%
\pgfsys@defobject{currentmarker}{\pgfqpoint{0.000000in}{-0.048611in}}{\pgfqpoint{0.000000in}{0.000000in}}{%
\pgfpathmoveto{\pgfqpoint{0.000000in}{0.000000in}}%
\pgfpathlineto{\pgfqpoint{0.000000in}{-0.048611in}}%
\pgfusepath{stroke,fill}%
}%
\begin{pgfscope}%
\pgfsys@transformshift{1.279526in}{3.220909in}%
\pgfsys@useobject{currentmarker}{}%
\end{pgfscope}%
\end{pgfscope}%
\begin{pgfscope}%
\pgfsetbuttcap%
\pgfsetroundjoin%
\definecolor{currentfill}{rgb}{0.000000,0.000000,0.000000}%
\pgfsetfillcolor{currentfill}%
\pgfsetlinewidth{0.803000pt}%
\definecolor{currentstroke}{rgb}{0.000000,0.000000,0.000000}%
\pgfsetstrokecolor{currentstroke}%
\pgfsetdash{}{0pt}%
\pgfsys@defobject{currentmarker}{\pgfqpoint{0.000000in}{-0.048611in}}{\pgfqpoint{0.000000in}{0.000000in}}{%
\pgfpathmoveto{\pgfqpoint{0.000000in}{0.000000in}}%
\pgfpathlineto{\pgfqpoint{0.000000in}{-0.048611in}}%
\pgfusepath{stroke,fill}%
}%
\begin{pgfscope}%
\pgfsys@transformshift{1.976075in}{3.220909in}%
\pgfsys@useobject{currentmarker}{}%
\end{pgfscope}%
\end{pgfscope}%
\begin{pgfscope}%
\pgfsetbuttcap%
\pgfsetroundjoin%
\definecolor{currentfill}{rgb}{0.000000,0.000000,0.000000}%
\pgfsetfillcolor{currentfill}%
\pgfsetlinewidth{0.803000pt}%
\definecolor{currentstroke}{rgb}{0.000000,0.000000,0.000000}%
\pgfsetstrokecolor{currentstroke}%
\pgfsetdash{}{0pt}%
\pgfsys@defobject{currentmarker}{\pgfqpoint{0.000000in}{-0.048611in}}{\pgfqpoint{0.000000in}{0.000000in}}{%
\pgfpathmoveto{\pgfqpoint{0.000000in}{0.000000in}}%
\pgfpathlineto{\pgfqpoint{0.000000in}{-0.048611in}}%
\pgfusepath{stroke,fill}%
}%
\begin{pgfscope}%
\pgfsys@transformshift{2.672625in}{3.220909in}%
\pgfsys@useobject{currentmarker}{}%
\end{pgfscope}%
\end{pgfscope}%
\begin{pgfscope}%
\pgfsetbuttcap%
\pgfsetroundjoin%
\definecolor{currentfill}{rgb}{0.000000,0.000000,0.000000}%
\pgfsetfillcolor{currentfill}%
\pgfsetlinewidth{0.803000pt}%
\definecolor{currentstroke}{rgb}{0.000000,0.000000,0.000000}%
\pgfsetstrokecolor{currentstroke}%
\pgfsetdash{}{0pt}%
\pgfsys@defobject{currentmarker}{\pgfqpoint{0.000000in}{-0.048611in}}{\pgfqpoint{0.000000in}{0.000000in}}{%
\pgfpathmoveto{\pgfqpoint{0.000000in}{0.000000in}}%
\pgfpathlineto{\pgfqpoint{0.000000in}{-0.048611in}}%
\pgfusepath{stroke,fill}%
}%
\begin{pgfscope}%
\pgfsys@transformshift{3.369174in}{3.220909in}%
\pgfsys@useobject{currentmarker}{}%
\end{pgfscope}%
\end{pgfscope}%
\begin{pgfscope}%
\pgfsetbuttcap%
\pgfsetroundjoin%
\definecolor{currentfill}{rgb}{0.000000,0.000000,0.000000}%
\pgfsetfillcolor{currentfill}%
\pgfsetlinewidth{0.803000pt}%
\definecolor{currentstroke}{rgb}{0.000000,0.000000,0.000000}%
\pgfsetstrokecolor{currentstroke}%
\pgfsetdash{}{0pt}%
\pgfsys@defobject{currentmarker}{\pgfqpoint{0.000000in}{-0.048611in}}{\pgfqpoint{0.000000in}{0.000000in}}{%
\pgfpathmoveto{\pgfqpoint{0.000000in}{0.000000in}}%
\pgfpathlineto{\pgfqpoint{0.000000in}{-0.048611in}}%
\pgfusepath{stroke,fill}%
}%
\begin{pgfscope}%
\pgfsys@transformshift{4.065724in}{3.220909in}%
\pgfsys@useobject{currentmarker}{}%
\end{pgfscope}%
\end{pgfscope}%
\begin{pgfscope}%
\pgfsetbuttcap%
\pgfsetroundjoin%
\definecolor{currentfill}{rgb}{0.000000,0.000000,0.000000}%
\pgfsetfillcolor{currentfill}%
\pgfsetlinewidth{0.803000pt}%
\definecolor{currentstroke}{rgb}{0.000000,0.000000,0.000000}%
\pgfsetstrokecolor{currentstroke}%
\pgfsetdash{}{0pt}%
\pgfsys@defobject{currentmarker}{\pgfqpoint{0.000000in}{-0.048611in}}{\pgfqpoint{0.000000in}{0.000000in}}{%
\pgfpathmoveto{\pgfqpoint{0.000000in}{0.000000in}}%
\pgfpathlineto{\pgfqpoint{0.000000in}{-0.048611in}}%
\pgfusepath{stroke,fill}%
}%
\begin{pgfscope}%
\pgfsys@transformshift{4.762273in}{3.220909in}%
\pgfsys@useobject{currentmarker}{}%
\end{pgfscope}%
\end{pgfscope}%
\begin{pgfscope}%
\pgfsetbuttcap%
\pgfsetroundjoin%
\definecolor{currentfill}{rgb}{0.000000,0.000000,0.000000}%
\pgfsetfillcolor{currentfill}%
\pgfsetlinewidth{0.803000pt}%
\definecolor{currentstroke}{rgb}{0.000000,0.000000,0.000000}%
\pgfsetstrokecolor{currentstroke}%
\pgfsetdash{}{0pt}%
\pgfsys@defobject{currentmarker}{\pgfqpoint{-0.048611in}{0.000000in}}{\pgfqpoint{-0.000000in}{0.000000in}}{%
\pgfpathmoveto{\pgfqpoint{-0.000000in}{0.000000in}}%
\pgfpathlineto{\pgfqpoint{-0.048611in}{0.000000in}}%
\pgfusepath{stroke,fill}%
}%
\begin{pgfscope}%
\pgfsys@transformshift{0.800000in}{3.340854in}%
\pgfsys@useobject{currentmarker}{}%
\end{pgfscope}%
\end{pgfscope}%
\begin{pgfscope}%
\definecolor{textcolor}{rgb}{0.000000,0.000000,0.000000}%
\pgfsetstrokecolor{textcolor}%
\pgfsetfillcolor{textcolor}%
\pgftext[x=0.455863in, y=3.292628in, left, base]{\color{textcolor}\sffamily\fontsize{10.000000}{12.000000}\selectfont \(\displaystyle {\ensuremath{-}20}\)}%
\end{pgfscope}%
\begin{pgfscope}%
\pgfsetbuttcap%
\pgfsetroundjoin%
\definecolor{currentfill}{rgb}{0.000000,0.000000,0.000000}%
\pgfsetfillcolor{currentfill}%
\pgfsetlinewidth{0.803000pt}%
\definecolor{currentstroke}{rgb}{0.000000,0.000000,0.000000}%
\pgfsetstrokecolor{currentstroke}%
\pgfsetdash{}{0pt}%
\pgfsys@defobject{currentmarker}{\pgfqpoint{-0.048611in}{0.000000in}}{\pgfqpoint{-0.000000in}{0.000000in}}{%
\pgfpathmoveto{\pgfqpoint{-0.000000in}{0.000000in}}%
\pgfpathlineto{\pgfqpoint{-0.048611in}{0.000000in}}%
\pgfusepath{stroke,fill}%
}%
\begin{pgfscope}%
\pgfsys@transformshift{0.800000in}{3.795654in}%
\pgfsys@useobject{currentmarker}{}%
\end{pgfscope}%
\end{pgfscope}%
\begin{pgfscope}%
\definecolor{textcolor}{rgb}{0.000000,0.000000,0.000000}%
\pgfsetstrokecolor{textcolor}%
\pgfsetfillcolor{textcolor}%
\pgftext[x=0.455863in, y=3.747429in, left, base]{\color{textcolor}\sffamily\fontsize{10.000000}{12.000000}\selectfont \(\displaystyle {\ensuremath{-}10}\)}%
\end{pgfscope}%
\begin{pgfscope}%
\pgfsetbuttcap%
\pgfsetroundjoin%
\definecolor{currentfill}{rgb}{0.000000,0.000000,0.000000}%
\pgfsetfillcolor{currentfill}%
\pgfsetlinewidth{0.803000pt}%
\definecolor{currentstroke}{rgb}{0.000000,0.000000,0.000000}%
\pgfsetstrokecolor{currentstroke}%
\pgfsetdash{}{0pt}%
\pgfsys@defobject{currentmarker}{\pgfqpoint{-0.048611in}{0.000000in}}{\pgfqpoint{-0.000000in}{0.000000in}}{%
\pgfpathmoveto{\pgfqpoint{-0.000000in}{0.000000in}}%
\pgfpathlineto{\pgfqpoint{-0.048611in}{0.000000in}}%
\pgfusepath{stroke,fill}%
}%
\begin{pgfscope}%
\pgfsys@transformshift{0.800000in}{4.250455in}%
\pgfsys@useobject{currentmarker}{}%
\end{pgfscope}%
\end{pgfscope}%
\begin{pgfscope}%
\definecolor{textcolor}{rgb}{0.000000,0.000000,0.000000}%
\pgfsetstrokecolor{textcolor}%
\pgfsetfillcolor{textcolor}%
\pgftext[x=0.633333in, y=4.202229in, left, base]{\color{textcolor}\sffamily\fontsize{10.000000}{12.000000}\selectfont \(\displaystyle {0}\)}%
\end{pgfscope}%
\begin{pgfscope}%
\pgfsetbuttcap%
\pgfsetroundjoin%
\definecolor{currentfill}{rgb}{0.000000,0.000000,0.000000}%
\pgfsetfillcolor{currentfill}%
\pgfsetlinewidth{0.803000pt}%
\definecolor{currentstroke}{rgb}{0.000000,0.000000,0.000000}%
\pgfsetstrokecolor{currentstroke}%
\pgfsetdash{}{0pt}%
\pgfsys@defobject{currentmarker}{\pgfqpoint{-0.048611in}{0.000000in}}{\pgfqpoint{-0.000000in}{0.000000in}}{%
\pgfpathmoveto{\pgfqpoint{-0.000000in}{0.000000in}}%
\pgfpathlineto{\pgfqpoint{-0.048611in}{0.000000in}}%
\pgfusepath{stroke,fill}%
}%
\begin{pgfscope}%
\pgfsys@transformshift{0.800000in}{4.705255in}%
\pgfsys@useobject{currentmarker}{}%
\end{pgfscope}%
\end{pgfscope}%
\begin{pgfscope}%
\definecolor{textcolor}{rgb}{0.000000,0.000000,0.000000}%
\pgfsetstrokecolor{textcolor}%
\pgfsetfillcolor{textcolor}%
\pgftext[x=0.563888in, y=4.657030in, left, base]{\color{textcolor}\sffamily\fontsize{10.000000}{12.000000}\selectfont \(\displaystyle {10}\)}%
\end{pgfscope}%
\begin{pgfscope}%
\pgfsetbuttcap%
\pgfsetroundjoin%
\definecolor{currentfill}{rgb}{0.000000,0.000000,0.000000}%
\pgfsetfillcolor{currentfill}%
\pgfsetlinewidth{0.803000pt}%
\definecolor{currentstroke}{rgb}{0.000000,0.000000,0.000000}%
\pgfsetstrokecolor{currentstroke}%
\pgfsetdash{}{0pt}%
\pgfsys@defobject{currentmarker}{\pgfqpoint{-0.048611in}{0.000000in}}{\pgfqpoint{-0.000000in}{0.000000in}}{%
\pgfpathmoveto{\pgfqpoint{-0.000000in}{0.000000in}}%
\pgfpathlineto{\pgfqpoint{-0.048611in}{0.000000in}}%
\pgfusepath{stroke,fill}%
}%
\begin{pgfscope}%
\pgfsys@transformshift{0.800000in}{5.160055in}%
\pgfsys@useobject{currentmarker}{}%
\end{pgfscope}%
\end{pgfscope}%
\begin{pgfscope}%
\definecolor{textcolor}{rgb}{0.000000,0.000000,0.000000}%
\pgfsetstrokecolor{textcolor}%
\pgfsetfillcolor{textcolor}%
\pgftext[x=0.563888in, y=5.111830in, left, base]{\color{textcolor}\sffamily\fontsize{10.000000}{12.000000}\selectfont \(\displaystyle {20}\)}%
\end{pgfscope}%
\begin{pgfscope}%
\definecolor{textcolor}{rgb}{0.000000,0.000000,0.000000}%
\pgfsetstrokecolor{textcolor}%
\pgfsetfillcolor{textcolor}%
\pgftext[x=0.400308in,y=4.250455in,,bottom,rotate=90.000000]{\color{textcolor}\sffamily\fontsize{10.000000}{12.000000}\selectfont \(\displaystyle z \, \mathrm{[\mu m]}\)}%
\end{pgfscope}%
\begin{pgfscope}%
\pgfsetrectcap%
\pgfsetmiterjoin%
\pgfsetlinewidth{0.803000pt}%
\definecolor{currentstroke}{rgb}{0.000000,0.000000,0.000000}%
\pgfsetstrokecolor{currentstroke}%
\pgfsetdash{}{0pt}%
\pgfpathmoveto{\pgfqpoint{0.800000in}{3.220909in}}%
\pgfpathlineto{\pgfqpoint{0.800000in}{5.280000in}}%
\pgfusepath{stroke}%
\end{pgfscope}%
\begin{pgfscope}%
\pgfsetrectcap%
\pgfsetmiterjoin%
\pgfsetlinewidth{0.803000pt}%
\definecolor{currentstroke}{rgb}{0.000000,0.000000,0.000000}%
\pgfsetstrokecolor{currentstroke}%
\pgfsetdash{}{0pt}%
\pgfpathmoveto{\pgfqpoint{5.120000in}{3.220909in}}%
\pgfpathlineto{\pgfqpoint{5.120000in}{5.280000in}}%
\pgfusepath{stroke}%
\end{pgfscope}%
\begin{pgfscope}%
\pgfsetrectcap%
\pgfsetmiterjoin%
\pgfsetlinewidth{0.803000pt}%
\definecolor{currentstroke}{rgb}{0.000000,0.000000,0.000000}%
\pgfsetstrokecolor{currentstroke}%
\pgfsetdash{}{0pt}%
\pgfpathmoveto{\pgfqpoint{0.800000in}{3.220909in}}%
\pgfpathlineto{\pgfqpoint{5.120000in}{3.220909in}}%
\pgfusepath{stroke}%
\end{pgfscope}%
\begin{pgfscope}%
\pgfsetrectcap%
\pgfsetmiterjoin%
\pgfsetlinewidth{0.803000pt}%
\definecolor{currentstroke}{rgb}{0.000000,0.000000,0.000000}%
\pgfsetstrokecolor{currentstroke}%
\pgfsetdash{}{0pt}%
\pgfpathmoveto{\pgfqpoint{0.800000in}{5.280000in}}%
\pgfpathlineto{\pgfqpoint{5.120000in}{5.280000in}}%
\pgfusepath{stroke}%
\end{pgfscope}%
\begin{pgfscope}%
\definecolor{textcolor}{rgb}{0.000000,0.000000,0.000000}%
\pgfsetstrokecolor{textcolor}%
\pgfsetfillcolor{textcolor}%
\pgftext[x=2.960000in,y=5.363333in,,base]{\color{textcolor}\sffamily\fontsize{12.000000}{14.400000}\selectfont a)}%
\end{pgfscope}%
\begin{pgfscope}%
\pgfsetbuttcap%
\pgfsetmiterjoin%
\definecolor{currentfill}{rgb}{1.000000,1.000000,1.000000}%
\pgfsetfillcolor{currentfill}%
\pgfsetlinewidth{0.000000pt}%
\definecolor{currentstroke}{rgb}{0.000000,0.000000,0.000000}%
\pgfsetstrokecolor{currentstroke}%
\pgfsetstrokeopacity{0.000000}%
\pgfsetdash{}{0pt}%
\pgfpathmoveto{\pgfqpoint{0.800000in}{0.750000in}}%
\pgfpathlineto{\pgfqpoint{5.120000in}{0.750000in}}%
\pgfpathlineto{\pgfqpoint{5.120000in}{2.809091in}}%
\pgfpathlineto{\pgfqpoint{0.800000in}{2.809091in}}%
\pgfpathlineto{\pgfqpoint{0.800000in}{0.750000in}}%
\pgfpathclose%
\pgfusepath{fill}%
\end{pgfscope}%
\begin{pgfscope}%
\pgfpathrectangle{\pgfqpoint{0.800000in}{0.750000in}}{\pgfqpoint{4.320000in}{2.059091in}}%
\pgfusepath{clip}%
\pgfsys@transformcm{4.333333}{0.000000}{0.000000}{2.069444}{0.800000in}{0.750000in}%
\pgftext[left,bottom]{\includegraphics[interpolate=false,width=1.000000in,height=1.000000in]{cavity_low-img2.png}}%
\end{pgfscope}%
\begin{pgfscope}%
\pgfsetbuttcap%
\pgfsetroundjoin%
\definecolor{currentfill}{rgb}{0.000000,0.000000,0.000000}%
\pgfsetfillcolor{currentfill}%
\pgfsetlinewidth{0.803000pt}%
\definecolor{currentstroke}{rgb}{0.000000,0.000000,0.000000}%
\pgfsetstrokecolor{currentstroke}%
\pgfsetdash{}{0pt}%
\pgfsys@defobject{currentmarker}{\pgfqpoint{0.000000in}{-0.048611in}}{\pgfqpoint{0.000000in}{0.000000in}}{%
\pgfpathmoveto{\pgfqpoint{0.000000in}{0.000000in}}%
\pgfpathlineto{\pgfqpoint{0.000000in}{-0.048611in}}%
\pgfusepath{stroke,fill}%
}%
\begin{pgfscope}%
\pgfsys@transformshift{1.279526in}{0.750000in}%
\pgfsys@useobject{currentmarker}{}%
\end{pgfscope}%
\end{pgfscope}%
\begin{pgfscope}%
\definecolor{textcolor}{rgb}{0.000000,0.000000,0.000000}%
\pgfsetstrokecolor{textcolor}%
\pgfsetfillcolor{textcolor}%
\pgftext[x=1.279526in,y=0.652778in,,top]{\color{textcolor}\sffamily\fontsize{10.000000}{12.000000}\selectfont \(\displaystyle {\ensuremath{-}40}\)}%
\end{pgfscope}%
\begin{pgfscope}%
\pgfsetbuttcap%
\pgfsetroundjoin%
\definecolor{currentfill}{rgb}{0.000000,0.000000,0.000000}%
\pgfsetfillcolor{currentfill}%
\pgfsetlinewidth{0.803000pt}%
\definecolor{currentstroke}{rgb}{0.000000,0.000000,0.000000}%
\pgfsetstrokecolor{currentstroke}%
\pgfsetdash{}{0pt}%
\pgfsys@defobject{currentmarker}{\pgfqpoint{0.000000in}{-0.048611in}}{\pgfqpoint{0.000000in}{0.000000in}}{%
\pgfpathmoveto{\pgfqpoint{0.000000in}{0.000000in}}%
\pgfpathlineto{\pgfqpoint{0.000000in}{-0.048611in}}%
\pgfusepath{stroke,fill}%
}%
\begin{pgfscope}%
\pgfsys@transformshift{1.976075in}{0.750000in}%
\pgfsys@useobject{currentmarker}{}%
\end{pgfscope}%
\end{pgfscope}%
\begin{pgfscope}%
\definecolor{textcolor}{rgb}{0.000000,0.000000,0.000000}%
\pgfsetstrokecolor{textcolor}%
\pgfsetfillcolor{textcolor}%
\pgftext[x=1.976075in,y=0.652778in,,top]{\color{textcolor}\sffamily\fontsize{10.000000}{12.000000}\selectfont \(\displaystyle {\ensuremath{-}30}\)}%
\end{pgfscope}%
\begin{pgfscope}%
\pgfsetbuttcap%
\pgfsetroundjoin%
\definecolor{currentfill}{rgb}{0.000000,0.000000,0.000000}%
\pgfsetfillcolor{currentfill}%
\pgfsetlinewidth{0.803000pt}%
\definecolor{currentstroke}{rgb}{0.000000,0.000000,0.000000}%
\pgfsetstrokecolor{currentstroke}%
\pgfsetdash{}{0pt}%
\pgfsys@defobject{currentmarker}{\pgfqpoint{0.000000in}{-0.048611in}}{\pgfqpoint{0.000000in}{0.000000in}}{%
\pgfpathmoveto{\pgfqpoint{0.000000in}{0.000000in}}%
\pgfpathlineto{\pgfqpoint{0.000000in}{-0.048611in}}%
\pgfusepath{stroke,fill}%
}%
\begin{pgfscope}%
\pgfsys@transformshift{2.672625in}{0.750000in}%
\pgfsys@useobject{currentmarker}{}%
\end{pgfscope}%
\end{pgfscope}%
\begin{pgfscope}%
\definecolor{textcolor}{rgb}{0.000000,0.000000,0.000000}%
\pgfsetstrokecolor{textcolor}%
\pgfsetfillcolor{textcolor}%
\pgftext[x=2.672625in,y=0.652778in,,top]{\color{textcolor}\sffamily\fontsize{10.000000}{12.000000}\selectfont \(\displaystyle {\ensuremath{-}20}\)}%
\end{pgfscope}%
\begin{pgfscope}%
\pgfsetbuttcap%
\pgfsetroundjoin%
\definecolor{currentfill}{rgb}{0.000000,0.000000,0.000000}%
\pgfsetfillcolor{currentfill}%
\pgfsetlinewidth{0.803000pt}%
\definecolor{currentstroke}{rgb}{0.000000,0.000000,0.000000}%
\pgfsetstrokecolor{currentstroke}%
\pgfsetdash{}{0pt}%
\pgfsys@defobject{currentmarker}{\pgfqpoint{0.000000in}{-0.048611in}}{\pgfqpoint{0.000000in}{0.000000in}}{%
\pgfpathmoveto{\pgfqpoint{0.000000in}{0.000000in}}%
\pgfpathlineto{\pgfqpoint{0.000000in}{-0.048611in}}%
\pgfusepath{stroke,fill}%
}%
\begin{pgfscope}%
\pgfsys@transformshift{3.369174in}{0.750000in}%
\pgfsys@useobject{currentmarker}{}%
\end{pgfscope}%
\end{pgfscope}%
\begin{pgfscope}%
\definecolor{textcolor}{rgb}{0.000000,0.000000,0.000000}%
\pgfsetstrokecolor{textcolor}%
\pgfsetfillcolor{textcolor}%
\pgftext[x=3.369174in,y=0.652778in,,top]{\color{textcolor}\sffamily\fontsize{10.000000}{12.000000}\selectfont \(\displaystyle {\ensuremath{-}10}\)}%
\end{pgfscope}%
\begin{pgfscope}%
\pgfsetbuttcap%
\pgfsetroundjoin%
\definecolor{currentfill}{rgb}{0.000000,0.000000,0.000000}%
\pgfsetfillcolor{currentfill}%
\pgfsetlinewidth{0.803000pt}%
\definecolor{currentstroke}{rgb}{0.000000,0.000000,0.000000}%
\pgfsetstrokecolor{currentstroke}%
\pgfsetdash{}{0pt}%
\pgfsys@defobject{currentmarker}{\pgfqpoint{0.000000in}{-0.048611in}}{\pgfqpoint{0.000000in}{0.000000in}}{%
\pgfpathmoveto{\pgfqpoint{0.000000in}{0.000000in}}%
\pgfpathlineto{\pgfqpoint{0.000000in}{-0.048611in}}%
\pgfusepath{stroke,fill}%
}%
\begin{pgfscope}%
\pgfsys@transformshift{4.065724in}{0.750000in}%
\pgfsys@useobject{currentmarker}{}%
\end{pgfscope}%
\end{pgfscope}%
\begin{pgfscope}%
\definecolor{textcolor}{rgb}{0.000000,0.000000,0.000000}%
\pgfsetstrokecolor{textcolor}%
\pgfsetfillcolor{textcolor}%
\pgftext[x=4.065724in,y=0.652778in,,top]{\color{textcolor}\sffamily\fontsize{10.000000}{12.000000}\selectfont \(\displaystyle {0}\)}%
\end{pgfscope}%
\begin{pgfscope}%
\pgfsetbuttcap%
\pgfsetroundjoin%
\definecolor{currentfill}{rgb}{0.000000,0.000000,0.000000}%
\pgfsetfillcolor{currentfill}%
\pgfsetlinewidth{0.803000pt}%
\definecolor{currentstroke}{rgb}{0.000000,0.000000,0.000000}%
\pgfsetstrokecolor{currentstroke}%
\pgfsetdash{}{0pt}%
\pgfsys@defobject{currentmarker}{\pgfqpoint{0.000000in}{-0.048611in}}{\pgfqpoint{0.000000in}{0.000000in}}{%
\pgfpathmoveto{\pgfqpoint{0.000000in}{0.000000in}}%
\pgfpathlineto{\pgfqpoint{0.000000in}{-0.048611in}}%
\pgfusepath{stroke,fill}%
}%
\begin{pgfscope}%
\pgfsys@transformshift{4.762273in}{0.750000in}%
\pgfsys@useobject{currentmarker}{}%
\end{pgfscope}%
\end{pgfscope}%
\begin{pgfscope}%
\definecolor{textcolor}{rgb}{0.000000,0.000000,0.000000}%
\pgfsetstrokecolor{textcolor}%
\pgfsetfillcolor{textcolor}%
\pgftext[x=4.762273in,y=0.652778in,,top]{\color{textcolor}\sffamily\fontsize{10.000000}{12.000000}\selectfont \(\displaystyle {10}\)}%
\end{pgfscope}%
\begin{pgfscope}%
\definecolor{textcolor}{rgb}{0.000000,0.000000,0.000000}%
\pgfsetstrokecolor{textcolor}%
\pgfsetfillcolor{textcolor}%
\pgftext[x=2.960000in,y=0.473766in,,top]{\color{textcolor}\sffamily\fontsize{10.000000}{12.000000}\selectfont \(\displaystyle \zeta \, \mathrm{[\mu m]}\)}%
\end{pgfscope}%
\begin{pgfscope}%
\pgfsetbuttcap%
\pgfsetroundjoin%
\definecolor{currentfill}{rgb}{0.000000,0.000000,0.000000}%
\pgfsetfillcolor{currentfill}%
\pgfsetlinewidth{0.803000pt}%
\definecolor{currentstroke}{rgb}{0.000000,0.000000,0.000000}%
\pgfsetstrokecolor{currentstroke}%
\pgfsetdash{}{0pt}%
\pgfsys@defobject{currentmarker}{\pgfqpoint{-0.048611in}{0.000000in}}{\pgfqpoint{-0.000000in}{0.000000in}}{%
\pgfpathmoveto{\pgfqpoint{-0.000000in}{0.000000in}}%
\pgfpathlineto{\pgfqpoint{-0.048611in}{0.000000in}}%
\pgfusepath{stroke,fill}%
}%
\begin{pgfscope}%
\pgfsys@transformshift{0.800000in}{0.813812in}%
\pgfsys@useobject{currentmarker}{}%
\end{pgfscope}%
\end{pgfscope}%
\begin{pgfscope}%
\definecolor{textcolor}{rgb}{0.000000,0.000000,0.000000}%
\pgfsetstrokecolor{textcolor}%
\pgfsetfillcolor{textcolor}%
\pgftext[x=0.633333in, y=0.765586in, left, base]{\color{textcolor}\sffamily\fontsize{10.000000}{12.000000}\selectfont \(\displaystyle {0}\)}%
\end{pgfscope}%
\begin{pgfscope}%
\pgfsetbuttcap%
\pgfsetroundjoin%
\definecolor{currentfill}{rgb}{0.000000,0.000000,0.000000}%
\pgfsetfillcolor{currentfill}%
\pgfsetlinewidth{0.803000pt}%
\definecolor{currentstroke}{rgb}{0.000000,0.000000,0.000000}%
\pgfsetstrokecolor{currentstroke}%
\pgfsetdash{}{0pt}%
\pgfsys@defobject{currentmarker}{\pgfqpoint{-0.048611in}{0.000000in}}{\pgfqpoint{-0.000000in}{0.000000in}}{%
\pgfpathmoveto{\pgfqpoint{-0.000000in}{0.000000in}}%
\pgfpathlineto{\pgfqpoint{-0.048611in}{0.000000in}}%
\pgfusepath{stroke,fill}%
}%
\begin{pgfscope}%
\pgfsys@transformshift{0.800000in}{1.137481in}%
\pgfsys@useobject{currentmarker}{}%
\end{pgfscope}%
\end{pgfscope}%
\begin{pgfscope}%
\definecolor{textcolor}{rgb}{0.000000,0.000000,0.000000}%
\pgfsetstrokecolor{textcolor}%
\pgfsetfillcolor{textcolor}%
\pgftext[x=0.633333in, y=1.089256in, left, base]{\color{textcolor}\sffamily\fontsize{10.000000}{12.000000}\selectfont \(\displaystyle {1}\)}%
\end{pgfscope}%
\begin{pgfscope}%
\pgfsetbuttcap%
\pgfsetroundjoin%
\definecolor{currentfill}{rgb}{0.000000,0.000000,0.000000}%
\pgfsetfillcolor{currentfill}%
\pgfsetlinewidth{0.803000pt}%
\definecolor{currentstroke}{rgb}{0.000000,0.000000,0.000000}%
\pgfsetstrokecolor{currentstroke}%
\pgfsetdash{}{0pt}%
\pgfsys@defobject{currentmarker}{\pgfqpoint{-0.048611in}{0.000000in}}{\pgfqpoint{-0.000000in}{0.000000in}}{%
\pgfpathmoveto{\pgfqpoint{-0.000000in}{0.000000in}}%
\pgfpathlineto{\pgfqpoint{-0.048611in}{0.000000in}}%
\pgfusepath{stroke,fill}%
}%
\begin{pgfscope}%
\pgfsys@transformshift{0.800000in}{1.461150in}%
\pgfsys@useobject{currentmarker}{}%
\end{pgfscope}%
\end{pgfscope}%
\begin{pgfscope}%
\definecolor{textcolor}{rgb}{0.000000,0.000000,0.000000}%
\pgfsetstrokecolor{textcolor}%
\pgfsetfillcolor{textcolor}%
\pgftext[x=0.633333in, y=1.412925in, left, base]{\color{textcolor}\sffamily\fontsize{10.000000}{12.000000}\selectfont \(\displaystyle {2}\)}%
\end{pgfscope}%
\begin{pgfscope}%
\pgfsetbuttcap%
\pgfsetroundjoin%
\definecolor{currentfill}{rgb}{0.000000,0.000000,0.000000}%
\pgfsetfillcolor{currentfill}%
\pgfsetlinewidth{0.803000pt}%
\definecolor{currentstroke}{rgb}{0.000000,0.000000,0.000000}%
\pgfsetstrokecolor{currentstroke}%
\pgfsetdash{}{0pt}%
\pgfsys@defobject{currentmarker}{\pgfqpoint{-0.048611in}{0.000000in}}{\pgfqpoint{-0.000000in}{0.000000in}}{%
\pgfpathmoveto{\pgfqpoint{-0.000000in}{0.000000in}}%
\pgfpathlineto{\pgfqpoint{-0.048611in}{0.000000in}}%
\pgfusepath{stroke,fill}%
}%
\begin{pgfscope}%
\pgfsys@transformshift{0.800000in}{1.784820in}%
\pgfsys@useobject{currentmarker}{}%
\end{pgfscope}%
\end{pgfscope}%
\begin{pgfscope}%
\definecolor{textcolor}{rgb}{0.000000,0.000000,0.000000}%
\pgfsetstrokecolor{textcolor}%
\pgfsetfillcolor{textcolor}%
\pgftext[x=0.633333in, y=1.736594in, left, base]{\color{textcolor}\sffamily\fontsize{10.000000}{12.000000}\selectfont \(\displaystyle {3}\)}%
\end{pgfscope}%
\begin{pgfscope}%
\pgfsetbuttcap%
\pgfsetroundjoin%
\definecolor{currentfill}{rgb}{0.000000,0.000000,0.000000}%
\pgfsetfillcolor{currentfill}%
\pgfsetlinewidth{0.803000pt}%
\definecolor{currentstroke}{rgb}{0.000000,0.000000,0.000000}%
\pgfsetstrokecolor{currentstroke}%
\pgfsetdash{}{0pt}%
\pgfsys@defobject{currentmarker}{\pgfqpoint{-0.048611in}{0.000000in}}{\pgfqpoint{-0.000000in}{0.000000in}}{%
\pgfpathmoveto{\pgfqpoint{-0.000000in}{0.000000in}}%
\pgfpathlineto{\pgfqpoint{-0.048611in}{0.000000in}}%
\pgfusepath{stroke,fill}%
}%
\begin{pgfscope}%
\pgfsys@transformshift{0.800000in}{2.108489in}%
\pgfsys@useobject{currentmarker}{}%
\end{pgfscope}%
\end{pgfscope}%
\begin{pgfscope}%
\definecolor{textcolor}{rgb}{0.000000,0.000000,0.000000}%
\pgfsetstrokecolor{textcolor}%
\pgfsetfillcolor{textcolor}%
\pgftext[x=0.633333in, y=2.060264in, left, base]{\color{textcolor}\sffamily\fontsize{10.000000}{12.000000}\selectfont \(\displaystyle {4}\)}%
\end{pgfscope}%
\begin{pgfscope}%
\pgfsetbuttcap%
\pgfsetroundjoin%
\definecolor{currentfill}{rgb}{0.000000,0.000000,0.000000}%
\pgfsetfillcolor{currentfill}%
\pgfsetlinewidth{0.803000pt}%
\definecolor{currentstroke}{rgb}{0.000000,0.000000,0.000000}%
\pgfsetstrokecolor{currentstroke}%
\pgfsetdash{}{0pt}%
\pgfsys@defobject{currentmarker}{\pgfqpoint{-0.048611in}{0.000000in}}{\pgfqpoint{-0.000000in}{0.000000in}}{%
\pgfpathmoveto{\pgfqpoint{-0.000000in}{0.000000in}}%
\pgfpathlineto{\pgfqpoint{-0.048611in}{0.000000in}}%
\pgfusepath{stroke,fill}%
}%
\begin{pgfscope}%
\pgfsys@transformshift{0.800000in}{2.432158in}%
\pgfsys@useobject{currentmarker}{}%
\end{pgfscope}%
\end{pgfscope}%
\begin{pgfscope}%
\definecolor{textcolor}{rgb}{0.000000,0.000000,0.000000}%
\pgfsetstrokecolor{textcolor}%
\pgfsetfillcolor{textcolor}%
\pgftext[x=0.633333in, y=2.383933in, left, base]{\color{textcolor}\sffamily\fontsize{10.000000}{12.000000}\selectfont \(\displaystyle {5}\)}%
\end{pgfscope}%
\begin{pgfscope}%
\pgfsetbuttcap%
\pgfsetroundjoin%
\definecolor{currentfill}{rgb}{0.000000,0.000000,0.000000}%
\pgfsetfillcolor{currentfill}%
\pgfsetlinewidth{0.803000pt}%
\definecolor{currentstroke}{rgb}{0.000000,0.000000,0.000000}%
\pgfsetstrokecolor{currentstroke}%
\pgfsetdash{}{0pt}%
\pgfsys@defobject{currentmarker}{\pgfqpoint{-0.048611in}{0.000000in}}{\pgfqpoint{-0.000000in}{0.000000in}}{%
\pgfpathmoveto{\pgfqpoint{-0.000000in}{0.000000in}}%
\pgfpathlineto{\pgfqpoint{-0.048611in}{0.000000in}}%
\pgfusepath{stroke,fill}%
}%
\begin{pgfscope}%
\pgfsys@transformshift{0.800000in}{2.755827in}%
\pgfsys@useobject{currentmarker}{}%
\end{pgfscope}%
\end{pgfscope}%
\begin{pgfscope}%
\definecolor{textcolor}{rgb}{0.000000,0.000000,0.000000}%
\pgfsetstrokecolor{textcolor}%
\pgfsetfillcolor{textcolor}%
\pgftext[x=0.633333in, y=2.707602in, left, base]{\color{textcolor}\sffamily\fontsize{10.000000}{12.000000}\selectfont \(\displaystyle {6}\)}%
\end{pgfscope}%
\begin{pgfscope}%
\definecolor{textcolor}{rgb}{0.000000,0.000000,0.000000}%
\pgfsetstrokecolor{textcolor}%
\pgfsetfillcolor{textcolor}%
\pgftext[x=0.577778in,y=1.779545in,,bottom,rotate=90.000000]{\color{textcolor}\sffamily\fontsize{10.000000}{12.000000}\selectfont \(\displaystyle y \,[mm]\)}%
\end{pgfscope}%
\begin{pgfscope}%
\pgfsetrectcap%
\pgfsetmiterjoin%
\pgfsetlinewidth{0.803000pt}%
\definecolor{currentstroke}{rgb}{0.000000,0.000000,0.000000}%
\pgfsetstrokecolor{currentstroke}%
\pgfsetdash{}{0pt}%
\pgfpathmoveto{\pgfqpoint{0.800000in}{0.750000in}}%
\pgfpathlineto{\pgfqpoint{0.800000in}{2.809091in}}%
\pgfusepath{stroke}%
\end{pgfscope}%
\begin{pgfscope}%
\pgfsetrectcap%
\pgfsetmiterjoin%
\pgfsetlinewidth{0.803000pt}%
\definecolor{currentstroke}{rgb}{0.000000,0.000000,0.000000}%
\pgfsetstrokecolor{currentstroke}%
\pgfsetdash{}{0pt}%
\pgfpathmoveto{\pgfqpoint{5.120000in}{0.750000in}}%
\pgfpathlineto{\pgfqpoint{5.120000in}{2.809091in}}%
\pgfusepath{stroke}%
\end{pgfscope}%
\begin{pgfscope}%
\pgfsetrectcap%
\pgfsetmiterjoin%
\pgfsetlinewidth{0.803000pt}%
\definecolor{currentstroke}{rgb}{0.000000,0.000000,0.000000}%
\pgfsetstrokecolor{currentstroke}%
\pgfsetdash{}{0pt}%
\pgfpathmoveto{\pgfqpoint{0.800000in}{0.750000in}}%
\pgfpathlineto{\pgfqpoint{5.120000in}{0.750000in}}%
\pgfusepath{stroke}%
\end{pgfscope}%
\begin{pgfscope}%
\pgfsetrectcap%
\pgfsetmiterjoin%
\pgfsetlinewidth{0.803000pt}%
\definecolor{currentstroke}{rgb}{0.000000,0.000000,0.000000}%
\pgfsetstrokecolor{currentstroke}%
\pgfsetdash{}{0pt}%
\pgfpathmoveto{\pgfqpoint{0.800000in}{2.809091in}}%
\pgfpathlineto{\pgfqpoint{5.120000in}{2.809091in}}%
\pgfusepath{stroke}%
\end{pgfscope}%
\begin{pgfscope}%
\definecolor{textcolor}{rgb}{0.000000,0.000000,0.000000}%
\pgfsetstrokecolor{textcolor}%
\pgfsetfillcolor{textcolor}%
\pgftext[x=2.960000in,y=2.892424in,,base]{\color{textcolor}\sffamily\fontsize{12.000000}{14.400000}\selectfont b)}%
\end{pgfscope}%
\begin{pgfscope}%
\pgfsetbuttcap%
\pgfsetmiterjoin%
\definecolor{currentfill}{rgb}{1.000000,1.000000,1.000000}%
\pgfsetfillcolor{currentfill}%
\pgfsetlinewidth{0.000000pt}%
\definecolor{currentstroke}{rgb}{0.000000,0.000000,0.000000}%
\pgfsetstrokecolor{currentstroke}%
\pgfsetstrokeopacity{0.000000}%
\pgfsetdash{}{0pt}%
\pgfpathmoveto{\pgfqpoint{5.312000in}{0.900000in}}%
\pgfpathlineto{\pgfqpoint{5.504000in}{0.900000in}}%
\pgfpathlineto{\pgfqpoint{5.504000in}{2.700000in}}%
\pgfpathlineto{\pgfqpoint{5.312000in}{2.700000in}}%
\pgfpathlineto{\pgfqpoint{5.312000in}{0.900000in}}%
\pgfpathclose%
\pgfusepath{fill}%
\end{pgfscope}%
\begin{pgfscope}%
\pgfpathrectangle{\pgfqpoint{5.312000in}{0.900000in}}{\pgfqpoint{0.192000in}{1.800000in}}%
\pgfusepath{clip}%
\pgfsetbuttcap%
\pgfsetmiterjoin%
\definecolor{currentfill}{rgb}{1.000000,1.000000,1.000000}%
\pgfsetfillcolor{currentfill}%
\pgfsetlinewidth{0.010037pt}%
\definecolor{currentstroke}{rgb}{1.000000,1.000000,1.000000}%
\pgfsetstrokecolor{currentstroke}%
\pgfsetdash{}{0pt}%
\pgfusepath{stroke,fill}%
\end{pgfscope}%
\begin{pgfscope}%
\pgfsys@transformshift{5.305556in}{0.902778in}%
\pgftext[left,bottom]{\includegraphics[interpolate=true,width=0.194444in,height=1.791667in]{cavity_low-img3.png}}%
\end{pgfscope}%
\begin{pgfscope}%
\pgfsetbuttcap%
\pgfsetroundjoin%
\definecolor{currentfill}{rgb}{0.000000,0.000000,0.000000}%
\pgfsetfillcolor{currentfill}%
\pgfsetlinewidth{0.803000pt}%
\definecolor{currentstroke}{rgb}{0.000000,0.000000,0.000000}%
\pgfsetstrokecolor{currentstroke}%
\pgfsetdash{}{0pt}%
\pgfsys@defobject{currentmarker}{\pgfqpoint{0.000000in}{0.000000in}}{\pgfqpoint{0.048611in}{0.000000in}}{%
\pgfpathmoveto{\pgfqpoint{0.000000in}{0.000000in}}%
\pgfpathlineto{\pgfqpoint{0.048611in}{0.000000in}}%
\pgfusepath{stroke,fill}%
}%
\begin{pgfscope}%
\pgfsys@transformshift{5.504000in}{0.900000in}%
\pgfsys@useobject{currentmarker}{}%
\end{pgfscope}%
\end{pgfscope}%
\begin{pgfscope}%
\definecolor{textcolor}{rgb}{0.000000,0.000000,0.000000}%
\pgfsetstrokecolor{textcolor}%
\pgfsetfillcolor{textcolor}%
\pgftext[x=5.601222in, y=0.851775in, left, base]{\color{textcolor}\sffamily\fontsize{10.000000}{12.000000}\selectfont \(\displaystyle {10^{-3}}\)}%
\end{pgfscope}%
\begin{pgfscope}%
\pgfsetbuttcap%
\pgfsetroundjoin%
\definecolor{currentfill}{rgb}{0.000000,0.000000,0.000000}%
\pgfsetfillcolor{currentfill}%
\pgfsetlinewidth{0.803000pt}%
\definecolor{currentstroke}{rgb}{0.000000,0.000000,0.000000}%
\pgfsetstrokecolor{currentstroke}%
\pgfsetdash{}{0pt}%
\pgfsys@defobject{currentmarker}{\pgfqpoint{0.000000in}{0.000000in}}{\pgfqpoint{0.048611in}{0.000000in}}{%
\pgfpathmoveto{\pgfqpoint{0.000000in}{0.000000in}}%
\pgfpathlineto{\pgfqpoint{0.048611in}{0.000000in}}%
\pgfusepath{stroke,fill}%
}%
\begin{pgfscope}%
\pgfsys@transformshift{5.504000in}{1.239557in}%
\pgfsys@useobject{currentmarker}{}%
\end{pgfscope}%
\end{pgfscope}%
\begin{pgfscope}%
\definecolor{textcolor}{rgb}{0.000000,0.000000,0.000000}%
\pgfsetstrokecolor{textcolor}%
\pgfsetfillcolor{textcolor}%
\pgftext[x=5.601222in, y=1.191331in, left, base]{\color{textcolor}\sffamily\fontsize{10.000000}{12.000000}\selectfont \(\displaystyle {10^{-2}}\)}%
\end{pgfscope}%
\begin{pgfscope}%
\pgfsetbuttcap%
\pgfsetroundjoin%
\definecolor{currentfill}{rgb}{0.000000,0.000000,0.000000}%
\pgfsetfillcolor{currentfill}%
\pgfsetlinewidth{0.803000pt}%
\definecolor{currentstroke}{rgb}{0.000000,0.000000,0.000000}%
\pgfsetstrokecolor{currentstroke}%
\pgfsetdash{}{0pt}%
\pgfsys@defobject{currentmarker}{\pgfqpoint{0.000000in}{0.000000in}}{\pgfqpoint{0.048611in}{0.000000in}}{%
\pgfpathmoveto{\pgfqpoint{0.000000in}{0.000000in}}%
\pgfpathlineto{\pgfqpoint{0.048611in}{0.000000in}}%
\pgfusepath{stroke,fill}%
}%
\begin{pgfscope}%
\pgfsys@transformshift{5.504000in}{1.579113in}%
\pgfsys@useobject{currentmarker}{}%
\end{pgfscope}%
\end{pgfscope}%
\begin{pgfscope}%
\definecolor{textcolor}{rgb}{0.000000,0.000000,0.000000}%
\pgfsetstrokecolor{textcolor}%
\pgfsetfillcolor{textcolor}%
\pgftext[x=5.601222in, y=1.530888in, left, base]{\color{textcolor}\sffamily\fontsize{10.000000}{12.000000}\selectfont \(\displaystyle {10^{-1}}\)}%
\end{pgfscope}%
\begin{pgfscope}%
\pgfsetbuttcap%
\pgfsetroundjoin%
\definecolor{currentfill}{rgb}{0.000000,0.000000,0.000000}%
\pgfsetfillcolor{currentfill}%
\pgfsetlinewidth{0.803000pt}%
\definecolor{currentstroke}{rgb}{0.000000,0.000000,0.000000}%
\pgfsetstrokecolor{currentstroke}%
\pgfsetdash{}{0pt}%
\pgfsys@defobject{currentmarker}{\pgfqpoint{0.000000in}{0.000000in}}{\pgfqpoint{0.048611in}{0.000000in}}{%
\pgfpathmoveto{\pgfqpoint{0.000000in}{0.000000in}}%
\pgfpathlineto{\pgfqpoint{0.048611in}{0.000000in}}%
\pgfusepath{stroke,fill}%
}%
\begin{pgfscope}%
\pgfsys@transformshift{5.504000in}{1.918670in}%
\pgfsys@useobject{currentmarker}{}%
\end{pgfscope}%
\end{pgfscope}%
\begin{pgfscope}%
\definecolor{textcolor}{rgb}{0.000000,0.000000,0.000000}%
\pgfsetstrokecolor{textcolor}%
\pgfsetfillcolor{textcolor}%
\pgftext[x=5.601222in, y=1.870445in, left, base]{\color{textcolor}\sffamily\fontsize{10.000000}{12.000000}\selectfont \(\displaystyle {10^{0}}\)}%
\end{pgfscope}%
\begin{pgfscope}%
\pgfsetbuttcap%
\pgfsetroundjoin%
\definecolor{currentfill}{rgb}{0.000000,0.000000,0.000000}%
\pgfsetfillcolor{currentfill}%
\pgfsetlinewidth{0.803000pt}%
\definecolor{currentstroke}{rgb}{0.000000,0.000000,0.000000}%
\pgfsetstrokecolor{currentstroke}%
\pgfsetdash{}{0pt}%
\pgfsys@defobject{currentmarker}{\pgfqpoint{0.000000in}{0.000000in}}{\pgfqpoint{0.048611in}{0.000000in}}{%
\pgfpathmoveto{\pgfqpoint{0.000000in}{0.000000in}}%
\pgfpathlineto{\pgfqpoint{0.048611in}{0.000000in}}%
\pgfusepath{stroke,fill}%
}%
\begin{pgfscope}%
\pgfsys@transformshift{5.504000in}{2.258227in}%
\pgfsys@useobject{currentmarker}{}%
\end{pgfscope}%
\end{pgfscope}%
\begin{pgfscope}%
\definecolor{textcolor}{rgb}{0.000000,0.000000,0.000000}%
\pgfsetstrokecolor{textcolor}%
\pgfsetfillcolor{textcolor}%
\pgftext[x=5.601222in, y=2.210001in, left, base]{\color{textcolor}\sffamily\fontsize{10.000000}{12.000000}\selectfont \(\displaystyle {10^{1}}\)}%
\end{pgfscope}%
\begin{pgfscope}%
\pgfsetbuttcap%
\pgfsetroundjoin%
\definecolor{currentfill}{rgb}{0.000000,0.000000,0.000000}%
\pgfsetfillcolor{currentfill}%
\pgfsetlinewidth{0.803000pt}%
\definecolor{currentstroke}{rgb}{0.000000,0.000000,0.000000}%
\pgfsetstrokecolor{currentstroke}%
\pgfsetdash{}{0pt}%
\pgfsys@defobject{currentmarker}{\pgfqpoint{0.000000in}{0.000000in}}{\pgfqpoint{0.048611in}{0.000000in}}{%
\pgfpathmoveto{\pgfqpoint{0.000000in}{0.000000in}}%
\pgfpathlineto{\pgfqpoint{0.048611in}{0.000000in}}%
\pgfusepath{stroke,fill}%
}%
\begin{pgfscope}%
\pgfsys@transformshift{5.504000in}{2.597783in}%
\pgfsys@useobject{currentmarker}{}%
\end{pgfscope}%
\end{pgfscope}%
\begin{pgfscope}%
\definecolor{textcolor}{rgb}{0.000000,0.000000,0.000000}%
\pgfsetstrokecolor{textcolor}%
\pgfsetfillcolor{textcolor}%
\pgftext[x=5.601222in, y=2.549558in, left, base]{\color{textcolor}\sffamily\fontsize{10.000000}{12.000000}\selectfont \(\displaystyle {10^{2}}\)}%
\end{pgfscope}%
\begin{pgfscope}%
\pgfsetbuttcap%
\pgfsetroundjoin%
\definecolor{currentfill}{rgb}{0.000000,0.000000,0.000000}%
\pgfsetfillcolor{currentfill}%
\pgfsetlinewidth{0.602250pt}%
\definecolor{currentstroke}{rgb}{0.000000,0.000000,0.000000}%
\pgfsetstrokecolor{currentstroke}%
\pgfsetdash{}{0pt}%
\pgfsys@defobject{currentmarker}{\pgfqpoint{0.000000in}{0.000000in}}{\pgfqpoint{0.027778in}{0.000000in}}{%
\pgfpathmoveto{\pgfqpoint{0.000000in}{0.000000in}}%
\pgfpathlineto{\pgfqpoint{0.027778in}{0.000000in}}%
\pgfusepath{stroke,fill}%
}%
\begin{pgfscope}%
\pgfsys@transformshift{5.504000in}{1.002217in}%
\pgfsys@useobject{currentmarker}{}%
\end{pgfscope}%
\end{pgfscope}%
\begin{pgfscope}%
\pgfsetbuttcap%
\pgfsetroundjoin%
\definecolor{currentfill}{rgb}{0.000000,0.000000,0.000000}%
\pgfsetfillcolor{currentfill}%
\pgfsetlinewidth{0.602250pt}%
\definecolor{currentstroke}{rgb}{0.000000,0.000000,0.000000}%
\pgfsetstrokecolor{currentstroke}%
\pgfsetdash{}{0pt}%
\pgfsys@defobject{currentmarker}{\pgfqpoint{0.000000in}{0.000000in}}{\pgfqpoint{0.027778in}{0.000000in}}{%
\pgfpathmoveto{\pgfqpoint{0.000000in}{0.000000in}}%
\pgfpathlineto{\pgfqpoint{0.027778in}{0.000000in}}%
\pgfusepath{stroke,fill}%
}%
\begin{pgfscope}%
\pgfsys@transformshift{5.504000in}{1.062010in}%
\pgfsys@useobject{currentmarker}{}%
\end{pgfscope}%
\end{pgfscope}%
\begin{pgfscope}%
\pgfsetbuttcap%
\pgfsetroundjoin%
\definecolor{currentfill}{rgb}{0.000000,0.000000,0.000000}%
\pgfsetfillcolor{currentfill}%
\pgfsetlinewidth{0.602250pt}%
\definecolor{currentstroke}{rgb}{0.000000,0.000000,0.000000}%
\pgfsetstrokecolor{currentstroke}%
\pgfsetdash{}{0pt}%
\pgfsys@defobject{currentmarker}{\pgfqpoint{0.000000in}{0.000000in}}{\pgfqpoint{0.027778in}{0.000000in}}{%
\pgfpathmoveto{\pgfqpoint{0.000000in}{0.000000in}}%
\pgfpathlineto{\pgfqpoint{0.027778in}{0.000000in}}%
\pgfusepath{stroke,fill}%
}%
\begin{pgfscope}%
\pgfsys@transformshift{5.504000in}{1.104433in}%
\pgfsys@useobject{currentmarker}{}%
\end{pgfscope}%
\end{pgfscope}%
\begin{pgfscope}%
\pgfsetbuttcap%
\pgfsetroundjoin%
\definecolor{currentfill}{rgb}{0.000000,0.000000,0.000000}%
\pgfsetfillcolor{currentfill}%
\pgfsetlinewidth{0.602250pt}%
\definecolor{currentstroke}{rgb}{0.000000,0.000000,0.000000}%
\pgfsetstrokecolor{currentstroke}%
\pgfsetdash{}{0pt}%
\pgfsys@defobject{currentmarker}{\pgfqpoint{0.000000in}{0.000000in}}{\pgfqpoint{0.027778in}{0.000000in}}{%
\pgfpathmoveto{\pgfqpoint{0.000000in}{0.000000in}}%
\pgfpathlineto{\pgfqpoint{0.027778in}{0.000000in}}%
\pgfusepath{stroke,fill}%
}%
\begin{pgfscope}%
\pgfsys@transformshift{5.504000in}{1.137340in}%
\pgfsys@useobject{currentmarker}{}%
\end{pgfscope}%
\end{pgfscope}%
\begin{pgfscope}%
\pgfsetbuttcap%
\pgfsetroundjoin%
\definecolor{currentfill}{rgb}{0.000000,0.000000,0.000000}%
\pgfsetfillcolor{currentfill}%
\pgfsetlinewidth{0.602250pt}%
\definecolor{currentstroke}{rgb}{0.000000,0.000000,0.000000}%
\pgfsetstrokecolor{currentstroke}%
\pgfsetdash{}{0pt}%
\pgfsys@defobject{currentmarker}{\pgfqpoint{0.000000in}{0.000000in}}{\pgfqpoint{0.027778in}{0.000000in}}{%
\pgfpathmoveto{\pgfqpoint{0.000000in}{0.000000in}}%
\pgfpathlineto{\pgfqpoint{0.027778in}{0.000000in}}%
\pgfusepath{stroke,fill}%
}%
\begin{pgfscope}%
\pgfsys@transformshift{5.504000in}{1.164226in}%
\pgfsys@useobject{currentmarker}{}%
\end{pgfscope}%
\end{pgfscope}%
\begin{pgfscope}%
\pgfsetbuttcap%
\pgfsetroundjoin%
\definecolor{currentfill}{rgb}{0.000000,0.000000,0.000000}%
\pgfsetfillcolor{currentfill}%
\pgfsetlinewidth{0.602250pt}%
\definecolor{currentstroke}{rgb}{0.000000,0.000000,0.000000}%
\pgfsetstrokecolor{currentstroke}%
\pgfsetdash{}{0pt}%
\pgfsys@defobject{currentmarker}{\pgfqpoint{0.000000in}{0.000000in}}{\pgfqpoint{0.027778in}{0.000000in}}{%
\pgfpathmoveto{\pgfqpoint{0.000000in}{0.000000in}}%
\pgfpathlineto{\pgfqpoint{0.027778in}{0.000000in}}%
\pgfusepath{stroke,fill}%
}%
\begin{pgfscope}%
\pgfsys@transformshift{5.504000in}{1.186959in}%
\pgfsys@useobject{currentmarker}{}%
\end{pgfscope}%
\end{pgfscope}%
\begin{pgfscope}%
\pgfsetbuttcap%
\pgfsetroundjoin%
\definecolor{currentfill}{rgb}{0.000000,0.000000,0.000000}%
\pgfsetfillcolor{currentfill}%
\pgfsetlinewidth{0.602250pt}%
\definecolor{currentstroke}{rgb}{0.000000,0.000000,0.000000}%
\pgfsetstrokecolor{currentstroke}%
\pgfsetdash{}{0pt}%
\pgfsys@defobject{currentmarker}{\pgfqpoint{0.000000in}{0.000000in}}{\pgfqpoint{0.027778in}{0.000000in}}{%
\pgfpathmoveto{\pgfqpoint{0.000000in}{0.000000in}}%
\pgfpathlineto{\pgfqpoint{0.027778in}{0.000000in}}%
\pgfusepath{stroke,fill}%
}%
\begin{pgfscope}%
\pgfsys@transformshift{5.504000in}{1.206650in}%
\pgfsys@useobject{currentmarker}{}%
\end{pgfscope}%
\end{pgfscope}%
\begin{pgfscope}%
\pgfsetbuttcap%
\pgfsetroundjoin%
\definecolor{currentfill}{rgb}{0.000000,0.000000,0.000000}%
\pgfsetfillcolor{currentfill}%
\pgfsetlinewidth{0.602250pt}%
\definecolor{currentstroke}{rgb}{0.000000,0.000000,0.000000}%
\pgfsetstrokecolor{currentstroke}%
\pgfsetdash{}{0pt}%
\pgfsys@defobject{currentmarker}{\pgfqpoint{0.000000in}{0.000000in}}{\pgfqpoint{0.027778in}{0.000000in}}{%
\pgfpathmoveto{\pgfqpoint{0.000000in}{0.000000in}}%
\pgfpathlineto{\pgfqpoint{0.027778in}{0.000000in}}%
\pgfusepath{stroke,fill}%
}%
\begin{pgfscope}%
\pgfsys@transformshift{5.504000in}{1.224019in}%
\pgfsys@useobject{currentmarker}{}%
\end{pgfscope}%
\end{pgfscope}%
\begin{pgfscope}%
\pgfsetbuttcap%
\pgfsetroundjoin%
\definecolor{currentfill}{rgb}{0.000000,0.000000,0.000000}%
\pgfsetfillcolor{currentfill}%
\pgfsetlinewidth{0.602250pt}%
\definecolor{currentstroke}{rgb}{0.000000,0.000000,0.000000}%
\pgfsetstrokecolor{currentstroke}%
\pgfsetdash{}{0pt}%
\pgfsys@defobject{currentmarker}{\pgfqpoint{0.000000in}{0.000000in}}{\pgfqpoint{0.027778in}{0.000000in}}{%
\pgfpathmoveto{\pgfqpoint{0.000000in}{0.000000in}}%
\pgfpathlineto{\pgfqpoint{0.027778in}{0.000000in}}%
\pgfusepath{stroke,fill}%
}%
\begin{pgfscope}%
\pgfsys@transformshift{5.504000in}{1.341773in}%
\pgfsys@useobject{currentmarker}{}%
\end{pgfscope}%
\end{pgfscope}%
\begin{pgfscope}%
\pgfsetbuttcap%
\pgfsetroundjoin%
\definecolor{currentfill}{rgb}{0.000000,0.000000,0.000000}%
\pgfsetfillcolor{currentfill}%
\pgfsetlinewidth{0.602250pt}%
\definecolor{currentstroke}{rgb}{0.000000,0.000000,0.000000}%
\pgfsetstrokecolor{currentstroke}%
\pgfsetdash{}{0pt}%
\pgfsys@defobject{currentmarker}{\pgfqpoint{0.000000in}{0.000000in}}{\pgfqpoint{0.027778in}{0.000000in}}{%
\pgfpathmoveto{\pgfqpoint{0.000000in}{0.000000in}}%
\pgfpathlineto{\pgfqpoint{0.027778in}{0.000000in}}%
\pgfusepath{stroke,fill}%
}%
\begin{pgfscope}%
\pgfsys@transformshift{5.504000in}{1.401566in}%
\pgfsys@useobject{currentmarker}{}%
\end{pgfscope}%
\end{pgfscope}%
\begin{pgfscope}%
\pgfsetbuttcap%
\pgfsetroundjoin%
\definecolor{currentfill}{rgb}{0.000000,0.000000,0.000000}%
\pgfsetfillcolor{currentfill}%
\pgfsetlinewidth{0.602250pt}%
\definecolor{currentstroke}{rgb}{0.000000,0.000000,0.000000}%
\pgfsetstrokecolor{currentstroke}%
\pgfsetdash{}{0pt}%
\pgfsys@defobject{currentmarker}{\pgfqpoint{0.000000in}{0.000000in}}{\pgfqpoint{0.027778in}{0.000000in}}{%
\pgfpathmoveto{\pgfqpoint{0.000000in}{0.000000in}}%
\pgfpathlineto{\pgfqpoint{0.027778in}{0.000000in}}%
\pgfusepath{stroke,fill}%
}%
\begin{pgfscope}%
\pgfsys@transformshift{5.504000in}{1.443990in}%
\pgfsys@useobject{currentmarker}{}%
\end{pgfscope}%
\end{pgfscope}%
\begin{pgfscope}%
\pgfsetbuttcap%
\pgfsetroundjoin%
\definecolor{currentfill}{rgb}{0.000000,0.000000,0.000000}%
\pgfsetfillcolor{currentfill}%
\pgfsetlinewidth{0.602250pt}%
\definecolor{currentstroke}{rgb}{0.000000,0.000000,0.000000}%
\pgfsetstrokecolor{currentstroke}%
\pgfsetdash{}{0pt}%
\pgfsys@defobject{currentmarker}{\pgfqpoint{0.000000in}{0.000000in}}{\pgfqpoint{0.027778in}{0.000000in}}{%
\pgfpathmoveto{\pgfqpoint{0.000000in}{0.000000in}}%
\pgfpathlineto{\pgfqpoint{0.027778in}{0.000000in}}%
\pgfusepath{stroke,fill}%
}%
\begin{pgfscope}%
\pgfsys@transformshift{5.504000in}{1.476897in}%
\pgfsys@useobject{currentmarker}{}%
\end{pgfscope}%
\end{pgfscope}%
\begin{pgfscope}%
\pgfsetbuttcap%
\pgfsetroundjoin%
\definecolor{currentfill}{rgb}{0.000000,0.000000,0.000000}%
\pgfsetfillcolor{currentfill}%
\pgfsetlinewidth{0.602250pt}%
\definecolor{currentstroke}{rgb}{0.000000,0.000000,0.000000}%
\pgfsetstrokecolor{currentstroke}%
\pgfsetdash{}{0pt}%
\pgfsys@defobject{currentmarker}{\pgfqpoint{0.000000in}{0.000000in}}{\pgfqpoint{0.027778in}{0.000000in}}{%
\pgfpathmoveto{\pgfqpoint{0.000000in}{0.000000in}}%
\pgfpathlineto{\pgfqpoint{0.027778in}{0.000000in}}%
\pgfusepath{stroke,fill}%
}%
\begin{pgfscope}%
\pgfsys@transformshift{5.504000in}{1.503783in}%
\pgfsys@useobject{currentmarker}{}%
\end{pgfscope}%
\end{pgfscope}%
\begin{pgfscope}%
\pgfsetbuttcap%
\pgfsetroundjoin%
\definecolor{currentfill}{rgb}{0.000000,0.000000,0.000000}%
\pgfsetfillcolor{currentfill}%
\pgfsetlinewidth{0.602250pt}%
\definecolor{currentstroke}{rgb}{0.000000,0.000000,0.000000}%
\pgfsetstrokecolor{currentstroke}%
\pgfsetdash{}{0pt}%
\pgfsys@defobject{currentmarker}{\pgfqpoint{0.000000in}{0.000000in}}{\pgfqpoint{0.027778in}{0.000000in}}{%
\pgfpathmoveto{\pgfqpoint{0.000000in}{0.000000in}}%
\pgfpathlineto{\pgfqpoint{0.027778in}{0.000000in}}%
\pgfusepath{stroke,fill}%
}%
\begin{pgfscope}%
\pgfsys@transformshift{5.504000in}{1.526515in}%
\pgfsys@useobject{currentmarker}{}%
\end{pgfscope}%
\end{pgfscope}%
\begin{pgfscope}%
\pgfsetbuttcap%
\pgfsetroundjoin%
\definecolor{currentfill}{rgb}{0.000000,0.000000,0.000000}%
\pgfsetfillcolor{currentfill}%
\pgfsetlinewidth{0.602250pt}%
\definecolor{currentstroke}{rgb}{0.000000,0.000000,0.000000}%
\pgfsetstrokecolor{currentstroke}%
\pgfsetdash{}{0pt}%
\pgfsys@defobject{currentmarker}{\pgfqpoint{0.000000in}{0.000000in}}{\pgfqpoint{0.027778in}{0.000000in}}{%
\pgfpathmoveto{\pgfqpoint{0.000000in}{0.000000in}}%
\pgfpathlineto{\pgfqpoint{0.027778in}{0.000000in}}%
\pgfusepath{stroke,fill}%
}%
\begin{pgfscope}%
\pgfsys@transformshift{5.504000in}{1.546207in}%
\pgfsys@useobject{currentmarker}{}%
\end{pgfscope}%
\end{pgfscope}%
\begin{pgfscope}%
\pgfsetbuttcap%
\pgfsetroundjoin%
\definecolor{currentfill}{rgb}{0.000000,0.000000,0.000000}%
\pgfsetfillcolor{currentfill}%
\pgfsetlinewidth{0.602250pt}%
\definecolor{currentstroke}{rgb}{0.000000,0.000000,0.000000}%
\pgfsetstrokecolor{currentstroke}%
\pgfsetdash{}{0pt}%
\pgfsys@defobject{currentmarker}{\pgfqpoint{0.000000in}{0.000000in}}{\pgfqpoint{0.027778in}{0.000000in}}{%
\pgfpathmoveto{\pgfqpoint{0.000000in}{0.000000in}}%
\pgfpathlineto{\pgfqpoint{0.027778in}{0.000000in}}%
\pgfusepath{stroke,fill}%
}%
\begin{pgfscope}%
\pgfsys@transformshift{5.504000in}{1.563576in}%
\pgfsys@useobject{currentmarker}{}%
\end{pgfscope}%
\end{pgfscope}%
\begin{pgfscope}%
\pgfsetbuttcap%
\pgfsetroundjoin%
\definecolor{currentfill}{rgb}{0.000000,0.000000,0.000000}%
\pgfsetfillcolor{currentfill}%
\pgfsetlinewidth{0.602250pt}%
\definecolor{currentstroke}{rgb}{0.000000,0.000000,0.000000}%
\pgfsetstrokecolor{currentstroke}%
\pgfsetdash{}{0pt}%
\pgfsys@defobject{currentmarker}{\pgfqpoint{0.000000in}{0.000000in}}{\pgfqpoint{0.027778in}{0.000000in}}{%
\pgfpathmoveto{\pgfqpoint{0.000000in}{0.000000in}}%
\pgfpathlineto{\pgfqpoint{0.027778in}{0.000000in}}%
\pgfusepath{stroke,fill}%
}%
\begin{pgfscope}%
\pgfsys@transformshift{5.504000in}{1.681330in}%
\pgfsys@useobject{currentmarker}{}%
\end{pgfscope}%
\end{pgfscope}%
\begin{pgfscope}%
\pgfsetbuttcap%
\pgfsetroundjoin%
\definecolor{currentfill}{rgb}{0.000000,0.000000,0.000000}%
\pgfsetfillcolor{currentfill}%
\pgfsetlinewidth{0.602250pt}%
\definecolor{currentstroke}{rgb}{0.000000,0.000000,0.000000}%
\pgfsetstrokecolor{currentstroke}%
\pgfsetdash{}{0pt}%
\pgfsys@defobject{currentmarker}{\pgfqpoint{0.000000in}{0.000000in}}{\pgfqpoint{0.027778in}{0.000000in}}{%
\pgfpathmoveto{\pgfqpoint{0.000000in}{0.000000in}}%
\pgfpathlineto{\pgfqpoint{0.027778in}{0.000000in}}%
\pgfusepath{stroke,fill}%
}%
\begin{pgfscope}%
\pgfsys@transformshift{5.504000in}{1.741123in}%
\pgfsys@useobject{currentmarker}{}%
\end{pgfscope}%
\end{pgfscope}%
\begin{pgfscope}%
\pgfsetbuttcap%
\pgfsetroundjoin%
\definecolor{currentfill}{rgb}{0.000000,0.000000,0.000000}%
\pgfsetfillcolor{currentfill}%
\pgfsetlinewidth{0.602250pt}%
\definecolor{currentstroke}{rgb}{0.000000,0.000000,0.000000}%
\pgfsetstrokecolor{currentstroke}%
\pgfsetdash{}{0pt}%
\pgfsys@defobject{currentmarker}{\pgfqpoint{0.000000in}{0.000000in}}{\pgfqpoint{0.027778in}{0.000000in}}{%
\pgfpathmoveto{\pgfqpoint{0.000000in}{0.000000in}}%
\pgfpathlineto{\pgfqpoint{0.027778in}{0.000000in}}%
\pgfusepath{stroke,fill}%
}%
\begin{pgfscope}%
\pgfsys@transformshift{5.504000in}{1.783547in}%
\pgfsys@useobject{currentmarker}{}%
\end{pgfscope}%
\end{pgfscope}%
\begin{pgfscope}%
\pgfsetbuttcap%
\pgfsetroundjoin%
\definecolor{currentfill}{rgb}{0.000000,0.000000,0.000000}%
\pgfsetfillcolor{currentfill}%
\pgfsetlinewidth{0.602250pt}%
\definecolor{currentstroke}{rgb}{0.000000,0.000000,0.000000}%
\pgfsetstrokecolor{currentstroke}%
\pgfsetdash{}{0pt}%
\pgfsys@defobject{currentmarker}{\pgfqpoint{0.000000in}{0.000000in}}{\pgfqpoint{0.027778in}{0.000000in}}{%
\pgfpathmoveto{\pgfqpoint{0.000000in}{0.000000in}}%
\pgfpathlineto{\pgfqpoint{0.027778in}{0.000000in}}%
\pgfusepath{stroke,fill}%
}%
\begin{pgfscope}%
\pgfsys@transformshift{5.504000in}{1.816453in}%
\pgfsys@useobject{currentmarker}{}%
\end{pgfscope}%
\end{pgfscope}%
\begin{pgfscope}%
\pgfsetbuttcap%
\pgfsetroundjoin%
\definecolor{currentfill}{rgb}{0.000000,0.000000,0.000000}%
\pgfsetfillcolor{currentfill}%
\pgfsetlinewidth{0.602250pt}%
\definecolor{currentstroke}{rgb}{0.000000,0.000000,0.000000}%
\pgfsetstrokecolor{currentstroke}%
\pgfsetdash{}{0pt}%
\pgfsys@defobject{currentmarker}{\pgfqpoint{0.000000in}{0.000000in}}{\pgfqpoint{0.027778in}{0.000000in}}{%
\pgfpathmoveto{\pgfqpoint{0.000000in}{0.000000in}}%
\pgfpathlineto{\pgfqpoint{0.027778in}{0.000000in}}%
\pgfusepath{stroke,fill}%
}%
\begin{pgfscope}%
\pgfsys@transformshift{5.504000in}{1.843340in}%
\pgfsys@useobject{currentmarker}{}%
\end{pgfscope}%
\end{pgfscope}%
\begin{pgfscope}%
\pgfsetbuttcap%
\pgfsetroundjoin%
\definecolor{currentfill}{rgb}{0.000000,0.000000,0.000000}%
\pgfsetfillcolor{currentfill}%
\pgfsetlinewidth{0.602250pt}%
\definecolor{currentstroke}{rgb}{0.000000,0.000000,0.000000}%
\pgfsetstrokecolor{currentstroke}%
\pgfsetdash{}{0pt}%
\pgfsys@defobject{currentmarker}{\pgfqpoint{0.000000in}{0.000000in}}{\pgfqpoint{0.027778in}{0.000000in}}{%
\pgfpathmoveto{\pgfqpoint{0.000000in}{0.000000in}}%
\pgfpathlineto{\pgfqpoint{0.027778in}{0.000000in}}%
\pgfusepath{stroke,fill}%
}%
\begin{pgfscope}%
\pgfsys@transformshift{5.504000in}{1.866072in}%
\pgfsys@useobject{currentmarker}{}%
\end{pgfscope}%
\end{pgfscope}%
\begin{pgfscope}%
\pgfsetbuttcap%
\pgfsetroundjoin%
\definecolor{currentfill}{rgb}{0.000000,0.000000,0.000000}%
\pgfsetfillcolor{currentfill}%
\pgfsetlinewidth{0.602250pt}%
\definecolor{currentstroke}{rgb}{0.000000,0.000000,0.000000}%
\pgfsetstrokecolor{currentstroke}%
\pgfsetdash{}{0pt}%
\pgfsys@defobject{currentmarker}{\pgfqpoint{0.000000in}{0.000000in}}{\pgfqpoint{0.027778in}{0.000000in}}{%
\pgfpathmoveto{\pgfqpoint{0.000000in}{0.000000in}}%
\pgfpathlineto{\pgfqpoint{0.027778in}{0.000000in}}%
\pgfusepath{stroke,fill}%
}%
\begin{pgfscope}%
\pgfsys@transformshift{5.504000in}{1.885764in}%
\pgfsys@useobject{currentmarker}{}%
\end{pgfscope}%
\end{pgfscope}%
\begin{pgfscope}%
\pgfsetbuttcap%
\pgfsetroundjoin%
\definecolor{currentfill}{rgb}{0.000000,0.000000,0.000000}%
\pgfsetfillcolor{currentfill}%
\pgfsetlinewidth{0.602250pt}%
\definecolor{currentstroke}{rgb}{0.000000,0.000000,0.000000}%
\pgfsetstrokecolor{currentstroke}%
\pgfsetdash{}{0pt}%
\pgfsys@defobject{currentmarker}{\pgfqpoint{0.000000in}{0.000000in}}{\pgfqpoint{0.027778in}{0.000000in}}{%
\pgfpathmoveto{\pgfqpoint{0.000000in}{0.000000in}}%
\pgfpathlineto{\pgfqpoint{0.027778in}{0.000000in}}%
\pgfusepath{stroke,fill}%
}%
\begin{pgfscope}%
\pgfsys@transformshift{5.504000in}{1.903133in}%
\pgfsys@useobject{currentmarker}{}%
\end{pgfscope}%
\end{pgfscope}%
\begin{pgfscope}%
\pgfsetbuttcap%
\pgfsetroundjoin%
\definecolor{currentfill}{rgb}{0.000000,0.000000,0.000000}%
\pgfsetfillcolor{currentfill}%
\pgfsetlinewidth{0.602250pt}%
\definecolor{currentstroke}{rgb}{0.000000,0.000000,0.000000}%
\pgfsetstrokecolor{currentstroke}%
\pgfsetdash{}{0pt}%
\pgfsys@defobject{currentmarker}{\pgfqpoint{0.000000in}{0.000000in}}{\pgfqpoint{0.027778in}{0.000000in}}{%
\pgfpathmoveto{\pgfqpoint{0.000000in}{0.000000in}}%
\pgfpathlineto{\pgfqpoint{0.027778in}{0.000000in}}%
\pgfusepath{stroke,fill}%
}%
\begin{pgfscope}%
\pgfsys@transformshift{5.504000in}{2.020887in}%
\pgfsys@useobject{currentmarker}{}%
\end{pgfscope}%
\end{pgfscope}%
\begin{pgfscope}%
\pgfsetbuttcap%
\pgfsetroundjoin%
\definecolor{currentfill}{rgb}{0.000000,0.000000,0.000000}%
\pgfsetfillcolor{currentfill}%
\pgfsetlinewidth{0.602250pt}%
\definecolor{currentstroke}{rgb}{0.000000,0.000000,0.000000}%
\pgfsetstrokecolor{currentstroke}%
\pgfsetdash{}{0pt}%
\pgfsys@defobject{currentmarker}{\pgfqpoint{0.000000in}{0.000000in}}{\pgfqpoint{0.027778in}{0.000000in}}{%
\pgfpathmoveto{\pgfqpoint{0.000000in}{0.000000in}}%
\pgfpathlineto{\pgfqpoint{0.027778in}{0.000000in}}%
\pgfusepath{stroke,fill}%
}%
\begin{pgfscope}%
\pgfsys@transformshift{5.504000in}{2.080680in}%
\pgfsys@useobject{currentmarker}{}%
\end{pgfscope}%
\end{pgfscope}%
\begin{pgfscope}%
\pgfsetbuttcap%
\pgfsetroundjoin%
\definecolor{currentfill}{rgb}{0.000000,0.000000,0.000000}%
\pgfsetfillcolor{currentfill}%
\pgfsetlinewidth{0.602250pt}%
\definecolor{currentstroke}{rgb}{0.000000,0.000000,0.000000}%
\pgfsetstrokecolor{currentstroke}%
\pgfsetdash{}{0pt}%
\pgfsys@defobject{currentmarker}{\pgfqpoint{0.000000in}{0.000000in}}{\pgfqpoint{0.027778in}{0.000000in}}{%
\pgfpathmoveto{\pgfqpoint{0.000000in}{0.000000in}}%
\pgfpathlineto{\pgfqpoint{0.027778in}{0.000000in}}%
\pgfusepath{stroke,fill}%
}%
\begin{pgfscope}%
\pgfsys@transformshift{5.504000in}{2.123103in}%
\pgfsys@useobject{currentmarker}{}%
\end{pgfscope}%
\end{pgfscope}%
\begin{pgfscope}%
\pgfsetbuttcap%
\pgfsetroundjoin%
\definecolor{currentfill}{rgb}{0.000000,0.000000,0.000000}%
\pgfsetfillcolor{currentfill}%
\pgfsetlinewidth{0.602250pt}%
\definecolor{currentstroke}{rgb}{0.000000,0.000000,0.000000}%
\pgfsetstrokecolor{currentstroke}%
\pgfsetdash{}{0pt}%
\pgfsys@defobject{currentmarker}{\pgfqpoint{0.000000in}{0.000000in}}{\pgfqpoint{0.027778in}{0.000000in}}{%
\pgfpathmoveto{\pgfqpoint{0.000000in}{0.000000in}}%
\pgfpathlineto{\pgfqpoint{0.027778in}{0.000000in}}%
\pgfusepath{stroke,fill}%
}%
\begin{pgfscope}%
\pgfsys@transformshift{5.504000in}{2.156010in}%
\pgfsys@useobject{currentmarker}{}%
\end{pgfscope}%
\end{pgfscope}%
\begin{pgfscope}%
\pgfsetbuttcap%
\pgfsetroundjoin%
\definecolor{currentfill}{rgb}{0.000000,0.000000,0.000000}%
\pgfsetfillcolor{currentfill}%
\pgfsetlinewidth{0.602250pt}%
\definecolor{currentstroke}{rgb}{0.000000,0.000000,0.000000}%
\pgfsetstrokecolor{currentstroke}%
\pgfsetdash{}{0pt}%
\pgfsys@defobject{currentmarker}{\pgfqpoint{0.000000in}{0.000000in}}{\pgfqpoint{0.027778in}{0.000000in}}{%
\pgfpathmoveto{\pgfqpoint{0.000000in}{0.000000in}}%
\pgfpathlineto{\pgfqpoint{0.027778in}{0.000000in}}%
\pgfusepath{stroke,fill}%
}%
\begin{pgfscope}%
\pgfsys@transformshift{5.504000in}{2.182896in}%
\pgfsys@useobject{currentmarker}{}%
\end{pgfscope}%
\end{pgfscope}%
\begin{pgfscope}%
\pgfsetbuttcap%
\pgfsetroundjoin%
\definecolor{currentfill}{rgb}{0.000000,0.000000,0.000000}%
\pgfsetfillcolor{currentfill}%
\pgfsetlinewidth{0.602250pt}%
\definecolor{currentstroke}{rgb}{0.000000,0.000000,0.000000}%
\pgfsetstrokecolor{currentstroke}%
\pgfsetdash{}{0pt}%
\pgfsys@defobject{currentmarker}{\pgfqpoint{0.000000in}{0.000000in}}{\pgfqpoint{0.027778in}{0.000000in}}{%
\pgfpathmoveto{\pgfqpoint{0.000000in}{0.000000in}}%
\pgfpathlineto{\pgfqpoint{0.027778in}{0.000000in}}%
\pgfusepath{stroke,fill}%
}%
\begin{pgfscope}%
\pgfsys@transformshift{5.504000in}{2.205629in}%
\pgfsys@useobject{currentmarker}{}%
\end{pgfscope}%
\end{pgfscope}%
\begin{pgfscope}%
\pgfsetbuttcap%
\pgfsetroundjoin%
\definecolor{currentfill}{rgb}{0.000000,0.000000,0.000000}%
\pgfsetfillcolor{currentfill}%
\pgfsetlinewidth{0.602250pt}%
\definecolor{currentstroke}{rgb}{0.000000,0.000000,0.000000}%
\pgfsetstrokecolor{currentstroke}%
\pgfsetdash{}{0pt}%
\pgfsys@defobject{currentmarker}{\pgfqpoint{0.000000in}{0.000000in}}{\pgfqpoint{0.027778in}{0.000000in}}{%
\pgfpathmoveto{\pgfqpoint{0.000000in}{0.000000in}}%
\pgfpathlineto{\pgfqpoint{0.027778in}{0.000000in}}%
\pgfusepath{stroke,fill}%
}%
\begin{pgfscope}%
\pgfsys@transformshift{5.504000in}{2.225320in}%
\pgfsys@useobject{currentmarker}{}%
\end{pgfscope}%
\end{pgfscope}%
\begin{pgfscope}%
\pgfsetbuttcap%
\pgfsetroundjoin%
\definecolor{currentfill}{rgb}{0.000000,0.000000,0.000000}%
\pgfsetfillcolor{currentfill}%
\pgfsetlinewidth{0.602250pt}%
\definecolor{currentstroke}{rgb}{0.000000,0.000000,0.000000}%
\pgfsetstrokecolor{currentstroke}%
\pgfsetdash{}{0pt}%
\pgfsys@defobject{currentmarker}{\pgfqpoint{0.000000in}{0.000000in}}{\pgfqpoint{0.027778in}{0.000000in}}{%
\pgfpathmoveto{\pgfqpoint{0.000000in}{0.000000in}}%
\pgfpathlineto{\pgfqpoint{0.027778in}{0.000000in}}%
\pgfusepath{stroke,fill}%
}%
\begin{pgfscope}%
\pgfsys@transformshift{5.504000in}{2.242689in}%
\pgfsys@useobject{currentmarker}{}%
\end{pgfscope}%
\end{pgfscope}%
\begin{pgfscope}%
\pgfsetbuttcap%
\pgfsetroundjoin%
\definecolor{currentfill}{rgb}{0.000000,0.000000,0.000000}%
\pgfsetfillcolor{currentfill}%
\pgfsetlinewidth{0.602250pt}%
\definecolor{currentstroke}{rgb}{0.000000,0.000000,0.000000}%
\pgfsetstrokecolor{currentstroke}%
\pgfsetdash{}{0pt}%
\pgfsys@defobject{currentmarker}{\pgfqpoint{0.000000in}{0.000000in}}{\pgfqpoint{0.027778in}{0.000000in}}{%
\pgfpathmoveto{\pgfqpoint{0.000000in}{0.000000in}}%
\pgfpathlineto{\pgfqpoint{0.027778in}{0.000000in}}%
\pgfusepath{stroke,fill}%
}%
\begin{pgfscope}%
\pgfsys@transformshift{5.504000in}{2.360443in}%
\pgfsys@useobject{currentmarker}{}%
\end{pgfscope}%
\end{pgfscope}%
\begin{pgfscope}%
\pgfsetbuttcap%
\pgfsetroundjoin%
\definecolor{currentfill}{rgb}{0.000000,0.000000,0.000000}%
\pgfsetfillcolor{currentfill}%
\pgfsetlinewidth{0.602250pt}%
\definecolor{currentstroke}{rgb}{0.000000,0.000000,0.000000}%
\pgfsetstrokecolor{currentstroke}%
\pgfsetdash{}{0pt}%
\pgfsys@defobject{currentmarker}{\pgfqpoint{0.000000in}{0.000000in}}{\pgfqpoint{0.027778in}{0.000000in}}{%
\pgfpathmoveto{\pgfqpoint{0.000000in}{0.000000in}}%
\pgfpathlineto{\pgfqpoint{0.027778in}{0.000000in}}%
\pgfusepath{stroke,fill}%
}%
\begin{pgfscope}%
\pgfsys@transformshift{5.504000in}{2.420236in}%
\pgfsys@useobject{currentmarker}{}%
\end{pgfscope}%
\end{pgfscope}%
\begin{pgfscope}%
\pgfsetbuttcap%
\pgfsetroundjoin%
\definecolor{currentfill}{rgb}{0.000000,0.000000,0.000000}%
\pgfsetfillcolor{currentfill}%
\pgfsetlinewidth{0.602250pt}%
\definecolor{currentstroke}{rgb}{0.000000,0.000000,0.000000}%
\pgfsetstrokecolor{currentstroke}%
\pgfsetdash{}{0pt}%
\pgfsys@defobject{currentmarker}{\pgfqpoint{0.000000in}{0.000000in}}{\pgfqpoint{0.027778in}{0.000000in}}{%
\pgfpathmoveto{\pgfqpoint{0.000000in}{0.000000in}}%
\pgfpathlineto{\pgfqpoint{0.027778in}{0.000000in}}%
\pgfusepath{stroke,fill}%
}%
\begin{pgfscope}%
\pgfsys@transformshift{5.504000in}{2.462660in}%
\pgfsys@useobject{currentmarker}{}%
\end{pgfscope}%
\end{pgfscope}%
\begin{pgfscope}%
\pgfsetbuttcap%
\pgfsetroundjoin%
\definecolor{currentfill}{rgb}{0.000000,0.000000,0.000000}%
\pgfsetfillcolor{currentfill}%
\pgfsetlinewidth{0.602250pt}%
\definecolor{currentstroke}{rgb}{0.000000,0.000000,0.000000}%
\pgfsetstrokecolor{currentstroke}%
\pgfsetdash{}{0pt}%
\pgfsys@defobject{currentmarker}{\pgfqpoint{0.000000in}{0.000000in}}{\pgfqpoint{0.027778in}{0.000000in}}{%
\pgfpathmoveto{\pgfqpoint{0.000000in}{0.000000in}}%
\pgfpathlineto{\pgfqpoint{0.027778in}{0.000000in}}%
\pgfusepath{stroke,fill}%
}%
\begin{pgfscope}%
\pgfsys@transformshift{5.504000in}{2.495567in}%
\pgfsys@useobject{currentmarker}{}%
\end{pgfscope}%
\end{pgfscope}%
\begin{pgfscope}%
\pgfsetbuttcap%
\pgfsetroundjoin%
\definecolor{currentfill}{rgb}{0.000000,0.000000,0.000000}%
\pgfsetfillcolor{currentfill}%
\pgfsetlinewidth{0.602250pt}%
\definecolor{currentstroke}{rgb}{0.000000,0.000000,0.000000}%
\pgfsetstrokecolor{currentstroke}%
\pgfsetdash{}{0pt}%
\pgfsys@defobject{currentmarker}{\pgfqpoint{0.000000in}{0.000000in}}{\pgfqpoint{0.027778in}{0.000000in}}{%
\pgfpathmoveto{\pgfqpoint{0.000000in}{0.000000in}}%
\pgfpathlineto{\pgfqpoint{0.027778in}{0.000000in}}%
\pgfusepath{stroke,fill}%
}%
\begin{pgfscope}%
\pgfsys@transformshift{5.504000in}{2.522453in}%
\pgfsys@useobject{currentmarker}{}%
\end{pgfscope}%
\end{pgfscope}%
\begin{pgfscope}%
\pgfsetbuttcap%
\pgfsetroundjoin%
\definecolor{currentfill}{rgb}{0.000000,0.000000,0.000000}%
\pgfsetfillcolor{currentfill}%
\pgfsetlinewidth{0.602250pt}%
\definecolor{currentstroke}{rgb}{0.000000,0.000000,0.000000}%
\pgfsetstrokecolor{currentstroke}%
\pgfsetdash{}{0pt}%
\pgfsys@defobject{currentmarker}{\pgfqpoint{0.000000in}{0.000000in}}{\pgfqpoint{0.027778in}{0.000000in}}{%
\pgfpathmoveto{\pgfqpoint{0.000000in}{0.000000in}}%
\pgfpathlineto{\pgfqpoint{0.027778in}{0.000000in}}%
\pgfusepath{stroke,fill}%
}%
\begin{pgfscope}%
\pgfsys@transformshift{5.504000in}{2.545185in}%
\pgfsys@useobject{currentmarker}{}%
\end{pgfscope}%
\end{pgfscope}%
\begin{pgfscope}%
\pgfsetbuttcap%
\pgfsetroundjoin%
\definecolor{currentfill}{rgb}{0.000000,0.000000,0.000000}%
\pgfsetfillcolor{currentfill}%
\pgfsetlinewidth{0.602250pt}%
\definecolor{currentstroke}{rgb}{0.000000,0.000000,0.000000}%
\pgfsetstrokecolor{currentstroke}%
\pgfsetdash{}{0pt}%
\pgfsys@defobject{currentmarker}{\pgfqpoint{0.000000in}{0.000000in}}{\pgfqpoint{0.027778in}{0.000000in}}{%
\pgfpathmoveto{\pgfqpoint{0.000000in}{0.000000in}}%
\pgfpathlineto{\pgfqpoint{0.027778in}{0.000000in}}%
\pgfusepath{stroke,fill}%
}%
\begin{pgfscope}%
\pgfsys@transformshift{5.504000in}{2.564877in}%
\pgfsys@useobject{currentmarker}{}%
\end{pgfscope}%
\end{pgfscope}%
\begin{pgfscope}%
\pgfsetbuttcap%
\pgfsetroundjoin%
\definecolor{currentfill}{rgb}{0.000000,0.000000,0.000000}%
\pgfsetfillcolor{currentfill}%
\pgfsetlinewidth{0.602250pt}%
\definecolor{currentstroke}{rgb}{0.000000,0.000000,0.000000}%
\pgfsetstrokecolor{currentstroke}%
\pgfsetdash{}{0pt}%
\pgfsys@defobject{currentmarker}{\pgfqpoint{0.000000in}{0.000000in}}{\pgfqpoint{0.027778in}{0.000000in}}{%
\pgfpathmoveto{\pgfqpoint{0.000000in}{0.000000in}}%
\pgfpathlineto{\pgfqpoint{0.027778in}{0.000000in}}%
\pgfusepath{stroke,fill}%
}%
\begin{pgfscope}%
\pgfsys@transformshift{5.504000in}{2.582246in}%
\pgfsys@useobject{currentmarker}{}%
\end{pgfscope}%
\end{pgfscope}%
\begin{pgfscope}%
\pgfsetbuttcap%
\pgfsetroundjoin%
\definecolor{currentfill}{rgb}{0.000000,0.000000,0.000000}%
\pgfsetfillcolor{currentfill}%
\pgfsetlinewidth{0.602250pt}%
\definecolor{currentstroke}{rgb}{0.000000,0.000000,0.000000}%
\pgfsetstrokecolor{currentstroke}%
\pgfsetdash{}{0pt}%
\pgfsys@defobject{currentmarker}{\pgfqpoint{0.000000in}{0.000000in}}{\pgfqpoint{0.027778in}{0.000000in}}{%
\pgfpathmoveto{\pgfqpoint{0.000000in}{0.000000in}}%
\pgfpathlineto{\pgfqpoint{0.027778in}{0.000000in}}%
\pgfusepath{stroke,fill}%
}%
\begin{pgfscope}%
\pgfsys@transformshift{5.504000in}{2.700000in}%
\pgfsys@useobject{currentmarker}{}%
\end{pgfscope}%
\end{pgfscope}%
\begin{pgfscope}%
\definecolor{textcolor}{rgb}{0.000000,0.000000,0.000000}%
\pgfsetstrokecolor{textcolor}%
\pgfsetfillcolor{textcolor}%
\pgftext[x=5.944780in,y=1.800000in,,top,rotate=90.000000]{\color{textcolor}\sffamily\fontsize{10.000000}{12.000000}\selectfont \(\displaystyle -dQ_{e^-}/dx/dy/dz \, \mathrm{[pC/\mu m^3]}\)}%
\end{pgfscope}%
\begin{pgfscope}%
\pgfsetrectcap%
\pgfsetmiterjoin%
\pgfsetlinewidth{0.803000pt}%
\definecolor{currentstroke}{rgb}{0.000000,0.000000,0.000000}%
\pgfsetstrokecolor{currentstroke}%
\pgfsetdash{}{0pt}%
\pgfpathmoveto{\pgfqpoint{5.312000in}{0.900000in}}%
\pgfpathlineto{\pgfqpoint{5.408000in}{0.900000in}}%
\pgfpathlineto{\pgfqpoint{5.504000in}{0.900000in}}%
\pgfpathlineto{\pgfqpoint{5.504000in}{2.700000in}}%
\pgfpathlineto{\pgfqpoint{5.408000in}{2.700000in}}%
\pgfpathlineto{\pgfqpoint{5.312000in}{2.700000in}}%
\pgfpathlineto{\pgfqpoint{5.312000in}{0.900000in}}%
\pgfpathclose%
\pgfusepath{stroke}%
\end{pgfscope}%
\begin{pgfscope}%
\pgfsetbuttcap%
\pgfsetmiterjoin%
\definecolor{currentfill}{rgb}{1.000000,1.000000,1.000000}%
\pgfsetfillcolor{currentfill}%
\pgfsetlinewidth{0.000000pt}%
\definecolor{currentstroke}{rgb}{0.000000,0.000000,0.000000}%
\pgfsetstrokecolor{currentstroke}%
\pgfsetstrokeopacity{0.000000}%
\pgfsetdash{}{0pt}%
\pgfpathmoveto{\pgfqpoint{5.312000in}{3.360000in}}%
\pgfpathlineto{\pgfqpoint{5.504000in}{3.360000in}}%
\pgfpathlineto{\pgfqpoint{5.504000in}{5.160000in}}%
\pgfpathlineto{\pgfqpoint{5.312000in}{5.160000in}}%
\pgfpathlineto{\pgfqpoint{5.312000in}{3.360000in}}%
\pgfpathclose%
\pgfusepath{fill}%
\end{pgfscope}%
\begin{pgfscope}%
\pgfpathrectangle{\pgfqpoint{5.312000in}{3.360000in}}{\pgfqpoint{0.192000in}{1.800000in}}%
\pgfusepath{clip}%
\pgfsetbuttcap%
\pgfsetmiterjoin%
\definecolor{currentfill}{rgb}{1.000000,1.000000,1.000000}%
\pgfsetfillcolor{currentfill}%
\pgfsetlinewidth{0.010037pt}%
\definecolor{currentstroke}{rgb}{1.000000,1.000000,1.000000}%
\pgfsetstrokecolor{currentstroke}%
\pgfsetdash{}{0pt}%
\pgfusepath{stroke,fill}%
\end{pgfscope}%
\begin{pgfscope}%
\pgfsys@transformshift{5.305556in}{3.361111in}%
\pgftext[left,bottom]{\includegraphics[interpolate=true,width=0.194444in,height=1.805556in]{cavity_low-img4.png}}%
\end{pgfscope}%
\begin{pgfscope}%
\pgfsetbuttcap%
\pgfsetroundjoin%
\definecolor{currentfill}{rgb}{0.000000,0.000000,0.000000}%
\pgfsetfillcolor{currentfill}%
\pgfsetlinewidth{0.803000pt}%
\definecolor{currentstroke}{rgb}{0.000000,0.000000,0.000000}%
\pgfsetstrokecolor{currentstroke}%
\pgfsetdash{}{0pt}%
\pgfsys@defobject{currentmarker}{\pgfqpoint{0.000000in}{0.000000in}}{\pgfqpoint{0.048611in}{0.000000in}}{%
\pgfpathmoveto{\pgfqpoint{0.000000in}{0.000000in}}%
\pgfpathlineto{\pgfqpoint{0.048611in}{0.000000in}}%
\pgfusepath{stroke,fill}%
}%
\begin{pgfscope}%
\pgfsys@transformshift{5.504000in}{3.360000in}%
\pgfsys@useobject{currentmarker}{}%
\end{pgfscope}%
\end{pgfscope}%
\begin{pgfscope}%
\definecolor{textcolor}{rgb}{0.000000,0.000000,0.000000}%
\pgfsetstrokecolor{textcolor}%
\pgfsetfillcolor{textcolor}%
\pgftext[x=5.601222in, y=3.311775in, left, base]{\color{textcolor}\sffamily\fontsize{10.000000}{12.000000}\selectfont \(\displaystyle {10^{-3}}\)}%
\end{pgfscope}%
\begin{pgfscope}%
\pgfsetbuttcap%
\pgfsetroundjoin%
\definecolor{currentfill}{rgb}{0.000000,0.000000,0.000000}%
\pgfsetfillcolor{currentfill}%
\pgfsetlinewidth{0.803000pt}%
\definecolor{currentstroke}{rgb}{0.000000,0.000000,0.000000}%
\pgfsetstrokecolor{currentstroke}%
\pgfsetdash{}{0pt}%
\pgfsys@defobject{currentmarker}{\pgfqpoint{0.000000in}{0.000000in}}{\pgfqpoint{0.048611in}{0.000000in}}{%
\pgfpathmoveto{\pgfqpoint{0.000000in}{0.000000in}}%
\pgfpathlineto{\pgfqpoint{0.048611in}{0.000000in}}%
\pgfusepath{stroke,fill}%
}%
\begin{pgfscope}%
\pgfsys@transformshift{5.504000in}{3.810000in}%
\pgfsys@useobject{currentmarker}{}%
\end{pgfscope}%
\end{pgfscope}%
\begin{pgfscope}%
\definecolor{textcolor}{rgb}{0.000000,0.000000,0.000000}%
\pgfsetstrokecolor{textcolor}%
\pgfsetfillcolor{textcolor}%
\pgftext[x=5.601222in, y=3.761775in, left, base]{\color{textcolor}\sffamily\fontsize{10.000000}{12.000000}\selectfont \(\displaystyle {10^{-2}}\)}%
\end{pgfscope}%
\begin{pgfscope}%
\pgfsetbuttcap%
\pgfsetroundjoin%
\definecolor{currentfill}{rgb}{0.000000,0.000000,0.000000}%
\pgfsetfillcolor{currentfill}%
\pgfsetlinewidth{0.803000pt}%
\definecolor{currentstroke}{rgb}{0.000000,0.000000,0.000000}%
\pgfsetstrokecolor{currentstroke}%
\pgfsetdash{}{0pt}%
\pgfsys@defobject{currentmarker}{\pgfqpoint{0.000000in}{0.000000in}}{\pgfqpoint{0.048611in}{0.000000in}}{%
\pgfpathmoveto{\pgfqpoint{0.000000in}{0.000000in}}%
\pgfpathlineto{\pgfqpoint{0.048611in}{0.000000in}}%
\pgfusepath{stroke,fill}%
}%
\begin{pgfscope}%
\pgfsys@transformshift{5.504000in}{4.260000in}%
\pgfsys@useobject{currentmarker}{}%
\end{pgfscope}%
\end{pgfscope}%
\begin{pgfscope}%
\definecolor{textcolor}{rgb}{0.000000,0.000000,0.000000}%
\pgfsetstrokecolor{textcolor}%
\pgfsetfillcolor{textcolor}%
\pgftext[x=5.601222in, y=4.211775in, left, base]{\color{textcolor}\sffamily\fontsize{10.000000}{12.000000}\selectfont \(\displaystyle {10^{-1}}\)}%
\end{pgfscope}%
\begin{pgfscope}%
\pgfsetbuttcap%
\pgfsetroundjoin%
\definecolor{currentfill}{rgb}{0.000000,0.000000,0.000000}%
\pgfsetfillcolor{currentfill}%
\pgfsetlinewidth{0.803000pt}%
\definecolor{currentstroke}{rgb}{0.000000,0.000000,0.000000}%
\pgfsetstrokecolor{currentstroke}%
\pgfsetdash{}{0pt}%
\pgfsys@defobject{currentmarker}{\pgfqpoint{0.000000in}{0.000000in}}{\pgfqpoint{0.048611in}{0.000000in}}{%
\pgfpathmoveto{\pgfqpoint{0.000000in}{0.000000in}}%
\pgfpathlineto{\pgfqpoint{0.048611in}{0.000000in}}%
\pgfusepath{stroke,fill}%
}%
\begin{pgfscope}%
\pgfsys@transformshift{5.504000in}{4.710000in}%
\pgfsys@useobject{currentmarker}{}%
\end{pgfscope}%
\end{pgfscope}%
\begin{pgfscope}%
\definecolor{textcolor}{rgb}{0.000000,0.000000,0.000000}%
\pgfsetstrokecolor{textcolor}%
\pgfsetfillcolor{textcolor}%
\pgftext[x=5.601222in, y=4.661775in, left, base]{\color{textcolor}\sffamily\fontsize{10.000000}{12.000000}\selectfont \(\displaystyle {10^{0}}\)}%
\end{pgfscope}%
\begin{pgfscope}%
\pgfsetbuttcap%
\pgfsetroundjoin%
\definecolor{currentfill}{rgb}{0.000000,0.000000,0.000000}%
\pgfsetfillcolor{currentfill}%
\pgfsetlinewidth{0.803000pt}%
\definecolor{currentstroke}{rgb}{0.000000,0.000000,0.000000}%
\pgfsetstrokecolor{currentstroke}%
\pgfsetdash{}{0pt}%
\pgfsys@defobject{currentmarker}{\pgfqpoint{0.000000in}{0.000000in}}{\pgfqpoint{0.048611in}{0.000000in}}{%
\pgfpathmoveto{\pgfqpoint{0.000000in}{0.000000in}}%
\pgfpathlineto{\pgfqpoint{0.048611in}{0.000000in}}%
\pgfusepath{stroke,fill}%
}%
\begin{pgfscope}%
\pgfsys@transformshift{5.504000in}{5.160000in}%
\pgfsys@useobject{currentmarker}{}%
\end{pgfscope}%
\end{pgfscope}%
\begin{pgfscope}%
\definecolor{textcolor}{rgb}{0.000000,0.000000,0.000000}%
\pgfsetstrokecolor{textcolor}%
\pgfsetfillcolor{textcolor}%
\pgftext[x=5.601222in, y=5.111775in, left, base]{\color{textcolor}\sffamily\fontsize{10.000000}{12.000000}\selectfont \(\displaystyle {10^{1}}\)}%
\end{pgfscope}%
\begin{pgfscope}%
\pgfsetbuttcap%
\pgfsetroundjoin%
\definecolor{currentfill}{rgb}{0.000000,0.000000,0.000000}%
\pgfsetfillcolor{currentfill}%
\pgfsetlinewidth{0.602250pt}%
\definecolor{currentstroke}{rgb}{0.000000,0.000000,0.000000}%
\pgfsetstrokecolor{currentstroke}%
\pgfsetdash{}{0pt}%
\pgfsys@defobject{currentmarker}{\pgfqpoint{0.000000in}{0.000000in}}{\pgfqpoint{0.027778in}{0.000000in}}{%
\pgfpathmoveto{\pgfqpoint{0.000000in}{0.000000in}}%
\pgfpathlineto{\pgfqpoint{0.027778in}{0.000000in}}%
\pgfusepath{stroke,fill}%
}%
\begin{pgfscope}%
\pgfsys@transformshift{5.504000in}{3.495463in}%
\pgfsys@useobject{currentmarker}{}%
\end{pgfscope}%
\end{pgfscope}%
\begin{pgfscope}%
\pgfsetbuttcap%
\pgfsetroundjoin%
\definecolor{currentfill}{rgb}{0.000000,0.000000,0.000000}%
\pgfsetfillcolor{currentfill}%
\pgfsetlinewidth{0.602250pt}%
\definecolor{currentstroke}{rgb}{0.000000,0.000000,0.000000}%
\pgfsetstrokecolor{currentstroke}%
\pgfsetdash{}{0pt}%
\pgfsys@defobject{currentmarker}{\pgfqpoint{0.000000in}{0.000000in}}{\pgfqpoint{0.027778in}{0.000000in}}{%
\pgfpathmoveto{\pgfqpoint{0.000000in}{0.000000in}}%
\pgfpathlineto{\pgfqpoint{0.027778in}{0.000000in}}%
\pgfusepath{stroke,fill}%
}%
\begin{pgfscope}%
\pgfsys@transformshift{5.504000in}{3.574705in}%
\pgfsys@useobject{currentmarker}{}%
\end{pgfscope}%
\end{pgfscope}%
\begin{pgfscope}%
\pgfsetbuttcap%
\pgfsetroundjoin%
\definecolor{currentfill}{rgb}{0.000000,0.000000,0.000000}%
\pgfsetfillcolor{currentfill}%
\pgfsetlinewidth{0.602250pt}%
\definecolor{currentstroke}{rgb}{0.000000,0.000000,0.000000}%
\pgfsetstrokecolor{currentstroke}%
\pgfsetdash{}{0pt}%
\pgfsys@defobject{currentmarker}{\pgfqpoint{0.000000in}{0.000000in}}{\pgfqpoint{0.027778in}{0.000000in}}{%
\pgfpathmoveto{\pgfqpoint{0.000000in}{0.000000in}}%
\pgfpathlineto{\pgfqpoint{0.027778in}{0.000000in}}%
\pgfusepath{stroke,fill}%
}%
\begin{pgfscope}%
\pgfsys@transformshift{5.504000in}{3.630927in}%
\pgfsys@useobject{currentmarker}{}%
\end{pgfscope}%
\end{pgfscope}%
\begin{pgfscope}%
\pgfsetbuttcap%
\pgfsetroundjoin%
\definecolor{currentfill}{rgb}{0.000000,0.000000,0.000000}%
\pgfsetfillcolor{currentfill}%
\pgfsetlinewidth{0.602250pt}%
\definecolor{currentstroke}{rgb}{0.000000,0.000000,0.000000}%
\pgfsetstrokecolor{currentstroke}%
\pgfsetdash{}{0pt}%
\pgfsys@defobject{currentmarker}{\pgfqpoint{0.000000in}{0.000000in}}{\pgfqpoint{0.027778in}{0.000000in}}{%
\pgfpathmoveto{\pgfqpoint{0.000000in}{0.000000in}}%
\pgfpathlineto{\pgfqpoint{0.027778in}{0.000000in}}%
\pgfusepath{stroke,fill}%
}%
\begin{pgfscope}%
\pgfsys@transformshift{5.504000in}{3.674537in}%
\pgfsys@useobject{currentmarker}{}%
\end{pgfscope}%
\end{pgfscope}%
\begin{pgfscope}%
\pgfsetbuttcap%
\pgfsetroundjoin%
\definecolor{currentfill}{rgb}{0.000000,0.000000,0.000000}%
\pgfsetfillcolor{currentfill}%
\pgfsetlinewidth{0.602250pt}%
\definecolor{currentstroke}{rgb}{0.000000,0.000000,0.000000}%
\pgfsetstrokecolor{currentstroke}%
\pgfsetdash{}{0pt}%
\pgfsys@defobject{currentmarker}{\pgfqpoint{0.000000in}{0.000000in}}{\pgfqpoint{0.027778in}{0.000000in}}{%
\pgfpathmoveto{\pgfqpoint{0.000000in}{0.000000in}}%
\pgfpathlineto{\pgfqpoint{0.027778in}{0.000000in}}%
\pgfusepath{stroke,fill}%
}%
\begin{pgfscope}%
\pgfsys@transformshift{5.504000in}{3.710168in}%
\pgfsys@useobject{currentmarker}{}%
\end{pgfscope}%
\end{pgfscope}%
\begin{pgfscope}%
\pgfsetbuttcap%
\pgfsetroundjoin%
\definecolor{currentfill}{rgb}{0.000000,0.000000,0.000000}%
\pgfsetfillcolor{currentfill}%
\pgfsetlinewidth{0.602250pt}%
\definecolor{currentstroke}{rgb}{0.000000,0.000000,0.000000}%
\pgfsetstrokecolor{currentstroke}%
\pgfsetdash{}{0pt}%
\pgfsys@defobject{currentmarker}{\pgfqpoint{0.000000in}{0.000000in}}{\pgfqpoint{0.027778in}{0.000000in}}{%
\pgfpathmoveto{\pgfqpoint{0.000000in}{0.000000in}}%
\pgfpathlineto{\pgfqpoint{0.027778in}{0.000000in}}%
\pgfusepath{stroke,fill}%
}%
\begin{pgfscope}%
\pgfsys@transformshift{5.504000in}{3.740294in}%
\pgfsys@useobject{currentmarker}{}%
\end{pgfscope}%
\end{pgfscope}%
\begin{pgfscope}%
\pgfsetbuttcap%
\pgfsetroundjoin%
\definecolor{currentfill}{rgb}{0.000000,0.000000,0.000000}%
\pgfsetfillcolor{currentfill}%
\pgfsetlinewidth{0.602250pt}%
\definecolor{currentstroke}{rgb}{0.000000,0.000000,0.000000}%
\pgfsetstrokecolor{currentstroke}%
\pgfsetdash{}{0pt}%
\pgfsys@defobject{currentmarker}{\pgfqpoint{0.000000in}{0.000000in}}{\pgfqpoint{0.027778in}{0.000000in}}{%
\pgfpathmoveto{\pgfqpoint{0.000000in}{0.000000in}}%
\pgfpathlineto{\pgfqpoint{0.027778in}{0.000000in}}%
\pgfusepath{stroke,fill}%
}%
\begin{pgfscope}%
\pgfsys@transformshift{5.504000in}{3.766390in}%
\pgfsys@useobject{currentmarker}{}%
\end{pgfscope}%
\end{pgfscope}%
\begin{pgfscope}%
\pgfsetbuttcap%
\pgfsetroundjoin%
\definecolor{currentfill}{rgb}{0.000000,0.000000,0.000000}%
\pgfsetfillcolor{currentfill}%
\pgfsetlinewidth{0.602250pt}%
\definecolor{currentstroke}{rgb}{0.000000,0.000000,0.000000}%
\pgfsetstrokecolor{currentstroke}%
\pgfsetdash{}{0pt}%
\pgfsys@defobject{currentmarker}{\pgfqpoint{0.000000in}{0.000000in}}{\pgfqpoint{0.027778in}{0.000000in}}{%
\pgfpathmoveto{\pgfqpoint{0.000000in}{0.000000in}}%
\pgfpathlineto{\pgfqpoint{0.027778in}{0.000000in}}%
\pgfusepath{stroke,fill}%
}%
\begin{pgfscope}%
\pgfsys@transformshift{5.504000in}{3.789409in}%
\pgfsys@useobject{currentmarker}{}%
\end{pgfscope}%
\end{pgfscope}%
\begin{pgfscope}%
\pgfsetbuttcap%
\pgfsetroundjoin%
\definecolor{currentfill}{rgb}{0.000000,0.000000,0.000000}%
\pgfsetfillcolor{currentfill}%
\pgfsetlinewidth{0.602250pt}%
\definecolor{currentstroke}{rgb}{0.000000,0.000000,0.000000}%
\pgfsetstrokecolor{currentstroke}%
\pgfsetdash{}{0pt}%
\pgfsys@defobject{currentmarker}{\pgfqpoint{0.000000in}{0.000000in}}{\pgfqpoint{0.027778in}{0.000000in}}{%
\pgfpathmoveto{\pgfqpoint{0.000000in}{0.000000in}}%
\pgfpathlineto{\pgfqpoint{0.027778in}{0.000000in}}%
\pgfusepath{stroke,fill}%
}%
\begin{pgfscope}%
\pgfsys@transformshift{5.504000in}{3.945463in}%
\pgfsys@useobject{currentmarker}{}%
\end{pgfscope}%
\end{pgfscope}%
\begin{pgfscope}%
\pgfsetbuttcap%
\pgfsetroundjoin%
\definecolor{currentfill}{rgb}{0.000000,0.000000,0.000000}%
\pgfsetfillcolor{currentfill}%
\pgfsetlinewidth{0.602250pt}%
\definecolor{currentstroke}{rgb}{0.000000,0.000000,0.000000}%
\pgfsetstrokecolor{currentstroke}%
\pgfsetdash{}{0pt}%
\pgfsys@defobject{currentmarker}{\pgfqpoint{0.000000in}{0.000000in}}{\pgfqpoint{0.027778in}{0.000000in}}{%
\pgfpathmoveto{\pgfqpoint{0.000000in}{0.000000in}}%
\pgfpathlineto{\pgfqpoint{0.027778in}{0.000000in}}%
\pgfusepath{stroke,fill}%
}%
\begin{pgfscope}%
\pgfsys@transformshift{5.504000in}{4.024705in}%
\pgfsys@useobject{currentmarker}{}%
\end{pgfscope}%
\end{pgfscope}%
\begin{pgfscope}%
\pgfsetbuttcap%
\pgfsetroundjoin%
\definecolor{currentfill}{rgb}{0.000000,0.000000,0.000000}%
\pgfsetfillcolor{currentfill}%
\pgfsetlinewidth{0.602250pt}%
\definecolor{currentstroke}{rgb}{0.000000,0.000000,0.000000}%
\pgfsetstrokecolor{currentstroke}%
\pgfsetdash{}{0pt}%
\pgfsys@defobject{currentmarker}{\pgfqpoint{0.000000in}{0.000000in}}{\pgfqpoint{0.027778in}{0.000000in}}{%
\pgfpathmoveto{\pgfqpoint{0.000000in}{0.000000in}}%
\pgfpathlineto{\pgfqpoint{0.027778in}{0.000000in}}%
\pgfusepath{stroke,fill}%
}%
\begin{pgfscope}%
\pgfsys@transformshift{5.504000in}{4.080927in}%
\pgfsys@useobject{currentmarker}{}%
\end{pgfscope}%
\end{pgfscope}%
\begin{pgfscope}%
\pgfsetbuttcap%
\pgfsetroundjoin%
\definecolor{currentfill}{rgb}{0.000000,0.000000,0.000000}%
\pgfsetfillcolor{currentfill}%
\pgfsetlinewidth{0.602250pt}%
\definecolor{currentstroke}{rgb}{0.000000,0.000000,0.000000}%
\pgfsetstrokecolor{currentstroke}%
\pgfsetdash{}{0pt}%
\pgfsys@defobject{currentmarker}{\pgfqpoint{0.000000in}{0.000000in}}{\pgfqpoint{0.027778in}{0.000000in}}{%
\pgfpathmoveto{\pgfqpoint{0.000000in}{0.000000in}}%
\pgfpathlineto{\pgfqpoint{0.027778in}{0.000000in}}%
\pgfusepath{stroke,fill}%
}%
\begin{pgfscope}%
\pgfsys@transformshift{5.504000in}{4.124537in}%
\pgfsys@useobject{currentmarker}{}%
\end{pgfscope}%
\end{pgfscope}%
\begin{pgfscope}%
\pgfsetbuttcap%
\pgfsetroundjoin%
\definecolor{currentfill}{rgb}{0.000000,0.000000,0.000000}%
\pgfsetfillcolor{currentfill}%
\pgfsetlinewidth{0.602250pt}%
\definecolor{currentstroke}{rgb}{0.000000,0.000000,0.000000}%
\pgfsetstrokecolor{currentstroke}%
\pgfsetdash{}{0pt}%
\pgfsys@defobject{currentmarker}{\pgfqpoint{0.000000in}{0.000000in}}{\pgfqpoint{0.027778in}{0.000000in}}{%
\pgfpathmoveto{\pgfqpoint{0.000000in}{0.000000in}}%
\pgfpathlineto{\pgfqpoint{0.027778in}{0.000000in}}%
\pgfusepath{stroke,fill}%
}%
\begin{pgfscope}%
\pgfsys@transformshift{5.504000in}{4.160168in}%
\pgfsys@useobject{currentmarker}{}%
\end{pgfscope}%
\end{pgfscope}%
\begin{pgfscope}%
\pgfsetbuttcap%
\pgfsetroundjoin%
\definecolor{currentfill}{rgb}{0.000000,0.000000,0.000000}%
\pgfsetfillcolor{currentfill}%
\pgfsetlinewidth{0.602250pt}%
\definecolor{currentstroke}{rgb}{0.000000,0.000000,0.000000}%
\pgfsetstrokecolor{currentstroke}%
\pgfsetdash{}{0pt}%
\pgfsys@defobject{currentmarker}{\pgfqpoint{0.000000in}{0.000000in}}{\pgfqpoint{0.027778in}{0.000000in}}{%
\pgfpathmoveto{\pgfqpoint{0.000000in}{0.000000in}}%
\pgfpathlineto{\pgfqpoint{0.027778in}{0.000000in}}%
\pgfusepath{stroke,fill}%
}%
\begin{pgfscope}%
\pgfsys@transformshift{5.504000in}{4.190294in}%
\pgfsys@useobject{currentmarker}{}%
\end{pgfscope}%
\end{pgfscope}%
\begin{pgfscope}%
\pgfsetbuttcap%
\pgfsetroundjoin%
\definecolor{currentfill}{rgb}{0.000000,0.000000,0.000000}%
\pgfsetfillcolor{currentfill}%
\pgfsetlinewidth{0.602250pt}%
\definecolor{currentstroke}{rgb}{0.000000,0.000000,0.000000}%
\pgfsetstrokecolor{currentstroke}%
\pgfsetdash{}{0pt}%
\pgfsys@defobject{currentmarker}{\pgfqpoint{0.000000in}{0.000000in}}{\pgfqpoint{0.027778in}{0.000000in}}{%
\pgfpathmoveto{\pgfqpoint{0.000000in}{0.000000in}}%
\pgfpathlineto{\pgfqpoint{0.027778in}{0.000000in}}%
\pgfusepath{stroke,fill}%
}%
\begin{pgfscope}%
\pgfsys@transformshift{5.504000in}{4.216390in}%
\pgfsys@useobject{currentmarker}{}%
\end{pgfscope}%
\end{pgfscope}%
\begin{pgfscope}%
\pgfsetbuttcap%
\pgfsetroundjoin%
\definecolor{currentfill}{rgb}{0.000000,0.000000,0.000000}%
\pgfsetfillcolor{currentfill}%
\pgfsetlinewidth{0.602250pt}%
\definecolor{currentstroke}{rgb}{0.000000,0.000000,0.000000}%
\pgfsetstrokecolor{currentstroke}%
\pgfsetdash{}{0pt}%
\pgfsys@defobject{currentmarker}{\pgfqpoint{0.000000in}{0.000000in}}{\pgfqpoint{0.027778in}{0.000000in}}{%
\pgfpathmoveto{\pgfqpoint{0.000000in}{0.000000in}}%
\pgfpathlineto{\pgfqpoint{0.027778in}{0.000000in}}%
\pgfusepath{stroke,fill}%
}%
\begin{pgfscope}%
\pgfsys@transformshift{5.504000in}{4.239409in}%
\pgfsys@useobject{currentmarker}{}%
\end{pgfscope}%
\end{pgfscope}%
\begin{pgfscope}%
\pgfsetbuttcap%
\pgfsetroundjoin%
\definecolor{currentfill}{rgb}{0.000000,0.000000,0.000000}%
\pgfsetfillcolor{currentfill}%
\pgfsetlinewidth{0.602250pt}%
\definecolor{currentstroke}{rgb}{0.000000,0.000000,0.000000}%
\pgfsetstrokecolor{currentstroke}%
\pgfsetdash{}{0pt}%
\pgfsys@defobject{currentmarker}{\pgfqpoint{0.000000in}{0.000000in}}{\pgfqpoint{0.027778in}{0.000000in}}{%
\pgfpathmoveto{\pgfqpoint{0.000000in}{0.000000in}}%
\pgfpathlineto{\pgfqpoint{0.027778in}{0.000000in}}%
\pgfusepath{stroke,fill}%
}%
\begin{pgfscope}%
\pgfsys@transformshift{5.504000in}{4.395463in}%
\pgfsys@useobject{currentmarker}{}%
\end{pgfscope}%
\end{pgfscope}%
\begin{pgfscope}%
\pgfsetbuttcap%
\pgfsetroundjoin%
\definecolor{currentfill}{rgb}{0.000000,0.000000,0.000000}%
\pgfsetfillcolor{currentfill}%
\pgfsetlinewidth{0.602250pt}%
\definecolor{currentstroke}{rgb}{0.000000,0.000000,0.000000}%
\pgfsetstrokecolor{currentstroke}%
\pgfsetdash{}{0pt}%
\pgfsys@defobject{currentmarker}{\pgfqpoint{0.000000in}{0.000000in}}{\pgfqpoint{0.027778in}{0.000000in}}{%
\pgfpathmoveto{\pgfqpoint{0.000000in}{0.000000in}}%
\pgfpathlineto{\pgfqpoint{0.027778in}{0.000000in}}%
\pgfusepath{stroke,fill}%
}%
\begin{pgfscope}%
\pgfsys@transformshift{5.504000in}{4.474705in}%
\pgfsys@useobject{currentmarker}{}%
\end{pgfscope}%
\end{pgfscope}%
\begin{pgfscope}%
\pgfsetbuttcap%
\pgfsetroundjoin%
\definecolor{currentfill}{rgb}{0.000000,0.000000,0.000000}%
\pgfsetfillcolor{currentfill}%
\pgfsetlinewidth{0.602250pt}%
\definecolor{currentstroke}{rgb}{0.000000,0.000000,0.000000}%
\pgfsetstrokecolor{currentstroke}%
\pgfsetdash{}{0pt}%
\pgfsys@defobject{currentmarker}{\pgfqpoint{0.000000in}{0.000000in}}{\pgfqpoint{0.027778in}{0.000000in}}{%
\pgfpathmoveto{\pgfqpoint{0.000000in}{0.000000in}}%
\pgfpathlineto{\pgfqpoint{0.027778in}{0.000000in}}%
\pgfusepath{stroke,fill}%
}%
\begin{pgfscope}%
\pgfsys@transformshift{5.504000in}{4.530927in}%
\pgfsys@useobject{currentmarker}{}%
\end{pgfscope}%
\end{pgfscope}%
\begin{pgfscope}%
\pgfsetbuttcap%
\pgfsetroundjoin%
\definecolor{currentfill}{rgb}{0.000000,0.000000,0.000000}%
\pgfsetfillcolor{currentfill}%
\pgfsetlinewidth{0.602250pt}%
\definecolor{currentstroke}{rgb}{0.000000,0.000000,0.000000}%
\pgfsetstrokecolor{currentstroke}%
\pgfsetdash{}{0pt}%
\pgfsys@defobject{currentmarker}{\pgfqpoint{0.000000in}{0.000000in}}{\pgfqpoint{0.027778in}{0.000000in}}{%
\pgfpathmoveto{\pgfqpoint{0.000000in}{0.000000in}}%
\pgfpathlineto{\pgfqpoint{0.027778in}{0.000000in}}%
\pgfusepath{stroke,fill}%
}%
\begin{pgfscope}%
\pgfsys@transformshift{5.504000in}{4.574537in}%
\pgfsys@useobject{currentmarker}{}%
\end{pgfscope}%
\end{pgfscope}%
\begin{pgfscope}%
\pgfsetbuttcap%
\pgfsetroundjoin%
\definecolor{currentfill}{rgb}{0.000000,0.000000,0.000000}%
\pgfsetfillcolor{currentfill}%
\pgfsetlinewidth{0.602250pt}%
\definecolor{currentstroke}{rgb}{0.000000,0.000000,0.000000}%
\pgfsetstrokecolor{currentstroke}%
\pgfsetdash{}{0pt}%
\pgfsys@defobject{currentmarker}{\pgfqpoint{0.000000in}{0.000000in}}{\pgfqpoint{0.027778in}{0.000000in}}{%
\pgfpathmoveto{\pgfqpoint{0.000000in}{0.000000in}}%
\pgfpathlineto{\pgfqpoint{0.027778in}{0.000000in}}%
\pgfusepath{stroke,fill}%
}%
\begin{pgfscope}%
\pgfsys@transformshift{5.504000in}{4.610168in}%
\pgfsys@useobject{currentmarker}{}%
\end{pgfscope}%
\end{pgfscope}%
\begin{pgfscope}%
\pgfsetbuttcap%
\pgfsetroundjoin%
\definecolor{currentfill}{rgb}{0.000000,0.000000,0.000000}%
\pgfsetfillcolor{currentfill}%
\pgfsetlinewidth{0.602250pt}%
\definecolor{currentstroke}{rgb}{0.000000,0.000000,0.000000}%
\pgfsetstrokecolor{currentstroke}%
\pgfsetdash{}{0pt}%
\pgfsys@defobject{currentmarker}{\pgfqpoint{0.000000in}{0.000000in}}{\pgfqpoint{0.027778in}{0.000000in}}{%
\pgfpathmoveto{\pgfqpoint{0.000000in}{0.000000in}}%
\pgfpathlineto{\pgfqpoint{0.027778in}{0.000000in}}%
\pgfusepath{stroke,fill}%
}%
\begin{pgfscope}%
\pgfsys@transformshift{5.504000in}{4.640294in}%
\pgfsys@useobject{currentmarker}{}%
\end{pgfscope}%
\end{pgfscope}%
\begin{pgfscope}%
\pgfsetbuttcap%
\pgfsetroundjoin%
\definecolor{currentfill}{rgb}{0.000000,0.000000,0.000000}%
\pgfsetfillcolor{currentfill}%
\pgfsetlinewidth{0.602250pt}%
\definecolor{currentstroke}{rgb}{0.000000,0.000000,0.000000}%
\pgfsetstrokecolor{currentstroke}%
\pgfsetdash{}{0pt}%
\pgfsys@defobject{currentmarker}{\pgfqpoint{0.000000in}{0.000000in}}{\pgfqpoint{0.027778in}{0.000000in}}{%
\pgfpathmoveto{\pgfqpoint{0.000000in}{0.000000in}}%
\pgfpathlineto{\pgfqpoint{0.027778in}{0.000000in}}%
\pgfusepath{stroke,fill}%
}%
\begin{pgfscope}%
\pgfsys@transformshift{5.504000in}{4.666390in}%
\pgfsys@useobject{currentmarker}{}%
\end{pgfscope}%
\end{pgfscope}%
\begin{pgfscope}%
\pgfsetbuttcap%
\pgfsetroundjoin%
\definecolor{currentfill}{rgb}{0.000000,0.000000,0.000000}%
\pgfsetfillcolor{currentfill}%
\pgfsetlinewidth{0.602250pt}%
\definecolor{currentstroke}{rgb}{0.000000,0.000000,0.000000}%
\pgfsetstrokecolor{currentstroke}%
\pgfsetdash{}{0pt}%
\pgfsys@defobject{currentmarker}{\pgfqpoint{0.000000in}{0.000000in}}{\pgfqpoint{0.027778in}{0.000000in}}{%
\pgfpathmoveto{\pgfqpoint{0.000000in}{0.000000in}}%
\pgfpathlineto{\pgfqpoint{0.027778in}{0.000000in}}%
\pgfusepath{stroke,fill}%
}%
\begin{pgfscope}%
\pgfsys@transformshift{5.504000in}{4.689409in}%
\pgfsys@useobject{currentmarker}{}%
\end{pgfscope}%
\end{pgfscope}%
\begin{pgfscope}%
\pgfsetbuttcap%
\pgfsetroundjoin%
\definecolor{currentfill}{rgb}{0.000000,0.000000,0.000000}%
\pgfsetfillcolor{currentfill}%
\pgfsetlinewidth{0.602250pt}%
\definecolor{currentstroke}{rgb}{0.000000,0.000000,0.000000}%
\pgfsetstrokecolor{currentstroke}%
\pgfsetdash{}{0pt}%
\pgfsys@defobject{currentmarker}{\pgfqpoint{0.000000in}{0.000000in}}{\pgfqpoint{0.027778in}{0.000000in}}{%
\pgfpathmoveto{\pgfqpoint{0.000000in}{0.000000in}}%
\pgfpathlineto{\pgfqpoint{0.027778in}{0.000000in}}%
\pgfusepath{stroke,fill}%
}%
\begin{pgfscope}%
\pgfsys@transformshift{5.504000in}{4.845463in}%
\pgfsys@useobject{currentmarker}{}%
\end{pgfscope}%
\end{pgfscope}%
\begin{pgfscope}%
\pgfsetbuttcap%
\pgfsetroundjoin%
\definecolor{currentfill}{rgb}{0.000000,0.000000,0.000000}%
\pgfsetfillcolor{currentfill}%
\pgfsetlinewidth{0.602250pt}%
\definecolor{currentstroke}{rgb}{0.000000,0.000000,0.000000}%
\pgfsetstrokecolor{currentstroke}%
\pgfsetdash{}{0pt}%
\pgfsys@defobject{currentmarker}{\pgfqpoint{0.000000in}{0.000000in}}{\pgfqpoint{0.027778in}{0.000000in}}{%
\pgfpathmoveto{\pgfqpoint{0.000000in}{0.000000in}}%
\pgfpathlineto{\pgfqpoint{0.027778in}{0.000000in}}%
\pgfusepath{stroke,fill}%
}%
\begin{pgfscope}%
\pgfsys@transformshift{5.504000in}{4.924705in}%
\pgfsys@useobject{currentmarker}{}%
\end{pgfscope}%
\end{pgfscope}%
\begin{pgfscope}%
\pgfsetbuttcap%
\pgfsetroundjoin%
\definecolor{currentfill}{rgb}{0.000000,0.000000,0.000000}%
\pgfsetfillcolor{currentfill}%
\pgfsetlinewidth{0.602250pt}%
\definecolor{currentstroke}{rgb}{0.000000,0.000000,0.000000}%
\pgfsetstrokecolor{currentstroke}%
\pgfsetdash{}{0pt}%
\pgfsys@defobject{currentmarker}{\pgfqpoint{0.000000in}{0.000000in}}{\pgfqpoint{0.027778in}{0.000000in}}{%
\pgfpathmoveto{\pgfqpoint{0.000000in}{0.000000in}}%
\pgfpathlineto{\pgfqpoint{0.027778in}{0.000000in}}%
\pgfusepath{stroke,fill}%
}%
\begin{pgfscope}%
\pgfsys@transformshift{5.504000in}{4.980927in}%
\pgfsys@useobject{currentmarker}{}%
\end{pgfscope}%
\end{pgfscope}%
\begin{pgfscope}%
\pgfsetbuttcap%
\pgfsetroundjoin%
\definecolor{currentfill}{rgb}{0.000000,0.000000,0.000000}%
\pgfsetfillcolor{currentfill}%
\pgfsetlinewidth{0.602250pt}%
\definecolor{currentstroke}{rgb}{0.000000,0.000000,0.000000}%
\pgfsetstrokecolor{currentstroke}%
\pgfsetdash{}{0pt}%
\pgfsys@defobject{currentmarker}{\pgfqpoint{0.000000in}{0.000000in}}{\pgfqpoint{0.027778in}{0.000000in}}{%
\pgfpathmoveto{\pgfqpoint{0.000000in}{0.000000in}}%
\pgfpathlineto{\pgfqpoint{0.027778in}{0.000000in}}%
\pgfusepath{stroke,fill}%
}%
\begin{pgfscope}%
\pgfsys@transformshift{5.504000in}{5.024537in}%
\pgfsys@useobject{currentmarker}{}%
\end{pgfscope}%
\end{pgfscope}%
\begin{pgfscope}%
\pgfsetbuttcap%
\pgfsetroundjoin%
\definecolor{currentfill}{rgb}{0.000000,0.000000,0.000000}%
\pgfsetfillcolor{currentfill}%
\pgfsetlinewidth{0.602250pt}%
\definecolor{currentstroke}{rgb}{0.000000,0.000000,0.000000}%
\pgfsetstrokecolor{currentstroke}%
\pgfsetdash{}{0pt}%
\pgfsys@defobject{currentmarker}{\pgfqpoint{0.000000in}{0.000000in}}{\pgfqpoint{0.027778in}{0.000000in}}{%
\pgfpathmoveto{\pgfqpoint{0.000000in}{0.000000in}}%
\pgfpathlineto{\pgfqpoint{0.027778in}{0.000000in}}%
\pgfusepath{stroke,fill}%
}%
\begin{pgfscope}%
\pgfsys@transformshift{5.504000in}{5.060168in}%
\pgfsys@useobject{currentmarker}{}%
\end{pgfscope}%
\end{pgfscope}%
\begin{pgfscope}%
\pgfsetbuttcap%
\pgfsetroundjoin%
\definecolor{currentfill}{rgb}{0.000000,0.000000,0.000000}%
\pgfsetfillcolor{currentfill}%
\pgfsetlinewidth{0.602250pt}%
\definecolor{currentstroke}{rgb}{0.000000,0.000000,0.000000}%
\pgfsetstrokecolor{currentstroke}%
\pgfsetdash{}{0pt}%
\pgfsys@defobject{currentmarker}{\pgfqpoint{0.000000in}{0.000000in}}{\pgfqpoint{0.027778in}{0.000000in}}{%
\pgfpathmoveto{\pgfqpoint{0.000000in}{0.000000in}}%
\pgfpathlineto{\pgfqpoint{0.027778in}{0.000000in}}%
\pgfusepath{stroke,fill}%
}%
\begin{pgfscope}%
\pgfsys@transformshift{5.504000in}{5.090294in}%
\pgfsys@useobject{currentmarker}{}%
\end{pgfscope}%
\end{pgfscope}%
\begin{pgfscope}%
\pgfsetbuttcap%
\pgfsetroundjoin%
\definecolor{currentfill}{rgb}{0.000000,0.000000,0.000000}%
\pgfsetfillcolor{currentfill}%
\pgfsetlinewidth{0.602250pt}%
\definecolor{currentstroke}{rgb}{0.000000,0.000000,0.000000}%
\pgfsetstrokecolor{currentstroke}%
\pgfsetdash{}{0pt}%
\pgfsys@defobject{currentmarker}{\pgfqpoint{0.000000in}{0.000000in}}{\pgfqpoint{0.027778in}{0.000000in}}{%
\pgfpathmoveto{\pgfqpoint{0.000000in}{0.000000in}}%
\pgfpathlineto{\pgfqpoint{0.027778in}{0.000000in}}%
\pgfusepath{stroke,fill}%
}%
\begin{pgfscope}%
\pgfsys@transformshift{5.504000in}{5.116390in}%
\pgfsys@useobject{currentmarker}{}%
\end{pgfscope}%
\end{pgfscope}%
\begin{pgfscope}%
\pgfsetbuttcap%
\pgfsetroundjoin%
\definecolor{currentfill}{rgb}{0.000000,0.000000,0.000000}%
\pgfsetfillcolor{currentfill}%
\pgfsetlinewidth{0.602250pt}%
\definecolor{currentstroke}{rgb}{0.000000,0.000000,0.000000}%
\pgfsetstrokecolor{currentstroke}%
\pgfsetdash{}{0pt}%
\pgfsys@defobject{currentmarker}{\pgfqpoint{0.000000in}{0.000000in}}{\pgfqpoint{0.027778in}{0.000000in}}{%
\pgfpathmoveto{\pgfqpoint{0.000000in}{0.000000in}}%
\pgfpathlineto{\pgfqpoint{0.027778in}{0.000000in}}%
\pgfusepath{stroke,fill}%
}%
\begin{pgfscope}%
\pgfsys@transformshift{5.504000in}{5.139409in}%
\pgfsys@useobject{currentmarker}{}%
\end{pgfscope}%
\end{pgfscope}%
\begin{pgfscope}%
\definecolor{textcolor}{rgb}{0.000000,0.000000,0.000000}%
\pgfsetstrokecolor{textcolor}%
\pgfsetfillcolor{textcolor}%
\pgftext[x=5.944780in,y=4.260000in,,top,rotate=90.000000]{\color{textcolor}\sffamily\fontsize{10.000000}{12.000000}\selectfont \(\displaystyle -dQ_{driver}/dx/dy/dz \, \mathrm{[pC/\mu m^3]}\)}%
\end{pgfscope}%
\begin{pgfscope}%
\pgfsetrectcap%
\pgfsetmiterjoin%
\pgfsetlinewidth{0.803000pt}%
\definecolor{currentstroke}{rgb}{0.000000,0.000000,0.000000}%
\pgfsetstrokecolor{currentstroke}%
\pgfsetdash{}{0pt}%
\pgfpathmoveto{\pgfqpoint{5.312000in}{3.360000in}}%
\pgfpathlineto{\pgfqpoint{5.408000in}{3.360000in}}%
\pgfpathlineto{\pgfqpoint{5.504000in}{3.360000in}}%
\pgfpathlineto{\pgfqpoint{5.504000in}{5.160000in}}%
\pgfpathlineto{\pgfqpoint{5.408000in}{5.160000in}}%
\pgfpathlineto{\pgfqpoint{5.312000in}{5.160000in}}%
\pgfpathlineto{\pgfqpoint{5.312000in}{3.360000in}}%
\pgfpathclose%
\pgfusepath{stroke}%
\end{pgfscope}%
\end{pgfpicture}%
\makeatother%
\endgroup%

	\caption{Charge density of driver and plasma for a low divergent driver for different time steps.
	\textbf{a)} ($y=$ \qty{3.22}{mm}) Charge density histogram showing a small cavity forming behind the broken up part of the driver.
	\textbf{b)} Charge density slice of the centered grid points plotted over $y$. 
	}
	\label{fig:cavity_low}
\end{figure}
Even before the cavities fill and a linear regime sets in, the driver breaks up as the strong forces cause fast energy drain (see the high energy loss in the first minimum behind the driver in \autoref{fig:gain_div}.
Through beam loading, the energy of the wakefield was absorbed by the fallen back part of the driver, causing the cavities to transition into the linear regime. 
Still there remains a small blowout behind the fallen back driver, caused by its high charge density. The peak of the energy gain is positioned in this smaller wakefield.

For a driver with high divergence, no blowout is achieved so the wakefield remains in the linear regime with weak electric fields. These weak fields are not able to cause a bunch breakup, resulting in the bunch just diverging with time.
The resulting maximally gained energies can be found in \autoref{tab:gain_div}.
\begin{table}[h]
\begin{center}
\begin{tabular}{|c|c|} 
	\hline
 	$\sigma_{\theta} \, \mathrm{[mrad]}$ & $E_{gain} \, \mathrm{[MeV]}$ \\ 
 	\hline
	$1.7$ & $359.3$ \\ 
	$1.7$ (sec. peak) & $840.7$ \\ 
 	$4.2$ & $439.7$ \\
	$8.7$ & $176.9$ \\
	\hline
\end{tabular}
\caption{Maximally possible energy gain for different initial divergences.}\label{tab:gain_div}
\end{center}
\end{table}
While the normal maximum is highest for the \qty{4.2}{\mrad} driver, the second maximum of the low divergence driver achieves nearly double the energy gain, giving good reasons to strive for low divergent beams.
 
The question arises, why the change in divergence causes such an extreme difference for drivers. It is possible that, that a higher current is induced from the low divergence driver. The velocity in propagation direction is higher and the spatial distribution
is denser, as the particles do not diverge as much before entering the plasma, causing a higher current. 


\paragraph*{Distribution comparison}\hspace{0pt} \\
Even though in the experiment, the shape of a driver beam is hard to control when leaving the \gls{lwfa} towards the \gls{pwfa}, a comparison between different distributions can give new insights into the properties that are needed from a driver
to form high accelerating fields. Besides the driver with Gaussian distribution in all spatial direction, a driver with Gaussian distribution in transverse direction and uniform charge distribution in propagation direction  was simulated.

The energy gain graph \autoref{fig:gain_square} shows that only a small win of \qty{14}{\MeV} in maximal energy can be achieved compared to the Gaussian driver.
\begin{figure}
	\centering
	%% Creator: Matplotlib, PGF backend
%%
%% To include the figure in your LaTeX document, write
%%   \input{<filename>.pgf}
%%
%% Make sure the required packages are loaded in your preamble
%%   \usepackage{pgf}
%%
%% Also ensure that all the required font packages are loaded; for instance,
%% the lmodern package is sometimes necessary when using math font.
%%   \usepackage{lmodern}
%%
%% Figures using additional raster images can only be included by \input if
%% they are in the same directory as the main LaTeX file. For loading figures
%% from other directories you can use the `import` package
%%   \usepackage{import}
%%
%% and then include the figures with
%%   \import{<path to file>}{<filename>.pgf}
%%
%% Matplotlib used the following preamble
%%
\begingroup%
\makeatletter%
\begin{pgfpicture}%
\pgfpathrectangle{\pgfpointorigin}{\pgfqpoint{6.000000in}{4.000000in}}%
\pgfusepath{use as bounding box, clip}%
\begin{pgfscope}%
\pgfsetbuttcap%
\pgfsetmiterjoin%
\pgfsetlinewidth{0.000000pt}%
\definecolor{currentstroke}{rgb}{1.000000,1.000000,1.000000}%
\pgfsetstrokecolor{currentstroke}%
\pgfsetstrokeopacity{0.000000}%
\pgfsetdash{}{0pt}%
\pgfpathmoveto{\pgfqpoint{0.000000in}{0.000000in}}%
\pgfpathlineto{\pgfqpoint{6.000000in}{0.000000in}}%
\pgfpathlineto{\pgfqpoint{6.000000in}{4.000000in}}%
\pgfpathlineto{\pgfqpoint{0.000000in}{4.000000in}}%
\pgfpathlineto{\pgfqpoint{0.000000in}{0.000000in}}%
\pgfpathclose%
\pgfusepath{}%
\end{pgfscope}%
\begin{pgfscope}%
\pgfsetbuttcap%
\pgfsetmiterjoin%
\definecolor{currentfill}{rgb}{1.000000,1.000000,1.000000}%
\pgfsetfillcolor{currentfill}%
\pgfsetlinewidth{0.000000pt}%
\definecolor{currentstroke}{rgb}{0.000000,0.000000,0.000000}%
\pgfsetstrokecolor{currentstroke}%
\pgfsetstrokeopacity{0.000000}%
\pgfsetdash{}{0pt}%
\pgfpathmoveto{\pgfqpoint{0.750000in}{0.500000in}}%
\pgfpathlineto{\pgfqpoint{5.400000in}{0.500000in}}%
\pgfpathlineto{\pgfqpoint{5.400000in}{3.520000in}}%
\pgfpathlineto{\pgfqpoint{0.750000in}{3.520000in}}%
\pgfpathlineto{\pgfqpoint{0.750000in}{0.500000in}}%
\pgfpathclose%
\pgfusepath{fill}%
\end{pgfscope}%
\begin{pgfscope}%
\pgfpathrectangle{\pgfqpoint{0.750000in}{0.500000in}}{\pgfqpoint{4.650000in}{3.020000in}}%
\pgfusepath{clip}%
\pgfsetrectcap%
\pgfsetroundjoin%
\pgfsetlinewidth{0.803000pt}%
\definecolor{currentstroke}{rgb}{0.690196,0.690196,0.690196}%
\pgfsetstrokecolor{currentstroke}%
\pgfsetdash{}{0pt}%
\pgfpathmoveto{\pgfqpoint{0.812351in}{0.500000in}}%
\pgfpathlineto{\pgfqpoint{0.812351in}{3.520000in}}%
\pgfusepath{stroke}%
\end{pgfscope}%
\begin{pgfscope}%
\pgfsetbuttcap%
\pgfsetroundjoin%
\definecolor{currentfill}{rgb}{0.000000,0.000000,0.000000}%
\pgfsetfillcolor{currentfill}%
\pgfsetlinewidth{0.803000pt}%
\definecolor{currentstroke}{rgb}{0.000000,0.000000,0.000000}%
\pgfsetstrokecolor{currentstroke}%
\pgfsetdash{}{0pt}%
\pgfsys@defobject{currentmarker}{\pgfqpoint{0.000000in}{-0.048611in}}{\pgfqpoint{0.000000in}{0.000000in}}{%
\pgfpathmoveto{\pgfqpoint{0.000000in}{0.000000in}}%
\pgfpathlineto{\pgfqpoint{0.000000in}{-0.048611in}}%
\pgfusepath{stroke,fill}%
}%
\begin{pgfscope}%
\pgfsys@transformshift{0.812351in}{0.500000in}%
\pgfsys@useobject{currentmarker}{}%
\end{pgfscope}%
\end{pgfscope}%
\begin{pgfscope}%
\definecolor{textcolor}{rgb}{0.000000,0.000000,0.000000}%
\pgfsetstrokecolor{textcolor}%
\pgfsetfillcolor{textcolor}%
\pgftext[x=0.812351in,y=0.402778in,,top]{\color{textcolor}\sffamily\fontsize{10.000000}{12.000000}\selectfont \(\displaystyle {\ensuremath{-}40}\)}%
\end{pgfscope}%
\begin{pgfscope}%
\pgfpathrectangle{\pgfqpoint{0.750000in}{0.500000in}}{\pgfqpoint{4.650000in}{3.020000in}}%
\pgfusepath{clip}%
\pgfsetrectcap%
\pgfsetroundjoin%
\pgfsetlinewidth{0.803000pt}%
\definecolor{currentstroke}{rgb}{0.690196,0.690196,0.690196}%
\pgfsetstrokecolor{currentstroke}%
\pgfsetdash{}{0pt}%
\pgfpathmoveto{\pgfqpoint{1.608707in}{0.500000in}}%
\pgfpathlineto{\pgfqpoint{1.608707in}{3.520000in}}%
\pgfusepath{stroke}%
\end{pgfscope}%
\begin{pgfscope}%
\pgfsetbuttcap%
\pgfsetroundjoin%
\definecolor{currentfill}{rgb}{0.000000,0.000000,0.000000}%
\pgfsetfillcolor{currentfill}%
\pgfsetlinewidth{0.803000pt}%
\definecolor{currentstroke}{rgb}{0.000000,0.000000,0.000000}%
\pgfsetstrokecolor{currentstroke}%
\pgfsetdash{}{0pt}%
\pgfsys@defobject{currentmarker}{\pgfqpoint{0.000000in}{-0.048611in}}{\pgfqpoint{0.000000in}{0.000000in}}{%
\pgfpathmoveto{\pgfqpoint{0.000000in}{0.000000in}}%
\pgfpathlineto{\pgfqpoint{0.000000in}{-0.048611in}}%
\pgfusepath{stroke,fill}%
}%
\begin{pgfscope}%
\pgfsys@transformshift{1.608707in}{0.500000in}%
\pgfsys@useobject{currentmarker}{}%
\end{pgfscope}%
\end{pgfscope}%
\begin{pgfscope}%
\definecolor{textcolor}{rgb}{0.000000,0.000000,0.000000}%
\pgfsetstrokecolor{textcolor}%
\pgfsetfillcolor{textcolor}%
\pgftext[x=1.608707in,y=0.402778in,,top]{\color{textcolor}\sffamily\fontsize{10.000000}{12.000000}\selectfont \(\displaystyle {\ensuremath{-}30}\)}%
\end{pgfscope}%
\begin{pgfscope}%
\pgfpathrectangle{\pgfqpoint{0.750000in}{0.500000in}}{\pgfqpoint{4.650000in}{3.020000in}}%
\pgfusepath{clip}%
\pgfsetrectcap%
\pgfsetroundjoin%
\pgfsetlinewidth{0.803000pt}%
\definecolor{currentstroke}{rgb}{0.690196,0.690196,0.690196}%
\pgfsetstrokecolor{currentstroke}%
\pgfsetdash{}{0pt}%
\pgfpathmoveto{\pgfqpoint{2.405063in}{0.500000in}}%
\pgfpathlineto{\pgfqpoint{2.405063in}{3.520000in}}%
\pgfusepath{stroke}%
\end{pgfscope}%
\begin{pgfscope}%
\pgfsetbuttcap%
\pgfsetroundjoin%
\definecolor{currentfill}{rgb}{0.000000,0.000000,0.000000}%
\pgfsetfillcolor{currentfill}%
\pgfsetlinewidth{0.803000pt}%
\definecolor{currentstroke}{rgb}{0.000000,0.000000,0.000000}%
\pgfsetstrokecolor{currentstroke}%
\pgfsetdash{}{0pt}%
\pgfsys@defobject{currentmarker}{\pgfqpoint{0.000000in}{-0.048611in}}{\pgfqpoint{0.000000in}{0.000000in}}{%
\pgfpathmoveto{\pgfqpoint{0.000000in}{0.000000in}}%
\pgfpathlineto{\pgfqpoint{0.000000in}{-0.048611in}}%
\pgfusepath{stroke,fill}%
}%
\begin{pgfscope}%
\pgfsys@transformshift{2.405063in}{0.500000in}%
\pgfsys@useobject{currentmarker}{}%
\end{pgfscope}%
\end{pgfscope}%
\begin{pgfscope}%
\definecolor{textcolor}{rgb}{0.000000,0.000000,0.000000}%
\pgfsetstrokecolor{textcolor}%
\pgfsetfillcolor{textcolor}%
\pgftext[x=2.405063in,y=0.402778in,,top]{\color{textcolor}\sffamily\fontsize{10.000000}{12.000000}\selectfont \(\displaystyle {\ensuremath{-}20}\)}%
\end{pgfscope}%
\begin{pgfscope}%
\pgfpathrectangle{\pgfqpoint{0.750000in}{0.500000in}}{\pgfqpoint{4.650000in}{3.020000in}}%
\pgfusepath{clip}%
\pgfsetrectcap%
\pgfsetroundjoin%
\pgfsetlinewidth{0.803000pt}%
\definecolor{currentstroke}{rgb}{0.690196,0.690196,0.690196}%
\pgfsetstrokecolor{currentstroke}%
\pgfsetdash{}{0pt}%
\pgfpathmoveto{\pgfqpoint{3.201418in}{0.500000in}}%
\pgfpathlineto{\pgfqpoint{3.201418in}{3.520000in}}%
\pgfusepath{stroke}%
\end{pgfscope}%
\begin{pgfscope}%
\pgfsetbuttcap%
\pgfsetroundjoin%
\definecolor{currentfill}{rgb}{0.000000,0.000000,0.000000}%
\pgfsetfillcolor{currentfill}%
\pgfsetlinewidth{0.803000pt}%
\definecolor{currentstroke}{rgb}{0.000000,0.000000,0.000000}%
\pgfsetstrokecolor{currentstroke}%
\pgfsetdash{}{0pt}%
\pgfsys@defobject{currentmarker}{\pgfqpoint{0.000000in}{-0.048611in}}{\pgfqpoint{0.000000in}{0.000000in}}{%
\pgfpathmoveto{\pgfqpoint{0.000000in}{0.000000in}}%
\pgfpathlineto{\pgfqpoint{0.000000in}{-0.048611in}}%
\pgfusepath{stroke,fill}%
}%
\begin{pgfscope}%
\pgfsys@transformshift{3.201418in}{0.500000in}%
\pgfsys@useobject{currentmarker}{}%
\end{pgfscope}%
\end{pgfscope}%
\begin{pgfscope}%
\definecolor{textcolor}{rgb}{0.000000,0.000000,0.000000}%
\pgfsetstrokecolor{textcolor}%
\pgfsetfillcolor{textcolor}%
\pgftext[x=3.201418in,y=0.402778in,,top]{\color{textcolor}\sffamily\fontsize{10.000000}{12.000000}\selectfont \(\displaystyle {\ensuremath{-}10}\)}%
\end{pgfscope}%
\begin{pgfscope}%
\pgfpathrectangle{\pgfqpoint{0.750000in}{0.500000in}}{\pgfqpoint{4.650000in}{3.020000in}}%
\pgfusepath{clip}%
\pgfsetrectcap%
\pgfsetroundjoin%
\pgfsetlinewidth{0.803000pt}%
\definecolor{currentstroke}{rgb}{0.690196,0.690196,0.690196}%
\pgfsetstrokecolor{currentstroke}%
\pgfsetdash{}{0pt}%
\pgfpathmoveto{\pgfqpoint{3.997774in}{0.500000in}}%
\pgfpathlineto{\pgfqpoint{3.997774in}{3.520000in}}%
\pgfusepath{stroke}%
\end{pgfscope}%
\begin{pgfscope}%
\pgfsetbuttcap%
\pgfsetroundjoin%
\definecolor{currentfill}{rgb}{0.000000,0.000000,0.000000}%
\pgfsetfillcolor{currentfill}%
\pgfsetlinewidth{0.803000pt}%
\definecolor{currentstroke}{rgb}{0.000000,0.000000,0.000000}%
\pgfsetstrokecolor{currentstroke}%
\pgfsetdash{}{0pt}%
\pgfsys@defobject{currentmarker}{\pgfqpoint{0.000000in}{-0.048611in}}{\pgfqpoint{0.000000in}{0.000000in}}{%
\pgfpathmoveto{\pgfqpoint{0.000000in}{0.000000in}}%
\pgfpathlineto{\pgfqpoint{0.000000in}{-0.048611in}}%
\pgfusepath{stroke,fill}%
}%
\begin{pgfscope}%
\pgfsys@transformshift{3.997774in}{0.500000in}%
\pgfsys@useobject{currentmarker}{}%
\end{pgfscope}%
\end{pgfscope}%
\begin{pgfscope}%
\definecolor{textcolor}{rgb}{0.000000,0.000000,0.000000}%
\pgfsetstrokecolor{textcolor}%
\pgfsetfillcolor{textcolor}%
\pgftext[x=3.997774in,y=0.402778in,,top]{\color{textcolor}\sffamily\fontsize{10.000000}{12.000000}\selectfont \(\displaystyle {0}\)}%
\end{pgfscope}%
\begin{pgfscope}%
\pgfpathrectangle{\pgfqpoint{0.750000in}{0.500000in}}{\pgfqpoint{4.650000in}{3.020000in}}%
\pgfusepath{clip}%
\pgfsetrectcap%
\pgfsetroundjoin%
\pgfsetlinewidth{0.803000pt}%
\definecolor{currentstroke}{rgb}{0.690196,0.690196,0.690196}%
\pgfsetstrokecolor{currentstroke}%
\pgfsetdash{}{0pt}%
\pgfpathmoveto{\pgfqpoint{4.794129in}{0.500000in}}%
\pgfpathlineto{\pgfqpoint{4.794129in}{3.520000in}}%
\pgfusepath{stroke}%
\end{pgfscope}%
\begin{pgfscope}%
\pgfsetbuttcap%
\pgfsetroundjoin%
\definecolor{currentfill}{rgb}{0.000000,0.000000,0.000000}%
\pgfsetfillcolor{currentfill}%
\pgfsetlinewidth{0.803000pt}%
\definecolor{currentstroke}{rgb}{0.000000,0.000000,0.000000}%
\pgfsetstrokecolor{currentstroke}%
\pgfsetdash{}{0pt}%
\pgfsys@defobject{currentmarker}{\pgfqpoint{0.000000in}{-0.048611in}}{\pgfqpoint{0.000000in}{0.000000in}}{%
\pgfpathmoveto{\pgfqpoint{0.000000in}{0.000000in}}%
\pgfpathlineto{\pgfqpoint{0.000000in}{-0.048611in}}%
\pgfusepath{stroke,fill}%
}%
\begin{pgfscope}%
\pgfsys@transformshift{4.794129in}{0.500000in}%
\pgfsys@useobject{currentmarker}{}%
\end{pgfscope}%
\end{pgfscope}%
\begin{pgfscope}%
\definecolor{textcolor}{rgb}{0.000000,0.000000,0.000000}%
\pgfsetstrokecolor{textcolor}%
\pgfsetfillcolor{textcolor}%
\pgftext[x=4.794129in,y=0.402778in,,top]{\color{textcolor}\sffamily\fontsize{10.000000}{12.000000}\selectfont \(\displaystyle {10}\)}%
\end{pgfscope}%
\begin{pgfscope}%
\definecolor{textcolor}{rgb}{0.000000,0.000000,0.000000}%
\pgfsetstrokecolor{textcolor}%
\pgfsetfillcolor{textcolor}%
\pgftext[x=3.075000in,y=0.223766in,,top]{\color{textcolor}\sffamily\fontsize{10.000000}{12.000000}\selectfont \(\displaystyle \zeta \, \mathrm{[\mu m]}\)}%
\end{pgfscope}%
\begin{pgfscope}%
\pgfpathrectangle{\pgfqpoint{0.750000in}{0.500000in}}{\pgfqpoint{4.650000in}{3.020000in}}%
\pgfusepath{clip}%
\pgfsetrectcap%
\pgfsetroundjoin%
\pgfsetlinewidth{0.803000pt}%
\definecolor{currentstroke}{rgb}{0.690196,0.690196,0.690196}%
\pgfsetstrokecolor{currentstroke}%
\pgfsetdash{}{0pt}%
\pgfpathmoveto{\pgfqpoint{0.750000in}{0.950666in}}%
\pgfpathlineto{\pgfqpoint{5.400000in}{0.950666in}}%
\pgfusepath{stroke}%
\end{pgfscope}%
\begin{pgfscope}%
\pgfsetbuttcap%
\pgfsetroundjoin%
\definecolor{currentfill}{rgb}{0.000000,0.000000,0.000000}%
\pgfsetfillcolor{currentfill}%
\pgfsetlinewidth{0.803000pt}%
\definecolor{currentstroke}{rgb}{0.000000,0.000000,0.000000}%
\pgfsetstrokecolor{currentstroke}%
\pgfsetdash{}{0pt}%
\pgfsys@defobject{currentmarker}{\pgfqpoint{-0.048611in}{0.000000in}}{\pgfqpoint{-0.000000in}{0.000000in}}{%
\pgfpathmoveto{\pgfqpoint{-0.000000in}{0.000000in}}%
\pgfpathlineto{\pgfqpoint{-0.048611in}{0.000000in}}%
\pgfusepath{stroke,fill}%
}%
\begin{pgfscope}%
\pgfsys@transformshift{0.750000in}{0.950666in}%
\pgfsys@useobject{currentmarker}{}%
\end{pgfscope}%
\end{pgfscope}%
\begin{pgfscope}%
\definecolor{textcolor}{rgb}{0.000000,0.000000,0.000000}%
\pgfsetstrokecolor{textcolor}%
\pgfsetfillcolor{textcolor}%
\pgftext[x=0.336419in, y=0.902441in, left, base]{\color{textcolor}\sffamily\fontsize{10.000000}{12.000000}\selectfont \(\displaystyle {\ensuremath{-}600}\)}%
\end{pgfscope}%
\begin{pgfscope}%
\pgfpathrectangle{\pgfqpoint{0.750000in}{0.500000in}}{\pgfqpoint{4.650000in}{3.020000in}}%
\pgfusepath{clip}%
\pgfsetrectcap%
\pgfsetroundjoin%
\pgfsetlinewidth{0.803000pt}%
\definecolor{currentstroke}{rgb}{0.690196,0.690196,0.690196}%
\pgfsetstrokecolor{currentstroke}%
\pgfsetdash{}{0pt}%
\pgfpathmoveto{\pgfqpoint{0.750000in}{1.412273in}}%
\pgfpathlineto{\pgfqpoint{5.400000in}{1.412273in}}%
\pgfusepath{stroke}%
\end{pgfscope}%
\begin{pgfscope}%
\pgfsetbuttcap%
\pgfsetroundjoin%
\definecolor{currentfill}{rgb}{0.000000,0.000000,0.000000}%
\pgfsetfillcolor{currentfill}%
\pgfsetlinewidth{0.803000pt}%
\definecolor{currentstroke}{rgb}{0.000000,0.000000,0.000000}%
\pgfsetstrokecolor{currentstroke}%
\pgfsetdash{}{0pt}%
\pgfsys@defobject{currentmarker}{\pgfqpoint{-0.048611in}{0.000000in}}{\pgfqpoint{-0.000000in}{0.000000in}}{%
\pgfpathmoveto{\pgfqpoint{-0.000000in}{0.000000in}}%
\pgfpathlineto{\pgfqpoint{-0.048611in}{0.000000in}}%
\pgfusepath{stroke,fill}%
}%
\begin{pgfscope}%
\pgfsys@transformshift{0.750000in}{1.412273in}%
\pgfsys@useobject{currentmarker}{}%
\end{pgfscope}%
\end{pgfscope}%
\begin{pgfscope}%
\definecolor{textcolor}{rgb}{0.000000,0.000000,0.000000}%
\pgfsetstrokecolor{textcolor}%
\pgfsetfillcolor{textcolor}%
\pgftext[x=0.336419in, y=1.364047in, left, base]{\color{textcolor}\sffamily\fontsize{10.000000}{12.000000}\selectfont \(\displaystyle {\ensuremath{-}400}\)}%
\end{pgfscope}%
\begin{pgfscope}%
\pgfpathrectangle{\pgfqpoint{0.750000in}{0.500000in}}{\pgfqpoint{4.650000in}{3.020000in}}%
\pgfusepath{clip}%
\pgfsetrectcap%
\pgfsetroundjoin%
\pgfsetlinewidth{0.803000pt}%
\definecolor{currentstroke}{rgb}{0.690196,0.690196,0.690196}%
\pgfsetstrokecolor{currentstroke}%
\pgfsetdash{}{0pt}%
\pgfpathmoveto{\pgfqpoint{0.750000in}{1.873879in}}%
\pgfpathlineto{\pgfqpoint{5.400000in}{1.873879in}}%
\pgfusepath{stroke}%
\end{pgfscope}%
\begin{pgfscope}%
\pgfsetbuttcap%
\pgfsetroundjoin%
\definecolor{currentfill}{rgb}{0.000000,0.000000,0.000000}%
\pgfsetfillcolor{currentfill}%
\pgfsetlinewidth{0.803000pt}%
\definecolor{currentstroke}{rgb}{0.000000,0.000000,0.000000}%
\pgfsetstrokecolor{currentstroke}%
\pgfsetdash{}{0pt}%
\pgfsys@defobject{currentmarker}{\pgfqpoint{-0.048611in}{0.000000in}}{\pgfqpoint{-0.000000in}{0.000000in}}{%
\pgfpathmoveto{\pgfqpoint{-0.000000in}{0.000000in}}%
\pgfpathlineto{\pgfqpoint{-0.048611in}{0.000000in}}%
\pgfusepath{stroke,fill}%
}%
\begin{pgfscope}%
\pgfsys@transformshift{0.750000in}{1.873879in}%
\pgfsys@useobject{currentmarker}{}%
\end{pgfscope}%
\end{pgfscope}%
\begin{pgfscope}%
\definecolor{textcolor}{rgb}{0.000000,0.000000,0.000000}%
\pgfsetstrokecolor{textcolor}%
\pgfsetfillcolor{textcolor}%
\pgftext[x=0.336419in, y=1.825654in, left, base]{\color{textcolor}\sffamily\fontsize{10.000000}{12.000000}\selectfont \(\displaystyle {\ensuremath{-}200}\)}%
\end{pgfscope}%
\begin{pgfscope}%
\pgfpathrectangle{\pgfqpoint{0.750000in}{0.500000in}}{\pgfqpoint{4.650000in}{3.020000in}}%
\pgfusepath{clip}%
\pgfsetrectcap%
\pgfsetroundjoin%
\pgfsetlinewidth{0.803000pt}%
\definecolor{currentstroke}{rgb}{0.690196,0.690196,0.690196}%
\pgfsetstrokecolor{currentstroke}%
\pgfsetdash{}{0pt}%
\pgfpathmoveto{\pgfqpoint{0.750000in}{2.335486in}}%
\pgfpathlineto{\pgfqpoint{5.400000in}{2.335486in}}%
\pgfusepath{stroke}%
\end{pgfscope}%
\begin{pgfscope}%
\pgfsetbuttcap%
\pgfsetroundjoin%
\definecolor{currentfill}{rgb}{0.000000,0.000000,0.000000}%
\pgfsetfillcolor{currentfill}%
\pgfsetlinewidth{0.803000pt}%
\definecolor{currentstroke}{rgb}{0.000000,0.000000,0.000000}%
\pgfsetstrokecolor{currentstroke}%
\pgfsetdash{}{0pt}%
\pgfsys@defobject{currentmarker}{\pgfqpoint{-0.048611in}{0.000000in}}{\pgfqpoint{-0.000000in}{0.000000in}}{%
\pgfpathmoveto{\pgfqpoint{-0.000000in}{0.000000in}}%
\pgfpathlineto{\pgfqpoint{-0.048611in}{0.000000in}}%
\pgfusepath{stroke,fill}%
}%
\begin{pgfscope}%
\pgfsys@transformshift{0.750000in}{2.335486in}%
\pgfsys@useobject{currentmarker}{}%
\end{pgfscope}%
\end{pgfscope}%
\begin{pgfscope}%
\definecolor{textcolor}{rgb}{0.000000,0.000000,0.000000}%
\pgfsetstrokecolor{textcolor}%
\pgfsetfillcolor{textcolor}%
\pgftext[x=0.583333in, y=2.287261in, left, base]{\color{textcolor}\sffamily\fontsize{10.000000}{12.000000}\selectfont \(\displaystyle {0}\)}%
\end{pgfscope}%
\begin{pgfscope}%
\pgfpathrectangle{\pgfqpoint{0.750000in}{0.500000in}}{\pgfqpoint{4.650000in}{3.020000in}}%
\pgfusepath{clip}%
\pgfsetrectcap%
\pgfsetroundjoin%
\pgfsetlinewidth{0.803000pt}%
\definecolor{currentstroke}{rgb}{0.690196,0.690196,0.690196}%
\pgfsetstrokecolor{currentstroke}%
\pgfsetdash{}{0pt}%
\pgfpathmoveto{\pgfqpoint{0.750000in}{2.797093in}}%
\pgfpathlineto{\pgfqpoint{5.400000in}{2.797093in}}%
\pgfusepath{stroke}%
\end{pgfscope}%
\begin{pgfscope}%
\pgfsetbuttcap%
\pgfsetroundjoin%
\definecolor{currentfill}{rgb}{0.000000,0.000000,0.000000}%
\pgfsetfillcolor{currentfill}%
\pgfsetlinewidth{0.803000pt}%
\definecolor{currentstroke}{rgb}{0.000000,0.000000,0.000000}%
\pgfsetstrokecolor{currentstroke}%
\pgfsetdash{}{0pt}%
\pgfsys@defobject{currentmarker}{\pgfqpoint{-0.048611in}{0.000000in}}{\pgfqpoint{-0.000000in}{0.000000in}}{%
\pgfpathmoveto{\pgfqpoint{-0.000000in}{0.000000in}}%
\pgfpathlineto{\pgfqpoint{-0.048611in}{0.000000in}}%
\pgfusepath{stroke,fill}%
}%
\begin{pgfscope}%
\pgfsys@transformshift{0.750000in}{2.797093in}%
\pgfsys@useobject{currentmarker}{}%
\end{pgfscope}%
\end{pgfscope}%
\begin{pgfscope}%
\definecolor{textcolor}{rgb}{0.000000,0.000000,0.000000}%
\pgfsetstrokecolor{textcolor}%
\pgfsetfillcolor{textcolor}%
\pgftext[x=0.444444in, y=2.748868in, left, base]{\color{textcolor}\sffamily\fontsize{10.000000}{12.000000}\selectfont \(\displaystyle {200}\)}%
\end{pgfscope}%
\begin{pgfscope}%
\pgfpathrectangle{\pgfqpoint{0.750000in}{0.500000in}}{\pgfqpoint{4.650000in}{3.020000in}}%
\pgfusepath{clip}%
\pgfsetrectcap%
\pgfsetroundjoin%
\pgfsetlinewidth{0.803000pt}%
\definecolor{currentstroke}{rgb}{0.690196,0.690196,0.690196}%
\pgfsetstrokecolor{currentstroke}%
\pgfsetdash{}{0pt}%
\pgfpathmoveto{\pgfqpoint{0.750000in}{3.258700in}}%
\pgfpathlineto{\pgfqpoint{5.400000in}{3.258700in}}%
\pgfusepath{stroke}%
\end{pgfscope}%
\begin{pgfscope}%
\pgfsetbuttcap%
\pgfsetroundjoin%
\definecolor{currentfill}{rgb}{0.000000,0.000000,0.000000}%
\pgfsetfillcolor{currentfill}%
\pgfsetlinewidth{0.803000pt}%
\definecolor{currentstroke}{rgb}{0.000000,0.000000,0.000000}%
\pgfsetstrokecolor{currentstroke}%
\pgfsetdash{}{0pt}%
\pgfsys@defobject{currentmarker}{\pgfqpoint{-0.048611in}{0.000000in}}{\pgfqpoint{-0.000000in}{0.000000in}}{%
\pgfpathmoveto{\pgfqpoint{-0.000000in}{0.000000in}}%
\pgfpathlineto{\pgfqpoint{-0.048611in}{0.000000in}}%
\pgfusepath{stroke,fill}%
}%
\begin{pgfscope}%
\pgfsys@transformshift{0.750000in}{3.258700in}%
\pgfsys@useobject{currentmarker}{}%
\end{pgfscope}%
\end{pgfscope}%
\begin{pgfscope}%
\definecolor{textcolor}{rgb}{0.000000,0.000000,0.000000}%
\pgfsetstrokecolor{textcolor}%
\pgfsetfillcolor{textcolor}%
\pgftext[x=0.444444in, y=3.210474in, left, base]{\color{textcolor}\sffamily\fontsize{10.000000}{12.000000}\selectfont \(\displaystyle {400}\)}%
\end{pgfscope}%
\begin{pgfscope}%
\definecolor{textcolor}{rgb}{0.000000,0.000000,0.000000}%
\pgfsetstrokecolor{textcolor}%
\pgfsetfillcolor{textcolor}%
\pgftext[x=0.280863in,y=2.010000in,,bottom,rotate=90.000000]{\color{textcolor}\sffamily\fontsize{10.000000}{12.000000}\selectfont \(\displaystyle \mathrm{Energy \, gain \, [MeV]}\)}%
\end{pgfscope}%
\begin{pgfscope}%
\pgfpathrectangle{\pgfqpoint{0.750000in}{0.500000in}}{\pgfqpoint{4.650000in}{3.020000in}}%
\pgfusepath{clip}%
\pgfsetrectcap%
\pgfsetroundjoin%
\pgfsetlinewidth{1.505625pt}%
\definecolor{currentstroke}{rgb}{0.121569,0.466667,0.705882}%
\pgfsetstrokecolor{currentstroke}%
\pgfsetdash{}{0pt}%
\pgfpathmoveto{\pgfqpoint{0.961364in}{1.212153in}}%
\pgfpathlineto{\pgfqpoint{0.975475in}{1.160138in}}%
\pgfpathlineto{\pgfqpoint{0.982531in}{1.142126in}}%
\pgfpathlineto{\pgfqpoint{0.989586in}{1.109090in}}%
\pgfpathlineto{\pgfqpoint{1.010754in}{1.035460in}}%
\pgfpathlineto{\pgfqpoint{1.031921in}{0.958522in}}%
\pgfpathlineto{\pgfqpoint{1.053088in}{0.888055in}}%
\pgfpathlineto{\pgfqpoint{1.060144in}{0.856712in}}%
\pgfpathlineto{\pgfqpoint{1.074255in}{0.812710in}}%
\pgfpathlineto{\pgfqpoint{1.081311in}{0.783953in}}%
\pgfpathlineto{\pgfqpoint{1.102478in}{0.725900in}}%
\pgfpathlineto{\pgfqpoint{1.109534in}{0.708626in}}%
\pgfpathlineto{\pgfqpoint{1.116589in}{0.718202in}}%
\pgfpathlineto{\pgfqpoint{1.123645in}{0.701521in}}%
\pgfpathlineto{\pgfqpoint{1.130701in}{0.723623in}}%
\pgfpathlineto{\pgfqpoint{1.144812in}{0.878052in}}%
\pgfpathlineto{\pgfqpoint{1.158924in}{1.243908in}}%
\pgfpathlineto{\pgfqpoint{1.173035in}{1.695773in}}%
\pgfpathlineto{\pgfqpoint{1.187146in}{2.180895in}}%
\pgfpathlineto{\pgfqpoint{1.194202in}{2.326806in}}%
\pgfpathlineto{\pgfqpoint{1.208314in}{2.439826in}}%
\pgfpathlineto{\pgfqpoint{1.215369in}{2.474176in}}%
\pgfpathlineto{\pgfqpoint{1.229481in}{2.578951in}}%
\pgfpathlineto{\pgfqpoint{1.236536in}{2.631015in}}%
\pgfpathlineto{\pgfqpoint{1.250648in}{2.713079in}}%
\pgfpathlineto{\pgfqpoint{1.257703in}{2.739392in}}%
\pgfpathlineto{\pgfqpoint{1.285926in}{2.881859in}}%
\pgfpathlineto{\pgfqpoint{1.314149in}{2.983753in}}%
\pgfpathlineto{\pgfqpoint{1.321205in}{3.000778in}}%
\pgfpathlineto{\pgfqpoint{1.328261in}{3.025401in}}%
\pgfpathlineto{\pgfqpoint{1.349428in}{3.073447in}}%
\pgfpathlineto{\pgfqpoint{1.363539in}{3.105684in}}%
\pgfpathlineto{\pgfqpoint{1.370595in}{3.114292in}}%
\pgfpathlineto{\pgfqpoint{1.377651in}{3.135184in}}%
\pgfpathlineto{\pgfqpoint{1.384706in}{3.138851in}}%
\pgfpathlineto{\pgfqpoint{1.391762in}{3.159345in}}%
\pgfpathlineto{\pgfqpoint{1.398818in}{3.161056in}}%
\pgfpathlineto{\pgfqpoint{1.405873in}{3.182988in}}%
\pgfpathlineto{\pgfqpoint{1.412929in}{3.184653in}}%
\pgfpathlineto{\pgfqpoint{1.419985in}{3.188179in}}%
\pgfpathlineto{\pgfqpoint{1.427041in}{3.205971in}}%
\pgfpathlineto{\pgfqpoint{1.434096in}{3.201699in}}%
\pgfpathlineto{\pgfqpoint{1.441152in}{3.226805in}}%
\pgfpathlineto{\pgfqpoint{1.448208in}{3.221379in}}%
\pgfpathlineto{\pgfqpoint{1.455263in}{3.229759in}}%
\pgfpathlineto{\pgfqpoint{1.462319in}{3.231606in}}%
\pgfpathlineto{\pgfqpoint{1.476431in}{3.242567in}}%
\pgfpathlineto{\pgfqpoint{1.483486in}{3.247244in}}%
\pgfpathlineto{\pgfqpoint{1.490542in}{3.242496in}}%
\pgfpathlineto{\pgfqpoint{1.497598in}{3.253460in}}%
\pgfpathlineto{\pgfqpoint{1.504653in}{3.251847in}}%
\pgfpathlineto{\pgfqpoint{1.511709in}{3.255758in}}%
\pgfpathlineto{\pgfqpoint{1.518765in}{3.248279in}}%
\pgfpathlineto{\pgfqpoint{1.525820in}{3.255247in}}%
\pgfpathlineto{\pgfqpoint{1.532876in}{3.251854in}}%
\pgfpathlineto{\pgfqpoint{1.539932in}{3.257471in}}%
\pgfpathlineto{\pgfqpoint{1.546988in}{3.248242in}}%
\pgfpathlineto{\pgfqpoint{1.554043in}{3.248289in}}%
\pgfpathlineto{\pgfqpoint{1.561099in}{3.246376in}}%
\pgfpathlineto{\pgfqpoint{1.568155in}{3.237024in}}%
\pgfpathlineto{\pgfqpoint{1.575210in}{3.238176in}}%
\pgfpathlineto{\pgfqpoint{1.582266in}{3.231055in}}%
\pgfpathlineto{\pgfqpoint{1.596378in}{3.222451in}}%
\pgfpathlineto{\pgfqpoint{1.603433in}{3.212784in}}%
\pgfpathlineto{\pgfqpoint{1.610489in}{3.213796in}}%
\pgfpathlineto{\pgfqpoint{1.617545in}{3.194388in}}%
\pgfpathlineto{\pgfqpoint{1.624600in}{3.194303in}}%
\pgfpathlineto{\pgfqpoint{1.631656in}{3.188021in}}%
\pgfpathlineto{\pgfqpoint{1.638712in}{3.175513in}}%
\pgfpathlineto{\pgfqpoint{1.645768in}{3.169874in}}%
\pgfpathlineto{\pgfqpoint{1.659879in}{3.145761in}}%
\pgfpathlineto{\pgfqpoint{1.666935in}{3.147761in}}%
\pgfpathlineto{\pgfqpoint{1.673990in}{3.127401in}}%
\pgfpathlineto{\pgfqpoint{1.681046in}{3.115650in}}%
\pgfpathlineto{\pgfqpoint{1.688102in}{3.107375in}}%
\pgfpathlineto{\pgfqpoint{1.695158in}{3.089886in}}%
\pgfpathlineto{\pgfqpoint{1.702213in}{3.084119in}}%
\pgfpathlineto{\pgfqpoint{1.709269in}{3.074203in}}%
\pgfpathlineto{\pgfqpoint{1.716325in}{3.051132in}}%
\pgfpathlineto{\pgfqpoint{1.723380in}{3.048968in}}%
\pgfpathlineto{\pgfqpoint{1.730436in}{3.028147in}}%
\pgfpathlineto{\pgfqpoint{1.737492in}{3.015071in}}%
\pgfpathlineto{\pgfqpoint{1.744547in}{3.006618in}}%
\pgfpathlineto{\pgfqpoint{1.751603in}{2.983260in}}%
\pgfpathlineto{\pgfqpoint{1.758659in}{2.975474in}}%
\pgfpathlineto{\pgfqpoint{1.765715in}{2.962025in}}%
\pgfpathlineto{\pgfqpoint{1.772770in}{2.943134in}}%
\pgfpathlineto{\pgfqpoint{1.779826in}{2.931464in}}%
\pgfpathlineto{\pgfqpoint{1.786882in}{2.915657in}}%
\pgfpathlineto{\pgfqpoint{1.793937in}{2.895773in}}%
\pgfpathlineto{\pgfqpoint{1.800993in}{2.882933in}}%
\pgfpathlineto{\pgfqpoint{1.815105in}{2.844959in}}%
\pgfpathlineto{\pgfqpoint{1.822160in}{2.842195in}}%
\pgfpathlineto{\pgfqpoint{1.829216in}{2.805475in}}%
\pgfpathlineto{\pgfqpoint{1.836272in}{2.801645in}}%
\pgfpathlineto{\pgfqpoint{1.843327in}{2.775821in}}%
\pgfpathlineto{\pgfqpoint{1.850383in}{2.765834in}}%
\pgfpathlineto{\pgfqpoint{1.878606in}{2.689691in}}%
\pgfpathlineto{\pgfqpoint{1.885662in}{2.662271in}}%
\pgfpathlineto{\pgfqpoint{1.892717in}{2.660765in}}%
\pgfpathlineto{\pgfqpoint{1.899773in}{2.627361in}}%
\pgfpathlineto{\pgfqpoint{1.906829in}{2.617664in}}%
\pgfpathlineto{\pgfqpoint{1.913885in}{2.593102in}}%
\pgfpathlineto{\pgfqpoint{1.927996in}{2.559480in}}%
\pgfpathlineto{\pgfqpoint{1.935052in}{2.532244in}}%
\pgfpathlineto{\pgfqpoint{1.942107in}{2.517259in}}%
\pgfpathlineto{\pgfqpoint{1.949163in}{2.495704in}}%
\pgfpathlineto{\pgfqpoint{1.956219in}{2.480922in}}%
\pgfpathlineto{\pgfqpoint{1.970330in}{2.435202in}}%
\pgfpathlineto{\pgfqpoint{1.977386in}{2.417684in}}%
\pgfpathlineto{\pgfqpoint{1.984442in}{2.389331in}}%
\pgfpathlineto{\pgfqpoint{1.991497in}{2.381621in}}%
\pgfpathlineto{\pgfqpoint{1.998553in}{2.348040in}}%
\pgfpathlineto{\pgfqpoint{2.012664in}{2.316758in}}%
\pgfpathlineto{\pgfqpoint{2.019720in}{2.286955in}}%
\pgfpathlineto{\pgfqpoint{2.026776in}{2.278533in}}%
\pgfpathlineto{\pgfqpoint{2.033832in}{2.245413in}}%
\pgfpathlineto{\pgfqpoint{2.040887in}{2.229319in}}%
\pgfpathlineto{\pgfqpoint{2.047943in}{2.203850in}}%
\pgfpathlineto{\pgfqpoint{2.054999in}{2.189925in}}%
\pgfpathlineto{\pgfqpoint{2.062054in}{2.155156in}}%
\pgfpathlineto{\pgfqpoint{2.076166in}{2.122949in}}%
\pgfpathlineto{\pgfqpoint{2.083222in}{2.087941in}}%
\pgfpathlineto{\pgfqpoint{2.090277in}{2.088200in}}%
\pgfpathlineto{\pgfqpoint{2.097333in}{2.047735in}}%
\pgfpathlineto{\pgfqpoint{2.104389in}{2.034391in}}%
\pgfpathlineto{\pgfqpoint{2.111444in}{2.013932in}}%
\pgfpathlineto{\pgfqpoint{2.118500in}{1.981422in}}%
\pgfpathlineto{\pgfqpoint{2.125556in}{1.971592in}}%
\pgfpathlineto{\pgfqpoint{2.132612in}{1.947650in}}%
\pgfpathlineto{\pgfqpoint{2.139667in}{1.916112in}}%
\pgfpathlineto{\pgfqpoint{2.146723in}{1.903471in}}%
\pgfpathlineto{\pgfqpoint{2.153779in}{1.874015in}}%
\pgfpathlineto{\pgfqpoint{2.174946in}{1.814231in}}%
\pgfpathlineto{\pgfqpoint{2.182002in}{1.782864in}}%
\pgfpathlineto{\pgfqpoint{2.189057in}{1.764843in}}%
\pgfpathlineto{\pgfqpoint{2.196113in}{1.737451in}}%
\pgfpathlineto{\pgfqpoint{2.203169in}{1.723735in}}%
\pgfpathlineto{\pgfqpoint{2.210224in}{1.688218in}}%
\pgfpathlineto{\pgfqpoint{2.217280in}{1.677438in}}%
\pgfpathlineto{\pgfqpoint{2.224336in}{1.652300in}}%
\pgfpathlineto{\pgfqpoint{2.231392in}{1.621299in}}%
\pgfpathlineto{\pgfqpoint{2.238447in}{1.611696in}}%
\pgfpathlineto{\pgfqpoint{2.245503in}{1.579186in}}%
\pgfpathlineto{\pgfqpoint{2.252559in}{1.566714in}}%
\pgfpathlineto{\pgfqpoint{2.259614in}{1.526346in}}%
\pgfpathlineto{\pgfqpoint{2.266670in}{1.516197in}}%
\pgfpathlineto{\pgfqpoint{2.273726in}{1.490165in}}%
\pgfpathlineto{\pgfqpoint{2.280781in}{1.470126in}}%
\pgfpathlineto{\pgfqpoint{2.287837in}{1.439968in}}%
\pgfpathlineto{\pgfqpoint{2.294893in}{1.428161in}}%
\pgfpathlineto{\pgfqpoint{2.301949in}{1.394173in}}%
\pgfpathlineto{\pgfqpoint{2.309004in}{1.376786in}}%
\pgfpathlineto{\pgfqpoint{2.316060in}{1.354009in}}%
\pgfpathlineto{\pgfqpoint{2.323116in}{1.325632in}}%
\pgfpathlineto{\pgfqpoint{2.330171in}{1.315437in}}%
\pgfpathlineto{\pgfqpoint{2.337227in}{1.278776in}}%
\pgfpathlineto{\pgfqpoint{2.344283in}{1.263506in}}%
\pgfpathlineto{\pgfqpoint{2.351339in}{1.233312in}}%
\pgfpathlineto{\pgfqpoint{2.358394in}{1.214666in}}%
\pgfpathlineto{\pgfqpoint{2.372506in}{1.168268in}}%
\pgfpathlineto{\pgfqpoint{2.379561in}{1.148350in}}%
\pgfpathlineto{\pgfqpoint{2.386617in}{1.118062in}}%
\pgfpathlineto{\pgfqpoint{2.393673in}{1.106493in}}%
\pgfpathlineto{\pgfqpoint{2.407784in}{1.051336in}}%
\pgfpathlineto{\pgfqpoint{2.421896in}{1.011837in}}%
\pgfpathlineto{\pgfqpoint{2.428951in}{0.977869in}}%
\pgfpathlineto{\pgfqpoint{2.436007in}{0.966896in}}%
\pgfpathlineto{\pgfqpoint{2.443063in}{0.943444in}}%
\pgfpathlineto{\pgfqpoint{2.450119in}{0.908065in}}%
\pgfpathlineto{\pgfqpoint{2.457174in}{0.911905in}}%
\pgfpathlineto{\pgfqpoint{2.464230in}{0.868488in}}%
\pgfpathlineto{\pgfqpoint{2.471286in}{0.849214in}}%
\pgfpathlineto{\pgfqpoint{2.478341in}{0.849215in}}%
\pgfpathlineto{\pgfqpoint{2.485397in}{0.802106in}}%
\pgfpathlineto{\pgfqpoint{2.492453in}{0.798157in}}%
\pgfpathlineto{\pgfqpoint{2.499509in}{0.782492in}}%
\pgfpathlineto{\pgfqpoint{2.506564in}{0.742922in}}%
\pgfpathlineto{\pgfqpoint{2.513620in}{0.761204in}}%
\pgfpathlineto{\pgfqpoint{2.520676in}{0.719949in}}%
\pgfpathlineto{\pgfqpoint{2.527731in}{0.724430in}}%
\pgfpathlineto{\pgfqpoint{2.534787in}{0.723546in}}%
\pgfpathlineto{\pgfqpoint{2.541843in}{0.721106in}}%
\pgfpathlineto{\pgfqpoint{2.548898in}{0.753120in}}%
\pgfpathlineto{\pgfqpoint{2.555954in}{0.794060in}}%
\pgfpathlineto{\pgfqpoint{2.563010in}{0.862226in}}%
\pgfpathlineto{\pgfqpoint{2.570066in}{1.017902in}}%
\pgfpathlineto{\pgfqpoint{2.577121in}{1.221136in}}%
\pgfpathlineto{\pgfqpoint{2.584177in}{1.610101in}}%
\pgfpathlineto{\pgfqpoint{2.598288in}{2.517878in}}%
\pgfpathlineto{\pgfqpoint{2.605344in}{2.725634in}}%
\pgfpathlineto{\pgfqpoint{2.619456in}{2.770630in}}%
\pgfpathlineto{\pgfqpoint{2.633567in}{2.905017in}}%
\pgfpathlineto{\pgfqpoint{2.640623in}{2.914361in}}%
\pgfpathlineto{\pgfqpoint{2.654734in}{2.922060in}}%
\pgfpathlineto{\pgfqpoint{2.661790in}{2.959266in}}%
\pgfpathlineto{\pgfqpoint{2.668846in}{2.960677in}}%
\pgfpathlineto{\pgfqpoint{2.675901in}{3.000733in}}%
\pgfpathlineto{\pgfqpoint{2.682957in}{3.015503in}}%
\pgfpathlineto{\pgfqpoint{2.690013in}{3.034749in}}%
\pgfpathlineto{\pgfqpoint{2.697068in}{3.073876in}}%
\pgfpathlineto{\pgfqpoint{2.704124in}{3.075082in}}%
\pgfpathlineto{\pgfqpoint{2.718236in}{3.127665in}}%
\pgfpathlineto{\pgfqpoint{2.725291in}{3.135093in}}%
\pgfpathlineto{\pgfqpoint{2.732347in}{3.165111in}}%
\pgfpathlineto{\pgfqpoint{2.739403in}{3.172649in}}%
\pgfpathlineto{\pgfqpoint{2.746458in}{3.197687in}}%
\pgfpathlineto{\pgfqpoint{2.753514in}{3.197128in}}%
\pgfpathlineto{\pgfqpoint{2.760570in}{3.238734in}}%
\pgfpathlineto{\pgfqpoint{2.767626in}{3.222978in}}%
\pgfpathlineto{\pgfqpoint{2.774681in}{3.258017in}}%
\pgfpathlineto{\pgfqpoint{2.781737in}{3.253009in}}%
\pgfpathlineto{\pgfqpoint{2.795848in}{3.288593in}}%
\pgfpathlineto{\pgfqpoint{2.802904in}{3.272863in}}%
\pgfpathlineto{\pgfqpoint{2.809960in}{3.311537in}}%
\pgfpathlineto{\pgfqpoint{2.817015in}{3.296972in}}%
\pgfpathlineto{\pgfqpoint{2.831127in}{3.328488in}}%
\pgfpathlineto{\pgfqpoint{2.838183in}{3.311826in}}%
\pgfpathlineto{\pgfqpoint{2.852294in}{3.341237in}}%
\pgfpathlineto{\pgfqpoint{2.859350in}{3.320116in}}%
\pgfpathlineto{\pgfqpoint{2.866405in}{3.350027in}}%
\pgfpathlineto{\pgfqpoint{2.873461in}{3.336667in}}%
\pgfpathlineto{\pgfqpoint{2.880517in}{3.336816in}}%
\pgfpathlineto{\pgfqpoint{2.887573in}{3.341939in}}%
\pgfpathlineto{\pgfqpoint{2.894628in}{3.340283in}}%
\pgfpathlineto{\pgfqpoint{2.901684in}{3.330091in}}%
\pgfpathlineto{\pgfqpoint{2.908740in}{3.350262in}}%
\pgfpathlineto{\pgfqpoint{2.915795in}{3.317590in}}%
\pgfpathlineto{\pgfqpoint{2.922851in}{3.342859in}}%
\pgfpathlineto{\pgfqpoint{2.936963in}{3.314578in}}%
\pgfpathlineto{\pgfqpoint{2.944018in}{3.325002in}}%
\pgfpathlineto{\pgfqpoint{2.951074in}{3.303998in}}%
\pgfpathlineto{\pgfqpoint{2.965185in}{3.309603in}}%
\pgfpathlineto{\pgfqpoint{2.972241in}{3.282557in}}%
\pgfpathlineto{\pgfqpoint{2.979297in}{3.293719in}}%
\pgfpathlineto{\pgfqpoint{2.986353in}{3.276328in}}%
\pgfpathlineto{\pgfqpoint{2.993408in}{3.265253in}}%
\pgfpathlineto{\pgfqpoint{3.000464in}{3.262442in}}%
\pgfpathlineto{\pgfqpoint{3.028687in}{3.217881in}}%
\pgfpathlineto{\pgfqpoint{3.035743in}{3.213798in}}%
\pgfpathlineto{\pgfqpoint{3.042798in}{3.184803in}}%
\pgfpathlineto{\pgfqpoint{3.049854in}{3.200294in}}%
\pgfpathlineto{\pgfqpoint{3.056910in}{3.155898in}}%
\pgfpathlineto{\pgfqpoint{3.063965in}{3.169619in}}%
\pgfpathlineto{\pgfqpoint{3.071021in}{3.138750in}}%
\pgfpathlineto{\pgfqpoint{3.078077in}{3.136171in}}%
\pgfpathlineto{\pgfqpoint{3.085132in}{3.126078in}}%
\pgfpathlineto{\pgfqpoint{3.092188in}{3.105036in}}%
\pgfpathlineto{\pgfqpoint{3.099244in}{3.111956in}}%
\pgfpathlineto{\pgfqpoint{3.106300in}{3.087531in}}%
\pgfpathlineto{\pgfqpoint{3.113355in}{3.072825in}}%
\pgfpathlineto{\pgfqpoint{3.120411in}{3.075478in}}%
\pgfpathlineto{\pgfqpoint{3.127467in}{3.047267in}}%
\pgfpathlineto{\pgfqpoint{3.134522in}{3.043382in}}%
\pgfpathlineto{\pgfqpoint{3.141578in}{3.033395in}}%
\pgfpathlineto{\pgfqpoint{3.148634in}{3.006593in}}%
\pgfpathlineto{\pgfqpoint{3.155690in}{3.007659in}}%
\pgfpathlineto{\pgfqpoint{3.162745in}{2.984072in}}%
\pgfpathlineto{\pgfqpoint{3.169801in}{2.978642in}}%
\pgfpathlineto{\pgfqpoint{3.176857in}{2.957124in}}%
\pgfpathlineto{\pgfqpoint{3.183912in}{2.946652in}}%
\pgfpathlineto{\pgfqpoint{3.190968in}{2.929367in}}%
\pgfpathlineto{\pgfqpoint{3.198024in}{2.917065in}}%
\pgfpathlineto{\pgfqpoint{3.205080in}{2.894930in}}%
\pgfpathlineto{\pgfqpoint{3.219191in}{2.866856in}}%
\pgfpathlineto{\pgfqpoint{3.226247in}{2.843732in}}%
\pgfpathlineto{\pgfqpoint{3.233302in}{2.842405in}}%
\pgfpathlineto{\pgfqpoint{3.240358in}{2.802361in}}%
\pgfpathlineto{\pgfqpoint{3.247414in}{2.808469in}}%
\pgfpathlineto{\pgfqpoint{3.254470in}{2.772268in}}%
\pgfpathlineto{\pgfqpoint{3.261525in}{2.757276in}}%
\pgfpathlineto{\pgfqpoint{3.268581in}{2.754569in}}%
\pgfpathlineto{\pgfqpoint{3.275637in}{2.706188in}}%
\pgfpathlineto{\pgfqpoint{3.282692in}{2.720272in}}%
\pgfpathlineto{\pgfqpoint{3.289748in}{2.680671in}}%
\pgfpathlineto{\pgfqpoint{3.303860in}{2.651105in}}%
\pgfpathlineto{\pgfqpoint{3.310915in}{2.623668in}}%
\pgfpathlineto{\pgfqpoint{3.317971in}{2.605951in}}%
\pgfpathlineto{\pgfqpoint{3.325027in}{2.596010in}}%
\pgfpathlineto{\pgfqpoint{3.332082in}{2.548930in}}%
\pgfpathlineto{\pgfqpoint{3.339138in}{2.564374in}}%
\pgfpathlineto{\pgfqpoint{3.346194in}{2.522016in}}%
\pgfpathlineto{\pgfqpoint{3.353249in}{2.504926in}}%
\pgfpathlineto{\pgfqpoint{3.360305in}{2.497740in}}%
\pgfpathlineto{\pgfqpoint{3.367361in}{2.464159in}}%
\pgfpathlineto{\pgfqpoint{3.374417in}{2.443097in}}%
\pgfpathlineto{\pgfqpoint{3.381472in}{2.439400in}}%
\pgfpathlineto{\pgfqpoint{3.388528in}{2.392951in}}%
\pgfpathlineto{\pgfqpoint{3.395584in}{2.391836in}}%
\pgfpathlineto{\pgfqpoint{3.409695in}{2.340599in}}%
\pgfpathlineto{\pgfqpoint{3.416751in}{2.331162in}}%
\pgfpathlineto{\pgfqpoint{3.430862in}{2.282181in}}%
\pgfpathlineto{\pgfqpoint{3.437918in}{2.276536in}}%
\pgfpathlineto{\pgfqpoint{3.444974in}{2.237957in}}%
\pgfpathlineto{\pgfqpoint{3.452029in}{2.233370in}}%
\pgfpathlineto{\pgfqpoint{3.459085in}{2.215993in}}%
\pgfpathlineto{\pgfqpoint{3.466141in}{2.178200in}}%
\pgfpathlineto{\pgfqpoint{3.473197in}{2.191611in}}%
\pgfpathlineto{\pgfqpoint{3.480252in}{2.143578in}}%
\pgfpathlineto{\pgfqpoint{3.487308in}{2.147486in}}%
\pgfpathlineto{\pgfqpoint{3.494364in}{2.122533in}}%
\pgfpathlineto{\pgfqpoint{3.501419in}{2.112215in}}%
\pgfpathlineto{\pgfqpoint{3.508475in}{2.088767in}}%
\pgfpathlineto{\pgfqpoint{3.515531in}{2.079574in}}%
\pgfpathlineto{\pgfqpoint{3.522587in}{2.059953in}}%
\pgfpathlineto{\pgfqpoint{3.529642in}{2.051645in}}%
\pgfpathlineto{\pgfqpoint{3.536698in}{2.020439in}}%
\pgfpathlineto{\pgfqpoint{3.543754in}{2.026333in}}%
\pgfpathlineto{\pgfqpoint{3.557865in}{1.974525in}}%
\pgfpathlineto{\pgfqpoint{3.564921in}{1.977502in}}%
\pgfpathlineto{\pgfqpoint{3.571977in}{1.958238in}}%
\pgfpathlineto{\pgfqpoint{3.579032in}{1.925968in}}%
\pgfpathlineto{\pgfqpoint{3.586088in}{1.941131in}}%
\pgfpathlineto{\pgfqpoint{3.593144in}{1.887195in}}%
\pgfpathlineto{\pgfqpoint{3.600199in}{1.905831in}}%
\pgfpathlineto{\pgfqpoint{3.614311in}{1.850818in}}%
\pgfpathlineto{\pgfqpoint{3.621366in}{1.867371in}}%
\pgfpathlineto{\pgfqpoint{3.628422in}{1.809697in}}%
\pgfpathlineto{\pgfqpoint{3.635478in}{1.835671in}}%
\pgfpathlineto{\pgfqpoint{3.642534in}{1.786990in}}%
\pgfpathlineto{\pgfqpoint{3.649589in}{1.797832in}}%
\pgfpathlineto{\pgfqpoint{3.656645in}{1.766019in}}%
\pgfpathlineto{\pgfqpoint{3.663701in}{1.776395in}}%
\pgfpathlineto{\pgfqpoint{3.670756in}{1.729245in}}%
\pgfpathlineto{\pgfqpoint{3.677812in}{1.753569in}}%
\pgfpathlineto{\pgfqpoint{3.684868in}{1.699811in}}%
\pgfpathlineto{\pgfqpoint{3.691924in}{1.724959in}}%
\pgfpathlineto{\pgfqpoint{3.698979in}{1.682005in}}%
\pgfpathlineto{\pgfqpoint{3.706035in}{1.693084in}}%
\pgfpathlineto{\pgfqpoint{3.713091in}{1.667517in}}%
\pgfpathlineto{\pgfqpoint{3.720146in}{1.658569in}}%
\pgfpathlineto{\pgfqpoint{3.727202in}{1.652581in}}%
\pgfpathlineto{\pgfqpoint{3.734258in}{1.643821in}}%
\pgfpathlineto{\pgfqpoint{3.741314in}{1.620749in}}%
\pgfpathlineto{\pgfqpoint{3.748369in}{1.638253in}}%
\pgfpathlineto{\pgfqpoint{3.755425in}{1.593568in}}%
\pgfpathlineto{\pgfqpoint{3.762481in}{1.616010in}}%
\pgfpathlineto{\pgfqpoint{3.769536in}{1.591347in}}%
\pgfpathlineto{\pgfqpoint{3.776592in}{1.597661in}}%
\pgfpathlineto{\pgfqpoint{3.783648in}{1.584808in}}%
\pgfpathlineto{\pgfqpoint{3.790704in}{1.580449in}}%
\pgfpathlineto{\pgfqpoint{3.797759in}{1.569536in}}%
\pgfpathlineto{\pgfqpoint{3.804815in}{1.579515in}}%
\pgfpathlineto{\pgfqpoint{3.811871in}{1.549335in}}%
\pgfpathlineto{\pgfqpoint{3.818926in}{1.593647in}}%
\pgfpathlineto{\pgfqpoint{3.825982in}{1.541909in}}%
\pgfpathlineto{\pgfqpoint{3.833038in}{1.593085in}}%
\pgfpathlineto{\pgfqpoint{3.840094in}{1.540204in}}%
\pgfpathlineto{\pgfqpoint{3.847149in}{1.584513in}}%
\pgfpathlineto{\pgfqpoint{3.854205in}{1.552514in}}%
\pgfpathlineto{\pgfqpoint{3.861261in}{1.574034in}}%
\pgfpathlineto{\pgfqpoint{3.868316in}{1.578764in}}%
\pgfpathlineto{\pgfqpoint{3.875372in}{1.577015in}}%
\pgfpathlineto{\pgfqpoint{3.882428in}{1.590610in}}%
\pgfpathlineto{\pgfqpoint{3.889483in}{1.581392in}}%
\pgfpathlineto{\pgfqpoint{3.896539in}{1.614814in}}%
\pgfpathlineto{\pgfqpoint{3.903595in}{1.583627in}}%
\pgfpathlineto{\pgfqpoint{3.910651in}{1.624171in}}%
\pgfpathlineto{\pgfqpoint{3.917706in}{1.623256in}}%
\pgfpathlineto{\pgfqpoint{3.924762in}{1.635971in}}%
\pgfpathlineto{\pgfqpoint{3.931818in}{1.640686in}}%
\pgfpathlineto{\pgfqpoint{3.938873in}{1.667694in}}%
\pgfpathlineto{\pgfqpoint{3.945929in}{1.667491in}}%
\pgfpathlineto{\pgfqpoint{3.952985in}{1.686634in}}%
\pgfpathlineto{\pgfqpoint{3.960041in}{1.697857in}}%
\pgfpathlineto{\pgfqpoint{3.967096in}{1.718669in}}%
\pgfpathlineto{\pgfqpoint{3.974152in}{1.734018in}}%
\pgfpathlineto{\pgfqpoint{3.981208in}{1.756641in}}%
\pgfpathlineto{\pgfqpoint{4.016486in}{1.841046in}}%
\pgfpathlineto{\pgfqpoint{4.023542in}{1.863210in}}%
\pgfpathlineto{\pgfqpoint{4.030598in}{1.877376in}}%
\pgfpathlineto{\pgfqpoint{4.037653in}{1.902333in}}%
\pgfpathlineto{\pgfqpoint{4.044709in}{1.916637in}}%
\pgfpathlineto{\pgfqpoint{4.051765in}{1.943511in}}%
\pgfpathlineto{\pgfqpoint{4.058821in}{1.956899in}}%
\pgfpathlineto{\pgfqpoint{4.065876in}{1.982022in}}%
\pgfpathlineto{\pgfqpoint{4.072932in}{1.994406in}}%
\pgfpathlineto{\pgfqpoint{4.087043in}{2.028093in}}%
\pgfpathlineto{\pgfqpoint{4.094099in}{2.039710in}}%
\pgfpathlineto{\pgfqpoint{4.101155in}{2.065504in}}%
\pgfpathlineto{\pgfqpoint{4.108211in}{2.081856in}}%
\pgfpathlineto{\pgfqpoint{4.115266in}{2.087941in}}%
\pgfpathlineto{\pgfqpoint{4.122322in}{2.115159in}}%
\pgfpathlineto{\pgfqpoint{4.129378in}{2.113848in}}%
\pgfpathlineto{\pgfqpoint{4.136433in}{2.147466in}}%
\pgfpathlineto{\pgfqpoint{4.143489in}{2.141714in}}%
\pgfpathlineto{\pgfqpoint{4.150545in}{2.171669in}}%
\pgfpathlineto{\pgfqpoint{4.157600in}{2.171451in}}%
\pgfpathlineto{\pgfqpoint{4.171712in}{2.194346in}}%
\pgfpathlineto{\pgfqpoint{4.178768in}{2.203069in}}%
\pgfpathlineto{\pgfqpoint{4.185823in}{2.215482in}}%
\pgfpathlineto{\pgfqpoint{4.192879in}{2.217420in}}%
\pgfpathlineto{\pgfqpoint{4.199935in}{2.231055in}}%
\pgfpathlineto{\pgfqpoint{4.206990in}{2.232472in}}%
\pgfpathlineto{\pgfqpoint{4.214046in}{2.246412in}}%
\pgfpathlineto{\pgfqpoint{4.221102in}{2.247119in}}%
\pgfpathlineto{\pgfqpoint{4.235213in}{2.264450in}}%
\pgfpathlineto{\pgfqpoint{4.242269in}{2.266256in}}%
\pgfpathlineto{\pgfqpoint{4.249325in}{2.271194in}}%
\pgfpathlineto{\pgfqpoint{4.256380in}{2.281231in}}%
\pgfpathlineto{\pgfqpoint{4.263436in}{2.278824in}}%
\pgfpathlineto{\pgfqpoint{4.270492in}{2.286841in}}%
\pgfpathlineto{\pgfqpoint{4.277548in}{2.286592in}}%
\pgfpathlineto{\pgfqpoint{4.284603in}{2.293715in}}%
\pgfpathlineto{\pgfqpoint{4.291659in}{2.294575in}}%
\pgfpathlineto{\pgfqpoint{4.305770in}{2.301997in}}%
\pgfpathlineto{\pgfqpoint{4.312826in}{2.302030in}}%
\pgfpathlineto{\pgfqpoint{4.319882in}{2.307348in}}%
\pgfpathlineto{\pgfqpoint{4.326938in}{2.307080in}}%
\pgfpathlineto{\pgfqpoint{4.333993in}{2.312581in}}%
\pgfpathlineto{\pgfqpoint{4.341049in}{2.310991in}}%
\pgfpathlineto{\pgfqpoint{4.348105in}{2.315584in}}%
\pgfpathlineto{\pgfqpoint{4.355160in}{2.314568in}}%
\pgfpathlineto{\pgfqpoint{4.362216in}{2.318072in}}%
\pgfpathlineto{\pgfqpoint{4.376328in}{2.319395in}}%
\pgfpathlineto{\pgfqpoint{4.383383in}{2.322030in}}%
\pgfpathlineto{\pgfqpoint{4.390439in}{2.320167in}}%
\pgfpathlineto{\pgfqpoint{4.397495in}{2.326326in}}%
\pgfpathlineto{\pgfqpoint{4.404550in}{2.321935in}}%
\pgfpathlineto{\pgfqpoint{4.411606in}{2.326059in}}%
\pgfpathlineto{\pgfqpoint{4.432773in}{2.325985in}}%
\pgfpathlineto{\pgfqpoint{4.439829in}{2.327836in}}%
\pgfpathlineto{\pgfqpoint{4.453940in}{2.328446in}}%
\pgfpathlineto{\pgfqpoint{4.496275in}{2.332613in}}%
\pgfpathlineto{\pgfqpoint{4.517442in}{2.332283in}}%
\pgfpathlineto{\pgfqpoint{4.524497in}{2.334075in}}%
\pgfpathlineto{\pgfqpoint{4.531553in}{2.332742in}}%
\pgfpathlineto{\pgfqpoint{4.538609in}{2.334851in}}%
\pgfpathlineto{\pgfqpoint{4.545665in}{2.333581in}}%
\pgfpathlineto{\pgfqpoint{4.559776in}{2.334305in}}%
\pgfpathlineto{\pgfqpoint{4.623277in}{2.335127in}}%
\pgfpathlineto{\pgfqpoint{4.637389in}{2.335466in}}%
\pgfpathlineto{\pgfqpoint{4.672667in}{2.335382in}}%
\pgfpathlineto{\pgfqpoint{4.743224in}{2.335443in}}%
\pgfpathlineto{\pgfqpoint{5.060731in}{2.335470in}}%
\pgfpathlineto{\pgfqpoint{5.187734in}{2.335483in}}%
\pgfpathlineto{\pgfqpoint{5.187734in}{2.335483in}}%
\pgfusepath{stroke}%
\end{pgfscope}%
\begin{pgfscope}%
\pgfpathrectangle{\pgfqpoint{0.750000in}{0.500000in}}{\pgfqpoint{4.650000in}{3.020000in}}%
\pgfusepath{clip}%
\pgfsetrectcap%
\pgfsetroundjoin%
\pgfsetlinewidth{1.505625pt}%
\definecolor{currentstroke}{rgb}{1.000000,0.498039,0.054902}%
\pgfsetstrokecolor{currentstroke}%
\pgfsetdash{}{0pt}%
\pgfpathmoveto{\pgfqpoint{0.962266in}{1.186946in}}%
\pgfpathlineto{\pgfqpoint{0.990489in}{1.090124in}}%
\pgfpathlineto{\pgfqpoint{0.997544in}{1.059909in}}%
\pgfpathlineto{\pgfqpoint{1.004600in}{1.037427in}}%
\pgfpathlineto{\pgfqpoint{1.011656in}{1.005987in}}%
\pgfpathlineto{\pgfqpoint{1.025767in}{0.959664in}}%
\pgfpathlineto{\pgfqpoint{1.032823in}{0.926288in}}%
\pgfpathlineto{\pgfqpoint{1.039879in}{0.912896in}}%
\pgfpathlineto{\pgfqpoint{1.053990in}{0.852744in}}%
\pgfpathlineto{\pgfqpoint{1.061046in}{0.834055in}}%
\pgfpathlineto{\pgfqpoint{1.075157in}{0.776614in}}%
\pgfpathlineto{\pgfqpoint{1.082213in}{0.762683in}}%
\pgfpathlineto{\pgfqpoint{1.096324in}{0.701326in}}%
\pgfpathlineto{\pgfqpoint{1.103380in}{0.693880in}}%
\pgfpathlineto{\pgfqpoint{1.110436in}{0.664999in}}%
\pgfpathlineto{\pgfqpoint{1.117491in}{0.662222in}}%
\pgfpathlineto{\pgfqpoint{1.124547in}{0.643388in}}%
\pgfpathlineto{\pgfqpoint{1.131603in}{0.643453in}}%
\pgfpathlineto{\pgfqpoint{1.138659in}{0.664004in}}%
\pgfpathlineto{\pgfqpoint{1.145714in}{0.703599in}}%
\pgfpathlineto{\pgfqpoint{1.152770in}{0.785340in}}%
\pgfpathlineto{\pgfqpoint{1.159826in}{0.943812in}}%
\pgfpathlineto{\pgfqpoint{1.166881in}{1.149571in}}%
\pgfpathlineto{\pgfqpoint{1.188049in}{1.885039in}}%
\pgfpathlineto{\pgfqpoint{1.202160in}{2.296018in}}%
\pgfpathlineto{\pgfqpoint{1.209216in}{2.391729in}}%
\pgfpathlineto{\pgfqpoint{1.216271in}{2.431198in}}%
\pgfpathlineto{\pgfqpoint{1.223327in}{2.456357in}}%
\pgfpathlineto{\pgfqpoint{1.237438in}{2.558202in}}%
\pgfpathlineto{\pgfqpoint{1.251550in}{2.695156in}}%
\pgfpathlineto{\pgfqpoint{1.265661in}{2.798583in}}%
\pgfpathlineto{\pgfqpoint{1.272717in}{2.842178in}}%
\pgfpathlineto{\pgfqpoint{1.279773in}{2.869347in}}%
\pgfpathlineto{\pgfqpoint{1.286828in}{2.907337in}}%
\pgfpathlineto{\pgfqpoint{1.307996in}{2.987283in}}%
\pgfpathlineto{\pgfqpoint{1.315051in}{3.002448in}}%
\pgfpathlineto{\pgfqpoint{1.336218in}{3.065678in}}%
\pgfpathlineto{\pgfqpoint{1.343274in}{3.077066in}}%
\pgfpathlineto{\pgfqpoint{1.357386in}{3.110349in}}%
\pgfpathlineto{\pgfqpoint{1.364441in}{3.123184in}}%
\pgfpathlineto{\pgfqpoint{1.371497in}{3.143009in}}%
\pgfpathlineto{\pgfqpoint{1.378553in}{3.151243in}}%
\pgfpathlineto{\pgfqpoint{1.399720in}{3.195655in}}%
\pgfpathlineto{\pgfqpoint{1.413831in}{3.216540in}}%
\pgfpathlineto{\pgfqpoint{1.434998in}{3.239500in}}%
\pgfpathlineto{\pgfqpoint{1.442054in}{3.254469in}}%
\pgfpathlineto{\pgfqpoint{1.449110in}{3.255564in}}%
\pgfpathlineto{\pgfqpoint{1.456166in}{3.267962in}}%
\pgfpathlineto{\pgfqpoint{1.463221in}{3.269037in}}%
\pgfpathlineto{\pgfqpoint{1.470277in}{3.274248in}}%
\pgfpathlineto{\pgfqpoint{1.477333in}{3.284844in}}%
\pgfpathlineto{\pgfqpoint{1.484388in}{3.279646in}}%
\pgfpathlineto{\pgfqpoint{1.498500in}{3.292197in}}%
\pgfpathlineto{\pgfqpoint{1.505555in}{3.293447in}}%
\pgfpathlineto{\pgfqpoint{1.512611in}{3.299977in}}%
\pgfpathlineto{\pgfqpoint{1.519667in}{3.294820in}}%
\pgfpathlineto{\pgfqpoint{1.526723in}{3.292125in}}%
\pgfpathlineto{\pgfqpoint{1.533778in}{3.304838in}}%
\pgfpathlineto{\pgfqpoint{1.540834in}{3.292375in}}%
\pgfpathlineto{\pgfqpoint{1.547890in}{3.303758in}}%
\pgfpathlineto{\pgfqpoint{1.554945in}{3.289399in}}%
\pgfpathlineto{\pgfqpoint{1.562001in}{3.287019in}}%
\pgfpathlineto{\pgfqpoint{1.569057in}{3.291286in}}%
\pgfpathlineto{\pgfqpoint{1.576113in}{3.280701in}}%
\pgfpathlineto{\pgfqpoint{1.583168in}{3.283631in}}%
\pgfpathlineto{\pgfqpoint{1.597280in}{3.268196in}}%
\pgfpathlineto{\pgfqpoint{1.604335in}{3.264809in}}%
\pgfpathlineto{\pgfqpoint{1.618447in}{3.247219in}}%
\pgfpathlineto{\pgfqpoint{1.639614in}{3.226374in}}%
\pgfpathlineto{\pgfqpoint{1.646670in}{3.213594in}}%
\pgfpathlineto{\pgfqpoint{1.653725in}{3.206306in}}%
\pgfpathlineto{\pgfqpoint{1.660781in}{3.195022in}}%
\pgfpathlineto{\pgfqpoint{1.667837in}{3.188137in}}%
\pgfpathlineto{\pgfqpoint{1.717227in}{3.103430in}}%
\pgfpathlineto{\pgfqpoint{1.724283in}{3.083491in}}%
\pgfpathlineto{\pgfqpoint{1.731338in}{3.073341in}}%
\pgfpathlineto{\pgfqpoint{1.752505in}{3.027221in}}%
\pgfpathlineto{\pgfqpoint{1.759561in}{3.017467in}}%
\pgfpathlineto{\pgfqpoint{1.766617in}{2.996096in}}%
\pgfpathlineto{\pgfqpoint{1.773672in}{2.991119in}}%
\pgfpathlineto{\pgfqpoint{1.780728in}{2.965418in}}%
\pgfpathlineto{\pgfqpoint{1.787784in}{2.949360in}}%
\pgfpathlineto{\pgfqpoint{1.794840in}{2.941797in}}%
\pgfpathlineto{\pgfqpoint{1.808951in}{2.897108in}}%
\pgfpathlineto{\pgfqpoint{1.816007in}{2.889504in}}%
\pgfpathlineto{\pgfqpoint{1.823062in}{2.862782in}}%
\pgfpathlineto{\pgfqpoint{1.830118in}{2.850077in}}%
\pgfpathlineto{\pgfqpoint{1.837174in}{2.825772in}}%
\pgfpathlineto{\pgfqpoint{1.844230in}{2.811275in}}%
\pgfpathlineto{\pgfqpoint{1.851285in}{2.789052in}}%
\pgfpathlineto{\pgfqpoint{1.858341in}{2.779211in}}%
\pgfpathlineto{\pgfqpoint{1.865397in}{2.748323in}}%
\pgfpathlineto{\pgfqpoint{1.872452in}{2.736274in}}%
\pgfpathlineto{\pgfqpoint{1.879508in}{2.720922in}}%
\pgfpathlineto{\pgfqpoint{1.886564in}{2.689126in}}%
\pgfpathlineto{\pgfqpoint{1.893620in}{2.686935in}}%
\pgfpathlineto{\pgfqpoint{1.900675in}{2.652063in}}%
\pgfpathlineto{\pgfqpoint{1.907731in}{2.640399in}}%
\pgfpathlineto{\pgfqpoint{1.957121in}{2.491737in}}%
\pgfpathlineto{\pgfqpoint{1.964177in}{2.476504in}}%
\pgfpathlineto{\pgfqpoint{1.971232in}{2.448342in}}%
\pgfpathlineto{\pgfqpoint{1.985344in}{2.408680in}}%
\pgfpathlineto{\pgfqpoint{1.992400in}{2.383217in}}%
\pgfpathlineto{\pgfqpoint{1.999455in}{2.371693in}}%
\pgfpathlineto{\pgfqpoint{2.006511in}{2.337072in}}%
\pgfpathlineto{\pgfqpoint{2.013567in}{2.327072in}}%
\pgfpathlineto{\pgfqpoint{2.020622in}{2.297948in}}%
\pgfpathlineto{\pgfqpoint{2.027678in}{2.280301in}}%
\pgfpathlineto{\pgfqpoint{2.041789in}{2.234033in}}%
\pgfpathlineto{\pgfqpoint{2.091179in}{2.074010in}}%
\pgfpathlineto{\pgfqpoint{2.098235in}{2.051827in}}%
\pgfpathlineto{\pgfqpoint{2.105291in}{2.020734in}}%
\pgfpathlineto{\pgfqpoint{2.112347in}{2.012109in}}%
\pgfpathlineto{\pgfqpoint{2.119402in}{1.969469in}}%
\pgfpathlineto{\pgfqpoint{2.126458in}{1.962608in}}%
\pgfpathlineto{\pgfqpoint{2.133514in}{1.929312in}}%
\pgfpathlineto{\pgfqpoint{2.140569in}{1.911326in}}%
\pgfpathlineto{\pgfqpoint{2.161737in}{1.832513in}}%
\pgfpathlineto{\pgfqpoint{2.175848in}{1.792651in}}%
\pgfpathlineto{\pgfqpoint{2.182904in}{1.757068in}}%
\pgfpathlineto{\pgfqpoint{2.189959in}{1.750750in}}%
\pgfpathlineto{\pgfqpoint{2.197015in}{1.713960in}}%
\pgfpathlineto{\pgfqpoint{2.204071in}{1.698546in}}%
\pgfpathlineto{\pgfqpoint{2.218182in}{1.644397in}}%
\pgfpathlineto{\pgfqpoint{2.232294in}{1.598674in}}%
\pgfpathlineto{\pgfqpoint{2.260517in}{1.501679in}}%
\pgfpathlineto{\pgfqpoint{2.267572in}{1.468084in}}%
\pgfpathlineto{\pgfqpoint{2.274628in}{1.463538in}}%
\pgfpathlineto{\pgfqpoint{2.281684in}{1.427338in}}%
\pgfpathlineto{\pgfqpoint{2.288739in}{1.400540in}}%
\pgfpathlineto{\pgfqpoint{2.295795in}{1.392926in}}%
\pgfpathlineto{\pgfqpoint{2.302851in}{1.348106in}}%
\pgfpathlineto{\pgfqpoint{2.309906in}{1.342738in}}%
\pgfpathlineto{\pgfqpoint{2.324018in}{1.279547in}}%
\pgfpathlineto{\pgfqpoint{2.331074in}{1.264233in}}%
\pgfpathlineto{\pgfqpoint{2.345185in}{1.211663in}}%
\pgfpathlineto{\pgfqpoint{2.352241in}{1.196608in}}%
\pgfpathlineto{\pgfqpoint{2.359296in}{1.157227in}}%
\pgfpathlineto{\pgfqpoint{2.366352in}{1.140575in}}%
\pgfpathlineto{\pgfqpoint{2.387519in}{1.064498in}}%
\pgfpathlineto{\pgfqpoint{2.394575in}{1.049360in}}%
\pgfpathlineto{\pgfqpoint{2.415742in}{0.967538in}}%
\pgfpathlineto{\pgfqpoint{2.422798in}{0.958963in}}%
\pgfpathlineto{\pgfqpoint{2.429854in}{0.918862in}}%
\pgfpathlineto{\pgfqpoint{2.436909in}{0.903311in}}%
\pgfpathlineto{\pgfqpoint{2.443965in}{0.883251in}}%
\pgfpathlineto{\pgfqpoint{2.451021in}{0.849212in}}%
\pgfpathlineto{\pgfqpoint{2.458076in}{0.835350in}}%
\pgfpathlineto{\pgfqpoint{2.465132in}{0.805264in}}%
\pgfpathlineto{\pgfqpoint{2.479244in}{0.761446in}}%
\pgfpathlineto{\pgfqpoint{2.486299in}{0.751020in}}%
\pgfpathlineto{\pgfqpoint{2.493355in}{0.722200in}}%
\pgfpathlineto{\pgfqpoint{2.500411in}{0.703106in}}%
\pgfpathlineto{\pgfqpoint{2.507466in}{0.693271in}}%
\pgfpathlineto{\pgfqpoint{2.521578in}{0.653732in}}%
\pgfpathlineto{\pgfqpoint{2.528634in}{0.654643in}}%
\pgfpathlineto{\pgfqpoint{2.535689in}{0.637273in}}%
\pgfpathlineto{\pgfqpoint{2.549801in}{0.665295in}}%
\pgfpathlineto{\pgfqpoint{2.556856in}{0.695279in}}%
\pgfpathlineto{\pgfqpoint{2.570968in}{0.888783in}}%
\pgfpathlineto{\pgfqpoint{2.578023in}{1.151710in}}%
\pgfpathlineto{\pgfqpoint{2.585079in}{1.565294in}}%
\pgfpathlineto{\pgfqpoint{2.599191in}{2.587085in}}%
\pgfpathlineto{\pgfqpoint{2.606246in}{2.726925in}}%
\pgfpathlineto{\pgfqpoint{2.627413in}{2.938174in}}%
\pgfpathlineto{\pgfqpoint{2.634469in}{2.904658in}}%
\pgfpathlineto{\pgfqpoint{2.641525in}{2.917587in}}%
\pgfpathlineto{\pgfqpoint{2.648581in}{2.920803in}}%
\pgfpathlineto{\pgfqpoint{2.655636in}{2.946435in}}%
\pgfpathlineto{\pgfqpoint{2.662692in}{2.963811in}}%
\pgfpathlineto{\pgfqpoint{2.705026in}{3.104978in}}%
\pgfpathlineto{\pgfqpoint{2.712082in}{3.122899in}}%
\pgfpathlineto{\pgfqpoint{2.726193in}{3.169252in}}%
\pgfpathlineto{\pgfqpoint{2.733249in}{3.178725in}}%
\pgfpathlineto{\pgfqpoint{2.740305in}{3.203222in}}%
\pgfpathlineto{\pgfqpoint{2.747361in}{3.214415in}}%
\pgfpathlineto{\pgfqpoint{2.754416in}{3.236791in}}%
\pgfpathlineto{\pgfqpoint{2.761472in}{3.251492in}}%
\pgfpathlineto{\pgfqpoint{2.768528in}{3.260192in}}%
\pgfpathlineto{\pgfqpoint{2.775583in}{3.283470in}}%
\pgfpathlineto{\pgfqpoint{2.782639in}{3.276927in}}%
\pgfpathlineto{\pgfqpoint{2.789695in}{3.313166in}}%
\pgfpathlineto{\pgfqpoint{2.796751in}{3.298954in}}%
\pgfpathlineto{\pgfqpoint{2.803806in}{3.327906in}}%
\pgfpathlineto{\pgfqpoint{2.810862in}{3.332144in}}%
\pgfpathlineto{\pgfqpoint{2.817918in}{3.330145in}}%
\pgfpathlineto{\pgfqpoint{2.824973in}{3.356196in}}%
\pgfpathlineto{\pgfqpoint{2.832029in}{3.346804in}}%
\pgfpathlineto{\pgfqpoint{2.846140in}{3.368222in}}%
\pgfpathlineto{\pgfqpoint{2.853196in}{3.361899in}}%
\pgfpathlineto{\pgfqpoint{2.860252in}{3.372807in}}%
\pgfpathlineto{\pgfqpoint{2.867308in}{3.378243in}}%
\pgfpathlineto{\pgfqpoint{2.888475in}{3.376998in}}%
\pgfpathlineto{\pgfqpoint{2.895530in}{3.380862in}}%
\pgfpathlineto{\pgfqpoint{2.902586in}{3.375329in}}%
\pgfpathlineto{\pgfqpoint{2.909642in}{3.374344in}}%
\pgfpathlineto{\pgfqpoint{2.916698in}{3.382727in}}%
\pgfpathlineto{\pgfqpoint{2.923753in}{3.362790in}}%
\pgfpathlineto{\pgfqpoint{2.930809in}{3.380651in}}%
\pgfpathlineto{\pgfqpoint{2.937865in}{3.355776in}}%
\pgfpathlineto{\pgfqpoint{2.944920in}{3.360433in}}%
\pgfpathlineto{\pgfqpoint{2.951976in}{3.358860in}}%
\pgfpathlineto{\pgfqpoint{2.959032in}{3.336926in}}%
\pgfpathlineto{\pgfqpoint{2.966088in}{3.354589in}}%
\pgfpathlineto{\pgfqpoint{2.973143in}{3.324383in}}%
\pgfpathlineto{\pgfqpoint{2.980199in}{3.335691in}}%
\pgfpathlineto{\pgfqpoint{2.987255in}{3.313836in}}%
\pgfpathlineto{\pgfqpoint{2.994310in}{3.310938in}}%
\pgfpathlineto{\pgfqpoint{3.001366in}{3.295135in}}%
\pgfpathlineto{\pgfqpoint{3.008422in}{3.302018in}}%
\pgfpathlineto{\pgfqpoint{3.015478in}{3.268885in}}%
\pgfpathlineto{\pgfqpoint{3.022533in}{3.276207in}}%
\pgfpathlineto{\pgfqpoint{3.029589in}{3.250804in}}%
\pgfpathlineto{\pgfqpoint{3.036645in}{3.254181in}}%
\pgfpathlineto{\pgfqpoint{3.043700in}{3.229140in}}%
\pgfpathlineto{\pgfqpoint{3.050756in}{3.224353in}}%
\pgfpathlineto{\pgfqpoint{3.064868in}{3.189913in}}%
\pgfpathlineto{\pgfqpoint{3.071923in}{3.186658in}}%
\pgfpathlineto{\pgfqpoint{3.078979in}{3.160374in}}%
\pgfpathlineto{\pgfqpoint{3.086035in}{3.157313in}}%
\pgfpathlineto{\pgfqpoint{3.093090in}{3.132179in}}%
\pgfpathlineto{\pgfqpoint{3.100146in}{3.137399in}}%
\pgfpathlineto{\pgfqpoint{3.107202in}{3.099992in}}%
\pgfpathlineto{\pgfqpoint{3.114257in}{3.101804in}}%
\pgfpathlineto{\pgfqpoint{3.121313in}{3.074815in}}%
\pgfpathlineto{\pgfqpoint{3.128369in}{3.060842in}}%
\pgfpathlineto{\pgfqpoint{3.135425in}{3.054393in}}%
\pgfpathlineto{\pgfqpoint{3.142480in}{3.023646in}}%
\pgfpathlineto{\pgfqpoint{3.149536in}{3.018434in}}%
\pgfpathlineto{\pgfqpoint{3.156592in}{2.993888in}}%
\pgfpathlineto{\pgfqpoint{3.163647in}{2.991199in}}%
\pgfpathlineto{\pgfqpoint{3.170703in}{2.953995in}}%
\pgfpathlineto{\pgfqpoint{3.177759in}{2.960910in}}%
\pgfpathlineto{\pgfqpoint{3.184815in}{2.926089in}}%
\pgfpathlineto{\pgfqpoint{3.191870in}{2.913605in}}%
\pgfpathlineto{\pgfqpoint{3.198926in}{2.908233in}}%
\pgfpathlineto{\pgfqpoint{3.205982in}{2.864447in}}%
\pgfpathlineto{\pgfqpoint{3.213037in}{2.870571in}}%
\pgfpathlineto{\pgfqpoint{3.220093in}{2.835363in}}%
\pgfpathlineto{\pgfqpoint{3.234205in}{2.812342in}}%
\pgfpathlineto{\pgfqpoint{3.241260in}{2.777481in}}%
\pgfpathlineto{\pgfqpoint{3.248316in}{2.771701in}}%
\pgfpathlineto{\pgfqpoint{3.255372in}{2.738011in}}%
\pgfpathlineto{\pgfqpoint{3.262427in}{2.735062in}}%
\pgfpathlineto{\pgfqpoint{3.269483in}{2.695020in}}%
\pgfpathlineto{\pgfqpoint{3.276539in}{2.691131in}}%
\pgfpathlineto{\pgfqpoint{3.283595in}{2.668492in}}%
\pgfpathlineto{\pgfqpoint{3.290650in}{2.631531in}}%
\pgfpathlineto{\pgfqpoint{3.297706in}{2.648865in}}%
\pgfpathlineto{\pgfqpoint{3.304762in}{2.574668in}}%
\pgfpathlineto{\pgfqpoint{3.311817in}{2.605790in}}%
\pgfpathlineto{\pgfqpoint{3.318873in}{2.548231in}}%
\pgfpathlineto{\pgfqpoint{3.332985in}{2.526510in}}%
\pgfpathlineto{\pgfqpoint{3.340040in}{2.486783in}}%
\pgfpathlineto{\pgfqpoint{3.347096in}{2.484473in}}%
\pgfpathlineto{\pgfqpoint{3.361207in}{2.433084in}}%
\pgfpathlineto{\pgfqpoint{3.368263in}{2.409093in}}%
\pgfpathlineto{\pgfqpoint{3.375319in}{2.403714in}}%
\pgfpathlineto{\pgfqpoint{3.382374in}{2.356121in}}%
\pgfpathlineto{\pgfqpoint{3.389430in}{2.350342in}}%
\pgfpathlineto{\pgfqpoint{3.403542in}{2.298770in}}%
\pgfpathlineto{\pgfqpoint{3.410597in}{2.291134in}}%
\pgfpathlineto{\pgfqpoint{3.417653in}{2.248544in}}%
\pgfpathlineto{\pgfqpoint{3.424709in}{2.250389in}}%
\pgfpathlineto{\pgfqpoint{3.431764in}{2.211956in}}%
\pgfpathlineto{\pgfqpoint{3.438820in}{2.203145in}}%
\pgfpathlineto{\pgfqpoint{3.445876in}{2.172530in}}%
\pgfpathlineto{\pgfqpoint{3.452932in}{2.155750in}}%
\pgfpathlineto{\pgfqpoint{3.459987in}{2.145614in}}%
\pgfpathlineto{\pgfqpoint{3.467043in}{2.105794in}}%
\pgfpathlineto{\pgfqpoint{3.474099in}{2.099523in}}%
\pgfpathlineto{\pgfqpoint{3.481154in}{2.082039in}}%
\pgfpathlineto{\pgfqpoint{3.495266in}{2.032668in}}%
\pgfpathlineto{\pgfqpoint{3.502322in}{2.029856in}}%
\pgfpathlineto{\pgfqpoint{3.509377in}{1.989221in}}%
\pgfpathlineto{\pgfqpoint{3.516433in}{1.987801in}}%
\pgfpathlineto{\pgfqpoint{3.523489in}{1.959369in}}%
\pgfpathlineto{\pgfqpoint{3.530544in}{1.942643in}}%
\pgfpathlineto{\pgfqpoint{3.537600in}{1.932686in}}%
\pgfpathlineto{\pgfqpoint{3.544656in}{1.903586in}}%
\pgfpathlineto{\pgfqpoint{3.551712in}{1.886394in}}%
\pgfpathlineto{\pgfqpoint{3.558767in}{1.880792in}}%
\pgfpathlineto{\pgfqpoint{3.565823in}{1.852207in}}%
\pgfpathlineto{\pgfqpoint{3.572879in}{1.845306in}}%
\pgfpathlineto{\pgfqpoint{3.579934in}{1.830929in}}%
\pgfpathlineto{\pgfqpoint{3.586990in}{1.802582in}}%
\pgfpathlineto{\pgfqpoint{3.594046in}{1.797401in}}%
\pgfpathlineto{\pgfqpoint{3.601102in}{1.776885in}}%
\pgfpathlineto{\pgfqpoint{3.608157in}{1.785892in}}%
\pgfpathlineto{\pgfqpoint{3.615213in}{1.727683in}}%
\pgfpathlineto{\pgfqpoint{3.622269in}{1.773290in}}%
\pgfpathlineto{\pgfqpoint{3.629324in}{1.717827in}}%
\pgfpathlineto{\pgfqpoint{3.636380in}{1.730988in}}%
\pgfpathlineto{\pgfqpoint{3.643436in}{1.717529in}}%
\pgfpathlineto{\pgfqpoint{3.650491in}{1.713754in}}%
\pgfpathlineto{\pgfqpoint{3.657547in}{1.728219in}}%
\pgfpathlineto{\pgfqpoint{3.671659in}{1.699961in}}%
\pgfpathlineto{\pgfqpoint{3.678714in}{1.727578in}}%
\pgfpathlineto{\pgfqpoint{3.685770in}{1.692488in}}%
\pgfpathlineto{\pgfqpoint{3.692826in}{1.707243in}}%
\pgfpathlineto{\pgfqpoint{3.699881in}{1.697659in}}%
\pgfpathlineto{\pgfqpoint{3.706937in}{1.706376in}}%
\pgfpathlineto{\pgfqpoint{3.713993in}{1.678676in}}%
\pgfpathlineto{\pgfqpoint{3.721049in}{1.708663in}}%
\pgfpathlineto{\pgfqpoint{3.728104in}{1.685341in}}%
\pgfpathlineto{\pgfqpoint{3.735160in}{1.697764in}}%
\pgfpathlineto{\pgfqpoint{3.742216in}{1.682265in}}%
\pgfpathlineto{\pgfqpoint{3.749271in}{1.711406in}}%
\pgfpathlineto{\pgfqpoint{3.756327in}{1.665167in}}%
\pgfpathlineto{\pgfqpoint{3.763383in}{1.700503in}}%
\pgfpathlineto{\pgfqpoint{3.770439in}{1.691480in}}%
\pgfpathlineto{\pgfqpoint{3.777494in}{1.688185in}}%
\pgfpathlineto{\pgfqpoint{3.784550in}{1.706184in}}%
\pgfpathlineto{\pgfqpoint{3.791606in}{1.674476in}}%
\pgfpathlineto{\pgfqpoint{3.798661in}{1.707882in}}%
\pgfpathlineto{\pgfqpoint{3.805717in}{1.690189in}}%
\pgfpathlineto{\pgfqpoint{3.812773in}{1.689256in}}%
\pgfpathlineto{\pgfqpoint{3.819829in}{1.714703in}}%
\pgfpathlineto{\pgfqpoint{3.826884in}{1.694432in}}%
\pgfpathlineto{\pgfqpoint{3.833940in}{1.712591in}}%
\pgfpathlineto{\pgfqpoint{3.840996in}{1.702800in}}%
\pgfpathlineto{\pgfqpoint{3.848051in}{1.710316in}}%
\pgfpathlineto{\pgfqpoint{3.855107in}{1.724313in}}%
\pgfpathlineto{\pgfqpoint{3.862163in}{1.714737in}}%
\pgfpathlineto{\pgfqpoint{3.869219in}{1.723998in}}%
\pgfpathlineto{\pgfqpoint{3.876274in}{1.737995in}}%
\pgfpathlineto{\pgfqpoint{3.883330in}{1.719513in}}%
\pgfpathlineto{\pgfqpoint{3.890386in}{1.741483in}}%
\pgfpathlineto{\pgfqpoint{3.897441in}{1.740675in}}%
\pgfpathlineto{\pgfqpoint{3.904497in}{1.742892in}}%
\pgfpathlineto{\pgfqpoint{3.918608in}{1.759201in}}%
\pgfpathlineto{\pgfqpoint{3.925664in}{1.763471in}}%
\pgfpathlineto{\pgfqpoint{3.932720in}{1.774449in}}%
\pgfpathlineto{\pgfqpoint{3.939776in}{1.772225in}}%
\pgfpathlineto{\pgfqpoint{3.946831in}{1.798673in}}%
\pgfpathlineto{\pgfqpoint{3.953887in}{1.791098in}}%
\pgfpathlineto{\pgfqpoint{3.967998in}{1.813300in}}%
\pgfpathlineto{\pgfqpoint{3.975054in}{1.832178in}}%
\pgfpathlineto{\pgfqpoint{3.982110in}{1.819391in}}%
\pgfpathlineto{\pgfqpoint{3.989166in}{1.858435in}}%
\pgfpathlineto{\pgfqpoint{3.996221in}{1.844461in}}%
\pgfpathlineto{\pgfqpoint{4.003277in}{1.872702in}}%
\pgfpathlineto{\pgfqpoint{4.010333in}{1.866741in}}%
\pgfpathlineto{\pgfqpoint{4.017388in}{1.898730in}}%
\pgfpathlineto{\pgfqpoint{4.024444in}{1.894842in}}%
\pgfpathlineto{\pgfqpoint{4.038556in}{1.930693in}}%
\pgfpathlineto{\pgfqpoint{4.045611in}{1.931381in}}%
\pgfpathlineto{\pgfqpoint{4.052667in}{1.954415in}}%
\pgfpathlineto{\pgfqpoint{4.059723in}{1.954098in}}%
\pgfpathlineto{\pgfqpoint{4.066778in}{1.976105in}}%
\pgfpathlineto{\pgfqpoint{4.073834in}{1.980359in}}%
\pgfpathlineto{\pgfqpoint{4.080890in}{1.995475in}}%
\pgfpathlineto{\pgfqpoint{4.087946in}{2.001939in}}%
\pgfpathlineto{\pgfqpoint{4.095001in}{2.012935in}}%
\pgfpathlineto{\pgfqpoint{4.102057in}{2.031490in}}%
\pgfpathlineto{\pgfqpoint{4.109113in}{2.033044in}}%
\pgfpathlineto{\pgfqpoint{4.116168in}{2.060628in}}%
\pgfpathlineto{\pgfqpoint{4.123224in}{2.056960in}}%
\pgfpathlineto{\pgfqpoint{4.130280in}{2.078963in}}%
\pgfpathlineto{\pgfqpoint{4.137336in}{2.089481in}}%
\pgfpathlineto{\pgfqpoint{4.144391in}{2.095005in}}%
\pgfpathlineto{\pgfqpoint{4.151447in}{2.115257in}}%
\pgfpathlineto{\pgfqpoint{4.158503in}{2.117300in}}%
\pgfpathlineto{\pgfqpoint{4.165558in}{2.134496in}}%
\pgfpathlineto{\pgfqpoint{4.186725in}{2.163467in}}%
\pgfpathlineto{\pgfqpoint{4.193781in}{2.170972in}}%
\pgfpathlineto{\pgfqpoint{4.200837in}{2.181369in}}%
\pgfpathlineto{\pgfqpoint{4.207893in}{2.195067in}}%
\pgfpathlineto{\pgfqpoint{4.214948in}{2.198152in}}%
\pgfpathlineto{\pgfqpoint{4.222004in}{2.211678in}}%
\pgfpathlineto{\pgfqpoint{4.229060in}{2.217444in}}%
\pgfpathlineto{\pgfqpoint{4.236115in}{2.231674in}}%
\pgfpathlineto{\pgfqpoint{4.243171in}{2.238248in}}%
\pgfpathlineto{\pgfqpoint{4.250227in}{2.249180in}}%
\pgfpathlineto{\pgfqpoint{4.257283in}{2.251396in}}%
\pgfpathlineto{\pgfqpoint{4.264338in}{2.264178in}}%
\pgfpathlineto{\pgfqpoint{4.271394in}{2.264860in}}%
\pgfpathlineto{\pgfqpoint{4.278450in}{2.281938in}}%
\pgfpathlineto{\pgfqpoint{4.285505in}{2.277025in}}%
\pgfpathlineto{\pgfqpoint{4.292561in}{2.289659in}}%
\pgfpathlineto{\pgfqpoint{4.306673in}{2.300313in}}%
\pgfpathlineto{\pgfqpoint{4.313728in}{2.307657in}}%
\pgfpathlineto{\pgfqpoint{4.320784in}{2.312526in}}%
\pgfpathlineto{\pgfqpoint{4.327840in}{2.321336in}}%
\pgfpathlineto{\pgfqpoint{4.349007in}{2.335593in}}%
\pgfpathlineto{\pgfqpoint{4.398397in}{2.335327in}}%
\pgfpathlineto{\pgfqpoint{5.188636in}{2.335497in}}%
\pgfpathlineto{\pgfqpoint{5.188636in}{2.335497in}}%
\pgfusepath{stroke}%
\end{pgfscope}%
\begin{pgfscope}%
\pgfsetrectcap%
\pgfsetmiterjoin%
\pgfsetlinewidth{0.803000pt}%
\definecolor{currentstroke}{rgb}{0.000000,0.000000,0.000000}%
\pgfsetstrokecolor{currentstroke}%
\pgfsetdash{}{0pt}%
\pgfpathmoveto{\pgfqpoint{0.750000in}{0.500000in}}%
\pgfpathlineto{\pgfqpoint{0.750000in}{3.520000in}}%
\pgfusepath{stroke}%
\end{pgfscope}%
\begin{pgfscope}%
\pgfsetrectcap%
\pgfsetmiterjoin%
\pgfsetlinewidth{0.803000pt}%
\definecolor{currentstroke}{rgb}{0.000000,0.000000,0.000000}%
\pgfsetstrokecolor{currentstroke}%
\pgfsetdash{}{0pt}%
\pgfpathmoveto{\pgfqpoint{5.400000in}{0.500000in}}%
\pgfpathlineto{\pgfqpoint{5.400000in}{3.520000in}}%
\pgfusepath{stroke}%
\end{pgfscope}%
\begin{pgfscope}%
\pgfsetrectcap%
\pgfsetmiterjoin%
\pgfsetlinewidth{0.803000pt}%
\definecolor{currentstroke}{rgb}{0.000000,0.000000,0.000000}%
\pgfsetstrokecolor{currentstroke}%
\pgfsetdash{}{0pt}%
\pgfpathmoveto{\pgfqpoint{0.750000in}{0.500000in}}%
\pgfpathlineto{\pgfqpoint{5.400000in}{0.500000in}}%
\pgfusepath{stroke}%
\end{pgfscope}%
\begin{pgfscope}%
\pgfsetrectcap%
\pgfsetmiterjoin%
\pgfsetlinewidth{0.803000pt}%
\definecolor{currentstroke}{rgb}{0.000000,0.000000,0.000000}%
\pgfsetstrokecolor{currentstroke}%
\pgfsetdash{}{0pt}%
\pgfpathmoveto{\pgfqpoint{0.750000in}{3.520000in}}%
\pgfpathlineto{\pgfqpoint{5.400000in}{3.520000in}}%
\pgfusepath{stroke}%
\end{pgfscope}%
\begin{pgfscope}%
\pgfsetbuttcap%
\pgfsetmiterjoin%
\definecolor{currentfill}{rgb}{1.000000,1.000000,1.000000}%
\pgfsetfillcolor{currentfill}%
\pgfsetfillopacity{0.800000}%
\pgfsetlinewidth{1.003750pt}%
\definecolor{currentstroke}{rgb}{0.800000,0.800000,0.800000}%
\pgfsetstrokecolor{currentstroke}%
\pgfsetstrokeopacity{0.800000}%
\pgfsetdash{}{0pt}%
\pgfpathmoveto{\pgfqpoint{3.622607in}{3.021543in}}%
\pgfpathlineto{\pgfqpoint{5.302778in}{3.021543in}}%
\pgfpathquadraticcurveto{\pgfqpoint{5.330556in}{3.021543in}}{\pgfqpoint{5.330556in}{3.049321in}}%
\pgfpathlineto{\pgfqpoint{5.330556in}{3.422778in}}%
\pgfpathquadraticcurveto{\pgfqpoint{5.330556in}{3.450556in}}{\pgfqpoint{5.302778in}{3.450556in}}%
\pgfpathlineto{\pgfqpoint{3.622607in}{3.450556in}}%
\pgfpathquadraticcurveto{\pgfqpoint{3.594829in}{3.450556in}}{\pgfqpoint{3.594829in}{3.422778in}}%
\pgfpathlineto{\pgfqpoint{3.594829in}{3.049321in}}%
\pgfpathquadraticcurveto{\pgfqpoint{3.594829in}{3.021543in}}{\pgfqpoint{3.622607in}{3.021543in}}%
\pgfpathlineto{\pgfqpoint{3.622607in}{3.021543in}}%
\pgfpathclose%
\pgfusepath{stroke,fill}%
\end{pgfscope}%
\begin{pgfscope}%
\pgfsetrectcap%
\pgfsetroundjoin%
\pgfsetlinewidth{1.505625pt}%
\definecolor{currentstroke}{rgb}{0.121569,0.466667,0.705882}%
\pgfsetstrokecolor{currentstroke}%
\pgfsetdash{}{0pt}%
\pgfpathmoveto{\pgfqpoint{3.650385in}{3.346389in}}%
\pgfpathlineto{\pgfqpoint{3.789274in}{3.346389in}}%
\pgfpathlineto{\pgfqpoint{3.928162in}{3.346389in}}%
\pgfusepath{stroke}%
\end{pgfscope}%
\begin{pgfscope}%
\definecolor{textcolor}{rgb}{0.000000,0.000000,0.000000}%
\pgfsetstrokecolor{textcolor}%
\pgfsetfillcolor{textcolor}%
\pgftext[x=4.039274in,y=3.297778in,left,base]{\color{textcolor}\sffamily\fontsize{10.000000}{12.000000}\selectfont gaussian distribution  }%
\end{pgfscope}%
\begin{pgfscope}%
\pgfsetrectcap%
\pgfsetroundjoin%
\pgfsetlinewidth{1.505625pt}%
\definecolor{currentstroke}{rgb}{1.000000,0.498039,0.054902}%
\pgfsetstrokecolor{currentstroke}%
\pgfsetdash{}{0pt}%
\pgfpathmoveto{\pgfqpoint{3.650385in}{3.152716in}}%
\pgfpathlineto{\pgfqpoint{3.789274in}{3.152716in}}%
\pgfpathlineto{\pgfqpoint{3.928162in}{3.152716in}}%
\pgfusepath{stroke}%
\end{pgfscope}%
\begin{pgfscope}%
\definecolor{textcolor}{rgb}{0.000000,0.000000,0.000000}%
\pgfsetstrokecolor{textcolor}%
\pgfsetfillcolor{textcolor}%
\pgftext[x=4.039274in,y=3.104105in,left,base]{\color{textcolor}\sffamily\fontsize{10.000000}{12.000000}\selectfont uniform distribution}%
\end{pgfscope}%
\end{pgfpicture}%
\makeatother%
\endgroup%

	\caption{Witness energy gain for two different initial charge distributions.}
	\label{fig:gain_square}
\end{figure}
The biggest difference in the two curves lies in the higher minimum positioned at the backside of the driver, that results in smaller energy loss and therefore a \qty{0.7}{mm} longer traversed distance before breakup.

\autoref{fig:q_series_square} shows a timeseries of the uniform driver. When compared to \autoref{fig:q_series}, the similarity in the transformation is visible. The formed tail does not get as thin as for the fully Gaussian driver, as the overshoot happens before the outer particles reach $z=0$.
\begin{figure}
	\centering
	%% Creator: Matplotlib, PGF backend
%%
%% To include the figure in your LaTeX document, write
%%   \input{<filename>.pgf}
%%
%% Make sure the required packages are loaded in your preamble
%%   \usepackage{pgf}
%%
%% Also ensure that all the required font packages are loaded; for instance,
%% the lmodern package is sometimes necessary when using math font.
%%   \usepackage{lmodern}
%%
%% Figures using additional raster images can only be included by \input if
%% they are in the same directory as the main LaTeX file. For loading figures
%% from other directories you can use the `import` package
%%   \usepackage{import}
%%
%% and then include the figures with
%%   \import{<path to file>}{<filename>.pgf}
%%
%% Matplotlib used the following preamble
%%
\begingroup%
\makeatletter%
\begin{pgfpicture}%
\pgfpathrectangle{\pgfpointorigin}{\pgfqpoint{6.400000in}{7.000000in}}%
\pgfusepath{use as bounding box, clip}%
\begin{pgfscope}%
\pgfsetbuttcap%
\pgfsetmiterjoin%
\pgfsetlinewidth{0.000000pt}%
\definecolor{currentstroke}{rgb}{1.000000,1.000000,1.000000}%
\pgfsetstrokecolor{currentstroke}%
\pgfsetstrokeopacity{0.000000}%
\pgfsetdash{}{0pt}%
\pgfpathmoveto{\pgfqpoint{0.000000in}{0.000000in}}%
\pgfpathlineto{\pgfqpoint{6.400000in}{0.000000in}}%
\pgfpathlineto{\pgfqpoint{6.400000in}{7.000000in}}%
\pgfpathlineto{\pgfqpoint{0.000000in}{7.000000in}}%
\pgfpathlineto{\pgfqpoint{0.000000in}{0.000000in}}%
\pgfpathclose%
\pgfusepath{}%
\end{pgfscope}%
\begin{pgfscope}%
\pgfsetbuttcap%
\pgfsetmiterjoin%
\definecolor{currentfill}{rgb}{1.000000,1.000000,1.000000}%
\pgfsetfillcolor{currentfill}%
\pgfsetlinewidth{0.000000pt}%
\definecolor{currentstroke}{rgb}{0.000000,0.000000,0.000000}%
\pgfsetstrokecolor{currentstroke}%
\pgfsetstrokeopacity{0.000000}%
\pgfsetdash{}{0pt}%
\pgfpathmoveto{\pgfqpoint{0.800000in}{5.105882in}}%
\pgfpathlineto{\pgfqpoint{3.207767in}{5.105882in}}%
\pgfpathlineto{\pgfqpoint{3.207767in}{6.650000in}}%
\pgfpathlineto{\pgfqpoint{0.800000in}{6.650000in}}%
\pgfpathlineto{\pgfqpoint{0.800000in}{5.105882in}}%
\pgfpathclose%
\pgfusepath{fill}%
\end{pgfscope}%
\begin{pgfscope}%
\pgfpathrectangle{\pgfqpoint{0.800000in}{5.105882in}}{\pgfqpoint{2.407767in}{1.544118in}}%
\pgfusepath{clip}%
\pgfsys@transformcm{2.416667}{0.000000}{0.000000}{1.555556}{0.800000in}{5.105882in}%
\pgftext[left,bottom]{\includegraphics[interpolate=false,width=1.000000in,height=1.000000in]{q_series_square-img0.png}}%
\end{pgfscope}%
\begin{pgfscope}%
\pgfsetbuttcap%
\pgfsetroundjoin%
\definecolor{currentfill}{rgb}{0.000000,0.000000,0.000000}%
\pgfsetfillcolor{currentfill}%
\pgfsetlinewidth{0.803000pt}%
\definecolor{currentstroke}{rgb}{0.000000,0.000000,0.000000}%
\pgfsetstrokecolor{currentstroke}%
\pgfsetdash{}{0pt}%
\pgfsys@defobject{currentmarker}{\pgfqpoint{0.000000in}{-0.048611in}}{\pgfqpoint{0.000000in}{0.000000in}}{%
\pgfpathmoveto{\pgfqpoint{0.000000in}{0.000000in}}%
\pgfpathlineto{\pgfqpoint{0.000000in}{-0.048611in}}%
\pgfusepath{stroke,fill}%
}%
\begin{pgfscope}%
\pgfsys@transformshift{1.233659in}{5.105882in}%
\pgfsys@useobject{currentmarker}{}%
\end{pgfscope}%
\end{pgfscope}%
\begin{pgfscope}%
\pgfsetbuttcap%
\pgfsetroundjoin%
\definecolor{currentfill}{rgb}{0.000000,0.000000,0.000000}%
\pgfsetfillcolor{currentfill}%
\pgfsetlinewidth{0.803000pt}%
\definecolor{currentstroke}{rgb}{0.000000,0.000000,0.000000}%
\pgfsetstrokecolor{currentstroke}%
\pgfsetdash{}{0pt}%
\pgfsys@defobject{currentmarker}{\pgfqpoint{0.000000in}{-0.048611in}}{\pgfqpoint{0.000000in}{0.000000in}}{%
\pgfpathmoveto{\pgfqpoint{0.000000in}{0.000000in}}%
\pgfpathlineto{\pgfqpoint{0.000000in}{-0.048611in}}%
\pgfusepath{stroke,fill}%
}%
\begin{pgfscope}%
\pgfsys@transformshift{1.739160in}{5.105882in}%
\pgfsys@useobject{currentmarker}{}%
\end{pgfscope}%
\end{pgfscope}%
\begin{pgfscope}%
\pgfsetbuttcap%
\pgfsetroundjoin%
\definecolor{currentfill}{rgb}{0.000000,0.000000,0.000000}%
\pgfsetfillcolor{currentfill}%
\pgfsetlinewidth{0.803000pt}%
\definecolor{currentstroke}{rgb}{0.000000,0.000000,0.000000}%
\pgfsetstrokecolor{currentstroke}%
\pgfsetdash{}{0pt}%
\pgfsys@defobject{currentmarker}{\pgfqpoint{0.000000in}{-0.048611in}}{\pgfqpoint{0.000000in}{0.000000in}}{%
\pgfpathmoveto{\pgfqpoint{0.000000in}{0.000000in}}%
\pgfpathlineto{\pgfqpoint{0.000000in}{-0.048611in}}%
\pgfusepath{stroke,fill}%
}%
\begin{pgfscope}%
\pgfsys@transformshift{2.244660in}{5.105882in}%
\pgfsys@useobject{currentmarker}{}%
\end{pgfscope}%
\end{pgfscope}%
\begin{pgfscope}%
\pgfsetbuttcap%
\pgfsetroundjoin%
\definecolor{currentfill}{rgb}{0.000000,0.000000,0.000000}%
\pgfsetfillcolor{currentfill}%
\pgfsetlinewidth{0.803000pt}%
\definecolor{currentstroke}{rgb}{0.000000,0.000000,0.000000}%
\pgfsetstrokecolor{currentstroke}%
\pgfsetdash{}{0pt}%
\pgfsys@defobject{currentmarker}{\pgfqpoint{0.000000in}{-0.048611in}}{\pgfqpoint{0.000000in}{0.000000in}}{%
\pgfpathmoveto{\pgfqpoint{0.000000in}{0.000000in}}%
\pgfpathlineto{\pgfqpoint{0.000000in}{-0.048611in}}%
\pgfusepath{stroke,fill}%
}%
\begin{pgfscope}%
\pgfsys@transformshift{2.750161in}{5.105882in}%
\pgfsys@useobject{currentmarker}{}%
\end{pgfscope}%
\end{pgfscope}%
\begin{pgfscope}%
\pgfsetbuttcap%
\pgfsetroundjoin%
\definecolor{currentfill}{rgb}{0.000000,0.000000,0.000000}%
\pgfsetfillcolor{currentfill}%
\pgfsetlinewidth{0.803000pt}%
\definecolor{currentstroke}{rgb}{0.000000,0.000000,0.000000}%
\pgfsetstrokecolor{currentstroke}%
\pgfsetdash{}{0pt}%
\pgfsys@defobject{currentmarker}{\pgfqpoint{-0.048611in}{0.000000in}}{\pgfqpoint{-0.000000in}{0.000000in}}{%
\pgfpathmoveto{\pgfqpoint{-0.000000in}{0.000000in}}%
\pgfpathlineto{\pgfqpoint{-0.048611in}{0.000000in}}%
\pgfusepath{stroke,fill}%
}%
\begin{pgfscope}%
\pgfsys@transformshift{0.800000in}{5.367356in}%
\pgfsys@useobject{currentmarker}{}%
\end{pgfscope}%
\end{pgfscope}%
\begin{pgfscope}%
\definecolor{textcolor}{rgb}{0.000000,0.000000,0.000000}%
\pgfsetstrokecolor{textcolor}%
\pgfsetfillcolor{textcolor}%
\pgftext[x=0.455863in, y=5.319131in, left, base]{\color{textcolor}\sffamily\fontsize{10.000000}{12.000000}\selectfont \(\displaystyle {\ensuremath{-}20}\)}%
\end{pgfscope}%
\begin{pgfscope}%
\pgfsetbuttcap%
\pgfsetroundjoin%
\definecolor{currentfill}{rgb}{0.000000,0.000000,0.000000}%
\pgfsetfillcolor{currentfill}%
\pgfsetlinewidth{0.803000pt}%
\definecolor{currentstroke}{rgb}{0.000000,0.000000,0.000000}%
\pgfsetstrokecolor{currentstroke}%
\pgfsetdash{}{0pt}%
\pgfsys@defobject{currentmarker}{\pgfqpoint{-0.048611in}{0.000000in}}{\pgfqpoint{-0.000000in}{0.000000in}}{%
\pgfpathmoveto{\pgfqpoint{-0.000000in}{0.000000in}}%
\pgfpathlineto{\pgfqpoint{-0.048611in}{0.000000in}}%
\pgfusepath{stroke,fill}%
}%
\begin{pgfscope}%
\pgfsys@transformshift{0.800000in}{5.877941in}%
\pgfsys@useobject{currentmarker}{}%
\end{pgfscope}%
\end{pgfscope}%
\begin{pgfscope}%
\definecolor{textcolor}{rgb}{0.000000,0.000000,0.000000}%
\pgfsetstrokecolor{textcolor}%
\pgfsetfillcolor{textcolor}%
\pgftext[x=0.633333in, y=5.829716in, left, base]{\color{textcolor}\sffamily\fontsize{10.000000}{12.000000}\selectfont \(\displaystyle {0}\)}%
\end{pgfscope}%
\begin{pgfscope}%
\pgfsetbuttcap%
\pgfsetroundjoin%
\definecolor{currentfill}{rgb}{0.000000,0.000000,0.000000}%
\pgfsetfillcolor{currentfill}%
\pgfsetlinewidth{0.803000pt}%
\definecolor{currentstroke}{rgb}{0.000000,0.000000,0.000000}%
\pgfsetstrokecolor{currentstroke}%
\pgfsetdash{}{0pt}%
\pgfsys@defobject{currentmarker}{\pgfqpoint{-0.048611in}{0.000000in}}{\pgfqpoint{-0.000000in}{0.000000in}}{%
\pgfpathmoveto{\pgfqpoint{-0.000000in}{0.000000in}}%
\pgfpathlineto{\pgfqpoint{-0.048611in}{0.000000in}}%
\pgfusepath{stroke,fill}%
}%
\begin{pgfscope}%
\pgfsys@transformshift{0.800000in}{6.388526in}%
\pgfsys@useobject{currentmarker}{}%
\end{pgfscope}%
\end{pgfscope}%
\begin{pgfscope}%
\definecolor{textcolor}{rgb}{0.000000,0.000000,0.000000}%
\pgfsetstrokecolor{textcolor}%
\pgfsetfillcolor{textcolor}%
\pgftext[x=0.563888in, y=6.340301in, left, base]{\color{textcolor}\sffamily\fontsize{10.000000}{12.000000}\selectfont \(\displaystyle {20}\)}%
\end{pgfscope}%
\begin{pgfscope}%
\definecolor{textcolor}{rgb}{0.000000,0.000000,0.000000}%
\pgfsetstrokecolor{textcolor}%
\pgfsetfillcolor{textcolor}%
\pgftext[x=0.400308in,y=5.877941in,,bottom,rotate=90.000000]{\color{textcolor}\sffamily\fontsize{10.000000}{12.000000}\selectfont \(\displaystyle z \, \mathrm{[\mu m]}\)}%
\end{pgfscope}%
\begin{pgfscope}%
\pgfpathrectangle{\pgfqpoint{0.800000in}{5.105882in}}{\pgfqpoint{2.407767in}{1.544118in}}%
\pgfusepath{clip}%
\pgfsetbuttcap%
\pgfsetroundjoin%
\pgfsetlinewidth{0.501875pt}%
\definecolor{currentstroke}{rgb}{0.277941,0.056324,0.381191}%
\pgfsetstrokecolor{currentstroke}%
\pgfsetdash{}{0pt}%
\pgfpathmoveto{\pgfqpoint{1.949703in}{6.364387in}}%
\pgfpathlineto{\pgfqpoint{1.949703in}{6.364387in}}%
\pgfusepath{stroke}%
\end{pgfscope}%
\begin{pgfscope}%
\pgfpathrectangle{\pgfqpoint{0.800000in}{5.105882in}}{\pgfqpoint{2.407767in}{1.544118in}}%
\pgfusepath{clip}%
\pgfsetbuttcap%
\pgfsetroundjoin%
\pgfsetlinewidth{0.501875pt}%
\definecolor{currentstroke}{rgb}{0.277941,0.056324,0.381191}%
\pgfsetstrokecolor{currentstroke}%
\pgfsetdash{}{0pt}%
\pgfpathmoveto{\pgfqpoint{1.949703in}{6.364387in}}%
\pgfpathlineto{\pgfqpoint{1.942491in}{6.339480in}}%
\pgfusepath{stroke}%
\end{pgfscope}%
\begin{pgfscope}%
\pgfpathrectangle{\pgfqpoint{0.800000in}{5.105882in}}{\pgfqpoint{2.407767in}{1.544118in}}%
\pgfusepath{clip}%
\pgfsetbuttcap%
\pgfsetroundjoin%
\pgfsetlinewidth{0.501875pt}%
\definecolor{currentstroke}{rgb}{0.280267,0.073417,0.397163}%
\pgfsetstrokecolor{currentstroke}%
\pgfsetdash{}{0pt}%
\pgfpathmoveto{\pgfqpoint{1.942491in}{6.339480in}}%
\pgfpathlineto{\pgfqpoint{1.942491in}{6.339480in}}%
\pgfusepath{stroke}%
\end{pgfscope}%
\begin{pgfscope}%
\pgfpathrectangle{\pgfqpoint{0.800000in}{5.105882in}}{\pgfqpoint{2.407767in}{1.544118in}}%
\pgfusepath{clip}%
\pgfsetbuttcap%
\pgfsetroundjoin%
\pgfsetlinewidth{0.501875pt}%
\definecolor{currentstroke}{rgb}{0.280267,0.073417,0.397163}%
\pgfsetstrokecolor{currentstroke}%
\pgfsetdash{}{0pt}%
\pgfpathmoveto{\pgfqpoint{1.942491in}{6.339480in}}%
\pgfpathlineto{\pgfqpoint{1.930154in}{6.321695in}}%
\pgfusepath{stroke}%
\end{pgfscope}%
\begin{pgfscope}%
\pgfpathrectangle{\pgfqpoint{0.800000in}{5.105882in}}{\pgfqpoint{2.407767in}{1.544118in}}%
\pgfusepath{clip}%
\pgfsetbuttcap%
\pgfsetroundjoin%
\pgfsetlinewidth{0.501875pt}%
\definecolor{currentstroke}{rgb}{0.280267,0.073417,0.397163}%
\pgfsetstrokecolor{currentstroke}%
\pgfsetdash{}{0pt}%
\pgfpathmoveto{\pgfqpoint{1.930154in}{6.321695in}}%
\pgfpathlineto{\pgfqpoint{1.912423in}{6.305573in}}%
\pgfusepath{stroke}%
\end{pgfscope}%
\begin{pgfscope}%
\pgfpathrectangle{\pgfqpoint{0.800000in}{5.105882in}}{\pgfqpoint{2.407767in}{1.544118in}}%
\pgfusepath{clip}%
\pgfsetbuttcap%
\pgfsetroundjoin%
\pgfsetlinewidth{0.501875pt}%
\definecolor{currentstroke}{rgb}{0.283187,0.125848,0.444960}%
\pgfsetstrokecolor{currentstroke}%
\pgfsetdash{}{0pt}%
\pgfpathmoveto{\pgfqpoint{1.912423in}{6.305573in}}%
\pgfpathlineto{\pgfqpoint{1.892860in}{6.286909in}}%
\pgfusepath{stroke}%
\end{pgfscope}%
\begin{pgfscope}%
\pgfpathrectangle{\pgfqpoint{0.800000in}{5.105882in}}{\pgfqpoint{2.407767in}{1.544118in}}%
\pgfusepath{clip}%
\pgfsetbuttcap%
\pgfsetroundjoin%
\pgfsetlinewidth{0.501875pt}%
\definecolor{currentstroke}{rgb}{0.283197,0.115680,0.436115}%
\pgfsetstrokecolor{currentstroke}%
\pgfsetdash{}{0pt}%
\pgfpathmoveto{\pgfqpoint{1.892860in}{6.286909in}}%
\pgfpathlineto{\pgfqpoint{1.882429in}{6.267387in}}%
\pgfusepath{stroke}%
\end{pgfscope}%
\begin{pgfscope}%
\pgfpathrectangle{\pgfqpoint{0.800000in}{5.105882in}}{\pgfqpoint{2.407767in}{1.544118in}}%
\pgfusepath{clip}%
\pgfsetbuttcap%
\pgfsetroundjoin%
\pgfsetlinewidth{0.501875pt}%
\definecolor{currentstroke}{rgb}{0.283197,0.115680,0.436115}%
\pgfsetstrokecolor{currentstroke}%
\pgfsetdash{}{0pt}%
\pgfpathmoveto{\pgfqpoint{1.882429in}{6.267387in}}%
\pgfpathlineto{\pgfqpoint{1.868034in}{6.240697in}}%
\pgfusepath{stroke}%
\end{pgfscope}%
\begin{pgfscope}%
\pgfpathrectangle{\pgfqpoint{0.800000in}{5.105882in}}{\pgfqpoint{2.407767in}{1.544118in}}%
\pgfusepath{clip}%
\pgfsetbuttcap%
\pgfsetroundjoin%
\pgfsetlinewidth{0.501875pt}%
\definecolor{currentstroke}{rgb}{0.282884,0.135920,0.453427}%
\pgfsetstrokecolor{currentstroke}%
\pgfsetdash{}{0pt}%
\pgfpathmoveto{\pgfqpoint{1.868034in}{6.240697in}}%
\pgfpathlineto{\pgfqpoint{1.847343in}{6.209654in}}%
\pgfusepath{stroke}%
\end{pgfscope}%
\begin{pgfscope}%
\pgfpathrectangle{\pgfqpoint{0.800000in}{5.105882in}}{\pgfqpoint{2.407767in}{1.544118in}}%
\pgfusepath{clip}%
\pgfsetbuttcap%
\pgfsetroundjoin%
\pgfsetlinewidth{0.501875pt}%
\definecolor{currentstroke}{rgb}{0.281887,0.150881,0.465405}%
\pgfsetstrokecolor{currentstroke}%
\pgfsetdash{}{0pt}%
\pgfpathmoveto{\pgfqpoint{1.847343in}{6.209654in}}%
\pgfpathlineto{\pgfqpoint{1.826078in}{6.178643in}}%
\pgfusepath{stroke}%
\end{pgfscope}%
\begin{pgfscope}%
\pgfpathrectangle{\pgfqpoint{0.800000in}{5.105882in}}{\pgfqpoint{2.407767in}{1.544118in}}%
\pgfusepath{clip}%
\pgfsetbuttcap%
\pgfsetroundjoin%
\pgfsetlinewidth{0.501875pt}%
\definecolor{currentstroke}{rgb}{0.279566,0.067836,0.391917}%
\pgfsetstrokecolor{currentstroke}%
\pgfsetdash{}{0pt}%
\pgfpathmoveto{\pgfqpoint{1.895523in}{5.426242in}}%
\pgfpathlineto{\pgfqpoint{1.895523in}{5.426242in}}%
\pgfusepath{stroke}%
\end{pgfscope}%
\begin{pgfscope}%
\pgfpathrectangle{\pgfqpoint{0.800000in}{5.105882in}}{\pgfqpoint{2.407767in}{1.544118in}}%
\pgfusepath{clip}%
\pgfsetbuttcap%
\pgfsetroundjoin%
\pgfsetlinewidth{0.501875pt}%
\definecolor{currentstroke}{rgb}{0.279566,0.067836,0.391917}%
\pgfsetstrokecolor{currentstroke}%
\pgfsetdash{}{0pt}%
\pgfpathmoveto{\pgfqpoint{1.895523in}{5.426242in}}%
\pgfpathlineto{\pgfqpoint{1.895523in}{5.426242in}}%
\pgfusepath{stroke}%
\end{pgfscope}%
\begin{pgfscope}%
\pgfpathrectangle{\pgfqpoint{0.800000in}{5.105882in}}{\pgfqpoint{2.407767in}{1.544118in}}%
\pgfusepath{clip}%
\pgfsetbuttcap%
\pgfsetroundjoin%
\pgfsetlinewidth{0.501875pt}%
\definecolor{currentstroke}{rgb}{0.279566,0.067836,0.391917}%
\pgfsetstrokecolor{currentstroke}%
\pgfsetdash{}{0pt}%
\pgfpathmoveto{\pgfqpoint{1.895523in}{5.426242in}}%
\pgfpathlineto{\pgfqpoint{1.883719in}{5.446500in}}%
\pgfusepath{stroke}%
\end{pgfscope}%
\begin{pgfscope}%
\pgfpathrectangle{\pgfqpoint{0.800000in}{5.105882in}}{\pgfqpoint{2.407767in}{1.544118in}}%
\pgfusepath{clip}%
\pgfsetbuttcap%
\pgfsetroundjoin%
\pgfsetlinewidth{0.501875pt}%
\definecolor{currentstroke}{rgb}{0.280267,0.073417,0.397163}%
\pgfsetstrokecolor{currentstroke}%
\pgfsetdash{}{0pt}%
\pgfpathmoveto{\pgfqpoint{1.883719in}{5.446500in}}%
\pgfpathlineto{\pgfqpoint{1.871669in}{5.468564in}}%
\pgfusepath{stroke}%
\end{pgfscope}%
\begin{pgfscope}%
\pgfpathrectangle{\pgfqpoint{0.800000in}{5.105882in}}{\pgfqpoint{2.407767in}{1.544118in}}%
\pgfusepath{clip}%
\pgfsetbuttcap%
\pgfsetroundjoin%
\pgfsetlinewidth{0.501875pt}%
\definecolor{currentstroke}{rgb}{0.283091,0.110553,0.431554}%
\pgfsetstrokecolor{currentstroke}%
\pgfsetdash{}{0pt}%
\pgfpathmoveto{\pgfqpoint{1.871669in}{5.468564in}}%
\pgfpathlineto{\pgfqpoint{1.855154in}{5.500223in}}%
\pgfusepath{stroke}%
\end{pgfscope}%
\begin{pgfscope}%
\pgfpathrectangle{\pgfqpoint{0.800000in}{5.105882in}}{\pgfqpoint{2.407767in}{1.544118in}}%
\pgfusepath{clip}%
\pgfsetbuttcap%
\pgfsetroundjoin%
\pgfsetlinewidth{0.501875pt}%
\definecolor{currentstroke}{rgb}{0.283229,0.120777,0.440584}%
\pgfsetstrokecolor{currentstroke}%
\pgfsetdash{}{0pt}%
\pgfpathmoveto{\pgfqpoint{1.855154in}{5.500223in}}%
\pgfpathlineto{\pgfqpoint{1.840102in}{5.532682in}}%
\pgfusepath{stroke}%
\end{pgfscope}%
\begin{pgfscope}%
\pgfpathrectangle{\pgfqpoint{0.800000in}{5.105882in}}{\pgfqpoint{2.407767in}{1.544118in}}%
\pgfusepath{clip}%
\pgfsetbuttcap%
\pgfsetroundjoin%
\pgfsetlinewidth{0.501875pt}%
\definecolor{currentstroke}{rgb}{0.282884,0.135920,0.453427}%
\pgfsetstrokecolor{currentstroke}%
\pgfsetdash{}{0pt}%
\pgfpathmoveto{\pgfqpoint{1.840102in}{5.532682in}}%
\pgfpathlineto{\pgfqpoint{1.820952in}{5.564134in}}%
\pgfusepath{stroke}%
\end{pgfscope}%
\begin{pgfscope}%
\pgfpathrectangle{\pgfqpoint{0.800000in}{5.105882in}}{\pgfqpoint{2.407767in}{1.544118in}}%
\pgfusepath{clip}%
\pgfsetbuttcap%
\pgfsetroundjoin%
\pgfsetlinewidth{0.501875pt}%
\definecolor{currentstroke}{rgb}{0.278012,0.180367,0.486697}%
\pgfsetstrokecolor{currentstroke}%
\pgfsetdash{}{0pt}%
\pgfpathmoveto{\pgfqpoint{1.820952in}{5.564134in}}%
\pgfpathlineto{\pgfqpoint{1.803634in}{5.594267in}}%
\pgfusepath{stroke}%
\end{pgfscope}%
\begin{pgfscope}%
\pgfpathrectangle{\pgfqpoint{0.800000in}{5.105882in}}{\pgfqpoint{2.407767in}{1.544118in}}%
\pgfusepath{clip}%
\pgfsetbuttcap%
\pgfsetroundjoin%
\pgfsetlinewidth{0.501875pt}%
\definecolor{currentstroke}{rgb}{0.278791,0.062145,0.386592}%
\pgfsetstrokecolor{currentstroke}%
\pgfsetdash{}{0pt}%
\pgfpathmoveto{\pgfqpoint{2.058064in}{5.426242in}}%
\pgfpathlineto{\pgfqpoint{2.058064in}{5.426242in}}%
\pgfusepath{stroke}%
\end{pgfscope}%
\begin{pgfscope}%
\pgfpathrectangle{\pgfqpoint{0.800000in}{5.105882in}}{\pgfqpoint{2.407767in}{1.544118in}}%
\pgfusepath{clip}%
\pgfsetbuttcap%
\pgfsetroundjoin%
\pgfsetlinewidth{0.501875pt}%
\definecolor{currentstroke}{rgb}{0.278791,0.062145,0.386592}%
\pgfsetstrokecolor{currentstroke}%
\pgfsetdash{}{0pt}%
\pgfpathmoveto{\pgfqpoint{2.058064in}{5.426242in}}%
\pgfpathlineto{\pgfqpoint{2.030353in}{5.440682in}}%
\pgfusepath{stroke}%
\end{pgfscope}%
\begin{pgfscope}%
\pgfpathrectangle{\pgfqpoint{0.800000in}{5.105882in}}{\pgfqpoint{2.407767in}{1.544118in}}%
\pgfusepath{clip}%
\pgfsetbuttcap%
\pgfsetroundjoin%
\pgfsetlinewidth{0.501875pt}%
\definecolor{currentstroke}{rgb}{0.281446,0.084320,0.407414}%
\pgfsetstrokecolor{currentstroke}%
\pgfsetdash{}{0pt}%
\pgfpathmoveto{\pgfqpoint{2.030353in}{5.440682in}}%
\pgfpathlineto{\pgfqpoint{2.008128in}{5.455956in}}%
\pgfusepath{stroke}%
\end{pgfscope}%
\begin{pgfscope}%
\pgfpathrectangle{\pgfqpoint{0.800000in}{5.105882in}}{\pgfqpoint{2.407767in}{1.544118in}}%
\pgfusepath{clip}%
\pgfsetbuttcap%
\pgfsetroundjoin%
\pgfsetlinewidth{0.501875pt}%
\definecolor{currentstroke}{rgb}{0.281446,0.084320,0.407414}%
\pgfsetstrokecolor{currentstroke}%
\pgfsetdash{}{0pt}%
\pgfpathmoveto{\pgfqpoint{2.008128in}{5.455956in}}%
\pgfpathlineto{\pgfqpoint{1.973166in}{5.481106in}}%
\pgfusepath{stroke}%
\end{pgfscope}%
\begin{pgfscope}%
\pgfpathrectangle{\pgfqpoint{0.800000in}{5.105882in}}{\pgfqpoint{2.407767in}{1.544118in}}%
\pgfusepath{clip}%
\pgfsetbuttcap%
\pgfsetroundjoin%
\pgfsetlinewidth{0.501875pt}%
\definecolor{currentstroke}{rgb}{0.283197,0.115680,0.436115}%
\pgfsetstrokecolor{currentstroke}%
\pgfsetdash{}{0pt}%
\pgfpathmoveto{\pgfqpoint{1.973166in}{5.481106in}}%
\pgfpathlineto{\pgfqpoint{1.942860in}{5.508436in}}%
\pgfusepath{stroke}%
\end{pgfscope}%
\begin{pgfscope}%
\pgfpathrectangle{\pgfqpoint{0.800000in}{5.105882in}}{\pgfqpoint{2.407767in}{1.544118in}}%
\pgfusepath{clip}%
\pgfsetbuttcap%
\pgfsetroundjoin%
\pgfsetlinewidth{0.501875pt}%
\definecolor{currentstroke}{rgb}{0.283187,0.125848,0.444960}%
\pgfsetstrokecolor{currentstroke}%
\pgfsetdash{}{0pt}%
\pgfpathmoveto{\pgfqpoint{1.942860in}{5.508436in}}%
\pgfpathlineto{\pgfqpoint{1.913743in}{5.535417in}}%
\pgfusepath{stroke}%
\end{pgfscope}%
\begin{pgfscope}%
\pgfpathrectangle{\pgfqpoint{0.800000in}{5.105882in}}{\pgfqpoint{2.407767in}{1.544118in}}%
\pgfusepath{clip}%
\pgfsetbuttcap%
\pgfsetroundjoin%
\pgfsetlinewidth{0.501875pt}%
\definecolor{currentstroke}{rgb}{0.281412,0.155834,0.469201}%
\pgfsetstrokecolor{currentstroke}%
\pgfsetdash{}{0pt}%
\pgfpathmoveto{\pgfqpoint{1.913743in}{5.535417in}}%
\pgfpathlineto{\pgfqpoint{1.883839in}{5.563167in}}%
\pgfusepath{stroke}%
\end{pgfscope}%
\begin{pgfscope}%
\pgfpathrectangle{\pgfqpoint{0.800000in}{5.105882in}}{\pgfqpoint{2.407767in}{1.544118in}}%
\pgfusepath{clip}%
\pgfsetbuttcap%
\pgfsetroundjoin%
\pgfsetlinewidth{0.501875pt}%
\definecolor{currentstroke}{rgb}{0.281412,0.155834,0.469201}%
\pgfsetstrokecolor{currentstroke}%
\pgfsetdash{}{0pt}%
\pgfpathmoveto{\pgfqpoint{1.883839in}{5.563167in}}%
\pgfpathlineto{\pgfqpoint{1.855688in}{5.591920in}}%
\pgfusepath{stroke}%
\end{pgfscope}%
\begin{pgfscope}%
\pgfpathrectangle{\pgfqpoint{0.800000in}{5.105882in}}{\pgfqpoint{2.407767in}{1.544118in}}%
\pgfusepath{clip}%
\pgfsetbuttcap%
\pgfsetroundjoin%
\pgfsetlinewidth{0.501875pt}%
\definecolor{currentstroke}{rgb}{0.276194,0.190074,0.493001}%
\pgfsetstrokecolor{currentstroke}%
\pgfsetdash{}{0pt}%
\pgfpathmoveto{\pgfqpoint{1.855688in}{5.591920in}}%
\pgfpathlineto{\pgfqpoint{1.825864in}{5.619915in}}%
\pgfusepath{stroke}%
\end{pgfscope}%
\begin{pgfscope}%
\pgfpathrectangle{\pgfqpoint{0.800000in}{5.105882in}}{\pgfqpoint{2.407767in}{1.544118in}}%
\pgfusepath{clip}%
\pgfsetbuttcap%
\pgfsetroundjoin%
\pgfsetlinewidth{0.501875pt}%
\definecolor{currentstroke}{rgb}{0.271828,0.209303,0.504434}%
\pgfsetstrokecolor{currentstroke}%
\pgfsetdash{}{0pt}%
\pgfpathmoveto{\pgfqpoint{1.825864in}{5.619915in}}%
\pgfpathlineto{\pgfqpoint{1.796022in}{5.647924in}}%
\pgfusepath{stroke}%
\end{pgfscope}%
\begin{pgfscope}%
\pgfpathrectangle{\pgfqpoint{0.800000in}{5.105882in}}{\pgfqpoint{2.407767in}{1.544118in}}%
\pgfusepath{clip}%
\pgfsetbuttcap%
\pgfsetroundjoin%
\pgfsetlinewidth{0.501875pt}%
\definecolor{currentstroke}{rgb}{0.274952,0.037752,0.364543}%
\pgfsetstrokecolor{currentstroke}%
\pgfsetdash{}{0pt}%
\pgfpathmoveto{\pgfqpoint{2.545685in}{6.329641in}}%
\pgfpathlineto{\pgfqpoint{2.493039in}{6.325977in}}%
\pgfusepath{stroke}%
\end{pgfscope}%
\begin{pgfscope}%
\pgfpathrectangle{\pgfqpoint{0.800000in}{5.105882in}}{\pgfqpoint{2.407767in}{1.544118in}}%
\pgfusepath{clip}%
\pgfsetbuttcap%
\pgfsetroundjoin%
\pgfsetlinewidth{0.501875pt}%
\definecolor{currentstroke}{rgb}{0.274952,0.037752,0.364543}%
\pgfsetstrokecolor{currentstroke}%
\pgfsetdash{}{0pt}%
\pgfpathmoveto{\pgfqpoint{2.493039in}{6.325977in}}%
\pgfpathlineto{\pgfqpoint{2.440341in}{6.322618in}}%
\pgfusepath{stroke}%
\end{pgfscope}%
\begin{pgfscope}%
\pgfpathrectangle{\pgfqpoint{0.800000in}{5.105882in}}{\pgfqpoint{2.407767in}{1.544118in}}%
\pgfusepath{clip}%
\pgfsetbuttcap%
\pgfsetroundjoin%
\pgfsetlinewidth{0.501875pt}%
\definecolor{currentstroke}{rgb}{0.277941,0.056324,0.381191}%
\pgfsetstrokecolor{currentstroke}%
\pgfsetdash{}{0pt}%
\pgfpathmoveto{\pgfqpoint{2.440341in}{6.322618in}}%
\pgfpathlineto{\pgfqpoint{2.387530in}{6.320046in}}%
\pgfusepath{stroke}%
\end{pgfscope}%
\begin{pgfscope}%
\pgfpathrectangle{\pgfqpoint{0.800000in}{5.105882in}}{\pgfqpoint{2.407767in}{1.544118in}}%
\pgfusepath{clip}%
\pgfsetbuttcap%
\pgfsetroundjoin%
\pgfsetlinewidth{0.501875pt}%
\definecolor{currentstroke}{rgb}{0.277941,0.056324,0.381191}%
\pgfsetstrokecolor{currentstroke}%
\pgfsetdash{}{0pt}%
\pgfpathmoveto{\pgfqpoint{2.387530in}{6.320046in}}%
\pgfpathlineto{\pgfqpoint{2.334712in}{6.317511in}}%
\pgfusepath{stroke}%
\end{pgfscope}%
\begin{pgfscope}%
\pgfpathrectangle{\pgfqpoint{0.800000in}{5.105882in}}{\pgfqpoint{2.407767in}{1.544118in}}%
\pgfusepath{clip}%
\pgfsetbuttcap%
\pgfsetroundjoin%
\pgfsetlinewidth{0.501875pt}%
\definecolor{currentstroke}{rgb}{0.277941,0.056324,0.381191}%
\pgfsetstrokecolor{currentstroke}%
\pgfsetdash{}{0pt}%
\pgfpathmoveto{\pgfqpoint{2.334712in}{6.317511in}}%
\pgfpathlineto{\pgfqpoint{2.281931in}{6.314640in}}%
\pgfusepath{stroke}%
\end{pgfscope}%
\begin{pgfscope}%
\pgfpathrectangle{\pgfqpoint{0.800000in}{5.105882in}}{\pgfqpoint{2.407767in}{1.544118in}}%
\pgfusepath{clip}%
\pgfsetbuttcap%
\pgfsetroundjoin%
\pgfsetlinewidth{0.501875pt}%
\definecolor{currentstroke}{rgb}{0.278791,0.062145,0.386592}%
\pgfsetstrokecolor{currentstroke}%
\pgfsetdash{}{0pt}%
\pgfpathmoveto{\pgfqpoint{2.281931in}{6.314640in}}%
\pgfpathlineto{\pgfqpoint{2.229347in}{6.310753in}}%
\pgfusepath{stroke}%
\end{pgfscope}%
\begin{pgfscope}%
\pgfpathrectangle{\pgfqpoint{0.800000in}{5.105882in}}{\pgfqpoint{2.407767in}{1.544118in}}%
\pgfusepath{clip}%
\pgfsetbuttcap%
\pgfsetroundjoin%
\pgfsetlinewidth{0.501875pt}%
\definecolor{currentstroke}{rgb}{0.280267,0.073417,0.397163}%
\pgfsetstrokecolor{currentstroke}%
\pgfsetdash{}{0pt}%
\pgfpathmoveto{\pgfqpoint{2.229347in}{6.310753in}}%
\pgfpathlineto{\pgfqpoint{2.177641in}{6.303680in}}%
\pgfusepath{stroke}%
\end{pgfscope}%
\begin{pgfscope}%
\pgfpathrectangle{\pgfqpoint{0.800000in}{5.105882in}}{\pgfqpoint{2.407767in}{1.544118in}}%
\pgfusepath{clip}%
\pgfsetbuttcap%
\pgfsetroundjoin%
\pgfsetlinewidth{0.501875pt}%
\definecolor{currentstroke}{rgb}{0.279566,0.067836,0.391917}%
\pgfsetstrokecolor{currentstroke}%
\pgfsetdash{}{0pt}%
\pgfpathmoveto{\pgfqpoint{2.177641in}{6.303680in}}%
\pgfpathlineto{\pgfqpoint{2.127046in}{6.293746in}}%
\pgfusepath{stroke}%
\end{pgfscope}%
\begin{pgfscope}%
\pgfpathrectangle{\pgfqpoint{0.800000in}{5.105882in}}{\pgfqpoint{2.407767in}{1.544118in}}%
\pgfusepath{clip}%
\pgfsetbuttcap%
\pgfsetroundjoin%
\pgfsetlinewidth{0.501875pt}%
\definecolor{currentstroke}{rgb}{0.278791,0.062145,0.386592}%
\pgfsetstrokecolor{currentstroke}%
\pgfsetdash{}{0pt}%
\pgfpathmoveto{\pgfqpoint{2.127046in}{6.293746in}}%
\pgfpathlineto{\pgfqpoint{2.078664in}{6.280242in}}%
\pgfusepath{stroke}%
\end{pgfscope}%
\begin{pgfscope}%
\pgfpathrectangle{\pgfqpoint{0.800000in}{5.105882in}}{\pgfqpoint{2.407767in}{1.544118in}}%
\pgfusepath{clip}%
\pgfsetbuttcap%
\pgfsetroundjoin%
\pgfsetlinewidth{0.501875pt}%
\definecolor{currentstroke}{rgb}{0.280267,0.073417,0.397163}%
\pgfsetstrokecolor{currentstroke}%
\pgfsetdash{}{0pt}%
\pgfpathmoveto{\pgfqpoint{2.078664in}{6.280242in}}%
\pgfpathlineto{\pgfqpoint{2.032369in}{6.263794in}}%
\pgfusepath{stroke}%
\end{pgfscope}%
\begin{pgfscope}%
\pgfpathrectangle{\pgfqpoint{0.800000in}{5.105882in}}{\pgfqpoint{2.407767in}{1.544118in}}%
\pgfusepath{clip}%
\pgfsetbuttcap%
\pgfsetroundjoin%
\pgfsetlinewidth{0.501875pt}%
\definecolor{currentstroke}{rgb}{0.282910,0.105393,0.426902}%
\pgfsetstrokecolor{currentstroke}%
\pgfsetdash{}{0pt}%
\pgfpathmoveto{\pgfqpoint{2.032369in}{6.263794in}}%
\pgfpathlineto{\pgfqpoint{1.990299in}{6.243574in}}%
\pgfusepath{stroke}%
\end{pgfscope}%
\begin{pgfscope}%
\pgfpathrectangle{\pgfqpoint{0.800000in}{5.105882in}}{\pgfqpoint{2.407767in}{1.544118in}}%
\pgfusepath{clip}%
\pgfsetbuttcap%
\pgfsetroundjoin%
\pgfsetlinewidth{0.501875pt}%
\definecolor{currentstroke}{rgb}{0.283197,0.115680,0.436115}%
\pgfsetstrokecolor{currentstroke}%
\pgfsetdash{}{0pt}%
\pgfpathmoveto{\pgfqpoint{1.990299in}{6.243574in}}%
\pgfpathlineto{\pgfqpoint{1.954518in}{6.220922in}}%
\pgfusepath{stroke}%
\end{pgfscope}%
\begin{pgfscope}%
\pgfpathrectangle{\pgfqpoint{0.800000in}{5.105882in}}{\pgfqpoint{2.407767in}{1.544118in}}%
\pgfusepath{clip}%
\pgfsetbuttcap%
\pgfsetroundjoin%
\pgfsetlinewidth{0.501875pt}%
\definecolor{currentstroke}{rgb}{0.282884,0.135920,0.453427}%
\pgfsetstrokecolor{currentstroke}%
\pgfsetdash{}{0pt}%
\pgfpathmoveto{\pgfqpoint{1.954518in}{6.220922in}}%
\pgfpathlineto{\pgfqpoint{1.919167in}{6.195664in}}%
\pgfusepath{stroke}%
\end{pgfscope}%
\begin{pgfscope}%
\pgfpathrectangle{\pgfqpoint{0.800000in}{5.105882in}}{\pgfqpoint{2.407767in}{1.544118in}}%
\pgfusepath{clip}%
\pgfsetbuttcap%
\pgfsetroundjoin%
\pgfsetlinewidth{0.501875pt}%
\definecolor{currentstroke}{rgb}{0.278826,0.175490,0.483397}%
\pgfsetstrokecolor{currentstroke}%
\pgfsetdash{}{0pt}%
\pgfpathmoveto{\pgfqpoint{1.919167in}{6.195664in}}%
\pgfpathlineto{\pgfqpoint{1.884114in}{6.170216in}}%
\pgfusepath{stroke}%
\end{pgfscope}%
\begin{pgfscope}%
\pgfpathrectangle{\pgfqpoint{0.800000in}{5.105882in}}{\pgfqpoint{2.407767in}{1.544118in}}%
\pgfusepath{clip}%
\pgfsetbuttcap%
\pgfsetroundjoin%
\pgfsetlinewidth{0.501875pt}%
\definecolor{currentstroke}{rgb}{0.276194,0.190074,0.493001}%
\pgfsetstrokecolor{currentstroke}%
\pgfsetdash{}{0pt}%
\pgfpathmoveto{\pgfqpoint{1.884114in}{6.170216in}}%
\pgfpathlineto{\pgfqpoint{1.851204in}{6.143728in}}%
\pgfusepath{stroke}%
\end{pgfscope}%
\begin{pgfscope}%
\pgfpathrectangle{\pgfqpoint{0.800000in}{5.105882in}}{\pgfqpoint{2.407767in}{1.544118in}}%
\pgfusepath{clip}%
\pgfsetbuttcap%
\pgfsetroundjoin%
\pgfsetlinewidth{0.501875pt}%
\definecolor{currentstroke}{rgb}{0.276194,0.190074,0.493001}%
\pgfsetstrokecolor{currentstroke}%
\pgfsetdash{}{0pt}%
\pgfpathmoveto{\pgfqpoint{1.851204in}{6.143728in}}%
\pgfpathlineto{\pgfqpoint{1.819985in}{6.116356in}}%
\pgfusepath{stroke}%
\end{pgfscope}%
\begin{pgfscope}%
\pgfpathrectangle{\pgfqpoint{0.800000in}{5.105882in}}{\pgfqpoint{2.407767in}{1.544118in}}%
\pgfusepath{clip}%
\pgfsetbuttcap%
\pgfsetroundjoin%
\pgfsetlinewidth{0.501875pt}%
\definecolor{currentstroke}{rgb}{0.279566,0.067836,0.391917}%
\pgfsetstrokecolor{currentstroke}%
\pgfsetdash{}{0pt}%
\pgfpathmoveto{\pgfqpoint{2.161404in}{5.448826in}}%
\pgfpathlineto{\pgfqpoint{2.112244in}{5.460988in}}%
\pgfusepath{stroke}%
\end{pgfscope}%
\begin{pgfscope}%
\pgfpathrectangle{\pgfqpoint{0.800000in}{5.105882in}}{\pgfqpoint{2.407767in}{1.544118in}}%
\pgfusepath{clip}%
\pgfsetbuttcap%
\pgfsetroundjoin%
\pgfsetlinewidth{0.501875pt}%
\definecolor{currentstroke}{rgb}{0.279566,0.067836,0.391917}%
\pgfsetstrokecolor{currentstroke}%
\pgfsetdash{}{0pt}%
\pgfpathmoveto{\pgfqpoint{2.112244in}{5.460988in}}%
\pgfpathlineto{\pgfqpoint{2.065889in}{5.477212in}}%
\pgfusepath{stroke}%
\end{pgfscope}%
\begin{pgfscope}%
\pgfpathrectangle{\pgfqpoint{0.800000in}{5.105882in}}{\pgfqpoint{2.407767in}{1.544118in}}%
\pgfusepath{clip}%
\pgfsetbuttcap%
\pgfsetroundjoin%
\pgfsetlinewidth{0.501875pt}%
\definecolor{currentstroke}{rgb}{0.282327,0.094955,0.417331}%
\pgfsetstrokecolor{currentstroke}%
\pgfsetdash{}{0pt}%
\pgfpathmoveto{\pgfqpoint{2.065889in}{5.477212in}}%
\pgfpathlineto{\pgfqpoint{2.022854in}{5.496594in}}%
\pgfusepath{stroke}%
\end{pgfscope}%
\begin{pgfscope}%
\pgfpathrectangle{\pgfqpoint{0.800000in}{5.105882in}}{\pgfqpoint{2.407767in}{1.544118in}}%
\pgfusepath{clip}%
\pgfsetbuttcap%
\pgfsetroundjoin%
\pgfsetlinewidth{0.501875pt}%
\definecolor{currentstroke}{rgb}{0.282656,0.100196,0.422160}%
\pgfsetstrokecolor{currentstroke}%
\pgfsetdash{}{0pt}%
\pgfpathmoveto{\pgfqpoint{2.022854in}{5.496594in}}%
\pgfpathlineto{\pgfqpoint{1.981257in}{5.517376in}}%
\pgfusepath{stroke}%
\end{pgfscope}%
\begin{pgfscope}%
\pgfpathrectangle{\pgfqpoint{0.800000in}{5.105882in}}{\pgfqpoint{2.407767in}{1.544118in}}%
\pgfusepath{clip}%
\pgfsetbuttcap%
\pgfsetroundjoin%
\pgfsetlinewidth{0.501875pt}%
\definecolor{currentstroke}{rgb}{0.283072,0.130895,0.449241}%
\pgfsetstrokecolor{currentstroke}%
\pgfsetdash{}{0pt}%
\pgfpathmoveto{\pgfqpoint{1.981257in}{5.517376in}}%
\pgfpathlineto{\pgfqpoint{1.944568in}{5.541144in}}%
\pgfusepath{stroke}%
\end{pgfscope}%
\begin{pgfscope}%
\pgfpathrectangle{\pgfqpoint{0.800000in}{5.105882in}}{\pgfqpoint{2.407767in}{1.544118in}}%
\pgfusepath{clip}%
\pgfsetbuttcap%
\pgfsetroundjoin%
\pgfsetlinewidth{0.501875pt}%
\definecolor{currentstroke}{rgb}{0.276022,0.044167,0.370164}%
\pgfsetstrokecolor{currentstroke}%
\pgfsetdash{}{0pt}%
\pgfpathmoveto{\pgfqpoint{2.488375in}{5.453187in}}%
\pgfpathlineto{\pgfqpoint{2.435771in}{5.457157in}}%
\pgfusepath{stroke}%
\end{pgfscope}%
\begin{pgfscope}%
\pgfpathrectangle{\pgfqpoint{0.800000in}{5.105882in}}{\pgfqpoint{2.407767in}{1.544118in}}%
\pgfusepath{clip}%
\pgfsetbuttcap%
\pgfsetroundjoin%
\pgfsetlinewidth{0.501875pt}%
\definecolor{currentstroke}{rgb}{0.273809,0.031497,0.358853}%
\pgfsetstrokecolor{currentstroke}%
\pgfsetdash{}{0pt}%
\pgfpathmoveto{\pgfqpoint{2.435771in}{5.457157in}}%
\pgfpathlineto{\pgfqpoint{2.383145in}{5.460988in}}%
\pgfusepath{stroke}%
\end{pgfscope}%
\begin{pgfscope}%
\pgfpathrectangle{\pgfqpoint{0.800000in}{5.105882in}}{\pgfqpoint{2.407767in}{1.544118in}}%
\pgfusepath{clip}%
\pgfsetbuttcap%
\pgfsetroundjoin%
\pgfsetlinewidth{0.501875pt}%
\definecolor{currentstroke}{rgb}{0.281924,0.089666,0.412415}%
\pgfsetstrokecolor{currentstroke}%
\pgfsetdash{}{0pt}%
\pgfpathmoveto{\pgfqpoint{2.383145in}{5.460988in}}%
\pgfpathlineto{\pgfqpoint{2.330515in}{5.464788in}}%
\pgfusepath{stroke}%
\end{pgfscope}%
\begin{pgfscope}%
\pgfpathrectangle{\pgfqpoint{0.800000in}{5.105882in}}{\pgfqpoint{2.407767in}{1.544118in}}%
\pgfusepath{clip}%
\pgfsetbuttcap%
\pgfsetroundjoin%
\pgfsetlinewidth{0.501875pt}%
\definecolor{currentstroke}{rgb}{0.280894,0.078907,0.402329}%
\pgfsetstrokecolor{currentstroke}%
\pgfsetdash{}{0pt}%
\pgfpathmoveto{\pgfqpoint{2.330515in}{5.464788in}}%
\pgfpathlineto{\pgfqpoint{2.277940in}{5.468920in}}%
\pgfusepath{stroke}%
\end{pgfscope}%
\begin{pgfscope}%
\pgfpathrectangle{\pgfqpoint{0.800000in}{5.105882in}}{\pgfqpoint{2.407767in}{1.544118in}}%
\pgfusepath{clip}%
\pgfsetbuttcap%
\pgfsetroundjoin%
\pgfsetlinewidth{0.501875pt}%
\definecolor{currentstroke}{rgb}{0.280267,0.073417,0.397163}%
\pgfsetstrokecolor{currentstroke}%
\pgfsetdash{}{0pt}%
\pgfpathmoveto{\pgfqpoint{2.277940in}{5.468920in}}%
\pgfpathlineto{\pgfqpoint{2.225651in}{5.474218in}}%
\pgfusepath{stroke}%
\end{pgfscope}%
\begin{pgfscope}%
\pgfpathrectangle{\pgfqpoint{0.800000in}{5.105882in}}{\pgfqpoint{2.407767in}{1.544118in}}%
\pgfusepath{clip}%
\pgfsetbuttcap%
\pgfsetroundjoin%
\pgfsetlinewidth{0.501875pt}%
\definecolor{currentstroke}{rgb}{0.281446,0.084320,0.407414}%
\pgfsetstrokecolor{currentstroke}%
\pgfsetdash{}{0pt}%
\pgfpathmoveto{\pgfqpoint{2.225651in}{5.474218in}}%
\pgfpathlineto{\pgfqpoint{2.174044in}{5.481808in}}%
\pgfusepath{stroke}%
\end{pgfscope}%
\begin{pgfscope}%
\pgfpathrectangle{\pgfqpoint{0.800000in}{5.105882in}}{\pgfqpoint{2.407767in}{1.544118in}}%
\pgfusepath{clip}%
\pgfsetbuttcap%
\pgfsetroundjoin%
\pgfsetlinewidth{0.501875pt}%
\definecolor{currentstroke}{rgb}{0.282327,0.094955,0.417331}%
\pgfsetstrokecolor{currentstroke}%
\pgfsetdash{}{0pt}%
\pgfpathmoveto{\pgfqpoint{2.174044in}{5.481808in}}%
\pgfpathlineto{\pgfqpoint{2.123679in}{5.492143in}}%
\pgfusepath{stroke}%
\end{pgfscope}%
\begin{pgfscope}%
\pgfpathrectangle{\pgfqpoint{0.800000in}{5.105882in}}{\pgfqpoint{2.407767in}{1.544118in}}%
\pgfusepath{clip}%
\pgfsetbuttcap%
\pgfsetroundjoin%
\pgfsetlinewidth{0.501875pt}%
\definecolor{currentstroke}{rgb}{0.280894,0.078907,0.402329}%
\pgfsetstrokecolor{currentstroke}%
\pgfsetdash{}{0pt}%
\pgfpathmoveto{\pgfqpoint{2.123679in}{5.492143in}}%
\pgfpathlineto{\pgfqpoint{2.075643in}{5.506237in}}%
\pgfusepath{stroke}%
\end{pgfscope}%
\begin{pgfscope}%
\pgfpathrectangle{\pgfqpoint{0.800000in}{5.105882in}}{\pgfqpoint{2.407767in}{1.544118in}}%
\pgfusepath{clip}%
\pgfsetbuttcap%
\pgfsetroundjoin%
\pgfsetlinewidth{0.501875pt}%
\definecolor{currentstroke}{rgb}{0.283229,0.120777,0.440584}%
\pgfsetstrokecolor{currentstroke}%
\pgfsetdash{}{0pt}%
\pgfpathmoveto{\pgfqpoint{2.075643in}{5.506237in}}%
\pgfpathlineto{\pgfqpoint{2.031052in}{5.524308in}}%
\pgfusepath{stroke}%
\end{pgfscope}%
\begin{pgfscope}%
\pgfpathrectangle{\pgfqpoint{0.800000in}{5.105882in}}{\pgfqpoint{2.407767in}{1.544118in}}%
\pgfusepath{clip}%
\pgfsetbuttcap%
\pgfsetroundjoin%
\pgfsetlinewidth{0.501875pt}%
\definecolor{currentstroke}{rgb}{0.269944,0.014625,0.341379}%
\pgfsetstrokecolor{currentstroke}%
\pgfsetdash{}{0pt}%
\pgfpathmoveto{\pgfqpoint{2.654046in}{5.495734in}}%
\pgfpathlineto{\pgfqpoint{2.601207in}{5.497640in}}%
\pgfusepath{stroke}%
\end{pgfscope}%
\begin{pgfscope}%
\pgfpathrectangle{\pgfqpoint{0.800000in}{5.105882in}}{\pgfqpoint{2.407767in}{1.544118in}}%
\pgfusepath{clip}%
\pgfsetbuttcap%
\pgfsetroundjoin%
\pgfsetlinewidth{0.501875pt}%
\definecolor{currentstroke}{rgb}{0.272594,0.025563,0.353093}%
\pgfsetstrokecolor{currentstroke}%
\pgfsetdash{}{0pt}%
\pgfpathmoveto{\pgfqpoint{2.601207in}{5.497640in}}%
\pgfpathlineto{\pgfqpoint{2.548474in}{5.500874in}}%
\pgfusepath{stroke}%
\end{pgfscope}%
\begin{pgfscope}%
\pgfpathrectangle{\pgfqpoint{0.800000in}{5.105882in}}{\pgfqpoint{2.407767in}{1.544118in}}%
\pgfusepath{clip}%
\pgfsetbuttcap%
\pgfsetroundjoin%
\pgfsetlinewidth{0.501875pt}%
\definecolor{currentstroke}{rgb}{0.273809,0.031497,0.358853}%
\pgfsetstrokecolor{currentstroke}%
\pgfsetdash{}{0pt}%
\pgfpathmoveto{\pgfqpoint{2.548474in}{5.500874in}}%
\pgfpathlineto{\pgfqpoint{2.495703in}{5.503842in}}%
\pgfusepath{stroke}%
\end{pgfscope}%
\begin{pgfscope}%
\pgfpathrectangle{\pgfqpoint{0.800000in}{5.105882in}}{\pgfqpoint{2.407767in}{1.544118in}}%
\pgfusepath{clip}%
\pgfsetbuttcap%
\pgfsetroundjoin%
\pgfsetlinewidth{0.501875pt}%
\definecolor{currentstroke}{rgb}{0.278791,0.062145,0.386592}%
\pgfsetstrokecolor{currentstroke}%
\pgfsetdash{}{0pt}%
\pgfpathmoveto{\pgfqpoint{2.495703in}{5.503842in}}%
\pgfpathlineto{\pgfqpoint{2.442934in}{5.506822in}}%
\pgfusepath{stroke}%
\end{pgfscope}%
\begin{pgfscope}%
\pgfpathrectangle{\pgfqpoint{0.800000in}{5.105882in}}{\pgfqpoint{2.407767in}{1.544118in}}%
\pgfusepath{clip}%
\pgfsetbuttcap%
\pgfsetroundjoin%
\pgfsetlinewidth{0.501875pt}%
\definecolor{currentstroke}{rgb}{0.282327,0.094955,0.417331}%
\pgfsetstrokecolor{currentstroke}%
\pgfsetdash{}{0pt}%
\pgfpathmoveto{\pgfqpoint{2.442934in}{5.506822in}}%
\pgfpathlineto{\pgfqpoint{2.390161in}{5.509768in}}%
\pgfusepath{stroke}%
\end{pgfscope}%
\begin{pgfscope}%
\pgfpathrectangle{\pgfqpoint{0.800000in}{5.105882in}}{\pgfqpoint{2.407767in}{1.544118in}}%
\pgfusepath{clip}%
\pgfsetbuttcap%
\pgfsetroundjoin%
\pgfsetlinewidth{0.501875pt}%
\definecolor{currentstroke}{rgb}{0.282656,0.100196,0.422160}%
\pgfsetstrokecolor{currentstroke}%
\pgfsetdash{}{0pt}%
\pgfpathmoveto{\pgfqpoint{2.390161in}{5.509768in}}%
\pgfpathlineto{\pgfqpoint{2.337366in}{5.512575in}}%
\pgfusepath{stroke}%
\end{pgfscope}%
\begin{pgfscope}%
\pgfpathrectangle{\pgfqpoint{0.800000in}{5.105882in}}{\pgfqpoint{2.407767in}{1.544118in}}%
\pgfusepath{clip}%
\pgfsetbuttcap%
\pgfsetroundjoin%
\pgfsetlinewidth{0.501875pt}%
\definecolor{currentstroke}{rgb}{0.282656,0.100196,0.422160}%
\pgfsetstrokecolor{currentstroke}%
\pgfsetdash{}{0pt}%
\pgfpathmoveto{\pgfqpoint{2.337366in}{5.512575in}}%
\pgfpathlineto{\pgfqpoint{2.284671in}{5.516019in}}%
\pgfusepath{stroke}%
\end{pgfscope}%
\begin{pgfscope}%
\pgfpathrectangle{\pgfqpoint{0.800000in}{5.105882in}}{\pgfqpoint{2.407767in}{1.544118in}}%
\pgfusepath{clip}%
\pgfsetbuttcap%
\pgfsetroundjoin%
\pgfsetlinewidth{0.501875pt}%
\definecolor{currentstroke}{rgb}{0.283197,0.115680,0.436115}%
\pgfsetstrokecolor{currentstroke}%
\pgfsetdash{}{0pt}%
\pgfpathmoveto{\pgfqpoint{2.284671in}{5.516019in}}%
\pgfpathlineto{\pgfqpoint{2.232140in}{5.520392in}}%
\pgfusepath{stroke}%
\end{pgfscope}%
\begin{pgfscope}%
\pgfpathrectangle{\pgfqpoint{0.800000in}{5.105882in}}{\pgfqpoint{2.407767in}{1.544118in}}%
\pgfusepath{clip}%
\pgfsetbuttcap%
\pgfsetroundjoin%
\pgfsetlinewidth{0.501875pt}%
\definecolor{currentstroke}{rgb}{0.282910,0.105393,0.426902}%
\pgfsetstrokecolor{currentstroke}%
\pgfsetdash{}{0pt}%
\pgfpathmoveto{\pgfqpoint{2.232140in}{5.520392in}}%
\pgfpathlineto{\pgfqpoint{2.180005in}{5.526285in}}%
\pgfusepath{stroke}%
\end{pgfscope}%
\begin{pgfscope}%
\pgfpathrectangle{\pgfqpoint{0.800000in}{5.105882in}}{\pgfqpoint{2.407767in}{1.544118in}}%
\pgfusepath{clip}%
\pgfsetbuttcap%
\pgfsetroundjoin%
\pgfsetlinewidth{0.501875pt}%
\definecolor{currentstroke}{rgb}{0.282327,0.094955,0.417331}%
\pgfsetstrokecolor{currentstroke}%
\pgfsetdash{}{0pt}%
\pgfpathmoveto{\pgfqpoint{2.180005in}{5.526285in}}%
\pgfpathlineto{\pgfqpoint{2.128842in}{5.534961in}}%
\pgfusepath{stroke}%
\end{pgfscope}%
\begin{pgfscope}%
\pgfpathrectangle{\pgfqpoint{0.800000in}{5.105882in}}{\pgfqpoint{2.407767in}{1.544118in}}%
\pgfusepath{clip}%
\pgfsetbuttcap%
\pgfsetroundjoin%
\pgfsetlinewidth{0.501875pt}%
\definecolor{currentstroke}{rgb}{0.271305,0.019942,0.347269}%
\pgfsetstrokecolor{currentstroke}%
\pgfsetdash{}{0pt}%
\pgfpathmoveto{\pgfqpoint{2.654046in}{5.530480in}}%
\pgfpathlineto{\pgfqpoint{2.601178in}{5.532430in}}%
\pgfusepath{stroke}%
\end{pgfscope}%
\begin{pgfscope}%
\pgfpathrectangle{\pgfqpoint{0.800000in}{5.105882in}}{\pgfqpoint{2.407767in}{1.544118in}}%
\pgfusepath{clip}%
\pgfsetbuttcap%
\pgfsetroundjoin%
\pgfsetlinewidth{0.501875pt}%
\definecolor{currentstroke}{rgb}{0.274952,0.037752,0.364543}%
\pgfsetstrokecolor{currentstroke}%
\pgfsetdash{}{0pt}%
\pgfpathmoveto{\pgfqpoint{2.601178in}{5.532430in}}%
\pgfpathlineto{\pgfqpoint{2.548369in}{5.535117in}}%
\pgfusepath{stroke}%
\end{pgfscope}%
\begin{pgfscope}%
\pgfpathrectangle{\pgfqpoint{0.800000in}{5.105882in}}{\pgfqpoint{2.407767in}{1.544118in}}%
\pgfusepath{clip}%
\pgfsetbuttcap%
\pgfsetroundjoin%
\pgfsetlinewidth{0.501875pt}%
\definecolor{currentstroke}{rgb}{0.279566,0.067836,0.391917}%
\pgfsetstrokecolor{currentstroke}%
\pgfsetdash{}{0pt}%
\pgfpathmoveto{\pgfqpoint{2.548369in}{5.535117in}}%
\pgfpathlineto{\pgfqpoint{2.495569in}{5.537855in}}%
\pgfusepath{stroke}%
\end{pgfscope}%
\begin{pgfscope}%
\pgfpathrectangle{\pgfqpoint{0.800000in}{5.105882in}}{\pgfqpoint{2.407767in}{1.544118in}}%
\pgfusepath{clip}%
\pgfsetbuttcap%
\pgfsetroundjoin%
\pgfsetlinewidth{0.501875pt}%
\definecolor{currentstroke}{rgb}{0.282327,0.094955,0.417331}%
\pgfsetstrokecolor{currentstroke}%
\pgfsetdash{}{0pt}%
\pgfpathmoveto{\pgfqpoint{2.495569in}{5.537855in}}%
\pgfpathlineto{\pgfqpoint{2.442749in}{5.540437in}}%
\pgfusepath{stroke}%
\end{pgfscope}%
\begin{pgfscope}%
\pgfpathrectangle{\pgfqpoint{0.800000in}{5.105882in}}{\pgfqpoint{2.407767in}{1.544118in}}%
\pgfusepath{clip}%
\pgfsetbuttcap%
\pgfsetroundjoin%
\pgfsetlinewidth{0.501875pt}%
\definecolor{currentstroke}{rgb}{0.283197,0.115680,0.436115}%
\pgfsetstrokecolor{currentstroke}%
\pgfsetdash{}{0pt}%
\pgfpathmoveto{\pgfqpoint{2.442749in}{5.540437in}}%
\pgfpathlineto{\pgfqpoint{2.389915in}{5.542918in}}%
\pgfusepath{stroke}%
\end{pgfscope}%
\begin{pgfscope}%
\pgfpathrectangle{\pgfqpoint{0.800000in}{5.105882in}}{\pgfqpoint{2.407767in}{1.544118in}}%
\pgfusepath{clip}%
\pgfsetbuttcap%
\pgfsetroundjoin%
\pgfsetlinewidth{0.501875pt}%
\definecolor{currentstroke}{rgb}{0.283229,0.120777,0.440584}%
\pgfsetstrokecolor{currentstroke}%
\pgfsetdash{}{0pt}%
\pgfpathmoveto{\pgfqpoint{2.389915in}{5.542918in}}%
\pgfpathlineto{\pgfqpoint{2.337120in}{5.545693in}}%
\pgfusepath{stroke}%
\end{pgfscope}%
\begin{pgfscope}%
\pgfpathrectangle{\pgfqpoint{0.800000in}{5.105882in}}{\pgfqpoint{2.407767in}{1.544118in}}%
\pgfusepath{clip}%
\pgfsetbuttcap%
\pgfsetroundjoin%
\pgfsetlinewidth{0.501875pt}%
\definecolor{currentstroke}{rgb}{0.283187,0.125848,0.444960}%
\pgfsetstrokecolor{currentstroke}%
\pgfsetdash{}{0pt}%
\pgfpathmoveto{\pgfqpoint{2.337120in}{5.545693in}}%
\pgfpathlineto{\pgfqpoint{2.284499in}{5.549541in}}%
\pgfusepath{stroke}%
\end{pgfscope}%
\begin{pgfscope}%
\pgfpathrectangle{\pgfqpoint{0.800000in}{5.105882in}}{\pgfqpoint{2.407767in}{1.544118in}}%
\pgfusepath{clip}%
\pgfsetbuttcap%
\pgfsetroundjoin%
\pgfsetlinewidth{0.501875pt}%
\definecolor{currentstroke}{rgb}{0.282623,0.140926,0.457517}%
\pgfsetstrokecolor{currentstroke}%
\pgfsetdash{}{0pt}%
\pgfpathmoveto{\pgfqpoint{2.284499in}{5.549541in}}%
\pgfpathlineto{\pgfqpoint{2.232125in}{5.554594in}}%
\pgfusepath{stroke}%
\end{pgfscope}%
\begin{pgfscope}%
\pgfpathrectangle{\pgfqpoint{0.800000in}{5.105882in}}{\pgfqpoint{2.407767in}{1.544118in}}%
\pgfusepath{clip}%
\pgfsetbuttcap%
\pgfsetroundjoin%
\pgfsetlinewidth{0.501875pt}%
\definecolor{currentstroke}{rgb}{0.281887,0.150881,0.465405}%
\pgfsetstrokecolor{currentstroke}%
\pgfsetdash{}{0pt}%
\pgfpathmoveto{\pgfqpoint{2.232125in}{5.554594in}}%
\pgfpathlineto{\pgfqpoint{2.180129in}{5.561010in}}%
\pgfusepath{stroke}%
\end{pgfscope}%
\begin{pgfscope}%
\pgfpathrectangle{\pgfqpoint{0.800000in}{5.105882in}}{\pgfqpoint{2.407767in}{1.544118in}}%
\pgfusepath{clip}%
\pgfsetbuttcap%
\pgfsetroundjoin%
\pgfsetlinewidth{0.501875pt}%
\definecolor{currentstroke}{rgb}{0.281887,0.150881,0.465405}%
\pgfsetstrokecolor{currentstroke}%
\pgfsetdash{}{0pt}%
\pgfpathmoveto{\pgfqpoint{2.180129in}{5.561010in}}%
\pgfpathlineto{\pgfqpoint{2.129040in}{5.569856in}}%
\pgfusepath{stroke}%
\end{pgfscope}%
\begin{pgfscope}%
\pgfpathrectangle{\pgfqpoint{0.800000in}{5.105882in}}{\pgfqpoint{2.407767in}{1.544118in}}%
\pgfusepath{clip}%
\pgfsetbuttcap%
\pgfsetroundjoin%
\pgfsetlinewidth{0.501875pt}%
\definecolor{currentstroke}{rgb}{0.281887,0.150881,0.465405}%
\pgfsetstrokecolor{currentstroke}%
\pgfsetdash{}{0pt}%
\pgfpathmoveto{\pgfqpoint{2.129040in}{5.569856in}}%
\pgfpathlineto{\pgfqpoint{2.079229in}{5.581331in}}%
\pgfusepath{stroke}%
\end{pgfscope}%
\begin{pgfscope}%
\pgfpathrectangle{\pgfqpoint{0.800000in}{5.105882in}}{\pgfqpoint{2.407767in}{1.544118in}}%
\pgfusepath{clip}%
\pgfsetbuttcap%
\pgfsetroundjoin%
\pgfsetlinewidth{0.501875pt}%
\definecolor{currentstroke}{rgb}{0.280255,0.165693,0.476498}%
\pgfsetstrokecolor{currentstroke}%
\pgfsetdash{}{0pt}%
\pgfpathmoveto{\pgfqpoint{2.079229in}{5.581331in}}%
\pgfpathlineto{\pgfqpoint{2.031060in}{5.595314in}}%
\pgfusepath{stroke}%
\end{pgfscope}%
\begin{pgfscope}%
\pgfpathrectangle{\pgfqpoint{0.800000in}{5.105882in}}{\pgfqpoint{2.407767in}{1.544118in}}%
\pgfusepath{clip}%
\pgfsetbuttcap%
\pgfsetroundjoin%
\pgfsetlinewidth{0.501875pt}%
\definecolor{currentstroke}{rgb}{0.280255,0.165693,0.476498}%
\pgfsetstrokecolor{currentstroke}%
\pgfsetdash{}{0pt}%
\pgfpathmoveto{\pgfqpoint{2.031060in}{5.595314in}}%
\pgfpathlineto{\pgfqpoint{1.984370in}{5.611300in}}%
\pgfusepath{stroke}%
\end{pgfscope}%
\begin{pgfscope}%
\pgfpathrectangle{\pgfqpoint{0.800000in}{5.105882in}}{\pgfqpoint{2.407767in}{1.544118in}}%
\pgfusepath{clip}%
\pgfsetbuttcap%
\pgfsetroundjoin%
\pgfsetlinewidth{0.501875pt}%
\definecolor{currentstroke}{rgb}{0.275191,0.194905,0.496005}%
\pgfsetstrokecolor{currentstroke}%
\pgfsetdash{}{0pt}%
\pgfpathmoveto{\pgfqpoint{1.984370in}{5.611300in}}%
\pgfpathlineto{\pgfqpoint{1.939058in}{5.628834in}}%
\pgfusepath{stroke}%
\end{pgfscope}%
\begin{pgfscope}%
\pgfpathrectangle{\pgfqpoint{0.800000in}{5.105882in}}{\pgfqpoint{2.407767in}{1.544118in}}%
\pgfusepath{clip}%
\pgfsetbuttcap%
\pgfsetroundjoin%
\pgfsetlinewidth{0.501875pt}%
\definecolor{currentstroke}{rgb}{0.271828,0.209303,0.504434}%
\pgfsetstrokecolor{currentstroke}%
\pgfsetdash{}{0pt}%
\pgfpathmoveto{\pgfqpoint{1.939058in}{5.628834in}}%
\pgfpathlineto{\pgfqpoint{1.895926in}{5.648463in}}%
\pgfusepath{stroke}%
\end{pgfscope}%
\begin{pgfscope}%
\pgfpathrectangle{\pgfqpoint{0.800000in}{5.105882in}}{\pgfqpoint{2.407767in}{1.544118in}}%
\pgfusepath{clip}%
\pgfsetbuttcap%
\pgfsetroundjoin%
\pgfsetlinewidth{0.501875pt}%
\definecolor{currentstroke}{rgb}{0.274128,0.199721,0.498911}%
\pgfsetstrokecolor{currentstroke}%
\pgfsetdash{}{0pt}%
\pgfpathmoveto{\pgfqpoint{1.895926in}{5.648463in}}%
\pgfpathlineto{\pgfqpoint{1.854669in}{5.669759in}}%
\pgfusepath{stroke}%
\end{pgfscope}%
\begin{pgfscope}%
\pgfpathrectangle{\pgfqpoint{0.800000in}{5.105882in}}{\pgfqpoint{2.407767in}{1.544118in}}%
\pgfusepath{clip}%
\pgfsetbuttcap%
\pgfsetroundjoin%
\pgfsetlinewidth{0.501875pt}%
\definecolor{currentstroke}{rgb}{0.266580,0.228262,0.514349}%
\pgfsetstrokecolor{currentstroke}%
\pgfsetdash{}{0pt}%
\pgfpathmoveto{\pgfqpoint{1.854669in}{5.669759in}}%
\pgfpathlineto{\pgfqpoint{1.813977in}{5.691471in}}%
\pgfusepath{stroke}%
\end{pgfscope}%
\begin{pgfscope}%
\pgfpathrectangle{\pgfqpoint{0.800000in}{5.105882in}}{\pgfqpoint{2.407767in}{1.544118in}}%
\pgfusepath{clip}%
\pgfsetbuttcap%
\pgfsetroundjoin%
\pgfsetlinewidth{0.501875pt}%
\definecolor{currentstroke}{rgb}{0.272594,0.025563,0.353093}%
\pgfsetstrokecolor{currentstroke}%
\pgfsetdash{}{0pt}%
\pgfpathmoveto{\pgfqpoint{2.654046in}{5.565226in}}%
\pgfpathlineto{\pgfqpoint{2.601169in}{5.567246in}}%
\pgfusepath{stroke}%
\end{pgfscope}%
\begin{pgfscope}%
\pgfpathrectangle{\pgfqpoint{0.800000in}{5.105882in}}{\pgfqpoint{2.407767in}{1.544118in}}%
\pgfusepath{clip}%
\pgfsetbuttcap%
\pgfsetroundjoin%
\pgfsetlinewidth{0.501875pt}%
\definecolor{currentstroke}{rgb}{0.276022,0.044167,0.370164}%
\pgfsetstrokecolor{currentstroke}%
\pgfsetdash{}{0pt}%
\pgfpathmoveto{\pgfqpoint{2.601169in}{5.567246in}}%
\pgfpathlineto{\pgfqpoint{2.548314in}{5.569516in}}%
\pgfusepath{stroke}%
\end{pgfscope}%
\begin{pgfscope}%
\pgfpathrectangle{\pgfqpoint{0.800000in}{5.105882in}}{\pgfqpoint{2.407767in}{1.544118in}}%
\pgfusepath{clip}%
\pgfsetbuttcap%
\pgfsetroundjoin%
\pgfsetlinewidth{0.501875pt}%
\definecolor{currentstroke}{rgb}{0.280894,0.078907,0.402329}%
\pgfsetstrokecolor{currentstroke}%
\pgfsetdash{}{0pt}%
\pgfpathmoveto{\pgfqpoint{2.548314in}{5.569516in}}%
\pgfpathlineto{\pgfqpoint{2.495471in}{5.571923in}}%
\pgfusepath{stroke}%
\end{pgfscope}%
\begin{pgfscope}%
\pgfpathrectangle{\pgfqpoint{0.800000in}{5.105882in}}{\pgfqpoint{2.407767in}{1.544118in}}%
\pgfusepath{clip}%
\pgfsetbuttcap%
\pgfsetroundjoin%
\pgfsetlinewidth{0.501875pt}%
\definecolor{currentstroke}{rgb}{0.281446,0.084320,0.407414}%
\pgfsetstrokecolor{currentstroke}%
\pgfsetdash{}{0pt}%
\pgfpathmoveto{\pgfqpoint{2.495471in}{5.571923in}}%
\pgfpathlineto{\pgfqpoint{2.442631in}{5.574353in}}%
\pgfusepath{stroke}%
\end{pgfscope}%
\begin{pgfscope}%
\pgfpathrectangle{\pgfqpoint{0.800000in}{5.105882in}}{\pgfqpoint{2.407767in}{1.544118in}}%
\pgfusepath{clip}%
\pgfsetbuttcap%
\pgfsetroundjoin%
\pgfsetlinewidth{0.501875pt}%
\definecolor{currentstroke}{rgb}{0.283187,0.125848,0.444960}%
\pgfsetstrokecolor{currentstroke}%
\pgfsetdash{}{0pt}%
\pgfpathmoveto{\pgfqpoint{2.442631in}{5.574353in}}%
\pgfpathlineto{\pgfqpoint{2.389818in}{5.576995in}}%
\pgfusepath{stroke}%
\end{pgfscope}%
\begin{pgfscope}%
\pgfpathrectangle{\pgfqpoint{0.800000in}{5.105882in}}{\pgfqpoint{2.407767in}{1.544118in}}%
\pgfusepath{clip}%
\pgfsetbuttcap%
\pgfsetroundjoin%
\pgfsetlinewidth{0.501875pt}%
\definecolor{currentstroke}{rgb}{0.282290,0.145912,0.461510}%
\pgfsetstrokecolor{currentstroke}%
\pgfsetdash{}{0pt}%
\pgfpathmoveto{\pgfqpoint{2.389818in}{5.576995in}}%
\pgfpathlineto{\pgfqpoint{2.337108in}{5.580327in}}%
\pgfusepath{stroke}%
\end{pgfscope}%
\begin{pgfscope}%
\pgfpathrectangle{\pgfqpoint{0.800000in}{5.105882in}}{\pgfqpoint{2.407767in}{1.544118in}}%
\pgfusepath{clip}%
\pgfsetbuttcap%
\pgfsetroundjoin%
\pgfsetlinewidth{0.501875pt}%
\definecolor{currentstroke}{rgb}{0.282290,0.145912,0.461510}%
\pgfsetstrokecolor{currentstroke}%
\pgfsetdash{}{0pt}%
\pgfpathmoveto{\pgfqpoint{2.337108in}{5.580327in}}%
\pgfpathlineto{\pgfqpoint{2.284555in}{5.584564in}}%
\pgfusepath{stroke}%
\end{pgfscope}%
\begin{pgfscope}%
\pgfpathrectangle{\pgfqpoint{0.800000in}{5.105882in}}{\pgfqpoint{2.407767in}{1.544118in}}%
\pgfusepath{clip}%
\pgfsetbuttcap%
\pgfsetroundjoin%
\pgfsetlinewidth{0.501875pt}%
\definecolor{currentstroke}{rgb}{0.280255,0.165693,0.476498}%
\pgfsetstrokecolor{currentstroke}%
\pgfsetdash{}{0pt}%
\pgfpathmoveto{\pgfqpoint{2.284555in}{5.584564in}}%
\pgfpathlineto{\pgfqpoint{2.232252in}{5.589910in}}%
\pgfusepath{stroke}%
\end{pgfscope}%
\begin{pgfscope}%
\pgfpathrectangle{\pgfqpoint{0.800000in}{5.105882in}}{\pgfqpoint{2.407767in}{1.544118in}}%
\pgfusepath{clip}%
\pgfsetbuttcap%
\pgfsetroundjoin%
\pgfsetlinewidth{0.501875pt}%
\definecolor{currentstroke}{rgb}{0.279574,0.170599,0.479997}%
\pgfsetstrokecolor{currentstroke}%
\pgfsetdash{}{0pt}%
\pgfpathmoveto{\pgfqpoint{2.232252in}{5.589910in}}%
\pgfpathlineto{\pgfqpoint{2.180311in}{5.596547in}}%
\pgfusepath{stroke}%
\end{pgfscope}%
\begin{pgfscope}%
\pgfpathrectangle{\pgfqpoint{0.800000in}{5.105882in}}{\pgfqpoint{2.407767in}{1.544118in}}%
\pgfusepath{clip}%
\pgfsetbuttcap%
\pgfsetroundjoin%
\pgfsetlinewidth{0.501875pt}%
\definecolor{currentstroke}{rgb}{0.280255,0.165693,0.476498}%
\pgfsetstrokecolor{currentstroke}%
\pgfsetdash{}{0pt}%
\pgfpathmoveto{\pgfqpoint{2.180311in}{5.596547in}}%
\pgfpathlineto{\pgfqpoint{2.129061in}{5.605020in}}%
\pgfusepath{stroke}%
\end{pgfscope}%
\begin{pgfscope}%
\pgfpathrectangle{\pgfqpoint{0.800000in}{5.105882in}}{\pgfqpoint{2.407767in}{1.544118in}}%
\pgfusepath{clip}%
\pgfsetbuttcap%
\pgfsetroundjoin%
\pgfsetlinewidth{0.501875pt}%
\definecolor{currentstroke}{rgb}{0.272594,0.025563,0.353093}%
\pgfsetstrokecolor{currentstroke}%
\pgfsetdash{}{0pt}%
\pgfpathmoveto{\pgfqpoint{2.654046in}{5.599972in}}%
\pgfpathlineto{\pgfqpoint{2.601134in}{5.601477in}}%
\pgfusepath{stroke}%
\end{pgfscope}%
\begin{pgfscope}%
\pgfpathrectangle{\pgfqpoint{0.800000in}{5.105882in}}{\pgfqpoint{2.407767in}{1.544118in}}%
\pgfusepath{clip}%
\pgfsetbuttcap%
\pgfsetroundjoin%
\pgfsetlinewidth{0.501875pt}%
\definecolor{currentstroke}{rgb}{0.279566,0.067836,0.391917}%
\pgfsetstrokecolor{currentstroke}%
\pgfsetdash{}{0pt}%
\pgfpathmoveto{\pgfqpoint{2.601134in}{5.601477in}}%
\pgfpathlineto{\pgfqpoint{2.548254in}{5.603506in}}%
\pgfusepath{stroke}%
\end{pgfscope}%
\begin{pgfscope}%
\pgfpathrectangle{\pgfqpoint{0.800000in}{5.105882in}}{\pgfqpoint{2.407767in}{1.544118in}}%
\pgfusepath{clip}%
\pgfsetbuttcap%
\pgfsetroundjoin%
\pgfsetlinewidth{0.501875pt}%
\definecolor{currentstroke}{rgb}{0.281924,0.089666,0.412415}%
\pgfsetstrokecolor{currentstroke}%
\pgfsetdash{}{0pt}%
\pgfpathmoveto{\pgfqpoint{2.548254in}{5.603506in}}%
\pgfpathlineto{\pgfqpoint{2.495383in}{5.605635in}}%
\pgfusepath{stroke}%
\end{pgfscope}%
\begin{pgfscope}%
\pgfpathrectangle{\pgfqpoint{0.800000in}{5.105882in}}{\pgfqpoint{2.407767in}{1.544118in}}%
\pgfusepath{clip}%
\pgfsetbuttcap%
\pgfsetroundjoin%
\pgfsetlinewidth{0.501875pt}%
\definecolor{currentstroke}{rgb}{0.283187,0.125848,0.444960}%
\pgfsetstrokecolor{currentstroke}%
\pgfsetdash{}{0pt}%
\pgfpathmoveto{\pgfqpoint{2.495383in}{5.605635in}}%
\pgfpathlineto{\pgfqpoint{2.442512in}{5.607758in}}%
\pgfusepath{stroke}%
\end{pgfscope}%
\begin{pgfscope}%
\pgfpathrectangle{\pgfqpoint{0.800000in}{5.105882in}}{\pgfqpoint{2.407767in}{1.544118in}}%
\pgfusepath{clip}%
\pgfsetbuttcap%
\pgfsetroundjoin%
\pgfsetlinewidth{0.501875pt}%
\definecolor{currentstroke}{rgb}{0.282290,0.145912,0.461510}%
\pgfsetstrokecolor{currentstroke}%
\pgfsetdash{}{0pt}%
\pgfpathmoveto{\pgfqpoint{2.442512in}{5.607758in}}%
\pgfpathlineto{\pgfqpoint{2.389669in}{5.610127in}}%
\pgfusepath{stroke}%
\end{pgfscope}%
\begin{pgfscope}%
\pgfpathrectangle{\pgfqpoint{0.800000in}{5.105882in}}{\pgfqpoint{2.407767in}{1.544118in}}%
\pgfusepath{clip}%
\pgfsetbuttcap%
\pgfsetroundjoin%
\pgfsetlinewidth{0.501875pt}%
\definecolor{currentstroke}{rgb}{0.278012,0.180367,0.486697}%
\pgfsetstrokecolor{currentstroke}%
\pgfsetdash{}{0pt}%
\pgfpathmoveto{\pgfqpoint{2.389669in}{5.610127in}}%
\pgfpathlineto{\pgfqpoint{2.336878in}{5.612947in}}%
\pgfusepath{stroke}%
\end{pgfscope}%
\begin{pgfscope}%
\pgfpathrectangle{\pgfqpoint{0.800000in}{5.105882in}}{\pgfqpoint{2.407767in}{1.544118in}}%
\pgfusepath{clip}%
\pgfsetbuttcap%
\pgfsetroundjoin%
\pgfsetlinewidth{0.501875pt}%
\definecolor{currentstroke}{rgb}{0.273809,0.031497,0.358853}%
\pgfsetstrokecolor{currentstroke}%
\pgfsetdash{}{0pt}%
\pgfpathmoveto{\pgfqpoint{2.654046in}{5.634718in}}%
\pgfpathlineto{\pgfqpoint{2.601130in}{5.636224in}}%
\pgfusepath{stroke}%
\end{pgfscope}%
\begin{pgfscope}%
\pgfpathrectangle{\pgfqpoint{0.800000in}{5.105882in}}{\pgfqpoint{2.407767in}{1.544118in}}%
\pgfusepath{clip}%
\pgfsetbuttcap%
\pgfsetroundjoin%
\pgfsetlinewidth{0.501875pt}%
\definecolor{currentstroke}{rgb}{0.277941,0.056324,0.381191}%
\pgfsetstrokecolor{currentstroke}%
\pgfsetdash{}{0pt}%
\pgfpathmoveto{\pgfqpoint{2.601130in}{5.636224in}}%
\pgfpathlineto{\pgfqpoint{2.548234in}{5.638070in}}%
\pgfusepath{stroke}%
\end{pgfscope}%
\begin{pgfscope}%
\pgfpathrectangle{\pgfqpoint{0.800000in}{5.105882in}}{\pgfqpoint{2.407767in}{1.544118in}}%
\pgfusepath{clip}%
\pgfsetbuttcap%
\pgfsetroundjoin%
\pgfsetlinewidth{0.501875pt}%
\definecolor{currentstroke}{rgb}{0.282910,0.105393,0.426902}%
\pgfsetstrokecolor{currentstroke}%
\pgfsetdash{}{0pt}%
\pgfpathmoveto{\pgfqpoint{2.548234in}{5.638070in}}%
\pgfpathlineto{\pgfqpoint{2.495334in}{5.639889in}}%
\pgfusepath{stroke}%
\end{pgfscope}%
\begin{pgfscope}%
\pgfpathrectangle{\pgfqpoint{0.800000in}{5.105882in}}{\pgfqpoint{2.407767in}{1.544118in}}%
\pgfusepath{clip}%
\pgfsetbuttcap%
\pgfsetroundjoin%
\pgfsetlinewidth{0.501875pt}%
\definecolor{currentstroke}{rgb}{0.282884,0.135920,0.453427}%
\pgfsetstrokecolor{currentstroke}%
\pgfsetdash{}{0pt}%
\pgfpathmoveto{\pgfqpoint{2.495334in}{5.639889in}}%
\pgfpathlineto{\pgfqpoint{2.442447in}{5.641850in}}%
\pgfusepath{stroke}%
\end{pgfscope}%
\begin{pgfscope}%
\pgfpathrectangle{\pgfqpoint{0.800000in}{5.105882in}}{\pgfqpoint{2.407767in}{1.544118in}}%
\pgfusepath{clip}%
\pgfsetbuttcap%
\pgfsetroundjoin%
\pgfsetlinewidth{0.501875pt}%
\definecolor{currentstroke}{rgb}{0.278826,0.175490,0.483397}%
\pgfsetstrokecolor{currentstroke}%
\pgfsetdash{}{0pt}%
\pgfpathmoveto{\pgfqpoint{2.442447in}{5.641850in}}%
\pgfpathlineto{\pgfqpoint{2.389579in}{5.644010in}}%
\pgfusepath{stroke}%
\end{pgfscope}%
\begin{pgfscope}%
\pgfpathrectangle{\pgfqpoint{0.800000in}{5.105882in}}{\pgfqpoint{2.407767in}{1.544118in}}%
\pgfusepath{clip}%
\pgfsetbuttcap%
\pgfsetroundjoin%
\pgfsetlinewidth{0.501875pt}%
\definecolor{currentstroke}{rgb}{0.271828,0.209303,0.504434}%
\pgfsetstrokecolor{currentstroke}%
\pgfsetdash{}{0pt}%
\pgfpathmoveto{\pgfqpoint{2.389579in}{5.644010in}}%
\pgfpathlineto{\pgfqpoint{2.336753in}{5.646548in}}%
\pgfusepath{stroke}%
\end{pgfscope}%
\begin{pgfscope}%
\pgfpathrectangle{\pgfqpoint{0.800000in}{5.105882in}}{\pgfqpoint{2.407767in}{1.544118in}}%
\pgfusepath{clip}%
\pgfsetbuttcap%
\pgfsetroundjoin%
\pgfsetlinewidth{0.501875pt}%
\definecolor{currentstroke}{rgb}{0.267968,0.223549,0.512008}%
\pgfsetstrokecolor{currentstroke}%
\pgfsetdash{}{0pt}%
\pgfpathmoveto{\pgfqpoint{2.336753in}{5.646548in}}%
\pgfpathlineto{\pgfqpoint{2.284006in}{5.649675in}}%
\pgfusepath{stroke}%
\end{pgfscope}%
\begin{pgfscope}%
\pgfpathrectangle{\pgfqpoint{0.800000in}{5.105882in}}{\pgfqpoint{2.407767in}{1.544118in}}%
\pgfusepath{clip}%
\pgfsetbuttcap%
\pgfsetroundjoin%
\pgfsetlinewidth{0.501875pt}%
\definecolor{currentstroke}{rgb}{0.269308,0.218818,0.509577}%
\pgfsetstrokecolor{currentstroke}%
\pgfsetdash{}{0pt}%
\pgfpathmoveto{\pgfqpoint{2.284006in}{5.649675in}}%
\pgfpathlineto{\pgfqpoint{2.231455in}{5.653888in}}%
\pgfusepath{stroke}%
\end{pgfscope}%
\begin{pgfscope}%
\pgfpathrectangle{\pgfqpoint{0.800000in}{5.105882in}}{\pgfqpoint{2.407767in}{1.544118in}}%
\pgfusepath{clip}%
\pgfsetbuttcap%
\pgfsetroundjoin%
\pgfsetlinewidth{0.501875pt}%
\definecolor{currentstroke}{rgb}{0.265145,0.232956,0.516599}%
\pgfsetstrokecolor{currentstroke}%
\pgfsetdash{}{0pt}%
\pgfpathmoveto{\pgfqpoint{2.231455in}{5.653888in}}%
\pgfpathlineto{\pgfqpoint{2.179138in}{5.659194in}}%
\pgfusepath{stroke}%
\end{pgfscope}%
\begin{pgfscope}%
\pgfpathrectangle{\pgfqpoint{0.800000in}{5.105882in}}{\pgfqpoint{2.407767in}{1.544118in}}%
\pgfusepath{clip}%
\pgfsetbuttcap%
\pgfsetroundjoin%
\pgfsetlinewidth{0.501875pt}%
\definecolor{currentstroke}{rgb}{0.269308,0.218818,0.509577}%
\pgfsetstrokecolor{currentstroke}%
\pgfsetdash{}{0pt}%
\pgfpathmoveto{\pgfqpoint{2.179138in}{5.659194in}}%
\pgfpathlineto{\pgfqpoint{2.127160in}{5.665684in}}%
\pgfusepath{stroke}%
\end{pgfscope}%
\begin{pgfscope}%
\pgfpathrectangle{\pgfqpoint{0.800000in}{5.105882in}}{\pgfqpoint{2.407767in}{1.544118in}}%
\pgfusepath{clip}%
\pgfsetbuttcap%
\pgfsetroundjoin%
\pgfsetlinewidth{0.501875pt}%
\definecolor{currentstroke}{rgb}{0.257322,0.256130,0.526563}%
\pgfsetstrokecolor{currentstroke}%
\pgfsetdash{}{0pt}%
\pgfpathmoveto{\pgfqpoint{2.127160in}{5.665684in}}%
\pgfpathlineto{\pgfqpoint{2.075847in}{5.674048in}}%
\pgfusepath{stroke}%
\end{pgfscope}%
\begin{pgfscope}%
\pgfpathrectangle{\pgfqpoint{0.800000in}{5.105882in}}{\pgfqpoint{2.407767in}{1.544118in}}%
\pgfusepath{clip}%
\pgfsetbuttcap%
\pgfsetroundjoin%
\pgfsetlinewidth{0.501875pt}%
\definecolor{currentstroke}{rgb}{0.262138,0.242286,0.520837}%
\pgfsetstrokecolor{currentstroke}%
\pgfsetdash{}{0pt}%
\pgfpathmoveto{\pgfqpoint{2.075847in}{5.674048in}}%
\pgfpathlineto{\pgfqpoint{2.025196in}{5.683975in}}%
\pgfusepath{stroke}%
\end{pgfscope}%
\begin{pgfscope}%
\pgfpathrectangle{\pgfqpoint{0.800000in}{5.105882in}}{\pgfqpoint{2.407767in}{1.544118in}}%
\pgfusepath{clip}%
\pgfsetbuttcap%
\pgfsetroundjoin%
\pgfsetlinewidth{0.501875pt}%
\definecolor{currentstroke}{rgb}{0.255645,0.260703,0.528312}%
\pgfsetstrokecolor{currentstroke}%
\pgfsetdash{}{0pt}%
\pgfpathmoveto{\pgfqpoint{2.025196in}{5.683975in}}%
\pgfpathlineto{\pgfqpoint{1.975521in}{5.695681in}}%
\pgfusepath{stroke}%
\end{pgfscope}%
\begin{pgfscope}%
\pgfpathrectangle{\pgfqpoint{0.800000in}{5.105882in}}{\pgfqpoint{2.407767in}{1.544118in}}%
\pgfusepath{clip}%
\pgfsetbuttcap%
\pgfsetroundjoin%
\pgfsetlinewidth{0.501875pt}%
\definecolor{currentstroke}{rgb}{0.273006,0.204520,0.501721}%
\pgfsetstrokecolor{currentstroke}%
\pgfsetdash{}{0pt}%
\pgfpathmoveto{\pgfqpoint{1.975521in}{5.695681in}}%
\pgfpathlineto{\pgfqpoint{1.926549in}{5.708609in}}%
\pgfusepath{stroke}%
\end{pgfscope}%
\begin{pgfscope}%
\pgfpathrectangle{\pgfqpoint{0.800000in}{5.105882in}}{\pgfqpoint{2.407767in}{1.544118in}}%
\pgfusepath{clip}%
\pgfsetbuttcap%
\pgfsetroundjoin%
\pgfsetlinewidth{0.501875pt}%
\definecolor{currentstroke}{rgb}{0.260571,0.246922,0.522828}%
\pgfsetstrokecolor{currentstroke}%
\pgfsetdash{}{0pt}%
\pgfpathmoveto{\pgfqpoint{1.926549in}{5.708609in}}%
\pgfpathlineto{\pgfqpoint{1.878545in}{5.722874in}}%
\pgfusepath{stroke}%
\end{pgfscope}%
\begin{pgfscope}%
\pgfpathrectangle{\pgfqpoint{0.800000in}{5.105882in}}{\pgfqpoint{2.407767in}{1.544118in}}%
\pgfusepath{clip}%
\pgfsetbuttcap%
\pgfsetroundjoin%
\pgfsetlinewidth{0.501875pt}%
\definecolor{currentstroke}{rgb}{0.258965,0.251537,0.524736}%
\pgfsetstrokecolor{currentstroke}%
\pgfsetdash{}{0pt}%
\pgfpathmoveto{\pgfqpoint{1.878545in}{5.722874in}}%
\pgfpathlineto{\pgfqpoint{1.831765in}{5.738765in}}%
\pgfusepath{stroke}%
\end{pgfscope}%
\begin{pgfscope}%
\pgfpathrectangle{\pgfqpoint{0.800000in}{5.105882in}}{\pgfqpoint{2.407767in}{1.544118in}}%
\pgfusepath{clip}%
\pgfsetbuttcap%
\pgfsetroundjoin%
\pgfsetlinewidth{0.501875pt}%
\definecolor{currentstroke}{rgb}{0.273809,0.031497,0.358853}%
\pgfsetstrokecolor{currentstroke}%
\pgfsetdash{}{0pt}%
\pgfpathmoveto{\pgfqpoint{2.654046in}{5.669464in}}%
\pgfpathlineto{\pgfqpoint{2.601108in}{5.670637in}}%
\pgfusepath{stroke}%
\end{pgfscope}%
\begin{pgfscope}%
\pgfpathrectangle{\pgfqpoint{0.800000in}{5.105882in}}{\pgfqpoint{2.407767in}{1.544118in}}%
\pgfusepath{clip}%
\pgfsetbuttcap%
\pgfsetroundjoin%
\pgfsetlinewidth{0.501875pt}%
\definecolor{currentstroke}{rgb}{0.280267,0.073417,0.397163}%
\pgfsetstrokecolor{currentstroke}%
\pgfsetdash{}{0pt}%
\pgfpathmoveto{\pgfqpoint{2.601108in}{5.670637in}}%
\pgfpathlineto{\pgfqpoint{2.548178in}{5.672004in}}%
\pgfusepath{stroke}%
\end{pgfscope}%
\begin{pgfscope}%
\pgfpathrectangle{\pgfqpoint{0.800000in}{5.105882in}}{\pgfqpoint{2.407767in}{1.544118in}}%
\pgfusepath{clip}%
\pgfsetbuttcap%
\pgfsetroundjoin%
\pgfsetlinewidth{0.501875pt}%
\definecolor{currentstroke}{rgb}{0.283072,0.130895,0.449241}%
\pgfsetstrokecolor{currentstroke}%
\pgfsetdash{}{0pt}%
\pgfpathmoveto{\pgfqpoint{2.548178in}{5.672004in}}%
\pgfpathlineto{\pgfqpoint{2.495267in}{5.673681in}}%
\pgfusepath{stroke}%
\end{pgfscope}%
\begin{pgfscope}%
\pgfpathrectangle{\pgfqpoint{0.800000in}{5.105882in}}{\pgfqpoint{2.407767in}{1.544118in}}%
\pgfusepath{clip}%
\pgfsetbuttcap%
\pgfsetroundjoin%
\pgfsetlinewidth{0.501875pt}%
\definecolor{currentstroke}{rgb}{0.279574,0.170599,0.479997}%
\pgfsetstrokecolor{currentstroke}%
\pgfsetdash{}{0pt}%
\pgfpathmoveto{\pgfqpoint{2.495267in}{5.673681in}}%
\pgfpathlineto{\pgfqpoint{2.442365in}{5.675472in}}%
\pgfusepath{stroke}%
\end{pgfscope}%
\begin{pgfscope}%
\pgfpathrectangle{\pgfqpoint{0.800000in}{5.105882in}}{\pgfqpoint{2.407767in}{1.544118in}}%
\pgfusepath{clip}%
\pgfsetbuttcap%
\pgfsetroundjoin%
\pgfsetlinewidth{0.501875pt}%
\definecolor{currentstroke}{rgb}{0.269308,0.218818,0.509577}%
\pgfsetstrokecolor{currentstroke}%
\pgfsetdash{}{0pt}%
\pgfpathmoveto{\pgfqpoint{2.442365in}{5.675472in}}%
\pgfpathlineto{\pgfqpoint{2.389482in}{5.677481in}}%
\pgfusepath{stroke}%
\end{pgfscope}%
\begin{pgfscope}%
\pgfpathrectangle{\pgfqpoint{0.800000in}{5.105882in}}{\pgfqpoint{2.407767in}{1.544118in}}%
\pgfusepath{clip}%
\pgfsetbuttcap%
\pgfsetroundjoin%
\pgfsetlinewidth{0.501875pt}%
\definecolor{currentstroke}{rgb}{0.265145,0.232956,0.516599}%
\pgfsetstrokecolor{currentstroke}%
\pgfsetdash{}{0pt}%
\pgfpathmoveto{\pgfqpoint{2.389482in}{5.677481in}}%
\pgfpathlineto{\pgfqpoint{2.336640in}{5.679872in}}%
\pgfusepath{stroke}%
\end{pgfscope}%
\begin{pgfscope}%
\pgfpathrectangle{\pgfqpoint{0.800000in}{5.105882in}}{\pgfqpoint{2.407767in}{1.544118in}}%
\pgfusepath{clip}%
\pgfsetbuttcap%
\pgfsetroundjoin%
\pgfsetlinewidth{0.501875pt}%
\definecolor{currentstroke}{rgb}{0.260571,0.246922,0.522828}%
\pgfsetstrokecolor{currentstroke}%
\pgfsetdash{}{0pt}%
\pgfpathmoveto{\pgfqpoint{2.336640in}{5.679872in}}%
\pgfpathlineto{\pgfqpoint{2.283883in}{5.682936in}}%
\pgfusepath{stroke}%
\end{pgfscope}%
\begin{pgfscope}%
\pgfpathrectangle{\pgfqpoint{0.800000in}{5.105882in}}{\pgfqpoint{2.407767in}{1.544118in}}%
\pgfusepath{clip}%
\pgfsetbuttcap%
\pgfsetroundjoin%
\pgfsetlinewidth{0.501875pt}%
\definecolor{currentstroke}{rgb}{0.248629,0.278775,0.534556}%
\pgfsetstrokecolor{currentstroke}%
\pgfsetdash{}{0pt}%
\pgfpathmoveto{\pgfqpoint{2.283883in}{5.682936in}}%
\pgfpathlineto{\pgfqpoint{2.231280in}{5.686912in}}%
\pgfusepath{stroke}%
\end{pgfscope}%
\begin{pgfscope}%
\pgfpathrectangle{\pgfqpoint{0.800000in}{5.105882in}}{\pgfqpoint{2.407767in}{1.544118in}}%
\pgfusepath{clip}%
\pgfsetbuttcap%
\pgfsetroundjoin%
\pgfsetlinewidth{0.501875pt}%
\definecolor{currentstroke}{rgb}{0.252194,0.269783,0.531579}%
\pgfsetstrokecolor{currentstroke}%
\pgfsetdash{}{0pt}%
\pgfpathmoveto{\pgfqpoint{2.231280in}{5.686912in}}%
\pgfpathlineto{\pgfqpoint{2.178888in}{5.691909in}}%
\pgfusepath{stroke}%
\end{pgfscope}%
\begin{pgfscope}%
\pgfpathrectangle{\pgfqpoint{0.800000in}{5.105882in}}{\pgfqpoint{2.407767in}{1.544118in}}%
\pgfusepath{clip}%
\pgfsetbuttcap%
\pgfsetroundjoin%
\pgfsetlinewidth{0.501875pt}%
\definecolor{currentstroke}{rgb}{0.255645,0.260703,0.528312}%
\pgfsetstrokecolor{currentstroke}%
\pgfsetdash{}{0pt}%
\pgfpathmoveto{\pgfqpoint{2.178888in}{5.691909in}}%
\pgfpathlineto{\pgfqpoint{2.126701in}{5.697728in}}%
\pgfusepath{stroke}%
\end{pgfscope}%
\begin{pgfscope}%
\pgfpathrectangle{\pgfqpoint{0.800000in}{5.105882in}}{\pgfqpoint{2.407767in}{1.544118in}}%
\pgfusepath{clip}%
\pgfsetbuttcap%
\pgfsetroundjoin%
\pgfsetlinewidth{0.501875pt}%
\definecolor{currentstroke}{rgb}{0.252194,0.269783,0.531579}%
\pgfsetstrokecolor{currentstroke}%
\pgfsetdash{}{0pt}%
\pgfpathmoveto{\pgfqpoint{2.126701in}{5.697728in}}%
\pgfpathlineto{\pgfqpoint{2.074850in}{5.704643in}}%
\pgfusepath{stroke}%
\end{pgfscope}%
\begin{pgfscope}%
\pgfpathrectangle{\pgfqpoint{0.800000in}{5.105882in}}{\pgfqpoint{2.407767in}{1.544118in}}%
\pgfusepath{clip}%
\pgfsetbuttcap%
\pgfsetroundjoin%
\pgfsetlinewidth{0.501875pt}%
\definecolor{currentstroke}{rgb}{0.253935,0.265254,0.529983}%
\pgfsetstrokecolor{currentstroke}%
\pgfsetdash{}{0pt}%
\pgfpathmoveto{\pgfqpoint{2.074850in}{5.704643in}}%
\pgfpathlineto{\pgfqpoint{2.023498in}{5.712947in}}%
\pgfusepath{stroke}%
\end{pgfscope}%
\begin{pgfscope}%
\pgfpathrectangle{\pgfqpoint{0.800000in}{5.105882in}}{\pgfqpoint{2.407767in}{1.544118in}}%
\pgfusepath{clip}%
\pgfsetbuttcap%
\pgfsetroundjoin%
\pgfsetlinewidth{0.501875pt}%
\definecolor{currentstroke}{rgb}{0.250425,0.274290,0.533103}%
\pgfsetstrokecolor{currentstroke}%
\pgfsetdash{}{0pt}%
\pgfpathmoveto{\pgfqpoint{2.023498in}{5.712947in}}%
\pgfpathlineto{\pgfqpoint{1.972650in}{5.722432in}}%
\pgfusepath{stroke}%
\end{pgfscope}%
\begin{pgfscope}%
\pgfpathrectangle{\pgfqpoint{0.800000in}{5.105882in}}{\pgfqpoint{2.407767in}{1.544118in}}%
\pgfusepath{clip}%
\pgfsetbuttcap%
\pgfsetroundjoin%
\pgfsetlinewidth{0.501875pt}%
\definecolor{currentstroke}{rgb}{0.274952,0.037752,0.364543}%
\pgfsetstrokecolor{currentstroke}%
\pgfsetdash{}{0pt}%
\pgfpathmoveto{\pgfqpoint{2.654046in}{5.704211in}}%
\pgfpathlineto{\pgfqpoint{2.601084in}{5.704950in}}%
\pgfusepath{stroke}%
\end{pgfscope}%
\begin{pgfscope}%
\pgfpathrectangle{\pgfqpoint{0.800000in}{5.105882in}}{\pgfqpoint{2.407767in}{1.544118in}}%
\pgfusepath{clip}%
\pgfsetbuttcap%
\pgfsetroundjoin%
\pgfsetlinewidth{0.501875pt}%
\definecolor{currentstroke}{rgb}{0.281924,0.089666,0.412415}%
\pgfsetstrokecolor{currentstroke}%
\pgfsetdash{}{0pt}%
\pgfpathmoveto{\pgfqpoint{2.601084in}{5.704950in}}%
\pgfpathlineto{\pgfqpoint{2.548132in}{5.705939in}}%
\pgfusepath{stroke}%
\end{pgfscope}%
\begin{pgfscope}%
\pgfpathrectangle{\pgfqpoint{0.800000in}{5.105882in}}{\pgfqpoint{2.407767in}{1.544118in}}%
\pgfusepath{clip}%
\pgfsetbuttcap%
\pgfsetroundjoin%
\pgfsetlinewidth{0.501875pt}%
\definecolor{currentstroke}{rgb}{0.282290,0.145912,0.461510}%
\pgfsetstrokecolor{currentstroke}%
\pgfsetdash{}{0pt}%
\pgfpathmoveto{\pgfqpoint{2.548132in}{5.705939in}}%
\pgfpathlineto{\pgfqpoint{2.495188in}{5.707105in}}%
\pgfusepath{stroke}%
\end{pgfscope}%
\begin{pgfscope}%
\pgfpathrectangle{\pgfqpoint{0.800000in}{5.105882in}}{\pgfqpoint{2.407767in}{1.544118in}}%
\pgfusepath{clip}%
\pgfsetbuttcap%
\pgfsetroundjoin%
\pgfsetlinewidth{0.501875pt}%
\definecolor{currentstroke}{rgb}{0.274128,0.199721,0.498911}%
\pgfsetstrokecolor{currentstroke}%
\pgfsetdash{}{0pt}%
\pgfpathmoveto{\pgfqpoint{2.495188in}{5.707105in}}%
\pgfpathlineto{\pgfqpoint{2.442259in}{5.708521in}}%
\pgfusepath{stroke}%
\end{pgfscope}%
\begin{pgfscope}%
\pgfpathrectangle{\pgfqpoint{0.800000in}{5.105882in}}{\pgfqpoint{2.407767in}{1.544118in}}%
\pgfusepath{clip}%
\pgfsetbuttcap%
\pgfsetroundjoin%
\pgfsetlinewidth{0.501875pt}%
\definecolor{currentstroke}{rgb}{0.260571,0.246922,0.522828}%
\pgfsetstrokecolor{currentstroke}%
\pgfsetdash{}{0pt}%
\pgfpathmoveto{\pgfqpoint{2.442259in}{5.708521in}}%
\pgfpathlineto{\pgfqpoint{2.389346in}{5.710175in}}%
\pgfusepath{stroke}%
\end{pgfscope}%
\begin{pgfscope}%
\pgfpathrectangle{\pgfqpoint{0.800000in}{5.105882in}}{\pgfqpoint{2.407767in}{1.544118in}}%
\pgfusepath{clip}%
\pgfsetbuttcap%
\pgfsetroundjoin%
\pgfsetlinewidth{0.501875pt}%
\definecolor{currentstroke}{rgb}{0.257322,0.256130,0.526563}%
\pgfsetstrokecolor{currentstroke}%
\pgfsetdash{}{0pt}%
\pgfpathmoveto{\pgfqpoint{2.389346in}{5.710175in}}%
\pgfpathlineto{\pgfqpoint{2.336473in}{5.712261in}}%
\pgfusepath{stroke}%
\end{pgfscope}%
\begin{pgfscope}%
\pgfpathrectangle{\pgfqpoint{0.800000in}{5.105882in}}{\pgfqpoint{2.407767in}{1.544118in}}%
\pgfusepath{clip}%
\pgfsetbuttcap%
\pgfsetroundjoin%
\pgfsetlinewidth{0.501875pt}%
\definecolor{currentstroke}{rgb}{0.244972,0.287675,0.537260}%
\pgfsetstrokecolor{currentstroke}%
\pgfsetdash{}{0pt}%
\pgfpathmoveto{\pgfqpoint{2.336473in}{5.712261in}}%
\pgfpathlineto{\pgfqpoint{2.283658in}{5.714900in}}%
\pgfusepath{stroke}%
\end{pgfscope}%
\begin{pgfscope}%
\pgfpathrectangle{\pgfqpoint{0.800000in}{5.105882in}}{\pgfqpoint{2.407767in}{1.544118in}}%
\pgfusepath{clip}%
\pgfsetbuttcap%
\pgfsetroundjoin%
\pgfsetlinewidth{0.501875pt}%
\definecolor{currentstroke}{rgb}{0.274952,0.037752,0.364543}%
\pgfsetstrokecolor{currentstroke}%
\pgfsetdash{}{0pt}%
\pgfpathmoveto{\pgfqpoint{2.654046in}{5.738957in}}%
\pgfpathlineto{\pgfqpoint{2.601079in}{5.739523in}}%
\pgfusepath{stroke}%
\end{pgfscope}%
\begin{pgfscope}%
\pgfpathrectangle{\pgfqpoint{0.800000in}{5.105882in}}{\pgfqpoint{2.407767in}{1.544118in}}%
\pgfusepath{clip}%
\pgfsetbuttcap%
\pgfsetroundjoin%
\pgfsetlinewidth{0.501875pt}%
\definecolor{currentstroke}{rgb}{0.282327,0.094955,0.417331}%
\pgfsetstrokecolor{currentstroke}%
\pgfsetdash{}{0pt}%
\pgfpathmoveto{\pgfqpoint{2.601079in}{5.739523in}}%
\pgfpathlineto{\pgfqpoint{2.548115in}{5.740152in}}%
\pgfusepath{stroke}%
\end{pgfscope}%
\begin{pgfscope}%
\pgfpathrectangle{\pgfqpoint{0.800000in}{5.105882in}}{\pgfqpoint{2.407767in}{1.544118in}}%
\pgfusepath{clip}%
\pgfsetbuttcap%
\pgfsetroundjoin%
\pgfsetlinewidth{0.501875pt}%
\definecolor{currentstroke}{rgb}{0.281412,0.155834,0.469201}%
\pgfsetstrokecolor{currentstroke}%
\pgfsetdash{}{0pt}%
\pgfpathmoveto{\pgfqpoint{2.548115in}{5.740152in}}%
\pgfpathlineto{\pgfqpoint{2.495157in}{5.740910in}}%
\pgfusepath{stroke}%
\end{pgfscope}%
\begin{pgfscope}%
\pgfpathrectangle{\pgfqpoint{0.800000in}{5.105882in}}{\pgfqpoint{2.407767in}{1.544118in}}%
\pgfusepath{clip}%
\pgfsetbuttcap%
\pgfsetroundjoin%
\pgfsetlinewidth{0.501875pt}%
\definecolor{currentstroke}{rgb}{0.266580,0.228262,0.514349}%
\pgfsetstrokecolor{currentstroke}%
\pgfsetdash{}{0pt}%
\pgfpathmoveto{\pgfqpoint{2.495157in}{5.740910in}}%
\pgfpathlineto{\pgfqpoint{2.442217in}{5.742142in}}%
\pgfusepath{stroke}%
\end{pgfscope}%
\begin{pgfscope}%
\pgfpathrectangle{\pgfqpoint{0.800000in}{5.105882in}}{\pgfqpoint{2.407767in}{1.544118in}}%
\pgfusepath{clip}%
\pgfsetbuttcap%
\pgfsetroundjoin%
\pgfsetlinewidth{0.501875pt}%
\definecolor{currentstroke}{rgb}{0.257322,0.256130,0.526563}%
\pgfsetstrokecolor{currentstroke}%
\pgfsetdash{}{0pt}%
\pgfpathmoveto{\pgfqpoint{2.442217in}{5.742142in}}%
\pgfpathlineto{\pgfqpoint{2.389292in}{5.743628in}}%
\pgfusepath{stroke}%
\end{pgfscope}%
\begin{pgfscope}%
\pgfpathrectangle{\pgfqpoint{0.800000in}{5.105882in}}{\pgfqpoint{2.407767in}{1.544118in}}%
\pgfusepath{clip}%
\pgfsetbuttcap%
\pgfsetroundjoin%
\pgfsetlinewidth{0.501875pt}%
\definecolor{currentstroke}{rgb}{0.235526,0.309527,0.542944}%
\pgfsetstrokecolor{currentstroke}%
\pgfsetdash{}{0pt}%
\pgfpathmoveto{\pgfqpoint{2.389292in}{5.743628in}}%
\pgfpathlineto{\pgfqpoint{2.336380in}{5.745290in}}%
\pgfusepath{stroke}%
\end{pgfscope}%
\begin{pgfscope}%
\pgfpathrectangle{\pgfqpoint{0.800000in}{5.105882in}}{\pgfqpoint{2.407767in}{1.544118in}}%
\pgfusepath{clip}%
\pgfsetbuttcap%
\pgfsetroundjoin%
\pgfsetlinewidth{0.501875pt}%
\definecolor{currentstroke}{rgb}{0.227802,0.326594,0.546532}%
\pgfsetstrokecolor{currentstroke}%
\pgfsetdash{}{0pt}%
\pgfpathmoveto{\pgfqpoint{2.336380in}{5.745290in}}%
\pgfpathlineto{\pgfqpoint{2.283500in}{5.747308in}}%
\pgfusepath{stroke}%
\end{pgfscope}%
\begin{pgfscope}%
\pgfpathrectangle{\pgfqpoint{0.800000in}{5.105882in}}{\pgfqpoint{2.407767in}{1.544118in}}%
\pgfusepath{clip}%
\pgfsetbuttcap%
\pgfsetroundjoin%
\pgfsetlinewidth{0.501875pt}%
\definecolor{currentstroke}{rgb}{0.231674,0.318106,0.544834}%
\pgfsetstrokecolor{currentstroke}%
\pgfsetdash{}{0pt}%
\pgfpathmoveto{\pgfqpoint{2.283500in}{5.747308in}}%
\pgfpathlineto{\pgfqpoint{2.230693in}{5.749982in}}%
\pgfusepath{stroke}%
\end{pgfscope}%
\begin{pgfscope}%
\pgfpathrectangle{\pgfqpoint{0.800000in}{5.105882in}}{\pgfqpoint{2.407767in}{1.544118in}}%
\pgfusepath{clip}%
\pgfsetbuttcap%
\pgfsetroundjoin%
\pgfsetlinewidth{0.501875pt}%
\definecolor{currentstroke}{rgb}{0.220057,0.343307,0.549413}%
\pgfsetstrokecolor{currentstroke}%
\pgfsetdash{}{0pt}%
\pgfpathmoveto{\pgfqpoint{2.230693in}{5.749982in}}%
\pgfpathlineto{\pgfqpoint{2.177977in}{5.753313in}}%
\pgfusepath{stroke}%
\end{pgfscope}%
\begin{pgfscope}%
\pgfpathrectangle{\pgfqpoint{0.800000in}{5.105882in}}{\pgfqpoint{2.407767in}{1.544118in}}%
\pgfusepath{clip}%
\pgfsetbuttcap%
\pgfsetroundjoin%
\pgfsetlinewidth{0.501875pt}%
\definecolor{currentstroke}{rgb}{0.225863,0.330805,0.547314}%
\pgfsetstrokecolor{currentstroke}%
\pgfsetdash{}{0pt}%
\pgfpathmoveto{\pgfqpoint{2.177977in}{5.753313in}}%
\pgfpathlineto{\pgfqpoint{2.125361in}{5.757244in}}%
\pgfusepath{stroke}%
\end{pgfscope}%
\begin{pgfscope}%
\pgfpathrectangle{\pgfqpoint{0.800000in}{5.105882in}}{\pgfqpoint{2.407767in}{1.544118in}}%
\pgfusepath{clip}%
\pgfsetbuttcap%
\pgfsetroundjoin%
\pgfsetlinewidth{0.501875pt}%
\definecolor{currentstroke}{rgb}{0.235526,0.309527,0.542944}%
\pgfsetstrokecolor{currentstroke}%
\pgfsetdash{}{0pt}%
\pgfpathmoveto{\pgfqpoint{2.125361in}{5.757244in}}%
\pgfpathlineto{\pgfqpoint{2.072927in}{5.762047in}}%
\pgfusepath{stroke}%
\end{pgfscope}%
\begin{pgfscope}%
\pgfpathrectangle{\pgfqpoint{0.800000in}{5.105882in}}{\pgfqpoint{2.407767in}{1.544118in}}%
\pgfusepath{clip}%
\pgfsetbuttcap%
\pgfsetroundjoin%
\pgfsetlinewidth{0.501875pt}%
\definecolor{currentstroke}{rgb}{0.248629,0.278775,0.534556}%
\pgfsetstrokecolor{currentstroke}%
\pgfsetdash{}{0pt}%
\pgfpathmoveto{\pgfqpoint{2.072927in}{5.762047in}}%
\pgfpathlineto{\pgfqpoint{2.020690in}{5.767663in}}%
\pgfusepath{stroke}%
\end{pgfscope}%
\begin{pgfscope}%
\pgfpathrectangle{\pgfqpoint{0.800000in}{5.105882in}}{\pgfqpoint{2.407767in}{1.544118in}}%
\pgfusepath{clip}%
\pgfsetbuttcap%
\pgfsetroundjoin%
\pgfsetlinewidth{0.501875pt}%
\definecolor{currentstroke}{rgb}{0.244972,0.287675,0.537260}%
\pgfsetstrokecolor{currentstroke}%
\pgfsetdash{}{0pt}%
\pgfpathmoveto{\pgfqpoint{2.020690in}{5.767663in}}%
\pgfpathlineto{\pgfqpoint{1.968658in}{5.773997in}}%
\pgfusepath{stroke}%
\end{pgfscope}%
\begin{pgfscope}%
\pgfpathrectangle{\pgfqpoint{0.800000in}{5.105882in}}{\pgfqpoint{2.407767in}{1.544118in}}%
\pgfusepath{clip}%
\pgfsetbuttcap%
\pgfsetroundjoin%
\pgfsetlinewidth{0.501875pt}%
\definecolor{currentstroke}{rgb}{0.248629,0.278775,0.534556}%
\pgfsetstrokecolor{currentstroke}%
\pgfsetdash{}{0pt}%
\pgfpathmoveto{\pgfqpoint{1.968658in}{5.773997in}}%
\pgfpathlineto{\pgfqpoint{1.916877in}{5.781118in}}%
\pgfusepath{stroke}%
\end{pgfscope}%
\begin{pgfscope}%
\pgfpathrectangle{\pgfqpoint{0.800000in}{5.105882in}}{\pgfqpoint{2.407767in}{1.544118in}}%
\pgfusepath{clip}%
\pgfsetbuttcap%
\pgfsetroundjoin%
\pgfsetlinewidth{0.501875pt}%
\definecolor{currentstroke}{rgb}{0.258965,0.251537,0.524736}%
\pgfsetstrokecolor{currentstroke}%
\pgfsetdash{}{0pt}%
\pgfpathmoveto{\pgfqpoint{1.916877in}{5.781118in}}%
\pgfpathlineto{\pgfqpoint{1.865368in}{5.789006in}}%
\pgfusepath{stroke}%
\end{pgfscope}%
\begin{pgfscope}%
\pgfpathrectangle{\pgfqpoint{0.800000in}{5.105882in}}{\pgfqpoint{2.407767in}{1.544118in}}%
\pgfusepath{clip}%
\pgfsetbuttcap%
\pgfsetroundjoin%
\pgfsetlinewidth{0.501875pt}%
\definecolor{currentstroke}{rgb}{0.260571,0.246922,0.522828}%
\pgfsetstrokecolor{currentstroke}%
\pgfsetdash{}{0pt}%
\pgfpathmoveto{\pgfqpoint{1.865368in}{5.789006in}}%
\pgfpathlineto{\pgfqpoint{1.814143in}{5.797610in}}%
\pgfusepath{stroke}%
\end{pgfscope}%
\begin{pgfscope}%
\pgfpathrectangle{\pgfqpoint{0.800000in}{5.105882in}}{\pgfqpoint{2.407767in}{1.544118in}}%
\pgfusepath{clip}%
\pgfsetbuttcap%
\pgfsetroundjoin%
\pgfsetlinewidth{0.501875pt}%
\definecolor{currentstroke}{rgb}{0.276022,0.044167,0.370164}%
\pgfsetstrokecolor{currentstroke}%
\pgfsetdash{}{0pt}%
\pgfpathmoveto{\pgfqpoint{2.654046in}{5.773703in}}%
\pgfpathlineto{\pgfqpoint{2.601078in}{5.774275in}}%
\pgfusepath{stroke}%
\end{pgfscope}%
\begin{pgfscope}%
\pgfpathrectangle{\pgfqpoint{0.800000in}{5.105882in}}{\pgfqpoint{2.407767in}{1.544118in}}%
\pgfusepath{clip}%
\pgfsetbuttcap%
\pgfsetroundjoin%
\pgfsetlinewidth{0.501875pt}%
\definecolor{currentstroke}{rgb}{0.283197,0.115680,0.436115}%
\pgfsetstrokecolor{currentstroke}%
\pgfsetdash{}{0pt}%
\pgfpathmoveto{\pgfqpoint{2.601078in}{5.774275in}}%
\pgfpathlineto{\pgfqpoint{2.548113in}{5.774984in}}%
\pgfusepath{stroke}%
\end{pgfscope}%
\begin{pgfscope}%
\pgfpathrectangle{\pgfqpoint{0.800000in}{5.105882in}}{\pgfqpoint{2.407767in}{1.544118in}}%
\pgfusepath{clip}%
\pgfsetbuttcap%
\pgfsetroundjoin%
\pgfsetlinewidth{0.501875pt}%
\definecolor{currentstroke}{rgb}{0.276194,0.190074,0.493001}%
\pgfsetstrokecolor{currentstroke}%
\pgfsetdash{}{0pt}%
\pgfpathmoveto{\pgfqpoint{2.548113in}{5.774984in}}%
\pgfpathlineto{\pgfqpoint{2.495151in}{5.775758in}}%
\pgfusepath{stroke}%
\end{pgfscope}%
\begin{pgfscope}%
\pgfpathrectangle{\pgfqpoint{0.800000in}{5.105882in}}{\pgfqpoint{2.407767in}{1.544118in}}%
\pgfusepath{clip}%
\pgfsetbuttcap%
\pgfsetroundjoin%
\pgfsetlinewidth{0.501875pt}%
\definecolor{currentstroke}{rgb}{0.262138,0.242286,0.520837}%
\pgfsetstrokecolor{currentstroke}%
\pgfsetdash{}{0pt}%
\pgfpathmoveto{\pgfqpoint{2.495151in}{5.775758in}}%
\pgfpathlineto{\pgfqpoint{2.442192in}{5.776616in}}%
\pgfusepath{stroke}%
\end{pgfscope}%
\begin{pgfscope}%
\pgfpathrectangle{\pgfqpoint{0.800000in}{5.105882in}}{\pgfqpoint{2.407767in}{1.544118in}}%
\pgfusepath{clip}%
\pgfsetbuttcap%
\pgfsetroundjoin%
\pgfsetlinewidth{0.501875pt}%
\definecolor{currentstroke}{rgb}{0.243113,0.292092,0.538516}%
\pgfsetstrokecolor{currentstroke}%
\pgfsetdash{}{0pt}%
\pgfpathmoveto{\pgfqpoint{2.442192in}{5.776616in}}%
\pgfpathlineto{\pgfqpoint{2.389243in}{5.777686in}}%
\pgfusepath{stroke}%
\end{pgfscope}%
\begin{pgfscope}%
\pgfpathrectangle{\pgfqpoint{0.800000in}{5.105882in}}{\pgfqpoint{2.407767in}{1.544118in}}%
\pgfusepath{clip}%
\pgfsetbuttcap%
\pgfsetroundjoin%
\pgfsetlinewidth{0.501875pt}%
\definecolor{currentstroke}{rgb}{0.225863,0.330805,0.547314}%
\pgfsetstrokecolor{currentstroke}%
\pgfsetdash{}{0pt}%
\pgfpathmoveto{\pgfqpoint{2.389243in}{5.777686in}}%
\pgfpathlineto{\pgfqpoint{2.336311in}{5.779063in}}%
\pgfusepath{stroke}%
\end{pgfscope}%
\begin{pgfscope}%
\pgfpathrectangle{\pgfqpoint{0.800000in}{5.105882in}}{\pgfqpoint{2.407767in}{1.544118in}}%
\pgfusepath{clip}%
\pgfsetbuttcap%
\pgfsetroundjoin%
\pgfsetlinewidth{0.501875pt}%
\definecolor{currentstroke}{rgb}{0.221989,0.339161,0.548752}%
\pgfsetstrokecolor{currentstroke}%
\pgfsetdash{}{0pt}%
\pgfpathmoveto{\pgfqpoint{2.336311in}{5.779063in}}%
\pgfpathlineto{\pgfqpoint{2.283396in}{5.780683in}}%
\pgfusepath{stroke}%
\end{pgfscope}%
\begin{pgfscope}%
\pgfpathrectangle{\pgfqpoint{0.800000in}{5.105882in}}{\pgfqpoint{2.407767in}{1.544118in}}%
\pgfusepath{clip}%
\pgfsetbuttcap%
\pgfsetroundjoin%
\pgfsetlinewidth{0.501875pt}%
\definecolor{currentstroke}{rgb}{0.223925,0.334994,0.548053}%
\pgfsetstrokecolor{currentstroke}%
\pgfsetdash{}{0pt}%
\pgfpathmoveto{\pgfqpoint{2.283396in}{5.780683in}}%
\pgfpathlineto{\pgfqpoint{2.230503in}{5.782571in}}%
\pgfusepath{stroke}%
\end{pgfscope}%
\begin{pgfscope}%
\pgfpathrectangle{\pgfqpoint{0.800000in}{5.105882in}}{\pgfqpoint{2.407767in}{1.544118in}}%
\pgfusepath{clip}%
\pgfsetbuttcap%
\pgfsetroundjoin%
\pgfsetlinewidth{0.501875pt}%
\definecolor{currentstroke}{rgb}{0.218130,0.347432,0.550038}%
\pgfsetstrokecolor{currentstroke}%
\pgfsetdash{}{0pt}%
\pgfpathmoveto{\pgfqpoint{2.230503in}{5.782571in}}%
\pgfpathlineto{\pgfqpoint{2.177657in}{5.784926in}}%
\pgfusepath{stroke}%
\end{pgfscope}%
\begin{pgfscope}%
\pgfpathrectangle{\pgfqpoint{0.800000in}{5.105882in}}{\pgfqpoint{2.407767in}{1.544118in}}%
\pgfusepath{clip}%
\pgfsetbuttcap%
\pgfsetroundjoin%
\pgfsetlinewidth{0.501875pt}%
\definecolor{currentstroke}{rgb}{0.276022,0.044167,0.370164}%
\pgfsetstrokecolor{currentstroke}%
\pgfsetdash{}{0pt}%
\pgfpathmoveto{\pgfqpoint{2.654046in}{5.808449in}}%
\pgfpathlineto{\pgfqpoint{2.601074in}{5.808772in}}%
\pgfusepath{stroke}%
\end{pgfscope}%
\begin{pgfscope}%
\pgfpathrectangle{\pgfqpoint{0.800000in}{5.105882in}}{\pgfqpoint{2.407767in}{1.544118in}}%
\pgfusepath{clip}%
\pgfsetbuttcap%
\pgfsetroundjoin%
\pgfsetlinewidth{0.501875pt}%
\definecolor{currentstroke}{rgb}{0.283187,0.125848,0.444960}%
\pgfsetstrokecolor{currentstroke}%
\pgfsetdash{}{0pt}%
\pgfpathmoveto{\pgfqpoint{2.601074in}{5.808772in}}%
\pgfpathlineto{\pgfqpoint{2.548104in}{5.809251in}}%
\pgfusepath{stroke}%
\end{pgfscope}%
\begin{pgfscope}%
\pgfpathrectangle{\pgfqpoint{0.800000in}{5.105882in}}{\pgfqpoint{2.407767in}{1.544118in}}%
\pgfusepath{clip}%
\pgfsetbuttcap%
\pgfsetroundjoin%
\pgfsetlinewidth{0.501875pt}%
\definecolor{currentstroke}{rgb}{0.273006,0.204520,0.501721}%
\pgfsetstrokecolor{currentstroke}%
\pgfsetdash{}{0pt}%
\pgfpathmoveto{\pgfqpoint{2.548104in}{5.809251in}}%
\pgfpathlineto{\pgfqpoint{2.495135in}{5.809810in}}%
\pgfusepath{stroke}%
\end{pgfscope}%
\begin{pgfscope}%
\pgfpathrectangle{\pgfqpoint{0.800000in}{5.105882in}}{\pgfqpoint{2.407767in}{1.544118in}}%
\pgfusepath{clip}%
\pgfsetbuttcap%
\pgfsetroundjoin%
\pgfsetlinewidth{0.501875pt}%
\definecolor{currentstroke}{rgb}{0.252194,0.269783,0.531579}%
\pgfsetstrokecolor{currentstroke}%
\pgfsetdash{}{0pt}%
\pgfpathmoveto{\pgfqpoint{2.495135in}{5.809810in}}%
\pgfpathlineto{\pgfqpoint{2.442168in}{5.810418in}}%
\pgfusepath{stroke}%
\end{pgfscope}%
\begin{pgfscope}%
\pgfpathrectangle{\pgfqpoint{0.800000in}{5.105882in}}{\pgfqpoint{2.407767in}{1.544118in}}%
\pgfusepath{clip}%
\pgfsetbuttcap%
\pgfsetroundjoin%
\pgfsetlinewidth{0.501875pt}%
\definecolor{currentstroke}{rgb}{0.239346,0.300855,0.540844}%
\pgfsetstrokecolor{currentstroke}%
\pgfsetdash{}{0pt}%
\pgfpathmoveto{\pgfqpoint{2.442168in}{5.810418in}}%
\pgfpathlineto{\pgfqpoint{2.389205in}{5.811179in}}%
\pgfusepath{stroke}%
\end{pgfscope}%
\begin{pgfscope}%
\pgfpathrectangle{\pgfqpoint{0.800000in}{5.105882in}}{\pgfqpoint{2.407767in}{1.544118in}}%
\pgfusepath{clip}%
\pgfsetbuttcap%
\pgfsetroundjoin%
\pgfsetlinewidth{0.501875pt}%
\definecolor{currentstroke}{rgb}{0.223925,0.334994,0.548053}%
\pgfsetstrokecolor{currentstroke}%
\pgfsetdash{}{0pt}%
\pgfpathmoveto{\pgfqpoint{2.389205in}{5.811179in}}%
\pgfpathlineto{\pgfqpoint{2.336251in}{5.812137in}}%
\pgfusepath{stroke}%
\end{pgfscope}%
\begin{pgfscope}%
\pgfpathrectangle{\pgfqpoint{0.800000in}{5.105882in}}{\pgfqpoint{2.407767in}{1.544118in}}%
\pgfusepath{clip}%
\pgfsetbuttcap%
\pgfsetroundjoin%
\pgfsetlinewidth{0.501875pt}%
\definecolor{currentstroke}{rgb}{0.208623,0.367752,0.552675}%
\pgfsetstrokecolor{currentstroke}%
\pgfsetdash{}{0pt}%
\pgfpathmoveto{\pgfqpoint{2.336251in}{5.812137in}}%
\pgfpathlineto{\pgfqpoint{2.283305in}{5.813291in}}%
\pgfusepath{stroke}%
\end{pgfscope}%
\begin{pgfscope}%
\pgfpathrectangle{\pgfqpoint{0.800000in}{5.105882in}}{\pgfqpoint{2.407767in}{1.544118in}}%
\pgfusepath{clip}%
\pgfsetbuttcap%
\pgfsetroundjoin%
\pgfsetlinewidth{0.501875pt}%
\definecolor{currentstroke}{rgb}{0.210503,0.363727,0.552206}%
\pgfsetstrokecolor{currentstroke}%
\pgfsetdash{}{0pt}%
\pgfpathmoveto{\pgfqpoint{2.283305in}{5.813291in}}%
\pgfpathlineto{\pgfqpoint{2.230374in}{5.814670in}}%
\pgfusepath{stroke}%
\end{pgfscope}%
\begin{pgfscope}%
\pgfpathrectangle{\pgfqpoint{0.800000in}{5.105882in}}{\pgfqpoint{2.407767in}{1.544118in}}%
\pgfusepath{clip}%
\pgfsetbuttcap%
\pgfsetroundjoin%
\pgfsetlinewidth{0.501875pt}%
\definecolor{currentstroke}{rgb}{0.221989,0.339161,0.548752}%
\pgfsetstrokecolor{currentstroke}%
\pgfsetdash{}{0pt}%
\pgfpathmoveto{\pgfqpoint{2.230374in}{5.814670in}}%
\pgfpathlineto{\pgfqpoint{2.177462in}{5.816338in}}%
\pgfusepath{stroke}%
\end{pgfscope}%
\begin{pgfscope}%
\pgfpathrectangle{\pgfqpoint{0.800000in}{5.105882in}}{\pgfqpoint{2.407767in}{1.544118in}}%
\pgfusepath{clip}%
\pgfsetbuttcap%
\pgfsetroundjoin%
\pgfsetlinewidth{0.501875pt}%
\definecolor{currentstroke}{rgb}{0.210503,0.363727,0.552206}%
\pgfsetstrokecolor{currentstroke}%
\pgfsetdash{}{0pt}%
\pgfpathmoveto{\pgfqpoint{2.177462in}{5.816338in}}%
\pgfpathlineto{\pgfqpoint{2.124572in}{5.818267in}}%
\pgfusepath{stroke}%
\end{pgfscope}%
\begin{pgfscope}%
\pgfpathrectangle{\pgfqpoint{0.800000in}{5.105882in}}{\pgfqpoint{2.407767in}{1.544118in}}%
\pgfusepath{clip}%
\pgfsetbuttcap%
\pgfsetroundjoin%
\pgfsetlinewidth{0.501875pt}%
\definecolor{currentstroke}{rgb}{0.223925,0.334994,0.548053}%
\pgfsetstrokecolor{currentstroke}%
\pgfsetdash{}{0pt}%
\pgfpathmoveto{\pgfqpoint{2.124572in}{5.818267in}}%
\pgfpathlineto{\pgfqpoint{2.071726in}{5.820618in}}%
\pgfusepath{stroke}%
\end{pgfscope}%
\begin{pgfscope}%
\pgfpathrectangle{\pgfqpoint{0.800000in}{5.105882in}}{\pgfqpoint{2.407767in}{1.544118in}}%
\pgfusepath{clip}%
\pgfsetbuttcap%
\pgfsetroundjoin%
\pgfsetlinewidth{0.501875pt}%
\definecolor{currentstroke}{rgb}{0.227802,0.326594,0.546532}%
\pgfsetstrokecolor{currentstroke}%
\pgfsetdash{}{0pt}%
\pgfpathmoveto{\pgfqpoint{2.071726in}{5.820618in}}%
\pgfpathlineto{\pgfqpoint{2.018946in}{5.823503in}}%
\pgfusepath{stroke}%
\end{pgfscope}%
\begin{pgfscope}%
\pgfpathrectangle{\pgfqpoint{0.800000in}{5.105882in}}{\pgfqpoint{2.407767in}{1.544118in}}%
\pgfusepath{clip}%
\pgfsetbuttcap%
\pgfsetroundjoin%
\pgfsetlinewidth{0.501875pt}%
\definecolor{currentstroke}{rgb}{0.231674,0.318106,0.544834}%
\pgfsetstrokecolor{currentstroke}%
\pgfsetdash{}{0pt}%
\pgfpathmoveto{\pgfqpoint{2.018946in}{5.823503in}}%
\pgfpathlineto{\pgfqpoint{1.966232in}{5.826845in}}%
\pgfusepath{stroke}%
\end{pgfscope}%
\begin{pgfscope}%
\pgfpathrectangle{\pgfqpoint{0.800000in}{5.105882in}}{\pgfqpoint{2.407767in}{1.544118in}}%
\pgfusepath{clip}%
\pgfsetbuttcap%
\pgfsetroundjoin%
\pgfsetlinewidth{0.501875pt}%
\definecolor{currentstroke}{rgb}{0.248629,0.278775,0.534556}%
\pgfsetstrokecolor{currentstroke}%
\pgfsetdash{}{0pt}%
\pgfpathmoveto{\pgfqpoint{1.966232in}{5.826845in}}%
\pgfpathlineto{\pgfqpoint{1.913592in}{5.830635in}}%
\pgfusepath{stroke}%
\end{pgfscope}%
\begin{pgfscope}%
\pgfpathrectangle{\pgfqpoint{0.800000in}{5.105882in}}{\pgfqpoint{2.407767in}{1.544118in}}%
\pgfusepath{clip}%
\pgfsetbuttcap%
\pgfsetroundjoin%
\pgfsetlinewidth{0.501875pt}%
\definecolor{currentstroke}{rgb}{0.260571,0.246922,0.522828}%
\pgfsetstrokecolor{currentstroke}%
\pgfsetdash{}{0pt}%
\pgfpathmoveto{\pgfqpoint{1.913592in}{5.830635in}}%
\pgfpathlineto{\pgfqpoint{1.861008in}{5.834732in}}%
\pgfusepath{stroke}%
\end{pgfscope}%
\begin{pgfscope}%
\pgfpathrectangle{\pgfqpoint{0.800000in}{5.105882in}}{\pgfqpoint{2.407767in}{1.544118in}}%
\pgfusepath{clip}%
\pgfsetbuttcap%
\pgfsetroundjoin%
\pgfsetlinewidth{0.501875pt}%
\definecolor{currentstroke}{rgb}{0.267968,0.223549,0.512008}%
\pgfsetstrokecolor{currentstroke}%
\pgfsetdash{}{0pt}%
\pgfpathmoveto{\pgfqpoint{1.861008in}{5.834732in}}%
\pgfpathlineto{\pgfqpoint{1.808506in}{5.839175in}}%
\pgfusepath{stroke}%
\end{pgfscope}%
\begin{pgfscope}%
\pgfpathrectangle{\pgfqpoint{0.800000in}{5.105882in}}{\pgfqpoint{2.407767in}{1.544118in}}%
\pgfusepath{clip}%
\pgfsetbuttcap%
\pgfsetroundjoin%
\pgfsetlinewidth{0.501875pt}%
\definecolor{currentstroke}{rgb}{0.276022,0.044167,0.370164}%
\pgfsetstrokecolor{currentstroke}%
\pgfsetdash{}{0pt}%
\pgfpathmoveto{\pgfqpoint{2.654046in}{5.843195in}}%
\pgfpathlineto{\pgfqpoint{2.601073in}{5.843467in}}%
\pgfusepath{stroke}%
\end{pgfscope}%
\begin{pgfscope}%
\pgfpathrectangle{\pgfqpoint{0.800000in}{5.105882in}}{\pgfqpoint{2.407767in}{1.544118in}}%
\pgfusepath{clip}%
\pgfsetbuttcap%
\pgfsetroundjoin%
\pgfsetlinewidth{0.501875pt}%
\definecolor{currentstroke}{rgb}{0.283229,0.120777,0.440584}%
\pgfsetstrokecolor{currentstroke}%
\pgfsetdash{}{0pt}%
\pgfpathmoveto{\pgfqpoint{2.601073in}{5.843467in}}%
\pgfpathlineto{\pgfqpoint{2.548099in}{5.843760in}}%
\pgfusepath{stroke}%
\end{pgfscope}%
\begin{pgfscope}%
\pgfpathrectangle{\pgfqpoint{0.800000in}{5.105882in}}{\pgfqpoint{2.407767in}{1.544118in}}%
\pgfusepath{clip}%
\pgfsetbuttcap%
\pgfsetroundjoin%
\pgfsetlinewidth{0.501875pt}%
\definecolor{currentstroke}{rgb}{0.271828,0.209303,0.504434}%
\pgfsetstrokecolor{currentstroke}%
\pgfsetdash{}{0pt}%
\pgfpathmoveto{\pgfqpoint{2.548099in}{5.843760in}}%
\pgfpathlineto{\pgfqpoint{2.495125in}{5.844082in}}%
\pgfusepath{stroke}%
\end{pgfscope}%
\begin{pgfscope}%
\pgfpathrectangle{\pgfqpoint{0.800000in}{5.105882in}}{\pgfqpoint{2.407767in}{1.544118in}}%
\pgfusepath{clip}%
\pgfsetbuttcap%
\pgfsetroundjoin%
\pgfsetlinewidth{0.501875pt}%
\definecolor{currentstroke}{rgb}{0.248629,0.278775,0.534556}%
\pgfsetstrokecolor{currentstroke}%
\pgfsetdash{}{0pt}%
\pgfpathmoveto{\pgfqpoint{2.495125in}{5.844082in}}%
\pgfpathlineto{\pgfqpoint{2.442153in}{5.844475in}}%
\pgfusepath{stroke}%
\end{pgfscope}%
\begin{pgfscope}%
\pgfpathrectangle{\pgfqpoint{0.800000in}{5.105882in}}{\pgfqpoint{2.407767in}{1.544118in}}%
\pgfusepath{clip}%
\pgfsetbuttcap%
\pgfsetroundjoin%
\pgfsetlinewidth{0.501875pt}%
\definecolor{currentstroke}{rgb}{0.233603,0.313828,0.543914}%
\pgfsetstrokecolor{currentstroke}%
\pgfsetdash{}{0pt}%
\pgfpathmoveto{\pgfqpoint{2.442153in}{5.844475in}}%
\pgfpathlineto{\pgfqpoint{2.389183in}{5.844985in}}%
\pgfusepath{stroke}%
\end{pgfscope}%
\begin{pgfscope}%
\pgfpathrectangle{\pgfqpoint{0.800000in}{5.105882in}}{\pgfqpoint{2.407767in}{1.544118in}}%
\pgfusepath{clip}%
\pgfsetbuttcap%
\pgfsetroundjoin%
\pgfsetlinewidth{0.501875pt}%
\definecolor{currentstroke}{rgb}{0.210503,0.363727,0.552206}%
\pgfsetstrokecolor{currentstroke}%
\pgfsetdash{}{0pt}%
\pgfpathmoveto{\pgfqpoint{2.389183in}{5.844985in}}%
\pgfpathlineto{\pgfqpoint{2.336215in}{5.845572in}}%
\pgfusepath{stroke}%
\end{pgfscope}%
\begin{pgfscope}%
\pgfpathrectangle{\pgfqpoint{0.800000in}{5.105882in}}{\pgfqpoint{2.407767in}{1.544118in}}%
\pgfusepath{clip}%
\pgfsetbuttcap%
\pgfsetroundjoin%
\pgfsetlinewidth{0.501875pt}%
\definecolor{currentstroke}{rgb}{0.216210,0.351535,0.550627}%
\pgfsetstrokecolor{currentstroke}%
\pgfsetdash{}{0pt}%
\pgfpathmoveto{\pgfqpoint{2.336215in}{5.845572in}}%
\pgfpathlineto{\pgfqpoint{2.283250in}{5.846241in}}%
\pgfusepath{stroke}%
\end{pgfscope}%
\begin{pgfscope}%
\pgfpathrectangle{\pgfqpoint{0.800000in}{5.105882in}}{\pgfqpoint{2.407767in}{1.544118in}}%
\pgfusepath{clip}%
\pgfsetbuttcap%
\pgfsetroundjoin%
\pgfsetlinewidth{0.501875pt}%
\definecolor{currentstroke}{rgb}{0.203063,0.379716,0.553925}%
\pgfsetstrokecolor{currentstroke}%
\pgfsetdash{}{0pt}%
\pgfpathmoveto{\pgfqpoint{2.283250in}{5.846241in}}%
\pgfpathlineto{\pgfqpoint{2.230285in}{5.846947in}}%
\pgfusepath{stroke}%
\end{pgfscope}%
\begin{pgfscope}%
\pgfpathrectangle{\pgfqpoint{0.800000in}{5.105882in}}{\pgfqpoint{2.407767in}{1.544118in}}%
\pgfusepath{clip}%
\pgfsetbuttcap%
\pgfsetroundjoin%
\pgfsetlinewidth{0.501875pt}%
\definecolor{currentstroke}{rgb}{0.208623,0.367752,0.552675}%
\pgfsetstrokecolor{currentstroke}%
\pgfsetdash{}{0pt}%
\pgfpathmoveto{\pgfqpoint{2.230285in}{5.846947in}}%
\pgfpathlineto{\pgfqpoint{2.177323in}{5.847718in}}%
\pgfusepath{stroke}%
\end{pgfscope}%
\begin{pgfscope}%
\pgfpathrectangle{\pgfqpoint{0.800000in}{5.105882in}}{\pgfqpoint{2.407767in}{1.544118in}}%
\pgfusepath{clip}%
\pgfsetbuttcap%
\pgfsetroundjoin%
\pgfsetlinewidth{0.501875pt}%
\definecolor{currentstroke}{rgb}{0.212395,0.359683,0.551710}%
\pgfsetstrokecolor{currentstroke}%
\pgfsetdash{}{0pt}%
\pgfpathmoveto{\pgfqpoint{2.177323in}{5.847718in}}%
\pgfpathlineto{\pgfqpoint{2.124361in}{5.848498in}}%
\pgfusepath{stroke}%
\end{pgfscope}%
\begin{pgfscope}%
\pgfpathrectangle{\pgfqpoint{0.800000in}{5.105882in}}{\pgfqpoint{2.407767in}{1.544118in}}%
\pgfusepath{clip}%
\pgfsetbuttcap%
\pgfsetroundjoin%
\pgfsetlinewidth{0.501875pt}%
\definecolor{currentstroke}{rgb}{0.231674,0.318106,0.544834}%
\pgfsetstrokecolor{currentstroke}%
\pgfsetdash{}{0pt}%
\pgfpathmoveto{\pgfqpoint{2.124361in}{5.848498in}}%
\pgfpathlineto{\pgfqpoint{2.071409in}{5.849471in}}%
\pgfusepath{stroke}%
\end{pgfscope}%
\begin{pgfscope}%
\pgfpathrectangle{\pgfqpoint{0.800000in}{5.105882in}}{\pgfqpoint{2.407767in}{1.544118in}}%
\pgfusepath{clip}%
\pgfsetbuttcap%
\pgfsetroundjoin%
\pgfsetlinewidth{0.501875pt}%
\definecolor{currentstroke}{rgb}{0.225863,0.330805,0.547314}%
\pgfsetstrokecolor{currentstroke}%
\pgfsetdash{}{0pt}%
\pgfpathmoveto{\pgfqpoint{2.071409in}{5.849471in}}%
\pgfpathlineto{\pgfqpoint{2.018481in}{5.850856in}}%
\pgfusepath{stroke}%
\end{pgfscope}%
\begin{pgfscope}%
\pgfpathrectangle{\pgfqpoint{0.800000in}{5.105882in}}{\pgfqpoint{2.407767in}{1.544118in}}%
\pgfusepath{clip}%
\pgfsetbuttcap%
\pgfsetroundjoin%
\pgfsetlinewidth{0.501875pt}%
\definecolor{currentstroke}{rgb}{0.276022,0.044167,0.370164}%
\pgfsetstrokecolor{currentstroke}%
\pgfsetdash{}{0pt}%
\pgfpathmoveto{\pgfqpoint{2.654046in}{5.877941in}}%
\pgfpathlineto{\pgfqpoint{2.601070in}{5.877820in}}%
\pgfusepath{stroke}%
\end{pgfscope}%
\begin{pgfscope}%
\pgfpathrectangle{\pgfqpoint{0.800000in}{5.105882in}}{\pgfqpoint{2.407767in}{1.544118in}}%
\pgfusepath{clip}%
\pgfsetbuttcap%
\pgfsetroundjoin%
\pgfsetlinewidth{0.501875pt}%
\definecolor{currentstroke}{rgb}{0.282884,0.135920,0.453427}%
\pgfsetstrokecolor{currentstroke}%
\pgfsetdash{}{0pt}%
\pgfpathmoveto{\pgfqpoint{2.601070in}{5.877820in}}%
\pgfpathlineto{\pgfqpoint{2.548094in}{5.877774in}}%
\pgfusepath{stroke}%
\end{pgfscope}%
\begin{pgfscope}%
\pgfpathrectangle{\pgfqpoint{0.800000in}{5.105882in}}{\pgfqpoint{2.407767in}{1.544118in}}%
\pgfusepath{clip}%
\pgfsetbuttcap%
\pgfsetroundjoin%
\pgfsetlinewidth{0.501875pt}%
\definecolor{currentstroke}{rgb}{0.265145,0.232956,0.516599}%
\pgfsetstrokecolor{currentstroke}%
\pgfsetdash{}{0pt}%
\pgfpathmoveto{\pgfqpoint{2.548094in}{5.877774in}}%
\pgfpathlineto{\pgfqpoint{2.495118in}{5.877752in}}%
\pgfusepath{stroke}%
\end{pgfscope}%
\begin{pgfscope}%
\pgfpathrectangle{\pgfqpoint{0.800000in}{5.105882in}}{\pgfqpoint{2.407767in}{1.544118in}}%
\pgfusepath{clip}%
\pgfsetbuttcap%
\pgfsetroundjoin%
\pgfsetlinewidth{0.501875pt}%
\definecolor{currentstroke}{rgb}{0.244972,0.287675,0.537260}%
\pgfsetstrokecolor{currentstroke}%
\pgfsetdash{}{0pt}%
\pgfpathmoveto{\pgfqpoint{2.495118in}{5.877752in}}%
\pgfpathlineto{\pgfqpoint{2.442142in}{5.877657in}}%
\pgfusepath{stroke}%
\end{pgfscope}%
\begin{pgfscope}%
\pgfpathrectangle{\pgfqpoint{0.800000in}{5.105882in}}{\pgfqpoint{2.407767in}{1.544118in}}%
\pgfusepath{clip}%
\pgfsetbuttcap%
\pgfsetroundjoin%
\pgfsetlinewidth{0.501875pt}%
\definecolor{currentstroke}{rgb}{0.229739,0.322361,0.545706}%
\pgfsetstrokecolor{currentstroke}%
\pgfsetdash{}{0pt}%
\pgfpathmoveto{\pgfqpoint{2.442142in}{5.877657in}}%
\pgfpathlineto{\pgfqpoint{2.389166in}{5.877596in}}%
\pgfusepath{stroke}%
\end{pgfscope}%
\begin{pgfscope}%
\pgfpathrectangle{\pgfqpoint{0.800000in}{5.105882in}}{\pgfqpoint{2.407767in}{1.544118in}}%
\pgfusepath{clip}%
\pgfsetbuttcap%
\pgfsetroundjoin%
\pgfsetlinewidth{0.501875pt}%
\definecolor{currentstroke}{rgb}{0.206756,0.371758,0.553117}%
\pgfsetstrokecolor{currentstroke}%
\pgfsetdash{}{0pt}%
\pgfpathmoveto{\pgfqpoint{2.389166in}{5.877596in}}%
\pgfpathlineto{\pgfqpoint{2.336191in}{5.877478in}}%
\pgfusepath{stroke}%
\end{pgfscope}%
\begin{pgfscope}%
\pgfpathrectangle{\pgfqpoint{0.800000in}{5.105882in}}{\pgfqpoint{2.407767in}{1.544118in}}%
\pgfusepath{clip}%
\pgfsetbuttcap%
\pgfsetroundjoin%
\pgfsetlinewidth{0.501875pt}%
\definecolor{currentstroke}{rgb}{0.204903,0.375746,0.553533}%
\pgfsetstrokecolor{currentstroke}%
\pgfsetdash{}{0pt}%
\pgfpathmoveto{\pgfqpoint{2.336191in}{5.877478in}}%
\pgfpathlineto{\pgfqpoint{2.283215in}{5.877327in}}%
\pgfusepath{stroke}%
\end{pgfscope}%
\begin{pgfscope}%
\pgfpathrectangle{\pgfqpoint{0.800000in}{5.105882in}}{\pgfqpoint{2.407767in}{1.544118in}}%
\pgfusepath{clip}%
\pgfsetbuttcap%
\pgfsetroundjoin%
\pgfsetlinewidth{0.501875pt}%
\definecolor{currentstroke}{rgb}{0.204903,0.375746,0.553533}%
\pgfsetstrokecolor{currentstroke}%
\pgfsetdash{}{0pt}%
\pgfpathmoveto{\pgfqpoint{2.283215in}{5.877327in}}%
\pgfpathlineto{\pgfqpoint{2.230240in}{5.877122in}}%
\pgfusepath{stroke}%
\end{pgfscope}%
\begin{pgfscope}%
\pgfpathrectangle{\pgfqpoint{0.800000in}{5.105882in}}{\pgfqpoint{2.407767in}{1.544118in}}%
\pgfusepath{clip}%
\pgfsetbuttcap%
\pgfsetroundjoin%
\pgfsetlinewidth{0.501875pt}%
\definecolor{currentstroke}{rgb}{0.212395,0.359683,0.551710}%
\pgfsetstrokecolor{currentstroke}%
\pgfsetdash{}{0pt}%
\pgfpathmoveto{\pgfqpoint{2.230240in}{5.877122in}}%
\pgfpathlineto{\pgfqpoint{2.177266in}{5.876952in}}%
\pgfusepath{stroke}%
\end{pgfscope}%
\begin{pgfscope}%
\pgfpathrectangle{\pgfqpoint{0.800000in}{5.105882in}}{\pgfqpoint{2.407767in}{1.544118in}}%
\pgfusepath{clip}%
\pgfsetbuttcap%
\pgfsetroundjoin%
\pgfsetlinewidth{0.501875pt}%
\definecolor{currentstroke}{rgb}{0.206756,0.371758,0.553117}%
\pgfsetstrokecolor{currentstroke}%
\pgfsetdash{}{0pt}%
\pgfpathmoveto{\pgfqpoint{2.177266in}{5.876952in}}%
\pgfpathlineto{\pgfqpoint{2.124293in}{5.876663in}}%
\pgfusepath{stroke}%
\end{pgfscope}%
\begin{pgfscope}%
\pgfpathrectangle{\pgfqpoint{0.800000in}{5.105882in}}{\pgfqpoint{2.407767in}{1.544118in}}%
\pgfusepath{clip}%
\pgfsetbuttcap%
\pgfsetroundjoin%
\pgfsetlinewidth{0.501875pt}%
\definecolor{currentstroke}{rgb}{0.220057,0.343307,0.549413}%
\pgfsetstrokecolor{currentstroke}%
\pgfsetdash{}{0pt}%
\pgfpathmoveto{\pgfqpoint{2.124293in}{5.876663in}}%
\pgfpathlineto{\pgfqpoint{2.071324in}{5.876252in}}%
\pgfusepath{stroke}%
\end{pgfscope}%
\begin{pgfscope}%
\pgfpathrectangle{\pgfqpoint{0.800000in}{5.105882in}}{\pgfqpoint{2.407767in}{1.544118in}}%
\pgfusepath{clip}%
\pgfsetbuttcap%
\pgfsetroundjoin%
\pgfsetlinewidth{0.501875pt}%
\definecolor{currentstroke}{rgb}{0.223925,0.334994,0.548053}%
\pgfsetstrokecolor{currentstroke}%
\pgfsetdash{}{0pt}%
\pgfpathmoveto{\pgfqpoint{2.071324in}{5.876252in}}%
\pgfpathlineto{\pgfqpoint{2.018353in}{5.875932in}}%
\pgfusepath{stroke}%
\end{pgfscope}%
\begin{pgfscope}%
\pgfpathrectangle{\pgfqpoint{0.800000in}{5.105882in}}{\pgfqpoint{2.407767in}{1.544118in}}%
\pgfusepath{clip}%
\pgfsetbuttcap%
\pgfsetroundjoin%
\pgfsetlinewidth{0.501875pt}%
\definecolor{currentstroke}{rgb}{0.237441,0.305202,0.541921}%
\pgfsetstrokecolor{currentstroke}%
\pgfsetdash{}{0pt}%
\pgfpathmoveto{\pgfqpoint{2.018353in}{5.875932in}}%
\pgfpathlineto{\pgfqpoint{1.965382in}{5.875733in}}%
\pgfusepath{stroke}%
\end{pgfscope}%
\begin{pgfscope}%
\pgfpathrectangle{\pgfqpoint{0.800000in}{5.105882in}}{\pgfqpoint{2.407767in}{1.544118in}}%
\pgfusepath{clip}%
\pgfsetbuttcap%
\pgfsetroundjoin%
\pgfsetlinewidth{0.501875pt}%
\definecolor{currentstroke}{rgb}{0.243113,0.292092,0.538516}%
\pgfsetstrokecolor{currentstroke}%
\pgfsetdash{}{0pt}%
\pgfpathmoveto{\pgfqpoint{1.965382in}{5.875733in}}%
\pgfpathlineto{\pgfqpoint{1.912416in}{5.875498in}}%
\pgfusepath{stroke}%
\end{pgfscope}%
\begin{pgfscope}%
\pgfpathrectangle{\pgfqpoint{0.800000in}{5.105882in}}{\pgfqpoint{2.407767in}{1.544118in}}%
\pgfusepath{clip}%
\pgfsetbuttcap%
\pgfsetroundjoin%
\pgfsetlinewidth{0.501875pt}%
\definecolor{currentstroke}{rgb}{0.260571,0.246922,0.522828}%
\pgfsetstrokecolor{currentstroke}%
\pgfsetdash{}{0pt}%
\pgfpathmoveto{\pgfqpoint{1.912416in}{5.875498in}}%
\pgfpathlineto{\pgfqpoint{1.859453in}{5.875079in}}%
\pgfusepath{stroke}%
\end{pgfscope}%
\begin{pgfscope}%
\pgfpathrectangle{\pgfqpoint{0.800000in}{5.105882in}}{\pgfqpoint{2.407767in}{1.544118in}}%
\pgfusepath{clip}%
\pgfsetbuttcap%
\pgfsetroundjoin%
\pgfsetlinewidth{0.501875pt}%
\definecolor{currentstroke}{rgb}{0.269308,0.218818,0.509577}%
\pgfsetstrokecolor{currentstroke}%
\pgfsetdash{}{0pt}%
\pgfpathmoveto{\pgfqpoint{1.859453in}{5.875079in}}%
\pgfpathlineto{\pgfqpoint{1.806513in}{5.874480in}}%
\pgfusepath{stroke}%
\end{pgfscope}%
\begin{pgfscope}%
\pgfpathrectangle{\pgfqpoint{0.800000in}{5.105882in}}{\pgfqpoint{2.407767in}{1.544118in}}%
\pgfusepath{clip}%
\pgfsetbuttcap%
\pgfsetroundjoin%
\pgfsetlinewidth{0.501875pt}%
\definecolor{currentstroke}{rgb}{0.276022,0.044167,0.370164}%
\pgfsetstrokecolor{currentstroke}%
\pgfsetdash{}{0pt}%
\pgfpathmoveto{\pgfqpoint{2.654046in}{5.912687in}}%
\pgfpathlineto{\pgfqpoint{2.601071in}{5.912486in}}%
\pgfusepath{stroke}%
\end{pgfscope}%
\begin{pgfscope}%
\pgfpathrectangle{\pgfqpoint{0.800000in}{5.105882in}}{\pgfqpoint{2.407767in}{1.544118in}}%
\pgfusepath{clip}%
\pgfsetbuttcap%
\pgfsetroundjoin%
\pgfsetlinewidth{0.501875pt}%
\definecolor{currentstroke}{rgb}{0.283072,0.130895,0.449241}%
\pgfsetstrokecolor{currentstroke}%
\pgfsetdash{}{0pt}%
\pgfpathmoveto{\pgfqpoint{2.601071in}{5.912486in}}%
\pgfpathlineto{\pgfqpoint{2.548096in}{5.912292in}}%
\pgfusepath{stroke}%
\end{pgfscope}%
\begin{pgfscope}%
\pgfpathrectangle{\pgfqpoint{0.800000in}{5.105882in}}{\pgfqpoint{2.407767in}{1.544118in}}%
\pgfusepath{clip}%
\pgfsetbuttcap%
\pgfsetroundjoin%
\pgfsetlinewidth{0.501875pt}%
\definecolor{currentstroke}{rgb}{0.271828,0.209303,0.504434}%
\pgfsetstrokecolor{currentstroke}%
\pgfsetdash{}{0pt}%
\pgfpathmoveto{\pgfqpoint{2.548096in}{5.912292in}}%
\pgfpathlineto{\pgfqpoint{2.495121in}{5.912085in}}%
\pgfusepath{stroke}%
\end{pgfscope}%
\begin{pgfscope}%
\pgfpathrectangle{\pgfqpoint{0.800000in}{5.105882in}}{\pgfqpoint{2.407767in}{1.544118in}}%
\pgfusepath{clip}%
\pgfsetbuttcap%
\pgfsetroundjoin%
\pgfsetlinewidth{0.501875pt}%
\definecolor{currentstroke}{rgb}{0.250425,0.274290,0.533103}%
\pgfsetstrokecolor{currentstroke}%
\pgfsetdash{}{0pt}%
\pgfpathmoveto{\pgfqpoint{2.495121in}{5.912085in}}%
\pgfpathlineto{\pgfqpoint{2.442147in}{5.911790in}}%
\pgfusepath{stroke}%
\end{pgfscope}%
\begin{pgfscope}%
\pgfpathrectangle{\pgfqpoint{0.800000in}{5.105882in}}{\pgfqpoint{2.407767in}{1.544118in}}%
\pgfusepath{clip}%
\pgfsetbuttcap%
\pgfsetroundjoin%
\pgfsetlinewidth{0.501875pt}%
\definecolor{currentstroke}{rgb}{0.229739,0.322361,0.545706}%
\pgfsetstrokecolor{currentstroke}%
\pgfsetdash{}{0pt}%
\pgfpathmoveto{\pgfqpoint{2.442147in}{5.911790in}}%
\pgfpathlineto{\pgfqpoint{2.389174in}{5.911416in}}%
\pgfusepath{stroke}%
\end{pgfscope}%
\begin{pgfscope}%
\pgfpathrectangle{\pgfqpoint{0.800000in}{5.105882in}}{\pgfqpoint{2.407767in}{1.544118in}}%
\pgfusepath{clip}%
\pgfsetbuttcap%
\pgfsetroundjoin%
\pgfsetlinewidth{0.501875pt}%
\definecolor{currentstroke}{rgb}{0.223925,0.334994,0.548053}%
\pgfsetstrokecolor{currentstroke}%
\pgfsetdash{}{0pt}%
\pgfpathmoveto{\pgfqpoint{2.389174in}{5.911416in}}%
\pgfpathlineto{\pgfqpoint{2.336204in}{5.910941in}}%
\pgfusepath{stroke}%
\end{pgfscope}%
\begin{pgfscope}%
\pgfpathrectangle{\pgfqpoint{0.800000in}{5.105882in}}{\pgfqpoint{2.407767in}{1.544118in}}%
\pgfusepath{clip}%
\pgfsetbuttcap%
\pgfsetroundjoin%
\pgfsetlinewidth{0.501875pt}%
\definecolor{currentstroke}{rgb}{0.204903,0.375746,0.553533}%
\pgfsetstrokecolor{currentstroke}%
\pgfsetdash{}{0pt}%
\pgfpathmoveto{\pgfqpoint{2.336204in}{5.910941in}}%
\pgfpathlineto{\pgfqpoint{2.283237in}{5.910305in}}%
\pgfusepath{stroke}%
\end{pgfscope}%
\begin{pgfscope}%
\pgfpathrectangle{\pgfqpoint{0.800000in}{5.105882in}}{\pgfqpoint{2.407767in}{1.544118in}}%
\pgfusepath{clip}%
\pgfsetbuttcap%
\pgfsetroundjoin%
\pgfsetlinewidth{0.501875pt}%
\definecolor{currentstroke}{rgb}{0.201239,0.383670,0.554294}%
\pgfsetstrokecolor{currentstroke}%
\pgfsetdash{}{0pt}%
\pgfpathmoveto{\pgfqpoint{2.283237in}{5.910305in}}%
\pgfpathlineto{\pgfqpoint{2.230271in}{5.909636in}}%
\pgfusepath{stroke}%
\end{pgfscope}%
\begin{pgfscope}%
\pgfpathrectangle{\pgfqpoint{0.800000in}{5.105882in}}{\pgfqpoint{2.407767in}{1.544118in}}%
\pgfusepath{clip}%
\pgfsetbuttcap%
\pgfsetroundjoin%
\pgfsetlinewidth{0.501875pt}%
\definecolor{currentstroke}{rgb}{0.208623,0.367752,0.552675}%
\pgfsetstrokecolor{currentstroke}%
\pgfsetdash{}{0pt}%
\pgfpathmoveto{\pgfqpoint{2.230271in}{5.909636in}}%
\pgfpathlineto{\pgfqpoint{2.177309in}{5.908879in}}%
\pgfusepath{stroke}%
\end{pgfscope}%
\begin{pgfscope}%
\pgfpathrectangle{\pgfqpoint{0.800000in}{5.105882in}}{\pgfqpoint{2.407767in}{1.544118in}}%
\pgfusepath{clip}%
\pgfsetbuttcap%
\pgfsetroundjoin%
\pgfsetlinewidth{0.501875pt}%
\definecolor{currentstroke}{rgb}{0.216210,0.351535,0.550627}%
\pgfsetstrokecolor{currentstroke}%
\pgfsetdash{}{0pt}%
\pgfpathmoveto{\pgfqpoint{2.177309in}{5.908879in}}%
\pgfpathlineto{\pgfqpoint{2.124356in}{5.907890in}}%
\pgfusepath{stroke}%
\end{pgfscope}%
\begin{pgfscope}%
\pgfpathrectangle{\pgfqpoint{0.800000in}{5.105882in}}{\pgfqpoint{2.407767in}{1.544118in}}%
\pgfusepath{clip}%
\pgfsetbuttcap%
\pgfsetroundjoin%
\pgfsetlinewidth{0.501875pt}%
\definecolor{currentstroke}{rgb}{0.214298,0.355619,0.551184}%
\pgfsetstrokecolor{currentstroke}%
\pgfsetdash{}{0pt}%
\pgfpathmoveto{\pgfqpoint{2.124356in}{5.907890in}}%
\pgfpathlineto{\pgfqpoint{2.071407in}{5.906799in}}%
\pgfusepath{stroke}%
\end{pgfscope}%
\begin{pgfscope}%
\pgfpathrectangle{\pgfqpoint{0.800000in}{5.105882in}}{\pgfqpoint{2.407767in}{1.544118in}}%
\pgfusepath{clip}%
\pgfsetbuttcap%
\pgfsetroundjoin%
\pgfsetlinewidth{0.501875pt}%
\definecolor{currentstroke}{rgb}{0.233603,0.313828,0.543914}%
\pgfsetstrokecolor{currentstroke}%
\pgfsetdash{}{0pt}%
\pgfpathmoveto{\pgfqpoint{2.071407in}{5.906799in}}%
\pgfpathlineto{\pgfqpoint{2.018466in}{5.905575in}}%
\pgfusepath{stroke}%
\end{pgfscope}%
\begin{pgfscope}%
\pgfpathrectangle{\pgfqpoint{0.800000in}{5.105882in}}{\pgfqpoint{2.407767in}{1.544118in}}%
\pgfusepath{clip}%
\pgfsetbuttcap%
\pgfsetroundjoin%
\pgfsetlinewidth{0.501875pt}%
\definecolor{currentstroke}{rgb}{0.235526,0.309527,0.542944}%
\pgfsetstrokecolor{currentstroke}%
\pgfsetdash{}{0pt}%
\pgfpathmoveto{\pgfqpoint{2.018466in}{5.905575in}}%
\pgfpathlineto{\pgfqpoint{1.965543in}{5.904104in}}%
\pgfusepath{stroke}%
\end{pgfscope}%
\begin{pgfscope}%
\pgfpathrectangle{\pgfqpoint{0.800000in}{5.105882in}}{\pgfqpoint{2.407767in}{1.544118in}}%
\pgfusepath{clip}%
\pgfsetbuttcap%
\pgfsetroundjoin%
\pgfsetlinewidth{0.501875pt}%
\definecolor{currentstroke}{rgb}{0.250425,0.274290,0.533103}%
\pgfsetstrokecolor{currentstroke}%
\pgfsetdash{}{0pt}%
\pgfpathmoveto{\pgfqpoint{1.965543in}{5.904104in}}%
\pgfpathlineto{\pgfqpoint{1.912638in}{5.902375in}}%
\pgfusepath{stroke}%
\end{pgfscope}%
\begin{pgfscope}%
\pgfpathrectangle{\pgfqpoint{0.800000in}{5.105882in}}{\pgfqpoint{2.407767in}{1.544118in}}%
\pgfusepath{clip}%
\pgfsetbuttcap%
\pgfsetroundjoin%
\pgfsetlinewidth{0.501875pt}%
\definecolor{currentstroke}{rgb}{0.253935,0.265254,0.529983}%
\pgfsetstrokecolor{currentstroke}%
\pgfsetdash{}{0pt}%
\pgfpathmoveto{\pgfqpoint{1.912638in}{5.902375in}}%
\pgfpathlineto{\pgfqpoint{1.859774in}{5.900211in}}%
\pgfusepath{stroke}%
\end{pgfscope}%
\begin{pgfscope}%
\pgfpathrectangle{\pgfqpoint{0.800000in}{5.105882in}}{\pgfqpoint{2.407767in}{1.544118in}}%
\pgfusepath{clip}%
\pgfsetbuttcap%
\pgfsetroundjoin%
\pgfsetlinewidth{0.501875pt}%
\definecolor{currentstroke}{rgb}{0.266580,0.228262,0.514349}%
\pgfsetstrokecolor{currentstroke}%
\pgfsetdash{}{0pt}%
\pgfpathmoveto{\pgfqpoint{1.859774in}{5.900211in}}%
\pgfpathlineto{\pgfqpoint{1.806956in}{5.897638in}}%
\pgfusepath{stroke}%
\end{pgfscope}%
\begin{pgfscope}%
\pgfpathrectangle{\pgfqpoint{0.800000in}{5.105882in}}{\pgfqpoint{2.407767in}{1.544118in}}%
\pgfusepath{clip}%
\pgfsetbuttcap%
\pgfsetroundjoin%
\pgfsetlinewidth{0.501875pt}%
\definecolor{currentstroke}{rgb}{0.276022,0.044167,0.370164}%
\pgfsetstrokecolor{currentstroke}%
\pgfsetdash{}{0pt}%
\pgfpathmoveto{\pgfqpoint{2.654046in}{5.947433in}}%
\pgfpathlineto{\pgfqpoint{2.601073in}{5.947092in}}%
\pgfusepath{stroke}%
\end{pgfscope}%
\begin{pgfscope}%
\pgfpathrectangle{\pgfqpoint{0.800000in}{5.105882in}}{\pgfqpoint{2.407767in}{1.544118in}}%
\pgfusepath{clip}%
\pgfsetbuttcap%
\pgfsetroundjoin%
\pgfsetlinewidth{0.501875pt}%
\definecolor{currentstroke}{rgb}{0.283229,0.120777,0.440584}%
\pgfsetstrokecolor{currentstroke}%
\pgfsetdash{}{0pt}%
\pgfpathmoveto{\pgfqpoint{2.601073in}{5.947092in}}%
\pgfpathlineto{\pgfqpoint{2.548103in}{5.946588in}}%
\pgfusepath{stroke}%
\end{pgfscope}%
\begin{pgfscope}%
\pgfpathrectangle{\pgfqpoint{0.800000in}{5.105882in}}{\pgfqpoint{2.407767in}{1.544118in}}%
\pgfusepath{clip}%
\pgfsetbuttcap%
\pgfsetroundjoin%
\pgfsetlinewidth{0.501875pt}%
\definecolor{currentstroke}{rgb}{0.274128,0.199721,0.498911}%
\pgfsetstrokecolor{currentstroke}%
\pgfsetdash{}{0pt}%
\pgfpathmoveto{\pgfqpoint{2.548103in}{5.946588in}}%
\pgfpathlineto{\pgfqpoint{2.495134in}{5.946075in}}%
\pgfusepath{stroke}%
\end{pgfscope}%
\begin{pgfscope}%
\pgfpathrectangle{\pgfqpoint{0.800000in}{5.105882in}}{\pgfqpoint{2.407767in}{1.544118in}}%
\pgfusepath{clip}%
\pgfsetbuttcap%
\pgfsetroundjoin%
\pgfsetlinewidth{0.501875pt}%
\definecolor{currentstroke}{rgb}{0.258965,0.251537,0.524736}%
\pgfsetstrokecolor{currentstroke}%
\pgfsetdash{}{0pt}%
\pgfpathmoveto{\pgfqpoint{2.495134in}{5.946075in}}%
\pgfpathlineto{\pgfqpoint{2.442165in}{5.945518in}}%
\pgfusepath{stroke}%
\end{pgfscope}%
\begin{pgfscope}%
\pgfpathrectangle{\pgfqpoint{0.800000in}{5.105882in}}{\pgfqpoint{2.407767in}{1.544118in}}%
\pgfusepath{clip}%
\pgfsetbuttcap%
\pgfsetroundjoin%
\pgfsetlinewidth{0.501875pt}%
\definecolor{currentstroke}{rgb}{0.235526,0.309527,0.542944}%
\pgfsetstrokecolor{currentstroke}%
\pgfsetdash{}{0pt}%
\pgfpathmoveto{\pgfqpoint{2.442165in}{5.945518in}}%
\pgfpathlineto{\pgfqpoint{2.389199in}{5.944854in}}%
\pgfusepath{stroke}%
\end{pgfscope}%
\begin{pgfscope}%
\pgfpathrectangle{\pgfqpoint{0.800000in}{5.105882in}}{\pgfqpoint{2.407767in}{1.544118in}}%
\pgfusepath{clip}%
\pgfsetbuttcap%
\pgfsetroundjoin%
\pgfsetlinewidth{0.501875pt}%
\definecolor{currentstroke}{rgb}{0.221989,0.339161,0.548752}%
\pgfsetstrokecolor{currentstroke}%
\pgfsetdash{}{0pt}%
\pgfpathmoveto{\pgfqpoint{2.389199in}{5.944854in}}%
\pgfpathlineto{\pgfqpoint{2.336238in}{5.944054in}}%
\pgfusepath{stroke}%
\end{pgfscope}%
\begin{pgfscope}%
\pgfpathrectangle{\pgfqpoint{0.800000in}{5.105882in}}{\pgfqpoint{2.407767in}{1.544118in}}%
\pgfusepath{clip}%
\pgfsetbuttcap%
\pgfsetroundjoin%
\pgfsetlinewidth{0.501875pt}%
\definecolor{currentstroke}{rgb}{0.223925,0.334994,0.548053}%
\pgfsetstrokecolor{currentstroke}%
\pgfsetdash{}{0pt}%
\pgfpathmoveto{\pgfqpoint{2.336238in}{5.944054in}}%
\pgfpathlineto{\pgfqpoint{2.283287in}{5.943031in}}%
\pgfusepath{stroke}%
\end{pgfscope}%
\begin{pgfscope}%
\pgfpathrectangle{\pgfqpoint{0.800000in}{5.105882in}}{\pgfqpoint{2.407767in}{1.544118in}}%
\pgfusepath{clip}%
\pgfsetbuttcap%
\pgfsetroundjoin%
\pgfsetlinewidth{0.501875pt}%
\definecolor{currentstroke}{rgb}{0.206756,0.371758,0.553117}%
\pgfsetstrokecolor{currentstroke}%
\pgfsetdash{}{0pt}%
\pgfpathmoveto{\pgfqpoint{2.283287in}{5.943031in}}%
\pgfpathlineto{\pgfqpoint{2.230350in}{5.941761in}}%
\pgfusepath{stroke}%
\end{pgfscope}%
\begin{pgfscope}%
\pgfpathrectangle{\pgfqpoint{0.800000in}{5.105882in}}{\pgfqpoint{2.407767in}{1.544118in}}%
\pgfusepath{clip}%
\pgfsetbuttcap%
\pgfsetroundjoin%
\pgfsetlinewidth{0.501875pt}%
\definecolor{currentstroke}{rgb}{0.216210,0.351535,0.550627}%
\pgfsetstrokecolor{currentstroke}%
\pgfsetdash{}{0pt}%
\pgfpathmoveto{\pgfqpoint{2.230350in}{5.941761in}}%
\pgfpathlineto{\pgfqpoint{2.177430in}{5.940223in}}%
\pgfusepath{stroke}%
\end{pgfscope}%
\begin{pgfscope}%
\pgfpathrectangle{\pgfqpoint{0.800000in}{5.105882in}}{\pgfqpoint{2.407767in}{1.544118in}}%
\pgfusepath{clip}%
\pgfsetbuttcap%
\pgfsetroundjoin%
\pgfsetlinewidth{0.501875pt}%
\definecolor{currentstroke}{rgb}{0.214298,0.355619,0.551184}%
\pgfsetstrokecolor{currentstroke}%
\pgfsetdash{}{0pt}%
\pgfpathmoveto{\pgfqpoint{2.177430in}{5.940223in}}%
\pgfpathlineto{\pgfqpoint{2.124535in}{5.938352in}}%
\pgfusepath{stroke}%
\end{pgfscope}%
\begin{pgfscope}%
\pgfpathrectangle{\pgfqpoint{0.800000in}{5.105882in}}{\pgfqpoint{2.407767in}{1.544118in}}%
\pgfusepath{clip}%
\pgfsetbuttcap%
\pgfsetroundjoin%
\pgfsetlinewidth{0.501875pt}%
\definecolor{currentstroke}{rgb}{0.212395,0.359683,0.551710}%
\pgfsetstrokecolor{currentstroke}%
\pgfsetdash{}{0pt}%
\pgfpathmoveto{\pgfqpoint{2.124535in}{5.938352in}}%
\pgfpathlineto{\pgfqpoint{2.071674in}{5.936138in}}%
\pgfusepath{stroke}%
\end{pgfscope}%
\begin{pgfscope}%
\pgfpathrectangle{\pgfqpoint{0.800000in}{5.105882in}}{\pgfqpoint{2.407767in}{1.544118in}}%
\pgfusepath{clip}%
\pgfsetbuttcap%
\pgfsetroundjoin%
\pgfsetlinewidth{0.501875pt}%
\definecolor{currentstroke}{rgb}{0.229739,0.322361,0.545706}%
\pgfsetstrokecolor{currentstroke}%
\pgfsetdash{}{0pt}%
\pgfpathmoveto{\pgfqpoint{2.071674in}{5.936138in}}%
\pgfpathlineto{\pgfqpoint{2.018856in}{5.933547in}}%
\pgfusepath{stroke}%
\end{pgfscope}%
\begin{pgfscope}%
\pgfpathrectangle{\pgfqpoint{0.800000in}{5.105882in}}{\pgfqpoint{2.407767in}{1.544118in}}%
\pgfusepath{clip}%
\pgfsetbuttcap%
\pgfsetroundjoin%
\pgfsetlinewidth{0.501875pt}%
\definecolor{currentstroke}{rgb}{0.244972,0.287675,0.537260}%
\pgfsetstrokecolor{currentstroke}%
\pgfsetdash{}{0pt}%
\pgfpathmoveto{\pgfqpoint{2.018856in}{5.933547in}}%
\pgfpathlineto{\pgfqpoint{1.966093in}{5.930556in}}%
\pgfusepath{stroke}%
\end{pgfscope}%
\begin{pgfscope}%
\pgfpathrectangle{\pgfqpoint{0.800000in}{5.105882in}}{\pgfqpoint{2.407767in}{1.544118in}}%
\pgfusepath{clip}%
\pgfsetbuttcap%
\pgfsetroundjoin%
\pgfsetlinewidth{0.501875pt}%
\definecolor{currentstroke}{rgb}{0.276022,0.044167,0.370164}%
\pgfsetstrokecolor{currentstroke}%
\pgfsetdash{}{0pt}%
\pgfpathmoveto{\pgfqpoint{2.654046in}{5.982180in}}%
\pgfpathlineto{\pgfqpoint{2.601084in}{5.981405in}}%
\pgfusepath{stroke}%
\end{pgfscope}%
\begin{pgfscope}%
\pgfpathrectangle{\pgfqpoint{0.800000in}{5.105882in}}{\pgfqpoint{2.407767in}{1.544118in}}%
\pgfusepath{clip}%
\pgfsetbuttcap%
\pgfsetroundjoin%
\pgfsetlinewidth{0.501875pt}%
\definecolor{currentstroke}{rgb}{0.283197,0.115680,0.436115}%
\pgfsetstrokecolor{currentstroke}%
\pgfsetdash{}{0pt}%
\pgfpathmoveto{\pgfqpoint{2.601084in}{5.981405in}}%
\pgfpathlineto{\pgfqpoint{2.548121in}{5.980654in}}%
\pgfusepath{stroke}%
\end{pgfscope}%
\begin{pgfscope}%
\pgfpathrectangle{\pgfqpoint{0.800000in}{5.105882in}}{\pgfqpoint{2.407767in}{1.544118in}}%
\pgfusepath{clip}%
\pgfsetbuttcap%
\pgfsetroundjoin%
\pgfsetlinewidth{0.501875pt}%
\definecolor{currentstroke}{rgb}{0.275191,0.194905,0.496005}%
\pgfsetstrokecolor{currentstroke}%
\pgfsetdash{}{0pt}%
\pgfpathmoveto{\pgfqpoint{2.548121in}{5.980654in}}%
\pgfpathlineto{\pgfqpoint{2.495158in}{5.979899in}}%
\pgfusepath{stroke}%
\end{pgfscope}%
\begin{pgfscope}%
\pgfpathrectangle{\pgfqpoint{0.800000in}{5.105882in}}{\pgfqpoint{2.407767in}{1.544118in}}%
\pgfusepath{clip}%
\pgfsetbuttcap%
\pgfsetroundjoin%
\pgfsetlinewidth{0.501875pt}%
\definecolor{currentstroke}{rgb}{0.257322,0.256130,0.526563}%
\pgfsetstrokecolor{currentstroke}%
\pgfsetdash{}{0pt}%
\pgfpathmoveto{\pgfqpoint{2.495158in}{5.979899in}}%
\pgfpathlineto{\pgfqpoint{2.442199in}{5.979035in}}%
\pgfusepath{stroke}%
\end{pgfscope}%
\begin{pgfscope}%
\pgfpathrectangle{\pgfqpoint{0.800000in}{5.105882in}}{\pgfqpoint{2.407767in}{1.544118in}}%
\pgfusepath{clip}%
\pgfsetbuttcap%
\pgfsetroundjoin%
\pgfsetlinewidth{0.501875pt}%
\definecolor{currentstroke}{rgb}{0.244972,0.287675,0.537260}%
\pgfsetstrokecolor{currentstroke}%
\pgfsetdash{}{0pt}%
\pgfpathmoveto{\pgfqpoint{2.442199in}{5.979035in}}%
\pgfpathlineto{\pgfqpoint{2.389252in}{5.977932in}}%
\pgfusepath{stroke}%
\end{pgfscope}%
\begin{pgfscope}%
\pgfpathrectangle{\pgfqpoint{0.800000in}{5.105882in}}{\pgfqpoint{2.407767in}{1.544118in}}%
\pgfusepath{clip}%
\pgfsetbuttcap%
\pgfsetroundjoin%
\pgfsetlinewidth{0.501875pt}%
\definecolor{currentstroke}{rgb}{0.231674,0.318106,0.544834}%
\pgfsetstrokecolor{currentstroke}%
\pgfsetdash{}{0pt}%
\pgfpathmoveto{\pgfqpoint{2.389252in}{5.977932in}}%
\pgfpathlineto{\pgfqpoint{2.336318in}{5.976597in}}%
\pgfusepath{stroke}%
\end{pgfscope}%
\begin{pgfscope}%
\pgfpathrectangle{\pgfqpoint{0.800000in}{5.105882in}}{\pgfqpoint{2.407767in}{1.544118in}}%
\pgfusepath{clip}%
\pgfsetbuttcap%
\pgfsetroundjoin%
\pgfsetlinewidth{0.501875pt}%
\definecolor{currentstroke}{rgb}{0.220057,0.343307,0.549413}%
\pgfsetstrokecolor{currentstroke}%
\pgfsetdash{}{0pt}%
\pgfpathmoveto{\pgfqpoint{2.336318in}{5.976597in}}%
\pgfpathlineto{\pgfqpoint{2.283403in}{5.975000in}}%
\pgfusepath{stroke}%
\end{pgfscope}%
\begin{pgfscope}%
\pgfpathrectangle{\pgfqpoint{0.800000in}{5.105882in}}{\pgfqpoint{2.407767in}{1.544118in}}%
\pgfusepath{clip}%
\pgfsetbuttcap%
\pgfsetroundjoin%
\pgfsetlinewidth{0.501875pt}%
\definecolor{currentstroke}{rgb}{0.221989,0.339161,0.548752}%
\pgfsetstrokecolor{currentstroke}%
\pgfsetdash{}{0pt}%
\pgfpathmoveto{\pgfqpoint{2.283403in}{5.975000in}}%
\pgfpathlineto{\pgfqpoint{2.230524in}{5.972975in}}%
\pgfusepath{stroke}%
\end{pgfscope}%
\begin{pgfscope}%
\pgfpathrectangle{\pgfqpoint{0.800000in}{5.105882in}}{\pgfqpoint{2.407767in}{1.544118in}}%
\pgfusepath{clip}%
\pgfsetbuttcap%
\pgfsetroundjoin%
\pgfsetlinewidth{0.501875pt}%
\definecolor{currentstroke}{rgb}{0.229739,0.322361,0.545706}%
\pgfsetstrokecolor{currentstroke}%
\pgfsetdash{}{0pt}%
\pgfpathmoveto{\pgfqpoint{2.230524in}{5.972975in}}%
\pgfpathlineto{\pgfqpoint{2.177697in}{5.970472in}}%
\pgfusepath{stroke}%
\end{pgfscope}%
\begin{pgfscope}%
\pgfpathrectangle{\pgfqpoint{0.800000in}{5.105882in}}{\pgfqpoint{2.407767in}{1.544118in}}%
\pgfusepath{clip}%
\pgfsetbuttcap%
\pgfsetroundjoin%
\pgfsetlinewidth{0.501875pt}%
\definecolor{currentstroke}{rgb}{0.239346,0.300855,0.540844}%
\pgfsetstrokecolor{currentstroke}%
\pgfsetdash{}{0pt}%
\pgfpathmoveto{\pgfqpoint{2.177697in}{5.970472in}}%
\pgfpathlineto{\pgfqpoint{2.124924in}{5.967525in}}%
\pgfusepath{stroke}%
\end{pgfscope}%
\begin{pgfscope}%
\pgfpathrectangle{\pgfqpoint{0.800000in}{5.105882in}}{\pgfqpoint{2.407767in}{1.544118in}}%
\pgfusepath{clip}%
\pgfsetbuttcap%
\pgfsetroundjoin%
\pgfsetlinewidth{0.501875pt}%
\definecolor{currentstroke}{rgb}{0.274952,0.037752,0.364543}%
\pgfsetstrokecolor{currentstroke}%
\pgfsetdash{}{0pt}%
\pgfpathmoveto{\pgfqpoint{2.654046in}{6.016926in}}%
\pgfpathlineto{\pgfqpoint{2.601083in}{6.016195in}}%
\pgfusepath{stroke}%
\end{pgfscope}%
\begin{pgfscope}%
\pgfpathrectangle{\pgfqpoint{0.800000in}{5.105882in}}{\pgfqpoint{2.407767in}{1.544118in}}%
\pgfusepath{clip}%
\pgfsetbuttcap%
\pgfsetroundjoin%
\pgfsetlinewidth{0.501875pt}%
\definecolor{currentstroke}{rgb}{0.282910,0.105393,0.426902}%
\pgfsetstrokecolor{currentstroke}%
\pgfsetdash{}{0pt}%
\pgfpathmoveto{\pgfqpoint{2.601083in}{6.016195in}}%
\pgfpathlineto{\pgfqpoint{2.548133in}{6.015140in}}%
\pgfusepath{stroke}%
\end{pgfscope}%
\begin{pgfscope}%
\pgfpathrectangle{\pgfqpoint{0.800000in}{5.105882in}}{\pgfqpoint{2.407767in}{1.544118in}}%
\pgfusepath{clip}%
\pgfsetbuttcap%
\pgfsetroundjoin%
\pgfsetlinewidth{0.501875pt}%
\definecolor{currentstroke}{rgb}{0.280868,0.160771,0.472899}%
\pgfsetstrokecolor{currentstroke}%
\pgfsetdash{}{0pt}%
\pgfpathmoveto{\pgfqpoint{2.548133in}{6.015140in}}%
\pgfpathlineto{\pgfqpoint{2.495186in}{6.014014in}}%
\pgfusepath{stroke}%
\end{pgfscope}%
\begin{pgfscope}%
\pgfpathrectangle{\pgfqpoint{0.800000in}{5.105882in}}{\pgfqpoint{2.407767in}{1.544118in}}%
\pgfusepath{clip}%
\pgfsetbuttcap%
\pgfsetroundjoin%
\pgfsetlinewidth{0.501875pt}%
\definecolor{currentstroke}{rgb}{0.265145,0.232956,0.516599}%
\pgfsetstrokecolor{currentstroke}%
\pgfsetdash{}{0pt}%
\pgfpathmoveto{\pgfqpoint{2.495186in}{6.014014in}}%
\pgfpathlineto{\pgfqpoint{2.442247in}{6.012750in}}%
\pgfusepath{stroke}%
\end{pgfscope}%
\begin{pgfscope}%
\pgfpathrectangle{\pgfqpoint{0.800000in}{5.105882in}}{\pgfqpoint{2.407767in}{1.544118in}}%
\pgfusepath{clip}%
\pgfsetbuttcap%
\pgfsetroundjoin%
\pgfsetlinewidth{0.501875pt}%
\definecolor{currentstroke}{rgb}{0.248629,0.278775,0.534556}%
\pgfsetstrokecolor{currentstroke}%
\pgfsetdash{}{0pt}%
\pgfpathmoveto{\pgfqpoint{2.442247in}{6.012750in}}%
\pgfpathlineto{\pgfqpoint{2.389329in}{6.011180in}}%
\pgfusepath{stroke}%
\end{pgfscope}%
\begin{pgfscope}%
\pgfpathrectangle{\pgfqpoint{0.800000in}{5.105882in}}{\pgfqpoint{2.407767in}{1.544118in}}%
\pgfusepath{clip}%
\pgfsetbuttcap%
\pgfsetroundjoin%
\pgfsetlinewidth{0.501875pt}%
\definecolor{currentstroke}{rgb}{0.241237,0.296485,0.539709}%
\pgfsetstrokecolor{currentstroke}%
\pgfsetdash{}{0pt}%
\pgfpathmoveto{\pgfqpoint{2.389329in}{6.011180in}}%
\pgfpathlineto{\pgfqpoint{2.336430in}{6.009369in}}%
\pgfusepath{stroke}%
\end{pgfscope}%
\begin{pgfscope}%
\pgfpathrectangle{\pgfqpoint{0.800000in}{5.105882in}}{\pgfqpoint{2.407767in}{1.544118in}}%
\pgfusepath{clip}%
\pgfsetbuttcap%
\pgfsetroundjoin%
\pgfsetlinewidth{0.501875pt}%
\definecolor{currentstroke}{rgb}{0.235526,0.309527,0.542944}%
\pgfsetstrokecolor{currentstroke}%
\pgfsetdash{}{0pt}%
\pgfpathmoveto{\pgfqpoint{2.336430in}{6.009369in}}%
\pgfpathlineto{\pgfqpoint{2.283565in}{6.007198in}}%
\pgfusepath{stroke}%
\end{pgfscope}%
\begin{pgfscope}%
\pgfpathrectangle{\pgfqpoint{0.800000in}{5.105882in}}{\pgfqpoint{2.407767in}{1.544118in}}%
\pgfusepath{clip}%
\pgfsetbuttcap%
\pgfsetroundjoin%
\pgfsetlinewidth{0.501875pt}%
\definecolor{currentstroke}{rgb}{0.231674,0.318106,0.544834}%
\pgfsetstrokecolor{currentstroke}%
\pgfsetdash{}{0pt}%
\pgfpathmoveto{\pgfqpoint{2.283565in}{6.007198in}}%
\pgfpathlineto{\pgfqpoint{2.230755in}{6.004521in}}%
\pgfusepath{stroke}%
\end{pgfscope}%
\begin{pgfscope}%
\pgfpathrectangle{\pgfqpoint{0.800000in}{5.105882in}}{\pgfqpoint{2.407767in}{1.544118in}}%
\pgfusepath{clip}%
\pgfsetbuttcap%
\pgfsetroundjoin%
\pgfsetlinewidth{0.501875pt}%
\definecolor{currentstroke}{rgb}{0.229739,0.322361,0.545706}%
\pgfsetstrokecolor{currentstroke}%
\pgfsetdash{}{0pt}%
\pgfpathmoveto{\pgfqpoint{2.230755in}{6.004521in}}%
\pgfpathlineto{\pgfqpoint{2.178033in}{6.001232in}}%
\pgfusepath{stroke}%
\end{pgfscope}%
\begin{pgfscope}%
\pgfpathrectangle{\pgfqpoint{0.800000in}{5.105882in}}{\pgfqpoint{2.407767in}{1.544118in}}%
\pgfusepath{clip}%
\pgfsetbuttcap%
\pgfsetroundjoin%
\pgfsetlinewidth{0.501875pt}%
\definecolor{currentstroke}{rgb}{0.274952,0.037752,0.364543}%
\pgfsetstrokecolor{currentstroke}%
\pgfsetdash{}{0pt}%
\pgfpathmoveto{\pgfqpoint{2.654046in}{6.051672in}}%
\pgfpathlineto{\pgfqpoint{2.601098in}{6.050620in}}%
\pgfusepath{stroke}%
\end{pgfscope}%
\begin{pgfscope}%
\pgfpathrectangle{\pgfqpoint{0.800000in}{5.105882in}}{\pgfqpoint{2.407767in}{1.544118in}}%
\pgfusepath{clip}%
\pgfsetbuttcap%
\pgfsetroundjoin%
\pgfsetlinewidth{0.501875pt}%
\definecolor{currentstroke}{rgb}{0.282656,0.100196,0.422160}%
\pgfsetstrokecolor{currentstroke}%
\pgfsetdash{}{0pt}%
\pgfpathmoveto{\pgfqpoint{2.601098in}{6.050620in}}%
\pgfpathlineto{\pgfqpoint{2.548163in}{6.049294in}}%
\pgfusepath{stroke}%
\end{pgfscope}%
\begin{pgfscope}%
\pgfpathrectangle{\pgfqpoint{0.800000in}{5.105882in}}{\pgfqpoint{2.407767in}{1.544118in}}%
\pgfusepath{clip}%
\pgfsetbuttcap%
\pgfsetroundjoin%
\pgfsetlinewidth{0.501875pt}%
\definecolor{currentstroke}{rgb}{0.282290,0.145912,0.461510}%
\pgfsetstrokecolor{currentstroke}%
\pgfsetdash{}{0pt}%
\pgfpathmoveto{\pgfqpoint{2.548163in}{6.049294in}}%
\pgfpathlineto{\pgfqpoint{2.495233in}{6.047888in}}%
\pgfusepath{stroke}%
\end{pgfscope}%
\begin{pgfscope}%
\pgfpathrectangle{\pgfqpoint{0.800000in}{5.105882in}}{\pgfqpoint{2.407767in}{1.544118in}}%
\pgfusepath{clip}%
\pgfsetbuttcap%
\pgfsetroundjoin%
\pgfsetlinewidth{0.501875pt}%
\definecolor{currentstroke}{rgb}{0.273006,0.204520,0.501721}%
\pgfsetstrokecolor{currentstroke}%
\pgfsetdash{}{0pt}%
\pgfpathmoveto{\pgfqpoint{2.495233in}{6.047888in}}%
\pgfpathlineto{\pgfqpoint{2.442308in}{6.046391in}}%
\pgfusepath{stroke}%
\end{pgfscope}%
\begin{pgfscope}%
\pgfpathrectangle{\pgfqpoint{0.800000in}{5.105882in}}{\pgfqpoint{2.407767in}{1.544118in}}%
\pgfusepath{clip}%
\pgfsetbuttcap%
\pgfsetroundjoin%
\pgfsetlinewidth{0.501875pt}%
\definecolor{currentstroke}{rgb}{0.255645,0.260703,0.528312}%
\pgfsetstrokecolor{currentstroke}%
\pgfsetdash{}{0pt}%
\pgfpathmoveto{\pgfqpoint{2.442308in}{6.046391in}}%
\pgfpathlineto{\pgfqpoint{2.389396in}{6.044731in}}%
\pgfusepath{stroke}%
\end{pgfscope}%
\begin{pgfscope}%
\pgfpathrectangle{\pgfqpoint{0.800000in}{5.105882in}}{\pgfqpoint{2.407767in}{1.544118in}}%
\pgfusepath{clip}%
\pgfsetbuttcap%
\pgfsetroundjoin%
\pgfsetlinewidth{0.501875pt}%
\definecolor{currentstroke}{rgb}{0.252194,0.269783,0.531579}%
\pgfsetstrokecolor{currentstroke}%
\pgfsetdash{}{0pt}%
\pgfpathmoveto{\pgfqpoint{2.389396in}{6.044731in}}%
\pgfpathlineto{\pgfqpoint{2.336517in}{6.042690in}}%
\pgfusepath{stroke}%
\end{pgfscope}%
\begin{pgfscope}%
\pgfpathrectangle{\pgfqpoint{0.800000in}{5.105882in}}{\pgfqpoint{2.407767in}{1.544118in}}%
\pgfusepath{clip}%
\pgfsetbuttcap%
\pgfsetroundjoin%
\pgfsetlinewidth{0.501875pt}%
\definecolor{currentstroke}{rgb}{0.246811,0.283237,0.535941}%
\pgfsetstrokecolor{currentstroke}%
\pgfsetdash{}{0pt}%
\pgfpathmoveto{\pgfqpoint{2.336517in}{6.042690in}}%
\pgfpathlineto{\pgfqpoint{2.283693in}{6.040135in}}%
\pgfusepath{stroke}%
\end{pgfscope}%
\begin{pgfscope}%
\pgfpathrectangle{\pgfqpoint{0.800000in}{5.105882in}}{\pgfqpoint{2.407767in}{1.544118in}}%
\pgfusepath{clip}%
\pgfsetbuttcap%
\pgfsetroundjoin%
\pgfsetlinewidth{0.501875pt}%
\definecolor{currentstroke}{rgb}{0.237441,0.305202,0.541921}%
\pgfsetstrokecolor{currentstroke}%
\pgfsetdash{}{0pt}%
\pgfpathmoveto{\pgfqpoint{2.283693in}{6.040135in}}%
\pgfpathlineto{\pgfqpoint{2.230949in}{6.036993in}}%
\pgfusepath{stroke}%
\end{pgfscope}%
\begin{pgfscope}%
\pgfpathrectangle{\pgfqpoint{0.800000in}{5.105882in}}{\pgfqpoint{2.407767in}{1.544118in}}%
\pgfusepath{clip}%
\pgfsetbuttcap%
\pgfsetroundjoin%
\pgfsetlinewidth{0.501875pt}%
\definecolor{currentstroke}{rgb}{0.273809,0.031497,0.358853}%
\pgfsetstrokecolor{currentstroke}%
\pgfsetdash{}{0pt}%
\pgfpathmoveto{\pgfqpoint{2.654046in}{6.086418in}}%
\pgfpathlineto{\pgfqpoint{2.601108in}{6.085185in}}%
\pgfusepath{stroke}%
\end{pgfscope}%
\begin{pgfscope}%
\pgfpathrectangle{\pgfqpoint{0.800000in}{5.105882in}}{\pgfqpoint{2.407767in}{1.544118in}}%
\pgfusepath{clip}%
\pgfsetbuttcap%
\pgfsetroundjoin%
\pgfsetlinewidth{0.501875pt}%
\definecolor{currentstroke}{rgb}{0.281446,0.084320,0.407414}%
\pgfsetstrokecolor{currentstroke}%
\pgfsetdash{}{0pt}%
\pgfpathmoveto{\pgfqpoint{2.601108in}{6.085185in}}%
\pgfpathlineto{\pgfqpoint{2.548184in}{6.083691in}}%
\pgfusepath{stroke}%
\end{pgfscope}%
\begin{pgfscope}%
\pgfpathrectangle{\pgfqpoint{0.800000in}{5.105882in}}{\pgfqpoint{2.407767in}{1.544118in}}%
\pgfusepath{clip}%
\pgfsetbuttcap%
\pgfsetroundjoin%
\pgfsetlinewidth{0.501875pt}%
\definecolor{currentstroke}{rgb}{0.282884,0.135920,0.453427}%
\pgfsetstrokecolor{currentstroke}%
\pgfsetdash{}{0pt}%
\pgfpathmoveto{\pgfqpoint{2.548184in}{6.083691in}}%
\pgfpathlineto{\pgfqpoint{2.495263in}{6.082149in}}%
\pgfusepath{stroke}%
\end{pgfscope}%
\begin{pgfscope}%
\pgfpathrectangle{\pgfqpoint{0.800000in}{5.105882in}}{\pgfqpoint{2.407767in}{1.544118in}}%
\pgfusepath{clip}%
\pgfsetbuttcap%
\pgfsetroundjoin%
\pgfsetlinewidth{0.501875pt}%
\definecolor{currentstroke}{rgb}{0.278012,0.180367,0.486697}%
\pgfsetstrokecolor{currentstroke}%
\pgfsetdash{}{0pt}%
\pgfpathmoveto{\pgfqpoint{2.495263in}{6.082149in}}%
\pgfpathlineto{\pgfqpoint{2.442351in}{6.080484in}}%
\pgfusepath{stroke}%
\end{pgfscope}%
\begin{pgfscope}%
\pgfpathrectangle{\pgfqpoint{0.800000in}{5.105882in}}{\pgfqpoint{2.407767in}{1.544118in}}%
\pgfusepath{clip}%
\pgfsetbuttcap%
\pgfsetroundjoin%
\pgfsetlinewidth{0.501875pt}%
\definecolor{currentstroke}{rgb}{0.274128,0.199721,0.498911}%
\pgfsetstrokecolor{currentstroke}%
\pgfsetdash{}{0pt}%
\pgfpathmoveto{\pgfqpoint{2.442351in}{6.080484in}}%
\pgfpathlineto{\pgfqpoint{2.389456in}{6.078613in}}%
\pgfusepath{stroke}%
\end{pgfscope}%
\begin{pgfscope}%
\pgfpathrectangle{\pgfqpoint{0.800000in}{5.105882in}}{\pgfqpoint{2.407767in}{1.544118in}}%
\pgfusepath{clip}%
\pgfsetbuttcap%
\pgfsetroundjoin%
\pgfsetlinewidth{0.501875pt}%
\definecolor{currentstroke}{rgb}{0.262138,0.242286,0.520837}%
\pgfsetstrokecolor{currentstroke}%
\pgfsetdash{}{0pt}%
\pgfpathmoveto{\pgfqpoint{2.389456in}{6.078613in}}%
\pgfpathlineto{\pgfqpoint{2.336607in}{6.076280in}}%
\pgfusepath{stroke}%
\end{pgfscope}%
\begin{pgfscope}%
\pgfpathrectangle{\pgfqpoint{0.800000in}{5.105882in}}{\pgfqpoint{2.407767in}{1.544118in}}%
\pgfusepath{clip}%
\pgfsetbuttcap%
\pgfsetroundjoin%
\pgfsetlinewidth{0.501875pt}%
\definecolor{currentstroke}{rgb}{0.260571,0.246922,0.522828}%
\pgfsetstrokecolor{currentstroke}%
\pgfsetdash{}{0pt}%
\pgfpathmoveto{\pgfqpoint{2.336607in}{6.076280in}}%
\pgfpathlineto{\pgfqpoint{2.283835in}{6.073322in}}%
\pgfusepath{stroke}%
\end{pgfscope}%
\begin{pgfscope}%
\pgfpathrectangle{\pgfqpoint{0.800000in}{5.105882in}}{\pgfqpoint{2.407767in}{1.544118in}}%
\pgfusepath{clip}%
\pgfsetbuttcap%
\pgfsetroundjoin%
\pgfsetlinewidth{0.501875pt}%
\definecolor{currentstroke}{rgb}{0.252194,0.269783,0.531579}%
\pgfsetstrokecolor{currentstroke}%
\pgfsetdash{}{0pt}%
\pgfpathmoveto{\pgfqpoint{2.283835in}{6.073322in}}%
\pgfpathlineto{\pgfqpoint{2.231180in}{6.069618in}}%
\pgfusepath{stroke}%
\end{pgfscope}%
\begin{pgfscope}%
\pgfpathrectangle{\pgfqpoint{0.800000in}{5.105882in}}{\pgfqpoint{2.407767in}{1.544118in}}%
\pgfusepath{clip}%
\pgfsetbuttcap%
\pgfsetroundjoin%
\pgfsetlinewidth{0.501875pt}%
\definecolor{currentstroke}{rgb}{0.253935,0.265254,0.529983}%
\pgfsetstrokecolor{currentstroke}%
\pgfsetdash{}{0pt}%
\pgfpathmoveto{\pgfqpoint{2.231180in}{6.069618in}}%
\pgfpathlineto{\pgfqpoint{2.178699in}{6.065028in}}%
\pgfusepath{stroke}%
\end{pgfscope}%
\begin{pgfscope}%
\pgfpathrectangle{\pgfqpoint{0.800000in}{5.105882in}}{\pgfqpoint{2.407767in}{1.544118in}}%
\pgfusepath{clip}%
\pgfsetbuttcap%
\pgfsetroundjoin%
\pgfsetlinewidth{0.501875pt}%
\definecolor{currentstroke}{rgb}{0.253935,0.265254,0.529983}%
\pgfsetstrokecolor{currentstroke}%
\pgfsetdash{}{0pt}%
\pgfpathmoveto{\pgfqpoint{2.178699in}{6.065028in}}%
\pgfpathlineto{\pgfqpoint{2.126448in}{6.059449in}}%
\pgfusepath{stroke}%
\end{pgfscope}%
\begin{pgfscope}%
\pgfpathrectangle{\pgfqpoint{0.800000in}{5.105882in}}{\pgfqpoint{2.407767in}{1.544118in}}%
\pgfusepath{clip}%
\pgfsetbuttcap%
\pgfsetroundjoin%
\pgfsetlinewidth{0.501875pt}%
\definecolor{currentstroke}{rgb}{0.250425,0.274290,0.533103}%
\pgfsetstrokecolor{currentstroke}%
\pgfsetdash{}{0pt}%
\pgfpathmoveto{\pgfqpoint{2.126448in}{6.059449in}}%
\pgfpathlineto{\pgfqpoint{2.074584in}{6.052582in}}%
\pgfusepath{stroke}%
\end{pgfscope}%
\begin{pgfscope}%
\pgfpathrectangle{\pgfqpoint{0.800000in}{5.105882in}}{\pgfqpoint{2.407767in}{1.544118in}}%
\pgfusepath{clip}%
\pgfsetbuttcap%
\pgfsetroundjoin%
\pgfsetlinewidth{0.501875pt}%
\definecolor{currentstroke}{rgb}{0.244972,0.287675,0.537260}%
\pgfsetstrokecolor{currentstroke}%
\pgfsetdash{}{0pt}%
\pgfpathmoveto{\pgfqpoint{2.074584in}{6.052582in}}%
\pgfpathlineto{\pgfqpoint{2.023142in}{6.044496in}}%
\pgfusepath{stroke}%
\end{pgfscope}%
\begin{pgfscope}%
\pgfpathrectangle{\pgfqpoint{0.800000in}{5.105882in}}{\pgfqpoint{2.407767in}{1.544118in}}%
\pgfusepath{clip}%
\pgfsetbuttcap%
\pgfsetroundjoin%
\pgfsetlinewidth{0.501875pt}%
\definecolor{currentstroke}{rgb}{0.255645,0.260703,0.528312}%
\pgfsetstrokecolor{currentstroke}%
\pgfsetdash{}{0pt}%
\pgfpathmoveto{\pgfqpoint{2.023142in}{6.044496in}}%
\pgfpathlineto{\pgfqpoint{1.972186in}{6.035258in}}%
\pgfusepath{stroke}%
\end{pgfscope}%
\begin{pgfscope}%
\pgfpathrectangle{\pgfqpoint{0.800000in}{5.105882in}}{\pgfqpoint{2.407767in}{1.544118in}}%
\pgfusepath{clip}%
\pgfsetbuttcap%
\pgfsetroundjoin%
\pgfsetlinewidth{0.501875pt}%
\definecolor{currentstroke}{rgb}{0.248629,0.278775,0.534556}%
\pgfsetstrokecolor{currentstroke}%
\pgfsetdash{}{0pt}%
\pgfpathmoveto{\pgfqpoint{1.972186in}{6.035258in}}%
\pgfpathlineto{\pgfqpoint{1.922060in}{6.024356in}}%
\pgfusepath{stroke}%
\end{pgfscope}%
\begin{pgfscope}%
\pgfpathrectangle{\pgfqpoint{0.800000in}{5.105882in}}{\pgfqpoint{2.407767in}{1.544118in}}%
\pgfusepath{clip}%
\pgfsetbuttcap%
\pgfsetroundjoin%
\pgfsetlinewidth{0.501875pt}%
\definecolor{currentstroke}{rgb}{0.258965,0.251537,0.524736}%
\pgfsetstrokecolor{currentstroke}%
\pgfsetdash{}{0pt}%
\pgfpathmoveto{\pgfqpoint{1.922060in}{6.024356in}}%
\pgfpathlineto{\pgfqpoint{1.872873in}{6.011794in}}%
\pgfusepath{stroke}%
\end{pgfscope}%
\begin{pgfscope}%
\pgfpathrectangle{\pgfqpoint{0.800000in}{5.105882in}}{\pgfqpoint{2.407767in}{1.544118in}}%
\pgfusepath{clip}%
\pgfsetbuttcap%
\pgfsetroundjoin%
\pgfsetlinewidth{0.501875pt}%
\definecolor{currentstroke}{rgb}{0.257322,0.256130,0.526563}%
\pgfsetstrokecolor{currentstroke}%
\pgfsetdash{}{0pt}%
\pgfpathmoveto{\pgfqpoint{1.872873in}{6.011794in}}%
\pgfpathlineto{\pgfqpoint{1.824491in}{5.998012in}}%
\pgfusepath{stroke}%
\end{pgfscope}%
\begin{pgfscope}%
\pgfpathrectangle{\pgfqpoint{0.800000in}{5.105882in}}{\pgfqpoint{2.407767in}{1.544118in}}%
\pgfusepath{clip}%
\pgfsetbuttcap%
\pgfsetroundjoin%
\pgfsetlinewidth{0.501875pt}%
\definecolor{currentstroke}{rgb}{0.271305,0.019942,0.347269}%
\pgfsetstrokecolor{currentstroke}%
\pgfsetdash{}{0pt}%
\pgfpathmoveto{\pgfqpoint{2.654046in}{6.121164in}}%
\pgfpathlineto{\pgfqpoint{2.601105in}{6.119965in}}%
\pgfusepath{stroke}%
\end{pgfscope}%
\begin{pgfscope}%
\pgfpathrectangle{\pgfqpoint{0.800000in}{5.105882in}}{\pgfqpoint{2.407767in}{1.544118in}}%
\pgfusepath{clip}%
\pgfsetbuttcap%
\pgfsetroundjoin%
\pgfsetlinewidth{0.501875pt}%
\definecolor{currentstroke}{rgb}{0.278791,0.062145,0.386592}%
\pgfsetstrokecolor{currentstroke}%
\pgfsetdash{}{0pt}%
\pgfpathmoveto{\pgfqpoint{2.601105in}{6.119965in}}%
\pgfpathlineto{\pgfqpoint{2.548186in}{6.118408in}}%
\pgfusepath{stroke}%
\end{pgfscope}%
\begin{pgfscope}%
\pgfpathrectangle{\pgfqpoint{0.800000in}{5.105882in}}{\pgfqpoint{2.407767in}{1.544118in}}%
\pgfusepath{clip}%
\pgfsetbuttcap%
\pgfsetroundjoin%
\pgfsetlinewidth{0.501875pt}%
\definecolor{currentstroke}{rgb}{0.283091,0.110553,0.431554}%
\pgfsetstrokecolor{currentstroke}%
\pgfsetdash{}{0pt}%
\pgfpathmoveto{\pgfqpoint{2.548186in}{6.118408in}}%
\pgfpathlineto{\pgfqpoint{2.495282in}{6.116641in}}%
\pgfusepath{stroke}%
\end{pgfscope}%
\begin{pgfscope}%
\pgfpathrectangle{\pgfqpoint{0.800000in}{5.105882in}}{\pgfqpoint{2.407767in}{1.544118in}}%
\pgfusepath{clip}%
\pgfsetbuttcap%
\pgfsetroundjoin%
\pgfsetlinewidth{0.501875pt}%
\definecolor{currentstroke}{rgb}{0.281887,0.150881,0.465405}%
\pgfsetstrokecolor{currentstroke}%
\pgfsetdash{}{0pt}%
\pgfpathmoveto{\pgfqpoint{2.495282in}{6.116641in}}%
\pgfpathlineto{\pgfqpoint{2.442400in}{6.114632in}}%
\pgfusepath{stroke}%
\end{pgfscope}%
\begin{pgfscope}%
\pgfpathrectangle{\pgfqpoint{0.800000in}{5.105882in}}{\pgfqpoint{2.407767in}{1.544118in}}%
\pgfusepath{clip}%
\pgfsetbuttcap%
\pgfsetroundjoin%
\pgfsetlinewidth{0.501875pt}%
\definecolor{currentstroke}{rgb}{0.276194,0.190074,0.493001}%
\pgfsetstrokecolor{currentstroke}%
\pgfsetdash{}{0pt}%
\pgfpathmoveto{\pgfqpoint{2.442400in}{6.114632in}}%
\pgfpathlineto{\pgfqpoint{2.389535in}{6.112437in}}%
\pgfusepath{stroke}%
\end{pgfscope}%
\begin{pgfscope}%
\pgfpathrectangle{\pgfqpoint{0.800000in}{5.105882in}}{\pgfqpoint{2.407767in}{1.544118in}}%
\pgfusepath{clip}%
\pgfsetbuttcap%
\pgfsetroundjoin%
\pgfsetlinewidth{0.501875pt}%
\definecolor{currentstroke}{rgb}{0.273006,0.204520,0.501721}%
\pgfsetstrokecolor{currentstroke}%
\pgfsetdash{}{0pt}%
\pgfpathmoveto{\pgfqpoint{2.389535in}{6.112437in}}%
\pgfpathlineto{\pgfqpoint{2.336719in}{6.109832in}}%
\pgfusepath{stroke}%
\end{pgfscope}%
\begin{pgfscope}%
\pgfpathrectangle{\pgfqpoint{0.800000in}{5.105882in}}{\pgfqpoint{2.407767in}{1.544118in}}%
\pgfusepath{clip}%
\pgfsetbuttcap%
\pgfsetroundjoin%
\pgfsetlinewidth{0.501875pt}%
\definecolor{currentstroke}{rgb}{0.271828,0.209303,0.504434}%
\pgfsetstrokecolor{currentstroke}%
\pgfsetdash{}{0pt}%
\pgfpathmoveto{\pgfqpoint{2.336719in}{6.109832in}}%
\pgfpathlineto{\pgfqpoint{2.283994in}{6.106562in}}%
\pgfusepath{stroke}%
\end{pgfscope}%
\begin{pgfscope}%
\pgfpathrectangle{\pgfqpoint{0.800000in}{5.105882in}}{\pgfqpoint{2.407767in}{1.544118in}}%
\pgfusepath{clip}%
\pgfsetbuttcap%
\pgfsetroundjoin%
\pgfsetlinewidth{0.501875pt}%
\definecolor{currentstroke}{rgb}{0.272594,0.025563,0.353093}%
\pgfsetstrokecolor{currentstroke}%
\pgfsetdash{}{0pt}%
\pgfpathmoveto{\pgfqpoint{2.654046in}{6.155910in}}%
\pgfpathlineto{\pgfqpoint{2.601117in}{6.154582in}}%
\pgfusepath{stroke}%
\end{pgfscope}%
\begin{pgfscope}%
\pgfpathrectangle{\pgfqpoint{0.800000in}{5.105882in}}{\pgfqpoint{2.407767in}{1.544118in}}%
\pgfusepath{clip}%
\pgfsetbuttcap%
\pgfsetroundjoin%
\pgfsetlinewidth{0.501875pt}%
\definecolor{currentstroke}{rgb}{0.277941,0.056324,0.381191}%
\pgfsetstrokecolor{currentstroke}%
\pgfsetdash{}{0pt}%
\pgfpathmoveto{\pgfqpoint{2.601117in}{6.154582in}}%
\pgfpathlineto{\pgfqpoint{2.548270in}{6.152275in}}%
\pgfusepath{stroke}%
\end{pgfscope}%
\begin{pgfscope}%
\pgfpathrectangle{\pgfqpoint{0.800000in}{5.105882in}}{\pgfqpoint{2.407767in}{1.544118in}}%
\pgfusepath{clip}%
\pgfsetbuttcap%
\pgfsetroundjoin%
\pgfsetlinewidth{0.501875pt}%
\definecolor{currentstroke}{rgb}{0.281446,0.084320,0.407414}%
\pgfsetstrokecolor{currentstroke}%
\pgfsetdash{}{0pt}%
\pgfpathmoveto{\pgfqpoint{2.548270in}{6.152275in}}%
\pgfpathlineto{\pgfqpoint{2.495434in}{6.149853in}}%
\pgfusepath{stroke}%
\end{pgfscope}%
\begin{pgfscope}%
\pgfpathrectangle{\pgfqpoint{0.800000in}{5.105882in}}{\pgfqpoint{2.407767in}{1.544118in}}%
\pgfusepath{clip}%
\pgfsetbuttcap%
\pgfsetroundjoin%
\pgfsetlinewidth{0.501875pt}%
\definecolor{currentstroke}{rgb}{0.283072,0.130895,0.449241}%
\pgfsetstrokecolor{currentstroke}%
\pgfsetdash{}{0pt}%
\pgfpathmoveto{\pgfqpoint{2.495434in}{6.149853in}}%
\pgfpathlineto{\pgfqpoint{2.442568in}{6.147666in}}%
\pgfusepath{stroke}%
\end{pgfscope}%
\begin{pgfscope}%
\pgfpathrectangle{\pgfqpoint{0.800000in}{5.105882in}}{\pgfqpoint{2.407767in}{1.544118in}}%
\pgfusepath{clip}%
\pgfsetbuttcap%
\pgfsetroundjoin%
\pgfsetlinewidth{0.501875pt}%
\definecolor{currentstroke}{rgb}{0.280868,0.160771,0.472899}%
\pgfsetstrokecolor{currentstroke}%
\pgfsetdash{}{0pt}%
\pgfpathmoveto{\pgfqpoint{2.442568in}{6.147666in}}%
\pgfpathlineto{\pgfqpoint{2.389733in}{6.145204in}}%
\pgfusepath{stroke}%
\end{pgfscope}%
\begin{pgfscope}%
\pgfpathrectangle{\pgfqpoint{0.800000in}{5.105882in}}{\pgfqpoint{2.407767in}{1.544118in}}%
\pgfusepath{clip}%
\pgfsetbuttcap%
\pgfsetroundjoin%
\pgfsetlinewidth{0.501875pt}%
\definecolor{currentstroke}{rgb}{0.278012,0.180367,0.486697}%
\pgfsetstrokecolor{currentstroke}%
\pgfsetdash{}{0pt}%
\pgfpathmoveto{\pgfqpoint{2.389733in}{6.145204in}}%
\pgfpathlineto{\pgfqpoint{2.336985in}{6.142105in}}%
\pgfusepath{stroke}%
\end{pgfscope}%
\begin{pgfscope}%
\pgfpathrectangle{\pgfqpoint{0.800000in}{5.105882in}}{\pgfqpoint{2.407767in}{1.544118in}}%
\pgfusepath{clip}%
\pgfsetbuttcap%
\pgfsetroundjoin%
\pgfsetlinewidth{0.501875pt}%
\definecolor{currentstroke}{rgb}{0.271305,0.019942,0.347269}%
\pgfsetstrokecolor{currentstroke}%
\pgfsetdash{}{0pt}%
\pgfpathmoveto{\pgfqpoint{2.654046in}{6.190656in}}%
\pgfpathlineto{\pgfqpoint{2.601267in}{6.187784in}}%
\pgfusepath{stroke}%
\end{pgfscope}%
\begin{pgfscope}%
\pgfpathrectangle{\pgfqpoint{0.800000in}{5.105882in}}{\pgfqpoint{2.407767in}{1.544118in}}%
\pgfusepath{clip}%
\pgfsetbuttcap%
\pgfsetroundjoin%
\pgfsetlinewidth{0.501875pt}%
\definecolor{currentstroke}{rgb}{0.276022,0.044167,0.370164}%
\pgfsetstrokecolor{currentstroke}%
\pgfsetdash{}{0pt}%
\pgfpathmoveto{\pgfqpoint{2.601267in}{6.187784in}}%
\pgfpathlineto{\pgfqpoint{2.548484in}{6.184932in}}%
\pgfusepath{stroke}%
\end{pgfscope}%
\begin{pgfscope}%
\pgfpathrectangle{\pgfqpoint{0.800000in}{5.105882in}}{\pgfqpoint{2.407767in}{1.544118in}}%
\pgfusepath{clip}%
\pgfsetbuttcap%
\pgfsetroundjoin%
\pgfsetlinewidth{0.501875pt}%
\definecolor{currentstroke}{rgb}{0.279566,0.067836,0.391917}%
\pgfsetstrokecolor{currentstroke}%
\pgfsetdash{}{0pt}%
\pgfpathmoveto{\pgfqpoint{2.548484in}{6.184932in}}%
\pgfpathlineto{\pgfqpoint{2.495638in}{6.182559in}}%
\pgfusepath{stroke}%
\end{pgfscope}%
\begin{pgfscope}%
\pgfpathrectangle{\pgfqpoint{0.800000in}{5.105882in}}{\pgfqpoint{2.407767in}{1.544118in}}%
\pgfusepath{clip}%
\pgfsetbuttcap%
\pgfsetroundjoin%
\pgfsetlinewidth{0.501875pt}%
\definecolor{currentstroke}{rgb}{0.283091,0.110553,0.431554}%
\pgfsetstrokecolor{currentstroke}%
\pgfsetdash{}{0pt}%
\pgfpathmoveto{\pgfqpoint{2.495638in}{6.182559in}}%
\pgfpathlineto{\pgfqpoint{2.442799in}{6.180127in}}%
\pgfusepath{stroke}%
\end{pgfscope}%
\begin{pgfscope}%
\pgfpathrectangle{\pgfqpoint{0.800000in}{5.105882in}}{\pgfqpoint{2.407767in}{1.544118in}}%
\pgfusepath{clip}%
\pgfsetbuttcap%
\pgfsetroundjoin%
\pgfsetlinewidth{0.501875pt}%
\definecolor{currentstroke}{rgb}{0.283187,0.125848,0.444960}%
\pgfsetstrokecolor{currentstroke}%
\pgfsetdash{}{0pt}%
\pgfpathmoveto{\pgfqpoint{2.442799in}{6.180127in}}%
\pgfpathlineto{\pgfqpoint{2.389981in}{6.177511in}}%
\pgfusepath{stroke}%
\end{pgfscope}%
\begin{pgfscope}%
\pgfpathrectangle{\pgfqpoint{0.800000in}{5.105882in}}{\pgfqpoint{2.407767in}{1.544118in}}%
\pgfusepath{clip}%
\pgfsetbuttcap%
\pgfsetroundjoin%
\pgfsetlinewidth{0.501875pt}%
\definecolor{currentstroke}{rgb}{0.282623,0.140926,0.457517}%
\pgfsetstrokecolor{currentstroke}%
\pgfsetdash{}{0pt}%
\pgfpathmoveto{\pgfqpoint{2.389981in}{6.177511in}}%
\pgfpathlineto{\pgfqpoint{2.337200in}{6.174611in}}%
\pgfusepath{stroke}%
\end{pgfscope}%
\begin{pgfscope}%
\pgfpathrectangle{\pgfqpoint{0.800000in}{5.105882in}}{\pgfqpoint{2.407767in}{1.544118in}}%
\pgfusepath{clip}%
\pgfsetbuttcap%
\pgfsetroundjoin%
\pgfsetlinewidth{0.501875pt}%
\definecolor{currentstroke}{rgb}{0.280868,0.160771,0.472899}%
\pgfsetstrokecolor{currentstroke}%
\pgfsetdash{}{0pt}%
\pgfpathmoveto{\pgfqpoint{2.337200in}{6.174611in}}%
\pgfpathlineto{\pgfqpoint{2.284551in}{6.170897in}}%
\pgfusepath{stroke}%
\end{pgfscope}%
\begin{pgfscope}%
\pgfpathrectangle{\pgfqpoint{0.800000in}{5.105882in}}{\pgfqpoint{2.407767in}{1.544118in}}%
\pgfusepath{clip}%
\pgfsetbuttcap%
\pgfsetroundjoin%
\pgfsetlinewidth{0.501875pt}%
\definecolor{currentstroke}{rgb}{0.281412,0.155834,0.469201}%
\pgfsetstrokecolor{currentstroke}%
\pgfsetdash{}{0pt}%
\pgfpathmoveto{\pgfqpoint{2.284551in}{6.170897in}}%
\pgfpathlineto{\pgfqpoint{2.232163in}{6.165903in}}%
\pgfusepath{stroke}%
\end{pgfscope}%
\begin{pgfscope}%
\pgfpathrectangle{\pgfqpoint{0.800000in}{5.105882in}}{\pgfqpoint{2.407767in}{1.544118in}}%
\pgfusepath{clip}%
\pgfsetbuttcap%
\pgfsetroundjoin%
\pgfsetlinewidth{0.501875pt}%
\definecolor{currentstroke}{rgb}{0.280255,0.165693,0.476498}%
\pgfsetstrokecolor{currentstroke}%
\pgfsetdash{}{0pt}%
\pgfpathmoveto{\pgfqpoint{2.232163in}{6.165903in}}%
\pgfpathlineto{\pgfqpoint{2.180260in}{6.159191in}}%
\pgfusepath{stroke}%
\end{pgfscope}%
\begin{pgfscope}%
\pgfpathrectangle{\pgfqpoint{0.800000in}{5.105882in}}{\pgfqpoint{2.407767in}{1.544118in}}%
\pgfusepath{clip}%
\pgfsetbuttcap%
\pgfsetroundjoin%
\pgfsetlinewidth{0.501875pt}%
\definecolor{currentstroke}{rgb}{0.278012,0.180367,0.486697}%
\pgfsetstrokecolor{currentstroke}%
\pgfsetdash{}{0pt}%
\pgfpathmoveto{\pgfqpoint{2.180260in}{6.159191in}}%
\pgfpathlineto{\pgfqpoint{2.129154in}{6.150336in}}%
\pgfusepath{stroke}%
\end{pgfscope}%
\begin{pgfscope}%
\pgfpathrectangle{\pgfqpoint{0.800000in}{5.105882in}}{\pgfqpoint{2.407767in}{1.544118in}}%
\pgfusepath{clip}%
\pgfsetbuttcap%
\pgfsetroundjoin%
\pgfsetlinewidth{0.501875pt}%
\definecolor{currentstroke}{rgb}{0.273006,0.204520,0.501721}%
\pgfsetstrokecolor{currentstroke}%
\pgfsetdash{}{0pt}%
\pgfpathmoveto{\pgfqpoint{2.129154in}{6.150336in}}%
\pgfpathlineto{\pgfqpoint{2.078743in}{6.139935in}}%
\pgfusepath{stroke}%
\end{pgfscope}%
\begin{pgfscope}%
\pgfpathrectangle{\pgfqpoint{0.800000in}{5.105882in}}{\pgfqpoint{2.407767in}{1.544118in}}%
\pgfusepath{clip}%
\pgfsetbuttcap%
\pgfsetroundjoin%
\pgfsetlinewidth{0.501875pt}%
\definecolor{currentstroke}{rgb}{0.278012,0.180367,0.486697}%
\pgfsetstrokecolor{currentstroke}%
\pgfsetdash{}{0pt}%
\pgfpathmoveto{\pgfqpoint{2.078743in}{6.139935in}}%
\pgfpathlineto{\pgfqpoint{2.029185in}{6.128012in}}%
\pgfusepath{stroke}%
\end{pgfscope}%
\begin{pgfscope}%
\pgfpathrectangle{\pgfqpoint{0.800000in}{5.105882in}}{\pgfqpoint{2.407767in}{1.544118in}}%
\pgfusepath{clip}%
\pgfsetbuttcap%
\pgfsetroundjoin%
\pgfsetlinewidth{0.501875pt}%
\definecolor{currentstroke}{rgb}{0.271828,0.209303,0.504434}%
\pgfsetstrokecolor{currentstroke}%
\pgfsetdash{}{0pt}%
\pgfpathmoveto{\pgfqpoint{2.029185in}{6.128012in}}%
\pgfpathlineto{\pgfqpoint{1.981153in}{6.113789in}}%
\pgfusepath{stroke}%
\end{pgfscope}%
\begin{pgfscope}%
\pgfpathrectangle{\pgfqpoint{0.800000in}{5.105882in}}{\pgfqpoint{2.407767in}{1.544118in}}%
\pgfusepath{clip}%
\pgfsetbuttcap%
\pgfsetroundjoin%
\pgfsetlinewidth{0.501875pt}%
\definecolor{currentstroke}{rgb}{0.274128,0.199721,0.498911}%
\pgfsetstrokecolor{currentstroke}%
\pgfsetdash{}{0pt}%
\pgfpathmoveto{\pgfqpoint{1.981153in}{6.113789in}}%
\pgfpathlineto{\pgfqpoint{1.934688in}{6.097535in}}%
\pgfusepath{stroke}%
\end{pgfscope}%
\begin{pgfscope}%
\pgfpathrectangle{\pgfqpoint{0.800000in}{5.105882in}}{\pgfqpoint{2.407767in}{1.544118in}}%
\pgfusepath{clip}%
\pgfsetbuttcap%
\pgfsetroundjoin%
\pgfsetlinewidth{0.501875pt}%
\definecolor{currentstroke}{rgb}{0.267968,0.223549,0.512008}%
\pgfsetstrokecolor{currentstroke}%
\pgfsetdash{}{0pt}%
\pgfpathmoveto{\pgfqpoint{1.934688in}{6.097535in}}%
\pgfpathlineto{\pgfqpoint{1.889055in}{6.080327in}}%
\pgfusepath{stroke}%
\end{pgfscope}%
\begin{pgfscope}%
\pgfpathrectangle{\pgfqpoint{0.800000in}{5.105882in}}{\pgfqpoint{2.407767in}{1.544118in}}%
\pgfusepath{clip}%
\pgfsetbuttcap%
\pgfsetroundjoin%
\pgfsetlinewidth{0.501875pt}%
\definecolor{currentstroke}{rgb}{0.269308,0.218818,0.509577}%
\pgfsetstrokecolor{currentstroke}%
\pgfsetdash{}{0pt}%
\pgfpathmoveto{\pgfqpoint{1.889055in}{6.080327in}}%
\pgfpathlineto{\pgfqpoint{1.844745in}{6.061793in}}%
\pgfusepath{stroke}%
\end{pgfscope}%
\begin{pgfscope}%
\pgfpathrectangle{\pgfqpoint{0.800000in}{5.105882in}}{\pgfqpoint{2.407767in}{1.544118in}}%
\pgfusepath{clip}%
\pgfsetbuttcap%
\pgfsetroundjoin%
\pgfsetlinewidth{0.501875pt}%
\definecolor{currentstroke}{rgb}{0.258965,0.251537,0.524736}%
\pgfsetstrokecolor{currentstroke}%
\pgfsetdash{}{0pt}%
\pgfpathmoveto{\pgfqpoint{1.844745in}{6.061793in}}%
\pgfpathlineto{\pgfqpoint{1.801950in}{6.041813in}}%
\pgfusepath{stroke}%
\end{pgfscope}%
\begin{pgfscope}%
\pgfpathrectangle{\pgfqpoint{0.800000in}{5.105882in}}{\pgfqpoint{2.407767in}{1.544118in}}%
\pgfusepath{clip}%
\pgfsetbuttcap%
\pgfsetroundjoin%
\pgfsetlinewidth{0.501875pt}%
\definecolor{currentstroke}{rgb}{0.268510,0.009605,0.335427}%
\pgfsetstrokecolor{currentstroke}%
\pgfsetdash{}{0pt}%
\pgfpathmoveto{\pgfqpoint{2.654046in}{6.225402in}}%
\pgfpathlineto{\pgfqpoint{2.601266in}{6.222638in}}%
\pgfusepath{stroke}%
\end{pgfscope}%
\begin{pgfscope}%
\pgfpathrectangle{\pgfqpoint{0.800000in}{5.105882in}}{\pgfqpoint{2.407767in}{1.544118in}}%
\pgfusepath{clip}%
\pgfsetbuttcap%
\pgfsetroundjoin%
\pgfsetlinewidth{0.501875pt}%
\definecolor{currentstroke}{rgb}{0.273809,0.031497,0.358853}%
\pgfsetstrokecolor{currentstroke}%
\pgfsetdash{}{0pt}%
\pgfpathmoveto{\pgfqpoint{2.601266in}{6.222638in}}%
\pgfpathlineto{\pgfqpoint{2.548527in}{6.219459in}}%
\pgfusepath{stroke}%
\end{pgfscope}%
\begin{pgfscope}%
\pgfpathrectangle{\pgfqpoint{0.800000in}{5.105882in}}{\pgfqpoint{2.407767in}{1.544118in}}%
\pgfusepath{clip}%
\pgfsetbuttcap%
\pgfsetroundjoin%
\pgfsetlinewidth{0.501875pt}%
\definecolor{currentstroke}{rgb}{0.277018,0.050344,0.375715}%
\pgfsetstrokecolor{currentstroke}%
\pgfsetdash{}{0pt}%
\pgfpathmoveto{\pgfqpoint{2.548527in}{6.219459in}}%
\pgfpathlineto{\pgfqpoint{2.495782in}{6.216311in}}%
\pgfusepath{stroke}%
\end{pgfscope}%
\begin{pgfscope}%
\pgfpathrectangle{\pgfqpoint{0.800000in}{5.105882in}}{\pgfqpoint{2.407767in}{1.544118in}}%
\pgfusepath{clip}%
\pgfsetbuttcap%
\pgfsetroundjoin%
\pgfsetlinewidth{0.501875pt}%
\definecolor{currentstroke}{rgb}{0.281446,0.084320,0.407414}%
\pgfsetstrokecolor{currentstroke}%
\pgfsetdash{}{0pt}%
\pgfpathmoveto{\pgfqpoint{2.495782in}{6.216311in}}%
\pgfpathlineto{\pgfqpoint{2.442993in}{6.213491in}}%
\pgfusepath{stroke}%
\end{pgfscope}%
\begin{pgfscope}%
\pgfpathrectangle{\pgfqpoint{0.800000in}{5.105882in}}{\pgfqpoint{2.407767in}{1.544118in}}%
\pgfusepath{clip}%
\pgfsetbuttcap%
\pgfsetroundjoin%
\pgfsetlinewidth{0.501875pt}%
\definecolor{currentstroke}{rgb}{0.282910,0.105393,0.426902}%
\pgfsetstrokecolor{currentstroke}%
\pgfsetdash{}{0pt}%
\pgfpathmoveto{\pgfqpoint{2.442993in}{6.213491in}}%
\pgfpathlineto{\pgfqpoint{2.390182in}{6.210824in}}%
\pgfusepath{stroke}%
\end{pgfscope}%
\begin{pgfscope}%
\pgfpathrectangle{\pgfqpoint{0.800000in}{5.105882in}}{\pgfqpoint{2.407767in}{1.544118in}}%
\pgfusepath{clip}%
\pgfsetbuttcap%
\pgfsetroundjoin%
\pgfsetlinewidth{0.501875pt}%
\definecolor{currentstroke}{rgb}{0.267004,0.004874,0.329415}%
\pgfsetstrokecolor{currentstroke}%
\pgfsetdash{}{0pt}%
\pgfpathmoveto{\pgfqpoint{2.654046in}{6.260149in}}%
\pgfpathlineto{\pgfqpoint{2.602702in}{6.255198in}}%
\pgfusepath{stroke}%
\end{pgfscope}%
\begin{pgfscope}%
\pgfpathrectangle{\pgfqpoint{0.800000in}{5.105882in}}{\pgfqpoint{2.407767in}{1.544118in}}%
\pgfusepath{clip}%
\pgfsetbuttcap%
\pgfsetroundjoin%
\pgfsetlinewidth{0.501875pt}%
\definecolor{currentstroke}{rgb}{0.273809,0.031497,0.358853}%
\pgfsetstrokecolor{currentstroke}%
\pgfsetdash{}{0pt}%
\pgfpathmoveto{\pgfqpoint{2.602702in}{6.255198in}}%
\pgfpathlineto{\pgfqpoint{2.550014in}{6.251714in}}%
\pgfusepath{stroke}%
\end{pgfscope}%
\begin{pgfscope}%
\pgfpathrectangle{\pgfqpoint{0.800000in}{5.105882in}}{\pgfqpoint{2.407767in}{1.544118in}}%
\pgfusepath{clip}%
\pgfsetbuttcap%
\pgfsetroundjoin%
\pgfsetlinewidth{0.501875pt}%
\definecolor{currentstroke}{rgb}{0.277018,0.050344,0.375715}%
\pgfsetstrokecolor{currentstroke}%
\pgfsetdash{}{0pt}%
\pgfpathmoveto{\pgfqpoint{2.550014in}{6.251714in}}%
\pgfpathlineto{\pgfqpoint{2.497272in}{6.248622in}}%
\pgfusepath{stroke}%
\end{pgfscope}%
\begin{pgfscope}%
\pgfpathrectangle{\pgfqpoint{0.800000in}{5.105882in}}{\pgfqpoint{2.407767in}{1.544118in}}%
\pgfusepath{clip}%
\pgfsetbuttcap%
\pgfsetroundjoin%
\pgfsetlinewidth{0.501875pt}%
\definecolor{currentstroke}{rgb}{0.280267,0.073417,0.397163}%
\pgfsetstrokecolor{currentstroke}%
\pgfsetdash{}{0pt}%
\pgfpathmoveto{\pgfqpoint{2.497272in}{6.248622in}}%
\pgfpathlineto{\pgfqpoint{2.444451in}{6.246052in}}%
\pgfusepath{stroke}%
\end{pgfscope}%
\begin{pgfscope}%
\pgfpathrectangle{\pgfqpoint{0.800000in}{5.105882in}}{\pgfqpoint{2.407767in}{1.544118in}}%
\pgfusepath{clip}%
\pgfsetbuttcap%
\pgfsetroundjoin%
\pgfsetlinewidth{0.501875pt}%
\definecolor{currentstroke}{rgb}{0.281446,0.084320,0.407414}%
\pgfsetstrokecolor{currentstroke}%
\pgfsetdash{}{0pt}%
\pgfpathmoveto{\pgfqpoint{2.444451in}{6.246052in}}%
\pgfpathlineto{\pgfqpoint{2.391656in}{6.243267in}}%
\pgfusepath{stroke}%
\end{pgfscope}%
\begin{pgfscope}%
\pgfpathrectangle{\pgfqpoint{0.800000in}{5.105882in}}{\pgfqpoint{2.407767in}{1.544118in}}%
\pgfusepath{clip}%
\pgfsetbuttcap%
\pgfsetroundjoin%
\pgfsetlinewidth{0.501875pt}%
\definecolor{currentstroke}{rgb}{0.283229,0.120777,0.440584}%
\pgfsetstrokecolor{currentstroke}%
\pgfsetdash{}{0pt}%
\pgfpathmoveto{\pgfqpoint{2.391656in}{6.243267in}}%
\pgfpathlineto{\pgfqpoint{2.338881in}{6.240316in}}%
\pgfusepath{stroke}%
\end{pgfscope}%
\begin{pgfscope}%
\pgfpathrectangle{\pgfqpoint{0.800000in}{5.105882in}}{\pgfqpoint{2.407767in}{1.544118in}}%
\pgfusepath{clip}%
\pgfsetbuttcap%
\pgfsetroundjoin%
\pgfsetlinewidth{0.501875pt}%
\definecolor{currentstroke}{rgb}{0.281924,0.089666,0.412415}%
\pgfsetstrokecolor{currentstroke}%
\pgfsetdash{}{0pt}%
\pgfpathmoveto{\pgfqpoint{2.338881in}{6.240316in}}%
\pgfpathlineto{\pgfqpoint{2.286185in}{6.236915in}}%
\pgfusepath{stroke}%
\end{pgfscope}%
\begin{pgfscope}%
\pgfpathrectangle{\pgfqpoint{0.800000in}{5.105882in}}{\pgfqpoint{2.407767in}{1.544118in}}%
\pgfusepath{clip}%
\pgfsetbuttcap%
\pgfsetroundjoin%
\pgfsetlinewidth{0.501875pt}%
\definecolor{currentstroke}{rgb}{0.282910,0.105393,0.426902}%
\pgfsetstrokecolor{currentstroke}%
\pgfsetdash{}{0pt}%
\pgfpathmoveto{\pgfqpoint{2.286185in}{6.236915in}}%
\pgfpathlineto{\pgfqpoint{2.233824in}{6.231922in}}%
\pgfusepath{stroke}%
\end{pgfscope}%
\begin{pgfscope}%
\pgfpathrectangle{\pgfqpoint{0.800000in}{5.105882in}}{\pgfqpoint{2.407767in}{1.544118in}}%
\pgfusepath{clip}%
\pgfsetbuttcap%
\pgfsetroundjoin%
\pgfsetlinewidth{0.501875pt}%
\definecolor{currentstroke}{rgb}{0.283229,0.120777,0.440584}%
\pgfsetstrokecolor{currentstroke}%
\pgfsetdash{}{0pt}%
\pgfpathmoveto{\pgfqpoint{2.233824in}{6.231922in}}%
\pgfpathlineto{\pgfqpoint{2.182166in}{6.224614in}}%
\pgfusepath{stroke}%
\end{pgfscope}%
\begin{pgfscope}%
\pgfpathrectangle{\pgfqpoint{0.800000in}{5.105882in}}{\pgfqpoint{2.407767in}{1.544118in}}%
\pgfusepath{clip}%
\pgfsetbuttcap%
\pgfsetroundjoin%
\pgfsetlinewidth{0.501875pt}%
\definecolor{currentstroke}{rgb}{0.283197,0.115680,0.436115}%
\pgfsetstrokecolor{currentstroke}%
\pgfsetdash{}{0pt}%
\pgfpathmoveto{\pgfqpoint{2.182166in}{6.224614in}}%
\pgfpathlineto{\pgfqpoint{2.131434in}{6.214997in}}%
\pgfusepath{stroke}%
\end{pgfscope}%
\begin{pgfscope}%
\pgfpathrectangle{\pgfqpoint{0.800000in}{5.105882in}}{\pgfqpoint{2.407767in}{1.544118in}}%
\pgfusepath{clip}%
\pgfsetbuttcap%
\pgfsetroundjoin%
\pgfsetlinewidth{0.501875pt}%
\definecolor{currentstroke}{rgb}{0.282884,0.135920,0.453427}%
\pgfsetstrokecolor{currentstroke}%
\pgfsetdash{}{0pt}%
\pgfpathmoveto{\pgfqpoint{2.131434in}{6.214997in}}%
\pgfpathlineto{\pgfqpoint{2.082097in}{6.202740in}}%
\pgfusepath{stroke}%
\end{pgfscope}%
\begin{pgfscope}%
\pgfpathrectangle{\pgfqpoint{0.800000in}{5.105882in}}{\pgfqpoint{2.407767in}{1.544118in}}%
\pgfusepath{clip}%
\pgfsetbuttcap%
\pgfsetroundjoin%
\pgfsetlinewidth{0.501875pt}%
\definecolor{currentstroke}{rgb}{0.282623,0.140926,0.457517}%
\pgfsetstrokecolor{currentstroke}%
\pgfsetdash{}{0pt}%
\pgfpathmoveto{\pgfqpoint{2.082097in}{6.202740in}}%
\pgfpathlineto{\pgfqpoint{2.034902in}{6.187459in}}%
\pgfusepath{stroke}%
\end{pgfscope}%
\begin{pgfscope}%
\pgfpathrectangle{\pgfqpoint{0.800000in}{5.105882in}}{\pgfqpoint{2.407767in}{1.544118in}}%
\pgfusepath{clip}%
\pgfsetbuttcap%
\pgfsetroundjoin%
\pgfsetlinewidth{0.501875pt}%
\definecolor{currentstroke}{rgb}{0.280868,0.160771,0.472899}%
\pgfsetstrokecolor{currentstroke}%
\pgfsetdash{}{0pt}%
\pgfpathmoveto{\pgfqpoint{2.034902in}{6.187459in}}%
\pgfpathlineto{\pgfqpoint{1.989221in}{6.170335in}}%
\pgfusepath{stroke}%
\end{pgfscope}%
\begin{pgfscope}%
\pgfpathrectangle{\pgfqpoint{0.800000in}{5.105882in}}{\pgfqpoint{2.407767in}{1.544118in}}%
\pgfusepath{clip}%
\pgfsetbuttcap%
\pgfsetroundjoin%
\pgfsetlinewidth{0.501875pt}%
\definecolor{currentstroke}{rgb}{0.281412,0.155834,0.469201}%
\pgfsetstrokecolor{currentstroke}%
\pgfsetdash{}{0pt}%
\pgfpathmoveto{\pgfqpoint{1.989221in}{6.170335in}}%
\pgfpathlineto{\pgfqpoint{1.945779in}{6.151085in}}%
\pgfusepath{stroke}%
\end{pgfscope}%
\begin{pgfscope}%
\pgfpathrectangle{\pgfqpoint{0.800000in}{5.105882in}}{\pgfqpoint{2.407767in}{1.544118in}}%
\pgfusepath{clip}%
\pgfsetbuttcap%
\pgfsetroundjoin%
\pgfsetlinewidth{0.501875pt}%
\definecolor{currentstroke}{rgb}{0.273006,0.204520,0.501721}%
\pgfsetstrokecolor{currentstroke}%
\pgfsetdash{}{0pt}%
\pgfpathmoveto{\pgfqpoint{1.945779in}{6.151085in}}%
\pgfpathlineto{\pgfqpoint{1.904331in}{6.129972in}}%
\pgfusepath{stroke}%
\end{pgfscope}%
\begin{pgfscope}%
\pgfpathrectangle{\pgfqpoint{0.800000in}{5.105882in}}{\pgfqpoint{2.407767in}{1.544118in}}%
\pgfusepath{clip}%
\pgfsetbuttcap%
\pgfsetroundjoin%
\pgfsetlinewidth{0.501875pt}%
\definecolor{currentstroke}{rgb}{0.268510,0.009605,0.335427}%
\pgfsetstrokecolor{currentstroke}%
\pgfsetdash{}{0pt}%
\pgfpathmoveto{\pgfqpoint{2.650581in}{6.302700in}}%
\pgfpathlineto{\pgfqpoint{2.597813in}{6.300366in}}%
\pgfusepath{stroke}%
\end{pgfscope}%
\begin{pgfscope}%
\pgfpathrectangle{\pgfqpoint{0.800000in}{5.105882in}}{\pgfqpoint{2.407767in}{1.544118in}}%
\pgfusepath{clip}%
\pgfsetbuttcap%
\pgfsetroundjoin%
\pgfsetlinewidth{0.501875pt}%
\definecolor{currentstroke}{rgb}{0.272594,0.025563,0.353093}%
\pgfsetstrokecolor{currentstroke}%
\pgfsetdash{}{0pt}%
\pgfpathmoveto{\pgfqpoint{2.597813in}{6.300366in}}%
\pgfpathlineto{\pgfqpoint{2.545685in}{6.294895in}}%
\pgfusepath{stroke}%
\end{pgfscope}%
\begin{pgfscope}%
\pgfpathrectangle{\pgfqpoint{0.800000in}{5.105882in}}{\pgfqpoint{2.407767in}{1.544118in}}%
\pgfusepath{clip}%
\pgfsetbuttcap%
\pgfsetroundjoin%
\pgfsetlinewidth{0.501875pt}%
\definecolor{currentstroke}{rgb}{0.271305,0.019942,0.347269}%
\pgfsetstrokecolor{currentstroke}%
\pgfsetdash{}{0pt}%
\pgfpathmoveto{\pgfqpoint{2.545685in}{6.294895in}}%
\pgfpathlineto{\pgfqpoint{2.493541in}{6.289479in}}%
\pgfusepath{stroke}%
\end{pgfscope}%
\begin{pgfscope}%
\pgfpathrectangle{\pgfqpoint{0.800000in}{5.105882in}}{\pgfqpoint{2.407767in}{1.544118in}}%
\pgfusepath{clip}%
\pgfsetbuttcap%
\pgfsetroundjoin%
\pgfsetlinewidth{0.501875pt}%
\definecolor{currentstroke}{rgb}{0.277018,0.050344,0.375715}%
\pgfsetstrokecolor{currentstroke}%
\pgfsetdash{}{0pt}%
\pgfpathmoveto{\pgfqpoint{2.493541in}{6.289479in}}%
\pgfpathlineto{\pgfqpoint{2.440893in}{6.285753in}}%
\pgfusepath{stroke}%
\end{pgfscope}%
\begin{pgfscope}%
\pgfpathrectangle{\pgfqpoint{0.800000in}{5.105882in}}{\pgfqpoint{2.407767in}{1.544118in}}%
\pgfusepath{clip}%
\pgfsetbuttcap%
\pgfsetroundjoin%
\pgfsetlinewidth{0.501875pt}%
\definecolor{currentstroke}{rgb}{0.277018,0.050344,0.375715}%
\pgfsetstrokecolor{currentstroke}%
\pgfsetdash{}{0pt}%
\pgfpathmoveto{\pgfqpoint{2.440893in}{6.285753in}}%
\pgfpathlineto{\pgfqpoint{2.388100in}{6.283115in}}%
\pgfusepath{stroke}%
\end{pgfscope}%
\begin{pgfscope}%
\pgfpathrectangle{\pgfqpoint{0.800000in}{5.105882in}}{\pgfqpoint{2.407767in}{1.544118in}}%
\pgfusepath{clip}%
\pgfsetbuttcap%
\pgfsetroundjoin%
\pgfsetlinewidth{0.501875pt}%
\definecolor{currentstroke}{rgb}{0.282327,0.094955,0.417331}%
\pgfsetstrokecolor{currentstroke}%
\pgfsetdash{}{0pt}%
\pgfpathmoveto{\pgfqpoint{2.270160in}{6.273478in}}%
\pgfpathlineto{\pgfqpoint{2.217905in}{6.267940in}}%
\pgfusepath{stroke}%
\end{pgfscope}%
\begin{pgfscope}%
\pgfpathrectangle{\pgfqpoint{0.800000in}{5.105882in}}{\pgfqpoint{2.407767in}{1.544118in}}%
\pgfusepath{clip}%
\pgfsetbuttcap%
\pgfsetroundjoin%
\pgfsetlinewidth{0.501875pt}%
\definecolor{currentstroke}{rgb}{0.281924,0.089666,0.412415}%
\pgfsetstrokecolor{currentstroke}%
\pgfsetdash{}{0pt}%
\pgfpathmoveto{\pgfqpoint{2.217905in}{6.267940in}}%
\pgfpathlineto{\pgfqpoint{2.166424in}{6.260149in}}%
\pgfusepath{stroke}%
\end{pgfscope}%
\begin{pgfscope}%
\pgfpathrectangle{\pgfqpoint{0.800000in}{5.105882in}}{\pgfqpoint{2.407767in}{1.544118in}}%
\pgfusepath{clip}%
\pgfsetbuttcap%
\pgfsetroundjoin%
\pgfsetlinewidth{0.501875pt}%
\definecolor{currentstroke}{rgb}{0.282656,0.100196,0.422160}%
\pgfsetstrokecolor{currentstroke}%
\pgfsetdash{}{0pt}%
\pgfpathmoveto{\pgfqpoint{2.166424in}{6.260149in}}%
\pgfpathlineto{\pgfqpoint{2.116425in}{6.249138in}}%
\pgfusepath{stroke}%
\end{pgfscope}%
\begin{pgfscope}%
\pgfpathrectangle{\pgfqpoint{0.800000in}{5.105882in}}{\pgfqpoint{2.407767in}{1.544118in}}%
\pgfusepath{clip}%
\pgfsetbuttcap%
\pgfsetroundjoin%
\pgfsetlinewidth{0.501875pt}%
\definecolor{currentstroke}{rgb}{0.282327,0.094955,0.417331}%
\pgfsetstrokecolor{currentstroke}%
\pgfsetdash{}{0pt}%
\pgfpathmoveto{\pgfqpoint{2.116425in}{6.249138in}}%
\pgfpathlineto{\pgfqpoint{2.068296in}{6.235237in}}%
\pgfusepath{stroke}%
\end{pgfscope}%
\begin{pgfscope}%
\pgfpathrectangle{\pgfqpoint{0.800000in}{5.105882in}}{\pgfqpoint{2.407767in}{1.544118in}}%
\pgfusepath{clip}%
\pgfsetbuttcap%
\pgfsetroundjoin%
\pgfsetlinewidth{0.501875pt}%
\definecolor{currentstroke}{rgb}{0.282327,0.094955,0.417331}%
\pgfsetstrokecolor{currentstroke}%
\pgfsetdash{}{0pt}%
\pgfpathmoveto{\pgfqpoint{2.068296in}{6.235237in}}%
\pgfpathlineto{\pgfqpoint{2.022325in}{6.218538in}}%
\pgfusepath{stroke}%
\end{pgfscope}%
\begin{pgfscope}%
\pgfpathrectangle{\pgfqpoint{0.800000in}{5.105882in}}{\pgfqpoint{2.407767in}{1.544118in}}%
\pgfusepath{clip}%
\pgfsetbuttcap%
\pgfsetroundjoin%
\pgfsetlinewidth{0.501875pt}%
\definecolor{currentstroke}{rgb}{0.283187,0.125848,0.444960}%
\pgfsetstrokecolor{currentstroke}%
\pgfsetdash{}{0pt}%
\pgfpathmoveto{\pgfqpoint{2.022325in}{6.218538in}}%
\pgfpathlineto{\pgfqpoint{1.978835in}{6.199232in}}%
\pgfusepath{stroke}%
\end{pgfscope}%
\begin{pgfscope}%
\pgfpathrectangle{\pgfqpoint{0.800000in}{5.105882in}}{\pgfqpoint{2.407767in}{1.544118in}}%
\pgfusepath{clip}%
\pgfsetbuttcap%
\pgfsetroundjoin%
\pgfsetlinewidth{0.501875pt}%
\definecolor{currentstroke}{rgb}{0.282290,0.145912,0.461510}%
\pgfsetstrokecolor{currentstroke}%
\pgfsetdash{}{0pt}%
\pgfpathmoveto{\pgfqpoint{1.978835in}{6.199232in}}%
\pgfpathlineto{\pgfqpoint{1.938394in}{6.177414in}}%
\pgfusepath{stroke}%
\end{pgfscope}%
\begin{pgfscope}%
\pgfpathrectangle{\pgfqpoint{0.800000in}{5.105882in}}{\pgfqpoint{2.407767in}{1.544118in}}%
\pgfusepath{clip}%
\pgfsetbuttcap%
\pgfsetroundjoin%
\pgfsetlinewidth{0.501875pt}%
\definecolor{currentstroke}{rgb}{0.275191,0.194905,0.496005}%
\pgfsetstrokecolor{currentstroke}%
\pgfsetdash{}{0pt}%
\pgfpathmoveto{\pgfqpoint{2.108480in}{5.624395in}}%
\pgfpathlineto{\pgfqpoint{2.058064in}{5.634718in}}%
\pgfusepath{stroke}%
\end{pgfscope}%
\begin{pgfscope}%
\pgfpathrectangle{\pgfqpoint{0.800000in}{5.105882in}}{\pgfqpoint{2.407767in}{1.544118in}}%
\pgfusepath{clip}%
\pgfsetbuttcap%
\pgfsetroundjoin%
\pgfsetlinewidth{0.501875pt}%
\definecolor{currentstroke}{rgb}{0.275191,0.194905,0.496005}%
\pgfsetstrokecolor{currentstroke}%
\pgfsetdash{}{0pt}%
\pgfpathmoveto{\pgfqpoint{2.058064in}{5.634718in}}%
\pgfpathlineto{\pgfqpoint{2.008712in}{5.647021in}}%
\pgfusepath{stroke}%
\end{pgfscope}%
\begin{pgfscope}%
\pgfpathrectangle{\pgfqpoint{0.800000in}{5.105882in}}{\pgfqpoint{2.407767in}{1.544118in}}%
\pgfusepath{clip}%
\pgfsetbuttcap%
\pgfsetroundjoin%
\pgfsetlinewidth{0.501875pt}%
\definecolor{currentstroke}{rgb}{0.270595,0.214069,0.507052}%
\pgfsetstrokecolor{currentstroke}%
\pgfsetdash{}{0pt}%
\pgfpathmoveto{\pgfqpoint{2.008712in}{5.647021in}}%
\pgfpathlineto{\pgfqpoint{1.960567in}{5.661132in}}%
\pgfusepath{stroke}%
\end{pgfscope}%
\begin{pgfscope}%
\pgfpathrectangle{\pgfqpoint{0.800000in}{5.105882in}}{\pgfqpoint{2.407767in}{1.544118in}}%
\pgfusepath{clip}%
\pgfsetbuttcap%
\pgfsetroundjoin%
\pgfsetlinewidth{0.501875pt}%
\definecolor{currentstroke}{rgb}{0.267968,0.223549,0.512008}%
\pgfsetstrokecolor{currentstroke}%
\pgfsetdash{}{0pt}%
\pgfpathmoveto{\pgfqpoint{1.960567in}{5.661132in}}%
\pgfpathlineto{\pgfqpoint{1.913731in}{5.676958in}}%
\pgfusepath{stroke}%
\end{pgfscope}%
\begin{pgfscope}%
\pgfpathrectangle{\pgfqpoint{0.800000in}{5.105882in}}{\pgfqpoint{2.407767in}{1.544118in}}%
\pgfusepath{clip}%
\pgfsetbuttcap%
\pgfsetroundjoin%
\pgfsetlinewidth{0.501875pt}%
\definecolor{currentstroke}{rgb}{0.266580,0.228262,0.514349}%
\pgfsetstrokecolor{currentstroke}%
\pgfsetdash{}{0pt}%
\pgfpathmoveto{\pgfqpoint{1.913731in}{5.676958in}}%
\pgfpathlineto{\pgfqpoint{1.868717in}{5.694768in}}%
\pgfusepath{stroke}%
\end{pgfscope}%
\begin{pgfscope}%
\pgfpathrectangle{\pgfqpoint{0.800000in}{5.105882in}}{\pgfqpoint{2.407767in}{1.544118in}}%
\pgfusepath{clip}%
\pgfsetbuttcap%
\pgfsetroundjoin%
\pgfsetlinewidth{0.501875pt}%
\definecolor{currentstroke}{rgb}{0.267968,0.223549,0.512008}%
\pgfsetstrokecolor{currentstroke}%
\pgfsetdash{}{0pt}%
\pgfpathmoveto{\pgfqpoint{1.868717in}{5.694768in}}%
\pgfpathlineto{\pgfqpoint{1.824953in}{5.713847in}}%
\pgfusepath{stroke}%
\end{pgfscope}%
\begin{pgfscope}%
\pgfpathrectangle{\pgfqpoint{0.800000in}{5.105882in}}{\pgfqpoint{2.407767in}{1.544118in}}%
\pgfusepath{clip}%
\pgfsetbuttcap%
\pgfsetroundjoin%
\pgfsetlinewidth{0.501875pt}%
\definecolor{currentstroke}{rgb}{0.241237,0.296485,0.539709}%
\pgfsetstrokecolor{currentstroke}%
\pgfsetdash{}{0pt}%
\pgfpathmoveto{\pgfqpoint{2.058064in}{5.982180in}}%
\pgfpathlineto{\pgfqpoint{2.005702in}{5.977043in}}%
\pgfusepath{stroke}%
\end{pgfscope}%
\begin{pgfscope}%
\pgfpathrectangle{\pgfqpoint{0.800000in}{5.105882in}}{\pgfqpoint{2.407767in}{1.544118in}}%
\pgfusepath{clip}%
\pgfsetbuttcap%
\pgfsetroundjoin%
\pgfsetlinewidth{0.501875pt}%
\definecolor{currentstroke}{rgb}{0.250425,0.274290,0.533103}%
\pgfsetstrokecolor{currentstroke}%
\pgfsetdash{}{0pt}%
\pgfpathmoveto{\pgfqpoint{2.005702in}{5.977043in}}%
\pgfpathlineto{\pgfqpoint{1.953498in}{5.971290in}}%
\pgfusepath{stroke}%
\end{pgfscope}%
\begin{pgfscope}%
\pgfpathrectangle{\pgfqpoint{0.800000in}{5.105882in}}{\pgfqpoint{2.407767in}{1.544118in}}%
\pgfusepath{clip}%
\pgfsetbuttcap%
\pgfsetroundjoin%
\pgfsetlinewidth{0.501875pt}%
\definecolor{currentstroke}{rgb}{0.252194,0.269783,0.531579}%
\pgfsetstrokecolor{currentstroke}%
\pgfsetdash{}{0pt}%
\pgfpathmoveto{\pgfqpoint{1.953498in}{5.971290in}}%
\pgfpathlineto{\pgfqpoint{1.901520in}{5.964763in}}%
\pgfusepath{stroke}%
\end{pgfscope}%
\begin{pgfscope}%
\pgfpathrectangle{\pgfqpoint{0.800000in}{5.105882in}}{\pgfqpoint{2.407767in}{1.544118in}}%
\pgfusepath{clip}%
\pgfsetbuttcap%
\pgfsetroundjoin%
\pgfsetlinewidth{0.501875pt}%
\definecolor{currentstroke}{rgb}{0.253935,0.265254,0.529983}%
\pgfsetstrokecolor{currentstroke}%
\pgfsetdash{}{0pt}%
\pgfpathmoveto{\pgfqpoint{1.901520in}{5.964763in}}%
\pgfpathlineto{\pgfqpoint{1.849897in}{5.957175in}}%
\pgfusepath{stroke}%
\end{pgfscope}%
\begin{pgfscope}%
\pgfpathrectangle{\pgfqpoint{0.800000in}{5.105882in}}{\pgfqpoint{2.407767in}{1.544118in}}%
\pgfusepath{clip}%
\pgfsetbuttcap%
\pgfsetroundjoin%
\pgfsetlinewidth{0.501875pt}%
\definecolor{currentstroke}{rgb}{0.265145,0.232956,0.516599}%
\pgfsetstrokecolor{currentstroke}%
\pgfsetdash{}{0pt}%
\pgfpathmoveto{\pgfqpoint{1.849897in}{5.957175in}}%
\pgfpathlineto{\pgfqpoint{1.798649in}{5.948684in}}%
\pgfusepath{stroke}%
\end{pgfscope}%
\begin{pgfscope}%
\pgfpathrectangle{\pgfqpoint{0.800000in}{5.105882in}}{\pgfqpoint{2.407767in}{1.544118in}}%
\pgfusepath{clip}%
\pgfsetroundcap%
\pgfsetroundjoin%
\pgfsetlinewidth{0.501875pt}%
\definecolor{currentstroke}{rgb}{0.283197,0.115680,0.436115}%
\pgfsetstrokecolor{currentstroke}%
\pgfsetdash{}{0pt}%
\pgfpathmoveto{\pgfqpoint{1.892860in}{6.286909in}}%
\pgfpathquadraticcurveto{\pgfqpoint{1.890252in}{6.282028in}}{\pgfqpoint{1.891304in}{6.283996in}}%
\pgfusepath{stroke}%
\end{pgfscope}%
\begin{pgfscope}%
\pgfpathrectangle{\pgfqpoint{0.800000in}{5.105882in}}{\pgfqpoint{2.407767in}{1.544118in}}%
\pgfusepath{clip}%
\pgfsetroundcap%
\pgfsetroundjoin%
\definecolor{currentfill}{rgb}{0.283197,0.115680,0.436115}%
\pgfsetfillcolor{currentfill}%
\pgfsetlinewidth{0.501875pt}%
\definecolor{currentstroke}{rgb}{0.283197,0.115680,0.436115}%
\pgfsetstrokecolor{currentstroke}%
\pgfsetdash{}{0pt}%
\pgfpathmoveto{\pgfqpoint{1.916645in}{6.301950in}}%
\pgfpathlineto{\pgfqpoint{1.891304in}{6.283996in}}%
\pgfpathlineto{\pgfqpoint{1.892145in}{6.315041in}}%
\pgfpathlineto{\pgfqpoint{1.916645in}{6.301950in}}%
\pgfpathlineto{\pgfqpoint{1.916645in}{6.301950in}}%
\pgfpathclose%
\pgfusepath{stroke,fill}%
\end{pgfscope}%
\begin{pgfscope}%
\pgfpathrectangle{\pgfqpoint{0.800000in}{5.105882in}}{\pgfqpoint{2.407767in}{1.544118in}}%
\pgfusepath{clip}%
\pgfsetroundcap%
\pgfsetroundjoin%
\pgfsetlinewidth{0.501875pt}%
\definecolor{currentstroke}{rgb}{0.283229,0.120777,0.440584}%
\pgfsetstrokecolor{currentstroke}%
\pgfsetdash{}{0pt}%
\pgfpathmoveto{\pgfqpoint{1.855154in}{5.500223in}}%
\pgfpathquadraticcurveto{\pgfqpoint{1.851391in}{5.508338in}}{\pgfqpoint{1.850895in}{5.509409in}}%
\pgfusepath{stroke}%
\end{pgfscope}%
\begin{pgfscope}%
\pgfpathrectangle{\pgfqpoint{0.800000in}{5.105882in}}{\pgfqpoint{2.407767in}{1.544118in}}%
\pgfusepath{clip}%
\pgfsetroundcap%
\pgfsetroundjoin%
\definecolor{currentfill}{rgb}{0.283229,0.120777,0.440584}%
\pgfsetfillcolor{currentfill}%
\pgfsetlinewidth{0.501875pt}%
\definecolor{currentstroke}{rgb}{0.283229,0.120777,0.440584}%
\pgfsetstrokecolor{currentstroke}%
\pgfsetdash{}{0pt}%
\pgfpathmoveto{\pgfqpoint{1.849981in}{5.478366in}}%
\pgfpathlineto{\pgfqpoint{1.850895in}{5.509409in}}%
\pgfpathlineto{\pgfqpoint{1.875181in}{5.490052in}}%
\pgfpathlineto{\pgfqpoint{1.849981in}{5.478366in}}%
\pgfpathlineto{\pgfqpoint{1.849981in}{5.478366in}}%
\pgfpathclose%
\pgfusepath{stroke,fill}%
\end{pgfscope}%
\begin{pgfscope}%
\pgfpathrectangle{\pgfqpoint{0.800000in}{5.105882in}}{\pgfqpoint{2.407767in}{1.544118in}}%
\pgfusepath{clip}%
\pgfsetroundcap%
\pgfsetroundjoin%
\pgfsetlinewidth{0.501875pt}%
\definecolor{currentstroke}{rgb}{0.281412,0.155834,0.469201}%
\pgfsetstrokecolor{currentstroke}%
\pgfsetdash{}{0pt}%
\pgfpathmoveto{\pgfqpoint{1.913743in}{5.535417in}}%
\pgfpathquadraticcurveto{\pgfqpoint{1.906267in}{5.542355in}}{\pgfqpoint{1.904482in}{5.544011in}}%
\pgfusepath{stroke}%
\end{pgfscope}%
\begin{pgfscope}%
\pgfpathrectangle{\pgfqpoint{0.800000in}{5.105882in}}{\pgfqpoint{2.407767in}{1.544118in}}%
\pgfusepath{clip}%
\pgfsetroundcap%
\pgfsetroundjoin%
\definecolor{currentfill}{rgb}{0.281412,0.155834,0.469201}%
\pgfsetfillcolor{currentfill}%
\pgfsetlinewidth{0.501875pt}%
\definecolor{currentstroke}{rgb}{0.281412,0.155834,0.469201}%
\pgfsetstrokecolor{currentstroke}%
\pgfsetdash{}{0pt}%
\pgfpathmoveto{\pgfqpoint{1.915397in}{5.514936in}}%
\pgfpathlineto{\pgfqpoint{1.904482in}{5.544011in}}%
\pgfpathlineto{\pgfqpoint{1.934291in}{5.535297in}}%
\pgfpathlineto{\pgfqpoint{1.915397in}{5.514936in}}%
\pgfpathlineto{\pgfqpoint{1.915397in}{5.514936in}}%
\pgfpathclose%
\pgfusepath{stroke,fill}%
\end{pgfscope}%
\begin{pgfscope}%
\pgfpathrectangle{\pgfqpoint{0.800000in}{5.105882in}}{\pgfqpoint{2.407767in}{1.544118in}}%
\pgfusepath{clip}%
\pgfsetroundcap%
\pgfsetroundjoin%
\pgfsetlinewidth{0.501875pt}%
\definecolor{currentstroke}{rgb}{0.282910,0.105393,0.426902}%
\pgfsetstrokecolor{currentstroke}%
\pgfsetdash{}{0pt}%
\pgfpathmoveto{\pgfqpoint{2.032369in}{6.263794in}}%
\pgfpathquadraticcurveto{\pgfqpoint{2.021851in}{6.258739in}}{\pgfqpoint{2.018332in}{6.257047in}}%
\pgfusepath{stroke}%
\end{pgfscope}%
\begin{pgfscope}%
\pgfpathrectangle{\pgfqpoint{0.800000in}{5.105882in}}{\pgfqpoint{2.407767in}{1.544118in}}%
\pgfusepath{clip}%
\pgfsetroundcap%
\pgfsetroundjoin%
\definecolor{currentfill}{rgb}{0.282910,0.105393,0.426902}%
\pgfsetfillcolor{currentfill}%
\pgfsetlinewidth{0.501875pt}%
\definecolor{currentstroke}{rgb}{0.282910,0.105393,0.426902}%
\pgfsetstrokecolor{currentstroke}%
\pgfsetdash{}{0pt}%
\pgfpathmoveto{\pgfqpoint{2.049385in}{6.256563in}}%
\pgfpathlineto{\pgfqpoint{2.018332in}{6.257047in}}%
\pgfpathlineto{\pgfqpoint{2.037351in}{6.281599in}}%
\pgfpathlineto{\pgfqpoint{2.049385in}{6.256563in}}%
\pgfpathlineto{\pgfqpoint{2.049385in}{6.256563in}}%
\pgfpathclose%
\pgfusepath{stroke,fill}%
\end{pgfscope}%
\begin{pgfscope}%
\pgfpathrectangle{\pgfqpoint{0.800000in}{5.105882in}}{\pgfqpoint{2.407767in}{1.544118in}}%
\pgfusepath{clip}%
\pgfsetroundcap%
\pgfsetroundjoin%
\pgfsetlinewidth{0.501875pt}%
\definecolor{currentstroke}{rgb}{0.282327,0.094955,0.417331}%
\pgfsetstrokecolor{currentstroke}%
\pgfsetdash{}{0pt}%
\pgfpathmoveto{\pgfqpoint{2.065889in}{5.477212in}}%
\pgfpathquadraticcurveto{\pgfqpoint{2.055130in}{5.482057in}}{\pgfqpoint{2.051451in}{5.483714in}}%
\pgfusepath{stroke}%
\end{pgfscope}%
\begin{pgfscope}%
\pgfpathrectangle{\pgfqpoint{0.800000in}{5.105882in}}{\pgfqpoint{2.407767in}{1.544118in}}%
\pgfusepath{clip}%
\pgfsetroundcap%
\pgfsetroundjoin%
\definecolor{currentfill}{rgb}{0.282327,0.094955,0.417331}%
\pgfsetfillcolor{currentfill}%
\pgfsetlinewidth{0.501875pt}%
\definecolor{currentstroke}{rgb}{0.282327,0.094955,0.417331}%
\pgfsetstrokecolor{currentstroke}%
\pgfsetdash{}{0pt}%
\pgfpathmoveto{\pgfqpoint{2.071075in}{5.459643in}}%
\pgfpathlineto{\pgfqpoint{2.051451in}{5.483714in}}%
\pgfpathlineto{\pgfqpoint{2.082482in}{5.484971in}}%
\pgfpathlineto{\pgfqpoint{2.071075in}{5.459643in}}%
\pgfpathlineto{\pgfqpoint{2.071075in}{5.459643in}}%
\pgfpathclose%
\pgfusepath{stroke,fill}%
\end{pgfscope}%
\begin{pgfscope}%
\pgfpathrectangle{\pgfqpoint{0.800000in}{5.105882in}}{\pgfqpoint{2.407767in}{1.544118in}}%
\pgfusepath{clip}%
\pgfsetroundcap%
\pgfsetroundjoin%
\pgfsetlinewidth{0.501875pt}%
\definecolor{currentstroke}{rgb}{0.281446,0.084320,0.407414}%
\pgfsetstrokecolor{currentstroke}%
\pgfsetdash{}{0pt}%
\pgfpathmoveto{\pgfqpoint{2.225651in}{5.474218in}}%
\pgfpathquadraticcurveto{\pgfqpoint{2.212749in}{5.476115in}}{\pgfqpoint{2.207529in}{5.476883in}}%
\pgfusepath{stroke}%
\end{pgfscope}%
\begin{pgfscope}%
\pgfpathrectangle{\pgfqpoint{0.800000in}{5.105882in}}{\pgfqpoint{2.407767in}{1.544118in}}%
\pgfusepath{clip}%
\pgfsetroundcap%
\pgfsetroundjoin%
\definecolor{currentfill}{rgb}{0.281446,0.084320,0.407414}%
\pgfsetfillcolor{currentfill}%
\pgfsetlinewidth{0.501875pt}%
\definecolor{currentstroke}{rgb}{0.281446,0.084320,0.407414}%
\pgfsetstrokecolor{currentstroke}%
\pgfsetdash{}{0pt}%
\pgfpathmoveto{\pgfqpoint{2.232990in}{5.459100in}}%
\pgfpathlineto{\pgfqpoint{2.207529in}{5.476883in}}%
\pgfpathlineto{\pgfqpoint{2.237032in}{5.486582in}}%
\pgfpathlineto{\pgfqpoint{2.232990in}{5.459100in}}%
\pgfpathlineto{\pgfqpoint{2.232990in}{5.459100in}}%
\pgfpathclose%
\pgfusepath{stroke,fill}%
\end{pgfscope}%
\begin{pgfscope}%
\pgfpathrectangle{\pgfqpoint{0.800000in}{5.105882in}}{\pgfqpoint{2.407767in}{1.544118in}}%
\pgfusepath{clip}%
\pgfsetroundcap%
\pgfsetroundjoin%
\pgfsetlinewidth{0.501875pt}%
\definecolor{currentstroke}{rgb}{0.282656,0.100196,0.422160}%
\pgfsetstrokecolor{currentstroke}%
\pgfsetdash{}{0pt}%
\pgfpathmoveto{\pgfqpoint{2.390161in}{5.509768in}}%
\pgfpathquadraticcurveto{\pgfqpoint{2.376962in}{5.510470in}}{\pgfqpoint{2.371516in}{5.510759in}}%
\pgfusepath{stroke}%
\end{pgfscope}%
\begin{pgfscope}%
\pgfpathrectangle{\pgfqpoint{0.800000in}{5.105882in}}{\pgfqpoint{2.407767in}{1.544118in}}%
\pgfusepath{clip}%
\pgfsetroundcap%
\pgfsetroundjoin%
\definecolor{currentfill}{rgb}{0.282656,0.100196,0.422160}%
\pgfsetfillcolor{currentfill}%
\pgfsetlinewidth{0.501875pt}%
\definecolor{currentstroke}{rgb}{0.282656,0.100196,0.422160}%
\pgfsetstrokecolor{currentstroke}%
\pgfsetdash{}{0pt}%
\pgfpathmoveto{\pgfqpoint{2.398517in}{5.495415in}}%
\pgfpathlineto{\pgfqpoint{2.371516in}{5.510759in}}%
\pgfpathlineto{\pgfqpoint{2.399993in}{5.523154in}}%
\pgfpathlineto{\pgfqpoint{2.398517in}{5.495415in}}%
\pgfpathlineto{\pgfqpoint{2.398517in}{5.495415in}}%
\pgfpathclose%
\pgfusepath{stroke,fill}%
\end{pgfscope}%
\begin{pgfscope}%
\pgfpathrectangle{\pgfqpoint{0.800000in}{5.105882in}}{\pgfqpoint{2.407767in}{1.544118in}}%
\pgfusepath{clip}%
\pgfsetroundcap%
\pgfsetroundjoin%
\pgfsetlinewidth{0.501875pt}%
\definecolor{currentstroke}{rgb}{0.281887,0.150881,0.465405}%
\pgfsetstrokecolor{currentstroke}%
\pgfsetdash{}{0pt}%
\pgfpathmoveto{\pgfqpoint{2.129040in}{5.569856in}}%
\pgfpathquadraticcurveto{\pgfqpoint{2.116587in}{5.572725in}}{\pgfqpoint{2.111700in}{5.573850in}}%
\pgfusepath{stroke}%
\end{pgfscope}%
\begin{pgfscope}%
\pgfpathrectangle{\pgfqpoint{0.800000in}{5.105882in}}{\pgfqpoint{2.407767in}{1.544118in}}%
\pgfusepath{clip}%
\pgfsetroundcap%
\pgfsetroundjoin%
\definecolor{currentfill}{rgb}{0.281887,0.150881,0.465405}%
\pgfsetfillcolor{currentfill}%
\pgfsetlinewidth{0.501875pt}%
\definecolor{currentstroke}{rgb}{0.281887,0.150881,0.465405}%
\pgfsetstrokecolor{currentstroke}%
\pgfsetdash{}{0pt}%
\pgfpathmoveto{\pgfqpoint{2.135651in}{5.554080in}}%
\pgfpathlineto{\pgfqpoint{2.111700in}{5.573850in}}%
\pgfpathlineto{\pgfqpoint{2.141887in}{5.581149in}}%
\pgfpathlineto{\pgfqpoint{2.135651in}{5.554080in}}%
\pgfpathlineto{\pgfqpoint{2.135651in}{5.554080in}}%
\pgfpathclose%
\pgfusepath{stroke,fill}%
\end{pgfscope}%
\begin{pgfscope}%
\pgfpathrectangle{\pgfqpoint{0.800000in}{5.105882in}}{\pgfqpoint{2.407767in}{1.544118in}}%
\pgfusepath{clip}%
\pgfsetroundcap%
\pgfsetroundjoin%
\pgfsetlinewidth{0.501875pt}%
\definecolor{currentstroke}{rgb}{0.282290,0.145912,0.461510}%
\pgfsetstrokecolor{currentstroke}%
\pgfsetdash{}{0pt}%
\pgfpathmoveto{\pgfqpoint{2.389818in}{5.576995in}}%
\pgfpathquadraticcurveto{\pgfqpoint{2.376641in}{5.577828in}}{\pgfqpoint{2.371212in}{5.578171in}}%
\pgfusepath{stroke}%
\end{pgfscope}%
\begin{pgfscope}%
\pgfpathrectangle{\pgfqpoint{0.800000in}{5.105882in}}{\pgfqpoint{2.407767in}{1.544118in}}%
\pgfusepath{clip}%
\pgfsetroundcap%
\pgfsetroundjoin%
\definecolor{currentfill}{rgb}{0.282290,0.145912,0.461510}%
\pgfsetfillcolor{currentfill}%
\pgfsetlinewidth{0.501875pt}%
\definecolor{currentstroke}{rgb}{0.282290,0.145912,0.461510}%
\pgfsetstrokecolor{currentstroke}%
\pgfsetdash{}{0pt}%
\pgfpathmoveto{\pgfqpoint{2.398058in}{5.562558in}}%
\pgfpathlineto{\pgfqpoint{2.371212in}{5.578171in}}%
\pgfpathlineto{\pgfqpoint{2.399811in}{5.590280in}}%
\pgfpathlineto{\pgfqpoint{2.398058in}{5.562558in}}%
\pgfpathlineto{\pgfqpoint{2.398058in}{5.562558in}}%
\pgfpathclose%
\pgfusepath{stroke,fill}%
\end{pgfscope}%
\begin{pgfscope}%
\pgfpathrectangle{\pgfqpoint{0.800000in}{5.105882in}}{\pgfqpoint{2.407767in}{1.544118in}}%
\pgfusepath{clip}%
\pgfsetroundcap%
\pgfsetroundjoin%
\pgfsetlinewidth{0.501875pt}%
\definecolor{currentstroke}{rgb}{0.283187,0.125848,0.444960}%
\pgfsetstrokecolor{currentstroke}%
\pgfsetdash{}{0pt}%
\pgfpathmoveto{\pgfqpoint{2.495383in}{5.605635in}}%
\pgfpathquadraticcurveto{\pgfqpoint{2.482165in}{5.606165in}}{\pgfqpoint{2.476705in}{5.606385in}}%
\pgfusepath{stroke}%
\end{pgfscope}%
\begin{pgfscope}%
\pgfpathrectangle{\pgfqpoint{0.800000in}{5.105882in}}{\pgfqpoint{2.407767in}{1.544118in}}%
\pgfusepath{clip}%
\pgfsetroundcap%
\pgfsetroundjoin%
\definecolor{currentfill}{rgb}{0.283187,0.125848,0.444960}%
\pgfsetfillcolor{currentfill}%
\pgfsetlinewidth{0.501875pt}%
\definecolor{currentstroke}{rgb}{0.283187,0.125848,0.444960}%
\pgfsetstrokecolor{currentstroke}%
\pgfsetdash{}{0pt}%
\pgfpathmoveto{\pgfqpoint{2.503904in}{5.591392in}}%
\pgfpathlineto{\pgfqpoint{2.476705in}{5.606385in}}%
\pgfpathlineto{\pgfqpoint{2.505018in}{5.619148in}}%
\pgfpathlineto{\pgfqpoint{2.503904in}{5.591392in}}%
\pgfpathlineto{\pgfqpoint{2.503904in}{5.591392in}}%
\pgfpathclose%
\pgfusepath{stroke,fill}%
\end{pgfscope}%
\begin{pgfscope}%
\pgfpathrectangle{\pgfqpoint{0.800000in}{5.105882in}}{\pgfqpoint{2.407767in}{1.544118in}}%
\pgfusepath{clip}%
\pgfsetroundcap%
\pgfsetroundjoin%
\pgfsetlinewidth{0.501875pt}%
\definecolor{currentstroke}{rgb}{0.265145,0.232956,0.516599}%
\pgfsetstrokecolor{currentstroke}%
\pgfsetdash{}{0pt}%
\pgfpathmoveto{\pgfqpoint{2.231455in}{5.653888in}}%
\pgfpathquadraticcurveto{\pgfqpoint{2.218376in}{5.655214in}}{\pgfqpoint{2.213021in}{5.655757in}}%
\pgfusepath{stroke}%
\end{pgfscope}%
\begin{pgfscope}%
\pgfpathrectangle{\pgfqpoint{0.800000in}{5.105882in}}{\pgfqpoint{2.407767in}{1.544118in}}%
\pgfusepath{clip}%
\pgfsetroundcap%
\pgfsetroundjoin%
\definecolor{currentfill}{rgb}{0.265145,0.232956,0.516599}%
\pgfsetfillcolor{currentfill}%
\pgfsetlinewidth{0.501875pt}%
\definecolor{currentstroke}{rgb}{0.265145,0.232956,0.516599}%
\pgfsetstrokecolor{currentstroke}%
\pgfsetdash{}{0pt}%
\pgfpathmoveto{\pgfqpoint{2.239256in}{5.639136in}}%
\pgfpathlineto{\pgfqpoint{2.213021in}{5.655757in}}%
\pgfpathlineto{\pgfqpoint{2.242059in}{5.666772in}}%
\pgfpathlineto{\pgfqpoint{2.239256in}{5.639136in}}%
\pgfpathlineto{\pgfqpoint{2.239256in}{5.639136in}}%
\pgfpathclose%
\pgfusepath{stroke,fill}%
\end{pgfscope}%
\begin{pgfscope}%
\pgfpathrectangle{\pgfqpoint{0.800000in}{5.105882in}}{\pgfqpoint{2.407767in}{1.544118in}}%
\pgfusepath{clip}%
\pgfsetroundcap%
\pgfsetroundjoin%
\pgfsetlinewidth{0.501875pt}%
\definecolor{currentstroke}{rgb}{0.260571,0.246922,0.522828}%
\pgfsetstrokecolor{currentstroke}%
\pgfsetdash{}{0pt}%
\pgfpathmoveto{\pgfqpoint{2.336640in}{5.679872in}}%
\pgfpathquadraticcurveto{\pgfqpoint{2.323450in}{5.680638in}}{\pgfqpoint{2.318012in}{5.680954in}}%
\pgfusepath{stroke}%
\end{pgfscope}%
\begin{pgfscope}%
\pgfpathrectangle{\pgfqpoint{0.800000in}{5.105882in}}{\pgfqpoint{2.407767in}{1.544118in}}%
\pgfusepath{clip}%
\pgfsetroundcap%
\pgfsetroundjoin%
\definecolor{currentfill}{rgb}{0.260571,0.246922,0.522828}%
\pgfsetfillcolor{currentfill}%
\pgfsetlinewidth{0.501875pt}%
\definecolor{currentstroke}{rgb}{0.260571,0.246922,0.522828}%
\pgfsetstrokecolor{currentstroke}%
\pgfsetdash{}{0pt}%
\pgfpathmoveto{\pgfqpoint{2.344938in}{5.665478in}}%
\pgfpathlineto{\pgfqpoint{2.318012in}{5.680954in}}%
\pgfpathlineto{\pgfqpoint{2.346548in}{5.693209in}}%
\pgfpathlineto{\pgfqpoint{2.344938in}{5.665478in}}%
\pgfpathlineto{\pgfqpoint{2.344938in}{5.665478in}}%
\pgfpathclose%
\pgfusepath{stroke,fill}%
\end{pgfscope}%
\begin{pgfscope}%
\pgfpathrectangle{\pgfqpoint{0.800000in}{5.105882in}}{\pgfqpoint{2.407767in}{1.544118in}}%
\pgfusepath{clip}%
\pgfsetroundcap%
\pgfsetroundjoin%
\pgfsetlinewidth{0.501875pt}%
\definecolor{currentstroke}{rgb}{0.274128,0.199721,0.498911}%
\pgfsetstrokecolor{currentstroke}%
\pgfsetdash{}{0pt}%
\pgfpathmoveto{\pgfqpoint{2.495188in}{5.707105in}}%
\pgfpathquadraticcurveto{\pgfqpoint{2.481956in}{5.707459in}}{\pgfqpoint{2.476485in}{5.707605in}}%
\pgfusepath{stroke}%
\end{pgfscope}%
\begin{pgfscope}%
\pgfpathrectangle{\pgfqpoint{0.800000in}{5.105882in}}{\pgfqpoint{2.407767in}{1.544118in}}%
\pgfusepath{clip}%
\pgfsetroundcap%
\pgfsetroundjoin%
\definecolor{currentfill}{rgb}{0.274128,0.199721,0.498911}%
\pgfsetfillcolor{currentfill}%
\pgfsetlinewidth{0.501875pt}%
\definecolor{currentstroke}{rgb}{0.274128,0.199721,0.498911}%
\pgfsetstrokecolor{currentstroke}%
\pgfsetdash{}{0pt}%
\pgfpathmoveto{\pgfqpoint{2.503881in}{5.692978in}}%
\pgfpathlineto{\pgfqpoint{2.476485in}{5.707605in}}%
\pgfpathlineto{\pgfqpoint{2.504624in}{5.720746in}}%
\pgfpathlineto{\pgfqpoint{2.503881in}{5.692978in}}%
\pgfpathlineto{\pgfqpoint{2.503881in}{5.692978in}}%
\pgfpathclose%
\pgfusepath{stroke,fill}%
\end{pgfscope}%
\begin{pgfscope}%
\pgfpathrectangle{\pgfqpoint{0.800000in}{5.105882in}}{\pgfqpoint{2.407767in}{1.544118in}}%
\pgfusepath{clip}%
\pgfsetroundcap%
\pgfsetroundjoin%
\pgfsetlinewidth{0.501875pt}%
\definecolor{currentstroke}{rgb}{0.220057,0.343307,0.549413}%
\pgfsetstrokecolor{currentstroke}%
\pgfsetdash{}{0pt}%
\pgfpathmoveto{\pgfqpoint{2.230693in}{5.749982in}}%
\pgfpathquadraticcurveto{\pgfqpoint{2.217514in}{5.750815in}}{\pgfqpoint{2.212083in}{5.751158in}}%
\pgfusepath{stroke}%
\end{pgfscope}%
\begin{pgfscope}%
\pgfpathrectangle{\pgfqpoint{0.800000in}{5.105882in}}{\pgfqpoint{2.407767in}{1.544118in}}%
\pgfusepath{clip}%
\pgfsetroundcap%
\pgfsetroundjoin%
\definecolor{currentfill}{rgb}{0.220057,0.343307,0.549413}%
\pgfsetfillcolor{currentfill}%
\pgfsetlinewidth{0.501875pt}%
\definecolor{currentstroke}{rgb}{0.220057,0.343307,0.549413}%
\pgfsetstrokecolor{currentstroke}%
\pgfsetdash{}{0pt}%
\pgfpathmoveto{\pgfqpoint{2.238930in}{5.735545in}}%
\pgfpathlineto{\pgfqpoint{2.212083in}{5.751158in}}%
\pgfpathlineto{\pgfqpoint{2.240682in}{5.763267in}}%
\pgfpathlineto{\pgfqpoint{2.238930in}{5.735545in}}%
\pgfpathlineto{\pgfqpoint{2.238930in}{5.735545in}}%
\pgfpathclose%
\pgfusepath{stroke,fill}%
\end{pgfscope}%
\begin{pgfscope}%
\pgfpathrectangle{\pgfqpoint{0.800000in}{5.105882in}}{\pgfqpoint{2.407767in}{1.544118in}}%
\pgfusepath{clip}%
\pgfsetroundcap%
\pgfsetroundjoin%
\pgfsetlinewidth{0.501875pt}%
\definecolor{currentstroke}{rgb}{0.243113,0.292092,0.538516}%
\pgfsetstrokecolor{currentstroke}%
\pgfsetdash{}{0pt}%
\pgfpathmoveto{\pgfqpoint{2.442192in}{5.776616in}}%
\pgfpathquadraticcurveto{\pgfqpoint{2.428955in}{5.776884in}}{\pgfqpoint{2.423480in}{5.776994in}}%
\pgfusepath{stroke}%
\end{pgfscope}%
\begin{pgfscope}%
\pgfpathrectangle{\pgfqpoint{0.800000in}{5.105882in}}{\pgfqpoint{2.407767in}{1.544118in}}%
\pgfusepath{clip}%
\pgfsetroundcap%
\pgfsetroundjoin%
\definecolor{currentfill}{rgb}{0.243113,0.292092,0.538516}%
\pgfsetfillcolor{currentfill}%
\pgfsetlinewidth{0.501875pt}%
\definecolor{currentstroke}{rgb}{0.243113,0.292092,0.538516}%
\pgfsetstrokecolor{currentstroke}%
\pgfsetdash{}{0pt}%
\pgfpathmoveto{\pgfqpoint{2.450971in}{5.762547in}}%
\pgfpathlineto{\pgfqpoint{2.423480in}{5.776994in}}%
\pgfpathlineto{\pgfqpoint{2.451533in}{5.790319in}}%
\pgfpathlineto{\pgfqpoint{2.450971in}{5.762547in}}%
\pgfpathlineto{\pgfqpoint{2.450971in}{5.762547in}}%
\pgfpathclose%
\pgfusepath{stroke,fill}%
\end{pgfscope}%
\begin{pgfscope}%
\pgfpathrectangle{\pgfqpoint{0.800000in}{5.105882in}}{\pgfqpoint{2.407767in}{1.544118in}}%
\pgfusepath{clip}%
\pgfsetroundcap%
\pgfsetroundjoin%
\pgfsetlinewidth{0.501875pt}%
\definecolor{currentstroke}{rgb}{0.221989,0.339161,0.548752}%
\pgfsetstrokecolor{currentstroke}%
\pgfsetdash{}{0pt}%
\pgfpathmoveto{\pgfqpoint{2.230374in}{5.814670in}}%
\pgfpathquadraticcurveto{\pgfqpoint{2.217146in}{5.815087in}}{\pgfqpoint{2.211678in}{5.815259in}}%
\pgfusepath{stroke}%
\end{pgfscope}%
\begin{pgfscope}%
\pgfpathrectangle{\pgfqpoint{0.800000in}{5.105882in}}{\pgfqpoint{2.407767in}{1.544118in}}%
\pgfusepath{clip}%
\pgfsetroundcap%
\pgfsetroundjoin%
\definecolor{currentfill}{rgb}{0.221989,0.339161,0.548752}%
\pgfsetfillcolor{currentfill}%
\pgfsetlinewidth{0.501875pt}%
\definecolor{currentstroke}{rgb}{0.221989,0.339161,0.548752}%
\pgfsetstrokecolor{currentstroke}%
\pgfsetdash{}{0pt}%
\pgfpathmoveto{\pgfqpoint{2.239005in}{5.800502in}}%
\pgfpathlineto{\pgfqpoint{2.211678in}{5.815259in}}%
\pgfpathlineto{\pgfqpoint{2.239880in}{5.828266in}}%
\pgfpathlineto{\pgfqpoint{2.239005in}{5.800502in}}%
\pgfpathlineto{\pgfqpoint{2.239005in}{5.800502in}}%
\pgfpathclose%
\pgfusepath{stroke,fill}%
\end{pgfscope}%
\begin{pgfscope}%
\pgfpathrectangle{\pgfqpoint{0.800000in}{5.105882in}}{\pgfqpoint{2.407767in}{1.544118in}}%
\pgfusepath{clip}%
\pgfsetroundcap%
\pgfsetroundjoin%
\pgfsetlinewidth{0.501875pt}%
\definecolor{currentstroke}{rgb}{0.216210,0.351535,0.550627}%
\pgfsetstrokecolor{currentstroke}%
\pgfsetdash{}{0pt}%
\pgfpathmoveto{\pgfqpoint{2.336215in}{5.845572in}}%
\pgfpathquadraticcurveto{\pgfqpoint{2.322974in}{5.845739in}}{\pgfqpoint{2.317496in}{5.845808in}}%
\pgfusepath{stroke}%
\end{pgfscope}%
\begin{pgfscope}%
\pgfpathrectangle{\pgfqpoint{0.800000in}{5.105882in}}{\pgfqpoint{2.407767in}{1.544118in}}%
\pgfusepath{clip}%
\pgfsetroundcap%
\pgfsetroundjoin%
\definecolor{currentfill}{rgb}{0.216210,0.351535,0.550627}%
\pgfsetfillcolor{currentfill}%
\pgfsetlinewidth{0.501875pt}%
\definecolor{currentstroke}{rgb}{0.216210,0.351535,0.550627}%
\pgfsetstrokecolor{currentstroke}%
\pgfsetdash{}{0pt}%
\pgfpathmoveto{\pgfqpoint{2.345096in}{5.831570in}}%
\pgfpathlineto{\pgfqpoint{2.317496in}{5.845808in}}%
\pgfpathlineto{\pgfqpoint{2.345447in}{5.859345in}}%
\pgfpathlineto{\pgfqpoint{2.345096in}{5.831570in}}%
\pgfpathlineto{\pgfqpoint{2.345096in}{5.831570in}}%
\pgfpathclose%
\pgfusepath{stroke,fill}%
\end{pgfscope}%
\begin{pgfscope}%
\pgfpathrectangle{\pgfqpoint{0.800000in}{5.105882in}}{\pgfqpoint{2.407767in}{1.544118in}}%
\pgfusepath{clip}%
\pgfsetroundcap%
\pgfsetroundjoin%
\pgfsetlinewidth{0.501875pt}%
\definecolor{currentstroke}{rgb}{0.212395,0.359683,0.551710}%
\pgfsetstrokecolor{currentstroke}%
\pgfsetdash{}{0pt}%
\pgfpathmoveto{\pgfqpoint{2.230240in}{5.877122in}}%
\pgfpathquadraticcurveto{\pgfqpoint{2.216997in}{5.877080in}}{\pgfqpoint{2.211517in}{5.877062in}}%
\pgfusepath{stroke}%
\end{pgfscope}%
\begin{pgfscope}%
\pgfpathrectangle{\pgfqpoint{0.800000in}{5.105882in}}{\pgfqpoint{2.407767in}{1.544118in}}%
\pgfusepath{clip}%
\pgfsetroundcap%
\pgfsetroundjoin%
\definecolor{currentfill}{rgb}{0.212395,0.359683,0.551710}%
\pgfsetfillcolor{currentfill}%
\pgfsetlinewidth{0.501875pt}%
\definecolor{currentstroke}{rgb}{0.212395,0.359683,0.551710}%
\pgfsetstrokecolor{currentstroke}%
\pgfsetdash{}{0pt}%
\pgfpathmoveto{\pgfqpoint{2.239339in}{5.863262in}}%
\pgfpathlineto{\pgfqpoint{2.211517in}{5.877062in}}%
\pgfpathlineto{\pgfqpoint{2.239250in}{5.891040in}}%
\pgfpathlineto{\pgfqpoint{2.239339in}{5.863262in}}%
\pgfpathlineto{\pgfqpoint{2.239339in}{5.863262in}}%
\pgfpathclose%
\pgfusepath{stroke,fill}%
\end{pgfscope}%
\begin{pgfscope}%
\pgfpathrectangle{\pgfqpoint{0.800000in}{5.105882in}}{\pgfqpoint{2.407767in}{1.544118in}}%
\pgfusepath{clip}%
\pgfsetroundcap%
\pgfsetroundjoin%
\pgfsetlinewidth{0.501875pt}%
\definecolor{currentstroke}{rgb}{0.208623,0.367752,0.552675}%
\pgfsetstrokecolor{currentstroke}%
\pgfsetdash{}{0pt}%
\pgfpathmoveto{\pgfqpoint{2.230271in}{5.909636in}}%
\pgfpathquadraticcurveto{\pgfqpoint{2.217031in}{5.909447in}}{\pgfqpoint{2.211553in}{5.909368in}}%
\pgfusepath{stroke}%
\end{pgfscope}%
\begin{pgfscope}%
\pgfpathrectangle{\pgfqpoint{0.800000in}{5.105882in}}{\pgfqpoint{2.407767in}{1.544118in}}%
\pgfusepath{clip}%
\pgfsetroundcap%
\pgfsetroundjoin%
\definecolor{currentfill}{rgb}{0.208623,0.367752,0.552675}%
\pgfsetfillcolor{currentfill}%
\pgfsetlinewidth{0.501875pt}%
\definecolor{currentstroke}{rgb}{0.208623,0.367752,0.552675}%
\pgfsetstrokecolor{currentstroke}%
\pgfsetdash{}{0pt}%
\pgfpathmoveto{\pgfqpoint{2.239527in}{5.895878in}}%
\pgfpathlineto{\pgfqpoint{2.211553in}{5.909368in}}%
\pgfpathlineto{\pgfqpoint{2.239130in}{5.923653in}}%
\pgfpathlineto{\pgfqpoint{2.239527in}{5.895878in}}%
\pgfpathlineto{\pgfqpoint{2.239527in}{5.895878in}}%
\pgfpathclose%
\pgfusepath{stroke,fill}%
\end{pgfscope}%
\begin{pgfscope}%
\pgfpathrectangle{\pgfqpoint{0.800000in}{5.105882in}}{\pgfqpoint{2.407767in}{1.544118in}}%
\pgfusepath{clip}%
\pgfsetroundcap%
\pgfsetroundjoin%
\pgfsetlinewidth{0.501875pt}%
\definecolor{currentstroke}{rgb}{0.223925,0.334994,0.548053}%
\pgfsetstrokecolor{currentstroke}%
\pgfsetdash{}{0pt}%
\pgfpathmoveto{\pgfqpoint{2.336238in}{5.944054in}}%
\pgfpathquadraticcurveto{\pgfqpoint{2.323000in}{5.943799in}}{\pgfqpoint{2.317525in}{5.943693in}}%
\pgfusepath{stroke}%
\end{pgfscope}%
\begin{pgfscope}%
\pgfpathrectangle{\pgfqpoint{0.800000in}{5.105882in}}{\pgfqpoint{2.407767in}{1.544118in}}%
\pgfusepath{clip}%
\pgfsetroundcap%
\pgfsetroundjoin%
\definecolor{currentfill}{rgb}{0.223925,0.334994,0.548053}%
\pgfsetfillcolor{currentfill}%
\pgfsetlinewidth{0.501875pt}%
\definecolor{currentstroke}{rgb}{0.223925,0.334994,0.548053}%
\pgfsetstrokecolor{currentstroke}%
\pgfsetdash{}{0pt}%
\pgfpathmoveto{\pgfqpoint{2.345566in}{5.930343in}}%
\pgfpathlineto{\pgfqpoint{2.317525in}{5.943693in}}%
\pgfpathlineto{\pgfqpoint{2.345029in}{5.958116in}}%
\pgfpathlineto{\pgfqpoint{2.345566in}{5.930343in}}%
\pgfpathlineto{\pgfqpoint{2.345566in}{5.930343in}}%
\pgfpathclose%
\pgfusepath{stroke,fill}%
\end{pgfscope}%
\begin{pgfscope}%
\pgfpathrectangle{\pgfqpoint{0.800000in}{5.105882in}}{\pgfqpoint{2.407767in}{1.544118in}}%
\pgfusepath{clip}%
\pgfsetroundcap%
\pgfsetroundjoin%
\pgfsetlinewidth{0.501875pt}%
\definecolor{currentstroke}{rgb}{0.231674,0.318106,0.544834}%
\pgfsetstrokecolor{currentstroke}%
\pgfsetdash{}{0pt}%
\pgfpathmoveto{\pgfqpoint{2.389252in}{5.977932in}}%
\pgfpathquadraticcurveto{\pgfqpoint{2.376019in}{5.977599in}}{\pgfqpoint{2.370547in}{5.977461in}}%
\pgfusepath{stroke}%
\end{pgfscope}%
\begin{pgfscope}%
\pgfpathrectangle{\pgfqpoint{0.800000in}{5.105882in}}{\pgfqpoint{2.407767in}{1.544118in}}%
\pgfusepath{clip}%
\pgfsetroundcap%
\pgfsetroundjoin%
\definecolor{currentfill}{rgb}{0.231674,0.318106,0.544834}%
\pgfsetfillcolor{currentfill}%
\pgfsetlinewidth{0.501875pt}%
\definecolor{currentstroke}{rgb}{0.231674,0.318106,0.544834}%
\pgfsetstrokecolor{currentstroke}%
\pgfsetdash{}{0pt}%
\pgfpathmoveto{\pgfqpoint{2.398666in}{5.964276in}}%
\pgfpathlineto{\pgfqpoint{2.370547in}{5.977461in}}%
\pgfpathlineto{\pgfqpoint{2.397966in}{5.992045in}}%
\pgfpathlineto{\pgfqpoint{2.398666in}{5.964276in}}%
\pgfpathlineto{\pgfqpoint{2.398666in}{5.964276in}}%
\pgfpathclose%
\pgfusepath{stroke,fill}%
\end{pgfscope}%
\begin{pgfscope}%
\pgfpathrectangle{\pgfqpoint{0.800000in}{5.105882in}}{\pgfqpoint{2.407767in}{1.544118in}}%
\pgfusepath{clip}%
\pgfsetroundcap%
\pgfsetroundjoin%
\pgfsetlinewidth{0.501875pt}%
\definecolor{currentstroke}{rgb}{0.248629,0.278775,0.534556}%
\pgfsetstrokecolor{currentstroke}%
\pgfsetdash{}{0pt}%
\pgfpathmoveto{\pgfqpoint{2.442247in}{6.012750in}}%
\pgfpathquadraticcurveto{\pgfqpoint{2.429018in}{6.012358in}}{\pgfqpoint{2.423549in}{6.012196in}}%
\pgfusepath{stroke}%
\end{pgfscope}%
\begin{pgfscope}%
\pgfpathrectangle{\pgfqpoint{0.800000in}{5.105882in}}{\pgfqpoint{2.407767in}{1.544118in}}%
\pgfusepath{clip}%
\pgfsetroundcap%
\pgfsetroundjoin%
\definecolor{currentfill}{rgb}{0.248629,0.278775,0.534556}%
\pgfsetfillcolor{currentfill}%
\pgfsetlinewidth{0.501875pt}%
\definecolor{currentstroke}{rgb}{0.248629,0.278775,0.534556}%
\pgfsetstrokecolor{currentstroke}%
\pgfsetdash{}{0pt}%
\pgfpathmoveto{\pgfqpoint{2.451726in}{5.999136in}}%
\pgfpathlineto{\pgfqpoint{2.423549in}{6.012196in}}%
\pgfpathlineto{\pgfqpoint{2.450902in}{6.026902in}}%
\pgfpathlineto{\pgfqpoint{2.451726in}{5.999136in}}%
\pgfpathlineto{\pgfqpoint{2.451726in}{5.999136in}}%
\pgfpathclose%
\pgfusepath{stroke,fill}%
\end{pgfscope}%
\begin{pgfscope}%
\pgfpathrectangle{\pgfqpoint{0.800000in}{5.105882in}}{\pgfqpoint{2.407767in}{1.544118in}}%
\pgfusepath{clip}%
\pgfsetroundcap%
\pgfsetroundjoin%
\pgfsetlinewidth{0.501875pt}%
\definecolor{currentstroke}{rgb}{0.255645,0.260703,0.528312}%
\pgfsetstrokecolor{currentstroke}%
\pgfsetdash{}{0pt}%
\pgfpathmoveto{\pgfqpoint{2.442308in}{6.046391in}}%
\pgfpathquadraticcurveto{\pgfqpoint{2.429080in}{6.045976in}}{\pgfqpoint{2.423613in}{6.045805in}}%
\pgfusepath{stroke}%
\end{pgfscope}%
\begin{pgfscope}%
\pgfpathrectangle{\pgfqpoint{0.800000in}{5.105882in}}{\pgfqpoint{2.407767in}{1.544118in}}%
\pgfusepath{clip}%
\pgfsetroundcap%
\pgfsetroundjoin%
\definecolor{currentfill}{rgb}{0.255645,0.260703,0.528312}%
\pgfsetfillcolor{currentfill}%
\pgfsetlinewidth{0.501875pt}%
\definecolor{currentstroke}{rgb}{0.255645,0.260703,0.528312}%
\pgfsetstrokecolor{currentstroke}%
\pgfsetdash{}{0pt}%
\pgfpathmoveto{\pgfqpoint{2.451812in}{6.032794in}}%
\pgfpathlineto{\pgfqpoint{2.423613in}{6.045805in}}%
\pgfpathlineto{\pgfqpoint{2.450941in}{6.060558in}}%
\pgfpathlineto{\pgfqpoint{2.451812in}{6.032794in}}%
\pgfpathlineto{\pgfqpoint{2.451812in}{6.032794in}}%
\pgfpathclose%
\pgfusepath{stroke,fill}%
\end{pgfscope}%
\begin{pgfscope}%
\pgfpathrectangle{\pgfqpoint{0.800000in}{5.105882in}}{\pgfqpoint{2.407767in}{1.544118in}}%
\pgfusepath{clip}%
\pgfsetroundcap%
\pgfsetroundjoin%
\pgfsetlinewidth{0.501875pt}%
\definecolor{currentstroke}{rgb}{0.253935,0.265254,0.529983}%
\pgfsetstrokecolor{currentstroke}%
\pgfsetdash{}{0pt}%
\pgfpathmoveto{\pgfqpoint{2.231180in}{6.069618in}}%
\pgfpathquadraticcurveto{\pgfqpoint{2.218060in}{6.068471in}}{\pgfqpoint{2.212674in}{6.068000in}}%
\pgfusepath{stroke}%
\end{pgfscope}%
\begin{pgfscope}%
\pgfpathrectangle{\pgfqpoint{0.800000in}{5.105882in}}{\pgfqpoint{2.407767in}{1.544118in}}%
\pgfusepath{clip}%
\pgfsetroundcap%
\pgfsetroundjoin%
\definecolor{currentfill}{rgb}{0.253935,0.265254,0.529983}%
\pgfsetfillcolor{currentfill}%
\pgfsetlinewidth{0.501875pt}%
\definecolor{currentstroke}{rgb}{0.253935,0.265254,0.529983}%
\pgfsetstrokecolor{currentstroke}%
\pgfsetdash{}{0pt}%
\pgfpathmoveto{\pgfqpoint{2.241556in}{6.056584in}}%
\pgfpathlineto{\pgfqpoint{2.212674in}{6.068000in}}%
\pgfpathlineto{\pgfqpoint{2.239136in}{6.084256in}}%
\pgfpathlineto{\pgfqpoint{2.241556in}{6.056584in}}%
\pgfpathlineto{\pgfqpoint{2.241556in}{6.056584in}}%
\pgfpathclose%
\pgfusepath{stroke,fill}%
\end{pgfscope}%
\begin{pgfscope}%
\pgfpathrectangle{\pgfqpoint{0.800000in}{5.105882in}}{\pgfqpoint{2.407767in}{1.544118in}}%
\pgfusepath{clip}%
\pgfsetroundcap%
\pgfsetroundjoin%
\pgfsetlinewidth{0.501875pt}%
\definecolor{currentstroke}{rgb}{0.281887,0.150881,0.465405}%
\pgfsetstrokecolor{currentstroke}%
\pgfsetdash{}{0pt}%
\pgfpathmoveto{\pgfqpoint{2.495282in}{6.116641in}}%
\pgfpathquadraticcurveto{\pgfqpoint{2.482061in}{6.116139in}}{\pgfqpoint{2.476600in}{6.115931in}}%
\pgfusepath{stroke}%
\end{pgfscope}%
\begin{pgfscope}%
\pgfpathrectangle{\pgfqpoint{0.800000in}{5.105882in}}{\pgfqpoint{2.407767in}{1.544118in}}%
\pgfusepath{clip}%
\pgfsetroundcap%
\pgfsetroundjoin%
\definecolor{currentfill}{rgb}{0.281887,0.150881,0.465405}%
\pgfsetfillcolor{currentfill}%
\pgfsetlinewidth{0.501875pt}%
\definecolor{currentstroke}{rgb}{0.281887,0.150881,0.465405}%
\pgfsetstrokecolor{currentstroke}%
\pgfsetdash{}{0pt}%
\pgfpathmoveto{\pgfqpoint{2.504885in}{6.103107in}}%
\pgfpathlineto{\pgfqpoint{2.476600in}{6.115931in}}%
\pgfpathlineto{\pgfqpoint{2.503830in}{6.130865in}}%
\pgfpathlineto{\pgfqpoint{2.504885in}{6.103107in}}%
\pgfpathlineto{\pgfqpoint{2.504885in}{6.103107in}}%
\pgfpathclose%
\pgfusepath{stroke,fill}%
\end{pgfscope}%
\begin{pgfscope}%
\pgfpathrectangle{\pgfqpoint{0.800000in}{5.105882in}}{\pgfqpoint{2.407767in}{1.544118in}}%
\pgfusepath{clip}%
\pgfsetroundcap%
\pgfsetroundjoin%
\pgfsetlinewidth{0.501875pt}%
\definecolor{currentstroke}{rgb}{0.283072,0.130895,0.449241}%
\pgfsetstrokecolor{currentstroke}%
\pgfsetdash{}{0pt}%
\pgfpathmoveto{\pgfqpoint{2.495434in}{6.149853in}}%
\pgfpathquadraticcurveto{\pgfqpoint{2.482217in}{6.149306in}}{\pgfqpoint{2.476758in}{6.149080in}}%
\pgfusepath{stroke}%
\end{pgfscope}%
\begin{pgfscope}%
\pgfpathrectangle{\pgfqpoint{0.800000in}{5.105882in}}{\pgfqpoint{2.407767in}{1.544118in}}%
\pgfusepath{clip}%
\pgfsetroundcap%
\pgfsetroundjoin%
\definecolor{currentfill}{rgb}{0.283072,0.130895,0.449241}%
\pgfsetfillcolor{currentfill}%
\pgfsetlinewidth{0.501875pt}%
\definecolor{currentstroke}{rgb}{0.283072,0.130895,0.449241}%
\pgfsetstrokecolor{currentstroke}%
\pgfsetdash{}{0pt}%
\pgfpathmoveto{\pgfqpoint{2.505086in}{6.136351in}}%
\pgfpathlineto{\pgfqpoint{2.476758in}{6.149080in}}%
\pgfpathlineto{\pgfqpoint{2.503939in}{6.164105in}}%
\pgfpathlineto{\pgfqpoint{2.505086in}{6.136351in}}%
\pgfpathlineto{\pgfqpoint{2.505086in}{6.136351in}}%
\pgfpathclose%
\pgfusepath{stroke,fill}%
\end{pgfscope}%
\begin{pgfscope}%
\pgfpathrectangle{\pgfqpoint{0.800000in}{5.105882in}}{\pgfqpoint{2.407767in}{1.544118in}}%
\pgfusepath{clip}%
\pgfsetroundcap%
\pgfsetroundjoin%
\pgfsetlinewidth{0.501875pt}%
\definecolor{currentstroke}{rgb}{0.278012,0.180367,0.486697}%
\pgfsetstrokecolor{currentstroke}%
\pgfsetdash{}{0pt}%
\pgfpathmoveto{\pgfqpoint{2.180260in}{6.159191in}}%
\pgfpathquadraticcurveto{\pgfqpoint{2.167484in}{6.156977in}}{\pgfqpoint{2.162357in}{6.156089in}}%
\pgfusepath{stroke}%
\end{pgfscope}%
\begin{pgfscope}%
\pgfpathrectangle{\pgfqpoint{0.800000in}{5.105882in}}{\pgfqpoint{2.407767in}{1.544118in}}%
\pgfusepath{clip}%
\pgfsetroundcap%
\pgfsetroundjoin%
\definecolor{currentfill}{rgb}{0.278012,0.180367,0.486697}%
\pgfsetfillcolor{currentfill}%
\pgfsetlinewidth{0.501875pt}%
\definecolor{currentstroke}{rgb}{0.278012,0.180367,0.486697}%
\pgfsetstrokecolor{currentstroke}%
\pgfsetdash{}{0pt}%
\pgfpathmoveto{\pgfqpoint{2.192098in}{6.147146in}}%
\pgfpathlineto{\pgfqpoint{2.162357in}{6.156089in}}%
\pgfpathlineto{\pgfqpoint{2.187356in}{6.174516in}}%
\pgfpathlineto{\pgfqpoint{2.192098in}{6.147146in}}%
\pgfpathlineto{\pgfqpoint{2.192098in}{6.147146in}}%
\pgfpathclose%
\pgfusepath{stroke,fill}%
\end{pgfscope}%
\begin{pgfscope}%
\pgfpathrectangle{\pgfqpoint{0.800000in}{5.105882in}}{\pgfqpoint{2.407767in}{1.544118in}}%
\pgfusepath{clip}%
\pgfsetroundcap%
\pgfsetroundjoin%
\pgfsetlinewidth{0.501875pt}%
\definecolor{currentstroke}{rgb}{0.277018,0.050344,0.375715}%
\pgfsetstrokecolor{currentstroke}%
\pgfsetdash{}{0pt}%
\pgfpathmoveto{\pgfqpoint{2.548527in}{6.219459in}}%
\pgfpathquadraticcurveto{\pgfqpoint{2.535341in}{6.218672in}}{\pgfqpoint{2.529905in}{6.218348in}}%
\pgfusepath{stroke}%
\end{pgfscope}%
\begin{pgfscope}%
\pgfpathrectangle{\pgfqpoint{0.800000in}{5.105882in}}{\pgfqpoint{2.407767in}{1.544118in}}%
\pgfusepath{clip}%
\pgfsetroundcap%
\pgfsetroundjoin%
\definecolor{currentfill}{rgb}{0.277018,0.050344,0.375715}%
\pgfsetfillcolor{currentfill}%
\pgfsetlinewidth{0.501875pt}%
\definecolor{currentstroke}{rgb}{0.277018,0.050344,0.375715}%
\pgfsetstrokecolor{currentstroke}%
\pgfsetdash{}{0pt}%
\pgfpathmoveto{\pgfqpoint{2.558461in}{6.206139in}}%
\pgfpathlineto{\pgfqpoint{2.529905in}{6.218348in}}%
\pgfpathlineto{\pgfqpoint{2.556806in}{6.233867in}}%
\pgfpathlineto{\pgfqpoint{2.558461in}{6.206139in}}%
\pgfpathlineto{\pgfqpoint{2.558461in}{6.206139in}}%
\pgfpathclose%
\pgfusepath{stroke,fill}%
\end{pgfscope}%
\begin{pgfscope}%
\pgfpathrectangle{\pgfqpoint{0.800000in}{5.105882in}}{\pgfqpoint{2.407767in}{1.544118in}}%
\pgfusepath{clip}%
\pgfsetroundcap%
\pgfsetroundjoin%
\pgfsetlinewidth{0.501875pt}%
\definecolor{currentstroke}{rgb}{0.283229,0.120777,0.440584}%
\pgfsetstrokecolor{currentstroke}%
\pgfsetdash{}{0pt}%
\pgfpathmoveto{\pgfqpoint{2.233824in}{6.231922in}}%
\pgfpathquadraticcurveto{\pgfqpoint{2.220910in}{6.230095in}}{\pgfqpoint{2.215683in}{6.229356in}}%
\pgfusepath{stroke}%
\end{pgfscope}%
\begin{pgfscope}%
\pgfpathrectangle{\pgfqpoint{0.800000in}{5.105882in}}{\pgfqpoint{2.407767in}{1.544118in}}%
\pgfusepath{clip}%
\pgfsetroundcap%
\pgfsetroundjoin%
\definecolor{currentfill}{rgb}{0.283229,0.120777,0.440584}%
\pgfsetfillcolor{currentfill}%
\pgfsetlinewidth{0.501875pt}%
\definecolor{currentstroke}{rgb}{0.283229,0.120777,0.440584}%
\pgfsetstrokecolor{currentstroke}%
\pgfsetdash{}{0pt}%
\pgfpathmoveto{\pgfqpoint{2.245132in}{6.219495in}}%
\pgfpathlineto{\pgfqpoint{2.215683in}{6.229356in}}%
\pgfpathlineto{\pgfqpoint{2.241241in}{6.246998in}}%
\pgfpathlineto{\pgfqpoint{2.245132in}{6.219495in}}%
\pgfpathlineto{\pgfqpoint{2.245132in}{6.219495in}}%
\pgfpathclose%
\pgfusepath{stroke,fill}%
\end{pgfscope}%
\begin{pgfscope}%
\pgfpathrectangle{\pgfqpoint{0.800000in}{5.105882in}}{\pgfqpoint{2.407767in}{1.544118in}}%
\pgfusepath{clip}%
\pgfsetroundcap%
\pgfsetroundjoin%
\pgfsetlinewidth{0.501875pt}%
\definecolor{currentstroke}{rgb}{0.271305,0.019942,0.347269}%
\pgfsetstrokecolor{currentstroke}%
\pgfsetdash{}{0pt}%
\pgfpathmoveto{\pgfqpoint{2.545685in}{6.294895in}}%
\pgfpathquadraticcurveto{\pgfqpoint{2.532649in}{6.293541in}}{\pgfqpoint{2.527336in}{6.292989in}}%
\pgfusepath{stroke}%
\end{pgfscope}%
\begin{pgfscope}%
\pgfpathrectangle{\pgfqpoint{0.800000in}{5.105882in}}{\pgfqpoint{2.407767in}{1.544118in}}%
\pgfusepath{clip}%
\pgfsetroundcap%
\pgfsetroundjoin%
\definecolor{currentfill}{rgb}{0.271305,0.019942,0.347269}%
\pgfsetfillcolor{currentfill}%
\pgfsetlinewidth{0.501875pt}%
\definecolor{currentstroke}{rgb}{0.271305,0.019942,0.347269}%
\pgfsetstrokecolor{currentstroke}%
\pgfsetdash{}{0pt}%
\pgfpathmoveto{\pgfqpoint{2.556400in}{6.282044in}}%
\pgfpathlineto{\pgfqpoint{2.527336in}{6.292989in}}%
\pgfpathlineto{\pgfqpoint{2.553530in}{6.309673in}}%
\pgfpathlineto{\pgfqpoint{2.556400in}{6.282044in}}%
\pgfpathlineto{\pgfqpoint{2.556400in}{6.282044in}}%
\pgfpathclose%
\pgfusepath{stroke,fill}%
\end{pgfscope}%
\begin{pgfscope}%
\pgfpathrectangle{\pgfqpoint{0.800000in}{5.105882in}}{\pgfqpoint{2.407767in}{1.544118in}}%
\pgfusepath{clip}%
\pgfsetroundcap%
\pgfsetroundjoin%
\pgfsetlinewidth{0.501875pt}%
\definecolor{currentstroke}{rgb}{0.282327,0.094955,0.417331}%
\pgfsetstrokecolor{currentstroke}%
\pgfsetdash{}{0pt}%
\pgfpathmoveto{\pgfqpoint{2.068296in}{6.235237in}}%
\pgfpathquadraticcurveto{\pgfqpoint{2.056803in}{6.231062in}}{\pgfqpoint{2.052608in}{6.229538in}}%
\pgfusepath{stroke}%
\end{pgfscope}%
\begin{pgfscope}%
\pgfpathrectangle{\pgfqpoint{0.800000in}{5.105882in}}{\pgfqpoint{2.407767in}{1.544118in}}%
\pgfusepath{clip}%
\pgfsetroundcap%
\pgfsetroundjoin%
\definecolor{currentfill}{rgb}{0.282327,0.094955,0.417331}%
\pgfsetfillcolor{currentfill}%
\pgfsetlinewidth{0.501875pt}%
\definecolor{currentstroke}{rgb}{0.282327,0.094955,0.417331}%
\pgfsetstrokecolor{currentstroke}%
\pgfsetdash{}{0pt}%
\pgfpathmoveto{\pgfqpoint{2.083459in}{6.225968in}}%
\pgfpathlineto{\pgfqpoint{2.052608in}{6.229538in}}%
\pgfpathlineto{\pgfqpoint{2.073975in}{6.252077in}}%
\pgfpathlineto{\pgfqpoint{2.083459in}{6.225968in}}%
\pgfpathlineto{\pgfqpoint{2.083459in}{6.225968in}}%
\pgfpathclose%
\pgfusepath{stroke,fill}%
\end{pgfscope}%
\begin{pgfscope}%
\pgfpathrectangle{\pgfqpoint{0.800000in}{5.105882in}}{\pgfqpoint{2.407767in}{1.544118in}}%
\pgfusepath{clip}%
\pgfsetroundcap%
\pgfsetroundjoin%
\pgfsetlinewidth{0.501875pt}%
\definecolor{currentstroke}{rgb}{0.267968,0.223549,0.512008}%
\pgfsetstrokecolor{currentstroke}%
\pgfsetdash{}{0pt}%
\pgfpathmoveto{\pgfqpoint{1.960567in}{5.661132in}}%
\pgfpathquadraticcurveto{\pgfqpoint{1.948858in}{5.665088in}}{\pgfqpoint{1.944505in}{5.666559in}}%
\pgfusepath{stroke}%
\end{pgfscope}%
\begin{pgfscope}%
\pgfpathrectangle{\pgfqpoint{0.800000in}{5.105882in}}{\pgfqpoint{2.407767in}{1.544118in}}%
\pgfusepath{clip}%
\pgfsetroundcap%
\pgfsetroundjoin%
\definecolor{currentfill}{rgb}{0.267968,0.223549,0.512008}%
\pgfsetfillcolor{currentfill}%
\pgfsetlinewidth{0.501875pt}%
\definecolor{currentstroke}{rgb}{0.267968,0.223549,0.512008}%
\pgfsetstrokecolor{currentstroke}%
\pgfsetdash{}{0pt}%
\pgfpathmoveto{\pgfqpoint{1.966374in}{5.644509in}}%
\pgfpathlineto{\pgfqpoint{1.944505in}{5.666559in}}%
\pgfpathlineto{\pgfqpoint{1.975267in}{5.670825in}}%
\pgfpathlineto{\pgfqpoint{1.966374in}{5.644509in}}%
\pgfpathlineto{\pgfqpoint{1.966374in}{5.644509in}}%
\pgfpathclose%
\pgfusepath{stroke,fill}%
\end{pgfscope}%
\begin{pgfscope}%
\pgfpathrectangle{\pgfqpoint{0.800000in}{5.105882in}}{\pgfqpoint{2.407767in}{1.544118in}}%
\pgfusepath{clip}%
\pgfsetroundcap%
\pgfsetroundjoin%
\pgfsetlinewidth{0.501875pt}%
\definecolor{currentstroke}{rgb}{0.252194,0.269783,0.531579}%
\pgfsetstrokecolor{currentstroke}%
\pgfsetdash{}{0pt}%
\pgfpathmoveto{\pgfqpoint{1.953498in}{5.971290in}}%
\pgfpathquadraticcurveto{\pgfqpoint{1.940503in}{5.969658in}}{\pgfqpoint{1.935212in}{5.968994in}}%
\pgfusepath{stroke}%
\end{pgfscope}%
\begin{pgfscope}%
\pgfpathrectangle{\pgfqpoint{0.800000in}{5.105882in}}{\pgfqpoint{2.407767in}{1.544118in}}%
\pgfusepath{clip}%
\pgfsetroundcap%
\pgfsetroundjoin%
\definecolor{currentfill}{rgb}{0.252194,0.269783,0.531579}%
\pgfsetfillcolor{currentfill}%
\pgfsetlinewidth{0.501875pt}%
\definecolor{currentstroke}{rgb}{0.252194,0.269783,0.531579}%
\pgfsetstrokecolor{currentstroke}%
\pgfsetdash{}{0pt}%
\pgfpathmoveto{\pgfqpoint{1.964504in}{5.958674in}}%
\pgfpathlineto{\pgfqpoint{1.935212in}{5.968994in}}%
\pgfpathlineto{\pgfqpoint{1.961043in}{5.986235in}}%
\pgfpathlineto{\pgfqpoint{1.964504in}{5.958674in}}%
\pgfpathlineto{\pgfqpoint{1.964504in}{5.958674in}}%
\pgfpathclose%
\pgfusepath{stroke,fill}%
\end{pgfscope}%
\begin{pgfscope}%
\pgfpathrectangle{\pgfqpoint{0.800000in}{5.105882in}}{\pgfqpoint{2.407767in}{1.544118in}}%
\pgfusepath{clip}%
\pgfsetbuttcap%
\pgfsetroundjoin%
\pgfsetlinewidth{1.505625pt}%
\definecolor{currentstroke}{rgb}{0.000000,0.000000,0.000000}%
\pgfsetstrokecolor{currentstroke}%
\pgfsetdash{}{0pt}%
\pgfpathmoveto{\pgfqpoint{1.789710in}{5.367356in}}%
\pgfpathlineto{\pgfqpoint{1.789710in}{6.388526in}}%
\pgfusepath{stroke}%
\end{pgfscope}%
\begin{pgfscope}%
\pgfpathrectangle{\pgfqpoint{0.800000in}{5.105882in}}{\pgfqpoint{2.407767in}{1.544118in}}%
\pgfusepath{clip}%
\pgfsetbuttcap%
\pgfsetroundjoin%
\pgfsetlinewidth{1.505625pt}%
\definecolor{currentstroke}{rgb}{0.000000,0.000000,0.000000}%
\pgfsetstrokecolor{currentstroke}%
\pgfsetdash{}{0pt}%
\pgfpathmoveto{\pgfqpoint{2.699611in}{5.367356in}}%
\pgfpathlineto{\pgfqpoint{2.699611in}{6.388526in}}%
\pgfusepath{stroke}%
\end{pgfscope}%
\begin{pgfscope}%
\pgfsetrectcap%
\pgfsetmiterjoin%
\pgfsetlinewidth{0.803000pt}%
\definecolor{currentstroke}{rgb}{0.000000,0.000000,0.000000}%
\pgfsetstrokecolor{currentstroke}%
\pgfsetdash{}{0pt}%
\pgfpathmoveto{\pgfqpoint{0.800000in}{5.105882in}}%
\pgfpathlineto{\pgfqpoint{0.800000in}{6.650000in}}%
\pgfusepath{stroke}%
\end{pgfscope}%
\begin{pgfscope}%
\pgfsetrectcap%
\pgfsetmiterjoin%
\pgfsetlinewidth{0.803000pt}%
\definecolor{currentstroke}{rgb}{0.000000,0.000000,0.000000}%
\pgfsetstrokecolor{currentstroke}%
\pgfsetdash{}{0pt}%
\pgfpathmoveto{\pgfqpoint{3.207767in}{5.105882in}}%
\pgfpathlineto{\pgfqpoint{3.207767in}{6.650000in}}%
\pgfusepath{stroke}%
\end{pgfscope}%
\begin{pgfscope}%
\pgfsetrectcap%
\pgfsetmiterjoin%
\pgfsetlinewidth{0.803000pt}%
\definecolor{currentstroke}{rgb}{0.000000,0.000000,0.000000}%
\pgfsetstrokecolor{currentstroke}%
\pgfsetdash{}{0pt}%
\pgfpathmoveto{\pgfqpoint{0.800000in}{5.105882in}}%
\pgfpathlineto{\pgfqpoint{3.207767in}{5.105882in}}%
\pgfusepath{stroke}%
\end{pgfscope}%
\begin{pgfscope}%
\pgfsetrectcap%
\pgfsetmiterjoin%
\pgfsetlinewidth{0.803000pt}%
\definecolor{currentstroke}{rgb}{0.000000,0.000000,0.000000}%
\pgfsetstrokecolor{currentstroke}%
\pgfsetdash{}{0pt}%
\pgfpathmoveto{\pgfqpoint{0.800000in}{6.650000in}}%
\pgfpathlineto{\pgfqpoint{3.207767in}{6.650000in}}%
\pgfusepath{stroke}%
\end{pgfscope}%
\begin{pgfscope}%
\definecolor{textcolor}{rgb}{0.000000,0.000000,0.000000}%
\pgfsetstrokecolor{textcolor}%
\pgfsetfillcolor{textcolor}%
\pgftext[x=2.003883in,y=6.733333in,,base]{\color{textcolor}\sffamily\fontsize{12.000000}{14.400000}\selectfont a)}%
\end{pgfscope}%
\begin{pgfscope}%
\pgfsetbuttcap%
\pgfsetmiterjoin%
\definecolor{currentfill}{rgb}{1.000000,1.000000,1.000000}%
\pgfsetfillcolor{currentfill}%
\pgfsetlinewidth{0.000000pt}%
\definecolor{currentstroke}{rgb}{0.000000,0.000000,0.000000}%
\pgfsetstrokecolor{currentstroke}%
\pgfsetstrokeopacity{0.000000}%
\pgfsetdash{}{0pt}%
\pgfpathmoveto{\pgfqpoint{3.352233in}{5.105882in}}%
\pgfpathlineto{\pgfqpoint{5.760000in}{5.105882in}}%
\pgfpathlineto{\pgfqpoint{5.760000in}{6.650000in}}%
\pgfpathlineto{\pgfqpoint{3.352233in}{6.650000in}}%
\pgfpathlineto{\pgfqpoint{3.352233in}{5.105882in}}%
\pgfpathclose%
\pgfusepath{fill}%
\end{pgfscope}%
\begin{pgfscope}%
\pgfpathrectangle{\pgfqpoint{3.352233in}{5.105882in}}{\pgfqpoint{2.407767in}{1.544118in}}%
\pgfusepath{clip}%
\pgfsys@transformcm{2.416667}{0.000000}{0.000000}{1.555556}{3.352233in}{5.105882in}%
\pgftext[left,bottom]{\includegraphics[interpolate=false,width=1.000000in,height=1.000000in]{q_series_square-img1.png}}%
\end{pgfscope}%
\begin{pgfscope}%
\pgfsetbuttcap%
\pgfsetroundjoin%
\definecolor{currentfill}{rgb}{0.000000,0.000000,0.000000}%
\pgfsetfillcolor{currentfill}%
\pgfsetlinewidth{0.803000pt}%
\definecolor{currentstroke}{rgb}{0.000000,0.000000,0.000000}%
\pgfsetstrokecolor{currentstroke}%
\pgfsetdash{}{0pt}%
\pgfsys@defobject{currentmarker}{\pgfqpoint{0.000000in}{-0.048611in}}{\pgfqpoint{0.000000in}{0.000000in}}{%
\pgfpathmoveto{\pgfqpoint{0.000000in}{0.000000in}}%
\pgfpathlineto{\pgfqpoint{0.000000in}{-0.048611in}}%
\pgfusepath{stroke,fill}%
}%
\begin{pgfscope}%
\pgfsys@transformshift{3.785892in}{5.105882in}%
\pgfsys@useobject{currentmarker}{}%
\end{pgfscope}%
\end{pgfscope}%
\begin{pgfscope}%
\pgfsetbuttcap%
\pgfsetroundjoin%
\definecolor{currentfill}{rgb}{0.000000,0.000000,0.000000}%
\pgfsetfillcolor{currentfill}%
\pgfsetlinewidth{0.803000pt}%
\definecolor{currentstroke}{rgb}{0.000000,0.000000,0.000000}%
\pgfsetstrokecolor{currentstroke}%
\pgfsetdash{}{0pt}%
\pgfsys@defobject{currentmarker}{\pgfqpoint{0.000000in}{-0.048611in}}{\pgfqpoint{0.000000in}{0.000000in}}{%
\pgfpathmoveto{\pgfqpoint{0.000000in}{0.000000in}}%
\pgfpathlineto{\pgfqpoint{0.000000in}{-0.048611in}}%
\pgfusepath{stroke,fill}%
}%
\begin{pgfscope}%
\pgfsys@transformshift{4.291393in}{5.105882in}%
\pgfsys@useobject{currentmarker}{}%
\end{pgfscope}%
\end{pgfscope}%
\begin{pgfscope}%
\pgfsetbuttcap%
\pgfsetroundjoin%
\definecolor{currentfill}{rgb}{0.000000,0.000000,0.000000}%
\pgfsetfillcolor{currentfill}%
\pgfsetlinewidth{0.803000pt}%
\definecolor{currentstroke}{rgb}{0.000000,0.000000,0.000000}%
\pgfsetstrokecolor{currentstroke}%
\pgfsetdash{}{0pt}%
\pgfsys@defobject{currentmarker}{\pgfqpoint{0.000000in}{-0.048611in}}{\pgfqpoint{0.000000in}{0.000000in}}{%
\pgfpathmoveto{\pgfqpoint{0.000000in}{0.000000in}}%
\pgfpathlineto{\pgfqpoint{0.000000in}{-0.048611in}}%
\pgfusepath{stroke,fill}%
}%
\begin{pgfscope}%
\pgfsys@transformshift{4.796893in}{5.105882in}%
\pgfsys@useobject{currentmarker}{}%
\end{pgfscope}%
\end{pgfscope}%
\begin{pgfscope}%
\pgfsetbuttcap%
\pgfsetroundjoin%
\definecolor{currentfill}{rgb}{0.000000,0.000000,0.000000}%
\pgfsetfillcolor{currentfill}%
\pgfsetlinewidth{0.803000pt}%
\definecolor{currentstroke}{rgb}{0.000000,0.000000,0.000000}%
\pgfsetstrokecolor{currentstroke}%
\pgfsetdash{}{0pt}%
\pgfsys@defobject{currentmarker}{\pgfqpoint{0.000000in}{-0.048611in}}{\pgfqpoint{0.000000in}{0.000000in}}{%
\pgfpathmoveto{\pgfqpoint{0.000000in}{0.000000in}}%
\pgfpathlineto{\pgfqpoint{0.000000in}{-0.048611in}}%
\pgfusepath{stroke,fill}%
}%
\begin{pgfscope}%
\pgfsys@transformshift{5.302394in}{5.105882in}%
\pgfsys@useobject{currentmarker}{}%
\end{pgfscope}%
\end{pgfscope}%
\begin{pgfscope}%
\pgfsetbuttcap%
\pgfsetroundjoin%
\definecolor{currentfill}{rgb}{0.000000,0.000000,0.000000}%
\pgfsetfillcolor{currentfill}%
\pgfsetlinewidth{0.803000pt}%
\definecolor{currentstroke}{rgb}{0.000000,0.000000,0.000000}%
\pgfsetstrokecolor{currentstroke}%
\pgfsetdash{}{0pt}%
\pgfsys@defobject{currentmarker}{\pgfqpoint{-0.048611in}{0.000000in}}{\pgfqpoint{-0.000000in}{0.000000in}}{%
\pgfpathmoveto{\pgfqpoint{-0.000000in}{0.000000in}}%
\pgfpathlineto{\pgfqpoint{-0.048611in}{0.000000in}}%
\pgfusepath{stroke,fill}%
}%
\begin{pgfscope}%
\pgfsys@transformshift{3.352233in}{5.367356in}%
\pgfsys@useobject{currentmarker}{}%
\end{pgfscope}%
\end{pgfscope}%
\begin{pgfscope}%
\pgfsetbuttcap%
\pgfsetroundjoin%
\definecolor{currentfill}{rgb}{0.000000,0.000000,0.000000}%
\pgfsetfillcolor{currentfill}%
\pgfsetlinewidth{0.803000pt}%
\definecolor{currentstroke}{rgb}{0.000000,0.000000,0.000000}%
\pgfsetstrokecolor{currentstroke}%
\pgfsetdash{}{0pt}%
\pgfsys@defobject{currentmarker}{\pgfqpoint{-0.048611in}{0.000000in}}{\pgfqpoint{-0.000000in}{0.000000in}}{%
\pgfpathmoveto{\pgfqpoint{-0.000000in}{0.000000in}}%
\pgfpathlineto{\pgfqpoint{-0.048611in}{0.000000in}}%
\pgfusepath{stroke,fill}%
}%
\begin{pgfscope}%
\pgfsys@transformshift{3.352233in}{5.877941in}%
\pgfsys@useobject{currentmarker}{}%
\end{pgfscope}%
\end{pgfscope}%
\begin{pgfscope}%
\pgfsetbuttcap%
\pgfsetroundjoin%
\definecolor{currentfill}{rgb}{0.000000,0.000000,0.000000}%
\pgfsetfillcolor{currentfill}%
\pgfsetlinewidth{0.803000pt}%
\definecolor{currentstroke}{rgb}{0.000000,0.000000,0.000000}%
\pgfsetstrokecolor{currentstroke}%
\pgfsetdash{}{0pt}%
\pgfsys@defobject{currentmarker}{\pgfqpoint{-0.048611in}{0.000000in}}{\pgfqpoint{-0.000000in}{0.000000in}}{%
\pgfpathmoveto{\pgfqpoint{-0.000000in}{0.000000in}}%
\pgfpathlineto{\pgfqpoint{-0.048611in}{0.000000in}}%
\pgfusepath{stroke,fill}%
}%
\begin{pgfscope}%
\pgfsys@transformshift{3.352233in}{6.388526in}%
\pgfsys@useobject{currentmarker}{}%
\end{pgfscope}%
\end{pgfscope}%
\begin{pgfscope}%
\pgfpathrectangle{\pgfqpoint{3.352233in}{5.105882in}}{\pgfqpoint{2.407767in}{1.544118in}}%
\pgfusepath{clip}%
\pgfsetbuttcap%
\pgfsetroundjoin%
\pgfsetlinewidth{0.501875pt}%
\definecolor{currentstroke}{rgb}{0.277941,0.056324,0.381191}%
\pgfsetstrokecolor{currentstroke}%
\pgfsetdash{}{0pt}%
\pgfpathmoveto{\pgfqpoint{4.556117in}{5.426242in}}%
\pgfpathlineto{\pgfqpoint{4.556117in}{5.426242in}}%
\pgfusepath{stroke}%
\end{pgfscope}%
\begin{pgfscope}%
\pgfpathrectangle{\pgfqpoint{3.352233in}{5.105882in}}{\pgfqpoint{2.407767in}{1.544118in}}%
\pgfusepath{clip}%
\pgfsetbuttcap%
\pgfsetroundjoin%
\pgfsetlinewidth{0.501875pt}%
\definecolor{currentstroke}{rgb}{0.277941,0.056324,0.381191}%
\pgfsetstrokecolor{currentstroke}%
\pgfsetdash{}{0pt}%
\pgfpathmoveto{\pgfqpoint{4.556117in}{5.426242in}}%
\pgfpathlineto{\pgfqpoint{4.556117in}{5.426242in}}%
\pgfusepath{stroke}%
\end{pgfscope}%
\begin{pgfscope}%
\pgfpathrectangle{\pgfqpoint{3.352233in}{5.105882in}}{\pgfqpoint{2.407767in}{1.544118in}}%
\pgfusepath{clip}%
\pgfsetbuttcap%
\pgfsetroundjoin%
\pgfsetlinewidth{0.501875pt}%
\definecolor{currentstroke}{rgb}{0.277941,0.056324,0.381191}%
\pgfsetstrokecolor{currentstroke}%
\pgfsetdash{}{0pt}%
\pgfpathmoveto{\pgfqpoint{4.556117in}{5.426242in}}%
\pgfpathlineto{\pgfqpoint{4.535157in}{5.442133in}}%
\pgfusepath{stroke}%
\end{pgfscope}%
\begin{pgfscope}%
\pgfpathrectangle{\pgfqpoint{3.352233in}{5.105882in}}{\pgfqpoint{2.407767in}{1.544118in}}%
\pgfusepath{clip}%
\pgfsetbuttcap%
\pgfsetroundjoin%
\pgfsetlinewidth{0.501875pt}%
\definecolor{currentstroke}{rgb}{0.279566,0.067836,0.391917}%
\pgfsetstrokecolor{currentstroke}%
\pgfsetdash{}{0pt}%
\pgfpathmoveto{\pgfqpoint{4.535157in}{5.442133in}}%
\pgfpathlineto{\pgfqpoint{4.519435in}{5.457732in}}%
\pgfusepath{stroke}%
\end{pgfscope}%
\begin{pgfscope}%
\pgfpathrectangle{\pgfqpoint{3.352233in}{5.105882in}}{\pgfqpoint{2.407767in}{1.544118in}}%
\pgfusepath{clip}%
\pgfsetbuttcap%
\pgfsetroundjoin%
\pgfsetlinewidth{0.501875pt}%
\definecolor{currentstroke}{rgb}{0.280267,0.073417,0.397163}%
\pgfsetstrokecolor{currentstroke}%
\pgfsetdash{}{0pt}%
\pgfpathmoveto{\pgfqpoint{4.519435in}{5.457732in}}%
\pgfpathlineto{\pgfqpoint{4.493110in}{5.485260in}}%
\pgfusepath{stroke}%
\end{pgfscope}%
\begin{pgfscope}%
\pgfpathrectangle{\pgfqpoint{3.352233in}{5.105882in}}{\pgfqpoint{2.407767in}{1.544118in}}%
\pgfusepath{clip}%
\pgfsetbuttcap%
\pgfsetroundjoin%
\pgfsetlinewidth{0.501875pt}%
\definecolor{currentstroke}{rgb}{0.279566,0.067836,0.391917}%
\pgfsetstrokecolor{currentstroke}%
\pgfsetdash{}{0pt}%
\pgfpathmoveto{\pgfqpoint{4.493110in}{5.485260in}}%
\pgfpathlineto{\pgfqpoint{4.469025in}{5.514171in}}%
\pgfusepath{stroke}%
\end{pgfscope}%
\begin{pgfscope}%
\pgfpathrectangle{\pgfqpoint{3.352233in}{5.105882in}}{\pgfqpoint{2.407767in}{1.544118in}}%
\pgfusepath{clip}%
\pgfsetbuttcap%
\pgfsetroundjoin%
\pgfsetlinewidth{0.501875pt}%
\definecolor{currentstroke}{rgb}{0.280894,0.078907,0.402329}%
\pgfsetstrokecolor{currentstroke}%
\pgfsetdash{}{0pt}%
\pgfpathmoveto{\pgfqpoint{4.469025in}{5.514171in}}%
\pgfpathlineto{\pgfqpoint{4.447645in}{5.544925in}}%
\pgfusepath{stroke}%
\end{pgfscope}%
\begin{pgfscope}%
\pgfpathrectangle{\pgfqpoint{3.352233in}{5.105882in}}{\pgfqpoint{2.407767in}{1.544118in}}%
\pgfusepath{clip}%
\pgfsetbuttcap%
\pgfsetroundjoin%
\pgfsetlinewidth{0.501875pt}%
\definecolor{currentstroke}{rgb}{0.283229,0.120777,0.440584}%
\pgfsetstrokecolor{currentstroke}%
\pgfsetdash{}{0pt}%
\pgfpathmoveto{\pgfqpoint{4.447645in}{5.544925in}}%
\pgfpathlineto{\pgfqpoint{4.432701in}{5.577072in}}%
\pgfusepath{stroke}%
\end{pgfscope}%
\begin{pgfscope}%
\pgfpathrectangle{\pgfqpoint{3.352233in}{5.105882in}}{\pgfqpoint{2.407767in}{1.544118in}}%
\pgfusepath{clip}%
\pgfsetbuttcap%
\pgfsetroundjoin%
\pgfsetlinewidth{0.501875pt}%
\definecolor{currentstroke}{rgb}{0.283197,0.115680,0.436115}%
\pgfsetstrokecolor{currentstroke}%
\pgfsetdash{}{0pt}%
\pgfpathmoveto{\pgfqpoint{4.432701in}{5.577072in}}%
\pgfpathlineto{\pgfqpoint{4.417478in}{5.606733in}}%
\pgfusepath{stroke}%
\end{pgfscope}%
\begin{pgfscope}%
\pgfpathrectangle{\pgfqpoint{3.352233in}{5.105882in}}{\pgfqpoint{2.407767in}{1.544118in}}%
\pgfusepath{clip}%
\pgfsetbuttcap%
\pgfsetroundjoin%
\pgfsetlinewidth{0.501875pt}%
\definecolor{currentstroke}{rgb}{0.280255,0.165693,0.476498}%
\pgfsetstrokecolor{currentstroke}%
\pgfsetdash{}{0pt}%
\pgfpathmoveto{\pgfqpoint{4.417478in}{5.606733in}}%
\pgfpathlineto{\pgfqpoint{4.394870in}{5.637140in}}%
\pgfusepath{stroke}%
\end{pgfscope}%
\begin{pgfscope}%
\pgfpathrectangle{\pgfqpoint{3.352233in}{5.105882in}}{\pgfqpoint{2.407767in}{1.544118in}}%
\pgfusepath{clip}%
\pgfsetbuttcap%
\pgfsetroundjoin%
\pgfsetlinewidth{0.501875pt}%
\definecolor{currentstroke}{rgb}{0.269308,0.218818,0.509577}%
\pgfsetstrokecolor{currentstroke}%
\pgfsetdash{}{0pt}%
\pgfpathmoveto{\pgfqpoint{4.394870in}{5.637140in}}%
\pgfpathlineto{\pgfqpoint{4.369181in}{5.666696in}}%
\pgfusepath{stroke}%
\end{pgfscope}%
\begin{pgfscope}%
\pgfpathrectangle{\pgfqpoint{3.352233in}{5.105882in}}{\pgfqpoint{2.407767in}{1.544118in}}%
\pgfusepath{clip}%
\pgfsetbuttcap%
\pgfsetroundjoin%
\pgfsetlinewidth{0.501875pt}%
\definecolor{currentstroke}{rgb}{0.262138,0.242286,0.520837}%
\pgfsetstrokecolor{currentstroke}%
\pgfsetdash{}{0pt}%
\pgfpathmoveto{\pgfqpoint{4.369181in}{5.666696in}}%
\pgfpathlineto{\pgfqpoint{4.344600in}{5.695675in}}%
\pgfusepath{stroke}%
\end{pgfscope}%
\begin{pgfscope}%
\pgfpathrectangle{\pgfqpoint{3.352233in}{5.105882in}}{\pgfqpoint{2.407767in}{1.544118in}}%
\pgfusepath{clip}%
\pgfsetbuttcap%
\pgfsetroundjoin%
\pgfsetlinewidth{0.501875pt}%
\definecolor{currentstroke}{rgb}{0.272594,0.025563,0.353093}%
\pgfsetstrokecolor{currentstroke}%
\pgfsetdash{}{0pt}%
\pgfpathmoveto{\pgfqpoint{5.093034in}{5.414644in}}%
\pgfpathlineto{\pgfqpoint{5.040335in}{5.417215in}}%
\pgfusepath{stroke}%
\end{pgfscope}%
\begin{pgfscope}%
\pgfpathrectangle{\pgfqpoint{3.352233in}{5.105882in}}{\pgfqpoint{2.407767in}{1.544118in}}%
\pgfusepath{clip}%
\pgfsetbuttcap%
\pgfsetroundjoin%
\pgfsetlinewidth{0.501875pt}%
\definecolor{currentstroke}{rgb}{0.272594,0.025563,0.353093}%
\pgfsetstrokecolor{currentstroke}%
\pgfsetdash{}{0pt}%
\pgfpathmoveto{\pgfqpoint{5.040335in}{5.417215in}}%
\pgfpathlineto{\pgfqpoint{4.987881in}{5.421792in}}%
\pgfusepath{stroke}%
\end{pgfscope}%
\begin{pgfscope}%
\pgfpathrectangle{\pgfqpoint{3.352233in}{5.105882in}}{\pgfqpoint{2.407767in}{1.544118in}}%
\pgfusepath{clip}%
\pgfsetbuttcap%
\pgfsetroundjoin%
\pgfsetlinewidth{0.501875pt}%
\definecolor{currentstroke}{rgb}{0.273809,0.031497,0.358853}%
\pgfsetstrokecolor{currentstroke}%
\pgfsetdash{}{0pt}%
\pgfpathmoveto{\pgfqpoint{4.987881in}{5.421792in}}%
\pgfpathlineto{\pgfqpoint{4.935378in}{5.426242in}}%
\pgfusepath{stroke}%
\end{pgfscope}%
\begin{pgfscope}%
\pgfpathrectangle{\pgfqpoint{3.352233in}{5.105882in}}{\pgfqpoint{2.407767in}{1.544118in}}%
\pgfusepath{clip}%
\pgfsetbuttcap%
\pgfsetroundjoin%
\pgfsetlinewidth{0.501875pt}%
\definecolor{currentstroke}{rgb}{0.278791,0.062145,0.386592}%
\pgfsetstrokecolor{currentstroke}%
\pgfsetdash{}{0pt}%
\pgfpathmoveto{\pgfqpoint{4.935378in}{5.426242in}}%
\pgfpathlineto{\pgfqpoint{4.883092in}{5.431482in}}%
\pgfusepath{stroke}%
\end{pgfscope}%
\begin{pgfscope}%
\pgfpathrectangle{\pgfqpoint{3.352233in}{5.105882in}}{\pgfqpoint{2.407767in}{1.544118in}}%
\pgfusepath{clip}%
\pgfsetbuttcap%
\pgfsetroundjoin%
\pgfsetlinewidth{0.501875pt}%
\definecolor{currentstroke}{rgb}{0.277018,0.050344,0.375715}%
\pgfsetstrokecolor{currentstroke}%
\pgfsetdash{}{0pt}%
\pgfpathmoveto{\pgfqpoint{4.883092in}{5.431482in}}%
\pgfpathlineto{\pgfqpoint{4.831110in}{5.437858in}}%
\pgfusepath{stroke}%
\end{pgfscope}%
\begin{pgfscope}%
\pgfpathrectangle{\pgfqpoint{3.352233in}{5.105882in}}{\pgfqpoint{2.407767in}{1.544118in}}%
\pgfusepath{clip}%
\pgfsetbuttcap%
\pgfsetroundjoin%
\pgfsetlinewidth{0.501875pt}%
\definecolor{currentstroke}{rgb}{0.278791,0.062145,0.386592}%
\pgfsetstrokecolor{currentstroke}%
\pgfsetdash{}{0pt}%
\pgfpathmoveto{\pgfqpoint{4.827017in}{6.329641in}}%
\pgfpathlineto{\pgfqpoint{4.775706in}{6.321394in}}%
\pgfusepath{stroke}%
\end{pgfscope}%
\begin{pgfscope}%
\pgfpathrectangle{\pgfqpoint{3.352233in}{5.105882in}}{\pgfqpoint{2.407767in}{1.544118in}}%
\pgfusepath{clip}%
\pgfsetbuttcap%
\pgfsetroundjoin%
\pgfsetlinewidth{0.501875pt}%
\definecolor{currentstroke}{rgb}{0.278791,0.062145,0.386592}%
\pgfsetstrokecolor{currentstroke}%
\pgfsetdash{}{0pt}%
\pgfpathmoveto{\pgfqpoint{4.775706in}{6.321394in}}%
\pgfpathlineto{\pgfqpoint{4.725239in}{6.311233in}}%
\pgfusepath{stroke}%
\end{pgfscope}%
\begin{pgfscope}%
\pgfpathrectangle{\pgfqpoint{3.352233in}{5.105882in}}{\pgfqpoint{2.407767in}{1.544118in}}%
\pgfusepath{clip}%
\pgfsetbuttcap%
\pgfsetroundjoin%
\pgfsetlinewidth{0.501875pt}%
\definecolor{currentstroke}{rgb}{0.279566,0.067836,0.391917}%
\pgfsetstrokecolor{currentstroke}%
\pgfsetdash{}{0pt}%
\pgfpathmoveto{\pgfqpoint{4.725239in}{6.311233in}}%
\pgfpathlineto{\pgfqpoint{4.675271in}{6.300126in}}%
\pgfusepath{stroke}%
\end{pgfscope}%
\begin{pgfscope}%
\pgfpathrectangle{\pgfqpoint{3.352233in}{5.105882in}}{\pgfqpoint{2.407767in}{1.544118in}}%
\pgfusepath{clip}%
\pgfsetbuttcap%
\pgfsetroundjoin%
\pgfsetlinewidth{0.501875pt}%
\definecolor{currentstroke}{rgb}{0.277941,0.056324,0.381191}%
\pgfsetstrokecolor{currentstroke}%
\pgfsetdash{}{0pt}%
\pgfpathmoveto{\pgfqpoint{4.675271in}{6.300126in}}%
\pgfpathlineto{\pgfqpoint{4.625112in}{6.289426in}}%
\pgfusepath{stroke}%
\end{pgfscope}%
\begin{pgfscope}%
\pgfpathrectangle{\pgfqpoint{3.352233in}{5.105882in}}{\pgfqpoint{2.407767in}{1.544118in}}%
\pgfusepath{clip}%
\pgfsetbuttcap%
\pgfsetroundjoin%
\pgfsetlinewidth{0.501875pt}%
\definecolor{currentstroke}{rgb}{0.281446,0.084320,0.407414}%
\pgfsetstrokecolor{currentstroke}%
\pgfsetdash{}{0pt}%
\pgfpathmoveto{\pgfqpoint{4.625112in}{6.289426in}}%
\pgfpathlineto{\pgfqpoint{4.625112in}{6.289426in}}%
\pgfusepath{stroke}%
\end{pgfscope}%
\begin{pgfscope}%
\pgfpathrectangle{\pgfqpoint{3.352233in}{5.105882in}}{\pgfqpoint{2.407767in}{1.544118in}}%
\pgfusepath{clip}%
\pgfsetbuttcap%
\pgfsetroundjoin%
\pgfsetlinewidth{0.501875pt}%
\definecolor{currentstroke}{rgb}{0.281446,0.084320,0.407414}%
\pgfsetstrokecolor{currentstroke}%
\pgfsetdash{}{0pt}%
\pgfpathmoveto{\pgfqpoint{4.625112in}{6.289426in}}%
\pgfpathlineto{\pgfqpoint{4.590582in}{6.279403in}}%
\pgfusepath{stroke}%
\end{pgfscope}%
\begin{pgfscope}%
\pgfpathrectangle{\pgfqpoint{3.352233in}{5.105882in}}{\pgfqpoint{2.407767in}{1.544118in}}%
\pgfusepath{clip}%
\pgfsetbuttcap%
\pgfsetroundjoin%
\pgfsetlinewidth{0.501875pt}%
\definecolor{currentstroke}{rgb}{0.278791,0.062145,0.386592}%
\pgfsetstrokecolor{currentstroke}%
\pgfsetdash{}{0pt}%
\pgfpathmoveto{\pgfqpoint{4.590582in}{6.279403in}}%
\pgfpathlineto{\pgfqpoint{4.561497in}{6.264191in}}%
\pgfusepath{stroke}%
\end{pgfscope}%
\begin{pgfscope}%
\pgfpathrectangle{\pgfqpoint{3.352233in}{5.105882in}}{\pgfqpoint{2.407767in}{1.544118in}}%
\pgfusepath{clip}%
\pgfsetbuttcap%
\pgfsetroundjoin%
\pgfsetlinewidth{0.501875pt}%
\definecolor{currentstroke}{rgb}{0.279566,0.067836,0.391917}%
\pgfsetstrokecolor{currentstroke}%
\pgfsetdash{}{0pt}%
\pgfpathmoveto{\pgfqpoint{4.561497in}{6.264191in}}%
\pgfpathlineto{\pgfqpoint{4.531460in}{6.245013in}}%
\pgfusepath{stroke}%
\end{pgfscope}%
\begin{pgfscope}%
\pgfpathrectangle{\pgfqpoint{3.352233in}{5.105882in}}{\pgfqpoint{2.407767in}{1.544118in}}%
\pgfusepath{clip}%
\pgfsetbuttcap%
\pgfsetroundjoin%
\pgfsetlinewidth{0.501875pt}%
\definecolor{currentstroke}{rgb}{0.280894,0.078907,0.402329}%
\pgfsetstrokecolor{currentstroke}%
\pgfsetdash{}{0pt}%
\pgfpathmoveto{\pgfqpoint{4.531460in}{6.245013in}}%
\pgfpathlineto{\pgfqpoint{4.497732in}{6.219591in}}%
\pgfusepath{stroke}%
\end{pgfscope}%
\begin{pgfscope}%
\pgfpathrectangle{\pgfqpoint{3.352233in}{5.105882in}}{\pgfqpoint{2.407767in}{1.544118in}}%
\pgfusepath{clip}%
\pgfsetbuttcap%
\pgfsetroundjoin%
\pgfsetlinewidth{0.501875pt}%
\definecolor{currentstroke}{rgb}{0.282327,0.094955,0.417331}%
\pgfsetstrokecolor{currentstroke}%
\pgfsetdash{}{0pt}%
\pgfpathmoveto{\pgfqpoint{4.497732in}{6.219591in}}%
\pgfpathlineto{\pgfqpoint{4.467903in}{6.192477in}}%
\pgfusepath{stroke}%
\end{pgfscope}%
\begin{pgfscope}%
\pgfpathrectangle{\pgfqpoint{3.352233in}{5.105882in}}{\pgfqpoint{2.407767in}{1.544118in}}%
\pgfusepath{clip}%
\pgfsetbuttcap%
\pgfsetroundjoin%
\pgfsetlinewidth{0.501875pt}%
\definecolor{currentstroke}{rgb}{0.283187,0.125848,0.444960}%
\pgfsetstrokecolor{currentstroke}%
\pgfsetdash{}{0pt}%
\pgfpathmoveto{\pgfqpoint{4.467903in}{6.192477in}}%
\pgfpathlineto{\pgfqpoint{4.443753in}{6.162581in}}%
\pgfusepath{stroke}%
\end{pgfscope}%
\begin{pgfscope}%
\pgfpathrectangle{\pgfqpoint{3.352233in}{5.105882in}}{\pgfqpoint{2.407767in}{1.544118in}}%
\pgfusepath{clip}%
\pgfsetbuttcap%
\pgfsetroundjoin%
\pgfsetlinewidth{0.501875pt}%
\definecolor{currentstroke}{rgb}{0.280868,0.160771,0.472899}%
\pgfsetstrokecolor{currentstroke}%
\pgfsetdash{}{0pt}%
\pgfpathmoveto{\pgfqpoint{4.443753in}{6.162581in}}%
\pgfpathlineto{\pgfqpoint{4.421431in}{6.131990in}}%
\pgfusepath{stroke}%
\end{pgfscope}%
\begin{pgfscope}%
\pgfpathrectangle{\pgfqpoint{3.352233in}{5.105882in}}{\pgfqpoint{2.407767in}{1.544118in}}%
\pgfusepath{clip}%
\pgfsetbuttcap%
\pgfsetroundjoin%
\pgfsetlinewidth{0.501875pt}%
\definecolor{currentstroke}{rgb}{0.282884,0.135920,0.453427}%
\pgfsetstrokecolor{currentstroke}%
\pgfsetdash{}{0pt}%
\pgfpathmoveto{\pgfqpoint{4.421431in}{6.131990in}}%
\pgfpathlineto{\pgfqpoint{4.397684in}{6.101712in}}%
\pgfusepath{stroke}%
\end{pgfscope}%
\begin{pgfscope}%
\pgfpathrectangle{\pgfqpoint{3.352233in}{5.105882in}}{\pgfqpoint{2.407767in}{1.544118in}}%
\pgfusepath{clip}%
\pgfsetbuttcap%
\pgfsetroundjoin%
\pgfsetlinewidth{0.501875pt}%
\definecolor{currentstroke}{rgb}{0.273006,0.204520,0.501721}%
\pgfsetstrokecolor{currentstroke}%
\pgfsetdash{}{0pt}%
\pgfpathmoveto{\pgfqpoint{4.397684in}{6.101712in}}%
\pgfpathlineto{\pgfqpoint{4.368018in}{6.073841in}}%
\pgfusepath{stroke}%
\end{pgfscope}%
\begin{pgfscope}%
\pgfpathrectangle{\pgfqpoint{3.352233in}{5.105882in}}{\pgfqpoint{2.407767in}{1.544118in}}%
\pgfusepath{clip}%
\pgfsetbuttcap%
\pgfsetroundjoin%
\pgfsetlinewidth{0.501875pt}%
\definecolor{currentstroke}{rgb}{0.277941,0.056324,0.381191}%
\pgfsetstrokecolor{currentstroke}%
\pgfsetdash{}{0pt}%
\pgfpathmoveto{\pgfqpoint{4.714186in}{6.340481in}}%
\pgfpathlineto{\pgfqpoint{4.664477in}{6.329641in}}%
\pgfusepath{stroke}%
\end{pgfscope}%
\begin{pgfscope}%
\pgfpathrectangle{\pgfqpoint{3.352233in}{5.105882in}}{\pgfqpoint{2.407767in}{1.544118in}}%
\pgfusepath{clip}%
\pgfsetbuttcap%
\pgfsetroundjoin%
\pgfsetlinewidth{0.501875pt}%
\definecolor{currentstroke}{rgb}{0.278791,0.062145,0.386592}%
\pgfsetstrokecolor{currentstroke}%
\pgfsetdash{}{0pt}%
\pgfpathmoveto{\pgfqpoint{4.664477in}{6.329641in}}%
\pgfpathlineto{\pgfqpoint{4.664477in}{6.329641in}}%
\pgfusepath{stroke}%
\end{pgfscope}%
\begin{pgfscope}%
\pgfpathrectangle{\pgfqpoint{3.352233in}{5.105882in}}{\pgfqpoint{2.407767in}{1.544118in}}%
\pgfusepath{clip}%
\pgfsetbuttcap%
\pgfsetroundjoin%
\pgfsetlinewidth{0.501875pt}%
\definecolor{currentstroke}{rgb}{0.278791,0.062145,0.386592}%
\pgfsetstrokecolor{currentstroke}%
\pgfsetdash{}{0pt}%
\pgfpathmoveto{\pgfqpoint{4.664477in}{6.329641in}}%
\pgfpathlineto{\pgfqpoint{4.625898in}{6.316598in}}%
\pgfusepath{stroke}%
\end{pgfscope}%
\begin{pgfscope}%
\pgfpathrectangle{\pgfqpoint{3.352233in}{5.105882in}}{\pgfqpoint{2.407767in}{1.544118in}}%
\pgfusepath{clip}%
\pgfsetbuttcap%
\pgfsetroundjoin%
\pgfsetlinewidth{0.501875pt}%
\definecolor{currentstroke}{rgb}{0.273809,0.031497,0.358853}%
\pgfsetstrokecolor{currentstroke}%
\pgfsetdash{}{0pt}%
\pgfpathmoveto{\pgfqpoint{4.625898in}{6.316598in}}%
\pgfpathlineto{\pgfqpoint{4.581964in}{6.299055in}}%
\pgfusepath{stroke}%
\end{pgfscope}%
\begin{pgfscope}%
\pgfpathrectangle{\pgfqpoint{3.352233in}{5.105882in}}{\pgfqpoint{2.407767in}{1.544118in}}%
\pgfusepath{clip}%
\pgfsetbuttcap%
\pgfsetroundjoin%
\pgfsetlinewidth{0.501875pt}%
\definecolor{currentstroke}{rgb}{0.278791,0.062145,0.386592}%
\pgfsetstrokecolor{currentstroke}%
\pgfsetdash{}{0pt}%
\pgfpathmoveto{\pgfqpoint{4.581964in}{6.299055in}}%
\pgfpathlineto{\pgfqpoint{4.537529in}{6.281813in}}%
\pgfusepath{stroke}%
\end{pgfscope}%
\begin{pgfscope}%
\pgfpathrectangle{\pgfqpoint{3.352233in}{5.105882in}}{\pgfqpoint{2.407767in}{1.544118in}}%
\pgfusepath{clip}%
\pgfsetbuttcap%
\pgfsetroundjoin%
\pgfsetlinewidth{0.501875pt}%
\definecolor{currentstroke}{rgb}{0.277941,0.056324,0.381191}%
\pgfsetstrokecolor{currentstroke}%
\pgfsetdash{}{0pt}%
\pgfpathmoveto{\pgfqpoint{4.537529in}{6.281813in}}%
\pgfpathlineto{\pgfqpoint{4.537529in}{6.281813in}}%
\pgfusepath{stroke}%
\end{pgfscope}%
\begin{pgfscope}%
\pgfpathrectangle{\pgfqpoint{3.352233in}{5.105882in}}{\pgfqpoint{2.407767in}{1.544118in}}%
\pgfusepath{clip}%
\pgfsetbuttcap%
\pgfsetroundjoin%
\pgfsetlinewidth{0.501875pt}%
\definecolor{currentstroke}{rgb}{0.277941,0.056324,0.381191}%
\pgfsetstrokecolor{currentstroke}%
\pgfsetdash{}{0pt}%
\pgfpathmoveto{\pgfqpoint{4.537529in}{6.281813in}}%
\pgfpathlineto{\pgfqpoint{4.537529in}{6.281813in}}%
\pgfusepath{stroke}%
\end{pgfscope}%
\begin{pgfscope}%
\pgfpathrectangle{\pgfqpoint{3.352233in}{5.105882in}}{\pgfqpoint{2.407767in}{1.544118in}}%
\pgfusepath{clip}%
\pgfsetbuttcap%
\pgfsetroundjoin%
\pgfsetlinewidth{0.501875pt}%
\definecolor{currentstroke}{rgb}{0.277941,0.056324,0.381191}%
\pgfsetstrokecolor{currentstroke}%
\pgfsetdash{}{0pt}%
\pgfpathmoveto{\pgfqpoint{4.537529in}{6.281813in}}%
\pgfpathlineto{\pgfqpoint{4.524226in}{6.272514in}}%
\pgfusepath{stroke}%
\end{pgfscope}%
\begin{pgfscope}%
\pgfpathrectangle{\pgfqpoint{3.352233in}{5.105882in}}{\pgfqpoint{2.407767in}{1.544118in}}%
\pgfusepath{clip}%
\pgfsetbuttcap%
\pgfsetroundjoin%
\pgfsetlinewidth{0.501875pt}%
\definecolor{currentstroke}{rgb}{0.279566,0.067836,0.391917}%
\pgfsetstrokecolor{currentstroke}%
\pgfsetdash{}{0pt}%
\pgfpathmoveto{\pgfqpoint{4.524226in}{6.272514in}}%
\pgfpathlineto{\pgfqpoint{4.512750in}{6.262665in}}%
\pgfusepath{stroke}%
\end{pgfscope}%
\begin{pgfscope}%
\pgfpathrectangle{\pgfqpoint{3.352233in}{5.105882in}}{\pgfqpoint{2.407767in}{1.544118in}}%
\pgfusepath{clip}%
\pgfsetbuttcap%
\pgfsetroundjoin%
\pgfsetlinewidth{0.501875pt}%
\definecolor{currentstroke}{rgb}{0.280267,0.073417,0.397163}%
\pgfsetstrokecolor{currentstroke}%
\pgfsetdash{}{0pt}%
\pgfpathmoveto{\pgfqpoint{4.512750in}{6.262665in}}%
\pgfpathlineto{\pgfqpoint{4.495984in}{6.244245in}}%
\pgfusepath{stroke}%
\end{pgfscope}%
\begin{pgfscope}%
\pgfpathrectangle{\pgfqpoint{3.352233in}{5.105882in}}{\pgfqpoint{2.407767in}{1.544118in}}%
\pgfusepath{clip}%
\pgfsetbuttcap%
\pgfsetroundjoin%
\pgfsetlinewidth{0.501875pt}%
\definecolor{currentstroke}{rgb}{0.278791,0.062145,0.386592}%
\pgfsetstrokecolor{currentstroke}%
\pgfsetdash{}{0pt}%
\pgfpathmoveto{\pgfqpoint{4.769075in}{5.450683in}}%
\pgfpathlineto{\pgfqpoint{4.718657in}{5.460988in}}%
\pgfusepath{stroke}%
\end{pgfscope}%
\begin{pgfscope}%
\pgfpathrectangle{\pgfqpoint{3.352233in}{5.105882in}}{\pgfqpoint{2.407767in}{1.544118in}}%
\pgfusepath{clip}%
\pgfsetbuttcap%
\pgfsetroundjoin%
\pgfsetlinewidth{0.501875pt}%
\definecolor{currentstroke}{rgb}{0.279566,0.067836,0.391917}%
\pgfsetstrokecolor{currentstroke}%
\pgfsetdash{}{0pt}%
\pgfpathmoveto{\pgfqpoint{4.718657in}{5.460988in}}%
\pgfpathlineto{\pgfqpoint{4.669453in}{5.473125in}}%
\pgfusepath{stroke}%
\end{pgfscope}%
\begin{pgfscope}%
\pgfpathrectangle{\pgfqpoint{3.352233in}{5.105882in}}{\pgfqpoint{2.407767in}{1.544118in}}%
\pgfusepath{clip}%
\pgfsetbuttcap%
\pgfsetroundjoin%
\pgfsetlinewidth{0.501875pt}%
\definecolor{currentstroke}{rgb}{0.278791,0.062145,0.386592}%
\pgfsetstrokecolor{currentstroke}%
\pgfsetdash{}{0pt}%
\pgfpathmoveto{\pgfqpoint{4.669453in}{5.473125in}}%
\pgfpathlineto{\pgfqpoint{4.622966in}{5.489203in}}%
\pgfusepath{stroke}%
\end{pgfscope}%
\begin{pgfscope}%
\pgfpathrectangle{\pgfqpoint{3.352233in}{5.105882in}}{\pgfqpoint{2.407767in}{1.544118in}}%
\pgfusepath{clip}%
\pgfsetbuttcap%
\pgfsetroundjoin%
\pgfsetlinewidth{0.501875pt}%
\definecolor{currentstroke}{rgb}{0.282327,0.094955,0.417331}%
\pgfsetstrokecolor{currentstroke}%
\pgfsetdash{}{0pt}%
\pgfpathmoveto{\pgfqpoint{4.622966in}{5.489203in}}%
\pgfpathlineto{\pgfqpoint{4.575830in}{5.504302in}}%
\pgfusepath{stroke}%
\end{pgfscope}%
\begin{pgfscope}%
\pgfpathrectangle{\pgfqpoint{3.352233in}{5.105882in}}{\pgfqpoint{2.407767in}{1.544118in}}%
\pgfusepath{clip}%
\pgfsetbuttcap%
\pgfsetroundjoin%
\pgfsetlinewidth{0.501875pt}%
\definecolor{currentstroke}{rgb}{0.283197,0.115680,0.436115}%
\pgfsetstrokecolor{currentstroke}%
\pgfsetdash{}{0pt}%
\pgfpathmoveto{\pgfqpoint{4.575830in}{5.504302in}}%
\pgfpathlineto{\pgfqpoint{4.575830in}{5.504302in}}%
\pgfusepath{stroke}%
\end{pgfscope}%
\begin{pgfscope}%
\pgfpathrectangle{\pgfqpoint{3.352233in}{5.105882in}}{\pgfqpoint{2.407767in}{1.544118in}}%
\pgfusepath{clip}%
\pgfsetbuttcap%
\pgfsetroundjoin%
\pgfsetlinewidth{0.501875pt}%
\definecolor{currentstroke}{rgb}{0.283197,0.115680,0.436115}%
\pgfsetstrokecolor{currentstroke}%
\pgfsetdash{}{0pt}%
\pgfpathmoveto{\pgfqpoint{4.575830in}{5.504302in}}%
\pgfpathlineto{\pgfqpoint{4.544075in}{5.516763in}}%
\pgfusepath{stroke}%
\end{pgfscope}%
\begin{pgfscope}%
\pgfpathrectangle{\pgfqpoint{3.352233in}{5.105882in}}{\pgfqpoint{2.407767in}{1.544118in}}%
\pgfusepath{clip}%
\pgfsetbuttcap%
\pgfsetroundjoin%
\pgfsetlinewidth{0.501875pt}%
\definecolor{currentstroke}{rgb}{0.282327,0.094955,0.417331}%
\pgfsetstrokecolor{currentstroke}%
\pgfsetdash{}{0pt}%
\pgfpathmoveto{\pgfqpoint{4.544075in}{5.516763in}}%
\pgfpathlineto{\pgfqpoint{4.517083in}{5.533724in}}%
\pgfusepath{stroke}%
\end{pgfscope}%
\begin{pgfscope}%
\pgfpathrectangle{\pgfqpoint{3.352233in}{5.105882in}}{\pgfqpoint{2.407767in}{1.544118in}}%
\pgfusepath{clip}%
\pgfsetbuttcap%
\pgfsetroundjoin%
\pgfsetlinewidth{0.501875pt}%
\definecolor{currentstroke}{rgb}{0.283197,0.115680,0.436115}%
\pgfsetstrokecolor{currentstroke}%
\pgfsetdash{}{0pt}%
\pgfpathmoveto{\pgfqpoint{4.517083in}{5.533724in}}%
\pgfpathlineto{\pgfqpoint{4.486166in}{5.559405in}}%
\pgfusepath{stroke}%
\end{pgfscope}%
\begin{pgfscope}%
\pgfpathrectangle{\pgfqpoint{3.352233in}{5.105882in}}{\pgfqpoint{2.407767in}{1.544118in}}%
\pgfusepath{clip}%
\pgfsetbuttcap%
\pgfsetroundjoin%
\pgfsetlinewidth{0.501875pt}%
\definecolor{currentstroke}{rgb}{0.283187,0.125848,0.444960}%
\pgfsetstrokecolor{currentstroke}%
\pgfsetdash{}{0pt}%
\pgfpathmoveto{\pgfqpoint{4.486166in}{5.559405in}}%
\pgfpathlineto{\pgfqpoint{4.457548in}{5.587957in}}%
\pgfusepath{stroke}%
\end{pgfscope}%
\begin{pgfscope}%
\pgfpathrectangle{\pgfqpoint{3.352233in}{5.105882in}}{\pgfqpoint{2.407767in}{1.544118in}}%
\pgfusepath{clip}%
\pgfsetbuttcap%
\pgfsetroundjoin%
\pgfsetlinewidth{0.501875pt}%
\definecolor{currentstroke}{rgb}{0.277134,0.185228,0.489898}%
\pgfsetstrokecolor{currentstroke}%
\pgfsetdash{}{0pt}%
\pgfpathmoveto{\pgfqpoint{4.457548in}{5.587957in}}%
\pgfpathlineto{\pgfqpoint{4.430084in}{5.616914in}}%
\pgfusepath{stroke}%
\end{pgfscope}%
\begin{pgfscope}%
\pgfpathrectangle{\pgfqpoint{3.352233in}{5.105882in}}{\pgfqpoint{2.407767in}{1.544118in}}%
\pgfusepath{clip}%
\pgfsetbuttcap%
\pgfsetroundjoin%
\pgfsetlinewidth{0.501875pt}%
\definecolor{currentstroke}{rgb}{0.272594,0.025563,0.353093}%
\pgfsetstrokecolor{currentstroke}%
\pgfsetdash{}{0pt}%
\pgfpathmoveto{\pgfqpoint{5.206279in}{5.495734in}}%
\pgfpathlineto{\pgfqpoint{5.153321in}{5.496582in}}%
\pgfusepath{stroke}%
\end{pgfscope}%
\begin{pgfscope}%
\pgfpathrectangle{\pgfqpoint{3.352233in}{5.105882in}}{\pgfqpoint{2.407767in}{1.544118in}}%
\pgfusepath{clip}%
\pgfsetbuttcap%
\pgfsetroundjoin%
\pgfsetlinewidth{0.501875pt}%
\definecolor{currentstroke}{rgb}{0.273809,0.031497,0.358853}%
\pgfsetstrokecolor{currentstroke}%
\pgfsetdash{}{0pt}%
\pgfpathmoveto{\pgfqpoint{5.153321in}{5.496582in}}%
\pgfpathlineto{\pgfqpoint{5.100377in}{5.497697in}}%
\pgfusepath{stroke}%
\end{pgfscope}%
\begin{pgfscope}%
\pgfpathrectangle{\pgfqpoint{3.352233in}{5.105882in}}{\pgfqpoint{2.407767in}{1.544118in}}%
\pgfusepath{clip}%
\pgfsetbuttcap%
\pgfsetroundjoin%
\pgfsetlinewidth{0.501875pt}%
\definecolor{currentstroke}{rgb}{0.276022,0.044167,0.370164}%
\pgfsetstrokecolor{currentstroke}%
\pgfsetdash{}{0pt}%
\pgfpathmoveto{\pgfqpoint{5.100377in}{5.497697in}}%
\pgfpathlineto{\pgfqpoint{5.047490in}{5.499544in}}%
\pgfusepath{stroke}%
\end{pgfscope}%
\begin{pgfscope}%
\pgfpathrectangle{\pgfqpoint{3.352233in}{5.105882in}}{\pgfqpoint{2.407767in}{1.544118in}}%
\pgfusepath{clip}%
\pgfsetbuttcap%
\pgfsetroundjoin%
\pgfsetlinewidth{0.501875pt}%
\definecolor{currentstroke}{rgb}{0.277941,0.056324,0.381191}%
\pgfsetstrokecolor{currentstroke}%
\pgfsetdash{}{0pt}%
\pgfpathmoveto{\pgfqpoint{5.047490in}{5.499544in}}%
\pgfpathlineto{\pgfqpoint{4.994730in}{5.502538in}}%
\pgfusepath{stroke}%
\end{pgfscope}%
\begin{pgfscope}%
\pgfpathrectangle{\pgfqpoint{3.352233in}{5.105882in}}{\pgfqpoint{2.407767in}{1.544118in}}%
\pgfusepath{clip}%
\pgfsetbuttcap%
\pgfsetroundjoin%
\pgfsetlinewidth{0.501875pt}%
\definecolor{currentstroke}{rgb}{0.280894,0.078907,0.402329}%
\pgfsetstrokecolor{currentstroke}%
\pgfsetdash{}{0pt}%
\pgfpathmoveto{\pgfqpoint{4.994730in}{5.502538in}}%
\pgfpathlineto{\pgfqpoint{4.942069in}{5.506224in}}%
\pgfusepath{stroke}%
\end{pgfscope}%
\begin{pgfscope}%
\pgfpathrectangle{\pgfqpoint{3.352233in}{5.105882in}}{\pgfqpoint{2.407767in}{1.544118in}}%
\pgfusepath{clip}%
\pgfsetbuttcap%
\pgfsetroundjoin%
\pgfsetlinewidth{0.501875pt}%
\definecolor{currentstroke}{rgb}{0.280267,0.073417,0.397163}%
\pgfsetstrokecolor{currentstroke}%
\pgfsetdash{}{0pt}%
\pgfpathmoveto{\pgfqpoint{4.942069in}{5.506224in}}%
\pgfpathlineto{\pgfqpoint{4.889541in}{5.510589in}}%
\pgfusepath{stroke}%
\end{pgfscope}%
\begin{pgfscope}%
\pgfpathrectangle{\pgfqpoint{3.352233in}{5.105882in}}{\pgfqpoint{2.407767in}{1.544118in}}%
\pgfusepath{clip}%
\pgfsetbuttcap%
\pgfsetroundjoin%
\pgfsetlinewidth{0.501875pt}%
\definecolor{currentstroke}{rgb}{0.283091,0.110553,0.431554}%
\pgfsetstrokecolor{currentstroke}%
\pgfsetdash{}{0pt}%
\pgfpathmoveto{\pgfqpoint{4.889541in}{5.510589in}}%
\pgfpathlineto{\pgfqpoint{4.837327in}{5.516195in}}%
\pgfusepath{stroke}%
\end{pgfscope}%
\begin{pgfscope}%
\pgfpathrectangle{\pgfqpoint{3.352233in}{5.105882in}}{\pgfqpoint{2.407767in}{1.544118in}}%
\pgfusepath{clip}%
\pgfsetbuttcap%
\pgfsetroundjoin%
\pgfsetlinewidth{0.501875pt}%
\definecolor{currentstroke}{rgb}{0.281446,0.084320,0.407414}%
\pgfsetstrokecolor{currentstroke}%
\pgfsetdash{}{0pt}%
\pgfpathmoveto{\pgfqpoint{4.837327in}{5.516195in}}%
\pgfpathlineto{\pgfqpoint{4.785543in}{5.523191in}}%
\pgfusepath{stroke}%
\end{pgfscope}%
\begin{pgfscope}%
\pgfpathrectangle{\pgfqpoint{3.352233in}{5.105882in}}{\pgfqpoint{2.407767in}{1.544118in}}%
\pgfusepath{clip}%
\pgfsetbuttcap%
\pgfsetroundjoin%
\pgfsetlinewidth{0.501875pt}%
\definecolor{currentstroke}{rgb}{0.281924,0.089666,0.412415}%
\pgfsetstrokecolor{currentstroke}%
\pgfsetdash{}{0pt}%
\pgfpathmoveto{\pgfqpoint{4.785543in}{5.523191in}}%
\pgfpathlineto{\pgfqpoint{4.734500in}{5.532047in}}%
\pgfusepath{stroke}%
\end{pgfscope}%
\begin{pgfscope}%
\pgfpathrectangle{\pgfqpoint{3.352233in}{5.105882in}}{\pgfqpoint{2.407767in}{1.544118in}}%
\pgfusepath{clip}%
\pgfsetbuttcap%
\pgfsetroundjoin%
\pgfsetlinewidth{0.501875pt}%
\definecolor{currentstroke}{rgb}{0.283229,0.120777,0.440584}%
\pgfsetstrokecolor{currentstroke}%
\pgfsetdash{}{0pt}%
\pgfpathmoveto{\pgfqpoint{4.734500in}{5.532047in}}%
\pgfpathlineto{\pgfqpoint{4.684306in}{5.542730in}}%
\pgfusepath{stroke}%
\end{pgfscope}%
\begin{pgfscope}%
\pgfpathrectangle{\pgfqpoint{3.352233in}{5.105882in}}{\pgfqpoint{2.407767in}{1.544118in}}%
\pgfusepath{clip}%
\pgfsetbuttcap%
\pgfsetroundjoin%
\pgfsetlinewidth{0.501875pt}%
\definecolor{currentstroke}{rgb}{0.282290,0.145912,0.461510}%
\pgfsetstrokecolor{currentstroke}%
\pgfsetdash{}{0pt}%
\pgfpathmoveto{\pgfqpoint{4.684306in}{5.542730in}}%
\pgfpathlineto{\pgfqpoint{4.634866in}{5.554874in}}%
\pgfusepath{stroke}%
\end{pgfscope}%
\begin{pgfscope}%
\pgfpathrectangle{\pgfqpoint{3.352233in}{5.105882in}}{\pgfqpoint{2.407767in}{1.544118in}}%
\pgfusepath{clip}%
\pgfsetbuttcap%
\pgfsetroundjoin%
\pgfsetlinewidth{0.501875pt}%
\definecolor{currentstroke}{rgb}{0.281887,0.150881,0.465405}%
\pgfsetstrokecolor{currentstroke}%
\pgfsetdash{}{0pt}%
\pgfpathmoveto{\pgfqpoint{4.634866in}{5.554874in}}%
\pgfpathlineto{\pgfqpoint{4.587011in}{5.569280in}}%
\pgfusepath{stroke}%
\end{pgfscope}%
\begin{pgfscope}%
\pgfpathrectangle{\pgfqpoint{3.352233in}{5.105882in}}{\pgfqpoint{2.407767in}{1.544118in}}%
\pgfusepath{clip}%
\pgfsetbuttcap%
\pgfsetroundjoin%
\pgfsetlinewidth{0.501875pt}%
\definecolor{currentstroke}{rgb}{0.282623,0.140926,0.457517}%
\pgfsetstrokecolor{currentstroke}%
\pgfsetdash{}{0pt}%
\pgfpathmoveto{\pgfqpoint{4.587011in}{5.569280in}}%
\pgfpathlineto{\pgfqpoint{4.541937in}{5.586905in}}%
\pgfusepath{stroke}%
\end{pgfscope}%
\begin{pgfscope}%
\pgfpathrectangle{\pgfqpoint{3.352233in}{5.105882in}}{\pgfqpoint{2.407767in}{1.544118in}}%
\pgfusepath{clip}%
\pgfsetbuttcap%
\pgfsetroundjoin%
\pgfsetlinewidth{0.501875pt}%
\definecolor{currentstroke}{rgb}{0.281887,0.150881,0.465405}%
\pgfsetstrokecolor{currentstroke}%
\pgfsetdash{}{0pt}%
\pgfpathmoveto{\pgfqpoint{4.541937in}{5.586905in}}%
\pgfpathlineto{\pgfqpoint{4.500649in}{5.607987in}}%
\pgfusepath{stroke}%
\end{pgfscope}%
\begin{pgfscope}%
\pgfpathrectangle{\pgfqpoint{3.352233in}{5.105882in}}{\pgfqpoint{2.407767in}{1.544118in}}%
\pgfusepath{clip}%
\pgfsetbuttcap%
\pgfsetroundjoin%
\pgfsetlinewidth{0.501875pt}%
\definecolor{currentstroke}{rgb}{0.281412,0.155834,0.469201}%
\pgfsetstrokecolor{currentstroke}%
\pgfsetdash{}{0pt}%
\pgfpathmoveto{\pgfqpoint{4.500649in}{5.607987in}}%
\pgfpathlineto{\pgfqpoint{4.463442in}{5.632039in}}%
\pgfusepath{stroke}%
\end{pgfscope}%
\begin{pgfscope}%
\pgfpathrectangle{\pgfqpoint{3.352233in}{5.105882in}}{\pgfqpoint{2.407767in}{1.544118in}}%
\pgfusepath{clip}%
\pgfsetbuttcap%
\pgfsetroundjoin%
\pgfsetlinewidth{0.501875pt}%
\definecolor{currentstroke}{rgb}{0.282623,0.140926,0.457517}%
\pgfsetstrokecolor{currentstroke}%
\pgfsetdash{}{0pt}%
\pgfpathmoveto{\pgfqpoint{4.463442in}{5.632039in}}%
\pgfpathlineto{\pgfqpoint{4.428297in}{5.657371in}}%
\pgfusepath{stroke}%
\end{pgfscope}%
\begin{pgfscope}%
\pgfpathrectangle{\pgfqpoint{3.352233in}{5.105882in}}{\pgfqpoint{2.407767in}{1.544118in}}%
\pgfusepath{clip}%
\pgfsetbuttcap%
\pgfsetroundjoin%
\pgfsetlinewidth{0.501875pt}%
\definecolor{currentstroke}{rgb}{0.271305,0.019942,0.347269}%
\pgfsetstrokecolor{currentstroke}%
\pgfsetdash{}{0pt}%
\pgfpathmoveto{\pgfqpoint{5.206279in}{5.530480in}}%
\pgfpathlineto{\pgfqpoint{5.153362in}{5.531506in}}%
\pgfusepath{stroke}%
\end{pgfscope}%
\begin{pgfscope}%
\pgfpathrectangle{\pgfqpoint{3.352233in}{5.105882in}}{\pgfqpoint{2.407767in}{1.544118in}}%
\pgfusepath{clip}%
\pgfsetbuttcap%
\pgfsetroundjoin%
\pgfsetlinewidth{0.501875pt}%
\definecolor{currentstroke}{rgb}{0.276022,0.044167,0.370164}%
\pgfsetstrokecolor{currentstroke}%
\pgfsetdash{}{0pt}%
\pgfpathmoveto{\pgfqpoint{5.153362in}{5.531506in}}%
\pgfpathlineto{\pgfqpoint{5.100416in}{5.532262in}}%
\pgfusepath{stroke}%
\end{pgfscope}%
\begin{pgfscope}%
\pgfpathrectangle{\pgfqpoint{3.352233in}{5.105882in}}{\pgfqpoint{2.407767in}{1.544118in}}%
\pgfusepath{clip}%
\pgfsetbuttcap%
\pgfsetroundjoin%
\pgfsetlinewidth{0.501875pt}%
\definecolor{currentstroke}{rgb}{0.277941,0.056324,0.381191}%
\pgfsetstrokecolor{currentstroke}%
\pgfsetdash{}{0pt}%
\pgfpathmoveto{\pgfqpoint{5.100416in}{5.532262in}}%
\pgfpathlineto{\pgfqpoint{5.047513in}{5.533945in}}%
\pgfusepath{stroke}%
\end{pgfscope}%
\begin{pgfscope}%
\pgfpathrectangle{\pgfqpoint{3.352233in}{5.105882in}}{\pgfqpoint{2.407767in}{1.544118in}}%
\pgfusepath{clip}%
\pgfsetbuttcap%
\pgfsetroundjoin%
\pgfsetlinewidth{0.501875pt}%
\definecolor{currentstroke}{rgb}{0.281446,0.084320,0.407414}%
\pgfsetstrokecolor{currentstroke}%
\pgfsetdash{}{0pt}%
\pgfpathmoveto{\pgfqpoint{5.047513in}{5.533945in}}%
\pgfpathlineto{\pgfqpoint{4.994705in}{5.536555in}}%
\pgfusepath{stroke}%
\end{pgfscope}%
\begin{pgfscope}%
\pgfpathrectangle{\pgfqpoint{3.352233in}{5.105882in}}{\pgfqpoint{2.407767in}{1.544118in}}%
\pgfusepath{clip}%
\pgfsetbuttcap%
\pgfsetroundjoin%
\pgfsetlinewidth{0.501875pt}%
\definecolor{currentstroke}{rgb}{0.281446,0.084320,0.407414}%
\pgfsetstrokecolor{currentstroke}%
\pgfsetdash{}{0pt}%
\pgfpathmoveto{\pgfqpoint{4.994705in}{5.536555in}}%
\pgfpathlineto{\pgfqpoint{4.941997in}{5.539951in}}%
\pgfusepath{stroke}%
\end{pgfscope}%
\begin{pgfscope}%
\pgfpathrectangle{\pgfqpoint{3.352233in}{5.105882in}}{\pgfqpoint{2.407767in}{1.544118in}}%
\pgfusepath{clip}%
\pgfsetbuttcap%
\pgfsetroundjoin%
\pgfsetlinewidth{0.501875pt}%
\definecolor{currentstroke}{rgb}{0.283091,0.110553,0.431554}%
\pgfsetstrokecolor{currentstroke}%
\pgfsetdash{}{0pt}%
\pgfpathmoveto{\pgfqpoint{4.941997in}{5.539951in}}%
\pgfpathlineto{\pgfqpoint{4.889423in}{5.544098in}}%
\pgfusepath{stroke}%
\end{pgfscope}%
\begin{pgfscope}%
\pgfpathrectangle{\pgfqpoint{3.352233in}{5.105882in}}{\pgfqpoint{2.407767in}{1.544118in}}%
\pgfusepath{clip}%
\pgfsetbuttcap%
\pgfsetroundjoin%
\pgfsetlinewidth{0.501875pt}%
\definecolor{currentstroke}{rgb}{0.281924,0.089666,0.412415}%
\pgfsetstrokecolor{currentstroke}%
\pgfsetdash{}{0pt}%
\pgfpathmoveto{\pgfqpoint{4.889423in}{5.544098in}}%
\pgfpathlineto{\pgfqpoint{4.837195in}{5.549677in}}%
\pgfusepath{stroke}%
\end{pgfscope}%
\begin{pgfscope}%
\pgfpathrectangle{\pgfqpoint{3.352233in}{5.105882in}}{\pgfqpoint{2.407767in}{1.544118in}}%
\pgfusepath{clip}%
\pgfsetbuttcap%
\pgfsetroundjoin%
\pgfsetlinewidth{0.501875pt}%
\definecolor{currentstroke}{rgb}{0.282884,0.135920,0.453427}%
\pgfsetstrokecolor{currentstroke}%
\pgfsetdash{}{0pt}%
\pgfpathmoveto{\pgfqpoint{4.837195in}{5.549677in}}%
\pgfpathlineto{\pgfqpoint{4.785276in}{5.556395in}}%
\pgfusepath{stroke}%
\end{pgfscope}%
\begin{pgfscope}%
\pgfpathrectangle{\pgfqpoint{3.352233in}{5.105882in}}{\pgfqpoint{2.407767in}{1.544118in}}%
\pgfusepath{clip}%
\pgfsetbuttcap%
\pgfsetroundjoin%
\pgfsetlinewidth{0.501875pt}%
\definecolor{currentstroke}{rgb}{0.283187,0.125848,0.444960}%
\pgfsetstrokecolor{currentstroke}%
\pgfsetdash{}{0pt}%
\pgfpathmoveto{\pgfqpoint{4.785276in}{5.556395in}}%
\pgfpathlineto{\pgfqpoint{4.733821in}{5.564356in}}%
\pgfusepath{stroke}%
\end{pgfscope}%
\begin{pgfscope}%
\pgfpathrectangle{\pgfqpoint{3.352233in}{5.105882in}}{\pgfqpoint{2.407767in}{1.544118in}}%
\pgfusepath{clip}%
\pgfsetbuttcap%
\pgfsetroundjoin%
\pgfsetlinewidth{0.501875pt}%
\definecolor{currentstroke}{rgb}{0.283072,0.130895,0.449241}%
\pgfsetstrokecolor{currentstroke}%
\pgfsetdash{}{0pt}%
\pgfpathmoveto{\pgfqpoint{4.733821in}{5.564356in}}%
\pgfpathlineto{\pgfqpoint{4.683221in}{5.574340in}}%
\pgfusepath{stroke}%
\end{pgfscope}%
\begin{pgfscope}%
\pgfpathrectangle{\pgfqpoint{3.352233in}{5.105882in}}{\pgfqpoint{2.407767in}{1.544118in}}%
\pgfusepath{clip}%
\pgfsetbuttcap%
\pgfsetroundjoin%
\pgfsetlinewidth{0.501875pt}%
\definecolor{currentstroke}{rgb}{0.278826,0.175490,0.483397}%
\pgfsetstrokecolor{currentstroke}%
\pgfsetdash{}{0pt}%
\pgfpathmoveto{\pgfqpoint{4.683221in}{5.574340in}}%
\pgfpathlineto{\pgfqpoint{4.633678in}{5.586295in}}%
\pgfusepath{stroke}%
\end{pgfscope}%
\begin{pgfscope}%
\pgfpathrectangle{\pgfqpoint{3.352233in}{5.105882in}}{\pgfqpoint{2.407767in}{1.544118in}}%
\pgfusepath{clip}%
\pgfsetbuttcap%
\pgfsetroundjoin%
\pgfsetlinewidth{0.501875pt}%
\definecolor{currentstroke}{rgb}{0.281887,0.150881,0.465405}%
\pgfsetstrokecolor{currentstroke}%
\pgfsetdash{}{0pt}%
\pgfpathmoveto{\pgfqpoint{4.633678in}{5.586295in}}%
\pgfpathlineto{\pgfqpoint{4.585357in}{5.600143in}}%
\pgfusepath{stroke}%
\end{pgfscope}%
\begin{pgfscope}%
\pgfpathrectangle{\pgfqpoint{3.352233in}{5.105882in}}{\pgfqpoint{2.407767in}{1.544118in}}%
\pgfusepath{clip}%
\pgfsetbuttcap%
\pgfsetroundjoin%
\pgfsetlinewidth{0.501875pt}%
\definecolor{currentstroke}{rgb}{0.272594,0.025563,0.353093}%
\pgfsetstrokecolor{currentstroke}%
\pgfsetdash{}{0pt}%
\pgfpathmoveto{\pgfqpoint{5.206279in}{5.565226in}}%
\pgfpathlineto{\pgfqpoint{5.153316in}{5.565773in}}%
\pgfusepath{stroke}%
\end{pgfscope}%
\begin{pgfscope}%
\pgfpathrectangle{\pgfqpoint{3.352233in}{5.105882in}}{\pgfqpoint{2.407767in}{1.544118in}}%
\pgfusepath{clip}%
\pgfsetbuttcap%
\pgfsetroundjoin%
\pgfsetlinewidth{0.501875pt}%
\definecolor{currentstroke}{rgb}{0.277941,0.056324,0.381191}%
\pgfsetstrokecolor{currentstroke}%
\pgfsetdash{}{0pt}%
\pgfpathmoveto{\pgfqpoint{5.153316in}{5.565773in}}%
\pgfpathlineto{\pgfqpoint{5.100369in}{5.566827in}}%
\pgfusepath{stroke}%
\end{pgfscope}%
\begin{pgfscope}%
\pgfpathrectangle{\pgfqpoint{3.352233in}{5.105882in}}{\pgfqpoint{2.407767in}{1.544118in}}%
\pgfusepath{clip}%
\pgfsetbuttcap%
\pgfsetroundjoin%
\pgfsetlinewidth{0.501875pt}%
\definecolor{currentstroke}{rgb}{0.280894,0.078907,0.402329}%
\pgfsetstrokecolor{currentstroke}%
\pgfsetdash{}{0pt}%
\pgfpathmoveto{\pgfqpoint{5.100369in}{5.566827in}}%
\pgfpathlineto{\pgfqpoint{5.047477in}{5.568678in}}%
\pgfusepath{stroke}%
\end{pgfscope}%
\begin{pgfscope}%
\pgfpathrectangle{\pgfqpoint{3.352233in}{5.105882in}}{\pgfqpoint{2.407767in}{1.544118in}}%
\pgfusepath{clip}%
\pgfsetbuttcap%
\pgfsetroundjoin%
\pgfsetlinewidth{0.501875pt}%
\definecolor{currentstroke}{rgb}{0.282327,0.094955,0.417331}%
\pgfsetstrokecolor{currentstroke}%
\pgfsetdash{}{0pt}%
\pgfpathmoveto{\pgfqpoint{5.047477in}{5.568678in}}%
\pgfpathlineto{\pgfqpoint{4.994657in}{5.571255in}}%
\pgfusepath{stroke}%
\end{pgfscope}%
\begin{pgfscope}%
\pgfpathrectangle{\pgfqpoint{3.352233in}{5.105882in}}{\pgfqpoint{2.407767in}{1.544118in}}%
\pgfusepath{clip}%
\pgfsetbuttcap%
\pgfsetroundjoin%
\pgfsetlinewidth{0.501875pt}%
\definecolor{currentstroke}{rgb}{0.283187,0.125848,0.444960}%
\pgfsetstrokecolor{currentstroke}%
\pgfsetdash{}{0pt}%
\pgfpathmoveto{\pgfqpoint{4.994657in}{5.571255in}}%
\pgfpathlineto{\pgfqpoint{4.941903in}{5.574356in}}%
\pgfusepath{stroke}%
\end{pgfscope}%
\begin{pgfscope}%
\pgfpathrectangle{\pgfqpoint{3.352233in}{5.105882in}}{\pgfqpoint{2.407767in}{1.544118in}}%
\pgfusepath{clip}%
\pgfsetbuttcap%
\pgfsetroundjoin%
\pgfsetlinewidth{0.501875pt}%
\definecolor{currentstroke}{rgb}{0.282623,0.140926,0.457517}%
\pgfsetstrokecolor{currentstroke}%
\pgfsetdash{}{0pt}%
\pgfpathmoveto{\pgfqpoint{4.941903in}{5.574356in}}%
\pgfpathlineto{\pgfqpoint{4.889276in}{5.578212in}}%
\pgfusepath{stroke}%
\end{pgfscope}%
\begin{pgfscope}%
\pgfpathrectangle{\pgfqpoint{3.352233in}{5.105882in}}{\pgfqpoint{2.407767in}{1.544118in}}%
\pgfusepath{clip}%
\pgfsetbuttcap%
\pgfsetroundjoin%
\pgfsetlinewidth{0.501875pt}%
\definecolor{currentstroke}{rgb}{0.283187,0.125848,0.444960}%
\pgfsetstrokecolor{currentstroke}%
\pgfsetdash{}{0pt}%
\pgfpathmoveto{\pgfqpoint{4.889276in}{5.578212in}}%
\pgfpathlineto{\pgfqpoint{4.836865in}{5.583116in}}%
\pgfusepath{stroke}%
\end{pgfscope}%
\begin{pgfscope}%
\pgfpathrectangle{\pgfqpoint{3.352233in}{5.105882in}}{\pgfqpoint{2.407767in}{1.544118in}}%
\pgfusepath{clip}%
\pgfsetbuttcap%
\pgfsetroundjoin%
\pgfsetlinewidth{0.501875pt}%
\definecolor{currentstroke}{rgb}{0.282290,0.145912,0.461510}%
\pgfsetstrokecolor{currentstroke}%
\pgfsetdash{}{0pt}%
\pgfpathmoveto{\pgfqpoint{4.836865in}{5.583116in}}%
\pgfpathlineto{\pgfqpoint{4.784766in}{5.589215in}}%
\pgfusepath{stroke}%
\end{pgfscope}%
\begin{pgfscope}%
\pgfpathrectangle{\pgfqpoint{3.352233in}{5.105882in}}{\pgfqpoint{2.407767in}{1.544118in}}%
\pgfusepath{clip}%
\pgfsetbuttcap%
\pgfsetroundjoin%
\pgfsetlinewidth{0.501875pt}%
\definecolor{currentstroke}{rgb}{0.282623,0.140926,0.457517}%
\pgfsetstrokecolor{currentstroke}%
\pgfsetdash{}{0pt}%
\pgfpathmoveto{\pgfqpoint{4.784766in}{5.589215in}}%
\pgfpathlineto{\pgfqpoint{4.733033in}{5.596478in}}%
\pgfusepath{stroke}%
\end{pgfscope}%
\begin{pgfscope}%
\pgfpathrectangle{\pgfqpoint{3.352233in}{5.105882in}}{\pgfqpoint{2.407767in}{1.544118in}}%
\pgfusepath{clip}%
\pgfsetbuttcap%
\pgfsetroundjoin%
\pgfsetlinewidth{0.501875pt}%
\definecolor{currentstroke}{rgb}{0.281887,0.150881,0.465405}%
\pgfsetstrokecolor{currentstroke}%
\pgfsetdash{}{0pt}%
\pgfpathmoveto{\pgfqpoint{4.733033in}{5.596478in}}%
\pgfpathlineto{\pgfqpoint{4.681720in}{5.604868in}}%
\pgfusepath{stroke}%
\end{pgfscope}%
\begin{pgfscope}%
\pgfpathrectangle{\pgfqpoint{3.352233in}{5.105882in}}{\pgfqpoint{2.407767in}{1.544118in}}%
\pgfusepath{clip}%
\pgfsetbuttcap%
\pgfsetroundjoin%
\pgfsetlinewidth{0.501875pt}%
\definecolor{currentstroke}{rgb}{0.274952,0.037752,0.364543}%
\pgfsetstrokecolor{currentstroke}%
\pgfsetdash{}{0pt}%
\pgfpathmoveto{\pgfqpoint{5.206279in}{5.599972in}}%
\pgfpathlineto{\pgfqpoint{5.153312in}{5.600519in}}%
\pgfusepath{stroke}%
\end{pgfscope}%
\begin{pgfscope}%
\pgfpathrectangle{\pgfqpoint{3.352233in}{5.105882in}}{\pgfqpoint{2.407767in}{1.544118in}}%
\pgfusepath{clip}%
\pgfsetbuttcap%
\pgfsetroundjoin%
\pgfsetlinewidth{0.501875pt}%
\definecolor{currentstroke}{rgb}{0.277941,0.056324,0.381191}%
\pgfsetstrokecolor{currentstroke}%
\pgfsetdash{}{0pt}%
\pgfpathmoveto{\pgfqpoint{5.153312in}{5.600519in}}%
\pgfpathlineto{\pgfqpoint{5.100363in}{5.601578in}}%
\pgfusepath{stroke}%
\end{pgfscope}%
\begin{pgfscope}%
\pgfpathrectangle{\pgfqpoint{3.352233in}{5.105882in}}{\pgfqpoint{2.407767in}{1.544118in}}%
\pgfusepath{clip}%
\pgfsetbuttcap%
\pgfsetroundjoin%
\pgfsetlinewidth{0.501875pt}%
\definecolor{currentstroke}{rgb}{0.281446,0.084320,0.407414}%
\pgfsetstrokecolor{currentstroke}%
\pgfsetdash{}{0pt}%
\pgfpathmoveto{\pgfqpoint{5.100363in}{5.601578in}}%
\pgfpathlineto{\pgfqpoint{5.047433in}{5.602975in}}%
\pgfusepath{stroke}%
\end{pgfscope}%
\begin{pgfscope}%
\pgfpathrectangle{\pgfqpoint{3.352233in}{5.105882in}}{\pgfqpoint{2.407767in}{1.544118in}}%
\pgfusepath{clip}%
\pgfsetbuttcap%
\pgfsetroundjoin%
\pgfsetlinewidth{0.501875pt}%
\definecolor{currentstroke}{rgb}{0.283197,0.115680,0.436115}%
\pgfsetstrokecolor{currentstroke}%
\pgfsetdash{}{0pt}%
\pgfpathmoveto{\pgfqpoint{5.047433in}{5.602975in}}%
\pgfpathlineto{\pgfqpoint{4.994576in}{5.605154in}}%
\pgfusepath{stroke}%
\end{pgfscope}%
\begin{pgfscope}%
\pgfpathrectangle{\pgfqpoint{3.352233in}{5.105882in}}{\pgfqpoint{2.407767in}{1.544118in}}%
\pgfusepath{clip}%
\pgfsetbuttcap%
\pgfsetroundjoin%
\pgfsetlinewidth{0.501875pt}%
\definecolor{currentstroke}{rgb}{0.283187,0.125848,0.444960}%
\pgfsetstrokecolor{currentstroke}%
\pgfsetdash{}{0pt}%
\pgfpathmoveto{\pgfqpoint{4.994576in}{5.605154in}}%
\pgfpathlineto{\pgfqpoint{4.941794in}{5.608055in}}%
\pgfusepath{stroke}%
\end{pgfscope}%
\begin{pgfscope}%
\pgfpathrectangle{\pgfqpoint{3.352233in}{5.105882in}}{\pgfqpoint{2.407767in}{1.544118in}}%
\pgfusepath{clip}%
\pgfsetbuttcap%
\pgfsetroundjoin%
\pgfsetlinewidth{0.501875pt}%
\definecolor{currentstroke}{rgb}{0.280255,0.165693,0.476498}%
\pgfsetstrokecolor{currentstroke}%
\pgfsetdash{}{0pt}%
\pgfpathmoveto{\pgfqpoint{4.941794in}{5.608055in}}%
\pgfpathlineto{\pgfqpoint{4.889116in}{5.611602in}}%
\pgfusepath{stroke}%
\end{pgfscope}%
\begin{pgfscope}%
\pgfpathrectangle{\pgfqpoint{3.352233in}{5.105882in}}{\pgfqpoint{2.407767in}{1.544118in}}%
\pgfusepath{clip}%
\pgfsetbuttcap%
\pgfsetroundjoin%
\pgfsetlinewidth{0.501875pt}%
\definecolor{currentstroke}{rgb}{0.273809,0.031497,0.358853}%
\pgfsetstrokecolor{currentstroke}%
\pgfsetdash{}{0pt}%
\pgfpathmoveto{\pgfqpoint{5.206279in}{5.634718in}}%
\pgfpathlineto{\pgfqpoint{5.153312in}{5.635307in}}%
\pgfusepath{stroke}%
\end{pgfscope}%
\begin{pgfscope}%
\pgfpathrectangle{\pgfqpoint{3.352233in}{5.105882in}}{\pgfqpoint{2.407767in}{1.544118in}}%
\pgfusepath{clip}%
\pgfsetbuttcap%
\pgfsetroundjoin%
\pgfsetlinewidth{0.501875pt}%
\definecolor{currentstroke}{rgb}{0.280267,0.073417,0.397163}%
\pgfsetstrokecolor{currentstroke}%
\pgfsetdash{}{0pt}%
\pgfpathmoveto{\pgfqpoint{5.153312in}{5.635307in}}%
\pgfpathlineto{\pgfqpoint{5.100360in}{5.636259in}}%
\pgfusepath{stroke}%
\end{pgfscope}%
\begin{pgfscope}%
\pgfpathrectangle{\pgfqpoint{3.352233in}{5.105882in}}{\pgfqpoint{2.407767in}{1.544118in}}%
\pgfusepath{clip}%
\pgfsetbuttcap%
\pgfsetroundjoin%
\pgfsetlinewidth{0.501875pt}%
\definecolor{currentstroke}{rgb}{0.282656,0.100196,0.422160}%
\pgfsetstrokecolor{currentstroke}%
\pgfsetdash{}{0pt}%
\pgfpathmoveto{\pgfqpoint{5.100360in}{5.636259in}}%
\pgfpathlineto{\pgfqpoint{5.047443in}{5.637819in}}%
\pgfusepath{stroke}%
\end{pgfscope}%
\begin{pgfscope}%
\pgfpathrectangle{\pgfqpoint{3.352233in}{5.105882in}}{\pgfqpoint{2.407767in}{1.544118in}}%
\pgfusepath{clip}%
\pgfsetbuttcap%
\pgfsetroundjoin%
\pgfsetlinewidth{0.501875pt}%
\definecolor{currentstroke}{rgb}{0.283072,0.130895,0.449241}%
\pgfsetstrokecolor{currentstroke}%
\pgfsetdash{}{0pt}%
\pgfpathmoveto{\pgfqpoint{5.047443in}{5.637819in}}%
\pgfpathlineto{\pgfqpoint{4.994571in}{5.639911in}}%
\pgfusepath{stroke}%
\end{pgfscope}%
\begin{pgfscope}%
\pgfpathrectangle{\pgfqpoint{3.352233in}{5.105882in}}{\pgfqpoint{2.407767in}{1.544118in}}%
\pgfusepath{clip}%
\pgfsetbuttcap%
\pgfsetroundjoin%
\pgfsetlinewidth{0.501875pt}%
\definecolor{currentstroke}{rgb}{0.281412,0.155834,0.469201}%
\pgfsetstrokecolor{currentstroke}%
\pgfsetdash{}{0pt}%
\pgfpathmoveto{\pgfqpoint{4.994571in}{5.639911in}}%
\pgfpathlineto{\pgfqpoint{4.941743in}{5.642437in}}%
\pgfusepath{stroke}%
\end{pgfscope}%
\begin{pgfscope}%
\pgfpathrectangle{\pgfqpoint{3.352233in}{5.105882in}}{\pgfqpoint{2.407767in}{1.544118in}}%
\pgfusepath{clip}%
\pgfsetbuttcap%
\pgfsetroundjoin%
\pgfsetlinewidth{0.501875pt}%
\definecolor{currentstroke}{rgb}{0.277134,0.185228,0.489898}%
\pgfsetstrokecolor{currentstroke}%
\pgfsetdash{}{0pt}%
\pgfpathmoveto{\pgfqpoint{4.941743in}{5.642437in}}%
\pgfpathlineto{\pgfqpoint{4.888992in}{5.645532in}}%
\pgfusepath{stroke}%
\end{pgfscope}%
\begin{pgfscope}%
\pgfpathrectangle{\pgfqpoint{3.352233in}{5.105882in}}{\pgfqpoint{2.407767in}{1.544118in}}%
\pgfusepath{clip}%
\pgfsetbuttcap%
\pgfsetroundjoin%
\pgfsetlinewidth{0.501875pt}%
\definecolor{currentstroke}{rgb}{0.277134,0.185228,0.489898}%
\pgfsetstrokecolor{currentstroke}%
\pgfsetdash{}{0pt}%
\pgfpathmoveto{\pgfqpoint{4.888992in}{5.645532in}}%
\pgfpathlineto{\pgfqpoint{4.836387in}{5.649514in}}%
\pgfusepath{stroke}%
\end{pgfscope}%
\begin{pgfscope}%
\pgfpathrectangle{\pgfqpoint{3.352233in}{5.105882in}}{\pgfqpoint{2.407767in}{1.544118in}}%
\pgfusepath{clip}%
\pgfsetbuttcap%
\pgfsetroundjoin%
\pgfsetlinewidth{0.501875pt}%
\definecolor{currentstroke}{rgb}{0.266580,0.228262,0.514349}%
\pgfsetstrokecolor{currentstroke}%
\pgfsetdash{}{0pt}%
\pgfpathmoveto{\pgfqpoint{4.836387in}{5.649514in}}%
\pgfpathlineto{\pgfqpoint{4.784001in}{5.654536in}}%
\pgfusepath{stroke}%
\end{pgfscope}%
\begin{pgfscope}%
\pgfpathrectangle{\pgfqpoint{3.352233in}{5.105882in}}{\pgfqpoint{2.407767in}{1.544118in}}%
\pgfusepath{clip}%
\pgfsetbuttcap%
\pgfsetroundjoin%
\pgfsetlinewidth{0.501875pt}%
\definecolor{currentstroke}{rgb}{0.265145,0.232956,0.516599}%
\pgfsetstrokecolor{currentstroke}%
\pgfsetdash{}{0pt}%
\pgfpathmoveto{\pgfqpoint{4.784001in}{5.654536in}}%
\pgfpathlineto{\pgfqpoint{4.731834in}{5.660433in}}%
\pgfusepath{stroke}%
\end{pgfscope}%
\begin{pgfscope}%
\pgfpathrectangle{\pgfqpoint{3.352233in}{5.105882in}}{\pgfqpoint{2.407767in}{1.544118in}}%
\pgfusepath{clip}%
\pgfsetbuttcap%
\pgfsetroundjoin%
\pgfsetlinewidth{0.501875pt}%
\definecolor{currentstroke}{rgb}{0.267968,0.223549,0.512008}%
\pgfsetstrokecolor{currentstroke}%
\pgfsetdash{}{0pt}%
\pgfpathmoveto{\pgfqpoint{4.731834in}{5.660433in}}%
\pgfpathlineto{\pgfqpoint{4.679973in}{5.667322in}}%
\pgfusepath{stroke}%
\end{pgfscope}%
\begin{pgfscope}%
\pgfpathrectangle{\pgfqpoint{3.352233in}{5.105882in}}{\pgfqpoint{2.407767in}{1.544118in}}%
\pgfusepath{clip}%
\pgfsetbuttcap%
\pgfsetroundjoin%
\pgfsetlinewidth{0.501875pt}%
\definecolor{currentstroke}{rgb}{0.250425,0.274290,0.533103}%
\pgfsetstrokecolor{currentstroke}%
\pgfsetdash{}{0pt}%
\pgfpathmoveto{\pgfqpoint{4.679973in}{5.667322in}}%
\pgfpathlineto{\pgfqpoint{4.628626in}{5.675632in}}%
\pgfusepath{stroke}%
\end{pgfscope}%
\begin{pgfscope}%
\pgfpathrectangle{\pgfqpoint{3.352233in}{5.105882in}}{\pgfqpoint{2.407767in}{1.544118in}}%
\pgfusepath{clip}%
\pgfsetbuttcap%
\pgfsetroundjoin%
\pgfsetlinewidth{0.501875pt}%
\definecolor{currentstroke}{rgb}{0.248629,0.278775,0.534556}%
\pgfsetstrokecolor{currentstroke}%
\pgfsetdash{}{0pt}%
\pgfpathmoveto{\pgfqpoint{4.628626in}{5.675632in}}%
\pgfpathlineto{\pgfqpoint{4.578029in}{5.685636in}}%
\pgfusepath{stroke}%
\end{pgfscope}%
\begin{pgfscope}%
\pgfpathrectangle{\pgfqpoint{3.352233in}{5.105882in}}{\pgfqpoint{2.407767in}{1.544118in}}%
\pgfusepath{clip}%
\pgfsetbuttcap%
\pgfsetroundjoin%
\pgfsetlinewidth{0.501875pt}%
\definecolor{currentstroke}{rgb}{0.250425,0.274290,0.533103}%
\pgfsetstrokecolor{currentstroke}%
\pgfsetdash{}{0pt}%
\pgfpathmoveto{\pgfqpoint{4.578029in}{5.685636in}}%
\pgfpathlineto{\pgfqpoint{4.528514in}{5.697633in}}%
\pgfusepath{stroke}%
\end{pgfscope}%
\begin{pgfscope}%
\pgfpathrectangle{\pgfqpoint{3.352233in}{5.105882in}}{\pgfqpoint{2.407767in}{1.544118in}}%
\pgfusepath{clip}%
\pgfsetbuttcap%
\pgfsetroundjoin%
\pgfsetlinewidth{0.501875pt}%
\definecolor{currentstroke}{rgb}{0.241237,0.296485,0.539709}%
\pgfsetstrokecolor{currentstroke}%
\pgfsetdash{}{0pt}%
\pgfpathmoveto{\pgfqpoint{4.528514in}{5.697633in}}%
\pgfpathlineto{\pgfqpoint{4.480046in}{5.711274in}}%
\pgfusepath{stroke}%
\end{pgfscope}%
\begin{pgfscope}%
\pgfpathrectangle{\pgfqpoint{3.352233in}{5.105882in}}{\pgfqpoint{2.407767in}{1.544118in}}%
\pgfusepath{clip}%
\pgfsetbuttcap%
\pgfsetroundjoin%
\pgfsetlinewidth{0.501875pt}%
\definecolor{currentstroke}{rgb}{0.250425,0.274290,0.533103}%
\pgfsetstrokecolor{currentstroke}%
\pgfsetdash{}{0pt}%
\pgfpathmoveto{\pgfqpoint{4.480046in}{5.711274in}}%
\pgfpathlineto{\pgfqpoint{4.432472in}{5.726150in}}%
\pgfusepath{stroke}%
\end{pgfscope}%
\begin{pgfscope}%
\pgfpathrectangle{\pgfqpoint{3.352233in}{5.105882in}}{\pgfqpoint{2.407767in}{1.544118in}}%
\pgfusepath{clip}%
\pgfsetbuttcap%
\pgfsetroundjoin%
\pgfsetlinewidth{0.501875pt}%
\definecolor{currentstroke}{rgb}{0.237441,0.305202,0.541921}%
\pgfsetstrokecolor{currentstroke}%
\pgfsetdash{}{0pt}%
\pgfpathmoveto{\pgfqpoint{4.432472in}{5.726150in}}%
\pgfpathlineto{\pgfqpoint{4.385110in}{5.741341in}}%
\pgfusepath{stroke}%
\end{pgfscope}%
\begin{pgfscope}%
\pgfpathrectangle{\pgfqpoint{3.352233in}{5.105882in}}{\pgfqpoint{2.407767in}{1.544118in}}%
\pgfusepath{clip}%
\pgfsetbuttcap%
\pgfsetroundjoin%
\pgfsetlinewidth{0.501875pt}%
\definecolor{currentstroke}{rgb}{0.274952,0.037752,0.364543}%
\pgfsetstrokecolor{currentstroke}%
\pgfsetdash{}{0pt}%
\pgfpathmoveto{\pgfqpoint{5.206279in}{5.669464in}}%
\pgfpathlineto{\pgfqpoint{5.153315in}{5.670075in}}%
\pgfusepath{stroke}%
\end{pgfscope}%
\begin{pgfscope}%
\pgfpathrectangle{\pgfqpoint{3.352233in}{5.105882in}}{\pgfqpoint{2.407767in}{1.544118in}}%
\pgfusepath{clip}%
\pgfsetbuttcap%
\pgfsetroundjoin%
\pgfsetlinewidth{0.501875pt}%
\definecolor{currentstroke}{rgb}{0.280267,0.073417,0.397163}%
\pgfsetstrokecolor{currentstroke}%
\pgfsetdash{}{0pt}%
\pgfpathmoveto{\pgfqpoint{5.153315in}{5.670075in}}%
\pgfpathlineto{\pgfqpoint{5.100368in}{5.671170in}}%
\pgfusepath{stroke}%
\end{pgfscope}%
\begin{pgfscope}%
\pgfpathrectangle{\pgfqpoint{3.352233in}{5.105882in}}{\pgfqpoint{2.407767in}{1.544118in}}%
\pgfusepath{clip}%
\pgfsetbuttcap%
\pgfsetroundjoin%
\pgfsetlinewidth{0.501875pt}%
\definecolor{currentstroke}{rgb}{0.283187,0.125848,0.444960}%
\pgfsetstrokecolor{currentstroke}%
\pgfsetdash{}{0pt}%
\pgfpathmoveto{\pgfqpoint{5.100368in}{5.671170in}}%
\pgfpathlineto{\pgfqpoint{5.047445in}{5.672667in}}%
\pgfusepath{stroke}%
\end{pgfscope}%
\begin{pgfscope}%
\pgfpathrectangle{\pgfqpoint{3.352233in}{5.105882in}}{\pgfqpoint{2.407767in}{1.544118in}}%
\pgfusepath{clip}%
\pgfsetbuttcap%
\pgfsetroundjoin%
\pgfsetlinewidth{0.501875pt}%
\definecolor{currentstroke}{rgb}{0.281887,0.150881,0.465405}%
\pgfsetstrokecolor{currentstroke}%
\pgfsetdash{}{0pt}%
\pgfpathmoveto{\pgfqpoint{5.047445in}{5.672667in}}%
\pgfpathlineto{\pgfqpoint{4.994558in}{5.674615in}}%
\pgfusepath{stroke}%
\end{pgfscope}%
\begin{pgfscope}%
\pgfpathrectangle{\pgfqpoint{3.352233in}{5.105882in}}{\pgfqpoint{2.407767in}{1.544118in}}%
\pgfusepath{clip}%
\pgfsetbuttcap%
\pgfsetroundjoin%
\pgfsetlinewidth{0.501875pt}%
\definecolor{currentstroke}{rgb}{0.278826,0.175490,0.483397}%
\pgfsetstrokecolor{currentstroke}%
\pgfsetdash{}{0pt}%
\pgfpathmoveto{\pgfqpoint{4.994558in}{5.674615in}}%
\pgfpathlineto{\pgfqpoint{4.941725in}{5.677080in}}%
\pgfusepath{stroke}%
\end{pgfscope}%
\begin{pgfscope}%
\pgfpathrectangle{\pgfqpoint{3.352233in}{5.105882in}}{\pgfqpoint{2.407767in}{1.544118in}}%
\pgfusepath{clip}%
\pgfsetbuttcap%
\pgfsetroundjoin%
\pgfsetlinewidth{0.501875pt}%
\definecolor{currentstroke}{rgb}{0.270595,0.214069,0.507052}%
\pgfsetstrokecolor{currentstroke}%
\pgfsetdash{}{0pt}%
\pgfpathmoveto{\pgfqpoint{4.941725in}{5.677080in}}%
\pgfpathlineto{\pgfqpoint{4.888946in}{5.680002in}}%
\pgfusepath{stroke}%
\end{pgfscope}%
\begin{pgfscope}%
\pgfpathrectangle{\pgfqpoint{3.352233in}{5.105882in}}{\pgfqpoint{2.407767in}{1.544118in}}%
\pgfusepath{clip}%
\pgfsetbuttcap%
\pgfsetroundjoin%
\pgfsetlinewidth{0.501875pt}%
\definecolor{currentstroke}{rgb}{0.257322,0.256130,0.526563}%
\pgfsetstrokecolor{currentstroke}%
\pgfsetdash{}{0pt}%
\pgfpathmoveto{\pgfqpoint{4.888946in}{5.680002in}}%
\pgfpathlineto{\pgfqpoint{4.836261in}{5.683525in}}%
\pgfusepath{stroke}%
\end{pgfscope}%
\begin{pgfscope}%
\pgfpathrectangle{\pgfqpoint{3.352233in}{5.105882in}}{\pgfqpoint{2.407767in}{1.544118in}}%
\pgfusepath{clip}%
\pgfsetbuttcap%
\pgfsetroundjoin%
\pgfsetlinewidth{0.501875pt}%
\definecolor{currentstroke}{rgb}{0.260571,0.246922,0.522828}%
\pgfsetstrokecolor{currentstroke}%
\pgfsetdash{}{0pt}%
\pgfpathmoveto{\pgfqpoint{4.836261in}{5.683525in}}%
\pgfpathlineto{\pgfqpoint{4.783737in}{5.687921in}}%
\pgfusepath{stroke}%
\end{pgfscope}%
\begin{pgfscope}%
\pgfpathrectangle{\pgfqpoint{3.352233in}{5.105882in}}{\pgfqpoint{2.407767in}{1.544118in}}%
\pgfusepath{clip}%
\pgfsetbuttcap%
\pgfsetroundjoin%
\pgfsetlinewidth{0.501875pt}%
\definecolor{currentstroke}{rgb}{0.250425,0.274290,0.533103}%
\pgfsetstrokecolor{currentstroke}%
\pgfsetdash{}{0pt}%
\pgfpathmoveto{\pgfqpoint{4.783737in}{5.687921in}}%
\pgfpathlineto{\pgfqpoint{4.731383in}{5.693094in}}%
\pgfusepath{stroke}%
\end{pgfscope}%
\begin{pgfscope}%
\pgfpathrectangle{\pgfqpoint{3.352233in}{5.105882in}}{\pgfqpoint{2.407767in}{1.544118in}}%
\pgfusepath{clip}%
\pgfsetbuttcap%
\pgfsetroundjoin%
\pgfsetlinewidth{0.501875pt}%
\definecolor{currentstroke}{rgb}{0.255645,0.260703,0.528312}%
\pgfsetstrokecolor{currentstroke}%
\pgfsetdash{}{0pt}%
\pgfpathmoveto{\pgfqpoint{4.731383in}{5.693094in}}%
\pgfpathlineto{\pgfqpoint{4.679201in}{5.698940in}}%
\pgfusepath{stroke}%
\end{pgfscope}%
\begin{pgfscope}%
\pgfpathrectangle{\pgfqpoint{3.352233in}{5.105882in}}{\pgfqpoint{2.407767in}{1.544118in}}%
\pgfusepath{clip}%
\pgfsetbuttcap%
\pgfsetroundjoin%
\pgfsetlinewidth{0.501875pt}%
\definecolor{currentstroke}{rgb}{0.244972,0.287675,0.537260}%
\pgfsetstrokecolor{currentstroke}%
\pgfsetdash{}{0pt}%
\pgfpathmoveto{\pgfqpoint{4.679201in}{5.698940in}}%
\pgfpathlineto{\pgfqpoint{4.627269in}{5.705619in}}%
\pgfusepath{stroke}%
\end{pgfscope}%
\begin{pgfscope}%
\pgfpathrectangle{\pgfqpoint{3.352233in}{5.105882in}}{\pgfqpoint{2.407767in}{1.544118in}}%
\pgfusepath{clip}%
\pgfsetbuttcap%
\pgfsetroundjoin%
\pgfsetlinewidth{0.501875pt}%
\definecolor{currentstroke}{rgb}{0.239346,0.300855,0.540844}%
\pgfsetstrokecolor{currentstroke}%
\pgfsetdash{}{0pt}%
\pgfpathmoveto{\pgfqpoint{4.627269in}{5.705619in}}%
\pgfpathlineto{\pgfqpoint{4.575829in}{5.713680in}}%
\pgfusepath{stroke}%
\end{pgfscope}%
\begin{pgfscope}%
\pgfpathrectangle{\pgfqpoint{3.352233in}{5.105882in}}{\pgfqpoint{2.407767in}{1.544118in}}%
\pgfusepath{clip}%
\pgfsetbuttcap%
\pgfsetroundjoin%
\pgfsetlinewidth{0.501875pt}%
\definecolor{currentstroke}{rgb}{0.227802,0.326594,0.546532}%
\pgfsetstrokecolor{currentstroke}%
\pgfsetdash{}{0pt}%
\pgfpathmoveto{\pgfqpoint{4.575829in}{5.713680in}}%
\pgfpathlineto{\pgfqpoint{4.525077in}{5.723374in}}%
\pgfusepath{stroke}%
\end{pgfscope}%
\begin{pgfscope}%
\pgfpathrectangle{\pgfqpoint{3.352233in}{5.105882in}}{\pgfqpoint{2.407767in}{1.544118in}}%
\pgfusepath{clip}%
\pgfsetbuttcap%
\pgfsetroundjoin%
\pgfsetlinewidth{0.501875pt}%
\definecolor{currentstroke}{rgb}{0.250425,0.274290,0.533103}%
\pgfsetstrokecolor{currentstroke}%
\pgfsetdash{}{0pt}%
\pgfpathmoveto{\pgfqpoint{4.525077in}{5.723374in}}%
\pgfpathlineto{\pgfqpoint{4.474951in}{5.734337in}}%
\pgfusepath{stroke}%
\end{pgfscope}%
\begin{pgfscope}%
\pgfpathrectangle{\pgfqpoint{3.352233in}{5.105882in}}{\pgfqpoint{2.407767in}{1.544118in}}%
\pgfusepath{clip}%
\pgfsetbuttcap%
\pgfsetroundjoin%
\pgfsetlinewidth{0.501875pt}%
\definecolor{currentstroke}{rgb}{0.276022,0.044167,0.370164}%
\pgfsetstrokecolor{currentstroke}%
\pgfsetdash{}{0pt}%
\pgfpathmoveto{\pgfqpoint{5.206279in}{5.704211in}}%
\pgfpathlineto{\pgfqpoint{5.153311in}{5.704778in}}%
\pgfusepath{stroke}%
\end{pgfscope}%
\begin{pgfscope}%
\pgfpathrectangle{\pgfqpoint{3.352233in}{5.105882in}}{\pgfqpoint{2.407767in}{1.544118in}}%
\pgfusepath{clip}%
\pgfsetbuttcap%
\pgfsetroundjoin%
\pgfsetlinewidth{0.501875pt}%
\definecolor{currentstroke}{rgb}{0.282656,0.100196,0.422160}%
\pgfsetstrokecolor{currentstroke}%
\pgfsetdash{}{0pt}%
\pgfpathmoveto{\pgfqpoint{5.153311in}{5.704778in}}%
\pgfpathlineto{\pgfqpoint{5.100354in}{5.705666in}}%
\pgfusepath{stroke}%
\end{pgfscope}%
\begin{pgfscope}%
\pgfpathrectangle{\pgfqpoint{3.352233in}{5.105882in}}{\pgfqpoint{2.407767in}{1.544118in}}%
\pgfusepath{clip}%
\pgfsetbuttcap%
\pgfsetroundjoin%
\pgfsetlinewidth{0.501875pt}%
\definecolor{currentstroke}{rgb}{0.282623,0.140926,0.457517}%
\pgfsetstrokecolor{currentstroke}%
\pgfsetdash{}{0pt}%
\pgfpathmoveto{\pgfqpoint{5.100354in}{5.705666in}}%
\pgfpathlineto{\pgfqpoint{5.047413in}{5.706874in}}%
\pgfusepath{stroke}%
\end{pgfscope}%
\begin{pgfscope}%
\pgfpathrectangle{\pgfqpoint{3.352233in}{5.105882in}}{\pgfqpoint{2.407767in}{1.544118in}}%
\pgfusepath{clip}%
\pgfsetbuttcap%
\pgfsetroundjoin%
\pgfsetlinewidth{0.501875pt}%
\definecolor{currentstroke}{rgb}{0.277134,0.185228,0.489898}%
\pgfsetstrokecolor{currentstroke}%
\pgfsetdash{}{0pt}%
\pgfpathmoveto{\pgfqpoint{5.047413in}{5.706874in}}%
\pgfpathlineto{\pgfqpoint{4.994493in}{5.708422in}}%
\pgfusepath{stroke}%
\end{pgfscope}%
\begin{pgfscope}%
\pgfpathrectangle{\pgfqpoint{3.352233in}{5.105882in}}{\pgfqpoint{2.407767in}{1.544118in}}%
\pgfusepath{clip}%
\pgfsetbuttcap%
\pgfsetroundjoin%
\pgfsetlinewidth{0.501875pt}%
\definecolor{currentstroke}{rgb}{0.266580,0.228262,0.514349}%
\pgfsetstrokecolor{currentstroke}%
\pgfsetdash{}{0pt}%
\pgfpathmoveto{\pgfqpoint{4.994493in}{5.708422in}}%
\pgfpathlineto{\pgfqpoint{4.941605in}{5.710371in}}%
\pgfusepath{stroke}%
\end{pgfscope}%
\begin{pgfscope}%
\pgfpathrectangle{\pgfqpoint{3.352233in}{5.105882in}}{\pgfqpoint{2.407767in}{1.544118in}}%
\pgfusepath{clip}%
\pgfsetbuttcap%
\pgfsetroundjoin%
\pgfsetlinewidth{0.501875pt}%
\definecolor{currentstroke}{rgb}{0.255645,0.260703,0.528312}%
\pgfsetstrokecolor{currentstroke}%
\pgfsetdash{}{0pt}%
\pgfpathmoveto{\pgfqpoint{4.941605in}{5.710371in}}%
\pgfpathlineto{\pgfqpoint{4.888767in}{5.712791in}}%
\pgfusepath{stroke}%
\end{pgfscope}%
\begin{pgfscope}%
\pgfpathrectangle{\pgfqpoint{3.352233in}{5.105882in}}{\pgfqpoint{2.407767in}{1.544118in}}%
\pgfusepath{clip}%
\pgfsetbuttcap%
\pgfsetroundjoin%
\pgfsetlinewidth{0.501875pt}%
\definecolor{currentstroke}{rgb}{0.250425,0.274290,0.533103}%
\pgfsetstrokecolor{currentstroke}%
\pgfsetdash{}{0pt}%
\pgfpathmoveto{\pgfqpoint{4.888767in}{5.712791in}}%
\pgfpathlineto{\pgfqpoint{4.835971in}{5.715567in}}%
\pgfusepath{stroke}%
\end{pgfscope}%
\begin{pgfscope}%
\pgfpathrectangle{\pgfqpoint{3.352233in}{5.105882in}}{\pgfqpoint{2.407767in}{1.544118in}}%
\pgfusepath{clip}%
\pgfsetbuttcap%
\pgfsetroundjoin%
\pgfsetlinewidth{0.501875pt}%
\definecolor{currentstroke}{rgb}{0.276022,0.044167,0.370164}%
\pgfsetstrokecolor{currentstroke}%
\pgfsetdash{}{0pt}%
\pgfpathmoveto{\pgfqpoint{5.206279in}{5.738957in}}%
\pgfpathlineto{\pgfqpoint{5.153306in}{5.739316in}}%
\pgfusepath{stroke}%
\end{pgfscope}%
\begin{pgfscope}%
\pgfpathrectangle{\pgfqpoint{3.352233in}{5.105882in}}{\pgfqpoint{2.407767in}{1.544118in}}%
\pgfusepath{clip}%
\pgfsetbuttcap%
\pgfsetroundjoin%
\pgfsetlinewidth{0.501875pt}%
\definecolor{currentstroke}{rgb}{0.282656,0.100196,0.422160}%
\pgfsetstrokecolor{currentstroke}%
\pgfsetdash{}{0pt}%
\pgfpathmoveto{\pgfqpoint{5.153306in}{5.739316in}}%
\pgfpathlineto{\pgfqpoint{5.100344in}{5.740051in}}%
\pgfusepath{stroke}%
\end{pgfscope}%
\begin{pgfscope}%
\pgfpathrectangle{\pgfqpoint{3.352233in}{5.105882in}}{\pgfqpoint{2.407767in}{1.544118in}}%
\pgfusepath{clip}%
\pgfsetbuttcap%
\pgfsetroundjoin%
\pgfsetlinewidth{0.501875pt}%
\definecolor{currentstroke}{rgb}{0.281412,0.155834,0.469201}%
\pgfsetstrokecolor{currentstroke}%
\pgfsetdash{}{0pt}%
\pgfpathmoveto{\pgfqpoint{5.100344in}{5.740051in}}%
\pgfpathlineto{\pgfqpoint{5.047393in}{5.741089in}}%
\pgfusepath{stroke}%
\end{pgfscope}%
\begin{pgfscope}%
\pgfpathrectangle{\pgfqpoint{3.352233in}{5.105882in}}{\pgfqpoint{2.407767in}{1.544118in}}%
\pgfusepath{clip}%
\pgfsetbuttcap%
\pgfsetroundjoin%
\pgfsetlinewidth{0.501875pt}%
\definecolor{currentstroke}{rgb}{0.275191,0.194905,0.496005}%
\pgfsetstrokecolor{currentstroke}%
\pgfsetdash{}{0pt}%
\pgfpathmoveto{\pgfqpoint{5.047393in}{5.741089in}}%
\pgfpathlineto{\pgfqpoint{4.994449in}{5.742264in}}%
\pgfusepath{stroke}%
\end{pgfscope}%
\begin{pgfscope}%
\pgfpathrectangle{\pgfqpoint{3.352233in}{5.105882in}}{\pgfqpoint{2.407767in}{1.544118in}}%
\pgfusepath{clip}%
\pgfsetbuttcap%
\pgfsetroundjoin%
\pgfsetlinewidth{0.501875pt}%
\definecolor{currentstroke}{rgb}{0.262138,0.242286,0.520837}%
\pgfsetstrokecolor{currentstroke}%
\pgfsetdash{}{0pt}%
\pgfpathmoveto{\pgfqpoint{4.994449in}{5.742264in}}%
\pgfpathlineto{\pgfqpoint{4.941519in}{5.743662in}}%
\pgfusepath{stroke}%
\end{pgfscope}%
\begin{pgfscope}%
\pgfpathrectangle{\pgfqpoint{3.352233in}{5.105882in}}{\pgfqpoint{2.407767in}{1.544118in}}%
\pgfusepath{clip}%
\pgfsetbuttcap%
\pgfsetroundjoin%
\pgfsetlinewidth{0.501875pt}%
\definecolor{currentstroke}{rgb}{0.248629,0.278775,0.534556}%
\pgfsetstrokecolor{currentstroke}%
\pgfsetdash{}{0pt}%
\pgfpathmoveto{\pgfqpoint{4.941519in}{5.743662in}}%
\pgfpathlineto{\pgfqpoint{4.888629in}{5.745572in}}%
\pgfusepath{stroke}%
\end{pgfscope}%
\begin{pgfscope}%
\pgfpathrectangle{\pgfqpoint{3.352233in}{5.105882in}}{\pgfqpoint{2.407767in}{1.544118in}}%
\pgfusepath{clip}%
\pgfsetbuttcap%
\pgfsetroundjoin%
\pgfsetlinewidth{0.501875pt}%
\definecolor{currentstroke}{rgb}{0.235526,0.309527,0.542944}%
\pgfsetstrokecolor{currentstroke}%
\pgfsetdash{}{0pt}%
\pgfpathmoveto{\pgfqpoint{4.888629in}{5.745572in}}%
\pgfpathlineto{\pgfqpoint{4.835792in}{5.748015in}}%
\pgfusepath{stroke}%
\end{pgfscope}%
\begin{pgfscope}%
\pgfpathrectangle{\pgfqpoint{3.352233in}{5.105882in}}{\pgfqpoint{2.407767in}{1.544118in}}%
\pgfusepath{clip}%
\pgfsetbuttcap%
\pgfsetroundjoin%
\pgfsetlinewidth{0.501875pt}%
\definecolor{currentstroke}{rgb}{0.223925,0.334994,0.548053}%
\pgfsetstrokecolor{currentstroke}%
\pgfsetdash{}{0pt}%
\pgfpathmoveto{\pgfqpoint{4.835792in}{5.748015in}}%
\pgfpathlineto{\pgfqpoint{4.783010in}{5.750909in}}%
\pgfusepath{stroke}%
\end{pgfscope}%
\begin{pgfscope}%
\pgfpathrectangle{\pgfqpoint{3.352233in}{5.105882in}}{\pgfqpoint{2.407767in}{1.544118in}}%
\pgfusepath{clip}%
\pgfsetbuttcap%
\pgfsetroundjoin%
\pgfsetlinewidth{0.501875pt}%
\definecolor{currentstroke}{rgb}{0.206756,0.371758,0.553117}%
\pgfsetstrokecolor{currentstroke}%
\pgfsetdash{}{0pt}%
\pgfpathmoveto{\pgfqpoint{4.783010in}{5.750909in}}%
\pgfpathlineto{\pgfqpoint{4.730305in}{5.754324in}}%
\pgfusepath{stroke}%
\end{pgfscope}%
\begin{pgfscope}%
\pgfpathrectangle{\pgfqpoint{3.352233in}{5.105882in}}{\pgfqpoint{2.407767in}{1.544118in}}%
\pgfusepath{clip}%
\pgfsetbuttcap%
\pgfsetroundjoin%
\pgfsetlinewidth{0.501875pt}%
\definecolor{currentstroke}{rgb}{0.194100,0.399323,0.555565}%
\pgfsetstrokecolor{currentstroke}%
\pgfsetdash{}{0pt}%
\pgfpathmoveto{\pgfqpoint{4.730305in}{5.754324in}}%
\pgfpathlineto{\pgfqpoint{4.677698in}{5.758312in}}%
\pgfusepath{stroke}%
\end{pgfscope}%
\begin{pgfscope}%
\pgfpathrectangle{\pgfqpoint{3.352233in}{5.105882in}}{\pgfqpoint{2.407767in}{1.544118in}}%
\pgfusepath{clip}%
\pgfsetbuttcap%
\pgfsetroundjoin%
\pgfsetlinewidth{0.501875pt}%
\definecolor{currentstroke}{rgb}{0.188923,0.410910,0.556326}%
\pgfsetstrokecolor{currentstroke}%
\pgfsetdash{}{0pt}%
\pgfpathmoveto{\pgfqpoint{4.677698in}{5.758312in}}%
\pgfpathlineto{\pgfqpoint{4.625257in}{5.763085in}}%
\pgfusepath{stroke}%
\end{pgfscope}%
\begin{pgfscope}%
\pgfpathrectangle{\pgfqpoint{3.352233in}{5.105882in}}{\pgfqpoint{2.407767in}{1.544118in}}%
\pgfusepath{clip}%
\pgfsetbuttcap%
\pgfsetroundjoin%
\pgfsetlinewidth{0.501875pt}%
\definecolor{currentstroke}{rgb}{0.194100,0.399323,0.555565}%
\pgfsetstrokecolor{currentstroke}%
\pgfsetdash{}{0pt}%
\pgfpathmoveto{\pgfqpoint{4.625257in}{5.763085in}}%
\pgfpathlineto{\pgfqpoint{4.573000in}{5.768640in}}%
\pgfusepath{stroke}%
\end{pgfscope}%
\begin{pgfscope}%
\pgfpathrectangle{\pgfqpoint{3.352233in}{5.105882in}}{\pgfqpoint{2.407767in}{1.544118in}}%
\pgfusepath{clip}%
\pgfsetbuttcap%
\pgfsetroundjoin%
\pgfsetlinewidth{0.501875pt}%
\definecolor{currentstroke}{rgb}{0.180629,0.429975,0.557282}%
\pgfsetstrokecolor{currentstroke}%
\pgfsetdash{}{0pt}%
\pgfpathmoveto{\pgfqpoint{4.573000in}{5.768640in}}%
\pgfpathlineto{\pgfqpoint{4.520915in}{5.774825in}}%
\pgfusepath{stroke}%
\end{pgfscope}%
\begin{pgfscope}%
\pgfpathrectangle{\pgfqpoint{3.352233in}{5.105882in}}{\pgfqpoint{2.407767in}{1.544118in}}%
\pgfusepath{clip}%
\pgfsetbuttcap%
\pgfsetroundjoin%
\pgfsetlinewidth{0.501875pt}%
\definecolor{currentstroke}{rgb}{0.179019,0.433756,0.557430}%
\pgfsetstrokecolor{currentstroke}%
\pgfsetdash{}{0pt}%
\pgfpathmoveto{\pgfqpoint{4.520915in}{5.774825in}}%
\pgfpathlineto{\pgfqpoint{4.469089in}{5.781824in}}%
\pgfusepath{stroke}%
\end{pgfscope}%
\begin{pgfscope}%
\pgfpathrectangle{\pgfqpoint{3.352233in}{5.105882in}}{\pgfqpoint{2.407767in}{1.544118in}}%
\pgfusepath{clip}%
\pgfsetbuttcap%
\pgfsetroundjoin%
\pgfsetlinewidth{0.501875pt}%
\definecolor{currentstroke}{rgb}{0.163625,0.471133,0.558148}%
\pgfsetstrokecolor{currentstroke}%
\pgfsetdash{}{0pt}%
\pgfpathmoveto{\pgfqpoint{4.469089in}{5.781824in}}%
\pgfpathlineto{\pgfqpoint{4.417571in}{5.789700in}}%
\pgfusepath{stroke}%
\end{pgfscope}%
\begin{pgfscope}%
\pgfpathrectangle{\pgfqpoint{3.352233in}{5.105882in}}{\pgfqpoint{2.407767in}{1.544118in}}%
\pgfusepath{clip}%
\pgfsetbuttcap%
\pgfsetroundjoin%
\pgfsetlinewidth{0.501875pt}%
\definecolor{currentstroke}{rgb}{0.136408,0.541173,0.554483}%
\pgfsetstrokecolor{currentstroke}%
\pgfsetdash{}{0pt}%
\pgfpathmoveto{\pgfqpoint{4.417571in}{5.789700in}}%
\pgfpathlineto{\pgfqpoint{4.366331in}{5.798285in}}%
\pgfusepath{stroke}%
\end{pgfscope}%
\begin{pgfscope}%
\pgfpathrectangle{\pgfqpoint{3.352233in}{5.105882in}}{\pgfqpoint{2.407767in}{1.544118in}}%
\pgfusepath{clip}%
\pgfsetbuttcap%
\pgfsetroundjoin%
\pgfsetlinewidth{0.501875pt}%
\definecolor{currentstroke}{rgb}{0.277941,0.056324,0.381191}%
\pgfsetstrokecolor{currentstroke}%
\pgfsetdash{}{0pt}%
\pgfpathmoveto{\pgfqpoint{5.206279in}{5.773703in}}%
\pgfpathlineto{\pgfqpoint{5.153304in}{5.773886in}}%
\pgfusepath{stroke}%
\end{pgfscope}%
\begin{pgfscope}%
\pgfpathrectangle{\pgfqpoint{3.352233in}{5.105882in}}{\pgfqpoint{2.407767in}{1.544118in}}%
\pgfusepath{clip}%
\pgfsetbuttcap%
\pgfsetroundjoin%
\pgfsetlinewidth{0.501875pt}%
\definecolor{currentstroke}{rgb}{0.283229,0.120777,0.440584}%
\pgfsetstrokecolor{currentstroke}%
\pgfsetdash{}{0pt}%
\pgfpathmoveto{\pgfqpoint{5.153304in}{5.773886in}}%
\pgfpathlineto{\pgfqpoint{5.100333in}{5.774289in}}%
\pgfusepath{stroke}%
\end{pgfscope}%
\begin{pgfscope}%
\pgfpathrectangle{\pgfqpoint{3.352233in}{5.105882in}}{\pgfqpoint{2.407767in}{1.544118in}}%
\pgfusepath{clip}%
\pgfsetbuttcap%
\pgfsetroundjoin%
\pgfsetlinewidth{0.501875pt}%
\definecolor{currentstroke}{rgb}{0.280255,0.165693,0.476498}%
\pgfsetstrokecolor{currentstroke}%
\pgfsetdash{}{0pt}%
\pgfpathmoveto{\pgfqpoint{5.100333in}{5.774289in}}%
\pgfpathlineto{\pgfqpoint{5.047367in}{5.774946in}}%
\pgfusepath{stroke}%
\end{pgfscope}%
\begin{pgfscope}%
\pgfpathrectangle{\pgfqpoint{3.352233in}{5.105882in}}{\pgfqpoint{2.407767in}{1.544118in}}%
\pgfusepath{clip}%
\pgfsetbuttcap%
\pgfsetroundjoin%
\pgfsetlinewidth{0.501875pt}%
\definecolor{currentstroke}{rgb}{0.269308,0.218818,0.509577}%
\pgfsetstrokecolor{currentstroke}%
\pgfsetdash{}{0pt}%
\pgfpathmoveto{\pgfqpoint{5.047367in}{5.774946in}}%
\pgfpathlineto{\pgfqpoint{4.994416in}{5.775946in}}%
\pgfusepath{stroke}%
\end{pgfscope}%
\begin{pgfscope}%
\pgfpathrectangle{\pgfqpoint{3.352233in}{5.105882in}}{\pgfqpoint{2.407767in}{1.544118in}}%
\pgfusepath{clip}%
\pgfsetbuttcap%
\pgfsetroundjoin%
\pgfsetlinewidth{0.501875pt}%
\definecolor{currentstroke}{rgb}{0.253935,0.265254,0.529983}%
\pgfsetstrokecolor{currentstroke}%
\pgfsetdash{}{0pt}%
\pgfpathmoveto{\pgfqpoint{4.994416in}{5.775946in}}%
\pgfpathlineto{\pgfqpoint{4.941481in}{5.777276in}}%
\pgfusepath{stroke}%
\end{pgfscope}%
\begin{pgfscope}%
\pgfpathrectangle{\pgfqpoint{3.352233in}{5.105882in}}{\pgfqpoint{2.407767in}{1.544118in}}%
\pgfusepath{clip}%
\pgfsetbuttcap%
\pgfsetroundjoin%
\pgfsetlinewidth{0.501875pt}%
\definecolor{currentstroke}{rgb}{0.229739,0.322361,0.545706}%
\pgfsetstrokecolor{currentstroke}%
\pgfsetdash{}{0pt}%
\pgfpathmoveto{\pgfqpoint{4.941481in}{5.777276in}}%
\pgfpathlineto{\pgfqpoint{4.888562in}{5.778847in}}%
\pgfusepath{stroke}%
\end{pgfscope}%
\begin{pgfscope}%
\pgfpathrectangle{\pgfqpoint{3.352233in}{5.105882in}}{\pgfqpoint{2.407767in}{1.544118in}}%
\pgfusepath{clip}%
\pgfsetbuttcap%
\pgfsetroundjoin%
\pgfsetlinewidth{0.501875pt}%
\definecolor{currentstroke}{rgb}{0.210503,0.363727,0.552206}%
\pgfsetstrokecolor{currentstroke}%
\pgfsetdash{}{0pt}%
\pgfpathmoveto{\pgfqpoint{4.888562in}{5.778847in}}%
\pgfpathlineto{\pgfqpoint{4.835666in}{5.780707in}}%
\pgfusepath{stroke}%
\end{pgfscope}%
\begin{pgfscope}%
\pgfpathrectangle{\pgfqpoint{3.352233in}{5.105882in}}{\pgfqpoint{2.407767in}{1.544118in}}%
\pgfusepath{clip}%
\pgfsetbuttcap%
\pgfsetroundjoin%
\pgfsetlinewidth{0.501875pt}%
\definecolor{currentstroke}{rgb}{0.212395,0.359683,0.551710}%
\pgfsetstrokecolor{currentstroke}%
\pgfsetdash{}{0pt}%
\pgfpathmoveto{\pgfqpoint{4.835666in}{5.780707in}}%
\pgfpathlineto{\pgfqpoint{4.782805in}{5.782927in}}%
\pgfusepath{stroke}%
\end{pgfscope}%
\begin{pgfscope}%
\pgfpathrectangle{\pgfqpoint{3.352233in}{5.105882in}}{\pgfqpoint{2.407767in}{1.544118in}}%
\pgfusepath{clip}%
\pgfsetbuttcap%
\pgfsetroundjoin%
\pgfsetlinewidth{0.501875pt}%
\definecolor{currentstroke}{rgb}{0.187231,0.414746,0.556547}%
\pgfsetstrokecolor{currentstroke}%
\pgfsetdash{}{0pt}%
\pgfpathmoveto{\pgfqpoint{4.782805in}{5.782927in}}%
\pgfpathlineto{\pgfqpoint{4.730000in}{5.785638in}}%
\pgfusepath{stroke}%
\end{pgfscope}%
\begin{pgfscope}%
\pgfpathrectangle{\pgfqpoint{3.352233in}{5.105882in}}{\pgfqpoint{2.407767in}{1.544118in}}%
\pgfusepath{clip}%
\pgfsetbuttcap%
\pgfsetroundjoin%
\pgfsetlinewidth{0.501875pt}%
\definecolor{currentstroke}{rgb}{0.277941,0.056324,0.381191}%
\pgfsetstrokecolor{currentstroke}%
\pgfsetdash{}{0pt}%
\pgfpathmoveto{\pgfqpoint{5.206279in}{5.808449in}}%
\pgfpathlineto{\pgfqpoint{5.153304in}{5.808645in}}%
\pgfusepath{stroke}%
\end{pgfscope}%
\begin{pgfscope}%
\pgfpathrectangle{\pgfqpoint{3.352233in}{5.105882in}}{\pgfqpoint{2.407767in}{1.544118in}}%
\pgfusepath{clip}%
\pgfsetbuttcap%
\pgfsetroundjoin%
\pgfsetlinewidth{0.501875pt}%
\definecolor{currentstroke}{rgb}{0.283187,0.125848,0.444960}%
\pgfsetstrokecolor{currentstroke}%
\pgfsetdash{}{0pt}%
\pgfpathmoveto{\pgfqpoint{5.153304in}{5.808645in}}%
\pgfpathlineto{\pgfqpoint{5.100331in}{5.808984in}}%
\pgfusepath{stroke}%
\end{pgfscope}%
\begin{pgfscope}%
\pgfpathrectangle{\pgfqpoint{3.352233in}{5.105882in}}{\pgfqpoint{2.407767in}{1.544118in}}%
\pgfusepath{clip}%
\pgfsetbuttcap%
\pgfsetroundjoin%
\pgfsetlinewidth{0.501875pt}%
\definecolor{currentstroke}{rgb}{0.278826,0.175490,0.483397}%
\pgfsetstrokecolor{currentstroke}%
\pgfsetdash{}{0pt}%
\pgfpathmoveto{\pgfqpoint{5.100331in}{5.808984in}}%
\pgfpathlineto{\pgfqpoint{5.047359in}{5.809416in}}%
\pgfusepath{stroke}%
\end{pgfscope}%
\begin{pgfscope}%
\pgfpathrectangle{\pgfqpoint{3.352233in}{5.105882in}}{\pgfqpoint{2.407767in}{1.544118in}}%
\pgfusepath{clip}%
\pgfsetbuttcap%
\pgfsetroundjoin%
\pgfsetlinewidth{0.501875pt}%
\definecolor{currentstroke}{rgb}{0.263663,0.237631,0.518762}%
\pgfsetstrokecolor{currentstroke}%
\pgfsetdash{}{0pt}%
\pgfpathmoveto{\pgfqpoint{5.047359in}{5.809416in}}%
\pgfpathlineto{\pgfqpoint{4.994392in}{5.810033in}}%
\pgfusepath{stroke}%
\end{pgfscope}%
\begin{pgfscope}%
\pgfpathrectangle{\pgfqpoint{3.352233in}{5.105882in}}{\pgfqpoint{2.407767in}{1.544118in}}%
\pgfusepath{clip}%
\pgfsetbuttcap%
\pgfsetroundjoin%
\pgfsetlinewidth{0.501875pt}%
\definecolor{currentstroke}{rgb}{0.244972,0.287675,0.537260}%
\pgfsetstrokecolor{currentstroke}%
\pgfsetdash{}{0pt}%
\pgfpathmoveto{\pgfqpoint{4.994392in}{5.810033in}}%
\pgfpathlineto{\pgfqpoint{4.941433in}{5.810879in}}%
\pgfusepath{stroke}%
\end{pgfscope}%
\begin{pgfscope}%
\pgfpathrectangle{\pgfqpoint{3.352233in}{5.105882in}}{\pgfqpoint{2.407767in}{1.544118in}}%
\pgfusepath{clip}%
\pgfsetbuttcap%
\pgfsetroundjoin%
\pgfsetlinewidth{0.501875pt}%
\definecolor{currentstroke}{rgb}{0.220057,0.343307,0.549413}%
\pgfsetstrokecolor{currentstroke}%
\pgfsetdash{}{0pt}%
\pgfpathmoveto{\pgfqpoint{4.941433in}{5.810879in}}%
\pgfpathlineto{\pgfqpoint{4.888479in}{5.811844in}}%
\pgfusepath{stroke}%
\end{pgfscope}%
\begin{pgfscope}%
\pgfpathrectangle{\pgfqpoint{3.352233in}{5.105882in}}{\pgfqpoint{2.407767in}{1.544118in}}%
\pgfusepath{clip}%
\pgfsetbuttcap%
\pgfsetroundjoin%
\pgfsetlinewidth{0.501875pt}%
\definecolor{currentstroke}{rgb}{0.208623,0.367752,0.552675}%
\pgfsetstrokecolor{currentstroke}%
\pgfsetdash{}{0pt}%
\pgfpathmoveto{\pgfqpoint{4.888479in}{5.811844in}}%
\pgfpathlineto{\pgfqpoint{4.835539in}{5.813066in}}%
\pgfusepath{stroke}%
\end{pgfscope}%
\begin{pgfscope}%
\pgfpathrectangle{\pgfqpoint{3.352233in}{5.105882in}}{\pgfqpoint{2.407767in}{1.544118in}}%
\pgfusepath{clip}%
\pgfsetbuttcap%
\pgfsetroundjoin%
\pgfsetlinewidth{0.501875pt}%
\definecolor{currentstroke}{rgb}{0.185556,0.418570,0.556753}%
\pgfsetstrokecolor{currentstroke}%
\pgfsetdash{}{0pt}%
\pgfpathmoveto{\pgfqpoint{4.835539in}{5.813066in}}%
\pgfpathlineto{\pgfqpoint{4.782618in}{5.814604in}}%
\pgfusepath{stroke}%
\end{pgfscope}%
\begin{pgfscope}%
\pgfpathrectangle{\pgfqpoint{3.352233in}{5.105882in}}{\pgfqpoint{2.407767in}{1.544118in}}%
\pgfusepath{clip}%
\pgfsetbuttcap%
\pgfsetroundjoin%
\pgfsetlinewidth{0.501875pt}%
\definecolor{currentstroke}{rgb}{0.175841,0.441290,0.557685}%
\pgfsetstrokecolor{currentstroke}%
\pgfsetdash{}{0pt}%
\pgfpathmoveto{\pgfqpoint{4.782618in}{5.814604in}}%
\pgfpathlineto{\pgfqpoint{4.729710in}{5.816312in}}%
\pgfusepath{stroke}%
\end{pgfscope}%
\begin{pgfscope}%
\pgfpathrectangle{\pgfqpoint{3.352233in}{5.105882in}}{\pgfqpoint{2.407767in}{1.544118in}}%
\pgfusepath{clip}%
\pgfsetbuttcap%
\pgfsetroundjoin%
\pgfsetlinewidth{0.501875pt}%
\definecolor{currentstroke}{rgb}{0.154815,0.493313,0.557840}%
\pgfsetstrokecolor{currentstroke}%
\pgfsetdash{}{0pt}%
\pgfpathmoveto{\pgfqpoint{4.729710in}{5.816312in}}%
\pgfpathlineto{\pgfqpoint{4.676830in}{5.818344in}}%
\pgfusepath{stroke}%
\end{pgfscope}%
\begin{pgfscope}%
\pgfpathrectangle{\pgfqpoint{3.352233in}{5.105882in}}{\pgfqpoint{2.407767in}{1.544118in}}%
\pgfusepath{clip}%
\pgfsetbuttcap%
\pgfsetroundjoin%
\pgfsetlinewidth{0.501875pt}%
\definecolor{currentstroke}{rgb}{0.157729,0.485932,0.558013}%
\pgfsetstrokecolor{currentstroke}%
\pgfsetdash{}{0pt}%
\pgfpathmoveto{\pgfqpoint{4.676830in}{5.818344in}}%
\pgfpathlineto{\pgfqpoint{4.623997in}{5.820822in}}%
\pgfusepath{stroke}%
\end{pgfscope}%
\begin{pgfscope}%
\pgfpathrectangle{\pgfqpoint{3.352233in}{5.105882in}}{\pgfqpoint{2.407767in}{1.544118in}}%
\pgfusepath{clip}%
\pgfsetbuttcap%
\pgfsetroundjoin%
\pgfsetlinewidth{0.501875pt}%
\definecolor{currentstroke}{rgb}{0.140536,0.530132,0.555659}%
\pgfsetstrokecolor{currentstroke}%
\pgfsetdash{}{0pt}%
\pgfpathmoveto{\pgfqpoint{4.623997in}{5.820822in}}%
\pgfpathlineto{\pgfqpoint{4.571210in}{5.823666in}}%
\pgfusepath{stroke}%
\end{pgfscope}%
\begin{pgfscope}%
\pgfpathrectangle{\pgfqpoint{3.352233in}{5.105882in}}{\pgfqpoint{2.407767in}{1.544118in}}%
\pgfusepath{clip}%
\pgfsetbuttcap%
\pgfsetroundjoin%
\pgfsetlinewidth{0.501875pt}%
\definecolor{currentstroke}{rgb}{0.127568,0.566949,0.550556}%
\pgfsetstrokecolor{currentstroke}%
\pgfsetdash{}{0pt}%
\pgfpathmoveto{\pgfqpoint{4.571210in}{5.823666in}}%
\pgfpathlineto{\pgfqpoint{4.518487in}{5.826941in}}%
\pgfusepath{stroke}%
\end{pgfscope}%
\begin{pgfscope}%
\pgfpathrectangle{\pgfqpoint{3.352233in}{5.105882in}}{\pgfqpoint{2.407767in}{1.544118in}}%
\pgfusepath{clip}%
\pgfsetbuttcap%
\pgfsetroundjoin%
\pgfsetlinewidth{0.501875pt}%
\definecolor{currentstroke}{rgb}{0.119738,0.603785,0.541400}%
\pgfsetstrokecolor{currentstroke}%
\pgfsetdash{}{0pt}%
\pgfpathmoveto{\pgfqpoint{4.518487in}{5.826941in}}%
\pgfpathlineto{\pgfqpoint{4.465833in}{5.830655in}}%
\pgfusepath{stroke}%
\end{pgfscope}%
\begin{pgfscope}%
\pgfpathrectangle{\pgfqpoint{3.352233in}{5.105882in}}{\pgfqpoint{2.407767in}{1.544118in}}%
\pgfusepath{clip}%
\pgfsetbuttcap%
\pgfsetroundjoin%
\pgfsetlinewidth{0.501875pt}%
\definecolor{currentstroke}{rgb}{0.137339,0.662252,0.515571}%
\pgfsetstrokecolor{currentstroke}%
\pgfsetdash{}{0pt}%
\pgfpathmoveto{\pgfqpoint{4.465833in}{5.830655in}}%
\pgfpathlineto{\pgfqpoint{4.413241in}{5.834709in}}%
\pgfusepath{stroke}%
\end{pgfscope}%
\begin{pgfscope}%
\pgfpathrectangle{\pgfqpoint{3.352233in}{5.105882in}}{\pgfqpoint{2.407767in}{1.544118in}}%
\pgfusepath{clip}%
\pgfsetbuttcap%
\pgfsetroundjoin%
\pgfsetlinewidth{0.501875pt}%
\definecolor{currentstroke}{rgb}{0.175707,0.697900,0.491033}%
\pgfsetstrokecolor{currentstroke}%
\pgfsetdash{}{0pt}%
\pgfpathmoveto{\pgfqpoint{4.413241in}{5.834709in}}%
\pgfpathlineto{\pgfqpoint{4.360677in}{5.838882in}}%
\pgfusepath{stroke}%
\end{pgfscope}%
\begin{pgfscope}%
\pgfpathrectangle{\pgfqpoint{3.352233in}{5.105882in}}{\pgfqpoint{2.407767in}{1.544118in}}%
\pgfusepath{clip}%
\pgfsetbuttcap%
\pgfsetroundjoin%
\pgfsetlinewidth{0.501875pt}%
\definecolor{currentstroke}{rgb}{0.277941,0.056324,0.381191}%
\pgfsetstrokecolor{currentstroke}%
\pgfsetdash{}{0pt}%
\pgfpathmoveto{\pgfqpoint{5.206279in}{5.843195in}}%
\pgfpathlineto{\pgfqpoint{5.153305in}{5.843385in}}%
\pgfusepath{stroke}%
\end{pgfscope}%
\begin{pgfscope}%
\pgfpathrectangle{\pgfqpoint{3.352233in}{5.105882in}}{\pgfqpoint{2.407767in}{1.544118in}}%
\pgfusepath{clip}%
\pgfsetbuttcap%
\pgfsetroundjoin%
\pgfsetlinewidth{0.501875pt}%
\definecolor{currentstroke}{rgb}{0.282884,0.135920,0.453427}%
\pgfsetstrokecolor{currentstroke}%
\pgfsetdash{}{0pt}%
\pgfpathmoveto{\pgfqpoint{5.153305in}{5.843385in}}%
\pgfpathlineto{\pgfqpoint{5.100331in}{5.843660in}}%
\pgfusepath{stroke}%
\end{pgfscope}%
\begin{pgfscope}%
\pgfpathrectangle{\pgfqpoint{3.352233in}{5.105882in}}{\pgfqpoint{2.407767in}{1.544118in}}%
\pgfusepath{clip}%
\pgfsetbuttcap%
\pgfsetroundjoin%
\pgfsetlinewidth{0.501875pt}%
\definecolor{currentstroke}{rgb}{0.275191,0.194905,0.496005}%
\pgfsetstrokecolor{currentstroke}%
\pgfsetdash{}{0pt}%
\pgfpathmoveto{\pgfqpoint{5.100331in}{5.843660in}}%
\pgfpathlineto{\pgfqpoint{5.047357in}{5.843942in}}%
\pgfusepath{stroke}%
\end{pgfscope}%
\begin{pgfscope}%
\pgfpathrectangle{\pgfqpoint{3.352233in}{5.105882in}}{\pgfqpoint{2.407767in}{1.544118in}}%
\pgfusepath{clip}%
\pgfsetbuttcap%
\pgfsetroundjoin%
\pgfsetlinewidth{0.501875pt}%
\definecolor{currentstroke}{rgb}{0.258965,0.251537,0.524736}%
\pgfsetstrokecolor{currentstroke}%
\pgfsetdash{}{0pt}%
\pgfpathmoveto{\pgfqpoint{5.047357in}{5.843942in}}%
\pgfpathlineto{\pgfqpoint{4.994385in}{5.844353in}}%
\pgfusepath{stroke}%
\end{pgfscope}%
\begin{pgfscope}%
\pgfpathrectangle{\pgfqpoint{3.352233in}{5.105882in}}{\pgfqpoint{2.407767in}{1.544118in}}%
\pgfusepath{clip}%
\pgfsetbuttcap%
\pgfsetroundjoin%
\pgfsetlinewidth{0.501875pt}%
\definecolor{currentstroke}{rgb}{0.235526,0.309527,0.542944}%
\pgfsetstrokecolor{currentstroke}%
\pgfsetdash{}{0pt}%
\pgfpathmoveto{\pgfqpoint{4.994385in}{5.844353in}}%
\pgfpathlineto{\pgfqpoint{4.941416in}{5.844894in}}%
\pgfusepath{stroke}%
\end{pgfscope}%
\begin{pgfscope}%
\pgfpathrectangle{\pgfqpoint{3.352233in}{5.105882in}}{\pgfqpoint{2.407767in}{1.544118in}}%
\pgfusepath{clip}%
\pgfsetbuttcap%
\pgfsetroundjoin%
\pgfsetlinewidth{0.501875pt}%
\definecolor{currentstroke}{rgb}{0.218130,0.347432,0.550038}%
\pgfsetstrokecolor{currentstroke}%
\pgfsetdash{}{0pt}%
\pgfpathmoveto{\pgfqpoint{4.941416in}{5.844894in}}%
\pgfpathlineto{\pgfqpoint{4.888450in}{5.845533in}}%
\pgfusepath{stroke}%
\end{pgfscope}%
\begin{pgfscope}%
\pgfpathrectangle{\pgfqpoint{3.352233in}{5.105882in}}{\pgfqpoint{2.407767in}{1.544118in}}%
\pgfusepath{clip}%
\pgfsetbuttcap%
\pgfsetroundjoin%
\pgfsetlinewidth{0.501875pt}%
\definecolor{currentstroke}{rgb}{0.197636,0.391528,0.554969}%
\pgfsetstrokecolor{currentstroke}%
\pgfsetdash{}{0pt}%
\pgfpathmoveto{\pgfqpoint{4.888450in}{5.845533in}}%
\pgfpathlineto{\pgfqpoint{4.835487in}{5.846307in}}%
\pgfusepath{stroke}%
\end{pgfscope}%
\begin{pgfscope}%
\pgfpathrectangle{\pgfqpoint{3.352233in}{5.105882in}}{\pgfqpoint{2.407767in}{1.544118in}}%
\pgfusepath{clip}%
\pgfsetbuttcap%
\pgfsetroundjoin%
\pgfsetlinewidth{0.501875pt}%
\definecolor{currentstroke}{rgb}{0.182256,0.426184,0.557120}%
\pgfsetstrokecolor{currentstroke}%
\pgfsetdash{}{0pt}%
\pgfpathmoveto{\pgfqpoint{4.835487in}{5.846307in}}%
\pgfpathlineto{\pgfqpoint{4.782531in}{5.847216in}}%
\pgfusepath{stroke}%
\end{pgfscope}%
\begin{pgfscope}%
\pgfpathrectangle{\pgfqpoint{3.352233in}{5.105882in}}{\pgfqpoint{2.407767in}{1.544118in}}%
\pgfusepath{clip}%
\pgfsetbuttcap%
\pgfsetroundjoin%
\pgfsetlinewidth{0.501875pt}%
\definecolor{currentstroke}{rgb}{0.169646,0.456262,0.558030}%
\pgfsetstrokecolor{currentstroke}%
\pgfsetdash{}{0pt}%
\pgfpathmoveto{\pgfqpoint{4.782531in}{5.847216in}}%
\pgfpathlineto{\pgfqpoint{4.729582in}{5.848292in}}%
\pgfusepath{stroke}%
\end{pgfscope}%
\begin{pgfscope}%
\pgfpathrectangle{\pgfqpoint{3.352233in}{5.105882in}}{\pgfqpoint{2.407767in}{1.544118in}}%
\pgfusepath{clip}%
\pgfsetbuttcap%
\pgfsetroundjoin%
\pgfsetlinewidth{0.501875pt}%
\definecolor{currentstroke}{rgb}{0.139147,0.533812,0.555298}%
\pgfsetstrokecolor{currentstroke}%
\pgfsetdash{}{0pt}%
\pgfpathmoveto{\pgfqpoint{4.729582in}{5.848292in}}%
\pgfpathlineto{\pgfqpoint{4.676640in}{5.849488in}}%
\pgfusepath{stroke}%
\end{pgfscope}%
\begin{pgfscope}%
\pgfpathrectangle{\pgfqpoint{3.352233in}{5.105882in}}{\pgfqpoint{2.407767in}{1.544118in}}%
\pgfusepath{clip}%
\pgfsetbuttcap%
\pgfsetroundjoin%
\pgfsetlinewidth{0.501875pt}%
\definecolor{currentstroke}{rgb}{0.135066,0.544853,0.554029}%
\pgfsetstrokecolor{currentstroke}%
\pgfsetdash{}{0pt}%
\pgfpathmoveto{\pgfqpoint{4.676640in}{5.849488in}}%
\pgfpathlineto{\pgfqpoint{4.623708in}{5.850841in}}%
\pgfusepath{stroke}%
\end{pgfscope}%
\begin{pgfscope}%
\pgfpathrectangle{\pgfqpoint{3.352233in}{5.105882in}}{\pgfqpoint{2.407767in}{1.544118in}}%
\pgfusepath{clip}%
\pgfsetbuttcap%
\pgfsetroundjoin%
\pgfsetlinewidth{0.501875pt}%
\definecolor{currentstroke}{rgb}{0.124395,0.578002,0.548287}%
\pgfsetstrokecolor{currentstroke}%
\pgfsetdash{}{0pt}%
\pgfpathmoveto{\pgfqpoint{4.623708in}{5.850841in}}%
\pgfpathlineto{\pgfqpoint{4.570791in}{5.852363in}}%
\pgfusepath{stroke}%
\end{pgfscope}%
\begin{pgfscope}%
\pgfpathrectangle{\pgfqpoint{3.352233in}{5.105882in}}{\pgfqpoint{2.407767in}{1.544118in}}%
\pgfusepath{clip}%
\pgfsetbuttcap%
\pgfsetroundjoin%
\pgfsetlinewidth{0.501875pt}%
\definecolor{currentstroke}{rgb}{0.278791,0.062145,0.386592}%
\pgfsetstrokecolor{currentstroke}%
\pgfsetdash{}{0pt}%
\pgfpathmoveto{\pgfqpoint{5.206279in}{5.877941in}}%
\pgfpathlineto{\pgfqpoint{5.153304in}{5.878167in}}%
\pgfusepath{stroke}%
\end{pgfscope}%
\begin{pgfscope}%
\pgfpathrectangle{\pgfqpoint{3.352233in}{5.105882in}}{\pgfqpoint{2.407767in}{1.544118in}}%
\pgfusepath{clip}%
\pgfsetbuttcap%
\pgfsetroundjoin%
\pgfsetlinewidth{0.501875pt}%
\definecolor{currentstroke}{rgb}{0.283072,0.130895,0.449241}%
\pgfsetstrokecolor{currentstroke}%
\pgfsetdash{}{0pt}%
\pgfpathmoveto{\pgfqpoint{5.153304in}{5.878167in}}%
\pgfpathlineto{\pgfqpoint{5.100329in}{5.878250in}}%
\pgfusepath{stroke}%
\end{pgfscope}%
\begin{pgfscope}%
\pgfpathrectangle{\pgfqpoint{3.352233in}{5.105882in}}{\pgfqpoint{2.407767in}{1.544118in}}%
\pgfusepath{clip}%
\pgfsetbuttcap%
\pgfsetroundjoin%
\pgfsetlinewidth{0.501875pt}%
\definecolor{currentstroke}{rgb}{0.275191,0.194905,0.496005}%
\pgfsetstrokecolor{currentstroke}%
\pgfsetdash{}{0pt}%
\pgfpathmoveto{\pgfqpoint{5.100329in}{5.878250in}}%
\pgfpathlineto{\pgfqpoint{5.047354in}{5.878107in}}%
\pgfusepath{stroke}%
\end{pgfscope}%
\begin{pgfscope}%
\pgfpathrectangle{\pgfqpoint{3.352233in}{5.105882in}}{\pgfqpoint{2.407767in}{1.544118in}}%
\pgfusepath{clip}%
\pgfsetbuttcap%
\pgfsetroundjoin%
\pgfsetlinewidth{0.501875pt}%
\definecolor{currentstroke}{rgb}{0.262138,0.242286,0.520837}%
\pgfsetstrokecolor{currentstroke}%
\pgfsetdash{}{0pt}%
\pgfpathmoveto{\pgfqpoint{5.047354in}{5.878107in}}%
\pgfpathlineto{\pgfqpoint{4.994379in}{5.877940in}}%
\pgfusepath{stroke}%
\end{pgfscope}%
\begin{pgfscope}%
\pgfpathrectangle{\pgfqpoint{3.352233in}{5.105882in}}{\pgfqpoint{2.407767in}{1.544118in}}%
\pgfusepath{clip}%
\pgfsetbuttcap%
\pgfsetroundjoin%
\pgfsetlinewidth{0.501875pt}%
\definecolor{currentstroke}{rgb}{0.239346,0.300855,0.540844}%
\pgfsetstrokecolor{currentstroke}%
\pgfsetdash{}{0pt}%
\pgfpathmoveto{\pgfqpoint{4.994379in}{5.877940in}}%
\pgfpathlineto{\pgfqpoint{4.941403in}{5.877818in}}%
\pgfusepath{stroke}%
\end{pgfscope}%
\begin{pgfscope}%
\pgfpathrectangle{\pgfqpoint{3.352233in}{5.105882in}}{\pgfqpoint{2.407767in}{1.544118in}}%
\pgfusepath{clip}%
\pgfsetbuttcap%
\pgfsetroundjoin%
\pgfsetlinewidth{0.501875pt}%
\definecolor{currentstroke}{rgb}{0.212395,0.359683,0.551710}%
\pgfsetstrokecolor{currentstroke}%
\pgfsetdash{}{0pt}%
\pgfpathmoveto{\pgfqpoint{4.941403in}{5.877818in}}%
\pgfpathlineto{\pgfqpoint{4.888427in}{5.877784in}}%
\pgfusepath{stroke}%
\end{pgfscope}%
\begin{pgfscope}%
\pgfpathrectangle{\pgfqpoint{3.352233in}{5.105882in}}{\pgfqpoint{2.407767in}{1.544118in}}%
\pgfusepath{clip}%
\pgfsetbuttcap%
\pgfsetroundjoin%
\pgfsetlinewidth{0.501875pt}%
\definecolor{currentstroke}{rgb}{0.199430,0.387607,0.554642}%
\pgfsetstrokecolor{currentstroke}%
\pgfsetdash{}{0pt}%
\pgfpathmoveto{\pgfqpoint{4.888427in}{5.877784in}}%
\pgfpathlineto{\pgfqpoint{4.835451in}{5.877695in}}%
\pgfusepath{stroke}%
\end{pgfscope}%
\begin{pgfscope}%
\pgfpathrectangle{\pgfqpoint{3.352233in}{5.105882in}}{\pgfqpoint{2.407767in}{1.544118in}}%
\pgfusepath{clip}%
\pgfsetbuttcap%
\pgfsetroundjoin%
\pgfsetlinewidth{0.501875pt}%
\definecolor{currentstroke}{rgb}{0.175841,0.441290,0.557685}%
\pgfsetstrokecolor{currentstroke}%
\pgfsetdash{}{0pt}%
\pgfpathmoveto{\pgfqpoint{4.835451in}{5.877695in}}%
\pgfpathlineto{\pgfqpoint{4.782476in}{5.877499in}}%
\pgfusepath{stroke}%
\end{pgfscope}%
\begin{pgfscope}%
\pgfpathrectangle{\pgfqpoint{3.352233in}{5.105882in}}{\pgfqpoint{2.407767in}{1.544118in}}%
\pgfusepath{clip}%
\pgfsetbuttcap%
\pgfsetroundjoin%
\pgfsetlinewidth{0.501875pt}%
\definecolor{currentstroke}{rgb}{0.153364,0.497000,0.557724}%
\pgfsetstrokecolor{currentstroke}%
\pgfsetdash{}{0pt}%
\pgfpathmoveto{\pgfqpoint{4.782476in}{5.877499in}}%
\pgfpathlineto{\pgfqpoint{4.729502in}{5.877294in}}%
\pgfusepath{stroke}%
\end{pgfscope}%
\begin{pgfscope}%
\pgfpathrectangle{\pgfqpoint{3.352233in}{5.105882in}}{\pgfqpoint{2.407767in}{1.544118in}}%
\pgfusepath{clip}%
\pgfsetbuttcap%
\pgfsetroundjoin%
\pgfsetlinewidth{0.501875pt}%
\definecolor{currentstroke}{rgb}{0.144759,0.519093,0.556572}%
\pgfsetstrokecolor{currentstroke}%
\pgfsetdash{}{0pt}%
\pgfpathmoveto{\pgfqpoint{4.729502in}{5.877294in}}%
\pgfpathlineto{\pgfqpoint{4.676527in}{5.877122in}}%
\pgfusepath{stroke}%
\end{pgfscope}%
\begin{pgfscope}%
\pgfpathrectangle{\pgfqpoint{3.352233in}{5.105882in}}{\pgfqpoint{2.407767in}{1.544118in}}%
\pgfusepath{clip}%
\pgfsetbuttcap%
\pgfsetroundjoin%
\pgfsetlinewidth{0.501875pt}%
\definecolor{currentstroke}{rgb}{0.144759,0.519093,0.556572}%
\pgfsetstrokecolor{currentstroke}%
\pgfsetdash{}{0pt}%
\pgfpathmoveto{\pgfqpoint{4.676527in}{5.877122in}}%
\pgfpathlineto{\pgfqpoint{4.623552in}{5.876962in}}%
\pgfusepath{stroke}%
\end{pgfscope}%
\begin{pgfscope}%
\pgfpathrectangle{\pgfqpoint{3.352233in}{5.105882in}}{\pgfqpoint{2.407767in}{1.544118in}}%
\pgfusepath{clip}%
\pgfsetbuttcap%
\pgfsetroundjoin%
\pgfsetlinewidth{0.501875pt}%
\definecolor{currentstroke}{rgb}{0.121148,0.592739,0.544641}%
\pgfsetstrokecolor{currentstroke}%
\pgfsetdash{}{0pt}%
\pgfpathmoveto{\pgfqpoint{4.623552in}{5.876962in}}%
\pgfpathlineto{\pgfqpoint{4.570578in}{5.876823in}}%
\pgfusepath{stroke}%
\end{pgfscope}%
\begin{pgfscope}%
\pgfpathrectangle{\pgfqpoint{3.352233in}{5.105882in}}{\pgfqpoint{2.407767in}{1.544118in}}%
\pgfusepath{clip}%
\pgfsetbuttcap%
\pgfsetroundjoin%
\pgfsetlinewidth{0.501875pt}%
\definecolor{currentstroke}{rgb}{0.124780,0.640461,0.527068}%
\pgfsetstrokecolor{currentstroke}%
\pgfsetdash{}{0pt}%
\pgfpathmoveto{\pgfqpoint{4.570578in}{5.876823in}}%
\pgfpathlineto{\pgfqpoint{4.517604in}{5.876718in}}%
\pgfusepath{stroke}%
\end{pgfscope}%
\begin{pgfscope}%
\pgfpathrectangle{\pgfqpoint{3.352233in}{5.105882in}}{\pgfqpoint{2.407767in}{1.544118in}}%
\pgfusepath{clip}%
\pgfsetbuttcap%
\pgfsetroundjoin%
\pgfsetlinewidth{0.501875pt}%
\definecolor{currentstroke}{rgb}{0.191090,0.708366,0.482284}%
\pgfsetstrokecolor{currentstroke}%
\pgfsetdash{}{0pt}%
\pgfpathmoveto{\pgfqpoint{4.517604in}{5.876718in}}%
\pgfpathlineto{\pgfqpoint{4.464632in}{5.876605in}}%
\pgfusepath{stroke}%
\end{pgfscope}%
\begin{pgfscope}%
\pgfpathrectangle{\pgfqpoint{3.352233in}{5.105882in}}{\pgfqpoint{2.407767in}{1.544118in}}%
\pgfusepath{clip}%
\pgfsetbuttcap%
\pgfsetroundjoin%
\pgfsetlinewidth{0.501875pt}%
\definecolor{currentstroke}{rgb}{0.246070,0.738910,0.452024}%
\pgfsetstrokecolor{currentstroke}%
\pgfsetdash{}{0pt}%
\pgfpathmoveto{\pgfqpoint{4.464632in}{5.876605in}}%
\pgfpathlineto{\pgfqpoint{4.411660in}{5.876476in}}%
\pgfusepath{stroke}%
\end{pgfscope}%
\begin{pgfscope}%
\pgfpathrectangle{\pgfqpoint{3.352233in}{5.105882in}}{\pgfqpoint{2.407767in}{1.544118in}}%
\pgfusepath{clip}%
\pgfsetbuttcap%
\pgfsetroundjoin%
\pgfsetlinewidth{0.501875pt}%
\definecolor{currentstroke}{rgb}{0.386433,0.794644,0.372886}%
\pgfsetstrokecolor{currentstroke}%
\pgfsetdash{}{0pt}%
\pgfpathmoveto{\pgfqpoint{4.411660in}{5.876476in}}%
\pgfpathlineto{\pgfqpoint{4.358691in}{5.876342in}}%
\pgfusepath{stroke}%
\end{pgfscope}%
\begin{pgfscope}%
\pgfpathrectangle{\pgfqpoint{3.352233in}{5.105882in}}{\pgfqpoint{2.407767in}{1.544118in}}%
\pgfusepath{clip}%
\pgfsetbuttcap%
\pgfsetroundjoin%
\pgfsetlinewidth{0.501875pt}%
\definecolor{currentstroke}{rgb}{0.278791,0.062145,0.386592}%
\pgfsetstrokecolor{currentstroke}%
\pgfsetdash{}{0pt}%
\pgfpathmoveto{\pgfqpoint{5.206279in}{5.912687in}}%
\pgfpathlineto{\pgfqpoint{5.153303in}{5.912686in}}%
\pgfusepath{stroke}%
\end{pgfscope}%
\begin{pgfscope}%
\pgfpathrectangle{\pgfqpoint{3.352233in}{5.105882in}}{\pgfqpoint{2.407767in}{1.544118in}}%
\pgfusepath{clip}%
\pgfsetbuttcap%
\pgfsetroundjoin%
\pgfsetlinewidth{0.501875pt}%
\definecolor{currentstroke}{rgb}{0.283187,0.125848,0.444960}%
\pgfsetstrokecolor{currentstroke}%
\pgfsetdash{}{0pt}%
\pgfpathmoveto{\pgfqpoint{5.153303in}{5.912686in}}%
\pgfpathlineto{\pgfqpoint{5.100328in}{5.912587in}}%
\pgfusepath{stroke}%
\end{pgfscope}%
\begin{pgfscope}%
\pgfpathrectangle{\pgfqpoint{3.352233in}{5.105882in}}{\pgfqpoint{2.407767in}{1.544118in}}%
\pgfusepath{clip}%
\pgfsetbuttcap%
\pgfsetroundjoin%
\pgfsetlinewidth{0.501875pt}%
\definecolor{currentstroke}{rgb}{0.277134,0.185228,0.489898}%
\pgfsetstrokecolor{currentstroke}%
\pgfsetdash{}{0pt}%
\pgfpathmoveto{\pgfqpoint{5.100328in}{5.912587in}}%
\pgfpathlineto{\pgfqpoint{5.047354in}{5.912414in}}%
\pgfusepath{stroke}%
\end{pgfscope}%
\begin{pgfscope}%
\pgfpathrectangle{\pgfqpoint{3.352233in}{5.105882in}}{\pgfqpoint{2.407767in}{1.544118in}}%
\pgfusepath{clip}%
\pgfsetbuttcap%
\pgfsetroundjoin%
\pgfsetlinewidth{0.501875pt}%
\definecolor{currentstroke}{rgb}{0.260571,0.246922,0.522828}%
\pgfsetstrokecolor{currentstroke}%
\pgfsetdash{}{0pt}%
\pgfpathmoveto{\pgfqpoint{5.047354in}{5.912414in}}%
\pgfpathlineto{\pgfqpoint{4.994380in}{5.912092in}}%
\pgfusepath{stroke}%
\end{pgfscope}%
\begin{pgfscope}%
\pgfpathrectangle{\pgfqpoint{3.352233in}{5.105882in}}{\pgfqpoint{2.407767in}{1.544118in}}%
\pgfusepath{clip}%
\pgfsetbuttcap%
\pgfsetroundjoin%
\pgfsetlinewidth{0.501875pt}%
\definecolor{currentstroke}{rgb}{0.243113,0.292092,0.538516}%
\pgfsetstrokecolor{currentstroke}%
\pgfsetdash{}{0pt}%
\pgfpathmoveto{\pgfqpoint{4.994380in}{5.912092in}}%
\pgfpathlineto{\pgfqpoint{4.941409in}{5.911638in}}%
\pgfusepath{stroke}%
\end{pgfscope}%
\begin{pgfscope}%
\pgfpathrectangle{\pgfqpoint{3.352233in}{5.105882in}}{\pgfqpoint{2.407767in}{1.544118in}}%
\pgfusepath{clip}%
\pgfsetbuttcap%
\pgfsetroundjoin%
\pgfsetlinewidth{0.501875pt}%
\definecolor{currentstroke}{rgb}{0.221989,0.339161,0.548752}%
\pgfsetstrokecolor{currentstroke}%
\pgfsetdash{}{0pt}%
\pgfpathmoveto{\pgfqpoint{4.941409in}{5.911638in}}%
\pgfpathlineto{\pgfqpoint{4.888441in}{5.911033in}}%
\pgfusepath{stroke}%
\end{pgfscope}%
\begin{pgfscope}%
\pgfpathrectangle{\pgfqpoint{3.352233in}{5.105882in}}{\pgfqpoint{2.407767in}{1.544118in}}%
\pgfusepath{clip}%
\pgfsetbuttcap%
\pgfsetroundjoin%
\pgfsetlinewidth{0.501875pt}%
\definecolor{currentstroke}{rgb}{0.195860,0.395433,0.555276}%
\pgfsetstrokecolor{currentstroke}%
\pgfsetdash{}{0pt}%
\pgfpathmoveto{\pgfqpoint{4.888441in}{5.911033in}}%
\pgfpathlineto{\pgfqpoint{4.835478in}{5.910282in}}%
\pgfusepath{stroke}%
\end{pgfscope}%
\begin{pgfscope}%
\pgfpathrectangle{\pgfqpoint{3.352233in}{5.105882in}}{\pgfqpoint{2.407767in}{1.544118in}}%
\pgfusepath{clip}%
\pgfsetbuttcap%
\pgfsetroundjoin%
\pgfsetlinewidth{0.501875pt}%
\definecolor{currentstroke}{rgb}{0.188923,0.410910,0.556326}%
\pgfsetstrokecolor{currentstroke}%
\pgfsetdash{}{0pt}%
\pgfpathmoveto{\pgfqpoint{4.835478in}{5.910282in}}%
\pgfpathlineto{\pgfqpoint{4.782521in}{5.909403in}}%
\pgfusepath{stroke}%
\end{pgfscope}%
\begin{pgfscope}%
\pgfpathrectangle{\pgfqpoint{3.352233in}{5.105882in}}{\pgfqpoint{2.407767in}{1.544118in}}%
\pgfusepath{clip}%
\pgfsetbuttcap%
\pgfsetroundjoin%
\pgfsetlinewidth{0.501875pt}%
\definecolor{currentstroke}{rgb}{0.174274,0.445044,0.557792}%
\pgfsetstrokecolor{currentstroke}%
\pgfsetdash{}{0pt}%
\pgfpathmoveto{\pgfqpoint{4.782521in}{5.909403in}}%
\pgfpathlineto{\pgfqpoint{4.729569in}{5.908377in}}%
\pgfusepath{stroke}%
\end{pgfscope}%
\begin{pgfscope}%
\pgfpathrectangle{\pgfqpoint{3.352233in}{5.105882in}}{\pgfqpoint{2.407767in}{1.544118in}}%
\pgfusepath{clip}%
\pgfsetbuttcap%
\pgfsetroundjoin%
\pgfsetlinewidth{0.501875pt}%
\definecolor{currentstroke}{rgb}{0.162142,0.474838,0.558140}%
\pgfsetstrokecolor{currentstroke}%
\pgfsetdash{}{0pt}%
\pgfpathmoveto{\pgfqpoint{4.729569in}{5.908377in}}%
\pgfpathlineto{\pgfqpoint{4.676626in}{5.907180in}}%
\pgfusepath{stroke}%
\end{pgfscope}%
\begin{pgfscope}%
\pgfpathrectangle{\pgfqpoint{3.352233in}{5.105882in}}{\pgfqpoint{2.407767in}{1.544118in}}%
\pgfusepath{clip}%
\pgfsetbuttcap%
\pgfsetroundjoin%
\pgfsetlinewidth{0.501875pt}%
\definecolor{currentstroke}{rgb}{0.136408,0.541173,0.554483}%
\pgfsetstrokecolor{currentstroke}%
\pgfsetdash{}{0pt}%
\pgfpathmoveto{\pgfqpoint{4.676626in}{5.907180in}}%
\pgfpathlineto{\pgfqpoint{4.623702in}{5.905716in}}%
\pgfusepath{stroke}%
\end{pgfscope}%
\begin{pgfscope}%
\pgfpathrectangle{\pgfqpoint{3.352233in}{5.105882in}}{\pgfqpoint{2.407767in}{1.544118in}}%
\pgfusepath{clip}%
\pgfsetbuttcap%
\pgfsetroundjoin%
\pgfsetlinewidth{0.501875pt}%
\definecolor{currentstroke}{rgb}{0.132444,0.552216,0.553018}%
\pgfsetstrokecolor{currentstroke}%
\pgfsetdash{}{0pt}%
\pgfpathmoveto{\pgfqpoint{4.623702in}{5.905716in}}%
\pgfpathlineto{\pgfqpoint{4.570791in}{5.904067in}}%
\pgfusepath{stroke}%
\end{pgfscope}%
\begin{pgfscope}%
\pgfpathrectangle{\pgfqpoint{3.352233in}{5.105882in}}{\pgfqpoint{2.407767in}{1.544118in}}%
\pgfusepath{clip}%
\pgfsetbuttcap%
\pgfsetroundjoin%
\pgfsetlinewidth{0.501875pt}%
\definecolor{currentstroke}{rgb}{0.120092,0.600104,0.542530}%
\pgfsetstrokecolor{currentstroke}%
\pgfsetdash{}{0pt}%
\pgfpathmoveto{\pgfqpoint{4.570791in}{5.904067in}}%
\pgfpathlineto{\pgfqpoint{4.517897in}{5.902226in}}%
\pgfusepath{stroke}%
\end{pgfscope}%
\begin{pgfscope}%
\pgfpathrectangle{\pgfqpoint{3.352233in}{5.105882in}}{\pgfqpoint{2.407767in}{1.544118in}}%
\pgfusepath{clip}%
\pgfsetbuttcap%
\pgfsetroundjoin%
\pgfsetlinewidth{0.501875pt}%
\definecolor{currentstroke}{rgb}{0.140210,0.665859,0.513427}%
\pgfsetstrokecolor{currentstroke}%
\pgfsetdash{}{0pt}%
\pgfpathmoveto{\pgfqpoint{4.517897in}{5.902226in}}%
\pgfpathlineto{\pgfqpoint{4.465018in}{5.900225in}}%
\pgfusepath{stroke}%
\end{pgfscope}%
\begin{pgfscope}%
\pgfpathrectangle{\pgfqpoint{3.352233in}{5.105882in}}{\pgfqpoint{2.407767in}{1.544118in}}%
\pgfusepath{clip}%
\pgfsetbuttcap%
\pgfsetroundjoin%
\pgfsetlinewidth{0.501875pt}%
\definecolor{currentstroke}{rgb}{0.208030,0.718701,0.472873}%
\pgfsetstrokecolor{currentstroke}%
\pgfsetdash{}{0pt}%
\pgfpathmoveto{\pgfqpoint{4.465018in}{5.900225in}}%
\pgfpathlineto{\pgfqpoint{4.412156in}{5.898087in}}%
\pgfusepath{stroke}%
\end{pgfscope}%
\begin{pgfscope}%
\pgfpathrectangle{\pgfqpoint{3.352233in}{5.105882in}}{\pgfqpoint{2.407767in}{1.544118in}}%
\pgfusepath{clip}%
\pgfsetbuttcap%
\pgfsetroundjoin%
\pgfsetlinewidth{0.501875pt}%
\definecolor{currentstroke}{rgb}{0.281477,0.755203,0.432552}%
\pgfsetstrokecolor{currentstroke}%
\pgfsetdash{}{0pt}%
\pgfpathmoveto{\pgfqpoint{4.412156in}{5.898087in}}%
\pgfpathlineto{\pgfqpoint{4.359306in}{5.895845in}}%
\pgfusepath{stroke}%
\end{pgfscope}%
\begin{pgfscope}%
\pgfpathrectangle{\pgfqpoint{3.352233in}{5.105882in}}{\pgfqpoint{2.407767in}{1.544118in}}%
\pgfusepath{clip}%
\pgfsetbuttcap%
\pgfsetroundjoin%
\pgfsetlinewidth{0.501875pt}%
\definecolor{currentstroke}{rgb}{0.278791,0.062145,0.386592}%
\pgfsetstrokecolor{currentstroke}%
\pgfsetdash{}{0pt}%
\pgfpathmoveto{\pgfqpoint{5.206279in}{5.947433in}}%
\pgfpathlineto{\pgfqpoint{5.153304in}{5.947205in}}%
\pgfusepath{stroke}%
\end{pgfscope}%
\begin{pgfscope}%
\pgfpathrectangle{\pgfqpoint{3.352233in}{5.105882in}}{\pgfqpoint{2.407767in}{1.544118in}}%
\pgfusepath{clip}%
\pgfsetbuttcap%
\pgfsetroundjoin%
\pgfsetlinewidth{0.501875pt}%
\definecolor{currentstroke}{rgb}{0.283187,0.125848,0.444960}%
\pgfsetstrokecolor{currentstroke}%
\pgfsetdash{}{0pt}%
\pgfpathmoveto{\pgfqpoint{5.153304in}{5.947205in}}%
\pgfpathlineto{\pgfqpoint{5.100330in}{5.946940in}}%
\pgfusepath{stroke}%
\end{pgfscope}%
\begin{pgfscope}%
\pgfpathrectangle{\pgfqpoint{3.352233in}{5.105882in}}{\pgfqpoint{2.407767in}{1.544118in}}%
\pgfusepath{clip}%
\pgfsetbuttcap%
\pgfsetroundjoin%
\pgfsetlinewidth{0.501875pt}%
\definecolor{currentstroke}{rgb}{0.276194,0.190074,0.493001}%
\pgfsetstrokecolor{currentstroke}%
\pgfsetdash{}{0pt}%
\pgfpathmoveto{\pgfqpoint{5.100330in}{5.946940in}}%
\pgfpathlineto{\pgfqpoint{5.047358in}{5.946530in}}%
\pgfusepath{stroke}%
\end{pgfscope}%
\begin{pgfscope}%
\pgfpathrectangle{\pgfqpoint{3.352233in}{5.105882in}}{\pgfqpoint{2.407767in}{1.544118in}}%
\pgfusepath{clip}%
\pgfsetbuttcap%
\pgfsetroundjoin%
\pgfsetlinewidth{0.501875pt}%
\definecolor{currentstroke}{rgb}{0.267968,0.223549,0.512008}%
\pgfsetstrokecolor{currentstroke}%
\pgfsetdash{}{0pt}%
\pgfpathmoveto{\pgfqpoint{5.047358in}{5.946530in}}%
\pgfpathlineto{\pgfqpoint{4.994388in}{5.946047in}}%
\pgfusepath{stroke}%
\end{pgfscope}%
\begin{pgfscope}%
\pgfpathrectangle{\pgfqpoint{3.352233in}{5.105882in}}{\pgfqpoint{2.407767in}{1.544118in}}%
\pgfusepath{clip}%
\pgfsetbuttcap%
\pgfsetroundjoin%
\pgfsetlinewidth{0.501875pt}%
\definecolor{currentstroke}{rgb}{0.243113,0.292092,0.538516}%
\pgfsetstrokecolor{currentstroke}%
\pgfsetdash{}{0pt}%
\pgfpathmoveto{\pgfqpoint{4.994388in}{5.946047in}}%
\pgfpathlineto{\pgfqpoint{4.941420in}{5.945479in}}%
\pgfusepath{stroke}%
\end{pgfscope}%
\begin{pgfscope}%
\pgfpathrectangle{\pgfqpoint{3.352233in}{5.105882in}}{\pgfqpoint{2.407767in}{1.544118in}}%
\pgfusepath{clip}%
\pgfsetbuttcap%
\pgfsetroundjoin%
\pgfsetlinewidth{0.501875pt}%
\definecolor{currentstroke}{rgb}{0.223925,0.334994,0.548053}%
\pgfsetstrokecolor{currentstroke}%
\pgfsetdash{}{0pt}%
\pgfpathmoveto{\pgfqpoint{4.941420in}{5.945479in}}%
\pgfpathlineto{\pgfqpoint{4.888458in}{5.944698in}}%
\pgfusepath{stroke}%
\end{pgfscope}%
\begin{pgfscope}%
\pgfpathrectangle{\pgfqpoint{3.352233in}{5.105882in}}{\pgfqpoint{2.407767in}{1.544118in}}%
\pgfusepath{clip}%
\pgfsetbuttcap%
\pgfsetroundjoin%
\pgfsetlinewidth{0.501875pt}%
\definecolor{currentstroke}{rgb}{0.201239,0.383670,0.554294}%
\pgfsetstrokecolor{currentstroke}%
\pgfsetdash{}{0pt}%
\pgfpathmoveto{\pgfqpoint{4.888458in}{5.944698in}}%
\pgfpathlineto{\pgfqpoint{4.835507in}{5.943675in}}%
\pgfusepath{stroke}%
\end{pgfscope}%
\begin{pgfscope}%
\pgfpathrectangle{\pgfqpoint{3.352233in}{5.105882in}}{\pgfqpoint{2.407767in}{1.544118in}}%
\pgfusepath{clip}%
\pgfsetbuttcap%
\pgfsetroundjoin%
\pgfsetlinewidth{0.501875pt}%
\definecolor{currentstroke}{rgb}{0.192357,0.403199,0.555836}%
\pgfsetstrokecolor{currentstroke}%
\pgfsetdash{}{0pt}%
\pgfpathmoveto{\pgfqpoint{4.835507in}{5.943675in}}%
\pgfpathlineto{\pgfqpoint{4.782572in}{5.942349in}}%
\pgfusepath{stroke}%
\end{pgfscope}%
\begin{pgfscope}%
\pgfpathrectangle{\pgfqpoint{3.352233in}{5.105882in}}{\pgfqpoint{2.407767in}{1.544118in}}%
\pgfusepath{clip}%
\pgfsetbuttcap%
\pgfsetroundjoin%
\pgfsetlinewidth{0.501875pt}%
\definecolor{currentstroke}{rgb}{0.177423,0.437527,0.557565}%
\pgfsetstrokecolor{currentstroke}%
\pgfsetdash{}{0pt}%
\pgfpathmoveto{\pgfqpoint{4.782572in}{5.942349in}}%
\pgfpathlineto{\pgfqpoint{4.729656in}{5.940747in}}%
\pgfusepath{stroke}%
\end{pgfscope}%
\begin{pgfscope}%
\pgfpathrectangle{\pgfqpoint{3.352233in}{5.105882in}}{\pgfqpoint{2.407767in}{1.544118in}}%
\pgfusepath{clip}%
\pgfsetbuttcap%
\pgfsetroundjoin%
\pgfsetlinewidth{0.501875pt}%
\definecolor{currentstroke}{rgb}{0.162142,0.474838,0.558140}%
\pgfsetstrokecolor{currentstroke}%
\pgfsetdash{}{0pt}%
\pgfpathmoveto{\pgfqpoint{4.729656in}{5.940747in}}%
\pgfpathlineto{\pgfqpoint{4.676764in}{5.938845in}}%
\pgfusepath{stroke}%
\end{pgfscope}%
\begin{pgfscope}%
\pgfpathrectangle{\pgfqpoint{3.352233in}{5.105882in}}{\pgfqpoint{2.407767in}{1.544118in}}%
\pgfusepath{clip}%
\pgfsetbuttcap%
\pgfsetroundjoin%
\pgfsetlinewidth{0.501875pt}%
\definecolor{currentstroke}{rgb}{0.151918,0.500685,0.557587}%
\pgfsetstrokecolor{currentstroke}%
\pgfsetdash{}{0pt}%
\pgfpathmoveto{\pgfqpoint{4.676764in}{5.938845in}}%
\pgfpathlineto{\pgfqpoint{4.623900in}{5.936641in}}%
\pgfusepath{stroke}%
\end{pgfscope}%
\begin{pgfscope}%
\pgfpathrectangle{\pgfqpoint{3.352233in}{5.105882in}}{\pgfqpoint{2.407767in}{1.544118in}}%
\pgfusepath{clip}%
\pgfsetbuttcap%
\pgfsetroundjoin%
\pgfsetlinewidth{0.501875pt}%
\definecolor{currentstroke}{rgb}{0.149039,0.508051,0.557250}%
\pgfsetstrokecolor{currentstroke}%
\pgfsetdash{}{0pt}%
\pgfpathmoveto{\pgfqpoint{4.623900in}{5.936641in}}%
\pgfpathlineto{\pgfqpoint{4.571087in}{5.934018in}}%
\pgfusepath{stroke}%
\end{pgfscope}%
\begin{pgfscope}%
\pgfpathrectangle{\pgfqpoint{3.352233in}{5.105882in}}{\pgfqpoint{2.407767in}{1.544118in}}%
\pgfusepath{clip}%
\pgfsetbuttcap%
\pgfsetroundjoin%
\pgfsetlinewidth{0.501875pt}%
\definecolor{currentstroke}{rgb}{0.140536,0.530132,0.555659}%
\pgfsetstrokecolor{currentstroke}%
\pgfsetdash{}{0pt}%
\pgfpathmoveto{\pgfqpoint{4.571087in}{5.934018in}}%
\pgfpathlineto{\pgfqpoint{4.518355in}{5.930800in}}%
\pgfusepath{stroke}%
\end{pgfscope}%
\begin{pgfscope}%
\pgfpathrectangle{\pgfqpoint{3.352233in}{5.105882in}}{\pgfqpoint{2.407767in}{1.544118in}}%
\pgfusepath{clip}%
\pgfsetbuttcap%
\pgfsetroundjoin%
\pgfsetlinewidth{0.501875pt}%
\definecolor{currentstroke}{rgb}{0.277941,0.056324,0.381191}%
\pgfsetstrokecolor{currentstroke}%
\pgfsetdash{}{0pt}%
\pgfpathmoveto{\pgfqpoint{5.206279in}{5.982180in}}%
\pgfpathlineto{\pgfqpoint{5.153306in}{5.981942in}}%
\pgfusepath{stroke}%
\end{pgfscope}%
\begin{pgfscope}%
\pgfpathrectangle{\pgfqpoint{3.352233in}{5.105882in}}{\pgfqpoint{2.407767in}{1.544118in}}%
\pgfusepath{clip}%
\pgfsetbuttcap%
\pgfsetroundjoin%
\pgfsetlinewidth{0.501875pt}%
\definecolor{currentstroke}{rgb}{0.283091,0.110553,0.431554}%
\pgfsetstrokecolor{currentstroke}%
\pgfsetdash{}{0pt}%
\pgfpathmoveto{\pgfqpoint{5.153306in}{5.981942in}}%
\pgfpathlineto{\pgfqpoint{5.100343in}{5.981251in}}%
\pgfusepath{stroke}%
\end{pgfscope}%
\begin{pgfscope}%
\pgfpathrectangle{\pgfqpoint{3.352233in}{5.105882in}}{\pgfqpoint{2.407767in}{1.544118in}}%
\pgfusepath{clip}%
\pgfsetbuttcap%
\pgfsetroundjoin%
\pgfsetlinewidth{0.501875pt}%
\definecolor{currentstroke}{rgb}{0.279574,0.170599,0.479997}%
\pgfsetstrokecolor{currentstroke}%
\pgfsetdash{}{0pt}%
\pgfpathmoveto{\pgfqpoint{5.100343in}{5.981251in}}%
\pgfpathlineto{\pgfqpoint{5.047385in}{5.980348in}}%
\pgfusepath{stroke}%
\end{pgfscope}%
\begin{pgfscope}%
\pgfpathrectangle{\pgfqpoint{3.352233in}{5.105882in}}{\pgfqpoint{2.407767in}{1.544118in}}%
\pgfusepath{clip}%
\pgfsetbuttcap%
\pgfsetroundjoin%
\pgfsetlinewidth{0.501875pt}%
\definecolor{currentstroke}{rgb}{0.270595,0.214069,0.507052}%
\pgfsetstrokecolor{currentstroke}%
\pgfsetdash{}{0pt}%
\pgfpathmoveto{\pgfqpoint{5.047385in}{5.980348in}}%
\pgfpathlineto{\pgfqpoint{4.994431in}{5.979373in}}%
\pgfusepath{stroke}%
\end{pgfscope}%
\begin{pgfscope}%
\pgfpathrectangle{\pgfqpoint{3.352233in}{5.105882in}}{\pgfqpoint{2.407767in}{1.544118in}}%
\pgfusepath{clip}%
\pgfsetbuttcap%
\pgfsetroundjoin%
\pgfsetlinewidth{0.501875pt}%
\definecolor{currentstroke}{rgb}{0.253935,0.265254,0.529983}%
\pgfsetstrokecolor{currentstroke}%
\pgfsetdash{}{0pt}%
\pgfpathmoveto{\pgfqpoint{4.994431in}{5.979373in}}%
\pgfpathlineto{\pgfqpoint{4.941488in}{5.978203in}}%
\pgfusepath{stroke}%
\end{pgfscope}%
\begin{pgfscope}%
\pgfpathrectangle{\pgfqpoint{3.352233in}{5.105882in}}{\pgfqpoint{2.407767in}{1.544118in}}%
\pgfusepath{clip}%
\pgfsetbuttcap%
\pgfsetroundjoin%
\pgfsetlinewidth{0.501875pt}%
\definecolor{currentstroke}{rgb}{0.235526,0.309527,0.542944}%
\pgfsetstrokecolor{currentstroke}%
\pgfsetdash{}{0pt}%
\pgfpathmoveto{\pgfqpoint{4.941488in}{5.978203in}}%
\pgfpathlineto{\pgfqpoint{4.888563in}{5.976723in}}%
\pgfusepath{stroke}%
\end{pgfscope}%
\begin{pgfscope}%
\pgfpathrectangle{\pgfqpoint{3.352233in}{5.105882in}}{\pgfqpoint{2.407767in}{1.544118in}}%
\pgfusepath{clip}%
\pgfsetbuttcap%
\pgfsetroundjoin%
\pgfsetlinewidth{0.501875pt}%
\definecolor{currentstroke}{rgb}{0.221989,0.339161,0.548752}%
\pgfsetstrokecolor{currentstroke}%
\pgfsetdash{}{0pt}%
\pgfpathmoveto{\pgfqpoint{4.888563in}{5.976723in}}%
\pgfpathlineto{\pgfqpoint{4.835667in}{5.974880in}}%
\pgfusepath{stroke}%
\end{pgfscope}%
\begin{pgfscope}%
\pgfpathrectangle{\pgfqpoint{3.352233in}{5.105882in}}{\pgfqpoint{2.407767in}{1.544118in}}%
\pgfusepath{clip}%
\pgfsetbuttcap%
\pgfsetroundjoin%
\pgfsetlinewidth{0.501875pt}%
\definecolor{currentstroke}{rgb}{0.203063,0.379716,0.553925}%
\pgfsetstrokecolor{currentstroke}%
\pgfsetdash{}{0pt}%
\pgfpathmoveto{\pgfqpoint{4.835667in}{5.974880in}}%
\pgfpathlineto{\pgfqpoint{4.782813in}{5.972597in}}%
\pgfusepath{stroke}%
\end{pgfscope}%
\begin{pgfscope}%
\pgfpathrectangle{\pgfqpoint{3.352233in}{5.105882in}}{\pgfqpoint{2.407767in}{1.544118in}}%
\pgfusepath{clip}%
\pgfsetbuttcap%
\pgfsetroundjoin%
\pgfsetlinewidth{0.501875pt}%
\definecolor{currentstroke}{rgb}{0.192357,0.403199,0.555836}%
\pgfsetstrokecolor{currentstroke}%
\pgfsetdash{}{0pt}%
\pgfpathmoveto{\pgfqpoint{4.782813in}{5.972597in}}%
\pgfpathlineto{\pgfqpoint{4.729998in}{5.969961in}}%
\pgfusepath{stroke}%
\end{pgfscope}%
\begin{pgfscope}%
\pgfpathrectangle{\pgfqpoint{3.352233in}{5.105882in}}{\pgfqpoint{2.407767in}{1.544118in}}%
\pgfusepath{clip}%
\pgfsetbuttcap%
\pgfsetroundjoin%
\pgfsetlinewidth{0.501875pt}%
\definecolor{currentstroke}{rgb}{0.180629,0.429975,0.557282}%
\pgfsetstrokecolor{currentstroke}%
\pgfsetdash{}{0pt}%
\pgfpathmoveto{\pgfqpoint{4.729998in}{5.969961in}}%
\pgfpathlineto{\pgfqpoint{4.677247in}{5.966864in}}%
\pgfusepath{stroke}%
\end{pgfscope}%
\begin{pgfscope}%
\pgfpathrectangle{\pgfqpoint{3.352233in}{5.105882in}}{\pgfqpoint{2.407767in}{1.544118in}}%
\pgfusepath{clip}%
\pgfsetbuttcap%
\pgfsetroundjoin%
\pgfsetlinewidth{0.501875pt}%
\definecolor{currentstroke}{rgb}{0.277018,0.050344,0.375715}%
\pgfsetstrokecolor{currentstroke}%
\pgfsetdash{}{0pt}%
\pgfpathmoveto{\pgfqpoint{5.206279in}{6.016926in}}%
\pgfpathlineto{\pgfqpoint{5.153305in}{6.016683in}}%
\pgfusepath{stroke}%
\end{pgfscope}%
\begin{pgfscope}%
\pgfpathrectangle{\pgfqpoint{3.352233in}{5.105882in}}{\pgfqpoint{2.407767in}{1.544118in}}%
\pgfusepath{clip}%
\pgfsetbuttcap%
\pgfsetroundjoin%
\pgfsetlinewidth{0.501875pt}%
\definecolor{currentstroke}{rgb}{0.282656,0.100196,0.422160}%
\pgfsetstrokecolor{currentstroke}%
\pgfsetdash{}{0pt}%
\pgfpathmoveto{\pgfqpoint{5.153305in}{6.016683in}}%
\pgfpathlineto{\pgfqpoint{5.100336in}{6.016274in}}%
\pgfusepath{stroke}%
\end{pgfscope}%
\begin{pgfscope}%
\pgfpathrectangle{\pgfqpoint{3.352233in}{5.105882in}}{\pgfqpoint{2.407767in}{1.544118in}}%
\pgfusepath{clip}%
\pgfsetbuttcap%
\pgfsetroundjoin%
\pgfsetlinewidth{0.501875pt}%
\definecolor{currentstroke}{rgb}{0.281412,0.155834,0.469201}%
\pgfsetstrokecolor{currentstroke}%
\pgfsetdash{}{0pt}%
\pgfpathmoveto{\pgfqpoint{5.100336in}{6.016274in}}%
\pgfpathlineto{\pgfqpoint{5.047385in}{6.015277in}}%
\pgfusepath{stroke}%
\end{pgfscope}%
\begin{pgfscope}%
\pgfpathrectangle{\pgfqpoint{3.352233in}{5.105882in}}{\pgfqpoint{2.407767in}{1.544118in}}%
\pgfusepath{clip}%
\pgfsetbuttcap%
\pgfsetroundjoin%
\pgfsetlinewidth{0.501875pt}%
\definecolor{currentstroke}{rgb}{0.270595,0.214069,0.507052}%
\pgfsetstrokecolor{currentstroke}%
\pgfsetdash{}{0pt}%
\pgfpathmoveto{\pgfqpoint{5.047385in}{6.015277in}}%
\pgfpathlineto{\pgfqpoint{4.994452in}{6.013912in}}%
\pgfusepath{stroke}%
\end{pgfscope}%
\begin{pgfscope}%
\pgfpathrectangle{\pgfqpoint{3.352233in}{5.105882in}}{\pgfqpoint{2.407767in}{1.544118in}}%
\pgfusepath{clip}%
\pgfsetbuttcap%
\pgfsetroundjoin%
\pgfsetlinewidth{0.501875pt}%
\definecolor{currentstroke}{rgb}{0.267968,0.223549,0.512008}%
\pgfsetstrokecolor{currentstroke}%
\pgfsetdash{}{0pt}%
\pgfpathmoveto{\pgfqpoint{4.994452in}{6.013912in}}%
\pgfpathlineto{\pgfqpoint{4.941553in}{6.012134in}}%
\pgfusepath{stroke}%
\end{pgfscope}%
\begin{pgfscope}%
\pgfpathrectangle{\pgfqpoint{3.352233in}{5.105882in}}{\pgfqpoint{2.407767in}{1.544118in}}%
\pgfusepath{clip}%
\pgfsetbuttcap%
\pgfsetroundjoin%
\pgfsetlinewidth{0.501875pt}%
\definecolor{currentstroke}{rgb}{0.253935,0.265254,0.529983}%
\pgfsetstrokecolor{currentstroke}%
\pgfsetdash{}{0pt}%
\pgfpathmoveto{\pgfqpoint{4.941553in}{6.012134in}}%
\pgfpathlineto{\pgfqpoint{4.888679in}{6.010037in}}%
\pgfusepath{stroke}%
\end{pgfscope}%
\begin{pgfscope}%
\pgfpathrectangle{\pgfqpoint{3.352233in}{5.105882in}}{\pgfqpoint{2.407767in}{1.544118in}}%
\pgfusepath{clip}%
\pgfsetbuttcap%
\pgfsetroundjoin%
\pgfsetlinewidth{0.501875pt}%
\definecolor{currentstroke}{rgb}{0.243113,0.292092,0.538516}%
\pgfsetstrokecolor{currentstroke}%
\pgfsetdash{}{0pt}%
\pgfpathmoveto{\pgfqpoint{4.888679in}{6.010037in}}%
\pgfpathlineto{\pgfqpoint{4.835846in}{6.007574in}}%
\pgfusepath{stroke}%
\end{pgfscope}%
\begin{pgfscope}%
\pgfpathrectangle{\pgfqpoint{3.352233in}{5.105882in}}{\pgfqpoint{2.407767in}{1.544118in}}%
\pgfusepath{clip}%
\pgfsetbuttcap%
\pgfsetroundjoin%
\pgfsetlinewidth{0.501875pt}%
\definecolor{currentstroke}{rgb}{0.223925,0.334994,0.548053}%
\pgfsetstrokecolor{currentstroke}%
\pgfsetdash{}{0pt}%
\pgfpathmoveto{\pgfqpoint{4.835846in}{6.007574in}}%
\pgfpathlineto{\pgfqpoint{4.783079in}{6.004571in}}%
\pgfusepath{stroke}%
\end{pgfscope}%
\begin{pgfscope}%
\pgfpathrectangle{\pgfqpoint{3.352233in}{5.105882in}}{\pgfqpoint{2.407767in}{1.544118in}}%
\pgfusepath{clip}%
\pgfsetbuttcap%
\pgfsetroundjoin%
\pgfsetlinewidth{0.501875pt}%
\definecolor{currentstroke}{rgb}{0.214298,0.355619,0.551184}%
\pgfsetstrokecolor{currentstroke}%
\pgfsetdash{}{0pt}%
\pgfpathmoveto{\pgfqpoint{4.783079in}{6.004571in}}%
\pgfpathlineto{\pgfqpoint{4.730391in}{6.001053in}}%
\pgfusepath{stroke}%
\end{pgfscope}%
\begin{pgfscope}%
\pgfpathrectangle{\pgfqpoint{3.352233in}{5.105882in}}{\pgfqpoint{2.407767in}{1.544118in}}%
\pgfusepath{clip}%
\pgfsetbuttcap%
\pgfsetroundjoin%
\pgfsetlinewidth{0.501875pt}%
\definecolor{currentstroke}{rgb}{0.276022,0.044167,0.370164}%
\pgfsetstrokecolor{currentstroke}%
\pgfsetdash{}{0pt}%
\pgfpathmoveto{\pgfqpoint{5.206279in}{6.051672in}}%
\pgfpathlineto{\pgfqpoint{5.153312in}{6.051197in}}%
\pgfusepath{stroke}%
\end{pgfscope}%
\begin{pgfscope}%
\pgfpathrectangle{\pgfqpoint{3.352233in}{5.105882in}}{\pgfqpoint{2.407767in}{1.544118in}}%
\pgfusepath{clip}%
\pgfsetbuttcap%
\pgfsetroundjoin%
\pgfsetlinewidth{0.501875pt}%
\definecolor{currentstroke}{rgb}{0.282327,0.094955,0.417331}%
\pgfsetstrokecolor{currentstroke}%
\pgfsetdash{}{0pt}%
\pgfpathmoveto{\pgfqpoint{5.153312in}{6.051197in}}%
\pgfpathlineto{\pgfqpoint{5.100359in}{6.050211in}}%
\pgfusepath{stroke}%
\end{pgfscope}%
\begin{pgfscope}%
\pgfpathrectangle{\pgfqpoint{3.352233in}{5.105882in}}{\pgfqpoint{2.407767in}{1.544118in}}%
\pgfusepath{clip}%
\pgfsetbuttcap%
\pgfsetroundjoin%
\pgfsetlinewidth{0.501875pt}%
\definecolor{currentstroke}{rgb}{0.282623,0.140926,0.457517}%
\pgfsetstrokecolor{currentstroke}%
\pgfsetdash{}{0pt}%
\pgfpathmoveto{\pgfqpoint{5.100359in}{6.050211in}}%
\pgfpathlineto{\pgfqpoint{5.047425in}{6.048889in}}%
\pgfusepath{stroke}%
\end{pgfscope}%
\begin{pgfscope}%
\pgfpathrectangle{\pgfqpoint{3.352233in}{5.105882in}}{\pgfqpoint{2.407767in}{1.544118in}}%
\pgfusepath{clip}%
\pgfsetbuttcap%
\pgfsetroundjoin%
\pgfsetlinewidth{0.501875pt}%
\definecolor{currentstroke}{rgb}{0.278012,0.180367,0.486697}%
\pgfsetstrokecolor{currentstroke}%
\pgfsetdash{}{0pt}%
\pgfpathmoveto{\pgfqpoint{5.047425in}{6.048889in}}%
\pgfpathlineto{\pgfqpoint{4.994513in}{6.047221in}}%
\pgfusepath{stroke}%
\end{pgfscope}%
\begin{pgfscope}%
\pgfpathrectangle{\pgfqpoint{3.352233in}{5.105882in}}{\pgfqpoint{2.407767in}{1.544118in}}%
\pgfusepath{clip}%
\pgfsetbuttcap%
\pgfsetroundjoin%
\pgfsetlinewidth{0.501875pt}%
\definecolor{currentstroke}{rgb}{0.267968,0.223549,0.512008}%
\pgfsetstrokecolor{currentstroke}%
\pgfsetdash{}{0pt}%
\pgfpathmoveto{\pgfqpoint{4.994513in}{6.047221in}}%
\pgfpathlineto{\pgfqpoint{4.941635in}{6.045166in}}%
\pgfusepath{stroke}%
\end{pgfscope}%
\begin{pgfscope}%
\pgfpathrectangle{\pgfqpoint{3.352233in}{5.105882in}}{\pgfqpoint{2.407767in}{1.544118in}}%
\pgfusepath{clip}%
\pgfsetbuttcap%
\pgfsetroundjoin%
\pgfsetlinewidth{0.501875pt}%
\definecolor{currentstroke}{rgb}{0.258965,0.251537,0.524736}%
\pgfsetstrokecolor{currentstroke}%
\pgfsetdash{}{0pt}%
\pgfpathmoveto{\pgfqpoint{4.941635in}{6.045166in}}%
\pgfpathlineto{\pgfqpoint{4.888815in}{6.042581in}}%
\pgfusepath{stroke}%
\end{pgfscope}%
\begin{pgfscope}%
\pgfpathrectangle{\pgfqpoint{3.352233in}{5.105882in}}{\pgfqpoint{2.407767in}{1.544118in}}%
\pgfusepath{clip}%
\pgfsetbuttcap%
\pgfsetroundjoin%
\pgfsetlinewidth{0.501875pt}%
\definecolor{currentstroke}{rgb}{0.252194,0.269783,0.531579}%
\pgfsetstrokecolor{currentstroke}%
\pgfsetdash{}{0pt}%
\pgfpathmoveto{\pgfqpoint{4.888815in}{6.042581in}}%
\pgfpathlineto{\pgfqpoint{4.836057in}{6.039511in}}%
\pgfusepath{stroke}%
\end{pgfscope}%
\begin{pgfscope}%
\pgfpathrectangle{\pgfqpoint{3.352233in}{5.105882in}}{\pgfqpoint{2.407767in}{1.544118in}}%
\pgfusepath{clip}%
\pgfsetbuttcap%
\pgfsetroundjoin%
\pgfsetlinewidth{0.501875pt}%
\definecolor{currentstroke}{rgb}{0.248629,0.278775,0.534556}%
\pgfsetstrokecolor{currentstroke}%
\pgfsetdash{}{0pt}%
\pgfpathmoveto{\pgfqpoint{4.836057in}{6.039511in}}%
\pgfpathlineto{\pgfqpoint{4.783389in}{6.035882in}}%
\pgfusepath{stroke}%
\end{pgfscope}%
\begin{pgfscope}%
\pgfpathrectangle{\pgfqpoint{3.352233in}{5.105882in}}{\pgfqpoint{2.407767in}{1.544118in}}%
\pgfusepath{clip}%
\pgfsetbuttcap%
\pgfsetroundjoin%
\pgfsetlinewidth{0.501875pt}%
\definecolor{currentstroke}{rgb}{0.274952,0.037752,0.364543}%
\pgfsetstrokecolor{currentstroke}%
\pgfsetdash{}{0pt}%
\pgfpathmoveto{\pgfqpoint{5.206279in}{6.086418in}}%
\pgfpathlineto{\pgfqpoint{5.153316in}{6.085701in}}%
\pgfusepath{stroke}%
\end{pgfscope}%
\begin{pgfscope}%
\pgfpathrectangle{\pgfqpoint{3.352233in}{5.105882in}}{\pgfqpoint{2.407767in}{1.544118in}}%
\pgfusepath{clip}%
\pgfsetbuttcap%
\pgfsetroundjoin%
\pgfsetlinewidth{0.501875pt}%
\definecolor{currentstroke}{rgb}{0.281446,0.084320,0.407414}%
\pgfsetstrokecolor{currentstroke}%
\pgfsetdash{}{0pt}%
\pgfpathmoveto{\pgfqpoint{5.153316in}{6.085701in}}%
\pgfpathlineto{\pgfqpoint{5.100366in}{6.084632in}}%
\pgfusepath{stroke}%
\end{pgfscope}%
\begin{pgfscope}%
\pgfpathrectangle{\pgfqpoint{3.352233in}{5.105882in}}{\pgfqpoint{2.407767in}{1.544118in}}%
\pgfusepath{clip}%
\pgfsetbuttcap%
\pgfsetroundjoin%
\pgfsetlinewidth{0.501875pt}%
\definecolor{currentstroke}{rgb}{0.283229,0.120777,0.440584}%
\pgfsetstrokecolor{currentstroke}%
\pgfsetdash{}{0pt}%
\pgfpathmoveto{\pgfqpoint{5.100366in}{6.084632in}}%
\pgfpathlineto{\pgfqpoint{5.047437in}{6.083220in}}%
\pgfusepath{stroke}%
\end{pgfscope}%
\begin{pgfscope}%
\pgfpathrectangle{\pgfqpoint{3.352233in}{5.105882in}}{\pgfqpoint{2.407767in}{1.544118in}}%
\pgfusepath{clip}%
\pgfsetbuttcap%
\pgfsetroundjoin%
\pgfsetlinewidth{0.501875pt}%
\definecolor{currentstroke}{rgb}{0.280868,0.160771,0.472899}%
\pgfsetstrokecolor{currentstroke}%
\pgfsetdash{}{0pt}%
\pgfpathmoveto{\pgfqpoint{5.047437in}{6.083220in}}%
\pgfpathlineto{\pgfqpoint{4.994536in}{6.081433in}}%
\pgfusepath{stroke}%
\end{pgfscope}%
\begin{pgfscope}%
\pgfpathrectangle{\pgfqpoint{3.352233in}{5.105882in}}{\pgfqpoint{2.407767in}{1.544118in}}%
\pgfusepath{clip}%
\pgfsetbuttcap%
\pgfsetroundjoin%
\pgfsetlinewidth{0.501875pt}%
\definecolor{currentstroke}{rgb}{0.276194,0.190074,0.493001}%
\pgfsetstrokecolor{currentstroke}%
\pgfsetdash{}{0pt}%
\pgfpathmoveto{\pgfqpoint{4.994536in}{6.081433in}}%
\pgfpathlineto{\pgfqpoint{4.941685in}{6.079128in}}%
\pgfusepath{stroke}%
\end{pgfscope}%
\begin{pgfscope}%
\pgfpathrectangle{\pgfqpoint{3.352233in}{5.105882in}}{\pgfqpoint{2.407767in}{1.544118in}}%
\pgfusepath{clip}%
\pgfsetbuttcap%
\pgfsetroundjoin%
\pgfsetlinewidth{0.501875pt}%
\definecolor{currentstroke}{rgb}{0.266580,0.228262,0.514349}%
\pgfsetstrokecolor{currentstroke}%
\pgfsetdash{}{0pt}%
\pgfpathmoveto{\pgfqpoint{4.941685in}{6.079128in}}%
\pgfpathlineto{\pgfqpoint{4.888906in}{6.076218in}}%
\pgfusepath{stroke}%
\end{pgfscope}%
\begin{pgfscope}%
\pgfpathrectangle{\pgfqpoint{3.352233in}{5.105882in}}{\pgfqpoint{2.407767in}{1.544118in}}%
\pgfusepath{clip}%
\pgfsetbuttcap%
\pgfsetroundjoin%
\pgfsetlinewidth{0.501875pt}%
\definecolor{currentstroke}{rgb}{0.265145,0.232956,0.516599}%
\pgfsetstrokecolor{currentstroke}%
\pgfsetdash{}{0pt}%
\pgfpathmoveto{\pgfqpoint{4.888906in}{6.076218in}}%
\pgfpathlineto{\pgfqpoint{4.836216in}{6.072721in}}%
\pgfusepath{stroke}%
\end{pgfscope}%
\begin{pgfscope}%
\pgfpathrectangle{\pgfqpoint{3.352233in}{5.105882in}}{\pgfqpoint{2.407767in}{1.544118in}}%
\pgfusepath{clip}%
\pgfsetbuttcap%
\pgfsetroundjoin%
\pgfsetlinewidth{0.501875pt}%
\definecolor{currentstroke}{rgb}{0.273809,0.031497,0.358853}%
\pgfsetstrokecolor{currentstroke}%
\pgfsetdash{}{0pt}%
\pgfpathmoveto{\pgfqpoint{5.206279in}{6.121164in}}%
\pgfpathlineto{\pgfqpoint{5.153319in}{6.120454in}}%
\pgfusepath{stroke}%
\end{pgfscope}%
\begin{pgfscope}%
\pgfpathrectangle{\pgfqpoint{3.352233in}{5.105882in}}{\pgfqpoint{2.407767in}{1.544118in}}%
\pgfusepath{clip}%
\pgfsetbuttcap%
\pgfsetroundjoin%
\pgfsetlinewidth{0.501875pt}%
\definecolor{currentstroke}{rgb}{0.279566,0.067836,0.391917}%
\pgfsetstrokecolor{currentstroke}%
\pgfsetdash{}{0pt}%
\pgfpathmoveto{\pgfqpoint{5.153319in}{6.120454in}}%
\pgfpathlineto{\pgfqpoint{5.100377in}{6.119252in}}%
\pgfusepath{stroke}%
\end{pgfscope}%
\begin{pgfscope}%
\pgfpathrectangle{\pgfqpoint{3.352233in}{5.105882in}}{\pgfqpoint{2.407767in}{1.544118in}}%
\pgfusepath{clip}%
\pgfsetbuttcap%
\pgfsetroundjoin%
\pgfsetlinewidth{0.501875pt}%
\definecolor{currentstroke}{rgb}{0.283091,0.110553,0.431554}%
\pgfsetstrokecolor{currentstroke}%
\pgfsetdash{}{0pt}%
\pgfpathmoveto{\pgfqpoint{5.100377in}{6.119252in}}%
\pgfpathlineto{\pgfqpoint{5.047459in}{6.117689in}}%
\pgfusepath{stroke}%
\end{pgfscope}%
\begin{pgfscope}%
\pgfpathrectangle{\pgfqpoint{3.352233in}{5.105882in}}{\pgfqpoint{2.407767in}{1.544118in}}%
\pgfusepath{clip}%
\pgfsetbuttcap%
\pgfsetroundjoin%
\pgfsetlinewidth{0.501875pt}%
\definecolor{currentstroke}{rgb}{0.282290,0.145912,0.461510}%
\pgfsetstrokecolor{currentstroke}%
\pgfsetdash{}{0pt}%
\pgfpathmoveto{\pgfqpoint{5.047459in}{6.117689in}}%
\pgfpathlineto{\pgfqpoint{4.994596in}{6.115503in}}%
\pgfusepath{stroke}%
\end{pgfscope}%
\begin{pgfscope}%
\pgfpathrectangle{\pgfqpoint{3.352233in}{5.105882in}}{\pgfqpoint{2.407767in}{1.544118in}}%
\pgfusepath{clip}%
\pgfsetbuttcap%
\pgfsetroundjoin%
\pgfsetlinewidth{0.501875pt}%
\definecolor{currentstroke}{rgb}{0.280255,0.165693,0.476498}%
\pgfsetstrokecolor{currentstroke}%
\pgfsetdash{}{0pt}%
\pgfpathmoveto{\pgfqpoint{4.994596in}{6.115503in}}%
\pgfpathlineto{\pgfqpoint{4.941795in}{6.112757in}}%
\pgfusepath{stroke}%
\end{pgfscope}%
\begin{pgfscope}%
\pgfpathrectangle{\pgfqpoint{3.352233in}{5.105882in}}{\pgfqpoint{2.407767in}{1.544118in}}%
\pgfusepath{clip}%
\pgfsetbuttcap%
\pgfsetroundjoin%
\pgfsetlinewidth{0.501875pt}%
\definecolor{currentstroke}{rgb}{0.278012,0.180367,0.486697}%
\pgfsetstrokecolor{currentstroke}%
\pgfsetdash{}{0pt}%
\pgfpathmoveto{\pgfqpoint{4.941795in}{6.112757in}}%
\pgfpathlineto{\pgfqpoint{4.889066in}{6.109502in}}%
\pgfusepath{stroke}%
\end{pgfscope}%
\begin{pgfscope}%
\pgfpathrectangle{\pgfqpoint{3.352233in}{5.105882in}}{\pgfqpoint{2.407767in}{1.544118in}}%
\pgfusepath{clip}%
\pgfsetbuttcap%
\pgfsetroundjoin%
\pgfsetlinewidth{0.501875pt}%
\definecolor{currentstroke}{rgb}{0.277134,0.185228,0.489898}%
\pgfsetstrokecolor{currentstroke}%
\pgfsetdash{}{0pt}%
\pgfpathmoveto{\pgfqpoint{4.889066in}{6.109502in}}%
\pgfpathlineto{\pgfqpoint{4.836471in}{6.105465in}}%
\pgfusepath{stroke}%
\end{pgfscope}%
\begin{pgfscope}%
\pgfpathrectangle{\pgfqpoint{3.352233in}{5.105882in}}{\pgfqpoint{2.407767in}{1.544118in}}%
\pgfusepath{clip}%
\pgfsetbuttcap%
\pgfsetroundjoin%
\pgfsetlinewidth{0.501875pt}%
\definecolor{currentstroke}{rgb}{0.266580,0.228262,0.514349}%
\pgfsetstrokecolor{currentstroke}%
\pgfsetdash{}{0pt}%
\pgfpathmoveto{\pgfqpoint{4.836471in}{6.105465in}}%
\pgfpathlineto{\pgfqpoint{4.784027in}{6.100680in}}%
\pgfusepath{stroke}%
\end{pgfscope}%
\begin{pgfscope}%
\pgfpathrectangle{\pgfqpoint{3.352233in}{5.105882in}}{\pgfqpoint{2.407767in}{1.544118in}}%
\pgfusepath{clip}%
\pgfsetbuttcap%
\pgfsetroundjoin%
\pgfsetlinewidth{0.501875pt}%
\definecolor{currentstroke}{rgb}{0.269308,0.218818,0.509577}%
\pgfsetstrokecolor{currentstroke}%
\pgfsetdash{}{0pt}%
\pgfpathmoveto{\pgfqpoint{4.784027in}{6.100680in}}%
\pgfpathlineto{\pgfqpoint{4.731777in}{6.095108in}}%
\pgfusepath{stroke}%
\end{pgfscope}%
\begin{pgfscope}%
\pgfpathrectangle{\pgfqpoint{3.352233in}{5.105882in}}{\pgfqpoint{2.407767in}{1.544118in}}%
\pgfusepath{clip}%
\pgfsetbuttcap%
\pgfsetroundjoin%
\pgfsetlinewidth{0.501875pt}%
\definecolor{currentstroke}{rgb}{0.260571,0.246922,0.522828}%
\pgfsetstrokecolor{currentstroke}%
\pgfsetdash{}{0pt}%
\pgfpathmoveto{\pgfqpoint{4.731777in}{6.095108in}}%
\pgfpathlineto{\pgfqpoint{4.679820in}{6.088501in}}%
\pgfusepath{stroke}%
\end{pgfscope}%
\begin{pgfscope}%
\pgfpathrectangle{\pgfqpoint{3.352233in}{5.105882in}}{\pgfqpoint{2.407767in}{1.544118in}}%
\pgfusepath{clip}%
\pgfsetbuttcap%
\pgfsetroundjoin%
\pgfsetlinewidth{0.501875pt}%
\definecolor{currentstroke}{rgb}{0.257322,0.256130,0.526563}%
\pgfsetstrokecolor{currentstroke}%
\pgfsetdash{}{0pt}%
\pgfpathmoveto{\pgfqpoint{4.679820in}{6.088501in}}%
\pgfpathlineto{\pgfqpoint{4.628292in}{6.080655in}}%
\pgfusepath{stroke}%
\end{pgfscope}%
\begin{pgfscope}%
\pgfpathrectangle{\pgfqpoint{3.352233in}{5.105882in}}{\pgfqpoint{2.407767in}{1.544118in}}%
\pgfusepath{clip}%
\pgfsetbuttcap%
\pgfsetroundjoin%
\pgfsetlinewidth{0.501875pt}%
\definecolor{currentstroke}{rgb}{0.258965,0.251537,0.524736}%
\pgfsetstrokecolor{currentstroke}%
\pgfsetdash{}{0pt}%
\pgfpathmoveto{\pgfqpoint{4.628292in}{6.080655in}}%
\pgfpathlineto{\pgfqpoint{4.577553in}{6.070967in}}%
\pgfusepath{stroke}%
\end{pgfscope}%
\begin{pgfscope}%
\pgfpathrectangle{\pgfqpoint{3.352233in}{5.105882in}}{\pgfqpoint{2.407767in}{1.544118in}}%
\pgfusepath{clip}%
\pgfsetbuttcap%
\pgfsetroundjoin%
\pgfsetlinewidth{0.501875pt}%
\definecolor{currentstroke}{rgb}{0.257322,0.256130,0.526563}%
\pgfsetstrokecolor{currentstroke}%
\pgfsetdash{}{0pt}%
\pgfpathmoveto{\pgfqpoint{4.577553in}{6.070967in}}%
\pgfpathlineto{\pgfqpoint{4.527746in}{6.059440in}}%
\pgfusepath{stroke}%
\end{pgfscope}%
\begin{pgfscope}%
\pgfpathrectangle{\pgfqpoint{3.352233in}{5.105882in}}{\pgfqpoint{2.407767in}{1.544118in}}%
\pgfusepath{clip}%
\pgfsetbuttcap%
\pgfsetroundjoin%
\pgfsetlinewidth{0.501875pt}%
\definecolor{currentstroke}{rgb}{0.235526,0.309527,0.542944}%
\pgfsetstrokecolor{currentstroke}%
\pgfsetdash{}{0pt}%
\pgfpathmoveto{\pgfqpoint{4.527746in}{6.059440in}}%
\pgfpathlineto{\pgfqpoint{4.479075in}{6.046100in}}%
\pgfusepath{stroke}%
\end{pgfscope}%
\begin{pgfscope}%
\pgfpathrectangle{\pgfqpoint{3.352233in}{5.105882in}}{\pgfqpoint{2.407767in}{1.544118in}}%
\pgfusepath{clip}%
\pgfsetbuttcap%
\pgfsetroundjoin%
\pgfsetlinewidth{0.501875pt}%
\definecolor{currentstroke}{rgb}{0.262138,0.242286,0.520837}%
\pgfsetstrokecolor{currentstroke}%
\pgfsetdash{}{0pt}%
\pgfpathmoveto{\pgfqpoint{4.479075in}{6.046100in}}%
\pgfpathlineto{\pgfqpoint{4.431646in}{6.031042in}}%
\pgfusepath{stroke}%
\end{pgfscope}%
\begin{pgfscope}%
\pgfpathrectangle{\pgfqpoint{3.352233in}{5.105882in}}{\pgfqpoint{2.407767in}{1.544118in}}%
\pgfusepath{clip}%
\pgfsetbuttcap%
\pgfsetroundjoin%
\pgfsetlinewidth{0.501875pt}%
\definecolor{currentstroke}{rgb}{0.216210,0.351535,0.550627}%
\pgfsetstrokecolor{currentstroke}%
\pgfsetdash{}{0pt}%
\pgfpathmoveto{\pgfqpoint{4.431646in}{6.031042in}}%
\pgfpathlineto{\pgfqpoint{4.385057in}{6.014921in}}%
\pgfusepath{stroke}%
\end{pgfscope}%
\begin{pgfscope}%
\pgfpathrectangle{\pgfqpoint{3.352233in}{5.105882in}}{\pgfqpoint{2.407767in}{1.544118in}}%
\pgfusepath{clip}%
\pgfsetbuttcap%
\pgfsetroundjoin%
\pgfsetlinewidth{0.501875pt}%
\definecolor{currentstroke}{rgb}{0.273809,0.031497,0.358853}%
\pgfsetstrokecolor{currentstroke}%
\pgfsetdash{}{0pt}%
\pgfpathmoveto{\pgfqpoint{5.206279in}{6.155910in}}%
\pgfpathlineto{\pgfqpoint{5.153331in}{6.154872in}}%
\pgfusepath{stroke}%
\end{pgfscope}%
\begin{pgfscope}%
\pgfpathrectangle{\pgfqpoint{3.352233in}{5.105882in}}{\pgfqpoint{2.407767in}{1.544118in}}%
\pgfusepath{clip}%
\pgfsetbuttcap%
\pgfsetroundjoin%
\pgfsetlinewidth{0.501875pt}%
\definecolor{currentstroke}{rgb}{0.277941,0.056324,0.381191}%
\pgfsetstrokecolor{currentstroke}%
\pgfsetdash{}{0pt}%
\pgfpathmoveto{\pgfqpoint{5.153331in}{6.154872in}}%
\pgfpathlineto{\pgfqpoint{5.100384in}{6.153773in}}%
\pgfusepath{stroke}%
\end{pgfscope}%
\begin{pgfscope}%
\pgfpathrectangle{\pgfqpoint{3.352233in}{5.105882in}}{\pgfqpoint{2.407767in}{1.544118in}}%
\pgfusepath{clip}%
\pgfsetbuttcap%
\pgfsetroundjoin%
\pgfsetlinewidth{0.501875pt}%
\definecolor{currentstroke}{rgb}{0.281924,0.089666,0.412415}%
\pgfsetstrokecolor{currentstroke}%
\pgfsetdash{}{0pt}%
\pgfpathmoveto{\pgfqpoint{5.100384in}{6.153773in}}%
\pgfpathlineto{\pgfqpoint{5.047464in}{6.152252in}}%
\pgfusepath{stroke}%
\end{pgfscope}%
\begin{pgfscope}%
\pgfpathrectangle{\pgfqpoint{3.352233in}{5.105882in}}{\pgfqpoint{2.407767in}{1.544118in}}%
\pgfusepath{clip}%
\pgfsetbuttcap%
\pgfsetroundjoin%
\pgfsetlinewidth{0.501875pt}%
\definecolor{currentstroke}{rgb}{0.283091,0.110553,0.431554}%
\pgfsetstrokecolor{currentstroke}%
\pgfsetdash{}{0pt}%
\pgfpathmoveto{\pgfqpoint{5.047464in}{6.152252in}}%
\pgfpathlineto{\pgfqpoint{4.994622in}{6.149907in}}%
\pgfusepath{stroke}%
\end{pgfscope}%
\begin{pgfscope}%
\pgfpathrectangle{\pgfqpoint{3.352233in}{5.105882in}}{\pgfqpoint{2.407767in}{1.544118in}}%
\pgfusepath{clip}%
\pgfsetbuttcap%
\pgfsetroundjoin%
\pgfsetlinewidth{0.501875pt}%
\definecolor{currentstroke}{rgb}{0.283072,0.130895,0.449241}%
\pgfsetstrokecolor{currentstroke}%
\pgfsetdash{}{0pt}%
\pgfpathmoveto{\pgfqpoint{4.994622in}{6.149907in}}%
\pgfpathlineto{\pgfqpoint{4.941895in}{6.146656in}}%
\pgfusepath{stroke}%
\end{pgfscope}%
\begin{pgfscope}%
\pgfpathrectangle{\pgfqpoint{3.352233in}{5.105882in}}{\pgfqpoint{2.407767in}{1.544118in}}%
\pgfusepath{clip}%
\pgfsetbuttcap%
\pgfsetroundjoin%
\pgfsetlinewidth{0.501875pt}%
\definecolor{currentstroke}{rgb}{0.281412,0.155834,0.469201}%
\pgfsetstrokecolor{currentstroke}%
\pgfsetdash{}{0pt}%
\pgfpathmoveto{\pgfqpoint{4.941895in}{6.146656in}}%
\pgfpathlineto{\pgfqpoint{4.889285in}{6.142693in}}%
\pgfusepath{stroke}%
\end{pgfscope}%
\begin{pgfscope}%
\pgfpathrectangle{\pgfqpoint{3.352233in}{5.105882in}}{\pgfqpoint{2.407767in}{1.544118in}}%
\pgfusepath{clip}%
\pgfsetbuttcap%
\pgfsetroundjoin%
\pgfsetlinewidth{0.501875pt}%
\definecolor{currentstroke}{rgb}{0.271305,0.019942,0.347269}%
\pgfsetstrokecolor{currentstroke}%
\pgfsetdash{}{0pt}%
\pgfpathmoveto{\pgfqpoint{5.206279in}{6.190656in}}%
\pgfpathlineto{\pgfqpoint{5.153309in}{6.190272in}}%
\pgfusepath{stroke}%
\end{pgfscope}%
\begin{pgfscope}%
\pgfpathrectangle{\pgfqpoint{3.352233in}{5.105882in}}{\pgfqpoint{2.407767in}{1.544118in}}%
\pgfusepath{clip}%
\pgfsetbuttcap%
\pgfsetroundjoin%
\pgfsetlinewidth{0.501875pt}%
\definecolor{currentstroke}{rgb}{0.277018,0.050344,0.375715}%
\pgfsetstrokecolor{currentstroke}%
\pgfsetdash{}{0pt}%
\pgfpathmoveto{\pgfqpoint{5.153309in}{6.190272in}}%
\pgfpathlineto{\pgfqpoint{5.100368in}{6.189337in}}%
\pgfusepath{stroke}%
\end{pgfscope}%
\begin{pgfscope}%
\pgfpathrectangle{\pgfqpoint{3.352233in}{5.105882in}}{\pgfqpoint{2.407767in}{1.544118in}}%
\pgfusepath{clip}%
\pgfsetbuttcap%
\pgfsetroundjoin%
\pgfsetlinewidth{0.501875pt}%
\definecolor{currentstroke}{rgb}{0.279566,0.067836,0.391917}%
\pgfsetstrokecolor{currentstroke}%
\pgfsetdash{}{0pt}%
\pgfpathmoveto{\pgfqpoint{5.100368in}{6.189337in}}%
\pgfpathlineto{\pgfqpoint{5.047489in}{6.187328in}}%
\pgfusepath{stroke}%
\end{pgfscope}%
\begin{pgfscope}%
\pgfpathrectangle{\pgfqpoint{3.352233in}{5.105882in}}{\pgfqpoint{2.407767in}{1.544118in}}%
\pgfusepath{clip}%
\pgfsetbuttcap%
\pgfsetroundjoin%
\pgfsetlinewidth{0.501875pt}%
\definecolor{currentstroke}{rgb}{0.281924,0.089666,0.412415}%
\pgfsetstrokecolor{currentstroke}%
\pgfsetdash{}{0pt}%
\pgfpathmoveto{\pgfqpoint{5.047489in}{6.187328in}}%
\pgfpathlineto{\pgfqpoint{4.994701in}{6.184505in}}%
\pgfusepath{stroke}%
\end{pgfscope}%
\begin{pgfscope}%
\pgfpathrectangle{\pgfqpoint{3.352233in}{5.105882in}}{\pgfqpoint{2.407767in}{1.544118in}}%
\pgfusepath{clip}%
\pgfsetbuttcap%
\pgfsetroundjoin%
\pgfsetlinewidth{0.501875pt}%
\definecolor{currentstroke}{rgb}{0.282910,0.105393,0.426902}%
\pgfsetstrokecolor{currentstroke}%
\pgfsetdash{}{0pt}%
\pgfpathmoveto{\pgfqpoint{4.994701in}{6.184505in}}%
\pgfpathlineto{\pgfqpoint{4.942015in}{6.180968in}}%
\pgfusepath{stroke}%
\end{pgfscope}%
\begin{pgfscope}%
\pgfpathrectangle{\pgfqpoint{3.352233in}{5.105882in}}{\pgfqpoint{2.407767in}{1.544118in}}%
\pgfusepath{clip}%
\pgfsetbuttcap%
\pgfsetroundjoin%
\pgfsetlinewidth{0.501875pt}%
\definecolor{currentstroke}{rgb}{0.282884,0.135920,0.453427}%
\pgfsetstrokecolor{currentstroke}%
\pgfsetdash{}{0pt}%
\pgfpathmoveto{\pgfqpoint{4.942015in}{6.180968in}}%
\pgfpathlineto{\pgfqpoint{4.889395in}{6.177041in}}%
\pgfusepath{stroke}%
\end{pgfscope}%
\begin{pgfscope}%
\pgfpathrectangle{\pgfqpoint{3.352233in}{5.105882in}}{\pgfqpoint{2.407767in}{1.544118in}}%
\pgfusepath{clip}%
\pgfsetbuttcap%
\pgfsetroundjoin%
\pgfsetlinewidth{0.501875pt}%
\definecolor{currentstroke}{rgb}{0.280868,0.160771,0.472899}%
\pgfsetstrokecolor{currentstroke}%
\pgfsetdash{}{0pt}%
\pgfpathmoveto{\pgfqpoint{4.889395in}{6.177041in}}%
\pgfpathlineto{\pgfqpoint{4.837043in}{6.171966in}}%
\pgfusepath{stroke}%
\end{pgfscope}%
\begin{pgfscope}%
\pgfpathrectangle{\pgfqpoint{3.352233in}{5.105882in}}{\pgfqpoint{2.407767in}{1.544118in}}%
\pgfusepath{clip}%
\pgfsetbuttcap%
\pgfsetroundjoin%
\pgfsetlinewidth{0.501875pt}%
\definecolor{currentstroke}{rgb}{0.281412,0.155834,0.469201}%
\pgfsetstrokecolor{currentstroke}%
\pgfsetdash{}{0pt}%
\pgfpathmoveto{\pgfqpoint{4.837043in}{6.171966in}}%
\pgfpathlineto{\pgfqpoint{4.785042in}{6.165530in}}%
\pgfusepath{stroke}%
\end{pgfscope}%
\begin{pgfscope}%
\pgfpathrectangle{\pgfqpoint{3.352233in}{5.105882in}}{\pgfqpoint{2.407767in}{1.544118in}}%
\pgfusepath{clip}%
\pgfsetbuttcap%
\pgfsetroundjoin%
\pgfsetlinewidth{0.501875pt}%
\definecolor{currentstroke}{rgb}{0.282290,0.145912,0.461510}%
\pgfsetstrokecolor{currentstroke}%
\pgfsetdash{}{0pt}%
\pgfpathmoveto{\pgfqpoint{4.785042in}{6.165530in}}%
\pgfpathlineto{\pgfqpoint{4.733355in}{6.158121in}}%
\pgfusepath{stroke}%
\end{pgfscope}%
\begin{pgfscope}%
\pgfpathrectangle{\pgfqpoint{3.352233in}{5.105882in}}{\pgfqpoint{2.407767in}{1.544118in}}%
\pgfusepath{clip}%
\pgfsetbuttcap%
\pgfsetroundjoin%
\pgfsetlinewidth{0.501875pt}%
\definecolor{currentstroke}{rgb}{0.278012,0.180367,0.486697}%
\pgfsetstrokecolor{currentstroke}%
\pgfsetdash{}{0pt}%
\pgfpathmoveto{\pgfqpoint{4.733355in}{6.158121in}}%
\pgfpathlineto{\pgfqpoint{4.682202in}{6.149331in}}%
\pgfusepath{stroke}%
\end{pgfscope}%
\begin{pgfscope}%
\pgfpathrectangle{\pgfqpoint{3.352233in}{5.105882in}}{\pgfqpoint{2.407767in}{1.544118in}}%
\pgfusepath{clip}%
\pgfsetbuttcap%
\pgfsetroundjoin%
\pgfsetlinewidth{0.501875pt}%
\definecolor{currentstroke}{rgb}{0.277134,0.185228,0.489898}%
\pgfsetstrokecolor{currentstroke}%
\pgfsetdash{}{0pt}%
\pgfpathmoveto{\pgfqpoint{4.682202in}{6.149331in}}%
\pgfpathlineto{\pgfqpoint{4.631779in}{6.138952in}}%
\pgfusepath{stroke}%
\end{pgfscope}%
\begin{pgfscope}%
\pgfpathrectangle{\pgfqpoint{3.352233in}{5.105882in}}{\pgfqpoint{2.407767in}{1.544118in}}%
\pgfusepath{clip}%
\pgfsetbuttcap%
\pgfsetroundjoin%
\pgfsetlinewidth{0.501875pt}%
\definecolor{currentstroke}{rgb}{0.277134,0.185228,0.489898}%
\pgfsetstrokecolor{currentstroke}%
\pgfsetdash{}{0pt}%
\pgfpathmoveto{\pgfqpoint{4.631779in}{6.138952in}}%
\pgfpathlineto{\pgfqpoint{4.582788in}{6.126218in}}%
\pgfusepath{stroke}%
\end{pgfscope}%
\begin{pgfscope}%
\pgfpathrectangle{\pgfqpoint{3.352233in}{5.105882in}}{\pgfqpoint{2.407767in}{1.544118in}}%
\pgfusepath{clip}%
\pgfsetbuttcap%
\pgfsetroundjoin%
\pgfsetlinewidth{0.501875pt}%
\definecolor{currentstroke}{rgb}{0.271828,0.209303,0.504434}%
\pgfsetstrokecolor{currentstroke}%
\pgfsetdash{}{0pt}%
\pgfpathmoveto{\pgfqpoint{4.582788in}{6.126218in}}%
\pgfpathlineto{\pgfqpoint{4.535387in}{6.111168in}}%
\pgfusepath{stroke}%
\end{pgfscope}%
\begin{pgfscope}%
\pgfpathrectangle{\pgfqpoint{3.352233in}{5.105882in}}{\pgfqpoint{2.407767in}{1.544118in}}%
\pgfusepath{clip}%
\pgfsetbuttcap%
\pgfsetroundjoin%
\pgfsetlinewidth{0.501875pt}%
\definecolor{currentstroke}{rgb}{0.275191,0.194905,0.496005}%
\pgfsetstrokecolor{currentstroke}%
\pgfsetdash{}{0pt}%
\pgfpathmoveto{\pgfqpoint{4.535387in}{6.111168in}}%
\pgfpathlineto{\pgfqpoint{4.489834in}{6.093934in}}%
\pgfusepath{stroke}%
\end{pgfscope}%
\begin{pgfscope}%
\pgfpathrectangle{\pgfqpoint{3.352233in}{5.105882in}}{\pgfqpoint{2.407767in}{1.544118in}}%
\pgfusepath{clip}%
\pgfsetbuttcap%
\pgfsetroundjoin%
\pgfsetlinewidth{0.501875pt}%
\definecolor{currentstroke}{rgb}{0.271828,0.209303,0.504434}%
\pgfsetstrokecolor{currentstroke}%
\pgfsetdash{}{0pt}%
\pgfpathmoveto{\pgfqpoint{4.489834in}{6.093934in}}%
\pgfpathlineto{\pgfqpoint{4.445896in}{6.075055in}}%
\pgfusepath{stroke}%
\end{pgfscope}%
\begin{pgfscope}%
\pgfpathrectangle{\pgfqpoint{3.352233in}{5.105882in}}{\pgfqpoint{2.407767in}{1.544118in}}%
\pgfusepath{clip}%
\pgfsetbuttcap%
\pgfsetroundjoin%
\pgfsetlinewidth{0.501875pt}%
\definecolor{currentstroke}{rgb}{0.243113,0.292092,0.538516}%
\pgfsetstrokecolor{currentstroke}%
\pgfsetdash{}{0pt}%
\pgfpathmoveto{\pgfqpoint{4.445896in}{6.075055in}}%
\pgfpathlineto{\pgfqpoint{4.403845in}{6.054550in}}%
\pgfusepath{stroke}%
\end{pgfscope}%
\begin{pgfscope}%
\pgfpathrectangle{\pgfqpoint{3.352233in}{5.105882in}}{\pgfqpoint{2.407767in}{1.544118in}}%
\pgfusepath{clip}%
\pgfsetbuttcap%
\pgfsetroundjoin%
\pgfsetlinewidth{0.501875pt}%
\definecolor{currentstroke}{rgb}{0.239346,0.300855,0.540844}%
\pgfsetstrokecolor{currentstroke}%
\pgfsetdash{}{0pt}%
\pgfpathmoveto{\pgfqpoint{4.403845in}{6.054550in}}%
\pgfpathlineto{\pgfqpoint{4.363439in}{6.032703in}}%
\pgfusepath{stroke}%
\end{pgfscope}%
\begin{pgfscope}%
\pgfpathrectangle{\pgfqpoint{3.352233in}{5.105882in}}{\pgfqpoint{2.407767in}{1.544118in}}%
\pgfusepath{clip}%
\pgfsetbuttcap%
\pgfsetroundjoin%
\pgfsetlinewidth{0.501875pt}%
\definecolor{currentstroke}{rgb}{0.271305,0.019942,0.347269}%
\pgfsetstrokecolor{currentstroke}%
\pgfsetdash{}{0pt}%
\pgfpathmoveto{\pgfqpoint{5.206279in}{6.225402in}}%
\pgfpathlineto{\pgfqpoint{5.153326in}{6.224668in}}%
\pgfusepath{stroke}%
\end{pgfscope}%
\begin{pgfscope}%
\pgfpathrectangle{\pgfqpoint{3.352233in}{5.105882in}}{\pgfqpoint{2.407767in}{1.544118in}}%
\pgfusepath{clip}%
\pgfsetbuttcap%
\pgfsetroundjoin%
\pgfsetlinewidth{0.501875pt}%
\definecolor{currentstroke}{rgb}{0.276022,0.044167,0.370164}%
\pgfsetstrokecolor{currentstroke}%
\pgfsetdash{}{0pt}%
\pgfpathmoveto{\pgfqpoint{5.153326in}{6.224668in}}%
\pgfpathlineto{\pgfqpoint{5.100391in}{6.223378in}}%
\pgfusepath{stroke}%
\end{pgfscope}%
\begin{pgfscope}%
\pgfpathrectangle{\pgfqpoint{3.352233in}{5.105882in}}{\pgfqpoint{2.407767in}{1.544118in}}%
\pgfusepath{clip}%
\pgfsetbuttcap%
\pgfsetroundjoin%
\pgfsetlinewidth{0.501875pt}%
\definecolor{currentstroke}{rgb}{0.277018,0.050344,0.375715}%
\pgfsetstrokecolor{currentstroke}%
\pgfsetdash{}{0pt}%
\pgfpathmoveto{\pgfqpoint{5.100391in}{6.223378in}}%
\pgfpathlineto{\pgfqpoint{5.047509in}{6.221459in}}%
\pgfusepath{stroke}%
\end{pgfscope}%
\begin{pgfscope}%
\pgfpathrectangle{\pgfqpoint{3.352233in}{5.105882in}}{\pgfqpoint{2.407767in}{1.544118in}}%
\pgfusepath{clip}%
\pgfsetbuttcap%
\pgfsetroundjoin%
\pgfsetlinewidth{0.501875pt}%
\definecolor{currentstroke}{rgb}{0.279566,0.067836,0.391917}%
\pgfsetstrokecolor{currentstroke}%
\pgfsetdash{}{0pt}%
\pgfpathmoveto{\pgfqpoint{5.047509in}{6.221459in}}%
\pgfpathlineto{\pgfqpoint{4.994712in}{6.218715in}}%
\pgfusepath{stroke}%
\end{pgfscope}%
\begin{pgfscope}%
\pgfpathrectangle{\pgfqpoint{3.352233in}{5.105882in}}{\pgfqpoint{2.407767in}{1.544118in}}%
\pgfusepath{clip}%
\pgfsetbuttcap%
\pgfsetroundjoin%
\pgfsetlinewidth{0.501875pt}%
\definecolor{currentstroke}{rgb}{0.281924,0.089666,0.412415}%
\pgfsetstrokecolor{currentstroke}%
\pgfsetdash{}{0pt}%
\pgfpathmoveto{\pgfqpoint{4.994712in}{6.218715in}}%
\pgfpathlineto{\pgfqpoint{4.942035in}{6.215164in}}%
\pgfusepath{stroke}%
\end{pgfscope}%
\begin{pgfscope}%
\pgfpathrectangle{\pgfqpoint{3.352233in}{5.105882in}}{\pgfqpoint{2.407767in}{1.544118in}}%
\pgfusepath{clip}%
\pgfsetbuttcap%
\pgfsetroundjoin%
\pgfsetlinewidth{0.501875pt}%
\definecolor{currentstroke}{rgb}{0.283229,0.120777,0.440584}%
\pgfsetstrokecolor{currentstroke}%
\pgfsetdash{}{0pt}%
\pgfpathmoveto{\pgfqpoint{4.942035in}{6.215164in}}%
\pgfpathlineto{\pgfqpoint{4.889359in}{6.211635in}}%
\pgfusepath{stroke}%
\end{pgfscope}%
\begin{pgfscope}%
\pgfpathrectangle{\pgfqpoint{3.352233in}{5.105882in}}{\pgfqpoint{2.407767in}{1.544118in}}%
\pgfusepath{clip}%
\pgfsetbuttcap%
\pgfsetroundjoin%
\pgfsetlinewidth{0.501875pt}%
\definecolor{currentstroke}{rgb}{0.283091,0.110553,0.431554}%
\pgfsetstrokecolor{currentstroke}%
\pgfsetdash{}{0pt}%
\pgfpathmoveto{\pgfqpoint{4.889359in}{6.211635in}}%
\pgfpathlineto{\pgfqpoint{4.836911in}{6.207137in}}%
\pgfusepath{stroke}%
\end{pgfscope}%
\begin{pgfscope}%
\pgfpathrectangle{\pgfqpoint{3.352233in}{5.105882in}}{\pgfqpoint{2.407767in}{1.544118in}}%
\pgfusepath{clip}%
\pgfsetbuttcap%
\pgfsetroundjoin%
\pgfsetlinewidth{0.501875pt}%
\definecolor{currentstroke}{rgb}{0.283229,0.120777,0.440584}%
\pgfsetstrokecolor{currentstroke}%
\pgfsetdash{}{0pt}%
\pgfpathmoveto{\pgfqpoint{4.836911in}{6.207137in}}%
\pgfpathlineto{\pgfqpoint{4.784924in}{6.200628in}}%
\pgfusepath{stroke}%
\end{pgfscope}%
\begin{pgfscope}%
\pgfpathrectangle{\pgfqpoint{3.352233in}{5.105882in}}{\pgfqpoint{2.407767in}{1.544118in}}%
\pgfusepath{clip}%
\pgfsetbuttcap%
\pgfsetroundjoin%
\pgfsetlinewidth{0.501875pt}%
\definecolor{currentstroke}{rgb}{0.283072,0.130895,0.449241}%
\pgfsetstrokecolor{currentstroke}%
\pgfsetdash{}{0pt}%
\pgfpathmoveto{\pgfqpoint{4.784924in}{6.200628in}}%
\pgfpathlineto{\pgfqpoint{4.733479in}{6.192629in}}%
\pgfusepath{stroke}%
\end{pgfscope}%
\begin{pgfscope}%
\pgfpathrectangle{\pgfqpoint{3.352233in}{5.105882in}}{\pgfqpoint{2.407767in}{1.544118in}}%
\pgfusepath{clip}%
\pgfsetbuttcap%
\pgfsetroundjoin%
\pgfsetlinewidth{0.501875pt}%
\definecolor{currentstroke}{rgb}{0.283187,0.125848,0.444960}%
\pgfsetstrokecolor{currentstroke}%
\pgfsetdash{}{0pt}%
\pgfpathmoveto{\pgfqpoint{4.733479in}{6.192629in}}%
\pgfpathlineto{\pgfqpoint{4.682932in}{6.182553in}}%
\pgfusepath{stroke}%
\end{pgfscope}%
\begin{pgfscope}%
\pgfpathrectangle{\pgfqpoint{3.352233in}{5.105882in}}{\pgfqpoint{2.407767in}{1.544118in}}%
\pgfusepath{clip}%
\pgfsetbuttcap%
\pgfsetroundjoin%
\pgfsetlinewidth{0.501875pt}%
\definecolor{currentstroke}{rgb}{0.268510,0.009605,0.335427}%
\pgfsetstrokecolor{currentstroke}%
\pgfsetdash{}{0pt}%
\pgfpathmoveto{\pgfqpoint{5.206279in}{6.260149in}}%
\pgfpathlineto{\pgfqpoint{5.153342in}{6.259772in}}%
\pgfusepath{stroke}%
\end{pgfscope}%
\begin{pgfscope}%
\pgfpathrectangle{\pgfqpoint{3.352233in}{5.105882in}}{\pgfqpoint{2.407767in}{1.544118in}}%
\pgfusepath{clip}%
\pgfsetbuttcap%
\pgfsetroundjoin%
\pgfsetlinewidth{0.501875pt}%
\definecolor{currentstroke}{rgb}{0.274952,0.037752,0.364543}%
\pgfsetstrokecolor{currentstroke}%
\pgfsetdash{}{0pt}%
\pgfpathmoveto{\pgfqpoint{5.153342in}{6.259772in}}%
\pgfpathlineto{\pgfqpoint{5.100392in}{6.259591in}}%
\pgfusepath{stroke}%
\end{pgfscope}%
\begin{pgfscope}%
\pgfpathrectangle{\pgfqpoint{3.352233in}{5.105882in}}{\pgfqpoint{2.407767in}{1.544118in}}%
\pgfusepath{clip}%
\pgfsetbuttcap%
\pgfsetroundjoin%
\pgfsetlinewidth{0.501875pt}%
\definecolor{currentstroke}{rgb}{0.278791,0.062145,0.386592}%
\pgfsetstrokecolor{currentstroke}%
\pgfsetdash{}{0pt}%
\pgfpathmoveto{\pgfqpoint{5.100392in}{6.259591in}}%
\pgfpathlineto{\pgfqpoint{5.047633in}{6.256974in}}%
\pgfusepath{stroke}%
\end{pgfscope}%
\begin{pgfscope}%
\pgfpathrectangle{\pgfqpoint{3.352233in}{5.105882in}}{\pgfqpoint{2.407767in}{1.544118in}}%
\pgfusepath{clip}%
\pgfsetbuttcap%
\pgfsetroundjoin%
\pgfsetlinewidth{0.501875pt}%
\definecolor{currentstroke}{rgb}{0.277941,0.056324,0.381191}%
\pgfsetstrokecolor{currentstroke}%
\pgfsetdash{}{0pt}%
\pgfpathmoveto{\pgfqpoint{5.047633in}{6.256974in}}%
\pgfpathlineto{\pgfqpoint{4.994982in}{6.253399in}}%
\pgfusepath{stroke}%
\end{pgfscope}%
\begin{pgfscope}%
\pgfpathrectangle{\pgfqpoint{3.352233in}{5.105882in}}{\pgfqpoint{2.407767in}{1.544118in}}%
\pgfusepath{clip}%
\pgfsetbuttcap%
\pgfsetroundjoin%
\pgfsetlinewidth{0.501875pt}%
\definecolor{currentstroke}{rgb}{0.279566,0.067836,0.391917}%
\pgfsetstrokecolor{currentstroke}%
\pgfsetdash{}{0pt}%
\pgfpathmoveto{\pgfqpoint{4.994982in}{6.253399in}}%
\pgfpathlineto{\pgfqpoint{4.942336in}{6.249725in}}%
\pgfusepath{stroke}%
\end{pgfscope}%
\begin{pgfscope}%
\pgfpathrectangle{\pgfqpoint{3.352233in}{5.105882in}}{\pgfqpoint{2.407767in}{1.544118in}}%
\pgfusepath{clip}%
\pgfsetbuttcap%
\pgfsetroundjoin%
\pgfsetlinewidth{0.501875pt}%
\definecolor{currentstroke}{rgb}{0.281446,0.084320,0.407414}%
\pgfsetstrokecolor{currentstroke}%
\pgfsetdash{}{0pt}%
\pgfpathmoveto{\pgfqpoint{4.942336in}{6.249725in}}%
\pgfpathlineto{\pgfqpoint{4.889791in}{6.245414in}}%
\pgfusepath{stroke}%
\end{pgfscope}%
\begin{pgfscope}%
\pgfpathrectangle{\pgfqpoint{3.352233in}{5.105882in}}{\pgfqpoint{2.407767in}{1.544118in}}%
\pgfusepath{clip}%
\pgfsetbuttcap%
\pgfsetroundjoin%
\pgfsetlinewidth{0.501875pt}%
\definecolor{currentstroke}{rgb}{0.279566,0.067836,0.391917}%
\pgfsetstrokecolor{currentstroke}%
\pgfsetdash{}{0pt}%
\pgfpathmoveto{\pgfqpoint{4.889791in}{6.245414in}}%
\pgfpathlineto{\pgfqpoint{4.837696in}{6.239550in}}%
\pgfusepath{stroke}%
\end{pgfscope}%
\begin{pgfscope}%
\pgfpathrectangle{\pgfqpoint{3.352233in}{5.105882in}}{\pgfqpoint{2.407767in}{1.544118in}}%
\pgfusepath{clip}%
\pgfsetbuttcap%
\pgfsetroundjoin%
\pgfsetlinewidth{0.501875pt}%
\definecolor{currentstroke}{rgb}{0.280894,0.078907,0.402329}%
\pgfsetstrokecolor{currentstroke}%
\pgfsetdash{}{0pt}%
\pgfpathmoveto{\pgfqpoint{4.837696in}{6.239550in}}%
\pgfpathlineto{\pgfqpoint{4.786085in}{6.232025in}}%
\pgfusepath{stroke}%
\end{pgfscope}%
\begin{pgfscope}%
\pgfpathrectangle{\pgfqpoint{3.352233in}{5.105882in}}{\pgfqpoint{2.407767in}{1.544118in}}%
\pgfusepath{clip}%
\pgfsetbuttcap%
\pgfsetroundjoin%
\pgfsetlinewidth{0.501875pt}%
\definecolor{currentstroke}{rgb}{0.283197,0.115680,0.436115}%
\pgfsetstrokecolor{currentstroke}%
\pgfsetdash{}{0pt}%
\pgfpathmoveto{\pgfqpoint{4.786085in}{6.232025in}}%
\pgfpathlineto{\pgfqpoint{4.734827in}{6.223473in}}%
\pgfusepath{stroke}%
\end{pgfscope}%
\begin{pgfscope}%
\pgfpathrectangle{\pgfqpoint{3.352233in}{5.105882in}}{\pgfqpoint{2.407767in}{1.544118in}}%
\pgfusepath{clip}%
\pgfsetbuttcap%
\pgfsetroundjoin%
\pgfsetlinewidth{0.501875pt}%
\definecolor{currentstroke}{rgb}{0.283091,0.110553,0.431554}%
\pgfsetstrokecolor{currentstroke}%
\pgfsetdash{}{0pt}%
\pgfpathmoveto{\pgfqpoint{4.734827in}{6.223473in}}%
\pgfpathlineto{\pgfqpoint{4.683907in}{6.214136in}}%
\pgfusepath{stroke}%
\end{pgfscope}%
\begin{pgfscope}%
\pgfpathrectangle{\pgfqpoint{3.352233in}{5.105882in}}{\pgfqpoint{2.407767in}{1.544118in}}%
\pgfusepath{clip}%
\pgfsetbuttcap%
\pgfsetroundjoin%
\pgfsetlinewidth{0.501875pt}%
\definecolor{currentstroke}{rgb}{0.282327,0.094955,0.417331}%
\pgfsetstrokecolor{currentstroke}%
\pgfsetdash{}{0pt}%
\pgfpathmoveto{\pgfqpoint{4.683907in}{6.214136in}}%
\pgfpathlineto{\pgfqpoint{4.634007in}{6.202901in}}%
\pgfusepath{stroke}%
\end{pgfscope}%
\begin{pgfscope}%
\pgfpathrectangle{\pgfqpoint{3.352233in}{5.105882in}}{\pgfqpoint{2.407767in}{1.544118in}}%
\pgfusepath{clip}%
\pgfsetbuttcap%
\pgfsetroundjoin%
\pgfsetlinewidth{0.501875pt}%
\definecolor{currentstroke}{rgb}{0.283091,0.110553,0.431554}%
\pgfsetstrokecolor{currentstroke}%
\pgfsetdash{}{0pt}%
\pgfpathmoveto{\pgfqpoint{4.634007in}{6.202901in}}%
\pgfpathlineto{\pgfqpoint{4.586278in}{6.188345in}}%
\pgfusepath{stroke}%
\end{pgfscope}%
\begin{pgfscope}%
\pgfpathrectangle{\pgfqpoint{3.352233in}{5.105882in}}{\pgfqpoint{2.407767in}{1.544118in}}%
\pgfusepath{clip}%
\pgfsetbuttcap%
\pgfsetroundjoin%
\pgfsetlinewidth{0.501875pt}%
\definecolor{currentstroke}{rgb}{0.283229,0.120777,0.440584}%
\pgfsetstrokecolor{currentstroke}%
\pgfsetdash{}{0pt}%
\pgfpathmoveto{\pgfqpoint{4.586278in}{6.188345in}}%
\pgfpathlineto{\pgfqpoint{4.541565in}{6.170245in}}%
\pgfusepath{stroke}%
\end{pgfscope}%
\begin{pgfscope}%
\pgfpathrectangle{\pgfqpoint{3.352233in}{5.105882in}}{\pgfqpoint{2.407767in}{1.544118in}}%
\pgfusepath{clip}%
\pgfsetbuttcap%
\pgfsetroundjoin%
\pgfsetlinewidth{0.501875pt}%
\definecolor{currentstroke}{rgb}{0.282910,0.105393,0.426902}%
\pgfsetstrokecolor{currentstroke}%
\pgfsetdash{}{0pt}%
\pgfpathmoveto{\pgfqpoint{4.541565in}{6.170245in}}%
\pgfpathlineto{\pgfqpoint{4.502468in}{6.147719in}}%
\pgfusepath{stroke}%
\end{pgfscope}%
\begin{pgfscope}%
\pgfpathrectangle{\pgfqpoint{3.352233in}{5.105882in}}{\pgfqpoint{2.407767in}{1.544118in}}%
\pgfusepath{clip}%
\pgfsetbuttcap%
\pgfsetroundjoin%
\pgfsetlinewidth{0.501875pt}%
\definecolor{currentstroke}{rgb}{0.269944,0.014625,0.341379}%
\pgfsetstrokecolor{currentstroke}%
\pgfsetdash{}{0pt}%
\pgfpathmoveto{\pgfqpoint{5.206279in}{6.294895in}}%
\pgfpathlineto{\pgfqpoint{5.153369in}{6.294221in}}%
\pgfusepath{stroke}%
\end{pgfscope}%
\begin{pgfscope}%
\pgfpathrectangle{\pgfqpoint{3.352233in}{5.105882in}}{\pgfqpoint{2.407767in}{1.544118in}}%
\pgfusepath{clip}%
\pgfsetbuttcap%
\pgfsetroundjoin%
\pgfsetlinewidth{0.501875pt}%
\definecolor{currentstroke}{rgb}{0.272594,0.025563,0.353093}%
\pgfsetstrokecolor{currentstroke}%
\pgfsetdash{}{0pt}%
\pgfpathmoveto{\pgfqpoint{5.153369in}{6.294221in}}%
\pgfpathlineto{\pgfqpoint{5.100491in}{6.292479in}}%
\pgfusepath{stroke}%
\end{pgfscope}%
\begin{pgfscope}%
\pgfpathrectangle{\pgfqpoint{3.352233in}{5.105882in}}{\pgfqpoint{2.407767in}{1.544118in}}%
\pgfusepath{clip}%
\pgfsetbuttcap%
\pgfsetroundjoin%
\pgfsetlinewidth{0.501875pt}%
\definecolor{currentstroke}{rgb}{0.273809,0.031497,0.358853}%
\pgfsetstrokecolor{currentstroke}%
\pgfsetdash{}{0pt}%
\pgfpathmoveto{\pgfqpoint{5.100491in}{6.292479in}}%
\pgfpathlineto{\pgfqpoint{5.047644in}{6.290232in}}%
\pgfusepath{stroke}%
\end{pgfscope}%
\begin{pgfscope}%
\pgfpathrectangle{\pgfqpoint{3.352233in}{5.105882in}}{\pgfqpoint{2.407767in}{1.544118in}}%
\pgfusepath{clip}%
\pgfsetbuttcap%
\pgfsetroundjoin%
\pgfsetlinewidth{0.501875pt}%
\definecolor{currentstroke}{rgb}{0.273809,0.031497,0.358853}%
\pgfsetstrokecolor{currentstroke}%
\pgfsetdash{}{0pt}%
\pgfpathmoveto{\pgfqpoint{5.047644in}{6.290232in}}%
\pgfpathlineto{\pgfqpoint{4.994914in}{6.287259in}}%
\pgfusepath{stroke}%
\end{pgfscope}%
\begin{pgfscope}%
\pgfpathrectangle{\pgfqpoint{3.352233in}{5.105882in}}{\pgfqpoint{2.407767in}{1.544118in}}%
\pgfusepath{clip}%
\pgfsetbuttcap%
\pgfsetroundjoin%
\pgfsetlinewidth{0.501875pt}%
\definecolor{currentstroke}{rgb}{0.278791,0.062145,0.386592}%
\pgfsetstrokecolor{currentstroke}%
\pgfsetdash{}{0pt}%
\pgfpathmoveto{\pgfqpoint{4.994914in}{6.287259in}}%
\pgfpathlineto{\pgfqpoint{4.942329in}{6.283199in}}%
\pgfusepath{stroke}%
\end{pgfscope}%
\begin{pgfscope}%
\pgfpathrectangle{\pgfqpoint{3.352233in}{5.105882in}}{\pgfqpoint{2.407767in}{1.544118in}}%
\pgfusepath{clip}%
\pgfsetbuttcap%
\pgfsetroundjoin%
\pgfsetlinewidth{0.501875pt}%
\definecolor{currentstroke}{rgb}{0.279566,0.067836,0.391917}%
\pgfsetstrokecolor{currentstroke}%
\pgfsetdash{}{0pt}%
\pgfpathmoveto{\pgfqpoint{4.924976in}{5.466533in}}%
\pgfpathlineto{\pgfqpoint{4.872587in}{5.471347in}}%
\pgfusepath{stroke}%
\end{pgfscope}%
\begin{pgfscope}%
\pgfpathrectangle{\pgfqpoint{3.352233in}{5.105882in}}{\pgfqpoint{2.407767in}{1.544118in}}%
\pgfusepath{clip}%
\pgfsetbuttcap%
\pgfsetroundjoin%
\pgfsetlinewidth{0.501875pt}%
\definecolor{currentstroke}{rgb}{0.280267,0.073417,0.397163}%
\pgfsetstrokecolor{currentstroke}%
\pgfsetdash{}{0pt}%
\pgfpathmoveto{\pgfqpoint{4.872587in}{5.471347in}}%
\pgfpathlineto{\pgfqpoint{4.820669in}{5.477793in}}%
\pgfusepath{stroke}%
\end{pgfscope}%
\begin{pgfscope}%
\pgfpathrectangle{\pgfqpoint{3.352233in}{5.105882in}}{\pgfqpoint{2.407767in}{1.544118in}}%
\pgfusepath{clip}%
\pgfsetbuttcap%
\pgfsetroundjoin%
\pgfsetlinewidth{0.501875pt}%
\definecolor{currentstroke}{rgb}{0.281446,0.084320,0.407414}%
\pgfsetstrokecolor{currentstroke}%
\pgfsetdash{}{0pt}%
\pgfpathmoveto{\pgfqpoint{4.820669in}{5.477793in}}%
\pgfpathlineto{\pgfqpoint{4.769359in}{5.486051in}}%
\pgfusepath{stroke}%
\end{pgfscope}%
\begin{pgfscope}%
\pgfpathrectangle{\pgfqpoint{3.352233in}{5.105882in}}{\pgfqpoint{2.407767in}{1.544118in}}%
\pgfusepath{clip}%
\pgfsetbuttcap%
\pgfsetroundjoin%
\pgfsetlinewidth{0.501875pt}%
\definecolor{currentstroke}{rgb}{0.281446,0.084320,0.407414}%
\pgfsetstrokecolor{currentstroke}%
\pgfsetdash{}{0pt}%
\pgfpathmoveto{\pgfqpoint{4.769359in}{5.486051in}}%
\pgfpathlineto{\pgfqpoint{4.718657in}{5.495734in}}%
\pgfusepath{stroke}%
\end{pgfscope}%
\begin{pgfscope}%
\pgfpathrectangle{\pgfqpoint{3.352233in}{5.105882in}}{\pgfqpoint{2.407767in}{1.544118in}}%
\pgfusepath{clip}%
\pgfsetbuttcap%
\pgfsetroundjoin%
\pgfsetlinewidth{0.501875pt}%
\definecolor{currentstroke}{rgb}{0.282327,0.094955,0.417331}%
\pgfsetstrokecolor{currentstroke}%
\pgfsetdash{}{0pt}%
\pgfpathmoveto{\pgfqpoint{4.718657in}{5.495734in}}%
\pgfpathlineto{\pgfqpoint{4.669075in}{5.507522in}}%
\pgfusepath{stroke}%
\end{pgfscope}%
\begin{pgfscope}%
\pgfpathrectangle{\pgfqpoint{3.352233in}{5.105882in}}{\pgfqpoint{2.407767in}{1.544118in}}%
\pgfusepath{clip}%
\pgfsetbuttcap%
\pgfsetroundjoin%
\pgfsetlinewidth{0.501875pt}%
\definecolor{currentstroke}{rgb}{0.282656,0.100196,0.422160}%
\pgfsetstrokecolor{currentstroke}%
\pgfsetdash{}{0pt}%
\pgfpathmoveto{\pgfqpoint{4.669075in}{5.507522in}}%
\pgfpathlineto{\pgfqpoint{4.620792in}{5.521295in}}%
\pgfusepath{stroke}%
\end{pgfscope}%
\begin{pgfscope}%
\pgfpathrectangle{\pgfqpoint{3.352233in}{5.105882in}}{\pgfqpoint{2.407767in}{1.544118in}}%
\pgfusepath{clip}%
\pgfsetbuttcap%
\pgfsetroundjoin%
\pgfsetlinewidth{0.501875pt}%
\definecolor{currentstroke}{rgb}{0.280267,0.073417,0.397163}%
\pgfsetstrokecolor{currentstroke}%
\pgfsetdash{}{0pt}%
\pgfpathmoveto{\pgfqpoint{4.821740in}{6.275686in}}%
\pgfpathlineto{\pgfqpoint{4.770034in}{6.268325in}}%
\pgfusepath{stroke}%
\end{pgfscope}%
\begin{pgfscope}%
\pgfpathrectangle{\pgfqpoint{3.352233in}{5.105882in}}{\pgfqpoint{2.407767in}{1.544118in}}%
\pgfusepath{clip}%
\pgfsetbuttcap%
\pgfsetroundjoin%
\pgfsetlinewidth{0.501875pt}%
\definecolor{currentstroke}{rgb}{0.281446,0.084320,0.407414}%
\pgfsetstrokecolor{currentstroke}%
\pgfsetdash{}{0pt}%
\pgfpathmoveto{\pgfqpoint{4.770034in}{6.268325in}}%
\pgfpathlineto{\pgfqpoint{4.718657in}{6.260149in}}%
\pgfusepath{stroke}%
\end{pgfscope}%
\begin{pgfscope}%
\pgfpathrectangle{\pgfqpoint{3.352233in}{5.105882in}}{\pgfqpoint{2.407767in}{1.544118in}}%
\pgfusepath{clip}%
\pgfsetbuttcap%
\pgfsetroundjoin%
\pgfsetlinewidth{0.501875pt}%
\definecolor{currentstroke}{rgb}{0.281446,0.084320,0.407414}%
\pgfsetstrokecolor{currentstroke}%
\pgfsetdash{}{0pt}%
\pgfpathmoveto{\pgfqpoint{4.718657in}{6.260149in}}%
\pgfpathlineto{\pgfqpoint{4.668891in}{6.248769in}}%
\pgfusepath{stroke}%
\end{pgfscope}%
\begin{pgfscope}%
\pgfpathrectangle{\pgfqpoint{3.352233in}{5.105882in}}{\pgfqpoint{2.407767in}{1.544118in}}%
\pgfusepath{clip}%
\pgfsetbuttcap%
\pgfsetroundjoin%
\pgfsetlinewidth{0.501875pt}%
\definecolor{currentstroke}{rgb}{0.282327,0.094955,0.417331}%
\pgfsetstrokecolor{currentstroke}%
\pgfsetdash{}{0pt}%
\pgfpathmoveto{\pgfqpoint{4.668891in}{6.248769in}}%
\pgfpathlineto{\pgfqpoint{4.620542in}{6.234977in}}%
\pgfusepath{stroke}%
\end{pgfscope}%
\begin{pgfscope}%
\pgfpathrectangle{\pgfqpoint{3.352233in}{5.105882in}}{\pgfqpoint{2.407767in}{1.544118in}}%
\pgfusepath{clip}%
\pgfsetbuttcap%
\pgfsetroundjoin%
\pgfsetlinewidth{0.501875pt}%
\definecolor{currentstroke}{rgb}{0.282656,0.100196,0.422160}%
\pgfsetstrokecolor{currentstroke}%
\pgfsetdash{}{0pt}%
\pgfpathmoveto{\pgfqpoint{4.620542in}{6.234977in}}%
\pgfpathlineto{\pgfqpoint{4.574511in}{6.218533in}}%
\pgfusepath{stroke}%
\end{pgfscope}%
\begin{pgfscope}%
\pgfpathrectangle{\pgfqpoint{3.352233in}{5.105882in}}{\pgfqpoint{2.407767in}{1.544118in}}%
\pgfusepath{clip}%
\pgfsetbuttcap%
\pgfsetroundjoin%
\pgfsetlinewidth{0.501875pt}%
\definecolor{currentstroke}{rgb}{0.283229,0.120777,0.440584}%
\pgfsetstrokecolor{currentstroke}%
\pgfsetdash{}{0pt}%
\pgfpathmoveto{\pgfqpoint{4.574511in}{6.218533in}}%
\pgfpathlineto{\pgfqpoint{4.534590in}{6.196688in}}%
\pgfusepath{stroke}%
\end{pgfscope}%
\begin{pgfscope}%
\pgfpathrectangle{\pgfqpoint{3.352233in}{5.105882in}}{\pgfqpoint{2.407767in}{1.544118in}}%
\pgfusepath{clip}%
\pgfsetbuttcap%
\pgfsetroundjoin%
\pgfsetlinewidth{0.501875pt}%
\definecolor{currentstroke}{rgb}{0.282910,0.105393,0.426902}%
\pgfsetstrokecolor{currentstroke}%
\pgfsetdash{}{0pt}%
\pgfpathmoveto{\pgfqpoint{4.534590in}{6.196688in}}%
\pgfpathlineto{\pgfqpoint{4.503129in}{6.175512in}}%
\pgfusepath{stroke}%
\end{pgfscope}%
\begin{pgfscope}%
\pgfpathrectangle{\pgfqpoint{3.352233in}{5.105882in}}{\pgfqpoint{2.407767in}{1.544118in}}%
\pgfusepath{clip}%
\pgfsetbuttcap%
\pgfsetroundjoin%
\pgfsetlinewidth{0.501875pt}%
\definecolor{currentstroke}{rgb}{0.277941,0.056324,0.381191}%
\pgfsetstrokecolor{currentstroke}%
\pgfsetdash{}{0pt}%
\pgfpathmoveto{\pgfqpoint{4.378230in}{6.294631in}}%
\pgfpathlineto{\pgfqpoint{4.390287in}{6.280016in}}%
\pgfusepath{stroke}%
\end{pgfscope}%
\begin{pgfscope}%
\pgfpathrectangle{\pgfqpoint{3.352233in}{5.105882in}}{\pgfqpoint{2.407767in}{1.544118in}}%
\pgfusepath{clip}%
\pgfsetbuttcap%
\pgfsetroundjoin%
\pgfsetlinewidth{0.501875pt}%
\definecolor{currentstroke}{rgb}{0.281446,0.084320,0.407414}%
\pgfsetstrokecolor{currentstroke}%
\pgfsetdash{}{0pt}%
\pgfpathmoveto{\pgfqpoint{4.390287in}{6.280016in}}%
\pgfpathlineto{\pgfqpoint{4.390287in}{6.280016in}}%
\pgfusepath{stroke}%
\end{pgfscope}%
\begin{pgfscope}%
\pgfpathrectangle{\pgfqpoint{3.352233in}{5.105882in}}{\pgfqpoint{2.407767in}{1.544118in}}%
\pgfusepath{clip}%
\pgfsetbuttcap%
\pgfsetroundjoin%
\pgfsetlinewidth{0.501875pt}%
\definecolor{currentstroke}{rgb}{0.281446,0.084320,0.407414}%
\pgfsetstrokecolor{currentstroke}%
\pgfsetdash{}{0pt}%
\pgfpathmoveto{\pgfqpoint{4.390287in}{6.280016in}}%
\pgfpathlineto{\pgfqpoint{4.393576in}{6.260149in}}%
\pgfusepath{stroke}%
\end{pgfscope}%
\begin{pgfscope}%
\pgfpathrectangle{\pgfqpoint{3.352233in}{5.105882in}}{\pgfqpoint{2.407767in}{1.544118in}}%
\pgfusepath{clip}%
\pgfsetbuttcap%
\pgfsetroundjoin%
\pgfsetlinewidth{0.501875pt}%
\definecolor{currentstroke}{rgb}{0.282327,0.094955,0.417331}%
\pgfsetstrokecolor{currentstroke}%
\pgfsetdash{}{0pt}%
\pgfpathmoveto{\pgfqpoint{4.393576in}{6.260149in}}%
\pgfpathlineto{\pgfqpoint{4.393576in}{6.260149in}}%
\pgfusepath{stroke}%
\end{pgfscope}%
\begin{pgfscope}%
\pgfpathrectangle{\pgfqpoint{3.352233in}{5.105882in}}{\pgfqpoint{2.407767in}{1.544118in}}%
\pgfusepath{clip}%
\pgfsetbuttcap%
\pgfsetroundjoin%
\pgfsetlinewidth{0.501875pt}%
\definecolor{currentstroke}{rgb}{0.282327,0.094955,0.417331}%
\pgfsetstrokecolor{currentstroke}%
\pgfsetdash{}{0pt}%
\pgfpathmoveto{\pgfqpoint{4.393576in}{6.260149in}}%
\pgfpathlineto{\pgfqpoint{4.393576in}{6.260149in}}%
\pgfusepath{stroke}%
\end{pgfscope}%
\begin{pgfscope}%
\pgfpathrectangle{\pgfqpoint{3.352233in}{5.105882in}}{\pgfqpoint{2.407767in}{1.544118in}}%
\pgfusepath{clip}%
\pgfsetbuttcap%
\pgfsetroundjoin%
\pgfsetlinewidth{0.501875pt}%
\definecolor{currentstroke}{rgb}{0.282327,0.094955,0.417331}%
\pgfsetstrokecolor{currentstroke}%
\pgfsetdash{}{0pt}%
\pgfpathmoveto{\pgfqpoint{4.393576in}{6.260149in}}%
\pgfpathlineto{\pgfqpoint{4.398358in}{6.232716in}}%
\pgfusepath{stroke}%
\end{pgfscope}%
\begin{pgfscope}%
\pgfpathrectangle{\pgfqpoint{3.352233in}{5.105882in}}{\pgfqpoint{2.407767in}{1.544118in}}%
\pgfusepath{clip}%
\pgfsetbuttcap%
\pgfsetroundjoin%
\pgfsetlinewidth{0.501875pt}%
\definecolor{currentstroke}{rgb}{0.281446,0.084320,0.407414}%
\pgfsetstrokecolor{currentstroke}%
\pgfsetdash{}{0pt}%
\pgfpathmoveto{\pgfqpoint{4.398358in}{6.232716in}}%
\pgfpathlineto{\pgfqpoint{4.404332in}{6.206948in}}%
\pgfusepath{stroke}%
\end{pgfscope}%
\begin{pgfscope}%
\pgfpathrectangle{\pgfqpoint{3.352233in}{5.105882in}}{\pgfqpoint{2.407767in}{1.544118in}}%
\pgfusepath{clip}%
\pgfsetbuttcap%
\pgfsetroundjoin%
\pgfsetlinewidth{0.501875pt}%
\definecolor{currentstroke}{rgb}{0.283229,0.120777,0.440584}%
\pgfsetstrokecolor{currentstroke}%
\pgfsetdash{}{0pt}%
\pgfpathmoveto{\pgfqpoint{4.404332in}{6.206948in}}%
\pgfpathlineto{\pgfqpoint{4.404531in}{6.173340in}}%
\pgfusepath{stroke}%
\end{pgfscope}%
\begin{pgfscope}%
\pgfpathrectangle{\pgfqpoint{3.352233in}{5.105882in}}{\pgfqpoint{2.407767in}{1.544118in}}%
\pgfusepath{clip}%
\pgfsetbuttcap%
\pgfsetroundjoin%
\pgfsetlinewidth{0.501875pt}%
\definecolor{currentstroke}{rgb}{0.282290,0.145912,0.461510}%
\pgfsetstrokecolor{currentstroke}%
\pgfsetdash{}{0pt}%
\pgfpathmoveto{\pgfqpoint{4.404531in}{6.173340in}}%
\pgfpathlineto{\pgfqpoint{4.396491in}{6.141809in}}%
\pgfusepath{stroke}%
\end{pgfscope}%
\begin{pgfscope}%
\pgfpathrectangle{\pgfqpoint{3.352233in}{5.105882in}}{\pgfqpoint{2.407767in}{1.544118in}}%
\pgfusepath{clip}%
\pgfsetbuttcap%
\pgfsetroundjoin%
\pgfsetlinewidth{0.501875pt}%
\definecolor{currentstroke}{rgb}{0.280868,0.160771,0.472899}%
\pgfsetstrokecolor{currentstroke}%
\pgfsetdash{}{0pt}%
\pgfpathmoveto{\pgfqpoint{4.396491in}{6.141809in}}%
\pgfpathlineto{\pgfqpoint{4.380659in}{6.109642in}}%
\pgfusepath{stroke}%
\end{pgfscope}%
\begin{pgfscope}%
\pgfpathrectangle{\pgfqpoint{3.352233in}{5.105882in}}{\pgfqpoint{2.407767in}{1.544118in}}%
\pgfusepath{clip}%
\pgfsetbuttcap%
\pgfsetroundjoin%
\pgfsetlinewidth{0.501875pt}%
\definecolor{currentstroke}{rgb}{0.273006,0.204520,0.501721}%
\pgfsetstrokecolor{currentstroke}%
\pgfsetdash{}{0pt}%
\pgfpathmoveto{\pgfqpoint{4.380659in}{6.109642in}}%
\pgfpathlineto{\pgfqpoint{4.357301in}{6.080471in}}%
\pgfusepath{stroke}%
\end{pgfscope}%
\begin{pgfscope}%
\pgfpathrectangle{\pgfqpoint{3.352233in}{5.105882in}}{\pgfqpoint{2.407767in}{1.544118in}}%
\pgfusepath{clip}%
\pgfsetbuttcap%
\pgfsetroundjoin%
\pgfsetlinewidth{0.501875pt}%
\definecolor{currentstroke}{rgb}{0.220057,0.343307,0.549413}%
\pgfsetstrokecolor{currentstroke}%
\pgfsetdash{}{0pt}%
\pgfpathmoveto{\pgfqpoint{4.664477in}{6.016926in}}%
\pgfpathlineto{\pgfqpoint{4.612330in}{6.010980in}}%
\pgfusepath{stroke}%
\end{pgfscope}%
\begin{pgfscope}%
\pgfpathrectangle{\pgfqpoint{3.352233in}{5.105882in}}{\pgfqpoint{2.407767in}{1.544118in}}%
\pgfusepath{clip}%
\pgfsetbuttcap%
\pgfsetroundjoin%
\pgfsetlinewidth{0.501875pt}%
\definecolor{currentstroke}{rgb}{0.210503,0.363727,0.552206}%
\pgfsetstrokecolor{currentstroke}%
\pgfsetdash{}{0pt}%
\pgfpathmoveto{\pgfqpoint{4.612330in}{6.010980in}}%
\pgfpathlineto{\pgfqpoint{4.560517in}{6.003946in}}%
\pgfusepath{stroke}%
\end{pgfscope}%
\begin{pgfscope}%
\pgfpathrectangle{\pgfqpoint{3.352233in}{5.105882in}}{\pgfqpoint{2.407767in}{1.544118in}}%
\pgfusepath{clip}%
\pgfsetbuttcap%
\pgfsetroundjoin%
\pgfsetlinewidth{0.501875pt}%
\definecolor{currentstroke}{rgb}{0.204903,0.375746,0.553533}%
\pgfsetstrokecolor{currentstroke}%
\pgfsetdash{}{0pt}%
\pgfpathmoveto{\pgfqpoint{4.560517in}{6.003946in}}%
\pgfpathlineto{\pgfqpoint{4.509235in}{5.995488in}}%
\pgfusepath{stroke}%
\end{pgfscope}%
\begin{pgfscope}%
\pgfpathrectangle{\pgfqpoint{3.352233in}{5.105882in}}{\pgfqpoint{2.407767in}{1.544118in}}%
\pgfusepath{clip}%
\pgfsetbuttcap%
\pgfsetroundjoin%
\pgfsetlinewidth{0.501875pt}%
\definecolor{currentstroke}{rgb}{0.195860,0.395433,0.555276}%
\pgfsetstrokecolor{currentstroke}%
\pgfsetdash{}{0pt}%
\pgfpathmoveto{\pgfqpoint{4.509235in}{5.995488in}}%
\pgfpathlineto{\pgfqpoint{4.458333in}{5.986100in}}%
\pgfusepath{stroke}%
\end{pgfscope}%
\begin{pgfscope}%
\pgfpathrectangle{\pgfqpoint{3.352233in}{5.105882in}}{\pgfqpoint{2.407767in}{1.544118in}}%
\pgfusepath{clip}%
\pgfsetbuttcap%
\pgfsetroundjoin%
\pgfsetlinewidth{0.501875pt}%
\definecolor{currentstroke}{rgb}{0.171176,0.452530,0.557965}%
\pgfsetstrokecolor{currentstroke}%
\pgfsetdash{}{0pt}%
\pgfpathmoveto{\pgfqpoint{4.458333in}{5.986100in}}%
\pgfpathlineto{\pgfqpoint{4.407701in}{5.976151in}}%
\pgfusepath{stroke}%
\end{pgfscope}%
\begin{pgfscope}%
\pgfpathrectangle{\pgfqpoint{3.352233in}{5.105882in}}{\pgfqpoint{2.407767in}{1.544118in}}%
\pgfusepath{clip}%
\pgfsetbuttcap%
\pgfsetroundjoin%
\pgfsetlinewidth{0.501875pt}%
\definecolor{currentstroke}{rgb}{0.169646,0.456262,0.558030}%
\pgfsetstrokecolor{currentstroke}%
\pgfsetdash{}{0pt}%
\pgfpathmoveto{\pgfqpoint{4.407701in}{5.976151in}}%
\pgfpathlineto{\pgfqpoint{4.357216in}{5.965902in}}%
\pgfusepath{stroke}%
\end{pgfscope}%
\begin{pgfscope}%
\pgfpathrectangle{\pgfqpoint{3.352233in}{5.105882in}}{\pgfqpoint{2.407767in}{1.544118in}}%
\pgfusepath{clip}%
\pgfsetroundcap%
\pgfsetroundjoin%
\pgfsetlinewidth{0.501875pt}%
\definecolor{currentstroke}{rgb}{0.283229,0.120777,0.440584}%
\pgfsetstrokecolor{currentstroke}%
\pgfsetdash{}{0pt}%
\pgfpathmoveto{\pgfqpoint{4.447645in}{5.544925in}}%
\pgfpathquadraticcurveto{\pgfqpoint{4.443909in}{5.552962in}}{\pgfqpoint{4.443446in}{5.553958in}}%
\pgfusepath{stroke}%
\end{pgfscope}%
\begin{pgfscope}%
\pgfpathrectangle{\pgfqpoint{3.352233in}{5.105882in}}{\pgfqpoint{2.407767in}{1.544118in}}%
\pgfusepath{clip}%
\pgfsetroundcap%
\pgfsetroundjoin%
\definecolor{currentfill}{rgb}{0.283229,0.120777,0.440584}%
\pgfsetfillcolor{currentfill}%
\pgfsetlinewidth{0.501875pt}%
\definecolor{currentstroke}{rgb}{0.283229,0.120777,0.440584}%
\pgfsetstrokecolor{currentstroke}%
\pgfsetdash{}{0pt}%
\pgfpathmoveto{\pgfqpoint{4.442561in}{5.522914in}}%
\pgfpathlineto{\pgfqpoint{4.443446in}{5.553958in}}%
\pgfpathlineto{\pgfqpoint{4.467750in}{5.534624in}}%
\pgfpathlineto{\pgfqpoint{4.442561in}{5.522914in}}%
\pgfpathlineto{\pgfqpoint{4.442561in}{5.522914in}}%
\pgfpathclose%
\pgfusepath{stroke,fill}%
\end{pgfscope}%
\begin{pgfscope}%
\pgfpathrectangle{\pgfqpoint{3.352233in}{5.105882in}}{\pgfqpoint{2.407767in}{1.544118in}}%
\pgfusepath{clip}%
\pgfsetroundcap%
\pgfsetroundjoin%
\pgfsetlinewidth{0.501875pt}%
\definecolor{currentstroke}{rgb}{0.273809,0.031497,0.358853}%
\pgfsetstrokecolor{currentstroke}%
\pgfsetdash{}{0pt}%
\pgfpathmoveto{\pgfqpoint{4.987881in}{5.421792in}}%
\pgfpathquadraticcurveto{\pgfqpoint{4.974756in}{5.422904in}}{\pgfqpoint{4.969366in}{5.423361in}}%
\pgfusepath{stroke}%
\end{pgfscope}%
\begin{pgfscope}%
\pgfpathrectangle{\pgfqpoint{3.352233in}{5.105882in}}{\pgfqpoint{2.407767in}{1.544118in}}%
\pgfusepath{clip}%
\pgfsetroundcap%
\pgfsetroundjoin%
\definecolor{currentfill}{rgb}{0.273809,0.031497,0.358853}%
\pgfsetfillcolor{currentfill}%
\pgfsetlinewidth{0.501875pt}%
\definecolor{currentstroke}{rgb}{0.273809,0.031497,0.358853}%
\pgfsetstrokecolor{currentstroke}%
\pgfsetdash{}{0pt}%
\pgfpathmoveto{\pgfqpoint{4.995872in}{5.407176in}}%
\pgfpathlineto{\pgfqpoint{4.969366in}{5.423361in}}%
\pgfpathlineto{\pgfqpoint{4.998217in}{5.434854in}}%
\pgfpathlineto{\pgfqpoint{4.995872in}{5.407176in}}%
\pgfpathlineto{\pgfqpoint{4.995872in}{5.407176in}}%
\pgfpathclose%
\pgfusepath{stroke,fill}%
\end{pgfscope}%
\begin{pgfscope}%
\pgfpathrectangle{\pgfqpoint{3.352233in}{5.105882in}}{\pgfqpoint{2.407767in}{1.544118in}}%
\pgfusepath{clip}%
\pgfsetroundcap%
\pgfsetroundjoin%
\pgfsetlinewidth{0.501875pt}%
\definecolor{currentstroke}{rgb}{0.282327,0.094955,0.417331}%
\pgfsetstrokecolor{currentstroke}%
\pgfsetdash{}{0pt}%
\pgfpathmoveto{\pgfqpoint{4.497732in}{6.219591in}}%
\pgfpathquadraticcurveto{\pgfqpoint{4.490275in}{6.212813in}}{\pgfqpoint{4.488563in}{6.211256in}}%
\pgfusepath{stroke}%
\end{pgfscope}%
\begin{pgfscope}%
\pgfpathrectangle{\pgfqpoint{3.352233in}{5.105882in}}{\pgfqpoint{2.407767in}{1.544118in}}%
\pgfusepath{clip}%
\pgfsetroundcap%
\pgfsetroundjoin%
\definecolor{currentfill}{rgb}{0.282327,0.094955,0.417331}%
\pgfsetfillcolor{currentfill}%
\pgfsetlinewidth{0.501875pt}%
\definecolor{currentstroke}{rgb}{0.282327,0.094955,0.417331}%
\pgfsetstrokecolor{currentstroke}%
\pgfsetdash{}{0pt}%
\pgfpathmoveto{\pgfqpoint{4.518460in}{6.219663in}}%
\pgfpathlineto{\pgfqpoint{4.488563in}{6.211256in}}%
\pgfpathlineto{\pgfqpoint{4.499776in}{6.240218in}}%
\pgfpathlineto{\pgfqpoint{4.518460in}{6.219663in}}%
\pgfpathlineto{\pgfqpoint{4.518460in}{6.219663in}}%
\pgfpathclose%
\pgfusepath{stroke,fill}%
\end{pgfscope}%
\begin{pgfscope}%
\pgfpathrectangle{\pgfqpoint{3.352233in}{5.105882in}}{\pgfqpoint{2.407767in}{1.544118in}}%
\pgfusepath{clip}%
\pgfsetroundcap%
\pgfsetroundjoin%
\pgfsetlinewidth{0.501875pt}%
\definecolor{currentstroke}{rgb}{0.278791,0.062145,0.386592}%
\pgfsetstrokecolor{currentstroke}%
\pgfsetdash{}{0pt}%
\pgfpathmoveto{\pgfqpoint{4.581964in}{6.299055in}}%
\pgfpathquadraticcurveto{\pgfqpoint{4.570855in}{6.294745in}}{\pgfqpoint{4.566985in}{6.293243in}}%
\pgfusepath{stroke}%
\end{pgfscope}%
\begin{pgfscope}%
\pgfpathrectangle{\pgfqpoint{3.352233in}{5.105882in}}{\pgfqpoint{2.407767in}{1.544118in}}%
\pgfusepath{clip}%
\pgfsetroundcap%
\pgfsetroundjoin%
\definecolor{currentfill}{rgb}{0.278791,0.062145,0.386592}%
\pgfsetfillcolor{currentfill}%
\pgfsetlinewidth{0.501875pt}%
\definecolor{currentstroke}{rgb}{0.278791,0.062145,0.386592}%
\pgfsetstrokecolor{currentstroke}%
\pgfsetdash{}{0pt}%
\pgfpathmoveto{\pgfqpoint{4.597906in}{6.290343in}}%
\pgfpathlineto{\pgfqpoint{4.566985in}{6.293243in}}%
\pgfpathlineto{\pgfqpoint{4.587857in}{6.316240in}}%
\pgfpathlineto{\pgfqpoint{4.597906in}{6.290343in}}%
\pgfpathlineto{\pgfqpoint{4.597906in}{6.290343in}}%
\pgfpathclose%
\pgfusepath{stroke,fill}%
\end{pgfscope}%
\begin{pgfscope}%
\pgfpathrectangle{\pgfqpoint{3.352233in}{5.105882in}}{\pgfqpoint{2.407767in}{1.544118in}}%
\pgfusepath{clip}%
\pgfsetroundcap%
\pgfsetroundjoin%
\pgfsetlinewidth{0.501875pt}%
\definecolor{currentstroke}{rgb}{0.282327,0.094955,0.417331}%
\pgfsetstrokecolor{currentstroke}%
\pgfsetdash{}{0pt}%
\pgfpathmoveto{\pgfqpoint{4.544075in}{5.516763in}}%
\pgfpathquadraticcurveto{\pgfqpoint{4.537327in}{5.521004in}}{\pgfqpoint{4.537153in}{5.521113in}}%
\pgfusepath{stroke}%
\end{pgfscope}%
\begin{pgfscope}%
\pgfpathrectangle{\pgfqpoint{3.352233in}{5.105882in}}{\pgfqpoint{2.407767in}{1.544118in}}%
\pgfusepath{clip}%
\pgfsetroundcap%
\pgfsetroundjoin%
\definecolor{currentfill}{rgb}{0.282327,0.094955,0.417331}%
\pgfsetfillcolor{currentfill}%
\pgfsetlinewidth{0.501875pt}%
\definecolor{currentstroke}{rgb}{0.282327,0.094955,0.417331}%
\pgfsetstrokecolor{currentstroke}%
\pgfsetdash{}{0pt}%
\pgfpathmoveto{\pgfqpoint{4.553283in}{5.494574in}}%
\pgfpathlineto{\pgfqpoint{4.537153in}{5.521113in}}%
\pgfpathlineto{\pgfqpoint{4.568063in}{5.518093in}}%
\pgfpathlineto{\pgfqpoint{4.553283in}{5.494574in}}%
\pgfpathlineto{\pgfqpoint{4.553283in}{5.494574in}}%
\pgfpathclose%
\pgfusepath{stroke,fill}%
\end{pgfscope}%
\begin{pgfscope}%
\pgfpathrectangle{\pgfqpoint{3.352233in}{5.105882in}}{\pgfqpoint{2.407767in}{1.544118in}}%
\pgfusepath{clip}%
\pgfsetroundcap%
\pgfsetroundjoin%
\pgfsetlinewidth{0.501875pt}%
\definecolor{currentstroke}{rgb}{0.283229,0.120777,0.440584}%
\pgfsetstrokecolor{currentstroke}%
\pgfsetdash{}{0pt}%
\pgfpathmoveto{\pgfqpoint{4.734500in}{5.532047in}}%
\pgfpathquadraticcurveto{\pgfqpoint{4.721951in}{5.534718in}}{\pgfqpoint{4.716997in}{5.535772in}}%
\pgfusepath{stroke}%
\end{pgfscope}%
\begin{pgfscope}%
\pgfpathrectangle{\pgfqpoint{3.352233in}{5.105882in}}{\pgfqpoint{2.407767in}{1.544118in}}%
\pgfusepath{clip}%
\pgfsetroundcap%
\pgfsetroundjoin%
\definecolor{currentfill}{rgb}{0.283229,0.120777,0.440584}%
\pgfsetfillcolor{currentfill}%
\pgfsetlinewidth{0.501875pt}%
\definecolor{currentstroke}{rgb}{0.283229,0.120777,0.440584}%
\pgfsetstrokecolor{currentstroke}%
\pgfsetdash{}{0pt}%
\pgfpathmoveto{\pgfqpoint{4.741275in}{5.516405in}}%
\pgfpathlineto{\pgfqpoint{4.716997in}{5.535772in}}%
\pgfpathlineto{\pgfqpoint{4.747057in}{5.543575in}}%
\pgfpathlineto{\pgfqpoint{4.741275in}{5.516405in}}%
\pgfpathlineto{\pgfqpoint{4.741275in}{5.516405in}}%
\pgfpathclose%
\pgfusepath{stroke,fill}%
\end{pgfscope}%
\begin{pgfscope}%
\pgfpathrectangle{\pgfqpoint{3.352233in}{5.105882in}}{\pgfqpoint{2.407767in}{1.544118in}}%
\pgfusepath{clip}%
\pgfsetroundcap%
\pgfsetroundjoin%
\pgfsetlinewidth{0.501875pt}%
\definecolor{currentstroke}{rgb}{0.281924,0.089666,0.412415}%
\pgfsetstrokecolor{currentstroke}%
\pgfsetdash{}{0pt}%
\pgfpathmoveto{\pgfqpoint{4.889423in}{5.544098in}}%
\pgfpathquadraticcurveto{\pgfqpoint{4.876366in}{5.545493in}}{\pgfqpoint{4.871030in}{5.546063in}}%
\pgfusepath{stroke}%
\end{pgfscope}%
\begin{pgfscope}%
\pgfpathrectangle{\pgfqpoint{3.352233in}{5.105882in}}{\pgfqpoint{2.407767in}{1.544118in}}%
\pgfusepath{clip}%
\pgfsetroundcap%
\pgfsetroundjoin%
\definecolor{currentfill}{rgb}{0.281924,0.089666,0.412415}%
\pgfsetfillcolor{currentfill}%
\pgfsetlinewidth{0.501875pt}%
\definecolor{currentstroke}{rgb}{0.281924,0.089666,0.412415}%
\pgfsetstrokecolor{currentstroke}%
\pgfsetdash{}{0pt}%
\pgfpathmoveto{\pgfqpoint{4.897175in}{5.529302in}}%
\pgfpathlineto{\pgfqpoint{4.871030in}{5.546063in}}%
\pgfpathlineto{\pgfqpoint{4.900125in}{5.556923in}}%
\pgfpathlineto{\pgfqpoint{4.897175in}{5.529302in}}%
\pgfpathlineto{\pgfqpoint{4.897175in}{5.529302in}}%
\pgfpathclose%
\pgfusepath{stroke,fill}%
\end{pgfscope}%
\begin{pgfscope}%
\pgfpathrectangle{\pgfqpoint{3.352233in}{5.105882in}}{\pgfqpoint{2.407767in}{1.544118in}}%
\pgfusepath{clip}%
\pgfsetroundcap%
\pgfsetroundjoin%
\pgfsetlinewidth{0.501875pt}%
\definecolor{currentstroke}{rgb}{0.282623,0.140926,0.457517}%
\pgfsetstrokecolor{currentstroke}%
\pgfsetdash{}{0pt}%
\pgfpathmoveto{\pgfqpoint{4.941903in}{5.574356in}}%
\pgfpathquadraticcurveto{\pgfqpoint{4.928747in}{5.575320in}}{\pgfqpoint{4.923333in}{5.575717in}}%
\pgfusepath{stroke}%
\end{pgfscope}%
\begin{pgfscope}%
\pgfpathrectangle{\pgfqpoint{3.352233in}{5.105882in}}{\pgfqpoint{2.407767in}{1.544118in}}%
\pgfusepath{clip}%
\pgfsetroundcap%
\pgfsetroundjoin%
\definecolor{currentfill}{rgb}{0.282623,0.140926,0.457517}%
\pgfsetfillcolor{currentfill}%
\pgfsetlinewidth{0.501875pt}%
\definecolor{currentstroke}{rgb}{0.282623,0.140926,0.457517}%
\pgfsetstrokecolor{currentstroke}%
\pgfsetdash{}{0pt}%
\pgfpathmoveto{\pgfqpoint{4.950022in}{5.559835in}}%
\pgfpathlineto{\pgfqpoint{4.923333in}{5.575717in}}%
\pgfpathlineto{\pgfqpoint{4.952052in}{5.587538in}}%
\pgfpathlineto{\pgfqpoint{4.950022in}{5.559835in}}%
\pgfpathlineto{\pgfqpoint{4.950022in}{5.559835in}}%
\pgfpathclose%
\pgfusepath{stroke,fill}%
\end{pgfscope}%
\begin{pgfscope}%
\pgfpathrectangle{\pgfqpoint{3.352233in}{5.105882in}}{\pgfqpoint{2.407767in}{1.544118in}}%
\pgfusepath{clip}%
\pgfsetroundcap%
\pgfsetroundjoin%
\pgfsetlinewidth{0.501875pt}%
\definecolor{currentstroke}{rgb}{0.283197,0.115680,0.436115}%
\pgfsetstrokecolor{currentstroke}%
\pgfsetdash{}{0pt}%
\pgfpathmoveto{\pgfqpoint{5.047433in}{5.602975in}}%
\pgfpathquadraticcurveto{\pgfqpoint{5.034219in}{5.603520in}}{\pgfqpoint{5.028762in}{5.603745in}}%
\pgfusepath{stroke}%
\end{pgfscope}%
\begin{pgfscope}%
\pgfpathrectangle{\pgfqpoint{3.352233in}{5.105882in}}{\pgfqpoint{2.407767in}{1.544118in}}%
\pgfusepath{clip}%
\pgfsetroundcap%
\pgfsetroundjoin%
\definecolor{currentfill}{rgb}{0.283197,0.115680,0.436115}%
\pgfsetfillcolor{currentfill}%
\pgfsetlinewidth{0.501875pt}%
\definecolor{currentstroke}{rgb}{0.283197,0.115680,0.436115}%
\pgfsetstrokecolor{currentstroke}%
\pgfsetdash{}{0pt}%
\pgfpathmoveto{\pgfqpoint{5.055944in}{5.588723in}}%
\pgfpathlineto{\pgfqpoint{5.028762in}{5.603745in}}%
\pgfpathlineto{\pgfqpoint{5.057089in}{5.616477in}}%
\pgfpathlineto{\pgfqpoint{5.055944in}{5.588723in}}%
\pgfpathlineto{\pgfqpoint{5.055944in}{5.588723in}}%
\pgfpathclose%
\pgfusepath{stroke,fill}%
\end{pgfscope}%
\begin{pgfscope}%
\pgfpathrectangle{\pgfqpoint{3.352233in}{5.105882in}}{\pgfqpoint{2.407767in}{1.544118in}}%
\pgfusepath{clip}%
\pgfsetroundcap%
\pgfsetroundjoin%
\pgfsetlinewidth{0.501875pt}%
\definecolor{currentstroke}{rgb}{0.265145,0.232956,0.516599}%
\pgfsetstrokecolor{currentstroke}%
\pgfsetdash{}{0pt}%
\pgfpathmoveto{\pgfqpoint{4.784001in}{5.654536in}}%
\pgfpathquadraticcurveto{\pgfqpoint{4.770959in}{5.656010in}}{\pgfqpoint{4.765632in}{5.656612in}}%
\pgfusepath{stroke}%
\end{pgfscope}%
\begin{pgfscope}%
\pgfpathrectangle{\pgfqpoint{3.352233in}{5.105882in}}{\pgfqpoint{2.407767in}{1.544118in}}%
\pgfusepath{clip}%
\pgfsetroundcap%
\pgfsetroundjoin%
\definecolor{currentfill}{rgb}{0.265145,0.232956,0.516599}%
\pgfsetfillcolor{currentfill}%
\pgfsetlinewidth{0.501875pt}%
\definecolor{currentstroke}{rgb}{0.265145,0.232956,0.516599}%
\pgfsetstrokecolor{currentstroke}%
\pgfsetdash{}{0pt}%
\pgfpathmoveto{\pgfqpoint{4.791674in}{5.639691in}}%
\pgfpathlineto{\pgfqpoint{4.765632in}{5.656612in}}%
\pgfpathlineto{\pgfqpoint{4.794794in}{5.667293in}}%
\pgfpathlineto{\pgfqpoint{4.791674in}{5.639691in}}%
\pgfpathlineto{\pgfqpoint{4.791674in}{5.639691in}}%
\pgfpathclose%
\pgfusepath{stroke,fill}%
\end{pgfscope}%
\begin{pgfscope}%
\pgfpathrectangle{\pgfqpoint{3.352233in}{5.105882in}}{\pgfqpoint{2.407767in}{1.544118in}}%
\pgfusepath{clip}%
\pgfsetroundcap%
\pgfsetroundjoin%
\pgfsetlinewidth{0.501875pt}%
\definecolor{currentstroke}{rgb}{0.260571,0.246922,0.522828}%
\pgfsetstrokecolor{currentstroke}%
\pgfsetdash{}{0pt}%
\pgfpathmoveto{\pgfqpoint{4.836261in}{5.683525in}}%
\pgfpathquadraticcurveto{\pgfqpoint{4.823130in}{5.684624in}}{\pgfqpoint{4.817736in}{5.685076in}}%
\pgfusepath{stroke}%
\end{pgfscope}%
\begin{pgfscope}%
\pgfpathrectangle{\pgfqpoint{3.352233in}{5.105882in}}{\pgfqpoint{2.407767in}{1.544118in}}%
\pgfusepath{clip}%
\pgfsetroundcap%
\pgfsetroundjoin%
\definecolor{currentfill}{rgb}{0.260571,0.246922,0.522828}%
\pgfsetfillcolor{currentfill}%
\pgfsetlinewidth{0.501875pt}%
\definecolor{currentstroke}{rgb}{0.260571,0.246922,0.522828}%
\pgfsetstrokecolor{currentstroke}%
\pgfsetdash{}{0pt}%
\pgfpathmoveto{\pgfqpoint{4.844258in}{5.668918in}}%
\pgfpathlineto{\pgfqpoint{4.817736in}{5.685076in}}%
\pgfpathlineto{\pgfqpoint{4.846575in}{5.696599in}}%
\pgfpathlineto{\pgfqpoint{4.844258in}{5.668918in}}%
\pgfpathlineto{\pgfqpoint{4.844258in}{5.668918in}}%
\pgfpathclose%
\pgfusepath{stroke,fill}%
\end{pgfscope}%
\begin{pgfscope}%
\pgfpathrectangle{\pgfqpoint{3.352233in}{5.105882in}}{\pgfqpoint{2.407767in}{1.544118in}}%
\pgfusepath{clip}%
\pgfsetroundcap%
\pgfsetroundjoin%
\pgfsetlinewidth{0.501875pt}%
\definecolor{currentstroke}{rgb}{0.277134,0.185228,0.489898}%
\pgfsetstrokecolor{currentstroke}%
\pgfsetdash{}{0pt}%
\pgfpathmoveto{\pgfqpoint{5.047413in}{5.706874in}}%
\pgfpathquadraticcurveto{\pgfqpoint{5.034183in}{5.707261in}}{\pgfqpoint{5.028714in}{5.707421in}}%
\pgfusepath{stroke}%
\end{pgfscope}%
\begin{pgfscope}%
\pgfpathrectangle{\pgfqpoint{3.352233in}{5.105882in}}{\pgfqpoint{2.407767in}{1.544118in}}%
\pgfusepath{clip}%
\pgfsetroundcap%
\pgfsetroundjoin%
\definecolor{currentfill}{rgb}{0.277134,0.185228,0.489898}%
\pgfsetfillcolor{currentfill}%
\pgfsetlinewidth{0.501875pt}%
\definecolor{currentstroke}{rgb}{0.277134,0.185228,0.489898}%
\pgfsetstrokecolor{currentstroke}%
\pgfsetdash{}{0pt}%
\pgfpathmoveto{\pgfqpoint{5.056073in}{5.692725in}}%
\pgfpathlineto{\pgfqpoint{5.028714in}{5.707421in}}%
\pgfpathlineto{\pgfqpoint{5.056886in}{5.720491in}}%
\pgfpathlineto{\pgfqpoint{5.056073in}{5.692725in}}%
\pgfpathlineto{\pgfqpoint{5.056073in}{5.692725in}}%
\pgfpathclose%
\pgfusepath{stroke,fill}%
\end{pgfscope}%
\begin{pgfscope}%
\pgfpathrectangle{\pgfqpoint{3.352233in}{5.105882in}}{\pgfqpoint{2.407767in}{1.544118in}}%
\pgfusepath{clip}%
\pgfsetroundcap%
\pgfsetroundjoin%
\pgfsetlinewidth{0.501875pt}%
\definecolor{currentstroke}{rgb}{0.206756,0.371758,0.553117}%
\pgfsetstrokecolor{currentstroke}%
\pgfsetdash{}{0pt}%
\pgfpathmoveto{\pgfqpoint{4.783010in}{5.750909in}}%
\pgfpathquadraticcurveto{\pgfqpoint{4.769834in}{5.751763in}}{\pgfqpoint{4.764406in}{5.752115in}}%
\pgfusepath{stroke}%
\end{pgfscope}%
\begin{pgfscope}%
\pgfpathrectangle{\pgfqpoint{3.352233in}{5.105882in}}{\pgfqpoint{2.407767in}{1.544118in}}%
\pgfusepath{clip}%
\pgfsetroundcap%
\pgfsetroundjoin%
\definecolor{currentfill}{rgb}{0.206756,0.371758,0.553117}%
\pgfsetfillcolor{currentfill}%
\pgfsetlinewidth{0.501875pt}%
\definecolor{currentstroke}{rgb}{0.206756,0.371758,0.553117}%
\pgfsetstrokecolor{currentstroke}%
\pgfsetdash{}{0pt}%
\pgfpathmoveto{\pgfqpoint{4.791227in}{5.736458in}}%
\pgfpathlineto{\pgfqpoint{4.764406in}{5.752115in}}%
\pgfpathlineto{\pgfqpoint{4.793023in}{5.764178in}}%
\pgfpathlineto{\pgfqpoint{4.791227in}{5.736458in}}%
\pgfpathlineto{\pgfqpoint{4.791227in}{5.736458in}}%
\pgfpathclose%
\pgfusepath{stroke,fill}%
\end{pgfscope}%
\begin{pgfscope}%
\pgfpathrectangle{\pgfqpoint{3.352233in}{5.105882in}}{\pgfqpoint{2.407767in}{1.544118in}}%
\pgfusepath{clip}%
\pgfsetroundcap%
\pgfsetroundjoin%
\pgfsetlinewidth{0.501875pt}%
\definecolor{currentstroke}{rgb}{0.253935,0.265254,0.529983}%
\pgfsetstrokecolor{currentstroke}%
\pgfsetdash{}{0pt}%
\pgfpathmoveto{\pgfqpoint{4.994416in}{5.775946in}}%
\pgfpathquadraticcurveto{\pgfqpoint{4.981182in}{5.776278in}}{\pgfqpoint{4.975710in}{5.776416in}}%
\pgfusepath{stroke}%
\end{pgfscope}%
\begin{pgfscope}%
\pgfpathrectangle{\pgfqpoint{3.352233in}{5.105882in}}{\pgfqpoint{2.407767in}{1.544118in}}%
\pgfusepath{clip}%
\pgfsetroundcap%
\pgfsetroundjoin%
\definecolor{currentfill}{rgb}{0.253935,0.265254,0.529983}%
\pgfsetfillcolor{currentfill}%
\pgfsetlinewidth{0.501875pt}%
\definecolor{currentstroke}{rgb}{0.253935,0.265254,0.529983}%
\pgfsetstrokecolor{currentstroke}%
\pgfsetdash{}{0pt}%
\pgfpathmoveto{\pgfqpoint{5.003130in}{5.761834in}}%
\pgfpathlineto{\pgfqpoint{4.975710in}{5.776416in}}%
\pgfpathlineto{\pgfqpoint{5.003828in}{5.789603in}}%
\pgfpathlineto{\pgfqpoint{5.003130in}{5.761834in}}%
\pgfpathlineto{\pgfqpoint{5.003130in}{5.761834in}}%
\pgfpathclose%
\pgfusepath{stroke,fill}%
\end{pgfscope}%
\begin{pgfscope}%
\pgfpathrectangle{\pgfqpoint{3.352233in}{5.105882in}}{\pgfqpoint{2.407767in}{1.544118in}}%
\pgfusepath{clip}%
\pgfsetroundcap%
\pgfsetroundjoin%
\pgfsetlinewidth{0.501875pt}%
\definecolor{currentstroke}{rgb}{0.175841,0.441290,0.557685}%
\pgfsetstrokecolor{currentstroke}%
\pgfsetdash{}{0pt}%
\pgfpathmoveto{\pgfqpoint{4.782618in}{5.814604in}}%
\pgfpathquadraticcurveto{\pgfqpoint{4.769391in}{5.815031in}}{\pgfqpoint{4.763924in}{5.815208in}}%
\pgfusepath{stroke}%
\end{pgfscope}%
\begin{pgfscope}%
\pgfpathrectangle{\pgfqpoint{3.352233in}{5.105882in}}{\pgfqpoint{2.407767in}{1.544118in}}%
\pgfusepath{clip}%
\pgfsetroundcap%
\pgfsetroundjoin%
\definecolor{currentfill}{rgb}{0.175841,0.441290,0.557685}%
\pgfsetfillcolor{currentfill}%
\pgfsetlinewidth{0.501875pt}%
\definecolor{currentstroke}{rgb}{0.175841,0.441290,0.557685}%
\pgfsetstrokecolor{currentstroke}%
\pgfsetdash{}{0pt}%
\pgfpathmoveto{\pgfqpoint{4.791239in}{5.800430in}}%
\pgfpathlineto{\pgfqpoint{4.763924in}{5.815208in}}%
\pgfpathlineto{\pgfqpoint{4.792136in}{5.828193in}}%
\pgfpathlineto{\pgfqpoint{4.791239in}{5.800430in}}%
\pgfpathlineto{\pgfqpoint{4.791239in}{5.800430in}}%
\pgfpathclose%
\pgfusepath{stroke,fill}%
\end{pgfscope}%
\begin{pgfscope}%
\pgfpathrectangle{\pgfqpoint{3.352233in}{5.105882in}}{\pgfqpoint{2.407767in}{1.544118in}}%
\pgfusepath{clip}%
\pgfsetroundcap%
\pgfsetroundjoin%
\pgfsetlinewidth{0.501875pt}%
\definecolor{currentstroke}{rgb}{0.197636,0.391528,0.554969}%
\pgfsetstrokecolor{currentstroke}%
\pgfsetdash{}{0pt}%
\pgfpathmoveto{\pgfqpoint{4.888450in}{5.845533in}}%
\pgfpathquadraticcurveto{\pgfqpoint{4.875209in}{5.845726in}}{\pgfqpoint{4.869732in}{5.845806in}}%
\pgfusepath{stroke}%
\end{pgfscope}%
\begin{pgfscope}%
\pgfpathrectangle{\pgfqpoint{3.352233in}{5.105882in}}{\pgfqpoint{2.407767in}{1.544118in}}%
\pgfusepath{clip}%
\pgfsetroundcap%
\pgfsetroundjoin%
\definecolor{currentfill}{rgb}{0.197636,0.391528,0.554969}%
\pgfsetfillcolor{currentfill}%
\pgfsetlinewidth{0.501875pt}%
\definecolor{currentstroke}{rgb}{0.197636,0.391528,0.554969}%
\pgfsetstrokecolor{currentstroke}%
\pgfsetdash{}{0pt}%
\pgfpathmoveto{\pgfqpoint{4.897304in}{5.831513in}}%
\pgfpathlineto{\pgfqpoint{4.869732in}{5.845806in}}%
\pgfpathlineto{\pgfqpoint{4.897709in}{5.859288in}}%
\pgfpathlineto{\pgfqpoint{4.897304in}{5.831513in}}%
\pgfpathlineto{\pgfqpoint{4.897304in}{5.831513in}}%
\pgfpathclose%
\pgfusepath{stroke,fill}%
\end{pgfscope}%
\begin{pgfscope}%
\pgfpathrectangle{\pgfqpoint{3.352233in}{5.105882in}}{\pgfqpoint{2.407767in}{1.544118in}}%
\pgfusepath{clip}%
\pgfsetroundcap%
\pgfsetroundjoin%
\pgfsetlinewidth{0.501875pt}%
\definecolor{currentstroke}{rgb}{0.175841,0.441290,0.557685}%
\pgfsetstrokecolor{currentstroke}%
\pgfsetdash{}{0pt}%
\pgfpathmoveto{\pgfqpoint{4.835451in}{5.877695in}}%
\pgfpathquadraticcurveto{\pgfqpoint{4.822208in}{5.877646in}}{\pgfqpoint{4.816728in}{5.877626in}}%
\pgfusepath{stroke}%
\end{pgfscope}%
\begin{pgfscope}%
\pgfpathrectangle{\pgfqpoint{3.352233in}{5.105882in}}{\pgfqpoint{2.407767in}{1.544118in}}%
\pgfusepath{clip}%
\pgfsetroundcap%
\pgfsetroundjoin%
\definecolor{currentfill}{rgb}{0.175841,0.441290,0.557685}%
\pgfsetfillcolor{currentfill}%
\pgfsetlinewidth{0.501875pt}%
\definecolor{currentstroke}{rgb}{0.175841,0.441290,0.557685}%
\pgfsetstrokecolor{currentstroke}%
\pgfsetdash{}{0pt}%
\pgfpathmoveto{\pgfqpoint{4.844557in}{5.863840in}}%
\pgfpathlineto{\pgfqpoint{4.816728in}{5.877626in}}%
\pgfpathlineto{\pgfqpoint{4.844454in}{5.891618in}}%
\pgfpathlineto{\pgfqpoint{4.844557in}{5.863840in}}%
\pgfpathlineto{\pgfqpoint{4.844557in}{5.863840in}}%
\pgfpathclose%
\pgfusepath{stroke,fill}%
\end{pgfscope}%
\begin{pgfscope}%
\pgfpathrectangle{\pgfqpoint{3.352233in}{5.105882in}}{\pgfqpoint{2.407767in}{1.544118in}}%
\pgfusepath{clip}%
\pgfsetroundcap%
\pgfsetroundjoin%
\pgfsetlinewidth{0.501875pt}%
\definecolor{currentstroke}{rgb}{0.174274,0.445044,0.557792}%
\pgfsetstrokecolor{currentstroke}%
\pgfsetdash{}{0pt}%
\pgfpathmoveto{\pgfqpoint{4.782521in}{5.909403in}}%
\pgfpathquadraticcurveto{\pgfqpoint{4.769283in}{5.909146in}}{\pgfqpoint{4.763808in}{5.909040in}}%
\pgfusepath{stroke}%
\end{pgfscope}%
\begin{pgfscope}%
\pgfpathrectangle{\pgfqpoint{3.352233in}{5.105882in}}{\pgfqpoint{2.407767in}{1.544118in}}%
\pgfusepath{clip}%
\pgfsetroundcap%
\pgfsetroundjoin%
\definecolor{currentfill}{rgb}{0.174274,0.445044,0.557792}%
\pgfsetfillcolor{currentfill}%
\pgfsetlinewidth{0.501875pt}%
\definecolor{currentstroke}{rgb}{0.174274,0.445044,0.557792}%
\pgfsetstrokecolor{currentstroke}%
\pgfsetdash{}{0pt}%
\pgfpathmoveto{\pgfqpoint{4.791849in}{5.895692in}}%
\pgfpathlineto{\pgfqpoint{4.763808in}{5.909040in}}%
\pgfpathlineto{\pgfqpoint{4.791311in}{5.923465in}}%
\pgfpathlineto{\pgfqpoint{4.791849in}{5.895692in}}%
\pgfpathlineto{\pgfqpoint{4.791849in}{5.895692in}}%
\pgfpathclose%
\pgfusepath{stroke,fill}%
\end{pgfscope}%
\begin{pgfscope}%
\pgfpathrectangle{\pgfqpoint{3.352233in}{5.105882in}}{\pgfqpoint{2.407767in}{1.544118in}}%
\pgfusepath{clip}%
\pgfsetroundcap%
\pgfsetroundjoin%
\pgfsetlinewidth{0.501875pt}%
\definecolor{currentstroke}{rgb}{0.201239,0.383670,0.554294}%
\pgfsetstrokecolor{currentstroke}%
\pgfsetdash{}{0pt}%
\pgfpathmoveto{\pgfqpoint{4.888458in}{5.944698in}}%
\pgfpathquadraticcurveto{\pgfqpoint{4.875220in}{5.944442in}}{\pgfqpoint{4.869745in}{5.944336in}}%
\pgfusepath{stroke}%
\end{pgfscope}%
\begin{pgfscope}%
\pgfpathrectangle{\pgfqpoint{3.352233in}{5.105882in}}{\pgfqpoint{2.407767in}{1.544118in}}%
\pgfusepath{clip}%
\pgfsetroundcap%
\pgfsetroundjoin%
\definecolor{currentfill}{rgb}{0.201239,0.383670,0.554294}%
\pgfsetfillcolor{currentfill}%
\pgfsetlinewidth{0.501875pt}%
\definecolor{currentstroke}{rgb}{0.201239,0.383670,0.554294}%
\pgfsetstrokecolor{currentstroke}%
\pgfsetdash{}{0pt}%
\pgfpathmoveto{\pgfqpoint{4.897786in}{5.930987in}}%
\pgfpathlineto{\pgfqpoint{4.869745in}{5.944336in}}%
\pgfpathlineto{\pgfqpoint{4.897249in}{5.958759in}}%
\pgfpathlineto{\pgfqpoint{4.897786in}{5.930987in}}%
\pgfpathlineto{\pgfqpoint{4.897786in}{5.930987in}}%
\pgfpathclose%
\pgfusepath{stroke,fill}%
\end{pgfscope}%
\begin{pgfscope}%
\pgfpathrectangle{\pgfqpoint{3.352233in}{5.105882in}}{\pgfqpoint{2.407767in}{1.544118in}}%
\pgfusepath{clip}%
\pgfsetroundcap%
\pgfsetroundjoin%
\pgfsetlinewidth{0.501875pt}%
\definecolor{currentstroke}{rgb}{0.235526,0.309527,0.542944}%
\pgfsetstrokecolor{currentstroke}%
\pgfsetdash{}{0pt}%
\pgfpathmoveto{\pgfqpoint{4.941488in}{5.978203in}}%
\pgfpathquadraticcurveto{\pgfqpoint{4.928257in}{5.977833in}}{\pgfqpoint{4.922787in}{5.977680in}}%
\pgfusepath{stroke}%
\end{pgfscope}%
\begin{pgfscope}%
\pgfpathrectangle{\pgfqpoint{3.352233in}{5.105882in}}{\pgfqpoint{2.407767in}{1.544118in}}%
\pgfusepath{clip}%
\pgfsetroundcap%
\pgfsetroundjoin%
\definecolor{currentfill}{rgb}{0.235526,0.309527,0.542944}%
\pgfsetfillcolor{currentfill}%
\pgfsetlinewidth{0.501875pt}%
\definecolor{currentstroke}{rgb}{0.235526,0.309527,0.542944}%
\pgfsetstrokecolor{currentstroke}%
\pgfsetdash{}{0pt}%
\pgfpathmoveto{\pgfqpoint{4.950942in}{5.964573in}}%
\pgfpathlineto{\pgfqpoint{4.922787in}{5.977680in}}%
\pgfpathlineto{\pgfqpoint{4.950165in}{5.992340in}}%
\pgfpathlineto{\pgfqpoint{4.950942in}{5.964573in}}%
\pgfpathlineto{\pgfqpoint{4.950942in}{5.964573in}}%
\pgfpathclose%
\pgfusepath{stroke,fill}%
\end{pgfscope}%
\begin{pgfscope}%
\pgfpathrectangle{\pgfqpoint{3.352233in}{5.105882in}}{\pgfqpoint{2.407767in}{1.544118in}}%
\pgfusepath{clip}%
\pgfsetroundcap%
\pgfsetroundjoin%
\pgfsetlinewidth{0.501875pt}%
\definecolor{currentstroke}{rgb}{0.267968,0.223549,0.512008}%
\pgfsetstrokecolor{currentstroke}%
\pgfsetdash{}{0pt}%
\pgfpathmoveto{\pgfqpoint{4.994452in}{6.013912in}}%
\pgfpathquadraticcurveto{\pgfqpoint{4.981227in}{6.013467in}}{\pgfqpoint{4.975762in}{6.013284in}}%
\pgfusepath{stroke}%
\end{pgfscope}%
\begin{pgfscope}%
\pgfpathrectangle{\pgfqpoint{3.352233in}{5.105882in}}{\pgfqpoint{2.407767in}{1.544118in}}%
\pgfusepath{clip}%
\pgfsetroundcap%
\pgfsetroundjoin%
\definecolor{currentfill}{rgb}{0.267968,0.223549,0.512008}%
\pgfsetfillcolor{currentfill}%
\pgfsetlinewidth{0.501875pt}%
\definecolor{currentstroke}{rgb}{0.267968,0.223549,0.512008}%
\pgfsetstrokecolor{currentstroke}%
\pgfsetdash{}{0pt}%
\pgfpathmoveto{\pgfqpoint{5.003991in}{6.000336in}}%
\pgfpathlineto{\pgfqpoint{4.975762in}{6.013284in}}%
\pgfpathlineto{\pgfqpoint{5.003058in}{6.028098in}}%
\pgfpathlineto{\pgfqpoint{5.003991in}{6.000336in}}%
\pgfpathlineto{\pgfqpoint{5.003991in}{6.000336in}}%
\pgfpathclose%
\pgfusepath{stroke,fill}%
\end{pgfscope}%
\begin{pgfscope}%
\pgfpathrectangle{\pgfqpoint{3.352233in}{5.105882in}}{\pgfqpoint{2.407767in}{1.544118in}}%
\pgfusepath{clip}%
\pgfsetroundcap%
\pgfsetroundjoin%
\pgfsetlinewidth{0.501875pt}%
\definecolor{currentstroke}{rgb}{0.267968,0.223549,0.512008}%
\pgfsetstrokecolor{currentstroke}%
\pgfsetdash{}{0pt}%
\pgfpathmoveto{\pgfqpoint{4.994513in}{6.047221in}}%
\pgfpathquadraticcurveto{\pgfqpoint{4.981294in}{6.046708in}}{\pgfqpoint{4.975832in}{6.046495in}}%
\pgfusepath{stroke}%
\end{pgfscope}%
\begin{pgfscope}%
\pgfpathrectangle{\pgfqpoint{3.352233in}{5.105882in}}{\pgfqpoint{2.407767in}{1.544118in}}%
\pgfusepath{clip}%
\pgfsetroundcap%
\pgfsetroundjoin%
\definecolor{currentfill}{rgb}{0.267968,0.223549,0.512008}%
\pgfsetfillcolor{currentfill}%
\pgfsetlinewidth{0.501875pt}%
\definecolor{currentstroke}{rgb}{0.267968,0.223549,0.512008}%
\pgfsetstrokecolor{currentstroke}%
\pgfsetdash{}{0pt}%
\pgfpathmoveto{\pgfqpoint{5.004129in}{6.033696in}}%
\pgfpathlineto{\pgfqpoint{4.975832in}{6.046495in}}%
\pgfpathlineto{\pgfqpoint{5.003050in}{6.061453in}}%
\pgfpathlineto{\pgfqpoint{5.004129in}{6.033696in}}%
\pgfpathlineto{\pgfqpoint{5.004129in}{6.033696in}}%
\pgfpathclose%
\pgfusepath{stroke,fill}%
\end{pgfscope}%
\begin{pgfscope}%
\pgfpathrectangle{\pgfqpoint{3.352233in}{5.105882in}}{\pgfqpoint{2.407767in}{1.544118in}}%
\pgfusepath{clip}%
\pgfsetroundcap%
\pgfsetroundjoin%
\pgfsetlinewidth{0.501875pt}%
\definecolor{currentstroke}{rgb}{0.280868,0.160771,0.472899}%
\pgfsetstrokecolor{currentstroke}%
\pgfsetdash{}{0pt}%
\pgfpathmoveto{\pgfqpoint{5.047437in}{6.083220in}}%
\pgfpathquadraticcurveto{\pgfqpoint{5.034212in}{6.082774in}}{\pgfqpoint{5.028746in}{6.082589in}}%
\pgfusepath{stroke}%
\end{pgfscope}%
\begin{pgfscope}%
\pgfpathrectangle{\pgfqpoint{3.352233in}{5.105882in}}{\pgfqpoint{2.407767in}{1.544118in}}%
\pgfusepath{clip}%
\pgfsetroundcap%
\pgfsetroundjoin%
\definecolor{currentfill}{rgb}{0.280868,0.160771,0.472899}%
\pgfsetfillcolor{currentfill}%
\pgfsetlinewidth{0.501875pt}%
\definecolor{currentstroke}{rgb}{0.280868,0.160771,0.472899}%
\pgfsetstrokecolor{currentstroke}%
\pgfsetdash{}{0pt}%
\pgfpathmoveto{\pgfqpoint{5.056977in}{6.069646in}}%
\pgfpathlineto{\pgfqpoint{5.028746in}{6.082589in}}%
\pgfpathlineto{\pgfqpoint{5.056039in}{6.097408in}}%
\pgfpathlineto{\pgfqpoint{5.056977in}{6.069646in}}%
\pgfpathlineto{\pgfqpoint{5.056977in}{6.069646in}}%
\pgfpathclose%
\pgfusepath{stroke,fill}%
\end{pgfscope}%
\begin{pgfscope}%
\pgfpathrectangle{\pgfqpoint{3.352233in}{5.105882in}}{\pgfqpoint{2.407767in}{1.544118in}}%
\pgfusepath{clip}%
\pgfsetroundcap%
\pgfsetroundjoin%
\pgfsetlinewidth{0.501875pt}%
\definecolor{currentstroke}{rgb}{0.269308,0.218818,0.509577}%
\pgfsetstrokecolor{currentstroke}%
\pgfsetdash{}{0pt}%
\pgfpathmoveto{\pgfqpoint{4.784027in}{6.100680in}}%
\pgfpathquadraticcurveto{\pgfqpoint{4.770964in}{6.099287in}}{\pgfqpoint{4.765622in}{6.098717in}}%
\pgfusepath{stroke}%
\end{pgfscope}%
\begin{pgfscope}%
\pgfpathrectangle{\pgfqpoint{3.352233in}{5.105882in}}{\pgfqpoint{2.407767in}{1.544118in}}%
\pgfusepath{clip}%
\pgfsetroundcap%
\pgfsetroundjoin%
\definecolor{currentfill}{rgb}{0.269308,0.218818,0.509577}%
\pgfsetfillcolor{currentfill}%
\pgfsetlinewidth{0.501875pt}%
\definecolor{currentstroke}{rgb}{0.269308,0.218818,0.509577}%
\pgfsetstrokecolor{currentstroke}%
\pgfsetdash{}{0pt}%
\pgfpathmoveto{\pgfqpoint{4.794716in}{6.087852in}}%
\pgfpathlineto{\pgfqpoint{4.765622in}{6.098717in}}%
\pgfpathlineto{\pgfqpoint{4.791771in}{6.115474in}}%
\pgfpathlineto{\pgfqpoint{4.794716in}{6.087852in}}%
\pgfpathlineto{\pgfqpoint{4.794716in}{6.087852in}}%
\pgfpathclose%
\pgfusepath{stroke,fill}%
\end{pgfscope}%
\begin{pgfscope}%
\pgfpathrectangle{\pgfqpoint{3.352233in}{5.105882in}}{\pgfqpoint{2.407767in}{1.544118in}}%
\pgfusepath{clip}%
\pgfsetroundcap%
\pgfsetroundjoin%
\pgfsetlinewidth{0.501875pt}%
\definecolor{currentstroke}{rgb}{0.283091,0.110553,0.431554}%
\pgfsetstrokecolor{currentstroke}%
\pgfsetdash{}{0pt}%
\pgfpathmoveto{\pgfqpoint{5.047464in}{6.152252in}}%
\pgfpathquadraticcurveto{\pgfqpoint{5.034254in}{6.151666in}}{\pgfqpoint{5.028800in}{6.151424in}}%
\pgfusepath{stroke}%
\end{pgfscope}%
\begin{pgfscope}%
\pgfpathrectangle{\pgfqpoint{3.352233in}{5.105882in}}{\pgfqpoint{2.407767in}{1.544118in}}%
\pgfusepath{clip}%
\pgfsetroundcap%
\pgfsetroundjoin%
\definecolor{currentfill}{rgb}{0.283091,0.110553,0.431554}%
\pgfsetfillcolor{currentfill}%
\pgfsetlinewidth{0.501875pt}%
\definecolor{currentstroke}{rgb}{0.283091,0.110553,0.431554}%
\pgfsetstrokecolor{currentstroke}%
\pgfsetdash{}{0pt}%
\pgfpathmoveto{\pgfqpoint{5.057166in}{6.138780in}}%
\pgfpathlineto{\pgfqpoint{5.028800in}{6.151424in}}%
\pgfpathlineto{\pgfqpoint{5.055934in}{6.166531in}}%
\pgfpathlineto{\pgfqpoint{5.057166in}{6.138780in}}%
\pgfpathlineto{\pgfqpoint{5.057166in}{6.138780in}}%
\pgfpathclose%
\pgfusepath{stroke,fill}%
\end{pgfscope}%
\begin{pgfscope}%
\pgfpathrectangle{\pgfqpoint{3.352233in}{5.105882in}}{\pgfqpoint{2.407767in}{1.544118in}}%
\pgfusepath{clip}%
\pgfsetroundcap%
\pgfsetroundjoin%
\pgfsetlinewidth{0.501875pt}%
\definecolor{currentstroke}{rgb}{0.277134,0.185228,0.489898}%
\pgfsetstrokecolor{currentstroke}%
\pgfsetdash{}{0pt}%
\pgfpathmoveto{\pgfqpoint{4.682202in}{6.149331in}}%
\pgfpathquadraticcurveto{\pgfqpoint{4.669596in}{6.146736in}}{\pgfqpoint{4.664595in}{6.145707in}}%
\pgfusepath{stroke}%
\end{pgfscope}%
\begin{pgfscope}%
\pgfpathrectangle{\pgfqpoint{3.352233in}{5.105882in}}{\pgfqpoint{2.407767in}{1.544118in}}%
\pgfusepath{clip}%
\pgfsetroundcap%
\pgfsetroundjoin%
\definecolor{currentfill}{rgb}{0.277134,0.185228,0.489898}%
\pgfsetfillcolor{currentfill}%
\pgfsetlinewidth{0.501875pt}%
\definecolor{currentstroke}{rgb}{0.277134,0.185228,0.489898}%
\pgfsetstrokecolor{currentstroke}%
\pgfsetdash{}{0pt}%
\pgfpathmoveto{\pgfqpoint{4.694603in}{6.137703in}}%
\pgfpathlineto{\pgfqpoint{4.664595in}{6.145707in}}%
\pgfpathlineto{\pgfqpoint{4.689003in}{6.164911in}}%
\pgfpathlineto{\pgfqpoint{4.694603in}{6.137703in}}%
\pgfpathlineto{\pgfqpoint{4.694603in}{6.137703in}}%
\pgfpathclose%
\pgfusepath{stroke,fill}%
\end{pgfscope}%
\begin{pgfscope}%
\pgfpathrectangle{\pgfqpoint{3.352233in}{5.105882in}}{\pgfqpoint{2.407767in}{1.544118in}}%
\pgfusepath{clip}%
\pgfsetroundcap%
\pgfsetroundjoin%
\pgfsetlinewidth{0.501875pt}%
\definecolor{currentstroke}{rgb}{0.283229,0.120777,0.440584}%
\pgfsetstrokecolor{currentstroke}%
\pgfsetdash{}{0pt}%
\pgfpathmoveto{\pgfqpoint{4.942035in}{6.215164in}}%
\pgfpathquadraticcurveto{\pgfqpoint{4.928866in}{6.214282in}}{\pgfqpoint{4.923444in}{6.213919in}}%
\pgfusepath{stroke}%
\end{pgfscope}%
\begin{pgfscope}%
\pgfpathrectangle{\pgfqpoint{3.352233in}{5.105882in}}{\pgfqpoint{2.407767in}{1.544118in}}%
\pgfusepath{clip}%
\pgfsetroundcap%
\pgfsetroundjoin%
\definecolor{currentfill}{rgb}{0.283229,0.120777,0.440584}%
\pgfsetfillcolor{currentfill}%
\pgfsetlinewidth{0.501875pt}%
\definecolor{currentstroke}{rgb}{0.283229,0.120777,0.440584}%
\pgfsetstrokecolor{currentstroke}%
\pgfsetdash{}{0pt}%
\pgfpathmoveto{\pgfqpoint{4.952088in}{6.201918in}}%
\pgfpathlineto{\pgfqpoint{4.923444in}{6.213919in}}%
\pgfpathlineto{\pgfqpoint{4.950231in}{6.229633in}}%
\pgfpathlineto{\pgfqpoint{4.952088in}{6.201918in}}%
\pgfpathlineto{\pgfqpoint{4.952088in}{6.201918in}}%
\pgfpathclose%
\pgfusepath{stroke,fill}%
\end{pgfscope}%
\begin{pgfscope}%
\pgfpathrectangle{\pgfqpoint{3.352233in}{5.105882in}}{\pgfqpoint{2.407767in}{1.544118in}}%
\pgfusepath{clip}%
\pgfsetroundcap%
\pgfsetroundjoin%
\pgfsetlinewidth{0.501875pt}%
\definecolor{currentstroke}{rgb}{0.283197,0.115680,0.436115}%
\pgfsetstrokecolor{currentstroke}%
\pgfsetdash{}{0pt}%
\pgfpathmoveto{\pgfqpoint{4.786085in}{6.232025in}}%
\pgfpathquadraticcurveto{\pgfqpoint{4.773271in}{6.229887in}}{\pgfqpoint{4.768114in}{6.229027in}}%
\pgfusepath{stroke}%
\end{pgfscope}%
\begin{pgfscope}%
\pgfpathrectangle{\pgfqpoint{3.352233in}{5.105882in}}{\pgfqpoint{2.407767in}{1.544118in}}%
\pgfusepath{clip}%
\pgfsetroundcap%
\pgfsetroundjoin%
\definecolor{currentfill}{rgb}{0.283197,0.115680,0.436115}%
\pgfsetfillcolor{currentfill}%
\pgfsetlinewidth{0.501875pt}%
\definecolor{currentstroke}{rgb}{0.283197,0.115680,0.436115}%
\pgfsetstrokecolor{currentstroke}%
\pgfsetdash{}{0pt}%
\pgfpathmoveto{\pgfqpoint{4.797799in}{6.219898in}}%
\pgfpathlineto{\pgfqpoint{4.768114in}{6.229027in}}%
\pgfpathlineto{\pgfqpoint{4.793228in}{6.247297in}}%
\pgfpathlineto{\pgfqpoint{4.797799in}{6.219898in}}%
\pgfpathlineto{\pgfqpoint{4.797799in}{6.219898in}}%
\pgfpathclose%
\pgfusepath{stroke,fill}%
\end{pgfscope}%
\begin{pgfscope}%
\pgfpathrectangle{\pgfqpoint{3.352233in}{5.105882in}}{\pgfqpoint{2.407767in}{1.544118in}}%
\pgfusepath{clip}%
\pgfsetroundcap%
\pgfsetroundjoin%
\pgfsetlinewidth{0.501875pt}%
\definecolor{currentstroke}{rgb}{0.273809,0.031497,0.358853}%
\pgfsetstrokecolor{currentstroke}%
\pgfsetdash{}{0pt}%
\pgfpathmoveto{\pgfqpoint{5.100491in}{6.292479in}}%
\pgfpathquadraticcurveto{\pgfqpoint{5.087279in}{6.291917in}}{\pgfqpoint{5.081824in}{6.291685in}}%
\pgfusepath{stroke}%
\end{pgfscope}%
\begin{pgfscope}%
\pgfpathrectangle{\pgfqpoint{3.352233in}{5.105882in}}{\pgfqpoint{2.407767in}{1.544118in}}%
\pgfusepath{clip}%
\pgfsetroundcap%
\pgfsetroundjoin%
\definecolor{currentfill}{rgb}{0.273809,0.031497,0.358853}%
\pgfsetfillcolor{currentfill}%
\pgfsetlinewidth{0.501875pt}%
\definecolor{currentstroke}{rgb}{0.273809,0.031497,0.358853}%
\pgfsetstrokecolor{currentstroke}%
\pgfsetdash{}{0pt}%
\pgfpathmoveto{\pgfqpoint{5.110167in}{6.278989in}}%
\pgfpathlineto{\pgfqpoint{5.081824in}{6.291685in}}%
\pgfpathlineto{\pgfqpoint{5.108987in}{6.306742in}}%
\pgfpathlineto{\pgfqpoint{5.110167in}{6.278989in}}%
\pgfpathlineto{\pgfqpoint{5.110167in}{6.278989in}}%
\pgfpathclose%
\pgfusepath{stroke,fill}%
\end{pgfscope}%
\begin{pgfscope}%
\pgfpathrectangle{\pgfqpoint{3.352233in}{5.105882in}}{\pgfqpoint{2.407767in}{1.544118in}}%
\pgfusepath{clip}%
\pgfsetroundcap%
\pgfsetroundjoin%
\pgfsetlinewidth{0.501875pt}%
\definecolor{currentstroke}{rgb}{0.281446,0.084320,0.407414}%
\pgfsetstrokecolor{currentstroke}%
\pgfsetdash{}{0pt}%
\pgfpathmoveto{\pgfqpoint{4.769359in}{5.486051in}}%
\pgfpathquadraticcurveto{\pgfqpoint{4.756683in}{5.488472in}}{\pgfqpoint{4.751634in}{5.489436in}}%
\pgfusepath{stroke}%
\end{pgfscope}%
\begin{pgfscope}%
\pgfpathrectangle{\pgfqpoint{3.352233in}{5.105882in}}{\pgfqpoint{2.407767in}{1.544118in}}%
\pgfusepath{clip}%
\pgfsetroundcap%
\pgfsetroundjoin%
\definecolor{currentfill}{rgb}{0.281446,0.084320,0.407414}%
\pgfsetfillcolor{currentfill}%
\pgfsetlinewidth{0.501875pt}%
\definecolor{currentstroke}{rgb}{0.281446,0.084320,0.407414}%
\pgfsetstrokecolor{currentstroke}%
\pgfsetdash{}{0pt}%
\pgfpathmoveto{\pgfqpoint{4.776314in}{5.470583in}}%
\pgfpathlineto{\pgfqpoint{4.751634in}{5.489436in}}%
\pgfpathlineto{\pgfqpoint{4.781524in}{5.497868in}}%
\pgfpathlineto{\pgfqpoint{4.776314in}{5.470583in}}%
\pgfpathlineto{\pgfqpoint{4.776314in}{5.470583in}}%
\pgfpathclose%
\pgfusepath{stroke,fill}%
\end{pgfscope}%
\begin{pgfscope}%
\pgfpathrectangle{\pgfqpoint{3.352233in}{5.105882in}}{\pgfqpoint{2.407767in}{1.544118in}}%
\pgfusepath{clip}%
\pgfsetroundcap%
\pgfsetroundjoin%
\pgfsetlinewidth{0.501875pt}%
\definecolor{currentstroke}{rgb}{0.282656,0.100196,0.422160}%
\pgfsetstrokecolor{currentstroke}%
\pgfsetdash{}{0pt}%
\pgfpathmoveto{\pgfqpoint{4.620542in}{6.234977in}}%
\pgfpathquadraticcurveto{\pgfqpoint{4.609035in}{6.230866in}}{\pgfqpoint{4.604838in}{6.229367in}}%
\pgfusepath{stroke}%
\end{pgfscope}%
\begin{pgfscope}%
\pgfpathrectangle{\pgfqpoint{3.352233in}{5.105882in}}{\pgfqpoint{2.407767in}{1.544118in}}%
\pgfusepath{clip}%
\pgfsetroundcap%
\pgfsetroundjoin%
\definecolor{currentfill}{rgb}{0.282656,0.100196,0.422160}%
\pgfsetfillcolor{currentfill}%
\pgfsetlinewidth{0.501875pt}%
\definecolor{currentstroke}{rgb}{0.282656,0.100196,0.422160}%
\pgfsetstrokecolor{currentstroke}%
\pgfsetdash{}{0pt}%
\pgfpathmoveto{\pgfqpoint{4.635669in}{6.225633in}}%
\pgfpathlineto{\pgfqpoint{4.604838in}{6.229367in}}%
\pgfpathlineto{\pgfqpoint{4.626324in}{6.251792in}}%
\pgfpathlineto{\pgfqpoint{4.635669in}{6.225633in}}%
\pgfpathlineto{\pgfqpoint{4.635669in}{6.225633in}}%
\pgfpathclose%
\pgfusepath{stroke,fill}%
\end{pgfscope}%
\begin{pgfscope}%
\pgfpathrectangle{\pgfqpoint{3.352233in}{5.105882in}}{\pgfqpoint{2.407767in}{1.544118in}}%
\pgfusepath{clip}%
\pgfsetroundcap%
\pgfsetroundjoin%
\pgfsetlinewidth{0.501875pt}%
\definecolor{currentstroke}{rgb}{0.283229,0.120777,0.440584}%
\pgfsetstrokecolor{currentstroke}%
\pgfsetdash{}{0pt}%
\pgfpathmoveto{\pgfqpoint{4.404332in}{6.206948in}}%
\pgfpathquadraticcurveto{\pgfqpoint{4.404382in}{6.198546in}}{\pgfqpoint{4.404386in}{6.197908in}}%
\pgfusepath{stroke}%
\end{pgfscope}%
\begin{pgfscope}%
\pgfpathrectangle{\pgfqpoint{3.352233in}{5.105882in}}{\pgfqpoint{2.407767in}{1.544118in}}%
\pgfusepath{clip}%
\pgfsetroundcap%
\pgfsetroundjoin%
\definecolor{currentfill}{rgb}{0.283229,0.120777,0.440584}%
\pgfsetfillcolor{currentfill}%
\pgfsetlinewidth{0.501875pt}%
\definecolor{currentstroke}{rgb}{0.283229,0.120777,0.440584}%
\pgfsetstrokecolor{currentstroke}%
\pgfsetdash{}{0pt}%
\pgfpathmoveto{\pgfqpoint{4.418110in}{6.225768in}}%
\pgfpathlineto{\pgfqpoint{4.404386in}{6.197908in}}%
\pgfpathlineto{\pgfqpoint{4.390333in}{6.225603in}}%
\pgfpathlineto{\pgfqpoint{4.418110in}{6.225768in}}%
\pgfpathlineto{\pgfqpoint{4.418110in}{6.225768in}}%
\pgfpathclose%
\pgfusepath{stroke,fill}%
\end{pgfscope}%
\begin{pgfscope}%
\pgfpathrectangle{\pgfqpoint{3.352233in}{5.105882in}}{\pgfqpoint{2.407767in}{1.544118in}}%
\pgfusepath{clip}%
\pgfsetroundcap%
\pgfsetroundjoin%
\pgfsetlinewidth{0.501875pt}%
\definecolor{currentstroke}{rgb}{0.195860,0.395433,0.555276}%
\pgfsetstrokecolor{currentstroke}%
\pgfsetdash{}{0pt}%
\pgfpathmoveto{\pgfqpoint{4.509235in}{5.995488in}}%
\pgfpathquadraticcurveto{\pgfqpoint{4.496510in}{5.993141in}}{\pgfqpoint{4.491419in}{5.992202in}}%
\pgfusepath{stroke}%
\end{pgfscope}%
\begin{pgfscope}%
\pgfpathrectangle{\pgfqpoint{3.352233in}{5.105882in}}{\pgfqpoint{2.407767in}{1.544118in}}%
\pgfusepath{clip}%
\pgfsetroundcap%
\pgfsetroundjoin%
\definecolor{currentfill}{rgb}{0.195860,0.395433,0.555276}%
\pgfsetfillcolor{currentfill}%
\pgfsetlinewidth{0.501875pt}%
\definecolor{currentstroke}{rgb}{0.195860,0.395433,0.555276}%
\pgfsetstrokecolor{currentstroke}%
\pgfsetdash{}{0pt}%
\pgfpathmoveto{\pgfqpoint{4.521256in}{5.983582in}}%
\pgfpathlineto{\pgfqpoint{4.491419in}{5.992202in}}%
\pgfpathlineto{\pgfqpoint{4.516217in}{6.010899in}}%
\pgfpathlineto{\pgfqpoint{4.521256in}{5.983582in}}%
\pgfpathlineto{\pgfqpoint{4.521256in}{5.983582in}}%
\pgfpathclose%
\pgfusepath{stroke,fill}%
\end{pgfscope}%
\begin{pgfscope}%
\pgfpathrectangle{\pgfqpoint{3.352233in}{5.105882in}}{\pgfqpoint{2.407767in}{1.544118in}}%
\pgfusepath{clip}%
\pgfsetbuttcap%
\pgfsetroundjoin%
\pgfsetlinewidth{1.505625pt}%
\definecolor{currentstroke}{rgb}{0.000000,0.000000,0.000000}%
\pgfsetstrokecolor{currentstroke}%
\pgfsetdash{}{0pt}%
\pgfpathmoveto{\pgfqpoint{4.341943in}{5.367356in}}%
\pgfpathlineto{\pgfqpoint{4.341943in}{6.388526in}}%
\pgfusepath{stroke}%
\end{pgfscope}%
\begin{pgfscope}%
\pgfpathrectangle{\pgfqpoint{3.352233in}{5.105882in}}{\pgfqpoint{2.407767in}{1.544118in}}%
\pgfusepath{clip}%
\pgfsetbuttcap%
\pgfsetroundjoin%
\pgfsetlinewidth{1.505625pt}%
\definecolor{currentstroke}{rgb}{0.000000,0.000000,0.000000}%
\pgfsetstrokecolor{currentstroke}%
\pgfsetdash{}{0pt}%
\pgfpathmoveto{\pgfqpoint{5.251844in}{5.367356in}}%
\pgfpathlineto{\pgfqpoint{5.251844in}{6.388526in}}%
\pgfusepath{stroke}%
\end{pgfscope}%
\begin{pgfscope}%
\pgfsetrectcap%
\pgfsetmiterjoin%
\pgfsetlinewidth{0.803000pt}%
\definecolor{currentstroke}{rgb}{0.000000,0.000000,0.000000}%
\pgfsetstrokecolor{currentstroke}%
\pgfsetdash{}{0pt}%
\pgfpathmoveto{\pgfqpoint{3.352233in}{5.105882in}}%
\pgfpathlineto{\pgfqpoint{3.352233in}{6.650000in}}%
\pgfusepath{stroke}%
\end{pgfscope}%
\begin{pgfscope}%
\pgfsetrectcap%
\pgfsetmiterjoin%
\pgfsetlinewidth{0.803000pt}%
\definecolor{currentstroke}{rgb}{0.000000,0.000000,0.000000}%
\pgfsetstrokecolor{currentstroke}%
\pgfsetdash{}{0pt}%
\pgfpathmoveto{\pgfqpoint{5.760000in}{5.105882in}}%
\pgfpathlineto{\pgfqpoint{5.760000in}{6.650000in}}%
\pgfusepath{stroke}%
\end{pgfscope}%
\begin{pgfscope}%
\pgfsetrectcap%
\pgfsetmiterjoin%
\pgfsetlinewidth{0.803000pt}%
\definecolor{currentstroke}{rgb}{0.000000,0.000000,0.000000}%
\pgfsetstrokecolor{currentstroke}%
\pgfsetdash{}{0pt}%
\pgfpathmoveto{\pgfqpoint{3.352233in}{5.105882in}}%
\pgfpathlineto{\pgfqpoint{5.760000in}{5.105882in}}%
\pgfusepath{stroke}%
\end{pgfscope}%
\begin{pgfscope}%
\pgfsetrectcap%
\pgfsetmiterjoin%
\pgfsetlinewidth{0.803000pt}%
\definecolor{currentstroke}{rgb}{0.000000,0.000000,0.000000}%
\pgfsetstrokecolor{currentstroke}%
\pgfsetdash{}{0pt}%
\pgfpathmoveto{\pgfqpoint{3.352233in}{6.650000in}}%
\pgfpathlineto{\pgfqpoint{5.760000in}{6.650000in}}%
\pgfusepath{stroke}%
\end{pgfscope}%
\begin{pgfscope}%
\definecolor{textcolor}{rgb}{0.000000,0.000000,0.000000}%
\pgfsetstrokecolor{textcolor}%
\pgfsetfillcolor{textcolor}%
\pgftext[x=4.556117in,y=6.733333in,,base]{\color{textcolor}\sffamily\fontsize{12.000000}{14.400000}\selectfont b)}%
\end{pgfscope}%
\begin{pgfscope}%
\pgfsetbuttcap%
\pgfsetmiterjoin%
\definecolor{currentfill}{rgb}{1.000000,1.000000,1.000000}%
\pgfsetfillcolor{currentfill}%
\pgfsetlinewidth{0.000000pt}%
\definecolor{currentstroke}{rgb}{0.000000,0.000000,0.000000}%
\pgfsetstrokecolor{currentstroke}%
\pgfsetstrokeopacity{0.000000}%
\pgfsetdash{}{0pt}%
\pgfpathmoveto{\pgfqpoint{0.800000in}{3.252941in}}%
\pgfpathlineto{\pgfqpoint{3.207767in}{3.252941in}}%
\pgfpathlineto{\pgfqpoint{3.207767in}{4.797059in}}%
\pgfpathlineto{\pgfqpoint{0.800000in}{4.797059in}}%
\pgfpathlineto{\pgfqpoint{0.800000in}{3.252941in}}%
\pgfpathclose%
\pgfusepath{fill}%
\end{pgfscope}%
\begin{pgfscope}%
\pgfpathrectangle{\pgfqpoint{0.800000in}{3.252941in}}{\pgfqpoint{2.407767in}{1.544118in}}%
\pgfusepath{clip}%
\pgfsys@transformcm{2.416667}{0.000000}{0.000000}{1.555556}{0.800000in}{3.252941in}%
\pgftext[left,bottom]{\includegraphics[interpolate=false,width=1.000000in,height=1.000000in]{q_series_square-img2.png}}%
\end{pgfscope}%
\begin{pgfscope}%
\pgfsetbuttcap%
\pgfsetroundjoin%
\definecolor{currentfill}{rgb}{0.000000,0.000000,0.000000}%
\pgfsetfillcolor{currentfill}%
\pgfsetlinewidth{0.803000pt}%
\definecolor{currentstroke}{rgb}{0.000000,0.000000,0.000000}%
\pgfsetstrokecolor{currentstroke}%
\pgfsetdash{}{0pt}%
\pgfsys@defobject{currentmarker}{\pgfqpoint{0.000000in}{-0.048611in}}{\pgfqpoint{0.000000in}{0.000000in}}{%
\pgfpathmoveto{\pgfqpoint{0.000000in}{0.000000in}}%
\pgfpathlineto{\pgfqpoint{0.000000in}{-0.048611in}}%
\pgfusepath{stroke,fill}%
}%
\begin{pgfscope}%
\pgfsys@transformshift{1.233659in}{3.252941in}%
\pgfsys@useobject{currentmarker}{}%
\end{pgfscope}%
\end{pgfscope}%
\begin{pgfscope}%
\pgfsetbuttcap%
\pgfsetroundjoin%
\definecolor{currentfill}{rgb}{0.000000,0.000000,0.000000}%
\pgfsetfillcolor{currentfill}%
\pgfsetlinewidth{0.803000pt}%
\definecolor{currentstroke}{rgb}{0.000000,0.000000,0.000000}%
\pgfsetstrokecolor{currentstroke}%
\pgfsetdash{}{0pt}%
\pgfsys@defobject{currentmarker}{\pgfqpoint{0.000000in}{-0.048611in}}{\pgfqpoint{0.000000in}{0.000000in}}{%
\pgfpathmoveto{\pgfqpoint{0.000000in}{0.000000in}}%
\pgfpathlineto{\pgfqpoint{0.000000in}{-0.048611in}}%
\pgfusepath{stroke,fill}%
}%
\begin{pgfscope}%
\pgfsys@transformshift{1.739160in}{3.252941in}%
\pgfsys@useobject{currentmarker}{}%
\end{pgfscope}%
\end{pgfscope}%
\begin{pgfscope}%
\pgfsetbuttcap%
\pgfsetroundjoin%
\definecolor{currentfill}{rgb}{0.000000,0.000000,0.000000}%
\pgfsetfillcolor{currentfill}%
\pgfsetlinewidth{0.803000pt}%
\definecolor{currentstroke}{rgb}{0.000000,0.000000,0.000000}%
\pgfsetstrokecolor{currentstroke}%
\pgfsetdash{}{0pt}%
\pgfsys@defobject{currentmarker}{\pgfqpoint{0.000000in}{-0.048611in}}{\pgfqpoint{0.000000in}{0.000000in}}{%
\pgfpathmoveto{\pgfqpoint{0.000000in}{0.000000in}}%
\pgfpathlineto{\pgfqpoint{0.000000in}{-0.048611in}}%
\pgfusepath{stroke,fill}%
}%
\begin{pgfscope}%
\pgfsys@transformshift{2.244660in}{3.252941in}%
\pgfsys@useobject{currentmarker}{}%
\end{pgfscope}%
\end{pgfscope}%
\begin{pgfscope}%
\pgfsetbuttcap%
\pgfsetroundjoin%
\definecolor{currentfill}{rgb}{0.000000,0.000000,0.000000}%
\pgfsetfillcolor{currentfill}%
\pgfsetlinewidth{0.803000pt}%
\definecolor{currentstroke}{rgb}{0.000000,0.000000,0.000000}%
\pgfsetstrokecolor{currentstroke}%
\pgfsetdash{}{0pt}%
\pgfsys@defobject{currentmarker}{\pgfqpoint{0.000000in}{-0.048611in}}{\pgfqpoint{0.000000in}{0.000000in}}{%
\pgfpathmoveto{\pgfqpoint{0.000000in}{0.000000in}}%
\pgfpathlineto{\pgfqpoint{0.000000in}{-0.048611in}}%
\pgfusepath{stroke,fill}%
}%
\begin{pgfscope}%
\pgfsys@transformshift{2.750161in}{3.252941in}%
\pgfsys@useobject{currentmarker}{}%
\end{pgfscope}%
\end{pgfscope}%
\begin{pgfscope}%
\pgfsetbuttcap%
\pgfsetroundjoin%
\definecolor{currentfill}{rgb}{0.000000,0.000000,0.000000}%
\pgfsetfillcolor{currentfill}%
\pgfsetlinewidth{0.803000pt}%
\definecolor{currentstroke}{rgb}{0.000000,0.000000,0.000000}%
\pgfsetstrokecolor{currentstroke}%
\pgfsetdash{}{0pt}%
\pgfsys@defobject{currentmarker}{\pgfqpoint{-0.048611in}{0.000000in}}{\pgfqpoint{-0.000000in}{0.000000in}}{%
\pgfpathmoveto{\pgfqpoint{-0.000000in}{0.000000in}}%
\pgfpathlineto{\pgfqpoint{-0.048611in}{0.000000in}}%
\pgfusepath{stroke,fill}%
}%
\begin{pgfscope}%
\pgfsys@transformshift{0.800000in}{3.514415in}%
\pgfsys@useobject{currentmarker}{}%
\end{pgfscope}%
\end{pgfscope}%
\begin{pgfscope}%
\definecolor{textcolor}{rgb}{0.000000,0.000000,0.000000}%
\pgfsetstrokecolor{textcolor}%
\pgfsetfillcolor{textcolor}%
\pgftext[x=0.455863in, y=3.466190in, left, base]{\color{textcolor}\sffamily\fontsize{10.000000}{12.000000}\selectfont \(\displaystyle {\ensuremath{-}20}\)}%
\end{pgfscope}%
\begin{pgfscope}%
\pgfsetbuttcap%
\pgfsetroundjoin%
\definecolor{currentfill}{rgb}{0.000000,0.000000,0.000000}%
\pgfsetfillcolor{currentfill}%
\pgfsetlinewidth{0.803000pt}%
\definecolor{currentstroke}{rgb}{0.000000,0.000000,0.000000}%
\pgfsetstrokecolor{currentstroke}%
\pgfsetdash{}{0pt}%
\pgfsys@defobject{currentmarker}{\pgfqpoint{-0.048611in}{0.000000in}}{\pgfqpoint{-0.000000in}{0.000000in}}{%
\pgfpathmoveto{\pgfqpoint{-0.000000in}{0.000000in}}%
\pgfpathlineto{\pgfqpoint{-0.048611in}{0.000000in}}%
\pgfusepath{stroke,fill}%
}%
\begin{pgfscope}%
\pgfsys@transformshift{0.800000in}{4.025000in}%
\pgfsys@useobject{currentmarker}{}%
\end{pgfscope}%
\end{pgfscope}%
\begin{pgfscope}%
\definecolor{textcolor}{rgb}{0.000000,0.000000,0.000000}%
\pgfsetstrokecolor{textcolor}%
\pgfsetfillcolor{textcolor}%
\pgftext[x=0.633333in, y=3.976775in, left, base]{\color{textcolor}\sffamily\fontsize{10.000000}{12.000000}\selectfont \(\displaystyle {0}\)}%
\end{pgfscope}%
\begin{pgfscope}%
\pgfsetbuttcap%
\pgfsetroundjoin%
\definecolor{currentfill}{rgb}{0.000000,0.000000,0.000000}%
\pgfsetfillcolor{currentfill}%
\pgfsetlinewidth{0.803000pt}%
\definecolor{currentstroke}{rgb}{0.000000,0.000000,0.000000}%
\pgfsetstrokecolor{currentstroke}%
\pgfsetdash{}{0pt}%
\pgfsys@defobject{currentmarker}{\pgfqpoint{-0.048611in}{0.000000in}}{\pgfqpoint{-0.000000in}{0.000000in}}{%
\pgfpathmoveto{\pgfqpoint{-0.000000in}{0.000000in}}%
\pgfpathlineto{\pgfqpoint{-0.048611in}{0.000000in}}%
\pgfusepath{stroke,fill}%
}%
\begin{pgfscope}%
\pgfsys@transformshift{0.800000in}{4.535585in}%
\pgfsys@useobject{currentmarker}{}%
\end{pgfscope}%
\end{pgfscope}%
\begin{pgfscope}%
\definecolor{textcolor}{rgb}{0.000000,0.000000,0.000000}%
\pgfsetstrokecolor{textcolor}%
\pgfsetfillcolor{textcolor}%
\pgftext[x=0.563888in, y=4.487360in, left, base]{\color{textcolor}\sffamily\fontsize{10.000000}{12.000000}\selectfont \(\displaystyle {20}\)}%
\end{pgfscope}%
\begin{pgfscope}%
\definecolor{textcolor}{rgb}{0.000000,0.000000,0.000000}%
\pgfsetstrokecolor{textcolor}%
\pgfsetfillcolor{textcolor}%
\pgftext[x=0.400308in,y=4.025000in,,bottom,rotate=90.000000]{\color{textcolor}\sffamily\fontsize{10.000000}{12.000000}\selectfont \(\displaystyle z \, \mathrm{[\mu m]}\)}%
\end{pgfscope}%
\begin{pgfscope}%
\pgfpathrectangle{\pgfqpoint{0.800000in}{3.252941in}}{\pgfqpoint{2.407767in}{1.544118in}}%
\pgfusepath{clip}%
\pgfsetbuttcap%
\pgfsetroundjoin%
\pgfsetlinewidth{0.501875pt}%
\definecolor{currentstroke}{rgb}{0.268510,0.009605,0.335427}%
\pgfsetstrokecolor{currentstroke}%
\pgfsetdash{}{0pt}%
\pgfpathmoveto{\pgfqpoint{2.647145in}{3.564534in}}%
\pgfpathlineto{\pgfqpoint{2.594309in}{3.564303in}}%
\pgfusepath{stroke}%
\end{pgfscope}%
\begin{pgfscope}%
\pgfpathrectangle{\pgfqpoint{0.800000in}{3.252941in}}{\pgfqpoint{2.407767in}{1.544118in}}%
\pgfusepath{clip}%
\pgfsetbuttcap%
\pgfsetroundjoin%
\pgfsetlinewidth{0.501875pt}%
\definecolor{currentstroke}{rgb}{0.271305,0.019942,0.347269}%
\pgfsetstrokecolor{currentstroke}%
\pgfsetdash{}{0pt}%
\pgfpathmoveto{\pgfqpoint{2.594309in}{3.564303in}}%
\pgfpathlineto{\pgfqpoint{2.541406in}{3.565540in}}%
\pgfusepath{stroke}%
\end{pgfscope}%
\begin{pgfscope}%
\pgfpathrectangle{\pgfqpoint{0.800000in}{3.252941in}}{\pgfqpoint{2.407767in}{1.544118in}}%
\pgfusepath{clip}%
\pgfsetbuttcap%
\pgfsetroundjoin%
\pgfsetlinewidth{0.501875pt}%
\definecolor{currentstroke}{rgb}{0.273809,0.031497,0.358853}%
\pgfsetstrokecolor{currentstroke}%
\pgfsetdash{}{0pt}%
\pgfpathmoveto{\pgfqpoint{2.541406in}{3.565540in}}%
\pgfpathlineto{\pgfqpoint{2.488567in}{3.567571in}}%
\pgfusepath{stroke}%
\end{pgfscope}%
\begin{pgfscope}%
\pgfpathrectangle{\pgfqpoint{0.800000in}{3.252941in}}{\pgfqpoint{2.407767in}{1.544118in}}%
\pgfusepath{clip}%
\pgfsetbuttcap%
\pgfsetroundjoin%
\pgfsetlinewidth{0.501875pt}%
\definecolor{currentstroke}{rgb}{0.276022,0.044167,0.370164}%
\pgfsetstrokecolor{currentstroke}%
\pgfsetdash{}{0pt}%
\pgfpathmoveto{\pgfqpoint{2.488567in}{3.567571in}}%
\pgfpathlineto{\pgfqpoint{2.435752in}{3.569968in}}%
\pgfusepath{stroke}%
\end{pgfscope}%
\begin{pgfscope}%
\pgfpathrectangle{\pgfqpoint{0.800000in}{3.252941in}}{\pgfqpoint{2.407767in}{1.544118in}}%
\pgfusepath{clip}%
\pgfsetbuttcap%
\pgfsetroundjoin%
\pgfsetlinewidth{0.501875pt}%
\definecolor{currentstroke}{rgb}{0.276022,0.044167,0.370164}%
\pgfsetstrokecolor{currentstroke}%
\pgfsetdash{}{0pt}%
\pgfpathmoveto{\pgfqpoint{2.435752in}{3.569968in}}%
\pgfpathlineto{\pgfqpoint{2.383145in}{3.573300in}}%
\pgfusepath{stroke}%
\end{pgfscope}%
\begin{pgfscope}%
\pgfpathrectangle{\pgfqpoint{0.800000in}{3.252941in}}{\pgfqpoint{2.407767in}{1.544118in}}%
\pgfusepath{clip}%
\pgfsetbuttcap%
\pgfsetroundjoin%
\pgfsetlinewidth{0.501875pt}%
\definecolor{currentstroke}{rgb}{0.277018,0.050344,0.375715}%
\pgfsetstrokecolor{currentstroke}%
\pgfsetdash{}{0pt}%
\pgfpathmoveto{\pgfqpoint{2.383145in}{3.573300in}}%
\pgfpathlineto{\pgfqpoint{2.330933in}{3.578788in}}%
\pgfusepath{stroke}%
\end{pgfscope}%
\begin{pgfscope}%
\pgfpathrectangle{\pgfqpoint{0.800000in}{3.252941in}}{\pgfqpoint{2.407767in}{1.544118in}}%
\pgfusepath{clip}%
\pgfsetbuttcap%
\pgfsetroundjoin%
\pgfsetlinewidth{0.501875pt}%
\definecolor{currentstroke}{rgb}{0.271305,0.019942,0.347269}%
\pgfsetstrokecolor{currentstroke}%
\pgfsetdash{}{0pt}%
\pgfpathmoveto{\pgfqpoint{2.654046in}{4.476700in}}%
\pgfpathlineto{\pgfqpoint{2.601083in}{4.476172in}}%
\pgfusepath{stroke}%
\end{pgfscope}%
\begin{pgfscope}%
\pgfpathrectangle{\pgfqpoint{0.800000in}{3.252941in}}{\pgfqpoint{2.407767in}{1.544118in}}%
\pgfusepath{clip}%
\pgfsetbuttcap%
\pgfsetroundjoin%
\pgfsetlinewidth{0.501875pt}%
\definecolor{currentstroke}{rgb}{0.271305,0.019942,0.347269}%
\pgfsetstrokecolor{currentstroke}%
\pgfsetdash{}{0pt}%
\pgfpathmoveto{\pgfqpoint{2.601083in}{4.476172in}}%
\pgfpathlineto{\pgfqpoint{2.548151in}{4.475532in}}%
\pgfusepath{stroke}%
\end{pgfscope}%
\begin{pgfscope}%
\pgfpathrectangle{\pgfqpoint{0.800000in}{3.252941in}}{\pgfqpoint{2.407767in}{1.544118in}}%
\pgfusepath{clip}%
\pgfsetbuttcap%
\pgfsetroundjoin%
\pgfsetlinewidth{0.501875pt}%
\definecolor{currentstroke}{rgb}{0.272594,0.025563,0.353093}%
\pgfsetstrokecolor{currentstroke}%
\pgfsetdash{}{0pt}%
\pgfpathmoveto{\pgfqpoint{2.548151in}{4.475532in}}%
\pgfpathlineto{\pgfqpoint{2.495298in}{4.473940in}}%
\pgfusepath{stroke}%
\end{pgfscope}%
\begin{pgfscope}%
\pgfpathrectangle{\pgfqpoint{0.800000in}{3.252941in}}{\pgfqpoint{2.407767in}{1.544118in}}%
\pgfusepath{clip}%
\pgfsetbuttcap%
\pgfsetroundjoin%
\pgfsetlinewidth{0.501875pt}%
\definecolor{currentstroke}{rgb}{0.274952,0.037752,0.364543}%
\pgfsetstrokecolor{currentstroke}%
\pgfsetdash{}{0pt}%
\pgfpathmoveto{\pgfqpoint{2.495298in}{4.473940in}}%
\pgfpathlineto{\pgfqpoint{2.442518in}{4.471104in}}%
\pgfusepath{stroke}%
\end{pgfscope}%
\begin{pgfscope}%
\pgfpathrectangle{\pgfqpoint{0.800000in}{3.252941in}}{\pgfqpoint{2.407767in}{1.544118in}}%
\pgfusepath{clip}%
\pgfsetbuttcap%
\pgfsetroundjoin%
\pgfsetlinewidth{0.501875pt}%
\definecolor{currentstroke}{rgb}{0.277018,0.050344,0.375715}%
\pgfsetstrokecolor{currentstroke}%
\pgfsetdash{}{0pt}%
\pgfpathmoveto{\pgfqpoint{2.442518in}{4.471104in}}%
\pgfpathlineto{\pgfqpoint{2.389721in}{4.468327in}}%
\pgfusepath{stroke}%
\end{pgfscope}%
\begin{pgfscope}%
\pgfpathrectangle{\pgfqpoint{0.800000in}{3.252941in}}{\pgfqpoint{2.407767in}{1.544118in}}%
\pgfusepath{clip}%
\pgfsetbuttcap%
\pgfsetroundjoin%
\pgfsetlinewidth{0.501875pt}%
\definecolor{currentstroke}{rgb}{0.273809,0.031497,0.358853}%
\pgfsetstrokecolor{currentstroke}%
\pgfsetdash{}{0pt}%
\pgfpathmoveto{\pgfqpoint{2.537807in}{3.590825in}}%
\pgfpathlineto{\pgfqpoint{2.484911in}{3.592465in}}%
\pgfusepath{stroke}%
\end{pgfscope}%
\begin{pgfscope}%
\pgfpathrectangle{\pgfqpoint{0.800000in}{3.252941in}}{\pgfqpoint{2.407767in}{1.544118in}}%
\pgfusepath{clip}%
\pgfsetbuttcap%
\pgfsetroundjoin%
\pgfsetlinewidth{0.501875pt}%
\definecolor{currentstroke}{rgb}{0.274952,0.037752,0.364543}%
\pgfsetstrokecolor{currentstroke}%
\pgfsetdash{}{0pt}%
\pgfpathmoveto{\pgfqpoint{2.484911in}{3.592465in}}%
\pgfpathlineto{\pgfqpoint{2.432045in}{3.594495in}}%
\pgfusepath{stroke}%
\end{pgfscope}%
\begin{pgfscope}%
\pgfpathrectangle{\pgfqpoint{0.800000in}{3.252941in}}{\pgfqpoint{2.407767in}{1.544118in}}%
\pgfusepath{clip}%
\pgfsetbuttcap%
\pgfsetroundjoin%
\pgfsetlinewidth{0.501875pt}%
\definecolor{currentstroke}{rgb}{0.277018,0.050344,0.375715}%
\pgfsetstrokecolor{currentstroke}%
\pgfsetdash{}{0pt}%
\pgfpathmoveto{\pgfqpoint{2.432045in}{3.594495in}}%
\pgfpathlineto{\pgfqpoint{2.379421in}{3.597903in}}%
\pgfusepath{stroke}%
\end{pgfscope}%
\begin{pgfscope}%
\pgfpathrectangle{\pgfqpoint{0.800000in}{3.252941in}}{\pgfqpoint{2.407767in}{1.544118in}}%
\pgfusepath{clip}%
\pgfsetbuttcap%
\pgfsetroundjoin%
\pgfsetlinewidth{0.501875pt}%
\definecolor{currentstroke}{rgb}{0.278791,0.062145,0.386592}%
\pgfsetstrokecolor{currentstroke}%
\pgfsetdash{}{0pt}%
\pgfpathmoveto{\pgfqpoint{2.379421in}{3.597903in}}%
\pgfpathlineto{\pgfqpoint{2.326985in}{3.602508in}}%
\pgfusepath{stroke}%
\end{pgfscope}%
\begin{pgfscope}%
\pgfpathrectangle{\pgfqpoint{0.800000in}{3.252941in}}{\pgfqpoint{2.407767in}{1.544118in}}%
\pgfusepath{clip}%
\pgfsetbuttcap%
\pgfsetroundjoin%
\pgfsetlinewidth{0.501875pt}%
\definecolor{currentstroke}{rgb}{0.277941,0.056324,0.381191}%
\pgfsetstrokecolor{currentstroke}%
\pgfsetdash{}{0pt}%
\pgfpathmoveto{\pgfqpoint{2.326985in}{3.602508in}}%
\pgfpathlineto{\pgfqpoint{2.274784in}{3.608047in}}%
\pgfusepath{stroke}%
\end{pgfscope}%
\begin{pgfscope}%
\pgfpathrectangle{\pgfqpoint{0.800000in}{3.252941in}}{\pgfqpoint{2.407767in}{1.544118in}}%
\pgfusepath{clip}%
\pgfsetbuttcap%
\pgfsetroundjoin%
\pgfsetlinewidth{0.501875pt}%
\definecolor{currentstroke}{rgb}{0.276022,0.044167,0.370164}%
\pgfsetstrokecolor{currentstroke}%
\pgfsetdash{}{0pt}%
\pgfpathmoveto{\pgfqpoint{2.274784in}{3.608047in}}%
\pgfpathlineto{\pgfqpoint{2.223188in}{3.615603in}}%
\pgfusepath{stroke}%
\end{pgfscope}%
\begin{pgfscope}%
\pgfpathrectangle{\pgfqpoint{0.800000in}{3.252941in}}{\pgfqpoint{2.407767in}{1.544118in}}%
\pgfusepath{clip}%
\pgfsetbuttcap%
\pgfsetroundjoin%
\pgfsetlinewidth{0.501875pt}%
\definecolor{currentstroke}{rgb}{0.271305,0.019942,0.347269}%
\pgfsetstrokecolor{currentstroke}%
\pgfsetdash{}{0pt}%
\pgfpathmoveto{\pgfqpoint{2.654046in}{3.642793in}}%
\pgfpathlineto{\pgfqpoint{2.601104in}{3.643944in}}%
\pgfusepath{stroke}%
\end{pgfscope}%
\begin{pgfscope}%
\pgfpathrectangle{\pgfqpoint{0.800000in}{3.252941in}}{\pgfqpoint{2.407767in}{1.544118in}}%
\pgfusepath{clip}%
\pgfsetbuttcap%
\pgfsetroundjoin%
\pgfsetlinewidth{0.501875pt}%
\definecolor{currentstroke}{rgb}{0.273809,0.031497,0.358853}%
\pgfsetstrokecolor{currentstroke}%
\pgfsetdash{}{0pt}%
\pgfpathmoveto{\pgfqpoint{2.601104in}{3.643944in}}%
\pgfpathlineto{\pgfqpoint{2.548183in}{3.645450in}}%
\pgfusepath{stroke}%
\end{pgfscope}%
\begin{pgfscope}%
\pgfpathrectangle{\pgfqpoint{0.800000in}{3.252941in}}{\pgfqpoint{2.407767in}{1.544118in}}%
\pgfusepath{clip}%
\pgfsetbuttcap%
\pgfsetroundjoin%
\pgfsetlinewidth{0.501875pt}%
\definecolor{currentstroke}{rgb}{0.276022,0.044167,0.370164}%
\pgfsetstrokecolor{currentstroke}%
\pgfsetdash{}{0pt}%
\pgfpathmoveto{\pgfqpoint{2.548183in}{3.645450in}}%
\pgfpathlineto{\pgfqpoint{2.495305in}{3.647371in}}%
\pgfusepath{stroke}%
\end{pgfscope}%
\begin{pgfscope}%
\pgfpathrectangle{\pgfqpoint{0.800000in}{3.252941in}}{\pgfqpoint{2.407767in}{1.544118in}}%
\pgfusepath{clip}%
\pgfsetbuttcap%
\pgfsetroundjoin%
\pgfsetlinewidth{0.501875pt}%
\definecolor{currentstroke}{rgb}{0.277018,0.050344,0.375715}%
\pgfsetstrokecolor{currentstroke}%
\pgfsetdash{}{0pt}%
\pgfpathmoveto{\pgfqpoint{2.495305in}{3.647371in}}%
\pgfpathlineto{\pgfqpoint{2.442508in}{3.650028in}}%
\pgfusepath{stroke}%
\end{pgfscope}%
\begin{pgfscope}%
\pgfpathrectangle{\pgfqpoint{0.800000in}{3.252941in}}{\pgfqpoint{2.407767in}{1.544118in}}%
\pgfusepath{clip}%
\pgfsetbuttcap%
\pgfsetroundjoin%
\pgfsetlinewidth{0.501875pt}%
\definecolor{currentstroke}{rgb}{0.279566,0.067836,0.391917}%
\pgfsetstrokecolor{currentstroke}%
\pgfsetdash{}{0pt}%
\pgfpathmoveto{\pgfqpoint{2.442508in}{3.650028in}}%
\pgfpathlineto{\pgfqpoint{2.389773in}{3.653153in}}%
\pgfusepath{stroke}%
\end{pgfscope}%
\begin{pgfscope}%
\pgfpathrectangle{\pgfqpoint{0.800000in}{3.252941in}}{\pgfqpoint{2.407767in}{1.544118in}}%
\pgfusepath{clip}%
\pgfsetbuttcap%
\pgfsetroundjoin%
\pgfsetlinewidth{0.501875pt}%
\definecolor{currentstroke}{rgb}{0.278791,0.062145,0.386592}%
\pgfsetstrokecolor{currentstroke}%
\pgfsetdash{}{0pt}%
\pgfpathmoveto{\pgfqpoint{2.389773in}{3.653153in}}%
\pgfpathlineto{\pgfqpoint{2.337277in}{3.657478in}}%
\pgfusepath{stroke}%
\end{pgfscope}%
\begin{pgfscope}%
\pgfpathrectangle{\pgfqpoint{0.800000in}{3.252941in}}{\pgfqpoint{2.407767in}{1.544118in}}%
\pgfusepath{clip}%
\pgfsetbuttcap%
\pgfsetroundjoin%
\pgfsetlinewidth{0.501875pt}%
\definecolor{currentstroke}{rgb}{0.271305,0.019942,0.347269}%
\pgfsetstrokecolor{currentstroke}%
\pgfsetdash{}{0pt}%
\pgfpathmoveto{\pgfqpoint{2.654046in}{3.677539in}}%
\pgfpathlineto{\pgfqpoint{2.601084in}{3.678150in}}%
\pgfusepath{stroke}%
\end{pgfscope}%
\begin{pgfscope}%
\pgfpathrectangle{\pgfqpoint{0.800000in}{3.252941in}}{\pgfqpoint{2.407767in}{1.544118in}}%
\pgfusepath{clip}%
\pgfsetbuttcap%
\pgfsetroundjoin%
\pgfsetlinewidth{0.501875pt}%
\definecolor{currentstroke}{rgb}{0.276022,0.044167,0.370164}%
\pgfsetstrokecolor{currentstroke}%
\pgfsetdash{}{0pt}%
\pgfpathmoveto{\pgfqpoint{2.601084in}{3.678150in}}%
\pgfpathlineto{\pgfqpoint{2.548132in}{3.679081in}}%
\pgfusepath{stroke}%
\end{pgfscope}%
\begin{pgfscope}%
\pgfpathrectangle{\pgfqpoint{0.800000in}{3.252941in}}{\pgfqpoint{2.407767in}{1.544118in}}%
\pgfusepath{clip}%
\pgfsetbuttcap%
\pgfsetroundjoin%
\pgfsetlinewidth{0.501875pt}%
\definecolor{currentstroke}{rgb}{0.277941,0.056324,0.381191}%
\pgfsetstrokecolor{currentstroke}%
\pgfsetdash{}{0pt}%
\pgfpathmoveto{\pgfqpoint{2.548132in}{3.679081in}}%
\pgfpathlineto{\pgfqpoint{2.495221in}{3.680630in}}%
\pgfusepath{stroke}%
\end{pgfscope}%
\begin{pgfscope}%
\pgfpathrectangle{\pgfqpoint{0.800000in}{3.252941in}}{\pgfqpoint{2.407767in}{1.544118in}}%
\pgfusepath{clip}%
\pgfsetbuttcap%
\pgfsetroundjoin%
\pgfsetlinewidth{0.501875pt}%
\definecolor{currentstroke}{rgb}{0.280267,0.073417,0.397163}%
\pgfsetstrokecolor{currentstroke}%
\pgfsetdash{}{0pt}%
\pgfpathmoveto{\pgfqpoint{2.495221in}{3.680630in}}%
\pgfpathlineto{\pgfqpoint{2.442380in}{3.683033in}}%
\pgfusepath{stroke}%
\end{pgfscope}%
\begin{pgfscope}%
\pgfpathrectangle{\pgfqpoint{0.800000in}{3.252941in}}{\pgfqpoint{2.407767in}{1.544118in}}%
\pgfusepath{clip}%
\pgfsetbuttcap%
\pgfsetroundjoin%
\pgfsetlinewidth{0.501875pt}%
\definecolor{currentstroke}{rgb}{0.280894,0.078907,0.402329}%
\pgfsetstrokecolor{currentstroke}%
\pgfsetdash{}{0pt}%
\pgfpathmoveto{\pgfqpoint{2.442380in}{3.683033in}}%
\pgfpathlineto{\pgfqpoint{2.389633in}{3.686146in}}%
\pgfusepath{stroke}%
\end{pgfscope}%
\begin{pgfscope}%
\pgfpathrectangle{\pgfqpoint{0.800000in}{3.252941in}}{\pgfqpoint{2.407767in}{1.544118in}}%
\pgfusepath{clip}%
\pgfsetbuttcap%
\pgfsetroundjoin%
\pgfsetlinewidth{0.501875pt}%
\definecolor{currentstroke}{rgb}{0.280267,0.073417,0.397163}%
\pgfsetstrokecolor{currentstroke}%
\pgfsetdash{}{0pt}%
\pgfpathmoveto{\pgfqpoint{2.389633in}{3.686146in}}%
\pgfpathlineto{\pgfqpoint{2.337049in}{3.690238in}}%
\pgfusepath{stroke}%
\end{pgfscope}%
\begin{pgfscope}%
\pgfpathrectangle{\pgfqpoint{0.800000in}{3.252941in}}{\pgfqpoint{2.407767in}{1.544118in}}%
\pgfusepath{clip}%
\pgfsetbuttcap%
\pgfsetroundjoin%
\pgfsetlinewidth{0.501875pt}%
\definecolor{currentstroke}{rgb}{0.283197,0.115680,0.436115}%
\pgfsetstrokecolor{currentstroke}%
\pgfsetdash{}{0pt}%
\pgfpathmoveto{\pgfqpoint{2.337049in}{3.690238in}}%
\pgfpathlineto{\pgfqpoint{2.284828in}{3.695837in}}%
\pgfusepath{stroke}%
\end{pgfscope}%
\begin{pgfscope}%
\pgfpathrectangle{\pgfqpoint{0.800000in}{3.252941in}}{\pgfqpoint{2.407767in}{1.544118in}}%
\pgfusepath{clip}%
\pgfsetbuttcap%
\pgfsetroundjoin%
\pgfsetlinewidth{0.501875pt}%
\definecolor{currentstroke}{rgb}{0.281446,0.084320,0.407414}%
\pgfsetstrokecolor{currentstroke}%
\pgfsetdash{}{0pt}%
\pgfpathmoveto{\pgfqpoint{2.284828in}{3.695837in}}%
\pgfpathlineto{\pgfqpoint{2.233011in}{3.702859in}}%
\pgfusepath{stroke}%
\end{pgfscope}%
\begin{pgfscope}%
\pgfpathrectangle{\pgfqpoint{0.800000in}{3.252941in}}{\pgfqpoint{2.407767in}{1.544118in}}%
\pgfusepath{clip}%
\pgfsetbuttcap%
\pgfsetroundjoin%
\pgfsetlinewidth{0.501875pt}%
\definecolor{currentstroke}{rgb}{0.282327,0.094955,0.417331}%
\pgfsetstrokecolor{currentstroke}%
\pgfsetdash{}{0pt}%
\pgfpathmoveto{\pgfqpoint{2.233011in}{3.702859in}}%
\pgfpathlineto{\pgfqpoint{2.181798in}{3.711425in}}%
\pgfusepath{stroke}%
\end{pgfscope}%
\begin{pgfscope}%
\pgfpathrectangle{\pgfqpoint{0.800000in}{3.252941in}}{\pgfqpoint{2.407767in}{1.544118in}}%
\pgfusepath{clip}%
\pgfsetbuttcap%
\pgfsetroundjoin%
\pgfsetlinewidth{0.501875pt}%
\definecolor{currentstroke}{rgb}{0.282327,0.094955,0.417331}%
\pgfsetstrokecolor{currentstroke}%
\pgfsetdash{}{0pt}%
\pgfpathmoveto{\pgfqpoint{2.181798in}{3.711425in}}%
\pgfpathlineto{\pgfqpoint{2.132539in}{3.723459in}}%
\pgfusepath{stroke}%
\end{pgfscope}%
\begin{pgfscope}%
\pgfpathrectangle{\pgfqpoint{0.800000in}{3.252941in}}{\pgfqpoint{2.407767in}{1.544118in}}%
\pgfusepath{clip}%
\pgfsetbuttcap%
\pgfsetroundjoin%
\pgfsetlinewidth{0.501875pt}%
\definecolor{currentstroke}{rgb}{0.282327,0.094955,0.417331}%
\pgfsetstrokecolor{currentstroke}%
\pgfsetdash{}{0pt}%
\pgfpathmoveto{\pgfqpoint{2.132539in}{3.723459in}}%
\pgfpathlineto{\pgfqpoint{2.084748in}{3.737717in}}%
\pgfusepath{stroke}%
\end{pgfscope}%
\begin{pgfscope}%
\pgfpathrectangle{\pgfqpoint{0.800000in}{3.252941in}}{\pgfqpoint{2.407767in}{1.544118in}}%
\pgfusepath{clip}%
\pgfsetbuttcap%
\pgfsetroundjoin%
\pgfsetlinewidth{0.501875pt}%
\definecolor{currentstroke}{rgb}{0.281924,0.089666,0.412415}%
\pgfsetstrokecolor{currentstroke}%
\pgfsetdash{}{0pt}%
\pgfpathmoveto{\pgfqpoint{2.084748in}{3.737717in}}%
\pgfpathlineto{\pgfqpoint{2.039427in}{3.754935in}}%
\pgfusepath{stroke}%
\end{pgfscope}%
\begin{pgfscope}%
\pgfpathrectangle{\pgfqpoint{0.800000in}{3.252941in}}{\pgfqpoint{2.407767in}{1.544118in}}%
\pgfusepath{clip}%
\pgfsetbuttcap%
\pgfsetroundjoin%
\pgfsetlinewidth{0.501875pt}%
\definecolor{currentstroke}{rgb}{0.282910,0.105393,0.426902}%
\pgfsetstrokecolor{currentstroke}%
\pgfsetdash{}{0pt}%
\pgfpathmoveto{\pgfqpoint{2.039427in}{3.754935in}}%
\pgfpathlineto{\pgfqpoint{1.996823in}{3.774825in}}%
\pgfusepath{stroke}%
\end{pgfscope}%
\begin{pgfscope}%
\pgfpathrectangle{\pgfqpoint{0.800000in}{3.252941in}}{\pgfqpoint{2.407767in}{1.544118in}}%
\pgfusepath{clip}%
\pgfsetbuttcap%
\pgfsetroundjoin%
\pgfsetlinewidth{0.501875pt}%
\definecolor{currentstroke}{rgb}{0.276194,0.190074,0.493001}%
\pgfsetstrokecolor{currentstroke}%
\pgfsetdash{}{0pt}%
\pgfpathmoveto{\pgfqpoint{1.996823in}{3.774825in}}%
\pgfpathlineto{\pgfqpoint{1.996823in}{3.774825in}}%
\pgfusepath{stroke}%
\end{pgfscope}%
\begin{pgfscope}%
\pgfpathrectangle{\pgfqpoint{0.800000in}{3.252941in}}{\pgfqpoint{2.407767in}{1.544118in}}%
\pgfusepath{clip}%
\pgfsetbuttcap%
\pgfsetroundjoin%
\pgfsetlinewidth{0.501875pt}%
\definecolor{currentstroke}{rgb}{0.276194,0.190074,0.493001}%
\pgfsetstrokecolor{currentstroke}%
\pgfsetdash{}{0pt}%
\pgfpathmoveto{\pgfqpoint{1.996823in}{3.774825in}}%
\pgfpathlineto{\pgfqpoint{1.975423in}{3.789490in}}%
\pgfusepath{stroke}%
\end{pgfscope}%
\begin{pgfscope}%
\pgfpathrectangle{\pgfqpoint{0.800000in}{3.252941in}}{\pgfqpoint{2.407767in}{1.544118in}}%
\pgfusepath{clip}%
\pgfsetbuttcap%
\pgfsetroundjoin%
\pgfsetlinewidth{0.501875pt}%
\definecolor{currentstroke}{rgb}{0.281412,0.155834,0.469201}%
\pgfsetstrokecolor{currentstroke}%
\pgfsetdash{}{0pt}%
\pgfpathmoveto{\pgfqpoint{1.975423in}{3.789490in}}%
\pgfpathlineto{\pgfqpoint{1.955363in}{3.806234in}}%
\pgfusepath{stroke}%
\end{pgfscope}%
\begin{pgfscope}%
\pgfpathrectangle{\pgfqpoint{0.800000in}{3.252941in}}{\pgfqpoint{2.407767in}{1.544118in}}%
\pgfusepath{clip}%
\pgfsetbuttcap%
\pgfsetroundjoin%
\pgfsetlinewidth{0.501875pt}%
\definecolor{currentstroke}{rgb}{0.276194,0.190074,0.493001}%
\pgfsetstrokecolor{currentstroke}%
\pgfsetdash{}{0pt}%
\pgfpathmoveto{\pgfqpoint{1.955363in}{3.806234in}}%
\pgfpathlineto{\pgfqpoint{1.919572in}{3.830614in}}%
\pgfusepath{stroke}%
\end{pgfscope}%
\begin{pgfscope}%
\pgfpathrectangle{\pgfqpoint{0.800000in}{3.252941in}}{\pgfqpoint{2.407767in}{1.544118in}}%
\pgfusepath{clip}%
\pgfsetbuttcap%
\pgfsetroundjoin%
\pgfsetlinewidth{0.501875pt}%
\definecolor{currentstroke}{rgb}{0.267968,0.223549,0.512008}%
\pgfsetstrokecolor{currentstroke}%
\pgfsetdash{}{0pt}%
\pgfpathmoveto{\pgfqpoint{1.919572in}{3.830614in}}%
\pgfpathlineto{\pgfqpoint{1.877962in}{3.851373in}}%
\pgfusepath{stroke}%
\end{pgfscope}%
\begin{pgfscope}%
\pgfpathrectangle{\pgfqpoint{0.800000in}{3.252941in}}{\pgfqpoint{2.407767in}{1.544118in}}%
\pgfusepath{clip}%
\pgfsetbuttcap%
\pgfsetroundjoin%
\pgfsetlinewidth{0.501875pt}%
\definecolor{currentstroke}{rgb}{0.150476,0.504369,0.557430}%
\pgfsetstrokecolor{currentstroke}%
\pgfsetdash{}{0pt}%
\pgfpathmoveto{\pgfqpoint{1.877962in}{3.851373in}}%
\pgfpathlineto{\pgfqpoint{1.834061in}{3.870343in}}%
\pgfusepath{stroke}%
\end{pgfscope}%
\begin{pgfscope}%
\pgfpathrectangle{\pgfqpoint{0.800000in}{3.252941in}}{\pgfqpoint{2.407767in}{1.544118in}}%
\pgfusepath{clip}%
\pgfsetbuttcap%
\pgfsetroundjoin%
\pgfsetlinewidth{0.501875pt}%
\definecolor{currentstroke}{rgb}{0.146616,0.673050,0.508936}%
\pgfsetstrokecolor{currentstroke}%
\pgfsetdash{}{0pt}%
\pgfpathmoveto{\pgfqpoint{1.834061in}{3.870343in}}%
\pgfpathlineto{\pgfqpoint{1.789704in}{3.888900in}}%
\pgfusepath{stroke}%
\end{pgfscope}%
\begin{pgfscope}%
\pgfpathrectangle{\pgfqpoint{0.800000in}{3.252941in}}{\pgfqpoint{2.407767in}{1.544118in}}%
\pgfusepath{clip}%
\pgfsetbuttcap%
\pgfsetroundjoin%
\pgfsetlinewidth{0.501875pt}%
\definecolor{currentstroke}{rgb}{0.272594,0.025563,0.353093}%
\pgfsetstrokecolor{currentstroke}%
\pgfsetdash{}{0pt}%
\pgfpathmoveto{\pgfqpoint{2.654046in}{3.712285in}}%
\pgfpathlineto{\pgfqpoint{2.601086in}{3.713004in}}%
\pgfusepath{stroke}%
\end{pgfscope}%
\begin{pgfscope}%
\pgfpathrectangle{\pgfqpoint{0.800000in}{3.252941in}}{\pgfqpoint{2.407767in}{1.544118in}}%
\pgfusepath{clip}%
\pgfsetbuttcap%
\pgfsetroundjoin%
\pgfsetlinewidth{0.501875pt}%
\definecolor{currentstroke}{rgb}{0.277018,0.050344,0.375715}%
\pgfsetstrokecolor{currentstroke}%
\pgfsetdash{}{0pt}%
\pgfpathmoveto{\pgfqpoint{2.601086in}{3.713004in}}%
\pgfpathlineto{\pgfqpoint{2.548150in}{3.714308in}}%
\pgfusepath{stroke}%
\end{pgfscope}%
\begin{pgfscope}%
\pgfpathrectangle{\pgfqpoint{0.800000in}{3.252941in}}{\pgfqpoint{2.407767in}{1.544118in}}%
\pgfusepath{clip}%
\pgfsetbuttcap%
\pgfsetroundjoin%
\pgfsetlinewidth{0.501875pt}%
\definecolor{currentstroke}{rgb}{0.278791,0.062145,0.386592}%
\pgfsetstrokecolor{currentstroke}%
\pgfsetdash{}{0pt}%
\pgfpathmoveto{\pgfqpoint{2.548150in}{3.714308in}}%
\pgfpathlineto{\pgfqpoint{2.495256in}{3.716130in}}%
\pgfusepath{stroke}%
\end{pgfscope}%
\begin{pgfscope}%
\pgfpathrectangle{\pgfqpoint{0.800000in}{3.252941in}}{\pgfqpoint{2.407767in}{1.544118in}}%
\pgfusepath{clip}%
\pgfsetbuttcap%
\pgfsetroundjoin%
\pgfsetlinewidth{0.501875pt}%
\definecolor{currentstroke}{rgb}{0.281446,0.084320,0.407414}%
\pgfsetstrokecolor{currentstroke}%
\pgfsetdash{}{0pt}%
\pgfpathmoveto{\pgfqpoint{2.495256in}{3.716130in}}%
\pgfpathlineto{\pgfqpoint{2.442418in}{3.718539in}}%
\pgfusepath{stroke}%
\end{pgfscope}%
\begin{pgfscope}%
\pgfpathrectangle{\pgfqpoint{0.800000in}{3.252941in}}{\pgfqpoint{2.407767in}{1.544118in}}%
\pgfusepath{clip}%
\pgfsetbuttcap%
\pgfsetroundjoin%
\pgfsetlinewidth{0.501875pt}%
\definecolor{currentstroke}{rgb}{0.281924,0.089666,0.412415}%
\pgfsetstrokecolor{currentstroke}%
\pgfsetdash{}{0pt}%
\pgfpathmoveto{\pgfqpoint{2.442418in}{3.718539in}}%
\pgfpathlineto{\pgfqpoint{2.389680in}{3.721700in}}%
\pgfusepath{stroke}%
\end{pgfscope}%
\begin{pgfscope}%
\pgfpathrectangle{\pgfqpoint{0.800000in}{3.252941in}}{\pgfqpoint{2.407767in}{1.544118in}}%
\pgfusepath{clip}%
\pgfsetbuttcap%
\pgfsetroundjoin%
\pgfsetlinewidth{0.501875pt}%
\definecolor{currentstroke}{rgb}{0.283072,0.130895,0.449241}%
\pgfsetstrokecolor{currentstroke}%
\pgfsetdash{}{0pt}%
\pgfpathmoveto{\pgfqpoint{2.389680in}{3.721700in}}%
\pgfpathlineto{\pgfqpoint{2.337127in}{3.725946in}}%
\pgfusepath{stroke}%
\end{pgfscope}%
\begin{pgfscope}%
\pgfpathrectangle{\pgfqpoint{0.800000in}{3.252941in}}{\pgfqpoint{2.407767in}{1.544118in}}%
\pgfusepath{clip}%
\pgfsetbuttcap%
\pgfsetroundjoin%
\pgfsetlinewidth{0.501875pt}%
\definecolor{currentstroke}{rgb}{0.283072,0.130895,0.449241}%
\pgfsetstrokecolor{currentstroke}%
\pgfsetdash{}{0pt}%
\pgfpathmoveto{\pgfqpoint{2.337127in}{3.725946in}}%
\pgfpathlineto{\pgfqpoint{2.284740in}{3.730954in}}%
\pgfusepath{stroke}%
\end{pgfscope}%
\begin{pgfscope}%
\pgfpathrectangle{\pgfqpoint{0.800000in}{3.252941in}}{\pgfqpoint{2.407767in}{1.544118in}}%
\pgfusepath{clip}%
\pgfsetbuttcap%
\pgfsetroundjoin%
\pgfsetlinewidth{0.501875pt}%
\definecolor{currentstroke}{rgb}{0.282327,0.094955,0.417331}%
\pgfsetstrokecolor{currentstroke}%
\pgfsetdash{}{0pt}%
\pgfpathmoveto{\pgfqpoint{2.284740in}{3.730954in}}%
\pgfpathlineto{\pgfqpoint{2.232632in}{3.736937in}}%
\pgfusepath{stroke}%
\end{pgfscope}%
\begin{pgfscope}%
\pgfpathrectangle{\pgfqpoint{0.800000in}{3.252941in}}{\pgfqpoint{2.407767in}{1.544118in}}%
\pgfusepath{clip}%
\pgfsetbuttcap%
\pgfsetroundjoin%
\pgfsetlinewidth{0.501875pt}%
\definecolor{currentstroke}{rgb}{0.282884,0.135920,0.453427}%
\pgfsetstrokecolor{currentstroke}%
\pgfsetdash{}{0pt}%
\pgfpathmoveto{\pgfqpoint{2.232632in}{3.736937in}}%
\pgfpathlineto{\pgfqpoint{2.181281in}{3.745124in}}%
\pgfusepath{stroke}%
\end{pgfscope}%
\begin{pgfscope}%
\pgfpathrectangle{\pgfqpoint{0.800000in}{3.252941in}}{\pgfqpoint{2.407767in}{1.544118in}}%
\pgfusepath{clip}%
\pgfsetbuttcap%
\pgfsetroundjoin%
\pgfsetlinewidth{0.501875pt}%
\definecolor{currentstroke}{rgb}{0.282884,0.135920,0.453427}%
\pgfsetstrokecolor{currentstroke}%
\pgfsetdash{}{0pt}%
\pgfpathmoveto{\pgfqpoint{2.181281in}{3.745124in}}%
\pgfpathlineto{\pgfqpoint{2.131567in}{3.756503in}}%
\pgfusepath{stroke}%
\end{pgfscope}%
\begin{pgfscope}%
\pgfpathrectangle{\pgfqpoint{0.800000in}{3.252941in}}{\pgfqpoint{2.407767in}{1.544118in}}%
\pgfusepath{clip}%
\pgfsetbuttcap%
\pgfsetroundjoin%
\pgfsetlinewidth{0.501875pt}%
\definecolor{currentstroke}{rgb}{0.273809,0.031497,0.358853}%
\pgfsetstrokecolor{currentstroke}%
\pgfsetdash{}{0pt}%
\pgfpathmoveto{\pgfqpoint{2.654046in}{3.747031in}}%
\pgfpathlineto{\pgfqpoint{2.601081in}{3.747666in}}%
\pgfusepath{stroke}%
\end{pgfscope}%
\begin{pgfscope}%
\pgfpathrectangle{\pgfqpoint{0.800000in}{3.252941in}}{\pgfqpoint{2.407767in}{1.544118in}}%
\pgfusepath{clip}%
\pgfsetbuttcap%
\pgfsetroundjoin%
\pgfsetlinewidth{0.501875pt}%
\definecolor{currentstroke}{rgb}{0.277941,0.056324,0.381191}%
\pgfsetstrokecolor{currentstroke}%
\pgfsetdash{}{0pt}%
\pgfpathmoveto{\pgfqpoint{2.601081in}{3.747666in}}%
\pgfpathlineto{\pgfqpoint{2.548123in}{3.748540in}}%
\pgfusepath{stroke}%
\end{pgfscope}%
\begin{pgfscope}%
\pgfpathrectangle{\pgfqpoint{0.800000in}{3.252941in}}{\pgfqpoint{2.407767in}{1.544118in}}%
\pgfusepath{clip}%
\pgfsetbuttcap%
\pgfsetroundjoin%
\pgfsetlinewidth{0.501875pt}%
\definecolor{currentstroke}{rgb}{0.280894,0.078907,0.402329}%
\pgfsetstrokecolor{currentstroke}%
\pgfsetdash{}{0pt}%
\pgfpathmoveto{\pgfqpoint{2.548123in}{3.748540in}}%
\pgfpathlineto{\pgfqpoint{2.495197in}{3.749950in}}%
\pgfusepath{stroke}%
\end{pgfscope}%
\begin{pgfscope}%
\pgfpathrectangle{\pgfqpoint{0.800000in}{3.252941in}}{\pgfqpoint{2.407767in}{1.544118in}}%
\pgfusepath{clip}%
\pgfsetbuttcap%
\pgfsetroundjoin%
\pgfsetlinewidth{0.501875pt}%
\definecolor{currentstroke}{rgb}{0.283091,0.110553,0.431554}%
\pgfsetstrokecolor{currentstroke}%
\pgfsetdash{}{0pt}%
\pgfpathmoveto{\pgfqpoint{2.495197in}{3.749950in}}%
\pgfpathlineto{\pgfqpoint{2.442334in}{3.752128in}}%
\pgfusepath{stroke}%
\end{pgfscope}%
\begin{pgfscope}%
\pgfpathrectangle{\pgfqpoint{0.800000in}{3.252941in}}{\pgfqpoint{2.407767in}{1.544118in}}%
\pgfusepath{clip}%
\pgfsetbuttcap%
\pgfsetroundjoin%
\pgfsetlinewidth{0.501875pt}%
\definecolor{currentstroke}{rgb}{0.283187,0.125848,0.444960}%
\pgfsetstrokecolor{currentstroke}%
\pgfsetdash{}{0pt}%
\pgfpathmoveto{\pgfqpoint{2.442334in}{3.752128in}}%
\pgfpathlineto{\pgfqpoint{2.389556in}{3.755012in}}%
\pgfusepath{stroke}%
\end{pgfscope}%
\begin{pgfscope}%
\pgfpathrectangle{\pgfqpoint{0.800000in}{3.252941in}}{\pgfqpoint{2.407767in}{1.544118in}}%
\pgfusepath{clip}%
\pgfsetbuttcap%
\pgfsetroundjoin%
\pgfsetlinewidth{0.501875pt}%
\definecolor{currentstroke}{rgb}{0.282290,0.145912,0.461510}%
\pgfsetstrokecolor{currentstroke}%
\pgfsetdash{}{0pt}%
\pgfpathmoveto{\pgfqpoint{2.389556in}{3.755012in}}%
\pgfpathlineto{\pgfqpoint{2.336942in}{3.758897in}}%
\pgfusepath{stroke}%
\end{pgfscope}%
\begin{pgfscope}%
\pgfpathrectangle{\pgfqpoint{0.800000in}{3.252941in}}{\pgfqpoint{2.407767in}{1.544118in}}%
\pgfusepath{clip}%
\pgfsetbuttcap%
\pgfsetroundjoin%
\pgfsetlinewidth{0.501875pt}%
\definecolor{currentstroke}{rgb}{0.273809,0.031497,0.358853}%
\pgfsetstrokecolor{currentstroke}%
\pgfsetdash{}{0pt}%
\pgfpathmoveto{\pgfqpoint{2.654046in}{3.781777in}}%
\pgfpathlineto{\pgfqpoint{2.601073in}{3.782167in}}%
\pgfusepath{stroke}%
\end{pgfscope}%
\begin{pgfscope}%
\pgfpathrectangle{\pgfqpoint{0.800000in}{3.252941in}}{\pgfqpoint{2.407767in}{1.544118in}}%
\pgfusepath{clip}%
\pgfsetbuttcap%
\pgfsetroundjoin%
\pgfsetlinewidth{0.501875pt}%
\definecolor{currentstroke}{rgb}{0.279566,0.067836,0.391917}%
\pgfsetstrokecolor{currentstroke}%
\pgfsetdash{}{0pt}%
\pgfpathmoveto{\pgfqpoint{2.601073in}{3.782167in}}%
\pgfpathlineto{\pgfqpoint{2.548117in}{3.782972in}}%
\pgfusepath{stroke}%
\end{pgfscope}%
\begin{pgfscope}%
\pgfpathrectangle{\pgfqpoint{0.800000in}{3.252941in}}{\pgfqpoint{2.407767in}{1.544118in}}%
\pgfusepath{clip}%
\pgfsetbuttcap%
\pgfsetroundjoin%
\pgfsetlinewidth{0.501875pt}%
\definecolor{currentstroke}{rgb}{0.282656,0.100196,0.422160}%
\pgfsetstrokecolor{currentstroke}%
\pgfsetdash{}{0pt}%
\pgfpathmoveto{\pgfqpoint{2.548117in}{3.782972in}}%
\pgfpathlineto{\pgfqpoint{2.495211in}{3.784640in}}%
\pgfusepath{stroke}%
\end{pgfscope}%
\begin{pgfscope}%
\pgfpathrectangle{\pgfqpoint{0.800000in}{3.252941in}}{\pgfqpoint{2.407767in}{1.544118in}}%
\pgfusepath{clip}%
\pgfsetbuttcap%
\pgfsetroundjoin%
\pgfsetlinewidth{0.501875pt}%
\definecolor{currentstroke}{rgb}{0.283072,0.130895,0.449241}%
\pgfsetstrokecolor{currentstroke}%
\pgfsetdash{}{0pt}%
\pgfpathmoveto{\pgfqpoint{2.495211in}{3.784640in}}%
\pgfpathlineto{\pgfqpoint{2.442348in}{3.786844in}}%
\pgfusepath{stroke}%
\end{pgfscope}%
\begin{pgfscope}%
\pgfpathrectangle{\pgfqpoint{0.800000in}{3.252941in}}{\pgfqpoint{2.407767in}{1.544118in}}%
\pgfusepath{clip}%
\pgfsetbuttcap%
\pgfsetroundjoin%
\pgfsetlinewidth{0.501875pt}%
\definecolor{currentstroke}{rgb}{0.282884,0.135920,0.453427}%
\pgfsetstrokecolor{currentstroke}%
\pgfsetdash{}{0pt}%
\pgfpathmoveto{\pgfqpoint{2.442348in}{3.786844in}}%
\pgfpathlineto{\pgfqpoint{2.389532in}{3.789457in}}%
\pgfusepath{stroke}%
\end{pgfscope}%
\begin{pgfscope}%
\pgfpathrectangle{\pgfqpoint{0.800000in}{3.252941in}}{\pgfqpoint{2.407767in}{1.544118in}}%
\pgfusepath{clip}%
\pgfsetbuttcap%
\pgfsetroundjoin%
\pgfsetlinewidth{0.501875pt}%
\definecolor{currentstroke}{rgb}{0.279574,0.170599,0.479997}%
\pgfsetstrokecolor{currentstroke}%
\pgfsetdash{}{0pt}%
\pgfpathmoveto{\pgfqpoint{2.389532in}{3.789457in}}%
\pgfpathlineto{\pgfqpoint{2.336825in}{3.792832in}}%
\pgfusepath{stroke}%
\end{pgfscope}%
\begin{pgfscope}%
\pgfpathrectangle{\pgfqpoint{0.800000in}{3.252941in}}{\pgfqpoint{2.407767in}{1.544118in}}%
\pgfusepath{clip}%
\pgfsetbuttcap%
\pgfsetroundjoin%
\pgfsetlinewidth{0.501875pt}%
\definecolor{currentstroke}{rgb}{0.276194,0.190074,0.493001}%
\pgfsetstrokecolor{currentstroke}%
\pgfsetdash{}{0pt}%
\pgfpathmoveto{\pgfqpoint{2.336825in}{3.792832in}}%
\pgfpathlineto{\pgfqpoint{2.284293in}{3.797189in}}%
\pgfusepath{stroke}%
\end{pgfscope}%
\begin{pgfscope}%
\pgfpathrectangle{\pgfqpoint{0.800000in}{3.252941in}}{\pgfqpoint{2.407767in}{1.544118in}}%
\pgfusepath{clip}%
\pgfsetbuttcap%
\pgfsetroundjoin%
\pgfsetlinewidth{0.501875pt}%
\definecolor{currentstroke}{rgb}{0.277134,0.185228,0.489898}%
\pgfsetstrokecolor{currentstroke}%
\pgfsetdash{}{0pt}%
\pgfpathmoveto{\pgfqpoint{2.284293in}{3.797189in}}%
\pgfpathlineto{\pgfqpoint{2.231976in}{3.802491in}}%
\pgfusepath{stroke}%
\end{pgfscope}%
\begin{pgfscope}%
\pgfpathrectangle{\pgfqpoint{0.800000in}{3.252941in}}{\pgfqpoint{2.407767in}{1.544118in}}%
\pgfusepath{clip}%
\pgfsetbuttcap%
\pgfsetroundjoin%
\pgfsetlinewidth{0.501875pt}%
\definecolor{currentstroke}{rgb}{0.281887,0.150881,0.465405}%
\pgfsetstrokecolor{currentstroke}%
\pgfsetdash{}{0pt}%
\pgfpathmoveto{\pgfqpoint{2.231976in}{3.802491in}}%
\pgfpathlineto{\pgfqpoint{2.180114in}{3.809311in}}%
\pgfusepath{stroke}%
\end{pgfscope}%
\begin{pgfscope}%
\pgfpathrectangle{\pgfqpoint{0.800000in}{3.252941in}}{\pgfqpoint{2.407767in}{1.544118in}}%
\pgfusepath{clip}%
\pgfsetbuttcap%
\pgfsetroundjoin%
\pgfsetlinewidth{0.501875pt}%
\definecolor{currentstroke}{rgb}{0.281887,0.150881,0.465405}%
\pgfsetstrokecolor{currentstroke}%
\pgfsetdash{}{0pt}%
\pgfpathmoveto{\pgfqpoint{2.180114in}{3.809311in}}%
\pgfpathlineto{\pgfqpoint{2.128874in}{3.817898in}}%
\pgfusepath{stroke}%
\end{pgfscope}%
\begin{pgfscope}%
\pgfpathrectangle{\pgfqpoint{0.800000in}{3.252941in}}{\pgfqpoint{2.407767in}{1.544118in}}%
\pgfusepath{clip}%
\pgfsetbuttcap%
\pgfsetroundjoin%
\pgfsetlinewidth{0.501875pt}%
\definecolor{currentstroke}{rgb}{0.265145,0.232956,0.516599}%
\pgfsetstrokecolor{currentstroke}%
\pgfsetdash{}{0pt}%
\pgfpathmoveto{\pgfqpoint{2.128874in}{3.817898in}}%
\pgfpathlineto{\pgfqpoint{2.078675in}{3.828615in}}%
\pgfusepath{stroke}%
\end{pgfscope}%
\begin{pgfscope}%
\pgfpathrectangle{\pgfqpoint{0.800000in}{3.252941in}}{\pgfqpoint{2.407767in}{1.544118in}}%
\pgfusepath{clip}%
\pgfsetbuttcap%
\pgfsetroundjoin%
\pgfsetlinewidth{0.501875pt}%
\definecolor{currentstroke}{rgb}{0.269308,0.218818,0.509577}%
\pgfsetstrokecolor{currentstroke}%
\pgfsetdash{}{0pt}%
\pgfpathmoveto{\pgfqpoint{2.078675in}{3.828615in}}%
\pgfpathlineto{\pgfqpoint{2.031089in}{3.842720in}}%
\pgfusepath{stroke}%
\end{pgfscope}%
\begin{pgfscope}%
\pgfpathrectangle{\pgfqpoint{0.800000in}{3.252941in}}{\pgfqpoint{2.407767in}{1.544118in}}%
\pgfusepath{clip}%
\pgfsetbuttcap%
\pgfsetroundjoin%
\pgfsetlinewidth{0.501875pt}%
\definecolor{currentstroke}{rgb}{0.283091,0.110553,0.431554}%
\pgfsetstrokecolor{currentstroke}%
\pgfsetdash{}{0pt}%
\pgfpathmoveto{\pgfqpoint{2.031089in}{3.842720in}}%
\pgfpathlineto{\pgfqpoint{1.984341in}{3.858192in}}%
\pgfusepath{stroke}%
\end{pgfscope}%
\begin{pgfscope}%
\pgfpathrectangle{\pgfqpoint{0.800000in}{3.252941in}}{\pgfqpoint{2.407767in}{1.544118in}}%
\pgfusepath{clip}%
\pgfsetbuttcap%
\pgfsetroundjoin%
\pgfsetlinewidth{0.501875pt}%
\definecolor{currentstroke}{rgb}{0.270595,0.214069,0.507052}%
\pgfsetstrokecolor{currentstroke}%
\pgfsetdash{}{0pt}%
\pgfpathmoveto{\pgfqpoint{1.984341in}{3.858192in}}%
\pgfpathlineto{\pgfqpoint{1.936951in}{3.873292in}}%
\pgfusepath{stroke}%
\end{pgfscope}%
\begin{pgfscope}%
\pgfpathrectangle{\pgfqpoint{0.800000in}{3.252941in}}{\pgfqpoint{2.407767in}{1.544118in}}%
\pgfusepath{clip}%
\pgfsetbuttcap%
\pgfsetroundjoin%
\pgfsetlinewidth{0.501875pt}%
\definecolor{currentstroke}{rgb}{0.121380,0.629492,0.531973}%
\pgfsetstrokecolor{currentstroke}%
\pgfsetdash{}{0pt}%
\pgfpathmoveto{\pgfqpoint{1.936951in}{3.873292in}}%
\pgfpathlineto{\pgfqpoint{1.889320in}{3.888152in}}%
\pgfusepath{stroke}%
\end{pgfscope}%
\begin{pgfscope}%
\pgfpathrectangle{\pgfqpoint{0.800000in}{3.252941in}}{\pgfqpoint{2.407767in}{1.544118in}}%
\pgfusepath{clip}%
\pgfsetbuttcap%
\pgfsetroundjoin%
\pgfsetlinewidth{0.501875pt}%
\definecolor{currentstroke}{rgb}{0.274952,0.037752,0.364543}%
\pgfsetstrokecolor{currentstroke}%
\pgfsetdash{}{0pt}%
\pgfpathmoveto{\pgfqpoint{2.654046in}{3.816523in}}%
\pgfpathlineto{\pgfqpoint{2.601082in}{3.817219in}}%
\pgfusepath{stroke}%
\end{pgfscope}%
\begin{pgfscope}%
\pgfpathrectangle{\pgfqpoint{0.800000in}{3.252941in}}{\pgfqpoint{2.407767in}{1.544118in}}%
\pgfusepath{clip}%
\pgfsetbuttcap%
\pgfsetroundjoin%
\pgfsetlinewidth{0.501875pt}%
\definecolor{currentstroke}{rgb}{0.281446,0.084320,0.407414}%
\pgfsetstrokecolor{currentstroke}%
\pgfsetdash{}{0pt}%
\pgfpathmoveto{\pgfqpoint{2.601082in}{3.817219in}}%
\pgfpathlineto{\pgfqpoint{2.548129in}{3.818197in}}%
\pgfusepath{stroke}%
\end{pgfscope}%
\begin{pgfscope}%
\pgfpathrectangle{\pgfqpoint{0.800000in}{3.252941in}}{\pgfqpoint{2.407767in}{1.544118in}}%
\pgfusepath{clip}%
\pgfsetbuttcap%
\pgfsetroundjoin%
\pgfsetlinewidth{0.501875pt}%
\definecolor{currentstroke}{rgb}{0.283197,0.115680,0.436115}%
\pgfsetstrokecolor{currentstroke}%
\pgfsetdash{}{0pt}%
\pgfpathmoveto{\pgfqpoint{2.548129in}{3.818197in}}%
\pgfpathlineto{\pgfqpoint{2.495188in}{3.819410in}}%
\pgfusepath{stroke}%
\end{pgfscope}%
\begin{pgfscope}%
\pgfpathrectangle{\pgfqpoint{0.800000in}{3.252941in}}{\pgfqpoint{2.407767in}{1.544118in}}%
\pgfusepath{clip}%
\pgfsetbuttcap%
\pgfsetroundjoin%
\pgfsetlinewidth{0.501875pt}%
\definecolor{currentstroke}{rgb}{0.281412,0.155834,0.469201}%
\pgfsetstrokecolor{currentstroke}%
\pgfsetdash{}{0pt}%
\pgfpathmoveto{\pgfqpoint{2.495188in}{3.819410in}}%
\pgfpathlineto{\pgfqpoint{2.442273in}{3.820999in}}%
\pgfusepath{stroke}%
\end{pgfscope}%
\begin{pgfscope}%
\pgfpathrectangle{\pgfqpoint{0.800000in}{3.252941in}}{\pgfqpoint{2.407767in}{1.544118in}}%
\pgfusepath{clip}%
\pgfsetbuttcap%
\pgfsetroundjoin%
\pgfsetlinewidth{0.501875pt}%
\definecolor{currentstroke}{rgb}{0.278826,0.175490,0.483397}%
\pgfsetstrokecolor{currentstroke}%
\pgfsetdash{}{0pt}%
\pgfpathmoveto{\pgfqpoint{2.442273in}{3.820999in}}%
\pgfpathlineto{\pgfqpoint{2.389394in}{3.823030in}}%
\pgfusepath{stroke}%
\end{pgfscope}%
\begin{pgfscope}%
\pgfpathrectangle{\pgfqpoint{0.800000in}{3.252941in}}{\pgfqpoint{2.407767in}{1.544118in}}%
\pgfusepath{clip}%
\pgfsetbuttcap%
\pgfsetroundjoin%
\pgfsetlinewidth{0.501875pt}%
\definecolor{currentstroke}{rgb}{0.277134,0.185228,0.489898}%
\pgfsetstrokecolor{currentstroke}%
\pgfsetdash{}{0pt}%
\pgfpathmoveto{\pgfqpoint{2.389394in}{3.823030in}}%
\pgfpathlineto{\pgfqpoint{2.336608in}{3.825834in}}%
\pgfusepath{stroke}%
\end{pgfscope}%
\begin{pgfscope}%
\pgfpathrectangle{\pgfqpoint{0.800000in}{3.252941in}}{\pgfqpoint{2.407767in}{1.544118in}}%
\pgfusepath{clip}%
\pgfsetbuttcap%
\pgfsetroundjoin%
\pgfsetlinewidth{0.501875pt}%
\definecolor{currentstroke}{rgb}{0.269308,0.218818,0.509577}%
\pgfsetstrokecolor{currentstroke}%
\pgfsetdash{}{0pt}%
\pgfpathmoveto{\pgfqpoint{2.336608in}{3.825834in}}%
\pgfpathlineto{\pgfqpoint{2.283943in}{3.829483in}}%
\pgfusepath{stroke}%
\end{pgfscope}%
\begin{pgfscope}%
\pgfpathrectangle{\pgfqpoint{0.800000in}{3.252941in}}{\pgfqpoint{2.407767in}{1.544118in}}%
\pgfusepath{clip}%
\pgfsetbuttcap%
\pgfsetroundjoin%
\pgfsetlinewidth{0.501875pt}%
\definecolor{currentstroke}{rgb}{0.266580,0.228262,0.514349}%
\pgfsetstrokecolor{currentstroke}%
\pgfsetdash{}{0pt}%
\pgfpathmoveto{\pgfqpoint{2.283943in}{3.829483in}}%
\pgfpathlineto{\pgfqpoint{2.231464in}{3.834035in}}%
\pgfusepath{stroke}%
\end{pgfscope}%
\begin{pgfscope}%
\pgfpathrectangle{\pgfqpoint{0.800000in}{3.252941in}}{\pgfqpoint{2.407767in}{1.544118in}}%
\pgfusepath{clip}%
\pgfsetbuttcap%
\pgfsetroundjoin%
\pgfsetlinewidth{0.501875pt}%
\definecolor{currentstroke}{rgb}{0.275191,0.194905,0.496005}%
\pgfsetstrokecolor{currentstroke}%
\pgfsetdash{}{0pt}%
\pgfpathmoveto{\pgfqpoint{2.231464in}{3.834035in}}%
\pgfpathlineto{\pgfqpoint{2.179278in}{3.839837in}}%
\pgfusepath{stroke}%
\end{pgfscope}%
\begin{pgfscope}%
\pgfpathrectangle{\pgfqpoint{0.800000in}{3.252941in}}{\pgfqpoint{2.407767in}{1.544118in}}%
\pgfusepath{clip}%
\pgfsetbuttcap%
\pgfsetroundjoin%
\pgfsetlinewidth{0.501875pt}%
\definecolor{currentstroke}{rgb}{0.255645,0.260703,0.528312}%
\pgfsetstrokecolor{currentstroke}%
\pgfsetdash{}{0pt}%
\pgfpathmoveto{\pgfqpoint{2.179278in}{3.839837in}}%
\pgfpathlineto{\pgfqpoint{2.127670in}{3.847392in}}%
\pgfusepath{stroke}%
\end{pgfscope}%
\begin{pgfscope}%
\pgfpathrectangle{\pgfqpoint{0.800000in}{3.252941in}}{\pgfqpoint{2.407767in}{1.544118in}}%
\pgfusepath{clip}%
\pgfsetbuttcap%
\pgfsetroundjoin%
\pgfsetlinewidth{0.501875pt}%
\definecolor{currentstroke}{rgb}{0.276022,0.044167,0.370164}%
\pgfsetstrokecolor{currentstroke}%
\pgfsetdash{}{0pt}%
\pgfpathmoveto{\pgfqpoint{2.654046in}{3.851269in}}%
\pgfpathlineto{\pgfqpoint{2.601074in}{3.851505in}}%
\pgfusepath{stroke}%
\end{pgfscope}%
\begin{pgfscope}%
\pgfpathrectangle{\pgfqpoint{0.800000in}{3.252941in}}{\pgfqpoint{2.407767in}{1.544118in}}%
\pgfusepath{clip}%
\pgfsetbuttcap%
\pgfsetroundjoin%
\pgfsetlinewidth{0.501875pt}%
\definecolor{currentstroke}{rgb}{0.282327,0.094955,0.417331}%
\pgfsetstrokecolor{currentstroke}%
\pgfsetdash{}{0pt}%
\pgfpathmoveto{\pgfqpoint{2.601074in}{3.851505in}}%
\pgfpathlineto{\pgfqpoint{2.548109in}{3.852167in}}%
\pgfusepath{stroke}%
\end{pgfscope}%
\begin{pgfscope}%
\pgfpathrectangle{\pgfqpoint{0.800000in}{3.252941in}}{\pgfqpoint{2.407767in}{1.544118in}}%
\pgfusepath{clip}%
\pgfsetbuttcap%
\pgfsetroundjoin%
\pgfsetlinewidth{0.501875pt}%
\definecolor{currentstroke}{rgb}{0.282290,0.145912,0.461510}%
\pgfsetstrokecolor{currentstroke}%
\pgfsetdash{}{0pt}%
\pgfpathmoveto{\pgfqpoint{2.548109in}{3.852167in}}%
\pgfpathlineto{\pgfqpoint{2.495172in}{3.853390in}}%
\pgfusepath{stroke}%
\end{pgfscope}%
\begin{pgfscope}%
\pgfpathrectangle{\pgfqpoint{0.800000in}{3.252941in}}{\pgfqpoint{2.407767in}{1.544118in}}%
\pgfusepath{clip}%
\pgfsetbuttcap%
\pgfsetroundjoin%
\pgfsetlinewidth{0.501875pt}%
\definecolor{currentstroke}{rgb}{0.281412,0.155834,0.469201}%
\pgfsetstrokecolor{currentstroke}%
\pgfsetdash{}{0pt}%
\pgfpathmoveto{\pgfqpoint{2.495172in}{3.853390in}}%
\pgfpathlineto{\pgfqpoint{2.442259in}{3.855045in}}%
\pgfusepath{stroke}%
\end{pgfscope}%
\begin{pgfscope}%
\pgfpathrectangle{\pgfqpoint{0.800000in}{3.252941in}}{\pgfqpoint{2.407767in}{1.544118in}}%
\pgfusepath{clip}%
\pgfsetbuttcap%
\pgfsetroundjoin%
\pgfsetlinewidth{0.501875pt}%
\definecolor{currentstroke}{rgb}{0.273006,0.204520,0.501721}%
\pgfsetstrokecolor{currentstroke}%
\pgfsetdash{}{0pt}%
\pgfpathmoveto{\pgfqpoint{2.442259in}{3.855045in}}%
\pgfpathlineto{\pgfqpoint{2.389388in}{3.857128in}}%
\pgfusepath{stroke}%
\end{pgfscope}%
\begin{pgfscope}%
\pgfpathrectangle{\pgfqpoint{0.800000in}{3.252941in}}{\pgfqpoint{2.407767in}{1.544118in}}%
\pgfusepath{clip}%
\pgfsetbuttcap%
\pgfsetroundjoin%
\pgfsetlinewidth{0.501875pt}%
\definecolor{currentstroke}{rgb}{0.258965,0.251537,0.524736}%
\pgfsetstrokecolor{currentstroke}%
\pgfsetdash{}{0pt}%
\pgfpathmoveto{\pgfqpoint{2.389388in}{3.857128in}}%
\pgfpathlineto{\pgfqpoint{2.336579in}{3.859816in}}%
\pgfusepath{stroke}%
\end{pgfscope}%
\begin{pgfscope}%
\pgfpathrectangle{\pgfqpoint{0.800000in}{3.252941in}}{\pgfqpoint{2.407767in}{1.544118in}}%
\pgfusepath{clip}%
\pgfsetbuttcap%
\pgfsetroundjoin%
\pgfsetlinewidth{0.501875pt}%
\definecolor{currentstroke}{rgb}{0.239346,0.300855,0.540844}%
\pgfsetstrokecolor{currentstroke}%
\pgfsetdash{}{0pt}%
\pgfpathmoveto{\pgfqpoint{2.336579in}{3.859816in}}%
\pgfpathlineto{\pgfqpoint{2.283858in}{3.863101in}}%
\pgfusepath{stroke}%
\end{pgfscope}%
\begin{pgfscope}%
\pgfpathrectangle{\pgfqpoint{0.800000in}{3.252941in}}{\pgfqpoint{2.407767in}{1.544118in}}%
\pgfusepath{clip}%
\pgfsetbuttcap%
\pgfsetroundjoin%
\pgfsetlinewidth{0.501875pt}%
\definecolor{currentstroke}{rgb}{0.277018,0.050344,0.375715}%
\pgfsetstrokecolor{currentstroke}%
\pgfsetdash{}{0pt}%
\pgfpathmoveto{\pgfqpoint{2.654046in}{3.886016in}}%
\pgfpathlineto{\pgfqpoint{2.601076in}{3.886477in}}%
\pgfusepath{stroke}%
\end{pgfscope}%
\begin{pgfscope}%
\pgfpathrectangle{\pgfqpoint{0.800000in}{3.252941in}}{\pgfqpoint{2.407767in}{1.544118in}}%
\pgfusepath{clip}%
\pgfsetbuttcap%
\pgfsetroundjoin%
\pgfsetlinewidth{0.501875pt}%
\definecolor{currentstroke}{rgb}{0.282327,0.094955,0.417331}%
\pgfsetstrokecolor{currentstroke}%
\pgfsetdash{}{0pt}%
\pgfpathmoveto{\pgfqpoint{2.601076in}{3.886477in}}%
\pgfpathlineto{\pgfqpoint{2.548108in}{3.887041in}}%
\pgfusepath{stroke}%
\end{pgfscope}%
\begin{pgfscope}%
\pgfpathrectangle{\pgfqpoint{0.800000in}{3.252941in}}{\pgfqpoint{2.407767in}{1.544118in}}%
\pgfusepath{clip}%
\pgfsetbuttcap%
\pgfsetroundjoin%
\pgfsetlinewidth{0.501875pt}%
\definecolor{currentstroke}{rgb}{0.282290,0.145912,0.461510}%
\pgfsetstrokecolor{currentstroke}%
\pgfsetdash{}{0pt}%
\pgfpathmoveto{\pgfqpoint{2.548108in}{3.887041in}}%
\pgfpathlineto{\pgfqpoint{2.495148in}{3.887849in}}%
\pgfusepath{stroke}%
\end{pgfscope}%
\begin{pgfscope}%
\pgfpathrectangle{\pgfqpoint{0.800000in}{3.252941in}}{\pgfqpoint{2.407767in}{1.544118in}}%
\pgfusepath{clip}%
\pgfsetbuttcap%
\pgfsetroundjoin%
\pgfsetlinewidth{0.501875pt}%
\definecolor{currentstroke}{rgb}{0.275191,0.194905,0.496005}%
\pgfsetstrokecolor{currentstroke}%
\pgfsetdash{}{0pt}%
\pgfpathmoveto{\pgfqpoint{2.495148in}{3.887849in}}%
\pgfpathlineto{\pgfqpoint{2.442203in}{3.888993in}}%
\pgfusepath{stroke}%
\end{pgfscope}%
\begin{pgfscope}%
\pgfpathrectangle{\pgfqpoint{0.800000in}{3.252941in}}{\pgfqpoint{2.407767in}{1.544118in}}%
\pgfusepath{clip}%
\pgfsetbuttcap%
\pgfsetroundjoin%
\pgfsetlinewidth{0.501875pt}%
\definecolor{currentstroke}{rgb}{0.262138,0.242286,0.520837}%
\pgfsetstrokecolor{currentstroke}%
\pgfsetdash{}{0pt}%
\pgfpathmoveto{\pgfqpoint{2.442203in}{3.888993in}}%
\pgfpathlineto{\pgfqpoint{2.389286in}{3.890579in}}%
\pgfusepath{stroke}%
\end{pgfscope}%
\begin{pgfscope}%
\pgfpathrectangle{\pgfqpoint{0.800000in}{3.252941in}}{\pgfqpoint{2.407767in}{1.544118in}}%
\pgfusepath{clip}%
\pgfsetbuttcap%
\pgfsetroundjoin%
\pgfsetlinewidth{0.501875pt}%
\definecolor{currentstroke}{rgb}{0.252194,0.269783,0.531579}%
\pgfsetstrokecolor{currentstroke}%
\pgfsetdash{}{0pt}%
\pgfpathmoveto{\pgfqpoint{2.389286in}{3.890579in}}%
\pgfpathlineto{\pgfqpoint{2.336417in}{3.892716in}}%
\pgfusepath{stroke}%
\end{pgfscope}%
\begin{pgfscope}%
\pgfpathrectangle{\pgfqpoint{0.800000in}{3.252941in}}{\pgfqpoint{2.407767in}{1.544118in}}%
\pgfusepath{clip}%
\pgfsetbuttcap%
\pgfsetroundjoin%
\pgfsetlinewidth{0.501875pt}%
\definecolor{currentstroke}{rgb}{0.235526,0.309527,0.542944}%
\pgfsetstrokecolor{currentstroke}%
\pgfsetdash{}{0pt}%
\pgfpathmoveto{\pgfqpoint{2.336417in}{3.892716in}}%
\pgfpathlineto{\pgfqpoint{2.283621in}{3.895485in}}%
\pgfusepath{stroke}%
\end{pgfscope}%
\begin{pgfscope}%
\pgfpathrectangle{\pgfqpoint{0.800000in}{3.252941in}}{\pgfqpoint{2.407767in}{1.544118in}}%
\pgfusepath{clip}%
\pgfsetbuttcap%
\pgfsetroundjoin%
\pgfsetlinewidth{0.501875pt}%
\definecolor{currentstroke}{rgb}{0.204903,0.375746,0.553533}%
\pgfsetstrokecolor{currentstroke}%
\pgfsetdash{}{0pt}%
\pgfpathmoveto{\pgfqpoint{2.283621in}{3.895485in}}%
\pgfpathlineto{\pgfqpoint{2.230947in}{3.899073in}}%
\pgfusepath{stroke}%
\end{pgfscope}%
\begin{pgfscope}%
\pgfpathrectangle{\pgfqpoint{0.800000in}{3.252941in}}{\pgfqpoint{2.407767in}{1.544118in}}%
\pgfusepath{clip}%
\pgfsetbuttcap%
\pgfsetroundjoin%
\pgfsetlinewidth{0.501875pt}%
\definecolor{currentstroke}{rgb}{0.199430,0.387607,0.554642}%
\pgfsetstrokecolor{currentstroke}%
\pgfsetdash{}{0pt}%
\pgfpathmoveto{\pgfqpoint{2.230947in}{3.899073in}}%
\pgfpathlineto{\pgfqpoint{2.178463in}{3.903661in}}%
\pgfusepath{stroke}%
\end{pgfscope}%
\begin{pgfscope}%
\pgfpathrectangle{\pgfqpoint{0.800000in}{3.252941in}}{\pgfqpoint{2.407767in}{1.544118in}}%
\pgfusepath{clip}%
\pgfsetbuttcap%
\pgfsetroundjoin%
\pgfsetlinewidth{0.501875pt}%
\definecolor{currentstroke}{rgb}{0.194100,0.399323,0.555565}%
\pgfsetstrokecolor{currentstroke}%
\pgfsetdash{}{0pt}%
\pgfpathmoveto{\pgfqpoint{2.178463in}{3.903661in}}%
\pgfpathlineto{\pgfqpoint{2.126280in}{3.909467in}}%
\pgfusepath{stroke}%
\end{pgfscope}%
\begin{pgfscope}%
\pgfpathrectangle{\pgfqpoint{0.800000in}{3.252941in}}{\pgfqpoint{2.407767in}{1.544118in}}%
\pgfusepath{clip}%
\pgfsetbuttcap%
\pgfsetroundjoin%
\pgfsetlinewidth{0.501875pt}%
\definecolor{currentstroke}{rgb}{0.129933,0.559582,0.551864}%
\pgfsetstrokecolor{currentstroke}%
\pgfsetdash{}{0pt}%
\pgfpathmoveto{\pgfqpoint{2.126280in}{3.909467in}}%
\pgfpathlineto{\pgfqpoint{2.074456in}{3.916482in}}%
\pgfusepath{stroke}%
\end{pgfscope}%
\begin{pgfscope}%
\pgfpathrectangle{\pgfqpoint{0.800000in}{3.252941in}}{\pgfqpoint{2.407767in}{1.544118in}}%
\pgfusepath{clip}%
\pgfsetbuttcap%
\pgfsetroundjoin%
\pgfsetlinewidth{0.501875pt}%
\definecolor{currentstroke}{rgb}{0.185783,0.704891,0.485273}%
\pgfsetstrokecolor{currentstroke}%
\pgfsetdash{}{0pt}%
\pgfpathmoveto{\pgfqpoint{2.074456in}{3.916482in}}%
\pgfpathlineto{\pgfqpoint{2.022859in}{3.924165in}}%
\pgfusepath{stroke}%
\end{pgfscope}%
\begin{pgfscope}%
\pgfpathrectangle{\pgfqpoint{0.800000in}{3.252941in}}{\pgfqpoint{2.407767in}{1.544118in}}%
\pgfusepath{clip}%
\pgfsetbuttcap%
\pgfsetroundjoin%
\pgfsetlinewidth{0.501875pt}%
\definecolor{currentstroke}{rgb}{0.395174,0.797475,0.367757}%
\pgfsetstrokecolor{currentstroke}%
\pgfsetdash{}{0pt}%
\pgfpathmoveto{\pgfqpoint{2.022859in}{3.924165in}}%
\pgfpathlineto{\pgfqpoint{1.971423in}{3.932270in}}%
\pgfusepath{stroke}%
\end{pgfscope}%
\begin{pgfscope}%
\pgfpathrectangle{\pgfqpoint{0.800000in}{3.252941in}}{\pgfqpoint{2.407767in}{1.544118in}}%
\pgfusepath{clip}%
\pgfsetbuttcap%
\pgfsetroundjoin%
\pgfsetlinewidth{0.501875pt}%
\definecolor{currentstroke}{rgb}{0.377779,0.791781,0.377939}%
\pgfsetstrokecolor{currentstroke}%
\pgfsetdash{}{0pt}%
\pgfpathmoveto{\pgfqpoint{1.971423in}{3.932270in}}%
\pgfpathlineto{\pgfqpoint{1.920108in}{3.940683in}}%
\pgfusepath{stroke}%
\end{pgfscope}%
\begin{pgfscope}%
\pgfpathrectangle{\pgfqpoint{0.800000in}{3.252941in}}{\pgfqpoint{2.407767in}{1.544118in}}%
\pgfusepath{clip}%
\pgfsetbuttcap%
\pgfsetroundjoin%
\pgfsetlinewidth{0.501875pt}%
\definecolor{currentstroke}{rgb}{0.606045,0.850733,0.236712}%
\pgfsetstrokecolor{currentstroke}%
\pgfsetdash{}{0pt}%
\pgfpathmoveto{\pgfqpoint{1.920108in}{3.940683in}}%
\pgfpathlineto{\pgfqpoint{1.868799in}{3.949114in}}%
\pgfusepath{stroke}%
\end{pgfscope}%
\begin{pgfscope}%
\pgfpathrectangle{\pgfqpoint{0.800000in}{3.252941in}}{\pgfqpoint{2.407767in}{1.544118in}}%
\pgfusepath{clip}%
\pgfsetbuttcap%
\pgfsetroundjoin%
\pgfsetlinewidth{0.501875pt}%
\definecolor{currentstroke}{rgb}{0.824940,0.884720,0.106217}%
\pgfsetstrokecolor{currentstroke}%
\pgfsetdash{}{0pt}%
\pgfpathmoveto{\pgfqpoint{1.868799in}{3.949114in}}%
\pgfpathlineto{\pgfqpoint{1.817431in}{3.957386in}}%
\pgfusepath{stroke}%
\end{pgfscope}%
\begin{pgfscope}%
\pgfpathrectangle{\pgfqpoint{0.800000in}{3.252941in}}{\pgfqpoint{2.407767in}{1.544118in}}%
\pgfusepath{clip}%
\pgfsetbuttcap%
\pgfsetroundjoin%
\pgfsetlinewidth{0.501875pt}%
\definecolor{currentstroke}{rgb}{0.845561,0.887322,0.099702}%
\pgfsetstrokecolor{currentstroke}%
\pgfsetdash{}{0pt}%
\pgfpathmoveto{\pgfqpoint{1.817431in}{3.957386in}}%
\pgfpathlineto{\pgfqpoint{1.766166in}{3.965900in}}%
\pgfusepath{stroke}%
\end{pgfscope}%
\begin{pgfscope}%
\pgfpathrectangle{\pgfqpoint{0.800000in}{3.252941in}}{\pgfqpoint{2.407767in}{1.544118in}}%
\pgfusepath{clip}%
\pgfsetbuttcap%
\pgfsetroundjoin%
\pgfsetlinewidth{0.501875pt}%
\definecolor{currentstroke}{rgb}{0.277018,0.050344,0.375715}%
\pgfsetstrokecolor{currentstroke}%
\pgfsetdash{}{0pt}%
\pgfpathmoveto{\pgfqpoint{2.654046in}{3.920762in}}%
\pgfpathlineto{\pgfqpoint{2.601074in}{3.921138in}}%
\pgfusepath{stroke}%
\end{pgfscope}%
\begin{pgfscope}%
\pgfpathrectangle{\pgfqpoint{0.800000in}{3.252941in}}{\pgfqpoint{2.407767in}{1.544118in}}%
\pgfusepath{clip}%
\pgfsetbuttcap%
\pgfsetroundjoin%
\pgfsetlinewidth{0.501875pt}%
\definecolor{currentstroke}{rgb}{0.281924,0.089666,0.412415}%
\pgfsetstrokecolor{currentstroke}%
\pgfsetdash{}{0pt}%
\pgfpathmoveto{\pgfqpoint{2.601074in}{3.921138in}}%
\pgfpathlineto{\pgfqpoint{2.548102in}{3.921568in}}%
\pgfusepath{stroke}%
\end{pgfscope}%
\begin{pgfscope}%
\pgfpathrectangle{\pgfqpoint{0.800000in}{3.252941in}}{\pgfqpoint{2.407767in}{1.544118in}}%
\pgfusepath{clip}%
\pgfsetbuttcap%
\pgfsetroundjoin%
\pgfsetlinewidth{0.501875pt}%
\definecolor{currentstroke}{rgb}{0.280868,0.160771,0.472899}%
\pgfsetstrokecolor{currentstroke}%
\pgfsetdash{}{0pt}%
\pgfpathmoveto{\pgfqpoint{2.548102in}{3.921568in}}%
\pgfpathlineto{\pgfqpoint{2.495136in}{3.922193in}}%
\pgfusepath{stroke}%
\end{pgfscope}%
\begin{pgfscope}%
\pgfpathrectangle{\pgfqpoint{0.800000in}{3.252941in}}{\pgfqpoint{2.407767in}{1.544118in}}%
\pgfusepath{clip}%
\pgfsetbuttcap%
\pgfsetroundjoin%
\pgfsetlinewidth{0.501875pt}%
\definecolor{currentstroke}{rgb}{0.270595,0.214069,0.507052}%
\pgfsetstrokecolor{currentstroke}%
\pgfsetdash{}{0pt}%
\pgfpathmoveto{\pgfqpoint{2.495136in}{3.922193in}}%
\pgfpathlineto{\pgfqpoint{2.442179in}{3.923085in}}%
\pgfusepath{stroke}%
\end{pgfscope}%
\begin{pgfscope}%
\pgfpathrectangle{\pgfqpoint{0.800000in}{3.252941in}}{\pgfqpoint{2.407767in}{1.544118in}}%
\pgfusepath{clip}%
\pgfsetbuttcap%
\pgfsetroundjoin%
\pgfsetlinewidth{0.501875pt}%
\definecolor{currentstroke}{rgb}{0.248629,0.278775,0.534556}%
\pgfsetstrokecolor{currentstroke}%
\pgfsetdash{}{0pt}%
\pgfpathmoveto{\pgfqpoint{2.442179in}{3.923085in}}%
\pgfpathlineto{\pgfqpoint{2.389237in}{3.924291in}}%
\pgfusepath{stroke}%
\end{pgfscope}%
\begin{pgfscope}%
\pgfpathrectangle{\pgfqpoint{0.800000in}{3.252941in}}{\pgfqpoint{2.407767in}{1.544118in}}%
\pgfusepath{clip}%
\pgfsetbuttcap%
\pgfsetroundjoin%
\pgfsetlinewidth{0.501875pt}%
\definecolor{currentstroke}{rgb}{0.221989,0.339161,0.548752}%
\pgfsetstrokecolor{currentstroke}%
\pgfsetdash{}{0pt}%
\pgfpathmoveto{\pgfqpoint{2.389237in}{3.924291in}}%
\pgfpathlineto{\pgfqpoint{2.336323in}{3.925912in}}%
\pgfusepath{stroke}%
\end{pgfscope}%
\begin{pgfscope}%
\pgfpathrectangle{\pgfqpoint{0.800000in}{3.252941in}}{\pgfqpoint{2.407767in}{1.544118in}}%
\pgfusepath{clip}%
\pgfsetbuttcap%
\pgfsetroundjoin%
\pgfsetlinewidth{0.501875pt}%
\definecolor{currentstroke}{rgb}{0.188923,0.410910,0.556326}%
\pgfsetstrokecolor{currentstroke}%
\pgfsetdash{}{0pt}%
\pgfpathmoveto{\pgfqpoint{2.336323in}{3.925912in}}%
\pgfpathlineto{\pgfqpoint{2.283461in}{3.928102in}}%
\pgfusepath{stroke}%
\end{pgfscope}%
\begin{pgfscope}%
\pgfpathrectangle{\pgfqpoint{0.800000in}{3.252941in}}{\pgfqpoint{2.407767in}{1.544118in}}%
\pgfusepath{clip}%
\pgfsetbuttcap%
\pgfsetroundjoin%
\pgfsetlinewidth{0.501875pt}%
\definecolor{currentstroke}{rgb}{0.165117,0.467423,0.558141}%
\pgfsetstrokecolor{currentstroke}%
\pgfsetdash{}{0pt}%
\pgfpathmoveto{\pgfqpoint{2.283461in}{3.928102in}}%
\pgfpathlineto{\pgfqpoint{2.230683in}{3.930995in}}%
\pgfusepath{stroke}%
\end{pgfscope}%
\begin{pgfscope}%
\pgfpathrectangle{\pgfqpoint{0.800000in}{3.252941in}}{\pgfqpoint{2.407767in}{1.544118in}}%
\pgfusepath{clip}%
\pgfsetbuttcap%
\pgfsetroundjoin%
\pgfsetlinewidth{0.501875pt}%
\definecolor{currentstroke}{rgb}{0.277018,0.050344,0.375715}%
\pgfsetstrokecolor{currentstroke}%
\pgfsetdash{}{0pt}%
\pgfpathmoveto{\pgfqpoint{2.654046in}{3.955508in}}%
\pgfpathlineto{\pgfqpoint{2.601072in}{3.955663in}}%
\pgfusepath{stroke}%
\end{pgfscope}%
\begin{pgfscope}%
\pgfpathrectangle{\pgfqpoint{0.800000in}{3.252941in}}{\pgfqpoint{2.407767in}{1.544118in}}%
\pgfusepath{clip}%
\pgfsetbuttcap%
\pgfsetroundjoin%
\pgfsetlinewidth{0.501875pt}%
\definecolor{currentstroke}{rgb}{0.283197,0.115680,0.436115}%
\pgfsetstrokecolor{currentstroke}%
\pgfsetdash{}{0pt}%
\pgfpathmoveto{\pgfqpoint{2.601072in}{3.955663in}}%
\pgfpathlineto{\pgfqpoint{2.548098in}{3.956006in}}%
\pgfusepath{stroke}%
\end{pgfscope}%
\begin{pgfscope}%
\pgfpathrectangle{\pgfqpoint{0.800000in}{3.252941in}}{\pgfqpoint{2.407767in}{1.544118in}}%
\pgfusepath{clip}%
\pgfsetbuttcap%
\pgfsetroundjoin%
\pgfsetlinewidth{0.501875pt}%
\definecolor{currentstroke}{rgb}{0.278826,0.175490,0.483397}%
\pgfsetstrokecolor{currentstroke}%
\pgfsetdash{}{0pt}%
\pgfpathmoveto{\pgfqpoint{2.548098in}{3.956006in}}%
\pgfpathlineto{\pgfqpoint{2.495127in}{3.956415in}}%
\pgfusepath{stroke}%
\end{pgfscope}%
\begin{pgfscope}%
\pgfpathrectangle{\pgfqpoint{0.800000in}{3.252941in}}{\pgfqpoint{2.407767in}{1.544118in}}%
\pgfusepath{clip}%
\pgfsetbuttcap%
\pgfsetroundjoin%
\pgfsetlinewidth{0.501875pt}%
\definecolor{currentstroke}{rgb}{0.262138,0.242286,0.520837}%
\pgfsetstrokecolor{currentstroke}%
\pgfsetdash{}{0pt}%
\pgfpathmoveto{\pgfqpoint{2.495127in}{3.956415in}}%
\pgfpathlineto{\pgfqpoint{2.442158in}{3.956968in}}%
\pgfusepath{stroke}%
\end{pgfscope}%
\begin{pgfscope}%
\pgfpathrectangle{\pgfqpoint{0.800000in}{3.252941in}}{\pgfqpoint{2.407767in}{1.544118in}}%
\pgfusepath{clip}%
\pgfsetbuttcap%
\pgfsetroundjoin%
\pgfsetlinewidth{0.501875pt}%
\definecolor{currentstroke}{rgb}{0.239346,0.300855,0.540844}%
\pgfsetstrokecolor{currentstroke}%
\pgfsetdash{}{0pt}%
\pgfpathmoveto{\pgfqpoint{2.442158in}{3.956968in}}%
\pgfpathlineto{\pgfqpoint{2.389192in}{3.957645in}}%
\pgfusepath{stroke}%
\end{pgfscope}%
\begin{pgfscope}%
\pgfpathrectangle{\pgfqpoint{0.800000in}{3.252941in}}{\pgfqpoint{2.407767in}{1.544118in}}%
\pgfusepath{clip}%
\pgfsetbuttcap%
\pgfsetroundjoin%
\pgfsetlinewidth{0.501875pt}%
\definecolor{currentstroke}{rgb}{0.204903,0.375746,0.553533}%
\pgfsetstrokecolor{currentstroke}%
\pgfsetdash{}{0pt}%
\pgfpathmoveto{\pgfqpoint{2.389192in}{3.957645in}}%
\pgfpathlineto{\pgfqpoint{2.336240in}{3.958613in}}%
\pgfusepath{stroke}%
\end{pgfscope}%
\begin{pgfscope}%
\pgfpathrectangle{\pgfqpoint{0.800000in}{3.252941in}}{\pgfqpoint{2.407767in}{1.544118in}}%
\pgfusepath{clip}%
\pgfsetbuttcap%
\pgfsetroundjoin%
\pgfsetlinewidth{0.501875pt}%
\definecolor{currentstroke}{rgb}{0.159194,0.482237,0.558073}%
\pgfsetstrokecolor{currentstroke}%
\pgfsetdash{}{0pt}%
\pgfpathmoveto{\pgfqpoint{2.336240in}{3.958613in}}%
\pgfpathlineto{\pgfqpoint{2.283309in}{3.959992in}}%
\pgfusepath{stroke}%
\end{pgfscope}%
\begin{pgfscope}%
\pgfpathrectangle{\pgfqpoint{0.800000in}{3.252941in}}{\pgfqpoint{2.407767in}{1.544118in}}%
\pgfusepath{clip}%
\pgfsetbuttcap%
\pgfsetroundjoin%
\pgfsetlinewidth{0.501875pt}%
\definecolor{currentstroke}{rgb}{0.127568,0.566949,0.550556}%
\pgfsetstrokecolor{currentstroke}%
\pgfsetdash{}{0pt}%
\pgfpathmoveto{\pgfqpoint{2.283309in}{3.959992in}}%
\pgfpathlineto{\pgfqpoint{2.230413in}{3.961822in}}%
\pgfusepath{stroke}%
\end{pgfscope}%
\begin{pgfscope}%
\pgfpathrectangle{\pgfqpoint{0.800000in}{3.252941in}}{\pgfqpoint{2.407767in}{1.544118in}}%
\pgfusepath{clip}%
\pgfsetbuttcap%
\pgfsetroundjoin%
\pgfsetlinewidth{0.501875pt}%
\definecolor{currentstroke}{rgb}{0.143303,0.669459,0.511215}%
\pgfsetstrokecolor{currentstroke}%
\pgfsetdash{}{0pt}%
\pgfpathmoveto{\pgfqpoint{2.230413in}{3.961822in}}%
\pgfpathlineto{\pgfqpoint{2.177567in}{3.964181in}}%
\pgfusepath{stroke}%
\end{pgfscope}%
\begin{pgfscope}%
\pgfpathrectangle{\pgfqpoint{0.800000in}{3.252941in}}{\pgfqpoint{2.407767in}{1.544118in}}%
\pgfusepath{clip}%
\pgfsetbuttcap%
\pgfsetroundjoin%
\pgfsetlinewidth{0.501875pt}%
\definecolor{currentstroke}{rgb}{0.281477,0.755203,0.432552}%
\pgfsetstrokecolor{currentstroke}%
\pgfsetdash{}{0pt}%
\pgfpathmoveto{\pgfqpoint{2.177567in}{3.964181in}}%
\pgfpathlineto{\pgfqpoint{2.124812in}{3.967240in}}%
\pgfusepath{stroke}%
\end{pgfscope}%
\begin{pgfscope}%
\pgfpathrectangle{\pgfqpoint{0.800000in}{3.252941in}}{\pgfqpoint{2.407767in}{1.544118in}}%
\pgfusepath{clip}%
\pgfsetbuttcap%
\pgfsetroundjoin%
\pgfsetlinewidth{0.501875pt}%
\definecolor{currentstroke}{rgb}{0.626579,0.854645,0.223353}%
\pgfsetstrokecolor{currentstroke}%
\pgfsetdash{}{0pt}%
\pgfpathmoveto{\pgfqpoint{2.124812in}{3.967240in}}%
\pgfpathlineto{\pgfqpoint{2.072147in}{3.970891in}}%
\pgfusepath{stroke}%
\end{pgfscope}%
\begin{pgfscope}%
\pgfpathrectangle{\pgfqpoint{0.800000in}{3.252941in}}{\pgfqpoint{2.407767in}{1.544118in}}%
\pgfusepath{clip}%
\pgfsetbuttcap%
\pgfsetroundjoin%
\pgfsetlinewidth{0.501875pt}%
\definecolor{currentstroke}{rgb}{0.668054,0.861999,0.196293}%
\pgfsetstrokecolor{currentstroke}%
\pgfsetdash{}{0pt}%
\pgfpathmoveto{\pgfqpoint{2.072147in}{3.970891in}}%
\pgfpathlineto{\pgfqpoint{2.019539in}{3.974854in}}%
\pgfusepath{stroke}%
\end{pgfscope}%
\begin{pgfscope}%
\pgfpathrectangle{\pgfqpoint{0.800000in}{3.252941in}}{\pgfqpoint{2.407767in}{1.544118in}}%
\pgfusepath{clip}%
\pgfsetbuttcap%
\pgfsetroundjoin%
\pgfsetlinewidth{0.501875pt}%
\definecolor{currentstroke}{rgb}{0.793760,0.880678,0.120005}%
\pgfsetstrokecolor{currentstroke}%
\pgfsetdash{}{0pt}%
\pgfpathmoveto{\pgfqpoint{2.019539in}{3.974854in}}%
\pgfpathlineto{\pgfqpoint{1.966970in}{3.979038in}}%
\pgfusepath{stroke}%
\end{pgfscope}%
\begin{pgfscope}%
\pgfpathrectangle{\pgfqpoint{0.800000in}{3.252941in}}{\pgfqpoint{2.407767in}{1.544118in}}%
\pgfusepath{clip}%
\pgfsetbuttcap%
\pgfsetroundjoin%
\pgfsetlinewidth{0.501875pt}%
\definecolor{currentstroke}{rgb}{0.845561,0.887322,0.099702}%
\pgfsetstrokecolor{currentstroke}%
\pgfsetdash{}{0pt}%
\pgfpathmoveto{\pgfqpoint{1.966970in}{3.979038in}}%
\pgfpathlineto{\pgfqpoint{1.914418in}{3.983292in}}%
\pgfusepath{stroke}%
\end{pgfscope}%
\begin{pgfscope}%
\pgfpathrectangle{\pgfqpoint{0.800000in}{3.252941in}}{\pgfqpoint{2.407767in}{1.544118in}}%
\pgfusepath{clip}%
\pgfsetbuttcap%
\pgfsetroundjoin%
\pgfsetlinewidth{0.501875pt}%
\definecolor{currentstroke}{rgb}{0.974417,0.903590,0.130215}%
\pgfsetstrokecolor{currentstroke}%
\pgfsetdash{}{0pt}%
\pgfpathmoveto{\pgfqpoint{1.914418in}{3.983292in}}%
\pgfpathlineto{\pgfqpoint{1.861868in}{3.987520in}}%
\pgfusepath{stroke}%
\end{pgfscope}%
\begin{pgfscope}%
\pgfpathrectangle{\pgfqpoint{0.800000in}{3.252941in}}{\pgfqpoint{2.407767in}{1.544118in}}%
\pgfusepath{clip}%
\pgfsetbuttcap%
\pgfsetroundjoin%
\pgfsetlinewidth{0.501875pt}%
\definecolor{currentstroke}{rgb}{0.993248,0.906157,0.143936}%
\pgfsetstrokecolor{currentstroke}%
\pgfsetdash{}{0pt}%
\pgfpathmoveto{\pgfqpoint{1.861868in}{3.987520in}}%
\pgfpathlineto{\pgfqpoint{1.809316in}{3.991741in}}%
\pgfusepath{stroke}%
\end{pgfscope}%
\begin{pgfscope}%
\pgfpathrectangle{\pgfqpoint{0.800000in}{3.252941in}}{\pgfqpoint{2.407767in}{1.544118in}}%
\pgfusepath{clip}%
\pgfsetbuttcap%
\pgfsetroundjoin%
\pgfsetlinewidth{0.501875pt}%
\definecolor{currentstroke}{rgb}{0.277941,0.056324,0.381191}%
\pgfsetstrokecolor{currentstroke}%
\pgfsetdash{}{0pt}%
\pgfpathmoveto{\pgfqpoint{2.654046in}{3.990254in}}%
\pgfpathlineto{\pgfqpoint{2.601071in}{3.990262in}}%
\pgfusepath{stroke}%
\end{pgfscope}%
\begin{pgfscope}%
\pgfpathrectangle{\pgfqpoint{0.800000in}{3.252941in}}{\pgfqpoint{2.407767in}{1.544118in}}%
\pgfusepath{clip}%
\pgfsetbuttcap%
\pgfsetroundjoin%
\pgfsetlinewidth{0.501875pt}%
\definecolor{currentstroke}{rgb}{0.283197,0.115680,0.436115}%
\pgfsetstrokecolor{currentstroke}%
\pgfsetdash{}{0pt}%
\pgfpathmoveto{\pgfqpoint{2.601071in}{3.990262in}}%
\pgfpathlineto{\pgfqpoint{2.548096in}{3.990342in}}%
\pgfusepath{stroke}%
\end{pgfscope}%
\begin{pgfscope}%
\pgfpathrectangle{\pgfqpoint{0.800000in}{3.252941in}}{\pgfqpoint{2.407767in}{1.544118in}}%
\pgfusepath{clip}%
\pgfsetbuttcap%
\pgfsetroundjoin%
\pgfsetlinewidth{0.501875pt}%
\definecolor{currentstroke}{rgb}{0.278012,0.180367,0.486697}%
\pgfsetstrokecolor{currentstroke}%
\pgfsetdash{}{0pt}%
\pgfpathmoveto{\pgfqpoint{2.548096in}{3.990342in}}%
\pgfpathlineto{\pgfqpoint{2.495121in}{3.990532in}}%
\pgfusepath{stroke}%
\end{pgfscope}%
\begin{pgfscope}%
\pgfpathrectangle{\pgfqpoint{0.800000in}{3.252941in}}{\pgfqpoint{2.407767in}{1.544118in}}%
\pgfusepath{clip}%
\pgfsetbuttcap%
\pgfsetroundjoin%
\pgfsetlinewidth{0.501875pt}%
\definecolor{currentstroke}{rgb}{0.262138,0.242286,0.520837}%
\pgfsetstrokecolor{currentstroke}%
\pgfsetdash{}{0pt}%
\pgfpathmoveto{\pgfqpoint{2.495121in}{3.990532in}}%
\pgfpathlineto{\pgfqpoint{2.442146in}{3.990774in}}%
\pgfusepath{stroke}%
\end{pgfscope}%
\begin{pgfscope}%
\pgfpathrectangle{\pgfqpoint{0.800000in}{3.252941in}}{\pgfqpoint{2.407767in}{1.544118in}}%
\pgfusepath{clip}%
\pgfsetbuttcap%
\pgfsetroundjoin%
\pgfsetlinewidth{0.501875pt}%
\definecolor{currentstroke}{rgb}{0.229739,0.322361,0.545706}%
\pgfsetstrokecolor{currentstroke}%
\pgfsetdash{}{0pt}%
\pgfpathmoveto{\pgfqpoint{2.442146in}{3.990774in}}%
\pgfpathlineto{\pgfqpoint{2.389175in}{3.991215in}}%
\pgfusepath{stroke}%
\end{pgfscope}%
\begin{pgfscope}%
\pgfpathrectangle{\pgfqpoint{0.800000in}{3.252941in}}{\pgfqpoint{2.407767in}{1.544118in}}%
\pgfusepath{clip}%
\pgfsetbuttcap%
\pgfsetroundjoin%
\pgfsetlinewidth{0.501875pt}%
\definecolor{currentstroke}{rgb}{0.188923,0.410910,0.556326}%
\pgfsetstrokecolor{currentstroke}%
\pgfsetdash{}{0pt}%
\pgfpathmoveto{\pgfqpoint{2.389175in}{3.991215in}}%
\pgfpathlineto{\pgfqpoint{2.336209in}{3.991881in}}%
\pgfusepath{stroke}%
\end{pgfscope}%
\begin{pgfscope}%
\pgfpathrectangle{\pgfqpoint{0.800000in}{3.252941in}}{\pgfqpoint{2.407767in}{1.544118in}}%
\pgfusepath{clip}%
\pgfsetbuttcap%
\pgfsetroundjoin%
\pgfsetlinewidth{0.501875pt}%
\definecolor{currentstroke}{rgb}{0.143343,0.522773,0.556295}%
\pgfsetstrokecolor{currentstroke}%
\pgfsetdash{}{0pt}%
\pgfpathmoveto{\pgfqpoint{2.336209in}{3.991881in}}%
\pgfpathlineto{\pgfqpoint{2.283252in}{3.992788in}}%
\pgfusepath{stroke}%
\end{pgfscope}%
\begin{pgfscope}%
\pgfpathrectangle{\pgfqpoint{0.800000in}{3.252941in}}{\pgfqpoint{2.407767in}{1.544118in}}%
\pgfusepath{clip}%
\pgfsetbuttcap%
\pgfsetroundjoin%
\pgfsetlinewidth{0.501875pt}%
\definecolor{currentstroke}{rgb}{0.132268,0.655014,0.519661}%
\pgfsetstrokecolor{currentstroke}%
\pgfsetdash{}{0pt}%
\pgfpathmoveto{\pgfqpoint{2.283252in}{3.992788in}}%
\pgfpathlineto{\pgfqpoint{2.230307in}{3.993942in}}%
\pgfusepath{stroke}%
\end{pgfscope}%
\begin{pgfscope}%
\pgfpathrectangle{\pgfqpoint{0.800000in}{3.252941in}}{\pgfqpoint{2.407767in}{1.544118in}}%
\pgfusepath{clip}%
\pgfsetbuttcap%
\pgfsetroundjoin%
\pgfsetlinewidth{0.501875pt}%
\definecolor{currentstroke}{rgb}{0.344074,0.780029,0.397381}%
\pgfsetstrokecolor{currentstroke}%
\pgfsetdash{}{0pt}%
\pgfpathmoveto{\pgfqpoint{2.230307in}{3.993942in}}%
\pgfpathlineto{\pgfqpoint{2.177376in}{3.995325in}}%
\pgfusepath{stroke}%
\end{pgfscope}%
\begin{pgfscope}%
\pgfpathrectangle{\pgfqpoint{0.800000in}{3.252941in}}{\pgfqpoint{2.407767in}{1.544118in}}%
\pgfusepath{clip}%
\pgfsetbuttcap%
\pgfsetroundjoin%
\pgfsetlinewidth{0.501875pt}%
\definecolor{currentstroke}{rgb}{0.525776,0.833491,0.288127}%
\pgfsetstrokecolor{currentstroke}%
\pgfsetdash{}{0pt}%
\pgfpathmoveto{\pgfqpoint{2.177376in}{3.995325in}}%
\pgfpathlineto{\pgfqpoint{2.124465in}{3.996981in}}%
\pgfusepath{stroke}%
\end{pgfscope}%
\begin{pgfscope}%
\pgfpathrectangle{\pgfqpoint{0.800000in}{3.252941in}}{\pgfqpoint{2.407767in}{1.544118in}}%
\pgfusepath{clip}%
\pgfsetbuttcap%
\pgfsetroundjoin%
\pgfsetlinewidth{0.501875pt}%
\definecolor{currentstroke}{rgb}{0.678489,0.863742,0.189503}%
\pgfsetstrokecolor{currentstroke}%
\pgfsetdash{}{0pt}%
\pgfpathmoveto{\pgfqpoint{2.124465in}{3.996981in}}%
\pgfpathlineto{\pgfqpoint{2.071576in}{3.998893in}}%
\pgfusepath{stroke}%
\end{pgfscope}%
\begin{pgfscope}%
\pgfpathrectangle{\pgfqpoint{0.800000in}{3.252941in}}{\pgfqpoint{2.407767in}{1.544118in}}%
\pgfusepath{clip}%
\pgfsetbuttcap%
\pgfsetroundjoin%
\pgfsetlinewidth{0.501875pt}%
\definecolor{currentstroke}{rgb}{0.277941,0.056324,0.381191}%
\pgfsetstrokecolor{currentstroke}%
\pgfsetdash{}{0pt}%
\pgfpathmoveto{\pgfqpoint{2.654046in}{4.025000in}}%
\pgfpathlineto{\pgfqpoint{2.601070in}{4.024966in}}%
\pgfusepath{stroke}%
\end{pgfscope}%
\begin{pgfscope}%
\pgfpathrectangle{\pgfqpoint{0.800000in}{3.252941in}}{\pgfqpoint{2.407767in}{1.544118in}}%
\pgfusepath{clip}%
\pgfsetbuttcap%
\pgfsetroundjoin%
\pgfsetlinewidth{0.501875pt}%
\definecolor{currentstroke}{rgb}{0.283187,0.125848,0.444960}%
\pgfsetstrokecolor{currentstroke}%
\pgfsetdash{}{0pt}%
\pgfpathmoveto{\pgfqpoint{2.601070in}{4.024966in}}%
\pgfpathlineto{\pgfqpoint{2.548094in}{4.024872in}}%
\pgfusepath{stroke}%
\end{pgfscope}%
\begin{pgfscope}%
\pgfpathrectangle{\pgfqpoint{0.800000in}{3.252941in}}{\pgfqpoint{2.407767in}{1.544118in}}%
\pgfusepath{clip}%
\pgfsetbuttcap%
\pgfsetroundjoin%
\pgfsetlinewidth{0.501875pt}%
\definecolor{currentstroke}{rgb}{0.277134,0.185228,0.489898}%
\pgfsetstrokecolor{currentstroke}%
\pgfsetdash{}{0pt}%
\pgfpathmoveto{\pgfqpoint{2.548094in}{4.024872in}}%
\pgfpathlineto{\pgfqpoint{2.495118in}{4.024893in}}%
\pgfusepath{stroke}%
\end{pgfscope}%
\begin{pgfscope}%
\pgfpathrectangle{\pgfqpoint{0.800000in}{3.252941in}}{\pgfqpoint{2.407767in}{1.544118in}}%
\pgfusepath{clip}%
\pgfsetbuttcap%
\pgfsetroundjoin%
\pgfsetlinewidth{0.501875pt}%
\definecolor{currentstroke}{rgb}{0.258965,0.251537,0.524736}%
\pgfsetstrokecolor{currentstroke}%
\pgfsetdash{}{0pt}%
\pgfpathmoveto{\pgfqpoint{2.495118in}{4.024893in}}%
\pgfpathlineto{\pgfqpoint{2.442143in}{4.025005in}}%
\pgfusepath{stroke}%
\end{pgfscope}%
\begin{pgfscope}%
\pgfpathrectangle{\pgfqpoint{0.800000in}{3.252941in}}{\pgfqpoint{2.407767in}{1.544118in}}%
\pgfusepath{clip}%
\pgfsetbuttcap%
\pgfsetroundjoin%
\pgfsetlinewidth{0.501875pt}%
\definecolor{currentstroke}{rgb}{0.227802,0.326594,0.546532}%
\pgfsetstrokecolor{currentstroke}%
\pgfsetdash{}{0pt}%
\pgfpathmoveto{\pgfqpoint{2.442143in}{4.025005in}}%
\pgfpathlineto{\pgfqpoint{2.389167in}{4.025153in}}%
\pgfusepath{stroke}%
\end{pgfscope}%
\begin{pgfscope}%
\pgfpathrectangle{\pgfqpoint{0.800000in}{3.252941in}}{\pgfqpoint{2.407767in}{1.544118in}}%
\pgfusepath{clip}%
\pgfsetbuttcap%
\pgfsetroundjoin%
\pgfsetlinewidth{0.501875pt}%
\definecolor{currentstroke}{rgb}{0.187231,0.414746,0.556547}%
\pgfsetstrokecolor{currentstroke}%
\pgfsetdash{}{0pt}%
\pgfpathmoveto{\pgfqpoint{2.389167in}{4.025153in}}%
\pgfpathlineto{\pgfqpoint{2.336191in}{4.025266in}}%
\pgfusepath{stroke}%
\end{pgfscope}%
\begin{pgfscope}%
\pgfpathrectangle{\pgfqpoint{0.800000in}{3.252941in}}{\pgfqpoint{2.407767in}{1.544118in}}%
\pgfusepath{clip}%
\pgfsetbuttcap%
\pgfsetroundjoin%
\pgfsetlinewidth{0.501875pt}%
\definecolor{currentstroke}{rgb}{0.137770,0.537492,0.554906}%
\pgfsetstrokecolor{currentstroke}%
\pgfsetdash{}{0pt}%
\pgfpathmoveto{\pgfqpoint{2.336191in}{4.025266in}}%
\pgfpathlineto{\pgfqpoint{2.283215in}{4.025369in}}%
\pgfusepath{stroke}%
\end{pgfscope}%
\begin{pgfscope}%
\pgfpathrectangle{\pgfqpoint{0.800000in}{3.252941in}}{\pgfqpoint{2.407767in}{1.544118in}}%
\pgfusepath{clip}%
\pgfsetbuttcap%
\pgfsetroundjoin%
\pgfsetlinewidth{0.501875pt}%
\definecolor{currentstroke}{rgb}{0.166383,0.690856,0.496502}%
\pgfsetstrokecolor{currentstroke}%
\pgfsetdash{}{0pt}%
\pgfpathmoveto{\pgfqpoint{2.283215in}{4.025369in}}%
\pgfpathlineto{\pgfqpoint{2.230240in}{4.025478in}}%
\pgfusepath{stroke}%
\end{pgfscope}%
\begin{pgfscope}%
\pgfpathrectangle{\pgfqpoint{0.800000in}{3.252941in}}{\pgfqpoint{2.407767in}{1.544118in}}%
\pgfusepath{clip}%
\pgfsetbuttcap%
\pgfsetroundjoin%
\pgfsetlinewidth{0.501875pt}%
\definecolor{currentstroke}{rgb}{0.377779,0.791781,0.377939}%
\pgfsetstrokecolor{currentstroke}%
\pgfsetdash{}{0pt}%
\pgfpathmoveto{\pgfqpoint{2.230240in}{4.025478in}}%
\pgfpathlineto{\pgfqpoint{2.177265in}{4.025566in}}%
\pgfusepath{stroke}%
\end{pgfscope}%
\begin{pgfscope}%
\pgfpathrectangle{\pgfqpoint{0.800000in}{3.252941in}}{\pgfqpoint{2.407767in}{1.544118in}}%
\pgfusepath{clip}%
\pgfsetbuttcap%
\pgfsetroundjoin%
\pgfsetlinewidth{0.501875pt}%
\definecolor{currentstroke}{rgb}{0.699415,0.867117,0.175971}%
\pgfsetstrokecolor{currentstroke}%
\pgfsetdash{}{0pt}%
\pgfpathmoveto{\pgfqpoint{2.177265in}{4.025566in}}%
\pgfpathlineto{\pgfqpoint{2.124290in}{4.025712in}}%
\pgfusepath{stroke}%
\end{pgfscope}%
\begin{pgfscope}%
\pgfpathrectangle{\pgfqpoint{0.800000in}{3.252941in}}{\pgfqpoint{2.407767in}{1.544118in}}%
\pgfusepath{clip}%
\pgfsetbuttcap%
\pgfsetroundjoin%
\pgfsetlinewidth{0.501875pt}%
\definecolor{currentstroke}{rgb}{0.845561,0.887322,0.099702}%
\pgfsetstrokecolor{currentstroke}%
\pgfsetdash{}{0pt}%
\pgfpathmoveto{\pgfqpoint{2.124290in}{4.025712in}}%
\pgfpathlineto{\pgfqpoint{2.071317in}{4.025917in}}%
\pgfusepath{stroke}%
\end{pgfscope}%
\begin{pgfscope}%
\pgfpathrectangle{\pgfqpoint{0.800000in}{3.252941in}}{\pgfqpoint{2.407767in}{1.544118in}}%
\pgfusepath{clip}%
\pgfsetbuttcap%
\pgfsetroundjoin%
\pgfsetlinewidth{0.501875pt}%
\definecolor{currentstroke}{rgb}{0.886271,0.892374,0.095374}%
\pgfsetstrokecolor{currentstroke}%
\pgfsetdash{}{0pt}%
\pgfpathmoveto{\pgfqpoint{2.071317in}{4.025917in}}%
\pgfpathlineto{\pgfqpoint{2.018344in}{4.026066in}}%
\pgfusepath{stroke}%
\end{pgfscope}%
\begin{pgfscope}%
\pgfpathrectangle{\pgfqpoint{0.800000in}{3.252941in}}{\pgfqpoint{2.407767in}{1.544118in}}%
\pgfusepath{clip}%
\pgfsetbuttcap%
\pgfsetroundjoin%
\pgfsetlinewidth{0.501875pt}%
\definecolor{currentstroke}{rgb}{0.993248,0.906157,0.143936}%
\pgfsetstrokecolor{currentstroke}%
\pgfsetdash{}{0pt}%
\pgfpathmoveto{\pgfqpoint{2.018344in}{4.026066in}}%
\pgfpathlineto{\pgfqpoint{1.965371in}{4.026150in}}%
\pgfusepath{stroke}%
\end{pgfscope}%
\begin{pgfscope}%
\pgfpathrectangle{\pgfqpoint{0.800000in}{3.252941in}}{\pgfqpoint{2.407767in}{1.544118in}}%
\pgfusepath{clip}%
\pgfsetbuttcap%
\pgfsetroundjoin%
\pgfsetlinewidth{0.501875pt}%
\definecolor{currentstroke}{rgb}{0.993248,0.906157,0.143936}%
\pgfsetstrokecolor{currentstroke}%
\pgfsetdash{}{0pt}%
\pgfpathmoveto{\pgfqpoint{1.965371in}{4.026150in}}%
\pgfpathlineto{\pgfqpoint{1.912399in}{4.026189in}}%
\pgfusepath{stroke}%
\end{pgfscope}%
\begin{pgfscope}%
\pgfpathrectangle{\pgfqpoint{0.800000in}{3.252941in}}{\pgfqpoint{2.407767in}{1.544118in}}%
\pgfusepath{clip}%
\pgfsetbuttcap%
\pgfsetroundjoin%
\pgfsetlinewidth{0.501875pt}%
\definecolor{currentstroke}{rgb}{0.993248,0.906157,0.143936}%
\pgfsetstrokecolor{currentstroke}%
\pgfsetdash{}{0pt}%
\pgfpathmoveto{\pgfqpoint{1.912399in}{4.026189in}}%
\pgfpathlineto{\pgfqpoint{1.859427in}{4.026158in}}%
\pgfusepath{stroke}%
\end{pgfscope}%
\begin{pgfscope}%
\pgfpathrectangle{\pgfqpoint{0.800000in}{3.252941in}}{\pgfqpoint{2.407767in}{1.544118in}}%
\pgfusepath{clip}%
\pgfsetbuttcap%
\pgfsetroundjoin%
\pgfsetlinewidth{0.501875pt}%
\definecolor{currentstroke}{rgb}{0.993248,0.906157,0.143936}%
\pgfsetstrokecolor{currentstroke}%
\pgfsetdash{}{0pt}%
\pgfpathmoveto{\pgfqpoint{1.859427in}{4.026158in}}%
\pgfpathlineto{\pgfqpoint{1.806457in}{4.026109in}}%
\pgfusepath{stroke}%
\end{pgfscope}%
\begin{pgfscope}%
\pgfpathrectangle{\pgfqpoint{0.800000in}{3.252941in}}{\pgfqpoint{2.407767in}{1.544118in}}%
\pgfusepath{clip}%
\pgfsetbuttcap%
\pgfsetroundjoin%
\pgfsetlinewidth{0.501875pt}%
\definecolor{currentstroke}{rgb}{0.277941,0.056324,0.381191}%
\pgfsetstrokecolor{currentstroke}%
\pgfsetdash{}{0pt}%
\pgfpathmoveto{\pgfqpoint{2.654046in}{4.059746in}}%
\pgfpathlineto{\pgfqpoint{2.601071in}{4.059571in}}%
\pgfusepath{stroke}%
\end{pgfscope}%
\begin{pgfscope}%
\pgfpathrectangle{\pgfqpoint{0.800000in}{3.252941in}}{\pgfqpoint{2.407767in}{1.544118in}}%
\pgfusepath{clip}%
\pgfsetbuttcap%
\pgfsetroundjoin%
\pgfsetlinewidth{0.501875pt}%
\definecolor{currentstroke}{rgb}{0.283197,0.115680,0.436115}%
\pgfsetstrokecolor{currentstroke}%
\pgfsetdash{}{0pt}%
\pgfpathmoveto{\pgfqpoint{2.601071in}{4.059571in}}%
\pgfpathlineto{\pgfqpoint{2.548095in}{4.059383in}}%
\pgfusepath{stroke}%
\end{pgfscope}%
\begin{pgfscope}%
\pgfpathrectangle{\pgfqpoint{0.800000in}{3.252941in}}{\pgfqpoint{2.407767in}{1.544118in}}%
\pgfusepath{clip}%
\pgfsetbuttcap%
\pgfsetroundjoin%
\pgfsetlinewidth{0.501875pt}%
\definecolor{currentstroke}{rgb}{0.278826,0.175490,0.483397}%
\pgfsetstrokecolor{currentstroke}%
\pgfsetdash{}{0pt}%
\pgfpathmoveto{\pgfqpoint{2.548095in}{4.059383in}}%
\pgfpathlineto{\pgfqpoint{2.495122in}{4.059077in}}%
\pgfusepath{stroke}%
\end{pgfscope}%
\begin{pgfscope}%
\pgfpathrectangle{\pgfqpoint{0.800000in}{3.252941in}}{\pgfqpoint{2.407767in}{1.544118in}}%
\pgfusepath{clip}%
\pgfsetbuttcap%
\pgfsetroundjoin%
\pgfsetlinewidth{0.501875pt}%
\definecolor{currentstroke}{rgb}{0.258965,0.251537,0.524736}%
\pgfsetstrokecolor{currentstroke}%
\pgfsetdash{}{0pt}%
\pgfpathmoveto{\pgfqpoint{2.495122in}{4.059077in}}%
\pgfpathlineto{\pgfqpoint{2.442150in}{4.058681in}}%
\pgfusepath{stroke}%
\end{pgfscope}%
\begin{pgfscope}%
\pgfpathrectangle{\pgfqpoint{0.800000in}{3.252941in}}{\pgfqpoint{2.407767in}{1.544118in}}%
\pgfusepath{clip}%
\pgfsetbuttcap%
\pgfsetroundjoin%
\pgfsetlinewidth{0.501875pt}%
\definecolor{currentstroke}{rgb}{0.229739,0.322361,0.545706}%
\pgfsetstrokecolor{currentstroke}%
\pgfsetdash{}{0pt}%
\pgfpathmoveto{\pgfqpoint{2.442150in}{4.058681in}}%
\pgfpathlineto{\pgfqpoint{2.389180in}{4.058156in}}%
\pgfusepath{stroke}%
\end{pgfscope}%
\begin{pgfscope}%
\pgfpathrectangle{\pgfqpoint{0.800000in}{3.252941in}}{\pgfqpoint{2.407767in}{1.544118in}}%
\pgfusepath{clip}%
\pgfsetbuttcap%
\pgfsetroundjoin%
\pgfsetlinewidth{0.501875pt}%
\definecolor{currentstroke}{rgb}{0.187231,0.414746,0.556547}%
\pgfsetstrokecolor{currentstroke}%
\pgfsetdash{}{0pt}%
\pgfpathmoveto{\pgfqpoint{2.389180in}{4.058156in}}%
\pgfpathlineto{\pgfqpoint{2.336216in}{4.057437in}}%
\pgfusepath{stroke}%
\end{pgfscope}%
\begin{pgfscope}%
\pgfpathrectangle{\pgfqpoint{0.800000in}{3.252941in}}{\pgfqpoint{2.407767in}{1.544118in}}%
\pgfusepath{clip}%
\pgfsetbuttcap%
\pgfsetroundjoin%
\pgfsetlinewidth{0.501875pt}%
\definecolor{currentstroke}{rgb}{0.137770,0.537492,0.554906}%
\pgfsetstrokecolor{currentstroke}%
\pgfsetdash{}{0pt}%
\pgfpathmoveto{\pgfqpoint{2.336216in}{4.057437in}}%
\pgfpathlineto{\pgfqpoint{2.283259in}{4.056544in}}%
\pgfusepath{stroke}%
\end{pgfscope}%
\begin{pgfscope}%
\pgfpathrectangle{\pgfqpoint{0.800000in}{3.252941in}}{\pgfqpoint{2.407767in}{1.544118in}}%
\pgfusepath{clip}%
\pgfsetbuttcap%
\pgfsetroundjoin%
\pgfsetlinewidth{0.501875pt}%
\definecolor{currentstroke}{rgb}{0.134692,0.658636,0.517649}%
\pgfsetstrokecolor{currentstroke}%
\pgfsetdash{}{0pt}%
\pgfpathmoveto{\pgfqpoint{2.283259in}{4.056544in}}%
\pgfpathlineto{\pgfqpoint{2.230316in}{4.055351in}}%
\pgfusepath{stroke}%
\end{pgfscope}%
\begin{pgfscope}%
\pgfpathrectangle{\pgfqpoint{0.800000in}{3.252941in}}{\pgfqpoint{2.407767in}{1.544118in}}%
\pgfusepath{clip}%
\pgfsetbuttcap%
\pgfsetroundjoin%
\pgfsetlinewidth{0.501875pt}%
\definecolor{currentstroke}{rgb}{0.344074,0.780029,0.397381}%
\pgfsetstrokecolor{currentstroke}%
\pgfsetdash{}{0pt}%
\pgfpathmoveto{\pgfqpoint{2.230316in}{4.055351in}}%
\pgfpathlineto{\pgfqpoint{2.177390in}{4.053888in}}%
\pgfusepath{stroke}%
\end{pgfscope}%
\begin{pgfscope}%
\pgfpathrectangle{\pgfqpoint{0.800000in}{3.252941in}}{\pgfqpoint{2.407767in}{1.544118in}}%
\pgfusepath{clip}%
\pgfsetbuttcap%
\pgfsetroundjoin%
\pgfsetlinewidth{0.501875pt}%
\definecolor{currentstroke}{rgb}{0.525776,0.833491,0.288127}%
\pgfsetstrokecolor{currentstroke}%
\pgfsetdash{}{0pt}%
\pgfpathmoveto{\pgfqpoint{2.177390in}{4.053888in}}%
\pgfpathlineto{\pgfqpoint{2.124479in}{4.052258in}}%
\pgfusepath{stroke}%
\end{pgfscope}%
\begin{pgfscope}%
\pgfpathrectangle{\pgfqpoint{0.800000in}{3.252941in}}{\pgfqpoint{2.407767in}{1.544118in}}%
\pgfusepath{clip}%
\pgfsetbuttcap%
\pgfsetroundjoin%
\pgfsetlinewidth{0.501875pt}%
\definecolor{currentstroke}{rgb}{0.636902,0.856542,0.216620}%
\pgfsetstrokecolor{currentstroke}%
\pgfsetdash{}{0pt}%
\pgfpathmoveto{\pgfqpoint{2.124479in}{4.052258in}}%
\pgfpathlineto{\pgfqpoint{2.071585in}{4.050442in}}%
\pgfusepath{stroke}%
\end{pgfscope}%
\begin{pgfscope}%
\pgfpathrectangle{\pgfqpoint{0.800000in}{3.252941in}}{\pgfqpoint{2.407767in}{1.544118in}}%
\pgfusepath{clip}%
\pgfsetbuttcap%
\pgfsetroundjoin%
\pgfsetlinewidth{0.501875pt}%
\definecolor{currentstroke}{rgb}{0.804182,0.882046,0.114965}%
\pgfsetstrokecolor{currentstroke}%
\pgfsetdash{}{0pt}%
\pgfpathmoveto{\pgfqpoint{2.071585in}{4.050442in}}%
\pgfpathlineto{\pgfqpoint{2.018707in}{4.048490in}}%
\pgfusepath{stroke}%
\end{pgfscope}%
\begin{pgfscope}%
\pgfpathrectangle{\pgfqpoint{0.800000in}{3.252941in}}{\pgfqpoint{2.407767in}{1.544118in}}%
\pgfusepath{clip}%
\pgfsetbuttcap%
\pgfsetroundjoin%
\pgfsetlinewidth{0.501875pt}%
\definecolor{currentstroke}{rgb}{0.955300,0.901065,0.118128}%
\pgfsetstrokecolor{currentstroke}%
\pgfsetdash{}{0pt}%
\pgfpathmoveto{\pgfqpoint{2.018707in}{4.048490in}}%
\pgfpathlineto{\pgfqpoint{1.965836in}{4.046510in}}%
\pgfusepath{stroke}%
\end{pgfscope}%
\begin{pgfscope}%
\pgfpathrectangle{\pgfqpoint{0.800000in}{3.252941in}}{\pgfqpoint{2.407767in}{1.544118in}}%
\pgfusepath{clip}%
\pgfsetbuttcap%
\pgfsetroundjoin%
\pgfsetlinewidth{0.501875pt}%
\definecolor{currentstroke}{rgb}{0.993248,0.906157,0.143936}%
\pgfsetstrokecolor{currentstroke}%
\pgfsetdash{}{0pt}%
\pgfpathmoveto{\pgfqpoint{1.965836in}{4.046510in}}%
\pgfpathlineto{\pgfqpoint{1.912968in}{4.044510in}}%
\pgfusepath{stroke}%
\end{pgfscope}%
\begin{pgfscope}%
\pgfpathrectangle{\pgfqpoint{0.800000in}{3.252941in}}{\pgfqpoint{2.407767in}{1.544118in}}%
\pgfusepath{clip}%
\pgfsetbuttcap%
\pgfsetroundjoin%
\pgfsetlinewidth{0.501875pt}%
\definecolor{currentstroke}{rgb}{0.993248,0.906157,0.143936}%
\pgfsetstrokecolor{currentstroke}%
\pgfsetdash{}{0pt}%
\pgfpathmoveto{\pgfqpoint{1.912968in}{4.044510in}}%
\pgfpathlineto{\pgfqpoint{1.860096in}{4.042507in}}%
\pgfusepath{stroke}%
\end{pgfscope}%
\begin{pgfscope}%
\pgfpathrectangle{\pgfqpoint{0.800000in}{3.252941in}}{\pgfqpoint{2.407767in}{1.544118in}}%
\pgfusepath{clip}%
\pgfsetbuttcap%
\pgfsetroundjoin%
\pgfsetlinewidth{0.501875pt}%
\definecolor{currentstroke}{rgb}{0.277018,0.050344,0.375715}%
\pgfsetstrokecolor{currentstroke}%
\pgfsetdash{}{0pt}%
\pgfpathmoveto{\pgfqpoint{2.654046in}{4.094492in}}%
\pgfpathlineto{\pgfqpoint{2.601072in}{4.094362in}}%
\pgfusepath{stroke}%
\end{pgfscope}%
\begin{pgfscope}%
\pgfpathrectangle{\pgfqpoint{0.800000in}{3.252941in}}{\pgfqpoint{2.407767in}{1.544118in}}%
\pgfusepath{clip}%
\pgfsetbuttcap%
\pgfsetroundjoin%
\pgfsetlinewidth{0.501875pt}%
\definecolor{currentstroke}{rgb}{0.283091,0.110553,0.431554}%
\pgfsetstrokecolor{currentstroke}%
\pgfsetdash{}{0pt}%
\pgfpathmoveto{\pgfqpoint{2.601072in}{4.094362in}}%
\pgfpathlineto{\pgfqpoint{2.548098in}{4.094071in}}%
\pgfusepath{stroke}%
\end{pgfscope}%
\begin{pgfscope}%
\pgfpathrectangle{\pgfqpoint{0.800000in}{3.252941in}}{\pgfqpoint{2.407767in}{1.544118in}}%
\pgfusepath{clip}%
\pgfsetbuttcap%
\pgfsetroundjoin%
\pgfsetlinewidth{0.501875pt}%
\definecolor{currentstroke}{rgb}{0.278012,0.180367,0.486697}%
\pgfsetstrokecolor{currentstroke}%
\pgfsetdash{}{0pt}%
\pgfpathmoveto{\pgfqpoint{2.548098in}{4.094071in}}%
\pgfpathlineto{\pgfqpoint{2.495127in}{4.093630in}}%
\pgfusepath{stroke}%
\end{pgfscope}%
\begin{pgfscope}%
\pgfpathrectangle{\pgfqpoint{0.800000in}{3.252941in}}{\pgfqpoint{2.407767in}{1.544118in}}%
\pgfusepath{clip}%
\pgfsetbuttcap%
\pgfsetroundjoin%
\pgfsetlinewidth{0.501875pt}%
\definecolor{currentstroke}{rgb}{0.266580,0.228262,0.514349}%
\pgfsetstrokecolor{currentstroke}%
\pgfsetdash{}{0pt}%
\pgfpathmoveto{\pgfqpoint{2.495127in}{4.093630in}}%
\pgfpathlineto{\pgfqpoint{2.442162in}{4.092966in}}%
\pgfusepath{stroke}%
\end{pgfscope}%
\begin{pgfscope}%
\pgfpathrectangle{\pgfqpoint{0.800000in}{3.252941in}}{\pgfqpoint{2.407767in}{1.544118in}}%
\pgfusepath{clip}%
\pgfsetbuttcap%
\pgfsetroundjoin%
\pgfsetlinewidth{0.501875pt}%
\definecolor{currentstroke}{rgb}{0.239346,0.300855,0.540844}%
\pgfsetstrokecolor{currentstroke}%
\pgfsetdash{}{0pt}%
\pgfpathmoveto{\pgfqpoint{2.442162in}{4.092966in}}%
\pgfpathlineto{\pgfqpoint{2.389203in}{4.092114in}}%
\pgfusepath{stroke}%
\end{pgfscope}%
\begin{pgfscope}%
\pgfpathrectangle{\pgfqpoint{0.800000in}{3.252941in}}{\pgfqpoint{2.407767in}{1.544118in}}%
\pgfusepath{clip}%
\pgfsetbuttcap%
\pgfsetroundjoin%
\pgfsetlinewidth{0.501875pt}%
\definecolor{currentstroke}{rgb}{0.195860,0.395433,0.555276}%
\pgfsetstrokecolor{currentstroke}%
\pgfsetdash{}{0pt}%
\pgfpathmoveto{\pgfqpoint{2.389203in}{4.092114in}}%
\pgfpathlineto{\pgfqpoint{2.336255in}{4.091042in}}%
\pgfusepath{stroke}%
\end{pgfscope}%
\begin{pgfscope}%
\pgfpathrectangle{\pgfqpoint{0.800000in}{3.252941in}}{\pgfqpoint{2.407767in}{1.544118in}}%
\pgfusepath{clip}%
\pgfsetbuttcap%
\pgfsetroundjoin%
\pgfsetlinewidth{0.501875pt}%
\definecolor{currentstroke}{rgb}{0.169646,0.456262,0.558030}%
\pgfsetstrokecolor{currentstroke}%
\pgfsetdash{}{0pt}%
\pgfpathmoveto{\pgfqpoint{2.336255in}{4.091042in}}%
\pgfpathlineto{\pgfqpoint{2.283330in}{4.089586in}}%
\pgfusepath{stroke}%
\end{pgfscope}%
\begin{pgfscope}%
\pgfpathrectangle{\pgfqpoint{0.800000in}{3.252941in}}{\pgfqpoint{2.407767in}{1.544118in}}%
\pgfusepath{clip}%
\pgfsetbuttcap%
\pgfsetroundjoin%
\pgfsetlinewidth{0.501875pt}%
\definecolor{currentstroke}{rgb}{0.132444,0.552216,0.553018}%
\pgfsetstrokecolor{currentstroke}%
\pgfsetdash{}{0pt}%
\pgfpathmoveto{\pgfqpoint{2.283330in}{4.089586in}}%
\pgfpathlineto{\pgfqpoint{2.230457in}{4.087512in}}%
\pgfusepath{stroke}%
\end{pgfscope}%
\begin{pgfscope}%
\pgfpathrectangle{\pgfqpoint{0.800000in}{3.252941in}}{\pgfqpoint{2.407767in}{1.544118in}}%
\pgfusepath{clip}%
\pgfsetbuttcap%
\pgfsetroundjoin%
\pgfsetlinewidth{0.501875pt}%
\definecolor{currentstroke}{rgb}{0.191090,0.708366,0.482284}%
\pgfsetstrokecolor{currentstroke}%
\pgfsetdash{}{0pt}%
\pgfpathmoveto{\pgfqpoint{2.230457in}{4.087512in}}%
\pgfpathlineto{\pgfqpoint{2.177651in}{4.084810in}}%
\pgfusepath{stroke}%
\end{pgfscope}%
\begin{pgfscope}%
\pgfpathrectangle{\pgfqpoint{0.800000in}{3.252941in}}{\pgfqpoint{2.407767in}{1.544118in}}%
\pgfusepath{clip}%
\pgfsetbuttcap%
\pgfsetroundjoin%
\pgfsetlinewidth{0.501875pt}%
\definecolor{currentstroke}{rgb}{0.232815,0.732247,0.459277}%
\pgfsetstrokecolor{currentstroke}%
\pgfsetdash{}{0pt}%
\pgfpathmoveto{\pgfqpoint{2.177651in}{4.084810in}}%
\pgfpathlineto{\pgfqpoint{2.124919in}{4.081585in}}%
\pgfusepath{stroke}%
\end{pgfscope}%
\begin{pgfscope}%
\pgfpathrectangle{\pgfqpoint{0.800000in}{3.252941in}}{\pgfqpoint{2.407767in}{1.544118in}}%
\pgfusepath{clip}%
\pgfsetbuttcap%
\pgfsetroundjoin%
\pgfsetlinewidth{0.501875pt}%
\definecolor{currentstroke}{rgb}{0.525776,0.833491,0.288127}%
\pgfsetstrokecolor{currentstroke}%
\pgfsetdash{}{0pt}%
\pgfpathmoveto{\pgfqpoint{2.124919in}{4.081585in}}%
\pgfpathlineto{\pgfqpoint{2.072264in}{4.077888in}}%
\pgfusepath{stroke}%
\end{pgfscope}%
\begin{pgfscope}%
\pgfpathrectangle{\pgfqpoint{0.800000in}{3.252941in}}{\pgfqpoint{2.407767in}{1.544118in}}%
\pgfusepath{clip}%
\pgfsetbuttcap%
\pgfsetroundjoin%
\pgfsetlinewidth{0.501875pt}%
\definecolor{currentstroke}{rgb}{0.276022,0.044167,0.370164}%
\pgfsetstrokecolor{currentstroke}%
\pgfsetdash{}{0pt}%
\pgfpathmoveto{\pgfqpoint{2.654046in}{4.129238in}}%
\pgfpathlineto{\pgfqpoint{2.601072in}{4.128990in}}%
\pgfusepath{stroke}%
\end{pgfscope}%
\begin{pgfscope}%
\pgfpathrectangle{\pgfqpoint{0.800000in}{3.252941in}}{\pgfqpoint{2.407767in}{1.544118in}}%
\pgfusepath{clip}%
\pgfsetbuttcap%
\pgfsetroundjoin%
\pgfsetlinewidth{0.501875pt}%
\definecolor{currentstroke}{rgb}{0.283091,0.110553,0.431554}%
\pgfsetstrokecolor{currentstroke}%
\pgfsetdash{}{0pt}%
\pgfpathmoveto{\pgfqpoint{2.601072in}{4.128990in}}%
\pgfpathlineto{\pgfqpoint{2.548103in}{4.128474in}}%
\pgfusepath{stroke}%
\end{pgfscope}%
\begin{pgfscope}%
\pgfpathrectangle{\pgfqpoint{0.800000in}{3.252941in}}{\pgfqpoint{2.407767in}{1.544118in}}%
\pgfusepath{clip}%
\pgfsetbuttcap%
\pgfsetroundjoin%
\pgfsetlinewidth{0.501875pt}%
\definecolor{currentstroke}{rgb}{0.281412,0.155834,0.469201}%
\pgfsetstrokecolor{currentstroke}%
\pgfsetdash{}{0pt}%
\pgfpathmoveto{\pgfqpoint{2.548103in}{4.128474in}}%
\pgfpathlineto{\pgfqpoint{2.495140in}{4.127745in}}%
\pgfusepath{stroke}%
\end{pgfscope}%
\begin{pgfscope}%
\pgfpathrectangle{\pgfqpoint{0.800000in}{3.252941in}}{\pgfqpoint{2.407767in}{1.544118in}}%
\pgfusepath{clip}%
\pgfsetbuttcap%
\pgfsetroundjoin%
\pgfsetlinewidth{0.501875pt}%
\definecolor{currentstroke}{rgb}{0.270595,0.214069,0.507052}%
\pgfsetstrokecolor{currentstroke}%
\pgfsetdash{}{0pt}%
\pgfpathmoveto{\pgfqpoint{2.495140in}{4.127745in}}%
\pgfpathlineto{\pgfqpoint{2.442186in}{4.126768in}}%
\pgfusepath{stroke}%
\end{pgfscope}%
\begin{pgfscope}%
\pgfpathrectangle{\pgfqpoint{0.800000in}{3.252941in}}{\pgfqpoint{2.407767in}{1.544118in}}%
\pgfusepath{clip}%
\pgfsetbuttcap%
\pgfsetroundjoin%
\pgfsetlinewidth{0.501875pt}%
\definecolor{currentstroke}{rgb}{0.248629,0.278775,0.534556}%
\pgfsetstrokecolor{currentstroke}%
\pgfsetdash{}{0pt}%
\pgfpathmoveto{\pgfqpoint{2.442186in}{4.126768in}}%
\pgfpathlineto{\pgfqpoint{2.389246in}{4.125535in}}%
\pgfusepath{stroke}%
\end{pgfscope}%
\begin{pgfscope}%
\pgfpathrectangle{\pgfqpoint{0.800000in}{3.252941in}}{\pgfqpoint{2.407767in}{1.544118in}}%
\pgfusepath{clip}%
\pgfsetbuttcap%
\pgfsetroundjoin%
\pgfsetlinewidth{0.501875pt}%
\definecolor{currentstroke}{rgb}{0.225863,0.330805,0.547314}%
\pgfsetstrokecolor{currentstroke}%
\pgfsetdash{}{0pt}%
\pgfpathmoveto{\pgfqpoint{2.389246in}{4.125535in}}%
\pgfpathlineto{\pgfqpoint{2.336330in}{4.123935in}}%
\pgfusepath{stroke}%
\end{pgfscope}%
\begin{pgfscope}%
\pgfpathrectangle{\pgfqpoint{0.800000in}{3.252941in}}{\pgfqpoint{2.407767in}{1.544118in}}%
\pgfusepath{clip}%
\pgfsetbuttcap%
\pgfsetroundjoin%
\pgfsetlinewidth{0.501875pt}%
\definecolor{currentstroke}{rgb}{0.192357,0.403199,0.555836}%
\pgfsetstrokecolor{currentstroke}%
\pgfsetdash{}{0pt}%
\pgfpathmoveto{\pgfqpoint{2.336330in}{4.123935in}}%
\pgfpathlineto{\pgfqpoint{2.283462in}{4.121790in}}%
\pgfusepath{stroke}%
\end{pgfscope}%
\begin{pgfscope}%
\pgfpathrectangle{\pgfqpoint{0.800000in}{3.252941in}}{\pgfqpoint{2.407767in}{1.544118in}}%
\pgfusepath{clip}%
\pgfsetbuttcap%
\pgfsetroundjoin%
\pgfsetlinewidth{0.501875pt}%
\definecolor{currentstroke}{rgb}{0.166617,0.463708,0.558119}%
\pgfsetstrokecolor{currentstroke}%
\pgfsetdash{}{0pt}%
\pgfpathmoveto{\pgfqpoint{2.283462in}{4.121790in}}%
\pgfpathlineto{\pgfqpoint{2.230686in}{4.118882in}}%
\pgfusepath{stroke}%
\end{pgfscope}%
\begin{pgfscope}%
\pgfpathrectangle{\pgfqpoint{0.800000in}{3.252941in}}{\pgfqpoint{2.407767in}{1.544118in}}%
\pgfusepath{clip}%
\pgfsetbuttcap%
\pgfsetroundjoin%
\pgfsetlinewidth{0.501875pt}%
\definecolor{currentstroke}{rgb}{0.149039,0.508051,0.557250}%
\pgfsetstrokecolor{currentstroke}%
\pgfsetdash{}{0pt}%
\pgfpathmoveto{\pgfqpoint{2.230686in}{4.118882in}}%
\pgfpathlineto{\pgfqpoint{2.178046in}{4.115095in}}%
\pgfusepath{stroke}%
\end{pgfscope}%
\begin{pgfscope}%
\pgfpathrectangle{\pgfqpoint{0.800000in}{3.252941in}}{\pgfqpoint{2.407767in}{1.544118in}}%
\pgfusepath{clip}%
\pgfsetbuttcap%
\pgfsetroundjoin%
\pgfsetlinewidth{0.501875pt}%
\definecolor{currentstroke}{rgb}{0.276022,0.044167,0.370164}%
\pgfsetstrokecolor{currentstroke}%
\pgfsetdash{}{0pt}%
\pgfpathmoveto{\pgfqpoint{2.654046in}{4.163984in}}%
\pgfpathlineto{\pgfqpoint{2.601074in}{4.163637in}}%
\pgfusepath{stroke}%
\end{pgfscope}%
\begin{pgfscope}%
\pgfpathrectangle{\pgfqpoint{0.800000in}{3.252941in}}{\pgfqpoint{2.407767in}{1.544118in}}%
\pgfusepath{clip}%
\pgfsetbuttcap%
\pgfsetroundjoin%
\pgfsetlinewidth{0.501875pt}%
\definecolor{currentstroke}{rgb}{0.282656,0.100196,0.422160}%
\pgfsetstrokecolor{currentstroke}%
\pgfsetdash{}{0pt}%
\pgfpathmoveto{\pgfqpoint{2.601074in}{4.163637in}}%
\pgfpathlineto{\pgfqpoint{2.548108in}{4.162987in}}%
\pgfusepath{stroke}%
\end{pgfscope}%
\begin{pgfscope}%
\pgfpathrectangle{\pgfqpoint{0.800000in}{3.252941in}}{\pgfqpoint{2.407767in}{1.544118in}}%
\pgfusepath{clip}%
\pgfsetbuttcap%
\pgfsetroundjoin%
\pgfsetlinewidth{0.501875pt}%
\definecolor{currentstroke}{rgb}{0.282623,0.140926,0.457517}%
\pgfsetstrokecolor{currentstroke}%
\pgfsetdash{}{0pt}%
\pgfpathmoveto{\pgfqpoint{2.548108in}{4.162987in}}%
\pgfpathlineto{\pgfqpoint{2.495157in}{4.161981in}}%
\pgfusepath{stroke}%
\end{pgfscope}%
\begin{pgfscope}%
\pgfpathrectangle{\pgfqpoint{0.800000in}{3.252941in}}{\pgfqpoint{2.407767in}{1.544118in}}%
\pgfusepath{clip}%
\pgfsetbuttcap%
\pgfsetroundjoin%
\pgfsetlinewidth{0.501875pt}%
\definecolor{currentstroke}{rgb}{0.276194,0.190074,0.493001}%
\pgfsetstrokecolor{currentstroke}%
\pgfsetdash{}{0pt}%
\pgfpathmoveto{\pgfqpoint{2.495157in}{4.161981in}}%
\pgfpathlineto{\pgfqpoint{2.442226in}{4.160585in}}%
\pgfusepath{stroke}%
\end{pgfscope}%
\begin{pgfscope}%
\pgfpathrectangle{\pgfqpoint{0.800000in}{3.252941in}}{\pgfqpoint{2.407767in}{1.544118in}}%
\pgfusepath{clip}%
\pgfsetbuttcap%
\pgfsetroundjoin%
\pgfsetlinewidth{0.501875pt}%
\definecolor{currentstroke}{rgb}{0.265145,0.232956,0.516599}%
\pgfsetstrokecolor{currentstroke}%
\pgfsetdash{}{0pt}%
\pgfpathmoveto{\pgfqpoint{2.442226in}{4.160585in}}%
\pgfpathlineto{\pgfqpoint{2.389311in}{4.158958in}}%
\pgfusepath{stroke}%
\end{pgfscope}%
\begin{pgfscope}%
\pgfpathrectangle{\pgfqpoint{0.800000in}{3.252941in}}{\pgfqpoint{2.407767in}{1.544118in}}%
\pgfusepath{clip}%
\pgfsetbuttcap%
\pgfsetroundjoin%
\pgfsetlinewidth{0.501875pt}%
\definecolor{currentstroke}{rgb}{0.250425,0.274290,0.533103}%
\pgfsetstrokecolor{currentstroke}%
\pgfsetdash{}{0pt}%
\pgfpathmoveto{\pgfqpoint{2.389311in}{4.158958in}}%
\pgfpathlineto{\pgfqpoint{2.336440in}{4.156863in}}%
\pgfusepath{stroke}%
\end{pgfscope}%
\begin{pgfscope}%
\pgfpathrectangle{\pgfqpoint{0.800000in}{3.252941in}}{\pgfqpoint{2.407767in}{1.544118in}}%
\pgfusepath{clip}%
\pgfsetbuttcap%
\pgfsetroundjoin%
\pgfsetlinewidth{0.501875pt}%
\definecolor{currentstroke}{rgb}{0.237441,0.305202,0.541921}%
\pgfsetstrokecolor{currentstroke}%
\pgfsetdash{}{0pt}%
\pgfpathmoveto{\pgfqpoint{2.336440in}{4.156863in}}%
\pgfpathlineto{\pgfqpoint{2.283654in}{4.154034in}}%
\pgfusepath{stroke}%
\end{pgfscope}%
\begin{pgfscope}%
\pgfpathrectangle{\pgfqpoint{0.800000in}{3.252941in}}{\pgfqpoint{2.407767in}{1.544118in}}%
\pgfusepath{clip}%
\pgfsetbuttcap%
\pgfsetroundjoin%
\pgfsetlinewidth{0.501875pt}%
\definecolor{currentstroke}{rgb}{0.204903,0.375746,0.553533}%
\pgfsetstrokecolor{currentstroke}%
\pgfsetdash{}{0pt}%
\pgfpathmoveto{\pgfqpoint{2.283654in}{4.154034in}}%
\pgfpathlineto{\pgfqpoint{2.230995in}{4.150373in}}%
\pgfusepath{stroke}%
\end{pgfscope}%
\begin{pgfscope}%
\pgfpathrectangle{\pgfqpoint{0.800000in}{3.252941in}}{\pgfqpoint{2.407767in}{1.544118in}}%
\pgfusepath{clip}%
\pgfsetbuttcap%
\pgfsetroundjoin%
\pgfsetlinewidth{0.501875pt}%
\definecolor{currentstroke}{rgb}{0.276022,0.044167,0.370164}%
\pgfsetstrokecolor{currentstroke}%
\pgfsetdash{}{0pt}%
\pgfpathmoveto{\pgfqpoint{2.654046in}{4.198731in}}%
\pgfpathlineto{\pgfqpoint{2.601076in}{4.198311in}}%
\pgfusepath{stroke}%
\end{pgfscope}%
\begin{pgfscope}%
\pgfpathrectangle{\pgfqpoint{0.800000in}{3.252941in}}{\pgfqpoint{2.407767in}{1.544118in}}%
\pgfusepath{clip}%
\pgfsetbuttcap%
\pgfsetroundjoin%
\pgfsetlinewidth{0.501875pt}%
\definecolor{currentstroke}{rgb}{0.281446,0.084320,0.407414}%
\pgfsetstrokecolor{currentstroke}%
\pgfsetdash{}{0pt}%
\pgfpathmoveto{\pgfqpoint{2.601076in}{4.198311in}}%
\pgfpathlineto{\pgfqpoint{2.548118in}{4.197457in}}%
\pgfusepath{stroke}%
\end{pgfscope}%
\begin{pgfscope}%
\pgfpathrectangle{\pgfqpoint{0.800000in}{3.252941in}}{\pgfqpoint{2.407767in}{1.544118in}}%
\pgfusepath{clip}%
\pgfsetbuttcap%
\pgfsetroundjoin%
\pgfsetlinewidth{0.501875pt}%
\definecolor{currentstroke}{rgb}{0.283072,0.130895,0.449241}%
\pgfsetstrokecolor{currentstroke}%
\pgfsetdash{}{0pt}%
\pgfpathmoveto{\pgfqpoint{2.548118in}{4.197457in}}%
\pgfpathlineto{\pgfqpoint{2.495174in}{4.196298in}}%
\pgfusepath{stroke}%
\end{pgfscope}%
\begin{pgfscope}%
\pgfpathrectangle{\pgfqpoint{0.800000in}{3.252941in}}{\pgfqpoint{2.407767in}{1.544118in}}%
\pgfusepath{clip}%
\pgfsetbuttcap%
\pgfsetroundjoin%
\pgfsetlinewidth{0.501875pt}%
\definecolor{currentstroke}{rgb}{0.281412,0.155834,0.469201}%
\pgfsetstrokecolor{currentstroke}%
\pgfsetdash{}{0pt}%
\pgfpathmoveto{\pgfqpoint{2.495174in}{4.196298in}}%
\pgfpathlineto{\pgfqpoint{2.442250in}{4.194816in}}%
\pgfusepath{stroke}%
\end{pgfscope}%
\begin{pgfscope}%
\pgfpathrectangle{\pgfqpoint{0.800000in}{3.252941in}}{\pgfqpoint{2.407767in}{1.544118in}}%
\pgfusepath{clip}%
\pgfsetbuttcap%
\pgfsetroundjoin%
\pgfsetlinewidth{0.501875pt}%
\definecolor{currentstroke}{rgb}{0.270595,0.214069,0.507052}%
\pgfsetstrokecolor{currentstroke}%
\pgfsetdash{}{0pt}%
\pgfpathmoveto{\pgfqpoint{2.442250in}{4.194816in}}%
\pgfpathlineto{\pgfqpoint{2.389357in}{4.192929in}}%
\pgfusepath{stroke}%
\end{pgfscope}%
\begin{pgfscope}%
\pgfpathrectangle{\pgfqpoint{0.800000in}{3.252941in}}{\pgfqpoint{2.407767in}{1.544118in}}%
\pgfusepath{clip}%
\pgfsetbuttcap%
\pgfsetroundjoin%
\pgfsetlinewidth{0.501875pt}%
\definecolor{currentstroke}{rgb}{0.250425,0.274290,0.533103}%
\pgfsetstrokecolor{currentstroke}%
\pgfsetdash{}{0pt}%
\pgfpathmoveto{\pgfqpoint{2.389357in}{4.192929in}}%
\pgfpathlineto{\pgfqpoint{2.336546in}{4.190302in}}%
\pgfusepath{stroke}%
\end{pgfscope}%
\begin{pgfscope}%
\pgfpathrectangle{\pgfqpoint{0.800000in}{3.252941in}}{\pgfqpoint{2.407767in}{1.544118in}}%
\pgfusepath{clip}%
\pgfsetbuttcap%
\pgfsetroundjoin%
\pgfsetlinewidth{0.501875pt}%
\definecolor{currentstroke}{rgb}{0.252194,0.269783,0.531579}%
\pgfsetstrokecolor{currentstroke}%
\pgfsetdash{}{0pt}%
\pgfpathmoveto{\pgfqpoint{2.336546in}{4.190302in}}%
\pgfpathlineto{\pgfqpoint{2.283852in}{4.186824in}}%
\pgfusepath{stroke}%
\end{pgfscope}%
\begin{pgfscope}%
\pgfpathrectangle{\pgfqpoint{0.800000in}{3.252941in}}{\pgfqpoint{2.407767in}{1.544118in}}%
\pgfusepath{clip}%
\pgfsetbuttcap%
\pgfsetroundjoin%
\pgfsetlinewidth{0.501875pt}%
\definecolor{currentstroke}{rgb}{0.252194,0.269783,0.531579}%
\pgfsetstrokecolor{currentstroke}%
\pgfsetdash{}{0pt}%
\pgfpathmoveto{\pgfqpoint{2.283852in}{4.186824in}}%
\pgfpathlineto{\pgfqpoint{2.231336in}{4.182402in}}%
\pgfusepath{stroke}%
\end{pgfscope}%
\begin{pgfscope}%
\pgfpathrectangle{\pgfqpoint{0.800000in}{3.252941in}}{\pgfqpoint{2.407767in}{1.544118in}}%
\pgfusepath{clip}%
\pgfsetbuttcap%
\pgfsetroundjoin%
\pgfsetlinewidth{0.501875pt}%
\definecolor{currentstroke}{rgb}{0.246811,0.283237,0.535941}%
\pgfsetstrokecolor{currentstroke}%
\pgfsetdash{}{0pt}%
\pgfpathmoveto{\pgfqpoint{2.231336in}{4.182402in}}%
\pgfpathlineto{\pgfqpoint{2.179092in}{4.176805in}}%
\pgfusepath{stroke}%
\end{pgfscope}%
\begin{pgfscope}%
\pgfpathrectangle{\pgfqpoint{0.800000in}{3.252941in}}{\pgfqpoint{2.407767in}{1.544118in}}%
\pgfusepath{clip}%
\pgfsetbuttcap%
\pgfsetroundjoin%
\pgfsetlinewidth{0.501875pt}%
\definecolor{currentstroke}{rgb}{0.231674,0.318106,0.544834}%
\pgfsetstrokecolor{currentstroke}%
\pgfsetdash{}{0pt}%
\pgfpathmoveto{\pgfqpoint{2.179092in}{4.176805in}}%
\pgfpathlineto{\pgfqpoint{2.127262in}{4.169837in}}%
\pgfusepath{stroke}%
\end{pgfscope}%
\begin{pgfscope}%
\pgfpathrectangle{\pgfqpoint{0.800000in}{3.252941in}}{\pgfqpoint{2.407767in}{1.544118in}}%
\pgfusepath{clip}%
\pgfsetbuttcap%
\pgfsetroundjoin%
\pgfsetlinewidth{0.501875pt}%
\definecolor{currentstroke}{rgb}{0.194100,0.399323,0.555565}%
\pgfsetstrokecolor{currentstroke}%
\pgfsetdash{}{0pt}%
\pgfpathmoveto{\pgfqpoint{2.127262in}{4.169837in}}%
\pgfpathlineto{\pgfqpoint{2.075889in}{4.161572in}}%
\pgfusepath{stroke}%
\end{pgfscope}%
\begin{pgfscope}%
\pgfpathrectangle{\pgfqpoint{0.800000in}{3.252941in}}{\pgfqpoint{2.407767in}{1.544118in}}%
\pgfusepath{clip}%
\pgfsetbuttcap%
\pgfsetroundjoin%
\pgfsetlinewidth{0.501875pt}%
\definecolor{currentstroke}{rgb}{0.156270,0.489624,0.557936}%
\pgfsetstrokecolor{currentstroke}%
\pgfsetdash{}{0pt}%
\pgfpathmoveto{\pgfqpoint{2.075889in}{4.161572in}}%
\pgfpathlineto{\pgfqpoint{2.025007in}{4.152154in}}%
\pgfusepath{stroke}%
\end{pgfscope}%
\begin{pgfscope}%
\pgfpathrectangle{\pgfqpoint{0.800000in}{3.252941in}}{\pgfqpoint{2.407767in}{1.544118in}}%
\pgfusepath{clip}%
\pgfsetbuttcap%
\pgfsetroundjoin%
\pgfsetlinewidth{0.501875pt}%
\definecolor{currentstroke}{rgb}{0.120565,0.596422,0.543611}%
\pgfsetstrokecolor{currentstroke}%
\pgfsetdash{}{0pt}%
\pgfpathmoveto{\pgfqpoint{2.025007in}{4.152154in}}%
\pgfpathlineto{\pgfqpoint{1.974555in}{4.141820in}}%
\pgfusepath{stroke}%
\end{pgfscope}%
\begin{pgfscope}%
\pgfpathrectangle{\pgfqpoint{0.800000in}{3.252941in}}{\pgfqpoint{2.407767in}{1.544118in}}%
\pgfusepath{clip}%
\pgfsetbuttcap%
\pgfsetroundjoin%
\pgfsetlinewidth{0.501875pt}%
\definecolor{currentstroke}{rgb}{0.214000,0.722114,0.469588}%
\pgfsetstrokecolor{currentstroke}%
\pgfsetdash{}{0pt}%
\pgfpathmoveto{\pgfqpoint{1.974555in}{4.141820in}}%
\pgfpathlineto{\pgfqpoint{1.924438in}{4.130835in}}%
\pgfusepath{stroke}%
\end{pgfscope}%
\begin{pgfscope}%
\pgfpathrectangle{\pgfqpoint{0.800000in}{3.252941in}}{\pgfqpoint{2.407767in}{1.544118in}}%
\pgfusepath{clip}%
\pgfsetbuttcap%
\pgfsetroundjoin%
\pgfsetlinewidth{0.501875pt}%
\definecolor{currentstroke}{rgb}{0.440137,0.811138,0.340967}%
\pgfsetstrokecolor{currentstroke}%
\pgfsetdash{}{0pt}%
\pgfpathmoveto{\pgfqpoint{1.924438in}{4.130835in}}%
\pgfpathlineto{\pgfqpoint{1.874506in}{4.119504in}}%
\pgfusepath{stroke}%
\end{pgfscope}%
\begin{pgfscope}%
\pgfpathrectangle{\pgfqpoint{0.800000in}{3.252941in}}{\pgfqpoint{2.407767in}{1.544118in}}%
\pgfusepath{clip}%
\pgfsetbuttcap%
\pgfsetroundjoin%
\pgfsetlinewidth{0.501875pt}%
\definecolor{currentstroke}{rgb}{0.565498,0.842430,0.262877}%
\pgfsetstrokecolor{currentstroke}%
\pgfsetdash{}{0pt}%
\pgfpathmoveto{\pgfqpoint{1.874506in}{4.119504in}}%
\pgfpathlineto{\pgfqpoint{1.824498in}{4.108336in}}%
\pgfusepath{stroke}%
\end{pgfscope}%
\begin{pgfscope}%
\pgfpathrectangle{\pgfqpoint{0.800000in}{3.252941in}}{\pgfqpoint{2.407767in}{1.544118in}}%
\pgfusepath{clip}%
\pgfsetbuttcap%
\pgfsetroundjoin%
\pgfsetlinewidth{0.501875pt}%
\definecolor{currentstroke}{rgb}{0.525776,0.833491,0.288127}%
\pgfsetstrokecolor{currentstroke}%
\pgfsetdash{}{0pt}%
\pgfpathmoveto{\pgfqpoint{1.824498in}{4.108336in}}%
\pgfpathlineto{\pgfqpoint{1.774273in}{4.097582in}}%
\pgfusepath{stroke}%
\end{pgfscope}%
\begin{pgfscope}%
\pgfpathrectangle{\pgfqpoint{0.800000in}{3.252941in}}{\pgfqpoint{2.407767in}{1.544118in}}%
\pgfusepath{clip}%
\pgfsetbuttcap%
\pgfsetroundjoin%
\pgfsetlinewidth{0.501875pt}%
\definecolor{currentstroke}{rgb}{0.273809,0.031497,0.358853}%
\pgfsetstrokecolor{currentstroke}%
\pgfsetdash{}{0pt}%
\pgfpathmoveto{\pgfqpoint{2.654046in}{4.233477in}}%
\pgfpathlineto{\pgfqpoint{2.601082in}{4.232741in}}%
\pgfusepath{stroke}%
\end{pgfscope}%
\begin{pgfscope}%
\pgfpathrectangle{\pgfqpoint{0.800000in}{3.252941in}}{\pgfqpoint{2.407767in}{1.544118in}}%
\pgfusepath{clip}%
\pgfsetbuttcap%
\pgfsetroundjoin%
\pgfsetlinewidth{0.501875pt}%
\definecolor{currentstroke}{rgb}{0.280894,0.078907,0.402329}%
\pgfsetstrokecolor{currentstroke}%
\pgfsetdash{}{0pt}%
\pgfpathmoveto{\pgfqpoint{2.601082in}{4.232741in}}%
\pgfpathlineto{\pgfqpoint{2.548128in}{4.231784in}}%
\pgfusepath{stroke}%
\end{pgfscope}%
\begin{pgfscope}%
\pgfpathrectangle{\pgfqpoint{0.800000in}{3.252941in}}{\pgfqpoint{2.407767in}{1.544118in}}%
\pgfusepath{clip}%
\pgfsetbuttcap%
\pgfsetroundjoin%
\pgfsetlinewidth{0.501875pt}%
\definecolor{currentstroke}{rgb}{0.283229,0.120777,0.440584}%
\pgfsetstrokecolor{currentstroke}%
\pgfsetdash{}{0pt}%
\pgfpathmoveto{\pgfqpoint{2.548128in}{4.231784in}}%
\pgfpathlineto{\pgfqpoint{2.495192in}{4.230487in}}%
\pgfusepath{stroke}%
\end{pgfscope}%
\begin{pgfscope}%
\pgfpathrectangle{\pgfqpoint{0.800000in}{3.252941in}}{\pgfqpoint{2.407767in}{1.544118in}}%
\pgfusepath{clip}%
\pgfsetbuttcap%
\pgfsetroundjoin%
\pgfsetlinewidth{0.501875pt}%
\definecolor{currentstroke}{rgb}{0.282290,0.145912,0.461510}%
\pgfsetstrokecolor{currentstroke}%
\pgfsetdash{}{0pt}%
\pgfpathmoveto{\pgfqpoint{2.495192in}{4.230487in}}%
\pgfpathlineto{\pgfqpoint{2.442279in}{4.228861in}}%
\pgfusepath{stroke}%
\end{pgfscope}%
\begin{pgfscope}%
\pgfpathrectangle{\pgfqpoint{0.800000in}{3.252941in}}{\pgfqpoint{2.407767in}{1.544118in}}%
\pgfusepath{clip}%
\pgfsetbuttcap%
\pgfsetroundjoin%
\pgfsetlinewidth{0.501875pt}%
\definecolor{currentstroke}{rgb}{0.278826,0.175490,0.483397}%
\pgfsetstrokecolor{currentstroke}%
\pgfsetdash{}{0pt}%
\pgfpathmoveto{\pgfqpoint{2.442279in}{4.228861in}}%
\pgfpathlineto{\pgfqpoint{2.389406in}{4.226780in}}%
\pgfusepath{stroke}%
\end{pgfscope}%
\begin{pgfscope}%
\pgfpathrectangle{\pgfqpoint{0.800000in}{3.252941in}}{\pgfqpoint{2.407767in}{1.544118in}}%
\pgfusepath{clip}%
\pgfsetbuttcap%
\pgfsetroundjoin%
\pgfsetlinewidth{0.501875pt}%
\definecolor{currentstroke}{rgb}{0.278012,0.180367,0.486697}%
\pgfsetstrokecolor{currentstroke}%
\pgfsetdash{}{0pt}%
\pgfpathmoveto{\pgfqpoint{2.389406in}{4.226780in}}%
\pgfpathlineto{\pgfqpoint{2.336621in}{4.223941in}}%
\pgfusepath{stroke}%
\end{pgfscope}%
\begin{pgfscope}%
\pgfpathrectangle{\pgfqpoint{0.800000in}{3.252941in}}{\pgfqpoint{2.407767in}{1.544118in}}%
\pgfusepath{clip}%
\pgfsetbuttcap%
\pgfsetroundjoin%
\pgfsetlinewidth{0.501875pt}%
\definecolor{currentstroke}{rgb}{0.277134,0.185228,0.489898}%
\pgfsetstrokecolor{currentstroke}%
\pgfsetdash{}{0pt}%
\pgfpathmoveto{\pgfqpoint{2.336621in}{4.223941in}}%
\pgfpathlineto{\pgfqpoint{2.284008in}{4.220033in}}%
\pgfusepath{stroke}%
\end{pgfscope}%
\begin{pgfscope}%
\pgfpathrectangle{\pgfqpoint{0.800000in}{3.252941in}}{\pgfqpoint{2.407767in}{1.544118in}}%
\pgfusepath{clip}%
\pgfsetbuttcap%
\pgfsetroundjoin%
\pgfsetlinewidth{0.501875pt}%
\definecolor{currentstroke}{rgb}{0.272594,0.025563,0.353093}%
\pgfsetstrokecolor{currentstroke}%
\pgfsetdash{}{0pt}%
\pgfpathmoveto{\pgfqpoint{2.654046in}{4.268223in}}%
\pgfpathlineto{\pgfqpoint{2.601077in}{4.267687in}}%
\pgfusepath{stroke}%
\end{pgfscope}%
\begin{pgfscope}%
\pgfpathrectangle{\pgfqpoint{0.800000in}{3.252941in}}{\pgfqpoint{2.407767in}{1.544118in}}%
\pgfusepath{clip}%
\pgfsetbuttcap%
\pgfsetroundjoin%
\pgfsetlinewidth{0.501875pt}%
\definecolor{currentstroke}{rgb}{0.278791,0.062145,0.386592}%
\pgfsetstrokecolor{currentstroke}%
\pgfsetdash{}{0pt}%
\pgfpathmoveto{\pgfqpoint{2.601077in}{4.267687in}}%
\pgfpathlineto{\pgfqpoint{2.548125in}{4.266727in}}%
\pgfusepath{stroke}%
\end{pgfscope}%
\begin{pgfscope}%
\pgfpathrectangle{\pgfqpoint{0.800000in}{3.252941in}}{\pgfqpoint{2.407767in}{1.544118in}}%
\pgfusepath{clip}%
\pgfsetbuttcap%
\pgfsetroundjoin%
\pgfsetlinewidth{0.501875pt}%
\definecolor{currentstroke}{rgb}{0.282327,0.094955,0.417331}%
\pgfsetstrokecolor{currentstroke}%
\pgfsetdash{}{0pt}%
\pgfpathmoveto{\pgfqpoint{2.548125in}{4.266727in}}%
\pgfpathlineto{\pgfqpoint{2.495200in}{4.265265in}}%
\pgfusepath{stroke}%
\end{pgfscope}%
\begin{pgfscope}%
\pgfpathrectangle{\pgfqpoint{0.800000in}{3.252941in}}{\pgfqpoint{2.407767in}{1.544118in}}%
\pgfusepath{clip}%
\pgfsetbuttcap%
\pgfsetroundjoin%
\pgfsetlinewidth{0.501875pt}%
\definecolor{currentstroke}{rgb}{0.283072,0.130895,0.449241}%
\pgfsetstrokecolor{currentstroke}%
\pgfsetdash{}{0pt}%
\pgfpathmoveto{\pgfqpoint{2.495200in}{4.265265in}}%
\pgfpathlineto{\pgfqpoint{2.442315in}{4.263317in}}%
\pgfusepath{stroke}%
\end{pgfscope}%
\begin{pgfscope}%
\pgfpathrectangle{\pgfqpoint{0.800000in}{3.252941in}}{\pgfqpoint{2.407767in}{1.544118in}}%
\pgfusepath{clip}%
\pgfsetbuttcap%
\pgfsetroundjoin%
\pgfsetlinewidth{0.501875pt}%
\definecolor{currentstroke}{rgb}{0.282290,0.145912,0.461510}%
\pgfsetstrokecolor{currentstroke}%
\pgfsetdash{}{0pt}%
\pgfpathmoveto{\pgfqpoint{2.442315in}{4.263317in}}%
\pgfpathlineto{\pgfqpoint{2.389532in}{4.260501in}}%
\pgfusepath{stroke}%
\end{pgfscope}%
\begin{pgfscope}%
\pgfpathrectangle{\pgfqpoint{0.800000in}{3.252941in}}{\pgfqpoint{2.407767in}{1.544118in}}%
\pgfusepath{clip}%
\pgfsetbuttcap%
\pgfsetroundjoin%
\pgfsetlinewidth{0.501875pt}%
\definecolor{currentstroke}{rgb}{0.283072,0.130895,0.449241}%
\pgfsetstrokecolor{currentstroke}%
\pgfsetdash{}{0pt}%
\pgfpathmoveto{\pgfqpoint{2.389532in}{4.260501in}}%
\pgfpathlineto{\pgfqpoint{2.336849in}{4.256957in}}%
\pgfusepath{stroke}%
\end{pgfscope}%
\begin{pgfscope}%
\pgfpathrectangle{\pgfqpoint{0.800000in}{3.252941in}}{\pgfqpoint{2.407767in}{1.544118in}}%
\pgfusepath{clip}%
\pgfsetbuttcap%
\pgfsetroundjoin%
\pgfsetlinewidth{0.501875pt}%
\definecolor{currentstroke}{rgb}{0.280868,0.160771,0.472899}%
\pgfsetstrokecolor{currentstroke}%
\pgfsetdash{}{0pt}%
\pgfpathmoveto{\pgfqpoint{2.336849in}{4.256957in}}%
\pgfpathlineto{\pgfqpoint{2.284269in}{4.252837in}}%
\pgfusepath{stroke}%
\end{pgfscope}%
\begin{pgfscope}%
\pgfpathrectangle{\pgfqpoint{0.800000in}{3.252941in}}{\pgfqpoint{2.407767in}{1.544118in}}%
\pgfusepath{clip}%
\pgfsetbuttcap%
\pgfsetroundjoin%
\pgfsetlinewidth{0.501875pt}%
\definecolor{currentstroke}{rgb}{0.269308,0.218818,0.509577}%
\pgfsetstrokecolor{currentstroke}%
\pgfsetdash{}{0pt}%
\pgfpathmoveto{\pgfqpoint{2.284269in}{4.252837in}}%
\pgfpathlineto{\pgfqpoint{2.231945in}{4.247582in}}%
\pgfusepath{stroke}%
\end{pgfscope}%
\begin{pgfscope}%
\pgfpathrectangle{\pgfqpoint{0.800000in}{3.252941in}}{\pgfqpoint{2.407767in}{1.544118in}}%
\pgfusepath{clip}%
\pgfsetbuttcap%
\pgfsetroundjoin%
\pgfsetlinewidth{0.501875pt}%
\definecolor{currentstroke}{rgb}{0.275191,0.194905,0.496005}%
\pgfsetstrokecolor{currentstroke}%
\pgfsetdash{}{0pt}%
\pgfpathmoveto{\pgfqpoint{2.231945in}{4.247582in}}%
\pgfpathlineto{\pgfqpoint{2.180108in}{4.240667in}}%
\pgfusepath{stroke}%
\end{pgfscope}%
\begin{pgfscope}%
\pgfpathrectangle{\pgfqpoint{0.800000in}{3.252941in}}{\pgfqpoint{2.407767in}{1.544118in}}%
\pgfusepath{clip}%
\pgfsetbuttcap%
\pgfsetroundjoin%
\pgfsetlinewidth{0.501875pt}%
\definecolor{currentstroke}{rgb}{0.267968,0.223549,0.512008}%
\pgfsetstrokecolor{currentstroke}%
\pgfsetdash{}{0pt}%
\pgfpathmoveto{\pgfqpoint{2.180108in}{4.240667in}}%
\pgfpathlineto{\pgfqpoint{2.129002in}{4.231813in}}%
\pgfusepath{stroke}%
\end{pgfscope}%
\begin{pgfscope}%
\pgfpathrectangle{\pgfqpoint{0.800000in}{3.252941in}}{\pgfqpoint{2.407767in}{1.544118in}}%
\pgfusepath{clip}%
\pgfsetbuttcap%
\pgfsetroundjoin%
\pgfsetlinewidth{0.501875pt}%
\definecolor{currentstroke}{rgb}{0.276194,0.190074,0.493001}%
\pgfsetstrokecolor{currentstroke}%
\pgfsetdash{}{0pt}%
\pgfpathmoveto{\pgfqpoint{2.129002in}{4.231813in}}%
\pgfpathlineto{\pgfqpoint{2.078942in}{4.220790in}}%
\pgfusepath{stroke}%
\end{pgfscope}%
\begin{pgfscope}%
\pgfpathrectangle{\pgfqpoint{0.800000in}{3.252941in}}{\pgfqpoint{2.407767in}{1.544118in}}%
\pgfusepath{clip}%
\pgfsetbuttcap%
\pgfsetroundjoin%
\pgfsetlinewidth{0.501875pt}%
\definecolor{currentstroke}{rgb}{0.263663,0.237631,0.518762}%
\pgfsetstrokecolor{currentstroke}%
\pgfsetdash{}{0pt}%
\pgfpathmoveto{\pgfqpoint{2.078942in}{4.220790in}}%
\pgfpathlineto{\pgfqpoint{2.030756in}{4.206839in}}%
\pgfusepath{stroke}%
\end{pgfscope}%
\begin{pgfscope}%
\pgfpathrectangle{\pgfqpoint{0.800000in}{3.252941in}}{\pgfqpoint{2.407767in}{1.544118in}}%
\pgfusepath{clip}%
\pgfsetbuttcap%
\pgfsetroundjoin%
\pgfsetlinewidth{0.501875pt}%
\definecolor{currentstroke}{rgb}{0.260571,0.246922,0.522828}%
\pgfsetstrokecolor{currentstroke}%
\pgfsetdash{}{0pt}%
\pgfpathmoveto{\pgfqpoint{2.030756in}{4.206839in}}%
\pgfpathlineto{\pgfqpoint{1.983383in}{4.191703in}}%
\pgfusepath{stroke}%
\end{pgfscope}%
\begin{pgfscope}%
\pgfpathrectangle{\pgfqpoint{0.800000in}{3.252941in}}{\pgfqpoint{2.407767in}{1.544118in}}%
\pgfusepath{clip}%
\pgfsetbuttcap%
\pgfsetroundjoin%
\pgfsetlinewidth{0.501875pt}%
\definecolor{currentstroke}{rgb}{0.225863,0.330805,0.547314}%
\pgfsetstrokecolor{currentstroke}%
\pgfsetdash{}{0pt}%
\pgfpathmoveto{\pgfqpoint{1.983383in}{4.191703in}}%
\pgfpathlineto{\pgfqpoint{1.936008in}{4.176521in}}%
\pgfusepath{stroke}%
\end{pgfscope}%
\begin{pgfscope}%
\pgfpathrectangle{\pgfqpoint{0.800000in}{3.252941in}}{\pgfqpoint{2.407767in}{1.544118in}}%
\pgfusepath{clip}%
\pgfsetbuttcap%
\pgfsetroundjoin%
\pgfsetlinewidth{0.501875pt}%
\definecolor{currentstroke}{rgb}{0.132268,0.655014,0.519661}%
\pgfsetstrokecolor{currentstroke}%
\pgfsetdash{}{0pt}%
\pgfpathmoveto{\pgfqpoint{1.936008in}{4.176521in}}%
\pgfpathlineto{\pgfqpoint{1.888602in}{4.161367in}}%
\pgfusepath{stroke}%
\end{pgfscope}%
\begin{pgfscope}%
\pgfpathrectangle{\pgfqpoint{0.800000in}{3.252941in}}{\pgfqpoint{2.407767in}{1.544118in}}%
\pgfusepath{clip}%
\pgfsetbuttcap%
\pgfsetroundjoin%
\pgfsetlinewidth{0.501875pt}%
\definecolor{currentstroke}{rgb}{0.344074,0.780029,0.397381}%
\pgfsetstrokecolor{currentstroke}%
\pgfsetdash{}{0pt}%
\pgfpathmoveto{\pgfqpoint{1.888602in}{4.161367in}}%
\pgfpathlineto{\pgfqpoint{1.841291in}{4.146093in}}%
\pgfusepath{stroke}%
\end{pgfscope}%
\begin{pgfscope}%
\pgfpathrectangle{\pgfqpoint{0.800000in}{3.252941in}}{\pgfqpoint{2.407767in}{1.544118in}}%
\pgfusepath{clip}%
\pgfsetbuttcap%
\pgfsetroundjoin%
\pgfsetlinewidth{0.501875pt}%
\definecolor{currentstroke}{rgb}{0.377779,0.791781,0.377939}%
\pgfsetstrokecolor{currentstroke}%
\pgfsetdash{}{0pt}%
\pgfpathmoveto{\pgfqpoint{1.841291in}{4.146093in}}%
\pgfpathlineto{\pgfqpoint{1.793772in}{4.131091in}}%
\pgfusepath{stroke}%
\end{pgfscope}%
\begin{pgfscope}%
\pgfpathrectangle{\pgfqpoint{0.800000in}{3.252941in}}{\pgfqpoint{2.407767in}{1.544118in}}%
\pgfusepath{clip}%
\pgfsetbuttcap%
\pgfsetroundjoin%
\pgfsetlinewidth{0.501875pt}%
\definecolor{currentstroke}{rgb}{0.272594,0.025563,0.353093}%
\pgfsetstrokecolor{currentstroke}%
\pgfsetdash{}{0pt}%
\pgfpathmoveto{\pgfqpoint{2.654046in}{4.302969in}}%
\pgfpathlineto{\pgfqpoint{2.601085in}{4.302357in}}%
\pgfusepath{stroke}%
\end{pgfscope}%
\begin{pgfscope}%
\pgfpathrectangle{\pgfqpoint{0.800000in}{3.252941in}}{\pgfqpoint{2.407767in}{1.544118in}}%
\pgfusepath{clip}%
\pgfsetbuttcap%
\pgfsetroundjoin%
\pgfsetlinewidth{0.501875pt}%
\definecolor{currentstroke}{rgb}{0.277018,0.050344,0.375715}%
\pgfsetstrokecolor{currentstroke}%
\pgfsetdash{}{0pt}%
\pgfpathmoveto{\pgfqpoint{2.601085in}{4.302357in}}%
\pgfpathlineto{\pgfqpoint{2.548140in}{4.301190in}}%
\pgfusepath{stroke}%
\end{pgfscope}%
\begin{pgfscope}%
\pgfpathrectangle{\pgfqpoint{0.800000in}{3.252941in}}{\pgfqpoint{2.407767in}{1.544118in}}%
\pgfusepath{clip}%
\pgfsetbuttcap%
\pgfsetroundjoin%
\pgfsetlinewidth{0.501875pt}%
\definecolor{currentstroke}{rgb}{0.279566,0.067836,0.391917}%
\pgfsetstrokecolor{currentstroke}%
\pgfsetdash{}{0pt}%
\pgfpathmoveto{\pgfqpoint{2.548140in}{4.301190in}}%
\pgfpathlineto{\pgfqpoint{2.495231in}{4.299528in}}%
\pgfusepath{stroke}%
\end{pgfscope}%
\begin{pgfscope}%
\pgfpathrectangle{\pgfqpoint{0.800000in}{3.252941in}}{\pgfqpoint{2.407767in}{1.544118in}}%
\pgfusepath{clip}%
\pgfsetbuttcap%
\pgfsetroundjoin%
\pgfsetlinewidth{0.501875pt}%
\definecolor{currentstroke}{rgb}{0.282656,0.100196,0.422160}%
\pgfsetstrokecolor{currentstroke}%
\pgfsetdash{}{0pt}%
\pgfpathmoveto{\pgfqpoint{2.495231in}{4.299528in}}%
\pgfpathlineto{\pgfqpoint{2.442368in}{4.297345in}}%
\pgfusepath{stroke}%
\end{pgfscope}%
\begin{pgfscope}%
\pgfpathrectangle{\pgfqpoint{0.800000in}{3.252941in}}{\pgfqpoint{2.407767in}{1.544118in}}%
\pgfusepath{clip}%
\pgfsetbuttcap%
\pgfsetroundjoin%
\pgfsetlinewidth{0.501875pt}%
\definecolor{currentstroke}{rgb}{0.283187,0.125848,0.444960}%
\pgfsetstrokecolor{currentstroke}%
\pgfsetdash{}{0pt}%
\pgfpathmoveto{\pgfqpoint{2.442368in}{4.297345in}}%
\pgfpathlineto{\pgfqpoint{2.389575in}{4.294582in}}%
\pgfusepath{stroke}%
\end{pgfscope}%
\begin{pgfscope}%
\pgfpathrectangle{\pgfqpoint{0.800000in}{3.252941in}}{\pgfqpoint{2.407767in}{1.544118in}}%
\pgfusepath{clip}%
\pgfsetbuttcap%
\pgfsetroundjoin%
\pgfsetlinewidth{0.501875pt}%
\definecolor{currentstroke}{rgb}{0.283187,0.125848,0.444960}%
\pgfsetstrokecolor{currentstroke}%
\pgfsetdash{}{0pt}%
\pgfpathmoveto{\pgfqpoint{2.389575in}{4.294582in}}%
\pgfpathlineto{\pgfqpoint{2.336959in}{4.290696in}}%
\pgfusepath{stroke}%
\end{pgfscope}%
\begin{pgfscope}%
\pgfpathrectangle{\pgfqpoint{0.800000in}{3.252941in}}{\pgfqpoint{2.407767in}{1.544118in}}%
\pgfusepath{clip}%
\pgfsetbuttcap%
\pgfsetroundjoin%
\pgfsetlinewidth{0.501875pt}%
\definecolor{currentstroke}{rgb}{0.281887,0.150881,0.465405}%
\pgfsetstrokecolor{currentstroke}%
\pgfsetdash{}{0pt}%
\pgfpathmoveto{\pgfqpoint{2.336959in}{4.290696in}}%
\pgfpathlineto{\pgfqpoint{2.284517in}{4.285910in}}%
\pgfusepath{stroke}%
\end{pgfscope}%
\begin{pgfscope}%
\pgfpathrectangle{\pgfqpoint{0.800000in}{3.252941in}}{\pgfqpoint{2.407767in}{1.544118in}}%
\pgfusepath{clip}%
\pgfsetbuttcap%
\pgfsetroundjoin%
\pgfsetlinewidth{0.501875pt}%
\definecolor{currentstroke}{rgb}{0.283229,0.120777,0.440584}%
\pgfsetstrokecolor{currentstroke}%
\pgfsetdash{}{0pt}%
\pgfpathmoveto{\pgfqpoint{2.284517in}{4.285910in}}%
\pgfpathlineto{\pgfqpoint{2.232381in}{4.279985in}}%
\pgfusepath{stroke}%
\end{pgfscope}%
\begin{pgfscope}%
\pgfpathrectangle{\pgfqpoint{0.800000in}{3.252941in}}{\pgfqpoint{2.407767in}{1.544118in}}%
\pgfusepath{clip}%
\pgfsetbuttcap%
\pgfsetroundjoin%
\pgfsetlinewidth{0.501875pt}%
\definecolor{currentstroke}{rgb}{0.271305,0.019942,0.347269}%
\pgfsetstrokecolor{currentstroke}%
\pgfsetdash{}{0pt}%
\pgfpathmoveto{\pgfqpoint{2.654046in}{4.337715in}}%
\pgfpathlineto{\pgfqpoint{2.601077in}{4.337260in}}%
\pgfusepath{stroke}%
\end{pgfscope}%
\begin{pgfscope}%
\pgfpathrectangle{\pgfqpoint{0.800000in}{3.252941in}}{\pgfqpoint{2.407767in}{1.544118in}}%
\pgfusepath{clip}%
\pgfsetbuttcap%
\pgfsetroundjoin%
\pgfsetlinewidth{0.501875pt}%
\definecolor{currentstroke}{rgb}{0.277018,0.050344,0.375715}%
\pgfsetstrokecolor{currentstroke}%
\pgfsetdash{}{0pt}%
\pgfpathmoveto{\pgfqpoint{2.601077in}{4.337260in}}%
\pgfpathlineto{\pgfqpoint{2.548133in}{4.336214in}}%
\pgfusepath{stroke}%
\end{pgfscope}%
\begin{pgfscope}%
\pgfpathrectangle{\pgfqpoint{0.800000in}{3.252941in}}{\pgfqpoint{2.407767in}{1.544118in}}%
\pgfusepath{clip}%
\pgfsetbuttcap%
\pgfsetroundjoin%
\pgfsetlinewidth{0.501875pt}%
\definecolor{currentstroke}{rgb}{0.280267,0.073417,0.397163}%
\pgfsetstrokecolor{currentstroke}%
\pgfsetdash{}{0pt}%
\pgfpathmoveto{\pgfqpoint{2.548133in}{4.336214in}}%
\pgfpathlineto{\pgfqpoint{2.495238in}{4.334465in}}%
\pgfusepath{stroke}%
\end{pgfscope}%
\begin{pgfscope}%
\pgfpathrectangle{\pgfqpoint{0.800000in}{3.252941in}}{\pgfqpoint{2.407767in}{1.544118in}}%
\pgfusepath{clip}%
\pgfsetbuttcap%
\pgfsetroundjoin%
\pgfsetlinewidth{0.501875pt}%
\definecolor{currentstroke}{rgb}{0.281924,0.089666,0.412415}%
\pgfsetstrokecolor{currentstroke}%
\pgfsetdash{}{0pt}%
\pgfpathmoveto{\pgfqpoint{2.495238in}{4.334465in}}%
\pgfpathlineto{\pgfqpoint{2.442389in}{4.332175in}}%
\pgfusepath{stroke}%
\end{pgfscope}%
\begin{pgfscope}%
\pgfpathrectangle{\pgfqpoint{0.800000in}{3.252941in}}{\pgfqpoint{2.407767in}{1.544118in}}%
\pgfusepath{clip}%
\pgfsetbuttcap%
\pgfsetroundjoin%
\pgfsetlinewidth{0.501875pt}%
\definecolor{currentstroke}{rgb}{0.282656,0.100196,0.422160}%
\pgfsetstrokecolor{currentstroke}%
\pgfsetdash{}{0pt}%
\pgfpathmoveto{\pgfqpoint{2.442389in}{4.332175in}}%
\pgfpathlineto{\pgfqpoint{2.389651in}{4.329050in}}%
\pgfusepath{stroke}%
\end{pgfscope}%
\begin{pgfscope}%
\pgfpathrectangle{\pgfqpoint{0.800000in}{3.252941in}}{\pgfqpoint{2.407767in}{1.544118in}}%
\pgfusepath{clip}%
\pgfsetbuttcap%
\pgfsetroundjoin%
\pgfsetlinewidth{0.501875pt}%
\definecolor{currentstroke}{rgb}{0.282910,0.105393,0.426902}%
\pgfsetstrokecolor{currentstroke}%
\pgfsetdash{}{0pt}%
\pgfpathmoveto{\pgfqpoint{2.389651in}{4.329050in}}%
\pgfpathlineto{\pgfqpoint{2.337148in}{4.324596in}}%
\pgfusepath{stroke}%
\end{pgfscope}%
\begin{pgfscope}%
\pgfpathrectangle{\pgfqpoint{0.800000in}{3.252941in}}{\pgfqpoint{2.407767in}{1.544118in}}%
\pgfusepath{clip}%
\pgfsetbuttcap%
\pgfsetroundjoin%
\pgfsetlinewidth{0.501875pt}%
\definecolor{currentstroke}{rgb}{0.271305,0.019942,0.347269}%
\pgfsetstrokecolor{currentstroke}%
\pgfsetdash{}{0pt}%
\pgfpathmoveto{\pgfqpoint{2.654046in}{4.372461in}}%
\pgfpathlineto{\pgfqpoint{2.601082in}{4.372025in}}%
\pgfusepath{stroke}%
\end{pgfscope}%
\begin{pgfscope}%
\pgfpathrectangle{\pgfqpoint{0.800000in}{3.252941in}}{\pgfqpoint{2.407767in}{1.544118in}}%
\pgfusepath{clip}%
\pgfsetbuttcap%
\pgfsetroundjoin%
\pgfsetlinewidth{0.501875pt}%
\definecolor{currentstroke}{rgb}{0.273809,0.031497,0.358853}%
\pgfsetstrokecolor{currentstroke}%
\pgfsetdash{}{0pt}%
\pgfpathmoveto{\pgfqpoint{2.601082in}{4.372025in}}%
\pgfpathlineto{\pgfqpoint{2.548145in}{4.370800in}}%
\pgfusepath{stroke}%
\end{pgfscope}%
\begin{pgfscope}%
\pgfpathrectangle{\pgfqpoint{0.800000in}{3.252941in}}{\pgfqpoint{2.407767in}{1.544118in}}%
\pgfusepath{clip}%
\pgfsetbuttcap%
\pgfsetroundjoin%
\pgfsetlinewidth{0.501875pt}%
\definecolor{currentstroke}{rgb}{0.277941,0.056324,0.381191}%
\pgfsetstrokecolor{currentstroke}%
\pgfsetdash{}{0pt}%
\pgfpathmoveto{\pgfqpoint{2.548145in}{4.370800in}}%
\pgfpathlineto{\pgfqpoint{2.495263in}{4.368867in}}%
\pgfusepath{stroke}%
\end{pgfscope}%
\begin{pgfscope}%
\pgfpathrectangle{\pgfqpoint{0.800000in}{3.252941in}}{\pgfqpoint{2.407767in}{1.544118in}}%
\pgfusepath{clip}%
\pgfsetbuttcap%
\pgfsetroundjoin%
\pgfsetlinewidth{0.501875pt}%
\definecolor{currentstroke}{rgb}{0.280267,0.073417,0.397163}%
\pgfsetstrokecolor{currentstroke}%
\pgfsetdash{}{0pt}%
\pgfpathmoveto{\pgfqpoint{2.495263in}{4.368867in}}%
\pgfpathlineto{\pgfqpoint{2.442454in}{4.366270in}}%
\pgfusepath{stroke}%
\end{pgfscope}%
\begin{pgfscope}%
\pgfpathrectangle{\pgfqpoint{0.800000in}{3.252941in}}{\pgfqpoint{2.407767in}{1.544118in}}%
\pgfusepath{clip}%
\pgfsetbuttcap%
\pgfsetroundjoin%
\pgfsetlinewidth{0.501875pt}%
\definecolor{currentstroke}{rgb}{0.280894,0.078907,0.402329}%
\pgfsetstrokecolor{currentstroke}%
\pgfsetdash{}{0pt}%
\pgfpathmoveto{\pgfqpoint{2.442454in}{4.366270in}}%
\pgfpathlineto{\pgfqpoint{2.389752in}{4.362967in}}%
\pgfusepath{stroke}%
\end{pgfscope}%
\begin{pgfscope}%
\pgfpathrectangle{\pgfqpoint{0.800000in}{3.252941in}}{\pgfqpoint{2.407767in}{1.544118in}}%
\pgfusepath{clip}%
\pgfsetbuttcap%
\pgfsetroundjoin%
\pgfsetlinewidth{0.501875pt}%
\definecolor{currentstroke}{rgb}{0.281446,0.084320,0.407414}%
\pgfsetstrokecolor{currentstroke}%
\pgfsetdash{}{0pt}%
\pgfpathmoveto{\pgfqpoint{2.389752in}{4.362967in}}%
\pgfpathlineto{\pgfqpoint{2.337136in}{4.359122in}}%
\pgfusepath{stroke}%
\end{pgfscope}%
\begin{pgfscope}%
\pgfpathrectangle{\pgfqpoint{0.800000in}{3.252941in}}{\pgfqpoint{2.407767in}{1.544118in}}%
\pgfusepath{clip}%
\pgfsetbuttcap%
\pgfsetroundjoin%
\pgfsetlinewidth{0.501875pt}%
\definecolor{currentstroke}{rgb}{0.279566,0.067836,0.391917}%
\pgfsetstrokecolor{currentstroke}%
\pgfsetdash{}{0pt}%
\pgfpathmoveto{\pgfqpoint{2.337136in}{4.359122in}}%
\pgfpathlineto{\pgfqpoint{2.284690in}{4.354457in}}%
\pgfusepath{stroke}%
\end{pgfscope}%
\begin{pgfscope}%
\pgfpathrectangle{\pgfqpoint{0.800000in}{3.252941in}}{\pgfqpoint{2.407767in}{1.544118in}}%
\pgfusepath{clip}%
\pgfsetbuttcap%
\pgfsetroundjoin%
\pgfsetlinewidth{0.501875pt}%
\definecolor{currentstroke}{rgb}{0.281446,0.084320,0.407414}%
\pgfsetstrokecolor{currentstroke}%
\pgfsetdash{}{0pt}%
\pgfpathmoveto{\pgfqpoint{2.284690in}{4.354457in}}%
\pgfpathlineto{\pgfqpoint{2.233072in}{4.347274in}}%
\pgfusepath{stroke}%
\end{pgfscope}%
\begin{pgfscope}%
\pgfpathrectangle{\pgfqpoint{0.800000in}{3.252941in}}{\pgfqpoint{2.407767in}{1.544118in}}%
\pgfusepath{clip}%
\pgfsetbuttcap%
\pgfsetroundjoin%
\pgfsetlinewidth{0.501875pt}%
\definecolor{currentstroke}{rgb}{0.282910,0.105393,0.426902}%
\pgfsetstrokecolor{currentstroke}%
\pgfsetdash{}{0pt}%
\pgfpathmoveto{\pgfqpoint{2.233072in}{4.347274in}}%
\pgfpathlineto{\pgfqpoint{2.182010in}{4.338467in}}%
\pgfusepath{stroke}%
\end{pgfscope}%
\begin{pgfscope}%
\pgfpathrectangle{\pgfqpoint{0.800000in}{3.252941in}}{\pgfqpoint{2.407767in}{1.544118in}}%
\pgfusepath{clip}%
\pgfsetbuttcap%
\pgfsetroundjoin%
\pgfsetlinewidth{0.501875pt}%
\definecolor{currentstroke}{rgb}{0.280894,0.078907,0.402329}%
\pgfsetstrokecolor{currentstroke}%
\pgfsetdash{}{0pt}%
\pgfpathmoveto{\pgfqpoint{2.182010in}{4.338467in}}%
\pgfpathlineto{\pgfqpoint{2.132460in}{4.326965in}}%
\pgfusepath{stroke}%
\end{pgfscope}%
\begin{pgfscope}%
\pgfpathrectangle{\pgfqpoint{0.800000in}{3.252941in}}{\pgfqpoint{2.407767in}{1.544118in}}%
\pgfusepath{clip}%
\pgfsetbuttcap%
\pgfsetroundjoin%
\pgfsetlinewidth{0.501875pt}%
\definecolor{currentstroke}{rgb}{0.282656,0.100196,0.422160}%
\pgfsetstrokecolor{currentstroke}%
\pgfsetdash{}{0pt}%
\pgfpathmoveto{\pgfqpoint{2.132460in}{4.326965in}}%
\pgfpathlineto{\pgfqpoint{2.085562in}{4.311587in}}%
\pgfusepath{stroke}%
\end{pgfscope}%
\begin{pgfscope}%
\pgfpathrectangle{\pgfqpoint{0.800000in}{3.252941in}}{\pgfqpoint{2.407767in}{1.544118in}}%
\pgfusepath{clip}%
\pgfsetbuttcap%
\pgfsetroundjoin%
\pgfsetlinewidth{0.501875pt}%
\definecolor{currentstroke}{rgb}{0.269944,0.014625,0.341379}%
\pgfsetstrokecolor{currentstroke}%
\pgfsetdash{}{0pt}%
\pgfpathmoveto{\pgfqpoint{2.654046in}{4.407207in}}%
\pgfpathlineto{\pgfqpoint{2.601128in}{4.405648in}}%
\pgfusepath{stroke}%
\end{pgfscope}%
\begin{pgfscope}%
\pgfpathrectangle{\pgfqpoint{0.800000in}{3.252941in}}{\pgfqpoint{2.407767in}{1.544118in}}%
\pgfusepath{clip}%
\pgfsetbuttcap%
\pgfsetroundjoin%
\pgfsetlinewidth{0.501875pt}%
\definecolor{currentstroke}{rgb}{0.273809,0.031497,0.358853}%
\pgfsetstrokecolor{currentstroke}%
\pgfsetdash{}{0pt}%
\pgfpathmoveto{\pgfqpoint{2.601128in}{4.405648in}}%
\pgfpathlineto{\pgfqpoint{2.548204in}{4.404179in}}%
\pgfusepath{stroke}%
\end{pgfscope}%
\begin{pgfscope}%
\pgfpathrectangle{\pgfqpoint{0.800000in}{3.252941in}}{\pgfqpoint{2.407767in}{1.544118in}}%
\pgfusepath{clip}%
\pgfsetbuttcap%
\pgfsetroundjoin%
\pgfsetlinewidth{0.501875pt}%
\definecolor{currentstroke}{rgb}{0.274952,0.037752,0.364543}%
\pgfsetstrokecolor{currentstroke}%
\pgfsetdash{}{0pt}%
\pgfpathmoveto{\pgfqpoint{2.548204in}{4.404179in}}%
\pgfpathlineto{\pgfqpoint{2.495318in}{4.402307in}}%
\pgfusepath{stroke}%
\end{pgfscope}%
\begin{pgfscope}%
\pgfpathrectangle{\pgfqpoint{0.800000in}{3.252941in}}{\pgfqpoint{2.407767in}{1.544118in}}%
\pgfusepath{clip}%
\pgfsetbuttcap%
\pgfsetroundjoin%
\pgfsetlinewidth{0.501875pt}%
\definecolor{currentstroke}{rgb}{0.277941,0.056324,0.381191}%
\pgfsetstrokecolor{currentstroke}%
\pgfsetdash{}{0pt}%
\pgfpathmoveto{\pgfqpoint{2.495318in}{4.402307in}}%
\pgfpathlineto{\pgfqpoint{2.442517in}{4.399629in}}%
\pgfusepath{stroke}%
\end{pgfscope}%
\begin{pgfscope}%
\pgfpathrectangle{\pgfqpoint{0.800000in}{3.252941in}}{\pgfqpoint{2.407767in}{1.544118in}}%
\pgfusepath{clip}%
\pgfsetbuttcap%
\pgfsetroundjoin%
\pgfsetlinewidth{0.501875pt}%
\definecolor{currentstroke}{rgb}{0.276022,0.044167,0.370164}%
\pgfsetstrokecolor{currentstroke}%
\pgfsetdash{}{0pt}%
\pgfpathmoveto{\pgfqpoint{2.442517in}{4.399629in}}%
\pgfpathlineto{\pgfqpoint{2.389842in}{4.396115in}}%
\pgfusepath{stroke}%
\end{pgfscope}%
\begin{pgfscope}%
\pgfpathrectangle{\pgfqpoint{0.800000in}{3.252941in}}{\pgfqpoint{2.407767in}{1.544118in}}%
\pgfusepath{clip}%
\pgfsetbuttcap%
\pgfsetroundjoin%
\pgfsetlinewidth{0.501875pt}%
\definecolor{currentstroke}{rgb}{0.281446,0.084320,0.407414}%
\pgfsetstrokecolor{currentstroke}%
\pgfsetdash{}{0pt}%
\pgfpathmoveto{\pgfqpoint{2.389842in}{4.396115in}}%
\pgfpathlineto{\pgfqpoint{2.337531in}{4.390906in}}%
\pgfusepath{stroke}%
\end{pgfscope}%
\begin{pgfscope}%
\pgfpathrectangle{\pgfqpoint{0.800000in}{3.252941in}}{\pgfqpoint{2.407767in}{1.544118in}}%
\pgfusepath{clip}%
\pgfsetbuttcap%
\pgfsetroundjoin%
\pgfsetlinewidth{0.501875pt}%
\definecolor{currentstroke}{rgb}{0.276022,0.044167,0.370164}%
\pgfsetstrokecolor{currentstroke}%
\pgfsetdash{}{0pt}%
\pgfpathmoveto{\pgfqpoint{2.337531in}{4.390906in}}%
\pgfpathlineto{\pgfqpoint{2.285527in}{4.384472in}}%
\pgfusepath{stroke}%
\end{pgfscope}%
\begin{pgfscope}%
\pgfpathrectangle{\pgfqpoint{0.800000in}{3.252941in}}{\pgfqpoint{2.407767in}{1.544118in}}%
\pgfusepath{clip}%
\pgfsetbuttcap%
\pgfsetroundjoin%
\pgfsetlinewidth{0.501875pt}%
\definecolor{currentstroke}{rgb}{0.280894,0.078907,0.402329}%
\pgfsetstrokecolor{currentstroke}%
\pgfsetdash{}{0pt}%
\pgfpathmoveto{\pgfqpoint{2.285527in}{4.384472in}}%
\pgfpathlineto{\pgfqpoint{2.233738in}{4.377530in}}%
\pgfusepath{stroke}%
\end{pgfscope}%
\begin{pgfscope}%
\pgfpathrectangle{\pgfqpoint{0.800000in}{3.252941in}}{\pgfqpoint{2.407767in}{1.544118in}}%
\pgfusepath{clip}%
\pgfsetbuttcap%
\pgfsetroundjoin%
\pgfsetlinewidth{0.501875pt}%
\definecolor{currentstroke}{rgb}{0.281446,0.084320,0.407414}%
\pgfsetstrokecolor{currentstroke}%
\pgfsetdash{}{0pt}%
\pgfpathmoveto{\pgfqpoint{2.233738in}{4.377530in}}%
\pgfpathlineto{\pgfqpoint{2.183076in}{4.368365in}}%
\pgfusepath{stroke}%
\end{pgfscope}%
\begin{pgfscope}%
\pgfpathrectangle{\pgfqpoint{0.800000in}{3.252941in}}{\pgfqpoint{2.407767in}{1.544118in}}%
\pgfusepath{clip}%
\pgfsetbuttcap%
\pgfsetroundjoin%
\pgfsetlinewidth{0.501875pt}%
\definecolor{currentstroke}{rgb}{0.269944,0.014625,0.341379}%
\pgfsetstrokecolor{currentstroke}%
\pgfsetdash{}{0pt}%
\pgfpathmoveto{\pgfqpoint{2.654046in}{4.441953in}}%
\pgfpathlineto{\pgfqpoint{2.601100in}{4.441617in}}%
\pgfusepath{stroke}%
\end{pgfscope}%
\begin{pgfscope}%
\pgfpathrectangle{\pgfqpoint{0.800000in}{3.252941in}}{\pgfqpoint{2.407767in}{1.544118in}}%
\pgfusepath{clip}%
\pgfsetbuttcap%
\pgfsetroundjoin%
\pgfsetlinewidth{0.501875pt}%
\definecolor{currentstroke}{rgb}{0.272594,0.025563,0.353093}%
\pgfsetstrokecolor{currentstroke}%
\pgfsetdash{}{0pt}%
\pgfpathmoveto{\pgfqpoint{2.601100in}{4.441617in}}%
\pgfpathlineto{\pgfqpoint{2.548168in}{4.440876in}}%
\pgfusepath{stroke}%
\end{pgfscope}%
\begin{pgfscope}%
\pgfpathrectangle{\pgfqpoint{0.800000in}{3.252941in}}{\pgfqpoint{2.407767in}{1.544118in}}%
\pgfusepath{clip}%
\pgfsetbuttcap%
\pgfsetroundjoin%
\pgfsetlinewidth{0.501875pt}%
\definecolor{currentstroke}{rgb}{0.274952,0.037752,0.364543}%
\pgfsetstrokecolor{currentstroke}%
\pgfsetdash{}{0pt}%
\pgfpathmoveto{\pgfqpoint{2.548168in}{4.440876in}}%
\pgfpathlineto{\pgfqpoint{2.495272in}{4.439214in}}%
\pgfusepath{stroke}%
\end{pgfscope}%
\begin{pgfscope}%
\pgfpathrectangle{\pgfqpoint{0.800000in}{3.252941in}}{\pgfqpoint{2.407767in}{1.544118in}}%
\pgfusepath{clip}%
\pgfsetbuttcap%
\pgfsetroundjoin%
\pgfsetlinewidth{0.501875pt}%
\definecolor{currentstroke}{rgb}{0.277018,0.050344,0.375715}%
\pgfsetstrokecolor{currentstroke}%
\pgfsetdash{}{0pt}%
\pgfpathmoveto{\pgfqpoint{2.495272in}{4.439214in}}%
\pgfpathlineto{\pgfqpoint{2.442443in}{4.436861in}}%
\pgfusepath{stroke}%
\end{pgfscope}%
\begin{pgfscope}%
\pgfpathrectangle{\pgfqpoint{0.800000in}{3.252941in}}{\pgfqpoint{2.407767in}{1.544118in}}%
\pgfusepath{clip}%
\pgfsetbuttcap%
\pgfsetroundjoin%
\pgfsetlinewidth{0.501875pt}%
\definecolor{currentstroke}{rgb}{0.277018,0.050344,0.375715}%
\pgfsetstrokecolor{currentstroke}%
\pgfsetdash{}{0pt}%
\pgfpathmoveto{\pgfqpoint{2.442443in}{4.436861in}}%
\pgfpathlineto{\pgfqpoint{2.389738in}{4.433553in}}%
\pgfusepath{stroke}%
\end{pgfscope}%
\begin{pgfscope}%
\pgfpathrectangle{\pgfqpoint{0.800000in}{3.252941in}}{\pgfqpoint{2.407767in}{1.544118in}}%
\pgfusepath{clip}%
\pgfsetbuttcap%
\pgfsetroundjoin%
\pgfsetlinewidth{0.501875pt}%
\definecolor{currentstroke}{rgb}{0.277941,0.056324,0.381191}%
\pgfsetstrokecolor{currentstroke}%
\pgfsetdash{}{0pt}%
\pgfpathmoveto{\pgfqpoint{2.058064in}{4.372461in}}%
\pgfpathlineto{\pgfqpoint{2.021576in}{4.348561in}}%
\pgfusepath{stroke}%
\end{pgfscope}%
\begin{pgfscope}%
\pgfpathrectangle{\pgfqpoint{0.800000in}{3.252941in}}{\pgfqpoint{2.407767in}{1.544118in}}%
\pgfusepath{clip}%
\pgfsetbuttcap%
\pgfsetroundjoin%
\pgfsetlinewidth{0.501875pt}%
\definecolor{currentstroke}{rgb}{0.280894,0.078907,0.402329}%
\pgfsetstrokecolor{currentstroke}%
\pgfsetdash{}{0pt}%
\pgfpathmoveto{\pgfqpoint{2.021576in}{4.348561in}}%
\pgfpathlineto{\pgfqpoint{1.986997in}{4.323549in}}%
\pgfusepath{stroke}%
\end{pgfscope}%
\begin{pgfscope}%
\pgfpathrectangle{\pgfqpoint{0.800000in}{3.252941in}}{\pgfqpoint{2.407767in}{1.544118in}}%
\pgfusepath{clip}%
\pgfsetbuttcap%
\pgfsetroundjoin%
\pgfsetlinewidth{0.501875pt}%
\definecolor{currentstroke}{rgb}{0.283091,0.110553,0.431554}%
\pgfsetstrokecolor{currentstroke}%
\pgfsetdash{}{0pt}%
\pgfpathmoveto{\pgfqpoint{1.986997in}{4.323549in}}%
\pgfpathlineto{\pgfqpoint{1.961366in}{4.294479in}}%
\pgfusepath{stroke}%
\end{pgfscope}%
\begin{pgfscope}%
\pgfpathrectangle{\pgfqpoint{0.800000in}{3.252941in}}{\pgfqpoint{2.407767in}{1.544118in}}%
\pgfusepath{clip}%
\pgfsetbuttcap%
\pgfsetroundjoin%
\pgfsetlinewidth{0.501875pt}%
\definecolor{currentstroke}{rgb}{0.283187,0.125848,0.444960}%
\pgfsetstrokecolor{currentstroke}%
\pgfsetdash{}{0pt}%
\pgfpathmoveto{\pgfqpoint{1.961366in}{4.294479in}}%
\pgfpathlineto{\pgfqpoint{1.946225in}{4.269338in}}%
\pgfusepath{stroke}%
\end{pgfscope}%
\begin{pgfscope}%
\pgfpathrectangle{\pgfqpoint{0.800000in}{3.252941in}}{\pgfqpoint{2.407767in}{1.544118in}}%
\pgfusepath{clip}%
\pgfsetbuttcap%
\pgfsetroundjoin%
\pgfsetlinewidth{0.501875pt}%
\definecolor{currentstroke}{rgb}{0.276022,0.044167,0.370164}%
\pgfsetstrokecolor{currentstroke}%
\pgfsetdash{}{0pt}%
\pgfpathmoveto{\pgfqpoint{1.946225in}{4.269338in}}%
\pgfpathlineto{\pgfqpoint{1.946225in}{4.269338in}}%
\pgfusepath{stroke}%
\end{pgfscope}%
\begin{pgfscope}%
\pgfpathrectangle{\pgfqpoint{0.800000in}{3.252941in}}{\pgfqpoint{2.407767in}{1.544118in}}%
\pgfusepath{clip}%
\pgfsetbuttcap%
\pgfsetroundjoin%
\pgfsetlinewidth{0.501875pt}%
\definecolor{currentstroke}{rgb}{0.276022,0.044167,0.370164}%
\pgfsetstrokecolor{currentstroke}%
\pgfsetdash{}{0pt}%
\pgfpathmoveto{\pgfqpoint{1.946225in}{4.269338in}}%
\pgfpathlineto{\pgfqpoint{1.946225in}{4.269338in}}%
\pgfusepath{stroke}%
\end{pgfscope}%
\begin{pgfscope}%
\pgfpathrectangle{\pgfqpoint{0.800000in}{3.252941in}}{\pgfqpoint{2.407767in}{1.544118in}}%
\pgfusepath{clip}%
\pgfsetbuttcap%
\pgfsetroundjoin%
\pgfsetlinewidth{0.501875pt}%
\definecolor{currentstroke}{rgb}{0.276022,0.044167,0.370164}%
\pgfsetstrokecolor{currentstroke}%
\pgfsetdash{}{0pt}%
\pgfpathmoveto{\pgfqpoint{1.946225in}{4.269338in}}%
\pgfpathlineto{\pgfqpoint{1.946225in}{4.269338in}}%
\pgfusepath{stroke}%
\end{pgfscope}%
\begin{pgfscope}%
\pgfpathrectangle{\pgfqpoint{0.800000in}{3.252941in}}{\pgfqpoint{2.407767in}{1.544118in}}%
\pgfusepath{clip}%
\pgfsetbuttcap%
\pgfsetroundjoin%
\pgfsetlinewidth{0.501875pt}%
\definecolor{currentstroke}{rgb}{0.276022,0.044167,0.370164}%
\pgfsetstrokecolor{currentstroke}%
\pgfsetdash{}{0pt}%
\pgfpathmoveto{\pgfqpoint{1.946225in}{4.269338in}}%
\pgfpathlineto{\pgfqpoint{1.943163in}{4.262196in}}%
\pgfusepath{stroke}%
\end{pgfscope}%
\begin{pgfscope}%
\pgfpathrectangle{\pgfqpoint{0.800000in}{3.252941in}}{\pgfqpoint{2.407767in}{1.544118in}}%
\pgfusepath{clip}%
\pgfsetbuttcap%
\pgfsetroundjoin%
\pgfsetlinewidth{0.501875pt}%
\definecolor{currentstroke}{rgb}{0.282884,0.135920,0.453427}%
\pgfsetstrokecolor{currentstroke}%
\pgfsetdash{}{0pt}%
\pgfpathmoveto{\pgfqpoint{1.943163in}{4.262196in}}%
\pgfpathlineto{\pgfqpoint{1.935436in}{4.255737in}}%
\pgfusepath{stroke}%
\end{pgfscope}%
\begin{pgfscope}%
\pgfpathrectangle{\pgfqpoint{0.800000in}{3.252941in}}{\pgfqpoint{2.407767in}{1.544118in}}%
\pgfusepath{clip}%
\pgfsetbuttcap%
\pgfsetroundjoin%
\pgfsetlinewidth{0.501875pt}%
\definecolor{currentstroke}{rgb}{0.276194,0.190074,0.493001}%
\pgfsetstrokecolor{currentstroke}%
\pgfsetdash{}{0pt}%
\pgfpathmoveto{\pgfqpoint{1.935436in}{4.255737in}}%
\pgfpathlineto{\pgfqpoint{1.935436in}{4.255737in}}%
\pgfusepath{stroke}%
\end{pgfscope}%
\begin{pgfscope}%
\pgfpathrectangle{\pgfqpoint{0.800000in}{3.252941in}}{\pgfqpoint{2.407767in}{1.544118in}}%
\pgfusepath{clip}%
\pgfsetbuttcap%
\pgfsetroundjoin%
\pgfsetlinewidth{0.501875pt}%
\definecolor{currentstroke}{rgb}{0.276194,0.190074,0.493001}%
\pgfsetstrokecolor{currentstroke}%
\pgfsetdash{}{0pt}%
\pgfpathmoveto{\pgfqpoint{1.935436in}{4.255737in}}%
\pgfpathlineto{\pgfqpoint{1.916225in}{4.231934in}}%
\pgfusepath{stroke}%
\end{pgfscope}%
\begin{pgfscope}%
\pgfpathrectangle{\pgfqpoint{0.800000in}{3.252941in}}{\pgfqpoint{2.407767in}{1.544118in}}%
\pgfusepath{clip}%
\pgfsetbuttcap%
\pgfsetroundjoin%
\pgfsetlinewidth{0.501875pt}%
\definecolor{currentstroke}{rgb}{0.279574,0.170599,0.479997}%
\pgfsetstrokecolor{currentstroke}%
\pgfsetdash{}{0pt}%
\pgfpathmoveto{\pgfqpoint{1.916225in}{4.231934in}}%
\pgfpathlineto{\pgfqpoint{1.916225in}{4.231934in}}%
\pgfusepath{stroke}%
\end{pgfscope}%
\begin{pgfscope}%
\pgfpathrectangle{\pgfqpoint{0.800000in}{3.252941in}}{\pgfqpoint{2.407767in}{1.544118in}}%
\pgfusepath{clip}%
\pgfsetbuttcap%
\pgfsetroundjoin%
\pgfsetlinewidth{0.501875pt}%
\definecolor{currentstroke}{rgb}{0.279574,0.170599,0.479997}%
\pgfsetstrokecolor{currentstroke}%
\pgfsetdash{}{0pt}%
\pgfpathmoveto{\pgfqpoint{1.916225in}{4.231934in}}%
\pgfpathlineto{\pgfqpoint{1.900344in}{4.217082in}}%
\pgfusepath{stroke}%
\end{pgfscope}%
\begin{pgfscope}%
\pgfpathrectangle{\pgfqpoint{0.800000in}{3.252941in}}{\pgfqpoint{2.407767in}{1.544118in}}%
\pgfusepath{clip}%
\pgfsetbuttcap%
\pgfsetroundjoin%
\pgfsetlinewidth{0.501875pt}%
\definecolor{currentstroke}{rgb}{0.229739,0.322361,0.545706}%
\pgfsetstrokecolor{currentstroke}%
\pgfsetdash{}{0pt}%
\pgfpathmoveto{\pgfqpoint{1.900344in}{4.217082in}}%
\pgfpathlineto{\pgfqpoint{1.879404in}{4.205312in}}%
\pgfusepath{stroke}%
\end{pgfscope}%
\begin{pgfscope}%
\pgfpathrectangle{\pgfqpoint{0.800000in}{3.252941in}}{\pgfqpoint{2.407767in}{1.544118in}}%
\pgfusepath{clip}%
\pgfsetbuttcap%
\pgfsetroundjoin%
\pgfsetlinewidth{0.501875pt}%
\definecolor{currentstroke}{rgb}{0.147607,0.511733,0.557049}%
\pgfsetstrokecolor{currentstroke}%
\pgfsetdash{}{0pt}%
\pgfpathmoveto{\pgfqpoint{1.879404in}{4.205312in}}%
\pgfpathlineto{\pgfqpoint{1.841112in}{4.186711in}}%
\pgfusepath{stroke}%
\end{pgfscope}%
\begin{pgfscope}%
\pgfpathrectangle{\pgfqpoint{0.800000in}{3.252941in}}{\pgfqpoint{2.407767in}{1.544118in}}%
\pgfusepath{clip}%
\pgfsetbuttcap%
\pgfsetroundjoin%
\pgfsetlinewidth{0.501875pt}%
\definecolor{currentstroke}{rgb}{0.130067,0.651384,0.521608}%
\pgfsetstrokecolor{currentstroke}%
\pgfsetdash{}{0pt}%
\pgfpathmoveto{\pgfqpoint{1.841112in}{4.186711in}}%
\pgfpathlineto{\pgfqpoint{1.798133in}{4.166885in}}%
\pgfusepath{stroke}%
\end{pgfscope}%
\begin{pgfscope}%
\pgfpathrectangle{\pgfqpoint{0.800000in}{3.252941in}}{\pgfqpoint{2.407767in}{1.544118in}}%
\pgfusepath{clip}%
\pgfsetbuttcap%
\pgfsetroundjoin%
\pgfsetlinewidth{0.501875pt}%
\definecolor{currentstroke}{rgb}{0.278791,0.062145,0.386592}%
\pgfsetstrokecolor{currentstroke}%
\pgfsetdash{}{0pt}%
\pgfpathmoveto{\pgfqpoint{1.831191in}{3.731280in}}%
\pgfpathlineto{\pgfqpoint{1.844804in}{3.738719in}}%
\pgfusepath{stroke}%
\end{pgfscope}%
\begin{pgfscope}%
\pgfpathrectangle{\pgfqpoint{0.800000in}{3.252941in}}{\pgfqpoint{2.407767in}{1.544118in}}%
\pgfusepath{clip}%
\pgfsetbuttcap%
\pgfsetroundjoin%
\pgfsetlinewidth{0.501875pt}%
\definecolor{currentstroke}{rgb}{0.283197,0.115680,0.436115}%
\pgfsetstrokecolor{currentstroke}%
\pgfsetdash{}{0pt}%
\pgfpathmoveto{\pgfqpoint{1.844804in}{3.738719in}}%
\pgfpathlineto{\pgfqpoint{1.856638in}{3.749810in}}%
\pgfusepath{stroke}%
\end{pgfscope}%
\begin{pgfscope}%
\pgfpathrectangle{\pgfqpoint{0.800000in}{3.252941in}}{\pgfqpoint{2.407767in}{1.544118in}}%
\pgfusepath{clip}%
\pgfsetbuttcap%
\pgfsetroundjoin%
\pgfsetlinewidth{0.501875pt}%
\definecolor{currentstroke}{rgb}{0.278826,0.175490,0.483397}%
\pgfsetstrokecolor{currentstroke}%
\pgfsetdash{}{0pt}%
\pgfpathmoveto{\pgfqpoint{1.856638in}{3.749810in}}%
\pgfpathlineto{\pgfqpoint{1.856638in}{3.749810in}}%
\pgfusepath{stroke}%
\end{pgfscope}%
\begin{pgfscope}%
\pgfpathrectangle{\pgfqpoint{0.800000in}{3.252941in}}{\pgfqpoint{2.407767in}{1.544118in}}%
\pgfusepath{clip}%
\pgfsetbuttcap%
\pgfsetroundjoin%
\pgfsetlinewidth{0.501875pt}%
\definecolor{currentstroke}{rgb}{0.278826,0.175490,0.483397}%
\pgfsetstrokecolor{currentstroke}%
\pgfsetdash{}{0pt}%
\pgfpathmoveto{\pgfqpoint{1.856638in}{3.749810in}}%
\pgfpathlineto{\pgfqpoint{1.856638in}{3.749810in}}%
\pgfusepath{stroke}%
\end{pgfscope}%
\begin{pgfscope}%
\pgfpathrectangle{\pgfqpoint{0.800000in}{3.252941in}}{\pgfqpoint{2.407767in}{1.544118in}}%
\pgfusepath{clip}%
\pgfsetbuttcap%
\pgfsetroundjoin%
\pgfsetlinewidth{0.501875pt}%
\definecolor{currentstroke}{rgb}{0.278826,0.175490,0.483397}%
\pgfsetstrokecolor{currentstroke}%
\pgfsetdash{}{0pt}%
\pgfpathmoveto{\pgfqpoint{1.856638in}{3.749810in}}%
\pgfpathlineto{\pgfqpoint{1.861599in}{3.771719in}}%
\pgfusepath{stroke}%
\end{pgfscope}%
\begin{pgfscope}%
\pgfpathrectangle{\pgfqpoint{0.800000in}{3.252941in}}{\pgfqpoint{2.407767in}{1.544118in}}%
\pgfusepath{clip}%
\pgfsetbuttcap%
\pgfsetroundjoin%
\pgfsetlinewidth{0.501875pt}%
\definecolor{currentstroke}{rgb}{0.275191,0.194905,0.496005}%
\pgfsetstrokecolor{currentstroke}%
\pgfsetdash{}{0pt}%
\pgfpathmoveto{\pgfqpoint{1.861599in}{3.771719in}}%
\pgfpathlineto{\pgfqpoint{1.857220in}{3.794820in}}%
\pgfusepath{stroke}%
\end{pgfscope}%
\begin{pgfscope}%
\pgfpathrectangle{\pgfqpoint{0.800000in}{3.252941in}}{\pgfqpoint{2.407767in}{1.544118in}}%
\pgfusepath{clip}%
\pgfsetbuttcap%
\pgfsetroundjoin%
\pgfsetlinewidth{0.501875pt}%
\definecolor{currentstroke}{rgb}{0.280255,0.165693,0.476498}%
\pgfsetstrokecolor{currentstroke}%
\pgfsetdash{}{0pt}%
\pgfpathmoveto{\pgfqpoint{1.857220in}{3.794820in}}%
\pgfpathlineto{\pgfqpoint{1.841343in}{3.816523in}}%
\pgfusepath{stroke}%
\end{pgfscope}%
\begin{pgfscope}%
\pgfpathrectangle{\pgfqpoint{0.800000in}{3.252941in}}{\pgfqpoint{2.407767in}{1.544118in}}%
\pgfusepath{clip}%
\pgfsetbuttcap%
\pgfsetroundjoin%
\pgfsetlinewidth{0.501875pt}%
\definecolor{currentstroke}{rgb}{0.220057,0.343307,0.549413}%
\pgfsetstrokecolor{currentstroke}%
\pgfsetdash{}{0pt}%
\pgfpathmoveto{\pgfqpoint{1.841343in}{3.816523in}}%
\pgfpathlineto{\pgfqpoint{1.841343in}{3.816523in}}%
\pgfusepath{stroke}%
\end{pgfscope}%
\begin{pgfscope}%
\pgfpathrectangle{\pgfqpoint{0.800000in}{3.252941in}}{\pgfqpoint{2.407767in}{1.544118in}}%
\pgfusepath{clip}%
\pgfsetbuttcap%
\pgfsetroundjoin%
\pgfsetlinewidth{0.501875pt}%
\definecolor{currentstroke}{rgb}{0.220057,0.343307,0.549413}%
\pgfsetstrokecolor{currentstroke}%
\pgfsetdash{}{0pt}%
\pgfpathmoveto{\pgfqpoint{1.841343in}{3.816523in}}%
\pgfpathlineto{\pgfqpoint{1.809044in}{3.842140in}}%
\pgfusepath{stroke}%
\end{pgfscope}%
\begin{pgfscope}%
\pgfpathrectangle{\pgfqpoint{0.800000in}{3.252941in}}{\pgfqpoint{2.407767in}{1.544118in}}%
\pgfusepath{clip}%
\pgfsetbuttcap%
\pgfsetroundjoin%
\pgfsetlinewidth{0.501875pt}%
\definecolor{currentstroke}{rgb}{0.188923,0.410910,0.556326}%
\pgfsetstrokecolor{currentstroke}%
\pgfsetdash{}{0pt}%
\pgfpathmoveto{\pgfqpoint{1.809044in}{3.842140in}}%
\pgfpathlineto{\pgfqpoint{1.771163in}{3.865792in}}%
\pgfusepath{stroke}%
\end{pgfscope}%
\begin{pgfscope}%
\pgfpathrectangle{\pgfqpoint{0.800000in}{3.252941in}}{\pgfqpoint{2.407767in}{1.544118in}}%
\pgfusepath{clip}%
\pgfsetroundcap%
\pgfsetroundjoin%
\pgfsetlinewidth{0.501875pt}%
\definecolor{currentstroke}{rgb}{0.276022,0.044167,0.370164}%
\pgfsetstrokecolor{currentstroke}%
\pgfsetdash{}{0pt}%
\pgfpathmoveto{\pgfqpoint{2.488567in}{3.567571in}}%
\pgfpathquadraticcurveto{\pgfqpoint{2.475363in}{3.568170in}}{\pgfqpoint{2.469916in}{3.568417in}}%
\pgfusepath{stroke}%
\end{pgfscope}%
\begin{pgfscope}%
\pgfpathrectangle{\pgfqpoint{0.800000in}{3.252941in}}{\pgfqpoint{2.407767in}{1.544118in}}%
\pgfusepath{clip}%
\pgfsetroundcap%
\pgfsetroundjoin%
\definecolor{currentfill}{rgb}{0.276022,0.044167,0.370164}%
\pgfsetfillcolor{currentfill}%
\pgfsetlinewidth{0.501875pt}%
\definecolor{currentstroke}{rgb}{0.276022,0.044167,0.370164}%
\pgfsetstrokecolor{currentstroke}%
\pgfsetdash{}{0pt}%
\pgfpathmoveto{\pgfqpoint{2.497035in}{3.553284in}}%
\pgfpathlineto{\pgfqpoint{2.469916in}{3.568417in}}%
\pgfpathlineto{\pgfqpoint{2.498294in}{3.581033in}}%
\pgfpathlineto{\pgfqpoint{2.497035in}{3.553284in}}%
\pgfpathlineto{\pgfqpoint{2.497035in}{3.553284in}}%
\pgfpathclose%
\pgfusepath{stroke,fill}%
\end{pgfscope}%
\begin{pgfscope}%
\pgfpathrectangle{\pgfqpoint{0.800000in}{3.252941in}}{\pgfqpoint{2.407767in}{1.544118in}}%
\pgfusepath{clip}%
\pgfsetroundcap%
\pgfsetroundjoin%
\pgfsetlinewidth{0.501875pt}%
\definecolor{currentstroke}{rgb}{0.272594,0.025563,0.353093}%
\pgfsetstrokecolor{currentstroke}%
\pgfsetdash{}{0pt}%
\pgfpathmoveto{\pgfqpoint{2.548151in}{4.475532in}}%
\pgfpathquadraticcurveto{\pgfqpoint{2.534937in}{4.475134in}}{\pgfqpoint{2.529485in}{4.474970in}}%
\pgfusepath{stroke}%
\end{pgfscope}%
\begin{pgfscope}%
\pgfpathrectangle{\pgfqpoint{0.800000in}{3.252941in}}{\pgfqpoint{2.407767in}{1.544118in}}%
\pgfusepath{clip}%
\pgfsetroundcap%
\pgfsetroundjoin%
\definecolor{currentfill}{rgb}{0.272594,0.025563,0.353093}%
\pgfsetfillcolor{currentfill}%
\pgfsetlinewidth{0.501875pt}%
\definecolor{currentstroke}{rgb}{0.272594,0.025563,0.353093}%
\pgfsetstrokecolor{currentstroke}%
\pgfsetdash{}{0pt}%
\pgfpathmoveto{\pgfqpoint{2.557668in}{4.461923in}}%
\pgfpathlineto{\pgfqpoint{2.529485in}{4.474970in}}%
\pgfpathlineto{\pgfqpoint{2.556832in}{4.489688in}}%
\pgfpathlineto{\pgfqpoint{2.557668in}{4.461923in}}%
\pgfpathlineto{\pgfqpoint{2.557668in}{4.461923in}}%
\pgfpathclose%
\pgfusepath{stroke,fill}%
\end{pgfscope}%
\begin{pgfscope}%
\pgfpathrectangle{\pgfqpoint{0.800000in}{3.252941in}}{\pgfqpoint{2.407767in}{1.544118in}}%
\pgfusepath{clip}%
\pgfsetroundcap%
\pgfsetroundjoin%
\pgfsetlinewidth{0.501875pt}%
\definecolor{currentstroke}{rgb}{0.278791,0.062145,0.386592}%
\pgfsetstrokecolor{currentstroke}%
\pgfsetdash{}{0pt}%
\pgfpathmoveto{\pgfqpoint{2.379421in}{3.597903in}}%
\pgfpathquadraticcurveto{\pgfqpoint{2.366312in}{3.599054in}}{\pgfqpoint{2.360938in}{3.599526in}}%
\pgfusepath{stroke}%
\end{pgfscope}%
\begin{pgfscope}%
\pgfpathrectangle{\pgfqpoint{0.800000in}{3.252941in}}{\pgfqpoint{2.407767in}{1.544118in}}%
\pgfusepath{clip}%
\pgfsetroundcap%
\pgfsetroundjoin%
\definecolor{currentfill}{rgb}{0.278791,0.062145,0.386592}%
\pgfsetfillcolor{currentfill}%
\pgfsetlinewidth{0.501875pt}%
\definecolor{currentstroke}{rgb}{0.278791,0.062145,0.386592}%
\pgfsetstrokecolor{currentstroke}%
\pgfsetdash{}{0pt}%
\pgfpathmoveto{\pgfqpoint{2.387394in}{3.583261in}}%
\pgfpathlineto{\pgfqpoint{2.360938in}{3.599526in}}%
\pgfpathlineto{\pgfqpoint{2.389824in}{3.610932in}}%
\pgfpathlineto{\pgfqpoint{2.387394in}{3.583261in}}%
\pgfpathlineto{\pgfqpoint{2.387394in}{3.583261in}}%
\pgfpathclose%
\pgfusepath{stroke,fill}%
\end{pgfscope}%
\begin{pgfscope}%
\pgfpathrectangle{\pgfqpoint{0.800000in}{3.252941in}}{\pgfqpoint{2.407767in}{1.544118in}}%
\pgfusepath{clip}%
\pgfsetroundcap%
\pgfsetroundjoin%
\pgfsetlinewidth{0.501875pt}%
\definecolor{currentstroke}{rgb}{0.277018,0.050344,0.375715}%
\pgfsetstrokecolor{currentstroke}%
\pgfsetdash{}{0pt}%
\pgfpathmoveto{\pgfqpoint{2.495305in}{3.647371in}}%
\pgfpathquadraticcurveto{\pgfqpoint{2.482105in}{3.648035in}}{\pgfqpoint{2.476660in}{3.648309in}}%
\pgfusepath{stroke}%
\end{pgfscope}%
\begin{pgfscope}%
\pgfpathrectangle{\pgfqpoint{0.800000in}{3.252941in}}{\pgfqpoint{2.407767in}{1.544118in}}%
\pgfusepath{clip}%
\pgfsetroundcap%
\pgfsetroundjoin%
\definecolor{currentfill}{rgb}{0.277018,0.050344,0.375715}%
\pgfsetfillcolor{currentfill}%
\pgfsetlinewidth{0.501875pt}%
\definecolor{currentstroke}{rgb}{0.277018,0.050344,0.375715}%
\pgfsetstrokecolor{currentstroke}%
\pgfsetdash{}{0pt}%
\pgfpathmoveto{\pgfqpoint{2.503705in}{3.633041in}}%
\pgfpathlineto{\pgfqpoint{2.476660in}{3.648309in}}%
\pgfpathlineto{\pgfqpoint{2.505101in}{3.660784in}}%
\pgfpathlineto{\pgfqpoint{2.503705in}{3.633041in}}%
\pgfpathlineto{\pgfqpoint{2.503705in}{3.633041in}}%
\pgfpathclose%
\pgfusepath{stroke,fill}%
\end{pgfscope}%
\begin{pgfscope}%
\pgfpathrectangle{\pgfqpoint{0.800000in}{3.252941in}}{\pgfqpoint{2.407767in}{1.544118in}}%
\pgfusepath{clip}%
\pgfsetroundcap%
\pgfsetroundjoin%
\pgfsetlinewidth{0.501875pt}%
\definecolor{currentstroke}{rgb}{0.281924,0.089666,0.412415}%
\pgfsetstrokecolor{currentstroke}%
\pgfsetdash{}{0pt}%
\pgfpathmoveto{\pgfqpoint{2.084748in}{3.737717in}}%
\pgfpathquadraticcurveto{\pgfqpoint{2.073418in}{3.742022in}}{\pgfqpoint{2.069345in}{3.743569in}}%
\pgfusepath{stroke}%
\end{pgfscope}%
\begin{pgfscope}%
\pgfpathrectangle{\pgfqpoint{0.800000in}{3.252941in}}{\pgfqpoint{2.407767in}{1.544118in}}%
\pgfusepath{clip}%
\pgfsetroundcap%
\pgfsetroundjoin%
\definecolor{currentfill}{rgb}{0.281924,0.089666,0.412415}%
\pgfsetfillcolor{currentfill}%
\pgfsetlinewidth{0.501875pt}%
\definecolor{currentstroke}{rgb}{0.281924,0.089666,0.412415}%
\pgfsetstrokecolor{currentstroke}%
\pgfsetdash{}{0pt}%
\pgfpathmoveto{\pgfqpoint{2.090380in}{3.720720in}}%
\pgfpathlineto{\pgfqpoint{2.069345in}{3.743569in}}%
\pgfpathlineto{\pgfqpoint{2.100245in}{3.746687in}}%
\pgfpathlineto{\pgfqpoint{2.090380in}{3.720720in}}%
\pgfpathlineto{\pgfqpoint{2.090380in}{3.720720in}}%
\pgfpathclose%
\pgfusepath{stroke,fill}%
\end{pgfscope}%
\begin{pgfscope}%
\pgfpathrectangle{\pgfqpoint{0.800000in}{3.252941in}}{\pgfqpoint{2.407767in}{1.544118in}}%
\pgfusepath{clip}%
\pgfsetroundcap%
\pgfsetroundjoin%
\pgfsetlinewidth{0.501875pt}%
\definecolor{currentstroke}{rgb}{0.283072,0.130895,0.449241}%
\pgfsetstrokecolor{currentstroke}%
\pgfsetdash{}{0pt}%
\pgfpathmoveto{\pgfqpoint{2.389680in}{3.721700in}}%
\pgfpathquadraticcurveto{\pgfqpoint{2.376542in}{3.722762in}}{\pgfqpoint{2.371142in}{3.723198in}}%
\pgfusepath{stroke}%
\end{pgfscope}%
\begin{pgfscope}%
\pgfpathrectangle{\pgfqpoint{0.800000in}{3.252941in}}{\pgfqpoint{2.407767in}{1.544118in}}%
\pgfusepath{clip}%
\pgfsetroundcap%
\pgfsetroundjoin%
\definecolor{currentfill}{rgb}{0.283072,0.130895,0.449241}%
\pgfsetfillcolor{currentfill}%
\pgfsetlinewidth{0.501875pt}%
\definecolor{currentstroke}{rgb}{0.283072,0.130895,0.449241}%
\pgfsetstrokecolor{currentstroke}%
\pgfsetdash{}{0pt}%
\pgfpathmoveto{\pgfqpoint{2.397711in}{3.707117in}}%
\pgfpathlineto{\pgfqpoint{2.371142in}{3.723198in}}%
\pgfpathlineto{\pgfqpoint{2.399948in}{3.734805in}}%
\pgfpathlineto{\pgfqpoint{2.397711in}{3.707117in}}%
\pgfpathlineto{\pgfqpoint{2.397711in}{3.707117in}}%
\pgfpathclose%
\pgfusepath{stroke,fill}%
\end{pgfscope}%
\begin{pgfscope}%
\pgfpathrectangle{\pgfqpoint{0.800000in}{3.252941in}}{\pgfqpoint{2.407767in}{1.544118in}}%
\pgfusepath{clip}%
\pgfsetroundcap%
\pgfsetroundjoin%
\pgfsetlinewidth{0.501875pt}%
\definecolor{currentstroke}{rgb}{0.283091,0.110553,0.431554}%
\pgfsetstrokecolor{currentstroke}%
\pgfsetdash{}{0pt}%
\pgfpathmoveto{\pgfqpoint{2.495197in}{3.749950in}}%
\pgfpathquadraticcurveto{\pgfqpoint{2.481981in}{3.750494in}}{\pgfqpoint{2.476523in}{3.750719in}}%
\pgfusepath{stroke}%
\end{pgfscope}%
\begin{pgfscope}%
\pgfpathrectangle{\pgfqpoint{0.800000in}{3.252941in}}{\pgfqpoint{2.407767in}{1.544118in}}%
\pgfusepath{clip}%
\pgfsetroundcap%
\pgfsetroundjoin%
\definecolor{currentfill}{rgb}{0.283091,0.110553,0.431554}%
\pgfsetfillcolor{currentfill}%
\pgfsetlinewidth{0.501875pt}%
\definecolor{currentstroke}{rgb}{0.283091,0.110553,0.431554}%
\pgfsetstrokecolor{currentstroke}%
\pgfsetdash{}{0pt}%
\pgfpathmoveto{\pgfqpoint{2.503706in}{3.735698in}}%
\pgfpathlineto{\pgfqpoint{2.476523in}{3.750719in}}%
\pgfpathlineto{\pgfqpoint{2.504849in}{3.763453in}}%
\pgfpathlineto{\pgfqpoint{2.503706in}{3.735698in}}%
\pgfpathlineto{\pgfqpoint{2.503706in}{3.735698in}}%
\pgfpathclose%
\pgfusepath{stroke,fill}%
\end{pgfscope}%
\begin{pgfscope}%
\pgfpathrectangle{\pgfqpoint{0.800000in}{3.252941in}}{\pgfqpoint{2.407767in}{1.544118in}}%
\pgfusepath{clip}%
\pgfsetroundcap%
\pgfsetroundjoin%
\pgfsetlinewidth{0.501875pt}%
\definecolor{currentstroke}{rgb}{0.281887,0.150881,0.465405}%
\pgfsetstrokecolor{currentstroke}%
\pgfsetdash{}{0pt}%
\pgfpathmoveto{\pgfqpoint{2.231976in}{3.802491in}}%
\pgfpathquadraticcurveto{\pgfqpoint{2.219010in}{3.804196in}}{\pgfqpoint{2.213743in}{3.804889in}}%
\pgfusepath{stroke}%
\end{pgfscope}%
\begin{pgfscope}%
\pgfpathrectangle{\pgfqpoint{0.800000in}{3.252941in}}{\pgfqpoint{2.407767in}{1.544118in}}%
\pgfusepath{clip}%
\pgfsetroundcap%
\pgfsetroundjoin%
\definecolor{currentfill}{rgb}{0.281887,0.150881,0.465405}%
\pgfsetfillcolor{currentfill}%
\pgfsetlinewidth{0.501875pt}%
\definecolor{currentstroke}{rgb}{0.281887,0.150881,0.465405}%
\pgfsetstrokecolor{currentstroke}%
\pgfsetdash{}{0pt}%
\pgfpathmoveto{\pgfqpoint{2.239473in}{3.787497in}}%
\pgfpathlineto{\pgfqpoint{2.213743in}{3.804889in}}%
\pgfpathlineto{\pgfqpoint{2.243094in}{3.815038in}}%
\pgfpathlineto{\pgfqpoint{2.239473in}{3.787497in}}%
\pgfpathlineto{\pgfqpoint{2.239473in}{3.787497in}}%
\pgfpathclose%
\pgfusepath{stroke,fill}%
\end{pgfscope}%
\begin{pgfscope}%
\pgfpathrectangle{\pgfqpoint{0.800000in}{3.252941in}}{\pgfqpoint{2.407767in}{1.544118in}}%
\pgfusepath{clip}%
\pgfsetroundcap%
\pgfsetroundjoin%
\pgfsetlinewidth{0.501875pt}%
\definecolor{currentstroke}{rgb}{0.277134,0.185228,0.489898}%
\pgfsetstrokecolor{currentstroke}%
\pgfsetdash{}{0pt}%
\pgfpathmoveto{\pgfqpoint{2.389394in}{3.823030in}}%
\pgfpathquadraticcurveto{\pgfqpoint{2.376197in}{3.823731in}}{\pgfqpoint{2.370754in}{3.824020in}}%
\pgfusepath{stroke}%
\end{pgfscope}%
\begin{pgfscope}%
\pgfpathrectangle{\pgfqpoint{0.800000in}{3.252941in}}{\pgfqpoint{2.407767in}{1.544118in}}%
\pgfusepath{clip}%
\pgfsetroundcap%
\pgfsetroundjoin%
\definecolor{currentfill}{rgb}{0.277134,0.185228,0.489898}%
\pgfsetfillcolor{currentfill}%
\pgfsetlinewidth{0.501875pt}%
\definecolor{currentstroke}{rgb}{0.277134,0.185228,0.489898}%
\pgfsetstrokecolor{currentstroke}%
\pgfsetdash{}{0pt}%
\pgfpathmoveto{\pgfqpoint{2.397756in}{3.808677in}}%
\pgfpathlineto{\pgfqpoint{2.370754in}{3.824020in}}%
\pgfpathlineto{\pgfqpoint{2.399230in}{3.836416in}}%
\pgfpathlineto{\pgfqpoint{2.397756in}{3.808677in}}%
\pgfpathlineto{\pgfqpoint{2.397756in}{3.808677in}}%
\pgfpathclose%
\pgfusepath{stroke,fill}%
\end{pgfscope}%
\begin{pgfscope}%
\pgfpathrectangle{\pgfqpoint{0.800000in}{3.252941in}}{\pgfqpoint{2.407767in}{1.544118in}}%
\pgfusepath{clip}%
\pgfsetroundcap%
\pgfsetroundjoin%
\pgfsetlinewidth{0.501875pt}%
\definecolor{currentstroke}{rgb}{0.281412,0.155834,0.469201}%
\pgfsetstrokecolor{currentstroke}%
\pgfsetdash{}{0pt}%
\pgfpathmoveto{\pgfqpoint{2.495172in}{3.853390in}}%
\pgfpathquadraticcurveto{\pgfqpoint{2.481944in}{3.853804in}}{\pgfqpoint{2.476476in}{3.853975in}}%
\pgfusepath{stroke}%
\end{pgfscope}%
\begin{pgfscope}%
\pgfpathrectangle{\pgfqpoint{0.800000in}{3.252941in}}{\pgfqpoint{2.407767in}{1.544118in}}%
\pgfusepath{clip}%
\pgfsetroundcap%
\pgfsetroundjoin%
\definecolor{currentfill}{rgb}{0.281412,0.155834,0.469201}%
\pgfsetfillcolor{currentfill}%
\pgfsetlinewidth{0.501875pt}%
\definecolor{currentstroke}{rgb}{0.281412,0.155834,0.469201}%
\pgfsetstrokecolor{currentstroke}%
\pgfsetdash{}{0pt}%
\pgfpathmoveto{\pgfqpoint{2.503806in}{3.839225in}}%
\pgfpathlineto{\pgfqpoint{2.476476in}{3.853975in}}%
\pgfpathlineto{\pgfqpoint{2.504674in}{3.866989in}}%
\pgfpathlineto{\pgfqpoint{2.503806in}{3.839225in}}%
\pgfpathlineto{\pgfqpoint{2.503806in}{3.839225in}}%
\pgfpathclose%
\pgfusepath{stroke,fill}%
\end{pgfscope}%
\begin{pgfscope}%
\pgfpathrectangle{\pgfqpoint{0.800000in}{3.252941in}}{\pgfqpoint{2.407767in}{1.544118in}}%
\pgfusepath{clip}%
\pgfsetroundcap%
\pgfsetroundjoin%
\pgfsetlinewidth{0.501875pt}%
\definecolor{currentstroke}{rgb}{0.199430,0.387607,0.554642}%
\pgfsetstrokecolor{currentstroke}%
\pgfsetdash{}{0pt}%
\pgfpathmoveto{\pgfqpoint{2.230947in}{3.899073in}}%
\pgfpathquadraticcurveto{\pgfqpoint{2.217826in}{3.900220in}}{\pgfqpoint{2.212440in}{3.900691in}}%
\pgfusepath{stroke}%
\end{pgfscope}%
\begin{pgfscope}%
\pgfpathrectangle{\pgfqpoint{0.800000in}{3.252941in}}{\pgfqpoint{2.407767in}{1.544118in}}%
\pgfusepath{clip}%
\pgfsetroundcap%
\pgfsetroundjoin%
\definecolor{currentfill}{rgb}{0.199430,0.387607,0.554642}%
\pgfsetfillcolor{currentfill}%
\pgfsetlinewidth{0.501875pt}%
\definecolor{currentstroke}{rgb}{0.199430,0.387607,0.554642}%
\pgfsetstrokecolor{currentstroke}%
\pgfsetdash{}{0pt}%
\pgfpathmoveto{\pgfqpoint{2.238903in}{3.884436in}}%
\pgfpathlineto{\pgfqpoint{2.212440in}{3.900691in}}%
\pgfpathlineto{\pgfqpoint{2.241321in}{3.912108in}}%
\pgfpathlineto{\pgfqpoint{2.238903in}{3.884436in}}%
\pgfpathlineto{\pgfqpoint{2.238903in}{3.884436in}}%
\pgfpathclose%
\pgfusepath{stroke,fill}%
\end{pgfscope}%
\begin{pgfscope}%
\pgfpathrectangle{\pgfqpoint{0.800000in}{3.252941in}}{\pgfqpoint{2.407767in}{1.544118in}}%
\pgfusepath{clip}%
\pgfsetroundcap%
\pgfsetroundjoin%
\pgfsetlinewidth{0.501875pt}%
\definecolor{currentstroke}{rgb}{0.248629,0.278775,0.534556}%
\pgfsetstrokecolor{currentstroke}%
\pgfsetdash{}{0pt}%
\pgfpathmoveto{\pgfqpoint{2.442179in}{3.923085in}}%
\pgfpathquadraticcurveto{\pgfqpoint{2.428943in}{3.923387in}}{\pgfqpoint{2.423470in}{3.923511in}}%
\pgfusepath{stroke}%
\end{pgfscope}%
\begin{pgfscope}%
\pgfpathrectangle{\pgfqpoint{0.800000in}{3.252941in}}{\pgfqpoint{2.407767in}{1.544118in}}%
\pgfusepath{clip}%
\pgfsetroundcap%
\pgfsetroundjoin%
\definecolor{currentfill}{rgb}{0.248629,0.278775,0.534556}%
\pgfsetfillcolor{currentfill}%
\pgfsetlinewidth{0.501875pt}%
\definecolor{currentstroke}{rgb}{0.248629,0.278775,0.534556}%
\pgfsetstrokecolor{currentstroke}%
\pgfsetdash{}{0pt}%
\pgfpathmoveto{\pgfqpoint{2.450924in}{3.908993in}}%
\pgfpathlineto{\pgfqpoint{2.423470in}{3.923511in}}%
\pgfpathlineto{\pgfqpoint{2.451557in}{3.936764in}}%
\pgfpathlineto{\pgfqpoint{2.450924in}{3.908993in}}%
\pgfpathlineto{\pgfqpoint{2.450924in}{3.908993in}}%
\pgfpathclose%
\pgfusepath{stroke,fill}%
\end{pgfscope}%
\begin{pgfscope}%
\pgfpathrectangle{\pgfqpoint{0.800000in}{3.252941in}}{\pgfqpoint{2.407767in}{1.544118in}}%
\pgfusepath{clip}%
\pgfsetroundcap%
\pgfsetroundjoin%
\pgfsetlinewidth{0.501875pt}%
\definecolor{currentstroke}{rgb}{0.143303,0.669459,0.511215}%
\pgfsetstrokecolor{currentstroke}%
\pgfsetdash{}{0pt}%
\pgfpathmoveto{\pgfqpoint{2.230413in}{3.961822in}}%
\pgfpathquadraticcurveto{\pgfqpoint{2.217201in}{3.962412in}}{\pgfqpoint{2.211746in}{3.962655in}}%
\pgfusepath{stroke}%
\end{pgfscope}%
\begin{pgfscope}%
\pgfpathrectangle{\pgfqpoint{0.800000in}{3.252941in}}{\pgfqpoint{2.407767in}{1.544118in}}%
\pgfusepath{clip}%
\pgfsetroundcap%
\pgfsetroundjoin%
\definecolor{currentfill}{rgb}{0.143303,0.669459,0.511215}%
\pgfsetfillcolor{currentfill}%
\pgfsetlinewidth{0.501875pt}%
\definecolor{currentstroke}{rgb}{0.143303,0.669459,0.511215}%
\pgfsetstrokecolor{currentstroke}%
\pgfsetdash{}{0pt}%
\pgfpathmoveto{\pgfqpoint{2.238877in}{3.947541in}}%
\pgfpathlineto{\pgfqpoint{2.211746in}{3.962655in}}%
\pgfpathlineto{\pgfqpoint{2.240116in}{3.975291in}}%
\pgfpathlineto{\pgfqpoint{2.238877in}{3.947541in}}%
\pgfpathlineto{\pgfqpoint{2.238877in}{3.947541in}}%
\pgfpathclose%
\pgfusepath{stroke,fill}%
\end{pgfscope}%
\begin{pgfscope}%
\pgfpathrectangle{\pgfqpoint{0.800000in}{3.252941in}}{\pgfqpoint{2.407767in}{1.544118in}}%
\pgfusepath{clip}%
\pgfsetroundcap%
\pgfsetroundjoin%
\pgfsetlinewidth{0.501875pt}%
\definecolor{currentstroke}{rgb}{0.188923,0.410910,0.556326}%
\pgfsetstrokecolor{currentstroke}%
\pgfsetdash{}{0pt}%
\pgfpathmoveto{\pgfqpoint{2.389175in}{3.991215in}}%
\pgfpathquadraticcurveto{\pgfqpoint{2.375933in}{3.991381in}}{\pgfqpoint{2.370455in}{3.991450in}}%
\pgfusepath{stroke}%
\end{pgfscope}%
\begin{pgfscope}%
\pgfpathrectangle{\pgfqpoint{0.800000in}{3.252941in}}{\pgfqpoint{2.407767in}{1.544118in}}%
\pgfusepath{clip}%
\pgfsetroundcap%
\pgfsetroundjoin%
\definecolor{currentfill}{rgb}{0.188923,0.410910,0.556326}%
\pgfsetfillcolor{currentfill}%
\pgfsetlinewidth{0.501875pt}%
\definecolor{currentstroke}{rgb}{0.188923,0.410910,0.556326}%
\pgfsetstrokecolor{currentstroke}%
\pgfsetdash{}{0pt}%
\pgfpathmoveto{\pgfqpoint{2.398056in}{3.977213in}}%
\pgfpathlineto{\pgfqpoint{2.370455in}{3.991450in}}%
\pgfpathlineto{\pgfqpoint{2.398406in}{4.004989in}}%
\pgfpathlineto{\pgfqpoint{2.398056in}{3.977213in}}%
\pgfpathlineto{\pgfqpoint{2.398056in}{3.977213in}}%
\pgfpathclose%
\pgfusepath{stroke,fill}%
\end{pgfscope}%
\begin{pgfscope}%
\pgfpathrectangle{\pgfqpoint{0.800000in}{3.252941in}}{\pgfqpoint{2.407767in}{1.544118in}}%
\pgfusepath{clip}%
\pgfsetroundcap%
\pgfsetroundjoin%
\pgfsetlinewidth{0.501875pt}%
\definecolor{currentstroke}{rgb}{0.166383,0.690856,0.496502}%
\pgfsetstrokecolor{currentstroke}%
\pgfsetdash{}{0pt}%
\pgfpathmoveto{\pgfqpoint{2.283215in}{4.025369in}}%
\pgfpathquadraticcurveto{\pgfqpoint{2.269972in}{4.025396in}}{\pgfqpoint{2.264492in}{4.025408in}}%
\pgfusepath{stroke}%
\end{pgfscope}%
\begin{pgfscope}%
\pgfpathrectangle{\pgfqpoint{0.800000in}{3.252941in}}{\pgfqpoint{2.407767in}{1.544118in}}%
\pgfusepath{clip}%
\pgfsetroundcap%
\pgfsetroundjoin%
\definecolor{currentfill}{rgb}{0.166383,0.690856,0.496502}%
\pgfsetfillcolor{currentfill}%
\pgfsetlinewidth{0.501875pt}%
\definecolor{currentstroke}{rgb}{0.166383,0.690856,0.496502}%
\pgfsetstrokecolor{currentstroke}%
\pgfsetdash{}{0pt}%
\pgfpathmoveto{\pgfqpoint{2.292241in}{4.011462in}}%
\pgfpathlineto{\pgfqpoint{2.264492in}{4.025408in}}%
\pgfpathlineto{\pgfqpoint{2.292298in}{4.039239in}}%
\pgfpathlineto{\pgfqpoint{2.292241in}{4.011462in}}%
\pgfpathlineto{\pgfqpoint{2.292241in}{4.011462in}}%
\pgfpathclose%
\pgfusepath{stroke,fill}%
\end{pgfscope}%
\begin{pgfscope}%
\pgfpathrectangle{\pgfqpoint{0.800000in}{3.252941in}}{\pgfqpoint{2.407767in}{1.544118in}}%
\pgfusepath{clip}%
\pgfsetroundcap%
\pgfsetroundjoin%
\pgfsetlinewidth{0.501875pt}%
\definecolor{currentstroke}{rgb}{0.134692,0.658636,0.517649}%
\pgfsetstrokecolor{currentstroke}%
\pgfsetdash{}{0pt}%
\pgfpathmoveto{\pgfqpoint{2.283259in}{4.056544in}}%
\pgfpathquadraticcurveto{\pgfqpoint{2.270023in}{4.056246in}}{\pgfqpoint{2.264550in}{4.056122in}}%
\pgfusepath{stroke}%
\end{pgfscope}%
\begin{pgfscope}%
\pgfpathrectangle{\pgfqpoint{0.800000in}{3.252941in}}{\pgfqpoint{2.407767in}{1.544118in}}%
\pgfusepath{clip}%
\pgfsetroundcap%
\pgfsetroundjoin%
\definecolor{currentfill}{rgb}{0.134692,0.658636,0.517649}%
\pgfsetfillcolor{currentfill}%
\pgfsetlinewidth{0.501875pt}%
\definecolor{currentstroke}{rgb}{0.134692,0.658636,0.517649}%
\pgfsetstrokecolor{currentstroke}%
\pgfsetdash{}{0pt}%
\pgfpathmoveto{\pgfqpoint{2.292633in}{4.042862in}}%
\pgfpathlineto{\pgfqpoint{2.264550in}{4.056122in}}%
\pgfpathlineto{\pgfqpoint{2.292008in}{4.070633in}}%
\pgfpathlineto{\pgfqpoint{2.292633in}{4.042862in}}%
\pgfpathlineto{\pgfqpoint{2.292633in}{4.042862in}}%
\pgfpathclose%
\pgfusepath{stroke,fill}%
\end{pgfscope}%
\begin{pgfscope}%
\pgfpathrectangle{\pgfqpoint{0.800000in}{3.252941in}}{\pgfqpoint{2.407767in}{1.544118in}}%
\pgfusepath{clip}%
\pgfsetroundcap%
\pgfsetroundjoin%
\pgfsetlinewidth{0.501875pt}%
\definecolor{currentstroke}{rgb}{0.195860,0.395433,0.555276}%
\pgfsetstrokecolor{currentstroke}%
\pgfsetdash{}{0pt}%
\pgfpathmoveto{\pgfqpoint{2.389203in}{4.092114in}}%
\pgfpathquadraticcurveto{\pgfqpoint{2.375966in}{4.091846in}}{\pgfqpoint{2.370492in}{4.091735in}}%
\pgfusepath{stroke}%
\end{pgfscope}%
\begin{pgfscope}%
\pgfpathrectangle{\pgfqpoint{0.800000in}{3.252941in}}{\pgfqpoint{2.407767in}{1.544118in}}%
\pgfusepath{clip}%
\pgfsetroundcap%
\pgfsetroundjoin%
\definecolor{currentfill}{rgb}{0.195860,0.395433,0.555276}%
\pgfsetfillcolor{currentfill}%
\pgfsetlinewidth{0.501875pt}%
\definecolor{currentstroke}{rgb}{0.195860,0.395433,0.555276}%
\pgfsetstrokecolor{currentstroke}%
\pgfsetdash{}{0pt}%
\pgfpathmoveto{\pgfqpoint{2.398545in}{4.078412in}}%
\pgfpathlineto{\pgfqpoint{2.370492in}{4.091735in}}%
\pgfpathlineto{\pgfqpoint{2.397982in}{4.106184in}}%
\pgfpathlineto{\pgfqpoint{2.398545in}{4.078412in}}%
\pgfpathlineto{\pgfqpoint{2.398545in}{4.078412in}}%
\pgfpathclose%
\pgfusepath{stroke,fill}%
\end{pgfscope}%
\begin{pgfscope}%
\pgfpathrectangle{\pgfqpoint{0.800000in}{3.252941in}}{\pgfqpoint{2.407767in}{1.544118in}}%
\pgfusepath{clip}%
\pgfsetroundcap%
\pgfsetroundjoin%
\pgfsetlinewidth{0.501875pt}%
\definecolor{currentstroke}{rgb}{0.248629,0.278775,0.534556}%
\pgfsetstrokecolor{currentstroke}%
\pgfsetdash{}{0pt}%
\pgfpathmoveto{\pgfqpoint{2.442186in}{4.126768in}}%
\pgfpathquadraticcurveto{\pgfqpoint{2.428951in}{4.126460in}}{\pgfqpoint{2.423478in}{4.126333in}}%
\pgfusepath{stroke}%
\end{pgfscope}%
\begin{pgfscope}%
\pgfpathrectangle{\pgfqpoint{0.800000in}{3.252941in}}{\pgfqpoint{2.407767in}{1.544118in}}%
\pgfusepath{clip}%
\pgfsetroundcap%
\pgfsetroundjoin%
\definecolor{currentfill}{rgb}{0.248629,0.278775,0.534556}%
\pgfsetfillcolor{currentfill}%
\pgfsetlinewidth{0.501875pt}%
\definecolor{currentstroke}{rgb}{0.248629,0.278775,0.534556}%
\pgfsetstrokecolor{currentstroke}%
\pgfsetdash{}{0pt}%
\pgfpathmoveto{\pgfqpoint{2.451572in}{4.113094in}}%
\pgfpathlineto{\pgfqpoint{2.423478in}{4.126333in}}%
\pgfpathlineto{\pgfqpoint{2.450925in}{4.140864in}}%
\pgfpathlineto{\pgfqpoint{2.451572in}{4.113094in}}%
\pgfpathlineto{\pgfqpoint{2.451572in}{4.113094in}}%
\pgfpathclose%
\pgfusepath{stroke,fill}%
\end{pgfscope}%
\begin{pgfscope}%
\pgfpathrectangle{\pgfqpoint{0.800000in}{3.252941in}}{\pgfqpoint{2.407767in}{1.544118in}}%
\pgfusepath{clip}%
\pgfsetroundcap%
\pgfsetroundjoin%
\pgfsetlinewidth{0.501875pt}%
\definecolor{currentstroke}{rgb}{0.265145,0.232956,0.516599}%
\pgfsetstrokecolor{currentstroke}%
\pgfsetdash{}{0pt}%
\pgfpathmoveto{\pgfqpoint{2.442226in}{4.160585in}}%
\pgfpathquadraticcurveto{\pgfqpoint{2.428997in}{4.160178in}}{\pgfqpoint{2.423529in}{4.160010in}}%
\pgfusepath{stroke}%
\end{pgfscope}%
\begin{pgfscope}%
\pgfpathrectangle{\pgfqpoint{0.800000in}{3.252941in}}{\pgfqpoint{2.407767in}{1.544118in}}%
\pgfusepath{clip}%
\pgfsetroundcap%
\pgfsetroundjoin%
\definecolor{currentfill}{rgb}{0.265145,0.232956,0.516599}%
\pgfsetfillcolor{currentfill}%
\pgfsetlinewidth{0.501875pt}%
\definecolor{currentstroke}{rgb}{0.265145,0.232956,0.516599}%
\pgfsetstrokecolor{currentstroke}%
\pgfsetdash{}{0pt}%
\pgfpathmoveto{\pgfqpoint{2.451721in}{4.146981in}}%
\pgfpathlineto{\pgfqpoint{2.423529in}{4.160010in}}%
\pgfpathlineto{\pgfqpoint{2.450867in}{4.174746in}}%
\pgfpathlineto{\pgfqpoint{2.451721in}{4.146981in}}%
\pgfpathlineto{\pgfqpoint{2.451721in}{4.146981in}}%
\pgfpathclose%
\pgfusepath{stroke,fill}%
\end{pgfscope}%
\begin{pgfscope}%
\pgfpathrectangle{\pgfqpoint{0.800000in}{3.252941in}}{\pgfqpoint{2.407767in}{1.544118in}}%
\pgfusepath{clip}%
\pgfsetroundcap%
\pgfsetroundjoin%
\pgfsetlinewidth{0.501875pt}%
\definecolor{currentstroke}{rgb}{0.231674,0.318106,0.544834}%
\pgfsetstrokecolor{currentstroke}%
\pgfsetdash{}{0pt}%
\pgfpathmoveto{\pgfqpoint{2.179092in}{4.176805in}}%
\pgfpathquadraticcurveto{\pgfqpoint{2.166134in}{4.175063in}}{\pgfqpoint{2.160872in}{4.174355in}}%
\pgfusepath{stroke}%
\end{pgfscope}%
\begin{pgfscope}%
\pgfpathrectangle{\pgfqpoint{0.800000in}{3.252941in}}{\pgfqpoint{2.407767in}{1.544118in}}%
\pgfusepath{clip}%
\pgfsetroundcap%
\pgfsetroundjoin%
\definecolor{currentfill}{rgb}{0.231674,0.318106,0.544834}%
\pgfsetfillcolor{currentfill}%
\pgfsetlinewidth{0.501875pt}%
\definecolor{currentstroke}{rgb}{0.231674,0.318106,0.544834}%
\pgfsetstrokecolor{currentstroke}%
\pgfsetdash{}{0pt}%
\pgfpathmoveto{\pgfqpoint{2.190253in}{4.164291in}}%
\pgfpathlineto{\pgfqpoint{2.160872in}{4.174355in}}%
\pgfpathlineto{\pgfqpoint{2.186552in}{4.191821in}}%
\pgfpathlineto{\pgfqpoint{2.190253in}{4.164291in}}%
\pgfpathlineto{\pgfqpoint{2.190253in}{4.164291in}}%
\pgfpathclose%
\pgfusepath{stroke,fill}%
\end{pgfscope}%
\begin{pgfscope}%
\pgfpathrectangle{\pgfqpoint{0.800000in}{3.252941in}}{\pgfqpoint{2.407767in}{1.544118in}}%
\pgfusepath{clip}%
\pgfsetroundcap%
\pgfsetroundjoin%
\pgfsetlinewidth{0.501875pt}%
\definecolor{currentstroke}{rgb}{0.282290,0.145912,0.461510}%
\pgfsetstrokecolor{currentstroke}%
\pgfsetdash{}{0pt}%
\pgfpathmoveto{\pgfqpoint{2.495192in}{4.230487in}}%
\pgfpathquadraticcurveto{\pgfqpoint{2.481964in}{4.230080in}}{\pgfqpoint{2.476496in}{4.229912in}}%
\pgfusepath{stroke}%
\end{pgfscope}%
\begin{pgfscope}%
\pgfpathrectangle{\pgfqpoint{0.800000in}{3.252941in}}{\pgfqpoint{2.407767in}{1.544118in}}%
\pgfusepath{clip}%
\pgfsetroundcap%
\pgfsetroundjoin%
\definecolor{currentfill}{rgb}{0.282290,0.145912,0.461510}%
\pgfsetfillcolor{currentfill}%
\pgfsetlinewidth{0.501875pt}%
\definecolor{currentstroke}{rgb}{0.282290,0.145912,0.461510}%
\pgfsetstrokecolor{currentstroke}%
\pgfsetdash{}{0pt}%
\pgfpathmoveto{\pgfqpoint{2.504687in}{4.216883in}}%
\pgfpathlineto{\pgfqpoint{2.476496in}{4.229912in}}%
\pgfpathlineto{\pgfqpoint{2.503834in}{4.244648in}}%
\pgfpathlineto{\pgfqpoint{2.504687in}{4.216883in}}%
\pgfpathlineto{\pgfqpoint{2.504687in}{4.216883in}}%
\pgfpathclose%
\pgfusepath{stroke,fill}%
\end{pgfscope}%
\begin{pgfscope}%
\pgfpathrectangle{\pgfqpoint{0.800000in}{3.252941in}}{\pgfqpoint{2.407767in}{1.544118in}}%
\pgfusepath{clip}%
\pgfsetroundcap%
\pgfsetroundjoin%
\pgfsetlinewidth{0.501875pt}%
\definecolor{currentstroke}{rgb}{0.267968,0.223549,0.512008}%
\pgfsetstrokecolor{currentstroke}%
\pgfsetdash{}{0pt}%
\pgfpathmoveto{\pgfqpoint{2.180108in}{4.240667in}}%
\pgfpathquadraticcurveto{\pgfqpoint{2.167331in}{4.238453in}}{\pgfqpoint{2.162205in}{4.237565in}}%
\pgfusepath{stroke}%
\end{pgfscope}%
\begin{pgfscope}%
\pgfpathrectangle{\pgfqpoint{0.800000in}{3.252941in}}{\pgfqpoint{2.407767in}{1.544118in}}%
\pgfusepath{clip}%
\pgfsetroundcap%
\pgfsetroundjoin%
\definecolor{currentfill}{rgb}{0.267968,0.223549,0.512008}%
\pgfsetfillcolor{currentfill}%
\pgfsetlinewidth{0.501875pt}%
\definecolor{currentstroke}{rgb}{0.267968,0.223549,0.512008}%
\pgfsetstrokecolor{currentstroke}%
\pgfsetdash{}{0pt}%
\pgfpathmoveto{\pgfqpoint{2.191946in}{4.228622in}}%
\pgfpathlineto{\pgfqpoint{2.162205in}{4.237565in}}%
\pgfpathlineto{\pgfqpoint{2.187205in}{4.255992in}}%
\pgfpathlineto{\pgfqpoint{2.191946in}{4.228622in}}%
\pgfpathlineto{\pgfqpoint{2.191946in}{4.228622in}}%
\pgfpathclose%
\pgfusepath{stroke,fill}%
\end{pgfscope}%
\begin{pgfscope}%
\pgfpathrectangle{\pgfqpoint{0.800000in}{3.252941in}}{\pgfqpoint{2.407767in}{1.544118in}}%
\pgfusepath{clip}%
\pgfsetroundcap%
\pgfsetroundjoin%
\pgfsetlinewidth{0.501875pt}%
\definecolor{currentstroke}{rgb}{0.283187,0.125848,0.444960}%
\pgfsetstrokecolor{currentstroke}%
\pgfsetdash{}{0pt}%
\pgfpathmoveto{\pgfqpoint{2.442368in}{4.297345in}}%
\pgfpathquadraticcurveto{\pgfqpoint{2.429170in}{4.296654in}}{\pgfqpoint{2.423725in}{4.296369in}}%
\pgfusepath{stroke}%
\end{pgfscope}%
\begin{pgfscope}%
\pgfpathrectangle{\pgfqpoint{0.800000in}{3.252941in}}{\pgfqpoint{2.407767in}{1.544118in}}%
\pgfusepath{clip}%
\pgfsetroundcap%
\pgfsetroundjoin%
\definecolor{currentfill}{rgb}{0.283187,0.125848,0.444960}%
\pgfsetfillcolor{currentfill}%
\pgfsetlinewidth{0.501875pt}%
\definecolor{currentstroke}{rgb}{0.283187,0.125848,0.444960}%
\pgfsetstrokecolor{currentstroke}%
\pgfsetdash{}{0pt}%
\pgfpathmoveto{\pgfqpoint{2.452191in}{4.283951in}}%
\pgfpathlineto{\pgfqpoint{2.423725in}{4.296369in}}%
\pgfpathlineto{\pgfqpoint{2.450739in}{4.311691in}}%
\pgfpathlineto{\pgfqpoint{2.452191in}{4.283951in}}%
\pgfpathlineto{\pgfqpoint{2.452191in}{4.283951in}}%
\pgfpathclose%
\pgfusepath{stroke,fill}%
\end{pgfscope}%
\begin{pgfscope}%
\pgfpathrectangle{\pgfqpoint{0.800000in}{3.252941in}}{\pgfqpoint{2.407767in}{1.544118in}}%
\pgfusepath{clip}%
\pgfsetroundcap%
\pgfsetroundjoin%
\pgfsetlinewidth{0.501875pt}%
\definecolor{currentstroke}{rgb}{0.281924,0.089666,0.412415}%
\pgfsetstrokecolor{currentstroke}%
\pgfsetdash{}{0pt}%
\pgfpathmoveto{\pgfqpoint{2.495238in}{4.334465in}}%
\pgfpathquadraticcurveto{\pgfqpoint{2.482026in}{4.333892in}}{\pgfqpoint{2.476570in}{4.333656in}}%
\pgfusepath{stroke}%
\end{pgfscope}%
\begin{pgfscope}%
\pgfpathrectangle{\pgfqpoint{0.800000in}{3.252941in}}{\pgfqpoint{2.407767in}{1.544118in}}%
\pgfusepath{clip}%
\pgfsetroundcap%
\pgfsetroundjoin%
\definecolor{currentfill}{rgb}{0.281924,0.089666,0.412415}%
\pgfsetfillcolor{currentfill}%
\pgfsetlinewidth{0.501875pt}%
\definecolor{currentstroke}{rgb}{0.281924,0.089666,0.412415}%
\pgfsetstrokecolor{currentstroke}%
\pgfsetdash{}{0pt}%
\pgfpathmoveto{\pgfqpoint{2.504923in}{4.320983in}}%
\pgfpathlineto{\pgfqpoint{2.476570in}{4.333656in}}%
\pgfpathlineto{\pgfqpoint{2.503721in}{4.348734in}}%
\pgfpathlineto{\pgfqpoint{2.504923in}{4.320983in}}%
\pgfpathlineto{\pgfqpoint{2.504923in}{4.320983in}}%
\pgfpathclose%
\pgfusepath{stroke,fill}%
\end{pgfscope}%
\begin{pgfscope}%
\pgfpathrectangle{\pgfqpoint{0.800000in}{3.252941in}}{\pgfqpoint{2.407767in}{1.544118in}}%
\pgfusepath{clip}%
\pgfsetroundcap%
\pgfsetroundjoin%
\pgfsetlinewidth{0.501875pt}%
\definecolor{currentstroke}{rgb}{0.281446,0.084320,0.407414}%
\pgfsetstrokecolor{currentstroke}%
\pgfsetdash{}{0pt}%
\pgfpathmoveto{\pgfqpoint{2.389752in}{4.362967in}}%
\pgfpathquadraticcurveto{\pgfqpoint{2.376598in}{4.362006in}}{\pgfqpoint{2.371187in}{4.361611in}}%
\pgfusepath{stroke}%
\end{pgfscope}%
\begin{pgfscope}%
\pgfpathrectangle{\pgfqpoint{0.800000in}{3.252941in}}{\pgfqpoint{2.407767in}{1.544118in}}%
\pgfusepath{clip}%
\pgfsetroundcap%
\pgfsetroundjoin%
\definecolor{currentfill}{rgb}{0.281446,0.084320,0.407414}%
\pgfsetfillcolor{currentfill}%
\pgfsetlinewidth{0.501875pt}%
\definecolor{currentstroke}{rgb}{0.281446,0.084320,0.407414}%
\pgfsetstrokecolor{currentstroke}%
\pgfsetdash{}{0pt}%
\pgfpathmoveto{\pgfqpoint{2.399903in}{4.349783in}}%
\pgfpathlineto{\pgfqpoint{2.371187in}{4.361611in}}%
\pgfpathlineto{\pgfqpoint{2.397879in}{4.377487in}}%
\pgfpathlineto{\pgfqpoint{2.399903in}{4.349783in}}%
\pgfpathlineto{\pgfqpoint{2.399903in}{4.349783in}}%
\pgfpathclose%
\pgfusepath{stroke,fill}%
\end{pgfscope}%
\begin{pgfscope}%
\pgfpathrectangle{\pgfqpoint{0.800000in}{3.252941in}}{\pgfqpoint{2.407767in}{1.544118in}}%
\pgfusepath{clip}%
\pgfsetroundcap%
\pgfsetroundjoin%
\pgfsetlinewidth{0.501875pt}%
\definecolor{currentstroke}{rgb}{0.276022,0.044167,0.370164}%
\pgfsetstrokecolor{currentstroke}%
\pgfsetdash{}{0pt}%
\pgfpathmoveto{\pgfqpoint{2.442517in}{4.399629in}}%
\pgfpathquadraticcurveto{\pgfqpoint{2.429348in}{4.398751in}}{\pgfqpoint{2.423926in}{4.398389in}}%
\pgfusepath{stroke}%
\end{pgfscope}%
\begin{pgfscope}%
\pgfpathrectangle{\pgfqpoint{0.800000in}{3.252941in}}{\pgfqpoint{2.407767in}{1.544118in}}%
\pgfusepath{clip}%
\pgfsetroundcap%
\pgfsetroundjoin%
\definecolor{currentfill}{rgb}{0.276022,0.044167,0.370164}%
\pgfsetfillcolor{currentfill}%
\pgfsetlinewidth{0.501875pt}%
\definecolor{currentstroke}{rgb}{0.276022,0.044167,0.370164}%
\pgfsetstrokecolor{currentstroke}%
\pgfsetdash{}{0pt}%
\pgfpathmoveto{\pgfqpoint{2.452567in}{4.386380in}}%
\pgfpathlineto{\pgfqpoint{2.423926in}{4.398389in}}%
\pgfpathlineto{\pgfqpoint{2.450718in}{4.414096in}}%
\pgfpathlineto{\pgfqpoint{2.452567in}{4.386380in}}%
\pgfpathlineto{\pgfqpoint{2.452567in}{4.386380in}}%
\pgfpathclose%
\pgfusepath{stroke,fill}%
\end{pgfscope}%
\begin{pgfscope}%
\pgfpathrectangle{\pgfqpoint{0.800000in}{3.252941in}}{\pgfqpoint{2.407767in}{1.544118in}}%
\pgfusepath{clip}%
\pgfsetroundcap%
\pgfsetroundjoin%
\pgfsetlinewidth{0.501875pt}%
\definecolor{currentstroke}{rgb}{0.274952,0.037752,0.364543}%
\pgfsetstrokecolor{currentstroke}%
\pgfsetdash{}{0pt}%
\pgfpathmoveto{\pgfqpoint{2.548168in}{4.440876in}}%
\pgfpathquadraticcurveto{\pgfqpoint{2.534944in}{4.440460in}}{\pgfqpoint{2.529480in}{4.440289in}}%
\pgfusepath{stroke}%
\end{pgfscope}%
\begin{pgfscope}%
\pgfpathrectangle{\pgfqpoint{0.800000in}{3.252941in}}{\pgfqpoint{2.407767in}{1.544118in}}%
\pgfusepath{clip}%
\pgfsetroundcap%
\pgfsetroundjoin%
\definecolor{currentfill}{rgb}{0.274952,0.037752,0.364543}%
\pgfsetfillcolor{currentfill}%
\pgfsetlinewidth{0.501875pt}%
\definecolor{currentstroke}{rgb}{0.274952,0.037752,0.364543}%
\pgfsetstrokecolor{currentstroke}%
\pgfsetdash{}{0pt}%
\pgfpathmoveto{\pgfqpoint{2.557681in}{4.427279in}}%
\pgfpathlineto{\pgfqpoint{2.529480in}{4.440289in}}%
\pgfpathlineto{\pgfqpoint{2.556808in}{4.455043in}}%
\pgfpathlineto{\pgfqpoint{2.557681in}{4.427279in}}%
\pgfpathlineto{\pgfqpoint{2.557681in}{4.427279in}}%
\pgfpathclose%
\pgfusepath{stroke,fill}%
\end{pgfscope}%
\begin{pgfscope}%
\pgfpathrectangle{\pgfqpoint{0.800000in}{3.252941in}}{\pgfqpoint{2.407767in}{1.544118in}}%
\pgfusepath{clip}%
\pgfsetroundcap%
\pgfsetroundjoin%
\pgfsetlinewidth{0.501875pt}%
\definecolor{currentstroke}{rgb}{0.276022,0.044167,0.370164}%
\pgfsetstrokecolor{currentstroke}%
\pgfsetdash{}{0pt}%
\pgfpathmoveto{\pgfqpoint{1.946225in}{4.269338in}}%
\pgfpathquadraticcurveto{\pgfqpoint{1.945460in}{4.267553in}}{\pgfqpoint{1.947754in}{4.272903in}}%
\pgfusepath{stroke}%
\end{pgfscope}%
\begin{pgfscope}%
\pgfpathrectangle{\pgfqpoint{0.800000in}{3.252941in}}{\pgfqpoint{2.407767in}{1.544118in}}%
\pgfusepath{clip}%
\pgfsetroundcap%
\pgfsetroundjoin%
\definecolor{currentfill}{rgb}{0.276022,0.044167,0.370164}%
\pgfsetfillcolor{currentfill}%
\pgfsetlinewidth{0.501875pt}%
\definecolor{currentstroke}{rgb}{0.276022,0.044167,0.370164}%
\pgfsetstrokecolor{currentstroke}%
\pgfsetdash{}{0pt}%
\pgfpathmoveto{\pgfqpoint{1.971465in}{4.292959in}}%
\pgfpathlineto{\pgfqpoint{1.947754in}{4.272903in}}%
\pgfpathlineto{\pgfqpoint{1.945935in}{4.303906in}}%
\pgfpathlineto{\pgfqpoint{1.971465in}{4.292959in}}%
\pgfpathlineto{\pgfqpoint{1.971465in}{4.292959in}}%
\pgfpathclose%
\pgfusepath{stroke,fill}%
\end{pgfscope}%
\begin{pgfscope}%
\pgfpathrectangle{\pgfqpoint{0.800000in}{3.252941in}}{\pgfqpoint{2.407767in}{1.544118in}}%
\pgfusepath{clip}%
\pgfsetroundcap%
\pgfsetroundjoin%
\pgfsetlinewidth{0.501875pt}%
\definecolor{currentstroke}{rgb}{0.280255,0.165693,0.476498}%
\pgfsetstrokecolor{currentstroke}%
\pgfsetdash{}{0pt}%
\pgfpathmoveto{\pgfqpoint{1.857220in}{3.794820in}}%
\pgfpathquadraticcurveto{\pgfqpoint{1.853251in}{3.800246in}}{\pgfqpoint{1.853866in}{3.799406in}}%
\pgfusepath{stroke}%
\end{pgfscope}%
\begin{pgfscope}%
\pgfpathrectangle{\pgfqpoint{0.800000in}{3.252941in}}{\pgfqpoint{2.407767in}{1.544118in}}%
\pgfusepath{clip}%
\pgfsetroundcap%
\pgfsetroundjoin%
\definecolor{currentfill}{rgb}{0.280255,0.165693,0.476498}%
\pgfsetfillcolor{currentfill}%
\pgfsetlinewidth{0.501875pt}%
\definecolor{currentstroke}{rgb}{0.280255,0.165693,0.476498}%
\pgfsetstrokecolor{currentstroke}%
\pgfsetdash{}{0pt}%
\pgfpathmoveto{\pgfqpoint{1.859057in}{3.768786in}}%
\pgfpathlineto{\pgfqpoint{1.853866in}{3.799406in}}%
\pgfpathlineto{\pgfqpoint{1.881476in}{3.785187in}}%
\pgfpathlineto{\pgfqpoint{1.859057in}{3.768786in}}%
\pgfpathlineto{\pgfqpoint{1.859057in}{3.768786in}}%
\pgfpathclose%
\pgfusepath{stroke,fill}%
\end{pgfscope}%
\begin{pgfscope}%
\pgfpathrectangle{\pgfqpoint{0.800000in}{3.252941in}}{\pgfqpoint{2.407767in}{1.544118in}}%
\pgfusepath{clip}%
\pgfsetbuttcap%
\pgfsetroundjoin%
\pgfsetlinewidth{1.505625pt}%
\definecolor{currentstroke}{rgb}{0.000000,0.000000,0.000000}%
\pgfsetstrokecolor{currentstroke}%
\pgfsetdash{}{0pt}%
\pgfpathmoveto{\pgfqpoint{1.789710in}{3.514415in}}%
\pgfpathlineto{\pgfqpoint{1.789710in}{4.535585in}}%
\pgfusepath{stroke}%
\end{pgfscope}%
\begin{pgfscope}%
\pgfpathrectangle{\pgfqpoint{0.800000in}{3.252941in}}{\pgfqpoint{2.407767in}{1.544118in}}%
\pgfusepath{clip}%
\pgfsetbuttcap%
\pgfsetroundjoin%
\pgfsetlinewidth{1.505625pt}%
\definecolor{currentstroke}{rgb}{0.000000,0.000000,0.000000}%
\pgfsetstrokecolor{currentstroke}%
\pgfsetdash{}{0pt}%
\pgfpathmoveto{\pgfqpoint{2.699611in}{3.514415in}}%
\pgfpathlineto{\pgfqpoint{2.699611in}{4.535585in}}%
\pgfusepath{stroke}%
\end{pgfscope}%
\begin{pgfscope}%
\pgfsetrectcap%
\pgfsetmiterjoin%
\pgfsetlinewidth{0.803000pt}%
\definecolor{currentstroke}{rgb}{0.000000,0.000000,0.000000}%
\pgfsetstrokecolor{currentstroke}%
\pgfsetdash{}{0pt}%
\pgfpathmoveto{\pgfqpoint{0.800000in}{3.252941in}}%
\pgfpathlineto{\pgfqpoint{0.800000in}{4.797059in}}%
\pgfusepath{stroke}%
\end{pgfscope}%
\begin{pgfscope}%
\pgfsetrectcap%
\pgfsetmiterjoin%
\pgfsetlinewidth{0.803000pt}%
\definecolor{currentstroke}{rgb}{0.000000,0.000000,0.000000}%
\pgfsetstrokecolor{currentstroke}%
\pgfsetdash{}{0pt}%
\pgfpathmoveto{\pgfqpoint{3.207767in}{3.252941in}}%
\pgfpathlineto{\pgfqpoint{3.207767in}{4.797059in}}%
\pgfusepath{stroke}%
\end{pgfscope}%
\begin{pgfscope}%
\pgfsetrectcap%
\pgfsetmiterjoin%
\pgfsetlinewidth{0.803000pt}%
\definecolor{currentstroke}{rgb}{0.000000,0.000000,0.000000}%
\pgfsetstrokecolor{currentstroke}%
\pgfsetdash{}{0pt}%
\pgfpathmoveto{\pgfqpoint{0.800000in}{3.252941in}}%
\pgfpathlineto{\pgfqpoint{3.207767in}{3.252941in}}%
\pgfusepath{stroke}%
\end{pgfscope}%
\begin{pgfscope}%
\pgfsetrectcap%
\pgfsetmiterjoin%
\pgfsetlinewidth{0.803000pt}%
\definecolor{currentstroke}{rgb}{0.000000,0.000000,0.000000}%
\pgfsetstrokecolor{currentstroke}%
\pgfsetdash{}{0pt}%
\pgfpathmoveto{\pgfqpoint{0.800000in}{4.797059in}}%
\pgfpathlineto{\pgfqpoint{3.207767in}{4.797059in}}%
\pgfusepath{stroke}%
\end{pgfscope}%
\begin{pgfscope}%
\definecolor{textcolor}{rgb}{0.000000,0.000000,0.000000}%
\pgfsetstrokecolor{textcolor}%
\pgfsetfillcolor{textcolor}%
\pgftext[x=2.003883in,y=4.880392in,,base]{\color{textcolor}\sffamily\fontsize{12.000000}{14.400000}\selectfont c)}%
\end{pgfscope}%
\begin{pgfscope}%
\pgfsetbuttcap%
\pgfsetmiterjoin%
\definecolor{currentfill}{rgb}{1.000000,1.000000,1.000000}%
\pgfsetfillcolor{currentfill}%
\pgfsetlinewidth{0.000000pt}%
\definecolor{currentstroke}{rgb}{0.000000,0.000000,0.000000}%
\pgfsetstrokecolor{currentstroke}%
\pgfsetstrokeopacity{0.000000}%
\pgfsetdash{}{0pt}%
\pgfpathmoveto{\pgfqpoint{3.352233in}{3.252941in}}%
\pgfpathlineto{\pgfqpoint{5.760000in}{3.252941in}}%
\pgfpathlineto{\pgfqpoint{5.760000in}{4.797059in}}%
\pgfpathlineto{\pgfqpoint{3.352233in}{4.797059in}}%
\pgfpathlineto{\pgfqpoint{3.352233in}{3.252941in}}%
\pgfpathclose%
\pgfusepath{fill}%
\end{pgfscope}%
\begin{pgfscope}%
\pgfpathrectangle{\pgfqpoint{3.352233in}{3.252941in}}{\pgfqpoint{2.407767in}{1.544118in}}%
\pgfusepath{clip}%
\pgfsys@transformcm{2.416667}{0.000000}{0.000000}{1.555556}{3.352233in}{3.252941in}%
\pgftext[left,bottom]{\includegraphics[interpolate=false,width=1.000000in,height=1.000000in]{q_series_square-img3.png}}%
\end{pgfscope}%
\begin{pgfscope}%
\pgfsetbuttcap%
\pgfsetroundjoin%
\definecolor{currentfill}{rgb}{0.000000,0.000000,0.000000}%
\pgfsetfillcolor{currentfill}%
\pgfsetlinewidth{0.803000pt}%
\definecolor{currentstroke}{rgb}{0.000000,0.000000,0.000000}%
\pgfsetstrokecolor{currentstroke}%
\pgfsetdash{}{0pt}%
\pgfsys@defobject{currentmarker}{\pgfqpoint{0.000000in}{-0.048611in}}{\pgfqpoint{0.000000in}{0.000000in}}{%
\pgfpathmoveto{\pgfqpoint{0.000000in}{0.000000in}}%
\pgfpathlineto{\pgfqpoint{0.000000in}{-0.048611in}}%
\pgfusepath{stroke,fill}%
}%
\begin{pgfscope}%
\pgfsys@transformshift{3.785892in}{3.252941in}%
\pgfsys@useobject{currentmarker}{}%
\end{pgfscope}%
\end{pgfscope}%
\begin{pgfscope}%
\pgfsetbuttcap%
\pgfsetroundjoin%
\definecolor{currentfill}{rgb}{0.000000,0.000000,0.000000}%
\pgfsetfillcolor{currentfill}%
\pgfsetlinewidth{0.803000pt}%
\definecolor{currentstroke}{rgb}{0.000000,0.000000,0.000000}%
\pgfsetstrokecolor{currentstroke}%
\pgfsetdash{}{0pt}%
\pgfsys@defobject{currentmarker}{\pgfqpoint{0.000000in}{-0.048611in}}{\pgfqpoint{0.000000in}{0.000000in}}{%
\pgfpathmoveto{\pgfqpoint{0.000000in}{0.000000in}}%
\pgfpathlineto{\pgfqpoint{0.000000in}{-0.048611in}}%
\pgfusepath{stroke,fill}%
}%
\begin{pgfscope}%
\pgfsys@transformshift{4.291393in}{3.252941in}%
\pgfsys@useobject{currentmarker}{}%
\end{pgfscope}%
\end{pgfscope}%
\begin{pgfscope}%
\pgfsetbuttcap%
\pgfsetroundjoin%
\definecolor{currentfill}{rgb}{0.000000,0.000000,0.000000}%
\pgfsetfillcolor{currentfill}%
\pgfsetlinewidth{0.803000pt}%
\definecolor{currentstroke}{rgb}{0.000000,0.000000,0.000000}%
\pgfsetstrokecolor{currentstroke}%
\pgfsetdash{}{0pt}%
\pgfsys@defobject{currentmarker}{\pgfqpoint{0.000000in}{-0.048611in}}{\pgfqpoint{0.000000in}{0.000000in}}{%
\pgfpathmoveto{\pgfqpoint{0.000000in}{0.000000in}}%
\pgfpathlineto{\pgfqpoint{0.000000in}{-0.048611in}}%
\pgfusepath{stroke,fill}%
}%
\begin{pgfscope}%
\pgfsys@transformshift{4.796893in}{3.252941in}%
\pgfsys@useobject{currentmarker}{}%
\end{pgfscope}%
\end{pgfscope}%
\begin{pgfscope}%
\pgfsetbuttcap%
\pgfsetroundjoin%
\definecolor{currentfill}{rgb}{0.000000,0.000000,0.000000}%
\pgfsetfillcolor{currentfill}%
\pgfsetlinewidth{0.803000pt}%
\definecolor{currentstroke}{rgb}{0.000000,0.000000,0.000000}%
\pgfsetstrokecolor{currentstroke}%
\pgfsetdash{}{0pt}%
\pgfsys@defobject{currentmarker}{\pgfqpoint{0.000000in}{-0.048611in}}{\pgfqpoint{0.000000in}{0.000000in}}{%
\pgfpathmoveto{\pgfqpoint{0.000000in}{0.000000in}}%
\pgfpathlineto{\pgfqpoint{0.000000in}{-0.048611in}}%
\pgfusepath{stroke,fill}%
}%
\begin{pgfscope}%
\pgfsys@transformshift{5.302394in}{3.252941in}%
\pgfsys@useobject{currentmarker}{}%
\end{pgfscope}%
\end{pgfscope}%
\begin{pgfscope}%
\pgfsetbuttcap%
\pgfsetroundjoin%
\definecolor{currentfill}{rgb}{0.000000,0.000000,0.000000}%
\pgfsetfillcolor{currentfill}%
\pgfsetlinewidth{0.803000pt}%
\definecolor{currentstroke}{rgb}{0.000000,0.000000,0.000000}%
\pgfsetstrokecolor{currentstroke}%
\pgfsetdash{}{0pt}%
\pgfsys@defobject{currentmarker}{\pgfqpoint{-0.048611in}{0.000000in}}{\pgfqpoint{-0.000000in}{0.000000in}}{%
\pgfpathmoveto{\pgfqpoint{-0.000000in}{0.000000in}}%
\pgfpathlineto{\pgfqpoint{-0.048611in}{0.000000in}}%
\pgfusepath{stroke,fill}%
}%
\begin{pgfscope}%
\pgfsys@transformshift{3.352233in}{3.514415in}%
\pgfsys@useobject{currentmarker}{}%
\end{pgfscope}%
\end{pgfscope}%
\begin{pgfscope}%
\pgfsetbuttcap%
\pgfsetroundjoin%
\definecolor{currentfill}{rgb}{0.000000,0.000000,0.000000}%
\pgfsetfillcolor{currentfill}%
\pgfsetlinewidth{0.803000pt}%
\definecolor{currentstroke}{rgb}{0.000000,0.000000,0.000000}%
\pgfsetstrokecolor{currentstroke}%
\pgfsetdash{}{0pt}%
\pgfsys@defobject{currentmarker}{\pgfqpoint{-0.048611in}{0.000000in}}{\pgfqpoint{-0.000000in}{0.000000in}}{%
\pgfpathmoveto{\pgfqpoint{-0.000000in}{0.000000in}}%
\pgfpathlineto{\pgfqpoint{-0.048611in}{0.000000in}}%
\pgfusepath{stroke,fill}%
}%
\begin{pgfscope}%
\pgfsys@transformshift{3.352233in}{4.025000in}%
\pgfsys@useobject{currentmarker}{}%
\end{pgfscope}%
\end{pgfscope}%
\begin{pgfscope}%
\pgfsetbuttcap%
\pgfsetroundjoin%
\definecolor{currentfill}{rgb}{0.000000,0.000000,0.000000}%
\pgfsetfillcolor{currentfill}%
\pgfsetlinewidth{0.803000pt}%
\definecolor{currentstroke}{rgb}{0.000000,0.000000,0.000000}%
\pgfsetstrokecolor{currentstroke}%
\pgfsetdash{}{0pt}%
\pgfsys@defobject{currentmarker}{\pgfqpoint{-0.048611in}{0.000000in}}{\pgfqpoint{-0.000000in}{0.000000in}}{%
\pgfpathmoveto{\pgfqpoint{-0.000000in}{0.000000in}}%
\pgfpathlineto{\pgfqpoint{-0.048611in}{0.000000in}}%
\pgfusepath{stroke,fill}%
}%
\begin{pgfscope}%
\pgfsys@transformshift{3.352233in}{4.535585in}%
\pgfsys@useobject{currentmarker}{}%
\end{pgfscope}%
\end{pgfscope}%
\begin{pgfscope}%
\pgfpathrectangle{\pgfqpoint{3.352233in}{3.252941in}}{\pgfqpoint{2.407767in}{1.544118in}}%
\pgfusepath{clip}%
\pgfsetbuttcap%
\pgfsetroundjoin%
\pgfsetlinewidth{0.501875pt}%
\definecolor{currentstroke}{rgb}{0.269944,0.014625,0.341379}%
\pgfsetstrokecolor{currentstroke}%
\pgfsetdash{}{0pt}%
\pgfpathmoveto{\pgfqpoint{5.206279in}{4.511446in}}%
\pgfpathlineto{\pgfqpoint{5.153326in}{4.510859in}}%
\pgfusepath{stroke}%
\end{pgfscope}%
\begin{pgfscope}%
\pgfpathrectangle{\pgfqpoint{3.352233in}{3.252941in}}{\pgfqpoint{2.407767in}{1.544118in}}%
\pgfusepath{clip}%
\pgfsetbuttcap%
\pgfsetroundjoin%
\pgfsetlinewidth{0.501875pt}%
\definecolor{currentstroke}{rgb}{0.269944,0.014625,0.341379}%
\pgfsetstrokecolor{currentstroke}%
\pgfsetdash{}{0pt}%
\pgfpathmoveto{\pgfqpoint{5.153326in}{4.510859in}}%
\pgfpathlineto{\pgfqpoint{5.100381in}{4.510005in}}%
\pgfusepath{stroke}%
\end{pgfscope}%
\begin{pgfscope}%
\pgfpathrectangle{\pgfqpoint{3.352233in}{3.252941in}}{\pgfqpoint{2.407767in}{1.544118in}}%
\pgfusepath{clip}%
\pgfsetbuttcap%
\pgfsetroundjoin%
\pgfsetlinewidth{0.501875pt}%
\definecolor{currentstroke}{rgb}{0.273809,0.031497,0.358853}%
\pgfsetstrokecolor{currentstroke}%
\pgfsetdash{}{0pt}%
\pgfpathmoveto{\pgfqpoint{5.100381in}{4.510005in}}%
\pgfpathlineto{\pgfqpoint{5.047615in}{4.507979in}}%
\pgfusepath{stroke}%
\end{pgfscope}%
\begin{pgfscope}%
\pgfpathrectangle{\pgfqpoint{3.352233in}{3.252941in}}{\pgfqpoint{2.407767in}{1.544118in}}%
\pgfusepath{clip}%
\pgfsetbuttcap%
\pgfsetroundjoin%
\pgfsetlinewidth{0.501875pt}%
\definecolor{currentstroke}{rgb}{0.273809,0.031497,0.358853}%
\pgfsetstrokecolor{currentstroke}%
\pgfsetdash{}{0pt}%
\pgfpathmoveto{\pgfqpoint{5.047615in}{4.507979in}}%
\pgfpathlineto{\pgfqpoint{4.994919in}{4.505196in}}%
\pgfusepath{stroke}%
\end{pgfscope}%
\begin{pgfscope}%
\pgfpathrectangle{\pgfqpoint{3.352233in}{3.252941in}}{\pgfqpoint{2.407767in}{1.544118in}}%
\pgfusepath{clip}%
\pgfsetbuttcap%
\pgfsetroundjoin%
\pgfsetlinewidth{0.501875pt}%
\definecolor{currentstroke}{rgb}{0.274952,0.037752,0.364543}%
\pgfsetstrokecolor{currentstroke}%
\pgfsetdash{}{0pt}%
\pgfpathmoveto{\pgfqpoint{4.994919in}{4.505196in}}%
\pgfpathlineto{\pgfqpoint{4.942254in}{4.501782in}}%
\pgfusepath{stroke}%
\end{pgfscope}%
\begin{pgfscope}%
\pgfpathrectangle{\pgfqpoint{3.352233in}{3.252941in}}{\pgfqpoint{2.407767in}{1.544118in}}%
\pgfusepath{clip}%
\pgfsetbuttcap%
\pgfsetroundjoin%
\pgfsetlinewidth{0.501875pt}%
\definecolor{currentstroke}{rgb}{0.268510,0.009605,0.335427}%
\pgfsetstrokecolor{currentstroke}%
\pgfsetdash{}{0pt}%
\pgfpathmoveto{\pgfqpoint{5.206279in}{4.476700in}}%
\pgfpathlineto{\pgfqpoint{5.153465in}{4.475080in}}%
\pgfusepath{stroke}%
\end{pgfscope}%
\begin{pgfscope}%
\pgfpathrectangle{\pgfqpoint{3.352233in}{3.252941in}}{\pgfqpoint{2.407767in}{1.544118in}}%
\pgfusepath{clip}%
\pgfsetbuttcap%
\pgfsetroundjoin%
\pgfsetlinewidth{0.501875pt}%
\definecolor{currentstroke}{rgb}{0.272594,0.025563,0.353093}%
\pgfsetstrokecolor{currentstroke}%
\pgfsetdash{}{0pt}%
\pgfpathmoveto{\pgfqpoint{5.153465in}{4.475080in}}%
\pgfpathlineto{\pgfqpoint{5.100609in}{4.473439in}}%
\pgfusepath{stroke}%
\end{pgfscope}%
\begin{pgfscope}%
\pgfpathrectangle{\pgfqpoint{3.352233in}{3.252941in}}{\pgfqpoint{2.407767in}{1.544118in}}%
\pgfusepath{clip}%
\pgfsetbuttcap%
\pgfsetroundjoin%
\pgfsetlinewidth{0.501875pt}%
\definecolor{currentstroke}{rgb}{0.273809,0.031497,0.358853}%
\pgfsetstrokecolor{currentstroke}%
\pgfsetdash{}{0pt}%
\pgfpathmoveto{\pgfqpoint{5.100609in}{4.473439in}}%
\pgfpathlineto{\pgfqpoint{5.047676in}{4.472463in}}%
\pgfusepath{stroke}%
\end{pgfscope}%
\begin{pgfscope}%
\pgfpathrectangle{\pgfqpoint{3.352233in}{3.252941in}}{\pgfqpoint{2.407767in}{1.544118in}}%
\pgfusepath{clip}%
\pgfsetbuttcap%
\pgfsetroundjoin%
\pgfsetlinewidth{0.501875pt}%
\definecolor{currentstroke}{rgb}{0.274952,0.037752,0.364543}%
\pgfsetstrokecolor{currentstroke}%
\pgfsetdash{}{0pt}%
\pgfpathmoveto{\pgfqpoint{5.047676in}{4.472463in}}%
\pgfpathlineto{\pgfqpoint{4.994801in}{4.470699in}}%
\pgfusepath{stroke}%
\end{pgfscope}%
\begin{pgfscope}%
\pgfpathrectangle{\pgfqpoint{3.352233in}{3.252941in}}{\pgfqpoint{2.407767in}{1.544118in}}%
\pgfusepath{clip}%
\pgfsetbuttcap%
\pgfsetroundjoin%
\pgfsetlinewidth{0.501875pt}%
\definecolor{currentstroke}{rgb}{0.277018,0.050344,0.375715}%
\pgfsetstrokecolor{currentstroke}%
\pgfsetdash{}{0pt}%
\pgfpathmoveto{\pgfqpoint{4.994801in}{4.470699in}}%
\pgfpathlineto{\pgfqpoint{4.942017in}{4.467901in}}%
\pgfusepath{stroke}%
\end{pgfscope}%
\begin{pgfscope}%
\pgfpathrectangle{\pgfqpoint{3.352233in}{3.252941in}}{\pgfqpoint{2.407767in}{1.544118in}}%
\pgfusepath{clip}%
\pgfsetbuttcap%
\pgfsetroundjoin%
\pgfsetlinewidth{0.501875pt}%
\definecolor{currentstroke}{rgb}{0.274952,0.037752,0.364543}%
\pgfsetstrokecolor{currentstroke}%
\pgfsetdash{}{0pt}%
\pgfpathmoveto{\pgfqpoint{4.942017in}{4.467901in}}%
\pgfpathlineto{\pgfqpoint{4.890899in}{4.462038in}}%
\pgfusepath{stroke}%
\end{pgfscope}%
\begin{pgfscope}%
\pgfpathrectangle{\pgfqpoint{3.352233in}{3.252941in}}{\pgfqpoint{2.407767in}{1.544118in}}%
\pgfusepath{clip}%
\pgfsetbuttcap%
\pgfsetroundjoin%
\pgfsetlinewidth{0.501875pt}%
\definecolor{currentstroke}{rgb}{0.269944,0.014625,0.341379}%
\pgfsetstrokecolor{currentstroke}%
\pgfsetdash{}{0pt}%
\pgfpathmoveto{\pgfqpoint{5.206279in}{3.642793in}}%
\pgfpathlineto{\pgfqpoint{5.153465in}{3.644668in}}%
\pgfusepath{stroke}%
\end{pgfscope}%
\begin{pgfscope}%
\pgfpathrectangle{\pgfqpoint{3.352233in}{3.252941in}}{\pgfqpoint{2.407767in}{1.544118in}}%
\pgfusepath{clip}%
\pgfsetbuttcap%
\pgfsetroundjoin%
\pgfsetlinewidth{0.501875pt}%
\definecolor{currentstroke}{rgb}{0.272594,0.025563,0.353093}%
\pgfsetstrokecolor{currentstroke}%
\pgfsetdash{}{0pt}%
\pgfpathmoveto{\pgfqpoint{5.153465in}{3.644668in}}%
\pgfpathlineto{\pgfqpoint{5.100614in}{3.646489in}}%
\pgfusepath{stroke}%
\end{pgfscope}%
\begin{pgfscope}%
\pgfpathrectangle{\pgfqpoint{3.352233in}{3.252941in}}{\pgfqpoint{2.407767in}{1.544118in}}%
\pgfusepath{clip}%
\pgfsetbuttcap%
\pgfsetroundjoin%
\pgfsetlinewidth{0.501875pt}%
\definecolor{currentstroke}{rgb}{0.276022,0.044167,0.370164}%
\pgfsetstrokecolor{currentstroke}%
\pgfsetdash{}{0pt}%
\pgfpathmoveto{\pgfqpoint{5.100614in}{3.646489in}}%
\pgfpathlineto{\pgfqpoint{5.047715in}{3.648098in}}%
\pgfusepath{stroke}%
\end{pgfscope}%
\begin{pgfscope}%
\pgfpathrectangle{\pgfqpoint{3.352233in}{3.252941in}}{\pgfqpoint{2.407767in}{1.544118in}}%
\pgfusepath{clip}%
\pgfsetbuttcap%
\pgfsetroundjoin%
\pgfsetlinewidth{0.501875pt}%
\definecolor{currentstroke}{rgb}{0.278791,0.062145,0.386592}%
\pgfsetstrokecolor{currentstroke}%
\pgfsetdash{}{0pt}%
\pgfpathmoveto{\pgfqpoint{5.047715in}{3.648098in}}%
\pgfpathlineto{\pgfqpoint{4.994919in}{3.650703in}}%
\pgfusepath{stroke}%
\end{pgfscope}%
\begin{pgfscope}%
\pgfpathrectangle{\pgfqpoint{3.352233in}{3.252941in}}{\pgfqpoint{2.407767in}{1.544118in}}%
\pgfusepath{clip}%
\pgfsetbuttcap%
\pgfsetroundjoin%
\pgfsetlinewidth{0.501875pt}%
\definecolor{currentstroke}{rgb}{0.278791,0.062145,0.386592}%
\pgfsetstrokecolor{currentstroke}%
\pgfsetdash{}{0pt}%
\pgfpathmoveto{\pgfqpoint{4.994919in}{3.650703in}}%
\pgfpathlineto{\pgfqpoint{4.942251in}{3.654185in}}%
\pgfusepath{stroke}%
\end{pgfscope}%
\begin{pgfscope}%
\pgfpathrectangle{\pgfqpoint{3.352233in}{3.252941in}}{\pgfqpoint{2.407767in}{1.544118in}}%
\pgfusepath{clip}%
\pgfsetbuttcap%
\pgfsetroundjoin%
\pgfsetlinewidth{0.501875pt}%
\definecolor{currentstroke}{rgb}{0.278791,0.062145,0.386592}%
\pgfsetstrokecolor{currentstroke}%
\pgfsetdash{}{0pt}%
\pgfpathmoveto{\pgfqpoint{4.942251in}{3.654185in}}%
\pgfpathlineto{\pgfqpoint{4.889603in}{3.657895in}}%
\pgfusepath{stroke}%
\end{pgfscope}%
\begin{pgfscope}%
\pgfpathrectangle{\pgfqpoint{3.352233in}{3.252941in}}{\pgfqpoint{2.407767in}{1.544118in}}%
\pgfusepath{clip}%
\pgfsetbuttcap%
\pgfsetroundjoin%
\pgfsetlinewidth{0.501875pt}%
\definecolor{currentstroke}{rgb}{0.271305,0.019942,0.347269}%
\pgfsetstrokecolor{currentstroke}%
\pgfsetdash{}{0pt}%
\pgfpathmoveto{\pgfqpoint{5.206279in}{3.677539in}}%
\pgfpathlineto{\pgfqpoint{5.153309in}{3.678016in}}%
\pgfusepath{stroke}%
\end{pgfscope}%
\begin{pgfscope}%
\pgfpathrectangle{\pgfqpoint{3.352233in}{3.252941in}}{\pgfqpoint{2.407767in}{1.544118in}}%
\pgfusepath{clip}%
\pgfsetbuttcap%
\pgfsetroundjoin%
\pgfsetlinewidth{0.501875pt}%
\definecolor{currentstroke}{rgb}{0.274952,0.037752,0.364543}%
\pgfsetstrokecolor{currentstroke}%
\pgfsetdash{}{0pt}%
\pgfpathmoveto{\pgfqpoint{5.153309in}{3.678016in}}%
\pgfpathlineto{\pgfqpoint{5.100358in}{3.678970in}}%
\pgfusepath{stroke}%
\end{pgfscope}%
\begin{pgfscope}%
\pgfpathrectangle{\pgfqpoint{3.352233in}{3.252941in}}{\pgfqpoint{2.407767in}{1.544118in}}%
\pgfusepath{clip}%
\pgfsetbuttcap%
\pgfsetroundjoin%
\pgfsetlinewidth{0.501875pt}%
\definecolor{currentstroke}{rgb}{0.276022,0.044167,0.370164}%
\pgfsetstrokecolor{currentstroke}%
\pgfsetdash{}{0pt}%
\pgfpathmoveto{\pgfqpoint{5.100358in}{3.678970in}}%
\pgfpathlineto{\pgfqpoint{5.047452in}{3.680655in}}%
\pgfusepath{stroke}%
\end{pgfscope}%
\begin{pgfscope}%
\pgfpathrectangle{\pgfqpoint{3.352233in}{3.252941in}}{\pgfqpoint{2.407767in}{1.544118in}}%
\pgfusepath{clip}%
\pgfsetbuttcap%
\pgfsetroundjoin%
\pgfsetlinewidth{0.501875pt}%
\definecolor{currentstroke}{rgb}{0.278791,0.062145,0.386592}%
\pgfsetstrokecolor{currentstroke}%
\pgfsetdash{}{0pt}%
\pgfpathmoveto{\pgfqpoint{5.047452in}{3.680655in}}%
\pgfpathlineto{\pgfqpoint{4.994632in}{3.683188in}}%
\pgfusepath{stroke}%
\end{pgfscope}%
\begin{pgfscope}%
\pgfpathrectangle{\pgfqpoint{3.352233in}{3.252941in}}{\pgfqpoint{2.407767in}{1.544118in}}%
\pgfusepath{clip}%
\pgfsetbuttcap%
\pgfsetroundjoin%
\pgfsetlinewidth{0.501875pt}%
\definecolor{currentstroke}{rgb}{0.280894,0.078907,0.402329}%
\pgfsetstrokecolor{currentstroke}%
\pgfsetdash{}{0pt}%
\pgfpathmoveto{\pgfqpoint{4.994632in}{3.683188in}}%
\pgfpathlineto{\pgfqpoint{4.941884in}{3.686307in}}%
\pgfusepath{stroke}%
\end{pgfscope}%
\begin{pgfscope}%
\pgfpathrectangle{\pgfqpoint{3.352233in}{3.252941in}}{\pgfqpoint{2.407767in}{1.544118in}}%
\pgfusepath{clip}%
\pgfsetbuttcap%
\pgfsetroundjoin%
\pgfsetlinewidth{0.501875pt}%
\definecolor{currentstroke}{rgb}{0.280267,0.073417,0.397163}%
\pgfsetstrokecolor{currentstroke}%
\pgfsetdash{}{0pt}%
\pgfpathmoveto{\pgfqpoint{4.941884in}{3.686307in}}%
\pgfpathlineto{\pgfqpoint{4.889421in}{3.690834in}}%
\pgfusepath{stroke}%
\end{pgfscope}%
\begin{pgfscope}%
\pgfpathrectangle{\pgfqpoint{3.352233in}{3.252941in}}{\pgfqpoint{2.407767in}{1.544118in}}%
\pgfusepath{clip}%
\pgfsetbuttcap%
\pgfsetroundjoin%
\pgfsetlinewidth{0.501875pt}%
\definecolor{currentstroke}{rgb}{0.272594,0.025563,0.353093}%
\pgfsetstrokecolor{currentstroke}%
\pgfsetdash{}{0pt}%
\pgfpathmoveto{\pgfqpoint{5.206279in}{3.712285in}}%
\pgfpathlineto{\pgfqpoint{5.153320in}{3.713128in}}%
\pgfusepath{stroke}%
\end{pgfscope}%
\begin{pgfscope}%
\pgfpathrectangle{\pgfqpoint{3.352233in}{3.252941in}}{\pgfqpoint{2.407767in}{1.544118in}}%
\pgfusepath{clip}%
\pgfsetbuttcap%
\pgfsetroundjoin%
\pgfsetlinewidth{0.501875pt}%
\definecolor{currentstroke}{rgb}{0.276022,0.044167,0.370164}%
\pgfsetstrokecolor{currentstroke}%
\pgfsetdash{}{0pt}%
\pgfpathmoveto{\pgfqpoint{5.153320in}{3.713128in}}%
\pgfpathlineto{\pgfqpoint{5.100369in}{3.714083in}}%
\pgfusepath{stroke}%
\end{pgfscope}%
\begin{pgfscope}%
\pgfpathrectangle{\pgfqpoint{3.352233in}{3.252941in}}{\pgfqpoint{2.407767in}{1.544118in}}%
\pgfusepath{clip}%
\pgfsetbuttcap%
\pgfsetroundjoin%
\pgfsetlinewidth{0.501875pt}%
\definecolor{currentstroke}{rgb}{0.280267,0.073417,0.397163}%
\pgfsetstrokecolor{currentstroke}%
\pgfsetdash{}{0pt}%
\pgfpathmoveto{\pgfqpoint{5.100369in}{3.714083in}}%
\pgfpathlineto{\pgfqpoint{5.047442in}{3.715396in}}%
\pgfusepath{stroke}%
\end{pgfscope}%
\begin{pgfscope}%
\pgfpathrectangle{\pgfqpoint{3.352233in}{3.252941in}}{\pgfqpoint{2.407767in}{1.544118in}}%
\pgfusepath{clip}%
\pgfsetbuttcap%
\pgfsetroundjoin%
\pgfsetlinewidth{0.501875pt}%
\definecolor{currentstroke}{rgb}{0.281446,0.084320,0.407414}%
\pgfsetstrokecolor{currentstroke}%
\pgfsetdash{}{0pt}%
\pgfpathmoveto{\pgfqpoint{5.047442in}{3.715396in}}%
\pgfpathlineto{\pgfqpoint{4.994577in}{3.717359in}}%
\pgfusepath{stroke}%
\end{pgfscope}%
\begin{pgfscope}%
\pgfpathrectangle{\pgfqpoint{3.352233in}{3.252941in}}{\pgfqpoint{2.407767in}{1.544118in}}%
\pgfusepath{clip}%
\pgfsetbuttcap%
\pgfsetroundjoin%
\pgfsetlinewidth{0.501875pt}%
\definecolor{currentstroke}{rgb}{0.280267,0.073417,0.397163}%
\pgfsetstrokecolor{currentstroke}%
\pgfsetdash{}{0pt}%
\pgfpathmoveto{\pgfqpoint{4.994577in}{3.717359in}}%
\pgfpathlineto{\pgfqpoint{4.941767in}{3.719916in}}%
\pgfusepath{stroke}%
\end{pgfscope}%
\begin{pgfscope}%
\pgfpathrectangle{\pgfqpoint{3.352233in}{3.252941in}}{\pgfqpoint{2.407767in}{1.544118in}}%
\pgfusepath{clip}%
\pgfsetbuttcap%
\pgfsetroundjoin%
\pgfsetlinewidth{0.501875pt}%
\definecolor{currentstroke}{rgb}{0.282884,0.135920,0.453427}%
\pgfsetstrokecolor{currentstroke}%
\pgfsetdash{}{0pt}%
\pgfpathmoveto{\pgfqpoint{4.941767in}{3.719916in}}%
\pgfpathlineto{\pgfqpoint{4.889161in}{3.723765in}}%
\pgfusepath{stroke}%
\end{pgfscope}%
\begin{pgfscope}%
\pgfpathrectangle{\pgfqpoint{3.352233in}{3.252941in}}{\pgfqpoint{2.407767in}{1.544118in}}%
\pgfusepath{clip}%
\pgfsetbuttcap%
\pgfsetroundjoin%
\pgfsetlinewidth{0.501875pt}%
\definecolor{currentstroke}{rgb}{0.278791,0.062145,0.386592}%
\pgfsetstrokecolor{currentstroke}%
\pgfsetdash{}{0pt}%
\pgfpathmoveto{\pgfqpoint{4.889161in}{3.723765in}}%
\pgfpathlineto{\pgfqpoint{4.836716in}{3.728421in}}%
\pgfusepath{stroke}%
\end{pgfscope}%
\begin{pgfscope}%
\pgfpathrectangle{\pgfqpoint{3.352233in}{3.252941in}}{\pgfqpoint{2.407767in}{1.544118in}}%
\pgfusepath{clip}%
\pgfsetbuttcap%
\pgfsetroundjoin%
\pgfsetlinewidth{0.501875pt}%
\definecolor{currentstroke}{rgb}{0.272594,0.025563,0.353093}%
\pgfsetstrokecolor{currentstroke}%
\pgfsetdash{}{0pt}%
\pgfpathmoveto{\pgfqpoint{5.206279in}{3.747031in}}%
\pgfpathlineto{\pgfqpoint{5.153304in}{3.747164in}}%
\pgfusepath{stroke}%
\end{pgfscope}%
\begin{pgfscope}%
\pgfpathrectangle{\pgfqpoint{3.352233in}{3.252941in}}{\pgfqpoint{2.407767in}{1.544118in}}%
\pgfusepath{clip}%
\pgfsetbuttcap%
\pgfsetroundjoin%
\pgfsetlinewidth{0.501875pt}%
\definecolor{currentstroke}{rgb}{0.277018,0.050344,0.375715}%
\pgfsetstrokecolor{currentstroke}%
\pgfsetdash{}{0pt}%
\pgfpathmoveto{\pgfqpoint{5.153304in}{3.747164in}}%
\pgfpathlineto{\pgfqpoint{5.100337in}{3.747637in}}%
\pgfusepath{stroke}%
\end{pgfscope}%
\begin{pgfscope}%
\pgfpathrectangle{\pgfqpoint{3.352233in}{3.252941in}}{\pgfqpoint{2.407767in}{1.544118in}}%
\pgfusepath{clip}%
\pgfsetbuttcap%
\pgfsetroundjoin%
\pgfsetlinewidth{0.501875pt}%
\definecolor{currentstroke}{rgb}{0.280894,0.078907,0.402329}%
\pgfsetstrokecolor{currentstroke}%
\pgfsetdash{}{0pt}%
\pgfpathmoveto{\pgfqpoint{5.100337in}{3.747637in}}%
\pgfpathlineto{\pgfqpoint{5.047430in}{3.749033in}}%
\pgfusepath{stroke}%
\end{pgfscope}%
\begin{pgfscope}%
\pgfpathrectangle{\pgfqpoint{3.352233in}{3.252941in}}{\pgfqpoint{2.407767in}{1.544118in}}%
\pgfusepath{clip}%
\pgfsetbuttcap%
\pgfsetroundjoin%
\pgfsetlinewidth{0.501875pt}%
\definecolor{currentstroke}{rgb}{0.282656,0.100196,0.422160}%
\pgfsetstrokecolor{currentstroke}%
\pgfsetdash{}{0pt}%
\pgfpathmoveto{\pgfqpoint{5.047430in}{3.749033in}}%
\pgfpathlineto{\pgfqpoint{4.994608in}{3.751515in}}%
\pgfusepath{stroke}%
\end{pgfscope}%
\begin{pgfscope}%
\pgfpathrectangle{\pgfqpoint{3.352233in}{3.252941in}}{\pgfqpoint{2.407767in}{1.544118in}}%
\pgfusepath{clip}%
\pgfsetbuttcap%
\pgfsetroundjoin%
\pgfsetlinewidth{0.501875pt}%
\definecolor{currentstroke}{rgb}{0.283072,0.130895,0.449241}%
\pgfsetstrokecolor{currentstroke}%
\pgfsetdash{}{0pt}%
\pgfpathmoveto{\pgfqpoint{4.994608in}{3.751515in}}%
\pgfpathlineto{\pgfqpoint{4.941842in}{3.754518in}}%
\pgfusepath{stroke}%
\end{pgfscope}%
\begin{pgfscope}%
\pgfpathrectangle{\pgfqpoint{3.352233in}{3.252941in}}{\pgfqpoint{2.407767in}{1.544118in}}%
\pgfusepath{clip}%
\pgfsetbuttcap%
\pgfsetroundjoin%
\pgfsetlinewidth{0.501875pt}%
\definecolor{currentstroke}{rgb}{0.281887,0.150881,0.465405}%
\pgfsetstrokecolor{currentstroke}%
\pgfsetdash{}{0pt}%
\pgfpathmoveto{\pgfqpoint{4.941842in}{3.754518in}}%
\pgfpathlineto{\pgfqpoint{4.889238in}{3.758452in}}%
\pgfusepath{stroke}%
\end{pgfscope}%
\begin{pgfscope}%
\pgfpathrectangle{\pgfqpoint{3.352233in}{3.252941in}}{\pgfqpoint{2.407767in}{1.544118in}}%
\pgfusepath{clip}%
\pgfsetbuttcap%
\pgfsetroundjoin%
\pgfsetlinewidth{0.501875pt}%
\definecolor{currentstroke}{rgb}{0.283072,0.130895,0.449241}%
\pgfsetstrokecolor{currentstroke}%
\pgfsetdash{}{0pt}%
\pgfpathmoveto{\pgfqpoint{4.889238in}{3.758452in}}%
\pgfpathlineto{\pgfqpoint{4.836950in}{3.763844in}}%
\pgfusepath{stroke}%
\end{pgfscope}%
\begin{pgfscope}%
\pgfpathrectangle{\pgfqpoint{3.352233in}{3.252941in}}{\pgfqpoint{2.407767in}{1.544118in}}%
\pgfusepath{clip}%
\pgfsetbuttcap%
\pgfsetroundjoin%
\pgfsetlinewidth{0.501875pt}%
\definecolor{currentstroke}{rgb}{0.282656,0.100196,0.422160}%
\pgfsetstrokecolor{currentstroke}%
\pgfsetdash{}{0pt}%
\pgfpathmoveto{\pgfqpoint{4.836950in}{3.763844in}}%
\pgfpathlineto{\pgfqpoint{4.785009in}{3.770436in}}%
\pgfusepath{stroke}%
\end{pgfscope}%
\begin{pgfscope}%
\pgfpathrectangle{\pgfqpoint{3.352233in}{3.252941in}}{\pgfqpoint{2.407767in}{1.544118in}}%
\pgfusepath{clip}%
\pgfsetbuttcap%
\pgfsetroundjoin%
\pgfsetlinewidth{0.501875pt}%
\definecolor{currentstroke}{rgb}{0.283091,0.110553,0.431554}%
\pgfsetstrokecolor{currentstroke}%
\pgfsetdash{}{0pt}%
\pgfpathmoveto{\pgfqpoint{4.785009in}{3.770436in}}%
\pgfpathlineto{\pgfqpoint{4.733581in}{3.778461in}}%
\pgfusepath{stroke}%
\end{pgfscope}%
\begin{pgfscope}%
\pgfpathrectangle{\pgfqpoint{3.352233in}{3.252941in}}{\pgfqpoint{2.407767in}{1.544118in}}%
\pgfusepath{clip}%
\pgfsetbuttcap%
\pgfsetroundjoin%
\pgfsetlinewidth{0.501875pt}%
\definecolor{currentstroke}{rgb}{0.273809,0.031497,0.358853}%
\pgfsetstrokecolor{currentstroke}%
\pgfsetdash{}{0pt}%
\pgfpathmoveto{\pgfqpoint{5.206279in}{3.781777in}}%
\pgfpathlineto{\pgfqpoint{5.153308in}{3.782152in}}%
\pgfusepath{stroke}%
\end{pgfscope}%
\begin{pgfscope}%
\pgfpathrectangle{\pgfqpoint{3.352233in}{3.252941in}}{\pgfqpoint{2.407767in}{1.544118in}}%
\pgfusepath{clip}%
\pgfsetbuttcap%
\pgfsetroundjoin%
\pgfsetlinewidth{0.501875pt}%
\definecolor{currentstroke}{rgb}{0.277941,0.056324,0.381191}%
\pgfsetstrokecolor{currentstroke}%
\pgfsetdash{}{0pt}%
\pgfpathmoveto{\pgfqpoint{5.153308in}{3.782152in}}%
\pgfpathlineto{\pgfqpoint{5.100358in}{3.783115in}}%
\pgfusepath{stroke}%
\end{pgfscope}%
\begin{pgfscope}%
\pgfpathrectangle{\pgfqpoint{3.352233in}{3.252941in}}{\pgfqpoint{2.407767in}{1.544118in}}%
\pgfusepath{clip}%
\pgfsetbuttcap%
\pgfsetroundjoin%
\pgfsetlinewidth{0.501875pt}%
\definecolor{currentstroke}{rgb}{0.282656,0.100196,0.422160}%
\pgfsetstrokecolor{currentstroke}%
\pgfsetdash{}{0pt}%
\pgfpathmoveto{\pgfqpoint{5.100358in}{3.783115in}}%
\pgfpathlineto{\pgfqpoint{5.047448in}{3.784752in}}%
\pgfusepath{stroke}%
\end{pgfscope}%
\begin{pgfscope}%
\pgfpathrectangle{\pgfqpoint{3.352233in}{3.252941in}}{\pgfqpoint{2.407767in}{1.544118in}}%
\pgfusepath{clip}%
\pgfsetbuttcap%
\pgfsetroundjoin%
\pgfsetlinewidth{0.501875pt}%
\definecolor{currentstroke}{rgb}{0.283187,0.125848,0.444960}%
\pgfsetstrokecolor{currentstroke}%
\pgfsetdash{}{0pt}%
\pgfpathmoveto{\pgfqpoint{5.047448in}{3.784752in}}%
\pgfpathlineto{\pgfqpoint{4.994573in}{3.786795in}}%
\pgfusepath{stroke}%
\end{pgfscope}%
\begin{pgfscope}%
\pgfpathrectangle{\pgfqpoint{3.352233in}{3.252941in}}{\pgfqpoint{2.407767in}{1.544118in}}%
\pgfusepath{clip}%
\pgfsetbuttcap%
\pgfsetroundjoin%
\pgfsetlinewidth{0.501875pt}%
\definecolor{currentstroke}{rgb}{0.282884,0.135920,0.453427}%
\pgfsetstrokecolor{currentstroke}%
\pgfsetdash{}{0pt}%
\pgfpathmoveto{\pgfqpoint{4.994573in}{3.786795in}}%
\pgfpathlineto{\pgfqpoint{4.941756in}{3.789372in}}%
\pgfusepath{stroke}%
\end{pgfscope}%
\begin{pgfscope}%
\pgfpathrectangle{\pgfqpoint{3.352233in}{3.252941in}}{\pgfqpoint{2.407767in}{1.544118in}}%
\pgfusepath{clip}%
\pgfsetbuttcap%
\pgfsetroundjoin%
\pgfsetlinewidth{0.501875pt}%
\definecolor{currentstroke}{rgb}{0.282290,0.145912,0.461510}%
\pgfsetstrokecolor{currentstroke}%
\pgfsetdash{}{0pt}%
\pgfpathmoveto{\pgfqpoint{4.941756in}{3.789372in}}%
\pgfpathlineto{\pgfqpoint{4.889046in}{3.792734in}}%
\pgfusepath{stroke}%
\end{pgfscope}%
\begin{pgfscope}%
\pgfpathrectangle{\pgfqpoint{3.352233in}{3.252941in}}{\pgfqpoint{2.407767in}{1.544118in}}%
\pgfusepath{clip}%
\pgfsetbuttcap%
\pgfsetroundjoin%
\pgfsetlinewidth{0.501875pt}%
\definecolor{currentstroke}{rgb}{0.273809,0.031497,0.358853}%
\pgfsetstrokecolor{currentstroke}%
\pgfsetdash{}{0pt}%
\pgfpathmoveto{\pgfqpoint{5.206279in}{3.816523in}}%
\pgfpathlineto{\pgfqpoint{5.153308in}{3.816926in}}%
\pgfusepath{stroke}%
\end{pgfscope}%
\begin{pgfscope}%
\pgfpathrectangle{\pgfqpoint{3.352233in}{3.252941in}}{\pgfqpoint{2.407767in}{1.544118in}}%
\pgfusepath{clip}%
\pgfsetbuttcap%
\pgfsetroundjoin%
\pgfsetlinewidth{0.501875pt}%
\definecolor{currentstroke}{rgb}{0.280894,0.078907,0.402329}%
\pgfsetstrokecolor{currentstroke}%
\pgfsetdash{}{0pt}%
\pgfpathmoveto{\pgfqpoint{5.153308in}{3.816926in}}%
\pgfpathlineto{\pgfqpoint{5.100345in}{3.817610in}}%
\pgfusepath{stroke}%
\end{pgfscope}%
\begin{pgfscope}%
\pgfpathrectangle{\pgfqpoint{3.352233in}{3.252941in}}{\pgfqpoint{2.407767in}{1.544118in}}%
\pgfusepath{clip}%
\pgfsetbuttcap%
\pgfsetroundjoin%
\pgfsetlinewidth{0.501875pt}%
\definecolor{currentstroke}{rgb}{0.283197,0.115680,0.436115}%
\pgfsetstrokecolor{currentstroke}%
\pgfsetdash{}{0pt}%
\pgfpathmoveto{\pgfqpoint{5.100345in}{3.817610in}}%
\pgfpathlineto{\pgfqpoint{5.047401in}{3.818751in}}%
\pgfusepath{stroke}%
\end{pgfscope}%
\begin{pgfscope}%
\pgfpathrectangle{\pgfqpoint{3.352233in}{3.252941in}}{\pgfqpoint{2.407767in}{1.544118in}}%
\pgfusepath{clip}%
\pgfsetbuttcap%
\pgfsetroundjoin%
\pgfsetlinewidth{0.501875pt}%
\definecolor{currentstroke}{rgb}{0.282623,0.140926,0.457517}%
\pgfsetstrokecolor{currentstroke}%
\pgfsetdash{}{0pt}%
\pgfpathmoveto{\pgfqpoint{5.047401in}{3.818751in}}%
\pgfpathlineto{\pgfqpoint{4.994481in}{3.820301in}}%
\pgfusepath{stroke}%
\end{pgfscope}%
\begin{pgfscope}%
\pgfpathrectangle{\pgfqpoint{3.352233in}{3.252941in}}{\pgfqpoint{2.407767in}{1.544118in}}%
\pgfusepath{clip}%
\pgfsetbuttcap%
\pgfsetroundjoin%
\pgfsetlinewidth{0.501875pt}%
\definecolor{currentstroke}{rgb}{0.280255,0.165693,0.476498}%
\pgfsetstrokecolor{currentstroke}%
\pgfsetdash{}{0pt}%
\pgfpathmoveto{\pgfqpoint{4.994481in}{3.820301in}}%
\pgfpathlineto{\pgfqpoint{4.941632in}{3.822582in}}%
\pgfusepath{stroke}%
\end{pgfscope}%
\begin{pgfscope}%
\pgfpathrectangle{\pgfqpoint{3.352233in}{3.252941in}}{\pgfqpoint{2.407767in}{1.544118in}}%
\pgfusepath{clip}%
\pgfsetbuttcap%
\pgfsetroundjoin%
\pgfsetlinewidth{0.501875pt}%
\definecolor{currentstroke}{rgb}{0.271828,0.209303,0.504434}%
\pgfsetstrokecolor{currentstroke}%
\pgfsetdash{}{0pt}%
\pgfpathmoveto{\pgfqpoint{4.941632in}{3.822582in}}%
\pgfpathlineto{\pgfqpoint{4.888978in}{3.826218in}}%
\pgfusepath{stroke}%
\end{pgfscope}%
\begin{pgfscope}%
\pgfpathrectangle{\pgfqpoint{3.352233in}{3.252941in}}{\pgfqpoint{2.407767in}{1.544118in}}%
\pgfusepath{clip}%
\pgfsetbuttcap%
\pgfsetroundjoin%
\pgfsetlinewidth{0.501875pt}%
\definecolor{currentstroke}{rgb}{0.270595,0.214069,0.507052}%
\pgfsetstrokecolor{currentstroke}%
\pgfsetdash{}{0pt}%
\pgfpathmoveto{\pgfqpoint{4.888978in}{3.826218in}}%
\pgfpathlineto{\pgfqpoint{4.836482in}{3.830776in}}%
\pgfusepath{stroke}%
\end{pgfscope}%
\begin{pgfscope}%
\pgfpathrectangle{\pgfqpoint{3.352233in}{3.252941in}}{\pgfqpoint{2.407767in}{1.544118in}}%
\pgfusepath{clip}%
\pgfsetbuttcap%
\pgfsetroundjoin%
\pgfsetlinewidth{0.501875pt}%
\definecolor{currentstroke}{rgb}{0.252194,0.269783,0.531579}%
\pgfsetstrokecolor{currentstroke}%
\pgfsetdash{}{0pt}%
\pgfpathmoveto{\pgfqpoint{4.836482in}{3.830776in}}%
\pgfpathlineto{\pgfqpoint{4.784123in}{3.835909in}}%
\pgfusepath{stroke}%
\end{pgfscope}%
\begin{pgfscope}%
\pgfpathrectangle{\pgfqpoint{3.352233in}{3.252941in}}{\pgfqpoint{2.407767in}{1.544118in}}%
\pgfusepath{clip}%
\pgfsetbuttcap%
\pgfsetroundjoin%
\pgfsetlinewidth{0.501875pt}%
\definecolor{currentstroke}{rgb}{0.248629,0.278775,0.534556}%
\pgfsetstrokecolor{currentstroke}%
\pgfsetdash{}{0pt}%
\pgfpathmoveto{\pgfqpoint{4.784123in}{3.835909in}}%
\pgfpathlineto{\pgfqpoint{4.732094in}{3.842258in}}%
\pgfusepath{stroke}%
\end{pgfscope}%
\begin{pgfscope}%
\pgfpathrectangle{\pgfqpoint{3.352233in}{3.252941in}}{\pgfqpoint{2.407767in}{1.544118in}}%
\pgfusepath{clip}%
\pgfsetbuttcap%
\pgfsetroundjoin%
\pgfsetlinewidth{0.501875pt}%
\definecolor{currentstroke}{rgb}{0.210503,0.363727,0.552206}%
\pgfsetstrokecolor{currentstroke}%
\pgfsetdash{}{0pt}%
\pgfpathmoveto{\pgfqpoint{4.732094in}{3.842258in}}%
\pgfpathlineto{\pgfqpoint{4.680530in}{3.850008in}}%
\pgfusepath{stroke}%
\end{pgfscope}%
\begin{pgfscope}%
\pgfpathrectangle{\pgfqpoint{3.352233in}{3.252941in}}{\pgfqpoint{2.407767in}{1.544118in}}%
\pgfusepath{clip}%
\pgfsetbuttcap%
\pgfsetroundjoin%
\pgfsetlinewidth{0.501875pt}%
\definecolor{currentstroke}{rgb}{0.187231,0.414746,0.556547}%
\pgfsetstrokecolor{currentstroke}%
\pgfsetdash{}{0pt}%
\pgfpathmoveto{\pgfqpoint{4.680530in}{3.850008in}}%
\pgfpathlineto{\pgfqpoint{4.629557in}{3.859221in}}%
\pgfusepath{stroke}%
\end{pgfscope}%
\begin{pgfscope}%
\pgfpathrectangle{\pgfqpoint{3.352233in}{3.252941in}}{\pgfqpoint{2.407767in}{1.544118in}}%
\pgfusepath{clip}%
\pgfsetbuttcap%
\pgfsetroundjoin%
\pgfsetlinewidth{0.501875pt}%
\definecolor{currentstroke}{rgb}{0.194100,0.399323,0.555565}%
\pgfsetstrokecolor{currentstroke}%
\pgfsetdash{}{0pt}%
\pgfpathmoveto{\pgfqpoint{4.629557in}{3.859221in}}%
\pgfpathlineto{\pgfqpoint{4.579228in}{3.869796in}}%
\pgfusepath{stroke}%
\end{pgfscope}%
\begin{pgfscope}%
\pgfpathrectangle{\pgfqpoint{3.352233in}{3.252941in}}{\pgfqpoint{2.407767in}{1.544118in}}%
\pgfusepath{clip}%
\pgfsetbuttcap%
\pgfsetroundjoin%
\pgfsetlinewidth{0.501875pt}%
\definecolor{currentstroke}{rgb}{0.179019,0.433756,0.557430}%
\pgfsetstrokecolor{currentstroke}%
\pgfsetdash{}{0pt}%
\pgfpathmoveto{\pgfqpoint{4.579228in}{3.869796in}}%
\pgfpathlineto{\pgfqpoint{4.529541in}{3.881547in}}%
\pgfusepath{stroke}%
\end{pgfscope}%
\begin{pgfscope}%
\pgfpathrectangle{\pgfqpoint{3.352233in}{3.252941in}}{\pgfqpoint{2.407767in}{1.544118in}}%
\pgfusepath{clip}%
\pgfsetbuttcap%
\pgfsetroundjoin%
\pgfsetlinewidth{0.501875pt}%
\definecolor{currentstroke}{rgb}{0.151918,0.500685,0.557587}%
\pgfsetstrokecolor{currentstroke}%
\pgfsetdash{}{0pt}%
\pgfpathmoveto{\pgfqpoint{4.529541in}{3.881547in}}%
\pgfpathlineto{\pgfqpoint{4.480214in}{3.893905in}}%
\pgfusepath{stroke}%
\end{pgfscope}%
\begin{pgfscope}%
\pgfpathrectangle{\pgfqpoint{3.352233in}{3.252941in}}{\pgfqpoint{2.407767in}{1.544118in}}%
\pgfusepath{clip}%
\pgfsetbuttcap%
\pgfsetroundjoin%
\pgfsetlinewidth{0.501875pt}%
\definecolor{currentstroke}{rgb}{0.122606,0.585371,0.546557}%
\pgfsetstrokecolor{currentstroke}%
\pgfsetdash{}{0pt}%
\pgfpathmoveto{\pgfqpoint{4.480214in}{3.893905in}}%
\pgfpathlineto{\pgfqpoint{4.431034in}{3.906489in}}%
\pgfusepath{stroke}%
\end{pgfscope}%
\begin{pgfscope}%
\pgfpathrectangle{\pgfqpoint{3.352233in}{3.252941in}}{\pgfqpoint{2.407767in}{1.544118in}}%
\pgfusepath{clip}%
\pgfsetbuttcap%
\pgfsetroundjoin%
\pgfsetlinewidth{0.501875pt}%
\definecolor{currentstroke}{rgb}{0.202219,0.715272,0.476084}%
\pgfsetstrokecolor{currentstroke}%
\pgfsetdash{}{0pt}%
\pgfpathmoveto{\pgfqpoint{4.431034in}{3.906489in}}%
\pgfpathlineto{\pgfqpoint{4.381982in}{3.919291in}}%
\pgfusepath{stroke}%
\end{pgfscope}%
\begin{pgfscope}%
\pgfpathrectangle{\pgfqpoint{3.352233in}{3.252941in}}{\pgfqpoint{2.407767in}{1.544118in}}%
\pgfusepath{clip}%
\pgfsetbuttcap%
\pgfsetroundjoin%
\pgfsetlinewidth{0.501875pt}%
\definecolor{currentstroke}{rgb}{0.430983,0.808473,0.346476}%
\pgfsetstrokecolor{currentstroke}%
\pgfsetdash{}{0pt}%
\pgfpathmoveto{\pgfqpoint{4.381982in}{3.919291in}}%
\pgfpathlineto{\pgfqpoint{4.332833in}{3.931950in}}%
\pgfusepath{stroke}%
\end{pgfscope}%
\begin{pgfscope}%
\pgfpathrectangle{\pgfqpoint{3.352233in}{3.252941in}}{\pgfqpoint{2.407767in}{1.544118in}}%
\pgfusepath{clip}%
\pgfsetbuttcap%
\pgfsetroundjoin%
\pgfsetlinewidth{0.501875pt}%
\definecolor{currentstroke}{rgb}{0.274952,0.037752,0.364543}%
\pgfsetstrokecolor{currentstroke}%
\pgfsetdash{}{0pt}%
\pgfpathmoveto{\pgfqpoint{5.206279in}{3.851269in}}%
\pgfpathlineto{\pgfqpoint{5.153306in}{3.851557in}}%
\pgfusepath{stroke}%
\end{pgfscope}%
\begin{pgfscope}%
\pgfpathrectangle{\pgfqpoint{3.352233in}{3.252941in}}{\pgfqpoint{2.407767in}{1.544118in}}%
\pgfusepath{clip}%
\pgfsetbuttcap%
\pgfsetroundjoin%
\pgfsetlinewidth{0.501875pt}%
\definecolor{currentstroke}{rgb}{0.281446,0.084320,0.407414}%
\pgfsetstrokecolor{currentstroke}%
\pgfsetdash{}{0pt}%
\pgfpathmoveto{\pgfqpoint{5.153306in}{3.851557in}}%
\pgfpathlineto{\pgfqpoint{5.100339in}{3.852159in}}%
\pgfusepath{stroke}%
\end{pgfscope}%
\begin{pgfscope}%
\pgfpathrectangle{\pgfqpoint{3.352233in}{3.252941in}}{\pgfqpoint{2.407767in}{1.544118in}}%
\pgfusepath{clip}%
\pgfsetbuttcap%
\pgfsetroundjoin%
\pgfsetlinewidth{0.501875pt}%
\definecolor{currentstroke}{rgb}{0.283187,0.125848,0.444960}%
\pgfsetstrokecolor{currentstroke}%
\pgfsetdash{}{0pt}%
\pgfpathmoveto{\pgfqpoint{5.100339in}{3.852159in}}%
\pgfpathlineto{\pgfqpoint{5.047384in}{3.853067in}}%
\pgfusepath{stroke}%
\end{pgfscope}%
\begin{pgfscope}%
\pgfpathrectangle{\pgfqpoint{3.352233in}{3.252941in}}{\pgfqpoint{2.407767in}{1.544118in}}%
\pgfusepath{clip}%
\pgfsetbuttcap%
\pgfsetroundjoin%
\pgfsetlinewidth{0.501875pt}%
\definecolor{currentstroke}{rgb}{0.280868,0.160771,0.472899}%
\pgfsetstrokecolor{currentstroke}%
\pgfsetdash{}{0pt}%
\pgfpathmoveto{\pgfqpoint{5.047384in}{3.853067in}}%
\pgfpathlineto{\pgfqpoint{4.994453in}{3.854421in}}%
\pgfusepath{stroke}%
\end{pgfscope}%
\begin{pgfscope}%
\pgfpathrectangle{\pgfqpoint{3.352233in}{3.252941in}}{\pgfqpoint{2.407767in}{1.544118in}}%
\pgfusepath{clip}%
\pgfsetbuttcap%
\pgfsetroundjoin%
\pgfsetlinewidth{0.501875pt}%
\definecolor{currentstroke}{rgb}{0.274128,0.199721,0.498911}%
\pgfsetstrokecolor{currentstroke}%
\pgfsetdash{}{0pt}%
\pgfpathmoveto{\pgfqpoint{4.994453in}{3.854421in}}%
\pgfpathlineto{\pgfqpoint{4.941558in}{3.856279in}}%
\pgfusepath{stroke}%
\end{pgfscope}%
\begin{pgfscope}%
\pgfpathrectangle{\pgfqpoint{3.352233in}{3.252941in}}{\pgfqpoint{2.407767in}{1.544118in}}%
\pgfusepath{clip}%
\pgfsetbuttcap%
\pgfsetroundjoin%
\pgfsetlinewidth{0.501875pt}%
\definecolor{currentstroke}{rgb}{0.270595,0.214069,0.507052}%
\pgfsetstrokecolor{currentstroke}%
\pgfsetdash{}{0pt}%
\pgfpathmoveto{\pgfqpoint{4.941558in}{3.856279in}}%
\pgfpathlineto{\pgfqpoint{4.888750in}{3.858925in}}%
\pgfusepath{stroke}%
\end{pgfscope}%
\begin{pgfscope}%
\pgfpathrectangle{\pgfqpoint{3.352233in}{3.252941in}}{\pgfqpoint{2.407767in}{1.544118in}}%
\pgfusepath{clip}%
\pgfsetbuttcap%
\pgfsetroundjoin%
\pgfsetlinewidth{0.501875pt}%
\definecolor{currentstroke}{rgb}{0.243113,0.292092,0.538516}%
\pgfsetstrokecolor{currentstroke}%
\pgfsetdash{}{0pt}%
\pgfpathmoveto{\pgfqpoint{4.888750in}{3.858925in}}%
\pgfpathlineto{\pgfqpoint{4.836085in}{3.862561in}}%
\pgfusepath{stroke}%
\end{pgfscope}%
\begin{pgfscope}%
\pgfpathrectangle{\pgfqpoint{3.352233in}{3.252941in}}{\pgfqpoint{2.407767in}{1.544118in}}%
\pgfusepath{clip}%
\pgfsetbuttcap%
\pgfsetroundjoin%
\pgfsetlinewidth{0.501875pt}%
\definecolor{currentstroke}{rgb}{0.276022,0.044167,0.370164}%
\pgfsetstrokecolor{currentstroke}%
\pgfsetdash{}{0pt}%
\pgfpathmoveto{\pgfqpoint{5.206279in}{3.886016in}}%
\pgfpathlineto{\pgfqpoint{5.153305in}{3.886269in}}%
\pgfusepath{stroke}%
\end{pgfscope}%
\begin{pgfscope}%
\pgfpathrectangle{\pgfqpoint{3.352233in}{3.252941in}}{\pgfqpoint{2.407767in}{1.544118in}}%
\pgfusepath{clip}%
\pgfsetbuttcap%
\pgfsetroundjoin%
\pgfsetlinewidth{0.501875pt}%
\definecolor{currentstroke}{rgb}{0.282327,0.094955,0.417331}%
\pgfsetstrokecolor{currentstroke}%
\pgfsetdash{}{0pt}%
\pgfpathmoveto{\pgfqpoint{5.153305in}{3.886269in}}%
\pgfpathlineto{\pgfqpoint{5.100337in}{3.886818in}}%
\pgfusepath{stroke}%
\end{pgfscope}%
\begin{pgfscope}%
\pgfpathrectangle{\pgfqpoint{3.352233in}{3.252941in}}{\pgfqpoint{2.407767in}{1.544118in}}%
\pgfusepath{clip}%
\pgfsetbuttcap%
\pgfsetroundjoin%
\pgfsetlinewidth{0.501875pt}%
\definecolor{currentstroke}{rgb}{0.282290,0.145912,0.461510}%
\pgfsetstrokecolor{currentstroke}%
\pgfsetdash{}{0pt}%
\pgfpathmoveto{\pgfqpoint{5.100337in}{3.886818in}}%
\pgfpathlineto{\pgfqpoint{5.047372in}{3.887496in}}%
\pgfusepath{stroke}%
\end{pgfscope}%
\begin{pgfscope}%
\pgfpathrectangle{\pgfqpoint{3.352233in}{3.252941in}}{\pgfqpoint{2.407767in}{1.544118in}}%
\pgfusepath{clip}%
\pgfsetbuttcap%
\pgfsetroundjoin%
\pgfsetlinewidth{0.501875pt}%
\definecolor{currentstroke}{rgb}{0.277134,0.185228,0.489898}%
\pgfsetstrokecolor{currentstroke}%
\pgfsetdash{}{0pt}%
\pgfpathmoveto{\pgfqpoint{5.047372in}{3.887496in}}%
\pgfpathlineto{\pgfqpoint{4.994423in}{3.888531in}}%
\pgfusepath{stroke}%
\end{pgfscope}%
\begin{pgfscope}%
\pgfpathrectangle{\pgfqpoint{3.352233in}{3.252941in}}{\pgfqpoint{2.407767in}{1.544118in}}%
\pgfusepath{clip}%
\pgfsetbuttcap%
\pgfsetroundjoin%
\pgfsetlinewidth{0.501875pt}%
\definecolor{currentstroke}{rgb}{0.262138,0.242286,0.520837}%
\pgfsetstrokecolor{currentstroke}%
\pgfsetdash{}{0pt}%
\pgfpathmoveto{\pgfqpoint{4.994423in}{3.888531in}}%
\pgfpathlineto{\pgfqpoint{4.941504in}{3.890086in}}%
\pgfusepath{stroke}%
\end{pgfscope}%
\begin{pgfscope}%
\pgfpathrectangle{\pgfqpoint{3.352233in}{3.252941in}}{\pgfqpoint{2.407767in}{1.544118in}}%
\pgfusepath{clip}%
\pgfsetbuttcap%
\pgfsetroundjoin%
\pgfsetlinewidth{0.501875pt}%
\definecolor{currentstroke}{rgb}{0.231674,0.318106,0.544834}%
\pgfsetstrokecolor{currentstroke}%
\pgfsetdash{}{0pt}%
\pgfpathmoveto{\pgfqpoint{4.941504in}{3.890086in}}%
\pgfpathlineto{\pgfqpoint{4.888647in}{3.892314in}}%
\pgfusepath{stroke}%
\end{pgfscope}%
\begin{pgfscope}%
\pgfpathrectangle{\pgfqpoint{3.352233in}{3.252941in}}{\pgfqpoint{2.407767in}{1.544118in}}%
\pgfusepath{clip}%
\pgfsetbuttcap%
\pgfsetroundjoin%
\pgfsetlinewidth{0.501875pt}%
\definecolor{currentstroke}{rgb}{0.218130,0.347432,0.550038}%
\pgfsetstrokecolor{currentstroke}%
\pgfsetdash{}{0pt}%
\pgfpathmoveto{\pgfqpoint{4.888647in}{3.892314in}}%
\pgfpathlineto{\pgfqpoint{4.835890in}{3.895361in}}%
\pgfusepath{stroke}%
\end{pgfscope}%
\begin{pgfscope}%
\pgfpathrectangle{\pgfqpoint{3.352233in}{3.252941in}}{\pgfqpoint{2.407767in}{1.544118in}}%
\pgfusepath{clip}%
\pgfsetbuttcap%
\pgfsetroundjoin%
\pgfsetlinewidth{0.501875pt}%
\definecolor{currentstroke}{rgb}{0.174274,0.445044,0.557792}%
\pgfsetstrokecolor{currentstroke}%
\pgfsetdash{}{0pt}%
\pgfpathmoveto{\pgfqpoint{4.835890in}{3.895361in}}%
\pgfpathlineto{\pgfqpoint{4.783264in}{3.899224in}}%
\pgfusepath{stroke}%
\end{pgfscope}%
\begin{pgfscope}%
\pgfpathrectangle{\pgfqpoint{3.352233in}{3.252941in}}{\pgfqpoint{2.407767in}{1.544118in}}%
\pgfusepath{clip}%
\pgfsetbuttcap%
\pgfsetroundjoin%
\pgfsetlinewidth{0.501875pt}%
\definecolor{currentstroke}{rgb}{0.150476,0.504369,0.557430}%
\pgfsetstrokecolor{currentstroke}%
\pgfsetdash{}{0pt}%
\pgfpathmoveto{\pgfqpoint{4.783264in}{3.899224in}}%
\pgfpathlineto{\pgfqpoint{4.730801in}{3.903910in}}%
\pgfusepath{stroke}%
\end{pgfscope}%
\begin{pgfscope}%
\pgfpathrectangle{\pgfqpoint{3.352233in}{3.252941in}}{\pgfqpoint{2.407767in}{1.544118in}}%
\pgfusepath{clip}%
\pgfsetbuttcap%
\pgfsetroundjoin%
\pgfsetlinewidth{0.501875pt}%
\definecolor{currentstroke}{rgb}{0.143343,0.522773,0.556295}%
\pgfsetstrokecolor{currentstroke}%
\pgfsetdash{}{0pt}%
\pgfpathmoveto{\pgfqpoint{4.730801in}{3.903910in}}%
\pgfpathlineto{\pgfqpoint{4.678554in}{3.909498in}}%
\pgfusepath{stroke}%
\end{pgfscope}%
\begin{pgfscope}%
\pgfpathrectangle{\pgfqpoint{3.352233in}{3.252941in}}{\pgfqpoint{2.407767in}{1.544118in}}%
\pgfusepath{clip}%
\pgfsetbuttcap%
\pgfsetroundjoin%
\pgfsetlinewidth{0.501875pt}%
\definecolor{currentstroke}{rgb}{0.137770,0.537492,0.554906}%
\pgfsetstrokecolor{currentstroke}%
\pgfsetdash{}{0pt}%
\pgfpathmoveto{\pgfqpoint{4.678554in}{3.909498in}}%
\pgfpathlineto{\pgfqpoint{4.626562in}{3.915995in}}%
\pgfusepath{stroke}%
\end{pgfscope}%
\begin{pgfscope}%
\pgfpathrectangle{\pgfqpoint{3.352233in}{3.252941in}}{\pgfqpoint{2.407767in}{1.544118in}}%
\pgfusepath{clip}%
\pgfsetbuttcap%
\pgfsetroundjoin%
\pgfsetlinewidth{0.501875pt}%
\definecolor{currentstroke}{rgb}{0.140536,0.530132,0.555659}%
\pgfsetstrokecolor{currentstroke}%
\pgfsetdash{}{0pt}%
\pgfpathmoveto{\pgfqpoint{4.626562in}{3.915995in}}%
\pgfpathlineto{\pgfqpoint{4.574826in}{3.923273in}}%
\pgfusepath{stroke}%
\end{pgfscope}%
\begin{pgfscope}%
\pgfpathrectangle{\pgfqpoint{3.352233in}{3.252941in}}{\pgfqpoint{2.407767in}{1.544118in}}%
\pgfusepath{clip}%
\pgfsetbuttcap%
\pgfsetroundjoin%
\pgfsetlinewidth{0.501875pt}%
\definecolor{currentstroke}{rgb}{0.137339,0.662252,0.515571}%
\pgfsetstrokecolor{currentstroke}%
\pgfsetdash{}{0pt}%
\pgfpathmoveto{\pgfqpoint{4.574826in}{3.923273in}}%
\pgfpathlineto{\pgfqpoint{4.523266in}{3.931048in}}%
\pgfusepath{stroke}%
\end{pgfscope}%
\begin{pgfscope}%
\pgfpathrectangle{\pgfqpoint{3.352233in}{3.252941in}}{\pgfqpoint{2.407767in}{1.544118in}}%
\pgfusepath{clip}%
\pgfsetbuttcap%
\pgfsetroundjoin%
\pgfsetlinewidth{0.501875pt}%
\definecolor{currentstroke}{rgb}{0.395174,0.797475,0.367757}%
\pgfsetstrokecolor{currentstroke}%
\pgfsetdash{}{0pt}%
\pgfpathmoveto{\pgfqpoint{4.523266in}{3.931048in}}%
\pgfpathlineto{\pgfqpoint{4.471803in}{3.939084in}}%
\pgfusepath{stroke}%
\end{pgfscope}%
\begin{pgfscope}%
\pgfpathrectangle{\pgfqpoint{3.352233in}{3.252941in}}{\pgfqpoint{2.407767in}{1.544118in}}%
\pgfusepath{clip}%
\pgfsetbuttcap%
\pgfsetroundjoin%
\pgfsetlinewidth{0.501875pt}%
\definecolor{currentstroke}{rgb}{0.477504,0.821444,0.318195}%
\pgfsetstrokecolor{currentstroke}%
\pgfsetdash{}{0pt}%
\pgfpathmoveto{\pgfqpoint{4.471803in}{3.939084in}}%
\pgfpathlineto{\pgfqpoint{4.420410in}{3.947297in}}%
\pgfusepath{stroke}%
\end{pgfscope}%
\begin{pgfscope}%
\pgfpathrectangle{\pgfqpoint{3.352233in}{3.252941in}}{\pgfqpoint{2.407767in}{1.544118in}}%
\pgfusepath{clip}%
\pgfsetbuttcap%
\pgfsetroundjoin%
\pgfsetlinewidth{0.501875pt}%
\definecolor{currentstroke}{rgb}{0.720391,0.870350,0.162603}%
\pgfsetstrokecolor{currentstroke}%
\pgfsetdash{}{0pt}%
\pgfpathmoveto{\pgfqpoint{4.420410in}{3.947297in}}%
\pgfpathlineto{\pgfqpoint{4.369056in}{3.955614in}}%
\pgfusepath{stroke}%
\end{pgfscope}%
\begin{pgfscope}%
\pgfpathrectangle{\pgfqpoint{3.352233in}{3.252941in}}{\pgfqpoint{2.407767in}{1.544118in}}%
\pgfusepath{clip}%
\pgfsetbuttcap%
\pgfsetroundjoin%
\pgfsetlinewidth{0.501875pt}%
\definecolor{currentstroke}{rgb}{0.699415,0.867117,0.175971}%
\pgfsetstrokecolor{currentstroke}%
\pgfsetdash{}{0pt}%
\pgfpathmoveto{\pgfqpoint{4.369056in}{3.955614in}}%
\pgfpathlineto{\pgfqpoint{4.317784in}{3.964140in}}%
\pgfusepath{stroke}%
\end{pgfscope}%
\begin{pgfscope}%
\pgfpathrectangle{\pgfqpoint{3.352233in}{3.252941in}}{\pgfqpoint{2.407767in}{1.544118in}}%
\pgfusepath{clip}%
\pgfsetbuttcap%
\pgfsetroundjoin%
\pgfsetlinewidth{0.501875pt}%
\definecolor{currentstroke}{rgb}{0.276022,0.044167,0.370164}%
\pgfsetstrokecolor{currentstroke}%
\pgfsetdash{}{0pt}%
\pgfpathmoveto{\pgfqpoint{5.206279in}{3.920762in}}%
\pgfpathlineto{\pgfqpoint{5.153305in}{3.920995in}}%
\pgfusepath{stroke}%
\end{pgfscope}%
\begin{pgfscope}%
\pgfpathrectangle{\pgfqpoint{3.352233in}{3.252941in}}{\pgfqpoint{2.407767in}{1.544118in}}%
\pgfusepath{clip}%
\pgfsetbuttcap%
\pgfsetroundjoin%
\pgfsetlinewidth{0.501875pt}%
\definecolor{currentstroke}{rgb}{0.281924,0.089666,0.412415}%
\pgfsetstrokecolor{currentstroke}%
\pgfsetdash{}{0pt}%
\pgfpathmoveto{\pgfqpoint{5.153305in}{3.920995in}}%
\pgfpathlineto{\pgfqpoint{5.100332in}{3.921305in}}%
\pgfusepath{stroke}%
\end{pgfscope}%
\begin{pgfscope}%
\pgfpathrectangle{\pgfqpoint{3.352233in}{3.252941in}}{\pgfqpoint{2.407767in}{1.544118in}}%
\pgfusepath{clip}%
\pgfsetbuttcap%
\pgfsetroundjoin%
\pgfsetlinewidth{0.501875pt}%
\definecolor{currentstroke}{rgb}{0.280868,0.160771,0.472899}%
\pgfsetstrokecolor{currentstroke}%
\pgfsetdash{}{0pt}%
\pgfpathmoveto{\pgfqpoint{5.100332in}{3.921305in}}%
\pgfpathlineto{\pgfqpoint{5.047364in}{3.921876in}}%
\pgfusepath{stroke}%
\end{pgfscope}%
\begin{pgfscope}%
\pgfpathrectangle{\pgfqpoint{3.352233in}{3.252941in}}{\pgfqpoint{2.407767in}{1.544118in}}%
\pgfusepath{clip}%
\pgfsetbuttcap%
\pgfsetroundjoin%
\pgfsetlinewidth{0.501875pt}%
\definecolor{currentstroke}{rgb}{0.273006,0.204520,0.501721}%
\pgfsetstrokecolor{currentstroke}%
\pgfsetdash{}{0pt}%
\pgfpathmoveto{\pgfqpoint{5.047364in}{3.921876in}}%
\pgfpathlineto{\pgfqpoint{4.994406in}{3.922738in}}%
\pgfusepath{stroke}%
\end{pgfscope}%
\begin{pgfscope}%
\pgfpathrectangle{\pgfqpoint{3.352233in}{3.252941in}}{\pgfqpoint{2.407767in}{1.544118in}}%
\pgfusepath{clip}%
\pgfsetbuttcap%
\pgfsetroundjoin%
\pgfsetlinewidth{0.501875pt}%
\definecolor{currentstroke}{rgb}{0.250425,0.274290,0.533103}%
\pgfsetstrokecolor{currentstroke}%
\pgfsetdash{}{0pt}%
\pgfpathmoveto{\pgfqpoint{4.994406in}{3.922738in}}%
\pgfpathlineto{\pgfqpoint{4.941470in}{3.924018in}}%
\pgfusepath{stroke}%
\end{pgfscope}%
\begin{pgfscope}%
\pgfpathrectangle{\pgfqpoint{3.352233in}{3.252941in}}{\pgfqpoint{2.407767in}{1.544118in}}%
\pgfusepath{clip}%
\pgfsetbuttcap%
\pgfsetroundjoin%
\pgfsetlinewidth{0.501875pt}%
\definecolor{currentstroke}{rgb}{0.214298,0.355619,0.551184}%
\pgfsetstrokecolor{currentstroke}%
\pgfsetdash{}{0pt}%
\pgfpathmoveto{\pgfqpoint{4.941470in}{3.924018in}}%
\pgfpathlineto{\pgfqpoint{4.888578in}{3.925905in}}%
\pgfusepath{stroke}%
\end{pgfscope}%
\begin{pgfscope}%
\pgfpathrectangle{\pgfqpoint{3.352233in}{3.252941in}}{\pgfqpoint{2.407767in}{1.544118in}}%
\pgfusepath{clip}%
\pgfsetbuttcap%
\pgfsetroundjoin%
\pgfsetlinewidth{0.501875pt}%
\definecolor{currentstroke}{rgb}{0.168126,0.459988,0.558082}%
\pgfsetstrokecolor{currentstroke}%
\pgfsetdash{}{0pt}%
\pgfpathmoveto{\pgfqpoint{4.888578in}{3.925905in}}%
\pgfpathlineto{\pgfqpoint{4.835742in}{3.928347in}}%
\pgfusepath{stroke}%
\end{pgfscope}%
\begin{pgfscope}%
\pgfpathrectangle{\pgfqpoint{3.352233in}{3.252941in}}{\pgfqpoint{2.407767in}{1.544118in}}%
\pgfusepath{clip}%
\pgfsetbuttcap%
\pgfsetroundjoin%
\pgfsetlinewidth{0.501875pt}%
\definecolor{currentstroke}{rgb}{0.139147,0.533812,0.555298}%
\pgfsetstrokecolor{currentstroke}%
\pgfsetdash{}{0pt}%
\pgfpathmoveto{\pgfqpoint{4.835742in}{3.928347in}}%
\pgfpathlineto{\pgfqpoint{4.782986in}{3.931415in}}%
\pgfusepath{stroke}%
\end{pgfscope}%
\begin{pgfscope}%
\pgfpathrectangle{\pgfqpoint{3.352233in}{3.252941in}}{\pgfqpoint{2.407767in}{1.544118in}}%
\pgfusepath{clip}%
\pgfsetbuttcap%
\pgfsetroundjoin%
\pgfsetlinewidth{0.501875pt}%
\definecolor{currentstroke}{rgb}{0.276022,0.044167,0.370164}%
\pgfsetstrokecolor{currentstroke}%
\pgfsetdash{}{0pt}%
\pgfpathmoveto{\pgfqpoint{5.206279in}{3.955508in}}%
\pgfpathlineto{\pgfqpoint{5.153304in}{3.955694in}}%
\pgfusepath{stroke}%
\end{pgfscope}%
\begin{pgfscope}%
\pgfpathrectangle{\pgfqpoint{3.352233in}{3.252941in}}{\pgfqpoint{2.407767in}{1.544118in}}%
\pgfusepath{clip}%
\pgfsetbuttcap%
\pgfsetroundjoin%
\pgfsetlinewidth{0.501875pt}%
\definecolor{currentstroke}{rgb}{0.282656,0.100196,0.422160}%
\pgfsetstrokecolor{currentstroke}%
\pgfsetdash{}{0pt}%
\pgfpathmoveto{\pgfqpoint{5.153304in}{3.955694in}}%
\pgfpathlineto{\pgfqpoint{5.100331in}{3.955978in}}%
\pgfusepath{stroke}%
\end{pgfscope}%
\begin{pgfscope}%
\pgfpathrectangle{\pgfqpoint{3.352233in}{3.252941in}}{\pgfqpoint{2.407767in}{1.544118in}}%
\pgfusepath{clip}%
\pgfsetbuttcap%
\pgfsetroundjoin%
\pgfsetlinewidth{0.501875pt}%
\definecolor{currentstroke}{rgb}{0.281887,0.150881,0.465405}%
\pgfsetstrokecolor{currentstroke}%
\pgfsetdash{}{0pt}%
\pgfpathmoveto{\pgfqpoint{5.100331in}{3.955978in}}%
\pgfpathlineto{\pgfqpoint{5.047361in}{3.956492in}}%
\pgfusepath{stroke}%
\end{pgfscope}%
\begin{pgfscope}%
\pgfpathrectangle{\pgfqpoint{3.352233in}{3.252941in}}{\pgfqpoint{2.407767in}{1.544118in}}%
\pgfusepath{clip}%
\pgfsetbuttcap%
\pgfsetroundjoin%
\pgfsetlinewidth{0.501875pt}%
\definecolor{currentstroke}{rgb}{0.263663,0.237631,0.518762}%
\pgfsetstrokecolor{currentstroke}%
\pgfsetdash{}{0pt}%
\pgfpathmoveto{\pgfqpoint{5.047361in}{3.956492in}}%
\pgfpathlineto{\pgfqpoint{4.994393in}{3.957067in}}%
\pgfusepath{stroke}%
\end{pgfscope}%
\begin{pgfscope}%
\pgfpathrectangle{\pgfqpoint{3.352233in}{3.252941in}}{\pgfqpoint{2.407767in}{1.544118in}}%
\pgfusepath{clip}%
\pgfsetbuttcap%
\pgfsetroundjoin%
\pgfsetlinewidth{0.501875pt}%
\definecolor{currentstroke}{rgb}{0.233603,0.313828,0.543914}%
\pgfsetstrokecolor{currentstroke}%
\pgfsetdash{}{0pt}%
\pgfpathmoveto{\pgfqpoint{4.994393in}{3.957067in}}%
\pgfpathlineto{\pgfqpoint{4.941431in}{3.957828in}}%
\pgfusepath{stroke}%
\end{pgfscope}%
\begin{pgfscope}%
\pgfpathrectangle{\pgfqpoint{3.352233in}{3.252941in}}{\pgfqpoint{2.407767in}{1.544118in}}%
\pgfusepath{clip}%
\pgfsetbuttcap%
\pgfsetroundjoin%
\pgfsetlinewidth{0.501875pt}%
\definecolor{currentstroke}{rgb}{0.195860,0.395433,0.555276}%
\pgfsetstrokecolor{currentstroke}%
\pgfsetdash{}{0pt}%
\pgfpathmoveto{\pgfqpoint{4.941431in}{3.957828in}}%
\pgfpathlineto{\pgfqpoint{4.888488in}{3.958992in}}%
\pgfusepath{stroke}%
\end{pgfscope}%
\begin{pgfscope}%
\pgfpathrectangle{\pgfqpoint{3.352233in}{3.252941in}}{\pgfqpoint{2.407767in}{1.544118in}}%
\pgfusepath{clip}%
\pgfsetbuttcap%
\pgfsetroundjoin%
\pgfsetlinewidth{0.501875pt}%
\definecolor{currentstroke}{rgb}{0.146180,0.515413,0.556823}%
\pgfsetstrokecolor{currentstroke}%
\pgfsetdash{}{0pt}%
\pgfpathmoveto{\pgfqpoint{4.888488in}{3.958992in}}%
\pgfpathlineto{\pgfqpoint{4.835570in}{3.960557in}}%
\pgfusepath{stroke}%
\end{pgfscope}%
\begin{pgfscope}%
\pgfpathrectangle{\pgfqpoint{3.352233in}{3.252941in}}{\pgfqpoint{2.407767in}{1.544118in}}%
\pgfusepath{clip}%
\pgfsetbuttcap%
\pgfsetroundjoin%
\pgfsetlinewidth{0.501875pt}%
\definecolor{currentstroke}{rgb}{0.119699,0.618490,0.536347}%
\pgfsetstrokecolor{currentstroke}%
\pgfsetdash{}{0pt}%
\pgfpathmoveto{\pgfqpoint{4.835570in}{3.960557in}}%
\pgfpathlineto{\pgfqpoint{4.782687in}{3.962540in}}%
\pgfusepath{stroke}%
\end{pgfscope}%
\begin{pgfscope}%
\pgfpathrectangle{\pgfqpoint{3.352233in}{3.252941in}}{\pgfqpoint{2.407767in}{1.544118in}}%
\pgfusepath{clip}%
\pgfsetbuttcap%
\pgfsetroundjoin%
\pgfsetlinewidth{0.501875pt}%
\definecolor{currentstroke}{rgb}{0.128087,0.647749,0.523491}%
\pgfsetstrokecolor{currentstroke}%
\pgfsetdash{}{0pt}%
\pgfpathmoveto{\pgfqpoint{4.782687in}{3.962540in}}%
\pgfpathlineto{\pgfqpoint{4.729853in}{3.965005in}}%
\pgfusepath{stroke}%
\end{pgfscope}%
\begin{pgfscope}%
\pgfpathrectangle{\pgfqpoint{3.352233in}{3.252941in}}{\pgfqpoint{2.407767in}{1.544118in}}%
\pgfusepath{clip}%
\pgfsetbuttcap%
\pgfsetroundjoin%
\pgfsetlinewidth{0.501875pt}%
\definecolor{currentstroke}{rgb}{0.153894,0.680203,0.504172}%
\pgfsetstrokecolor{currentstroke}%
\pgfsetdash{}{0pt}%
\pgfpathmoveto{\pgfqpoint{4.729853in}{3.965005in}}%
\pgfpathlineto{\pgfqpoint{4.677064in}{3.967841in}}%
\pgfusepath{stroke}%
\end{pgfscope}%
\begin{pgfscope}%
\pgfpathrectangle{\pgfqpoint{3.352233in}{3.252941in}}{\pgfqpoint{2.407767in}{1.544118in}}%
\pgfusepath{clip}%
\pgfsetbuttcap%
\pgfsetroundjoin%
\pgfsetlinewidth{0.501875pt}%
\definecolor{currentstroke}{rgb}{0.335885,0.777018,0.402049}%
\pgfsetstrokecolor{currentstroke}%
\pgfsetdash{}{0pt}%
\pgfpathmoveto{\pgfqpoint{4.677064in}{3.967841in}}%
\pgfpathlineto{\pgfqpoint{4.624327in}{3.971038in}}%
\pgfusepath{stroke}%
\end{pgfscope}%
\begin{pgfscope}%
\pgfpathrectangle{\pgfqpoint{3.352233in}{3.252941in}}{\pgfqpoint{2.407767in}{1.544118in}}%
\pgfusepath{clip}%
\pgfsetbuttcap%
\pgfsetroundjoin%
\pgfsetlinewidth{0.501875pt}%
\definecolor{currentstroke}{rgb}{0.458674,0.816363,0.329727}%
\pgfsetstrokecolor{currentstroke}%
\pgfsetdash{}{0pt}%
\pgfpathmoveto{\pgfqpoint{4.624327in}{3.971038in}}%
\pgfpathlineto{\pgfqpoint{4.571661in}{3.974671in}}%
\pgfusepath{stroke}%
\end{pgfscope}%
\begin{pgfscope}%
\pgfpathrectangle{\pgfqpoint{3.352233in}{3.252941in}}{\pgfqpoint{2.407767in}{1.544118in}}%
\pgfusepath{clip}%
\pgfsetbuttcap%
\pgfsetroundjoin%
\pgfsetlinewidth{0.501875pt}%
\definecolor{currentstroke}{rgb}{0.565498,0.842430,0.262877}%
\pgfsetstrokecolor{currentstroke}%
\pgfsetdash{}{0pt}%
\pgfpathmoveto{\pgfqpoint{4.571661in}{3.974671in}}%
\pgfpathlineto{\pgfqpoint{4.519054in}{3.978652in}}%
\pgfusepath{stroke}%
\end{pgfscope}%
\begin{pgfscope}%
\pgfpathrectangle{\pgfqpoint{3.352233in}{3.252941in}}{\pgfqpoint{2.407767in}{1.544118in}}%
\pgfusepath{clip}%
\pgfsetbuttcap%
\pgfsetroundjoin%
\pgfsetlinewidth{0.501875pt}%
\definecolor{currentstroke}{rgb}{0.606045,0.850733,0.236712}%
\pgfsetstrokecolor{currentstroke}%
\pgfsetdash{}{0pt}%
\pgfpathmoveto{\pgfqpoint{4.519054in}{3.978652in}}%
\pgfpathlineto{\pgfqpoint{4.466483in}{3.982814in}}%
\pgfusepath{stroke}%
\end{pgfscope}%
\begin{pgfscope}%
\pgfpathrectangle{\pgfqpoint{3.352233in}{3.252941in}}{\pgfqpoint{2.407767in}{1.544118in}}%
\pgfusepath{clip}%
\pgfsetbuttcap%
\pgfsetroundjoin%
\pgfsetlinewidth{0.501875pt}%
\definecolor{currentstroke}{rgb}{0.866013,0.889868,0.095953}%
\pgfsetstrokecolor{currentstroke}%
\pgfsetdash{}{0pt}%
\pgfpathmoveto{\pgfqpoint{4.466483in}{3.982814in}}%
\pgfpathlineto{\pgfqpoint{4.413932in}{3.987039in}}%
\pgfusepath{stroke}%
\end{pgfscope}%
\begin{pgfscope}%
\pgfpathrectangle{\pgfqpoint{3.352233in}{3.252941in}}{\pgfqpoint{2.407767in}{1.544118in}}%
\pgfusepath{clip}%
\pgfsetbuttcap%
\pgfsetroundjoin%
\pgfsetlinewidth{0.501875pt}%
\definecolor{currentstroke}{rgb}{0.916242,0.896091,0.100717}%
\pgfsetstrokecolor{currentstroke}%
\pgfsetdash{}{0pt}%
\pgfpathmoveto{\pgfqpoint{4.413932in}{3.987039in}}%
\pgfpathlineto{\pgfqpoint{4.361392in}{3.991313in}}%
\pgfusepath{stroke}%
\end{pgfscope}%
\begin{pgfscope}%
\pgfpathrectangle{\pgfqpoint{3.352233in}{3.252941in}}{\pgfqpoint{2.407767in}{1.544118in}}%
\pgfusepath{clip}%
\pgfsetbuttcap%
\pgfsetroundjoin%
\pgfsetlinewidth{0.501875pt}%
\definecolor{currentstroke}{rgb}{0.896320,0.893616,0.096335}%
\pgfsetstrokecolor{currentstroke}%
\pgfsetdash{}{0pt}%
\pgfpathmoveto{\pgfqpoint{4.361392in}{3.991313in}}%
\pgfpathlineto{\pgfqpoint{4.308842in}{3.995562in}}%
\pgfusepath{stroke}%
\end{pgfscope}%
\begin{pgfscope}%
\pgfpathrectangle{\pgfqpoint{3.352233in}{3.252941in}}{\pgfqpoint{2.407767in}{1.544118in}}%
\pgfusepath{clip}%
\pgfsetbuttcap%
\pgfsetroundjoin%
\pgfsetlinewidth{0.501875pt}%
\definecolor{currentstroke}{rgb}{0.276022,0.044167,0.370164}%
\pgfsetstrokecolor{currentstroke}%
\pgfsetdash{}{0pt}%
\pgfpathmoveto{\pgfqpoint{5.206279in}{3.990254in}}%
\pgfpathlineto{\pgfqpoint{5.153303in}{3.990256in}}%
\pgfusepath{stroke}%
\end{pgfscope}%
\begin{pgfscope}%
\pgfpathrectangle{\pgfqpoint{3.352233in}{3.252941in}}{\pgfqpoint{2.407767in}{1.544118in}}%
\pgfusepath{clip}%
\pgfsetbuttcap%
\pgfsetroundjoin%
\pgfsetlinewidth{0.501875pt}%
\definecolor{currentstroke}{rgb}{0.282910,0.105393,0.426902}%
\pgfsetstrokecolor{currentstroke}%
\pgfsetdash{}{0pt}%
\pgfpathmoveto{\pgfqpoint{5.153303in}{3.990256in}}%
\pgfpathlineto{\pgfqpoint{5.100327in}{3.990343in}}%
\pgfusepath{stroke}%
\end{pgfscope}%
\begin{pgfscope}%
\pgfpathrectangle{\pgfqpoint{3.352233in}{3.252941in}}{\pgfqpoint{2.407767in}{1.544118in}}%
\pgfusepath{clip}%
\pgfsetbuttcap%
\pgfsetroundjoin%
\pgfsetlinewidth{0.501875pt}%
\definecolor{currentstroke}{rgb}{0.280255,0.165693,0.476498}%
\pgfsetstrokecolor{currentstroke}%
\pgfsetdash{}{0pt}%
\pgfpathmoveto{\pgfqpoint{5.100327in}{3.990343in}}%
\pgfpathlineto{\pgfqpoint{5.047352in}{3.990526in}}%
\pgfusepath{stroke}%
\end{pgfscope}%
\begin{pgfscope}%
\pgfpathrectangle{\pgfqpoint{3.352233in}{3.252941in}}{\pgfqpoint{2.407767in}{1.544118in}}%
\pgfusepath{clip}%
\pgfsetbuttcap%
\pgfsetroundjoin%
\pgfsetlinewidth{0.501875pt}%
\definecolor{currentstroke}{rgb}{0.262138,0.242286,0.520837}%
\pgfsetstrokecolor{currentstroke}%
\pgfsetdash{}{0pt}%
\pgfpathmoveto{\pgfqpoint{5.047352in}{3.990526in}}%
\pgfpathlineto{\pgfqpoint{4.994378in}{3.990845in}}%
\pgfusepath{stroke}%
\end{pgfscope}%
\begin{pgfscope}%
\pgfpathrectangle{\pgfqpoint{3.352233in}{3.252941in}}{\pgfqpoint{2.407767in}{1.544118in}}%
\pgfusepath{clip}%
\pgfsetbuttcap%
\pgfsetroundjoin%
\pgfsetlinewidth{0.501875pt}%
\definecolor{currentstroke}{rgb}{0.220057,0.343307,0.549413}%
\pgfsetstrokecolor{currentstroke}%
\pgfsetdash{}{0pt}%
\pgfpathmoveto{\pgfqpoint{4.994378in}{3.990845in}}%
\pgfpathlineto{\pgfqpoint{4.941407in}{3.991316in}}%
\pgfusepath{stroke}%
\end{pgfscope}%
\begin{pgfscope}%
\pgfpathrectangle{\pgfqpoint{3.352233in}{3.252941in}}{\pgfqpoint{2.407767in}{1.544118in}}%
\pgfusepath{clip}%
\pgfsetbuttcap%
\pgfsetroundjoin%
\pgfsetlinewidth{0.501875pt}%
\definecolor{currentstroke}{rgb}{0.169646,0.456262,0.558030}%
\pgfsetstrokecolor{currentstroke}%
\pgfsetdash{}{0pt}%
\pgfpathmoveto{\pgfqpoint{4.941407in}{3.991316in}}%
\pgfpathlineto{\pgfqpoint{4.888440in}{3.991927in}}%
\pgfusepath{stroke}%
\end{pgfscope}%
\begin{pgfscope}%
\pgfpathrectangle{\pgfqpoint{3.352233in}{3.252941in}}{\pgfqpoint{2.407767in}{1.544118in}}%
\pgfusepath{clip}%
\pgfsetbuttcap%
\pgfsetroundjoin%
\pgfsetlinewidth{0.501875pt}%
\definecolor{currentstroke}{rgb}{0.127568,0.566949,0.550556}%
\pgfsetstrokecolor{currentstroke}%
\pgfsetdash{}{0pt}%
\pgfpathmoveto{\pgfqpoint{4.888440in}{3.991927in}}%
\pgfpathlineto{\pgfqpoint{4.835480in}{3.992720in}}%
\pgfusepath{stroke}%
\end{pgfscope}%
\begin{pgfscope}%
\pgfpathrectangle{\pgfqpoint{3.352233in}{3.252941in}}{\pgfqpoint{2.407767in}{1.544118in}}%
\pgfusepath{clip}%
\pgfsetbuttcap%
\pgfsetroundjoin%
\pgfsetlinewidth{0.501875pt}%
\definecolor{currentstroke}{rgb}{0.124780,0.640461,0.527068}%
\pgfsetstrokecolor{currentstroke}%
\pgfsetdash{}{0pt}%
\pgfpathmoveto{\pgfqpoint{4.835480in}{3.992720in}}%
\pgfpathlineto{\pgfqpoint{4.782527in}{3.993705in}}%
\pgfusepath{stroke}%
\end{pgfscope}%
\begin{pgfscope}%
\pgfpathrectangle{\pgfqpoint{3.352233in}{3.252941in}}{\pgfqpoint{2.407767in}{1.544118in}}%
\pgfusepath{clip}%
\pgfsetbuttcap%
\pgfsetroundjoin%
\pgfsetlinewidth{0.501875pt}%
\definecolor{currentstroke}{rgb}{0.220124,0.725509,0.466226}%
\pgfsetstrokecolor{currentstroke}%
\pgfsetdash{}{0pt}%
\pgfpathmoveto{\pgfqpoint{4.782527in}{3.993705in}}%
\pgfpathlineto{\pgfqpoint{4.729584in}{3.994899in}}%
\pgfusepath{stroke}%
\end{pgfscope}%
\begin{pgfscope}%
\pgfpathrectangle{\pgfqpoint{3.352233in}{3.252941in}}{\pgfqpoint{2.407767in}{1.544118in}}%
\pgfusepath{clip}%
\pgfsetbuttcap%
\pgfsetroundjoin%
\pgfsetlinewidth{0.501875pt}%
\definecolor{currentstroke}{rgb}{0.281477,0.755203,0.432552}%
\pgfsetstrokecolor{currentstroke}%
\pgfsetdash{}{0pt}%
\pgfpathmoveto{\pgfqpoint{4.729584in}{3.994899in}}%
\pgfpathlineto{\pgfqpoint{4.676656in}{3.996325in}}%
\pgfusepath{stroke}%
\end{pgfscope}%
\begin{pgfscope}%
\pgfpathrectangle{\pgfqpoint{3.352233in}{3.252941in}}{\pgfqpoint{2.407767in}{1.544118in}}%
\pgfusepath{clip}%
\pgfsetbuttcap%
\pgfsetroundjoin%
\pgfsetlinewidth{0.501875pt}%
\definecolor{currentstroke}{rgb}{0.606045,0.850733,0.236712}%
\pgfsetstrokecolor{currentstroke}%
\pgfsetdash{}{0pt}%
\pgfpathmoveto{\pgfqpoint{4.676656in}{3.996325in}}%
\pgfpathlineto{\pgfqpoint{4.623744in}{3.997949in}}%
\pgfusepath{stroke}%
\end{pgfscope}%
\begin{pgfscope}%
\pgfpathrectangle{\pgfqpoint{3.352233in}{3.252941in}}{\pgfqpoint{2.407767in}{1.544118in}}%
\pgfusepath{clip}%
\pgfsetbuttcap%
\pgfsetroundjoin%
\pgfsetlinewidth{0.501875pt}%
\definecolor{currentstroke}{rgb}{0.276022,0.044167,0.370164}%
\pgfsetstrokecolor{currentstroke}%
\pgfsetdash{}{0pt}%
\pgfpathmoveto{\pgfqpoint{5.206279in}{4.025000in}}%
\pgfpathlineto{\pgfqpoint{5.153303in}{4.025045in}}%
\pgfusepath{stroke}%
\end{pgfscope}%
\begin{pgfscope}%
\pgfpathrectangle{\pgfqpoint{3.352233in}{3.252941in}}{\pgfqpoint{2.407767in}{1.544118in}}%
\pgfusepath{clip}%
\pgfsetbuttcap%
\pgfsetroundjoin%
\pgfsetlinewidth{0.501875pt}%
\definecolor{currentstroke}{rgb}{0.283091,0.110553,0.431554}%
\pgfsetstrokecolor{currentstroke}%
\pgfsetdash{}{0pt}%
\pgfpathmoveto{\pgfqpoint{5.153303in}{4.025045in}}%
\pgfpathlineto{\pgfqpoint{5.100328in}{4.025107in}}%
\pgfusepath{stroke}%
\end{pgfscope}%
\begin{pgfscope}%
\pgfpathrectangle{\pgfqpoint{3.352233in}{3.252941in}}{\pgfqpoint{2.407767in}{1.544118in}}%
\pgfusepath{clip}%
\pgfsetbuttcap%
\pgfsetroundjoin%
\pgfsetlinewidth{0.501875pt}%
\definecolor{currentstroke}{rgb}{0.280255,0.165693,0.476498}%
\pgfsetstrokecolor{currentstroke}%
\pgfsetdash{}{0pt}%
\pgfpathmoveto{\pgfqpoint{5.100328in}{4.025107in}}%
\pgfpathlineto{\pgfqpoint{5.047352in}{4.025264in}}%
\pgfusepath{stroke}%
\end{pgfscope}%
\begin{pgfscope}%
\pgfpathrectangle{\pgfqpoint{3.352233in}{3.252941in}}{\pgfqpoint{2.407767in}{1.544118in}}%
\pgfusepath{clip}%
\pgfsetbuttcap%
\pgfsetroundjoin%
\pgfsetlinewidth{0.501875pt}%
\definecolor{currentstroke}{rgb}{0.260571,0.246922,0.522828}%
\pgfsetstrokecolor{currentstroke}%
\pgfsetdash{}{0pt}%
\pgfpathmoveto{\pgfqpoint{5.047352in}{4.025264in}}%
\pgfpathlineto{\pgfqpoint{4.994377in}{4.025308in}}%
\pgfusepath{stroke}%
\end{pgfscope}%
\begin{pgfscope}%
\pgfpathrectangle{\pgfqpoint{3.352233in}{3.252941in}}{\pgfqpoint{2.407767in}{1.544118in}}%
\pgfusepath{clip}%
\pgfsetbuttcap%
\pgfsetroundjoin%
\pgfsetlinewidth{0.501875pt}%
\definecolor{currentstroke}{rgb}{0.223925,0.334994,0.548053}%
\pgfsetstrokecolor{currentstroke}%
\pgfsetdash{}{0pt}%
\pgfpathmoveto{\pgfqpoint{4.994377in}{4.025308in}}%
\pgfpathlineto{\pgfqpoint{4.941400in}{4.025311in}}%
\pgfusepath{stroke}%
\end{pgfscope}%
\begin{pgfscope}%
\pgfpathrectangle{\pgfqpoint{3.352233in}{3.252941in}}{\pgfqpoint{2.407767in}{1.544118in}}%
\pgfusepath{clip}%
\pgfsetbuttcap%
\pgfsetroundjoin%
\pgfsetlinewidth{0.501875pt}%
\definecolor{currentstroke}{rgb}{0.160665,0.478540,0.558115}%
\pgfsetstrokecolor{currentstroke}%
\pgfsetdash{}{0pt}%
\pgfpathmoveto{\pgfqpoint{4.941400in}{4.025311in}}%
\pgfpathlineto{\pgfqpoint{4.888424in}{4.025346in}}%
\pgfusepath{stroke}%
\end{pgfscope}%
\begin{pgfscope}%
\pgfpathrectangle{\pgfqpoint{3.352233in}{3.252941in}}{\pgfqpoint{2.407767in}{1.544118in}}%
\pgfusepath{clip}%
\pgfsetbuttcap%
\pgfsetroundjoin%
\pgfsetlinewidth{0.501875pt}%
\definecolor{currentstroke}{rgb}{0.124395,0.578002,0.548287}%
\pgfsetstrokecolor{currentstroke}%
\pgfsetdash{}{0pt}%
\pgfpathmoveto{\pgfqpoint{4.888424in}{4.025346in}}%
\pgfpathlineto{\pgfqpoint{4.835448in}{4.025389in}}%
\pgfusepath{stroke}%
\end{pgfscope}%
\begin{pgfscope}%
\pgfpathrectangle{\pgfqpoint{3.352233in}{3.252941in}}{\pgfqpoint{2.407767in}{1.544118in}}%
\pgfusepath{clip}%
\pgfsetbuttcap%
\pgfsetroundjoin%
\pgfsetlinewidth{0.501875pt}%
\definecolor{currentstroke}{rgb}{0.132268,0.655014,0.519661}%
\pgfsetstrokecolor{currentstroke}%
\pgfsetdash{}{0pt}%
\pgfpathmoveto{\pgfqpoint{4.835448in}{4.025389in}}%
\pgfpathlineto{\pgfqpoint{4.782472in}{4.025414in}}%
\pgfusepath{stroke}%
\end{pgfscope}%
\begin{pgfscope}%
\pgfpathrectangle{\pgfqpoint{3.352233in}{3.252941in}}{\pgfqpoint{2.407767in}{1.544118in}}%
\pgfusepath{clip}%
\pgfsetbuttcap%
\pgfsetroundjoin%
\pgfsetlinewidth{0.501875pt}%
\definecolor{currentstroke}{rgb}{0.232815,0.732247,0.459277}%
\pgfsetstrokecolor{currentstroke}%
\pgfsetdash{}{0pt}%
\pgfpathmoveto{\pgfqpoint{4.782472in}{4.025414in}}%
\pgfpathlineto{\pgfqpoint{4.729496in}{4.025460in}}%
\pgfusepath{stroke}%
\end{pgfscope}%
\begin{pgfscope}%
\pgfpathrectangle{\pgfqpoint{3.352233in}{3.252941in}}{\pgfqpoint{2.407767in}{1.544118in}}%
\pgfusepath{clip}%
\pgfsetbuttcap%
\pgfsetroundjoin%
\pgfsetlinewidth{0.501875pt}%
\definecolor{currentstroke}{rgb}{0.360741,0.785964,0.387814}%
\pgfsetstrokecolor{currentstroke}%
\pgfsetdash{}{0pt}%
\pgfpathmoveto{\pgfqpoint{4.729496in}{4.025460in}}%
\pgfpathlineto{\pgfqpoint{4.676521in}{4.025502in}}%
\pgfusepath{stroke}%
\end{pgfscope}%
\begin{pgfscope}%
\pgfpathrectangle{\pgfqpoint{3.352233in}{3.252941in}}{\pgfqpoint{2.407767in}{1.544118in}}%
\pgfusepath{clip}%
\pgfsetbuttcap%
\pgfsetroundjoin%
\pgfsetlinewidth{0.501875pt}%
\definecolor{currentstroke}{rgb}{0.595839,0.848717,0.243329}%
\pgfsetstrokecolor{currentstroke}%
\pgfsetdash{}{0pt}%
\pgfpathmoveto{\pgfqpoint{4.676521in}{4.025502in}}%
\pgfpathlineto{\pgfqpoint{4.623545in}{4.025526in}}%
\pgfusepath{stroke}%
\end{pgfscope}%
\begin{pgfscope}%
\pgfpathrectangle{\pgfqpoint{3.352233in}{3.252941in}}{\pgfqpoint{2.407767in}{1.544118in}}%
\pgfusepath{clip}%
\pgfsetbuttcap%
\pgfsetroundjoin%
\pgfsetlinewidth{0.501875pt}%
\definecolor{currentstroke}{rgb}{0.688944,0.865448,0.182725}%
\pgfsetstrokecolor{currentstroke}%
\pgfsetdash{}{0pt}%
\pgfpathmoveto{\pgfqpoint{4.623545in}{4.025526in}}%
\pgfpathlineto{\pgfqpoint{4.570569in}{4.025579in}}%
\pgfusepath{stroke}%
\end{pgfscope}%
\begin{pgfscope}%
\pgfpathrectangle{\pgfqpoint{3.352233in}{3.252941in}}{\pgfqpoint{2.407767in}{1.544118in}}%
\pgfusepath{clip}%
\pgfsetbuttcap%
\pgfsetroundjoin%
\pgfsetlinewidth{0.501875pt}%
\definecolor{currentstroke}{rgb}{0.855810,0.888601,0.097452}%
\pgfsetstrokecolor{currentstroke}%
\pgfsetdash{}{0pt}%
\pgfpathmoveto{\pgfqpoint{4.570569in}{4.025579in}}%
\pgfpathlineto{\pgfqpoint{4.517595in}{4.025664in}}%
\pgfusepath{stroke}%
\end{pgfscope}%
\begin{pgfscope}%
\pgfpathrectangle{\pgfqpoint{3.352233in}{3.252941in}}{\pgfqpoint{2.407767in}{1.544118in}}%
\pgfusepath{clip}%
\pgfsetbuttcap%
\pgfsetroundjoin%
\pgfsetlinewidth{0.501875pt}%
\definecolor{currentstroke}{rgb}{0.876168,0.891125,0.095250}%
\pgfsetstrokecolor{currentstroke}%
\pgfsetdash{}{0pt}%
\pgfpathmoveto{\pgfqpoint{4.517595in}{4.025664in}}%
\pgfpathlineto{\pgfqpoint{4.464621in}{4.025820in}}%
\pgfusepath{stroke}%
\end{pgfscope}%
\begin{pgfscope}%
\pgfpathrectangle{\pgfqpoint{3.352233in}{3.252941in}}{\pgfqpoint{2.407767in}{1.544118in}}%
\pgfusepath{clip}%
\pgfsetbuttcap%
\pgfsetroundjoin%
\pgfsetlinewidth{0.501875pt}%
\definecolor{currentstroke}{rgb}{0.974417,0.903590,0.130215}%
\pgfsetstrokecolor{currentstroke}%
\pgfsetdash{}{0pt}%
\pgfpathmoveto{\pgfqpoint{4.464621in}{4.025820in}}%
\pgfpathlineto{\pgfqpoint{4.411649in}{4.026041in}}%
\pgfusepath{stroke}%
\end{pgfscope}%
\begin{pgfscope}%
\pgfpathrectangle{\pgfqpoint{3.352233in}{3.252941in}}{\pgfqpoint{2.407767in}{1.544118in}}%
\pgfusepath{clip}%
\pgfsetbuttcap%
\pgfsetroundjoin%
\pgfsetlinewidth{0.501875pt}%
\definecolor{currentstroke}{rgb}{0.916242,0.896091,0.100717}%
\pgfsetstrokecolor{currentstroke}%
\pgfsetdash{}{0pt}%
\pgfpathmoveto{\pgfqpoint{4.411649in}{4.026041in}}%
\pgfpathlineto{\pgfqpoint{4.358680in}{4.026249in}}%
\pgfusepath{stroke}%
\end{pgfscope}%
\begin{pgfscope}%
\pgfpathrectangle{\pgfqpoint{3.352233in}{3.252941in}}{\pgfqpoint{2.407767in}{1.544118in}}%
\pgfusepath{clip}%
\pgfsetbuttcap%
\pgfsetroundjoin%
\pgfsetlinewidth{0.501875pt}%
\definecolor{currentstroke}{rgb}{0.955300,0.901065,0.118128}%
\pgfsetstrokecolor{currentstroke}%
\pgfsetdash{}{0pt}%
\pgfpathmoveto{\pgfqpoint{4.358680in}{4.026249in}}%
\pgfpathlineto{\pgfqpoint{4.305713in}{4.026336in}}%
\pgfusepath{stroke}%
\end{pgfscope}%
\begin{pgfscope}%
\pgfpathrectangle{\pgfqpoint{3.352233in}{3.252941in}}{\pgfqpoint{2.407767in}{1.544118in}}%
\pgfusepath{clip}%
\pgfsetbuttcap%
\pgfsetroundjoin%
\pgfsetlinewidth{0.501875pt}%
\definecolor{currentstroke}{rgb}{0.276022,0.044167,0.370164}%
\pgfsetstrokecolor{currentstroke}%
\pgfsetdash{}{0pt}%
\pgfpathmoveto{\pgfqpoint{5.206279in}{4.059746in}}%
\pgfpathlineto{\pgfqpoint{5.153303in}{4.059745in}}%
\pgfusepath{stroke}%
\end{pgfscope}%
\begin{pgfscope}%
\pgfpathrectangle{\pgfqpoint{3.352233in}{3.252941in}}{\pgfqpoint{2.407767in}{1.544118in}}%
\pgfusepath{clip}%
\pgfsetbuttcap%
\pgfsetroundjoin%
\pgfsetlinewidth{0.501875pt}%
\definecolor{currentstroke}{rgb}{0.282910,0.105393,0.426902}%
\pgfsetstrokecolor{currentstroke}%
\pgfsetdash{}{0pt}%
\pgfpathmoveto{\pgfqpoint{5.153303in}{4.059745in}}%
\pgfpathlineto{\pgfqpoint{5.100327in}{4.059658in}}%
\pgfusepath{stroke}%
\end{pgfscope}%
\begin{pgfscope}%
\pgfpathrectangle{\pgfqpoint{3.352233in}{3.252941in}}{\pgfqpoint{2.407767in}{1.544118in}}%
\pgfusepath{clip}%
\pgfsetbuttcap%
\pgfsetroundjoin%
\pgfsetlinewidth{0.501875pt}%
\definecolor{currentstroke}{rgb}{0.279574,0.170599,0.479997}%
\pgfsetstrokecolor{currentstroke}%
\pgfsetdash{}{0pt}%
\pgfpathmoveto{\pgfqpoint{5.100327in}{4.059658in}}%
\pgfpathlineto{\pgfqpoint{5.047353in}{4.059473in}}%
\pgfusepath{stroke}%
\end{pgfscope}%
\begin{pgfscope}%
\pgfpathrectangle{\pgfqpoint{3.352233in}{3.252941in}}{\pgfqpoint{2.407767in}{1.544118in}}%
\pgfusepath{clip}%
\pgfsetbuttcap%
\pgfsetroundjoin%
\pgfsetlinewidth{0.501875pt}%
\definecolor{currentstroke}{rgb}{0.265145,0.232956,0.516599}%
\pgfsetstrokecolor{currentstroke}%
\pgfsetdash{}{0pt}%
\pgfpathmoveto{\pgfqpoint{5.047353in}{4.059473in}}%
\pgfpathlineto{\pgfqpoint{4.994380in}{4.059136in}}%
\pgfusepath{stroke}%
\end{pgfscope}%
\begin{pgfscope}%
\pgfpathrectangle{\pgfqpoint{3.352233in}{3.252941in}}{\pgfqpoint{2.407767in}{1.544118in}}%
\pgfusepath{clip}%
\pgfsetbuttcap%
\pgfsetroundjoin%
\pgfsetlinewidth{0.501875pt}%
\definecolor{currentstroke}{rgb}{0.221989,0.339161,0.548752}%
\pgfsetstrokecolor{currentstroke}%
\pgfsetdash{}{0pt}%
\pgfpathmoveto{\pgfqpoint{4.994380in}{4.059136in}}%
\pgfpathlineto{\pgfqpoint{4.941410in}{4.058611in}}%
\pgfusepath{stroke}%
\end{pgfscope}%
\begin{pgfscope}%
\pgfpathrectangle{\pgfqpoint{3.352233in}{3.252941in}}{\pgfqpoint{2.407767in}{1.544118in}}%
\pgfusepath{clip}%
\pgfsetbuttcap%
\pgfsetroundjoin%
\pgfsetlinewidth{0.501875pt}%
\definecolor{currentstroke}{rgb}{0.168126,0.459988,0.558082}%
\pgfsetstrokecolor{currentstroke}%
\pgfsetdash{}{0pt}%
\pgfpathmoveto{\pgfqpoint{4.941410in}{4.058611in}}%
\pgfpathlineto{\pgfqpoint{4.888447in}{4.057869in}}%
\pgfusepath{stroke}%
\end{pgfscope}%
\begin{pgfscope}%
\pgfpathrectangle{\pgfqpoint{3.352233in}{3.252941in}}{\pgfqpoint{2.407767in}{1.544118in}}%
\pgfusepath{clip}%
\pgfsetbuttcap%
\pgfsetroundjoin%
\pgfsetlinewidth{0.501875pt}%
\definecolor{currentstroke}{rgb}{0.126453,0.570633,0.549841}%
\pgfsetstrokecolor{currentstroke}%
\pgfsetdash{}{0pt}%
\pgfpathmoveto{\pgfqpoint{4.888447in}{4.057869in}}%
\pgfpathlineto{\pgfqpoint{4.835491in}{4.056952in}}%
\pgfusepath{stroke}%
\end{pgfscope}%
\begin{pgfscope}%
\pgfpathrectangle{\pgfqpoint{3.352233in}{3.252941in}}{\pgfqpoint{2.407767in}{1.544118in}}%
\pgfusepath{clip}%
\pgfsetbuttcap%
\pgfsetroundjoin%
\pgfsetlinewidth{0.501875pt}%
\definecolor{currentstroke}{rgb}{0.128087,0.647749,0.523491}%
\pgfsetstrokecolor{currentstroke}%
\pgfsetdash{}{0pt}%
\pgfpathmoveto{\pgfqpoint{4.835491in}{4.056952in}}%
\pgfpathlineto{\pgfqpoint{4.782541in}{4.055891in}}%
\pgfusepath{stroke}%
\end{pgfscope}%
\begin{pgfscope}%
\pgfpathrectangle{\pgfqpoint{3.352233in}{3.252941in}}{\pgfqpoint{2.407767in}{1.544118in}}%
\pgfusepath{clip}%
\pgfsetbuttcap%
\pgfsetroundjoin%
\pgfsetlinewidth{0.501875pt}%
\definecolor{currentstroke}{rgb}{0.153894,0.680203,0.504172}%
\pgfsetstrokecolor{currentstroke}%
\pgfsetdash{}{0pt}%
\pgfpathmoveto{\pgfqpoint{4.782541in}{4.055891in}}%
\pgfpathlineto{\pgfqpoint{4.729599in}{4.054690in}}%
\pgfusepath{stroke}%
\end{pgfscope}%
\begin{pgfscope}%
\pgfpathrectangle{\pgfqpoint{3.352233in}{3.252941in}}{\pgfqpoint{2.407767in}{1.544118in}}%
\pgfusepath{clip}%
\pgfsetbuttcap%
\pgfsetroundjoin%
\pgfsetlinewidth{0.501875pt}%
\definecolor{currentstroke}{rgb}{0.304148,0.764704,0.419943}%
\pgfsetstrokecolor{currentstroke}%
\pgfsetdash{}{0pt}%
\pgfpathmoveto{\pgfqpoint{4.729599in}{4.054690in}}%
\pgfpathlineto{\pgfqpoint{4.676669in}{4.053298in}}%
\pgfusepath{stroke}%
\end{pgfscope}%
\begin{pgfscope}%
\pgfpathrectangle{\pgfqpoint{3.352233in}{3.252941in}}{\pgfqpoint{2.407767in}{1.544118in}}%
\pgfusepath{clip}%
\pgfsetbuttcap%
\pgfsetroundjoin%
\pgfsetlinewidth{0.501875pt}%
\definecolor{currentstroke}{rgb}{0.545524,0.838039,0.275626}%
\pgfsetstrokecolor{currentstroke}%
\pgfsetdash{}{0pt}%
\pgfpathmoveto{\pgfqpoint{4.676669in}{4.053298in}}%
\pgfpathlineto{\pgfqpoint{4.623755in}{4.051708in}}%
\pgfusepath{stroke}%
\end{pgfscope}%
\begin{pgfscope}%
\pgfpathrectangle{\pgfqpoint{3.352233in}{3.252941in}}{\pgfqpoint{2.407767in}{1.544118in}}%
\pgfusepath{clip}%
\pgfsetbuttcap%
\pgfsetroundjoin%
\pgfsetlinewidth{0.501875pt}%
\definecolor{currentstroke}{rgb}{0.688944,0.865448,0.182725}%
\pgfsetstrokecolor{currentstroke}%
\pgfsetdash{}{0pt}%
\pgfpathmoveto{\pgfqpoint{4.623755in}{4.051708in}}%
\pgfpathlineto{\pgfqpoint{4.570851in}{4.050035in}}%
\pgfusepath{stroke}%
\end{pgfscope}%
\begin{pgfscope}%
\pgfpathrectangle{\pgfqpoint{3.352233in}{3.252941in}}{\pgfqpoint{2.407767in}{1.544118in}}%
\pgfusepath{clip}%
\pgfsetbuttcap%
\pgfsetroundjoin%
\pgfsetlinewidth{0.501875pt}%
\definecolor{currentstroke}{rgb}{0.741388,0.873449,0.149561}%
\pgfsetstrokecolor{currentstroke}%
\pgfsetdash{}{0pt}%
\pgfpathmoveto{\pgfqpoint{4.570851in}{4.050035in}}%
\pgfpathlineto{\pgfqpoint{4.517955in}{4.048315in}}%
\pgfusepath{stroke}%
\end{pgfscope}%
\begin{pgfscope}%
\pgfpathrectangle{\pgfqpoint{3.352233in}{3.252941in}}{\pgfqpoint{2.407767in}{1.544118in}}%
\pgfusepath{clip}%
\pgfsetbuttcap%
\pgfsetroundjoin%
\pgfsetlinewidth{0.501875pt}%
\definecolor{currentstroke}{rgb}{0.793760,0.880678,0.120005}%
\pgfsetstrokecolor{currentstroke}%
\pgfsetdash{}{0pt}%
\pgfpathmoveto{\pgfqpoint{4.517955in}{4.048315in}}%
\pgfpathlineto{\pgfqpoint{4.465067in}{4.046516in}}%
\pgfusepath{stroke}%
\end{pgfscope}%
\begin{pgfscope}%
\pgfpathrectangle{\pgfqpoint{3.352233in}{3.252941in}}{\pgfqpoint{2.407767in}{1.544118in}}%
\pgfusepath{clip}%
\pgfsetbuttcap%
\pgfsetroundjoin%
\pgfsetlinewidth{0.501875pt}%
\definecolor{currentstroke}{rgb}{0.916242,0.896091,0.100717}%
\pgfsetstrokecolor{currentstroke}%
\pgfsetdash{}{0pt}%
\pgfpathmoveto{\pgfqpoint{4.465067in}{4.046516in}}%
\pgfpathlineto{\pgfqpoint{4.412190in}{4.044611in}}%
\pgfusepath{stroke}%
\end{pgfscope}%
\begin{pgfscope}%
\pgfpathrectangle{\pgfqpoint{3.352233in}{3.252941in}}{\pgfqpoint{2.407767in}{1.544118in}}%
\pgfusepath{clip}%
\pgfsetbuttcap%
\pgfsetroundjoin%
\pgfsetlinewidth{0.501875pt}%
\definecolor{currentstroke}{rgb}{0.896320,0.893616,0.096335}%
\pgfsetstrokecolor{currentstroke}%
\pgfsetdash{}{0pt}%
\pgfpathmoveto{\pgfqpoint{4.412190in}{4.044611in}}%
\pgfpathlineto{\pgfqpoint{4.359312in}{4.042687in}}%
\pgfusepath{stroke}%
\end{pgfscope}%
\begin{pgfscope}%
\pgfpathrectangle{\pgfqpoint{3.352233in}{3.252941in}}{\pgfqpoint{2.407767in}{1.544118in}}%
\pgfusepath{clip}%
\pgfsetbuttcap%
\pgfsetroundjoin%
\pgfsetlinewidth{0.501875pt}%
\definecolor{currentstroke}{rgb}{0.276022,0.044167,0.370164}%
\pgfsetstrokecolor{currentstroke}%
\pgfsetdash{}{0pt}%
\pgfpathmoveto{\pgfqpoint{5.206279in}{4.094492in}}%
\pgfpathlineto{\pgfqpoint{5.153304in}{4.094284in}}%
\pgfusepath{stroke}%
\end{pgfscope}%
\begin{pgfscope}%
\pgfpathrectangle{\pgfqpoint{3.352233in}{3.252941in}}{\pgfqpoint{2.407767in}{1.544118in}}%
\pgfusepath{clip}%
\pgfsetbuttcap%
\pgfsetroundjoin%
\pgfsetlinewidth{0.501875pt}%
\definecolor{currentstroke}{rgb}{0.282656,0.100196,0.422160}%
\pgfsetstrokecolor{currentstroke}%
\pgfsetdash{}{0pt}%
\pgfpathmoveto{\pgfqpoint{5.153304in}{4.094284in}}%
\pgfpathlineto{\pgfqpoint{5.100331in}{4.093956in}}%
\pgfusepath{stroke}%
\end{pgfscope}%
\begin{pgfscope}%
\pgfpathrectangle{\pgfqpoint{3.352233in}{3.252941in}}{\pgfqpoint{2.407767in}{1.544118in}}%
\pgfusepath{clip}%
\pgfsetbuttcap%
\pgfsetroundjoin%
\pgfsetlinewidth{0.501875pt}%
\definecolor{currentstroke}{rgb}{0.280255,0.165693,0.476498}%
\pgfsetstrokecolor{currentstroke}%
\pgfsetdash{}{0pt}%
\pgfpathmoveto{\pgfqpoint{5.100331in}{4.093956in}}%
\pgfpathlineto{\pgfqpoint{5.047361in}{4.093452in}}%
\pgfusepath{stroke}%
\end{pgfscope}%
\begin{pgfscope}%
\pgfpathrectangle{\pgfqpoint{3.352233in}{3.252941in}}{\pgfqpoint{2.407767in}{1.544118in}}%
\pgfusepath{clip}%
\pgfsetbuttcap%
\pgfsetroundjoin%
\pgfsetlinewidth{0.501875pt}%
\definecolor{currentstroke}{rgb}{0.267968,0.223549,0.512008}%
\pgfsetstrokecolor{currentstroke}%
\pgfsetdash{}{0pt}%
\pgfpathmoveto{\pgfqpoint{5.047361in}{4.093452in}}%
\pgfpathlineto{\pgfqpoint{4.994398in}{4.092717in}}%
\pgfusepath{stroke}%
\end{pgfscope}%
\begin{pgfscope}%
\pgfpathrectangle{\pgfqpoint{3.352233in}{3.252941in}}{\pgfqpoint{2.407767in}{1.544118in}}%
\pgfusepath{clip}%
\pgfsetbuttcap%
\pgfsetroundjoin%
\pgfsetlinewidth{0.501875pt}%
\definecolor{currentstroke}{rgb}{0.237441,0.305202,0.541921}%
\pgfsetstrokecolor{currentstroke}%
\pgfsetdash{}{0pt}%
\pgfpathmoveto{\pgfqpoint{4.994398in}{4.092717in}}%
\pgfpathlineto{\pgfqpoint{4.941444in}{4.091739in}}%
\pgfusepath{stroke}%
\end{pgfscope}%
\begin{pgfscope}%
\pgfpathrectangle{\pgfqpoint{3.352233in}{3.252941in}}{\pgfqpoint{2.407767in}{1.544118in}}%
\pgfusepath{clip}%
\pgfsetbuttcap%
\pgfsetroundjoin%
\pgfsetlinewidth{0.501875pt}%
\definecolor{currentstroke}{rgb}{0.188923,0.410910,0.556326}%
\pgfsetstrokecolor{currentstroke}%
\pgfsetdash{}{0pt}%
\pgfpathmoveto{\pgfqpoint{4.941444in}{4.091739in}}%
\pgfpathlineto{\pgfqpoint{4.888509in}{4.090426in}}%
\pgfusepath{stroke}%
\end{pgfscope}%
\begin{pgfscope}%
\pgfpathrectangle{\pgfqpoint{3.352233in}{3.252941in}}{\pgfqpoint{2.407767in}{1.544118in}}%
\pgfusepath{clip}%
\pgfsetbuttcap%
\pgfsetroundjoin%
\pgfsetlinewidth{0.501875pt}%
\definecolor{currentstroke}{rgb}{0.139147,0.533812,0.555298}%
\pgfsetstrokecolor{currentstroke}%
\pgfsetdash{}{0pt}%
\pgfpathmoveto{\pgfqpoint{4.888509in}{4.090426in}}%
\pgfpathlineto{\pgfqpoint{4.835604in}{4.088688in}}%
\pgfusepath{stroke}%
\end{pgfscope}%
\begin{pgfscope}%
\pgfpathrectangle{\pgfqpoint{3.352233in}{3.252941in}}{\pgfqpoint{2.407767in}{1.544118in}}%
\pgfusepath{clip}%
\pgfsetbuttcap%
\pgfsetroundjoin%
\pgfsetlinewidth{0.501875pt}%
\definecolor{currentstroke}{rgb}{0.119512,0.607464,0.540218}%
\pgfsetstrokecolor{currentstroke}%
\pgfsetdash{}{0pt}%
\pgfpathmoveto{\pgfqpoint{4.835604in}{4.088688in}}%
\pgfpathlineto{\pgfqpoint{4.782738in}{4.086512in}}%
\pgfusepath{stroke}%
\end{pgfscope}%
\begin{pgfscope}%
\pgfpathrectangle{\pgfqpoint{3.352233in}{3.252941in}}{\pgfqpoint{2.407767in}{1.544118in}}%
\pgfusepath{clip}%
\pgfsetbuttcap%
\pgfsetroundjoin%
\pgfsetlinewidth{0.501875pt}%
\definecolor{currentstroke}{rgb}{0.128087,0.647749,0.523491}%
\pgfsetstrokecolor{currentstroke}%
\pgfsetdash{}{0pt}%
\pgfpathmoveto{\pgfqpoint{4.782738in}{4.086512in}}%
\pgfpathlineto{\pgfqpoint{4.729913in}{4.083962in}}%
\pgfusepath{stroke}%
\end{pgfscope}%
\begin{pgfscope}%
\pgfpathrectangle{\pgfqpoint{3.352233in}{3.252941in}}{\pgfqpoint{2.407767in}{1.544118in}}%
\pgfusepath{clip}%
\pgfsetbuttcap%
\pgfsetroundjoin%
\pgfsetlinewidth{0.501875pt}%
\definecolor{currentstroke}{rgb}{0.150148,0.676631,0.506589}%
\pgfsetstrokecolor{currentstroke}%
\pgfsetdash{}{0pt}%
\pgfpathmoveto{\pgfqpoint{4.729913in}{4.083962in}}%
\pgfpathlineto{\pgfqpoint{4.677141in}{4.081017in}}%
\pgfusepath{stroke}%
\end{pgfscope}%
\begin{pgfscope}%
\pgfpathrectangle{\pgfqpoint{3.352233in}{3.252941in}}{\pgfqpoint{2.407767in}{1.544118in}}%
\pgfusepath{clip}%
\pgfsetbuttcap%
\pgfsetroundjoin%
\pgfsetlinewidth{0.501875pt}%
\definecolor{currentstroke}{rgb}{0.266941,0.748751,0.440573}%
\pgfsetstrokecolor{currentstroke}%
\pgfsetdash{}{0pt}%
\pgfpathmoveto{\pgfqpoint{4.677141in}{4.081017in}}%
\pgfpathlineto{\pgfqpoint{4.624435in}{4.077637in}}%
\pgfusepath{stroke}%
\end{pgfscope}%
\begin{pgfscope}%
\pgfpathrectangle{\pgfqpoint{3.352233in}{3.252941in}}{\pgfqpoint{2.407767in}{1.544118in}}%
\pgfusepath{clip}%
\pgfsetbuttcap%
\pgfsetroundjoin%
\pgfsetlinewidth{0.501875pt}%
\definecolor{currentstroke}{rgb}{0.276022,0.044167,0.370164}%
\pgfsetstrokecolor{currentstroke}%
\pgfsetdash{}{0pt}%
\pgfpathmoveto{\pgfqpoint{5.206279in}{4.129238in}}%
\pgfpathlineto{\pgfqpoint{5.153304in}{4.129030in}}%
\pgfusepath{stroke}%
\end{pgfscope}%
\begin{pgfscope}%
\pgfpathrectangle{\pgfqpoint{3.352233in}{3.252941in}}{\pgfqpoint{2.407767in}{1.544118in}}%
\pgfusepath{clip}%
\pgfsetbuttcap%
\pgfsetroundjoin%
\pgfsetlinewidth{0.501875pt}%
\definecolor{currentstroke}{rgb}{0.282327,0.094955,0.417331}%
\pgfsetstrokecolor{currentstroke}%
\pgfsetdash{}{0pt}%
\pgfpathmoveto{\pgfqpoint{5.153304in}{4.129030in}}%
\pgfpathlineto{\pgfqpoint{5.100332in}{4.128672in}}%
\pgfusepath{stroke}%
\end{pgfscope}%
\begin{pgfscope}%
\pgfpathrectangle{\pgfqpoint{3.352233in}{3.252941in}}{\pgfqpoint{2.407767in}{1.544118in}}%
\pgfusepath{clip}%
\pgfsetbuttcap%
\pgfsetroundjoin%
\pgfsetlinewidth{0.501875pt}%
\definecolor{currentstroke}{rgb}{0.281887,0.150881,0.465405}%
\pgfsetstrokecolor{currentstroke}%
\pgfsetdash{}{0pt}%
\pgfpathmoveto{\pgfqpoint{5.100332in}{4.128672in}}%
\pgfpathlineto{\pgfqpoint{5.047367in}{4.128011in}}%
\pgfusepath{stroke}%
\end{pgfscope}%
\begin{pgfscope}%
\pgfpathrectangle{\pgfqpoint{3.352233in}{3.252941in}}{\pgfqpoint{2.407767in}{1.544118in}}%
\pgfusepath{clip}%
\pgfsetbuttcap%
\pgfsetroundjoin%
\pgfsetlinewidth{0.501875pt}%
\definecolor{currentstroke}{rgb}{0.270595,0.214069,0.507052}%
\pgfsetstrokecolor{currentstroke}%
\pgfsetdash{}{0pt}%
\pgfpathmoveto{\pgfqpoint{5.047367in}{4.128011in}}%
\pgfpathlineto{\pgfqpoint{4.994415in}{4.127007in}}%
\pgfusepath{stroke}%
\end{pgfscope}%
\begin{pgfscope}%
\pgfpathrectangle{\pgfqpoint{3.352233in}{3.252941in}}{\pgfqpoint{2.407767in}{1.544118in}}%
\pgfusepath{clip}%
\pgfsetbuttcap%
\pgfsetroundjoin%
\pgfsetlinewidth{0.501875pt}%
\definecolor{currentstroke}{rgb}{0.248629,0.278775,0.534556}%
\pgfsetstrokecolor{currentstroke}%
\pgfsetdash{}{0pt}%
\pgfpathmoveto{\pgfqpoint{4.994415in}{4.127007in}}%
\pgfpathlineto{\pgfqpoint{4.941480in}{4.125691in}}%
\pgfusepath{stroke}%
\end{pgfscope}%
\begin{pgfscope}%
\pgfpathrectangle{\pgfqpoint{3.352233in}{3.252941in}}{\pgfqpoint{2.407767in}{1.544118in}}%
\pgfusepath{clip}%
\pgfsetbuttcap%
\pgfsetroundjoin%
\pgfsetlinewidth{0.501875pt}%
\definecolor{currentstroke}{rgb}{0.220057,0.343307,0.549413}%
\pgfsetstrokecolor{currentstroke}%
\pgfsetdash{}{0pt}%
\pgfpathmoveto{\pgfqpoint{4.941480in}{4.125691in}}%
\pgfpathlineto{\pgfqpoint{4.888580in}{4.123890in}}%
\pgfusepath{stroke}%
\end{pgfscope}%
\begin{pgfscope}%
\pgfpathrectangle{\pgfqpoint{3.352233in}{3.252941in}}{\pgfqpoint{2.407767in}{1.544118in}}%
\pgfusepath{clip}%
\pgfsetbuttcap%
\pgfsetroundjoin%
\pgfsetlinewidth{0.501875pt}%
\definecolor{currentstroke}{rgb}{0.182256,0.426184,0.557120}%
\pgfsetstrokecolor{currentstroke}%
\pgfsetdash{}{0pt}%
\pgfpathmoveto{\pgfqpoint{4.888580in}{4.123890in}}%
\pgfpathlineto{\pgfqpoint{4.835746in}{4.121450in}}%
\pgfusepath{stroke}%
\end{pgfscope}%
\begin{pgfscope}%
\pgfpathrectangle{\pgfqpoint{3.352233in}{3.252941in}}{\pgfqpoint{2.407767in}{1.544118in}}%
\pgfusepath{clip}%
\pgfsetbuttcap%
\pgfsetroundjoin%
\pgfsetlinewidth{0.501875pt}%
\definecolor{currentstroke}{rgb}{0.136408,0.541173,0.554483}%
\pgfsetstrokecolor{currentstroke}%
\pgfsetdash{}{0pt}%
\pgfpathmoveto{\pgfqpoint{4.835746in}{4.121450in}}%
\pgfpathlineto{\pgfqpoint{4.783002in}{4.118303in}}%
\pgfusepath{stroke}%
\end{pgfscope}%
\begin{pgfscope}%
\pgfpathrectangle{\pgfqpoint{3.352233in}{3.252941in}}{\pgfqpoint{2.407767in}{1.544118in}}%
\pgfusepath{clip}%
\pgfsetbuttcap%
\pgfsetroundjoin%
\pgfsetlinewidth{0.501875pt}%
\definecolor{currentstroke}{rgb}{0.122606,0.585371,0.546557}%
\pgfsetstrokecolor{currentstroke}%
\pgfsetdash{}{0pt}%
\pgfpathmoveto{\pgfqpoint{4.783002in}{4.118303in}}%
\pgfpathlineto{\pgfqpoint{4.730354in}{4.114542in}}%
\pgfusepath{stroke}%
\end{pgfscope}%
\begin{pgfscope}%
\pgfpathrectangle{\pgfqpoint{3.352233in}{3.252941in}}{\pgfqpoint{2.407767in}{1.544118in}}%
\pgfusepath{clip}%
\pgfsetbuttcap%
\pgfsetroundjoin%
\pgfsetlinewidth{0.501875pt}%
\definecolor{currentstroke}{rgb}{0.273809,0.031497,0.358853}%
\pgfsetstrokecolor{currentstroke}%
\pgfsetdash{}{0pt}%
\pgfpathmoveto{\pgfqpoint{5.206279in}{4.163984in}}%
\pgfpathlineto{\pgfqpoint{5.153305in}{4.163861in}}%
\pgfusepath{stroke}%
\end{pgfscope}%
\begin{pgfscope}%
\pgfpathrectangle{\pgfqpoint{3.352233in}{3.252941in}}{\pgfqpoint{2.407767in}{1.544118in}}%
\pgfusepath{clip}%
\pgfsetbuttcap%
\pgfsetroundjoin%
\pgfsetlinewidth{0.501875pt}%
\definecolor{currentstroke}{rgb}{0.281924,0.089666,0.412415}%
\pgfsetstrokecolor{currentstroke}%
\pgfsetdash{}{0pt}%
\pgfpathmoveto{\pgfqpoint{5.153305in}{4.163861in}}%
\pgfpathlineto{\pgfqpoint{5.100339in}{4.163247in}}%
\pgfusepath{stroke}%
\end{pgfscope}%
\begin{pgfscope}%
\pgfpathrectangle{\pgfqpoint{3.352233in}{3.252941in}}{\pgfqpoint{2.407767in}{1.544118in}}%
\pgfusepath{clip}%
\pgfsetbuttcap%
\pgfsetroundjoin%
\pgfsetlinewidth{0.501875pt}%
\definecolor{currentstroke}{rgb}{0.283072,0.130895,0.449241}%
\pgfsetstrokecolor{currentstroke}%
\pgfsetdash{}{0pt}%
\pgfpathmoveto{\pgfqpoint{5.100339in}{4.163247in}}%
\pgfpathlineto{\pgfqpoint{5.047383in}{4.162329in}}%
\pgfusepath{stroke}%
\end{pgfscope}%
\begin{pgfscope}%
\pgfpathrectangle{\pgfqpoint{3.352233in}{3.252941in}}{\pgfqpoint{2.407767in}{1.544118in}}%
\pgfusepath{clip}%
\pgfsetbuttcap%
\pgfsetroundjoin%
\pgfsetlinewidth{0.501875pt}%
\definecolor{currentstroke}{rgb}{0.277134,0.185228,0.489898}%
\pgfsetstrokecolor{currentstroke}%
\pgfsetdash{}{0pt}%
\pgfpathmoveto{\pgfqpoint{5.047383in}{4.162329in}}%
\pgfpathlineto{\pgfqpoint{4.994445in}{4.161070in}}%
\pgfusepath{stroke}%
\end{pgfscope}%
\begin{pgfscope}%
\pgfpathrectangle{\pgfqpoint{3.352233in}{3.252941in}}{\pgfqpoint{2.407767in}{1.544118in}}%
\pgfusepath{clip}%
\pgfsetbuttcap%
\pgfsetroundjoin%
\pgfsetlinewidth{0.501875pt}%
\definecolor{currentstroke}{rgb}{0.265145,0.232956,0.516599}%
\pgfsetstrokecolor{currentstroke}%
\pgfsetdash{}{0pt}%
\pgfpathmoveto{\pgfqpoint{4.994445in}{4.161070in}}%
\pgfpathlineto{\pgfqpoint{4.941551in}{4.159216in}}%
\pgfusepath{stroke}%
\end{pgfscope}%
\begin{pgfscope}%
\pgfpathrectangle{\pgfqpoint{3.352233in}{3.252941in}}{\pgfqpoint{2.407767in}{1.544118in}}%
\pgfusepath{clip}%
\pgfsetbuttcap%
\pgfsetroundjoin%
\pgfsetlinewidth{0.501875pt}%
\definecolor{currentstroke}{rgb}{0.241237,0.296485,0.539709}%
\pgfsetstrokecolor{currentstroke}%
\pgfsetdash{}{0pt}%
\pgfpathmoveto{\pgfqpoint{4.941551in}{4.159216in}}%
\pgfpathlineto{\pgfqpoint{4.888734in}{4.156617in}}%
\pgfusepath{stroke}%
\end{pgfscope}%
\begin{pgfscope}%
\pgfpathrectangle{\pgfqpoint{3.352233in}{3.252941in}}{\pgfqpoint{2.407767in}{1.544118in}}%
\pgfusepath{clip}%
\pgfsetbuttcap%
\pgfsetroundjoin%
\pgfsetlinewidth{0.501875pt}%
\definecolor{currentstroke}{rgb}{0.216210,0.351535,0.550627}%
\pgfsetstrokecolor{currentstroke}%
\pgfsetdash{}{0pt}%
\pgfpathmoveto{\pgfqpoint{4.888734in}{4.156617in}}%
\pgfpathlineto{\pgfqpoint{4.836017in}{4.153290in}}%
\pgfusepath{stroke}%
\end{pgfscope}%
\begin{pgfscope}%
\pgfpathrectangle{\pgfqpoint{3.352233in}{3.252941in}}{\pgfqpoint{2.407767in}{1.544118in}}%
\pgfusepath{clip}%
\pgfsetbuttcap%
\pgfsetroundjoin%
\pgfsetlinewidth{0.501875pt}%
\definecolor{currentstroke}{rgb}{0.177423,0.437527,0.557565}%
\pgfsetstrokecolor{currentstroke}%
\pgfsetdash{}{0pt}%
\pgfpathmoveto{\pgfqpoint{4.836017in}{4.153290in}}%
\pgfpathlineto{\pgfqpoint{4.783445in}{4.149147in}}%
\pgfusepath{stroke}%
\end{pgfscope}%
\begin{pgfscope}%
\pgfpathrectangle{\pgfqpoint{3.352233in}{3.252941in}}{\pgfqpoint{2.407767in}{1.544118in}}%
\pgfusepath{clip}%
\pgfsetbuttcap%
\pgfsetroundjoin%
\pgfsetlinewidth{0.501875pt}%
\definecolor{currentstroke}{rgb}{0.274952,0.037752,0.364543}%
\pgfsetstrokecolor{currentstroke}%
\pgfsetdash{}{0pt}%
\pgfpathmoveto{\pgfqpoint{5.206279in}{4.198731in}}%
\pgfpathlineto{\pgfqpoint{5.153309in}{4.198338in}}%
\pgfusepath{stroke}%
\end{pgfscope}%
\begin{pgfscope}%
\pgfpathrectangle{\pgfqpoint{3.352233in}{3.252941in}}{\pgfqpoint{2.407767in}{1.544118in}}%
\pgfusepath{clip}%
\pgfsetbuttcap%
\pgfsetroundjoin%
\pgfsetlinewidth{0.501875pt}%
\definecolor{currentstroke}{rgb}{0.280267,0.073417,0.397163}%
\pgfsetstrokecolor{currentstroke}%
\pgfsetdash{}{0pt}%
\pgfpathmoveto{\pgfqpoint{5.153309in}{4.198338in}}%
\pgfpathlineto{\pgfqpoint{5.100342in}{4.197801in}}%
\pgfusepath{stroke}%
\end{pgfscope}%
\begin{pgfscope}%
\pgfpathrectangle{\pgfqpoint{3.352233in}{3.252941in}}{\pgfqpoint{2.407767in}{1.544118in}}%
\pgfusepath{clip}%
\pgfsetbuttcap%
\pgfsetroundjoin%
\pgfsetlinewidth{0.501875pt}%
\definecolor{currentstroke}{rgb}{0.283197,0.115680,0.436115}%
\pgfsetstrokecolor{currentstroke}%
\pgfsetdash{}{0pt}%
\pgfpathmoveto{\pgfqpoint{5.100342in}{4.197801in}}%
\pgfpathlineto{\pgfqpoint{5.047380in}{4.197038in}}%
\pgfusepath{stroke}%
\end{pgfscope}%
\begin{pgfscope}%
\pgfpathrectangle{\pgfqpoint{3.352233in}{3.252941in}}{\pgfqpoint{2.407767in}{1.544118in}}%
\pgfusepath{clip}%
\pgfsetbuttcap%
\pgfsetroundjoin%
\pgfsetlinewidth{0.501875pt}%
\definecolor{currentstroke}{rgb}{0.280868,0.160771,0.472899}%
\pgfsetstrokecolor{currentstroke}%
\pgfsetdash{}{0pt}%
\pgfpathmoveto{\pgfqpoint{5.047380in}{4.197038in}}%
\pgfpathlineto{\pgfqpoint{4.994443in}{4.195781in}}%
\pgfusepath{stroke}%
\end{pgfscope}%
\begin{pgfscope}%
\pgfpathrectangle{\pgfqpoint{3.352233in}{3.252941in}}{\pgfqpoint{2.407767in}{1.544118in}}%
\pgfusepath{clip}%
\pgfsetbuttcap%
\pgfsetroundjoin%
\pgfsetlinewidth{0.501875pt}%
\definecolor{currentstroke}{rgb}{0.273006,0.204520,0.501721}%
\pgfsetstrokecolor{currentstroke}%
\pgfsetdash{}{0pt}%
\pgfpathmoveto{\pgfqpoint{4.994443in}{4.195781in}}%
\pgfpathlineto{\pgfqpoint{4.941549in}{4.193916in}}%
\pgfusepath{stroke}%
\end{pgfscope}%
\begin{pgfscope}%
\pgfpathrectangle{\pgfqpoint{3.352233in}{3.252941in}}{\pgfqpoint{2.407767in}{1.544118in}}%
\pgfusepath{clip}%
\pgfsetbuttcap%
\pgfsetroundjoin%
\pgfsetlinewidth{0.501875pt}%
\definecolor{currentstroke}{rgb}{0.267968,0.223549,0.512008}%
\pgfsetstrokecolor{currentstroke}%
\pgfsetdash{}{0pt}%
\pgfpathmoveto{\pgfqpoint{4.941549in}{4.193916in}}%
\pgfpathlineto{\pgfqpoint{4.888717in}{4.191428in}}%
\pgfusepath{stroke}%
\end{pgfscope}%
\begin{pgfscope}%
\pgfpathrectangle{\pgfqpoint{3.352233in}{3.252941in}}{\pgfqpoint{2.407767in}{1.544118in}}%
\pgfusepath{clip}%
\pgfsetbuttcap%
\pgfsetroundjoin%
\pgfsetlinewidth{0.501875pt}%
\definecolor{currentstroke}{rgb}{0.248629,0.278775,0.534556}%
\pgfsetstrokecolor{currentstroke}%
\pgfsetdash{}{0pt}%
\pgfpathmoveto{\pgfqpoint{4.888717in}{4.191428in}}%
\pgfpathlineto{\pgfqpoint{4.836008in}{4.188069in}}%
\pgfusepath{stroke}%
\end{pgfscope}%
\begin{pgfscope}%
\pgfpathrectangle{\pgfqpoint{3.352233in}{3.252941in}}{\pgfqpoint{2.407767in}{1.544118in}}%
\pgfusepath{clip}%
\pgfsetbuttcap%
\pgfsetroundjoin%
\pgfsetlinewidth{0.501875pt}%
\definecolor{currentstroke}{rgb}{0.248629,0.278775,0.534556}%
\pgfsetstrokecolor{currentstroke}%
\pgfsetdash{}{0pt}%
\pgfpathmoveto{\pgfqpoint{4.836008in}{4.188069in}}%
\pgfpathlineto{\pgfqpoint{4.783505in}{4.183579in}}%
\pgfusepath{stroke}%
\end{pgfscope}%
\begin{pgfscope}%
\pgfpathrectangle{\pgfqpoint{3.352233in}{3.252941in}}{\pgfqpoint{2.407767in}{1.544118in}}%
\pgfusepath{clip}%
\pgfsetbuttcap%
\pgfsetroundjoin%
\pgfsetlinewidth{0.501875pt}%
\definecolor{currentstroke}{rgb}{0.206756,0.371758,0.553117}%
\pgfsetstrokecolor{currentstroke}%
\pgfsetdash{}{0pt}%
\pgfpathmoveto{\pgfqpoint{4.783505in}{4.183579in}}%
\pgfpathlineto{\pgfqpoint{4.731311in}{4.177816in}}%
\pgfusepath{stroke}%
\end{pgfscope}%
\begin{pgfscope}%
\pgfpathrectangle{\pgfqpoint{3.352233in}{3.252941in}}{\pgfqpoint{2.407767in}{1.544118in}}%
\pgfusepath{clip}%
\pgfsetbuttcap%
\pgfsetroundjoin%
\pgfsetlinewidth{0.501875pt}%
\definecolor{currentstroke}{rgb}{0.157729,0.485932,0.558013}%
\pgfsetstrokecolor{currentstroke}%
\pgfsetdash{}{0pt}%
\pgfpathmoveto{\pgfqpoint{4.731311in}{4.177816in}}%
\pgfpathlineto{\pgfqpoint{4.679507in}{4.170748in}}%
\pgfusepath{stroke}%
\end{pgfscope}%
\begin{pgfscope}%
\pgfpathrectangle{\pgfqpoint{3.352233in}{3.252941in}}{\pgfqpoint{2.407767in}{1.544118in}}%
\pgfusepath{clip}%
\pgfsetbuttcap%
\pgfsetroundjoin%
\pgfsetlinewidth{0.501875pt}%
\definecolor{currentstroke}{rgb}{0.151918,0.500685,0.557587}%
\pgfsetstrokecolor{currentstroke}%
\pgfsetdash{}{0pt}%
\pgfpathmoveto{\pgfqpoint{4.679507in}{4.170748in}}%
\pgfpathlineto{\pgfqpoint{4.628060in}{4.162660in}}%
\pgfusepath{stroke}%
\end{pgfscope}%
\begin{pgfscope}%
\pgfpathrectangle{\pgfqpoint{3.352233in}{3.252941in}}{\pgfqpoint{2.407767in}{1.544118in}}%
\pgfusepath{clip}%
\pgfsetbuttcap%
\pgfsetroundjoin%
\pgfsetlinewidth{0.501875pt}%
\definecolor{currentstroke}{rgb}{0.171176,0.452530,0.557965}%
\pgfsetstrokecolor{currentstroke}%
\pgfsetdash{}{0pt}%
\pgfpathmoveto{\pgfqpoint{4.628060in}{4.162660in}}%
\pgfpathlineto{\pgfqpoint{4.577063in}{4.153498in}}%
\pgfusepath{stroke}%
\end{pgfscope}%
\begin{pgfscope}%
\pgfpathrectangle{\pgfqpoint{3.352233in}{3.252941in}}{\pgfqpoint{2.407767in}{1.544118in}}%
\pgfusepath{clip}%
\pgfsetbuttcap%
\pgfsetroundjoin%
\pgfsetlinewidth{0.501875pt}%
\definecolor{currentstroke}{rgb}{0.169646,0.456262,0.558030}%
\pgfsetstrokecolor{currentstroke}%
\pgfsetdash{}{0pt}%
\pgfpathmoveto{\pgfqpoint{4.577063in}{4.153498in}}%
\pgfpathlineto{\pgfqpoint{4.526481in}{4.143432in}}%
\pgfusepath{stroke}%
\end{pgfscope}%
\begin{pgfscope}%
\pgfpathrectangle{\pgfqpoint{3.352233in}{3.252941in}}{\pgfqpoint{2.407767in}{1.544118in}}%
\pgfusepath{clip}%
\pgfsetbuttcap%
\pgfsetroundjoin%
\pgfsetlinewidth{0.501875pt}%
\definecolor{currentstroke}{rgb}{0.119423,0.611141,0.538982}%
\pgfsetstrokecolor{currentstroke}%
\pgfsetdash{}{0pt}%
\pgfpathmoveto{\pgfqpoint{4.526481in}{4.143432in}}%
\pgfpathlineto{\pgfqpoint{4.476126in}{4.132907in}}%
\pgfusepath{stroke}%
\end{pgfscope}%
\begin{pgfscope}%
\pgfpathrectangle{\pgfqpoint{3.352233in}{3.252941in}}{\pgfqpoint{2.407767in}{1.544118in}}%
\pgfusepath{clip}%
\pgfsetbuttcap%
\pgfsetroundjoin%
\pgfsetlinewidth{0.501875pt}%
\definecolor{currentstroke}{rgb}{0.259857,0.745492,0.444467}%
\pgfsetstrokecolor{currentstroke}%
\pgfsetdash{}{0pt}%
\pgfpathmoveto{\pgfqpoint{4.476126in}{4.132907in}}%
\pgfpathlineto{\pgfqpoint{4.425941in}{4.122055in}}%
\pgfusepath{stroke}%
\end{pgfscope}%
\begin{pgfscope}%
\pgfpathrectangle{\pgfqpoint{3.352233in}{3.252941in}}{\pgfqpoint{2.407767in}{1.544118in}}%
\pgfusepath{clip}%
\pgfsetbuttcap%
\pgfsetroundjoin%
\pgfsetlinewidth{0.501875pt}%
\definecolor{currentstroke}{rgb}{0.296479,0.761561,0.424223}%
\pgfsetstrokecolor{currentstroke}%
\pgfsetdash{}{0pt}%
\pgfpathmoveto{\pgfqpoint{4.425941in}{4.122055in}}%
\pgfpathlineto{\pgfqpoint{4.375792in}{4.111156in}}%
\pgfusepath{stroke}%
\end{pgfscope}%
\begin{pgfscope}%
\pgfpathrectangle{\pgfqpoint{3.352233in}{3.252941in}}{\pgfqpoint{2.407767in}{1.544118in}}%
\pgfusepath{clip}%
\pgfsetbuttcap%
\pgfsetroundjoin%
\pgfsetlinewidth{0.501875pt}%
\definecolor{currentstroke}{rgb}{0.535621,0.835785,0.281908}%
\pgfsetstrokecolor{currentstroke}%
\pgfsetdash{}{0pt}%
\pgfpathmoveto{\pgfqpoint{4.375792in}{4.111156in}}%
\pgfpathlineto{\pgfqpoint{4.325471in}{4.100579in}}%
\pgfusepath{stroke}%
\end{pgfscope}%
\begin{pgfscope}%
\pgfpathrectangle{\pgfqpoint{3.352233in}{3.252941in}}{\pgfqpoint{2.407767in}{1.544118in}}%
\pgfusepath{clip}%
\pgfsetbuttcap%
\pgfsetroundjoin%
\pgfsetlinewidth{0.501875pt}%
\definecolor{currentstroke}{rgb}{0.273809,0.031497,0.358853}%
\pgfsetstrokecolor{currentstroke}%
\pgfsetdash{}{0pt}%
\pgfpathmoveto{\pgfqpoint{5.206279in}{4.233477in}}%
\pgfpathlineto{\pgfqpoint{5.153307in}{4.233106in}}%
\pgfusepath{stroke}%
\end{pgfscope}%
\begin{pgfscope}%
\pgfpathrectangle{\pgfqpoint{3.352233in}{3.252941in}}{\pgfqpoint{2.407767in}{1.544118in}}%
\pgfusepath{clip}%
\pgfsetbuttcap%
\pgfsetroundjoin%
\pgfsetlinewidth{0.501875pt}%
\definecolor{currentstroke}{rgb}{0.280267,0.073417,0.397163}%
\pgfsetstrokecolor{currentstroke}%
\pgfsetdash{}{0pt}%
\pgfpathmoveto{\pgfqpoint{5.153307in}{4.233106in}}%
\pgfpathlineto{\pgfqpoint{5.100345in}{4.232359in}}%
\pgfusepath{stroke}%
\end{pgfscope}%
\begin{pgfscope}%
\pgfpathrectangle{\pgfqpoint{3.352233in}{3.252941in}}{\pgfqpoint{2.407767in}{1.544118in}}%
\pgfusepath{clip}%
\pgfsetbuttcap%
\pgfsetroundjoin%
\pgfsetlinewidth{0.501875pt}%
\definecolor{currentstroke}{rgb}{0.283091,0.110553,0.431554}%
\pgfsetstrokecolor{currentstroke}%
\pgfsetdash{}{0pt}%
\pgfpathmoveto{\pgfqpoint{5.100345in}{4.232359in}}%
\pgfpathlineto{\pgfqpoint{5.047403in}{4.231178in}}%
\pgfusepath{stroke}%
\end{pgfscope}%
\begin{pgfscope}%
\pgfpathrectangle{\pgfqpoint{3.352233in}{3.252941in}}{\pgfqpoint{2.407767in}{1.544118in}}%
\pgfusepath{clip}%
\pgfsetbuttcap%
\pgfsetroundjoin%
\pgfsetlinewidth{0.501875pt}%
\definecolor{currentstroke}{rgb}{0.282623,0.140926,0.457517}%
\pgfsetstrokecolor{currentstroke}%
\pgfsetdash{}{0pt}%
\pgfpathmoveto{\pgfqpoint{5.047403in}{4.231178in}}%
\pgfpathlineto{\pgfqpoint{4.994487in}{4.229586in}}%
\pgfusepath{stroke}%
\end{pgfscope}%
\begin{pgfscope}%
\pgfpathrectangle{\pgfqpoint{3.352233in}{3.252941in}}{\pgfqpoint{2.407767in}{1.544118in}}%
\pgfusepath{clip}%
\pgfsetbuttcap%
\pgfsetroundjoin%
\pgfsetlinewidth{0.501875pt}%
\definecolor{currentstroke}{rgb}{0.280255,0.165693,0.476498}%
\pgfsetstrokecolor{currentstroke}%
\pgfsetdash{}{0pt}%
\pgfpathmoveto{\pgfqpoint{4.994487in}{4.229586in}}%
\pgfpathlineto{\pgfqpoint{4.941615in}{4.227490in}}%
\pgfusepath{stroke}%
\end{pgfscope}%
\begin{pgfscope}%
\pgfpathrectangle{\pgfqpoint{3.352233in}{3.252941in}}{\pgfqpoint{2.407767in}{1.544118in}}%
\pgfusepath{clip}%
\pgfsetbuttcap%
\pgfsetroundjoin%
\pgfsetlinewidth{0.501875pt}%
\definecolor{currentstroke}{rgb}{0.274128,0.199721,0.498911}%
\pgfsetstrokecolor{currentstroke}%
\pgfsetdash{}{0pt}%
\pgfpathmoveto{\pgfqpoint{4.941615in}{4.227490in}}%
\pgfpathlineto{\pgfqpoint{4.888868in}{4.224413in}}%
\pgfusepath{stroke}%
\end{pgfscope}%
\begin{pgfscope}%
\pgfpathrectangle{\pgfqpoint{3.352233in}{3.252941in}}{\pgfqpoint{2.407767in}{1.544118in}}%
\pgfusepath{clip}%
\pgfsetbuttcap%
\pgfsetroundjoin%
\pgfsetlinewidth{0.501875pt}%
\definecolor{currentstroke}{rgb}{0.267968,0.223549,0.512008}%
\pgfsetstrokecolor{currentstroke}%
\pgfsetdash{}{0pt}%
\pgfpathmoveto{\pgfqpoint{4.888868in}{4.224413in}}%
\pgfpathlineto{\pgfqpoint{4.836273in}{4.220361in}}%
\pgfusepath{stroke}%
\end{pgfscope}%
\begin{pgfscope}%
\pgfpathrectangle{\pgfqpoint{3.352233in}{3.252941in}}{\pgfqpoint{2.407767in}{1.544118in}}%
\pgfusepath{clip}%
\pgfsetbuttcap%
\pgfsetroundjoin%
\pgfsetlinewidth{0.501875pt}%
\definecolor{currentstroke}{rgb}{0.273809,0.031497,0.358853}%
\pgfsetstrokecolor{currentstroke}%
\pgfsetdash{}{0pt}%
\pgfpathmoveto{\pgfqpoint{5.206279in}{4.268223in}}%
\pgfpathlineto{\pgfqpoint{5.153310in}{4.267719in}}%
\pgfusepath{stroke}%
\end{pgfscope}%
\begin{pgfscope}%
\pgfpathrectangle{\pgfqpoint{3.352233in}{3.252941in}}{\pgfqpoint{2.407767in}{1.544118in}}%
\pgfusepath{clip}%
\pgfsetbuttcap%
\pgfsetroundjoin%
\pgfsetlinewidth{0.501875pt}%
\definecolor{currentstroke}{rgb}{0.279566,0.067836,0.391917}%
\pgfsetstrokecolor{currentstroke}%
\pgfsetdash{}{0pt}%
\pgfpathmoveto{\pgfqpoint{5.153310in}{4.267719in}}%
\pgfpathlineto{\pgfqpoint{5.100353in}{4.266835in}}%
\pgfusepath{stroke}%
\end{pgfscope}%
\begin{pgfscope}%
\pgfpathrectangle{\pgfqpoint{3.352233in}{3.252941in}}{\pgfqpoint{2.407767in}{1.544118in}}%
\pgfusepath{clip}%
\pgfsetbuttcap%
\pgfsetroundjoin%
\pgfsetlinewidth{0.501875pt}%
\definecolor{currentstroke}{rgb}{0.281446,0.084320,0.407414}%
\pgfsetstrokecolor{currentstroke}%
\pgfsetdash{}{0pt}%
\pgfpathmoveto{\pgfqpoint{5.100353in}{4.266835in}}%
\pgfpathlineto{\pgfqpoint{5.047409in}{4.265675in}}%
\pgfusepath{stroke}%
\end{pgfscope}%
\begin{pgfscope}%
\pgfpathrectangle{\pgfqpoint{3.352233in}{3.252941in}}{\pgfqpoint{2.407767in}{1.544118in}}%
\pgfusepath{clip}%
\pgfsetbuttcap%
\pgfsetroundjoin%
\pgfsetlinewidth{0.501875pt}%
\definecolor{currentstroke}{rgb}{0.283072,0.130895,0.449241}%
\pgfsetstrokecolor{currentstroke}%
\pgfsetdash{}{0pt}%
\pgfpathmoveto{\pgfqpoint{5.047409in}{4.265675in}}%
\pgfpathlineto{\pgfqpoint{4.994494in}{4.264070in}}%
\pgfusepath{stroke}%
\end{pgfscope}%
\begin{pgfscope}%
\pgfpathrectangle{\pgfqpoint{3.352233in}{3.252941in}}{\pgfqpoint{2.407767in}{1.544118in}}%
\pgfusepath{clip}%
\pgfsetbuttcap%
\pgfsetroundjoin%
\pgfsetlinewidth{0.501875pt}%
\definecolor{currentstroke}{rgb}{0.282290,0.145912,0.461510}%
\pgfsetstrokecolor{currentstroke}%
\pgfsetdash{}{0pt}%
\pgfpathmoveto{\pgfqpoint{4.994494in}{4.264070in}}%
\pgfpathlineto{\pgfqpoint{4.941637in}{4.261831in}}%
\pgfusepath{stroke}%
\end{pgfscope}%
\begin{pgfscope}%
\pgfpathrectangle{\pgfqpoint{3.352233in}{3.252941in}}{\pgfqpoint{2.407767in}{1.544118in}}%
\pgfusepath{clip}%
\pgfsetbuttcap%
\pgfsetroundjoin%
\pgfsetlinewidth{0.501875pt}%
\definecolor{currentstroke}{rgb}{0.282884,0.135920,0.453427}%
\pgfsetstrokecolor{currentstroke}%
\pgfsetdash{}{0pt}%
\pgfpathmoveto{\pgfqpoint{4.941637in}{4.261831in}}%
\pgfpathlineto{\pgfqpoint{4.888887in}{4.258740in}}%
\pgfusepath{stroke}%
\end{pgfscope}%
\begin{pgfscope}%
\pgfpathrectangle{\pgfqpoint{3.352233in}{3.252941in}}{\pgfqpoint{2.407767in}{1.544118in}}%
\pgfusepath{clip}%
\pgfsetbuttcap%
\pgfsetroundjoin%
\pgfsetlinewidth{0.501875pt}%
\definecolor{currentstroke}{rgb}{0.281412,0.155834,0.469201}%
\pgfsetstrokecolor{currentstroke}%
\pgfsetdash{}{0pt}%
\pgfpathmoveto{\pgfqpoint{4.888887in}{4.258740in}}%
\pgfpathlineto{\pgfqpoint{4.836366in}{4.254372in}}%
\pgfusepath{stroke}%
\end{pgfscope}%
\begin{pgfscope}%
\pgfpathrectangle{\pgfqpoint{3.352233in}{3.252941in}}{\pgfqpoint{2.407767in}{1.544118in}}%
\pgfusepath{clip}%
\pgfsetbuttcap%
\pgfsetroundjoin%
\pgfsetlinewidth{0.501875pt}%
\definecolor{currentstroke}{rgb}{0.280255,0.165693,0.476498}%
\pgfsetstrokecolor{currentstroke}%
\pgfsetdash{}{0pt}%
\pgfpathmoveto{\pgfqpoint{4.836366in}{4.254372in}}%
\pgfpathlineto{\pgfqpoint{4.784176in}{4.248607in}}%
\pgfusepath{stroke}%
\end{pgfscope}%
\begin{pgfscope}%
\pgfpathrectangle{\pgfqpoint{3.352233in}{3.252941in}}{\pgfqpoint{2.407767in}{1.544118in}}%
\pgfusepath{clip}%
\pgfsetbuttcap%
\pgfsetroundjoin%
\pgfsetlinewidth{0.501875pt}%
\definecolor{currentstroke}{rgb}{0.273006,0.204520,0.501721}%
\pgfsetstrokecolor{currentstroke}%
\pgfsetdash{}{0pt}%
\pgfpathmoveto{\pgfqpoint{4.784176in}{4.248607in}}%
\pgfpathlineto{\pgfqpoint{4.732404in}{4.241534in}}%
\pgfusepath{stroke}%
\end{pgfscope}%
\begin{pgfscope}%
\pgfpathrectangle{\pgfqpoint{3.352233in}{3.252941in}}{\pgfqpoint{2.407767in}{1.544118in}}%
\pgfusepath{clip}%
\pgfsetbuttcap%
\pgfsetroundjoin%
\pgfsetlinewidth{0.501875pt}%
\definecolor{currentstroke}{rgb}{0.271828,0.209303,0.504434}%
\pgfsetstrokecolor{currentstroke}%
\pgfsetdash{}{0pt}%
\pgfpathmoveto{\pgfqpoint{4.732404in}{4.241534in}}%
\pgfpathlineto{\pgfqpoint{4.681330in}{4.232648in}}%
\pgfusepath{stroke}%
\end{pgfscope}%
\begin{pgfscope}%
\pgfpathrectangle{\pgfqpoint{3.352233in}{3.252941in}}{\pgfqpoint{2.407767in}{1.544118in}}%
\pgfusepath{clip}%
\pgfsetbuttcap%
\pgfsetroundjoin%
\pgfsetlinewidth{0.501875pt}%
\definecolor{currentstroke}{rgb}{0.237441,0.305202,0.541921}%
\pgfsetstrokecolor{currentstroke}%
\pgfsetdash{}{0pt}%
\pgfpathmoveto{\pgfqpoint{4.681330in}{4.232648in}}%
\pgfpathlineto{\pgfqpoint{4.631047in}{4.221987in}}%
\pgfusepath{stroke}%
\end{pgfscope}%
\begin{pgfscope}%
\pgfpathrectangle{\pgfqpoint{3.352233in}{3.252941in}}{\pgfqpoint{2.407767in}{1.544118in}}%
\pgfusepath{clip}%
\pgfsetbuttcap%
\pgfsetroundjoin%
\pgfsetlinewidth{0.501875pt}%
\definecolor{currentstroke}{rgb}{0.231674,0.318106,0.544834}%
\pgfsetstrokecolor{currentstroke}%
\pgfsetdash{}{0pt}%
\pgfpathmoveto{\pgfqpoint{4.631047in}{4.221987in}}%
\pgfpathlineto{\pgfqpoint{4.581678in}{4.209716in}}%
\pgfusepath{stroke}%
\end{pgfscope}%
\begin{pgfscope}%
\pgfpathrectangle{\pgfqpoint{3.352233in}{3.252941in}}{\pgfqpoint{2.407767in}{1.544118in}}%
\pgfusepath{clip}%
\pgfsetbuttcap%
\pgfsetroundjoin%
\pgfsetlinewidth{0.501875pt}%
\definecolor{currentstroke}{rgb}{0.212395,0.359683,0.551710}%
\pgfsetstrokecolor{currentstroke}%
\pgfsetdash{}{0pt}%
\pgfpathmoveto{\pgfqpoint{4.581678in}{4.209716in}}%
\pgfpathlineto{\pgfqpoint{4.533349in}{4.195831in}}%
\pgfusepath{stroke}%
\end{pgfscope}%
\begin{pgfscope}%
\pgfpathrectangle{\pgfqpoint{3.352233in}{3.252941in}}{\pgfqpoint{2.407767in}{1.544118in}}%
\pgfusepath{clip}%
\pgfsetbuttcap%
\pgfsetroundjoin%
\pgfsetlinewidth{0.501875pt}%
\definecolor{currentstroke}{rgb}{0.208623,0.367752,0.552675}%
\pgfsetstrokecolor{currentstroke}%
\pgfsetdash{}{0pt}%
\pgfpathmoveto{\pgfqpoint{4.533349in}{4.195831in}}%
\pgfpathlineto{\pgfqpoint{4.485906in}{4.180733in}}%
\pgfusepath{stroke}%
\end{pgfscope}%
\begin{pgfscope}%
\pgfpathrectangle{\pgfqpoint{3.352233in}{3.252941in}}{\pgfqpoint{2.407767in}{1.544118in}}%
\pgfusepath{clip}%
\pgfsetbuttcap%
\pgfsetroundjoin%
\pgfsetlinewidth{0.501875pt}%
\definecolor{currentstroke}{rgb}{0.146180,0.515413,0.556823}%
\pgfsetstrokecolor{currentstroke}%
\pgfsetdash{}{0pt}%
\pgfpathmoveto{\pgfqpoint{4.485906in}{4.180733in}}%
\pgfpathlineto{\pgfqpoint{4.438697in}{4.165338in}}%
\pgfusepath{stroke}%
\end{pgfscope}%
\begin{pgfscope}%
\pgfpathrectangle{\pgfqpoint{3.352233in}{3.252941in}}{\pgfqpoint{2.407767in}{1.544118in}}%
\pgfusepath{clip}%
\pgfsetbuttcap%
\pgfsetroundjoin%
\pgfsetlinewidth{0.501875pt}%
\definecolor{currentstroke}{rgb}{0.124780,0.640461,0.527068}%
\pgfsetstrokecolor{currentstroke}%
\pgfsetdash{}{0pt}%
\pgfpathmoveto{\pgfqpoint{4.438697in}{4.165338in}}%
\pgfpathlineto{\pgfqpoint{4.391416in}{4.150044in}}%
\pgfusepath{stroke}%
\end{pgfscope}%
\begin{pgfscope}%
\pgfpathrectangle{\pgfqpoint{3.352233in}{3.252941in}}{\pgfqpoint{2.407767in}{1.544118in}}%
\pgfusepath{clip}%
\pgfsetbuttcap%
\pgfsetroundjoin%
\pgfsetlinewidth{0.501875pt}%
\definecolor{currentstroke}{rgb}{0.274149,0.751988,0.436601}%
\pgfsetstrokecolor{currentstroke}%
\pgfsetdash{}{0pt}%
\pgfpathmoveto{\pgfqpoint{4.391416in}{4.150044in}}%
\pgfpathlineto{\pgfqpoint{4.344087in}{4.134820in}}%
\pgfusepath{stroke}%
\end{pgfscope}%
\begin{pgfscope}%
\pgfpathrectangle{\pgfqpoint{3.352233in}{3.252941in}}{\pgfqpoint{2.407767in}{1.544118in}}%
\pgfusepath{clip}%
\pgfsetbuttcap%
\pgfsetroundjoin%
\pgfsetlinewidth{0.501875pt}%
\definecolor{currentstroke}{rgb}{0.259857,0.745492,0.444467}%
\pgfsetstrokecolor{currentstroke}%
\pgfsetdash{}{0pt}%
\pgfpathmoveto{\pgfqpoint{4.344087in}{4.134820in}}%
\pgfpathlineto{\pgfqpoint{4.296443in}{4.120028in}}%
\pgfusepath{stroke}%
\end{pgfscope}%
\begin{pgfscope}%
\pgfpathrectangle{\pgfqpoint{3.352233in}{3.252941in}}{\pgfqpoint{2.407767in}{1.544118in}}%
\pgfusepath{clip}%
\pgfsetbuttcap%
\pgfsetroundjoin%
\pgfsetlinewidth{0.501875pt}%
\definecolor{currentstroke}{rgb}{0.272594,0.025563,0.353093}%
\pgfsetstrokecolor{currentstroke}%
\pgfsetdash{}{0pt}%
\pgfpathmoveto{\pgfqpoint{5.206279in}{4.302969in}}%
\pgfpathlineto{\pgfqpoint{5.153331in}{4.301892in}}%
\pgfusepath{stroke}%
\end{pgfscope}%
\begin{pgfscope}%
\pgfpathrectangle{\pgfqpoint{3.352233in}{3.252941in}}{\pgfqpoint{2.407767in}{1.544118in}}%
\pgfusepath{clip}%
\pgfsetbuttcap%
\pgfsetroundjoin%
\pgfsetlinewidth{0.501875pt}%
\definecolor{currentstroke}{rgb}{0.277018,0.050344,0.375715}%
\pgfsetstrokecolor{currentstroke}%
\pgfsetdash{}{0pt}%
\pgfpathmoveto{\pgfqpoint{5.153331in}{4.301892in}}%
\pgfpathlineto{\pgfqpoint{5.100394in}{4.300619in}}%
\pgfusepath{stroke}%
\end{pgfscope}%
\begin{pgfscope}%
\pgfpathrectangle{\pgfqpoint{3.352233in}{3.252941in}}{\pgfqpoint{2.407767in}{1.544118in}}%
\pgfusepath{clip}%
\pgfsetbuttcap%
\pgfsetroundjoin%
\pgfsetlinewidth{0.501875pt}%
\definecolor{currentstroke}{rgb}{0.280894,0.078907,0.402329}%
\pgfsetstrokecolor{currentstroke}%
\pgfsetdash{}{0pt}%
\pgfpathmoveto{\pgfqpoint{5.100394in}{4.300619in}}%
\pgfpathlineto{\pgfqpoint{5.047478in}{4.299052in}}%
\pgfusepath{stroke}%
\end{pgfscope}%
\begin{pgfscope}%
\pgfpathrectangle{\pgfqpoint{3.352233in}{3.252941in}}{\pgfqpoint{2.407767in}{1.544118in}}%
\pgfusepath{clip}%
\pgfsetbuttcap%
\pgfsetroundjoin%
\pgfsetlinewidth{0.501875pt}%
\definecolor{currentstroke}{rgb}{0.282656,0.100196,0.422160}%
\pgfsetstrokecolor{currentstroke}%
\pgfsetdash{}{0pt}%
\pgfpathmoveto{\pgfqpoint{5.047478in}{4.299052in}}%
\pgfpathlineto{\pgfqpoint{4.994618in}{4.296902in}}%
\pgfusepath{stroke}%
\end{pgfscope}%
\begin{pgfscope}%
\pgfpathrectangle{\pgfqpoint{3.352233in}{3.252941in}}{\pgfqpoint{2.407767in}{1.544118in}}%
\pgfusepath{clip}%
\pgfsetbuttcap%
\pgfsetroundjoin%
\pgfsetlinewidth{0.501875pt}%
\definecolor{currentstroke}{rgb}{0.283187,0.125848,0.444960}%
\pgfsetstrokecolor{currentstroke}%
\pgfsetdash{}{0pt}%
\pgfpathmoveto{\pgfqpoint{4.994618in}{4.296902in}}%
\pgfpathlineto{\pgfqpoint{4.941808in}{4.294216in}}%
\pgfusepath{stroke}%
\end{pgfscope}%
\begin{pgfscope}%
\pgfpathrectangle{\pgfqpoint{3.352233in}{3.252941in}}{\pgfqpoint{2.407767in}{1.544118in}}%
\pgfusepath{clip}%
\pgfsetbuttcap%
\pgfsetroundjoin%
\pgfsetlinewidth{0.501875pt}%
\definecolor{currentstroke}{rgb}{0.282327,0.094955,0.417331}%
\pgfsetstrokecolor{currentstroke}%
\pgfsetdash{}{0pt}%
\pgfpathmoveto{\pgfqpoint{4.941808in}{4.294216in}}%
\pgfpathlineto{\pgfqpoint{4.889126in}{4.290736in}}%
\pgfusepath{stroke}%
\end{pgfscope}%
\begin{pgfscope}%
\pgfpathrectangle{\pgfqpoint{3.352233in}{3.252941in}}{\pgfqpoint{2.407767in}{1.544118in}}%
\pgfusepath{clip}%
\pgfsetbuttcap%
\pgfsetroundjoin%
\pgfsetlinewidth{0.501875pt}%
\definecolor{currentstroke}{rgb}{0.283091,0.110553,0.431554}%
\pgfsetstrokecolor{currentstroke}%
\pgfsetdash{}{0pt}%
\pgfpathmoveto{\pgfqpoint{4.889126in}{4.290736in}}%
\pgfpathlineto{\pgfqpoint{4.836605in}{4.286391in}}%
\pgfusepath{stroke}%
\end{pgfscope}%
\begin{pgfscope}%
\pgfpathrectangle{\pgfqpoint{3.352233in}{3.252941in}}{\pgfqpoint{2.407767in}{1.544118in}}%
\pgfusepath{clip}%
\pgfsetbuttcap%
\pgfsetroundjoin%
\pgfsetlinewidth{0.501875pt}%
\definecolor{currentstroke}{rgb}{0.282884,0.135920,0.453427}%
\pgfsetstrokecolor{currentstroke}%
\pgfsetdash{}{0pt}%
\pgfpathmoveto{\pgfqpoint{4.836605in}{4.286391in}}%
\pgfpathlineto{\pgfqpoint{4.784775in}{4.279872in}}%
\pgfusepath{stroke}%
\end{pgfscope}%
\begin{pgfscope}%
\pgfpathrectangle{\pgfqpoint{3.352233in}{3.252941in}}{\pgfqpoint{2.407767in}{1.544118in}}%
\pgfusepath{clip}%
\pgfsetbuttcap%
\pgfsetroundjoin%
\pgfsetlinewidth{0.501875pt}%
\definecolor{currentstroke}{rgb}{0.271305,0.019942,0.347269}%
\pgfsetstrokecolor{currentstroke}%
\pgfsetdash{}{0pt}%
\pgfpathmoveto{\pgfqpoint{5.206279in}{4.337715in}}%
\pgfpathlineto{\pgfqpoint{5.153309in}{4.337366in}}%
\pgfusepath{stroke}%
\end{pgfscope}%
\begin{pgfscope}%
\pgfpathrectangle{\pgfqpoint{3.352233in}{3.252941in}}{\pgfqpoint{2.407767in}{1.544118in}}%
\pgfusepath{clip}%
\pgfsetbuttcap%
\pgfsetroundjoin%
\pgfsetlinewidth{0.501875pt}%
\definecolor{currentstroke}{rgb}{0.277018,0.050344,0.375715}%
\pgfsetstrokecolor{currentstroke}%
\pgfsetdash{}{0pt}%
\pgfpathmoveto{\pgfqpoint{5.153309in}{4.337366in}}%
\pgfpathlineto{\pgfqpoint{5.100364in}{4.336291in}}%
\pgfusepath{stroke}%
\end{pgfscope}%
\begin{pgfscope}%
\pgfpathrectangle{\pgfqpoint{3.352233in}{3.252941in}}{\pgfqpoint{2.407767in}{1.544118in}}%
\pgfusepath{clip}%
\pgfsetbuttcap%
\pgfsetroundjoin%
\pgfsetlinewidth{0.501875pt}%
\definecolor{currentstroke}{rgb}{0.278791,0.062145,0.386592}%
\pgfsetstrokecolor{currentstroke}%
\pgfsetdash{}{0pt}%
\pgfpathmoveto{\pgfqpoint{5.100364in}{4.336291in}}%
\pgfpathlineto{\pgfqpoint{5.047456in}{4.334611in}}%
\pgfusepath{stroke}%
\end{pgfscope}%
\begin{pgfscope}%
\pgfpathrectangle{\pgfqpoint{3.352233in}{3.252941in}}{\pgfqpoint{2.407767in}{1.544118in}}%
\pgfusepath{clip}%
\pgfsetbuttcap%
\pgfsetroundjoin%
\pgfsetlinewidth{0.501875pt}%
\definecolor{currentstroke}{rgb}{0.280894,0.078907,0.402329}%
\pgfsetstrokecolor{currentstroke}%
\pgfsetdash{}{0pt}%
\pgfpathmoveto{\pgfqpoint{5.047456in}{4.334611in}}%
\pgfpathlineto{\pgfqpoint{4.994587in}{4.332495in}}%
\pgfusepath{stroke}%
\end{pgfscope}%
\begin{pgfscope}%
\pgfpathrectangle{\pgfqpoint{3.352233in}{3.252941in}}{\pgfqpoint{2.407767in}{1.544118in}}%
\pgfusepath{clip}%
\pgfsetbuttcap%
\pgfsetroundjoin%
\pgfsetlinewidth{0.501875pt}%
\definecolor{currentstroke}{rgb}{0.280267,0.073417,0.397163}%
\pgfsetstrokecolor{currentstroke}%
\pgfsetdash{}{0pt}%
\pgfpathmoveto{\pgfqpoint{4.994587in}{4.332495in}}%
\pgfpathlineto{\pgfqpoint{4.941754in}{4.330021in}}%
\pgfusepath{stroke}%
\end{pgfscope}%
\begin{pgfscope}%
\pgfpathrectangle{\pgfqpoint{3.352233in}{3.252941in}}{\pgfqpoint{2.407767in}{1.544118in}}%
\pgfusepath{clip}%
\pgfsetbuttcap%
\pgfsetroundjoin%
\pgfsetlinewidth{0.501875pt}%
\definecolor{currentstroke}{rgb}{0.282656,0.100196,0.422160}%
\pgfsetstrokecolor{currentstroke}%
\pgfsetdash{}{0pt}%
\pgfpathmoveto{\pgfqpoint{4.941754in}{4.330021in}}%
\pgfpathlineto{\pgfqpoint{4.889033in}{4.326761in}}%
\pgfusepath{stroke}%
\end{pgfscope}%
\begin{pgfscope}%
\pgfpathrectangle{\pgfqpoint{3.352233in}{3.252941in}}{\pgfqpoint{2.407767in}{1.544118in}}%
\pgfusepath{clip}%
\pgfsetbuttcap%
\pgfsetroundjoin%
\pgfsetlinewidth{0.501875pt}%
\definecolor{currentstroke}{rgb}{0.283197,0.115680,0.436115}%
\pgfsetstrokecolor{currentstroke}%
\pgfsetdash{}{0pt}%
\pgfpathmoveto{\pgfqpoint{4.889033in}{4.326761in}}%
\pgfpathlineto{\pgfqpoint{4.836622in}{4.321959in}}%
\pgfusepath{stroke}%
\end{pgfscope}%
\begin{pgfscope}%
\pgfpathrectangle{\pgfqpoint{3.352233in}{3.252941in}}{\pgfqpoint{2.407767in}{1.544118in}}%
\pgfusepath{clip}%
\pgfsetbuttcap%
\pgfsetroundjoin%
\pgfsetlinewidth{0.501875pt}%
\definecolor{currentstroke}{rgb}{0.282656,0.100196,0.422160}%
\pgfsetstrokecolor{currentstroke}%
\pgfsetdash{}{0pt}%
\pgfpathmoveto{\pgfqpoint{4.836622in}{4.321959in}}%
\pgfpathlineto{\pgfqpoint{4.785109in}{4.314419in}}%
\pgfusepath{stroke}%
\end{pgfscope}%
\begin{pgfscope}%
\pgfpathrectangle{\pgfqpoint{3.352233in}{3.252941in}}{\pgfqpoint{2.407767in}{1.544118in}}%
\pgfusepath{clip}%
\pgfsetbuttcap%
\pgfsetroundjoin%
\pgfsetlinewidth{0.501875pt}%
\definecolor{currentstroke}{rgb}{0.269944,0.014625,0.341379}%
\pgfsetstrokecolor{currentstroke}%
\pgfsetdash{}{0pt}%
\pgfpathmoveto{\pgfqpoint{5.206279in}{4.441953in}}%
\pgfpathlineto{\pgfqpoint{5.153323in}{4.441424in}}%
\pgfusepath{stroke}%
\end{pgfscope}%
\begin{pgfscope}%
\pgfpathrectangle{\pgfqpoint{3.352233in}{3.252941in}}{\pgfqpoint{2.407767in}{1.544118in}}%
\pgfusepath{clip}%
\pgfsetbuttcap%
\pgfsetroundjoin%
\pgfsetlinewidth{0.501875pt}%
\definecolor{currentstroke}{rgb}{0.272594,0.025563,0.353093}%
\pgfsetstrokecolor{currentstroke}%
\pgfsetdash{}{0pt}%
\pgfpathmoveto{\pgfqpoint{5.153323in}{4.441424in}}%
\pgfpathlineto{\pgfqpoint{5.100400in}{4.440293in}}%
\pgfusepath{stroke}%
\end{pgfscope}%
\begin{pgfscope}%
\pgfpathrectangle{\pgfqpoint{3.352233in}{3.252941in}}{\pgfqpoint{2.407767in}{1.544118in}}%
\pgfusepath{clip}%
\pgfsetbuttcap%
\pgfsetroundjoin%
\pgfsetlinewidth{0.501875pt}%
\definecolor{currentstroke}{rgb}{0.276022,0.044167,0.370164}%
\pgfsetstrokecolor{currentstroke}%
\pgfsetdash{}{0pt}%
\pgfpathmoveto{\pgfqpoint{5.100400in}{4.440293in}}%
\pgfpathlineto{\pgfqpoint{5.047522in}{4.438258in}}%
\pgfusepath{stroke}%
\end{pgfscope}%
\begin{pgfscope}%
\pgfpathrectangle{\pgfqpoint{3.352233in}{3.252941in}}{\pgfqpoint{2.407767in}{1.544118in}}%
\pgfusepath{clip}%
\pgfsetbuttcap%
\pgfsetroundjoin%
\pgfsetlinewidth{0.501875pt}%
\definecolor{currentstroke}{rgb}{0.276022,0.044167,0.370164}%
\pgfsetstrokecolor{currentstroke}%
\pgfsetdash{}{0pt}%
\pgfpathmoveto{\pgfqpoint{5.047522in}{4.438258in}}%
\pgfpathlineto{\pgfqpoint{4.994805in}{4.435135in}}%
\pgfusepath{stroke}%
\end{pgfscope}%
\begin{pgfscope}%
\pgfpathrectangle{\pgfqpoint{3.352233in}{3.252941in}}{\pgfqpoint{2.407767in}{1.544118in}}%
\pgfusepath{clip}%
\pgfsetbuttcap%
\pgfsetroundjoin%
\pgfsetlinewidth{0.501875pt}%
\definecolor{currentstroke}{rgb}{0.277018,0.050344,0.375715}%
\pgfsetstrokecolor{currentstroke}%
\pgfsetdash{}{0pt}%
\pgfpathmoveto{\pgfqpoint{4.994805in}{4.435135in}}%
\pgfpathlineto{\pgfqpoint{4.942162in}{4.431496in}}%
\pgfusepath{stroke}%
\end{pgfscope}%
\begin{pgfscope}%
\pgfpathrectangle{\pgfqpoint{3.352233in}{3.252941in}}{\pgfqpoint{2.407767in}{1.544118in}}%
\pgfusepath{clip}%
\pgfsetbuttcap%
\pgfsetroundjoin%
\pgfsetlinewidth{0.501875pt}%
\definecolor{currentstroke}{rgb}{0.277018,0.050344,0.375715}%
\pgfsetstrokecolor{currentstroke}%
\pgfsetdash{}{0pt}%
\pgfpathmoveto{\pgfqpoint{4.942162in}{4.431496in}}%
\pgfpathlineto{\pgfqpoint{4.889803in}{4.426755in}}%
\pgfusepath{stroke}%
\end{pgfscope}%
\begin{pgfscope}%
\pgfpathrectangle{\pgfqpoint{3.352233in}{3.252941in}}{\pgfqpoint{2.407767in}{1.544118in}}%
\pgfusepath{clip}%
\pgfsetbuttcap%
\pgfsetroundjoin%
\pgfsetlinewidth{0.501875pt}%
\definecolor{currentstroke}{rgb}{0.277018,0.050344,0.375715}%
\pgfsetstrokecolor{currentstroke}%
\pgfsetdash{}{0pt}%
\pgfpathmoveto{\pgfqpoint{4.889803in}{4.426755in}}%
\pgfpathlineto{\pgfqpoint{4.837669in}{4.421052in}}%
\pgfusepath{stroke}%
\end{pgfscope}%
\begin{pgfscope}%
\pgfpathrectangle{\pgfqpoint{3.352233in}{3.252941in}}{\pgfqpoint{2.407767in}{1.544118in}}%
\pgfusepath{clip}%
\pgfsetbuttcap%
\pgfsetroundjoin%
\pgfsetlinewidth{0.501875pt}%
\definecolor{currentstroke}{rgb}{0.271305,0.019942,0.347269}%
\pgfsetstrokecolor{currentstroke}%
\pgfsetdash{}{0pt}%
\pgfpathmoveto{\pgfqpoint{5.199518in}{4.418354in}}%
\pgfpathlineto{\pgfqpoint{5.146581in}{4.417101in}}%
\pgfusepath{stroke}%
\end{pgfscope}%
\begin{pgfscope}%
\pgfpathrectangle{\pgfqpoint{3.352233in}{3.252941in}}{\pgfqpoint{2.407767in}{1.544118in}}%
\pgfusepath{clip}%
\pgfsetbuttcap%
\pgfsetroundjoin%
\pgfsetlinewidth{0.501875pt}%
\definecolor{currentstroke}{rgb}{0.273809,0.031497,0.358853}%
\pgfsetstrokecolor{currentstroke}%
\pgfsetdash{}{0pt}%
\pgfpathmoveto{\pgfqpoint{5.146581in}{4.417101in}}%
\pgfpathlineto{\pgfqpoint{5.093646in}{4.415893in}}%
\pgfusepath{stroke}%
\end{pgfscope}%
\begin{pgfscope}%
\pgfpathrectangle{\pgfqpoint{3.352233in}{3.252941in}}{\pgfqpoint{2.407767in}{1.544118in}}%
\pgfusepath{clip}%
\pgfsetbuttcap%
\pgfsetroundjoin%
\pgfsetlinewidth{0.501875pt}%
\definecolor{currentstroke}{rgb}{0.276022,0.044167,0.370164}%
\pgfsetstrokecolor{currentstroke}%
\pgfsetdash{}{0pt}%
\pgfpathmoveto{\pgfqpoint{5.093646in}{4.415893in}}%
\pgfpathlineto{\pgfqpoint{5.040830in}{4.413521in}}%
\pgfusepath{stroke}%
\end{pgfscope}%
\begin{pgfscope}%
\pgfpathrectangle{\pgfqpoint{3.352233in}{3.252941in}}{\pgfqpoint{2.407767in}{1.544118in}}%
\pgfusepath{clip}%
\pgfsetbuttcap%
\pgfsetroundjoin%
\pgfsetlinewidth{0.501875pt}%
\definecolor{currentstroke}{rgb}{0.277018,0.050344,0.375715}%
\pgfsetstrokecolor{currentstroke}%
\pgfsetdash{}{0pt}%
\pgfpathmoveto{\pgfqpoint{5.040830in}{4.413521in}}%
\pgfpathlineto{\pgfqpoint{4.988107in}{4.410356in}}%
\pgfusepath{stroke}%
\end{pgfscope}%
\begin{pgfscope}%
\pgfpathrectangle{\pgfqpoint{3.352233in}{3.252941in}}{\pgfqpoint{2.407767in}{1.544118in}}%
\pgfusepath{clip}%
\pgfsetbuttcap%
\pgfsetroundjoin%
\pgfsetlinewidth{0.501875pt}%
\definecolor{currentstroke}{rgb}{0.280894,0.078907,0.402329}%
\pgfsetstrokecolor{currentstroke}%
\pgfsetdash{}{0pt}%
\pgfpathmoveto{\pgfqpoint{4.988107in}{4.410356in}}%
\pgfpathlineto{\pgfqpoint{4.935378in}{4.407207in}}%
\pgfusepath{stroke}%
\end{pgfscope}%
\begin{pgfscope}%
\pgfpathrectangle{\pgfqpoint{3.352233in}{3.252941in}}{\pgfqpoint{2.407767in}{1.544118in}}%
\pgfusepath{clip}%
\pgfsetbuttcap%
\pgfsetroundjoin%
\pgfsetlinewidth{0.501875pt}%
\definecolor{currentstroke}{rgb}{0.277941,0.056324,0.381191}%
\pgfsetstrokecolor{currentstroke}%
\pgfsetdash{}{0pt}%
\pgfpathmoveto{\pgfqpoint{4.935378in}{4.407207in}}%
\pgfpathlineto{\pgfqpoint{4.882815in}{4.403097in}}%
\pgfusepath{stroke}%
\end{pgfscope}%
\begin{pgfscope}%
\pgfpathrectangle{\pgfqpoint{3.352233in}{3.252941in}}{\pgfqpoint{2.407767in}{1.544118in}}%
\pgfusepath{clip}%
\pgfsetbuttcap%
\pgfsetroundjoin%
\pgfsetlinewidth{0.501875pt}%
\definecolor{currentstroke}{rgb}{0.272594,0.025563,0.353093}%
\pgfsetstrokecolor{currentstroke}%
\pgfsetdash{}{0pt}%
\pgfpathmoveto{\pgfqpoint{5.201113in}{4.379132in}}%
\pgfpathlineto{\pgfqpoint{5.148164in}{4.378484in}}%
\pgfusepath{stroke}%
\end{pgfscope}%
\begin{pgfscope}%
\pgfpathrectangle{\pgfqpoint{3.352233in}{3.252941in}}{\pgfqpoint{2.407767in}{1.544118in}}%
\pgfusepath{clip}%
\pgfsetbuttcap%
\pgfsetroundjoin%
\pgfsetlinewidth{0.501875pt}%
\definecolor{currentstroke}{rgb}{0.274952,0.037752,0.364543}%
\pgfsetstrokecolor{currentstroke}%
\pgfsetdash{}{0pt}%
\pgfpathmoveto{\pgfqpoint{5.148164in}{4.378484in}}%
\pgfpathlineto{\pgfqpoint{5.095233in}{4.377160in}}%
\pgfusepath{stroke}%
\end{pgfscope}%
\begin{pgfscope}%
\pgfpathrectangle{\pgfqpoint{3.352233in}{3.252941in}}{\pgfqpoint{2.407767in}{1.544118in}}%
\pgfusepath{clip}%
\pgfsetbuttcap%
\pgfsetroundjoin%
\pgfsetlinewidth{0.501875pt}%
\definecolor{currentstroke}{rgb}{0.277941,0.056324,0.381191}%
\pgfsetstrokecolor{currentstroke}%
\pgfsetdash{}{0pt}%
\pgfpathmoveto{\pgfqpoint{5.095233in}{4.377160in}}%
\pgfpathlineto{\pgfqpoint{5.042367in}{4.375111in}}%
\pgfusepath{stroke}%
\end{pgfscope}%
\begin{pgfscope}%
\pgfpathrectangle{\pgfqpoint{3.352233in}{3.252941in}}{\pgfqpoint{2.407767in}{1.544118in}}%
\pgfusepath{clip}%
\pgfsetbuttcap%
\pgfsetroundjoin%
\pgfsetlinewidth{0.501875pt}%
\definecolor{currentstroke}{rgb}{0.280894,0.078907,0.402329}%
\pgfsetstrokecolor{currentstroke}%
\pgfsetdash{}{0pt}%
\pgfpathmoveto{\pgfqpoint{5.042367in}{4.375111in}}%
\pgfpathlineto{\pgfqpoint{4.989558in}{4.372461in}}%
\pgfusepath{stroke}%
\end{pgfscope}%
\begin{pgfscope}%
\pgfpathrectangle{\pgfqpoint{3.352233in}{3.252941in}}{\pgfqpoint{2.407767in}{1.544118in}}%
\pgfusepath{clip}%
\pgfsetbuttcap%
\pgfsetroundjoin%
\pgfsetlinewidth{0.501875pt}%
\definecolor{currentstroke}{rgb}{0.280894,0.078907,0.402329}%
\pgfsetstrokecolor{currentstroke}%
\pgfsetdash{}{0pt}%
\pgfpathmoveto{\pgfqpoint{4.989558in}{4.372461in}}%
\pgfpathlineto{\pgfqpoint{4.936837in}{4.369214in}}%
\pgfusepath{stroke}%
\end{pgfscope}%
\begin{pgfscope}%
\pgfpathrectangle{\pgfqpoint{3.352233in}{3.252941in}}{\pgfqpoint{2.407767in}{1.544118in}}%
\pgfusepath{clip}%
\pgfsetbuttcap%
\pgfsetroundjoin%
\pgfsetlinewidth{0.501875pt}%
\definecolor{currentstroke}{rgb}{0.282656,0.100196,0.422160}%
\pgfsetstrokecolor{currentstroke}%
\pgfsetdash{}{0pt}%
\pgfpathmoveto{\pgfqpoint{4.936837in}{4.369214in}}%
\pgfpathlineto{\pgfqpoint{4.884503in}{4.364122in}}%
\pgfusepath{stroke}%
\end{pgfscope}%
\begin{pgfscope}%
\pgfpathrectangle{\pgfqpoint{3.352233in}{3.252941in}}{\pgfqpoint{2.407767in}{1.544118in}}%
\pgfusepath{clip}%
\pgfsetbuttcap%
\pgfsetroundjoin%
\pgfsetlinewidth{0.501875pt}%
\definecolor{currentstroke}{rgb}{0.282327,0.094955,0.417331}%
\pgfsetstrokecolor{currentstroke}%
\pgfsetdash{}{0pt}%
\pgfpathmoveto{\pgfqpoint{4.884503in}{4.364122in}}%
\pgfpathlineto{\pgfqpoint{4.832867in}{4.356727in}}%
\pgfusepath{stroke}%
\end{pgfscope}%
\begin{pgfscope}%
\pgfpathrectangle{\pgfqpoint{3.352233in}{3.252941in}}{\pgfqpoint{2.407767in}{1.544118in}}%
\pgfusepath{clip}%
\pgfsetbuttcap%
\pgfsetroundjoin%
\pgfsetlinewidth{0.501875pt}%
\definecolor{currentstroke}{rgb}{0.282327,0.094955,0.417331}%
\pgfsetstrokecolor{currentstroke}%
\pgfsetdash{}{0pt}%
\pgfpathmoveto{\pgfqpoint{4.832867in}{4.356727in}}%
\pgfpathlineto{\pgfqpoint{4.781246in}{4.349343in}}%
\pgfusepath{stroke}%
\end{pgfscope}%
\begin{pgfscope}%
\pgfpathrectangle{\pgfqpoint{3.352233in}{3.252941in}}{\pgfqpoint{2.407767in}{1.544118in}}%
\pgfusepath{clip}%
\pgfsetbuttcap%
\pgfsetroundjoin%
\pgfsetlinewidth{0.501875pt}%
\definecolor{currentstroke}{rgb}{0.282623,0.140926,0.457517}%
\pgfsetstrokecolor{currentstroke}%
\pgfsetdash{}{0pt}%
\pgfpathmoveto{\pgfqpoint{4.501936in}{4.302969in}}%
\pgfpathlineto{\pgfqpoint{4.501936in}{4.302969in}}%
\pgfusepath{stroke}%
\end{pgfscope}%
\begin{pgfscope}%
\pgfpathrectangle{\pgfqpoint{3.352233in}{3.252941in}}{\pgfqpoint{2.407767in}{1.544118in}}%
\pgfusepath{clip}%
\pgfsetbuttcap%
\pgfsetroundjoin%
\pgfsetlinewidth{0.501875pt}%
\definecolor{currentstroke}{rgb}{0.282623,0.140926,0.457517}%
\pgfsetstrokecolor{currentstroke}%
\pgfsetdash{}{0pt}%
\pgfpathmoveto{\pgfqpoint{4.501936in}{4.302969in}}%
\pgfpathlineto{\pgfqpoint{4.486677in}{4.282820in}}%
\pgfusepath{stroke}%
\end{pgfscope}%
\begin{pgfscope}%
\pgfpathrectangle{\pgfqpoint{3.352233in}{3.252941in}}{\pgfqpoint{2.407767in}{1.544118in}}%
\pgfusepath{clip}%
\pgfsetbuttcap%
\pgfsetroundjoin%
\pgfsetlinewidth{0.501875pt}%
\definecolor{currentstroke}{rgb}{0.270595,0.214069,0.507052}%
\pgfsetstrokecolor{currentstroke}%
\pgfsetdash{}{0pt}%
\pgfpathmoveto{\pgfqpoint{4.486677in}{4.282820in}}%
\pgfpathlineto{\pgfqpoint{4.466657in}{4.265784in}}%
\pgfusepath{stroke}%
\end{pgfscope}%
\begin{pgfscope}%
\pgfpathrectangle{\pgfqpoint{3.352233in}{3.252941in}}{\pgfqpoint{2.407767in}{1.544118in}}%
\pgfusepath{clip}%
\pgfsetbuttcap%
\pgfsetroundjoin%
\pgfsetlinewidth{0.501875pt}%
\definecolor{currentstroke}{rgb}{0.246811,0.283237,0.535941}%
\pgfsetstrokecolor{currentstroke}%
\pgfsetdash{}{0pt}%
\pgfpathmoveto{\pgfqpoint{4.466657in}{4.265784in}}%
\pgfpathlineto{\pgfqpoint{4.436118in}{4.242598in}}%
\pgfusepath{stroke}%
\end{pgfscope}%
\begin{pgfscope}%
\pgfpathrectangle{\pgfqpoint{3.352233in}{3.252941in}}{\pgfqpoint{2.407767in}{1.544118in}}%
\pgfusepath{clip}%
\pgfsetbuttcap%
\pgfsetroundjoin%
\pgfsetlinewidth{0.501875pt}%
\definecolor{currentstroke}{rgb}{0.229739,0.322361,0.545706}%
\pgfsetstrokecolor{currentstroke}%
\pgfsetdash{}{0pt}%
\pgfpathmoveto{\pgfqpoint{4.436118in}{4.242598in}}%
\pgfpathlineto{\pgfqpoint{4.401229in}{4.217096in}}%
\pgfusepath{stroke}%
\end{pgfscope}%
\begin{pgfscope}%
\pgfpathrectangle{\pgfqpoint{3.352233in}{3.252941in}}{\pgfqpoint{2.407767in}{1.544118in}}%
\pgfusepath{clip}%
\pgfsetbuttcap%
\pgfsetroundjoin%
\pgfsetlinewidth{0.501875pt}%
\definecolor{currentstroke}{rgb}{0.192357,0.403199,0.555836}%
\pgfsetstrokecolor{currentstroke}%
\pgfsetdash{}{0pt}%
\pgfpathmoveto{\pgfqpoint{4.401229in}{4.217096in}}%
\pgfpathlineto{\pgfqpoint{4.363577in}{4.193313in}}%
\pgfusepath{stroke}%
\end{pgfscope}%
\begin{pgfscope}%
\pgfpathrectangle{\pgfqpoint{3.352233in}{3.252941in}}{\pgfqpoint{2.407767in}{1.544118in}}%
\pgfusepath{clip}%
\pgfsetbuttcap%
\pgfsetroundjoin%
\pgfsetlinewidth{0.501875pt}%
\definecolor{currentstroke}{rgb}{0.160665,0.478540,0.558115}%
\pgfsetstrokecolor{currentstroke}%
\pgfsetdash{}{0pt}%
\pgfpathmoveto{\pgfqpoint{4.363577in}{4.193313in}}%
\pgfpathlineto{\pgfqpoint{4.323281in}{4.171345in}}%
\pgfusepath{stroke}%
\end{pgfscope}%
\begin{pgfscope}%
\pgfpathrectangle{\pgfqpoint{3.352233in}{3.252941in}}{\pgfqpoint{2.407767in}{1.544118in}}%
\pgfusepath{clip}%
\pgfsetbuttcap%
\pgfsetroundjoin%
\pgfsetlinewidth{0.501875pt}%
\definecolor{currentstroke}{rgb}{0.235526,0.309527,0.542944}%
\pgfsetstrokecolor{currentstroke}%
\pgfsetdash{}{0pt}%
\pgfpathmoveto{\pgfqpoint{4.501936in}{3.781777in}}%
\pgfpathlineto{\pgfqpoint{4.462975in}{3.804704in}}%
\pgfusepath{stroke}%
\end{pgfscope}%
\begin{pgfscope}%
\pgfpathrectangle{\pgfqpoint{3.352233in}{3.252941in}}{\pgfqpoint{2.407767in}{1.544118in}}%
\pgfusepath{clip}%
\pgfsetbuttcap%
\pgfsetroundjoin%
\pgfsetlinewidth{0.501875pt}%
\definecolor{currentstroke}{rgb}{0.229739,0.322361,0.545706}%
\pgfsetstrokecolor{currentstroke}%
\pgfsetdash{}{0pt}%
\pgfpathmoveto{\pgfqpoint{4.462975in}{3.804704in}}%
\pgfpathlineto{\pgfqpoint{4.424872in}{3.828212in}}%
\pgfusepath{stroke}%
\end{pgfscope}%
\begin{pgfscope}%
\pgfpathrectangle{\pgfqpoint{3.352233in}{3.252941in}}{\pgfqpoint{2.407767in}{1.544118in}}%
\pgfusepath{clip}%
\pgfsetbuttcap%
\pgfsetroundjoin%
\pgfsetlinewidth{0.501875pt}%
\definecolor{currentstroke}{rgb}{0.221989,0.339161,0.548752}%
\pgfsetstrokecolor{currentstroke}%
\pgfsetdash{}{0pt}%
\pgfpathmoveto{\pgfqpoint{4.424872in}{3.828212in}}%
\pgfpathlineto{\pgfqpoint{4.386001in}{3.851206in}}%
\pgfusepath{stroke}%
\end{pgfscope}%
\begin{pgfscope}%
\pgfpathrectangle{\pgfqpoint{3.352233in}{3.252941in}}{\pgfqpoint{2.407767in}{1.544118in}}%
\pgfusepath{clip}%
\pgfsetbuttcap%
\pgfsetroundjoin%
\pgfsetlinewidth{0.501875pt}%
\definecolor{currentstroke}{rgb}{0.180629,0.429975,0.557282}%
\pgfsetstrokecolor{currentstroke}%
\pgfsetdash{}{0pt}%
\pgfpathmoveto{\pgfqpoint{4.386001in}{3.851206in}}%
\pgfpathlineto{\pgfqpoint{4.344617in}{3.872303in}}%
\pgfusepath{stroke}%
\end{pgfscope}%
\begin{pgfscope}%
\pgfpathrectangle{\pgfqpoint{3.352233in}{3.252941in}}{\pgfqpoint{2.407767in}{1.544118in}}%
\pgfusepath{clip}%
\pgfsetbuttcap%
\pgfsetroundjoin%
\pgfsetlinewidth{0.501875pt}%
\definecolor{currentstroke}{rgb}{0.232815,0.732247,0.459277}%
\pgfsetstrokecolor{currentstroke}%
\pgfsetdash{}{0pt}%
\pgfpathmoveto{\pgfqpoint{4.344617in}{3.872303in}}%
\pgfpathlineto{\pgfqpoint{4.301981in}{3.892397in}}%
\pgfusepath{stroke}%
\end{pgfscope}%
\begin{pgfscope}%
\pgfpathrectangle{\pgfqpoint{3.352233in}{3.252941in}}{\pgfqpoint{2.407767in}{1.544118in}}%
\pgfusepath{clip}%
\pgfsetbuttcap%
\pgfsetroundjoin%
\pgfsetlinewidth{0.501875pt}%
\definecolor{currentstroke}{rgb}{0.281887,0.150881,0.465405}%
\pgfsetstrokecolor{currentstroke}%
\pgfsetdash{}{0pt}%
\pgfpathmoveto{\pgfqpoint{4.664477in}{4.268223in}}%
\pgfpathlineto{\pgfqpoint{4.615297in}{4.255731in}}%
\pgfusepath{stroke}%
\end{pgfscope}%
\begin{pgfscope}%
\pgfpathrectangle{\pgfqpoint{3.352233in}{3.252941in}}{\pgfqpoint{2.407767in}{1.544118in}}%
\pgfusepath{clip}%
\pgfsetbuttcap%
\pgfsetroundjoin%
\pgfsetlinewidth{0.501875pt}%
\definecolor{currentstroke}{rgb}{0.273006,0.204520,0.501721}%
\pgfsetstrokecolor{currentstroke}%
\pgfsetdash{}{0pt}%
\pgfpathmoveto{\pgfqpoint{4.615297in}{4.255731in}}%
\pgfpathlineto{\pgfqpoint{4.567806in}{4.240782in}}%
\pgfusepath{stroke}%
\end{pgfscope}%
\begin{pgfscope}%
\pgfpathrectangle{\pgfqpoint{3.352233in}{3.252941in}}{\pgfqpoint{2.407767in}{1.544118in}}%
\pgfusepath{clip}%
\pgfsetbuttcap%
\pgfsetroundjoin%
\pgfsetlinewidth{0.501875pt}%
\definecolor{currentstroke}{rgb}{0.237441,0.305202,0.541921}%
\pgfsetstrokecolor{currentstroke}%
\pgfsetdash{}{0pt}%
\pgfpathmoveto{\pgfqpoint{4.567806in}{4.240782in}}%
\pgfpathlineto{\pgfqpoint{4.521553in}{4.224257in}}%
\pgfusepath{stroke}%
\end{pgfscope}%
\begin{pgfscope}%
\pgfpathrectangle{\pgfqpoint{3.352233in}{3.252941in}}{\pgfqpoint{2.407767in}{1.544118in}}%
\pgfusepath{clip}%
\pgfsetbuttcap%
\pgfsetroundjoin%
\pgfsetlinewidth{0.501875pt}%
\definecolor{currentstroke}{rgb}{0.201239,0.383670,0.554294}%
\pgfsetstrokecolor{currentstroke}%
\pgfsetdash{}{0pt}%
\pgfpathmoveto{\pgfqpoint{4.521553in}{4.224257in}}%
\pgfpathlineto{\pgfqpoint{4.476930in}{4.206038in}}%
\pgfusepath{stroke}%
\end{pgfscope}%
\begin{pgfscope}%
\pgfpathrectangle{\pgfqpoint{3.352233in}{3.252941in}}{\pgfqpoint{2.407767in}{1.544118in}}%
\pgfusepath{clip}%
\pgfsetbuttcap%
\pgfsetroundjoin%
\pgfsetlinewidth{0.501875pt}%
\definecolor{currentstroke}{rgb}{0.227802,0.326594,0.546532}%
\pgfsetstrokecolor{currentstroke}%
\pgfsetdash{}{0pt}%
\pgfpathmoveto{\pgfqpoint{4.476930in}{4.206038in}}%
\pgfpathlineto{\pgfqpoint{4.433062in}{4.187062in}}%
\pgfusepath{stroke}%
\end{pgfscope}%
\begin{pgfscope}%
\pgfpathrectangle{\pgfqpoint{3.352233in}{3.252941in}}{\pgfqpoint{2.407767in}{1.544118in}}%
\pgfusepath{clip}%
\pgfsetbuttcap%
\pgfsetroundjoin%
\pgfsetlinewidth{0.501875pt}%
\definecolor{currentstroke}{rgb}{0.243113,0.292092,0.538516}%
\pgfsetstrokecolor{currentstroke}%
\pgfsetdash{}{0pt}%
\pgfpathmoveto{\pgfqpoint{4.603481in}{3.801333in}}%
\pgfpathlineto{\pgfqpoint{4.556117in}{3.816523in}}%
\pgfusepath{stroke}%
\end{pgfscope}%
\begin{pgfscope}%
\pgfpathrectangle{\pgfqpoint{3.352233in}{3.252941in}}{\pgfqpoint{2.407767in}{1.544118in}}%
\pgfusepath{clip}%
\pgfsetbuttcap%
\pgfsetroundjoin%
\pgfsetlinewidth{0.501875pt}%
\definecolor{currentstroke}{rgb}{0.210503,0.363727,0.552206}%
\pgfsetstrokecolor{currentstroke}%
\pgfsetdash{}{0pt}%
\pgfpathmoveto{\pgfqpoint{4.556117in}{3.816523in}}%
\pgfpathlineto{\pgfqpoint{4.509919in}{3.833088in}}%
\pgfusepath{stroke}%
\end{pgfscope}%
\begin{pgfscope}%
\pgfpathrectangle{\pgfqpoint{3.352233in}{3.252941in}}{\pgfqpoint{2.407767in}{1.544118in}}%
\pgfusepath{clip}%
\pgfsetbuttcap%
\pgfsetroundjoin%
\pgfsetlinewidth{0.501875pt}%
\definecolor{currentstroke}{rgb}{0.206756,0.371758,0.553117}%
\pgfsetstrokecolor{currentstroke}%
\pgfsetdash{}{0pt}%
\pgfpathmoveto{\pgfqpoint{4.509919in}{3.833088in}}%
\pgfpathlineto{\pgfqpoint{4.465021in}{3.851049in}}%
\pgfusepath{stroke}%
\end{pgfscope}%
\begin{pgfscope}%
\pgfpathrectangle{\pgfqpoint{3.352233in}{3.252941in}}{\pgfqpoint{2.407767in}{1.544118in}}%
\pgfusepath{clip}%
\pgfsetbuttcap%
\pgfsetroundjoin%
\pgfsetlinewidth{0.501875pt}%
\definecolor{currentstroke}{rgb}{0.199430,0.387607,0.554642}%
\pgfsetstrokecolor{currentstroke}%
\pgfsetdash{}{0pt}%
\pgfpathmoveto{\pgfqpoint{4.465021in}{3.851049in}}%
\pgfpathlineto{\pgfqpoint{4.420401in}{3.869295in}}%
\pgfusepath{stroke}%
\end{pgfscope}%
\begin{pgfscope}%
\pgfpathrectangle{\pgfqpoint{3.352233in}{3.252941in}}{\pgfqpoint{2.407767in}{1.544118in}}%
\pgfusepath{clip}%
\pgfsetbuttcap%
\pgfsetroundjoin%
\pgfsetlinewidth{0.501875pt}%
\definecolor{currentstroke}{rgb}{0.150476,0.504369,0.557430}%
\pgfsetstrokecolor{currentstroke}%
\pgfsetdash{}{0pt}%
\pgfpathmoveto{\pgfqpoint{4.420401in}{3.869295in}}%
\pgfpathlineto{\pgfqpoint{4.374967in}{3.886698in}}%
\pgfusepath{stroke}%
\end{pgfscope}%
\begin{pgfscope}%
\pgfpathrectangle{\pgfqpoint{3.352233in}{3.252941in}}{\pgfqpoint{2.407767in}{1.544118in}}%
\pgfusepath{clip}%
\pgfsetroundcap%
\pgfsetroundjoin%
\pgfsetlinewidth{0.501875pt}%
\definecolor{currentstroke}{rgb}{0.273809,0.031497,0.358853}%
\pgfsetstrokecolor{currentstroke}%
\pgfsetdash{}{0pt}%
\pgfpathmoveto{\pgfqpoint{5.100381in}{4.510005in}}%
\pgfpathquadraticcurveto{\pgfqpoint{5.087189in}{4.509499in}}{\pgfqpoint{5.081756in}{4.509290in}}%
\pgfusepath{stroke}%
\end{pgfscope}%
\begin{pgfscope}%
\pgfpathrectangle{\pgfqpoint{3.352233in}{3.252941in}}{\pgfqpoint{2.407767in}{1.544118in}}%
\pgfusepath{clip}%
\pgfsetroundcap%
\pgfsetroundjoin%
\definecolor{currentfill}{rgb}{0.273809,0.031497,0.358853}%
\pgfsetfillcolor{currentfill}%
\pgfsetlinewidth{0.501875pt}%
\definecolor{currentstroke}{rgb}{0.273809,0.031497,0.358853}%
\pgfsetstrokecolor{currentstroke}%
\pgfsetdash{}{0pt}%
\pgfpathmoveto{\pgfqpoint{5.110047in}{4.496477in}}%
\pgfpathlineto{\pgfqpoint{5.081756in}{4.509290in}}%
\pgfpathlineto{\pgfqpoint{5.108981in}{4.524234in}}%
\pgfpathlineto{\pgfqpoint{5.110047in}{4.496477in}}%
\pgfpathlineto{\pgfqpoint{5.110047in}{4.496477in}}%
\pgfpathclose%
\pgfusepath{stroke,fill}%
\end{pgfscope}%
\begin{pgfscope}%
\pgfpathrectangle{\pgfqpoint{3.352233in}{3.252941in}}{\pgfqpoint{2.407767in}{1.544118in}}%
\pgfusepath{clip}%
\pgfsetroundcap%
\pgfsetroundjoin%
\pgfsetlinewidth{0.501875pt}%
\definecolor{currentstroke}{rgb}{0.274952,0.037752,0.364543}%
\pgfsetstrokecolor{currentstroke}%
\pgfsetdash{}{0pt}%
\pgfpathmoveto{\pgfqpoint{5.047676in}{4.472463in}}%
\pgfpathquadraticcurveto{\pgfqpoint{5.034457in}{4.472022in}}{\pgfqpoint{5.028998in}{4.471840in}}%
\pgfusepath{stroke}%
\end{pgfscope}%
\begin{pgfscope}%
\pgfpathrectangle{\pgfqpoint{3.352233in}{3.252941in}}{\pgfqpoint{2.407767in}{1.544118in}}%
\pgfusepath{clip}%
\pgfsetroundcap%
\pgfsetroundjoin%
\definecolor{currentfill}{rgb}{0.274952,0.037752,0.364543}%
\pgfsetfillcolor{currentfill}%
\pgfsetlinewidth{0.501875pt}%
\definecolor{currentstroke}{rgb}{0.274952,0.037752,0.364543}%
\pgfsetstrokecolor{currentstroke}%
\pgfsetdash{}{0pt}%
\pgfpathmoveto{\pgfqpoint{5.057224in}{4.458885in}}%
\pgfpathlineto{\pgfqpoint{5.028998in}{4.471840in}}%
\pgfpathlineto{\pgfqpoint{5.056298in}{4.486647in}}%
\pgfpathlineto{\pgfqpoint{5.057224in}{4.458885in}}%
\pgfpathlineto{\pgfqpoint{5.057224in}{4.458885in}}%
\pgfpathclose%
\pgfusepath{stroke,fill}%
\end{pgfscope}%
\begin{pgfscope}%
\pgfpathrectangle{\pgfqpoint{3.352233in}{3.252941in}}{\pgfqpoint{2.407767in}{1.544118in}}%
\pgfusepath{clip}%
\pgfsetroundcap%
\pgfsetroundjoin%
\pgfsetlinewidth{0.501875pt}%
\definecolor{currentstroke}{rgb}{0.278791,0.062145,0.386592}%
\pgfsetstrokecolor{currentstroke}%
\pgfsetdash{}{0pt}%
\pgfpathmoveto{\pgfqpoint{5.047715in}{3.648098in}}%
\pgfpathquadraticcurveto{\pgfqpoint{5.034516in}{3.648749in}}{\pgfqpoint{5.029072in}{3.649017in}}%
\pgfusepath{stroke}%
\end{pgfscope}%
\begin{pgfscope}%
\pgfpathrectangle{\pgfqpoint{3.352233in}{3.252941in}}{\pgfqpoint{2.407767in}{1.544118in}}%
\pgfusepath{clip}%
\pgfsetroundcap%
\pgfsetroundjoin%
\definecolor{currentfill}{rgb}{0.278791,0.062145,0.386592}%
\pgfsetfillcolor{currentfill}%
\pgfsetlinewidth{0.501875pt}%
\definecolor{currentstroke}{rgb}{0.278791,0.062145,0.386592}%
\pgfsetstrokecolor{currentstroke}%
\pgfsetdash{}{0pt}%
\pgfpathmoveto{\pgfqpoint{5.056131in}{3.633776in}}%
\pgfpathlineto{\pgfqpoint{5.029072in}{3.649017in}}%
\pgfpathlineto{\pgfqpoint{5.057500in}{3.661520in}}%
\pgfpathlineto{\pgfqpoint{5.056131in}{3.633776in}}%
\pgfpathlineto{\pgfqpoint{5.056131in}{3.633776in}}%
\pgfpathclose%
\pgfusepath{stroke,fill}%
\end{pgfscope}%
\begin{pgfscope}%
\pgfpathrectangle{\pgfqpoint{3.352233in}{3.252941in}}{\pgfqpoint{2.407767in}{1.544118in}}%
\pgfusepath{clip}%
\pgfsetroundcap%
\pgfsetroundjoin%
\pgfsetlinewidth{0.501875pt}%
\definecolor{currentstroke}{rgb}{0.278791,0.062145,0.386592}%
\pgfsetstrokecolor{currentstroke}%
\pgfsetdash{}{0pt}%
\pgfpathmoveto{\pgfqpoint{5.047452in}{3.680655in}}%
\pgfpathquadraticcurveto{\pgfqpoint{5.034247in}{3.681288in}}{\pgfqpoint{5.028797in}{3.681550in}}%
\pgfusepath{stroke}%
\end{pgfscope}%
\begin{pgfscope}%
\pgfpathrectangle{\pgfqpoint{3.352233in}{3.252941in}}{\pgfqpoint{2.407767in}{1.544118in}}%
\pgfusepath{clip}%
\pgfsetroundcap%
\pgfsetroundjoin%
\definecolor{currentfill}{rgb}{0.278791,0.062145,0.386592}%
\pgfsetfillcolor{currentfill}%
\pgfsetlinewidth{0.501875pt}%
\definecolor{currentstroke}{rgb}{0.278791,0.062145,0.386592}%
\pgfsetstrokecolor{currentstroke}%
\pgfsetdash{}{0pt}%
\pgfpathmoveto{\pgfqpoint{5.055878in}{3.666347in}}%
\pgfpathlineto{\pgfqpoint{5.028797in}{3.681550in}}%
\pgfpathlineto{\pgfqpoint{5.057208in}{3.694092in}}%
\pgfpathlineto{\pgfqpoint{5.055878in}{3.666347in}}%
\pgfpathlineto{\pgfqpoint{5.055878in}{3.666347in}}%
\pgfpathclose%
\pgfusepath{stroke,fill}%
\end{pgfscope}%
\begin{pgfscope}%
\pgfpathrectangle{\pgfqpoint{3.352233in}{3.252941in}}{\pgfqpoint{2.407767in}{1.544118in}}%
\pgfusepath{clip}%
\pgfsetroundcap%
\pgfsetroundjoin%
\pgfsetlinewidth{0.501875pt}%
\definecolor{currentstroke}{rgb}{0.281446,0.084320,0.407414}%
\pgfsetstrokecolor{currentstroke}%
\pgfsetdash{}{0pt}%
\pgfpathmoveto{\pgfqpoint{5.047442in}{3.715396in}}%
\pgfpathquadraticcurveto{\pgfqpoint{5.034226in}{3.715887in}}{\pgfqpoint{5.028768in}{3.716090in}}%
\pgfusepath{stroke}%
\end{pgfscope}%
\begin{pgfscope}%
\pgfpathrectangle{\pgfqpoint{3.352233in}{3.252941in}}{\pgfqpoint{2.407767in}{1.544118in}}%
\pgfusepath{clip}%
\pgfsetroundcap%
\pgfsetroundjoin%
\definecolor{currentfill}{rgb}{0.281446,0.084320,0.407414}%
\pgfsetfillcolor{currentfill}%
\pgfsetlinewidth{0.501875pt}%
\definecolor{currentstroke}{rgb}{0.281446,0.084320,0.407414}%
\pgfsetstrokecolor{currentstroke}%
\pgfsetdash{}{0pt}%
\pgfpathmoveto{\pgfqpoint{5.056012in}{3.701180in}}%
\pgfpathlineto{\pgfqpoint{5.028768in}{3.716090in}}%
\pgfpathlineto{\pgfqpoint{5.057042in}{3.728939in}}%
\pgfpathlineto{\pgfqpoint{5.056012in}{3.701180in}}%
\pgfpathlineto{\pgfqpoint{5.056012in}{3.701180in}}%
\pgfpathclose%
\pgfusepath{stroke,fill}%
\end{pgfscope}%
\begin{pgfscope}%
\pgfpathrectangle{\pgfqpoint{3.352233in}{3.252941in}}{\pgfqpoint{2.407767in}{1.544118in}}%
\pgfusepath{clip}%
\pgfsetroundcap%
\pgfsetroundjoin%
\pgfsetlinewidth{0.501875pt}%
\definecolor{currentstroke}{rgb}{0.283072,0.130895,0.449241}%
\pgfsetstrokecolor{currentstroke}%
\pgfsetdash{}{0pt}%
\pgfpathmoveto{\pgfqpoint{4.994608in}{3.751515in}}%
\pgfpathquadraticcurveto{\pgfqpoint{4.981417in}{3.752266in}}{\pgfqpoint{4.975977in}{3.752575in}}%
\pgfusepath{stroke}%
\end{pgfscope}%
\begin{pgfscope}%
\pgfpathrectangle{\pgfqpoint{3.352233in}{3.252941in}}{\pgfqpoint{2.407767in}{1.544118in}}%
\pgfusepath{clip}%
\pgfsetroundcap%
\pgfsetroundjoin%
\definecolor{currentfill}{rgb}{0.283072,0.130895,0.449241}%
\pgfsetfillcolor{currentfill}%
\pgfsetlinewidth{0.501875pt}%
\definecolor{currentstroke}{rgb}{0.283072,0.130895,0.449241}%
\pgfsetstrokecolor{currentstroke}%
\pgfsetdash{}{0pt}%
\pgfpathmoveto{\pgfqpoint{5.002920in}{3.737130in}}%
\pgfpathlineto{\pgfqpoint{4.975977in}{3.752575in}}%
\pgfpathlineto{\pgfqpoint{5.004499in}{3.764863in}}%
\pgfpathlineto{\pgfqpoint{5.002920in}{3.737130in}}%
\pgfpathlineto{\pgfqpoint{5.002920in}{3.737130in}}%
\pgfpathclose%
\pgfusepath{stroke,fill}%
\end{pgfscope}%
\begin{pgfscope}%
\pgfpathrectangle{\pgfqpoint{3.352233in}{3.252941in}}{\pgfqpoint{2.407767in}{1.544118in}}%
\pgfusepath{clip}%
\pgfsetroundcap%
\pgfsetroundjoin%
\pgfsetlinewidth{0.501875pt}%
\definecolor{currentstroke}{rgb}{0.283187,0.125848,0.444960}%
\pgfsetstrokecolor{currentstroke}%
\pgfsetdash{}{0pt}%
\pgfpathmoveto{\pgfqpoint{5.047448in}{3.784752in}}%
\pgfpathquadraticcurveto{\pgfqpoint{5.034230in}{3.785262in}}{\pgfqpoint{5.028769in}{3.785474in}}%
\pgfusepath{stroke}%
\end{pgfscope}%
\begin{pgfscope}%
\pgfpathrectangle{\pgfqpoint{3.352233in}{3.252941in}}{\pgfqpoint{2.407767in}{1.544118in}}%
\pgfusepath{clip}%
\pgfsetroundcap%
\pgfsetroundjoin%
\definecolor{currentfill}{rgb}{0.283187,0.125848,0.444960}%
\pgfsetfillcolor{currentfill}%
\pgfsetlinewidth{0.501875pt}%
\definecolor{currentstroke}{rgb}{0.283187,0.125848,0.444960}%
\pgfsetstrokecolor{currentstroke}%
\pgfsetdash{}{0pt}%
\pgfpathmoveto{\pgfqpoint{5.055989in}{3.770522in}}%
\pgfpathlineto{\pgfqpoint{5.028769in}{3.785474in}}%
\pgfpathlineto{\pgfqpoint{5.057062in}{3.798279in}}%
\pgfpathlineto{\pgfqpoint{5.055989in}{3.770522in}}%
\pgfpathlineto{\pgfqpoint{5.055989in}{3.770522in}}%
\pgfpathclose%
\pgfusepath{stroke,fill}%
\end{pgfscope}%
\begin{pgfscope}%
\pgfpathrectangle{\pgfqpoint{3.352233in}{3.252941in}}{\pgfqpoint{2.407767in}{1.544118in}}%
\pgfusepath{clip}%
\pgfsetroundcap%
\pgfsetroundjoin%
\pgfsetlinewidth{0.501875pt}%
\definecolor{currentstroke}{rgb}{0.210503,0.363727,0.552206}%
\pgfsetstrokecolor{currentstroke}%
\pgfsetdash{}{0pt}%
\pgfpathmoveto{\pgfqpoint{4.732094in}{3.842258in}}%
\pgfpathquadraticcurveto{\pgfqpoint{4.719203in}{3.844196in}}{\pgfqpoint{4.713990in}{3.844979in}}%
\pgfusepath{stroke}%
\end{pgfscope}%
\begin{pgfscope}%
\pgfpathrectangle{\pgfqpoint{3.352233in}{3.252941in}}{\pgfqpoint{2.407767in}{1.544118in}}%
\pgfusepath{clip}%
\pgfsetroundcap%
\pgfsetroundjoin%
\definecolor{currentfill}{rgb}{0.210503,0.363727,0.552206}%
\pgfsetfillcolor{currentfill}%
\pgfsetlinewidth{0.501875pt}%
\definecolor{currentstroke}{rgb}{0.210503,0.363727,0.552206}%
\pgfsetstrokecolor{currentstroke}%
\pgfsetdash{}{0pt}%
\pgfpathmoveto{\pgfqpoint{4.739395in}{3.827116in}}%
\pgfpathlineto{\pgfqpoint{4.713990in}{3.844979in}}%
\pgfpathlineto{\pgfqpoint{4.743523in}{3.854585in}}%
\pgfpathlineto{\pgfqpoint{4.739395in}{3.827116in}}%
\pgfpathlineto{\pgfqpoint{4.739395in}{3.827116in}}%
\pgfpathclose%
\pgfusepath{stroke,fill}%
\end{pgfscope}%
\begin{pgfscope}%
\pgfpathrectangle{\pgfqpoint{3.352233in}{3.252941in}}{\pgfqpoint{2.407767in}{1.544118in}}%
\pgfusepath{clip}%
\pgfsetroundcap%
\pgfsetroundjoin%
\pgfsetlinewidth{0.501875pt}%
\definecolor{currentstroke}{rgb}{0.280868,0.160771,0.472899}%
\pgfsetstrokecolor{currentstroke}%
\pgfsetdash{}{0pt}%
\pgfpathmoveto{\pgfqpoint{5.047384in}{3.853067in}}%
\pgfpathquadraticcurveto{\pgfqpoint{5.034151in}{3.853406in}}{\pgfqpoint{5.028680in}{3.853546in}}%
\pgfusepath{stroke}%
\end{pgfscope}%
\begin{pgfscope}%
\pgfpathrectangle{\pgfqpoint{3.352233in}{3.252941in}}{\pgfqpoint{2.407767in}{1.544118in}}%
\pgfusepath{clip}%
\pgfsetroundcap%
\pgfsetroundjoin%
\definecolor{currentfill}{rgb}{0.280868,0.160771,0.472899}%
\pgfsetfillcolor{currentfill}%
\pgfsetlinewidth{0.501875pt}%
\definecolor{currentstroke}{rgb}{0.280868,0.160771,0.472899}%
\pgfsetstrokecolor{currentstroke}%
\pgfsetdash{}{0pt}%
\pgfpathmoveto{\pgfqpoint{5.056093in}{3.838951in}}%
\pgfpathlineto{\pgfqpoint{5.028680in}{3.853546in}}%
\pgfpathlineto{\pgfqpoint{5.056804in}{3.866720in}}%
\pgfpathlineto{\pgfqpoint{5.056093in}{3.838951in}}%
\pgfpathlineto{\pgfqpoint{5.056093in}{3.838951in}}%
\pgfpathclose%
\pgfusepath{stroke,fill}%
\end{pgfscope}%
\begin{pgfscope}%
\pgfpathrectangle{\pgfqpoint{3.352233in}{3.252941in}}{\pgfqpoint{2.407767in}{1.544118in}}%
\pgfusepath{clip}%
\pgfsetroundcap%
\pgfsetroundjoin%
\pgfsetlinewidth{0.501875pt}%
\definecolor{currentstroke}{rgb}{0.150476,0.504369,0.557430}%
\pgfsetstrokecolor{currentstroke}%
\pgfsetdash{}{0pt}%
\pgfpathmoveto{\pgfqpoint{4.783264in}{3.899224in}}%
\pgfpathquadraticcurveto{\pgfqpoint{4.770148in}{3.900395in}}{\pgfqpoint{4.764766in}{3.900876in}}%
\pgfusepath{stroke}%
\end{pgfscope}%
\begin{pgfscope}%
\pgfpathrectangle{\pgfqpoint{3.352233in}{3.252941in}}{\pgfqpoint{2.407767in}{1.544118in}}%
\pgfusepath{clip}%
\pgfsetroundcap%
\pgfsetroundjoin%
\definecolor{currentfill}{rgb}{0.150476,0.504369,0.557430}%
\pgfsetfillcolor{currentfill}%
\pgfsetlinewidth{0.501875pt}%
\definecolor{currentstroke}{rgb}{0.150476,0.504369,0.557430}%
\pgfsetstrokecolor{currentstroke}%
\pgfsetdash{}{0pt}%
\pgfpathmoveto{\pgfqpoint{4.791198in}{3.884571in}}%
\pgfpathlineto{\pgfqpoint{4.764766in}{3.900876in}}%
\pgfpathlineto{\pgfqpoint{4.793669in}{3.912239in}}%
\pgfpathlineto{\pgfqpoint{4.791198in}{3.884571in}}%
\pgfpathlineto{\pgfqpoint{4.791198in}{3.884571in}}%
\pgfpathclose%
\pgfusepath{stroke,fill}%
\end{pgfscope}%
\begin{pgfscope}%
\pgfpathrectangle{\pgfqpoint{3.352233in}{3.252941in}}{\pgfqpoint{2.407767in}{1.544118in}}%
\pgfusepath{clip}%
\pgfsetroundcap%
\pgfsetroundjoin%
\pgfsetlinewidth{0.501875pt}%
\definecolor{currentstroke}{rgb}{0.250425,0.274290,0.533103}%
\pgfsetstrokecolor{currentstroke}%
\pgfsetdash{}{0pt}%
\pgfpathmoveto{\pgfqpoint{4.994406in}{3.922738in}}%
\pgfpathquadraticcurveto{\pgfqpoint{4.981172in}{3.923058in}}{\pgfqpoint{4.975700in}{3.923190in}}%
\pgfusepath{stroke}%
\end{pgfscope}%
\begin{pgfscope}%
\pgfpathrectangle{\pgfqpoint{3.352233in}{3.252941in}}{\pgfqpoint{2.407767in}{1.544118in}}%
\pgfusepath{clip}%
\pgfsetroundcap%
\pgfsetroundjoin%
\definecolor{currentfill}{rgb}{0.250425,0.274290,0.533103}%
\pgfsetfillcolor{currentfill}%
\pgfsetlinewidth{0.501875pt}%
\definecolor{currentstroke}{rgb}{0.250425,0.274290,0.533103}%
\pgfsetstrokecolor{currentstroke}%
\pgfsetdash{}{0pt}%
\pgfpathmoveto{\pgfqpoint{5.003134in}{3.908634in}}%
\pgfpathlineto{\pgfqpoint{4.975700in}{3.923190in}}%
\pgfpathlineto{\pgfqpoint{5.003805in}{3.936404in}}%
\pgfpathlineto{\pgfqpoint{5.003134in}{3.908634in}}%
\pgfpathlineto{\pgfqpoint{5.003134in}{3.908634in}}%
\pgfpathclose%
\pgfusepath{stroke,fill}%
\end{pgfscope}%
\begin{pgfscope}%
\pgfpathrectangle{\pgfqpoint{3.352233in}{3.252941in}}{\pgfqpoint{2.407767in}{1.544118in}}%
\pgfusepath{clip}%
\pgfsetroundcap%
\pgfsetroundjoin%
\pgfsetlinewidth{0.501875pt}%
\definecolor{currentstroke}{rgb}{0.128087,0.647749,0.523491}%
\pgfsetstrokecolor{currentstroke}%
\pgfsetdash{}{0pt}%
\pgfpathmoveto{\pgfqpoint{4.782687in}{3.962540in}}%
\pgfpathquadraticcurveto{\pgfqpoint{4.769478in}{3.963156in}}{\pgfqpoint{4.764026in}{3.963411in}}%
\pgfusepath{stroke}%
\end{pgfscope}%
\begin{pgfscope}%
\pgfpathrectangle{\pgfqpoint{3.352233in}{3.252941in}}{\pgfqpoint{2.407767in}{1.544118in}}%
\pgfusepath{clip}%
\pgfsetroundcap%
\pgfsetroundjoin%
\definecolor{currentfill}{rgb}{0.128087,0.647749,0.523491}%
\pgfsetfillcolor{currentfill}%
\pgfsetlinewidth{0.501875pt}%
\definecolor{currentstroke}{rgb}{0.128087,0.647749,0.523491}%
\pgfsetstrokecolor{currentstroke}%
\pgfsetdash{}{0pt}%
\pgfpathmoveto{\pgfqpoint{4.791126in}{3.948242in}}%
\pgfpathlineto{\pgfqpoint{4.764026in}{3.963411in}}%
\pgfpathlineto{\pgfqpoint{4.792421in}{3.975990in}}%
\pgfpathlineto{\pgfqpoint{4.791126in}{3.948242in}}%
\pgfpathlineto{\pgfqpoint{4.791126in}{3.948242in}}%
\pgfpathclose%
\pgfusepath{stroke,fill}%
\end{pgfscope}%
\begin{pgfscope}%
\pgfpathrectangle{\pgfqpoint{3.352233in}{3.252941in}}{\pgfqpoint{2.407767in}{1.544118in}}%
\pgfusepath{clip}%
\pgfsetroundcap%
\pgfsetroundjoin%
\pgfsetlinewidth{0.501875pt}%
\definecolor{currentstroke}{rgb}{0.169646,0.456262,0.558030}%
\pgfsetstrokecolor{currentstroke}%
\pgfsetdash{}{0pt}%
\pgfpathmoveto{\pgfqpoint{4.941407in}{3.991316in}}%
\pgfpathquadraticcurveto{\pgfqpoint{4.928166in}{3.991469in}}{\pgfqpoint{4.922687in}{3.991532in}}%
\pgfusepath{stroke}%
\end{pgfscope}%
\begin{pgfscope}%
\pgfpathrectangle{\pgfqpoint{3.352233in}{3.252941in}}{\pgfqpoint{2.407767in}{1.544118in}}%
\pgfusepath{clip}%
\pgfsetroundcap%
\pgfsetroundjoin%
\definecolor{currentfill}{rgb}{0.169646,0.456262,0.558030}%
\pgfsetfillcolor{currentfill}%
\pgfsetlinewidth{0.501875pt}%
\definecolor{currentstroke}{rgb}{0.169646,0.456262,0.558030}%
\pgfsetstrokecolor{currentstroke}%
\pgfsetdash{}{0pt}%
\pgfpathmoveto{\pgfqpoint{4.950303in}{3.977324in}}%
\pgfpathlineto{\pgfqpoint{4.922687in}{3.991532in}}%
\pgfpathlineto{\pgfqpoint{4.950623in}{4.005100in}}%
\pgfpathlineto{\pgfqpoint{4.950303in}{3.977324in}}%
\pgfpathlineto{\pgfqpoint{4.950303in}{3.977324in}}%
\pgfpathclose%
\pgfusepath{stroke,fill}%
\end{pgfscope}%
\begin{pgfscope}%
\pgfpathrectangle{\pgfqpoint{3.352233in}{3.252941in}}{\pgfqpoint{2.407767in}{1.544118in}}%
\pgfusepath{clip}%
\pgfsetroundcap%
\pgfsetroundjoin%
\pgfsetlinewidth{0.501875pt}%
\definecolor{currentstroke}{rgb}{0.232815,0.732247,0.459277}%
\pgfsetstrokecolor{currentstroke}%
\pgfsetdash{}{0pt}%
\pgfpathmoveto{\pgfqpoint{4.782472in}{4.025414in}}%
\pgfpathquadraticcurveto{\pgfqpoint{4.769228in}{4.025425in}}{\pgfqpoint{4.763749in}{4.025430in}}%
\pgfusepath{stroke}%
\end{pgfscope}%
\begin{pgfscope}%
\pgfpathrectangle{\pgfqpoint{3.352233in}{3.252941in}}{\pgfqpoint{2.407767in}{1.544118in}}%
\pgfusepath{clip}%
\pgfsetroundcap%
\pgfsetroundjoin%
\definecolor{currentfill}{rgb}{0.232815,0.732247,0.459277}%
\pgfsetfillcolor{currentfill}%
\pgfsetlinewidth{0.501875pt}%
\definecolor{currentstroke}{rgb}{0.232815,0.732247,0.459277}%
\pgfsetstrokecolor{currentstroke}%
\pgfsetdash{}{0pt}%
\pgfpathmoveto{\pgfqpoint{4.791514in}{4.011517in}}%
\pgfpathlineto{\pgfqpoint{4.763749in}{4.025430in}}%
\pgfpathlineto{\pgfqpoint{4.791539in}{4.039295in}}%
\pgfpathlineto{\pgfqpoint{4.791514in}{4.011517in}}%
\pgfpathlineto{\pgfqpoint{4.791514in}{4.011517in}}%
\pgfpathclose%
\pgfusepath{stroke,fill}%
\end{pgfscope}%
\begin{pgfscope}%
\pgfpathrectangle{\pgfqpoint{3.352233in}{3.252941in}}{\pgfqpoint{2.407767in}{1.544118in}}%
\pgfusepath{clip}%
\pgfsetroundcap%
\pgfsetroundjoin%
\pgfsetlinewidth{0.501875pt}%
\definecolor{currentstroke}{rgb}{0.153894,0.680203,0.504172}%
\pgfsetstrokecolor{currentstroke}%
\pgfsetdash{}{0pt}%
\pgfpathmoveto{\pgfqpoint{4.782541in}{4.055891in}}%
\pgfpathquadraticcurveto{\pgfqpoint{4.769305in}{4.055590in}}{\pgfqpoint{4.763832in}{4.055466in}}%
\pgfusepath{stroke}%
\end{pgfscope}%
\begin{pgfscope}%
\pgfpathrectangle{\pgfqpoint{3.352233in}{3.252941in}}{\pgfqpoint{2.407767in}{1.544118in}}%
\pgfusepath{clip}%
\pgfsetroundcap%
\pgfsetroundjoin%
\definecolor{currentfill}{rgb}{0.153894,0.680203,0.504172}%
\pgfsetfillcolor{currentfill}%
\pgfsetlinewidth{0.501875pt}%
\definecolor{currentstroke}{rgb}{0.153894,0.680203,0.504172}%
\pgfsetstrokecolor{currentstroke}%
\pgfsetdash{}{0pt}%
\pgfpathmoveto{\pgfqpoint{4.791917in}{4.042211in}}%
\pgfpathlineto{\pgfqpoint{4.763832in}{4.055466in}}%
\pgfpathlineto{\pgfqpoint{4.791288in}{4.069981in}}%
\pgfpathlineto{\pgfqpoint{4.791917in}{4.042211in}}%
\pgfpathlineto{\pgfqpoint{4.791917in}{4.042211in}}%
\pgfpathclose%
\pgfusepath{stroke,fill}%
\end{pgfscope}%
\begin{pgfscope}%
\pgfpathrectangle{\pgfqpoint{3.352233in}{3.252941in}}{\pgfqpoint{2.407767in}{1.544118in}}%
\pgfusepath{clip}%
\pgfsetroundcap%
\pgfsetroundjoin%
\pgfsetlinewidth{0.501875pt}%
\definecolor{currentstroke}{rgb}{0.188923,0.410910,0.556326}%
\pgfsetstrokecolor{currentstroke}%
\pgfsetdash{}{0pt}%
\pgfpathmoveto{\pgfqpoint{4.941444in}{4.091739in}}%
\pgfpathquadraticcurveto{\pgfqpoint{4.928211in}{4.091411in}}{\pgfqpoint{4.922739in}{4.091275in}}%
\pgfusepath{stroke}%
\end{pgfscope}%
\begin{pgfscope}%
\pgfpathrectangle{\pgfqpoint{3.352233in}{3.252941in}}{\pgfqpoint{2.407767in}{1.544118in}}%
\pgfusepath{clip}%
\pgfsetroundcap%
\pgfsetroundjoin%
\definecolor{currentfill}{rgb}{0.188923,0.410910,0.556326}%
\pgfsetfillcolor{currentfill}%
\pgfsetlinewidth{0.501875pt}%
\definecolor{currentstroke}{rgb}{0.188923,0.410910,0.556326}%
\pgfsetstrokecolor{currentstroke}%
\pgfsetdash{}{0pt}%
\pgfpathmoveto{\pgfqpoint{4.950852in}{4.078079in}}%
\pgfpathlineto{\pgfqpoint{4.922739in}{4.091275in}}%
\pgfpathlineto{\pgfqpoint{4.950164in}{4.105848in}}%
\pgfpathlineto{\pgfqpoint{4.950852in}{4.078079in}}%
\pgfpathlineto{\pgfqpoint{4.950852in}{4.078079in}}%
\pgfpathclose%
\pgfusepath{stroke,fill}%
\end{pgfscope}%
\begin{pgfscope}%
\pgfpathrectangle{\pgfqpoint{3.352233in}{3.252941in}}{\pgfqpoint{2.407767in}{1.544118in}}%
\pgfusepath{clip}%
\pgfsetroundcap%
\pgfsetroundjoin%
\pgfsetlinewidth{0.501875pt}%
\definecolor{currentstroke}{rgb}{0.248629,0.278775,0.534556}%
\pgfsetstrokecolor{currentstroke}%
\pgfsetdash{}{0pt}%
\pgfpathmoveto{\pgfqpoint{4.994415in}{4.127007in}}%
\pgfpathquadraticcurveto{\pgfqpoint{4.981181in}{4.126678in}}{\pgfqpoint{4.975709in}{4.126542in}}%
\pgfusepath{stroke}%
\end{pgfscope}%
\begin{pgfscope}%
\pgfpathrectangle{\pgfqpoint{3.352233in}{3.252941in}}{\pgfqpoint{2.407767in}{1.544118in}}%
\pgfusepath{clip}%
\pgfsetroundcap%
\pgfsetroundjoin%
\definecolor{currentfill}{rgb}{0.248629,0.278775,0.534556}%
\pgfsetfillcolor{currentfill}%
\pgfsetlinewidth{0.501875pt}%
\definecolor{currentstroke}{rgb}{0.248629,0.278775,0.534556}%
\pgfsetstrokecolor{currentstroke}%
\pgfsetdash{}{0pt}%
\pgfpathmoveto{\pgfqpoint{5.003824in}{4.113348in}}%
\pgfpathlineto{\pgfqpoint{4.975709in}{4.126542in}}%
\pgfpathlineto{\pgfqpoint{5.003133in}{4.141117in}}%
\pgfpathlineto{\pgfqpoint{5.003824in}{4.113348in}}%
\pgfpathlineto{\pgfqpoint{5.003824in}{4.113348in}}%
\pgfpathclose%
\pgfusepath{stroke,fill}%
\end{pgfscope}%
\begin{pgfscope}%
\pgfpathrectangle{\pgfqpoint{3.352233in}{3.252941in}}{\pgfqpoint{2.407767in}{1.544118in}}%
\pgfusepath{clip}%
\pgfsetroundcap%
\pgfsetroundjoin%
\pgfsetlinewidth{0.501875pt}%
\definecolor{currentstroke}{rgb}{0.265145,0.232956,0.516599}%
\pgfsetstrokecolor{currentstroke}%
\pgfsetdash{}{0pt}%
\pgfpathmoveto{\pgfqpoint{4.994445in}{4.161070in}}%
\pgfpathquadraticcurveto{\pgfqpoint{4.981222in}{4.160606in}}{\pgfqpoint{4.975758in}{4.160415in}}%
\pgfusepath{stroke}%
\end{pgfscope}%
\begin{pgfscope}%
\pgfpathrectangle{\pgfqpoint{3.352233in}{3.252941in}}{\pgfqpoint{2.407767in}{1.544118in}}%
\pgfusepath{clip}%
\pgfsetroundcap%
\pgfsetroundjoin%
\definecolor{currentfill}{rgb}{0.265145,0.232956,0.516599}%
\pgfsetfillcolor{currentfill}%
\pgfsetlinewidth{0.501875pt}%
\definecolor{currentstroke}{rgb}{0.265145,0.232956,0.516599}%
\pgfsetstrokecolor{currentstroke}%
\pgfsetdash{}{0pt}%
\pgfpathmoveto{\pgfqpoint{5.004005in}{4.147508in}}%
\pgfpathlineto{\pgfqpoint{4.975758in}{4.160415in}}%
\pgfpathlineto{\pgfqpoint{5.003032in}{4.175269in}}%
\pgfpathlineto{\pgfqpoint{5.004005in}{4.147508in}}%
\pgfpathlineto{\pgfqpoint{5.004005in}{4.147508in}}%
\pgfpathclose%
\pgfusepath{stroke,fill}%
\end{pgfscope}%
\begin{pgfscope}%
\pgfpathrectangle{\pgfqpoint{3.352233in}{3.252941in}}{\pgfqpoint{2.407767in}{1.544118in}}%
\pgfusepath{clip}%
\pgfsetroundcap%
\pgfsetroundjoin%
\pgfsetlinewidth{0.501875pt}%
\definecolor{currentstroke}{rgb}{0.157729,0.485932,0.558013}%
\pgfsetstrokecolor{currentstroke}%
\pgfsetdash{}{0pt}%
\pgfpathmoveto{\pgfqpoint{4.731311in}{4.177816in}}%
\pgfpathquadraticcurveto{\pgfqpoint{4.718360in}{4.176049in}}{\pgfqpoint{4.713102in}{4.175332in}}%
\pgfusepath{stroke}%
\end{pgfscope}%
\begin{pgfscope}%
\pgfpathrectangle{\pgfqpoint{3.352233in}{3.252941in}}{\pgfqpoint{2.407767in}{1.544118in}}%
\pgfusepath{clip}%
\pgfsetroundcap%
\pgfsetroundjoin%
\definecolor{currentfill}{rgb}{0.157729,0.485932,0.558013}%
\pgfsetfillcolor{currentfill}%
\pgfsetlinewidth{0.501875pt}%
\definecolor{currentstroke}{rgb}{0.157729,0.485932,0.558013}%
\pgfsetstrokecolor{currentstroke}%
\pgfsetdash{}{0pt}%
\pgfpathmoveto{\pgfqpoint{4.742502in}{4.165325in}}%
\pgfpathlineto{\pgfqpoint{4.713102in}{4.175332in}}%
\pgfpathlineto{\pgfqpoint{4.738747in}{4.192848in}}%
\pgfpathlineto{\pgfqpoint{4.742502in}{4.165325in}}%
\pgfpathlineto{\pgfqpoint{4.742502in}{4.165325in}}%
\pgfpathclose%
\pgfusepath{stroke,fill}%
\end{pgfscope}%
\begin{pgfscope}%
\pgfpathrectangle{\pgfqpoint{3.352233in}{3.252941in}}{\pgfqpoint{2.407767in}{1.544118in}}%
\pgfusepath{clip}%
\pgfsetroundcap%
\pgfsetroundjoin%
\pgfsetlinewidth{0.501875pt}%
\definecolor{currentstroke}{rgb}{0.282623,0.140926,0.457517}%
\pgfsetstrokecolor{currentstroke}%
\pgfsetdash{}{0pt}%
\pgfpathmoveto{\pgfqpoint{5.047403in}{4.231178in}}%
\pgfpathquadraticcurveto{\pgfqpoint{5.034174in}{4.230780in}}{\pgfqpoint{5.028706in}{4.230615in}}%
\pgfusepath{stroke}%
\end{pgfscope}%
\begin{pgfscope}%
\pgfpathrectangle{\pgfqpoint{3.352233in}{3.252941in}}{\pgfqpoint{2.407767in}{1.544118in}}%
\pgfusepath{clip}%
\pgfsetroundcap%
\pgfsetroundjoin%
\definecolor{currentfill}{rgb}{0.282623,0.140926,0.457517}%
\pgfsetfillcolor{currentfill}%
\pgfsetlinewidth{0.501875pt}%
\definecolor{currentstroke}{rgb}{0.282623,0.140926,0.457517}%
\pgfsetstrokecolor{currentstroke}%
\pgfsetdash{}{0pt}%
\pgfpathmoveto{\pgfqpoint{5.056888in}{4.217568in}}%
\pgfpathlineto{\pgfqpoint{5.028706in}{4.230615in}}%
\pgfpathlineto{\pgfqpoint{5.056053in}{4.245333in}}%
\pgfpathlineto{\pgfqpoint{5.056888in}{4.217568in}}%
\pgfpathlineto{\pgfqpoint{5.056888in}{4.217568in}}%
\pgfpathclose%
\pgfusepath{stroke,fill}%
\end{pgfscope}%
\begin{pgfscope}%
\pgfpathrectangle{\pgfqpoint{3.352233in}{3.252941in}}{\pgfqpoint{2.407767in}{1.544118in}}%
\pgfusepath{clip}%
\pgfsetroundcap%
\pgfsetroundjoin%
\pgfsetlinewidth{0.501875pt}%
\definecolor{currentstroke}{rgb}{0.237441,0.305202,0.541921}%
\pgfsetstrokecolor{currentstroke}%
\pgfsetdash{}{0pt}%
\pgfpathmoveto{\pgfqpoint{4.681330in}{4.232648in}}%
\pgfpathquadraticcurveto{\pgfqpoint{4.668759in}{4.229982in}}{\pgfqpoint{4.663784in}{4.228928in}}%
\pgfusepath{stroke}%
\end{pgfscope}%
\begin{pgfscope}%
\pgfpathrectangle{\pgfqpoint{3.352233in}{3.252941in}}{\pgfqpoint{2.407767in}{1.544118in}}%
\pgfusepath{clip}%
\pgfsetroundcap%
\pgfsetroundjoin%
\definecolor{currentfill}{rgb}{0.237441,0.305202,0.541921}%
\pgfsetfillcolor{currentfill}%
\pgfsetlinewidth{0.501875pt}%
\definecolor{currentstroke}{rgb}{0.237441,0.305202,0.541921}%
\pgfsetstrokecolor{currentstroke}%
\pgfsetdash{}{0pt}%
\pgfpathmoveto{\pgfqpoint{4.693838in}{4.221102in}}%
\pgfpathlineto{\pgfqpoint{4.663784in}{4.228928in}}%
\pgfpathlineto{\pgfqpoint{4.688077in}{4.248276in}}%
\pgfpathlineto{\pgfqpoint{4.693838in}{4.221102in}}%
\pgfpathlineto{\pgfqpoint{4.693838in}{4.221102in}}%
\pgfpathclose%
\pgfusepath{stroke,fill}%
\end{pgfscope}%
\begin{pgfscope}%
\pgfpathrectangle{\pgfqpoint{3.352233in}{3.252941in}}{\pgfqpoint{2.407767in}{1.544118in}}%
\pgfusepath{clip}%
\pgfsetroundcap%
\pgfsetroundjoin%
\pgfsetlinewidth{0.501875pt}%
\definecolor{currentstroke}{rgb}{0.283187,0.125848,0.444960}%
\pgfsetstrokecolor{currentstroke}%
\pgfsetdash{}{0pt}%
\pgfpathmoveto{\pgfqpoint{4.994618in}{4.296902in}}%
\pgfpathquadraticcurveto{\pgfqpoint{4.981415in}{4.296230in}}{\pgfqpoint{4.975967in}{4.295953in}}%
\pgfusepath{stroke}%
\end{pgfscope}%
\begin{pgfscope}%
\pgfpathrectangle{\pgfqpoint{3.352233in}{3.252941in}}{\pgfqpoint{2.407767in}{1.544118in}}%
\pgfusepath{clip}%
\pgfsetroundcap%
\pgfsetroundjoin%
\definecolor{currentfill}{rgb}{0.283187,0.125848,0.444960}%
\pgfsetfillcolor{currentfill}%
\pgfsetlinewidth{0.501875pt}%
\definecolor{currentstroke}{rgb}{0.283187,0.125848,0.444960}%
\pgfsetstrokecolor{currentstroke}%
\pgfsetdash{}{0pt}%
\pgfpathmoveto{\pgfqpoint{5.004414in}{4.283493in}}%
\pgfpathlineto{\pgfqpoint{4.975967in}{4.295953in}}%
\pgfpathlineto{\pgfqpoint{5.003003in}{4.311235in}}%
\pgfpathlineto{\pgfqpoint{5.004414in}{4.283493in}}%
\pgfpathlineto{\pgfqpoint{5.004414in}{4.283493in}}%
\pgfpathclose%
\pgfusepath{stroke,fill}%
\end{pgfscope}%
\begin{pgfscope}%
\pgfpathrectangle{\pgfqpoint{3.352233in}{3.252941in}}{\pgfqpoint{2.407767in}{1.544118in}}%
\pgfusepath{clip}%
\pgfsetroundcap%
\pgfsetroundjoin%
\pgfsetlinewidth{0.501875pt}%
\definecolor{currentstroke}{rgb}{0.280267,0.073417,0.397163}%
\pgfsetstrokecolor{currentstroke}%
\pgfsetdash{}{0pt}%
\pgfpathmoveto{\pgfqpoint{4.994587in}{4.332495in}}%
\pgfpathquadraticcurveto{\pgfqpoint{4.981378in}{4.331876in}}{\pgfqpoint{4.975926in}{4.331621in}}%
\pgfusepath{stroke}%
\end{pgfscope}%
\begin{pgfscope}%
\pgfpathrectangle{\pgfqpoint{3.352233in}{3.252941in}}{\pgfqpoint{2.407767in}{1.544118in}}%
\pgfusepath{clip}%
\pgfsetroundcap%
\pgfsetroundjoin%
\definecolor{currentfill}{rgb}{0.280267,0.073417,0.397163}%
\pgfsetfillcolor{currentfill}%
\pgfsetlinewidth{0.501875pt}%
\definecolor{currentstroke}{rgb}{0.280267,0.073417,0.397163}%
\pgfsetstrokecolor{currentstroke}%
\pgfsetdash{}{0pt}%
\pgfpathmoveto{\pgfqpoint{5.004323in}{4.319046in}}%
\pgfpathlineto{\pgfqpoint{4.975926in}{4.331621in}}%
\pgfpathlineto{\pgfqpoint{5.003024in}{4.346794in}}%
\pgfpathlineto{\pgfqpoint{5.004323in}{4.319046in}}%
\pgfpathlineto{\pgfqpoint{5.004323in}{4.319046in}}%
\pgfpathclose%
\pgfusepath{stroke,fill}%
\end{pgfscope}%
\begin{pgfscope}%
\pgfpathrectangle{\pgfqpoint{3.352233in}{3.252941in}}{\pgfqpoint{2.407767in}{1.544118in}}%
\pgfusepath{clip}%
\pgfsetroundcap%
\pgfsetroundjoin%
\pgfsetlinewidth{0.501875pt}%
\definecolor{currentstroke}{rgb}{0.276022,0.044167,0.370164}%
\pgfsetstrokecolor{currentstroke}%
\pgfsetdash{}{0pt}%
\pgfpathmoveto{\pgfqpoint{5.047522in}{4.438258in}}%
\pgfpathquadraticcurveto{\pgfqpoint{5.034343in}{4.437477in}}{\pgfqpoint{5.028914in}{4.437156in}}%
\pgfusepath{stroke}%
\end{pgfscope}%
\begin{pgfscope}%
\pgfpathrectangle{\pgfqpoint{3.352233in}{3.252941in}}{\pgfqpoint{2.407767in}{1.544118in}}%
\pgfusepath{clip}%
\pgfsetroundcap%
\pgfsetroundjoin%
\definecolor{currentfill}{rgb}{0.276022,0.044167,0.370164}%
\pgfsetfillcolor{currentfill}%
\pgfsetlinewidth{0.501875pt}%
\definecolor{currentstroke}{rgb}{0.276022,0.044167,0.370164}%
\pgfsetstrokecolor{currentstroke}%
\pgfsetdash{}{0pt}%
\pgfpathmoveto{\pgfqpoint{5.057465in}{4.424934in}}%
\pgfpathlineto{\pgfqpoint{5.028914in}{4.437156in}}%
\pgfpathlineto{\pgfqpoint{5.055822in}{4.452663in}}%
\pgfpathlineto{\pgfqpoint{5.057465in}{4.424934in}}%
\pgfpathlineto{\pgfqpoint{5.057465in}{4.424934in}}%
\pgfpathclose%
\pgfusepath{stroke,fill}%
\end{pgfscope}%
\begin{pgfscope}%
\pgfpathrectangle{\pgfqpoint{3.352233in}{3.252941in}}{\pgfqpoint{2.407767in}{1.544118in}}%
\pgfusepath{clip}%
\pgfsetroundcap%
\pgfsetroundjoin%
\pgfsetlinewidth{0.501875pt}%
\definecolor{currentstroke}{rgb}{0.277018,0.050344,0.375715}%
\pgfsetstrokecolor{currentstroke}%
\pgfsetdash{}{0pt}%
\pgfpathmoveto{\pgfqpoint{5.040830in}{4.413521in}}%
\pgfpathquadraticcurveto{\pgfqpoint{5.027649in}{4.412730in}}{\pgfqpoint{5.022219in}{4.412404in}}%
\pgfusepath{stroke}%
\end{pgfscope}%
\begin{pgfscope}%
\pgfpathrectangle{\pgfqpoint{3.352233in}{3.252941in}}{\pgfqpoint{2.407767in}{1.544118in}}%
\pgfusepath{clip}%
\pgfsetroundcap%
\pgfsetroundjoin%
\definecolor{currentfill}{rgb}{0.277018,0.050344,0.375715}%
\pgfsetfillcolor{currentfill}%
\pgfsetlinewidth{0.501875pt}%
\definecolor{currentstroke}{rgb}{0.277018,0.050344,0.375715}%
\pgfsetstrokecolor{currentstroke}%
\pgfsetdash{}{0pt}%
\pgfpathmoveto{\pgfqpoint{5.050779in}{4.400205in}}%
\pgfpathlineto{\pgfqpoint{5.022219in}{4.412404in}}%
\pgfpathlineto{\pgfqpoint{5.049114in}{4.427933in}}%
\pgfpathlineto{\pgfqpoint{5.050779in}{4.400205in}}%
\pgfpathlineto{\pgfqpoint{5.050779in}{4.400205in}}%
\pgfpathclose%
\pgfusepath{stroke,fill}%
\end{pgfscope}%
\begin{pgfscope}%
\pgfpathrectangle{\pgfqpoint{3.352233in}{3.252941in}}{\pgfqpoint{2.407767in}{1.544118in}}%
\pgfusepath{clip}%
\pgfsetroundcap%
\pgfsetroundjoin%
\pgfsetlinewidth{0.501875pt}%
\definecolor{currentstroke}{rgb}{0.280894,0.078907,0.402329}%
\pgfsetstrokecolor{currentstroke}%
\pgfsetdash{}{0pt}%
\pgfpathmoveto{\pgfqpoint{4.989558in}{4.372461in}}%
\pgfpathquadraticcurveto{\pgfqpoint{4.976378in}{4.371649in}}{\pgfqpoint{4.970947in}{4.371315in}}%
\pgfusepath{stroke}%
\end{pgfscope}%
\begin{pgfscope}%
\pgfpathrectangle{\pgfqpoint{3.352233in}{3.252941in}}{\pgfqpoint{2.407767in}{1.544118in}}%
\pgfusepath{clip}%
\pgfsetroundcap%
\pgfsetroundjoin%
\definecolor{currentfill}{rgb}{0.280894,0.078907,0.402329}%
\pgfsetfillcolor{currentfill}%
\pgfsetlinewidth{0.501875pt}%
\definecolor{currentstroke}{rgb}{0.280894,0.078907,0.402329}%
\pgfsetstrokecolor{currentstroke}%
\pgfsetdash{}{0pt}%
\pgfpathmoveto{\pgfqpoint{4.999526in}{4.359160in}}%
\pgfpathlineto{\pgfqpoint{4.970947in}{4.371315in}}%
\pgfpathlineto{\pgfqpoint{4.997818in}{4.386885in}}%
\pgfpathlineto{\pgfqpoint{4.999526in}{4.359160in}}%
\pgfpathlineto{\pgfqpoint{4.999526in}{4.359160in}}%
\pgfpathclose%
\pgfusepath{stroke,fill}%
\end{pgfscope}%
\begin{pgfscope}%
\pgfpathrectangle{\pgfqpoint{3.352233in}{3.252941in}}{\pgfqpoint{2.407767in}{1.544118in}}%
\pgfusepath{clip}%
\pgfsetroundcap%
\pgfsetroundjoin%
\pgfsetlinewidth{0.501875pt}%
\definecolor{currentstroke}{rgb}{0.229739,0.322361,0.545706}%
\pgfsetstrokecolor{currentstroke}%
\pgfsetdash{}{0pt}%
\pgfpathmoveto{\pgfqpoint{4.436118in}{4.242598in}}%
\pgfpathquadraticcurveto{\pgfqpoint{4.427396in}{4.236223in}}{\pgfqpoint{4.424941in}{4.234429in}}%
\pgfusepath{stroke}%
\end{pgfscope}%
\begin{pgfscope}%
\pgfpathrectangle{\pgfqpoint{3.352233in}{3.252941in}}{\pgfqpoint{2.407767in}{1.544118in}}%
\pgfusepath{clip}%
\pgfsetroundcap%
\pgfsetroundjoin%
\definecolor{currentfill}{rgb}{0.229739,0.322361,0.545706}%
\pgfsetfillcolor{currentfill}%
\pgfsetlinewidth{0.501875pt}%
\definecolor{currentstroke}{rgb}{0.229739,0.322361,0.545706}%
\pgfsetstrokecolor{currentstroke}%
\pgfsetdash{}{0pt}%
\pgfpathmoveto{\pgfqpoint{4.455563in}{4.239608in}}%
\pgfpathlineto{\pgfqpoint{4.424941in}{4.234429in}}%
\pgfpathlineto{\pgfqpoint{4.439171in}{4.262034in}}%
\pgfpathlineto{\pgfqpoint{4.455563in}{4.239608in}}%
\pgfpathlineto{\pgfqpoint{4.455563in}{4.239608in}}%
\pgfpathclose%
\pgfusepath{stroke,fill}%
\end{pgfscope}%
\begin{pgfscope}%
\pgfpathrectangle{\pgfqpoint{3.352233in}{3.252941in}}{\pgfqpoint{2.407767in}{1.544118in}}%
\pgfusepath{clip}%
\pgfsetroundcap%
\pgfsetroundjoin%
\pgfsetlinewidth{0.501875pt}%
\definecolor{currentstroke}{rgb}{0.221989,0.339161,0.548752}%
\pgfsetstrokecolor{currentstroke}%
\pgfsetdash{}{0pt}%
\pgfpathmoveto{\pgfqpoint{4.424872in}{3.828212in}}%
\pgfpathquadraticcurveto{\pgfqpoint{4.415154in}{3.833961in}}{\pgfqpoint{4.412119in}{3.835756in}}%
\pgfusepath{stroke}%
\end{pgfscope}%
\begin{pgfscope}%
\pgfpathrectangle{\pgfqpoint{3.352233in}{3.252941in}}{\pgfqpoint{2.407767in}{1.544118in}}%
\pgfusepath{clip}%
\pgfsetroundcap%
\pgfsetroundjoin%
\definecolor{currentfill}{rgb}{0.221989,0.339161,0.548752}%
\pgfsetfillcolor{currentfill}%
\pgfsetlinewidth{0.501875pt}%
\definecolor{currentstroke}{rgb}{0.221989,0.339161,0.548752}%
\pgfsetstrokecolor{currentstroke}%
\pgfsetdash{}{0pt}%
\pgfpathmoveto{\pgfqpoint{4.428956in}{3.809660in}}%
\pgfpathlineto{\pgfqpoint{4.412119in}{3.835756in}}%
\pgfpathlineto{\pgfqpoint{4.443098in}{3.833568in}}%
\pgfpathlineto{\pgfqpoint{4.428956in}{3.809660in}}%
\pgfpathlineto{\pgfqpoint{4.428956in}{3.809660in}}%
\pgfpathclose%
\pgfusepath{stroke,fill}%
\end{pgfscope}%
\begin{pgfscope}%
\pgfpathrectangle{\pgfqpoint{3.352233in}{3.252941in}}{\pgfqpoint{2.407767in}{1.544118in}}%
\pgfusepath{clip}%
\pgfsetroundcap%
\pgfsetroundjoin%
\pgfsetlinewidth{0.501875pt}%
\definecolor{currentstroke}{rgb}{0.237441,0.305202,0.541921}%
\pgfsetstrokecolor{currentstroke}%
\pgfsetdash{}{0pt}%
\pgfpathmoveto{\pgfqpoint{4.567806in}{4.240782in}}%
\pgfpathquadraticcurveto{\pgfqpoint{4.556243in}{4.236651in}}{\pgfqpoint{4.551991in}{4.235132in}}%
\pgfusepath{stroke}%
\end{pgfscope}%
\begin{pgfscope}%
\pgfpathrectangle{\pgfqpoint{3.352233in}{3.252941in}}{\pgfqpoint{2.407767in}{1.544118in}}%
\pgfusepath{clip}%
\pgfsetroundcap%
\pgfsetroundjoin%
\definecolor{currentfill}{rgb}{0.237441,0.305202,0.541921}%
\pgfsetfillcolor{currentfill}%
\pgfsetlinewidth{0.501875pt}%
\definecolor{currentstroke}{rgb}{0.237441,0.305202,0.541921}%
\pgfsetstrokecolor{currentstroke}%
\pgfsetdash{}{0pt}%
\pgfpathmoveto{\pgfqpoint{4.582822in}{4.231399in}}%
\pgfpathlineto{\pgfqpoint{4.551991in}{4.235132in}}%
\pgfpathlineto{\pgfqpoint{4.573477in}{4.257557in}}%
\pgfpathlineto{\pgfqpoint{4.582822in}{4.231399in}}%
\pgfpathlineto{\pgfqpoint{4.582822in}{4.231399in}}%
\pgfpathclose%
\pgfusepath{stroke,fill}%
\end{pgfscope}%
\begin{pgfscope}%
\pgfpathrectangle{\pgfqpoint{3.352233in}{3.252941in}}{\pgfqpoint{2.407767in}{1.544118in}}%
\pgfusepath{clip}%
\pgfsetroundcap%
\pgfsetroundjoin%
\pgfsetlinewidth{0.501875pt}%
\definecolor{currentstroke}{rgb}{0.206756,0.371758,0.553117}%
\pgfsetstrokecolor{currentstroke}%
\pgfsetdash{}{0pt}%
\pgfpathmoveto{\pgfqpoint{4.509919in}{3.833088in}}%
\pgfpathquadraticcurveto{\pgfqpoint{4.498694in}{3.837578in}}{\pgfqpoint{4.494678in}{3.839185in}}%
\pgfusepath{stroke}%
\end{pgfscope}%
\begin{pgfscope}%
\pgfpathrectangle{\pgfqpoint{3.352233in}{3.252941in}}{\pgfqpoint{2.407767in}{1.544118in}}%
\pgfusepath{clip}%
\pgfsetroundcap%
\pgfsetroundjoin%
\definecolor{currentfill}{rgb}{0.206756,0.371758,0.553117}%
\pgfsetfillcolor{currentfill}%
\pgfsetlinewidth{0.501875pt}%
\definecolor{currentstroke}{rgb}{0.206756,0.371758,0.553117}%
\pgfsetstrokecolor{currentstroke}%
\pgfsetdash{}{0pt}%
\pgfpathmoveto{\pgfqpoint{4.515310in}{3.815972in}}%
\pgfpathlineto{\pgfqpoint{4.494678in}{3.839185in}}%
\pgfpathlineto{\pgfqpoint{4.525628in}{3.841763in}}%
\pgfpathlineto{\pgfqpoint{4.515310in}{3.815972in}}%
\pgfpathlineto{\pgfqpoint{4.515310in}{3.815972in}}%
\pgfpathclose%
\pgfusepath{stroke,fill}%
\end{pgfscope}%
\begin{pgfscope}%
\pgfpathrectangle{\pgfqpoint{3.352233in}{3.252941in}}{\pgfqpoint{2.407767in}{1.544118in}}%
\pgfusepath{clip}%
\pgfsetbuttcap%
\pgfsetroundjoin%
\pgfsetlinewidth{1.505625pt}%
\definecolor{currentstroke}{rgb}{0.000000,0.000000,0.000000}%
\pgfsetstrokecolor{currentstroke}%
\pgfsetdash{}{0pt}%
\pgfpathmoveto{\pgfqpoint{4.341943in}{3.514415in}}%
\pgfpathlineto{\pgfqpoint{4.341943in}{4.535585in}}%
\pgfusepath{stroke}%
\end{pgfscope}%
\begin{pgfscope}%
\pgfpathrectangle{\pgfqpoint{3.352233in}{3.252941in}}{\pgfqpoint{2.407767in}{1.544118in}}%
\pgfusepath{clip}%
\pgfsetbuttcap%
\pgfsetroundjoin%
\pgfsetlinewidth{1.505625pt}%
\definecolor{currentstroke}{rgb}{0.000000,0.000000,0.000000}%
\pgfsetstrokecolor{currentstroke}%
\pgfsetdash{}{0pt}%
\pgfpathmoveto{\pgfqpoint{5.251844in}{3.514415in}}%
\pgfpathlineto{\pgfqpoint{5.251844in}{4.535585in}}%
\pgfusepath{stroke}%
\end{pgfscope}%
\begin{pgfscope}%
\pgfsetrectcap%
\pgfsetmiterjoin%
\pgfsetlinewidth{0.803000pt}%
\definecolor{currentstroke}{rgb}{0.000000,0.000000,0.000000}%
\pgfsetstrokecolor{currentstroke}%
\pgfsetdash{}{0pt}%
\pgfpathmoveto{\pgfqpoint{3.352233in}{3.252941in}}%
\pgfpathlineto{\pgfqpoint{3.352233in}{4.797059in}}%
\pgfusepath{stroke}%
\end{pgfscope}%
\begin{pgfscope}%
\pgfsetrectcap%
\pgfsetmiterjoin%
\pgfsetlinewidth{0.803000pt}%
\definecolor{currentstroke}{rgb}{0.000000,0.000000,0.000000}%
\pgfsetstrokecolor{currentstroke}%
\pgfsetdash{}{0pt}%
\pgfpathmoveto{\pgfqpoint{5.760000in}{3.252941in}}%
\pgfpathlineto{\pgfqpoint{5.760000in}{4.797059in}}%
\pgfusepath{stroke}%
\end{pgfscope}%
\begin{pgfscope}%
\pgfsetrectcap%
\pgfsetmiterjoin%
\pgfsetlinewidth{0.803000pt}%
\definecolor{currentstroke}{rgb}{0.000000,0.000000,0.000000}%
\pgfsetstrokecolor{currentstroke}%
\pgfsetdash{}{0pt}%
\pgfpathmoveto{\pgfqpoint{3.352233in}{3.252941in}}%
\pgfpathlineto{\pgfqpoint{5.760000in}{3.252941in}}%
\pgfusepath{stroke}%
\end{pgfscope}%
\begin{pgfscope}%
\pgfsetrectcap%
\pgfsetmiterjoin%
\pgfsetlinewidth{0.803000pt}%
\definecolor{currentstroke}{rgb}{0.000000,0.000000,0.000000}%
\pgfsetstrokecolor{currentstroke}%
\pgfsetdash{}{0pt}%
\pgfpathmoveto{\pgfqpoint{3.352233in}{4.797059in}}%
\pgfpathlineto{\pgfqpoint{5.760000in}{4.797059in}}%
\pgfusepath{stroke}%
\end{pgfscope}%
\begin{pgfscope}%
\definecolor{textcolor}{rgb}{0.000000,0.000000,0.000000}%
\pgfsetstrokecolor{textcolor}%
\pgfsetfillcolor{textcolor}%
\pgftext[x=4.556117in,y=4.880392in,,base]{\color{textcolor}\sffamily\fontsize{12.000000}{14.400000}\selectfont d)}%
\end{pgfscope}%
\begin{pgfscope}%
\pgfsetbuttcap%
\pgfsetmiterjoin%
\definecolor{currentfill}{rgb}{1.000000,1.000000,1.000000}%
\pgfsetfillcolor{currentfill}%
\pgfsetlinewidth{0.000000pt}%
\definecolor{currentstroke}{rgb}{0.000000,0.000000,0.000000}%
\pgfsetstrokecolor{currentstroke}%
\pgfsetstrokeopacity{0.000000}%
\pgfsetdash{}{0pt}%
\pgfpathmoveto{\pgfqpoint{0.800000in}{1.400000in}}%
\pgfpathlineto{\pgfqpoint{3.207767in}{1.400000in}}%
\pgfpathlineto{\pgfqpoint{3.207767in}{2.944118in}}%
\pgfpathlineto{\pgfqpoint{0.800000in}{2.944118in}}%
\pgfpathlineto{\pgfqpoint{0.800000in}{1.400000in}}%
\pgfpathclose%
\pgfusepath{fill}%
\end{pgfscope}%
\begin{pgfscope}%
\pgfpathrectangle{\pgfqpoint{0.800000in}{1.400000in}}{\pgfqpoint{2.407767in}{1.544118in}}%
\pgfusepath{clip}%
\pgfsys@transformcm{2.416667}{0.000000}{0.000000}{1.555556}{0.800000in}{1.400000in}%
\pgftext[left,bottom]{\includegraphics[interpolate=false,width=1.000000in,height=1.000000in]{q_series_square-img4.png}}%
\end{pgfscope}%
\begin{pgfscope}%
\pgfsetbuttcap%
\pgfsetroundjoin%
\definecolor{currentfill}{rgb}{0.000000,0.000000,0.000000}%
\pgfsetfillcolor{currentfill}%
\pgfsetlinewidth{0.803000pt}%
\definecolor{currentstroke}{rgb}{0.000000,0.000000,0.000000}%
\pgfsetstrokecolor{currentstroke}%
\pgfsetdash{}{0pt}%
\pgfsys@defobject{currentmarker}{\pgfqpoint{0.000000in}{-0.048611in}}{\pgfqpoint{0.000000in}{0.000000in}}{%
\pgfpathmoveto{\pgfqpoint{0.000000in}{0.000000in}}%
\pgfpathlineto{\pgfqpoint{0.000000in}{-0.048611in}}%
\pgfusepath{stroke,fill}%
}%
\begin{pgfscope}%
\pgfsys@transformshift{1.233659in}{1.400000in}%
\pgfsys@useobject{currentmarker}{}%
\end{pgfscope}%
\end{pgfscope}%
\begin{pgfscope}%
\definecolor{textcolor}{rgb}{0.000000,0.000000,0.000000}%
\pgfsetstrokecolor{textcolor}%
\pgfsetfillcolor{textcolor}%
\pgftext[x=1.233659in,y=1.302778in,,top]{\color{textcolor}\sffamily\fontsize{10.000000}{12.000000}\selectfont \(\displaystyle {\ensuremath{-}10}\)}%
\end{pgfscope}%
\begin{pgfscope}%
\pgfsetbuttcap%
\pgfsetroundjoin%
\definecolor{currentfill}{rgb}{0.000000,0.000000,0.000000}%
\pgfsetfillcolor{currentfill}%
\pgfsetlinewidth{0.803000pt}%
\definecolor{currentstroke}{rgb}{0.000000,0.000000,0.000000}%
\pgfsetstrokecolor{currentstroke}%
\pgfsetdash{}{0pt}%
\pgfsys@defobject{currentmarker}{\pgfqpoint{0.000000in}{-0.048611in}}{\pgfqpoint{0.000000in}{0.000000in}}{%
\pgfpathmoveto{\pgfqpoint{0.000000in}{0.000000in}}%
\pgfpathlineto{\pgfqpoint{0.000000in}{-0.048611in}}%
\pgfusepath{stroke,fill}%
}%
\begin{pgfscope}%
\pgfsys@transformshift{1.739160in}{1.400000in}%
\pgfsys@useobject{currentmarker}{}%
\end{pgfscope}%
\end{pgfscope}%
\begin{pgfscope}%
\definecolor{textcolor}{rgb}{0.000000,0.000000,0.000000}%
\pgfsetstrokecolor{textcolor}%
\pgfsetfillcolor{textcolor}%
\pgftext[x=1.739160in,y=1.302778in,,top]{\color{textcolor}\sffamily\fontsize{10.000000}{12.000000}\selectfont \(\displaystyle {\ensuremath{-}5}\)}%
\end{pgfscope}%
\begin{pgfscope}%
\pgfsetbuttcap%
\pgfsetroundjoin%
\definecolor{currentfill}{rgb}{0.000000,0.000000,0.000000}%
\pgfsetfillcolor{currentfill}%
\pgfsetlinewidth{0.803000pt}%
\definecolor{currentstroke}{rgb}{0.000000,0.000000,0.000000}%
\pgfsetstrokecolor{currentstroke}%
\pgfsetdash{}{0pt}%
\pgfsys@defobject{currentmarker}{\pgfqpoint{0.000000in}{-0.048611in}}{\pgfqpoint{0.000000in}{0.000000in}}{%
\pgfpathmoveto{\pgfqpoint{0.000000in}{0.000000in}}%
\pgfpathlineto{\pgfqpoint{0.000000in}{-0.048611in}}%
\pgfusepath{stroke,fill}%
}%
\begin{pgfscope}%
\pgfsys@transformshift{2.244660in}{1.400000in}%
\pgfsys@useobject{currentmarker}{}%
\end{pgfscope}%
\end{pgfscope}%
\begin{pgfscope}%
\definecolor{textcolor}{rgb}{0.000000,0.000000,0.000000}%
\pgfsetstrokecolor{textcolor}%
\pgfsetfillcolor{textcolor}%
\pgftext[x=2.244660in,y=1.302778in,,top]{\color{textcolor}\sffamily\fontsize{10.000000}{12.000000}\selectfont \(\displaystyle {0}\)}%
\end{pgfscope}%
\begin{pgfscope}%
\pgfsetbuttcap%
\pgfsetroundjoin%
\definecolor{currentfill}{rgb}{0.000000,0.000000,0.000000}%
\pgfsetfillcolor{currentfill}%
\pgfsetlinewidth{0.803000pt}%
\definecolor{currentstroke}{rgb}{0.000000,0.000000,0.000000}%
\pgfsetstrokecolor{currentstroke}%
\pgfsetdash{}{0pt}%
\pgfsys@defobject{currentmarker}{\pgfqpoint{0.000000in}{-0.048611in}}{\pgfqpoint{0.000000in}{0.000000in}}{%
\pgfpathmoveto{\pgfqpoint{0.000000in}{0.000000in}}%
\pgfpathlineto{\pgfqpoint{0.000000in}{-0.048611in}}%
\pgfusepath{stroke,fill}%
}%
\begin{pgfscope}%
\pgfsys@transformshift{2.750161in}{1.400000in}%
\pgfsys@useobject{currentmarker}{}%
\end{pgfscope}%
\end{pgfscope}%
\begin{pgfscope}%
\definecolor{textcolor}{rgb}{0.000000,0.000000,0.000000}%
\pgfsetstrokecolor{textcolor}%
\pgfsetfillcolor{textcolor}%
\pgftext[x=2.750161in,y=1.302778in,,top]{\color{textcolor}\sffamily\fontsize{10.000000}{12.000000}\selectfont \(\displaystyle {5}\)}%
\end{pgfscope}%
\begin{pgfscope}%
\definecolor{textcolor}{rgb}{0.000000,0.000000,0.000000}%
\pgfsetstrokecolor{textcolor}%
\pgfsetfillcolor{textcolor}%
\pgftext[x=2.003883in,y=1.123766in,,top]{\color{textcolor}\sffamily\fontsize{10.000000}{12.000000}\selectfont \(\displaystyle \zeta \, \mathrm{[\mu m]}\)}%
\end{pgfscope}%
\begin{pgfscope}%
\pgfsetbuttcap%
\pgfsetroundjoin%
\definecolor{currentfill}{rgb}{0.000000,0.000000,0.000000}%
\pgfsetfillcolor{currentfill}%
\pgfsetlinewidth{0.803000pt}%
\definecolor{currentstroke}{rgb}{0.000000,0.000000,0.000000}%
\pgfsetstrokecolor{currentstroke}%
\pgfsetdash{}{0pt}%
\pgfsys@defobject{currentmarker}{\pgfqpoint{-0.048611in}{0.000000in}}{\pgfqpoint{-0.000000in}{0.000000in}}{%
\pgfpathmoveto{\pgfqpoint{-0.000000in}{0.000000in}}%
\pgfpathlineto{\pgfqpoint{-0.048611in}{0.000000in}}%
\pgfusepath{stroke,fill}%
}%
\begin{pgfscope}%
\pgfsys@transformshift{0.800000in}{1.661474in}%
\pgfsys@useobject{currentmarker}{}%
\end{pgfscope}%
\end{pgfscope}%
\begin{pgfscope}%
\definecolor{textcolor}{rgb}{0.000000,0.000000,0.000000}%
\pgfsetstrokecolor{textcolor}%
\pgfsetfillcolor{textcolor}%
\pgftext[x=0.455863in, y=1.613249in, left, base]{\color{textcolor}\sffamily\fontsize{10.000000}{12.000000}\selectfont \(\displaystyle {\ensuremath{-}20}\)}%
\end{pgfscope}%
\begin{pgfscope}%
\pgfsetbuttcap%
\pgfsetroundjoin%
\definecolor{currentfill}{rgb}{0.000000,0.000000,0.000000}%
\pgfsetfillcolor{currentfill}%
\pgfsetlinewidth{0.803000pt}%
\definecolor{currentstroke}{rgb}{0.000000,0.000000,0.000000}%
\pgfsetstrokecolor{currentstroke}%
\pgfsetdash{}{0pt}%
\pgfsys@defobject{currentmarker}{\pgfqpoint{-0.048611in}{0.000000in}}{\pgfqpoint{-0.000000in}{0.000000in}}{%
\pgfpathmoveto{\pgfqpoint{-0.000000in}{0.000000in}}%
\pgfpathlineto{\pgfqpoint{-0.048611in}{0.000000in}}%
\pgfusepath{stroke,fill}%
}%
\begin{pgfscope}%
\pgfsys@transformshift{0.800000in}{2.172059in}%
\pgfsys@useobject{currentmarker}{}%
\end{pgfscope}%
\end{pgfscope}%
\begin{pgfscope}%
\definecolor{textcolor}{rgb}{0.000000,0.000000,0.000000}%
\pgfsetstrokecolor{textcolor}%
\pgfsetfillcolor{textcolor}%
\pgftext[x=0.633333in, y=2.123834in, left, base]{\color{textcolor}\sffamily\fontsize{10.000000}{12.000000}\selectfont \(\displaystyle {0}\)}%
\end{pgfscope}%
\begin{pgfscope}%
\pgfsetbuttcap%
\pgfsetroundjoin%
\definecolor{currentfill}{rgb}{0.000000,0.000000,0.000000}%
\pgfsetfillcolor{currentfill}%
\pgfsetlinewidth{0.803000pt}%
\definecolor{currentstroke}{rgb}{0.000000,0.000000,0.000000}%
\pgfsetstrokecolor{currentstroke}%
\pgfsetdash{}{0pt}%
\pgfsys@defobject{currentmarker}{\pgfqpoint{-0.048611in}{0.000000in}}{\pgfqpoint{-0.000000in}{0.000000in}}{%
\pgfpathmoveto{\pgfqpoint{-0.000000in}{0.000000in}}%
\pgfpathlineto{\pgfqpoint{-0.048611in}{0.000000in}}%
\pgfusepath{stroke,fill}%
}%
\begin{pgfscope}%
\pgfsys@transformshift{0.800000in}{2.682644in}%
\pgfsys@useobject{currentmarker}{}%
\end{pgfscope}%
\end{pgfscope}%
\begin{pgfscope}%
\definecolor{textcolor}{rgb}{0.000000,0.000000,0.000000}%
\pgfsetstrokecolor{textcolor}%
\pgfsetfillcolor{textcolor}%
\pgftext[x=0.563888in, y=2.634418in, left, base]{\color{textcolor}\sffamily\fontsize{10.000000}{12.000000}\selectfont \(\displaystyle {20}\)}%
\end{pgfscope}%
\begin{pgfscope}%
\definecolor{textcolor}{rgb}{0.000000,0.000000,0.000000}%
\pgfsetstrokecolor{textcolor}%
\pgfsetfillcolor{textcolor}%
\pgftext[x=0.400308in,y=2.172059in,,bottom,rotate=90.000000]{\color{textcolor}\sffamily\fontsize{10.000000}{12.000000}\selectfont \(\displaystyle z \, \mathrm{[\mu m]}\)}%
\end{pgfscope}%
\begin{pgfscope}%
\pgfpathrectangle{\pgfqpoint{0.800000in}{1.400000in}}{\pgfqpoint{2.407767in}{1.544118in}}%
\pgfusepath{clip}%
\pgfsetbuttcap%
\pgfsetroundjoin%
\pgfsetlinewidth{0.000000pt}%
\definecolor{currentstroke}{rgb}{0.000000,0.000000,0.000000}%
\pgfsetstrokecolor{currentstroke}%
\pgfsetdash{}{0pt}%
\pgfpathmoveto{\pgfqpoint{2.672823in}{2.833724in}}%
\pgfpathlineto{\pgfqpoint{2.654046in}{2.832235in}}%
\pgfusepath{}%
\end{pgfscope}%
\begin{pgfscope}%
\pgfpathrectangle{\pgfqpoint{0.800000in}{1.400000in}}{\pgfqpoint{2.407767in}{1.544118in}}%
\pgfusepath{clip}%
\pgfsetbuttcap%
\pgfsetroundjoin%
\pgfsetlinewidth{0.501875pt}%
\definecolor{currentstroke}{rgb}{0.268510,0.009605,0.335427}%
\pgfsetstrokecolor{currentstroke}%
\pgfsetdash{}{0pt}%
\pgfpathmoveto{\pgfqpoint{2.654046in}{2.832235in}}%
\pgfpathlineto{\pgfqpoint{2.654046in}{2.832235in}}%
\pgfusepath{stroke}%
\end{pgfscope}%
\begin{pgfscope}%
\pgfpathrectangle{\pgfqpoint{0.800000in}{1.400000in}}{\pgfqpoint{2.407767in}{1.544118in}}%
\pgfusepath{clip}%
\pgfsetbuttcap%
\pgfsetroundjoin%
\pgfsetlinewidth{0.501875pt}%
\definecolor{currentstroke}{rgb}{0.268510,0.009605,0.335427}%
\pgfsetstrokecolor{currentstroke}%
\pgfsetdash{}{0pt}%
\pgfpathmoveto{\pgfqpoint{2.654046in}{2.832235in}}%
\pgfpathlineto{\pgfqpoint{2.640831in}{2.830553in}}%
\pgfusepath{stroke}%
\end{pgfscope}%
\begin{pgfscope}%
\pgfpathrectangle{\pgfqpoint{0.800000in}{1.400000in}}{\pgfqpoint{2.407767in}{1.544118in}}%
\pgfusepath{clip}%
\pgfsetbuttcap%
\pgfsetroundjoin%
\pgfsetlinewidth{0.501875pt}%
\definecolor{currentstroke}{rgb}{0.267004,0.004874,0.329415}%
\pgfsetstrokecolor{currentstroke}%
\pgfsetdash{}{0pt}%
\pgfpathmoveto{\pgfqpoint{2.640831in}{2.830553in}}%
\pgfpathlineto{\pgfqpoint{2.640831in}{2.830553in}}%
\pgfusepath{stroke}%
\end{pgfscope}%
\begin{pgfscope}%
\pgfpathrectangle{\pgfqpoint{0.800000in}{1.400000in}}{\pgfqpoint{2.407767in}{1.544118in}}%
\pgfusepath{clip}%
\pgfsetbuttcap%
\pgfsetroundjoin%
\pgfsetlinewidth{0.501875pt}%
\definecolor{currentstroke}{rgb}{0.267004,0.004874,0.329415}%
\pgfsetstrokecolor{currentstroke}%
\pgfsetdash{}{0pt}%
\pgfpathmoveto{\pgfqpoint{2.640831in}{2.830553in}}%
\pgfpathlineto{\pgfqpoint{2.632111in}{2.829357in}}%
\pgfusepath{stroke}%
\end{pgfscope}%
\begin{pgfscope}%
\pgfpathrectangle{\pgfqpoint{0.800000in}{1.400000in}}{\pgfqpoint{2.407767in}{1.544118in}}%
\pgfusepath{clip}%
\pgfsetbuttcap%
\pgfsetroundjoin%
\pgfsetlinewidth{0.501875pt}%
\definecolor{currentstroke}{rgb}{0.267004,0.004874,0.329415}%
\pgfsetstrokecolor{currentstroke}%
\pgfsetdash{}{0pt}%
\pgfpathmoveto{\pgfqpoint{2.632111in}{2.829357in}}%
\pgfpathlineto{\pgfqpoint{2.632111in}{2.829357in}}%
\pgfusepath{stroke}%
\end{pgfscope}%
\begin{pgfscope}%
\pgfpathrectangle{\pgfqpoint{0.800000in}{1.400000in}}{\pgfqpoint{2.407767in}{1.544118in}}%
\pgfusepath{clip}%
\pgfsetbuttcap%
\pgfsetroundjoin%
\pgfsetlinewidth{0.501875pt}%
\definecolor{currentstroke}{rgb}{0.267004,0.004874,0.329415}%
\pgfsetstrokecolor{currentstroke}%
\pgfsetdash{}{0pt}%
\pgfpathmoveto{\pgfqpoint{2.632111in}{2.829357in}}%
\pgfpathlineto{\pgfqpoint{2.632111in}{2.829357in}}%
\pgfusepath{stroke}%
\end{pgfscope}%
\begin{pgfscope}%
\pgfpathrectangle{\pgfqpoint{0.800000in}{1.400000in}}{\pgfqpoint{2.407767in}{1.544118in}}%
\pgfusepath{clip}%
\pgfsetbuttcap%
\pgfsetroundjoin%
\pgfsetlinewidth{0.501875pt}%
\definecolor{currentstroke}{rgb}{0.267004,0.004874,0.329415}%
\pgfsetstrokecolor{currentstroke}%
\pgfsetdash{}{0pt}%
\pgfpathmoveto{\pgfqpoint{2.632111in}{2.829357in}}%
\pgfpathlineto{\pgfqpoint{2.609825in}{2.828895in}}%
\pgfusepath{stroke}%
\end{pgfscope}%
\begin{pgfscope}%
\pgfpathrectangle{\pgfqpoint{0.800000in}{1.400000in}}{\pgfqpoint{2.407767in}{1.544118in}}%
\pgfusepath{clip}%
\pgfsetbuttcap%
\pgfsetroundjoin%
\pgfsetlinewidth{0.501875pt}%
\definecolor{currentstroke}{rgb}{0.268510,0.009605,0.335427}%
\pgfsetstrokecolor{currentstroke}%
\pgfsetdash{}{0pt}%
\pgfpathmoveto{\pgfqpoint{2.609825in}{2.828895in}}%
\pgfpathlineto{\pgfqpoint{2.583826in}{2.828278in}}%
\pgfusepath{stroke}%
\end{pgfscope}%
\begin{pgfscope}%
\pgfpathrectangle{\pgfqpoint{0.800000in}{1.400000in}}{\pgfqpoint{2.407767in}{1.544118in}}%
\pgfusepath{clip}%
\pgfsetbuttcap%
\pgfsetroundjoin%
\pgfsetlinewidth{0.501875pt}%
\definecolor{currentstroke}{rgb}{0.269944,0.014625,0.341379}%
\pgfsetstrokecolor{currentstroke}%
\pgfsetdash{}{0pt}%
\pgfpathmoveto{\pgfqpoint{2.583826in}{2.828278in}}%
\pgfpathlineto{\pgfqpoint{2.536740in}{2.825246in}}%
\pgfusepath{stroke}%
\end{pgfscope}%
\begin{pgfscope}%
\pgfpathrectangle{\pgfqpoint{0.800000in}{1.400000in}}{\pgfqpoint{2.407767in}{1.544118in}}%
\pgfusepath{clip}%
\pgfsetbuttcap%
\pgfsetroundjoin%
\pgfsetlinewidth{0.501875pt}%
\definecolor{currentstroke}{rgb}{0.268510,0.009605,0.335427}%
\pgfsetstrokecolor{currentstroke}%
\pgfsetdash{}{0pt}%
\pgfpathmoveto{\pgfqpoint{2.654046in}{2.797489in}}%
\pgfpathlineto{\pgfqpoint{2.601121in}{2.797551in}}%
\pgfusepath{stroke}%
\end{pgfscope}%
\begin{pgfscope}%
\pgfpathrectangle{\pgfqpoint{0.800000in}{1.400000in}}{\pgfqpoint{2.407767in}{1.544118in}}%
\pgfusepath{clip}%
\pgfsetbuttcap%
\pgfsetroundjoin%
\pgfsetlinewidth{0.501875pt}%
\definecolor{currentstroke}{rgb}{0.268510,0.009605,0.335427}%
\pgfsetstrokecolor{currentstroke}%
\pgfsetdash{}{0pt}%
\pgfpathmoveto{\pgfqpoint{2.601121in}{2.797551in}}%
\pgfpathlineto{\pgfqpoint{2.548213in}{2.797753in}}%
\pgfusepath{stroke}%
\end{pgfscope}%
\begin{pgfscope}%
\pgfpathrectangle{\pgfqpoint{0.800000in}{1.400000in}}{\pgfqpoint{2.407767in}{1.544118in}}%
\pgfusepath{clip}%
\pgfsetbuttcap%
\pgfsetroundjoin%
\pgfsetlinewidth{0.501875pt}%
\definecolor{currentstroke}{rgb}{0.271305,0.019942,0.347269}%
\pgfsetstrokecolor{currentstroke}%
\pgfsetdash{}{0pt}%
\pgfpathmoveto{\pgfqpoint{2.548213in}{2.797753in}}%
\pgfpathlineto{\pgfqpoint{2.495377in}{2.795934in}}%
\pgfusepath{stroke}%
\end{pgfscope}%
\begin{pgfscope}%
\pgfpathrectangle{\pgfqpoint{0.800000in}{1.400000in}}{\pgfqpoint{2.407767in}{1.544118in}}%
\pgfusepath{clip}%
\pgfsetbuttcap%
\pgfsetroundjoin%
\pgfsetlinewidth{0.501875pt}%
\definecolor{currentstroke}{rgb}{0.271305,0.019942,0.347269}%
\pgfsetstrokecolor{currentstroke}%
\pgfsetdash{}{0pt}%
\pgfpathmoveto{\pgfqpoint{2.495377in}{2.795934in}}%
\pgfpathlineto{\pgfqpoint{2.442561in}{2.793706in}}%
\pgfusepath{stroke}%
\end{pgfscope}%
\begin{pgfscope}%
\pgfpathrectangle{\pgfqpoint{0.800000in}{1.400000in}}{\pgfqpoint{2.407767in}{1.544118in}}%
\pgfusepath{clip}%
\pgfsetbuttcap%
\pgfsetroundjoin%
\pgfsetlinewidth{0.501875pt}%
\definecolor{currentstroke}{rgb}{0.272594,0.025563,0.353093}%
\pgfsetstrokecolor{currentstroke}%
\pgfsetdash{}{0pt}%
\pgfpathmoveto{\pgfqpoint{2.442561in}{2.793706in}}%
\pgfpathlineto{\pgfqpoint{2.389809in}{2.790903in}}%
\pgfusepath{stroke}%
\end{pgfscope}%
\begin{pgfscope}%
\pgfpathrectangle{\pgfqpoint{0.800000in}{1.400000in}}{\pgfqpoint{2.407767in}{1.544118in}}%
\pgfusepath{clip}%
\pgfsetbuttcap%
\pgfsetroundjoin%
\pgfsetlinewidth{0.501875pt}%
\definecolor{currentstroke}{rgb}{0.269944,0.014625,0.341379}%
\pgfsetstrokecolor{currentstroke}%
\pgfsetdash{}{0pt}%
\pgfpathmoveto{\pgfqpoint{1.852853in}{1.547629in}}%
\pgfpathlineto{\pgfqpoint{1.859406in}{1.554021in}}%
\pgfusepath{stroke}%
\end{pgfscope}%
\begin{pgfscope}%
\pgfpathrectangle{\pgfqpoint{0.800000in}{1.400000in}}{\pgfqpoint{2.407767in}{1.544118in}}%
\pgfusepath{clip}%
\pgfsetbuttcap%
\pgfsetroundjoin%
\pgfsetlinewidth{0.501875pt}%
\definecolor{currentstroke}{rgb}{0.269944,0.014625,0.341379}%
\pgfsetstrokecolor{currentstroke}%
\pgfsetdash{}{0pt}%
\pgfpathmoveto{\pgfqpoint{1.859406in}{1.554021in}}%
\pgfpathlineto{\pgfqpoint{1.861190in}{1.560892in}}%
\pgfusepath{stroke}%
\end{pgfscope}%
\begin{pgfscope}%
\pgfpathrectangle{\pgfqpoint{0.800000in}{1.400000in}}{\pgfqpoint{2.407767in}{1.544118in}}%
\pgfusepath{clip}%
\pgfsetbuttcap%
\pgfsetroundjoin%
\pgfsetlinewidth{0.501875pt}%
\definecolor{currentstroke}{rgb}{0.269944,0.014625,0.341379}%
\pgfsetstrokecolor{currentstroke}%
\pgfsetdash{}{0pt}%
\pgfpathmoveto{\pgfqpoint{1.861190in}{1.560892in}}%
\pgfpathlineto{\pgfqpoint{1.862440in}{1.566418in}}%
\pgfusepath{stroke}%
\end{pgfscope}%
\begin{pgfscope}%
\pgfpathrectangle{\pgfqpoint{0.800000in}{1.400000in}}{\pgfqpoint{2.407767in}{1.544118in}}%
\pgfusepath{clip}%
\pgfsetbuttcap%
\pgfsetroundjoin%
\pgfsetlinewidth{0.501875pt}%
\definecolor{currentstroke}{rgb}{0.269944,0.014625,0.341379}%
\pgfsetstrokecolor{currentstroke}%
\pgfsetdash{}{0pt}%
\pgfpathmoveto{\pgfqpoint{1.862440in}{1.566418in}}%
\pgfpathlineto{\pgfqpoint{1.862440in}{1.566418in}}%
\pgfusepath{stroke}%
\end{pgfscope}%
\begin{pgfscope}%
\pgfpathrectangle{\pgfqpoint{0.800000in}{1.400000in}}{\pgfqpoint{2.407767in}{1.544118in}}%
\pgfusepath{clip}%
\pgfsetbuttcap%
\pgfsetroundjoin%
\pgfsetlinewidth{0.501875pt}%
\definecolor{currentstroke}{rgb}{0.269944,0.014625,0.341379}%
\pgfsetstrokecolor{currentstroke}%
\pgfsetdash{}{0pt}%
\pgfpathmoveto{\pgfqpoint{1.862440in}{1.566418in}}%
\pgfpathlineto{\pgfqpoint{1.862440in}{1.566418in}}%
\pgfusepath{stroke}%
\end{pgfscope}%
\begin{pgfscope}%
\pgfpathrectangle{\pgfqpoint{0.800000in}{1.400000in}}{\pgfqpoint{2.407767in}{1.544118in}}%
\pgfusepath{clip}%
\pgfsetbuttcap%
\pgfsetroundjoin%
\pgfsetlinewidth{0.501875pt}%
\definecolor{currentstroke}{rgb}{0.269944,0.014625,0.341379}%
\pgfsetstrokecolor{currentstroke}%
\pgfsetdash{}{0pt}%
\pgfpathmoveto{\pgfqpoint{1.862440in}{1.566418in}}%
\pgfpathlineto{\pgfqpoint{1.862440in}{1.566418in}}%
\pgfusepath{stroke}%
\end{pgfscope}%
\begin{pgfscope}%
\pgfpathrectangle{\pgfqpoint{0.800000in}{1.400000in}}{\pgfqpoint{2.407767in}{1.544118in}}%
\pgfusepath{clip}%
\pgfsetbuttcap%
\pgfsetroundjoin%
\pgfsetlinewidth{0.501875pt}%
\definecolor{currentstroke}{rgb}{0.269944,0.014625,0.341379}%
\pgfsetstrokecolor{currentstroke}%
\pgfsetdash{}{0pt}%
\pgfpathmoveto{\pgfqpoint{1.862440in}{1.566418in}}%
\pgfpathlineto{\pgfqpoint{1.895523in}{1.581375in}}%
\pgfusepath{stroke}%
\end{pgfscope}%
\begin{pgfscope}%
\pgfpathrectangle{\pgfqpoint{0.800000in}{1.400000in}}{\pgfqpoint{2.407767in}{1.544118in}}%
\pgfusepath{clip}%
\pgfsetbuttcap%
\pgfsetroundjoin%
\pgfsetlinewidth{0.501875pt}%
\definecolor{currentstroke}{rgb}{0.271305,0.019942,0.347269}%
\pgfsetstrokecolor{currentstroke}%
\pgfsetdash{}{0pt}%
\pgfpathmoveto{\pgfqpoint{1.895523in}{1.581375in}}%
\pgfpathlineto{\pgfqpoint{1.895523in}{1.581375in}}%
\pgfusepath{stroke}%
\end{pgfscope}%
\begin{pgfscope}%
\pgfpathrectangle{\pgfqpoint{0.800000in}{1.400000in}}{\pgfqpoint{2.407767in}{1.544118in}}%
\pgfusepath{clip}%
\pgfsetbuttcap%
\pgfsetroundjoin%
\pgfsetlinewidth{0.501875pt}%
\definecolor{currentstroke}{rgb}{0.271305,0.019942,0.347269}%
\pgfsetstrokecolor{currentstroke}%
\pgfsetdash{}{0pt}%
\pgfpathmoveto{\pgfqpoint{1.895523in}{1.581375in}}%
\pgfpathlineto{\pgfqpoint{1.895523in}{1.581375in}}%
\pgfusepath{stroke}%
\end{pgfscope}%
\begin{pgfscope}%
\pgfpathrectangle{\pgfqpoint{0.800000in}{1.400000in}}{\pgfqpoint{2.407767in}{1.544118in}}%
\pgfusepath{clip}%
\pgfsetbuttcap%
\pgfsetroundjoin%
\pgfsetlinewidth{0.501875pt}%
\definecolor{currentstroke}{rgb}{0.271305,0.019942,0.347269}%
\pgfsetstrokecolor{currentstroke}%
\pgfsetdash{}{0pt}%
\pgfpathmoveto{\pgfqpoint{1.895523in}{1.581375in}}%
\pgfpathlineto{\pgfqpoint{1.895523in}{1.581375in}}%
\pgfusepath{stroke}%
\end{pgfscope}%
\begin{pgfscope}%
\pgfpathrectangle{\pgfqpoint{0.800000in}{1.400000in}}{\pgfqpoint{2.407767in}{1.544118in}}%
\pgfusepath{clip}%
\pgfsetbuttcap%
\pgfsetroundjoin%
\pgfsetlinewidth{0.501875pt}%
\definecolor{currentstroke}{rgb}{0.271305,0.019942,0.347269}%
\pgfsetstrokecolor{currentstroke}%
\pgfsetdash{}{0pt}%
\pgfpathmoveto{\pgfqpoint{1.895523in}{1.581375in}}%
\pgfpathlineto{\pgfqpoint{1.900277in}{1.585709in}}%
\pgfusepath{stroke}%
\end{pgfscope}%
\begin{pgfscope}%
\pgfpathrectangle{\pgfqpoint{0.800000in}{1.400000in}}{\pgfqpoint{2.407767in}{1.544118in}}%
\pgfusepath{clip}%
\pgfsetbuttcap%
\pgfsetroundjoin%
\pgfsetlinewidth{0.501875pt}%
\definecolor{currentstroke}{rgb}{0.272594,0.025563,0.353093}%
\pgfsetstrokecolor{currentstroke}%
\pgfsetdash{}{0pt}%
\pgfpathmoveto{\pgfqpoint{1.900277in}{1.585709in}}%
\pgfpathlineto{\pgfqpoint{1.899059in}{1.591205in}}%
\pgfusepath{stroke}%
\end{pgfscope}%
\begin{pgfscope}%
\pgfpathrectangle{\pgfqpoint{0.800000in}{1.400000in}}{\pgfqpoint{2.407767in}{1.544118in}}%
\pgfusepath{clip}%
\pgfsetbuttcap%
\pgfsetroundjoin%
\pgfsetlinewidth{0.501875pt}%
\definecolor{currentstroke}{rgb}{0.272594,0.025563,0.353093}%
\pgfsetstrokecolor{currentstroke}%
\pgfsetdash{}{0pt}%
\pgfpathmoveto{\pgfqpoint{1.899059in}{1.591205in}}%
\pgfpathlineto{\pgfqpoint{1.900888in}{1.596771in}}%
\pgfusepath{stroke}%
\end{pgfscope}%
\begin{pgfscope}%
\pgfpathrectangle{\pgfqpoint{0.800000in}{1.400000in}}{\pgfqpoint{2.407767in}{1.544118in}}%
\pgfusepath{clip}%
\pgfsetbuttcap%
\pgfsetroundjoin%
\pgfsetlinewidth{0.501875pt}%
\definecolor{currentstroke}{rgb}{0.271305,0.019942,0.347269}%
\pgfsetstrokecolor{currentstroke}%
\pgfsetdash{}{0pt}%
\pgfpathmoveto{\pgfqpoint{1.900888in}{1.596771in}}%
\pgfpathlineto{\pgfqpoint{1.907950in}{1.600972in}}%
\pgfusepath{stroke}%
\end{pgfscope}%
\begin{pgfscope}%
\pgfpathrectangle{\pgfqpoint{0.800000in}{1.400000in}}{\pgfqpoint{2.407767in}{1.544118in}}%
\pgfusepath{clip}%
\pgfsetbuttcap%
\pgfsetroundjoin%
\pgfsetlinewidth{0.501875pt}%
\definecolor{currentstroke}{rgb}{0.272594,0.025563,0.353093}%
\pgfsetstrokecolor{currentstroke}%
\pgfsetdash{}{0pt}%
\pgfpathmoveto{\pgfqpoint{1.907950in}{1.600972in}}%
\pgfpathlineto{\pgfqpoint{1.917212in}{1.604356in}}%
\pgfusepath{stroke}%
\end{pgfscope}%
\begin{pgfscope}%
\pgfpathrectangle{\pgfqpoint{0.800000in}{1.400000in}}{\pgfqpoint{2.407767in}{1.544118in}}%
\pgfusepath{clip}%
\pgfsetbuttcap%
\pgfsetroundjoin%
\pgfsetlinewidth{0.501875pt}%
\definecolor{currentstroke}{rgb}{0.271305,0.019942,0.347269}%
\pgfsetstrokecolor{currentstroke}%
\pgfsetdash{}{0pt}%
\pgfpathmoveto{\pgfqpoint{1.917212in}{1.604356in}}%
\pgfpathlineto{\pgfqpoint{1.917212in}{1.604356in}}%
\pgfusepath{stroke}%
\end{pgfscope}%
\begin{pgfscope}%
\pgfpathrectangle{\pgfqpoint{0.800000in}{1.400000in}}{\pgfqpoint{2.407767in}{1.544118in}}%
\pgfusepath{clip}%
\pgfsetbuttcap%
\pgfsetroundjoin%
\pgfsetlinewidth{0.501875pt}%
\definecolor{currentstroke}{rgb}{0.271305,0.019942,0.347269}%
\pgfsetstrokecolor{currentstroke}%
\pgfsetdash{}{0pt}%
\pgfpathmoveto{\pgfqpoint{1.917212in}{1.604356in}}%
\pgfpathlineto{\pgfqpoint{1.917212in}{1.604356in}}%
\pgfusepath{stroke}%
\end{pgfscope}%
\begin{pgfscope}%
\pgfpathrectangle{\pgfqpoint{0.800000in}{1.400000in}}{\pgfqpoint{2.407767in}{1.544118in}}%
\pgfusepath{clip}%
\pgfsetbuttcap%
\pgfsetroundjoin%
\pgfsetlinewidth{0.501875pt}%
\definecolor{currentstroke}{rgb}{0.271305,0.019942,0.347269}%
\pgfsetstrokecolor{currentstroke}%
\pgfsetdash{}{0pt}%
\pgfpathmoveto{\pgfqpoint{1.917212in}{1.604356in}}%
\pgfpathlineto{\pgfqpoint{1.921339in}{1.608816in}}%
\pgfusepath{stroke}%
\end{pgfscope}%
\begin{pgfscope}%
\pgfpathrectangle{\pgfqpoint{0.800000in}{1.400000in}}{\pgfqpoint{2.407767in}{1.544118in}}%
\pgfusepath{clip}%
\pgfsetbuttcap%
\pgfsetroundjoin%
\pgfsetlinewidth{0.501875pt}%
\definecolor{currentstroke}{rgb}{0.271305,0.019942,0.347269}%
\pgfsetstrokecolor{currentstroke}%
\pgfsetdash{}{0pt}%
\pgfpathmoveto{\pgfqpoint{1.921339in}{1.608816in}}%
\pgfpathlineto{\pgfqpoint{1.926034in}{1.614925in}}%
\pgfusepath{stroke}%
\end{pgfscope}%
\begin{pgfscope}%
\pgfpathrectangle{\pgfqpoint{0.800000in}{1.400000in}}{\pgfqpoint{2.407767in}{1.544118in}}%
\pgfusepath{clip}%
\pgfsetbuttcap%
\pgfsetroundjoin%
\pgfsetlinewidth{0.501875pt}%
\definecolor{currentstroke}{rgb}{0.271305,0.019942,0.347269}%
\pgfsetstrokecolor{currentstroke}%
\pgfsetdash{}{0pt}%
\pgfpathmoveto{\pgfqpoint{1.926034in}{1.614925in}}%
\pgfpathlineto{\pgfqpoint{1.933675in}{1.625395in}}%
\pgfusepath{stroke}%
\end{pgfscope}%
\begin{pgfscope}%
\pgfpathrectangle{\pgfqpoint{0.800000in}{1.400000in}}{\pgfqpoint{2.407767in}{1.544118in}}%
\pgfusepath{clip}%
\pgfsetbuttcap%
\pgfsetroundjoin%
\pgfsetlinewidth{0.501875pt}%
\definecolor{currentstroke}{rgb}{0.269944,0.014625,0.341379}%
\pgfsetstrokecolor{currentstroke}%
\pgfsetdash{}{0pt}%
\pgfpathmoveto{\pgfqpoint{1.933675in}{1.625395in}}%
\pgfpathlineto{\pgfqpoint{1.933675in}{1.625395in}}%
\pgfusepath{stroke}%
\end{pgfscope}%
\begin{pgfscope}%
\pgfpathrectangle{\pgfqpoint{0.800000in}{1.400000in}}{\pgfqpoint{2.407767in}{1.544118in}}%
\pgfusepath{clip}%
\pgfsetbuttcap%
\pgfsetroundjoin%
\pgfsetlinewidth{0.501875pt}%
\definecolor{currentstroke}{rgb}{0.269944,0.014625,0.341379}%
\pgfsetstrokecolor{currentstroke}%
\pgfsetdash{}{0pt}%
\pgfpathmoveto{\pgfqpoint{1.933675in}{1.625395in}}%
\pgfpathlineto{\pgfqpoint{1.933675in}{1.625395in}}%
\pgfusepath{stroke}%
\end{pgfscope}%
\begin{pgfscope}%
\pgfpathrectangle{\pgfqpoint{0.800000in}{1.400000in}}{\pgfqpoint{2.407767in}{1.544118in}}%
\pgfusepath{clip}%
\pgfsetbuttcap%
\pgfsetroundjoin%
\pgfsetlinewidth{0.501875pt}%
\definecolor{currentstroke}{rgb}{0.269944,0.014625,0.341379}%
\pgfsetstrokecolor{currentstroke}%
\pgfsetdash{}{0pt}%
\pgfpathmoveto{\pgfqpoint{1.933675in}{1.625395in}}%
\pgfpathlineto{\pgfqpoint{1.932882in}{1.632801in}}%
\pgfusepath{stroke}%
\end{pgfscope}%
\begin{pgfscope}%
\pgfpathrectangle{\pgfqpoint{0.800000in}{1.400000in}}{\pgfqpoint{2.407767in}{1.544118in}}%
\pgfusepath{clip}%
\pgfsetbuttcap%
\pgfsetroundjoin%
\pgfsetlinewidth{0.501875pt}%
\definecolor{currentstroke}{rgb}{0.269944,0.014625,0.341379}%
\pgfsetstrokecolor{currentstroke}%
\pgfsetdash{}{0pt}%
\pgfpathmoveto{\pgfqpoint{2.642773in}{1.599425in}}%
\pgfpathlineto{\pgfqpoint{2.589816in}{1.599050in}}%
\pgfusepath{stroke}%
\end{pgfscope}%
\begin{pgfscope}%
\pgfpathrectangle{\pgfqpoint{0.800000in}{1.400000in}}{\pgfqpoint{2.407767in}{1.544118in}}%
\pgfusepath{clip}%
\pgfsetbuttcap%
\pgfsetroundjoin%
\pgfsetlinewidth{0.501875pt}%
\definecolor{currentstroke}{rgb}{0.271305,0.019942,0.347269}%
\pgfsetstrokecolor{currentstroke}%
\pgfsetdash{}{0pt}%
\pgfpathmoveto{\pgfqpoint{2.589816in}{1.599050in}}%
\pgfpathlineto{\pgfqpoint{2.536857in}{1.599336in}}%
\pgfusepath{stroke}%
\end{pgfscope}%
\begin{pgfscope}%
\pgfpathrectangle{\pgfqpoint{0.800000in}{1.400000in}}{\pgfqpoint{2.407767in}{1.544118in}}%
\pgfusepath{clip}%
\pgfsetbuttcap%
\pgfsetroundjoin%
\pgfsetlinewidth{0.501875pt}%
\definecolor{currentstroke}{rgb}{0.272594,0.025563,0.353093}%
\pgfsetstrokecolor{currentstroke}%
\pgfsetdash{}{0pt}%
\pgfpathmoveto{\pgfqpoint{2.536857in}{1.599336in}}%
\pgfpathlineto{\pgfqpoint{2.483919in}{1.600044in}}%
\pgfusepath{stroke}%
\end{pgfscope}%
\begin{pgfscope}%
\pgfpathrectangle{\pgfqpoint{0.800000in}{1.400000in}}{\pgfqpoint{2.407767in}{1.544118in}}%
\pgfusepath{clip}%
\pgfsetbuttcap%
\pgfsetroundjoin%
\pgfsetlinewidth{0.501875pt}%
\definecolor{currentstroke}{rgb}{0.272594,0.025563,0.353093}%
\pgfsetstrokecolor{currentstroke}%
\pgfsetdash{}{0pt}%
\pgfpathmoveto{\pgfqpoint{2.483919in}{1.600044in}}%
\pgfpathlineto{\pgfqpoint{2.430994in}{1.601226in}}%
\pgfusepath{stroke}%
\end{pgfscope}%
\begin{pgfscope}%
\pgfpathrectangle{\pgfqpoint{0.800000in}{1.400000in}}{\pgfqpoint{2.407767in}{1.544118in}}%
\pgfusepath{clip}%
\pgfsetbuttcap%
\pgfsetroundjoin%
\pgfsetlinewidth{0.501875pt}%
\definecolor{currentstroke}{rgb}{0.273809,0.031497,0.358853}%
\pgfsetstrokecolor{currentstroke}%
\pgfsetdash{}{0pt}%
\pgfpathmoveto{\pgfqpoint{2.430994in}{1.601226in}}%
\pgfpathlineto{\pgfqpoint{2.378209in}{1.603760in}}%
\pgfusepath{stroke}%
\end{pgfscope}%
\begin{pgfscope}%
\pgfpathrectangle{\pgfqpoint{0.800000in}{1.400000in}}{\pgfqpoint{2.407767in}{1.544118in}}%
\pgfusepath{clip}%
\pgfsetbuttcap%
\pgfsetroundjoin%
\pgfsetlinewidth{0.501875pt}%
\definecolor{currentstroke}{rgb}{0.274952,0.037752,0.364543}%
\pgfsetstrokecolor{currentstroke}%
\pgfsetdash{}{0pt}%
\pgfpathmoveto{\pgfqpoint{2.378209in}{1.603760in}}%
\pgfpathlineto{\pgfqpoint{2.325550in}{1.607198in}}%
\pgfusepath{stroke}%
\end{pgfscope}%
\begin{pgfscope}%
\pgfpathrectangle{\pgfqpoint{0.800000in}{1.400000in}}{\pgfqpoint{2.407767in}{1.544118in}}%
\pgfusepath{clip}%
\pgfsetbuttcap%
\pgfsetroundjoin%
\pgfsetlinewidth{0.501875pt}%
\definecolor{currentstroke}{rgb}{0.272594,0.025563,0.353093}%
\pgfsetstrokecolor{currentstroke}%
\pgfsetdash{}{0pt}%
\pgfpathmoveto{\pgfqpoint{2.325550in}{1.607198in}}%
\pgfpathlineto{\pgfqpoint{2.272942in}{1.610986in}}%
\pgfusepath{stroke}%
\end{pgfscope}%
\begin{pgfscope}%
\pgfpathrectangle{\pgfqpoint{0.800000in}{1.400000in}}{\pgfqpoint{2.407767in}{1.544118in}}%
\pgfusepath{clip}%
\pgfsetbuttcap%
\pgfsetroundjoin%
\pgfsetlinewidth{0.501875pt}%
\definecolor{currentstroke}{rgb}{0.273809,0.031497,0.358853}%
\pgfsetstrokecolor{currentstroke}%
\pgfsetdash{}{0pt}%
\pgfpathmoveto{\pgfqpoint{2.272942in}{1.610986in}}%
\pgfpathlineto{\pgfqpoint{2.220604in}{1.616121in}}%
\pgfusepath{stroke}%
\end{pgfscope}%
\begin{pgfscope}%
\pgfpathrectangle{\pgfqpoint{0.800000in}{1.400000in}}{\pgfqpoint{2.407767in}{1.544118in}}%
\pgfusepath{clip}%
\pgfsetbuttcap%
\pgfsetroundjoin%
\pgfsetlinewidth{0.501875pt}%
\definecolor{currentstroke}{rgb}{0.271305,0.019942,0.347269}%
\pgfsetstrokecolor{currentstroke}%
\pgfsetdash{}{0pt}%
\pgfpathmoveto{\pgfqpoint{1.967886in}{1.633520in}}%
\pgfpathlineto{\pgfqpoint{1.949703in}{1.650867in}}%
\pgfusepath{stroke}%
\end{pgfscope}%
\begin{pgfscope}%
\pgfpathrectangle{\pgfqpoint{0.800000in}{1.400000in}}{\pgfqpoint{2.407767in}{1.544118in}}%
\pgfusepath{clip}%
\pgfsetbuttcap%
\pgfsetroundjoin%
\pgfsetlinewidth{0.501875pt}%
\definecolor{currentstroke}{rgb}{0.272594,0.025563,0.353093}%
\pgfsetstrokecolor{currentstroke}%
\pgfsetdash{}{0pt}%
\pgfpathmoveto{\pgfqpoint{1.949703in}{1.650867in}}%
\pgfpathlineto{\pgfqpoint{1.949703in}{1.650867in}}%
\pgfusepath{stroke}%
\end{pgfscope}%
\begin{pgfscope}%
\pgfpathrectangle{\pgfqpoint{0.800000in}{1.400000in}}{\pgfqpoint{2.407767in}{1.544118in}}%
\pgfusepath{clip}%
\pgfsetbuttcap%
\pgfsetroundjoin%
\pgfsetlinewidth{0.501875pt}%
\definecolor{currentstroke}{rgb}{0.272594,0.025563,0.353093}%
\pgfsetstrokecolor{currentstroke}%
\pgfsetdash{}{0pt}%
\pgfpathmoveto{\pgfqpoint{1.949703in}{1.650867in}}%
\pgfpathlineto{\pgfqpoint{1.949703in}{1.650867in}}%
\pgfusepath{stroke}%
\end{pgfscope}%
\begin{pgfscope}%
\pgfpathrectangle{\pgfqpoint{0.800000in}{1.400000in}}{\pgfqpoint{2.407767in}{1.544118in}}%
\pgfusepath{clip}%
\pgfsetbuttcap%
\pgfsetroundjoin%
\pgfsetlinewidth{0.501875pt}%
\definecolor{currentstroke}{rgb}{0.272594,0.025563,0.353093}%
\pgfsetstrokecolor{currentstroke}%
\pgfsetdash{}{0pt}%
\pgfpathmoveto{\pgfqpoint{1.949703in}{1.650867in}}%
\pgfpathlineto{\pgfqpoint{1.929101in}{1.670247in}}%
\pgfusepath{stroke}%
\end{pgfscope}%
\begin{pgfscope}%
\pgfpathrectangle{\pgfqpoint{0.800000in}{1.400000in}}{\pgfqpoint{2.407767in}{1.544118in}}%
\pgfusepath{clip}%
\pgfsetbuttcap%
\pgfsetroundjoin%
\pgfsetlinewidth{0.501875pt}%
\definecolor{currentstroke}{rgb}{0.271305,0.019942,0.347269}%
\pgfsetstrokecolor{currentstroke}%
\pgfsetdash{}{0pt}%
\pgfpathmoveto{\pgfqpoint{1.929101in}{1.670247in}}%
\pgfpathlineto{\pgfqpoint{1.929101in}{1.670247in}}%
\pgfusepath{stroke}%
\end{pgfscope}%
\begin{pgfscope}%
\pgfpathrectangle{\pgfqpoint{0.800000in}{1.400000in}}{\pgfqpoint{2.407767in}{1.544118in}}%
\pgfusepath{clip}%
\pgfsetbuttcap%
\pgfsetroundjoin%
\pgfsetlinewidth{0.501875pt}%
\definecolor{currentstroke}{rgb}{0.271305,0.019942,0.347269}%
\pgfsetstrokecolor{currentstroke}%
\pgfsetdash{}{0pt}%
\pgfpathmoveto{\pgfqpoint{1.929101in}{1.670247in}}%
\pgfpathlineto{\pgfqpoint{1.929101in}{1.670247in}}%
\pgfusepath{stroke}%
\end{pgfscope}%
\begin{pgfscope}%
\pgfpathrectangle{\pgfqpoint{0.800000in}{1.400000in}}{\pgfqpoint{2.407767in}{1.544118in}}%
\pgfusepath{clip}%
\pgfsetbuttcap%
\pgfsetroundjoin%
\pgfsetlinewidth{0.501875pt}%
\definecolor{currentstroke}{rgb}{0.271305,0.019942,0.347269}%
\pgfsetstrokecolor{currentstroke}%
\pgfsetdash{}{0pt}%
\pgfpathmoveto{\pgfqpoint{1.929101in}{1.670247in}}%
\pgfpathlineto{\pgfqpoint{1.914749in}{1.678684in}}%
\pgfusepath{stroke}%
\end{pgfscope}%
\begin{pgfscope}%
\pgfpathrectangle{\pgfqpoint{0.800000in}{1.400000in}}{\pgfqpoint{2.407767in}{1.544118in}}%
\pgfusepath{clip}%
\pgfsetbuttcap%
\pgfsetroundjoin%
\pgfsetlinewidth{0.501875pt}%
\definecolor{currentstroke}{rgb}{0.269944,0.014625,0.341379}%
\pgfsetstrokecolor{currentstroke}%
\pgfsetdash{}{0pt}%
\pgfpathmoveto{\pgfqpoint{1.914749in}{1.678684in}}%
\pgfpathlineto{\pgfqpoint{1.914749in}{1.678684in}}%
\pgfusepath{stroke}%
\end{pgfscope}%
\begin{pgfscope}%
\pgfpathrectangle{\pgfqpoint{0.800000in}{1.400000in}}{\pgfqpoint{2.407767in}{1.544118in}}%
\pgfusepath{clip}%
\pgfsetbuttcap%
\pgfsetroundjoin%
\pgfsetlinewidth{0.501875pt}%
\definecolor{currentstroke}{rgb}{0.269944,0.014625,0.341379}%
\pgfsetstrokecolor{currentstroke}%
\pgfsetdash{}{0pt}%
\pgfpathmoveto{\pgfqpoint{1.914749in}{1.678684in}}%
\pgfpathlineto{\pgfqpoint{1.905105in}{1.681273in}}%
\pgfusepath{stroke}%
\end{pgfscope}%
\begin{pgfscope}%
\pgfpathrectangle{\pgfqpoint{0.800000in}{1.400000in}}{\pgfqpoint{2.407767in}{1.544118in}}%
\pgfusepath{clip}%
\pgfsetbuttcap%
\pgfsetroundjoin%
\pgfsetlinewidth{0.501875pt}%
\definecolor{currentstroke}{rgb}{0.269944,0.014625,0.341379}%
\pgfsetstrokecolor{currentstroke}%
\pgfsetdash{}{0pt}%
\pgfpathmoveto{\pgfqpoint{1.905105in}{1.681273in}}%
\pgfpathlineto{\pgfqpoint{1.905105in}{1.681273in}}%
\pgfusepath{stroke}%
\end{pgfscope}%
\begin{pgfscope}%
\pgfpathrectangle{\pgfqpoint{0.800000in}{1.400000in}}{\pgfqpoint{2.407767in}{1.544118in}}%
\pgfusepath{clip}%
\pgfsetbuttcap%
\pgfsetroundjoin%
\pgfsetlinewidth{0.501875pt}%
\definecolor{currentstroke}{rgb}{0.269944,0.014625,0.341379}%
\pgfsetstrokecolor{currentstroke}%
\pgfsetdash{}{0pt}%
\pgfpathmoveto{\pgfqpoint{1.905105in}{1.681273in}}%
\pgfpathlineto{\pgfqpoint{1.901034in}{1.683528in}}%
\pgfusepath{stroke}%
\end{pgfscope}%
\begin{pgfscope}%
\pgfpathrectangle{\pgfqpoint{0.800000in}{1.400000in}}{\pgfqpoint{2.407767in}{1.544118in}}%
\pgfusepath{clip}%
\pgfsetbuttcap%
\pgfsetroundjoin%
\pgfsetlinewidth{0.501875pt}%
\definecolor{currentstroke}{rgb}{0.271305,0.019942,0.347269}%
\pgfsetstrokecolor{currentstroke}%
\pgfsetdash{}{0pt}%
\pgfpathmoveto{\pgfqpoint{1.901034in}{1.683528in}}%
\pgfpathlineto{\pgfqpoint{1.898881in}{1.686334in}}%
\pgfusepath{stroke}%
\end{pgfscope}%
\begin{pgfscope}%
\pgfpathrectangle{\pgfqpoint{0.800000in}{1.400000in}}{\pgfqpoint{2.407767in}{1.544118in}}%
\pgfusepath{clip}%
\pgfsetbuttcap%
\pgfsetroundjoin%
\pgfsetlinewidth{0.501875pt}%
\definecolor{currentstroke}{rgb}{0.271305,0.019942,0.347269}%
\pgfsetstrokecolor{currentstroke}%
\pgfsetdash{}{0pt}%
\pgfpathmoveto{\pgfqpoint{1.898881in}{1.686334in}}%
\pgfpathlineto{\pgfqpoint{1.897176in}{1.692267in}}%
\pgfusepath{stroke}%
\end{pgfscope}%
\begin{pgfscope}%
\pgfpathrectangle{\pgfqpoint{0.800000in}{1.400000in}}{\pgfqpoint{2.407767in}{1.544118in}}%
\pgfusepath{clip}%
\pgfsetbuttcap%
\pgfsetroundjoin%
\pgfsetlinewidth{0.501875pt}%
\definecolor{currentstroke}{rgb}{0.272594,0.025563,0.353093}%
\pgfsetstrokecolor{currentstroke}%
\pgfsetdash{}{0pt}%
\pgfpathmoveto{\pgfqpoint{1.897176in}{1.692267in}}%
\pgfpathlineto{\pgfqpoint{1.900425in}{1.705576in}}%
\pgfusepath{stroke}%
\end{pgfscope}%
\begin{pgfscope}%
\pgfpathrectangle{\pgfqpoint{0.800000in}{1.400000in}}{\pgfqpoint{2.407767in}{1.544118in}}%
\pgfusepath{clip}%
\pgfsetbuttcap%
\pgfsetroundjoin%
\pgfsetlinewidth{0.501875pt}%
\definecolor{currentstroke}{rgb}{0.272594,0.025563,0.353093}%
\pgfsetstrokecolor{currentstroke}%
\pgfsetdash{}{0pt}%
\pgfpathmoveto{\pgfqpoint{1.900425in}{1.705576in}}%
\pgfpathlineto{\pgfqpoint{1.900425in}{1.705576in}}%
\pgfusepath{stroke}%
\end{pgfscope}%
\begin{pgfscope}%
\pgfpathrectangle{\pgfqpoint{0.800000in}{1.400000in}}{\pgfqpoint{2.407767in}{1.544118in}}%
\pgfusepath{clip}%
\pgfsetbuttcap%
\pgfsetroundjoin%
\pgfsetlinewidth{0.501875pt}%
\definecolor{currentstroke}{rgb}{0.272594,0.025563,0.353093}%
\pgfsetstrokecolor{currentstroke}%
\pgfsetdash{}{0pt}%
\pgfpathmoveto{\pgfqpoint{1.900425in}{1.705576in}}%
\pgfpathlineto{\pgfqpoint{1.909476in}{1.715382in}}%
\pgfusepath{stroke}%
\end{pgfscope}%
\begin{pgfscope}%
\pgfpathrectangle{\pgfqpoint{0.800000in}{1.400000in}}{\pgfqpoint{2.407767in}{1.544118in}}%
\pgfusepath{clip}%
\pgfsetbuttcap%
\pgfsetroundjoin%
\pgfsetlinewidth{0.501875pt}%
\definecolor{currentstroke}{rgb}{0.273809,0.031497,0.358853}%
\pgfsetstrokecolor{currentstroke}%
\pgfsetdash{}{0pt}%
\pgfpathmoveto{\pgfqpoint{1.909476in}{1.715382in}}%
\pgfpathlineto{\pgfqpoint{1.909476in}{1.715382in}}%
\pgfusepath{stroke}%
\end{pgfscope}%
\begin{pgfscope}%
\pgfpathrectangle{\pgfqpoint{0.800000in}{1.400000in}}{\pgfqpoint{2.407767in}{1.544118in}}%
\pgfusepath{clip}%
\pgfsetbuttcap%
\pgfsetroundjoin%
\pgfsetlinewidth{0.501875pt}%
\definecolor{currentstroke}{rgb}{0.273809,0.031497,0.358853}%
\pgfsetstrokecolor{currentstroke}%
\pgfsetdash{}{0pt}%
\pgfpathmoveto{\pgfqpoint{1.909476in}{1.715382in}}%
\pgfpathlineto{\pgfqpoint{1.909476in}{1.715382in}}%
\pgfusepath{stroke}%
\end{pgfscope}%
\begin{pgfscope}%
\pgfpathrectangle{\pgfqpoint{0.800000in}{1.400000in}}{\pgfqpoint{2.407767in}{1.544118in}}%
\pgfusepath{clip}%
\pgfsetbuttcap%
\pgfsetroundjoin%
\pgfsetlinewidth{0.501875pt}%
\definecolor{currentstroke}{rgb}{0.273809,0.031497,0.358853}%
\pgfsetstrokecolor{currentstroke}%
\pgfsetdash{}{0pt}%
\pgfpathmoveto{\pgfqpoint{1.909476in}{1.715382in}}%
\pgfpathlineto{\pgfqpoint{1.914982in}{1.722442in}}%
\pgfusepath{stroke}%
\end{pgfscope}%
\begin{pgfscope}%
\pgfpathrectangle{\pgfqpoint{0.800000in}{1.400000in}}{\pgfqpoint{2.407767in}{1.544118in}}%
\pgfusepath{clip}%
\pgfsetbuttcap%
\pgfsetroundjoin%
\pgfsetlinewidth{0.501875pt}%
\definecolor{currentstroke}{rgb}{0.273809,0.031497,0.358853}%
\pgfsetstrokecolor{currentstroke}%
\pgfsetdash{}{0pt}%
\pgfpathmoveto{\pgfqpoint{1.914982in}{1.722442in}}%
\pgfpathlineto{\pgfqpoint{1.921217in}{1.730118in}}%
\pgfusepath{stroke}%
\end{pgfscope}%
\begin{pgfscope}%
\pgfpathrectangle{\pgfqpoint{0.800000in}{1.400000in}}{\pgfqpoint{2.407767in}{1.544118in}}%
\pgfusepath{clip}%
\pgfsetbuttcap%
\pgfsetroundjoin%
\pgfsetlinewidth{0.501875pt}%
\definecolor{currentstroke}{rgb}{0.273809,0.031497,0.358853}%
\pgfsetstrokecolor{currentstroke}%
\pgfsetdash{}{0pt}%
\pgfpathmoveto{\pgfqpoint{1.921217in}{1.730118in}}%
\pgfpathlineto{\pgfqpoint{1.921217in}{1.730118in}}%
\pgfusepath{stroke}%
\end{pgfscope}%
\begin{pgfscope}%
\pgfpathrectangle{\pgfqpoint{0.800000in}{1.400000in}}{\pgfqpoint{2.407767in}{1.544118in}}%
\pgfusepath{clip}%
\pgfsetbuttcap%
\pgfsetroundjoin%
\pgfsetlinewidth{0.501875pt}%
\definecolor{currentstroke}{rgb}{0.273809,0.031497,0.358853}%
\pgfsetstrokecolor{currentstroke}%
\pgfsetdash{}{0pt}%
\pgfpathmoveto{\pgfqpoint{1.921217in}{1.730118in}}%
\pgfpathlineto{\pgfqpoint{1.921105in}{1.738425in}}%
\pgfusepath{stroke}%
\end{pgfscope}%
\begin{pgfscope}%
\pgfpathrectangle{\pgfqpoint{0.800000in}{1.400000in}}{\pgfqpoint{2.407767in}{1.544118in}}%
\pgfusepath{clip}%
\pgfsetbuttcap%
\pgfsetroundjoin%
\pgfsetlinewidth{0.501875pt}%
\definecolor{currentstroke}{rgb}{0.276022,0.044167,0.370164}%
\pgfsetstrokecolor{currentstroke}%
\pgfsetdash{}{0pt}%
\pgfpathmoveto{\pgfqpoint{1.921105in}{1.738425in}}%
\pgfpathlineto{\pgfqpoint{1.916540in}{1.746638in}}%
\pgfusepath{stroke}%
\end{pgfscope}%
\begin{pgfscope}%
\pgfpathrectangle{\pgfqpoint{0.800000in}{1.400000in}}{\pgfqpoint{2.407767in}{1.544118in}}%
\pgfusepath{clip}%
\pgfsetbuttcap%
\pgfsetroundjoin%
\pgfsetlinewidth{0.501875pt}%
\definecolor{currentstroke}{rgb}{0.277018,0.050344,0.375715}%
\pgfsetstrokecolor{currentstroke}%
\pgfsetdash{}{0pt}%
\pgfpathmoveto{\pgfqpoint{1.916540in}{1.746638in}}%
\pgfpathlineto{\pgfqpoint{1.907850in}{1.765483in}}%
\pgfusepath{stroke}%
\end{pgfscope}%
\begin{pgfscope}%
\pgfpathrectangle{\pgfqpoint{0.800000in}{1.400000in}}{\pgfqpoint{2.407767in}{1.544118in}}%
\pgfusepath{clip}%
\pgfsetbuttcap%
\pgfsetroundjoin%
\pgfsetlinewidth{0.501875pt}%
\definecolor{currentstroke}{rgb}{0.277941,0.056324,0.381191}%
\pgfsetstrokecolor{currentstroke}%
\pgfsetdash{}{0pt}%
\pgfpathmoveto{\pgfqpoint{1.907850in}{1.765483in}}%
\pgfpathlineto{\pgfqpoint{1.894891in}{1.791597in}}%
\pgfusepath{stroke}%
\end{pgfscope}%
\begin{pgfscope}%
\pgfpathrectangle{\pgfqpoint{0.800000in}{1.400000in}}{\pgfqpoint{2.407767in}{1.544118in}}%
\pgfusepath{clip}%
\pgfsetbuttcap%
\pgfsetroundjoin%
\pgfsetlinewidth{0.501875pt}%
\definecolor{currentstroke}{rgb}{0.277941,0.056324,0.381191}%
\pgfsetstrokecolor{currentstroke}%
\pgfsetdash{}{0pt}%
\pgfpathmoveto{\pgfqpoint{1.894891in}{1.791597in}}%
\pgfpathlineto{\pgfqpoint{1.874755in}{1.820945in}}%
\pgfusepath{stroke}%
\end{pgfscope}%
\begin{pgfscope}%
\pgfpathrectangle{\pgfqpoint{0.800000in}{1.400000in}}{\pgfqpoint{2.407767in}{1.544118in}}%
\pgfusepath{clip}%
\pgfsetbuttcap%
\pgfsetroundjoin%
\pgfsetlinewidth{0.501875pt}%
\definecolor{currentstroke}{rgb}{0.280267,0.073417,0.397163}%
\pgfsetstrokecolor{currentstroke}%
\pgfsetdash{}{0pt}%
\pgfpathmoveto{\pgfqpoint{1.874755in}{1.820945in}}%
\pgfpathlineto{\pgfqpoint{1.853245in}{1.850793in}}%
\pgfusepath{stroke}%
\end{pgfscope}%
\begin{pgfscope}%
\pgfpathrectangle{\pgfqpoint{0.800000in}{1.400000in}}{\pgfqpoint{2.407767in}{1.544118in}}%
\pgfusepath{clip}%
\pgfsetbuttcap%
\pgfsetroundjoin%
\pgfsetlinewidth{0.501875pt}%
\definecolor{currentstroke}{rgb}{0.282910,0.105393,0.426902}%
\pgfsetstrokecolor{currentstroke}%
\pgfsetdash{}{0pt}%
\pgfpathmoveto{\pgfqpoint{1.853245in}{1.850793in}}%
\pgfpathlineto{\pgfqpoint{1.829517in}{1.880161in}}%
\pgfusepath{stroke}%
\end{pgfscope}%
\begin{pgfscope}%
\pgfpathrectangle{\pgfqpoint{0.800000in}{1.400000in}}{\pgfqpoint{2.407767in}{1.544118in}}%
\pgfusepath{clip}%
\pgfsetbuttcap%
\pgfsetroundjoin%
\pgfsetlinewidth{0.501875pt}%
\definecolor{currentstroke}{rgb}{0.283072,0.130895,0.449241}%
\pgfsetstrokecolor{currentstroke}%
\pgfsetdash{}{0pt}%
\pgfpathmoveto{\pgfqpoint{1.829517in}{1.880161in}}%
\pgfpathlineto{\pgfqpoint{1.804533in}{1.909517in}}%
\pgfusepath{stroke}%
\end{pgfscope}%
\begin{pgfscope}%
\pgfpathrectangle{\pgfqpoint{0.800000in}{1.400000in}}{\pgfqpoint{2.407767in}{1.544118in}}%
\pgfusepath{clip}%
\pgfsetbuttcap%
\pgfsetroundjoin%
\pgfsetlinewidth{0.501875pt}%
\definecolor{currentstroke}{rgb}{0.282623,0.140926,0.457517}%
\pgfsetstrokecolor{currentstroke}%
\pgfsetdash{}{0pt}%
\pgfpathmoveto{\pgfqpoint{1.804533in}{1.909517in}}%
\pgfpathlineto{\pgfqpoint{1.773831in}{1.936792in}}%
\pgfusepath{stroke}%
\end{pgfscope}%
\begin{pgfscope}%
\pgfpathrectangle{\pgfqpoint{0.800000in}{1.400000in}}{\pgfqpoint{2.407767in}{1.544118in}}%
\pgfusepath{clip}%
\pgfsetbuttcap%
\pgfsetroundjoin%
\pgfsetlinewidth{0.501875pt}%
\definecolor{currentstroke}{rgb}{0.279574,0.170599,0.479997}%
\pgfsetstrokecolor{currentstroke}%
\pgfsetdash{}{0pt}%
\pgfpathmoveto{\pgfqpoint{1.773831in}{1.936792in}}%
\pgfpathlineto{\pgfqpoint{1.741559in}{1.961429in}}%
\pgfusepath{stroke}%
\end{pgfscope}%
\begin{pgfscope}%
\pgfpathrectangle{\pgfqpoint{0.800000in}{1.400000in}}{\pgfqpoint{2.407767in}{1.544118in}}%
\pgfusepath{clip}%
\pgfsetbuttcap%
\pgfsetroundjoin%
\pgfsetlinewidth{0.501875pt}%
\definecolor{currentstroke}{rgb}{0.278826,0.175490,0.483397}%
\pgfsetstrokecolor{currentstroke}%
\pgfsetdash{}{0pt}%
\pgfpathmoveto{\pgfqpoint{1.741559in}{1.961429in}}%
\pgfpathlineto{\pgfqpoint{1.705821in}{1.986381in}}%
\pgfusepath{stroke}%
\end{pgfscope}%
\begin{pgfscope}%
\pgfpathrectangle{\pgfqpoint{0.800000in}{1.400000in}}{\pgfqpoint{2.407767in}{1.544118in}}%
\pgfusepath{clip}%
\pgfsetbuttcap%
\pgfsetroundjoin%
\pgfsetlinewidth{0.501875pt}%
\definecolor{currentstroke}{rgb}{0.270595,0.214069,0.507052}%
\pgfsetstrokecolor{currentstroke}%
\pgfsetdash{}{0pt}%
\pgfpathmoveto{\pgfqpoint{1.705821in}{1.986381in}}%
\pgfpathlineto{\pgfqpoint{1.667982in}{2.009944in}}%
\pgfusepath{stroke}%
\end{pgfscope}%
\begin{pgfscope}%
\pgfpathrectangle{\pgfqpoint{0.800000in}{1.400000in}}{\pgfqpoint{2.407767in}{1.544118in}}%
\pgfusepath{clip}%
\pgfsetbuttcap%
\pgfsetroundjoin%
\pgfsetlinewidth{0.501875pt}%
\definecolor{currentstroke}{rgb}{0.255645,0.260703,0.528312}%
\pgfsetstrokecolor{currentstroke}%
\pgfsetdash{}{0pt}%
\pgfpathmoveto{\pgfqpoint{1.667982in}{2.009944in}}%
\pgfpathlineto{\pgfqpoint{1.629545in}{2.032929in}}%
\pgfusepath{stroke}%
\end{pgfscope}%
\begin{pgfscope}%
\pgfpathrectangle{\pgfqpoint{0.800000in}{1.400000in}}{\pgfqpoint{2.407767in}{1.544118in}}%
\pgfusepath{clip}%
\pgfsetbuttcap%
\pgfsetroundjoin%
\pgfsetlinewidth{0.501875pt}%
\definecolor{currentstroke}{rgb}{0.266580,0.228262,0.514349}%
\pgfsetstrokecolor{currentstroke}%
\pgfsetdash{}{0pt}%
\pgfpathmoveto{\pgfqpoint{1.629545in}{2.032929in}}%
\pgfpathlineto{\pgfqpoint{1.593712in}{2.056669in}}%
\pgfusepath{stroke}%
\end{pgfscope}%
\begin{pgfscope}%
\pgfpathrectangle{\pgfqpoint{0.800000in}{1.400000in}}{\pgfqpoint{2.407767in}{1.544118in}}%
\pgfusepath{clip}%
\pgfsetbuttcap%
\pgfsetroundjoin%
\pgfsetlinewidth{0.501875pt}%
\definecolor{currentstroke}{rgb}{0.277018,0.050344,0.375715}%
\pgfsetstrokecolor{currentstroke}%
\pgfsetdash{}{0pt}%
\pgfpathmoveto{\pgfqpoint{2.434716in}{1.682346in}}%
\pgfpathlineto{\pgfqpoint{2.381781in}{1.683620in}}%
\pgfusepath{stroke}%
\end{pgfscope}%
\begin{pgfscope}%
\pgfpathrectangle{\pgfqpoint{0.800000in}{1.400000in}}{\pgfqpoint{2.407767in}{1.544118in}}%
\pgfusepath{clip}%
\pgfsetbuttcap%
\pgfsetroundjoin%
\pgfsetlinewidth{0.501875pt}%
\definecolor{currentstroke}{rgb}{0.278791,0.062145,0.386592}%
\pgfsetstrokecolor{currentstroke}%
\pgfsetdash{}{0pt}%
\pgfpathmoveto{\pgfqpoint{2.381781in}{1.683620in}}%
\pgfpathlineto{\pgfqpoint{2.328965in}{1.685613in}}%
\pgfusepath{stroke}%
\end{pgfscope}%
\begin{pgfscope}%
\pgfpathrectangle{\pgfqpoint{0.800000in}{1.400000in}}{\pgfqpoint{2.407767in}{1.544118in}}%
\pgfusepath{clip}%
\pgfsetbuttcap%
\pgfsetroundjoin%
\pgfsetlinewidth{0.501875pt}%
\definecolor{currentstroke}{rgb}{0.277018,0.050344,0.375715}%
\pgfsetstrokecolor{currentstroke}%
\pgfsetdash{}{0pt}%
\pgfpathmoveto{\pgfqpoint{2.328965in}{1.685613in}}%
\pgfpathlineto{\pgfqpoint{2.276278in}{1.688736in}}%
\pgfusepath{stroke}%
\end{pgfscope}%
\begin{pgfscope}%
\pgfpathrectangle{\pgfqpoint{0.800000in}{1.400000in}}{\pgfqpoint{2.407767in}{1.544118in}}%
\pgfusepath{clip}%
\pgfsetbuttcap%
\pgfsetroundjoin%
\pgfsetlinewidth{0.501875pt}%
\definecolor{currentstroke}{rgb}{0.277941,0.056324,0.381191}%
\pgfsetstrokecolor{currentstroke}%
\pgfsetdash{}{0pt}%
\pgfpathmoveto{\pgfqpoint{2.276278in}{1.688736in}}%
\pgfpathlineto{\pgfqpoint{2.223794in}{1.693107in}}%
\pgfusepath{stroke}%
\end{pgfscope}%
\begin{pgfscope}%
\pgfpathrectangle{\pgfqpoint{0.800000in}{1.400000in}}{\pgfqpoint{2.407767in}{1.544118in}}%
\pgfusepath{clip}%
\pgfsetbuttcap%
\pgfsetroundjoin%
\pgfsetlinewidth{0.501875pt}%
\definecolor{currentstroke}{rgb}{0.276022,0.044167,0.370164}%
\pgfsetstrokecolor{currentstroke}%
\pgfsetdash{}{0pt}%
\pgfpathmoveto{\pgfqpoint{2.223794in}{1.693107in}}%
\pgfpathlineto{\pgfqpoint{2.171531in}{1.698564in}}%
\pgfusepath{stroke}%
\end{pgfscope}%
\begin{pgfscope}%
\pgfpathrectangle{\pgfqpoint{0.800000in}{1.400000in}}{\pgfqpoint{2.407767in}{1.544118in}}%
\pgfusepath{clip}%
\pgfsetbuttcap%
\pgfsetroundjoin%
\pgfsetlinewidth{0.501875pt}%
\definecolor{currentstroke}{rgb}{0.277018,0.050344,0.375715}%
\pgfsetstrokecolor{currentstroke}%
\pgfsetdash{}{0pt}%
\pgfpathmoveto{\pgfqpoint{2.171531in}{1.698564in}}%
\pgfpathlineto{\pgfqpoint{2.119668in}{1.705292in}}%
\pgfusepath{stroke}%
\end{pgfscope}%
\begin{pgfscope}%
\pgfpathrectangle{\pgfqpoint{0.800000in}{1.400000in}}{\pgfqpoint{2.407767in}{1.544118in}}%
\pgfusepath{clip}%
\pgfsetbuttcap%
\pgfsetroundjoin%
\pgfsetlinewidth{0.501875pt}%
\definecolor{currentstroke}{rgb}{0.276022,0.044167,0.370164}%
\pgfsetstrokecolor{currentstroke}%
\pgfsetdash{}{0pt}%
\pgfpathmoveto{\pgfqpoint{2.119668in}{1.705292in}}%
\pgfpathlineto{\pgfqpoint{2.070027in}{1.716178in}}%
\pgfusepath{stroke}%
\end{pgfscope}%
\begin{pgfscope}%
\pgfpathrectangle{\pgfqpoint{0.800000in}{1.400000in}}{\pgfqpoint{2.407767in}{1.544118in}}%
\pgfusepath{clip}%
\pgfsetbuttcap%
\pgfsetroundjoin%
\pgfsetlinewidth{0.501875pt}%
\definecolor{currentstroke}{rgb}{0.277941,0.056324,0.381191}%
\pgfsetstrokecolor{currentstroke}%
\pgfsetdash{}{0pt}%
\pgfpathmoveto{\pgfqpoint{2.070027in}{1.716178in}}%
\pgfpathlineto{\pgfqpoint{2.026949in}{1.732512in}}%
\pgfusepath{stroke}%
\end{pgfscope}%
\begin{pgfscope}%
\pgfpathrectangle{\pgfqpoint{0.800000in}{1.400000in}}{\pgfqpoint{2.407767in}{1.544118in}}%
\pgfusepath{clip}%
\pgfsetbuttcap%
\pgfsetroundjoin%
\pgfsetlinewidth{0.501875pt}%
\definecolor{currentstroke}{rgb}{0.273809,0.031497,0.358853}%
\pgfsetstrokecolor{currentstroke}%
\pgfsetdash{}{0pt}%
\pgfpathmoveto{\pgfqpoint{2.026949in}{1.732512in}}%
\pgfpathlineto{\pgfqpoint{1.990240in}{1.746454in}}%
\pgfusepath{stroke}%
\end{pgfscope}%
\begin{pgfscope}%
\pgfpathrectangle{\pgfqpoint{0.800000in}{1.400000in}}{\pgfqpoint{2.407767in}{1.544118in}}%
\pgfusepath{clip}%
\pgfsetbuttcap%
\pgfsetroundjoin%
\pgfsetlinewidth{0.501875pt}%
\definecolor{currentstroke}{rgb}{0.272594,0.025563,0.353093}%
\pgfsetstrokecolor{currentstroke}%
\pgfsetdash{}{0pt}%
\pgfpathmoveto{\pgfqpoint{1.990240in}{1.746454in}}%
\pgfpathlineto{\pgfqpoint{1.990240in}{1.746454in}}%
\pgfusepath{stroke}%
\end{pgfscope}%
\begin{pgfscope}%
\pgfpathrectangle{\pgfqpoint{0.800000in}{1.400000in}}{\pgfqpoint{2.407767in}{1.544118in}}%
\pgfusepath{clip}%
\pgfsetbuttcap%
\pgfsetroundjoin%
\pgfsetlinewidth{0.501875pt}%
\definecolor{currentstroke}{rgb}{0.272594,0.025563,0.353093}%
\pgfsetstrokecolor{currentstroke}%
\pgfsetdash{}{0pt}%
\pgfpathmoveto{\pgfqpoint{1.990240in}{1.746454in}}%
\pgfpathlineto{\pgfqpoint{1.974576in}{1.756743in}}%
\pgfusepath{stroke}%
\end{pgfscope}%
\begin{pgfscope}%
\pgfpathrectangle{\pgfqpoint{0.800000in}{1.400000in}}{\pgfqpoint{2.407767in}{1.544118in}}%
\pgfusepath{clip}%
\pgfsetbuttcap%
\pgfsetroundjoin%
\pgfsetlinewidth{0.501875pt}%
\definecolor{currentstroke}{rgb}{0.277018,0.050344,0.375715}%
\pgfsetstrokecolor{currentstroke}%
\pgfsetdash{}{0pt}%
\pgfpathmoveto{\pgfqpoint{1.974576in}{1.756743in}}%
\pgfpathlineto{\pgfqpoint{1.965941in}{1.768344in}}%
\pgfusepath{stroke}%
\end{pgfscope}%
\begin{pgfscope}%
\pgfpathrectangle{\pgfqpoint{0.800000in}{1.400000in}}{\pgfqpoint{2.407767in}{1.544118in}}%
\pgfusepath{clip}%
\pgfsetbuttcap%
\pgfsetroundjoin%
\pgfsetlinewidth{0.501875pt}%
\definecolor{currentstroke}{rgb}{0.274952,0.037752,0.364543}%
\pgfsetstrokecolor{currentstroke}%
\pgfsetdash{}{0pt}%
\pgfpathmoveto{\pgfqpoint{1.965941in}{1.768344in}}%
\pgfpathlineto{\pgfqpoint{1.957927in}{1.782005in}}%
\pgfusepath{stroke}%
\end{pgfscope}%
\begin{pgfscope}%
\pgfpathrectangle{\pgfqpoint{0.800000in}{1.400000in}}{\pgfqpoint{2.407767in}{1.544118in}}%
\pgfusepath{clip}%
\pgfsetbuttcap%
\pgfsetroundjoin%
\pgfsetlinewidth{0.501875pt}%
\definecolor{currentstroke}{rgb}{0.272594,0.025563,0.353093}%
\pgfsetstrokecolor{currentstroke}%
\pgfsetdash{}{0pt}%
\pgfpathmoveto{\pgfqpoint{1.957927in}{1.782005in}}%
\pgfpathlineto{\pgfqpoint{1.937283in}{1.799696in}}%
\pgfusepath{stroke}%
\end{pgfscope}%
\begin{pgfscope}%
\pgfpathrectangle{\pgfqpoint{0.800000in}{1.400000in}}{\pgfqpoint{2.407767in}{1.544118in}}%
\pgfusepath{clip}%
\pgfsetbuttcap%
\pgfsetroundjoin%
\pgfsetlinewidth{0.501875pt}%
\definecolor{currentstroke}{rgb}{0.268510,0.009605,0.335427}%
\pgfsetstrokecolor{currentstroke}%
\pgfsetdash{}{0pt}%
\pgfpathmoveto{\pgfqpoint{2.654046in}{2.658505in}}%
\pgfpathlineto{\pgfqpoint{2.601092in}{2.657920in}}%
\pgfusepath{stroke}%
\end{pgfscope}%
\begin{pgfscope}%
\pgfpathrectangle{\pgfqpoint{0.800000in}{1.400000in}}{\pgfqpoint{2.407767in}{1.544118in}}%
\pgfusepath{clip}%
\pgfsetbuttcap%
\pgfsetroundjoin%
\pgfsetlinewidth{0.501875pt}%
\definecolor{currentstroke}{rgb}{0.271305,0.019942,0.347269}%
\pgfsetstrokecolor{currentstroke}%
\pgfsetdash{}{0pt}%
\pgfpathmoveto{\pgfqpoint{2.601092in}{2.657920in}}%
\pgfpathlineto{\pgfqpoint{2.548124in}{2.657508in}}%
\pgfusepath{stroke}%
\end{pgfscope}%
\begin{pgfscope}%
\pgfpathrectangle{\pgfqpoint{0.800000in}{1.400000in}}{\pgfqpoint{2.407767in}{1.544118in}}%
\pgfusepath{clip}%
\pgfsetbuttcap%
\pgfsetroundjoin%
\pgfsetlinewidth{0.501875pt}%
\definecolor{currentstroke}{rgb}{0.273809,0.031497,0.358853}%
\pgfsetstrokecolor{currentstroke}%
\pgfsetdash{}{0pt}%
\pgfpathmoveto{\pgfqpoint{2.548124in}{2.657508in}}%
\pgfpathlineto{\pgfqpoint{2.495165in}{2.656864in}}%
\pgfusepath{stroke}%
\end{pgfscope}%
\begin{pgfscope}%
\pgfpathrectangle{\pgfqpoint{0.800000in}{1.400000in}}{\pgfqpoint{2.407767in}{1.544118in}}%
\pgfusepath{clip}%
\pgfsetbuttcap%
\pgfsetroundjoin%
\pgfsetlinewidth{0.501875pt}%
\definecolor{currentstroke}{rgb}{0.274952,0.037752,0.364543}%
\pgfsetstrokecolor{currentstroke}%
\pgfsetdash{}{0pt}%
\pgfpathmoveto{\pgfqpoint{2.495165in}{2.656864in}}%
\pgfpathlineto{\pgfqpoint{2.442231in}{2.655746in}}%
\pgfusepath{stroke}%
\end{pgfscope}%
\begin{pgfscope}%
\pgfpathrectangle{\pgfqpoint{0.800000in}{1.400000in}}{\pgfqpoint{2.407767in}{1.544118in}}%
\pgfusepath{clip}%
\pgfsetbuttcap%
\pgfsetroundjoin%
\pgfsetlinewidth{0.501875pt}%
\definecolor{currentstroke}{rgb}{0.274952,0.037752,0.364543}%
\pgfsetstrokecolor{currentstroke}%
\pgfsetdash{}{0pt}%
\pgfpathmoveto{\pgfqpoint{2.442231in}{2.655746in}}%
\pgfpathlineto{\pgfqpoint{2.389310in}{2.654415in}}%
\pgfusepath{stroke}%
\end{pgfscope}%
\begin{pgfscope}%
\pgfpathrectangle{\pgfqpoint{0.800000in}{1.400000in}}{\pgfqpoint{2.407767in}{1.544118in}}%
\pgfusepath{clip}%
\pgfsetbuttcap%
\pgfsetroundjoin%
\pgfsetlinewidth{0.501875pt}%
\definecolor{currentstroke}{rgb}{0.277941,0.056324,0.381191}%
\pgfsetstrokecolor{currentstroke}%
\pgfsetdash{}{0pt}%
\pgfpathmoveto{\pgfqpoint{2.389310in}{2.654415in}}%
\pgfpathlineto{\pgfqpoint{2.336391in}{2.652984in}}%
\pgfusepath{stroke}%
\end{pgfscope}%
\begin{pgfscope}%
\pgfpathrectangle{\pgfqpoint{0.800000in}{1.400000in}}{\pgfqpoint{2.407767in}{1.544118in}}%
\pgfusepath{clip}%
\pgfsetbuttcap%
\pgfsetroundjoin%
\pgfsetlinewidth{0.501875pt}%
\definecolor{currentstroke}{rgb}{0.277018,0.050344,0.375715}%
\pgfsetstrokecolor{currentstroke}%
\pgfsetdash{}{0pt}%
\pgfpathmoveto{\pgfqpoint{2.336391in}{2.652984in}}%
\pgfpathlineto{\pgfqpoint{2.283537in}{2.650984in}}%
\pgfusepath{stroke}%
\end{pgfscope}%
\begin{pgfscope}%
\pgfpathrectangle{\pgfqpoint{0.800000in}{1.400000in}}{\pgfqpoint{2.407767in}{1.544118in}}%
\pgfusepath{clip}%
\pgfsetbuttcap%
\pgfsetroundjoin%
\pgfsetlinewidth{0.501875pt}%
\definecolor{currentstroke}{rgb}{0.271305,0.019942,0.347269}%
\pgfsetstrokecolor{currentstroke}%
\pgfsetdash{}{0pt}%
\pgfpathmoveto{\pgfqpoint{2.643347in}{1.705943in}}%
\pgfpathlineto{\pgfqpoint{2.590388in}{1.706334in}}%
\pgfusepath{stroke}%
\end{pgfscope}%
\begin{pgfscope}%
\pgfpathrectangle{\pgfqpoint{0.800000in}{1.400000in}}{\pgfqpoint{2.407767in}{1.544118in}}%
\pgfusepath{clip}%
\pgfsetbuttcap%
\pgfsetroundjoin%
\pgfsetlinewidth{0.501875pt}%
\definecolor{currentstroke}{rgb}{0.271305,0.019942,0.347269}%
\pgfsetstrokecolor{currentstroke}%
\pgfsetdash{}{0pt}%
\pgfpathmoveto{\pgfqpoint{2.590388in}{1.706334in}}%
\pgfpathlineto{\pgfqpoint{2.537427in}{1.706705in}}%
\pgfusepath{stroke}%
\end{pgfscope}%
\begin{pgfscope}%
\pgfpathrectangle{\pgfqpoint{0.800000in}{1.400000in}}{\pgfqpoint{2.407767in}{1.544118in}}%
\pgfusepath{clip}%
\pgfsetbuttcap%
\pgfsetroundjoin%
\pgfsetlinewidth{0.501875pt}%
\definecolor{currentstroke}{rgb}{0.272594,0.025563,0.353093}%
\pgfsetstrokecolor{currentstroke}%
\pgfsetdash{}{0pt}%
\pgfpathmoveto{\pgfqpoint{2.537427in}{1.706705in}}%
\pgfpathlineto{\pgfqpoint{2.484487in}{1.707849in}}%
\pgfusepath{stroke}%
\end{pgfscope}%
\begin{pgfscope}%
\pgfpathrectangle{\pgfqpoint{0.800000in}{1.400000in}}{\pgfqpoint{2.407767in}{1.544118in}}%
\pgfusepath{clip}%
\pgfsetbuttcap%
\pgfsetroundjoin%
\pgfsetlinewidth{0.501875pt}%
\definecolor{currentstroke}{rgb}{0.276022,0.044167,0.370164}%
\pgfsetstrokecolor{currentstroke}%
\pgfsetdash{}{0pt}%
\pgfpathmoveto{\pgfqpoint{2.484487in}{1.707849in}}%
\pgfpathlineto{\pgfqpoint{2.431571in}{1.709404in}}%
\pgfusepath{stroke}%
\end{pgfscope}%
\begin{pgfscope}%
\pgfpathrectangle{\pgfqpoint{0.800000in}{1.400000in}}{\pgfqpoint{2.407767in}{1.544118in}}%
\pgfusepath{clip}%
\pgfsetbuttcap%
\pgfsetroundjoin%
\pgfsetlinewidth{0.501875pt}%
\definecolor{currentstroke}{rgb}{0.277941,0.056324,0.381191}%
\pgfsetstrokecolor{currentstroke}%
\pgfsetdash{}{0pt}%
\pgfpathmoveto{\pgfqpoint{2.431571in}{1.709404in}}%
\pgfpathlineto{\pgfqpoint{2.378716in}{1.711597in}}%
\pgfusepath{stroke}%
\end{pgfscope}%
\begin{pgfscope}%
\pgfpathrectangle{\pgfqpoint{0.800000in}{1.400000in}}{\pgfqpoint{2.407767in}{1.544118in}}%
\pgfusepath{clip}%
\pgfsetbuttcap%
\pgfsetroundjoin%
\pgfsetlinewidth{0.501875pt}%
\definecolor{currentstroke}{rgb}{0.279566,0.067836,0.391917}%
\pgfsetstrokecolor{currentstroke}%
\pgfsetdash{}{0pt}%
\pgfpathmoveto{\pgfqpoint{2.378716in}{1.711597in}}%
\pgfpathlineto{\pgfqpoint{2.325873in}{1.713901in}}%
\pgfusepath{stroke}%
\end{pgfscope}%
\begin{pgfscope}%
\pgfpathrectangle{\pgfqpoint{0.800000in}{1.400000in}}{\pgfqpoint{2.407767in}{1.544118in}}%
\pgfusepath{clip}%
\pgfsetbuttcap%
\pgfsetroundjoin%
\pgfsetlinewidth{0.501875pt}%
\definecolor{currentstroke}{rgb}{0.277941,0.056324,0.381191}%
\pgfsetstrokecolor{currentstroke}%
\pgfsetdash{}{0pt}%
\pgfpathmoveto{\pgfqpoint{2.325873in}{1.713901in}}%
\pgfpathlineto{\pgfqpoint{2.273044in}{1.716128in}}%
\pgfusepath{stroke}%
\end{pgfscope}%
\begin{pgfscope}%
\pgfpathrectangle{\pgfqpoint{0.800000in}{1.400000in}}{\pgfqpoint{2.407767in}{1.544118in}}%
\pgfusepath{clip}%
\pgfsetbuttcap%
\pgfsetroundjoin%
\pgfsetlinewidth{0.501875pt}%
\definecolor{currentstroke}{rgb}{0.276022,0.044167,0.370164}%
\pgfsetstrokecolor{currentstroke}%
\pgfsetdash{}{0pt}%
\pgfpathmoveto{\pgfqpoint{2.273044in}{1.716128in}}%
\pgfpathlineto{\pgfqpoint{2.220604in}{1.720359in}}%
\pgfusepath{stroke}%
\end{pgfscope}%
\begin{pgfscope}%
\pgfpathrectangle{\pgfqpoint{0.800000in}{1.400000in}}{\pgfqpoint{2.407767in}{1.544118in}}%
\pgfusepath{clip}%
\pgfsetbuttcap%
\pgfsetroundjoin%
\pgfsetlinewidth{0.501875pt}%
\definecolor{currentstroke}{rgb}{0.278791,0.062145,0.386592}%
\pgfsetstrokecolor{currentstroke}%
\pgfsetdash{}{0pt}%
\pgfpathmoveto{\pgfqpoint{2.220604in}{1.720359in}}%
\pgfpathlineto{\pgfqpoint{2.168476in}{1.726290in}}%
\pgfusepath{stroke}%
\end{pgfscope}%
\begin{pgfscope}%
\pgfpathrectangle{\pgfqpoint{0.800000in}{1.400000in}}{\pgfqpoint{2.407767in}{1.544118in}}%
\pgfusepath{clip}%
\pgfsetbuttcap%
\pgfsetroundjoin%
\pgfsetlinewidth{0.501875pt}%
\definecolor{currentstroke}{rgb}{0.269944,0.014625,0.341379}%
\pgfsetstrokecolor{currentstroke}%
\pgfsetdash{}{0pt}%
\pgfpathmoveto{\pgfqpoint{2.654046in}{2.623758in}}%
\pgfpathlineto{\pgfqpoint{2.601089in}{2.623116in}}%
\pgfusepath{stroke}%
\end{pgfscope}%
\begin{pgfscope}%
\pgfpathrectangle{\pgfqpoint{0.800000in}{1.400000in}}{\pgfqpoint{2.407767in}{1.544118in}}%
\pgfusepath{clip}%
\pgfsetbuttcap%
\pgfsetroundjoin%
\pgfsetlinewidth{0.501875pt}%
\definecolor{currentstroke}{rgb}{0.271305,0.019942,0.347269}%
\pgfsetstrokecolor{currentstroke}%
\pgfsetdash{}{0pt}%
\pgfpathmoveto{\pgfqpoint{2.601089in}{2.623116in}}%
\pgfpathlineto{\pgfqpoint{2.548122in}{2.622573in}}%
\pgfusepath{stroke}%
\end{pgfscope}%
\begin{pgfscope}%
\pgfpathrectangle{\pgfqpoint{0.800000in}{1.400000in}}{\pgfqpoint{2.407767in}{1.544118in}}%
\pgfusepath{clip}%
\pgfsetbuttcap%
\pgfsetroundjoin%
\pgfsetlinewidth{0.501875pt}%
\definecolor{currentstroke}{rgb}{0.273809,0.031497,0.358853}%
\pgfsetstrokecolor{currentstroke}%
\pgfsetdash{}{0pt}%
\pgfpathmoveto{\pgfqpoint{2.548122in}{2.622573in}}%
\pgfpathlineto{\pgfqpoint{2.495167in}{2.621777in}}%
\pgfusepath{stroke}%
\end{pgfscope}%
\begin{pgfscope}%
\pgfpathrectangle{\pgfqpoint{0.800000in}{1.400000in}}{\pgfqpoint{2.407767in}{1.544118in}}%
\pgfusepath{clip}%
\pgfsetbuttcap%
\pgfsetroundjoin%
\pgfsetlinewidth{0.501875pt}%
\definecolor{currentstroke}{rgb}{0.276022,0.044167,0.370164}%
\pgfsetstrokecolor{currentstroke}%
\pgfsetdash{}{0pt}%
\pgfpathmoveto{\pgfqpoint{2.495167in}{2.621777in}}%
\pgfpathlineto{\pgfqpoint{2.442225in}{2.620659in}}%
\pgfusepath{stroke}%
\end{pgfscope}%
\begin{pgfscope}%
\pgfpathrectangle{\pgfqpoint{0.800000in}{1.400000in}}{\pgfqpoint{2.407767in}{1.544118in}}%
\pgfusepath{clip}%
\pgfsetbuttcap%
\pgfsetroundjoin%
\pgfsetlinewidth{0.501875pt}%
\definecolor{currentstroke}{rgb}{0.277018,0.050344,0.375715}%
\pgfsetstrokecolor{currentstroke}%
\pgfsetdash{}{0pt}%
\pgfpathmoveto{\pgfqpoint{2.442225in}{2.620659in}}%
\pgfpathlineto{\pgfqpoint{2.389316in}{2.619038in}}%
\pgfusepath{stroke}%
\end{pgfscope}%
\begin{pgfscope}%
\pgfpathrectangle{\pgfqpoint{0.800000in}{1.400000in}}{\pgfqpoint{2.407767in}{1.544118in}}%
\pgfusepath{clip}%
\pgfsetbuttcap%
\pgfsetroundjoin%
\pgfsetlinewidth{0.501875pt}%
\definecolor{currentstroke}{rgb}{0.277941,0.056324,0.381191}%
\pgfsetstrokecolor{currentstroke}%
\pgfsetdash{}{0pt}%
\pgfpathmoveto{\pgfqpoint{2.389316in}{2.619038in}}%
\pgfpathlineto{\pgfqpoint{2.336552in}{2.616175in}}%
\pgfusepath{stroke}%
\end{pgfscope}%
\begin{pgfscope}%
\pgfpathrectangle{\pgfqpoint{0.800000in}{1.400000in}}{\pgfqpoint{2.407767in}{1.544118in}}%
\pgfusepath{clip}%
\pgfsetbuttcap%
\pgfsetroundjoin%
\pgfsetlinewidth{0.501875pt}%
\definecolor{currentstroke}{rgb}{0.279566,0.067836,0.391917}%
\pgfsetstrokecolor{currentstroke}%
\pgfsetdash{}{0pt}%
\pgfpathmoveto{\pgfqpoint{2.336552in}{2.616175in}}%
\pgfpathlineto{\pgfqpoint{2.283831in}{2.612925in}}%
\pgfusepath{stroke}%
\end{pgfscope}%
\begin{pgfscope}%
\pgfpathrectangle{\pgfqpoint{0.800000in}{1.400000in}}{\pgfqpoint{2.407767in}{1.544118in}}%
\pgfusepath{clip}%
\pgfsetbuttcap%
\pgfsetroundjoin%
\pgfsetlinewidth{0.501875pt}%
\definecolor{currentstroke}{rgb}{0.273809,0.031497,0.358853}%
\pgfsetstrokecolor{currentstroke}%
\pgfsetdash{}{0pt}%
\pgfpathmoveto{\pgfqpoint{2.112244in}{2.623758in}}%
\pgfpathlineto{\pgfqpoint{2.065499in}{2.609524in}}%
\pgfusepath{stroke}%
\end{pgfscope}%
\begin{pgfscope}%
\pgfpathrectangle{\pgfqpoint{0.800000in}{1.400000in}}{\pgfqpoint{2.407767in}{1.544118in}}%
\pgfusepath{clip}%
\pgfsetbuttcap%
\pgfsetroundjoin%
\pgfsetlinewidth{0.501875pt}%
\definecolor{currentstroke}{rgb}{0.273809,0.031497,0.358853}%
\pgfsetstrokecolor{currentstroke}%
\pgfsetdash{}{0pt}%
\pgfpathmoveto{\pgfqpoint{2.065499in}{2.609524in}}%
\pgfpathlineto{\pgfqpoint{2.022493in}{2.596680in}}%
\pgfusepath{stroke}%
\end{pgfscope}%
\begin{pgfscope}%
\pgfpathrectangle{\pgfqpoint{0.800000in}{1.400000in}}{\pgfqpoint{2.407767in}{1.544118in}}%
\pgfusepath{clip}%
\pgfsetbuttcap%
\pgfsetroundjoin%
\pgfsetlinewidth{0.501875pt}%
\definecolor{currentstroke}{rgb}{0.274952,0.037752,0.364543}%
\pgfsetstrokecolor{currentstroke}%
\pgfsetdash{}{0pt}%
\pgfpathmoveto{\pgfqpoint{2.022493in}{2.596680in}}%
\pgfpathlineto{\pgfqpoint{1.979976in}{2.581456in}}%
\pgfusepath{stroke}%
\end{pgfscope}%
\begin{pgfscope}%
\pgfpathrectangle{\pgfqpoint{0.800000in}{1.400000in}}{\pgfqpoint{2.407767in}{1.544118in}}%
\pgfusepath{clip}%
\pgfsetbuttcap%
\pgfsetroundjoin%
\pgfsetlinewidth{0.501875pt}%
\definecolor{currentstroke}{rgb}{0.271305,0.019942,0.347269}%
\pgfsetstrokecolor{currentstroke}%
\pgfsetdash{}{0pt}%
\pgfpathmoveto{\pgfqpoint{1.979976in}{2.581456in}}%
\pgfpathlineto{\pgfqpoint{1.979976in}{2.581456in}}%
\pgfusepath{stroke}%
\end{pgfscope}%
\begin{pgfscope}%
\pgfpathrectangle{\pgfqpoint{0.800000in}{1.400000in}}{\pgfqpoint{2.407767in}{1.544118in}}%
\pgfusepath{clip}%
\pgfsetbuttcap%
\pgfsetroundjoin%
\pgfsetlinewidth{0.501875pt}%
\definecolor{currentstroke}{rgb}{0.271305,0.019942,0.347269}%
\pgfsetstrokecolor{currentstroke}%
\pgfsetdash{}{0pt}%
\pgfpathmoveto{\pgfqpoint{1.979976in}{2.581456in}}%
\pgfpathlineto{\pgfqpoint{1.970684in}{2.573095in}}%
\pgfusepath{stroke}%
\end{pgfscope}%
\begin{pgfscope}%
\pgfpathrectangle{\pgfqpoint{0.800000in}{1.400000in}}{\pgfqpoint{2.407767in}{1.544118in}}%
\pgfusepath{clip}%
\pgfsetbuttcap%
\pgfsetroundjoin%
\pgfsetlinewidth{0.501875pt}%
\definecolor{currentstroke}{rgb}{0.269944,0.014625,0.341379}%
\pgfsetstrokecolor{currentstroke}%
\pgfsetdash{}{0pt}%
\pgfpathmoveto{\pgfqpoint{1.970684in}{2.573095in}}%
\pgfpathlineto{\pgfqpoint{1.969399in}{2.564098in}}%
\pgfusepath{stroke}%
\end{pgfscope}%
\begin{pgfscope}%
\pgfpathrectangle{\pgfqpoint{0.800000in}{1.400000in}}{\pgfqpoint{2.407767in}{1.544118in}}%
\pgfusepath{clip}%
\pgfsetbuttcap%
\pgfsetroundjoin%
\pgfsetlinewidth{0.501875pt}%
\definecolor{currentstroke}{rgb}{0.271305,0.019942,0.347269}%
\pgfsetstrokecolor{currentstroke}%
\pgfsetdash{}{0pt}%
\pgfpathmoveto{\pgfqpoint{1.969399in}{2.564098in}}%
\pgfpathlineto{\pgfqpoint{1.970832in}{2.553189in}}%
\pgfusepath{stroke}%
\end{pgfscope}%
\begin{pgfscope}%
\pgfpathrectangle{\pgfqpoint{0.800000in}{1.400000in}}{\pgfqpoint{2.407767in}{1.544118in}}%
\pgfusepath{clip}%
\pgfsetbuttcap%
\pgfsetroundjoin%
\pgfsetlinewidth{0.501875pt}%
\definecolor{currentstroke}{rgb}{0.272594,0.025563,0.353093}%
\pgfsetstrokecolor{currentstroke}%
\pgfsetdash{}{0pt}%
\pgfpathmoveto{\pgfqpoint{1.970832in}{2.553189in}}%
\pgfpathlineto{\pgfqpoint{1.970277in}{2.534283in}}%
\pgfusepath{stroke}%
\end{pgfscope}%
\begin{pgfscope}%
\pgfpathrectangle{\pgfqpoint{0.800000in}{1.400000in}}{\pgfqpoint{2.407767in}{1.544118in}}%
\pgfusepath{clip}%
\pgfsetbuttcap%
\pgfsetroundjoin%
\pgfsetlinewidth{0.501875pt}%
\definecolor{currentstroke}{rgb}{0.273809,0.031497,0.358853}%
\pgfsetstrokecolor{currentstroke}%
\pgfsetdash{}{0pt}%
\pgfpathmoveto{\pgfqpoint{1.970277in}{2.534283in}}%
\pgfpathlineto{\pgfqpoint{1.970277in}{2.534283in}}%
\pgfusepath{stroke}%
\end{pgfscope}%
\begin{pgfscope}%
\pgfpathrectangle{\pgfqpoint{0.800000in}{1.400000in}}{\pgfqpoint{2.407767in}{1.544118in}}%
\pgfusepath{clip}%
\pgfsetbuttcap%
\pgfsetroundjoin%
\pgfsetlinewidth{0.501875pt}%
\definecolor{currentstroke}{rgb}{0.273809,0.031497,0.358853}%
\pgfsetstrokecolor{currentstroke}%
\pgfsetdash{}{0pt}%
\pgfpathmoveto{\pgfqpoint{1.970277in}{2.534283in}}%
\pgfpathlineto{\pgfqpoint{1.970277in}{2.534283in}}%
\pgfusepath{stroke}%
\end{pgfscope}%
\begin{pgfscope}%
\pgfpathrectangle{\pgfqpoint{0.800000in}{1.400000in}}{\pgfqpoint{2.407767in}{1.544118in}}%
\pgfusepath{clip}%
\pgfsetbuttcap%
\pgfsetroundjoin%
\pgfsetlinewidth{0.501875pt}%
\definecolor{currentstroke}{rgb}{0.273809,0.031497,0.358853}%
\pgfsetstrokecolor{currentstroke}%
\pgfsetdash{}{0pt}%
\pgfpathmoveto{\pgfqpoint{1.970277in}{2.534283in}}%
\pgfpathlineto{\pgfqpoint{1.961588in}{2.525258in}}%
\pgfusepath{stroke}%
\end{pgfscope}%
\begin{pgfscope}%
\pgfpathrectangle{\pgfqpoint{0.800000in}{1.400000in}}{\pgfqpoint{2.407767in}{1.544118in}}%
\pgfusepath{clip}%
\pgfsetbuttcap%
\pgfsetroundjoin%
\pgfsetlinewidth{0.501875pt}%
\definecolor{currentstroke}{rgb}{0.276022,0.044167,0.370164}%
\pgfsetstrokecolor{currentstroke}%
\pgfsetdash{}{0pt}%
\pgfpathmoveto{\pgfqpoint{1.961588in}{2.525258in}}%
\pgfpathlineto{\pgfqpoint{1.948337in}{2.517703in}}%
\pgfusepath{stroke}%
\end{pgfscope}%
\begin{pgfscope}%
\pgfpathrectangle{\pgfqpoint{0.800000in}{1.400000in}}{\pgfqpoint{2.407767in}{1.544118in}}%
\pgfusepath{clip}%
\pgfsetbuttcap%
\pgfsetroundjoin%
\pgfsetlinewidth{0.501875pt}%
\definecolor{currentstroke}{rgb}{0.278791,0.062145,0.386592}%
\pgfsetstrokecolor{currentstroke}%
\pgfsetdash{}{0pt}%
\pgfpathmoveto{\pgfqpoint{1.948337in}{2.517703in}}%
\pgfpathlineto{\pgfqpoint{1.948337in}{2.517703in}}%
\pgfusepath{stroke}%
\end{pgfscope}%
\begin{pgfscope}%
\pgfpathrectangle{\pgfqpoint{0.800000in}{1.400000in}}{\pgfqpoint{2.407767in}{1.544118in}}%
\pgfusepath{clip}%
\pgfsetbuttcap%
\pgfsetroundjoin%
\pgfsetlinewidth{0.501875pt}%
\definecolor{currentstroke}{rgb}{0.278791,0.062145,0.386592}%
\pgfsetstrokecolor{currentstroke}%
\pgfsetdash{}{0pt}%
\pgfpathmoveto{\pgfqpoint{1.948337in}{2.517703in}}%
\pgfpathlineto{\pgfqpoint{1.948337in}{2.517703in}}%
\pgfusepath{stroke}%
\end{pgfscope}%
\begin{pgfscope}%
\pgfpathrectangle{\pgfqpoint{0.800000in}{1.400000in}}{\pgfqpoint{2.407767in}{1.544118in}}%
\pgfusepath{clip}%
\pgfsetbuttcap%
\pgfsetroundjoin%
\pgfsetlinewidth{0.501875pt}%
\definecolor{currentstroke}{rgb}{0.278791,0.062145,0.386592}%
\pgfsetstrokecolor{currentstroke}%
\pgfsetdash{}{0pt}%
\pgfpathmoveto{\pgfqpoint{1.948337in}{2.517703in}}%
\pgfpathlineto{\pgfqpoint{1.938713in}{2.508904in}}%
\pgfusepath{stroke}%
\end{pgfscope}%
\begin{pgfscope}%
\pgfpathrectangle{\pgfqpoint{0.800000in}{1.400000in}}{\pgfqpoint{2.407767in}{1.544118in}}%
\pgfusepath{clip}%
\pgfsetbuttcap%
\pgfsetroundjoin%
\pgfsetlinewidth{0.501875pt}%
\definecolor{currentstroke}{rgb}{0.277941,0.056324,0.381191}%
\pgfsetstrokecolor{currentstroke}%
\pgfsetdash{}{0pt}%
\pgfpathmoveto{\pgfqpoint{1.938713in}{2.508904in}}%
\pgfpathlineto{\pgfqpoint{1.931977in}{2.497498in}}%
\pgfusepath{stroke}%
\end{pgfscope}%
\begin{pgfscope}%
\pgfpathrectangle{\pgfqpoint{0.800000in}{1.400000in}}{\pgfqpoint{2.407767in}{1.544118in}}%
\pgfusepath{clip}%
\pgfsetbuttcap%
\pgfsetroundjoin%
\pgfsetlinewidth{0.501875pt}%
\definecolor{currentstroke}{rgb}{0.277018,0.050344,0.375715}%
\pgfsetstrokecolor{currentstroke}%
\pgfsetdash{}{0pt}%
\pgfpathmoveto{\pgfqpoint{1.931977in}{2.497498in}}%
\pgfpathlineto{\pgfqpoint{1.922520in}{2.481677in}}%
\pgfusepath{stroke}%
\end{pgfscope}%
\begin{pgfscope}%
\pgfpathrectangle{\pgfqpoint{0.800000in}{1.400000in}}{\pgfqpoint{2.407767in}{1.544118in}}%
\pgfusepath{clip}%
\pgfsetbuttcap%
\pgfsetroundjoin%
\pgfsetlinewidth{0.501875pt}%
\definecolor{currentstroke}{rgb}{0.278791,0.062145,0.386592}%
\pgfsetstrokecolor{currentstroke}%
\pgfsetdash{}{0pt}%
\pgfpathmoveto{\pgfqpoint{1.922520in}{2.481677in}}%
\pgfpathlineto{\pgfqpoint{1.909915in}{2.458548in}}%
\pgfusepath{stroke}%
\end{pgfscope}%
\begin{pgfscope}%
\pgfpathrectangle{\pgfqpoint{0.800000in}{1.400000in}}{\pgfqpoint{2.407767in}{1.544118in}}%
\pgfusepath{clip}%
\pgfsetbuttcap%
\pgfsetroundjoin%
\pgfsetlinewidth{0.501875pt}%
\definecolor{currentstroke}{rgb}{0.280894,0.078907,0.402329}%
\pgfsetstrokecolor{currentstroke}%
\pgfsetdash{}{0pt}%
\pgfpathmoveto{\pgfqpoint{1.909915in}{2.458548in}}%
\pgfpathlineto{\pgfqpoint{1.909915in}{2.458548in}}%
\pgfusepath{stroke}%
\end{pgfscope}%
\begin{pgfscope}%
\pgfpathrectangle{\pgfqpoint{0.800000in}{1.400000in}}{\pgfqpoint{2.407767in}{1.544118in}}%
\pgfusepath{clip}%
\pgfsetbuttcap%
\pgfsetroundjoin%
\pgfsetlinewidth{0.501875pt}%
\definecolor{currentstroke}{rgb}{0.280894,0.078907,0.402329}%
\pgfsetstrokecolor{currentstroke}%
\pgfsetdash{}{0pt}%
\pgfpathmoveto{\pgfqpoint{1.909915in}{2.458548in}}%
\pgfpathlineto{\pgfqpoint{1.891557in}{2.436001in}}%
\pgfusepath{stroke}%
\end{pgfscope}%
\begin{pgfscope}%
\pgfpathrectangle{\pgfqpoint{0.800000in}{1.400000in}}{\pgfqpoint{2.407767in}{1.544118in}}%
\pgfusepath{clip}%
\pgfsetbuttcap%
\pgfsetroundjoin%
\pgfsetlinewidth{0.501875pt}%
\definecolor{currentstroke}{rgb}{0.281446,0.084320,0.407414}%
\pgfsetstrokecolor{currentstroke}%
\pgfsetdash{}{0pt}%
\pgfpathmoveto{\pgfqpoint{1.891557in}{2.436001in}}%
\pgfpathlineto{\pgfqpoint{1.874701in}{2.416248in}}%
\pgfusepath{stroke}%
\end{pgfscope}%
\begin{pgfscope}%
\pgfpathrectangle{\pgfqpoint{0.800000in}{1.400000in}}{\pgfqpoint{2.407767in}{1.544118in}}%
\pgfusepath{clip}%
\pgfsetbuttcap%
\pgfsetroundjoin%
\pgfsetlinewidth{0.501875pt}%
\definecolor{currentstroke}{rgb}{0.280894,0.078907,0.402329}%
\pgfsetstrokecolor{currentstroke}%
\pgfsetdash{}{0pt}%
\pgfpathmoveto{\pgfqpoint{1.874701in}{2.416248in}}%
\pgfpathlineto{\pgfqpoint{1.848937in}{2.388643in}}%
\pgfusepath{stroke}%
\end{pgfscope}%
\begin{pgfscope}%
\pgfpathrectangle{\pgfqpoint{0.800000in}{1.400000in}}{\pgfqpoint{2.407767in}{1.544118in}}%
\pgfusepath{clip}%
\pgfsetbuttcap%
\pgfsetroundjoin%
\pgfsetlinewidth{0.501875pt}%
\definecolor{currentstroke}{rgb}{0.282327,0.094955,0.417331}%
\pgfsetstrokecolor{currentstroke}%
\pgfsetdash{}{0pt}%
\pgfpathmoveto{\pgfqpoint{1.848937in}{2.388643in}}%
\pgfpathlineto{\pgfqpoint{1.822287in}{2.360596in}}%
\pgfusepath{stroke}%
\end{pgfscope}%
\begin{pgfscope}%
\pgfpathrectangle{\pgfqpoint{0.800000in}{1.400000in}}{\pgfqpoint{2.407767in}{1.544118in}}%
\pgfusepath{clip}%
\pgfsetbuttcap%
\pgfsetroundjoin%
\pgfsetlinewidth{0.501875pt}%
\definecolor{currentstroke}{rgb}{0.283072,0.130895,0.449241}%
\pgfsetstrokecolor{currentstroke}%
\pgfsetdash{}{0pt}%
\pgfpathmoveto{\pgfqpoint{1.822287in}{2.360596in}}%
\pgfpathlineto{\pgfqpoint{1.790243in}{2.334436in}}%
\pgfusepath{stroke}%
\end{pgfscope}%
\begin{pgfscope}%
\pgfpathrectangle{\pgfqpoint{0.800000in}{1.400000in}}{\pgfqpoint{2.407767in}{1.544118in}}%
\pgfusepath{clip}%
\pgfsetbuttcap%
\pgfsetroundjoin%
\pgfsetlinewidth{0.501875pt}%
\definecolor{currentstroke}{rgb}{0.275191,0.194905,0.496005}%
\pgfsetstrokecolor{currentstroke}%
\pgfsetdash{}{0pt}%
\pgfpathmoveto{\pgfqpoint{1.790243in}{2.334436in}}%
\pgfpathlineto{\pgfqpoint{1.758303in}{2.312276in}}%
\pgfusepath{stroke}%
\end{pgfscope}%
\begin{pgfscope}%
\pgfpathrectangle{\pgfqpoint{0.800000in}{1.400000in}}{\pgfqpoint{2.407767in}{1.544118in}}%
\pgfusepath{clip}%
\pgfsetbuttcap%
\pgfsetroundjoin%
\pgfsetlinewidth{0.501875pt}%
\definecolor{currentstroke}{rgb}{0.273006,0.204520,0.501721}%
\pgfsetstrokecolor{currentstroke}%
\pgfsetdash{}{0pt}%
\pgfpathmoveto{\pgfqpoint{1.758303in}{2.312276in}}%
\pgfpathlineto{\pgfqpoint{1.722996in}{2.287151in}}%
\pgfusepath{stroke}%
\end{pgfscope}%
\begin{pgfscope}%
\pgfpathrectangle{\pgfqpoint{0.800000in}{1.400000in}}{\pgfqpoint{2.407767in}{1.544118in}}%
\pgfusepath{clip}%
\pgfsetbuttcap%
\pgfsetroundjoin%
\pgfsetlinewidth{0.501875pt}%
\definecolor{currentstroke}{rgb}{0.273006,0.204520,0.501721}%
\pgfsetstrokecolor{currentstroke}%
\pgfsetdash{}{0pt}%
\pgfpathmoveto{\pgfqpoint{1.722996in}{2.287151in}}%
\pgfpathlineto{\pgfqpoint{1.686564in}{2.262695in}}%
\pgfusepath{stroke}%
\end{pgfscope}%
\begin{pgfscope}%
\pgfpathrectangle{\pgfqpoint{0.800000in}{1.400000in}}{\pgfqpoint{2.407767in}{1.544118in}}%
\pgfusepath{clip}%
\pgfsetbuttcap%
\pgfsetroundjoin%
\pgfsetlinewidth{0.501875pt}%
\definecolor{currentstroke}{rgb}{0.269308,0.218818,0.509577}%
\pgfsetstrokecolor{currentstroke}%
\pgfsetdash{}{0pt}%
\pgfpathmoveto{\pgfqpoint{1.686564in}{2.262695in}}%
\pgfpathlineto{\pgfqpoint{1.647088in}{2.240238in}}%
\pgfusepath{stroke}%
\end{pgfscope}%
\begin{pgfscope}%
\pgfpathrectangle{\pgfqpoint{0.800000in}{1.400000in}}{\pgfqpoint{2.407767in}{1.544118in}}%
\pgfusepath{clip}%
\pgfsetbuttcap%
\pgfsetroundjoin%
\pgfsetlinewidth{0.501875pt}%
\definecolor{currentstroke}{rgb}{0.253935,0.265254,0.529983}%
\pgfsetstrokecolor{currentstroke}%
\pgfsetdash{}{0pt}%
\pgfpathmoveto{\pgfqpoint{1.647088in}{2.240238in}}%
\pgfpathlineto{\pgfqpoint{1.606195in}{2.218798in}}%
\pgfusepath{stroke}%
\end{pgfscope}%
\begin{pgfscope}%
\pgfpathrectangle{\pgfqpoint{0.800000in}{1.400000in}}{\pgfqpoint{2.407767in}{1.544118in}}%
\pgfusepath{clip}%
\pgfsetbuttcap%
\pgfsetroundjoin%
\pgfsetlinewidth{0.501875pt}%
\definecolor{currentstroke}{rgb}{0.265145,0.232956,0.516599}%
\pgfsetstrokecolor{currentstroke}%
\pgfsetdash{}{0pt}%
\pgfpathmoveto{\pgfqpoint{1.606195in}{2.218798in}}%
\pgfpathlineto{\pgfqpoint{1.606195in}{2.218798in}}%
\pgfusepath{stroke}%
\end{pgfscope}%
\begin{pgfscope}%
\pgfpathrectangle{\pgfqpoint{0.800000in}{1.400000in}}{\pgfqpoint{2.407767in}{1.544118in}}%
\pgfusepath{clip}%
\pgfsetbuttcap%
\pgfsetroundjoin%
\pgfsetlinewidth{0.501875pt}%
\definecolor{currentstroke}{rgb}{0.265145,0.232956,0.516599}%
\pgfsetstrokecolor{currentstroke}%
\pgfsetdash{}{0pt}%
\pgfpathmoveto{\pgfqpoint{1.606195in}{2.218798in}}%
\pgfpathlineto{\pgfqpoint{1.587109in}{2.208189in}}%
\pgfusepath{stroke}%
\end{pgfscope}%
\begin{pgfscope}%
\pgfpathrectangle{\pgfqpoint{0.800000in}{1.400000in}}{\pgfqpoint{2.407767in}{1.544118in}}%
\pgfusepath{clip}%
\pgfsetbuttcap%
\pgfsetroundjoin%
\pgfsetlinewidth{0.501875pt}%
\definecolor{currentstroke}{rgb}{0.278791,0.062145,0.386592}%
\pgfsetstrokecolor{currentstroke}%
\pgfsetdash{}{0pt}%
\pgfpathmoveto{\pgfqpoint{1.697247in}{1.720715in}}%
\pgfpathlineto{\pgfqpoint{1.743504in}{1.736239in}}%
\pgfusepath{stroke}%
\end{pgfscope}%
\begin{pgfscope}%
\pgfpathrectangle{\pgfqpoint{0.800000in}{1.400000in}}{\pgfqpoint{2.407767in}{1.544118in}}%
\pgfusepath{clip}%
\pgfsetbuttcap%
\pgfsetroundjoin%
\pgfsetlinewidth{0.501875pt}%
\definecolor{currentstroke}{rgb}{0.277941,0.056324,0.381191}%
\pgfsetstrokecolor{currentstroke}%
\pgfsetdash{}{0pt}%
\pgfpathmoveto{\pgfqpoint{1.743504in}{1.736239in}}%
\pgfpathlineto{\pgfqpoint{1.787163in}{1.755105in}}%
\pgfusepath{stroke}%
\end{pgfscope}%
\begin{pgfscope}%
\pgfpathrectangle{\pgfqpoint{0.800000in}{1.400000in}}{\pgfqpoint{2.407767in}{1.544118in}}%
\pgfusepath{clip}%
\pgfsetbuttcap%
\pgfsetroundjoin%
\pgfsetlinewidth{0.501875pt}%
\definecolor{currentstroke}{rgb}{0.279566,0.067836,0.391917}%
\pgfsetstrokecolor{currentstroke}%
\pgfsetdash{}{0pt}%
\pgfpathmoveto{\pgfqpoint{1.787163in}{1.755105in}}%
\pgfpathlineto{\pgfqpoint{1.787163in}{1.755105in}}%
\pgfusepath{stroke}%
\end{pgfscope}%
\begin{pgfscope}%
\pgfpathrectangle{\pgfqpoint{0.800000in}{1.400000in}}{\pgfqpoint{2.407767in}{1.544118in}}%
\pgfusepath{clip}%
\pgfsetbuttcap%
\pgfsetroundjoin%
\pgfsetlinewidth{0.501875pt}%
\definecolor{currentstroke}{rgb}{0.279566,0.067836,0.391917}%
\pgfsetstrokecolor{currentstroke}%
\pgfsetdash{}{0pt}%
\pgfpathmoveto{\pgfqpoint{1.787163in}{1.755105in}}%
\pgfpathlineto{\pgfqpoint{1.815815in}{1.765916in}}%
\pgfusepath{stroke}%
\end{pgfscope}%
\begin{pgfscope}%
\pgfpathrectangle{\pgfqpoint{0.800000in}{1.400000in}}{\pgfqpoint{2.407767in}{1.544118in}}%
\pgfusepath{clip}%
\pgfsetbuttcap%
\pgfsetroundjoin%
\pgfsetlinewidth{0.501875pt}%
\definecolor{currentstroke}{rgb}{0.277018,0.050344,0.375715}%
\pgfsetstrokecolor{currentstroke}%
\pgfsetdash{}{0pt}%
\pgfpathmoveto{\pgfqpoint{1.815815in}{1.765916in}}%
\pgfpathlineto{\pgfqpoint{1.815815in}{1.765916in}}%
\pgfusepath{stroke}%
\end{pgfscope}%
\begin{pgfscope}%
\pgfpathrectangle{\pgfqpoint{0.800000in}{1.400000in}}{\pgfqpoint{2.407767in}{1.544118in}}%
\pgfusepath{clip}%
\pgfsetbuttcap%
\pgfsetroundjoin%
\pgfsetlinewidth{0.501875pt}%
\definecolor{currentstroke}{rgb}{0.277018,0.050344,0.375715}%
\pgfsetstrokecolor{currentstroke}%
\pgfsetdash{}{0pt}%
\pgfpathmoveto{\pgfqpoint{1.815815in}{1.765916in}}%
\pgfpathlineto{\pgfqpoint{1.815815in}{1.765916in}}%
\pgfusepath{stroke}%
\end{pgfscope}%
\begin{pgfscope}%
\pgfpathrectangle{\pgfqpoint{0.800000in}{1.400000in}}{\pgfqpoint{2.407767in}{1.544118in}}%
\pgfusepath{clip}%
\pgfsetbuttcap%
\pgfsetroundjoin%
\pgfsetlinewidth{0.501875pt}%
\definecolor{currentstroke}{rgb}{0.277018,0.050344,0.375715}%
\pgfsetstrokecolor{currentstroke}%
\pgfsetdash{}{0pt}%
\pgfpathmoveto{\pgfqpoint{1.815815in}{1.765916in}}%
\pgfpathlineto{\pgfqpoint{1.828578in}{1.773582in}}%
\pgfusepath{stroke}%
\end{pgfscope}%
\begin{pgfscope}%
\pgfpathrectangle{\pgfqpoint{0.800000in}{1.400000in}}{\pgfqpoint{2.407767in}{1.544118in}}%
\pgfusepath{clip}%
\pgfsetbuttcap%
\pgfsetroundjoin%
\pgfsetlinewidth{0.501875pt}%
\definecolor{currentstroke}{rgb}{0.281924,0.089666,0.412415}%
\pgfsetstrokecolor{currentstroke}%
\pgfsetdash{}{0pt}%
\pgfpathmoveto{\pgfqpoint{1.828578in}{1.773582in}}%
\pgfpathlineto{\pgfqpoint{1.836913in}{1.783953in}}%
\pgfusepath{stroke}%
\end{pgfscope}%
\begin{pgfscope}%
\pgfpathrectangle{\pgfqpoint{0.800000in}{1.400000in}}{\pgfqpoint{2.407767in}{1.544118in}}%
\pgfusepath{clip}%
\pgfsetbuttcap%
\pgfsetroundjoin%
\pgfsetlinewidth{0.501875pt}%
\definecolor{currentstroke}{rgb}{0.280267,0.073417,0.397163}%
\pgfsetstrokecolor{currentstroke}%
\pgfsetdash{}{0pt}%
\pgfpathmoveto{\pgfqpoint{1.836913in}{1.783953in}}%
\pgfpathlineto{\pgfqpoint{1.849291in}{1.804234in}}%
\pgfusepath{stroke}%
\end{pgfscope}%
\begin{pgfscope}%
\pgfpathrectangle{\pgfqpoint{0.800000in}{1.400000in}}{\pgfqpoint{2.407767in}{1.544118in}}%
\pgfusepath{clip}%
\pgfsetbuttcap%
\pgfsetroundjoin%
\pgfsetlinewidth{0.501875pt}%
\definecolor{currentstroke}{rgb}{0.277018,0.050344,0.375715}%
\pgfsetstrokecolor{currentstroke}%
\pgfsetdash{}{0pt}%
\pgfpathmoveto{\pgfqpoint{1.849291in}{1.804234in}}%
\pgfpathlineto{\pgfqpoint{1.849291in}{1.804234in}}%
\pgfusepath{stroke}%
\end{pgfscope}%
\begin{pgfscope}%
\pgfpathrectangle{\pgfqpoint{0.800000in}{1.400000in}}{\pgfqpoint{2.407767in}{1.544118in}}%
\pgfusepath{clip}%
\pgfsetbuttcap%
\pgfsetroundjoin%
\pgfsetlinewidth{0.501875pt}%
\definecolor{currentstroke}{rgb}{0.277018,0.050344,0.375715}%
\pgfsetstrokecolor{currentstroke}%
\pgfsetdash{}{0pt}%
\pgfpathmoveto{\pgfqpoint{1.849291in}{1.804234in}}%
\pgfpathlineto{\pgfqpoint{1.856594in}{1.827350in}}%
\pgfusepath{stroke}%
\end{pgfscope}%
\begin{pgfscope}%
\pgfpathrectangle{\pgfqpoint{0.800000in}{1.400000in}}{\pgfqpoint{2.407767in}{1.544118in}}%
\pgfusepath{clip}%
\pgfsetbuttcap%
\pgfsetroundjoin%
\pgfsetlinewidth{0.501875pt}%
\definecolor{currentstroke}{rgb}{0.277941,0.056324,0.381191}%
\pgfsetstrokecolor{currentstroke}%
\pgfsetdash{}{0pt}%
\pgfpathmoveto{\pgfqpoint{1.856594in}{1.827350in}}%
\pgfpathlineto{\pgfqpoint{1.856594in}{1.827350in}}%
\pgfusepath{stroke}%
\end{pgfscope}%
\begin{pgfscope}%
\pgfpathrectangle{\pgfqpoint{0.800000in}{1.400000in}}{\pgfqpoint{2.407767in}{1.544118in}}%
\pgfusepath{clip}%
\pgfsetbuttcap%
\pgfsetroundjoin%
\pgfsetlinewidth{0.501875pt}%
\definecolor{currentstroke}{rgb}{0.277018,0.050344,0.375715}%
\pgfsetstrokecolor{currentstroke}%
\pgfsetdash{}{0pt}%
\pgfpathmoveto{\pgfqpoint{2.482774in}{1.738271in}}%
\pgfpathlineto{\pgfqpoint{2.429839in}{1.739539in}}%
\pgfusepath{stroke}%
\end{pgfscope}%
\begin{pgfscope}%
\pgfpathrectangle{\pgfqpoint{0.800000in}{1.400000in}}{\pgfqpoint{2.407767in}{1.544118in}}%
\pgfusepath{clip}%
\pgfsetbuttcap%
\pgfsetroundjoin%
\pgfsetlinewidth{0.501875pt}%
\definecolor{currentstroke}{rgb}{0.279566,0.067836,0.391917}%
\pgfsetstrokecolor{currentstroke}%
\pgfsetdash{}{0pt}%
\pgfpathmoveto{\pgfqpoint{2.429839in}{1.739539in}}%
\pgfpathlineto{\pgfqpoint{2.376967in}{1.741591in}}%
\pgfusepath{stroke}%
\end{pgfscope}%
\begin{pgfscope}%
\pgfpathrectangle{\pgfqpoint{0.800000in}{1.400000in}}{\pgfqpoint{2.407767in}{1.544118in}}%
\pgfusepath{clip}%
\pgfsetbuttcap%
\pgfsetroundjoin%
\pgfsetlinewidth{0.501875pt}%
\definecolor{currentstroke}{rgb}{0.277941,0.056324,0.381191}%
\pgfsetstrokecolor{currentstroke}%
\pgfsetdash{}{0pt}%
\pgfpathmoveto{\pgfqpoint{2.376967in}{1.741591in}}%
\pgfpathlineto{\pgfqpoint{2.324177in}{1.744420in}}%
\pgfusepath{stroke}%
\end{pgfscope}%
\begin{pgfscope}%
\pgfpathrectangle{\pgfqpoint{0.800000in}{1.400000in}}{\pgfqpoint{2.407767in}{1.544118in}}%
\pgfusepath{clip}%
\pgfsetbuttcap%
\pgfsetroundjoin%
\pgfsetlinewidth{0.501875pt}%
\definecolor{currentstroke}{rgb}{0.277018,0.050344,0.375715}%
\pgfsetstrokecolor{currentstroke}%
\pgfsetdash{}{0pt}%
\pgfpathmoveto{\pgfqpoint{2.324177in}{1.744420in}}%
\pgfpathlineto{\pgfqpoint{2.271459in}{1.747527in}}%
\pgfusepath{stroke}%
\end{pgfscope}%
\begin{pgfscope}%
\pgfpathrectangle{\pgfqpoint{0.800000in}{1.400000in}}{\pgfqpoint{2.407767in}{1.544118in}}%
\pgfusepath{clip}%
\pgfsetbuttcap%
\pgfsetroundjoin%
\pgfsetlinewidth{0.501875pt}%
\definecolor{currentstroke}{rgb}{0.277941,0.056324,0.381191}%
\pgfsetstrokecolor{currentstroke}%
\pgfsetdash{}{0pt}%
\pgfpathmoveto{\pgfqpoint{2.271459in}{1.747527in}}%
\pgfpathlineto{\pgfqpoint{2.218761in}{1.750467in}}%
\pgfusepath{stroke}%
\end{pgfscope}%
\begin{pgfscope}%
\pgfpathrectangle{\pgfqpoint{0.800000in}{1.400000in}}{\pgfqpoint{2.407767in}{1.544118in}}%
\pgfusepath{clip}%
\pgfsetbuttcap%
\pgfsetroundjoin%
\pgfsetlinewidth{0.501875pt}%
\definecolor{currentstroke}{rgb}{0.278791,0.062145,0.386592}%
\pgfsetstrokecolor{currentstroke}%
\pgfsetdash{}{0pt}%
\pgfpathmoveto{\pgfqpoint{2.218761in}{1.750467in}}%
\pgfpathlineto{\pgfqpoint{2.166424in}{1.755105in}}%
\pgfusepath{stroke}%
\end{pgfscope}%
\begin{pgfscope}%
\pgfpathrectangle{\pgfqpoint{0.800000in}{1.400000in}}{\pgfqpoint{2.407767in}{1.544118in}}%
\pgfusepath{clip}%
\pgfsetbuttcap%
\pgfsetroundjoin%
\pgfsetlinewidth{0.501875pt}%
\definecolor{currentstroke}{rgb}{0.278791,0.062145,0.386592}%
\pgfsetstrokecolor{currentstroke}%
\pgfsetdash{}{0pt}%
\pgfpathmoveto{\pgfqpoint{2.166424in}{1.755105in}}%
\pgfpathlineto{\pgfqpoint{2.116401in}{1.764719in}}%
\pgfusepath{stroke}%
\end{pgfscope}%
\begin{pgfscope}%
\pgfpathrectangle{\pgfqpoint{0.800000in}{1.400000in}}{\pgfqpoint{2.407767in}{1.544118in}}%
\pgfusepath{clip}%
\pgfsetbuttcap%
\pgfsetroundjoin%
\pgfsetlinewidth{0.501875pt}%
\definecolor{currentstroke}{rgb}{0.277941,0.056324,0.381191}%
\pgfsetstrokecolor{currentstroke}%
\pgfsetdash{}{0pt}%
\pgfpathmoveto{\pgfqpoint{2.116401in}{1.764719in}}%
\pgfpathlineto{\pgfqpoint{2.069723in}{1.776269in}}%
\pgfusepath{stroke}%
\end{pgfscope}%
\begin{pgfscope}%
\pgfpathrectangle{\pgfqpoint{0.800000in}{1.400000in}}{\pgfqpoint{2.407767in}{1.544118in}}%
\pgfusepath{clip}%
\pgfsetbuttcap%
\pgfsetroundjoin%
\pgfsetlinewidth{0.501875pt}%
\definecolor{currentstroke}{rgb}{0.280267,0.073417,0.397163}%
\pgfsetstrokecolor{currentstroke}%
\pgfsetdash{}{0pt}%
\pgfpathmoveto{\pgfqpoint{2.069723in}{1.776269in}}%
\pgfpathlineto{\pgfqpoint{2.021759in}{1.790575in}}%
\pgfusepath{stroke}%
\end{pgfscope}%
\begin{pgfscope}%
\pgfpathrectangle{\pgfqpoint{0.800000in}{1.400000in}}{\pgfqpoint{2.407767in}{1.544118in}}%
\pgfusepath{clip}%
\pgfsetbuttcap%
\pgfsetroundjoin%
\pgfsetlinewidth{0.501875pt}%
\definecolor{currentstroke}{rgb}{0.269944,0.014625,0.341379}%
\pgfsetstrokecolor{currentstroke}%
\pgfsetdash{}{0pt}%
\pgfpathmoveto{\pgfqpoint{2.654046in}{1.789851in}}%
\pgfpathlineto{\pgfqpoint{2.601080in}{1.790280in}}%
\pgfusepath{stroke}%
\end{pgfscope}%
\begin{pgfscope}%
\pgfpathrectangle{\pgfqpoint{0.800000in}{1.400000in}}{\pgfqpoint{2.407767in}{1.544118in}}%
\pgfusepath{clip}%
\pgfsetbuttcap%
\pgfsetroundjoin%
\pgfsetlinewidth{0.501875pt}%
\definecolor{currentstroke}{rgb}{0.271305,0.019942,0.347269}%
\pgfsetstrokecolor{currentstroke}%
\pgfsetdash{}{0pt}%
\pgfpathmoveto{\pgfqpoint{2.601080in}{1.790280in}}%
\pgfpathlineto{\pgfqpoint{2.548120in}{1.790837in}}%
\pgfusepath{stroke}%
\end{pgfscope}%
\begin{pgfscope}%
\pgfpathrectangle{\pgfqpoint{0.800000in}{1.400000in}}{\pgfqpoint{2.407767in}{1.544118in}}%
\pgfusepath{clip}%
\pgfsetbuttcap%
\pgfsetroundjoin%
\pgfsetlinewidth{0.501875pt}%
\definecolor{currentstroke}{rgb}{0.274952,0.037752,0.364543}%
\pgfsetstrokecolor{currentstroke}%
\pgfsetdash{}{0pt}%
\pgfpathmoveto{\pgfqpoint{2.548120in}{1.790837in}}%
\pgfpathlineto{\pgfqpoint{2.495165in}{1.791568in}}%
\pgfusepath{stroke}%
\end{pgfscope}%
\begin{pgfscope}%
\pgfpathrectangle{\pgfqpoint{0.800000in}{1.400000in}}{\pgfqpoint{2.407767in}{1.544118in}}%
\pgfusepath{clip}%
\pgfsetbuttcap%
\pgfsetroundjoin%
\pgfsetlinewidth{0.501875pt}%
\definecolor{currentstroke}{rgb}{0.277941,0.056324,0.381191}%
\pgfsetstrokecolor{currentstroke}%
\pgfsetdash{}{0pt}%
\pgfpathmoveto{\pgfqpoint{2.495165in}{1.791568in}}%
\pgfpathlineto{\pgfqpoint{2.442230in}{1.792595in}}%
\pgfusepath{stroke}%
\end{pgfscope}%
\begin{pgfscope}%
\pgfpathrectangle{\pgfqpoint{0.800000in}{1.400000in}}{\pgfqpoint{2.407767in}{1.544118in}}%
\pgfusepath{clip}%
\pgfsetbuttcap%
\pgfsetroundjoin%
\pgfsetlinewidth{0.501875pt}%
\definecolor{currentstroke}{rgb}{0.279566,0.067836,0.391917}%
\pgfsetstrokecolor{currentstroke}%
\pgfsetdash{}{0pt}%
\pgfpathmoveto{\pgfqpoint{2.442230in}{1.792595in}}%
\pgfpathlineto{\pgfqpoint{2.389306in}{1.793945in}}%
\pgfusepath{stroke}%
\end{pgfscope}%
\begin{pgfscope}%
\pgfpathrectangle{\pgfqpoint{0.800000in}{1.400000in}}{\pgfqpoint{2.407767in}{1.544118in}}%
\pgfusepath{clip}%
\pgfsetbuttcap%
\pgfsetroundjoin%
\pgfsetlinewidth{0.501875pt}%
\definecolor{currentstroke}{rgb}{0.279566,0.067836,0.391917}%
\pgfsetstrokecolor{currentstroke}%
\pgfsetdash{}{0pt}%
\pgfpathmoveto{\pgfqpoint{2.389306in}{1.793945in}}%
\pgfpathlineto{\pgfqpoint{2.336433in}{1.795817in}}%
\pgfusepath{stroke}%
\end{pgfscope}%
\begin{pgfscope}%
\pgfpathrectangle{\pgfqpoint{0.800000in}{1.400000in}}{\pgfqpoint{2.407767in}{1.544118in}}%
\pgfusepath{clip}%
\pgfsetbuttcap%
\pgfsetroundjoin%
\pgfsetlinewidth{0.501875pt}%
\definecolor{currentstroke}{rgb}{0.280894,0.078907,0.402329}%
\pgfsetstrokecolor{currentstroke}%
\pgfsetdash{}{0pt}%
\pgfpathmoveto{\pgfqpoint{2.336433in}{1.795817in}}%
\pgfpathlineto{\pgfqpoint{2.283704in}{1.799003in}}%
\pgfusepath{stroke}%
\end{pgfscope}%
\begin{pgfscope}%
\pgfpathrectangle{\pgfqpoint{0.800000in}{1.400000in}}{\pgfqpoint{2.407767in}{1.544118in}}%
\pgfusepath{clip}%
\pgfsetbuttcap%
\pgfsetroundjoin%
\pgfsetlinewidth{0.501875pt}%
\definecolor{currentstroke}{rgb}{0.280894,0.078907,0.402329}%
\pgfsetstrokecolor{currentstroke}%
\pgfsetdash{}{0pt}%
\pgfpathmoveto{\pgfqpoint{2.283704in}{1.799003in}}%
\pgfpathlineto{\pgfqpoint{2.231090in}{1.802902in}}%
\pgfusepath{stroke}%
\end{pgfscope}%
\begin{pgfscope}%
\pgfpathrectangle{\pgfqpoint{0.800000in}{1.400000in}}{\pgfqpoint{2.407767in}{1.544118in}}%
\pgfusepath{clip}%
\pgfsetbuttcap%
\pgfsetroundjoin%
\pgfsetlinewidth{0.501875pt}%
\definecolor{currentstroke}{rgb}{0.281924,0.089666,0.412415}%
\pgfsetstrokecolor{currentstroke}%
\pgfsetdash{}{0pt}%
\pgfpathmoveto{\pgfqpoint{2.231090in}{1.802902in}}%
\pgfpathlineto{\pgfqpoint{2.178622in}{1.807547in}}%
\pgfusepath{stroke}%
\end{pgfscope}%
\begin{pgfscope}%
\pgfpathrectangle{\pgfqpoint{0.800000in}{1.400000in}}{\pgfqpoint{2.407767in}{1.544118in}}%
\pgfusepath{clip}%
\pgfsetbuttcap%
\pgfsetroundjoin%
\pgfsetlinewidth{0.501875pt}%
\definecolor{currentstroke}{rgb}{0.269944,0.014625,0.341379}%
\pgfsetstrokecolor{currentstroke}%
\pgfsetdash{}{0pt}%
\pgfpathmoveto{\pgfqpoint{2.654046in}{1.824598in}}%
\pgfpathlineto{\pgfqpoint{2.601080in}{1.824912in}}%
\pgfusepath{stroke}%
\end{pgfscope}%
\begin{pgfscope}%
\pgfpathrectangle{\pgfqpoint{0.800000in}{1.400000in}}{\pgfqpoint{2.407767in}{1.544118in}}%
\pgfusepath{clip}%
\pgfsetbuttcap%
\pgfsetroundjoin%
\pgfsetlinewidth{0.501875pt}%
\definecolor{currentstroke}{rgb}{0.273809,0.031497,0.358853}%
\pgfsetstrokecolor{currentstroke}%
\pgfsetdash{}{0pt}%
\pgfpathmoveto{\pgfqpoint{2.601080in}{1.824912in}}%
\pgfpathlineto{\pgfqpoint{2.548120in}{1.825512in}}%
\pgfusepath{stroke}%
\end{pgfscope}%
\begin{pgfscope}%
\pgfpathrectangle{\pgfqpoint{0.800000in}{1.400000in}}{\pgfqpoint{2.407767in}{1.544118in}}%
\pgfusepath{clip}%
\pgfsetbuttcap%
\pgfsetroundjoin%
\pgfsetlinewidth{0.501875pt}%
\definecolor{currentstroke}{rgb}{0.276022,0.044167,0.370164}%
\pgfsetstrokecolor{currentstroke}%
\pgfsetdash{}{0pt}%
\pgfpathmoveto{\pgfqpoint{2.548120in}{1.825512in}}%
\pgfpathlineto{\pgfqpoint{2.495161in}{1.826334in}}%
\pgfusepath{stroke}%
\end{pgfscope}%
\begin{pgfscope}%
\pgfpathrectangle{\pgfqpoint{0.800000in}{1.400000in}}{\pgfqpoint{2.407767in}{1.544118in}}%
\pgfusepath{clip}%
\pgfsetbuttcap%
\pgfsetroundjoin%
\pgfsetlinewidth{0.501875pt}%
\definecolor{currentstroke}{rgb}{0.279566,0.067836,0.391917}%
\pgfsetstrokecolor{currentstroke}%
\pgfsetdash{}{0pt}%
\pgfpathmoveto{\pgfqpoint{2.495161in}{1.826334in}}%
\pgfpathlineto{\pgfqpoint{2.442216in}{1.827486in}}%
\pgfusepath{stroke}%
\end{pgfscope}%
\begin{pgfscope}%
\pgfpathrectangle{\pgfqpoint{0.800000in}{1.400000in}}{\pgfqpoint{2.407767in}{1.544118in}}%
\pgfusepath{clip}%
\pgfsetbuttcap%
\pgfsetroundjoin%
\pgfsetlinewidth{0.501875pt}%
\definecolor{currentstroke}{rgb}{0.280894,0.078907,0.402329}%
\pgfsetstrokecolor{currentstroke}%
\pgfsetdash{}{0pt}%
\pgfpathmoveto{\pgfqpoint{2.442216in}{1.827486in}}%
\pgfpathlineto{\pgfqpoint{2.389297in}{1.829038in}}%
\pgfusepath{stroke}%
\end{pgfscope}%
\begin{pgfscope}%
\pgfpathrectangle{\pgfqpoint{0.800000in}{1.400000in}}{\pgfqpoint{2.407767in}{1.544118in}}%
\pgfusepath{clip}%
\pgfsetbuttcap%
\pgfsetroundjoin%
\pgfsetlinewidth{0.501875pt}%
\definecolor{currentstroke}{rgb}{0.280894,0.078907,0.402329}%
\pgfsetstrokecolor{currentstroke}%
\pgfsetdash{}{0pt}%
\pgfpathmoveto{\pgfqpoint{2.389297in}{1.829038in}}%
\pgfpathlineto{\pgfqpoint{2.336400in}{1.830883in}}%
\pgfusepath{stroke}%
\end{pgfscope}%
\begin{pgfscope}%
\pgfpathrectangle{\pgfqpoint{0.800000in}{1.400000in}}{\pgfqpoint{2.407767in}{1.544118in}}%
\pgfusepath{clip}%
\pgfsetbuttcap%
\pgfsetroundjoin%
\pgfsetlinewidth{0.501875pt}%
\definecolor{currentstroke}{rgb}{0.282910,0.105393,0.426902}%
\pgfsetstrokecolor{currentstroke}%
\pgfsetdash{}{0pt}%
\pgfpathmoveto{\pgfqpoint{2.336400in}{1.830883in}}%
\pgfpathlineto{\pgfqpoint{2.283563in}{1.833251in}}%
\pgfusepath{stroke}%
\end{pgfscope}%
\begin{pgfscope}%
\pgfpathrectangle{\pgfqpoint{0.800000in}{1.400000in}}{\pgfqpoint{2.407767in}{1.544118in}}%
\pgfusepath{clip}%
\pgfsetbuttcap%
\pgfsetroundjoin%
\pgfsetlinewidth{0.501875pt}%
\definecolor{currentstroke}{rgb}{0.283091,0.110553,0.431554}%
\pgfsetstrokecolor{currentstroke}%
\pgfsetdash{}{0pt}%
\pgfpathmoveto{\pgfqpoint{2.283563in}{1.833251in}}%
\pgfpathlineto{\pgfqpoint{2.230800in}{1.836247in}}%
\pgfusepath{stroke}%
\end{pgfscope}%
\begin{pgfscope}%
\pgfpathrectangle{\pgfqpoint{0.800000in}{1.400000in}}{\pgfqpoint{2.407767in}{1.544118in}}%
\pgfusepath{clip}%
\pgfsetbuttcap%
\pgfsetroundjoin%
\pgfsetlinewidth{0.501875pt}%
\definecolor{currentstroke}{rgb}{0.269944,0.014625,0.341379}%
\pgfsetstrokecolor{currentstroke}%
\pgfsetdash{}{0pt}%
\pgfpathmoveto{\pgfqpoint{2.654046in}{1.859344in}}%
\pgfpathlineto{\pgfqpoint{2.601081in}{1.859572in}}%
\pgfusepath{stroke}%
\end{pgfscope}%
\begin{pgfscope}%
\pgfpathrectangle{\pgfqpoint{0.800000in}{1.400000in}}{\pgfqpoint{2.407767in}{1.544118in}}%
\pgfusepath{clip}%
\pgfsetbuttcap%
\pgfsetroundjoin%
\pgfsetlinewidth{0.501875pt}%
\definecolor{currentstroke}{rgb}{0.273809,0.031497,0.358853}%
\pgfsetstrokecolor{currentstroke}%
\pgfsetdash{}{0pt}%
\pgfpathmoveto{\pgfqpoint{2.601081in}{1.859572in}}%
\pgfpathlineto{\pgfqpoint{2.548117in}{1.860277in}}%
\pgfusepath{stroke}%
\end{pgfscope}%
\begin{pgfscope}%
\pgfpathrectangle{\pgfqpoint{0.800000in}{1.400000in}}{\pgfqpoint{2.407767in}{1.544118in}}%
\pgfusepath{clip}%
\pgfsetbuttcap%
\pgfsetroundjoin%
\pgfsetlinewidth{0.501875pt}%
\definecolor{currentstroke}{rgb}{0.277941,0.056324,0.381191}%
\pgfsetstrokecolor{currentstroke}%
\pgfsetdash{}{0pt}%
\pgfpathmoveto{\pgfqpoint{2.548117in}{1.860277in}}%
\pgfpathlineto{\pgfqpoint{2.495163in}{1.861137in}}%
\pgfusepath{stroke}%
\end{pgfscope}%
\begin{pgfscope}%
\pgfpathrectangle{\pgfqpoint{0.800000in}{1.400000in}}{\pgfqpoint{2.407767in}{1.544118in}}%
\pgfusepath{clip}%
\pgfsetbuttcap%
\pgfsetroundjoin%
\pgfsetlinewidth{0.501875pt}%
\definecolor{currentstroke}{rgb}{0.280267,0.073417,0.397163}%
\pgfsetstrokecolor{currentstroke}%
\pgfsetdash{}{0pt}%
\pgfpathmoveto{\pgfqpoint{2.495163in}{1.861137in}}%
\pgfpathlineto{\pgfqpoint{2.442227in}{1.862376in}}%
\pgfusepath{stroke}%
\end{pgfscope}%
\begin{pgfscope}%
\pgfpathrectangle{\pgfqpoint{0.800000in}{1.400000in}}{\pgfqpoint{2.407767in}{1.544118in}}%
\pgfusepath{clip}%
\pgfsetbuttcap%
\pgfsetroundjoin%
\pgfsetlinewidth{0.501875pt}%
\definecolor{currentstroke}{rgb}{0.281924,0.089666,0.412415}%
\pgfsetstrokecolor{currentstroke}%
\pgfsetdash{}{0pt}%
\pgfpathmoveto{\pgfqpoint{2.442227in}{1.862376in}}%
\pgfpathlineto{\pgfqpoint{2.389282in}{1.863417in}}%
\pgfusepath{stroke}%
\end{pgfscope}%
\begin{pgfscope}%
\pgfpathrectangle{\pgfqpoint{0.800000in}{1.400000in}}{\pgfqpoint{2.407767in}{1.544118in}}%
\pgfusepath{clip}%
\pgfsetbuttcap%
\pgfsetroundjoin%
\pgfsetlinewidth{0.501875pt}%
\definecolor{currentstroke}{rgb}{0.282656,0.100196,0.422160}%
\pgfsetstrokecolor{currentstroke}%
\pgfsetdash{}{0pt}%
\pgfpathmoveto{\pgfqpoint{2.389282in}{1.863417in}}%
\pgfpathlineto{\pgfqpoint{2.336376in}{1.864930in}}%
\pgfusepath{stroke}%
\end{pgfscope}%
\begin{pgfscope}%
\pgfpathrectangle{\pgfqpoint{0.800000in}{1.400000in}}{\pgfqpoint{2.407767in}{1.544118in}}%
\pgfusepath{clip}%
\pgfsetbuttcap%
\pgfsetroundjoin%
\pgfsetlinewidth{0.501875pt}%
\definecolor{currentstroke}{rgb}{0.283197,0.115680,0.436115}%
\pgfsetstrokecolor{currentstroke}%
\pgfsetdash{}{0pt}%
\pgfpathmoveto{\pgfqpoint{2.336376in}{1.864930in}}%
\pgfpathlineto{\pgfqpoint{2.283533in}{1.867323in}}%
\pgfusepath{stroke}%
\end{pgfscope}%
\begin{pgfscope}%
\pgfpathrectangle{\pgfqpoint{0.800000in}{1.400000in}}{\pgfqpoint{2.407767in}{1.544118in}}%
\pgfusepath{clip}%
\pgfsetbuttcap%
\pgfsetroundjoin%
\pgfsetlinewidth{0.501875pt}%
\definecolor{currentstroke}{rgb}{0.283187,0.125848,0.444960}%
\pgfsetstrokecolor{currentstroke}%
\pgfsetdash{}{0pt}%
\pgfpathmoveto{\pgfqpoint{2.283533in}{1.867323in}}%
\pgfpathlineto{\pgfqpoint{2.230768in}{1.870292in}}%
\pgfusepath{stroke}%
\end{pgfscope}%
\begin{pgfscope}%
\pgfpathrectangle{\pgfqpoint{0.800000in}{1.400000in}}{\pgfqpoint{2.407767in}{1.544118in}}%
\pgfusepath{clip}%
\pgfsetbuttcap%
\pgfsetroundjoin%
\pgfsetlinewidth{0.501875pt}%
\definecolor{currentstroke}{rgb}{0.283229,0.120777,0.440584}%
\pgfsetstrokecolor{currentstroke}%
\pgfsetdash{}{0pt}%
\pgfpathmoveto{\pgfqpoint{2.230768in}{1.870292in}}%
\pgfpathlineto{\pgfqpoint{2.178202in}{1.874425in}}%
\pgfusepath{stroke}%
\end{pgfscope}%
\begin{pgfscope}%
\pgfpathrectangle{\pgfqpoint{0.800000in}{1.400000in}}{\pgfqpoint{2.407767in}{1.544118in}}%
\pgfusepath{clip}%
\pgfsetbuttcap%
\pgfsetroundjoin%
\pgfsetlinewidth{0.501875pt}%
\definecolor{currentstroke}{rgb}{0.283197,0.115680,0.436115}%
\pgfsetstrokecolor{currentstroke}%
\pgfsetdash{}{0pt}%
\pgfpathmoveto{\pgfqpoint{2.178202in}{1.874425in}}%
\pgfpathlineto{\pgfqpoint{2.125990in}{1.880086in}}%
\pgfusepath{stroke}%
\end{pgfscope}%
\begin{pgfscope}%
\pgfpathrectangle{\pgfqpoint{0.800000in}{1.400000in}}{\pgfqpoint{2.407767in}{1.544118in}}%
\pgfusepath{clip}%
\pgfsetbuttcap%
\pgfsetroundjoin%
\pgfsetlinewidth{0.501875pt}%
\definecolor{currentstroke}{rgb}{0.283197,0.115680,0.436115}%
\pgfsetstrokecolor{currentstroke}%
\pgfsetdash{}{0pt}%
\pgfpathmoveto{\pgfqpoint{2.125990in}{1.880086in}}%
\pgfpathlineto{\pgfqpoint{2.074223in}{1.887245in}}%
\pgfusepath{stroke}%
\end{pgfscope}%
\begin{pgfscope}%
\pgfpathrectangle{\pgfqpoint{0.800000in}{1.400000in}}{\pgfqpoint{2.407767in}{1.544118in}}%
\pgfusepath{clip}%
\pgfsetbuttcap%
\pgfsetroundjoin%
\pgfsetlinewidth{0.501875pt}%
\definecolor{currentstroke}{rgb}{0.283072,0.130895,0.449241}%
\pgfsetstrokecolor{currentstroke}%
\pgfsetdash{}{0pt}%
\pgfpathmoveto{\pgfqpoint{2.074223in}{1.887245in}}%
\pgfpathlineto{\pgfqpoint{2.023044in}{1.895938in}}%
\pgfusepath{stroke}%
\end{pgfscope}%
\begin{pgfscope}%
\pgfpathrectangle{\pgfqpoint{0.800000in}{1.400000in}}{\pgfqpoint{2.407767in}{1.544118in}}%
\pgfusepath{clip}%
\pgfsetbuttcap%
\pgfsetroundjoin%
\pgfsetlinewidth{0.501875pt}%
\definecolor{currentstroke}{rgb}{0.279574,0.170599,0.479997}%
\pgfsetstrokecolor{currentstroke}%
\pgfsetdash{}{0pt}%
\pgfpathmoveto{\pgfqpoint{2.023044in}{1.895938in}}%
\pgfpathlineto{\pgfqpoint{1.973079in}{1.907069in}}%
\pgfusepath{stroke}%
\end{pgfscope}%
\begin{pgfscope}%
\pgfpathrectangle{\pgfqpoint{0.800000in}{1.400000in}}{\pgfqpoint{2.407767in}{1.544118in}}%
\pgfusepath{clip}%
\pgfsetbuttcap%
\pgfsetroundjoin%
\pgfsetlinewidth{0.501875pt}%
\definecolor{currentstroke}{rgb}{0.282290,0.145912,0.461510}%
\pgfsetstrokecolor{currentstroke}%
\pgfsetdash{}{0pt}%
\pgfpathmoveto{\pgfqpoint{1.973079in}{1.907069in}}%
\pgfpathlineto{\pgfqpoint{1.924626in}{1.920689in}}%
\pgfusepath{stroke}%
\end{pgfscope}%
\begin{pgfscope}%
\pgfpathrectangle{\pgfqpoint{0.800000in}{1.400000in}}{\pgfqpoint{2.407767in}{1.544118in}}%
\pgfusepath{clip}%
\pgfsetbuttcap%
\pgfsetroundjoin%
\pgfsetlinewidth{0.501875pt}%
\definecolor{currentstroke}{rgb}{0.282884,0.135920,0.453427}%
\pgfsetstrokecolor{currentstroke}%
\pgfsetdash{}{0pt}%
\pgfpathmoveto{\pgfqpoint{1.924626in}{1.920689in}}%
\pgfpathlineto{\pgfqpoint{1.877290in}{1.935777in}}%
\pgfusepath{stroke}%
\end{pgfscope}%
\begin{pgfscope}%
\pgfpathrectangle{\pgfqpoint{0.800000in}{1.400000in}}{\pgfqpoint{2.407767in}{1.544118in}}%
\pgfusepath{clip}%
\pgfsetbuttcap%
\pgfsetroundjoin%
\pgfsetlinewidth{0.501875pt}%
\definecolor{currentstroke}{rgb}{0.280868,0.160771,0.472899}%
\pgfsetstrokecolor{currentstroke}%
\pgfsetdash{}{0pt}%
\pgfpathmoveto{\pgfqpoint{1.877290in}{1.935777in}}%
\pgfpathlineto{\pgfqpoint{1.832061in}{1.953193in}}%
\pgfusepath{stroke}%
\end{pgfscope}%
\begin{pgfscope}%
\pgfpathrectangle{\pgfqpoint{0.800000in}{1.400000in}}{\pgfqpoint{2.407767in}{1.544118in}}%
\pgfusepath{clip}%
\pgfsetbuttcap%
\pgfsetroundjoin%
\pgfsetlinewidth{0.501875pt}%
\definecolor{currentstroke}{rgb}{0.282290,0.145912,0.461510}%
\pgfsetstrokecolor{currentstroke}%
\pgfsetdash{}{0pt}%
\pgfpathmoveto{\pgfqpoint{1.832061in}{1.953193in}}%
\pgfpathlineto{\pgfqpoint{1.788518in}{1.972522in}}%
\pgfusepath{stroke}%
\end{pgfscope}%
\begin{pgfscope}%
\pgfpathrectangle{\pgfqpoint{0.800000in}{1.400000in}}{\pgfqpoint{2.407767in}{1.544118in}}%
\pgfusepath{clip}%
\pgfsetbuttcap%
\pgfsetroundjoin%
\pgfsetlinewidth{0.501875pt}%
\definecolor{currentstroke}{rgb}{0.277134,0.185228,0.489898}%
\pgfsetstrokecolor{currentstroke}%
\pgfsetdash{}{0pt}%
\pgfpathmoveto{\pgfqpoint{1.788518in}{1.972522in}}%
\pgfpathlineto{\pgfqpoint{1.745019in}{1.991853in}}%
\pgfusepath{stroke}%
\end{pgfscope}%
\begin{pgfscope}%
\pgfpathrectangle{\pgfqpoint{0.800000in}{1.400000in}}{\pgfqpoint{2.407767in}{1.544118in}}%
\pgfusepath{clip}%
\pgfsetbuttcap%
\pgfsetroundjoin%
\pgfsetlinewidth{0.501875pt}%
\definecolor{currentstroke}{rgb}{0.269944,0.014625,0.341379}%
\pgfsetstrokecolor{currentstroke}%
\pgfsetdash{}{0pt}%
\pgfpathmoveto{\pgfqpoint{2.654046in}{1.894090in}}%
\pgfpathlineto{\pgfqpoint{2.601084in}{1.894478in}}%
\pgfusepath{stroke}%
\end{pgfscope}%
\begin{pgfscope}%
\pgfpathrectangle{\pgfqpoint{0.800000in}{1.400000in}}{\pgfqpoint{2.407767in}{1.544118in}}%
\pgfusepath{clip}%
\pgfsetbuttcap%
\pgfsetroundjoin%
\pgfsetlinewidth{0.501875pt}%
\definecolor{currentstroke}{rgb}{0.274952,0.037752,0.364543}%
\pgfsetstrokecolor{currentstroke}%
\pgfsetdash{}{0pt}%
\pgfpathmoveto{\pgfqpoint{2.601084in}{1.894478in}}%
\pgfpathlineto{\pgfqpoint{2.548114in}{1.894837in}}%
\pgfusepath{stroke}%
\end{pgfscope}%
\begin{pgfscope}%
\pgfpathrectangle{\pgfqpoint{0.800000in}{1.400000in}}{\pgfqpoint{2.407767in}{1.544118in}}%
\pgfusepath{clip}%
\pgfsetbuttcap%
\pgfsetroundjoin%
\pgfsetlinewidth{0.501875pt}%
\definecolor{currentstroke}{rgb}{0.277941,0.056324,0.381191}%
\pgfsetstrokecolor{currentstroke}%
\pgfsetdash{}{0pt}%
\pgfpathmoveto{\pgfqpoint{2.548114in}{1.894837in}}%
\pgfpathlineto{\pgfqpoint{2.495156in}{1.895640in}}%
\pgfusepath{stroke}%
\end{pgfscope}%
\begin{pgfscope}%
\pgfpathrectangle{\pgfqpoint{0.800000in}{1.400000in}}{\pgfqpoint{2.407767in}{1.544118in}}%
\pgfusepath{clip}%
\pgfsetbuttcap%
\pgfsetroundjoin%
\pgfsetlinewidth{0.501875pt}%
\definecolor{currentstroke}{rgb}{0.281446,0.084320,0.407414}%
\pgfsetstrokecolor{currentstroke}%
\pgfsetdash{}{0pt}%
\pgfpathmoveto{\pgfqpoint{2.495156in}{1.895640in}}%
\pgfpathlineto{\pgfqpoint{2.442203in}{1.896606in}}%
\pgfusepath{stroke}%
\end{pgfscope}%
\begin{pgfscope}%
\pgfpathrectangle{\pgfqpoint{0.800000in}{1.400000in}}{\pgfqpoint{2.407767in}{1.544118in}}%
\pgfusepath{clip}%
\pgfsetbuttcap%
\pgfsetroundjoin%
\pgfsetlinewidth{0.501875pt}%
\definecolor{currentstroke}{rgb}{0.282656,0.100196,0.422160}%
\pgfsetstrokecolor{currentstroke}%
\pgfsetdash{}{0pt}%
\pgfpathmoveto{\pgfqpoint{2.442203in}{1.896606in}}%
\pgfpathlineto{\pgfqpoint{2.389255in}{1.897687in}}%
\pgfusepath{stroke}%
\end{pgfscope}%
\begin{pgfscope}%
\pgfpathrectangle{\pgfqpoint{0.800000in}{1.400000in}}{\pgfqpoint{2.407767in}{1.544118in}}%
\pgfusepath{clip}%
\pgfsetbuttcap%
\pgfsetroundjoin%
\pgfsetlinewidth{0.501875pt}%
\definecolor{currentstroke}{rgb}{0.282327,0.094955,0.417331}%
\pgfsetstrokecolor{currentstroke}%
\pgfsetdash{}{0pt}%
\pgfpathmoveto{\pgfqpoint{2.389255in}{1.897687in}}%
\pgfpathlineto{\pgfqpoint{2.336332in}{1.899162in}}%
\pgfusepath{stroke}%
\end{pgfscope}%
\begin{pgfscope}%
\pgfpathrectangle{\pgfqpoint{0.800000in}{1.400000in}}{\pgfqpoint{2.407767in}{1.544118in}}%
\pgfusepath{clip}%
\pgfsetbuttcap%
\pgfsetroundjoin%
\pgfsetlinewidth{0.501875pt}%
\definecolor{currentstroke}{rgb}{0.282884,0.135920,0.453427}%
\pgfsetstrokecolor{currentstroke}%
\pgfsetdash{}{0pt}%
\pgfpathmoveto{\pgfqpoint{2.336332in}{1.899162in}}%
\pgfpathlineto{\pgfqpoint{2.283485in}{1.901469in}}%
\pgfusepath{stroke}%
\end{pgfscope}%
\begin{pgfscope}%
\pgfpathrectangle{\pgfqpoint{0.800000in}{1.400000in}}{\pgfqpoint{2.407767in}{1.544118in}}%
\pgfusepath{clip}%
\pgfsetbuttcap%
\pgfsetroundjoin%
\pgfsetlinewidth{0.501875pt}%
\definecolor{currentstroke}{rgb}{0.282623,0.140926,0.457517}%
\pgfsetstrokecolor{currentstroke}%
\pgfsetdash{}{0pt}%
\pgfpathmoveto{\pgfqpoint{2.283485in}{1.901469in}}%
\pgfpathlineto{\pgfqpoint{2.230725in}{1.904512in}}%
\pgfusepath{stroke}%
\end{pgfscope}%
\begin{pgfscope}%
\pgfpathrectangle{\pgfqpoint{0.800000in}{1.400000in}}{\pgfqpoint{2.407767in}{1.544118in}}%
\pgfusepath{clip}%
\pgfsetbuttcap%
\pgfsetroundjoin%
\pgfsetlinewidth{0.501875pt}%
\definecolor{currentstroke}{rgb}{0.282623,0.140926,0.457517}%
\pgfsetstrokecolor{currentstroke}%
\pgfsetdash{}{0pt}%
\pgfpathmoveto{\pgfqpoint{2.230725in}{1.904512in}}%
\pgfpathlineto{\pgfqpoint{2.178069in}{1.908191in}}%
\pgfusepath{stroke}%
\end{pgfscope}%
\begin{pgfscope}%
\pgfpathrectangle{\pgfqpoint{0.800000in}{1.400000in}}{\pgfqpoint{2.407767in}{1.544118in}}%
\pgfusepath{clip}%
\pgfsetbuttcap%
\pgfsetroundjoin%
\pgfsetlinewidth{0.501875pt}%
\definecolor{currentstroke}{rgb}{0.280868,0.160771,0.472899}%
\pgfsetstrokecolor{currentstroke}%
\pgfsetdash{}{0pt}%
\pgfpathmoveto{\pgfqpoint{2.178069in}{1.908191in}}%
\pgfpathlineto{\pgfqpoint{2.125667in}{1.913095in}}%
\pgfusepath{stroke}%
\end{pgfscope}%
\begin{pgfscope}%
\pgfpathrectangle{\pgfqpoint{0.800000in}{1.400000in}}{\pgfqpoint{2.407767in}{1.544118in}}%
\pgfusepath{clip}%
\pgfsetbuttcap%
\pgfsetroundjoin%
\pgfsetlinewidth{0.501875pt}%
\definecolor{currentstroke}{rgb}{0.278012,0.180367,0.486697}%
\pgfsetstrokecolor{currentstroke}%
\pgfsetdash{}{0pt}%
\pgfpathmoveto{\pgfqpoint{2.125667in}{1.913095in}}%
\pgfpathlineto{\pgfqpoint{2.073629in}{1.919417in}}%
\pgfusepath{stroke}%
\end{pgfscope}%
\begin{pgfscope}%
\pgfpathrectangle{\pgfqpoint{0.800000in}{1.400000in}}{\pgfqpoint{2.407767in}{1.544118in}}%
\pgfusepath{clip}%
\pgfsetbuttcap%
\pgfsetroundjoin%
\pgfsetlinewidth{0.501875pt}%
\definecolor{currentstroke}{rgb}{0.278012,0.180367,0.486697}%
\pgfsetstrokecolor{currentstroke}%
\pgfsetdash{}{0pt}%
\pgfpathmoveto{\pgfqpoint{2.073629in}{1.919417in}}%
\pgfpathlineto{\pgfqpoint{2.021929in}{1.926806in}}%
\pgfusepath{stroke}%
\end{pgfscope}%
\begin{pgfscope}%
\pgfpathrectangle{\pgfqpoint{0.800000in}{1.400000in}}{\pgfqpoint{2.407767in}{1.544118in}}%
\pgfusepath{clip}%
\pgfsetbuttcap%
\pgfsetroundjoin%
\pgfsetlinewidth{0.501875pt}%
\definecolor{currentstroke}{rgb}{0.269944,0.014625,0.341379}%
\pgfsetstrokecolor{currentstroke}%
\pgfsetdash{}{0pt}%
\pgfpathmoveto{\pgfqpoint{2.654046in}{1.928836in}}%
\pgfpathlineto{\pgfqpoint{2.601076in}{1.929290in}}%
\pgfusepath{stroke}%
\end{pgfscope}%
\begin{pgfscope}%
\pgfpathrectangle{\pgfqpoint{0.800000in}{1.400000in}}{\pgfqpoint{2.407767in}{1.544118in}}%
\pgfusepath{clip}%
\pgfsetbuttcap%
\pgfsetroundjoin%
\pgfsetlinewidth{0.501875pt}%
\definecolor{currentstroke}{rgb}{0.274952,0.037752,0.364543}%
\pgfsetstrokecolor{currentstroke}%
\pgfsetdash{}{0pt}%
\pgfpathmoveto{\pgfqpoint{2.601076in}{1.929290in}}%
\pgfpathlineto{\pgfqpoint{2.548111in}{1.929935in}}%
\pgfusepath{stroke}%
\end{pgfscope}%
\begin{pgfscope}%
\pgfpathrectangle{\pgfqpoint{0.800000in}{1.400000in}}{\pgfqpoint{2.407767in}{1.544118in}}%
\pgfusepath{clip}%
\pgfsetbuttcap%
\pgfsetroundjoin%
\pgfsetlinewidth{0.501875pt}%
\definecolor{currentstroke}{rgb}{0.277941,0.056324,0.381191}%
\pgfsetstrokecolor{currentstroke}%
\pgfsetdash{}{0pt}%
\pgfpathmoveto{\pgfqpoint{2.548111in}{1.929935in}}%
\pgfpathlineto{\pgfqpoint{2.495146in}{1.930579in}}%
\pgfusepath{stroke}%
\end{pgfscope}%
\begin{pgfscope}%
\pgfpathrectangle{\pgfqpoint{0.800000in}{1.400000in}}{\pgfqpoint{2.407767in}{1.544118in}}%
\pgfusepath{clip}%
\pgfsetbuttcap%
\pgfsetroundjoin%
\pgfsetlinewidth{0.501875pt}%
\definecolor{currentstroke}{rgb}{0.282327,0.094955,0.417331}%
\pgfsetstrokecolor{currentstroke}%
\pgfsetdash{}{0pt}%
\pgfpathmoveto{\pgfqpoint{2.495146in}{1.930579in}}%
\pgfpathlineto{\pgfqpoint{2.442182in}{1.931284in}}%
\pgfusepath{stroke}%
\end{pgfscope}%
\begin{pgfscope}%
\pgfpathrectangle{\pgfqpoint{0.800000in}{1.400000in}}{\pgfqpoint{2.407767in}{1.544118in}}%
\pgfusepath{clip}%
\pgfsetbuttcap%
\pgfsetroundjoin%
\pgfsetlinewidth{0.501875pt}%
\definecolor{currentstroke}{rgb}{0.283197,0.115680,0.436115}%
\pgfsetstrokecolor{currentstroke}%
\pgfsetdash{}{0pt}%
\pgfpathmoveto{\pgfqpoint{2.442182in}{1.931284in}}%
\pgfpathlineto{\pgfqpoint{2.389229in}{1.932272in}}%
\pgfusepath{stroke}%
\end{pgfscope}%
\begin{pgfscope}%
\pgfpathrectangle{\pgfqpoint{0.800000in}{1.400000in}}{\pgfqpoint{2.407767in}{1.544118in}}%
\pgfusepath{clip}%
\pgfsetbuttcap%
\pgfsetroundjoin%
\pgfsetlinewidth{0.501875pt}%
\definecolor{currentstroke}{rgb}{0.283187,0.125848,0.444960}%
\pgfsetstrokecolor{currentstroke}%
\pgfsetdash{}{0pt}%
\pgfpathmoveto{\pgfqpoint{2.389229in}{1.932272in}}%
\pgfpathlineto{\pgfqpoint{2.336296in}{1.933620in}}%
\pgfusepath{stroke}%
\end{pgfscope}%
\begin{pgfscope}%
\pgfpathrectangle{\pgfqpoint{0.800000in}{1.400000in}}{\pgfqpoint{2.407767in}{1.544118in}}%
\pgfusepath{clip}%
\pgfsetbuttcap%
\pgfsetroundjoin%
\pgfsetlinewidth{0.501875pt}%
\definecolor{currentstroke}{rgb}{0.279574,0.170599,0.479997}%
\pgfsetstrokecolor{currentstroke}%
\pgfsetdash{}{0pt}%
\pgfpathmoveto{\pgfqpoint{2.336296in}{1.933620in}}%
\pgfpathlineto{\pgfqpoint{2.283402in}{1.935473in}}%
\pgfusepath{stroke}%
\end{pgfscope}%
\begin{pgfscope}%
\pgfpathrectangle{\pgfqpoint{0.800000in}{1.400000in}}{\pgfqpoint{2.407767in}{1.544118in}}%
\pgfusepath{clip}%
\pgfsetbuttcap%
\pgfsetroundjoin%
\pgfsetlinewidth{0.501875pt}%
\definecolor{currentstroke}{rgb}{0.278012,0.180367,0.486697}%
\pgfsetstrokecolor{currentstroke}%
\pgfsetdash{}{0pt}%
\pgfpathmoveto{\pgfqpoint{2.283402in}{1.935473in}}%
\pgfpathlineto{\pgfqpoint{2.230550in}{1.937769in}}%
\pgfusepath{stroke}%
\end{pgfscope}%
\begin{pgfscope}%
\pgfpathrectangle{\pgfqpoint{0.800000in}{1.400000in}}{\pgfqpoint{2.407767in}{1.544118in}}%
\pgfusepath{clip}%
\pgfsetbuttcap%
\pgfsetroundjoin%
\pgfsetlinewidth{0.501875pt}%
\definecolor{currentstroke}{rgb}{0.277134,0.185228,0.489898}%
\pgfsetstrokecolor{currentstroke}%
\pgfsetdash{}{0pt}%
\pgfpathmoveto{\pgfqpoint{2.230550in}{1.937769in}}%
\pgfpathlineto{\pgfqpoint{2.177759in}{1.940540in}}%
\pgfusepath{stroke}%
\end{pgfscope}%
\begin{pgfscope}%
\pgfpathrectangle{\pgfqpoint{0.800000in}{1.400000in}}{\pgfqpoint{2.407767in}{1.544118in}}%
\pgfusepath{clip}%
\pgfsetbuttcap%
\pgfsetroundjoin%
\pgfsetlinewidth{0.501875pt}%
\definecolor{currentstroke}{rgb}{0.271305,0.019942,0.347269}%
\pgfsetstrokecolor{currentstroke}%
\pgfsetdash{}{0pt}%
\pgfpathmoveto{\pgfqpoint{2.654046in}{1.963582in}}%
\pgfpathlineto{\pgfqpoint{2.601079in}{1.963946in}}%
\pgfusepath{stroke}%
\end{pgfscope}%
\begin{pgfscope}%
\pgfpathrectangle{\pgfqpoint{0.800000in}{1.400000in}}{\pgfqpoint{2.407767in}{1.544118in}}%
\pgfusepath{clip}%
\pgfsetbuttcap%
\pgfsetroundjoin%
\pgfsetlinewidth{0.501875pt}%
\definecolor{currentstroke}{rgb}{0.273809,0.031497,0.358853}%
\pgfsetstrokecolor{currentstroke}%
\pgfsetdash{}{0pt}%
\pgfpathmoveto{\pgfqpoint{2.601079in}{1.963946in}}%
\pgfpathlineto{\pgfqpoint{2.548115in}{1.964492in}}%
\pgfusepath{stroke}%
\end{pgfscope}%
\begin{pgfscope}%
\pgfpathrectangle{\pgfqpoint{0.800000in}{1.400000in}}{\pgfqpoint{2.407767in}{1.544118in}}%
\pgfusepath{clip}%
\pgfsetbuttcap%
\pgfsetroundjoin%
\pgfsetlinewidth{0.501875pt}%
\definecolor{currentstroke}{rgb}{0.278791,0.062145,0.386592}%
\pgfsetstrokecolor{currentstroke}%
\pgfsetdash{}{0pt}%
\pgfpathmoveto{\pgfqpoint{2.548115in}{1.964492in}}%
\pgfpathlineto{\pgfqpoint{2.495143in}{1.964857in}}%
\pgfusepath{stroke}%
\end{pgfscope}%
\begin{pgfscope}%
\pgfpathrectangle{\pgfqpoint{0.800000in}{1.400000in}}{\pgfqpoint{2.407767in}{1.544118in}}%
\pgfusepath{clip}%
\pgfsetbuttcap%
\pgfsetroundjoin%
\pgfsetlinewidth{0.501875pt}%
\definecolor{currentstroke}{rgb}{0.282327,0.094955,0.417331}%
\pgfsetstrokecolor{currentstroke}%
\pgfsetdash{}{0pt}%
\pgfpathmoveto{\pgfqpoint{2.495143in}{1.964857in}}%
\pgfpathlineto{\pgfqpoint{2.442171in}{1.965213in}}%
\pgfusepath{stroke}%
\end{pgfscope}%
\begin{pgfscope}%
\pgfpathrectangle{\pgfqpoint{0.800000in}{1.400000in}}{\pgfqpoint{2.407767in}{1.544118in}}%
\pgfusepath{clip}%
\pgfsetbuttcap%
\pgfsetroundjoin%
\pgfsetlinewidth{0.501875pt}%
\definecolor{currentstroke}{rgb}{0.283197,0.115680,0.436115}%
\pgfsetstrokecolor{currentstroke}%
\pgfsetdash{}{0pt}%
\pgfpathmoveto{\pgfqpoint{2.442171in}{1.965213in}}%
\pgfpathlineto{\pgfqpoint{2.389205in}{1.965835in}}%
\pgfusepath{stroke}%
\end{pgfscope}%
\begin{pgfscope}%
\pgfpathrectangle{\pgfqpoint{0.800000in}{1.400000in}}{\pgfqpoint{2.407767in}{1.544118in}}%
\pgfusepath{clip}%
\pgfsetbuttcap%
\pgfsetroundjoin%
\pgfsetlinewidth{0.501875pt}%
\definecolor{currentstroke}{rgb}{0.282623,0.140926,0.457517}%
\pgfsetstrokecolor{currentstroke}%
\pgfsetdash{}{0pt}%
\pgfpathmoveto{\pgfqpoint{2.389205in}{1.965835in}}%
\pgfpathlineto{\pgfqpoint{2.336254in}{1.966834in}}%
\pgfusepath{stroke}%
\end{pgfscope}%
\begin{pgfscope}%
\pgfpathrectangle{\pgfqpoint{0.800000in}{1.400000in}}{\pgfqpoint{2.407767in}{1.544118in}}%
\pgfusepath{clip}%
\pgfsetbuttcap%
\pgfsetroundjoin%
\pgfsetlinewidth{0.501875pt}%
\definecolor{currentstroke}{rgb}{0.281412,0.155834,0.469201}%
\pgfsetstrokecolor{currentstroke}%
\pgfsetdash{}{0pt}%
\pgfpathmoveto{\pgfqpoint{2.336254in}{1.966834in}}%
\pgfpathlineto{\pgfqpoint{2.283317in}{1.968133in}}%
\pgfusepath{stroke}%
\end{pgfscope}%
\begin{pgfscope}%
\pgfpathrectangle{\pgfqpoint{0.800000in}{1.400000in}}{\pgfqpoint{2.407767in}{1.544118in}}%
\pgfusepath{clip}%
\pgfsetbuttcap%
\pgfsetroundjoin%
\pgfsetlinewidth{0.501875pt}%
\definecolor{currentstroke}{rgb}{0.276194,0.190074,0.493001}%
\pgfsetstrokecolor{currentstroke}%
\pgfsetdash{}{0pt}%
\pgfpathmoveto{\pgfqpoint{2.283317in}{1.968133in}}%
\pgfpathlineto{\pgfqpoint{2.230402in}{1.969728in}}%
\pgfusepath{stroke}%
\end{pgfscope}%
\begin{pgfscope}%
\pgfpathrectangle{\pgfqpoint{0.800000in}{1.400000in}}{\pgfqpoint{2.407767in}{1.544118in}}%
\pgfusepath{clip}%
\pgfsetbuttcap%
\pgfsetroundjoin%
\pgfsetlinewidth{0.501875pt}%
\definecolor{currentstroke}{rgb}{0.271828,0.209303,0.504434}%
\pgfsetstrokecolor{currentstroke}%
\pgfsetdash{}{0pt}%
\pgfpathmoveto{\pgfqpoint{2.230402in}{1.969728in}}%
\pgfpathlineto{\pgfqpoint{2.177530in}{1.971827in}}%
\pgfusepath{stroke}%
\end{pgfscope}%
\begin{pgfscope}%
\pgfpathrectangle{\pgfqpoint{0.800000in}{1.400000in}}{\pgfqpoint{2.407767in}{1.544118in}}%
\pgfusepath{clip}%
\pgfsetbuttcap%
\pgfsetroundjoin%
\pgfsetlinewidth{0.501875pt}%
\definecolor{currentstroke}{rgb}{0.274128,0.199721,0.498911}%
\pgfsetstrokecolor{currentstroke}%
\pgfsetdash{}{0pt}%
\pgfpathmoveto{\pgfqpoint{2.177530in}{1.971827in}}%
\pgfpathlineto{\pgfqpoint{2.124722in}{1.974505in}}%
\pgfusepath{stroke}%
\end{pgfscope}%
\begin{pgfscope}%
\pgfpathrectangle{\pgfqpoint{0.800000in}{1.400000in}}{\pgfqpoint{2.407767in}{1.544118in}}%
\pgfusepath{clip}%
\pgfsetbuttcap%
\pgfsetroundjoin%
\pgfsetlinewidth{0.501875pt}%
\definecolor{currentstroke}{rgb}{0.270595,0.214069,0.507052}%
\pgfsetstrokecolor{currentstroke}%
\pgfsetdash{}{0pt}%
\pgfpathmoveto{\pgfqpoint{2.124722in}{1.974505in}}%
\pgfpathlineto{\pgfqpoint{2.071994in}{1.977770in}}%
\pgfusepath{stroke}%
\end{pgfscope}%
\begin{pgfscope}%
\pgfpathrectangle{\pgfqpoint{0.800000in}{1.400000in}}{\pgfqpoint{2.407767in}{1.544118in}}%
\pgfusepath{clip}%
\pgfsetbuttcap%
\pgfsetroundjoin%
\pgfsetlinewidth{0.501875pt}%
\definecolor{currentstroke}{rgb}{0.276194,0.190074,0.493001}%
\pgfsetstrokecolor{currentstroke}%
\pgfsetdash{}{0pt}%
\pgfpathmoveto{\pgfqpoint{2.071994in}{1.977770in}}%
\pgfpathlineto{\pgfqpoint{2.019514in}{1.982293in}}%
\pgfusepath{stroke}%
\end{pgfscope}%
\begin{pgfscope}%
\pgfpathrectangle{\pgfqpoint{0.800000in}{1.400000in}}{\pgfqpoint{2.407767in}{1.544118in}}%
\pgfusepath{clip}%
\pgfsetbuttcap%
\pgfsetroundjoin%
\pgfsetlinewidth{0.501875pt}%
\definecolor{currentstroke}{rgb}{0.278826,0.175490,0.483397}%
\pgfsetstrokecolor{currentstroke}%
\pgfsetdash{}{0pt}%
\pgfpathmoveto{\pgfqpoint{2.019514in}{1.982293in}}%
\pgfpathlineto{\pgfqpoint{1.967450in}{1.988497in}}%
\pgfusepath{stroke}%
\end{pgfscope}%
\begin{pgfscope}%
\pgfpathrectangle{\pgfqpoint{0.800000in}{1.400000in}}{\pgfqpoint{2.407767in}{1.544118in}}%
\pgfusepath{clip}%
\pgfsetbuttcap%
\pgfsetroundjoin%
\pgfsetlinewidth{0.501875pt}%
\definecolor{currentstroke}{rgb}{0.269308,0.218818,0.509577}%
\pgfsetstrokecolor{currentstroke}%
\pgfsetdash{}{0pt}%
\pgfpathmoveto{\pgfqpoint{1.967450in}{1.988497in}}%
\pgfpathlineto{\pgfqpoint{1.915788in}{1.995950in}}%
\pgfusepath{stroke}%
\end{pgfscope}%
\begin{pgfscope}%
\pgfpathrectangle{\pgfqpoint{0.800000in}{1.400000in}}{\pgfqpoint{2.407767in}{1.544118in}}%
\pgfusepath{clip}%
\pgfsetbuttcap%
\pgfsetroundjoin%
\pgfsetlinewidth{0.501875pt}%
\definecolor{currentstroke}{rgb}{0.263663,0.237631,0.518762}%
\pgfsetstrokecolor{currentstroke}%
\pgfsetdash{}{0pt}%
\pgfpathmoveto{\pgfqpoint{1.915788in}{1.995950in}}%
\pgfpathlineto{\pgfqpoint{1.864921in}{2.005314in}}%
\pgfusepath{stroke}%
\end{pgfscope}%
\begin{pgfscope}%
\pgfpathrectangle{\pgfqpoint{0.800000in}{1.400000in}}{\pgfqpoint{2.407767in}{1.544118in}}%
\pgfusepath{clip}%
\pgfsetbuttcap%
\pgfsetroundjoin%
\pgfsetlinewidth{0.501875pt}%
\definecolor{currentstroke}{rgb}{0.270595,0.214069,0.507052}%
\pgfsetstrokecolor{currentstroke}%
\pgfsetdash{}{0pt}%
\pgfpathmoveto{\pgfqpoint{1.864921in}{2.005314in}}%
\pgfpathlineto{\pgfqpoint{1.814559in}{2.015802in}}%
\pgfusepath{stroke}%
\end{pgfscope}%
\begin{pgfscope}%
\pgfpathrectangle{\pgfqpoint{0.800000in}{1.400000in}}{\pgfqpoint{2.407767in}{1.544118in}}%
\pgfusepath{clip}%
\pgfsetbuttcap%
\pgfsetroundjoin%
\pgfsetlinewidth{0.501875pt}%
\definecolor{currentstroke}{rgb}{0.260571,0.246922,0.522828}%
\pgfsetstrokecolor{currentstroke}%
\pgfsetdash{}{0pt}%
\pgfpathmoveto{\pgfqpoint{1.814559in}{2.015802in}}%
\pgfpathlineto{\pgfqpoint{1.764682in}{2.027170in}}%
\pgfusepath{stroke}%
\end{pgfscope}%
\begin{pgfscope}%
\pgfpathrectangle{\pgfqpoint{0.800000in}{1.400000in}}{\pgfqpoint{2.407767in}{1.544118in}}%
\pgfusepath{clip}%
\pgfsetbuttcap%
\pgfsetroundjoin%
\pgfsetlinewidth{0.501875pt}%
\definecolor{currentstroke}{rgb}{0.260571,0.246922,0.522828}%
\pgfsetstrokecolor{currentstroke}%
\pgfsetdash{}{0pt}%
\pgfpathmoveto{\pgfqpoint{1.764682in}{2.027170in}}%
\pgfpathlineto{\pgfqpoint{1.715564in}{2.039843in}}%
\pgfusepath{stroke}%
\end{pgfscope}%
\begin{pgfscope}%
\pgfpathrectangle{\pgfqpoint{0.800000in}{1.400000in}}{\pgfqpoint{2.407767in}{1.544118in}}%
\pgfusepath{clip}%
\pgfsetbuttcap%
\pgfsetroundjoin%
\pgfsetlinewidth{0.501875pt}%
\definecolor{currentstroke}{rgb}{0.257322,0.256130,0.526563}%
\pgfsetstrokecolor{currentstroke}%
\pgfsetdash{}{0pt}%
\pgfpathmoveto{\pgfqpoint{1.715564in}{2.039843in}}%
\pgfpathlineto{\pgfqpoint{1.667127in}{2.053535in}}%
\pgfusepath{stroke}%
\end{pgfscope}%
\begin{pgfscope}%
\pgfpathrectangle{\pgfqpoint{0.800000in}{1.400000in}}{\pgfqpoint{2.407767in}{1.544118in}}%
\pgfusepath{clip}%
\pgfsetbuttcap%
\pgfsetroundjoin%
\pgfsetlinewidth{0.501875pt}%
\definecolor{currentstroke}{rgb}{0.253935,0.265254,0.529983}%
\pgfsetstrokecolor{currentstroke}%
\pgfsetdash{}{0pt}%
\pgfpathmoveto{\pgfqpoint{1.667127in}{2.053535in}}%
\pgfpathlineto{\pgfqpoint{1.619507in}{2.068261in}}%
\pgfusepath{stroke}%
\end{pgfscope}%
\begin{pgfscope}%
\pgfpathrectangle{\pgfqpoint{0.800000in}{1.400000in}}{\pgfqpoint{2.407767in}{1.544118in}}%
\pgfusepath{clip}%
\pgfsetbuttcap%
\pgfsetroundjoin%
\pgfsetlinewidth{0.501875pt}%
\definecolor{currentstroke}{rgb}{0.269944,0.014625,0.341379}%
\pgfsetstrokecolor{currentstroke}%
\pgfsetdash{}{0pt}%
\pgfpathmoveto{\pgfqpoint{2.654046in}{1.998328in}}%
\pgfpathlineto{\pgfqpoint{2.601075in}{1.998504in}}%
\pgfusepath{stroke}%
\end{pgfscope}%
\begin{pgfscope}%
\pgfpathrectangle{\pgfqpoint{0.800000in}{1.400000in}}{\pgfqpoint{2.407767in}{1.544118in}}%
\pgfusepath{clip}%
\pgfsetbuttcap%
\pgfsetroundjoin%
\pgfsetlinewidth{0.501875pt}%
\definecolor{currentstroke}{rgb}{0.274952,0.037752,0.364543}%
\pgfsetstrokecolor{currentstroke}%
\pgfsetdash{}{0pt}%
\pgfpathmoveto{\pgfqpoint{2.601075in}{1.998504in}}%
\pgfpathlineto{\pgfqpoint{2.548099in}{1.998676in}}%
\pgfusepath{stroke}%
\end{pgfscope}%
\begin{pgfscope}%
\pgfpathrectangle{\pgfqpoint{0.800000in}{1.400000in}}{\pgfqpoint{2.407767in}{1.544118in}}%
\pgfusepath{clip}%
\pgfsetbuttcap%
\pgfsetroundjoin%
\pgfsetlinewidth{0.501875pt}%
\definecolor{currentstroke}{rgb}{0.279566,0.067836,0.391917}%
\pgfsetstrokecolor{currentstroke}%
\pgfsetdash{}{0pt}%
\pgfpathmoveto{\pgfqpoint{2.548099in}{1.998676in}}%
\pgfpathlineto{\pgfqpoint{2.495127in}{1.999024in}}%
\pgfusepath{stroke}%
\end{pgfscope}%
\begin{pgfscope}%
\pgfpathrectangle{\pgfqpoint{0.800000in}{1.400000in}}{\pgfqpoint{2.407767in}{1.544118in}}%
\pgfusepath{clip}%
\pgfsetbuttcap%
\pgfsetroundjoin%
\pgfsetlinewidth{0.501875pt}%
\definecolor{currentstroke}{rgb}{0.282327,0.094955,0.417331}%
\pgfsetstrokecolor{currentstroke}%
\pgfsetdash{}{0pt}%
\pgfpathmoveto{\pgfqpoint{2.495127in}{1.999024in}}%
\pgfpathlineto{\pgfqpoint{2.442157in}{1.999516in}}%
\pgfusepath{stroke}%
\end{pgfscope}%
\begin{pgfscope}%
\pgfpathrectangle{\pgfqpoint{0.800000in}{1.400000in}}{\pgfqpoint{2.407767in}{1.544118in}}%
\pgfusepath{clip}%
\pgfsetbuttcap%
\pgfsetroundjoin%
\pgfsetlinewidth{0.501875pt}%
\definecolor{currentstroke}{rgb}{0.283197,0.115680,0.436115}%
\pgfsetstrokecolor{currentstroke}%
\pgfsetdash{}{0pt}%
\pgfpathmoveto{\pgfqpoint{2.442157in}{1.999516in}}%
\pgfpathlineto{\pgfqpoint{2.389189in}{2.000101in}}%
\pgfusepath{stroke}%
\end{pgfscope}%
\begin{pgfscope}%
\pgfpathrectangle{\pgfqpoint{0.800000in}{1.400000in}}{\pgfqpoint{2.407767in}{1.544118in}}%
\pgfusepath{clip}%
\pgfsetbuttcap%
\pgfsetroundjoin%
\pgfsetlinewidth{0.501875pt}%
\definecolor{currentstroke}{rgb}{0.280868,0.160771,0.472899}%
\pgfsetstrokecolor{currentstroke}%
\pgfsetdash{}{0pt}%
\pgfpathmoveto{\pgfqpoint{2.389189in}{2.000101in}}%
\pgfpathlineto{\pgfqpoint{2.336227in}{2.000881in}}%
\pgfusepath{stroke}%
\end{pgfscope}%
\begin{pgfscope}%
\pgfpathrectangle{\pgfqpoint{0.800000in}{1.400000in}}{\pgfqpoint{2.407767in}{1.544118in}}%
\pgfusepath{clip}%
\pgfsetbuttcap%
\pgfsetroundjoin%
\pgfsetlinewidth{0.501875pt}%
\definecolor{currentstroke}{rgb}{0.278012,0.180367,0.486697}%
\pgfsetstrokecolor{currentstroke}%
\pgfsetdash{}{0pt}%
\pgfpathmoveto{\pgfqpoint{2.336227in}{2.000881in}}%
\pgfpathlineto{\pgfqpoint{2.283275in}{2.001895in}}%
\pgfusepath{stroke}%
\end{pgfscope}%
\begin{pgfscope}%
\pgfpathrectangle{\pgfqpoint{0.800000in}{1.400000in}}{\pgfqpoint{2.407767in}{1.544118in}}%
\pgfusepath{clip}%
\pgfsetbuttcap%
\pgfsetroundjoin%
\pgfsetlinewidth{0.501875pt}%
\definecolor{currentstroke}{rgb}{0.273006,0.204520,0.501721}%
\pgfsetstrokecolor{currentstroke}%
\pgfsetdash{}{0pt}%
\pgfpathmoveto{\pgfqpoint{2.283275in}{2.001895in}}%
\pgfpathlineto{\pgfqpoint{2.230347in}{2.003308in}}%
\pgfusepath{stroke}%
\end{pgfscope}%
\begin{pgfscope}%
\pgfpathrectangle{\pgfqpoint{0.800000in}{1.400000in}}{\pgfqpoint{2.407767in}{1.544118in}}%
\pgfusepath{clip}%
\pgfsetbuttcap%
\pgfsetroundjoin%
\pgfsetlinewidth{0.501875pt}%
\definecolor{currentstroke}{rgb}{0.262138,0.242286,0.520837}%
\pgfsetstrokecolor{currentstroke}%
\pgfsetdash{}{0pt}%
\pgfpathmoveto{\pgfqpoint{2.230347in}{2.003308in}}%
\pgfpathlineto{\pgfqpoint{2.177450in}{2.005146in}}%
\pgfusepath{stroke}%
\end{pgfscope}%
\begin{pgfscope}%
\pgfpathrectangle{\pgfqpoint{0.800000in}{1.400000in}}{\pgfqpoint{2.407767in}{1.544118in}}%
\pgfusepath{clip}%
\pgfsetbuttcap%
\pgfsetroundjoin%
\pgfsetlinewidth{0.501875pt}%
\definecolor{currentstroke}{rgb}{0.257322,0.256130,0.526563}%
\pgfsetstrokecolor{currentstroke}%
\pgfsetdash{}{0pt}%
\pgfpathmoveto{\pgfqpoint{2.177450in}{2.005146in}}%
\pgfpathlineto{\pgfqpoint{2.124590in}{2.007384in}}%
\pgfusepath{stroke}%
\end{pgfscope}%
\begin{pgfscope}%
\pgfpathrectangle{\pgfqpoint{0.800000in}{1.400000in}}{\pgfqpoint{2.407767in}{1.544118in}}%
\pgfusepath{clip}%
\pgfsetbuttcap%
\pgfsetroundjoin%
\pgfsetlinewidth{0.501875pt}%
\definecolor{currentstroke}{rgb}{0.262138,0.242286,0.520837}%
\pgfsetstrokecolor{currentstroke}%
\pgfsetdash{}{0pt}%
\pgfpathmoveto{\pgfqpoint{2.124590in}{2.007384in}}%
\pgfpathlineto{\pgfqpoint{2.071841in}{2.010453in}}%
\pgfusepath{stroke}%
\end{pgfscope}%
\begin{pgfscope}%
\pgfpathrectangle{\pgfqpoint{0.800000in}{1.400000in}}{\pgfqpoint{2.407767in}{1.544118in}}%
\pgfusepath{clip}%
\pgfsetbuttcap%
\pgfsetroundjoin%
\pgfsetlinewidth{0.501875pt}%
\definecolor{currentstroke}{rgb}{0.269944,0.014625,0.341379}%
\pgfsetstrokecolor{currentstroke}%
\pgfsetdash{}{0pt}%
\pgfpathmoveto{\pgfqpoint{2.654046in}{2.033074in}}%
\pgfpathlineto{\pgfqpoint{2.601081in}{2.033708in}}%
\pgfusepath{stroke}%
\end{pgfscope}%
\begin{pgfscope}%
\pgfpathrectangle{\pgfqpoint{0.800000in}{1.400000in}}{\pgfqpoint{2.407767in}{1.544118in}}%
\pgfusepath{clip}%
\pgfsetbuttcap%
\pgfsetroundjoin%
\pgfsetlinewidth{0.501875pt}%
\definecolor{currentstroke}{rgb}{0.274952,0.037752,0.364543}%
\pgfsetstrokecolor{currentstroke}%
\pgfsetdash{}{0pt}%
\pgfpathmoveto{\pgfqpoint{2.601081in}{2.033708in}}%
\pgfpathlineto{\pgfqpoint{2.548111in}{2.034211in}}%
\pgfusepath{stroke}%
\end{pgfscope}%
\begin{pgfscope}%
\pgfpathrectangle{\pgfqpoint{0.800000in}{1.400000in}}{\pgfqpoint{2.407767in}{1.544118in}}%
\pgfusepath{clip}%
\pgfsetbuttcap%
\pgfsetroundjoin%
\pgfsetlinewidth{0.501875pt}%
\definecolor{currentstroke}{rgb}{0.280267,0.073417,0.397163}%
\pgfsetstrokecolor{currentstroke}%
\pgfsetdash{}{0pt}%
\pgfpathmoveto{\pgfqpoint{2.548111in}{2.034211in}}%
\pgfpathlineto{\pgfqpoint{2.495141in}{2.034732in}}%
\pgfusepath{stroke}%
\end{pgfscope}%
\begin{pgfscope}%
\pgfpathrectangle{\pgfqpoint{0.800000in}{1.400000in}}{\pgfqpoint{2.407767in}{1.544118in}}%
\pgfusepath{clip}%
\pgfsetbuttcap%
\pgfsetroundjoin%
\pgfsetlinewidth{0.501875pt}%
\definecolor{currentstroke}{rgb}{0.282327,0.094955,0.417331}%
\pgfsetstrokecolor{currentstroke}%
\pgfsetdash{}{0pt}%
\pgfpathmoveto{\pgfqpoint{2.495141in}{2.034732in}}%
\pgfpathlineto{\pgfqpoint{2.442173in}{2.035297in}}%
\pgfusepath{stroke}%
\end{pgfscope}%
\begin{pgfscope}%
\pgfpathrectangle{\pgfqpoint{0.800000in}{1.400000in}}{\pgfqpoint{2.407767in}{1.544118in}}%
\pgfusepath{clip}%
\pgfsetbuttcap%
\pgfsetroundjoin%
\pgfsetlinewidth{0.501875pt}%
\definecolor{currentstroke}{rgb}{0.283187,0.125848,0.444960}%
\pgfsetstrokecolor{currentstroke}%
\pgfsetdash{}{0pt}%
\pgfpathmoveto{\pgfqpoint{2.442173in}{2.035297in}}%
\pgfpathlineto{\pgfqpoint{2.389206in}{2.035924in}}%
\pgfusepath{stroke}%
\end{pgfscope}%
\begin{pgfscope}%
\pgfpathrectangle{\pgfqpoint{0.800000in}{1.400000in}}{\pgfqpoint{2.407767in}{1.544118in}}%
\pgfusepath{clip}%
\pgfsetbuttcap%
\pgfsetroundjoin%
\pgfsetlinewidth{0.501875pt}%
\definecolor{currentstroke}{rgb}{0.280255,0.165693,0.476498}%
\pgfsetstrokecolor{currentstroke}%
\pgfsetdash{}{0pt}%
\pgfpathmoveto{\pgfqpoint{2.389206in}{2.035924in}}%
\pgfpathlineto{\pgfqpoint{2.336241in}{2.036598in}}%
\pgfusepath{stroke}%
\end{pgfscope}%
\begin{pgfscope}%
\pgfpathrectangle{\pgfqpoint{0.800000in}{1.400000in}}{\pgfqpoint{2.407767in}{1.544118in}}%
\pgfusepath{clip}%
\pgfsetbuttcap%
\pgfsetroundjoin%
\pgfsetlinewidth{0.501875pt}%
\definecolor{currentstroke}{rgb}{0.277134,0.185228,0.489898}%
\pgfsetstrokecolor{currentstroke}%
\pgfsetdash{}{0pt}%
\pgfpathmoveto{\pgfqpoint{2.336241in}{2.036598in}}%
\pgfpathlineto{\pgfqpoint{2.283281in}{2.037417in}}%
\pgfusepath{stroke}%
\end{pgfscope}%
\begin{pgfscope}%
\pgfpathrectangle{\pgfqpoint{0.800000in}{1.400000in}}{\pgfqpoint{2.407767in}{1.544118in}}%
\pgfusepath{clip}%
\pgfsetbuttcap%
\pgfsetroundjoin%
\pgfsetlinewidth{0.501875pt}%
\definecolor{currentstroke}{rgb}{0.267968,0.223549,0.512008}%
\pgfsetstrokecolor{currentstroke}%
\pgfsetdash{}{0pt}%
\pgfpathmoveto{\pgfqpoint{2.283281in}{2.037417in}}%
\pgfpathlineto{\pgfqpoint{2.230333in}{2.038502in}}%
\pgfusepath{stroke}%
\end{pgfscope}%
\begin{pgfscope}%
\pgfpathrectangle{\pgfqpoint{0.800000in}{1.400000in}}{\pgfqpoint{2.407767in}{1.544118in}}%
\pgfusepath{clip}%
\pgfsetbuttcap%
\pgfsetroundjoin%
\pgfsetlinewidth{0.501875pt}%
\definecolor{currentstroke}{rgb}{0.255645,0.260703,0.528312}%
\pgfsetstrokecolor{currentstroke}%
\pgfsetdash{}{0pt}%
\pgfpathmoveto{\pgfqpoint{2.230333in}{2.038502in}}%
\pgfpathlineto{\pgfqpoint{2.177409in}{2.039984in}}%
\pgfusepath{stroke}%
\end{pgfscope}%
\begin{pgfscope}%
\pgfpathrectangle{\pgfqpoint{0.800000in}{1.400000in}}{\pgfqpoint{2.407767in}{1.544118in}}%
\pgfusepath{clip}%
\pgfsetbuttcap%
\pgfsetroundjoin%
\pgfsetlinewidth{0.501875pt}%
\definecolor{currentstroke}{rgb}{0.253935,0.265254,0.529983}%
\pgfsetstrokecolor{currentstroke}%
\pgfsetdash{}{0pt}%
\pgfpathmoveto{\pgfqpoint{2.177409in}{2.039984in}}%
\pgfpathlineto{\pgfqpoint{2.124524in}{2.041960in}}%
\pgfusepath{stroke}%
\end{pgfscope}%
\begin{pgfscope}%
\pgfpathrectangle{\pgfqpoint{0.800000in}{1.400000in}}{\pgfqpoint{2.407767in}{1.544118in}}%
\pgfusepath{clip}%
\pgfsetbuttcap%
\pgfsetroundjoin%
\pgfsetlinewidth{0.501875pt}%
\definecolor{currentstroke}{rgb}{0.250425,0.274290,0.533103}%
\pgfsetstrokecolor{currentstroke}%
\pgfsetdash{}{0pt}%
\pgfpathmoveto{\pgfqpoint{2.124524in}{2.041960in}}%
\pgfpathlineto{\pgfqpoint{2.071692in}{2.044441in}}%
\pgfusepath{stroke}%
\end{pgfscope}%
\begin{pgfscope}%
\pgfpathrectangle{\pgfqpoint{0.800000in}{1.400000in}}{\pgfqpoint{2.407767in}{1.544118in}}%
\pgfusepath{clip}%
\pgfsetbuttcap%
\pgfsetroundjoin%
\pgfsetlinewidth{0.501875pt}%
\definecolor{currentstroke}{rgb}{0.244972,0.287675,0.537260}%
\pgfsetstrokecolor{currentstroke}%
\pgfsetdash{}{0pt}%
\pgfpathmoveto{\pgfqpoint{2.071692in}{2.044441in}}%
\pgfpathlineto{\pgfqpoint{2.018941in}{2.047546in}}%
\pgfusepath{stroke}%
\end{pgfscope}%
\begin{pgfscope}%
\pgfpathrectangle{\pgfqpoint{0.800000in}{1.400000in}}{\pgfqpoint{2.407767in}{1.544118in}}%
\pgfusepath{clip}%
\pgfsetbuttcap%
\pgfsetroundjoin%
\pgfsetlinewidth{0.501875pt}%
\definecolor{currentstroke}{rgb}{0.248629,0.278775,0.534556}%
\pgfsetstrokecolor{currentstroke}%
\pgfsetdash{}{0pt}%
\pgfpathmoveto{\pgfqpoint{2.018941in}{2.047546in}}%
\pgfpathlineto{\pgfqpoint{1.966302in}{2.051348in}}%
\pgfusepath{stroke}%
\end{pgfscope}%
\begin{pgfscope}%
\pgfpathrectangle{\pgfqpoint{0.800000in}{1.400000in}}{\pgfqpoint{2.407767in}{1.544118in}}%
\pgfusepath{clip}%
\pgfsetbuttcap%
\pgfsetroundjoin%
\pgfsetlinewidth{0.501875pt}%
\definecolor{currentstroke}{rgb}{0.246811,0.283237,0.535941}%
\pgfsetstrokecolor{currentstroke}%
\pgfsetdash{}{0pt}%
\pgfpathmoveto{\pgfqpoint{1.966302in}{2.051348in}}%
\pgfpathlineto{\pgfqpoint{1.913808in}{2.055879in}}%
\pgfusepath{stroke}%
\end{pgfscope}%
\begin{pgfscope}%
\pgfpathrectangle{\pgfqpoint{0.800000in}{1.400000in}}{\pgfqpoint{2.407767in}{1.544118in}}%
\pgfusepath{clip}%
\pgfsetbuttcap%
\pgfsetroundjoin%
\pgfsetlinewidth{0.501875pt}%
\definecolor{currentstroke}{rgb}{0.244972,0.287675,0.537260}%
\pgfsetstrokecolor{currentstroke}%
\pgfsetdash{}{0pt}%
\pgfpathmoveto{\pgfqpoint{1.913808in}{2.055879in}}%
\pgfpathlineto{\pgfqpoint{1.861585in}{2.061527in}}%
\pgfusepath{stroke}%
\end{pgfscope}%
\begin{pgfscope}%
\pgfpathrectangle{\pgfqpoint{0.800000in}{1.400000in}}{\pgfqpoint{2.407767in}{1.544118in}}%
\pgfusepath{clip}%
\pgfsetbuttcap%
\pgfsetroundjoin%
\pgfsetlinewidth{0.501875pt}%
\definecolor{currentstroke}{rgb}{0.257322,0.256130,0.526563}%
\pgfsetstrokecolor{currentstroke}%
\pgfsetdash{}{0pt}%
\pgfpathmoveto{\pgfqpoint{1.861585in}{2.061527in}}%
\pgfpathlineto{\pgfqpoint{1.809586in}{2.068007in}}%
\pgfusepath{stroke}%
\end{pgfscope}%
\begin{pgfscope}%
\pgfpathrectangle{\pgfqpoint{0.800000in}{1.400000in}}{\pgfqpoint{2.407767in}{1.544118in}}%
\pgfusepath{clip}%
\pgfsetbuttcap%
\pgfsetroundjoin%
\pgfsetlinewidth{0.501875pt}%
\definecolor{currentstroke}{rgb}{0.250425,0.274290,0.533103}%
\pgfsetstrokecolor{currentstroke}%
\pgfsetdash{}{0pt}%
\pgfpathmoveto{\pgfqpoint{1.809586in}{2.068007in}}%
\pgfpathlineto{\pgfqpoint{1.757755in}{2.074976in}}%
\pgfusepath{stroke}%
\end{pgfscope}%
\begin{pgfscope}%
\pgfpathrectangle{\pgfqpoint{0.800000in}{1.400000in}}{\pgfqpoint{2.407767in}{1.544118in}}%
\pgfusepath{clip}%
\pgfsetbuttcap%
\pgfsetroundjoin%
\pgfsetlinewidth{0.501875pt}%
\definecolor{currentstroke}{rgb}{0.246811,0.283237,0.535941}%
\pgfsetstrokecolor{currentstroke}%
\pgfsetdash{}{0pt}%
\pgfpathmoveto{\pgfqpoint{1.757755in}{2.074976in}}%
\pgfpathlineto{\pgfqpoint{1.706108in}{2.082485in}}%
\pgfusepath{stroke}%
\end{pgfscope}%
\begin{pgfscope}%
\pgfpathrectangle{\pgfqpoint{0.800000in}{1.400000in}}{\pgfqpoint{2.407767in}{1.544118in}}%
\pgfusepath{clip}%
\pgfsetbuttcap%
\pgfsetroundjoin%
\pgfsetlinewidth{0.501875pt}%
\definecolor{currentstroke}{rgb}{0.257322,0.256130,0.526563}%
\pgfsetstrokecolor{currentstroke}%
\pgfsetdash{}{0pt}%
\pgfpathmoveto{\pgfqpoint{1.706108in}{2.082485in}}%
\pgfpathlineto{\pgfqpoint{1.654834in}{2.090915in}}%
\pgfusepath{stroke}%
\end{pgfscope}%
\begin{pgfscope}%
\pgfpathrectangle{\pgfqpoint{0.800000in}{1.400000in}}{\pgfqpoint{2.407767in}{1.544118in}}%
\pgfusepath{clip}%
\pgfsetbuttcap%
\pgfsetroundjoin%
\pgfsetlinewidth{0.501875pt}%
\definecolor{currentstroke}{rgb}{0.257322,0.256130,0.526563}%
\pgfsetstrokecolor{currentstroke}%
\pgfsetdash{}{0pt}%
\pgfpathmoveto{\pgfqpoint{1.654834in}{2.090915in}}%
\pgfpathlineto{\pgfqpoint{1.604500in}{2.101247in}}%
\pgfusepath{stroke}%
\end{pgfscope}%
\begin{pgfscope}%
\pgfpathrectangle{\pgfqpoint{0.800000in}{1.400000in}}{\pgfqpoint{2.407767in}{1.544118in}}%
\pgfusepath{clip}%
\pgfsetbuttcap%
\pgfsetroundjoin%
\pgfsetlinewidth{0.501875pt}%
\definecolor{currentstroke}{rgb}{0.271305,0.019942,0.347269}%
\pgfsetstrokecolor{currentstroke}%
\pgfsetdash{}{0pt}%
\pgfpathmoveto{\pgfqpoint{2.654046in}{2.067820in}}%
\pgfpathlineto{\pgfqpoint{2.601072in}{2.068035in}}%
\pgfusepath{stroke}%
\end{pgfscope}%
\begin{pgfscope}%
\pgfpathrectangle{\pgfqpoint{0.800000in}{1.400000in}}{\pgfqpoint{2.407767in}{1.544118in}}%
\pgfusepath{clip}%
\pgfsetbuttcap%
\pgfsetroundjoin%
\pgfsetlinewidth{0.501875pt}%
\definecolor{currentstroke}{rgb}{0.274952,0.037752,0.364543}%
\pgfsetstrokecolor{currentstroke}%
\pgfsetdash{}{0pt}%
\pgfpathmoveto{\pgfqpoint{2.601072in}{2.068035in}}%
\pgfpathlineto{\pgfqpoint{2.548097in}{2.068112in}}%
\pgfusepath{stroke}%
\end{pgfscope}%
\begin{pgfscope}%
\pgfpathrectangle{\pgfqpoint{0.800000in}{1.400000in}}{\pgfqpoint{2.407767in}{1.544118in}}%
\pgfusepath{clip}%
\pgfsetbuttcap%
\pgfsetroundjoin%
\pgfsetlinewidth{0.501875pt}%
\definecolor{currentstroke}{rgb}{0.278791,0.062145,0.386592}%
\pgfsetstrokecolor{currentstroke}%
\pgfsetdash{}{0pt}%
\pgfpathmoveto{\pgfqpoint{2.548097in}{2.068112in}}%
\pgfpathlineto{\pgfqpoint{2.495124in}{2.068389in}}%
\pgfusepath{stroke}%
\end{pgfscope}%
\begin{pgfscope}%
\pgfpathrectangle{\pgfqpoint{0.800000in}{1.400000in}}{\pgfqpoint{2.407767in}{1.544118in}}%
\pgfusepath{clip}%
\pgfsetbuttcap%
\pgfsetroundjoin%
\pgfsetlinewidth{0.501875pt}%
\definecolor{currentstroke}{rgb}{0.282910,0.105393,0.426902}%
\pgfsetstrokecolor{currentstroke}%
\pgfsetdash{}{0pt}%
\pgfpathmoveto{\pgfqpoint{2.495124in}{2.068389in}}%
\pgfpathlineto{\pgfqpoint{2.442155in}{2.068921in}}%
\pgfusepath{stroke}%
\end{pgfscope}%
\begin{pgfscope}%
\pgfpathrectangle{\pgfqpoint{0.800000in}{1.400000in}}{\pgfqpoint{2.407767in}{1.544118in}}%
\pgfusepath{clip}%
\pgfsetbuttcap%
\pgfsetroundjoin%
\pgfsetlinewidth{0.501875pt}%
\definecolor{currentstroke}{rgb}{0.283072,0.130895,0.449241}%
\pgfsetstrokecolor{currentstroke}%
\pgfsetdash{}{0pt}%
\pgfpathmoveto{\pgfqpoint{2.442155in}{2.068921in}}%
\pgfpathlineto{\pgfqpoint{2.389187in}{2.069512in}}%
\pgfusepath{stroke}%
\end{pgfscope}%
\begin{pgfscope}%
\pgfpathrectangle{\pgfqpoint{0.800000in}{1.400000in}}{\pgfqpoint{2.407767in}{1.544118in}}%
\pgfusepath{clip}%
\pgfsetbuttcap%
\pgfsetroundjoin%
\pgfsetlinewidth{0.501875pt}%
\definecolor{currentstroke}{rgb}{0.278012,0.180367,0.486697}%
\pgfsetstrokecolor{currentstroke}%
\pgfsetdash{}{0pt}%
\pgfpathmoveto{\pgfqpoint{2.389187in}{2.069512in}}%
\pgfpathlineto{\pgfqpoint{2.336224in}{2.070224in}}%
\pgfusepath{stroke}%
\end{pgfscope}%
\begin{pgfscope}%
\pgfpathrectangle{\pgfqpoint{0.800000in}{1.400000in}}{\pgfqpoint{2.407767in}{1.544118in}}%
\pgfusepath{clip}%
\pgfsetbuttcap%
\pgfsetroundjoin%
\pgfsetlinewidth{0.501875pt}%
\definecolor{currentstroke}{rgb}{0.273006,0.204520,0.501721}%
\pgfsetstrokecolor{currentstroke}%
\pgfsetdash{}{0pt}%
\pgfpathmoveto{\pgfqpoint{2.336224in}{2.070224in}}%
\pgfpathlineto{\pgfqpoint{2.283268in}{2.071133in}}%
\pgfusepath{stroke}%
\end{pgfscope}%
\begin{pgfscope}%
\pgfpathrectangle{\pgfqpoint{0.800000in}{1.400000in}}{\pgfqpoint{2.407767in}{1.544118in}}%
\pgfusepath{clip}%
\pgfsetbuttcap%
\pgfsetroundjoin%
\pgfsetlinewidth{0.501875pt}%
\definecolor{currentstroke}{rgb}{0.258965,0.251537,0.524736}%
\pgfsetstrokecolor{currentstroke}%
\pgfsetdash{}{0pt}%
\pgfpathmoveto{\pgfqpoint{2.283268in}{2.071133in}}%
\pgfpathlineto{\pgfqpoint{2.230322in}{2.072268in}}%
\pgfusepath{stroke}%
\end{pgfscope}%
\begin{pgfscope}%
\pgfpathrectangle{\pgfqpoint{0.800000in}{1.400000in}}{\pgfqpoint{2.407767in}{1.544118in}}%
\pgfusepath{clip}%
\pgfsetbuttcap%
\pgfsetroundjoin%
\pgfsetlinewidth{0.501875pt}%
\definecolor{currentstroke}{rgb}{0.253935,0.265254,0.529983}%
\pgfsetstrokecolor{currentstroke}%
\pgfsetdash{}{0pt}%
\pgfpathmoveto{\pgfqpoint{2.230322in}{2.072268in}}%
\pgfpathlineto{\pgfqpoint{2.177391in}{2.073648in}}%
\pgfusepath{stroke}%
\end{pgfscope}%
\begin{pgfscope}%
\pgfpathrectangle{\pgfqpoint{0.800000in}{1.400000in}}{\pgfqpoint{2.407767in}{1.544118in}}%
\pgfusepath{clip}%
\pgfsetbuttcap%
\pgfsetroundjoin%
\pgfsetlinewidth{0.501875pt}%
\definecolor{currentstroke}{rgb}{0.244972,0.287675,0.537260}%
\pgfsetstrokecolor{currentstroke}%
\pgfsetdash{}{0pt}%
\pgfpathmoveto{\pgfqpoint{2.177391in}{2.073648in}}%
\pgfpathlineto{\pgfqpoint{2.124477in}{2.075268in}}%
\pgfusepath{stroke}%
\end{pgfscope}%
\begin{pgfscope}%
\pgfpathrectangle{\pgfqpoint{0.800000in}{1.400000in}}{\pgfqpoint{2.407767in}{1.544118in}}%
\pgfusepath{clip}%
\pgfsetbuttcap%
\pgfsetroundjoin%
\pgfsetlinewidth{0.501875pt}%
\definecolor{currentstroke}{rgb}{0.237441,0.305202,0.541921}%
\pgfsetstrokecolor{currentstroke}%
\pgfsetdash{}{0pt}%
\pgfpathmoveto{\pgfqpoint{2.124477in}{2.075268in}}%
\pgfpathlineto{\pgfqpoint{2.071596in}{2.077271in}}%
\pgfusepath{stroke}%
\end{pgfscope}%
\begin{pgfscope}%
\pgfpathrectangle{\pgfqpoint{0.800000in}{1.400000in}}{\pgfqpoint{2.407767in}{1.544118in}}%
\pgfusepath{clip}%
\pgfsetbuttcap%
\pgfsetroundjoin%
\pgfsetlinewidth{0.501875pt}%
\definecolor{currentstroke}{rgb}{0.223925,0.334994,0.548053}%
\pgfsetstrokecolor{currentstroke}%
\pgfsetdash{}{0pt}%
\pgfpathmoveto{\pgfqpoint{2.071596in}{2.077271in}}%
\pgfpathlineto{\pgfqpoint{2.018756in}{2.079695in}}%
\pgfusepath{stroke}%
\end{pgfscope}%
\begin{pgfscope}%
\pgfpathrectangle{\pgfqpoint{0.800000in}{1.400000in}}{\pgfqpoint{2.407767in}{1.544118in}}%
\pgfusepath{clip}%
\pgfsetbuttcap%
\pgfsetroundjoin%
\pgfsetlinewidth{0.501875pt}%
\definecolor{currentstroke}{rgb}{0.271305,0.019942,0.347269}%
\pgfsetstrokecolor{currentstroke}%
\pgfsetdash{}{0pt}%
\pgfpathmoveto{\pgfqpoint{2.654046in}{2.102567in}}%
\pgfpathlineto{\pgfqpoint{2.601074in}{2.102530in}}%
\pgfusepath{stroke}%
\end{pgfscope}%
\begin{pgfscope}%
\pgfpathrectangle{\pgfqpoint{0.800000in}{1.400000in}}{\pgfqpoint{2.407767in}{1.544118in}}%
\pgfusepath{clip}%
\pgfsetbuttcap%
\pgfsetroundjoin%
\pgfsetlinewidth{0.501875pt}%
\definecolor{currentstroke}{rgb}{0.274952,0.037752,0.364543}%
\pgfsetstrokecolor{currentstroke}%
\pgfsetdash{}{0pt}%
\pgfpathmoveto{\pgfqpoint{2.601074in}{2.102530in}}%
\pgfpathlineto{\pgfqpoint{2.548102in}{2.102528in}}%
\pgfusepath{stroke}%
\end{pgfscope}%
\begin{pgfscope}%
\pgfpathrectangle{\pgfqpoint{0.800000in}{1.400000in}}{\pgfqpoint{2.407767in}{1.544118in}}%
\pgfusepath{clip}%
\pgfsetbuttcap%
\pgfsetroundjoin%
\pgfsetlinewidth{0.501875pt}%
\definecolor{currentstroke}{rgb}{0.280267,0.073417,0.397163}%
\pgfsetstrokecolor{currentstroke}%
\pgfsetdash{}{0pt}%
\pgfpathmoveto{\pgfqpoint{2.548102in}{2.102528in}}%
\pgfpathlineto{\pgfqpoint{2.495128in}{2.102703in}}%
\pgfusepath{stroke}%
\end{pgfscope}%
\begin{pgfscope}%
\pgfpathrectangle{\pgfqpoint{0.800000in}{1.400000in}}{\pgfqpoint{2.407767in}{1.544118in}}%
\pgfusepath{clip}%
\pgfsetbuttcap%
\pgfsetroundjoin%
\pgfsetlinewidth{0.501875pt}%
\definecolor{currentstroke}{rgb}{0.283091,0.110553,0.431554}%
\pgfsetstrokecolor{currentstroke}%
\pgfsetdash{}{0pt}%
\pgfpathmoveto{\pgfqpoint{2.495128in}{2.102703in}}%
\pgfpathlineto{\pgfqpoint{2.442153in}{2.102787in}}%
\pgfusepath{stroke}%
\end{pgfscope}%
\begin{pgfscope}%
\pgfpathrectangle{\pgfqpoint{0.800000in}{1.400000in}}{\pgfqpoint{2.407767in}{1.544118in}}%
\pgfusepath{clip}%
\pgfsetbuttcap%
\pgfsetroundjoin%
\pgfsetlinewidth{0.501875pt}%
\definecolor{currentstroke}{rgb}{0.282623,0.140926,0.457517}%
\pgfsetstrokecolor{currentstroke}%
\pgfsetdash{}{0pt}%
\pgfpathmoveto{\pgfqpoint{2.442153in}{2.102787in}}%
\pgfpathlineto{\pgfqpoint{2.389178in}{2.102964in}}%
\pgfusepath{stroke}%
\end{pgfscope}%
\begin{pgfscope}%
\pgfpathrectangle{\pgfqpoint{0.800000in}{1.400000in}}{\pgfqpoint{2.407767in}{1.544118in}}%
\pgfusepath{clip}%
\pgfsetbuttcap%
\pgfsetroundjoin%
\pgfsetlinewidth{0.501875pt}%
\definecolor{currentstroke}{rgb}{0.278012,0.180367,0.486697}%
\pgfsetstrokecolor{currentstroke}%
\pgfsetdash{}{0pt}%
\pgfpathmoveto{\pgfqpoint{2.389178in}{2.102964in}}%
\pgfpathlineto{\pgfqpoint{2.336205in}{2.103228in}}%
\pgfusepath{stroke}%
\end{pgfscope}%
\begin{pgfscope}%
\pgfpathrectangle{\pgfqpoint{0.800000in}{1.400000in}}{\pgfqpoint{2.407767in}{1.544118in}}%
\pgfusepath{clip}%
\pgfsetbuttcap%
\pgfsetroundjoin%
\pgfsetlinewidth{0.501875pt}%
\definecolor{currentstroke}{rgb}{0.262138,0.242286,0.520837}%
\pgfsetstrokecolor{currentstroke}%
\pgfsetdash{}{0pt}%
\pgfpathmoveto{\pgfqpoint{2.336205in}{2.103228in}}%
\pgfpathlineto{\pgfqpoint{2.283234in}{2.103580in}}%
\pgfusepath{stroke}%
\end{pgfscope}%
\begin{pgfscope}%
\pgfpathrectangle{\pgfqpoint{0.800000in}{1.400000in}}{\pgfqpoint{2.407767in}{1.544118in}}%
\pgfusepath{clip}%
\pgfsetbuttcap%
\pgfsetroundjoin%
\pgfsetlinewidth{0.501875pt}%
\definecolor{currentstroke}{rgb}{0.253935,0.265254,0.529983}%
\pgfsetstrokecolor{currentstroke}%
\pgfsetdash{}{0pt}%
\pgfpathmoveto{\pgfqpoint{2.283234in}{2.103580in}}%
\pgfpathlineto{\pgfqpoint{2.230267in}{2.104159in}}%
\pgfusepath{stroke}%
\end{pgfscope}%
\begin{pgfscope}%
\pgfpathrectangle{\pgfqpoint{0.800000in}{1.400000in}}{\pgfqpoint{2.407767in}{1.544118in}}%
\pgfusepath{clip}%
\pgfsetbuttcap%
\pgfsetroundjoin%
\pgfsetlinewidth{0.501875pt}%
\definecolor{currentstroke}{rgb}{0.244972,0.287675,0.537260}%
\pgfsetstrokecolor{currentstroke}%
\pgfsetdash{}{0pt}%
\pgfpathmoveto{\pgfqpoint{2.230267in}{2.104159in}}%
\pgfpathlineto{\pgfqpoint{2.177305in}{2.104933in}}%
\pgfusepath{stroke}%
\end{pgfscope}%
\begin{pgfscope}%
\pgfpathrectangle{\pgfqpoint{0.800000in}{1.400000in}}{\pgfqpoint{2.407767in}{1.544118in}}%
\pgfusepath{clip}%
\pgfsetbuttcap%
\pgfsetroundjoin%
\pgfsetlinewidth{0.501875pt}%
\definecolor{currentstroke}{rgb}{0.229739,0.322361,0.545706}%
\pgfsetstrokecolor{currentstroke}%
\pgfsetdash{}{0pt}%
\pgfpathmoveto{\pgfqpoint{2.177305in}{2.104933in}}%
\pgfpathlineto{\pgfqpoint{2.124347in}{2.105796in}}%
\pgfusepath{stroke}%
\end{pgfscope}%
\begin{pgfscope}%
\pgfpathrectangle{\pgfqpoint{0.800000in}{1.400000in}}{\pgfqpoint{2.407767in}{1.544118in}}%
\pgfusepath{clip}%
\pgfsetbuttcap%
\pgfsetroundjoin%
\pgfsetlinewidth{0.501875pt}%
\definecolor{currentstroke}{rgb}{0.237441,0.305202,0.541921}%
\pgfsetstrokecolor{currentstroke}%
\pgfsetdash{}{0pt}%
\pgfpathmoveto{\pgfqpoint{2.124347in}{2.105796in}}%
\pgfpathlineto{\pgfqpoint{2.071394in}{2.106738in}}%
\pgfusepath{stroke}%
\end{pgfscope}%
\begin{pgfscope}%
\pgfpathrectangle{\pgfqpoint{0.800000in}{1.400000in}}{\pgfqpoint{2.407767in}{1.544118in}}%
\pgfusepath{clip}%
\pgfsetbuttcap%
\pgfsetroundjoin%
\pgfsetlinewidth{0.501875pt}%
\definecolor{currentstroke}{rgb}{0.227802,0.326594,0.546532}%
\pgfsetstrokecolor{currentstroke}%
\pgfsetdash{}{0pt}%
\pgfpathmoveto{\pgfqpoint{2.071394in}{2.106738in}}%
\pgfpathlineto{\pgfqpoint{2.018454in}{2.107931in}}%
\pgfusepath{stroke}%
\end{pgfscope}%
\begin{pgfscope}%
\pgfpathrectangle{\pgfqpoint{0.800000in}{1.400000in}}{\pgfqpoint{2.407767in}{1.544118in}}%
\pgfusepath{clip}%
\pgfsetbuttcap%
\pgfsetroundjoin%
\pgfsetlinewidth{0.501875pt}%
\definecolor{currentstroke}{rgb}{0.221989,0.339161,0.548752}%
\pgfsetstrokecolor{currentstroke}%
\pgfsetdash{}{0pt}%
\pgfpathmoveto{\pgfqpoint{2.018454in}{2.107931in}}%
\pgfpathlineto{\pgfqpoint{1.965528in}{2.109377in}}%
\pgfusepath{stroke}%
\end{pgfscope}%
\begin{pgfscope}%
\pgfpathrectangle{\pgfqpoint{0.800000in}{1.400000in}}{\pgfqpoint{2.407767in}{1.544118in}}%
\pgfusepath{clip}%
\pgfsetbuttcap%
\pgfsetroundjoin%
\pgfsetlinewidth{0.501875pt}%
\definecolor{currentstroke}{rgb}{0.225863,0.330805,0.547314}%
\pgfsetstrokecolor{currentstroke}%
\pgfsetdash{}{0pt}%
\pgfpathmoveto{\pgfqpoint{1.965528in}{2.109377in}}%
\pgfpathlineto{\pgfqpoint{1.912628in}{2.111156in}}%
\pgfusepath{stroke}%
\end{pgfscope}%
\begin{pgfscope}%
\pgfpathrectangle{\pgfqpoint{0.800000in}{1.400000in}}{\pgfqpoint{2.407767in}{1.544118in}}%
\pgfusepath{clip}%
\pgfsetbuttcap%
\pgfsetroundjoin%
\pgfsetlinewidth{0.501875pt}%
\definecolor{currentstroke}{rgb}{0.235526,0.309527,0.542944}%
\pgfsetstrokecolor{currentstroke}%
\pgfsetdash{}{0pt}%
\pgfpathmoveto{\pgfqpoint{1.912628in}{2.111156in}}%
\pgfpathlineto{\pgfqpoint{1.859767in}{2.113357in}}%
\pgfusepath{stroke}%
\end{pgfscope}%
\begin{pgfscope}%
\pgfpathrectangle{\pgfqpoint{0.800000in}{1.400000in}}{\pgfqpoint{2.407767in}{1.544118in}}%
\pgfusepath{clip}%
\pgfsetbuttcap%
\pgfsetroundjoin%
\pgfsetlinewidth{0.501875pt}%
\definecolor{currentstroke}{rgb}{0.241237,0.296485,0.539709}%
\pgfsetstrokecolor{currentstroke}%
\pgfsetdash{}{0pt}%
\pgfpathmoveto{\pgfqpoint{1.859767in}{2.113357in}}%
\pgfpathlineto{\pgfqpoint{1.806941in}{2.115844in}}%
\pgfusepath{stroke}%
\end{pgfscope}%
\begin{pgfscope}%
\pgfpathrectangle{\pgfqpoint{0.800000in}{1.400000in}}{\pgfqpoint{2.407767in}{1.544118in}}%
\pgfusepath{clip}%
\pgfsetbuttcap%
\pgfsetroundjoin%
\pgfsetlinewidth{0.501875pt}%
\definecolor{currentstroke}{rgb}{0.248629,0.278775,0.534556}%
\pgfsetstrokecolor{currentstroke}%
\pgfsetdash{}{0pt}%
\pgfpathmoveto{\pgfqpoint{1.806941in}{2.115844in}}%
\pgfpathlineto{\pgfqpoint{1.754144in}{2.118554in}}%
\pgfusepath{stroke}%
\end{pgfscope}%
\begin{pgfscope}%
\pgfpathrectangle{\pgfqpoint{0.800000in}{1.400000in}}{\pgfqpoint{2.407767in}{1.544118in}}%
\pgfusepath{clip}%
\pgfsetbuttcap%
\pgfsetroundjoin%
\pgfsetlinewidth{0.501875pt}%
\definecolor{currentstroke}{rgb}{0.250425,0.274290,0.533103}%
\pgfsetstrokecolor{currentstroke}%
\pgfsetdash{}{0pt}%
\pgfpathmoveto{\pgfqpoint{1.754144in}{2.118554in}}%
\pgfpathlineto{\pgfqpoint{1.701374in}{2.121438in}}%
\pgfusepath{stroke}%
\end{pgfscope}%
\begin{pgfscope}%
\pgfpathrectangle{\pgfqpoint{0.800000in}{1.400000in}}{\pgfqpoint{2.407767in}{1.544118in}}%
\pgfusepath{clip}%
\pgfsetbuttcap%
\pgfsetroundjoin%
\pgfsetlinewidth{0.501875pt}%
\definecolor{currentstroke}{rgb}{0.253935,0.265254,0.529983}%
\pgfsetstrokecolor{currentstroke}%
\pgfsetdash{}{0pt}%
\pgfpathmoveto{\pgfqpoint{1.701374in}{2.121438in}}%
\pgfpathlineto{\pgfqpoint{1.648672in}{2.124759in}}%
\pgfusepath{stroke}%
\end{pgfscope}%
\begin{pgfscope}%
\pgfpathrectangle{\pgfqpoint{0.800000in}{1.400000in}}{\pgfqpoint{2.407767in}{1.544118in}}%
\pgfusepath{clip}%
\pgfsetbuttcap%
\pgfsetroundjoin%
\pgfsetlinewidth{0.501875pt}%
\definecolor{currentstroke}{rgb}{0.271828,0.209303,0.504434}%
\pgfsetstrokecolor{currentstroke}%
\pgfsetdash{}{0pt}%
\pgfpathmoveto{\pgfqpoint{1.648672in}{2.124759in}}%
\pgfpathlineto{\pgfqpoint{1.596812in}{2.129957in}}%
\pgfusepath{stroke}%
\end{pgfscope}%
\begin{pgfscope}%
\pgfpathrectangle{\pgfqpoint{0.800000in}{1.400000in}}{\pgfqpoint{2.407767in}{1.544118in}}%
\pgfusepath{clip}%
\pgfsetbuttcap%
\pgfsetroundjoin%
\pgfsetlinewidth{0.501875pt}%
\definecolor{currentstroke}{rgb}{0.271305,0.019942,0.347269}%
\pgfsetstrokecolor{currentstroke}%
\pgfsetdash{}{0pt}%
\pgfpathmoveto{\pgfqpoint{2.654046in}{2.137313in}}%
\pgfpathlineto{\pgfqpoint{2.601071in}{2.137181in}}%
\pgfusepath{stroke}%
\end{pgfscope}%
\begin{pgfscope}%
\pgfpathrectangle{\pgfqpoint{0.800000in}{1.400000in}}{\pgfqpoint{2.407767in}{1.544118in}}%
\pgfusepath{clip}%
\pgfsetbuttcap%
\pgfsetroundjoin%
\pgfsetlinewidth{0.501875pt}%
\definecolor{currentstroke}{rgb}{0.274952,0.037752,0.364543}%
\pgfsetstrokecolor{currentstroke}%
\pgfsetdash{}{0pt}%
\pgfpathmoveto{\pgfqpoint{2.601071in}{2.137181in}}%
\pgfpathlineto{\pgfqpoint{2.548095in}{2.137083in}}%
\pgfusepath{stroke}%
\end{pgfscope}%
\begin{pgfscope}%
\pgfpathrectangle{\pgfqpoint{0.800000in}{1.400000in}}{\pgfqpoint{2.407767in}{1.544118in}}%
\pgfusepath{clip}%
\pgfsetbuttcap%
\pgfsetroundjoin%
\pgfsetlinewidth{0.501875pt}%
\definecolor{currentstroke}{rgb}{0.280267,0.073417,0.397163}%
\pgfsetstrokecolor{currentstroke}%
\pgfsetdash{}{0pt}%
\pgfpathmoveto{\pgfqpoint{2.548095in}{2.137083in}}%
\pgfpathlineto{\pgfqpoint{2.495120in}{2.137030in}}%
\pgfusepath{stroke}%
\end{pgfscope}%
\begin{pgfscope}%
\pgfpathrectangle{\pgfqpoint{0.800000in}{1.400000in}}{\pgfqpoint{2.407767in}{1.544118in}}%
\pgfusepath{clip}%
\pgfsetbuttcap%
\pgfsetroundjoin%
\pgfsetlinewidth{0.501875pt}%
\definecolor{currentstroke}{rgb}{0.283091,0.110553,0.431554}%
\pgfsetstrokecolor{currentstroke}%
\pgfsetdash{}{0pt}%
\pgfpathmoveto{\pgfqpoint{2.495120in}{2.137030in}}%
\pgfpathlineto{\pgfqpoint{2.442145in}{2.136933in}}%
\pgfusepath{stroke}%
\end{pgfscope}%
\begin{pgfscope}%
\pgfpathrectangle{\pgfqpoint{0.800000in}{1.400000in}}{\pgfqpoint{2.407767in}{1.544118in}}%
\pgfusepath{clip}%
\pgfsetbuttcap%
\pgfsetroundjoin%
\pgfsetlinewidth{0.501875pt}%
\definecolor{currentstroke}{rgb}{0.282623,0.140926,0.457517}%
\pgfsetstrokecolor{currentstroke}%
\pgfsetdash{}{0pt}%
\pgfpathmoveto{\pgfqpoint{2.442145in}{2.136933in}}%
\pgfpathlineto{\pgfqpoint{2.389170in}{2.136828in}}%
\pgfusepath{stroke}%
\end{pgfscope}%
\begin{pgfscope}%
\pgfpathrectangle{\pgfqpoint{0.800000in}{1.400000in}}{\pgfqpoint{2.407767in}{1.544118in}}%
\pgfusepath{clip}%
\pgfsetbuttcap%
\pgfsetroundjoin%
\pgfsetlinewidth{0.501875pt}%
\definecolor{currentstroke}{rgb}{0.277134,0.185228,0.489898}%
\pgfsetstrokecolor{currentstroke}%
\pgfsetdash{}{0pt}%
\pgfpathmoveto{\pgfqpoint{2.389170in}{2.136828in}}%
\pgfpathlineto{\pgfqpoint{2.336194in}{2.136810in}}%
\pgfusepath{stroke}%
\end{pgfscope}%
\begin{pgfscope}%
\pgfpathrectangle{\pgfqpoint{0.800000in}{1.400000in}}{\pgfqpoint{2.407767in}{1.544118in}}%
\pgfusepath{clip}%
\pgfsetbuttcap%
\pgfsetroundjoin%
\pgfsetlinewidth{0.501875pt}%
\definecolor{currentstroke}{rgb}{0.267968,0.223549,0.512008}%
\pgfsetstrokecolor{currentstroke}%
\pgfsetdash{}{0pt}%
\pgfpathmoveto{\pgfqpoint{2.336194in}{2.136810in}}%
\pgfpathlineto{\pgfqpoint{2.283218in}{2.136863in}}%
\pgfusepath{stroke}%
\end{pgfscope}%
\begin{pgfscope}%
\pgfpathrectangle{\pgfqpoint{0.800000in}{1.400000in}}{\pgfqpoint{2.407767in}{1.544118in}}%
\pgfusepath{clip}%
\pgfsetbuttcap%
\pgfsetroundjoin%
\pgfsetlinewidth{0.501875pt}%
\definecolor{currentstroke}{rgb}{0.255645,0.260703,0.528312}%
\pgfsetstrokecolor{currentstroke}%
\pgfsetdash{}{0pt}%
\pgfpathmoveto{\pgfqpoint{2.283218in}{2.136863in}}%
\pgfpathlineto{\pgfqpoint{2.230242in}{2.136903in}}%
\pgfusepath{stroke}%
\end{pgfscope}%
\begin{pgfscope}%
\pgfpathrectangle{\pgfqpoint{0.800000in}{1.400000in}}{\pgfqpoint{2.407767in}{1.544118in}}%
\pgfusepath{clip}%
\pgfsetbuttcap%
\pgfsetroundjoin%
\pgfsetlinewidth{0.501875pt}%
\definecolor{currentstroke}{rgb}{0.239346,0.300855,0.540844}%
\pgfsetstrokecolor{currentstroke}%
\pgfsetdash{}{0pt}%
\pgfpathmoveto{\pgfqpoint{2.230242in}{2.136903in}}%
\pgfpathlineto{\pgfqpoint{2.177267in}{2.136951in}}%
\pgfusepath{stroke}%
\end{pgfscope}%
\begin{pgfscope}%
\pgfpathrectangle{\pgfqpoint{0.800000in}{1.400000in}}{\pgfqpoint{2.407767in}{1.544118in}}%
\pgfusepath{clip}%
\pgfsetbuttcap%
\pgfsetroundjoin%
\pgfsetlinewidth{0.501875pt}%
\definecolor{currentstroke}{rgb}{0.233603,0.313828,0.543914}%
\pgfsetstrokecolor{currentstroke}%
\pgfsetdash{}{0pt}%
\pgfpathmoveto{\pgfqpoint{2.177267in}{2.136951in}}%
\pgfpathlineto{\pgfqpoint{2.124291in}{2.136979in}}%
\pgfusepath{stroke}%
\end{pgfscope}%
\begin{pgfscope}%
\pgfpathrectangle{\pgfqpoint{0.800000in}{1.400000in}}{\pgfqpoint{2.407767in}{1.544118in}}%
\pgfusepath{clip}%
\pgfsetbuttcap%
\pgfsetroundjoin%
\pgfsetlinewidth{0.501875pt}%
\definecolor{currentstroke}{rgb}{0.227802,0.326594,0.546532}%
\pgfsetstrokecolor{currentstroke}%
\pgfsetdash{}{0pt}%
\pgfpathmoveto{\pgfqpoint{2.124291in}{2.136979in}}%
\pgfpathlineto{\pgfqpoint{2.071317in}{2.137138in}}%
\pgfusepath{stroke}%
\end{pgfscope}%
\begin{pgfscope}%
\pgfpathrectangle{\pgfqpoint{0.800000in}{1.400000in}}{\pgfqpoint{2.407767in}{1.544118in}}%
\pgfusepath{clip}%
\pgfsetbuttcap%
\pgfsetroundjoin%
\pgfsetlinewidth{0.501875pt}%
\definecolor{currentstroke}{rgb}{0.223925,0.334994,0.548053}%
\pgfsetstrokecolor{currentstroke}%
\pgfsetdash{}{0pt}%
\pgfpathmoveto{\pgfqpoint{2.071317in}{2.137138in}}%
\pgfpathlineto{\pgfqpoint{2.018345in}{2.137400in}}%
\pgfusepath{stroke}%
\end{pgfscope}%
\begin{pgfscope}%
\pgfpathrectangle{\pgfqpoint{0.800000in}{1.400000in}}{\pgfqpoint{2.407767in}{1.544118in}}%
\pgfusepath{clip}%
\pgfsetbuttcap%
\pgfsetroundjoin%
\pgfsetlinewidth{0.501875pt}%
\definecolor{currentstroke}{rgb}{0.229739,0.322361,0.545706}%
\pgfsetstrokecolor{currentstroke}%
\pgfsetdash{}{0pt}%
\pgfpathmoveto{\pgfqpoint{2.018345in}{2.137400in}}%
\pgfpathlineto{\pgfqpoint{1.965373in}{2.137669in}}%
\pgfusepath{stroke}%
\end{pgfscope}%
\begin{pgfscope}%
\pgfpathrectangle{\pgfqpoint{0.800000in}{1.400000in}}{\pgfqpoint{2.407767in}{1.544118in}}%
\pgfusepath{clip}%
\pgfsetbuttcap%
\pgfsetroundjoin%
\pgfsetlinewidth{0.501875pt}%
\definecolor{currentstroke}{rgb}{0.218130,0.347432,0.550038}%
\pgfsetstrokecolor{currentstroke}%
\pgfsetdash{}{0pt}%
\pgfpathmoveto{\pgfqpoint{1.965373in}{2.137669in}}%
\pgfpathlineto{\pgfqpoint{1.912402in}{2.137988in}}%
\pgfusepath{stroke}%
\end{pgfscope}%
\begin{pgfscope}%
\pgfpathrectangle{\pgfqpoint{0.800000in}{1.400000in}}{\pgfqpoint{2.407767in}{1.544118in}}%
\pgfusepath{clip}%
\pgfsetbuttcap%
\pgfsetroundjoin%
\pgfsetlinewidth{0.501875pt}%
\definecolor{currentstroke}{rgb}{0.233603,0.313828,0.543914}%
\pgfsetstrokecolor{currentstroke}%
\pgfsetdash{}{0pt}%
\pgfpathmoveto{\pgfqpoint{1.912402in}{2.137988in}}%
\pgfpathlineto{\pgfqpoint{1.859434in}{2.138421in}}%
\pgfusepath{stroke}%
\end{pgfscope}%
\begin{pgfscope}%
\pgfpathrectangle{\pgfqpoint{0.800000in}{1.400000in}}{\pgfqpoint{2.407767in}{1.544118in}}%
\pgfusepath{clip}%
\pgfsetbuttcap%
\pgfsetroundjoin%
\pgfsetlinewidth{0.501875pt}%
\definecolor{currentstroke}{rgb}{0.241237,0.296485,0.539709}%
\pgfsetstrokecolor{currentstroke}%
\pgfsetdash{}{0pt}%
\pgfpathmoveto{\pgfqpoint{1.859434in}{2.138421in}}%
\pgfpathlineto{\pgfqpoint{1.806471in}{2.138684in}}%
\pgfusepath{stroke}%
\end{pgfscope}%
\begin{pgfscope}%
\pgfpathrectangle{\pgfqpoint{0.800000in}{1.400000in}}{\pgfqpoint{2.407767in}{1.544118in}}%
\pgfusepath{clip}%
\pgfsetbuttcap%
\pgfsetroundjoin%
\pgfsetlinewidth{0.501875pt}%
\definecolor{currentstroke}{rgb}{0.246811,0.283237,0.535941}%
\pgfsetstrokecolor{currentstroke}%
\pgfsetdash{}{0pt}%
\pgfpathmoveto{\pgfqpoint{1.806471in}{2.138684in}}%
\pgfpathlineto{\pgfqpoint{1.753513in}{2.138958in}}%
\pgfusepath{stroke}%
\end{pgfscope}%
\begin{pgfscope}%
\pgfpathrectangle{\pgfqpoint{0.800000in}{1.400000in}}{\pgfqpoint{2.407767in}{1.544118in}}%
\pgfusepath{clip}%
\pgfsetbuttcap%
\pgfsetroundjoin%
\pgfsetlinewidth{0.501875pt}%
\definecolor{currentstroke}{rgb}{0.271305,0.019942,0.347269}%
\pgfsetstrokecolor{currentstroke}%
\pgfsetdash{}{0pt}%
\pgfpathmoveto{\pgfqpoint{2.654046in}{2.172059in}}%
\pgfpathlineto{\pgfqpoint{2.601080in}{2.172027in}}%
\pgfusepath{stroke}%
\end{pgfscope}%
\begin{pgfscope}%
\pgfpathrectangle{\pgfqpoint{0.800000in}{1.400000in}}{\pgfqpoint{2.407767in}{1.544118in}}%
\pgfusepath{clip}%
\pgfsetbuttcap%
\pgfsetroundjoin%
\pgfsetlinewidth{0.501875pt}%
\definecolor{currentstroke}{rgb}{0.274952,0.037752,0.364543}%
\pgfsetstrokecolor{currentstroke}%
\pgfsetdash{}{0pt}%
\pgfpathmoveto{\pgfqpoint{2.601080in}{2.172027in}}%
\pgfpathlineto{\pgfqpoint{2.548107in}{2.172082in}}%
\pgfusepath{stroke}%
\end{pgfscope}%
\begin{pgfscope}%
\pgfpathrectangle{\pgfqpoint{0.800000in}{1.400000in}}{\pgfqpoint{2.407767in}{1.544118in}}%
\pgfusepath{clip}%
\pgfsetbuttcap%
\pgfsetroundjoin%
\pgfsetlinewidth{0.501875pt}%
\definecolor{currentstroke}{rgb}{0.280267,0.073417,0.397163}%
\pgfsetstrokecolor{currentstroke}%
\pgfsetdash{}{0pt}%
\pgfpathmoveto{\pgfqpoint{2.548107in}{2.172082in}}%
\pgfpathlineto{\pgfqpoint{2.495133in}{2.172293in}}%
\pgfusepath{stroke}%
\end{pgfscope}%
\begin{pgfscope}%
\pgfpathrectangle{\pgfqpoint{0.800000in}{1.400000in}}{\pgfqpoint{2.407767in}{1.544118in}}%
\pgfusepath{clip}%
\pgfsetbuttcap%
\pgfsetroundjoin%
\pgfsetlinewidth{0.501875pt}%
\definecolor{currentstroke}{rgb}{0.283091,0.110553,0.431554}%
\pgfsetstrokecolor{currentstroke}%
\pgfsetdash{}{0pt}%
\pgfpathmoveto{\pgfqpoint{2.495133in}{2.172293in}}%
\pgfpathlineto{\pgfqpoint{2.442157in}{2.172284in}}%
\pgfusepath{stroke}%
\end{pgfscope}%
\begin{pgfscope}%
\pgfpathrectangle{\pgfqpoint{0.800000in}{1.400000in}}{\pgfqpoint{2.407767in}{1.544118in}}%
\pgfusepath{clip}%
\pgfsetbuttcap%
\pgfsetroundjoin%
\pgfsetlinewidth{0.501875pt}%
\definecolor{currentstroke}{rgb}{0.282623,0.140926,0.457517}%
\pgfsetstrokecolor{currentstroke}%
\pgfsetdash{}{0pt}%
\pgfpathmoveto{\pgfqpoint{2.442157in}{2.172284in}}%
\pgfpathlineto{\pgfqpoint{2.389180in}{2.172246in}}%
\pgfusepath{stroke}%
\end{pgfscope}%
\begin{pgfscope}%
\pgfpathrectangle{\pgfqpoint{0.800000in}{1.400000in}}{\pgfqpoint{2.407767in}{1.544118in}}%
\pgfusepath{clip}%
\pgfsetbuttcap%
\pgfsetroundjoin%
\pgfsetlinewidth{0.501875pt}%
\definecolor{currentstroke}{rgb}{0.278012,0.180367,0.486697}%
\pgfsetstrokecolor{currentstroke}%
\pgfsetdash{}{0pt}%
\pgfpathmoveto{\pgfqpoint{2.389180in}{2.172246in}}%
\pgfpathlineto{\pgfqpoint{2.336204in}{2.172275in}}%
\pgfusepath{stroke}%
\end{pgfscope}%
\begin{pgfscope}%
\pgfpathrectangle{\pgfqpoint{0.800000in}{1.400000in}}{\pgfqpoint{2.407767in}{1.544118in}}%
\pgfusepath{clip}%
\pgfsetbuttcap%
\pgfsetroundjoin%
\pgfsetlinewidth{0.501875pt}%
\definecolor{currentstroke}{rgb}{0.267968,0.223549,0.512008}%
\pgfsetstrokecolor{currentstroke}%
\pgfsetdash{}{0pt}%
\pgfpathmoveto{\pgfqpoint{2.336204in}{2.172275in}}%
\pgfpathlineto{\pgfqpoint{2.283229in}{2.172248in}}%
\pgfusepath{stroke}%
\end{pgfscope}%
\begin{pgfscope}%
\pgfpathrectangle{\pgfqpoint{0.800000in}{1.400000in}}{\pgfqpoint{2.407767in}{1.544118in}}%
\pgfusepath{clip}%
\pgfsetbuttcap%
\pgfsetroundjoin%
\pgfsetlinewidth{0.501875pt}%
\definecolor{currentstroke}{rgb}{0.253935,0.265254,0.529983}%
\pgfsetstrokecolor{currentstroke}%
\pgfsetdash{}{0pt}%
\pgfpathmoveto{\pgfqpoint{2.283229in}{2.172248in}}%
\pgfpathlineto{\pgfqpoint{2.230254in}{2.172034in}}%
\pgfusepath{stroke}%
\end{pgfscope}%
\begin{pgfscope}%
\pgfpathrectangle{\pgfqpoint{0.800000in}{1.400000in}}{\pgfqpoint{2.407767in}{1.544118in}}%
\pgfusepath{clip}%
\pgfsetbuttcap%
\pgfsetroundjoin%
\pgfsetlinewidth{0.501875pt}%
\definecolor{currentstroke}{rgb}{0.248629,0.278775,0.534556}%
\pgfsetstrokecolor{currentstroke}%
\pgfsetdash{}{0pt}%
\pgfpathmoveto{\pgfqpoint{2.230254in}{2.172034in}}%
\pgfpathlineto{\pgfqpoint{2.177280in}{2.171703in}}%
\pgfusepath{stroke}%
\end{pgfscope}%
\begin{pgfscope}%
\pgfpathrectangle{\pgfqpoint{0.800000in}{1.400000in}}{\pgfqpoint{2.407767in}{1.544118in}}%
\pgfusepath{clip}%
\pgfsetbuttcap%
\pgfsetroundjoin%
\pgfsetlinewidth{0.501875pt}%
\definecolor{currentstroke}{rgb}{0.239346,0.300855,0.540844}%
\pgfsetstrokecolor{currentstroke}%
\pgfsetdash{}{0pt}%
\pgfpathmoveto{\pgfqpoint{2.177280in}{2.171703in}}%
\pgfpathlineto{\pgfqpoint{2.124308in}{2.171304in}}%
\pgfusepath{stroke}%
\end{pgfscope}%
\begin{pgfscope}%
\pgfpathrectangle{\pgfqpoint{0.800000in}{1.400000in}}{\pgfqpoint{2.407767in}{1.544118in}}%
\pgfusepath{clip}%
\pgfsetbuttcap%
\pgfsetroundjoin%
\pgfsetlinewidth{0.501875pt}%
\definecolor{currentstroke}{rgb}{0.227802,0.326594,0.546532}%
\pgfsetstrokecolor{currentstroke}%
\pgfsetdash{}{0pt}%
\pgfpathmoveto{\pgfqpoint{2.124308in}{2.171304in}}%
\pgfpathlineto{\pgfqpoint{2.071336in}{2.170900in}}%
\pgfusepath{stroke}%
\end{pgfscope}%
\begin{pgfscope}%
\pgfpathrectangle{\pgfqpoint{0.800000in}{1.400000in}}{\pgfqpoint{2.407767in}{1.544118in}}%
\pgfusepath{clip}%
\pgfsetbuttcap%
\pgfsetroundjoin%
\pgfsetlinewidth{0.501875pt}%
\definecolor{currentstroke}{rgb}{0.227802,0.326594,0.546532}%
\pgfsetstrokecolor{currentstroke}%
\pgfsetdash{}{0pt}%
\pgfpathmoveto{\pgfqpoint{2.071336in}{2.170900in}}%
\pgfpathlineto{\pgfqpoint{2.018372in}{2.170275in}}%
\pgfusepath{stroke}%
\end{pgfscope}%
\begin{pgfscope}%
\pgfpathrectangle{\pgfqpoint{0.800000in}{1.400000in}}{\pgfqpoint{2.407767in}{1.544118in}}%
\pgfusepath{clip}%
\pgfsetbuttcap%
\pgfsetroundjoin%
\pgfsetlinewidth{0.501875pt}%
\definecolor{currentstroke}{rgb}{0.210503,0.363727,0.552206}%
\pgfsetstrokecolor{currentstroke}%
\pgfsetdash{}{0pt}%
\pgfpathmoveto{\pgfqpoint{2.018372in}{2.170275in}}%
\pgfpathlineto{\pgfqpoint{1.965415in}{2.169395in}}%
\pgfusepath{stroke}%
\end{pgfscope}%
\begin{pgfscope}%
\pgfpathrectangle{\pgfqpoint{0.800000in}{1.400000in}}{\pgfqpoint{2.407767in}{1.544118in}}%
\pgfusepath{clip}%
\pgfsetbuttcap%
\pgfsetroundjoin%
\pgfsetlinewidth{0.501875pt}%
\definecolor{currentstroke}{rgb}{0.231674,0.318106,0.544834}%
\pgfsetstrokecolor{currentstroke}%
\pgfsetdash{}{0pt}%
\pgfpathmoveto{\pgfqpoint{1.965415in}{2.169395in}}%
\pgfpathlineto{\pgfqpoint{1.912470in}{2.168294in}}%
\pgfusepath{stroke}%
\end{pgfscope}%
\begin{pgfscope}%
\pgfpathrectangle{\pgfqpoint{0.800000in}{1.400000in}}{\pgfqpoint{2.407767in}{1.544118in}}%
\pgfusepath{clip}%
\pgfsetbuttcap%
\pgfsetroundjoin%
\pgfsetlinewidth{0.501875pt}%
\definecolor{currentstroke}{rgb}{0.233603,0.313828,0.543914}%
\pgfsetstrokecolor{currentstroke}%
\pgfsetdash{}{0pt}%
\pgfpathmoveto{\pgfqpoint{1.912470in}{2.168294in}}%
\pgfpathlineto{\pgfqpoint{1.859551in}{2.166818in}}%
\pgfusepath{stroke}%
\end{pgfscope}%
\begin{pgfscope}%
\pgfpathrectangle{\pgfqpoint{0.800000in}{1.400000in}}{\pgfqpoint{2.407767in}{1.544118in}}%
\pgfusepath{clip}%
\pgfsetbuttcap%
\pgfsetroundjoin%
\pgfsetlinewidth{0.501875pt}%
\definecolor{currentstroke}{rgb}{0.239346,0.300855,0.540844}%
\pgfsetstrokecolor{currentstroke}%
\pgfsetdash{}{0pt}%
\pgfpathmoveto{\pgfqpoint{1.859551in}{2.166818in}}%
\pgfpathlineto{\pgfqpoint{1.806660in}{2.165076in}}%
\pgfusepath{stroke}%
\end{pgfscope}%
\begin{pgfscope}%
\pgfpathrectangle{\pgfqpoint{0.800000in}{1.400000in}}{\pgfqpoint{2.407767in}{1.544118in}}%
\pgfusepath{clip}%
\pgfsetbuttcap%
\pgfsetroundjoin%
\pgfsetlinewidth{0.501875pt}%
\definecolor{currentstroke}{rgb}{0.243113,0.292092,0.538516}%
\pgfsetstrokecolor{currentstroke}%
\pgfsetdash{}{0pt}%
\pgfpathmoveto{\pgfqpoint{1.806660in}{2.165076in}}%
\pgfpathlineto{\pgfqpoint{1.753793in}{2.163112in}}%
\pgfusepath{stroke}%
\end{pgfscope}%
\begin{pgfscope}%
\pgfpathrectangle{\pgfqpoint{0.800000in}{1.400000in}}{\pgfqpoint{2.407767in}{1.544118in}}%
\pgfusepath{clip}%
\pgfsetbuttcap%
\pgfsetroundjoin%
\pgfsetlinewidth{0.501875pt}%
\definecolor{currentstroke}{rgb}{0.250425,0.274290,0.533103}%
\pgfsetstrokecolor{currentstroke}%
\pgfsetdash{}{0pt}%
\pgfpathmoveto{\pgfqpoint{1.753793in}{2.163112in}}%
\pgfpathlineto{\pgfqpoint{1.700964in}{2.160834in}}%
\pgfusepath{stroke}%
\end{pgfscope}%
\begin{pgfscope}%
\pgfpathrectangle{\pgfqpoint{0.800000in}{1.400000in}}{\pgfqpoint{2.407767in}{1.544118in}}%
\pgfusepath{clip}%
\pgfsetbuttcap%
\pgfsetroundjoin%
\pgfsetlinewidth{0.501875pt}%
\definecolor{currentstroke}{rgb}{0.260571,0.246922,0.522828}%
\pgfsetstrokecolor{currentstroke}%
\pgfsetdash{}{0pt}%
\pgfpathmoveto{\pgfqpoint{1.700964in}{2.160834in}}%
\pgfpathlineto{\pgfqpoint{1.648159in}{2.158385in}}%
\pgfusepath{stroke}%
\end{pgfscope}%
\begin{pgfscope}%
\pgfpathrectangle{\pgfqpoint{0.800000in}{1.400000in}}{\pgfqpoint{2.407767in}{1.544118in}}%
\pgfusepath{clip}%
\pgfsetbuttcap%
\pgfsetroundjoin%
\pgfsetlinewidth{0.501875pt}%
\definecolor{currentstroke}{rgb}{0.257322,0.256130,0.526563}%
\pgfsetstrokecolor{currentstroke}%
\pgfsetdash{}{0pt}%
\pgfpathmoveto{\pgfqpoint{1.648159in}{2.158385in}}%
\pgfpathlineto{\pgfqpoint{1.595443in}{2.155294in}}%
\pgfusepath{stroke}%
\end{pgfscope}%
\begin{pgfscope}%
\pgfpathrectangle{\pgfqpoint{0.800000in}{1.400000in}}{\pgfqpoint{2.407767in}{1.544118in}}%
\pgfusepath{clip}%
\pgfsetbuttcap%
\pgfsetroundjoin%
\pgfsetlinewidth{0.501875pt}%
\definecolor{currentstroke}{rgb}{0.269944,0.014625,0.341379}%
\pgfsetstrokecolor{currentstroke}%
\pgfsetdash{}{0pt}%
\pgfpathmoveto{\pgfqpoint{2.654046in}{2.206805in}}%
\pgfpathlineto{\pgfqpoint{2.601076in}{2.206721in}}%
\pgfusepath{stroke}%
\end{pgfscope}%
\begin{pgfscope}%
\pgfpathrectangle{\pgfqpoint{0.800000in}{1.400000in}}{\pgfqpoint{2.407767in}{1.544118in}}%
\pgfusepath{clip}%
\pgfsetbuttcap%
\pgfsetroundjoin%
\pgfsetlinewidth{0.501875pt}%
\definecolor{currentstroke}{rgb}{0.274952,0.037752,0.364543}%
\pgfsetstrokecolor{currentstroke}%
\pgfsetdash{}{0pt}%
\pgfpathmoveto{\pgfqpoint{2.601076in}{2.206721in}}%
\pgfpathlineto{\pgfqpoint{2.548104in}{2.206455in}}%
\pgfusepath{stroke}%
\end{pgfscope}%
\begin{pgfscope}%
\pgfpathrectangle{\pgfqpoint{0.800000in}{1.400000in}}{\pgfqpoint{2.407767in}{1.544118in}}%
\pgfusepath{clip}%
\pgfsetbuttcap%
\pgfsetroundjoin%
\pgfsetlinewidth{0.501875pt}%
\definecolor{currentstroke}{rgb}{0.279566,0.067836,0.391917}%
\pgfsetstrokecolor{currentstroke}%
\pgfsetdash{}{0pt}%
\pgfpathmoveto{\pgfqpoint{2.548104in}{2.206455in}}%
\pgfpathlineto{\pgfqpoint{2.495129in}{2.206329in}}%
\pgfusepath{stroke}%
\end{pgfscope}%
\begin{pgfscope}%
\pgfpathrectangle{\pgfqpoint{0.800000in}{1.400000in}}{\pgfqpoint{2.407767in}{1.544118in}}%
\pgfusepath{clip}%
\pgfsetbuttcap%
\pgfsetroundjoin%
\pgfsetlinewidth{0.501875pt}%
\definecolor{currentstroke}{rgb}{0.283091,0.110553,0.431554}%
\pgfsetstrokecolor{currentstroke}%
\pgfsetdash{}{0pt}%
\pgfpathmoveto{\pgfqpoint{2.495129in}{2.206329in}}%
\pgfpathlineto{\pgfqpoint{2.442156in}{2.206099in}}%
\pgfusepath{stroke}%
\end{pgfscope}%
\begin{pgfscope}%
\pgfpathrectangle{\pgfqpoint{0.800000in}{1.400000in}}{\pgfqpoint{2.407767in}{1.544118in}}%
\pgfusepath{clip}%
\pgfsetbuttcap%
\pgfsetroundjoin%
\pgfsetlinewidth{0.501875pt}%
\definecolor{currentstroke}{rgb}{0.283072,0.130895,0.449241}%
\pgfsetstrokecolor{currentstroke}%
\pgfsetdash{}{0pt}%
\pgfpathmoveto{\pgfqpoint{2.442156in}{2.206099in}}%
\pgfpathlineto{\pgfqpoint{2.389185in}{2.205728in}}%
\pgfusepath{stroke}%
\end{pgfscope}%
\begin{pgfscope}%
\pgfpathrectangle{\pgfqpoint{0.800000in}{1.400000in}}{\pgfqpoint{2.407767in}{1.544118in}}%
\pgfusepath{clip}%
\pgfsetbuttcap%
\pgfsetroundjoin%
\pgfsetlinewidth{0.501875pt}%
\definecolor{currentstroke}{rgb}{0.277134,0.185228,0.489898}%
\pgfsetstrokecolor{currentstroke}%
\pgfsetdash{}{0pt}%
\pgfpathmoveto{\pgfqpoint{2.389185in}{2.205728in}}%
\pgfpathlineto{\pgfqpoint{2.336215in}{2.205227in}}%
\pgfusepath{stroke}%
\end{pgfscope}%
\begin{pgfscope}%
\pgfpathrectangle{\pgfqpoint{0.800000in}{1.400000in}}{\pgfqpoint{2.407767in}{1.544118in}}%
\pgfusepath{clip}%
\pgfsetbuttcap%
\pgfsetroundjoin%
\pgfsetlinewidth{0.501875pt}%
\definecolor{currentstroke}{rgb}{0.274128,0.199721,0.498911}%
\pgfsetstrokecolor{currentstroke}%
\pgfsetdash{}{0pt}%
\pgfpathmoveto{\pgfqpoint{2.336215in}{2.205227in}}%
\pgfpathlineto{\pgfqpoint{2.283247in}{2.204632in}}%
\pgfusepath{stroke}%
\end{pgfscope}%
\begin{pgfscope}%
\pgfpathrectangle{\pgfqpoint{0.800000in}{1.400000in}}{\pgfqpoint{2.407767in}{1.544118in}}%
\pgfusepath{clip}%
\pgfsetbuttcap%
\pgfsetroundjoin%
\pgfsetlinewidth{0.501875pt}%
\definecolor{currentstroke}{rgb}{0.258965,0.251537,0.524736}%
\pgfsetstrokecolor{currentstroke}%
\pgfsetdash{}{0pt}%
\pgfpathmoveto{\pgfqpoint{2.283247in}{2.204632in}}%
\pgfpathlineto{\pgfqpoint{2.230285in}{2.203874in}}%
\pgfusepath{stroke}%
\end{pgfscope}%
\begin{pgfscope}%
\pgfpathrectangle{\pgfqpoint{0.800000in}{1.400000in}}{\pgfqpoint{2.407767in}{1.544118in}}%
\pgfusepath{clip}%
\pgfsetbuttcap%
\pgfsetroundjoin%
\pgfsetlinewidth{0.501875pt}%
\definecolor{currentstroke}{rgb}{0.248629,0.278775,0.534556}%
\pgfsetstrokecolor{currentstroke}%
\pgfsetdash{}{0pt}%
\pgfpathmoveto{\pgfqpoint{2.230285in}{2.203874in}}%
\pgfpathlineto{\pgfqpoint{2.177333in}{2.202858in}}%
\pgfusepath{stroke}%
\end{pgfscope}%
\begin{pgfscope}%
\pgfpathrectangle{\pgfqpoint{0.800000in}{1.400000in}}{\pgfqpoint{2.407767in}{1.544118in}}%
\pgfusepath{clip}%
\pgfsetbuttcap%
\pgfsetroundjoin%
\pgfsetlinewidth{0.501875pt}%
\definecolor{currentstroke}{rgb}{0.231674,0.318106,0.544834}%
\pgfsetstrokecolor{currentstroke}%
\pgfsetdash{}{0pt}%
\pgfpathmoveto{\pgfqpoint{2.177333in}{2.202858in}}%
\pgfpathlineto{\pgfqpoint{2.124390in}{2.201653in}}%
\pgfusepath{stroke}%
\end{pgfscope}%
\begin{pgfscope}%
\pgfpathrectangle{\pgfqpoint{0.800000in}{1.400000in}}{\pgfqpoint{2.407767in}{1.544118in}}%
\pgfusepath{clip}%
\pgfsetbuttcap%
\pgfsetroundjoin%
\pgfsetlinewidth{0.501875pt}%
\definecolor{currentstroke}{rgb}{0.237441,0.305202,0.541921}%
\pgfsetstrokecolor{currentstroke}%
\pgfsetdash{}{0pt}%
\pgfpathmoveto{\pgfqpoint{2.124390in}{2.201653in}}%
\pgfpathlineto{\pgfqpoint{2.071472in}{2.200096in}}%
\pgfusepath{stroke}%
\end{pgfscope}%
\begin{pgfscope}%
\pgfpathrectangle{\pgfqpoint{0.800000in}{1.400000in}}{\pgfqpoint{2.407767in}{1.544118in}}%
\pgfusepath{clip}%
\pgfsetbuttcap%
\pgfsetroundjoin%
\pgfsetlinewidth{0.501875pt}%
\definecolor{currentstroke}{rgb}{0.220057,0.343307,0.549413}%
\pgfsetstrokecolor{currentstroke}%
\pgfsetdash{}{0pt}%
\pgfpathmoveto{\pgfqpoint{2.071472in}{2.200096in}}%
\pgfpathlineto{\pgfqpoint{2.018605in}{2.197976in}}%
\pgfusepath{stroke}%
\end{pgfscope}%
\begin{pgfscope}%
\pgfpathrectangle{\pgfqpoint{0.800000in}{1.400000in}}{\pgfqpoint{2.407767in}{1.544118in}}%
\pgfusepath{clip}%
\pgfsetbuttcap%
\pgfsetroundjoin%
\pgfsetlinewidth{0.501875pt}%
\definecolor{currentstroke}{rgb}{0.241237,0.296485,0.539709}%
\pgfsetstrokecolor{currentstroke}%
\pgfsetdash{}{0pt}%
\pgfpathmoveto{\pgfqpoint{2.018605in}{2.197976in}}%
\pgfpathlineto{\pgfqpoint{1.965773in}{2.195514in}}%
\pgfusepath{stroke}%
\end{pgfscope}%
\begin{pgfscope}%
\pgfpathrectangle{\pgfqpoint{0.800000in}{1.400000in}}{\pgfqpoint{2.407767in}{1.544118in}}%
\pgfusepath{clip}%
\pgfsetbuttcap%
\pgfsetroundjoin%
\pgfsetlinewidth{0.501875pt}%
\definecolor{currentstroke}{rgb}{0.233603,0.313828,0.543914}%
\pgfsetstrokecolor{currentstroke}%
\pgfsetdash{}{0pt}%
\pgfpathmoveto{\pgfqpoint{1.965773in}{2.195514in}}%
\pgfpathlineto{\pgfqpoint{1.912993in}{2.192646in}}%
\pgfusepath{stroke}%
\end{pgfscope}%
\begin{pgfscope}%
\pgfpathrectangle{\pgfqpoint{0.800000in}{1.400000in}}{\pgfqpoint{2.407767in}{1.544118in}}%
\pgfusepath{clip}%
\pgfsetbuttcap%
\pgfsetroundjoin%
\pgfsetlinewidth{0.501875pt}%
\definecolor{currentstroke}{rgb}{0.271305,0.019942,0.347269}%
\pgfsetstrokecolor{currentstroke}%
\pgfsetdash{}{0pt}%
\pgfpathmoveto{\pgfqpoint{2.654046in}{2.241551in}}%
\pgfpathlineto{\pgfqpoint{2.601079in}{2.241396in}}%
\pgfusepath{stroke}%
\end{pgfscope}%
\begin{pgfscope}%
\pgfpathrectangle{\pgfqpoint{0.800000in}{1.400000in}}{\pgfqpoint{2.407767in}{1.544118in}}%
\pgfusepath{clip}%
\pgfsetbuttcap%
\pgfsetroundjoin%
\pgfsetlinewidth{0.501875pt}%
\definecolor{currentstroke}{rgb}{0.274952,0.037752,0.364543}%
\pgfsetstrokecolor{currentstroke}%
\pgfsetdash{}{0pt}%
\pgfpathmoveto{\pgfqpoint{2.601079in}{2.241396in}}%
\pgfpathlineto{\pgfqpoint{2.548111in}{2.241264in}}%
\pgfusepath{stroke}%
\end{pgfscope}%
\begin{pgfscope}%
\pgfpathrectangle{\pgfqpoint{0.800000in}{1.400000in}}{\pgfqpoint{2.407767in}{1.544118in}}%
\pgfusepath{clip}%
\pgfsetbuttcap%
\pgfsetroundjoin%
\pgfsetlinewidth{0.501875pt}%
\definecolor{currentstroke}{rgb}{0.279566,0.067836,0.391917}%
\pgfsetstrokecolor{currentstroke}%
\pgfsetdash{}{0pt}%
\pgfpathmoveto{\pgfqpoint{2.548111in}{2.241264in}}%
\pgfpathlineto{\pgfqpoint{2.495136in}{2.241099in}}%
\pgfusepath{stroke}%
\end{pgfscope}%
\begin{pgfscope}%
\pgfpathrectangle{\pgfqpoint{0.800000in}{1.400000in}}{\pgfqpoint{2.407767in}{1.544118in}}%
\pgfusepath{clip}%
\pgfsetbuttcap%
\pgfsetroundjoin%
\pgfsetlinewidth{0.501875pt}%
\definecolor{currentstroke}{rgb}{0.282910,0.105393,0.426902}%
\pgfsetstrokecolor{currentstroke}%
\pgfsetdash{}{0pt}%
\pgfpathmoveto{\pgfqpoint{2.495136in}{2.241099in}}%
\pgfpathlineto{\pgfqpoint{2.442164in}{2.240711in}}%
\pgfusepath{stroke}%
\end{pgfscope}%
\begin{pgfscope}%
\pgfpathrectangle{\pgfqpoint{0.800000in}{1.400000in}}{\pgfqpoint{2.407767in}{1.544118in}}%
\pgfusepath{clip}%
\pgfsetbuttcap%
\pgfsetroundjoin%
\pgfsetlinewidth{0.501875pt}%
\definecolor{currentstroke}{rgb}{0.282884,0.135920,0.453427}%
\pgfsetstrokecolor{currentstroke}%
\pgfsetdash{}{0pt}%
\pgfpathmoveto{\pgfqpoint{2.442164in}{2.240711in}}%
\pgfpathlineto{\pgfqpoint{2.389195in}{2.240216in}}%
\pgfusepath{stroke}%
\end{pgfscope}%
\begin{pgfscope}%
\pgfpathrectangle{\pgfqpoint{0.800000in}{1.400000in}}{\pgfqpoint{2.407767in}{1.544118in}}%
\pgfusepath{clip}%
\pgfsetbuttcap%
\pgfsetroundjoin%
\pgfsetlinewidth{0.501875pt}%
\definecolor{currentstroke}{rgb}{0.279574,0.170599,0.479997}%
\pgfsetstrokecolor{currentstroke}%
\pgfsetdash{}{0pt}%
\pgfpathmoveto{\pgfqpoint{2.389195in}{2.240216in}}%
\pgfpathlineto{\pgfqpoint{2.336229in}{2.239596in}}%
\pgfusepath{stroke}%
\end{pgfscope}%
\begin{pgfscope}%
\pgfpathrectangle{\pgfqpoint{0.800000in}{1.400000in}}{\pgfqpoint{2.407767in}{1.544118in}}%
\pgfusepath{clip}%
\pgfsetbuttcap%
\pgfsetroundjoin%
\pgfsetlinewidth{0.501875pt}%
\definecolor{currentstroke}{rgb}{0.271828,0.209303,0.504434}%
\pgfsetstrokecolor{currentstroke}%
\pgfsetdash{}{0pt}%
\pgfpathmoveto{\pgfqpoint{2.336229in}{2.239596in}}%
\pgfpathlineto{\pgfqpoint{2.283274in}{2.238658in}}%
\pgfusepath{stroke}%
\end{pgfscope}%
\begin{pgfscope}%
\pgfpathrectangle{\pgfqpoint{0.800000in}{1.400000in}}{\pgfqpoint{2.407767in}{1.544118in}}%
\pgfusepath{clip}%
\pgfsetbuttcap%
\pgfsetroundjoin%
\pgfsetlinewidth{0.501875pt}%
\definecolor{currentstroke}{rgb}{0.265145,0.232956,0.516599}%
\pgfsetstrokecolor{currentstroke}%
\pgfsetdash{}{0pt}%
\pgfpathmoveto{\pgfqpoint{2.283274in}{2.238658in}}%
\pgfpathlineto{\pgfqpoint{2.230331in}{2.237465in}}%
\pgfusepath{stroke}%
\end{pgfscope}%
\begin{pgfscope}%
\pgfpathrectangle{\pgfqpoint{0.800000in}{1.400000in}}{\pgfqpoint{2.407767in}{1.544118in}}%
\pgfusepath{clip}%
\pgfsetbuttcap%
\pgfsetroundjoin%
\pgfsetlinewidth{0.501875pt}%
\definecolor{currentstroke}{rgb}{0.260571,0.246922,0.522828}%
\pgfsetstrokecolor{currentstroke}%
\pgfsetdash{}{0pt}%
\pgfpathmoveto{\pgfqpoint{2.230331in}{2.237465in}}%
\pgfpathlineto{\pgfqpoint{2.177403in}{2.236038in}}%
\pgfusepath{stroke}%
\end{pgfscope}%
\begin{pgfscope}%
\pgfpathrectangle{\pgfqpoint{0.800000in}{1.400000in}}{\pgfqpoint{2.407767in}{1.544118in}}%
\pgfusepath{clip}%
\pgfsetbuttcap%
\pgfsetroundjoin%
\pgfsetlinewidth{0.501875pt}%
\definecolor{currentstroke}{rgb}{0.252194,0.269783,0.531579}%
\pgfsetstrokecolor{currentstroke}%
\pgfsetdash{}{0pt}%
\pgfpathmoveto{\pgfqpoint{2.177403in}{2.236038in}}%
\pgfpathlineto{\pgfqpoint{2.124505in}{2.234220in}}%
\pgfusepath{stroke}%
\end{pgfscope}%
\begin{pgfscope}%
\pgfpathrectangle{\pgfqpoint{0.800000in}{1.400000in}}{\pgfqpoint{2.407767in}{1.544118in}}%
\pgfusepath{clip}%
\pgfsetbuttcap%
\pgfsetroundjoin%
\pgfsetlinewidth{0.501875pt}%
\definecolor{currentstroke}{rgb}{0.244972,0.287675,0.537260}%
\pgfsetstrokecolor{currentstroke}%
\pgfsetdash{}{0pt}%
\pgfpathmoveto{\pgfqpoint{2.124505in}{2.234220in}}%
\pgfpathlineto{\pgfqpoint{2.071658in}{2.231877in}}%
\pgfusepath{stroke}%
\end{pgfscope}%
\begin{pgfscope}%
\pgfpathrectangle{\pgfqpoint{0.800000in}{1.400000in}}{\pgfqpoint{2.407767in}{1.544118in}}%
\pgfusepath{clip}%
\pgfsetbuttcap%
\pgfsetroundjoin%
\pgfsetlinewidth{0.501875pt}%
\definecolor{currentstroke}{rgb}{0.241237,0.296485,0.539709}%
\pgfsetstrokecolor{currentstroke}%
\pgfsetdash{}{0pt}%
\pgfpathmoveto{\pgfqpoint{2.071658in}{2.231877in}}%
\pgfpathlineto{\pgfqpoint{2.018864in}{2.229087in}}%
\pgfusepath{stroke}%
\end{pgfscope}%
\begin{pgfscope}%
\pgfpathrectangle{\pgfqpoint{0.800000in}{1.400000in}}{\pgfqpoint{2.407767in}{1.544118in}}%
\pgfusepath{clip}%
\pgfsetbuttcap%
\pgfsetroundjoin%
\pgfsetlinewidth{0.501875pt}%
\definecolor{currentstroke}{rgb}{0.237441,0.305202,0.541921}%
\pgfsetstrokecolor{currentstroke}%
\pgfsetdash{}{0pt}%
\pgfpathmoveto{\pgfqpoint{2.018864in}{2.229087in}}%
\pgfpathlineto{\pgfqpoint{1.966154in}{2.225739in}}%
\pgfusepath{stroke}%
\end{pgfscope}%
\begin{pgfscope}%
\pgfpathrectangle{\pgfqpoint{0.800000in}{1.400000in}}{\pgfqpoint{2.407767in}{1.544118in}}%
\pgfusepath{clip}%
\pgfsetbuttcap%
\pgfsetroundjoin%
\pgfsetlinewidth{0.501875pt}%
\definecolor{currentstroke}{rgb}{0.269944,0.014625,0.341379}%
\pgfsetstrokecolor{currentstroke}%
\pgfsetdash{}{0pt}%
\pgfpathmoveto{\pgfqpoint{2.654046in}{2.276297in}}%
\pgfpathlineto{\pgfqpoint{2.601076in}{2.276562in}}%
\pgfusepath{stroke}%
\end{pgfscope}%
\begin{pgfscope}%
\pgfpathrectangle{\pgfqpoint{0.800000in}{1.400000in}}{\pgfqpoint{2.407767in}{1.544118in}}%
\pgfusepath{clip}%
\pgfsetbuttcap%
\pgfsetroundjoin%
\pgfsetlinewidth{0.501875pt}%
\definecolor{currentstroke}{rgb}{0.274952,0.037752,0.364543}%
\pgfsetstrokecolor{currentstroke}%
\pgfsetdash{}{0pt}%
\pgfpathmoveto{\pgfqpoint{2.601076in}{2.276562in}}%
\pgfpathlineto{\pgfqpoint{2.548104in}{2.276737in}}%
\pgfusepath{stroke}%
\end{pgfscope}%
\begin{pgfscope}%
\pgfpathrectangle{\pgfqpoint{0.800000in}{1.400000in}}{\pgfqpoint{2.407767in}{1.544118in}}%
\pgfusepath{clip}%
\pgfsetbuttcap%
\pgfsetroundjoin%
\pgfsetlinewidth{0.501875pt}%
\definecolor{currentstroke}{rgb}{0.279566,0.067836,0.391917}%
\pgfsetstrokecolor{currentstroke}%
\pgfsetdash{}{0pt}%
\pgfpathmoveto{\pgfqpoint{2.548104in}{2.276737in}}%
\pgfpathlineto{\pgfqpoint{2.495131in}{2.276478in}}%
\pgfusepath{stroke}%
\end{pgfscope}%
\begin{pgfscope}%
\pgfpathrectangle{\pgfqpoint{0.800000in}{1.400000in}}{\pgfqpoint{2.407767in}{1.544118in}}%
\pgfusepath{clip}%
\pgfsetbuttcap%
\pgfsetroundjoin%
\pgfsetlinewidth{0.501875pt}%
\definecolor{currentstroke}{rgb}{0.282656,0.100196,0.422160}%
\pgfsetstrokecolor{currentstroke}%
\pgfsetdash{}{0pt}%
\pgfpathmoveto{\pgfqpoint{2.495131in}{2.276478in}}%
\pgfpathlineto{\pgfqpoint{2.442161in}{2.276003in}}%
\pgfusepath{stroke}%
\end{pgfscope}%
\begin{pgfscope}%
\pgfpathrectangle{\pgfqpoint{0.800000in}{1.400000in}}{\pgfqpoint{2.407767in}{1.544118in}}%
\pgfusepath{clip}%
\pgfsetbuttcap%
\pgfsetroundjoin%
\pgfsetlinewidth{0.501875pt}%
\definecolor{currentstroke}{rgb}{0.283229,0.120777,0.440584}%
\pgfsetstrokecolor{currentstroke}%
\pgfsetdash{}{0pt}%
\pgfpathmoveto{\pgfqpoint{2.442161in}{2.276003in}}%
\pgfpathlineto{\pgfqpoint{2.389196in}{2.275327in}}%
\pgfusepath{stroke}%
\end{pgfscope}%
\begin{pgfscope}%
\pgfpathrectangle{\pgfqpoint{0.800000in}{1.400000in}}{\pgfqpoint{2.407767in}{1.544118in}}%
\pgfusepath{clip}%
\pgfsetbuttcap%
\pgfsetroundjoin%
\pgfsetlinewidth{0.501875pt}%
\definecolor{currentstroke}{rgb}{0.280868,0.160771,0.472899}%
\pgfsetstrokecolor{currentstroke}%
\pgfsetdash{}{0pt}%
\pgfpathmoveto{\pgfqpoint{2.389196in}{2.275327in}}%
\pgfpathlineto{\pgfqpoint{2.336239in}{2.274412in}}%
\pgfusepath{stroke}%
\end{pgfscope}%
\begin{pgfscope}%
\pgfpathrectangle{\pgfqpoint{0.800000in}{1.400000in}}{\pgfqpoint{2.407767in}{1.544118in}}%
\pgfusepath{clip}%
\pgfsetbuttcap%
\pgfsetroundjoin%
\pgfsetlinewidth{0.501875pt}%
\definecolor{currentstroke}{rgb}{0.279574,0.170599,0.479997}%
\pgfsetstrokecolor{currentstroke}%
\pgfsetdash{}{0pt}%
\pgfpathmoveto{\pgfqpoint{2.336239in}{2.274412in}}%
\pgfpathlineto{\pgfqpoint{2.283300in}{2.273159in}}%
\pgfusepath{stroke}%
\end{pgfscope}%
\begin{pgfscope}%
\pgfpathrectangle{\pgfqpoint{0.800000in}{1.400000in}}{\pgfqpoint{2.407767in}{1.544118in}}%
\pgfusepath{clip}%
\pgfsetbuttcap%
\pgfsetroundjoin%
\pgfsetlinewidth{0.501875pt}%
\definecolor{currentstroke}{rgb}{0.269308,0.218818,0.509577}%
\pgfsetstrokecolor{currentstroke}%
\pgfsetdash{}{0pt}%
\pgfpathmoveto{\pgfqpoint{2.283300in}{2.273159in}}%
\pgfpathlineto{\pgfqpoint{2.230383in}{2.271562in}}%
\pgfusepath{stroke}%
\end{pgfscope}%
\begin{pgfscope}%
\pgfpathrectangle{\pgfqpoint{0.800000in}{1.400000in}}{\pgfqpoint{2.407767in}{1.544118in}}%
\pgfusepath{clip}%
\pgfsetbuttcap%
\pgfsetroundjoin%
\pgfsetlinewidth{0.501875pt}%
\definecolor{currentstroke}{rgb}{0.263663,0.237631,0.518762}%
\pgfsetstrokecolor{currentstroke}%
\pgfsetdash{}{0pt}%
\pgfpathmoveto{\pgfqpoint{2.230383in}{2.271562in}}%
\pgfpathlineto{\pgfqpoint{2.177505in}{2.269524in}}%
\pgfusepath{stroke}%
\end{pgfscope}%
\begin{pgfscope}%
\pgfpathrectangle{\pgfqpoint{0.800000in}{1.400000in}}{\pgfqpoint{2.407767in}{1.544118in}}%
\pgfusepath{clip}%
\pgfsetbuttcap%
\pgfsetroundjoin%
\pgfsetlinewidth{0.501875pt}%
\definecolor{currentstroke}{rgb}{0.265145,0.232956,0.516599}%
\pgfsetstrokecolor{currentstroke}%
\pgfsetdash{}{0pt}%
\pgfpathmoveto{\pgfqpoint{2.177505in}{2.269524in}}%
\pgfpathlineto{\pgfqpoint{2.124704in}{2.266810in}}%
\pgfusepath{stroke}%
\end{pgfscope}%
\begin{pgfscope}%
\pgfpathrectangle{\pgfqpoint{0.800000in}{1.400000in}}{\pgfqpoint{2.407767in}{1.544118in}}%
\pgfusepath{clip}%
\pgfsetbuttcap%
\pgfsetroundjoin%
\pgfsetlinewidth{0.501875pt}%
\definecolor{currentstroke}{rgb}{0.252194,0.269783,0.531579}%
\pgfsetstrokecolor{currentstroke}%
\pgfsetdash{}{0pt}%
\pgfpathmoveto{\pgfqpoint{2.124704in}{2.266810in}}%
\pgfpathlineto{\pgfqpoint{2.071992in}{2.263435in}}%
\pgfusepath{stroke}%
\end{pgfscope}%
\begin{pgfscope}%
\pgfpathrectangle{\pgfqpoint{0.800000in}{1.400000in}}{\pgfqpoint{2.407767in}{1.544118in}}%
\pgfusepath{clip}%
\pgfsetbuttcap%
\pgfsetroundjoin%
\pgfsetlinewidth{0.501875pt}%
\definecolor{currentstroke}{rgb}{0.253935,0.265254,0.529983}%
\pgfsetstrokecolor{currentstroke}%
\pgfsetdash{}{0pt}%
\pgfpathmoveto{\pgfqpoint{2.071992in}{2.263435in}}%
\pgfpathlineto{\pgfqpoint{2.019410in}{2.259338in}}%
\pgfusepath{stroke}%
\end{pgfscope}%
\begin{pgfscope}%
\pgfpathrectangle{\pgfqpoint{0.800000in}{1.400000in}}{\pgfqpoint{2.407767in}{1.544118in}}%
\pgfusepath{clip}%
\pgfsetbuttcap%
\pgfsetroundjoin%
\pgfsetlinewidth{0.501875pt}%
\definecolor{currentstroke}{rgb}{0.271305,0.019942,0.347269}%
\pgfsetstrokecolor{currentstroke}%
\pgfsetdash{}{0pt}%
\pgfpathmoveto{\pgfqpoint{2.654046in}{2.311043in}}%
\pgfpathlineto{\pgfqpoint{2.601073in}{2.311201in}}%
\pgfusepath{stroke}%
\end{pgfscope}%
\begin{pgfscope}%
\pgfpathrectangle{\pgfqpoint{0.800000in}{1.400000in}}{\pgfqpoint{2.407767in}{1.544118in}}%
\pgfusepath{clip}%
\pgfsetbuttcap%
\pgfsetroundjoin%
\pgfsetlinewidth{0.501875pt}%
\definecolor{currentstroke}{rgb}{0.274952,0.037752,0.364543}%
\pgfsetstrokecolor{currentstroke}%
\pgfsetdash{}{0pt}%
\pgfpathmoveto{\pgfqpoint{2.601073in}{2.311201in}}%
\pgfpathlineto{\pgfqpoint{2.548097in}{2.311073in}}%
\pgfusepath{stroke}%
\end{pgfscope}%
\begin{pgfscope}%
\pgfpathrectangle{\pgfqpoint{0.800000in}{1.400000in}}{\pgfqpoint{2.407767in}{1.544118in}}%
\pgfusepath{clip}%
\pgfsetbuttcap%
\pgfsetroundjoin%
\pgfsetlinewidth{0.501875pt}%
\definecolor{currentstroke}{rgb}{0.278791,0.062145,0.386592}%
\pgfsetstrokecolor{currentstroke}%
\pgfsetdash{}{0pt}%
\pgfpathmoveto{\pgfqpoint{2.548097in}{2.311073in}}%
\pgfpathlineto{\pgfqpoint{2.495123in}{2.310862in}}%
\pgfusepath{stroke}%
\end{pgfscope}%
\begin{pgfscope}%
\pgfpathrectangle{\pgfqpoint{0.800000in}{1.400000in}}{\pgfqpoint{2.407767in}{1.544118in}}%
\pgfusepath{clip}%
\pgfsetbuttcap%
\pgfsetroundjoin%
\pgfsetlinewidth{0.501875pt}%
\definecolor{currentstroke}{rgb}{0.282656,0.100196,0.422160}%
\pgfsetstrokecolor{currentstroke}%
\pgfsetdash{}{0pt}%
\pgfpathmoveto{\pgfqpoint{2.495123in}{2.310862in}}%
\pgfpathlineto{\pgfqpoint{2.442156in}{2.310307in}}%
\pgfusepath{stroke}%
\end{pgfscope}%
\begin{pgfscope}%
\pgfpathrectangle{\pgfqpoint{0.800000in}{1.400000in}}{\pgfqpoint{2.407767in}{1.544118in}}%
\pgfusepath{clip}%
\pgfsetbuttcap%
\pgfsetroundjoin%
\pgfsetlinewidth{0.501875pt}%
\definecolor{currentstroke}{rgb}{0.283197,0.115680,0.436115}%
\pgfsetstrokecolor{currentstroke}%
\pgfsetdash{}{0pt}%
\pgfpathmoveto{\pgfqpoint{2.442156in}{2.310307in}}%
\pgfpathlineto{\pgfqpoint{2.389205in}{2.309273in}}%
\pgfusepath{stroke}%
\end{pgfscope}%
\begin{pgfscope}%
\pgfpathrectangle{\pgfqpoint{0.800000in}{1.400000in}}{\pgfqpoint{2.407767in}{1.544118in}}%
\pgfusepath{clip}%
\pgfsetbuttcap%
\pgfsetroundjoin%
\pgfsetlinewidth{0.501875pt}%
\definecolor{currentstroke}{rgb}{0.282623,0.140926,0.457517}%
\pgfsetstrokecolor{currentstroke}%
\pgfsetdash{}{0pt}%
\pgfpathmoveto{\pgfqpoint{2.389205in}{2.309273in}}%
\pgfpathlineto{\pgfqpoint{2.336271in}{2.307930in}}%
\pgfusepath{stroke}%
\end{pgfscope}%
\begin{pgfscope}%
\pgfpathrectangle{\pgfqpoint{0.800000in}{1.400000in}}{\pgfqpoint{2.407767in}{1.544118in}}%
\pgfusepath{clip}%
\pgfsetbuttcap%
\pgfsetroundjoin%
\pgfsetlinewidth{0.501875pt}%
\definecolor{currentstroke}{rgb}{0.280868,0.160771,0.472899}%
\pgfsetstrokecolor{currentstroke}%
\pgfsetdash{}{0pt}%
\pgfpathmoveto{\pgfqpoint{2.336271in}{2.307930in}}%
\pgfpathlineto{\pgfqpoint{2.283359in}{2.306293in}}%
\pgfusepath{stroke}%
\end{pgfscope}%
\begin{pgfscope}%
\pgfpathrectangle{\pgfqpoint{0.800000in}{1.400000in}}{\pgfqpoint{2.407767in}{1.544118in}}%
\pgfusepath{clip}%
\pgfsetbuttcap%
\pgfsetroundjoin%
\pgfsetlinewidth{0.501875pt}%
\definecolor{currentstroke}{rgb}{0.278012,0.180367,0.486697}%
\pgfsetstrokecolor{currentstroke}%
\pgfsetdash{}{0pt}%
\pgfpathmoveto{\pgfqpoint{2.283359in}{2.306293in}}%
\pgfpathlineto{\pgfqpoint{2.230500in}{2.304099in}}%
\pgfusepath{stroke}%
\end{pgfscope}%
\begin{pgfscope}%
\pgfpathrectangle{\pgfqpoint{0.800000in}{1.400000in}}{\pgfqpoint{2.407767in}{1.544118in}}%
\pgfusepath{clip}%
\pgfsetbuttcap%
\pgfsetroundjoin%
\pgfsetlinewidth{0.501875pt}%
\definecolor{currentstroke}{rgb}{0.271828,0.209303,0.504434}%
\pgfsetstrokecolor{currentstroke}%
\pgfsetdash{}{0pt}%
\pgfpathmoveto{\pgfqpoint{2.230500in}{2.304099in}}%
\pgfpathlineto{\pgfqpoint{2.177732in}{2.301125in}}%
\pgfusepath{stroke}%
\end{pgfscope}%
\begin{pgfscope}%
\pgfpathrectangle{\pgfqpoint{0.800000in}{1.400000in}}{\pgfqpoint{2.407767in}{1.544118in}}%
\pgfusepath{clip}%
\pgfsetbuttcap%
\pgfsetroundjoin%
\pgfsetlinewidth{0.501875pt}%
\definecolor{currentstroke}{rgb}{0.271828,0.209303,0.504434}%
\pgfsetstrokecolor{currentstroke}%
\pgfsetdash{}{0pt}%
\pgfpathmoveto{\pgfqpoint{2.177732in}{2.301125in}}%
\pgfpathlineto{\pgfqpoint{2.125073in}{2.297453in}}%
\pgfusepath{stroke}%
\end{pgfscope}%
\begin{pgfscope}%
\pgfpathrectangle{\pgfqpoint{0.800000in}{1.400000in}}{\pgfqpoint{2.407767in}{1.544118in}}%
\pgfusepath{clip}%
\pgfsetbuttcap%
\pgfsetroundjoin%
\pgfsetlinewidth{0.501875pt}%
\definecolor{currentstroke}{rgb}{0.271305,0.019942,0.347269}%
\pgfsetstrokecolor{currentstroke}%
\pgfsetdash{}{0pt}%
\pgfpathmoveto{\pgfqpoint{2.654046in}{2.345789in}}%
\pgfpathlineto{\pgfqpoint{2.601078in}{2.345568in}}%
\pgfusepath{stroke}%
\end{pgfscope}%
\begin{pgfscope}%
\pgfpathrectangle{\pgfqpoint{0.800000in}{1.400000in}}{\pgfqpoint{2.407767in}{1.544118in}}%
\pgfusepath{clip}%
\pgfsetbuttcap%
\pgfsetroundjoin%
\pgfsetlinewidth{0.501875pt}%
\definecolor{currentstroke}{rgb}{0.273809,0.031497,0.358853}%
\pgfsetstrokecolor{currentstroke}%
\pgfsetdash{}{0pt}%
\pgfpathmoveto{\pgfqpoint{2.601078in}{2.345568in}}%
\pgfpathlineto{\pgfqpoint{2.548108in}{2.345209in}}%
\pgfusepath{stroke}%
\end{pgfscope}%
\begin{pgfscope}%
\pgfpathrectangle{\pgfqpoint{0.800000in}{1.400000in}}{\pgfqpoint{2.407767in}{1.544118in}}%
\pgfusepath{clip}%
\pgfsetbuttcap%
\pgfsetroundjoin%
\pgfsetlinewidth{0.501875pt}%
\definecolor{currentstroke}{rgb}{0.277941,0.056324,0.381191}%
\pgfsetstrokecolor{currentstroke}%
\pgfsetdash{}{0pt}%
\pgfpathmoveto{\pgfqpoint{2.548108in}{2.345209in}}%
\pgfpathlineto{\pgfqpoint{2.495135in}{2.344952in}}%
\pgfusepath{stroke}%
\end{pgfscope}%
\begin{pgfscope}%
\pgfpathrectangle{\pgfqpoint{0.800000in}{1.400000in}}{\pgfqpoint{2.407767in}{1.544118in}}%
\pgfusepath{clip}%
\pgfsetbuttcap%
\pgfsetroundjoin%
\pgfsetlinewidth{0.501875pt}%
\definecolor{currentstroke}{rgb}{0.281446,0.084320,0.407414}%
\pgfsetstrokecolor{currentstroke}%
\pgfsetdash{}{0pt}%
\pgfpathmoveto{\pgfqpoint{2.495135in}{2.344952in}}%
\pgfpathlineto{\pgfqpoint{2.442167in}{2.344416in}}%
\pgfusepath{stroke}%
\end{pgfscope}%
\begin{pgfscope}%
\pgfpathrectangle{\pgfqpoint{0.800000in}{1.400000in}}{\pgfqpoint{2.407767in}{1.544118in}}%
\pgfusepath{clip}%
\pgfsetbuttcap%
\pgfsetroundjoin%
\pgfsetlinewidth{0.501875pt}%
\definecolor{currentstroke}{rgb}{0.282656,0.100196,0.422160}%
\pgfsetstrokecolor{currentstroke}%
\pgfsetdash{}{0pt}%
\pgfpathmoveto{\pgfqpoint{2.442167in}{2.344416in}}%
\pgfpathlineto{\pgfqpoint{2.389214in}{2.343472in}}%
\pgfusepath{stroke}%
\end{pgfscope}%
\begin{pgfscope}%
\pgfpathrectangle{\pgfqpoint{0.800000in}{1.400000in}}{\pgfqpoint{2.407767in}{1.544118in}}%
\pgfusepath{clip}%
\pgfsetbuttcap%
\pgfsetroundjoin%
\pgfsetlinewidth{0.501875pt}%
\definecolor{currentstroke}{rgb}{0.283229,0.120777,0.440584}%
\pgfsetstrokecolor{currentstroke}%
\pgfsetdash{}{0pt}%
\pgfpathmoveto{\pgfqpoint{2.389214in}{2.343472in}}%
\pgfpathlineto{\pgfqpoint{2.336277in}{2.342175in}}%
\pgfusepath{stroke}%
\end{pgfscope}%
\begin{pgfscope}%
\pgfpathrectangle{\pgfqpoint{0.800000in}{1.400000in}}{\pgfqpoint{2.407767in}{1.544118in}}%
\pgfusepath{clip}%
\pgfsetbuttcap%
\pgfsetroundjoin%
\pgfsetlinewidth{0.501875pt}%
\definecolor{currentstroke}{rgb}{0.281887,0.150881,0.465405}%
\pgfsetstrokecolor{currentstroke}%
\pgfsetdash{}{0pt}%
\pgfpathmoveto{\pgfqpoint{2.336277in}{2.342175in}}%
\pgfpathlineto{\pgfqpoint{2.283382in}{2.340358in}}%
\pgfusepath{stroke}%
\end{pgfscope}%
\begin{pgfscope}%
\pgfpathrectangle{\pgfqpoint{0.800000in}{1.400000in}}{\pgfqpoint{2.407767in}{1.544118in}}%
\pgfusepath{clip}%
\pgfsetbuttcap%
\pgfsetroundjoin%
\pgfsetlinewidth{0.501875pt}%
\definecolor{currentstroke}{rgb}{0.280255,0.165693,0.476498}%
\pgfsetstrokecolor{currentstroke}%
\pgfsetdash{}{0pt}%
\pgfpathmoveto{\pgfqpoint{2.283382in}{2.340358in}}%
\pgfpathlineto{\pgfqpoint{2.230579in}{2.337651in}}%
\pgfusepath{stroke}%
\end{pgfscope}%
\begin{pgfscope}%
\pgfpathrectangle{\pgfqpoint{0.800000in}{1.400000in}}{\pgfqpoint{2.407767in}{1.544118in}}%
\pgfusepath{clip}%
\pgfsetbuttcap%
\pgfsetroundjoin%
\pgfsetlinewidth{0.501875pt}%
\definecolor{currentstroke}{rgb}{0.279574,0.170599,0.479997}%
\pgfsetstrokecolor{currentstroke}%
\pgfsetdash{}{0pt}%
\pgfpathmoveto{\pgfqpoint{2.230579in}{2.337651in}}%
\pgfpathlineto{\pgfqpoint{2.177900in}{2.334094in}}%
\pgfusepath{stroke}%
\end{pgfscope}%
\begin{pgfscope}%
\pgfpathrectangle{\pgfqpoint{0.800000in}{1.400000in}}{\pgfqpoint{2.407767in}{1.544118in}}%
\pgfusepath{clip}%
\pgfsetbuttcap%
\pgfsetroundjoin%
\pgfsetlinewidth{0.501875pt}%
\definecolor{currentstroke}{rgb}{0.280255,0.165693,0.476498}%
\pgfsetstrokecolor{currentstroke}%
\pgfsetdash{}{0pt}%
\pgfpathmoveto{\pgfqpoint{2.177900in}{2.334094in}}%
\pgfpathlineto{\pgfqpoint{2.125369in}{2.329723in}}%
\pgfusepath{stroke}%
\end{pgfscope}%
\begin{pgfscope}%
\pgfpathrectangle{\pgfqpoint{0.800000in}{1.400000in}}{\pgfqpoint{2.407767in}{1.544118in}}%
\pgfusepath{clip}%
\pgfsetbuttcap%
\pgfsetroundjoin%
\pgfsetlinewidth{0.501875pt}%
\definecolor{currentstroke}{rgb}{0.279574,0.170599,0.479997}%
\pgfsetstrokecolor{currentstroke}%
\pgfsetdash{}{0pt}%
\pgfpathmoveto{\pgfqpoint{2.125369in}{2.329723in}}%
\pgfpathlineto{\pgfqpoint{2.073032in}{2.324490in}}%
\pgfusepath{stroke}%
\end{pgfscope}%
\begin{pgfscope}%
\pgfpathrectangle{\pgfqpoint{0.800000in}{1.400000in}}{\pgfqpoint{2.407767in}{1.544118in}}%
\pgfusepath{clip}%
\pgfsetbuttcap%
\pgfsetroundjoin%
\pgfsetlinewidth{0.501875pt}%
\definecolor{currentstroke}{rgb}{0.274128,0.199721,0.498911}%
\pgfsetstrokecolor{currentstroke}%
\pgfsetdash{}{0pt}%
\pgfpathmoveto{\pgfqpoint{2.073032in}{2.324490in}}%
\pgfpathlineto{\pgfqpoint{2.021028in}{2.318067in}}%
\pgfusepath{stroke}%
\end{pgfscope}%
\begin{pgfscope}%
\pgfpathrectangle{\pgfqpoint{0.800000in}{1.400000in}}{\pgfqpoint{2.407767in}{1.544118in}}%
\pgfusepath{clip}%
\pgfsetbuttcap%
\pgfsetroundjoin%
\pgfsetlinewidth{0.501875pt}%
\definecolor{currentstroke}{rgb}{0.271828,0.209303,0.504434}%
\pgfsetstrokecolor{currentstroke}%
\pgfsetdash{}{0pt}%
\pgfpathmoveto{\pgfqpoint{2.021028in}{2.318067in}}%
\pgfpathlineto{\pgfqpoint{1.969504in}{2.310235in}}%
\pgfusepath{stroke}%
\end{pgfscope}%
\begin{pgfscope}%
\pgfpathrectangle{\pgfqpoint{0.800000in}{1.400000in}}{\pgfqpoint{2.407767in}{1.544118in}}%
\pgfusepath{clip}%
\pgfsetbuttcap%
\pgfsetroundjoin%
\pgfsetlinewidth{0.501875pt}%
\definecolor{currentstroke}{rgb}{0.271828,0.209303,0.504434}%
\pgfsetstrokecolor{currentstroke}%
\pgfsetdash{}{0pt}%
\pgfpathmoveto{\pgfqpoint{1.969504in}{2.310235in}}%
\pgfpathlineto{\pgfqpoint{1.918693in}{2.300728in}}%
\pgfusepath{stroke}%
\end{pgfscope}%
\begin{pgfscope}%
\pgfpathrectangle{\pgfqpoint{0.800000in}{1.400000in}}{\pgfqpoint{2.407767in}{1.544118in}}%
\pgfusepath{clip}%
\pgfsetbuttcap%
\pgfsetroundjoin%
\pgfsetlinewidth{0.501875pt}%
\definecolor{currentstroke}{rgb}{0.269308,0.218818,0.509577}%
\pgfsetstrokecolor{currentstroke}%
\pgfsetdash{}{0pt}%
\pgfpathmoveto{\pgfqpoint{1.918693in}{2.300728in}}%
\pgfpathlineto{\pgfqpoint{1.868839in}{2.289360in}}%
\pgfusepath{stroke}%
\end{pgfscope}%
\begin{pgfscope}%
\pgfpathrectangle{\pgfqpoint{0.800000in}{1.400000in}}{\pgfqpoint{2.407767in}{1.544118in}}%
\pgfusepath{clip}%
\pgfsetbuttcap%
\pgfsetroundjoin%
\pgfsetlinewidth{0.501875pt}%
\definecolor{currentstroke}{rgb}{0.266580,0.228262,0.514349}%
\pgfsetstrokecolor{currentstroke}%
\pgfsetdash{}{0pt}%
\pgfpathmoveto{\pgfqpoint{1.868839in}{2.289360in}}%
\pgfpathlineto{\pgfqpoint{1.819909in}{2.276436in}}%
\pgfusepath{stroke}%
\end{pgfscope}%
\begin{pgfscope}%
\pgfpathrectangle{\pgfqpoint{0.800000in}{1.400000in}}{\pgfqpoint{2.407767in}{1.544118in}}%
\pgfusepath{clip}%
\pgfsetbuttcap%
\pgfsetroundjoin%
\pgfsetlinewidth{0.501875pt}%
\definecolor{currentstroke}{rgb}{0.266580,0.228262,0.514349}%
\pgfsetstrokecolor{currentstroke}%
\pgfsetdash{}{0pt}%
\pgfpathmoveto{\pgfqpoint{1.819909in}{2.276436in}}%
\pgfpathlineto{\pgfqpoint{1.771693in}{2.262409in}}%
\pgfusepath{stroke}%
\end{pgfscope}%
\begin{pgfscope}%
\pgfpathrectangle{\pgfqpoint{0.800000in}{1.400000in}}{\pgfqpoint{2.407767in}{1.544118in}}%
\pgfusepath{clip}%
\pgfsetbuttcap%
\pgfsetroundjoin%
\pgfsetlinewidth{0.501875pt}%
\definecolor{currentstroke}{rgb}{0.263663,0.237631,0.518762}%
\pgfsetstrokecolor{currentstroke}%
\pgfsetdash{}{0pt}%
\pgfpathmoveto{\pgfqpoint{1.771693in}{2.262409in}}%
\pgfpathlineto{\pgfqpoint{1.724406in}{2.247148in}}%
\pgfusepath{stroke}%
\end{pgfscope}%
\begin{pgfscope}%
\pgfpathrectangle{\pgfqpoint{0.800000in}{1.400000in}}{\pgfqpoint{2.407767in}{1.544118in}}%
\pgfusepath{clip}%
\pgfsetbuttcap%
\pgfsetroundjoin%
\pgfsetlinewidth{0.501875pt}%
\definecolor{currentstroke}{rgb}{0.263663,0.237631,0.518762}%
\pgfsetstrokecolor{currentstroke}%
\pgfsetdash{}{0pt}%
\pgfpathmoveto{\pgfqpoint{1.724406in}{2.247148in}}%
\pgfpathlineto{\pgfqpoint{1.678103in}{2.230745in}}%
\pgfusepath{stroke}%
\end{pgfscope}%
\begin{pgfscope}%
\pgfpathrectangle{\pgfqpoint{0.800000in}{1.400000in}}{\pgfqpoint{2.407767in}{1.544118in}}%
\pgfusepath{clip}%
\pgfsetbuttcap%
\pgfsetroundjoin%
\pgfsetlinewidth{0.501875pt}%
\definecolor{currentstroke}{rgb}{0.271305,0.019942,0.347269}%
\pgfsetstrokecolor{currentstroke}%
\pgfsetdash{}{0pt}%
\pgfpathmoveto{\pgfqpoint{2.654046in}{2.380536in}}%
\pgfpathlineto{\pgfqpoint{2.601073in}{2.380553in}}%
\pgfusepath{stroke}%
\end{pgfscope}%
\begin{pgfscope}%
\pgfpathrectangle{\pgfqpoint{0.800000in}{1.400000in}}{\pgfqpoint{2.407767in}{1.544118in}}%
\pgfusepath{clip}%
\pgfsetbuttcap%
\pgfsetroundjoin%
\pgfsetlinewidth{0.501875pt}%
\definecolor{currentstroke}{rgb}{0.273809,0.031497,0.358853}%
\pgfsetstrokecolor{currentstroke}%
\pgfsetdash{}{0pt}%
\pgfpathmoveto{\pgfqpoint{2.601073in}{2.380553in}}%
\pgfpathlineto{\pgfqpoint{2.548099in}{2.380519in}}%
\pgfusepath{stroke}%
\end{pgfscope}%
\begin{pgfscope}%
\pgfpathrectangle{\pgfqpoint{0.800000in}{1.400000in}}{\pgfqpoint{2.407767in}{1.544118in}}%
\pgfusepath{clip}%
\pgfsetbuttcap%
\pgfsetroundjoin%
\pgfsetlinewidth{0.501875pt}%
\definecolor{currentstroke}{rgb}{0.277941,0.056324,0.381191}%
\pgfsetstrokecolor{currentstroke}%
\pgfsetdash{}{0pt}%
\pgfpathmoveto{\pgfqpoint{2.548099in}{2.380519in}}%
\pgfpathlineto{\pgfqpoint{2.495128in}{2.380375in}}%
\pgfusepath{stroke}%
\end{pgfscope}%
\begin{pgfscope}%
\pgfpathrectangle{\pgfqpoint{0.800000in}{1.400000in}}{\pgfqpoint{2.407767in}{1.544118in}}%
\pgfusepath{clip}%
\pgfsetbuttcap%
\pgfsetroundjoin%
\pgfsetlinewidth{0.501875pt}%
\definecolor{currentstroke}{rgb}{0.280894,0.078907,0.402329}%
\pgfsetstrokecolor{currentstroke}%
\pgfsetdash{}{0pt}%
\pgfpathmoveto{\pgfqpoint{2.495128in}{2.380375in}}%
\pgfpathlineto{\pgfqpoint{2.442170in}{2.379786in}}%
\pgfusepath{stroke}%
\end{pgfscope}%
\begin{pgfscope}%
\pgfpathrectangle{\pgfqpoint{0.800000in}{1.400000in}}{\pgfqpoint{2.407767in}{1.544118in}}%
\pgfusepath{clip}%
\pgfsetbuttcap%
\pgfsetroundjoin%
\pgfsetlinewidth{0.501875pt}%
\definecolor{currentstroke}{rgb}{0.282910,0.105393,0.426902}%
\pgfsetstrokecolor{currentstroke}%
\pgfsetdash{}{0pt}%
\pgfpathmoveto{\pgfqpoint{2.442170in}{2.379786in}}%
\pgfpathlineto{\pgfqpoint{2.389212in}{2.378946in}}%
\pgfusepath{stroke}%
\end{pgfscope}%
\begin{pgfscope}%
\pgfpathrectangle{\pgfqpoint{0.800000in}{1.400000in}}{\pgfqpoint{2.407767in}{1.544118in}}%
\pgfusepath{clip}%
\pgfsetbuttcap%
\pgfsetroundjoin%
\pgfsetlinewidth{0.501875pt}%
\definecolor{currentstroke}{rgb}{0.282910,0.105393,0.426902}%
\pgfsetstrokecolor{currentstroke}%
\pgfsetdash{}{0pt}%
\pgfpathmoveto{\pgfqpoint{2.389212in}{2.378946in}}%
\pgfpathlineto{\pgfqpoint{2.336253in}{2.378106in}}%
\pgfusepath{stroke}%
\end{pgfscope}%
\begin{pgfscope}%
\pgfpathrectangle{\pgfqpoint{0.800000in}{1.400000in}}{\pgfqpoint{2.407767in}{1.544118in}}%
\pgfusepath{clip}%
\pgfsetbuttcap%
\pgfsetroundjoin%
\pgfsetlinewidth{0.501875pt}%
\definecolor{currentstroke}{rgb}{0.282884,0.135920,0.453427}%
\pgfsetstrokecolor{currentstroke}%
\pgfsetdash{}{0pt}%
\pgfpathmoveto{\pgfqpoint{2.336253in}{2.378106in}}%
\pgfpathlineto{\pgfqpoint{2.283327in}{2.376701in}}%
\pgfusepath{stroke}%
\end{pgfscope}%
\begin{pgfscope}%
\pgfpathrectangle{\pgfqpoint{0.800000in}{1.400000in}}{\pgfqpoint{2.407767in}{1.544118in}}%
\pgfusepath{clip}%
\pgfsetbuttcap%
\pgfsetroundjoin%
\pgfsetlinewidth{0.501875pt}%
\definecolor{currentstroke}{rgb}{0.283072,0.130895,0.449241}%
\pgfsetstrokecolor{currentstroke}%
\pgfsetdash{}{0pt}%
\pgfpathmoveto{\pgfqpoint{2.283327in}{2.376701in}}%
\pgfpathlineto{\pgfqpoint{2.230492in}{2.374300in}}%
\pgfusepath{stroke}%
\end{pgfscope}%
\begin{pgfscope}%
\pgfpathrectangle{\pgfqpoint{0.800000in}{1.400000in}}{\pgfqpoint{2.407767in}{1.544118in}}%
\pgfusepath{clip}%
\pgfsetbuttcap%
\pgfsetroundjoin%
\pgfsetlinewidth{0.501875pt}%
\definecolor{currentstroke}{rgb}{0.282884,0.135920,0.453427}%
\pgfsetstrokecolor{currentstroke}%
\pgfsetdash{}{0pt}%
\pgfpathmoveto{\pgfqpoint{2.230492in}{2.374300in}}%
\pgfpathlineto{\pgfqpoint{2.177736in}{2.371213in}}%
\pgfusepath{stroke}%
\end{pgfscope}%
\begin{pgfscope}%
\pgfpathrectangle{\pgfqpoint{0.800000in}{1.400000in}}{\pgfqpoint{2.407767in}{1.544118in}}%
\pgfusepath{clip}%
\pgfsetbuttcap%
\pgfsetroundjoin%
\pgfsetlinewidth{0.501875pt}%
\definecolor{currentstroke}{rgb}{0.282623,0.140926,0.457517}%
\pgfsetstrokecolor{currentstroke}%
\pgfsetdash{}{0pt}%
\pgfpathmoveto{\pgfqpoint{2.177736in}{2.371213in}}%
\pgfpathlineto{\pgfqpoint{2.125173in}{2.367090in}}%
\pgfusepath{stroke}%
\end{pgfscope}%
\begin{pgfscope}%
\pgfpathrectangle{\pgfqpoint{0.800000in}{1.400000in}}{\pgfqpoint{2.407767in}{1.544118in}}%
\pgfusepath{clip}%
\pgfsetbuttcap%
\pgfsetroundjoin%
\pgfsetlinewidth{0.501875pt}%
\definecolor{currentstroke}{rgb}{0.283072,0.130895,0.449241}%
\pgfsetstrokecolor{currentstroke}%
\pgfsetdash{}{0pt}%
\pgfpathmoveto{\pgfqpoint{2.125173in}{2.367090in}}%
\pgfpathlineto{\pgfqpoint{2.073039in}{2.361131in}}%
\pgfusepath{stroke}%
\end{pgfscope}%
\begin{pgfscope}%
\pgfpathrectangle{\pgfqpoint{0.800000in}{1.400000in}}{\pgfqpoint{2.407767in}{1.544118in}}%
\pgfusepath{clip}%
\pgfsetbuttcap%
\pgfsetroundjoin%
\pgfsetlinewidth{0.501875pt}%
\definecolor{currentstroke}{rgb}{0.281412,0.155834,0.469201}%
\pgfsetstrokecolor{currentstroke}%
\pgfsetdash{}{0pt}%
\pgfpathmoveto{\pgfqpoint{2.073039in}{2.361131in}}%
\pgfpathlineto{\pgfqpoint{2.021347in}{2.353729in}}%
\pgfusepath{stroke}%
\end{pgfscope}%
\begin{pgfscope}%
\pgfpathrectangle{\pgfqpoint{0.800000in}{1.400000in}}{\pgfqpoint{2.407767in}{1.544118in}}%
\pgfusepath{clip}%
\pgfsetbuttcap%
\pgfsetroundjoin%
\pgfsetlinewidth{0.501875pt}%
\definecolor{currentstroke}{rgb}{0.282623,0.140926,0.457517}%
\pgfsetstrokecolor{currentstroke}%
\pgfsetdash{}{0pt}%
\pgfpathmoveto{\pgfqpoint{2.021347in}{2.353729in}}%
\pgfpathlineto{\pgfqpoint{1.970365in}{2.344623in}}%
\pgfusepath{stroke}%
\end{pgfscope}%
\begin{pgfscope}%
\pgfpathrectangle{\pgfqpoint{0.800000in}{1.400000in}}{\pgfqpoint{2.407767in}{1.544118in}}%
\pgfusepath{clip}%
\pgfsetbuttcap%
\pgfsetroundjoin%
\pgfsetlinewidth{0.501875pt}%
\definecolor{currentstroke}{rgb}{0.280255,0.165693,0.476498}%
\pgfsetstrokecolor{currentstroke}%
\pgfsetdash{}{0pt}%
\pgfpathmoveto{\pgfqpoint{1.970365in}{2.344623in}}%
\pgfpathlineto{\pgfqpoint{1.920891in}{2.332747in}}%
\pgfusepath{stroke}%
\end{pgfscope}%
\begin{pgfscope}%
\pgfpathrectangle{\pgfqpoint{0.800000in}{1.400000in}}{\pgfqpoint{2.407767in}{1.544118in}}%
\pgfusepath{clip}%
\pgfsetbuttcap%
\pgfsetroundjoin%
\pgfsetlinewidth{0.501875pt}%
\definecolor{currentstroke}{rgb}{0.269944,0.014625,0.341379}%
\pgfsetstrokecolor{currentstroke}%
\pgfsetdash{}{0pt}%
\pgfpathmoveto{\pgfqpoint{2.654046in}{2.415282in}}%
\pgfpathlineto{\pgfqpoint{2.601076in}{2.415409in}}%
\pgfusepath{stroke}%
\end{pgfscope}%
\begin{pgfscope}%
\pgfpathrectangle{\pgfqpoint{0.800000in}{1.400000in}}{\pgfqpoint{2.407767in}{1.544118in}}%
\pgfusepath{clip}%
\pgfsetbuttcap%
\pgfsetroundjoin%
\pgfsetlinewidth{0.501875pt}%
\definecolor{currentstroke}{rgb}{0.272594,0.025563,0.353093}%
\pgfsetstrokecolor{currentstroke}%
\pgfsetdash{}{0pt}%
\pgfpathmoveto{\pgfqpoint{2.601076in}{2.415409in}}%
\pgfpathlineto{\pgfqpoint{2.548103in}{2.415510in}}%
\pgfusepath{stroke}%
\end{pgfscope}%
\begin{pgfscope}%
\pgfpathrectangle{\pgfqpoint{0.800000in}{1.400000in}}{\pgfqpoint{2.407767in}{1.544118in}}%
\pgfusepath{clip}%
\pgfsetbuttcap%
\pgfsetroundjoin%
\pgfsetlinewidth{0.501875pt}%
\definecolor{currentstroke}{rgb}{0.277018,0.050344,0.375715}%
\pgfsetstrokecolor{currentstroke}%
\pgfsetdash{}{0pt}%
\pgfpathmoveto{\pgfqpoint{2.548103in}{2.415510in}}%
\pgfpathlineto{\pgfqpoint{2.495140in}{2.414889in}}%
\pgfusepath{stroke}%
\end{pgfscope}%
\begin{pgfscope}%
\pgfpathrectangle{\pgfqpoint{0.800000in}{1.400000in}}{\pgfqpoint{2.407767in}{1.544118in}}%
\pgfusepath{clip}%
\pgfsetbuttcap%
\pgfsetroundjoin%
\pgfsetlinewidth{0.501875pt}%
\definecolor{currentstroke}{rgb}{0.280267,0.073417,0.397163}%
\pgfsetstrokecolor{currentstroke}%
\pgfsetdash{}{0pt}%
\pgfpathmoveto{\pgfqpoint{2.495140in}{2.414889in}}%
\pgfpathlineto{\pgfqpoint{2.442184in}{2.413969in}}%
\pgfusepath{stroke}%
\end{pgfscope}%
\begin{pgfscope}%
\pgfpathrectangle{\pgfqpoint{0.800000in}{1.400000in}}{\pgfqpoint{2.407767in}{1.544118in}}%
\pgfusepath{clip}%
\pgfsetbuttcap%
\pgfsetroundjoin%
\pgfsetlinewidth{0.501875pt}%
\definecolor{currentstroke}{rgb}{0.281924,0.089666,0.412415}%
\pgfsetstrokecolor{currentstroke}%
\pgfsetdash{}{0pt}%
\pgfpathmoveto{\pgfqpoint{2.442184in}{2.413969in}}%
\pgfpathlineto{\pgfqpoint{2.389232in}{2.412975in}}%
\pgfusepath{stroke}%
\end{pgfscope}%
\begin{pgfscope}%
\pgfpathrectangle{\pgfqpoint{0.800000in}{1.400000in}}{\pgfqpoint{2.407767in}{1.544118in}}%
\pgfusepath{clip}%
\pgfsetbuttcap%
\pgfsetroundjoin%
\pgfsetlinewidth{0.501875pt}%
\definecolor{currentstroke}{rgb}{0.282327,0.094955,0.417331}%
\pgfsetstrokecolor{currentstroke}%
\pgfsetdash{}{0pt}%
\pgfpathmoveto{\pgfqpoint{2.389232in}{2.412975in}}%
\pgfpathlineto{\pgfqpoint{2.336300in}{2.411623in}}%
\pgfusepath{stroke}%
\end{pgfscope}%
\begin{pgfscope}%
\pgfpathrectangle{\pgfqpoint{0.800000in}{1.400000in}}{\pgfqpoint{2.407767in}{1.544118in}}%
\pgfusepath{clip}%
\pgfsetbuttcap%
\pgfsetroundjoin%
\pgfsetlinewidth{0.501875pt}%
\definecolor{currentstroke}{rgb}{0.283197,0.115680,0.436115}%
\pgfsetstrokecolor{currentstroke}%
\pgfsetdash{}{0pt}%
\pgfpathmoveto{\pgfqpoint{2.336300in}{2.411623in}}%
\pgfpathlineto{\pgfqpoint{2.283427in}{2.409602in}}%
\pgfusepath{stroke}%
\end{pgfscope}%
\begin{pgfscope}%
\pgfpathrectangle{\pgfqpoint{0.800000in}{1.400000in}}{\pgfqpoint{2.407767in}{1.544118in}}%
\pgfusepath{clip}%
\pgfsetbuttcap%
\pgfsetroundjoin%
\pgfsetlinewidth{0.501875pt}%
\definecolor{currentstroke}{rgb}{0.283229,0.120777,0.440584}%
\pgfsetstrokecolor{currentstroke}%
\pgfsetdash{}{0pt}%
\pgfpathmoveto{\pgfqpoint{2.283427in}{2.409602in}}%
\pgfpathlineto{\pgfqpoint{2.230665in}{2.406639in}}%
\pgfusepath{stroke}%
\end{pgfscope}%
\begin{pgfscope}%
\pgfpathrectangle{\pgfqpoint{0.800000in}{1.400000in}}{\pgfqpoint{2.407767in}{1.544118in}}%
\pgfusepath{clip}%
\pgfsetbuttcap%
\pgfsetroundjoin%
\pgfsetlinewidth{0.501875pt}%
\definecolor{currentstroke}{rgb}{0.283197,0.115680,0.436115}%
\pgfsetstrokecolor{currentstroke}%
\pgfsetdash{}{0pt}%
\pgfpathmoveto{\pgfqpoint{2.230665in}{2.406639in}}%
\pgfpathlineto{\pgfqpoint{2.178008in}{2.402966in}}%
\pgfusepath{stroke}%
\end{pgfscope}%
\begin{pgfscope}%
\pgfpathrectangle{\pgfqpoint{0.800000in}{1.400000in}}{\pgfqpoint{2.407767in}{1.544118in}}%
\pgfusepath{clip}%
\pgfsetbuttcap%
\pgfsetroundjoin%
\pgfsetlinewidth{0.501875pt}%
\definecolor{currentstroke}{rgb}{0.283072,0.130895,0.449241}%
\pgfsetstrokecolor{currentstroke}%
\pgfsetdash{}{0pt}%
\pgfpathmoveto{\pgfqpoint{2.178008in}{2.402966in}}%
\pgfpathlineto{\pgfqpoint{2.125516in}{2.398449in}}%
\pgfusepath{stroke}%
\end{pgfscope}%
\begin{pgfscope}%
\pgfpathrectangle{\pgfqpoint{0.800000in}{1.400000in}}{\pgfqpoint{2.407767in}{1.544118in}}%
\pgfusepath{clip}%
\pgfsetbuttcap%
\pgfsetroundjoin%
\pgfsetlinewidth{0.501875pt}%
\definecolor{currentstroke}{rgb}{0.282910,0.105393,0.426902}%
\pgfsetstrokecolor{currentstroke}%
\pgfsetdash{}{0pt}%
\pgfpathmoveto{\pgfqpoint{2.125516in}{2.398449in}}%
\pgfpathlineto{\pgfqpoint{2.073415in}{2.392420in}}%
\pgfusepath{stroke}%
\end{pgfscope}%
\begin{pgfscope}%
\pgfpathrectangle{\pgfqpoint{0.800000in}{1.400000in}}{\pgfqpoint{2.407767in}{1.544118in}}%
\pgfusepath{clip}%
\pgfsetbuttcap%
\pgfsetroundjoin%
\pgfsetlinewidth{0.501875pt}%
\definecolor{currentstroke}{rgb}{0.283229,0.120777,0.440584}%
\pgfsetstrokecolor{currentstroke}%
\pgfsetdash{}{0pt}%
\pgfpathmoveto{\pgfqpoint{2.073415in}{2.392420in}}%
\pgfpathlineto{\pgfqpoint{2.021902in}{2.384576in}}%
\pgfusepath{stroke}%
\end{pgfscope}%
\begin{pgfscope}%
\pgfpathrectangle{\pgfqpoint{0.800000in}{1.400000in}}{\pgfqpoint{2.407767in}{1.544118in}}%
\pgfusepath{clip}%
\pgfsetbuttcap%
\pgfsetroundjoin%
\pgfsetlinewidth{0.501875pt}%
\definecolor{currentstroke}{rgb}{0.283187,0.125848,0.444960}%
\pgfsetstrokecolor{currentstroke}%
\pgfsetdash{}{0pt}%
\pgfpathmoveto{\pgfqpoint{2.021902in}{2.384576in}}%
\pgfpathlineto{\pgfqpoint{1.971531in}{2.374231in}}%
\pgfusepath{stroke}%
\end{pgfscope}%
\begin{pgfscope}%
\pgfpathrectangle{\pgfqpoint{0.800000in}{1.400000in}}{\pgfqpoint{2.407767in}{1.544118in}}%
\pgfusepath{clip}%
\pgfsetbuttcap%
\pgfsetroundjoin%
\pgfsetlinewidth{0.501875pt}%
\definecolor{currentstroke}{rgb}{0.269944,0.014625,0.341379}%
\pgfsetstrokecolor{currentstroke}%
\pgfsetdash{}{0pt}%
\pgfpathmoveto{\pgfqpoint{2.654046in}{2.450028in}}%
\pgfpathlineto{\pgfqpoint{2.601074in}{2.449830in}}%
\pgfusepath{stroke}%
\end{pgfscope}%
\begin{pgfscope}%
\pgfpathrectangle{\pgfqpoint{0.800000in}{1.400000in}}{\pgfqpoint{2.407767in}{1.544118in}}%
\pgfusepath{clip}%
\pgfsetbuttcap%
\pgfsetroundjoin%
\pgfsetlinewidth{0.501875pt}%
\definecolor{currentstroke}{rgb}{0.272594,0.025563,0.353093}%
\pgfsetstrokecolor{currentstroke}%
\pgfsetdash{}{0pt}%
\pgfpathmoveto{\pgfqpoint{2.601074in}{2.449830in}}%
\pgfpathlineto{\pgfqpoint{2.548101in}{2.449560in}}%
\pgfusepath{stroke}%
\end{pgfscope}%
\begin{pgfscope}%
\pgfpathrectangle{\pgfqpoint{0.800000in}{1.400000in}}{\pgfqpoint{2.407767in}{1.544118in}}%
\pgfusepath{clip}%
\pgfsetbuttcap%
\pgfsetroundjoin%
\pgfsetlinewidth{0.501875pt}%
\definecolor{currentstroke}{rgb}{0.277941,0.056324,0.381191}%
\pgfsetstrokecolor{currentstroke}%
\pgfsetdash{}{0pt}%
\pgfpathmoveto{\pgfqpoint{2.548101in}{2.449560in}}%
\pgfpathlineto{\pgfqpoint{2.495133in}{2.449050in}}%
\pgfusepath{stroke}%
\end{pgfscope}%
\begin{pgfscope}%
\pgfpathrectangle{\pgfqpoint{0.800000in}{1.400000in}}{\pgfqpoint{2.407767in}{1.544118in}}%
\pgfusepath{clip}%
\pgfsetbuttcap%
\pgfsetroundjoin%
\pgfsetlinewidth{0.501875pt}%
\definecolor{currentstroke}{rgb}{0.280267,0.073417,0.397163}%
\pgfsetstrokecolor{currentstroke}%
\pgfsetdash{}{0pt}%
\pgfpathmoveto{\pgfqpoint{2.495133in}{2.449050in}}%
\pgfpathlineto{\pgfqpoint{2.442171in}{2.448343in}}%
\pgfusepath{stroke}%
\end{pgfscope}%
\begin{pgfscope}%
\pgfpathrectangle{\pgfqpoint{0.800000in}{1.400000in}}{\pgfqpoint{2.407767in}{1.544118in}}%
\pgfusepath{clip}%
\pgfsetbuttcap%
\pgfsetroundjoin%
\pgfsetlinewidth{0.501875pt}%
\definecolor{currentstroke}{rgb}{0.282327,0.094955,0.417331}%
\pgfsetstrokecolor{currentstroke}%
\pgfsetdash{}{0pt}%
\pgfpathmoveto{\pgfqpoint{2.442171in}{2.448343in}}%
\pgfpathlineto{\pgfqpoint{2.389232in}{2.447200in}}%
\pgfusepath{stroke}%
\end{pgfscope}%
\begin{pgfscope}%
\pgfpathrectangle{\pgfqpoint{0.800000in}{1.400000in}}{\pgfqpoint{2.407767in}{1.544118in}}%
\pgfusepath{clip}%
\pgfsetbuttcap%
\pgfsetroundjoin%
\pgfsetlinewidth{0.501875pt}%
\definecolor{currentstroke}{rgb}{0.282656,0.100196,0.422160}%
\pgfsetstrokecolor{currentstroke}%
\pgfsetdash{}{0pt}%
\pgfpathmoveto{\pgfqpoint{2.389232in}{2.447200in}}%
\pgfpathlineto{\pgfqpoint{2.336324in}{2.445510in}}%
\pgfusepath{stroke}%
\end{pgfscope}%
\begin{pgfscope}%
\pgfpathrectangle{\pgfqpoint{0.800000in}{1.400000in}}{\pgfqpoint{2.407767in}{1.544118in}}%
\pgfusepath{clip}%
\pgfsetbuttcap%
\pgfsetroundjoin%
\pgfsetlinewidth{0.501875pt}%
\definecolor{currentstroke}{rgb}{0.281446,0.084320,0.407414}%
\pgfsetstrokecolor{currentstroke}%
\pgfsetdash{}{0pt}%
\pgfpathmoveto{\pgfqpoint{2.336324in}{2.445510in}}%
\pgfpathlineto{\pgfqpoint{2.283490in}{2.443093in}}%
\pgfusepath{stroke}%
\end{pgfscope}%
\begin{pgfscope}%
\pgfpathrectangle{\pgfqpoint{0.800000in}{1.400000in}}{\pgfqpoint{2.407767in}{1.544118in}}%
\pgfusepath{clip}%
\pgfsetbuttcap%
\pgfsetroundjoin%
\pgfsetlinewidth{0.501875pt}%
\definecolor{currentstroke}{rgb}{0.282910,0.105393,0.426902}%
\pgfsetstrokecolor{currentstroke}%
\pgfsetdash{}{0pt}%
\pgfpathmoveto{\pgfqpoint{2.283490in}{2.443093in}}%
\pgfpathlineto{\pgfqpoint{2.230741in}{2.440024in}}%
\pgfusepath{stroke}%
\end{pgfscope}%
\begin{pgfscope}%
\pgfpathrectangle{\pgfqpoint{0.800000in}{1.400000in}}{\pgfqpoint{2.407767in}{1.544118in}}%
\pgfusepath{clip}%
\pgfsetbuttcap%
\pgfsetroundjoin%
\pgfsetlinewidth{0.501875pt}%
\definecolor{currentstroke}{rgb}{0.282656,0.100196,0.422160}%
\pgfsetstrokecolor{currentstroke}%
\pgfsetdash{}{0pt}%
\pgfpathmoveto{\pgfqpoint{2.230741in}{2.440024in}}%
\pgfpathlineto{\pgfqpoint{2.178170in}{2.435970in}}%
\pgfusepath{stroke}%
\end{pgfscope}%
\begin{pgfscope}%
\pgfpathrectangle{\pgfqpoint{0.800000in}{1.400000in}}{\pgfqpoint{2.407767in}{1.544118in}}%
\pgfusepath{clip}%
\pgfsetbuttcap%
\pgfsetroundjoin%
\pgfsetlinewidth{0.501875pt}%
\definecolor{currentstroke}{rgb}{0.271305,0.019942,0.347269}%
\pgfsetstrokecolor{currentstroke}%
\pgfsetdash{}{0pt}%
\pgfpathmoveto{\pgfqpoint{2.654046in}{2.484774in}}%
\pgfpathlineto{\pgfqpoint{2.601071in}{2.484788in}}%
\pgfusepath{stroke}%
\end{pgfscope}%
\begin{pgfscope}%
\pgfpathrectangle{\pgfqpoint{0.800000in}{1.400000in}}{\pgfqpoint{2.407767in}{1.544118in}}%
\pgfusepath{clip}%
\pgfsetbuttcap%
\pgfsetroundjoin%
\pgfsetlinewidth{0.501875pt}%
\definecolor{currentstroke}{rgb}{0.272594,0.025563,0.353093}%
\pgfsetstrokecolor{currentstroke}%
\pgfsetdash{}{0pt}%
\pgfpathmoveto{\pgfqpoint{2.601071in}{2.484788in}}%
\pgfpathlineto{\pgfqpoint{2.548102in}{2.484435in}}%
\pgfusepath{stroke}%
\end{pgfscope}%
\begin{pgfscope}%
\pgfpathrectangle{\pgfqpoint{0.800000in}{1.400000in}}{\pgfqpoint{2.407767in}{1.544118in}}%
\pgfusepath{clip}%
\pgfsetbuttcap%
\pgfsetroundjoin%
\pgfsetlinewidth{0.501875pt}%
\definecolor{currentstroke}{rgb}{0.276022,0.044167,0.370164}%
\pgfsetstrokecolor{currentstroke}%
\pgfsetdash{}{0pt}%
\pgfpathmoveto{\pgfqpoint{2.548102in}{2.484435in}}%
\pgfpathlineto{\pgfqpoint{2.495135in}{2.483808in}}%
\pgfusepath{stroke}%
\end{pgfscope}%
\begin{pgfscope}%
\pgfpathrectangle{\pgfqpoint{0.800000in}{1.400000in}}{\pgfqpoint{2.407767in}{1.544118in}}%
\pgfusepath{clip}%
\pgfsetbuttcap%
\pgfsetroundjoin%
\pgfsetlinewidth{0.501875pt}%
\definecolor{currentstroke}{rgb}{0.278791,0.062145,0.386592}%
\pgfsetstrokecolor{currentstroke}%
\pgfsetdash{}{0pt}%
\pgfpathmoveto{\pgfqpoint{2.495135in}{2.483808in}}%
\pgfpathlineto{\pgfqpoint{2.442179in}{2.482938in}}%
\pgfusepath{stroke}%
\end{pgfscope}%
\begin{pgfscope}%
\pgfpathrectangle{\pgfqpoint{0.800000in}{1.400000in}}{\pgfqpoint{2.407767in}{1.544118in}}%
\pgfusepath{clip}%
\pgfsetbuttcap%
\pgfsetroundjoin%
\pgfsetlinewidth{0.501875pt}%
\definecolor{currentstroke}{rgb}{0.281446,0.084320,0.407414}%
\pgfsetstrokecolor{currentstroke}%
\pgfsetdash{}{0pt}%
\pgfpathmoveto{\pgfqpoint{2.442179in}{2.482938in}}%
\pgfpathlineto{\pgfqpoint{2.389248in}{2.481580in}}%
\pgfusepath{stroke}%
\end{pgfscope}%
\begin{pgfscope}%
\pgfpathrectangle{\pgfqpoint{0.800000in}{1.400000in}}{\pgfqpoint{2.407767in}{1.544118in}}%
\pgfusepath{clip}%
\pgfsetbuttcap%
\pgfsetroundjoin%
\pgfsetlinewidth{0.501875pt}%
\definecolor{currentstroke}{rgb}{0.280267,0.073417,0.397163}%
\pgfsetstrokecolor{currentstroke}%
\pgfsetdash{}{0pt}%
\pgfpathmoveto{\pgfqpoint{2.389248in}{2.481580in}}%
\pgfpathlineto{\pgfqpoint{2.336320in}{2.480540in}}%
\pgfusepath{stroke}%
\end{pgfscope}%
\begin{pgfscope}%
\pgfpathrectangle{\pgfqpoint{0.800000in}{1.400000in}}{\pgfqpoint{2.407767in}{1.544118in}}%
\pgfusepath{clip}%
\pgfsetbuttcap%
\pgfsetroundjoin%
\pgfsetlinewidth{0.501875pt}%
\definecolor{currentstroke}{rgb}{0.278791,0.062145,0.386592}%
\pgfsetstrokecolor{currentstroke}%
\pgfsetdash{}{0pt}%
\pgfpathmoveto{\pgfqpoint{2.336320in}{2.480540in}}%
\pgfpathlineto{\pgfqpoint{2.283446in}{2.478945in}}%
\pgfusepath{stroke}%
\end{pgfscope}%
\begin{pgfscope}%
\pgfpathrectangle{\pgfqpoint{0.800000in}{1.400000in}}{\pgfqpoint{2.407767in}{1.544118in}}%
\pgfusepath{clip}%
\pgfsetbuttcap%
\pgfsetroundjoin%
\pgfsetlinewidth{0.501875pt}%
\definecolor{currentstroke}{rgb}{0.282327,0.094955,0.417331}%
\pgfsetstrokecolor{currentstroke}%
\pgfsetdash{}{0pt}%
\pgfpathmoveto{\pgfqpoint{2.283446in}{2.478945in}}%
\pgfpathlineto{\pgfqpoint{2.230755in}{2.475589in}}%
\pgfusepath{stroke}%
\end{pgfscope}%
\begin{pgfscope}%
\pgfpathrectangle{\pgfqpoint{0.800000in}{1.400000in}}{\pgfqpoint{2.407767in}{1.544118in}}%
\pgfusepath{clip}%
\pgfsetbuttcap%
\pgfsetroundjoin%
\pgfsetlinewidth{0.501875pt}%
\definecolor{currentstroke}{rgb}{0.282910,0.105393,0.426902}%
\pgfsetstrokecolor{currentstroke}%
\pgfsetdash{}{0pt}%
\pgfpathmoveto{\pgfqpoint{2.230755in}{2.475589in}}%
\pgfpathlineto{\pgfqpoint{2.178256in}{2.471101in}}%
\pgfusepath{stroke}%
\end{pgfscope}%
\begin{pgfscope}%
\pgfpathrectangle{\pgfqpoint{0.800000in}{1.400000in}}{\pgfqpoint{2.407767in}{1.544118in}}%
\pgfusepath{clip}%
\pgfsetbuttcap%
\pgfsetroundjoin%
\pgfsetlinewidth{0.501875pt}%
\definecolor{currentstroke}{rgb}{0.279566,0.067836,0.391917}%
\pgfsetstrokecolor{currentstroke}%
\pgfsetdash{}{0pt}%
\pgfpathmoveto{\pgfqpoint{2.178256in}{2.471101in}}%
\pgfpathlineto{\pgfqpoint{2.126305in}{2.464889in}}%
\pgfusepath{stroke}%
\end{pgfscope}%
\begin{pgfscope}%
\pgfpathrectangle{\pgfqpoint{0.800000in}{1.400000in}}{\pgfqpoint{2.407767in}{1.544118in}}%
\pgfusepath{clip}%
\pgfsetbuttcap%
\pgfsetroundjoin%
\pgfsetlinewidth{0.501875pt}%
\definecolor{currentstroke}{rgb}{0.281446,0.084320,0.407414}%
\pgfsetstrokecolor{currentstroke}%
\pgfsetdash{}{0pt}%
\pgfpathmoveto{\pgfqpoint{2.126305in}{2.464889in}}%
\pgfpathlineto{\pgfqpoint{2.075017in}{2.456596in}}%
\pgfusepath{stroke}%
\end{pgfscope}%
\begin{pgfscope}%
\pgfpathrectangle{\pgfqpoint{0.800000in}{1.400000in}}{\pgfqpoint{2.407767in}{1.544118in}}%
\pgfusepath{clip}%
\pgfsetbuttcap%
\pgfsetroundjoin%
\pgfsetlinewidth{0.501875pt}%
\definecolor{currentstroke}{rgb}{0.281924,0.089666,0.412415}%
\pgfsetstrokecolor{currentstroke}%
\pgfsetdash{}{0pt}%
\pgfpathmoveto{\pgfqpoint{2.075017in}{2.456596in}}%
\pgfpathlineto{\pgfqpoint{2.025142in}{2.445355in}}%
\pgfusepath{stroke}%
\end{pgfscope}%
\begin{pgfscope}%
\pgfpathrectangle{\pgfqpoint{0.800000in}{1.400000in}}{\pgfqpoint{2.407767in}{1.544118in}}%
\pgfusepath{clip}%
\pgfsetbuttcap%
\pgfsetroundjoin%
\pgfsetlinewidth{0.501875pt}%
\definecolor{currentstroke}{rgb}{0.281446,0.084320,0.407414}%
\pgfsetstrokecolor{currentstroke}%
\pgfsetdash{}{0pt}%
\pgfpathmoveto{\pgfqpoint{2.025142in}{2.445355in}}%
\pgfpathlineto{\pgfqpoint{1.977269in}{2.431018in}}%
\pgfusepath{stroke}%
\end{pgfscope}%
\begin{pgfscope}%
\pgfpathrectangle{\pgfqpoint{0.800000in}{1.400000in}}{\pgfqpoint{2.407767in}{1.544118in}}%
\pgfusepath{clip}%
\pgfsetbuttcap%
\pgfsetroundjoin%
\pgfsetlinewidth{0.501875pt}%
\definecolor{currentstroke}{rgb}{0.281446,0.084320,0.407414}%
\pgfsetstrokecolor{currentstroke}%
\pgfsetdash{}{0pt}%
\pgfpathmoveto{\pgfqpoint{1.977269in}{2.431018in}}%
\pgfpathlineto{\pgfqpoint{1.933513in}{2.412970in}}%
\pgfusepath{stroke}%
\end{pgfscope}%
\begin{pgfscope}%
\pgfpathrectangle{\pgfqpoint{0.800000in}{1.400000in}}{\pgfqpoint{2.407767in}{1.544118in}}%
\pgfusepath{clip}%
\pgfsetbuttcap%
\pgfsetroundjoin%
\pgfsetlinewidth{0.501875pt}%
\definecolor{currentstroke}{rgb}{0.278791,0.062145,0.386592}%
\pgfsetstrokecolor{currentstroke}%
\pgfsetdash{}{0pt}%
\pgfpathmoveto{\pgfqpoint{1.933513in}{2.412970in}}%
\pgfpathlineto{\pgfqpoint{1.893474in}{2.392571in}}%
\pgfusepath{stroke}%
\end{pgfscope}%
\begin{pgfscope}%
\pgfpathrectangle{\pgfqpoint{0.800000in}{1.400000in}}{\pgfqpoint{2.407767in}{1.544118in}}%
\pgfusepath{clip}%
\pgfsetbuttcap%
\pgfsetroundjoin%
\pgfsetlinewidth{0.501875pt}%
\definecolor{currentstroke}{rgb}{0.269944,0.014625,0.341379}%
\pgfsetstrokecolor{currentstroke}%
\pgfsetdash{}{0pt}%
\pgfpathmoveto{\pgfqpoint{2.654046in}{2.519520in}}%
\pgfpathlineto{\pgfqpoint{2.601077in}{2.519237in}}%
\pgfusepath{stroke}%
\end{pgfscope}%
\begin{pgfscope}%
\pgfpathrectangle{\pgfqpoint{0.800000in}{1.400000in}}{\pgfqpoint{2.407767in}{1.544118in}}%
\pgfusepath{clip}%
\pgfsetbuttcap%
\pgfsetroundjoin%
\pgfsetlinewidth{0.501875pt}%
\definecolor{currentstroke}{rgb}{0.271305,0.019942,0.347269}%
\pgfsetstrokecolor{currentstroke}%
\pgfsetdash{}{0pt}%
\pgfpathmoveto{\pgfqpoint{2.601077in}{2.519237in}}%
\pgfpathlineto{\pgfqpoint{2.548106in}{2.519074in}}%
\pgfusepath{stroke}%
\end{pgfscope}%
\begin{pgfscope}%
\pgfpathrectangle{\pgfqpoint{0.800000in}{1.400000in}}{\pgfqpoint{2.407767in}{1.544118in}}%
\pgfusepath{clip}%
\pgfsetbuttcap%
\pgfsetroundjoin%
\pgfsetlinewidth{0.501875pt}%
\definecolor{currentstroke}{rgb}{0.274952,0.037752,0.364543}%
\pgfsetstrokecolor{currentstroke}%
\pgfsetdash{}{0pt}%
\pgfpathmoveto{\pgfqpoint{2.548106in}{2.519074in}}%
\pgfpathlineto{\pgfqpoint{2.495144in}{2.518635in}}%
\pgfusepath{stroke}%
\end{pgfscope}%
\begin{pgfscope}%
\pgfpathrectangle{\pgfqpoint{0.800000in}{1.400000in}}{\pgfqpoint{2.407767in}{1.544118in}}%
\pgfusepath{clip}%
\pgfsetbuttcap%
\pgfsetroundjoin%
\pgfsetlinewidth{0.501875pt}%
\definecolor{currentstroke}{rgb}{0.277941,0.056324,0.381191}%
\pgfsetstrokecolor{currentstroke}%
\pgfsetdash{}{0pt}%
\pgfpathmoveto{\pgfqpoint{2.495144in}{2.518635in}}%
\pgfpathlineto{\pgfqpoint{2.442198in}{2.517686in}}%
\pgfusepath{stroke}%
\end{pgfscope}%
\begin{pgfscope}%
\pgfpathrectangle{\pgfqpoint{0.800000in}{1.400000in}}{\pgfqpoint{2.407767in}{1.544118in}}%
\pgfusepath{clip}%
\pgfsetbuttcap%
\pgfsetroundjoin%
\pgfsetlinewidth{0.501875pt}%
\definecolor{currentstroke}{rgb}{0.279566,0.067836,0.391917}%
\pgfsetstrokecolor{currentstroke}%
\pgfsetdash{}{0pt}%
\pgfpathmoveto{\pgfqpoint{2.442198in}{2.517686in}}%
\pgfpathlineto{\pgfqpoint{2.389273in}{2.516325in}}%
\pgfusepath{stroke}%
\end{pgfscope}%
\begin{pgfscope}%
\pgfpathrectangle{\pgfqpoint{0.800000in}{1.400000in}}{\pgfqpoint{2.407767in}{1.544118in}}%
\pgfusepath{clip}%
\pgfsetbuttcap%
\pgfsetroundjoin%
\pgfsetlinewidth{0.501875pt}%
\definecolor{currentstroke}{rgb}{0.280267,0.073417,0.397163}%
\pgfsetstrokecolor{currentstroke}%
\pgfsetdash{}{0pt}%
\pgfpathmoveto{\pgfqpoint{2.389273in}{2.516325in}}%
\pgfpathlineto{\pgfqpoint{2.336399in}{2.514317in}}%
\pgfusepath{stroke}%
\end{pgfscope}%
\begin{pgfscope}%
\pgfpathrectangle{\pgfqpoint{0.800000in}{1.400000in}}{\pgfqpoint{2.407767in}{1.544118in}}%
\pgfusepath{clip}%
\pgfsetbuttcap%
\pgfsetroundjoin%
\pgfsetlinewidth{0.501875pt}%
\definecolor{currentstroke}{rgb}{0.281446,0.084320,0.407414}%
\pgfsetstrokecolor{currentstroke}%
\pgfsetdash{}{0pt}%
\pgfpathmoveto{\pgfqpoint{2.336399in}{2.514317in}}%
\pgfpathlineto{\pgfqpoint{2.283626in}{2.511559in}}%
\pgfusepath{stroke}%
\end{pgfscope}%
\begin{pgfscope}%
\pgfpathrectangle{\pgfqpoint{0.800000in}{1.400000in}}{\pgfqpoint{2.407767in}{1.544118in}}%
\pgfusepath{clip}%
\pgfsetbuttcap%
\pgfsetroundjoin%
\pgfsetlinewidth{0.501875pt}%
\definecolor{currentstroke}{rgb}{0.276022,0.044167,0.370164}%
\pgfsetstrokecolor{currentstroke}%
\pgfsetdash{}{0pt}%
\pgfpathmoveto{\pgfqpoint{2.283626in}{2.511559in}}%
\pgfpathlineto{\pgfqpoint{2.231071in}{2.507617in}}%
\pgfusepath{stroke}%
\end{pgfscope}%
\begin{pgfscope}%
\pgfpathrectangle{\pgfqpoint{0.800000in}{1.400000in}}{\pgfqpoint{2.407767in}{1.544118in}}%
\pgfusepath{clip}%
\pgfsetbuttcap%
\pgfsetroundjoin%
\pgfsetlinewidth{0.501875pt}%
\definecolor{currentstroke}{rgb}{0.280894,0.078907,0.402329}%
\pgfsetstrokecolor{currentstroke}%
\pgfsetdash{}{0pt}%
\pgfpathmoveto{\pgfqpoint{2.231071in}{2.507617in}}%
\pgfpathlineto{\pgfqpoint{2.178713in}{2.502613in}}%
\pgfusepath{stroke}%
\end{pgfscope}%
\begin{pgfscope}%
\pgfpathrectangle{\pgfqpoint{0.800000in}{1.400000in}}{\pgfqpoint{2.407767in}{1.544118in}}%
\pgfusepath{clip}%
\pgfsetbuttcap%
\pgfsetroundjoin%
\pgfsetlinewidth{0.501875pt}%
\definecolor{currentstroke}{rgb}{0.280267,0.073417,0.397163}%
\pgfsetstrokecolor{currentstroke}%
\pgfsetdash{}{0pt}%
\pgfpathmoveto{\pgfqpoint{2.178713in}{2.502613in}}%
\pgfpathlineto{\pgfqpoint{2.126641in}{2.496526in}}%
\pgfusepath{stroke}%
\end{pgfscope}%
\begin{pgfscope}%
\pgfpathrectangle{\pgfqpoint{0.800000in}{1.400000in}}{\pgfqpoint{2.407767in}{1.544118in}}%
\pgfusepath{clip}%
\pgfsetbuttcap%
\pgfsetroundjoin%
\pgfsetlinewidth{0.501875pt}%
\definecolor{currentstroke}{rgb}{0.274952,0.037752,0.364543}%
\pgfsetstrokecolor{currentstroke}%
\pgfsetdash{}{0pt}%
\pgfpathmoveto{\pgfqpoint{2.126641in}{2.496526in}}%
\pgfpathlineto{\pgfqpoint{2.075155in}{2.488633in}}%
\pgfusepath{stroke}%
\end{pgfscope}%
\begin{pgfscope}%
\pgfpathrectangle{\pgfqpoint{0.800000in}{1.400000in}}{\pgfqpoint{2.407767in}{1.544118in}}%
\pgfusepath{clip}%
\pgfsetbuttcap%
\pgfsetroundjoin%
\pgfsetlinewidth{0.501875pt}%
\definecolor{currentstroke}{rgb}{0.277018,0.050344,0.375715}%
\pgfsetstrokecolor{currentstroke}%
\pgfsetdash{}{0pt}%
\pgfpathmoveto{\pgfqpoint{2.075155in}{2.488633in}}%
\pgfpathlineto{\pgfqpoint{2.025189in}{2.477664in}}%
\pgfusepath{stroke}%
\end{pgfscope}%
\begin{pgfscope}%
\pgfpathrectangle{\pgfqpoint{0.800000in}{1.400000in}}{\pgfqpoint{2.407767in}{1.544118in}}%
\pgfusepath{clip}%
\pgfsetbuttcap%
\pgfsetroundjoin%
\pgfsetlinewidth{0.501875pt}%
\definecolor{currentstroke}{rgb}{0.268510,0.009605,0.335427}%
\pgfsetstrokecolor{currentstroke}%
\pgfsetdash{}{0pt}%
\pgfpathmoveto{\pgfqpoint{2.654046in}{2.554266in}}%
\pgfpathlineto{\pgfqpoint{2.601113in}{2.555222in}}%
\pgfusepath{stroke}%
\end{pgfscope}%
\begin{pgfscope}%
\pgfpathrectangle{\pgfqpoint{0.800000in}{1.400000in}}{\pgfqpoint{2.407767in}{1.544118in}}%
\pgfusepath{clip}%
\pgfsetbuttcap%
\pgfsetroundjoin%
\pgfsetlinewidth{0.501875pt}%
\definecolor{currentstroke}{rgb}{0.272594,0.025563,0.353093}%
\pgfsetstrokecolor{currentstroke}%
\pgfsetdash{}{0pt}%
\pgfpathmoveto{\pgfqpoint{2.601113in}{2.555222in}}%
\pgfpathlineto{\pgfqpoint{2.548138in}{2.555162in}}%
\pgfusepath{stroke}%
\end{pgfscope}%
\begin{pgfscope}%
\pgfpathrectangle{\pgfqpoint{0.800000in}{1.400000in}}{\pgfqpoint{2.407767in}{1.544118in}}%
\pgfusepath{clip}%
\pgfsetbuttcap%
\pgfsetroundjoin%
\pgfsetlinewidth{0.501875pt}%
\definecolor{currentstroke}{rgb}{0.274952,0.037752,0.364543}%
\pgfsetstrokecolor{currentstroke}%
\pgfsetdash{}{0pt}%
\pgfpathmoveto{\pgfqpoint{2.548138in}{2.555162in}}%
\pgfpathlineto{\pgfqpoint{2.495163in}{2.555027in}}%
\pgfusepath{stroke}%
\end{pgfscope}%
\begin{pgfscope}%
\pgfpathrectangle{\pgfqpoint{0.800000in}{1.400000in}}{\pgfqpoint{2.407767in}{1.544118in}}%
\pgfusepath{clip}%
\pgfsetbuttcap%
\pgfsetroundjoin%
\pgfsetlinewidth{0.501875pt}%
\definecolor{currentstroke}{rgb}{0.276022,0.044167,0.370164}%
\pgfsetstrokecolor{currentstroke}%
\pgfsetdash{}{0pt}%
\pgfpathmoveto{\pgfqpoint{2.495163in}{2.555027in}}%
\pgfpathlineto{\pgfqpoint{2.442209in}{2.554306in}}%
\pgfusepath{stroke}%
\end{pgfscope}%
\begin{pgfscope}%
\pgfpathrectangle{\pgfqpoint{0.800000in}{1.400000in}}{\pgfqpoint{2.407767in}{1.544118in}}%
\pgfusepath{clip}%
\pgfsetbuttcap%
\pgfsetroundjoin%
\pgfsetlinewidth{0.501875pt}%
\definecolor{currentstroke}{rgb}{0.277941,0.056324,0.381191}%
\pgfsetstrokecolor{currentstroke}%
\pgfsetdash{}{0pt}%
\pgfpathmoveto{\pgfqpoint{2.442209in}{2.554306in}}%
\pgfpathlineto{\pgfqpoint{2.389275in}{2.552970in}}%
\pgfusepath{stroke}%
\end{pgfscope}%
\begin{pgfscope}%
\pgfpathrectangle{\pgfqpoint{0.800000in}{1.400000in}}{\pgfqpoint{2.407767in}{1.544118in}}%
\pgfusepath{clip}%
\pgfsetbuttcap%
\pgfsetroundjoin%
\pgfsetlinewidth{0.501875pt}%
\definecolor{currentstroke}{rgb}{0.278791,0.062145,0.386592}%
\pgfsetstrokecolor{currentstroke}%
\pgfsetdash{}{0pt}%
\pgfpathmoveto{\pgfqpoint{2.389275in}{2.552970in}}%
\pgfpathlineto{\pgfqpoint{2.336450in}{2.550641in}}%
\pgfusepath{stroke}%
\end{pgfscope}%
\begin{pgfscope}%
\pgfpathrectangle{\pgfqpoint{0.800000in}{1.400000in}}{\pgfqpoint{2.407767in}{1.544118in}}%
\pgfusepath{clip}%
\pgfsetbuttcap%
\pgfsetroundjoin%
\pgfsetlinewidth{0.501875pt}%
\definecolor{currentstroke}{rgb}{0.280267,0.073417,0.397163}%
\pgfsetstrokecolor{currentstroke}%
\pgfsetdash{}{0pt}%
\pgfpathmoveto{\pgfqpoint{2.336450in}{2.550641in}}%
\pgfpathlineto{\pgfqpoint{2.283663in}{2.547986in}}%
\pgfusepath{stroke}%
\end{pgfscope}%
\begin{pgfscope}%
\pgfpathrectangle{\pgfqpoint{0.800000in}{1.400000in}}{\pgfqpoint{2.407767in}{1.544118in}}%
\pgfusepath{clip}%
\pgfsetbuttcap%
\pgfsetroundjoin%
\pgfsetlinewidth{0.501875pt}%
\definecolor{currentstroke}{rgb}{0.280267,0.073417,0.397163}%
\pgfsetstrokecolor{currentstroke}%
\pgfsetdash{}{0pt}%
\pgfpathmoveto{\pgfqpoint{2.283663in}{2.547986in}}%
\pgfpathlineto{\pgfqpoint{2.230917in}{2.545060in}}%
\pgfusepath{stroke}%
\end{pgfscope}%
\begin{pgfscope}%
\pgfpathrectangle{\pgfqpoint{0.800000in}{1.400000in}}{\pgfqpoint{2.407767in}{1.544118in}}%
\pgfusepath{clip}%
\pgfsetbuttcap%
\pgfsetroundjoin%
\pgfsetlinewidth{0.501875pt}%
\definecolor{currentstroke}{rgb}{0.279566,0.067836,0.391917}%
\pgfsetstrokecolor{currentstroke}%
\pgfsetdash{}{0pt}%
\pgfpathmoveto{\pgfqpoint{2.230917in}{2.545060in}}%
\pgfpathlineto{\pgfqpoint{2.178399in}{2.540672in}}%
\pgfusepath{stroke}%
\end{pgfscope}%
\begin{pgfscope}%
\pgfpathrectangle{\pgfqpoint{0.800000in}{1.400000in}}{\pgfqpoint{2.407767in}{1.544118in}}%
\pgfusepath{clip}%
\pgfsetbuttcap%
\pgfsetroundjoin%
\pgfsetlinewidth{0.501875pt}%
\definecolor{currentstroke}{rgb}{0.279566,0.067836,0.391917}%
\pgfsetstrokecolor{currentstroke}%
\pgfsetdash{}{0pt}%
\pgfpathmoveto{\pgfqpoint{2.178399in}{2.540672in}}%
\pgfpathlineto{\pgfqpoint{2.126603in}{2.533909in}}%
\pgfusepath{stroke}%
\end{pgfscope}%
\begin{pgfscope}%
\pgfpathrectangle{\pgfqpoint{0.800000in}{1.400000in}}{\pgfqpoint{2.407767in}{1.544118in}}%
\pgfusepath{clip}%
\pgfsetbuttcap%
\pgfsetroundjoin%
\pgfsetlinewidth{0.501875pt}%
\definecolor{currentstroke}{rgb}{0.269944,0.014625,0.341379}%
\pgfsetstrokecolor{currentstroke}%
\pgfsetdash{}{0pt}%
\pgfpathmoveto{\pgfqpoint{2.654046in}{2.589012in}}%
\pgfpathlineto{\pgfqpoint{2.601087in}{2.588482in}}%
\pgfusepath{stroke}%
\end{pgfscope}%
\begin{pgfscope}%
\pgfpathrectangle{\pgfqpoint{0.800000in}{1.400000in}}{\pgfqpoint{2.407767in}{1.544118in}}%
\pgfusepath{clip}%
\pgfsetbuttcap%
\pgfsetroundjoin%
\pgfsetlinewidth{0.501875pt}%
\definecolor{currentstroke}{rgb}{0.272594,0.025563,0.353093}%
\pgfsetstrokecolor{currentstroke}%
\pgfsetdash{}{0pt}%
\pgfpathmoveto{\pgfqpoint{2.601087in}{2.588482in}}%
\pgfpathlineto{\pgfqpoint{2.548124in}{2.587910in}}%
\pgfusepath{stroke}%
\end{pgfscope}%
\begin{pgfscope}%
\pgfpathrectangle{\pgfqpoint{0.800000in}{1.400000in}}{\pgfqpoint{2.407767in}{1.544118in}}%
\pgfusepath{clip}%
\pgfsetbuttcap%
\pgfsetroundjoin%
\pgfsetlinewidth{0.501875pt}%
\definecolor{currentstroke}{rgb}{0.273809,0.031497,0.358853}%
\pgfsetstrokecolor{currentstroke}%
\pgfsetdash{}{0pt}%
\pgfpathmoveto{\pgfqpoint{2.548124in}{2.587910in}}%
\pgfpathlineto{\pgfqpoint{2.495175in}{2.586928in}}%
\pgfusepath{stroke}%
\end{pgfscope}%
\begin{pgfscope}%
\pgfpathrectangle{\pgfqpoint{0.800000in}{1.400000in}}{\pgfqpoint{2.407767in}{1.544118in}}%
\pgfusepath{clip}%
\pgfsetbuttcap%
\pgfsetroundjoin%
\pgfsetlinewidth{0.501875pt}%
\definecolor{currentstroke}{rgb}{0.277018,0.050344,0.375715}%
\pgfsetstrokecolor{currentstroke}%
\pgfsetdash{}{0pt}%
\pgfpathmoveto{\pgfqpoint{2.495175in}{2.586928in}}%
\pgfpathlineto{\pgfqpoint{2.442258in}{2.585445in}}%
\pgfusepath{stroke}%
\end{pgfscope}%
\begin{pgfscope}%
\pgfpathrectangle{\pgfqpoint{0.800000in}{1.400000in}}{\pgfqpoint{2.407767in}{1.544118in}}%
\pgfusepath{clip}%
\pgfsetbuttcap%
\pgfsetroundjoin%
\pgfsetlinewidth{0.501875pt}%
\definecolor{currentstroke}{rgb}{0.277941,0.056324,0.381191}%
\pgfsetstrokecolor{currentstroke}%
\pgfsetdash{}{0pt}%
\pgfpathmoveto{\pgfqpoint{2.442258in}{2.585445in}}%
\pgfpathlineto{\pgfqpoint{2.389346in}{2.583839in}}%
\pgfusepath{stroke}%
\end{pgfscope}%
\begin{pgfscope}%
\pgfpathrectangle{\pgfqpoint{0.800000in}{1.400000in}}{\pgfqpoint{2.407767in}{1.544118in}}%
\pgfusepath{clip}%
\pgfsetbuttcap%
\pgfsetroundjoin%
\pgfsetlinewidth{0.501875pt}%
\definecolor{currentstroke}{rgb}{0.277941,0.056324,0.381191}%
\pgfsetstrokecolor{currentstroke}%
\pgfsetdash{}{0pt}%
\pgfpathmoveto{\pgfqpoint{2.389346in}{2.583839in}}%
\pgfpathlineto{\pgfqpoint{2.336510in}{2.581556in}}%
\pgfusepath{stroke}%
\end{pgfscope}%
\begin{pgfscope}%
\pgfpathrectangle{\pgfqpoint{0.800000in}{1.400000in}}{\pgfqpoint{2.407767in}{1.544118in}}%
\pgfusepath{clip}%
\pgfsetbuttcap%
\pgfsetroundjoin%
\pgfsetlinewidth{0.501875pt}%
\definecolor{currentstroke}{rgb}{0.277941,0.056324,0.381191}%
\pgfsetstrokecolor{currentstroke}%
\pgfsetdash{}{0pt}%
\pgfpathmoveto{\pgfqpoint{2.336510in}{2.581556in}}%
\pgfpathlineto{\pgfqpoint{2.283832in}{2.578118in}}%
\pgfusepath{stroke}%
\end{pgfscope}%
\begin{pgfscope}%
\pgfpathrectangle{\pgfqpoint{0.800000in}{1.400000in}}{\pgfqpoint{2.407767in}{1.544118in}}%
\pgfusepath{clip}%
\pgfsetbuttcap%
\pgfsetroundjoin%
\pgfsetlinewidth{0.501875pt}%
\definecolor{currentstroke}{rgb}{0.278791,0.062145,0.386592}%
\pgfsetstrokecolor{currentstroke}%
\pgfsetdash{}{0pt}%
\pgfpathmoveto{\pgfqpoint{2.283832in}{2.578118in}}%
\pgfpathlineto{\pgfqpoint{2.231268in}{2.574026in}}%
\pgfusepath{stroke}%
\end{pgfscope}%
\begin{pgfscope}%
\pgfpathrectangle{\pgfqpoint{0.800000in}{1.400000in}}{\pgfqpoint{2.407767in}{1.544118in}}%
\pgfusepath{clip}%
\pgfsetbuttcap%
\pgfsetroundjoin%
\pgfsetlinewidth{0.501875pt}%
\definecolor{currentstroke}{rgb}{0.282656,0.100196,0.422160}%
\pgfsetstrokecolor{currentstroke}%
\pgfsetdash{}{0pt}%
\pgfpathmoveto{\pgfqpoint{1.630897in}{1.776062in}}%
\pgfpathlineto{\pgfqpoint{1.678802in}{1.789851in}}%
\pgfusepath{stroke}%
\end{pgfscope}%
\begin{pgfscope}%
\pgfpathrectangle{\pgfqpoint{0.800000in}{1.400000in}}{\pgfqpoint{2.407767in}{1.544118in}}%
\pgfusepath{clip}%
\pgfsetbuttcap%
\pgfsetroundjoin%
\pgfsetlinewidth{0.501875pt}%
\definecolor{currentstroke}{rgb}{0.283091,0.110553,0.431554}%
\pgfsetstrokecolor{currentstroke}%
\pgfsetdash{}{0pt}%
\pgfpathmoveto{\pgfqpoint{1.678802in}{1.789851in}}%
\pgfpathlineto{\pgfqpoint{1.678802in}{1.789851in}}%
\pgfusepath{stroke}%
\end{pgfscope}%
\begin{pgfscope}%
\pgfpathrectangle{\pgfqpoint{0.800000in}{1.400000in}}{\pgfqpoint{2.407767in}{1.544118in}}%
\pgfusepath{clip}%
\pgfsetbuttcap%
\pgfsetroundjoin%
\pgfsetlinewidth{0.501875pt}%
\definecolor{currentstroke}{rgb}{0.283091,0.110553,0.431554}%
\pgfsetstrokecolor{currentstroke}%
\pgfsetdash{}{0pt}%
\pgfpathmoveto{\pgfqpoint{1.678802in}{1.789851in}}%
\pgfpathlineto{\pgfqpoint{1.715616in}{1.802484in}}%
\pgfusepath{stroke}%
\end{pgfscope}%
\begin{pgfscope}%
\pgfpathrectangle{\pgfqpoint{0.800000in}{1.400000in}}{\pgfqpoint{2.407767in}{1.544118in}}%
\pgfusepath{clip}%
\pgfsetbuttcap%
\pgfsetroundjoin%
\pgfsetlinewidth{0.501875pt}%
\definecolor{currentstroke}{rgb}{0.281924,0.089666,0.412415}%
\pgfsetstrokecolor{currentstroke}%
\pgfsetdash{}{0pt}%
\pgfpathmoveto{\pgfqpoint{1.715616in}{1.802484in}}%
\pgfpathlineto{\pgfqpoint{1.751918in}{1.816092in}}%
\pgfusepath{stroke}%
\end{pgfscope}%
\begin{pgfscope}%
\pgfpathrectangle{\pgfqpoint{0.800000in}{1.400000in}}{\pgfqpoint{2.407767in}{1.544118in}}%
\pgfusepath{clip}%
\pgfsetbuttcap%
\pgfsetroundjoin%
\pgfsetlinewidth{0.501875pt}%
\definecolor{currentstroke}{rgb}{0.280267,0.073417,0.397163}%
\pgfsetstrokecolor{currentstroke}%
\pgfsetdash{}{0pt}%
\pgfpathmoveto{\pgfqpoint{1.751918in}{1.816092in}}%
\pgfpathlineto{\pgfqpoint{1.792801in}{1.835681in}}%
\pgfusepath{stroke}%
\end{pgfscope}%
\begin{pgfscope}%
\pgfpathrectangle{\pgfqpoint{0.800000in}{1.400000in}}{\pgfqpoint{2.407767in}{1.544118in}}%
\pgfusepath{clip}%
\pgfsetbuttcap%
\pgfsetroundjoin%
\pgfsetlinewidth{0.501875pt}%
\definecolor{currentstroke}{rgb}{0.280894,0.078907,0.402329}%
\pgfsetstrokecolor{currentstroke}%
\pgfsetdash{}{0pt}%
\pgfpathmoveto{\pgfqpoint{1.792801in}{1.835681in}}%
\pgfpathlineto{\pgfqpoint{1.792801in}{1.835681in}}%
\pgfusepath{stroke}%
\end{pgfscope}%
\begin{pgfscope}%
\pgfpathrectangle{\pgfqpoint{0.800000in}{1.400000in}}{\pgfqpoint{2.407767in}{1.544118in}}%
\pgfusepath{clip}%
\pgfsetbuttcap%
\pgfsetroundjoin%
\pgfsetlinewidth{0.501875pt}%
\definecolor{currentstroke}{rgb}{0.280894,0.078907,0.402329}%
\pgfsetstrokecolor{currentstroke}%
\pgfsetdash{}{0pt}%
\pgfpathmoveto{\pgfqpoint{1.792801in}{1.835681in}}%
\pgfpathlineto{\pgfqpoint{1.809739in}{1.856897in}}%
\pgfusepath{stroke}%
\end{pgfscope}%
\begin{pgfscope}%
\pgfpathrectangle{\pgfqpoint{0.800000in}{1.400000in}}{\pgfqpoint{2.407767in}{1.544118in}}%
\pgfusepath{clip}%
\pgfsetbuttcap%
\pgfsetroundjoin%
\pgfsetlinewidth{0.501875pt}%
\definecolor{currentstroke}{rgb}{0.280894,0.078907,0.402329}%
\pgfsetstrokecolor{currentstroke}%
\pgfsetdash{}{0pt}%
\pgfpathmoveto{\pgfqpoint{1.809739in}{1.856897in}}%
\pgfpathlineto{\pgfqpoint{1.809739in}{1.856897in}}%
\pgfusepath{stroke}%
\end{pgfscope}%
\begin{pgfscope}%
\pgfpathrectangle{\pgfqpoint{0.800000in}{1.400000in}}{\pgfqpoint{2.407767in}{1.544118in}}%
\pgfusepath{clip}%
\pgfsetbuttcap%
\pgfsetroundjoin%
\pgfsetlinewidth{0.501875pt}%
\definecolor{currentstroke}{rgb}{0.280894,0.078907,0.402329}%
\pgfsetstrokecolor{currentstroke}%
\pgfsetdash{}{0pt}%
\pgfpathmoveto{\pgfqpoint{1.809739in}{1.856897in}}%
\pgfpathlineto{\pgfqpoint{1.814206in}{1.873619in}}%
\pgfusepath{stroke}%
\end{pgfscope}%
\begin{pgfscope}%
\pgfpathrectangle{\pgfqpoint{0.800000in}{1.400000in}}{\pgfqpoint{2.407767in}{1.544118in}}%
\pgfusepath{clip}%
\pgfsetbuttcap%
\pgfsetroundjoin%
\pgfsetlinewidth{0.501875pt}%
\definecolor{currentstroke}{rgb}{0.281924,0.089666,0.412415}%
\pgfsetstrokecolor{currentstroke}%
\pgfsetdash{}{0pt}%
\pgfpathmoveto{\pgfqpoint{1.814206in}{1.873619in}}%
\pgfpathlineto{\pgfqpoint{1.814206in}{1.873619in}}%
\pgfusepath{stroke}%
\end{pgfscope}%
\begin{pgfscope}%
\pgfpathrectangle{\pgfqpoint{0.800000in}{1.400000in}}{\pgfqpoint{2.407767in}{1.544118in}}%
\pgfusepath{clip}%
\pgfsetbuttcap%
\pgfsetroundjoin%
\pgfsetlinewidth{0.501875pt}%
\definecolor{currentstroke}{rgb}{0.280894,0.078907,0.402329}%
\pgfsetstrokecolor{currentstroke}%
\pgfsetdash{}{0pt}%
\pgfpathmoveto{\pgfqpoint{2.153626in}{1.792352in}}%
\pgfpathlineto{\pgfqpoint{2.102015in}{1.799854in}}%
\pgfusepath{stroke}%
\end{pgfscope}%
\begin{pgfscope}%
\pgfpathrectangle{\pgfqpoint{0.800000in}{1.400000in}}{\pgfqpoint{2.407767in}{1.544118in}}%
\pgfusepath{clip}%
\pgfsetbuttcap%
\pgfsetroundjoin%
\pgfsetlinewidth{0.501875pt}%
\definecolor{currentstroke}{rgb}{0.280894,0.078907,0.402329}%
\pgfsetstrokecolor{currentstroke}%
\pgfsetdash{}{0pt}%
\pgfpathmoveto{\pgfqpoint{2.102015in}{1.799854in}}%
\pgfpathlineto{\pgfqpoint{2.051742in}{1.810321in}}%
\pgfusepath{stroke}%
\end{pgfscope}%
\begin{pgfscope}%
\pgfpathrectangle{\pgfqpoint{0.800000in}{1.400000in}}{\pgfqpoint{2.407767in}{1.544118in}}%
\pgfusepath{clip}%
\pgfsetbuttcap%
\pgfsetroundjoin%
\pgfsetlinewidth{0.501875pt}%
\definecolor{currentstroke}{rgb}{0.280267,0.073417,0.397163}%
\pgfsetstrokecolor{currentstroke}%
\pgfsetdash{}{0pt}%
\pgfpathmoveto{\pgfqpoint{2.051742in}{1.810321in}}%
\pgfpathlineto{\pgfqpoint{2.003883in}{1.824598in}}%
\pgfusepath{stroke}%
\end{pgfscope}%
\begin{pgfscope}%
\pgfpathrectangle{\pgfqpoint{0.800000in}{1.400000in}}{\pgfqpoint{2.407767in}{1.544118in}}%
\pgfusepath{clip}%
\pgfsetbuttcap%
\pgfsetroundjoin%
\pgfsetlinewidth{0.501875pt}%
\definecolor{currentstroke}{rgb}{0.279566,0.067836,0.391917}%
\pgfsetstrokecolor{currentstroke}%
\pgfsetdash{}{0pt}%
\pgfpathmoveto{\pgfqpoint{2.003883in}{1.824598in}}%
\pgfpathlineto{\pgfqpoint{1.958897in}{1.842253in}}%
\pgfusepath{stroke}%
\end{pgfscope}%
\begin{pgfscope}%
\pgfpathrectangle{\pgfqpoint{0.800000in}{1.400000in}}{\pgfqpoint{2.407767in}{1.544118in}}%
\pgfusepath{clip}%
\pgfsetbuttcap%
\pgfsetroundjoin%
\pgfsetlinewidth{0.501875pt}%
\definecolor{currentstroke}{rgb}{0.280894,0.078907,0.402329}%
\pgfsetstrokecolor{currentstroke}%
\pgfsetdash{}{0pt}%
\pgfpathmoveto{\pgfqpoint{1.958897in}{1.842253in}}%
\pgfpathlineto{\pgfqpoint{1.916968in}{1.862068in}}%
\pgfusepath{stroke}%
\end{pgfscope}%
\begin{pgfscope}%
\pgfpathrectangle{\pgfqpoint{0.800000in}{1.400000in}}{\pgfqpoint{2.407767in}{1.544118in}}%
\pgfusepath{clip}%
\pgfsetbuttcap%
\pgfsetroundjoin%
\pgfsetlinewidth{0.501875pt}%
\definecolor{currentstroke}{rgb}{0.282327,0.094955,0.417331}%
\pgfsetstrokecolor{currentstroke}%
\pgfsetdash{}{0pt}%
\pgfpathmoveto{\pgfqpoint{1.916968in}{1.862068in}}%
\pgfpathlineto{\pgfqpoint{1.879176in}{1.885106in}}%
\pgfusepath{stroke}%
\end{pgfscope}%
\begin{pgfscope}%
\pgfpathrectangle{\pgfqpoint{0.800000in}{1.400000in}}{\pgfqpoint{2.407767in}{1.544118in}}%
\pgfusepath{clip}%
\pgfsetbuttcap%
\pgfsetroundjoin%
\pgfsetlinewidth{0.501875pt}%
\definecolor{currentstroke}{rgb}{0.282623,0.140926,0.457517}%
\pgfsetstrokecolor{currentstroke}%
\pgfsetdash{}{0pt}%
\pgfpathmoveto{\pgfqpoint{1.628531in}{1.860576in}}%
\pgfpathlineto{\pgfqpoint{1.655500in}{1.874614in}}%
\pgfusepath{stroke}%
\end{pgfscope}%
\begin{pgfscope}%
\pgfpathrectangle{\pgfqpoint{0.800000in}{1.400000in}}{\pgfqpoint{2.407767in}{1.544118in}}%
\pgfusepath{clip}%
\pgfsetbuttcap%
\pgfsetroundjoin%
\pgfsetlinewidth{0.501875pt}%
\definecolor{currentstroke}{rgb}{0.280255,0.165693,0.476498}%
\pgfsetstrokecolor{currentstroke}%
\pgfsetdash{}{0pt}%
\pgfpathmoveto{\pgfqpoint{1.655500in}{1.874614in}}%
\pgfpathlineto{\pgfqpoint{1.678802in}{1.894090in}}%
\pgfusepath{stroke}%
\end{pgfscope}%
\begin{pgfscope}%
\pgfpathrectangle{\pgfqpoint{0.800000in}{1.400000in}}{\pgfqpoint{2.407767in}{1.544118in}}%
\pgfusepath{clip}%
\pgfsetbuttcap%
\pgfsetroundjoin%
\pgfsetlinewidth{0.501875pt}%
\definecolor{currentstroke}{rgb}{0.274128,0.199721,0.498911}%
\pgfsetstrokecolor{currentstroke}%
\pgfsetdash{}{0pt}%
\pgfpathmoveto{\pgfqpoint{1.678802in}{1.894090in}}%
\pgfpathlineto{\pgfqpoint{1.678802in}{1.894090in}}%
\pgfusepath{stroke}%
\end{pgfscope}%
\begin{pgfscope}%
\pgfpathrectangle{\pgfqpoint{0.800000in}{1.400000in}}{\pgfqpoint{2.407767in}{1.544118in}}%
\pgfusepath{clip}%
\pgfsetbuttcap%
\pgfsetroundjoin%
\pgfsetlinewidth{0.501875pt}%
\definecolor{currentstroke}{rgb}{0.274128,0.199721,0.498911}%
\pgfsetstrokecolor{currentstroke}%
\pgfsetdash{}{0pt}%
\pgfpathmoveto{\pgfqpoint{1.678802in}{1.894090in}}%
\pgfpathlineto{\pgfqpoint{1.678802in}{1.894090in}}%
\pgfusepath{stroke}%
\end{pgfscope}%
\begin{pgfscope}%
\pgfpathrectangle{\pgfqpoint{0.800000in}{1.400000in}}{\pgfqpoint{2.407767in}{1.544118in}}%
\pgfusepath{clip}%
\pgfsetbuttcap%
\pgfsetroundjoin%
\pgfsetlinewidth{0.501875pt}%
\definecolor{currentstroke}{rgb}{0.274128,0.199721,0.498911}%
\pgfsetstrokecolor{currentstroke}%
\pgfsetdash{}{0pt}%
\pgfpathmoveto{\pgfqpoint{1.678802in}{1.894090in}}%
\pgfpathlineto{\pgfqpoint{1.678802in}{1.894090in}}%
\pgfusepath{stroke}%
\end{pgfscope}%
\begin{pgfscope}%
\pgfpathrectangle{\pgfqpoint{0.800000in}{1.400000in}}{\pgfqpoint{2.407767in}{1.544118in}}%
\pgfusepath{clip}%
\pgfsetbuttcap%
\pgfsetroundjoin%
\pgfsetlinewidth{0.501875pt}%
\definecolor{currentstroke}{rgb}{0.274128,0.199721,0.498911}%
\pgfsetstrokecolor{currentstroke}%
\pgfsetdash{}{0pt}%
\pgfpathmoveto{\pgfqpoint{1.678802in}{1.894090in}}%
\pgfpathlineto{\pgfqpoint{1.687395in}{1.911752in}}%
\pgfusepath{stroke}%
\end{pgfscope}%
\begin{pgfscope}%
\pgfpathrectangle{\pgfqpoint{0.800000in}{1.400000in}}{\pgfqpoint{2.407767in}{1.544118in}}%
\pgfusepath{clip}%
\pgfsetbuttcap%
\pgfsetroundjoin%
\pgfsetlinewidth{0.501875pt}%
\definecolor{currentstroke}{rgb}{0.278012,0.180367,0.486697}%
\pgfsetstrokecolor{currentstroke}%
\pgfsetdash{}{0pt}%
\pgfpathmoveto{\pgfqpoint{1.687395in}{1.911752in}}%
\pgfpathlineto{\pgfqpoint{1.693195in}{1.929565in}}%
\pgfusepath{stroke}%
\end{pgfscope}%
\begin{pgfscope}%
\pgfpathrectangle{\pgfqpoint{0.800000in}{1.400000in}}{\pgfqpoint{2.407767in}{1.544118in}}%
\pgfusepath{clip}%
\pgfsetbuttcap%
\pgfsetroundjoin%
\pgfsetlinewidth{0.501875pt}%
\definecolor{currentstroke}{rgb}{0.279574,0.170599,0.479997}%
\pgfsetstrokecolor{currentstroke}%
\pgfsetdash{}{0pt}%
\pgfpathmoveto{\pgfqpoint{1.693195in}{1.929565in}}%
\pgfpathlineto{\pgfqpoint{1.693195in}{1.929565in}}%
\pgfusepath{stroke}%
\end{pgfscope}%
\begin{pgfscope}%
\pgfpathrectangle{\pgfqpoint{0.800000in}{1.400000in}}{\pgfqpoint{2.407767in}{1.544118in}}%
\pgfusepath{clip}%
\pgfsetbuttcap%
\pgfsetroundjoin%
\pgfsetlinewidth{0.501875pt}%
\definecolor{currentstroke}{rgb}{0.279574,0.170599,0.479997}%
\pgfsetstrokecolor{currentstroke}%
\pgfsetdash{}{0pt}%
\pgfpathmoveto{\pgfqpoint{1.693195in}{1.929565in}}%
\pgfpathlineto{\pgfqpoint{1.694297in}{1.946466in}}%
\pgfusepath{stroke}%
\end{pgfscope}%
\begin{pgfscope}%
\pgfpathrectangle{\pgfqpoint{0.800000in}{1.400000in}}{\pgfqpoint{2.407767in}{1.544118in}}%
\pgfusepath{clip}%
\pgfsetbuttcap%
\pgfsetroundjoin%
\pgfsetlinewidth{0.501875pt}%
\definecolor{currentstroke}{rgb}{0.278826,0.175490,0.483397}%
\pgfsetstrokecolor{currentstroke}%
\pgfsetdash{}{0pt}%
\pgfpathmoveto{\pgfqpoint{1.694297in}{1.946466in}}%
\pgfpathlineto{\pgfqpoint{1.687452in}{1.962881in}}%
\pgfusepath{stroke}%
\end{pgfscope}%
\begin{pgfscope}%
\pgfpathrectangle{\pgfqpoint{0.800000in}{1.400000in}}{\pgfqpoint{2.407767in}{1.544118in}}%
\pgfusepath{clip}%
\pgfsetbuttcap%
\pgfsetroundjoin%
\pgfsetlinewidth{0.501875pt}%
\definecolor{currentstroke}{rgb}{0.270595,0.214069,0.507052}%
\pgfsetstrokecolor{currentstroke}%
\pgfsetdash{}{0pt}%
\pgfpathmoveto{\pgfqpoint{1.687452in}{1.962881in}}%
\pgfpathlineto{\pgfqpoint{1.675084in}{1.978873in}}%
\pgfusepath{stroke}%
\end{pgfscope}%
\begin{pgfscope}%
\pgfpathrectangle{\pgfqpoint{0.800000in}{1.400000in}}{\pgfqpoint{2.407767in}{1.544118in}}%
\pgfusepath{clip}%
\pgfsetbuttcap%
\pgfsetroundjoin%
\pgfsetlinewidth{0.501875pt}%
\definecolor{currentstroke}{rgb}{0.270595,0.214069,0.507052}%
\pgfsetstrokecolor{currentstroke}%
\pgfsetdash{}{0pt}%
\pgfpathmoveto{\pgfqpoint{1.675084in}{1.978873in}}%
\pgfpathlineto{\pgfqpoint{1.645464in}{2.006476in}}%
\pgfusepath{stroke}%
\end{pgfscope}%
\begin{pgfscope}%
\pgfpathrectangle{\pgfqpoint{0.800000in}{1.400000in}}{\pgfqpoint{2.407767in}{1.544118in}}%
\pgfusepath{clip}%
\pgfsetbuttcap%
\pgfsetroundjoin%
\pgfsetlinewidth{0.000000pt}%
\definecolor{currentstroke}{rgb}{0.270595,0.214069,0.507052}%
\pgfsetstrokecolor{currentstroke}%
\pgfsetdash{}{0pt}%
\pgfpathmoveto{\pgfqpoint{1.574656in}{2.369342in}}%
\pgfpathlineto{\pgfqpoint{1.597978in}{2.350673in}}%
\pgfusepath{}%
\end{pgfscope}%
\begin{pgfscope}%
\pgfpathrectangle{\pgfqpoint{0.800000in}{1.400000in}}{\pgfqpoint{2.407767in}{1.544118in}}%
\pgfusepath{clip}%
\pgfsetbuttcap%
\pgfsetroundjoin%
\pgfsetlinewidth{0.501875pt}%
\definecolor{currentstroke}{rgb}{0.265145,0.232956,0.516599}%
\pgfsetstrokecolor{currentstroke}%
\pgfsetdash{}{0pt}%
\pgfpathmoveto{\pgfqpoint{1.597978in}{2.350673in}}%
\pgfpathlineto{\pgfqpoint{1.616217in}{2.330859in}}%
\pgfusepath{stroke}%
\end{pgfscope}%
\begin{pgfscope}%
\pgfpathrectangle{\pgfqpoint{0.800000in}{1.400000in}}{\pgfqpoint{2.407767in}{1.544118in}}%
\pgfusepath{clip}%
\pgfsetbuttcap%
\pgfsetroundjoin%
\pgfsetlinewidth{0.501875pt}%
\definecolor{currentstroke}{rgb}{0.260571,0.246922,0.522828}%
\pgfsetstrokecolor{currentstroke}%
\pgfsetdash{}{0pt}%
\pgfpathmoveto{\pgfqpoint{1.616217in}{2.330859in}}%
\pgfpathlineto{\pgfqpoint{1.624622in}{2.311043in}}%
\pgfusepath{stroke}%
\end{pgfscope}%
\begin{pgfscope}%
\pgfpathrectangle{\pgfqpoint{0.800000in}{1.400000in}}{\pgfqpoint{2.407767in}{1.544118in}}%
\pgfusepath{clip}%
\pgfsetbuttcap%
\pgfsetroundjoin%
\pgfsetlinewidth{0.501875pt}%
\definecolor{currentstroke}{rgb}{0.258965,0.251537,0.524736}%
\pgfsetstrokecolor{currentstroke}%
\pgfsetdash{}{0pt}%
\pgfpathmoveto{\pgfqpoint{1.624622in}{2.311043in}}%
\pgfpathlineto{\pgfqpoint{1.624622in}{2.311043in}}%
\pgfusepath{stroke}%
\end{pgfscope}%
\begin{pgfscope}%
\pgfpathrectangle{\pgfqpoint{0.800000in}{1.400000in}}{\pgfqpoint{2.407767in}{1.544118in}}%
\pgfusepath{clip}%
\pgfsetbuttcap%
\pgfsetroundjoin%
\pgfsetlinewidth{0.501875pt}%
\definecolor{currentstroke}{rgb}{0.258965,0.251537,0.524736}%
\pgfsetstrokecolor{currentstroke}%
\pgfsetdash{}{0pt}%
\pgfpathmoveto{\pgfqpoint{1.624622in}{2.311043in}}%
\pgfpathlineto{\pgfqpoint{1.624622in}{2.311043in}}%
\pgfusepath{stroke}%
\end{pgfscope}%
\begin{pgfscope}%
\pgfpathrectangle{\pgfqpoint{0.800000in}{1.400000in}}{\pgfqpoint{2.407767in}{1.544118in}}%
\pgfusepath{clip}%
\pgfsetbuttcap%
\pgfsetroundjoin%
\pgfsetlinewidth{0.501875pt}%
\definecolor{currentstroke}{rgb}{0.258965,0.251537,0.524736}%
\pgfsetstrokecolor{currentstroke}%
\pgfsetdash{}{0pt}%
\pgfpathmoveto{\pgfqpoint{1.624622in}{2.311043in}}%
\pgfpathlineto{\pgfqpoint{1.623584in}{2.291327in}}%
\pgfusepath{stroke}%
\end{pgfscope}%
\begin{pgfscope}%
\pgfpathrectangle{\pgfqpoint{0.800000in}{1.400000in}}{\pgfqpoint{2.407767in}{1.544118in}}%
\pgfusepath{clip}%
\pgfsetbuttcap%
\pgfsetroundjoin%
\pgfsetlinewidth{0.501875pt}%
\definecolor{currentstroke}{rgb}{0.241237,0.296485,0.539709}%
\pgfsetstrokecolor{currentstroke}%
\pgfsetdash{}{0pt}%
\pgfpathmoveto{\pgfqpoint{1.623584in}{2.291327in}}%
\pgfpathlineto{\pgfqpoint{1.614152in}{2.272898in}}%
\pgfusepath{stroke}%
\end{pgfscope}%
\begin{pgfscope}%
\pgfpathrectangle{\pgfqpoint{0.800000in}{1.400000in}}{\pgfqpoint{2.407767in}{1.544118in}}%
\pgfusepath{clip}%
\pgfsetbuttcap%
\pgfsetroundjoin%
\pgfsetlinewidth{0.501875pt}%
\definecolor{currentstroke}{rgb}{0.248629,0.278775,0.534556}%
\pgfsetstrokecolor{currentstroke}%
\pgfsetdash{}{0pt}%
\pgfpathmoveto{\pgfqpoint{1.614152in}{2.272898in}}%
\pgfpathlineto{\pgfqpoint{1.597244in}{2.251294in}}%
\pgfusepath{stroke}%
\end{pgfscope}%
\begin{pgfscope}%
\pgfpathrectangle{\pgfqpoint{0.800000in}{1.400000in}}{\pgfqpoint{2.407767in}{1.544118in}}%
\pgfusepath{clip}%
\pgfsetbuttcap%
\pgfsetroundjoin%
\pgfsetlinewidth{0.501875pt}%
\definecolor{currentstroke}{rgb}{0.253935,0.265254,0.529983}%
\pgfsetstrokecolor{currentstroke}%
\pgfsetdash{}{0pt}%
\pgfpathmoveto{\pgfqpoint{1.597244in}{2.251294in}}%
\pgfpathlineto{\pgfqpoint{1.572997in}{2.225156in}}%
\pgfusepath{stroke}%
\end{pgfscope}%
\begin{pgfscope}%
\pgfpathrectangle{\pgfqpoint{0.800000in}{1.400000in}}{\pgfqpoint{2.407767in}{1.544118in}}%
\pgfusepath{clip}%
\pgfsetroundcap%
\pgfsetroundjoin%
\pgfsetlinewidth{0.501875pt}%
\definecolor{currentstroke}{rgb}{0.268510,0.009605,0.335427}%
\pgfsetstrokecolor{currentstroke}%
\pgfsetdash{}{0pt}%
\pgfpathmoveto{\pgfqpoint{2.609825in}{2.828895in}}%
\pgfpathquadraticcurveto{\pgfqpoint{2.603325in}{2.828741in}}{\pgfqpoint{2.604588in}{2.828771in}}%
\pgfusepath{stroke}%
\end{pgfscope}%
\begin{pgfscope}%
\pgfpathrectangle{\pgfqpoint{0.800000in}{1.400000in}}{\pgfqpoint{2.407767in}{1.544118in}}%
\pgfusepath{clip}%
\pgfsetroundcap%
\pgfsetroundjoin%
\definecolor{currentfill}{rgb}{0.268510,0.009605,0.335427}%
\pgfsetfillcolor{currentfill}%
\pgfsetlinewidth{0.501875pt}%
\definecolor{currentstroke}{rgb}{0.268510,0.009605,0.335427}%
\pgfsetstrokecolor{currentstroke}%
\pgfsetdash{}{0pt}%
\pgfpathmoveto{\pgfqpoint{2.632687in}{2.815545in}}%
\pgfpathlineto{\pgfqpoint{2.604588in}{2.828771in}}%
\pgfpathlineto{\pgfqpoint{2.632028in}{2.843315in}}%
\pgfpathlineto{\pgfqpoint{2.632687in}{2.815545in}}%
\pgfpathlineto{\pgfqpoint{2.632687in}{2.815545in}}%
\pgfpathclose%
\pgfusepath{stroke,fill}%
\end{pgfscope}%
\begin{pgfscope}%
\pgfpathrectangle{\pgfqpoint{0.800000in}{1.400000in}}{\pgfqpoint{2.407767in}{1.544118in}}%
\pgfusepath{clip}%
\pgfsetroundcap%
\pgfsetroundjoin%
\pgfsetlinewidth{0.501875pt}%
\definecolor{currentstroke}{rgb}{0.271305,0.019942,0.347269}%
\pgfsetstrokecolor{currentstroke}%
\pgfsetdash{}{0pt}%
\pgfpathmoveto{\pgfqpoint{2.548213in}{2.797753in}}%
\pgfpathquadraticcurveto{\pgfqpoint{2.535004in}{2.797298in}}{\pgfqpoint{2.529555in}{2.797111in}}%
\pgfusepath{stroke}%
\end{pgfscope}%
\begin{pgfscope}%
\pgfpathrectangle{\pgfqpoint{0.800000in}{1.400000in}}{\pgfqpoint{2.407767in}{1.544118in}}%
\pgfusepath{clip}%
\pgfsetroundcap%
\pgfsetroundjoin%
\definecolor{currentfill}{rgb}{0.271305,0.019942,0.347269}%
\pgfsetfillcolor{currentfill}%
\pgfsetlinewidth{0.501875pt}%
\definecolor{currentstroke}{rgb}{0.271305,0.019942,0.347269}%
\pgfsetstrokecolor{currentstroke}%
\pgfsetdash{}{0pt}%
\pgfpathmoveto{\pgfqpoint{2.557794in}{2.784186in}}%
\pgfpathlineto{\pgfqpoint{2.529555in}{2.797111in}}%
\pgfpathlineto{\pgfqpoint{2.556838in}{2.811947in}}%
\pgfpathlineto{\pgfqpoint{2.557794in}{2.784186in}}%
\pgfpathlineto{\pgfqpoint{2.557794in}{2.784186in}}%
\pgfpathclose%
\pgfusepath{stroke,fill}%
\end{pgfscope}%
\begin{pgfscope}%
\pgfpathrectangle{\pgfqpoint{0.800000in}{1.400000in}}{\pgfqpoint{2.407767in}{1.544118in}}%
\pgfusepath{clip}%
\pgfsetroundcap%
\pgfsetroundjoin%
\pgfsetlinewidth{0.501875pt}%
\definecolor{currentstroke}{rgb}{0.272594,0.025563,0.353093}%
\pgfsetstrokecolor{currentstroke}%
\pgfsetdash{}{0pt}%
\pgfpathmoveto{\pgfqpoint{1.900277in}{1.585709in}}%
\pgfpathquadraticcurveto{\pgfqpoint{1.899972in}{1.587083in}}{\pgfqpoint{1.901347in}{1.580877in}}%
\pgfusepath{stroke}%
\end{pgfscope}%
\begin{pgfscope}%
\pgfpathrectangle{\pgfqpoint{0.800000in}{1.400000in}}{\pgfqpoint{2.407767in}{1.544118in}}%
\pgfusepath{clip}%
\pgfsetroundcap%
\pgfsetroundjoin%
\definecolor{currentfill}{rgb}{0.272594,0.025563,0.353093}%
\pgfsetfillcolor{currentfill}%
\pgfsetlinewidth{0.501875pt}%
\definecolor{currentstroke}{rgb}{0.272594,0.025563,0.353093}%
\pgfsetstrokecolor{currentstroke}%
\pgfsetdash{}{0pt}%
\pgfpathmoveto{\pgfqpoint{1.893796in}{1.550752in}}%
\pgfpathlineto{\pgfqpoint{1.901347in}{1.580877in}}%
\pgfpathlineto{\pgfqpoint{1.920916in}{1.556761in}}%
\pgfpathlineto{\pgfqpoint{1.893796in}{1.550752in}}%
\pgfpathlineto{\pgfqpoint{1.893796in}{1.550752in}}%
\pgfpathclose%
\pgfusepath{stroke,fill}%
\end{pgfscope}%
\begin{pgfscope}%
\pgfpathrectangle{\pgfqpoint{0.800000in}{1.400000in}}{\pgfqpoint{2.407767in}{1.544118in}}%
\pgfusepath{clip}%
\pgfsetroundcap%
\pgfsetroundjoin%
\pgfsetlinewidth{0.501875pt}%
\definecolor{currentstroke}{rgb}{0.273809,0.031497,0.358853}%
\pgfsetstrokecolor{currentstroke}%
\pgfsetdash{}{0pt}%
\pgfpathmoveto{\pgfqpoint{2.430994in}{1.601226in}}%
\pgfpathquadraticcurveto{\pgfqpoint{2.417798in}{1.601859in}}{\pgfqpoint{2.412357in}{1.602121in}}%
\pgfusepath{stroke}%
\end{pgfscope}%
\begin{pgfscope}%
\pgfpathrectangle{\pgfqpoint{0.800000in}{1.400000in}}{\pgfqpoint{2.407767in}{1.544118in}}%
\pgfusepath{clip}%
\pgfsetroundcap%
\pgfsetroundjoin%
\definecolor{currentfill}{rgb}{0.273809,0.031497,0.358853}%
\pgfsetfillcolor{currentfill}%
\pgfsetlinewidth{0.501875pt}%
\definecolor{currentstroke}{rgb}{0.273809,0.031497,0.358853}%
\pgfsetstrokecolor{currentstroke}%
\pgfsetdash{}{0pt}%
\pgfpathmoveto{\pgfqpoint{2.439437in}{1.586916in}}%
\pgfpathlineto{\pgfqpoint{2.412357in}{1.602121in}}%
\pgfpathlineto{\pgfqpoint{2.440769in}{1.614662in}}%
\pgfpathlineto{\pgfqpoint{2.439437in}{1.586916in}}%
\pgfpathlineto{\pgfqpoint{2.439437in}{1.586916in}}%
\pgfpathclose%
\pgfusepath{stroke,fill}%
\end{pgfscope}%
\begin{pgfscope}%
\pgfpathrectangle{\pgfqpoint{0.800000in}{1.400000in}}{\pgfqpoint{2.407767in}{1.544118in}}%
\pgfusepath{clip}%
\pgfsetroundcap%
\pgfsetroundjoin%
\pgfsetlinewidth{0.501875pt}%
\definecolor{currentstroke}{rgb}{0.280267,0.073417,0.397163}%
\pgfsetstrokecolor{currentstroke}%
\pgfsetdash{}{0pt}%
\pgfpathmoveto{\pgfqpoint{1.874755in}{1.820945in}}%
\pgfpathquadraticcurveto{\pgfqpoint{1.869378in}{1.828407in}}{\pgfqpoint{1.868539in}{1.829570in}}%
\pgfusepath{stroke}%
\end{pgfscope}%
\begin{pgfscope}%
\pgfpathrectangle{\pgfqpoint{0.800000in}{1.400000in}}{\pgfqpoint{2.407767in}{1.544118in}}%
\pgfusepath{clip}%
\pgfsetroundcap%
\pgfsetroundjoin%
\definecolor{currentfill}{rgb}{0.280267,0.073417,0.397163}%
\pgfsetfillcolor{currentfill}%
\pgfsetlinewidth{0.501875pt}%
\definecolor{currentstroke}{rgb}{0.280267,0.073417,0.397163}%
\pgfsetstrokecolor{currentstroke}%
\pgfsetdash{}{0pt}%
\pgfpathmoveto{\pgfqpoint{1.873513in}{1.798914in}}%
\pgfpathlineto{\pgfqpoint{1.868539in}{1.829570in}}%
\pgfpathlineto{\pgfqpoint{1.896048in}{1.815155in}}%
\pgfpathlineto{\pgfqpoint{1.873513in}{1.798914in}}%
\pgfpathlineto{\pgfqpoint{1.873513in}{1.798914in}}%
\pgfpathclose%
\pgfusepath{stroke,fill}%
\end{pgfscope}%
\begin{pgfscope}%
\pgfpathrectangle{\pgfqpoint{0.800000in}{1.400000in}}{\pgfqpoint{2.407767in}{1.544118in}}%
\pgfusepath{clip}%
\pgfsetroundcap%
\pgfsetroundjoin%
\pgfsetlinewidth{0.501875pt}%
\definecolor{currentstroke}{rgb}{0.276022,0.044167,0.370164}%
\pgfsetstrokecolor{currentstroke}%
\pgfsetdash{}{0pt}%
\pgfpathmoveto{\pgfqpoint{2.119668in}{1.705292in}}%
\pgfpathquadraticcurveto{\pgfqpoint{2.107258in}{1.708014in}}{\pgfqpoint{2.102432in}{1.709072in}}%
\pgfusepath{stroke}%
\end{pgfscope}%
\begin{pgfscope}%
\pgfpathrectangle{\pgfqpoint{0.800000in}{1.400000in}}{\pgfqpoint{2.407767in}{1.544118in}}%
\pgfusepath{clip}%
\pgfsetroundcap%
\pgfsetroundjoin%
\definecolor{currentfill}{rgb}{0.276022,0.044167,0.370164}%
\pgfsetfillcolor{currentfill}%
\pgfsetlinewidth{0.501875pt}%
\definecolor{currentstroke}{rgb}{0.276022,0.044167,0.370164}%
\pgfsetstrokecolor{currentstroke}%
\pgfsetdash{}{0pt}%
\pgfpathmoveto{\pgfqpoint{2.126590in}{1.689556in}}%
\pgfpathlineto{\pgfqpoint{2.102432in}{1.709072in}}%
\pgfpathlineto{\pgfqpoint{2.132540in}{1.716689in}}%
\pgfpathlineto{\pgfqpoint{2.126590in}{1.689556in}}%
\pgfpathlineto{\pgfqpoint{2.126590in}{1.689556in}}%
\pgfpathclose%
\pgfusepath{stroke,fill}%
\end{pgfscope}%
\begin{pgfscope}%
\pgfpathrectangle{\pgfqpoint{0.800000in}{1.400000in}}{\pgfqpoint{2.407767in}{1.544118in}}%
\pgfusepath{clip}%
\pgfsetroundcap%
\pgfsetroundjoin%
\pgfsetlinewidth{0.501875pt}%
\definecolor{currentstroke}{rgb}{0.274952,0.037752,0.364543}%
\pgfsetstrokecolor{currentstroke}%
\pgfsetdash{}{0pt}%
\pgfpathmoveto{\pgfqpoint{2.495165in}{2.656864in}}%
\pgfpathquadraticcurveto{\pgfqpoint{2.481931in}{2.656584in}}{\pgfqpoint{2.476460in}{2.656469in}}%
\pgfusepath{stroke}%
\end{pgfscope}%
\begin{pgfscope}%
\pgfpathrectangle{\pgfqpoint{0.800000in}{1.400000in}}{\pgfqpoint{2.407767in}{1.544118in}}%
\pgfusepath{clip}%
\pgfsetroundcap%
\pgfsetroundjoin%
\definecolor{currentfill}{rgb}{0.274952,0.037752,0.364543}%
\pgfsetfillcolor{currentfill}%
\pgfsetlinewidth{0.501875pt}%
\definecolor{currentstroke}{rgb}{0.274952,0.037752,0.364543}%
\pgfsetstrokecolor{currentstroke}%
\pgfsetdash{}{0pt}%
\pgfpathmoveto{\pgfqpoint{2.504525in}{2.643170in}}%
\pgfpathlineto{\pgfqpoint{2.476460in}{2.656469in}}%
\pgfpathlineto{\pgfqpoint{2.503939in}{2.670941in}}%
\pgfpathlineto{\pgfqpoint{2.504525in}{2.643170in}}%
\pgfpathlineto{\pgfqpoint{2.504525in}{2.643170in}}%
\pgfpathclose%
\pgfusepath{stroke,fill}%
\end{pgfscope}%
\begin{pgfscope}%
\pgfpathrectangle{\pgfqpoint{0.800000in}{1.400000in}}{\pgfqpoint{2.407767in}{1.544118in}}%
\pgfusepath{clip}%
\pgfsetroundcap%
\pgfsetroundjoin%
\pgfsetlinewidth{0.501875pt}%
\definecolor{currentstroke}{rgb}{0.277941,0.056324,0.381191}%
\pgfsetstrokecolor{currentstroke}%
\pgfsetdash{}{0pt}%
\pgfpathmoveto{\pgfqpoint{2.431571in}{1.709404in}}%
\pgfpathquadraticcurveto{\pgfqpoint{2.418358in}{1.709952in}}{\pgfqpoint{2.412901in}{1.710179in}}%
\pgfusepath{stroke}%
\end{pgfscope}%
\begin{pgfscope}%
\pgfpathrectangle{\pgfqpoint{0.800000in}{1.400000in}}{\pgfqpoint{2.407767in}{1.544118in}}%
\pgfusepath{clip}%
\pgfsetroundcap%
\pgfsetroundjoin%
\definecolor{currentfill}{rgb}{0.277941,0.056324,0.381191}%
\pgfsetfillcolor{currentfill}%
\pgfsetlinewidth{0.501875pt}%
\definecolor{currentstroke}{rgb}{0.277941,0.056324,0.381191}%
\pgfsetstrokecolor{currentstroke}%
\pgfsetdash{}{0pt}%
\pgfpathmoveto{\pgfqpoint{2.440079in}{1.695150in}}%
\pgfpathlineto{\pgfqpoint{2.412901in}{1.710179in}}%
\pgfpathlineto{\pgfqpoint{2.441231in}{1.722904in}}%
\pgfpathlineto{\pgfqpoint{2.440079in}{1.695150in}}%
\pgfpathlineto{\pgfqpoint{2.440079in}{1.695150in}}%
\pgfpathclose%
\pgfusepath{stroke,fill}%
\end{pgfscope}%
\begin{pgfscope}%
\pgfpathrectangle{\pgfqpoint{0.800000in}{1.400000in}}{\pgfqpoint{2.407767in}{1.544118in}}%
\pgfusepath{clip}%
\pgfsetroundcap%
\pgfsetroundjoin%
\pgfsetlinewidth{0.501875pt}%
\definecolor{currentstroke}{rgb}{0.276022,0.044167,0.370164}%
\pgfsetstrokecolor{currentstroke}%
\pgfsetdash{}{0pt}%
\pgfpathmoveto{\pgfqpoint{2.495167in}{2.621777in}}%
\pgfpathquadraticcurveto{\pgfqpoint{2.481932in}{2.621498in}}{\pgfqpoint{2.476458in}{2.621382in}}%
\pgfusepath{stroke}%
\end{pgfscope}%
\begin{pgfscope}%
\pgfpathrectangle{\pgfqpoint{0.800000in}{1.400000in}}{\pgfqpoint{2.407767in}{1.544118in}}%
\pgfusepath{clip}%
\pgfsetroundcap%
\pgfsetroundjoin%
\definecolor{currentfill}{rgb}{0.276022,0.044167,0.370164}%
\pgfsetfillcolor{currentfill}%
\pgfsetlinewidth{0.501875pt}%
\definecolor{currentstroke}{rgb}{0.276022,0.044167,0.370164}%
\pgfsetstrokecolor{currentstroke}%
\pgfsetdash{}{0pt}%
\pgfpathmoveto{\pgfqpoint{2.504523in}{2.608083in}}%
\pgfpathlineto{\pgfqpoint{2.476458in}{2.621382in}}%
\pgfpathlineto{\pgfqpoint{2.503937in}{2.635854in}}%
\pgfpathlineto{\pgfqpoint{2.504523in}{2.608083in}}%
\pgfpathlineto{\pgfqpoint{2.504523in}{2.608083in}}%
\pgfpathclose%
\pgfusepath{stroke,fill}%
\end{pgfscope}%
\begin{pgfscope}%
\pgfpathrectangle{\pgfqpoint{0.800000in}{1.400000in}}{\pgfqpoint{2.407767in}{1.544118in}}%
\pgfusepath{clip}%
\pgfsetroundcap%
\pgfsetroundjoin%
\pgfsetlinewidth{0.501875pt}%
\definecolor{currentstroke}{rgb}{0.281446,0.084320,0.407414}%
\pgfsetstrokecolor{currentstroke}%
\pgfsetdash{}{0pt}%
\pgfpathmoveto{\pgfqpoint{1.891557in}{2.436001in}}%
\pgfpathquadraticcurveto{\pgfqpoint{1.887343in}{2.431063in}}{\pgfqpoint{1.888169in}{2.432030in}}%
\pgfusepath{stroke}%
\end{pgfscope}%
\begin{pgfscope}%
\pgfpathrectangle{\pgfqpoint{0.800000in}{1.400000in}}{\pgfqpoint{2.407767in}{1.544118in}}%
\pgfusepath{clip}%
\pgfsetroundcap%
\pgfsetroundjoin%
\definecolor{currentfill}{rgb}{0.281446,0.084320,0.407414}%
\pgfsetfillcolor{currentfill}%
\pgfsetlinewidth{0.501875pt}%
\definecolor{currentstroke}{rgb}{0.281446,0.084320,0.407414}%
\pgfsetstrokecolor{currentstroke}%
\pgfsetdash{}{0pt}%
\pgfpathmoveto{\pgfqpoint{1.916766in}{2.444144in}}%
\pgfpathlineto{\pgfqpoint{1.888169in}{2.432030in}}%
\pgfpathlineto{\pgfqpoint{1.895636in}{2.462176in}}%
\pgfpathlineto{\pgfqpoint{1.916766in}{2.444144in}}%
\pgfpathlineto{\pgfqpoint{1.916766in}{2.444144in}}%
\pgfpathclose%
\pgfusepath{stroke,fill}%
\end{pgfscope}%
\begin{pgfscope}%
\pgfpathrectangle{\pgfqpoint{0.800000in}{1.400000in}}{\pgfqpoint{2.407767in}{1.544118in}}%
\pgfusepath{clip}%
\pgfsetroundcap%
\pgfsetroundjoin%
\pgfsetlinewidth{0.501875pt}%
\definecolor{currentstroke}{rgb}{0.277018,0.050344,0.375715}%
\pgfsetstrokecolor{currentstroke}%
\pgfsetdash{}{0pt}%
\pgfpathmoveto{\pgfqpoint{1.815815in}{1.765916in}}%
\pgfpathquadraticcurveto{\pgfqpoint{1.819005in}{1.767832in}}{\pgfqpoint{1.815540in}{1.765751in}}%
\pgfusepath{stroke}%
\end{pgfscope}%
\begin{pgfscope}%
\pgfpathrectangle{\pgfqpoint{0.800000in}{1.400000in}}{\pgfqpoint{2.407767in}{1.544118in}}%
\pgfusepath{clip}%
\pgfsetroundcap%
\pgfsetroundjoin%
\definecolor{currentfill}{rgb}{0.277018,0.050344,0.375715}%
\pgfsetfillcolor{currentfill}%
\pgfsetlinewidth{0.501875pt}%
\definecolor{currentstroke}{rgb}{0.277018,0.050344,0.375715}%
\pgfsetstrokecolor{currentstroke}%
\pgfsetdash{}{0pt}%
\pgfpathmoveto{\pgfqpoint{1.784576in}{1.763355in}}%
\pgfpathlineto{\pgfqpoint{1.815540in}{1.765751in}}%
\pgfpathlineto{\pgfqpoint{1.798879in}{1.739542in}}%
\pgfpathlineto{\pgfqpoint{1.784576in}{1.763355in}}%
\pgfpathlineto{\pgfqpoint{1.784576in}{1.763355in}}%
\pgfpathclose%
\pgfusepath{stroke,fill}%
\end{pgfscope}%
\begin{pgfscope}%
\pgfpathrectangle{\pgfqpoint{0.800000in}{1.400000in}}{\pgfqpoint{2.407767in}{1.544118in}}%
\pgfusepath{clip}%
\pgfsetroundcap%
\pgfsetroundjoin%
\pgfsetlinewidth{0.501875pt}%
\definecolor{currentstroke}{rgb}{0.277941,0.056324,0.381191}%
\pgfsetstrokecolor{currentstroke}%
\pgfsetdash{}{0pt}%
\pgfpathmoveto{\pgfqpoint{2.271459in}{1.747527in}}%
\pgfpathquadraticcurveto{\pgfqpoint{2.258284in}{1.748262in}}{\pgfqpoint{2.252862in}{1.748564in}}%
\pgfusepath{stroke}%
\end{pgfscope}%
\begin{pgfscope}%
\pgfpathrectangle{\pgfqpoint{0.800000in}{1.400000in}}{\pgfqpoint{2.407767in}{1.544118in}}%
\pgfusepath{clip}%
\pgfsetroundcap%
\pgfsetroundjoin%
\definecolor{currentfill}{rgb}{0.277941,0.056324,0.381191}%
\pgfsetfillcolor{currentfill}%
\pgfsetlinewidth{0.501875pt}%
\definecolor{currentstroke}{rgb}{0.277941,0.056324,0.381191}%
\pgfsetstrokecolor{currentstroke}%
\pgfsetdash{}{0pt}%
\pgfpathmoveto{\pgfqpoint{2.279823in}{1.733150in}}%
\pgfpathlineto{\pgfqpoint{2.252862in}{1.748564in}}%
\pgfpathlineto{\pgfqpoint{2.281370in}{1.760884in}}%
\pgfpathlineto{\pgfqpoint{2.279823in}{1.733150in}}%
\pgfpathlineto{\pgfqpoint{2.279823in}{1.733150in}}%
\pgfpathclose%
\pgfusepath{stroke,fill}%
\end{pgfscope}%
\begin{pgfscope}%
\pgfpathrectangle{\pgfqpoint{0.800000in}{1.400000in}}{\pgfqpoint{2.407767in}{1.544118in}}%
\pgfusepath{clip}%
\pgfsetroundcap%
\pgfsetroundjoin%
\pgfsetlinewidth{0.501875pt}%
\definecolor{currentstroke}{rgb}{0.279566,0.067836,0.391917}%
\pgfsetstrokecolor{currentstroke}%
\pgfsetdash{}{0pt}%
\pgfpathmoveto{\pgfqpoint{2.442230in}{1.792595in}}%
\pgfpathquadraticcurveto{\pgfqpoint{2.428999in}{1.792932in}}{\pgfqpoint{2.423530in}{1.793072in}}%
\pgfusepath{stroke}%
\end{pgfscope}%
\begin{pgfscope}%
\pgfpathrectangle{\pgfqpoint{0.800000in}{1.400000in}}{\pgfqpoint{2.407767in}{1.544118in}}%
\pgfusepath{clip}%
\pgfsetroundcap%
\pgfsetroundjoin%
\definecolor{currentfill}{rgb}{0.279566,0.067836,0.391917}%
\pgfsetfillcolor{currentfill}%
\pgfsetlinewidth{0.501875pt}%
\definecolor{currentstroke}{rgb}{0.279566,0.067836,0.391917}%
\pgfsetstrokecolor{currentstroke}%
\pgfsetdash{}{0pt}%
\pgfpathmoveto{\pgfqpoint{2.450944in}{1.778479in}}%
\pgfpathlineto{\pgfqpoint{2.423530in}{1.793072in}}%
\pgfpathlineto{\pgfqpoint{2.451653in}{1.806248in}}%
\pgfpathlineto{\pgfqpoint{2.450944in}{1.778479in}}%
\pgfpathlineto{\pgfqpoint{2.450944in}{1.778479in}}%
\pgfpathclose%
\pgfusepath{stroke,fill}%
\end{pgfscope}%
\begin{pgfscope}%
\pgfpathrectangle{\pgfqpoint{0.800000in}{1.400000in}}{\pgfqpoint{2.407767in}{1.544118in}}%
\pgfusepath{clip}%
\pgfsetroundcap%
\pgfsetroundjoin%
\pgfsetlinewidth{0.501875pt}%
\definecolor{currentstroke}{rgb}{0.280894,0.078907,0.402329}%
\pgfsetstrokecolor{currentstroke}%
\pgfsetdash{}{0pt}%
\pgfpathmoveto{\pgfqpoint{2.442216in}{1.827486in}}%
\pgfpathquadraticcurveto{\pgfqpoint{2.428986in}{1.827874in}}{\pgfqpoint{2.423517in}{1.828034in}}%
\pgfusepath{stroke}%
\end{pgfscope}%
\begin{pgfscope}%
\pgfpathrectangle{\pgfqpoint{0.800000in}{1.400000in}}{\pgfqpoint{2.407767in}{1.544118in}}%
\pgfusepath{clip}%
\pgfsetroundcap%
\pgfsetroundjoin%
\definecolor{currentfill}{rgb}{0.280894,0.078907,0.402329}%
\pgfsetfillcolor{currentfill}%
\pgfsetlinewidth{0.501875pt}%
\definecolor{currentstroke}{rgb}{0.280894,0.078907,0.402329}%
\pgfsetstrokecolor{currentstroke}%
\pgfsetdash{}{0pt}%
\pgfpathmoveto{\pgfqpoint{2.450876in}{1.813337in}}%
\pgfpathlineto{\pgfqpoint{2.423517in}{1.828034in}}%
\pgfpathlineto{\pgfqpoint{2.451690in}{1.841103in}}%
\pgfpathlineto{\pgfqpoint{2.450876in}{1.813337in}}%
\pgfpathlineto{\pgfqpoint{2.450876in}{1.813337in}}%
\pgfpathclose%
\pgfusepath{stroke,fill}%
\end{pgfscope}%
\begin{pgfscope}%
\pgfpathrectangle{\pgfqpoint{0.800000in}{1.400000in}}{\pgfqpoint{2.407767in}{1.544118in}}%
\pgfusepath{clip}%
\pgfsetroundcap%
\pgfsetroundjoin%
\pgfsetlinewidth{0.501875pt}%
\definecolor{currentstroke}{rgb}{0.283197,0.115680,0.436115}%
\pgfsetstrokecolor{currentstroke}%
\pgfsetdash{}{0pt}%
\pgfpathmoveto{\pgfqpoint{2.125990in}{1.880086in}}%
\pgfpathquadraticcurveto{\pgfqpoint{2.113048in}{1.881876in}}{\pgfqpoint{2.107797in}{1.882602in}}%
\pgfusepath{stroke}%
\end{pgfscope}%
\begin{pgfscope}%
\pgfpathrectangle{\pgfqpoint{0.800000in}{1.400000in}}{\pgfqpoint{2.407767in}{1.544118in}}%
\pgfusepath{clip}%
\pgfsetroundcap%
\pgfsetroundjoin%
\definecolor{currentfill}{rgb}{0.283197,0.115680,0.436115}%
\pgfsetfillcolor{currentfill}%
\pgfsetlinewidth{0.501875pt}%
\definecolor{currentstroke}{rgb}{0.283197,0.115680,0.436115}%
\pgfsetstrokecolor{currentstroke}%
\pgfsetdash{}{0pt}%
\pgfpathmoveto{\pgfqpoint{2.133411in}{1.865039in}}%
\pgfpathlineto{\pgfqpoint{2.107797in}{1.882602in}}%
\pgfpathlineto{\pgfqpoint{2.137216in}{1.892555in}}%
\pgfpathlineto{\pgfqpoint{2.133411in}{1.865039in}}%
\pgfpathlineto{\pgfqpoint{2.133411in}{1.865039in}}%
\pgfpathclose%
\pgfusepath{stroke,fill}%
\end{pgfscope}%
\begin{pgfscope}%
\pgfpathrectangle{\pgfqpoint{0.800000in}{1.400000in}}{\pgfqpoint{2.407767in}{1.544118in}}%
\pgfusepath{clip}%
\pgfsetroundcap%
\pgfsetroundjoin%
\pgfsetlinewidth{0.501875pt}%
\definecolor{currentstroke}{rgb}{0.282884,0.135920,0.453427}%
\pgfsetstrokecolor{currentstroke}%
\pgfsetdash{}{0pt}%
\pgfpathmoveto{\pgfqpoint{2.336332in}{1.899162in}}%
\pgfpathquadraticcurveto{\pgfqpoint{2.323120in}{1.899739in}}{\pgfqpoint{2.317665in}{1.899977in}}%
\pgfusepath{stroke}%
\end{pgfscope}%
\begin{pgfscope}%
\pgfpathrectangle{\pgfqpoint{0.800000in}{1.400000in}}{\pgfqpoint{2.407767in}{1.544118in}}%
\pgfusepath{clip}%
\pgfsetroundcap%
\pgfsetroundjoin%
\definecolor{currentfill}{rgb}{0.282884,0.135920,0.453427}%
\pgfsetfillcolor{currentfill}%
\pgfsetlinewidth{0.501875pt}%
\definecolor{currentstroke}{rgb}{0.282884,0.135920,0.453427}%
\pgfsetstrokecolor{currentstroke}%
\pgfsetdash{}{0pt}%
\pgfpathmoveto{\pgfqpoint{2.344811in}{1.884889in}}%
\pgfpathlineto{\pgfqpoint{2.317665in}{1.899977in}}%
\pgfpathlineto{\pgfqpoint{2.346022in}{1.912641in}}%
\pgfpathlineto{\pgfqpoint{2.344811in}{1.884889in}}%
\pgfpathlineto{\pgfqpoint{2.344811in}{1.884889in}}%
\pgfpathclose%
\pgfusepath{stroke,fill}%
\end{pgfscope}%
\begin{pgfscope}%
\pgfpathrectangle{\pgfqpoint{0.800000in}{1.400000in}}{\pgfqpoint{2.407767in}{1.544118in}}%
\pgfusepath{clip}%
\pgfsetroundcap%
\pgfsetroundjoin%
\pgfsetlinewidth{0.501875pt}%
\definecolor{currentstroke}{rgb}{0.283197,0.115680,0.436115}%
\pgfsetstrokecolor{currentstroke}%
\pgfsetdash{}{0pt}%
\pgfpathmoveto{\pgfqpoint{2.442182in}{1.931284in}}%
\pgfpathquadraticcurveto{\pgfqpoint{2.428944in}{1.931531in}}{\pgfqpoint{2.423468in}{1.931633in}}%
\pgfusepath{stroke}%
\end{pgfscope}%
\begin{pgfscope}%
\pgfpathrectangle{\pgfqpoint{0.800000in}{1.400000in}}{\pgfqpoint{2.407767in}{1.544118in}}%
\pgfusepath{clip}%
\pgfsetroundcap%
\pgfsetroundjoin%
\definecolor{currentfill}{rgb}{0.283197,0.115680,0.436115}%
\pgfsetfillcolor{currentfill}%
\pgfsetlinewidth{0.501875pt}%
\definecolor{currentstroke}{rgb}{0.283197,0.115680,0.436115}%
\pgfsetstrokecolor{currentstroke}%
\pgfsetdash{}{0pt}%
\pgfpathmoveto{\pgfqpoint{2.450982in}{1.917228in}}%
\pgfpathlineto{\pgfqpoint{2.423468in}{1.931633in}}%
\pgfpathlineto{\pgfqpoint{2.451501in}{1.945001in}}%
\pgfpathlineto{\pgfqpoint{2.450982in}{1.917228in}}%
\pgfpathlineto{\pgfqpoint{2.450982in}{1.917228in}}%
\pgfpathclose%
\pgfusepath{stroke,fill}%
\end{pgfscope}%
\begin{pgfscope}%
\pgfpathrectangle{\pgfqpoint{0.800000in}{1.400000in}}{\pgfqpoint{2.407767in}{1.544118in}}%
\pgfusepath{clip}%
\pgfsetroundcap%
\pgfsetroundjoin%
\pgfsetlinewidth{0.501875pt}%
\definecolor{currentstroke}{rgb}{0.270595,0.214069,0.507052}%
\pgfsetstrokecolor{currentstroke}%
\pgfsetdash{}{0pt}%
\pgfpathmoveto{\pgfqpoint{2.124722in}{1.974505in}}%
\pgfpathquadraticcurveto{\pgfqpoint{2.111540in}{1.975321in}}{\pgfqpoint{2.106107in}{1.975658in}}%
\pgfusepath{stroke}%
\end{pgfscope}%
\begin{pgfscope}%
\pgfpathrectangle{\pgfqpoint{0.800000in}{1.400000in}}{\pgfqpoint{2.407767in}{1.544118in}}%
\pgfusepath{clip}%
\pgfsetroundcap%
\pgfsetroundjoin%
\definecolor{currentfill}{rgb}{0.270595,0.214069,0.507052}%
\pgfsetfillcolor{currentfill}%
\pgfsetlinewidth{0.501875pt}%
\definecolor{currentstroke}{rgb}{0.270595,0.214069,0.507052}%
\pgfsetstrokecolor{currentstroke}%
\pgfsetdash{}{0pt}%
\pgfpathmoveto{\pgfqpoint{2.132974in}{1.960079in}}%
\pgfpathlineto{\pgfqpoint{2.106107in}{1.975658in}}%
\pgfpathlineto{\pgfqpoint{2.134690in}{1.987804in}}%
\pgfpathlineto{\pgfqpoint{2.132974in}{1.960079in}}%
\pgfpathlineto{\pgfqpoint{2.132974in}{1.960079in}}%
\pgfpathclose%
\pgfusepath{stroke,fill}%
\end{pgfscope}%
\begin{pgfscope}%
\pgfpathrectangle{\pgfqpoint{0.800000in}{1.400000in}}{\pgfqpoint{2.407767in}{1.544118in}}%
\pgfusepath{clip}%
\pgfsetroundcap%
\pgfsetroundjoin%
\pgfsetlinewidth{0.501875pt}%
\definecolor{currentstroke}{rgb}{0.280868,0.160771,0.472899}%
\pgfsetstrokecolor{currentstroke}%
\pgfsetdash{}{0pt}%
\pgfpathmoveto{\pgfqpoint{2.389189in}{2.000101in}}%
\pgfpathquadraticcurveto{\pgfqpoint{2.375949in}{2.000296in}}{\pgfqpoint{2.370472in}{2.000376in}}%
\pgfusepath{stroke}%
\end{pgfscope}%
\begin{pgfscope}%
\pgfpathrectangle{\pgfqpoint{0.800000in}{1.400000in}}{\pgfqpoint{2.407767in}{1.544118in}}%
\pgfusepath{clip}%
\pgfsetroundcap%
\pgfsetroundjoin%
\definecolor{currentfill}{rgb}{0.280868,0.160771,0.472899}%
\pgfsetfillcolor{currentfill}%
\pgfsetlinewidth{0.501875pt}%
\definecolor{currentstroke}{rgb}{0.280868,0.160771,0.472899}%
\pgfsetstrokecolor{currentstroke}%
\pgfsetdash{}{0pt}%
\pgfpathmoveto{\pgfqpoint{2.398042in}{1.986080in}}%
\pgfpathlineto{\pgfqpoint{2.370472in}{2.000376in}}%
\pgfpathlineto{\pgfqpoint{2.398451in}{2.013855in}}%
\pgfpathlineto{\pgfqpoint{2.398042in}{1.986080in}}%
\pgfpathlineto{\pgfqpoint{2.398042in}{1.986080in}}%
\pgfpathclose%
\pgfusepath{stroke,fill}%
\end{pgfscope}%
\begin{pgfscope}%
\pgfpathrectangle{\pgfqpoint{0.800000in}{1.400000in}}{\pgfqpoint{2.407767in}{1.544118in}}%
\pgfusepath{clip}%
\pgfsetroundcap%
\pgfsetroundjoin%
\pgfsetlinewidth{0.501875pt}%
\definecolor{currentstroke}{rgb}{0.250425,0.274290,0.533103}%
\pgfsetstrokecolor{currentstroke}%
\pgfsetdash{}{0pt}%
\pgfpathmoveto{\pgfqpoint{2.124524in}{2.041960in}}%
\pgfpathquadraticcurveto{\pgfqpoint{2.111316in}{2.042581in}}{\pgfqpoint{2.105864in}{2.042837in}}%
\pgfusepath{stroke}%
\end{pgfscope}%
\begin{pgfscope}%
\pgfpathrectangle{\pgfqpoint{0.800000in}{1.400000in}}{\pgfqpoint{2.407767in}{1.544118in}}%
\pgfusepath{clip}%
\pgfsetroundcap%
\pgfsetroundjoin%
\definecolor{currentfill}{rgb}{0.250425,0.274290,0.533103}%
\pgfsetfillcolor{currentfill}%
\pgfsetlinewidth{0.501875pt}%
\definecolor{currentstroke}{rgb}{0.250425,0.274290,0.533103}%
\pgfsetstrokecolor{currentstroke}%
\pgfsetdash{}{0pt}%
\pgfpathmoveto{\pgfqpoint{2.132959in}{2.027660in}}%
\pgfpathlineto{\pgfqpoint{2.105864in}{2.042837in}}%
\pgfpathlineto{\pgfqpoint{2.134262in}{2.055407in}}%
\pgfpathlineto{\pgfqpoint{2.132959in}{2.027660in}}%
\pgfpathlineto{\pgfqpoint{2.132959in}{2.027660in}}%
\pgfpathclose%
\pgfusepath{stroke,fill}%
\end{pgfscope}%
\begin{pgfscope}%
\pgfpathrectangle{\pgfqpoint{0.800000in}{1.400000in}}{\pgfqpoint{2.407767in}{1.544118in}}%
\pgfusepath{clip}%
\pgfsetroundcap%
\pgfsetroundjoin%
\pgfsetlinewidth{0.501875pt}%
\definecolor{currentstroke}{rgb}{0.273006,0.204520,0.501721}%
\pgfsetstrokecolor{currentstroke}%
\pgfsetdash{}{0pt}%
\pgfpathmoveto{\pgfqpoint{2.336224in}{2.070224in}}%
\pgfpathquadraticcurveto{\pgfqpoint{2.322985in}{2.070451in}}{\pgfqpoint{2.317509in}{2.070545in}}%
\pgfusepath{stroke}%
\end{pgfscope}%
\begin{pgfscope}%
\pgfpathrectangle{\pgfqpoint{0.800000in}{1.400000in}}{\pgfqpoint{2.407767in}{1.544118in}}%
\pgfusepath{clip}%
\pgfsetroundcap%
\pgfsetroundjoin%
\definecolor{currentfill}{rgb}{0.273006,0.204520,0.501721}%
\pgfsetfillcolor{currentfill}%
\pgfsetlinewidth{0.501875pt}%
\definecolor{currentstroke}{rgb}{0.273006,0.204520,0.501721}%
\pgfsetstrokecolor{currentstroke}%
\pgfsetdash{}{0pt}%
\pgfpathmoveto{\pgfqpoint{2.345044in}{2.056181in}}%
\pgfpathlineto{\pgfqpoint{2.317509in}{2.070545in}}%
\pgfpathlineto{\pgfqpoint{2.345521in}{2.083955in}}%
\pgfpathlineto{\pgfqpoint{2.345044in}{2.056181in}}%
\pgfpathlineto{\pgfqpoint{2.345044in}{2.056181in}}%
\pgfpathclose%
\pgfusepath{stroke,fill}%
\end{pgfscope}%
\begin{pgfscope}%
\pgfpathrectangle{\pgfqpoint{0.800000in}{1.400000in}}{\pgfqpoint{2.407767in}{1.544118in}}%
\pgfusepath{clip}%
\pgfsetroundcap%
\pgfsetroundjoin%
\pgfsetlinewidth{0.501875pt}%
\definecolor{currentstroke}{rgb}{0.237441,0.305202,0.541921}%
\pgfsetstrokecolor{currentstroke}%
\pgfsetdash{}{0pt}%
\pgfpathmoveto{\pgfqpoint{2.124347in}{2.105796in}}%
\pgfpathquadraticcurveto{\pgfqpoint{2.111109in}{2.106032in}}{\pgfqpoint{2.105633in}{2.106129in}}%
\pgfusepath{stroke}%
\end{pgfscope}%
\begin{pgfscope}%
\pgfpathrectangle{\pgfqpoint{0.800000in}{1.400000in}}{\pgfqpoint{2.407767in}{1.544118in}}%
\pgfusepath{clip}%
\pgfsetroundcap%
\pgfsetroundjoin%
\definecolor{currentfill}{rgb}{0.237441,0.305202,0.541921}%
\pgfsetfillcolor{currentfill}%
\pgfsetlinewidth{0.501875pt}%
\definecolor{currentstroke}{rgb}{0.237441,0.305202,0.541921}%
\pgfsetstrokecolor{currentstroke}%
\pgfsetdash{}{0pt}%
\pgfpathmoveto{\pgfqpoint{2.133160in}{2.091749in}}%
\pgfpathlineto{\pgfqpoint{2.105633in}{2.106129in}}%
\pgfpathlineto{\pgfqpoint{2.133653in}{2.119522in}}%
\pgfpathlineto{\pgfqpoint{2.133160in}{2.091749in}}%
\pgfpathlineto{\pgfqpoint{2.133160in}{2.091749in}}%
\pgfpathclose%
\pgfusepath{stroke,fill}%
\end{pgfscope}%
\begin{pgfscope}%
\pgfpathrectangle{\pgfqpoint{0.800000in}{1.400000in}}{\pgfqpoint{2.407767in}{1.544118in}}%
\pgfusepath{clip}%
\pgfsetroundcap%
\pgfsetroundjoin%
\pgfsetlinewidth{0.501875pt}%
\definecolor{currentstroke}{rgb}{0.239346,0.300855,0.540844}%
\pgfsetstrokecolor{currentstroke}%
\pgfsetdash{}{0pt}%
\pgfpathmoveto{\pgfqpoint{2.230242in}{2.136903in}}%
\pgfpathquadraticcurveto{\pgfqpoint{2.216999in}{2.136915in}}{\pgfqpoint{2.211519in}{2.136920in}}%
\pgfusepath{stroke}%
\end{pgfscope}%
\begin{pgfscope}%
\pgfpathrectangle{\pgfqpoint{0.800000in}{1.400000in}}{\pgfqpoint{2.407767in}{1.544118in}}%
\pgfusepath{clip}%
\pgfsetroundcap%
\pgfsetroundjoin%
\definecolor{currentfill}{rgb}{0.239346,0.300855,0.540844}%
\pgfsetfillcolor{currentfill}%
\pgfsetlinewidth{0.501875pt}%
\definecolor{currentstroke}{rgb}{0.239346,0.300855,0.540844}%
\pgfsetstrokecolor{currentstroke}%
\pgfsetdash{}{0pt}%
\pgfpathmoveto{\pgfqpoint{2.239284in}{2.123006in}}%
\pgfpathlineto{\pgfqpoint{2.211519in}{2.136920in}}%
\pgfpathlineto{\pgfqpoint{2.239309in}{2.150784in}}%
\pgfpathlineto{\pgfqpoint{2.239284in}{2.123006in}}%
\pgfpathlineto{\pgfqpoint{2.239284in}{2.123006in}}%
\pgfpathclose%
\pgfusepath{stroke,fill}%
\end{pgfscope}%
\begin{pgfscope}%
\pgfpathrectangle{\pgfqpoint{0.800000in}{1.400000in}}{\pgfqpoint{2.407767in}{1.544118in}}%
\pgfusepath{clip}%
\pgfsetroundcap%
\pgfsetroundjoin%
\pgfsetlinewidth{0.501875pt}%
\definecolor{currentstroke}{rgb}{0.227802,0.326594,0.546532}%
\pgfsetstrokecolor{currentstroke}%
\pgfsetdash{}{0pt}%
\pgfpathmoveto{\pgfqpoint{2.124308in}{2.171304in}}%
\pgfpathquadraticcurveto{\pgfqpoint{2.111065in}{2.171203in}}{\pgfqpoint{2.105586in}{2.171161in}}%
\pgfusepath{stroke}%
\end{pgfscope}%
\begin{pgfscope}%
\pgfpathrectangle{\pgfqpoint{0.800000in}{1.400000in}}{\pgfqpoint{2.407767in}{1.544118in}}%
\pgfusepath{clip}%
\pgfsetroundcap%
\pgfsetroundjoin%
\definecolor{currentfill}{rgb}{0.227802,0.326594,0.546532}%
\pgfsetfillcolor{currentfill}%
\pgfsetlinewidth{0.501875pt}%
\definecolor{currentstroke}{rgb}{0.227802,0.326594,0.546532}%
\pgfsetstrokecolor{currentstroke}%
\pgfsetdash{}{0pt}%
\pgfpathmoveto{\pgfqpoint{2.133469in}{2.157484in}}%
\pgfpathlineto{\pgfqpoint{2.105586in}{2.171161in}}%
\pgfpathlineto{\pgfqpoint{2.133257in}{2.185261in}}%
\pgfpathlineto{\pgfqpoint{2.133469in}{2.157484in}}%
\pgfpathlineto{\pgfqpoint{2.133469in}{2.157484in}}%
\pgfpathclose%
\pgfusepath{stroke,fill}%
\end{pgfscope}%
\begin{pgfscope}%
\pgfpathrectangle{\pgfqpoint{0.800000in}{1.400000in}}{\pgfqpoint{2.407767in}{1.544118in}}%
\pgfusepath{clip}%
\pgfsetroundcap%
\pgfsetroundjoin%
\pgfsetlinewidth{0.501875pt}%
\definecolor{currentstroke}{rgb}{0.258965,0.251537,0.524736}%
\pgfsetstrokecolor{currentstroke}%
\pgfsetdash{}{0pt}%
\pgfpathmoveto{\pgfqpoint{2.283247in}{2.204632in}}%
\pgfpathquadraticcurveto{\pgfqpoint{2.270006in}{2.204442in}}{\pgfqpoint{2.264529in}{2.204364in}}%
\pgfusepath{stroke}%
\end{pgfscope}%
\begin{pgfscope}%
\pgfpathrectangle{\pgfqpoint{0.800000in}{1.400000in}}{\pgfqpoint{2.407767in}{1.544118in}}%
\pgfusepath{clip}%
\pgfsetroundcap%
\pgfsetroundjoin%
\definecolor{currentfill}{rgb}{0.258965,0.251537,0.524736}%
\pgfsetfillcolor{currentfill}%
\pgfsetlinewidth{0.501875pt}%
\definecolor{currentstroke}{rgb}{0.258965,0.251537,0.524736}%
\pgfsetstrokecolor{currentstroke}%
\pgfsetdash{}{0pt}%
\pgfpathmoveto{\pgfqpoint{2.292503in}{2.190874in}}%
\pgfpathlineto{\pgfqpoint{2.264529in}{2.204364in}}%
\pgfpathlineto{\pgfqpoint{2.292105in}{2.218649in}}%
\pgfpathlineto{\pgfqpoint{2.292503in}{2.190874in}}%
\pgfpathlineto{\pgfqpoint{2.292503in}{2.190874in}}%
\pgfpathclose%
\pgfusepath{stroke,fill}%
\end{pgfscope}%
\begin{pgfscope}%
\pgfpathrectangle{\pgfqpoint{0.800000in}{1.400000in}}{\pgfqpoint{2.407767in}{1.544118in}}%
\pgfusepath{clip}%
\pgfsetroundcap%
\pgfsetroundjoin%
\pgfsetlinewidth{0.501875pt}%
\definecolor{currentstroke}{rgb}{0.271828,0.209303,0.504434}%
\pgfsetstrokecolor{currentstroke}%
\pgfsetdash{}{0pt}%
\pgfpathmoveto{\pgfqpoint{2.336229in}{2.239596in}}%
\pgfpathquadraticcurveto{\pgfqpoint{2.322990in}{2.239361in}}{\pgfqpoint{2.317514in}{2.239264in}}%
\pgfusepath{stroke}%
\end{pgfscope}%
\begin{pgfscope}%
\pgfpathrectangle{\pgfqpoint{0.800000in}{1.400000in}}{\pgfqpoint{2.407767in}{1.544118in}}%
\pgfusepath{clip}%
\pgfsetroundcap%
\pgfsetroundjoin%
\definecolor{currentfill}{rgb}{0.271828,0.209303,0.504434}%
\pgfsetfillcolor{currentfill}%
\pgfsetlinewidth{0.501875pt}%
\definecolor{currentstroke}{rgb}{0.271828,0.209303,0.504434}%
\pgfsetstrokecolor{currentstroke}%
\pgfsetdash{}{0pt}%
\pgfpathmoveto{\pgfqpoint{2.345534in}{2.225870in}}%
\pgfpathlineto{\pgfqpoint{2.317514in}{2.239264in}}%
\pgfpathlineto{\pgfqpoint{2.345041in}{2.253643in}}%
\pgfpathlineto{\pgfqpoint{2.345534in}{2.225870in}}%
\pgfpathlineto{\pgfqpoint{2.345534in}{2.225870in}}%
\pgfpathclose%
\pgfusepath{stroke,fill}%
\end{pgfscope}%
\begin{pgfscope}%
\pgfpathrectangle{\pgfqpoint{0.800000in}{1.400000in}}{\pgfqpoint{2.407767in}{1.544118in}}%
\pgfusepath{clip}%
\pgfsetroundcap%
\pgfsetroundjoin%
\pgfsetlinewidth{0.501875pt}%
\definecolor{currentstroke}{rgb}{0.279574,0.170599,0.479997}%
\pgfsetstrokecolor{currentstroke}%
\pgfsetdash{}{0pt}%
\pgfpathmoveto{\pgfqpoint{2.336239in}{2.274412in}}%
\pgfpathquadraticcurveto{\pgfqpoint{2.323004in}{2.274099in}}{\pgfqpoint{2.317532in}{2.273969in}}%
\pgfusepath{stroke}%
\end{pgfscope}%
\begin{pgfscope}%
\pgfpathrectangle{\pgfqpoint{0.800000in}{1.400000in}}{\pgfqpoint{2.407767in}{1.544118in}}%
\pgfusepath{clip}%
\pgfsetroundcap%
\pgfsetroundjoin%
\definecolor{currentfill}{rgb}{0.279574,0.170599,0.479997}%
\pgfsetfillcolor{currentfill}%
\pgfsetlinewidth{0.501875pt}%
\definecolor{currentstroke}{rgb}{0.279574,0.170599,0.479997}%
\pgfsetstrokecolor{currentstroke}%
\pgfsetdash{}{0pt}%
\pgfpathmoveto{\pgfqpoint{2.345630in}{2.260741in}}%
\pgfpathlineto{\pgfqpoint{2.317532in}{2.273969in}}%
\pgfpathlineto{\pgfqpoint{2.344973in}{2.288511in}}%
\pgfpathlineto{\pgfqpoint{2.345630in}{2.260741in}}%
\pgfpathlineto{\pgfqpoint{2.345630in}{2.260741in}}%
\pgfpathclose%
\pgfusepath{stroke,fill}%
\end{pgfscope}%
\begin{pgfscope}%
\pgfpathrectangle{\pgfqpoint{0.800000in}{1.400000in}}{\pgfqpoint{2.407767in}{1.544118in}}%
\pgfusepath{clip}%
\pgfsetroundcap%
\pgfsetroundjoin%
\pgfsetlinewidth{0.501875pt}%
\definecolor{currentstroke}{rgb}{0.282623,0.140926,0.457517}%
\pgfsetstrokecolor{currentstroke}%
\pgfsetdash{}{0pt}%
\pgfpathmoveto{\pgfqpoint{2.389205in}{2.309273in}}%
\pgfpathquadraticcurveto{\pgfqpoint{2.375972in}{2.308937in}}{\pgfqpoint{2.370500in}{2.308798in}}%
\pgfusepath{stroke}%
\end{pgfscope}%
\begin{pgfscope}%
\pgfpathrectangle{\pgfqpoint{0.800000in}{1.400000in}}{\pgfqpoint{2.407767in}{1.544118in}}%
\pgfusepath{clip}%
\pgfsetroundcap%
\pgfsetroundjoin%
\definecolor{currentfill}{rgb}{0.282623,0.140926,0.457517}%
\pgfsetfillcolor{currentfill}%
\pgfsetlinewidth{0.501875pt}%
\definecolor{currentstroke}{rgb}{0.282623,0.140926,0.457517}%
\pgfsetstrokecolor{currentstroke}%
\pgfsetdash{}{0pt}%
\pgfpathmoveto{\pgfqpoint{2.398621in}{2.295618in}}%
\pgfpathlineto{\pgfqpoint{2.370500in}{2.308798in}}%
\pgfpathlineto{\pgfqpoint{2.397917in}{2.323387in}}%
\pgfpathlineto{\pgfqpoint{2.398621in}{2.295618in}}%
\pgfpathlineto{\pgfqpoint{2.398621in}{2.295618in}}%
\pgfpathclose%
\pgfusepath{stroke,fill}%
\end{pgfscope}%
\begin{pgfscope}%
\pgfpathrectangle{\pgfqpoint{0.800000in}{1.400000in}}{\pgfqpoint{2.407767in}{1.544118in}}%
\pgfusepath{clip}%
\pgfsetroundcap%
\pgfsetroundjoin%
\pgfsetlinewidth{0.501875pt}%
\definecolor{currentstroke}{rgb}{0.279574,0.170599,0.479997}%
\pgfsetstrokecolor{currentstroke}%
\pgfsetdash{}{0pt}%
\pgfpathmoveto{\pgfqpoint{2.125369in}{2.329723in}}%
\pgfpathquadraticcurveto{\pgfqpoint{2.112285in}{2.328415in}}{\pgfqpoint{2.106926in}{2.327879in}}%
\pgfusepath{stroke}%
\end{pgfscope}%
\begin{pgfscope}%
\pgfpathrectangle{\pgfqpoint{0.800000in}{1.400000in}}{\pgfqpoint{2.407767in}{1.544118in}}%
\pgfusepath{clip}%
\pgfsetroundcap%
\pgfsetroundjoin%
\definecolor{currentfill}{rgb}{0.279574,0.170599,0.479997}%
\pgfsetfillcolor{currentfill}%
\pgfsetlinewidth{0.501875pt}%
\definecolor{currentstroke}{rgb}{0.279574,0.170599,0.479997}%
\pgfsetstrokecolor{currentstroke}%
\pgfsetdash{}{0pt}%
\pgfpathmoveto{\pgfqpoint{2.135948in}{2.316823in}}%
\pgfpathlineto{\pgfqpoint{2.106926in}{2.327879in}}%
\pgfpathlineto{\pgfqpoint{2.133184in}{2.344463in}}%
\pgfpathlineto{\pgfqpoint{2.135948in}{2.316823in}}%
\pgfpathlineto{\pgfqpoint{2.135948in}{2.316823in}}%
\pgfpathclose%
\pgfusepath{stroke,fill}%
\end{pgfscope}%
\begin{pgfscope}%
\pgfpathrectangle{\pgfqpoint{0.800000in}{1.400000in}}{\pgfqpoint{2.407767in}{1.544118in}}%
\pgfusepath{clip}%
\pgfsetroundcap%
\pgfsetroundjoin%
\pgfsetlinewidth{0.501875pt}%
\definecolor{currentstroke}{rgb}{0.283072,0.130895,0.449241}%
\pgfsetstrokecolor{currentstroke}%
\pgfsetdash{}{0pt}%
\pgfpathmoveto{\pgfqpoint{2.283327in}{2.376701in}}%
\pgfpathquadraticcurveto{\pgfqpoint{2.270118in}{2.376101in}}{\pgfqpoint{2.264666in}{2.375853in}}%
\pgfusepath{stroke}%
\end{pgfscope}%
\begin{pgfscope}%
\pgfpathrectangle{\pgfqpoint{0.800000in}{1.400000in}}{\pgfqpoint{2.407767in}{1.544118in}}%
\pgfusepath{clip}%
\pgfsetroundcap%
\pgfsetroundjoin%
\definecolor{currentfill}{rgb}{0.283072,0.130895,0.449241}%
\pgfsetfillcolor{currentfill}%
\pgfsetlinewidth{0.501875pt}%
\definecolor{currentstroke}{rgb}{0.283072,0.130895,0.449241}%
\pgfsetstrokecolor{currentstroke}%
\pgfsetdash{}{0pt}%
\pgfpathmoveto{\pgfqpoint{2.293045in}{2.363239in}}%
\pgfpathlineto{\pgfqpoint{2.264666in}{2.375853in}}%
\pgfpathlineto{\pgfqpoint{2.291785in}{2.390988in}}%
\pgfpathlineto{\pgfqpoint{2.293045in}{2.363239in}}%
\pgfpathlineto{\pgfqpoint{2.293045in}{2.363239in}}%
\pgfpathclose%
\pgfusepath{stroke,fill}%
\end{pgfscope}%
\begin{pgfscope}%
\pgfpathrectangle{\pgfqpoint{0.800000in}{1.400000in}}{\pgfqpoint{2.407767in}{1.544118in}}%
\pgfusepath{clip}%
\pgfsetroundcap%
\pgfsetroundjoin%
\pgfsetlinewidth{0.501875pt}%
\definecolor{currentstroke}{rgb}{0.283197,0.115680,0.436115}%
\pgfsetstrokecolor{currentstroke}%
\pgfsetdash{}{0pt}%
\pgfpathmoveto{\pgfqpoint{2.336300in}{2.411623in}}%
\pgfpathquadraticcurveto{\pgfqpoint{2.323081in}{2.411118in}}{\pgfqpoint{2.317622in}{2.410909in}}%
\pgfusepath{stroke}%
\end{pgfscope}%
\begin{pgfscope}%
\pgfpathrectangle{\pgfqpoint{0.800000in}{1.400000in}}{\pgfqpoint{2.407767in}{1.544118in}}%
\pgfusepath{clip}%
\pgfsetroundcap%
\pgfsetroundjoin%
\definecolor{currentfill}{rgb}{0.283197,0.115680,0.436115}%
\pgfsetfillcolor{currentfill}%
\pgfsetlinewidth{0.501875pt}%
\definecolor{currentstroke}{rgb}{0.283197,0.115680,0.436115}%
\pgfsetstrokecolor{currentstroke}%
\pgfsetdash{}{0pt}%
\pgfpathmoveto{\pgfqpoint{2.345910in}{2.398091in}}%
\pgfpathlineto{\pgfqpoint{2.317622in}{2.410909in}}%
\pgfpathlineto{\pgfqpoint{2.344849in}{2.425849in}}%
\pgfpathlineto{\pgfqpoint{2.345910in}{2.398091in}}%
\pgfpathlineto{\pgfqpoint{2.345910in}{2.398091in}}%
\pgfpathclose%
\pgfusepath{stroke,fill}%
\end{pgfscope}%
\begin{pgfscope}%
\pgfpathrectangle{\pgfqpoint{0.800000in}{1.400000in}}{\pgfqpoint{2.407767in}{1.544118in}}%
\pgfusepath{clip}%
\pgfsetroundcap%
\pgfsetroundjoin%
\pgfsetlinewidth{0.501875pt}%
\definecolor{currentstroke}{rgb}{0.282327,0.094955,0.417331}%
\pgfsetstrokecolor{currentstroke}%
\pgfsetdash{}{0pt}%
\pgfpathmoveto{\pgfqpoint{2.442171in}{2.448343in}}%
\pgfpathquadraticcurveto{\pgfqpoint{2.428937in}{2.448058in}}{\pgfqpoint{2.423464in}{2.447939in}}%
\pgfusepath{stroke}%
\end{pgfscope}%
\begin{pgfscope}%
\pgfpathrectangle{\pgfqpoint{0.800000in}{1.400000in}}{\pgfqpoint{2.407767in}{1.544118in}}%
\pgfusepath{clip}%
\pgfsetroundcap%
\pgfsetroundjoin%
\definecolor{currentfill}{rgb}{0.282327,0.094955,0.417331}%
\pgfsetfillcolor{currentfill}%
\pgfsetlinewidth{0.501875pt}%
\definecolor{currentstroke}{rgb}{0.282327,0.094955,0.417331}%
\pgfsetstrokecolor{currentstroke}%
\pgfsetdash{}{0pt}%
\pgfpathmoveto{\pgfqpoint{2.451535in}{2.434654in}}%
\pgfpathlineto{\pgfqpoint{2.423464in}{2.447939in}}%
\pgfpathlineto{\pgfqpoint{2.450935in}{2.462425in}}%
\pgfpathlineto{\pgfqpoint{2.451535in}{2.434654in}}%
\pgfpathlineto{\pgfqpoint{2.451535in}{2.434654in}}%
\pgfpathclose%
\pgfusepath{stroke,fill}%
\end{pgfscope}%
\begin{pgfscope}%
\pgfpathrectangle{\pgfqpoint{0.800000in}{1.400000in}}{\pgfqpoint{2.407767in}{1.544118in}}%
\pgfusepath{clip}%
\pgfsetroundcap%
\pgfsetroundjoin%
\pgfsetlinewidth{0.501875pt}%
\definecolor{currentstroke}{rgb}{0.282910,0.105393,0.426902}%
\pgfsetstrokecolor{currentstroke}%
\pgfsetdash{}{0pt}%
\pgfpathmoveto{\pgfqpoint{2.230755in}{2.475589in}}%
\pgfpathquadraticcurveto{\pgfqpoint{2.217631in}{2.474467in}}{\pgfqpoint{2.212242in}{2.474006in}}%
\pgfusepath{stroke}%
\end{pgfscope}%
\begin{pgfscope}%
\pgfpathrectangle{\pgfqpoint{0.800000in}{1.400000in}}{\pgfqpoint{2.407767in}{1.544118in}}%
\pgfusepath{clip}%
\pgfsetroundcap%
\pgfsetroundjoin%
\definecolor{currentfill}{rgb}{0.282910,0.105393,0.426902}%
\pgfsetfillcolor{currentfill}%
\pgfsetlinewidth{0.501875pt}%
\definecolor{currentstroke}{rgb}{0.282910,0.105393,0.426902}%
\pgfsetstrokecolor{currentstroke}%
\pgfsetdash{}{0pt}%
\pgfpathmoveto{\pgfqpoint{2.241102in}{2.462534in}}%
\pgfpathlineto{\pgfqpoint{2.212242in}{2.474006in}}%
\pgfpathlineto{\pgfqpoint{2.238736in}{2.490211in}}%
\pgfpathlineto{\pgfqpoint{2.241102in}{2.462534in}}%
\pgfpathlineto{\pgfqpoint{2.241102in}{2.462534in}}%
\pgfpathclose%
\pgfusepath{stroke,fill}%
\end{pgfscope}%
\begin{pgfscope}%
\pgfpathrectangle{\pgfqpoint{0.800000in}{1.400000in}}{\pgfqpoint{2.407767in}{1.544118in}}%
\pgfusepath{clip}%
\pgfsetroundcap%
\pgfsetroundjoin%
\pgfsetlinewidth{0.501875pt}%
\definecolor{currentstroke}{rgb}{0.281446,0.084320,0.407414}%
\pgfsetstrokecolor{currentstroke}%
\pgfsetdash{}{0pt}%
\pgfpathmoveto{\pgfqpoint{2.336399in}{2.514317in}}%
\pgfpathquadraticcurveto{\pgfqpoint{2.323206in}{2.513627in}}{\pgfqpoint{2.317766in}{2.513343in}}%
\pgfusepath{stroke}%
\end{pgfscope}%
\begin{pgfscope}%
\pgfpathrectangle{\pgfqpoint{0.800000in}{1.400000in}}{\pgfqpoint{2.407767in}{1.544118in}}%
\pgfusepath{clip}%
\pgfsetroundcap%
\pgfsetroundjoin%
\definecolor{currentfill}{rgb}{0.281446,0.084320,0.407414}%
\pgfsetfillcolor{currentfill}%
\pgfsetlinewidth{0.501875pt}%
\definecolor{currentstroke}{rgb}{0.281446,0.084320,0.407414}%
\pgfsetstrokecolor{currentstroke}%
\pgfsetdash{}{0pt}%
\pgfpathmoveto{\pgfqpoint{2.346231in}{2.500923in}}%
\pgfpathlineto{\pgfqpoint{2.317766in}{2.513343in}}%
\pgfpathlineto{\pgfqpoint{2.344781in}{2.528662in}}%
\pgfpathlineto{\pgfqpoint{2.346231in}{2.500923in}}%
\pgfpathlineto{\pgfqpoint{2.346231in}{2.500923in}}%
\pgfpathclose%
\pgfusepath{stroke,fill}%
\end{pgfscope}%
\begin{pgfscope}%
\pgfpathrectangle{\pgfqpoint{0.800000in}{1.400000in}}{\pgfqpoint{2.407767in}{1.544118in}}%
\pgfusepath{clip}%
\pgfsetroundcap%
\pgfsetroundjoin%
\pgfsetlinewidth{0.501875pt}%
\definecolor{currentstroke}{rgb}{0.278791,0.062145,0.386592}%
\pgfsetstrokecolor{currentstroke}%
\pgfsetdash{}{0pt}%
\pgfpathmoveto{\pgfqpoint{2.389275in}{2.552970in}}%
\pgfpathquadraticcurveto{\pgfqpoint{2.376069in}{2.552388in}}{\pgfqpoint{2.370619in}{2.552147in}}%
\pgfusepath{stroke}%
\end{pgfscope}%
\begin{pgfscope}%
\pgfpathrectangle{\pgfqpoint{0.800000in}{1.400000in}}{\pgfqpoint{2.407767in}{1.544118in}}%
\pgfusepath{clip}%
\pgfsetroundcap%
\pgfsetroundjoin%
\definecolor{currentfill}{rgb}{0.278791,0.062145,0.386592}%
\pgfsetfillcolor{currentfill}%
\pgfsetlinewidth{0.501875pt}%
\definecolor{currentstroke}{rgb}{0.278791,0.062145,0.386592}%
\pgfsetstrokecolor{currentstroke}%
\pgfsetdash{}{0pt}%
\pgfpathmoveto{\pgfqpoint{2.398982in}{2.539495in}}%
\pgfpathlineto{\pgfqpoint{2.370619in}{2.552147in}}%
\pgfpathlineto{\pgfqpoint{2.397758in}{2.567246in}}%
\pgfpathlineto{\pgfqpoint{2.398982in}{2.539495in}}%
\pgfpathlineto{\pgfqpoint{2.398982in}{2.539495in}}%
\pgfpathclose%
\pgfusepath{stroke,fill}%
\end{pgfscope}%
\begin{pgfscope}%
\pgfpathrectangle{\pgfqpoint{0.800000in}{1.400000in}}{\pgfqpoint{2.407767in}{1.544118in}}%
\pgfusepath{clip}%
\pgfsetroundcap%
\pgfsetroundjoin%
\pgfsetlinewidth{0.501875pt}%
\definecolor{currentstroke}{rgb}{0.277941,0.056324,0.381191}%
\pgfsetstrokecolor{currentstroke}%
\pgfsetdash{}{0pt}%
\pgfpathmoveto{\pgfqpoint{2.442258in}{2.585445in}}%
\pgfpathquadraticcurveto{\pgfqpoint{2.429030in}{2.585043in}}{\pgfqpoint{2.423563in}{2.584877in}}%
\pgfusepath{stroke}%
\end{pgfscope}%
\begin{pgfscope}%
\pgfpathrectangle{\pgfqpoint{0.800000in}{1.400000in}}{\pgfqpoint{2.407767in}{1.544118in}}%
\pgfusepath{clip}%
\pgfsetroundcap%
\pgfsetroundjoin%
\definecolor{currentfill}{rgb}{0.277941,0.056324,0.381191}%
\pgfsetfillcolor{currentfill}%
\pgfsetlinewidth{0.501875pt}%
\definecolor{currentstroke}{rgb}{0.277941,0.056324,0.381191}%
\pgfsetstrokecolor{currentstroke}%
\pgfsetdash{}{0pt}%
\pgfpathmoveto{\pgfqpoint{2.451749in}{2.571838in}}%
\pgfpathlineto{\pgfqpoint{2.423563in}{2.584877in}}%
\pgfpathlineto{\pgfqpoint{2.450906in}{2.599603in}}%
\pgfpathlineto{\pgfqpoint{2.451749in}{2.571838in}}%
\pgfpathlineto{\pgfqpoint{2.451749in}{2.571838in}}%
\pgfpathclose%
\pgfusepath{stroke,fill}%
\end{pgfscope}%
\begin{pgfscope}%
\pgfpathrectangle{\pgfqpoint{0.800000in}{1.400000in}}{\pgfqpoint{2.407767in}{1.544118in}}%
\pgfusepath{clip}%
\pgfsetroundcap%
\pgfsetroundjoin%
\pgfsetlinewidth{0.501875pt}%
\definecolor{currentstroke}{rgb}{0.280267,0.073417,0.397163}%
\pgfsetstrokecolor{currentstroke}%
\pgfsetdash{}{0pt}%
\pgfpathmoveto{\pgfqpoint{1.751918in}{1.816092in}}%
\pgfpathquadraticcurveto{\pgfqpoint{1.762138in}{1.820989in}}{\pgfqpoint{1.765357in}{1.822532in}}%
\pgfusepath{stroke}%
\end{pgfscope}%
\begin{pgfscope}%
\pgfpathrectangle{\pgfqpoint{0.800000in}{1.400000in}}{\pgfqpoint{2.407767in}{1.544118in}}%
\pgfusepath{clip}%
\pgfsetroundcap%
\pgfsetroundjoin%
\definecolor{currentfill}{rgb}{0.280267,0.073417,0.397163}%
\pgfsetfillcolor{currentfill}%
\pgfsetlinewidth{0.501875pt}%
\definecolor{currentstroke}{rgb}{0.280267,0.073417,0.397163}%
\pgfsetstrokecolor{currentstroke}%
\pgfsetdash{}{0pt}%
\pgfpathmoveto{\pgfqpoint{1.734305in}{1.823055in}}%
\pgfpathlineto{\pgfqpoint{1.765357in}{1.822532in}}%
\pgfpathlineto{\pgfqpoint{1.746308in}{1.798004in}}%
\pgfpathlineto{\pgfqpoint{1.734305in}{1.823055in}}%
\pgfpathlineto{\pgfqpoint{1.734305in}{1.823055in}}%
\pgfpathclose%
\pgfusepath{stroke,fill}%
\end{pgfscope}%
\begin{pgfscope}%
\pgfpathrectangle{\pgfqpoint{0.800000in}{1.400000in}}{\pgfqpoint{2.407767in}{1.544118in}}%
\pgfusepath{clip}%
\pgfsetroundcap%
\pgfsetroundjoin%
\pgfsetlinewidth{0.501875pt}%
\definecolor{currentstroke}{rgb}{0.279566,0.067836,0.391917}%
\pgfsetstrokecolor{currentstroke}%
\pgfsetdash{}{0pt}%
\pgfpathmoveto{\pgfqpoint{2.003883in}{1.824598in}}%
\pgfpathquadraticcurveto{\pgfqpoint{1.992637in}{1.829011in}}{\pgfqpoint{1.988618in}{1.830589in}}%
\pgfusepath{stroke}%
\end{pgfscope}%
\begin{pgfscope}%
\pgfpathrectangle{\pgfqpoint{0.800000in}{1.400000in}}{\pgfqpoint{2.407767in}{1.544118in}}%
\pgfusepath{clip}%
\pgfsetroundcap%
\pgfsetroundjoin%
\definecolor{currentfill}{rgb}{0.279566,0.067836,0.391917}%
\pgfsetfillcolor{currentfill}%
\pgfsetlinewidth{0.501875pt}%
\definecolor{currentstroke}{rgb}{0.279566,0.067836,0.391917}%
\pgfsetstrokecolor{currentstroke}%
\pgfsetdash{}{0pt}%
\pgfpathmoveto{\pgfqpoint{2.009402in}{1.807512in}}%
\pgfpathlineto{\pgfqpoint{1.988618in}{1.830589in}}%
\pgfpathlineto{\pgfqpoint{2.019550in}{1.833370in}}%
\pgfpathlineto{\pgfqpoint{2.009402in}{1.807512in}}%
\pgfpathlineto{\pgfqpoint{2.009402in}{1.807512in}}%
\pgfpathclose%
\pgfusepath{stroke,fill}%
\end{pgfscope}%
\begin{pgfscope}%
\pgfpathrectangle{\pgfqpoint{0.800000in}{1.400000in}}{\pgfqpoint{2.407767in}{1.544118in}}%
\pgfusepath{clip}%
\pgfsetroundcap%
\pgfsetroundjoin%
\pgfsetlinewidth{0.501875pt}%
\definecolor{currentstroke}{rgb}{0.279574,0.170599,0.479997}%
\pgfsetstrokecolor{currentstroke}%
\pgfsetdash{}{0pt}%
\pgfpathmoveto{\pgfqpoint{1.693195in}{1.929565in}}%
\pgfpathquadraticcurveto{\pgfqpoint{1.693471in}{1.933790in}}{\pgfqpoint{1.693241in}{1.930268in}}%
\pgfusepath{stroke}%
\end{pgfscope}%
\begin{pgfscope}%
\pgfpathrectangle{\pgfqpoint{0.800000in}{1.400000in}}{\pgfqpoint{2.407767in}{1.544118in}}%
\pgfusepath{clip}%
\pgfsetroundcap%
\pgfsetroundjoin%
\definecolor{currentfill}{rgb}{0.279574,0.170599,0.479997}%
\pgfsetfillcolor{currentfill}%
\pgfsetlinewidth{0.501875pt}%
\definecolor{currentstroke}{rgb}{0.279574,0.170599,0.479997}%
\pgfsetstrokecolor{currentstroke}%
\pgfsetdash{}{0pt}%
\pgfpathmoveto{\pgfqpoint{1.677573in}{1.903453in}}%
\pgfpathlineto{\pgfqpoint{1.693241in}{1.930268in}}%
\pgfpathlineto{\pgfqpoint{1.705292in}{1.901645in}}%
\pgfpathlineto{\pgfqpoint{1.677573in}{1.903453in}}%
\pgfpathlineto{\pgfqpoint{1.677573in}{1.903453in}}%
\pgfpathclose%
\pgfusepath{stroke,fill}%
\end{pgfscope}%
\begin{pgfscope}%
\pgfpathrectangle{\pgfqpoint{0.800000in}{1.400000in}}{\pgfqpoint{2.407767in}{1.544118in}}%
\pgfusepath{clip}%
\pgfsetroundcap%
\pgfsetroundjoin%
\pgfsetlinewidth{0.501875pt}%
\definecolor{currentstroke}{rgb}{0.258965,0.251537,0.524736}%
\pgfsetstrokecolor{currentstroke}%
\pgfsetdash{}{0pt}%
\pgfpathmoveto{\pgfqpoint{1.624622in}{2.311043in}}%
\pgfpathquadraticcurveto{\pgfqpoint{1.624363in}{2.306114in}}{\pgfqpoint{1.624511in}{2.308939in}}%
\pgfusepath{stroke}%
\end{pgfscope}%
\begin{pgfscope}%
\pgfpathrectangle{\pgfqpoint{0.800000in}{1.400000in}}{\pgfqpoint{2.407767in}{1.544118in}}%
\pgfusepath{clip}%
\pgfsetroundcap%
\pgfsetroundjoin%
\definecolor{currentfill}{rgb}{0.258965,0.251537,0.524736}%
\pgfsetfillcolor{currentfill}%
\pgfsetlinewidth{0.501875pt}%
\definecolor{currentstroke}{rgb}{0.258965,0.251537,0.524736}%
\pgfsetstrokecolor{currentstroke}%
\pgfsetdash{}{0pt}%
\pgfpathmoveto{\pgfqpoint{1.639842in}{2.335948in}}%
\pgfpathlineto{\pgfqpoint{1.624511in}{2.308939in}}%
\pgfpathlineto{\pgfqpoint{1.612103in}{2.337408in}}%
\pgfpathlineto{\pgfqpoint{1.639842in}{2.335948in}}%
\pgfpathlineto{\pgfqpoint{1.639842in}{2.335948in}}%
\pgfpathclose%
\pgfusepath{stroke,fill}%
\end{pgfscope}%
\begin{pgfscope}%
\pgfpathrectangle{\pgfqpoint{0.800000in}{1.400000in}}{\pgfqpoint{2.407767in}{1.544118in}}%
\pgfusepath{clip}%
\pgfsetbuttcap%
\pgfsetroundjoin%
\pgfsetlinewidth{1.505625pt}%
\definecolor{currentstroke}{rgb}{0.000000,0.000000,0.000000}%
\pgfsetstrokecolor{currentstroke}%
\pgfsetdash{}{0pt}%
\pgfpathmoveto{\pgfqpoint{1.789710in}{1.661474in}}%
\pgfpathlineto{\pgfqpoint{1.789710in}{2.682644in}}%
\pgfusepath{stroke}%
\end{pgfscope}%
\begin{pgfscope}%
\pgfpathrectangle{\pgfqpoint{0.800000in}{1.400000in}}{\pgfqpoint{2.407767in}{1.544118in}}%
\pgfusepath{clip}%
\pgfsetbuttcap%
\pgfsetroundjoin%
\pgfsetlinewidth{1.505625pt}%
\definecolor{currentstroke}{rgb}{0.000000,0.000000,0.000000}%
\pgfsetstrokecolor{currentstroke}%
\pgfsetdash{}{0pt}%
\pgfpathmoveto{\pgfqpoint{2.699611in}{1.661474in}}%
\pgfpathlineto{\pgfqpoint{2.699611in}{2.682644in}}%
\pgfusepath{stroke}%
\end{pgfscope}%
\begin{pgfscope}%
\pgfsetrectcap%
\pgfsetmiterjoin%
\pgfsetlinewidth{0.803000pt}%
\definecolor{currentstroke}{rgb}{0.000000,0.000000,0.000000}%
\pgfsetstrokecolor{currentstroke}%
\pgfsetdash{}{0pt}%
\pgfpathmoveto{\pgfqpoint{0.800000in}{1.400000in}}%
\pgfpathlineto{\pgfqpoint{0.800000in}{2.944118in}}%
\pgfusepath{stroke}%
\end{pgfscope}%
\begin{pgfscope}%
\pgfsetrectcap%
\pgfsetmiterjoin%
\pgfsetlinewidth{0.803000pt}%
\definecolor{currentstroke}{rgb}{0.000000,0.000000,0.000000}%
\pgfsetstrokecolor{currentstroke}%
\pgfsetdash{}{0pt}%
\pgfpathmoveto{\pgfqpoint{3.207767in}{1.400000in}}%
\pgfpathlineto{\pgfqpoint{3.207767in}{2.944118in}}%
\pgfusepath{stroke}%
\end{pgfscope}%
\begin{pgfscope}%
\pgfsetrectcap%
\pgfsetmiterjoin%
\pgfsetlinewidth{0.803000pt}%
\definecolor{currentstroke}{rgb}{0.000000,0.000000,0.000000}%
\pgfsetstrokecolor{currentstroke}%
\pgfsetdash{}{0pt}%
\pgfpathmoveto{\pgfqpoint{0.800000in}{1.400000in}}%
\pgfpathlineto{\pgfqpoint{3.207767in}{1.400000in}}%
\pgfusepath{stroke}%
\end{pgfscope}%
\begin{pgfscope}%
\pgfsetrectcap%
\pgfsetmiterjoin%
\pgfsetlinewidth{0.803000pt}%
\definecolor{currentstroke}{rgb}{0.000000,0.000000,0.000000}%
\pgfsetstrokecolor{currentstroke}%
\pgfsetdash{}{0pt}%
\pgfpathmoveto{\pgfqpoint{0.800000in}{2.944118in}}%
\pgfpathlineto{\pgfqpoint{3.207767in}{2.944118in}}%
\pgfusepath{stroke}%
\end{pgfscope}%
\begin{pgfscope}%
\definecolor{textcolor}{rgb}{0.000000,0.000000,0.000000}%
\pgfsetstrokecolor{textcolor}%
\pgfsetfillcolor{textcolor}%
\pgftext[x=2.003883in,y=3.027451in,,base]{\color{textcolor}\sffamily\fontsize{12.000000}{14.400000}\selectfont e)}%
\end{pgfscope}%
\begin{pgfscope}%
\pgfsetbuttcap%
\pgfsetmiterjoin%
\definecolor{currentfill}{rgb}{1.000000,1.000000,1.000000}%
\pgfsetfillcolor{currentfill}%
\pgfsetlinewidth{0.000000pt}%
\definecolor{currentstroke}{rgb}{0.000000,0.000000,0.000000}%
\pgfsetstrokecolor{currentstroke}%
\pgfsetstrokeopacity{0.000000}%
\pgfsetdash{}{0pt}%
\pgfpathmoveto{\pgfqpoint{3.352233in}{1.400000in}}%
\pgfpathlineto{\pgfqpoint{5.760000in}{1.400000in}}%
\pgfpathlineto{\pgfqpoint{5.760000in}{2.944118in}}%
\pgfpathlineto{\pgfqpoint{3.352233in}{2.944118in}}%
\pgfpathlineto{\pgfqpoint{3.352233in}{1.400000in}}%
\pgfpathclose%
\pgfusepath{fill}%
\end{pgfscope}%
\begin{pgfscope}%
\pgfpathrectangle{\pgfqpoint{3.352233in}{1.400000in}}{\pgfqpoint{2.407767in}{1.544118in}}%
\pgfusepath{clip}%
\pgfsys@transformcm{2.416667}{0.000000}{0.000000}{1.555556}{3.352233in}{1.400000in}%
\pgftext[left,bottom]{\includegraphics[interpolate=false,width=1.000000in,height=1.000000in]{q_series_square-img5.png}}%
\end{pgfscope}%
\begin{pgfscope}%
\pgfsetbuttcap%
\pgfsetroundjoin%
\definecolor{currentfill}{rgb}{0.000000,0.000000,0.000000}%
\pgfsetfillcolor{currentfill}%
\pgfsetlinewidth{0.803000pt}%
\definecolor{currentstroke}{rgb}{0.000000,0.000000,0.000000}%
\pgfsetstrokecolor{currentstroke}%
\pgfsetdash{}{0pt}%
\pgfsys@defobject{currentmarker}{\pgfqpoint{0.000000in}{-0.048611in}}{\pgfqpoint{0.000000in}{0.000000in}}{%
\pgfpathmoveto{\pgfqpoint{0.000000in}{0.000000in}}%
\pgfpathlineto{\pgfqpoint{0.000000in}{-0.048611in}}%
\pgfusepath{stroke,fill}%
}%
\begin{pgfscope}%
\pgfsys@transformshift{3.785892in}{1.400000in}%
\pgfsys@useobject{currentmarker}{}%
\end{pgfscope}%
\end{pgfscope}%
\begin{pgfscope}%
\definecolor{textcolor}{rgb}{0.000000,0.000000,0.000000}%
\pgfsetstrokecolor{textcolor}%
\pgfsetfillcolor{textcolor}%
\pgftext[x=3.785892in,y=1.302778in,,top]{\color{textcolor}\sffamily\fontsize{10.000000}{12.000000}\selectfont \(\displaystyle {\ensuremath{-}10}\)}%
\end{pgfscope}%
\begin{pgfscope}%
\pgfsetbuttcap%
\pgfsetroundjoin%
\definecolor{currentfill}{rgb}{0.000000,0.000000,0.000000}%
\pgfsetfillcolor{currentfill}%
\pgfsetlinewidth{0.803000pt}%
\definecolor{currentstroke}{rgb}{0.000000,0.000000,0.000000}%
\pgfsetstrokecolor{currentstroke}%
\pgfsetdash{}{0pt}%
\pgfsys@defobject{currentmarker}{\pgfqpoint{0.000000in}{-0.048611in}}{\pgfqpoint{0.000000in}{0.000000in}}{%
\pgfpathmoveto{\pgfqpoint{0.000000in}{0.000000in}}%
\pgfpathlineto{\pgfqpoint{0.000000in}{-0.048611in}}%
\pgfusepath{stroke,fill}%
}%
\begin{pgfscope}%
\pgfsys@transformshift{4.291393in}{1.400000in}%
\pgfsys@useobject{currentmarker}{}%
\end{pgfscope}%
\end{pgfscope}%
\begin{pgfscope}%
\definecolor{textcolor}{rgb}{0.000000,0.000000,0.000000}%
\pgfsetstrokecolor{textcolor}%
\pgfsetfillcolor{textcolor}%
\pgftext[x=4.291393in,y=1.302778in,,top]{\color{textcolor}\sffamily\fontsize{10.000000}{12.000000}\selectfont \(\displaystyle {\ensuremath{-}5}\)}%
\end{pgfscope}%
\begin{pgfscope}%
\pgfsetbuttcap%
\pgfsetroundjoin%
\definecolor{currentfill}{rgb}{0.000000,0.000000,0.000000}%
\pgfsetfillcolor{currentfill}%
\pgfsetlinewidth{0.803000pt}%
\definecolor{currentstroke}{rgb}{0.000000,0.000000,0.000000}%
\pgfsetstrokecolor{currentstroke}%
\pgfsetdash{}{0pt}%
\pgfsys@defobject{currentmarker}{\pgfqpoint{0.000000in}{-0.048611in}}{\pgfqpoint{0.000000in}{0.000000in}}{%
\pgfpathmoveto{\pgfqpoint{0.000000in}{0.000000in}}%
\pgfpathlineto{\pgfqpoint{0.000000in}{-0.048611in}}%
\pgfusepath{stroke,fill}%
}%
\begin{pgfscope}%
\pgfsys@transformshift{4.796893in}{1.400000in}%
\pgfsys@useobject{currentmarker}{}%
\end{pgfscope}%
\end{pgfscope}%
\begin{pgfscope}%
\definecolor{textcolor}{rgb}{0.000000,0.000000,0.000000}%
\pgfsetstrokecolor{textcolor}%
\pgfsetfillcolor{textcolor}%
\pgftext[x=4.796893in,y=1.302778in,,top]{\color{textcolor}\sffamily\fontsize{10.000000}{12.000000}\selectfont \(\displaystyle {0}\)}%
\end{pgfscope}%
\begin{pgfscope}%
\pgfsetbuttcap%
\pgfsetroundjoin%
\definecolor{currentfill}{rgb}{0.000000,0.000000,0.000000}%
\pgfsetfillcolor{currentfill}%
\pgfsetlinewidth{0.803000pt}%
\definecolor{currentstroke}{rgb}{0.000000,0.000000,0.000000}%
\pgfsetstrokecolor{currentstroke}%
\pgfsetdash{}{0pt}%
\pgfsys@defobject{currentmarker}{\pgfqpoint{0.000000in}{-0.048611in}}{\pgfqpoint{0.000000in}{0.000000in}}{%
\pgfpathmoveto{\pgfqpoint{0.000000in}{0.000000in}}%
\pgfpathlineto{\pgfqpoint{0.000000in}{-0.048611in}}%
\pgfusepath{stroke,fill}%
}%
\begin{pgfscope}%
\pgfsys@transformshift{5.302394in}{1.400000in}%
\pgfsys@useobject{currentmarker}{}%
\end{pgfscope}%
\end{pgfscope}%
\begin{pgfscope}%
\definecolor{textcolor}{rgb}{0.000000,0.000000,0.000000}%
\pgfsetstrokecolor{textcolor}%
\pgfsetfillcolor{textcolor}%
\pgftext[x=5.302394in,y=1.302778in,,top]{\color{textcolor}\sffamily\fontsize{10.000000}{12.000000}\selectfont \(\displaystyle {5}\)}%
\end{pgfscope}%
\begin{pgfscope}%
\definecolor{textcolor}{rgb}{0.000000,0.000000,0.000000}%
\pgfsetstrokecolor{textcolor}%
\pgfsetfillcolor{textcolor}%
\pgftext[x=4.556117in,y=1.123766in,,top]{\color{textcolor}\sffamily\fontsize{10.000000}{12.000000}\selectfont \(\displaystyle \zeta \, \mathrm{[\mu m]}\)}%
\end{pgfscope}%
\begin{pgfscope}%
\pgfsetbuttcap%
\pgfsetroundjoin%
\definecolor{currentfill}{rgb}{0.000000,0.000000,0.000000}%
\pgfsetfillcolor{currentfill}%
\pgfsetlinewidth{0.803000pt}%
\definecolor{currentstroke}{rgb}{0.000000,0.000000,0.000000}%
\pgfsetstrokecolor{currentstroke}%
\pgfsetdash{}{0pt}%
\pgfsys@defobject{currentmarker}{\pgfqpoint{-0.048611in}{0.000000in}}{\pgfqpoint{-0.000000in}{0.000000in}}{%
\pgfpathmoveto{\pgfqpoint{-0.000000in}{0.000000in}}%
\pgfpathlineto{\pgfqpoint{-0.048611in}{0.000000in}}%
\pgfusepath{stroke,fill}%
}%
\begin{pgfscope}%
\pgfsys@transformshift{3.352233in}{1.661474in}%
\pgfsys@useobject{currentmarker}{}%
\end{pgfscope}%
\end{pgfscope}%
\begin{pgfscope}%
\pgfsetbuttcap%
\pgfsetroundjoin%
\definecolor{currentfill}{rgb}{0.000000,0.000000,0.000000}%
\pgfsetfillcolor{currentfill}%
\pgfsetlinewidth{0.803000pt}%
\definecolor{currentstroke}{rgb}{0.000000,0.000000,0.000000}%
\pgfsetstrokecolor{currentstroke}%
\pgfsetdash{}{0pt}%
\pgfsys@defobject{currentmarker}{\pgfqpoint{-0.048611in}{0.000000in}}{\pgfqpoint{-0.000000in}{0.000000in}}{%
\pgfpathmoveto{\pgfqpoint{-0.000000in}{0.000000in}}%
\pgfpathlineto{\pgfqpoint{-0.048611in}{0.000000in}}%
\pgfusepath{stroke,fill}%
}%
\begin{pgfscope}%
\pgfsys@transformshift{3.352233in}{2.172059in}%
\pgfsys@useobject{currentmarker}{}%
\end{pgfscope}%
\end{pgfscope}%
\begin{pgfscope}%
\pgfsetbuttcap%
\pgfsetroundjoin%
\definecolor{currentfill}{rgb}{0.000000,0.000000,0.000000}%
\pgfsetfillcolor{currentfill}%
\pgfsetlinewidth{0.803000pt}%
\definecolor{currentstroke}{rgb}{0.000000,0.000000,0.000000}%
\pgfsetstrokecolor{currentstroke}%
\pgfsetdash{}{0pt}%
\pgfsys@defobject{currentmarker}{\pgfqpoint{-0.048611in}{0.000000in}}{\pgfqpoint{-0.000000in}{0.000000in}}{%
\pgfpathmoveto{\pgfqpoint{-0.000000in}{0.000000in}}%
\pgfpathlineto{\pgfqpoint{-0.048611in}{0.000000in}}%
\pgfusepath{stroke,fill}%
}%
\begin{pgfscope}%
\pgfsys@transformshift{3.352233in}{2.682644in}%
\pgfsys@useobject{currentmarker}{}%
\end{pgfscope}%
\end{pgfscope}%
\begin{pgfscope}%
\pgfpathrectangle{\pgfqpoint{3.352233in}{1.400000in}}{\pgfqpoint{2.407767in}{1.544118in}}%
\pgfusepath{clip}%
\pgfsetbuttcap%
\pgfsetroundjoin%
\pgfsetlinewidth{0.501875pt}%
\definecolor{currentstroke}{rgb}{0.268510,0.009605,0.335427}%
\pgfsetstrokecolor{currentstroke}%
\pgfsetdash{}{0pt}%
\pgfpathmoveto{\pgfqpoint{4.339396in}{1.442390in}}%
\pgfpathlineto{\pgfqpoint{4.339396in}{1.442390in}}%
\pgfusepath{stroke}%
\end{pgfscope}%
\begin{pgfscope}%
\pgfpathrectangle{\pgfqpoint{3.352233in}{1.400000in}}{\pgfqpoint{2.407767in}{1.544118in}}%
\pgfusepath{clip}%
\pgfsetbuttcap%
\pgfsetroundjoin%
\pgfsetlinewidth{0.501875pt}%
\definecolor{currentstroke}{rgb}{0.268510,0.009605,0.335427}%
\pgfsetstrokecolor{currentstroke}%
\pgfsetdash{}{0pt}%
\pgfpathmoveto{\pgfqpoint{4.339396in}{1.442390in}}%
\pgfpathlineto{\pgfqpoint{4.339396in}{1.442390in}}%
\pgfusepath{stroke}%
\end{pgfscope}%
\begin{pgfscope}%
\pgfpathrectangle{\pgfqpoint{3.352233in}{1.400000in}}{\pgfqpoint{2.407767in}{1.544118in}}%
\pgfusepath{clip}%
\pgfsetbuttcap%
\pgfsetroundjoin%
\pgfsetlinewidth{0.501875pt}%
\definecolor{currentstroke}{rgb}{0.268510,0.009605,0.335427}%
\pgfsetstrokecolor{currentstroke}%
\pgfsetdash{}{0pt}%
\pgfpathmoveto{\pgfqpoint{4.339396in}{1.442390in}}%
\pgfpathlineto{\pgfqpoint{4.353635in}{1.445254in}}%
\pgfusepath{stroke}%
\end{pgfscope}%
\begin{pgfscope}%
\pgfpathrectangle{\pgfqpoint{3.352233in}{1.400000in}}{\pgfqpoint{2.407767in}{1.544118in}}%
\pgfusepath{clip}%
\pgfsetbuttcap%
\pgfsetroundjoin%
\pgfsetlinewidth{0.501875pt}%
\definecolor{currentstroke}{rgb}{0.269944,0.014625,0.341379}%
\pgfsetstrokecolor{currentstroke}%
\pgfsetdash{}{0pt}%
\pgfpathmoveto{\pgfqpoint{4.353635in}{1.445254in}}%
\pgfpathlineto{\pgfqpoint{4.353635in}{1.445254in}}%
\pgfusepath{stroke}%
\end{pgfscope}%
\begin{pgfscope}%
\pgfpathrectangle{\pgfqpoint{3.352233in}{1.400000in}}{\pgfqpoint{2.407767in}{1.544118in}}%
\pgfusepath{clip}%
\pgfsetbuttcap%
\pgfsetroundjoin%
\pgfsetlinewidth{0.501875pt}%
\definecolor{currentstroke}{rgb}{0.269944,0.014625,0.341379}%
\pgfsetstrokecolor{currentstroke}%
\pgfsetdash{}{0pt}%
\pgfpathmoveto{\pgfqpoint{4.353635in}{1.445254in}}%
\pgfpathlineto{\pgfqpoint{4.357772in}{1.448087in}}%
\pgfusepath{stroke}%
\end{pgfscope}%
\begin{pgfscope}%
\pgfpathrectangle{\pgfqpoint{3.352233in}{1.400000in}}{\pgfqpoint{2.407767in}{1.544118in}}%
\pgfusepath{clip}%
\pgfsetbuttcap%
\pgfsetroundjoin%
\pgfsetlinewidth{0.501875pt}%
\definecolor{currentstroke}{rgb}{0.269944,0.014625,0.341379}%
\pgfsetstrokecolor{currentstroke}%
\pgfsetdash{}{0pt}%
\pgfpathmoveto{\pgfqpoint{4.357772in}{1.448087in}}%
\pgfpathlineto{\pgfqpoint{4.358466in}{1.451053in}}%
\pgfusepath{stroke}%
\end{pgfscope}%
\begin{pgfscope}%
\pgfpathrectangle{\pgfqpoint{3.352233in}{1.400000in}}{\pgfqpoint{2.407767in}{1.544118in}}%
\pgfusepath{clip}%
\pgfsetbuttcap%
\pgfsetroundjoin%
\pgfsetlinewidth{0.501875pt}%
\definecolor{currentstroke}{rgb}{0.269944,0.014625,0.341379}%
\pgfsetstrokecolor{currentstroke}%
\pgfsetdash{}{0pt}%
\pgfpathmoveto{\pgfqpoint{4.358466in}{1.451053in}}%
\pgfpathlineto{\pgfqpoint{4.358216in}{1.454349in}}%
\pgfusepath{stroke}%
\end{pgfscope}%
\begin{pgfscope}%
\pgfpathrectangle{\pgfqpoint{3.352233in}{1.400000in}}{\pgfqpoint{2.407767in}{1.544118in}}%
\pgfusepath{clip}%
\pgfsetbuttcap%
\pgfsetroundjoin%
\pgfsetlinewidth{0.501875pt}%
\definecolor{currentstroke}{rgb}{0.269944,0.014625,0.341379}%
\pgfsetstrokecolor{currentstroke}%
\pgfsetdash{}{0pt}%
\pgfpathmoveto{\pgfqpoint{4.358216in}{1.454349in}}%
\pgfpathlineto{\pgfqpoint{4.357992in}{1.458149in}}%
\pgfusepath{stroke}%
\end{pgfscope}%
\begin{pgfscope}%
\pgfpathrectangle{\pgfqpoint{3.352233in}{1.400000in}}{\pgfqpoint{2.407767in}{1.544118in}}%
\pgfusepath{clip}%
\pgfsetbuttcap%
\pgfsetroundjoin%
\pgfsetlinewidth{0.501875pt}%
\definecolor{currentstroke}{rgb}{0.269944,0.014625,0.341379}%
\pgfsetstrokecolor{currentstroke}%
\pgfsetdash{}{0pt}%
\pgfpathmoveto{\pgfqpoint{4.357992in}{1.458149in}}%
\pgfpathlineto{\pgfqpoint{4.355853in}{1.462444in}}%
\pgfusepath{stroke}%
\end{pgfscope}%
\begin{pgfscope}%
\pgfpathrectangle{\pgfqpoint{3.352233in}{1.400000in}}{\pgfqpoint{2.407767in}{1.544118in}}%
\pgfusepath{clip}%
\pgfsetbuttcap%
\pgfsetroundjoin%
\pgfsetlinewidth{0.501875pt}%
\definecolor{currentstroke}{rgb}{0.269944,0.014625,0.341379}%
\pgfsetstrokecolor{currentstroke}%
\pgfsetdash{}{0pt}%
\pgfpathmoveto{\pgfqpoint{4.355853in}{1.462444in}}%
\pgfpathlineto{\pgfqpoint{4.355853in}{1.462444in}}%
\pgfusepath{stroke}%
\end{pgfscope}%
\begin{pgfscope}%
\pgfpathrectangle{\pgfqpoint{3.352233in}{1.400000in}}{\pgfqpoint{2.407767in}{1.544118in}}%
\pgfusepath{clip}%
\pgfsetbuttcap%
\pgfsetroundjoin%
\pgfsetlinewidth{0.501875pt}%
\definecolor{currentstroke}{rgb}{0.269944,0.014625,0.341379}%
\pgfsetstrokecolor{currentstroke}%
\pgfsetdash{}{0pt}%
\pgfpathmoveto{\pgfqpoint{4.355853in}{1.462444in}}%
\pgfpathlineto{\pgfqpoint{4.357101in}{1.464391in}}%
\pgfusepath{stroke}%
\end{pgfscope}%
\begin{pgfscope}%
\pgfpathrectangle{\pgfqpoint{3.352233in}{1.400000in}}{\pgfqpoint{2.407767in}{1.544118in}}%
\pgfusepath{clip}%
\pgfsetbuttcap%
\pgfsetroundjoin%
\pgfsetlinewidth{0.501875pt}%
\definecolor{currentstroke}{rgb}{0.269944,0.014625,0.341379}%
\pgfsetstrokecolor{currentstroke}%
\pgfsetdash{}{0pt}%
\pgfpathmoveto{\pgfqpoint{4.357101in}{1.464391in}}%
\pgfpathlineto{\pgfqpoint{4.357492in}{1.466303in}}%
\pgfusepath{stroke}%
\end{pgfscope}%
\begin{pgfscope}%
\pgfpathrectangle{\pgfqpoint{3.352233in}{1.400000in}}{\pgfqpoint{2.407767in}{1.544118in}}%
\pgfusepath{clip}%
\pgfsetbuttcap%
\pgfsetroundjoin%
\pgfsetlinewidth{0.501875pt}%
\definecolor{currentstroke}{rgb}{0.268510,0.009605,0.335427}%
\pgfsetstrokecolor{currentstroke}%
\pgfsetdash{}{0pt}%
\pgfpathmoveto{\pgfqpoint{4.357492in}{1.466303in}}%
\pgfpathlineto{\pgfqpoint{4.357156in}{1.468185in}}%
\pgfusepath{stroke}%
\end{pgfscope}%
\begin{pgfscope}%
\pgfpathrectangle{\pgfqpoint{3.352233in}{1.400000in}}{\pgfqpoint{2.407767in}{1.544118in}}%
\pgfusepath{clip}%
\pgfsetbuttcap%
\pgfsetroundjoin%
\pgfsetlinewidth{0.501875pt}%
\definecolor{currentstroke}{rgb}{0.268510,0.009605,0.335427}%
\pgfsetstrokecolor{currentstroke}%
\pgfsetdash{}{0pt}%
\pgfpathmoveto{\pgfqpoint{4.357156in}{1.468185in}}%
\pgfpathlineto{\pgfqpoint{4.356713in}{1.469841in}}%
\pgfusepath{stroke}%
\end{pgfscope}%
\begin{pgfscope}%
\pgfpathrectangle{\pgfqpoint{3.352233in}{1.400000in}}{\pgfqpoint{2.407767in}{1.544118in}}%
\pgfusepath{clip}%
\pgfsetbuttcap%
\pgfsetroundjoin%
\pgfsetlinewidth{0.501875pt}%
\definecolor{currentstroke}{rgb}{0.268510,0.009605,0.335427}%
\pgfsetstrokecolor{currentstroke}%
\pgfsetdash{}{0pt}%
\pgfpathmoveto{\pgfqpoint{4.356713in}{1.469841in}}%
\pgfpathlineto{\pgfqpoint{4.357584in}{1.471433in}}%
\pgfusepath{stroke}%
\end{pgfscope}%
\begin{pgfscope}%
\pgfpathrectangle{\pgfqpoint{3.352233in}{1.400000in}}{\pgfqpoint{2.407767in}{1.544118in}}%
\pgfusepath{clip}%
\pgfsetbuttcap%
\pgfsetroundjoin%
\pgfsetlinewidth{0.501875pt}%
\definecolor{currentstroke}{rgb}{0.268510,0.009605,0.335427}%
\pgfsetstrokecolor{currentstroke}%
\pgfsetdash{}{0pt}%
\pgfpathmoveto{\pgfqpoint{4.357584in}{1.471433in}}%
\pgfpathlineto{\pgfqpoint{4.359748in}{1.473141in}}%
\pgfusepath{stroke}%
\end{pgfscope}%
\begin{pgfscope}%
\pgfpathrectangle{\pgfqpoint{3.352233in}{1.400000in}}{\pgfqpoint{2.407767in}{1.544118in}}%
\pgfusepath{clip}%
\pgfsetbuttcap%
\pgfsetroundjoin%
\pgfsetlinewidth{0.501875pt}%
\definecolor{currentstroke}{rgb}{0.268510,0.009605,0.335427}%
\pgfsetstrokecolor{currentstroke}%
\pgfsetdash{}{0pt}%
\pgfpathmoveto{\pgfqpoint{4.359748in}{1.473141in}}%
\pgfpathlineto{\pgfqpoint{4.363264in}{1.475418in}}%
\pgfusepath{stroke}%
\end{pgfscope}%
\begin{pgfscope}%
\pgfpathrectangle{\pgfqpoint{3.352233in}{1.400000in}}{\pgfqpoint{2.407767in}{1.544118in}}%
\pgfusepath{clip}%
\pgfsetbuttcap%
\pgfsetroundjoin%
\pgfsetlinewidth{0.501875pt}%
\definecolor{currentstroke}{rgb}{0.268510,0.009605,0.335427}%
\pgfsetstrokecolor{currentstroke}%
\pgfsetdash{}{0pt}%
\pgfpathmoveto{\pgfqpoint{4.363264in}{1.475418in}}%
\pgfpathlineto{\pgfqpoint{4.373610in}{1.478846in}}%
\pgfusepath{stroke}%
\end{pgfscope}%
\begin{pgfscope}%
\pgfpathrectangle{\pgfqpoint{3.352233in}{1.400000in}}{\pgfqpoint{2.407767in}{1.544118in}}%
\pgfusepath{clip}%
\pgfsetbuttcap%
\pgfsetroundjoin%
\pgfsetlinewidth{0.501875pt}%
\definecolor{currentstroke}{rgb}{0.268510,0.009605,0.335427}%
\pgfsetstrokecolor{currentstroke}%
\pgfsetdash{}{0pt}%
\pgfpathmoveto{\pgfqpoint{5.198345in}{1.438545in}}%
\pgfpathlineto{\pgfqpoint{5.145409in}{1.439348in}}%
\pgfusepath{stroke}%
\end{pgfscope}%
\begin{pgfscope}%
\pgfpathrectangle{\pgfqpoint{3.352233in}{1.400000in}}{\pgfqpoint{2.407767in}{1.544118in}}%
\pgfusepath{clip}%
\pgfsetbuttcap%
\pgfsetroundjoin%
\pgfsetlinewidth{0.501875pt}%
\definecolor{currentstroke}{rgb}{0.271305,0.019942,0.347269}%
\pgfsetstrokecolor{currentstroke}%
\pgfsetdash{}{0pt}%
\pgfpathmoveto{\pgfqpoint{5.145409in}{1.439348in}}%
\pgfpathlineto{\pgfqpoint{5.092441in}{1.439405in}}%
\pgfusepath{stroke}%
\end{pgfscope}%
\begin{pgfscope}%
\pgfpathrectangle{\pgfqpoint{3.352233in}{1.400000in}}{\pgfqpoint{2.407767in}{1.544118in}}%
\pgfusepath{clip}%
\pgfsetbuttcap%
\pgfsetroundjoin%
\pgfsetlinewidth{0.501875pt}%
\definecolor{currentstroke}{rgb}{0.271305,0.019942,0.347269}%
\pgfsetstrokecolor{currentstroke}%
\pgfsetdash{}{0pt}%
\pgfpathmoveto{\pgfqpoint{5.092441in}{1.439405in}}%
\pgfpathlineto{\pgfqpoint{5.039499in}{1.440102in}}%
\pgfusepath{stroke}%
\end{pgfscope}%
\begin{pgfscope}%
\pgfpathrectangle{\pgfqpoint{3.352233in}{1.400000in}}{\pgfqpoint{2.407767in}{1.544118in}}%
\pgfusepath{clip}%
\pgfsetbuttcap%
\pgfsetroundjoin%
\pgfsetlinewidth{0.501875pt}%
\definecolor{currentstroke}{rgb}{0.272594,0.025563,0.353093}%
\pgfsetstrokecolor{currentstroke}%
\pgfsetdash{}{0pt}%
\pgfpathmoveto{\pgfqpoint{5.039499in}{1.440102in}}%
\pgfpathlineto{\pgfqpoint{4.986715in}{1.439658in}}%
\pgfusepath{stroke}%
\end{pgfscope}%
\begin{pgfscope}%
\pgfpathrectangle{\pgfqpoint{3.352233in}{1.400000in}}{\pgfqpoint{2.407767in}{1.544118in}}%
\pgfusepath{clip}%
\pgfsetbuttcap%
\pgfsetroundjoin%
\pgfsetlinewidth{0.501875pt}%
\definecolor{currentstroke}{rgb}{0.269944,0.014625,0.341379}%
\pgfsetstrokecolor{currentstroke}%
\pgfsetdash{}{0pt}%
\pgfpathmoveto{\pgfqpoint{4.986715in}{1.439658in}}%
\pgfpathlineto{\pgfqpoint{4.934017in}{1.439816in}}%
\pgfusepath{stroke}%
\end{pgfscope}%
\begin{pgfscope}%
\pgfpathrectangle{\pgfqpoint{3.352233in}{1.400000in}}{\pgfqpoint{2.407767in}{1.544118in}}%
\pgfusepath{clip}%
\pgfsetbuttcap%
\pgfsetroundjoin%
\pgfsetlinewidth{0.501875pt}%
\definecolor{currentstroke}{rgb}{0.272594,0.025563,0.353093}%
\pgfsetstrokecolor{currentstroke}%
\pgfsetdash{}{0pt}%
\pgfpathmoveto{\pgfqpoint{4.934017in}{1.439816in}}%
\pgfpathlineto{\pgfqpoint{4.881198in}{1.442390in}}%
\pgfusepath{stroke}%
\end{pgfscope}%
\begin{pgfscope}%
\pgfpathrectangle{\pgfqpoint{3.352233in}{1.400000in}}{\pgfqpoint{2.407767in}{1.544118in}}%
\pgfusepath{clip}%
\pgfsetbuttcap%
\pgfsetroundjoin%
\pgfsetlinewidth{0.501875pt}%
\definecolor{currentstroke}{rgb}{0.268510,0.009605,0.335427}%
\pgfsetstrokecolor{currentstroke}%
\pgfsetdash{}{0pt}%
\pgfpathmoveto{\pgfqpoint{5.195642in}{1.464460in}}%
\pgfpathlineto{\pgfqpoint{5.142712in}{1.464471in}}%
\pgfusepath{stroke}%
\end{pgfscope}%
\begin{pgfscope}%
\pgfpathrectangle{\pgfqpoint{3.352233in}{1.400000in}}{\pgfqpoint{2.407767in}{1.544118in}}%
\pgfusepath{clip}%
\pgfsetbuttcap%
\pgfsetroundjoin%
\pgfsetlinewidth{0.501875pt}%
\definecolor{currentstroke}{rgb}{0.269944,0.014625,0.341379}%
\pgfsetstrokecolor{currentstroke}%
\pgfsetdash{}{0pt}%
\pgfpathmoveto{\pgfqpoint{5.142712in}{1.464471in}}%
\pgfpathlineto{\pgfqpoint{5.089782in}{1.465042in}}%
\pgfusepath{stroke}%
\end{pgfscope}%
\begin{pgfscope}%
\pgfpathrectangle{\pgfqpoint{3.352233in}{1.400000in}}{\pgfqpoint{2.407767in}{1.544118in}}%
\pgfusepath{clip}%
\pgfsetbuttcap%
\pgfsetroundjoin%
\pgfsetlinewidth{0.501875pt}%
\definecolor{currentstroke}{rgb}{0.269944,0.014625,0.341379}%
\pgfsetstrokecolor{currentstroke}%
\pgfsetdash{}{0pt}%
\pgfpathmoveto{\pgfqpoint{5.089782in}{1.465042in}}%
\pgfpathlineto{\pgfqpoint{5.036842in}{1.466169in}}%
\pgfusepath{stroke}%
\end{pgfscope}%
\begin{pgfscope}%
\pgfpathrectangle{\pgfqpoint{3.352233in}{1.400000in}}{\pgfqpoint{2.407767in}{1.544118in}}%
\pgfusepath{clip}%
\pgfsetbuttcap%
\pgfsetroundjoin%
\pgfsetlinewidth{0.501875pt}%
\definecolor{currentstroke}{rgb}{0.272594,0.025563,0.353093}%
\pgfsetstrokecolor{currentstroke}%
\pgfsetdash{}{0pt}%
\pgfpathmoveto{\pgfqpoint{5.036842in}{1.466169in}}%
\pgfpathlineto{\pgfqpoint{4.983959in}{1.467935in}}%
\pgfusepath{stroke}%
\end{pgfscope}%
\begin{pgfscope}%
\pgfpathrectangle{\pgfqpoint{3.352233in}{1.400000in}}{\pgfqpoint{2.407767in}{1.544118in}}%
\pgfusepath{clip}%
\pgfsetbuttcap%
\pgfsetroundjoin%
\pgfsetlinewidth{0.501875pt}%
\definecolor{currentstroke}{rgb}{0.272594,0.025563,0.353093}%
\pgfsetstrokecolor{currentstroke}%
\pgfsetdash{}{0pt}%
\pgfpathmoveto{\pgfqpoint{4.983959in}{1.467935in}}%
\pgfpathlineto{\pgfqpoint{4.931078in}{1.469802in}}%
\pgfusepath{stroke}%
\end{pgfscope}%
\begin{pgfscope}%
\pgfpathrectangle{\pgfqpoint{3.352233in}{1.400000in}}{\pgfqpoint{2.407767in}{1.544118in}}%
\pgfusepath{clip}%
\pgfsetbuttcap%
\pgfsetroundjoin%
\pgfsetlinewidth{0.501875pt}%
\definecolor{currentstroke}{rgb}{0.272594,0.025563,0.353093}%
\pgfsetstrokecolor{currentstroke}%
\pgfsetdash{}{0pt}%
\pgfpathmoveto{\pgfqpoint{4.931078in}{1.469802in}}%
\pgfpathlineto{\pgfqpoint{4.878316in}{1.471894in}}%
\pgfusepath{stroke}%
\end{pgfscope}%
\begin{pgfscope}%
\pgfpathrectangle{\pgfqpoint{3.352233in}{1.400000in}}{\pgfqpoint{2.407767in}{1.544118in}}%
\pgfusepath{clip}%
\pgfsetbuttcap%
\pgfsetroundjoin%
\pgfsetlinewidth{0.501875pt}%
\definecolor{currentstroke}{rgb}{0.272594,0.025563,0.353093}%
\pgfsetstrokecolor{currentstroke}%
\pgfsetdash{}{0pt}%
\pgfpathmoveto{\pgfqpoint{4.878316in}{1.471894in}}%
\pgfpathlineto{\pgfqpoint{4.825618in}{1.474550in}}%
\pgfusepath{stroke}%
\end{pgfscope}%
\begin{pgfscope}%
\pgfpathrectangle{\pgfqpoint{3.352233in}{1.400000in}}{\pgfqpoint{2.407767in}{1.544118in}}%
\pgfusepath{clip}%
\pgfsetbuttcap%
\pgfsetroundjoin%
\pgfsetlinewidth{0.501875pt}%
\definecolor{currentstroke}{rgb}{0.271305,0.019942,0.347269}%
\pgfsetstrokecolor{currentstroke}%
\pgfsetdash{}{0pt}%
\pgfpathmoveto{\pgfqpoint{4.825618in}{1.474550in}}%
\pgfpathlineto{\pgfqpoint{4.772837in}{1.477136in}}%
\pgfusepath{stroke}%
\end{pgfscope}%
\begin{pgfscope}%
\pgfpathrectangle{\pgfqpoint{3.352233in}{1.400000in}}{\pgfqpoint{2.407767in}{1.544118in}}%
\pgfusepath{clip}%
\pgfsetbuttcap%
\pgfsetroundjoin%
\pgfsetlinewidth{0.501875pt}%
\definecolor{currentstroke}{rgb}{0.268510,0.009605,0.335427}%
\pgfsetstrokecolor{currentstroke}%
\pgfsetdash{}{0pt}%
\pgfpathmoveto{\pgfqpoint{5.206279in}{2.866981in}}%
\pgfpathlineto{\pgfqpoint{5.153313in}{2.866680in}}%
\pgfusepath{stroke}%
\end{pgfscope}%
\begin{pgfscope}%
\pgfpathrectangle{\pgfqpoint{3.352233in}{1.400000in}}{\pgfqpoint{2.407767in}{1.544118in}}%
\pgfusepath{clip}%
\pgfsetbuttcap%
\pgfsetroundjoin%
\pgfsetlinewidth{0.501875pt}%
\definecolor{currentstroke}{rgb}{0.271305,0.019942,0.347269}%
\pgfsetstrokecolor{currentstroke}%
\pgfsetdash{}{0pt}%
\pgfpathmoveto{\pgfqpoint{5.153313in}{2.866680in}}%
\pgfpathlineto{\pgfqpoint{5.100420in}{2.866453in}}%
\pgfusepath{stroke}%
\end{pgfscope}%
\begin{pgfscope}%
\pgfpathrectangle{\pgfqpoint{3.352233in}{1.400000in}}{\pgfqpoint{2.407767in}{1.544118in}}%
\pgfusepath{clip}%
\pgfsetbuttcap%
\pgfsetroundjoin%
\pgfsetlinewidth{0.501875pt}%
\definecolor{currentstroke}{rgb}{0.269944,0.014625,0.341379}%
\pgfsetstrokecolor{currentstroke}%
\pgfsetdash{}{0pt}%
\pgfpathmoveto{\pgfqpoint{5.100420in}{2.866453in}}%
\pgfpathlineto{\pgfqpoint{5.047535in}{2.866345in}}%
\pgfusepath{stroke}%
\end{pgfscope}%
\begin{pgfscope}%
\pgfpathrectangle{\pgfqpoint{3.352233in}{1.400000in}}{\pgfqpoint{2.407767in}{1.544118in}}%
\pgfusepath{clip}%
\pgfsetbuttcap%
\pgfsetroundjoin%
\pgfsetlinewidth{0.501875pt}%
\definecolor{currentstroke}{rgb}{0.271305,0.019942,0.347269}%
\pgfsetstrokecolor{currentstroke}%
\pgfsetdash{}{0pt}%
\pgfpathmoveto{\pgfqpoint{5.047535in}{2.866345in}}%
\pgfpathlineto{\pgfqpoint{4.994683in}{2.864920in}}%
\pgfusepath{stroke}%
\end{pgfscope}%
\begin{pgfscope}%
\pgfpathrectangle{\pgfqpoint{3.352233in}{1.400000in}}{\pgfqpoint{2.407767in}{1.544118in}}%
\pgfusepath{clip}%
\pgfsetbuttcap%
\pgfsetroundjoin%
\pgfsetlinewidth{0.501875pt}%
\definecolor{currentstroke}{rgb}{0.272594,0.025563,0.353093}%
\pgfsetstrokecolor{currentstroke}%
\pgfsetdash{}{0pt}%
\pgfpathmoveto{\pgfqpoint{4.994683in}{2.864920in}}%
\pgfpathlineto{\pgfqpoint{4.941818in}{2.863622in}}%
\pgfusepath{stroke}%
\end{pgfscope}%
\begin{pgfscope}%
\pgfpathrectangle{\pgfqpoint{3.352233in}{1.400000in}}{\pgfqpoint{2.407767in}{1.544118in}}%
\pgfusepath{clip}%
\pgfsetbuttcap%
\pgfsetroundjoin%
\pgfsetlinewidth{0.501875pt}%
\definecolor{currentstroke}{rgb}{0.273809,0.031497,0.358853}%
\pgfsetstrokecolor{currentstroke}%
\pgfsetdash{}{0pt}%
\pgfpathmoveto{\pgfqpoint{4.941818in}{2.863622in}}%
\pgfpathlineto{\pgfqpoint{4.888931in}{2.862403in}}%
\pgfusepath{stroke}%
\end{pgfscope}%
\begin{pgfscope}%
\pgfpathrectangle{\pgfqpoint{3.352233in}{1.400000in}}{\pgfqpoint{2.407767in}{1.544118in}}%
\pgfusepath{clip}%
\pgfsetbuttcap%
\pgfsetroundjoin%
\pgfsetlinewidth{0.501875pt}%
\definecolor{currentstroke}{rgb}{0.268510,0.009605,0.335427}%
\pgfsetstrokecolor{currentstroke}%
\pgfsetdash{}{0pt}%
\pgfpathmoveto{\pgfqpoint{5.197136in}{1.503620in}}%
\pgfpathlineto{\pgfqpoint{5.144185in}{1.503733in}}%
\pgfusepath{stroke}%
\end{pgfscope}%
\begin{pgfscope}%
\pgfpathrectangle{\pgfqpoint{3.352233in}{1.400000in}}{\pgfqpoint{2.407767in}{1.544118in}}%
\pgfusepath{clip}%
\pgfsetbuttcap%
\pgfsetroundjoin%
\pgfsetlinewidth{0.501875pt}%
\definecolor{currentstroke}{rgb}{0.269944,0.014625,0.341379}%
\pgfsetstrokecolor{currentstroke}%
\pgfsetdash{}{0pt}%
\pgfpathmoveto{\pgfqpoint{5.144185in}{1.503733in}}%
\pgfpathlineto{\pgfqpoint{5.091232in}{1.504346in}}%
\pgfusepath{stroke}%
\end{pgfscope}%
\begin{pgfscope}%
\pgfpathrectangle{\pgfqpoint{3.352233in}{1.400000in}}{\pgfqpoint{2.407767in}{1.544118in}}%
\pgfusepath{clip}%
\pgfsetbuttcap%
\pgfsetroundjoin%
\pgfsetlinewidth{0.501875pt}%
\definecolor{currentstroke}{rgb}{0.271305,0.019942,0.347269}%
\pgfsetstrokecolor{currentstroke}%
\pgfsetdash{}{0pt}%
\pgfpathmoveto{\pgfqpoint{5.091232in}{1.504346in}}%
\pgfpathlineto{\pgfqpoint{5.038276in}{1.504836in}}%
\pgfusepath{stroke}%
\end{pgfscope}%
\begin{pgfscope}%
\pgfpathrectangle{\pgfqpoint{3.352233in}{1.400000in}}{\pgfqpoint{2.407767in}{1.544118in}}%
\pgfusepath{clip}%
\pgfsetbuttcap%
\pgfsetroundjoin%
\pgfsetlinewidth{0.501875pt}%
\definecolor{currentstroke}{rgb}{0.272594,0.025563,0.353093}%
\pgfsetstrokecolor{currentstroke}%
\pgfsetdash{}{0pt}%
\pgfpathmoveto{\pgfqpoint{5.038276in}{1.504836in}}%
\pgfpathlineto{\pgfqpoint{4.985340in}{1.505802in}}%
\pgfusepath{stroke}%
\end{pgfscope}%
\begin{pgfscope}%
\pgfpathrectangle{\pgfqpoint{3.352233in}{1.400000in}}{\pgfqpoint{2.407767in}{1.544118in}}%
\pgfusepath{clip}%
\pgfsetbuttcap%
\pgfsetroundjoin%
\pgfsetlinewidth{0.501875pt}%
\definecolor{currentstroke}{rgb}{0.274952,0.037752,0.364543}%
\pgfsetstrokecolor{currentstroke}%
\pgfsetdash{}{0pt}%
\pgfpathmoveto{\pgfqpoint{4.985340in}{1.505802in}}%
\pgfpathlineto{\pgfqpoint{4.932430in}{1.507300in}}%
\pgfusepath{stroke}%
\end{pgfscope}%
\begin{pgfscope}%
\pgfpathrectangle{\pgfqpoint{3.352233in}{1.400000in}}{\pgfqpoint{2.407767in}{1.544118in}}%
\pgfusepath{clip}%
\pgfsetbuttcap%
\pgfsetroundjoin%
\pgfsetlinewidth{0.501875pt}%
\definecolor{currentstroke}{rgb}{0.273809,0.031497,0.358853}%
\pgfsetstrokecolor{currentstroke}%
\pgfsetdash{}{0pt}%
\pgfpathmoveto{\pgfqpoint{4.932430in}{1.507300in}}%
\pgfpathlineto{\pgfqpoint{4.879705in}{1.509521in}}%
\pgfusepath{stroke}%
\end{pgfscope}%
\begin{pgfscope}%
\pgfpathrectangle{\pgfqpoint{3.352233in}{1.400000in}}{\pgfqpoint{2.407767in}{1.544118in}}%
\pgfusepath{clip}%
\pgfsetbuttcap%
\pgfsetroundjoin%
\pgfsetlinewidth{0.501875pt}%
\definecolor{currentstroke}{rgb}{0.273809,0.031497,0.358853}%
\pgfsetstrokecolor{currentstroke}%
\pgfsetdash{}{0pt}%
\pgfpathmoveto{\pgfqpoint{4.879705in}{1.509521in}}%
\pgfpathlineto{\pgfqpoint{4.827017in}{1.511883in}}%
\pgfusepath{stroke}%
\end{pgfscope}%
\begin{pgfscope}%
\pgfpathrectangle{\pgfqpoint{3.352233in}{1.400000in}}{\pgfqpoint{2.407767in}{1.544118in}}%
\pgfusepath{clip}%
\pgfsetbuttcap%
\pgfsetroundjoin%
\pgfsetlinewidth{0.501875pt}%
\definecolor{currentstroke}{rgb}{0.268510,0.009605,0.335427}%
\pgfsetstrokecolor{currentstroke}%
\pgfsetdash{}{0pt}%
\pgfpathmoveto{\pgfqpoint{5.206279in}{2.832235in}}%
\pgfpathlineto{\pgfqpoint{5.153311in}{2.831748in}}%
\pgfusepath{stroke}%
\end{pgfscope}%
\begin{pgfscope}%
\pgfpathrectangle{\pgfqpoint{3.352233in}{1.400000in}}{\pgfqpoint{2.407767in}{1.544118in}}%
\pgfusepath{clip}%
\pgfsetbuttcap%
\pgfsetroundjoin%
\pgfsetlinewidth{0.501875pt}%
\definecolor{currentstroke}{rgb}{0.271305,0.019942,0.347269}%
\pgfsetstrokecolor{currentstroke}%
\pgfsetdash{}{0pt}%
\pgfpathmoveto{\pgfqpoint{5.153311in}{2.831748in}}%
\pgfpathlineto{\pgfqpoint{5.100341in}{2.831522in}}%
\pgfusepath{stroke}%
\end{pgfscope}%
\begin{pgfscope}%
\pgfpathrectangle{\pgfqpoint{3.352233in}{1.400000in}}{\pgfqpoint{2.407767in}{1.544118in}}%
\pgfusepath{clip}%
\pgfsetbuttcap%
\pgfsetroundjoin%
\pgfsetlinewidth{0.501875pt}%
\definecolor{currentstroke}{rgb}{0.271305,0.019942,0.347269}%
\pgfsetstrokecolor{currentstroke}%
\pgfsetdash{}{0pt}%
\pgfpathmoveto{\pgfqpoint{5.100341in}{2.831522in}}%
\pgfpathlineto{\pgfqpoint{5.047391in}{2.831655in}}%
\pgfusepath{stroke}%
\end{pgfscope}%
\begin{pgfscope}%
\pgfpathrectangle{\pgfqpoint{3.352233in}{1.400000in}}{\pgfqpoint{2.407767in}{1.544118in}}%
\pgfusepath{clip}%
\pgfsetbuttcap%
\pgfsetroundjoin%
\pgfsetlinewidth{0.501875pt}%
\definecolor{currentstroke}{rgb}{0.273809,0.031497,0.358853}%
\pgfsetstrokecolor{currentstroke}%
\pgfsetdash{}{0pt}%
\pgfpathmoveto{\pgfqpoint{5.047391in}{2.831655in}}%
\pgfpathlineto{\pgfqpoint{4.994441in}{2.831889in}}%
\pgfusepath{stroke}%
\end{pgfscope}%
\begin{pgfscope}%
\pgfpathrectangle{\pgfqpoint{3.352233in}{1.400000in}}{\pgfqpoint{2.407767in}{1.544118in}}%
\pgfusepath{clip}%
\pgfsetbuttcap%
\pgfsetroundjoin%
\pgfsetlinewidth{0.501875pt}%
\definecolor{currentstroke}{rgb}{0.272594,0.025563,0.353093}%
\pgfsetstrokecolor{currentstroke}%
\pgfsetdash{}{0pt}%
\pgfpathmoveto{\pgfqpoint{4.994441in}{2.831889in}}%
\pgfpathlineto{\pgfqpoint{4.941491in}{2.831298in}}%
\pgfusepath{stroke}%
\end{pgfscope}%
\begin{pgfscope}%
\pgfpathrectangle{\pgfqpoint{3.352233in}{1.400000in}}{\pgfqpoint{2.407767in}{1.544118in}}%
\pgfusepath{clip}%
\pgfsetbuttcap%
\pgfsetroundjoin%
\pgfsetlinewidth{0.501875pt}%
\definecolor{currentstroke}{rgb}{0.274952,0.037752,0.364543}%
\pgfsetstrokecolor{currentstroke}%
\pgfsetdash{}{0pt}%
\pgfpathmoveto{\pgfqpoint{4.941491in}{2.831298in}}%
\pgfpathlineto{\pgfqpoint{4.888552in}{2.830027in}}%
\pgfusepath{stroke}%
\end{pgfscope}%
\begin{pgfscope}%
\pgfpathrectangle{\pgfqpoint{3.352233in}{1.400000in}}{\pgfqpoint{2.407767in}{1.544118in}}%
\pgfusepath{clip}%
\pgfsetbuttcap%
\pgfsetroundjoin%
\pgfsetlinewidth{0.501875pt}%
\definecolor{currentstroke}{rgb}{0.269944,0.014625,0.341379}%
\pgfsetstrokecolor{currentstroke}%
\pgfsetdash{}{0pt}%
\pgfpathmoveto{\pgfqpoint{4.501936in}{1.546629in}}%
\pgfpathlineto{\pgfqpoint{4.501936in}{1.546629in}}%
\pgfusepath{stroke}%
\end{pgfscope}%
\begin{pgfscope}%
\pgfpathrectangle{\pgfqpoint{3.352233in}{1.400000in}}{\pgfqpoint{2.407767in}{1.544118in}}%
\pgfusepath{clip}%
\pgfsetbuttcap%
\pgfsetroundjoin%
\pgfsetlinewidth{0.501875pt}%
\definecolor{currentstroke}{rgb}{0.269944,0.014625,0.341379}%
\pgfsetstrokecolor{currentstroke}%
\pgfsetdash{}{0pt}%
\pgfpathmoveto{\pgfqpoint{4.501936in}{1.546629in}}%
\pgfpathlineto{\pgfqpoint{4.501936in}{1.546629in}}%
\pgfusepath{stroke}%
\end{pgfscope}%
\begin{pgfscope}%
\pgfpathrectangle{\pgfqpoint{3.352233in}{1.400000in}}{\pgfqpoint{2.407767in}{1.544118in}}%
\pgfusepath{clip}%
\pgfsetbuttcap%
\pgfsetroundjoin%
\pgfsetlinewidth{0.501875pt}%
\definecolor{currentstroke}{rgb}{0.269944,0.014625,0.341379}%
\pgfsetstrokecolor{currentstroke}%
\pgfsetdash{}{0pt}%
\pgfpathmoveto{\pgfqpoint{4.501936in}{1.546629in}}%
\pgfpathlineto{\pgfqpoint{4.508444in}{1.551861in}}%
\pgfusepath{stroke}%
\end{pgfscope}%
\begin{pgfscope}%
\pgfpathrectangle{\pgfqpoint{3.352233in}{1.400000in}}{\pgfqpoint{2.407767in}{1.544118in}}%
\pgfusepath{clip}%
\pgfsetbuttcap%
\pgfsetroundjoin%
\pgfsetlinewidth{0.501875pt}%
\definecolor{currentstroke}{rgb}{0.268510,0.009605,0.335427}%
\pgfsetstrokecolor{currentstroke}%
\pgfsetdash{}{0pt}%
\pgfpathmoveto{\pgfqpoint{4.508444in}{1.551861in}}%
\pgfpathlineto{\pgfqpoint{4.508444in}{1.551861in}}%
\pgfusepath{stroke}%
\end{pgfscope}%
\begin{pgfscope}%
\pgfpathrectangle{\pgfqpoint{3.352233in}{1.400000in}}{\pgfqpoint{2.407767in}{1.544118in}}%
\pgfusepath{clip}%
\pgfsetbuttcap%
\pgfsetroundjoin%
\pgfsetlinewidth{0.501875pt}%
\definecolor{currentstroke}{rgb}{0.268510,0.009605,0.335427}%
\pgfsetstrokecolor{currentstroke}%
\pgfsetdash{}{0pt}%
\pgfpathmoveto{\pgfqpoint{4.508444in}{1.551861in}}%
\pgfpathlineto{\pgfqpoint{4.508761in}{1.554423in}}%
\pgfusepath{stroke}%
\end{pgfscope}%
\begin{pgfscope}%
\pgfpathrectangle{\pgfqpoint{3.352233in}{1.400000in}}{\pgfqpoint{2.407767in}{1.544118in}}%
\pgfusepath{clip}%
\pgfsetbuttcap%
\pgfsetroundjoin%
\pgfsetlinewidth{0.501875pt}%
\definecolor{currentstroke}{rgb}{0.268510,0.009605,0.335427}%
\pgfsetstrokecolor{currentstroke}%
\pgfsetdash{}{0pt}%
\pgfpathmoveto{\pgfqpoint{4.508761in}{1.554423in}}%
\pgfpathlineto{\pgfqpoint{4.509129in}{1.555890in}}%
\pgfusepath{stroke}%
\end{pgfscope}%
\begin{pgfscope}%
\pgfpathrectangle{\pgfqpoint{3.352233in}{1.400000in}}{\pgfqpoint{2.407767in}{1.544118in}}%
\pgfusepath{clip}%
\pgfsetbuttcap%
\pgfsetroundjoin%
\pgfsetlinewidth{0.501875pt}%
\definecolor{currentstroke}{rgb}{0.268510,0.009605,0.335427}%
\pgfsetstrokecolor{currentstroke}%
\pgfsetdash{}{0pt}%
\pgfpathmoveto{\pgfqpoint{4.509129in}{1.555890in}}%
\pgfpathlineto{\pgfqpoint{4.509261in}{1.556831in}}%
\pgfusepath{stroke}%
\end{pgfscope}%
\begin{pgfscope}%
\pgfpathrectangle{\pgfqpoint{3.352233in}{1.400000in}}{\pgfqpoint{2.407767in}{1.544118in}}%
\pgfusepath{clip}%
\pgfsetbuttcap%
\pgfsetroundjoin%
\pgfsetlinewidth{0.501875pt}%
\definecolor{currentstroke}{rgb}{0.268510,0.009605,0.335427}%
\pgfsetstrokecolor{currentstroke}%
\pgfsetdash{}{0pt}%
\pgfpathmoveto{\pgfqpoint{4.509261in}{1.556831in}}%
\pgfpathlineto{\pgfqpoint{4.509101in}{1.557496in}}%
\pgfusepath{stroke}%
\end{pgfscope}%
\begin{pgfscope}%
\pgfpathrectangle{\pgfqpoint{3.352233in}{1.400000in}}{\pgfqpoint{2.407767in}{1.544118in}}%
\pgfusepath{clip}%
\pgfsetbuttcap%
\pgfsetroundjoin%
\pgfsetlinewidth{0.501875pt}%
\definecolor{currentstroke}{rgb}{0.268510,0.009605,0.335427}%
\pgfsetstrokecolor{currentstroke}%
\pgfsetdash{}{0pt}%
\pgfpathmoveto{\pgfqpoint{4.509101in}{1.557496in}}%
\pgfpathlineto{\pgfqpoint{4.508884in}{1.557943in}}%
\pgfusepath{stroke}%
\end{pgfscope}%
\begin{pgfscope}%
\pgfpathrectangle{\pgfqpoint{3.352233in}{1.400000in}}{\pgfqpoint{2.407767in}{1.544118in}}%
\pgfusepath{clip}%
\pgfsetbuttcap%
\pgfsetroundjoin%
\pgfsetlinewidth{0.501875pt}%
\definecolor{currentstroke}{rgb}{0.268510,0.009605,0.335427}%
\pgfsetstrokecolor{currentstroke}%
\pgfsetdash{}{0pt}%
\pgfpathmoveto{\pgfqpoint{4.508884in}{1.557943in}}%
\pgfpathlineto{\pgfqpoint{4.509005in}{1.558180in}}%
\pgfusepath{stroke}%
\end{pgfscope}%
\begin{pgfscope}%
\pgfpathrectangle{\pgfqpoint{3.352233in}{1.400000in}}{\pgfqpoint{2.407767in}{1.544118in}}%
\pgfusepath{clip}%
\pgfsetbuttcap%
\pgfsetroundjoin%
\pgfsetlinewidth{0.501875pt}%
\definecolor{currentstroke}{rgb}{0.268510,0.009605,0.335427}%
\pgfsetstrokecolor{currentstroke}%
\pgfsetdash{}{0pt}%
\pgfpathmoveto{\pgfqpoint{4.509005in}{1.558180in}}%
\pgfpathlineto{\pgfqpoint{4.509252in}{1.558298in}}%
\pgfusepath{stroke}%
\end{pgfscope}%
\begin{pgfscope}%
\pgfpathrectangle{\pgfqpoint{3.352233in}{1.400000in}}{\pgfqpoint{2.407767in}{1.544118in}}%
\pgfusepath{clip}%
\pgfsetbuttcap%
\pgfsetroundjoin%
\pgfsetlinewidth{0.501875pt}%
\definecolor{currentstroke}{rgb}{0.268510,0.009605,0.335427}%
\pgfsetstrokecolor{currentstroke}%
\pgfsetdash{}{0pt}%
\pgfpathmoveto{\pgfqpoint{4.509252in}{1.558298in}}%
\pgfpathlineto{\pgfqpoint{4.509336in}{1.558383in}}%
\pgfusepath{stroke}%
\end{pgfscope}%
\begin{pgfscope}%
\pgfpathrectangle{\pgfqpoint{3.352233in}{1.400000in}}{\pgfqpoint{2.407767in}{1.544118in}}%
\pgfusepath{clip}%
\pgfsetbuttcap%
\pgfsetroundjoin%
\pgfsetlinewidth{0.501875pt}%
\definecolor{currentstroke}{rgb}{0.268510,0.009605,0.335427}%
\pgfsetstrokecolor{currentstroke}%
\pgfsetdash{}{0pt}%
\pgfpathmoveto{\pgfqpoint{4.509336in}{1.558383in}}%
\pgfpathlineto{\pgfqpoint{4.509177in}{1.558462in}}%
\pgfusepath{stroke}%
\end{pgfscope}%
\begin{pgfscope}%
\pgfpathrectangle{\pgfqpoint{3.352233in}{1.400000in}}{\pgfqpoint{2.407767in}{1.544118in}}%
\pgfusepath{clip}%
\pgfsetbuttcap%
\pgfsetroundjoin%
\pgfsetlinewidth{0.501875pt}%
\definecolor{currentstroke}{rgb}{0.268510,0.009605,0.335427}%
\pgfsetstrokecolor{currentstroke}%
\pgfsetdash{}{0pt}%
\pgfpathmoveto{\pgfqpoint{4.509177in}{1.558462in}}%
\pgfpathlineto{\pgfqpoint{4.508965in}{1.558520in}}%
\pgfusepath{stroke}%
\end{pgfscope}%
\begin{pgfscope}%
\pgfpathrectangle{\pgfqpoint{3.352233in}{1.400000in}}{\pgfqpoint{2.407767in}{1.544118in}}%
\pgfusepath{clip}%
\pgfsetbuttcap%
\pgfsetroundjoin%
\pgfsetlinewidth{0.501875pt}%
\definecolor{currentstroke}{rgb}{0.268510,0.009605,0.335427}%
\pgfsetstrokecolor{currentstroke}%
\pgfsetdash{}{0pt}%
\pgfpathmoveto{\pgfqpoint{4.508965in}{1.558520in}}%
\pgfpathlineto{\pgfqpoint{4.509041in}{1.558531in}}%
\pgfusepath{stroke}%
\end{pgfscope}%
\begin{pgfscope}%
\pgfpathrectangle{\pgfqpoint{3.352233in}{1.400000in}}{\pgfqpoint{2.407767in}{1.544118in}}%
\pgfusepath{clip}%
\pgfsetbuttcap%
\pgfsetroundjoin%
\pgfsetlinewidth{0.501875pt}%
\definecolor{currentstroke}{rgb}{0.268510,0.009605,0.335427}%
\pgfsetstrokecolor{currentstroke}%
\pgfsetdash{}{0pt}%
\pgfpathmoveto{\pgfqpoint{4.509041in}{1.558531in}}%
\pgfpathlineto{\pgfqpoint{4.509249in}{1.558525in}}%
\pgfusepath{stroke}%
\end{pgfscope}%
\begin{pgfscope}%
\pgfpathrectangle{\pgfqpoint{3.352233in}{1.400000in}}{\pgfqpoint{2.407767in}{1.544118in}}%
\pgfusepath{clip}%
\pgfsetbuttcap%
\pgfsetroundjoin%
\pgfsetlinewidth{0.501875pt}%
\definecolor{currentstroke}{rgb}{0.268510,0.009605,0.335427}%
\pgfsetstrokecolor{currentstroke}%
\pgfsetdash{}{0pt}%
\pgfpathmoveto{\pgfqpoint{4.509249in}{1.558525in}}%
\pgfpathlineto{\pgfqpoint{4.509324in}{1.558530in}}%
\pgfusepath{stroke}%
\end{pgfscope}%
\begin{pgfscope}%
\pgfpathrectangle{\pgfqpoint{3.352233in}{1.400000in}}{\pgfqpoint{2.407767in}{1.544118in}}%
\pgfusepath{clip}%
\pgfsetbuttcap%
\pgfsetroundjoin%
\pgfsetlinewidth{0.501875pt}%
\definecolor{currentstroke}{rgb}{0.268510,0.009605,0.335427}%
\pgfsetstrokecolor{currentstroke}%
\pgfsetdash{}{0pt}%
\pgfpathmoveto{\pgfqpoint{4.509324in}{1.558530in}}%
\pgfpathlineto{\pgfqpoint{4.509186in}{1.558552in}}%
\pgfusepath{stroke}%
\end{pgfscope}%
\begin{pgfscope}%
\pgfpathrectangle{\pgfqpoint{3.352233in}{1.400000in}}{\pgfqpoint{2.407767in}{1.544118in}}%
\pgfusepath{clip}%
\pgfsetbuttcap%
\pgfsetroundjoin%
\pgfsetlinewidth{0.501875pt}%
\definecolor{currentstroke}{rgb}{0.268510,0.009605,0.335427}%
\pgfsetstrokecolor{currentstroke}%
\pgfsetdash{}{0pt}%
\pgfpathmoveto{\pgfqpoint{4.509186in}{1.558552in}}%
\pgfpathlineto{\pgfqpoint{4.508998in}{1.558573in}}%
\pgfusepath{stroke}%
\end{pgfscope}%
\begin{pgfscope}%
\pgfpathrectangle{\pgfqpoint{3.352233in}{1.400000in}}{\pgfqpoint{2.407767in}{1.544118in}}%
\pgfusepath{clip}%
\pgfsetbuttcap%
\pgfsetroundjoin%
\pgfsetlinewidth{0.501875pt}%
\definecolor{currentstroke}{rgb}{0.268510,0.009605,0.335427}%
\pgfsetstrokecolor{currentstroke}%
\pgfsetdash{}{0pt}%
\pgfpathmoveto{\pgfqpoint{4.508998in}{1.558573in}}%
\pgfpathlineto{\pgfqpoint{4.509052in}{1.558565in}}%
\pgfusepath{stroke}%
\end{pgfscope}%
\begin{pgfscope}%
\pgfpathrectangle{\pgfqpoint{3.352233in}{1.400000in}}{\pgfqpoint{2.407767in}{1.544118in}}%
\pgfusepath{clip}%
\pgfsetbuttcap%
\pgfsetroundjoin%
\pgfsetlinewidth{0.501875pt}%
\definecolor{currentstroke}{rgb}{0.268510,0.009605,0.335427}%
\pgfsetstrokecolor{currentstroke}%
\pgfsetdash{}{0pt}%
\pgfpathmoveto{\pgfqpoint{4.509052in}{1.558565in}}%
\pgfpathlineto{\pgfqpoint{4.509234in}{1.558548in}}%
\pgfusepath{stroke}%
\end{pgfscope}%
\begin{pgfscope}%
\pgfpathrectangle{\pgfqpoint{3.352233in}{1.400000in}}{\pgfqpoint{2.407767in}{1.544118in}}%
\pgfusepath{clip}%
\pgfsetbuttcap%
\pgfsetroundjoin%
\pgfsetlinewidth{0.501875pt}%
\definecolor{currentstroke}{rgb}{0.268510,0.009605,0.335427}%
\pgfsetstrokecolor{currentstroke}%
\pgfsetdash{}{0pt}%
\pgfpathmoveto{\pgfqpoint{4.509234in}{1.558548in}}%
\pgfpathlineto{\pgfqpoint{4.509306in}{1.558545in}}%
\pgfusepath{stroke}%
\end{pgfscope}%
\begin{pgfscope}%
\pgfpathrectangle{\pgfqpoint{3.352233in}{1.400000in}}{\pgfqpoint{2.407767in}{1.544118in}}%
\pgfusepath{clip}%
\pgfsetbuttcap%
\pgfsetroundjoin%
\pgfsetlinewidth{0.501875pt}%
\definecolor{currentstroke}{rgb}{0.268510,0.009605,0.335427}%
\pgfsetstrokecolor{currentstroke}%
\pgfsetdash{}{0pt}%
\pgfpathmoveto{\pgfqpoint{4.509306in}{1.558545in}}%
\pgfpathlineto{\pgfqpoint{4.509189in}{1.558560in}}%
\pgfusepath{stroke}%
\end{pgfscope}%
\begin{pgfscope}%
\pgfpathrectangle{\pgfqpoint{3.352233in}{1.400000in}}{\pgfqpoint{2.407767in}{1.544118in}}%
\pgfusepath{clip}%
\pgfsetbuttcap%
\pgfsetroundjoin%
\pgfsetlinewidth{0.501875pt}%
\definecolor{currentstroke}{rgb}{0.268510,0.009605,0.335427}%
\pgfsetstrokecolor{currentstroke}%
\pgfsetdash{}{0pt}%
\pgfpathmoveto{\pgfqpoint{4.509189in}{1.558560in}}%
\pgfpathlineto{\pgfqpoint{4.509022in}{1.558575in}}%
\pgfusepath{stroke}%
\end{pgfscope}%
\begin{pgfscope}%
\pgfpathrectangle{\pgfqpoint{3.352233in}{1.400000in}}{\pgfqpoint{2.407767in}{1.544118in}}%
\pgfusepath{clip}%
\pgfsetbuttcap%
\pgfsetroundjoin%
\pgfsetlinewidth{0.501875pt}%
\definecolor{currentstroke}{rgb}{0.268510,0.009605,0.335427}%
\pgfsetstrokecolor{currentstroke}%
\pgfsetdash{}{0pt}%
\pgfpathmoveto{\pgfqpoint{4.509022in}{1.558575in}}%
\pgfpathlineto{\pgfqpoint{4.509059in}{1.558568in}}%
\pgfusepath{stroke}%
\end{pgfscope}%
\begin{pgfscope}%
\pgfpathrectangle{\pgfqpoint{3.352233in}{1.400000in}}{\pgfqpoint{2.407767in}{1.544118in}}%
\pgfusepath{clip}%
\pgfsetbuttcap%
\pgfsetroundjoin%
\pgfsetlinewidth{0.501875pt}%
\definecolor{currentstroke}{rgb}{0.268510,0.009605,0.335427}%
\pgfsetstrokecolor{currentstroke}%
\pgfsetdash{}{0pt}%
\pgfpathmoveto{\pgfqpoint{4.509059in}{1.558568in}}%
\pgfpathlineto{\pgfqpoint{4.509220in}{1.558552in}}%
\pgfusepath{stroke}%
\end{pgfscope}%
\begin{pgfscope}%
\pgfpathrectangle{\pgfqpoint{3.352233in}{1.400000in}}{\pgfqpoint{2.407767in}{1.544118in}}%
\pgfusepath{clip}%
\pgfsetbuttcap%
\pgfsetroundjoin%
\pgfsetlinewidth{0.501875pt}%
\definecolor{currentstroke}{rgb}{0.268510,0.009605,0.335427}%
\pgfsetstrokecolor{currentstroke}%
\pgfsetdash{}{0pt}%
\pgfpathmoveto{\pgfqpoint{4.509220in}{1.558552in}}%
\pgfpathlineto{\pgfqpoint{4.509290in}{1.558548in}}%
\pgfusepath{stroke}%
\end{pgfscope}%
\begin{pgfscope}%
\pgfpathrectangle{\pgfqpoint{3.352233in}{1.400000in}}{\pgfqpoint{2.407767in}{1.544118in}}%
\pgfusepath{clip}%
\pgfsetbuttcap%
\pgfsetroundjoin%
\pgfsetlinewidth{0.501875pt}%
\definecolor{currentstroke}{rgb}{0.268510,0.009605,0.335427}%
\pgfsetstrokecolor{currentstroke}%
\pgfsetdash{}{0pt}%
\pgfpathmoveto{\pgfqpoint{4.509290in}{1.558548in}}%
\pgfpathlineto{\pgfqpoint{4.509191in}{1.558560in}}%
\pgfusepath{stroke}%
\end{pgfscope}%
\begin{pgfscope}%
\pgfpathrectangle{\pgfqpoint{3.352233in}{1.400000in}}{\pgfqpoint{2.407767in}{1.544118in}}%
\pgfusepath{clip}%
\pgfsetbuttcap%
\pgfsetroundjoin%
\pgfsetlinewidth{0.501875pt}%
\definecolor{currentstroke}{rgb}{0.268510,0.009605,0.335427}%
\pgfsetstrokecolor{currentstroke}%
\pgfsetdash{}{0pt}%
\pgfpathmoveto{\pgfqpoint{4.509191in}{1.558560in}}%
\pgfpathlineto{\pgfqpoint{4.509043in}{1.558574in}}%
\pgfusepath{stroke}%
\end{pgfscope}%
\begin{pgfscope}%
\pgfpathrectangle{\pgfqpoint{3.352233in}{1.400000in}}{\pgfqpoint{2.407767in}{1.544118in}}%
\pgfusepath{clip}%
\pgfsetbuttcap%
\pgfsetroundjoin%
\pgfsetlinewidth{0.501875pt}%
\definecolor{currentstroke}{rgb}{0.268510,0.009605,0.335427}%
\pgfsetstrokecolor{currentstroke}%
\pgfsetdash{}{0pt}%
\pgfpathmoveto{\pgfqpoint{4.509043in}{1.558574in}}%
\pgfpathlineto{\pgfqpoint{4.509066in}{1.558568in}}%
\pgfusepath{stroke}%
\end{pgfscope}%
\begin{pgfscope}%
\pgfpathrectangle{\pgfqpoint{3.352233in}{1.400000in}}{\pgfqpoint{2.407767in}{1.544118in}}%
\pgfusepath{clip}%
\pgfsetbuttcap%
\pgfsetroundjoin%
\pgfsetlinewidth{0.501875pt}%
\definecolor{currentstroke}{rgb}{0.268510,0.009605,0.335427}%
\pgfsetstrokecolor{currentstroke}%
\pgfsetdash{}{0pt}%
\pgfpathmoveto{\pgfqpoint{4.509066in}{1.558568in}}%
\pgfpathlineto{\pgfqpoint{4.509208in}{1.558554in}}%
\pgfusepath{stroke}%
\end{pgfscope}%
\begin{pgfscope}%
\pgfpathrectangle{\pgfqpoint{3.352233in}{1.400000in}}{\pgfqpoint{2.407767in}{1.544118in}}%
\pgfusepath{clip}%
\pgfsetbuttcap%
\pgfsetroundjoin%
\pgfsetlinewidth{0.501875pt}%
\definecolor{currentstroke}{rgb}{0.268510,0.009605,0.335427}%
\pgfsetstrokecolor{currentstroke}%
\pgfsetdash{}{0pt}%
\pgfpathmoveto{\pgfqpoint{4.509208in}{1.558554in}}%
\pgfpathlineto{\pgfqpoint{4.509275in}{1.558550in}}%
\pgfusepath{stroke}%
\end{pgfscope}%
\begin{pgfscope}%
\pgfpathrectangle{\pgfqpoint{3.352233in}{1.400000in}}{\pgfqpoint{2.407767in}{1.544118in}}%
\pgfusepath{clip}%
\pgfsetbuttcap%
\pgfsetroundjoin%
\pgfsetlinewidth{0.501875pt}%
\definecolor{currentstroke}{rgb}{0.268510,0.009605,0.335427}%
\pgfsetstrokecolor{currentstroke}%
\pgfsetdash{}{0pt}%
\pgfpathmoveto{\pgfqpoint{4.509275in}{1.558550in}}%
\pgfpathlineto{\pgfqpoint{4.509192in}{1.558560in}}%
\pgfusepath{stroke}%
\end{pgfscope}%
\begin{pgfscope}%
\pgfpathrectangle{\pgfqpoint{3.352233in}{1.400000in}}{\pgfqpoint{2.407767in}{1.544118in}}%
\pgfusepath{clip}%
\pgfsetbuttcap%
\pgfsetroundjoin%
\pgfsetlinewidth{0.501875pt}%
\definecolor{currentstroke}{rgb}{0.268510,0.009605,0.335427}%
\pgfsetstrokecolor{currentstroke}%
\pgfsetdash{}{0pt}%
\pgfpathmoveto{\pgfqpoint{4.509192in}{1.558560in}}%
\pgfpathlineto{\pgfqpoint{4.509060in}{1.558572in}}%
\pgfusepath{stroke}%
\end{pgfscope}%
\begin{pgfscope}%
\pgfpathrectangle{\pgfqpoint{3.352233in}{1.400000in}}{\pgfqpoint{2.407767in}{1.544118in}}%
\pgfusepath{clip}%
\pgfsetbuttcap%
\pgfsetroundjoin%
\pgfsetlinewidth{0.501875pt}%
\definecolor{currentstroke}{rgb}{0.268510,0.009605,0.335427}%
\pgfsetstrokecolor{currentstroke}%
\pgfsetdash{}{0pt}%
\pgfpathmoveto{\pgfqpoint{4.509060in}{1.558572in}}%
\pgfpathlineto{\pgfqpoint{4.509073in}{1.558568in}}%
\pgfusepath{stroke}%
\end{pgfscope}%
\begin{pgfscope}%
\pgfpathrectangle{\pgfqpoint{3.352233in}{1.400000in}}{\pgfqpoint{2.407767in}{1.544118in}}%
\pgfusepath{clip}%
\pgfsetbuttcap%
\pgfsetroundjoin%
\pgfsetlinewidth{0.501875pt}%
\definecolor{currentstroke}{rgb}{0.268510,0.009605,0.335427}%
\pgfsetstrokecolor{currentstroke}%
\pgfsetdash{}{0pt}%
\pgfpathmoveto{\pgfqpoint{4.509073in}{1.558568in}}%
\pgfpathlineto{\pgfqpoint{4.509197in}{1.558555in}}%
\pgfusepath{stroke}%
\end{pgfscope}%
\begin{pgfscope}%
\pgfpathrectangle{\pgfqpoint{3.352233in}{1.400000in}}{\pgfqpoint{2.407767in}{1.544118in}}%
\pgfusepath{clip}%
\pgfsetbuttcap%
\pgfsetroundjoin%
\pgfsetlinewidth{0.501875pt}%
\definecolor{currentstroke}{rgb}{0.268510,0.009605,0.335427}%
\pgfsetstrokecolor{currentstroke}%
\pgfsetdash{}{0pt}%
\pgfpathmoveto{\pgfqpoint{4.509197in}{1.558555in}}%
\pgfpathlineto{\pgfqpoint{4.509262in}{1.558551in}}%
\pgfusepath{stroke}%
\end{pgfscope}%
\begin{pgfscope}%
\pgfpathrectangle{\pgfqpoint{3.352233in}{1.400000in}}{\pgfqpoint{2.407767in}{1.544118in}}%
\pgfusepath{clip}%
\pgfsetbuttcap%
\pgfsetroundjoin%
\pgfsetlinewidth{0.501875pt}%
\definecolor{currentstroke}{rgb}{0.268510,0.009605,0.335427}%
\pgfsetstrokecolor{currentstroke}%
\pgfsetdash{}{0pt}%
\pgfpathmoveto{\pgfqpoint{4.509262in}{1.558551in}}%
\pgfpathlineto{\pgfqpoint{4.509194in}{1.558560in}}%
\pgfusepath{stroke}%
\end{pgfscope}%
\begin{pgfscope}%
\pgfpathrectangle{\pgfqpoint{3.352233in}{1.400000in}}{\pgfqpoint{2.407767in}{1.544118in}}%
\pgfusepath{clip}%
\pgfsetbuttcap%
\pgfsetroundjoin%
\pgfsetlinewidth{0.501875pt}%
\definecolor{currentstroke}{rgb}{0.268510,0.009605,0.335427}%
\pgfsetstrokecolor{currentstroke}%
\pgfsetdash{}{0pt}%
\pgfpathmoveto{\pgfqpoint{4.509194in}{1.558560in}}%
\pgfpathlineto{\pgfqpoint{4.509076in}{1.558571in}}%
\pgfusepath{stroke}%
\end{pgfscope}%
\begin{pgfscope}%
\pgfpathrectangle{\pgfqpoint{3.352233in}{1.400000in}}{\pgfqpoint{2.407767in}{1.544118in}}%
\pgfusepath{clip}%
\pgfsetbuttcap%
\pgfsetroundjoin%
\pgfsetlinewidth{0.501875pt}%
\definecolor{currentstroke}{rgb}{0.268510,0.009605,0.335427}%
\pgfsetstrokecolor{currentstroke}%
\pgfsetdash{}{0pt}%
\pgfpathmoveto{\pgfqpoint{4.509076in}{1.558571in}}%
\pgfpathlineto{\pgfqpoint{4.509079in}{1.558568in}}%
\pgfusepath{stroke}%
\end{pgfscope}%
\begin{pgfscope}%
\pgfpathrectangle{\pgfqpoint{3.352233in}{1.400000in}}{\pgfqpoint{2.407767in}{1.544118in}}%
\pgfusepath{clip}%
\pgfsetbuttcap%
\pgfsetroundjoin%
\pgfsetlinewidth{0.501875pt}%
\definecolor{currentstroke}{rgb}{0.268510,0.009605,0.335427}%
\pgfsetstrokecolor{currentstroke}%
\pgfsetdash{}{0pt}%
\pgfpathmoveto{\pgfqpoint{4.509079in}{1.558568in}}%
\pgfpathlineto{\pgfqpoint{4.509188in}{1.558556in}}%
\pgfusepath{stroke}%
\end{pgfscope}%
\begin{pgfscope}%
\pgfpathrectangle{\pgfqpoint{3.352233in}{1.400000in}}{\pgfqpoint{2.407767in}{1.544118in}}%
\pgfusepath{clip}%
\pgfsetbuttcap%
\pgfsetroundjoin%
\pgfsetlinewidth{0.501875pt}%
\definecolor{currentstroke}{rgb}{0.268510,0.009605,0.335427}%
\pgfsetstrokecolor{currentstroke}%
\pgfsetdash{}{0pt}%
\pgfpathmoveto{\pgfqpoint{4.509188in}{1.558556in}}%
\pgfpathlineto{\pgfqpoint{4.509251in}{1.558552in}}%
\pgfusepath{stroke}%
\end{pgfscope}%
\begin{pgfscope}%
\pgfpathrectangle{\pgfqpoint{3.352233in}{1.400000in}}{\pgfqpoint{2.407767in}{1.544118in}}%
\pgfusepath{clip}%
\pgfsetbuttcap%
\pgfsetroundjoin%
\pgfsetlinewidth{0.501875pt}%
\definecolor{currentstroke}{rgb}{0.268510,0.009605,0.335427}%
\pgfsetstrokecolor{currentstroke}%
\pgfsetdash{}{0pt}%
\pgfpathmoveto{\pgfqpoint{4.509251in}{1.558552in}}%
\pgfpathlineto{\pgfqpoint{4.509194in}{1.558560in}}%
\pgfusepath{stroke}%
\end{pgfscope}%
\begin{pgfscope}%
\pgfpathrectangle{\pgfqpoint{3.352233in}{1.400000in}}{\pgfqpoint{2.407767in}{1.544118in}}%
\pgfusepath{clip}%
\pgfsetbuttcap%
\pgfsetroundjoin%
\pgfsetlinewidth{0.501875pt}%
\definecolor{currentstroke}{rgb}{0.268510,0.009605,0.335427}%
\pgfsetstrokecolor{currentstroke}%
\pgfsetdash{}{0pt}%
\pgfpathmoveto{\pgfqpoint{4.509194in}{1.558560in}}%
\pgfpathlineto{\pgfqpoint{4.509089in}{1.558570in}}%
\pgfusepath{stroke}%
\end{pgfscope}%
\begin{pgfscope}%
\pgfpathrectangle{\pgfqpoint{3.352233in}{1.400000in}}{\pgfqpoint{2.407767in}{1.544118in}}%
\pgfusepath{clip}%
\pgfsetbuttcap%
\pgfsetroundjoin%
\pgfsetlinewidth{0.501875pt}%
\definecolor{currentstroke}{rgb}{0.268510,0.009605,0.335427}%
\pgfsetstrokecolor{currentstroke}%
\pgfsetdash{}{0pt}%
\pgfpathmoveto{\pgfqpoint{4.509089in}{1.558570in}}%
\pgfpathlineto{\pgfqpoint{4.509085in}{1.558567in}}%
\pgfusepath{stroke}%
\end{pgfscope}%
\begin{pgfscope}%
\pgfpathrectangle{\pgfqpoint{3.352233in}{1.400000in}}{\pgfqpoint{2.407767in}{1.544118in}}%
\pgfusepath{clip}%
\pgfsetbuttcap%
\pgfsetroundjoin%
\pgfsetlinewidth{0.501875pt}%
\definecolor{currentstroke}{rgb}{0.268510,0.009605,0.335427}%
\pgfsetstrokecolor{currentstroke}%
\pgfsetdash{}{0pt}%
\pgfpathmoveto{\pgfqpoint{4.509085in}{1.558567in}}%
\pgfpathlineto{\pgfqpoint{4.509180in}{1.558557in}}%
\pgfusepath{stroke}%
\end{pgfscope}%
\begin{pgfscope}%
\pgfpathrectangle{\pgfqpoint{3.352233in}{1.400000in}}{\pgfqpoint{2.407767in}{1.544118in}}%
\pgfusepath{clip}%
\pgfsetbuttcap%
\pgfsetroundjoin%
\pgfsetlinewidth{0.501875pt}%
\definecolor{currentstroke}{rgb}{0.268510,0.009605,0.335427}%
\pgfsetstrokecolor{currentstroke}%
\pgfsetdash{}{0pt}%
\pgfpathmoveto{\pgfqpoint{4.509180in}{1.558557in}}%
\pgfpathlineto{\pgfqpoint{4.509240in}{1.558553in}}%
\pgfusepath{stroke}%
\end{pgfscope}%
\begin{pgfscope}%
\pgfpathrectangle{\pgfqpoint{3.352233in}{1.400000in}}{\pgfqpoint{2.407767in}{1.544118in}}%
\pgfusepath{clip}%
\pgfsetbuttcap%
\pgfsetroundjoin%
\pgfsetlinewidth{0.501875pt}%
\definecolor{currentstroke}{rgb}{0.268510,0.009605,0.335427}%
\pgfsetstrokecolor{currentstroke}%
\pgfsetdash{}{0pt}%
\pgfpathmoveto{\pgfqpoint{4.509240in}{1.558553in}}%
\pgfpathlineto{\pgfqpoint{4.509194in}{1.558559in}}%
\pgfusepath{stroke}%
\end{pgfscope}%
\begin{pgfscope}%
\pgfpathrectangle{\pgfqpoint{3.352233in}{1.400000in}}{\pgfqpoint{2.407767in}{1.544118in}}%
\pgfusepath{clip}%
\pgfsetbuttcap%
\pgfsetroundjoin%
\pgfsetlinewidth{0.501875pt}%
\definecolor{currentstroke}{rgb}{0.268510,0.009605,0.335427}%
\pgfsetstrokecolor{currentstroke}%
\pgfsetdash{}{0pt}%
\pgfpathmoveto{\pgfqpoint{4.509194in}{1.558559in}}%
\pgfpathlineto{\pgfqpoint{4.509100in}{1.558568in}}%
\pgfusepath{stroke}%
\end{pgfscope}%
\begin{pgfscope}%
\pgfpathrectangle{\pgfqpoint{3.352233in}{1.400000in}}{\pgfqpoint{2.407767in}{1.544118in}}%
\pgfusepath{clip}%
\pgfsetbuttcap%
\pgfsetroundjoin%
\pgfsetlinewidth{0.501875pt}%
\definecolor{currentstroke}{rgb}{0.268510,0.009605,0.335427}%
\pgfsetstrokecolor{currentstroke}%
\pgfsetdash{}{0pt}%
\pgfpathmoveto{\pgfqpoint{4.509100in}{1.558568in}}%
\pgfpathlineto{\pgfqpoint{4.509091in}{1.558567in}}%
\pgfusepath{stroke}%
\end{pgfscope}%
\begin{pgfscope}%
\pgfpathrectangle{\pgfqpoint{3.352233in}{1.400000in}}{\pgfqpoint{2.407767in}{1.544118in}}%
\pgfusepath{clip}%
\pgfsetbuttcap%
\pgfsetroundjoin%
\pgfsetlinewidth{0.501875pt}%
\definecolor{currentstroke}{rgb}{0.268510,0.009605,0.335427}%
\pgfsetstrokecolor{currentstroke}%
\pgfsetdash{}{0pt}%
\pgfpathmoveto{\pgfqpoint{4.509091in}{1.558567in}}%
\pgfpathlineto{\pgfqpoint{4.509174in}{1.558558in}}%
\pgfusepath{stroke}%
\end{pgfscope}%
\begin{pgfscope}%
\pgfpathrectangle{\pgfqpoint{3.352233in}{1.400000in}}{\pgfqpoint{2.407767in}{1.544118in}}%
\pgfusepath{clip}%
\pgfsetbuttcap%
\pgfsetroundjoin%
\pgfsetlinewidth{0.501875pt}%
\definecolor{currentstroke}{rgb}{0.268510,0.009605,0.335427}%
\pgfsetstrokecolor{currentstroke}%
\pgfsetdash{}{0pt}%
\pgfpathmoveto{\pgfqpoint{4.509174in}{1.558558in}}%
\pgfpathlineto{\pgfqpoint{4.509231in}{1.558554in}}%
\pgfusepath{stroke}%
\end{pgfscope}%
\begin{pgfscope}%
\pgfpathrectangle{\pgfqpoint{3.352233in}{1.400000in}}{\pgfqpoint{2.407767in}{1.544118in}}%
\pgfusepath{clip}%
\pgfsetbuttcap%
\pgfsetroundjoin%
\pgfsetlinewidth{0.501875pt}%
\definecolor{currentstroke}{rgb}{0.268510,0.009605,0.335427}%
\pgfsetstrokecolor{currentstroke}%
\pgfsetdash{}{0pt}%
\pgfpathmoveto{\pgfqpoint{4.509231in}{1.558554in}}%
\pgfpathlineto{\pgfqpoint{4.509194in}{1.558559in}}%
\pgfusepath{stroke}%
\end{pgfscope}%
\begin{pgfscope}%
\pgfpathrectangle{\pgfqpoint{3.352233in}{1.400000in}}{\pgfqpoint{2.407767in}{1.544118in}}%
\pgfusepath{clip}%
\pgfsetbuttcap%
\pgfsetroundjoin%
\pgfsetlinewidth{0.501875pt}%
\definecolor{currentstroke}{rgb}{0.268510,0.009605,0.335427}%
\pgfsetstrokecolor{currentstroke}%
\pgfsetdash{}{0pt}%
\pgfpathmoveto{\pgfqpoint{4.509194in}{1.558559in}}%
\pgfpathlineto{\pgfqpoint{4.509110in}{1.558567in}}%
\pgfusepath{stroke}%
\end{pgfscope}%
\begin{pgfscope}%
\pgfpathrectangle{\pgfqpoint{3.352233in}{1.400000in}}{\pgfqpoint{2.407767in}{1.544118in}}%
\pgfusepath{clip}%
\pgfsetbuttcap%
\pgfsetroundjoin%
\pgfsetlinewidth{0.501875pt}%
\definecolor{currentstroke}{rgb}{0.268510,0.009605,0.335427}%
\pgfsetstrokecolor{currentstroke}%
\pgfsetdash{}{0pt}%
\pgfpathmoveto{\pgfqpoint{4.509110in}{1.558567in}}%
\pgfpathlineto{\pgfqpoint{4.509097in}{1.558567in}}%
\pgfusepath{stroke}%
\end{pgfscope}%
\begin{pgfscope}%
\pgfpathrectangle{\pgfqpoint{3.352233in}{1.400000in}}{\pgfqpoint{2.407767in}{1.544118in}}%
\pgfusepath{clip}%
\pgfsetbuttcap%
\pgfsetroundjoin%
\pgfsetlinewidth{0.501875pt}%
\definecolor{currentstroke}{rgb}{0.268510,0.009605,0.335427}%
\pgfsetstrokecolor{currentstroke}%
\pgfsetdash{}{0pt}%
\pgfpathmoveto{\pgfqpoint{4.509097in}{1.558567in}}%
\pgfpathlineto{\pgfqpoint{4.509168in}{1.558559in}}%
\pgfusepath{stroke}%
\end{pgfscope}%
\begin{pgfscope}%
\pgfpathrectangle{\pgfqpoint{3.352233in}{1.400000in}}{\pgfqpoint{2.407767in}{1.544118in}}%
\pgfusepath{clip}%
\pgfsetbuttcap%
\pgfsetroundjoin%
\pgfsetlinewidth{0.501875pt}%
\definecolor{currentstroke}{rgb}{0.268510,0.009605,0.335427}%
\pgfsetstrokecolor{currentstroke}%
\pgfsetdash{}{0pt}%
\pgfpathmoveto{\pgfqpoint{4.509168in}{1.558559in}}%
\pgfpathlineto{\pgfqpoint{4.509222in}{1.558555in}}%
\pgfusepath{stroke}%
\end{pgfscope}%
\begin{pgfscope}%
\pgfpathrectangle{\pgfqpoint{3.352233in}{1.400000in}}{\pgfqpoint{2.407767in}{1.544118in}}%
\pgfusepath{clip}%
\pgfsetbuttcap%
\pgfsetroundjoin%
\pgfsetlinewidth{0.501875pt}%
\definecolor{currentstroke}{rgb}{0.268510,0.009605,0.335427}%
\pgfsetstrokecolor{currentstroke}%
\pgfsetdash{}{0pt}%
\pgfpathmoveto{\pgfqpoint{4.509222in}{1.558555in}}%
\pgfpathlineto{\pgfqpoint{4.509193in}{1.558559in}}%
\pgfusepath{stroke}%
\end{pgfscope}%
\begin{pgfscope}%
\pgfpathrectangle{\pgfqpoint{3.352233in}{1.400000in}}{\pgfqpoint{2.407767in}{1.544118in}}%
\pgfusepath{clip}%
\pgfsetbuttcap%
\pgfsetroundjoin%
\pgfsetlinewidth{0.501875pt}%
\definecolor{currentstroke}{rgb}{0.268510,0.009605,0.335427}%
\pgfsetstrokecolor{currentstroke}%
\pgfsetdash{}{0pt}%
\pgfpathmoveto{\pgfqpoint{4.509193in}{1.558559in}}%
\pgfpathlineto{\pgfqpoint{4.509118in}{1.558567in}}%
\pgfusepath{stroke}%
\end{pgfscope}%
\begin{pgfscope}%
\pgfpathrectangle{\pgfqpoint{3.352233in}{1.400000in}}{\pgfqpoint{2.407767in}{1.544118in}}%
\pgfusepath{clip}%
\pgfsetbuttcap%
\pgfsetroundjoin%
\pgfsetlinewidth{0.501875pt}%
\definecolor{currentstroke}{rgb}{0.268510,0.009605,0.335427}%
\pgfsetstrokecolor{currentstroke}%
\pgfsetdash{}{0pt}%
\pgfpathmoveto{\pgfqpoint{4.509118in}{1.558567in}}%
\pgfpathlineto{\pgfqpoint{4.509103in}{1.558566in}}%
\pgfusepath{stroke}%
\end{pgfscope}%
\begin{pgfscope}%
\pgfpathrectangle{\pgfqpoint{3.352233in}{1.400000in}}{\pgfqpoint{2.407767in}{1.544118in}}%
\pgfusepath{clip}%
\pgfsetbuttcap%
\pgfsetroundjoin%
\pgfsetlinewidth{0.501875pt}%
\definecolor{currentstroke}{rgb}{0.268510,0.009605,0.335427}%
\pgfsetstrokecolor{currentstroke}%
\pgfsetdash{}{0pt}%
\pgfpathmoveto{\pgfqpoint{4.509103in}{1.558566in}}%
\pgfpathlineto{\pgfqpoint{4.509164in}{1.558560in}}%
\pgfusepath{stroke}%
\end{pgfscope}%
\begin{pgfscope}%
\pgfpathrectangle{\pgfqpoint{3.352233in}{1.400000in}}{\pgfqpoint{2.407767in}{1.544118in}}%
\pgfusepath{clip}%
\pgfsetbuttcap%
\pgfsetroundjoin%
\pgfsetlinewidth{0.501875pt}%
\definecolor{currentstroke}{rgb}{0.268510,0.009605,0.335427}%
\pgfsetstrokecolor{currentstroke}%
\pgfsetdash{}{0pt}%
\pgfpathmoveto{\pgfqpoint{4.509164in}{1.558560in}}%
\pgfpathlineto{\pgfqpoint{4.509215in}{1.558556in}}%
\pgfusepath{stroke}%
\end{pgfscope}%
\begin{pgfscope}%
\pgfpathrectangle{\pgfqpoint{3.352233in}{1.400000in}}{\pgfqpoint{2.407767in}{1.544118in}}%
\pgfusepath{clip}%
\pgfsetbuttcap%
\pgfsetroundjoin%
\pgfsetlinewidth{0.501875pt}%
\definecolor{currentstroke}{rgb}{0.268510,0.009605,0.335427}%
\pgfsetstrokecolor{currentstroke}%
\pgfsetdash{}{0pt}%
\pgfpathmoveto{\pgfqpoint{4.509215in}{1.558556in}}%
\pgfpathlineto{\pgfqpoint{4.509192in}{1.558559in}}%
\pgfusepath{stroke}%
\end{pgfscope}%
\begin{pgfscope}%
\pgfpathrectangle{\pgfqpoint{3.352233in}{1.400000in}}{\pgfqpoint{2.407767in}{1.544118in}}%
\pgfusepath{clip}%
\pgfsetbuttcap%
\pgfsetroundjoin%
\pgfsetlinewidth{0.501875pt}%
\definecolor{currentstroke}{rgb}{0.268510,0.009605,0.335427}%
\pgfsetstrokecolor{currentstroke}%
\pgfsetdash{}{0pt}%
\pgfpathmoveto{\pgfqpoint{4.509192in}{1.558559in}}%
\pgfpathlineto{\pgfqpoint{4.509126in}{1.558566in}}%
\pgfusepath{stroke}%
\end{pgfscope}%
\begin{pgfscope}%
\pgfpathrectangle{\pgfqpoint{3.352233in}{1.400000in}}{\pgfqpoint{2.407767in}{1.544118in}}%
\pgfusepath{clip}%
\pgfsetbuttcap%
\pgfsetroundjoin%
\pgfsetlinewidth{0.501875pt}%
\definecolor{currentstroke}{rgb}{0.268510,0.009605,0.335427}%
\pgfsetstrokecolor{currentstroke}%
\pgfsetdash{}{0pt}%
\pgfpathmoveto{\pgfqpoint{4.509126in}{1.558566in}}%
\pgfpathlineto{\pgfqpoint{4.509108in}{1.558566in}}%
\pgfusepath{stroke}%
\end{pgfscope}%
\begin{pgfscope}%
\pgfpathrectangle{\pgfqpoint{3.352233in}{1.400000in}}{\pgfqpoint{2.407767in}{1.544118in}}%
\pgfusepath{clip}%
\pgfsetbuttcap%
\pgfsetroundjoin%
\pgfsetlinewidth{0.501875pt}%
\definecolor{currentstroke}{rgb}{0.268510,0.009605,0.335427}%
\pgfsetstrokecolor{currentstroke}%
\pgfsetdash{}{0pt}%
\pgfpathmoveto{\pgfqpoint{4.509108in}{1.558566in}}%
\pgfpathlineto{\pgfqpoint{4.509160in}{1.558560in}}%
\pgfusepath{stroke}%
\end{pgfscope}%
\begin{pgfscope}%
\pgfpathrectangle{\pgfqpoint{3.352233in}{1.400000in}}{\pgfqpoint{2.407767in}{1.544118in}}%
\pgfusepath{clip}%
\pgfsetbuttcap%
\pgfsetroundjoin%
\pgfsetlinewidth{0.501875pt}%
\definecolor{currentstroke}{rgb}{0.268510,0.009605,0.335427}%
\pgfsetstrokecolor{currentstroke}%
\pgfsetdash{}{0pt}%
\pgfpathmoveto{\pgfqpoint{4.509160in}{1.558560in}}%
\pgfpathlineto{\pgfqpoint{4.509208in}{1.558556in}}%
\pgfusepath{stroke}%
\end{pgfscope}%
\begin{pgfscope}%
\pgfpathrectangle{\pgfqpoint{3.352233in}{1.400000in}}{\pgfqpoint{2.407767in}{1.544118in}}%
\pgfusepath{clip}%
\pgfsetbuttcap%
\pgfsetroundjoin%
\pgfsetlinewidth{0.501875pt}%
\definecolor{currentstroke}{rgb}{0.268510,0.009605,0.335427}%
\pgfsetstrokecolor{currentstroke}%
\pgfsetdash{}{0pt}%
\pgfpathmoveto{\pgfqpoint{4.509208in}{1.558556in}}%
\pgfpathlineto{\pgfqpoint{4.509190in}{1.558559in}}%
\pgfusepath{stroke}%
\end{pgfscope}%
\begin{pgfscope}%
\pgfpathrectangle{\pgfqpoint{3.352233in}{1.400000in}}{\pgfqpoint{2.407767in}{1.544118in}}%
\pgfusepath{clip}%
\pgfsetbuttcap%
\pgfsetroundjoin%
\pgfsetlinewidth{0.501875pt}%
\definecolor{currentstroke}{rgb}{0.268510,0.009605,0.335427}%
\pgfsetstrokecolor{currentstroke}%
\pgfsetdash{}{0pt}%
\pgfpathmoveto{\pgfqpoint{4.509190in}{1.558559in}}%
\pgfpathlineto{\pgfqpoint{4.509132in}{1.558565in}}%
\pgfusepath{stroke}%
\end{pgfscope}%
\begin{pgfscope}%
\pgfpathrectangle{\pgfqpoint{3.352233in}{1.400000in}}{\pgfqpoint{2.407767in}{1.544118in}}%
\pgfusepath{clip}%
\pgfsetbuttcap%
\pgfsetroundjoin%
\pgfsetlinewidth{0.501875pt}%
\definecolor{currentstroke}{rgb}{0.268510,0.009605,0.335427}%
\pgfsetstrokecolor{currentstroke}%
\pgfsetdash{}{0pt}%
\pgfpathmoveto{\pgfqpoint{4.509132in}{1.558565in}}%
\pgfpathlineto{\pgfqpoint{4.509113in}{1.558566in}}%
\pgfusepath{stroke}%
\end{pgfscope}%
\begin{pgfscope}%
\pgfpathrectangle{\pgfqpoint{3.352233in}{1.400000in}}{\pgfqpoint{2.407767in}{1.544118in}}%
\pgfusepath{clip}%
\pgfsetbuttcap%
\pgfsetroundjoin%
\pgfsetlinewidth{0.501875pt}%
\definecolor{currentstroke}{rgb}{0.268510,0.009605,0.335427}%
\pgfsetstrokecolor{currentstroke}%
\pgfsetdash{}{0pt}%
\pgfpathmoveto{\pgfqpoint{4.509113in}{1.558566in}}%
\pgfpathlineto{\pgfqpoint{4.509157in}{1.558561in}}%
\pgfusepath{stroke}%
\end{pgfscope}%
\begin{pgfscope}%
\pgfpathrectangle{\pgfqpoint{3.352233in}{1.400000in}}{\pgfqpoint{2.407767in}{1.544118in}}%
\pgfusepath{clip}%
\pgfsetbuttcap%
\pgfsetroundjoin%
\pgfsetlinewidth{0.501875pt}%
\definecolor{currentstroke}{rgb}{0.268510,0.009605,0.335427}%
\pgfsetstrokecolor{currentstroke}%
\pgfsetdash{}{0pt}%
\pgfpathmoveto{\pgfqpoint{4.509157in}{1.558561in}}%
\pgfpathlineto{\pgfqpoint{4.509201in}{1.558557in}}%
\pgfusepath{stroke}%
\end{pgfscope}%
\begin{pgfscope}%
\pgfpathrectangle{\pgfqpoint{3.352233in}{1.400000in}}{\pgfqpoint{2.407767in}{1.544118in}}%
\pgfusepath{clip}%
\pgfsetbuttcap%
\pgfsetroundjoin%
\pgfsetlinewidth{0.501875pt}%
\definecolor{currentstroke}{rgb}{0.268510,0.009605,0.335427}%
\pgfsetstrokecolor{currentstroke}%
\pgfsetdash{}{0pt}%
\pgfpathmoveto{\pgfqpoint{4.509201in}{1.558557in}}%
\pgfpathlineto{\pgfqpoint{4.509189in}{1.558559in}}%
\pgfusepath{stroke}%
\end{pgfscope}%
\begin{pgfscope}%
\pgfpathrectangle{\pgfqpoint{3.352233in}{1.400000in}}{\pgfqpoint{2.407767in}{1.544118in}}%
\pgfusepath{clip}%
\pgfsetbuttcap%
\pgfsetroundjoin%
\pgfsetlinewidth{0.501875pt}%
\definecolor{currentstroke}{rgb}{0.268510,0.009605,0.335427}%
\pgfsetstrokecolor{currentstroke}%
\pgfsetdash{}{0pt}%
\pgfpathmoveto{\pgfqpoint{4.509189in}{1.558559in}}%
\pgfpathlineto{\pgfqpoint{4.509137in}{1.558564in}}%
\pgfusepath{stroke}%
\end{pgfscope}%
\begin{pgfscope}%
\pgfpathrectangle{\pgfqpoint{3.352233in}{1.400000in}}{\pgfqpoint{2.407767in}{1.544118in}}%
\pgfusepath{clip}%
\pgfsetbuttcap%
\pgfsetroundjoin%
\pgfsetlinewidth{0.501875pt}%
\definecolor{currentstroke}{rgb}{0.268510,0.009605,0.335427}%
\pgfsetstrokecolor{currentstroke}%
\pgfsetdash{}{0pt}%
\pgfpathmoveto{\pgfqpoint{4.509137in}{1.558564in}}%
\pgfpathlineto{\pgfqpoint{4.509117in}{1.558565in}}%
\pgfusepath{stroke}%
\end{pgfscope}%
\begin{pgfscope}%
\pgfpathrectangle{\pgfqpoint{3.352233in}{1.400000in}}{\pgfqpoint{2.407767in}{1.544118in}}%
\pgfusepath{clip}%
\pgfsetbuttcap%
\pgfsetroundjoin%
\pgfsetlinewidth{0.501875pt}%
\definecolor{currentstroke}{rgb}{0.268510,0.009605,0.335427}%
\pgfsetstrokecolor{currentstroke}%
\pgfsetdash{}{0pt}%
\pgfpathmoveto{\pgfqpoint{4.509117in}{1.558565in}}%
\pgfpathlineto{\pgfqpoint{4.509155in}{1.558561in}}%
\pgfusepath{stroke}%
\end{pgfscope}%
\begin{pgfscope}%
\pgfpathrectangle{\pgfqpoint{3.352233in}{1.400000in}}{\pgfqpoint{2.407767in}{1.544118in}}%
\pgfusepath{clip}%
\pgfsetbuttcap%
\pgfsetroundjoin%
\pgfsetlinewidth{0.501875pt}%
\definecolor{currentstroke}{rgb}{0.268510,0.009605,0.335427}%
\pgfsetstrokecolor{currentstroke}%
\pgfsetdash{}{0pt}%
\pgfpathmoveto{\pgfqpoint{4.509155in}{1.558561in}}%
\pgfpathlineto{\pgfqpoint{4.509196in}{1.558557in}}%
\pgfusepath{stroke}%
\end{pgfscope}%
\begin{pgfscope}%
\pgfpathrectangle{\pgfqpoint{3.352233in}{1.400000in}}{\pgfqpoint{2.407767in}{1.544118in}}%
\pgfusepath{clip}%
\pgfsetbuttcap%
\pgfsetroundjoin%
\pgfsetlinewidth{0.501875pt}%
\definecolor{currentstroke}{rgb}{0.268510,0.009605,0.335427}%
\pgfsetstrokecolor{currentstroke}%
\pgfsetdash{}{0pt}%
\pgfpathmoveto{\pgfqpoint{4.509196in}{1.558557in}}%
\pgfpathlineto{\pgfqpoint{4.509187in}{1.558559in}}%
\pgfusepath{stroke}%
\end{pgfscope}%
\begin{pgfscope}%
\pgfpathrectangle{\pgfqpoint{3.352233in}{1.400000in}}{\pgfqpoint{2.407767in}{1.544118in}}%
\pgfusepath{clip}%
\pgfsetbuttcap%
\pgfsetroundjoin%
\pgfsetlinewidth{0.501875pt}%
\definecolor{currentstroke}{rgb}{0.268510,0.009605,0.335427}%
\pgfsetstrokecolor{currentstroke}%
\pgfsetdash{}{0pt}%
\pgfpathmoveto{\pgfqpoint{4.509187in}{1.558559in}}%
\pgfpathlineto{\pgfqpoint{4.509142in}{1.558564in}}%
\pgfusepath{stroke}%
\end{pgfscope}%
\begin{pgfscope}%
\pgfpathrectangle{\pgfqpoint{3.352233in}{1.400000in}}{\pgfqpoint{2.407767in}{1.544118in}}%
\pgfusepath{clip}%
\pgfsetbuttcap%
\pgfsetroundjoin%
\pgfsetlinewidth{0.501875pt}%
\definecolor{currentstroke}{rgb}{0.268510,0.009605,0.335427}%
\pgfsetstrokecolor{currentstroke}%
\pgfsetdash{}{0pt}%
\pgfpathmoveto{\pgfqpoint{4.509142in}{1.558564in}}%
\pgfpathlineto{\pgfqpoint{4.509122in}{1.558565in}}%
\pgfusepath{stroke}%
\end{pgfscope}%
\begin{pgfscope}%
\pgfpathrectangle{\pgfqpoint{3.352233in}{1.400000in}}{\pgfqpoint{2.407767in}{1.544118in}}%
\pgfusepath{clip}%
\pgfsetbuttcap%
\pgfsetroundjoin%
\pgfsetlinewidth{0.501875pt}%
\definecolor{currentstroke}{rgb}{0.268510,0.009605,0.335427}%
\pgfsetstrokecolor{currentstroke}%
\pgfsetdash{}{0pt}%
\pgfpathmoveto{\pgfqpoint{4.509122in}{1.558565in}}%
\pgfpathlineto{\pgfqpoint{4.509153in}{1.558561in}}%
\pgfusepath{stroke}%
\end{pgfscope}%
\begin{pgfscope}%
\pgfpathrectangle{\pgfqpoint{3.352233in}{1.400000in}}{\pgfqpoint{2.407767in}{1.544118in}}%
\pgfusepath{clip}%
\pgfsetbuttcap%
\pgfsetroundjoin%
\pgfsetlinewidth{0.501875pt}%
\definecolor{currentstroke}{rgb}{0.268510,0.009605,0.335427}%
\pgfsetstrokecolor{currentstroke}%
\pgfsetdash{}{0pt}%
\pgfpathmoveto{\pgfqpoint{4.509153in}{1.558561in}}%
\pgfpathlineto{\pgfqpoint{4.509191in}{1.558558in}}%
\pgfusepath{stroke}%
\end{pgfscope}%
\begin{pgfscope}%
\pgfpathrectangle{\pgfqpoint{3.352233in}{1.400000in}}{\pgfqpoint{2.407767in}{1.544118in}}%
\pgfusepath{clip}%
\pgfsetbuttcap%
\pgfsetroundjoin%
\pgfsetlinewidth{0.501875pt}%
\definecolor{currentstroke}{rgb}{0.268510,0.009605,0.335427}%
\pgfsetstrokecolor{currentstroke}%
\pgfsetdash{}{0pt}%
\pgfpathmoveto{\pgfqpoint{4.509191in}{1.558558in}}%
\pgfpathlineto{\pgfqpoint{4.509186in}{1.558559in}}%
\pgfusepath{stroke}%
\end{pgfscope}%
\begin{pgfscope}%
\pgfpathrectangle{\pgfqpoint{3.352233in}{1.400000in}}{\pgfqpoint{2.407767in}{1.544118in}}%
\pgfusepath{clip}%
\pgfsetbuttcap%
\pgfsetroundjoin%
\pgfsetlinewidth{0.501875pt}%
\definecolor{currentstroke}{rgb}{0.268510,0.009605,0.335427}%
\pgfsetstrokecolor{currentstroke}%
\pgfsetdash{}{0pt}%
\pgfpathmoveto{\pgfqpoint{4.509186in}{1.558559in}}%
\pgfpathlineto{\pgfqpoint{4.509146in}{1.558564in}}%
\pgfusepath{stroke}%
\end{pgfscope}%
\begin{pgfscope}%
\pgfpathrectangle{\pgfqpoint{3.352233in}{1.400000in}}{\pgfqpoint{2.407767in}{1.544118in}}%
\pgfusepath{clip}%
\pgfsetbuttcap%
\pgfsetroundjoin%
\pgfsetlinewidth{0.501875pt}%
\definecolor{currentstroke}{rgb}{0.268510,0.009605,0.335427}%
\pgfsetstrokecolor{currentstroke}%
\pgfsetdash{}{0pt}%
\pgfpathmoveto{\pgfqpoint{4.509146in}{1.558564in}}%
\pgfpathlineto{\pgfqpoint{4.509126in}{1.558565in}}%
\pgfusepath{stroke}%
\end{pgfscope}%
\begin{pgfscope}%
\pgfpathrectangle{\pgfqpoint{3.352233in}{1.400000in}}{\pgfqpoint{2.407767in}{1.544118in}}%
\pgfusepath{clip}%
\pgfsetbuttcap%
\pgfsetroundjoin%
\pgfsetlinewidth{0.501875pt}%
\definecolor{currentstroke}{rgb}{0.268510,0.009605,0.335427}%
\pgfsetstrokecolor{currentstroke}%
\pgfsetdash{}{0pt}%
\pgfpathmoveto{\pgfqpoint{4.509126in}{1.558565in}}%
\pgfpathlineto{\pgfqpoint{4.509152in}{1.558561in}}%
\pgfusepath{stroke}%
\end{pgfscope}%
\begin{pgfscope}%
\pgfpathrectangle{\pgfqpoint{3.352233in}{1.400000in}}{\pgfqpoint{2.407767in}{1.544118in}}%
\pgfusepath{clip}%
\pgfsetbuttcap%
\pgfsetroundjoin%
\pgfsetlinewidth{0.501875pt}%
\definecolor{currentstroke}{rgb}{0.268510,0.009605,0.335427}%
\pgfsetstrokecolor{currentstroke}%
\pgfsetdash{}{0pt}%
\pgfpathmoveto{\pgfqpoint{4.509152in}{1.558561in}}%
\pgfpathlineto{\pgfqpoint{4.509187in}{1.558558in}}%
\pgfusepath{stroke}%
\end{pgfscope}%
\begin{pgfscope}%
\pgfpathrectangle{\pgfqpoint{3.352233in}{1.400000in}}{\pgfqpoint{2.407767in}{1.544118in}}%
\pgfusepath{clip}%
\pgfsetbuttcap%
\pgfsetroundjoin%
\pgfsetlinewidth{0.501875pt}%
\definecolor{currentstroke}{rgb}{0.268510,0.009605,0.335427}%
\pgfsetstrokecolor{currentstroke}%
\pgfsetdash{}{0pt}%
\pgfpathmoveto{\pgfqpoint{4.509187in}{1.558558in}}%
\pgfpathlineto{\pgfqpoint{4.509184in}{1.558559in}}%
\pgfusepath{stroke}%
\end{pgfscope}%
\begin{pgfscope}%
\pgfpathrectangle{\pgfqpoint{3.352233in}{1.400000in}}{\pgfqpoint{2.407767in}{1.544118in}}%
\pgfusepath{clip}%
\pgfsetbuttcap%
\pgfsetroundjoin%
\pgfsetlinewidth{0.501875pt}%
\definecolor{currentstroke}{rgb}{0.268510,0.009605,0.335427}%
\pgfsetstrokecolor{currentstroke}%
\pgfsetdash{}{0pt}%
\pgfpathmoveto{\pgfqpoint{4.509184in}{1.558559in}}%
\pgfpathlineto{\pgfqpoint{4.509149in}{1.558563in}}%
\pgfusepath{stroke}%
\end{pgfscope}%
\begin{pgfscope}%
\pgfpathrectangle{\pgfqpoint{3.352233in}{1.400000in}}{\pgfqpoint{2.407767in}{1.544118in}}%
\pgfusepath{clip}%
\pgfsetbuttcap%
\pgfsetroundjoin%
\pgfsetlinewidth{0.501875pt}%
\definecolor{currentstroke}{rgb}{0.268510,0.009605,0.335427}%
\pgfsetstrokecolor{currentstroke}%
\pgfsetdash{}{0pt}%
\pgfpathmoveto{\pgfqpoint{4.509149in}{1.558563in}}%
\pgfpathlineto{\pgfqpoint{4.509129in}{1.558564in}}%
\pgfusepath{stroke}%
\end{pgfscope}%
\begin{pgfscope}%
\pgfpathrectangle{\pgfqpoint{3.352233in}{1.400000in}}{\pgfqpoint{2.407767in}{1.544118in}}%
\pgfusepath{clip}%
\pgfsetbuttcap%
\pgfsetroundjoin%
\pgfsetlinewidth{0.501875pt}%
\definecolor{currentstroke}{rgb}{0.268510,0.009605,0.335427}%
\pgfsetstrokecolor{currentstroke}%
\pgfsetdash{}{0pt}%
\pgfpathmoveto{\pgfqpoint{4.509129in}{1.558564in}}%
\pgfpathlineto{\pgfqpoint{4.509151in}{1.558562in}}%
\pgfusepath{stroke}%
\end{pgfscope}%
\begin{pgfscope}%
\pgfpathrectangle{\pgfqpoint{3.352233in}{1.400000in}}{\pgfqpoint{2.407767in}{1.544118in}}%
\pgfusepath{clip}%
\pgfsetbuttcap%
\pgfsetroundjoin%
\pgfsetlinewidth{0.501875pt}%
\definecolor{currentstroke}{rgb}{0.268510,0.009605,0.335427}%
\pgfsetstrokecolor{currentstroke}%
\pgfsetdash{}{0pt}%
\pgfpathmoveto{\pgfqpoint{4.509151in}{1.558562in}}%
\pgfpathlineto{\pgfqpoint{4.509183in}{1.558559in}}%
\pgfusepath{stroke}%
\end{pgfscope}%
\begin{pgfscope}%
\pgfpathrectangle{\pgfqpoint{3.352233in}{1.400000in}}{\pgfqpoint{2.407767in}{1.544118in}}%
\pgfusepath{clip}%
\pgfsetbuttcap%
\pgfsetroundjoin%
\pgfsetlinewidth{0.501875pt}%
\definecolor{currentstroke}{rgb}{0.268510,0.009605,0.335427}%
\pgfsetstrokecolor{currentstroke}%
\pgfsetdash{}{0pt}%
\pgfpathmoveto{\pgfqpoint{4.509183in}{1.558559in}}%
\pgfpathlineto{\pgfqpoint{4.509182in}{1.558560in}}%
\pgfusepath{stroke}%
\end{pgfscope}%
\begin{pgfscope}%
\pgfpathrectangle{\pgfqpoint{3.352233in}{1.400000in}}{\pgfqpoint{2.407767in}{1.544118in}}%
\pgfusepath{clip}%
\pgfsetbuttcap%
\pgfsetroundjoin%
\pgfsetlinewidth{0.501875pt}%
\definecolor{currentstroke}{rgb}{0.268510,0.009605,0.335427}%
\pgfsetstrokecolor{currentstroke}%
\pgfsetdash{}{0pt}%
\pgfpathmoveto{\pgfqpoint{4.509182in}{1.558560in}}%
\pgfpathlineto{\pgfqpoint{4.509152in}{1.558563in}}%
\pgfusepath{stroke}%
\end{pgfscope}%
\begin{pgfscope}%
\pgfpathrectangle{\pgfqpoint{3.352233in}{1.400000in}}{\pgfqpoint{2.407767in}{1.544118in}}%
\pgfusepath{clip}%
\pgfsetbuttcap%
\pgfsetroundjoin%
\pgfsetlinewidth{0.501875pt}%
\definecolor{currentstroke}{rgb}{0.268510,0.009605,0.335427}%
\pgfsetstrokecolor{currentstroke}%
\pgfsetdash{}{0pt}%
\pgfpathmoveto{\pgfqpoint{4.509152in}{1.558563in}}%
\pgfpathlineto{\pgfqpoint{4.509133in}{1.558564in}}%
\pgfusepath{stroke}%
\end{pgfscope}%
\begin{pgfscope}%
\pgfpathrectangle{\pgfqpoint{3.352233in}{1.400000in}}{\pgfqpoint{2.407767in}{1.544118in}}%
\pgfusepath{clip}%
\pgfsetbuttcap%
\pgfsetroundjoin%
\pgfsetlinewidth{0.501875pt}%
\definecolor{currentstroke}{rgb}{0.268510,0.009605,0.335427}%
\pgfsetstrokecolor{currentstroke}%
\pgfsetdash{}{0pt}%
\pgfpathmoveto{\pgfqpoint{4.509133in}{1.558564in}}%
\pgfpathlineto{\pgfqpoint{4.509151in}{1.558562in}}%
\pgfusepath{stroke}%
\end{pgfscope}%
\begin{pgfscope}%
\pgfpathrectangle{\pgfqpoint{3.352233in}{1.400000in}}{\pgfqpoint{2.407767in}{1.544118in}}%
\pgfusepath{clip}%
\pgfsetbuttcap%
\pgfsetroundjoin%
\pgfsetlinewidth{0.501875pt}%
\definecolor{currentstroke}{rgb}{0.268510,0.009605,0.335427}%
\pgfsetstrokecolor{currentstroke}%
\pgfsetdash{}{0pt}%
\pgfpathmoveto{\pgfqpoint{4.509151in}{1.558562in}}%
\pgfpathlineto{\pgfqpoint{4.509179in}{1.558559in}}%
\pgfusepath{stroke}%
\end{pgfscope}%
\begin{pgfscope}%
\pgfpathrectangle{\pgfqpoint{3.352233in}{1.400000in}}{\pgfqpoint{2.407767in}{1.544118in}}%
\pgfusepath{clip}%
\pgfsetbuttcap%
\pgfsetroundjoin%
\pgfsetlinewidth{0.501875pt}%
\definecolor{currentstroke}{rgb}{0.268510,0.009605,0.335427}%
\pgfsetstrokecolor{currentstroke}%
\pgfsetdash{}{0pt}%
\pgfpathmoveto{\pgfqpoint{4.509179in}{1.558559in}}%
\pgfpathlineto{\pgfqpoint{4.509181in}{1.558560in}}%
\pgfusepath{stroke}%
\end{pgfscope}%
\begin{pgfscope}%
\pgfpathrectangle{\pgfqpoint{3.352233in}{1.400000in}}{\pgfqpoint{2.407767in}{1.544118in}}%
\pgfusepath{clip}%
\pgfsetbuttcap%
\pgfsetroundjoin%
\pgfsetlinewidth{0.501875pt}%
\definecolor{currentstroke}{rgb}{0.268510,0.009605,0.335427}%
\pgfsetstrokecolor{currentstroke}%
\pgfsetdash{}{0pt}%
\pgfpathmoveto{\pgfqpoint{4.509181in}{1.558560in}}%
\pgfpathlineto{\pgfqpoint{4.509154in}{1.558563in}}%
\pgfusepath{stroke}%
\end{pgfscope}%
\begin{pgfscope}%
\pgfpathrectangle{\pgfqpoint{3.352233in}{1.400000in}}{\pgfqpoint{2.407767in}{1.544118in}}%
\pgfusepath{clip}%
\pgfsetbuttcap%
\pgfsetroundjoin%
\pgfsetlinewidth{0.501875pt}%
\definecolor{currentstroke}{rgb}{0.268510,0.009605,0.335427}%
\pgfsetstrokecolor{currentstroke}%
\pgfsetdash{}{0pt}%
\pgfpathmoveto{\pgfqpoint{4.509154in}{1.558563in}}%
\pgfpathlineto{\pgfqpoint{4.509136in}{1.558564in}}%
\pgfusepath{stroke}%
\end{pgfscope}%
\begin{pgfscope}%
\pgfpathrectangle{\pgfqpoint{3.352233in}{1.400000in}}{\pgfqpoint{2.407767in}{1.544118in}}%
\pgfusepath{clip}%
\pgfsetbuttcap%
\pgfsetroundjoin%
\pgfsetlinewidth{0.501875pt}%
\definecolor{currentstroke}{rgb}{0.268510,0.009605,0.335427}%
\pgfsetstrokecolor{currentstroke}%
\pgfsetdash{}{0pt}%
\pgfpathmoveto{\pgfqpoint{4.509136in}{1.558564in}}%
\pgfpathlineto{\pgfqpoint{4.509150in}{1.558562in}}%
\pgfusepath{stroke}%
\end{pgfscope}%
\begin{pgfscope}%
\pgfpathrectangle{\pgfqpoint{3.352233in}{1.400000in}}{\pgfqpoint{2.407767in}{1.544118in}}%
\pgfusepath{clip}%
\pgfsetbuttcap%
\pgfsetroundjoin%
\pgfsetlinewidth{0.501875pt}%
\definecolor{currentstroke}{rgb}{0.268510,0.009605,0.335427}%
\pgfsetstrokecolor{currentstroke}%
\pgfsetdash{}{0pt}%
\pgfpathmoveto{\pgfqpoint{4.509150in}{1.558562in}}%
\pgfpathlineto{\pgfqpoint{4.509176in}{1.558560in}}%
\pgfusepath{stroke}%
\end{pgfscope}%
\begin{pgfscope}%
\pgfpathrectangle{\pgfqpoint{3.352233in}{1.400000in}}{\pgfqpoint{2.407767in}{1.544118in}}%
\pgfusepath{clip}%
\pgfsetbuttcap%
\pgfsetroundjoin%
\pgfsetlinewidth{0.501875pt}%
\definecolor{currentstroke}{rgb}{0.268510,0.009605,0.335427}%
\pgfsetstrokecolor{currentstroke}%
\pgfsetdash{}{0pt}%
\pgfpathmoveto{\pgfqpoint{4.509176in}{1.558560in}}%
\pgfpathlineto{\pgfqpoint{4.509179in}{1.558560in}}%
\pgfusepath{stroke}%
\end{pgfscope}%
\begin{pgfscope}%
\pgfpathrectangle{\pgfqpoint{3.352233in}{1.400000in}}{\pgfqpoint{2.407767in}{1.544118in}}%
\pgfusepath{clip}%
\pgfsetbuttcap%
\pgfsetroundjoin%
\pgfsetlinewidth{0.501875pt}%
\definecolor{currentstroke}{rgb}{0.268510,0.009605,0.335427}%
\pgfsetstrokecolor{currentstroke}%
\pgfsetdash{}{0pt}%
\pgfpathmoveto{\pgfqpoint{4.509179in}{1.558560in}}%
\pgfpathlineto{\pgfqpoint{4.509156in}{1.558562in}}%
\pgfusepath{stroke}%
\end{pgfscope}%
\begin{pgfscope}%
\pgfpathrectangle{\pgfqpoint{3.352233in}{1.400000in}}{\pgfqpoint{2.407767in}{1.544118in}}%
\pgfusepath{clip}%
\pgfsetbuttcap%
\pgfsetroundjoin%
\pgfsetlinewidth{0.501875pt}%
\definecolor{currentstroke}{rgb}{0.268510,0.009605,0.335427}%
\pgfsetstrokecolor{currentstroke}%
\pgfsetdash{}{0pt}%
\pgfpathmoveto{\pgfqpoint{4.509156in}{1.558562in}}%
\pgfpathlineto{\pgfqpoint{4.509139in}{1.558564in}}%
\pgfusepath{stroke}%
\end{pgfscope}%
\begin{pgfscope}%
\pgfpathrectangle{\pgfqpoint{3.352233in}{1.400000in}}{\pgfqpoint{2.407767in}{1.544118in}}%
\pgfusepath{clip}%
\pgfsetbuttcap%
\pgfsetroundjoin%
\pgfsetlinewidth{0.501875pt}%
\definecolor{currentstroke}{rgb}{0.268510,0.009605,0.335427}%
\pgfsetstrokecolor{currentstroke}%
\pgfsetdash{}{0pt}%
\pgfpathmoveto{\pgfqpoint{4.509139in}{1.558564in}}%
\pgfpathlineto{\pgfqpoint{4.509150in}{1.558562in}}%
\pgfusepath{stroke}%
\end{pgfscope}%
\begin{pgfscope}%
\pgfpathrectangle{\pgfqpoint{3.352233in}{1.400000in}}{\pgfqpoint{2.407767in}{1.544118in}}%
\pgfusepath{clip}%
\pgfsetbuttcap%
\pgfsetroundjoin%
\pgfsetlinewidth{0.501875pt}%
\definecolor{currentstroke}{rgb}{0.268510,0.009605,0.335427}%
\pgfsetstrokecolor{currentstroke}%
\pgfsetdash{}{0pt}%
\pgfpathmoveto{\pgfqpoint{4.509150in}{1.558562in}}%
\pgfpathlineto{\pgfqpoint{4.509173in}{1.558560in}}%
\pgfusepath{stroke}%
\end{pgfscope}%
\begin{pgfscope}%
\pgfpathrectangle{\pgfqpoint{3.352233in}{1.400000in}}{\pgfqpoint{2.407767in}{1.544118in}}%
\pgfusepath{clip}%
\pgfsetbuttcap%
\pgfsetroundjoin%
\pgfsetlinewidth{0.501875pt}%
\definecolor{currentstroke}{rgb}{0.268510,0.009605,0.335427}%
\pgfsetstrokecolor{currentstroke}%
\pgfsetdash{}{0pt}%
\pgfpathmoveto{\pgfqpoint{4.509173in}{1.558560in}}%
\pgfpathlineto{\pgfqpoint{4.509177in}{1.558560in}}%
\pgfusepath{stroke}%
\end{pgfscope}%
\begin{pgfscope}%
\pgfpathrectangle{\pgfqpoint{3.352233in}{1.400000in}}{\pgfqpoint{2.407767in}{1.544118in}}%
\pgfusepath{clip}%
\pgfsetbuttcap%
\pgfsetroundjoin%
\pgfsetlinewidth{0.501875pt}%
\definecolor{currentstroke}{rgb}{0.268510,0.009605,0.335427}%
\pgfsetstrokecolor{currentstroke}%
\pgfsetdash{}{0pt}%
\pgfpathmoveto{\pgfqpoint{4.509177in}{1.558560in}}%
\pgfpathlineto{\pgfqpoint{4.509157in}{1.558562in}}%
\pgfusepath{stroke}%
\end{pgfscope}%
\begin{pgfscope}%
\pgfpathrectangle{\pgfqpoint{3.352233in}{1.400000in}}{\pgfqpoint{2.407767in}{1.544118in}}%
\pgfusepath{clip}%
\pgfsetbuttcap%
\pgfsetroundjoin%
\pgfsetlinewidth{0.501875pt}%
\definecolor{currentstroke}{rgb}{0.268510,0.009605,0.335427}%
\pgfsetstrokecolor{currentstroke}%
\pgfsetdash{}{0pt}%
\pgfpathmoveto{\pgfqpoint{4.509157in}{1.558562in}}%
\pgfpathlineto{\pgfqpoint{4.509141in}{1.558563in}}%
\pgfusepath{stroke}%
\end{pgfscope}%
\begin{pgfscope}%
\pgfpathrectangle{\pgfqpoint{3.352233in}{1.400000in}}{\pgfqpoint{2.407767in}{1.544118in}}%
\pgfusepath{clip}%
\pgfsetbuttcap%
\pgfsetroundjoin%
\pgfsetlinewidth{0.501875pt}%
\definecolor{currentstroke}{rgb}{0.268510,0.009605,0.335427}%
\pgfsetstrokecolor{currentstroke}%
\pgfsetdash{}{0pt}%
\pgfpathmoveto{\pgfqpoint{4.509141in}{1.558563in}}%
\pgfpathlineto{\pgfqpoint{4.509150in}{1.558562in}}%
\pgfusepath{stroke}%
\end{pgfscope}%
\begin{pgfscope}%
\pgfpathrectangle{\pgfqpoint{3.352233in}{1.400000in}}{\pgfqpoint{2.407767in}{1.544118in}}%
\pgfusepath{clip}%
\pgfsetbuttcap%
\pgfsetroundjoin%
\pgfsetlinewidth{0.501875pt}%
\definecolor{currentstroke}{rgb}{0.268510,0.009605,0.335427}%
\pgfsetstrokecolor{currentstroke}%
\pgfsetdash{}{0pt}%
\pgfpathmoveto{\pgfqpoint{4.509150in}{1.558562in}}%
\pgfpathlineto{\pgfqpoint{4.509171in}{1.558560in}}%
\pgfusepath{stroke}%
\end{pgfscope}%
\begin{pgfscope}%
\pgfpathrectangle{\pgfqpoint{3.352233in}{1.400000in}}{\pgfqpoint{2.407767in}{1.544118in}}%
\pgfusepath{clip}%
\pgfsetbuttcap%
\pgfsetroundjoin%
\pgfsetlinewidth{0.501875pt}%
\definecolor{currentstroke}{rgb}{0.268510,0.009605,0.335427}%
\pgfsetstrokecolor{currentstroke}%
\pgfsetdash{}{0pt}%
\pgfpathmoveto{\pgfqpoint{4.509171in}{1.558560in}}%
\pgfpathlineto{\pgfqpoint{4.509176in}{1.558560in}}%
\pgfusepath{stroke}%
\end{pgfscope}%
\begin{pgfscope}%
\pgfpathrectangle{\pgfqpoint{3.352233in}{1.400000in}}{\pgfqpoint{2.407767in}{1.544118in}}%
\pgfusepath{clip}%
\pgfsetbuttcap%
\pgfsetroundjoin%
\pgfsetlinewidth{0.501875pt}%
\definecolor{currentstroke}{rgb}{0.268510,0.009605,0.335427}%
\pgfsetstrokecolor{currentstroke}%
\pgfsetdash{}{0pt}%
\pgfpathmoveto{\pgfqpoint{4.509176in}{1.558560in}}%
\pgfpathlineto{\pgfqpoint{4.509159in}{1.558562in}}%
\pgfusepath{stroke}%
\end{pgfscope}%
\begin{pgfscope}%
\pgfpathrectangle{\pgfqpoint{3.352233in}{1.400000in}}{\pgfqpoint{2.407767in}{1.544118in}}%
\pgfusepath{clip}%
\pgfsetbuttcap%
\pgfsetroundjoin%
\pgfsetlinewidth{0.501875pt}%
\definecolor{currentstroke}{rgb}{0.268510,0.009605,0.335427}%
\pgfsetstrokecolor{currentstroke}%
\pgfsetdash{}{0pt}%
\pgfpathmoveto{\pgfqpoint{4.509159in}{1.558562in}}%
\pgfpathlineto{\pgfqpoint{4.509143in}{1.558563in}}%
\pgfusepath{stroke}%
\end{pgfscope}%
\begin{pgfscope}%
\pgfpathrectangle{\pgfqpoint{3.352233in}{1.400000in}}{\pgfqpoint{2.407767in}{1.544118in}}%
\pgfusepath{clip}%
\pgfsetbuttcap%
\pgfsetroundjoin%
\pgfsetlinewidth{0.501875pt}%
\definecolor{currentstroke}{rgb}{0.268510,0.009605,0.335427}%
\pgfsetstrokecolor{currentstroke}%
\pgfsetdash{}{0pt}%
\pgfpathmoveto{\pgfqpoint{4.509143in}{1.558563in}}%
\pgfpathlineto{\pgfqpoint{4.509150in}{1.558562in}}%
\pgfusepath{stroke}%
\end{pgfscope}%
\begin{pgfscope}%
\pgfpathrectangle{\pgfqpoint{3.352233in}{1.400000in}}{\pgfqpoint{2.407767in}{1.544118in}}%
\pgfusepath{clip}%
\pgfsetbuttcap%
\pgfsetroundjoin%
\pgfsetlinewidth{0.501875pt}%
\definecolor{currentstroke}{rgb}{0.268510,0.009605,0.335427}%
\pgfsetstrokecolor{currentstroke}%
\pgfsetdash{}{0pt}%
\pgfpathmoveto{\pgfqpoint{4.509150in}{1.558562in}}%
\pgfpathlineto{\pgfqpoint{4.509169in}{1.558560in}}%
\pgfusepath{stroke}%
\end{pgfscope}%
\begin{pgfscope}%
\pgfpathrectangle{\pgfqpoint{3.352233in}{1.400000in}}{\pgfqpoint{2.407767in}{1.544118in}}%
\pgfusepath{clip}%
\pgfsetbuttcap%
\pgfsetroundjoin%
\pgfsetlinewidth{0.501875pt}%
\definecolor{currentstroke}{rgb}{0.268510,0.009605,0.335427}%
\pgfsetstrokecolor{currentstroke}%
\pgfsetdash{}{0pt}%
\pgfpathmoveto{\pgfqpoint{4.509169in}{1.558560in}}%
\pgfpathlineto{\pgfqpoint{4.509174in}{1.558560in}}%
\pgfusepath{stroke}%
\end{pgfscope}%
\begin{pgfscope}%
\pgfpathrectangle{\pgfqpoint{3.352233in}{1.400000in}}{\pgfqpoint{2.407767in}{1.544118in}}%
\pgfusepath{clip}%
\pgfsetbuttcap%
\pgfsetroundjoin%
\pgfsetlinewidth{0.501875pt}%
\definecolor{currentstroke}{rgb}{0.268510,0.009605,0.335427}%
\pgfsetstrokecolor{currentstroke}%
\pgfsetdash{}{0pt}%
\pgfpathmoveto{\pgfqpoint{4.509174in}{1.558560in}}%
\pgfpathlineto{\pgfqpoint{4.509160in}{1.558562in}}%
\pgfusepath{stroke}%
\end{pgfscope}%
\begin{pgfscope}%
\pgfpathrectangle{\pgfqpoint{3.352233in}{1.400000in}}{\pgfqpoint{2.407767in}{1.544118in}}%
\pgfusepath{clip}%
\pgfsetbuttcap%
\pgfsetroundjoin%
\pgfsetlinewidth{0.501875pt}%
\definecolor{currentstroke}{rgb}{0.268510,0.009605,0.335427}%
\pgfsetstrokecolor{currentstroke}%
\pgfsetdash{}{0pt}%
\pgfpathmoveto{\pgfqpoint{4.509160in}{1.558562in}}%
\pgfpathlineto{\pgfqpoint{4.509145in}{1.558563in}}%
\pgfusepath{stroke}%
\end{pgfscope}%
\begin{pgfscope}%
\pgfpathrectangle{\pgfqpoint{3.352233in}{1.400000in}}{\pgfqpoint{2.407767in}{1.544118in}}%
\pgfusepath{clip}%
\pgfsetbuttcap%
\pgfsetroundjoin%
\pgfsetlinewidth{0.501875pt}%
\definecolor{currentstroke}{rgb}{0.268510,0.009605,0.335427}%
\pgfsetstrokecolor{currentstroke}%
\pgfsetdash{}{0pt}%
\pgfpathmoveto{\pgfqpoint{4.509145in}{1.558563in}}%
\pgfpathlineto{\pgfqpoint{4.509151in}{1.558562in}}%
\pgfusepath{stroke}%
\end{pgfscope}%
\begin{pgfscope}%
\pgfpathrectangle{\pgfqpoint{3.352233in}{1.400000in}}{\pgfqpoint{2.407767in}{1.544118in}}%
\pgfusepath{clip}%
\pgfsetbuttcap%
\pgfsetroundjoin%
\pgfsetlinewidth{0.501875pt}%
\definecolor{currentstroke}{rgb}{0.268510,0.009605,0.335427}%
\pgfsetstrokecolor{currentstroke}%
\pgfsetdash{}{0pt}%
\pgfpathmoveto{\pgfqpoint{4.509151in}{1.558562in}}%
\pgfpathlineto{\pgfqpoint{4.509167in}{1.558560in}}%
\pgfusepath{stroke}%
\end{pgfscope}%
\begin{pgfscope}%
\pgfpathrectangle{\pgfqpoint{3.352233in}{1.400000in}}{\pgfqpoint{2.407767in}{1.544118in}}%
\pgfusepath{clip}%
\pgfsetbuttcap%
\pgfsetroundjoin%
\pgfsetlinewidth{0.501875pt}%
\definecolor{currentstroke}{rgb}{0.268510,0.009605,0.335427}%
\pgfsetstrokecolor{currentstroke}%
\pgfsetdash{}{0pt}%
\pgfpathmoveto{\pgfqpoint{4.509167in}{1.558560in}}%
\pgfpathlineto{\pgfqpoint{4.509173in}{1.558560in}}%
\pgfusepath{stroke}%
\end{pgfscope}%
\begin{pgfscope}%
\pgfpathrectangle{\pgfqpoint{3.352233in}{1.400000in}}{\pgfqpoint{2.407767in}{1.544118in}}%
\pgfusepath{clip}%
\pgfsetbuttcap%
\pgfsetroundjoin%
\pgfsetlinewidth{0.501875pt}%
\definecolor{currentstroke}{rgb}{0.268510,0.009605,0.335427}%
\pgfsetstrokecolor{currentstroke}%
\pgfsetdash{}{0pt}%
\pgfpathmoveto{\pgfqpoint{4.509173in}{1.558560in}}%
\pgfpathlineto{\pgfqpoint{4.509161in}{1.558562in}}%
\pgfusepath{stroke}%
\end{pgfscope}%
\begin{pgfscope}%
\pgfpathrectangle{\pgfqpoint{3.352233in}{1.400000in}}{\pgfqpoint{2.407767in}{1.544118in}}%
\pgfusepath{clip}%
\pgfsetbuttcap%
\pgfsetroundjoin%
\pgfsetlinewidth{0.501875pt}%
\definecolor{currentstroke}{rgb}{0.268510,0.009605,0.335427}%
\pgfsetstrokecolor{currentstroke}%
\pgfsetdash{}{0pt}%
\pgfpathmoveto{\pgfqpoint{4.509161in}{1.558562in}}%
\pgfpathlineto{\pgfqpoint{4.509147in}{1.558563in}}%
\pgfusepath{stroke}%
\end{pgfscope}%
\begin{pgfscope}%
\pgfpathrectangle{\pgfqpoint{3.352233in}{1.400000in}}{\pgfqpoint{2.407767in}{1.544118in}}%
\pgfusepath{clip}%
\pgfsetbuttcap%
\pgfsetroundjoin%
\pgfsetlinewidth{0.501875pt}%
\definecolor{currentstroke}{rgb}{0.268510,0.009605,0.335427}%
\pgfsetstrokecolor{currentstroke}%
\pgfsetdash{}{0pt}%
\pgfpathmoveto{\pgfqpoint{4.509147in}{1.558563in}}%
\pgfpathlineto{\pgfqpoint{4.509151in}{1.558562in}}%
\pgfusepath{stroke}%
\end{pgfscope}%
\begin{pgfscope}%
\pgfpathrectangle{\pgfqpoint{3.352233in}{1.400000in}}{\pgfqpoint{2.407767in}{1.544118in}}%
\pgfusepath{clip}%
\pgfsetbuttcap%
\pgfsetroundjoin%
\pgfsetlinewidth{0.501875pt}%
\definecolor{currentstroke}{rgb}{0.268510,0.009605,0.335427}%
\pgfsetstrokecolor{currentstroke}%
\pgfsetdash{}{0pt}%
\pgfpathmoveto{\pgfqpoint{4.509151in}{1.558562in}}%
\pgfpathlineto{\pgfqpoint{4.509166in}{1.558561in}}%
\pgfusepath{stroke}%
\end{pgfscope}%
\begin{pgfscope}%
\pgfpathrectangle{\pgfqpoint{3.352233in}{1.400000in}}{\pgfqpoint{2.407767in}{1.544118in}}%
\pgfusepath{clip}%
\pgfsetbuttcap%
\pgfsetroundjoin%
\pgfsetlinewidth{0.501875pt}%
\definecolor{currentstroke}{rgb}{0.268510,0.009605,0.335427}%
\pgfsetstrokecolor{currentstroke}%
\pgfsetdash{}{0pt}%
\pgfpathmoveto{\pgfqpoint{4.509166in}{1.558561in}}%
\pgfpathlineto{\pgfqpoint{4.509172in}{1.558560in}}%
\pgfusepath{stroke}%
\end{pgfscope}%
\begin{pgfscope}%
\pgfpathrectangle{\pgfqpoint{3.352233in}{1.400000in}}{\pgfqpoint{2.407767in}{1.544118in}}%
\pgfusepath{clip}%
\pgfsetbuttcap%
\pgfsetroundjoin%
\pgfsetlinewidth{0.501875pt}%
\definecolor{currentstroke}{rgb}{0.268510,0.009605,0.335427}%
\pgfsetstrokecolor{currentstroke}%
\pgfsetdash{}{0pt}%
\pgfpathmoveto{\pgfqpoint{4.509172in}{1.558560in}}%
\pgfpathlineto{\pgfqpoint{4.509161in}{1.558562in}}%
\pgfusepath{stroke}%
\end{pgfscope}%
\begin{pgfscope}%
\pgfpathrectangle{\pgfqpoint{3.352233in}{1.400000in}}{\pgfqpoint{2.407767in}{1.544118in}}%
\pgfusepath{clip}%
\pgfsetbuttcap%
\pgfsetroundjoin%
\pgfsetlinewidth{0.501875pt}%
\definecolor{currentstroke}{rgb}{0.268510,0.009605,0.335427}%
\pgfsetstrokecolor{currentstroke}%
\pgfsetdash{}{0pt}%
\pgfpathmoveto{\pgfqpoint{4.509161in}{1.558562in}}%
\pgfpathlineto{\pgfqpoint{4.509149in}{1.558563in}}%
\pgfusepath{stroke}%
\end{pgfscope}%
\begin{pgfscope}%
\pgfpathrectangle{\pgfqpoint{3.352233in}{1.400000in}}{\pgfqpoint{2.407767in}{1.544118in}}%
\pgfusepath{clip}%
\pgfsetbuttcap%
\pgfsetroundjoin%
\pgfsetlinewidth{0.501875pt}%
\definecolor{currentstroke}{rgb}{0.268510,0.009605,0.335427}%
\pgfsetstrokecolor{currentstroke}%
\pgfsetdash{}{0pt}%
\pgfpathmoveto{\pgfqpoint{4.509149in}{1.558563in}}%
\pgfpathlineto{\pgfqpoint{4.509151in}{1.558562in}}%
\pgfusepath{stroke}%
\end{pgfscope}%
\begin{pgfscope}%
\pgfpathrectangle{\pgfqpoint{3.352233in}{1.400000in}}{\pgfqpoint{2.407767in}{1.544118in}}%
\pgfusepath{clip}%
\pgfsetbuttcap%
\pgfsetroundjoin%
\pgfsetlinewidth{0.501875pt}%
\definecolor{currentstroke}{rgb}{0.268510,0.009605,0.335427}%
\pgfsetstrokecolor{currentstroke}%
\pgfsetdash{}{0pt}%
\pgfpathmoveto{\pgfqpoint{4.509151in}{1.558562in}}%
\pgfpathlineto{\pgfqpoint{4.509165in}{1.558561in}}%
\pgfusepath{stroke}%
\end{pgfscope}%
\begin{pgfscope}%
\pgfpathrectangle{\pgfqpoint{3.352233in}{1.400000in}}{\pgfqpoint{2.407767in}{1.544118in}}%
\pgfusepath{clip}%
\pgfsetbuttcap%
\pgfsetroundjoin%
\pgfsetlinewidth{0.501875pt}%
\definecolor{currentstroke}{rgb}{0.268510,0.009605,0.335427}%
\pgfsetstrokecolor{currentstroke}%
\pgfsetdash{}{0pt}%
\pgfpathmoveto{\pgfqpoint{4.509165in}{1.558561in}}%
\pgfpathlineto{\pgfqpoint{4.509170in}{1.558560in}}%
\pgfusepath{stroke}%
\end{pgfscope}%
\begin{pgfscope}%
\pgfpathrectangle{\pgfqpoint{3.352233in}{1.400000in}}{\pgfqpoint{2.407767in}{1.544118in}}%
\pgfusepath{clip}%
\pgfsetbuttcap%
\pgfsetroundjoin%
\pgfsetlinewidth{0.501875pt}%
\definecolor{currentstroke}{rgb}{0.268510,0.009605,0.335427}%
\pgfsetstrokecolor{currentstroke}%
\pgfsetdash{}{0pt}%
\pgfpathmoveto{\pgfqpoint{4.509170in}{1.558560in}}%
\pgfpathlineto{\pgfqpoint{4.509162in}{1.558562in}}%
\pgfusepath{stroke}%
\end{pgfscope}%
\begin{pgfscope}%
\pgfpathrectangle{\pgfqpoint{3.352233in}{1.400000in}}{\pgfqpoint{2.407767in}{1.544118in}}%
\pgfusepath{clip}%
\pgfsetbuttcap%
\pgfsetroundjoin%
\pgfsetlinewidth{0.501875pt}%
\definecolor{currentstroke}{rgb}{0.268510,0.009605,0.335427}%
\pgfsetstrokecolor{currentstroke}%
\pgfsetdash{}{0pt}%
\pgfpathmoveto{\pgfqpoint{4.509162in}{1.558562in}}%
\pgfpathlineto{\pgfqpoint{4.509150in}{1.558563in}}%
\pgfusepath{stroke}%
\end{pgfscope}%
\begin{pgfscope}%
\pgfpathrectangle{\pgfqpoint{3.352233in}{1.400000in}}{\pgfqpoint{2.407767in}{1.544118in}}%
\pgfusepath{clip}%
\pgfsetbuttcap%
\pgfsetroundjoin%
\pgfsetlinewidth{0.501875pt}%
\definecolor{currentstroke}{rgb}{0.268510,0.009605,0.335427}%
\pgfsetstrokecolor{currentstroke}%
\pgfsetdash{}{0pt}%
\pgfpathmoveto{\pgfqpoint{4.509150in}{1.558563in}}%
\pgfpathlineto{\pgfqpoint{4.509152in}{1.558562in}}%
\pgfusepath{stroke}%
\end{pgfscope}%
\begin{pgfscope}%
\pgfpathrectangle{\pgfqpoint{3.352233in}{1.400000in}}{\pgfqpoint{2.407767in}{1.544118in}}%
\pgfusepath{clip}%
\pgfsetbuttcap%
\pgfsetroundjoin%
\pgfsetlinewidth{0.501875pt}%
\definecolor{currentstroke}{rgb}{0.268510,0.009605,0.335427}%
\pgfsetstrokecolor{currentstroke}%
\pgfsetdash{}{0pt}%
\pgfpathmoveto{\pgfqpoint{4.509152in}{1.558562in}}%
\pgfpathlineto{\pgfqpoint{4.509163in}{1.558561in}}%
\pgfusepath{stroke}%
\end{pgfscope}%
\begin{pgfscope}%
\pgfpathrectangle{\pgfqpoint{3.352233in}{1.400000in}}{\pgfqpoint{2.407767in}{1.544118in}}%
\pgfusepath{clip}%
\pgfsetbuttcap%
\pgfsetroundjoin%
\pgfsetlinewidth{0.501875pt}%
\definecolor{currentstroke}{rgb}{0.268510,0.009605,0.335427}%
\pgfsetstrokecolor{currentstroke}%
\pgfsetdash{}{0pt}%
\pgfpathmoveto{\pgfqpoint{4.509163in}{1.558561in}}%
\pgfpathlineto{\pgfqpoint{4.509169in}{1.558561in}}%
\pgfusepath{stroke}%
\end{pgfscope}%
\begin{pgfscope}%
\pgfpathrectangle{\pgfqpoint{3.352233in}{1.400000in}}{\pgfqpoint{2.407767in}{1.544118in}}%
\pgfusepath{clip}%
\pgfsetbuttcap%
\pgfsetroundjoin%
\pgfsetlinewidth{0.501875pt}%
\definecolor{currentstroke}{rgb}{0.268510,0.009605,0.335427}%
\pgfsetstrokecolor{currentstroke}%
\pgfsetdash{}{0pt}%
\pgfpathmoveto{\pgfqpoint{4.509169in}{1.558561in}}%
\pgfpathlineto{\pgfqpoint{4.509162in}{1.558561in}}%
\pgfusepath{stroke}%
\end{pgfscope}%
\begin{pgfscope}%
\pgfpathrectangle{\pgfqpoint{3.352233in}{1.400000in}}{\pgfqpoint{2.407767in}{1.544118in}}%
\pgfusepath{clip}%
\pgfsetbuttcap%
\pgfsetroundjoin%
\pgfsetlinewidth{0.501875pt}%
\definecolor{currentstroke}{rgb}{0.268510,0.009605,0.335427}%
\pgfsetstrokecolor{currentstroke}%
\pgfsetdash{}{0pt}%
\pgfpathmoveto{\pgfqpoint{4.509162in}{1.558561in}}%
\pgfpathlineto{\pgfqpoint{4.509152in}{1.558562in}}%
\pgfusepath{stroke}%
\end{pgfscope}%
\begin{pgfscope}%
\pgfpathrectangle{\pgfqpoint{3.352233in}{1.400000in}}{\pgfqpoint{2.407767in}{1.544118in}}%
\pgfusepath{clip}%
\pgfsetbuttcap%
\pgfsetroundjoin%
\pgfsetlinewidth{0.501875pt}%
\definecolor{currentstroke}{rgb}{0.268510,0.009605,0.335427}%
\pgfsetstrokecolor{currentstroke}%
\pgfsetdash{}{0pt}%
\pgfpathmoveto{\pgfqpoint{4.509152in}{1.558562in}}%
\pgfpathlineto{\pgfqpoint{4.509152in}{1.558562in}}%
\pgfusepath{stroke}%
\end{pgfscope}%
\begin{pgfscope}%
\pgfpathrectangle{\pgfqpoint{3.352233in}{1.400000in}}{\pgfqpoint{2.407767in}{1.544118in}}%
\pgfusepath{clip}%
\pgfsetbuttcap%
\pgfsetroundjoin%
\pgfsetlinewidth{0.501875pt}%
\definecolor{currentstroke}{rgb}{0.268510,0.009605,0.335427}%
\pgfsetstrokecolor{currentstroke}%
\pgfsetdash{}{0pt}%
\pgfpathmoveto{\pgfqpoint{4.509152in}{1.558562in}}%
\pgfpathlineto{\pgfqpoint{4.509163in}{1.558561in}}%
\pgfusepath{stroke}%
\end{pgfscope}%
\begin{pgfscope}%
\pgfpathrectangle{\pgfqpoint{3.352233in}{1.400000in}}{\pgfqpoint{2.407767in}{1.544118in}}%
\pgfusepath{clip}%
\pgfsetbuttcap%
\pgfsetroundjoin%
\pgfsetlinewidth{0.501875pt}%
\definecolor{currentstroke}{rgb}{0.268510,0.009605,0.335427}%
\pgfsetstrokecolor{currentstroke}%
\pgfsetdash{}{0pt}%
\pgfpathmoveto{\pgfqpoint{4.509163in}{1.558561in}}%
\pgfpathlineto{\pgfqpoint{4.509168in}{1.558561in}}%
\pgfusepath{stroke}%
\end{pgfscope}%
\begin{pgfscope}%
\pgfpathrectangle{\pgfqpoint{3.352233in}{1.400000in}}{\pgfqpoint{2.407767in}{1.544118in}}%
\pgfusepath{clip}%
\pgfsetbuttcap%
\pgfsetroundjoin%
\pgfsetlinewidth{0.501875pt}%
\definecolor{currentstroke}{rgb}{0.268510,0.009605,0.335427}%
\pgfsetstrokecolor{currentstroke}%
\pgfsetdash{}{0pt}%
\pgfpathmoveto{\pgfqpoint{4.509168in}{1.558561in}}%
\pgfpathlineto{\pgfqpoint{4.509162in}{1.558561in}}%
\pgfusepath{stroke}%
\end{pgfscope}%
\begin{pgfscope}%
\pgfpathrectangle{\pgfqpoint{3.352233in}{1.400000in}}{\pgfqpoint{2.407767in}{1.544118in}}%
\pgfusepath{clip}%
\pgfsetbuttcap%
\pgfsetroundjoin%
\pgfsetlinewidth{0.501875pt}%
\definecolor{currentstroke}{rgb}{0.268510,0.009605,0.335427}%
\pgfsetstrokecolor{currentstroke}%
\pgfsetdash{}{0pt}%
\pgfpathmoveto{\pgfqpoint{4.509162in}{1.558561in}}%
\pgfpathlineto{\pgfqpoint{4.509153in}{1.558562in}}%
\pgfusepath{stroke}%
\end{pgfscope}%
\begin{pgfscope}%
\pgfpathrectangle{\pgfqpoint{3.352233in}{1.400000in}}{\pgfqpoint{2.407767in}{1.544118in}}%
\pgfusepath{clip}%
\pgfsetbuttcap%
\pgfsetroundjoin%
\pgfsetlinewidth{0.501875pt}%
\definecolor{currentstroke}{rgb}{0.268510,0.009605,0.335427}%
\pgfsetstrokecolor{currentstroke}%
\pgfsetdash{}{0pt}%
\pgfpathmoveto{\pgfqpoint{4.509153in}{1.558562in}}%
\pgfpathlineto{\pgfqpoint{4.509153in}{1.558562in}}%
\pgfusepath{stroke}%
\end{pgfscope}%
\begin{pgfscope}%
\pgfpathrectangle{\pgfqpoint{3.352233in}{1.400000in}}{\pgfqpoint{2.407767in}{1.544118in}}%
\pgfusepath{clip}%
\pgfsetbuttcap%
\pgfsetroundjoin%
\pgfsetlinewidth{0.501875pt}%
\definecolor{currentstroke}{rgb}{0.268510,0.009605,0.335427}%
\pgfsetstrokecolor{currentstroke}%
\pgfsetdash{}{0pt}%
\pgfpathmoveto{\pgfqpoint{4.509153in}{1.558562in}}%
\pgfpathlineto{\pgfqpoint{4.509162in}{1.558561in}}%
\pgfusepath{stroke}%
\end{pgfscope}%
\begin{pgfscope}%
\pgfpathrectangle{\pgfqpoint{3.352233in}{1.400000in}}{\pgfqpoint{2.407767in}{1.544118in}}%
\pgfusepath{clip}%
\pgfsetbuttcap%
\pgfsetroundjoin%
\pgfsetlinewidth{0.501875pt}%
\definecolor{currentstroke}{rgb}{0.268510,0.009605,0.335427}%
\pgfsetstrokecolor{currentstroke}%
\pgfsetdash{}{0pt}%
\pgfpathmoveto{\pgfqpoint{4.509162in}{1.558561in}}%
\pgfpathlineto{\pgfqpoint{4.509167in}{1.558561in}}%
\pgfusepath{stroke}%
\end{pgfscope}%
\begin{pgfscope}%
\pgfpathrectangle{\pgfqpoint{3.352233in}{1.400000in}}{\pgfqpoint{2.407767in}{1.544118in}}%
\pgfusepath{clip}%
\pgfsetbuttcap%
\pgfsetroundjoin%
\pgfsetlinewidth{0.501875pt}%
\definecolor{currentstroke}{rgb}{0.268510,0.009605,0.335427}%
\pgfsetstrokecolor{currentstroke}%
\pgfsetdash{}{0pt}%
\pgfpathmoveto{\pgfqpoint{4.509167in}{1.558561in}}%
\pgfpathlineto{\pgfqpoint{4.509162in}{1.558561in}}%
\pgfusepath{stroke}%
\end{pgfscope}%
\begin{pgfscope}%
\pgfpathrectangle{\pgfqpoint{3.352233in}{1.400000in}}{\pgfqpoint{2.407767in}{1.544118in}}%
\pgfusepath{clip}%
\pgfsetbuttcap%
\pgfsetroundjoin%
\pgfsetlinewidth{0.501875pt}%
\definecolor{currentstroke}{rgb}{0.268510,0.009605,0.335427}%
\pgfsetstrokecolor{currentstroke}%
\pgfsetdash{}{0pt}%
\pgfpathmoveto{\pgfqpoint{4.509162in}{1.558561in}}%
\pgfpathlineto{\pgfqpoint{4.509154in}{1.558562in}}%
\pgfusepath{stroke}%
\end{pgfscope}%
\begin{pgfscope}%
\pgfpathrectangle{\pgfqpoint{3.352233in}{1.400000in}}{\pgfqpoint{2.407767in}{1.544118in}}%
\pgfusepath{clip}%
\pgfsetbuttcap%
\pgfsetroundjoin%
\pgfsetlinewidth{0.501875pt}%
\definecolor{currentstroke}{rgb}{0.268510,0.009605,0.335427}%
\pgfsetstrokecolor{currentstroke}%
\pgfsetdash{}{0pt}%
\pgfpathmoveto{\pgfqpoint{4.509154in}{1.558562in}}%
\pgfpathlineto{\pgfqpoint{4.509153in}{1.558562in}}%
\pgfusepath{stroke}%
\end{pgfscope}%
\begin{pgfscope}%
\pgfpathrectangle{\pgfqpoint{3.352233in}{1.400000in}}{\pgfqpoint{2.407767in}{1.544118in}}%
\pgfusepath{clip}%
\pgfsetbuttcap%
\pgfsetroundjoin%
\pgfsetlinewidth{0.501875pt}%
\definecolor{currentstroke}{rgb}{0.268510,0.009605,0.335427}%
\pgfsetstrokecolor{currentstroke}%
\pgfsetdash{}{0pt}%
\pgfpathmoveto{\pgfqpoint{4.509153in}{1.558562in}}%
\pgfpathlineto{\pgfqpoint{4.509161in}{1.558561in}}%
\pgfusepath{stroke}%
\end{pgfscope}%
\begin{pgfscope}%
\pgfpathrectangle{\pgfqpoint{3.352233in}{1.400000in}}{\pgfqpoint{2.407767in}{1.544118in}}%
\pgfusepath{clip}%
\pgfsetbuttcap%
\pgfsetroundjoin%
\pgfsetlinewidth{0.501875pt}%
\definecolor{currentstroke}{rgb}{0.268510,0.009605,0.335427}%
\pgfsetstrokecolor{currentstroke}%
\pgfsetdash{}{0pt}%
\pgfpathmoveto{\pgfqpoint{4.509161in}{1.558561in}}%
\pgfpathlineto{\pgfqpoint{4.509166in}{1.558561in}}%
\pgfusepath{stroke}%
\end{pgfscope}%
\begin{pgfscope}%
\pgfpathrectangle{\pgfqpoint{3.352233in}{1.400000in}}{\pgfqpoint{2.407767in}{1.544118in}}%
\pgfusepath{clip}%
\pgfsetbuttcap%
\pgfsetroundjoin%
\pgfsetlinewidth{0.501875pt}%
\definecolor{currentstroke}{rgb}{0.268510,0.009605,0.335427}%
\pgfsetstrokecolor{currentstroke}%
\pgfsetdash{}{0pt}%
\pgfpathmoveto{\pgfqpoint{4.509166in}{1.558561in}}%
\pgfpathlineto{\pgfqpoint{4.509162in}{1.558561in}}%
\pgfusepath{stroke}%
\end{pgfscope}%
\begin{pgfscope}%
\pgfpathrectangle{\pgfqpoint{3.352233in}{1.400000in}}{\pgfqpoint{2.407767in}{1.544118in}}%
\pgfusepath{clip}%
\pgfsetbuttcap%
\pgfsetroundjoin%
\pgfsetlinewidth{0.501875pt}%
\definecolor{currentstroke}{rgb}{0.268510,0.009605,0.335427}%
\pgfsetstrokecolor{currentstroke}%
\pgfsetdash{}{0pt}%
\pgfpathmoveto{\pgfqpoint{4.509162in}{1.558561in}}%
\pgfpathlineto{\pgfqpoint{4.509155in}{1.558562in}}%
\pgfusepath{stroke}%
\end{pgfscope}%
\begin{pgfscope}%
\pgfpathrectangle{\pgfqpoint{3.352233in}{1.400000in}}{\pgfqpoint{2.407767in}{1.544118in}}%
\pgfusepath{clip}%
\pgfsetbuttcap%
\pgfsetroundjoin%
\pgfsetlinewidth{0.501875pt}%
\definecolor{currentstroke}{rgb}{0.268510,0.009605,0.335427}%
\pgfsetstrokecolor{currentstroke}%
\pgfsetdash{}{0pt}%
\pgfpathmoveto{\pgfqpoint{4.509155in}{1.558562in}}%
\pgfpathlineto{\pgfqpoint{4.509154in}{1.558562in}}%
\pgfusepath{stroke}%
\end{pgfscope}%
\begin{pgfscope}%
\pgfpathrectangle{\pgfqpoint{3.352233in}{1.400000in}}{\pgfqpoint{2.407767in}{1.544118in}}%
\pgfusepath{clip}%
\pgfsetbuttcap%
\pgfsetroundjoin%
\pgfsetlinewidth{0.501875pt}%
\definecolor{currentstroke}{rgb}{0.268510,0.009605,0.335427}%
\pgfsetstrokecolor{currentstroke}%
\pgfsetdash{}{0pt}%
\pgfpathmoveto{\pgfqpoint{4.509154in}{1.558562in}}%
\pgfpathlineto{\pgfqpoint{4.509161in}{1.558561in}}%
\pgfusepath{stroke}%
\end{pgfscope}%
\begin{pgfscope}%
\pgfpathrectangle{\pgfqpoint{3.352233in}{1.400000in}}{\pgfqpoint{2.407767in}{1.544118in}}%
\pgfusepath{clip}%
\pgfsetbuttcap%
\pgfsetroundjoin%
\pgfsetlinewidth{0.501875pt}%
\definecolor{currentstroke}{rgb}{0.268510,0.009605,0.335427}%
\pgfsetstrokecolor{currentstroke}%
\pgfsetdash{}{0pt}%
\pgfpathmoveto{\pgfqpoint{4.509161in}{1.558561in}}%
\pgfpathlineto{\pgfqpoint{4.509166in}{1.558561in}}%
\pgfusepath{stroke}%
\end{pgfscope}%
\begin{pgfscope}%
\pgfpathrectangle{\pgfqpoint{3.352233in}{1.400000in}}{\pgfqpoint{2.407767in}{1.544118in}}%
\pgfusepath{clip}%
\pgfsetbuttcap%
\pgfsetroundjoin%
\pgfsetlinewidth{0.501875pt}%
\definecolor{currentstroke}{rgb}{0.268510,0.009605,0.335427}%
\pgfsetstrokecolor{currentstroke}%
\pgfsetdash{}{0pt}%
\pgfpathmoveto{\pgfqpoint{4.509166in}{1.558561in}}%
\pgfpathlineto{\pgfqpoint{4.509162in}{1.558561in}}%
\pgfusepath{stroke}%
\end{pgfscope}%
\begin{pgfscope}%
\pgfpathrectangle{\pgfqpoint{3.352233in}{1.400000in}}{\pgfqpoint{2.407767in}{1.544118in}}%
\pgfusepath{clip}%
\pgfsetbuttcap%
\pgfsetroundjoin%
\pgfsetlinewidth{0.501875pt}%
\definecolor{currentstroke}{rgb}{0.268510,0.009605,0.335427}%
\pgfsetstrokecolor{currentstroke}%
\pgfsetdash{}{0pt}%
\pgfpathmoveto{\pgfqpoint{4.509162in}{1.558561in}}%
\pgfpathlineto{\pgfqpoint{4.509156in}{1.558562in}}%
\pgfusepath{stroke}%
\end{pgfscope}%
\begin{pgfscope}%
\pgfpathrectangle{\pgfqpoint{3.352233in}{1.400000in}}{\pgfqpoint{2.407767in}{1.544118in}}%
\pgfusepath{clip}%
\pgfsetbuttcap%
\pgfsetroundjoin%
\pgfsetlinewidth{0.501875pt}%
\definecolor{currentstroke}{rgb}{0.268510,0.009605,0.335427}%
\pgfsetstrokecolor{currentstroke}%
\pgfsetdash{}{0pt}%
\pgfpathmoveto{\pgfqpoint{4.509156in}{1.558562in}}%
\pgfpathlineto{\pgfqpoint{4.509154in}{1.558562in}}%
\pgfusepath{stroke}%
\end{pgfscope}%
\begin{pgfscope}%
\pgfpathrectangle{\pgfqpoint{3.352233in}{1.400000in}}{\pgfqpoint{2.407767in}{1.544118in}}%
\pgfusepath{clip}%
\pgfsetbuttcap%
\pgfsetroundjoin%
\pgfsetlinewidth{0.501875pt}%
\definecolor{currentstroke}{rgb}{0.268510,0.009605,0.335427}%
\pgfsetstrokecolor{currentstroke}%
\pgfsetdash{}{0pt}%
\pgfpathmoveto{\pgfqpoint{4.509154in}{1.558562in}}%
\pgfpathlineto{\pgfqpoint{4.509160in}{1.558561in}}%
\pgfusepath{stroke}%
\end{pgfscope}%
\begin{pgfscope}%
\pgfpathrectangle{\pgfqpoint{3.352233in}{1.400000in}}{\pgfqpoint{2.407767in}{1.544118in}}%
\pgfusepath{clip}%
\pgfsetbuttcap%
\pgfsetroundjoin%
\pgfsetlinewidth{0.501875pt}%
\definecolor{currentstroke}{rgb}{0.268510,0.009605,0.335427}%
\pgfsetstrokecolor{currentstroke}%
\pgfsetdash{}{0pt}%
\pgfpathmoveto{\pgfqpoint{4.509160in}{1.558561in}}%
\pgfpathlineto{\pgfqpoint{4.509165in}{1.558561in}}%
\pgfusepath{stroke}%
\end{pgfscope}%
\begin{pgfscope}%
\pgfpathrectangle{\pgfqpoint{3.352233in}{1.400000in}}{\pgfqpoint{2.407767in}{1.544118in}}%
\pgfusepath{clip}%
\pgfsetbuttcap%
\pgfsetroundjoin%
\pgfsetlinewidth{0.501875pt}%
\definecolor{currentstroke}{rgb}{0.268510,0.009605,0.335427}%
\pgfsetstrokecolor{currentstroke}%
\pgfsetdash{}{0pt}%
\pgfpathmoveto{\pgfqpoint{4.509165in}{1.558561in}}%
\pgfpathlineto{\pgfqpoint{4.509162in}{1.558561in}}%
\pgfusepath{stroke}%
\end{pgfscope}%
\begin{pgfscope}%
\pgfpathrectangle{\pgfqpoint{3.352233in}{1.400000in}}{\pgfqpoint{2.407767in}{1.544118in}}%
\pgfusepath{clip}%
\pgfsetbuttcap%
\pgfsetroundjoin%
\pgfsetlinewidth{0.501875pt}%
\definecolor{currentstroke}{rgb}{0.268510,0.009605,0.335427}%
\pgfsetstrokecolor{currentstroke}%
\pgfsetdash{}{0pt}%
\pgfpathmoveto{\pgfqpoint{4.509162in}{1.558561in}}%
\pgfpathlineto{\pgfqpoint{4.509156in}{1.558562in}}%
\pgfusepath{stroke}%
\end{pgfscope}%
\begin{pgfscope}%
\pgfpathrectangle{\pgfqpoint{3.352233in}{1.400000in}}{\pgfqpoint{2.407767in}{1.544118in}}%
\pgfusepath{clip}%
\pgfsetbuttcap%
\pgfsetroundjoin%
\pgfsetlinewidth{0.501875pt}%
\definecolor{currentstroke}{rgb}{0.268510,0.009605,0.335427}%
\pgfsetstrokecolor{currentstroke}%
\pgfsetdash{}{0pt}%
\pgfpathmoveto{\pgfqpoint{4.509156in}{1.558562in}}%
\pgfpathlineto{\pgfqpoint{4.509155in}{1.558562in}}%
\pgfusepath{stroke}%
\end{pgfscope}%
\begin{pgfscope}%
\pgfpathrectangle{\pgfqpoint{3.352233in}{1.400000in}}{\pgfqpoint{2.407767in}{1.544118in}}%
\pgfusepath{clip}%
\pgfsetbuttcap%
\pgfsetroundjoin%
\pgfsetlinewidth{0.501875pt}%
\definecolor{currentstroke}{rgb}{0.268510,0.009605,0.335427}%
\pgfsetstrokecolor{currentstroke}%
\pgfsetdash{}{0pt}%
\pgfpathmoveto{\pgfqpoint{4.509155in}{1.558562in}}%
\pgfpathlineto{\pgfqpoint{4.509160in}{1.558561in}}%
\pgfusepath{stroke}%
\end{pgfscope}%
\begin{pgfscope}%
\pgfpathrectangle{\pgfqpoint{3.352233in}{1.400000in}}{\pgfqpoint{2.407767in}{1.544118in}}%
\pgfusepath{clip}%
\pgfsetbuttcap%
\pgfsetroundjoin%
\pgfsetlinewidth{0.501875pt}%
\definecolor{currentstroke}{rgb}{0.268510,0.009605,0.335427}%
\pgfsetstrokecolor{currentstroke}%
\pgfsetdash{}{0pt}%
\pgfpathmoveto{\pgfqpoint{4.509160in}{1.558561in}}%
\pgfpathlineto{\pgfqpoint{4.509164in}{1.558561in}}%
\pgfusepath{stroke}%
\end{pgfscope}%
\begin{pgfscope}%
\pgfpathrectangle{\pgfqpoint{3.352233in}{1.400000in}}{\pgfqpoint{2.407767in}{1.544118in}}%
\pgfusepath{clip}%
\pgfsetbuttcap%
\pgfsetroundjoin%
\pgfsetlinewidth{0.501875pt}%
\definecolor{currentstroke}{rgb}{0.268510,0.009605,0.335427}%
\pgfsetstrokecolor{currentstroke}%
\pgfsetdash{}{0pt}%
\pgfpathmoveto{\pgfqpoint{4.509164in}{1.558561in}}%
\pgfpathlineto{\pgfqpoint{4.509162in}{1.558561in}}%
\pgfusepath{stroke}%
\end{pgfscope}%
\begin{pgfscope}%
\pgfpathrectangle{\pgfqpoint{3.352233in}{1.400000in}}{\pgfqpoint{2.407767in}{1.544118in}}%
\pgfusepath{clip}%
\pgfsetbuttcap%
\pgfsetroundjoin%
\pgfsetlinewidth{0.501875pt}%
\definecolor{currentstroke}{rgb}{0.268510,0.009605,0.335427}%
\pgfsetstrokecolor{currentstroke}%
\pgfsetdash{}{0pt}%
\pgfpathmoveto{\pgfqpoint{4.509162in}{1.558561in}}%
\pgfpathlineto{\pgfqpoint{4.509157in}{1.558562in}}%
\pgfusepath{stroke}%
\end{pgfscope}%
\begin{pgfscope}%
\pgfpathrectangle{\pgfqpoint{3.352233in}{1.400000in}}{\pgfqpoint{2.407767in}{1.544118in}}%
\pgfusepath{clip}%
\pgfsetbuttcap%
\pgfsetroundjoin%
\pgfsetlinewidth{0.501875pt}%
\definecolor{currentstroke}{rgb}{0.268510,0.009605,0.335427}%
\pgfsetstrokecolor{currentstroke}%
\pgfsetdash{}{0pt}%
\pgfpathmoveto{\pgfqpoint{4.509157in}{1.558562in}}%
\pgfpathlineto{\pgfqpoint{4.509155in}{1.558562in}}%
\pgfusepath{stroke}%
\end{pgfscope}%
\begin{pgfscope}%
\pgfpathrectangle{\pgfqpoint{3.352233in}{1.400000in}}{\pgfqpoint{2.407767in}{1.544118in}}%
\pgfusepath{clip}%
\pgfsetbuttcap%
\pgfsetroundjoin%
\pgfsetlinewidth{0.501875pt}%
\definecolor{currentstroke}{rgb}{0.268510,0.009605,0.335427}%
\pgfsetstrokecolor{currentstroke}%
\pgfsetdash{}{0pt}%
\pgfpathmoveto{\pgfqpoint{4.509155in}{1.558562in}}%
\pgfpathlineto{\pgfqpoint{4.509159in}{1.558561in}}%
\pgfusepath{stroke}%
\end{pgfscope}%
\begin{pgfscope}%
\pgfpathrectangle{\pgfqpoint{3.352233in}{1.400000in}}{\pgfqpoint{2.407767in}{1.544118in}}%
\pgfusepath{clip}%
\pgfsetbuttcap%
\pgfsetroundjoin%
\pgfsetlinewidth{0.501875pt}%
\definecolor{currentstroke}{rgb}{0.268510,0.009605,0.335427}%
\pgfsetstrokecolor{currentstroke}%
\pgfsetdash{}{0pt}%
\pgfpathmoveto{\pgfqpoint{4.509159in}{1.558561in}}%
\pgfpathlineto{\pgfqpoint{4.509164in}{1.558561in}}%
\pgfusepath{stroke}%
\end{pgfscope}%
\begin{pgfscope}%
\pgfpathrectangle{\pgfqpoint{3.352233in}{1.400000in}}{\pgfqpoint{2.407767in}{1.544118in}}%
\pgfusepath{clip}%
\pgfsetbuttcap%
\pgfsetroundjoin%
\pgfsetlinewidth{0.501875pt}%
\definecolor{currentstroke}{rgb}{0.268510,0.009605,0.335427}%
\pgfsetstrokecolor{currentstroke}%
\pgfsetdash{}{0pt}%
\pgfpathmoveto{\pgfqpoint{4.509164in}{1.558561in}}%
\pgfpathlineto{\pgfqpoint{4.509162in}{1.558561in}}%
\pgfusepath{stroke}%
\end{pgfscope}%
\begin{pgfscope}%
\pgfpathrectangle{\pgfqpoint{3.352233in}{1.400000in}}{\pgfqpoint{2.407767in}{1.544118in}}%
\pgfusepath{clip}%
\pgfsetbuttcap%
\pgfsetroundjoin%
\pgfsetlinewidth{0.501875pt}%
\definecolor{currentstroke}{rgb}{0.268510,0.009605,0.335427}%
\pgfsetstrokecolor{currentstroke}%
\pgfsetdash{}{0pt}%
\pgfpathmoveto{\pgfqpoint{4.509162in}{1.558561in}}%
\pgfpathlineto{\pgfqpoint{4.509157in}{1.558562in}}%
\pgfusepath{stroke}%
\end{pgfscope}%
\begin{pgfscope}%
\pgfpathrectangle{\pgfqpoint{3.352233in}{1.400000in}}{\pgfqpoint{2.407767in}{1.544118in}}%
\pgfusepath{clip}%
\pgfsetbuttcap%
\pgfsetroundjoin%
\pgfsetlinewidth{0.501875pt}%
\definecolor{currentstroke}{rgb}{0.268510,0.009605,0.335427}%
\pgfsetstrokecolor{currentstroke}%
\pgfsetdash{}{0pt}%
\pgfpathmoveto{\pgfqpoint{4.509157in}{1.558562in}}%
\pgfpathlineto{\pgfqpoint{4.509156in}{1.558562in}}%
\pgfusepath{stroke}%
\end{pgfscope}%
\begin{pgfscope}%
\pgfpathrectangle{\pgfqpoint{3.352233in}{1.400000in}}{\pgfqpoint{2.407767in}{1.544118in}}%
\pgfusepath{clip}%
\pgfsetbuttcap%
\pgfsetroundjoin%
\pgfsetlinewidth{0.501875pt}%
\definecolor{currentstroke}{rgb}{0.268510,0.009605,0.335427}%
\pgfsetstrokecolor{currentstroke}%
\pgfsetdash{}{0pt}%
\pgfpathmoveto{\pgfqpoint{4.509156in}{1.558562in}}%
\pgfpathlineto{\pgfqpoint{4.509159in}{1.558561in}}%
\pgfusepath{stroke}%
\end{pgfscope}%
\begin{pgfscope}%
\pgfpathrectangle{\pgfqpoint{3.352233in}{1.400000in}}{\pgfqpoint{2.407767in}{1.544118in}}%
\pgfusepath{clip}%
\pgfsetbuttcap%
\pgfsetroundjoin%
\pgfsetlinewidth{0.501875pt}%
\definecolor{currentstroke}{rgb}{0.268510,0.009605,0.335427}%
\pgfsetstrokecolor{currentstroke}%
\pgfsetdash{}{0pt}%
\pgfpathmoveto{\pgfqpoint{4.509159in}{1.558561in}}%
\pgfpathlineto{\pgfqpoint{4.509163in}{1.558561in}}%
\pgfusepath{stroke}%
\end{pgfscope}%
\begin{pgfscope}%
\pgfpathrectangle{\pgfqpoint{3.352233in}{1.400000in}}{\pgfqpoint{2.407767in}{1.544118in}}%
\pgfusepath{clip}%
\pgfsetbuttcap%
\pgfsetroundjoin%
\pgfsetlinewidth{0.501875pt}%
\definecolor{currentstroke}{rgb}{0.268510,0.009605,0.335427}%
\pgfsetstrokecolor{currentstroke}%
\pgfsetdash{}{0pt}%
\pgfpathmoveto{\pgfqpoint{4.509163in}{1.558561in}}%
\pgfpathlineto{\pgfqpoint{4.509162in}{1.558561in}}%
\pgfusepath{stroke}%
\end{pgfscope}%
\begin{pgfscope}%
\pgfpathrectangle{\pgfqpoint{3.352233in}{1.400000in}}{\pgfqpoint{2.407767in}{1.544118in}}%
\pgfusepath{clip}%
\pgfsetbuttcap%
\pgfsetroundjoin%
\pgfsetlinewidth{0.501875pt}%
\definecolor{currentstroke}{rgb}{0.268510,0.009605,0.335427}%
\pgfsetstrokecolor{currentstroke}%
\pgfsetdash{}{0pt}%
\pgfpathmoveto{\pgfqpoint{4.509162in}{1.558561in}}%
\pgfpathlineto{\pgfqpoint{4.509158in}{1.558562in}}%
\pgfusepath{stroke}%
\end{pgfscope}%
\begin{pgfscope}%
\pgfpathrectangle{\pgfqpoint{3.352233in}{1.400000in}}{\pgfqpoint{2.407767in}{1.544118in}}%
\pgfusepath{clip}%
\pgfsetbuttcap%
\pgfsetroundjoin%
\pgfsetlinewidth{0.501875pt}%
\definecolor{currentstroke}{rgb}{0.268510,0.009605,0.335427}%
\pgfsetstrokecolor{currentstroke}%
\pgfsetdash{}{0pt}%
\pgfpathmoveto{\pgfqpoint{4.509158in}{1.558562in}}%
\pgfpathlineto{\pgfqpoint{4.509156in}{1.558562in}}%
\pgfusepath{stroke}%
\end{pgfscope}%
\begin{pgfscope}%
\pgfpathrectangle{\pgfqpoint{3.352233in}{1.400000in}}{\pgfqpoint{2.407767in}{1.544118in}}%
\pgfusepath{clip}%
\pgfsetbuttcap%
\pgfsetroundjoin%
\pgfsetlinewidth{0.501875pt}%
\definecolor{currentstroke}{rgb}{0.268510,0.009605,0.335427}%
\pgfsetstrokecolor{currentstroke}%
\pgfsetdash{}{0pt}%
\pgfpathmoveto{\pgfqpoint{4.509156in}{1.558562in}}%
\pgfpathlineto{\pgfqpoint{4.509159in}{1.558561in}}%
\pgfusepath{stroke}%
\end{pgfscope}%
\begin{pgfscope}%
\pgfpathrectangle{\pgfqpoint{3.352233in}{1.400000in}}{\pgfqpoint{2.407767in}{1.544118in}}%
\pgfusepath{clip}%
\pgfsetbuttcap%
\pgfsetroundjoin%
\pgfsetlinewidth{0.501875pt}%
\definecolor{currentstroke}{rgb}{0.268510,0.009605,0.335427}%
\pgfsetstrokecolor{currentstroke}%
\pgfsetdash{}{0pt}%
\pgfpathmoveto{\pgfqpoint{4.509159in}{1.558561in}}%
\pgfpathlineto{\pgfqpoint{4.509163in}{1.558561in}}%
\pgfusepath{stroke}%
\end{pgfscope}%
\begin{pgfscope}%
\pgfpathrectangle{\pgfqpoint{3.352233in}{1.400000in}}{\pgfqpoint{2.407767in}{1.544118in}}%
\pgfusepath{clip}%
\pgfsetbuttcap%
\pgfsetroundjoin%
\pgfsetlinewidth{0.501875pt}%
\definecolor{currentstroke}{rgb}{0.268510,0.009605,0.335427}%
\pgfsetstrokecolor{currentstroke}%
\pgfsetdash{}{0pt}%
\pgfpathmoveto{\pgfqpoint{4.509163in}{1.558561in}}%
\pgfpathlineto{\pgfqpoint{4.509162in}{1.558561in}}%
\pgfusepath{stroke}%
\end{pgfscope}%
\begin{pgfscope}%
\pgfpathrectangle{\pgfqpoint{3.352233in}{1.400000in}}{\pgfqpoint{2.407767in}{1.544118in}}%
\pgfusepath{clip}%
\pgfsetbuttcap%
\pgfsetroundjoin%
\pgfsetlinewidth{0.501875pt}%
\definecolor{currentstroke}{rgb}{0.268510,0.009605,0.335427}%
\pgfsetstrokecolor{currentstroke}%
\pgfsetdash{}{0pt}%
\pgfpathmoveto{\pgfqpoint{4.509162in}{1.558561in}}%
\pgfpathlineto{\pgfqpoint{4.509158in}{1.558562in}}%
\pgfusepath{stroke}%
\end{pgfscope}%
\begin{pgfscope}%
\pgfpathrectangle{\pgfqpoint{3.352233in}{1.400000in}}{\pgfqpoint{2.407767in}{1.544118in}}%
\pgfusepath{clip}%
\pgfsetbuttcap%
\pgfsetroundjoin%
\pgfsetlinewidth{0.501875pt}%
\definecolor{currentstroke}{rgb}{0.268510,0.009605,0.335427}%
\pgfsetstrokecolor{currentstroke}%
\pgfsetdash{}{0pt}%
\pgfpathmoveto{\pgfqpoint{4.509158in}{1.558562in}}%
\pgfpathlineto{\pgfqpoint{4.509156in}{1.558562in}}%
\pgfusepath{stroke}%
\end{pgfscope}%
\begin{pgfscope}%
\pgfpathrectangle{\pgfqpoint{3.352233in}{1.400000in}}{\pgfqpoint{2.407767in}{1.544118in}}%
\pgfusepath{clip}%
\pgfsetbuttcap%
\pgfsetroundjoin%
\pgfsetlinewidth{0.501875pt}%
\definecolor{currentstroke}{rgb}{0.268510,0.009605,0.335427}%
\pgfsetstrokecolor{currentstroke}%
\pgfsetdash{}{0pt}%
\pgfpathmoveto{\pgfqpoint{4.509156in}{1.558562in}}%
\pgfpathlineto{\pgfqpoint{4.509159in}{1.558562in}}%
\pgfusepath{stroke}%
\end{pgfscope}%
\begin{pgfscope}%
\pgfpathrectangle{\pgfqpoint{3.352233in}{1.400000in}}{\pgfqpoint{2.407767in}{1.544118in}}%
\pgfusepath{clip}%
\pgfsetbuttcap%
\pgfsetroundjoin%
\pgfsetlinewidth{0.501875pt}%
\definecolor{currentstroke}{rgb}{0.268510,0.009605,0.335427}%
\pgfsetstrokecolor{currentstroke}%
\pgfsetdash{}{0pt}%
\pgfpathmoveto{\pgfqpoint{4.509159in}{1.558562in}}%
\pgfpathlineto{\pgfqpoint{4.509162in}{1.558561in}}%
\pgfusepath{stroke}%
\end{pgfscope}%
\begin{pgfscope}%
\pgfpathrectangle{\pgfqpoint{3.352233in}{1.400000in}}{\pgfqpoint{2.407767in}{1.544118in}}%
\pgfusepath{clip}%
\pgfsetbuttcap%
\pgfsetroundjoin%
\pgfsetlinewidth{0.501875pt}%
\definecolor{currentstroke}{rgb}{0.268510,0.009605,0.335427}%
\pgfsetstrokecolor{currentstroke}%
\pgfsetdash{}{0pt}%
\pgfpathmoveto{\pgfqpoint{4.509162in}{1.558561in}}%
\pgfpathlineto{\pgfqpoint{4.509162in}{1.558561in}}%
\pgfusepath{stroke}%
\end{pgfscope}%
\begin{pgfscope}%
\pgfpathrectangle{\pgfqpoint{3.352233in}{1.400000in}}{\pgfqpoint{2.407767in}{1.544118in}}%
\pgfusepath{clip}%
\pgfsetbuttcap%
\pgfsetroundjoin%
\pgfsetlinewidth{0.501875pt}%
\definecolor{currentstroke}{rgb}{0.268510,0.009605,0.335427}%
\pgfsetstrokecolor{currentstroke}%
\pgfsetdash{}{0pt}%
\pgfpathmoveto{\pgfqpoint{4.509162in}{1.558561in}}%
\pgfpathlineto{\pgfqpoint{4.509158in}{1.558562in}}%
\pgfusepath{stroke}%
\end{pgfscope}%
\begin{pgfscope}%
\pgfpathrectangle{\pgfqpoint{3.352233in}{1.400000in}}{\pgfqpoint{2.407767in}{1.544118in}}%
\pgfusepath{clip}%
\pgfsetbuttcap%
\pgfsetroundjoin%
\pgfsetlinewidth{0.501875pt}%
\definecolor{currentstroke}{rgb}{0.268510,0.009605,0.335427}%
\pgfsetstrokecolor{currentstroke}%
\pgfsetdash{}{0pt}%
\pgfpathmoveto{\pgfqpoint{4.509158in}{1.558562in}}%
\pgfpathlineto{\pgfqpoint{4.509157in}{1.558562in}}%
\pgfusepath{stroke}%
\end{pgfscope}%
\begin{pgfscope}%
\pgfpathrectangle{\pgfqpoint{3.352233in}{1.400000in}}{\pgfqpoint{2.407767in}{1.544118in}}%
\pgfusepath{clip}%
\pgfsetbuttcap%
\pgfsetroundjoin%
\pgfsetlinewidth{0.501875pt}%
\definecolor{currentstroke}{rgb}{0.268510,0.009605,0.335427}%
\pgfsetstrokecolor{currentstroke}%
\pgfsetdash{}{0pt}%
\pgfpathmoveto{\pgfqpoint{4.509157in}{1.558562in}}%
\pgfpathlineto{\pgfqpoint{4.509159in}{1.558562in}}%
\pgfusepath{stroke}%
\end{pgfscope}%
\begin{pgfscope}%
\pgfpathrectangle{\pgfqpoint{3.352233in}{1.400000in}}{\pgfqpoint{2.407767in}{1.544118in}}%
\pgfusepath{clip}%
\pgfsetbuttcap%
\pgfsetroundjoin%
\pgfsetlinewidth{0.501875pt}%
\definecolor{currentstroke}{rgb}{0.268510,0.009605,0.335427}%
\pgfsetstrokecolor{currentstroke}%
\pgfsetdash{}{0pt}%
\pgfpathmoveto{\pgfqpoint{4.509159in}{1.558562in}}%
\pgfpathlineto{\pgfqpoint{4.509162in}{1.558561in}}%
\pgfusepath{stroke}%
\end{pgfscope}%
\begin{pgfscope}%
\pgfpathrectangle{\pgfqpoint{3.352233in}{1.400000in}}{\pgfqpoint{2.407767in}{1.544118in}}%
\pgfusepath{clip}%
\pgfsetbuttcap%
\pgfsetroundjoin%
\pgfsetlinewidth{0.501875pt}%
\definecolor{currentstroke}{rgb}{0.268510,0.009605,0.335427}%
\pgfsetstrokecolor{currentstroke}%
\pgfsetdash{}{0pt}%
\pgfpathmoveto{\pgfqpoint{4.509162in}{1.558561in}}%
\pgfpathlineto{\pgfqpoint{4.509162in}{1.558561in}}%
\pgfusepath{stroke}%
\end{pgfscope}%
\begin{pgfscope}%
\pgfpathrectangle{\pgfqpoint{3.352233in}{1.400000in}}{\pgfqpoint{2.407767in}{1.544118in}}%
\pgfusepath{clip}%
\pgfsetbuttcap%
\pgfsetroundjoin%
\pgfsetlinewidth{0.501875pt}%
\definecolor{currentstroke}{rgb}{0.268510,0.009605,0.335427}%
\pgfsetstrokecolor{currentstroke}%
\pgfsetdash{}{0pt}%
\pgfpathmoveto{\pgfqpoint{4.509162in}{1.558561in}}%
\pgfpathlineto{\pgfqpoint{4.509159in}{1.558562in}}%
\pgfusepath{stroke}%
\end{pgfscope}%
\begin{pgfscope}%
\pgfpathrectangle{\pgfqpoint{3.352233in}{1.400000in}}{\pgfqpoint{2.407767in}{1.544118in}}%
\pgfusepath{clip}%
\pgfsetbuttcap%
\pgfsetroundjoin%
\pgfsetlinewidth{0.501875pt}%
\definecolor{currentstroke}{rgb}{0.268510,0.009605,0.335427}%
\pgfsetstrokecolor{currentstroke}%
\pgfsetdash{}{0pt}%
\pgfpathmoveto{\pgfqpoint{4.509159in}{1.558562in}}%
\pgfpathlineto{\pgfqpoint{4.509157in}{1.558562in}}%
\pgfusepath{stroke}%
\end{pgfscope}%
\begin{pgfscope}%
\pgfpathrectangle{\pgfqpoint{3.352233in}{1.400000in}}{\pgfqpoint{2.407767in}{1.544118in}}%
\pgfusepath{clip}%
\pgfsetbuttcap%
\pgfsetroundjoin%
\pgfsetlinewidth{0.501875pt}%
\definecolor{currentstroke}{rgb}{0.268510,0.009605,0.335427}%
\pgfsetstrokecolor{currentstroke}%
\pgfsetdash{}{0pt}%
\pgfpathmoveto{\pgfqpoint{4.509157in}{1.558562in}}%
\pgfpathlineto{\pgfqpoint{4.509159in}{1.558562in}}%
\pgfusepath{stroke}%
\end{pgfscope}%
\begin{pgfscope}%
\pgfpathrectangle{\pgfqpoint{3.352233in}{1.400000in}}{\pgfqpoint{2.407767in}{1.544118in}}%
\pgfusepath{clip}%
\pgfsetbuttcap%
\pgfsetroundjoin%
\pgfsetlinewidth{0.501875pt}%
\definecolor{currentstroke}{rgb}{0.268510,0.009605,0.335427}%
\pgfsetstrokecolor{currentstroke}%
\pgfsetdash{}{0pt}%
\pgfpathmoveto{\pgfqpoint{4.509159in}{1.558562in}}%
\pgfpathlineto{\pgfqpoint{4.509161in}{1.558561in}}%
\pgfusepath{stroke}%
\end{pgfscope}%
\begin{pgfscope}%
\pgfpathrectangle{\pgfqpoint{3.352233in}{1.400000in}}{\pgfqpoint{2.407767in}{1.544118in}}%
\pgfusepath{clip}%
\pgfsetbuttcap%
\pgfsetroundjoin%
\pgfsetlinewidth{0.501875pt}%
\definecolor{currentstroke}{rgb}{0.268510,0.009605,0.335427}%
\pgfsetstrokecolor{currentstroke}%
\pgfsetdash{}{0pt}%
\pgfpathmoveto{\pgfqpoint{4.509161in}{1.558561in}}%
\pgfpathlineto{\pgfqpoint{4.509161in}{1.558561in}}%
\pgfusepath{stroke}%
\end{pgfscope}%
\begin{pgfscope}%
\pgfpathrectangle{\pgfqpoint{3.352233in}{1.400000in}}{\pgfqpoint{2.407767in}{1.544118in}}%
\pgfusepath{clip}%
\pgfsetbuttcap%
\pgfsetroundjoin%
\pgfsetlinewidth{0.501875pt}%
\definecolor{currentstroke}{rgb}{0.268510,0.009605,0.335427}%
\pgfsetstrokecolor{currentstroke}%
\pgfsetdash{}{0pt}%
\pgfpathmoveto{\pgfqpoint{4.509161in}{1.558561in}}%
\pgfpathlineto{\pgfqpoint{4.509159in}{1.558562in}}%
\pgfusepath{stroke}%
\end{pgfscope}%
\begin{pgfscope}%
\pgfpathrectangle{\pgfqpoint{3.352233in}{1.400000in}}{\pgfqpoint{2.407767in}{1.544118in}}%
\pgfusepath{clip}%
\pgfsetbuttcap%
\pgfsetroundjoin%
\pgfsetlinewidth{0.501875pt}%
\definecolor{currentstroke}{rgb}{0.268510,0.009605,0.335427}%
\pgfsetstrokecolor{currentstroke}%
\pgfsetdash{}{0pt}%
\pgfpathmoveto{\pgfqpoint{4.509159in}{1.558562in}}%
\pgfpathlineto{\pgfqpoint{4.509157in}{1.558562in}}%
\pgfusepath{stroke}%
\end{pgfscope}%
\begin{pgfscope}%
\pgfpathrectangle{\pgfqpoint{3.352233in}{1.400000in}}{\pgfqpoint{2.407767in}{1.544118in}}%
\pgfusepath{clip}%
\pgfsetbuttcap%
\pgfsetroundjoin%
\pgfsetlinewidth{0.501875pt}%
\definecolor{currentstroke}{rgb}{0.268510,0.009605,0.335427}%
\pgfsetstrokecolor{currentstroke}%
\pgfsetdash{}{0pt}%
\pgfpathmoveto{\pgfqpoint{4.509157in}{1.558562in}}%
\pgfpathlineto{\pgfqpoint{4.509159in}{1.558562in}}%
\pgfusepath{stroke}%
\end{pgfscope}%
\begin{pgfscope}%
\pgfpathrectangle{\pgfqpoint{3.352233in}{1.400000in}}{\pgfqpoint{2.407767in}{1.544118in}}%
\pgfusepath{clip}%
\pgfsetbuttcap%
\pgfsetroundjoin%
\pgfsetlinewidth{0.501875pt}%
\definecolor{currentstroke}{rgb}{0.268510,0.009605,0.335427}%
\pgfsetstrokecolor{currentstroke}%
\pgfsetdash{}{0pt}%
\pgfpathmoveto{\pgfqpoint{4.509159in}{1.558562in}}%
\pgfpathlineto{\pgfqpoint{4.509161in}{1.558561in}}%
\pgfusepath{stroke}%
\end{pgfscope}%
\begin{pgfscope}%
\pgfpathrectangle{\pgfqpoint{3.352233in}{1.400000in}}{\pgfqpoint{2.407767in}{1.544118in}}%
\pgfusepath{clip}%
\pgfsetbuttcap%
\pgfsetroundjoin%
\pgfsetlinewidth{0.501875pt}%
\definecolor{currentstroke}{rgb}{0.268510,0.009605,0.335427}%
\pgfsetstrokecolor{currentstroke}%
\pgfsetdash{}{0pt}%
\pgfpathmoveto{\pgfqpoint{4.509161in}{1.558561in}}%
\pgfpathlineto{\pgfqpoint{4.509161in}{1.558561in}}%
\pgfusepath{stroke}%
\end{pgfscope}%
\begin{pgfscope}%
\pgfpathrectangle{\pgfqpoint{3.352233in}{1.400000in}}{\pgfqpoint{2.407767in}{1.544118in}}%
\pgfusepath{clip}%
\pgfsetbuttcap%
\pgfsetroundjoin%
\pgfsetlinewidth{0.501875pt}%
\definecolor{currentstroke}{rgb}{0.268510,0.009605,0.335427}%
\pgfsetstrokecolor{currentstroke}%
\pgfsetdash{}{0pt}%
\pgfpathmoveto{\pgfqpoint{4.509161in}{1.558561in}}%
\pgfpathlineto{\pgfqpoint{4.509159in}{1.558562in}}%
\pgfusepath{stroke}%
\end{pgfscope}%
\begin{pgfscope}%
\pgfpathrectangle{\pgfqpoint{3.352233in}{1.400000in}}{\pgfqpoint{2.407767in}{1.544118in}}%
\pgfusepath{clip}%
\pgfsetbuttcap%
\pgfsetroundjoin%
\pgfsetlinewidth{0.501875pt}%
\definecolor{currentstroke}{rgb}{0.268510,0.009605,0.335427}%
\pgfsetstrokecolor{currentstroke}%
\pgfsetdash{}{0pt}%
\pgfpathmoveto{\pgfqpoint{4.509159in}{1.558562in}}%
\pgfpathlineto{\pgfqpoint{4.509158in}{1.558562in}}%
\pgfusepath{stroke}%
\end{pgfscope}%
\begin{pgfscope}%
\pgfpathrectangle{\pgfqpoint{3.352233in}{1.400000in}}{\pgfqpoint{2.407767in}{1.544118in}}%
\pgfusepath{clip}%
\pgfsetbuttcap%
\pgfsetroundjoin%
\pgfsetlinewidth{0.501875pt}%
\definecolor{currentstroke}{rgb}{0.268510,0.009605,0.335427}%
\pgfsetstrokecolor{currentstroke}%
\pgfsetdash{}{0pt}%
\pgfpathmoveto{\pgfqpoint{4.509158in}{1.558562in}}%
\pgfpathlineto{\pgfqpoint{4.509159in}{1.558562in}}%
\pgfusepath{stroke}%
\end{pgfscope}%
\begin{pgfscope}%
\pgfpathrectangle{\pgfqpoint{3.352233in}{1.400000in}}{\pgfqpoint{2.407767in}{1.544118in}}%
\pgfusepath{clip}%
\pgfsetbuttcap%
\pgfsetroundjoin%
\pgfsetlinewidth{0.501875pt}%
\definecolor{currentstroke}{rgb}{0.268510,0.009605,0.335427}%
\pgfsetstrokecolor{currentstroke}%
\pgfsetdash{}{0pt}%
\pgfpathmoveto{\pgfqpoint{4.509159in}{1.558562in}}%
\pgfpathlineto{\pgfqpoint{4.509161in}{1.558561in}}%
\pgfusepath{stroke}%
\end{pgfscope}%
\begin{pgfscope}%
\pgfpathrectangle{\pgfqpoint{3.352233in}{1.400000in}}{\pgfqpoint{2.407767in}{1.544118in}}%
\pgfusepath{clip}%
\pgfsetbuttcap%
\pgfsetroundjoin%
\pgfsetlinewidth{0.501875pt}%
\definecolor{currentstroke}{rgb}{0.268510,0.009605,0.335427}%
\pgfsetstrokecolor{currentstroke}%
\pgfsetdash{}{0pt}%
\pgfpathmoveto{\pgfqpoint{4.509161in}{1.558561in}}%
\pgfpathlineto{\pgfqpoint{4.509161in}{1.558561in}}%
\pgfusepath{stroke}%
\end{pgfscope}%
\begin{pgfscope}%
\pgfpathrectangle{\pgfqpoint{3.352233in}{1.400000in}}{\pgfqpoint{2.407767in}{1.544118in}}%
\pgfusepath{clip}%
\pgfsetbuttcap%
\pgfsetroundjoin%
\pgfsetlinewidth{0.501875pt}%
\definecolor{currentstroke}{rgb}{0.268510,0.009605,0.335427}%
\pgfsetstrokecolor{currentstroke}%
\pgfsetdash{}{0pt}%
\pgfpathmoveto{\pgfqpoint{4.509161in}{1.558561in}}%
\pgfpathlineto{\pgfqpoint{4.509159in}{1.558562in}}%
\pgfusepath{stroke}%
\end{pgfscope}%
\begin{pgfscope}%
\pgfpathrectangle{\pgfqpoint{3.352233in}{1.400000in}}{\pgfqpoint{2.407767in}{1.544118in}}%
\pgfusepath{clip}%
\pgfsetbuttcap%
\pgfsetroundjoin%
\pgfsetlinewidth{0.501875pt}%
\definecolor{currentstroke}{rgb}{0.268510,0.009605,0.335427}%
\pgfsetstrokecolor{currentstroke}%
\pgfsetdash{}{0pt}%
\pgfpathmoveto{\pgfqpoint{4.509159in}{1.558562in}}%
\pgfpathlineto{\pgfqpoint{4.509158in}{1.558562in}}%
\pgfusepath{stroke}%
\end{pgfscope}%
\begin{pgfscope}%
\pgfpathrectangle{\pgfqpoint{3.352233in}{1.400000in}}{\pgfqpoint{2.407767in}{1.544118in}}%
\pgfusepath{clip}%
\pgfsetbuttcap%
\pgfsetroundjoin%
\pgfsetlinewidth{0.501875pt}%
\definecolor{currentstroke}{rgb}{0.268510,0.009605,0.335427}%
\pgfsetstrokecolor{currentstroke}%
\pgfsetdash{}{0pt}%
\pgfpathmoveto{\pgfqpoint{4.509158in}{1.558562in}}%
\pgfpathlineto{\pgfqpoint{4.509159in}{1.558562in}}%
\pgfusepath{stroke}%
\end{pgfscope}%
\begin{pgfscope}%
\pgfpathrectangle{\pgfqpoint{3.352233in}{1.400000in}}{\pgfqpoint{2.407767in}{1.544118in}}%
\pgfusepath{clip}%
\pgfsetbuttcap%
\pgfsetroundjoin%
\pgfsetlinewidth{0.501875pt}%
\definecolor{currentstroke}{rgb}{0.268510,0.009605,0.335427}%
\pgfsetstrokecolor{currentstroke}%
\pgfsetdash{}{0pt}%
\pgfpathmoveto{\pgfqpoint{4.509159in}{1.558562in}}%
\pgfpathlineto{\pgfqpoint{4.509161in}{1.558561in}}%
\pgfusepath{stroke}%
\end{pgfscope}%
\begin{pgfscope}%
\pgfpathrectangle{\pgfqpoint{3.352233in}{1.400000in}}{\pgfqpoint{2.407767in}{1.544118in}}%
\pgfusepath{clip}%
\pgfsetbuttcap%
\pgfsetroundjoin%
\pgfsetlinewidth{0.501875pt}%
\definecolor{currentstroke}{rgb}{0.268510,0.009605,0.335427}%
\pgfsetstrokecolor{currentstroke}%
\pgfsetdash{}{0pt}%
\pgfpathmoveto{\pgfqpoint{4.509161in}{1.558561in}}%
\pgfpathlineto{\pgfqpoint{4.509161in}{1.558561in}}%
\pgfusepath{stroke}%
\end{pgfscope}%
\begin{pgfscope}%
\pgfpathrectangle{\pgfqpoint{3.352233in}{1.400000in}}{\pgfqpoint{2.407767in}{1.544118in}}%
\pgfusepath{clip}%
\pgfsetbuttcap%
\pgfsetroundjoin%
\pgfsetlinewidth{0.501875pt}%
\definecolor{currentstroke}{rgb}{0.268510,0.009605,0.335427}%
\pgfsetstrokecolor{currentstroke}%
\pgfsetdash{}{0pt}%
\pgfpathmoveto{\pgfqpoint{4.509161in}{1.558561in}}%
\pgfpathlineto{\pgfqpoint{4.509159in}{1.558562in}}%
\pgfusepath{stroke}%
\end{pgfscope}%
\begin{pgfscope}%
\pgfpathrectangle{\pgfqpoint{3.352233in}{1.400000in}}{\pgfqpoint{2.407767in}{1.544118in}}%
\pgfusepath{clip}%
\pgfsetbuttcap%
\pgfsetroundjoin%
\pgfsetlinewidth{0.501875pt}%
\definecolor{currentstroke}{rgb}{0.268510,0.009605,0.335427}%
\pgfsetstrokecolor{currentstroke}%
\pgfsetdash{}{0pt}%
\pgfpathmoveto{\pgfqpoint{4.509159in}{1.558562in}}%
\pgfpathlineto{\pgfqpoint{4.509158in}{1.558562in}}%
\pgfusepath{stroke}%
\end{pgfscope}%
\begin{pgfscope}%
\pgfpathrectangle{\pgfqpoint{3.352233in}{1.400000in}}{\pgfqpoint{2.407767in}{1.544118in}}%
\pgfusepath{clip}%
\pgfsetbuttcap%
\pgfsetroundjoin%
\pgfsetlinewidth{0.501875pt}%
\definecolor{currentstroke}{rgb}{0.268510,0.009605,0.335427}%
\pgfsetstrokecolor{currentstroke}%
\pgfsetdash{}{0pt}%
\pgfpathmoveto{\pgfqpoint{4.509158in}{1.558562in}}%
\pgfpathlineto{\pgfqpoint{4.509159in}{1.558562in}}%
\pgfusepath{stroke}%
\end{pgfscope}%
\begin{pgfscope}%
\pgfpathrectangle{\pgfqpoint{3.352233in}{1.400000in}}{\pgfqpoint{2.407767in}{1.544118in}}%
\pgfusepath{clip}%
\pgfsetbuttcap%
\pgfsetroundjoin%
\pgfsetlinewidth{0.501875pt}%
\definecolor{currentstroke}{rgb}{0.268510,0.009605,0.335427}%
\pgfsetstrokecolor{currentstroke}%
\pgfsetdash{}{0pt}%
\pgfpathmoveto{\pgfqpoint{4.509159in}{1.558562in}}%
\pgfpathlineto{\pgfqpoint{4.509160in}{1.558561in}}%
\pgfusepath{stroke}%
\end{pgfscope}%
\begin{pgfscope}%
\pgfpathrectangle{\pgfqpoint{3.352233in}{1.400000in}}{\pgfqpoint{2.407767in}{1.544118in}}%
\pgfusepath{clip}%
\pgfsetbuttcap%
\pgfsetroundjoin%
\pgfsetlinewidth{0.501875pt}%
\definecolor{currentstroke}{rgb}{0.268510,0.009605,0.335427}%
\pgfsetstrokecolor{currentstroke}%
\pgfsetdash{}{0pt}%
\pgfpathmoveto{\pgfqpoint{4.509160in}{1.558561in}}%
\pgfpathlineto{\pgfqpoint{4.509161in}{1.558561in}}%
\pgfusepath{stroke}%
\end{pgfscope}%
\begin{pgfscope}%
\pgfpathrectangle{\pgfqpoint{3.352233in}{1.400000in}}{\pgfqpoint{2.407767in}{1.544118in}}%
\pgfusepath{clip}%
\pgfsetbuttcap%
\pgfsetroundjoin%
\pgfsetlinewidth{0.501875pt}%
\definecolor{currentstroke}{rgb}{0.268510,0.009605,0.335427}%
\pgfsetstrokecolor{currentstroke}%
\pgfsetdash{}{0pt}%
\pgfpathmoveto{\pgfqpoint{4.509161in}{1.558561in}}%
\pgfpathlineto{\pgfqpoint{4.509160in}{1.558562in}}%
\pgfusepath{stroke}%
\end{pgfscope}%
\begin{pgfscope}%
\pgfpathrectangle{\pgfqpoint{3.352233in}{1.400000in}}{\pgfqpoint{2.407767in}{1.544118in}}%
\pgfusepath{clip}%
\pgfsetbuttcap%
\pgfsetroundjoin%
\pgfsetlinewidth{0.501875pt}%
\definecolor{currentstroke}{rgb}{0.268510,0.009605,0.335427}%
\pgfsetstrokecolor{currentstroke}%
\pgfsetdash{}{0pt}%
\pgfpathmoveto{\pgfqpoint{4.509160in}{1.558562in}}%
\pgfpathlineto{\pgfqpoint{4.509158in}{1.558562in}}%
\pgfusepath{stroke}%
\end{pgfscope}%
\begin{pgfscope}%
\pgfpathrectangle{\pgfqpoint{3.352233in}{1.400000in}}{\pgfqpoint{2.407767in}{1.544118in}}%
\pgfusepath{clip}%
\pgfsetbuttcap%
\pgfsetroundjoin%
\pgfsetlinewidth{0.501875pt}%
\definecolor{currentstroke}{rgb}{0.268510,0.009605,0.335427}%
\pgfsetstrokecolor{currentstroke}%
\pgfsetdash{}{0pt}%
\pgfpathmoveto{\pgfqpoint{4.509158in}{1.558562in}}%
\pgfpathlineto{\pgfqpoint{4.509159in}{1.558562in}}%
\pgfusepath{stroke}%
\end{pgfscope}%
\begin{pgfscope}%
\pgfpathrectangle{\pgfqpoint{3.352233in}{1.400000in}}{\pgfqpoint{2.407767in}{1.544118in}}%
\pgfusepath{clip}%
\pgfsetbuttcap%
\pgfsetroundjoin%
\pgfsetlinewidth{0.501875pt}%
\definecolor{currentstroke}{rgb}{0.268510,0.009605,0.335427}%
\pgfsetstrokecolor{currentstroke}%
\pgfsetdash{}{0pt}%
\pgfpathmoveto{\pgfqpoint{4.509159in}{1.558562in}}%
\pgfpathlineto{\pgfqpoint{4.509160in}{1.558561in}}%
\pgfusepath{stroke}%
\end{pgfscope}%
\begin{pgfscope}%
\pgfpathrectangle{\pgfqpoint{3.352233in}{1.400000in}}{\pgfqpoint{2.407767in}{1.544118in}}%
\pgfusepath{clip}%
\pgfsetbuttcap%
\pgfsetroundjoin%
\pgfsetlinewidth{0.501875pt}%
\definecolor{currentstroke}{rgb}{0.268510,0.009605,0.335427}%
\pgfsetstrokecolor{currentstroke}%
\pgfsetdash{}{0pt}%
\pgfpathmoveto{\pgfqpoint{4.509160in}{1.558561in}}%
\pgfpathlineto{\pgfqpoint{4.509161in}{1.558561in}}%
\pgfusepath{stroke}%
\end{pgfscope}%
\begin{pgfscope}%
\pgfpathrectangle{\pgfqpoint{3.352233in}{1.400000in}}{\pgfqpoint{2.407767in}{1.544118in}}%
\pgfusepath{clip}%
\pgfsetbuttcap%
\pgfsetroundjoin%
\pgfsetlinewidth{0.501875pt}%
\definecolor{currentstroke}{rgb}{0.268510,0.009605,0.335427}%
\pgfsetstrokecolor{currentstroke}%
\pgfsetdash{}{0pt}%
\pgfpathmoveto{\pgfqpoint{4.509161in}{1.558561in}}%
\pgfpathlineto{\pgfqpoint{4.509160in}{1.558562in}}%
\pgfusepath{stroke}%
\end{pgfscope}%
\begin{pgfscope}%
\pgfpathrectangle{\pgfqpoint{3.352233in}{1.400000in}}{\pgfqpoint{2.407767in}{1.544118in}}%
\pgfusepath{clip}%
\pgfsetbuttcap%
\pgfsetroundjoin%
\pgfsetlinewidth{0.501875pt}%
\definecolor{currentstroke}{rgb}{0.268510,0.009605,0.335427}%
\pgfsetstrokecolor{currentstroke}%
\pgfsetdash{}{0pt}%
\pgfpathmoveto{\pgfqpoint{4.509160in}{1.558562in}}%
\pgfpathlineto{\pgfqpoint{4.509158in}{1.558562in}}%
\pgfusepath{stroke}%
\end{pgfscope}%
\begin{pgfscope}%
\pgfpathrectangle{\pgfqpoint{3.352233in}{1.400000in}}{\pgfqpoint{2.407767in}{1.544118in}}%
\pgfusepath{clip}%
\pgfsetbuttcap%
\pgfsetroundjoin%
\pgfsetlinewidth{0.501875pt}%
\definecolor{currentstroke}{rgb}{0.268510,0.009605,0.335427}%
\pgfsetstrokecolor{currentstroke}%
\pgfsetdash{}{0pt}%
\pgfpathmoveto{\pgfqpoint{4.509158in}{1.558562in}}%
\pgfpathlineto{\pgfqpoint{4.509159in}{1.558562in}}%
\pgfusepath{stroke}%
\end{pgfscope}%
\begin{pgfscope}%
\pgfpathrectangle{\pgfqpoint{3.352233in}{1.400000in}}{\pgfqpoint{2.407767in}{1.544118in}}%
\pgfusepath{clip}%
\pgfsetbuttcap%
\pgfsetroundjoin%
\pgfsetlinewidth{0.501875pt}%
\definecolor{currentstroke}{rgb}{0.268510,0.009605,0.335427}%
\pgfsetstrokecolor{currentstroke}%
\pgfsetdash{}{0pt}%
\pgfpathmoveto{\pgfqpoint{4.509159in}{1.558562in}}%
\pgfpathlineto{\pgfqpoint{4.509160in}{1.558561in}}%
\pgfusepath{stroke}%
\end{pgfscope}%
\begin{pgfscope}%
\pgfpathrectangle{\pgfqpoint{3.352233in}{1.400000in}}{\pgfqpoint{2.407767in}{1.544118in}}%
\pgfusepath{clip}%
\pgfsetbuttcap%
\pgfsetroundjoin%
\pgfsetlinewidth{0.501875pt}%
\definecolor{currentstroke}{rgb}{0.268510,0.009605,0.335427}%
\pgfsetstrokecolor{currentstroke}%
\pgfsetdash{}{0pt}%
\pgfpathmoveto{\pgfqpoint{4.509160in}{1.558561in}}%
\pgfpathlineto{\pgfqpoint{4.509161in}{1.558561in}}%
\pgfusepath{stroke}%
\end{pgfscope}%
\begin{pgfscope}%
\pgfpathrectangle{\pgfqpoint{3.352233in}{1.400000in}}{\pgfqpoint{2.407767in}{1.544118in}}%
\pgfusepath{clip}%
\pgfsetbuttcap%
\pgfsetroundjoin%
\pgfsetlinewidth{0.501875pt}%
\definecolor{currentstroke}{rgb}{0.268510,0.009605,0.335427}%
\pgfsetstrokecolor{currentstroke}%
\pgfsetdash{}{0pt}%
\pgfpathmoveto{\pgfqpoint{4.509161in}{1.558561in}}%
\pgfpathlineto{\pgfqpoint{4.509160in}{1.558562in}}%
\pgfusepath{stroke}%
\end{pgfscope}%
\begin{pgfscope}%
\pgfpathrectangle{\pgfqpoint{3.352233in}{1.400000in}}{\pgfqpoint{2.407767in}{1.544118in}}%
\pgfusepath{clip}%
\pgfsetbuttcap%
\pgfsetroundjoin%
\pgfsetlinewidth{0.501875pt}%
\definecolor{currentstroke}{rgb}{0.268510,0.009605,0.335427}%
\pgfsetstrokecolor{currentstroke}%
\pgfsetdash{}{0pt}%
\pgfpathmoveto{\pgfqpoint{4.509160in}{1.558562in}}%
\pgfpathlineto{\pgfqpoint{4.509159in}{1.558562in}}%
\pgfusepath{stroke}%
\end{pgfscope}%
\begin{pgfscope}%
\pgfpathrectangle{\pgfqpoint{3.352233in}{1.400000in}}{\pgfqpoint{2.407767in}{1.544118in}}%
\pgfusepath{clip}%
\pgfsetbuttcap%
\pgfsetroundjoin%
\pgfsetlinewidth{0.501875pt}%
\definecolor{currentstroke}{rgb}{0.268510,0.009605,0.335427}%
\pgfsetstrokecolor{currentstroke}%
\pgfsetdash{}{0pt}%
\pgfpathmoveto{\pgfqpoint{4.509159in}{1.558562in}}%
\pgfpathlineto{\pgfqpoint{4.509159in}{1.558562in}}%
\pgfusepath{stroke}%
\end{pgfscope}%
\begin{pgfscope}%
\pgfpathrectangle{\pgfqpoint{3.352233in}{1.400000in}}{\pgfqpoint{2.407767in}{1.544118in}}%
\pgfusepath{clip}%
\pgfsetbuttcap%
\pgfsetroundjoin%
\pgfsetlinewidth{0.501875pt}%
\definecolor{currentstroke}{rgb}{0.268510,0.009605,0.335427}%
\pgfsetstrokecolor{currentstroke}%
\pgfsetdash{}{0pt}%
\pgfpathmoveto{\pgfqpoint{4.509159in}{1.558562in}}%
\pgfpathlineto{\pgfqpoint{4.509160in}{1.558561in}}%
\pgfusepath{stroke}%
\end{pgfscope}%
\begin{pgfscope}%
\pgfpathrectangle{\pgfqpoint{3.352233in}{1.400000in}}{\pgfqpoint{2.407767in}{1.544118in}}%
\pgfusepath{clip}%
\pgfsetbuttcap%
\pgfsetroundjoin%
\pgfsetlinewidth{0.501875pt}%
\definecolor{currentstroke}{rgb}{0.268510,0.009605,0.335427}%
\pgfsetstrokecolor{currentstroke}%
\pgfsetdash{}{0pt}%
\pgfpathmoveto{\pgfqpoint{4.509160in}{1.558561in}}%
\pgfpathlineto{\pgfqpoint{4.509161in}{1.558561in}}%
\pgfusepath{stroke}%
\end{pgfscope}%
\begin{pgfscope}%
\pgfpathrectangle{\pgfqpoint{3.352233in}{1.400000in}}{\pgfqpoint{2.407767in}{1.544118in}}%
\pgfusepath{clip}%
\pgfsetbuttcap%
\pgfsetroundjoin%
\pgfsetlinewidth{0.501875pt}%
\definecolor{currentstroke}{rgb}{0.268510,0.009605,0.335427}%
\pgfsetstrokecolor{currentstroke}%
\pgfsetdash{}{0pt}%
\pgfpathmoveto{\pgfqpoint{4.509161in}{1.558561in}}%
\pgfpathlineto{\pgfqpoint{4.509160in}{1.558562in}}%
\pgfusepath{stroke}%
\end{pgfscope}%
\begin{pgfscope}%
\pgfpathrectangle{\pgfqpoint{3.352233in}{1.400000in}}{\pgfqpoint{2.407767in}{1.544118in}}%
\pgfusepath{clip}%
\pgfsetbuttcap%
\pgfsetroundjoin%
\pgfsetlinewidth{0.501875pt}%
\definecolor{currentstroke}{rgb}{0.268510,0.009605,0.335427}%
\pgfsetstrokecolor{currentstroke}%
\pgfsetdash{}{0pt}%
\pgfpathmoveto{\pgfqpoint{4.509160in}{1.558562in}}%
\pgfpathlineto{\pgfqpoint{4.509159in}{1.558562in}}%
\pgfusepath{stroke}%
\end{pgfscope}%
\begin{pgfscope}%
\pgfpathrectangle{\pgfqpoint{3.352233in}{1.400000in}}{\pgfqpoint{2.407767in}{1.544118in}}%
\pgfusepath{clip}%
\pgfsetbuttcap%
\pgfsetroundjoin%
\pgfsetlinewidth{0.501875pt}%
\definecolor{currentstroke}{rgb}{0.268510,0.009605,0.335427}%
\pgfsetstrokecolor{currentstroke}%
\pgfsetdash{}{0pt}%
\pgfpathmoveto{\pgfqpoint{4.509159in}{1.558562in}}%
\pgfpathlineto{\pgfqpoint{4.509159in}{1.558562in}}%
\pgfusepath{stroke}%
\end{pgfscope}%
\begin{pgfscope}%
\pgfpathrectangle{\pgfqpoint{3.352233in}{1.400000in}}{\pgfqpoint{2.407767in}{1.544118in}}%
\pgfusepath{clip}%
\pgfsetbuttcap%
\pgfsetroundjoin%
\pgfsetlinewidth{0.501875pt}%
\definecolor{currentstroke}{rgb}{0.268510,0.009605,0.335427}%
\pgfsetstrokecolor{currentstroke}%
\pgfsetdash{}{0pt}%
\pgfpathmoveto{\pgfqpoint{4.509159in}{1.558562in}}%
\pgfpathlineto{\pgfqpoint{4.509160in}{1.558561in}}%
\pgfusepath{stroke}%
\end{pgfscope}%
\begin{pgfscope}%
\pgfpathrectangle{\pgfqpoint{3.352233in}{1.400000in}}{\pgfqpoint{2.407767in}{1.544118in}}%
\pgfusepath{clip}%
\pgfsetbuttcap%
\pgfsetroundjoin%
\pgfsetlinewidth{0.501875pt}%
\definecolor{currentstroke}{rgb}{0.268510,0.009605,0.335427}%
\pgfsetstrokecolor{currentstroke}%
\pgfsetdash{}{0pt}%
\pgfpathmoveto{\pgfqpoint{4.509160in}{1.558561in}}%
\pgfpathlineto{\pgfqpoint{4.509160in}{1.558561in}}%
\pgfusepath{stroke}%
\end{pgfscope}%
\begin{pgfscope}%
\pgfpathrectangle{\pgfqpoint{3.352233in}{1.400000in}}{\pgfqpoint{2.407767in}{1.544118in}}%
\pgfusepath{clip}%
\pgfsetbuttcap%
\pgfsetroundjoin%
\pgfsetlinewidth{0.501875pt}%
\definecolor{currentstroke}{rgb}{0.268510,0.009605,0.335427}%
\pgfsetstrokecolor{currentstroke}%
\pgfsetdash{}{0pt}%
\pgfpathmoveto{\pgfqpoint{4.509160in}{1.558561in}}%
\pgfpathlineto{\pgfqpoint{4.509160in}{1.558562in}}%
\pgfusepath{stroke}%
\end{pgfscope}%
\begin{pgfscope}%
\pgfpathrectangle{\pgfqpoint{3.352233in}{1.400000in}}{\pgfqpoint{2.407767in}{1.544118in}}%
\pgfusepath{clip}%
\pgfsetbuttcap%
\pgfsetroundjoin%
\pgfsetlinewidth{0.501875pt}%
\definecolor{currentstroke}{rgb}{0.268510,0.009605,0.335427}%
\pgfsetstrokecolor{currentstroke}%
\pgfsetdash{}{0pt}%
\pgfpathmoveto{\pgfqpoint{4.509160in}{1.558562in}}%
\pgfpathlineto{\pgfqpoint{4.509159in}{1.558562in}}%
\pgfusepath{stroke}%
\end{pgfscope}%
\begin{pgfscope}%
\pgfpathrectangle{\pgfqpoint{3.352233in}{1.400000in}}{\pgfqpoint{2.407767in}{1.544118in}}%
\pgfusepath{clip}%
\pgfsetbuttcap%
\pgfsetroundjoin%
\pgfsetlinewidth{0.501875pt}%
\definecolor{currentstroke}{rgb}{0.268510,0.009605,0.335427}%
\pgfsetstrokecolor{currentstroke}%
\pgfsetdash{}{0pt}%
\pgfpathmoveto{\pgfqpoint{4.509159in}{1.558562in}}%
\pgfpathlineto{\pgfqpoint{4.509159in}{1.558562in}}%
\pgfusepath{stroke}%
\end{pgfscope}%
\begin{pgfscope}%
\pgfpathrectangle{\pgfqpoint{3.352233in}{1.400000in}}{\pgfqpoint{2.407767in}{1.544118in}}%
\pgfusepath{clip}%
\pgfsetbuttcap%
\pgfsetroundjoin%
\pgfsetlinewidth{0.501875pt}%
\definecolor{currentstroke}{rgb}{0.268510,0.009605,0.335427}%
\pgfsetstrokecolor{currentstroke}%
\pgfsetdash{}{0pt}%
\pgfpathmoveto{\pgfqpoint{4.509159in}{1.558562in}}%
\pgfpathlineto{\pgfqpoint{4.509160in}{1.558561in}}%
\pgfusepath{stroke}%
\end{pgfscope}%
\begin{pgfscope}%
\pgfpathrectangle{\pgfqpoint{3.352233in}{1.400000in}}{\pgfqpoint{2.407767in}{1.544118in}}%
\pgfusepath{clip}%
\pgfsetbuttcap%
\pgfsetroundjoin%
\pgfsetlinewidth{0.501875pt}%
\definecolor{currentstroke}{rgb}{0.268510,0.009605,0.335427}%
\pgfsetstrokecolor{currentstroke}%
\pgfsetdash{}{0pt}%
\pgfpathmoveto{\pgfqpoint{4.509160in}{1.558561in}}%
\pgfpathlineto{\pgfqpoint{4.509160in}{1.558561in}}%
\pgfusepath{stroke}%
\end{pgfscope}%
\begin{pgfscope}%
\pgfpathrectangle{\pgfqpoint{3.352233in}{1.400000in}}{\pgfqpoint{2.407767in}{1.544118in}}%
\pgfusepath{clip}%
\pgfsetbuttcap%
\pgfsetroundjoin%
\pgfsetlinewidth{0.501875pt}%
\definecolor{currentstroke}{rgb}{0.268510,0.009605,0.335427}%
\pgfsetstrokecolor{currentstroke}%
\pgfsetdash{}{0pt}%
\pgfpathmoveto{\pgfqpoint{4.509160in}{1.558561in}}%
\pgfpathlineto{\pgfqpoint{4.509160in}{1.558562in}}%
\pgfusepath{stroke}%
\end{pgfscope}%
\begin{pgfscope}%
\pgfpathrectangle{\pgfqpoint{3.352233in}{1.400000in}}{\pgfqpoint{2.407767in}{1.544118in}}%
\pgfusepath{clip}%
\pgfsetbuttcap%
\pgfsetroundjoin%
\pgfsetlinewidth{0.501875pt}%
\definecolor{currentstroke}{rgb}{0.268510,0.009605,0.335427}%
\pgfsetstrokecolor{currentstroke}%
\pgfsetdash{}{0pt}%
\pgfpathmoveto{\pgfqpoint{4.509160in}{1.558562in}}%
\pgfpathlineto{\pgfqpoint{4.509159in}{1.558562in}}%
\pgfusepath{stroke}%
\end{pgfscope}%
\begin{pgfscope}%
\pgfpathrectangle{\pgfqpoint{3.352233in}{1.400000in}}{\pgfqpoint{2.407767in}{1.544118in}}%
\pgfusepath{clip}%
\pgfsetbuttcap%
\pgfsetroundjoin%
\pgfsetlinewidth{0.501875pt}%
\definecolor{currentstroke}{rgb}{0.268510,0.009605,0.335427}%
\pgfsetstrokecolor{currentstroke}%
\pgfsetdash{}{0pt}%
\pgfpathmoveto{\pgfqpoint{4.509159in}{1.558562in}}%
\pgfpathlineto{\pgfqpoint{4.509159in}{1.558562in}}%
\pgfusepath{stroke}%
\end{pgfscope}%
\begin{pgfscope}%
\pgfpathrectangle{\pgfqpoint{3.352233in}{1.400000in}}{\pgfqpoint{2.407767in}{1.544118in}}%
\pgfusepath{clip}%
\pgfsetbuttcap%
\pgfsetroundjoin%
\pgfsetlinewidth{0.501875pt}%
\definecolor{currentstroke}{rgb}{0.268510,0.009605,0.335427}%
\pgfsetstrokecolor{currentstroke}%
\pgfsetdash{}{0pt}%
\pgfpathmoveto{\pgfqpoint{4.509159in}{1.558562in}}%
\pgfpathlineto{\pgfqpoint{4.509160in}{1.558561in}}%
\pgfusepath{stroke}%
\end{pgfscope}%
\begin{pgfscope}%
\pgfpathrectangle{\pgfqpoint{3.352233in}{1.400000in}}{\pgfqpoint{2.407767in}{1.544118in}}%
\pgfusepath{clip}%
\pgfsetbuttcap%
\pgfsetroundjoin%
\pgfsetlinewidth{0.501875pt}%
\definecolor{currentstroke}{rgb}{0.268510,0.009605,0.335427}%
\pgfsetstrokecolor{currentstroke}%
\pgfsetdash{}{0pt}%
\pgfpathmoveto{\pgfqpoint{4.509160in}{1.558561in}}%
\pgfpathlineto{\pgfqpoint{4.509160in}{1.558561in}}%
\pgfusepath{stroke}%
\end{pgfscope}%
\begin{pgfscope}%
\pgfpathrectangle{\pgfqpoint{3.352233in}{1.400000in}}{\pgfqpoint{2.407767in}{1.544118in}}%
\pgfusepath{clip}%
\pgfsetbuttcap%
\pgfsetroundjoin%
\pgfsetlinewidth{0.501875pt}%
\definecolor{currentstroke}{rgb}{0.268510,0.009605,0.335427}%
\pgfsetstrokecolor{currentstroke}%
\pgfsetdash{}{0pt}%
\pgfpathmoveto{\pgfqpoint{4.509160in}{1.558561in}}%
\pgfpathlineto{\pgfqpoint{4.509160in}{1.558561in}}%
\pgfusepath{stroke}%
\end{pgfscope}%
\begin{pgfscope}%
\pgfpathrectangle{\pgfqpoint{3.352233in}{1.400000in}}{\pgfqpoint{2.407767in}{1.544118in}}%
\pgfusepath{clip}%
\pgfsetbuttcap%
\pgfsetroundjoin%
\pgfsetlinewidth{0.501875pt}%
\definecolor{currentstroke}{rgb}{0.268510,0.009605,0.335427}%
\pgfsetstrokecolor{currentstroke}%
\pgfsetdash{}{0pt}%
\pgfpathmoveto{\pgfqpoint{4.509160in}{1.558561in}}%
\pgfpathlineto{\pgfqpoint{4.509159in}{1.558562in}}%
\pgfusepath{stroke}%
\end{pgfscope}%
\begin{pgfscope}%
\pgfpathrectangle{\pgfqpoint{3.352233in}{1.400000in}}{\pgfqpoint{2.407767in}{1.544118in}}%
\pgfusepath{clip}%
\pgfsetbuttcap%
\pgfsetroundjoin%
\pgfsetlinewidth{0.501875pt}%
\definecolor{currentstroke}{rgb}{0.268510,0.009605,0.335427}%
\pgfsetstrokecolor{currentstroke}%
\pgfsetdash{}{0pt}%
\pgfpathmoveto{\pgfqpoint{4.509159in}{1.558562in}}%
\pgfpathlineto{\pgfqpoint{4.509159in}{1.558562in}}%
\pgfusepath{stroke}%
\end{pgfscope}%
\begin{pgfscope}%
\pgfpathrectangle{\pgfqpoint{3.352233in}{1.400000in}}{\pgfqpoint{2.407767in}{1.544118in}}%
\pgfusepath{clip}%
\pgfsetbuttcap%
\pgfsetroundjoin%
\pgfsetlinewidth{0.501875pt}%
\definecolor{currentstroke}{rgb}{0.268510,0.009605,0.335427}%
\pgfsetstrokecolor{currentstroke}%
\pgfsetdash{}{0pt}%
\pgfpathmoveto{\pgfqpoint{4.509159in}{1.558562in}}%
\pgfpathlineto{\pgfqpoint{4.509160in}{1.558561in}}%
\pgfusepath{stroke}%
\end{pgfscope}%
\begin{pgfscope}%
\pgfpathrectangle{\pgfqpoint{3.352233in}{1.400000in}}{\pgfqpoint{2.407767in}{1.544118in}}%
\pgfusepath{clip}%
\pgfsetbuttcap%
\pgfsetroundjoin%
\pgfsetlinewidth{0.501875pt}%
\definecolor{currentstroke}{rgb}{0.268510,0.009605,0.335427}%
\pgfsetstrokecolor{currentstroke}%
\pgfsetdash{}{0pt}%
\pgfpathmoveto{\pgfqpoint{4.509160in}{1.558561in}}%
\pgfpathlineto{\pgfqpoint{4.509160in}{1.558561in}}%
\pgfusepath{stroke}%
\end{pgfscope}%
\begin{pgfscope}%
\pgfpathrectangle{\pgfqpoint{3.352233in}{1.400000in}}{\pgfqpoint{2.407767in}{1.544118in}}%
\pgfusepath{clip}%
\pgfsetbuttcap%
\pgfsetroundjoin%
\pgfsetlinewidth{0.501875pt}%
\definecolor{currentstroke}{rgb}{0.268510,0.009605,0.335427}%
\pgfsetstrokecolor{currentstroke}%
\pgfsetdash{}{0pt}%
\pgfpathmoveto{\pgfqpoint{4.509160in}{1.558561in}}%
\pgfpathlineto{\pgfqpoint{4.509160in}{1.558561in}}%
\pgfusepath{stroke}%
\end{pgfscope}%
\begin{pgfscope}%
\pgfpathrectangle{\pgfqpoint{3.352233in}{1.400000in}}{\pgfqpoint{2.407767in}{1.544118in}}%
\pgfusepath{clip}%
\pgfsetbuttcap%
\pgfsetroundjoin%
\pgfsetlinewidth{0.501875pt}%
\definecolor{currentstroke}{rgb}{0.268510,0.009605,0.335427}%
\pgfsetstrokecolor{currentstroke}%
\pgfsetdash{}{0pt}%
\pgfpathmoveto{\pgfqpoint{4.509160in}{1.558561in}}%
\pgfpathlineto{\pgfqpoint{4.509159in}{1.558562in}}%
\pgfusepath{stroke}%
\end{pgfscope}%
\begin{pgfscope}%
\pgfpathrectangle{\pgfqpoint{3.352233in}{1.400000in}}{\pgfqpoint{2.407767in}{1.544118in}}%
\pgfusepath{clip}%
\pgfsetbuttcap%
\pgfsetroundjoin%
\pgfsetlinewidth{0.501875pt}%
\definecolor{currentstroke}{rgb}{0.268510,0.009605,0.335427}%
\pgfsetstrokecolor{currentstroke}%
\pgfsetdash{}{0pt}%
\pgfpathmoveto{\pgfqpoint{4.509159in}{1.558562in}}%
\pgfpathlineto{\pgfqpoint{4.509159in}{1.558562in}}%
\pgfusepath{stroke}%
\end{pgfscope}%
\begin{pgfscope}%
\pgfpathrectangle{\pgfqpoint{3.352233in}{1.400000in}}{\pgfqpoint{2.407767in}{1.544118in}}%
\pgfusepath{clip}%
\pgfsetbuttcap%
\pgfsetroundjoin%
\pgfsetlinewidth{0.501875pt}%
\definecolor{currentstroke}{rgb}{0.268510,0.009605,0.335427}%
\pgfsetstrokecolor{currentstroke}%
\pgfsetdash{}{0pt}%
\pgfpathmoveto{\pgfqpoint{4.509159in}{1.558562in}}%
\pgfpathlineto{\pgfqpoint{4.509160in}{1.558561in}}%
\pgfusepath{stroke}%
\end{pgfscope}%
\begin{pgfscope}%
\pgfpathrectangle{\pgfqpoint{3.352233in}{1.400000in}}{\pgfqpoint{2.407767in}{1.544118in}}%
\pgfusepath{clip}%
\pgfsetbuttcap%
\pgfsetroundjoin%
\pgfsetlinewidth{0.501875pt}%
\definecolor{currentstroke}{rgb}{0.268510,0.009605,0.335427}%
\pgfsetstrokecolor{currentstroke}%
\pgfsetdash{}{0pt}%
\pgfpathmoveto{\pgfqpoint{4.509160in}{1.558561in}}%
\pgfpathlineto{\pgfqpoint{4.509160in}{1.558561in}}%
\pgfusepath{stroke}%
\end{pgfscope}%
\begin{pgfscope}%
\pgfpathrectangle{\pgfqpoint{3.352233in}{1.400000in}}{\pgfqpoint{2.407767in}{1.544118in}}%
\pgfusepath{clip}%
\pgfsetbuttcap%
\pgfsetroundjoin%
\pgfsetlinewidth{0.501875pt}%
\definecolor{currentstroke}{rgb}{0.268510,0.009605,0.335427}%
\pgfsetstrokecolor{currentstroke}%
\pgfsetdash{}{0pt}%
\pgfpathmoveto{\pgfqpoint{4.509160in}{1.558561in}}%
\pgfpathlineto{\pgfqpoint{4.509160in}{1.558561in}}%
\pgfusepath{stroke}%
\end{pgfscope}%
\begin{pgfscope}%
\pgfpathrectangle{\pgfqpoint{3.352233in}{1.400000in}}{\pgfqpoint{2.407767in}{1.544118in}}%
\pgfusepath{clip}%
\pgfsetbuttcap%
\pgfsetroundjoin%
\pgfsetlinewidth{0.501875pt}%
\definecolor{currentstroke}{rgb}{0.268510,0.009605,0.335427}%
\pgfsetstrokecolor{currentstroke}%
\pgfsetdash{}{0pt}%
\pgfpathmoveto{\pgfqpoint{4.509160in}{1.558561in}}%
\pgfpathlineto{\pgfqpoint{4.509159in}{1.558562in}}%
\pgfusepath{stroke}%
\end{pgfscope}%
\begin{pgfscope}%
\pgfpathrectangle{\pgfqpoint{3.352233in}{1.400000in}}{\pgfqpoint{2.407767in}{1.544118in}}%
\pgfusepath{clip}%
\pgfsetbuttcap%
\pgfsetroundjoin%
\pgfsetlinewidth{0.501875pt}%
\definecolor{currentstroke}{rgb}{0.268510,0.009605,0.335427}%
\pgfsetstrokecolor{currentstroke}%
\pgfsetdash{}{0pt}%
\pgfpathmoveto{\pgfqpoint{4.509159in}{1.558562in}}%
\pgfpathlineto{\pgfqpoint{4.509159in}{1.558562in}}%
\pgfusepath{stroke}%
\end{pgfscope}%
\begin{pgfscope}%
\pgfpathrectangle{\pgfqpoint{3.352233in}{1.400000in}}{\pgfqpoint{2.407767in}{1.544118in}}%
\pgfusepath{clip}%
\pgfsetbuttcap%
\pgfsetroundjoin%
\pgfsetlinewidth{0.501875pt}%
\definecolor{currentstroke}{rgb}{0.268510,0.009605,0.335427}%
\pgfsetstrokecolor{currentstroke}%
\pgfsetdash{}{0pt}%
\pgfpathmoveto{\pgfqpoint{4.509159in}{1.558562in}}%
\pgfpathlineto{\pgfqpoint{4.509160in}{1.558561in}}%
\pgfusepath{stroke}%
\end{pgfscope}%
\begin{pgfscope}%
\pgfpathrectangle{\pgfqpoint{3.352233in}{1.400000in}}{\pgfqpoint{2.407767in}{1.544118in}}%
\pgfusepath{clip}%
\pgfsetbuttcap%
\pgfsetroundjoin%
\pgfsetlinewidth{0.501875pt}%
\definecolor{currentstroke}{rgb}{0.268510,0.009605,0.335427}%
\pgfsetstrokecolor{currentstroke}%
\pgfsetdash{}{0pt}%
\pgfpathmoveto{\pgfqpoint{4.509160in}{1.558561in}}%
\pgfpathlineto{\pgfqpoint{4.509160in}{1.558561in}}%
\pgfusepath{stroke}%
\end{pgfscope}%
\begin{pgfscope}%
\pgfpathrectangle{\pgfqpoint{3.352233in}{1.400000in}}{\pgfqpoint{2.407767in}{1.544118in}}%
\pgfusepath{clip}%
\pgfsetbuttcap%
\pgfsetroundjoin%
\pgfsetlinewidth{0.501875pt}%
\definecolor{currentstroke}{rgb}{0.268510,0.009605,0.335427}%
\pgfsetstrokecolor{currentstroke}%
\pgfsetdash{}{0pt}%
\pgfpathmoveto{\pgfqpoint{4.509160in}{1.558561in}}%
\pgfpathlineto{\pgfqpoint{4.509160in}{1.558561in}}%
\pgfusepath{stroke}%
\end{pgfscope}%
\begin{pgfscope}%
\pgfpathrectangle{\pgfqpoint{3.352233in}{1.400000in}}{\pgfqpoint{2.407767in}{1.544118in}}%
\pgfusepath{clip}%
\pgfsetbuttcap%
\pgfsetroundjoin%
\pgfsetlinewidth{0.501875pt}%
\definecolor{currentstroke}{rgb}{0.268510,0.009605,0.335427}%
\pgfsetstrokecolor{currentstroke}%
\pgfsetdash{}{0pt}%
\pgfpathmoveto{\pgfqpoint{4.509160in}{1.558561in}}%
\pgfpathlineto{\pgfqpoint{4.509159in}{1.558562in}}%
\pgfusepath{stroke}%
\end{pgfscope}%
\begin{pgfscope}%
\pgfpathrectangle{\pgfqpoint{3.352233in}{1.400000in}}{\pgfqpoint{2.407767in}{1.544118in}}%
\pgfusepath{clip}%
\pgfsetbuttcap%
\pgfsetroundjoin%
\pgfsetlinewidth{0.501875pt}%
\definecolor{currentstroke}{rgb}{0.268510,0.009605,0.335427}%
\pgfsetstrokecolor{currentstroke}%
\pgfsetdash{}{0pt}%
\pgfpathmoveto{\pgfqpoint{4.509159in}{1.558562in}}%
\pgfpathlineto{\pgfqpoint{4.509159in}{1.558562in}}%
\pgfusepath{stroke}%
\end{pgfscope}%
\begin{pgfscope}%
\pgfpathrectangle{\pgfqpoint{3.352233in}{1.400000in}}{\pgfqpoint{2.407767in}{1.544118in}}%
\pgfusepath{clip}%
\pgfsetbuttcap%
\pgfsetroundjoin%
\pgfsetlinewidth{0.501875pt}%
\definecolor{currentstroke}{rgb}{0.268510,0.009605,0.335427}%
\pgfsetstrokecolor{currentstroke}%
\pgfsetdash{}{0pt}%
\pgfpathmoveto{\pgfqpoint{4.509159in}{1.558562in}}%
\pgfpathlineto{\pgfqpoint{4.509160in}{1.558561in}}%
\pgfusepath{stroke}%
\end{pgfscope}%
\begin{pgfscope}%
\pgfpathrectangle{\pgfqpoint{3.352233in}{1.400000in}}{\pgfqpoint{2.407767in}{1.544118in}}%
\pgfusepath{clip}%
\pgfsetbuttcap%
\pgfsetroundjoin%
\pgfsetlinewidth{0.501875pt}%
\definecolor{currentstroke}{rgb}{0.268510,0.009605,0.335427}%
\pgfsetstrokecolor{currentstroke}%
\pgfsetdash{}{0pt}%
\pgfpathmoveto{\pgfqpoint{4.509160in}{1.558561in}}%
\pgfpathlineto{\pgfqpoint{4.509160in}{1.558561in}}%
\pgfusepath{stroke}%
\end{pgfscope}%
\begin{pgfscope}%
\pgfpathrectangle{\pgfqpoint{3.352233in}{1.400000in}}{\pgfqpoint{2.407767in}{1.544118in}}%
\pgfusepath{clip}%
\pgfsetbuttcap%
\pgfsetroundjoin%
\pgfsetlinewidth{0.501875pt}%
\definecolor{currentstroke}{rgb}{0.268510,0.009605,0.335427}%
\pgfsetstrokecolor{currentstroke}%
\pgfsetdash{}{0pt}%
\pgfpathmoveto{\pgfqpoint{4.509160in}{1.558561in}}%
\pgfpathlineto{\pgfqpoint{4.509160in}{1.558561in}}%
\pgfusepath{stroke}%
\end{pgfscope}%
\begin{pgfscope}%
\pgfpathrectangle{\pgfqpoint{3.352233in}{1.400000in}}{\pgfqpoint{2.407767in}{1.544118in}}%
\pgfusepath{clip}%
\pgfsetbuttcap%
\pgfsetroundjoin%
\pgfsetlinewidth{0.501875pt}%
\definecolor{currentstroke}{rgb}{0.268510,0.009605,0.335427}%
\pgfsetstrokecolor{currentstroke}%
\pgfsetdash{}{0pt}%
\pgfpathmoveto{\pgfqpoint{4.509160in}{1.558561in}}%
\pgfpathlineto{\pgfqpoint{4.509159in}{1.558562in}}%
\pgfusepath{stroke}%
\end{pgfscope}%
\begin{pgfscope}%
\pgfpathrectangle{\pgfqpoint{3.352233in}{1.400000in}}{\pgfqpoint{2.407767in}{1.544118in}}%
\pgfusepath{clip}%
\pgfsetbuttcap%
\pgfsetroundjoin%
\pgfsetlinewidth{0.501875pt}%
\definecolor{currentstroke}{rgb}{0.268510,0.009605,0.335427}%
\pgfsetstrokecolor{currentstroke}%
\pgfsetdash{}{0pt}%
\pgfpathmoveto{\pgfqpoint{4.509159in}{1.558562in}}%
\pgfpathlineto{\pgfqpoint{4.509159in}{1.558562in}}%
\pgfusepath{stroke}%
\end{pgfscope}%
\begin{pgfscope}%
\pgfpathrectangle{\pgfqpoint{3.352233in}{1.400000in}}{\pgfqpoint{2.407767in}{1.544118in}}%
\pgfusepath{clip}%
\pgfsetbuttcap%
\pgfsetroundjoin%
\pgfsetlinewidth{0.501875pt}%
\definecolor{currentstroke}{rgb}{0.268510,0.009605,0.335427}%
\pgfsetstrokecolor{currentstroke}%
\pgfsetdash{}{0pt}%
\pgfpathmoveto{\pgfqpoint{4.509159in}{1.558562in}}%
\pgfpathlineto{\pgfqpoint{4.509160in}{1.558561in}}%
\pgfusepath{stroke}%
\end{pgfscope}%
\begin{pgfscope}%
\pgfpathrectangle{\pgfqpoint{3.352233in}{1.400000in}}{\pgfqpoint{2.407767in}{1.544118in}}%
\pgfusepath{clip}%
\pgfsetbuttcap%
\pgfsetroundjoin%
\pgfsetlinewidth{0.501875pt}%
\definecolor{currentstroke}{rgb}{0.268510,0.009605,0.335427}%
\pgfsetstrokecolor{currentstroke}%
\pgfsetdash{}{0pt}%
\pgfpathmoveto{\pgfqpoint{4.509160in}{1.558561in}}%
\pgfpathlineto{\pgfqpoint{4.509160in}{1.558561in}}%
\pgfusepath{stroke}%
\end{pgfscope}%
\begin{pgfscope}%
\pgfpathrectangle{\pgfqpoint{3.352233in}{1.400000in}}{\pgfqpoint{2.407767in}{1.544118in}}%
\pgfusepath{clip}%
\pgfsetbuttcap%
\pgfsetroundjoin%
\pgfsetlinewidth{0.501875pt}%
\definecolor{currentstroke}{rgb}{0.268510,0.009605,0.335427}%
\pgfsetstrokecolor{currentstroke}%
\pgfsetdash{}{0pt}%
\pgfpathmoveto{\pgfqpoint{4.509160in}{1.558561in}}%
\pgfpathlineto{\pgfqpoint{4.509160in}{1.558561in}}%
\pgfusepath{stroke}%
\end{pgfscope}%
\begin{pgfscope}%
\pgfpathrectangle{\pgfqpoint{3.352233in}{1.400000in}}{\pgfqpoint{2.407767in}{1.544118in}}%
\pgfusepath{clip}%
\pgfsetbuttcap%
\pgfsetroundjoin%
\pgfsetlinewidth{0.501875pt}%
\definecolor{currentstroke}{rgb}{0.268510,0.009605,0.335427}%
\pgfsetstrokecolor{currentstroke}%
\pgfsetdash{}{0pt}%
\pgfpathmoveto{\pgfqpoint{4.509160in}{1.558561in}}%
\pgfpathlineto{\pgfqpoint{4.509159in}{1.558562in}}%
\pgfusepath{stroke}%
\end{pgfscope}%
\begin{pgfscope}%
\pgfpathrectangle{\pgfqpoint{3.352233in}{1.400000in}}{\pgfqpoint{2.407767in}{1.544118in}}%
\pgfusepath{clip}%
\pgfsetbuttcap%
\pgfsetroundjoin%
\pgfsetlinewidth{0.501875pt}%
\definecolor{currentstroke}{rgb}{0.268510,0.009605,0.335427}%
\pgfsetstrokecolor{currentstroke}%
\pgfsetdash{}{0pt}%
\pgfpathmoveto{\pgfqpoint{4.509159in}{1.558562in}}%
\pgfpathlineto{\pgfqpoint{4.509159in}{1.558562in}}%
\pgfusepath{stroke}%
\end{pgfscope}%
\begin{pgfscope}%
\pgfpathrectangle{\pgfqpoint{3.352233in}{1.400000in}}{\pgfqpoint{2.407767in}{1.544118in}}%
\pgfusepath{clip}%
\pgfsetbuttcap%
\pgfsetroundjoin%
\pgfsetlinewidth{0.501875pt}%
\definecolor{currentstroke}{rgb}{0.268510,0.009605,0.335427}%
\pgfsetstrokecolor{currentstroke}%
\pgfsetdash{}{0pt}%
\pgfpathmoveto{\pgfqpoint{4.509159in}{1.558562in}}%
\pgfpathlineto{\pgfqpoint{4.509160in}{1.558562in}}%
\pgfusepath{stroke}%
\end{pgfscope}%
\begin{pgfscope}%
\pgfpathrectangle{\pgfqpoint{3.352233in}{1.400000in}}{\pgfqpoint{2.407767in}{1.544118in}}%
\pgfusepath{clip}%
\pgfsetbuttcap%
\pgfsetroundjoin%
\pgfsetlinewidth{0.501875pt}%
\definecolor{currentstroke}{rgb}{0.268510,0.009605,0.335427}%
\pgfsetstrokecolor{currentstroke}%
\pgfsetdash{}{0pt}%
\pgfpathmoveto{\pgfqpoint{4.509160in}{1.558562in}}%
\pgfpathlineto{\pgfqpoint{4.509160in}{1.558561in}}%
\pgfusepath{stroke}%
\end{pgfscope}%
\begin{pgfscope}%
\pgfpathrectangle{\pgfqpoint{3.352233in}{1.400000in}}{\pgfqpoint{2.407767in}{1.544118in}}%
\pgfusepath{clip}%
\pgfsetbuttcap%
\pgfsetroundjoin%
\pgfsetlinewidth{0.501875pt}%
\definecolor{currentstroke}{rgb}{0.268510,0.009605,0.335427}%
\pgfsetstrokecolor{currentstroke}%
\pgfsetdash{}{0pt}%
\pgfpathmoveto{\pgfqpoint{4.509160in}{1.558561in}}%
\pgfpathlineto{\pgfqpoint{4.509160in}{1.558561in}}%
\pgfusepath{stroke}%
\end{pgfscope}%
\begin{pgfscope}%
\pgfpathrectangle{\pgfqpoint{3.352233in}{1.400000in}}{\pgfqpoint{2.407767in}{1.544118in}}%
\pgfusepath{clip}%
\pgfsetbuttcap%
\pgfsetroundjoin%
\pgfsetlinewidth{0.501875pt}%
\definecolor{currentstroke}{rgb}{0.268510,0.009605,0.335427}%
\pgfsetstrokecolor{currentstroke}%
\pgfsetdash{}{0pt}%
\pgfpathmoveto{\pgfqpoint{4.509160in}{1.558561in}}%
\pgfpathlineto{\pgfqpoint{4.509159in}{1.558562in}}%
\pgfusepath{stroke}%
\end{pgfscope}%
\begin{pgfscope}%
\pgfpathrectangle{\pgfqpoint{3.352233in}{1.400000in}}{\pgfqpoint{2.407767in}{1.544118in}}%
\pgfusepath{clip}%
\pgfsetbuttcap%
\pgfsetroundjoin%
\pgfsetlinewidth{0.501875pt}%
\definecolor{currentstroke}{rgb}{0.268510,0.009605,0.335427}%
\pgfsetstrokecolor{currentstroke}%
\pgfsetdash{}{0pt}%
\pgfpathmoveto{\pgfqpoint{4.509159in}{1.558562in}}%
\pgfpathlineto{\pgfqpoint{4.509159in}{1.558562in}}%
\pgfusepath{stroke}%
\end{pgfscope}%
\begin{pgfscope}%
\pgfpathrectangle{\pgfqpoint{3.352233in}{1.400000in}}{\pgfqpoint{2.407767in}{1.544118in}}%
\pgfusepath{clip}%
\pgfsetbuttcap%
\pgfsetroundjoin%
\pgfsetlinewidth{0.501875pt}%
\definecolor{currentstroke}{rgb}{0.268510,0.009605,0.335427}%
\pgfsetstrokecolor{currentstroke}%
\pgfsetdash{}{0pt}%
\pgfpathmoveto{\pgfqpoint{4.509159in}{1.558562in}}%
\pgfpathlineto{\pgfqpoint{4.509160in}{1.558562in}}%
\pgfusepath{stroke}%
\end{pgfscope}%
\begin{pgfscope}%
\pgfpathrectangle{\pgfqpoint{3.352233in}{1.400000in}}{\pgfqpoint{2.407767in}{1.544118in}}%
\pgfusepath{clip}%
\pgfsetbuttcap%
\pgfsetroundjoin%
\pgfsetlinewidth{0.501875pt}%
\definecolor{currentstroke}{rgb}{0.268510,0.009605,0.335427}%
\pgfsetstrokecolor{currentstroke}%
\pgfsetdash{}{0pt}%
\pgfpathmoveto{\pgfqpoint{4.509160in}{1.558562in}}%
\pgfpathlineto{\pgfqpoint{4.509160in}{1.558561in}}%
\pgfusepath{stroke}%
\end{pgfscope}%
\begin{pgfscope}%
\pgfpathrectangle{\pgfqpoint{3.352233in}{1.400000in}}{\pgfqpoint{2.407767in}{1.544118in}}%
\pgfusepath{clip}%
\pgfsetbuttcap%
\pgfsetroundjoin%
\pgfsetlinewidth{0.501875pt}%
\definecolor{currentstroke}{rgb}{0.268510,0.009605,0.335427}%
\pgfsetstrokecolor{currentstroke}%
\pgfsetdash{}{0pt}%
\pgfpathmoveto{\pgfqpoint{4.509160in}{1.558561in}}%
\pgfpathlineto{\pgfqpoint{4.509160in}{1.558561in}}%
\pgfusepath{stroke}%
\end{pgfscope}%
\begin{pgfscope}%
\pgfpathrectangle{\pgfqpoint{3.352233in}{1.400000in}}{\pgfqpoint{2.407767in}{1.544118in}}%
\pgfusepath{clip}%
\pgfsetbuttcap%
\pgfsetroundjoin%
\pgfsetlinewidth{0.501875pt}%
\definecolor{currentstroke}{rgb}{0.268510,0.009605,0.335427}%
\pgfsetstrokecolor{currentstroke}%
\pgfsetdash{}{0pt}%
\pgfpathmoveto{\pgfqpoint{4.509160in}{1.558561in}}%
\pgfpathlineto{\pgfqpoint{4.509159in}{1.558562in}}%
\pgfusepath{stroke}%
\end{pgfscope}%
\begin{pgfscope}%
\pgfpathrectangle{\pgfqpoint{3.352233in}{1.400000in}}{\pgfqpoint{2.407767in}{1.544118in}}%
\pgfusepath{clip}%
\pgfsetbuttcap%
\pgfsetroundjoin%
\pgfsetlinewidth{0.501875pt}%
\definecolor{currentstroke}{rgb}{0.268510,0.009605,0.335427}%
\pgfsetstrokecolor{currentstroke}%
\pgfsetdash{}{0pt}%
\pgfpathmoveto{\pgfqpoint{4.509159in}{1.558562in}}%
\pgfpathlineto{\pgfqpoint{4.509159in}{1.558562in}}%
\pgfusepath{stroke}%
\end{pgfscope}%
\begin{pgfscope}%
\pgfpathrectangle{\pgfqpoint{3.352233in}{1.400000in}}{\pgfqpoint{2.407767in}{1.544118in}}%
\pgfusepath{clip}%
\pgfsetbuttcap%
\pgfsetroundjoin%
\pgfsetlinewidth{0.501875pt}%
\definecolor{currentstroke}{rgb}{0.268510,0.009605,0.335427}%
\pgfsetstrokecolor{currentstroke}%
\pgfsetdash{}{0pt}%
\pgfpathmoveto{\pgfqpoint{4.509159in}{1.558562in}}%
\pgfpathlineto{\pgfqpoint{4.509159in}{1.558562in}}%
\pgfusepath{stroke}%
\end{pgfscope}%
\begin{pgfscope}%
\pgfpathrectangle{\pgfqpoint{3.352233in}{1.400000in}}{\pgfqpoint{2.407767in}{1.544118in}}%
\pgfusepath{clip}%
\pgfsetbuttcap%
\pgfsetroundjoin%
\pgfsetlinewidth{0.501875pt}%
\definecolor{currentstroke}{rgb}{0.268510,0.009605,0.335427}%
\pgfsetstrokecolor{currentstroke}%
\pgfsetdash{}{0pt}%
\pgfpathmoveto{\pgfqpoint{4.509159in}{1.558562in}}%
\pgfpathlineto{\pgfqpoint{4.509160in}{1.558561in}}%
\pgfusepath{stroke}%
\end{pgfscope}%
\begin{pgfscope}%
\pgfpathrectangle{\pgfqpoint{3.352233in}{1.400000in}}{\pgfqpoint{2.407767in}{1.544118in}}%
\pgfusepath{clip}%
\pgfsetbuttcap%
\pgfsetroundjoin%
\pgfsetlinewidth{0.501875pt}%
\definecolor{currentstroke}{rgb}{0.268510,0.009605,0.335427}%
\pgfsetstrokecolor{currentstroke}%
\pgfsetdash{}{0pt}%
\pgfpathmoveto{\pgfqpoint{4.509160in}{1.558561in}}%
\pgfpathlineto{\pgfqpoint{4.509160in}{1.558561in}}%
\pgfusepath{stroke}%
\end{pgfscope}%
\begin{pgfscope}%
\pgfpathrectangle{\pgfqpoint{3.352233in}{1.400000in}}{\pgfqpoint{2.407767in}{1.544118in}}%
\pgfusepath{clip}%
\pgfsetbuttcap%
\pgfsetroundjoin%
\pgfsetlinewidth{0.501875pt}%
\definecolor{currentstroke}{rgb}{0.268510,0.009605,0.335427}%
\pgfsetstrokecolor{currentstroke}%
\pgfsetdash{}{0pt}%
\pgfpathmoveto{\pgfqpoint{4.509160in}{1.558561in}}%
\pgfpathlineto{\pgfqpoint{4.509159in}{1.558562in}}%
\pgfusepath{stroke}%
\end{pgfscope}%
\begin{pgfscope}%
\pgfpathrectangle{\pgfqpoint{3.352233in}{1.400000in}}{\pgfqpoint{2.407767in}{1.544118in}}%
\pgfusepath{clip}%
\pgfsetbuttcap%
\pgfsetroundjoin%
\pgfsetlinewidth{0.501875pt}%
\definecolor{currentstroke}{rgb}{0.268510,0.009605,0.335427}%
\pgfsetstrokecolor{currentstroke}%
\pgfsetdash{}{0pt}%
\pgfpathmoveto{\pgfqpoint{4.509159in}{1.558562in}}%
\pgfpathlineto{\pgfqpoint{4.509159in}{1.558562in}}%
\pgfusepath{stroke}%
\end{pgfscope}%
\begin{pgfscope}%
\pgfpathrectangle{\pgfqpoint{3.352233in}{1.400000in}}{\pgfqpoint{2.407767in}{1.544118in}}%
\pgfusepath{clip}%
\pgfsetbuttcap%
\pgfsetroundjoin%
\pgfsetlinewidth{0.501875pt}%
\definecolor{currentstroke}{rgb}{0.268510,0.009605,0.335427}%
\pgfsetstrokecolor{currentstroke}%
\pgfsetdash{}{0pt}%
\pgfpathmoveto{\pgfqpoint{4.509159in}{1.558562in}}%
\pgfpathlineto{\pgfqpoint{4.509159in}{1.558562in}}%
\pgfusepath{stroke}%
\end{pgfscope}%
\begin{pgfscope}%
\pgfpathrectangle{\pgfqpoint{3.352233in}{1.400000in}}{\pgfqpoint{2.407767in}{1.544118in}}%
\pgfusepath{clip}%
\pgfsetbuttcap%
\pgfsetroundjoin%
\pgfsetlinewidth{0.501875pt}%
\definecolor{currentstroke}{rgb}{0.268510,0.009605,0.335427}%
\pgfsetstrokecolor{currentstroke}%
\pgfsetdash{}{0pt}%
\pgfpathmoveto{\pgfqpoint{4.509159in}{1.558562in}}%
\pgfpathlineto{\pgfqpoint{4.509160in}{1.558561in}}%
\pgfusepath{stroke}%
\end{pgfscope}%
\begin{pgfscope}%
\pgfpathrectangle{\pgfqpoint{3.352233in}{1.400000in}}{\pgfqpoint{2.407767in}{1.544118in}}%
\pgfusepath{clip}%
\pgfsetbuttcap%
\pgfsetroundjoin%
\pgfsetlinewidth{0.501875pt}%
\definecolor{currentstroke}{rgb}{0.268510,0.009605,0.335427}%
\pgfsetstrokecolor{currentstroke}%
\pgfsetdash{}{0pt}%
\pgfpathmoveto{\pgfqpoint{4.509160in}{1.558561in}}%
\pgfpathlineto{\pgfqpoint{4.509160in}{1.558561in}}%
\pgfusepath{stroke}%
\end{pgfscope}%
\begin{pgfscope}%
\pgfpathrectangle{\pgfqpoint{3.352233in}{1.400000in}}{\pgfqpoint{2.407767in}{1.544118in}}%
\pgfusepath{clip}%
\pgfsetbuttcap%
\pgfsetroundjoin%
\pgfsetlinewidth{0.501875pt}%
\definecolor{currentstroke}{rgb}{0.268510,0.009605,0.335427}%
\pgfsetstrokecolor{currentstroke}%
\pgfsetdash{}{0pt}%
\pgfpathmoveto{\pgfqpoint{4.509160in}{1.558561in}}%
\pgfpathlineto{\pgfqpoint{4.509159in}{1.558562in}}%
\pgfusepath{stroke}%
\end{pgfscope}%
\begin{pgfscope}%
\pgfpathrectangle{\pgfqpoint{3.352233in}{1.400000in}}{\pgfqpoint{2.407767in}{1.544118in}}%
\pgfusepath{clip}%
\pgfsetbuttcap%
\pgfsetroundjoin%
\pgfsetlinewidth{0.501875pt}%
\definecolor{currentstroke}{rgb}{0.268510,0.009605,0.335427}%
\pgfsetstrokecolor{currentstroke}%
\pgfsetdash{}{0pt}%
\pgfpathmoveto{\pgfqpoint{4.509159in}{1.558562in}}%
\pgfpathlineto{\pgfqpoint{4.509159in}{1.558562in}}%
\pgfusepath{stroke}%
\end{pgfscope}%
\begin{pgfscope}%
\pgfpathrectangle{\pgfqpoint{3.352233in}{1.400000in}}{\pgfqpoint{2.407767in}{1.544118in}}%
\pgfusepath{clip}%
\pgfsetbuttcap%
\pgfsetroundjoin%
\pgfsetlinewidth{0.501875pt}%
\definecolor{currentstroke}{rgb}{0.268510,0.009605,0.335427}%
\pgfsetstrokecolor{currentstroke}%
\pgfsetdash{}{0pt}%
\pgfpathmoveto{\pgfqpoint{4.509159in}{1.558562in}}%
\pgfpathlineto{\pgfqpoint{4.509159in}{1.558562in}}%
\pgfusepath{stroke}%
\end{pgfscope}%
\begin{pgfscope}%
\pgfpathrectangle{\pgfqpoint{3.352233in}{1.400000in}}{\pgfqpoint{2.407767in}{1.544118in}}%
\pgfusepath{clip}%
\pgfsetbuttcap%
\pgfsetroundjoin%
\pgfsetlinewidth{0.501875pt}%
\definecolor{currentstroke}{rgb}{0.268510,0.009605,0.335427}%
\pgfsetstrokecolor{currentstroke}%
\pgfsetdash{}{0pt}%
\pgfpathmoveto{\pgfqpoint{4.509159in}{1.558562in}}%
\pgfpathlineto{\pgfqpoint{4.509160in}{1.558561in}}%
\pgfusepath{stroke}%
\end{pgfscope}%
\begin{pgfscope}%
\pgfpathrectangle{\pgfqpoint{3.352233in}{1.400000in}}{\pgfqpoint{2.407767in}{1.544118in}}%
\pgfusepath{clip}%
\pgfsetbuttcap%
\pgfsetroundjoin%
\pgfsetlinewidth{0.501875pt}%
\definecolor{currentstroke}{rgb}{0.268510,0.009605,0.335427}%
\pgfsetstrokecolor{currentstroke}%
\pgfsetdash{}{0pt}%
\pgfpathmoveto{\pgfqpoint{4.509160in}{1.558561in}}%
\pgfpathlineto{\pgfqpoint{4.509160in}{1.558561in}}%
\pgfusepath{stroke}%
\end{pgfscope}%
\begin{pgfscope}%
\pgfpathrectangle{\pgfqpoint{3.352233in}{1.400000in}}{\pgfqpoint{2.407767in}{1.544118in}}%
\pgfusepath{clip}%
\pgfsetbuttcap%
\pgfsetroundjoin%
\pgfsetlinewidth{0.501875pt}%
\definecolor{currentstroke}{rgb}{0.268510,0.009605,0.335427}%
\pgfsetstrokecolor{currentstroke}%
\pgfsetdash{}{0pt}%
\pgfpathmoveto{\pgfqpoint{4.509160in}{1.558561in}}%
\pgfpathlineto{\pgfqpoint{4.509159in}{1.558562in}}%
\pgfusepath{stroke}%
\end{pgfscope}%
\begin{pgfscope}%
\pgfpathrectangle{\pgfqpoint{3.352233in}{1.400000in}}{\pgfqpoint{2.407767in}{1.544118in}}%
\pgfusepath{clip}%
\pgfsetbuttcap%
\pgfsetroundjoin%
\pgfsetlinewidth{0.501875pt}%
\definecolor{currentstroke}{rgb}{0.268510,0.009605,0.335427}%
\pgfsetstrokecolor{currentstroke}%
\pgfsetdash{}{0pt}%
\pgfpathmoveto{\pgfqpoint{4.509159in}{1.558562in}}%
\pgfpathlineto{\pgfqpoint{4.509159in}{1.558562in}}%
\pgfusepath{stroke}%
\end{pgfscope}%
\begin{pgfscope}%
\pgfpathrectangle{\pgfqpoint{3.352233in}{1.400000in}}{\pgfqpoint{2.407767in}{1.544118in}}%
\pgfusepath{clip}%
\pgfsetbuttcap%
\pgfsetroundjoin%
\pgfsetlinewidth{0.501875pt}%
\definecolor{currentstroke}{rgb}{0.268510,0.009605,0.335427}%
\pgfsetstrokecolor{currentstroke}%
\pgfsetdash{}{0pt}%
\pgfpathmoveto{\pgfqpoint{4.509159in}{1.558562in}}%
\pgfpathlineto{\pgfqpoint{4.509159in}{1.558562in}}%
\pgfusepath{stroke}%
\end{pgfscope}%
\begin{pgfscope}%
\pgfpathrectangle{\pgfqpoint{3.352233in}{1.400000in}}{\pgfqpoint{2.407767in}{1.544118in}}%
\pgfusepath{clip}%
\pgfsetbuttcap%
\pgfsetroundjoin%
\pgfsetlinewidth{0.501875pt}%
\definecolor{currentstroke}{rgb}{0.268510,0.009605,0.335427}%
\pgfsetstrokecolor{currentstroke}%
\pgfsetdash{}{0pt}%
\pgfpathmoveto{\pgfqpoint{4.509159in}{1.558562in}}%
\pgfpathlineto{\pgfqpoint{4.509160in}{1.558561in}}%
\pgfusepath{stroke}%
\end{pgfscope}%
\begin{pgfscope}%
\pgfpathrectangle{\pgfqpoint{3.352233in}{1.400000in}}{\pgfqpoint{2.407767in}{1.544118in}}%
\pgfusepath{clip}%
\pgfsetbuttcap%
\pgfsetroundjoin%
\pgfsetlinewidth{0.501875pt}%
\definecolor{currentstroke}{rgb}{0.268510,0.009605,0.335427}%
\pgfsetstrokecolor{currentstroke}%
\pgfsetdash{}{0pt}%
\pgfpathmoveto{\pgfqpoint{4.509160in}{1.558561in}}%
\pgfpathlineto{\pgfqpoint{4.509160in}{1.558561in}}%
\pgfusepath{stroke}%
\end{pgfscope}%
\begin{pgfscope}%
\pgfpathrectangle{\pgfqpoint{3.352233in}{1.400000in}}{\pgfqpoint{2.407767in}{1.544118in}}%
\pgfusepath{clip}%
\pgfsetbuttcap%
\pgfsetroundjoin%
\pgfsetlinewidth{0.501875pt}%
\definecolor{currentstroke}{rgb}{0.268510,0.009605,0.335427}%
\pgfsetstrokecolor{currentstroke}%
\pgfsetdash{}{0pt}%
\pgfpathmoveto{\pgfqpoint{4.509160in}{1.558561in}}%
\pgfpathlineto{\pgfqpoint{4.509160in}{1.558562in}}%
\pgfusepath{stroke}%
\end{pgfscope}%
\begin{pgfscope}%
\pgfpathrectangle{\pgfqpoint{3.352233in}{1.400000in}}{\pgfqpoint{2.407767in}{1.544118in}}%
\pgfusepath{clip}%
\pgfsetbuttcap%
\pgfsetroundjoin%
\pgfsetlinewidth{0.501875pt}%
\definecolor{currentstroke}{rgb}{0.268510,0.009605,0.335427}%
\pgfsetstrokecolor{currentstroke}%
\pgfsetdash{}{0pt}%
\pgfpathmoveto{\pgfqpoint{4.509160in}{1.558562in}}%
\pgfpathlineto{\pgfqpoint{4.509159in}{1.558562in}}%
\pgfusepath{stroke}%
\end{pgfscope}%
\begin{pgfscope}%
\pgfpathrectangle{\pgfqpoint{3.352233in}{1.400000in}}{\pgfqpoint{2.407767in}{1.544118in}}%
\pgfusepath{clip}%
\pgfsetbuttcap%
\pgfsetroundjoin%
\pgfsetlinewidth{0.501875pt}%
\definecolor{currentstroke}{rgb}{0.268510,0.009605,0.335427}%
\pgfsetstrokecolor{currentstroke}%
\pgfsetdash{}{0pt}%
\pgfpathmoveto{\pgfqpoint{4.509159in}{1.558562in}}%
\pgfpathlineto{\pgfqpoint{4.509159in}{1.558562in}}%
\pgfusepath{stroke}%
\end{pgfscope}%
\begin{pgfscope}%
\pgfpathrectangle{\pgfqpoint{3.352233in}{1.400000in}}{\pgfqpoint{2.407767in}{1.544118in}}%
\pgfusepath{clip}%
\pgfsetbuttcap%
\pgfsetroundjoin%
\pgfsetlinewidth{0.501875pt}%
\definecolor{currentstroke}{rgb}{0.268510,0.009605,0.335427}%
\pgfsetstrokecolor{currentstroke}%
\pgfsetdash{}{0pt}%
\pgfpathmoveto{\pgfqpoint{4.509159in}{1.558562in}}%
\pgfpathlineto{\pgfqpoint{4.509160in}{1.558561in}}%
\pgfusepath{stroke}%
\end{pgfscope}%
\begin{pgfscope}%
\pgfpathrectangle{\pgfqpoint{3.352233in}{1.400000in}}{\pgfqpoint{2.407767in}{1.544118in}}%
\pgfusepath{clip}%
\pgfsetbuttcap%
\pgfsetroundjoin%
\pgfsetlinewidth{0.501875pt}%
\definecolor{currentstroke}{rgb}{0.268510,0.009605,0.335427}%
\pgfsetstrokecolor{currentstroke}%
\pgfsetdash{}{0pt}%
\pgfpathmoveto{\pgfqpoint{4.509160in}{1.558561in}}%
\pgfpathlineto{\pgfqpoint{4.509160in}{1.558561in}}%
\pgfusepath{stroke}%
\end{pgfscope}%
\begin{pgfscope}%
\pgfpathrectangle{\pgfqpoint{3.352233in}{1.400000in}}{\pgfqpoint{2.407767in}{1.544118in}}%
\pgfusepath{clip}%
\pgfsetbuttcap%
\pgfsetroundjoin%
\pgfsetlinewidth{0.501875pt}%
\definecolor{currentstroke}{rgb}{0.268510,0.009605,0.335427}%
\pgfsetstrokecolor{currentstroke}%
\pgfsetdash{}{0pt}%
\pgfpathmoveto{\pgfqpoint{4.509160in}{1.558561in}}%
\pgfpathlineto{\pgfqpoint{4.509160in}{1.558562in}}%
\pgfusepath{stroke}%
\end{pgfscope}%
\begin{pgfscope}%
\pgfpathrectangle{\pgfqpoint{3.352233in}{1.400000in}}{\pgfqpoint{2.407767in}{1.544118in}}%
\pgfusepath{clip}%
\pgfsetbuttcap%
\pgfsetroundjoin%
\pgfsetlinewidth{0.501875pt}%
\definecolor{currentstroke}{rgb}{0.268510,0.009605,0.335427}%
\pgfsetstrokecolor{currentstroke}%
\pgfsetdash{}{0pt}%
\pgfpathmoveto{\pgfqpoint{4.509160in}{1.558562in}}%
\pgfpathlineto{\pgfqpoint{4.509159in}{1.558562in}}%
\pgfusepath{stroke}%
\end{pgfscope}%
\begin{pgfscope}%
\pgfpathrectangle{\pgfqpoint{3.352233in}{1.400000in}}{\pgfqpoint{2.407767in}{1.544118in}}%
\pgfusepath{clip}%
\pgfsetbuttcap%
\pgfsetroundjoin%
\pgfsetlinewidth{0.501875pt}%
\definecolor{currentstroke}{rgb}{0.268510,0.009605,0.335427}%
\pgfsetstrokecolor{currentstroke}%
\pgfsetdash{}{0pt}%
\pgfpathmoveto{\pgfqpoint{4.509159in}{1.558562in}}%
\pgfpathlineto{\pgfqpoint{4.509159in}{1.558562in}}%
\pgfusepath{stroke}%
\end{pgfscope}%
\begin{pgfscope}%
\pgfpathrectangle{\pgfqpoint{3.352233in}{1.400000in}}{\pgfqpoint{2.407767in}{1.544118in}}%
\pgfusepath{clip}%
\pgfsetbuttcap%
\pgfsetroundjoin%
\pgfsetlinewidth{0.501875pt}%
\definecolor{currentstroke}{rgb}{0.268510,0.009605,0.335427}%
\pgfsetstrokecolor{currentstroke}%
\pgfsetdash{}{0pt}%
\pgfpathmoveto{\pgfqpoint{4.509159in}{1.558562in}}%
\pgfpathlineto{\pgfqpoint{4.509160in}{1.558561in}}%
\pgfusepath{stroke}%
\end{pgfscope}%
\begin{pgfscope}%
\pgfpathrectangle{\pgfqpoint{3.352233in}{1.400000in}}{\pgfqpoint{2.407767in}{1.544118in}}%
\pgfusepath{clip}%
\pgfsetbuttcap%
\pgfsetroundjoin%
\pgfsetlinewidth{0.501875pt}%
\definecolor{currentstroke}{rgb}{0.268510,0.009605,0.335427}%
\pgfsetstrokecolor{currentstroke}%
\pgfsetdash{}{0pt}%
\pgfpathmoveto{\pgfqpoint{4.509160in}{1.558561in}}%
\pgfpathlineto{\pgfqpoint{4.509160in}{1.558561in}}%
\pgfusepath{stroke}%
\end{pgfscope}%
\begin{pgfscope}%
\pgfpathrectangle{\pgfqpoint{3.352233in}{1.400000in}}{\pgfqpoint{2.407767in}{1.544118in}}%
\pgfusepath{clip}%
\pgfsetbuttcap%
\pgfsetroundjoin%
\pgfsetlinewidth{0.501875pt}%
\definecolor{currentstroke}{rgb}{0.268510,0.009605,0.335427}%
\pgfsetstrokecolor{currentstroke}%
\pgfsetdash{}{0pt}%
\pgfpathmoveto{\pgfqpoint{4.509160in}{1.558561in}}%
\pgfpathlineto{\pgfqpoint{4.509160in}{1.558562in}}%
\pgfusepath{stroke}%
\end{pgfscope}%
\begin{pgfscope}%
\pgfpathrectangle{\pgfqpoint{3.352233in}{1.400000in}}{\pgfqpoint{2.407767in}{1.544118in}}%
\pgfusepath{clip}%
\pgfsetbuttcap%
\pgfsetroundjoin%
\pgfsetlinewidth{0.501875pt}%
\definecolor{currentstroke}{rgb}{0.268510,0.009605,0.335427}%
\pgfsetstrokecolor{currentstroke}%
\pgfsetdash{}{0pt}%
\pgfpathmoveto{\pgfqpoint{4.509160in}{1.558562in}}%
\pgfpathlineto{\pgfqpoint{4.509159in}{1.558562in}}%
\pgfusepath{stroke}%
\end{pgfscope}%
\begin{pgfscope}%
\pgfpathrectangle{\pgfqpoint{3.352233in}{1.400000in}}{\pgfqpoint{2.407767in}{1.544118in}}%
\pgfusepath{clip}%
\pgfsetbuttcap%
\pgfsetroundjoin%
\pgfsetlinewidth{0.501875pt}%
\definecolor{currentstroke}{rgb}{0.268510,0.009605,0.335427}%
\pgfsetstrokecolor{currentstroke}%
\pgfsetdash{}{0pt}%
\pgfpathmoveto{\pgfqpoint{4.509159in}{1.558562in}}%
\pgfpathlineto{\pgfqpoint{4.509159in}{1.558562in}}%
\pgfusepath{stroke}%
\end{pgfscope}%
\begin{pgfscope}%
\pgfpathrectangle{\pgfqpoint{3.352233in}{1.400000in}}{\pgfqpoint{2.407767in}{1.544118in}}%
\pgfusepath{clip}%
\pgfsetbuttcap%
\pgfsetroundjoin%
\pgfsetlinewidth{0.501875pt}%
\definecolor{currentstroke}{rgb}{0.268510,0.009605,0.335427}%
\pgfsetstrokecolor{currentstroke}%
\pgfsetdash{}{0pt}%
\pgfpathmoveto{\pgfqpoint{4.509159in}{1.558562in}}%
\pgfpathlineto{\pgfqpoint{4.509160in}{1.558561in}}%
\pgfusepath{stroke}%
\end{pgfscope}%
\begin{pgfscope}%
\pgfpathrectangle{\pgfqpoint{3.352233in}{1.400000in}}{\pgfqpoint{2.407767in}{1.544118in}}%
\pgfusepath{clip}%
\pgfsetbuttcap%
\pgfsetroundjoin%
\pgfsetlinewidth{0.501875pt}%
\definecolor{currentstroke}{rgb}{0.268510,0.009605,0.335427}%
\pgfsetstrokecolor{currentstroke}%
\pgfsetdash{}{0pt}%
\pgfpathmoveto{\pgfqpoint{4.509160in}{1.558561in}}%
\pgfpathlineto{\pgfqpoint{4.509160in}{1.558561in}}%
\pgfusepath{stroke}%
\end{pgfscope}%
\begin{pgfscope}%
\pgfpathrectangle{\pgfqpoint{3.352233in}{1.400000in}}{\pgfqpoint{2.407767in}{1.544118in}}%
\pgfusepath{clip}%
\pgfsetbuttcap%
\pgfsetroundjoin%
\pgfsetlinewidth{0.501875pt}%
\definecolor{currentstroke}{rgb}{0.268510,0.009605,0.335427}%
\pgfsetstrokecolor{currentstroke}%
\pgfsetdash{}{0pt}%
\pgfpathmoveto{\pgfqpoint{4.509160in}{1.558561in}}%
\pgfpathlineto{\pgfqpoint{4.509160in}{1.558562in}}%
\pgfusepath{stroke}%
\end{pgfscope}%
\begin{pgfscope}%
\pgfpathrectangle{\pgfqpoint{3.352233in}{1.400000in}}{\pgfqpoint{2.407767in}{1.544118in}}%
\pgfusepath{clip}%
\pgfsetbuttcap%
\pgfsetroundjoin%
\pgfsetlinewidth{0.501875pt}%
\definecolor{currentstroke}{rgb}{0.268510,0.009605,0.335427}%
\pgfsetstrokecolor{currentstroke}%
\pgfsetdash{}{0pt}%
\pgfpathmoveto{\pgfqpoint{4.509160in}{1.558562in}}%
\pgfpathlineto{\pgfqpoint{4.509159in}{1.558562in}}%
\pgfusepath{stroke}%
\end{pgfscope}%
\begin{pgfscope}%
\pgfpathrectangle{\pgfqpoint{3.352233in}{1.400000in}}{\pgfqpoint{2.407767in}{1.544118in}}%
\pgfusepath{clip}%
\pgfsetbuttcap%
\pgfsetroundjoin%
\pgfsetlinewidth{0.501875pt}%
\definecolor{currentstroke}{rgb}{0.268510,0.009605,0.335427}%
\pgfsetstrokecolor{currentstroke}%
\pgfsetdash{}{0pt}%
\pgfpathmoveto{\pgfqpoint{4.509159in}{1.558562in}}%
\pgfpathlineto{\pgfqpoint{4.509159in}{1.558562in}}%
\pgfusepath{stroke}%
\end{pgfscope}%
\begin{pgfscope}%
\pgfpathrectangle{\pgfqpoint{3.352233in}{1.400000in}}{\pgfqpoint{2.407767in}{1.544118in}}%
\pgfusepath{clip}%
\pgfsetbuttcap%
\pgfsetroundjoin%
\pgfsetlinewidth{0.501875pt}%
\definecolor{currentstroke}{rgb}{0.268510,0.009605,0.335427}%
\pgfsetstrokecolor{currentstroke}%
\pgfsetdash{}{0pt}%
\pgfpathmoveto{\pgfqpoint{4.509159in}{1.558562in}}%
\pgfpathlineto{\pgfqpoint{4.509160in}{1.558561in}}%
\pgfusepath{stroke}%
\end{pgfscope}%
\begin{pgfscope}%
\pgfpathrectangle{\pgfqpoint{3.352233in}{1.400000in}}{\pgfqpoint{2.407767in}{1.544118in}}%
\pgfusepath{clip}%
\pgfsetbuttcap%
\pgfsetroundjoin%
\pgfsetlinewidth{0.501875pt}%
\definecolor{currentstroke}{rgb}{0.268510,0.009605,0.335427}%
\pgfsetstrokecolor{currentstroke}%
\pgfsetdash{}{0pt}%
\pgfpathmoveto{\pgfqpoint{4.509160in}{1.558561in}}%
\pgfpathlineto{\pgfqpoint{4.509160in}{1.558561in}}%
\pgfusepath{stroke}%
\end{pgfscope}%
\begin{pgfscope}%
\pgfpathrectangle{\pgfqpoint{3.352233in}{1.400000in}}{\pgfqpoint{2.407767in}{1.544118in}}%
\pgfusepath{clip}%
\pgfsetbuttcap%
\pgfsetroundjoin%
\pgfsetlinewidth{0.501875pt}%
\definecolor{currentstroke}{rgb}{0.268510,0.009605,0.335427}%
\pgfsetstrokecolor{currentstroke}%
\pgfsetdash{}{0pt}%
\pgfpathmoveto{\pgfqpoint{4.509160in}{1.558561in}}%
\pgfpathlineto{\pgfqpoint{4.509160in}{1.558562in}}%
\pgfusepath{stroke}%
\end{pgfscope}%
\begin{pgfscope}%
\pgfpathrectangle{\pgfqpoint{3.352233in}{1.400000in}}{\pgfqpoint{2.407767in}{1.544118in}}%
\pgfusepath{clip}%
\pgfsetbuttcap%
\pgfsetroundjoin%
\pgfsetlinewidth{0.501875pt}%
\definecolor{currentstroke}{rgb}{0.268510,0.009605,0.335427}%
\pgfsetstrokecolor{currentstroke}%
\pgfsetdash{}{0pt}%
\pgfpathmoveto{\pgfqpoint{4.509160in}{1.558562in}}%
\pgfpathlineto{\pgfqpoint{4.509159in}{1.558562in}}%
\pgfusepath{stroke}%
\end{pgfscope}%
\begin{pgfscope}%
\pgfpathrectangle{\pgfqpoint{3.352233in}{1.400000in}}{\pgfqpoint{2.407767in}{1.544118in}}%
\pgfusepath{clip}%
\pgfsetbuttcap%
\pgfsetroundjoin%
\pgfsetlinewidth{0.501875pt}%
\definecolor{currentstroke}{rgb}{0.268510,0.009605,0.335427}%
\pgfsetstrokecolor{currentstroke}%
\pgfsetdash{}{0pt}%
\pgfpathmoveto{\pgfqpoint{4.509159in}{1.558562in}}%
\pgfpathlineto{\pgfqpoint{4.509159in}{1.558562in}}%
\pgfusepath{stroke}%
\end{pgfscope}%
\begin{pgfscope}%
\pgfpathrectangle{\pgfqpoint{3.352233in}{1.400000in}}{\pgfqpoint{2.407767in}{1.544118in}}%
\pgfusepath{clip}%
\pgfsetbuttcap%
\pgfsetroundjoin%
\pgfsetlinewidth{0.501875pt}%
\definecolor{currentstroke}{rgb}{0.268510,0.009605,0.335427}%
\pgfsetstrokecolor{currentstroke}%
\pgfsetdash{}{0pt}%
\pgfpathmoveto{\pgfqpoint{4.509159in}{1.558562in}}%
\pgfpathlineto{\pgfqpoint{4.509160in}{1.558562in}}%
\pgfusepath{stroke}%
\end{pgfscope}%
\begin{pgfscope}%
\pgfpathrectangle{\pgfqpoint{3.352233in}{1.400000in}}{\pgfqpoint{2.407767in}{1.544118in}}%
\pgfusepath{clip}%
\pgfsetbuttcap%
\pgfsetroundjoin%
\pgfsetlinewidth{0.501875pt}%
\definecolor{currentstroke}{rgb}{0.268510,0.009605,0.335427}%
\pgfsetstrokecolor{currentstroke}%
\pgfsetdash{}{0pt}%
\pgfpathmoveto{\pgfqpoint{4.509160in}{1.558562in}}%
\pgfpathlineto{\pgfqpoint{4.509160in}{1.558561in}}%
\pgfusepath{stroke}%
\end{pgfscope}%
\begin{pgfscope}%
\pgfpathrectangle{\pgfqpoint{3.352233in}{1.400000in}}{\pgfqpoint{2.407767in}{1.544118in}}%
\pgfusepath{clip}%
\pgfsetbuttcap%
\pgfsetroundjoin%
\pgfsetlinewidth{0.501875pt}%
\definecolor{currentstroke}{rgb}{0.268510,0.009605,0.335427}%
\pgfsetstrokecolor{currentstroke}%
\pgfsetdash{}{0pt}%
\pgfpathmoveto{\pgfqpoint{4.509160in}{1.558561in}}%
\pgfpathlineto{\pgfqpoint{4.509160in}{1.558562in}}%
\pgfusepath{stroke}%
\end{pgfscope}%
\begin{pgfscope}%
\pgfpathrectangle{\pgfqpoint{3.352233in}{1.400000in}}{\pgfqpoint{2.407767in}{1.544118in}}%
\pgfusepath{clip}%
\pgfsetbuttcap%
\pgfsetroundjoin%
\pgfsetlinewidth{0.501875pt}%
\definecolor{currentstroke}{rgb}{0.268510,0.009605,0.335427}%
\pgfsetstrokecolor{currentstroke}%
\pgfsetdash{}{0pt}%
\pgfpathmoveto{\pgfqpoint{4.509160in}{1.558562in}}%
\pgfpathlineto{\pgfqpoint{4.509159in}{1.558562in}}%
\pgfusepath{stroke}%
\end{pgfscope}%
\begin{pgfscope}%
\pgfpathrectangle{\pgfqpoint{3.352233in}{1.400000in}}{\pgfqpoint{2.407767in}{1.544118in}}%
\pgfusepath{clip}%
\pgfsetbuttcap%
\pgfsetroundjoin%
\pgfsetlinewidth{0.501875pt}%
\definecolor{currentstroke}{rgb}{0.268510,0.009605,0.335427}%
\pgfsetstrokecolor{currentstroke}%
\pgfsetdash{}{0pt}%
\pgfpathmoveto{\pgfqpoint{4.509159in}{1.558562in}}%
\pgfpathlineto{\pgfqpoint{4.509159in}{1.558562in}}%
\pgfusepath{stroke}%
\end{pgfscope}%
\begin{pgfscope}%
\pgfpathrectangle{\pgfqpoint{3.352233in}{1.400000in}}{\pgfqpoint{2.407767in}{1.544118in}}%
\pgfusepath{clip}%
\pgfsetbuttcap%
\pgfsetroundjoin%
\pgfsetlinewidth{0.501875pt}%
\definecolor{currentstroke}{rgb}{0.268510,0.009605,0.335427}%
\pgfsetstrokecolor{currentstroke}%
\pgfsetdash{}{0pt}%
\pgfpathmoveto{\pgfqpoint{4.509159in}{1.558562in}}%
\pgfpathlineto{\pgfqpoint{4.509160in}{1.558562in}}%
\pgfusepath{stroke}%
\end{pgfscope}%
\begin{pgfscope}%
\pgfpathrectangle{\pgfqpoint{3.352233in}{1.400000in}}{\pgfqpoint{2.407767in}{1.544118in}}%
\pgfusepath{clip}%
\pgfsetbuttcap%
\pgfsetroundjoin%
\pgfsetlinewidth{0.501875pt}%
\definecolor{currentstroke}{rgb}{0.268510,0.009605,0.335427}%
\pgfsetstrokecolor{currentstroke}%
\pgfsetdash{}{0pt}%
\pgfpathmoveto{\pgfqpoint{4.509160in}{1.558562in}}%
\pgfpathlineto{\pgfqpoint{4.509160in}{1.558561in}}%
\pgfusepath{stroke}%
\end{pgfscope}%
\begin{pgfscope}%
\pgfpathrectangle{\pgfqpoint{3.352233in}{1.400000in}}{\pgfqpoint{2.407767in}{1.544118in}}%
\pgfusepath{clip}%
\pgfsetbuttcap%
\pgfsetroundjoin%
\pgfsetlinewidth{0.501875pt}%
\definecolor{currentstroke}{rgb}{0.268510,0.009605,0.335427}%
\pgfsetstrokecolor{currentstroke}%
\pgfsetdash{}{0pt}%
\pgfpathmoveto{\pgfqpoint{4.509160in}{1.558561in}}%
\pgfpathlineto{\pgfqpoint{4.509160in}{1.558562in}}%
\pgfusepath{stroke}%
\end{pgfscope}%
\begin{pgfscope}%
\pgfpathrectangle{\pgfqpoint{3.352233in}{1.400000in}}{\pgfqpoint{2.407767in}{1.544118in}}%
\pgfusepath{clip}%
\pgfsetbuttcap%
\pgfsetroundjoin%
\pgfsetlinewidth{0.501875pt}%
\definecolor{currentstroke}{rgb}{0.268510,0.009605,0.335427}%
\pgfsetstrokecolor{currentstroke}%
\pgfsetdash{}{0pt}%
\pgfpathmoveto{\pgfqpoint{4.509160in}{1.558562in}}%
\pgfpathlineto{\pgfqpoint{4.509159in}{1.558562in}}%
\pgfusepath{stroke}%
\end{pgfscope}%
\begin{pgfscope}%
\pgfpathrectangle{\pgfqpoint{3.352233in}{1.400000in}}{\pgfqpoint{2.407767in}{1.544118in}}%
\pgfusepath{clip}%
\pgfsetbuttcap%
\pgfsetroundjoin%
\pgfsetlinewidth{0.501875pt}%
\definecolor{currentstroke}{rgb}{0.268510,0.009605,0.335427}%
\pgfsetstrokecolor{currentstroke}%
\pgfsetdash{}{0pt}%
\pgfpathmoveto{\pgfqpoint{4.509159in}{1.558562in}}%
\pgfpathlineto{\pgfqpoint{4.509159in}{1.558562in}}%
\pgfusepath{stroke}%
\end{pgfscope}%
\begin{pgfscope}%
\pgfpathrectangle{\pgfqpoint{3.352233in}{1.400000in}}{\pgfqpoint{2.407767in}{1.544118in}}%
\pgfusepath{clip}%
\pgfsetbuttcap%
\pgfsetroundjoin%
\pgfsetlinewidth{0.501875pt}%
\definecolor{currentstroke}{rgb}{0.268510,0.009605,0.335427}%
\pgfsetstrokecolor{currentstroke}%
\pgfsetdash{}{0pt}%
\pgfpathmoveto{\pgfqpoint{4.509159in}{1.558562in}}%
\pgfpathlineto{\pgfqpoint{4.509160in}{1.558562in}}%
\pgfusepath{stroke}%
\end{pgfscope}%
\begin{pgfscope}%
\pgfpathrectangle{\pgfqpoint{3.352233in}{1.400000in}}{\pgfqpoint{2.407767in}{1.544118in}}%
\pgfusepath{clip}%
\pgfsetbuttcap%
\pgfsetroundjoin%
\pgfsetlinewidth{0.501875pt}%
\definecolor{currentstroke}{rgb}{0.268510,0.009605,0.335427}%
\pgfsetstrokecolor{currentstroke}%
\pgfsetdash{}{0pt}%
\pgfpathmoveto{\pgfqpoint{4.509160in}{1.558562in}}%
\pgfpathlineto{\pgfqpoint{4.509160in}{1.558562in}}%
\pgfusepath{stroke}%
\end{pgfscope}%
\begin{pgfscope}%
\pgfpathrectangle{\pgfqpoint{3.352233in}{1.400000in}}{\pgfqpoint{2.407767in}{1.544118in}}%
\pgfusepath{clip}%
\pgfsetbuttcap%
\pgfsetroundjoin%
\pgfsetlinewidth{0.501875pt}%
\definecolor{currentstroke}{rgb}{0.268510,0.009605,0.335427}%
\pgfsetstrokecolor{currentstroke}%
\pgfsetdash{}{0pt}%
\pgfpathmoveto{\pgfqpoint{4.509160in}{1.558562in}}%
\pgfpathlineto{\pgfqpoint{4.509160in}{1.558562in}}%
\pgfusepath{stroke}%
\end{pgfscope}%
\begin{pgfscope}%
\pgfpathrectangle{\pgfqpoint{3.352233in}{1.400000in}}{\pgfqpoint{2.407767in}{1.544118in}}%
\pgfusepath{clip}%
\pgfsetbuttcap%
\pgfsetroundjoin%
\pgfsetlinewidth{0.501875pt}%
\definecolor{currentstroke}{rgb}{0.268510,0.009605,0.335427}%
\pgfsetstrokecolor{currentstroke}%
\pgfsetdash{}{0pt}%
\pgfpathmoveto{\pgfqpoint{4.509160in}{1.558562in}}%
\pgfpathlineto{\pgfqpoint{4.509159in}{1.558562in}}%
\pgfusepath{stroke}%
\end{pgfscope}%
\begin{pgfscope}%
\pgfpathrectangle{\pgfqpoint{3.352233in}{1.400000in}}{\pgfqpoint{2.407767in}{1.544118in}}%
\pgfusepath{clip}%
\pgfsetbuttcap%
\pgfsetroundjoin%
\pgfsetlinewidth{0.501875pt}%
\definecolor{currentstroke}{rgb}{0.268510,0.009605,0.335427}%
\pgfsetstrokecolor{currentstroke}%
\pgfsetdash{}{0pt}%
\pgfpathmoveto{\pgfqpoint{4.509159in}{1.558562in}}%
\pgfpathlineto{\pgfqpoint{4.509159in}{1.558562in}}%
\pgfusepath{stroke}%
\end{pgfscope}%
\begin{pgfscope}%
\pgfpathrectangle{\pgfqpoint{3.352233in}{1.400000in}}{\pgfqpoint{2.407767in}{1.544118in}}%
\pgfusepath{clip}%
\pgfsetbuttcap%
\pgfsetroundjoin%
\pgfsetlinewidth{0.501875pt}%
\definecolor{currentstroke}{rgb}{0.268510,0.009605,0.335427}%
\pgfsetstrokecolor{currentstroke}%
\pgfsetdash{}{0pt}%
\pgfpathmoveto{\pgfqpoint{4.509159in}{1.558562in}}%
\pgfpathlineto{\pgfqpoint{4.509160in}{1.558562in}}%
\pgfusepath{stroke}%
\end{pgfscope}%
\begin{pgfscope}%
\pgfpathrectangle{\pgfqpoint{3.352233in}{1.400000in}}{\pgfqpoint{2.407767in}{1.544118in}}%
\pgfusepath{clip}%
\pgfsetbuttcap%
\pgfsetroundjoin%
\pgfsetlinewidth{0.501875pt}%
\definecolor{currentstroke}{rgb}{0.268510,0.009605,0.335427}%
\pgfsetstrokecolor{currentstroke}%
\pgfsetdash{}{0pt}%
\pgfpathmoveto{\pgfqpoint{4.509160in}{1.558562in}}%
\pgfpathlineto{\pgfqpoint{4.509160in}{1.558562in}}%
\pgfusepath{stroke}%
\end{pgfscope}%
\begin{pgfscope}%
\pgfpathrectangle{\pgfqpoint{3.352233in}{1.400000in}}{\pgfqpoint{2.407767in}{1.544118in}}%
\pgfusepath{clip}%
\pgfsetbuttcap%
\pgfsetroundjoin%
\pgfsetlinewidth{0.501875pt}%
\definecolor{currentstroke}{rgb}{0.268510,0.009605,0.335427}%
\pgfsetstrokecolor{currentstroke}%
\pgfsetdash{}{0pt}%
\pgfpathmoveto{\pgfqpoint{4.509160in}{1.558562in}}%
\pgfpathlineto{\pgfqpoint{4.509160in}{1.558562in}}%
\pgfusepath{stroke}%
\end{pgfscope}%
\begin{pgfscope}%
\pgfpathrectangle{\pgfqpoint{3.352233in}{1.400000in}}{\pgfqpoint{2.407767in}{1.544118in}}%
\pgfusepath{clip}%
\pgfsetbuttcap%
\pgfsetroundjoin%
\pgfsetlinewidth{0.501875pt}%
\definecolor{currentstroke}{rgb}{0.268510,0.009605,0.335427}%
\pgfsetstrokecolor{currentstroke}%
\pgfsetdash{}{0pt}%
\pgfpathmoveto{\pgfqpoint{4.509160in}{1.558562in}}%
\pgfpathlineto{\pgfqpoint{4.509159in}{1.558562in}}%
\pgfusepath{stroke}%
\end{pgfscope}%
\begin{pgfscope}%
\pgfpathrectangle{\pgfqpoint{3.352233in}{1.400000in}}{\pgfqpoint{2.407767in}{1.544118in}}%
\pgfusepath{clip}%
\pgfsetbuttcap%
\pgfsetroundjoin%
\pgfsetlinewidth{0.501875pt}%
\definecolor{currentstroke}{rgb}{0.268510,0.009605,0.335427}%
\pgfsetstrokecolor{currentstroke}%
\pgfsetdash{}{0pt}%
\pgfpathmoveto{\pgfqpoint{4.509159in}{1.558562in}}%
\pgfpathlineto{\pgfqpoint{4.509159in}{1.558562in}}%
\pgfusepath{stroke}%
\end{pgfscope}%
\begin{pgfscope}%
\pgfpathrectangle{\pgfqpoint{3.352233in}{1.400000in}}{\pgfqpoint{2.407767in}{1.544118in}}%
\pgfusepath{clip}%
\pgfsetbuttcap%
\pgfsetroundjoin%
\pgfsetlinewidth{0.501875pt}%
\definecolor{currentstroke}{rgb}{0.268510,0.009605,0.335427}%
\pgfsetstrokecolor{currentstroke}%
\pgfsetdash{}{0pt}%
\pgfpathmoveto{\pgfqpoint{4.509159in}{1.558562in}}%
\pgfpathlineto{\pgfqpoint{4.509160in}{1.558562in}}%
\pgfusepath{stroke}%
\end{pgfscope}%
\begin{pgfscope}%
\pgfpathrectangle{\pgfqpoint{3.352233in}{1.400000in}}{\pgfqpoint{2.407767in}{1.544118in}}%
\pgfusepath{clip}%
\pgfsetbuttcap%
\pgfsetroundjoin%
\pgfsetlinewidth{0.501875pt}%
\definecolor{currentstroke}{rgb}{0.268510,0.009605,0.335427}%
\pgfsetstrokecolor{currentstroke}%
\pgfsetdash{}{0pt}%
\pgfpathmoveto{\pgfqpoint{4.509160in}{1.558562in}}%
\pgfpathlineto{\pgfqpoint{4.509160in}{1.558562in}}%
\pgfusepath{stroke}%
\end{pgfscope}%
\begin{pgfscope}%
\pgfpathrectangle{\pgfqpoint{3.352233in}{1.400000in}}{\pgfqpoint{2.407767in}{1.544118in}}%
\pgfusepath{clip}%
\pgfsetbuttcap%
\pgfsetroundjoin%
\pgfsetlinewidth{0.501875pt}%
\definecolor{currentstroke}{rgb}{0.268510,0.009605,0.335427}%
\pgfsetstrokecolor{currentstroke}%
\pgfsetdash{}{0pt}%
\pgfpathmoveto{\pgfqpoint{4.509160in}{1.558562in}}%
\pgfpathlineto{\pgfqpoint{4.509160in}{1.558562in}}%
\pgfusepath{stroke}%
\end{pgfscope}%
\begin{pgfscope}%
\pgfpathrectangle{\pgfqpoint{3.352233in}{1.400000in}}{\pgfqpoint{2.407767in}{1.544118in}}%
\pgfusepath{clip}%
\pgfsetbuttcap%
\pgfsetroundjoin%
\pgfsetlinewidth{0.501875pt}%
\definecolor{currentstroke}{rgb}{0.268510,0.009605,0.335427}%
\pgfsetstrokecolor{currentstroke}%
\pgfsetdash{}{0pt}%
\pgfpathmoveto{\pgfqpoint{4.509160in}{1.558562in}}%
\pgfpathlineto{\pgfqpoint{4.509159in}{1.558562in}}%
\pgfusepath{stroke}%
\end{pgfscope}%
\begin{pgfscope}%
\pgfpathrectangle{\pgfqpoint{3.352233in}{1.400000in}}{\pgfqpoint{2.407767in}{1.544118in}}%
\pgfusepath{clip}%
\pgfsetbuttcap%
\pgfsetroundjoin%
\pgfsetlinewidth{0.501875pt}%
\definecolor{currentstroke}{rgb}{0.268510,0.009605,0.335427}%
\pgfsetstrokecolor{currentstroke}%
\pgfsetdash{}{0pt}%
\pgfpathmoveto{\pgfqpoint{4.509159in}{1.558562in}}%
\pgfpathlineto{\pgfqpoint{4.509159in}{1.558562in}}%
\pgfusepath{stroke}%
\end{pgfscope}%
\begin{pgfscope}%
\pgfpathrectangle{\pgfqpoint{3.352233in}{1.400000in}}{\pgfqpoint{2.407767in}{1.544118in}}%
\pgfusepath{clip}%
\pgfsetbuttcap%
\pgfsetroundjoin%
\pgfsetlinewidth{0.501875pt}%
\definecolor{currentstroke}{rgb}{0.268510,0.009605,0.335427}%
\pgfsetstrokecolor{currentstroke}%
\pgfsetdash{}{0pt}%
\pgfpathmoveto{\pgfqpoint{4.509159in}{1.558562in}}%
\pgfpathlineto{\pgfqpoint{4.509160in}{1.558562in}}%
\pgfusepath{stroke}%
\end{pgfscope}%
\begin{pgfscope}%
\pgfpathrectangle{\pgfqpoint{3.352233in}{1.400000in}}{\pgfqpoint{2.407767in}{1.544118in}}%
\pgfusepath{clip}%
\pgfsetbuttcap%
\pgfsetroundjoin%
\pgfsetlinewidth{0.501875pt}%
\definecolor{currentstroke}{rgb}{0.268510,0.009605,0.335427}%
\pgfsetstrokecolor{currentstroke}%
\pgfsetdash{}{0pt}%
\pgfpathmoveto{\pgfqpoint{4.509160in}{1.558562in}}%
\pgfpathlineto{\pgfqpoint{4.509160in}{1.558562in}}%
\pgfusepath{stroke}%
\end{pgfscope}%
\begin{pgfscope}%
\pgfpathrectangle{\pgfqpoint{3.352233in}{1.400000in}}{\pgfqpoint{2.407767in}{1.544118in}}%
\pgfusepath{clip}%
\pgfsetbuttcap%
\pgfsetroundjoin%
\pgfsetlinewidth{0.501875pt}%
\definecolor{currentstroke}{rgb}{0.268510,0.009605,0.335427}%
\pgfsetstrokecolor{currentstroke}%
\pgfsetdash{}{0pt}%
\pgfpathmoveto{\pgfqpoint{4.509160in}{1.558562in}}%
\pgfpathlineto{\pgfqpoint{4.509160in}{1.558562in}}%
\pgfusepath{stroke}%
\end{pgfscope}%
\begin{pgfscope}%
\pgfpathrectangle{\pgfqpoint{3.352233in}{1.400000in}}{\pgfqpoint{2.407767in}{1.544118in}}%
\pgfusepath{clip}%
\pgfsetbuttcap%
\pgfsetroundjoin%
\pgfsetlinewidth{0.501875pt}%
\definecolor{currentstroke}{rgb}{0.268510,0.009605,0.335427}%
\pgfsetstrokecolor{currentstroke}%
\pgfsetdash{}{0pt}%
\pgfpathmoveto{\pgfqpoint{4.509160in}{1.558562in}}%
\pgfpathlineto{\pgfqpoint{4.509160in}{1.558562in}}%
\pgfusepath{stroke}%
\end{pgfscope}%
\begin{pgfscope}%
\pgfpathrectangle{\pgfqpoint{3.352233in}{1.400000in}}{\pgfqpoint{2.407767in}{1.544118in}}%
\pgfusepath{clip}%
\pgfsetbuttcap%
\pgfsetroundjoin%
\pgfsetlinewidth{0.501875pt}%
\definecolor{currentstroke}{rgb}{0.268510,0.009605,0.335427}%
\pgfsetstrokecolor{currentstroke}%
\pgfsetdash{}{0pt}%
\pgfpathmoveto{\pgfqpoint{4.509160in}{1.558562in}}%
\pgfpathlineto{\pgfqpoint{4.509160in}{1.558562in}}%
\pgfusepath{stroke}%
\end{pgfscope}%
\begin{pgfscope}%
\pgfpathrectangle{\pgfqpoint{3.352233in}{1.400000in}}{\pgfqpoint{2.407767in}{1.544118in}}%
\pgfusepath{clip}%
\pgfsetbuttcap%
\pgfsetroundjoin%
\pgfsetlinewidth{0.501875pt}%
\definecolor{currentstroke}{rgb}{0.268510,0.009605,0.335427}%
\pgfsetstrokecolor{currentstroke}%
\pgfsetdash{}{0pt}%
\pgfpathmoveto{\pgfqpoint{4.509160in}{1.558562in}}%
\pgfpathlineto{\pgfqpoint{4.509160in}{1.558562in}}%
\pgfusepath{stroke}%
\end{pgfscope}%
\begin{pgfscope}%
\pgfpathrectangle{\pgfqpoint{3.352233in}{1.400000in}}{\pgfqpoint{2.407767in}{1.544118in}}%
\pgfusepath{clip}%
\pgfsetbuttcap%
\pgfsetroundjoin%
\pgfsetlinewidth{0.501875pt}%
\definecolor{currentstroke}{rgb}{0.268510,0.009605,0.335427}%
\pgfsetstrokecolor{currentstroke}%
\pgfsetdash{}{0pt}%
\pgfpathmoveto{\pgfqpoint{4.509160in}{1.558562in}}%
\pgfpathlineto{\pgfqpoint{4.509160in}{1.558562in}}%
\pgfusepath{stroke}%
\end{pgfscope}%
\begin{pgfscope}%
\pgfpathrectangle{\pgfqpoint{3.352233in}{1.400000in}}{\pgfqpoint{2.407767in}{1.544118in}}%
\pgfusepath{clip}%
\pgfsetbuttcap%
\pgfsetroundjoin%
\pgfsetlinewidth{0.501875pt}%
\definecolor{currentstroke}{rgb}{0.268510,0.009605,0.335427}%
\pgfsetstrokecolor{currentstroke}%
\pgfsetdash{}{0pt}%
\pgfpathmoveto{\pgfqpoint{4.509160in}{1.558562in}}%
\pgfpathlineto{\pgfqpoint{4.509160in}{1.558562in}}%
\pgfusepath{stroke}%
\end{pgfscope}%
\begin{pgfscope}%
\pgfpathrectangle{\pgfqpoint{3.352233in}{1.400000in}}{\pgfqpoint{2.407767in}{1.544118in}}%
\pgfusepath{clip}%
\pgfsetbuttcap%
\pgfsetroundjoin%
\pgfsetlinewidth{0.501875pt}%
\definecolor{currentstroke}{rgb}{0.268510,0.009605,0.335427}%
\pgfsetstrokecolor{currentstroke}%
\pgfsetdash{}{0pt}%
\pgfpathmoveto{\pgfqpoint{4.509160in}{1.558562in}}%
\pgfpathlineto{\pgfqpoint{4.509160in}{1.558562in}}%
\pgfusepath{stroke}%
\end{pgfscope}%
\begin{pgfscope}%
\pgfpathrectangle{\pgfqpoint{3.352233in}{1.400000in}}{\pgfqpoint{2.407767in}{1.544118in}}%
\pgfusepath{clip}%
\pgfsetbuttcap%
\pgfsetroundjoin%
\pgfsetlinewidth{0.501875pt}%
\definecolor{currentstroke}{rgb}{0.268510,0.009605,0.335427}%
\pgfsetstrokecolor{currentstroke}%
\pgfsetdash{}{0pt}%
\pgfpathmoveto{\pgfqpoint{4.509160in}{1.558562in}}%
\pgfpathlineto{\pgfqpoint{4.509160in}{1.558562in}}%
\pgfusepath{stroke}%
\end{pgfscope}%
\begin{pgfscope}%
\pgfpathrectangle{\pgfqpoint{3.352233in}{1.400000in}}{\pgfqpoint{2.407767in}{1.544118in}}%
\pgfusepath{clip}%
\pgfsetbuttcap%
\pgfsetroundjoin%
\pgfsetlinewidth{0.501875pt}%
\definecolor{currentstroke}{rgb}{0.268510,0.009605,0.335427}%
\pgfsetstrokecolor{currentstroke}%
\pgfsetdash{}{0pt}%
\pgfpathmoveto{\pgfqpoint{4.509160in}{1.558562in}}%
\pgfpathlineto{\pgfqpoint{4.509160in}{1.558562in}}%
\pgfusepath{stroke}%
\end{pgfscope}%
\begin{pgfscope}%
\pgfpathrectangle{\pgfqpoint{3.352233in}{1.400000in}}{\pgfqpoint{2.407767in}{1.544118in}}%
\pgfusepath{clip}%
\pgfsetbuttcap%
\pgfsetroundjoin%
\pgfsetlinewidth{0.501875pt}%
\definecolor{currentstroke}{rgb}{0.268510,0.009605,0.335427}%
\pgfsetstrokecolor{currentstroke}%
\pgfsetdash{}{0pt}%
\pgfpathmoveto{\pgfqpoint{4.509160in}{1.558562in}}%
\pgfpathlineto{\pgfqpoint{4.509160in}{1.558562in}}%
\pgfusepath{stroke}%
\end{pgfscope}%
\begin{pgfscope}%
\pgfpathrectangle{\pgfqpoint{3.352233in}{1.400000in}}{\pgfqpoint{2.407767in}{1.544118in}}%
\pgfusepath{clip}%
\pgfsetbuttcap%
\pgfsetroundjoin%
\pgfsetlinewidth{0.501875pt}%
\definecolor{currentstroke}{rgb}{0.268510,0.009605,0.335427}%
\pgfsetstrokecolor{currentstroke}%
\pgfsetdash{}{0pt}%
\pgfpathmoveto{\pgfqpoint{4.509160in}{1.558562in}}%
\pgfpathlineto{\pgfqpoint{4.509160in}{1.558562in}}%
\pgfusepath{stroke}%
\end{pgfscope}%
\begin{pgfscope}%
\pgfpathrectangle{\pgfqpoint{3.352233in}{1.400000in}}{\pgfqpoint{2.407767in}{1.544118in}}%
\pgfusepath{clip}%
\pgfsetbuttcap%
\pgfsetroundjoin%
\pgfsetlinewidth{0.501875pt}%
\definecolor{currentstroke}{rgb}{0.268510,0.009605,0.335427}%
\pgfsetstrokecolor{currentstroke}%
\pgfsetdash{}{0pt}%
\pgfpathmoveto{\pgfqpoint{4.509160in}{1.558562in}}%
\pgfpathlineto{\pgfqpoint{4.509160in}{1.558562in}}%
\pgfusepath{stroke}%
\end{pgfscope}%
\begin{pgfscope}%
\pgfpathrectangle{\pgfqpoint{3.352233in}{1.400000in}}{\pgfqpoint{2.407767in}{1.544118in}}%
\pgfusepath{clip}%
\pgfsetbuttcap%
\pgfsetroundjoin%
\pgfsetlinewidth{0.501875pt}%
\definecolor{currentstroke}{rgb}{0.268510,0.009605,0.335427}%
\pgfsetstrokecolor{currentstroke}%
\pgfsetdash{}{0pt}%
\pgfpathmoveto{\pgfqpoint{4.509160in}{1.558562in}}%
\pgfpathlineto{\pgfqpoint{4.509160in}{1.558562in}}%
\pgfusepath{stroke}%
\end{pgfscope}%
\begin{pgfscope}%
\pgfpathrectangle{\pgfqpoint{3.352233in}{1.400000in}}{\pgfqpoint{2.407767in}{1.544118in}}%
\pgfusepath{clip}%
\pgfsetbuttcap%
\pgfsetroundjoin%
\pgfsetlinewidth{0.501875pt}%
\definecolor{currentstroke}{rgb}{0.268510,0.009605,0.335427}%
\pgfsetstrokecolor{currentstroke}%
\pgfsetdash{}{0pt}%
\pgfpathmoveto{\pgfqpoint{4.509160in}{1.558562in}}%
\pgfpathlineto{\pgfqpoint{4.509160in}{1.558562in}}%
\pgfusepath{stroke}%
\end{pgfscope}%
\begin{pgfscope}%
\pgfpathrectangle{\pgfqpoint{3.352233in}{1.400000in}}{\pgfqpoint{2.407767in}{1.544118in}}%
\pgfusepath{clip}%
\pgfsetbuttcap%
\pgfsetroundjoin%
\pgfsetlinewidth{0.501875pt}%
\definecolor{currentstroke}{rgb}{0.268510,0.009605,0.335427}%
\pgfsetstrokecolor{currentstroke}%
\pgfsetdash{}{0pt}%
\pgfpathmoveto{\pgfqpoint{4.509160in}{1.558562in}}%
\pgfpathlineto{\pgfqpoint{4.509160in}{1.558562in}}%
\pgfusepath{stroke}%
\end{pgfscope}%
\begin{pgfscope}%
\pgfpathrectangle{\pgfqpoint{3.352233in}{1.400000in}}{\pgfqpoint{2.407767in}{1.544118in}}%
\pgfusepath{clip}%
\pgfsetbuttcap%
\pgfsetroundjoin%
\pgfsetlinewidth{0.501875pt}%
\definecolor{currentstroke}{rgb}{0.268510,0.009605,0.335427}%
\pgfsetstrokecolor{currentstroke}%
\pgfsetdash{}{0pt}%
\pgfpathmoveto{\pgfqpoint{4.509160in}{1.558562in}}%
\pgfpathlineto{\pgfqpoint{4.509160in}{1.558562in}}%
\pgfusepath{stroke}%
\end{pgfscope}%
\begin{pgfscope}%
\pgfpathrectangle{\pgfqpoint{3.352233in}{1.400000in}}{\pgfqpoint{2.407767in}{1.544118in}}%
\pgfusepath{clip}%
\pgfsetbuttcap%
\pgfsetroundjoin%
\pgfsetlinewidth{0.501875pt}%
\definecolor{currentstroke}{rgb}{0.268510,0.009605,0.335427}%
\pgfsetstrokecolor{currentstroke}%
\pgfsetdash{}{0pt}%
\pgfpathmoveto{\pgfqpoint{4.509160in}{1.558562in}}%
\pgfpathlineto{\pgfqpoint{4.509160in}{1.558562in}}%
\pgfusepath{stroke}%
\end{pgfscope}%
\begin{pgfscope}%
\pgfpathrectangle{\pgfqpoint{3.352233in}{1.400000in}}{\pgfqpoint{2.407767in}{1.544118in}}%
\pgfusepath{clip}%
\pgfsetbuttcap%
\pgfsetroundjoin%
\pgfsetlinewidth{0.501875pt}%
\definecolor{currentstroke}{rgb}{0.268510,0.009605,0.335427}%
\pgfsetstrokecolor{currentstroke}%
\pgfsetdash{}{0pt}%
\pgfpathmoveto{\pgfqpoint{4.509160in}{1.558562in}}%
\pgfpathlineto{\pgfqpoint{4.509160in}{1.558562in}}%
\pgfusepath{stroke}%
\end{pgfscope}%
\begin{pgfscope}%
\pgfpathrectangle{\pgfqpoint{3.352233in}{1.400000in}}{\pgfqpoint{2.407767in}{1.544118in}}%
\pgfusepath{clip}%
\pgfsetbuttcap%
\pgfsetroundjoin%
\pgfsetlinewidth{0.501875pt}%
\definecolor{currentstroke}{rgb}{0.268510,0.009605,0.335427}%
\pgfsetstrokecolor{currentstroke}%
\pgfsetdash{}{0pt}%
\pgfpathmoveto{\pgfqpoint{4.509160in}{1.558562in}}%
\pgfpathlineto{\pgfqpoint{4.509160in}{1.558562in}}%
\pgfusepath{stroke}%
\end{pgfscope}%
\begin{pgfscope}%
\pgfpathrectangle{\pgfqpoint{3.352233in}{1.400000in}}{\pgfqpoint{2.407767in}{1.544118in}}%
\pgfusepath{clip}%
\pgfsetbuttcap%
\pgfsetroundjoin%
\pgfsetlinewidth{0.501875pt}%
\definecolor{currentstroke}{rgb}{0.268510,0.009605,0.335427}%
\pgfsetstrokecolor{currentstroke}%
\pgfsetdash{}{0pt}%
\pgfpathmoveto{\pgfqpoint{4.509160in}{1.558562in}}%
\pgfpathlineto{\pgfqpoint{4.509160in}{1.558562in}}%
\pgfusepath{stroke}%
\end{pgfscope}%
\begin{pgfscope}%
\pgfpathrectangle{\pgfqpoint{3.352233in}{1.400000in}}{\pgfqpoint{2.407767in}{1.544118in}}%
\pgfusepath{clip}%
\pgfsetbuttcap%
\pgfsetroundjoin%
\pgfsetlinewidth{0.501875pt}%
\definecolor{currentstroke}{rgb}{0.268510,0.009605,0.335427}%
\pgfsetstrokecolor{currentstroke}%
\pgfsetdash{}{0pt}%
\pgfpathmoveto{\pgfqpoint{4.509160in}{1.558562in}}%
\pgfpathlineto{\pgfqpoint{4.509160in}{1.558562in}}%
\pgfusepath{stroke}%
\end{pgfscope}%
\begin{pgfscope}%
\pgfpathrectangle{\pgfqpoint{3.352233in}{1.400000in}}{\pgfqpoint{2.407767in}{1.544118in}}%
\pgfusepath{clip}%
\pgfsetbuttcap%
\pgfsetroundjoin%
\pgfsetlinewidth{0.501875pt}%
\definecolor{currentstroke}{rgb}{0.268510,0.009605,0.335427}%
\pgfsetstrokecolor{currentstroke}%
\pgfsetdash{}{0pt}%
\pgfpathmoveto{\pgfqpoint{4.509160in}{1.558562in}}%
\pgfpathlineto{\pgfqpoint{4.509160in}{1.558562in}}%
\pgfusepath{stroke}%
\end{pgfscope}%
\begin{pgfscope}%
\pgfpathrectangle{\pgfqpoint{3.352233in}{1.400000in}}{\pgfqpoint{2.407767in}{1.544118in}}%
\pgfusepath{clip}%
\pgfsetbuttcap%
\pgfsetroundjoin%
\pgfsetlinewidth{0.501875pt}%
\definecolor{currentstroke}{rgb}{0.268510,0.009605,0.335427}%
\pgfsetstrokecolor{currentstroke}%
\pgfsetdash{}{0pt}%
\pgfpathmoveto{\pgfqpoint{4.509160in}{1.558562in}}%
\pgfpathlineto{\pgfqpoint{4.509160in}{1.558562in}}%
\pgfusepath{stroke}%
\end{pgfscope}%
\begin{pgfscope}%
\pgfpathrectangle{\pgfqpoint{3.352233in}{1.400000in}}{\pgfqpoint{2.407767in}{1.544118in}}%
\pgfusepath{clip}%
\pgfsetbuttcap%
\pgfsetroundjoin%
\pgfsetlinewidth{0.501875pt}%
\definecolor{currentstroke}{rgb}{0.268510,0.009605,0.335427}%
\pgfsetstrokecolor{currentstroke}%
\pgfsetdash{}{0pt}%
\pgfpathmoveto{\pgfqpoint{4.509160in}{1.558562in}}%
\pgfpathlineto{\pgfqpoint{4.509160in}{1.558562in}}%
\pgfusepath{stroke}%
\end{pgfscope}%
\begin{pgfscope}%
\pgfpathrectangle{\pgfqpoint{3.352233in}{1.400000in}}{\pgfqpoint{2.407767in}{1.544118in}}%
\pgfusepath{clip}%
\pgfsetbuttcap%
\pgfsetroundjoin%
\pgfsetlinewidth{0.501875pt}%
\definecolor{currentstroke}{rgb}{0.268510,0.009605,0.335427}%
\pgfsetstrokecolor{currentstroke}%
\pgfsetdash{}{0pt}%
\pgfpathmoveto{\pgfqpoint{4.509160in}{1.558562in}}%
\pgfpathlineto{\pgfqpoint{4.509160in}{1.558562in}}%
\pgfusepath{stroke}%
\end{pgfscope}%
\begin{pgfscope}%
\pgfpathrectangle{\pgfqpoint{3.352233in}{1.400000in}}{\pgfqpoint{2.407767in}{1.544118in}}%
\pgfusepath{clip}%
\pgfsetbuttcap%
\pgfsetroundjoin%
\pgfsetlinewidth{0.501875pt}%
\definecolor{currentstroke}{rgb}{0.268510,0.009605,0.335427}%
\pgfsetstrokecolor{currentstroke}%
\pgfsetdash{}{0pt}%
\pgfpathmoveto{\pgfqpoint{4.509160in}{1.558562in}}%
\pgfpathlineto{\pgfqpoint{4.509160in}{1.558562in}}%
\pgfusepath{stroke}%
\end{pgfscope}%
\begin{pgfscope}%
\pgfpathrectangle{\pgfqpoint{3.352233in}{1.400000in}}{\pgfqpoint{2.407767in}{1.544118in}}%
\pgfusepath{clip}%
\pgfsetbuttcap%
\pgfsetroundjoin%
\pgfsetlinewidth{0.501875pt}%
\definecolor{currentstroke}{rgb}{0.268510,0.009605,0.335427}%
\pgfsetstrokecolor{currentstroke}%
\pgfsetdash{}{0pt}%
\pgfpathmoveto{\pgfqpoint{4.509160in}{1.558562in}}%
\pgfpathlineto{\pgfqpoint{4.509160in}{1.558562in}}%
\pgfusepath{stroke}%
\end{pgfscope}%
\begin{pgfscope}%
\pgfpathrectangle{\pgfqpoint{3.352233in}{1.400000in}}{\pgfqpoint{2.407767in}{1.544118in}}%
\pgfusepath{clip}%
\pgfsetbuttcap%
\pgfsetroundjoin%
\pgfsetlinewidth{0.501875pt}%
\definecolor{currentstroke}{rgb}{0.268510,0.009605,0.335427}%
\pgfsetstrokecolor{currentstroke}%
\pgfsetdash{}{0pt}%
\pgfpathmoveto{\pgfqpoint{4.509160in}{1.558562in}}%
\pgfpathlineto{\pgfqpoint{4.509160in}{1.558562in}}%
\pgfusepath{stroke}%
\end{pgfscope}%
\begin{pgfscope}%
\pgfpathrectangle{\pgfqpoint{3.352233in}{1.400000in}}{\pgfqpoint{2.407767in}{1.544118in}}%
\pgfusepath{clip}%
\pgfsetbuttcap%
\pgfsetroundjoin%
\pgfsetlinewidth{0.501875pt}%
\definecolor{currentstroke}{rgb}{0.268510,0.009605,0.335427}%
\pgfsetstrokecolor{currentstroke}%
\pgfsetdash{}{0pt}%
\pgfpathmoveto{\pgfqpoint{4.509160in}{1.558562in}}%
\pgfpathlineto{\pgfqpoint{4.509160in}{1.558562in}}%
\pgfusepath{stroke}%
\end{pgfscope}%
\begin{pgfscope}%
\pgfpathrectangle{\pgfqpoint{3.352233in}{1.400000in}}{\pgfqpoint{2.407767in}{1.544118in}}%
\pgfusepath{clip}%
\pgfsetbuttcap%
\pgfsetroundjoin%
\pgfsetlinewidth{0.501875pt}%
\definecolor{currentstroke}{rgb}{0.268510,0.009605,0.335427}%
\pgfsetstrokecolor{currentstroke}%
\pgfsetdash{}{0pt}%
\pgfpathmoveto{\pgfqpoint{4.509160in}{1.558562in}}%
\pgfpathlineto{\pgfqpoint{4.509160in}{1.558562in}}%
\pgfusepath{stroke}%
\end{pgfscope}%
\begin{pgfscope}%
\pgfpathrectangle{\pgfqpoint{3.352233in}{1.400000in}}{\pgfqpoint{2.407767in}{1.544118in}}%
\pgfusepath{clip}%
\pgfsetbuttcap%
\pgfsetroundjoin%
\pgfsetlinewidth{0.501875pt}%
\definecolor{currentstroke}{rgb}{0.268510,0.009605,0.335427}%
\pgfsetstrokecolor{currentstroke}%
\pgfsetdash{}{0pt}%
\pgfpathmoveto{\pgfqpoint{4.509160in}{1.558562in}}%
\pgfpathlineto{\pgfqpoint{4.509160in}{1.558562in}}%
\pgfusepath{stroke}%
\end{pgfscope}%
\begin{pgfscope}%
\pgfpathrectangle{\pgfqpoint{3.352233in}{1.400000in}}{\pgfqpoint{2.407767in}{1.544118in}}%
\pgfusepath{clip}%
\pgfsetbuttcap%
\pgfsetroundjoin%
\pgfsetlinewidth{0.501875pt}%
\definecolor{currentstroke}{rgb}{0.268510,0.009605,0.335427}%
\pgfsetstrokecolor{currentstroke}%
\pgfsetdash{}{0pt}%
\pgfpathmoveto{\pgfqpoint{4.509160in}{1.558562in}}%
\pgfpathlineto{\pgfqpoint{4.509160in}{1.558562in}}%
\pgfusepath{stroke}%
\end{pgfscope}%
\begin{pgfscope}%
\pgfpathrectangle{\pgfqpoint{3.352233in}{1.400000in}}{\pgfqpoint{2.407767in}{1.544118in}}%
\pgfusepath{clip}%
\pgfsetbuttcap%
\pgfsetroundjoin%
\pgfsetlinewidth{0.501875pt}%
\definecolor{currentstroke}{rgb}{0.268510,0.009605,0.335427}%
\pgfsetstrokecolor{currentstroke}%
\pgfsetdash{}{0pt}%
\pgfpathmoveto{\pgfqpoint{4.509160in}{1.558562in}}%
\pgfpathlineto{\pgfqpoint{4.509160in}{1.558562in}}%
\pgfusepath{stroke}%
\end{pgfscope}%
\begin{pgfscope}%
\pgfpathrectangle{\pgfqpoint{3.352233in}{1.400000in}}{\pgfqpoint{2.407767in}{1.544118in}}%
\pgfusepath{clip}%
\pgfsetbuttcap%
\pgfsetroundjoin%
\pgfsetlinewidth{0.501875pt}%
\definecolor{currentstroke}{rgb}{0.268510,0.009605,0.335427}%
\pgfsetstrokecolor{currentstroke}%
\pgfsetdash{}{0pt}%
\pgfpathmoveto{\pgfqpoint{4.509160in}{1.558562in}}%
\pgfpathlineto{\pgfqpoint{4.509160in}{1.558562in}}%
\pgfusepath{stroke}%
\end{pgfscope}%
\begin{pgfscope}%
\pgfpathrectangle{\pgfqpoint{3.352233in}{1.400000in}}{\pgfqpoint{2.407767in}{1.544118in}}%
\pgfusepath{clip}%
\pgfsetbuttcap%
\pgfsetroundjoin%
\pgfsetlinewidth{0.501875pt}%
\definecolor{currentstroke}{rgb}{0.268510,0.009605,0.335427}%
\pgfsetstrokecolor{currentstroke}%
\pgfsetdash{}{0pt}%
\pgfpathmoveto{\pgfqpoint{4.509160in}{1.558562in}}%
\pgfpathlineto{\pgfqpoint{4.509160in}{1.558562in}}%
\pgfusepath{stroke}%
\end{pgfscope}%
\begin{pgfscope}%
\pgfpathrectangle{\pgfqpoint{3.352233in}{1.400000in}}{\pgfqpoint{2.407767in}{1.544118in}}%
\pgfusepath{clip}%
\pgfsetbuttcap%
\pgfsetroundjoin%
\pgfsetlinewidth{0.501875pt}%
\definecolor{currentstroke}{rgb}{0.268510,0.009605,0.335427}%
\pgfsetstrokecolor{currentstroke}%
\pgfsetdash{}{0pt}%
\pgfpathmoveto{\pgfqpoint{4.509160in}{1.558562in}}%
\pgfpathlineto{\pgfqpoint{4.509160in}{1.558562in}}%
\pgfusepath{stroke}%
\end{pgfscope}%
\begin{pgfscope}%
\pgfpathrectangle{\pgfqpoint{3.352233in}{1.400000in}}{\pgfqpoint{2.407767in}{1.544118in}}%
\pgfusepath{clip}%
\pgfsetbuttcap%
\pgfsetroundjoin%
\pgfsetlinewidth{0.501875pt}%
\definecolor{currentstroke}{rgb}{0.268510,0.009605,0.335427}%
\pgfsetstrokecolor{currentstroke}%
\pgfsetdash{}{0pt}%
\pgfpathmoveto{\pgfqpoint{4.509160in}{1.558562in}}%
\pgfpathlineto{\pgfqpoint{4.509160in}{1.558562in}}%
\pgfusepath{stroke}%
\end{pgfscope}%
\begin{pgfscope}%
\pgfpathrectangle{\pgfqpoint{3.352233in}{1.400000in}}{\pgfqpoint{2.407767in}{1.544118in}}%
\pgfusepath{clip}%
\pgfsetbuttcap%
\pgfsetroundjoin%
\pgfsetlinewidth{0.501875pt}%
\definecolor{currentstroke}{rgb}{0.268510,0.009605,0.335427}%
\pgfsetstrokecolor{currentstroke}%
\pgfsetdash{}{0pt}%
\pgfpathmoveto{\pgfqpoint{4.509160in}{1.558562in}}%
\pgfpathlineto{\pgfqpoint{4.509160in}{1.558562in}}%
\pgfusepath{stroke}%
\end{pgfscope}%
\begin{pgfscope}%
\pgfpathrectangle{\pgfqpoint{3.352233in}{1.400000in}}{\pgfqpoint{2.407767in}{1.544118in}}%
\pgfusepath{clip}%
\pgfsetbuttcap%
\pgfsetroundjoin%
\pgfsetlinewidth{0.501875pt}%
\definecolor{currentstroke}{rgb}{0.268510,0.009605,0.335427}%
\pgfsetstrokecolor{currentstroke}%
\pgfsetdash{}{0pt}%
\pgfpathmoveto{\pgfqpoint{4.509160in}{1.558562in}}%
\pgfpathlineto{\pgfqpoint{4.509160in}{1.558562in}}%
\pgfusepath{stroke}%
\end{pgfscope}%
\begin{pgfscope}%
\pgfpathrectangle{\pgfqpoint{3.352233in}{1.400000in}}{\pgfqpoint{2.407767in}{1.544118in}}%
\pgfusepath{clip}%
\pgfsetbuttcap%
\pgfsetroundjoin%
\pgfsetlinewidth{0.501875pt}%
\definecolor{currentstroke}{rgb}{0.268510,0.009605,0.335427}%
\pgfsetstrokecolor{currentstroke}%
\pgfsetdash{}{0pt}%
\pgfpathmoveto{\pgfqpoint{4.509160in}{1.558562in}}%
\pgfpathlineto{\pgfqpoint{4.509160in}{1.558562in}}%
\pgfusepath{stroke}%
\end{pgfscope}%
\begin{pgfscope}%
\pgfpathrectangle{\pgfqpoint{3.352233in}{1.400000in}}{\pgfqpoint{2.407767in}{1.544118in}}%
\pgfusepath{clip}%
\pgfsetbuttcap%
\pgfsetroundjoin%
\pgfsetlinewidth{0.501875pt}%
\definecolor{currentstroke}{rgb}{0.268510,0.009605,0.335427}%
\pgfsetstrokecolor{currentstroke}%
\pgfsetdash{}{0pt}%
\pgfpathmoveto{\pgfqpoint{4.509160in}{1.558562in}}%
\pgfpathlineto{\pgfqpoint{4.509160in}{1.558562in}}%
\pgfusepath{stroke}%
\end{pgfscope}%
\begin{pgfscope}%
\pgfpathrectangle{\pgfqpoint{3.352233in}{1.400000in}}{\pgfqpoint{2.407767in}{1.544118in}}%
\pgfusepath{clip}%
\pgfsetbuttcap%
\pgfsetroundjoin%
\pgfsetlinewidth{0.501875pt}%
\definecolor{currentstroke}{rgb}{0.268510,0.009605,0.335427}%
\pgfsetstrokecolor{currentstroke}%
\pgfsetdash{}{0pt}%
\pgfpathmoveto{\pgfqpoint{4.509160in}{1.558562in}}%
\pgfpathlineto{\pgfqpoint{4.509160in}{1.558562in}}%
\pgfusepath{stroke}%
\end{pgfscope}%
\begin{pgfscope}%
\pgfpathrectangle{\pgfqpoint{3.352233in}{1.400000in}}{\pgfqpoint{2.407767in}{1.544118in}}%
\pgfusepath{clip}%
\pgfsetbuttcap%
\pgfsetroundjoin%
\pgfsetlinewidth{0.501875pt}%
\definecolor{currentstroke}{rgb}{0.268510,0.009605,0.335427}%
\pgfsetstrokecolor{currentstroke}%
\pgfsetdash{}{0pt}%
\pgfpathmoveto{\pgfqpoint{4.509160in}{1.558562in}}%
\pgfpathlineto{\pgfqpoint{4.509160in}{1.558562in}}%
\pgfusepath{stroke}%
\end{pgfscope}%
\begin{pgfscope}%
\pgfpathrectangle{\pgfqpoint{3.352233in}{1.400000in}}{\pgfqpoint{2.407767in}{1.544118in}}%
\pgfusepath{clip}%
\pgfsetbuttcap%
\pgfsetroundjoin%
\pgfsetlinewidth{0.501875pt}%
\definecolor{currentstroke}{rgb}{0.268510,0.009605,0.335427}%
\pgfsetstrokecolor{currentstroke}%
\pgfsetdash{}{0pt}%
\pgfpathmoveto{\pgfqpoint{4.509160in}{1.558562in}}%
\pgfpathlineto{\pgfqpoint{4.509160in}{1.558562in}}%
\pgfusepath{stroke}%
\end{pgfscope}%
\begin{pgfscope}%
\pgfpathrectangle{\pgfqpoint{3.352233in}{1.400000in}}{\pgfqpoint{2.407767in}{1.544118in}}%
\pgfusepath{clip}%
\pgfsetbuttcap%
\pgfsetroundjoin%
\pgfsetlinewidth{0.501875pt}%
\definecolor{currentstroke}{rgb}{0.268510,0.009605,0.335427}%
\pgfsetstrokecolor{currentstroke}%
\pgfsetdash{}{0pt}%
\pgfpathmoveto{\pgfqpoint{4.509160in}{1.558562in}}%
\pgfpathlineto{\pgfqpoint{4.509160in}{1.558562in}}%
\pgfusepath{stroke}%
\end{pgfscope}%
\begin{pgfscope}%
\pgfpathrectangle{\pgfqpoint{3.352233in}{1.400000in}}{\pgfqpoint{2.407767in}{1.544118in}}%
\pgfusepath{clip}%
\pgfsetbuttcap%
\pgfsetroundjoin%
\pgfsetlinewidth{0.501875pt}%
\definecolor{currentstroke}{rgb}{0.268510,0.009605,0.335427}%
\pgfsetstrokecolor{currentstroke}%
\pgfsetdash{}{0pt}%
\pgfpathmoveto{\pgfqpoint{4.509160in}{1.558562in}}%
\pgfpathlineto{\pgfqpoint{4.509160in}{1.558562in}}%
\pgfusepath{stroke}%
\end{pgfscope}%
\begin{pgfscope}%
\pgfpathrectangle{\pgfqpoint{3.352233in}{1.400000in}}{\pgfqpoint{2.407767in}{1.544118in}}%
\pgfusepath{clip}%
\pgfsetbuttcap%
\pgfsetroundjoin%
\pgfsetlinewidth{0.501875pt}%
\definecolor{currentstroke}{rgb}{0.268510,0.009605,0.335427}%
\pgfsetstrokecolor{currentstroke}%
\pgfsetdash{}{0pt}%
\pgfpathmoveto{\pgfqpoint{4.509160in}{1.558562in}}%
\pgfpathlineto{\pgfqpoint{4.509160in}{1.558562in}}%
\pgfusepath{stroke}%
\end{pgfscope}%
\begin{pgfscope}%
\pgfpathrectangle{\pgfqpoint{3.352233in}{1.400000in}}{\pgfqpoint{2.407767in}{1.544118in}}%
\pgfusepath{clip}%
\pgfsetbuttcap%
\pgfsetroundjoin%
\pgfsetlinewidth{0.501875pt}%
\definecolor{currentstroke}{rgb}{0.268510,0.009605,0.335427}%
\pgfsetstrokecolor{currentstroke}%
\pgfsetdash{}{0pt}%
\pgfpathmoveto{\pgfqpoint{4.509160in}{1.558562in}}%
\pgfpathlineto{\pgfqpoint{4.509160in}{1.558562in}}%
\pgfusepath{stroke}%
\end{pgfscope}%
\begin{pgfscope}%
\pgfpathrectangle{\pgfqpoint{3.352233in}{1.400000in}}{\pgfqpoint{2.407767in}{1.544118in}}%
\pgfusepath{clip}%
\pgfsetbuttcap%
\pgfsetroundjoin%
\pgfsetlinewidth{0.501875pt}%
\definecolor{currentstroke}{rgb}{0.268510,0.009605,0.335427}%
\pgfsetstrokecolor{currentstroke}%
\pgfsetdash{}{0pt}%
\pgfpathmoveto{\pgfqpoint{4.509160in}{1.558562in}}%
\pgfpathlineto{\pgfqpoint{4.509160in}{1.558562in}}%
\pgfusepath{stroke}%
\end{pgfscope}%
\begin{pgfscope}%
\pgfpathrectangle{\pgfqpoint{3.352233in}{1.400000in}}{\pgfqpoint{2.407767in}{1.544118in}}%
\pgfusepath{clip}%
\pgfsetbuttcap%
\pgfsetroundjoin%
\pgfsetlinewidth{0.501875pt}%
\definecolor{currentstroke}{rgb}{0.268510,0.009605,0.335427}%
\pgfsetstrokecolor{currentstroke}%
\pgfsetdash{}{0pt}%
\pgfpathmoveto{\pgfqpoint{4.509160in}{1.558562in}}%
\pgfpathlineto{\pgfqpoint{4.509160in}{1.558562in}}%
\pgfusepath{stroke}%
\end{pgfscope}%
\begin{pgfscope}%
\pgfpathrectangle{\pgfqpoint{3.352233in}{1.400000in}}{\pgfqpoint{2.407767in}{1.544118in}}%
\pgfusepath{clip}%
\pgfsetbuttcap%
\pgfsetroundjoin%
\pgfsetlinewidth{0.501875pt}%
\definecolor{currentstroke}{rgb}{0.268510,0.009605,0.335427}%
\pgfsetstrokecolor{currentstroke}%
\pgfsetdash{}{0pt}%
\pgfpathmoveto{\pgfqpoint{4.509160in}{1.558562in}}%
\pgfpathlineto{\pgfqpoint{4.509160in}{1.558562in}}%
\pgfusepath{stroke}%
\end{pgfscope}%
\begin{pgfscope}%
\pgfpathrectangle{\pgfqpoint{3.352233in}{1.400000in}}{\pgfqpoint{2.407767in}{1.544118in}}%
\pgfusepath{clip}%
\pgfsetbuttcap%
\pgfsetroundjoin%
\pgfsetlinewidth{0.501875pt}%
\definecolor{currentstroke}{rgb}{0.268510,0.009605,0.335427}%
\pgfsetstrokecolor{currentstroke}%
\pgfsetdash{}{0pt}%
\pgfpathmoveto{\pgfqpoint{4.509160in}{1.558562in}}%
\pgfpathlineto{\pgfqpoint{4.509160in}{1.558562in}}%
\pgfusepath{stroke}%
\end{pgfscope}%
\begin{pgfscope}%
\pgfpathrectangle{\pgfqpoint{3.352233in}{1.400000in}}{\pgfqpoint{2.407767in}{1.544118in}}%
\pgfusepath{clip}%
\pgfsetbuttcap%
\pgfsetroundjoin%
\pgfsetlinewidth{0.501875pt}%
\definecolor{currentstroke}{rgb}{0.268510,0.009605,0.335427}%
\pgfsetstrokecolor{currentstroke}%
\pgfsetdash{}{0pt}%
\pgfpathmoveto{\pgfqpoint{4.509160in}{1.558562in}}%
\pgfpathlineto{\pgfqpoint{4.509160in}{1.558562in}}%
\pgfusepath{stroke}%
\end{pgfscope}%
\begin{pgfscope}%
\pgfpathrectangle{\pgfqpoint{3.352233in}{1.400000in}}{\pgfqpoint{2.407767in}{1.544118in}}%
\pgfusepath{clip}%
\pgfsetbuttcap%
\pgfsetroundjoin%
\pgfsetlinewidth{0.501875pt}%
\definecolor{currentstroke}{rgb}{0.268510,0.009605,0.335427}%
\pgfsetstrokecolor{currentstroke}%
\pgfsetdash{}{0pt}%
\pgfpathmoveto{\pgfqpoint{4.509160in}{1.558562in}}%
\pgfpathlineto{\pgfqpoint{4.509160in}{1.558562in}}%
\pgfusepath{stroke}%
\end{pgfscope}%
\begin{pgfscope}%
\pgfpathrectangle{\pgfqpoint{3.352233in}{1.400000in}}{\pgfqpoint{2.407767in}{1.544118in}}%
\pgfusepath{clip}%
\pgfsetbuttcap%
\pgfsetroundjoin%
\pgfsetlinewidth{0.501875pt}%
\definecolor{currentstroke}{rgb}{0.268510,0.009605,0.335427}%
\pgfsetstrokecolor{currentstroke}%
\pgfsetdash{}{0pt}%
\pgfpathmoveto{\pgfqpoint{4.509160in}{1.558562in}}%
\pgfpathlineto{\pgfqpoint{4.509160in}{1.558562in}}%
\pgfusepath{stroke}%
\end{pgfscope}%
\begin{pgfscope}%
\pgfpathrectangle{\pgfqpoint{3.352233in}{1.400000in}}{\pgfqpoint{2.407767in}{1.544118in}}%
\pgfusepath{clip}%
\pgfsetbuttcap%
\pgfsetroundjoin%
\pgfsetlinewidth{0.501875pt}%
\definecolor{currentstroke}{rgb}{0.268510,0.009605,0.335427}%
\pgfsetstrokecolor{currentstroke}%
\pgfsetdash{}{0pt}%
\pgfpathmoveto{\pgfqpoint{4.509160in}{1.558562in}}%
\pgfpathlineto{\pgfqpoint{4.509160in}{1.558562in}}%
\pgfusepath{stroke}%
\end{pgfscope}%
\begin{pgfscope}%
\pgfpathrectangle{\pgfqpoint{3.352233in}{1.400000in}}{\pgfqpoint{2.407767in}{1.544118in}}%
\pgfusepath{clip}%
\pgfsetbuttcap%
\pgfsetroundjoin%
\pgfsetlinewidth{0.501875pt}%
\definecolor{currentstroke}{rgb}{0.268510,0.009605,0.335427}%
\pgfsetstrokecolor{currentstroke}%
\pgfsetdash{}{0pt}%
\pgfpathmoveto{\pgfqpoint{4.509160in}{1.558562in}}%
\pgfpathlineto{\pgfqpoint{4.509160in}{1.558562in}}%
\pgfusepath{stroke}%
\end{pgfscope}%
\begin{pgfscope}%
\pgfpathrectangle{\pgfqpoint{3.352233in}{1.400000in}}{\pgfqpoint{2.407767in}{1.544118in}}%
\pgfusepath{clip}%
\pgfsetbuttcap%
\pgfsetroundjoin%
\pgfsetlinewidth{0.501875pt}%
\definecolor{currentstroke}{rgb}{0.268510,0.009605,0.335427}%
\pgfsetstrokecolor{currentstroke}%
\pgfsetdash{}{0pt}%
\pgfpathmoveto{\pgfqpoint{4.509160in}{1.558562in}}%
\pgfpathlineto{\pgfqpoint{4.509160in}{1.558562in}}%
\pgfusepath{stroke}%
\end{pgfscope}%
\begin{pgfscope}%
\pgfpathrectangle{\pgfqpoint{3.352233in}{1.400000in}}{\pgfqpoint{2.407767in}{1.544118in}}%
\pgfusepath{clip}%
\pgfsetbuttcap%
\pgfsetroundjoin%
\pgfsetlinewidth{0.501875pt}%
\definecolor{currentstroke}{rgb}{0.268510,0.009605,0.335427}%
\pgfsetstrokecolor{currentstroke}%
\pgfsetdash{}{0pt}%
\pgfpathmoveto{\pgfqpoint{4.509160in}{1.558562in}}%
\pgfpathlineto{\pgfqpoint{4.509160in}{1.558562in}}%
\pgfusepath{stroke}%
\end{pgfscope}%
\begin{pgfscope}%
\pgfpathrectangle{\pgfqpoint{3.352233in}{1.400000in}}{\pgfqpoint{2.407767in}{1.544118in}}%
\pgfusepath{clip}%
\pgfsetbuttcap%
\pgfsetroundjoin%
\pgfsetlinewidth{0.501875pt}%
\definecolor{currentstroke}{rgb}{0.268510,0.009605,0.335427}%
\pgfsetstrokecolor{currentstroke}%
\pgfsetdash{}{0pt}%
\pgfpathmoveto{\pgfqpoint{4.509160in}{1.558562in}}%
\pgfpathlineto{\pgfqpoint{4.509160in}{1.558562in}}%
\pgfusepath{stroke}%
\end{pgfscope}%
\begin{pgfscope}%
\pgfpathrectangle{\pgfqpoint{3.352233in}{1.400000in}}{\pgfqpoint{2.407767in}{1.544118in}}%
\pgfusepath{clip}%
\pgfsetbuttcap%
\pgfsetroundjoin%
\pgfsetlinewidth{0.501875pt}%
\definecolor{currentstroke}{rgb}{0.268510,0.009605,0.335427}%
\pgfsetstrokecolor{currentstroke}%
\pgfsetdash{}{0pt}%
\pgfpathmoveto{\pgfqpoint{4.509160in}{1.558562in}}%
\pgfpathlineto{\pgfqpoint{4.509160in}{1.558562in}}%
\pgfusepath{stroke}%
\end{pgfscope}%
\begin{pgfscope}%
\pgfpathrectangle{\pgfqpoint{3.352233in}{1.400000in}}{\pgfqpoint{2.407767in}{1.544118in}}%
\pgfusepath{clip}%
\pgfsetbuttcap%
\pgfsetroundjoin%
\pgfsetlinewidth{0.501875pt}%
\definecolor{currentstroke}{rgb}{0.268510,0.009605,0.335427}%
\pgfsetstrokecolor{currentstroke}%
\pgfsetdash{}{0pt}%
\pgfpathmoveto{\pgfqpoint{4.509160in}{1.558562in}}%
\pgfpathlineto{\pgfqpoint{4.509160in}{1.558562in}}%
\pgfusepath{stroke}%
\end{pgfscope}%
\begin{pgfscope}%
\pgfpathrectangle{\pgfqpoint{3.352233in}{1.400000in}}{\pgfqpoint{2.407767in}{1.544118in}}%
\pgfusepath{clip}%
\pgfsetbuttcap%
\pgfsetroundjoin%
\pgfsetlinewidth{0.501875pt}%
\definecolor{currentstroke}{rgb}{0.268510,0.009605,0.335427}%
\pgfsetstrokecolor{currentstroke}%
\pgfsetdash{}{0pt}%
\pgfpathmoveto{\pgfqpoint{4.509160in}{1.558562in}}%
\pgfpathlineto{\pgfqpoint{4.509160in}{1.558562in}}%
\pgfusepath{stroke}%
\end{pgfscope}%
\begin{pgfscope}%
\pgfpathrectangle{\pgfqpoint{3.352233in}{1.400000in}}{\pgfqpoint{2.407767in}{1.544118in}}%
\pgfusepath{clip}%
\pgfsetbuttcap%
\pgfsetroundjoin%
\pgfsetlinewidth{0.501875pt}%
\definecolor{currentstroke}{rgb}{0.268510,0.009605,0.335427}%
\pgfsetstrokecolor{currentstroke}%
\pgfsetdash{}{0pt}%
\pgfpathmoveto{\pgfqpoint{4.509160in}{1.558562in}}%
\pgfpathlineto{\pgfqpoint{4.509160in}{1.558562in}}%
\pgfusepath{stroke}%
\end{pgfscope}%
\begin{pgfscope}%
\pgfpathrectangle{\pgfqpoint{3.352233in}{1.400000in}}{\pgfqpoint{2.407767in}{1.544118in}}%
\pgfusepath{clip}%
\pgfsetbuttcap%
\pgfsetroundjoin%
\pgfsetlinewidth{0.501875pt}%
\definecolor{currentstroke}{rgb}{0.268510,0.009605,0.335427}%
\pgfsetstrokecolor{currentstroke}%
\pgfsetdash{}{0pt}%
\pgfpathmoveto{\pgfqpoint{4.509160in}{1.558562in}}%
\pgfpathlineto{\pgfqpoint{4.509160in}{1.558562in}}%
\pgfusepath{stroke}%
\end{pgfscope}%
\begin{pgfscope}%
\pgfpathrectangle{\pgfqpoint{3.352233in}{1.400000in}}{\pgfqpoint{2.407767in}{1.544118in}}%
\pgfusepath{clip}%
\pgfsetbuttcap%
\pgfsetroundjoin%
\pgfsetlinewidth{0.501875pt}%
\definecolor{currentstroke}{rgb}{0.268510,0.009605,0.335427}%
\pgfsetstrokecolor{currentstroke}%
\pgfsetdash{}{0pt}%
\pgfpathmoveto{\pgfqpoint{4.509160in}{1.558562in}}%
\pgfpathlineto{\pgfqpoint{4.509160in}{1.558562in}}%
\pgfusepath{stroke}%
\end{pgfscope}%
\begin{pgfscope}%
\pgfpathrectangle{\pgfqpoint{3.352233in}{1.400000in}}{\pgfqpoint{2.407767in}{1.544118in}}%
\pgfusepath{clip}%
\pgfsetbuttcap%
\pgfsetroundjoin%
\pgfsetlinewidth{0.501875pt}%
\definecolor{currentstroke}{rgb}{0.268510,0.009605,0.335427}%
\pgfsetstrokecolor{currentstroke}%
\pgfsetdash{}{0pt}%
\pgfpathmoveto{\pgfqpoint{4.509160in}{1.558562in}}%
\pgfpathlineto{\pgfqpoint{4.509160in}{1.558562in}}%
\pgfusepath{stroke}%
\end{pgfscope}%
\begin{pgfscope}%
\pgfpathrectangle{\pgfqpoint{3.352233in}{1.400000in}}{\pgfqpoint{2.407767in}{1.544118in}}%
\pgfusepath{clip}%
\pgfsetbuttcap%
\pgfsetroundjoin%
\pgfsetlinewidth{0.501875pt}%
\definecolor{currentstroke}{rgb}{0.268510,0.009605,0.335427}%
\pgfsetstrokecolor{currentstroke}%
\pgfsetdash{}{0pt}%
\pgfpathmoveto{\pgfqpoint{4.509160in}{1.558562in}}%
\pgfpathlineto{\pgfqpoint{4.509160in}{1.558562in}}%
\pgfusepath{stroke}%
\end{pgfscope}%
\begin{pgfscope}%
\pgfpathrectangle{\pgfqpoint{3.352233in}{1.400000in}}{\pgfqpoint{2.407767in}{1.544118in}}%
\pgfusepath{clip}%
\pgfsetbuttcap%
\pgfsetroundjoin%
\pgfsetlinewidth{0.501875pt}%
\definecolor{currentstroke}{rgb}{0.268510,0.009605,0.335427}%
\pgfsetstrokecolor{currentstroke}%
\pgfsetdash{}{0pt}%
\pgfpathmoveto{\pgfqpoint{4.509160in}{1.558562in}}%
\pgfpathlineto{\pgfqpoint{4.509160in}{1.558562in}}%
\pgfusepath{stroke}%
\end{pgfscope}%
\begin{pgfscope}%
\pgfpathrectangle{\pgfqpoint{3.352233in}{1.400000in}}{\pgfqpoint{2.407767in}{1.544118in}}%
\pgfusepath{clip}%
\pgfsetbuttcap%
\pgfsetroundjoin%
\pgfsetlinewidth{0.501875pt}%
\definecolor{currentstroke}{rgb}{0.268510,0.009605,0.335427}%
\pgfsetstrokecolor{currentstroke}%
\pgfsetdash{}{0pt}%
\pgfpathmoveto{\pgfqpoint{4.509160in}{1.558562in}}%
\pgfpathlineto{\pgfqpoint{4.509160in}{1.558562in}}%
\pgfusepath{stroke}%
\end{pgfscope}%
\begin{pgfscope}%
\pgfpathrectangle{\pgfqpoint{3.352233in}{1.400000in}}{\pgfqpoint{2.407767in}{1.544118in}}%
\pgfusepath{clip}%
\pgfsetbuttcap%
\pgfsetroundjoin%
\pgfsetlinewidth{0.501875pt}%
\definecolor{currentstroke}{rgb}{0.268510,0.009605,0.335427}%
\pgfsetstrokecolor{currentstroke}%
\pgfsetdash{}{0pt}%
\pgfpathmoveto{\pgfqpoint{4.509160in}{1.558562in}}%
\pgfpathlineto{\pgfqpoint{4.509160in}{1.558562in}}%
\pgfusepath{stroke}%
\end{pgfscope}%
\begin{pgfscope}%
\pgfpathrectangle{\pgfqpoint{3.352233in}{1.400000in}}{\pgfqpoint{2.407767in}{1.544118in}}%
\pgfusepath{clip}%
\pgfsetbuttcap%
\pgfsetroundjoin%
\pgfsetlinewidth{0.501875pt}%
\definecolor{currentstroke}{rgb}{0.268510,0.009605,0.335427}%
\pgfsetstrokecolor{currentstroke}%
\pgfsetdash{}{0pt}%
\pgfpathmoveto{\pgfqpoint{4.509160in}{1.558562in}}%
\pgfpathlineto{\pgfqpoint{4.509160in}{1.558562in}}%
\pgfusepath{stroke}%
\end{pgfscope}%
\begin{pgfscope}%
\pgfpathrectangle{\pgfqpoint{3.352233in}{1.400000in}}{\pgfqpoint{2.407767in}{1.544118in}}%
\pgfusepath{clip}%
\pgfsetbuttcap%
\pgfsetroundjoin%
\pgfsetlinewidth{0.501875pt}%
\definecolor{currentstroke}{rgb}{0.268510,0.009605,0.335427}%
\pgfsetstrokecolor{currentstroke}%
\pgfsetdash{}{0pt}%
\pgfpathmoveto{\pgfqpoint{4.509160in}{1.558562in}}%
\pgfpathlineto{\pgfqpoint{4.509160in}{1.558562in}}%
\pgfusepath{stroke}%
\end{pgfscope}%
\begin{pgfscope}%
\pgfpathrectangle{\pgfqpoint{3.352233in}{1.400000in}}{\pgfqpoint{2.407767in}{1.544118in}}%
\pgfusepath{clip}%
\pgfsetbuttcap%
\pgfsetroundjoin%
\pgfsetlinewidth{0.501875pt}%
\definecolor{currentstroke}{rgb}{0.268510,0.009605,0.335427}%
\pgfsetstrokecolor{currentstroke}%
\pgfsetdash{}{0pt}%
\pgfpathmoveto{\pgfqpoint{4.509160in}{1.558562in}}%
\pgfpathlineto{\pgfqpoint{4.509160in}{1.558562in}}%
\pgfusepath{stroke}%
\end{pgfscope}%
\begin{pgfscope}%
\pgfpathrectangle{\pgfqpoint{3.352233in}{1.400000in}}{\pgfqpoint{2.407767in}{1.544118in}}%
\pgfusepath{clip}%
\pgfsetbuttcap%
\pgfsetroundjoin%
\pgfsetlinewidth{0.501875pt}%
\definecolor{currentstroke}{rgb}{0.268510,0.009605,0.335427}%
\pgfsetstrokecolor{currentstroke}%
\pgfsetdash{}{0pt}%
\pgfpathmoveto{\pgfqpoint{4.509160in}{1.558562in}}%
\pgfpathlineto{\pgfqpoint{4.509160in}{1.558562in}}%
\pgfusepath{stroke}%
\end{pgfscope}%
\begin{pgfscope}%
\pgfpathrectangle{\pgfqpoint{3.352233in}{1.400000in}}{\pgfqpoint{2.407767in}{1.544118in}}%
\pgfusepath{clip}%
\pgfsetbuttcap%
\pgfsetroundjoin%
\pgfsetlinewidth{0.501875pt}%
\definecolor{currentstroke}{rgb}{0.268510,0.009605,0.335427}%
\pgfsetstrokecolor{currentstroke}%
\pgfsetdash{}{0pt}%
\pgfpathmoveto{\pgfqpoint{4.509160in}{1.558562in}}%
\pgfpathlineto{\pgfqpoint{4.509160in}{1.558562in}}%
\pgfusepath{stroke}%
\end{pgfscope}%
\begin{pgfscope}%
\pgfpathrectangle{\pgfqpoint{3.352233in}{1.400000in}}{\pgfqpoint{2.407767in}{1.544118in}}%
\pgfusepath{clip}%
\pgfsetbuttcap%
\pgfsetroundjoin%
\pgfsetlinewidth{0.501875pt}%
\definecolor{currentstroke}{rgb}{0.268510,0.009605,0.335427}%
\pgfsetstrokecolor{currentstroke}%
\pgfsetdash{}{0pt}%
\pgfpathmoveto{\pgfqpoint{4.509160in}{1.558562in}}%
\pgfpathlineto{\pgfqpoint{4.509160in}{1.558562in}}%
\pgfusepath{stroke}%
\end{pgfscope}%
\begin{pgfscope}%
\pgfpathrectangle{\pgfqpoint{3.352233in}{1.400000in}}{\pgfqpoint{2.407767in}{1.544118in}}%
\pgfusepath{clip}%
\pgfsetbuttcap%
\pgfsetroundjoin%
\pgfsetlinewidth{0.501875pt}%
\definecolor{currentstroke}{rgb}{0.268510,0.009605,0.335427}%
\pgfsetstrokecolor{currentstroke}%
\pgfsetdash{}{0pt}%
\pgfpathmoveto{\pgfqpoint{4.509160in}{1.558562in}}%
\pgfpathlineto{\pgfqpoint{4.509160in}{1.558562in}}%
\pgfusepath{stroke}%
\end{pgfscope}%
\begin{pgfscope}%
\pgfpathrectangle{\pgfqpoint{3.352233in}{1.400000in}}{\pgfqpoint{2.407767in}{1.544118in}}%
\pgfusepath{clip}%
\pgfsetbuttcap%
\pgfsetroundjoin%
\pgfsetlinewidth{0.501875pt}%
\definecolor{currentstroke}{rgb}{0.268510,0.009605,0.335427}%
\pgfsetstrokecolor{currentstroke}%
\pgfsetdash{}{0pt}%
\pgfpathmoveto{\pgfqpoint{4.509160in}{1.558562in}}%
\pgfpathlineto{\pgfqpoint{4.509160in}{1.558562in}}%
\pgfusepath{stroke}%
\end{pgfscope}%
\begin{pgfscope}%
\pgfpathrectangle{\pgfqpoint{3.352233in}{1.400000in}}{\pgfqpoint{2.407767in}{1.544118in}}%
\pgfusepath{clip}%
\pgfsetbuttcap%
\pgfsetroundjoin%
\pgfsetlinewidth{0.501875pt}%
\definecolor{currentstroke}{rgb}{0.271305,0.019942,0.347269}%
\pgfsetstrokecolor{currentstroke}%
\pgfsetdash{}{0pt}%
\pgfpathmoveto{\pgfqpoint{5.141296in}{1.530192in}}%
\pgfpathlineto{\pgfqpoint{5.088360in}{1.531363in}}%
\pgfusepath{stroke}%
\end{pgfscope}%
\begin{pgfscope}%
\pgfpathrectangle{\pgfqpoint{3.352233in}{1.400000in}}{\pgfqpoint{2.407767in}{1.544118in}}%
\pgfusepath{clip}%
\pgfsetbuttcap%
\pgfsetroundjoin%
\pgfsetlinewidth{0.501875pt}%
\definecolor{currentstroke}{rgb}{0.272594,0.025563,0.353093}%
\pgfsetstrokecolor{currentstroke}%
\pgfsetdash{}{0pt}%
\pgfpathmoveto{\pgfqpoint{5.088360in}{1.531363in}}%
\pgfpathlineto{\pgfqpoint{5.035452in}{1.532846in}}%
\pgfusepath{stroke}%
\end{pgfscope}%
\begin{pgfscope}%
\pgfpathrectangle{\pgfqpoint{3.352233in}{1.400000in}}{\pgfqpoint{2.407767in}{1.544118in}}%
\pgfusepath{clip}%
\pgfsetbuttcap%
\pgfsetroundjoin%
\pgfsetlinewidth{0.501875pt}%
\definecolor{currentstroke}{rgb}{0.273809,0.031497,0.358853}%
\pgfsetstrokecolor{currentstroke}%
\pgfsetdash{}{0pt}%
\pgfpathmoveto{\pgfqpoint{5.035452in}{1.532846in}}%
\pgfpathlineto{\pgfqpoint{4.982586in}{1.534848in}}%
\pgfusepath{stroke}%
\end{pgfscope}%
\begin{pgfscope}%
\pgfpathrectangle{\pgfqpoint{3.352233in}{1.400000in}}{\pgfqpoint{2.407767in}{1.544118in}}%
\pgfusepath{clip}%
\pgfsetbuttcap%
\pgfsetroundjoin%
\pgfsetlinewidth{0.501875pt}%
\definecolor{currentstroke}{rgb}{0.273809,0.031497,0.358853}%
\pgfsetstrokecolor{currentstroke}%
\pgfsetdash{}{0pt}%
\pgfpathmoveto{\pgfqpoint{4.982586in}{1.534848in}}%
\pgfpathlineto{\pgfqpoint{4.929719in}{1.536968in}}%
\pgfusepath{stroke}%
\end{pgfscope}%
\begin{pgfscope}%
\pgfpathrectangle{\pgfqpoint{3.352233in}{1.400000in}}{\pgfqpoint{2.407767in}{1.544118in}}%
\pgfusepath{clip}%
\pgfsetbuttcap%
\pgfsetroundjoin%
\pgfsetlinewidth{0.501875pt}%
\definecolor{currentstroke}{rgb}{0.273809,0.031497,0.358853}%
\pgfsetstrokecolor{currentstroke}%
\pgfsetdash{}{0pt}%
\pgfpathmoveto{\pgfqpoint{4.929719in}{1.536968in}}%
\pgfpathlineto{\pgfqpoint{4.876830in}{1.538809in}}%
\pgfusepath{stroke}%
\end{pgfscope}%
\begin{pgfscope}%
\pgfpathrectangle{\pgfqpoint{3.352233in}{1.400000in}}{\pgfqpoint{2.407767in}{1.544118in}}%
\pgfusepath{clip}%
\pgfsetbuttcap%
\pgfsetroundjoin%
\pgfsetlinewidth{0.501875pt}%
\definecolor{currentstroke}{rgb}{0.274952,0.037752,0.364543}%
\pgfsetstrokecolor{currentstroke}%
\pgfsetdash{}{0pt}%
\pgfpathmoveto{\pgfqpoint{4.876830in}{1.538809in}}%
\pgfpathlineto{\pgfqpoint{4.824063in}{1.541229in}}%
\pgfusepath{stroke}%
\end{pgfscope}%
\begin{pgfscope}%
\pgfpathrectangle{\pgfqpoint{3.352233in}{1.400000in}}{\pgfqpoint{2.407767in}{1.544118in}}%
\pgfusepath{clip}%
\pgfsetbuttcap%
\pgfsetroundjoin%
\pgfsetlinewidth{0.501875pt}%
\definecolor{currentstroke}{rgb}{0.272594,0.025563,0.353093}%
\pgfsetstrokecolor{currentstroke}%
\pgfsetdash{}{0pt}%
\pgfpathmoveto{\pgfqpoint{4.824063in}{1.541229in}}%
\pgfpathlineto{\pgfqpoint{4.771266in}{1.543243in}}%
\pgfusepath{stroke}%
\end{pgfscope}%
\begin{pgfscope}%
\pgfpathrectangle{\pgfqpoint{3.352233in}{1.400000in}}{\pgfqpoint{2.407767in}{1.544118in}}%
\pgfusepath{clip}%
\pgfsetbuttcap%
\pgfsetroundjoin%
\pgfsetlinewidth{0.501875pt}%
\definecolor{currentstroke}{rgb}{0.271305,0.019942,0.347269}%
\pgfsetstrokecolor{currentstroke}%
\pgfsetdash{}{0pt}%
\pgfpathmoveto{\pgfqpoint{4.771266in}{1.543243in}}%
\pgfpathlineto{\pgfqpoint{4.718657in}{1.546629in}}%
\pgfusepath{stroke}%
\end{pgfscope}%
\begin{pgfscope}%
\pgfpathrectangle{\pgfqpoint{3.352233in}{1.400000in}}{\pgfqpoint{2.407767in}{1.544118in}}%
\pgfusepath{clip}%
\pgfsetbuttcap%
\pgfsetroundjoin%
\pgfsetlinewidth{0.501875pt}%
\definecolor{currentstroke}{rgb}{0.268510,0.009605,0.335427}%
\pgfsetstrokecolor{currentstroke}%
\pgfsetdash{}{0pt}%
\pgfpathmoveto{\pgfqpoint{5.206279in}{2.797489in}}%
\pgfpathlineto{\pgfqpoint{5.153309in}{2.797042in}}%
\pgfusepath{stroke}%
\end{pgfscope}%
\begin{pgfscope}%
\pgfpathrectangle{\pgfqpoint{3.352233in}{1.400000in}}{\pgfqpoint{2.407767in}{1.544118in}}%
\pgfusepath{clip}%
\pgfsetbuttcap%
\pgfsetroundjoin%
\pgfsetlinewidth{0.501875pt}%
\definecolor{currentstroke}{rgb}{0.268510,0.009605,0.335427}%
\pgfsetstrokecolor{currentstroke}%
\pgfsetdash{}{0pt}%
\pgfpathmoveto{\pgfqpoint{5.153309in}{2.797042in}}%
\pgfpathlineto{\pgfqpoint{5.100354in}{2.796711in}}%
\pgfusepath{stroke}%
\end{pgfscope}%
\begin{pgfscope}%
\pgfpathrectangle{\pgfqpoint{3.352233in}{1.400000in}}{\pgfqpoint{2.407767in}{1.544118in}}%
\pgfusepath{clip}%
\pgfsetbuttcap%
\pgfsetroundjoin%
\pgfsetlinewidth{0.501875pt}%
\definecolor{currentstroke}{rgb}{0.271305,0.019942,0.347269}%
\pgfsetstrokecolor{currentstroke}%
\pgfsetdash{}{0pt}%
\pgfpathmoveto{\pgfqpoint{5.100354in}{2.796711in}}%
\pgfpathlineto{\pgfqpoint{5.047418in}{2.796145in}}%
\pgfusepath{stroke}%
\end{pgfscope}%
\begin{pgfscope}%
\pgfpathrectangle{\pgfqpoint{3.352233in}{1.400000in}}{\pgfqpoint{2.407767in}{1.544118in}}%
\pgfusepath{clip}%
\pgfsetbuttcap%
\pgfsetroundjoin%
\pgfsetlinewidth{0.501875pt}%
\definecolor{currentstroke}{rgb}{0.271305,0.019942,0.347269}%
\pgfsetstrokecolor{currentstroke}%
\pgfsetdash{}{0pt}%
\pgfpathmoveto{\pgfqpoint{5.047418in}{2.796145in}}%
\pgfpathlineto{\pgfqpoint{4.994567in}{2.794858in}}%
\pgfusepath{stroke}%
\end{pgfscope}%
\begin{pgfscope}%
\pgfpathrectangle{\pgfqpoint{3.352233in}{1.400000in}}{\pgfqpoint{2.407767in}{1.544118in}}%
\pgfusepath{clip}%
\pgfsetbuttcap%
\pgfsetroundjoin%
\pgfsetlinewidth{0.501875pt}%
\definecolor{currentstroke}{rgb}{0.272594,0.025563,0.353093}%
\pgfsetstrokecolor{currentstroke}%
\pgfsetdash{}{0pt}%
\pgfpathmoveto{\pgfqpoint{4.994567in}{2.794858in}}%
\pgfpathlineto{\pgfqpoint{4.941739in}{2.793482in}}%
\pgfusepath{stroke}%
\end{pgfscope}%
\begin{pgfscope}%
\pgfpathrectangle{\pgfqpoint{3.352233in}{1.400000in}}{\pgfqpoint{2.407767in}{1.544118in}}%
\pgfusepath{clip}%
\pgfsetbuttcap%
\pgfsetroundjoin%
\pgfsetlinewidth{0.501875pt}%
\definecolor{currentstroke}{rgb}{0.273809,0.031497,0.358853}%
\pgfsetstrokecolor{currentstroke}%
\pgfsetdash{}{0pt}%
\pgfpathmoveto{\pgfqpoint{4.941739in}{2.793482in}}%
\pgfpathlineto{\pgfqpoint{4.888927in}{2.791343in}}%
\pgfusepath{stroke}%
\end{pgfscope}%
\begin{pgfscope}%
\pgfpathrectangle{\pgfqpoint{3.352233in}{1.400000in}}{\pgfqpoint{2.407767in}{1.544118in}}%
\pgfusepath{clip}%
\pgfsetbuttcap%
\pgfsetroundjoin%
\pgfsetlinewidth{0.501875pt}%
\definecolor{currentstroke}{rgb}{0.273809,0.031497,0.358853}%
\pgfsetstrokecolor{currentstroke}%
\pgfsetdash{}{0pt}%
\pgfpathmoveto{\pgfqpoint{4.888927in}{2.791343in}}%
\pgfpathlineto{\pgfqpoint{4.836129in}{2.789291in}}%
\pgfusepath{stroke}%
\end{pgfscope}%
\begin{pgfscope}%
\pgfpathrectangle{\pgfqpoint{3.352233in}{1.400000in}}{\pgfqpoint{2.407767in}{1.544118in}}%
\pgfusepath{clip}%
\pgfsetbuttcap%
\pgfsetroundjoin%
\pgfsetlinewidth{0.501875pt}%
\definecolor{currentstroke}{rgb}{0.273809,0.031497,0.358853}%
\pgfsetstrokecolor{currentstroke}%
\pgfsetdash{}{0pt}%
\pgfpathmoveto{\pgfqpoint{4.823356in}{1.573977in}}%
\pgfpathlineto{\pgfqpoint{4.770570in}{1.576756in}}%
\pgfusepath{stroke}%
\end{pgfscope}%
\begin{pgfscope}%
\pgfpathrectangle{\pgfqpoint{3.352233in}{1.400000in}}{\pgfqpoint{2.407767in}{1.544118in}}%
\pgfusepath{clip}%
\pgfsetbuttcap%
\pgfsetroundjoin%
\pgfsetlinewidth{0.501875pt}%
\definecolor{currentstroke}{rgb}{0.273809,0.031497,0.358853}%
\pgfsetstrokecolor{currentstroke}%
\pgfsetdash{}{0pt}%
\pgfpathmoveto{\pgfqpoint{4.770570in}{1.576756in}}%
\pgfpathlineto{\pgfqpoint{4.718657in}{1.581375in}}%
\pgfusepath{stroke}%
\end{pgfscope}%
\begin{pgfscope}%
\pgfpathrectangle{\pgfqpoint{3.352233in}{1.400000in}}{\pgfqpoint{2.407767in}{1.544118in}}%
\pgfusepath{clip}%
\pgfsetbuttcap%
\pgfsetroundjoin%
\pgfsetlinewidth{0.501875pt}%
\definecolor{currentstroke}{rgb}{0.272594,0.025563,0.353093}%
\pgfsetstrokecolor{currentstroke}%
\pgfsetdash{}{0pt}%
\pgfpathmoveto{\pgfqpoint{4.718657in}{1.581375in}}%
\pgfpathlineto{\pgfqpoint{4.669886in}{1.590786in}}%
\pgfusepath{stroke}%
\end{pgfscope}%
\begin{pgfscope}%
\pgfpathrectangle{\pgfqpoint{3.352233in}{1.400000in}}{\pgfqpoint{2.407767in}{1.544118in}}%
\pgfusepath{clip}%
\pgfsetbuttcap%
\pgfsetroundjoin%
\pgfsetlinewidth{0.501875pt}%
\definecolor{currentstroke}{rgb}{0.271305,0.019942,0.347269}%
\pgfsetstrokecolor{currentstroke}%
\pgfsetdash{}{0pt}%
\pgfpathmoveto{\pgfqpoint{4.669886in}{1.590786in}}%
\pgfpathlineto{\pgfqpoint{4.669886in}{1.590786in}}%
\pgfusepath{stroke}%
\end{pgfscope}%
\begin{pgfscope}%
\pgfpathrectangle{\pgfqpoint{3.352233in}{1.400000in}}{\pgfqpoint{2.407767in}{1.544118in}}%
\pgfusepath{clip}%
\pgfsetbuttcap%
\pgfsetroundjoin%
\pgfsetlinewidth{0.501875pt}%
\definecolor{currentstroke}{rgb}{0.271305,0.019942,0.347269}%
\pgfsetstrokecolor{currentstroke}%
\pgfsetdash{}{0pt}%
\pgfpathmoveto{\pgfqpoint{4.669886in}{1.590786in}}%
\pgfpathlineto{\pgfqpoint{4.669886in}{1.590786in}}%
\pgfusepath{stroke}%
\end{pgfscope}%
\begin{pgfscope}%
\pgfpathrectangle{\pgfqpoint{3.352233in}{1.400000in}}{\pgfqpoint{2.407767in}{1.544118in}}%
\pgfusepath{clip}%
\pgfsetbuttcap%
\pgfsetroundjoin%
\pgfsetlinewidth{0.501875pt}%
\definecolor{currentstroke}{rgb}{0.271305,0.019942,0.347269}%
\pgfsetstrokecolor{currentstroke}%
\pgfsetdash{}{0pt}%
\pgfpathmoveto{\pgfqpoint{4.669886in}{1.590786in}}%
\pgfpathlineto{\pgfqpoint{4.644153in}{1.598706in}}%
\pgfusepath{stroke}%
\end{pgfscope}%
\begin{pgfscope}%
\pgfpathrectangle{\pgfqpoint{3.352233in}{1.400000in}}{\pgfqpoint{2.407767in}{1.544118in}}%
\pgfusepath{clip}%
\pgfsetbuttcap%
\pgfsetroundjoin%
\pgfsetlinewidth{0.501875pt}%
\definecolor{currentstroke}{rgb}{0.272594,0.025563,0.353093}%
\pgfsetstrokecolor{currentstroke}%
\pgfsetdash{}{0pt}%
\pgfpathmoveto{\pgfqpoint{4.644153in}{1.598706in}}%
\pgfpathlineto{\pgfqpoint{4.618647in}{1.603341in}}%
\pgfusepath{stroke}%
\end{pgfscope}%
\begin{pgfscope}%
\pgfpathrectangle{\pgfqpoint{3.352233in}{1.400000in}}{\pgfqpoint{2.407767in}{1.544118in}}%
\pgfusepath{clip}%
\pgfsetbuttcap%
\pgfsetroundjoin%
\pgfsetlinewidth{0.501875pt}%
\definecolor{currentstroke}{rgb}{0.272594,0.025563,0.353093}%
\pgfsetstrokecolor{currentstroke}%
\pgfsetdash{}{0pt}%
\pgfpathmoveto{\pgfqpoint{4.618647in}{1.603341in}}%
\pgfpathlineto{\pgfqpoint{4.568857in}{1.614494in}}%
\pgfusepath{stroke}%
\end{pgfscope}%
\begin{pgfscope}%
\pgfpathrectangle{\pgfqpoint{3.352233in}{1.400000in}}{\pgfqpoint{2.407767in}{1.544118in}}%
\pgfusepath{clip}%
\pgfsetbuttcap%
\pgfsetroundjoin%
\pgfsetlinewidth{0.501875pt}%
\definecolor{currentstroke}{rgb}{0.272594,0.025563,0.353093}%
\pgfsetstrokecolor{currentstroke}%
\pgfsetdash{}{0pt}%
\pgfpathmoveto{\pgfqpoint{4.568857in}{1.614494in}}%
\pgfpathlineto{\pgfqpoint{4.520321in}{1.627526in}}%
\pgfusepath{stroke}%
\end{pgfscope}%
\begin{pgfscope}%
\pgfpathrectangle{\pgfqpoint{3.352233in}{1.400000in}}{\pgfqpoint{2.407767in}{1.544118in}}%
\pgfusepath{clip}%
\pgfsetbuttcap%
\pgfsetroundjoin%
\pgfsetlinewidth{0.501875pt}%
\definecolor{currentstroke}{rgb}{0.271305,0.019942,0.347269}%
\pgfsetstrokecolor{currentstroke}%
\pgfsetdash{}{0pt}%
\pgfpathmoveto{\pgfqpoint{4.520321in}{1.627526in}}%
\pgfpathlineto{\pgfqpoint{4.520321in}{1.627526in}}%
\pgfusepath{stroke}%
\end{pgfscope}%
\begin{pgfscope}%
\pgfpathrectangle{\pgfqpoint{3.352233in}{1.400000in}}{\pgfqpoint{2.407767in}{1.544118in}}%
\pgfusepath{clip}%
\pgfsetbuttcap%
\pgfsetroundjoin%
\pgfsetlinewidth{0.501875pt}%
\definecolor{currentstroke}{rgb}{0.271305,0.019942,0.347269}%
\pgfsetstrokecolor{currentstroke}%
\pgfsetdash{}{0pt}%
\pgfpathmoveto{\pgfqpoint{4.520321in}{1.627526in}}%
\pgfpathlineto{\pgfqpoint{4.503498in}{1.632784in}}%
\pgfusepath{stroke}%
\end{pgfscope}%
\begin{pgfscope}%
\pgfpathrectangle{\pgfqpoint{3.352233in}{1.400000in}}{\pgfqpoint{2.407767in}{1.544118in}}%
\pgfusepath{clip}%
\pgfsetbuttcap%
\pgfsetroundjoin%
\pgfsetlinewidth{0.501875pt}%
\definecolor{currentstroke}{rgb}{0.271305,0.019942,0.347269}%
\pgfsetstrokecolor{currentstroke}%
\pgfsetdash{}{0pt}%
\pgfpathmoveto{\pgfqpoint{4.503498in}{1.632784in}}%
\pgfpathlineto{\pgfqpoint{4.503498in}{1.632784in}}%
\pgfusepath{stroke}%
\end{pgfscope}%
\begin{pgfscope}%
\pgfpathrectangle{\pgfqpoint{3.352233in}{1.400000in}}{\pgfqpoint{2.407767in}{1.544118in}}%
\pgfusepath{clip}%
\pgfsetbuttcap%
\pgfsetroundjoin%
\pgfsetlinewidth{0.501875pt}%
\definecolor{currentstroke}{rgb}{0.269944,0.014625,0.341379}%
\pgfsetstrokecolor{currentstroke}%
\pgfsetdash{}{0pt}%
\pgfpathmoveto{\pgfqpoint{5.202639in}{1.580893in}}%
\pgfpathlineto{\pgfqpoint{5.149669in}{1.580875in}}%
\pgfusepath{stroke}%
\end{pgfscope}%
\begin{pgfscope}%
\pgfpathrectangle{\pgfqpoint{3.352233in}{1.400000in}}{\pgfqpoint{2.407767in}{1.544118in}}%
\pgfusepath{clip}%
\pgfsetbuttcap%
\pgfsetroundjoin%
\pgfsetlinewidth{0.501875pt}%
\definecolor{currentstroke}{rgb}{0.271305,0.019942,0.347269}%
\pgfsetstrokecolor{currentstroke}%
\pgfsetdash{}{0pt}%
\pgfpathmoveto{\pgfqpoint{5.149669in}{1.580875in}}%
\pgfpathlineto{\pgfqpoint{5.096703in}{1.581022in}}%
\pgfusepath{stroke}%
\end{pgfscope}%
\begin{pgfscope}%
\pgfpathrectangle{\pgfqpoint{3.352233in}{1.400000in}}{\pgfqpoint{2.407767in}{1.544118in}}%
\pgfusepath{clip}%
\pgfsetbuttcap%
\pgfsetroundjoin%
\pgfsetlinewidth{0.501875pt}%
\definecolor{currentstroke}{rgb}{0.271305,0.019942,0.347269}%
\pgfsetstrokecolor{currentstroke}%
\pgfsetdash{}{0pt}%
\pgfpathmoveto{\pgfqpoint{5.096703in}{1.581022in}}%
\pgfpathlineto{\pgfqpoint{5.043738in}{1.581375in}}%
\pgfusepath{stroke}%
\end{pgfscope}%
\begin{pgfscope}%
\pgfpathrectangle{\pgfqpoint{3.352233in}{1.400000in}}{\pgfqpoint{2.407767in}{1.544118in}}%
\pgfusepath{clip}%
\pgfsetbuttcap%
\pgfsetroundjoin%
\pgfsetlinewidth{0.501875pt}%
\definecolor{currentstroke}{rgb}{0.272594,0.025563,0.353093}%
\pgfsetstrokecolor{currentstroke}%
\pgfsetdash{}{0pt}%
\pgfpathmoveto{\pgfqpoint{5.043738in}{1.581375in}}%
\pgfpathlineto{\pgfqpoint{4.990784in}{1.581970in}}%
\pgfusepath{stroke}%
\end{pgfscope}%
\begin{pgfscope}%
\pgfpathrectangle{\pgfqpoint{3.352233in}{1.400000in}}{\pgfqpoint{2.407767in}{1.544118in}}%
\pgfusepath{clip}%
\pgfsetbuttcap%
\pgfsetroundjoin%
\pgfsetlinewidth{0.501875pt}%
\definecolor{currentstroke}{rgb}{0.274952,0.037752,0.364543}%
\pgfsetstrokecolor{currentstroke}%
\pgfsetdash{}{0pt}%
\pgfpathmoveto{\pgfqpoint{4.990784in}{1.581970in}}%
\pgfpathlineto{\pgfqpoint{4.937844in}{1.583105in}}%
\pgfusepath{stroke}%
\end{pgfscope}%
\begin{pgfscope}%
\pgfpathrectangle{\pgfqpoint{3.352233in}{1.400000in}}{\pgfqpoint{2.407767in}{1.544118in}}%
\pgfusepath{clip}%
\pgfsetbuttcap%
\pgfsetroundjoin%
\pgfsetlinewidth{0.501875pt}%
\definecolor{currentstroke}{rgb}{0.269944,0.014625,0.341379}%
\pgfsetstrokecolor{currentstroke}%
\pgfsetdash{}{0pt}%
\pgfpathmoveto{\pgfqpoint{5.206279in}{2.762743in}}%
\pgfpathlineto{\pgfqpoint{5.153398in}{2.761474in}}%
\pgfusepath{stroke}%
\end{pgfscope}%
\begin{pgfscope}%
\pgfpathrectangle{\pgfqpoint{3.352233in}{1.400000in}}{\pgfqpoint{2.407767in}{1.544118in}}%
\pgfusepath{clip}%
\pgfsetbuttcap%
\pgfsetroundjoin%
\pgfsetlinewidth{0.501875pt}%
\definecolor{currentstroke}{rgb}{0.269944,0.014625,0.341379}%
\pgfsetstrokecolor{currentstroke}%
\pgfsetdash{}{0pt}%
\pgfpathmoveto{\pgfqpoint{5.153398in}{2.761474in}}%
\pgfpathlineto{\pgfqpoint{5.100518in}{2.760086in}}%
\pgfusepath{stroke}%
\end{pgfscope}%
\begin{pgfscope}%
\pgfpathrectangle{\pgfqpoint{3.352233in}{1.400000in}}{\pgfqpoint{2.407767in}{1.544118in}}%
\pgfusepath{clip}%
\pgfsetbuttcap%
\pgfsetroundjoin%
\pgfsetlinewidth{0.501875pt}%
\definecolor{currentstroke}{rgb}{0.272594,0.025563,0.353093}%
\pgfsetstrokecolor{currentstroke}%
\pgfsetdash{}{0pt}%
\pgfpathmoveto{\pgfqpoint{5.100518in}{2.760086in}}%
\pgfpathlineto{\pgfqpoint{5.047563in}{2.759919in}}%
\pgfusepath{stroke}%
\end{pgfscope}%
\begin{pgfscope}%
\pgfpathrectangle{\pgfqpoint{3.352233in}{1.400000in}}{\pgfqpoint{2.407767in}{1.544118in}}%
\pgfusepath{clip}%
\pgfsetbuttcap%
\pgfsetroundjoin%
\pgfsetlinewidth{0.501875pt}%
\definecolor{currentstroke}{rgb}{0.271305,0.019942,0.347269}%
\pgfsetstrokecolor{currentstroke}%
\pgfsetdash{}{0pt}%
\pgfpathmoveto{\pgfqpoint{5.047563in}{2.759919in}}%
\pgfpathlineto{\pgfqpoint{4.994621in}{2.759120in}}%
\pgfusepath{stroke}%
\end{pgfscope}%
\begin{pgfscope}%
\pgfpathrectangle{\pgfqpoint{3.352233in}{1.400000in}}{\pgfqpoint{2.407767in}{1.544118in}}%
\pgfusepath{clip}%
\pgfsetbuttcap%
\pgfsetroundjoin%
\pgfsetlinewidth{0.501875pt}%
\definecolor{currentstroke}{rgb}{0.273809,0.031497,0.358853}%
\pgfsetstrokecolor{currentstroke}%
\pgfsetdash{}{0pt}%
\pgfpathmoveto{\pgfqpoint{4.994621in}{2.759120in}}%
\pgfpathlineto{\pgfqpoint{4.941706in}{2.757692in}}%
\pgfusepath{stroke}%
\end{pgfscope}%
\begin{pgfscope}%
\pgfpathrectangle{\pgfqpoint{3.352233in}{1.400000in}}{\pgfqpoint{2.407767in}{1.544118in}}%
\pgfusepath{clip}%
\pgfsetbuttcap%
\pgfsetroundjoin%
\pgfsetlinewidth{0.501875pt}%
\definecolor{currentstroke}{rgb}{0.274952,0.037752,0.364543}%
\pgfsetstrokecolor{currentstroke}%
\pgfsetdash{}{0pt}%
\pgfpathmoveto{\pgfqpoint{4.941706in}{2.757692in}}%
\pgfpathlineto{\pgfqpoint{4.888892in}{2.755452in}}%
\pgfusepath{stroke}%
\end{pgfscope}%
\begin{pgfscope}%
\pgfpathrectangle{\pgfqpoint{3.352233in}{1.400000in}}{\pgfqpoint{2.407767in}{1.544118in}}%
\pgfusepath{clip}%
\pgfsetbuttcap%
\pgfsetroundjoin%
\pgfsetlinewidth{0.501875pt}%
\definecolor{currentstroke}{rgb}{0.268510,0.009605,0.335427}%
\pgfsetstrokecolor{currentstroke}%
\pgfsetdash{}{0pt}%
\pgfpathmoveto{\pgfqpoint{5.193942in}{1.600819in}}%
\pgfpathlineto{\pgfqpoint{5.140996in}{1.601031in}}%
\pgfusepath{stroke}%
\end{pgfscope}%
\begin{pgfscope}%
\pgfpathrectangle{\pgfqpoint{3.352233in}{1.400000in}}{\pgfqpoint{2.407767in}{1.544118in}}%
\pgfusepath{clip}%
\pgfsetbuttcap%
\pgfsetroundjoin%
\pgfsetlinewidth{0.501875pt}%
\definecolor{currentstroke}{rgb}{0.271305,0.019942,0.347269}%
\pgfsetstrokecolor{currentstroke}%
\pgfsetdash{}{0pt}%
\pgfpathmoveto{\pgfqpoint{5.140996in}{1.601031in}}%
\pgfpathlineto{\pgfqpoint{5.088066in}{1.601609in}}%
\pgfusepath{stroke}%
\end{pgfscope}%
\begin{pgfscope}%
\pgfpathrectangle{\pgfqpoint{3.352233in}{1.400000in}}{\pgfqpoint{2.407767in}{1.544118in}}%
\pgfusepath{clip}%
\pgfsetbuttcap%
\pgfsetroundjoin%
\pgfsetlinewidth{0.501875pt}%
\definecolor{currentstroke}{rgb}{0.271305,0.019942,0.347269}%
\pgfsetstrokecolor{currentstroke}%
\pgfsetdash{}{0pt}%
\pgfpathmoveto{\pgfqpoint{5.088066in}{1.601609in}}%
\pgfpathlineto{\pgfqpoint{5.035129in}{1.601639in}}%
\pgfusepath{stroke}%
\end{pgfscope}%
\begin{pgfscope}%
\pgfpathrectangle{\pgfqpoint{3.352233in}{1.400000in}}{\pgfqpoint{2.407767in}{1.544118in}}%
\pgfusepath{clip}%
\pgfsetbuttcap%
\pgfsetroundjoin%
\pgfsetlinewidth{0.501875pt}%
\definecolor{currentstroke}{rgb}{0.273809,0.031497,0.358853}%
\pgfsetstrokecolor{currentstroke}%
\pgfsetdash{}{0pt}%
\pgfpathmoveto{\pgfqpoint{5.035129in}{1.601639in}}%
\pgfpathlineto{\pgfqpoint{4.982175in}{1.601907in}}%
\pgfusepath{stroke}%
\end{pgfscope}%
\begin{pgfscope}%
\pgfpathrectangle{\pgfqpoint{3.352233in}{1.400000in}}{\pgfqpoint{2.407767in}{1.544118in}}%
\pgfusepath{clip}%
\pgfsetbuttcap%
\pgfsetroundjoin%
\pgfsetlinewidth{0.501875pt}%
\definecolor{currentstroke}{rgb}{0.274952,0.037752,0.364543}%
\pgfsetstrokecolor{currentstroke}%
\pgfsetdash{}{0pt}%
\pgfpathmoveto{\pgfqpoint{4.982175in}{1.601907in}}%
\pgfpathlineto{\pgfqpoint{4.929238in}{1.602877in}}%
\pgfusepath{stroke}%
\end{pgfscope}%
\begin{pgfscope}%
\pgfpathrectangle{\pgfqpoint{3.352233in}{1.400000in}}{\pgfqpoint{2.407767in}{1.544118in}}%
\pgfusepath{clip}%
\pgfsetbuttcap%
\pgfsetroundjoin%
\pgfsetlinewidth{0.501875pt}%
\definecolor{currentstroke}{rgb}{0.276022,0.044167,0.370164}%
\pgfsetstrokecolor{currentstroke}%
\pgfsetdash{}{0pt}%
\pgfpathmoveto{\pgfqpoint{4.929238in}{1.602877in}}%
\pgfpathlineto{\pgfqpoint{4.876374in}{1.604836in}}%
\pgfusepath{stroke}%
\end{pgfscope}%
\begin{pgfscope}%
\pgfpathrectangle{\pgfqpoint{3.352233in}{1.400000in}}{\pgfqpoint{2.407767in}{1.544118in}}%
\pgfusepath{clip}%
\pgfsetbuttcap%
\pgfsetroundjoin%
\pgfsetlinewidth{0.501875pt}%
\definecolor{currentstroke}{rgb}{0.274952,0.037752,0.364543}%
\pgfsetstrokecolor{currentstroke}%
\pgfsetdash{}{0pt}%
\pgfpathmoveto{\pgfqpoint{4.876374in}{1.604836in}}%
\pgfpathlineto{\pgfqpoint{4.823544in}{1.607242in}}%
\pgfusepath{stroke}%
\end{pgfscope}%
\begin{pgfscope}%
\pgfpathrectangle{\pgfqpoint{3.352233in}{1.400000in}}{\pgfqpoint{2.407767in}{1.544118in}}%
\pgfusepath{clip}%
\pgfsetbuttcap%
\pgfsetroundjoin%
\pgfsetlinewidth{0.501875pt}%
\definecolor{currentstroke}{rgb}{0.272594,0.025563,0.353093}%
\pgfsetstrokecolor{currentstroke}%
\pgfsetdash{}{0pt}%
\pgfpathmoveto{\pgfqpoint{4.823544in}{1.607242in}}%
\pgfpathlineto{\pgfqpoint{4.770983in}{1.610981in}}%
\pgfusepath{stroke}%
\end{pgfscope}%
\begin{pgfscope}%
\pgfpathrectangle{\pgfqpoint{3.352233in}{1.400000in}}{\pgfqpoint{2.407767in}{1.544118in}}%
\pgfusepath{clip}%
\pgfsetbuttcap%
\pgfsetroundjoin%
\pgfsetlinewidth{0.501875pt}%
\definecolor{currentstroke}{rgb}{0.272594,0.025563,0.353093}%
\pgfsetstrokecolor{currentstroke}%
\pgfsetdash{}{0pt}%
\pgfpathmoveto{\pgfqpoint{4.770983in}{1.610981in}}%
\pgfpathlineto{\pgfqpoint{4.718657in}{1.616121in}}%
\pgfusepath{stroke}%
\end{pgfscope}%
\begin{pgfscope}%
\pgfpathrectangle{\pgfqpoint{3.352233in}{1.400000in}}{\pgfqpoint{2.407767in}{1.544118in}}%
\pgfusepath{clip}%
\pgfsetbuttcap%
\pgfsetroundjoin%
\pgfsetlinewidth{0.501875pt}%
\definecolor{currentstroke}{rgb}{0.273809,0.031497,0.358853}%
\pgfsetstrokecolor{currentstroke}%
\pgfsetdash{}{0pt}%
\pgfpathmoveto{\pgfqpoint{4.718657in}{1.616121in}}%
\pgfpathlineto{\pgfqpoint{4.667237in}{1.623702in}}%
\pgfusepath{stroke}%
\end{pgfscope}%
\begin{pgfscope}%
\pgfpathrectangle{\pgfqpoint{3.352233in}{1.400000in}}{\pgfqpoint{2.407767in}{1.544118in}}%
\pgfusepath{clip}%
\pgfsetbuttcap%
\pgfsetroundjoin%
\pgfsetlinewidth{0.501875pt}%
\definecolor{currentstroke}{rgb}{0.271305,0.019942,0.347269}%
\pgfsetstrokecolor{currentstroke}%
\pgfsetdash{}{0pt}%
\pgfpathmoveto{\pgfqpoint{4.501936in}{2.727997in}}%
\pgfpathlineto{\pgfqpoint{4.501936in}{2.727997in}}%
\pgfusepath{stroke}%
\end{pgfscope}%
\begin{pgfscope}%
\pgfpathrectangle{\pgfqpoint{3.352233in}{1.400000in}}{\pgfqpoint{2.407767in}{1.544118in}}%
\pgfusepath{clip}%
\pgfsetbuttcap%
\pgfsetroundjoin%
\pgfsetlinewidth{0.501875pt}%
\definecolor{currentstroke}{rgb}{0.271305,0.019942,0.347269}%
\pgfsetstrokecolor{currentstroke}%
\pgfsetdash{}{0pt}%
\pgfpathmoveto{\pgfqpoint{4.501936in}{2.727997in}}%
\pgfpathlineto{\pgfqpoint{4.501936in}{2.727997in}}%
\pgfusepath{stroke}%
\end{pgfscope}%
\begin{pgfscope}%
\pgfpathrectangle{\pgfqpoint{3.352233in}{1.400000in}}{\pgfqpoint{2.407767in}{1.544118in}}%
\pgfusepath{clip}%
\pgfsetbuttcap%
\pgfsetroundjoin%
\pgfsetlinewidth{0.501875pt}%
\definecolor{currentstroke}{rgb}{0.271305,0.019942,0.347269}%
\pgfsetstrokecolor{currentstroke}%
\pgfsetdash{}{0pt}%
\pgfpathmoveto{\pgfqpoint{4.501936in}{2.727997in}}%
\pgfpathlineto{\pgfqpoint{4.501936in}{2.727997in}}%
\pgfusepath{stroke}%
\end{pgfscope}%
\begin{pgfscope}%
\pgfpathrectangle{\pgfqpoint{3.352233in}{1.400000in}}{\pgfqpoint{2.407767in}{1.544118in}}%
\pgfusepath{clip}%
\pgfsetbuttcap%
\pgfsetroundjoin%
\pgfsetlinewidth{0.501875pt}%
\definecolor{currentstroke}{rgb}{0.271305,0.019942,0.347269}%
\pgfsetstrokecolor{currentstroke}%
\pgfsetdash{}{0pt}%
\pgfpathmoveto{\pgfqpoint{4.501936in}{2.727997in}}%
\pgfpathlineto{\pgfqpoint{4.497483in}{2.721195in}}%
\pgfusepath{stroke}%
\end{pgfscope}%
\begin{pgfscope}%
\pgfpathrectangle{\pgfqpoint{3.352233in}{1.400000in}}{\pgfqpoint{2.407767in}{1.544118in}}%
\pgfusepath{clip}%
\pgfsetbuttcap%
\pgfsetroundjoin%
\pgfsetlinewidth{0.501875pt}%
\definecolor{currentstroke}{rgb}{0.271305,0.019942,0.347269}%
\pgfsetstrokecolor{currentstroke}%
\pgfsetdash{}{0pt}%
\pgfpathmoveto{\pgfqpoint{4.497483in}{2.721195in}}%
\pgfpathlineto{\pgfqpoint{4.498053in}{2.713651in}}%
\pgfusepath{stroke}%
\end{pgfscope}%
\begin{pgfscope}%
\pgfpathrectangle{\pgfqpoint{3.352233in}{1.400000in}}{\pgfqpoint{2.407767in}{1.544118in}}%
\pgfusepath{clip}%
\pgfsetbuttcap%
\pgfsetroundjoin%
\pgfsetlinewidth{0.501875pt}%
\definecolor{currentstroke}{rgb}{0.271305,0.019942,0.347269}%
\pgfsetstrokecolor{currentstroke}%
\pgfsetdash{}{0pt}%
\pgfpathmoveto{\pgfqpoint{4.498053in}{2.713651in}}%
\pgfpathlineto{\pgfqpoint{4.499564in}{2.701659in}}%
\pgfusepath{stroke}%
\end{pgfscope}%
\begin{pgfscope}%
\pgfpathrectangle{\pgfqpoint{3.352233in}{1.400000in}}{\pgfqpoint{2.407767in}{1.544118in}}%
\pgfusepath{clip}%
\pgfsetbuttcap%
\pgfsetroundjoin%
\pgfsetlinewidth{0.501875pt}%
\definecolor{currentstroke}{rgb}{0.271305,0.019942,0.347269}%
\pgfsetstrokecolor{currentstroke}%
\pgfsetdash{}{0pt}%
\pgfpathmoveto{\pgfqpoint{4.499564in}{2.701659in}}%
\pgfpathlineto{\pgfqpoint{4.499564in}{2.701659in}}%
\pgfusepath{stroke}%
\end{pgfscope}%
\begin{pgfscope}%
\pgfpathrectangle{\pgfqpoint{3.352233in}{1.400000in}}{\pgfqpoint{2.407767in}{1.544118in}}%
\pgfusepath{clip}%
\pgfsetbuttcap%
\pgfsetroundjoin%
\pgfsetlinewidth{0.501875pt}%
\definecolor{currentstroke}{rgb}{0.271305,0.019942,0.347269}%
\pgfsetstrokecolor{currentstroke}%
\pgfsetdash{}{0pt}%
\pgfpathmoveto{\pgfqpoint{4.499564in}{2.701659in}}%
\pgfpathlineto{\pgfqpoint{4.499564in}{2.701659in}}%
\pgfusepath{stroke}%
\end{pgfscope}%
\begin{pgfscope}%
\pgfpathrectangle{\pgfqpoint{3.352233in}{1.400000in}}{\pgfqpoint{2.407767in}{1.544118in}}%
\pgfusepath{clip}%
\pgfsetbuttcap%
\pgfsetroundjoin%
\pgfsetlinewidth{0.501875pt}%
\definecolor{currentstroke}{rgb}{0.271305,0.019942,0.347269}%
\pgfsetstrokecolor{currentstroke}%
\pgfsetdash{}{0pt}%
\pgfpathmoveto{\pgfqpoint{4.499564in}{2.701659in}}%
\pgfpathlineto{\pgfqpoint{4.494519in}{2.692503in}}%
\pgfusepath{stroke}%
\end{pgfscope}%
\begin{pgfscope}%
\pgfpathrectangle{\pgfqpoint{3.352233in}{1.400000in}}{\pgfqpoint{2.407767in}{1.544118in}}%
\pgfusepath{clip}%
\pgfsetbuttcap%
\pgfsetroundjoin%
\pgfsetlinewidth{0.501875pt}%
\definecolor{currentstroke}{rgb}{0.268510,0.009605,0.335427}%
\pgfsetstrokecolor{currentstroke}%
\pgfsetdash{}{0pt}%
\pgfpathmoveto{\pgfqpoint{4.494519in}{2.692503in}}%
\pgfpathlineto{\pgfqpoint{4.494144in}{2.687568in}}%
\pgfusepath{stroke}%
\end{pgfscope}%
\begin{pgfscope}%
\pgfpathrectangle{\pgfqpoint{3.352233in}{1.400000in}}{\pgfqpoint{2.407767in}{1.544118in}}%
\pgfusepath{clip}%
\pgfsetbuttcap%
\pgfsetroundjoin%
\pgfsetlinewidth{0.501875pt}%
\definecolor{currentstroke}{rgb}{0.268510,0.009605,0.335427}%
\pgfsetstrokecolor{currentstroke}%
\pgfsetdash{}{0pt}%
\pgfpathmoveto{\pgfqpoint{4.494144in}{2.687568in}}%
\pgfpathlineto{\pgfqpoint{4.495687in}{2.678397in}}%
\pgfusepath{stroke}%
\end{pgfscope}%
\begin{pgfscope}%
\pgfpathrectangle{\pgfqpoint{3.352233in}{1.400000in}}{\pgfqpoint{2.407767in}{1.544118in}}%
\pgfusepath{clip}%
\pgfsetbuttcap%
\pgfsetroundjoin%
\pgfsetlinewidth{0.501875pt}%
\definecolor{currentstroke}{rgb}{0.269944,0.014625,0.341379}%
\pgfsetstrokecolor{currentstroke}%
\pgfsetdash{}{0pt}%
\pgfpathmoveto{\pgfqpoint{4.495687in}{2.678397in}}%
\pgfpathlineto{\pgfqpoint{4.496011in}{2.670087in}}%
\pgfusepath{stroke}%
\end{pgfscope}%
\begin{pgfscope}%
\pgfpathrectangle{\pgfqpoint{3.352233in}{1.400000in}}{\pgfqpoint{2.407767in}{1.544118in}}%
\pgfusepath{clip}%
\pgfsetbuttcap%
\pgfsetroundjoin%
\pgfsetlinewidth{0.501875pt}%
\definecolor{currentstroke}{rgb}{0.269944,0.014625,0.341379}%
\pgfsetstrokecolor{currentstroke}%
\pgfsetdash{}{0pt}%
\pgfpathmoveto{\pgfqpoint{4.496011in}{2.670087in}}%
\pgfpathlineto{\pgfqpoint{4.496011in}{2.670087in}}%
\pgfusepath{stroke}%
\end{pgfscope}%
\begin{pgfscope}%
\pgfpathrectangle{\pgfqpoint{3.352233in}{1.400000in}}{\pgfqpoint{2.407767in}{1.544118in}}%
\pgfusepath{clip}%
\pgfsetbuttcap%
\pgfsetroundjoin%
\pgfsetlinewidth{0.501875pt}%
\definecolor{currentstroke}{rgb}{0.269944,0.014625,0.341379}%
\pgfsetstrokecolor{currentstroke}%
\pgfsetdash{}{0pt}%
\pgfpathmoveto{\pgfqpoint{4.496011in}{2.670087in}}%
\pgfpathlineto{\pgfqpoint{4.496473in}{2.665242in}}%
\pgfusepath{stroke}%
\end{pgfscope}%
\begin{pgfscope}%
\pgfpathrectangle{\pgfqpoint{3.352233in}{1.400000in}}{\pgfqpoint{2.407767in}{1.544118in}}%
\pgfusepath{clip}%
\pgfsetbuttcap%
\pgfsetroundjoin%
\pgfsetlinewidth{0.501875pt}%
\definecolor{currentstroke}{rgb}{0.269944,0.014625,0.341379}%
\pgfsetstrokecolor{currentstroke}%
\pgfsetdash{}{0pt}%
\pgfpathmoveto{\pgfqpoint{4.496473in}{2.665242in}}%
\pgfpathlineto{\pgfqpoint{4.496473in}{2.665242in}}%
\pgfusepath{stroke}%
\end{pgfscope}%
\begin{pgfscope}%
\pgfpathrectangle{\pgfqpoint{3.352233in}{1.400000in}}{\pgfqpoint{2.407767in}{1.544118in}}%
\pgfusepath{clip}%
\pgfsetbuttcap%
\pgfsetroundjoin%
\pgfsetlinewidth{0.501875pt}%
\definecolor{currentstroke}{rgb}{0.269944,0.014625,0.341379}%
\pgfsetstrokecolor{currentstroke}%
\pgfsetdash{}{0pt}%
\pgfpathmoveto{\pgfqpoint{4.496473in}{2.665242in}}%
\pgfpathlineto{\pgfqpoint{4.496402in}{2.662188in}}%
\pgfusepath{stroke}%
\end{pgfscope}%
\begin{pgfscope}%
\pgfpathrectangle{\pgfqpoint{3.352233in}{1.400000in}}{\pgfqpoint{2.407767in}{1.544118in}}%
\pgfusepath{clip}%
\pgfsetbuttcap%
\pgfsetroundjoin%
\pgfsetlinewidth{0.501875pt}%
\definecolor{currentstroke}{rgb}{0.269944,0.014625,0.341379}%
\pgfsetstrokecolor{currentstroke}%
\pgfsetdash{}{0pt}%
\pgfpathmoveto{\pgfqpoint{4.496402in}{2.662188in}}%
\pgfpathlineto{\pgfqpoint{4.496970in}{2.659333in}}%
\pgfusepath{stroke}%
\end{pgfscope}%
\begin{pgfscope}%
\pgfpathrectangle{\pgfqpoint{3.352233in}{1.400000in}}{\pgfqpoint{2.407767in}{1.544118in}}%
\pgfusepath{clip}%
\pgfsetbuttcap%
\pgfsetroundjoin%
\pgfsetlinewidth{0.501875pt}%
\definecolor{currentstroke}{rgb}{0.269944,0.014625,0.341379}%
\pgfsetstrokecolor{currentstroke}%
\pgfsetdash{}{0pt}%
\pgfpathmoveto{\pgfqpoint{4.496970in}{2.659333in}}%
\pgfpathlineto{\pgfqpoint{4.497264in}{2.656317in}}%
\pgfusepath{stroke}%
\end{pgfscope}%
\begin{pgfscope}%
\pgfpathrectangle{\pgfqpoint{3.352233in}{1.400000in}}{\pgfqpoint{2.407767in}{1.544118in}}%
\pgfusepath{clip}%
\pgfsetbuttcap%
\pgfsetroundjoin%
\pgfsetlinewidth{0.501875pt}%
\definecolor{currentstroke}{rgb}{0.269944,0.014625,0.341379}%
\pgfsetstrokecolor{currentstroke}%
\pgfsetdash{}{0pt}%
\pgfpathmoveto{\pgfqpoint{4.497264in}{2.656317in}}%
\pgfpathlineto{\pgfqpoint{4.498011in}{2.652538in}}%
\pgfusepath{stroke}%
\end{pgfscope}%
\begin{pgfscope}%
\pgfpathrectangle{\pgfqpoint{3.352233in}{1.400000in}}{\pgfqpoint{2.407767in}{1.544118in}}%
\pgfusepath{clip}%
\pgfsetbuttcap%
\pgfsetroundjoin%
\pgfsetlinewidth{0.501875pt}%
\definecolor{currentstroke}{rgb}{0.269944,0.014625,0.341379}%
\pgfsetstrokecolor{currentstroke}%
\pgfsetdash{}{0pt}%
\pgfpathmoveto{\pgfqpoint{4.498011in}{2.652538in}}%
\pgfpathlineto{\pgfqpoint{4.498011in}{2.652538in}}%
\pgfusepath{stroke}%
\end{pgfscope}%
\begin{pgfscope}%
\pgfpathrectangle{\pgfqpoint{3.352233in}{1.400000in}}{\pgfqpoint{2.407767in}{1.544118in}}%
\pgfusepath{clip}%
\pgfsetbuttcap%
\pgfsetroundjoin%
\pgfsetlinewidth{0.501875pt}%
\definecolor{currentstroke}{rgb}{0.269944,0.014625,0.341379}%
\pgfsetstrokecolor{currentstroke}%
\pgfsetdash{}{0pt}%
\pgfpathmoveto{\pgfqpoint{4.498011in}{2.652538in}}%
\pgfpathlineto{\pgfqpoint{4.502168in}{2.647078in}}%
\pgfusepath{stroke}%
\end{pgfscope}%
\begin{pgfscope}%
\pgfpathrectangle{\pgfqpoint{3.352233in}{1.400000in}}{\pgfqpoint{2.407767in}{1.544118in}}%
\pgfusepath{clip}%
\pgfsetbuttcap%
\pgfsetroundjoin%
\pgfsetlinewidth{0.501875pt}%
\definecolor{currentstroke}{rgb}{0.271305,0.019942,0.347269}%
\pgfsetstrokecolor{currentstroke}%
\pgfsetdash{}{0pt}%
\pgfpathmoveto{\pgfqpoint{4.231035in}{1.650867in}}%
\pgfpathlineto{\pgfqpoint{4.280593in}{1.660217in}}%
\pgfusepath{stroke}%
\end{pgfscope}%
\begin{pgfscope}%
\pgfpathrectangle{\pgfqpoint{3.352233in}{1.400000in}}{\pgfqpoint{2.407767in}{1.544118in}}%
\pgfusepath{clip}%
\pgfsetbuttcap%
\pgfsetroundjoin%
\pgfsetlinewidth{0.501875pt}%
\definecolor{currentstroke}{rgb}{0.274952,0.037752,0.364543}%
\pgfsetstrokecolor{currentstroke}%
\pgfsetdash{}{0pt}%
\pgfpathmoveto{\pgfqpoint{4.280593in}{1.660217in}}%
\pgfpathlineto{\pgfqpoint{4.280593in}{1.660217in}}%
\pgfusepath{stroke}%
\end{pgfscope}%
\begin{pgfscope}%
\pgfpathrectangle{\pgfqpoint{3.352233in}{1.400000in}}{\pgfqpoint{2.407767in}{1.544118in}}%
\pgfusepath{clip}%
\pgfsetbuttcap%
\pgfsetroundjoin%
\pgfsetlinewidth{0.501875pt}%
\definecolor{currentstroke}{rgb}{0.274952,0.037752,0.364543}%
\pgfsetstrokecolor{currentstroke}%
\pgfsetdash{}{0pt}%
\pgfpathmoveto{\pgfqpoint{4.280593in}{1.660217in}}%
\pgfpathlineto{\pgfqpoint{4.304652in}{1.665349in}}%
\pgfusepath{stroke}%
\end{pgfscope}%
\begin{pgfscope}%
\pgfpathrectangle{\pgfqpoint{3.352233in}{1.400000in}}{\pgfqpoint{2.407767in}{1.544118in}}%
\pgfusepath{clip}%
\pgfsetbuttcap%
\pgfsetroundjoin%
\pgfsetlinewidth{0.501875pt}%
\definecolor{currentstroke}{rgb}{0.269944,0.014625,0.341379}%
\pgfsetstrokecolor{currentstroke}%
\pgfsetdash{}{0pt}%
\pgfpathmoveto{\pgfqpoint{4.304652in}{1.665349in}}%
\pgfpathlineto{\pgfqpoint{4.304652in}{1.665349in}}%
\pgfusepath{stroke}%
\end{pgfscope}%
\begin{pgfscope}%
\pgfpathrectangle{\pgfqpoint{3.352233in}{1.400000in}}{\pgfqpoint{2.407767in}{1.544118in}}%
\pgfusepath{clip}%
\pgfsetbuttcap%
\pgfsetroundjoin%
\pgfsetlinewidth{0.501875pt}%
\definecolor{currentstroke}{rgb}{0.269944,0.014625,0.341379}%
\pgfsetstrokecolor{currentstroke}%
\pgfsetdash{}{0pt}%
\pgfpathmoveto{\pgfqpoint{4.304652in}{1.665349in}}%
\pgfpathlineto{\pgfqpoint{4.304652in}{1.665349in}}%
\pgfusepath{stroke}%
\end{pgfscope}%
\begin{pgfscope}%
\pgfpathrectangle{\pgfqpoint{3.352233in}{1.400000in}}{\pgfqpoint{2.407767in}{1.544118in}}%
\pgfusepath{clip}%
\pgfsetbuttcap%
\pgfsetroundjoin%
\pgfsetlinewidth{0.501875pt}%
\definecolor{currentstroke}{rgb}{0.269944,0.014625,0.341379}%
\pgfsetstrokecolor{currentstroke}%
\pgfsetdash{}{0pt}%
\pgfpathmoveto{\pgfqpoint{4.304652in}{1.665349in}}%
\pgfpathlineto{\pgfqpoint{4.319020in}{1.668927in}}%
\pgfusepath{stroke}%
\end{pgfscope}%
\begin{pgfscope}%
\pgfpathrectangle{\pgfqpoint{3.352233in}{1.400000in}}{\pgfqpoint{2.407767in}{1.544118in}}%
\pgfusepath{clip}%
\pgfsetbuttcap%
\pgfsetroundjoin%
\pgfsetlinewidth{0.501875pt}%
\definecolor{currentstroke}{rgb}{0.269944,0.014625,0.341379}%
\pgfsetstrokecolor{currentstroke}%
\pgfsetdash{}{0pt}%
\pgfpathmoveto{\pgfqpoint{4.319020in}{1.668927in}}%
\pgfpathlineto{\pgfqpoint{4.319020in}{1.668927in}}%
\pgfusepath{stroke}%
\end{pgfscope}%
\begin{pgfscope}%
\pgfpathrectangle{\pgfqpoint{3.352233in}{1.400000in}}{\pgfqpoint{2.407767in}{1.544118in}}%
\pgfusepath{clip}%
\pgfsetbuttcap%
\pgfsetroundjoin%
\pgfsetlinewidth{0.501875pt}%
\definecolor{currentstroke}{rgb}{0.269944,0.014625,0.341379}%
\pgfsetstrokecolor{currentstroke}%
\pgfsetdash{}{0pt}%
\pgfpathmoveto{\pgfqpoint{4.319020in}{1.668927in}}%
\pgfpathlineto{\pgfqpoint{4.319020in}{1.668927in}}%
\pgfusepath{stroke}%
\end{pgfscope}%
\begin{pgfscope}%
\pgfpathrectangle{\pgfqpoint{3.352233in}{1.400000in}}{\pgfqpoint{2.407767in}{1.544118in}}%
\pgfusepath{clip}%
\pgfsetbuttcap%
\pgfsetroundjoin%
\pgfsetlinewidth{0.501875pt}%
\definecolor{currentstroke}{rgb}{0.269944,0.014625,0.341379}%
\pgfsetstrokecolor{currentstroke}%
\pgfsetdash{}{0pt}%
\pgfpathmoveto{\pgfqpoint{4.319020in}{1.668927in}}%
\pgfpathlineto{\pgfqpoint{4.325686in}{1.673353in}}%
\pgfusepath{stroke}%
\end{pgfscope}%
\begin{pgfscope}%
\pgfpathrectangle{\pgfqpoint{3.352233in}{1.400000in}}{\pgfqpoint{2.407767in}{1.544118in}}%
\pgfusepath{clip}%
\pgfsetbuttcap%
\pgfsetroundjoin%
\pgfsetlinewidth{0.501875pt}%
\definecolor{currentstroke}{rgb}{0.272594,0.025563,0.353093}%
\pgfsetstrokecolor{currentstroke}%
\pgfsetdash{}{0pt}%
\pgfpathmoveto{\pgfqpoint{4.325686in}{1.673353in}}%
\pgfpathlineto{\pgfqpoint{4.329036in}{1.679727in}}%
\pgfusepath{stroke}%
\end{pgfscope}%
\begin{pgfscope}%
\pgfpathrectangle{\pgfqpoint{3.352233in}{1.400000in}}{\pgfqpoint{2.407767in}{1.544118in}}%
\pgfusepath{clip}%
\pgfsetbuttcap%
\pgfsetroundjoin%
\pgfsetlinewidth{0.501875pt}%
\definecolor{currentstroke}{rgb}{0.273809,0.031497,0.358853}%
\pgfsetstrokecolor{currentstroke}%
\pgfsetdash{}{0pt}%
\pgfpathmoveto{\pgfqpoint{4.329036in}{1.679727in}}%
\pgfpathlineto{\pgfqpoint{4.338403in}{1.689433in}}%
\pgfusepath{stroke}%
\end{pgfscope}%
\begin{pgfscope}%
\pgfpathrectangle{\pgfqpoint{3.352233in}{1.400000in}}{\pgfqpoint{2.407767in}{1.544118in}}%
\pgfusepath{clip}%
\pgfsetbuttcap%
\pgfsetroundjoin%
\pgfsetlinewidth{0.501875pt}%
\definecolor{currentstroke}{rgb}{0.277018,0.050344,0.375715}%
\pgfsetstrokecolor{currentstroke}%
\pgfsetdash{}{0pt}%
\pgfpathmoveto{\pgfqpoint{4.338403in}{1.689433in}}%
\pgfpathlineto{\pgfqpoint{4.351833in}{1.695690in}}%
\pgfusepath{stroke}%
\end{pgfscope}%
\begin{pgfscope}%
\pgfpathrectangle{\pgfqpoint{3.352233in}{1.400000in}}{\pgfqpoint{2.407767in}{1.544118in}}%
\pgfusepath{clip}%
\pgfsetbuttcap%
\pgfsetroundjoin%
\pgfsetlinewidth{0.501875pt}%
\definecolor{currentstroke}{rgb}{0.273809,0.031497,0.358853}%
\pgfsetstrokecolor{currentstroke}%
\pgfsetdash{}{0pt}%
\pgfpathmoveto{\pgfqpoint{4.351833in}{1.695690in}}%
\pgfpathlineto{\pgfqpoint{4.351833in}{1.695690in}}%
\pgfusepath{stroke}%
\end{pgfscope}%
\begin{pgfscope}%
\pgfpathrectangle{\pgfqpoint{3.352233in}{1.400000in}}{\pgfqpoint{2.407767in}{1.544118in}}%
\pgfusepath{clip}%
\pgfsetbuttcap%
\pgfsetroundjoin%
\pgfsetlinewidth{0.501875pt}%
\definecolor{currentstroke}{rgb}{0.273809,0.031497,0.358853}%
\pgfsetstrokecolor{currentstroke}%
\pgfsetdash{}{0pt}%
\pgfpathmoveto{\pgfqpoint{4.351833in}{1.695690in}}%
\pgfpathlineto{\pgfqpoint{4.361017in}{1.703368in}}%
\pgfusepath{stroke}%
\end{pgfscope}%
\begin{pgfscope}%
\pgfpathrectangle{\pgfqpoint{3.352233in}{1.400000in}}{\pgfqpoint{2.407767in}{1.544118in}}%
\pgfusepath{clip}%
\pgfsetbuttcap%
\pgfsetroundjoin%
\pgfsetlinewidth{0.501875pt}%
\definecolor{currentstroke}{rgb}{0.272594,0.025563,0.353093}%
\pgfsetstrokecolor{currentstroke}%
\pgfsetdash{}{0pt}%
\pgfpathmoveto{\pgfqpoint{4.361017in}{1.703368in}}%
\pgfpathlineto{\pgfqpoint{4.365271in}{1.711608in}}%
\pgfusepath{stroke}%
\end{pgfscope}%
\begin{pgfscope}%
\pgfpathrectangle{\pgfqpoint{3.352233in}{1.400000in}}{\pgfqpoint{2.407767in}{1.544118in}}%
\pgfusepath{clip}%
\pgfsetbuttcap%
\pgfsetroundjoin%
\pgfsetlinewidth{0.501875pt}%
\definecolor{currentstroke}{rgb}{0.273809,0.031497,0.358853}%
\pgfsetstrokecolor{currentstroke}%
\pgfsetdash{}{0pt}%
\pgfpathmoveto{\pgfqpoint{4.365271in}{1.711608in}}%
\pgfpathlineto{\pgfqpoint{4.365589in}{1.722153in}}%
\pgfusepath{stroke}%
\end{pgfscope}%
\begin{pgfscope}%
\pgfpathrectangle{\pgfqpoint{3.352233in}{1.400000in}}{\pgfqpoint{2.407767in}{1.544118in}}%
\pgfusepath{clip}%
\pgfsetbuttcap%
\pgfsetroundjoin%
\pgfsetlinewidth{0.501875pt}%
\definecolor{currentstroke}{rgb}{0.276022,0.044167,0.370164}%
\pgfsetstrokecolor{currentstroke}%
\pgfsetdash{}{0pt}%
\pgfpathmoveto{\pgfqpoint{4.365589in}{1.722153in}}%
\pgfpathlineto{\pgfqpoint{4.374237in}{1.754197in}}%
\pgfusepath{stroke}%
\end{pgfscope}%
\begin{pgfscope}%
\pgfpathrectangle{\pgfqpoint{3.352233in}{1.400000in}}{\pgfqpoint{2.407767in}{1.544118in}}%
\pgfusepath{clip}%
\pgfsetbuttcap%
\pgfsetroundjoin%
\pgfsetlinewidth{0.501875pt}%
\definecolor{currentstroke}{rgb}{0.274952,0.037752,0.364543}%
\pgfsetstrokecolor{currentstroke}%
\pgfsetdash{}{0pt}%
\pgfpathmoveto{\pgfqpoint{4.374237in}{1.754197in}}%
\pgfpathlineto{\pgfqpoint{4.374237in}{1.754197in}}%
\pgfusepath{stroke}%
\end{pgfscope}%
\begin{pgfscope}%
\pgfpathrectangle{\pgfqpoint{3.352233in}{1.400000in}}{\pgfqpoint{2.407767in}{1.544118in}}%
\pgfusepath{clip}%
\pgfsetbuttcap%
\pgfsetroundjoin%
\pgfsetlinewidth{0.501875pt}%
\definecolor{currentstroke}{rgb}{0.274952,0.037752,0.364543}%
\pgfsetstrokecolor{currentstroke}%
\pgfsetdash{}{0pt}%
\pgfpathmoveto{\pgfqpoint{4.374237in}{1.754197in}}%
\pgfpathlineto{\pgfqpoint{4.380507in}{1.767379in}}%
\pgfusepath{stroke}%
\end{pgfscope}%
\begin{pgfscope}%
\pgfpathrectangle{\pgfqpoint{3.352233in}{1.400000in}}{\pgfqpoint{2.407767in}{1.544118in}}%
\pgfusepath{clip}%
\pgfsetbuttcap%
\pgfsetroundjoin%
\pgfsetlinewidth{0.501875pt}%
\definecolor{currentstroke}{rgb}{0.276022,0.044167,0.370164}%
\pgfsetstrokecolor{currentstroke}%
\pgfsetdash{}{0pt}%
\pgfpathmoveto{\pgfqpoint{4.380507in}{1.767379in}}%
\pgfpathlineto{\pgfqpoint{4.380238in}{1.782755in}}%
\pgfusepath{stroke}%
\end{pgfscope}%
\begin{pgfscope}%
\pgfpathrectangle{\pgfqpoint{3.352233in}{1.400000in}}{\pgfqpoint{2.407767in}{1.544118in}}%
\pgfusepath{clip}%
\pgfsetbuttcap%
\pgfsetroundjoin%
\pgfsetlinewidth{0.501875pt}%
\definecolor{currentstroke}{rgb}{0.277018,0.050344,0.375715}%
\pgfsetstrokecolor{currentstroke}%
\pgfsetdash{}{0pt}%
\pgfpathmoveto{\pgfqpoint{4.380238in}{1.782755in}}%
\pgfpathlineto{\pgfqpoint{4.377043in}{1.797356in}}%
\pgfusepath{stroke}%
\end{pgfscope}%
\begin{pgfscope}%
\pgfpathrectangle{\pgfqpoint{3.352233in}{1.400000in}}{\pgfqpoint{2.407767in}{1.544118in}}%
\pgfusepath{clip}%
\pgfsetbuttcap%
\pgfsetroundjoin%
\pgfsetlinewidth{0.501875pt}%
\definecolor{currentstroke}{rgb}{0.278791,0.062145,0.386592}%
\pgfsetstrokecolor{currentstroke}%
\pgfsetdash{}{0pt}%
\pgfpathmoveto{\pgfqpoint{4.377043in}{1.797356in}}%
\pgfpathlineto{\pgfqpoint{4.368367in}{1.817770in}}%
\pgfusepath{stroke}%
\end{pgfscope}%
\begin{pgfscope}%
\pgfpathrectangle{\pgfqpoint{3.352233in}{1.400000in}}{\pgfqpoint{2.407767in}{1.544118in}}%
\pgfusepath{clip}%
\pgfsetbuttcap%
\pgfsetroundjoin%
\pgfsetlinewidth{0.501875pt}%
\definecolor{currentstroke}{rgb}{0.277941,0.056324,0.381191}%
\pgfsetstrokecolor{currentstroke}%
\pgfsetdash{}{0pt}%
\pgfpathmoveto{\pgfqpoint{4.368367in}{1.817770in}}%
\pgfpathlineto{\pgfqpoint{4.356502in}{1.849392in}}%
\pgfusepath{stroke}%
\end{pgfscope}%
\begin{pgfscope}%
\pgfpathrectangle{\pgfqpoint{3.352233in}{1.400000in}}{\pgfqpoint{2.407767in}{1.544118in}}%
\pgfusepath{clip}%
\pgfsetbuttcap%
\pgfsetroundjoin%
\pgfsetlinewidth{0.501875pt}%
\definecolor{currentstroke}{rgb}{0.278791,0.062145,0.386592}%
\pgfsetstrokecolor{currentstroke}%
\pgfsetdash{}{0pt}%
\pgfpathmoveto{\pgfqpoint{4.356502in}{1.849392in}}%
\pgfpathlineto{\pgfqpoint{4.338791in}{1.879862in}}%
\pgfusepath{stroke}%
\end{pgfscope}%
\begin{pgfscope}%
\pgfpathrectangle{\pgfqpoint{3.352233in}{1.400000in}}{\pgfqpoint{2.407767in}{1.544118in}}%
\pgfusepath{clip}%
\pgfsetbuttcap%
\pgfsetroundjoin%
\pgfsetlinewidth{0.501875pt}%
\definecolor{currentstroke}{rgb}{0.280267,0.073417,0.397163}%
\pgfsetstrokecolor{currentstroke}%
\pgfsetdash{}{0pt}%
\pgfpathmoveto{\pgfqpoint{4.338791in}{1.879862in}}%
\pgfpathlineto{\pgfqpoint{4.338791in}{1.879862in}}%
\pgfusepath{stroke}%
\end{pgfscope}%
\begin{pgfscope}%
\pgfpathrectangle{\pgfqpoint{3.352233in}{1.400000in}}{\pgfqpoint{2.407767in}{1.544118in}}%
\pgfusepath{clip}%
\pgfsetbuttcap%
\pgfsetroundjoin%
\pgfsetlinewidth{0.501875pt}%
\definecolor{currentstroke}{rgb}{0.280267,0.073417,0.397163}%
\pgfsetstrokecolor{currentstroke}%
\pgfsetdash{}{0pt}%
\pgfpathmoveto{\pgfqpoint{4.338791in}{1.879862in}}%
\pgfpathlineto{\pgfqpoint{4.324488in}{1.901579in}}%
\pgfusepath{stroke}%
\end{pgfscope}%
\begin{pgfscope}%
\pgfpathrectangle{\pgfqpoint{3.352233in}{1.400000in}}{\pgfqpoint{2.407767in}{1.544118in}}%
\pgfusepath{clip}%
\pgfsetbuttcap%
\pgfsetroundjoin%
\pgfsetlinewidth{0.501875pt}%
\definecolor{currentstroke}{rgb}{0.278791,0.062145,0.386592}%
\pgfsetstrokecolor{currentstroke}%
\pgfsetdash{}{0pt}%
\pgfpathmoveto{\pgfqpoint{4.324488in}{1.901579in}}%
\pgfpathlineto{\pgfqpoint{4.302061in}{1.930418in}}%
\pgfusepath{stroke}%
\end{pgfscope}%
\begin{pgfscope}%
\pgfpathrectangle{\pgfqpoint{3.352233in}{1.400000in}}{\pgfqpoint{2.407767in}{1.544118in}}%
\pgfusepath{clip}%
\pgfsetbuttcap%
\pgfsetroundjoin%
\pgfsetlinewidth{0.501875pt}%
\definecolor{currentstroke}{rgb}{0.280267,0.073417,0.397163}%
\pgfsetstrokecolor{currentstroke}%
\pgfsetdash{}{0pt}%
\pgfpathmoveto{\pgfqpoint{4.302061in}{1.930418in}}%
\pgfpathlineto{\pgfqpoint{4.281742in}{1.960461in}}%
\pgfusepath{stroke}%
\end{pgfscope}%
\begin{pgfscope}%
\pgfpathrectangle{\pgfqpoint{3.352233in}{1.400000in}}{\pgfqpoint{2.407767in}{1.544118in}}%
\pgfusepath{clip}%
\pgfsetbuttcap%
\pgfsetroundjoin%
\pgfsetlinewidth{0.501875pt}%
\definecolor{currentstroke}{rgb}{0.280894,0.078907,0.402329}%
\pgfsetstrokecolor{currentstroke}%
\pgfsetdash{}{0pt}%
\pgfpathmoveto{\pgfqpoint{4.281742in}{1.960461in}}%
\pgfpathlineto{\pgfqpoint{4.265957in}{1.990894in}}%
\pgfusepath{stroke}%
\end{pgfscope}%
\begin{pgfscope}%
\pgfpathrectangle{\pgfqpoint{3.352233in}{1.400000in}}{\pgfqpoint{2.407767in}{1.544118in}}%
\pgfusepath{clip}%
\pgfsetbuttcap%
\pgfsetroundjoin%
\pgfsetlinewidth{0.501875pt}%
\definecolor{currentstroke}{rgb}{0.280267,0.073417,0.397163}%
\pgfsetstrokecolor{currentstroke}%
\pgfsetdash{}{0pt}%
\pgfpathmoveto{\pgfqpoint{4.265957in}{1.990894in}}%
\pgfpathlineto{\pgfqpoint{4.252008in}{2.021942in}}%
\pgfusepath{stroke}%
\end{pgfscope}%
\begin{pgfscope}%
\pgfpathrectangle{\pgfqpoint{3.352233in}{1.400000in}}{\pgfqpoint{2.407767in}{1.544118in}}%
\pgfusepath{clip}%
\pgfsetbuttcap%
\pgfsetroundjoin%
\pgfsetlinewidth{0.501875pt}%
\definecolor{currentstroke}{rgb}{0.277018,0.050344,0.375715}%
\pgfsetstrokecolor{currentstroke}%
\pgfsetdash{}{0pt}%
\pgfpathmoveto{\pgfqpoint{4.252008in}{2.021942in}}%
\pgfpathlineto{\pgfqpoint{4.252008in}{2.021942in}}%
\pgfusepath{stroke}%
\end{pgfscope}%
\begin{pgfscope}%
\pgfpathrectangle{\pgfqpoint{3.352233in}{1.400000in}}{\pgfqpoint{2.407767in}{1.544118in}}%
\pgfusepath{clip}%
\pgfsetbuttcap%
\pgfsetroundjoin%
\pgfsetlinewidth{0.501875pt}%
\definecolor{currentstroke}{rgb}{0.277018,0.050344,0.375715}%
\pgfsetstrokecolor{currentstroke}%
\pgfsetdash{}{0pt}%
\pgfpathmoveto{\pgfqpoint{4.252008in}{2.021942in}}%
\pgfpathlineto{\pgfqpoint{4.240364in}{2.037760in}}%
\pgfusepath{stroke}%
\end{pgfscope}%
\begin{pgfscope}%
\pgfpathrectangle{\pgfqpoint{3.352233in}{1.400000in}}{\pgfqpoint{2.407767in}{1.544118in}}%
\pgfusepath{clip}%
\pgfsetbuttcap%
\pgfsetroundjoin%
\pgfsetlinewidth{0.501875pt}%
\definecolor{currentstroke}{rgb}{0.276022,0.044167,0.370164}%
\pgfsetstrokecolor{currentstroke}%
\pgfsetdash{}{0pt}%
\pgfpathmoveto{\pgfqpoint{4.240364in}{2.037760in}}%
\pgfpathlineto{\pgfqpoint{4.240364in}{2.037760in}}%
\pgfusepath{stroke}%
\end{pgfscope}%
\begin{pgfscope}%
\pgfpathrectangle{\pgfqpoint{3.352233in}{1.400000in}}{\pgfqpoint{2.407767in}{1.544118in}}%
\pgfusepath{clip}%
\pgfsetbuttcap%
\pgfsetroundjoin%
\pgfsetlinewidth{0.501875pt}%
\definecolor{currentstroke}{rgb}{0.276022,0.044167,0.370164}%
\pgfsetstrokecolor{currentstroke}%
\pgfsetdash{}{0pt}%
\pgfpathmoveto{\pgfqpoint{4.240364in}{2.037760in}}%
\pgfpathlineto{\pgfqpoint{4.240364in}{2.037760in}}%
\pgfusepath{stroke}%
\end{pgfscope}%
\begin{pgfscope}%
\pgfpathrectangle{\pgfqpoint{3.352233in}{1.400000in}}{\pgfqpoint{2.407767in}{1.544118in}}%
\pgfusepath{clip}%
\pgfsetbuttcap%
\pgfsetroundjoin%
\pgfsetlinewidth{0.501875pt}%
\definecolor{currentstroke}{rgb}{0.276022,0.044167,0.370164}%
\pgfsetstrokecolor{currentstroke}%
\pgfsetdash{}{0pt}%
\pgfpathmoveto{\pgfqpoint{4.240364in}{2.037760in}}%
\pgfpathlineto{\pgfqpoint{4.238431in}{2.047966in}}%
\pgfusepath{stroke}%
\end{pgfscope}%
\begin{pgfscope}%
\pgfpathrectangle{\pgfqpoint{3.352233in}{1.400000in}}{\pgfqpoint{2.407767in}{1.544118in}}%
\pgfusepath{clip}%
\pgfsetbuttcap%
\pgfsetroundjoin%
\pgfsetlinewidth{0.501875pt}%
\definecolor{currentstroke}{rgb}{0.273809,0.031497,0.358853}%
\pgfsetstrokecolor{currentstroke}%
\pgfsetdash{}{0pt}%
\pgfpathmoveto{\pgfqpoint{4.238431in}{2.047966in}}%
\pgfpathlineto{\pgfqpoint{4.235231in}{2.057619in}}%
\pgfusepath{stroke}%
\end{pgfscope}%
\begin{pgfscope}%
\pgfpathrectangle{\pgfqpoint{3.352233in}{1.400000in}}{\pgfqpoint{2.407767in}{1.544118in}}%
\pgfusepath{clip}%
\pgfsetbuttcap%
\pgfsetroundjoin%
\pgfsetlinewidth{0.501875pt}%
\definecolor{currentstroke}{rgb}{0.272594,0.025563,0.353093}%
\pgfsetstrokecolor{currentstroke}%
\pgfsetdash{}{0pt}%
\pgfpathmoveto{\pgfqpoint{4.235231in}{2.057619in}}%
\pgfpathlineto{\pgfqpoint{4.230135in}{2.071784in}}%
\pgfusepath{stroke}%
\end{pgfscope}%
\begin{pgfscope}%
\pgfpathrectangle{\pgfqpoint{3.352233in}{1.400000in}}{\pgfqpoint{2.407767in}{1.544118in}}%
\pgfusepath{clip}%
\pgfsetbuttcap%
\pgfsetroundjoin%
\pgfsetlinewidth{0.501875pt}%
\definecolor{currentstroke}{rgb}{0.271305,0.019942,0.347269}%
\pgfsetstrokecolor{currentstroke}%
\pgfsetdash{}{0pt}%
\pgfpathmoveto{\pgfqpoint{4.230135in}{2.071784in}}%
\pgfpathlineto{\pgfqpoint{4.230135in}{2.071784in}}%
\pgfusepath{stroke}%
\end{pgfscope}%
\begin{pgfscope}%
\pgfpathrectangle{\pgfqpoint{3.352233in}{1.400000in}}{\pgfqpoint{2.407767in}{1.544118in}}%
\pgfusepath{clip}%
\pgfsetbuttcap%
\pgfsetroundjoin%
\pgfsetlinewidth{0.501875pt}%
\definecolor{currentstroke}{rgb}{0.271305,0.019942,0.347269}%
\pgfsetstrokecolor{currentstroke}%
\pgfsetdash{}{0pt}%
\pgfpathmoveto{\pgfqpoint{4.230135in}{2.071784in}}%
\pgfpathlineto{\pgfqpoint{4.220703in}{2.084441in}}%
\pgfusepath{stroke}%
\end{pgfscope}%
\begin{pgfscope}%
\pgfpathrectangle{\pgfqpoint{3.352233in}{1.400000in}}{\pgfqpoint{2.407767in}{1.544118in}}%
\pgfusepath{clip}%
\pgfsetbuttcap%
\pgfsetroundjoin%
\pgfsetlinewidth{0.501875pt}%
\definecolor{currentstroke}{rgb}{0.271305,0.019942,0.347269}%
\pgfsetstrokecolor{currentstroke}%
\pgfsetdash{}{0pt}%
\pgfpathmoveto{\pgfqpoint{4.220703in}{2.084441in}}%
\pgfpathlineto{\pgfqpoint{4.220703in}{2.084441in}}%
\pgfusepath{stroke}%
\end{pgfscope}%
\begin{pgfscope}%
\pgfpathrectangle{\pgfqpoint{3.352233in}{1.400000in}}{\pgfqpoint{2.407767in}{1.544118in}}%
\pgfusepath{clip}%
\pgfsetbuttcap%
\pgfsetroundjoin%
\pgfsetlinewidth{0.501875pt}%
\definecolor{currentstroke}{rgb}{0.271305,0.019942,0.347269}%
\pgfsetstrokecolor{currentstroke}%
\pgfsetdash{}{0pt}%
\pgfpathmoveto{\pgfqpoint{4.220703in}{2.084441in}}%
\pgfpathlineto{\pgfqpoint{4.214575in}{2.091909in}}%
\pgfusepath{stroke}%
\end{pgfscope}%
\begin{pgfscope}%
\pgfpathrectangle{\pgfqpoint{3.352233in}{1.400000in}}{\pgfqpoint{2.407767in}{1.544118in}}%
\pgfusepath{clip}%
\pgfsetbuttcap%
\pgfsetroundjoin%
\pgfsetlinewidth{0.501875pt}%
\definecolor{currentstroke}{rgb}{0.272594,0.025563,0.353093}%
\pgfsetstrokecolor{currentstroke}%
\pgfsetdash{}{0pt}%
\pgfpathmoveto{\pgfqpoint{4.214575in}{2.091909in}}%
\pgfpathlineto{\pgfqpoint{4.214448in}{2.096396in}}%
\pgfusepath{stroke}%
\end{pgfscope}%
\begin{pgfscope}%
\pgfpathrectangle{\pgfqpoint{3.352233in}{1.400000in}}{\pgfqpoint{2.407767in}{1.544118in}}%
\pgfusepath{clip}%
\pgfsetbuttcap%
\pgfsetroundjoin%
\pgfsetlinewidth{0.501875pt}%
\definecolor{currentstroke}{rgb}{0.273809,0.031497,0.358853}%
\pgfsetstrokecolor{currentstroke}%
\pgfsetdash{}{0pt}%
\pgfpathmoveto{\pgfqpoint{4.214448in}{2.096396in}}%
\pgfpathlineto{\pgfqpoint{4.213321in}{2.107258in}}%
\pgfusepath{stroke}%
\end{pgfscope}%
\begin{pgfscope}%
\pgfpathrectangle{\pgfqpoint{3.352233in}{1.400000in}}{\pgfqpoint{2.407767in}{1.544118in}}%
\pgfusepath{clip}%
\pgfsetbuttcap%
\pgfsetroundjoin%
\pgfsetlinewidth{0.501875pt}%
\definecolor{currentstroke}{rgb}{0.269944,0.014625,0.341379}%
\pgfsetstrokecolor{currentstroke}%
\pgfsetdash{}{0pt}%
\pgfpathmoveto{\pgfqpoint{4.213321in}{2.107258in}}%
\pgfpathlineto{\pgfqpoint{4.213321in}{2.107258in}}%
\pgfusepath{stroke}%
\end{pgfscope}%
\begin{pgfscope}%
\pgfpathrectangle{\pgfqpoint{3.352233in}{1.400000in}}{\pgfqpoint{2.407767in}{1.544118in}}%
\pgfusepath{clip}%
\pgfsetbuttcap%
\pgfsetroundjoin%
\pgfsetlinewidth{0.501875pt}%
\definecolor{currentstroke}{rgb}{0.269944,0.014625,0.341379}%
\pgfsetstrokecolor{currentstroke}%
\pgfsetdash{}{0pt}%
\pgfpathmoveto{\pgfqpoint{4.213321in}{2.107258in}}%
\pgfpathlineto{\pgfqpoint{4.217022in}{2.112620in}}%
\pgfusepath{stroke}%
\end{pgfscope}%
\begin{pgfscope}%
\pgfpathrectangle{\pgfqpoint{3.352233in}{1.400000in}}{\pgfqpoint{2.407767in}{1.544118in}}%
\pgfusepath{clip}%
\pgfsetbuttcap%
\pgfsetroundjoin%
\pgfsetlinewidth{0.501875pt}%
\definecolor{currentstroke}{rgb}{0.268510,0.009605,0.335427}%
\pgfsetstrokecolor{currentstroke}%
\pgfsetdash{}{0pt}%
\pgfpathmoveto{\pgfqpoint{4.217022in}{2.112620in}}%
\pgfpathlineto{\pgfqpoint{4.217022in}{2.112620in}}%
\pgfusepath{stroke}%
\end{pgfscope}%
\begin{pgfscope}%
\pgfpathrectangle{\pgfqpoint{3.352233in}{1.400000in}}{\pgfqpoint{2.407767in}{1.544118in}}%
\pgfusepath{clip}%
\pgfsetbuttcap%
\pgfsetroundjoin%
\pgfsetlinewidth{0.501875pt}%
\definecolor{currentstroke}{rgb}{0.268510,0.009605,0.335427}%
\pgfsetstrokecolor{currentstroke}%
\pgfsetdash{}{0pt}%
\pgfpathmoveto{\pgfqpoint{4.217022in}{2.112620in}}%
\pgfpathlineto{\pgfqpoint{4.217356in}{2.115226in}}%
\pgfusepath{stroke}%
\end{pgfscope}%
\begin{pgfscope}%
\pgfpathrectangle{\pgfqpoint{3.352233in}{1.400000in}}{\pgfqpoint{2.407767in}{1.544118in}}%
\pgfusepath{clip}%
\pgfsetbuttcap%
\pgfsetroundjoin%
\pgfsetlinewidth{0.501875pt}%
\definecolor{currentstroke}{rgb}{0.269944,0.014625,0.341379}%
\pgfsetstrokecolor{currentstroke}%
\pgfsetdash{}{0pt}%
\pgfpathmoveto{\pgfqpoint{4.217356in}{2.115226in}}%
\pgfpathlineto{\pgfqpoint{4.217609in}{2.116312in}}%
\pgfusepath{stroke}%
\end{pgfscope}%
\begin{pgfscope}%
\pgfpathrectangle{\pgfqpoint{3.352233in}{1.400000in}}{\pgfqpoint{2.407767in}{1.544118in}}%
\pgfusepath{clip}%
\pgfsetbuttcap%
\pgfsetroundjoin%
\pgfsetlinewidth{0.501875pt}%
\definecolor{currentstroke}{rgb}{0.269944,0.014625,0.341379}%
\pgfsetstrokecolor{currentstroke}%
\pgfsetdash{}{0pt}%
\pgfpathmoveto{\pgfqpoint{4.217609in}{2.116312in}}%
\pgfpathlineto{\pgfqpoint{4.216309in}{2.116807in}}%
\pgfusepath{stroke}%
\end{pgfscope}%
\begin{pgfscope}%
\pgfpathrectangle{\pgfqpoint{3.352233in}{1.400000in}}{\pgfqpoint{2.407767in}{1.544118in}}%
\pgfusepath{clip}%
\pgfsetbuttcap%
\pgfsetroundjoin%
\pgfsetlinewidth{0.501875pt}%
\definecolor{currentstroke}{rgb}{0.269944,0.014625,0.341379}%
\pgfsetstrokecolor{currentstroke}%
\pgfsetdash{}{0pt}%
\pgfpathmoveto{\pgfqpoint{4.216309in}{2.116807in}}%
\pgfpathlineto{\pgfqpoint{4.216309in}{2.116807in}}%
\pgfusepath{stroke}%
\end{pgfscope}%
\begin{pgfscope}%
\pgfpathrectangle{\pgfqpoint{3.352233in}{1.400000in}}{\pgfqpoint{2.407767in}{1.544118in}}%
\pgfusepath{clip}%
\pgfsetbuttcap%
\pgfsetroundjoin%
\pgfsetlinewidth{0.501875pt}%
\definecolor{currentstroke}{rgb}{0.269944,0.014625,0.341379}%
\pgfsetstrokecolor{currentstroke}%
\pgfsetdash{}{0pt}%
\pgfpathmoveto{\pgfqpoint{4.216309in}{2.116807in}}%
\pgfpathlineto{\pgfqpoint{4.215918in}{2.117101in}}%
\pgfusepath{stroke}%
\end{pgfscope}%
\begin{pgfscope}%
\pgfpathrectangle{\pgfqpoint{3.352233in}{1.400000in}}{\pgfqpoint{2.407767in}{1.544118in}}%
\pgfusepath{clip}%
\pgfsetbuttcap%
\pgfsetroundjoin%
\pgfsetlinewidth{0.501875pt}%
\definecolor{currentstroke}{rgb}{0.269944,0.014625,0.341379}%
\pgfsetstrokecolor{currentstroke}%
\pgfsetdash{}{0pt}%
\pgfpathmoveto{\pgfqpoint{4.215918in}{2.117101in}}%
\pgfpathlineto{\pgfqpoint{4.214859in}{2.117217in}}%
\pgfusepath{stroke}%
\end{pgfscope}%
\begin{pgfscope}%
\pgfpathrectangle{\pgfqpoint{3.352233in}{1.400000in}}{\pgfqpoint{2.407767in}{1.544118in}}%
\pgfusepath{clip}%
\pgfsetbuttcap%
\pgfsetroundjoin%
\pgfsetlinewidth{0.501875pt}%
\definecolor{currentstroke}{rgb}{0.269944,0.014625,0.341379}%
\pgfsetstrokecolor{currentstroke}%
\pgfsetdash{}{0pt}%
\pgfpathmoveto{\pgfqpoint{4.214859in}{2.117217in}}%
\pgfpathlineto{\pgfqpoint{4.214498in}{2.117277in}}%
\pgfusepath{stroke}%
\end{pgfscope}%
\begin{pgfscope}%
\pgfpathrectangle{\pgfqpoint{3.352233in}{1.400000in}}{\pgfqpoint{2.407767in}{1.544118in}}%
\pgfusepath{clip}%
\pgfsetbuttcap%
\pgfsetroundjoin%
\pgfsetlinewidth{0.501875pt}%
\definecolor{currentstroke}{rgb}{0.269944,0.014625,0.341379}%
\pgfsetstrokecolor{currentstroke}%
\pgfsetdash{}{0pt}%
\pgfpathmoveto{\pgfqpoint{4.214498in}{2.117277in}}%
\pgfpathlineto{\pgfqpoint{4.215276in}{2.117340in}}%
\pgfusepath{stroke}%
\end{pgfscope}%
\begin{pgfscope}%
\pgfpathrectangle{\pgfqpoint{3.352233in}{1.400000in}}{\pgfqpoint{2.407767in}{1.544118in}}%
\pgfusepath{clip}%
\pgfsetbuttcap%
\pgfsetroundjoin%
\pgfsetlinewidth{0.501875pt}%
\definecolor{currentstroke}{rgb}{0.269944,0.014625,0.341379}%
\pgfsetstrokecolor{currentstroke}%
\pgfsetdash{}{0pt}%
\pgfpathmoveto{\pgfqpoint{4.215276in}{2.117340in}}%
\pgfpathlineto{\pgfqpoint{4.215879in}{2.117380in}}%
\pgfusepath{stroke}%
\end{pgfscope}%
\begin{pgfscope}%
\pgfpathrectangle{\pgfqpoint{3.352233in}{1.400000in}}{\pgfqpoint{2.407767in}{1.544118in}}%
\pgfusepath{clip}%
\pgfsetbuttcap%
\pgfsetroundjoin%
\pgfsetlinewidth{0.501875pt}%
\definecolor{currentstroke}{rgb}{0.269944,0.014625,0.341379}%
\pgfsetstrokecolor{currentstroke}%
\pgfsetdash{}{0pt}%
\pgfpathmoveto{\pgfqpoint{4.215879in}{2.117380in}}%
\pgfpathlineto{\pgfqpoint{4.215922in}{2.117392in}}%
\pgfusepath{stroke}%
\end{pgfscope}%
\begin{pgfscope}%
\pgfpathrectangle{\pgfqpoint{3.352233in}{1.400000in}}{\pgfqpoint{2.407767in}{1.544118in}}%
\pgfusepath{clip}%
\pgfsetbuttcap%
\pgfsetroundjoin%
\pgfsetlinewidth{0.501875pt}%
\definecolor{currentstroke}{rgb}{0.269944,0.014625,0.341379}%
\pgfsetstrokecolor{currentstroke}%
\pgfsetdash{}{0pt}%
\pgfpathmoveto{\pgfqpoint{4.215922in}{2.117392in}}%
\pgfpathlineto{\pgfqpoint{4.215499in}{2.117381in}}%
\pgfusepath{stroke}%
\end{pgfscope}%
\begin{pgfscope}%
\pgfpathrectangle{\pgfqpoint{3.352233in}{1.400000in}}{\pgfqpoint{2.407767in}{1.544118in}}%
\pgfusepath{clip}%
\pgfsetbuttcap%
\pgfsetroundjoin%
\pgfsetlinewidth{0.501875pt}%
\definecolor{currentstroke}{rgb}{0.269944,0.014625,0.341379}%
\pgfsetstrokecolor{currentstroke}%
\pgfsetdash{}{0pt}%
\pgfpathmoveto{\pgfqpoint{4.215499in}{2.117381in}}%
\pgfpathlineto{\pgfqpoint{4.215069in}{2.117367in}}%
\pgfusepath{stroke}%
\end{pgfscope}%
\begin{pgfscope}%
\pgfpathrectangle{\pgfqpoint{3.352233in}{1.400000in}}{\pgfqpoint{2.407767in}{1.544118in}}%
\pgfusepath{clip}%
\pgfsetbuttcap%
\pgfsetroundjoin%
\pgfsetlinewidth{0.501875pt}%
\definecolor{currentstroke}{rgb}{0.269944,0.014625,0.341379}%
\pgfsetstrokecolor{currentstroke}%
\pgfsetdash{}{0pt}%
\pgfpathmoveto{\pgfqpoint{4.215069in}{2.117367in}}%
\pgfpathlineto{\pgfqpoint{4.215178in}{2.117372in}}%
\pgfusepath{stroke}%
\end{pgfscope}%
\begin{pgfscope}%
\pgfpathrectangle{\pgfqpoint{3.352233in}{1.400000in}}{\pgfqpoint{2.407767in}{1.544118in}}%
\pgfusepath{clip}%
\pgfsetbuttcap%
\pgfsetroundjoin%
\pgfsetlinewidth{0.501875pt}%
\definecolor{currentstroke}{rgb}{0.269944,0.014625,0.341379}%
\pgfsetstrokecolor{currentstroke}%
\pgfsetdash{}{0pt}%
\pgfpathmoveto{\pgfqpoint{4.215178in}{2.117372in}}%
\pgfpathlineto{\pgfqpoint{4.215565in}{2.117387in}}%
\pgfusepath{stroke}%
\end{pgfscope}%
\begin{pgfscope}%
\pgfpathrectangle{\pgfqpoint{3.352233in}{1.400000in}}{\pgfqpoint{2.407767in}{1.544118in}}%
\pgfusepath{clip}%
\pgfsetbuttcap%
\pgfsetroundjoin%
\pgfsetlinewidth{0.501875pt}%
\definecolor{currentstroke}{rgb}{0.269944,0.014625,0.341379}%
\pgfsetstrokecolor{currentstroke}%
\pgfsetdash{}{0pt}%
\pgfpathmoveto{\pgfqpoint{4.215565in}{2.117387in}}%
\pgfpathlineto{\pgfqpoint{4.215756in}{2.117395in}}%
\pgfusepath{stroke}%
\end{pgfscope}%
\begin{pgfscope}%
\pgfpathrectangle{\pgfqpoint{3.352233in}{1.400000in}}{\pgfqpoint{2.407767in}{1.544118in}}%
\pgfusepath{clip}%
\pgfsetbuttcap%
\pgfsetroundjoin%
\pgfsetlinewidth{0.501875pt}%
\definecolor{currentstroke}{rgb}{0.269944,0.014625,0.341379}%
\pgfsetstrokecolor{currentstroke}%
\pgfsetdash{}{0pt}%
\pgfpathmoveto{\pgfqpoint{4.215756in}{2.117395in}}%
\pgfpathlineto{\pgfqpoint{4.215638in}{2.117391in}}%
\pgfusepath{stroke}%
\end{pgfscope}%
\begin{pgfscope}%
\pgfpathrectangle{\pgfqpoint{3.352233in}{1.400000in}}{\pgfqpoint{2.407767in}{1.544118in}}%
\pgfusepath{clip}%
\pgfsetbuttcap%
\pgfsetroundjoin%
\pgfsetlinewidth{0.501875pt}%
\definecolor{currentstroke}{rgb}{0.269944,0.014625,0.341379}%
\pgfsetstrokecolor{currentstroke}%
\pgfsetdash{}{0pt}%
\pgfpathmoveto{\pgfqpoint{4.215638in}{2.117391in}}%
\pgfpathlineto{\pgfqpoint{4.215374in}{2.117381in}}%
\pgfusepath{stroke}%
\end{pgfscope}%
\begin{pgfscope}%
\pgfpathrectangle{\pgfqpoint{3.352233in}{1.400000in}}{\pgfqpoint{2.407767in}{1.544118in}}%
\pgfusepath{clip}%
\pgfsetbuttcap%
\pgfsetroundjoin%
\pgfsetlinewidth{0.501875pt}%
\definecolor{currentstroke}{rgb}{0.269944,0.014625,0.341379}%
\pgfsetstrokecolor{currentstroke}%
\pgfsetdash{}{0pt}%
\pgfpathmoveto{\pgfqpoint{4.215374in}{2.117381in}}%
\pgfpathlineto{\pgfqpoint{4.215274in}{2.117377in}}%
\pgfusepath{stroke}%
\end{pgfscope}%
\begin{pgfscope}%
\pgfpathrectangle{\pgfqpoint{3.352233in}{1.400000in}}{\pgfqpoint{2.407767in}{1.544118in}}%
\pgfusepath{clip}%
\pgfsetbuttcap%
\pgfsetroundjoin%
\pgfsetlinewidth{0.501875pt}%
\definecolor{currentstroke}{rgb}{0.269944,0.014625,0.341379}%
\pgfsetstrokecolor{currentstroke}%
\pgfsetdash{}{0pt}%
\pgfpathmoveto{\pgfqpoint{4.215274in}{2.117377in}}%
\pgfpathlineto{\pgfqpoint{4.215425in}{2.117383in}}%
\pgfusepath{stroke}%
\end{pgfscope}%
\begin{pgfscope}%
\pgfpathrectangle{\pgfqpoint{3.352233in}{1.400000in}}{\pgfqpoint{2.407767in}{1.544118in}}%
\pgfusepath{clip}%
\pgfsetbuttcap%
\pgfsetroundjoin%
\pgfsetlinewidth{0.501875pt}%
\definecolor{currentstroke}{rgb}{0.269944,0.014625,0.341379}%
\pgfsetstrokecolor{currentstroke}%
\pgfsetdash{}{0pt}%
\pgfpathmoveto{\pgfqpoint{4.215425in}{2.117383in}}%
\pgfpathlineto{\pgfqpoint{4.215600in}{2.117390in}}%
\pgfusepath{stroke}%
\end{pgfscope}%
\begin{pgfscope}%
\pgfpathrectangle{\pgfqpoint{3.352233in}{1.400000in}}{\pgfqpoint{2.407767in}{1.544118in}}%
\pgfusepath{clip}%
\pgfsetbuttcap%
\pgfsetroundjoin%
\pgfsetlinewidth{0.501875pt}%
\definecolor{currentstroke}{rgb}{0.269944,0.014625,0.341379}%
\pgfsetstrokecolor{currentstroke}%
\pgfsetdash{}{0pt}%
\pgfpathmoveto{\pgfqpoint{4.215600in}{2.117390in}}%
\pgfpathlineto{\pgfqpoint{4.215622in}{2.117391in}}%
\pgfusepath{stroke}%
\end{pgfscope}%
\begin{pgfscope}%
\pgfpathrectangle{\pgfqpoint{3.352233in}{1.400000in}}{\pgfqpoint{2.407767in}{1.544118in}}%
\pgfusepath{clip}%
\pgfsetbuttcap%
\pgfsetroundjoin%
\pgfsetlinewidth{0.501875pt}%
\definecolor{currentstroke}{rgb}{0.269944,0.014625,0.341379}%
\pgfsetstrokecolor{currentstroke}%
\pgfsetdash{}{0pt}%
\pgfpathmoveto{\pgfqpoint{4.215622in}{2.117391in}}%
\pgfpathlineto{\pgfqpoint{4.215503in}{2.117386in}}%
\pgfusepath{stroke}%
\end{pgfscope}%
\begin{pgfscope}%
\pgfpathrectangle{\pgfqpoint{3.352233in}{1.400000in}}{\pgfqpoint{2.407767in}{1.544118in}}%
\pgfusepath{clip}%
\pgfsetbuttcap%
\pgfsetroundjoin%
\pgfsetlinewidth{0.501875pt}%
\definecolor{currentstroke}{rgb}{0.269944,0.014625,0.341379}%
\pgfsetstrokecolor{currentstroke}%
\pgfsetdash{}{0pt}%
\pgfpathmoveto{\pgfqpoint{4.215503in}{2.117386in}}%
\pgfpathlineto{\pgfqpoint{4.215389in}{2.117382in}}%
\pgfusepath{stroke}%
\end{pgfscope}%
\begin{pgfscope}%
\pgfpathrectangle{\pgfqpoint{3.352233in}{1.400000in}}{\pgfqpoint{2.407767in}{1.544118in}}%
\pgfusepath{clip}%
\pgfsetbuttcap%
\pgfsetroundjoin%
\pgfsetlinewidth{0.501875pt}%
\definecolor{currentstroke}{rgb}{0.269944,0.014625,0.341379}%
\pgfsetstrokecolor{currentstroke}%
\pgfsetdash{}{0pt}%
\pgfpathmoveto{\pgfqpoint{4.215389in}{2.117382in}}%
\pgfpathlineto{\pgfqpoint{4.215406in}{2.117382in}}%
\pgfusepath{stroke}%
\end{pgfscope}%
\begin{pgfscope}%
\pgfpathrectangle{\pgfqpoint{3.352233in}{1.400000in}}{\pgfqpoint{2.407767in}{1.544118in}}%
\pgfusepath{clip}%
\pgfsetbuttcap%
\pgfsetroundjoin%
\pgfsetlinewidth{0.501875pt}%
\definecolor{currentstroke}{rgb}{0.269944,0.014625,0.341379}%
\pgfsetstrokecolor{currentstroke}%
\pgfsetdash{}{0pt}%
\pgfpathmoveto{\pgfqpoint{4.215406in}{2.117382in}}%
\pgfpathlineto{\pgfqpoint{4.215507in}{2.117386in}}%
\pgfusepath{stroke}%
\end{pgfscope}%
\begin{pgfscope}%
\pgfpathrectangle{\pgfqpoint{3.352233in}{1.400000in}}{\pgfqpoint{2.407767in}{1.544118in}}%
\pgfusepath{clip}%
\pgfsetbuttcap%
\pgfsetroundjoin%
\pgfsetlinewidth{0.501875pt}%
\definecolor{currentstroke}{rgb}{0.269944,0.014625,0.341379}%
\pgfsetstrokecolor{currentstroke}%
\pgfsetdash{}{0pt}%
\pgfpathmoveto{\pgfqpoint{4.215507in}{2.117386in}}%
\pgfpathlineto{\pgfqpoint{4.215567in}{2.117389in}}%
\pgfusepath{stroke}%
\end{pgfscope}%
\begin{pgfscope}%
\pgfpathrectangle{\pgfqpoint{3.352233in}{1.400000in}}{\pgfqpoint{2.407767in}{1.544118in}}%
\pgfusepath{clip}%
\pgfsetbuttcap%
\pgfsetroundjoin%
\pgfsetlinewidth{0.501875pt}%
\definecolor{currentstroke}{rgb}{0.269944,0.014625,0.341379}%
\pgfsetstrokecolor{currentstroke}%
\pgfsetdash{}{0pt}%
\pgfpathmoveto{\pgfqpoint{4.215567in}{2.117389in}}%
\pgfpathlineto{\pgfqpoint{4.215536in}{2.117387in}}%
\pgfusepath{stroke}%
\end{pgfscope}%
\begin{pgfscope}%
\pgfpathrectangle{\pgfqpoint{3.352233in}{1.400000in}}{\pgfqpoint{2.407767in}{1.544118in}}%
\pgfusepath{clip}%
\pgfsetbuttcap%
\pgfsetroundjoin%
\pgfsetlinewidth{0.501875pt}%
\definecolor{currentstroke}{rgb}{0.269944,0.014625,0.341379}%
\pgfsetstrokecolor{currentstroke}%
\pgfsetdash{}{0pt}%
\pgfpathmoveto{\pgfqpoint{4.215536in}{2.117387in}}%
\pgfpathlineto{\pgfqpoint{4.215463in}{2.117385in}}%
\pgfusepath{stroke}%
\end{pgfscope}%
\begin{pgfscope}%
\pgfpathrectangle{\pgfqpoint{3.352233in}{1.400000in}}{\pgfqpoint{2.407767in}{1.544118in}}%
\pgfusepath{clip}%
\pgfsetbuttcap%
\pgfsetroundjoin%
\pgfsetlinewidth{0.501875pt}%
\definecolor{currentstroke}{rgb}{0.269944,0.014625,0.341379}%
\pgfsetstrokecolor{currentstroke}%
\pgfsetdash{}{0pt}%
\pgfpathmoveto{\pgfqpoint{4.215463in}{2.117385in}}%
\pgfpathlineto{\pgfqpoint{4.215433in}{2.117383in}}%
\pgfusepath{stroke}%
\end{pgfscope}%
\begin{pgfscope}%
\pgfpathrectangle{\pgfqpoint{3.352233in}{1.400000in}}{\pgfqpoint{2.407767in}{1.544118in}}%
\pgfusepath{clip}%
\pgfsetbuttcap%
\pgfsetroundjoin%
\pgfsetlinewidth{0.501875pt}%
\definecolor{currentstroke}{rgb}{0.269944,0.014625,0.341379}%
\pgfsetstrokecolor{currentstroke}%
\pgfsetdash{}{0pt}%
\pgfpathmoveto{\pgfqpoint{4.215433in}{2.117383in}}%
\pgfpathlineto{\pgfqpoint{4.215470in}{2.117385in}}%
\pgfusepath{stroke}%
\end{pgfscope}%
\begin{pgfscope}%
\pgfpathrectangle{\pgfqpoint{3.352233in}{1.400000in}}{\pgfqpoint{2.407767in}{1.544118in}}%
\pgfusepath{clip}%
\pgfsetbuttcap%
\pgfsetroundjoin%
\pgfsetlinewidth{0.501875pt}%
\definecolor{currentstroke}{rgb}{0.269944,0.014625,0.341379}%
\pgfsetstrokecolor{currentstroke}%
\pgfsetdash{}{0pt}%
\pgfpathmoveto{\pgfqpoint{4.215470in}{2.117385in}}%
\pgfpathlineto{\pgfqpoint{4.215520in}{2.117387in}}%
\pgfusepath{stroke}%
\end{pgfscope}%
\begin{pgfscope}%
\pgfpathrectangle{\pgfqpoint{3.352233in}{1.400000in}}{\pgfqpoint{2.407767in}{1.544118in}}%
\pgfusepath{clip}%
\pgfsetbuttcap%
\pgfsetroundjoin%
\pgfsetlinewidth{0.501875pt}%
\definecolor{currentstroke}{rgb}{0.269944,0.014625,0.341379}%
\pgfsetstrokecolor{currentstroke}%
\pgfsetdash{}{0pt}%
\pgfpathmoveto{\pgfqpoint{4.215520in}{2.117387in}}%
\pgfpathlineto{\pgfqpoint{4.215529in}{2.117387in}}%
\pgfusepath{stroke}%
\end{pgfscope}%
\begin{pgfscope}%
\pgfpathrectangle{\pgfqpoint{3.352233in}{1.400000in}}{\pgfqpoint{2.407767in}{1.544118in}}%
\pgfusepath{clip}%
\pgfsetbuttcap%
\pgfsetroundjoin%
\pgfsetlinewidth{0.501875pt}%
\definecolor{currentstroke}{rgb}{0.269944,0.014625,0.341379}%
\pgfsetstrokecolor{currentstroke}%
\pgfsetdash{}{0pt}%
\pgfpathmoveto{\pgfqpoint{4.215529in}{2.117387in}}%
\pgfpathlineto{\pgfqpoint{4.215497in}{2.117386in}}%
\pgfusepath{stroke}%
\end{pgfscope}%
\begin{pgfscope}%
\pgfpathrectangle{\pgfqpoint{3.352233in}{1.400000in}}{\pgfqpoint{2.407767in}{1.544118in}}%
\pgfusepath{clip}%
\pgfsetbuttcap%
\pgfsetroundjoin%
\pgfsetlinewidth{0.501875pt}%
\definecolor{currentstroke}{rgb}{0.269944,0.014625,0.341379}%
\pgfsetstrokecolor{currentstroke}%
\pgfsetdash{}{0pt}%
\pgfpathmoveto{\pgfqpoint{4.215497in}{2.117386in}}%
\pgfpathlineto{\pgfqpoint{4.215464in}{2.117385in}}%
\pgfusepath{stroke}%
\end{pgfscope}%
\begin{pgfscope}%
\pgfpathrectangle{\pgfqpoint{3.352233in}{1.400000in}}{\pgfqpoint{2.407767in}{1.544118in}}%
\pgfusepath{clip}%
\pgfsetbuttcap%
\pgfsetroundjoin%
\pgfsetlinewidth{0.501875pt}%
\definecolor{currentstroke}{rgb}{0.269944,0.014625,0.341379}%
\pgfsetstrokecolor{currentstroke}%
\pgfsetdash{}{0pt}%
\pgfpathmoveto{\pgfqpoint{4.215464in}{2.117385in}}%
\pgfpathlineto{\pgfqpoint{4.215467in}{2.117385in}}%
\pgfusepath{stroke}%
\end{pgfscope}%
\begin{pgfscope}%
\pgfpathrectangle{\pgfqpoint{3.352233in}{1.400000in}}{\pgfqpoint{2.407767in}{1.544118in}}%
\pgfusepath{clip}%
\pgfsetbuttcap%
\pgfsetroundjoin%
\pgfsetlinewidth{0.501875pt}%
\definecolor{currentstroke}{rgb}{0.269944,0.014625,0.341379}%
\pgfsetstrokecolor{currentstroke}%
\pgfsetdash{}{0pt}%
\pgfpathmoveto{\pgfqpoint{4.215467in}{2.117385in}}%
\pgfpathlineto{\pgfqpoint{4.215494in}{2.117386in}}%
\pgfusepath{stroke}%
\end{pgfscope}%
\begin{pgfscope}%
\pgfpathrectangle{\pgfqpoint{3.352233in}{1.400000in}}{\pgfqpoint{2.407767in}{1.544118in}}%
\pgfusepath{clip}%
\pgfsetbuttcap%
\pgfsetroundjoin%
\pgfsetlinewidth{0.501875pt}%
\definecolor{currentstroke}{rgb}{0.269944,0.014625,0.341379}%
\pgfsetstrokecolor{currentstroke}%
\pgfsetdash{}{0pt}%
\pgfpathmoveto{\pgfqpoint{4.215494in}{2.117386in}}%
\pgfpathlineto{\pgfqpoint{4.215512in}{2.117386in}}%
\pgfusepath{stroke}%
\end{pgfscope}%
\begin{pgfscope}%
\pgfpathrectangle{\pgfqpoint{3.352233in}{1.400000in}}{\pgfqpoint{2.407767in}{1.544118in}}%
\pgfusepath{clip}%
\pgfsetbuttcap%
\pgfsetroundjoin%
\pgfsetlinewidth{0.501875pt}%
\definecolor{currentstroke}{rgb}{0.269944,0.014625,0.341379}%
\pgfsetstrokecolor{currentstroke}%
\pgfsetdash{}{0pt}%
\pgfpathmoveto{\pgfqpoint{4.215512in}{2.117386in}}%
\pgfpathlineto{\pgfqpoint{4.215505in}{2.117386in}}%
\pgfusepath{stroke}%
\end{pgfscope}%
\begin{pgfscope}%
\pgfpathrectangle{\pgfqpoint{3.352233in}{1.400000in}}{\pgfqpoint{2.407767in}{1.544118in}}%
\pgfusepath{clip}%
\pgfsetbuttcap%
\pgfsetroundjoin%
\pgfsetlinewidth{0.501875pt}%
\definecolor{currentstroke}{rgb}{0.269944,0.014625,0.341379}%
\pgfsetstrokecolor{currentstroke}%
\pgfsetdash{}{0pt}%
\pgfpathmoveto{\pgfqpoint{4.215505in}{2.117386in}}%
\pgfpathlineto{\pgfqpoint{4.215484in}{2.117385in}}%
\pgfusepath{stroke}%
\end{pgfscope}%
\begin{pgfscope}%
\pgfpathrectangle{\pgfqpoint{3.352233in}{1.400000in}}{\pgfqpoint{2.407767in}{1.544118in}}%
\pgfusepath{clip}%
\pgfsetbuttcap%
\pgfsetroundjoin%
\pgfsetlinewidth{0.501875pt}%
\definecolor{currentstroke}{rgb}{0.269944,0.014625,0.341379}%
\pgfsetstrokecolor{currentstroke}%
\pgfsetdash{}{0pt}%
\pgfpathmoveto{\pgfqpoint{4.215484in}{2.117385in}}%
\pgfpathlineto{\pgfqpoint{4.215475in}{2.117385in}}%
\pgfusepath{stroke}%
\end{pgfscope}%
\begin{pgfscope}%
\pgfpathrectangle{\pgfqpoint{3.352233in}{1.400000in}}{\pgfqpoint{2.407767in}{1.544118in}}%
\pgfusepath{clip}%
\pgfsetbuttcap%
\pgfsetroundjoin%
\pgfsetlinewidth{0.501875pt}%
\definecolor{currentstroke}{rgb}{0.269944,0.014625,0.341379}%
\pgfsetstrokecolor{currentstroke}%
\pgfsetdash{}{0pt}%
\pgfpathmoveto{\pgfqpoint{4.215475in}{2.117385in}}%
\pgfpathlineto{\pgfqpoint{4.215484in}{2.117385in}}%
\pgfusepath{stroke}%
\end{pgfscope}%
\begin{pgfscope}%
\pgfpathrectangle{\pgfqpoint{3.352233in}{1.400000in}}{\pgfqpoint{2.407767in}{1.544118in}}%
\pgfusepath{clip}%
\pgfsetbuttcap%
\pgfsetroundjoin%
\pgfsetlinewidth{0.501875pt}%
\definecolor{currentstroke}{rgb}{0.269944,0.014625,0.341379}%
\pgfsetstrokecolor{currentstroke}%
\pgfsetdash{}{0pt}%
\pgfpathmoveto{\pgfqpoint{4.215484in}{2.117385in}}%
\pgfpathlineto{\pgfqpoint{4.215499in}{2.117386in}}%
\pgfusepath{stroke}%
\end{pgfscope}%
\begin{pgfscope}%
\pgfpathrectangle{\pgfqpoint{3.352233in}{1.400000in}}{\pgfqpoint{2.407767in}{1.544118in}}%
\pgfusepath{clip}%
\pgfsetbuttcap%
\pgfsetroundjoin%
\pgfsetlinewidth{0.501875pt}%
\definecolor{currentstroke}{rgb}{0.269944,0.014625,0.341379}%
\pgfsetstrokecolor{currentstroke}%
\pgfsetdash{}{0pt}%
\pgfpathmoveto{\pgfqpoint{4.215499in}{2.117386in}}%
\pgfpathlineto{\pgfqpoint{4.215502in}{2.117386in}}%
\pgfusepath{stroke}%
\end{pgfscope}%
\begin{pgfscope}%
\pgfpathrectangle{\pgfqpoint{3.352233in}{1.400000in}}{\pgfqpoint{2.407767in}{1.544118in}}%
\pgfusepath{clip}%
\pgfsetbuttcap%
\pgfsetroundjoin%
\pgfsetlinewidth{0.501875pt}%
\definecolor{currentstroke}{rgb}{0.269944,0.014625,0.341379}%
\pgfsetstrokecolor{currentstroke}%
\pgfsetdash{}{0pt}%
\pgfpathmoveto{\pgfqpoint{4.215502in}{2.117386in}}%
\pgfpathlineto{\pgfqpoint{4.215493in}{2.117386in}}%
\pgfusepath{stroke}%
\end{pgfscope}%
\begin{pgfscope}%
\pgfpathrectangle{\pgfqpoint{3.352233in}{1.400000in}}{\pgfqpoint{2.407767in}{1.544118in}}%
\pgfusepath{clip}%
\pgfsetbuttcap%
\pgfsetroundjoin%
\pgfsetlinewidth{0.501875pt}%
\definecolor{currentstroke}{rgb}{0.269944,0.014625,0.341379}%
\pgfsetstrokecolor{currentstroke}%
\pgfsetdash{}{0pt}%
\pgfpathmoveto{\pgfqpoint{4.215493in}{2.117386in}}%
\pgfpathlineto{\pgfqpoint{4.215484in}{2.117385in}}%
\pgfusepath{stroke}%
\end{pgfscope}%
\begin{pgfscope}%
\pgfpathrectangle{\pgfqpoint{3.352233in}{1.400000in}}{\pgfqpoint{2.407767in}{1.544118in}}%
\pgfusepath{clip}%
\pgfsetbuttcap%
\pgfsetroundjoin%
\pgfsetlinewidth{0.501875pt}%
\definecolor{currentstroke}{rgb}{0.269944,0.014625,0.341379}%
\pgfsetstrokecolor{currentstroke}%
\pgfsetdash{}{0pt}%
\pgfpathmoveto{\pgfqpoint{4.215484in}{2.117385in}}%
\pgfpathlineto{\pgfqpoint{4.215484in}{2.117385in}}%
\pgfusepath{stroke}%
\end{pgfscope}%
\begin{pgfscope}%
\pgfpathrectangle{\pgfqpoint{3.352233in}{1.400000in}}{\pgfqpoint{2.407767in}{1.544118in}}%
\pgfusepath{clip}%
\pgfsetbuttcap%
\pgfsetroundjoin%
\pgfsetlinewidth{0.501875pt}%
\definecolor{currentstroke}{rgb}{0.269944,0.014625,0.341379}%
\pgfsetstrokecolor{currentstroke}%
\pgfsetdash{}{0pt}%
\pgfpathmoveto{\pgfqpoint{4.215484in}{2.117385in}}%
\pgfpathlineto{\pgfqpoint{4.215491in}{2.117386in}}%
\pgfusepath{stroke}%
\end{pgfscope}%
\begin{pgfscope}%
\pgfpathrectangle{\pgfqpoint{3.352233in}{1.400000in}}{\pgfqpoint{2.407767in}{1.544118in}}%
\pgfusepath{clip}%
\pgfsetbuttcap%
\pgfsetroundjoin%
\pgfsetlinewidth{0.501875pt}%
\definecolor{currentstroke}{rgb}{0.269944,0.014625,0.341379}%
\pgfsetstrokecolor{currentstroke}%
\pgfsetdash{}{0pt}%
\pgfpathmoveto{\pgfqpoint{4.215491in}{2.117386in}}%
\pgfpathlineto{\pgfqpoint{4.215497in}{2.117386in}}%
\pgfusepath{stroke}%
\end{pgfscope}%
\begin{pgfscope}%
\pgfpathrectangle{\pgfqpoint{3.352233in}{1.400000in}}{\pgfqpoint{2.407767in}{1.544118in}}%
\pgfusepath{clip}%
\pgfsetbuttcap%
\pgfsetroundjoin%
\pgfsetlinewidth{0.501875pt}%
\definecolor{currentstroke}{rgb}{0.269944,0.014625,0.341379}%
\pgfsetstrokecolor{currentstroke}%
\pgfsetdash{}{0pt}%
\pgfpathmoveto{\pgfqpoint{4.215497in}{2.117386in}}%
\pgfpathlineto{\pgfqpoint{4.215495in}{2.117386in}}%
\pgfusepath{stroke}%
\end{pgfscope}%
\begin{pgfscope}%
\pgfpathrectangle{\pgfqpoint{3.352233in}{1.400000in}}{\pgfqpoint{2.407767in}{1.544118in}}%
\pgfusepath{clip}%
\pgfsetbuttcap%
\pgfsetroundjoin%
\pgfsetlinewidth{0.501875pt}%
\definecolor{currentstroke}{rgb}{0.269944,0.014625,0.341379}%
\pgfsetstrokecolor{currentstroke}%
\pgfsetdash{}{0pt}%
\pgfpathmoveto{\pgfqpoint{4.215495in}{2.117386in}}%
\pgfpathlineto{\pgfqpoint{4.215489in}{2.117386in}}%
\pgfusepath{stroke}%
\end{pgfscope}%
\begin{pgfscope}%
\pgfpathrectangle{\pgfqpoint{3.352233in}{1.400000in}}{\pgfqpoint{2.407767in}{1.544118in}}%
\pgfusepath{clip}%
\pgfsetbuttcap%
\pgfsetroundjoin%
\pgfsetlinewidth{0.501875pt}%
\definecolor{currentstroke}{rgb}{0.269944,0.014625,0.341379}%
\pgfsetstrokecolor{currentstroke}%
\pgfsetdash{}{0pt}%
\pgfpathmoveto{\pgfqpoint{4.215489in}{2.117386in}}%
\pgfpathlineto{\pgfqpoint{4.215487in}{2.117386in}}%
\pgfusepath{stroke}%
\end{pgfscope}%
\begin{pgfscope}%
\pgfpathrectangle{\pgfqpoint{3.352233in}{1.400000in}}{\pgfqpoint{2.407767in}{1.544118in}}%
\pgfusepath{clip}%
\pgfsetbuttcap%
\pgfsetroundjoin%
\pgfsetlinewidth{0.501875pt}%
\definecolor{currentstroke}{rgb}{0.269944,0.014625,0.341379}%
\pgfsetstrokecolor{currentstroke}%
\pgfsetdash{}{0pt}%
\pgfpathmoveto{\pgfqpoint{4.215487in}{2.117386in}}%
\pgfpathlineto{\pgfqpoint{4.215489in}{2.117386in}}%
\pgfusepath{stroke}%
\end{pgfscope}%
\begin{pgfscope}%
\pgfpathrectangle{\pgfqpoint{3.352233in}{1.400000in}}{\pgfqpoint{2.407767in}{1.544118in}}%
\pgfusepath{clip}%
\pgfsetbuttcap%
\pgfsetroundjoin%
\pgfsetlinewidth{0.501875pt}%
\definecolor{currentstroke}{rgb}{0.269944,0.014625,0.341379}%
\pgfsetstrokecolor{currentstroke}%
\pgfsetdash{}{0pt}%
\pgfpathmoveto{\pgfqpoint{4.215489in}{2.117386in}}%
\pgfpathlineto{\pgfqpoint{4.215493in}{2.117386in}}%
\pgfusepath{stroke}%
\end{pgfscope}%
\begin{pgfscope}%
\pgfpathrectangle{\pgfqpoint{3.352233in}{1.400000in}}{\pgfqpoint{2.407767in}{1.544118in}}%
\pgfusepath{clip}%
\pgfsetbuttcap%
\pgfsetroundjoin%
\pgfsetlinewidth{0.501875pt}%
\definecolor{currentstroke}{rgb}{0.269944,0.014625,0.341379}%
\pgfsetstrokecolor{currentstroke}%
\pgfsetdash{}{0pt}%
\pgfpathmoveto{\pgfqpoint{4.215493in}{2.117386in}}%
\pgfpathlineto{\pgfqpoint{4.215494in}{2.117386in}}%
\pgfusepath{stroke}%
\end{pgfscope}%
\begin{pgfscope}%
\pgfpathrectangle{\pgfqpoint{3.352233in}{1.400000in}}{\pgfqpoint{2.407767in}{1.544118in}}%
\pgfusepath{clip}%
\pgfsetbuttcap%
\pgfsetroundjoin%
\pgfsetlinewidth{0.501875pt}%
\definecolor{currentstroke}{rgb}{0.269944,0.014625,0.341379}%
\pgfsetstrokecolor{currentstroke}%
\pgfsetdash{}{0pt}%
\pgfpathmoveto{\pgfqpoint{4.215494in}{2.117386in}}%
\pgfpathlineto{\pgfqpoint{4.215492in}{2.117386in}}%
\pgfusepath{stroke}%
\end{pgfscope}%
\begin{pgfscope}%
\pgfpathrectangle{\pgfqpoint{3.352233in}{1.400000in}}{\pgfqpoint{2.407767in}{1.544118in}}%
\pgfusepath{clip}%
\pgfsetbuttcap%
\pgfsetroundjoin%
\pgfsetlinewidth{0.501875pt}%
\definecolor{currentstroke}{rgb}{0.269944,0.014625,0.341379}%
\pgfsetstrokecolor{currentstroke}%
\pgfsetdash{}{0pt}%
\pgfpathmoveto{\pgfqpoint{4.215492in}{2.117386in}}%
\pgfpathlineto{\pgfqpoint{4.215489in}{2.117386in}}%
\pgfusepath{stroke}%
\end{pgfscope}%
\begin{pgfscope}%
\pgfpathrectangle{\pgfqpoint{3.352233in}{1.400000in}}{\pgfqpoint{2.407767in}{1.544118in}}%
\pgfusepath{clip}%
\pgfsetbuttcap%
\pgfsetroundjoin%
\pgfsetlinewidth{0.501875pt}%
\definecolor{currentstroke}{rgb}{0.269944,0.014625,0.341379}%
\pgfsetstrokecolor{currentstroke}%
\pgfsetdash{}{0pt}%
\pgfpathmoveto{\pgfqpoint{4.215489in}{2.117386in}}%
\pgfpathlineto{\pgfqpoint{4.215489in}{2.117386in}}%
\pgfusepath{stroke}%
\end{pgfscope}%
\begin{pgfscope}%
\pgfpathrectangle{\pgfqpoint{3.352233in}{1.400000in}}{\pgfqpoint{2.407767in}{1.544118in}}%
\pgfusepath{clip}%
\pgfsetbuttcap%
\pgfsetroundjoin%
\pgfsetlinewidth{0.501875pt}%
\definecolor{currentstroke}{rgb}{0.269944,0.014625,0.341379}%
\pgfsetstrokecolor{currentstroke}%
\pgfsetdash{}{0pt}%
\pgfpathmoveto{\pgfqpoint{4.215489in}{2.117386in}}%
\pgfpathlineto{\pgfqpoint{4.215491in}{2.117386in}}%
\pgfusepath{stroke}%
\end{pgfscope}%
\begin{pgfscope}%
\pgfpathrectangle{\pgfqpoint{3.352233in}{1.400000in}}{\pgfqpoint{2.407767in}{1.544118in}}%
\pgfusepath{clip}%
\pgfsetbuttcap%
\pgfsetroundjoin%
\pgfsetlinewidth{0.501875pt}%
\definecolor{currentstroke}{rgb}{0.269944,0.014625,0.341379}%
\pgfsetstrokecolor{currentstroke}%
\pgfsetdash{}{0pt}%
\pgfpathmoveto{\pgfqpoint{4.215491in}{2.117386in}}%
\pgfpathlineto{\pgfqpoint{4.215493in}{2.117386in}}%
\pgfusepath{stroke}%
\end{pgfscope}%
\begin{pgfscope}%
\pgfpathrectangle{\pgfqpoint{3.352233in}{1.400000in}}{\pgfqpoint{2.407767in}{1.544118in}}%
\pgfusepath{clip}%
\pgfsetbuttcap%
\pgfsetroundjoin%
\pgfsetlinewidth{0.501875pt}%
\definecolor{currentstroke}{rgb}{0.269944,0.014625,0.341379}%
\pgfsetstrokecolor{currentstroke}%
\pgfsetdash{}{0pt}%
\pgfpathmoveto{\pgfqpoint{4.215493in}{2.117386in}}%
\pgfpathlineto{\pgfqpoint{4.215492in}{2.117386in}}%
\pgfusepath{stroke}%
\end{pgfscope}%
\begin{pgfscope}%
\pgfpathrectangle{\pgfqpoint{3.352233in}{1.400000in}}{\pgfqpoint{2.407767in}{1.544118in}}%
\pgfusepath{clip}%
\pgfsetbuttcap%
\pgfsetroundjoin%
\pgfsetlinewidth{0.501875pt}%
\definecolor{currentstroke}{rgb}{0.269944,0.014625,0.341379}%
\pgfsetstrokecolor{currentstroke}%
\pgfsetdash{}{0pt}%
\pgfpathmoveto{\pgfqpoint{4.215492in}{2.117386in}}%
\pgfpathlineto{\pgfqpoint{4.215491in}{2.117386in}}%
\pgfusepath{stroke}%
\end{pgfscope}%
\begin{pgfscope}%
\pgfpathrectangle{\pgfqpoint{3.352233in}{1.400000in}}{\pgfqpoint{2.407767in}{1.544118in}}%
\pgfusepath{clip}%
\pgfsetbuttcap%
\pgfsetroundjoin%
\pgfsetlinewidth{0.501875pt}%
\definecolor{currentstroke}{rgb}{0.269944,0.014625,0.341379}%
\pgfsetstrokecolor{currentstroke}%
\pgfsetdash{}{0pt}%
\pgfpathmoveto{\pgfqpoint{4.215491in}{2.117386in}}%
\pgfpathlineto{\pgfqpoint{4.215490in}{2.117386in}}%
\pgfusepath{stroke}%
\end{pgfscope}%
\begin{pgfscope}%
\pgfpathrectangle{\pgfqpoint{3.352233in}{1.400000in}}{\pgfqpoint{2.407767in}{1.544118in}}%
\pgfusepath{clip}%
\pgfsetbuttcap%
\pgfsetroundjoin%
\pgfsetlinewidth{0.501875pt}%
\definecolor{currentstroke}{rgb}{0.269944,0.014625,0.341379}%
\pgfsetstrokecolor{currentstroke}%
\pgfsetdash{}{0pt}%
\pgfpathmoveto{\pgfqpoint{4.215490in}{2.117386in}}%
\pgfpathlineto{\pgfqpoint{4.215490in}{2.117386in}}%
\pgfusepath{stroke}%
\end{pgfscope}%
\begin{pgfscope}%
\pgfpathrectangle{\pgfqpoint{3.352233in}{1.400000in}}{\pgfqpoint{2.407767in}{1.544118in}}%
\pgfusepath{clip}%
\pgfsetbuttcap%
\pgfsetroundjoin%
\pgfsetlinewidth{0.501875pt}%
\definecolor{currentstroke}{rgb}{0.269944,0.014625,0.341379}%
\pgfsetstrokecolor{currentstroke}%
\pgfsetdash{}{0pt}%
\pgfpathmoveto{\pgfqpoint{4.215490in}{2.117386in}}%
\pgfpathlineto{\pgfqpoint{4.215492in}{2.117386in}}%
\pgfusepath{stroke}%
\end{pgfscope}%
\begin{pgfscope}%
\pgfpathrectangle{\pgfqpoint{3.352233in}{1.400000in}}{\pgfqpoint{2.407767in}{1.544118in}}%
\pgfusepath{clip}%
\pgfsetbuttcap%
\pgfsetroundjoin%
\pgfsetlinewidth{0.501875pt}%
\definecolor{currentstroke}{rgb}{0.269944,0.014625,0.341379}%
\pgfsetstrokecolor{currentstroke}%
\pgfsetdash{}{0pt}%
\pgfpathmoveto{\pgfqpoint{4.215492in}{2.117386in}}%
\pgfpathlineto{\pgfqpoint{4.215492in}{2.117386in}}%
\pgfusepath{stroke}%
\end{pgfscope}%
\begin{pgfscope}%
\pgfpathrectangle{\pgfqpoint{3.352233in}{1.400000in}}{\pgfqpoint{2.407767in}{1.544118in}}%
\pgfusepath{clip}%
\pgfsetbuttcap%
\pgfsetroundjoin%
\pgfsetlinewidth{0.501875pt}%
\definecolor{currentstroke}{rgb}{0.269944,0.014625,0.341379}%
\pgfsetstrokecolor{currentstroke}%
\pgfsetdash{}{0pt}%
\pgfpathmoveto{\pgfqpoint{4.215492in}{2.117386in}}%
\pgfpathlineto{\pgfqpoint{4.215491in}{2.117386in}}%
\pgfusepath{stroke}%
\end{pgfscope}%
\begin{pgfscope}%
\pgfpathrectangle{\pgfqpoint{3.352233in}{1.400000in}}{\pgfqpoint{2.407767in}{1.544118in}}%
\pgfusepath{clip}%
\pgfsetbuttcap%
\pgfsetroundjoin%
\pgfsetlinewidth{0.501875pt}%
\definecolor{currentstroke}{rgb}{0.269944,0.014625,0.341379}%
\pgfsetstrokecolor{currentstroke}%
\pgfsetdash{}{0pt}%
\pgfpathmoveto{\pgfqpoint{4.215491in}{2.117386in}}%
\pgfpathlineto{\pgfqpoint{4.215491in}{2.117386in}}%
\pgfusepath{stroke}%
\end{pgfscope}%
\begin{pgfscope}%
\pgfpathrectangle{\pgfqpoint{3.352233in}{1.400000in}}{\pgfqpoint{2.407767in}{1.544118in}}%
\pgfusepath{clip}%
\pgfsetbuttcap%
\pgfsetroundjoin%
\pgfsetlinewidth{0.501875pt}%
\definecolor{currentstroke}{rgb}{0.269944,0.014625,0.341379}%
\pgfsetstrokecolor{currentstroke}%
\pgfsetdash{}{0pt}%
\pgfpathmoveto{\pgfqpoint{4.215491in}{2.117386in}}%
\pgfpathlineto{\pgfqpoint{4.215490in}{2.117386in}}%
\pgfusepath{stroke}%
\end{pgfscope}%
\begin{pgfscope}%
\pgfpathrectangle{\pgfqpoint{3.352233in}{1.400000in}}{\pgfqpoint{2.407767in}{1.544118in}}%
\pgfusepath{clip}%
\pgfsetbuttcap%
\pgfsetroundjoin%
\pgfsetlinewidth{0.501875pt}%
\definecolor{currentstroke}{rgb}{0.269944,0.014625,0.341379}%
\pgfsetstrokecolor{currentstroke}%
\pgfsetdash{}{0pt}%
\pgfpathmoveto{\pgfqpoint{4.215490in}{2.117386in}}%
\pgfpathlineto{\pgfqpoint{4.215491in}{2.117386in}}%
\pgfusepath{stroke}%
\end{pgfscope}%
\begin{pgfscope}%
\pgfpathrectangle{\pgfqpoint{3.352233in}{1.400000in}}{\pgfqpoint{2.407767in}{1.544118in}}%
\pgfusepath{clip}%
\pgfsetbuttcap%
\pgfsetroundjoin%
\pgfsetlinewidth{0.501875pt}%
\definecolor{currentstroke}{rgb}{0.269944,0.014625,0.341379}%
\pgfsetstrokecolor{currentstroke}%
\pgfsetdash{}{0pt}%
\pgfpathmoveto{\pgfqpoint{4.215491in}{2.117386in}}%
\pgfpathlineto{\pgfqpoint{4.215491in}{2.117386in}}%
\pgfusepath{stroke}%
\end{pgfscope}%
\begin{pgfscope}%
\pgfpathrectangle{\pgfqpoint{3.352233in}{1.400000in}}{\pgfqpoint{2.407767in}{1.544118in}}%
\pgfusepath{clip}%
\pgfsetbuttcap%
\pgfsetroundjoin%
\pgfsetlinewidth{0.501875pt}%
\definecolor{currentstroke}{rgb}{0.269944,0.014625,0.341379}%
\pgfsetstrokecolor{currentstroke}%
\pgfsetdash{}{0pt}%
\pgfpathmoveto{\pgfqpoint{4.215491in}{2.117386in}}%
\pgfpathlineto{\pgfqpoint{4.215491in}{2.117386in}}%
\pgfusepath{stroke}%
\end{pgfscope}%
\begin{pgfscope}%
\pgfpathrectangle{\pgfqpoint{3.352233in}{1.400000in}}{\pgfqpoint{2.407767in}{1.544118in}}%
\pgfusepath{clip}%
\pgfsetbuttcap%
\pgfsetroundjoin%
\pgfsetlinewidth{0.501875pt}%
\definecolor{currentstroke}{rgb}{0.269944,0.014625,0.341379}%
\pgfsetstrokecolor{currentstroke}%
\pgfsetdash{}{0pt}%
\pgfpathmoveto{\pgfqpoint{4.215491in}{2.117386in}}%
\pgfpathlineto{\pgfqpoint{4.215491in}{2.117386in}}%
\pgfusepath{stroke}%
\end{pgfscope}%
\begin{pgfscope}%
\pgfpathrectangle{\pgfqpoint{3.352233in}{1.400000in}}{\pgfqpoint{2.407767in}{1.544118in}}%
\pgfusepath{clip}%
\pgfsetbuttcap%
\pgfsetroundjoin%
\pgfsetlinewidth{0.501875pt}%
\definecolor{currentstroke}{rgb}{0.269944,0.014625,0.341379}%
\pgfsetstrokecolor{currentstroke}%
\pgfsetdash{}{0pt}%
\pgfpathmoveto{\pgfqpoint{4.215491in}{2.117386in}}%
\pgfpathlineto{\pgfqpoint{4.215491in}{2.117386in}}%
\pgfusepath{stroke}%
\end{pgfscope}%
\begin{pgfscope}%
\pgfpathrectangle{\pgfqpoint{3.352233in}{1.400000in}}{\pgfqpoint{2.407767in}{1.544118in}}%
\pgfusepath{clip}%
\pgfsetbuttcap%
\pgfsetroundjoin%
\pgfsetlinewidth{0.501875pt}%
\definecolor{currentstroke}{rgb}{0.269944,0.014625,0.341379}%
\pgfsetstrokecolor{currentstroke}%
\pgfsetdash{}{0pt}%
\pgfpathmoveto{\pgfqpoint{4.215491in}{2.117386in}}%
\pgfpathlineto{\pgfqpoint{4.215491in}{2.117386in}}%
\pgfusepath{stroke}%
\end{pgfscope}%
\begin{pgfscope}%
\pgfpathrectangle{\pgfqpoint{3.352233in}{1.400000in}}{\pgfqpoint{2.407767in}{1.544118in}}%
\pgfusepath{clip}%
\pgfsetbuttcap%
\pgfsetroundjoin%
\pgfsetlinewidth{0.501875pt}%
\definecolor{currentstroke}{rgb}{0.269944,0.014625,0.341379}%
\pgfsetstrokecolor{currentstroke}%
\pgfsetdash{}{0pt}%
\pgfpathmoveto{\pgfqpoint{4.215491in}{2.117386in}}%
\pgfpathlineto{\pgfqpoint{4.215491in}{2.117386in}}%
\pgfusepath{stroke}%
\end{pgfscope}%
\begin{pgfscope}%
\pgfpathrectangle{\pgfqpoint{3.352233in}{1.400000in}}{\pgfqpoint{2.407767in}{1.544118in}}%
\pgfusepath{clip}%
\pgfsetbuttcap%
\pgfsetroundjoin%
\pgfsetlinewidth{0.501875pt}%
\definecolor{currentstroke}{rgb}{0.269944,0.014625,0.341379}%
\pgfsetstrokecolor{currentstroke}%
\pgfsetdash{}{0pt}%
\pgfpathmoveto{\pgfqpoint{4.215491in}{2.117386in}}%
\pgfpathlineto{\pgfqpoint{4.215491in}{2.117386in}}%
\pgfusepath{stroke}%
\end{pgfscope}%
\begin{pgfscope}%
\pgfpathrectangle{\pgfqpoint{3.352233in}{1.400000in}}{\pgfqpoint{2.407767in}{1.544118in}}%
\pgfusepath{clip}%
\pgfsetbuttcap%
\pgfsetroundjoin%
\pgfsetlinewidth{0.501875pt}%
\definecolor{currentstroke}{rgb}{0.269944,0.014625,0.341379}%
\pgfsetstrokecolor{currentstroke}%
\pgfsetdash{}{0pt}%
\pgfpathmoveto{\pgfqpoint{4.215491in}{2.117386in}}%
\pgfpathlineto{\pgfqpoint{4.215491in}{2.117386in}}%
\pgfusepath{stroke}%
\end{pgfscope}%
\begin{pgfscope}%
\pgfpathrectangle{\pgfqpoint{3.352233in}{1.400000in}}{\pgfqpoint{2.407767in}{1.544118in}}%
\pgfusepath{clip}%
\pgfsetbuttcap%
\pgfsetroundjoin%
\pgfsetlinewidth{0.501875pt}%
\definecolor{currentstroke}{rgb}{0.269944,0.014625,0.341379}%
\pgfsetstrokecolor{currentstroke}%
\pgfsetdash{}{0pt}%
\pgfpathmoveto{\pgfqpoint{4.215491in}{2.117386in}}%
\pgfpathlineto{\pgfqpoint{4.215491in}{2.117386in}}%
\pgfusepath{stroke}%
\end{pgfscope}%
\begin{pgfscope}%
\pgfpathrectangle{\pgfqpoint{3.352233in}{1.400000in}}{\pgfqpoint{2.407767in}{1.544118in}}%
\pgfusepath{clip}%
\pgfsetbuttcap%
\pgfsetroundjoin%
\pgfsetlinewidth{0.501875pt}%
\definecolor{currentstroke}{rgb}{0.269944,0.014625,0.341379}%
\pgfsetstrokecolor{currentstroke}%
\pgfsetdash{}{0pt}%
\pgfpathmoveto{\pgfqpoint{4.215491in}{2.117386in}}%
\pgfpathlineto{\pgfqpoint{4.215491in}{2.117386in}}%
\pgfusepath{stroke}%
\end{pgfscope}%
\begin{pgfscope}%
\pgfpathrectangle{\pgfqpoint{3.352233in}{1.400000in}}{\pgfqpoint{2.407767in}{1.544118in}}%
\pgfusepath{clip}%
\pgfsetbuttcap%
\pgfsetroundjoin%
\pgfsetlinewidth{0.501875pt}%
\definecolor{currentstroke}{rgb}{0.269944,0.014625,0.341379}%
\pgfsetstrokecolor{currentstroke}%
\pgfsetdash{}{0pt}%
\pgfpathmoveto{\pgfqpoint{4.215491in}{2.117386in}}%
\pgfpathlineto{\pgfqpoint{4.215491in}{2.117386in}}%
\pgfusepath{stroke}%
\end{pgfscope}%
\begin{pgfscope}%
\pgfpathrectangle{\pgfqpoint{3.352233in}{1.400000in}}{\pgfqpoint{2.407767in}{1.544118in}}%
\pgfusepath{clip}%
\pgfsetbuttcap%
\pgfsetroundjoin%
\pgfsetlinewidth{0.501875pt}%
\definecolor{currentstroke}{rgb}{0.269944,0.014625,0.341379}%
\pgfsetstrokecolor{currentstroke}%
\pgfsetdash{}{0pt}%
\pgfpathmoveto{\pgfqpoint{4.215491in}{2.117386in}}%
\pgfpathlineto{\pgfqpoint{4.215491in}{2.117386in}}%
\pgfusepath{stroke}%
\end{pgfscope}%
\begin{pgfscope}%
\pgfpathrectangle{\pgfqpoint{3.352233in}{1.400000in}}{\pgfqpoint{2.407767in}{1.544118in}}%
\pgfusepath{clip}%
\pgfsetbuttcap%
\pgfsetroundjoin%
\pgfsetlinewidth{0.501875pt}%
\definecolor{currentstroke}{rgb}{0.269944,0.014625,0.341379}%
\pgfsetstrokecolor{currentstroke}%
\pgfsetdash{}{0pt}%
\pgfpathmoveto{\pgfqpoint{4.215491in}{2.117386in}}%
\pgfpathlineto{\pgfqpoint{4.215491in}{2.117386in}}%
\pgfusepath{stroke}%
\end{pgfscope}%
\begin{pgfscope}%
\pgfpathrectangle{\pgfqpoint{3.352233in}{1.400000in}}{\pgfqpoint{2.407767in}{1.544118in}}%
\pgfusepath{clip}%
\pgfsetbuttcap%
\pgfsetroundjoin%
\pgfsetlinewidth{0.501875pt}%
\definecolor{currentstroke}{rgb}{0.269944,0.014625,0.341379}%
\pgfsetstrokecolor{currentstroke}%
\pgfsetdash{}{0pt}%
\pgfpathmoveto{\pgfqpoint{4.215491in}{2.117386in}}%
\pgfpathlineto{\pgfqpoint{4.215491in}{2.117386in}}%
\pgfusepath{stroke}%
\end{pgfscope}%
\begin{pgfscope}%
\pgfpathrectangle{\pgfqpoint{3.352233in}{1.400000in}}{\pgfqpoint{2.407767in}{1.544118in}}%
\pgfusepath{clip}%
\pgfsetbuttcap%
\pgfsetroundjoin%
\pgfsetlinewidth{0.501875pt}%
\definecolor{currentstroke}{rgb}{0.269944,0.014625,0.341379}%
\pgfsetstrokecolor{currentstroke}%
\pgfsetdash{}{0pt}%
\pgfpathmoveto{\pgfqpoint{4.215491in}{2.117386in}}%
\pgfpathlineto{\pgfqpoint{4.215491in}{2.117386in}}%
\pgfusepath{stroke}%
\end{pgfscope}%
\begin{pgfscope}%
\pgfpathrectangle{\pgfqpoint{3.352233in}{1.400000in}}{\pgfqpoint{2.407767in}{1.544118in}}%
\pgfusepath{clip}%
\pgfsetbuttcap%
\pgfsetroundjoin%
\pgfsetlinewidth{0.501875pt}%
\definecolor{currentstroke}{rgb}{0.269944,0.014625,0.341379}%
\pgfsetstrokecolor{currentstroke}%
\pgfsetdash{}{0pt}%
\pgfpathmoveto{\pgfqpoint{4.215491in}{2.117386in}}%
\pgfpathlineto{\pgfqpoint{4.215491in}{2.117386in}}%
\pgfusepath{stroke}%
\end{pgfscope}%
\begin{pgfscope}%
\pgfpathrectangle{\pgfqpoint{3.352233in}{1.400000in}}{\pgfqpoint{2.407767in}{1.544118in}}%
\pgfusepath{clip}%
\pgfsetbuttcap%
\pgfsetroundjoin%
\pgfsetlinewidth{0.501875pt}%
\definecolor{currentstroke}{rgb}{0.269944,0.014625,0.341379}%
\pgfsetstrokecolor{currentstroke}%
\pgfsetdash{}{0pt}%
\pgfpathmoveto{\pgfqpoint{4.215491in}{2.117386in}}%
\pgfpathlineto{\pgfqpoint{4.215491in}{2.117386in}}%
\pgfusepath{stroke}%
\end{pgfscope}%
\begin{pgfscope}%
\pgfpathrectangle{\pgfqpoint{3.352233in}{1.400000in}}{\pgfqpoint{2.407767in}{1.544118in}}%
\pgfusepath{clip}%
\pgfsetbuttcap%
\pgfsetroundjoin%
\pgfsetlinewidth{0.501875pt}%
\definecolor{currentstroke}{rgb}{0.269944,0.014625,0.341379}%
\pgfsetstrokecolor{currentstroke}%
\pgfsetdash{}{0pt}%
\pgfpathmoveto{\pgfqpoint{4.215491in}{2.117386in}}%
\pgfpathlineto{\pgfqpoint{4.215491in}{2.117386in}}%
\pgfusepath{stroke}%
\end{pgfscope}%
\begin{pgfscope}%
\pgfpathrectangle{\pgfqpoint{3.352233in}{1.400000in}}{\pgfqpoint{2.407767in}{1.544118in}}%
\pgfusepath{clip}%
\pgfsetbuttcap%
\pgfsetroundjoin%
\pgfsetlinewidth{0.501875pt}%
\definecolor{currentstroke}{rgb}{0.269944,0.014625,0.341379}%
\pgfsetstrokecolor{currentstroke}%
\pgfsetdash{}{0pt}%
\pgfpathmoveto{\pgfqpoint{4.215491in}{2.117386in}}%
\pgfpathlineto{\pgfqpoint{4.215491in}{2.117386in}}%
\pgfusepath{stroke}%
\end{pgfscope}%
\begin{pgfscope}%
\pgfpathrectangle{\pgfqpoint{3.352233in}{1.400000in}}{\pgfqpoint{2.407767in}{1.544118in}}%
\pgfusepath{clip}%
\pgfsetbuttcap%
\pgfsetroundjoin%
\pgfsetlinewidth{0.501875pt}%
\definecolor{currentstroke}{rgb}{0.269944,0.014625,0.341379}%
\pgfsetstrokecolor{currentstroke}%
\pgfsetdash{}{0pt}%
\pgfpathmoveto{\pgfqpoint{4.215491in}{2.117386in}}%
\pgfpathlineto{\pgfqpoint{4.215491in}{2.117386in}}%
\pgfusepath{stroke}%
\end{pgfscope}%
\begin{pgfscope}%
\pgfpathrectangle{\pgfqpoint{3.352233in}{1.400000in}}{\pgfqpoint{2.407767in}{1.544118in}}%
\pgfusepath{clip}%
\pgfsetbuttcap%
\pgfsetroundjoin%
\pgfsetlinewidth{0.501875pt}%
\definecolor{currentstroke}{rgb}{0.269944,0.014625,0.341379}%
\pgfsetstrokecolor{currentstroke}%
\pgfsetdash{}{0pt}%
\pgfpathmoveto{\pgfqpoint{4.215491in}{2.117386in}}%
\pgfpathlineto{\pgfqpoint{4.215491in}{2.117386in}}%
\pgfusepath{stroke}%
\end{pgfscope}%
\begin{pgfscope}%
\pgfpathrectangle{\pgfqpoint{3.352233in}{1.400000in}}{\pgfqpoint{2.407767in}{1.544118in}}%
\pgfusepath{clip}%
\pgfsetbuttcap%
\pgfsetroundjoin%
\pgfsetlinewidth{0.501875pt}%
\definecolor{currentstroke}{rgb}{0.269944,0.014625,0.341379}%
\pgfsetstrokecolor{currentstroke}%
\pgfsetdash{}{0pt}%
\pgfpathmoveto{\pgfqpoint{4.215491in}{2.117386in}}%
\pgfpathlineto{\pgfqpoint{4.215491in}{2.117386in}}%
\pgfusepath{stroke}%
\end{pgfscope}%
\begin{pgfscope}%
\pgfpathrectangle{\pgfqpoint{3.352233in}{1.400000in}}{\pgfqpoint{2.407767in}{1.544118in}}%
\pgfusepath{clip}%
\pgfsetbuttcap%
\pgfsetroundjoin%
\pgfsetlinewidth{0.501875pt}%
\definecolor{currentstroke}{rgb}{0.269944,0.014625,0.341379}%
\pgfsetstrokecolor{currentstroke}%
\pgfsetdash{}{0pt}%
\pgfpathmoveto{\pgfqpoint{4.215491in}{2.117386in}}%
\pgfpathlineto{\pgfqpoint{4.215491in}{2.117386in}}%
\pgfusepath{stroke}%
\end{pgfscope}%
\begin{pgfscope}%
\pgfpathrectangle{\pgfqpoint{3.352233in}{1.400000in}}{\pgfqpoint{2.407767in}{1.544118in}}%
\pgfusepath{clip}%
\pgfsetbuttcap%
\pgfsetroundjoin%
\pgfsetlinewidth{0.501875pt}%
\definecolor{currentstroke}{rgb}{0.269944,0.014625,0.341379}%
\pgfsetstrokecolor{currentstroke}%
\pgfsetdash{}{0pt}%
\pgfpathmoveto{\pgfqpoint{4.215491in}{2.117386in}}%
\pgfpathlineto{\pgfqpoint{4.215491in}{2.117386in}}%
\pgfusepath{stroke}%
\end{pgfscope}%
\begin{pgfscope}%
\pgfpathrectangle{\pgfqpoint{3.352233in}{1.400000in}}{\pgfqpoint{2.407767in}{1.544118in}}%
\pgfusepath{clip}%
\pgfsetbuttcap%
\pgfsetroundjoin%
\pgfsetlinewidth{0.501875pt}%
\definecolor{currentstroke}{rgb}{0.269944,0.014625,0.341379}%
\pgfsetstrokecolor{currentstroke}%
\pgfsetdash{}{0pt}%
\pgfpathmoveto{\pgfqpoint{4.215491in}{2.117386in}}%
\pgfpathlineto{\pgfqpoint{4.215491in}{2.117386in}}%
\pgfusepath{stroke}%
\end{pgfscope}%
\begin{pgfscope}%
\pgfpathrectangle{\pgfqpoint{3.352233in}{1.400000in}}{\pgfqpoint{2.407767in}{1.544118in}}%
\pgfusepath{clip}%
\pgfsetbuttcap%
\pgfsetroundjoin%
\pgfsetlinewidth{0.501875pt}%
\definecolor{currentstroke}{rgb}{0.269944,0.014625,0.341379}%
\pgfsetstrokecolor{currentstroke}%
\pgfsetdash{}{0pt}%
\pgfpathmoveto{\pgfqpoint{4.215491in}{2.117386in}}%
\pgfpathlineto{\pgfqpoint{4.215491in}{2.117386in}}%
\pgfusepath{stroke}%
\end{pgfscope}%
\begin{pgfscope}%
\pgfpathrectangle{\pgfqpoint{3.352233in}{1.400000in}}{\pgfqpoint{2.407767in}{1.544118in}}%
\pgfusepath{clip}%
\pgfsetbuttcap%
\pgfsetroundjoin%
\pgfsetlinewidth{0.501875pt}%
\definecolor{currentstroke}{rgb}{0.269944,0.014625,0.341379}%
\pgfsetstrokecolor{currentstroke}%
\pgfsetdash{}{0pt}%
\pgfpathmoveto{\pgfqpoint{4.215491in}{2.117386in}}%
\pgfpathlineto{\pgfqpoint{4.215491in}{2.117386in}}%
\pgfusepath{stroke}%
\end{pgfscope}%
\begin{pgfscope}%
\pgfpathrectangle{\pgfqpoint{3.352233in}{1.400000in}}{\pgfqpoint{2.407767in}{1.544118in}}%
\pgfusepath{clip}%
\pgfsetbuttcap%
\pgfsetroundjoin%
\pgfsetlinewidth{0.501875pt}%
\definecolor{currentstroke}{rgb}{0.269944,0.014625,0.341379}%
\pgfsetstrokecolor{currentstroke}%
\pgfsetdash{}{0pt}%
\pgfpathmoveto{\pgfqpoint{4.215491in}{2.117386in}}%
\pgfpathlineto{\pgfqpoint{4.215491in}{2.117386in}}%
\pgfusepath{stroke}%
\end{pgfscope}%
\begin{pgfscope}%
\pgfpathrectangle{\pgfqpoint{3.352233in}{1.400000in}}{\pgfqpoint{2.407767in}{1.544118in}}%
\pgfusepath{clip}%
\pgfsetbuttcap%
\pgfsetroundjoin%
\pgfsetlinewidth{0.501875pt}%
\definecolor{currentstroke}{rgb}{0.269944,0.014625,0.341379}%
\pgfsetstrokecolor{currentstroke}%
\pgfsetdash{}{0pt}%
\pgfpathmoveto{\pgfqpoint{4.215491in}{2.117386in}}%
\pgfpathlineto{\pgfqpoint{4.215491in}{2.117386in}}%
\pgfusepath{stroke}%
\end{pgfscope}%
\begin{pgfscope}%
\pgfpathrectangle{\pgfqpoint{3.352233in}{1.400000in}}{\pgfqpoint{2.407767in}{1.544118in}}%
\pgfusepath{clip}%
\pgfsetbuttcap%
\pgfsetroundjoin%
\pgfsetlinewidth{0.501875pt}%
\definecolor{currentstroke}{rgb}{0.269944,0.014625,0.341379}%
\pgfsetstrokecolor{currentstroke}%
\pgfsetdash{}{0pt}%
\pgfpathmoveto{\pgfqpoint{4.215491in}{2.117386in}}%
\pgfpathlineto{\pgfqpoint{4.215491in}{2.117386in}}%
\pgfusepath{stroke}%
\end{pgfscope}%
\begin{pgfscope}%
\pgfpathrectangle{\pgfqpoint{3.352233in}{1.400000in}}{\pgfqpoint{2.407767in}{1.544118in}}%
\pgfusepath{clip}%
\pgfsetbuttcap%
\pgfsetroundjoin%
\pgfsetlinewidth{0.501875pt}%
\definecolor{currentstroke}{rgb}{0.269944,0.014625,0.341379}%
\pgfsetstrokecolor{currentstroke}%
\pgfsetdash{}{0pt}%
\pgfpathmoveto{\pgfqpoint{4.215491in}{2.117386in}}%
\pgfpathlineto{\pgfqpoint{4.215491in}{2.117386in}}%
\pgfusepath{stroke}%
\end{pgfscope}%
\begin{pgfscope}%
\pgfpathrectangle{\pgfqpoint{3.352233in}{1.400000in}}{\pgfqpoint{2.407767in}{1.544118in}}%
\pgfusepath{clip}%
\pgfsetbuttcap%
\pgfsetroundjoin%
\pgfsetlinewidth{0.501875pt}%
\definecolor{currentstroke}{rgb}{0.269944,0.014625,0.341379}%
\pgfsetstrokecolor{currentstroke}%
\pgfsetdash{}{0pt}%
\pgfpathmoveto{\pgfqpoint{4.215491in}{2.117386in}}%
\pgfpathlineto{\pgfqpoint{4.215491in}{2.117386in}}%
\pgfusepath{stroke}%
\end{pgfscope}%
\begin{pgfscope}%
\pgfpathrectangle{\pgfqpoint{3.352233in}{1.400000in}}{\pgfqpoint{2.407767in}{1.544118in}}%
\pgfusepath{clip}%
\pgfsetbuttcap%
\pgfsetroundjoin%
\pgfsetlinewidth{0.501875pt}%
\definecolor{currentstroke}{rgb}{0.269944,0.014625,0.341379}%
\pgfsetstrokecolor{currentstroke}%
\pgfsetdash{}{0pt}%
\pgfpathmoveto{\pgfqpoint{4.215491in}{2.117386in}}%
\pgfpathlineto{\pgfqpoint{4.215491in}{2.117386in}}%
\pgfusepath{stroke}%
\end{pgfscope}%
\begin{pgfscope}%
\pgfpathrectangle{\pgfqpoint{3.352233in}{1.400000in}}{\pgfqpoint{2.407767in}{1.544118in}}%
\pgfusepath{clip}%
\pgfsetbuttcap%
\pgfsetroundjoin%
\pgfsetlinewidth{0.501875pt}%
\definecolor{currentstroke}{rgb}{0.269944,0.014625,0.341379}%
\pgfsetstrokecolor{currentstroke}%
\pgfsetdash{}{0pt}%
\pgfpathmoveto{\pgfqpoint{4.215491in}{2.117386in}}%
\pgfpathlineto{\pgfqpoint{4.215491in}{2.117386in}}%
\pgfusepath{stroke}%
\end{pgfscope}%
\begin{pgfscope}%
\pgfpathrectangle{\pgfqpoint{3.352233in}{1.400000in}}{\pgfqpoint{2.407767in}{1.544118in}}%
\pgfusepath{clip}%
\pgfsetbuttcap%
\pgfsetroundjoin%
\pgfsetlinewidth{0.501875pt}%
\definecolor{currentstroke}{rgb}{0.269944,0.014625,0.341379}%
\pgfsetstrokecolor{currentstroke}%
\pgfsetdash{}{0pt}%
\pgfpathmoveto{\pgfqpoint{4.215491in}{2.117386in}}%
\pgfpathlineto{\pgfqpoint{4.215491in}{2.117386in}}%
\pgfusepath{stroke}%
\end{pgfscope}%
\begin{pgfscope}%
\pgfpathrectangle{\pgfqpoint{3.352233in}{1.400000in}}{\pgfqpoint{2.407767in}{1.544118in}}%
\pgfusepath{clip}%
\pgfsetbuttcap%
\pgfsetroundjoin%
\pgfsetlinewidth{0.501875pt}%
\definecolor{currentstroke}{rgb}{0.269944,0.014625,0.341379}%
\pgfsetstrokecolor{currentstroke}%
\pgfsetdash{}{0pt}%
\pgfpathmoveto{\pgfqpoint{4.215491in}{2.117386in}}%
\pgfpathlineto{\pgfqpoint{4.215491in}{2.117386in}}%
\pgfusepath{stroke}%
\end{pgfscope}%
\begin{pgfscope}%
\pgfpathrectangle{\pgfqpoint{3.352233in}{1.400000in}}{\pgfqpoint{2.407767in}{1.544118in}}%
\pgfusepath{clip}%
\pgfsetbuttcap%
\pgfsetroundjoin%
\pgfsetlinewidth{0.501875pt}%
\definecolor{currentstroke}{rgb}{0.269944,0.014625,0.341379}%
\pgfsetstrokecolor{currentstroke}%
\pgfsetdash{}{0pt}%
\pgfpathmoveto{\pgfqpoint{4.215491in}{2.117386in}}%
\pgfpathlineto{\pgfqpoint{4.215491in}{2.117386in}}%
\pgfusepath{stroke}%
\end{pgfscope}%
\begin{pgfscope}%
\pgfpathrectangle{\pgfqpoint{3.352233in}{1.400000in}}{\pgfqpoint{2.407767in}{1.544118in}}%
\pgfusepath{clip}%
\pgfsetbuttcap%
\pgfsetroundjoin%
\pgfsetlinewidth{0.501875pt}%
\definecolor{currentstroke}{rgb}{0.269944,0.014625,0.341379}%
\pgfsetstrokecolor{currentstroke}%
\pgfsetdash{}{0pt}%
\pgfpathmoveto{\pgfqpoint{4.215491in}{2.117386in}}%
\pgfpathlineto{\pgfqpoint{4.215491in}{2.117386in}}%
\pgfusepath{stroke}%
\end{pgfscope}%
\begin{pgfscope}%
\pgfpathrectangle{\pgfqpoint{3.352233in}{1.400000in}}{\pgfqpoint{2.407767in}{1.544118in}}%
\pgfusepath{clip}%
\pgfsetbuttcap%
\pgfsetroundjoin%
\pgfsetlinewidth{0.501875pt}%
\definecolor{currentstroke}{rgb}{0.269944,0.014625,0.341379}%
\pgfsetstrokecolor{currentstroke}%
\pgfsetdash{}{0pt}%
\pgfpathmoveto{\pgfqpoint{4.215491in}{2.117386in}}%
\pgfpathlineto{\pgfqpoint{4.215491in}{2.117386in}}%
\pgfusepath{stroke}%
\end{pgfscope}%
\begin{pgfscope}%
\pgfpathrectangle{\pgfqpoint{3.352233in}{1.400000in}}{\pgfqpoint{2.407767in}{1.544118in}}%
\pgfusepath{clip}%
\pgfsetbuttcap%
\pgfsetroundjoin%
\pgfsetlinewidth{0.501875pt}%
\definecolor{currentstroke}{rgb}{0.269944,0.014625,0.341379}%
\pgfsetstrokecolor{currentstroke}%
\pgfsetdash{}{0pt}%
\pgfpathmoveto{\pgfqpoint{4.215491in}{2.117386in}}%
\pgfpathlineto{\pgfqpoint{4.215491in}{2.117386in}}%
\pgfusepath{stroke}%
\end{pgfscope}%
\begin{pgfscope}%
\pgfpathrectangle{\pgfqpoint{3.352233in}{1.400000in}}{\pgfqpoint{2.407767in}{1.544118in}}%
\pgfusepath{clip}%
\pgfsetbuttcap%
\pgfsetroundjoin%
\pgfsetlinewidth{0.501875pt}%
\definecolor{currentstroke}{rgb}{0.269944,0.014625,0.341379}%
\pgfsetstrokecolor{currentstroke}%
\pgfsetdash{}{0pt}%
\pgfpathmoveto{\pgfqpoint{4.215491in}{2.117386in}}%
\pgfpathlineto{\pgfqpoint{4.215491in}{2.117386in}}%
\pgfusepath{stroke}%
\end{pgfscope}%
\begin{pgfscope}%
\pgfpathrectangle{\pgfqpoint{3.352233in}{1.400000in}}{\pgfqpoint{2.407767in}{1.544118in}}%
\pgfusepath{clip}%
\pgfsetbuttcap%
\pgfsetroundjoin%
\pgfsetlinewidth{0.501875pt}%
\definecolor{currentstroke}{rgb}{0.269944,0.014625,0.341379}%
\pgfsetstrokecolor{currentstroke}%
\pgfsetdash{}{0pt}%
\pgfpathmoveto{\pgfqpoint{4.215491in}{2.117386in}}%
\pgfpathlineto{\pgfqpoint{4.215491in}{2.117386in}}%
\pgfusepath{stroke}%
\end{pgfscope}%
\begin{pgfscope}%
\pgfpathrectangle{\pgfqpoint{3.352233in}{1.400000in}}{\pgfqpoint{2.407767in}{1.544118in}}%
\pgfusepath{clip}%
\pgfsetbuttcap%
\pgfsetroundjoin%
\pgfsetlinewidth{0.501875pt}%
\definecolor{currentstroke}{rgb}{0.269944,0.014625,0.341379}%
\pgfsetstrokecolor{currentstroke}%
\pgfsetdash{}{0pt}%
\pgfpathmoveto{\pgfqpoint{4.215491in}{2.117386in}}%
\pgfpathlineto{\pgfqpoint{4.215491in}{2.117386in}}%
\pgfusepath{stroke}%
\end{pgfscope}%
\begin{pgfscope}%
\pgfpathrectangle{\pgfqpoint{3.352233in}{1.400000in}}{\pgfqpoint{2.407767in}{1.544118in}}%
\pgfusepath{clip}%
\pgfsetbuttcap%
\pgfsetroundjoin%
\pgfsetlinewidth{0.501875pt}%
\definecolor{currentstroke}{rgb}{0.269944,0.014625,0.341379}%
\pgfsetstrokecolor{currentstroke}%
\pgfsetdash{}{0pt}%
\pgfpathmoveto{\pgfqpoint{4.215491in}{2.117386in}}%
\pgfpathlineto{\pgfqpoint{4.215491in}{2.117386in}}%
\pgfusepath{stroke}%
\end{pgfscope}%
\begin{pgfscope}%
\pgfpathrectangle{\pgfqpoint{3.352233in}{1.400000in}}{\pgfqpoint{2.407767in}{1.544118in}}%
\pgfusepath{clip}%
\pgfsetbuttcap%
\pgfsetroundjoin%
\pgfsetlinewidth{0.501875pt}%
\definecolor{currentstroke}{rgb}{0.269944,0.014625,0.341379}%
\pgfsetstrokecolor{currentstroke}%
\pgfsetdash{}{0pt}%
\pgfpathmoveto{\pgfqpoint{4.215491in}{2.117386in}}%
\pgfpathlineto{\pgfqpoint{4.215491in}{2.117386in}}%
\pgfusepath{stroke}%
\end{pgfscope}%
\begin{pgfscope}%
\pgfpathrectangle{\pgfqpoint{3.352233in}{1.400000in}}{\pgfqpoint{2.407767in}{1.544118in}}%
\pgfusepath{clip}%
\pgfsetbuttcap%
\pgfsetroundjoin%
\pgfsetlinewidth{0.501875pt}%
\definecolor{currentstroke}{rgb}{0.269944,0.014625,0.341379}%
\pgfsetstrokecolor{currentstroke}%
\pgfsetdash{}{0pt}%
\pgfpathmoveto{\pgfqpoint{4.215491in}{2.117386in}}%
\pgfpathlineto{\pgfqpoint{4.215491in}{2.117386in}}%
\pgfusepath{stroke}%
\end{pgfscope}%
\begin{pgfscope}%
\pgfpathrectangle{\pgfqpoint{3.352233in}{1.400000in}}{\pgfqpoint{2.407767in}{1.544118in}}%
\pgfusepath{clip}%
\pgfsetbuttcap%
\pgfsetroundjoin%
\pgfsetlinewidth{0.501875pt}%
\definecolor{currentstroke}{rgb}{0.269944,0.014625,0.341379}%
\pgfsetstrokecolor{currentstroke}%
\pgfsetdash{}{0pt}%
\pgfpathmoveto{\pgfqpoint{4.215491in}{2.117386in}}%
\pgfpathlineto{\pgfqpoint{4.215491in}{2.117386in}}%
\pgfusepath{stroke}%
\end{pgfscope}%
\begin{pgfscope}%
\pgfpathrectangle{\pgfqpoint{3.352233in}{1.400000in}}{\pgfqpoint{2.407767in}{1.544118in}}%
\pgfusepath{clip}%
\pgfsetbuttcap%
\pgfsetroundjoin%
\pgfsetlinewidth{0.501875pt}%
\definecolor{currentstroke}{rgb}{0.269944,0.014625,0.341379}%
\pgfsetstrokecolor{currentstroke}%
\pgfsetdash{}{0pt}%
\pgfpathmoveto{\pgfqpoint{4.215491in}{2.117386in}}%
\pgfpathlineto{\pgfqpoint{4.215491in}{2.117386in}}%
\pgfusepath{stroke}%
\end{pgfscope}%
\begin{pgfscope}%
\pgfpathrectangle{\pgfqpoint{3.352233in}{1.400000in}}{\pgfqpoint{2.407767in}{1.544118in}}%
\pgfusepath{clip}%
\pgfsetbuttcap%
\pgfsetroundjoin%
\pgfsetlinewidth{0.501875pt}%
\definecolor{currentstroke}{rgb}{0.269944,0.014625,0.341379}%
\pgfsetstrokecolor{currentstroke}%
\pgfsetdash{}{0pt}%
\pgfpathmoveto{\pgfqpoint{4.215491in}{2.117386in}}%
\pgfpathlineto{\pgfqpoint{4.215491in}{2.117386in}}%
\pgfusepath{stroke}%
\end{pgfscope}%
\begin{pgfscope}%
\pgfpathrectangle{\pgfqpoint{3.352233in}{1.400000in}}{\pgfqpoint{2.407767in}{1.544118in}}%
\pgfusepath{clip}%
\pgfsetbuttcap%
\pgfsetroundjoin%
\pgfsetlinewidth{0.501875pt}%
\definecolor{currentstroke}{rgb}{0.269944,0.014625,0.341379}%
\pgfsetstrokecolor{currentstroke}%
\pgfsetdash{}{0pt}%
\pgfpathmoveto{\pgfqpoint{4.215491in}{2.117386in}}%
\pgfpathlineto{\pgfqpoint{4.215491in}{2.117386in}}%
\pgfusepath{stroke}%
\end{pgfscope}%
\begin{pgfscope}%
\pgfpathrectangle{\pgfqpoint{3.352233in}{1.400000in}}{\pgfqpoint{2.407767in}{1.544118in}}%
\pgfusepath{clip}%
\pgfsetbuttcap%
\pgfsetroundjoin%
\pgfsetlinewidth{0.501875pt}%
\definecolor{currentstroke}{rgb}{0.269944,0.014625,0.341379}%
\pgfsetstrokecolor{currentstroke}%
\pgfsetdash{}{0pt}%
\pgfpathmoveto{\pgfqpoint{4.215491in}{2.117386in}}%
\pgfpathlineto{\pgfqpoint{4.215491in}{2.117386in}}%
\pgfusepath{stroke}%
\end{pgfscope}%
\begin{pgfscope}%
\pgfpathrectangle{\pgfqpoint{3.352233in}{1.400000in}}{\pgfqpoint{2.407767in}{1.544118in}}%
\pgfusepath{clip}%
\pgfsetbuttcap%
\pgfsetroundjoin%
\pgfsetlinewidth{0.501875pt}%
\definecolor{currentstroke}{rgb}{0.269944,0.014625,0.341379}%
\pgfsetstrokecolor{currentstroke}%
\pgfsetdash{}{0pt}%
\pgfpathmoveto{\pgfqpoint{4.215491in}{2.117386in}}%
\pgfpathlineto{\pgfqpoint{4.215491in}{2.117386in}}%
\pgfusepath{stroke}%
\end{pgfscope}%
\begin{pgfscope}%
\pgfpathrectangle{\pgfqpoint{3.352233in}{1.400000in}}{\pgfqpoint{2.407767in}{1.544118in}}%
\pgfusepath{clip}%
\pgfsetbuttcap%
\pgfsetroundjoin%
\pgfsetlinewidth{0.501875pt}%
\definecolor{currentstroke}{rgb}{0.269944,0.014625,0.341379}%
\pgfsetstrokecolor{currentstroke}%
\pgfsetdash{}{0pt}%
\pgfpathmoveto{\pgfqpoint{4.215491in}{2.117386in}}%
\pgfpathlineto{\pgfqpoint{4.215491in}{2.117386in}}%
\pgfusepath{stroke}%
\end{pgfscope}%
\begin{pgfscope}%
\pgfpathrectangle{\pgfqpoint{3.352233in}{1.400000in}}{\pgfqpoint{2.407767in}{1.544118in}}%
\pgfusepath{clip}%
\pgfsetbuttcap%
\pgfsetroundjoin%
\pgfsetlinewidth{0.501875pt}%
\definecolor{currentstroke}{rgb}{0.269944,0.014625,0.341379}%
\pgfsetstrokecolor{currentstroke}%
\pgfsetdash{}{0pt}%
\pgfpathmoveto{\pgfqpoint{4.215491in}{2.117386in}}%
\pgfpathlineto{\pgfqpoint{4.215491in}{2.117386in}}%
\pgfusepath{stroke}%
\end{pgfscope}%
\begin{pgfscope}%
\pgfpathrectangle{\pgfqpoint{3.352233in}{1.400000in}}{\pgfqpoint{2.407767in}{1.544118in}}%
\pgfusepath{clip}%
\pgfsetbuttcap%
\pgfsetroundjoin%
\pgfsetlinewidth{0.501875pt}%
\definecolor{currentstroke}{rgb}{0.269944,0.014625,0.341379}%
\pgfsetstrokecolor{currentstroke}%
\pgfsetdash{}{0pt}%
\pgfpathmoveto{\pgfqpoint{4.215491in}{2.117386in}}%
\pgfpathlineto{\pgfqpoint{4.215491in}{2.117386in}}%
\pgfusepath{stroke}%
\end{pgfscope}%
\begin{pgfscope}%
\pgfpathrectangle{\pgfqpoint{3.352233in}{1.400000in}}{\pgfqpoint{2.407767in}{1.544118in}}%
\pgfusepath{clip}%
\pgfsetbuttcap%
\pgfsetroundjoin%
\pgfsetlinewidth{0.501875pt}%
\definecolor{currentstroke}{rgb}{0.269944,0.014625,0.341379}%
\pgfsetstrokecolor{currentstroke}%
\pgfsetdash{}{0pt}%
\pgfpathmoveto{\pgfqpoint{4.215491in}{2.117386in}}%
\pgfpathlineto{\pgfqpoint{4.215491in}{2.117386in}}%
\pgfusepath{stroke}%
\end{pgfscope}%
\begin{pgfscope}%
\pgfpathrectangle{\pgfqpoint{3.352233in}{1.400000in}}{\pgfqpoint{2.407767in}{1.544118in}}%
\pgfusepath{clip}%
\pgfsetbuttcap%
\pgfsetroundjoin%
\pgfsetlinewidth{0.501875pt}%
\definecolor{currentstroke}{rgb}{0.269944,0.014625,0.341379}%
\pgfsetstrokecolor{currentstroke}%
\pgfsetdash{}{0pt}%
\pgfpathmoveto{\pgfqpoint{4.215491in}{2.117386in}}%
\pgfpathlineto{\pgfqpoint{4.215491in}{2.117386in}}%
\pgfusepath{stroke}%
\end{pgfscope}%
\begin{pgfscope}%
\pgfpathrectangle{\pgfqpoint{3.352233in}{1.400000in}}{\pgfqpoint{2.407767in}{1.544118in}}%
\pgfusepath{clip}%
\pgfsetbuttcap%
\pgfsetroundjoin%
\pgfsetlinewidth{0.501875pt}%
\definecolor{currentstroke}{rgb}{0.269944,0.014625,0.341379}%
\pgfsetstrokecolor{currentstroke}%
\pgfsetdash{}{0pt}%
\pgfpathmoveto{\pgfqpoint{4.215491in}{2.117386in}}%
\pgfpathlineto{\pgfqpoint{4.215491in}{2.117386in}}%
\pgfusepath{stroke}%
\end{pgfscope}%
\begin{pgfscope}%
\pgfpathrectangle{\pgfqpoint{3.352233in}{1.400000in}}{\pgfqpoint{2.407767in}{1.544118in}}%
\pgfusepath{clip}%
\pgfsetbuttcap%
\pgfsetroundjoin%
\pgfsetlinewidth{0.501875pt}%
\definecolor{currentstroke}{rgb}{0.269944,0.014625,0.341379}%
\pgfsetstrokecolor{currentstroke}%
\pgfsetdash{}{0pt}%
\pgfpathmoveto{\pgfqpoint{4.215491in}{2.117386in}}%
\pgfpathlineto{\pgfqpoint{4.215491in}{2.117386in}}%
\pgfusepath{stroke}%
\end{pgfscope}%
\begin{pgfscope}%
\pgfpathrectangle{\pgfqpoint{3.352233in}{1.400000in}}{\pgfqpoint{2.407767in}{1.544118in}}%
\pgfusepath{clip}%
\pgfsetbuttcap%
\pgfsetroundjoin%
\pgfsetlinewidth{0.501875pt}%
\definecolor{currentstroke}{rgb}{0.269944,0.014625,0.341379}%
\pgfsetstrokecolor{currentstroke}%
\pgfsetdash{}{0pt}%
\pgfpathmoveto{\pgfqpoint{4.215491in}{2.117386in}}%
\pgfpathlineto{\pgfqpoint{4.215491in}{2.117386in}}%
\pgfusepath{stroke}%
\end{pgfscope}%
\begin{pgfscope}%
\pgfpathrectangle{\pgfqpoint{3.352233in}{1.400000in}}{\pgfqpoint{2.407767in}{1.544118in}}%
\pgfusepath{clip}%
\pgfsetbuttcap%
\pgfsetroundjoin%
\pgfsetlinewidth{0.501875pt}%
\definecolor{currentstroke}{rgb}{0.269944,0.014625,0.341379}%
\pgfsetstrokecolor{currentstroke}%
\pgfsetdash{}{0pt}%
\pgfpathmoveto{\pgfqpoint{4.215491in}{2.117386in}}%
\pgfpathlineto{\pgfqpoint{4.215491in}{2.117386in}}%
\pgfusepath{stroke}%
\end{pgfscope}%
\begin{pgfscope}%
\pgfpathrectangle{\pgfqpoint{3.352233in}{1.400000in}}{\pgfqpoint{2.407767in}{1.544118in}}%
\pgfusepath{clip}%
\pgfsetbuttcap%
\pgfsetroundjoin%
\pgfsetlinewidth{0.501875pt}%
\definecolor{currentstroke}{rgb}{0.269944,0.014625,0.341379}%
\pgfsetstrokecolor{currentstroke}%
\pgfsetdash{}{0pt}%
\pgfpathmoveto{\pgfqpoint{4.215491in}{2.117386in}}%
\pgfpathlineto{\pgfqpoint{4.215491in}{2.117386in}}%
\pgfusepath{stroke}%
\end{pgfscope}%
\begin{pgfscope}%
\pgfpathrectangle{\pgfqpoint{3.352233in}{1.400000in}}{\pgfqpoint{2.407767in}{1.544118in}}%
\pgfusepath{clip}%
\pgfsetbuttcap%
\pgfsetroundjoin%
\pgfsetlinewidth{0.501875pt}%
\definecolor{currentstroke}{rgb}{0.269944,0.014625,0.341379}%
\pgfsetstrokecolor{currentstroke}%
\pgfsetdash{}{0pt}%
\pgfpathmoveto{\pgfqpoint{4.215491in}{2.117386in}}%
\pgfpathlineto{\pgfqpoint{4.215491in}{2.117386in}}%
\pgfusepath{stroke}%
\end{pgfscope}%
\begin{pgfscope}%
\pgfpathrectangle{\pgfqpoint{3.352233in}{1.400000in}}{\pgfqpoint{2.407767in}{1.544118in}}%
\pgfusepath{clip}%
\pgfsetbuttcap%
\pgfsetroundjoin%
\pgfsetlinewidth{0.501875pt}%
\definecolor{currentstroke}{rgb}{0.269944,0.014625,0.341379}%
\pgfsetstrokecolor{currentstroke}%
\pgfsetdash{}{0pt}%
\pgfpathmoveto{\pgfqpoint{4.215491in}{2.117386in}}%
\pgfpathlineto{\pgfqpoint{4.215491in}{2.117386in}}%
\pgfusepath{stroke}%
\end{pgfscope}%
\begin{pgfscope}%
\pgfpathrectangle{\pgfqpoint{3.352233in}{1.400000in}}{\pgfqpoint{2.407767in}{1.544118in}}%
\pgfusepath{clip}%
\pgfsetbuttcap%
\pgfsetroundjoin%
\pgfsetlinewidth{0.501875pt}%
\definecolor{currentstroke}{rgb}{0.269944,0.014625,0.341379}%
\pgfsetstrokecolor{currentstroke}%
\pgfsetdash{}{0pt}%
\pgfpathmoveto{\pgfqpoint{4.215491in}{2.117386in}}%
\pgfpathlineto{\pgfqpoint{4.215491in}{2.117386in}}%
\pgfusepath{stroke}%
\end{pgfscope}%
\begin{pgfscope}%
\pgfpathrectangle{\pgfqpoint{3.352233in}{1.400000in}}{\pgfqpoint{2.407767in}{1.544118in}}%
\pgfusepath{clip}%
\pgfsetbuttcap%
\pgfsetroundjoin%
\pgfsetlinewidth{0.501875pt}%
\definecolor{currentstroke}{rgb}{0.269944,0.014625,0.341379}%
\pgfsetstrokecolor{currentstroke}%
\pgfsetdash{}{0pt}%
\pgfpathmoveto{\pgfqpoint{4.215491in}{2.117386in}}%
\pgfpathlineto{\pgfqpoint{4.215491in}{2.117386in}}%
\pgfusepath{stroke}%
\end{pgfscope}%
\begin{pgfscope}%
\pgfpathrectangle{\pgfqpoint{3.352233in}{1.400000in}}{\pgfqpoint{2.407767in}{1.544118in}}%
\pgfusepath{clip}%
\pgfsetbuttcap%
\pgfsetroundjoin%
\pgfsetlinewidth{0.501875pt}%
\definecolor{currentstroke}{rgb}{0.269944,0.014625,0.341379}%
\pgfsetstrokecolor{currentstroke}%
\pgfsetdash{}{0pt}%
\pgfpathmoveto{\pgfqpoint{4.215491in}{2.117386in}}%
\pgfpathlineto{\pgfqpoint{4.215491in}{2.117386in}}%
\pgfusepath{stroke}%
\end{pgfscope}%
\begin{pgfscope}%
\pgfpathrectangle{\pgfqpoint{3.352233in}{1.400000in}}{\pgfqpoint{2.407767in}{1.544118in}}%
\pgfusepath{clip}%
\pgfsetbuttcap%
\pgfsetroundjoin%
\pgfsetlinewidth{0.501875pt}%
\definecolor{currentstroke}{rgb}{0.269944,0.014625,0.341379}%
\pgfsetstrokecolor{currentstroke}%
\pgfsetdash{}{0pt}%
\pgfpathmoveto{\pgfqpoint{4.215491in}{2.117386in}}%
\pgfpathlineto{\pgfqpoint{4.215491in}{2.117386in}}%
\pgfusepath{stroke}%
\end{pgfscope}%
\begin{pgfscope}%
\pgfpathrectangle{\pgfqpoint{3.352233in}{1.400000in}}{\pgfqpoint{2.407767in}{1.544118in}}%
\pgfusepath{clip}%
\pgfsetbuttcap%
\pgfsetroundjoin%
\pgfsetlinewidth{0.501875pt}%
\definecolor{currentstroke}{rgb}{0.269944,0.014625,0.341379}%
\pgfsetstrokecolor{currentstroke}%
\pgfsetdash{}{0pt}%
\pgfpathmoveto{\pgfqpoint{4.215491in}{2.117386in}}%
\pgfpathlineto{\pgfqpoint{4.215491in}{2.117386in}}%
\pgfusepath{stroke}%
\end{pgfscope}%
\begin{pgfscope}%
\pgfpathrectangle{\pgfqpoint{3.352233in}{1.400000in}}{\pgfqpoint{2.407767in}{1.544118in}}%
\pgfusepath{clip}%
\pgfsetbuttcap%
\pgfsetroundjoin%
\pgfsetlinewidth{0.501875pt}%
\definecolor{currentstroke}{rgb}{0.269944,0.014625,0.341379}%
\pgfsetstrokecolor{currentstroke}%
\pgfsetdash{}{0pt}%
\pgfpathmoveto{\pgfqpoint{4.215491in}{2.117386in}}%
\pgfpathlineto{\pgfqpoint{4.215491in}{2.117386in}}%
\pgfusepath{stroke}%
\end{pgfscope}%
\begin{pgfscope}%
\pgfpathrectangle{\pgfqpoint{3.352233in}{1.400000in}}{\pgfqpoint{2.407767in}{1.544118in}}%
\pgfusepath{clip}%
\pgfsetbuttcap%
\pgfsetroundjoin%
\pgfsetlinewidth{0.501875pt}%
\definecolor{currentstroke}{rgb}{0.269944,0.014625,0.341379}%
\pgfsetstrokecolor{currentstroke}%
\pgfsetdash{}{0pt}%
\pgfpathmoveto{\pgfqpoint{4.215491in}{2.117386in}}%
\pgfpathlineto{\pgfqpoint{4.215491in}{2.117386in}}%
\pgfusepath{stroke}%
\end{pgfscope}%
\begin{pgfscope}%
\pgfpathrectangle{\pgfqpoint{3.352233in}{1.400000in}}{\pgfqpoint{2.407767in}{1.544118in}}%
\pgfusepath{clip}%
\pgfsetbuttcap%
\pgfsetroundjoin%
\pgfsetlinewidth{0.501875pt}%
\definecolor{currentstroke}{rgb}{0.269944,0.014625,0.341379}%
\pgfsetstrokecolor{currentstroke}%
\pgfsetdash{}{0pt}%
\pgfpathmoveto{\pgfqpoint{4.215491in}{2.117386in}}%
\pgfpathlineto{\pgfqpoint{4.215491in}{2.117386in}}%
\pgfusepath{stroke}%
\end{pgfscope}%
\begin{pgfscope}%
\pgfpathrectangle{\pgfqpoint{3.352233in}{1.400000in}}{\pgfqpoint{2.407767in}{1.544118in}}%
\pgfusepath{clip}%
\pgfsetbuttcap%
\pgfsetroundjoin%
\pgfsetlinewidth{0.501875pt}%
\definecolor{currentstroke}{rgb}{0.269944,0.014625,0.341379}%
\pgfsetstrokecolor{currentstroke}%
\pgfsetdash{}{0pt}%
\pgfpathmoveto{\pgfqpoint{4.215491in}{2.117386in}}%
\pgfpathlineto{\pgfqpoint{4.215491in}{2.117386in}}%
\pgfusepath{stroke}%
\end{pgfscope}%
\begin{pgfscope}%
\pgfpathrectangle{\pgfqpoint{3.352233in}{1.400000in}}{\pgfqpoint{2.407767in}{1.544118in}}%
\pgfusepath{clip}%
\pgfsetbuttcap%
\pgfsetroundjoin%
\pgfsetlinewidth{0.501875pt}%
\definecolor{currentstroke}{rgb}{0.269944,0.014625,0.341379}%
\pgfsetstrokecolor{currentstroke}%
\pgfsetdash{}{0pt}%
\pgfpathmoveto{\pgfqpoint{4.215491in}{2.117386in}}%
\pgfpathlineto{\pgfqpoint{4.215491in}{2.117386in}}%
\pgfusepath{stroke}%
\end{pgfscope}%
\begin{pgfscope}%
\pgfpathrectangle{\pgfqpoint{3.352233in}{1.400000in}}{\pgfqpoint{2.407767in}{1.544118in}}%
\pgfusepath{clip}%
\pgfsetbuttcap%
\pgfsetroundjoin%
\pgfsetlinewidth{0.501875pt}%
\definecolor{currentstroke}{rgb}{0.269944,0.014625,0.341379}%
\pgfsetstrokecolor{currentstroke}%
\pgfsetdash{}{0pt}%
\pgfpathmoveto{\pgfqpoint{4.215491in}{2.117386in}}%
\pgfpathlineto{\pgfqpoint{4.215491in}{2.117386in}}%
\pgfusepath{stroke}%
\end{pgfscope}%
\begin{pgfscope}%
\pgfpathrectangle{\pgfqpoint{3.352233in}{1.400000in}}{\pgfqpoint{2.407767in}{1.544118in}}%
\pgfusepath{clip}%
\pgfsetbuttcap%
\pgfsetroundjoin%
\pgfsetlinewidth{0.501875pt}%
\definecolor{currentstroke}{rgb}{0.269944,0.014625,0.341379}%
\pgfsetstrokecolor{currentstroke}%
\pgfsetdash{}{0pt}%
\pgfpathmoveto{\pgfqpoint{4.215491in}{2.117386in}}%
\pgfpathlineto{\pgfqpoint{4.215491in}{2.117386in}}%
\pgfusepath{stroke}%
\end{pgfscope}%
\begin{pgfscope}%
\pgfpathrectangle{\pgfqpoint{3.352233in}{1.400000in}}{\pgfqpoint{2.407767in}{1.544118in}}%
\pgfusepath{clip}%
\pgfsetbuttcap%
\pgfsetroundjoin%
\pgfsetlinewidth{0.501875pt}%
\definecolor{currentstroke}{rgb}{0.269944,0.014625,0.341379}%
\pgfsetstrokecolor{currentstroke}%
\pgfsetdash{}{0pt}%
\pgfpathmoveto{\pgfqpoint{4.215491in}{2.117386in}}%
\pgfpathlineto{\pgfqpoint{4.215491in}{2.117386in}}%
\pgfusepath{stroke}%
\end{pgfscope}%
\begin{pgfscope}%
\pgfpathrectangle{\pgfqpoint{3.352233in}{1.400000in}}{\pgfqpoint{2.407767in}{1.544118in}}%
\pgfusepath{clip}%
\pgfsetbuttcap%
\pgfsetroundjoin%
\pgfsetlinewidth{0.501875pt}%
\definecolor{currentstroke}{rgb}{0.269944,0.014625,0.341379}%
\pgfsetstrokecolor{currentstroke}%
\pgfsetdash{}{0pt}%
\pgfpathmoveto{\pgfqpoint{4.215491in}{2.117386in}}%
\pgfpathlineto{\pgfqpoint{4.215491in}{2.117386in}}%
\pgfusepath{stroke}%
\end{pgfscope}%
\begin{pgfscope}%
\pgfpathrectangle{\pgfqpoint{3.352233in}{1.400000in}}{\pgfqpoint{2.407767in}{1.544118in}}%
\pgfusepath{clip}%
\pgfsetbuttcap%
\pgfsetroundjoin%
\pgfsetlinewidth{0.501875pt}%
\definecolor{currentstroke}{rgb}{0.269944,0.014625,0.341379}%
\pgfsetstrokecolor{currentstroke}%
\pgfsetdash{}{0pt}%
\pgfpathmoveto{\pgfqpoint{4.215491in}{2.117386in}}%
\pgfpathlineto{\pgfqpoint{4.215491in}{2.117386in}}%
\pgfusepath{stroke}%
\end{pgfscope}%
\begin{pgfscope}%
\pgfpathrectangle{\pgfqpoint{3.352233in}{1.400000in}}{\pgfqpoint{2.407767in}{1.544118in}}%
\pgfusepath{clip}%
\pgfsetbuttcap%
\pgfsetroundjoin%
\pgfsetlinewidth{0.501875pt}%
\definecolor{currentstroke}{rgb}{0.269944,0.014625,0.341379}%
\pgfsetstrokecolor{currentstroke}%
\pgfsetdash{}{0pt}%
\pgfpathmoveto{\pgfqpoint{4.215491in}{2.117386in}}%
\pgfpathlineto{\pgfqpoint{4.215491in}{2.117386in}}%
\pgfusepath{stroke}%
\end{pgfscope}%
\begin{pgfscope}%
\pgfpathrectangle{\pgfqpoint{3.352233in}{1.400000in}}{\pgfqpoint{2.407767in}{1.544118in}}%
\pgfusepath{clip}%
\pgfsetbuttcap%
\pgfsetroundjoin%
\pgfsetlinewidth{0.501875pt}%
\definecolor{currentstroke}{rgb}{0.269944,0.014625,0.341379}%
\pgfsetstrokecolor{currentstroke}%
\pgfsetdash{}{0pt}%
\pgfpathmoveto{\pgfqpoint{4.215491in}{2.117386in}}%
\pgfpathlineto{\pgfqpoint{4.215491in}{2.117386in}}%
\pgfusepath{stroke}%
\end{pgfscope}%
\begin{pgfscope}%
\pgfpathrectangle{\pgfqpoint{3.352233in}{1.400000in}}{\pgfqpoint{2.407767in}{1.544118in}}%
\pgfusepath{clip}%
\pgfsetbuttcap%
\pgfsetroundjoin%
\pgfsetlinewidth{0.501875pt}%
\definecolor{currentstroke}{rgb}{0.269944,0.014625,0.341379}%
\pgfsetstrokecolor{currentstroke}%
\pgfsetdash{}{0pt}%
\pgfpathmoveto{\pgfqpoint{4.215491in}{2.117386in}}%
\pgfpathlineto{\pgfqpoint{4.215491in}{2.117386in}}%
\pgfusepath{stroke}%
\end{pgfscope}%
\begin{pgfscope}%
\pgfpathrectangle{\pgfqpoint{3.352233in}{1.400000in}}{\pgfqpoint{2.407767in}{1.544118in}}%
\pgfusepath{clip}%
\pgfsetbuttcap%
\pgfsetroundjoin%
\pgfsetlinewidth{0.501875pt}%
\definecolor{currentstroke}{rgb}{0.269944,0.014625,0.341379}%
\pgfsetstrokecolor{currentstroke}%
\pgfsetdash{}{0pt}%
\pgfpathmoveto{\pgfqpoint{4.215491in}{2.117386in}}%
\pgfpathlineto{\pgfqpoint{4.215491in}{2.117386in}}%
\pgfusepath{stroke}%
\end{pgfscope}%
\begin{pgfscope}%
\pgfpathrectangle{\pgfqpoint{3.352233in}{1.400000in}}{\pgfqpoint{2.407767in}{1.544118in}}%
\pgfusepath{clip}%
\pgfsetbuttcap%
\pgfsetroundjoin%
\pgfsetlinewidth{0.501875pt}%
\definecolor{currentstroke}{rgb}{0.269944,0.014625,0.341379}%
\pgfsetstrokecolor{currentstroke}%
\pgfsetdash{}{0pt}%
\pgfpathmoveto{\pgfqpoint{4.215491in}{2.117386in}}%
\pgfpathlineto{\pgfqpoint{4.215491in}{2.117386in}}%
\pgfusepath{stroke}%
\end{pgfscope}%
\begin{pgfscope}%
\pgfpathrectangle{\pgfqpoint{3.352233in}{1.400000in}}{\pgfqpoint{2.407767in}{1.544118in}}%
\pgfusepath{clip}%
\pgfsetbuttcap%
\pgfsetroundjoin%
\pgfsetlinewidth{0.501875pt}%
\definecolor{currentstroke}{rgb}{0.269944,0.014625,0.341379}%
\pgfsetstrokecolor{currentstroke}%
\pgfsetdash{}{0pt}%
\pgfpathmoveto{\pgfqpoint{4.215491in}{2.117386in}}%
\pgfpathlineto{\pgfqpoint{4.215491in}{2.117386in}}%
\pgfusepath{stroke}%
\end{pgfscope}%
\begin{pgfscope}%
\pgfpathrectangle{\pgfqpoint{3.352233in}{1.400000in}}{\pgfqpoint{2.407767in}{1.544118in}}%
\pgfusepath{clip}%
\pgfsetbuttcap%
\pgfsetroundjoin%
\pgfsetlinewidth{0.501875pt}%
\definecolor{currentstroke}{rgb}{0.269944,0.014625,0.341379}%
\pgfsetstrokecolor{currentstroke}%
\pgfsetdash{}{0pt}%
\pgfpathmoveto{\pgfqpoint{4.215491in}{2.117386in}}%
\pgfpathlineto{\pgfqpoint{4.215491in}{2.117386in}}%
\pgfusepath{stroke}%
\end{pgfscope}%
\begin{pgfscope}%
\pgfpathrectangle{\pgfqpoint{3.352233in}{1.400000in}}{\pgfqpoint{2.407767in}{1.544118in}}%
\pgfusepath{clip}%
\pgfsetbuttcap%
\pgfsetroundjoin%
\pgfsetlinewidth{0.501875pt}%
\definecolor{currentstroke}{rgb}{0.269944,0.014625,0.341379}%
\pgfsetstrokecolor{currentstroke}%
\pgfsetdash{}{0pt}%
\pgfpathmoveto{\pgfqpoint{4.215491in}{2.117386in}}%
\pgfpathlineto{\pgfqpoint{4.215491in}{2.117386in}}%
\pgfusepath{stroke}%
\end{pgfscope}%
\begin{pgfscope}%
\pgfpathrectangle{\pgfqpoint{3.352233in}{1.400000in}}{\pgfqpoint{2.407767in}{1.544118in}}%
\pgfusepath{clip}%
\pgfsetbuttcap%
\pgfsetroundjoin%
\pgfsetlinewidth{0.501875pt}%
\definecolor{currentstroke}{rgb}{0.269944,0.014625,0.341379}%
\pgfsetstrokecolor{currentstroke}%
\pgfsetdash{}{0pt}%
\pgfpathmoveto{\pgfqpoint{4.215491in}{2.117386in}}%
\pgfpathlineto{\pgfqpoint{4.215491in}{2.117386in}}%
\pgfusepath{stroke}%
\end{pgfscope}%
\begin{pgfscope}%
\pgfpathrectangle{\pgfqpoint{3.352233in}{1.400000in}}{\pgfqpoint{2.407767in}{1.544118in}}%
\pgfusepath{clip}%
\pgfsetbuttcap%
\pgfsetroundjoin%
\pgfsetlinewidth{0.501875pt}%
\definecolor{currentstroke}{rgb}{0.269944,0.014625,0.341379}%
\pgfsetstrokecolor{currentstroke}%
\pgfsetdash{}{0pt}%
\pgfpathmoveto{\pgfqpoint{4.215491in}{2.117386in}}%
\pgfpathlineto{\pgfqpoint{4.215491in}{2.117386in}}%
\pgfusepath{stroke}%
\end{pgfscope}%
\begin{pgfscope}%
\pgfpathrectangle{\pgfqpoint{3.352233in}{1.400000in}}{\pgfqpoint{2.407767in}{1.544118in}}%
\pgfusepath{clip}%
\pgfsetbuttcap%
\pgfsetroundjoin%
\pgfsetlinewidth{0.501875pt}%
\definecolor{currentstroke}{rgb}{0.269944,0.014625,0.341379}%
\pgfsetstrokecolor{currentstroke}%
\pgfsetdash{}{0pt}%
\pgfpathmoveto{\pgfqpoint{4.215491in}{2.117386in}}%
\pgfpathlineto{\pgfqpoint{4.215491in}{2.117386in}}%
\pgfusepath{stroke}%
\end{pgfscope}%
\begin{pgfscope}%
\pgfpathrectangle{\pgfqpoint{3.352233in}{1.400000in}}{\pgfqpoint{2.407767in}{1.544118in}}%
\pgfusepath{clip}%
\pgfsetbuttcap%
\pgfsetroundjoin%
\pgfsetlinewidth{0.501875pt}%
\definecolor{currentstroke}{rgb}{0.269944,0.014625,0.341379}%
\pgfsetstrokecolor{currentstroke}%
\pgfsetdash{}{0pt}%
\pgfpathmoveto{\pgfqpoint{4.215491in}{2.117386in}}%
\pgfpathlineto{\pgfqpoint{4.215491in}{2.117386in}}%
\pgfusepath{stroke}%
\end{pgfscope}%
\begin{pgfscope}%
\pgfpathrectangle{\pgfqpoint{3.352233in}{1.400000in}}{\pgfqpoint{2.407767in}{1.544118in}}%
\pgfusepath{clip}%
\pgfsetbuttcap%
\pgfsetroundjoin%
\pgfsetlinewidth{0.501875pt}%
\definecolor{currentstroke}{rgb}{0.269944,0.014625,0.341379}%
\pgfsetstrokecolor{currentstroke}%
\pgfsetdash{}{0pt}%
\pgfpathmoveto{\pgfqpoint{4.215491in}{2.117386in}}%
\pgfpathlineto{\pgfqpoint{4.215491in}{2.117386in}}%
\pgfusepath{stroke}%
\end{pgfscope}%
\begin{pgfscope}%
\pgfpathrectangle{\pgfqpoint{3.352233in}{1.400000in}}{\pgfqpoint{2.407767in}{1.544118in}}%
\pgfusepath{clip}%
\pgfsetbuttcap%
\pgfsetroundjoin%
\pgfsetlinewidth{0.501875pt}%
\definecolor{currentstroke}{rgb}{0.269944,0.014625,0.341379}%
\pgfsetstrokecolor{currentstroke}%
\pgfsetdash{}{0pt}%
\pgfpathmoveto{\pgfqpoint{4.215491in}{2.117386in}}%
\pgfpathlineto{\pgfqpoint{4.215491in}{2.117386in}}%
\pgfusepath{stroke}%
\end{pgfscope}%
\begin{pgfscope}%
\pgfpathrectangle{\pgfqpoint{3.352233in}{1.400000in}}{\pgfqpoint{2.407767in}{1.544118in}}%
\pgfusepath{clip}%
\pgfsetbuttcap%
\pgfsetroundjoin%
\pgfsetlinewidth{0.501875pt}%
\definecolor{currentstroke}{rgb}{0.269944,0.014625,0.341379}%
\pgfsetstrokecolor{currentstroke}%
\pgfsetdash{}{0pt}%
\pgfpathmoveto{\pgfqpoint{4.215491in}{2.117386in}}%
\pgfpathlineto{\pgfqpoint{4.215491in}{2.117386in}}%
\pgfusepath{stroke}%
\end{pgfscope}%
\begin{pgfscope}%
\pgfpathrectangle{\pgfqpoint{3.352233in}{1.400000in}}{\pgfqpoint{2.407767in}{1.544118in}}%
\pgfusepath{clip}%
\pgfsetbuttcap%
\pgfsetroundjoin%
\pgfsetlinewidth{0.501875pt}%
\definecolor{currentstroke}{rgb}{0.269944,0.014625,0.341379}%
\pgfsetstrokecolor{currentstroke}%
\pgfsetdash{}{0pt}%
\pgfpathmoveto{\pgfqpoint{4.215491in}{2.117386in}}%
\pgfpathlineto{\pgfqpoint{4.215491in}{2.117386in}}%
\pgfusepath{stroke}%
\end{pgfscope}%
\begin{pgfscope}%
\pgfpathrectangle{\pgfqpoint{3.352233in}{1.400000in}}{\pgfqpoint{2.407767in}{1.544118in}}%
\pgfusepath{clip}%
\pgfsetbuttcap%
\pgfsetroundjoin%
\pgfsetlinewidth{0.501875pt}%
\definecolor{currentstroke}{rgb}{0.269944,0.014625,0.341379}%
\pgfsetstrokecolor{currentstroke}%
\pgfsetdash{}{0pt}%
\pgfpathmoveto{\pgfqpoint{4.215491in}{2.117386in}}%
\pgfpathlineto{\pgfqpoint{4.215491in}{2.117386in}}%
\pgfusepath{stroke}%
\end{pgfscope}%
\begin{pgfscope}%
\pgfpathrectangle{\pgfqpoint{3.352233in}{1.400000in}}{\pgfqpoint{2.407767in}{1.544118in}}%
\pgfusepath{clip}%
\pgfsetbuttcap%
\pgfsetroundjoin%
\pgfsetlinewidth{0.501875pt}%
\definecolor{currentstroke}{rgb}{0.269944,0.014625,0.341379}%
\pgfsetstrokecolor{currentstroke}%
\pgfsetdash{}{0pt}%
\pgfpathmoveto{\pgfqpoint{4.215491in}{2.117386in}}%
\pgfpathlineto{\pgfqpoint{4.215491in}{2.117386in}}%
\pgfusepath{stroke}%
\end{pgfscope}%
\begin{pgfscope}%
\pgfpathrectangle{\pgfqpoint{3.352233in}{1.400000in}}{\pgfqpoint{2.407767in}{1.544118in}}%
\pgfusepath{clip}%
\pgfsetbuttcap%
\pgfsetroundjoin%
\pgfsetlinewidth{0.501875pt}%
\definecolor{currentstroke}{rgb}{0.269944,0.014625,0.341379}%
\pgfsetstrokecolor{currentstroke}%
\pgfsetdash{}{0pt}%
\pgfpathmoveto{\pgfqpoint{4.215491in}{2.117386in}}%
\pgfpathlineto{\pgfqpoint{4.215491in}{2.117386in}}%
\pgfusepath{stroke}%
\end{pgfscope}%
\begin{pgfscope}%
\pgfpathrectangle{\pgfqpoint{3.352233in}{1.400000in}}{\pgfqpoint{2.407767in}{1.544118in}}%
\pgfusepath{clip}%
\pgfsetbuttcap%
\pgfsetroundjoin%
\pgfsetlinewidth{0.501875pt}%
\definecolor{currentstroke}{rgb}{0.269944,0.014625,0.341379}%
\pgfsetstrokecolor{currentstroke}%
\pgfsetdash{}{0pt}%
\pgfpathmoveto{\pgfqpoint{4.215491in}{2.117386in}}%
\pgfpathlineto{\pgfqpoint{4.215491in}{2.117386in}}%
\pgfusepath{stroke}%
\end{pgfscope}%
\begin{pgfscope}%
\pgfpathrectangle{\pgfqpoint{3.352233in}{1.400000in}}{\pgfqpoint{2.407767in}{1.544118in}}%
\pgfusepath{clip}%
\pgfsetbuttcap%
\pgfsetroundjoin%
\pgfsetlinewidth{0.501875pt}%
\definecolor{currentstroke}{rgb}{0.269944,0.014625,0.341379}%
\pgfsetstrokecolor{currentstroke}%
\pgfsetdash{}{0pt}%
\pgfpathmoveto{\pgfqpoint{4.215491in}{2.117386in}}%
\pgfpathlineto{\pgfqpoint{4.215491in}{2.117386in}}%
\pgfusepath{stroke}%
\end{pgfscope}%
\begin{pgfscope}%
\pgfpathrectangle{\pgfqpoint{3.352233in}{1.400000in}}{\pgfqpoint{2.407767in}{1.544118in}}%
\pgfusepath{clip}%
\pgfsetbuttcap%
\pgfsetroundjoin%
\pgfsetlinewidth{0.501875pt}%
\definecolor{currentstroke}{rgb}{0.269944,0.014625,0.341379}%
\pgfsetstrokecolor{currentstroke}%
\pgfsetdash{}{0pt}%
\pgfpathmoveto{\pgfqpoint{4.215491in}{2.117386in}}%
\pgfpathlineto{\pgfqpoint{4.215491in}{2.117386in}}%
\pgfusepath{stroke}%
\end{pgfscope}%
\begin{pgfscope}%
\pgfpathrectangle{\pgfqpoint{3.352233in}{1.400000in}}{\pgfqpoint{2.407767in}{1.544118in}}%
\pgfusepath{clip}%
\pgfsetbuttcap%
\pgfsetroundjoin%
\pgfsetlinewidth{0.501875pt}%
\definecolor{currentstroke}{rgb}{0.269944,0.014625,0.341379}%
\pgfsetstrokecolor{currentstroke}%
\pgfsetdash{}{0pt}%
\pgfpathmoveto{\pgfqpoint{4.215491in}{2.117386in}}%
\pgfpathlineto{\pgfqpoint{4.215491in}{2.117386in}}%
\pgfusepath{stroke}%
\end{pgfscope}%
\begin{pgfscope}%
\pgfpathrectangle{\pgfqpoint{3.352233in}{1.400000in}}{\pgfqpoint{2.407767in}{1.544118in}}%
\pgfusepath{clip}%
\pgfsetbuttcap%
\pgfsetroundjoin%
\pgfsetlinewidth{0.501875pt}%
\definecolor{currentstroke}{rgb}{0.269944,0.014625,0.341379}%
\pgfsetstrokecolor{currentstroke}%
\pgfsetdash{}{0pt}%
\pgfpathmoveto{\pgfqpoint{4.215491in}{2.117386in}}%
\pgfpathlineto{\pgfqpoint{4.215491in}{2.117386in}}%
\pgfusepath{stroke}%
\end{pgfscope}%
\begin{pgfscope}%
\pgfpathrectangle{\pgfqpoint{3.352233in}{1.400000in}}{\pgfqpoint{2.407767in}{1.544118in}}%
\pgfusepath{clip}%
\pgfsetbuttcap%
\pgfsetroundjoin%
\pgfsetlinewidth{0.501875pt}%
\definecolor{currentstroke}{rgb}{0.269944,0.014625,0.341379}%
\pgfsetstrokecolor{currentstroke}%
\pgfsetdash{}{0pt}%
\pgfpathmoveto{\pgfqpoint{4.215491in}{2.117386in}}%
\pgfpathlineto{\pgfqpoint{4.215491in}{2.117386in}}%
\pgfusepath{stroke}%
\end{pgfscope}%
\begin{pgfscope}%
\pgfpathrectangle{\pgfqpoint{3.352233in}{1.400000in}}{\pgfqpoint{2.407767in}{1.544118in}}%
\pgfusepath{clip}%
\pgfsetbuttcap%
\pgfsetroundjoin%
\pgfsetlinewidth{0.501875pt}%
\definecolor{currentstroke}{rgb}{0.269944,0.014625,0.341379}%
\pgfsetstrokecolor{currentstroke}%
\pgfsetdash{}{0pt}%
\pgfpathmoveto{\pgfqpoint{4.215491in}{2.117386in}}%
\pgfpathlineto{\pgfqpoint{4.215491in}{2.117386in}}%
\pgfusepath{stroke}%
\end{pgfscope}%
\begin{pgfscope}%
\pgfpathrectangle{\pgfqpoint{3.352233in}{1.400000in}}{\pgfqpoint{2.407767in}{1.544118in}}%
\pgfusepath{clip}%
\pgfsetbuttcap%
\pgfsetroundjoin%
\pgfsetlinewidth{0.501875pt}%
\definecolor{currentstroke}{rgb}{0.269944,0.014625,0.341379}%
\pgfsetstrokecolor{currentstroke}%
\pgfsetdash{}{0pt}%
\pgfpathmoveto{\pgfqpoint{4.215491in}{2.117386in}}%
\pgfpathlineto{\pgfqpoint{4.215491in}{2.117386in}}%
\pgfusepath{stroke}%
\end{pgfscope}%
\begin{pgfscope}%
\pgfpathrectangle{\pgfqpoint{3.352233in}{1.400000in}}{\pgfqpoint{2.407767in}{1.544118in}}%
\pgfusepath{clip}%
\pgfsetbuttcap%
\pgfsetroundjoin%
\pgfsetlinewidth{0.501875pt}%
\definecolor{currentstroke}{rgb}{0.269944,0.014625,0.341379}%
\pgfsetstrokecolor{currentstroke}%
\pgfsetdash{}{0pt}%
\pgfpathmoveto{\pgfqpoint{4.215491in}{2.117386in}}%
\pgfpathlineto{\pgfqpoint{4.215491in}{2.117386in}}%
\pgfusepath{stroke}%
\end{pgfscope}%
\begin{pgfscope}%
\pgfpathrectangle{\pgfqpoint{3.352233in}{1.400000in}}{\pgfqpoint{2.407767in}{1.544118in}}%
\pgfusepath{clip}%
\pgfsetbuttcap%
\pgfsetroundjoin%
\pgfsetlinewidth{0.501875pt}%
\definecolor{currentstroke}{rgb}{0.269944,0.014625,0.341379}%
\pgfsetstrokecolor{currentstroke}%
\pgfsetdash{}{0pt}%
\pgfpathmoveto{\pgfqpoint{4.215491in}{2.117386in}}%
\pgfpathlineto{\pgfqpoint{4.215491in}{2.117386in}}%
\pgfusepath{stroke}%
\end{pgfscope}%
\begin{pgfscope}%
\pgfpathrectangle{\pgfqpoint{3.352233in}{1.400000in}}{\pgfqpoint{2.407767in}{1.544118in}}%
\pgfusepath{clip}%
\pgfsetbuttcap%
\pgfsetroundjoin%
\pgfsetlinewidth{0.501875pt}%
\definecolor{currentstroke}{rgb}{0.269944,0.014625,0.341379}%
\pgfsetstrokecolor{currentstroke}%
\pgfsetdash{}{0pt}%
\pgfpathmoveto{\pgfqpoint{4.215491in}{2.117386in}}%
\pgfpathlineto{\pgfqpoint{4.215491in}{2.117386in}}%
\pgfusepath{stroke}%
\end{pgfscope}%
\begin{pgfscope}%
\pgfpathrectangle{\pgfqpoint{3.352233in}{1.400000in}}{\pgfqpoint{2.407767in}{1.544118in}}%
\pgfusepath{clip}%
\pgfsetbuttcap%
\pgfsetroundjoin%
\pgfsetlinewidth{0.501875pt}%
\definecolor{currentstroke}{rgb}{0.269944,0.014625,0.341379}%
\pgfsetstrokecolor{currentstroke}%
\pgfsetdash{}{0pt}%
\pgfpathmoveto{\pgfqpoint{4.215491in}{2.117386in}}%
\pgfpathlineto{\pgfqpoint{4.215491in}{2.117386in}}%
\pgfusepath{stroke}%
\end{pgfscope}%
\begin{pgfscope}%
\pgfpathrectangle{\pgfqpoint{3.352233in}{1.400000in}}{\pgfqpoint{2.407767in}{1.544118in}}%
\pgfusepath{clip}%
\pgfsetbuttcap%
\pgfsetroundjoin%
\pgfsetlinewidth{0.501875pt}%
\definecolor{currentstroke}{rgb}{0.269944,0.014625,0.341379}%
\pgfsetstrokecolor{currentstroke}%
\pgfsetdash{}{0pt}%
\pgfpathmoveto{\pgfqpoint{4.215491in}{2.117386in}}%
\pgfpathlineto{\pgfqpoint{4.215491in}{2.117386in}}%
\pgfusepath{stroke}%
\end{pgfscope}%
\begin{pgfscope}%
\pgfpathrectangle{\pgfqpoint{3.352233in}{1.400000in}}{\pgfqpoint{2.407767in}{1.544118in}}%
\pgfusepath{clip}%
\pgfsetbuttcap%
\pgfsetroundjoin%
\pgfsetlinewidth{0.501875pt}%
\definecolor{currentstroke}{rgb}{0.269944,0.014625,0.341379}%
\pgfsetstrokecolor{currentstroke}%
\pgfsetdash{}{0pt}%
\pgfpathmoveto{\pgfqpoint{4.215491in}{2.117386in}}%
\pgfpathlineto{\pgfqpoint{4.215491in}{2.117386in}}%
\pgfusepath{stroke}%
\end{pgfscope}%
\begin{pgfscope}%
\pgfpathrectangle{\pgfqpoint{3.352233in}{1.400000in}}{\pgfqpoint{2.407767in}{1.544118in}}%
\pgfusepath{clip}%
\pgfsetbuttcap%
\pgfsetroundjoin%
\pgfsetlinewidth{0.501875pt}%
\definecolor{currentstroke}{rgb}{0.269944,0.014625,0.341379}%
\pgfsetstrokecolor{currentstroke}%
\pgfsetdash{}{0pt}%
\pgfpathmoveto{\pgfqpoint{4.215491in}{2.117386in}}%
\pgfpathlineto{\pgfqpoint{4.215491in}{2.117386in}}%
\pgfusepath{stroke}%
\end{pgfscope}%
\begin{pgfscope}%
\pgfpathrectangle{\pgfqpoint{3.352233in}{1.400000in}}{\pgfqpoint{2.407767in}{1.544118in}}%
\pgfusepath{clip}%
\pgfsetbuttcap%
\pgfsetroundjoin%
\pgfsetlinewidth{0.501875pt}%
\definecolor{currentstroke}{rgb}{0.269944,0.014625,0.341379}%
\pgfsetstrokecolor{currentstroke}%
\pgfsetdash{}{0pt}%
\pgfpathmoveto{\pgfqpoint{4.215491in}{2.117386in}}%
\pgfpathlineto{\pgfqpoint{4.215491in}{2.117386in}}%
\pgfusepath{stroke}%
\end{pgfscope}%
\begin{pgfscope}%
\pgfpathrectangle{\pgfqpoint{3.352233in}{1.400000in}}{\pgfqpoint{2.407767in}{1.544118in}}%
\pgfusepath{clip}%
\pgfsetbuttcap%
\pgfsetroundjoin%
\pgfsetlinewidth{0.501875pt}%
\definecolor{currentstroke}{rgb}{0.269944,0.014625,0.341379}%
\pgfsetstrokecolor{currentstroke}%
\pgfsetdash{}{0pt}%
\pgfpathmoveto{\pgfqpoint{4.215491in}{2.117386in}}%
\pgfpathlineto{\pgfqpoint{4.215491in}{2.117386in}}%
\pgfusepath{stroke}%
\end{pgfscope}%
\begin{pgfscope}%
\pgfpathrectangle{\pgfqpoint{3.352233in}{1.400000in}}{\pgfqpoint{2.407767in}{1.544118in}}%
\pgfusepath{clip}%
\pgfsetbuttcap%
\pgfsetroundjoin%
\pgfsetlinewidth{0.501875pt}%
\definecolor{currentstroke}{rgb}{0.269944,0.014625,0.341379}%
\pgfsetstrokecolor{currentstroke}%
\pgfsetdash{}{0pt}%
\pgfpathmoveto{\pgfqpoint{4.215491in}{2.117386in}}%
\pgfpathlineto{\pgfqpoint{4.215491in}{2.117386in}}%
\pgfusepath{stroke}%
\end{pgfscope}%
\begin{pgfscope}%
\pgfpathrectangle{\pgfqpoint{3.352233in}{1.400000in}}{\pgfqpoint{2.407767in}{1.544118in}}%
\pgfusepath{clip}%
\pgfsetbuttcap%
\pgfsetroundjoin%
\pgfsetlinewidth{0.501875pt}%
\definecolor{currentstroke}{rgb}{0.269944,0.014625,0.341379}%
\pgfsetstrokecolor{currentstroke}%
\pgfsetdash{}{0pt}%
\pgfpathmoveto{\pgfqpoint{4.215491in}{2.117386in}}%
\pgfpathlineto{\pgfqpoint{4.215491in}{2.117386in}}%
\pgfusepath{stroke}%
\end{pgfscope}%
\begin{pgfscope}%
\pgfpathrectangle{\pgfqpoint{3.352233in}{1.400000in}}{\pgfqpoint{2.407767in}{1.544118in}}%
\pgfusepath{clip}%
\pgfsetbuttcap%
\pgfsetroundjoin%
\pgfsetlinewidth{0.501875pt}%
\definecolor{currentstroke}{rgb}{0.269944,0.014625,0.341379}%
\pgfsetstrokecolor{currentstroke}%
\pgfsetdash{}{0pt}%
\pgfpathmoveto{\pgfqpoint{4.215491in}{2.117386in}}%
\pgfpathlineto{\pgfqpoint{4.215491in}{2.117386in}}%
\pgfusepath{stroke}%
\end{pgfscope}%
\begin{pgfscope}%
\pgfpathrectangle{\pgfqpoint{3.352233in}{1.400000in}}{\pgfqpoint{2.407767in}{1.544118in}}%
\pgfusepath{clip}%
\pgfsetbuttcap%
\pgfsetroundjoin%
\pgfsetlinewidth{0.501875pt}%
\definecolor{currentstroke}{rgb}{0.269944,0.014625,0.341379}%
\pgfsetstrokecolor{currentstroke}%
\pgfsetdash{}{0pt}%
\pgfpathmoveto{\pgfqpoint{4.215491in}{2.117386in}}%
\pgfpathlineto{\pgfqpoint{4.215491in}{2.117386in}}%
\pgfusepath{stroke}%
\end{pgfscope}%
\begin{pgfscope}%
\pgfpathrectangle{\pgfqpoint{3.352233in}{1.400000in}}{\pgfqpoint{2.407767in}{1.544118in}}%
\pgfusepath{clip}%
\pgfsetbuttcap%
\pgfsetroundjoin%
\pgfsetlinewidth{0.501875pt}%
\definecolor{currentstroke}{rgb}{0.269944,0.014625,0.341379}%
\pgfsetstrokecolor{currentstroke}%
\pgfsetdash{}{0pt}%
\pgfpathmoveto{\pgfqpoint{4.215491in}{2.117386in}}%
\pgfpathlineto{\pgfqpoint{4.215491in}{2.117386in}}%
\pgfusepath{stroke}%
\end{pgfscope}%
\begin{pgfscope}%
\pgfpathrectangle{\pgfqpoint{3.352233in}{1.400000in}}{\pgfqpoint{2.407767in}{1.544118in}}%
\pgfusepath{clip}%
\pgfsetbuttcap%
\pgfsetroundjoin%
\pgfsetlinewidth{0.501875pt}%
\definecolor{currentstroke}{rgb}{0.269944,0.014625,0.341379}%
\pgfsetstrokecolor{currentstroke}%
\pgfsetdash{}{0pt}%
\pgfpathmoveto{\pgfqpoint{4.215491in}{2.117386in}}%
\pgfpathlineto{\pgfqpoint{4.215491in}{2.117386in}}%
\pgfusepath{stroke}%
\end{pgfscope}%
\begin{pgfscope}%
\pgfpathrectangle{\pgfqpoint{3.352233in}{1.400000in}}{\pgfqpoint{2.407767in}{1.544118in}}%
\pgfusepath{clip}%
\pgfsetbuttcap%
\pgfsetroundjoin%
\pgfsetlinewidth{0.501875pt}%
\definecolor{currentstroke}{rgb}{0.269944,0.014625,0.341379}%
\pgfsetstrokecolor{currentstroke}%
\pgfsetdash{}{0pt}%
\pgfpathmoveto{\pgfqpoint{4.215491in}{2.117386in}}%
\pgfpathlineto{\pgfqpoint{4.215491in}{2.117386in}}%
\pgfusepath{stroke}%
\end{pgfscope}%
\begin{pgfscope}%
\pgfpathrectangle{\pgfqpoint{3.352233in}{1.400000in}}{\pgfqpoint{2.407767in}{1.544118in}}%
\pgfusepath{clip}%
\pgfsetbuttcap%
\pgfsetroundjoin%
\pgfsetlinewidth{0.501875pt}%
\definecolor{currentstroke}{rgb}{0.269944,0.014625,0.341379}%
\pgfsetstrokecolor{currentstroke}%
\pgfsetdash{}{0pt}%
\pgfpathmoveto{\pgfqpoint{4.215491in}{2.117386in}}%
\pgfpathlineto{\pgfqpoint{4.215491in}{2.117386in}}%
\pgfusepath{stroke}%
\end{pgfscope}%
\begin{pgfscope}%
\pgfpathrectangle{\pgfqpoint{3.352233in}{1.400000in}}{\pgfqpoint{2.407767in}{1.544118in}}%
\pgfusepath{clip}%
\pgfsetbuttcap%
\pgfsetroundjoin%
\pgfsetlinewidth{0.501875pt}%
\definecolor{currentstroke}{rgb}{0.273809,0.031497,0.358853}%
\pgfsetstrokecolor{currentstroke}%
\pgfsetdash{}{0pt}%
\pgfpathmoveto{\pgfqpoint{4.713898in}{1.637359in}}%
\pgfpathlineto{\pgfqpoint{4.661754in}{1.643078in}}%
\pgfusepath{stroke}%
\end{pgfscope}%
\begin{pgfscope}%
\pgfpathrectangle{\pgfqpoint{3.352233in}{1.400000in}}{\pgfqpoint{2.407767in}{1.544118in}}%
\pgfusepath{clip}%
\pgfsetbuttcap%
\pgfsetroundjoin%
\pgfsetlinewidth{0.501875pt}%
\definecolor{currentstroke}{rgb}{0.274952,0.037752,0.364543}%
\pgfsetstrokecolor{currentstroke}%
\pgfsetdash{}{0pt}%
\pgfpathmoveto{\pgfqpoint{4.661754in}{1.643078in}}%
\pgfpathlineto{\pgfqpoint{4.610297in}{1.650867in}}%
\pgfusepath{stroke}%
\end{pgfscope}%
\begin{pgfscope}%
\pgfpathrectangle{\pgfqpoint{3.352233in}{1.400000in}}{\pgfqpoint{2.407767in}{1.544118in}}%
\pgfusepath{clip}%
\pgfsetbuttcap%
\pgfsetroundjoin%
\pgfsetlinewidth{0.501875pt}%
\definecolor{currentstroke}{rgb}{0.273809,0.031497,0.358853}%
\pgfsetstrokecolor{currentstroke}%
\pgfsetdash{}{0pt}%
\pgfpathmoveto{\pgfqpoint{4.610297in}{1.650867in}}%
\pgfpathlineto{\pgfqpoint{4.561176in}{1.661616in}}%
\pgfusepath{stroke}%
\end{pgfscope}%
\begin{pgfscope}%
\pgfpathrectangle{\pgfqpoint{3.352233in}{1.400000in}}{\pgfqpoint{2.407767in}{1.544118in}}%
\pgfusepath{clip}%
\pgfsetbuttcap%
\pgfsetroundjoin%
\pgfsetlinewidth{0.501875pt}%
\definecolor{currentstroke}{rgb}{0.273809,0.031497,0.358853}%
\pgfsetstrokecolor{currentstroke}%
\pgfsetdash{}{0pt}%
\pgfpathmoveto{\pgfqpoint{4.561176in}{1.661616in}}%
\pgfpathlineto{\pgfqpoint{4.561176in}{1.661616in}}%
\pgfusepath{stroke}%
\end{pgfscope}%
\begin{pgfscope}%
\pgfpathrectangle{\pgfqpoint{3.352233in}{1.400000in}}{\pgfqpoint{2.407767in}{1.544118in}}%
\pgfusepath{clip}%
\pgfsetbuttcap%
\pgfsetroundjoin%
\pgfsetlinewidth{0.501875pt}%
\definecolor{currentstroke}{rgb}{0.273809,0.031497,0.358853}%
\pgfsetstrokecolor{currentstroke}%
\pgfsetdash{}{0pt}%
\pgfpathmoveto{\pgfqpoint{4.561176in}{1.661616in}}%
\pgfpathlineto{\pgfqpoint{4.561176in}{1.661616in}}%
\pgfusepath{stroke}%
\end{pgfscope}%
\begin{pgfscope}%
\pgfpathrectangle{\pgfqpoint{3.352233in}{1.400000in}}{\pgfqpoint{2.407767in}{1.544118in}}%
\pgfusepath{clip}%
\pgfsetbuttcap%
\pgfsetroundjoin%
\pgfsetlinewidth{0.501875pt}%
\definecolor{currentstroke}{rgb}{0.273809,0.031497,0.358853}%
\pgfsetstrokecolor{currentstroke}%
\pgfsetdash{}{0pt}%
\pgfpathmoveto{\pgfqpoint{4.561176in}{1.661616in}}%
\pgfpathlineto{\pgfqpoint{4.561176in}{1.661616in}}%
\pgfusepath{stroke}%
\end{pgfscope}%
\begin{pgfscope}%
\pgfpathrectangle{\pgfqpoint{3.352233in}{1.400000in}}{\pgfqpoint{2.407767in}{1.544118in}}%
\pgfusepath{clip}%
\pgfsetbuttcap%
\pgfsetroundjoin%
\pgfsetlinewidth{0.501875pt}%
\definecolor{currentstroke}{rgb}{0.273809,0.031497,0.358853}%
\pgfsetstrokecolor{currentstroke}%
\pgfsetdash{}{0pt}%
\pgfpathmoveto{\pgfqpoint{4.561176in}{1.661616in}}%
\pgfpathlineto{\pgfqpoint{4.547426in}{1.670135in}}%
\pgfusepath{stroke}%
\end{pgfscope}%
\begin{pgfscope}%
\pgfpathrectangle{\pgfqpoint{3.352233in}{1.400000in}}{\pgfqpoint{2.407767in}{1.544118in}}%
\pgfusepath{clip}%
\pgfsetbuttcap%
\pgfsetroundjoin%
\pgfsetlinewidth{0.501875pt}%
\definecolor{currentstroke}{rgb}{0.273809,0.031497,0.358853}%
\pgfsetstrokecolor{currentstroke}%
\pgfsetdash{}{0pt}%
\pgfpathmoveto{\pgfqpoint{4.547426in}{1.670135in}}%
\pgfpathlineto{\pgfqpoint{4.534679in}{1.678154in}}%
\pgfusepath{stroke}%
\end{pgfscope}%
\begin{pgfscope}%
\pgfpathrectangle{\pgfqpoint{3.352233in}{1.400000in}}{\pgfqpoint{2.407767in}{1.544118in}}%
\pgfusepath{clip}%
\pgfsetbuttcap%
\pgfsetroundjoin%
\pgfsetlinewidth{0.501875pt}%
\definecolor{currentstroke}{rgb}{0.273809,0.031497,0.358853}%
\pgfsetstrokecolor{currentstroke}%
\pgfsetdash{}{0pt}%
\pgfpathmoveto{\pgfqpoint{4.534679in}{1.678154in}}%
\pgfpathlineto{\pgfqpoint{4.504268in}{1.693349in}}%
\pgfusepath{stroke}%
\end{pgfscope}%
\begin{pgfscope}%
\pgfpathrectangle{\pgfqpoint{3.352233in}{1.400000in}}{\pgfqpoint{2.407767in}{1.544118in}}%
\pgfusepath{clip}%
\pgfsetbuttcap%
\pgfsetroundjoin%
\pgfsetlinewidth{0.501875pt}%
\definecolor{currentstroke}{rgb}{0.273809,0.031497,0.358853}%
\pgfsetstrokecolor{currentstroke}%
\pgfsetdash{}{0pt}%
\pgfpathmoveto{\pgfqpoint{4.504268in}{1.693349in}}%
\pgfpathlineto{\pgfqpoint{4.504268in}{1.693349in}}%
\pgfusepath{stroke}%
\end{pgfscope}%
\begin{pgfscope}%
\pgfpathrectangle{\pgfqpoint{3.352233in}{1.400000in}}{\pgfqpoint{2.407767in}{1.544118in}}%
\pgfusepath{clip}%
\pgfsetbuttcap%
\pgfsetroundjoin%
\pgfsetlinewidth{0.501875pt}%
\definecolor{currentstroke}{rgb}{0.273809,0.031497,0.358853}%
\pgfsetstrokecolor{currentstroke}%
\pgfsetdash{}{0pt}%
\pgfpathmoveto{\pgfqpoint{4.504268in}{1.693349in}}%
\pgfpathlineto{\pgfqpoint{4.477006in}{1.709730in}}%
\pgfusepath{stroke}%
\end{pgfscope}%
\begin{pgfscope}%
\pgfpathrectangle{\pgfqpoint{3.352233in}{1.400000in}}{\pgfqpoint{2.407767in}{1.544118in}}%
\pgfusepath{clip}%
\pgfsetbuttcap%
\pgfsetroundjoin%
\pgfsetlinewidth{0.501875pt}%
\definecolor{currentstroke}{rgb}{0.274952,0.037752,0.364543}%
\pgfsetstrokecolor{currentstroke}%
\pgfsetdash{}{0pt}%
\pgfpathmoveto{\pgfqpoint{4.477006in}{1.709730in}}%
\pgfpathlineto{\pgfqpoint{4.453800in}{1.728422in}}%
\pgfusepath{stroke}%
\end{pgfscope}%
\begin{pgfscope}%
\pgfpathrectangle{\pgfqpoint{3.352233in}{1.400000in}}{\pgfqpoint{2.407767in}{1.544118in}}%
\pgfusepath{clip}%
\pgfsetbuttcap%
\pgfsetroundjoin%
\pgfsetlinewidth{0.501875pt}%
\definecolor{currentstroke}{rgb}{0.272594,0.025563,0.353093}%
\pgfsetstrokecolor{currentstroke}%
\pgfsetdash{}{0pt}%
\pgfpathmoveto{\pgfqpoint{4.453800in}{1.728422in}}%
\pgfpathlineto{\pgfqpoint{4.453800in}{1.728422in}}%
\pgfusepath{stroke}%
\end{pgfscope}%
\begin{pgfscope}%
\pgfpathrectangle{\pgfqpoint{3.352233in}{1.400000in}}{\pgfqpoint{2.407767in}{1.544118in}}%
\pgfusepath{clip}%
\pgfsetbuttcap%
\pgfsetroundjoin%
\pgfsetlinewidth{0.501875pt}%
\definecolor{currentstroke}{rgb}{0.272594,0.025563,0.353093}%
\pgfsetstrokecolor{currentstroke}%
\pgfsetdash{}{0pt}%
\pgfpathmoveto{\pgfqpoint{4.453800in}{1.728422in}}%
\pgfpathlineto{\pgfqpoint{4.440309in}{1.743242in}}%
\pgfusepath{stroke}%
\end{pgfscope}%
\begin{pgfscope}%
\pgfpathrectangle{\pgfqpoint{3.352233in}{1.400000in}}{\pgfqpoint{2.407767in}{1.544118in}}%
\pgfusepath{clip}%
\pgfsetbuttcap%
\pgfsetroundjoin%
\pgfsetlinewidth{0.501875pt}%
\definecolor{currentstroke}{rgb}{0.274952,0.037752,0.364543}%
\pgfsetstrokecolor{currentstroke}%
\pgfsetdash{}{0pt}%
\pgfpathmoveto{\pgfqpoint{4.440309in}{1.743242in}}%
\pgfpathlineto{\pgfqpoint{4.440309in}{1.743242in}}%
\pgfusepath{stroke}%
\end{pgfscope}%
\begin{pgfscope}%
\pgfpathrectangle{\pgfqpoint{3.352233in}{1.400000in}}{\pgfqpoint{2.407767in}{1.544118in}}%
\pgfusepath{clip}%
\pgfsetbuttcap%
\pgfsetroundjoin%
\pgfsetlinewidth{0.501875pt}%
\definecolor{currentstroke}{rgb}{0.274952,0.037752,0.364543}%
\pgfsetstrokecolor{currentstroke}%
\pgfsetdash{}{0pt}%
\pgfpathmoveto{\pgfqpoint{4.440309in}{1.743242in}}%
\pgfpathlineto{\pgfqpoint{4.440309in}{1.743242in}}%
\pgfusepath{stroke}%
\end{pgfscope}%
\begin{pgfscope}%
\pgfpathrectangle{\pgfqpoint{3.352233in}{1.400000in}}{\pgfqpoint{2.407767in}{1.544118in}}%
\pgfusepath{clip}%
\pgfsetbuttcap%
\pgfsetroundjoin%
\pgfsetlinewidth{0.501875pt}%
\definecolor{currentstroke}{rgb}{0.274952,0.037752,0.364543}%
\pgfsetstrokecolor{currentstroke}%
\pgfsetdash{}{0pt}%
\pgfpathmoveto{\pgfqpoint{4.440309in}{1.743242in}}%
\pgfpathlineto{\pgfqpoint{4.438236in}{1.751912in}}%
\pgfusepath{stroke}%
\end{pgfscope}%
\begin{pgfscope}%
\pgfpathrectangle{\pgfqpoint{3.352233in}{1.400000in}}{\pgfqpoint{2.407767in}{1.544118in}}%
\pgfusepath{clip}%
\pgfsetbuttcap%
\pgfsetroundjoin%
\pgfsetlinewidth{0.501875pt}%
\definecolor{currentstroke}{rgb}{0.273809,0.031497,0.358853}%
\pgfsetstrokecolor{currentstroke}%
\pgfsetdash{}{0pt}%
\pgfpathmoveto{\pgfqpoint{4.438236in}{1.751912in}}%
\pgfpathlineto{\pgfqpoint{4.444123in}{1.757317in}}%
\pgfusepath{stroke}%
\end{pgfscope}%
\begin{pgfscope}%
\pgfpathrectangle{\pgfqpoint{3.352233in}{1.400000in}}{\pgfqpoint{2.407767in}{1.544118in}}%
\pgfusepath{clip}%
\pgfsetbuttcap%
\pgfsetroundjoin%
\pgfsetlinewidth{0.501875pt}%
\definecolor{currentstroke}{rgb}{0.272594,0.025563,0.353093}%
\pgfsetstrokecolor{currentstroke}%
\pgfsetdash{}{0pt}%
\pgfpathmoveto{\pgfqpoint{4.444123in}{1.757317in}}%
\pgfpathlineto{\pgfqpoint{4.445164in}{1.764467in}}%
\pgfusepath{stroke}%
\end{pgfscope}%
\begin{pgfscope}%
\pgfpathrectangle{\pgfqpoint{3.352233in}{1.400000in}}{\pgfqpoint{2.407767in}{1.544118in}}%
\pgfusepath{clip}%
\pgfsetbuttcap%
\pgfsetroundjoin%
\pgfsetlinewidth{0.501875pt}%
\definecolor{currentstroke}{rgb}{0.273809,0.031497,0.358853}%
\pgfsetstrokecolor{currentstroke}%
\pgfsetdash{}{0pt}%
\pgfpathmoveto{\pgfqpoint{4.445164in}{1.764467in}}%
\pgfpathlineto{\pgfqpoint{4.437581in}{1.773212in}}%
\pgfusepath{stroke}%
\end{pgfscope}%
\begin{pgfscope}%
\pgfpathrectangle{\pgfqpoint{3.352233in}{1.400000in}}{\pgfqpoint{2.407767in}{1.544118in}}%
\pgfusepath{clip}%
\pgfsetbuttcap%
\pgfsetroundjoin%
\pgfsetlinewidth{0.501875pt}%
\definecolor{currentstroke}{rgb}{0.278791,0.062145,0.386592}%
\pgfsetstrokecolor{currentstroke}%
\pgfsetdash{}{0pt}%
\pgfpathmoveto{\pgfqpoint{4.437581in}{1.773212in}}%
\pgfpathlineto{\pgfqpoint{4.424766in}{1.783552in}}%
\pgfusepath{stroke}%
\end{pgfscope}%
\begin{pgfscope}%
\pgfpathrectangle{\pgfqpoint{3.352233in}{1.400000in}}{\pgfqpoint{2.407767in}{1.544118in}}%
\pgfusepath{clip}%
\pgfsetbuttcap%
\pgfsetroundjoin%
\pgfsetlinewidth{0.501875pt}%
\definecolor{currentstroke}{rgb}{0.277941,0.056324,0.381191}%
\pgfsetstrokecolor{currentstroke}%
\pgfsetdash{}{0pt}%
\pgfpathmoveto{\pgfqpoint{4.424766in}{1.783552in}}%
\pgfpathlineto{\pgfqpoint{4.424766in}{1.783552in}}%
\pgfusepath{stroke}%
\end{pgfscope}%
\begin{pgfscope}%
\pgfpathrectangle{\pgfqpoint{3.352233in}{1.400000in}}{\pgfqpoint{2.407767in}{1.544118in}}%
\pgfusepath{clip}%
\pgfsetbuttcap%
\pgfsetroundjoin%
\pgfsetlinewidth{0.501875pt}%
\definecolor{currentstroke}{rgb}{0.268510,0.009605,0.335427}%
\pgfsetstrokecolor{currentstroke}%
\pgfsetdash{}{0pt}%
\pgfpathmoveto{\pgfqpoint{5.195777in}{1.640618in}}%
\pgfpathlineto{\pgfqpoint{5.142853in}{1.640397in}}%
\pgfusepath{stroke}%
\end{pgfscope}%
\begin{pgfscope}%
\pgfpathrectangle{\pgfqpoint{3.352233in}{1.400000in}}{\pgfqpoint{2.407767in}{1.544118in}}%
\pgfusepath{clip}%
\pgfsetbuttcap%
\pgfsetroundjoin%
\pgfsetlinewidth{0.501875pt}%
\definecolor{currentstroke}{rgb}{0.271305,0.019942,0.347269}%
\pgfsetstrokecolor{currentstroke}%
\pgfsetdash{}{0pt}%
\pgfpathmoveto{\pgfqpoint{5.142853in}{1.640397in}}%
\pgfpathlineto{\pgfqpoint{5.089899in}{1.640979in}}%
\pgfusepath{stroke}%
\end{pgfscope}%
\begin{pgfscope}%
\pgfpathrectangle{\pgfqpoint{3.352233in}{1.400000in}}{\pgfqpoint{2.407767in}{1.544118in}}%
\pgfusepath{clip}%
\pgfsetbuttcap%
\pgfsetroundjoin%
\pgfsetlinewidth{0.501875pt}%
\definecolor{currentstroke}{rgb}{0.272594,0.025563,0.353093}%
\pgfsetstrokecolor{currentstroke}%
\pgfsetdash{}{0pt}%
\pgfpathmoveto{\pgfqpoint{5.089899in}{1.640979in}}%
\pgfpathlineto{\pgfqpoint{5.036941in}{1.641584in}}%
\pgfusepath{stroke}%
\end{pgfscope}%
\begin{pgfscope}%
\pgfpathrectangle{\pgfqpoint{3.352233in}{1.400000in}}{\pgfqpoint{2.407767in}{1.544118in}}%
\pgfusepath{clip}%
\pgfsetbuttcap%
\pgfsetroundjoin%
\pgfsetlinewidth{0.501875pt}%
\definecolor{currentstroke}{rgb}{0.273809,0.031497,0.358853}%
\pgfsetstrokecolor{currentstroke}%
\pgfsetdash{}{0pt}%
\pgfpathmoveto{\pgfqpoint{5.036941in}{1.641584in}}%
\pgfpathlineto{\pgfqpoint{4.983986in}{1.642408in}}%
\pgfusepath{stroke}%
\end{pgfscope}%
\begin{pgfscope}%
\pgfpathrectangle{\pgfqpoint{3.352233in}{1.400000in}}{\pgfqpoint{2.407767in}{1.544118in}}%
\pgfusepath{clip}%
\pgfsetbuttcap%
\pgfsetroundjoin%
\pgfsetlinewidth{0.501875pt}%
\definecolor{currentstroke}{rgb}{0.276022,0.044167,0.370164}%
\pgfsetstrokecolor{currentstroke}%
\pgfsetdash{}{0pt}%
\pgfpathmoveto{\pgfqpoint{4.983986in}{1.642408in}}%
\pgfpathlineto{\pgfqpoint{4.931033in}{1.643221in}}%
\pgfusepath{stroke}%
\end{pgfscope}%
\begin{pgfscope}%
\pgfpathrectangle{\pgfqpoint{3.352233in}{1.400000in}}{\pgfqpoint{2.407767in}{1.544118in}}%
\pgfusepath{clip}%
\pgfsetbuttcap%
\pgfsetroundjoin%
\pgfsetlinewidth{0.501875pt}%
\definecolor{currentstroke}{rgb}{0.276022,0.044167,0.370164}%
\pgfsetstrokecolor{currentstroke}%
\pgfsetdash{}{0pt}%
\pgfpathmoveto{\pgfqpoint{4.931033in}{1.643221in}}%
\pgfpathlineto{\pgfqpoint{4.878177in}{1.644722in}}%
\pgfusepath{stroke}%
\end{pgfscope}%
\begin{pgfscope}%
\pgfpathrectangle{\pgfqpoint{3.352233in}{1.400000in}}{\pgfqpoint{2.407767in}{1.544118in}}%
\pgfusepath{clip}%
\pgfsetbuttcap%
\pgfsetroundjoin%
\pgfsetlinewidth{0.501875pt}%
\definecolor{currentstroke}{rgb}{0.274952,0.037752,0.364543}%
\pgfsetstrokecolor{currentstroke}%
\pgfsetdash{}{0pt}%
\pgfpathmoveto{\pgfqpoint{4.878177in}{1.644722in}}%
\pgfpathlineto{\pgfqpoint{4.825360in}{1.647000in}}%
\pgfusepath{stroke}%
\end{pgfscope}%
\begin{pgfscope}%
\pgfpathrectangle{\pgfqpoint{3.352233in}{1.400000in}}{\pgfqpoint{2.407767in}{1.544118in}}%
\pgfusepath{clip}%
\pgfsetbuttcap%
\pgfsetroundjoin%
\pgfsetlinewidth{0.501875pt}%
\definecolor{currentstroke}{rgb}{0.277018,0.050344,0.375715}%
\pgfsetstrokecolor{currentstroke}%
\pgfsetdash{}{0pt}%
\pgfpathmoveto{\pgfqpoint{4.825360in}{1.647000in}}%
\pgfpathlineto{\pgfqpoint{4.772837in}{1.650867in}}%
\pgfusepath{stroke}%
\end{pgfscope}%
\begin{pgfscope}%
\pgfpathrectangle{\pgfqpoint{3.352233in}{1.400000in}}{\pgfqpoint{2.407767in}{1.544118in}}%
\pgfusepath{clip}%
\pgfsetbuttcap%
\pgfsetroundjoin%
\pgfsetlinewidth{0.501875pt}%
\definecolor{currentstroke}{rgb}{0.267004,0.004874,0.329415}%
\pgfsetstrokecolor{currentstroke}%
\pgfsetdash{}{0pt}%
\pgfpathmoveto{\pgfqpoint{5.206279in}{2.693251in}}%
\pgfpathlineto{\pgfqpoint{5.153459in}{2.692130in}}%
\pgfusepath{stroke}%
\end{pgfscope}%
\begin{pgfscope}%
\pgfpathrectangle{\pgfqpoint{3.352233in}{1.400000in}}{\pgfqpoint{2.407767in}{1.544118in}}%
\pgfusepath{clip}%
\pgfsetbuttcap%
\pgfsetroundjoin%
\pgfsetlinewidth{0.501875pt}%
\definecolor{currentstroke}{rgb}{0.272594,0.025563,0.353093}%
\pgfsetstrokecolor{currentstroke}%
\pgfsetdash{}{0pt}%
\pgfpathmoveto{\pgfqpoint{5.153459in}{2.692130in}}%
\pgfpathlineto{\pgfqpoint{5.100486in}{2.691892in}}%
\pgfusepath{stroke}%
\end{pgfscope}%
\begin{pgfscope}%
\pgfpathrectangle{\pgfqpoint{3.352233in}{1.400000in}}{\pgfqpoint{2.407767in}{1.544118in}}%
\pgfusepath{clip}%
\pgfsetbuttcap%
\pgfsetroundjoin%
\pgfsetlinewidth{0.501875pt}%
\definecolor{currentstroke}{rgb}{0.271305,0.019942,0.347269}%
\pgfsetstrokecolor{currentstroke}%
\pgfsetdash{}{0pt}%
\pgfpathmoveto{\pgfqpoint{5.100486in}{2.691892in}}%
\pgfpathlineto{\pgfqpoint{5.047520in}{2.691410in}}%
\pgfusepath{stroke}%
\end{pgfscope}%
\begin{pgfscope}%
\pgfpathrectangle{\pgfqpoint{3.352233in}{1.400000in}}{\pgfqpoint{2.407767in}{1.544118in}}%
\pgfusepath{clip}%
\pgfsetbuttcap%
\pgfsetroundjoin%
\pgfsetlinewidth{0.501875pt}%
\definecolor{currentstroke}{rgb}{0.273809,0.031497,0.358853}%
\pgfsetstrokecolor{currentstroke}%
\pgfsetdash{}{0pt}%
\pgfpathmoveto{\pgfqpoint{5.047520in}{2.691410in}}%
\pgfpathlineto{\pgfqpoint{4.994564in}{2.690535in}}%
\pgfusepath{stroke}%
\end{pgfscope}%
\begin{pgfscope}%
\pgfpathrectangle{\pgfqpoint{3.352233in}{1.400000in}}{\pgfqpoint{2.407767in}{1.544118in}}%
\pgfusepath{clip}%
\pgfsetbuttcap%
\pgfsetroundjoin%
\pgfsetlinewidth{0.501875pt}%
\definecolor{currentstroke}{rgb}{0.274952,0.037752,0.364543}%
\pgfsetstrokecolor{currentstroke}%
\pgfsetdash{}{0pt}%
\pgfpathmoveto{\pgfqpoint{4.994564in}{2.690535in}}%
\pgfpathlineto{\pgfqpoint{4.941612in}{2.689630in}}%
\pgfusepath{stroke}%
\end{pgfscope}%
\begin{pgfscope}%
\pgfpathrectangle{\pgfqpoint{3.352233in}{1.400000in}}{\pgfqpoint{2.407767in}{1.544118in}}%
\pgfusepath{clip}%
\pgfsetbuttcap%
\pgfsetroundjoin%
\pgfsetlinewidth{0.501875pt}%
\definecolor{currentstroke}{rgb}{0.274952,0.037752,0.364543}%
\pgfsetstrokecolor{currentstroke}%
\pgfsetdash{}{0pt}%
\pgfpathmoveto{\pgfqpoint{4.941612in}{2.689630in}}%
\pgfpathlineto{\pgfqpoint{4.888702in}{2.688188in}}%
\pgfusepath{stroke}%
\end{pgfscope}%
\begin{pgfscope}%
\pgfpathrectangle{\pgfqpoint{3.352233in}{1.400000in}}{\pgfqpoint{2.407767in}{1.544118in}}%
\pgfusepath{clip}%
\pgfsetbuttcap%
\pgfsetroundjoin%
\pgfsetlinewidth{0.501875pt}%
\definecolor{currentstroke}{rgb}{0.229739,0.322361,0.545706}%
\pgfsetstrokecolor{currentstroke}%
\pgfsetdash{}{0pt}%
\pgfpathmoveto{\pgfqpoint{3.743414in}{2.172059in}}%
\pgfpathlineto{\pgfqpoint{3.796373in}{2.171248in}}%
\pgfusepath{stroke}%
\end{pgfscope}%
\begin{pgfscope}%
\pgfpathrectangle{\pgfqpoint{3.352233in}{1.400000in}}{\pgfqpoint{2.407767in}{1.544118in}}%
\pgfusepath{clip}%
\pgfsetbuttcap%
\pgfsetroundjoin%
\pgfsetlinewidth{0.501875pt}%
\definecolor{currentstroke}{rgb}{0.235526,0.309527,0.542944}%
\pgfsetstrokecolor{currentstroke}%
\pgfsetdash{}{0pt}%
\pgfpathmoveto{\pgfqpoint{3.796373in}{2.171248in}}%
\pgfpathlineto{\pgfqpoint{3.849326in}{2.170827in}}%
\pgfusepath{stroke}%
\end{pgfscope}%
\begin{pgfscope}%
\pgfpathrectangle{\pgfqpoint{3.352233in}{1.400000in}}{\pgfqpoint{2.407767in}{1.544118in}}%
\pgfusepath{clip}%
\pgfsetbuttcap%
\pgfsetroundjoin%
\pgfsetlinewidth{0.501875pt}%
\definecolor{currentstroke}{rgb}{0.246811,0.283237,0.535941}%
\pgfsetstrokecolor{currentstroke}%
\pgfsetdash{}{0pt}%
\pgfpathmoveto{\pgfqpoint{3.849326in}{2.170827in}}%
\pgfpathlineto{\pgfqpoint{3.902278in}{2.170654in}}%
\pgfusepath{stroke}%
\end{pgfscope}%
\begin{pgfscope}%
\pgfpathrectangle{\pgfqpoint{3.352233in}{1.400000in}}{\pgfqpoint{2.407767in}{1.544118in}}%
\pgfusepath{clip}%
\pgfsetbuttcap%
\pgfsetroundjoin%
\pgfsetlinewidth{0.501875pt}%
\definecolor{currentstroke}{rgb}{0.262138,0.242286,0.520837}%
\pgfsetstrokecolor{currentstroke}%
\pgfsetdash{}{0pt}%
\pgfpathmoveto{\pgfqpoint{3.902278in}{2.170654in}}%
\pgfpathlineto{\pgfqpoint{3.955227in}{2.170287in}}%
\pgfusepath{stroke}%
\end{pgfscope}%
\begin{pgfscope}%
\pgfpathrectangle{\pgfqpoint{3.352233in}{1.400000in}}{\pgfqpoint{2.407767in}{1.544118in}}%
\pgfusepath{clip}%
\pgfsetbuttcap%
\pgfsetroundjoin%
\pgfsetlinewidth{0.501875pt}%
\definecolor{currentstroke}{rgb}{0.276194,0.190074,0.493001}%
\pgfsetstrokecolor{currentstroke}%
\pgfsetdash{}{0pt}%
\pgfpathmoveto{\pgfqpoint{3.955227in}{2.170287in}}%
\pgfpathlineto{\pgfqpoint{4.008157in}{2.169363in}}%
\pgfusepath{stroke}%
\end{pgfscope}%
\begin{pgfscope}%
\pgfpathrectangle{\pgfqpoint{3.352233in}{1.400000in}}{\pgfqpoint{2.407767in}{1.544118in}}%
\pgfusepath{clip}%
\pgfsetbuttcap%
\pgfsetroundjoin%
\pgfsetlinewidth{0.501875pt}%
\definecolor{currentstroke}{rgb}{0.283091,0.110553,0.431554}%
\pgfsetstrokecolor{currentstroke}%
\pgfsetdash{}{0pt}%
\pgfpathmoveto{\pgfqpoint{4.008157in}{2.169363in}}%
\pgfpathlineto{\pgfqpoint{4.061071in}{2.167850in}}%
\pgfusepath{stroke}%
\end{pgfscope}%
\begin{pgfscope}%
\pgfpathrectangle{\pgfqpoint{3.352233in}{1.400000in}}{\pgfqpoint{2.407767in}{1.544118in}}%
\pgfusepath{clip}%
\pgfsetbuttcap%
\pgfsetroundjoin%
\pgfsetlinewidth{0.501875pt}%
\definecolor{currentstroke}{rgb}{0.280894,0.078907,0.402329}%
\pgfsetstrokecolor{currentstroke}%
\pgfsetdash{}{0pt}%
\pgfpathmoveto{\pgfqpoint{4.061071in}{2.167850in}}%
\pgfpathlineto{\pgfqpoint{4.113942in}{2.166118in}}%
\pgfusepath{stroke}%
\end{pgfscope}%
\begin{pgfscope}%
\pgfpathrectangle{\pgfqpoint{3.352233in}{1.400000in}}{\pgfqpoint{2.407767in}{1.544118in}}%
\pgfusepath{clip}%
\pgfsetbuttcap%
\pgfsetroundjoin%
\pgfsetlinewidth{0.501875pt}%
\definecolor{currentstroke}{rgb}{0.277018,0.050344,0.375715}%
\pgfsetstrokecolor{currentstroke}%
\pgfsetdash{}{0pt}%
\pgfpathmoveto{\pgfqpoint{4.113942in}{2.166118in}}%
\pgfpathlineto{\pgfqpoint{4.166575in}{2.163059in}}%
\pgfusepath{stroke}%
\end{pgfscope}%
\begin{pgfscope}%
\pgfpathrectangle{\pgfqpoint{3.352233in}{1.400000in}}{\pgfqpoint{2.407767in}{1.544118in}}%
\pgfusepath{clip}%
\pgfsetbuttcap%
\pgfsetroundjoin%
\pgfsetlinewidth{0.501875pt}%
\definecolor{currentstroke}{rgb}{0.272594,0.025563,0.353093}%
\pgfsetstrokecolor{currentstroke}%
\pgfsetdash{}{0pt}%
\pgfpathmoveto{\pgfqpoint{4.166575in}{2.163059in}}%
\pgfpathlineto{\pgfqpoint{4.166575in}{2.163059in}}%
\pgfusepath{stroke}%
\end{pgfscope}%
\begin{pgfscope}%
\pgfpathrectangle{\pgfqpoint{3.352233in}{1.400000in}}{\pgfqpoint{2.407767in}{1.544118in}}%
\pgfusepath{clip}%
\pgfsetbuttcap%
\pgfsetroundjoin%
\pgfsetlinewidth{0.501875pt}%
\definecolor{currentstroke}{rgb}{0.272594,0.025563,0.353093}%
\pgfsetstrokecolor{currentstroke}%
\pgfsetdash{}{0pt}%
\pgfpathmoveto{\pgfqpoint{4.166575in}{2.163059in}}%
\pgfpathlineto{\pgfqpoint{4.166575in}{2.163059in}}%
\pgfusepath{stroke}%
\end{pgfscope}%
\begin{pgfscope}%
\pgfpathrectangle{\pgfqpoint{3.352233in}{1.400000in}}{\pgfqpoint{2.407767in}{1.544118in}}%
\pgfusepath{clip}%
\pgfsetbuttcap%
\pgfsetroundjoin%
\pgfsetlinewidth{0.501875pt}%
\definecolor{currentstroke}{rgb}{0.272594,0.025563,0.353093}%
\pgfsetstrokecolor{currentstroke}%
\pgfsetdash{}{0pt}%
\pgfpathmoveto{\pgfqpoint{4.166575in}{2.163059in}}%
\pgfpathlineto{\pgfqpoint{4.166575in}{2.163059in}}%
\pgfusepath{stroke}%
\end{pgfscope}%
\begin{pgfscope}%
\pgfpathrectangle{\pgfqpoint{3.352233in}{1.400000in}}{\pgfqpoint{2.407767in}{1.544118in}}%
\pgfusepath{clip}%
\pgfsetbuttcap%
\pgfsetroundjoin%
\pgfsetlinewidth{0.501875pt}%
\definecolor{currentstroke}{rgb}{0.272594,0.025563,0.353093}%
\pgfsetstrokecolor{currentstroke}%
\pgfsetdash{}{0pt}%
\pgfpathmoveto{\pgfqpoint{4.166575in}{2.163059in}}%
\pgfpathlineto{\pgfqpoint{4.186460in}{2.162153in}}%
\pgfusepath{stroke}%
\end{pgfscope}%
\begin{pgfscope}%
\pgfpathrectangle{\pgfqpoint{3.352233in}{1.400000in}}{\pgfqpoint{2.407767in}{1.544118in}}%
\pgfusepath{clip}%
\pgfsetbuttcap%
\pgfsetroundjoin%
\pgfsetlinewidth{0.501875pt}%
\definecolor{currentstroke}{rgb}{0.271305,0.019942,0.347269}%
\pgfsetstrokecolor{currentstroke}%
\pgfsetdash{}{0pt}%
\pgfpathmoveto{\pgfqpoint{4.186460in}{2.162153in}}%
\pgfpathlineto{\pgfqpoint{4.214056in}{2.162635in}}%
\pgfusepath{stroke}%
\end{pgfscope}%
\begin{pgfscope}%
\pgfpathrectangle{\pgfqpoint{3.352233in}{1.400000in}}{\pgfqpoint{2.407767in}{1.544118in}}%
\pgfusepath{clip}%
\pgfsetbuttcap%
\pgfsetroundjoin%
\pgfsetlinewidth{0.501875pt}%
\definecolor{currentstroke}{rgb}{0.271305,0.019942,0.347269}%
\pgfsetstrokecolor{currentstroke}%
\pgfsetdash{}{0pt}%
\pgfpathmoveto{\pgfqpoint{4.214056in}{2.162635in}}%
\pgfpathlineto{\pgfqpoint{4.214056in}{2.162635in}}%
\pgfusepath{stroke}%
\end{pgfscope}%
\begin{pgfscope}%
\pgfpathrectangle{\pgfqpoint{3.352233in}{1.400000in}}{\pgfqpoint{2.407767in}{1.544118in}}%
\pgfusepath{clip}%
\pgfsetbuttcap%
\pgfsetroundjoin%
\pgfsetlinewidth{0.501875pt}%
\definecolor{currentstroke}{rgb}{0.271305,0.019942,0.347269}%
\pgfsetstrokecolor{currentstroke}%
\pgfsetdash{}{0pt}%
\pgfpathmoveto{\pgfqpoint{4.214056in}{2.162635in}}%
\pgfpathlineto{\pgfqpoint{4.225316in}{2.161858in}}%
\pgfusepath{stroke}%
\end{pgfscope}%
\begin{pgfscope}%
\pgfpathrectangle{\pgfqpoint{3.352233in}{1.400000in}}{\pgfqpoint{2.407767in}{1.544118in}}%
\pgfusepath{clip}%
\pgfsetbuttcap%
\pgfsetroundjoin%
\pgfsetlinewidth{0.501875pt}%
\definecolor{currentstroke}{rgb}{0.271305,0.019942,0.347269}%
\pgfsetstrokecolor{currentstroke}%
\pgfsetdash{}{0pt}%
\pgfpathmoveto{\pgfqpoint{4.225316in}{2.161858in}}%
\pgfpathlineto{\pgfqpoint{4.225316in}{2.161858in}}%
\pgfusepath{stroke}%
\end{pgfscope}%
\begin{pgfscope}%
\pgfpathrectangle{\pgfqpoint{3.352233in}{1.400000in}}{\pgfqpoint{2.407767in}{1.544118in}}%
\pgfusepath{clip}%
\pgfsetbuttcap%
\pgfsetroundjoin%
\pgfsetlinewidth{0.501875pt}%
\definecolor{currentstroke}{rgb}{0.271305,0.019942,0.347269}%
\pgfsetstrokecolor{currentstroke}%
\pgfsetdash{}{0pt}%
\pgfpathmoveto{\pgfqpoint{4.225316in}{2.161858in}}%
\pgfpathlineto{\pgfqpoint{4.231192in}{2.160169in}}%
\pgfusepath{stroke}%
\end{pgfscope}%
\begin{pgfscope}%
\pgfpathrectangle{\pgfqpoint{3.352233in}{1.400000in}}{\pgfqpoint{2.407767in}{1.544118in}}%
\pgfusepath{clip}%
\pgfsetbuttcap%
\pgfsetroundjoin%
\pgfsetlinewidth{0.501875pt}%
\definecolor{currentstroke}{rgb}{0.269944,0.014625,0.341379}%
\pgfsetstrokecolor{currentstroke}%
\pgfsetdash{}{0pt}%
\pgfpathmoveto{\pgfqpoint{4.231192in}{2.160169in}}%
\pgfpathlineto{\pgfqpoint{4.235982in}{2.157808in}}%
\pgfusepath{stroke}%
\end{pgfscope}%
\begin{pgfscope}%
\pgfpathrectangle{\pgfqpoint{3.352233in}{1.400000in}}{\pgfqpoint{2.407767in}{1.544118in}}%
\pgfusepath{clip}%
\pgfsetbuttcap%
\pgfsetroundjoin%
\pgfsetlinewidth{0.501875pt}%
\definecolor{currentstroke}{rgb}{0.269944,0.014625,0.341379}%
\pgfsetstrokecolor{currentstroke}%
\pgfsetdash{}{0pt}%
\pgfpathmoveto{\pgfqpoint{4.235982in}{2.157808in}}%
\pgfpathlineto{\pgfqpoint{4.235982in}{2.157808in}}%
\pgfusepath{stroke}%
\end{pgfscope}%
\begin{pgfscope}%
\pgfpathrectangle{\pgfqpoint{3.352233in}{1.400000in}}{\pgfqpoint{2.407767in}{1.544118in}}%
\pgfusepath{clip}%
\pgfsetbuttcap%
\pgfsetroundjoin%
\pgfsetlinewidth{0.501875pt}%
\definecolor{currentstroke}{rgb}{0.269944,0.014625,0.341379}%
\pgfsetstrokecolor{currentstroke}%
\pgfsetdash{}{0pt}%
\pgfpathmoveto{\pgfqpoint{4.235982in}{2.157808in}}%
\pgfpathlineto{\pgfqpoint{4.237424in}{2.154795in}}%
\pgfusepath{stroke}%
\end{pgfscope}%
\begin{pgfscope}%
\pgfpathrectangle{\pgfqpoint{3.352233in}{1.400000in}}{\pgfqpoint{2.407767in}{1.544118in}}%
\pgfusepath{clip}%
\pgfsetbuttcap%
\pgfsetroundjoin%
\pgfsetlinewidth{0.501875pt}%
\definecolor{currentstroke}{rgb}{0.269944,0.014625,0.341379}%
\pgfsetstrokecolor{currentstroke}%
\pgfsetdash{}{0pt}%
\pgfpathmoveto{\pgfqpoint{4.237424in}{2.154795in}}%
\pgfpathlineto{\pgfqpoint{4.235480in}{2.153190in}}%
\pgfusepath{stroke}%
\end{pgfscope}%
\begin{pgfscope}%
\pgfpathrectangle{\pgfqpoint{3.352233in}{1.400000in}}{\pgfqpoint{2.407767in}{1.544118in}}%
\pgfusepath{clip}%
\pgfsetbuttcap%
\pgfsetroundjoin%
\pgfsetlinewidth{0.501875pt}%
\definecolor{currentstroke}{rgb}{0.269944,0.014625,0.341379}%
\pgfsetstrokecolor{currentstroke}%
\pgfsetdash{}{0pt}%
\pgfpathmoveto{\pgfqpoint{4.235480in}{2.153190in}}%
\pgfpathlineto{\pgfqpoint{4.233104in}{2.151861in}}%
\pgfusepath{stroke}%
\end{pgfscope}%
\begin{pgfscope}%
\pgfpathrectangle{\pgfqpoint{3.352233in}{1.400000in}}{\pgfqpoint{2.407767in}{1.544118in}}%
\pgfusepath{clip}%
\pgfsetbuttcap%
\pgfsetroundjoin%
\pgfsetlinewidth{0.501875pt}%
\definecolor{currentstroke}{rgb}{0.269944,0.014625,0.341379}%
\pgfsetstrokecolor{currentstroke}%
\pgfsetdash{}{0pt}%
\pgfpathmoveto{\pgfqpoint{4.233104in}{2.151861in}}%
\pgfpathlineto{\pgfqpoint{4.231304in}{2.149894in}}%
\pgfusepath{stroke}%
\end{pgfscope}%
\begin{pgfscope}%
\pgfpathrectangle{\pgfqpoint{3.352233in}{1.400000in}}{\pgfqpoint{2.407767in}{1.544118in}}%
\pgfusepath{clip}%
\pgfsetbuttcap%
\pgfsetroundjoin%
\pgfsetlinewidth{0.501875pt}%
\definecolor{currentstroke}{rgb}{0.269944,0.014625,0.341379}%
\pgfsetstrokecolor{currentstroke}%
\pgfsetdash{}{0pt}%
\pgfpathmoveto{\pgfqpoint{4.231304in}{2.149894in}}%
\pgfpathlineto{\pgfqpoint{4.231304in}{2.149894in}}%
\pgfusepath{stroke}%
\end{pgfscope}%
\begin{pgfscope}%
\pgfpathrectangle{\pgfqpoint{3.352233in}{1.400000in}}{\pgfqpoint{2.407767in}{1.544118in}}%
\pgfusepath{clip}%
\pgfsetbuttcap%
\pgfsetroundjoin%
\pgfsetlinewidth{0.501875pt}%
\definecolor{currentstroke}{rgb}{0.269944,0.014625,0.341379}%
\pgfsetstrokecolor{currentstroke}%
\pgfsetdash{}{0pt}%
\pgfpathmoveto{\pgfqpoint{4.231304in}{2.149894in}}%
\pgfpathlineto{\pgfqpoint{4.232434in}{2.148546in}}%
\pgfusepath{stroke}%
\end{pgfscope}%
\begin{pgfscope}%
\pgfpathrectangle{\pgfqpoint{3.352233in}{1.400000in}}{\pgfqpoint{2.407767in}{1.544118in}}%
\pgfusepath{clip}%
\pgfsetbuttcap%
\pgfsetroundjoin%
\pgfsetlinewidth{0.501875pt}%
\definecolor{currentstroke}{rgb}{0.269944,0.014625,0.341379}%
\pgfsetstrokecolor{currentstroke}%
\pgfsetdash{}{0pt}%
\pgfpathmoveto{\pgfqpoint{4.232434in}{2.148546in}}%
\pgfpathlineto{\pgfqpoint{4.232434in}{2.148546in}}%
\pgfusepath{stroke}%
\end{pgfscope}%
\begin{pgfscope}%
\pgfpathrectangle{\pgfqpoint{3.352233in}{1.400000in}}{\pgfqpoint{2.407767in}{1.544118in}}%
\pgfusepath{clip}%
\pgfsetbuttcap%
\pgfsetroundjoin%
\pgfsetlinewidth{0.501875pt}%
\definecolor{currentstroke}{rgb}{0.269944,0.014625,0.341379}%
\pgfsetstrokecolor{currentstroke}%
\pgfsetdash{}{0pt}%
\pgfpathmoveto{\pgfqpoint{4.232434in}{2.148546in}}%
\pgfpathlineto{\pgfqpoint{4.231348in}{2.147874in}}%
\pgfusepath{stroke}%
\end{pgfscope}%
\begin{pgfscope}%
\pgfpathrectangle{\pgfqpoint{3.352233in}{1.400000in}}{\pgfqpoint{2.407767in}{1.544118in}}%
\pgfusepath{clip}%
\pgfsetbuttcap%
\pgfsetroundjoin%
\pgfsetlinewidth{0.501875pt}%
\definecolor{currentstroke}{rgb}{0.269944,0.014625,0.341379}%
\pgfsetstrokecolor{currentstroke}%
\pgfsetdash{}{0pt}%
\pgfpathmoveto{\pgfqpoint{4.231348in}{2.147874in}}%
\pgfpathlineto{\pgfqpoint{4.231270in}{2.146982in}}%
\pgfusepath{stroke}%
\end{pgfscope}%
\begin{pgfscope}%
\pgfpathrectangle{\pgfqpoint{3.352233in}{1.400000in}}{\pgfqpoint{2.407767in}{1.544118in}}%
\pgfusepath{clip}%
\pgfsetbuttcap%
\pgfsetroundjoin%
\pgfsetlinewidth{0.501875pt}%
\definecolor{currentstroke}{rgb}{0.269944,0.014625,0.341379}%
\pgfsetstrokecolor{currentstroke}%
\pgfsetdash{}{0pt}%
\pgfpathmoveto{\pgfqpoint{4.231270in}{2.146982in}}%
\pgfpathlineto{\pgfqpoint{4.232117in}{2.145840in}}%
\pgfusepath{stroke}%
\end{pgfscope}%
\begin{pgfscope}%
\pgfpathrectangle{\pgfqpoint{3.352233in}{1.400000in}}{\pgfqpoint{2.407767in}{1.544118in}}%
\pgfusepath{clip}%
\pgfsetbuttcap%
\pgfsetroundjoin%
\pgfsetlinewidth{0.501875pt}%
\definecolor{currentstroke}{rgb}{0.271305,0.019942,0.347269}%
\pgfsetstrokecolor{currentstroke}%
\pgfsetdash{}{0pt}%
\pgfpathmoveto{\pgfqpoint{4.232117in}{2.145840in}}%
\pgfpathlineto{\pgfqpoint{4.232117in}{2.145840in}}%
\pgfusepath{stroke}%
\end{pgfscope}%
\begin{pgfscope}%
\pgfpathrectangle{\pgfqpoint{3.352233in}{1.400000in}}{\pgfqpoint{2.407767in}{1.544118in}}%
\pgfusepath{clip}%
\pgfsetbuttcap%
\pgfsetroundjoin%
\pgfsetlinewidth{0.501875pt}%
\definecolor{currentstroke}{rgb}{0.271305,0.019942,0.347269}%
\pgfsetstrokecolor{currentstroke}%
\pgfsetdash{}{0pt}%
\pgfpathmoveto{\pgfqpoint{4.232117in}{2.145840in}}%
\pgfpathlineto{\pgfqpoint{4.230449in}{2.145074in}}%
\pgfusepath{stroke}%
\end{pgfscope}%
\begin{pgfscope}%
\pgfpathrectangle{\pgfqpoint{3.352233in}{1.400000in}}{\pgfqpoint{2.407767in}{1.544118in}}%
\pgfusepath{clip}%
\pgfsetbuttcap%
\pgfsetroundjoin%
\pgfsetlinewidth{0.501875pt}%
\definecolor{currentstroke}{rgb}{0.269944,0.014625,0.341379}%
\pgfsetstrokecolor{currentstroke}%
\pgfsetdash{}{0pt}%
\pgfpathmoveto{\pgfqpoint{4.230449in}{2.145074in}}%
\pgfpathlineto{\pgfqpoint{4.229885in}{2.143992in}}%
\pgfusepath{stroke}%
\end{pgfscope}%
\begin{pgfscope}%
\pgfpathrectangle{\pgfqpoint{3.352233in}{1.400000in}}{\pgfqpoint{2.407767in}{1.544118in}}%
\pgfusepath{clip}%
\pgfsetbuttcap%
\pgfsetroundjoin%
\pgfsetlinewidth{0.501875pt}%
\definecolor{currentstroke}{rgb}{0.269944,0.014625,0.341379}%
\pgfsetstrokecolor{currentstroke}%
\pgfsetdash{}{0pt}%
\pgfpathmoveto{\pgfqpoint{4.229885in}{2.143992in}}%
\pgfpathlineto{\pgfqpoint{4.230273in}{2.142473in}}%
\pgfusepath{stroke}%
\end{pgfscope}%
\begin{pgfscope}%
\pgfpathrectangle{\pgfqpoint{3.352233in}{1.400000in}}{\pgfqpoint{2.407767in}{1.544118in}}%
\pgfusepath{clip}%
\pgfsetbuttcap%
\pgfsetroundjoin%
\pgfsetlinewidth{0.501875pt}%
\definecolor{currentstroke}{rgb}{0.269944,0.014625,0.341379}%
\pgfsetstrokecolor{currentstroke}%
\pgfsetdash{}{0pt}%
\pgfpathmoveto{\pgfqpoint{4.230273in}{2.142473in}}%
\pgfpathlineto{\pgfqpoint{4.230273in}{2.142473in}}%
\pgfusepath{stroke}%
\end{pgfscope}%
\begin{pgfscope}%
\pgfpathrectangle{\pgfqpoint{3.352233in}{1.400000in}}{\pgfqpoint{2.407767in}{1.544118in}}%
\pgfusepath{clip}%
\pgfsetbuttcap%
\pgfsetroundjoin%
\pgfsetlinewidth{0.501875pt}%
\definecolor{currentstroke}{rgb}{0.269944,0.014625,0.341379}%
\pgfsetstrokecolor{currentstroke}%
\pgfsetdash{}{0pt}%
\pgfpathmoveto{\pgfqpoint{4.230273in}{2.142473in}}%
\pgfpathlineto{\pgfqpoint{4.228950in}{2.141269in}}%
\pgfusepath{stroke}%
\end{pgfscope}%
\begin{pgfscope}%
\pgfpathrectangle{\pgfqpoint{3.352233in}{1.400000in}}{\pgfqpoint{2.407767in}{1.544118in}}%
\pgfusepath{clip}%
\pgfsetbuttcap%
\pgfsetroundjoin%
\pgfsetlinewidth{0.501875pt}%
\definecolor{currentstroke}{rgb}{0.269944,0.014625,0.341379}%
\pgfsetstrokecolor{currentstroke}%
\pgfsetdash{}{0pt}%
\pgfpathmoveto{\pgfqpoint{4.228950in}{2.141269in}}%
\pgfpathlineto{\pgfqpoint{4.228895in}{2.139450in}}%
\pgfusepath{stroke}%
\end{pgfscope}%
\begin{pgfscope}%
\pgfpathrectangle{\pgfqpoint{3.352233in}{1.400000in}}{\pgfqpoint{2.407767in}{1.544118in}}%
\pgfusepath{clip}%
\pgfsetbuttcap%
\pgfsetroundjoin%
\pgfsetlinewidth{0.501875pt}%
\definecolor{currentstroke}{rgb}{0.269944,0.014625,0.341379}%
\pgfsetstrokecolor{currentstroke}%
\pgfsetdash{}{0pt}%
\pgfpathmoveto{\pgfqpoint{4.228895in}{2.139450in}}%
\pgfpathlineto{\pgfqpoint{4.228167in}{2.137435in}}%
\pgfusepath{stroke}%
\end{pgfscope}%
\begin{pgfscope}%
\pgfpathrectangle{\pgfqpoint{3.352233in}{1.400000in}}{\pgfqpoint{2.407767in}{1.544118in}}%
\pgfusepath{clip}%
\pgfsetbuttcap%
\pgfsetroundjoin%
\pgfsetlinewidth{0.501875pt}%
\definecolor{currentstroke}{rgb}{0.269944,0.014625,0.341379}%
\pgfsetstrokecolor{currentstroke}%
\pgfsetdash{}{0pt}%
\pgfpathmoveto{\pgfqpoint{4.228167in}{2.137435in}}%
\pgfpathlineto{\pgfqpoint{4.226965in}{2.134912in}}%
\pgfusepath{stroke}%
\end{pgfscope}%
\begin{pgfscope}%
\pgfpathrectangle{\pgfqpoint{3.352233in}{1.400000in}}{\pgfqpoint{2.407767in}{1.544118in}}%
\pgfusepath{clip}%
\pgfsetbuttcap%
\pgfsetroundjoin%
\pgfsetlinewidth{0.501875pt}%
\definecolor{currentstroke}{rgb}{0.269944,0.014625,0.341379}%
\pgfsetstrokecolor{currentstroke}%
\pgfsetdash{}{0pt}%
\pgfpathmoveto{\pgfqpoint{4.226965in}{2.134912in}}%
\pgfpathlineto{\pgfqpoint{4.224639in}{2.131765in}}%
\pgfusepath{stroke}%
\end{pgfscope}%
\begin{pgfscope}%
\pgfpathrectangle{\pgfqpoint{3.352233in}{1.400000in}}{\pgfqpoint{2.407767in}{1.544118in}}%
\pgfusepath{clip}%
\pgfsetbuttcap%
\pgfsetroundjoin%
\pgfsetlinewidth{0.501875pt}%
\definecolor{currentstroke}{rgb}{0.269944,0.014625,0.341379}%
\pgfsetstrokecolor{currentstroke}%
\pgfsetdash{}{0pt}%
\pgfpathmoveto{\pgfqpoint{4.224639in}{2.131765in}}%
\pgfpathlineto{\pgfqpoint{4.222112in}{2.127956in}}%
\pgfusepath{stroke}%
\end{pgfscope}%
\begin{pgfscope}%
\pgfpathrectangle{\pgfqpoint{3.352233in}{1.400000in}}{\pgfqpoint{2.407767in}{1.544118in}}%
\pgfusepath{clip}%
\pgfsetbuttcap%
\pgfsetroundjoin%
\pgfsetlinewidth{0.501875pt}%
\definecolor{currentstroke}{rgb}{0.271305,0.019942,0.347269}%
\pgfsetstrokecolor{currentstroke}%
\pgfsetdash{}{0pt}%
\pgfpathmoveto{\pgfqpoint{4.222112in}{2.127956in}}%
\pgfpathlineto{\pgfqpoint{4.222112in}{2.127956in}}%
\pgfusepath{stroke}%
\end{pgfscope}%
\begin{pgfscope}%
\pgfpathrectangle{\pgfqpoint{3.352233in}{1.400000in}}{\pgfqpoint{2.407767in}{1.544118in}}%
\pgfusepath{clip}%
\pgfsetbuttcap%
\pgfsetroundjoin%
\pgfsetlinewidth{0.501875pt}%
\definecolor{currentstroke}{rgb}{0.271305,0.019942,0.347269}%
\pgfsetstrokecolor{currentstroke}%
\pgfsetdash{}{0pt}%
\pgfpathmoveto{\pgfqpoint{4.222112in}{2.127956in}}%
\pgfpathlineto{\pgfqpoint{4.220740in}{2.125115in}}%
\pgfusepath{stroke}%
\end{pgfscope}%
\begin{pgfscope}%
\pgfpathrectangle{\pgfqpoint{3.352233in}{1.400000in}}{\pgfqpoint{2.407767in}{1.544118in}}%
\pgfusepath{clip}%
\pgfsetbuttcap%
\pgfsetroundjoin%
\pgfsetlinewidth{0.501875pt}%
\definecolor{currentstroke}{rgb}{0.271305,0.019942,0.347269}%
\pgfsetstrokecolor{currentstroke}%
\pgfsetdash{}{0pt}%
\pgfpathmoveto{\pgfqpoint{4.220740in}{2.125115in}}%
\pgfpathlineto{\pgfqpoint{4.219807in}{2.123047in}}%
\pgfusepath{stroke}%
\end{pgfscope}%
\begin{pgfscope}%
\pgfpathrectangle{\pgfqpoint{3.352233in}{1.400000in}}{\pgfqpoint{2.407767in}{1.544118in}}%
\pgfusepath{clip}%
\pgfsetbuttcap%
\pgfsetroundjoin%
\pgfsetlinewidth{0.501875pt}%
\definecolor{currentstroke}{rgb}{0.271305,0.019942,0.347269}%
\pgfsetstrokecolor{currentstroke}%
\pgfsetdash{}{0pt}%
\pgfpathmoveto{\pgfqpoint{4.219807in}{2.123047in}}%
\pgfpathlineto{\pgfqpoint{4.219375in}{2.121301in}}%
\pgfusepath{stroke}%
\end{pgfscope}%
\begin{pgfscope}%
\pgfpathrectangle{\pgfqpoint{3.352233in}{1.400000in}}{\pgfqpoint{2.407767in}{1.544118in}}%
\pgfusepath{clip}%
\pgfsetbuttcap%
\pgfsetroundjoin%
\pgfsetlinewidth{0.501875pt}%
\definecolor{currentstroke}{rgb}{0.269944,0.014625,0.341379}%
\pgfsetstrokecolor{currentstroke}%
\pgfsetdash{}{0pt}%
\pgfpathmoveto{\pgfqpoint{5.194556in}{1.673713in}}%
\pgfpathlineto{\pgfqpoint{5.141588in}{1.673861in}}%
\pgfusepath{stroke}%
\end{pgfscope}%
\begin{pgfscope}%
\pgfpathrectangle{\pgfqpoint{3.352233in}{1.400000in}}{\pgfqpoint{2.407767in}{1.544118in}}%
\pgfusepath{clip}%
\pgfsetbuttcap%
\pgfsetroundjoin%
\pgfsetlinewidth{0.501875pt}%
\definecolor{currentstroke}{rgb}{0.271305,0.019942,0.347269}%
\pgfsetstrokecolor{currentstroke}%
\pgfsetdash{}{0pt}%
\pgfpathmoveto{\pgfqpoint{5.141588in}{1.673861in}}%
\pgfpathlineto{\pgfqpoint{5.088636in}{1.673968in}}%
\pgfusepath{stroke}%
\end{pgfscope}%
\begin{pgfscope}%
\pgfpathrectangle{\pgfqpoint{3.352233in}{1.400000in}}{\pgfqpoint{2.407767in}{1.544118in}}%
\pgfusepath{clip}%
\pgfsetbuttcap%
\pgfsetroundjoin%
\pgfsetlinewidth{0.501875pt}%
\definecolor{currentstroke}{rgb}{0.272594,0.025563,0.353093}%
\pgfsetstrokecolor{currentstroke}%
\pgfsetdash{}{0pt}%
\pgfpathmoveto{\pgfqpoint{5.088636in}{1.673968in}}%
\pgfpathlineto{\pgfqpoint{5.035689in}{1.674385in}}%
\pgfusepath{stroke}%
\end{pgfscope}%
\begin{pgfscope}%
\pgfpathrectangle{\pgfqpoint{3.352233in}{1.400000in}}{\pgfqpoint{2.407767in}{1.544118in}}%
\pgfusepath{clip}%
\pgfsetbuttcap%
\pgfsetroundjoin%
\pgfsetlinewidth{0.501875pt}%
\definecolor{currentstroke}{rgb}{0.276022,0.044167,0.370164}%
\pgfsetstrokecolor{currentstroke}%
\pgfsetdash{}{0pt}%
\pgfpathmoveto{\pgfqpoint{5.035689in}{1.674385in}}%
\pgfpathlineto{\pgfqpoint{4.982731in}{1.675190in}}%
\pgfusepath{stroke}%
\end{pgfscope}%
\begin{pgfscope}%
\pgfpathrectangle{\pgfqpoint{3.352233in}{1.400000in}}{\pgfqpoint{2.407767in}{1.544118in}}%
\pgfusepath{clip}%
\pgfsetbuttcap%
\pgfsetroundjoin%
\pgfsetlinewidth{0.501875pt}%
\definecolor{currentstroke}{rgb}{0.274952,0.037752,0.364543}%
\pgfsetstrokecolor{currentstroke}%
\pgfsetdash{}{0pt}%
\pgfpathmoveto{\pgfqpoint{4.982731in}{1.675190in}}%
\pgfpathlineto{\pgfqpoint{4.929779in}{1.676059in}}%
\pgfusepath{stroke}%
\end{pgfscope}%
\begin{pgfscope}%
\pgfpathrectangle{\pgfqpoint{3.352233in}{1.400000in}}{\pgfqpoint{2.407767in}{1.544118in}}%
\pgfusepath{clip}%
\pgfsetbuttcap%
\pgfsetroundjoin%
\pgfsetlinewidth{0.501875pt}%
\definecolor{currentstroke}{rgb}{0.276022,0.044167,0.370164}%
\pgfsetstrokecolor{currentstroke}%
\pgfsetdash{}{0pt}%
\pgfpathmoveto{\pgfqpoint{4.929779in}{1.676059in}}%
\pgfpathlineto{\pgfqpoint{4.876862in}{1.677143in}}%
\pgfusepath{stroke}%
\end{pgfscope}%
\begin{pgfscope}%
\pgfpathrectangle{\pgfqpoint{3.352233in}{1.400000in}}{\pgfqpoint{2.407767in}{1.544118in}}%
\pgfusepath{clip}%
\pgfsetbuttcap%
\pgfsetroundjoin%
\pgfsetlinewidth{0.501875pt}%
\definecolor{currentstroke}{rgb}{0.276022,0.044167,0.370164}%
\pgfsetstrokecolor{currentstroke}%
\pgfsetdash{}{0pt}%
\pgfpathmoveto{\pgfqpoint{4.876862in}{1.677143in}}%
\pgfpathlineto{\pgfqpoint{4.824001in}{1.678943in}}%
\pgfusepath{stroke}%
\end{pgfscope}%
\begin{pgfscope}%
\pgfpathrectangle{\pgfqpoint{3.352233in}{1.400000in}}{\pgfqpoint{2.407767in}{1.544118in}}%
\pgfusepath{clip}%
\pgfsetbuttcap%
\pgfsetroundjoin%
\pgfsetlinewidth{0.501875pt}%
\definecolor{currentstroke}{rgb}{0.276022,0.044167,0.370164}%
\pgfsetstrokecolor{currentstroke}%
\pgfsetdash{}{0pt}%
\pgfpathmoveto{\pgfqpoint{4.824001in}{1.678943in}}%
\pgfpathlineto{\pgfqpoint{4.771199in}{1.681586in}}%
\pgfusepath{stroke}%
\end{pgfscope}%
\begin{pgfscope}%
\pgfpathrectangle{\pgfqpoint{3.352233in}{1.400000in}}{\pgfqpoint{2.407767in}{1.544118in}}%
\pgfusepath{clip}%
\pgfsetbuttcap%
\pgfsetroundjoin%
\pgfsetlinewidth{0.501875pt}%
\definecolor{currentstroke}{rgb}{0.273809,0.031497,0.358853}%
\pgfsetstrokecolor{currentstroke}%
\pgfsetdash{}{0pt}%
\pgfpathmoveto{\pgfqpoint{4.771199in}{1.681586in}}%
\pgfpathlineto{\pgfqpoint{4.718657in}{1.685613in}}%
\pgfusepath{stroke}%
\end{pgfscope}%
\begin{pgfscope}%
\pgfpathrectangle{\pgfqpoint{3.352233in}{1.400000in}}{\pgfqpoint{2.407767in}{1.544118in}}%
\pgfusepath{clip}%
\pgfsetbuttcap%
\pgfsetroundjoin%
\pgfsetlinewidth{0.501875pt}%
\definecolor{currentstroke}{rgb}{0.268510,0.009605,0.335427}%
\pgfsetstrokecolor{currentstroke}%
\pgfsetdash{}{0pt}%
\pgfpathmoveto{\pgfqpoint{5.206279in}{2.658505in}}%
\pgfpathlineto{\pgfqpoint{5.153345in}{2.659330in}}%
\pgfusepath{stroke}%
\end{pgfscope}%
\begin{pgfscope}%
\pgfpathrectangle{\pgfqpoint{3.352233in}{1.400000in}}{\pgfqpoint{2.407767in}{1.544118in}}%
\pgfusepath{clip}%
\pgfsetbuttcap%
\pgfsetroundjoin%
\pgfsetlinewidth{0.501875pt}%
\definecolor{currentstroke}{rgb}{0.271305,0.019942,0.347269}%
\pgfsetstrokecolor{currentstroke}%
\pgfsetdash{}{0pt}%
\pgfpathmoveto{\pgfqpoint{5.153345in}{2.659330in}}%
\pgfpathlineto{\pgfqpoint{5.100395in}{2.660342in}}%
\pgfusepath{stroke}%
\end{pgfscope}%
\begin{pgfscope}%
\pgfpathrectangle{\pgfqpoint{3.352233in}{1.400000in}}{\pgfqpoint{2.407767in}{1.544118in}}%
\pgfusepath{clip}%
\pgfsetbuttcap%
\pgfsetroundjoin%
\pgfsetlinewidth{0.501875pt}%
\definecolor{currentstroke}{rgb}{0.272594,0.025563,0.353093}%
\pgfsetstrokecolor{currentstroke}%
\pgfsetdash{}{0pt}%
\pgfpathmoveto{\pgfqpoint{5.100395in}{2.660342in}}%
\pgfpathlineto{\pgfqpoint{5.047460in}{2.660180in}}%
\pgfusepath{stroke}%
\end{pgfscope}%
\begin{pgfscope}%
\pgfpathrectangle{\pgfqpoint{3.352233in}{1.400000in}}{\pgfqpoint{2.407767in}{1.544118in}}%
\pgfusepath{clip}%
\pgfsetbuttcap%
\pgfsetroundjoin%
\pgfsetlinewidth{0.501875pt}%
\definecolor{currentstroke}{rgb}{0.273809,0.031497,0.358853}%
\pgfsetstrokecolor{currentstroke}%
\pgfsetdash{}{0pt}%
\pgfpathmoveto{\pgfqpoint{5.047460in}{2.660180in}}%
\pgfpathlineto{\pgfqpoint{4.994532in}{2.658932in}}%
\pgfusepath{stroke}%
\end{pgfscope}%
\begin{pgfscope}%
\pgfpathrectangle{\pgfqpoint{3.352233in}{1.400000in}}{\pgfqpoint{2.407767in}{1.544118in}}%
\pgfusepath{clip}%
\pgfsetbuttcap%
\pgfsetroundjoin%
\pgfsetlinewidth{0.501875pt}%
\definecolor{currentstroke}{rgb}{0.274952,0.037752,0.364543}%
\pgfsetstrokecolor{currentstroke}%
\pgfsetdash{}{0pt}%
\pgfpathmoveto{\pgfqpoint{4.994532in}{2.658932in}}%
\pgfpathlineto{\pgfqpoint{4.941616in}{2.657453in}}%
\pgfusepath{stroke}%
\end{pgfscope}%
\begin{pgfscope}%
\pgfpathrectangle{\pgfqpoint{3.352233in}{1.400000in}}{\pgfqpoint{2.407767in}{1.544118in}}%
\pgfusepath{clip}%
\pgfsetbuttcap%
\pgfsetroundjoin%
\pgfsetlinewidth{0.501875pt}%
\definecolor{currentstroke}{rgb}{0.273809,0.031497,0.358853}%
\pgfsetstrokecolor{currentstroke}%
\pgfsetdash{}{0pt}%
\pgfpathmoveto{\pgfqpoint{4.819290in}{2.674100in}}%
\pgfpathlineto{\pgfqpoint{4.766695in}{2.670655in}}%
\pgfusepath{stroke}%
\end{pgfscope}%
\begin{pgfscope}%
\pgfpathrectangle{\pgfqpoint{3.352233in}{1.400000in}}{\pgfqpoint{2.407767in}{1.544118in}}%
\pgfusepath{clip}%
\pgfsetbuttcap%
\pgfsetroundjoin%
\pgfsetlinewidth{0.501875pt}%
\definecolor{currentstroke}{rgb}{0.274952,0.037752,0.364543}%
\pgfsetstrokecolor{currentstroke}%
\pgfsetdash{}{0pt}%
\pgfpathmoveto{\pgfqpoint{4.766695in}{2.670655in}}%
\pgfpathlineto{\pgfqpoint{4.713845in}{2.668888in}}%
\pgfusepath{stroke}%
\end{pgfscope}%
\begin{pgfscope}%
\pgfpathrectangle{\pgfqpoint{3.352233in}{1.400000in}}{\pgfqpoint{2.407767in}{1.544118in}}%
\pgfusepath{clip}%
\pgfsetbuttcap%
\pgfsetroundjoin%
\pgfsetlinewidth{0.501875pt}%
\definecolor{currentstroke}{rgb}{0.273809,0.031497,0.358853}%
\pgfsetstrokecolor{currentstroke}%
\pgfsetdash{}{0pt}%
\pgfpathmoveto{\pgfqpoint{4.713845in}{2.668888in}}%
\pgfpathlineto{\pgfqpoint{4.661412in}{2.666184in}}%
\pgfusepath{stroke}%
\end{pgfscope}%
\begin{pgfscope}%
\pgfpathrectangle{\pgfqpoint{3.352233in}{1.400000in}}{\pgfqpoint{2.407767in}{1.544118in}}%
\pgfusepath{clip}%
\pgfsetbuttcap%
\pgfsetroundjoin%
\pgfsetlinewidth{0.501875pt}%
\definecolor{currentstroke}{rgb}{0.272594,0.025563,0.353093}%
\pgfsetstrokecolor{currentstroke}%
\pgfsetdash{}{0pt}%
\pgfpathmoveto{\pgfqpoint{4.661412in}{2.666184in}}%
\pgfpathlineto{\pgfqpoint{4.610297in}{2.658505in}}%
\pgfusepath{stroke}%
\end{pgfscope}%
\begin{pgfscope}%
\pgfpathrectangle{\pgfqpoint{3.352233in}{1.400000in}}{\pgfqpoint{2.407767in}{1.544118in}}%
\pgfusepath{clip}%
\pgfsetbuttcap%
\pgfsetroundjoin%
\pgfsetlinewidth{0.501875pt}%
\definecolor{currentstroke}{rgb}{0.272594,0.025563,0.353093}%
\pgfsetstrokecolor{currentstroke}%
\pgfsetdash{}{0pt}%
\pgfpathmoveto{\pgfqpoint{4.610297in}{2.658505in}}%
\pgfpathlineto{\pgfqpoint{4.610297in}{2.658505in}}%
\pgfusepath{stroke}%
\end{pgfscope}%
\begin{pgfscope}%
\pgfpathrectangle{\pgfqpoint{3.352233in}{1.400000in}}{\pgfqpoint{2.407767in}{1.544118in}}%
\pgfusepath{clip}%
\pgfsetbuttcap%
\pgfsetroundjoin%
\pgfsetlinewidth{0.501875pt}%
\definecolor{currentstroke}{rgb}{0.272594,0.025563,0.353093}%
\pgfsetstrokecolor{currentstroke}%
\pgfsetdash{}{0pt}%
\pgfpathmoveto{\pgfqpoint{4.610297in}{2.658505in}}%
\pgfpathlineto{\pgfqpoint{4.610297in}{2.658505in}}%
\pgfusepath{stroke}%
\end{pgfscope}%
\begin{pgfscope}%
\pgfpathrectangle{\pgfqpoint{3.352233in}{1.400000in}}{\pgfqpoint{2.407767in}{1.544118in}}%
\pgfusepath{clip}%
\pgfsetbuttcap%
\pgfsetroundjoin%
\pgfsetlinewidth{0.501875pt}%
\definecolor{currentstroke}{rgb}{0.272594,0.025563,0.353093}%
\pgfsetstrokecolor{currentstroke}%
\pgfsetdash{}{0pt}%
\pgfpathmoveto{\pgfqpoint{4.610297in}{2.658505in}}%
\pgfpathlineto{\pgfqpoint{4.590778in}{2.655704in}}%
\pgfusepath{stroke}%
\end{pgfscope}%
\begin{pgfscope}%
\pgfpathrectangle{\pgfqpoint{3.352233in}{1.400000in}}{\pgfqpoint{2.407767in}{1.544118in}}%
\pgfusepath{clip}%
\pgfsetbuttcap%
\pgfsetroundjoin%
\pgfsetlinewidth{0.501875pt}%
\definecolor{currentstroke}{rgb}{0.271305,0.019942,0.347269}%
\pgfsetstrokecolor{currentstroke}%
\pgfsetdash{}{0pt}%
\pgfpathmoveto{\pgfqpoint{4.590778in}{2.655704in}}%
\pgfpathlineto{\pgfqpoint{4.568305in}{2.653259in}}%
\pgfusepath{stroke}%
\end{pgfscope}%
\begin{pgfscope}%
\pgfpathrectangle{\pgfqpoint{3.352233in}{1.400000in}}{\pgfqpoint{2.407767in}{1.544118in}}%
\pgfusepath{clip}%
\pgfsetbuttcap%
\pgfsetroundjoin%
\pgfsetlinewidth{0.501875pt}%
\definecolor{currentstroke}{rgb}{0.269944,0.014625,0.341379}%
\pgfsetstrokecolor{currentstroke}%
\pgfsetdash{}{0pt}%
\pgfpathmoveto{\pgfqpoint{4.568305in}{2.653259in}}%
\pgfpathlineto{\pgfqpoint{4.568305in}{2.653259in}}%
\pgfusepath{stroke}%
\end{pgfscope}%
\begin{pgfscope}%
\pgfpathrectangle{\pgfqpoint{3.352233in}{1.400000in}}{\pgfqpoint{2.407767in}{1.544118in}}%
\pgfusepath{clip}%
\pgfsetbuttcap%
\pgfsetroundjoin%
\pgfsetlinewidth{0.501875pt}%
\definecolor{currentstroke}{rgb}{0.269944,0.014625,0.341379}%
\pgfsetstrokecolor{currentstroke}%
\pgfsetdash{}{0pt}%
\pgfpathmoveto{\pgfqpoint{4.568305in}{2.653259in}}%
\pgfpathlineto{\pgfqpoint{4.540045in}{2.648738in}}%
\pgfusepath{stroke}%
\end{pgfscope}%
\begin{pgfscope}%
\pgfpathrectangle{\pgfqpoint{3.352233in}{1.400000in}}{\pgfqpoint{2.407767in}{1.544118in}}%
\pgfusepath{clip}%
\pgfsetbuttcap%
\pgfsetroundjoin%
\pgfsetlinewidth{0.501875pt}%
\definecolor{currentstroke}{rgb}{0.235526,0.309527,0.542944}%
\pgfsetstrokecolor{currentstroke}%
\pgfsetdash{}{0pt}%
\pgfpathmoveto{\pgfqpoint{3.797594in}{2.137313in}}%
\pgfpathlineto{\pgfqpoint{3.850540in}{2.138222in}}%
\pgfusepath{stroke}%
\end{pgfscope}%
\begin{pgfscope}%
\pgfpathrectangle{\pgfqpoint{3.352233in}{1.400000in}}{\pgfqpoint{2.407767in}{1.544118in}}%
\pgfusepath{clip}%
\pgfsetbuttcap%
\pgfsetroundjoin%
\pgfsetlinewidth{0.501875pt}%
\definecolor{currentstroke}{rgb}{0.250425,0.274290,0.533103}%
\pgfsetstrokecolor{currentstroke}%
\pgfsetdash{}{0pt}%
\pgfpathmoveto{\pgfqpoint{3.850540in}{2.138222in}}%
\pgfpathlineto{\pgfqpoint{3.903481in}{2.138229in}}%
\pgfusepath{stroke}%
\end{pgfscope}%
\begin{pgfscope}%
\pgfpathrectangle{\pgfqpoint{3.352233in}{1.400000in}}{\pgfqpoint{2.407767in}{1.544118in}}%
\pgfusepath{clip}%
\pgfsetbuttcap%
\pgfsetroundjoin%
\pgfsetlinewidth{0.501875pt}%
\definecolor{currentstroke}{rgb}{0.270595,0.214069,0.507052}%
\pgfsetstrokecolor{currentstroke}%
\pgfsetdash{}{0pt}%
\pgfpathmoveto{\pgfqpoint{3.903481in}{2.138229in}}%
\pgfpathlineto{\pgfqpoint{3.956406in}{2.137792in}}%
\pgfusepath{stroke}%
\end{pgfscope}%
\begin{pgfscope}%
\pgfpathrectangle{\pgfqpoint{3.352233in}{1.400000in}}{\pgfqpoint{2.407767in}{1.544118in}}%
\pgfusepath{clip}%
\pgfsetbuttcap%
\pgfsetroundjoin%
\pgfsetlinewidth{0.501875pt}%
\definecolor{currentstroke}{rgb}{0.279574,0.170599,0.479997}%
\pgfsetstrokecolor{currentstroke}%
\pgfsetdash{}{0pt}%
\pgfpathmoveto{\pgfqpoint{3.956406in}{2.137792in}}%
\pgfpathlineto{\pgfqpoint{4.009340in}{2.137387in}}%
\pgfusepath{stroke}%
\end{pgfscope}%
\begin{pgfscope}%
\pgfpathrectangle{\pgfqpoint{3.352233in}{1.400000in}}{\pgfqpoint{2.407767in}{1.544118in}}%
\pgfusepath{clip}%
\pgfsetbuttcap%
\pgfsetroundjoin%
\pgfsetlinewidth{0.501875pt}%
\definecolor{currentstroke}{rgb}{0.282623,0.140926,0.457517}%
\pgfsetstrokecolor{currentstroke}%
\pgfsetdash{}{0pt}%
\pgfpathmoveto{\pgfqpoint{4.009340in}{2.137387in}}%
\pgfpathlineto{\pgfqpoint{4.062250in}{2.137369in}}%
\pgfusepath{stroke}%
\end{pgfscope}%
\begin{pgfscope}%
\pgfpathrectangle{\pgfqpoint{3.352233in}{1.400000in}}{\pgfqpoint{2.407767in}{1.544118in}}%
\pgfusepath{clip}%
\pgfsetbuttcap%
\pgfsetroundjoin%
\pgfsetlinewidth{0.501875pt}%
\definecolor{currentstroke}{rgb}{0.279566,0.067836,0.391917}%
\pgfsetstrokecolor{currentstroke}%
\pgfsetdash{}{0pt}%
\pgfpathmoveto{\pgfqpoint{4.062250in}{2.137369in}}%
\pgfpathlineto{\pgfqpoint{4.115058in}{2.137003in}}%
\pgfusepath{stroke}%
\end{pgfscope}%
\begin{pgfscope}%
\pgfpathrectangle{\pgfqpoint{3.352233in}{1.400000in}}{\pgfqpoint{2.407767in}{1.544118in}}%
\pgfusepath{clip}%
\pgfsetbuttcap%
\pgfsetroundjoin%
\pgfsetlinewidth{0.501875pt}%
\definecolor{currentstroke}{rgb}{0.277018,0.050344,0.375715}%
\pgfsetstrokecolor{currentstroke}%
\pgfsetdash{}{0pt}%
\pgfpathmoveto{\pgfqpoint{4.115058in}{2.137003in}}%
\pgfpathlineto{\pgfqpoint{4.167729in}{2.134486in}}%
\pgfusepath{stroke}%
\end{pgfscope}%
\begin{pgfscope}%
\pgfpathrectangle{\pgfqpoint{3.352233in}{1.400000in}}{\pgfqpoint{2.407767in}{1.544118in}}%
\pgfusepath{clip}%
\pgfsetbuttcap%
\pgfsetroundjoin%
\pgfsetlinewidth{0.501875pt}%
\definecolor{currentstroke}{rgb}{0.273809,0.031497,0.358853}%
\pgfsetstrokecolor{currentstroke}%
\pgfsetdash{}{0pt}%
\pgfpathmoveto{\pgfqpoint{4.167729in}{2.134486in}}%
\pgfpathlineto{\pgfqpoint{4.167729in}{2.134486in}}%
\pgfusepath{stroke}%
\end{pgfscope}%
\begin{pgfscope}%
\pgfpathrectangle{\pgfqpoint{3.352233in}{1.400000in}}{\pgfqpoint{2.407767in}{1.544118in}}%
\pgfusepath{clip}%
\pgfsetbuttcap%
\pgfsetroundjoin%
\pgfsetlinewidth{0.501875pt}%
\definecolor{currentstroke}{rgb}{0.273809,0.031497,0.358853}%
\pgfsetstrokecolor{currentstroke}%
\pgfsetdash{}{0pt}%
\pgfpathmoveto{\pgfqpoint{4.167729in}{2.134486in}}%
\pgfpathlineto{\pgfqpoint{4.190994in}{2.133384in}}%
\pgfusepath{stroke}%
\end{pgfscope}%
\begin{pgfscope}%
\pgfpathrectangle{\pgfqpoint{3.352233in}{1.400000in}}{\pgfqpoint{2.407767in}{1.544118in}}%
\pgfusepath{clip}%
\pgfsetbuttcap%
\pgfsetroundjoin%
\pgfsetlinewidth{0.501875pt}%
\definecolor{currentstroke}{rgb}{0.272594,0.025563,0.353093}%
\pgfsetstrokecolor{currentstroke}%
\pgfsetdash{}{0pt}%
\pgfpathmoveto{\pgfqpoint{4.190994in}{2.133384in}}%
\pgfpathlineto{\pgfqpoint{4.190994in}{2.133384in}}%
\pgfusepath{stroke}%
\end{pgfscope}%
\begin{pgfscope}%
\pgfpathrectangle{\pgfqpoint{3.352233in}{1.400000in}}{\pgfqpoint{2.407767in}{1.544118in}}%
\pgfusepath{clip}%
\pgfsetbuttcap%
\pgfsetroundjoin%
\pgfsetlinewidth{0.501875pt}%
\definecolor{currentstroke}{rgb}{0.235526,0.309527,0.542944}%
\pgfsetstrokecolor{currentstroke}%
\pgfsetdash{}{0pt}%
\pgfpathmoveto{\pgfqpoint{3.797594in}{2.102567in}}%
\pgfpathlineto{\pgfqpoint{3.850547in}{2.103284in}}%
\pgfusepath{stroke}%
\end{pgfscope}%
\begin{pgfscope}%
\pgfpathrectangle{\pgfqpoint{3.352233in}{1.400000in}}{\pgfqpoint{2.407767in}{1.544118in}}%
\pgfusepath{clip}%
\pgfsetbuttcap%
\pgfsetroundjoin%
\pgfsetlinewidth{0.501875pt}%
\definecolor{currentstroke}{rgb}{0.248629,0.278775,0.534556}%
\pgfsetstrokecolor{currentstroke}%
\pgfsetdash{}{0pt}%
\pgfpathmoveto{\pgfqpoint{3.850547in}{2.103284in}}%
\pgfpathlineto{\pgfqpoint{3.903485in}{2.104281in}}%
\pgfusepath{stroke}%
\end{pgfscope}%
\begin{pgfscope}%
\pgfpathrectangle{\pgfqpoint{3.352233in}{1.400000in}}{\pgfqpoint{2.407767in}{1.544118in}}%
\pgfusepath{clip}%
\pgfsetbuttcap%
\pgfsetroundjoin%
\pgfsetlinewidth{0.501875pt}%
\definecolor{currentstroke}{rgb}{0.265145,0.232956,0.516599}%
\pgfsetstrokecolor{currentstroke}%
\pgfsetdash{}{0pt}%
\pgfpathmoveto{\pgfqpoint{3.903485in}{2.104281in}}%
\pgfpathlineto{\pgfqpoint{3.956378in}{2.105604in}}%
\pgfusepath{stroke}%
\end{pgfscope}%
\begin{pgfscope}%
\pgfpathrectangle{\pgfqpoint{3.352233in}{1.400000in}}{\pgfqpoint{2.407767in}{1.544118in}}%
\pgfusepath{clip}%
\pgfsetbuttcap%
\pgfsetroundjoin%
\pgfsetlinewidth{0.501875pt}%
\definecolor{currentstroke}{rgb}{0.278012,0.180367,0.486697}%
\pgfsetstrokecolor{currentstroke}%
\pgfsetdash{}{0pt}%
\pgfpathmoveto{\pgfqpoint{3.956378in}{2.105604in}}%
\pgfpathlineto{\pgfqpoint{4.009289in}{2.106640in}}%
\pgfusepath{stroke}%
\end{pgfscope}%
\begin{pgfscope}%
\pgfpathrectangle{\pgfqpoint{3.352233in}{1.400000in}}{\pgfqpoint{2.407767in}{1.544118in}}%
\pgfusepath{clip}%
\pgfsetbuttcap%
\pgfsetroundjoin%
\pgfsetlinewidth{0.501875pt}%
\definecolor{currentstroke}{rgb}{0.283187,0.125848,0.444960}%
\pgfsetstrokecolor{currentstroke}%
\pgfsetdash{}{0pt}%
\pgfpathmoveto{\pgfqpoint{4.009289in}{2.106640in}}%
\pgfpathlineto{\pgfqpoint{4.062204in}{2.107905in}}%
\pgfusepath{stroke}%
\end{pgfscope}%
\begin{pgfscope}%
\pgfpathrectangle{\pgfqpoint{3.352233in}{1.400000in}}{\pgfqpoint{2.407767in}{1.544118in}}%
\pgfusepath{clip}%
\pgfsetbuttcap%
\pgfsetroundjoin%
\pgfsetlinewidth{0.501875pt}%
\definecolor{currentstroke}{rgb}{0.280894,0.078907,0.402329}%
\pgfsetstrokecolor{currentstroke}%
\pgfsetdash{}{0pt}%
\pgfpathmoveto{\pgfqpoint{4.062204in}{2.107905in}}%
\pgfpathlineto{\pgfqpoint{4.114953in}{2.110740in}}%
\pgfusepath{stroke}%
\end{pgfscope}%
\begin{pgfscope}%
\pgfpathrectangle{\pgfqpoint{3.352233in}{1.400000in}}{\pgfqpoint{2.407767in}{1.544118in}}%
\pgfusepath{clip}%
\pgfsetbuttcap%
\pgfsetroundjoin%
\pgfsetlinewidth{0.501875pt}%
\definecolor{currentstroke}{rgb}{0.277018,0.050344,0.375715}%
\pgfsetstrokecolor{currentstroke}%
\pgfsetdash{}{0pt}%
\pgfpathmoveto{\pgfqpoint{4.114953in}{2.110740in}}%
\pgfpathlineto{\pgfqpoint{4.167670in}{2.113336in}}%
\pgfusepath{stroke}%
\end{pgfscope}%
\begin{pgfscope}%
\pgfpathrectangle{\pgfqpoint{3.352233in}{1.400000in}}{\pgfqpoint{2.407767in}{1.544118in}}%
\pgfusepath{clip}%
\pgfsetbuttcap%
\pgfsetroundjoin%
\pgfsetlinewidth{0.501875pt}%
\definecolor{currentstroke}{rgb}{0.274952,0.037752,0.364543}%
\pgfsetstrokecolor{currentstroke}%
\pgfsetdash{}{0pt}%
\pgfpathmoveto{\pgfqpoint{4.167670in}{2.113336in}}%
\pgfpathlineto{\pgfqpoint{4.167670in}{2.113336in}}%
\pgfusepath{stroke}%
\end{pgfscope}%
\begin{pgfscope}%
\pgfpathrectangle{\pgfqpoint{3.352233in}{1.400000in}}{\pgfqpoint{2.407767in}{1.544118in}}%
\pgfusepath{clip}%
\pgfsetbuttcap%
\pgfsetroundjoin%
\pgfsetlinewidth{0.501875pt}%
\definecolor{currentstroke}{rgb}{0.274952,0.037752,0.364543}%
\pgfsetstrokecolor{currentstroke}%
\pgfsetdash{}{0pt}%
\pgfpathmoveto{\pgfqpoint{4.167670in}{2.113336in}}%
\pgfpathlineto{\pgfqpoint{4.189119in}{2.113905in}}%
\pgfusepath{stroke}%
\end{pgfscope}%
\begin{pgfscope}%
\pgfpathrectangle{\pgfqpoint{3.352233in}{1.400000in}}{\pgfqpoint{2.407767in}{1.544118in}}%
\pgfusepath{clip}%
\pgfsetbuttcap%
\pgfsetroundjoin%
\pgfsetlinewidth{0.501875pt}%
\definecolor{currentstroke}{rgb}{0.271305,0.019942,0.347269}%
\pgfsetstrokecolor{currentstroke}%
\pgfsetdash{}{0pt}%
\pgfpathmoveto{\pgfqpoint{4.189119in}{2.113905in}}%
\pgfpathlineto{\pgfqpoint{4.189119in}{2.113905in}}%
\pgfusepath{stroke}%
\end{pgfscope}%
\begin{pgfscope}%
\pgfpathrectangle{\pgfqpoint{3.352233in}{1.400000in}}{\pgfqpoint{2.407767in}{1.544118in}}%
\pgfusepath{clip}%
\pgfsetbuttcap%
\pgfsetroundjoin%
\pgfsetlinewidth{0.501875pt}%
\definecolor{currentstroke}{rgb}{0.271305,0.019942,0.347269}%
\pgfsetstrokecolor{currentstroke}%
\pgfsetdash{}{0pt}%
\pgfpathmoveto{\pgfqpoint{4.189119in}{2.113905in}}%
\pgfpathlineto{\pgfqpoint{4.203396in}{2.114765in}}%
\pgfusepath{stroke}%
\end{pgfscope}%
\begin{pgfscope}%
\pgfpathrectangle{\pgfqpoint{3.352233in}{1.400000in}}{\pgfqpoint{2.407767in}{1.544118in}}%
\pgfusepath{clip}%
\pgfsetbuttcap%
\pgfsetroundjoin%
\pgfsetlinewidth{0.501875pt}%
\definecolor{currentstroke}{rgb}{0.268510,0.009605,0.335427}%
\pgfsetstrokecolor{currentstroke}%
\pgfsetdash{}{0pt}%
\pgfpathmoveto{\pgfqpoint{4.203396in}{2.114765in}}%
\pgfpathlineto{\pgfqpoint{4.203396in}{2.114765in}}%
\pgfusepath{stroke}%
\end{pgfscope}%
\begin{pgfscope}%
\pgfpathrectangle{\pgfqpoint{3.352233in}{1.400000in}}{\pgfqpoint{2.407767in}{1.544118in}}%
\pgfusepath{clip}%
\pgfsetbuttcap%
\pgfsetroundjoin%
\pgfsetlinewidth{0.501875pt}%
\definecolor{currentstroke}{rgb}{0.237441,0.305202,0.541921}%
\pgfsetstrokecolor{currentstroke}%
\pgfsetdash{}{0pt}%
\pgfpathmoveto{\pgfqpoint{3.797594in}{2.067820in}}%
\pgfpathlineto{\pgfqpoint{3.850536in}{2.068885in}}%
\pgfusepath{stroke}%
\end{pgfscope}%
\begin{pgfscope}%
\pgfpathrectangle{\pgfqpoint{3.352233in}{1.400000in}}{\pgfqpoint{2.407767in}{1.544118in}}%
\pgfusepath{clip}%
\pgfsetbuttcap%
\pgfsetroundjoin%
\pgfsetlinewidth{0.501875pt}%
\definecolor{currentstroke}{rgb}{0.250425,0.274290,0.533103}%
\pgfsetstrokecolor{currentstroke}%
\pgfsetdash{}{0pt}%
\pgfpathmoveto{\pgfqpoint{3.850536in}{2.068885in}}%
\pgfpathlineto{\pgfqpoint{3.903483in}{2.069791in}}%
\pgfusepath{stroke}%
\end{pgfscope}%
\begin{pgfscope}%
\pgfpathrectangle{\pgfqpoint{3.352233in}{1.400000in}}{\pgfqpoint{2.407767in}{1.544118in}}%
\pgfusepath{clip}%
\pgfsetbuttcap%
\pgfsetroundjoin%
\pgfsetlinewidth{0.501875pt}%
\definecolor{currentstroke}{rgb}{0.262138,0.242286,0.520837}%
\pgfsetstrokecolor{currentstroke}%
\pgfsetdash{}{0pt}%
\pgfpathmoveto{\pgfqpoint{3.903483in}{2.069791in}}%
\pgfpathlineto{\pgfqpoint{3.956406in}{2.071076in}}%
\pgfusepath{stroke}%
\end{pgfscope}%
\begin{pgfscope}%
\pgfpathrectangle{\pgfqpoint{3.352233in}{1.400000in}}{\pgfqpoint{2.407767in}{1.544118in}}%
\pgfusepath{clip}%
\pgfsetbuttcap%
\pgfsetroundjoin%
\pgfsetlinewidth{0.501875pt}%
\definecolor{currentstroke}{rgb}{0.277134,0.185228,0.489898}%
\pgfsetstrokecolor{currentstroke}%
\pgfsetdash{}{0pt}%
\pgfpathmoveto{\pgfqpoint{3.956406in}{2.071076in}}%
\pgfpathlineto{\pgfqpoint{4.009325in}{2.072353in}}%
\pgfusepath{stroke}%
\end{pgfscope}%
\begin{pgfscope}%
\pgfpathrectangle{\pgfqpoint{3.352233in}{1.400000in}}{\pgfqpoint{2.407767in}{1.544118in}}%
\pgfusepath{clip}%
\pgfsetbuttcap%
\pgfsetroundjoin%
\pgfsetlinewidth{0.501875pt}%
\definecolor{currentstroke}{rgb}{0.283187,0.125848,0.444960}%
\pgfsetstrokecolor{currentstroke}%
\pgfsetdash{}{0pt}%
\pgfpathmoveto{\pgfqpoint{4.009325in}{2.072353in}}%
\pgfpathlineto{\pgfqpoint{4.061972in}{2.075250in}}%
\pgfusepath{stroke}%
\end{pgfscope}%
\begin{pgfscope}%
\pgfpathrectangle{\pgfqpoint{3.352233in}{1.400000in}}{\pgfqpoint{2.407767in}{1.544118in}}%
\pgfusepath{clip}%
\pgfsetbuttcap%
\pgfsetroundjoin%
\pgfsetlinewidth{0.501875pt}%
\definecolor{currentstroke}{rgb}{0.281446,0.084320,0.407414}%
\pgfsetstrokecolor{currentstroke}%
\pgfsetdash{}{0pt}%
\pgfpathmoveto{\pgfqpoint{4.061972in}{2.075250in}}%
\pgfpathlineto{\pgfqpoint{4.114548in}{2.079196in}}%
\pgfusepath{stroke}%
\end{pgfscope}%
\begin{pgfscope}%
\pgfpathrectangle{\pgfqpoint{3.352233in}{1.400000in}}{\pgfqpoint{2.407767in}{1.544118in}}%
\pgfusepath{clip}%
\pgfsetbuttcap%
\pgfsetroundjoin%
\pgfsetlinewidth{0.501875pt}%
\definecolor{currentstroke}{rgb}{0.277941,0.056324,0.381191}%
\pgfsetstrokecolor{currentstroke}%
\pgfsetdash{}{0pt}%
\pgfpathmoveto{\pgfqpoint{4.114548in}{2.079196in}}%
\pgfpathlineto{\pgfqpoint{4.167185in}{2.081888in}}%
\pgfusepath{stroke}%
\end{pgfscope}%
\begin{pgfscope}%
\pgfpathrectangle{\pgfqpoint{3.352233in}{1.400000in}}{\pgfqpoint{2.407767in}{1.544118in}}%
\pgfusepath{clip}%
\pgfsetbuttcap%
\pgfsetroundjoin%
\pgfsetlinewidth{0.501875pt}%
\definecolor{currentstroke}{rgb}{0.272594,0.025563,0.353093}%
\pgfsetstrokecolor{currentstroke}%
\pgfsetdash{}{0pt}%
\pgfpathmoveto{\pgfqpoint{4.167185in}{2.081888in}}%
\pgfpathlineto{\pgfqpoint{4.167185in}{2.081888in}}%
\pgfusepath{stroke}%
\end{pgfscope}%
\begin{pgfscope}%
\pgfpathrectangle{\pgfqpoint{3.352233in}{1.400000in}}{\pgfqpoint{2.407767in}{1.544118in}}%
\pgfusepath{clip}%
\pgfsetbuttcap%
\pgfsetroundjoin%
\pgfsetlinewidth{0.501875pt}%
\definecolor{currentstroke}{rgb}{0.272594,0.025563,0.353093}%
\pgfsetstrokecolor{currentstroke}%
\pgfsetdash{}{0pt}%
\pgfpathmoveto{\pgfqpoint{4.167185in}{2.081888in}}%
\pgfpathlineto{\pgfqpoint{4.186106in}{2.082852in}}%
\pgfusepath{stroke}%
\end{pgfscope}%
\begin{pgfscope}%
\pgfpathrectangle{\pgfqpoint{3.352233in}{1.400000in}}{\pgfqpoint{2.407767in}{1.544118in}}%
\pgfusepath{clip}%
\pgfsetbuttcap%
\pgfsetroundjoin%
\pgfsetlinewidth{0.501875pt}%
\definecolor{currentstroke}{rgb}{0.271305,0.019942,0.347269}%
\pgfsetstrokecolor{currentstroke}%
\pgfsetdash{}{0pt}%
\pgfpathmoveto{\pgfqpoint{4.186106in}{2.082852in}}%
\pgfpathlineto{\pgfqpoint{4.186106in}{2.082852in}}%
\pgfusepath{stroke}%
\end{pgfscope}%
\begin{pgfscope}%
\pgfpathrectangle{\pgfqpoint{3.352233in}{1.400000in}}{\pgfqpoint{2.407767in}{1.544118in}}%
\pgfusepath{clip}%
\pgfsetbuttcap%
\pgfsetroundjoin%
\pgfsetlinewidth{0.501875pt}%
\definecolor{currentstroke}{rgb}{0.278791,0.062145,0.386592}%
\pgfsetstrokecolor{currentstroke}%
\pgfsetdash{}{0pt}%
\pgfpathmoveto{\pgfqpoint{4.176855in}{1.720359in}}%
\pgfpathlineto{\pgfqpoint{4.225988in}{1.731593in}}%
\pgfusepath{stroke}%
\end{pgfscope}%
\begin{pgfscope}%
\pgfpathrectangle{\pgfqpoint{3.352233in}{1.400000in}}{\pgfqpoint{2.407767in}{1.544118in}}%
\pgfusepath{clip}%
\pgfsetbuttcap%
\pgfsetroundjoin%
\pgfsetlinewidth{0.501875pt}%
\definecolor{currentstroke}{rgb}{0.276022,0.044167,0.370164}%
\pgfsetstrokecolor{currentstroke}%
\pgfsetdash{}{0pt}%
\pgfpathmoveto{\pgfqpoint{4.225988in}{1.731593in}}%
\pgfpathlineto{\pgfqpoint{4.270942in}{1.744368in}}%
\pgfusepath{stroke}%
\end{pgfscope}%
\begin{pgfscope}%
\pgfpathrectangle{\pgfqpoint{3.352233in}{1.400000in}}{\pgfqpoint{2.407767in}{1.544118in}}%
\pgfusepath{clip}%
\pgfsetbuttcap%
\pgfsetroundjoin%
\pgfsetlinewidth{0.501875pt}%
\definecolor{currentstroke}{rgb}{0.277941,0.056324,0.381191}%
\pgfsetstrokecolor{currentstroke}%
\pgfsetdash{}{0pt}%
\pgfpathmoveto{\pgfqpoint{4.270942in}{1.744368in}}%
\pgfpathlineto{\pgfqpoint{4.270942in}{1.744368in}}%
\pgfusepath{stroke}%
\end{pgfscope}%
\begin{pgfscope}%
\pgfpathrectangle{\pgfqpoint{3.352233in}{1.400000in}}{\pgfqpoint{2.407767in}{1.544118in}}%
\pgfusepath{clip}%
\pgfsetbuttcap%
\pgfsetroundjoin%
\pgfsetlinewidth{0.501875pt}%
\definecolor{currentstroke}{rgb}{0.277941,0.056324,0.381191}%
\pgfsetstrokecolor{currentstroke}%
\pgfsetdash{}{0pt}%
\pgfpathmoveto{\pgfqpoint{4.270942in}{1.744368in}}%
\pgfpathlineto{\pgfqpoint{4.294139in}{1.753362in}}%
\pgfusepath{stroke}%
\end{pgfscope}%
\begin{pgfscope}%
\pgfpathrectangle{\pgfqpoint{3.352233in}{1.400000in}}{\pgfqpoint{2.407767in}{1.544118in}}%
\pgfusepath{clip}%
\pgfsetbuttcap%
\pgfsetroundjoin%
\pgfsetlinewidth{0.501875pt}%
\definecolor{currentstroke}{rgb}{0.277941,0.056324,0.381191}%
\pgfsetstrokecolor{currentstroke}%
\pgfsetdash{}{0pt}%
\pgfpathmoveto{\pgfqpoint{4.294139in}{1.753362in}}%
\pgfpathlineto{\pgfqpoint{4.311095in}{1.766208in}}%
\pgfusepath{stroke}%
\end{pgfscope}%
\begin{pgfscope}%
\pgfpathrectangle{\pgfqpoint{3.352233in}{1.400000in}}{\pgfqpoint{2.407767in}{1.544118in}}%
\pgfusepath{clip}%
\pgfsetbuttcap%
\pgfsetroundjoin%
\pgfsetlinewidth{0.501875pt}%
\definecolor{currentstroke}{rgb}{0.277941,0.056324,0.381191}%
\pgfsetstrokecolor{currentstroke}%
\pgfsetdash{}{0pt}%
\pgfpathmoveto{\pgfqpoint{4.311095in}{1.766208in}}%
\pgfpathlineto{\pgfqpoint{4.326026in}{1.782293in}}%
\pgfusepath{stroke}%
\end{pgfscope}%
\begin{pgfscope}%
\pgfpathrectangle{\pgfqpoint{3.352233in}{1.400000in}}{\pgfqpoint{2.407767in}{1.544118in}}%
\pgfusepath{clip}%
\pgfsetbuttcap%
\pgfsetroundjoin%
\pgfsetlinewidth{0.501875pt}%
\definecolor{currentstroke}{rgb}{0.277941,0.056324,0.381191}%
\pgfsetstrokecolor{currentstroke}%
\pgfsetdash{}{0pt}%
\pgfpathmoveto{\pgfqpoint{4.326026in}{1.782293in}}%
\pgfpathlineto{\pgfqpoint{4.326026in}{1.782293in}}%
\pgfusepath{stroke}%
\end{pgfscope}%
\begin{pgfscope}%
\pgfpathrectangle{\pgfqpoint{3.352233in}{1.400000in}}{\pgfqpoint{2.407767in}{1.544118in}}%
\pgfusepath{clip}%
\pgfsetbuttcap%
\pgfsetroundjoin%
\pgfsetlinewidth{0.501875pt}%
\definecolor{currentstroke}{rgb}{0.277941,0.056324,0.381191}%
\pgfsetstrokecolor{currentstroke}%
\pgfsetdash{}{0pt}%
\pgfpathmoveto{\pgfqpoint{4.326026in}{1.782293in}}%
\pgfpathlineto{\pgfqpoint{4.326026in}{1.782293in}}%
\pgfusepath{stroke}%
\end{pgfscope}%
\begin{pgfscope}%
\pgfpathrectangle{\pgfqpoint{3.352233in}{1.400000in}}{\pgfqpoint{2.407767in}{1.544118in}}%
\pgfusepath{clip}%
\pgfsetbuttcap%
\pgfsetroundjoin%
\pgfsetlinewidth{0.501875pt}%
\definecolor{currentstroke}{rgb}{0.277941,0.056324,0.381191}%
\pgfsetstrokecolor{currentstroke}%
\pgfsetdash{}{0pt}%
\pgfpathmoveto{\pgfqpoint{4.326026in}{1.782293in}}%
\pgfpathlineto{\pgfqpoint{4.331874in}{1.795261in}}%
\pgfusepath{stroke}%
\end{pgfscope}%
\begin{pgfscope}%
\pgfpathrectangle{\pgfqpoint{3.352233in}{1.400000in}}{\pgfqpoint{2.407767in}{1.544118in}}%
\pgfusepath{clip}%
\pgfsetbuttcap%
\pgfsetroundjoin%
\pgfsetlinewidth{0.501875pt}%
\definecolor{currentstroke}{rgb}{0.279566,0.067836,0.391917}%
\pgfsetstrokecolor{currentstroke}%
\pgfsetdash{}{0pt}%
\pgfpathmoveto{\pgfqpoint{4.331874in}{1.795261in}}%
\pgfpathlineto{\pgfqpoint{4.337624in}{1.808609in}}%
\pgfusepath{stroke}%
\end{pgfscope}%
\begin{pgfscope}%
\pgfpathrectangle{\pgfqpoint{3.352233in}{1.400000in}}{\pgfqpoint{2.407767in}{1.544118in}}%
\pgfusepath{clip}%
\pgfsetbuttcap%
\pgfsetroundjoin%
\pgfsetlinewidth{0.501875pt}%
\definecolor{currentstroke}{rgb}{0.277941,0.056324,0.381191}%
\pgfsetstrokecolor{currentstroke}%
\pgfsetdash{}{0pt}%
\pgfpathmoveto{\pgfqpoint{4.337624in}{1.808609in}}%
\pgfpathlineto{\pgfqpoint{4.337624in}{1.808609in}}%
\pgfusepath{stroke}%
\end{pgfscope}%
\begin{pgfscope}%
\pgfpathrectangle{\pgfqpoint{3.352233in}{1.400000in}}{\pgfqpoint{2.407767in}{1.544118in}}%
\pgfusepath{clip}%
\pgfsetbuttcap%
\pgfsetroundjoin%
\pgfsetlinewidth{0.501875pt}%
\definecolor{currentstroke}{rgb}{0.277941,0.056324,0.381191}%
\pgfsetstrokecolor{currentstroke}%
\pgfsetdash{}{0pt}%
\pgfpathmoveto{\pgfqpoint{4.337624in}{1.808609in}}%
\pgfpathlineto{\pgfqpoint{4.337624in}{1.808609in}}%
\pgfusepath{stroke}%
\end{pgfscope}%
\begin{pgfscope}%
\pgfpathrectangle{\pgfqpoint{3.352233in}{1.400000in}}{\pgfqpoint{2.407767in}{1.544118in}}%
\pgfusepath{clip}%
\pgfsetbuttcap%
\pgfsetroundjoin%
\pgfsetlinewidth{0.501875pt}%
\definecolor{currentstroke}{rgb}{0.277941,0.056324,0.381191}%
\pgfsetstrokecolor{currentstroke}%
\pgfsetdash{}{0pt}%
\pgfpathmoveto{\pgfqpoint{4.337624in}{1.808609in}}%
\pgfpathlineto{\pgfqpoint{4.337068in}{1.823940in}}%
\pgfusepath{stroke}%
\end{pgfscope}%
\begin{pgfscope}%
\pgfpathrectangle{\pgfqpoint{3.352233in}{1.400000in}}{\pgfqpoint{2.407767in}{1.544118in}}%
\pgfusepath{clip}%
\pgfsetbuttcap%
\pgfsetroundjoin%
\pgfsetlinewidth{0.501875pt}%
\definecolor{currentstroke}{rgb}{0.278791,0.062145,0.386592}%
\pgfsetstrokecolor{currentstroke}%
\pgfsetdash{}{0pt}%
\pgfpathmoveto{\pgfqpoint{4.337068in}{1.823940in}}%
\pgfpathlineto{\pgfqpoint{4.332889in}{1.838192in}}%
\pgfusepath{stroke}%
\end{pgfscope}%
\begin{pgfscope}%
\pgfpathrectangle{\pgfqpoint{3.352233in}{1.400000in}}{\pgfqpoint{2.407767in}{1.544118in}}%
\pgfusepath{clip}%
\pgfsetbuttcap%
\pgfsetroundjoin%
\pgfsetlinewidth{0.501875pt}%
\definecolor{currentstroke}{rgb}{0.277018,0.050344,0.375715}%
\pgfsetstrokecolor{currentstroke}%
\pgfsetdash{}{0pt}%
\pgfpathmoveto{\pgfqpoint{4.980754in}{1.704040in}}%
\pgfpathlineto{\pgfqpoint{4.927819in}{1.705253in}}%
\pgfusepath{stroke}%
\end{pgfscope}%
\begin{pgfscope}%
\pgfpathrectangle{\pgfqpoint{3.352233in}{1.400000in}}{\pgfqpoint{2.407767in}{1.544118in}}%
\pgfusepath{clip}%
\pgfsetbuttcap%
\pgfsetroundjoin%
\pgfsetlinewidth{0.501875pt}%
\definecolor{currentstroke}{rgb}{0.277018,0.050344,0.375715}%
\pgfsetstrokecolor{currentstroke}%
\pgfsetdash{}{0pt}%
\pgfpathmoveto{\pgfqpoint{4.927819in}{1.705253in}}%
\pgfpathlineto{\pgfqpoint{4.874915in}{1.706891in}}%
\pgfusepath{stroke}%
\end{pgfscope}%
\begin{pgfscope}%
\pgfpathrectangle{\pgfqpoint{3.352233in}{1.400000in}}{\pgfqpoint{2.407767in}{1.544118in}}%
\pgfusepath{clip}%
\pgfsetbuttcap%
\pgfsetroundjoin%
\pgfsetlinewidth{0.501875pt}%
\definecolor{currentstroke}{rgb}{0.278791,0.062145,0.386592}%
\pgfsetstrokecolor{currentstroke}%
\pgfsetdash{}{0pt}%
\pgfpathmoveto{\pgfqpoint{4.874915in}{1.706891in}}%
\pgfpathlineto{\pgfqpoint{4.822051in}{1.709073in}}%
\pgfusepath{stroke}%
\end{pgfscope}%
\begin{pgfscope}%
\pgfpathrectangle{\pgfqpoint{3.352233in}{1.400000in}}{\pgfqpoint{2.407767in}{1.544118in}}%
\pgfusepath{clip}%
\pgfsetbuttcap%
\pgfsetroundjoin%
\pgfsetlinewidth{0.501875pt}%
\definecolor{currentstroke}{rgb}{0.277941,0.056324,0.381191}%
\pgfsetstrokecolor{currentstroke}%
\pgfsetdash{}{0pt}%
\pgfpathmoveto{\pgfqpoint{4.822051in}{1.709073in}}%
\pgfpathlineto{\pgfqpoint{4.769170in}{1.710955in}}%
\pgfusepath{stroke}%
\end{pgfscope}%
\begin{pgfscope}%
\pgfpathrectangle{\pgfqpoint{3.352233in}{1.400000in}}{\pgfqpoint{2.407767in}{1.544118in}}%
\pgfusepath{clip}%
\pgfsetbuttcap%
\pgfsetroundjoin%
\pgfsetlinewidth{0.501875pt}%
\definecolor{currentstroke}{rgb}{0.277018,0.050344,0.375715}%
\pgfsetstrokecolor{currentstroke}%
\pgfsetdash{}{0pt}%
\pgfpathmoveto{\pgfqpoint{4.769170in}{1.710955in}}%
\pgfpathlineto{\pgfqpoint{4.716495in}{1.714136in}}%
\pgfusepath{stroke}%
\end{pgfscope}%
\begin{pgfscope}%
\pgfpathrectangle{\pgfqpoint{3.352233in}{1.400000in}}{\pgfqpoint{2.407767in}{1.544118in}}%
\pgfusepath{clip}%
\pgfsetbuttcap%
\pgfsetroundjoin%
\pgfsetlinewidth{0.501875pt}%
\definecolor{currentstroke}{rgb}{0.274952,0.037752,0.364543}%
\pgfsetstrokecolor{currentstroke}%
\pgfsetdash{}{0pt}%
\pgfpathmoveto{\pgfqpoint{4.716495in}{1.714136in}}%
\pgfpathlineto{\pgfqpoint{4.664477in}{1.720359in}}%
\pgfusepath{stroke}%
\end{pgfscope}%
\begin{pgfscope}%
\pgfpathrectangle{\pgfqpoint{3.352233in}{1.400000in}}{\pgfqpoint{2.407767in}{1.544118in}}%
\pgfusepath{clip}%
\pgfsetbuttcap%
\pgfsetroundjoin%
\pgfsetlinewidth{0.501875pt}%
\definecolor{currentstroke}{rgb}{0.277941,0.056324,0.381191}%
\pgfsetstrokecolor{currentstroke}%
\pgfsetdash{}{0pt}%
\pgfpathmoveto{\pgfqpoint{4.664477in}{1.720359in}}%
\pgfpathlineto{\pgfqpoint{4.613674in}{1.729628in}}%
\pgfusepath{stroke}%
\end{pgfscope}%
\begin{pgfscope}%
\pgfpathrectangle{\pgfqpoint{3.352233in}{1.400000in}}{\pgfqpoint{2.407767in}{1.544118in}}%
\pgfusepath{clip}%
\pgfsetbuttcap%
\pgfsetroundjoin%
\pgfsetlinewidth{0.501875pt}%
\definecolor{currentstroke}{rgb}{0.276022,0.044167,0.370164}%
\pgfsetstrokecolor{currentstroke}%
\pgfsetdash{}{0pt}%
\pgfpathmoveto{\pgfqpoint{4.613674in}{1.729628in}}%
\pgfpathlineto{\pgfqpoint{4.563880in}{1.740940in}}%
\pgfusepath{stroke}%
\end{pgfscope}%
\begin{pgfscope}%
\pgfpathrectangle{\pgfqpoint{3.352233in}{1.400000in}}{\pgfqpoint{2.407767in}{1.544118in}}%
\pgfusepath{clip}%
\pgfsetbuttcap%
\pgfsetroundjoin%
\pgfsetlinewidth{0.501875pt}%
\definecolor{currentstroke}{rgb}{0.276022,0.044167,0.370164}%
\pgfsetstrokecolor{currentstroke}%
\pgfsetdash{}{0pt}%
\pgfpathmoveto{\pgfqpoint{4.563880in}{1.740940in}}%
\pgfpathlineto{\pgfqpoint{4.515301in}{1.754122in}}%
\pgfusepath{stroke}%
\end{pgfscope}%
\begin{pgfscope}%
\pgfpathrectangle{\pgfqpoint{3.352233in}{1.400000in}}{\pgfqpoint{2.407767in}{1.544118in}}%
\pgfusepath{clip}%
\pgfsetbuttcap%
\pgfsetroundjoin%
\pgfsetlinewidth{0.501875pt}%
\definecolor{currentstroke}{rgb}{0.277018,0.050344,0.375715}%
\pgfsetstrokecolor{currentstroke}%
\pgfsetdash{}{0pt}%
\pgfpathmoveto{\pgfqpoint{4.515301in}{1.754122in}}%
\pgfpathlineto{\pgfqpoint{4.515301in}{1.754122in}}%
\pgfusepath{stroke}%
\end{pgfscope}%
\begin{pgfscope}%
\pgfpathrectangle{\pgfqpoint{3.352233in}{1.400000in}}{\pgfqpoint{2.407767in}{1.544118in}}%
\pgfusepath{clip}%
\pgfsetbuttcap%
\pgfsetroundjoin%
\pgfsetlinewidth{0.501875pt}%
\definecolor{currentstroke}{rgb}{0.277018,0.050344,0.375715}%
\pgfsetstrokecolor{currentstroke}%
\pgfsetdash{}{0pt}%
\pgfpathmoveto{\pgfqpoint{4.515301in}{1.754122in}}%
\pgfpathlineto{\pgfqpoint{4.483929in}{1.766230in}}%
\pgfusepath{stroke}%
\end{pgfscope}%
\begin{pgfscope}%
\pgfpathrectangle{\pgfqpoint{3.352233in}{1.400000in}}{\pgfqpoint{2.407767in}{1.544118in}}%
\pgfusepath{clip}%
\pgfsetbuttcap%
\pgfsetroundjoin%
\pgfsetlinewidth{0.501875pt}%
\definecolor{currentstroke}{rgb}{0.268510,0.009605,0.335427}%
\pgfsetstrokecolor{currentstroke}%
\pgfsetdash{}{0pt}%
\pgfpathmoveto{\pgfqpoint{5.206279in}{2.623758in}}%
\pgfpathlineto{\pgfqpoint{5.153362in}{2.623953in}}%
\pgfusepath{stroke}%
\end{pgfscope}%
\begin{pgfscope}%
\pgfpathrectangle{\pgfqpoint{3.352233in}{1.400000in}}{\pgfqpoint{2.407767in}{1.544118in}}%
\pgfusepath{clip}%
\pgfsetbuttcap%
\pgfsetroundjoin%
\pgfsetlinewidth{0.501875pt}%
\definecolor{currentstroke}{rgb}{0.271305,0.019942,0.347269}%
\pgfsetstrokecolor{currentstroke}%
\pgfsetdash{}{0pt}%
\pgfpathmoveto{\pgfqpoint{5.153362in}{2.623953in}}%
\pgfpathlineto{\pgfqpoint{5.100440in}{2.624452in}}%
\pgfusepath{stroke}%
\end{pgfscope}%
\begin{pgfscope}%
\pgfpathrectangle{\pgfqpoint{3.352233in}{1.400000in}}{\pgfqpoint{2.407767in}{1.544118in}}%
\pgfusepath{clip}%
\pgfsetbuttcap%
\pgfsetroundjoin%
\pgfsetlinewidth{0.501875pt}%
\definecolor{currentstroke}{rgb}{0.272594,0.025563,0.353093}%
\pgfsetstrokecolor{currentstroke}%
\pgfsetdash{}{0pt}%
\pgfpathmoveto{\pgfqpoint{5.100440in}{2.624452in}}%
\pgfpathlineto{\pgfqpoint{5.047470in}{2.624352in}}%
\pgfusepath{stroke}%
\end{pgfscope}%
\begin{pgfscope}%
\pgfpathrectangle{\pgfqpoint{3.352233in}{1.400000in}}{\pgfqpoint{2.407767in}{1.544118in}}%
\pgfusepath{clip}%
\pgfsetbuttcap%
\pgfsetroundjoin%
\pgfsetlinewidth{0.501875pt}%
\definecolor{currentstroke}{rgb}{0.273809,0.031497,0.358853}%
\pgfsetstrokecolor{currentstroke}%
\pgfsetdash{}{0pt}%
\pgfpathmoveto{\pgfqpoint{5.047470in}{2.624352in}}%
\pgfpathlineto{\pgfqpoint{4.994507in}{2.623765in}}%
\pgfusepath{stroke}%
\end{pgfscope}%
\begin{pgfscope}%
\pgfpathrectangle{\pgfqpoint{3.352233in}{1.400000in}}{\pgfqpoint{2.407767in}{1.544118in}}%
\pgfusepath{clip}%
\pgfsetbuttcap%
\pgfsetroundjoin%
\pgfsetlinewidth{0.501875pt}%
\definecolor{currentstroke}{rgb}{0.276022,0.044167,0.370164}%
\pgfsetstrokecolor{currentstroke}%
\pgfsetdash{}{0pt}%
\pgfpathmoveto{\pgfqpoint{4.994507in}{2.623765in}}%
\pgfpathlineto{\pgfqpoint{4.941573in}{2.622602in}}%
\pgfusepath{stroke}%
\end{pgfscope}%
\begin{pgfscope}%
\pgfpathrectangle{\pgfqpoint{3.352233in}{1.400000in}}{\pgfqpoint{2.407767in}{1.544118in}}%
\pgfusepath{clip}%
\pgfsetbuttcap%
\pgfsetroundjoin%
\pgfsetlinewidth{0.501875pt}%
\definecolor{currentstroke}{rgb}{0.276022,0.044167,0.370164}%
\pgfsetstrokecolor{currentstroke}%
\pgfsetdash{}{0pt}%
\pgfpathmoveto{\pgfqpoint{4.941573in}{2.622602in}}%
\pgfpathlineto{\pgfqpoint{4.888643in}{2.621485in}}%
\pgfusepath{stroke}%
\end{pgfscope}%
\begin{pgfscope}%
\pgfpathrectangle{\pgfqpoint{3.352233in}{1.400000in}}{\pgfqpoint{2.407767in}{1.544118in}}%
\pgfusepath{clip}%
\pgfsetbuttcap%
\pgfsetroundjoin%
\pgfsetlinewidth{0.501875pt}%
\definecolor{currentstroke}{rgb}{0.277018,0.050344,0.375715}%
\pgfsetstrokecolor{currentstroke}%
\pgfsetdash{}{0pt}%
\pgfpathmoveto{\pgfqpoint{4.888643in}{2.621485in}}%
\pgfpathlineto{\pgfqpoint{4.835718in}{2.620460in}}%
\pgfusepath{stroke}%
\end{pgfscope}%
\begin{pgfscope}%
\pgfpathrectangle{\pgfqpoint{3.352233in}{1.400000in}}{\pgfqpoint{2.407767in}{1.544118in}}%
\pgfusepath{clip}%
\pgfsetbuttcap%
\pgfsetroundjoin%
\pgfsetlinewidth{0.501875pt}%
\definecolor{currentstroke}{rgb}{0.276022,0.044167,0.370164}%
\pgfsetstrokecolor{currentstroke}%
\pgfsetdash{}{0pt}%
\pgfpathmoveto{\pgfqpoint{4.835718in}{2.620460in}}%
\pgfpathlineto{\pgfqpoint{4.782846in}{2.618752in}}%
\pgfusepath{stroke}%
\end{pgfscope}%
\begin{pgfscope}%
\pgfpathrectangle{\pgfqpoint{3.352233in}{1.400000in}}{\pgfqpoint{2.407767in}{1.544118in}}%
\pgfusepath{clip}%
\pgfsetbuttcap%
\pgfsetroundjoin%
\pgfsetlinewidth{0.501875pt}%
\definecolor{currentstroke}{rgb}{0.277018,0.050344,0.375715}%
\pgfsetstrokecolor{currentstroke}%
\pgfsetdash{}{0pt}%
\pgfpathmoveto{\pgfqpoint{4.782846in}{2.618752in}}%
\pgfpathlineto{\pgfqpoint{4.730347in}{2.615280in}}%
\pgfusepath{stroke}%
\end{pgfscope}%
\begin{pgfscope}%
\pgfpathrectangle{\pgfqpoint{3.352233in}{1.400000in}}{\pgfqpoint{2.407767in}{1.544118in}}%
\pgfusepath{clip}%
\pgfsetbuttcap%
\pgfsetroundjoin%
\pgfsetlinewidth{0.501875pt}%
\definecolor{currentstroke}{rgb}{0.269944,0.014625,0.341379}%
\pgfsetstrokecolor{currentstroke}%
\pgfsetdash{}{0pt}%
\pgfpathmoveto{\pgfqpoint{4.393576in}{2.623758in}}%
\pgfpathlineto{\pgfqpoint{4.393576in}{2.623758in}}%
\pgfusepath{stroke}%
\end{pgfscope}%
\begin{pgfscope}%
\pgfpathrectangle{\pgfqpoint{3.352233in}{1.400000in}}{\pgfqpoint{2.407767in}{1.544118in}}%
\pgfusepath{clip}%
\pgfsetbuttcap%
\pgfsetroundjoin%
\pgfsetlinewidth{0.501875pt}%
\definecolor{currentstroke}{rgb}{0.269944,0.014625,0.341379}%
\pgfsetstrokecolor{currentstroke}%
\pgfsetdash{}{0pt}%
\pgfpathmoveto{\pgfqpoint{4.393576in}{2.623758in}}%
\pgfpathlineto{\pgfqpoint{4.427917in}{2.610358in}}%
\pgfusepath{stroke}%
\end{pgfscope}%
\begin{pgfscope}%
\pgfpathrectangle{\pgfqpoint{3.352233in}{1.400000in}}{\pgfqpoint{2.407767in}{1.544118in}}%
\pgfusepath{clip}%
\pgfsetbuttcap%
\pgfsetroundjoin%
\pgfsetlinewidth{0.501875pt}%
\definecolor{currentstroke}{rgb}{0.271305,0.019942,0.347269}%
\pgfsetstrokecolor{currentstroke}%
\pgfsetdash{}{0pt}%
\pgfpathmoveto{\pgfqpoint{4.427917in}{2.610358in}}%
\pgfpathlineto{\pgfqpoint{4.446688in}{2.594348in}}%
\pgfusepath{stroke}%
\end{pgfscope}%
\begin{pgfscope}%
\pgfpathrectangle{\pgfqpoint{3.352233in}{1.400000in}}{\pgfqpoint{2.407767in}{1.544118in}}%
\pgfusepath{clip}%
\pgfsetbuttcap%
\pgfsetroundjoin%
\pgfsetlinewidth{0.501875pt}%
\definecolor{currentstroke}{rgb}{0.271305,0.019942,0.347269}%
\pgfsetstrokecolor{currentstroke}%
\pgfsetdash{}{0pt}%
\pgfpathmoveto{\pgfqpoint{4.446688in}{2.594348in}}%
\pgfpathlineto{\pgfqpoint{4.446688in}{2.594348in}}%
\pgfusepath{stroke}%
\end{pgfscope}%
\begin{pgfscope}%
\pgfpathrectangle{\pgfqpoint{3.352233in}{1.400000in}}{\pgfqpoint{2.407767in}{1.544118in}}%
\pgfusepath{clip}%
\pgfsetbuttcap%
\pgfsetroundjoin%
\pgfsetlinewidth{0.501875pt}%
\definecolor{currentstroke}{rgb}{0.271305,0.019942,0.347269}%
\pgfsetstrokecolor{currentstroke}%
\pgfsetdash{}{0pt}%
\pgfpathmoveto{\pgfqpoint{4.446688in}{2.594348in}}%
\pgfpathlineto{\pgfqpoint{4.446688in}{2.594348in}}%
\pgfusepath{stroke}%
\end{pgfscope}%
\begin{pgfscope}%
\pgfpathrectangle{\pgfqpoint{3.352233in}{1.400000in}}{\pgfqpoint{2.407767in}{1.544118in}}%
\pgfusepath{clip}%
\pgfsetbuttcap%
\pgfsetroundjoin%
\pgfsetlinewidth{0.501875pt}%
\definecolor{currentstroke}{rgb}{0.271305,0.019942,0.347269}%
\pgfsetstrokecolor{currentstroke}%
\pgfsetdash{}{0pt}%
\pgfpathmoveto{\pgfqpoint{4.446688in}{2.594348in}}%
\pgfpathlineto{\pgfqpoint{4.446688in}{2.594348in}}%
\pgfusepath{stroke}%
\end{pgfscope}%
\begin{pgfscope}%
\pgfpathrectangle{\pgfqpoint{3.352233in}{1.400000in}}{\pgfqpoint{2.407767in}{1.544118in}}%
\pgfusepath{clip}%
\pgfsetbuttcap%
\pgfsetroundjoin%
\pgfsetlinewidth{0.501875pt}%
\definecolor{currentstroke}{rgb}{0.271305,0.019942,0.347269}%
\pgfsetstrokecolor{currentstroke}%
\pgfsetdash{}{0pt}%
\pgfpathmoveto{\pgfqpoint{4.446688in}{2.594348in}}%
\pgfpathlineto{\pgfqpoint{4.445243in}{2.585828in}}%
\pgfusepath{stroke}%
\end{pgfscope}%
\begin{pgfscope}%
\pgfpathrectangle{\pgfqpoint{3.352233in}{1.400000in}}{\pgfqpoint{2.407767in}{1.544118in}}%
\pgfusepath{clip}%
\pgfsetbuttcap%
\pgfsetroundjoin%
\pgfsetlinewidth{0.501875pt}%
\definecolor{currentstroke}{rgb}{0.271305,0.019942,0.347269}%
\pgfsetstrokecolor{currentstroke}%
\pgfsetdash{}{0pt}%
\pgfpathmoveto{\pgfqpoint{4.445243in}{2.585828in}}%
\pgfpathlineto{\pgfqpoint{4.445747in}{2.577604in}}%
\pgfusepath{stroke}%
\end{pgfscope}%
\begin{pgfscope}%
\pgfpathrectangle{\pgfqpoint{3.352233in}{1.400000in}}{\pgfqpoint{2.407767in}{1.544118in}}%
\pgfusepath{clip}%
\pgfsetbuttcap%
\pgfsetroundjoin%
\pgfsetlinewidth{0.501875pt}%
\definecolor{currentstroke}{rgb}{0.271305,0.019942,0.347269}%
\pgfsetstrokecolor{currentstroke}%
\pgfsetdash{}{0pt}%
\pgfpathmoveto{\pgfqpoint{4.445747in}{2.577604in}}%
\pgfpathlineto{\pgfqpoint{4.451561in}{2.567289in}}%
\pgfusepath{stroke}%
\end{pgfscope}%
\begin{pgfscope}%
\pgfpathrectangle{\pgfqpoint{3.352233in}{1.400000in}}{\pgfqpoint{2.407767in}{1.544118in}}%
\pgfusepath{clip}%
\pgfsetbuttcap%
\pgfsetroundjoin%
\pgfsetlinewidth{0.501875pt}%
\definecolor{currentstroke}{rgb}{0.272594,0.025563,0.353093}%
\pgfsetstrokecolor{currentstroke}%
\pgfsetdash{}{0pt}%
\pgfpathmoveto{\pgfqpoint{4.451561in}{2.567289in}}%
\pgfpathlineto{\pgfqpoint{4.451561in}{2.567289in}}%
\pgfusepath{stroke}%
\end{pgfscope}%
\begin{pgfscope}%
\pgfpathrectangle{\pgfqpoint{3.352233in}{1.400000in}}{\pgfqpoint{2.407767in}{1.544118in}}%
\pgfusepath{clip}%
\pgfsetbuttcap%
\pgfsetroundjoin%
\pgfsetlinewidth{0.501875pt}%
\definecolor{currentstroke}{rgb}{0.272594,0.025563,0.353093}%
\pgfsetstrokecolor{currentstroke}%
\pgfsetdash{}{0pt}%
\pgfpathmoveto{\pgfqpoint{4.451561in}{2.567289in}}%
\pgfpathlineto{\pgfqpoint{4.451327in}{2.558377in}}%
\pgfusepath{stroke}%
\end{pgfscope}%
\begin{pgfscope}%
\pgfpathrectangle{\pgfqpoint{3.352233in}{1.400000in}}{\pgfqpoint{2.407767in}{1.544118in}}%
\pgfusepath{clip}%
\pgfsetbuttcap%
\pgfsetroundjoin%
\pgfsetlinewidth{0.501875pt}%
\definecolor{currentstroke}{rgb}{0.272594,0.025563,0.353093}%
\pgfsetstrokecolor{currentstroke}%
\pgfsetdash{}{0pt}%
\pgfpathmoveto{\pgfqpoint{4.451327in}{2.558377in}}%
\pgfpathlineto{\pgfqpoint{4.450097in}{2.550388in}}%
\pgfusepath{stroke}%
\end{pgfscope}%
\begin{pgfscope}%
\pgfpathrectangle{\pgfqpoint{3.352233in}{1.400000in}}{\pgfqpoint{2.407767in}{1.544118in}}%
\pgfusepath{clip}%
\pgfsetbuttcap%
\pgfsetroundjoin%
\pgfsetlinewidth{0.501875pt}%
\definecolor{currentstroke}{rgb}{0.273809,0.031497,0.358853}%
\pgfsetstrokecolor{currentstroke}%
\pgfsetdash{}{0pt}%
\pgfpathmoveto{\pgfqpoint{4.450097in}{2.550388in}}%
\pgfpathlineto{\pgfqpoint{4.450097in}{2.550388in}}%
\pgfusepath{stroke}%
\end{pgfscope}%
\begin{pgfscope}%
\pgfpathrectangle{\pgfqpoint{3.352233in}{1.400000in}}{\pgfqpoint{2.407767in}{1.544118in}}%
\pgfusepath{clip}%
\pgfsetbuttcap%
\pgfsetroundjoin%
\pgfsetlinewidth{0.501875pt}%
\definecolor{currentstroke}{rgb}{0.273809,0.031497,0.358853}%
\pgfsetstrokecolor{currentstroke}%
\pgfsetdash{}{0pt}%
\pgfpathmoveto{\pgfqpoint{4.450097in}{2.550388in}}%
\pgfpathlineto{\pgfqpoint{4.450097in}{2.550388in}}%
\pgfusepath{stroke}%
\end{pgfscope}%
\begin{pgfscope}%
\pgfpathrectangle{\pgfqpoint{3.352233in}{1.400000in}}{\pgfqpoint{2.407767in}{1.544118in}}%
\pgfusepath{clip}%
\pgfsetbuttcap%
\pgfsetroundjoin%
\pgfsetlinewidth{0.501875pt}%
\definecolor{currentstroke}{rgb}{0.273809,0.031497,0.358853}%
\pgfsetstrokecolor{currentstroke}%
\pgfsetdash{}{0pt}%
\pgfpathmoveto{\pgfqpoint{4.450097in}{2.550388in}}%
\pgfpathlineto{\pgfqpoint{4.455275in}{2.539012in}}%
\pgfusepath{stroke}%
\end{pgfscope}%
\begin{pgfscope}%
\pgfpathrectangle{\pgfqpoint{3.352233in}{1.400000in}}{\pgfqpoint{2.407767in}{1.544118in}}%
\pgfusepath{clip}%
\pgfsetbuttcap%
\pgfsetroundjoin%
\pgfsetlinewidth{0.501875pt}%
\definecolor{currentstroke}{rgb}{0.273809,0.031497,0.358853}%
\pgfsetstrokecolor{currentstroke}%
\pgfsetdash{}{0pt}%
\pgfpathmoveto{\pgfqpoint{4.455275in}{2.539012in}}%
\pgfpathlineto{\pgfqpoint{4.455275in}{2.539012in}}%
\pgfusepath{stroke}%
\end{pgfscope}%
\begin{pgfscope}%
\pgfpathrectangle{\pgfqpoint{3.352233in}{1.400000in}}{\pgfqpoint{2.407767in}{1.544118in}}%
\pgfusepath{clip}%
\pgfsetbuttcap%
\pgfsetroundjoin%
\pgfsetlinewidth{0.501875pt}%
\definecolor{currentstroke}{rgb}{0.273809,0.031497,0.358853}%
\pgfsetstrokecolor{currentstroke}%
\pgfsetdash{}{0pt}%
\pgfpathmoveto{\pgfqpoint{4.455275in}{2.539012in}}%
\pgfpathlineto{\pgfqpoint{4.457489in}{2.529705in}}%
\pgfusepath{stroke}%
\end{pgfscope}%
\begin{pgfscope}%
\pgfpathrectangle{\pgfqpoint{3.352233in}{1.400000in}}{\pgfqpoint{2.407767in}{1.544118in}}%
\pgfusepath{clip}%
\pgfsetbuttcap%
\pgfsetroundjoin%
\pgfsetlinewidth{0.501875pt}%
\definecolor{currentstroke}{rgb}{0.273809,0.031497,0.358853}%
\pgfsetstrokecolor{currentstroke}%
\pgfsetdash{}{0pt}%
\pgfpathmoveto{\pgfqpoint{4.457489in}{2.529705in}}%
\pgfpathlineto{\pgfqpoint{4.453727in}{2.520647in}}%
\pgfusepath{stroke}%
\end{pgfscope}%
\begin{pgfscope}%
\pgfpathrectangle{\pgfqpoint{3.352233in}{1.400000in}}{\pgfqpoint{2.407767in}{1.544118in}}%
\pgfusepath{clip}%
\pgfsetbuttcap%
\pgfsetroundjoin%
\pgfsetlinewidth{0.501875pt}%
\definecolor{currentstroke}{rgb}{0.272594,0.025563,0.353093}%
\pgfsetstrokecolor{currentstroke}%
\pgfsetdash{}{0pt}%
\pgfpathmoveto{\pgfqpoint{4.453727in}{2.520647in}}%
\pgfpathlineto{\pgfqpoint{4.448319in}{2.513417in}}%
\pgfusepath{stroke}%
\end{pgfscope}%
\begin{pgfscope}%
\pgfpathrectangle{\pgfqpoint{3.352233in}{1.400000in}}{\pgfqpoint{2.407767in}{1.544118in}}%
\pgfusepath{clip}%
\pgfsetbuttcap%
\pgfsetroundjoin%
\pgfsetlinewidth{0.501875pt}%
\definecolor{currentstroke}{rgb}{0.273809,0.031497,0.358853}%
\pgfsetstrokecolor{currentstroke}%
\pgfsetdash{}{0pt}%
\pgfpathmoveto{\pgfqpoint{4.448319in}{2.513417in}}%
\pgfpathlineto{\pgfqpoint{4.439198in}{2.502467in}}%
\pgfusepath{stroke}%
\end{pgfscope}%
\begin{pgfscope}%
\pgfpathrectangle{\pgfqpoint{3.352233in}{1.400000in}}{\pgfqpoint{2.407767in}{1.544118in}}%
\pgfusepath{clip}%
\pgfsetbuttcap%
\pgfsetroundjoin%
\pgfsetlinewidth{0.501875pt}%
\definecolor{currentstroke}{rgb}{0.274952,0.037752,0.364543}%
\pgfsetstrokecolor{currentstroke}%
\pgfsetdash{}{0pt}%
\pgfpathmoveto{\pgfqpoint{4.439198in}{2.502467in}}%
\pgfpathlineto{\pgfqpoint{4.439198in}{2.502467in}}%
\pgfusepath{stroke}%
\end{pgfscope}%
\begin{pgfscope}%
\pgfpathrectangle{\pgfqpoint{3.352233in}{1.400000in}}{\pgfqpoint{2.407767in}{1.544118in}}%
\pgfusepath{clip}%
\pgfsetbuttcap%
\pgfsetroundjoin%
\pgfsetlinewidth{0.501875pt}%
\definecolor{currentstroke}{rgb}{0.274952,0.037752,0.364543}%
\pgfsetstrokecolor{currentstroke}%
\pgfsetdash{}{0pt}%
\pgfpathmoveto{\pgfqpoint{4.439198in}{2.502467in}}%
\pgfpathlineto{\pgfqpoint{4.433808in}{2.488711in}}%
\pgfusepath{stroke}%
\end{pgfscope}%
\begin{pgfscope}%
\pgfpathrectangle{\pgfqpoint{3.352233in}{1.400000in}}{\pgfqpoint{2.407767in}{1.544118in}}%
\pgfusepath{clip}%
\pgfsetbuttcap%
\pgfsetroundjoin%
\pgfsetlinewidth{0.501875pt}%
\definecolor{currentstroke}{rgb}{0.274952,0.037752,0.364543}%
\pgfsetstrokecolor{currentstroke}%
\pgfsetdash{}{0pt}%
\pgfpathmoveto{\pgfqpoint{4.433808in}{2.488711in}}%
\pgfpathlineto{\pgfqpoint{4.432916in}{2.474515in}}%
\pgfusepath{stroke}%
\end{pgfscope}%
\begin{pgfscope}%
\pgfpathrectangle{\pgfqpoint{3.352233in}{1.400000in}}{\pgfqpoint{2.407767in}{1.544118in}}%
\pgfusepath{clip}%
\pgfsetbuttcap%
\pgfsetroundjoin%
\pgfsetlinewidth{0.501875pt}%
\definecolor{currentstroke}{rgb}{0.274952,0.037752,0.364543}%
\pgfsetstrokecolor{currentstroke}%
\pgfsetdash{}{0pt}%
\pgfpathmoveto{\pgfqpoint{4.432916in}{2.474515in}}%
\pgfpathlineto{\pgfqpoint{4.428212in}{2.459678in}}%
\pgfusepath{stroke}%
\end{pgfscope}%
\begin{pgfscope}%
\pgfpathrectangle{\pgfqpoint{3.352233in}{1.400000in}}{\pgfqpoint{2.407767in}{1.544118in}}%
\pgfusepath{clip}%
\pgfsetbuttcap%
\pgfsetroundjoin%
\pgfsetlinewidth{0.501875pt}%
\definecolor{currentstroke}{rgb}{0.276022,0.044167,0.370164}%
\pgfsetstrokecolor{currentstroke}%
\pgfsetdash{}{0pt}%
\pgfpathmoveto{\pgfqpoint{4.428212in}{2.459678in}}%
\pgfpathlineto{\pgfqpoint{4.420074in}{2.447442in}}%
\pgfusepath{stroke}%
\end{pgfscope}%
\begin{pgfscope}%
\pgfpathrectangle{\pgfqpoint{3.352233in}{1.400000in}}{\pgfqpoint{2.407767in}{1.544118in}}%
\pgfusepath{clip}%
\pgfsetbuttcap%
\pgfsetroundjoin%
\pgfsetlinewidth{0.501875pt}%
\definecolor{currentstroke}{rgb}{0.279566,0.067836,0.391917}%
\pgfsetstrokecolor{currentstroke}%
\pgfsetdash{}{0pt}%
\pgfpathmoveto{\pgfqpoint{4.420074in}{2.447442in}}%
\pgfpathlineto{\pgfqpoint{4.412854in}{2.434877in}}%
\pgfusepath{stroke}%
\end{pgfscope}%
\begin{pgfscope}%
\pgfpathrectangle{\pgfqpoint{3.352233in}{1.400000in}}{\pgfqpoint{2.407767in}{1.544118in}}%
\pgfusepath{clip}%
\pgfsetbuttcap%
\pgfsetroundjoin%
\pgfsetlinewidth{0.501875pt}%
\definecolor{currentstroke}{rgb}{0.277941,0.056324,0.381191}%
\pgfsetstrokecolor{currentstroke}%
\pgfsetdash{}{0pt}%
\pgfpathmoveto{\pgfqpoint{4.412854in}{2.434877in}}%
\pgfpathlineto{\pgfqpoint{4.403979in}{2.419700in}}%
\pgfusepath{stroke}%
\end{pgfscope}%
\begin{pgfscope}%
\pgfpathrectangle{\pgfqpoint{3.352233in}{1.400000in}}{\pgfqpoint{2.407767in}{1.544118in}}%
\pgfusepath{clip}%
\pgfsetbuttcap%
\pgfsetroundjoin%
\pgfsetlinewidth{0.501875pt}%
\definecolor{currentstroke}{rgb}{0.279566,0.067836,0.391917}%
\pgfsetstrokecolor{currentstroke}%
\pgfsetdash{}{0pt}%
\pgfpathmoveto{\pgfqpoint{4.403979in}{2.419700in}}%
\pgfpathlineto{\pgfqpoint{4.390384in}{2.407815in}}%
\pgfusepath{stroke}%
\end{pgfscope}%
\begin{pgfscope}%
\pgfpathrectangle{\pgfqpoint{3.352233in}{1.400000in}}{\pgfqpoint{2.407767in}{1.544118in}}%
\pgfusepath{clip}%
\pgfsetbuttcap%
\pgfsetroundjoin%
\pgfsetlinewidth{0.501875pt}%
\definecolor{currentstroke}{rgb}{0.280267,0.073417,0.397163}%
\pgfsetstrokecolor{currentstroke}%
\pgfsetdash{}{0pt}%
\pgfpathmoveto{\pgfqpoint{4.390384in}{2.407815in}}%
\pgfpathlineto{\pgfqpoint{4.363530in}{2.382667in}}%
\pgfusepath{stroke}%
\end{pgfscope}%
\begin{pgfscope}%
\pgfpathrectangle{\pgfqpoint{3.352233in}{1.400000in}}{\pgfqpoint{2.407767in}{1.544118in}}%
\pgfusepath{clip}%
\pgfsetbuttcap%
\pgfsetroundjoin%
\pgfsetlinewidth{0.501875pt}%
\definecolor{currentstroke}{rgb}{0.281446,0.084320,0.407414}%
\pgfsetstrokecolor{currentstroke}%
\pgfsetdash{}{0pt}%
\pgfpathmoveto{\pgfqpoint{4.363530in}{2.382667in}}%
\pgfpathlineto{\pgfqpoint{4.338602in}{2.353413in}}%
\pgfusepath{stroke}%
\end{pgfscope}%
\begin{pgfscope}%
\pgfpathrectangle{\pgfqpoint{3.352233in}{1.400000in}}{\pgfqpoint{2.407767in}{1.544118in}}%
\pgfusepath{clip}%
\pgfsetbuttcap%
\pgfsetroundjoin%
\pgfsetlinewidth{0.501875pt}%
\definecolor{currentstroke}{rgb}{0.282327,0.094955,0.417331}%
\pgfsetstrokecolor{currentstroke}%
\pgfsetdash{}{0pt}%
\pgfpathmoveto{\pgfqpoint{4.338602in}{2.353413in}}%
\pgfpathlineto{\pgfqpoint{4.322018in}{2.322032in}}%
\pgfusepath{stroke}%
\end{pgfscope}%
\begin{pgfscope}%
\pgfpathrectangle{\pgfqpoint{3.352233in}{1.400000in}}{\pgfqpoint{2.407767in}{1.544118in}}%
\pgfusepath{clip}%
\pgfsetbuttcap%
\pgfsetroundjoin%
\pgfsetlinewidth{0.501875pt}%
\definecolor{currentstroke}{rgb}{0.280267,0.073417,0.397163}%
\pgfsetstrokecolor{currentstroke}%
\pgfsetdash{}{0pt}%
\pgfpathmoveto{\pgfqpoint{4.322018in}{2.322032in}}%
\pgfpathlineto{\pgfqpoint{4.322018in}{2.322032in}}%
\pgfusepath{stroke}%
\end{pgfscope}%
\begin{pgfscope}%
\pgfpathrectangle{\pgfqpoint{3.352233in}{1.400000in}}{\pgfqpoint{2.407767in}{1.544118in}}%
\pgfusepath{clip}%
\pgfsetbuttcap%
\pgfsetroundjoin%
\pgfsetlinewidth{0.501875pt}%
\definecolor{currentstroke}{rgb}{0.280267,0.073417,0.397163}%
\pgfsetstrokecolor{currentstroke}%
\pgfsetdash{}{0pt}%
\pgfpathmoveto{\pgfqpoint{4.322018in}{2.322032in}}%
\pgfpathlineto{\pgfqpoint{4.322018in}{2.322032in}}%
\pgfusepath{stroke}%
\end{pgfscope}%
\begin{pgfscope}%
\pgfpathrectangle{\pgfqpoint{3.352233in}{1.400000in}}{\pgfqpoint{2.407767in}{1.544118in}}%
\pgfusepath{clip}%
\pgfsetbuttcap%
\pgfsetroundjoin%
\pgfsetlinewidth{0.501875pt}%
\definecolor{currentstroke}{rgb}{0.280267,0.073417,0.397163}%
\pgfsetstrokecolor{currentstroke}%
\pgfsetdash{}{0pt}%
\pgfpathmoveto{\pgfqpoint{4.322018in}{2.322032in}}%
\pgfpathlineto{\pgfqpoint{4.308091in}{2.307533in}}%
\pgfusepath{stroke}%
\end{pgfscope}%
\begin{pgfscope}%
\pgfpathrectangle{\pgfqpoint{3.352233in}{1.400000in}}{\pgfqpoint{2.407767in}{1.544118in}}%
\pgfusepath{clip}%
\pgfsetbuttcap%
\pgfsetroundjoin%
\pgfsetlinewidth{0.501875pt}%
\definecolor{currentstroke}{rgb}{0.280894,0.078907,0.402329}%
\pgfsetstrokecolor{currentstroke}%
\pgfsetdash{}{0pt}%
\pgfpathmoveto{\pgfqpoint{4.308091in}{2.307533in}}%
\pgfpathlineto{\pgfqpoint{4.294948in}{2.293393in}}%
\pgfusepath{stroke}%
\end{pgfscope}%
\begin{pgfscope}%
\pgfpathrectangle{\pgfqpoint{3.352233in}{1.400000in}}{\pgfqpoint{2.407767in}{1.544118in}}%
\pgfusepath{clip}%
\pgfsetbuttcap%
\pgfsetroundjoin%
\pgfsetlinewidth{0.501875pt}%
\definecolor{currentstroke}{rgb}{0.279566,0.067836,0.391917}%
\pgfsetstrokecolor{currentstroke}%
\pgfsetdash{}{0pt}%
\pgfpathmoveto{\pgfqpoint{4.294948in}{2.293393in}}%
\pgfpathlineto{\pgfqpoint{4.278929in}{2.274572in}}%
\pgfusepath{stroke}%
\end{pgfscope}%
\begin{pgfscope}%
\pgfpathrectangle{\pgfqpoint{3.352233in}{1.400000in}}{\pgfqpoint{2.407767in}{1.544118in}}%
\pgfusepath{clip}%
\pgfsetbuttcap%
\pgfsetroundjoin%
\pgfsetlinewidth{0.501875pt}%
\definecolor{currentstroke}{rgb}{0.279566,0.067836,0.391917}%
\pgfsetstrokecolor{currentstroke}%
\pgfsetdash{}{0pt}%
\pgfpathmoveto{\pgfqpoint{4.278929in}{2.274572in}}%
\pgfpathlineto{\pgfqpoint{4.278929in}{2.274572in}}%
\pgfusepath{stroke}%
\end{pgfscope}%
\begin{pgfscope}%
\pgfpathrectangle{\pgfqpoint{3.352233in}{1.400000in}}{\pgfqpoint{2.407767in}{1.544118in}}%
\pgfusepath{clip}%
\pgfsetbuttcap%
\pgfsetroundjoin%
\pgfsetlinewidth{0.501875pt}%
\definecolor{currentstroke}{rgb}{0.279566,0.067836,0.391917}%
\pgfsetstrokecolor{currentstroke}%
\pgfsetdash{}{0pt}%
\pgfpathmoveto{\pgfqpoint{4.278929in}{2.274572in}}%
\pgfpathlineto{\pgfqpoint{4.264063in}{2.248906in}}%
\pgfusepath{stroke}%
\end{pgfscope}%
\begin{pgfscope}%
\pgfpathrectangle{\pgfqpoint{3.352233in}{1.400000in}}{\pgfqpoint{2.407767in}{1.544118in}}%
\pgfusepath{clip}%
\pgfsetbuttcap%
\pgfsetroundjoin%
\pgfsetlinewidth{0.501875pt}%
\definecolor{currentstroke}{rgb}{0.279566,0.067836,0.391917}%
\pgfsetstrokecolor{currentstroke}%
\pgfsetdash{}{0pt}%
\pgfpathmoveto{\pgfqpoint{4.264063in}{2.248906in}}%
\pgfpathlineto{\pgfqpoint{4.246591in}{2.227699in}}%
\pgfusepath{stroke}%
\end{pgfscope}%
\begin{pgfscope}%
\pgfpathrectangle{\pgfqpoint{3.352233in}{1.400000in}}{\pgfqpoint{2.407767in}{1.544118in}}%
\pgfusepath{clip}%
\pgfsetbuttcap%
\pgfsetroundjoin%
\pgfsetlinewidth{0.501875pt}%
\definecolor{currentstroke}{rgb}{0.274952,0.037752,0.364543}%
\pgfsetstrokecolor{currentstroke}%
\pgfsetdash{}{0pt}%
\pgfpathmoveto{\pgfqpoint{4.246591in}{2.227699in}}%
\pgfpathlineto{\pgfqpoint{4.233619in}{2.205058in}}%
\pgfusepath{stroke}%
\end{pgfscope}%
\begin{pgfscope}%
\pgfpathrectangle{\pgfqpoint{3.352233in}{1.400000in}}{\pgfqpoint{2.407767in}{1.544118in}}%
\pgfusepath{clip}%
\pgfsetbuttcap%
\pgfsetroundjoin%
\pgfsetlinewidth{0.501875pt}%
\definecolor{currentstroke}{rgb}{0.273809,0.031497,0.358853}%
\pgfsetstrokecolor{currentstroke}%
\pgfsetdash{}{0pt}%
\pgfpathmoveto{\pgfqpoint{4.233619in}{2.205058in}}%
\pgfpathlineto{\pgfqpoint{4.233619in}{2.205058in}}%
\pgfusepath{stroke}%
\end{pgfscope}%
\begin{pgfscope}%
\pgfpathrectangle{\pgfqpoint{3.352233in}{1.400000in}}{\pgfqpoint{2.407767in}{1.544118in}}%
\pgfusepath{clip}%
\pgfsetbuttcap%
\pgfsetroundjoin%
\pgfsetlinewidth{0.501875pt}%
\definecolor{currentstroke}{rgb}{0.273809,0.031497,0.358853}%
\pgfsetstrokecolor{currentstroke}%
\pgfsetdash{}{0pt}%
\pgfpathmoveto{\pgfqpoint{4.233619in}{2.205058in}}%
\pgfpathlineto{\pgfqpoint{4.232520in}{2.195646in}}%
\pgfusepath{stroke}%
\end{pgfscope}%
\begin{pgfscope}%
\pgfpathrectangle{\pgfqpoint{3.352233in}{1.400000in}}{\pgfqpoint{2.407767in}{1.544118in}}%
\pgfusepath{clip}%
\pgfsetbuttcap%
\pgfsetroundjoin%
\pgfsetlinewidth{0.501875pt}%
\definecolor{currentstroke}{rgb}{0.272594,0.025563,0.353093}%
\pgfsetstrokecolor{currentstroke}%
\pgfsetdash{}{0pt}%
\pgfpathmoveto{\pgfqpoint{4.232520in}{2.195646in}}%
\pgfpathlineto{\pgfqpoint{4.234174in}{2.189459in}}%
\pgfusepath{stroke}%
\end{pgfscope}%
\begin{pgfscope}%
\pgfpathrectangle{\pgfqpoint{3.352233in}{1.400000in}}{\pgfqpoint{2.407767in}{1.544118in}}%
\pgfusepath{clip}%
\pgfsetbuttcap%
\pgfsetroundjoin%
\pgfsetlinewidth{0.501875pt}%
\definecolor{currentstroke}{rgb}{0.282327,0.094955,0.417331}%
\pgfsetstrokecolor{currentstroke}%
\pgfsetdash{}{0pt}%
\pgfpathmoveto{\pgfqpoint{4.122675in}{1.755105in}}%
\pgfpathlineto{\pgfqpoint{4.171883in}{1.767449in}}%
\pgfusepath{stroke}%
\end{pgfscope}%
\begin{pgfscope}%
\pgfpathrectangle{\pgfqpoint{3.352233in}{1.400000in}}{\pgfqpoint{2.407767in}{1.544118in}}%
\pgfusepath{clip}%
\pgfsetbuttcap%
\pgfsetroundjoin%
\pgfsetlinewidth{0.501875pt}%
\definecolor{currentstroke}{rgb}{0.280894,0.078907,0.402329}%
\pgfsetstrokecolor{currentstroke}%
\pgfsetdash{}{0pt}%
\pgfpathmoveto{\pgfqpoint{4.171883in}{1.767449in}}%
\pgfpathlineto{\pgfqpoint{4.220013in}{1.781191in}}%
\pgfusepath{stroke}%
\end{pgfscope}%
\begin{pgfscope}%
\pgfpathrectangle{\pgfqpoint{3.352233in}{1.400000in}}{\pgfqpoint{2.407767in}{1.544118in}}%
\pgfusepath{clip}%
\pgfsetbuttcap%
\pgfsetroundjoin%
\pgfsetlinewidth{0.501875pt}%
\definecolor{currentstroke}{rgb}{0.281924,0.089666,0.412415}%
\pgfsetstrokecolor{currentstroke}%
\pgfsetdash{}{0pt}%
\pgfpathmoveto{\pgfqpoint{4.220013in}{1.781191in}}%
\pgfpathlineto{\pgfqpoint{4.220013in}{1.781191in}}%
\pgfusepath{stroke}%
\end{pgfscope}%
\begin{pgfscope}%
\pgfpathrectangle{\pgfqpoint{3.352233in}{1.400000in}}{\pgfqpoint{2.407767in}{1.544118in}}%
\pgfusepath{clip}%
\pgfsetbuttcap%
\pgfsetroundjoin%
\pgfsetlinewidth{0.501875pt}%
\definecolor{currentstroke}{rgb}{0.281924,0.089666,0.412415}%
\pgfsetstrokecolor{currentstroke}%
\pgfsetdash{}{0pt}%
\pgfpathmoveto{\pgfqpoint{4.220013in}{1.781191in}}%
\pgfpathlineto{\pgfqpoint{4.252936in}{1.795600in}}%
\pgfusepath{stroke}%
\end{pgfscope}%
\begin{pgfscope}%
\pgfpathrectangle{\pgfqpoint{3.352233in}{1.400000in}}{\pgfqpoint{2.407767in}{1.544118in}}%
\pgfusepath{clip}%
\pgfsetbuttcap%
\pgfsetroundjoin%
\pgfsetlinewidth{0.501875pt}%
\definecolor{currentstroke}{rgb}{0.280267,0.073417,0.397163}%
\pgfsetstrokecolor{currentstroke}%
\pgfsetdash{}{0pt}%
\pgfpathmoveto{\pgfqpoint{4.252936in}{1.795600in}}%
\pgfpathlineto{\pgfqpoint{4.278874in}{1.814441in}}%
\pgfusepath{stroke}%
\end{pgfscope}%
\begin{pgfscope}%
\pgfpathrectangle{\pgfqpoint{3.352233in}{1.400000in}}{\pgfqpoint{2.407767in}{1.544118in}}%
\pgfusepath{clip}%
\pgfsetbuttcap%
\pgfsetroundjoin%
\pgfsetlinewidth{0.501875pt}%
\definecolor{currentstroke}{rgb}{0.280894,0.078907,0.402329}%
\pgfsetstrokecolor{currentstroke}%
\pgfsetdash{}{0pt}%
\pgfpathmoveto{\pgfqpoint{4.278874in}{1.814441in}}%
\pgfpathlineto{\pgfqpoint{4.278874in}{1.814441in}}%
\pgfusepath{stroke}%
\end{pgfscope}%
\begin{pgfscope}%
\pgfpathrectangle{\pgfqpoint{3.352233in}{1.400000in}}{\pgfqpoint{2.407767in}{1.544118in}}%
\pgfusepath{clip}%
\pgfsetbuttcap%
\pgfsetroundjoin%
\pgfsetlinewidth{0.501875pt}%
\definecolor{currentstroke}{rgb}{0.280894,0.078907,0.402329}%
\pgfsetstrokecolor{currentstroke}%
\pgfsetdash{}{0pt}%
\pgfpathmoveto{\pgfqpoint{4.278874in}{1.814441in}}%
\pgfpathlineto{\pgfqpoint{4.293581in}{1.831638in}}%
\pgfusepath{stroke}%
\end{pgfscope}%
\begin{pgfscope}%
\pgfpathrectangle{\pgfqpoint{3.352233in}{1.400000in}}{\pgfqpoint{2.407767in}{1.544118in}}%
\pgfusepath{clip}%
\pgfsetbuttcap%
\pgfsetroundjoin%
\pgfsetlinewidth{0.501875pt}%
\definecolor{currentstroke}{rgb}{0.280267,0.073417,0.397163}%
\pgfsetstrokecolor{currentstroke}%
\pgfsetdash{}{0pt}%
\pgfpathmoveto{\pgfqpoint{4.293581in}{1.831638in}}%
\pgfpathlineto{\pgfqpoint{4.293581in}{1.831638in}}%
\pgfusepath{stroke}%
\end{pgfscope}%
\begin{pgfscope}%
\pgfpathrectangle{\pgfqpoint{3.352233in}{1.400000in}}{\pgfqpoint{2.407767in}{1.544118in}}%
\pgfusepath{clip}%
\pgfsetbuttcap%
\pgfsetroundjoin%
\pgfsetlinewidth{0.501875pt}%
\definecolor{currentstroke}{rgb}{0.280267,0.073417,0.397163}%
\pgfsetstrokecolor{currentstroke}%
\pgfsetdash{}{0pt}%
\pgfpathmoveto{\pgfqpoint{4.293581in}{1.831638in}}%
\pgfpathlineto{\pgfqpoint{4.296692in}{1.844662in}}%
\pgfusepath{stroke}%
\end{pgfscope}%
\begin{pgfscope}%
\pgfpathrectangle{\pgfqpoint{3.352233in}{1.400000in}}{\pgfqpoint{2.407767in}{1.544118in}}%
\pgfusepath{clip}%
\pgfsetbuttcap%
\pgfsetroundjoin%
\pgfsetlinewidth{0.501875pt}%
\definecolor{currentstroke}{rgb}{0.281446,0.084320,0.407414}%
\pgfsetstrokecolor{currentstroke}%
\pgfsetdash{}{0pt}%
\pgfpathmoveto{\pgfqpoint{4.296692in}{1.844662in}}%
\pgfpathlineto{\pgfqpoint{4.291953in}{1.856650in}}%
\pgfusepath{stroke}%
\end{pgfscope}%
\begin{pgfscope}%
\pgfpathrectangle{\pgfqpoint{3.352233in}{1.400000in}}{\pgfqpoint{2.407767in}{1.544118in}}%
\pgfusepath{clip}%
\pgfsetbuttcap%
\pgfsetroundjoin%
\pgfsetlinewidth{0.501875pt}%
\definecolor{currentstroke}{rgb}{0.281924,0.089666,0.412415}%
\pgfsetstrokecolor{currentstroke}%
\pgfsetdash{}{0pt}%
\pgfpathmoveto{\pgfqpoint{4.291953in}{1.856650in}}%
\pgfpathlineto{\pgfqpoint{4.291953in}{1.856650in}}%
\pgfusepath{stroke}%
\end{pgfscope}%
\begin{pgfscope}%
\pgfpathrectangle{\pgfqpoint{3.352233in}{1.400000in}}{\pgfqpoint{2.407767in}{1.544118in}}%
\pgfusepath{clip}%
\pgfsetbuttcap%
\pgfsetroundjoin%
\pgfsetlinewidth{0.501875pt}%
\definecolor{currentstroke}{rgb}{0.281924,0.089666,0.412415}%
\pgfsetstrokecolor{currentstroke}%
\pgfsetdash{}{0pt}%
\pgfpathmoveto{\pgfqpoint{4.291953in}{1.856650in}}%
\pgfpathlineto{\pgfqpoint{4.292390in}{1.870933in}}%
\pgfusepath{stroke}%
\end{pgfscope}%
\begin{pgfscope}%
\pgfpathrectangle{\pgfqpoint{3.352233in}{1.400000in}}{\pgfqpoint{2.407767in}{1.544118in}}%
\pgfusepath{clip}%
\pgfsetbuttcap%
\pgfsetroundjoin%
\pgfsetlinewidth{0.501875pt}%
\definecolor{currentstroke}{rgb}{0.281446,0.084320,0.407414}%
\pgfsetstrokecolor{currentstroke}%
\pgfsetdash{}{0pt}%
\pgfpathmoveto{\pgfqpoint{4.292390in}{1.870933in}}%
\pgfpathlineto{\pgfqpoint{4.295573in}{1.884757in}}%
\pgfusepath{stroke}%
\end{pgfscope}%
\begin{pgfscope}%
\pgfpathrectangle{\pgfqpoint{3.352233in}{1.400000in}}{\pgfqpoint{2.407767in}{1.544118in}}%
\pgfusepath{clip}%
\pgfsetbuttcap%
\pgfsetroundjoin%
\pgfsetlinewidth{0.501875pt}%
\definecolor{currentstroke}{rgb}{0.279566,0.067836,0.391917}%
\pgfsetstrokecolor{currentstroke}%
\pgfsetdash{}{0pt}%
\pgfpathmoveto{\pgfqpoint{4.295573in}{1.884757in}}%
\pgfpathlineto{\pgfqpoint{4.295573in}{1.884757in}}%
\pgfusepath{stroke}%
\end{pgfscope}%
\begin{pgfscope}%
\pgfpathrectangle{\pgfqpoint{3.352233in}{1.400000in}}{\pgfqpoint{2.407767in}{1.544118in}}%
\pgfusepath{clip}%
\pgfsetbuttcap%
\pgfsetroundjoin%
\pgfsetlinewidth{0.501875pt}%
\definecolor{currentstroke}{rgb}{0.279566,0.067836,0.391917}%
\pgfsetstrokecolor{currentstroke}%
\pgfsetdash{}{0pt}%
\pgfpathmoveto{\pgfqpoint{4.295573in}{1.884757in}}%
\pgfpathlineto{\pgfqpoint{4.294903in}{1.902337in}}%
\pgfusepath{stroke}%
\end{pgfscope}%
\begin{pgfscope}%
\pgfpathrectangle{\pgfqpoint{3.352233in}{1.400000in}}{\pgfqpoint{2.407767in}{1.544118in}}%
\pgfusepath{clip}%
\pgfsetbuttcap%
\pgfsetroundjoin%
\pgfsetlinewidth{0.501875pt}%
\definecolor{currentstroke}{rgb}{0.280267,0.073417,0.397163}%
\pgfsetstrokecolor{currentstroke}%
\pgfsetdash{}{0pt}%
\pgfpathmoveto{\pgfqpoint{4.874718in}{1.739403in}}%
\pgfpathlineto{\pgfqpoint{4.821878in}{1.741781in}}%
\pgfusepath{stroke}%
\end{pgfscope}%
\begin{pgfscope}%
\pgfpathrectangle{\pgfqpoint{3.352233in}{1.400000in}}{\pgfqpoint{2.407767in}{1.544118in}}%
\pgfusepath{clip}%
\pgfsetbuttcap%
\pgfsetroundjoin%
\pgfsetlinewidth{0.501875pt}%
\definecolor{currentstroke}{rgb}{0.279566,0.067836,0.391917}%
\pgfsetstrokecolor{currentstroke}%
\pgfsetdash{}{0pt}%
\pgfpathmoveto{\pgfqpoint{4.821878in}{1.741781in}}%
\pgfpathlineto{\pgfqpoint{4.769128in}{1.744811in}}%
\pgfusepath{stroke}%
\end{pgfscope}%
\begin{pgfscope}%
\pgfpathrectangle{\pgfqpoint{3.352233in}{1.400000in}}{\pgfqpoint{2.407767in}{1.544118in}}%
\pgfusepath{clip}%
\pgfsetbuttcap%
\pgfsetroundjoin%
\pgfsetlinewidth{0.501875pt}%
\definecolor{currentstroke}{rgb}{0.278791,0.062145,0.386592}%
\pgfsetstrokecolor{currentstroke}%
\pgfsetdash{}{0pt}%
\pgfpathmoveto{\pgfqpoint{4.769128in}{1.744811in}}%
\pgfpathlineto{\pgfqpoint{4.716618in}{1.749248in}}%
\pgfusepath{stroke}%
\end{pgfscope}%
\begin{pgfscope}%
\pgfpathrectangle{\pgfqpoint{3.352233in}{1.400000in}}{\pgfqpoint{2.407767in}{1.544118in}}%
\pgfusepath{clip}%
\pgfsetbuttcap%
\pgfsetroundjoin%
\pgfsetlinewidth{0.501875pt}%
\definecolor{currentstroke}{rgb}{0.280267,0.073417,0.397163}%
\pgfsetstrokecolor{currentstroke}%
\pgfsetdash{}{0pt}%
\pgfpathmoveto{\pgfqpoint{4.716618in}{1.749248in}}%
\pgfpathlineto{\pgfqpoint{4.664477in}{1.755105in}}%
\pgfusepath{stroke}%
\end{pgfscope}%
\begin{pgfscope}%
\pgfpathrectangle{\pgfqpoint{3.352233in}{1.400000in}}{\pgfqpoint{2.407767in}{1.544118in}}%
\pgfusepath{clip}%
\pgfsetbuttcap%
\pgfsetroundjoin%
\pgfsetlinewidth{0.501875pt}%
\definecolor{currentstroke}{rgb}{0.278791,0.062145,0.386592}%
\pgfsetstrokecolor{currentstroke}%
\pgfsetdash{}{0pt}%
\pgfpathmoveto{\pgfqpoint{4.664477in}{1.755105in}}%
\pgfpathlineto{\pgfqpoint{4.612647in}{1.762035in}}%
\pgfusepath{stroke}%
\end{pgfscope}%
\begin{pgfscope}%
\pgfpathrectangle{\pgfqpoint{3.352233in}{1.400000in}}{\pgfqpoint{2.407767in}{1.544118in}}%
\pgfusepath{clip}%
\pgfsetbuttcap%
\pgfsetroundjoin%
\pgfsetlinewidth{0.501875pt}%
\definecolor{currentstroke}{rgb}{0.269944,0.014625,0.341379}%
\pgfsetstrokecolor{currentstroke}%
\pgfsetdash{}{0pt}%
\pgfpathmoveto{\pgfqpoint{5.200208in}{1.753080in}}%
\pgfpathlineto{\pgfqpoint{5.147238in}{1.753217in}}%
\pgfusepath{stroke}%
\end{pgfscope}%
\begin{pgfscope}%
\pgfpathrectangle{\pgfqpoint{3.352233in}{1.400000in}}{\pgfqpoint{2.407767in}{1.544118in}}%
\pgfusepath{clip}%
\pgfsetbuttcap%
\pgfsetroundjoin%
\pgfsetlinewidth{0.501875pt}%
\definecolor{currentstroke}{rgb}{0.272594,0.025563,0.353093}%
\pgfsetstrokecolor{currentstroke}%
\pgfsetdash{}{0pt}%
\pgfpathmoveto{\pgfqpoint{5.147238in}{1.753217in}}%
\pgfpathlineto{\pgfqpoint{5.094269in}{1.753528in}}%
\pgfusepath{stroke}%
\end{pgfscope}%
\begin{pgfscope}%
\pgfpathrectangle{\pgfqpoint{3.352233in}{1.400000in}}{\pgfqpoint{2.407767in}{1.544118in}}%
\pgfusepath{clip}%
\pgfsetbuttcap%
\pgfsetroundjoin%
\pgfsetlinewidth{0.501875pt}%
\definecolor{currentstroke}{rgb}{0.273809,0.031497,0.358853}%
\pgfsetstrokecolor{currentstroke}%
\pgfsetdash{}{0pt}%
\pgfpathmoveto{\pgfqpoint{5.094269in}{1.753528in}}%
\pgfpathlineto{\pgfqpoint{5.041302in}{1.753841in}}%
\pgfusepath{stroke}%
\end{pgfscope}%
\begin{pgfscope}%
\pgfpathrectangle{\pgfqpoint{3.352233in}{1.400000in}}{\pgfqpoint{2.407767in}{1.544118in}}%
\pgfusepath{clip}%
\pgfsetbuttcap%
\pgfsetroundjoin%
\pgfsetlinewidth{0.501875pt}%
\definecolor{currentstroke}{rgb}{0.274952,0.037752,0.364543}%
\pgfsetstrokecolor{currentstroke}%
\pgfsetdash{}{0pt}%
\pgfpathmoveto{\pgfqpoint{5.041302in}{1.753841in}}%
\pgfpathlineto{\pgfqpoint{4.988336in}{1.754308in}}%
\pgfusepath{stroke}%
\end{pgfscope}%
\begin{pgfscope}%
\pgfpathrectangle{\pgfqpoint{3.352233in}{1.400000in}}{\pgfqpoint{2.407767in}{1.544118in}}%
\pgfusepath{clip}%
\pgfsetbuttcap%
\pgfsetroundjoin%
\pgfsetlinewidth{0.501875pt}%
\definecolor{currentstroke}{rgb}{0.277018,0.050344,0.375715}%
\pgfsetstrokecolor{currentstroke}%
\pgfsetdash{}{0pt}%
\pgfpathmoveto{\pgfqpoint{4.988336in}{1.754308in}}%
\pgfpathlineto{\pgfqpoint{4.935378in}{1.755105in}}%
\pgfusepath{stroke}%
\end{pgfscope}%
\begin{pgfscope}%
\pgfpathrectangle{\pgfqpoint{3.352233in}{1.400000in}}{\pgfqpoint{2.407767in}{1.544118in}}%
\pgfusepath{clip}%
\pgfsetbuttcap%
\pgfsetroundjoin%
\pgfsetlinewidth{0.501875pt}%
\definecolor{currentstroke}{rgb}{0.269944,0.014625,0.341379}%
\pgfsetstrokecolor{currentstroke}%
\pgfsetdash{}{0pt}%
\pgfpathmoveto{\pgfqpoint{5.206279in}{1.789851in}}%
\pgfpathlineto{\pgfqpoint{5.153315in}{1.789555in}}%
\pgfusepath{stroke}%
\end{pgfscope}%
\begin{pgfscope}%
\pgfpathrectangle{\pgfqpoint{3.352233in}{1.400000in}}{\pgfqpoint{2.407767in}{1.544118in}}%
\pgfusepath{clip}%
\pgfsetbuttcap%
\pgfsetroundjoin%
\pgfsetlinewidth{0.501875pt}%
\definecolor{currentstroke}{rgb}{0.272594,0.025563,0.353093}%
\pgfsetstrokecolor{currentstroke}%
\pgfsetdash{}{0pt}%
\pgfpathmoveto{\pgfqpoint{5.153315in}{1.789555in}}%
\pgfpathlineto{\pgfqpoint{5.100343in}{1.789871in}}%
\pgfusepath{stroke}%
\end{pgfscope}%
\begin{pgfscope}%
\pgfpathrectangle{\pgfqpoint{3.352233in}{1.400000in}}{\pgfqpoint{2.407767in}{1.544118in}}%
\pgfusepath{clip}%
\pgfsetbuttcap%
\pgfsetroundjoin%
\pgfsetlinewidth{0.501875pt}%
\definecolor{currentstroke}{rgb}{0.273809,0.031497,0.358853}%
\pgfsetstrokecolor{currentstroke}%
\pgfsetdash{}{0pt}%
\pgfpathmoveto{\pgfqpoint{5.100343in}{1.789871in}}%
\pgfpathlineto{\pgfqpoint{5.047371in}{1.790164in}}%
\pgfusepath{stroke}%
\end{pgfscope}%
\begin{pgfscope}%
\pgfpathrectangle{\pgfqpoint{3.352233in}{1.400000in}}{\pgfqpoint{2.407767in}{1.544118in}}%
\pgfusepath{clip}%
\pgfsetbuttcap%
\pgfsetroundjoin%
\pgfsetlinewidth{0.501875pt}%
\definecolor{currentstroke}{rgb}{0.276022,0.044167,0.370164}%
\pgfsetstrokecolor{currentstroke}%
\pgfsetdash{}{0pt}%
\pgfpathmoveto{\pgfqpoint{5.047371in}{1.790164in}}%
\pgfpathlineto{\pgfqpoint{4.994416in}{1.790901in}}%
\pgfusepath{stroke}%
\end{pgfscope}%
\begin{pgfscope}%
\pgfpathrectangle{\pgfqpoint{3.352233in}{1.400000in}}{\pgfqpoint{2.407767in}{1.544118in}}%
\pgfusepath{clip}%
\pgfsetbuttcap%
\pgfsetroundjoin%
\pgfsetlinewidth{0.501875pt}%
\definecolor{currentstroke}{rgb}{0.277941,0.056324,0.381191}%
\pgfsetstrokecolor{currentstroke}%
\pgfsetdash{}{0pt}%
\pgfpathmoveto{\pgfqpoint{4.994416in}{1.790901in}}%
\pgfpathlineto{\pgfqpoint{4.941470in}{1.792025in}}%
\pgfusepath{stroke}%
\end{pgfscope}%
\begin{pgfscope}%
\pgfpathrectangle{\pgfqpoint{3.352233in}{1.400000in}}{\pgfqpoint{2.407767in}{1.544118in}}%
\pgfusepath{clip}%
\pgfsetbuttcap%
\pgfsetroundjoin%
\pgfsetlinewidth{0.501875pt}%
\definecolor{currentstroke}{rgb}{0.280267,0.073417,0.397163}%
\pgfsetstrokecolor{currentstroke}%
\pgfsetdash{}{0pt}%
\pgfpathmoveto{\pgfqpoint{4.941470in}{1.792025in}}%
\pgfpathlineto{\pgfqpoint{4.888525in}{1.793056in}}%
\pgfusepath{stroke}%
\end{pgfscope}%
\begin{pgfscope}%
\pgfpathrectangle{\pgfqpoint{3.352233in}{1.400000in}}{\pgfqpoint{2.407767in}{1.544118in}}%
\pgfusepath{clip}%
\pgfsetbuttcap%
\pgfsetroundjoin%
\pgfsetlinewidth{0.501875pt}%
\definecolor{currentstroke}{rgb}{0.279566,0.067836,0.391917}%
\pgfsetstrokecolor{currentstroke}%
\pgfsetdash{}{0pt}%
\pgfpathmoveto{\pgfqpoint{4.888525in}{1.793056in}}%
\pgfpathlineto{\pgfqpoint{4.835616in}{1.794550in}}%
\pgfusepath{stroke}%
\end{pgfscope}%
\begin{pgfscope}%
\pgfpathrectangle{\pgfqpoint{3.352233in}{1.400000in}}{\pgfqpoint{2.407767in}{1.544118in}}%
\pgfusepath{clip}%
\pgfsetbuttcap%
\pgfsetroundjoin%
\pgfsetlinewidth{0.501875pt}%
\definecolor{currentstroke}{rgb}{0.281446,0.084320,0.407414}%
\pgfsetstrokecolor{currentstroke}%
\pgfsetdash{}{0pt}%
\pgfpathmoveto{\pgfqpoint{4.835616in}{1.794550in}}%
\pgfpathlineto{\pgfqpoint{4.782751in}{1.796687in}}%
\pgfusepath{stroke}%
\end{pgfscope}%
\begin{pgfscope}%
\pgfpathrectangle{\pgfqpoint{3.352233in}{1.400000in}}{\pgfqpoint{2.407767in}{1.544118in}}%
\pgfusepath{clip}%
\pgfsetbuttcap%
\pgfsetroundjoin%
\pgfsetlinewidth{0.501875pt}%
\definecolor{currentstroke}{rgb}{0.281446,0.084320,0.407414}%
\pgfsetstrokecolor{currentstroke}%
\pgfsetdash{}{0pt}%
\pgfpathmoveto{\pgfqpoint{4.782751in}{1.796687in}}%
\pgfpathlineto{\pgfqpoint{4.729936in}{1.799297in}}%
\pgfusepath{stroke}%
\end{pgfscope}%
\begin{pgfscope}%
\pgfpathrectangle{\pgfqpoint{3.352233in}{1.400000in}}{\pgfqpoint{2.407767in}{1.544118in}}%
\pgfusepath{clip}%
\pgfsetbuttcap%
\pgfsetroundjoin%
\pgfsetlinewidth{0.501875pt}%
\definecolor{currentstroke}{rgb}{0.282327,0.094955,0.417331}%
\pgfsetstrokecolor{currentstroke}%
\pgfsetdash{}{0pt}%
\pgfpathmoveto{\pgfqpoint{4.729936in}{1.799297in}}%
\pgfpathlineto{\pgfqpoint{4.677342in}{1.803192in}}%
\pgfusepath{stroke}%
\end{pgfscope}%
\begin{pgfscope}%
\pgfpathrectangle{\pgfqpoint{3.352233in}{1.400000in}}{\pgfqpoint{2.407767in}{1.544118in}}%
\pgfusepath{clip}%
\pgfsetbuttcap%
\pgfsetroundjoin%
\pgfsetlinewidth{0.501875pt}%
\definecolor{currentstroke}{rgb}{0.282656,0.100196,0.422160}%
\pgfsetstrokecolor{currentstroke}%
\pgfsetdash{}{0pt}%
\pgfpathmoveto{\pgfqpoint{4.677342in}{1.803192in}}%
\pgfpathlineto{\pgfqpoint{4.625152in}{1.808924in}}%
\pgfusepath{stroke}%
\end{pgfscope}%
\begin{pgfscope}%
\pgfpathrectangle{\pgfqpoint{3.352233in}{1.400000in}}{\pgfqpoint{2.407767in}{1.544118in}}%
\pgfusepath{clip}%
\pgfsetbuttcap%
\pgfsetroundjoin%
\pgfsetlinewidth{0.501875pt}%
\definecolor{currentstroke}{rgb}{0.280894,0.078907,0.402329}%
\pgfsetstrokecolor{currentstroke}%
\pgfsetdash{}{0pt}%
\pgfpathmoveto{\pgfqpoint{4.625152in}{1.808924in}}%
\pgfpathlineto{\pgfqpoint{4.573580in}{1.816513in}}%
\pgfusepath{stroke}%
\end{pgfscope}%
\begin{pgfscope}%
\pgfpathrectangle{\pgfqpoint{3.352233in}{1.400000in}}{\pgfqpoint{2.407767in}{1.544118in}}%
\pgfusepath{clip}%
\pgfsetbuttcap%
\pgfsetroundjoin%
\pgfsetlinewidth{0.501875pt}%
\definecolor{currentstroke}{rgb}{0.280267,0.073417,0.397163}%
\pgfsetstrokecolor{currentstroke}%
\pgfsetdash{}{0pt}%
\pgfpathmoveto{\pgfqpoint{4.573580in}{1.816513in}}%
\pgfpathlineto{\pgfqpoint{4.522540in}{1.825553in}}%
\pgfusepath{stroke}%
\end{pgfscope}%
\begin{pgfscope}%
\pgfpathrectangle{\pgfqpoint{3.352233in}{1.400000in}}{\pgfqpoint{2.407767in}{1.544118in}}%
\pgfusepath{clip}%
\pgfsetbuttcap%
\pgfsetroundjoin%
\pgfsetlinewidth{0.501875pt}%
\definecolor{currentstroke}{rgb}{0.280267,0.073417,0.397163}%
\pgfsetstrokecolor{currentstroke}%
\pgfsetdash{}{0pt}%
\pgfpathmoveto{\pgfqpoint{4.522540in}{1.825553in}}%
\pgfpathlineto{\pgfqpoint{4.472684in}{1.836684in}}%
\pgfusepath{stroke}%
\end{pgfscope}%
\begin{pgfscope}%
\pgfpathrectangle{\pgfqpoint{3.352233in}{1.400000in}}{\pgfqpoint{2.407767in}{1.544118in}}%
\pgfusepath{clip}%
\pgfsetbuttcap%
\pgfsetroundjoin%
\pgfsetlinewidth{0.501875pt}%
\definecolor{currentstroke}{rgb}{0.282327,0.094955,0.417331}%
\pgfsetstrokecolor{currentstroke}%
\pgfsetdash{}{0pt}%
\pgfpathmoveto{\pgfqpoint{4.472684in}{1.836684in}}%
\pgfpathlineto{\pgfqpoint{4.424873in}{1.851023in}}%
\pgfusepath{stroke}%
\end{pgfscope}%
\begin{pgfscope}%
\pgfpathrectangle{\pgfqpoint{3.352233in}{1.400000in}}{\pgfqpoint{2.407767in}{1.544118in}}%
\pgfusepath{clip}%
\pgfsetbuttcap%
\pgfsetroundjoin%
\pgfsetlinewidth{0.501875pt}%
\definecolor{currentstroke}{rgb}{0.281446,0.084320,0.407414}%
\pgfsetstrokecolor{currentstroke}%
\pgfsetdash{}{0pt}%
\pgfpathmoveto{\pgfqpoint{4.424873in}{1.851023in}}%
\pgfpathlineto{\pgfqpoint{4.380126in}{1.868950in}}%
\pgfusepath{stroke}%
\end{pgfscope}%
\begin{pgfscope}%
\pgfpathrectangle{\pgfqpoint{3.352233in}{1.400000in}}{\pgfqpoint{2.407767in}{1.544118in}}%
\pgfusepath{clip}%
\pgfsetbuttcap%
\pgfsetroundjoin%
\pgfsetlinewidth{0.501875pt}%
\definecolor{currentstroke}{rgb}{0.281446,0.084320,0.407414}%
\pgfsetstrokecolor{currentstroke}%
\pgfsetdash{}{0pt}%
\pgfpathmoveto{\pgfqpoint{4.380126in}{1.868950in}}%
\pgfpathlineto{\pgfqpoint{4.380126in}{1.868950in}}%
\pgfusepath{stroke}%
\end{pgfscope}%
\begin{pgfscope}%
\pgfpathrectangle{\pgfqpoint{3.352233in}{1.400000in}}{\pgfqpoint{2.407767in}{1.544118in}}%
\pgfusepath{clip}%
\pgfsetbuttcap%
\pgfsetroundjoin%
\pgfsetlinewidth{0.501875pt}%
\definecolor{currentstroke}{rgb}{0.269944,0.014625,0.341379}%
\pgfsetstrokecolor{currentstroke}%
\pgfsetdash{}{0pt}%
\pgfpathmoveto{\pgfqpoint{5.206279in}{1.824598in}}%
\pgfpathlineto{\pgfqpoint{5.153341in}{1.825004in}}%
\pgfusepath{stroke}%
\end{pgfscope}%
\begin{pgfscope}%
\pgfpathrectangle{\pgfqpoint{3.352233in}{1.400000in}}{\pgfqpoint{2.407767in}{1.544118in}}%
\pgfusepath{clip}%
\pgfsetbuttcap%
\pgfsetroundjoin%
\pgfsetlinewidth{0.501875pt}%
\definecolor{currentstroke}{rgb}{0.271305,0.019942,0.347269}%
\pgfsetstrokecolor{currentstroke}%
\pgfsetdash{}{0pt}%
\pgfpathmoveto{\pgfqpoint{5.153341in}{1.825004in}}%
\pgfpathlineto{\pgfqpoint{5.100378in}{1.824601in}}%
\pgfusepath{stroke}%
\end{pgfscope}%
\begin{pgfscope}%
\pgfpathrectangle{\pgfqpoint{3.352233in}{1.400000in}}{\pgfqpoint{2.407767in}{1.544118in}}%
\pgfusepath{clip}%
\pgfsetbuttcap%
\pgfsetroundjoin%
\pgfsetlinewidth{0.501875pt}%
\definecolor{currentstroke}{rgb}{0.274952,0.037752,0.364543}%
\pgfsetstrokecolor{currentstroke}%
\pgfsetdash{}{0pt}%
\pgfpathmoveto{\pgfqpoint{5.100378in}{1.824601in}}%
\pgfpathlineto{\pgfqpoint{5.047405in}{1.824592in}}%
\pgfusepath{stroke}%
\end{pgfscope}%
\begin{pgfscope}%
\pgfpathrectangle{\pgfqpoint{3.352233in}{1.400000in}}{\pgfqpoint{2.407767in}{1.544118in}}%
\pgfusepath{clip}%
\pgfsetbuttcap%
\pgfsetroundjoin%
\pgfsetlinewidth{0.501875pt}%
\definecolor{currentstroke}{rgb}{0.277018,0.050344,0.375715}%
\pgfsetstrokecolor{currentstroke}%
\pgfsetdash{}{0pt}%
\pgfpathmoveto{\pgfqpoint{5.047405in}{1.824592in}}%
\pgfpathlineto{\pgfqpoint{4.994436in}{1.825005in}}%
\pgfusepath{stroke}%
\end{pgfscope}%
\begin{pgfscope}%
\pgfpathrectangle{\pgfqpoint{3.352233in}{1.400000in}}{\pgfqpoint{2.407767in}{1.544118in}}%
\pgfusepath{clip}%
\pgfsetbuttcap%
\pgfsetroundjoin%
\pgfsetlinewidth{0.501875pt}%
\definecolor{currentstroke}{rgb}{0.278791,0.062145,0.386592}%
\pgfsetstrokecolor{currentstroke}%
\pgfsetdash{}{0pt}%
\pgfpathmoveto{\pgfqpoint{4.994436in}{1.825005in}}%
\pgfpathlineto{\pgfqpoint{4.941472in}{1.825666in}}%
\pgfusepath{stroke}%
\end{pgfscope}%
\begin{pgfscope}%
\pgfpathrectangle{\pgfqpoint{3.352233in}{1.400000in}}{\pgfqpoint{2.407767in}{1.544118in}}%
\pgfusepath{clip}%
\pgfsetbuttcap%
\pgfsetroundjoin%
\pgfsetlinewidth{0.501875pt}%
\definecolor{currentstroke}{rgb}{0.281446,0.084320,0.407414}%
\pgfsetstrokecolor{currentstroke}%
\pgfsetdash{}{0pt}%
\pgfpathmoveto{\pgfqpoint{4.941472in}{1.825666in}}%
\pgfpathlineto{\pgfqpoint{4.888514in}{1.826423in}}%
\pgfusepath{stroke}%
\end{pgfscope}%
\begin{pgfscope}%
\pgfpathrectangle{\pgfqpoint{3.352233in}{1.400000in}}{\pgfqpoint{2.407767in}{1.544118in}}%
\pgfusepath{clip}%
\pgfsetbuttcap%
\pgfsetroundjoin%
\pgfsetlinewidth{0.501875pt}%
\definecolor{currentstroke}{rgb}{0.281924,0.089666,0.412415}%
\pgfsetstrokecolor{currentstroke}%
\pgfsetdash{}{0pt}%
\pgfpathmoveto{\pgfqpoint{4.888514in}{1.826423in}}%
\pgfpathlineto{\pgfqpoint{4.835560in}{1.827284in}}%
\pgfusepath{stroke}%
\end{pgfscope}%
\begin{pgfscope}%
\pgfpathrectangle{\pgfqpoint{3.352233in}{1.400000in}}{\pgfqpoint{2.407767in}{1.544118in}}%
\pgfusepath{clip}%
\pgfsetbuttcap%
\pgfsetroundjoin%
\pgfsetlinewidth{0.501875pt}%
\definecolor{currentstroke}{rgb}{0.282327,0.094955,0.417331}%
\pgfsetstrokecolor{currentstroke}%
\pgfsetdash{}{0pt}%
\pgfpathmoveto{\pgfqpoint{4.835560in}{1.827284in}}%
\pgfpathlineto{\pgfqpoint{4.782628in}{1.828597in}}%
\pgfusepath{stroke}%
\end{pgfscope}%
\begin{pgfscope}%
\pgfpathrectangle{\pgfqpoint{3.352233in}{1.400000in}}{\pgfqpoint{2.407767in}{1.544118in}}%
\pgfusepath{clip}%
\pgfsetbuttcap%
\pgfsetroundjoin%
\pgfsetlinewidth{0.501875pt}%
\definecolor{currentstroke}{rgb}{0.282656,0.100196,0.422160}%
\pgfsetstrokecolor{currentstroke}%
\pgfsetdash{}{0pt}%
\pgfpathmoveto{\pgfqpoint{4.782628in}{1.828597in}}%
\pgfpathlineto{\pgfqpoint{4.729742in}{1.830513in}}%
\pgfusepath{stroke}%
\end{pgfscope}%
\begin{pgfscope}%
\pgfpathrectangle{\pgfqpoint{3.352233in}{1.400000in}}{\pgfqpoint{2.407767in}{1.544118in}}%
\pgfusepath{clip}%
\pgfsetbuttcap%
\pgfsetroundjoin%
\pgfsetlinewidth{0.501875pt}%
\definecolor{currentstroke}{rgb}{0.281924,0.089666,0.412415}%
\pgfsetstrokecolor{currentstroke}%
\pgfsetdash{}{0pt}%
\pgfpathmoveto{\pgfqpoint{4.729742in}{1.830513in}}%
\pgfpathlineto{\pgfqpoint{4.677008in}{1.833591in}}%
\pgfusepath{stroke}%
\end{pgfscope}%
\begin{pgfscope}%
\pgfpathrectangle{\pgfqpoint{3.352233in}{1.400000in}}{\pgfqpoint{2.407767in}{1.544118in}}%
\pgfusepath{clip}%
\pgfsetbuttcap%
\pgfsetroundjoin%
\pgfsetlinewidth{0.501875pt}%
\definecolor{currentstroke}{rgb}{0.268510,0.009605,0.335427}%
\pgfsetstrokecolor{currentstroke}%
\pgfsetdash{}{0pt}%
\pgfpathmoveto{\pgfqpoint{5.206279in}{1.859344in}}%
\pgfpathlineto{\pgfqpoint{5.153324in}{1.859527in}}%
\pgfusepath{stroke}%
\end{pgfscope}%
\begin{pgfscope}%
\pgfpathrectangle{\pgfqpoint{3.352233in}{1.400000in}}{\pgfqpoint{2.407767in}{1.544118in}}%
\pgfusepath{clip}%
\pgfsetbuttcap%
\pgfsetroundjoin%
\pgfsetlinewidth{0.501875pt}%
\definecolor{currentstroke}{rgb}{0.272594,0.025563,0.353093}%
\pgfsetstrokecolor{currentstroke}%
\pgfsetdash{}{0pt}%
\pgfpathmoveto{\pgfqpoint{5.153324in}{1.859527in}}%
\pgfpathlineto{\pgfqpoint{5.100350in}{1.859678in}}%
\pgfusepath{stroke}%
\end{pgfscope}%
\begin{pgfscope}%
\pgfpathrectangle{\pgfqpoint{3.352233in}{1.400000in}}{\pgfqpoint{2.407767in}{1.544118in}}%
\pgfusepath{clip}%
\pgfsetbuttcap%
\pgfsetroundjoin%
\pgfsetlinewidth{0.501875pt}%
\definecolor{currentstroke}{rgb}{0.274952,0.037752,0.364543}%
\pgfsetstrokecolor{currentstroke}%
\pgfsetdash{}{0pt}%
\pgfpathmoveto{\pgfqpoint{5.100350in}{1.859678in}}%
\pgfpathlineto{\pgfqpoint{5.047378in}{1.859994in}}%
\pgfusepath{stroke}%
\end{pgfscope}%
\begin{pgfscope}%
\pgfpathrectangle{\pgfqpoint{3.352233in}{1.400000in}}{\pgfqpoint{2.407767in}{1.544118in}}%
\pgfusepath{clip}%
\pgfsetbuttcap%
\pgfsetroundjoin%
\pgfsetlinewidth{0.501875pt}%
\definecolor{currentstroke}{rgb}{0.277018,0.050344,0.375715}%
\pgfsetstrokecolor{currentstroke}%
\pgfsetdash{}{0pt}%
\pgfpathmoveto{\pgfqpoint{5.047378in}{1.859994in}}%
\pgfpathlineto{\pgfqpoint{4.994412in}{1.860623in}}%
\pgfusepath{stroke}%
\end{pgfscope}%
\begin{pgfscope}%
\pgfpathrectangle{\pgfqpoint{3.352233in}{1.400000in}}{\pgfqpoint{2.407767in}{1.544118in}}%
\pgfusepath{clip}%
\pgfsetbuttcap%
\pgfsetroundjoin%
\pgfsetlinewidth{0.501875pt}%
\definecolor{currentstroke}{rgb}{0.280267,0.073417,0.397163}%
\pgfsetstrokecolor{currentstroke}%
\pgfsetdash{}{0pt}%
\pgfpathmoveto{\pgfqpoint{4.994412in}{1.860623in}}%
\pgfpathlineto{\pgfqpoint{4.941461in}{1.861537in}}%
\pgfusepath{stroke}%
\end{pgfscope}%
\begin{pgfscope}%
\pgfpathrectangle{\pgfqpoint{3.352233in}{1.400000in}}{\pgfqpoint{2.407767in}{1.544118in}}%
\pgfusepath{clip}%
\pgfsetbuttcap%
\pgfsetroundjoin%
\pgfsetlinewidth{0.501875pt}%
\definecolor{currentstroke}{rgb}{0.280894,0.078907,0.402329}%
\pgfsetstrokecolor{currentstroke}%
\pgfsetdash{}{0pt}%
\pgfpathmoveto{\pgfqpoint{4.941461in}{1.861537in}}%
\pgfpathlineto{\pgfqpoint{4.888516in}{1.862604in}}%
\pgfusepath{stroke}%
\end{pgfscope}%
\begin{pgfscope}%
\pgfpathrectangle{\pgfqpoint{3.352233in}{1.400000in}}{\pgfqpoint{2.407767in}{1.544118in}}%
\pgfusepath{clip}%
\pgfsetbuttcap%
\pgfsetroundjoin%
\pgfsetlinewidth{0.501875pt}%
\definecolor{currentstroke}{rgb}{0.282327,0.094955,0.417331}%
\pgfsetstrokecolor{currentstroke}%
\pgfsetdash{}{0pt}%
\pgfpathmoveto{\pgfqpoint{4.888516in}{1.862604in}}%
\pgfpathlineto{\pgfqpoint{4.835567in}{1.863701in}}%
\pgfusepath{stroke}%
\end{pgfscope}%
\begin{pgfscope}%
\pgfpathrectangle{\pgfqpoint{3.352233in}{1.400000in}}{\pgfqpoint{2.407767in}{1.544118in}}%
\pgfusepath{clip}%
\pgfsetbuttcap%
\pgfsetroundjoin%
\pgfsetlinewidth{0.501875pt}%
\definecolor{currentstroke}{rgb}{0.283229,0.120777,0.440584}%
\pgfsetstrokecolor{currentstroke}%
\pgfsetdash{}{0pt}%
\pgfpathmoveto{\pgfqpoint{4.835567in}{1.863701in}}%
\pgfpathlineto{\pgfqpoint{4.782652in}{1.865267in}}%
\pgfusepath{stroke}%
\end{pgfscope}%
\begin{pgfscope}%
\pgfpathrectangle{\pgfqpoint{3.352233in}{1.400000in}}{\pgfqpoint{2.407767in}{1.544118in}}%
\pgfusepath{clip}%
\pgfsetbuttcap%
\pgfsetroundjoin%
\pgfsetlinewidth{0.501875pt}%
\definecolor{currentstroke}{rgb}{0.283091,0.110553,0.431554}%
\pgfsetstrokecolor{currentstroke}%
\pgfsetdash{}{0pt}%
\pgfpathmoveto{\pgfqpoint{4.782652in}{1.865267in}}%
\pgfpathlineto{\pgfqpoint{4.729779in}{1.867367in}}%
\pgfusepath{stroke}%
\end{pgfscope}%
\begin{pgfscope}%
\pgfpathrectangle{\pgfqpoint{3.352233in}{1.400000in}}{\pgfqpoint{2.407767in}{1.544118in}}%
\pgfusepath{clip}%
\pgfsetbuttcap%
\pgfsetroundjoin%
\pgfsetlinewidth{0.501875pt}%
\definecolor{currentstroke}{rgb}{0.283187,0.125848,0.444960}%
\pgfsetstrokecolor{currentstroke}%
\pgfsetdash{}{0pt}%
\pgfpathmoveto{\pgfqpoint{4.729779in}{1.867367in}}%
\pgfpathlineto{\pgfqpoint{4.677001in}{1.870234in}}%
\pgfusepath{stroke}%
\end{pgfscope}%
\begin{pgfscope}%
\pgfpathrectangle{\pgfqpoint{3.352233in}{1.400000in}}{\pgfqpoint{2.407767in}{1.544118in}}%
\pgfusepath{clip}%
\pgfsetbuttcap%
\pgfsetroundjoin%
\pgfsetlinewidth{0.501875pt}%
\definecolor{currentstroke}{rgb}{0.283187,0.125848,0.444960}%
\pgfsetstrokecolor{currentstroke}%
\pgfsetdash{}{0pt}%
\pgfpathmoveto{\pgfqpoint{4.677001in}{1.870234in}}%
\pgfpathlineto{\pgfqpoint{4.624361in}{1.874029in}}%
\pgfusepath{stroke}%
\end{pgfscope}%
\begin{pgfscope}%
\pgfpathrectangle{\pgfqpoint{3.352233in}{1.400000in}}{\pgfqpoint{2.407767in}{1.544118in}}%
\pgfusepath{clip}%
\pgfsetbuttcap%
\pgfsetroundjoin%
\pgfsetlinewidth{0.501875pt}%
\definecolor{currentstroke}{rgb}{0.283197,0.115680,0.436115}%
\pgfsetstrokecolor{currentstroke}%
\pgfsetdash{}{0pt}%
\pgfpathmoveto{\pgfqpoint{4.624361in}{1.874029in}}%
\pgfpathlineto{\pgfqpoint{4.572026in}{1.879183in}}%
\pgfusepath{stroke}%
\end{pgfscope}%
\begin{pgfscope}%
\pgfpathrectangle{\pgfqpoint{3.352233in}{1.400000in}}{\pgfqpoint{2.407767in}{1.544118in}}%
\pgfusepath{clip}%
\pgfsetbuttcap%
\pgfsetroundjoin%
\pgfsetlinewidth{0.501875pt}%
\definecolor{currentstroke}{rgb}{0.283197,0.115680,0.436115}%
\pgfsetstrokecolor{currentstroke}%
\pgfsetdash{}{0pt}%
\pgfpathmoveto{\pgfqpoint{4.572026in}{1.879183in}}%
\pgfpathlineto{\pgfqpoint{4.520104in}{1.885875in}}%
\pgfusepath{stroke}%
\end{pgfscope}%
\begin{pgfscope}%
\pgfpathrectangle{\pgfqpoint{3.352233in}{1.400000in}}{\pgfqpoint{2.407767in}{1.544118in}}%
\pgfusepath{clip}%
\pgfsetbuttcap%
\pgfsetroundjoin%
\pgfsetlinewidth{0.501875pt}%
\definecolor{currentstroke}{rgb}{0.281924,0.089666,0.412415}%
\pgfsetstrokecolor{currentstroke}%
\pgfsetdash{}{0pt}%
\pgfpathmoveto{\pgfqpoint{4.520104in}{1.885875in}}%
\pgfpathlineto{\pgfqpoint{4.469092in}{1.894787in}}%
\pgfusepath{stroke}%
\end{pgfscope}%
\begin{pgfscope}%
\pgfpathrectangle{\pgfqpoint{3.352233in}{1.400000in}}{\pgfqpoint{2.407767in}{1.544118in}}%
\pgfusepath{clip}%
\pgfsetbuttcap%
\pgfsetroundjoin%
\pgfsetlinewidth{0.501875pt}%
\definecolor{currentstroke}{rgb}{0.278791,0.062145,0.386592}%
\pgfsetstrokecolor{currentstroke}%
\pgfsetdash{}{0pt}%
\pgfpathmoveto{\pgfqpoint{4.469092in}{1.894787in}}%
\pgfpathlineto{\pgfqpoint{4.420158in}{1.907091in}}%
\pgfusepath{stroke}%
\end{pgfscope}%
\begin{pgfscope}%
\pgfpathrectangle{\pgfqpoint{3.352233in}{1.400000in}}{\pgfqpoint{2.407767in}{1.544118in}}%
\pgfusepath{clip}%
\pgfsetbuttcap%
\pgfsetroundjoin%
\pgfsetlinewidth{0.501875pt}%
\definecolor{currentstroke}{rgb}{0.280267,0.073417,0.397163}%
\pgfsetstrokecolor{currentstroke}%
\pgfsetdash{}{0pt}%
\pgfpathmoveto{\pgfqpoint{4.420158in}{1.907091in}}%
\pgfpathlineto{\pgfqpoint{4.372549in}{1.921159in}}%
\pgfusepath{stroke}%
\end{pgfscope}%
\begin{pgfscope}%
\pgfpathrectangle{\pgfqpoint{3.352233in}{1.400000in}}{\pgfqpoint{2.407767in}{1.544118in}}%
\pgfusepath{clip}%
\pgfsetbuttcap%
\pgfsetroundjoin%
\pgfsetlinewidth{0.501875pt}%
\definecolor{currentstroke}{rgb}{0.281924,0.089666,0.412415}%
\pgfsetstrokecolor{currentstroke}%
\pgfsetdash{}{0pt}%
\pgfpathmoveto{\pgfqpoint{4.372549in}{1.921159in}}%
\pgfpathlineto{\pgfqpoint{4.372549in}{1.921159in}}%
\pgfusepath{stroke}%
\end{pgfscope}%
\begin{pgfscope}%
\pgfpathrectangle{\pgfqpoint{3.352233in}{1.400000in}}{\pgfqpoint{2.407767in}{1.544118in}}%
\pgfusepath{clip}%
\pgfsetbuttcap%
\pgfsetroundjoin%
\pgfsetlinewidth{0.501875pt}%
\definecolor{currentstroke}{rgb}{0.281924,0.089666,0.412415}%
\pgfsetstrokecolor{currentstroke}%
\pgfsetdash{}{0pt}%
\pgfpathmoveto{\pgfqpoint{4.372549in}{1.921159in}}%
\pgfpathlineto{\pgfqpoint{4.342275in}{1.934650in}}%
\pgfusepath{stroke}%
\end{pgfscope}%
\begin{pgfscope}%
\pgfpathrectangle{\pgfqpoint{3.352233in}{1.400000in}}{\pgfqpoint{2.407767in}{1.544118in}}%
\pgfusepath{clip}%
\pgfsetbuttcap%
\pgfsetroundjoin%
\pgfsetlinewidth{0.501875pt}%
\definecolor{currentstroke}{rgb}{0.278791,0.062145,0.386592}%
\pgfsetstrokecolor{currentstroke}%
\pgfsetdash{}{0pt}%
\pgfpathmoveto{\pgfqpoint{4.342275in}{1.934650in}}%
\pgfpathlineto{\pgfqpoint{4.318885in}{1.949664in}}%
\pgfusepath{stroke}%
\end{pgfscope}%
\begin{pgfscope}%
\pgfpathrectangle{\pgfqpoint{3.352233in}{1.400000in}}{\pgfqpoint{2.407767in}{1.544118in}}%
\pgfusepath{clip}%
\pgfsetbuttcap%
\pgfsetroundjoin%
\pgfsetlinewidth{0.501875pt}%
\definecolor{currentstroke}{rgb}{0.280894,0.078907,0.402329}%
\pgfsetstrokecolor{currentstroke}%
\pgfsetdash{}{0pt}%
\pgfpathmoveto{\pgfqpoint{4.318885in}{1.949664in}}%
\pgfpathlineto{\pgfqpoint{4.318885in}{1.949664in}}%
\pgfusepath{stroke}%
\end{pgfscope}%
\begin{pgfscope}%
\pgfpathrectangle{\pgfqpoint{3.352233in}{1.400000in}}{\pgfqpoint{2.407767in}{1.544118in}}%
\pgfusepath{clip}%
\pgfsetbuttcap%
\pgfsetroundjoin%
\pgfsetlinewidth{0.501875pt}%
\definecolor{currentstroke}{rgb}{0.268510,0.009605,0.335427}%
\pgfsetstrokecolor{currentstroke}%
\pgfsetdash{}{0pt}%
\pgfpathmoveto{\pgfqpoint{5.206279in}{1.894090in}}%
\pgfpathlineto{\pgfqpoint{5.153308in}{1.894355in}}%
\pgfusepath{stroke}%
\end{pgfscope}%
\begin{pgfscope}%
\pgfpathrectangle{\pgfqpoint{3.352233in}{1.400000in}}{\pgfqpoint{2.407767in}{1.544118in}}%
\pgfusepath{clip}%
\pgfsetbuttcap%
\pgfsetroundjoin%
\pgfsetlinewidth{0.501875pt}%
\definecolor{currentstroke}{rgb}{0.271305,0.019942,0.347269}%
\pgfsetstrokecolor{currentstroke}%
\pgfsetdash{}{0pt}%
\pgfpathmoveto{\pgfqpoint{5.153308in}{1.894355in}}%
\pgfpathlineto{\pgfqpoint{5.100337in}{1.894678in}}%
\pgfusepath{stroke}%
\end{pgfscope}%
\begin{pgfscope}%
\pgfpathrectangle{\pgfqpoint{3.352233in}{1.400000in}}{\pgfqpoint{2.407767in}{1.544118in}}%
\pgfusepath{clip}%
\pgfsetbuttcap%
\pgfsetroundjoin%
\pgfsetlinewidth{0.501875pt}%
\definecolor{currentstroke}{rgb}{0.274952,0.037752,0.364543}%
\pgfsetstrokecolor{currentstroke}%
\pgfsetdash{}{0pt}%
\pgfpathmoveto{\pgfqpoint{5.100337in}{1.894678in}}%
\pgfpathlineto{\pgfqpoint{5.047367in}{1.894982in}}%
\pgfusepath{stroke}%
\end{pgfscope}%
\begin{pgfscope}%
\pgfpathrectangle{\pgfqpoint{3.352233in}{1.400000in}}{\pgfqpoint{2.407767in}{1.544118in}}%
\pgfusepath{clip}%
\pgfsetbuttcap%
\pgfsetroundjoin%
\pgfsetlinewidth{0.501875pt}%
\definecolor{currentstroke}{rgb}{0.277941,0.056324,0.381191}%
\pgfsetstrokecolor{currentstroke}%
\pgfsetdash{}{0pt}%
\pgfpathmoveto{\pgfqpoint{5.047367in}{1.894982in}}%
\pgfpathlineto{\pgfqpoint{4.994394in}{1.895234in}}%
\pgfusepath{stroke}%
\end{pgfscope}%
\begin{pgfscope}%
\pgfpathrectangle{\pgfqpoint{3.352233in}{1.400000in}}{\pgfqpoint{2.407767in}{1.544118in}}%
\pgfusepath{clip}%
\pgfsetbuttcap%
\pgfsetroundjoin%
\pgfsetlinewidth{0.501875pt}%
\definecolor{currentstroke}{rgb}{0.280267,0.073417,0.397163}%
\pgfsetstrokecolor{currentstroke}%
\pgfsetdash{}{0pt}%
\pgfpathmoveto{\pgfqpoint{4.994394in}{1.895234in}}%
\pgfpathlineto{\pgfqpoint{4.941422in}{1.895618in}}%
\pgfusepath{stroke}%
\end{pgfscope}%
\begin{pgfscope}%
\pgfpathrectangle{\pgfqpoint{3.352233in}{1.400000in}}{\pgfqpoint{2.407767in}{1.544118in}}%
\pgfusepath{clip}%
\pgfsetbuttcap%
\pgfsetroundjoin%
\pgfsetlinewidth{0.501875pt}%
\definecolor{currentstroke}{rgb}{0.282327,0.094955,0.417331}%
\pgfsetstrokecolor{currentstroke}%
\pgfsetdash{}{0pt}%
\pgfpathmoveto{\pgfqpoint{4.941422in}{1.895618in}}%
\pgfpathlineto{\pgfqpoint{4.888459in}{1.896282in}}%
\pgfusepath{stroke}%
\end{pgfscope}%
\begin{pgfscope}%
\pgfpathrectangle{\pgfqpoint{3.352233in}{1.400000in}}{\pgfqpoint{2.407767in}{1.544118in}}%
\pgfusepath{clip}%
\pgfsetbuttcap%
\pgfsetroundjoin%
\pgfsetlinewidth{0.501875pt}%
\definecolor{currentstroke}{rgb}{0.282656,0.100196,0.422160}%
\pgfsetstrokecolor{currentstroke}%
\pgfsetdash{}{0pt}%
\pgfpathmoveto{\pgfqpoint{4.888459in}{1.896282in}}%
\pgfpathlineto{\pgfqpoint{4.835521in}{1.897537in}}%
\pgfusepath{stroke}%
\end{pgfscope}%
\begin{pgfscope}%
\pgfpathrectangle{\pgfqpoint{3.352233in}{1.400000in}}{\pgfqpoint{2.407767in}{1.544118in}}%
\pgfusepath{clip}%
\pgfsetbuttcap%
\pgfsetroundjoin%
\pgfsetlinewidth{0.501875pt}%
\definecolor{currentstroke}{rgb}{0.283229,0.120777,0.440584}%
\pgfsetstrokecolor{currentstroke}%
\pgfsetdash{}{0pt}%
\pgfpathmoveto{\pgfqpoint{4.835521in}{1.897537in}}%
\pgfpathlineto{\pgfqpoint{4.782608in}{1.899173in}}%
\pgfusepath{stroke}%
\end{pgfscope}%
\begin{pgfscope}%
\pgfpathrectangle{\pgfqpoint{3.352233in}{1.400000in}}{\pgfqpoint{2.407767in}{1.544118in}}%
\pgfusepath{clip}%
\pgfsetbuttcap%
\pgfsetroundjoin%
\pgfsetlinewidth{0.501875pt}%
\definecolor{currentstroke}{rgb}{0.283229,0.120777,0.440584}%
\pgfsetstrokecolor{currentstroke}%
\pgfsetdash{}{0pt}%
\pgfpathmoveto{\pgfqpoint{4.782608in}{1.899173in}}%
\pgfpathlineto{\pgfqpoint{4.729703in}{1.900915in}}%
\pgfusepath{stroke}%
\end{pgfscope}%
\begin{pgfscope}%
\pgfpathrectangle{\pgfqpoint{3.352233in}{1.400000in}}{\pgfqpoint{2.407767in}{1.544118in}}%
\pgfusepath{clip}%
\pgfsetbuttcap%
\pgfsetroundjoin%
\pgfsetlinewidth{0.501875pt}%
\definecolor{currentstroke}{rgb}{0.283187,0.125848,0.444960}%
\pgfsetstrokecolor{currentstroke}%
\pgfsetdash{}{0pt}%
\pgfpathmoveto{\pgfqpoint{4.729703in}{1.900915in}}%
\pgfpathlineto{\pgfqpoint{4.676862in}{1.903251in}}%
\pgfusepath{stroke}%
\end{pgfscope}%
\begin{pgfscope}%
\pgfpathrectangle{\pgfqpoint{3.352233in}{1.400000in}}{\pgfqpoint{2.407767in}{1.544118in}}%
\pgfusepath{clip}%
\pgfsetbuttcap%
\pgfsetroundjoin%
\pgfsetlinewidth{0.501875pt}%
\definecolor{currentstroke}{rgb}{0.283072,0.130895,0.449241}%
\pgfsetstrokecolor{currentstroke}%
\pgfsetdash{}{0pt}%
\pgfpathmoveto{\pgfqpoint{4.676862in}{1.903251in}}%
\pgfpathlineto{\pgfqpoint{4.624133in}{1.906486in}}%
\pgfusepath{stroke}%
\end{pgfscope}%
\begin{pgfscope}%
\pgfpathrectangle{\pgfqpoint{3.352233in}{1.400000in}}{\pgfqpoint{2.407767in}{1.544118in}}%
\pgfusepath{clip}%
\pgfsetbuttcap%
\pgfsetroundjoin%
\pgfsetlinewidth{0.501875pt}%
\definecolor{currentstroke}{rgb}{0.269944,0.014625,0.341379}%
\pgfsetstrokecolor{currentstroke}%
\pgfsetdash{}{0pt}%
\pgfpathmoveto{\pgfqpoint{5.206279in}{1.928836in}}%
\pgfpathlineto{\pgfqpoint{5.153313in}{1.929255in}}%
\pgfusepath{stroke}%
\end{pgfscope}%
\begin{pgfscope}%
\pgfpathrectangle{\pgfqpoint{3.352233in}{1.400000in}}{\pgfqpoint{2.407767in}{1.544118in}}%
\pgfusepath{clip}%
\pgfsetbuttcap%
\pgfsetroundjoin%
\pgfsetlinewidth{0.501875pt}%
\definecolor{currentstroke}{rgb}{0.271305,0.019942,0.347269}%
\pgfsetstrokecolor{currentstroke}%
\pgfsetdash{}{0pt}%
\pgfpathmoveto{\pgfqpoint{5.153313in}{1.929255in}}%
\pgfpathlineto{\pgfqpoint{5.100353in}{1.929994in}}%
\pgfusepath{stroke}%
\end{pgfscope}%
\begin{pgfscope}%
\pgfpathrectangle{\pgfqpoint{3.352233in}{1.400000in}}{\pgfqpoint{2.407767in}{1.544118in}}%
\pgfusepath{clip}%
\pgfsetbuttcap%
\pgfsetroundjoin%
\pgfsetlinewidth{0.501875pt}%
\definecolor{currentstroke}{rgb}{0.274952,0.037752,0.364543}%
\pgfsetstrokecolor{currentstroke}%
\pgfsetdash{}{0pt}%
\pgfpathmoveto{\pgfqpoint{5.100353in}{1.929994in}}%
\pgfpathlineto{\pgfqpoint{5.047386in}{1.930486in}}%
\pgfusepath{stroke}%
\end{pgfscope}%
\begin{pgfscope}%
\pgfpathrectangle{\pgfqpoint{3.352233in}{1.400000in}}{\pgfqpoint{2.407767in}{1.544118in}}%
\pgfusepath{clip}%
\pgfsetbuttcap%
\pgfsetroundjoin%
\pgfsetlinewidth{0.501875pt}%
\definecolor{currentstroke}{rgb}{0.278791,0.062145,0.386592}%
\pgfsetstrokecolor{currentstroke}%
\pgfsetdash{}{0pt}%
\pgfpathmoveto{\pgfqpoint{5.047386in}{1.930486in}}%
\pgfpathlineto{\pgfqpoint{4.994423in}{1.931049in}}%
\pgfusepath{stroke}%
\end{pgfscope}%
\begin{pgfscope}%
\pgfpathrectangle{\pgfqpoint{3.352233in}{1.400000in}}{\pgfqpoint{2.407767in}{1.544118in}}%
\pgfusepath{clip}%
\pgfsetbuttcap%
\pgfsetroundjoin%
\pgfsetlinewidth{0.501875pt}%
\definecolor{currentstroke}{rgb}{0.280267,0.073417,0.397163}%
\pgfsetstrokecolor{currentstroke}%
\pgfsetdash{}{0pt}%
\pgfpathmoveto{\pgfqpoint{4.994423in}{1.931049in}}%
\pgfpathlineto{\pgfqpoint{4.941466in}{1.931826in}}%
\pgfusepath{stroke}%
\end{pgfscope}%
\begin{pgfscope}%
\pgfpathrectangle{\pgfqpoint{3.352233in}{1.400000in}}{\pgfqpoint{2.407767in}{1.544118in}}%
\pgfusepath{clip}%
\pgfsetbuttcap%
\pgfsetroundjoin%
\pgfsetlinewidth{0.501875pt}%
\definecolor{currentstroke}{rgb}{0.281924,0.089666,0.412415}%
\pgfsetstrokecolor{currentstroke}%
\pgfsetdash{}{0pt}%
\pgfpathmoveto{\pgfqpoint{4.941466in}{1.931826in}}%
\pgfpathlineto{\pgfqpoint{4.888510in}{1.932735in}}%
\pgfusepath{stroke}%
\end{pgfscope}%
\begin{pgfscope}%
\pgfpathrectangle{\pgfqpoint{3.352233in}{1.400000in}}{\pgfqpoint{2.407767in}{1.544118in}}%
\pgfusepath{clip}%
\pgfsetbuttcap%
\pgfsetroundjoin%
\pgfsetlinewidth{0.501875pt}%
\definecolor{currentstroke}{rgb}{0.282910,0.105393,0.426902}%
\pgfsetstrokecolor{currentstroke}%
\pgfsetdash{}{0pt}%
\pgfpathmoveto{\pgfqpoint{4.888510in}{1.932735in}}%
\pgfpathlineto{\pgfqpoint{4.835567in}{1.933904in}}%
\pgfusepath{stroke}%
\end{pgfscope}%
\begin{pgfscope}%
\pgfpathrectangle{\pgfqpoint{3.352233in}{1.400000in}}{\pgfqpoint{2.407767in}{1.544118in}}%
\pgfusepath{clip}%
\pgfsetbuttcap%
\pgfsetroundjoin%
\pgfsetlinewidth{0.501875pt}%
\definecolor{currentstroke}{rgb}{0.283187,0.125848,0.444960}%
\pgfsetstrokecolor{currentstroke}%
\pgfsetdash{}{0pt}%
\pgfpathmoveto{\pgfqpoint{4.835567in}{1.933904in}}%
\pgfpathlineto{\pgfqpoint{4.782635in}{1.935262in}}%
\pgfusepath{stroke}%
\end{pgfscope}%
\begin{pgfscope}%
\pgfpathrectangle{\pgfqpoint{3.352233in}{1.400000in}}{\pgfqpoint{2.407767in}{1.544118in}}%
\pgfusepath{clip}%
\pgfsetbuttcap%
\pgfsetroundjoin%
\pgfsetlinewidth{0.501875pt}%
\definecolor{currentstroke}{rgb}{0.283072,0.130895,0.449241}%
\pgfsetstrokecolor{currentstroke}%
\pgfsetdash{}{0pt}%
\pgfpathmoveto{\pgfqpoint{4.782635in}{1.935262in}}%
\pgfpathlineto{\pgfqpoint{4.729728in}{1.936939in}}%
\pgfusepath{stroke}%
\end{pgfscope}%
\begin{pgfscope}%
\pgfpathrectangle{\pgfqpoint{3.352233in}{1.400000in}}{\pgfqpoint{2.407767in}{1.544118in}}%
\pgfusepath{clip}%
\pgfsetbuttcap%
\pgfsetroundjoin%
\pgfsetlinewidth{0.501875pt}%
\definecolor{currentstroke}{rgb}{0.282290,0.145912,0.461510}%
\pgfsetstrokecolor{currentstroke}%
\pgfsetdash{}{0pt}%
\pgfpathmoveto{\pgfqpoint{4.729728in}{1.936939in}}%
\pgfpathlineto{\pgfqpoint{4.676844in}{1.938911in}}%
\pgfusepath{stroke}%
\end{pgfscope}%
\begin{pgfscope}%
\pgfpathrectangle{\pgfqpoint{3.352233in}{1.400000in}}{\pgfqpoint{2.407767in}{1.544118in}}%
\pgfusepath{clip}%
\pgfsetbuttcap%
\pgfsetroundjoin%
\pgfsetlinewidth{0.501875pt}%
\definecolor{currentstroke}{rgb}{0.282290,0.145912,0.461510}%
\pgfsetstrokecolor{currentstroke}%
\pgfsetdash{}{0pt}%
\pgfpathmoveto{\pgfqpoint{4.676844in}{1.938911in}}%
\pgfpathlineto{\pgfqpoint{4.624075in}{1.941772in}}%
\pgfusepath{stroke}%
\end{pgfscope}%
\begin{pgfscope}%
\pgfpathrectangle{\pgfqpoint{3.352233in}{1.400000in}}{\pgfqpoint{2.407767in}{1.544118in}}%
\pgfusepath{clip}%
\pgfsetbuttcap%
\pgfsetroundjoin%
\pgfsetlinewidth{0.501875pt}%
\definecolor{currentstroke}{rgb}{0.282623,0.140926,0.457517}%
\pgfsetstrokecolor{currentstroke}%
\pgfsetdash{}{0pt}%
\pgfpathmoveto{\pgfqpoint{4.624075in}{1.941772in}}%
\pgfpathlineto{\pgfqpoint{4.571435in}{1.945573in}}%
\pgfusepath{stroke}%
\end{pgfscope}%
\begin{pgfscope}%
\pgfpathrectangle{\pgfqpoint{3.352233in}{1.400000in}}{\pgfqpoint{2.407767in}{1.544118in}}%
\pgfusepath{clip}%
\pgfsetbuttcap%
\pgfsetroundjoin%
\pgfsetlinewidth{0.501875pt}%
\definecolor{currentstroke}{rgb}{0.282884,0.135920,0.453427}%
\pgfsetstrokecolor{currentstroke}%
\pgfsetdash{}{0pt}%
\pgfpathmoveto{\pgfqpoint{4.571435in}{1.945573in}}%
\pgfpathlineto{\pgfqpoint{4.518982in}{1.950229in}}%
\pgfusepath{stroke}%
\end{pgfscope}%
\begin{pgfscope}%
\pgfpathrectangle{\pgfqpoint{3.352233in}{1.400000in}}{\pgfqpoint{2.407767in}{1.544118in}}%
\pgfusepath{clip}%
\pgfsetbuttcap%
\pgfsetroundjoin%
\pgfsetlinewidth{0.501875pt}%
\definecolor{currentstroke}{rgb}{0.283187,0.125848,0.444960}%
\pgfsetstrokecolor{currentstroke}%
\pgfsetdash{}{0pt}%
\pgfpathmoveto{\pgfqpoint{4.518982in}{1.950229in}}%
\pgfpathlineto{\pgfqpoint{4.467015in}{1.956706in}}%
\pgfusepath{stroke}%
\end{pgfscope}%
\begin{pgfscope}%
\pgfpathrectangle{\pgfqpoint{3.352233in}{1.400000in}}{\pgfqpoint{2.407767in}{1.544118in}}%
\pgfusepath{clip}%
\pgfsetbuttcap%
\pgfsetroundjoin%
\pgfsetlinewidth{0.501875pt}%
\definecolor{currentstroke}{rgb}{0.283229,0.120777,0.440584}%
\pgfsetstrokecolor{currentstroke}%
\pgfsetdash{}{0pt}%
\pgfpathmoveto{\pgfqpoint{4.467015in}{1.956706in}}%
\pgfpathlineto{\pgfqpoint{4.415873in}{1.965438in}}%
\pgfusepath{stroke}%
\end{pgfscope}%
\begin{pgfscope}%
\pgfpathrectangle{\pgfqpoint{3.352233in}{1.400000in}}{\pgfqpoint{2.407767in}{1.544118in}}%
\pgfusepath{clip}%
\pgfsetbuttcap%
\pgfsetroundjoin%
\pgfsetlinewidth{0.501875pt}%
\definecolor{currentstroke}{rgb}{0.269944,0.014625,0.341379}%
\pgfsetstrokecolor{currentstroke}%
\pgfsetdash{}{0pt}%
\pgfpathmoveto{\pgfqpoint{5.206279in}{1.963582in}}%
\pgfpathlineto{\pgfqpoint{5.153315in}{1.963870in}}%
\pgfusepath{stroke}%
\end{pgfscope}%
\begin{pgfscope}%
\pgfpathrectangle{\pgfqpoint{3.352233in}{1.400000in}}{\pgfqpoint{2.407767in}{1.544118in}}%
\pgfusepath{clip}%
\pgfsetbuttcap%
\pgfsetroundjoin%
\pgfsetlinewidth{0.501875pt}%
\definecolor{currentstroke}{rgb}{0.272594,0.025563,0.353093}%
\pgfsetstrokecolor{currentstroke}%
\pgfsetdash{}{0pt}%
\pgfpathmoveto{\pgfqpoint{5.153315in}{1.963870in}}%
\pgfpathlineto{\pgfqpoint{5.100349in}{1.964337in}}%
\pgfusepath{stroke}%
\end{pgfscope}%
\begin{pgfscope}%
\pgfpathrectangle{\pgfqpoint{3.352233in}{1.400000in}}{\pgfqpoint{2.407767in}{1.544118in}}%
\pgfusepath{clip}%
\pgfsetbuttcap%
\pgfsetroundjoin%
\pgfsetlinewidth{0.501875pt}%
\definecolor{currentstroke}{rgb}{0.276022,0.044167,0.370164}%
\pgfsetstrokecolor{currentstroke}%
\pgfsetdash{}{0pt}%
\pgfpathmoveto{\pgfqpoint{5.100349in}{1.964337in}}%
\pgfpathlineto{\pgfqpoint{5.047378in}{1.964715in}}%
\pgfusepath{stroke}%
\end{pgfscope}%
\begin{pgfscope}%
\pgfpathrectangle{\pgfqpoint{3.352233in}{1.400000in}}{\pgfqpoint{2.407767in}{1.544118in}}%
\pgfusepath{clip}%
\pgfsetbuttcap%
\pgfsetroundjoin%
\pgfsetlinewidth{0.501875pt}%
\definecolor{currentstroke}{rgb}{0.277018,0.050344,0.375715}%
\pgfsetstrokecolor{currentstroke}%
\pgfsetdash{}{0pt}%
\pgfpathmoveto{\pgfqpoint{5.047378in}{1.964715in}}%
\pgfpathlineto{\pgfqpoint{4.994409in}{1.965227in}}%
\pgfusepath{stroke}%
\end{pgfscope}%
\begin{pgfscope}%
\pgfpathrectangle{\pgfqpoint{3.352233in}{1.400000in}}{\pgfqpoint{2.407767in}{1.544118in}}%
\pgfusepath{clip}%
\pgfsetbuttcap%
\pgfsetroundjoin%
\pgfsetlinewidth{0.501875pt}%
\definecolor{currentstroke}{rgb}{0.281446,0.084320,0.407414}%
\pgfsetstrokecolor{currentstroke}%
\pgfsetdash{}{0pt}%
\pgfpathmoveto{\pgfqpoint{4.994409in}{1.965227in}}%
\pgfpathlineto{\pgfqpoint{4.941440in}{1.965757in}}%
\pgfusepath{stroke}%
\end{pgfscope}%
\begin{pgfscope}%
\pgfpathrectangle{\pgfqpoint{3.352233in}{1.400000in}}{\pgfqpoint{2.407767in}{1.544118in}}%
\pgfusepath{clip}%
\pgfsetbuttcap%
\pgfsetroundjoin%
\pgfsetlinewidth{0.501875pt}%
\definecolor{currentstroke}{rgb}{0.283091,0.110553,0.431554}%
\pgfsetstrokecolor{currentstroke}%
\pgfsetdash{}{0pt}%
\pgfpathmoveto{\pgfqpoint{4.941440in}{1.965757in}}%
\pgfpathlineto{\pgfqpoint{4.888474in}{1.966340in}}%
\pgfusepath{stroke}%
\end{pgfscope}%
\begin{pgfscope}%
\pgfpathrectangle{\pgfqpoint{3.352233in}{1.400000in}}{\pgfqpoint{2.407767in}{1.544118in}}%
\pgfusepath{clip}%
\pgfsetbuttcap%
\pgfsetroundjoin%
\pgfsetlinewidth{0.501875pt}%
\definecolor{currentstroke}{rgb}{0.283187,0.125848,0.444960}%
\pgfsetstrokecolor{currentstroke}%
\pgfsetdash{}{0pt}%
\pgfpathmoveto{\pgfqpoint{4.888474in}{1.966340in}}%
\pgfpathlineto{\pgfqpoint{4.835518in}{1.967209in}}%
\pgfusepath{stroke}%
\end{pgfscope}%
\begin{pgfscope}%
\pgfpathrectangle{\pgfqpoint{3.352233in}{1.400000in}}{\pgfqpoint{2.407767in}{1.544118in}}%
\pgfusepath{clip}%
\pgfsetbuttcap%
\pgfsetroundjoin%
\pgfsetlinewidth{0.501875pt}%
\definecolor{currentstroke}{rgb}{0.283072,0.130895,0.449241}%
\pgfsetstrokecolor{currentstroke}%
\pgfsetdash{}{0pt}%
\pgfpathmoveto{\pgfqpoint{4.835518in}{1.967209in}}%
\pgfpathlineto{\pgfqpoint{4.782577in}{1.968421in}}%
\pgfusepath{stroke}%
\end{pgfscope}%
\begin{pgfscope}%
\pgfpathrectangle{\pgfqpoint{3.352233in}{1.400000in}}{\pgfqpoint{2.407767in}{1.544118in}}%
\pgfusepath{clip}%
\pgfsetbuttcap%
\pgfsetroundjoin%
\pgfsetlinewidth{0.501875pt}%
\definecolor{currentstroke}{rgb}{0.282623,0.140926,0.457517}%
\pgfsetstrokecolor{currentstroke}%
\pgfsetdash{}{0pt}%
\pgfpathmoveto{\pgfqpoint{4.782577in}{1.968421in}}%
\pgfpathlineto{\pgfqpoint{4.729656in}{1.969926in}}%
\pgfusepath{stroke}%
\end{pgfscope}%
\begin{pgfscope}%
\pgfpathrectangle{\pgfqpoint{3.352233in}{1.400000in}}{\pgfqpoint{2.407767in}{1.544118in}}%
\pgfusepath{clip}%
\pgfsetbuttcap%
\pgfsetroundjoin%
\pgfsetlinewidth{0.501875pt}%
\definecolor{currentstroke}{rgb}{0.280255,0.165693,0.476498}%
\pgfsetstrokecolor{currentstroke}%
\pgfsetdash{}{0pt}%
\pgfpathmoveto{\pgfqpoint{4.729656in}{1.969926in}}%
\pgfpathlineto{\pgfqpoint{4.676758in}{1.971738in}}%
\pgfusepath{stroke}%
\end{pgfscope}%
\begin{pgfscope}%
\pgfpathrectangle{\pgfqpoint{3.352233in}{1.400000in}}{\pgfqpoint{2.407767in}{1.544118in}}%
\pgfusepath{clip}%
\pgfsetbuttcap%
\pgfsetroundjoin%
\pgfsetlinewidth{0.501875pt}%
\definecolor{currentstroke}{rgb}{0.277134,0.185228,0.489898}%
\pgfsetstrokecolor{currentstroke}%
\pgfsetdash{}{0pt}%
\pgfpathmoveto{\pgfqpoint{4.676758in}{1.971738in}}%
\pgfpathlineto{\pgfqpoint{4.623914in}{1.974090in}}%
\pgfusepath{stroke}%
\end{pgfscope}%
\begin{pgfscope}%
\pgfpathrectangle{\pgfqpoint{3.352233in}{1.400000in}}{\pgfqpoint{2.407767in}{1.544118in}}%
\pgfusepath{clip}%
\pgfsetbuttcap%
\pgfsetroundjoin%
\pgfsetlinewidth{0.501875pt}%
\definecolor{currentstroke}{rgb}{0.281887,0.150881,0.465405}%
\pgfsetstrokecolor{currentstroke}%
\pgfsetdash{}{0pt}%
\pgfpathmoveto{\pgfqpoint{4.623914in}{1.974090in}}%
\pgfpathlineto{\pgfqpoint{4.571164in}{1.977191in}}%
\pgfusepath{stroke}%
\end{pgfscope}%
\begin{pgfscope}%
\pgfpathrectangle{\pgfqpoint{3.352233in}{1.400000in}}{\pgfqpoint{2.407767in}{1.544118in}}%
\pgfusepath{clip}%
\pgfsetbuttcap%
\pgfsetroundjoin%
\pgfsetlinewidth{0.501875pt}%
\definecolor{currentstroke}{rgb}{0.269944,0.014625,0.341379}%
\pgfsetstrokecolor{currentstroke}%
\pgfsetdash{}{0pt}%
\pgfpathmoveto{\pgfqpoint{5.206279in}{1.998328in}}%
\pgfpathlineto{\pgfqpoint{5.153305in}{1.998151in}}%
\pgfusepath{stroke}%
\end{pgfscope}%
\begin{pgfscope}%
\pgfpathrectangle{\pgfqpoint{3.352233in}{1.400000in}}{\pgfqpoint{2.407767in}{1.544118in}}%
\pgfusepath{clip}%
\pgfsetbuttcap%
\pgfsetroundjoin%
\pgfsetlinewidth{0.501875pt}%
\definecolor{currentstroke}{rgb}{0.272594,0.025563,0.353093}%
\pgfsetstrokecolor{currentstroke}%
\pgfsetdash{}{0pt}%
\pgfpathmoveto{\pgfqpoint{5.153305in}{1.998151in}}%
\pgfpathlineto{\pgfqpoint{5.100331in}{1.998004in}}%
\pgfusepath{stroke}%
\end{pgfscope}%
\begin{pgfscope}%
\pgfpathrectangle{\pgfqpoint{3.352233in}{1.400000in}}{\pgfqpoint{2.407767in}{1.544118in}}%
\pgfusepath{clip}%
\pgfsetbuttcap%
\pgfsetroundjoin%
\pgfsetlinewidth{0.501875pt}%
\definecolor{currentstroke}{rgb}{0.276022,0.044167,0.370164}%
\pgfsetstrokecolor{currentstroke}%
\pgfsetdash{}{0pt}%
\pgfpathmoveto{\pgfqpoint{5.100331in}{1.998004in}}%
\pgfpathlineto{\pgfqpoint{5.047363in}{1.998251in}}%
\pgfusepath{stroke}%
\end{pgfscope}%
\begin{pgfscope}%
\pgfpathrectangle{\pgfqpoint{3.352233in}{1.400000in}}{\pgfqpoint{2.407767in}{1.544118in}}%
\pgfusepath{clip}%
\pgfsetbuttcap%
\pgfsetroundjoin%
\pgfsetlinewidth{0.501875pt}%
\definecolor{currentstroke}{rgb}{0.279566,0.067836,0.391917}%
\pgfsetstrokecolor{currentstroke}%
\pgfsetdash{}{0pt}%
\pgfpathmoveto{\pgfqpoint{5.047363in}{1.998251in}}%
\pgfpathlineto{\pgfqpoint{4.994396in}{1.998759in}}%
\pgfusepath{stroke}%
\end{pgfscope}%
\begin{pgfscope}%
\pgfpathrectangle{\pgfqpoint{3.352233in}{1.400000in}}{\pgfqpoint{2.407767in}{1.544118in}}%
\pgfusepath{clip}%
\pgfsetbuttcap%
\pgfsetroundjoin%
\pgfsetlinewidth{0.501875pt}%
\definecolor{currentstroke}{rgb}{0.281446,0.084320,0.407414}%
\pgfsetstrokecolor{currentstroke}%
\pgfsetdash{}{0pt}%
\pgfpathmoveto{\pgfqpoint{4.994396in}{1.998759in}}%
\pgfpathlineto{\pgfqpoint{4.941423in}{1.999019in}}%
\pgfusepath{stroke}%
\end{pgfscope}%
\begin{pgfscope}%
\pgfpathrectangle{\pgfqpoint{3.352233in}{1.400000in}}{\pgfqpoint{2.407767in}{1.544118in}}%
\pgfusepath{clip}%
\pgfsetbuttcap%
\pgfsetroundjoin%
\pgfsetlinewidth{0.501875pt}%
\definecolor{currentstroke}{rgb}{0.283091,0.110553,0.431554}%
\pgfsetstrokecolor{currentstroke}%
\pgfsetdash{}{0pt}%
\pgfpathmoveto{\pgfqpoint{4.941423in}{1.999019in}}%
\pgfpathlineto{\pgfqpoint{4.888454in}{1.999445in}}%
\pgfusepath{stroke}%
\end{pgfscope}%
\begin{pgfscope}%
\pgfpathrectangle{\pgfqpoint{3.352233in}{1.400000in}}{\pgfqpoint{2.407767in}{1.544118in}}%
\pgfusepath{clip}%
\pgfsetbuttcap%
\pgfsetroundjoin%
\pgfsetlinewidth{0.501875pt}%
\definecolor{currentstroke}{rgb}{0.283229,0.120777,0.440584}%
\pgfsetstrokecolor{currentstroke}%
\pgfsetdash{}{0pt}%
\pgfpathmoveto{\pgfqpoint{4.888454in}{1.999445in}}%
\pgfpathlineto{\pgfqpoint{4.835492in}{2.000183in}}%
\pgfusepath{stroke}%
\end{pgfscope}%
\begin{pgfscope}%
\pgfpathrectangle{\pgfqpoint{3.352233in}{1.400000in}}{\pgfqpoint{2.407767in}{1.544118in}}%
\pgfusepath{clip}%
\pgfsetbuttcap%
\pgfsetroundjoin%
\pgfsetlinewidth{0.501875pt}%
\definecolor{currentstroke}{rgb}{0.282884,0.135920,0.453427}%
\pgfsetstrokecolor{currentstroke}%
\pgfsetdash{}{0pt}%
\pgfpathmoveto{\pgfqpoint{4.835492in}{2.000183in}}%
\pgfpathlineto{\pgfqpoint{4.782527in}{2.000863in}}%
\pgfusepath{stroke}%
\end{pgfscope}%
\begin{pgfscope}%
\pgfpathrectangle{\pgfqpoint{3.352233in}{1.400000in}}{\pgfqpoint{2.407767in}{1.544118in}}%
\pgfusepath{clip}%
\pgfsetbuttcap%
\pgfsetroundjoin%
\pgfsetlinewidth{0.501875pt}%
\definecolor{currentstroke}{rgb}{0.280255,0.165693,0.476498}%
\pgfsetstrokecolor{currentstroke}%
\pgfsetdash{}{0pt}%
\pgfpathmoveto{\pgfqpoint{4.782527in}{2.000863in}}%
\pgfpathlineto{\pgfqpoint{4.729565in}{2.001627in}}%
\pgfusepath{stroke}%
\end{pgfscope}%
\begin{pgfscope}%
\pgfpathrectangle{\pgfqpoint{3.352233in}{1.400000in}}{\pgfqpoint{2.407767in}{1.544118in}}%
\pgfusepath{clip}%
\pgfsetbuttcap%
\pgfsetroundjoin%
\pgfsetlinewidth{0.501875pt}%
\definecolor{currentstroke}{rgb}{0.278826,0.175490,0.483397}%
\pgfsetstrokecolor{currentstroke}%
\pgfsetdash{}{0pt}%
\pgfpathmoveto{\pgfqpoint{4.729565in}{2.001627in}}%
\pgfpathlineto{\pgfqpoint{4.676618in}{2.002701in}}%
\pgfusepath{stroke}%
\end{pgfscope}%
\begin{pgfscope}%
\pgfpathrectangle{\pgfqpoint{3.352233in}{1.400000in}}{\pgfqpoint{2.407767in}{1.544118in}}%
\pgfusepath{clip}%
\pgfsetbuttcap%
\pgfsetroundjoin%
\pgfsetlinewidth{0.501875pt}%
\definecolor{currentstroke}{rgb}{0.281887,0.150881,0.465405}%
\pgfsetstrokecolor{currentstroke}%
\pgfsetdash{}{0pt}%
\pgfpathmoveto{\pgfqpoint{4.676618in}{2.002701in}}%
\pgfpathlineto{\pgfqpoint{4.623706in}{2.004317in}}%
\pgfusepath{stroke}%
\end{pgfscope}%
\begin{pgfscope}%
\pgfpathrectangle{\pgfqpoint{3.352233in}{1.400000in}}{\pgfqpoint{2.407767in}{1.544118in}}%
\pgfusepath{clip}%
\pgfsetbuttcap%
\pgfsetroundjoin%
\pgfsetlinewidth{0.501875pt}%
\definecolor{currentstroke}{rgb}{0.280868,0.160771,0.472899}%
\pgfsetstrokecolor{currentstroke}%
\pgfsetdash{}{0pt}%
\pgfpathmoveto{\pgfqpoint{4.623706in}{2.004317in}}%
\pgfpathlineto{\pgfqpoint{4.570839in}{2.006474in}}%
\pgfusepath{stroke}%
\end{pgfscope}%
\begin{pgfscope}%
\pgfpathrectangle{\pgfqpoint{3.352233in}{1.400000in}}{\pgfqpoint{2.407767in}{1.544118in}}%
\pgfusepath{clip}%
\pgfsetbuttcap%
\pgfsetroundjoin%
\pgfsetlinewidth{0.501875pt}%
\definecolor{currentstroke}{rgb}{0.283072,0.130895,0.449241}%
\pgfsetstrokecolor{currentstroke}%
\pgfsetdash{}{0pt}%
\pgfpathmoveto{\pgfqpoint{4.570839in}{2.006474in}}%
\pgfpathlineto{\pgfqpoint{4.518017in}{2.009038in}}%
\pgfusepath{stroke}%
\end{pgfscope}%
\begin{pgfscope}%
\pgfpathrectangle{\pgfqpoint{3.352233in}{1.400000in}}{\pgfqpoint{2.407767in}{1.544118in}}%
\pgfusepath{clip}%
\pgfsetbuttcap%
\pgfsetroundjoin%
\pgfsetlinewidth{0.501875pt}%
\definecolor{currentstroke}{rgb}{0.282623,0.140926,0.457517}%
\pgfsetstrokecolor{currentstroke}%
\pgfsetdash{}{0pt}%
\pgfpathmoveto{\pgfqpoint{4.518017in}{2.009038in}}%
\pgfpathlineto{\pgfqpoint{4.465279in}{2.012221in}}%
\pgfusepath{stroke}%
\end{pgfscope}%
\begin{pgfscope}%
\pgfpathrectangle{\pgfqpoint{3.352233in}{1.400000in}}{\pgfqpoint{2.407767in}{1.544118in}}%
\pgfusepath{clip}%
\pgfsetbuttcap%
\pgfsetroundjoin%
\pgfsetlinewidth{0.501875pt}%
\definecolor{currentstroke}{rgb}{0.283072,0.130895,0.449241}%
\pgfsetstrokecolor{currentstroke}%
\pgfsetdash{}{0pt}%
\pgfpathmoveto{\pgfqpoint{4.465279in}{2.012221in}}%
\pgfpathlineto{\pgfqpoint{4.412874in}{2.017035in}}%
\pgfusepath{stroke}%
\end{pgfscope}%
\begin{pgfscope}%
\pgfpathrectangle{\pgfqpoint{3.352233in}{1.400000in}}{\pgfqpoint{2.407767in}{1.544118in}}%
\pgfusepath{clip}%
\pgfsetbuttcap%
\pgfsetroundjoin%
\pgfsetlinewidth{0.501875pt}%
\definecolor{currentstroke}{rgb}{0.282327,0.094955,0.417331}%
\pgfsetstrokecolor{currentstroke}%
\pgfsetdash{}{0pt}%
\pgfpathmoveto{\pgfqpoint{4.412874in}{2.017035in}}%
\pgfpathlineto{\pgfqpoint{4.361012in}{2.023801in}}%
\pgfusepath{stroke}%
\end{pgfscope}%
\begin{pgfscope}%
\pgfpathrectangle{\pgfqpoint{3.352233in}{1.400000in}}{\pgfqpoint{2.407767in}{1.544118in}}%
\pgfusepath{clip}%
\pgfsetbuttcap%
\pgfsetroundjoin%
\pgfsetlinewidth{0.501875pt}%
\definecolor{currentstroke}{rgb}{0.282327,0.094955,0.417331}%
\pgfsetstrokecolor{currentstroke}%
\pgfsetdash{}{0pt}%
\pgfpathmoveto{\pgfqpoint{4.361012in}{2.023801in}}%
\pgfpathlineto{\pgfqpoint{4.310721in}{2.033876in}}%
\pgfusepath{stroke}%
\end{pgfscope}%
\begin{pgfscope}%
\pgfpathrectangle{\pgfqpoint{3.352233in}{1.400000in}}{\pgfqpoint{2.407767in}{1.544118in}}%
\pgfusepath{clip}%
\pgfsetbuttcap%
\pgfsetroundjoin%
\pgfsetlinewidth{0.501875pt}%
\definecolor{currentstroke}{rgb}{0.277018,0.050344,0.375715}%
\pgfsetstrokecolor{currentstroke}%
\pgfsetdash{}{0pt}%
\pgfpathmoveto{\pgfqpoint{4.310721in}{2.033876in}}%
\pgfpathlineto{\pgfqpoint{4.267280in}{2.048859in}}%
\pgfusepath{stroke}%
\end{pgfscope}%
\begin{pgfscope}%
\pgfpathrectangle{\pgfqpoint{3.352233in}{1.400000in}}{\pgfqpoint{2.407767in}{1.544118in}}%
\pgfusepath{clip}%
\pgfsetbuttcap%
\pgfsetroundjoin%
\pgfsetlinewidth{0.501875pt}%
\definecolor{currentstroke}{rgb}{0.273809,0.031497,0.358853}%
\pgfsetstrokecolor{currentstroke}%
\pgfsetdash{}{0pt}%
\pgfpathmoveto{\pgfqpoint{4.267280in}{2.048859in}}%
\pgfpathlineto{\pgfqpoint{4.267280in}{2.048859in}}%
\pgfusepath{stroke}%
\end{pgfscope}%
\begin{pgfscope}%
\pgfpathrectangle{\pgfqpoint{3.352233in}{1.400000in}}{\pgfqpoint{2.407767in}{1.544118in}}%
\pgfusepath{clip}%
\pgfsetbuttcap%
\pgfsetroundjoin%
\pgfsetlinewidth{0.501875pt}%
\definecolor{currentstroke}{rgb}{0.268510,0.009605,0.335427}%
\pgfsetstrokecolor{currentstroke}%
\pgfsetdash{}{0pt}%
\pgfpathmoveto{\pgfqpoint{5.206279in}{2.033074in}}%
\pgfpathlineto{\pgfqpoint{5.153306in}{2.032872in}}%
\pgfusepath{stroke}%
\end{pgfscope}%
\begin{pgfscope}%
\pgfpathrectangle{\pgfqpoint{3.352233in}{1.400000in}}{\pgfqpoint{2.407767in}{1.544118in}}%
\pgfusepath{clip}%
\pgfsetbuttcap%
\pgfsetroundjoin%
\pgfsetlinewidth{0.501875pt}%
\definecolor{currentstroke}{rgb}{0.271305,0.019942,0.347269}%
\pgfsetstrokecolor{currentstroke}%
\pgfsetdash{}{0pt}%
\pgfpathmoveto{\pgfqpoint{5.153306in}{2.032872in}}%
\pgfpathlineto{\pgfqpoint{5.100335in}{2.032899in}}%
\pgfusepath{stroke}%
\end{pgfscope}%
\begin{pgfscope}%
\pgfpathrectangle{\pgfqpoint{3.352233in}{1.400000in}}{\pgfqpoint{2.407767in}{1.544118in}}%
\pgfusepath{clip}%
\pgfsetbuttcap%
\pgfsetroundjoin%
\pgfsetlinewidth{0.501875pt}%
\definecolor{currentstroke}{rgb}{0.277018,0.050344,0.375715}%
\pgfsetstrokecolor{currentstroke}%
\pgfsetdash{}{0pt}%
\pgfpathmoveto{\pgfqpoint{5.100335in}{2.032899in}}%
\pgfpathlineto{\pgfqpoint{5.047362in}{2.033177in}}%
\pgfusepath{stroke}%
\end{pgfscope}%
\begin{pgfscope}%
\pgfpathrectangle{\pgfqpoint{3.352233in}{1.400000in}}{\pgfqpoint{2.407767in}{1.544118in}}%
\pgfusepath{clip}%
\pgfsetbuttcap%
\pgfsetroundjoin%
\pgfsetlinewidth{0.501875pt}%
\definecolor{currentstroke}{rgb}{0.279566,0.067836,0.391917}%
\pgfsetstrokecolor{currentstroke}%
\pgfsetdash{}{0pt}%
\pgfpathmoveto{\pgfqpoint{5.047362in}{2.033177in}}%
\pgfpathlineto{\pgfqpoint{4.994387in}{2.033322in}}%
\pgfusepath{stroke}%
\end{pgfscope}%
\begin{pgfscope}%
\pgfpathrectangle{\pgfqpoint{3.352233in}{1.400000in}}{\pgfqpoint{2.407767in}{1.544118in}}%
\pgfusepath{clip}%
\pgfsetbuttcap%
\pgfsetroundjoin%
\pgfsetlinewidth{0.501875pt}%
\definecolor{currentstroke}{rgb}{0.280267,0.073417,0.397163}%
\pgfsetstrokecolor{currentstroke}%
\pgfsetdash{}{0pt}%
\pgfpathmoveto{\pgfqpoint{4.994387in}{2.033322in}}%
\pgfpathlineto{\pgfqpoint{4.941412in}{2.033402in}}%
\pgfusepath{stroke}%
\end{pgfscope}%
\begin{pgfscope}%
\pgfpathrectangle{\pgfqpoint{3.352233in}{1.400000in}}{\pgfqpoint{2.407767in}{1.544118in}}%
\pgfusepath{clip}%
\pgfsetbuttcap%
\pgfsetroundjoin%
\pgfsetlinewidth{0.501875pt}%
\definecolor{currentstroke}{rgb}{0.283091,0.110553,0.431554}%
\pgfsetstrokecolor{currentstroke}%
\pgfsetdash{}{0pt}%
\pgfpathmoveto{\pgfqpoint{4.941412in}{2.033402in}}%
\pgfpathlineto{\pgfqpoint{4.888437in}{2.033571in}}%
\pgfusepath{stroke}%
\end{pgfscope}%
\begin{pgfscope}%
\pgfpathrectangle{\pgfqpoint{3.352233in}{1.400000in}}{\pgfqpoint{2.407767in}{1.544118in}}%
\pgfusepath{clip}%
\pgfsetbuttcap%
\pgfsetroundjoin%
\pgfsetlinewidth{0.501875pt}%
\definecolor{currentstroke}{rgb}{0.283072,0.130895,0.449241}%
\pgfsetstrokecolor{currentstroke}%
\pgfsetdash{}{0pt}%
\pgfpathmoveto{\pgfqpoint{4.888437in}{2.033571in}}%
\pgfpathlineto{\pgfqpoint{4.835462in}{2.033725in}}%
\pgfusepath{stroke}%
\end{pgfscope}%
\begin{pgfscope}%
\pgfpathrectangle{\pgfqpoint{3.352233in}{1.400000in}}{\pgfqpoint{2.407767in}{1.544118in}}%
\pgfusepath{clip}%
\pgfsetbuttcap%
\pgfsetroundjoin%
\pgfsetlinewidth{0.501875pt}%
\definecolor{currentstroke}{rgb}{0.280868,0.160771,0.472899}%
\pgfsetstrokecolor{currentstroke}%
\pgfsetdash{}{0pt}%
\pgfpathmoveto{\pgfqpoint{4.835462in}{2.033725in}}%
\pgfpathlineto{\pgfqpoint{4.782486in}{2.033696in}}%
\pgfusepath{stroke}%
\end{pgfscope}%
\begin{pgfscope}%
\pgfpathrectangle{\pgfqpoint{3.352233in}{1.400000in}}{\pgfqpoint{2.407767in}{1.544118in}}%
\pgfusepath{clip}%
\pgfsetbuttcap%
\pgfsetroundjoin%
\pgfsetlinewidth{0.501875pt}%
\definecolor{currentstroke}{rgb}{0.278826,0.175490,0.483397}%
\pgfsetstrokecolor{currentstroke}%
\pgfsetdash{}{0pt}%
\pgfpathmoveto{\pgfqpoint{4.782486in}{2.033696in}}%
\pgfpathlineto{\pgfqpoint{4.729511in}{2.033759in}}%
\pgfusepath{stroke}%
\end{pgfscope}%
\begin{pgfscope}%
\pgfpathrectangle{\pgfqpoint{3.352233in}{1.400000in}}{\pgfqpoint{2.407767in}{1.544118in}}%
\pgfusepath{clip}%
\pgfsetbuttcap%
\pgfsetroundjoin%
\pgfsetlinewidth{0.501875pt}%
\definecolor{currentstroke}{rgb}{0.278012,0.180367,0.486697}%
\pgfsetstrokecolor{currentstroke}%
\pgfsetdash{}{0pt}%
\pgfpathmoveto{\pgfqpoint{4.729511in}{2.033759in}}%
\pgfpathlineto{\pgfqpoint{4.676543in}{2.034237in}}%
\pgfusepath{stroke}%
\end{pgfscope}%
\begin{pgfscope}%
\pgfpathrectangle{\pgfqpoint{3.352233in}{1.400000in}}{\pgfqpoint{2.407767in}{1.544118in}}%
\pgfusepath{clip}%
\pgfsetbuttcap%
\pgfsetroundjoin%
\pgfsetlinewidth{0.501875pt}%
\definecolor{currentstroke}{rgb}{0.276194,0.190074,0.493001}%
\pgfsetstrokecolor{currentstroke}%
\pgfsetdash{}{0pt}%
\pgfpathmoveto{\pgfqpoint{4.676543in}{2.034237in}}%
\pgfpathlineto{\pgfqpoint{4.623585in}{2.035117in}}%
\pgfusepath{stroke}%
\end{pgfscope}%
\begin{pgfscope}%
\pgfpathrectangle{\pgfqpoint{3.352233in}{1.400000in}}{\pgfqpoint{2.407767in}{1.544118in}}%
\pgfusepath{clip}%
\pgfsetbuttcap%
\pgfsetroundjoin%
\pgfsetlinewidth{0.501875pt}%
\definecolor{currentstroke}{rgb}{0.278826,0.175490,0.483397}%
\pgfsetstrokecolor{currentstroke}%
\pgfsetdash{}{0pt}%
\pgfpathmoveto{\pgfqpoint{4.623585in}{2.035117in}}%
\pgfpathlineto{\pgfqpoint{4.570645in}{2.036316in}}%
\pgfusepath{stroke}%
\end{pgfscope}%
\begin{pgfscope}%
\pgfpathrectangle{\pgfqpoint{3.352233in}{1.400000in}}{\pgfqpoint{2.407767in}{1.544118in}}%
\pgfusepath{clip}%
\pgfsetbuttcap%
\pgfsetroundjoin%
\pgfsetlinewidth{0.501875pt}%
\definecolor{currentstroke}{rgb}{0.281412,0.155834,0.469201}%
\pgfsetstrokecolor{currentstroke}%
\pgfsetdash{}{0pt}%
\pgfpathmoveto{\pgfqpoint{4.570645in}{2.036316in}}%
\pgfpathlineto{\pgfqpoint{4.517744in}{2.038072in}}%
\pgfusepath{stroke}%
\end{pgfscope}%
\begin{pgfscope}%
\pgfpathrectangle{\pgfqpoint{3.352233in}{1.400000in}}{\pgfqpoint{2.407767in}{1.544118in}}%
\pgfusepath{clip}%
\pgfsetbuttcap%
\pgfsetroundjoin%
\pgfsetlinewidth{0.501875pt}%
\definecolor{currentstroke}{rgb}{0.282884,0.135920,0.453427}%
\pgfsetstrokecolor{currentstroke}%
\pgfsetdash{}{0pt}%
\pgfpathmoveto{\pgfqpoint{4.517744in}{2.038072in}}%
\pgfpathlineto{\pgfqpoint{4.464895in}{2.040394in}}%
\pgfusepath{stroke}%
\end{pgfscope}%
\begin{pgfscope}%
\pgfpathrectangle{\pgfqpoint{3.352233in}{1.400000in}}{\pgfqpoint{2.407767in}{1.544118in}}%
\pgfusepath{clip}%
\pgfsetbuttcap%
\pgfsetroundjoin%
\pgfsetlinewidth{0.501875pt}%
\definecolor{currentstroke}{rgb}{0.269944,0.014625,0.341379}%
\pgfsetstrokecolor{currentstroke}%
\pgfsetdash{}{0pt}%
\pgfpathmoveto{\pgfqpoint{5.206279in}{2.067820in}}%
\pgfpathlineto{\pgfqpoint{5.153315in}{2.067477in}}%
\pgfusepath{stroke}%
\end{pgfscope}%
\begin{pgfscope}%
\pgfpathrectangle{\pgfqpoint{3.352233in}{1.400000in}}{\pgfqpoint{2.407767in}{1.544118in}}%
\pgfusepath{clip}%
\pgfsetbuttcap%
\pgfsetroundjoin%
\pgfsetlinewidth{0.501875pt}%
\definecolor{currentstroke}{rgb}{0.273809,0.031497,0.358853}%
\pgfsetstrokecolor{currentstroke}%
\pgfsetdash{}{0pt}%
\pgfpathmoveto{\pgfqpoint{5.153315in}{2.067477in}}%
\pgfpathlineto{\pgfqpoint{5.100344in}{2.067619in}}%
\pgfusepath{stroke}%
\end{pgfscope}%
\begin{pgfscope}%
\pgfpathrectangle{\pgfqpoint{3.352233in}{1.400000in}}{\pgfqpoint{2.407767in}{1.544118in}}%
\pgfusepath{clip}%
\pgfsetbuttcap%
\pgfsetroundjoin%
\pgfsetlinewidth{0.501875pt}%
\definecolor{currentstroke}{rgb}{0.277018,0.050344,0.375715}%
\pgfsetstrokecolor{currentstroke}%
\pgfsetdash{}{0pt}%
\pgfpathmoveto{\pgfqpoint{5.100344in}{2.067619in}}%
\pgfpathlineto{\pgfqpoint{5.047373in}{2.067928in}}%
\pgfusepath{stroke}%
\end{pgfscope}%
\begin{pgfscope}%
\pgfpathrectangle{\pgfqpoint{3.352233in}{1.400000in}}{\pgfqpoint{2.407767in}{1.544118in}}%
\pgfusepath{clip}%
\pgfsetbuttcap%
\pgfsetroundjoin%
\pgfsetlinewidth{0.501875pt}%
\definecolor{currentstroke}{rgb}{0.277941,0.056324,0.381191}%
\pgfsetstrokecolor{currentstroke}%
\pgfsetdash{}{0pt}%
\pgfpathmoveto{\pgfqpoint{5.047373in}{2.067928in}}%
\pgfpathlineto{\pgfqpoint{4.994398in}{2.067965in}}%
\pgfusepath{stroke}%
\end{pgfscope}%
\begin{pgfscope}%
\pgfpathrectangle{\pgfqpoint{3.352233in}{1.400000in}}{\pgfqpoint{2.407767in}{1.544118in}}%
\pgfusepath{clip}%
\pgfsetbuttcap%
\pgfsetroundjoin%
\pgfsetlinewidth{0.501875pt}%
\definecolor{currentstroke}{rgb}{0.280894,0.078907,0.402329}%
\pgfsetstrokecolor{currentstroke}%
\pgfsetdash{}{0pt}%
\pgfpathmoveto{\pgfqpoint{4.994398in}{2.067965in}}%
\pgfpathlineto{\pgfqpoint{4.941423in}{2.067979in}}%
\pgfusepath{stroke}%
\end{pgfscope}%
\begin{pgfscope}%
\pgfpathrectangle{\pgfqpoint{3.352233in}{1.400000in}}{\pgfqpoint{2.407767in}{1.544118in}}%
\pgfusepath{clip}%
\pgfsetbuttcap%
\pgfsetroundjoin%
\pgfsetlinewidth{0.501875pt}%
\definecolor{currentstroke}{rgb}{0.283091,0.110553,0.431554}%
\pgfsetstrokecolor{currentstroke}%
\pgfsetdash{}{0pt}%
\pgfpathmoveto{\pgfqpoint{4.941423in}{2.067979in}}%
\pgfpathlineto{\pgfqpoint{4.888448in}{2.067919in}}%
\pgfusepath{stroke}%
\end{pgfscope}%
\begin{pgfscope}%
\pgfpathrectangle{\pgfqpoint{3.352233in}{1.400000in}}{\pgfqpoint{2.407767in}{1.544118in}}%
\pgfusepath{clip}%
\pgfsetbuttcap%
\pgfsetroundjoin%
\pgfsetlinewidth{0.501875pt}%
\definecolor{currentstroke}{rgb}{0.282884,0.135920,0.453427}%
\pgfsetstrokecolor{currentstroke}%
\pgfsetdash{}{0pt}%
\pgfpathmoveto{\pgfqpoint{4.888448in}{2.067919in}}%
\pgfpathlineto{\pgfqpoint{4.835474in}{2.067810in}}%
\pgfusepath{stroke}%
\end{pgfscope}%
\begin{pgfscope}%
\pgfpathrectangle{\pgfqpoint{3.352233in}{1.400000in}}{\pgfqpoint{2.407767in}{1.544118in}}%
\pgfusepath{clip}%
\pgfsetbuttcap%
\pgfsetroundjoin%
\pgfsetlinewidth{0.501875pt}%
\definecolor{currentstroke}{rgb}{0.282290,0.145912,0.461510}%
\pgfsetstrokecolor{currentstroke}%
\pgfsetdash{}{0pt}%
\pgfpathmoveto{\pgfqpoint{4.835474in}{2.067810in}}%
\pgfpathlineto{\pgfqpoint{4.782500in}{2.067872in}}%
\pgfusepath{stroke}%
\end{pgfscope}%
\begin{pgfscope}%
\pgfpathrectangle{\pgfqpoint{3.352233in}{1.400000in}}{\pgfqpoint{2.407767in}{1.544118in}}%
\pgfusepath{clip}%
\pgfsetbuttcap%
\pgfsetroundjoin%
\pgfsetlinewidth{0.501875pt}%
\definecolor{currentstroke}{rgb}{0.279574,0.170599,0.479997}%
\pgfsetstrokecolor{currentstroke}%
\pgfsetdash{}{0pt}%
\pgfpathmoveto{\pgfqpoint{4.782500in}{2.067872in}}%
\pgfpathlineto{\pgfqpoint{4.729525in}{2.068037in}}%
\pgfusepath{stroke}%
\end{pgfscope}%
\begin{pgfscope}%
\pgfpathrectangle{\pgfqpoint{3.352233in}{1.400000in}}{\pgfqpoint{2.407767in}{1.544118in}}%
\pgfusepath{clip}%
\pgfsetbuttcap%
\pgfsetroundjoin%
\pgfsetlinewidth{0.501875pt}%
\definecolor{currentstroke}{rgb}{0.276194,0.190074,0.493001}%
\pgfsetstrokecolor{currentstroke}%
\pgfsetdash{}{0pt}%
\pgfpathmoveto{\pgfqpoint{4.729525in}{2.068037in}}%
\pgfpathlineto{\pgfqpoint{4.676551in}{2.068205in}}%
\pgfusepath{stroke}%
\end{pgfscope}%
\begin{pgfscope}%
\pgfpathrectangle{\pgfqpoint{3.352233in}{1.400000in}}{\pgfqpoint{2.407767in}{1.544118in}}%
\pgfusepath{clip}%
\pgfsetbuttcap%
\pgfsetroundjoin%
\pgfsetlinewidth{0.501875pt}%
\definecolor{currentstroke}{rgb}{0.278012,0.180367,0.486697}%
\pgfsetstrokecolor{currentstroke}%
\pgfsetdash{}{0pt}%
\pgfpathmoveto{\pgfqpoint{4.676551in}{2.068205in}}%
\pgfpathlineto{\pgfqpoint{4.623580in}{2.068476in}}%
\pgfusepath{stroke}%
\end{pgfscope}%
\begin{pgfscope}%
\pgfpathrectangle{\pgfqpoint{3.352233in}{1.400000in}}{\pgfqpoint{2.407767in}{1.544118in}}%
\pgfusepath{clip}%
\pgfsetbuttcap%
\pgfsetroundjoin%
\pgfsetlinewidth{0.501875pt}%
\definecolor{currentstroke}{rgb}{0.280868,0.160771,0.472899}%
\pgfsetstrokecolor{currentstroke}%
\pgfsetdash{}{0pt}%
\pgfpathmoveto{\pgfqpoint{4.623580in}{2.068476in}}%
\pgfpathlineto{\pgfqpoint{4.570619in}{2.069086in}}%
\pgfusepath{stroke}%
\end{pgfscope}%
\begin{pgfscope}%
\pgfpathrectangle{\pgfqpoint{3.352233in}{1.400000in}}{\pgfqpoint{2.407767in}{1.544118in}}%
\pgfusepath{clip}%
\pgfsetbuttcap%
\pgfsetroundjoin%
\pgfsetlinewidth{0.501875pt}%
\definecolor{currentstroke}{rgb}{0.278826,0.175490,0.483397}%
\pgfsetstrokecolor{currentstroke}%
\pgfsetdash{}{0pt}%
\pgfpathmoveto{\pgfqpoint{4.570619in}{2.069086in}}%
\pgfpathlineto{\pgfqpoint{4.517666in}{2.069956in}}%
\pgfusepath{stroke}%
\end{pgfscope}%
\begin{pgfscope}%
\pgfpathrectangle{\pgfqpoint{3.352233in}{1.400000in}}{\pgfqpoint{2.407767in}{1.544118in}}%
\pgfusepath{clip}%
\pgfsetbuttcap%
\pgfsetroundjoin%
\pgfsetlinewidth{0.501875pt}%
\definecolor{currentstroke}{rgb}{0.283229,0.120777,0.440584}%
\pgfsetstrokecolor{currentstroke}%
\pgfsetdash{}{0pt}%
\pgfpathmoveto{\pgfqpoint{4.517666in}{2.069956in}}%
\pgfpathlineto{\pgfqpoint{4.464726in}{2.071100in}}%
\pgfusepath{stroke}%
\end{pgfscope}%
\begin{pgfscope}%
\pgfpathrectangle{\pgfqpoint{3.352233in}{1.400000in}}{\pgfqpoint{2.407767in}{1.544118in}}%
\pgfusepath{clip}%
\pgfsetbuttcap%
\pgfsetroundjoin%
\pgfsetlinewidth{0.501875pt}%
\definecolor{currentstroke}{rgb}{0.283091,0.110553,0.431554}%
\pgfsetstrokecolor{currentstroke}%
\pgfsetdash{}{0pt}%
\pgfpathmoveto{\pgfqpoint{4.464726in}{2.071100in}}%
\pgfpathlineto{\pgfqpoint{4.411866in}{2.073160in}}%
\pgfusepath{stroke}%
\end{pgfscope}%
\begin{pgfscope}%
\pgfpathrectangle{\pgfqpoint{3.352233in}{1.400000in}}{\pgfqpoint{2.407767in}{1.544118in}}%
\pgfusepath{clip}%
\pgfsetbuttcap%
\pgfsetroundjoin%
\pgfsetlinewidth{0.501875pt}%
\definecolor{currentstroke}{rgb}{0.281446,0.084320,0.407414}%
\pgfsetstrokecolor{currentstroke}%
\pgfsetdash{}{0pt}%
\pgfpathmoveto{\pgfqpoint{4.411866in}{2.073160in}}%
\pgfpathlineto{\pgfqpoint{4.359078in}{2.075983in}}%
\pgfusepath{stroke}%
\end{pgfscope}%
\begin{pgfscope}%
\pgfpathrectangle{\pgfqpoint{3.352233in}{1.400000in}}{\pgfqpoint{2.407767in}{1.544118in}}%
\pgfusepath{clip}%
\pgfsetbuttcap%
\pgfsetroundjoin%
\pgfsetlinewidth{0.501875pt}%
\definecolor{currentstroke}{rgb}{0.280267,0.073417,0.397163}%
\pgfsetstrokecolor{currentstroke}%
\pgfsetdash{}{0pt}%
\pgfpathmoveto{\pgfqpoint{4.359078in}{2.075983in}}%
\pgfpathlineto{\pgfqpoint{4.306527in}{2.079791in}}%
\pgfusepath{stroke}%
\end{pgfscope}%
\begin{pgfscope}%
\pgfpathrectangle{\pgfqpoint{3.352233in}{1.400000in}}{\pgfqpoint{2.407767in}{1.544118in}}%
\pgfusepath{clip}%
\pgfsetbuttcap%
\pgfsetroundjoin%
\pgfsetlinewidth{0.501875pt}%
\definecolor{currentstroke}{rgb}{0.277018,0.050344,0.375715}%
\pgfsetstrokecolor{currentstroke}%
\pgfsetdash{}{0pt}%
\pgfpathmoveto{\pgfqpoint{4.306527in}{2.079791in}}%
\pgfpathlineto{\pgfqpoint{4.306527in}{2.079791in}}%
\pgfusepath{stroke}%
\end{pgfscope}%
\begin{pgfscope}%
\pgfpathrectangle{\pgfqpoint{3.352233in}{1.400000in}}{\pgfqpoint{2.407767in}{1.544118in}}%
\pgfusepath{clip}%
\pgfsetbuttcap%
\pgfsetroundjoin%
\pgfsetlinewidth{0.501875pt}%
\definecolor{currentstroke}{rgb}{0.277018,0.050344,0.375715}%
\pgfsetstrokecolor{currentstroke}%
\pgfsetdash{}{0pt}%
\pgfpathmoveto{\pgfqpoint{4.306527in}{2.079791in}}%
\pgfpathlineto{\pgfqpoint{4.306527in}{2.079791in}}%
\pgfusepath{stroke}%
\end{pgfscope}%
\begin{pgfscope}%
\pgfpathrectangle{\pgfqpoint{3.352233in}{1.400000in}}{\pgfqpoint{2.407767in}{1.544118in}}%
\pgfusepath{clip}%
\pgfsetbuttcap%
\pgfsetroundjoin%
\pgfsetlinewidth{0.501875pt}%
\definecolor{currentstroke}{rgb}{0.277018,0.050344,0.375715}%
\pgfsetstrokecolor{currentstroke}%
\pgfsetdash{}{0pt}%
\pgfpathmoveto{\pgfqpoint{4.306527in}{2.079791in}}%
\pgfpathlineto{\pgfqpoint{4.282697in}{2.083143in}}%
\pgfusepath{stroke}%
\end{pgfscope}%
\begin{pgfscope}%
\pgfpathrectangle{\pgfqpoint{3.352233in}{1.400000in}}{\pgfqpoint{2.407767in}{1.544118in}}%
\pgfusepath{clip}%
\pgfsetbuttcap%
\pgfsetroundjoin%
\pgfsetlinewidth{0.501875pt}%
\definecolor{currentstroke}{rgb}{0.269944,0.014625,0.341379}%
\pgfsetstrokecolor{currentstroke}%
\pgfsetdash{}{0pt}%
\pgfpathmoveto{\pgfqpoint{5.206279in}{2.102567in}}%
\pgfpathlineto{\pgfqpoint{5.153308in}{2.102207in}}%
\pgfusepath{stroke}%
\end{pgfscope}%
\begin{pgfscope}%
\pgfpathrectangle{\pgfqpoint{3.352233in}{1.400000in}}{\pgfqpoint{2.407767in}{1.544118in}}%
\pgfusepath{clip}%
\pgfsetbuttcap%
\pgfsetroundjoin%
\pgfsetlinewidth{0.501875pt}%
\definecolor{currentstroke}{rgb}{0.272594,0.025563,0.353093}%
\pgfsetstrokecolor{currentstroke}%
\pgfsetdash{}{0pt}%
\pgfpathmoveto{\pgfqpoint{5.153308in}{2.102207in}}%
\pgfpathlineto{\pgfqpoint{5.100337in}{2.101923in}}%
\pgfusepath{stroke}%
\end{pgfscope}%
\begin{pgfscope}%
\pgfpathrectangle{\pgfqpoint{3.352233in}{1.400000in}}{\pgfqpoint{2.407767in}{1.544118in}}%
\pgfusepath{clip}%
\pgfsetbuttcap%
\pgfsetroundjoin%
\pgfsetlinewidth{0.501875pt}%
\definecolor{currentstroke}{rgb}{0.276022,0.044167,0.370164}%
\pgfsetstrokecolor{currentstroke}%
\pgfsetdash{}{0pt}%
\pgfpathmoveto{\pgfqpoint{5.100337in}{2.101923in}}%
\pgfpathlineto{\pgfqpoint{5.047365in}{2.101730in}}%
\pgfusepath{stroke}%
\end{pgfscope}%
\begin{pgfscope}%
\pgfpathrectangle{\pgfqpoint{3.352233in}{1.400000in}}{\pgfqpoint{2.407767in}{1.544118in}}%
\pgfusepath{clip}%
\pgfsetbuttcap%
\pgfsetroundjoin%
\pgfsetlinewidth{0.501875pt}%
\definecolor{currentstroke}{rgb}{0.279566,0.067836,0.391917}%
\pgfsetstrokecolor{currentstroke}%
\pgfsetdash{}{0pt}%
\pgfpathmoveto{\pgfqpoint{5.047365in}{2.101730in}}%
\pgfpathlineto{\pgfqpoint{4.994390in}{2.101663in}}%
\pgfusepath{stroke}%
\end{pgfscope}%
\begin{pgfscope}%
\pgfpathrectangle{\pgfqpoint{3.352233in}{1.400000in}}{\pgfqpoint{2.407767in}{1.544118in}}%
\pgfusepath{clip}%
\pgfsetbuttcap%
\pgfsetroundjoin%
\pgfsetlinewidth{0.501875pt}%
\definecolor{currentstroke}{rgb}{0.280894,0.078907,0.402329}%
\pgfsetstrokecolor{currentstroke}%
\pgfsetdash{}{0pt}%
\pgfpathmoveto{\pgfqpoint{4.994390in}{2.101663in}}%
\pgfpathlineto{\pgfqpoint{4.941414in}{2.101610in}}%
\pgfusepath{stroke}%
\end{pgfscope}%
\begin{pgfscope}%
\pgfpathrectangle{\pgfqpoint{3.352233in}{1.400000in}}{\pgfqpoint{2.407767in}{1.544118in}}%
\pgfusepath{clip}%
\pgfsetbuttcap%
\pgfsetroundjoin%
\pgfsetlinewidth{0.501875pt}%
\definecolor{currentstroke}{rgb}{0.283091,0.110553,0.431554}%
\pgfsetstrokecolor{currentstroke}%
\pgfsetdash{}{0pt}%
\pgfpathmoveto{\pgfqpoint{4.941414in}{2.101610in}}%
\pgfpathlineto{\pgfqpoint{4.888439in}{2.101705in}}%
\pgfusepath{stroke}%
\end{pgfscope}%
\begin{pgfscope}%
\pgfpathrectangle{\pgfqpoint{3.352233in}{1.400000in}}{\pgfqpoint{2.407767in}{1.544118in}}%
\pgfusepath{clip}%
\pgfsetbuttcap%
\pgfsetroundjoin%
\pgfsetlinewidth{0.501875pt}%
\definecolor{currentstroke}{rgb}{0.282884,0.135920,0.453427}%
\pgfsetstrokecolor{currentstroke}%
\pgfsetdash{}{0pt}%
\pgfpathmoveto{\pgfqpoint{4.888439in}{2.101705in}}%
\pgfpathlineto{\pgfqpoint{4.835465in}{2.101750in}}%
\pgfusepath{stroke}%
\end{pgfscope}%
\begin{pgfscope}%
\pgfpathrectangle{\pgfqpoint{3.352233in}{1.400000in}}{\pgfqpoint{2.407767in}{1.544118in}}%
\pgfusepath{clip}%
\pgfsetbuttcap%
\pgfsetroundjoin%
\pgfsetlinewidth{0.501875pt}%
\definecolor{currentstroke}{rgb}{0.281887,0.150881,0.465405}%
\pgfsetstrokecolor{currentstroke}%
\pgfsetdash{}{0pt}%
\pgfpathmoveto{\pgfqpoint{4.835465in}{2.101750in}}%
\pgfpathlineto{\pgfqpoint{4.782489in}{2.101704in}}%
\pgfusepath{stroke}%
\end{pgfscope}%
\begin{pgfscope}%
\pgfpathrectangle{\pgfqpoint{3.352233in}{1.400000in}}{\pgfqpoint{2.407767in}{1.544118in}}%
\pgfusepath{clip}%
\pgfsetbuttcap%
\pgfsetroundjoin%
\pgfsetlinewidth{0.501875pt}%
\definecolor{currentstroke}{rgb}{0.278826,0.175490,0.483397}%
\pgfsetstrokecolor{currentstroke}%
\pgfsetdash{}{0pt}%
\pgfpathmoveto{\pgfqpoint{4.782489in}{2.101704in}}%
\pgfpathlineto{\pgfqpoint{4.729514in}{2.101840in}}%
\pgfusepath{stroke}%
\end{pgfscope}%
\begin{pgfscope}%
\pgfpathrectangle{\pgfqpoint{3.352233in}{1.400000in}}{\pgfqpoint{2.407767in}{1.544118in}}%
\pgfusepath{clip}%
\pgfsetbuttcap%
\pgfsetroundjoin%
\pgfsetlinewidth{0.501875pt}%
\definecolor{currentstroke}{rgb}{0.280868,0.160771,0.472899}%
\pgfsetstrokecolor{currentstroke}%
\pgfsetdash{}{0pt}%
\pgfpathmoveto{\pgfqpoint{4.729514in}{2.101840in}}%
\pgfpathlineto{\pgfqpoint{4.676539in}{2.101986in}}%
\pgfusepath{stroke}%
\end{pgfscope}%
\begin{pgfscope}%
\pgfpathrectangle{\pgfqpoint{3.352233in}{1.400000in}}{\pgfqpoint{2.407767in}{1.544118in}}%
\pgfusepath{clip}%
\pgfsetbuttcap%
\pgfsetroundjoin%
\pgfsetlinewidth{0.501875pt}%
\definecolor{currentstroke}{rgb}{0.278826,0.175490,0.483397}%
\pgfsetstrokecolor{currentstroke}%
\pgfsetdash{}{0pt}%
\pgfpathmoveto{\pgfqpoint{4.676539in}{2.101986in}}%
\pgfpathlineto{\pgfqpoint{4.623566in}{2.102163in}}%
\pgfusepath{stroke}%
\end{pgfscope}%
\begin{pgfscope}%
\pgfpathrectangle{\pgfqpoint{3.352233in}{1.400000in}}{\pgfqpoint{2.407767in}{1.544118in}}%
\pgfusepath{clip}%
\pgfsetbuttcap%
\pgfsetroundjoin%
\pgfsetlinewidth{0.501875pt}%
\definecolor{currentstroke}{rgb}{0.281887,0.150881,0.465405}%
\pgfsetstrokecolor{currentstroke}%
\pgfsetdash{}{0pt}%
\pgfpathmoveto{\pgfqpoint{4.623566in}{2.102163in}}%
\pgfpathlineto{\pgfqpoint{4.570595in}{2.102505in}}%
\pgfusepath{stroke}%
\end{pgfscope}%
\begin{pgfscope}%
\pgfpathrectangle{\pgfqpoint{3.352233in}{1.400000in}}{\pgfqpoint{2.407767in}{1.544118in}}%
\pgfusepath{clip}%
\pgfsetbuttcap%
\pgfsetroundjoin%
\pgfsetlinewidth{0.501875pt}%
\definecolor{currentstroke}{rgb}{0.281412,0.155834,0.469201}%
\pgfsetstrokecolor{currentstroke}%
\pgfsetdash{}{0pt}%
\pgfpathmoveto{\pgfqpoint{4.570595in}{2.102505in}}%
\pgfpathlineto{\pgfqpoint{4.517625in}{2.102895in}}%
\pgfusepath{stroke}%
\end{pgfscope}%
\begin{pgfscope}%
\pgfpathrectangle{\pgfqpoint{3.352233in}{1.400000in}}{\pgfqpoint{2.407767in}{1.544118in}}%
\pgfusepath{clip}%
\pgfsetbuttcap%
\pgfsetroundjoin%
\pgfsetlinewidth{0.501875pt}%
\definecolor{currentstroke}{rgb}{0.281412,0.155834,0.469201}%
\pgfsetstrokecolor{currentstroke}%
\pgfsetdash{}{0pt}%
\pgfpathmoveto{\pgfqpoint{4.517625in}{2.102895in}}%
\pgfpathlineto{\pgfqpoint{4.464653in}{2.103169in}}%
\pgfusepath{stroke}%
\end{pgfscope}%
\begin{pgfscope}%
\pgfpathrectangle{\pgfqpoint{3.352233in}{1.400000in}}{\pgfqpoint{2.407767in}{1.544118in}}%
\pgfusepath{clip}%
\pgfsetbuttcap%
\pgfsetroundjoin%
\pgfsetlinewidth{0.501875pt}%
\definecolor{currentstroke}{rgb}{0.283072,0.130895,0.449241}%
\pgfsetstrokecolor{currentstroke}%
\pgfsetdash{}{0pt}%
\pgfpathmoveto{\pgfqpoint{4.464653in}{2.103169in}}%
\pgfpathlineto{\pgfqpoint{4.411687in}{2.103543in}}%
\pgfusepath{stroke}%
\end{pgfscope}%
\begin{pgfscope}%
\pgfpathrectangle{\pgfqpoint{3.352233in}{1.400000in}}{\pgfqpoint{2.407767in}{1.544118in}}%
\pgfusepath{clip}%
\pgfsetbuttcap%
\pgfsetroundjoin%
\pgfsetlinewidth{0.501875pt}%
\definecolor{currentstroke}{rgb}{0.283091,0.110553,0.431554}%
\pgfsetstrokecolor{currentstroke}%
\pgfsetdash{}{0pt}%
\pgfpathmoveto{\pgfqpoint{4.411687in}{2.103543in}}%
\pgfpathlineto{\pgfqpoint{4.358766in}{2.104518in}}%
\pgfusepath{stroke}%
\end{pgfscope}%
\begin{pgfscope}%
\pgfpathrectangle{\pgfqpoint{3.352233in}{1.400000in}}{\pgfqpoint{2.407767in}{1.544118in}}%
\pgfusepath{clip}%
\pgfsetbuttcap%
\pgfsetroundjoin%
\pgfsetlinewidth{0.501875pt}%
\definecolor{currentstroke}{rgb}{0.280267,0.073417,0.397163}%
\pgfsetstrokecolor{currentstroke}%
\pgfsetdash{}{0pt}%
\pgfpathmoveto{\pgfqpoint{4.358766in}{2.104518in}}%
\pgfpathlineto{\pgfqpoint{4.305917in}{2.106219in}}%
\pgfusepath{stroke}%
\end{pgfscope}%
\begin{pgfscope}%
\pgfpathrectangle{\pgfqpoint{3.352233in}{1.400000in}}{\pgfqpoint{2.407767in}{1.544118in}}%
\pgfusepath{clip}%
\pgfsetbuttcap%
\pgfsetroundjoin%
\pgfsetlinewidth{0.501875pt}%
\definecolor{currentstroke}{rgb}{0.269944,0.014625,0.341379}%
\pgfsetstrokecolor{currentstroke}%
\pgfsetdash{}{0pt}%
\pgfpathmoveto{\pgfqpoint{5.206279in}{2.137313in}}%
\pgfpathlineto{\pgfqpoint{5.153322in}{2.136634in}}%
\pgfusepath{stroke}%
\end{pgfscope}%
\begin{pgfscope}%
\pgfpathrectangle{\pgfqpoint{3.352233in}{1.400000in}}{\pgfqpoint{2.407767in}{1.544118in}}%
\pgfusepath{clip}%
\pgfsetbuttcap%
\pgfsetroundjoin%
\pgfsetlinewidth{0.501875pt}%
\definecolor{currentstroke}{rgb}{0.272594,0.025563,0.353093}%
\pgfsetstrokecolor{currentstroke}%
\pgfsetdash{}{0pt}%
\pgfpathmoveto{\pgfqpoint{5.153322in}{2.136634in}}%
\pgfpathlineto{\pgfqpoint{5.100350in}{2.136681in}}%
\pgfusepath{stroke}%
\end{pgfscope}%
\begin{pgfscope}%
\pgfpathrectangle{\pgfqpoint{3.352233in}{1.400000in}}{\pgfqpoint{2.407767in}{1.544118in}}%
\pgfusepath{clip}%
\pgfsetbuttcap%
\pgfsetroundjoin%
\pgfsetlinewidth{0.501875pt}%
\definecolor{currentstroke}{rgb}{0.276022,0.044167,0.370164}%
\pgfsetstrokecolor{currentstroke}%
\pgfsetdash{}{0pt}%
\pgfpathmoveto{\pgfqpoint{5.100350in}{2.136681in}}%
\pgfpathlineto{\pgfqpoint{5.047376in}{2.136783in}}%
\pgfusepath{stroke}%
\end{pgfscope}%
\begin{pgfscope}%
\pgfpathrectangle{\pgfqpoint{3.352233in}{1.400000in}}{\pgfqpoint{2.407767in}{1.544118in}}%
\pgfusepath{clip}%
\pgfsetbuttcap%
\pgfsetroundjoin%
\pgfsetlinewidth{0.501875pt}%
\definecolor{currentstroke}{rgb}{0.278791,0.062145,0.386592}%
\pgfsetstrokecolor{currentstroke}%
\pgfsetdash{}{0pt}%
\pgfpathmoveto{\pgfqpoint{5.047376in}{2.136783in}}%
\pgfpathlineto{\pgfqpoint{4.994402in}{2.136706in}}%
\pgfusepath{stroke}%
\end{pgfscope}%
\begin{pgfscope}%
\pgfpathrectangle{\pgfqpoint{3.352233in}{1.400000in}}{\pgfqpoint{2.407767in}{1.544118in}}%
\pgfusepath{clip}%
\pgfsetbuttcap%
\pgfsetroundjoin%
\pgfsetlinewidth{0.501875pt}%
\definecolor{currentstroke}{rgb}{0.281446,0.084320,0.407414}%
\pgfsetstrokecolor{currentstroke}%
\pgfsetdash{}{0pt}%
\pgfpathmoveto{\pgfqpoint{4.994402in}{2.136706in}}%
\pgfpathlineto{\pgfqpoint{4.941426in}{2.136755in}}%
\pgfusepath{stroke}%
\end{pgfscope}%
\begin{pgfscope}%
\pgfpathrectangle{\pgfqpoint{3.352233in}{1.400000in}}{\pgfqpoint{2.407767in}{1.544118in}}%
\pgfusepath{clip}%
\pgfsetbuttcap%
\pgfsetroundjoin%
\pgfsetlinewidth{0.501875pt}%
\definecolor{currentstroke}{rgb}{0.283091,0.110553,0.431554}%
\pgfsetstrokecolor{currentstroke}%
\pgfsetdash{}{0pt}%
\pgfpathmoveto{\pgfqpoint{4.941426in}{2.136755in}}%
\pgfpathlineto{\pgfqpoint{4.888452in}{2.136845in}}%
\pgfusepath{stroke}%
\end{pgfscope}%
\begin{pgfscope}%
\pgfpathrectangle{\pgfqpoint{3.352233in}{1.400000in}}{\pgfqpoint{2.407767in}{1.544118in}}%
\pgfusepath{clip}%
\pgfsetbuttcap%
\pgfsetroundjoin%
\pgfsetlinewidth{0.501875pt}%
\definecolor{currentstroke}{rgb}{0.283197,0.115680,0.436115}%
\pgfsetstrokecolor{currentstroke}%
\pgfsetdash{}{0pt}%
\pgfpathmoveto{\pgfqpoint{4.888452in}{2.136845in}}%
\pgfpathlineto{\pgfqpoint{4.835478in}{2.136922in}}%
\pgfusepath{stroke}%
\end{pgfscope}%
\begin{pgfscope}%
\pgfpathrectangle{\pgfqpoint{3.352233in}{1.400000in}}{\pgfqpoint{2.407767in}{1.544118in}}%
\pgfusepath{clip}%
\pgfsetbuttcap%
\pgfsetroundjoin%
\pgfsetlinewidth{0.501875pt}%
\definecolor{currentstroke}{rgb}{0.281412,0.155834,0.469201}%
\pgfsetstrokecolor{currentstroke}%
\pgfsetdash{}{0pt}%
\pgfpathmoveto{\pgfqpoint{4.835478in}{2.136922in}}%
\pgfpathlineto{\pgfqpoint{4.782503in}{2.136894in}}%
\pgfusepath{stroke}%
\end{pgfscope}%
\begin{pgfscope}%
\pgfpathrectangle{\pgfqpoint{3.352233in}{1.400000in}}{\pgfqpoint{2.407767in}{1.544118in}}%
\pgfusepath{clip}%
\pgfsetbuttcap%
\pgfsetroundjoin%
\pgfsetlinewidth{0.501875pt}%
\definecolor{currentstroke}{rgb}{0.280868,0.160771,0.472899}%
\pgfsetstrokecolor{currentstroke}%
\pgfsetdash{}{0pt}%
\pgfpathmoveto{\pgfqpoint{4.782503in}{2.136894in}}%
\pgfpathlineto{\pgfqpoint{4.729527in}{2.136847in}}%
\pgfusepath{stroke}%
\end{pgfscope}%
\begin{pgfscope}%
\pgfpathrectangle{\pgfqpoint{3.352233in}{1.400000in}}{\pgfqpoint{2.407767in}{1.544118in}}%
\pgfusepath{clip}%
\pgfsetbuttcap%
\pgfsetroundjoin%
\pgfsetlinewidth{0.501875pt}%
\definecolor{currentstroke}{rgb}{0.277134,0.185228,0.489898}%
\pgfsetstrokecolor{currentstroke}%
\pgfsetdash{}{0pt}%
\pgfpathmoveto{\pgfqpoint{4.729527in}{2.136847in}}%
\pgfpathlineto{\pgfqpoint{4.676552in}{2.136796in}}%
\pgfusepath{stroke}%
\end{pgfscope}%
\begin{pgfscope}%
\pgfpathrectangle{\pgfqpoint{3.352233in}{1.400000in}}{\pgfqpoint{2.407767in}{1.544118in}}%
\pgfusepath{clip}%
\pgfsetbuttcap%
\pgfsetroundjoin%
\pgfsetlinewidth{0.501875pt}%
\definecolor{currentstroke}{rgb}{0.278826,0.175490,0.483397}%
\pgfsetstrokecolor{currentstroke}%
\pgfsetdash{}{0pt}%
\pgfpathmoveto{\pgfqpoint{4.676552in}{2.136796in}}%
\pgfpathlineto{\pgfqpoint{4.623577in}{2.136772in}}%
\pgfusepath{stroke}%
\end{pgfscope}%
\begin{pgfscope}%
\pgfpathrectangle{\pgfqpoint{3.352233in}{1.400000in}}{\pgfqpoint{2.407767in}{1.544118in}}%
\pgfusepath{clip}%
\pgfsetbuttcap%
\pgfsetroundjoin%
\pgfsetlinewidth{0.501875pt}%
\definecolor{currentstroke}{rgb}{0.278826,0.175490,0.483397}%
\pgfsetstrokecolor{currentstroke}%
\pgfsetdash{}{0pt}%
\pgfpathmoveto{\pgfqpoint{4.623577in}{2.136772in}}%
\pgfpathlineto{\pgfqpoint{4.570603in}{2.136678in}}%
\pgfusepath{stroke}%
\end{pgfscope}%
\begin{pgfscope}%
\pgfpathrectangle{\pgfqpoint{3.352233in}{1.400000in}}{\pgfqpoint{2.407767in}{1.544118in}}%
\pgfusepath{clip}%
\pgfsetbuttcap%
\pgfsetroundjoin%
\pgfsetlinewidth{0.501875pt}%
\definecolor{currentstroke}{rgb}{0.280868,0.160771,0.472899}%
\pgfsetstrokecolor{currentstroke}%
\pgfsetdash{}{0pt}%
\pgfpathmoveto{\pgfqpoint{4.570603in}{2.136678in}}%
\pgfpathlineto{\pgfqpoint{4.517632in}{2.136387in}}%
\pgfusepath{stroke}%
\end{pgfscope}%
\begin{pgfscope}%
\pgfpathrectangle{\pgfqpoint{3.352233in}{1.400000in}}{\pgfqpoint{2.407767in}{1.544118in}}%
\pgfusepath{clip}%
\pgfsetbuttcap%
\pgfsetroundjoin%
\pgfsetlinewidth{0.501875pt}%
\definecolor{currentstroke}{rgb}{0.281887,0.150881,0.465405}%
\pgfsetstrokecolor{currentstroke}%
\pgfsetdash{}{0pt}%
\pgfpathmoveto{\pgfqpoint{4.517632in}{2.136387in}}%
\pgfpathlineto{\pgfqpoint{4.464661in}{2.136404in}}%
\pgfusepath{stroke}%
\end{pgfscope}%
\begin{pgfscope}%
\pgfpathrectangle{\pgfqpoint{3.352233in}{1.400000in}}{\pgfqpoint{2.407767in}{1.544118in}}%
\pgfusepath{clip}%
\pgfsetbuttcap%
\pgfsetroundjoin%
\pgfsetlinewidth{0.501875pt}%
\definecolor{currentstroke}{rgb}{0.282884,0.135920,0.453427}%
\pgfsetstrokecolor{currentstroke}%
\pgfsetdash{}{0pt}%
\pgfpathmoveto{\pgfqpoint{4.464661in}{2.136404in}}%
\pgfpathlineto{\pgfqpoint{4.411695in}{2.136636in}}%
\pgfusepath{stroke}%
\end{pgfscope}%
\begin{pgfscope}%
\pgfpathrectangle{\pgfqpoint{3.352233in}{1.400000in}}{\pgfqpoint{2.407767in}{1.544118in}}%
\pgfusepath{clip}%
\pgfsetbuttcap%
\pgfsetroundjoin%
\pgfsetlinewidth{0.501875pt}%
\definecolor{currentstroke}{rgb}{0.282910,0.105393,0.426902}%
\pgfsetstrokecolor{currentstroke}%
\pgfsetdash{}{0pt}%
\pgfpathmoveto{\pgfqpoint{4.411695in}{2.136636in}}%
\pgfpathlineto{\pgfqpoint{4.358744in}{2.136221in}}%
\pgfusepath{stroke}%
\end{pgfscope}%
\begin{pgfscope}%
\pgfpathrectangle{\pgfqpoint{3.352233in}{1.400000in}}{\pgfqpoint{2.407767in}{1.544118in}}%
\pgfusepath{clip}%
\pgfsetbuttcap%
\pgfsetroundjoin%
\pgfsetlinewidth{0.501875pt}%
\definecolor{currentstroke}{rgb}{0.280894,0.078907,0.402329}%
\pgfsetstrokecolor{currentstroke}%
\pgfsetdash{}{0pt}%
\pgfpathmoveto{\pgfqpoint{4.358744in}{2.136221in}}%
\pgfpathlineto{\pgfqpoint{4.305813in}{2.135142in}}%
\pgfusepath{stroke}%
\end{pgfscope}%
\begin{pgfscope}%
\pgfpathrectangle{\pgfqpoint{3.352233in}{1.400000in}}{\pgfqpoint{2.407767in}{1.544118in}}%
\pgfusepath{clip}%
\pgfsetbuttcap%
\pgfsetroundjoin%
\pgfsetlinewidth{0.501875pt}%
\definecolor{currentstroke}{rgb}{0.269944,0.014625,0.341379}%
\pgfsetstrokecolor{currentstroke}%
\pgfsetdash{}{0pt}%
\pgfpathmoveto{\pgfqpoint{5.206279in}{2.172059in}}%
\pgfpathlineto{\pgfqpoint{5.153314in}{2.172498in}}%
\pgfusepath{stroke}%
\end{pgfscope}%
\begin{pgfscope}%
\pgfpathrectangle{\pgfqpoint{3.352233in}{1.400000in}}{\pgfqpoint{2.407767in}{1.544118in}}%
\pgfusepath{clip}%
\pgfsetbuttcap%
\pgfsetroundjoin%
\pgfsetlinewidth{0.501875pt}%
\definecolor{currentstroke}{rgb}{0.272594,0.025563,0.353093}%
\pgfsetstrokecolor{currentstroke}%
\pgfsetdash{}{0pt}%
\pgfpathmoveto{\pgfqpoint{5.153314in}{2.172498in}}%
\pgfpathlineto{\pgfqpoint{5.100347in}{2.172784in}}%
\pgfusepath{stroke}%
\end{pgfscope}%
\begin{pgfscope}%
\pgfpathrectangle{\pgfqpoint{3.352233in}{1.400000in}}{\pgfqpoint{2.407767in}{1.544118in}}%
\pgfusepath{clip}%
\pgfsetbuttcap%
\pgfsetroundjoin%
\pgfsetlinewidth{0.501875pt}%
\definecolor{currentstroke}{rgb}{0.277018,0.050344,0.375715}%
\pgfsetstrokecolor{currentstroke}%
\pgfsetdash{}{0pt}%
\pgfpathmoveto{\pgfqpoint{5.100347in}{2.172784in}}%
\pgfpathlineto{\pgfqpoint{5.047372in}{2.172656in}}%
\pgfusepath{stroke}%
\end{pgfscope}%
\begin{pgfscope}%
\pgfpathrectangle{\pgfqpoint{3.352233in}{1.400000in}}{\pgfqpoint{2.407767in}{1.544118in}}%
\pgfusepath{clip}%
\pgfsetbuttcap%
\pgfsetroundjoin%
\pgfsetlinewidth{0.501875pt}%
\definecolor{currentstroke}{rgb}{0.277941,0.056324,0.381191}%
\pgfsetstrokecolor{currentstroke}%
\pgfsetdash{}{0pt}%
\pgfpathmoveto{\pgfqpoint{5.047372in}{2.172656in}}%
\pgfpathlineto{\pgfqpoint{4.994396in}{2.172682in}}%
\pgfusepath{stroke}%
\end{pgfscope}%
\begin{pgfscope}%
\pgfpathrectangle{\pgfqpoint{3.352233in}{1.400000in}}{\pgfqpoint{2.407767in}{1.544118in}}%
\pgfusepath{clip}%
\pgfsetbuttcap%
\pgfsetroundjoin%
\pgfsetlinewidth{0.501875pt}%
\definecolor{currentstroke}{rgb}{0.281446,0.084320,0.407414}%
\pgfsetstrokecolor{currentstroke}%
\pgfsetdash{}{0pt}%
\pgfpathmoveto{\pgfqpoint{4.994396in}{2.172682in}}%
\pgfpathlineto{\pgfqpoint{4.941420in}{2.172751in}}%
\pgfusepath{stroke}%
\end{pgfscope}%
\begin{pgfscope}%
\pgfpathrectangle{\pgfqpoint{3.352233in}{1.400000in}}{\pgfqpoint{2.407767in}{1.544118in}}%
\pgfusepath{clip}%
\pgfsetbuttcap%
\pgfsetroundjoin%
\pgfsetlinewidth{0.501875pt}%
\definecolor{currentstroke}{rgb}{0.283197,0.115680,0.436115}%
\pgfsetstrokecolor{currentstroke}%
\pgfsetdash{}{0pt}%
\pgfpathmoveto{\pgfqpoint{4.941420in}{2.172751in}}%
\pgfpathlineto{\pgfqpoint{4.888446in}{2.172955in}}%
\pgfusepath{stroke}%
\end{pgfscope}%
\begin{pgfscope}%
\pgfpathrectangle{\pgfqpoint{3.352233in}{1.400000in}}{\pgfqpoint{2.407767in}{1.544118in}}%
\pgfusepath{clip}%
\pgfsetbuttcap%
\pgfsetroundjoin%
\pgfsetlinewidth{0.501875pt}%
\definecolor{currentstroke}{rgb}{0.283072,0.130895,0.449241}%
\pgfsetstrokecolor{currentstroke}%
\pgfsetdash{}{0pt}%
\pgfpathmoveto{\pgfqpoint{4.888446in}{2.172955in}}%
\pgfpathlineto{\pgfqpoint{4.835474in}{2.172882in}}%
\pgfusepath{stroke}%
\end{pgfscope}%
\begin{pgfscope}%
\pgfpathrectangle{\pgfqpoint{3.352233in}{1.400000in}}{\pgfqpoint{2.407767in}{1.544118in}}%
\pgfusepath{clip}%
\pgfsetbuttcap%
\pgfsetroundjoin%
\pgfsetlinewidth{0.501875pt}%
\definecolor{currentstroke}{rgb}{0.282290,0.145912,0.461510}%
\pgfsetstrokecolor{currentstroke}%
\pgfsetdash{}{0pt}%
\pgfpathmoveto{\pgfqpoint{4.835474in}{2.172882in}}%
\pgfpathlineto{\pgfqpoint{4.782501in}{2.172643in}}%
\pgfusepath{stroke}%
\end{pgfscope}%
\begin{pgfscope}%
\pgfpathrectangle{\pgfqpoint{3.352233in}{1.400000in}}{\pgfqpoint{2.407767in}{1.544118in}}%
\pgfusepath{clip}%
\pgfsetbuttcap%
\pgfsetroundjoin%
\pgfsetlinewidth{0.501875pt}%
\definecolor{currentstroke}{rgb}{0.281412,0.155834,0.469201}%
\pgfsetstrokecolor{currentstroke}%
\pgfsetdash{}{0pt}%
\pgfpathmoveto{\pgfqpoint{4.782501in}{2.172643in}}%
\pgfpathlineto{\pgfqpoint{4.729526in}{2.172504in}}%
\pgfusepath{stroke}%
\end{pgfscope}%
\begin{pgfscope}%
\pgfpathrectangle{\pgfqpoint{3.352233in}{1.400000in}}{\pgfqpoint{2.407767in}{1.544118in}}%
\pgfusepath{clip}%
\pgfsetbuttcap%
\pgfsetroundjoin%
\pgfsetlinewidth{0.501875pt}%
\definecolor{currentstroke}{rgb}{0.280255,0.165693,0.476498}%
\pgfsetstrokecolor{currentstroke}%
\pgfsetdash{}{0pt}%
\pgfpathmoveto{\pgfqpoint{4.729526in}{2.172504in}}%
\pgfpathlineto{\pgfqpoint{4.676551in}{2.172465in}}%
\pgfusepath{stroke}%
\end{pgfscope}%
\begin{pgfscope}%
\pgfpathrectangle{\pgfqpoint{3.352233in}{1.400000in}}{\pgfqpoint{2.407767in}{1.544118in}}%
\pgfusepath{clip}%
\pgfsetbuttcap%
\pgfsetroundjoin%
\pgfsetlinewidth{0.501875pt}%
\definecolor{currentstroke}{rgb}{0.276194,0.190074,0.493001}%
\pgfsetstrokecolor{currentstroke}%
\pgfsetdash{}{0pt}%
\pgfpathmoveto{\pgfqpoint{4.676551in}{2.172465in}}%
\pgfpathlineto{\pgfqpoint{4.623578in}{2.172352in}}%
\pgfusepath{stroke}%
\end{pgfscope}%
\begin{pgfscope}%
\pgfpathrectangle{\pgfqpoint{3.352233in}{1.400000in}}{\pgfqpoint{2.407767in}{1.544118in}}%
\pgfusepath{clip}%
\pgfsetbuttcap%
\pgfsetroundjoin%
\pgfsetlinewidth{0.501875pt}%
\definecolor{currentstroke}{rgb}{0.280255,0.165693,0.476498}%
\pgfsetstrokecolor{currentstroke}%
\pgfsetdash{}{0pt}%
\pgfpathmoveto{\pgfqpoint{4.623578in}{2.172352in}}%
\pgfpathlineto{\pgfqpoint{4.570615in}{2.171732in}}%
\pgfusepath{stroke}%
\end{pgfscope}%
\begin{pgfscope}%
\pgfpathrectangle{\pgfqpoint{3.352233in}{1.400000in}}{\pgfqpoint{2.407767in}{1.544118in}}%
\pgfusepath{clip}%
\pgfsetbuttcap%
\pgfsetroundjoin%
\pgfsetlinewidth{0.501875pt}%
\definecolor{currentstroke}{rgb}{0.280868,0.160771,0.472899}%
\pgfsetstrokecolor{currentstroke}%
\pgfsetdash{}{0pt}%
\pgfpathmoveto{\pgfqpoint{4.570615in}{2.171732in}}%
\pgfpathlineto{\pgfqpoint{4.517653in}{2.171224in}}%
\pgfusepath{stroke}%
\end{pgfscope}%
\begin{pgfscope}%
\pgfpathrectangle{\pgfqpoint{3.352233in}{1.400000in}}{\pgfqpoint{2.407767in}{1.544118in}}%
\pgfusepath{clip}%
\pgfsetbuttcap%
\pgfsetroundjoin%
\pgfsetlinewidth{0.501875pt}%
\definecolor{currentstroke}{rgb}{0.281412,0.155834,0.469201}%
\pgfsetstrokecolor{currentstroke}%
\pgfsetdash{}{0pt}%
\pgfpathmoveto{\pgfqpoint{4.517653in}{2.171224in}}%
\pgfpathlineto{\pgfqpoint{4.464684in}{2.170877in}}%
\pgfusepath{stroke}%
\end{pgfscope}%
\begin{pgfscope}%
\pgfpathrectangle{\pgfqpoint{3.352233in}{1.400000in}}{\pgfqpoint{2.407767in}{1.544118in}}%
\pgfusepath{clip}%
\pgfsetbuttcap%
\pgfsetroundjoin%
\pgfsetlinewidth{0.501875pt}%
\definecolor{currentstroke}{rgb}{0.282656,0.100196,0.422160}%
\pgfsetstrokecolor{currentstroke}%
\pgfsetdash{}{0pt}%
\pgfpathmoveto{\pgfqpoint{4.464684in}{2.170877in}}%
\pgfpathlineto{\pgfqpoint{4.411716in}{2.170747in}}%
\pgfusepath{stroke}%
\end{pgfscope}%
\begin{pgfscope}%
\pgfpathrectangle{\pgfqpoint{3.352233in}{1.400000in}}{\pgfqpoint{2.407767in}{1.544118in}}%
\pgfusepath{clip}%
\pgfsetbuttcap%
\pgfsetroundjoin%
\pgfsetlinewidth{0.501875pt}%
\definecolor{currentstroke}{rgb}{0.282327,0.094955,0.417331}%
\pgfsetstrokecolor{currentstroke}%
\pgfsetdash{}{0pt}%
\pgfpathmoveto{\pgfqpoint{4.411716in}{2.170747in}}%
\pgfpathlineto{\pgfqpoint{4.358756in}{2.170396in}}%
\pgfusepath{stroke}%
\end{pgfscope}%
\begin{pgfscope}%
\pgfpathrectangle{\pgfqpoint{3.352233in}{1.400000in}}{\pgfqpoint{2.407767in}{1.544118in}}%
\pgfusepath{clip}%
\pgfsetbuttcap%
\pgfsetroundjoin%
\pgfsetlinewidth{0.501875pt}%
\definecolor{currentstroke}{rgb}{0.281446,0.084320,0.407414}%
\pgfsetstrokecolor{currentstroke}%
\pgfsetdash{}{0pt}%
\pgfpathmoveto{\pgfqpoint{4.358756in}{2.170396in}}%
\pgfpathlineto{\pgfqpoint{4.305799in}{2.169653in}}%
\pgfusepath{stroke}%
\end{pgfscope}%
\begin{pgfscope}%
\pgfpathrectangle{\pgfqpoint{3.352233in}{1.400000in}}{\pgfqpoint{2.407767in}{1.544118in}}%
\pgfusepath{clip}%
\pgfsetbuttcap%
\pgfsetroundjoin%
\pgfsetlinewidth{0.501875pt}%
\definecolor{currentstroke}{rgb}{0.268510,0.009605,0.335427}%
\pgfsetstrokecolor{currentstroke}%
\pgfsetdash{}{0pt}%
\pgfpathmoveto{\pgfqpoint{5.206279in}{2.206805in}}%
\pgfpathlineto{\pgfqpoint{5.153314in}{2.206729in}}%
\pgfusepath{stroke}%
\end{pgfscope}%
\begin{pgfscope}%
\pgfpathrectangle{\pgfqpoint{3.352233in}{1.400000in}}{\pgfqpoint{2.407767in}{1.544118in}}%
\pgfusepath{clip}%
\pgfsetbuttcap%
\pgfsetroundjoin%
\pgfsetlinewidth{0.501875pt}%
\definecolor{currentstroke}{rgb}{0.272594,0.025563,0.353093}%
\pgfsetstrokecolor{currentstroke}%
\pgfsetdash{}{0pt}%
\pgfpathmoveto{\pgfqpoint{5.153314in}{2.206729in}}%
\pgfpathlineto{\pgfqpoint{5.100359in}{2.206509in}}%
\pgfusepath{stroke}%
\end{pgfscope}%
\begin{pgfscope}%
\pgfpathrectangle{\pgfqpoint{3.352233in}{1.400000in}}{\pgfqpoint{2.407767in}{1.544118in}}%
\pgfusepath{clip}%
\pgfsetbuttcap%
\pgfsetroundjoin%
\pgfsetlinewidth{0.501875pt}%
\definecolor{currentstroke}{rgb}{0.276022,0.044167,0.370164}%
\pgfsetstrokecolor{currentstroke}%
\pgfsetdash{}{0pt}%
\pgfpathmoveto{\pgfqpoint{5.100359in}{2.206509in}}%
\pgfpathlineto{\pgfqpoint{5.047400in}{2.206024in}}%
\pgfusepath{stroke}%
\end{pgfscope}%
\begin{pgfscope}%
\pgfpathrectangle{\pgfqpoint{3.352233in}{1.400000in}}{\pgfqpoint{2.407767in}{1.544118in}}%
\pgfusepath{clip}%
\pgfsetbuttcap%
\pgfsetroundjoin%
\pgfsetlinewidth{0.501875pt}%
\definecolor{currentstroke}{rgb}{0.278791,0.062145,0.386592}%
\pgfsetstrokecolor{currentstroke}%
\pgfsetdash{}{0pt}%
\pgfpathmoveto{\pgfqpoint{5.047400in}{2.206024in}}%
\pgfpathlineto{\pgfqpoint{4.994425in}{2.205968in}}%
\pgfusepath{stroke}%
\end{pgfscope}%
\begin{pgfscope}%
\pgfpathrectangle{\pgfqpoint{3.352233in}{1.400000in}}{\pgfqpoint{2.407767in}{1.544118in}}%
\pgfusepath{clip}%
\pgfsetbuttcap%
\pgfsetroundjoin%
\pgfsetlinewidth{0.501875pt}%
\definecolor{currentstroke}{rgb}{0.281446,0.084320,0.407414}%
\pgfsetstrokecolor{currentstroke}%
\pgfsetdash{}{0pt}%
\pgfpathmoveto{\pgfqpoint{4.994425in}{2.205968in}}%
\pgfpathlineto{\pgfqpoint{4.941451in}{2.205771in}}%
\pgfusepath{stroke}%
\end{pgfscope}%
\begin{pgfscope}%
\pgfpathrectangle{\pgfqpoint{3.352233in}{1.400000in}}{\pgfqpoint{2.407767in}{1.544118in}}%
\pgfusepath{clip}%
\pgfsetbuttcap%
\pgfsetroundjoin%
\pgfsetlinewidth{0.501875pt}%
\definecolor{currentstroke}{rgb}{0.283091,0.110553,0.431554}%
\pgfsetstrokecolor{currentstroke}%
\pgfsetdash{}{0pt}%
\pgfpathmoveto{\pgfqpoint{4.941451in}{2.205771in}}%
\pgfpathlineto{\pgfqpoint{4.888479in}{2.205396in}}%
\pgfusepath{stroke}%
\end{pgfscope}%
\begin{pgfscope}%
\pgfpathrectangle{\pgfqpoint{3.352233in}{1.400000in}}{\pgfqpoint{2.407767in}{1.544118in}}%
\pgfusepath{clip}%
\pgfsetbuttcap%
\pgfsetroundjoin%
\pgfsetlinewidth{0.501875pt}%
\definecolor{currentstroke}{rgb}{0.283229,0.120777,0.440584}%
\pgfsetstrokecolor{currentstroke}%
\pgfsetdash{}{0pt}%
\pgfpathmoveto{\pgfqpoint{4.888479in}{2.205396in}}%
\pgfpathlineto{\pgfqpoint{4.835506in}{2.205068in}}%
\pgfusepath{stroke}%
\end{pgfscope}%
\begin{pgfscope}%
\pgfpathrectangle{\pgfqpoint{3.352233in}{1.400000in}}{\pgfqpoint{2.407767in}{1.544118in}}%
\pgfusepath{clip}%
\pgfsetbuttcap%
\pgfsetroundjoin%
\pgfsetlinewidth{0.501875pt}%
\definecolor{currentstroke}{rgb}{0.281412,0.155834,0.469201}%
\pgfsetstrokecolor{currentstroke}%
\pgfsetdash{}{0pt}%
\pgfpathmoveto{\pgfqpoint{4.835506in}{2.205068in}}%
\pgfpathlineto{\pgfqpoint{4.782531in}{2.204930in}}%
\pgfusepath{stroke}%
\end{pgfscope}%
\begin{pgfscope}%
\pgfpathrectangle{\pgfqpoint{3.352233in}{1.400000in}}{\pgfqpoint{2.407767in}{1.544118in}}%
\pgfusepath{clip}%
\pgfsetbuttcap%
\pgfsetroundjoin%
\pgfsetlinewidth{0.501875pt}%
\definecolor{currentstroke}{rgb}{0.278826,0.175490,0.483397}%
\pgfsetstrokecolor{currentstroke}%
\pgfsetdash{}{0pt}%
\pgfpathmoveto{\pgfqpoint{4.782531in}{2.204930in}}%
\pgfpathlineto{\pgfqpoint{4.729557in}{2.204686in}}%
\pgfusepath{stroke}%
\end{pgfscope}%
\begin{pgfscope}%
\pgfpathrectangle{\pgfqpoint{3.352233in}{1.400000in}}{\pgfqpoint{2.407767in}{1.544118in}}%
\pgfusepath{clip}%
\pgfsetbuttcap%
\pgfsetroundjoin%
\pgfsetlinewidth{0.501875pt}%
\definecolor{currentstroke}{rgb}{0.278012,0.180367,0.486697}%
\pgfsetstrokecolor{currentstroke}%
\pgfsetdash{}{0pt}%
\pgfpathmoveto{\pgfqpoint{4.729557in}{2.204686in}}%
\pgfpathlineto{\pgfqpoint{4.676591in}{2.204083in}}%
\pgfusepath{stroke}%
\end{pgfscope}%
\begin{pgfscope}%
\pgfpathrectangle{\pgfqpoint{3.352233in}{1.400000in}}{\pgfqpoint{2.407767in}{1.544118in}}%
\pgfusepath{clip}%
\pgfsetbuttcap%
\pgfsetroundjoin%
\pgfsetlinewidth{0.501875pt}%
\definecolor{currentstroke}{rgb}{0.280868,0.160771,0.472899}%
\pgfsetstrokecolor{currentstroke}%
\pgfsetdash{}{0pt}%
\pgfpathmoveto{\pgfqpoint{4.676591in}{2.204083in}}%
\pgfpathlineto{\pgfqpoint{4.623630in}{2.203268in}}%
\pgfusepath{stroke}%
\end{pgfscope}%
\begin{pgfscope}%
\pgfpathrectangle{\pgfqpoint{3.352233in}{1.400000in}}{\pgfqpoint{2.407767in}{1.544118in}}%
\pgfusepath{clip}%
\pgfsetbuttcap%
\pgfsetroundjoin%
\pgfsetlinewidth{0.501875pt}%
\definecolor{currentstroke}{rgb}{0.279574,0.170599,0.479997}%
\pgfsetstrokecolor{currentstroke}%
\pgfsetdash{}{0pt}%
\pgfpathmoveto{\pgfqpoint{4.623630in}{2.203268in}}%
\pgfpathlineto{\pgfqpoint{4.570669in}{2.202482in}}%
\pgfusepath{stroke}%
\end{pgfscope}%
\begin{pgfscope}%
\pgfpathrectangle{\pgfqpoint{3.352233in}{1.400000in}}{\pgfqpoint{2.407767in}{1.544118in}}%
\pgfusepath{clip}%
\pgfsetbuttcap%
\pgfsetroundjoin%
\pgfsetlinewidth{0.501875pt}%
\definecolor{currentstroke}{rgb}{0.280255,0.165693,0.476498}%
\pgfsetstrokecolor{currentstroke}%
\pgfsetdash{}{0pt}%
\pgfpathmoveto{\pgfqpoint{4.570669in}{2.202482in}}%
\pgfpathlineto{\pgfqpoint{4.517735in}{2.201270in}}%
\pgfusepath{stroke}%
\end{pgfscope}%
\begin{pgfscope}%
\pgfpathrectangle{\pgfqpoint{3.352233in}{1.400000in}}{\pgfqpoint{2.407767in}{1.544118in}}%
\pgfusepath{clip}%
\pgfsetbuttcap%
\pgfsetroundjoin%
\pgfsetlinewidth{0.501875pt}%
\definecolor{currentstroke}{rgb}{0.282884,0.135920,0.453427}%
\pgfsetstrokecolor{currentstroke}%
\pgfsetdash{}{0pt}%
\pgfpathmoveto{\pgfqpoint{4.517735in}{2.201270in}}%
\pgfpathlineto{\pgfqpoint{4.464865in}{2.199223in}}%
\pgfusepath{stroke}%
\end{pgfscope}%
\begin{pgfscope}%
\pgfpathrectangle{\pgfqpoint{3.352233in}{1.400000in}}{\pgfqpoint{2.407767in}{1.544118in}}%
\pgfusepath{clip}%
\pgfsetbuttcap%
\pgfsetroundjoin%
\pgfsetlinewidth{0.501875pt}%
\definecolor{currentstroke}{rgb}{0.283187,0.125848,0.444960}%
\pgfsetstrokecolor{currentstroke}%
\pgfsetdash{}{0pt}%
\pgfpathmoveto{\pgfqpoint{4.464865in}{2.199223in}}%
\pgfpathlineto{\pgfqpoint{4.412001in}{2.197069in}}%
\pgfusepath{stroke}%
\end{pgfscope}%
\begin{pgfscope}%
\pgfpathrectangle{\pgfqpoint{3.352233in}{1.400000in}}{\pgfqpoint{2.407767in}{1.544118in}}%
\pgfusepath{clip}%
\pgfsetbuttcap%
\pgfsetroundjoin%
\pgfsetlinewidth{0.501875pt}%
\definecolor{currentstroke}{rgb}{0.282910,0.105393,0.426902}%
\pgfsetstrokecolor{currentstroke}%
\pgfsetdash{}{0pt}%
\pgfpathmoveto{\pgfqpoint{4.412001in}{2.197069in}}%
\pgfpathlineto{\pgfqpoint{4.359137in}{2.194900in}}%
\pgfusepath{stroke}%
\end{pgfscope}%
\begin{pgfscope}%
\pgfpathrectangle{\pgfqpoint{3.352233in}{1.400000in}}{\pgfqpoint{2.407767in}{1.544118in}}%
\pgfusepath{clip}%
\pgfsetbuttcap%
\pgfsetroundjoin%
\pgfsetlinewidth{0.501875pt}%
\definecolor{currentstroke}{rgb}{0.281446,0.084320,0.407414}%
\pgfsetstrokecolor{currentstroke}%
\pgfsetdash{}{0pt}%
\pgfpathmoveto{\pgfqpoint{4.359137in}{2.194900in}}%
\pgfpathlineto{\pgfqpoint{4.306524in}{2.191466in}}%
\pgfusepath{stroke}%
\end{pgfscope}%
\begin{pgfscope}%
\pgfpathrectangle{\pgfqpoint{3.352233in}{1.400000in}}{\pgfqpoint{2.407767in}{1.544118in}}%
\pgfusepath{clip}%
\pgfsetbuttcap%
\pgfsetroundjoin%
\pgfsetlinewidth{0.501875pt}%
\definecolor{currentstroke}{rgb}{0.277941,0.056324,0.381191}%
\pgfsetstrokecolor{currentstroke}%
\pgfsetdash{}{0pt}%
\pgfpathmoveto{\pgfqpoint{4.306524in}{2.191466in}}%
\pgfpathlineto{\pgfqpoint{4.306524in}{2.191466in}}%
\pgfusepath{stroke}%
\end{pgfscope}%
\begin{pgfscope}%
\pgfpathrectangle{\pgfqpoint{3.352233in}{1.400000in}}{\pgfqpoint{2.407767in}{1.544118in}}%
\pgfusepath{clip}%
\pgfsetbuttcap%
\pgfsetroundjoin%
\pgfsetlinewidth{0.501875pt}%
\definecolor{currentstroke}{rgb}{0.268510,0.009605,0.335427}%
\pgfsetstrokecolor{currentstroke}%
\pgfsetdash{}{0pt}%
\pgfpathmoveto{\pgfqpoint{5.206279in}{2.241551in}}%
\pgfpathlineto{\pgfqpoint{5.153318in}{2.241862in}}%
\pgfusepath{stroke}%
\end{pgfscope}%
\begin{pgfscope}%
\pgfpathrectangle{\pgfqpoint{3.352233in}{1.400000in}}{\pgfqpoint{2.407767in}{1.544118in}}%
\pgfusepath{clip}%
\pgfsetbuttcap%
\pgfsetroundjoin%
\pgfsetlinewidth{0.501875pt}%
\definecolor{currentstroke}{rgb}{0.272594,0.025563,0.353093}%
\pgfsetstrokecolor{currentstroke}%
\pgfsetdash{}{0pt}%
\pgfpathmoveto{\pgfqpoint{5.153318in}{2.241862in}}%
\pgfpathlineto{\pgfqpoint{5.100349in}{2.242096in}}%
\pgfusepath{stroke}%
\end{pgfscope}%
\begin{pgfscope}%
\pgfpathrectangle{\pgfqpoint{3.352233in}{1.400000in}}{\pgfqpoint{2.407767in}{1.544118in}}%
\pgfusepath{clip}%
\pgfsetbuttcap%
\pgfsetroundjoin%
\pgfsetlinewidth{0.501875pt}%
\definecolor{currentstroke}{rgb}{0.276022,0.044167,0.370164}%
\pgfsetstrokecolor{currentstroke}%
\pgfsetdash{}{0pt}%
\pgfpathmoveto{\pgfqpoint{5.100349in}{2.242096in}}%
\pgfpathlineto{\pgfqpoint{5.047375in}{2.242322in}}%
\pgfusepath{stroke}%
\end{pgfscope}%
\begin{pgfscope}%
\pgfpathrectangle{\pgfqpoint{3.352233in}{1.400000in}}{\pgfqpoint{2.407767in}{1.544118in}}%
\pgfusepath{clip}%
\pgfsetbuttcap%
\pgfsetroundjoin%
\pgfsetlinewidth{0.501875pt}%
\definecolor{currentstroke}{rgb}{0.278791,0.062145,0.386592}%
\pgfsetstrokecolor{currentstroke}%
\pgfsetdash{}{0pt}%
\pgfpathmoveto{\pgfqpoint{5.047375in}{2.242322in}}%
\pgfpathlineto{\pgfqpoint{4.994400in}{2.242486in}}%
\pgfusepath{stroke}%
\end{pgfscope}%
\begin{pgfscope}%
\pgfpathrectangle{\pgfqpoint{3.352233in}{1.400000in}}{\pgfqpoint{2.407767in}{1.544118in}}%
\pgfusepath{clip}%
\pgfsetbuttcap%
\pgfsetroundjoin%
\pgfsetlinewidth{0.501875pt}%
\definecolor{currentstroke}{rgb}{0.280894,0.078907,0.402329}%
\pgfsetstrokecolor{currentstroke}%
\pgfsetdash{}{0pt}%
\pgfpathmoveto{\pgfqpoint{4.994400in}{2.242486in}}%
\pgfpathlineto{\pgfqpoint{4.941425in}{2.242489in}}%
\pgfusepath{stroke}%
\end{pgfscope}%
\begin{pgfscope}%
\pgfpathrectangle{\pgfqpoint{3.352233in}{1.400000in}}{\pgfqpoint{2.407767in}{1.544118in}}%
\pgfusepath{clip}%
\pgfsetbuttcap%
\pgfsetroundjoin%
\pgfsetlinewidth{0.501875pt}%
\definecolor{currentstroke}{rgb}{0.283197,0.115680,0.436115}%
\pgfsetstrokecolor{currentstroke}%
\pgfsetdash{}{0pt}%
\pgfpathmoveto{\pgfqpoint{4.941425in}{2.242489in}}%
\pgfpathlineto{\pgfqpoint{4.888450in}{2.242533in}}%
\pgfusepath{stroke}%
\end{pgfscope}%
\begin{pgfscope}%
\pgfpathrectangle{\pgfqpoint{3.352233in}{1.400000in}}{\pgfqpoint{2.407767in}{1.544118in}}%
\pgfusepath{clip}%
\pgfsetbuttcap%
\pgfsetroundjoin%
\pgfsetlinewidth{0.501875pt}%
\definecolor{currentstroke}{rgb}{0.283229,0.120777,0.440584}%
\pgfsetstrokecolor{currentstroke}%
\pgfsetdash{}{0pt}%
\pgfpathmoveto{\pgfqpoint{4.888450in}{2.242533in}}%
\pgfpathlineto{\pgfqpoint{4.835475in}{2.242550in}}%
\pgfusepath{stroke}%
\end{pgfscope}%
\begin{pgfscope}%
\pgfpathrectangle{\pgfqpoint{3.352233in}{1.400000in}}{\pgfqpoint{2.407767in}{1.544118in}}%
\pgfusepath{clip}%
\pgfsetbuttcap%
\pgfsetroundjoin%
\pgfsetlinewidth{0.501875pt}%
\definecolor{currentstroke}{rgb}{0.282623,0.140926,0.457517}%
\pgfsetstrokecolor{currentstroke}%
\pgfsetdash{}{0pt}%
\pgfpathmoveto{\pgfqpoint{4.835475in}{2.242550in}}%
\pgfpathlineto{\pgfqpoint{4.782504in}{2.242194in}}%
\pgfusepath{stroke}%
\end{pgfscope}%
\begin{pgfscope}%
\pgfpathrectangle{\pgfqpoint{3.352233in}{1.400000in}}{\pgfqpoint{2.407767in}{1.544118in}}%
\pgfusepath{clip}%
\pgfsetbuttcap%
\pgfsetroundjoin%
\pgfsetlinewidth{0.501875pt}%
\definecolor{currentstroke}{rgb}{0.279574,0.170599,0.479997}%
\pgfsetstrokecolor{currentstroke}%
\pgfsetdash{}{0pt}%
\pgfpathmoveto{\pgfqpoint{4.782504in}{2.242194in}}%
\pgfpathlineto{\pgfqpoint{4.729534in}{2.241749in}}%
\pgfusepath{stroke}%
\end{pgfscope}%
\begin{pgfscope}%
\pgfpathrectangle{\pgfqpoint{3.352233in}{1.400000in}}{\pgfqpoint{2.407767in}{1.544118in}}%
\pgfusepath{clip}%
\pgfsetbuttcap%
\pgfsetroundjoin%
\pgfsetlinewidth{0.501875pt}%
\definecolor{currentstroke}{rgb}{0.280868,0.160771,0.472899}%
\pgfsetstrokecolor{currentstroke}%
\pgfsetdash{}{0pt}%
\pgfpathmoveto{\pgfqpoint{4.729534in}{2.241749in}}%
\pgfpathlineto{\pgfqpoint{4.676563in}{2.241299in}}%
\pgfusepath{stroke}%
\end{pgfscope}%
\begin{pgfscope}%
\pgfpathrectangle{\pgfqpoint{3.352233in}{1.400000in}}{\pgfqpoint{2.407767in}{1.544118in}}%
\pgfusepath{clip}%
\pgfsetbuttcap%
\pgfsetroundjoin%
\pgfsetlinewidth{0.501875pt}%
\definecolor{currentstroke}{rgb}{0.278826,0.175490,0.483397}%
\pgfsetstrokecolor{currentstroke}%
\pgfsetdash{}{0pt}%
\pgfpathmoveto{\pgfqpoint{4.676563in}{2.241299in}}%
\pgfpathlineto{\pgfqpoint{4.623610in}{2.240388in}}%
\pgfusepath{stroke}%
\end{pgfscope}%
\begin{pgfscope}%
\pgfpathrectangle{\pgfqpoint{3.352233in}{1.400000in}}{\pgfqpoint{2.407767in}{1.544118in}}%
\pgfusepath{clip}%
\pgfsetbuttcap%
\pgfsetroundjoin%
\pgfsetlinewidth{0.501875pt}%
\definecolor{currentstroke}{rgb}{0.279574,0.170599,0.479997}%
\pgfsetstrokecolor{currentstroke}%
\pgfsetdash{}{0pt}%
\pgfpathmoveto{\pgfqpoint{4.623610in}{2.240388in}}%
\pgfpathlineto{\pgfqpoint{4.570677in}{2.239114in}}%
\pgfusepath{stroke}%
\end{pgfscope}%
\begin{pgfscope}%
\pgfpathrectangle{\pgfqpoint{3.352233in}{1.400000in}}{\pgfqpoint{2.407767in}{1.544118in}}%
\pgfusepath{clip}%
\pgfsetbuttcap%
\pgfsetroundjoin%
\pgfsetlinewidth{0.501875pt}%
\definecolor{currentstroke}{rgb}{0.282623,0.140926,0.457517}%
\pgfsetstrokecolor{currentstroke}%
\pgfsetdash{}{0pt}%
\pgfpathmoveto{\pgfqpoint{4.570677in}{2.239114in}}%
\pgfpathlineto{\pgfqpoint{4.517785in}{2.237326in}}%
\pgfusepath{stroke}%
\end{pgfscope}%
\begin{pgfscope}%
\pgfpathrectangle{\pgfqpoint{3.352233in}{1.400000in}}{\pgfqpoint{2.407767in}{1.544118in}}%
\pgfusepath{clip}%
\pgfsetbuttcap%
\pgfsetroundjoin%
\pgfsetlinewidth{0.501875pt}%
\definecolor{currentstroke}{rgb}{0.283072,0.130895,0.449241}%
\pgfsetstrokecolor{currentstroke}%
\pgfsetdash{}{0pt}%
\pgfpathmoveto{\pgfqpoint{4.517785in}{2.237326in}}%
\pgfpathlineto{\pgfqpoint{4.464959in}{2.234830in}}%
\pgfusepath{stroke}%
\end{pgfscope}%
\begin{pgfscope}%
\pgfpathrectangle{\pgfqpoint{3.352233in}{1.400000in}}{\pgfqpoint{2.407767in}{1.544118in}}%
\pgfusepath{clip}%
\pgfsetbuttcap%
\pgfsetroundjoin%
\pgfsetlinewidth{0.501875pt}%
\definecolor{currentstroke}{rgb}{0.283187,0.125848,0.444960}%
\pgfsetstrokecolor{currentstroke}%
\pgfsetdash{}{0pt}%
\pgfpathmoveto{\pgfqpoint{4.464959in}{2.234830in}}%
\pgfpathlineto{\pgfqpoint{4.412195in}{2.231842in}}%
\pgfusepath{stroke}%
\end{pgfscope}%
\begin{pgfscope}%
\pgfpathrectangle{\pgfqpoint{3.352233in}{1.400000in}}{\pgfqpoint{2.407767in}{1.544118in}}%
\pgfusepath{clip}%
\pgfsetbuttcap%
\pgfsetroundjoin%
\pgfsetlinewidth{0.501875pt}%
\definecolor{currentstroke}{rgb}{0.282327,0.094955,0.417331}%
\pgfsetstrokecolor{currentstroke}%
\pgfsetdash{}{0pt}%
\pgfpathmoveto{\pgfqpoint{4.412195in}{2.231842in}}%
\pgfpathlineto{\pgfqpoint{4.359839in}{2.226952in}}%
\pgfusepath{stroke}%
\end{pgfscope}%
\begin{pgfscope}%
\pgfpathrectangle{\pgfqpoint{3.352233in}{1.400000in}}{\pgfqpoint{2.407767in}{1.544118in}}%
\pgfusepath{clip}%
\pgfsetbuttcap%
\pgfsetroundjoin%
\pgfsetlinewidth{0.501875pt}%
\definecolor{currentstroke}{rgb}{0.269944,0.014625,0.341379}%
\pgfsetstrokecolor{currentstroke}%
\pgfsetdash{}{0pt}%
\pgfpathmoveto{\pgfqpoint{5.206279in}{2.276297in}}%
\pgfpathlineto{\pgfqpoint{5.153325in}{2.275884in}}%
\pgfusepath{stroke}%
\end{pgfscope}%
\begin{pgfscope}%
\pgfpathrectangle{\pgfqpoint{3.352233in}{1.400000in}}{\pgfqpoint{2.407767in}{1.544118in}}%
\pgfusepath{clip}%
\pgfsetbuttcap%
\pgfsetroundjoin%
\pgfsetlinewidth{0.501875pt}%
\definecolor{currentstroke}{rgb}{0.272594,0.025563,0.353093}%
\pgfsetstrokecolor{currentstroke}%
\pgfsetdash{}{0pt}%
\pgfpathmoveto{\pgfqpoint{5.153325in}{2.275884in}}%
\pgfpathlineto{\pgfqpoint{5.100370in}{2.275553in}}%
\pgfusepath{stroke}%
\end{pgfscope}%
\begin{pgfscope}%
\pgfpathrectangle{\pgfqpoint{3.352233in}{1.400000in}}{\pgfqpoint{2.407767in}{1.544118in}}%
\pgfusepath{clip}%
\pgfsetbuttcap%
\pgfsetroundjoin%
\pgfsetlinewidth{0.501875pt}%
\definecolor{currentstroke}{rgb}{0.276022,0.044167,0.370164}%
\pgfsetstrokecolor{currentstroke}%
\pgfsetdash{}{0pt}%
\pgfpathmoveto{\pgfqpoint{5.100370in}{2.275553in}}%
\pgfpathlineto{\pgfqpoint{5.047395in}{2.275563in}}%
\pgfusepath{stroke}%
\end{pgfscope}%
\begin{pgfscope}%
\pgfpathrectangle{\pgfqpoint{3.352233in}{1.400000in}}{\pgfqpoint{2.407767in}{1.544118in}}%
\pgfusepath{clip}%
\pgfsetbuttcap%
\pgfsetroundjoin%
\pgfsetlinewidth{0.501875pt}%
\definecolor{currentstroke}{rgb}{0.278791,0.062145,0.386592}%
\pgfsetstrokecolor{currentstroke}%
\pgfsetdash{}{0pt}%
\pgfpathmoveto{\pgfqpoint{5.047395in}{2.275563in}}%
\pgfpathlineto{\pgfqpoint{4.994420in}{2.275562in}}%
\pgfusepath{stroke}%
\end{pgfscope}%
\begin{pgfscope}%
\pgfpathrectangle{\pgfqpoint{3.352233in}{1.400000in}}{\pgfqpoint{2.407767in}{1.544118in}}%
\pgfusepath{clip}%
\pgfsetbuttcap%
\pgfsetroundjoin%
\pgfsetlinewidth{0.501875pt}%
\definecolor{currentstroke}{rgb}{0.281924,0.089666,0.412415}%
\pgfsetstrokecolor{currentstroke}%
\pgfsetdash{}{0pt}%
\pgfpathmoveto{\pgfqpoint{4.994420in}{2.275562in}}%
\pgfpathlineto{\pgfqpoint{4.941444in}{2.275435in}}%
\pgfusepath{stroke}%
\end{pgfscope}%
\begin{pgfscope}%
\pgfpathrectangle{\pgfqpoint{3.352233in}{1.400000in}}{\pgfqpoint{2.407767in}{1.544118in}}%
\pgfusepath{clip}%
\pgfsetbuttcap%
\pgfsetroundjoin%
\pgfsetlinewidth{0.501875pt}%
\definecolor{currentstroke}{rgb}{0.283091,0.110553,0.431554}%
\pgfsetstrokecolor{currentstroke}%
\pgfsetdash{}{0pt}%
\pgfpathmoveto{\pgfqpoint{4.941444in}{2.275435in}}%
\pgfpathlineto{\pgfqpoint{4.888469in}{2.275280in}}%
\pgfusepath{stroke}%
\end{pgfscope}%
\begin{pgfscope}%
\pgfpathrectangle{\pgfqpoint{3.352233in}{1.400000in}}{\pgfqpoint{2.407767in}{1.544118in}}%
\pgfusepath{clip}%
\pgfsetbuttcap%
\pgfsetroundjoin%
\pgfsetlinewidth{0.501875pt}%
\definecolor{currentstroke}{rgb}{0.283197,0.115680,0.436115}%
\pgfsetstrokecolor{currentstroke}%
\pgfsetdash{}{0pt}%
\pgfpathmoveto{\pgfqpoint{4.888469in}{2.275280in}}%
\pgfpathlineto{\pgfqpoint{4.835496in}{2.274974in}}%
\pgfusepath{stroke}%
\end{pgfscope}%
\begin{pgfscope}%
\pgfpathrectangle{\pgfqpoint{3.352233in}{1.400000in}}{\pgfqpoint{2.407767in}{1.544118in}}%
\pgfusepath{clip}%
\pgfsetbuttcap%
\pgfsetroundjoin%
\pgfsetlinewidth{0.501875pt}%
\definecolor{currentstroke}{rgb}{0.282623,0.140926,0.457517}%
\pgfsetstrokecolor{currentstroke}%
\pgfsetdash{}{0pt}%
\pgfpathmoveto{\pgfqpoint{4.835496in}{2.274974in}}%
\pgfpathlineto{\pgfqpoint{4.782530in}{2.274348in}}%
\pgfusepath{stroke}%
\end{pgfscope}%
\begin{pgfscope}%
\pgfpathrectangle{\pgfqpoint{3.352233in}{1.400000in}}{\pgfqpoint{2.407767in}{1.544118in}}%
\pgfusepath{clip}%
\pgfsetbuttcap%
\pgfsetroundjoin%
\pgfsetlinewidth{0.501875pt}%
\definecolor{currentstroke}{rgb}{0.281887,0.150881,0.465405}%
\pgfsetstrokecolor{currentstroke}%
\pgfsetdash{}{0pt}%
\pgfpathmoveto{\pgfqpoint{4.782530in}{2.274348in}}%
\pgfpathlineto{\pgfqpoint{4.729572in}{2.273477in}}%
\pgfusepath{stroke}%
\end{pgfscope}%
\begin{pgfscope}%
\pgfpathrectangle{\pgfqpoint{3.352233in}{1.400000in}}{\pgfqpoint{2.407767in}{1.544118in}}%
\pgfusepath{clip}%
\pgfsetbuttcap%
\pgfsetroundjoin%
\pgfsetlinewidth{0.501875pt}%
\definecolor{currentstroke}{rgb}{0.281412,0.155834,0.469201}%
\pgfsetstrokecolor{currentstroke}%
\pgfsetdash{}{0pt}%
\pgfpathmoveto{\pgfqpoint{4.729572in}{2.273477in}}%
\pgfpathlineto{\pgfqpoint{4.676631in}{2.272282in}}%
\pgfusepath{stroke}%
\end{pgfscope}%
\begin{pgfscope}%
\pgfpathrectangle{\pgfqpoint{3.352233in}{1.400000in}}{\pgfqpoint{2.407767in}{1.544118in}}%
\pgfusepath{clip}%
\pgfsetbuttcap%
\pgfsetroundjoin%
\pgfsetlinewidth{0.501875pt}%
\definecolor{currentstroke}{rgb}{0.278012,0.180367,0.486697}%
\pgfsetstrokecolor{currentstroke}%
\pgfsetdash{}{0pt}%
\pgfpathmoveto{\pgfqpoint{4.676631in}{2.272282in}}%
\pgfpathlineto{\pgfqpoint{4.623717in}{2.270668in}}%
\pgfusepath{stroke}%
\end{pgfscope}%
\begin{pgfscope}%
\pgfpathrectangle{\pgfqpoint{3.352233in}{1.400000in}}{\pgfqpoint{2.407767in}{1.544118in}}%
\pgfusepath{clip}%
\pgfsetbuttcap%
\pgfsetroundjoin%
\pgfsetlinewidth{0.501875pt}%
\definecolor{currentstroke}{rgb}{0.281412,0.155834,0.469201}%
\pgfsetstrokecolor{currentstroke}%
\pgfsetdash{}{0pt}%
\pgfpathmoveto{\pgfqpoint{4.623717in}{2.270668in}}%
\pgfpathlineto{\pgfqpoint{4.570856in}{2.268490in}}%
\pgfusepath{stroke}%
\end{pgfscope}%
\begin{pgfscope}%
\pgfpathrectangle{\pgfqpoint{3.352233in}{1.400000in}}{\pgfqpoint{2.407767in}{1.544118in}}%
\pgfusepath{clip}%
\pgfsetbuttcap%
\pgfsetroundjoin%
\pgfsetlinewidth{0.501875pt}%
\definecolor{currentstroke}{rgb}{0.282623,0.140926,0.457517}%
\pgfsetstrokecolor{currentstroke}%
\pgfsetdash{}{0pt}%
\pgfpathmoveto{\pgfqpoint{4.570856in}{2.268490in}}%
\pgfpathlineto{\pgfqpoint{4.518067in}{2.265698in}}%
\pgfusepath{stroke}%
\end{pgfscope}%
\begin{pgfscope}%
\pgfpathrectangle{\pgfqpoint{3.352233in}{1.400000in}}{\pgfqpoint{2.407767in}{1.544118in}}%
\pgfusepath{clip}%
\pgfsetbuttcap%
\pgfsetroundjoin%
\pgfsetlinewidth{0.501875pt}%
\definecolor{currentstroke}{rgb}{0.282290,0.145912,0.461510}%
\pgfsetstrokecolor{currentstroke}%
\pgfsetdash{}{0pt}%
\pgfpathmoveto{\pgfqpoint{4.518067in}{2.265698in}}%
\pgfpathlineto{\pgfqpoint{4.465430in}{2.261954in}}%
\pgfusepath{stroke}%
\end{pgfscope}%
\begin{pgfscope}%
\pgfpathrectangle{\pgfqpoint{3.352233in}{1.400000in}}{\pgfqpoint{2.407767in}{1.544118in}}%
\pgfusepath{clip}%
\pgfsetbuttcap%
\pgfsetroundjoin%
\pgfsetlinewidth{0.501875pt}%
\definecolor{currentstroke}{rgb}{0.269944,0.014625,0.341379}%
\pgfsetstrokecolor{currentstroke}%
\pgfsetdash{}{0pt}%
\pgfpathmoveto{\pgfqpoint{5.206279in}{2.311043in}}%
\pgfpathlineto{\pgfqpoint{5.153304in}{2.311073in}}%
\pgfusepath{stroke}%
\end{pgfscope}%
\begin{pgfscope}%
\pgfpathrectangle{\pgfqpoint{3.352233in}{1.400000in}}{\pgfqpoint{2.407767in}{1.544118in}}%
\pgfusepath{clip}%
\pgfsetbuttcap%
\pgfsetroundjoin%
\pgfsetlinewidth{0.501875pt}%
\definecolor{currentstroke}{rgb}{0.273809,0.031497,0.358853}%
\pgfsetstrokecolor{currentstroke}%
\pgfsetdash{}{0pt}%
\pgfpathmoveto{\pgfqpoint{5.153304in}{2.311073in}}%
\pgfpathlineto{\pgfqpoint{5.100329in}{2.311134in}}%
\pgfusepath{stroke}%
\end{pgfscope}%
\begin{pgfscope}%
\pgfpathrectangle{\pgfqpoint{3.352233in}{1.400000in}}{\pgfqpoint{2.407767in}{1.544118in}}%
\pgfusepath{clip}%
\pgfsetbuttcap%
\pgfsetroundjoin%
\pgfsetlinewidth{0.501875pt}%
\definecolor{currentstroke}{rgb}{0.276022,0.044167,0.370164}%
\pgfsetstrokecolor{currentstroke}%
\pgfsetdash{}{0pt}%
\pgfpathmoveto{\pgfqpoint{5.100329in}{2.311134in}}%
\pgfpathlineto{\pgfqpoint{5.047355in}{2.311003in}}%
\pgfusepath{stroke}%
\end{pgfscope}%
\begin{pgfscope}%
\pgfpathrectangle{\pgfqpoint{3.352233in}{1.400000in}}{\pgfqpoint{2.407767in}{1.544118in}}%
\pgfusepath{clip}%
\pgfsetbuttcap%
\pgfsetroundjoin%
\pgfsetlinewidth{0.501875pt}%
\definecolor{currentstroke}{rgb}{0.278791,0.062145,0.386592}%
\pgfsetstrokecolor{currentstroke}%
\pgfsetdash{}{0pt}%
\pgfpathmoveto{\pgfqpoint{5.047355in}{2.311003in}}%
\pgfpathlineto{\pgfqpoint{4.994385in}{2.310586in}}%
\pgfusepath{stroke}%
\end{pgfscope}%
\begin{pgfscope}%
\pgfpathrectangle{\pgfqpoint{3.352233in}{1.400000in}}{\pgfqpoint{2.407767in}{1.544118in}}%
\pgfusepath{clip}%
\pgfsetbuttcap%
\pgfsetroundjoin%
\pgfsetlinewidth{0.501875pt}%
\definecolor{currentstroke}{rgb}{0.281446,0.084320,0.407414}%
\pgfsetstrokecolor{currentstroke}%
\pgfsetdash{}{0pt}%
\pgfpathmoveto{\pgfqpoint{4.994385in}{2.310586in}}%
\pgfpathlineto{\pgfqpoint{4.941419in}{2.309951in}}%
\pgfusepath{stroke}%
\end{pgfscope}%
\begin{pgfscope}%
\pgfpathrectangle{\pgfqpoint{3.352233in}{1.400000in}}{\pgfqpoint{2.407767in}{1.544118in}}%
\pgfusepath{clip}%
\pgfsetbuttcap%
\pgfsetroundjoin%
\pgfsetlinewidth{0.501875pt}%
\definecolor{currentstroke}{rgb}{0.282910,0.105393,0.426902}%
\pgfsetstrokecolor{currentstroke}%
\pgfsetdash{}{0pt}%
\pgfpathmoveto{\pgfqpoint{4.941419in}{2.309951in}}%
\pgfpathlineto{\pgfqpoint{4.888453in}{2.309298in}}%
\pgfusepath{stroke}%
\end{pgfscope}%
\begin{pgfscope}%
\pgfpathrectangle{\pgfqpoint{3.352233in}{1.400000in}}{\pgfqpoint{2.407767in}{1.544118in}}%
\pgfusepath{clip}%
\pgfsetbuttcap%
\pgfsetroundjoin%
\pgfsetlinewidth{0.501875pt}%
\definecolor{currentstroke}{rgb}{0.282910,0.105393,0.426902}%
\pgfsetstrokecolor{currentstroke}%
\pgfsetdash{}{0pt}%
\pgfpathmoveto{\pgfqpoint{4.888453in}{2.309298in}}%
\pgfpathlineto{\pgfqpoint{4.835496in}{2.308424in}}%
\pgfusepath{stroke}%
\end{pgfscope}%
\begin{pgfscope}%
\pgfpathrectangle{\pgfqpoint{3.352233in}{1.400000in}}{\pgfqpoint{2.407767in}{1.544118in}}%
\pgfusepath{clip}%
\pgfsetbuttcap%
\pgfsetroundjoin%
\pgfsetlinewidth{0.501875pt}%
\definecolor{currentstroke}{rgb}{0.282623,0.140926,0.457517}%
\pgfsetstrokecolor{currentstroke}%
\pgfsetdash{}{0pt}%
\pgfpathmoveto{\pgfqpoint{4.835496in}{2.308424in}}%
\pgfpathlineto{\pgfqpoint{4.782568in}{2.307060in}}%
\pgfusepath{stroke}%
\end{pgfscope}%
\begin{pgfscope}%
\pgfpathrectangle{\pgfqpoint{3.352233in}{1.400000in}}{\pgfqpoint{2.407767in}{1.544118in}}%
\pgfusepath{clip}%
\pgfsetbuttcap%
\pgfsetroundjoin%
\pgfsetlinewidth{0.501875pt}%
\definecolor{currentstroke}{rgb}{0.282884,0.135920,0.453427}%
\pgfsetstrokecolor{currentstroke}%
\pgfsetdash{}{0pt}%
\pgfpathmoveto{\pgfqpoint{4.782568in}{2.307060in}}%
\pgfpathlineto{\pgfqpoint{4.729661in}{2.305383in}}%
\pgfusepath{stroke}%
\end{pgfscope}%
\begin{pgfscope}%
\pgfpathrectangle{\pgfqpoint{3.352233in}{1.400000in}}{\pgfqpoint{2.407767in}{1.544118in}}%
\pgfusepath{clip}%
\pgfsetbuttcap%
\pgfsetroundjoin%
\pgfsetlinewidth{0.501875pt}%
\definecolor{currentstroke}{rgb}{0.281887,0.150881,0.465405}%
\pgfsetstrokecolor{currentstroke}%
\pgfsetdash{}{0pt}%
\pgfpathmoveto{\pgfqpoint{4.729661in}{2.305383in}}%
\pgfpathlineto{\pgfqpoint{4.676789in}{2.303292in}}%
\pgfusepath{stroke}%
\end{pgfscope}%
\begin{pgfscope}%
\pgfpathrectangle{\pgfqpoint{3.352233in}{1.400000in}}{\pgfqpoint{2.407767in}{1.544118in}}%
\pgfusepath{clip}%
\pgfsetbuttcap%
\pgfsetroundjoin%
\pgfsetlinewidth{0.501875pt}%
\definecolor{currentstroke}{rgb}{0.282290,0.145912,0.461510}%
\pgfsetstrokecolor{currentstroke}%
\pgfsetdash{}{0pt}%
\pgfpathmoveto{\pgfqpoint{4.676789in}{2.303292in}}%
\pgfpathlineto{\pgfqpoint{4.623993in}{2.300537in}}%
\pgfusepath{stroke}%
\end{pgfscope}%
\begin{pgfscope}%
\pgfpathrectangle{\pgfqpoint{3.352233in}{1.400000in}}{\pgfqpoint{2.407767in}{1.544118in}}%
\pgfusepath{clip}%
\pgfsetbuttcap%
\pgfsetroundjoin%
\pgfsetlinewidth{0.501875pt}%
\definecolor{currentstroke}{rgb}{0.283072,0.130895,0.449241}%
\pgfsetstrokecolor{currentstroke}%
\pgfsetdash{}{0pt}%
\pgfpathmoveto{\pgfqpoint{4.623993in}{2.300537in}}%
\pgfpathlineto{\pgfqpoint{4.571294in}{2.297137in}}%
\pgfusepath{stroke}%
\end{pgfscope}%
\begin{pgfscope}%
\pgfpathrectangle{\pgfqpoint{3.352233in}{1.400000in}}{\pgfqpoint{2.407767in}{1.544118in}}%
\pgfusepath{clip}%
\pgfsetbuttcap%
\pgfsetroundjoin%
\pgfsetlinewidth{0.501875pt}%
\definecolor{currentstroke}{rgb}{0.268510,0.009605,0.335427}%
\pgfsetstrokecolor{currentstroke}%
\pgfsetdash{}{0pt}%
\pgfpathmoveto{\pgfqpoint{5.206279in}{2.345789in}}%
\pgfpathlineto{\pgfqpoint{5.153328in}{2.345800in}}%
\pgfusepath{stroke}%
\end{pgfscope}%
\begin{pgfscope}%
\pgfpathrectangle{\pgfqpoint{3.352233in}{1.400000in}}{\pgfqpoint{2.407767in}{1.544118in}}%
\pgfusepath{clip}%
\pgfsetbuttcap%
\pgfsetroundjoin%
\pgfsetlinewidth{0.501875pt}%
\definecolor{currentstroke}{rgb}{0.272594,0.025563,0.353093}%
\pgfsetstrokecolor{currentstroke}%
\pgfsetdash{}{0pt}%
\pgfpathmoveto{\pgfqpoint{5.153328in}{2.345800in}}%
\pgfpathlineto{\pgfqpoint{5.100354in}{2.345765in}}%
\pgfusepath{stroke}%
\end{pgfscope}%
\begin{pgfscope}%
\pgfpathrectangle{\pgfqpoint{3.352233in}{1.400000in}}{\pgfqpoint{2.407767in}{1.544118in}}%
\pgfusepath{clip}%
\pgfsetbuttcap%
\pgfsetroundjoin%
\pgfsetlinewidth{0.501875pt}%
\definecolor{currentstroke}{rgb}{0.274952,0.037752,0.364543}%
\pgfsetstrokecolor{currentstroke}%
\pgfsetdash{}{0pt}%
\pgfpathmoveto{\pgfqpoint{5.100354in}{2.345765in}}%
\pgfpathlineto{\pgfqpoint{5.047383in}{2.345369in}}%
\pgfusepath{stroke}%
\end{pgfscope}%
\begin{pgfscope}%
\pgfpathrectangle{\pgfqpoint{3.352233in}{1.400000in}}{\pgfqpoint{2.407767in}{1.544118in}}%
\pgfusepath{clip}%
\pgfsetbuttcap%
\pgfsetroundjoin%
\pgfsetlinewidth{0.501875pt}%
\definecolor{currentstroke}{rgb}{0.277941,0.056324,0.381191}%
\pgfsetstrokecolor{currentstroke}%
\pgfsetdash{}{0pt}%
\pgfpathmoveto{\pgfqpoint{5.047383in}{2.345369in}}%
\pgfpathlineto{\pgfqpoint{4.994414in}{2.344865in}}%
\pgfusepath{stroke}%
\end{pgfscope}%
\begin{pgfscope}%
\pgfpathrectangle{\pgfqpoint{3.352233in}{1.400000in}}{\pgfqpoint{2.407767in}{1.544118in}}%
\pgfusepath{clip}%
\pgfsetbuttcap%
\pgfsetroundjoin%
\pgfsetlinewidth{0.501875pt}%
\definecolor{currentstroke}{rgb}{0.280894,0.078907,0.402329}%
\pgfsetstrokecolor{currentstroke}%
\pgfsetdash{}{0pt}%
\pgfpathmoveto{\pgfqpoint{4.994414in}{2.344865in}}%
\pgfpathlineto{\pgfqpoint{4.941455in}{2.344069in}}%
\pgfusepath{stroke}%
\end{pgfscope}%
\begin{pgfscope}%
\pgfpathrectangle{\pgfqpoint{3.352233in}{1.400000in}}{\pgfqpoint{2.407767in}{1.544118in}}%
\pgfusepath{clip}%
\pgfsetbuttcap%
\pgfsetroundjoin%
\pgfsetlinewidth{0.501875pt}%
\definecolor{currentstroke}{rgb}{0.282327,0.094955,0.417331}%
\pgfsetstrokecolor{currentstroke}%
\pgfsetdash{}{0pt}%
\pgfpathmoveto{\pgfqpoint{4.941455in}{2.344069in}}%
\pgfpathlineto{\pgfqpoint{4.888510in}{2.342913in}}%
\pgfusepath{stroke}%
\end{pgfscope}%
\begin{pgfscope}%
\pgfpathrectangle{\pgfqpoint{3.352233in}{1.400000in}}{\pgfqpoint{2.407767in}{1.544118in}}%
\pgfusepath{clip}%
\pgfsetbuttcap%
\pgfsetroundjoin%
\pgfsetlinewidth{0.501875pt}%
\definecolor{currentstroke}{rgb}{0.282656,0.100196,0.422160}%
\pgfsetstrokecolor{currentstroke}%
\pgfsetdash{}{0pt}%
\pgfpathmoveto{\pgfqpoint{4.888510in}{2.342913in}}%
\pgfpathlineto{\pgfqpoint{4.835588in}{2.341429in}}%
\pgfusepath{stroke}%
\end{pgfscope}%
\begin{pgfscope}%
\pgfpathrectangle{\pgfqpoint{3.352233in}{1.400000in}}{\pgfqpoint{2.407767in}{1.544118in}}%
\pgfusepath{clip}%
\pgfsetbuttcap%
\pgfsetroundjoin%
\pgfsetlinewidth{0.501875pt}%
\definecolor{currentstroke}{rgb}{0.283229,0.120777,0.440584}%
\pgfsetstrokecolor{currentstroke}%
\pgfsetdash{}{0pt}%
\pgfpathmoveto{\pgfqpoint{4.835588in}{2.341429in}}%
\pgfpathlineto{\pgfqpoint{4.782716in}{2.339348in}}%
\pgfusepath{stroke}%
\end{pgfscope}%
\begin{pgfscope}%
\pgfpathrectangle{\pgfqpoint{3.352233in}{1.400000in}}{\pgfqpoint{2.407767in}{1.544118in}}%
\pgfusepath{clip}%
\pgfsetbuttcap%
\pgfsetroundjoin%
\pgfsetlinewidth{0.501875pt}%
\definecolor{currentstroke}{rgb}{0.282884,0.135920,0.453427}%
\pgfsetstrokecolor{currentstroke}%
\pgfsetdash{}{0pt}%
\pgfpathmoveto{\pgfqpoint{4.782716in}{2.339348in}}%
\pgfpathlineto{\pgfqpoint{4.729887in}{2.336828in}}%
\pgfusepath{stroke}%
\end{pgfscope}%
\begin{pgfscope}%
\pgfpathrectangle{\pgfqpoint{3.352233in}{1.400000in}}{\pgfqpoint{2.407767in}{1.544118in}}%
\pgfusepath{clip}%
\pgfsetbuttcap%
\pgfsetroundjoin%
\pgfsetlinewidth{0.501875pt}%
\definecolor{currentstroke}{rgb}{0.283072,0.130895,0.449241}%
\pgfsetstrokecolor{currentstroke}%
\pgfsetdash{}{0pt}%
\pgfpathmoveto{\pgfqpoint{4.729887in}{2.336828in}}%
\pgfpathlineto{\pgfqpoint{4.677101in}{2.333980in}}%
\pgfusepath{stroke}%
\end{pgfscope}%
\begin{pgfscope}%
\pgfpathrectangle{\pgfqpoint{3.352233in}{1.400000in}}{\pgfqpoint{2.407767in}{1.544118in}}%
\pgfusepath{clip}%
\pgfsetbuttcap%
\pgfsetroundjoin%
\pgfsetlinewidth{0.501875pt}%
\definecolor{currentstroke}{rgb}{0.282884,0.135920,0.453427}%
\pgfsetstrokecolor{currentstroke}%
\pgfsetdash{}{0pt}%
\pgfpathmoveto{\pgfqpoint{4.677101in}{2.333980in}}%
\pgfpathlineto{\pgfqpoint{4.624462in}{2.330232in}}%
\pgfusepath{stroke}%
\end{pgfscope}%
\begin{pgfscope}%
\pgfpathrectangle{\pgfqpoint{3.352233in}{1.400000in}}{\pgfqpoint{2.407767in}{1.544118in}}%
\pgfusepath{clip}%
\pgfsetbuttcap%
\pgfsetroundjoin%
\pgfsetlinewidth{0.501875pt}%
\definecolor{currentstroke}{rgb}{0.269944,0.014625,0.341379}%
\pgfsetstrokecolor{currentstroke}%
\pgfsetdash{}{0pt}%
\pgfpathmoveto{\pgfqpoint{5.206279in}{2.380536in}}%
\pgfpathlineto{\pgfqpoint{5.153311in}{2.380073in}}%
\pgfusepath{stroke}%
\end{pgfscope}%
\begin{pgfscope}%
\pgfpathrectangle{\pgfqpoint{3.352233in}{1.400000in}}{\pgfqpoint{2.407767in}{1.544118in}}%
\pgfusepath{clip}%
\pgfsetbuttcap%
\pgfsetroundjoin%
\pgfsetlinewidth{0.501875pt}%
\definecolor{currentstroke}{rgb}{0.272594,0.025563,0.353093}%
\pgfsetstrokecolor{currentstroke}%
\pgfsetdash{}{0pt}%
\pgfpathmoveto{\pgfqpoint{5.153311in}{2.380073in}}%
\pgfpathlineto{\pgfqpoint{5.100338in}{2.379917in}}%
\pgfusepath{stroke}%
\end{pgfscope}%
\begin{pgfscope}%
\pgfpathrectangle{\pgfqpoint{3.352233in}{1.400000in}}{\pgfqpoint{2.407767in}{1.544118in}}%
\pgfusepath{clip}%
\pgfsetbuttcap%
\pgfsetroundjoin%
\pgfsetlinewidth{0.501875pt}%
\definecolor{currentstroke}{rgb}{0.274952,0.037752,0.364543}%
\pgfsetstrokecolor{currentstroke}%
\pgfsetdash{}{0pt}%
\pgfpathmoveto{\pgfqpoint{5.100338in}{2.379917in}}%
\pgfpathlineto{\pgfqpoint{5.047367in}{2.379615in}}%
\pgfusepath{stroke}%
\end{pgfscope}%
\begin{pgfscope}%
\pgfpathrectangle{\pgfqpoint{3.352233in}{1.400000in}}{\pgfqpoint{2.407767in}{1.544118in}}%
\pgfusepath{clip}%
\pgfsetbuttcap%
\pgfsetroundjoin%
\pgfsetlinewidth{0.501875pt}%
\definecolor{currentstroke}{rgb}{0.277018,0.050344,0.375715}%
\pgfsetstrokecolor{currentstroke}%
\pgfsetdash{}{0pt}%
\pgfpathmoveto{\pgfqpoint{5.047367in}{2.379615in}}%
\pgfpathlineto{\pgfqpoint{4.994397in}{2.379136in}}%
\pgfusepath{stroke}%
\end{pgfscope}%
\begin{pgfscope}%
\pgfpathrectangle{\pgfqpoint{3.352233in}{1.400000in}}{\pgfqpoint{2.407767in}{1.544118in}}%
\pgfusepath{clip}%
\pgfsetbuttcap%
\pgfsetroundjoin%
\pgfsetlinewidth{0.501875pt}%
\definecolor{currentstroke}{rgb}{0.278791,0.062145,0.386592}%
\pgfsetstrokecolor{currentstroke}%
\pgfsetdash{}{0pt}%
\pgfpathmoveto{\pgfqpoint{4.994397in}{2.379136in}}%
\pgfpathlineto{\pgfqpoint{4.941430in}{2.378596in}}%
\pgfusepath{stroke}%
\end{pgfscope}%
\begin{pgfscope}%
\pgfpathrectangle{\pgfqpoint{3.352233in}{1.400000in}}{\pgfqpoint{2.407767in}{1.544118in}}%
\pgfusepath{clip}%
\pgfsetbuttcap%
\pgfsetroundjoin%
\pgfsetlinewidth{0.501875pt}%
\definecolor{currentstroke}{rgb}{0.280894,0.078907,0.402329}%
\pgfsetstrokecolor{currentstroke}%
\pgfsetdash{}{0pt}%
\pgfpathmoveto{\pgfqpoint{4.941430in}{2.378596in}}%
\pgfpathlineto{\pgfqpoint{4.888472in}{2.377729in}}%
\pgfusepath{stroke}%
\end{pgfscope}%
\begin{pgfscope}%
\pgfpathrectangle{\pgfqpoint{3.352233in}{1.400000in}}{\pgfqpoint{2.407767in}{1.544118in}}%
\pgfusepath{clip}%
\pgfsetbuttcap%
\pgfsetroundjoin%
\pgfsetlinewidth{0.501875pt}%
\definecolor{currentstroke}{rgb}{0.282656,0.100196,0.422160}%
\pgfsetstrokecolor{currentstroke}%
\pgfsetdash{}{0pt}%
\pgfpathmoveto{\pgfqpoint{4.888472in}{2.377729in}}%
\pgfpathlineto{\pgfqpoint{4.835543in}{2.376379in}}%
\pgfusepath{stroke}%
\end{pgfscope}%
\begin{pgfscope}%
\pgfpathrectangle{\pgfqpoint{3.352233in}{1.400000in}}{\pgfqpoint{2.407767in}{1.544118in}}%
\pgfusepath{clip}%
\pgfsetbuttcap%
\pgfsetroundjoin%
\pgfsetlinewidth{0.501875pt}%
\definecolor{currentstroke}{rgb}{0.283091,0.110553,0.431554}%
\pgfsetstrokecolor{currentstroke}%
\pgfsetdash{}{0pt}%
\pgfpathmoveto{\pgfqpoint{4.835543in}{2.376379in}}%
\pgfpathlineto{\pgfqpoint{4.782665in}{2.374405in}}%
\pgfusepath{stroke}%
\end{pgfscope}%
\begin{pgfscope}%
\pgfpathrectangle{\pgfqpoint{3.352233in}{1.400000in}}{\pgfqpoint{2.407767in}{1.544118in}}%
\pgfusepath{clip}%
\pgfsetbuttcap%
\pgfsetroundjoin%
\pgfsetlinewidth{0.501875pt}%
\definecolor{currentstroke}{rgb}{0.282656,0.100196,0.422160}%
\pgfsetstrokecolor{currentstroke}%
\pgfsetdash{}{0pt}%
\pgfpathmoveto{\pgfqpoint{4.782665in}{2.374405in}}%
\pgfpathlineto{\pgfqpoint{4.729801in}{2.372316in}}%
\pgfusepath{stroke}%
\end{pgfscope}%
\begin{pgfscope}%
\pgfpathrectangle{\pgfqpoint{3.352233in}{1.400000in}}{\pgfqpoint{2.407767in}{1.544118in}}%
\pgfusepath{clip}%
\pgfsetbuttcap%
\pgfsetroundjoin%
\pgfsetlinewidth{0.501875pt}%
\definecolor{currentstroke}{rgb}{0.283197,0.115680,0.436115}%
\pgfsetstrokecolor{currentstroke}%
\pgfsetdash{}{0pt}%
\pgfpathmoveto{\pgfqpoint{4.729801in}{2.372316in}}%
\pgfpathlineto{\pgfqpoint{4.677033in}{2.369494in}}%
\pgfusepath{stroke}%
\end{pgfscope}%
\begin{pgfscope}%
\pgfpathrectangle{\pgfqpoint{3.352233in}{1.400000in}}{\pgfqpoint{2.407767in}{1.544118in}}%
\pgfusepath{clip}%
\pgfsetbuttcap%
\pgfsetroundjoin%
\pgfsetlinewidth{0.501875pt}%
\definecolor{currentstroke}{rgb}{0.282656,0.100196,0.422160}%
\pgfsetstrokecolor{currentstroke}%
\pgfsetdash{}{0pt}%
\pgfpathmoveto{\pgfqpoint{4.677033in}{2.369494in}}%
\pgfpathlineto{\pgfqpoint{4.624491in}{2.365208in}}%
\pgfusepath{stroke}%
\end{pgfscope}%
\begin{pgfscope}%
\pgfpathrectangle{\pgfqpoint{3.352233in}{1.400000in}}{\pgfqpoint{2.407767in}{1.544118in}}%
\pgfusepath{clip}%
\pgfsetbuttcap%
\pgfsetroundjoin%
\pgfsetlinewidth{0.501875pt}%
\definecolor{currentstroke}{rgb}{0.283197,0.115680,0.436115}%
\pgfsetstrokecolor{currentstroke}%
\pgfsetdash{}{0pt}%
\pgfpathmoveto{\pgfqpoint{4.624491in}{2.365208in}}%
\pgfpathlineto{\pgfqpoint{4.572404in}{2.359189in}}%
\pgfusepath{stroke}%
\end{pgfscope}%
\begin{pgfscope}%
\pgfpathrectangle{\pgfqpoint{3.352233in}{1.400000in}}{\pgfqpoint{2.407767in}{1.544118in}}%
\pgfusepath{clip}%
\pgfsetbuttcap%
\pgfsetroundjoin%
\pgfsetlinewidth{0.501875pt}%
\definecolor{currentstroke}{rgb}{0.283091,0.110553,0.431554}%
\pgfsetstrokecolor{currentstroke}%
\pgfsetdash{}{0pt}%
\pgfpathmoveto{\pgfqpoint{4.572404in}{2.359189in}}%
\pgfpathlineto{\pgfqpoint{4.520847in}{2.351462in}}%
\pgfusepath{stroke}%
\end{pgfscope}%
\begin{pgfscope}%
\pgfpathrectangle{\pgfqpoint{3.352233in}{1.400000in}}{\pgfqpoint{2.407767in}{1.544118in}}%
\pgfusepath{clip}%
\pgfsetbuttcap%
\pgfsetroundjoin%
\pgfsetlinewidth{0.501875pt}%
\definecolor{currentstroke}{rgb}{0.282910,0.105393,0.426902}%
\pgfsetstrokecolor{currentstroke}%
\pgfsetdash{}{0pt}%
\pgfpathmoveto{\pgfqpoint{4.520847in}{2.351462in}}%
\pgfpathlineto{\pgfqpoint{4.470035in}{2.341946in}}%
\pgfusepath{stroke}%
\end{pgfscope}%
\begin{pgfscope}%
\pgfpathrectangle{\pgfqpoint{3.352233in}{1.400000in}}{\pgfqpoint{2.407767in}{1.544118in}}%
\pgfusepath{clip}%
\pgfsetbuttcap%
\pgfsetroundjoin%
\pgfsetlinewidth{0.501875pt}%
\definecolor{currentstroke}{rgb}{0.282910,0.105393,0.426902}%
\pgfsetstrokecolor{currentstroke}%
\pgfsetdash{}{0pt}%
\pgfpathmoveto{\pgfqpoint{4.470035in}{2.341946in}}%
\pgfpathlineto{\pgfqpoint{4.421040in}{2.329323in}}%
\pgfusepath{stroke}%
\end{pgfscope}%
\begin{pgfscope}%
\pgfpathrectangle{\pgfqpoint{3.352233in}{1.400000in}}{\pgfqpoint{2.407767in}{1.544118in}}%
\pgfusepath{clip}%
\pgfsetbuttcap%
\pgfsetroundjoin%
\pgfsetlinewidth{0.501875pt}%
\definecolor{currentstroke}{rgb}{0.282327,0.094955,0.417331}%
\pgfsetstrokecolor{currentstroke}%
\pgfsetdash{}{0pt}%
\pgfpathmoveto{\pgfqpoint{4.421040in}{2.329323in}}%
\pgfpathlineto{\pgfqpoint{4.374506in}{2.313412in}}%
\pgfusepath{stroke}%
\end{pgfscope}%
\begin{pgfscope}%
\pgfpathrectangle{\pgfqpoint{3.352233in}{1.400000in}}{\pgfqpoint{2.407767in}{1.544118in}}%
\pgfusepath{clip}%
\pgfsetbuttcap%
\pgfsetroundjoin%
\pgfsetlinewidth{0.501875pt}%
\definecolor{currentstroke}{rgb}{0.269944,0.014625,0.341379}%
\pgfsetstrokecolor{currentstroke}%
\pgfsetdash{}{0pt}%
\pgfpathmoveto{\pgfqpoint{5.206279in}{2.415282in}}%
\pgfpathlineto{\pgfqpoint{5.153349in}{2.414387in}}%
\pgfusepath{stroke}%
\end{pgfscope}%
\begin{pgfscope}%
\pgfpathrectangle{\pgfqpoint{3.352233in}{1.400000in}}{\pgfqpoint{2.407767in}{1.544118in}}%
\pgfusepath{clip}%
\pgfsetbuttcap%
\pgfsetroundjoin%
\pgfsetlinewidth{0.501875pt}%
\definecolor{currentstroke}{rgb}{0.271305,0.019942,0.347269}%
\pgfsetstrokecolor{currentstroke}%
\pgfsetdash{}{0pt}%
\pgfpathmoveto{\pgfqpoint{5.153349in}{2.414387in}}%
\pgfpathlineto{\pgfqpoint{5.100418in}{2.413529in}}%
\pgfusepath{stroke}%
\end{pgfscope}%
\begin{pgfscope}%
\pgfpathrectangle{\pgfqpoint{3.352233in}{1.400000in}}{\pgfqpoint{2.407767in}{1.544118in}}%
\pgfusepath{clip}%
\pgfsetbuttcap%
\pgfsetroundjoin%
\pgfsetlinewidth{0.501875pt}%
\definecolor{currentstroke}{rgb}{0.273809,0.031497,0.358853}%
\pgfsetstrokecolor{currentstroke}%
\pgfsetdash{}{0pt}%
\pgfpathmoveto{\pgfqpoint{5.100418in}{2.413529in}}%
\pgfpathlineto{\pgfqpoint{5.047446in}{2.413269in}}%
\pgfusepath{stroke}%
\end{pgfscope}%
\begin{pgfscope}%
\pgfpathrectangle{\pgfqpoint{3.352233in}{1.400000in}}{\pgfqpoint{2.407767in}{1.544118in}}%
\pgfusepath{clip}%
\pgfsetbuttcap%
\pgfsetroundjoin%
\pgfsetlinewidth{0.501875pt}%
\definecolor{currentstroke}{rgb}{0.277018,0.050344,0.375715}%
\pgfsetstrokecolor{currentstroke}%
\pgfsetdash{}{0pt}%
\pgfpathmoveto{\pgfqpoint{5.047446in}{2.413269in}}%
\pgfpathlineto{\pgfqpoint{4.994484in}{2.412788in}}%
\pgfusepath{stroke}%
\end{pgfscope}%
\begin{pgfscope}%
\pgfpathrectangle{\pgfqpoint{3.352233in}{1.400000in}}{\pgfqpoint{2.407767in}{1.544118in}}%
\pgfusepath{clip}%
\pgfsetbuttcap%
\pgfsetroundjoin%
\pgfsetlinewidth{0.501875pt}%
\definecolor{currentstroke}{rgb}{0.277941,0.056324,0.381191}%
\pgfsetstrokecolor{currentstroke}%
\pgfsetdash{}{0pt}%
\pgfpathmoveto{\pgfqpoint{4.994484in}{2.412788in}}%
\pgfpathlineto{\pgfqpoint{4.941524in}{2.412105in}}%
\pgfusepath{stroke}%
\end{pgfscope}%
\begin{pgfscope}%
\pgfpathrectangle{\pgfqpoint{3.352233in}{1.400000in}}{\pgfqpoint{2.407767in}{1.544118in}}%
\pgfusepath{clip}%
\pgfsetbuttcap%
\pgfsetroundjoin%
\pgfsetlinewidth{0.501875pt}%
\definecolor{currentstroke}{rgb}{0.281924,0.089666,0.412415}%
\pgfsetstrokecolor{currentstroke}%
\pgfsetdash{}{0pt}%
\pgfpathmoveto{\pgfqpoint{4.941524in}{2.412105in}}%
\pgfpathlineto{\pgfqpoint{4.888561in}{2.411422in}}%
\pgfusepath{stroke}%
\end{pgfscope}%
\begin{pgfscope}%
\pgfpathrectangle{\pgfqpoint{3.352233in}{1.400000in}}{\pgfqpoint{2.407767in}{1.544118in}}%
\pgfusepath{clip}%
\pgfsetbuttcap%
\pgfsetroundjoin%
\pgfsetlinewidth{0.501875pt}%
\definecolor{currentstroke}{rgb}{0.281446,0.084320,0.407414}%
\pgfsetstrokecolor{currentstroke}%
\pgfsetdash{}{0pt}%
\pgfpathmoveto{\pgfqpoint{4.888561in}{2.411422in}}%
\pgfpathlineto{\pgfqpoint{4.835630in}{2.410136in}}%
\pgfusepath{stroke}%
\end{pgfscope}%
\begin{pgfscope}%
\pgfpathrectangle{\pgfqpoint{3.352233in}{1.400000in}}{\pgfqpoint{2.407767in}{1.544118in}}%
\pgfusepath{clip}%
\pgfsetbuttcap%
\pgfsetroundjoin%
\pgfsetlinewidth{0.501875pt}%
\definecolor{currentstroke}{rgb}{0.282327,0.094955,0.417331}%
\pgfsetstrokecolor{currentstroke}%
\pgfsetdash{}{0pt}%
\pgfpathmoveto{\pgfqpoint{4.835630in}{2.410136in}}%
\pgfpathlineto{\pgfqpoint{4.782782in}{2.407879in}}%
\pgfusepath{stroke}%
\end{pgfscope}%
\begin{pgfscope}%
\pgfpathrectangle{\pgfqpoint{3.352233in}{1.400000in}}{\pgfqpoint{2.407767in}{1.544118in}}%
\pgfusepath{clip}%
\pgfsetbuttcap%
\pgfsetroundjoin%
\pgfsetlinewidth{0.501875pt}%
\definecolor{currentstroke}{rgb}{0.281924,0.089666,0.412415}%
\pgfsetstrokecolor{currentstroke}%
\pgfsetdash{}{0pt}%
\pgfpathmoveto{\pgfqpoint{4.782782in}{2.407879in}}%
\pgfpathlineto{\pgfqpoint{4.730006in}{2.404992in}}%
\pgfusepath{stroke}%
\end{pgfscope}%
\begin{pgfscope}%
\pgfpathrectangle{\pgfqpoint{3.352233in}{1.400000in}}{\pgfqpoint{2.407767in}{1.544118in}}%
\pgfusepath{clip}%
\pgfsetbuttcap%
\pgfsetroundjoin%
\pgfsetlinewidth{0.501875pt}%
\definecolor{currentstroke}{rgb}{0.282656,0.100196,0.422160}%
\pgfsetstrokecolor{currentstroke}%
\pgfsetdash{}{0pt}%
\pgfpathmoveto{\pgfqpoint{4.730006in}{2.404992in}}%
\pgfpathlineto{\pgfqpoint{4.677364in}{2.401265in}}%
\pgfusepath{stroke}%
\end{pgfscope}%
\begin{pgfscope}%
\pgfpathrectangle{\pgfqpoint{3.352233in}{1.400000in}}{\pgfqpoint{2.407767in}{1.544118in}}%
\pgfusepath{clip}%
\pgfsetbuttcap%
\pgfsetroundjoin%
\pgfsetlinewidth{0.501875pt}%
\definecolor{currentstroke}{rgb}{0.269944,0.014625,0.341379}%
\pgfsetstrokecolor{currentstroke}%
\pgfsetdash{}{0pt}%
\pgfpathmoveto{\pgfqpoint{5.206279in}{2.450028in}}%
\pgfpathlineto{\pgfqpoint{5.153312in}{2.449619in}}%
\pgfusepath{stroke}%
\end{pgfscope}%
\begin{pgfscope}%
\pgfpathrectangle{\pgfqpoint{3.352233in}{1.400000in}}{\pgfqpoint{2.407767in}{1.544118in}}%
\pgfusepath{clip}%
\pgfsetbuttcap%
\pgfsetroundjoin%
\pgfsetlinewidth{0.501875pt}%
\definecolor{currentstroke}{rgb}{0.272594,0.025563,0.353093}%
\pgfsetstrokecolor{currentstroke}%
\pgfsetdash{}{0pt}%
\pgfpathmoveto{\pgfqpoint{5.153312in}{2.449619in}}%
\pgfpathlineto{\pgfqpoint{5.100346in}{2.449578in}}%
\pgfusepath{stroke}%
\end{pgfscope}%
\begin{pgfscope}%
\pgfpathrectangle{\pgfqpoint{3.352233in}{1.400000in}}{\pgfqpoint{2.407767in}{1.544118in}}%
\pgfusepath{clip}%
\pgfsetbuttcap%
\pgfsetroundjoin%
\pgfsetlinewidth{0.501875pt}%
\definecolor{currentstroke}{rgb}{0.274952,0.037752,0.364543}%
\pgfsetstrokecolor{currentstroke}%
\pgfsetdash{}{0pt}%
\pgfpathmoveto{\pgfqpoint{5.100346in}{2.449578in}}%
\pgfpathlineto{\pgfqpoint{5.047379in}{2.449649in}}%
\pgfusepath{stroke}%
\end{pgfscope}%
\begin{pgfscope}%
\pgfpathrectangle{\pgfqpoint{3.352233in}{1.400000in}}{\pgfqpoint{2.407767in}{1.544118in}}%
\pgfusepath{clip}%
\pgfsetbuttcap%
\pgfsetroundjoin%
\pgfsetlinewidth{0.501875pt}%
\definecolor{currentstroke}{rgb}{0.277941,0.056324,0.381191}%
\pgfsetstrokecolor{currentstroke}%
\pgfsetdash{}{0pt}%
\pgfpathmoveto{\pgfqpoint{5.047379in}{2.449649in}}%
\pgfpathlineto{\pgfqpoint{4.994414in}{2.449130in}}%
\pgfusepath{stroke}%
\end{pgfscope}%
\begin{pgfscope}%
\pgfpathrectangle{\pgfqpoint{3.352233in}{1.400000in}}{\pgfqpoint{2.407767in}{1.544118in}}%
\pgfusepath{clip}%
\pgfsetbuttcap%
\pgfsetroundjoin%
\pgfsetlinewidth{0.501875pt}%
\definecolor{currentstroke}{rgb}{0.277941,0.056324,0.381191}%
\pgfsetstrokecolor{currentstroke}%
\pgfsetdash{}{0pt}%
\pgfpathmoveto{\pgfqpoint{4.994414in}{2.449130in}}%
\pgfpathlineto{\pgfqpoint{4.941452in}{2.448404in}}%
\pgfusepath{stroke}%
\end{pgfscope}%
\begin{pgfscope}%
\pgfpathrectangle{\pgfqpoint{3.352233in}{1.400000in}}{\pgfqpoint{2.407767in}{1.544118in}}%
\pgfusepath{clip}%
\pgfsetbuttcap%
\pgfsetroundjoin%
\pgfsetlinewidth{0.501875pt}%
\definecolor{currentstroke}{rgb}{0.280267,0.073417,0.397163}%
\pgfsetstrokecolor{currentstroke}%
\pgfsetdash{}{0pt}%
\pgfpathmoveto{\pgfqpoint{4.941452in}{2.448404in}}%
\pgfpathlineto{\pgfqpoint{4.888497in}{2.447481in}}%
\pgfusepath{stroke}%
\end{pgfscope}%
\begin{pgfscope}%
\pgfpathrectangle{\pgfqpoint{3.352233in}{1.400000in}}{\pgfqpoint{2.407767in}{1.544118in}}%
\pgfusepath{clip}%
\pgfsetbuttcap%
\pgfsetroundjoin%
\pgfsetlinewidth{0.501875pt}%
\definecolor{currentstroke}{rgb}{0.280267,0.073417,0.397163}%
\pgfsetstrokecolor{currentstroke}%
\pgfsetdash{}{0pt}%
\pgfpathmoveto{\pgfqpoint{4.888497in}{2.447481in}}%
\pgfpathlineto{\pgfqpoint{4.835595in}{2.445807in}}%
\pgfusepath{stroke}%
\end{pgfscope}%
\begin{pgfscope}%
\pgfpathrectangle{\pgfqpoint{3.352233in}{1.400000in}}{\pgfqpoint{2.407767in}{1.544118in}}%
\pgfusepath{clip}%
\pgfsetbuttcap%
\pgfsetroundjoin%
\pgfsetlinewidth{0.501875pt}%
\definecolor{currentstroke}{rgb}{0.281446,0.084320,0.407414}%
\pgfsetstrokecolor{currentstroke}%
\pgfsetdash{}{0pt}%
\pgfpathmoveto{\pgfqpoint{4.835595in}{2.445807in}}%
\pgfpathlineto{\pgfqpoint{4.782728in}{2.443677in}}%
\pgfusepath{stroke}%
\end{pgfscope}%
\begin{pgfscope}%
\pgfpathrectangle{\pgfqpoint{3.352233in}{1.400000in}}{\pgfqpoint{2.407767in}{1.544118in}}%
\pgfusepath{clip}%
\pgfsetbuttcap%
\pgfsetroundjoin%
\pgfsetlinewidth{0.501875pt}%
\definecolor{currentstroke}{rgb}{0.280894,0.078907,0.402329}%
\pgfsetstrokecolor{currentstroke}%
\pgfsetdash{}{0pt}%
\pgfpathmoveto{\pgfqpoint{4.782728in}{2.443677in}}%
\pgfpathlineto{\pgfqpoint{4.729922in}{2.441031in}}%
\pgfusepath{stroke}%
\end{pgfscope}%
\begin{pgfscope}%
\pgfpathrectangle{\pgfqpoint{3.352233in}{1.400000in}}{\pgfqpoint{2.407767in}{1.544118in}}%
\pgfusepath{clip}%
\pgfsetbuttcap%
\pgfsetroundjoin%
\pgfsetlinewidth{0.501875pt}%
\definecolor{currentstroke}{rgb}{0.280267,0.073417,0.397163}%
\pgfsetstrokecolor{currentstroke}%
\pgfsetdash{}{0pt}%
\pgfpathmoveto{\pgfqpoint{4.729922in}{2.441031in}}%
\pgfpathlineto{\pgfqpoint{4.677219in}{2.437624in}}%
\pgfusepath{stroke}%
\end{pgfscope}%
\begin{pgfscope}%
\pgfpathrectangle{\pgfqpoint{3.352233in}{1.400000in}}{\pgfqpoint{2.407767in}{1.544118in}}%
\pgfusepath{clip}%
\pgfsetbuttcap%
\pgfsetroundjoin%
\pgfsetlinewidth{0.501875pt}%
\definecolor{currentstroke}{rgb}{0.280267,0.073417,0.397163}%
\pgfsetstrokecolor{currentstroke}%
\pgfsetdash{}{0pt}%
\pgfpathmoveto{\pgfqpoint{4.677219in}{2.437624in}}%
\pgfpathlineto{\pgfqpoint{4.624864in}{2.432667in}}%
\pgfusepath{stroke}%
\end{pgfscope}%
\begin{pgfscope}%
\pgfpathrectangle{\pgfqpoint{3.352233in}{1.400000in}}{\pgfqpoint{2.407767in}{1.544118in}}%
\pgfusepath{clip}%
\pgfsetbuttcap%
\pgfsetroundjoin%
\pgfsetlinewidth{0.501875pt}%
\definecolor{currentstroke}{rgb}{0.279566,0.067836,0.391917}%
\pgfsetstrokecolor{currentstroke}%
\pgfsetdash{}{0pt}%
\pgfpathmoveto{\pgfqpoint{4.624864in}{2.432667in}}%
\pgfpathlineto{\pgfqpoint{4.572784in}{2.426612in}}%
\pgfusepath{stroke}%
\end{pgfscope}%
\begin{pgfscope}%
\pgfpathrectangle{\pgfqpoint{3.352233in}{1.400000in}}{\pgfqpoint{2.407767in}{1.544118in}}%
\pgfusepath{clip}%
\pgfsetbuttcap%
\pgfsetroundjoin%
\pgfsetlinewidth{0.501875pt}%
\definecolor{currentstroke}{rgb}{0.278791,0.062145,0.386592}%
\pgfsetstrokecolor{currentstroke}%
\pgfsetdash{}{0pt}%
\pgfpathmoveto{\pgfqpoint{4.572784in}{2.426612in}}%
\pgfpathlineto{\pgfqpoint{4.522091in}{2.417596in}}%
\pgfusepath{stroke}%
\end{pgfscope}%
\begin{pgfscope}%
\pgfpathrectangle{\pgfqpoint{3.352233in}{1.400000in}}{\pgfqpoint{2.407767in}{1.544118in}}%
\pgfusepath{clip}%
\pgfsetbuttcap%
\pgfsetroundjoin%
\pgfsetlinewidth{0.501875pt}%
\definecolor{currentstroke}{rgb}{0.277941,0.056324,0.381191}%
\pgfsetstrokecolor{currentstroke}%
\pgfsetdash{}{0pt}%
\pgfpathmoveto{\pgfqpoint{4.522091in}{2.417596in}}%
\pgfpathlineto{\pgfqpoint{4.475344in}{2.403676in}}%
\pgfusepath{stroke}%
\end{pgfscope}%
\begin{pgfscope}%
\pgfpathrectangle{\pgfqpoint{3.352233in}{1.400000in}}{\pgfqpoint{2.407767in}{1.544118in}}%
\pgfusepath{clip}%
\pgfsetbuttcap%
\pgfsetroundjoin%
\pgfsetlinewidth{0.501875pt}%
\definecolor{currentstroke}{rgb}{0.280894,0.078907,0.402329}%
\pgfsetstrokecolor{currentstroke}%
\pgfsetdash{}{0pt}%
\pgfpathmoveto{\pgfqpoint{4.475344in}{2.403676in}}%
\pgfpathlineto{\pgfqpoint{4.431402in}{2.385142in}}%
\pgfusepath{stroke}%
\end{pgfscope}%
\begin{pgfscope}%
\pgfpathrectangle{\pgfqpoint{3.352233in}{1.400000in}}{\pgfqpoint{2.407767in}{1.544118in}}%
\pgfusepath{clip}%
\pgfsetbuttcap%
\pgfsetroundjoin%
\pgfsetlinewidth{0.501875pt}%
\definecolor{currentstroke}{rgb}{0.280894,0.078907,0.402329}%
\pgfsetstrokecolor{currentstroke}%
\pgfsetdash{}{0pt}%
\pgfpathmoveto{\pgfqpoint{4.431402in}{2.385142in}}%
\pgfpathlineto{\pgfqpoint{4.389915in}{2.364295in}}%
\pgfusepath{stroke}%
\end{pgfscope}%
\begin{pgfscope}%
\pgfpathrectangle{\pgfqpoint{3.352233in}{1.400000in}}{\pgfqpoint{2.407767in}{1.544118in}}%
\pgfusepath{clip}%
\pgfsetbuttcap%
\pgfsetroundjoin%
\pgfsetlinewidth{0.501875pt}%
\definecolor{currentstroke}{rgb}{0.269944,0.014625,0.341379}%
\pgfsetstrokecolor{currentstroke}%
\pgfsetdash{}{0pt}%
\pgfpathmoveto{\pgfqpoint{5.206279in}{2.484774in}}%
\pgfpathlineto{\pgfqpoint{5.153321in}{2.484311in}}%
\pgfusepath{stroke}%
\end{pgfscope}%
\begin{pgfscope}%
\pgfpathrectangle{\pgfqpoint{3.352233in}{1.400000in}}{\pgfqpoint{2.407767in}{1.544118in}}%
\pgfusepath{clip}%
\pgfsetbuttcap%
\pgfsetroundjoin%
\pgfsetlinewidth{0.501875pt}%
\definecolor{currentstroke}{rgb}{0.271305,0.019942,0.347269}%
\pgfsetstrokecolor{currentstroke}%
\pgfsetdash{}{0pt}%
\pgfpathmoveto{\pgfqpoint{5.153321in}{2.484311in}}%
\pgfpathlineto{\pgfqpoint{5.100348in}{2.484021in}}%
\pgfusepath{stroke}%
\end{pgfscope}%
\begin{pgfscope}%
\pgfpathrectangle{\pgfqpoint{3.352233in}{1.400000in}}{\pgfqpoint{2.407767in}{1.544118in}}%
\pgfusepath{clip}%
\pgfsetbuttcap%
\pgfsetroundjoin%
\pgfsetlinewidth{0.501875pt}%
\definecolor{currentstroke}{rgb}{0.274952,0.037752,0.364543}%
\pgfsetstrokecolor{currentstroke}%
\pgfsetdash{}{0pt}%
\pgfpathmoveto{\pgfqpoint{5.100348in}{2.484021in}}%
\pgfpathlineto{\pgfqpoint{5.047376in}{2.483852in}}%
\pgfusepath{stroke}%
\end{pgfscope}%
\begin{pgfscope}%
\pgfpathrectangle{\pgfqpoint{3.352233in}{1.400000in}}{\pgfqpoint{2.407767in}{1.544118in}}%
\pgfusepath{clip}%
\pgfsetbuttcap%
\pgfsetroundjoin%
\pgfsetlinewidth{0.501875pt}%
\definecolor{currentstroke}{rgb}{0.276022,0.044167,0.370164}%
\pgfsetstrokecolor{currentstroke}%
\pgfsetdash{}{0pt}%
\pgfpathmoveto{\pgfqpoint{5.047376in}{2.483852in}}%
\pgfpathlineto{\pgfqpoint{4.994408in}{2.483534in}}%
\pgfusepath{stroke}%
\end{pgfscope}%
\begin{pgfscope}%
\pgfpathrectangle{\pgfqpoint{3.352233in}{1.400000in}}{\pgfqpoint{2.407767in}{1.544118in}}%
\pgfusepath{clip}%
\pgfsetbuttcap%
\pgfsetroundjoin%
\pgfsetlinewidth{0.501875pt}%
\definecolor{currentstroke}{rgb}{0.277941,0.056324,0.381191}%
\pgfsetstrokecolor{currentstroke}%
\pgfsetdash{}{0pt}%
\pgfpathmoveto{\pgfqpoint{4.994408in}{2.483534in}}%
\pgfpathlineto{\pgfqpoint{4.941453in}{2.482675in}}%
\pgfusepath{stroke}%
\end{pgfscope}%
\begin{pgfscope}%
\pgfpathrectangle{\pgfqpoint{3.352233in}{1.400000in}}{\pgfqpoint{2.407767in}{1.544118in}}%
\pgfusepath{clip}%
\pgfsetbuttcap%
\pgfsetroundjoin%
\pgfsetlinewidth{0.501875pt}%
\definecolor{currentstroke}{rgb}{0.278791,0.062145,0.386592}%
\pgfsetstrokecolor{currentstroke}%
\pgfsetdash{}{0pt}%
\pgfpathmoveto{\pgfqpoint{4.941453in}{2.482675in}}%
\pgfpathlineto{\pgfqpoint{4.888542in}{2.481087in}}%
\pgfusepath{stroke}%
\end{pgfscope}%
\begin{pgfscope}%
\pgfpathrectangle{\pgfqpoint{3.352233in}{1.400000in}}{\pgfqpoint{2.407767in}{1.544118in}}%
\pgfusepath{clip}%
\pgfsetbuttcap%
\pgfsetroundjoin%
\pgfsetlinewidth{0.501875pt}%
\definecolor{currentstroke}{rgb}{0.279566,0.067836,0.391917}%
\pgfsetstrokecolor{currentstroke}%
\pgfsetdash{}{0pt}%
\pgfpathmoveto{\pgfqpoint{4.888542in}{2.481087in}}%
\pgfpathlineto{\pgfqpoint{4.835679in}{2.478883in}}%
\pgfusepath{stroke}%
\end{pgfscope}%
\begin{pgfscope}%
\pgfpathrectangle{\pgfqpoint{3.352233in}{1.400000in}}{\pgfqpoint{2.407767in}{1.544118in}}%
\pgfusepath{clip}%
\pgfsetbuttcap%
\pgfsetroundjoin%
\pgfsetlinewidth{0.501875pt}%
\definecolor{currentstroke}{rgb}{0.277941,0.056324,0.381191}%
\pgfsetstrokecolor{currentstroke}%
\pgfsetdash{}{0pt}%
\pgfpathmoveto{\pgfqpoint{4.835679in}{2.478883in}}%
\pgfpathlineto{\pgfqpoint{4.782820in}{2.476646in}}%
\pgfusepath{stroke}%
\end{pgfscope}%
\begin{pgfscope}%
\pgfpathrectangle{\pgfqpoint{3.352233in}{1.400000in}}{\pgfqpoint{2.407767in}{1.544118in}}%
\pgfusepath{clip}%
\pgfsetbuttcap%
\pgfsetroundjoin%
\pgfsetlinewidth{0.501875pt}%
\definecolor{currentstroke}{rgb}{0.279566,0.067836,0.391917}%
\pgfsetstrokecolor{currentstroke}%
\pgfsetdash{}{0pt}%
\pgfpathmoveto{\pgfqpoint{4.782820in}{2.476646in}}%
\pgfpathlineto{\pgfqpoint{4.730074in}{2.473741in}}%
\pgfusepath{stroke}%
\end{pgfscope}%
\begin{pgfscope}%
\pgfpathrectangle{\pgfqpoint{3.352233in}{1.400000in}}{\pgfqpoint{2.407767in}{1.544118in}}%
\pgfusepath{clip}%
\pgfsetbuttcap%
\pgfsetroundjoin%
\pgfsetlinewidth{0.501875pt}%
\definecolor{currentstroke}{rgb}{0.279566,0.067836,0.391917}%
\pgfsetstrokecolor{currentstroke}%
\pgfsetdash{}{0pt}%
\pgfpathmoveto{\pgfqpoint{4.730074in}{2.473741in}}%
\pgfpathlineto{\pgfqpoint{4.677700in}{2.468894in}}%
\pgfusepath{stroke}%
\end{pgfscope}%
\begin{pgfscope}%
\pgfpathrectangle{\pgfqpoint{3.352233in}{1.400000in}}{\pgfqpoint{2.407767in}{1.544118in}}%
\pgfusepath{clip}%
\pgfsetbuttcap%
\pgfsetroundjoin%
\pgfsetlinewidth{0.501875pt}%
\definecolor{currentstroke}{rgb}{0.269944,0.014625,0.341379}%
\pgfsetstrokecolor{currentstroke}%
\pgfsetdash{}{0pt}%
\pgfpathmoveto{\pgfqpoint{5.206279in}{2.519520in}}%
\pgfpathlineto{\pgfqpoint{5.153353in}{2.519879in}}%
\pgfusepath{stroke}%
\end{pgfscope}%
\begin{pgfscope}%
\pgfpathrectangle{\pgfqpoint{3.352233in}{1.400000in}}{\pgfqpoint{2.407767in}{1.544118in}}%
\pgfusepath{clip}%
\pgfsetbuttcap%
\pgfsetroundjoin%
\pgfsetlinewidth{0.501875pt}%
\definecolor{currentstroke}{rgb}{0.272594,0.025563,0.353093}%
\pgfsetstrokecolor{currentstroke}%
\pgfsetdash{}{0pt}%
\pgfpathmoveto{\pgfqpoint{5.153353in}{2.519879in}}%
\pgfpathlineto{\pgfqpoint{5.100380in}{2.519702in}}%
\pgfusepath{stroke}%
\end{pgfscope}%
\begin{pgfscope}%
\pgfpathrectangle{\pgfqpoint{3.352233in}{1.400000in}}{\pgfqpoint{2.407767in}{1.544118in}}%
\pgfusepath{clip}%
\pgfsetbuttcap%
\pgfsetroundjoin%
\pgfsetlinewidth{0.501875pt}%
\definecolor{currentstroke}{rgb}{0.274952,0.037752,0.364543}%
\pgfsetstrokecolor{currentstroke}%
\pgfsetdash{}{0pt}%
\pgfpathmoveto{\pgfqpoint{5.100380in}{2.519702in}}%
\pgfpathlineto{\pgfqpoint{5.047410in}{2.519270in}}%
\pgfusepath{stroke}%
\end{pgfscope}%
\begin{pgfscope}%
\pgfpathrectangle{\pgfqpoint{3.352233in}{1.400000in}}{\pgfqpoint{2.407767in}{1.544118in}}%
\pgfusepath{clip}%
\pgfsetbuttcap%
\pgfsetroundjoin%
\pgfsetlinewidth{0.501875pt}%
\definecolor{currentstroke}{rgb}{0.274952,0.037752,0.364543}%
\pgfsetstrokecolor{currentstroke}%
\pgfsetdash{}{0pt}%
\pgfpathmoveto{\pgfqpoint{5.047410in}{2.519270in}}%
\pgfpathlineto{\pgfqpoint{4.994444in}{2.518679in}}%
\pgfusepath{stroke}%
\end{pgfscope}%
\begin{pgfscope}%
\pgfpathrectangle{\pgfqpoint{3.352233in}{1.400000in}}{\pgfqpoint{2.407767in}{1.544118in}}%
\pgfusepath{clip}%
\pgfsetbuttcap%
\pgfsetroundjoin%
\pgfsetlinewidth{0.501875pt}%
\definecolor{currentstroke}{rgb}{0.277018,0.050344,0.375715}%
\pgfsetstrokecolor{currentstroke}%
\pgfsetdash{}{0pt}%
\pgfpathmoveto{\pgfqpoint{4.994444in}{2.518679in}}%
\pgfpathlineto{\pgfqpoint{4.941480in}{2.518017in}}%
\pgfusepath{stroke}%
\end{pgfscope}%
\begin{pgfscope}%
\pgfpathrectangle{\pgfqpoint{3.352233in}{1.400000in}}{\pgfqpoint{2.407767in}{1.544118in}}%
\pgfusepath{clip}%
\pgfsetbuttcap%
\pgfsetroundjoin%
\pgfsetlinewidth{0.501875pt}%
\definecolor{currentstroke}{rgb}{0.278791,0.062145,0.386592}%
\pgfsetstrokecolor{currentstroke}%
\pgfsetdash{}{0pt}%
\pgfpathmoveto{\pgfqpoint{4.941480in}{2.518017in}}%
\pgfpathlineto{\pgfqpoint{4.888548in}{2.516761in}}%
\pgfusepath{stroke}%
\end{pgfscope}%
\begin{pgfscope}%
\pgfpathrectangle{\pgfqpoint{3.352233in}{1.400000in}}{\pgfqpoint{2.407767in}{1.544118in}}%
\pgfusepath{clip}%
\pgfsetbuttcap%
\pgfsetroundjoin%
\pgfsetlinewidth{0.501875pt}%
\definecolor{currentstroke}{rgb}{0.279566,0.067836,0.391917}%
\pgfsetstrokecolor{currentstroke}%
\pgfsetdash{}{0pt}%
\pgfpathmoveto{\pgfqpoint{4.888548in}{2.516761in}}%
\pgfpathlineto{\pgfqpoint{4.835670in}{2.514743in}}%
\pgfusepath{stroke}%
\end{pgfscope}%
\begin{pgfscope}%
\pgfpathrectangle{\pgfqpoint{3.352233in}{1.400000in}}{\pgfqpoint{2.407767in}{1.544118in}}%
\pgfusepath{clip}%
\pgfsetbuttcap%
\pgfsetroundjoin%
\pgfsetlinewidth{0.501875pt}%
\definecolor{currentstroke}{rgb}{0.278791,0.062145,0.386592}%
\pgfsetstrokecolor{currentstroke}%
\pgfsetdash{}{0pt}%
\pgfpathmoveto{\pgfqpoint{4.835670in}{2.514743in}}%
\pgfpathlineto{\pgfqpoint{4.782839in}{2.512377in}}%
\pgfusepath{stroke}%
\end{pgfscope}%
\begin{pgfscope}%
\pgfpathrectangle{\pgfqpoint{3.352233in}{1.400000in}}{\pgfqpoint{2.407767in}{1.544118in}}%
\pgfusepath{clip}%
\pgfsetbuttcap%
\pgfsetroundjoin%
\pgfsetlinewidth{0.501875pt}%
\definecolor{currentstroke}{rgb}{0.279566,0.067836,0.391917}%
\pgfsetstrokecolor{currentstroke}%
\pgfsetdash{}{0pt}%
\pgfpathmoveto{\pgfqpoint{4.782839in}{2.512377in}}%
\pgfpathlineto{\pgfqpoint{4.730124in}{2.509194in}}%
\pgfusepath{stroke}%
\end{pgfscope}%
\begin{pgfscope}%
\pgfpathrectangle{\pgfqpoint{3.352233in}{1.400000in}}{\pgfqpoint{2.407767in}{1.544118in}}%
\pgfusepath{clip}%
\pgfsetbuttcap%
\pgfsetroundjoin%
\pgfsetlinewidth{0.501875pt}%
\definecolor{currentstroke}{rgb}{0.279566,0.067836,0.391917}%
\pgfsetstrokecolor{currentstroke}%
\pgfsetdash{}{0pt}%
\pgfpathmoveto{\pgfqpoint{4.730124in}{2.509194in}}%
\pgfpathlineto{\pgfqpoint{4.677759in}{2.504216in}}%
\pgfusepath{stroke}%
\end{pgfscope}%
\begin{pgfscope}%
\pgfpathrectangle{\pgfqpoint{3.352233in}{1.400000in}}{\pgfqpoint{2.407767in}{1.544118in}}%
\pgfusepath{clip}%
\pgfsetbuttcap%
\pgfsetroundjoin%
\pgfsetlinewidth{0.501875pt}%
\definecolor{currentstroke}{rgb}{0.277941,0.056324,0.381191}%
\pgfsetstrokecolor{currentstroke}%
\pgfsetdash{}{0pt}%
\pgfpathmoveto{\pgfqpoint{4.677759in}{2.504216in}}%
\pgfpathlineto{\pgfqpoint{4.625844in}{2.497602in}}%
\pgfusepath{stroke}%
\end{pgfscope}%
\begin{pgfscope}%
\pgfpathrectangle{\pgfqpoint{3.352233in}{1.400000in}}{\pgfqpoint{2.407767in}{1.544118in}}%
\pgfusepath{clip}%
\pgfsetbuttcap%
\pgfsetroundjoin%
\pgfsetlinewidth{0.501875pt}%
\definecolor{currentstroke}{rgb}{0.274952,0.037752,0.364543}%
\pgfsetstrokecolor{currentstroke}%
\pgfsetdash{}{0pt}%
\pgfpathmoveto{\pgfqpoint{4.625844in}{2.497602in}}%
\pgfpathlineto{\pgfqpoint{4.575295in}{2.487863in}}%
\pgfusepath{stroke}%
\end{pgfscope}%
\begin{pgfscope}%
\pgfpathrectangle{\pgfqpoint{3.352233in}{1.400000in}}{\pgfqpoint{2.407767in}{1.544118in}}%
\pgfusepath{clip}%
\pgfsetbuttcap%
\pgfsetroundjoin%
\pgfsetlinewidth{0.501875pt}%
\definecolor{currentstroke}{rgb}{0.277018,0.050344,0.375715}%
\pgfsetstrokecolor{currentstroke}%
\pgfsetdash{}{0pt}%
\pgfpathmoveto{\pgfqpoint{4.575295in}{2.487863in}}%
\pgfpathlineto{\pgfqpoint{4.526428in}{2.474854in}}%
\pgfusepath{stroke}%
\end{pgfscope}%
\begin{pgfscope}%
\pgfpathrectangle{\pgfqpoint{3.352233in}{1.400000in}}{\pgfqpoint{2.407767in}{1.544118in}}%
\pgfusepath{clip}%
\pgfsetbuttcap%
\pgfsetroundjoin%
\pgfsetlinewidth{0.501875pt}%
\definecolor{currentstroke}{rgb}{0.276022,0.044167,0.370164}%
\pgfsetstrokecolor{currentstroke}%
\pgfsetdash{}{0pt}%
\pgfpathmoveto{\pgfqpoint{4.526428in}{2.474854in}}%
\pgfpathlineto{\pgfqpoint{4.481508in}{2.457353in}}%
\pgfusepath{stroke}%
\end{pgfscope}%
\begin{pgfscope}%
\pgfpathrectangle{\pgfqpoint{3.352233in}{1.400000in}}{\pgfqpoint{2.407767in}{1.544118in}}%
\pgfusepath{clip}%
\pgfsetbuttcap%
\pgfsetroundjoin%
\pgfsetlinewidth{0.501875pt}%
\definecolor{currentstroke}{rgb}{0.268510,0.009605,0.335427}%
\pgfsetstrokecolor{currentstroke}%
\pgfsetdash{}{0pt}%
\pgfpathmoveto{\pgfqpoint{5.206279in}{2.554266in}}%
\pgfpathlineto{\pgfqpoint{5.153309in}{2.553816in}}%
\pgfusepath{stroke}%
\end{pgfscope}%
\begin{pgfscope}%
\pgfpathrectangle{\pgfqpoint{3.352233in}{1.400000in}}{\pgfqpoint{2.407767in}{1.544118in}}%
\pgfusepath{clip}%
\pgfsetbuttcap%
\pgfsetroundjoin%
\pgfsetlinewidth{0.501875pt}%
\definecolor{currentstroke}{rgb}{0.271305,0.019942,0.347269}%
\pgfsetstrokecolor{currentstroke}%
\pgfsetdash{}{0pt}%
\pgfpathmoveto{\pgfqpoint{5.153309in}{2.553816in}}%
\pgfpathlineto{\pgfqpoint{5.100335in}{2.553551in}}%
\pgfusepath{stroke}%
\end{pgfscope}%
\begin{pgfscope}%
\pgfpathrectangle{\pgfqpoint{3.352233in}{1.400000in}}{\pgfqpoint{2.407767in}{1.544118in}}%
\pgfusepath{clip}%
\pgfsetbuttcap%
\pgfsetroundjoin%
\pgfsetlinewidth{0.501875pt}%
\definecolor{currentstroke}{rgb}{0.273809,0.031497,0.358853}%
\pgfsetstrokecolor{currentstroke}%
\pgfsetdash{}{0pt}%
\pgfpathmoveto{\pgfqpoint{5.100335in}{2.553551in}}%
\pgfpathlineto{\pgfqpoint{5.047376in}{2.552852in}}%
\pgfusepath{stroke}%
\end{pgfscope}%
\begin{pgfscope}%
\pgfpathrectangle{\pgfqpoint{3.352233in}{1.400000in}}{\pgfqpoint{2.407767in}{1.544118in}}%
\pgfusepath{clip}%
\pgfsetbuttcap%
\pgfsetroundjoin%
\pgfsetlinewidth{0.501875pt}%
\definecolor{currentstroke}{rgb}{0.274952,0.037752,0.364543}%
\pgfsetstrokecolor{currentstroke}%
\pgfsetdash{}{0pt}%
\pgfpathmoveto{\pgfqpoint{5.047376in}{2.552852in}}%
\pgfpathlineto{\pgfqpoint{4.994420in}{2.551993in}}%
\pgfusepath{stroke}%
\end{pgfscope}%
\begin{pgfscope}%
\pgfpathrectangle{\pgfqpoint{3.352233in}{1.400000in}}{\pgfqpoint{2.407767in}{1.544118in}}%
\pgfusepath{clip}%
\pgfsetbuttcap%
\pgfsetroundjoin%
\pgfsetlinewidth{0.501875pt}%
\definecolor{currentstroke}{rgb}{0.274952,0.037752,0.364543}%
\pgfsetstrokecolor{currentstroke}%
\pgfsetdash{}{0pt}%
\pgfpathmoveto{\pgfqpoint{4.994420in}{2.551993in}}%
\pgfpathlineto{\pgfqpoint{4.941467in}{2.551076in}}%
\pgfusepath{stroke}%
\end{pgfscope}%
\begin{pgfscope}%
\pgfpathrectangle{\pgfqpoint{3.352233in}{1.400000in}}{\pgfqpoint{2.407767in}{1.544118in}}%
\pgfusepath{clip}%
\pgfsetbuttcap%
\pgfsetroundjoin%
\pgfsetlinewidth{0.501875pt}%
\definecolor{currentstroke}{rgb}{0.277941,0.056324,0.381191}%
\pgfsetstrokecolor{currentstroke}%
\pgfsetdash{}{0pt}%
\pgfpathmoveto{\pgfqpoint{4.941467in}{2.551076in}}%
\pgfpathlineto{\pgfqpoint{4.888515in}{2.550130in}}%
\pgfusepath{stroke}%
\end{pgfscope}%
\begin{pgfscope}%
\pgfpathrectangle{\pgfqpoint{3.352233in}{1.400000in}}{\pgfqpoint{2.407767in}{1.544118in}}%
\pgfusepath{clip}%
\pgfsetbuttcap%
\pgfsetroundjoin%
\pgfsetlinewidth{0.501875pt}%
\definecolor{currentstroke}{rgb}{0.278791,0.062145,0.386592}%
\pgfsetstrokecolor{currentstroke}%
\pgfsetdash{}{0pt}%
\pgfpathmoveto{\pgfqpoint{4.888515in}{2.550130in}}%
\pgfpathlineto{\pgfqpoint{4.835641in}{2.548392in}}%
\pgfusepath{stroke}%
\end{pgfscope}%
\begin{pgfscope}%
\pgfpathrectangle{\pgfqpoint{3.352233in}{1.400000in}}{\pgfqpoint{2.407767in}{1.544118in}}%
\pgfusepath{clip}%
\pgfsetbuttcap%
\pgfsetroundjoin%
\pgfsetlinewidth{0.501875pt}%
\definecolor{currentstroke}{rgb}{0.277018,0.050344,0.375715}%
\pgfsetstrokecolor{currentstroke}%
\pgfsetdash{}{0pt}%
\pgfpathmoveto{\pgfqpoint{4.835641in}{2.548392in}}%
\pgfpathlineto{\pgfqpoint{4.782917in}{2.545318in}}%
\pgfusepath{stroke}%
\end{pgfscope}%
\begin{pgfscope}%
\pgfpathrectangle{\pgfqpoint{3.352233in}{1.400000in}}{\pgfqpoint{2.407767in}{1.544118in}}%
\pgfusepath{clip}%
\pgfsetbuttcap%
\pgfsetroundjoin%
\pgfsetlinewidth{0.501875pt}%
\definecolor{currentstroke}{rgb}{0.277941,0.056324,0.381191}%
\pgfsetstrokecolor{currentstroke}%
\pgfsetdash{}{0pt}%
\pgfpathmoveto{\pgfqpoint{4.782917in}{2.545318in}}%
\pgfpathlineto{\pgfqpoint{4.730225in}{2.542027in}}%
\pgfusepath{stroke}%
\end{pgfscope}%
\begin{pgfscope}%
\pgfpathrectangle{\pgfqpoint{3.352233in}{1.400000in}}{\pgfqpoint{2.407767in}{1.544118in}}%
\pgfusepath{clip}%
\pgfsetbuttcap%
\pgfsetroundjoin%
\pgfsetlinewidth{0.501875pt}%
\definecolor{currentstroke}{rgb}{0.277941,0.056324,0.381191}%
\pgfsetstrokecolor{currentstroke}%
\pgfsetdash{}{0pt}%
\pgfpathmoveto{\pgfqpoint{4.730225in}{2.542027in}}%
\pgfpathlineto{\pgfqpoint{4.677762in}{2.537537in}}%
\pgfusepath{stroke}%
\end{pgfscope}%
\begin{pgfscope}%
\pgfpathrectangle{\pgfqpoint{3.352233in}{1.400000in}}{\pgfqpoint{2.407767in}{1.544118in}}%
\pgfusepath{clip}%
\pgfsetbuttcap%
\pgfsetroundjoin%
\pgfsetlinewidth{0.501875pt}%
\definecolor{currentstroke}{rgb}{0.277018,0.050344,0.375715}%
\pgfsetstrokecolor{currentstroke}%
\pgfsetdash{}{0pt}%
\pgfpathmoveto{\pgfqpoint{4.677762in}{2.537537in}}%
\pgfpathlineto{\pgfqpoint{4.625699in}{2.531407in}}%
\pgfusepath{stroke}%
\end{pgfscope}%
\begin{pgfscope}%
\pgfpathrectangle{\pgfqpoint{3.352233in}{1.400000in}}{\pgfqpoint{2.407767in}{1.544118in}}%
\pgfusepath{clip}%
\pgfsetbuttcap%
\pgfsetroundjoin%
\pgfsetlinewidth{0.501875pt}%
\definecolor{currentstroke}{rgb}{0.276022,0.044167,0.370164}%
\pgfsetstrokecolor{currentstroke}%
\pgfsetdash{}{0pt}%
\pgfpathmoveto{\pgfqpoint{4.625699in}{2.531407in}}%
\pgfpathlineto{\pgfqpoint{4.574875in}{2.522278in}}%
\pgfusepath{stroke}%
\end{pgfscope}%
\begin{pgfscope}%
\pgfpathrectangle{\pgfqpoint{3.352233in}{1.400000in}}{\pgfqpoint{2.407767in}{1.544118in}}%
\pgfusepath{clip}%
\pgfsetbuttcap%
\pgfsetroundjoin%
\pgfsetlinewidth{0.501875pt}%
\definecolor{currentstroke}{rgb}{0.269944,0.014625,0.341379}%
\pgfsetstrokecolor{currentstroke}%
\pgfsetdash{}{0pt}%
\pgfpathmoveto{\pgfqpoint{5.206279in}{2.589012in}}%
\pgfpathlineto{\pgfqpoint{5.153325in}{2.588286in}}%
\pgfusepath{stroke}%
\end{pgfscope}%
\begin{pgfscope}%
\pgfpathrectangle{\pgfqpoint{3.352233in}{1.400000in}}{\pgfqpoint{2.407767in}{1.544118in}}%
\pgfusepath{clip}%
\pgfsetbuttcap%
\pgfsetroundjoin%
\pgfsetlinewidth{0.501875pt}%
\definecolor{currentstroke}{rgb}{0.269944,0.014625,0.341379}%
\pgfsetstrokecolor{currentstroke}%
\pgfsetdash{}{0pt}%
\pgfpathmoveto{\pgfqpoint{5.153325in}{2.588286in}}%
\pgfpathlineto{\pgfqpoint{5.100374in}{2.587767in}}%
\pgfusepath{stroke}%
\end{pgfscope}%
\begin{pgfscope}%
\pgfpathrectangle{\pgfqpoint{3.352233in}{1.400000in}}{\pgfqpoint{2.407767in}{1.544118in}}%
\pgfusepath{clip}%
\pgfsetbuttcap%
\pgfsetroundjoin%
\pgfsetlinewidth{0.501875pt}%
\definecolor{currentstroke}{rgb}{0.272594,0.025563,0.353093}%
\pgfsetstrokecolor{currentstroke}%
\pgfsetdash{}{0pt}%
\pgfpathmoveto{\pgfqpoint{5.100374in}{2.587767in}}%
\pgfpathlineto{\pgfqpoint{5.047413in}{2.587465in}}%
\pgfusepath{stroke}%
\end{pgfscope}%
\begin{pgfscope}%
\pgfpathrectangle{\pgfqpoint{3.352233in}{1.400000in}}{\pgfqpoint{2.407767in}{1.544118in}}%
\pgfusepath{clip}%
\pgfsetbuttcap%
\pgfsetroundjoin%
\pgfsetlinewidth{0.501875pt}%
\definecolor{currentstroke}{rgb}{0.273809,0.031497,0.358853}%
\pgfsetstrokecolor{currentstroke}%
\pgfsetdash{}{0pt}%
\pgfpathmoveto{\pgfqpoint{5.047413in}{2.587465in}}%
\pgfpathlineto{\pgfqpoint{4.994450in}{2.586952in}}%
\pgfusepath{stroke}%
\end{pgfscope}%
\begin{pgfscope}%
\pgfpathrectangle{\pgfqpoint{3.352233in}{1.400000in}}{\pgfqpoint{2.407767in}{1.544118in}}%
\pgfusepath{clip}%
\pgfsetbuttcap%
\pgfsetroundjoin%
\pgfsetlinewidth{0.501875pt}%
\definecolor{currentstroke}{rgb}{0.274952,0.037752,0.364543}%
\pgfsetstrokecolor{currentstroke}%
\pgfsetdash{}{0pt}%
\pgfpathmoveto{\pgfqpoint{4.994450in}{2.586952in}}%
\pgfpathlineto{\pgfqpoint{4.941521in}{2.585899in}}%
\pgfusepath{stroke}%
\end{pgfscope}%
\begin{pgfscope}%
\pgfpathrectangle{\pgfqpoint{3.352233in}{1.400000in}}{\pgfqpoint{2.407767in}{1.544118in}}%
\pgfusepath{clip}%
\pgfsetbuttcap%
\pgfsetroundjoin%
\pgfsetlinewidth{0.501875pt}%
\definecolor{currentstroke}{rgb}{0.274952,0.037752,0.364543}%
\pgfsetstrokecolor{currentstroke}%
\pgfsetdash{}{0pt}%
\pgfpathmoveto{\pgfqpoint{4.941521in}{2.585899in}}%
\pgfpathlineto{\pgfqpoint{4.888788in}{2.582956in}}%
\pgfusepath{stroke}%
\end{pgfscope}%
\begin{pgfscope}%
\pgfpathrectangle{\pgfqpoint{3.352233in}{1.400000in}}{\pgfqpoint{2.407767in}{1.544118in}}%
\pgfusepath{clip}%
\pgfsetbuttcap%
\pgfsetroundjoin%
\pgfsetlinewidth{0.501875pt}%
\definecolor{currentstroke}{rgb}{0.279566,0.067836,0.391917}%
\pgfsetstrokecolor{currentstroke}%
\pgfsetdash{}{0pt}%
\pgfpathmoveto{\pgfqpoint{4.888788in}{2.582956in}}%
\pgfpathlineto{\pgfqpoint{4.836050in}{2.580139in}}%
\pgfusepath{stroke}%
\end{pgfscope}%
\begin{pgfscope}%
\pgfpathrectangle{\pgfqpoint{3.352233in}{1.400000in}}{\pgfqpoint{2.407767in}{1.544118in}}%
\pgfusepath{clip}%
\pgfsetbuttcap%
\pgfsetroundjoin%
\pgfsetlinewidth{0.501875pt}%
\definecolor{currentstroke}{rgb}{0.278791,0.062145,0.386592}%
\pgfsetstrokecolor{currentstroke}%
\pgfsetdash{}{0pt}%
\pgfpathmoveto{\pgfqpoint{4.836050in}{2.580139in}}%
\pgfpathlineto{\pgfqpoint{4.783181in}{2.578256in}}%
\pgfusepath{stroke}%
\end{pgfscope}%
\begin{pgfscope}%
\pgfpathrectangle{\pgfqpoint{3.352233in}{1.400000in}}{\pgfqpoint{2.407767in}{1.544118in}}%
\pgfusepath{clip}%
\pgfsetbuttcap%
\pgfsetroundjoin%
\pgfsetlinewidth{0.501875pt}%
\definecolor{currentstroke}{rgb}{0.274952,0.037752,0.364543}%
\pgfsetstrokecolor{currentstroke}%
\pgfsetdash{}{0pt}%
\pgfpathmoveto{\pgfqpoint{4.783181in}{2.578256in}}%
\pgfpathlineto{\pgfqpoint{4.730499in}{2.575030in}}%
\pgfusepath{stroke}%
\end{pgfscope}%
\begin{pgfscope}%
\pgfpathrectangle{\pgfqpoint{3.352233in}{1.400000in}}{\pgfqpoint{2.407767in}{1.544118in}}%
\pgfusepath{clip}%
\pgfsetbuttcap%
\pgfsetroundjoin%
\pgfsetlinewidth{0.501875pt}%
\definecolor{currentstroke}{rgb}{0.276022,0.044167,0.370164}%
\pgfsetstrokecolor{currentstroke}%
\pgfsetdash{}{0pt}%
\pgfpathmoveto{\pgfqpoint{4.132067in}{2.615053in}}%
\pgfpathlineto{\pgfqpoint{4.183858in}{2.608118in}}%
\pgfusepath{stroke}%
\end{pgfscope}%
\begin{pgfscope}%
\pgfpathrectangle{\pgfqpoint{3.352233in}{1.400000in}}{\pgfqpoint{2.407767in}{1.544118in}}%
\pgfusepath{clip}%
\pgfsetbuttcap%
\pgfsetroundjoin%
\pgfsetlinewidth{0.501875pt}%
\definecolor{currentstroke}{rgb}{0.276022,0.044167,0.370164}%
\pgfsetstrokecolor{currentstroke}%
\pgfsetdash{}{0pt}%
\pgfpathmoveto{\pgfqpoint{4.183858in}{2.608118in}}%
\pgfpathlineto{\pgfqpoint{4.234919in}{2.599457in}}%
\pgfusepath{stroke}%
\end{pgfscope}%
\begin{pgfscope}%
\pgfpathrectangle{\pgfqpoint{3.352233in}{1.400000in}}{\pgfqpoint{2.407767in}{1.544118in}}%
\pgfusepath{clip}%
\pgfsetbuttcap%
\pgfsetroundjoin%
\pgfsetlinewidth{0.501875pt}%
\definecolor{currentstroke}{rgb}{0.274952,0.037752,0.364543}%
\pgfsetstrokecolor{currentstroke}%
\pgfsetdash{}{0pt}%
\pgfpathmoveto{\pgfqpoint{4.234919in}{2.599457in}}%
\pgfpathlineto{\pgfqpoint{4.285216in}{2.589012in}}%
\pgfusepath{stroke}%
\end{pgfscope}%
\begin{pgfscope}%
\pgfpathrectangle{\pgfqpoint{3.352233in}{1.400000in}}{\pgfqpoint{2.407767in}{1.544118in}}%
\pgfusepath{clip}%
\pgfsetbuttcap%
\pgfsetroundjoin%
\pgfsetlinewidth{0.501875pt}%
\definecolor{currentstroke}{rgb}{0.276022,0.044167,0.370164}%
\pgfsetstrokecolor{currentstroke}%
\pgfsetdash{}{0pt}%
\pgfpathmoveto{\pgfqpoint{4.285216in}{2.589012in}}%
\pgfpathlineto{\pgfqpoint{4.334185in}{2.576654in}}%
\pgfusepath{stroke}%
\end{pgfscope}%
\begin{pgfscope}%
\pgfpathrectangle{\pgfqpoint{3.352233in}{1.400000in}}{\pgfqpoint{2.407767in}{1.544118in}}%
\pgfusepath{clip}%
\pgfsetbuttcap%
\pgfsetroundjoin%
\pgfsetlinewidth{0.501875pt}%
\definecolor{currentstroke}{rgb}{0.272594,0.025563,0.353093}%
\pgfsetstrokecolor{currentstroke}%
\pgfsetdash{}{0pt}%
\pgfpathmoveto{\pgfqpoint{4.334185in}{2.576654in}}%
\pgfpathlineto{\pgfqpoint{4.378195in}{2.559997in}}%
\pgfusepath{stroke}%
\end{pgfscope}%
\begin{pgfscope}%
\pgfpathrectangle{\pgfqpoint{3.352233in}{1.400000in}}{\pgfqpoint{2.407767in}{1.544118in}}%
\pgfusepath{clip}%
\pgfsetbuttcap%
\pgfsetroundjoin%
\pgfsetlinewidth{0.501875pt}%
\definecolor{currentstroke}{rgb}{0.271305,0.019942,0.347269}%
\pgfsetstrokecolor{currentstroke}%
\pgfsetdash{}{0pt}%
\pgfpathmoveto{\pgfqpoint{4.378195in}{2.559997in}}%
\pgfpathlineto{\pgfqpoint{4.415321in}{2.538224in}}%
\pgfusepath{stroke}%
\end{pgfscope}%
\begin{pgfscope}%
\pgfpathrectangle{\pgfqpoint{3.352233in}{1.400000in}}{\pgfqpoint{2.407767in}{1.544118in}}%
\pgfusepath{clip}%
\pgfsetbuttcap%
\pgfsetroundjoin%
\pgfsetlinewidth{0.501875pt}%
\definecolor{currentstroke}{rgb}{0.273809,0.031497,0.358853}%
\pgfsetstrokecolor{currentstroke}%
\pgfsetdash{}{0pt}%
\pgfpathmoveto{\pgfqpoint{4.415321in}{2.538224in}}%
\pgfpathlineto{\pgfqpoint{4.415321in}{2.538224in}}%
\pgfusepath{stroke}%
\end{pgfscope}%
\begin{pgfscope}%
\pgfpathrectangle{\pgfqpoint{3.352233in}{1.400000in}}{\pgfqpoint{2.407767in}{1.544118in}}%
\pgfusepath{clip}%
\pgfsetbuttcap%
\pgfsetroundjoin%
\pgfsetlinewidth{0.501875pt}%
\definecolor{currentstroke}{rgb}{0.277941,0.056324,0.381191}%
\pgfsetstrokecolor{currentstroke}%
\pgfsetdash{}{0pt}%
\pgfpathmoveto{\pgfqpoint{4.176855in}{2.554266in}}%
\pgfpathlineto{\pgfqpoint{4.228482in}{2.547000in}}%
\pgfusepath{stroke}%
\end{pgfscope}%
\begin{pgfscope}%
\pgfpathrectangle{\pgfqpoint{3.352233in}{1.400000in}}{\pgfqpoint{2.407767in}{1.544118in}}%
\pgfusepath{clip}%
\pgfsetbuttcap%
\pgfsetroundjoin%
\pgfsetlinewidth{0.501875pt}%
\definecolor{currentstroke}{rgb}{0.277941,0.056324,0.381191}%
\pgfsetstrokecolor{currentstroke}%
\pgfsetdash{}{0pt}%
\pgfpathmoveto{\pgfqpoint{4.228482in}{2.547000in}}%
\pgfpathlineto{\pgfqpoint{4.280518in}{2.542365in}}%
\pgfusepath{stroke}%
\end{pgfscope}%
\begin{pgfscope}%
\pgfpathrectangle{\pgfqpoint{3.352233in}{1.400000in}}{\pgfqpoint{2.407767in}{1.544118in}}%
\pgfusepath{clip}%
\pgfsetbuttcap%
\pgfsetroundjoin%
\pgfsetlinewidth{0.501875pt}%
\definecolor{currentstroke}{rgb}{0.273809,0.031497,0.358853}%
\pgfsetstrokecolor{currentstroke}%
\pgfsetdash{}{0pt}%
\pgfpathmoveto{\pgfqpoint{4.280518in}{2.542365in}}%
\pgfpathlineto{\pgfqpoint{4.280518in}{2.542365in}}%
\pgfusepath{stroke}%
\end{pgfscope}%
\begin{pgfscope}%
\pgfpathrectangle{\pgfqpoint{3.352233in}{1.400000in}}{\pgfqpoint{2.407767in}{1.544118in}}%
\pgfusepath{clip}%
\pgfsetbuttcap%
\pgfsetroundjoin%
\pgfsetlinewidth{0.501875pt}%
\definecolor{currentstroke}{rgb}{0.273809,0.031497,0.358853}%
\pgfsetstrokecolor{currentstroke}%
\pgfsetdash{}{0pt}%
\pgfpathmoveto{\pgfqpoint{4.280518in}{2.542365in}}%
\pgfpathlineto{\pgfqpoint{4.280518in}{2.542365in}}%
\pgfusepath{stroke}%
\end{pgfscope}%
\begin{pgfscope}%
\pgfpathrectangle{\pgfqpoint{3.352233in}{1.400000in}}{\pgfqpoint{2.407767in}{1.544118in}}%
\pgfusepath{clip}%
\pgfsetbuttcap%
\pgfsetroundjoin%
\pgfsetlinewidth{0.501875pt}%
\definecolor{currentstroke}{rgb}{0.273809,0.031497,0.358853}%
\pgfsetstrokecolor{currentstroke}%
\pgfsetdash{}{0pt}%
\pgfpathmoveto{\pgfqpoint{4.280518in}{2.542365in}}%
\pgfpathlineto{\pgfqpoint{4.303623in}{2.538863in}}%
\pgfusepath{stroke}%
\end{pgfscope}%
\begin{pgfscope}%
\pgfpathrectangle{\pgfqpoint{3.352233in}{1.400000in}}{\pgfqpoint{2.407767in}{1.544118in}}%
\pgfusepath{clip}%
\pgfsetbuttcap%
\pgfsetroundjoin%
\pgfsetlinewidth{0.501875pt}%
\definecolor{currentstroke}{rgb}{0.274952,0.037752,0.364543}%
\pgfsetstrokecolor{currentstroke}%
\pgfsetdash{}{0pt}%
\pgfpathmoveto{\pgfqpoint{4.303623in}{2.538863in}}%
\pgfpathlineto{\pgfqpoint{4.328656in}{2.531187in}}%
\pgfusepath{stroke}%
\end{pgfscope}%
\begin{pgfscope}%
\pgfpathrectangle{\pgfqpoint{3.352233in}{1.400000in}}{\pgfqpoint{2.407767in}{1.544118in}}%
\pgfusepath{clip}%
\pgfsetbuttcap%
\pgfsetroundjoin%
\pgfsetlinewidth{0.501875pt}%
\definecolor{currentstroke}{rgb}{0.277941,0.056324,0.381191}%
\pgfsetstrokecolor{currentstroke}%
\pgfsetdash{}{0pt}%
\pgfpathmoveto{\pgfqpoint{4.328656in}{2.531187in}}%
\pgfpathlineto{\pgfqpoint{4.328656in}{2.531187in}}%
\pgfusepath{stroke}%
\end{pgfscope}%
\begin{pgfscope}%
\pgfpathrectangle{\pgfqpoint{3.352233in}{1.400000in}}{\pgfqpoint{2.407767in}{1.544118in}}%
\pgfusepath{clip}%
\pgfsetbuttcap%
\pgfsetroundjoin%
\pgfsetlinewidth{0.501875pt}%
\definecolor{currentstroke}{rgb}{0.277941,0.056324,0.381191}%
\pgfsetstrokecolor{currentstroke}%
\pgfsetdash{}{0pt}%
\pgfpathmoveto{\pgfqpoint{4.328656in}{2.531187in}}%
\pgfpathlineto{\pgfqpoint{4.358538in}{2.518850in}}%
\pgfusepath{stroke}%
\end{pgfscope}%
\begin{pgfscope}%
\pgfpathrectangle{\pgfqpoint{3.352233in}{1.400000in}}{\pgfqpoint{2.407767in}{1.544118in}}%
\pgfusepath{clip}%
\pgfsetbuttcap%
\pgfsetroundjoin%
\pgfsetlinewidth{0.501875pt}%
\definecolor{currentstroke}{rgb}{0.274952,0.037752,0.364543}%
\pgfsetstrokecolor{currentstroke}%
\pgfsetdash{}{0pt}%
\pgfpathmoveto{\pgfqpoint{4.358538in}{2.518850in}}%
\pgfpathlineto{\pgfqpoint{4.395673in}{2.499892in}}%
\pgfusepath{stroke}%
\end{pgfscope}%
\begin{pgfscope}%
\pgfpathrectangle{\pgfqpoint{3.352233in}{1.400000in}}{\pgfqpoint{2.407767in}{1.544118in}}%
\pgfusepath{clip}%
\pgfsetbuttcap%
\pgfsetroundjoin%
\pgfsetlinewidth{0.501875pt}%
\definecolor{currentstroke}{rgb}{0.277018,0.050344,0.375715}%
\pgfsetstrokecolor{currentstroke}%
\pgfsetdash{}{0pt}%
\pgfpathmoveto{\pgfqpoint{4.395673in}{2.499892in}}%
\pgfpathlineto{\pgfqpoint{4.395673in}{2.499892in}}%
\pgfusepath{stroke}%
\end{pgfscope}%
\begin{pgfscope}%
\pgfpathrectangle{\pgfqpoint{3.352233in}{1.400000in}}{\pgfqpoint{2.407767in}{1.544118in}}%
\pgfusepath{clip}%
\pgfsetbuttcap%
\pgfsetroundjoin%
\pgfsetlinewidth{0.501875pt}%
\definecolor{currentstroke}{rgb}{0.277018,0.050344,0.375715}%
\pgfsetstrokecolor{currentstroke}%
\pgfsetdash{}{0pt}%
\pgfpathmoveto{\pgfqpoint{4.395673in}{2.499892in}}%
\pgfpathlineto{\pgfqpoint{4.413079in}{2.479939in}}%
\pgfusepath{stroke}%
\end{pgfscope}%
\begin{pgfscope}%
\pgfpathrectangle{\pgfqpoint{3.352233in}{1.400000in}}{\pgfqpoint{2.407767in}{1.544118in}}%
\pgfusepath{clip}%
\pgfsetbuttcap%
\pgfsetroundjoin%
\pgfsetlinewidth{0.501875pt}%
\definecolor{currentstroke}{rgb}{0.276022,0.044167,0.370164}%
\pgfsetstrokecolor{currentstroke}%
\pgfsetdash{}{0pt}%
\pgfpathmoveto{\pgfqpoint{4.413079in}{2.479939in}}%
\pgfpathlineto{\pgfqpoint{4.413079in}{2.479939in}}%
\pgfusepath{stroke}%
\end{pgfscope}%
\begin{pgfscope}%
\pgfpathrectangle{\pgfqpoint{3.352233in}{1.400000in}}{\pgfqpoint{2.407767in}{1.544118in}}%
\pgfusepath{clip}%
\pgfsetbuttcap%
\pgfsetroundjoin%
\pgfsetlinewidth{0.501875pt}%
\definecolor{currentstroke}{rgb}{0.252194,0.269783,0.531579}%
\pgfsetstrokecolor{currentstroke}%
\pgfsetdash{}{0pt}%
\pgfpathmoveto{\pgfqpoint{3.856802in}{2.020222in}}%
\pgfpathlineto{\pgfqpoint{3.909476in}{2.023609in}}%
\pgfusepath{stroke}%
\end{pgfscope}%
\begin{pgfscope}%
\pgfpathrectangle{\pgfqpoint{3.352233in}{1.400000in}}{\pgfqpoint{2.407767in}{1.544118in}}%
\pgfusepath{clip}%
\pgfsetbuttcap%
\pgfsetroundjoin%
\pgfsetlinewidth{0.501875pt}%
\definecolor{currentstroke}{rgb}{0.257322,0.256130,0.526563}%
\pgfsetstrokecolor{currentstroke}%
\pgfsetdash{}{0pt}%
\pgfpathmoveto{\pgfqpoint{3.909476in}{2.023609in}}%
\pgfpathlineto{\pgfqpoint{3.962025in}{2.027787in}}%
\pgfusepath{stroke}%
\end{pgfscope}%
\begin{pgfscope}%
\pgfpathrectangle{\pgfqpoint{3.352233in}{1.400000in}}{\pgfqpoint{2.407767in}{1.544118in}}%
\pgfusepath{clip}%
\pgfsetbuttcap%
\pgfsetroundjoin%
\pgfsetlinewidth{0.501875pt}%
\definecolor{currentstroke}{rgb}{0.279574,0.170599,0.479997}%
\pgfsetstrokecolor{currentstroke}%
\pgfsetdash{}{0pt}%
\pgfpathmoveto{\pgfqpoint{3.962025in}{2.027787in}}%
\pgfpathlineto{\pgfqpoint{4.014315in}{2.033074in}}%
\pgfusepath{stroke}%
\end{pgfscope}%
\begin{pgfscope}%
\pgfpathrectangle{\pgfqpoint{3.352233in}{1.400000in}}{\pgfqpoint{2.407767in}{1.544118in}}%
\pgfusepath{clip}%
\pgfsetbuttcap%
\pgfsetroundjoin%
\pgfsetlinewidth{0.501875pt}%
\definecolor{currentstroke}{rgb}{0.282884,0.135920,0.453427}%
\pgfsetstrokecolor{currentstroke}%
\pgfsetdash{}{0pt}%
\pgfpathmoveto{\pgfqpoint{4.014315in}{2.033074in}}%
\pgfpathlineto{\pgfqpoint{4.065820in}{2.040644in}}%
\pgfusepath{stroke}%
\end{pgfscope}%
\begin{pgfscope}%
\pgfpathrectangle{\pgfqpoint{3.352233in}{1.400000in}}{\pgfqpoint{2.407767in}{1.544118in}}%
\pgfusepath{clip}%
\pgfsetbuttcap%
\pgfsetroundjoin%
\pgfsetlinewidth{0.501875pt}%
\definecolor{currentstroke}{rgb}{0.281924,0.089666,0.412415}%
\pgfsetstrokecolor{currentstroke}%
\pgfsetdash{}{0pt}%
\pgfpathmoveto{\pgfqpoint{4.065820in}{2.040644in}}%
\pgfpathlineto{\pgfqpoint{4.117051in}{2.048986in}}%
\pgfusepath{stroke}%
\end{pgfscope}%
\begin{pgfscope}%
\pgfpathrectangle{\pgfqpoint{3.352233in}{1.400000in}}{\pgfqpoint{2.407767in}{1.544118in}}%
\pgfusepath{clip}%
\pgfsetbuttcap%
\pgfsetroundjoin%
\pgfsetlinewidth{0.501875pt}%
\definecolor{currentstroke}{rgb}{0.283072,0.130895,0.449241}%
\pgfsetstrokecolor{currentstroke}%
\pgfsetdash{}{0pt}%
\pgfpathmoveto{\pgfqpoint{4.068495in}{1.859344in}}%
\pgfpathlineto{\pgfqpoint{4.115947in}{1.874153in}}%
\pgfusepath{stroke}%
\end{pgfscope}%
\begin{pgfscope}%
\pgfpathrectangle{\pgfqpoint{3.352233in}{1.400000in}}{\pgfqpoint{2.407767in}{1.544118in}}%
\pgfusepath{clip}%
\pgfsetbuttcap%
\pgfsetroundjoin%
\pgfsetlinewidth{0.501875pt}%
\definecolor{currentstroke}{rgb}{0.283197,0.115680,0.436115}%
\pgfsetstrokecolor{currentstroke}%
\pgfsetdash{}{0pt}%
\pgfpathmoveto{\pgfqpoint{4.115947in}{1.874153in}}%
\pgfpathlineto{\pgfqpoint{4.163004in}{1.889536in}}%
\pgfusepath{stroke}%
\end{pgfscope}%
\begin{pgfscope}%
\pgfpathrectangle{\pgfqpoint{3.352233in}{1.400000in}}{\pgfqpoint{2.407767in}{1.544118in}}%
\pgfusepath{clip}%
\pgfsetbuttcap%
\pgfsetroundjoin%
\pgfsetlinewidth{0.501875pt}%
\definecolor{currentstroke}{rgb}{0.282656,0.100196,0.422160}%
\pgfsetstrokecolor{currentstroke}%
\pgfsetdash{}{0pt}%
\pgfpathmoveto{\pgfqpoint{4.163004in}{1.889536in}}%
\pgfpathlineto{\pgfqpoint{4.207469in}{1.907246in}}%
\pgfusepath{stroke}%
\end{pgfscope}%
\begin{pgfscope}%
\pgfpathrectangle{\pgfqpoint{3.352233in}{1.400000in}}{\pgfqpoint{2.407767in}{1.544118in}}%
\pgfusepath{clip}%
\pgfsetbuttcap%
\pgfsetroundjoin%
\pgfsetlinewidth{0.501875pt}%
\definecolor{currentstroke}{rgb}{0.281924,0.089666,0.412415}%
\pgfsetstrokecolor{currentstroke}%
\pgfsetdash{}{0pt}%
\pgfpathmoveto{\pgfqpoint{4.207469in}{1.907246in}}%
\pgfpathlineto{\pgfqpoint{4.207469in}{1.907246in}}%
\pgfusepath{stroke}%
\end{pgfscope}%
\begin{pgfscope}%
\pgfpathrectangle{\pgfqpoint{3.352233in}{1.400000in}}{\pgfqpoint{2.407767in}{1.544118in}}%
\pgfusepath{clip}%
\pgfsetbuttcap%
\pgfsetroundjoin%
\pgfsetlinewidth{0.501875pt}%
\definecolor{currentstroke}{rgb}{0.281924,0.089666,0.412415}%
\pgfsetstrokecolor{currentstroke}%
\pgfsetdash{}{0pt}%
\pgfpathmoveto{\pgfqpoint{4.207469in}{1.907246in}}%
\pgfpathlineto{\pgfqpoint{4.233230in}{1.924274in}}%
\pgfusepath{stroke}%
\end{pgfscope}%
\begin{pgfscope}%
\pgfpathrectangle{\pgfqpoint{3.352233in}{1.400000in}}{\pgfqpoint{2.407767in}{1.544118in}}%
\pgfusepath{clip}%
\pgfsetbuttcap%
\pgfsetroundjoin%
\pgfsetlinewidth{0.501875pt}%
\definecolor{currentstroke}{rgb}{0.281924,0.089666,0.412415}%
\pgfsetstrokecolor{currentstroke}%
\pgfsetdash{}{0pt}%
\pgfpathmoveto{\pgfqpoint{4.233230in}{1.924274in}}%
\pgfpathlineto{\pgfqpoint{4.233230in}{1.924274in}}%
\pgfusepath{stroke}%
\end{pgfscope}%
\begin{pgfscope}%
\pgfpathrectangle{\pgfqpoint{3.352233in}{1.400000in}}{\pgfqpoint{2.407767in}{1.544118in}}%
\pgfusepath{clip}%
\pgfsetbuttcap%
\pgfsetroundjoin%
\pgfsetlinewidth{0.501875pt}%
\definecolor{currentstroke}{rgb}{0.281924,0.089666,0.412415}%
\pgfsetstrokecolor{currentstroke}%
\pgfsetdash{}{0pt}%
\pgfpathmoveto{\pgfqpoint{4.233230in}{1.924274in}}%
\pgfpathlineto{\pgfqpoint{4.246350in}{1.940058in}}%
\pgfusepath{stroke}%
\end{pgfscope}%
\begin{pgfscope}%
\pgfpathrectangle{\pgfqpoint{3.352233in}{1.400000in}}{\pgfqpoint{2.407767in}{1.544118in}}%
\pgfusepath{clip}%
\pgfsetbuttcap%
\pgfsetroundjoin%
\pgfsetlinewidth{0.501875pt}%
\definecolor{currentstroke}{rgb}{0.281446,0.084320,0.407414}%
\pgfsetstrokecolor{currentstroke}%
\pgfsetdash{}{0pt}%
\pgfpathmoveto{\pgfqpoint{4.246350in}{1.940058in}}%
\pgfpathlineto{\pgfqpoint{4.254397in}{1.956628in}}%
\pgfusepath{stroke}%
\end{pgfscope}%
\begin{pgfscope}%
\pgfpathrectangle{\pgfqpoint{3.352233in}{1.400000in}}{\pgfqpoint{2.407767in}{1.544118in}}%
\pgfusepath{clip}%
\pgfsetbuttcap%
\pgfsetroundjoin%
\pgfsetlinewidth{0.501875pt}%
\definecolor{currentstroke}{rgb}{0.280267,0.073417,0.397163}%
\pgfsetstrokecolor{currentstroke}%
\pgfsetdash{}{0pt}%
\pgfpathmoveto{\pgfqpoint{4.254397in}{1.956628in}}%
\pgfpathlineto{\pgfqpoint{4.254397in}{1.956628in}}%
\pgfusepath{stroke}%
\end{pgfscope}%
\begin{pgfscope}%
\pgfpathrectangle{\pgfqpoint{3.352233in}{1.400000in}}{\pgfqpoint{2.407767in}{1.544118in}}%
\pgfusepath{clip}%
\pgfsetbuttcap%
\pgfsetroundjoin%
\pgfsetlinewidth{0.501875pt}%
\definecolor{currentstroke}{rgb}{0.280267,0.073417,0.397163}%
\pgfsetstrokecolor{currentstroke}%
\pgfsetdash{}{0pt}%
\pgfpathmoveto{\pgfqpoint{4.254397in}{1.956628in}}%
\pgfpathlineto{\pgfqpoint{4.257972in}{1.973236in}}%
\pgfusepath{stroke}%
\end{pgfscope}%
\begin{pgfscope}%
\pgfpathrectangle{\pgfqpoint{3.352233in}{1.400000in}}{\pgfqpoint{2.407767in}{1.544118in}}%
\pgfusepath{clip}%
\pgfsetbuttcap%
\pgfsetroundjoin%
\pgfsetlinewidth{0.501875pt}%
\definecolor{currentstroke}{rgb}{0.281924,0.089666,0.412415}%
\pgfsetstrokecolor{currentstroke}%
\pgfsetdash{}{0pt}%
\pgfpathmoveto{\pgfqpoint{4.257972in}{1.973236in}}%
\pgfpathlineto{\pgfqpoint{4.257211in}{1.991156in}}%
\pgfusepath{stroke}%
\end{pgfscope}%
\begin{pgfscope}%
\pgfpathrectangle{\pgfqpoint{3.352233in}{1.400000in}}{\pgfqpoint{2.407767in}{1.544118in}}%
\pgfusepath{clip}%
\pgfsetbuttcap%
\pgfsetroundjoin%
\pgfsetlinewidth{0.501875pt}%
\definecolor{currentstroke}{rgb}{0.280267,0.073417,0.397163}%
\pgfsetstrokecolor{currentstroke}%
\pgfsetdash{}{0pt}%
\pgfpathmoveto{\pgfqpoint{4.257211in}{1.991156in}}%
\pgfpathlineto{\pgfqpoint{4.257211in}{1.991156in}}%
\pgfusepath{stroke}%
\end{pgfscope}%
\begin{pgfscope}%
\pgfpathrectangle{\pgfqpoint{3.352233in}{1.400000in}}{\pgfqpoint{2.407767in}{1.544118in}}%
\pgfusepath{clip}%
\pgfsetbuttcap%
\pgfsetroundjoin%
\pgfsetlinewidth{0.501875pt}%
\definecolor{currentstroke}{rgb}{0.280267,0.073417,0.397163}%
\pgfsetstrokecolor{currentstroke}%
\pgfsetdash{}{0pt}%
\pgfpathmoveto{\pgfqpoint{4.257211in}{1.991156in}}%
\pgfpathlineto{\pgfqpoint{4.253103in}{2.008277in}}%
\pgfusepath{stroke}%
\end{pgfscope}%
\begin{pgfscope}%
\pgfpathrectangle{\pgfqpoint{3.352233in}{1.400000in}}{\pgfqpoint{2.407767in}{1.544118in}}%
\pgfusepath{clip}%
\pgfsetbuttcap%
\pgfsetroundjoin%
\pgfsetlinewidth{0.501875pt}%
\definecolor{currentstroke}{rgb}{0.277941,0.056324,0.381191}%
\pgfsetstrokecolor{currentstroke}%
\pgfsetdash{}{0pt}%
\pgfpathmoveto{\pgfqpoint{4.191512in}{2.519143in}}%
\pgfpathlineto{\pgfqpoint{4.242257in}{2.509821in}}%
\pgfusepath{stroke}%
\end{pgfscope}%
\begin{pgfscope}%
\pgfpathrectangle{\pgfqpoint{3.352233in}{1.400000in}}{\pgfqpoint{2.407767in}{1.544118in}}%
\pgfusepath{clip}%
\pgfsetbuttcap%
\pgfsetroundjoin%
\pgfsetlinewidth{0.501875pt}%
\definecolor{currentstroke}{rgb}{0.277018,0.050344,0.375715}%
\pgfsetstrokecolor{currentstroke}%
\pgfsetdash{}{0pt}%
\pgfpathmoveto{\pgfqpoint{4.242257in}{2.509821in}}%
\pgfpathlineto{\pgfqpoint{4.291588in}{2.498107in}}%
\pgfusepath{stroke}%
\end{pgfscope}%
\begin{pgfscope}%
\pgfpathrectangle{\pgfqpoint{3.352233in}{1.400000in}}{\pgfqpoint{2.407767in}{1.544118in}}%
\pgfusepath{clip}%
\pgfsetbuttcap%
\pgfsetroundjoin%
\pgfsetlinewidth{0.501875pt}%
\definecolor{currentstroke}{rgb}{0.277018,0.050344,0.375715}%
\pgfsetstrokecolor{currentstroke}%
\pgfsetdash{}{0pt}%
\pgfpathmoveto{\pgfqpoint{4.291588in}{2.498107in}}%
\pgfpathlineto{\pgfqpoint{4.339396in}{2.484774in}}%
\pgfusepath{stroke}%
\end{pgfscope}%
\begin{pgfscope}%
\pgfpathrectangle{\pgfqpoint{3.352233in}{1.400000in}}{\pgfqpoint{2.407767in}{1.544118in}}%
\pgfusepath{clip}%
\pgfsetbuttcap%
\pgfsetroundjoin%
\pgfsetlinewidth{0.501875pt}%
\definecolor{currentstroke}{rgb}{0.273809,0.031497,0.358853}%
\pgfsetstrokecolor{currentstroke}%
\pgfsetdash{}{0pt}%
\pgfpathmoveto{\pgfqpoint{4.339396in}{2.484774in}}%
\pgfpathlineto{\pgfqpoint{4.339396in}{2.484774in}}%
\pgfusepath{stroke}%
\end{pgfscope}%
\begin{pgfscope}%
\pgfpathrectangle{\pgfqpoint{3.352233in}{1.400000in}}{\pgfqpoint{2.407767in}{1.544118in}}%
\pgfusepath{clip}%
\pgfsetbuttcap%
\pgfsetroundjoin%
\pgfsetlinewidth{0.501875pt}%
\definecolor{currentstroke}{rgb}{0.273809,0.031497,0.358853}%
\pgfsetstrokecolor{currentstroke}%
\pgfsetdash{}{0pt}%
\pgfpathmoveto{\pgfqpoint{4.339396in}{2.484774in}}%
\pgfpathlineto{\pgfqpoint{4.358341in}{2.475964in}}%
\pgfusepath{stroke}%
\end{pgfscope}%
\begin{pgfscope}%
\pgfpathrectangle{\pgfqpoint{3.352233in}{1.400000in}}{\pgfqpoint{2.407767in}{1.544118in}}%
\pgfusepath{clip}%
\pgfsetbuttcap%
\pgfsetroundjoin%
\pgfsetlinewidth{0.501875pt}%
\definecolor{currentstroke}{rgb}{0.276022,0.044167,0.370164}%
\pgfsetstrokecolor{currentstroke}%
\pgfsetdash{}{0pt}%
\pgfpathmoveto{\pgfqpoint{4.358341in}{2.475964in}}%
\pgfpathlineto{\pgfqpoint{4.358341in}{2.475964in}}%
\pgfusepath{stroke}%
\end{pgfscope}%
\begin{pgfscope}%
\pgfpathrectangle{\pgfqpoint{3.352233in}{1.400000in}}{\pgfqpoint{2.407767in}{1.544118in}}%
\pgfusepath{clip}%
\pgfsetbuttcap%
\pgfsetroundjoin%
\pgfsetlinewidth{0.501875pt}%
\definecolor{currentstroke}{rgb}{0.276022,0.044167,0.370164}%
\pgfsetstrokecolor{currentstroke}%
\pgfsetdash{}{0pt}%
\pgfpathmoveto{\pgfqpoint{4.358341in}{2.475964in}}%
\pgfpathlineto{\pgfqpoint{4.362121in}{2.466231in}}%
\pgfusepath{stroke}%
\end{pgfscope}%
\begin{pgfscope}%
\pgfpathrectangle{\pgfqpoint{3.352233in}{1.400000in}}{\pgfqpoint{2.407767in}{1.544118in}}%
\pgfusepath{clip}%
\pgfsetbuttcap%
\pgfsetroundjoin%
\pgfsetlinewidth{0.501875pt}%
\definecolor{currentstroke}{rgb}{0.280267,0.073417,0.397163}%
\pgfsetstrokecolor{currentstroke}%
\pgfsetdash{}{0pt}%
\pgfpathmoveto{\pgfqpoint{4.362121in}{2.466231in}}%
\pgfpathlineto{\pgfqpoint{4.359763in}{2.457039in}}%
\pgfusepath{stroke}%
\end{pgfscope}%
\begin{pgfscope}%
\pgfpathrectangle{\pgfqpoint{3.352233in}{1.400000in}}{\pgfqpoint{2.407767in}{1.544118in}}%
\pgfusepath{clip}%
\pgfsetbuttcap%
\pgfsetroundjoin%
\pgfsetlinewidth{0.501875pt}%
\definecolor{currentstroke}{rgb}{0.280894,0.078907,0.402329}%
\pgfsetstrokecolor{currentstroke}%
\pgfsetdash{}{0pt}%
\pgfpathmoveto{\pgfqpoint{4.359763in}{2.457039in}}%
\pgfpathlineto{\pgfqpoint{4.359763in}{2.457039in}}%
\pgfusepath{stroke}%
\end{pgfscope}%
\begin{pgfscope}%
\pgfpathrectangle{\pgfqpoint{3.352233in}{1.400000in}}{\pgfqpoint{2.407767in}{1.544118in}}%
\pgfusepath{clip}%
\pgfsetbuttcap%
\pgfsetroundjoin%
\pgfsetlinewidth{0.501875pt}%
\definecolor{currentstroke}{rgb}{0.280894,0.078907,0.402329}%
\pgfsetstrokecolor{currentstroke}%
\pgfsetdash{}{0pt}%
\pgfpathmoveto{\pgfqpoint{4.359763in}{2.457039in}}%
\pgfpathlineto{\pgfqpoint{4.360919in}{2.437962in}}%
\pgfusepath{stroke}%
\end{pgfscope}%
\begin{pgfscope}%
\pgfpathrectangle{\pgfqpoint{3.352233in}{1.400000in}}{\pgfqpoint{2.407767in}{1.544118in}}%
\pgfusepath{clip}%
\pgfsetbuttcap%
\pgfsetroundjoin%
\pgfsetlinewidth{0.501875pt}%
\definecolor{currentstroke}{rgb}{0.278791,0.062145,0.386592}%
\pgfsetstrokecolor{currentstroke}%
\pgfsetdash{}{0pt}%
\pgfpathmoveto{\pgfqpoint{4.360919in}{2.437962in}}%
\pgfpathlineto{\pgfqpoint{4.361083in}{2.419280in}}%
\pgfusepath{stroke}%
\end{pgfscope}%
\begin{pgfscope}%
\pgfpathrectangle{\pgfqpoint{3.352233in}{1.400000in}}{\pgfqpoint{2.407767in}{1.544118in}}%
\pgfusepath{clip}%
\pgfsetbuttcap%
\pgfsetroundjoin%
\pgfsetlinewidth{0.501875pt}%
\definecolor{currentstroke}{rgb}{0.278791,0.062145,0.386592}%
\pgfsetstrokecolor{currentstroke}%
\pgfsetdash{}{0pt}%
\pgfpathmoveto{\pgfqpoint{4.361083in}{2.419280in}}%
\pgfpathlineto{\pgfqpoint{4.361083in}{2.419280in}}%
\pgfusepath{stroke}%
\end{pgfscope}%
\begin{pgfscope}%
\pgfpathrectangle{\pgfqpoint{3.352233in}{1.400000in}}{\pgfqpoint{2.407767in}{1.544118in}}%
\pgfusepath{clip}%
\pgfsetbuttcap%
\pgfsetroundjoin%
\pgfsetlinewidth{0.501875pt}%
\definecolor{currentstroke}{rgb}{0.278791,0.062145,0.386592}%
\pgfsetstrokecolor{currentstroke}%
\pgfsetdash{}{0pt}%
\pgfpathmoveto{\pgfqpoint{4.361083in}{2.419280in}}%
\pgfpathlineto{\pgfqpoint{4.356396in}{2.399672in}}%
\pgfusepath{stroke}%
\end{pgfscope}%
\begin{pgfscope}%
\pgfpathrectangle{\pgfqpoint{3.352233in}{1.400000in}}{\pgfqpoint{2.407767in}{1.544118in}}%
\pgfusepath{clip}%
\pgfsetbuttcap%
\pgfsetroundjoin%
\pgfsetlinewidth{0.501875pt}%
\definecolor{currentstroke}{rgb}{0.282884,0.135920,0.453427}%
\pgfsetstrokecolor{currentstroke}%
\pgfsetdash{}{0pt}%
\pgfpathmoveto{\pgfqpoint{4.068495in}{2.380536in}}%
\pgfpathlineto{\pgfqpoint{4.114962in}{2.364403in}}%
\pgfusepath{stroke}%
\end{pgfscope}%
\begin{pgfscope}%
\pgfpathrectangle{\pgfqpoint{3.352233in}{1.400000in}}{\pgfqpoint{2.407767in}{1.544118in}}%
\pgfusepath{clip}%
\pgfsetbuttcap%
\pgfsetroundjoin%
\pgfsetlinewidth{0.501875pt}%
\definecolor{currentstroke}{rgb}{0.283072,0.130895,0.449241}%
\pgfsetstrokecolor{currentstroke}%
\pgfsetdash{}{0pt}%
\pgfpathmoveto{\pgfqpoint{4.114962in}{2.364403in}}%
\pgfpathlineto{\pgfqpoint{4.158737in}{2.345461in}}%
\pgfusepath{stroke}%
\end{pgfscope}%
\begin{pgfscope}%
\pgfpathrectangle{\pgfqpoint{3.352233in}{1.400000in}}{\pgfqpoint{2.407767in}{1.544118in}}%
\pgfusepath{clip}%
\pgfsetbuttcap%
\pgfsetroundjoin%
\pgfsetlinewidth{0.501875pt}%
\definecolor{currentstroke}{rgb}{0.283197,0.115680,0.436115}%
\pgfsetstrokecolor{currentstroke}%
\pgfsetdash{}{0pt}%
\pgfpathmoveto{\pgfqpoint{4.158737in}{2.345461in}}%
\pgfpathlineto{\pgfqpoint{4.158737in}{2.345461in}}%
\pgfusepath{stroke}%
\end{pgfscope}%
\begin{pgfscope}%
\pgfpathrectangle{\pgfqpoint{3.352233in}{1.400000in}}{\pgfqpoint{2.407767in}{1.544118in}}%
\pgfusepath{clip}%
\pgfsetbuttcap%
\pgfsetroundjoin%
\pgfsetlinewidth{0.501875pt}%
\definecolor{currentstroke}{rgb}{0.283197,0.115680,0.436115}%
\pgfsetstrokecolor{currentstroke}%
\pgfsetdash{}{0pt}%
\pgfpathmoveto{\pgfqpoint{4.158737in}{2.345461in}}%
\pgfpathlineto{\pgfqpoint{4.189051in}{2.327901in}}%
\pgfusepath{stroke}%
\end{pgfscope}%
\begin{pgfscope}%
\pgfpathrectangle{\pgfqpoint{3.352233in}{1.400000in}}{\pgfqpoint{2.407767in}{1.544118in}}%
\pgfusepath{clip}%
\pgfsetbuttcap%
\pgfsetroundjoin%
\pgfsetlinewidth{0.501875pt}%
\definecolor{currentstroke}{rgb}{0.282910,0.105393,0.426902}%
\pgfsetstrokecolor{currentstroke}%
\pgfsetdash{}{0pt}%
\pgfpathmoveto{\pgfqpoint{4.189051in}{2.327901in}}%
\pgfpathlineto{\pgfqpoint{4.218945in}{2.304646in}}%
\pgfusepath{stroke}%
\end{pgfscope}%
\begin{pgfscope}%
\pgfpathrectangle{\pgfqpoint{3.352233in}{1.400000in}}{\pgfqpoint{2.407767in}{1.544118in}}%
\pgfusepath{clip}%
\pgfsetbuttcap%
\pgfsetroundjoin%
\pgfsetlinewidth{0.501875pt}%
\definecolor{currentstroke}{rgb}{0.281924,0.089666,0.412415}%
\pgfsetstrokecolor{currentstroke}%
\pgfsetdash{}{0pt}%
\pgfpathmoveto{\pgfqpoint{4.218945in}{2.304646in}}%
\pgfpathlineto{\pgfqpoint{4.245506in}{2.279656in}}%
\pgfusepath{stroke}%
\end{pgfscope}%
\begin{pgfscope}%
\pgfpathrectangle{\pgfqpoint{3.352233in}{1.400000in}}{\pgfqpoint{2.407767in}{1.544118in}}%
\pgfusepath{clip}%
\pgfsetbuttcap%
\pgfsetroundjoin%
\pgfsetlinewidth{0.501875pt}%
\definecolor{currentstroke}{rgb}{0.276022,0.044167,0.370164}%
\pgfsetstrokecolor{currentstroke}%
\pgfsetdash{}{0pt}%
\pgfpathmoveto{\pgfqpoint{4.245506in}{2.279656in}}%
\pgfpathlineto{\pgfqpoint{4.245506in}{2.279656in}}%
\pgfusepath{stroke}%
\end{pgfscope}%
\begin{pgfscope}%
\pgfpathrectangle{\pgfqpoint{3.352233in}{1.400000in}}{\pgfqpoint{2.407767in}{1.544118in}}%
\pgfusepath{clip}%
\pgfsetbuttcap%
\pgfsetroundjoin%
\pgfsetlinewidth{0.501875pt}%
\definecolor{currentstroke}{rgb}{0.276022,0.044167,0.370164}%
\pgfsetstrokecolor{currentstroke}%
\pgfsetdash{}{0pt}%
\pgfpathmoveto{\pgfqpoint{4.245506in}{2.279656in}}%
\pgfpathlineto{\pgfqpoint{4.252658in}{2.267961in}}%
\pgfusepath{stroke}%
\end{pgfscope}%
\begin{pgfscope}%
\pgfpathrectangle{\pgfqpoint{3.352233in}{1.400000in}}{\pgfqpoint{2.407767in}{1.544118in}}%
\pgfusepath{clip}%
\pgfsetbuttcap%
\pgfsetroundjoin%
\pgfsetlinewidth{0.000000pt}%
\definecolor{currentstroke}{rgb}{0.276022,0.044167,0.370164}%
\pgfsetstrokecolor{currentstroke}%
\pgfsetdash{}{0pt}%
\pgfpathmoveto{\pgfqpoint{4.078604in}{2.475152in}}%
\pgfpathlineto{\pgfqpoint{4.128149in}{2.463317in}}%
\pgfusepath{}%
\end{pgfscope}%
\begin{pgfscope}%
\pgfpathrectangle{\pgfqpoint{3.352233in}{1.400000in}}{\pgfqpoint{2.407767in}{1.544118in}}%
\pgfusepath{clip}%
\pgfsetbuttcap%
\pgfsetroundjoin%
\pgfsetlinewidth{0.501875pt}%
\definecolor{currentstroke}{rgb}{0.282910,0.105393,0.426902}%
\pgfsetstrokecolor{currentstroke}%
\pgfsetdash{}{0pt}%
\pgfpathmoveto{\pgfqpoint{4.128149in}{2.463317in}}%
\pgfpathlineto{\pgfqpoint{4.176855in}{2.450028in}}%
\pgfusepath{stroke}%
\end{pgfscope}%
\begin{pgfscope}%
\pgfpathrectangle{\pgfqpoint{3.352233in}{1.400000in}}{\pgfqpoint{2.407767in}{1.544118in}}%
\pgfusepath{clip}%
\pgfsetbuttcap%
\pgfsetroundjoin%
\pgfsetlinewidth{0.501875pt}%
\definecolor{currentstroke}{rgb}{0.280894,0.078907,0.402329}%
\pgfsetstrokecolor{currentstroke}%
\pgfsetdash{}{0pt}%
\pgfpathmoveto{\pgfqpoint{4.176855in}{2.450028in}}%
\pgfpathlineto{\pgfqpoint{4.224904in}{2.436081in}}%
\pgfusepath{stroke}%
\end{pgfscope}%
\begin{pgfscope}%
\pgfpathrectangle{\pgfqpoint{3.352233in}{1.400000in}}{\pgfqpoint{2.407767in}{1.544118in}}%
\pgfusepath{clip}%
\pgfsetbuttcap%
\pgfsetroundjoin%
\pgfsetlinewidth{0.501875pt}%
\definecolor{currentstroke}{rgb}{0.279566,0.067836,0.391917}%
\pgfsetstrokecolor{currentstroke}%
\pgfsetdash{}{0pt}%
\pgfpathmoveto{\pgfqpoint{4.224904in}{2.436081in}}%
\pgfpathlineto{\pgfqpoint{4.224904in}{2.436081in}}%
\pgfusepath{stroke}%
\end{pgfscope}%
\begin{pgfscope}%
\pgfpathrectangle{\pgfqpoint{3.352233in}{1.400000in}}{\pgfqpoint{2.407767in}{1.544118in}}%
\pgfusepath{clip}%
\pgfsetbuttcap%
\pgfsetroundjoin%
\pgfsetlinewidth{0.501875pt}%
\definecolor{currentstroke}{rgb}{0.279566,0.067836,0.391917}%
\pgfsetstrokecolor{currentstroke}%
\pgfsetdash{}{0pt}%
\pgfpathmoveto{\pgfqpoint{4.224904in}{2.436081in}}%
\pgfpathlineto{\pgfqpoint{4.257369in}{2.422035in}}%
\pgfusepath{stroke}%
\end{pgfscope}%
\begin{pgfscope}%
\pgfpathrectangle{\pgfqpoint{3.352233in}{1.400000in}}{\pgfqpoint{2.407767in}{1.544118in}}%
\pgfusepath{clip}%
\pgfsetbuttcap%
\pgfsetroundjoin%
\pgfsetlinewidth{0.501875pt}%
\definecolor{currentstroke}{rgb}{0.280894,0.078907,0.402329}%
\pgfsetstrokecolor{currentstroke}%
\pgfsetdash{}{0pt}%
\pgfpathmoveto{\pgfqpoint{4.257369in}{2.422035in}}%
\pgfpathlineto{\pgfqpoint{4.284662in}{2.404127in}}%
\pgfusepath{stroke}%
\end{pgfscope}%
\begin{pgfscope}%
\pgfpathrectangle{\pgfqpoint{3.352233in}{1.400000in}}{\pgfqpoint{2.407767in}{1.544118in}}%
\pgfusepath{clip}%
\pgfsetbuttcap%
\pgfsetroundjoin%
\pgfsetlinewidth{0.501875pt}%
\definecolor{currentstroke}{rgb}{0.280894,0.078907,0.402329}%
\pgfsetstrokecolor{currentstroke}%
\pgfsetdash{}{0pt}%
\pgfpathmoveto{\pgfqpoint{4.284662in}{2.404127in}}%
\pgfpathlineto{\pgfqpoint{4.284662in}{2.404127in}}%
\pgfusepath{stroke}%
\end{pgfscope}%
\begin{pgfscope}%
\pgfpathrectangle{\pgfqpoint{3.352233in}{1.400000in}}{\pgfqpoint{2.407767in}{1.544118in}}%
\pgfusepath{clip}%
\pgfsetbuttcap%
\pgfsetroundjoin%
\pgfsetlinewidth{0.501875pt}%
\definecolor{currentstroke}{rgb}{0.280894,0.078907,0.402329}%
\pgfsetstrokecolor{currentstroke}%
\pgfsetdash{}{0pt}%
\pgfpathmoveto{\pgfqpoint{4.284662in}{2.404127in}}%
\pgfpathlineto{\pgfqpoint{4.284662in}{2.404127in}}%
\pgfusepath{stroke}%
\end{pgfscope}%
\begin{pgfscope}%
\pgfpathrectangle{\pgfqpoint{3.352233in}{1.400000in}}{\pgfqpoint{2.407767in}{1.544118in}}%
\pgfusepath{clip}%
\pgfsetbuttcap%
\pgfsetroundjoin%
\pgfsetlinewidth{0.501875pt}%
\definecolor{currentstroke}{rgb}{0.280894,0.078907,0.402329}%
\pgfsetstrokecolor{currentstroke}%
\pgfsetdash{}{0pt}%
\pgfpathmoveto{\pgfqpoint{4.284662in}{2.404127in}}%
\pgfpathlineto{\pgfqpoint{4.293395in}{2.394290in}}%
\pgfusepath{stroke}%
\end{pgfscope}%
\begin{pgfscope}%
\pgfpathrectangle{\pgfqpoint{3.352233in}{1.400000in}}{\pgfqpoint{2.407767in}{1.544118in}}%
\pgfusepath{clip}%
\pgfsetbuttcap%
\pgfsetroundjoin%
\pgfsetlinewidth{0.501875pt}%
\definecolor{currentstroke}{rgb}{0.280894,0.078907,0.402329}%
\pgfsetstrokecolor{currentstroke}%
\pgfsetdash{}{0pt}%
\pgfpathmoveto{\pgfqpoint{4.293395in}{2.394290in}}%
\pgfpathlineto{\pgfqpoint{4.293046in}{2.382426in}}%
\pgfusepath{stroke}%
\end{pgfscope}%
\begin{pgfscope}%
\pgfpathrectangle{\pgfqpoint{3.352233in}{1.400000in}}{\pgfqpoint{2.407767in}{1.544118in}}%
\pgfusepath{clip}%
\pgfsetbuttcap%
\pgfsetroundjoin%
\pgfsetlinewidth{0.501875pt}%
\definecolor{currentstroke}{rgb}{0.282910,0.105393,0.426902}%
\pgfsetstrokecolor{currentstroke}%
\pgfsetdash{}{0pt}%
\pgfpathmoveto{\pgfqpoint{4.293046in}{2.382426in}}%
\pgfpathlineto{\pgfqpoint{4.294235in}{2.369278in}}%
\pgfusepath{stroke}%
\end{pgfscope}%
\begin{pgfscope}%
\pgfpathrectangle{\pgfqpoint{3.352233in}{1.400000in}}{\pgfqpoint{2.407767in}{1.544118in}}%
\pgfusepath{clip}%
\pgfsetbuttcap%
\pgfsetroundjoin%
\pgfsetlinewidth{0.501875pt}%
\definecolor{currentstroke}{rgb}{0.281924,0.089666,0.412415}%
\pgfsetstrokecolor{currentstroke}%
\pgfsetdash{}{0pt}%
\pgfpathmoveto{\pgfqpoint{4.294235in}{2.369278in}}%
\pgfpathlineto{\pgfqpoint{4.296008in}{2.355422in}}%
\pgfusepath{stroke}%
\end{pgfscope}%
\begin{pgfscope}%
\pgfpathrectangle{\pgfqpoint{3.352233in}{1.400000in}}{\pgfqpoint{2.407767in}{1.544118in}}%
\pgfusepath{clip}%
\pgfsetbuttcap%
\pgfsetroundjoin%
\pgfsetlinewidth{0.501875pt}%
\definecolor{currentstroke}{rgb}{0.281924,0.089666,0.412415}%
\pgfsetstrokecolor{currentstroke}%
\pgfsetdash{}{0pt}%
\pgfpathmoveto{\pgfqpoint{4.296008in}{2.355422in}}%
\pgfpathlineto{\pgfqpoint{4.294422in}{2.337962in}}%
\pgfusepath{stroke}%
\end{pgfscope}%
\begin{pgfscope}%
\pgfpathrectangle{\pgfqpoint{3.352233in}{1.400000in}}{\pgfqpoint{2.407767in}{1.544118in}}%
\pgfusepath{clip}%
\pgfsetroundcap%
\pgfsetroundjoin%
\pgfsetlinewidth{0.501875pt}%
\definecolor{currentstroke}{rgb}{0.269944,0.014625,0.341379}%
\pgfsetstrokecolor{currentstroke}%
\pgfsetdash{}{0pt}%
\pgfpathmoveto{\pgfqpoint{4.357992in}{1.458149in}}%
\pgfpathquadraticcurveto{\pgfqpoint{4.357457in}{1.459223in}}{\pgfqpoint{4.360384in}{1.453347in}}%
\pgfusepath{stroke}%
\end{pgfscope}%
\begin{pgfscope}%
\pgfpathrectangle{\pgfqpoint{3.352233in}{1.400000in}}{\pgfqpoint{2.407767in}{1.544118in}}%
\pgfusepath{clip}%
\pgfsetroundcap%
\pgfsetroundjoin%
\definecolor{currentfill}{rgb}{0.269944,0.014625,0.341379}%
\pgfsetfillcolor{currentfill}%
\pgfsetlinewidth{0.501875pt}%
\definecolor{currentstroke}{rgb}{0.269944,0.014625,0.341379}%
\pgfsetstrokecolor{currentstroke}%
\pgfsetdash{}{0pt}%
\pgfpathmoveto{\pgfqpoint{4.360338in}{1.422291in}}%
\pgfpathlineto{\pgfqpoint{4.360384in}{1.453347in}}%
\pgfpathlineto{\pgfqpoint{4.385202in}{1.434676in}}%
\pgfpathlineto{\pgfqpoint{4.360338in}{1.422291in}}%
\pgfpathlineto{\pgfqpoint{4.360338in}{1.422291in}}%
\pgfpathclose%
\pgfusepath{stroke,fill}%
\end{pgfscope}%
\begin{pgfscope}%
\pgfpathrectangle{\pgfqpoint{3.352233in}{1.400000in}}{\pgfqpoint{2.407767in}{1.544118in}}%
\pgfusepath{clip}%
\pgfsetroundcap%
\pgfsetroundjoin%
\pgfsetlinewidth{0.501875pt}%
\definecolor{currentstroke}{rgb}{0.272594,0.025563,0.353093}%
\pgfsetstrokecolor{currentstroke}%
\pgfsetdash{}{0pt}%
\pgfpathmoveto{\pgfqpoint{5.039499in}{1.440102in}}%
\pgfpathquadraticcurveto{\pgfqpoint{5.026303in}{1.439991in}}{\pgfqpoint{5.020871in}{1.439945in}}%
\pgfusepath{stroke}%
\end{pgfscope}%
\begin{pgfscope}%
\pgfpathrectangle{\pgfqpoint{3.352233in}{1.400000in}}{\pgfqpoint{2.407767in}{1.544118in}}%
\pgfusepath{clip}%
\pgfsetroundcap%
\pgfsetroundjoin%
\definecolor{currentfill}{rgb}{0.272594,0.025563,0.353093}%
\pgfsetfillcolor{currentfill}%
\pgfsetlinewidth{0.501875pt}%
\definecolor{currentstroke}{rgb}{0.272594,0.025563,0.353093}%
\pgfsetstrokecolor{currentstroke}%
\pgfsetdash{}{0pt}%
\pgfpathmoveto{\pgfqpoint{5.048764in}{1.426290in}}%
\pgfpathlineto{\pgfqpoint{5.020871in}{1.439945in}}%
\pgfpathlineto{\pgfqpoint{5.048531in}{1.454067in}}%
\pgfpathlineto{\pgfqpoint{5.048764in}{1.426290in}}%
\pgfpathlineto{\pgfqpoint{5.048764in}{1.426290in}}%
\pgfpathclose%
\pgfusepath{stroke,fill}%
\end{pgfscope}%
\begin{pgfscope}%
\pgfpathrectangle{\pgfqpoint{3.352233in}{1.400000in}}{\pgfqpoint{2.407767in}{1.544118in}}%
\pgfusepath{clip}%
\pgfsetroundcap%
\pgfsetroundjoin%
\pgfsetlinewidth{0.501875pt}%
\definecolor{currentstroke}{rgb}{0.272594,0.025563,0.353093}%
\pgfsetstrokecolor{currentstroke}%
\pgfsetdash{}{0pt}%
\pgfpathmoveto{\pgfqpoint{4.983959in}{1.467935in}}%
\pgfpathquadraticcurveto{\pgfqpoint{4.970739in}{1.468402in}}{\pgfqpoint{4.965278in}{1.468594in}}%
\pgfusepath{stroke}%
\end{pgfscope}%
\begin{pgfscope}%
\pgfpathrectangle{\pgfqpoint{3.352233in}{1.400000in}}{\pgfqpoint{2.407767in}{1.544118in}}%
\pgfusepath{clip}%
\pgfsetroundcap%
\pgfsetroundjoin%
\definecolor{currentfill}{rgb}{0.272594,0.025563,0.353093}%
\pgfsetfillcolor{currentfill}%
\pgfsetlinewidth{0.501875pt}%
\definecolor{currentstroke}{rgb}{0.272594,0.025563,0.353093}%
\pgfsetstrokecolor{currentstroke}%
\pgfsetdash{}{0pt}%
\pgfpathmoveto{\pgfqpoint{4.992548in}{1.453734in}}%
\pgfpathlineto{\pgfqpoint{4.965278in}{1.468594in}}%
\pgfpathlineto{\pgfqpoint{4.993528in}{1.481494in}}%
\pgfpathlineto{\pgfqpoint{4.992548in}{1.453734in}}%
\pgfpathlineto{\pgfqpoint{4.992548in}{1.453734in}}%
\pgfpathclose%
\pgfusepath{stroke,fill}%
\end{pgfscope}%
\begin{pgfscope}%
\pgfpathrectangle{\pgfqpoint{3.352233in}{1.400000in}}{\pgfqpoint{2.407767in}{1.544118in}}%
\pgfusepath{clip}%
\pgfsetroundcap%
\pgfsetroundjoin%
\pgfsetlinewidth{0.501875pt}%
\definecolor{currentstroke}{rgb}{0.271305,0.019942,0.347269}%
\pgfsetstrokecolor{currentstroke}%
\pgfsetdash{}{0pt}%
\pgfpathmoveto{\pgfqpoint{5.047535in}{2.866345in}}%
\pgfpathquadraticcurveto{\pgfqpoint{5.034322in}{2.865989in}}{\pgfqpoint{5.028870in}{2.865842in}}%
\pgfusepath{stroke}%
\end{pgfscope}%
\begin{pgfscope}%
\pgfpathrectangle{\pgfqpoint{3.352233in}{1.400000in}}{\pgfqpoint{2.407767in}{1.544118in}}%
\pgfusepath{clip}%
\pgfsetroundcap%
\pgfsetroundjoin%
\definecolor{currentfill}{rgb}{0.271305,0.019942,0.347269}%
\pgfsetfillcolor{currentfill}%
\pgfsetlinewidth{0.501875pt}%
\definecolor{currentstroke}{rgb}{0.271305,0.019942,0.347269}%
\pgfsetstrokecolor{currentstroke}%
\pgfsetdash{}{0pt}%
\pgfpathmoveto{\pgfqpoint{5.057012in}{2.852707in}}%
\pgfpathlineto{\pgfqpoint{5.028870in}{2.865842in}}%
\pgfpathlineto{\pgfqpoint{5.056264in}{2.880475in}}%
\pgfpathlineto{\pgfqpoint{5.057012in}{2.852707in}}%
\pgfpathlineto{\pgfqpoint{5.057012in}{2.852707in}}%
\pgfpathclose%
\pgfusepath{stroke,fill}%
\end{pgfscope}%
\begin{pgfscope}%
\pgfpathrectangle{\pgfqpoint{3.352233in}{1.400000in}}{\pgfqpoint{2.407767in}{1.544118in}}%
\pgfusepath{clip}%
\pgfsetroundcap%
\pgfsetroundjoin%
\pgfsetlinewidth{0.501875pt}%
\definecolor{currentstroke}{rgb}{0.272594,0.025563,0.353093}%
\pgfsetstrokecolor{currentstroke}%
\pgfsetdash{}{0pt}%
\pgfpathmoveto{\pgfqpoint{5.038276in}{1.504836in}}%
\pgfpathquadraticcurveto{\pgfqpoint{5.025042in}{1.505078in}}{\pgfqpoint{5.019571in}{1.505177in}}%
\pgfusepath{stroke}%
\end{pgfscope}%
\begin{pgfscope}%
\pgfpathrectangle{\pgfqpoint{3.352233in}{1.400000in}}{\pgfqpoint{2.407767in}{1.544118in}}%
\pgfusepath{clip}%
\pgfsetroundcap%
\pgfsetroundjoin%
\definecolor{currentfill}{rgb}{0.272594,0.025563,0.353093}%
\pgfsetfillcolor{currentfill}%
\pgfsetlinewidth{0.501875pt}%
\definecolor{currentstroke}{rgb}{0.272594,0.025563,0.353093}%
\pgfsetstrokecolor{currentstroke}%
\pgfsetdash{}{0pt}%
\pgfpathmoveto{\pgfqpoint{5.047090in}{1.490784in}}%
\pgfpathlineto{\pgfqpoint{5.019571in}{1.505177in}}%
\pgfpathlineto{\pgfqpoint{5.047597in}{1.518557in}}%
\pgfpathlineto{\pgfqpoint{5.047090in}{1.490784in}}%
\pgfpathlineto{\pgfqpoint{5.047090in}{1.490784in}}%
\pgfpathclose%
\pgfusepath{stroke,fill}%
\end{pgfscope}%
\begin{pgfscope}%
\pgfpathrectangle{\pgfqpoint{3.352233in}{1.400000in}}{\pgfqpoint{2.407767in}{1.544118in}}%
\pgfusepath{clip}%
\pgfsetroundcap%
\pgfsetroundjoin%
\pgfsetlinewidth{0.501875pt}%
\definecolor{currentstroke}{rgb}{0.273809,0.031497,0.358853}%
\pgfsetstrokecolor{currentstroke}%
\pgfsetdash{}{0pt}%
\pgfpathmoveto{\pgfqpoint{5.047391in}{2.831655in}}%
\pgfpathquadraticcurveto{\pgfqpoint{5.034154in}{2.831713in}}{\pgfqpoint{5.028680in}{2.831738in}}%
\pgfusepath{stroke}%
\end{pgfscope}%
\begin{pgfscope}%
\pgfpathrectangle{\pgfqpoint{3.352233in}{1.400000in}}{\pgfqpoint{2.407767in}{1.544118in}}%
\pgfusepath{clip}%
\pgfsetroundcap%
\pgfsetroundjoin%
\definecolor{currentfill}{rgb}{0.273809,0.031497,0.358853}%
\pgfsetfillcolor{currentfill}%
\pgfsetlinewidth{0.501875pt}%
\definecolor{currentstroke}{rgb}{0.273809,0.031497,0.358853}%
\pgfsetstrokecolor{currentstroke}%
\pgfsetdash{}{0pt}%
\pgfpathmoveto{\pgfqpoint{5.056396in}{2.817726in}}%
\pgfpathlineto{\pgfqpoint{5.028680in}{2.831738in}}%
\pgfpathlineto{\pgfqpoint{5.056519in}{2.845503in}}%
\pgfpathlineto{\pgfqpoint{5.056396in}{2.817726in}}%
\pgfpathlineto{\pgfqpoint{5.056396in}{2.817726in}}%
\pgfpathclose%
\pgfusepath{stroke,fill}%
\end{pgfscope}%
\begin{pgfscope}%
\pgfpathrectangle{\pgfqpoint{3.352233in}{1.400000in}}{\pgfqpoint{2.407767in}{1.544118in}}%
\pgfusepath{clip}%
\pgfsetroundcap%
\pgfsetroundjoin%
\pgfsetlinewidth{0.501875pt}%
\definecolor{currentstroke}{rgb}{0.268510,0.009605,0.335427}%
\pgfsetstrokecolor{currentstroke}%
\pgfsetdash{}{0pt}%
\pgfpathmoveto{\pgfqpoint{4.508444in}{1.551861in}}%
\pgfpathquadraticcurveto{\pgfqpoint{4.508523in}{1.552502in}}{\pgfqpoint{4.507647in}{1.545437in}}%
\pgfusepath{stroke}%
\end{pgfscope}%
\begin{pgfscope}%
\pgfpathrectangle{\pgfqpoint{3.352233in}{1.400000in}}{\pgfqpoint{2.407767in}{1.544118in}}%
\pgfusepath{clip}%
\pgfsetroundcap%
\pgfsetroundjoin%
\definecolor{currentfill}{rgb}{0.268510,0.009605,0.335427}%
\pgfsetfillcolor{currentfill}%
\pgfsetlinewidth{0.501875pt}%
\definecolor{currentstroke}{rgb}{0.268510,0.009605,0.335427}%
\pgfsetstrokecolor{currentstroke}%
\pgfsetdash{}{0pt}%
\pgfpathmoveto{\pgfqpoint{4.490448in}{1.519578in}}%
\pgfpathlineto{\pgfqpoint{4.507647in}{1.545437in}}%
\pgfpathlineto{\pgfqpoint{4.518015in}{1.516162in}}%
\pgfpathlineto{\pgfqpoint{4.490448in}{1.519578in}}%
\pgfpathlineto{\pgfqpoint{4.490448in}{1.519578in}}%
\pgfpathclose%
\pgfusepath{stroke,fill}%
\end{pgfscope}%
\begin{pgfscope}%
\pgfpathrectangle{\pgfqpoint{3.352233in}{1.400000in}}{\pgfqpoint{2.407767in}{1.544118in}}%
\pgfusepath{clip}%
\pgfsetroundcap%
\pgfsetroundjoin%
\pgfsetlinewidth{0.501875pt}%
\definecolor{currentstroke}{rgb}{0.273809,0.031497,0.358853}%
\pgfsetstrokecolor{currentstroke}%
\pgfsetdash{}{0pt}%
\pgfpathmoveto{\pgfqpoint{4.929719in}{1.536968in}}%
\pgfpathquadraticcurveto{\pgfqpoint{4.916497in}{1.537428in}}{\pgfqpoint{4.911034in}{1.537618in}}%
\pgfusepath{stroke}%
\end{pgfscope}%
\begin{pgfscope}%
\pgfpathrectangle{\pgfqpoint{3.352233in}{1.400000in}}{\pgfqpoint{2.407767in}{1.544118in}}%
\pgfusepath{clip}%
\pgfsetroundcap%
\pgfsetroundjoin%
\definecolor{currentfill}{rgb}{0.273809,0.031497,0.358853}%
\pgfsetfillcolor{currentfill}%
\pgfsetlinewidth{0.501875pt}%
\definecolor{currentstroke}{rgb}{0.273809,0.031497,0.358853}%
\pgfsetstrokecolor{currentstroke}%
\pgfsetdash{}{0pt}%
\pgfpathmoveto{\pgfqpoint{4.938312in}{1.522772in}}%
\pgfpathlineto{\pgfqpoint{4.911034in}{1.537618in}}%
\pgfpathlineto{\pgfqpoint{4.939278in}{1.550533in}}%
\pgfpathlineto{\pgfqpoint{4.938312in}{1.522772in}}%
\pgfpathlineto{\pgfqpoint{4.938312in}{1.522772in}}%
\pgfpathclose%
\pgfusepath{stroke,fill}%
\end{pgfscope}%
\begin{pgfscope}%
\pgfpathrectangle{\pgfqpoint{3.352233in}{1.400000in}}{\pgfqpoint{2.407767in}{1.544118in}}%
\pgfusepath{clip}%
\pgfsetroundcap%
\pgfsetroundjoin%
\pgfsetlinewidth{0.501875pt}%
\definecolor{currentstroke}{rgb}{0.271305,0.019942,0.347269}%
\pgfsetstrokecolor{currentstroke}%
\pgfsetdash{}{0pt}%
\pgfpathmoveto{\pgfqpoint{5.047418in}{2.796145in}}%
\pgfpathquadraticcurveto{\pgfqpoint{5.034205in}{2.795823in}}{\pgfqpoint{5.028754in}{2.795690in}}%
\pgfusepath{stroke}%
\end{pgfscope}%
\begin{pgfscope}%
\pgfpathrectangle{\pgfqpoint{3.352233in}{1.400000in}}{\pgfqpoint{2.407767in}{1.544118in}}%
\pgfusepath{clip}%
\pgfsetroundcap%
\pgfsetroundjoin%
\definecolor{currentfill}{rgb}{0.271305,0.019942,0.347269}%
\pgfsetfillcolor{currentfill}%
\pgfsetlinewidth{0.501875pt}%
\definecolor{currentstroke}{rgb}{0.271305,0.019942,0.347269}%
\pgfsetstrokecolor{currentstroke}%
\pgfsetdash{}{0pt}%
\pgfpathmoveto{\pgfqpoint{5.056862in}{2.782482in}}%
\pgfpathlineto{\pgfqpoint{5.028754in}{2.795690in}}%
\pgfpathlineto{\pgfqpoint{5.056186in}{2.810251in}}%
\pgfpathlineto{\pgfqpoint{5.056862in}{2.782482in}}%
\pgfpathlineto{\pgfqpoint{5.056862in}{2.782482in}}%
\pgfpathclose%
\pgfusepath{stroke,fill}%
\end{pgfscope}%
\begin{pgfscope}%
\pgfpathrectangle{\pgfqpoint{3.352233in}{1.400000in}}{\pgfqpoint{2.407767in}{1.544118in}}%
\pgfusepath{clip}%
\pgfsetroundcap%
\pgfsetroundjoin%
\pgfsetlinewidth{0.501875pt}%
\definecolor{currentstroke}{rgb}{0.271305,0.019942,0.347269}%
\pgfsetstrokecolor{currentstroke}%
\pgfsetdash{}{0pt}%
\pgfpathmoveto{\pgfqpoint{4.669886in}{1.590786in}}%
\pgfpathquadraticcurveto{\pgfqpoint{4.663453in}{1.592766in}}{\pgfqpoint{4.664440in}{1.592462in}}%
\pgfusepath{stroke}%
\end{pgfscope}%
\begin{pgfscope}%
\pgfpathrectangle{\pgfqpoint{3.352233in}{1.400000in}}{\pgfqpoint{2.407767in}{1.544118in}}%
\pgfusepath{clip}%
\pgfsetroundcap%
\pgfsetroundjoin%
\definecolor{currentfill}{rgb}{0.271305,0.019942,0.347269}%
\pgfsetfillcolor{currentfill}%
\pgfsetlinewidth{0.501875pt}%
\definecolor{currentstroke}{rgb}{0.271305,0.019942,0.347269}%
\pgfsetstrokecolor{currentstroke}%
\pgfsetdash{}{0pt}%
\pgfpathmoveto{\pgfqpoint{4.686904in}{1.571017in}}%
\pgfpathlineto{\pgfqpoint{4.664440in}{1.592462in}}%
\pgfpathlineto{\pgfqpoint{4.695074in}{1.597566in}}%
\pgfpathlineto{\pgfqpoint{4.686904in}{1.571017in}}%
\pgfpathlineto{\pgfqpoint{4.686904in}{1.571017in}}%
\pgfpathclose%
\pgfusepath{stroke,fill}%
\end{pgfscope}%
\begin{pgfscope}%
\pgfpathrectangle{\pgfqpoint{3.352233in}{1.400000in}}{\pgfqpoint{2.407767in}{1.544118in}}%
\pgfusepath{clip}%
\pgfsetroundcap%
\pgfsetroundjoin%
\pgfsetlinewidth{0.501875pt}%
\definecolor{currentstroke}{rgb}{0.271305,0.019942,0.347269}%
\pgfsetstrokecolor{currentstroke}%
\pgfsetdash{}{0pt}%
\pgfpathmoveto{\pgfqpoint{5.096703in}{1.581022in}}%
\pgfpathquadraticcurveto{\pgfqpoint{5.083462in}{1.581110in}}{\pgfqpoint{5.077985in}{1.581147in}}%
\pgfusepath{stroke}%
\end{pgfscope}%
\begin{pgfscope}%
\pgfpathrectangle{\pgfqpoint{3.352233in}{1.400000in}}{\pgfqpoint{2.407767in}{1.544118in}}%
\pgfusepath{clip}%
\pgfsetroundcap%
\pgfsetroundjoin%
\definecolor{currentfill}{rgb}{0.271305,0.019942,0.347269}%
\pgfsetfillcolor{currentfill}%
\pgfsetlinewidth{0.501875pt}%
\definecolor{currentstroke}{rgb}{0.271305,0.019942,0.347269}%
\pgfsetstrokecolor{currentstroke}%
\pgfsetdash{}{0pt}%
\pgfpathmoveto{\pgfqpoint{5.105669in}{1.567074in}}%
\pgfpathlineto{\pgfqpoint{5.077985in}{1.581147in}}%
\pgfpathlineto{\pgfqpoint{5.105854in}{1.594851in}}%
\pgfpathlineto{\pgfqpoint{5.105669in}{1.567074in}}%
\pgfpathlineto{\pgfqpoint{5.105669in}{1.567074in}}%
\pgfpathclose%
\pgfusepath{stroke,fill}%
\end{pgfscope}%
\begin{pgfscope}%
\pgfpathrectangle{\pgfqpoint{3.352233in}{1.400000in}}{\pgfqpoint{2.407767in}{1.544118in}}%
\pgfusepath{clip}%
\pgfsetroundcap%
\pgfsetroundjoin%
\pgfsetlinewidth{0.501875pt}%
\definecolor{currentstroke}{rgb}{0.271305,0.019942,0.347269}%
\pgfsetstrokecolor{currentstroke}%
\pgfsetdash{}{0pt}%
\pgfpathmoveto{\pgfqpoint{5.047563in}{2.759919in}}%
\pgfpathquadraticcurveto{\pgfqpoint{5.034328in}{2.759719in}}{\pgfqpoint{5.028855in}{2.759637in}}%
\pgfusepath{stroke}%
\end{pgfscope}%
\begin{pgfscope}%
\pgfpathrectangle{\pgfqpoint{3.352233in}{1.400000in}}{\pgfqpoint{2.407767in}{1.544118in}}%
\pgfusepath{clip}%
\pgfsetroundcap%
\pgfsetroundjoin%
\definecolor{currentfill}{rgb}{0.271305,0.019942,0.347269}%
\pgfsetfillcolor{currentfill}%
\pgfsetlinewidth{0.501875pt}%
\definecolor{currentstroke}{rgb}{0.271305,0.019942,0.347269}%
\pgfsetstrokecolor{currentstroke}%
\pgfsetdash{}{0pt}%
\pgfpathmoveto{\pgfqpoint{5.056839in}{2.746168in}}%
\pgfpathlineto{\pgfqpoint{5.028855in}{2.759637in}}%
\pgfpathlineto{\pgfqpoint{5.056420in}{2.773943in}}%
\pgfpathlineto{\pgfqpoint{5.056839in}{2.746168in}}%
\pgfpathlineto{\pgfqpoint{5.056839in}{2.746168in}}%
\pgfpathclose%
\pgfusepath{stroke,fill}%
\end{pgfscope}%
\begin{pgfscope}%
\pgfpathrectangle{\pgfqpoint{3.352233in}{1.400000in}}{\pgfqpoint{2.407767in}{1.544118in}}%
\pgfusepath{clip}%
\pgfsetroundcap%
\pgfsetroundjoin%
\pgfsetlinewidth{0.501875pt}%
\definecolor{currentstroke}{rgb}{0.276022,0.044167,0.370164}%
\pgfsetstrokecolor{currentstroke}%
\pgfsetdash{}{0pt}%
\pgfpathmoveto{\pgfqpoint{4.929238in}{1.602877in}}%
\pgfpathquadraticcurveto{\pgfqpoint{4.916022in}{1.603367in}}{\pgfqpoint{4.910565in}{1.603569in}}%
\pgfusepath{stroke}%
\end{pgfscope}%
\begin{pgfscope}%
\pgfpathrectangle{\pgfqpoint{3.352233in}{1.400000in}}{\pgfqpoint{2.407767in}{1.544118in}}%
\pgfusepath{clip}%
\pgfsetroundcap%
\pgfsetroundjoin%
\definecolor{currentfill}{rgb}{0.276022,0.044167,0.370164}%
\pgfsetfillcolor{currentfill}%
\pgfsetlinewidth{0.501875pt}%
\definecolor{currentstroke}{rgb}{0.276022,0.044167,0.370164}%
\pgfsetstrokecolor{currentstroke}%
\pgfsetdash{}{0pt}%
\pgfpathmoveto{\pgfqpoint{4.937809in}{1.588661in}}%
\pgfpathlineto{\pgfqpoint{4.910565in}{1.603569in}}%
\pgfpathlineto{\pgfqpoint{4.938838in}{1.616420in}}%
\pgfpathlineto{\pgfqpoint{4.937809in}{1.588661in}}%
\pgfpathlineto{\pgfqpoint{4.937809in}{1.588661in}}%
\pgfpathclose%
\pgfusepath{stroke,fill}%
\end{pgfscope}%
\begin{pgfscope}%
\pgfpathrectangle{\pgfqpoint{3.352233in}{1.400000in}}{\pgfqpoint{2.407767in}{1.544118in}}%
\pgfusepath{clip}%
\pgfsetroundcap%
\pgfsetroundjoin%
\pgfsetlinewidth{0.501875pt}%
\definecolor{currentstroke}{rgb}{0.268510,0.009605,0.335427}%
\pgfsetstrokecolor{currentstroke}%
\pgfsetdash{}{0pt}%
\pgfpathmoveto{\pgfqpoint{4.494519in}{2.692503in}}%
\pgfpathquadraticcurveto{\pgfqpoint{4.494425in}{2.691269in}}{\pgfqpoint{4.494918in}{2.697777in}}%
\pgfusepath{stroke}%
\end{pgfscope}%
\begin{pgfscope}%
\pgfpathrectangle{\pgfqpoint{3.352233in}{1.400000in}}{\pgfqpoint{2.407767in}{1.544118in}}%
\pgfusepath{clip}%
\pgfsetroundcap%
\pgfsetroundjoin%
\definecolor{currentfill}{rgb}{0.268510,0.009605,0.335427}%
\pgfsetfillcolor{currentfill}%
\pgfsetlinewidth{0.501875pt}%
\definecolor{currentstroke}{rgb}{0.268510,0.009605,0.335427}%
\pgfsetstrokecolor{currentstroke}%
\pgfsetdash{}{0pt}%
\pgfpathmoveto{\pgfqpoint{4.510867in}{2.724426in}}%
\pgfpathlineto{\pgfqpoint{4.494918in}{2.697777in}}%
\pgfpathlineto{\pgfqpoint{4.483168in}{2.726525in}}%
\pgfpathlineto{\pgfqpoint{4.510867in}{2.724426in}}%
\pgfpathlineto{\pgfqpoint{4.510867in}{2.724426in}}%
\pgfpathclose%
\pgfusepath{stroke,fill}%
\end{pgfscope}%
\begin{pgfscope}%
\pgfpathrectangle{\pgfqpoint{3.352233in}{1.400000in}}{\pgfqpoint{2.407767in}{1.544118in}}%
\pgfusepath{clip}%
\pgfsetroundcap%
\pgfsetroundjoin%
\pgfsetlinewidth{0.501875pt}%
\definecolor{currentstroke}{rgb}{0.278791,0.062145,0.386592}%
\pgfsetstrokecolor{currentstroke}%
\pgfsetdash{}{0pt}%
\pgfpathmoveto{\pgfqpoint{4.356502in}{1.849392in}}%
\pgfpathquadraticcurveto{\pgfqpoint{4.352074in}{1.857010in}}{\pgfqpoint{4.351548in}{1.857915in}}%
\pgfusepath{stroke}%
\end{pgfscope}%
\begin{pgfscope}%
\pgfpathrectangle{\pgfqpoint{3.352233in}{1.400000in}}{\pgfqpoint{2.407767in}{1.544118in}}%
\pgfusepath{clip}%
\pgfsetroundcap%
\pgfsetroundjoin%
\definecolor{currentfill}{rgb}{0.278791,0.062145,0.386592}%
\pgfsetfillcolor{currentfill}%
\pgfsetlinewidth{0.501875pt}%
\definecolor{currentstroke}{rgb}{0.278791,0.062145,0.386592}%
\pgfsetstrokecolor{currentstroke}%
\pgfsetdash{}{0pt}%
\pgfpathmoveto{\pgfqpoint{4.353500in}{1.826920in}}%
\pgfpathlineto{\pgfqpoint{4.351548in}{1.857915in}}%
\pgfpathlineto{\pgfqpoint{4.377515in}{1.840879in}}%
\pgfpathlineto{\pgfqpoint{4.353500in}{1.826920in}}%
\pgfpathlineto{\pgfqpoint{4.353500in}{1.826920in}}%
\pgfpathclose%
\pgfusepath{stroke,fill}%
\end{pgfscope}%
\begin{pgfscope}%
\pgfpathrectangle{\pgfqpoint{3.352233in}{1.400000in}}{\pgfqpoint{2.407767in}{1.544118in}}%
\pgfusepath{clip}%
\pgfsetroundcap%
\pgfsetroundjoin%
\pgfsetlinewidth{0.501875pt}%
\definecolor{currentstroke}{rgb}{0.273809,0.031497,0.358853}%
\pgfsetstrokecolor{currentstroke}%
\pgfsetdash{}{0pt}%
\pgfpathmoveto{\pgfqpoint{4.504268in}{1.693349in}}%
\pgfpathquadraticcurveto{\pgfqpoint{4.497452in}{1.697444in}}{\pgfqpoint{4.497292in}{1.697540in}}%
\pgfusepath{stroke}%
\end{pgfscope}%
\begin{pgfscope}%
\pgfpathrectangle{\pgfqpoint{3.352233in}{1.400000in}}{\pgfqpoint{2.407767in}{1.544118in}}%
\pgfusepath{clip}%
\pgfsetroundcap%
\pgfsetroundjoin%
\definecolor{currentfill}{rgb}{0.273809,0.031497,0.358853}%
\pgfsetfillcolor{currentfill}%
\pgfsetlinewidth{0.501875pt}%
\definecolor{currentstroke}{rgb}{0.273809,0.031497,0.358853}%
\pgfsetstrokecolor{currentstroke}%
\pgfsetdash{}{0pt}%
\pgfpathmoveto{\pgfqpoint{4.513949in}{1.671329in}}%
\pgfpathlineto{\pgfqpoint{4.497292in}{1.697540in}}%
\pgfpathlineto{\pgfqpoint{4.528256in}{1.695139in}}%
\pgfpathlineto{\pgfqpoint{4.513949in}{1.671329in}}%
\pgfpathlineto{\pgfqpoint{4.513949in}{1.671329in}}%
\pgfpathclose%
\pgfusepath{stroke,fill}%
\end{pgfscope}%
\begin{pgfscope}%
\pgfpathrectangle{\pgfqpoint{3.352233in}{1.400000in}}{\pgfqpoint{2.407767in}{1.544118in}}%
\pgfusepath{clip}%
\pgfsetroundcap%
\pgfsetroundjoin%
\pgfsetlinewidth{0.501875pt}%
\definecolor{currentstroke}{rgb}{0.276022,0.044167,0.370164}%
\pgfsetstrokecolor{currentstroke}%
\pgfsetdash{}{0pt}%
\pgfpathmoveto{\pgfqpoint{4.983986in}{1.642408in}}%
\pgfpathquadraticcurveto{\pgfqpoint{4.970748in}{1.642612in}}{\pgfqpoint{4.965273in}{1.642696in}}%
\pgfusepath{stroke}%
\end{pgfscope}%
\begin{pgfscope}%
\pgfpathrectangle{\pgfqpoint{3.352233in}{1.400000in}}{\pgfqpoint{2.407767in}{1.544118in}}%
\pgfusepath{clip}%
\pgfsetroundcap%
\pgfsetroundjoin%
\definecolor{currentfill}{rgb}{0.276022,0.044167,0.370164}%
\pgfsetfillcolor{currentfill}%
\pgfsetlinewidth{0.501875pt}%
\definecolor{currentstroke}{rgb}{0.276022,0.044167,0.370164}%
\pgfsetstrokecolor{currentstroke}%
\pgfsetdash{}{0pt}%
\pgfpathmoveto{\pgfqpoint{4.992834in}{1.628382in}}%
\pgfpathlineto{\pgfqpoint{4.965273in}{1.642696in}}%
\pgfpathlineto{\pgfqpoint{4.993261in}{1.656156in}}%
\pgfpathlineto{\pgfqpoint{4.992834in}{1.628382in}}%
\pgfpathlineto{\pgfqpoint{4.992834in}{1.628382in}}%
\pgfpathclose%
\pgfusepath{stroke,fill}%
\end{pgfscope}%
\begin{pgfscope}%
\pgfpathrectangle{\pgfqpoint{3.352233in}{1.400000in}}{\pgfqpoint{2.407767in}{1.544118in}}%
\pgfusepath{clip}%
\pgfsetroundcap%
\pgfsetroundjoin%
\pgfsetlinewidth{0.501875pt}%
\definecolor{currentstroke}{rgb}{0.273809,0.031497,0.358853}%
\pgfsetstrokecolor{currentstroke}%
\pgfsetdash{}{0pt}%
\pgfpathmoveto{\pgfqpoint{5.047520in}{2.691410in}}%
\pgfpathquadraticcurveto{\pgfqpoint{5.034281in}{2.691191in}}{\pgfqpoint{5.028805in}{2.691101in}}%
\pgfusepath{stroke}%
\end{pgfscope}%
\begin{pgfscope}%
\pgfpathrectangle{\pgfqpoint{3.352233in}{1.400000in}}{\pgfqpoint{2.407767in}{1.544118in}}%
\pgfusepath{clip}%
\pgfsetroundcap%
\pgfsetroundjoin%
\definecolor{currentfill}{rgb}{0.273809,0.031497,0.358853}%
\pgfsetfillcolor{currentfill}%
\pgfsetlinewidth{0.501875pt}%
\definecolor{currentstroke}{rgb}{0.273809,0.031497,0.358853}%
\pgfsetstrokecolor{currentstroke}%
\pgfsetdash{}{0pt}%
\pgfpathmoveto{\pgfqpoint{5.056809in}{2.677673in}}%
\pgfpathlineto{\pgfqpoint{5.028805in}{2.691101in}}%
\pgfpathlineto{\pgfqpoint{5.056349in}{2.705447in}}%
\pgfpathlineto{\pgfqpoint{5.056809in}{2.677673in}}%
\pgfpathlineto{\pgfqpoint{5.056809in}{2.677673in}}%
\pgfpathclose%
\pgfusepath{stroke,fill}%
\end{pgfscope}%
\begin{pgfscope}%
\pgfpathrectangle{\pgfqpoint{3.352233in}{1.400000in}}{\pgfqpoint{2.407767in}{1.544118in}}%
\pgfusepath{clip}%
\pgfsetroundcap%
\pgfsetroundjoin%
\pgfsetlinewidth{0.501875pt}%
\definecolor{currentstroke}{rgb}{0.280894,0.078907,0.402329}%
\pgfsetstrokecolor{currentstroke}%
\pgfsetdash{}{0pt}%
\pgfpathmoveto{\pgfqpoint{4.061071in}{2.167850in}}%
\pgfpathquadraticcurveto{\pgfqpoint{4.074289in}{2.167417in}}{\pgfqpoint{4.079747in}{2.167238in}}%
\pgfusepath{stroke}%
\end{pgfscope}%
\begin{pgfscope}%
\pgfpathrectangle{\pgfqpoint{3.352233in}{1.400000in}}{\pgfqpoint{2.407767in}{1.544118in}}%
\pgfusepath{clip}%
\pgfsetroundcap%
\pgfsetroundjoin%
\definecolor{currentfill}{rgb}{0.280894,0.078907,0.402329}%
\pgfsetfillcolor{currentfill}%
\pgfsetlinewidth{0.501875pt}%
\definecolor{currentstroke}{rgb}{0.280894,0.078907,0.402329}%
\pgfsetstrokecolor{currentstroke}%
\pgfsetdash{}{0pt}%
\pgfpathmoveto{\pgfqpoint{4.052439in}{2.182029in}}%
\pgfpathlineto{\pgfqpoint{4.079747in}{2.167238in}}%
\pgfpathlineto{\pgfqpoint{4.051529in}{2.154266in}}%
\pgfpathlineto{\pgfqpoint{4.052439in}{2.182029in}}%
\pgfpathlineto{\pgfqpoint{4.052439in}{2.182029in}}%
\pgfpathclose%
\pgfusepath{stroke,fill}%
\end{pgfscope}%
\begin{pgfscope}%
\pgfpathrectangle{\pgfqpoint{3.352233in}{1.400000in}}{\pgfqpoint{2.407767in}{1.544118in}}%
\pgfusepath{clip}%
\pgfsetroundcap%
\pgfsetroundjoin%
\pgfsetlinewidth{0.501875pt}%
\definecolor{currentstroke}{rgb}{0.274952,0.037752,0.364543}%
\pgfsetstrokecolor{currentstroke}%
\pgfsetdash{}{0pt}%
\pgfpathmoveto{\pgfqpoint{4.982731in}{1.675190in}}%
\pgfpathquadraticcurveto{\pgfqpoint{4.969493in}{1.675407in}}{\pgfqpoint{4.964018in}{1.675497in}}%
\pgfusepath{stroke}%
\end{pgfscope}%
\begin{pgfscope}%
\pgfpathrectangle{\pgfqpoint{3.352233in}{1.400000in}}{\pgfqpoint{2.407767in}{1.544118in}}%
\pgfusepath{clip}%
\pgfsetroundcap%
\pgfsetroundjoin%
\definecolor{currentfill}{rgb}{0.274952,0.037752,0.364543}%
\pgfsetfillcolor{currentfill}%
\pgfsetlinewidth{0.501875pt}%
\definecolor{currentstroke}{rgb}{0.274952,0.037752,0.364543}%
\pgfsetstrokecolor{currentstroke}%
\pgfsetdash{}{0pt}%
\pgfpathmoveto{\pgfqpoint{4.991564in}{1.661154in}}%
\pgfpathlineto{\pgfqpoint{4.964018in}{1.675497in}}%
\pgfpathlineto{\pgfqpoint{4.992020in}{1.688928in}}%
\pgfpathlineto{\pgfqpoint{4.991564in}{1.661154in}}%
\pgfpathlineto{\pgfqpoint{4.991564in}{1.661154in}}%
\pgfpathclose%
\pgfusepath{stroke,fill}%
\end{pgfscope}%
\begin{pgfscope}%
\pgfpathrectangle{\pgfqpoint{3.352233in}{1.400000in}}{\pgfqpoint{2.407767in}{1.544118in}}%
\pgfusepath{clip}%
\pgfsetroundcap%
\pgfsetroundjoin%
\pgfsetlinewidth{0.501875pt}%
\definecolor{currentstroke}{rgb}{0.272594,0.025563,0.353093}%
\pgfsetstrokecolor{currentstroke}%
\pgfsetdash{}{0pt}%
\pgfpathmoveto{\pgfqpoint{5.100395in}{2.660342in}}%
\pgfpathquadraticcurveto{\pgfqpoint{5.087161in}{2.660301in}}{\pgfqpoint{5.081692in}{2.660284in}}%
\pgfusepath{stroke}%
\end{pgfscope}%
\begin{pgfscope}%
\pgfpathrectangle{\pgfqpoint{3.352233in}{1.400000in}}{\pgfqpoint{2.407767in}{1.544118in}}%
\pgfusepath{clip}%
\pgfsetroundcap%
\pgfsetroundjoin%
\definecolor{currentfill}{rgb}{0.272594,0.025563,0.353093}%
\pgfsetfillcolor{currentfill}%
\pgfsetlinewidth{0.501875pt}%
\definecolor{currentstroke}{rgb}{0.272594,0.025563,0.353093}%
\pgfsetstrokecolor{currentstroke}%
\pgfsetdash{}{0pt}%
\pgfpathmoveto{\pgfqpoint{5.109512in}{2.646481in}}%
\pgfpathlineto{\pgfqpoint{5.081692in}{2.660284in}}%
\pgfpathlineto{\pgfqpoint{5.109427in}{2.674258in}}%
\pgfpathlineto{\pgfqpoint{5.109512in}{2.646481in}}%
\pgfpathlineto{\pgfqpoint{5.109512in}{2.646481in}}%
\pgfpathclose%
\pgfusepath{stroke,fill}%
\end{pgfscope}%
\begin{pgfscope}%
\pgfpathrectangle{\pgfqpoint{3.352233in}{1.400000in}}{\pgfqpoint{2.407767in}{1.544118in}}%
\pgfusepath{clip}%
\pgfsetroundcap%
\pgfsetroundjoin%
\pgfsetlinewidth{0.501875pt}%
\definecolor{currentstroke}{rgb}{0.273809,0.031497,0.358853}%
\pgfsetstrokecolor{currentstroke}%
\pgfsetdash{}{0pt}%
\pgfpathmoveto{\pgfqpoint{4.713845in}{2.668888in}}%
\pgfpathquadraticcurveto{\pgfqpoint{4.700736in}{2.668212in}}{\pgfqpoint{4.695382in}{2.667936in}}%
\pgfusepath{stroke}%
\end{pgfscope}%
\begin{pgfscope}%
\pgfpathrectangle{\pgfqpoint{3.352233in}{1.400000in}}{\pgfqpoint{2.407767in}{1.544118in}}%
\pgfusepath{clip}%
\pgfsetroundcap%
\pgfsetroundjoin%
\definecolor{currentfill}{rgb}{0.273809,0.031497,0.358853}%
\pgfsetfillcolor{currentfill}%
\pgfsetlinewidth{0.501875pt}%
\definecolor{currentstroke}{rgb}{0.273809,0.031497,0.358853}%
\pgfsetstrokecolor{currentstroke}%
\pgfsetdash{}{0pt}%
\pgfpathmoveto{\pgfqpoint{4.723838in}{2.655496in}}%
\pgfpathlineto{\pgfqpoint{4.695382in}{2.667936in}}%
\pgfpathlineto{\pgfqpoint{4.722408in}{2.683237in}}%
\pgfpathlineto{\pgfqpoint{4.723838in}{2.655496in}}%
\pgfpathlineto{\pgfqpoint{4.723838in}{2.655496in}}%
\pgfpathclose%
\pgfusepath{stroke,fill}%
\end{pgfscope}%
\begin{pgfscope}%
\pgfpathrectangle{\pgfqpoint{3.352233in}{1.400000in}}{\pgfqpoint{2.407767in}{1.544118in}}%
\pgfusepath{clip}%
\pgfsetroundcap%
\pgfsetroundjoin%
\pgfsetlinewidth{0.501875pt}%
\definecolor{currentstroke}{rgb}{0.279574,0.170599,0.479997}%
\pgfsetstrokecolor{currentstroke}%
\pgfsetdash{}{0pt}%
\pgfpathmoveto{\pgfqpoint{3.956406in}{2.137792in}}%
\pgfpathquadraticcurveto{\pgfqpoint{3.969639in}{2.137691in}}{\pgfqpoint{3.975109in}{2.137649in}}%
\pgfusepath{stroke}%
\end{pgfscope}%
\begin{pgfscope}%
\pgfpathrectangle{\pgfqpoint{3.352233in}{1.400000in}}{\pgfqpoint{2.407767in}{1.544118in}}%
\pgfusepath{clip}%
\pgfsetroundcap%
\pgfsetroundjoin%
\definecolor{currentfill}{rgb}{0.279574,0.170599,0.479997}%
\pgfsetfillcolor{currentfill}%
\pgfsetlinewidth{0.501875pt}%
\definecolor{currentstroke}{rgb}{0.279574,0.170599,0.479997}%
\pgfsetstrokecolor{currentstroke}%
\pgfsetdash{}{0pt}%
\pgfpathmoveto{\pgfqpoint{3.947438in}{2.151750in}}%
\pgfpathlineto{\pgfqpoint{3.975109in}{2.137649in}}%
\pgfpathlineto{\pgfqpoint{3.947226in}{2.123973in}}%
\pgfpathlineto{\pgfqpoint{3.947438in}{2.151750in}}%
\pgfpathlineto{\pgfqpoint{3.947438in}{2.151750in}}%
\pgfpathclose%
\pgfusepath{stroke,fill}%
\end{pgfscope}%
\begin{pgfscope}%
\pgfpathrectangle{\pgfqpoint{3.352233in}{1.400000in}}{\pgfqpoint{2.407767in}{1.544118in}}%
\pgfusepath{clip}%
\pgfsetroundcap%
\pgfsetroundjoin%
\pgfsetlinewidth{0.501875pt}%
\definecolor{currentstroke}{rgb}{0.278012,0.180367,0.486697}%
\pgfsetstrokecolor{currentstroke}%
\pgfsetdash{}{0pt}%
\pgfpathmoveto{\pgfqpoint{3.956378in}{2.105604in}}%
\pgfpathquadraticcurveto{\pgfqpoint{3.969606in}{2.105863in}}{\pgfqpoint{3.975071in}{2.105970in}}%
\pgfusepath{stroke}%
\end{pgfscope}%
\begin{pgfscope}%
\pgfpathrectangle{\pgfqpoint{3.352233in}{1.400000in}}{\pgfqpoint{2.407767in}{1.544118in}}%
\pgfusepath{clip}%
\pgfsetroundcap%
\pgfsetroundjoin%
\definecolor{currentfill}{rgb}{0.278012,0.180367,0.486697}%
\pgfsetfillcolor{currentfill}%
\pgfsetlinewidth{0.501875pt}%
\definecolor{currentstroke}{rgb}{0.278012,0.180367,0.486697}%
\pgfsetstrokecolor{currentstroke}%
\pgfsetdash{}{0pt}%
\pgfpathmoveto{\pgfqpoint{3.947027in}{2.119313in}}%
\pgfpathlineto{\pgfqpoint{3.975071in}{2.105970in}}%
\pgfpathlineto{\pgfqpoint{3.947570in}{2.091541in}}%
\pgfpathlineto{\pgfqpoint{3.947027in}{2.119313in}}%
\pgfpathlineto{\pgfqpoint{3.947027in}{2.119313in}}%
\pgfpathclose%
\pgfusepath{stroke,fill}%
\end{pgfscope}%
\begin{pgfscope}%
\pgfpathrectangle{\pgfqpoint{3.352233in}{1.400000in}}{\pgfqpoint{2.407767in}{1.544118in}}%
\pgfusepath{clip}%
\pgfsetroundcap%
\pgfsetroundjoin%
\pgfsetlinewidth{0.501875pt}%
\definecolor{currentstroke}{rgb}{0.277134,0.185228,0.489898}%
\pgfsetstrokecolor{currentstroke}%
\pgfsetdash{}{0pt}%
\pgfpathmoveto{\pgfqpoint{3.956406in}{2.071076in}}%
\pgfpathquadraticcurveto{\pgfqpoint{3.969636in}{2.071395in}}{\pgfqpoint{3.975104in}{2.071527in}}%
\pgfusepath{stroke}%
\end{pgfscope}%
\begin{pgfscope}%
\pgfpathrectangle{\pgfqpoint{3.352233in}{1.400000in}}{\pgfqpoint{2.407767in}{1.544118in}}%
\pgfusepath{clip}%
\pgfsetroundcap%
\pgfsetroundjoin%
\definecolor{currentfill}{rgb}{0.277134,0.185228,0.489898}%
\pgfsetfillcolor{currentfill}%
\pgfsetlinewidth{0.501875pt}%
\definecolor{currentstroke}{rgb}{0.277134,0.185228,0.489898}%
\pgfsetstrokecolor{currentstroke}%
\pgfsetdash{}{0pt}%
\pgfpathmoveto{\pgfqpoint{3.946999in}{2.084742in}}%
\pgfpathlineto{\pgfqpoint{3.975104in}{2.071527in}}%
\pgfpathlineto{\pgfqpoint{3.947669in}{2.056972in}}%
\pgfpathlineto{\pgfqpoint{3.946999in}{2.084742in}}%
\pgfpathlineto{\pgfqpoint{3.946999in}{2.084742in}}%
\pgfpathclose%
\pgfusepath{stroke,fill}%
\end{pgfscope}%
\begin{pgfscope}%
\pgfpathrectangle{\pgfqpoint{3.352233in}{1.400000in}}{\pgfqpoint{2.407767in}{1.544118in}}%
\pgfusepath{clip}%
\pgfsetroundcap%
\pgfsetroundjoin%
\pgfsetlinewidth{0.501875pt}%
\definecolor{currentstroke}{rgb}{0.277941,0.056324,0.381191}%
\pgfsetstrokecolor{currentstroke}%
\pgfsetdash{}{0pt}%
\pgfpathmoveto{\pgfqpoint{4.311095in}{1.766208in}}%
\pgfpathquadraticcurveto{\pgfqpoint{4.314828in}{1.770229in}}{\pgfqpoint{4.313278in}{1.768560in}}%
\pgfusepath{stroke}%
\end{pgfscope}%
\begin{pgfscope}%
\pgfpathrectangle{\pgfqpoint{3.352233in}{1.400000in}}{\pgfqpoint{2.407767in}{1.544118in}}%
\pgfusepath{clip}%
\pgfsetroundcap%
\pgfsetroundjoin%
\definecolor{currentfill}{rgb}{0.277941,0.056324,0.381191}%
\pgfsetfillcolor{currentfill}%
\pgfsetlinewidth{0.501875pt}%
\definecolor{currentstroke}{rgb}{0.277941,0.056324,0.381191}%
\pgfsetstrokecolor{currentstroke}%
\pgfsetdash{}{0pt}%
\pgfpathmoveto{\pgfqpoint{4.284201in}{1.757650in}}%
\pgfpathlineto{\pgfqpoint{4.313278in}{1.768560in}}%
\pgfpathlineto{\pgfqpoint{4.304560in}{1.738752in}}%
\pgfpathlineto{\pgfqpoint{4.284201in}{1.757650in}}%
\pgfpathlineto{\pgfqpoint{4.284201in}{1.757650in}}%
\pgfpathclose%
\pgfusepath{stroke,fill}%
\end{pgfscope}%
\begin{pgfscope}%
\pgfpathrectangle{\pgfqpoint{3.352233in}{1.400000in}}{\pgfqpoint{2.407767in}{1.544118in}}%
\pgfusepath{clip}%
\pgfsetroundcap%
\pgfsetroundjoin%
\pgfsetlinewidth{0.501875pt}%
\definecolor{currentstroke}{rgb}{0.274952,0.037752,0.364543}%
\pgfsetstrokecolor{currentstroke}%
\pgfsetdash{}{0pt}%
\pgfpathmoveto{\pgfqpoint{4.716495in}{1.714136in}}%
\pgfpathquadraticcurveto{\pgfqpoint{4.703491in}{1.715692in}}{\pgfqpoint{4.698195in}{1.716326in}}%
\pgfusepath{stroke}%
\end{pgfscope}%
\begin{pgfscope}%
\pgfpathrectangle{\pgfqpoint{3.352233in}{1.400000in}}{\pgfqpoint{2.407767in}{1.544118in}}%
\pgfusepath{clip}%
\pgfsetroundcap%
\pgfsetroundjoin%
\definecolor{currentfill}{rgb}{0.274952,0.037752,0.364543}%
\pgfsetfillcolor{currentfill}%
\pgfsetlinewidth{0.501875pt}%
\definecolor{currentstroke}{rgb}{0.274952,0.037752,0.364543}%
\pgfsetstrokecolor{currentstroke}%
\pgfsetdash{}{0pt}%
\pgfpathmoveto{\pgfqpoint{4.724127in}{1.699236in}}%
\pgfpathlineto{\pgfqpoint{4.698195in}{1.716326in}}%
\pgfpathlineto{\pgfqpoint{4.727426in}{1.726817in}}%
\pgfpathlineto{\pgfqpoint{4.724127in}{1.699236in}}%
\pgfpathlineto{\pgfqpoint{4.724127in}{1.699236in}}%
\pgfpathclose%
\pgfusepath{stroke,fill}%
\end{pgfscope}%
\begin{pgfscope}%
\pgfpathrectangle{\pgfqpoint{3.352233in}{1.400000in}}{\pgfqpoint{2.407767in}{1.544118in}}%
\pgfusepath{clip}%
\pgfsetroundcap%
\pgfsetroundjoin%
\pgfsetlinewidth{0.501875pt}%
\definecolor{currentstroke}{rgb}{0.276022,0.044167,0.370164}%
\pgfsetstrokecolor{currentstroke}%
\pgfsetdash{}{0pt}%
\pgfpathmoveto{\pgfqpoint{4.994507in}{2.623765in}}%
\pgfpathquadraticcurveto{\pgfqpoint{4.981274in}{2.623474in}}{\pgfqpoint{4.975803in}{2.623354in}}%
\pgfusepath{stroke}%
\end{pgfscope}%
\begin{pgfscope}%
\pgfpathrectangle{\pgfqpoint{3.352233in}{1.400000in}}{\pgfqpoint{2.407767in}{1.544118in}}%
\pgfusepath{clip}%
\pgfsetroundcap%
\pgfsetroundjoin%
\definecolor{currentfill}{rgb}{0.276022,0.044167,0.370164}%
\pgfsetfillcolor{currentfill}%
\pgfsetlinewidth{0.501875pt}%
\definecolor{currentstroke}{rgb}{0.276022,0.044167,0.370164}%
\pgfsetstrokecolor{currentstroke}%
\pgfsetdash{}{0pt}%
\pgfpathmoveto{\pgfqpoint{5.003879in}{2.610078in}}%
\pgfpathlineto{\pgfqpoint{4.975803in}{2.623354in}}%
\pgfpathlineto{\pgfqpoint{5.003269in}{2.637849in}}%
\pgfpathlineto{\pgfqpoint{5.003879in}{2.610078in}}%
\pgfpathlineto{\pgfqpoint{5.003879in}{2.610078in}}%
\pgfpathclose%
\pgfusepath{stroke,fill}%
\end{pgfscope}%
\begin{pgfscope}%
\pgfpathrectangle{\pgfqpoint{3.352233in}{1.400000in}}{\pgfqpoint{2.407767in}{1.544118in}}%
\pgfusepath{clip}%
\pgfsetroundcap%
\pgfsetroundjoin%
\pgfsetlinewidth{0.501875pt}%
\definecolor{currentstroke}{rgb}{0.280267,0.073417,0.397163}%
\pgfsetstrokecolor{currentstroke}%
\pgfsetdash{}{0pt}%
\pgfpathmoveto{\pgfqpoint{4.390384in}{2.407815in}}%
\pgfpathquadraticcurveto{\pgfqpoint{4.383671in}{2.401528in}}{\pgfqpoint{4.382624in}{2.400548in}}%
\pgfusepath{stroke}%
\end{pgfscope}%
\begin{pgfscope}%
\pgfpathrectangle{\pgfqpoint{3.352233in}{1.400000in}}{\pgfqpoint{2.407767in}{1.544118in}}%
\pgfusepath{clip}%
\pgfsetroundcap%
\pgfsetroundjoin%
\definecolor{currentfill}{rgb}{0.280267,0.073417,0.397163}%
\pgfsetfillcolor{currentfill}%
\pgfsetlinewidth{0.501875pt}%
\definecolor{currentstroke}{rgb}{0.280267,0.073417,0.397163}%
\pgfsetstrokecolor{currentstroke}%
\pgfsetdash{}{0pt}%
\pgfpathmoveto{\pgfqpoint{4.412393in}{2.409398in}}%
\pgfpathlineto{\pgfqpoint{4.382624in}{2.400548in}}%
\pgfpathlineto{\pgfqpoint{4.393406in}{2.429673in}}%
\pgfpathlineto{\pgfqpoint{4.412393in}{2.409398in}}%
\pgfpathlineto{\pgfqpoint{4.412393in}{2.409398in}}%
\pgfpathclose%
\pgfusepath{stroke,fill}%
\end{pgfscope}%
\begin{pgfscope}%
\pgfpathrectangle{\pgfqpoint{3.352233in}{1.400000in}}{\pgfqpoint{2.407767in}{1.544118in}}%
\pgfusepath{clip}%
\pgfsetroundcap%
\pgfsetroundjoin%
\pgfsetlinewidth{0.501875pt}%
\definecolor{currentstroke}{rgb}{0.280894,0.078907,0.402329}%
\pgfsetstrokecolor{currentstroke}%
\pgfsetdash{}{0pt}%
\pgfpathmoveto{\pgfqpoint{4.278874in}{1.814441in}}%
\pgfpathquadraticcurveto{\pgfqpoint{4.282550in}{1.818740in}}{\pgfqpoint{4.281181in}{1.817139in}}%
\pgfusepath{stroke}%
\end{pgfscope}%
\begin{pgfscope}%
\pgfpathrectangle{\pgfqpoint{3.352233in}{1.400000in}}{\pgfqpoint{2.407767in}{1.544118in}}%
\pgfusepath{clip}%
\pgfsetroundcap%
\pgfsetroundjoin%
\definecolor{currentfill}{rgb}{0.280894,0.078907,0.402329}%
\pgfsetfillcolor{currentfill}%
\pgfsetlinewidth{0.501875pt}%
\definecolor{currentstroke}{rgb}{0.280894,0.078907,0.402329}%
\pgfsetstrokecolor{currentstroke}%
\pgfsetdash{}{0pt}%
\pgfpathmoveto{\pgfqpoint{4.252571in}{1.805056in}}%
\pgfpathlineto{\pgfqpoint{4.281181in}{1.817139in}}%
\pgfpathlineto{\pgfqpoint{4.273681in}{1.787002in}}%
\pgfpathlineto{\pgfqpoint{4.252571in}{1.805056in}}%
\pgfpathlineto{\pgfqpoint{4.252571in}{1.805056in}}%
\pgfpathclose%
\pgfusepath{stroke,fill}%
\end{pgfscope}%
\begin{pgfscope}%
\pgfpathrectangle{\pgfqpoint{3.352233in}{1.400000in}}{\pgfqpoint{2.407767in}{1.544118in}}%
\pgfusepath{clip}%
\pgfsetroundcap%
\pgfsetroundjoin%
\pgfsetlinewidth{0.501875pt}%
\definecolor{currentstroke}{rgb}{0.278791,0.062145,0.386592}%
\pgfsetstrokecolor{currentstroke}%
\pgfsetdash{}{0pt}%
\pgfpathmoveto{\pgfqpoint{4.769128in}{1.744811in}}%
\pgfpathquadraticcurveto{\pgfqpoint{4.756000in}{1.745920in}}{\pgfqpoint{4.750609in}{1.746376in}}%
\pgfusepath{stroke}%
\end{pgfscope}%
\begin{pgfscope}%
\pgfpathrectangle{\pgfqpoint{3.352233in}{1.400000in}}{\pgfqpoint{2.407767in}{1.544118in}}%
\pgfusepath{clip}%
\pgfsetroundcap%
\pgfsetroundjoin%
\definecolor{currentfill}{rgb}{0.278791,0.062145,0.386592}%
\pgfsetfillcolor{currentfill}%
\pgfsetlinewidth{0.501875pt}%
\definecolor{currentstroke}{rgb}{0.278791,0.062145,0.386592}%
\pgfsetstrokecolor{currentstroke}%
\pgfsetdash{}{0pt}%
\pgfpathmoveto{\pgfqpoint{4.777119in}{1.730197in}}%
\pgfpathlineto{\pgfqpoint{4.750609in}{1.746376in}}%
\pgfpathlineto{\pgfqpoint{4.779458in}{1.757876in}}%
\pgfpathlineto{\pgfqpoint{4.777119in}{1.730197in}}%
\pgfpathlineto{\pgfqpoint{4.777119in}{1.730197in}}%
\pgfpathclose%
\pgfusepath{stroke,fill}%
\end{pgfscope}%
\begin{pgfscope}%
\pgfpathrectangle{\pgfqpoint{3.352233in}{1.400000in}}{\pgfqpoint{2.407767in}{1.544118in}}%
\pgfusepath{clip}%
\pgfsetroundcap%
\pgfsetroundjoin%
\pgfsetlinewidth{0.501875pt}%
\definecolor{currentstroke}{rgb}{0.273809,0.031497,0.358853}%
\pgfsetstrokecolor{currentstroke}%
\pgfsetdash{}{0pt}%
\pgfpathmoveto{\pgfqpoint{5.094269in}{1.753528in}}%
\pgfpathquadraticcurveto{\pgfqpoint{5.081027in}{1.753606in}}{\pgfqpoint{5.075550in}{1.753639in}}%
\pgfusepath{stroke}%
\end{pgfscope}%
\begin{pgfscope}%
\pgfpathrectangle{\pgfqpoint{3.352233in}{1.400000in}}{\pgfqpoint{2.407767in}{1.544118in}}%
\pgfusepath{clip}%
\pgfsetroundcap%
\pgfsetroundjoin%
\definecolor{currentfill}{rgb}{0.273809,0.031497,0.358853}%
\pgfsetfillcolor{currentfill}%
\pgfsetlinewidth{0.501875pt}%
\definecolor{currentstroke}{rgb}{0.273809,0.031497,0.358853}%
\pgfsetstrokecolor{currentstroke}%
\pgfsetdash{}{0pt}%
\pgfpathmoveto{\pgfqpoint{5.103245in}{1.739586in}}%
\pgfpathlineto{\pgfqpoint{5.075550in}{1.753639in}}%
\pgfpathlineto{\pgfqpoint{5.103409in}{1.767363in}}%
\pgfpathlineto{\pgfqpoint{5.103245in}{1.739586in}}%
\pgfpathlineto{\pgfqpoint{5.103245in}{1.739586in}}%
\pgfpathclose%
\pgfusepath{stroke,fill}%
\end{pgfscope}%
\begin{pgfscope}%
\pgfpathrectangle{\pgfqpoint{3.352233in}{1.400000in}}{\pgfqpoint{2.407767in}{1.544118in}}%
\pgfusepath{clip}%
\pgfsetroundcap%
\pgfsetroundjoin%
\pgfsetlinewidth{0.501875pt}%
\definecolor{currentstroke}{rgb}{0.281446,0.084320,0.407414}%
\pgfsetstrokecolor{currentstroke}%
\pgfsetdash{}{0pt}%
\pgfpathmoveto{\pgfqpoint{4.782751in}{1.796687in}}%
\pgfpathquadraticcurveto{\pgfqpoint{4.769547in}{1.797340in}}{\pgfqpoint{4.764098in}{1.797609in}}%
\pgfusepath{stroke}%
\end{pgfscope}%
\begin{pgfscope}%
\pgfpathrectangle{\pgfqpoint{3.352233in}{1.400000in}}{\pgfqpoint{2.407767in}{1.544118in}}%
\pgfusepath{clip}%
\pgfsetroundcap%
\pgfsetroundjoin%
\definecolor{currentfill}{rgb}{0.281446,0.084320,0.407414}%
\pgfsetfillcolor{currentfill}%
\pgfsetlinewidth{0.501875pt}%
\definecolor{currentstroke}{rgb}{0.281446,0.084320,0.407414}%
\pgfsetstrokecolor{currentstroke}%
\pgfsetdash{}{0pt}%
\pgfpathmoveto{\pgfqpoint{4.791156in}{1.782366in}}%
\pgfpathlineto{\pgfqpoint{4.764098in}{1.797609in}}%
\pgfpathlineto{\pgfqpoint{4.792527in}{1.810110in}}%
\pgfpathlineto{\pgfqpoint{4.791156in}{1.782366in}}%
\pgfpathlineto{\pgfqpoint{4.791156in}{1.782366in}}%
\pgfpathclose%
\pgfusepath{stroke,fill}%
\end{pgfscope}%
\begin{pgfscope}%
\pgfpathrectangle{\pgfqpoint{3.352233in}{1.400000in}}{\pgfqpoint{2.407767in}{1.544118in}}%
\pgfusepath{clip}%
\pgfsetroundcap%
\pgfsetroundjoin%
\pgfsetlinewidth{0.501875pt}%
\definecolor{currentstroke}{rgb}{0.281446,0.084320,0.407414}%
\pgfsetstrokecolor{currentstroke}%
\pgfsetdash{}{0pt}%
\pgfpathmoveto{\pgfqpoint{4.941472in}{1.825666in}}%
\pgfpathquadraticcurveto{\pgfqpoint{4.928232in}{1.825855in}}{\pgfqpoint{4.922756in}{1.825934in}}%
\pgfusepath{stroke}%
\end{pgfscope}%
\begin{pgfscope}%
\pgfpathrectangle{\pgfqpoint{3.352233in}{1.400000in}}{\pgfqpoint{2.407767in}{1.544118in}}%
\pgfusepath{clip}%
\pgfsetroundcap%
\pgfsetroundjoin%
\definecolor{currentfill}{rgb}{0.281446,0.084320,0.407414}%
\pgfsetfillcolor{currentfill}%
\pgfsetlinewidth{0.501875pt}%
\definecolor{currentstroke}{rgb}{0.281446,0.084320,0.407414}%
\pgfsetstrokecolor{currentstroke}%
\pgfsetdash{}{0pt}%
\pgfpathmoveto{\pgfqpoint{4.950333in}{1.811649in}}%
\pgfpathlineto{\pgfqpoint{4.922756in}{1.825934in}}%
\pgfpathlineto{\pgfqpoint{4.950729in}{1.839424in}}%
\pgfpathlineto{\pgfqpoint{4.950333in}{1.811649in}}%
\pgfpathlineto{\pgfqpoint{4.950333in}{1.811649in}}%
\pgfpathclose%
\pgfusepath{stroke,fill}%
\end{pgfscope}%
\begin{pgfscope}%
\pgfpathrectangle{\pgfqpoint{3.352233in}{1.400000in}}{\pgfqpoint{2.407767in}{1.544118in}}%
\pgfusepath{clip}%
\pgfsetroundcap%
\pgfsetroundjoin%
\pgfsetlinewidth{0.501875pt}%
\definecolor{currentstroke}{rgb}{0.283187,0.125848,0.444960}%
\pgfsetstrokecolor{currentstroke}%
\pgfsetdash{}{0pt}%
\pgfpathmoveto{\pgfqpoint{4.729779in}{1.867367in}}%
\pgfpathquadraticcurveto{\pgfqpoint{4.716585in}{1.868083in}}{\pgfqpoint{4.711143in}{1.868379in}}%
\pgfusepath{stroke}%
\end{pgfscope}%
\begin{pgfscope}%
\pgfpathrectangle{\pgfqpoint{3.352233in}{1.400000in}}{\pgfqpoint{2.407767in}{1.544118in}}%
\pgfusepath{clip}%
\pgfsetroundcap%
\pgfsetroundjoin%
\definecolor{currentfill}{rgb}{0.283187,0.125848,0.444960}%
\pgfsetfillcolor{currentfill}%
\pgfsetlinewidth{0.501875pt}%
\definecolor{currentstroke}{rgb}{0.283187,0.125848,0.444960}%
\pgfsetstrokecolor{currentstroke}%
\pgfsetdash{}{0pt}%
\pgfpathmoveto{\pgfqpoint{4.738126in}{1.853004in}}%
\pgfpathlineto{\pgfqpoint{4.711143in}{1.868379in}}%
\pgfpathlineto{\pgfqpoint{4.739633in}{1.880740in}}%
\pgfpathlineto{\pgfqpoint{4.738126in}{1.853004in}}%
\pgfpathlineto{\pgfqpoint{4.738126in}{1.853004in}}%
\pgfpathclose%
\pgfusepath{stroke,fill}%
\end{pgfscope}%
\begin{pgfscope}%
\pgfpathrectangle{\pgfqpoint{3.352233in}{1.400000in}}{\pgfqpoint{2.407767in}{1.544118in}}%
\pgfusepath{clip}%
\pgfsetroundcap%
\pgfsetroundjoin%
\pgfsetlinewidth{0.501875pt}%
\definecolor{currentstroke}{rgb}{0.282327,0.094955,0.417331}%
\pgfsetstrokecolor{currentstroke}%
\pgfsetdash{}{0pt}%
\pgfpathmoveto{\pgfqpoint{4.941422in}{1.895618in}}%
\pgfpathquadraticcurveto{\pgfqpoint{4.928181in}{1.895784in}}{\pgfqpoint{4.922704in}{1.895853in}}%
\pgfusepath{stroke}%
\end{pgfscope}%
\begin{pgfscope}%
\pgfpathrectangle{\pgfqpoint{3.352233in}{1.400000in}}{\pgfqpoint{2.407767in}{1.544118in}}%
\pgfusepath{clip}%
\pgfsetroundcap%
\pgfsetroundjoin%
\definecolor{currentfill}{rgb}{0.282327,0.094955,0.417331}%
\pgfsetfillcolor{currentfill}%
\pgfsetlinewidth{0.501875pt}%
\definecolor{currentstroke}{rgb}{0.282327,0.094955,0.417331}%
\pgfsetstrokecolor{currentstroke}%
\pgfsetdash{}{0pt}%
\pgfpathmoveto{\pgfqpoint{4.950305in}{1.881616in}}%
\pgfpathlineto{\pgfqpoint{4.922704in}{1.895853in}}%
\pgfpathlineto{\pgfqpoint{4.950654in}{1.909392in}}%
\pgfpathlineto{\pgfqpoint{4.950305in}{1.881616in}}%
\pgfpathlineto{\pgfqpoint{4.950305in}{1.881616in}}%
\pgfpathclose%
\pgfusepath{stroke,fill}%
\end{pgfscope}%
\begin{pgfscope}%
\pgfpathrectangle{\pgfqpoint{3.352233in}{1.400000in}}{\pgfqpoint{2.407767in}{1.544118in}}%
\pgfusepath{clip}%
\pgfsetroundcap%
\pgfsetroundjoin%
\pgfsetlinewidth{0.501875pt}%
\definecolor{currentstroke}{rgb}{0.283187,0.125848,0.444960}%
\pgfsetstrokecolor{currentstroke}%
\pgfsetdash{}{0pt}%
\pgfpathmoveto{\pgfqpoint{4.835567in}{1.933904in}}%
\pgfpathquadraticcurveto{\pgfqpoint{4.822334in}{1.934244in}}{\pgfqpoint{4.816862in}{1.934384in}}%
\pgfusepath{stroke}%
\end{pgfscope}%
\begin{pgfscope}%
\pgfpathrectangle{\pgfqpoint{3.352233in}{1.400000in}}{\pgfqpoint{2.407767in}{1.544118in}}%
\pgfusepath{clip}%
\pgfsetroundcap%
\pgfsetroundjoin%
\definecolor{currentfill}{rgb}{0.283187,0.125848,0.444960}%
\pgfsetfillcolor{currentfill}%
\pgfsetlinewidth{0.501875pt}%
\definecolor{currentstroke}{rgb}{0.283187,0.125848,0.444960}%
\pgfsetstrokecolor{currentstroke}%
\pgfsetdash{}{0pt}%
\pgfpathmoveto{\pgfqpoint{4.844275in}{1.919787in}}%
\pgfpathlineto{\pgfqpoint{4.816862in}{1.934384in}}%
\pgfpathlineto{\pgfqpoint{4.844987in}{1.947556in}}%
\pgfpathlineto{\pgfqpoint{4.844275in}{1.919787in}}%
\pgfpathlineto{\pgfqpoint{4.844275in}{1.919787in}}%
\pgfpathclose%
\pgfusepath{stroke,fill}%
\end{pgfscope}%
\begin{pgfscope}%
\pgfpathrectangle{\pgfqpoint{3.352233in}{1.400000in}}{\pgfqpoint{2.407767in}{1.544118in}}%
\pgfusepath{clip}%
\pgfsetroundcap%
\pgfsetroundjoin%
\pgfsetlinewidth{0.501875pt}%
\definecolor{currentstroke}{rgb}{0.283187,0.125848,0.444960}%
\pgfsetstrokecolor{currentstroke}%
\pgfsetdash{}{0pt}%
\pgfpathmoveto{\pgfqpoint{4.888474in}{1.966340in}}%
\pgfpathquadraticcurveto{\pgfqpoint{4.875235in}{1.966557in}}{\pgfqpoint{4.869759in}{1.966647in}}%
\pgfusepath{stroke}%
\end{pgfscope}%
\begin{pgfscope}%
\pgfpathrectangle{\pgfqpoint{3.352233in}{1.400000in}}{\pgfqpoint{2.407767in}{1.544118in}}%
\pgfusepath{clip}%
\pgfsetroundcap%
\pgfsetroundjoin%
\definecolor{currentfill}{rgb}{0.283187,0.125848,0.444960}%
\pgfsetfillcolor{currentfill}%
\pgfsetlinewidth{0.501875pt}%
\definecolor{currentstroke}{rgb}{0.283187,0.125848,0.444960}%
\pgfsetstrokecolor{currentstroke}%
\pgfsetdash{}{0pt}%
\pgfpathmoveto{\pgfqpoint{4.897305in}{1.952304in}}%
\pgfpathlineto{\pgfqpoint{4.869759in}{1.966647in}}%
\pgfpathlineto{\pgfqpoint{4.897761in}{1.980078in}}%
\pgfpathlineto{\pgfqpoint{4.897305in}{1.952304in}}%
\pgfpathlineto{\pgfqpoint{4.897305in}{1.952304in}}%
\pgfpathclose%
\pgfusepath{stroke,fill}%
\end{pgfscope}%
\begin{pgfscope}%
\pgfpathrectangle{\pgfqpoint{3.352233in}{1.400000in}}{\pgfqpoint{2.407767in}{1.544118in}}%
\pgfusepath{clip}%
\pgfsetroundcap%
\pgfsetroundjoin%
\pgfsetlinewidth{0.501875pt}%
\definecolor{currentstroke}{rgb}{0.278826,0.175490,0.483397}%
\pgfsetstrokecolor{currentstroke}%
\pgfsetdash{}{0pt}%
\pgfpathmoveto{\pgfqpoint{4.729565in}{2.001627in}}%
\pgfpathquadraticcurveto{\pgfqpoint{4.716328in}{2.001896in}}{\pgfqpoint{4.710854in}{2.002007in}}%
\pgfusepath{stroke}%
\end{pgfscope}%
\begin{pgfscope}%
\pgfpathrectangle{\pgfqpoint{3.352233in}{1.400000in}}{\pgfqpoint{2.407767in}{1.544118in}}%
\pgfusepath{clip}%
\pgfsetroundcap%
\pgfsetroundjoin%
\definecolor{currentfill}{rgb}{0.278826,0.175490,0.483397}%
\pgfsetfillcolor{currentfill}%
\pgfsetlinewidth{0.501875pt}%
\definecolor{currentstroke}{rgb}{0.278826,0.175490,0.483397}%
\pgfsetstrokecolor{currentstroke}%
\pgfsetdash{}{0pt}%
\pgfpathmoveto{\pgfqpoint{4.738344in}{1.987557in}}%
\pgfpathlineto{\pgfqpoint{4.710854in}{2.002007in}}%
\pgfpathlineto{\pgfqpoint{4.738908in}{2.015329in}}%
\pgfpathlineto{\pgfqpoint{4.738344in}{1.987557in}}%
\pgfpathlineto{\pgfqpoint{4.738344in}{1.987557in}}%
\pgfpathclose%
\pgfusepath{stroke,fill}%
\end{pgfscope}%
\begin{pgfscope}%
\pgfpathrectangle{\pgfqpoint{3.352233in}{1.400000in}}{\pgfqpoint{2.407767in}{1.544118in}}%
\pgfusepath{clip}%
\pgfsetroundcap%
\pgfsetroundjoin%
\pgfsetlinewidth{0.501875pt}%
\definecolor{currentstroke}{rgb}{0.280868,0.160771,0.472899}%
\pgfsetstrokecolor{currentstroke}%
\pgfsetdash{}{0pt}%
\pgfpathmoveto{\pgfqpoint{4.835462in}{2.033725in}}%
\pgfpathquadraticcurveto{\pgfqpoint{4.822218in}{2.033717in}}{\pgfqpoint{4.816738in}{2.033714in}}%
\pgfusepath{stroke}%
\end{pgfscope}%
\begin{pgfscope}%
\pgfpathrectangle{\pgfqpoint{3.352233in}{1.400000in}}{\pgfqpoint{2.407767in}{1.544118in}}%
\pgfusepath{clip}%
\pgfsetroundcap%
\pgfsetroundjoin%
\definecolor{currentfill}{rgb}{0.280868,0.160771,0.472899}%
\pgfsetfillcolor{currentfill}%
\pgfsetlinewidth{0.501875pt}%
\definecolor{currentstroke}{rgb}{0.280868,0.160771,0.472899}%
\pgfsetstrokecolor{currentstroke}%
\pgfsetdash{}{0pt}%
\pgfpathmoveto{\pgfqpoint{4.844523in}{2.019840in}}%
\pgfpathlineto{\pgfqpoint{4.816738in}{2.033714in}}%
\pgfpathlineto{\pgfqpoint{4.844508in}{2.047618in}}%
\pgfpathlineto{\pgfqpoint{4.844523in}{2.019840in}}%
\pgfpathlineto{\pgfqpoint{4.844523in}{2.019840in}}%
\pgfpathclose%
\pgfusepath{stroke,fill}%
\end{pgfscope}%
\begin{pgfscope}%
\pgfpathrectangle{\pgfqpoint{3.352233in}{1.400000in}}{\pgfqpoint{2.407767in}{1.544118in}}%
\pgfusepath{clip}%
\pgfsetroundcap%
\pgfsetroundjoin%
\pgfsetlinewidth{0.501875pt}%
\definecolor{currentstroke}{rgb}{0.279574,0.170599,0.479997}%
\pgfsetstrokecolor{currentstroke}%
\pgfsetdash{}{0pt}%
\pgfpathmoveto{\pgfqpoint{4.782500in}{2.067872in}}%
\pgfpathquadraticcurveto{\pgfqpoint{4.769256in}{2.067913in}}{\pgfqpoint{4.763776in}{2.067930in}}%
\pgfusepath{stroke}%
\end{pgfscope}%
\begin{pgfscope}%
\pgfpathrectangle{\pgfqpoint{3.352233in}{1.400000in}}{\pgfqpoint{2.407767in}{1.544118in}}%
\pgfusepath{clip}%
\pgfsetroundcap%
\pgfsetroundjoin%
\definecolor{currentfill}{rgb}{0.279574,0.170599,0.479997}%
\pgfsetfillcolor{currentfill}%
\pgfsetlinewidth{0.501875pt}%
\definecolor{currentstroke}{rgb}{0.279574,0.170599,0.479997}%
\pgfsetstrokecolor{currentstroke}%
\pgfsetdash{}{0pt}%
\pgfpathmoveto{\pgfqpoint{4.791511in}{2.053955in}}%
\pgfpathlineto{\pgfqpoint{4.763776in}{2.067930in}}%
\pgfpathlineto{\pgfqpoint{4.791597in}{2.081733in}}%
\pgfpathlineto{\pgfqpoint{4.791511in}{2.053955in}}%
\pgfpathlineto{\pgfqpoint{4.791511in}{2.053955in}}%
\pgfpathclose%
\pgfusepath{stroke,fill}%
\end{pgfscope}%
\begin{pgfscope}%
\pgfpathrectangle{\pgfqpoint{3.352233in}{1.400000in}}{\pgfqpoint{2.407767in}{1.544118in}}%
\pgfusepath{clip}%
\pgfsetroundcap%
\pgfsetroundjoin%
\pgfsetlinewidth{0.501875pt}%
\definecolor{currentstroke}{rgb}{0.278826,0.175490,0.483397}%
\pgfsetstrokecolor{currentstroke}%
\pgfsetdash{}{0pt}%
\pgfpathmoveto{\pgfqpoint{4.782489in}{2.101704in}}%
\pgfpathquadraticcurveto{\pgfqpoint{4.769246in}{2.101738in}}{\pgfqpoint{4.763766in}{2.101752in}}%
\pgfusepath{stroke}%
\end{pgfscope}%
\begin{pgfscope}%
\pgfpathrectangle{\pgfqpoint{3.352233in}{1.400000in}}{\pgfqpoint{2.407767in}{1.544118in}}%
\pgfusepath{clip}%
\pgfsetroundcap%
\pgfsetroundjoin%
\definecolor{currentfill}{rgb}{0.278826,0.175490,0.483397}%
\pgfsetfillcolor{currentfill}%
\pgfsetlinewidth{0.501875pt}%
\definecolor{currentstroke}{rgb}{0.278826,0.175490,0.483397}%
\pgfsetstrokecolor{currentstroke}%
\pgfsetdash{}{0pt}%
\pgfpathmoveto{\pgfqpoint{4.791508in}{2.087792in}}%
\pgfpathlineto{\pgfqpoint{4.763766in}{2.101752in}}%
\pgfpathlineto{\pgfqpoint{4.791579in}{2.115570in}}%
\pgfpathlineto{\pgfqpoint{4.791508in}{2.087792in}}%
\pgfpathlineto{\pgfqpoint{4.791508in}{2.087792in}}%
\pgfpathclose%
\pgfusepath{stroke,fill}%
\end{pgfscope}%
\begin{pgfscope}%
\pgfpathrectangle{\pgfqpoint{3.352233in}{1.400000in}}{\pgfqpoint{2.407767in}{1.544118in}}%
\pgfusepath{clip}%
\pgfsetroundcap%
\pgfsetroundjoin%
\pgfsetlinewidth{0.501875pt}%
\definecolor{currentstroke}{rgb}{0.280868,0.160771,0.472899}%
\pgfsetstrokecolor{currentstroke}%
\pgfsetdash{}{0pt}%
\pgfpathmoveto{\pgfqpoint{4.782503in}{2.136894in}}%
\pgfpathquadraticcurveto{\pgfqpoint{4.769259in}{2.136882in}}{\pgfqpoint{4.763779in}{2.136877in}}%
\pgfusepath{stroke}%
\end{pgfscope}%
\begin{pgfscope}%
\pgfpathrectangle{\pgfqpoint{3.352233in}{1.400000in}}{\pgfqpoint{2.407767in}{1.544118in}}%
\pgfusepath{clip}%
\pgfsetroundcap%
\pgfsetroundjoin%
\definecolor{currentfill}{rgb}{0.280868,0.160771,0.472899}%
\pgfsetfillcolor{currentfill}%
\pgfsetlinewidth{0.501875pt}%
\definecolor{currentstroke}{rgb}{0.280868,0.160771,0.472899}%
\pgfsetstrokecolor{currentstroke}%
\pgfsetdash{}{0pt}%
\pgfpathmoveto{\pgfqpoint{4.791569in}{2.123013in}}%
\pgfpathlineto{\pgfqpoint{4.763779in}{2.136877in}}%
\pgfpathlineto{\pgfqpoint{4.791544in}{2.150790in}}%
\pgfpathlineto{\pgfqpoint{4.791569in}{2.123013in}}%
\pgfpathlineto{\pgfqpoint{4.791569in}{2.123013in}}%
\pgfpathclose%
\pgfusepath{stroke,fill}%
\end{pgfscope}%
\begin{pgfscope}%
\pgfpathrectangle{\pgfqpoint{3.352233in}{1.400000in}}{\pgfqpoint{2.407767in}{1.544118in}}%
\pgfusepath{clip}%
\pgfsetroundcap%
\pgfsetroundjoin%
\pgfsetlinewidth{0.501875pt}%
\definecolor{currentstroke}{rgb}{0.281412,0.155834,0.469201}%
\pgfsetstrokecolor{currentstroke}%
\pgfsetdash{}{0pt}%
\pgfpathmoveto{\pgfqpoint{4.782501in}{2.172643in}}%
\pgfpathquadraticcurveto{\pgfqpoint{4.769257in}{2.172608in}}{\pgfqpoint{4.763777in}{2.172594in}}%
\pgfusepath{stroke}%
\end{pgfscope}%
\begin{pgfscope}%
\pgfpathrectangle{\pgfqpoint{3.352233in}{1.400000in}}{\pgfqpoint{2.407767in}{1.544118in}}%
\pgfusepath{clip}%
\pgfsetroundcap%
\pgfsetroundjoin%
\definecolor{currentfill}{rgb}{0.281412,0.155834,0.469201}%
\pgfsetfillcolor{currentfill}%
\pgfsetlinewidth{0.501875pt}%
\definecolor{currentstroke}{rgb}{0.281412,0.155834,0.469201}%
\pgfsetstrokecolor{currentstroke}%
\pgfsetdash{}{0pt}%
\pgfpathmoveto{\pgfqpoint{4.791592in}{2.158778in}}%
\pgfpathlineto{\pgfqpoint{4.763777in}{2.172594in}}%
\pgfpathlineto{\pgfqpoint{4.791519in}{2.186556in}}%
\pgfpathlineto{\pgfqpoint{4.791592in}{2.158778in}}%
\pgfpathlineto{\pgfqpoint{4.791592in}{2.158778in}}%
\pgfpathclose%
\pgfusepath{stroke,fill}%
\end{pgfscope}%
\begin{pgfscope}%
\pgfpathrectangle{\pgfqpoint{3.352233in}{1.400000in}}{\pgfqpoint{2.407767in}{1.544118in}}%
\pgfusepath{clip}%
\pgfsetroundcap%
\pgfsetroundjoin%
\pgfsetlinewidth{0.501875pt}%
\definecolor{currentstroke}{rgb}{0.278826,0.175490,0.483397}%
\pgfsetstrokecolor{currentstroke}%
\pgfsetdash{}{0pt}%
\pgfpathmoveto{\pgfqpoint{4.782531in}{2.204930in}}%
\pgfpathquadraticcurveto{\pgfqpoint{4.769288in}{2.204869in}}{\pgfqpoint{4.763808in}{2.204844in}}%
\pgfusepath{stroke}%
\end{pgfscope}%
\begin{pgfscope}%
\pgfpathrectangle{\pgfqpoint{3.352233in}{1.400000in}}{\pgfqpoint{2.407767in}{1.544118in}}%
\pgfusepath{clip}%
\pgfsetroundcap%
\pgfsetroundjoin%
\definecolor{currentfill}{rgb}{0.278826,0.175490,0.483397}%
\pgfsetfillcolor{currentfill}%
\pgfsetlinewidth{0.501875pt}%
\definecolor{currentstroke}{rgb}{0.278826,0.175490,0.483397}%
\pgfsetstrokecolor{currentstroke}%
\pgfsetdash{}{0pt}%
\pgfpathmoveto{\pgfqpoint{4.791650in}{2.191083in}}%
\pgfpathlineto{\pgfqpoint{4.763808in}{2.204844in}}%
\pgfpathlineto{\pgfqpoint{4.791522in}{2.218860in}}%
\pgfpathlineto{\pgfqpoint{4.791650in}{2.191083in}}%
\pgfpathlineto{\pgfqpoint{4.791650in}{2.191083in}}%
\pgfpathclose%
\pgfusepath{stroke,fill}%
\end{pgfscope}%
\begin{pgfscope}%
\pgfpathrectangle{\pgfqpoint{3.352233in}{1.400000in}}{\pgfqpoint{2.407767in}{1.544118in}}%
\pgfusepath{clip}%
\pgfsetroundcap%
\pgfsetroundjoin%
\pgfsetlinewidth{0.501875pt}%
\definecolor{currentstroke}{rgb}{0.279574,0.170599,0.479997}%
\pgfsetstrokecolor{currentstroke}%
\pgfsetdash{}{0pt}%
\pgfpathmoveto{\pgfqpoint{4.782504in}{2.242194in}}%
\pgfpathquadraticcurveto{\pgfqpoint{4.769261in}{2.242083in}}{\pgfqpoint{4.763783in}{2.242037in}}%
\pgfusepath{stroke}%
\end{pgfscope}%
\begin{pgfscope}%
\pgfpathrectangle{\pgfqpoint{3.352233in}{1.400000in}}{\pgfqpoint{2.407767in}{1.544118in}}%
\pgfusepath{clip}%
\pgfsetroundcap%
\pgfsetroundjoin%
\definecolor{currentfill}{rgb}{0.279574,0.170599,0.479997}%
\pgfsetfillcolor{currentfill}%
\pgfsetlinewidth{0.501875pt}%
\definecolor{currentstroke}{rgb}{0.279574,0.170599,0.479997}%
\pgfsetstrokecolor{currentstroke}%
\pgfsetdash{}{0pt}%
\pgfpathmoveto{\pgfqpoint{4.791676in}{2.228381in}}%
\pgfpathlineto{\pgfqpoint{4.763783in}{2.242037in}}%
\pgfpathlineto{\pgfqpoint{4.791443in}{2.256158in}}%
\pgfpathlineto{\pgfqpoint{4.791676in}{2.228381in}}%
\pgfpathlineto{\pgfqpoint{4.791676in}{2.228381in}}%
\pgfpathclose%
\pgfusepath{stroke,fill}%
\end{pgfscope}%
\begin{pgfscope}%
\pgfpathrectangle{\pgfqpoint{3.352233in}{1.400000in}}{\pgfqpoint{2.407767in}{1.544118in}}%
\pgfusepath{clip}%
\pgfsetroundcap%
\pgfsetroundjoin%
\pgfsetlinewidth{0.501875pt}%
\definecolor{currentstroke}{rgb}{0.282623,0.140926,0.457517}%
\pgfsetstrokecolor{currentstroke}%
\pgfsetdash{}{0pt}%
\pgfpathmoveto{\pgfqpoint{4.835496in}{2.274974in}}%
\pgfpathquadraticcurveto{\pgfqpoint{4.822254in}{2.274817in}}{\pgfqpoint{4.816776in}{2.274753in}}%
\pgfusepath{stroke}%
\end{pgfscope}%
\begin{pgfscope}%
\pgfpathrectangle{\pgfqpoint{3.352233in}{1.400000in}}{\pgfqpoint{2.407767in}{1.544118in}}%
\pgfusepath{clip}%
\pgfsetroundcap%
\pgfsetroundjoin%
\definecolor{currentfill}{rgb}{0.282623,0.140926,0.457517}%
\pgfsetfillcolor{currentfill}%
\pgfsetlinewidth{0.501875pt}%
\definecolor{currentstroke}{rgb}{0.282623,0.140926,0.457517}%
\pgfsetstrokecolor{currentstroke}%
\pgfsetdash{}{0pt}%
\pgfpathmoveto{\pgfqpoint{4.844716in}{2.261193in}}%
\pgfpathlineto{\pgfqpoint{4.816776in}{2.274753in}}%
\pgfpathlineto{\pgfqpoint{4.844388in}{2.288969in}}%
\pgfpathlineto{\pgfqpoint{4.844716in}{2.261193in}}%
\pgfpathlineto{\pgfqpoint{4.844716in}{2.261193in}}%
\pgfpathclose%
\pgfusepath{stroke,fill}%
\end{pgfscope}%
\begin{pgfscope}%
\pgfpathrectangle{\pgfqpoint{3.352233in}{1.400000in}}{\pgfqpoint{2.407767in}{1.544118in}}%
\pgfusepath{clip}%
\pgfsetroundcap%
\pgfsetroundjoin%
\pgfsetlinewidth{0.501875pt}%
\definecolor{currentstroke}{rgb}{0.282910,0.105393,0.426902}%
\pgfsetstrokecolor{currentstroke}%
\pgfsetdash{}{0pt}%
\pgfpathmoveto{\pgfqpoint{4.888453in}{2.309298in}}%
\pgfpathquadraticcurveto{\pgfqpoint{4.875214in}{2.309080in}}{\pgfqpoint{4.869738in}{2.308989in}}%
\pgfusepath{stroke}%
\end{pgfscope}%
\begin{pgfscope}%
\pgfpathrectangle{\pgfqpoint{3.352233in}{1.400000in}}{\pgfqpoint{2.407767in}{1.544118in}}%
\pgfusepath{clip}%
\pgfsetroundcap%
\pgfsetroundjoin%
\definecolor{currentfill}{rgb}{0.282910,0.105393,0.426902}%
\pgfsetfillcolor{currentfill}%
\pgfsetlinewidth{0.501875pt}%
\definecolor{currentstroke}{rgb}{0.282910,0.105393,0.426902}%
\pgfsetstrokecolor{currentstroke}%
\pgfsetdash{}{0pt}%
\pgfpathmoveto{\pgfqpoint{4.897741in}{2.295561in}}%
\pgfpathlineto{\pgfqpoint{4.869738in}{2.308989in}}%
\pgfpathlineto{\pgfqpoint{4.897282in}{2.323335in}}%
\pgfpathlineto{\pgfqpoint{4.897741in}{2.295561in}}%
\pgfpathlineto{\pgfqpoint{4.897741in}{2.295561in}}%
\pgfpathclose%
\pgfusepath{stroke,fill}%
\end{pgfscope}%
\begin{pgfscope}%
\pgfpathrectangle{\pgfqpoint{3.352233in}{1.400000in}}{\pgfqpoint{2.407767in}{1.544118in}}%
\pgfusepath{clip}%
\pgfsetroundcap%
\pgfsetroundjoin%
\pgfsetlinewidth{0.501875pt}%
\definecolor{currentstroke}{rgb}{0.282327,0.094955,0.417331}%
\pgfsetstrokecolor{currentstroke}%
\pgfsetdash{}{0pt}%
\pgfpathmoveto{\pgfqpoint{4.941455in}{2.344069in}}%
\pgfpathquadraticcurveto{\pgfqpoint{4.928219in}{2.343780in}}{\pgfqpoint{4.922745in}{2.343660in}}%
\pgfusepath{stroke}%
\end{pgfscope}%
\begin{pgfscope}%
\pgfpathrectangle{\pgfqpoint{3.352233in}{1.400000in}}{\pgfqpoint{2.407767in}{1.544118in}}%
\pgfusepath{clip}%
\pgfsetroundcap%
\pgfsetroundjoin%
\definecolor{currentfill}{rgb}{0.282327,0.094955,0.417331}%
\pgfsetfillcolor{currentfill}%
\pgfsetlinewidth{0.501875pt}%
\definecolor{currentstroke}{rgb}{0.282327,0.094955,0.417331}%
\pgfsetstrokecolor{currentstroke}%
\pgfsetdash{}{0pt}%
\pgfpathmoveto{\pgfqpoint{4.950819in}{2.330381in}}%
\pgfpathlineto{\pgfqpoint{4.922745in}{2.343660in}}%
\pgfpathlineto{\pgfqpoint{4.950213in}{2.358152in}}%
\pgfpathlineto{\pgfqpoint{4.950819in}{2.330381in}}%
\pgfpathlineto{\pgfqpoint{4.950819in}{2.330381in}}%
\pgfpathclose%
\pgfusepath{stroke,fill}%
\end{pgfscope}%
\begin{pgfscope}%
\pgfpathrectangle{\pgfqpoint{3.352233in}{1.400000in}}{\pgfqpoint{2.407767in}{1.544118in}}%
\pgfusepath{clip}%
\pgfsetroundcap%
\pgfsetroundjoin%
\pgfsetlinewidth{0.501875pt}%
\definecolor{currentstroke}{rgb}{0.282656,0.100196,0.422160}%
\pgfsetstrokecolor{currentstroke}%
\pgfsetdash{}{0pt}%
\pgfpathmoveto{\pgfqpoint{4.782665in}{2.374405in}}%
\pgfpathquadraticcurveto{\pgfqpoint{4.769449in}{2.373883in}}{\pgfqpoint{4.763991in}{2.373667in}}%
\pgfusepath{stroke}%
\end{pgfscope}%
\begin{pgfscope}%
\pgfpathrectangle{\pgfqpoint{3.352233in}{1.400000in}}{\pgfqpoint{2.407767in}{1.544118in}}%
\pgfusepath{clip}%
\pgfsetroundcap%
\pgfsetroundjoin%
\definecolor{currentfill}{rgb}{0.282656,0.100196,0.422160}%
\pgfsetfillcolor{currentfill}%
\pgfsetlinewidth{0.501875pt}%
\definecolor{currentstroke}{rgb}{0.282656,0.100196,0.422160}%
\pgfsetstrokecolor{currentstroke}%
\pgfsetdash{}{0pt}%
\pgfpathmoveto{\pgfqpoint{4.792296in}{2.360886in}}%
\pgfpathlineto{\pgfqpoint{4.763991in}{2.373667in}}%
\pgfpathlineto{\pgfqpoint{4.791199in}{2.388642in}}%
\pgfpathlineto{\pgfqpoint{4.792296in}{2.360886in}}%
\pgfpathlineto{\pgfqpoint{4.792296in}{2.360886in}}%
\pgfpathclose%
\pgfusepath{stroke,fill}%
\end{pgfscope}%
\begin{pgfscope}%
\pgfpathrectangle{\pgfqpoint{3.352233in}{1.400000in}}{\pgfqpoint{2.407767in}{1.544118in}}%
\pgfusepath{clip}%
\pgfsetroundcap%
\pgfsetroundjoin%
\pgfsetlinewidth{0.501875pt}%
\definecolor{currentstroke}{rgb}{0.281924,0.089666,0.412415}%
\pgfsetstrokecolor{currentstroke}%
\pgfsetdash{}{0pt}%
\pgfpathmoveto{\pgfqpoint{4.941524in}{2.412105in}}%
\pgfpathquadraticcurveto{\pgfqpoint{4.928284in}{2.411934in}}{\pgfqpoint{4.922806in}{2.411864in}}%
\pgfusepath{stroke}%
\end{pgfscope}%
\begin{pgfscope}%
\pgfpathrectangle{\pgfqpoint{3.352233in}{1.400000in}}{\pgfqpoint{2.407767in}{1.544118in}}%
\pgfusepath{clip}%
\pgfsetroundcap%
\pgfsetroundjoin%
\definecolor{currentfill}{rgb}{0.281924,0.089666,0.412415}%
\pgfsetfillcolor{currentfill}%
\pgfsetlinewidth{0.501875pt}%
\definecolor{currentstroke}{rgb}{0.281924,0.089666,0.412415}%
\pgfsetstrokecolor{currentstroke}%
\pgfsetdash{}{0pt}%
\pgfpathmoveto{\pgfqpoint{4.950761in}{2.398334in}}%
\pgfpathlineto{\pgfqpoint{4.922806in}{2.411864in}}%
\pgfpathlineto{\pgfqpoint{4.950403in}{2.426109in}}%
\pgfpathlineto{\pgfqpoint{4.950761in}{2.398334in}}%
\pgfpathlineto{\pgfqpoint{4.950761in}{2.398334in}}%
\pgfpathclose%
\pgfusepath{stroke,fill}%
\end{pgfscope}%
\begin{pgfscope}%
\pgfpathrectangle{\pgfqpoint{3.352233in}{1.400000in}}{\pgfqpoint{2.407767in}{1.544118in}}%
\pgfusepath{clip}%
\pgfsetroundcap%
\pgfsetroundjoin%
\pgfsetlinewidth{0.501875pt}%
\definecolor{currentstroke}{rgb}{0.280267,0.073417,0.397163}%
\pgfsetstrokecolor{currentstroke}%
\pgfsetdash{}{0pt}%
\pgfpathmoveto{\pgfqpoint{4.729922in}{2.441031in}}%
\pgfpathquadraticcurveto{\pgfqpoint{4.716746in}{2.440180in}}{\pgfqpoint{4.711318in}{2.439829in}}%
\pgfusepath{stroke}%
\end{pgfscope}%
\begin{pgfscope}%
\pgfpathrectangle{\pgfqpoint{3.352233in}{1.400000in}}{\pgfqpoint{2.407767in}{1.544118in}}%
\pgfusepath{clip}%
\pgfsetroundcap%
\pgfsetroundjoin%
\definecolor{currentfill}{rgb}{0.280267,0.073417,0.397163}%
\pgfsetfillcolor{currentfill}%
\pgfsetlinewidth{0.501875pt}%
\definecolor{currentstroke}{rgb}{0.280267,0.073417,0.397163}%
\pgfsetstrokecolor{currentstroke}%
\pgfsetdash{}{0pt}%
\pgfpathmoveto{\pgfqpoint{4.739934in}{2.427761in}}%
\pgfpathlineto{\pgfqpoint{4.711318in}{2.439829in}}%
\pgfpathlineto{\pgfqpoint{4.738142in}{2.455481in}}%
\pgfpathlineto{\pgfqpoint{4.739934in}{2.427761in}}%
\pgfpathlineto{\pgfqpoint{4.739934in}{2.427761in}}%
\pgfpathclose%
\pgfusepath{stroke,fill}%
\end{pgfscope}%
\begin{pgfscope}%
\pgfpathrectangle{\pgfqpoint{3.352233in}{1.400000in}}{\pgfqpoint{2.407767in}{1.544118in}}%
\pgfusepath{clip}%
\pgfsetroundcap%
\pgfsetroundjoin%
\pgfsetlinewidth{0.501875pt}%
\definecolor{currentstroke}{rgb}{0.278791,0.062145,0.386592}%
\pgfsetstrokecolor{currentstroke}%
\pgfsetdash{}{0pt}%
\pgfpathmoveto{\pgfqpoint{4.941453in}{2.482675in}}%
\pgfpathquadraticcurveto{\pgfqpoint{4.928226in}{2.482278in}}{\pgfqpoint{4.922759in}{2.482114in}}%
\pgfusepath{stroke}%
\end{pgfscope}%
\begin{pgfscope}%
\pgfpathrectangle{\pgfqpoint{3.352233in}{1.400000in}}{\pgfqpoint{2.407767in}{1.544118in}}%
\pgfusepath{clip}%
\pgfsetroundcap%
\pgfsetroundjoin%
\definecolor{currentfill}{rgb}{0.278791,0.062145,0.386592}%
\pgfsetfillcolor{currentfill}%
\pgfsetlinewidth{0.501875pt}%
\definecolor{currentstroke}{rgb}{0.278791,0.062145,0.386592}%
\pgfsetstrokecolor{currentstroke}%
\pgfsetdash{}{0pt}%
\pgfpathmoveto{\pgfqpoint{4.950940in}{2.469064in}}%
\pgfpathlineto{\pgfqpoint{4.922759in}{2.482114in}}%
\pgfpathlineto{\pgfqpoint{4.950107in}{2.496829in}}%
\pgfpathlineto{\pgfqpoint{4.950940in}{2.469064in}}%
\pgfpathlineto{\pgfqpoint{4.950940in}{2.469064in}}%
\pgfpathclose%
\pgfusepath{stroke,fill}%
\end{pgfscope}%
\begin{pgfscope}%
\pgfpathrectangle{\pgfqpoint{3.352233in}{1.400000in}}{\pgfqpoint{2.407767in}{1.544118in}}%
\pgfusepath{clip}%
\pgfsetroundcap%
\pgfsetroundjoin%
\pgfsetlinewidth{0.501875pt}%
\definecolor{currentstroke}{rgb}{0.278791,0.062145,0.386592}%
\pgfsetstrokecolor{currentstroke}%
\pgfsetdash{}{0pt}%
\pgfpathmoveto{\pgfqpoint{4.835670in}{2.514743in}}%
\pgfpathquadraticcurveto{\pgfqpoint{4.822463in}{2.514151in}}{\pgfqpoint{4.817011in}{2.513907in}}%
\pgfusepath{stroke}%
\end{pgfscope}%
\begin{pgfscope}%
\pgfpathrectangle{\pgfqpoint{3.352233in}{1.400000in}}{\pgfqpoint{2.407767in}{1.544118in}}%
\pgfusepath{clip}%
\pgfsetroundcap%
\pgfsetroundjoin%
\definecolor{currentfill}{rgb}{0.278791,0.062145,0.386592}%
\pgfsetfillcolor{currentfill}%
\pgfsetlinewidth{0.501875pt}%
\definecolor{currentstroke}{rgb}{0.278791,0.062145,0.386592}%
\pgfsetstrokecolor{currentstroke}%
\pgfsetdash{}{0pt}%
\pgfpathmoveto{\pgfqpoint{4.845383in}{2.501275in}}%
\pgfpathlineto{\pgfqpoint{4.817011in}{2.513907in}}%
\pgfpathlineto{\pgfqpoint{4.844140in}{2.529025in}}%
\pgfpathlineto{\pgfqpoint{4.845383in}{2.501275in}}%
\pgfpathlineto{\pgfqpoint{4.845383in}{2.501275in}}%
\pgfpathclose%
\pgfusepath{stroke,fill}%
\end{pgfscope}%
\begin{pgfscope}%
\pgfpathrectangle{\pgfqpoint{3.352233in}{1.400000in}}{\pgfqpoint{2.407767in}{1.544118in}}%
\pgfusepath{clip}%
\pgfsetroundcap%
\pgfsetroundjoin%
\pgfsetlinewidth{0.501875pt}%
\definecolor{currentstroke}{rgb}{0.278791,0.062145,0.386592}%
\pgfsetstrokecolor{currentstroke}%
\pgfsetdash{}{0pt}%
\pgfpathmoveto{\pgfqpoint{4.888515in}{2.550130in}}%
\pgfpathquadraticcurveto{\pgfqpoint{4.875296in}{2.549696in}}{\pgfqpoint{4.869838in}{2.549516in}}%
\pgfusepath{stroke}%
\end{pgfscope}%
\begin{pgfscope}%
\pgfpathrectangle{\pgfqpoint{3.352233in}{1.400000in}}{\pgfqpoint{2.407767in}{1.544118in}}%
\pgfusepath{clip}%
\pgfsetroundcap%
\pgfsetroundjoin%
\definecolor{currentfill}{rgb}{0.278791,0.062145,0.386592}%
\pgfsetfillcolor{currentfill}%
\pgfsetlinewidth{0.501875pt}%
\definecolor{currentstroke}{rgb}{0.278791,0.062145,0.386592}%
\pgfsetstrokecolor{currentstroke}%
\pgfsetdash{}{0pt}%
\pgfpathmoveto{\pgfqpoint{4.898057in}{2.536548in}}%
\pgfpathlineto{\pgfqpoint{4.869838in}{2.549516in}}%
\pgfpathlineto{\pgfqpoint{4.897144in}{2.564310in}}%
\pgfpathlineto{\pgfqpoint{4.898057in}{2.536548in}}%
\pgfpathlineto{\pgfqpoint{4.898057in}{2.536548in}}%
\pgfpathclose%
\pgfusepath{stroke,fill}%
\end{pgfscope}%
\begin{pgfscope}%
\pgfpathrectangle{\pgfqpoint{3.352233in}{1.400000in}}{\pgfqpoint{2.407767in}{1.544118in}}%
\pgfusepath{clip}%
\pgfsetroundcap%
\pgfsetroundjoin%
\pgfsetlinewidth{0.501875pt}%
\definecolor{currentstroke}{rgb}{0.274952,0.037752,0.364543}%
\pgfsetstrokecolor{currentstroke}%
\pgfsetdash{}{0pt}%
\pgfpathmoveto{\pgfqpoint{4.994450in}{2.586952in}}%
\pgfpathquadraticcurveto{\pgfqpoint{4.981218in}{2.586689in}}{\pgfqpoint{4.975748in}{2.586580in}}%
\pgfusepath{stroke}%
\end{pgfscope}%
\begin{pgfscope}%
\pgfpathrectangle{\pgfqpoint{3.352233in}{1.400000in}}{\pgfqpoint{2.407767in}{1.544118in}}%
\pgfusepath{clip}%
\pgfsetroundcap%
\pgfsetroundjoin%
\definecolor{currentfill}{rgb}{0.274952,0.037752,0.364543}%
\pgfsetfillcolor{currentfill}%
\pgfsetlinewidth{0.501875pt}%
\definecolor{currentstroke}{rgb}{0.274952,0.037752,0.364543}%
\pgfsetstrokecolor{currentstroke}%
\pgfsetdash{}{0pt}%
\pgfpathmoveto{\pgfqpoint{5.003797in}{2.573247in}}%
\pgfpathlineto{\pgfqpoint{4.975748in}{2.586580in}}%
\pgfpathlineto{\pgfqpoint{5.003244in}{2.601019in}}%
\pgfpathlineto{\pgfqpoint{5.003797in}{2.573247in}}%
\pgfpathlineto{\pgfqpoint{5.003797in}{2.573247in}}%
\pgfpathclose%
\pgfusepath{stroke,fill}%
\end{pgfscope}%
\begin{pgfscope}%
\pgfpathrectangle{\pgfqpoint{3.352233in}{1.400000in}}{\pgfqpoint{2.407767in}{1.544118in}}%
\pgfusepath{clip}%
\pgfsetroundcap%
\pgfsetroundjoin%
\pgfsetlinewidth{0.501875pt}%
\definecolor{currentstroke}{rgb}{0.276022,0.044167,0.370164}%
\pgfsetstrokecolor{currentstroke}%
\pgfsetdash{}{0pt}%
\pgfpathmoveto{\pgfqpoint{4.285216in}{2.589012in}}%
\pgfpathquadraticcurveto{\pgfqpoint{4.297458in}{2.585923in}}{\pgfqpoint{4.302172in}{2.584733in}}%
\pgfusepath{stroke}%
\end{pgfscope}%
\begin{pgfscope}%
\pgfpathrectangle{\pgfqpoint{3.352233in}{1.400000in}}{\pgfqpoint{2.407767in}{1.544118in}}%
\pgfusepath{clip}%
\pgfsetroundcap%
\pgfsetroundjoin%
\definecolor{currentfill}{rgb}{0.276022,0.044167,0.370164}%
\pgfsetfillcolor{currentfill}%
\pgfsetlinewidth{0.501875pt}%
\definecolor{currentstroke}{rgb}{0.276022,0.044167,0.370164}%
\pgfsetstrokecolor{currentstroke}%
\pgfsetdash{}{0pt}%
\pgfpathmoveto{\pgfqpoint{4.278637in}{2.604997in}}%
\pgfpathlineto{\pgfqpoint{4.302172in}{2.584733in}}%
\pgfpathlineto{\pgfqpoint{4.271840in}{2.578064in}}%
\pgfpathlineto{\pgfqpoint{4.278637in}{2.604997in}}%
\pgfpathlineto{\pgfqpoint{4.278637in}{2.604997in}}%
\pgfpathclose%
\pgfusepath{stroke,fill}%
\end{pgfscope}%
\begin{pgfscope}%
\pgfpathrectangle{\pgfqpoint{3.352233in}{1.400000in}}{\pgfqpoint{2.407767in}{1.544118in}}%
\pgfusepath{clip}%
\pgfsetroundcap%
\pgfsetroundjoin%
\pgfsetlinewidth{0.501875pt}%
\definecolor{currentstroke}{rgb}{0.277941,0.056324,0.381191}%
\pgfsetstrokecolor{currentstroke}%
\pgfsetdash{}{0pt}%
\pgfpathmoveto{\pgfqpoint{4.328656in}{2.531187in}}%
\pgfpathquadraticcurveto{\pgfqpoint{4.336127in}{2.528102in}}{\pgfqpoint{4.336420in}{2.527981in}}%
\pgfusepath{stroke}%
\end{pgfscope}%
\begin{pgfscope}%
\pgfpathrectangle{\pgfqpoint{3.352233in}{1.400000in}}{\pgfqpoint{2.407767in}{1.544118in}}%
\pgfusepath{clip}%
\pgfsetroundcap%
\pgfsetroundjoin%
\definecolor{currentfill}{rgb}{0.277941,0.056324,0.381191}%
\pgfsetfillcolor{currentfill}%
\pgfsetlinewidth{0.501875pt}%
\definecolor{currentstroke}{rgb}{0.277941,0.056324,0.381191}%
\pgfsetstrokecolor{currentstroke}%
\pgfsetdash{}{0pt}%
\pgfpathmoveto{\pgfqpoint{4.316045in}{2.551419in}}%
\pgfpathlineto{\pgfqpoint{4.336420in}{2.527981in}}%
\pgfpathlineto{\pgfqpoint{4.305445in}{2.525743in}}%
\pgfpathlineto{\pgfqpoint{4.316045in}{2.551419in}}%
\pgfpathlineto{\pgfqpoint{4.316045in}{2.551419in}}%
\pgfpathclose%
\pgfusepath{stroke,fill}%
\end{pgfscope}%
\begin{pgfscope}%
\pgfpathrectangle{\pgfqpoint{3.352233in}{1.400000in}}{\pgfqpoint{2.407767in}{1.544118in}}%
\pgfusepath{clip}%
\pgfsetroundcap%
\pgfsetroundjoin%
\pgfsetlinewidth{0.501875pt}%
\definecolor{currentstroke}{rgb}{0.279574,0.170599,0.479997}%
\pgfsetstrokecolor{currentstroke}%
\pgfsetdash{}{0pt}%
\pgfpathmoveto{\pgfqpoint{3.962025in}{2.027787in}}%
\pgfpathquadraticcurveto{\pgfqpoint{3.975097in}{2.029109in}}{\pgfqpoint{3.980445in}{2.029650in}}%
\pgfusepath{stroke}%
\end{pgfscope}%
\begin{pgfscope}%
\pgfpathrectangle{\pgfqpoint{3.352233in}{1.400000in}}{\pgfqpoint{2.407767in}{1.544118in}}%
\pgfusepath{clip}%
\pgfsetroundcap%
\pgfsetroundjoin%
\definecolor{currentfill}{rgb}{0.279574,0.170599,0.479997}%
\pgfsetfillcolor{currentfill}%
\pgfsetlinewidth{0.501875pt}%
\definecolor{currentstroke}{rgb}{0.279574,0.170599,0.479997}%
\pgfsetstrokecolor{currentstroke}%
\pgfsetdash{}{0pt}%
\pgfpathmoveto{\pgfqpoint{3.951411in}{2.040674in}}%
\pgfpathlineto{\pgfqpoint{3.980445in}{2.029650in}}%
\pgfpathlineto{\pgfqpoint{3.954205in}{2.013037in}}%
\pgfpathlineto{\pgfqpoint{3.951411in}{2.040674in}}%
\pgfpathlineto{\pgfqpoint{3.951411in}{2.040674in}}%
\pgfpathclose%
\pgfusepath{stroke,fill}%
\end{pgfscope}%
\begin{pgfscope}%
\pgfpathrectangle{\pgfqpoint{3.352233in}{1.400000in}}{\pgfqpoint{2.407767in}{1.544118in}}%
\pgfusepath{clip}%
\pgfsetroundcap%
\pgfsetroundjoin%
\pgfsetlinewidth{0.501875pt}%
\definecolor{currentstroke}{rgb}{0.281924,0.089666,0.412415}%
\pgfsetstrokecolor{currentstroke}%
\pgfsetdash{}{0pt}%
\pgfpathmoveto{\pgfqpoint{4.233230in}{1.924274in}}%
\pgfpathquadraticcurveto{\pgfqpoint{4.236510in}{1.928220in}}{\pgfqpoint{4.234827in}{1.926195in}}%
\pgfusepath{stroke}%
\end{pgfscope}%
\begin{pgfscope}%
\pgfpathrectangle{\pgfqpoint{3.352233in}{1.400000in}}{\pgfqpoint{2.407767in}{1.544118in}}%
\pgfusepath{clip}%
\pgfsetroundcap%
\pgfsetroundjoin%
\definecolor{currentfill}{rgb}{0.281924,0.089666,0.412415}%
\pgfsetfillcolor{currentfill}%
\pgfsetlinewidth{0.501875pt}%
\definecolor{currentstroke}{rgb}{0.281924,0.089666,0.412415}%
\pgfsetstrokecolor{currentstroke}%
\pgfsetdash{}{0pt}%
\pgfpathmoveto{\pgfqpoint{4.206390in}{1.913711in}}%
\pgfpathlineto{\pgfqpoint{4.234827in}{1.926195in}}%
\pgfpathlineto{\pgfqpoint{4.227752in}{1.895955in}}%
\pgfpathlineto{\pgfqpoint{4.206390in}{1.913711in}}%
\pgfpathlineto{\pgfqpoint{4.206390in}{1.913711in}}%
\pgfpathclose%
\pgfusepath{stroke,fill}%
\end{pgfscope}%
\begin{pgfscope}%
\pgfpathrectangle{\pgfqpoint{3.352233in}{1.400000in}}{\pgfqpoint{2.407767in}{1.544118in}}%
\pgfusepath{clip}%
\pgfsetroundcap%
\pgfsetroundjoin%
\pgfsetlinewidth{0.501875pt}%
\definecolor{currentstroke}{rgb}{0.276022,0.044167,0.370164}%
\pgfsetstrokecolor{currentstroke}%
\pgfsetdash{}{0pt}%
\pgfpathmoveto{\pgfqpoint{4.358341in}{2.475964in}}%
\pgfpathquadraticcurveto{\pgfqpoint{4.359286in}{2.473531in}}{\pgfqpoint{4.357420in}{2.478335in}}%
\pgfusepath{stroke}%
\end{pgfscope}%
\begin{pgfscope}%
\pgfpathrectangle{\pgfqpoint{3.352233in}{1.400000in}}{\pgfqpoint{2.407767in}{1.544118in}}%
\pgfusepath{clip}%
\pgfsetroundcap%
\pgfsetroundjoin%
\definecolor{currentfill}{rgb}{0.276022,0.044167,0.370164}%
\pgfsetfillcolor{currentfill}%
\pgfsetlinewidth{0.501875pt}%
\definecolor{currentstroke}{rgb}{0.276022,0.044167,0.370164}%
\pgfsetstrokecolor{currentstroke}%
\pgfsetdash{}{0pt}%
\pgfpathmoveto{\pgfqpoint{4.360309in}{2.509257in}}%
\pgfpathlineto{\pgfqpoint{4.357420in}{2.478335in}}%
\pgfpathlineto{\pgfqpoint{4.334416in}{2.499200in}}%
\pgfpathlineto{\pgfqpoint{4.360309in}{2.509257in}}%
\pgfpathlineto{\pgfqpoint{4.360309in}{2.509257in}}%
\pgfpathclose%
\pgfusepath{stroke,fill}%
\end{pgfscope}%
\begin{pgfscope}%
\pgfpathrectangle{\pgfqpoint{3.352233in}{1.400000in}}{\pgfqpoint{2.407767in}{1.544118in}}%
\pgfusepath{clip}%
\pgfsetroundcap%
\pgfsetroundjoin%
\pgfsetlinewidth{0.501875pt}%
\definecolor{currentstroke}{rgb}{0.282910,0.105393,0.426902}%
\pgfsetstrokecolor{currentstroke}%
\pgfsetdash{}{0pt}%
\pgfpathmoveto{\pgfqpoint{4.189051in}{2.327901in}}%
\pgfpathquadraticcurveto{\pgfqpoint{4.196525in}{2.322087in}}{\pgfqpoint{4.197870in}{2.321041in}}%
\pgfusepath{stroke}%
\end{pgfscope}%
\begin{pgfscope}%
\pgfpathrectangle{\pgfqpoint{3.352233in}{1.400000in}}{\pgfqpoint{2.407767in}{1.544118in}}%
\pgfusepath{clip}%
\pgfsetroundcap%
\pgfsetroundjoin%
\definecolor{currentfill}{rgb}{0.282910,0.105393,0.426902}%
\pgfsetfillcolor{currentfill}%
\pgfsetlinewidth{0.501875pt}%
\definecolor{currentstroke}{rgb}{0.282910,0.105393,0.426902}%
\pgfsetstrokecolor{currentstroke}%
\pgfsetdash{}{0pt}%
\pgfpathmoveto{\pgfqpoint{4.184472in}{2.349059in}}%
\pgfpathlineto{\pgfqpoint{4.197870in}{2.321041in}}%
\pgfpathlineto{\pgfqpoint{4.167417in}{2.327134in}}%
\pgfpathlineto{\pgfqpoint{4.184472in}{2.349059in}}%
\pgfpathlineto{\pgfqpoint{4.184472in}{2.349059in}}%
\pgfpathclose%
\pgfusepath{stroke,fill}%
\end{pgfscope}%
\begin{pgfscope}%
\pgfpathrectangle{\pgfqpoint{3.352233in}{1.400000in}}{\pgfqpoint{2.407767in}{1.544118in}}%
\pgfusepath{clip}%
\pgfsetroundcap%
\pgfsetroundjoin%
\pgfsetlinewidth{0.501875pt}%
\definecolor{currentstroke}{rgb}{0.280894,0.078907,0.402329}%
\pgfsetstrokecolor{currentstroke}%
\pgfsetdash{}{0pt}%
\pgfpathmoveto{\pgfqpoint{4.257369in}{2.422035in}}%
\pgfpathquadraticcurveto{\pgfqpoint{4.264192in}{2.417558in}}{\pgfqpoint{4.264524in}{2.417341in}}%
\pgfusepath{stroke}%
\end{pgfscope}%
\begin{pgfscope}%
\pgfpathrectangle{\pgfqpoint{3.352233in}{1.400000in}}{\pgfqpoint{2.407767in}{1.544118in}}%
\pgfusepath{clip}%
\pgfsetroundcap%
\pgfsetroundjoin%
\definecolor{currentfill}{rgb}{0.280894,0.078907,0.402329}%
\pgfsetfillcolor{currentfill}%
\pgfsetlinewidth{0.501875pt}%
\definecolor{currentstroke}{rgb}{0.280894,0.078907,0.402329}%
\pgfsetstrokecolor{currentstroke}%
\pgfsetdash{}{0pt}%
\pgfpathmoveto{\pgfqpoint{4.248918in}{2.444192in}}%
\pgfpathlineto{\pgfqpoint{4.264524in}{2.417341in}}%
\pgfpathlineto{\pgfqpoint{4.233680in}{2.420967in}}%
\pgfpathlineto{\pgfqpoint{4.248918in}{2.444192in}}%
\pgfpathlineto{\pgfqpoint{4.248918in}{2.444192in}}%
\pgfpathclose%
\pgfusepath{stroke,fill}%
\end{pgfscope}%
\begin{pgfscope}%
\pgfpathrectangle{\pgfqpoint{3.352233in}{1.400000in}}{\pgfqpoint{2.407767in}{1.544118in}}%
\pgfusepath{clip}%
\pgfsetbuttcap%
\pgfsetroundjoin%
\pgfsetlinewidth{1.505625pt}%
\definecolor{currentstroke}{rgb}{0.000000,0.000000,0.000000}%
\pgfsetstrokecolor{currentstroke}%
\pgfsetdash{}{0pt}%
\pgfpathmoveto{\pgfqpoint{4.341943in}{1.661474in}}%
\pgfpathlineto{\pgfqpoint{4.341943in}{2.682644in}}%
\pgfusepath{stroke}%
\end{pgfscope}%
\begin{pgfscope}%
\pgfpathrectangle{\pgfqpoint{3.352233in}{1.400000in}}{\pgfqpoint{2.407767in}{1.544118in}}%
\pgfusepath{clip}%
\pgfsetbuttcap%
\pgfsetroundjoin%
\pgfsetlinewidth{1.505625pt}%
\definecolor{currentstroke}{rgb}{0.000000,0.000000,0.000000}%
\pgfsetstrokecolor{currentstroke}%
\pgfsetdash{}{0pt}%
\pgfpathmoveto{\pgfqpoint{5.251844in}{1.661474in}}%
\pgfpathlineto{\pgfqpoint{5.251844in}{2.682644in}}%
\pgfusepath{stroke}%
\end{pgfscope}%
\begin{pgfscope}%
\pgfsetrectcap%
\pgfsetmiterjoin%
\pgfsetlinewidth{0.803000pt}%
\definecolor{currentstroke}{rgb}{0.000000,0.000000,0.000000}%
\pgfsetstrokecolor{currentstroke}%
\pgfsetdash{}{0pt}%
\pgfpathmoveto{\pgfqpoint{3.352233in}{1.400000in}}%
\pgfpathlineto{\pgfqpoint{3.352233in}{2.944118in}}%
\pgfusepath{stroke}%
\end{pgfscope}%
\begin{pgfscope}%
\pgfsetrectcap%
\pgfsetmiterjoin%
\pgfsetlinewidth{0.803000pt}%
\definecolor{currentstroke}{rgb}{0.000000,0.000000,0.000000}%
\pgfsetstrokecolor{currentstroke}%
\pgfsetdash{}{0pt}%
\pgfpathmoveto{\pgfqpoint{5.760000in}{1.400000in}}%
\pgfpathlineto{\pgfqpoint{5.760000in}{2.944118in}}%
\pgfusepath{stroke}%
\end{pgfscope}%
\begin{pgfscope}%
\pgfsetrectcap%
\pgfsetmiterjoin%
\pgfsetlinewidth{0.803000pt}%
\definecolor{currentstroke}{rgb}{0.000000,0.000000,0.000000}%
\pgfsetstrokecolor{currentstroke}%
\pgfsetdash{}{0pt}%
\pgfpathmoveto{\pgfqpoint{3.352233in}{1.400000in}}%
\pgfpathlineto{\pgfqpoint{5.760000in}{1.400000in}}%
\pgfusepath{stroke}%
\end{pgfscope}%
\begin{pgfscope}%
\pgfsetrectcap%
\pgfsetmiterjoin%
\pgfsetlinewidth{0.803000pt}%
\definecolor{currentstroke}{rgb}{0.000000,0.000000,0.000000}%
\pgfsetstrokecolor{currentstroke}%
\pgfsetdash{}{0pt}%
\pgfpathmoveto{\pgfqpoint{3.352233in}{2.944118in}}%
\pgfpathlineto{\pgfqpoint{5.760000in}{2.944118in}}%
\pgfusepath{stroke}%
\end{pgfscope}%
\begin{pgfscope}%
\definecolor{textcolor}{rgb}{0.000000,0.000000,0.000000}%
\pgfsetstrokecolor{textcolor}%
\pgfsetfillcolor{textcolor}%
\pgftext[x=4.556117in,y=3.027451in,,base]{\color{textcolor}\sffamily\fontsize{12.000000}{14.400000}\selectfont f)}%
\end{pgfscope}%
\begin{pgfscope}%
\pgfsetbuttcap%
\pgfsetmiterjoin%
\definecolor{currentfill}{rgb}{1.000000,1.000000,1.000000}%
\pgfsetfillcolor{currentfill}%
\pgfsetlinewidth{0.000000pt}%
\definecolor{currentstroke}{rgb}{0.000000,0.000000,0.000000}%
\pgfsetstrokecolor{currentstroke}%
\pgfsetstrokeopacity{0.000000}%
\pgfsetdash{}{0pt}%
\pgfpathmoveto{\pgfqpoint{0.896000in}{0.700000in}}%
\pgfpathlineto{\pgfqpoint{2.944000in}{0.700000in}}%
\pgfpathlineto{\pgfqpoint{2.944000in}{0.910000in}}%
\pgfpathlineto{\pgfqpoint{0.896000in}{0.910000in}}%
\pgfpathlineto{\pgfqpoint{0.896000in}{0.700000in}}%
\pgfpathclose%
\pgfusepath{fill}%
\end{pgfscope}%
\begin{pgfscope}%
\pgfpathrectangle{\pgfqpoint{0.896000in}{0.700000in}}{\pgfqpoint{2.048000in}{0.210000in}}%
\pgfusepath{clip}%
\pgfsetbuttcap%
\pgfsetmiterjoin%
\definecolor{currentfill}{rgb}{1.000000,1.000000,1.000000}%
\pgfsetfillcolor{currentfill}%
\pgfsetlinewidth{0.010037pt}%
\definecolor{currentstroke}{rgb}{1.000000,1.000000,1.000000}%
\pgfsetstrokecolor{currentstroke}%
\pgfsetdash{}{0pt}%
\pgfusepath{stroke,fill}%
\end{pgfscope}%
\begin{pgfscope}%
\pgfsys@transformshift{0.902778in}{0.694444in}%
\pgftext[left,bottom]{\includegraphics[interpolate=true,width=2.041667in,height=0.222222in]{q_series_square-img6.png}}%
\end{pgfscope}%
\begin{pgfscope}%
\pgfsetbuttcap%
\pgfsetroundjoin%
\definecolor{currentfill}{rgb}{0.000000,0.000000,0.000000}%
\pgfsetfillcolor{currentfill}%
\pgfsetlinewidth{0.803000pt}%
\definecolor{currentstroke}{rgb}{0.000000,0.000000,0.000000}%
\pgfsetstrokecolor{currentstroke}%
\pgfsetdash{}{0pt}%
\pgfsys@defobject{currentmarker}{\pgfqpoint{0.000000in}{-0.048611in}}{\pgfqpoint{0.000000in}{0.000000in}}{%
\pgfpathmoveto{\pgfqpoint{0.000000in}{0.000000in}}%
\pgfpathlineto{\pgfqpoint{0.000000in}{-0.048611in}}%
\pgfusepath{stroke,fill}%
}%
\begin{pgfscope}%
\pgfsys@transformshift{0.896000in}{0.700000in}%
\pgfsys@useobject{currentmarker}{}%
\end{pgfscope}%
\end{pgfscope}%
\begin{pgfscope}%
\definecolor{textcolor}{rgb}{0.000000,0.000000,0.000000}%
\pgfsetstrokecolor{textcolor}%
\pgfsetfillcolor{textcolor}%
\pgftext[x=0.896000in,y=0.602778in,,top]{\color{textcolor}\sffamily\fontsize{10.000000}{12.000000}\selectfont \(\displaystyle {10^{-1}}\)}%
\end{pgfscope}%
\begin{pgfscope}%
\pgfsetbuttcap%
\pgfsetroundjoin%
\definecolor{currentfill}{rgb}{0.000000,0.000000,0.000000}%
\pgfsetfillcolor{currentfill}%
\pgfsetlinewidth{0.803000pt}%
\definecolor{currentstroke}{rgb}{0.000000,0.000000,0.000000}%
\pgfsetstrokecolor{currentstroke}%
\pgfsetdash{}{0pt}%
\pgfsys@defobject{currentmarker}{\pgfqpoint{0.000000in}{-0.048611in}}{\pgfqpoint{0.000000in}{0.000000in}}{%
\pgfpathmoveto{\pgfqpoint{0.000000in}{0.000000in}}%
\pgfpathlineto{\pgfqpoint{0.000000in}{-0.048611in}}%
\pgfusepath{stroke,fill}%
}%
\begin{pgfscope}%
\pgfsys@transformshift{1.920000in}{0.700000in}%
\pgfsys@useobject{currentmarker}{}%
\end{pgfscope}%
\end{pgfscope}%
\begin{pgfscope}%
\definecolor{textcolor}{rgb}{0.000000,0.000000,0.000000}%
\pgfsetstrokecolor{textcolor}%
\pgfsetfillcolor{textcolor}%
\pgftext[x=1.920000in,y=0.602778in,,top]{\color{textcolor}\sffamily\fontsize{10.000000}{12.000000}\selectfont \(\displaystyle {10^{0}}\)}%
\end{pgfscope}%
\begin{pgfscope}%
\pgfsetbuttcap%
\pgfsetroundjoin%
\definecolor{currentfill}{rgb}{0.000000,0.000000,0.000000}%
\pgfsetfillcolor{currentfill}%
\pgfsetlinewidth{0.803000pt}%
\definecolor{currentstroke}{rgb}{0.000000,0.000000,0.000000}%
\pgfsetstrokecolor{currentstroke}%
\pgfsetdash{}{0pt}%
\pgfsys@defobject{currentmarker}{\pgfqpoint{0.000000in}{-0.048611in}}{\pgfqpoint{0.000000in}{0.000000in}}{%
\pgfpathmoveto{\pgfqpoint{0.000000in}{0.000000in}}%
\pgfpathlineto{\pgfqpoint{0.000000in}{-0.048611in}}%
\pgfusepath{stroke,fill}%
}%
\begin{pgfscope}%
\pgfsys@transformshift{2.944000in}{0.700000in}%
\pgfsys@useobject{currentmarker}{}%
\end{pgfscope}%
\end{pgfscope}%
\begin{pgfscope}%
\definecolor{textcolor}{rgb}{0.000000,0.000000,0.000000}%
\pgfsetstrokecolor{textcolor}%
\pgfsetfillcolor{textcolor}%
\pgftext[x=2.944000in,y=0.602778in,,top]{\color{textcolor}\sffamily\fontsize{10.000000}{12.000000}\selectfont \(\displaystyle {10^{1}}\)}%
\end{pgfscope}%
\begin{pgfscope}%
\pgfsetbuttcap%
\pgfsetroundjoin%
\definecolor{currentfill}{rgb}{0.000000,0.000000,0.000000}%
\pgfsetfillcolor{currentfill}%
\pgfsetlinewidth{0.602250pt}%
\definecolor{currentstroke}{rgb}{0.000000,0.000000,0.000000}%
\pgfsetstrokecolor{currentstroke}%
\pgfsetdash{}{0pt}%
\pgfsys@defobject{currentmarker}{\pgfqpoint{0.000000in}{-0.027778in}}{\pgfqpoint{0.000000in}{0.000000in}}{%
\pgfpathmoveto{\pgfqpoint{0.000000in}{0.000000in}}%
\pgfpathlineto{\pgfqpoint{0.000000in}{-0.027778in}}%
\pgfusepath{stroke,fill}%
}%
\begin{pgfscope}%
\pgfsys@transformshift{1.204255in}{0.700000in}%
\pgfsys@useobject{currentmarker}{}%
\end{pgfscope}%
\end{pgfscope}%
\begin{pgfscope}%
\pgfsetbuttcap%
\pgfsetroundjoin%
\definecolor{currentfill}{rgb}{0.000000,0.000000,0.000000}%
\pgfsetfillcolor{currentfill}%
\pgfsetlinewidth{0.602250pt}%
\definecolor{currentstroke}{rgb}{0.000000,0.000000,0.000000}%
\pgfsetstrokecolor{currentstroke}%
\pgfsetdash{}{0pt}%
\pgfsys@defobject{currentmarker}{\pgfqpoint{0.000000in}{-0.027778in}}{\pgfqpoint{0.000000in}{0.000000in}}{%
\pgfpathmoveto{\pgfqpoint{0.000000in}{0.000000in}}%
\pgfpathlineto{\pgfqpoint{0.000000in}{-0.027778in}}%
\pgfusepath{stroke,fill}%
}%
\begin{pgfscope}%
\pgfsys@transformshift{1.384572in}{0.700000in}%
\pgfsys@useobject{currentmarker}{}%
\end{pgfscope}%
\end{pgfscope}%
\begin{pgfscope}%
\pgfsetbuttcap%
\pgfsetroundjoin%
\definecolor{currentfill}{rgb}{0.000000,0.000000,0.000000}%
\pgfsetfillcolor{currentfill}%
\pgfsetlinewidth{0.602250pt}%
\definecolor{currentstroke}{rgb}{0.000000,0.000000,0.000000}%
\pgfsetstrokecolor{currentstroke}%
\pgfsetdash{}{0pt}%
\pgfsys@defobject{currentmarker}{\pgfqpoint{0.000000in}{-0.027778in}}{\pgfqpoint{0.000000in}{0.000000in}}{%
\pgfpathmoveto{\pgfqpoint{0.000000in}{0.000000in}}%
\pgfpathlineto{\pgfqpoint{0.000000in}{-0.027778in}}%
\pgfusepath{stroke,fill}%
}%
\begin{pgfscope}%
\pgfsys@transformshift{1.512509in}{0.700000in}%
\pgfsys@useobject{currentmarker}{}%
\end{pgfscope}%
\end{pgfscope}%
\begin{pgfscope}%
\pgfsetbuttcap%
\pgfsetroundjoin%
\definecolor{currentfill}{rgb}{0.000000,0.000000,0.000000}%
\pgfsetfillcolor{currentfill}%
\pgfsetlinewidth{0.602250pt}%
\definecolor{currentstroke}{rgb}{0.000000,0.000000,0.000000}%
\pgfsetstrokecolor{currentstroke}%
\pgfsetdash{}{0pt}%
\pgfsys@defobject{currentmarker}{\pgfqpoint{0.000000in}{-0.027778in}}{\pgfqpoint{0.000000in}{0.000000in}}{%
\pgfpathmoveto{\pgfqpoint{0.000000in}{0.000000in}}%
\pgfpathlineto{\pgfqpoint{0.000000in}{-0.027778in}}%
\pgfusepath{stroke,fill}%
}%
\begin{pgfscope}%
\pgfsys@transformshift{1.611745in}{0.700000in}%
\pgfsys@useobject{currentmarker}{}%
\end{pgfscope}%
\end{pgfscope}%
\begin{pgfscope}%
\pgfsetbuttcap%
\pgfsetroundjoin%
\definecolor{currentfill}{rgb}{0.000000,0.000000,0.000000}%
\pgfsetfillcolor{currentfill}%
\pgfsetlinewidth{0.602250pt}%
\definecolor{currentstroke}{rgb}{0.000000,0.000000,0.000000}%
\pgfsetstrokecolor{currentstroke}%
\pgfsetdash{}{0pt}%
\pgfsys@defobject{currentmarker}{\pgfqpoint{0.000000in}{-0.027778in}}{\pgfqpoint{0.000000in}{0.000000in}}{%
\pgfpathmoveto{\pgfqpoint{0.000000in}{0.000000in}}%
\pgfpathlineto{\pgfqpoint{0.000000in}{-0.027778in}}%
\pgfusepath{stroke,fill}%
}%
\begin{pgfscope}%
\pgfsys@transformshift{1.692827in}{0.700000in}%
\pgfsys@useobject{currentmarker}{}%
\end{pgfscope}%
\end{pgfscope}%
\begin{pgfscope}%
\pgfsetbuttcap%
\pgfsetroundjoin%
\definecolor{currentfill}{rgb}{0.000000,0.000000,0.000000}%
\pgfsetfillcolor{currentfill}%
\pgfsetlinewidth{0.602250pt}%
\definecolor{currentstroke}{rgb}{0.000000,0.000000,0.000000}%
\pgfsetstrokecolor{currentstroke}%
\pgfsetdash{}{0pt}%
\pgfsys@defobject{currentmarker}{\pgfqpoint{0.000000in}{-0.027778in}}{\pgfqpoint{0.000000in}{0.000000in}}{%
\pgfpathmoveto{\pgfqpoint{0.000000in}{0.000000in}}%
\pgfpathlineto{\pgfqpoint{0.000000in}{-0.027778in}}%
\pgfusepath{stroke,fill}%
}%
\begin{pgfscope}%
\pgfsys@transformshift{1.761380in}{0.700000in}%
\pgfsys@useobject{currentmarker}{}%
\end{pgfscope}%
\end{pgfscope}%
\begin{pgfscope}%
\pgfsetbuttcap%
\pgfsetroundjoin%
\definecolor{currentfill}{rgb}{0.000000,0.000000,0.000000}%
\pgfsetfillcolor{currentfill}%
\pgfsetlinewidth{0.602250pt}%
\definecolor{currentstroke}{rgb}{0.000000,0.000000,0.000000}%
\pgfsetstrokecolor{currentstroke}%
\pgfsetdash{}{0pt}%
\pgfsys@defobject{currentmarker}{\pgfqpoint{0.000000in}{-0.027778in}}{\pgfqpoint{0.000000in}{0.000000in}}{%
\pgfpathmoveto{\pgfqpoint{0.000000in}{0.000000in}}%
\pgfpathlineto{\pgfqpoint{0.000000in}{-0.027778in}}%
\pgfusepath{stroke,fill}%
}%
\begin{pgfscope}%
\pgfsys@transformshift{1.820764in}{0.700000in}%
\pgfsys@useobject{currentmarker}{}%
\end{pgfscope}%
\end{pgfscope}%
\begin{pgfscope}%
\pgfsetbuttcap%
\pgfsetroundjoin%
\definecolor{currentfill}{rgb}{0.000000,0.000000,0.000000}%
\pgfsetfillcolor{currentfill}%
\pgfsetlinewidth{0.602250pt}%
\definecolor{currentstroke}{rgb}{0.000000,0.000000,0.000000}%
\pgfsetstrokecolor{currentstroke}%
\pgfsetdash{}{0pt}%
\pgfsys@defobject{currentmarker}{\pgfqpoint{0.000000in}{-0.027778in}}{\pgfqpoint{0.000000in}{0.000000in}}{%
\pgfpathmoveto{\pgfqpoint{0.000000in}{0.000000in}}%
\pgfpathlineto{\pgfqpoint{0.000000in}{-0.027778in}}%
\pgfusepath{stroke,fill}%
}%
\begin{pgfscope}%
\pgfsys@transformshift{1.873144in}{0.700000in}%
\pgfsys@useobject{currentmarker}{}%
\end{pgfscope}%
\end{pgfscope}%
\begin{pgfscope}%
\pgfsetbuttcap%
\pgfsetroundjoin%
\definecolor{currentfill}{rgb}{0.000000,0.000000,0.000000}%
\pgfsetfillcolor{currentfill}%
\pgfsetlinewidth{0.602250pt}%
\definecolor{currentstroke}{rgb}{0.000000,0.000000,0.000000}%
\pgfsetstrokecolor{currentstroke}%
\pgfsetdash{}{0pt}%
\pgfsys@defobject{currentmarker}{\pgfqpoint{0.000000in}{-0.027778in}}{\pgfqpoint{0.000000in}{0.000000in}}{%
\pgfpathmoveto{\pgfqpoint{0.000000in}{0.000000in}}%
\pgfpathlineto{\pgfqpoint{0.000000in}{-0.027778in}}%
\pgfusepath{stroke,fill}%
}%
\begin{pgfscope}%
\pgfsys@transformshift{2.228255in}{0.700000in}%
\pgfsys@useobject{currentmarker}{}%
\end{pgfscope}%
\end{pgfscope}%
\begin{pgfscope}%
\pgfsetbuttcap%
\pgfsetroundjoin%
\definecolor{currentfill}{rgb}{0.000000,0.000000,0.000000}%
\pgfsetfillcolor{currentfill}%
\pgfsetlinewidth{0.602250pt}%
\definecolor{currentstroke}{rgb}{0.000000,0.000000,0.000000}%
\pgfsetstrokecolor{currentstroke}%
\pgfsetdash{}{0pt}%
\pgfsys@defobject{currentmarker}{\pgfqpoint{0.000000in}{-0.027778in}}{\pgfqpoint{0.000000in}{0.000000in}}{%
\pgfpathmoveto{\pgfqpoint{0.000000in}{0.000000in}}%
\pgfpathlineto{\pgfqpoint{0.000000in}{-0.027778in}}%
\pgfusepath{stroke,fill}%
}%
\begin{pgfscope}%
\pgfsys@transformshift{2.408572in}{0.700000in}%
\pgfsys@useobject{currentmarker}{}%
\end{pgfscope}%
\end{pgfscope}%
\begin{pgfscope}%
\pgfsetbuttcap%
\pgfsetroundjoin%
\definecolor{currentfill}{rgb}{0.000000,0.000000,0.000000}%
\pgfsetfillcolor{currentfill}%
\pgfsetlinewidth{0.602250pt}%
\definecolor{currentstroke}{rgb}{0.000000,0.000000,0.000000}%
\pgfsetstrokecolor{currentstroke}%
\pgfsetdash{}{0pt}%
\pgfsys@defobject{currentmarker}{\pgfqpoint{0.000000in}{-0.027778in}}{\pgfqpoint{0.000000in}{0.000000in}}{%
\pgfpathmoveto{\pgfqpoint{0.000000in}{0.000000in}}%
\pgfpathlineto{\pgfqpoint{0.000000in}{-0.027778in}}%
\pgfusepath{stroke,fill}%
}%
\begin{pgfscope}%
\pgfsys@transformshift{2.536509in}{0.700000in}%
\pgfsys@useobject{currentmarker}{}%
\end{pgfscope}%
\end{pgfscope}%
\begin{pgfscope}%
\pgfsetbuttcap%
\pgfsetroundjoin%
\definecolor{currentfill}{rgb}{0.000000,0.000000,0.000000}%
\pgfsetfillcolor{currentfill}%
\pgfsetlinewidth{0.602250pt}%
\definecolor{currentstroke}{rgb}{0.000000,0.000000,0.000000}%
\pgfsetstrokecolor{currentstroke}%
\pgfsetdash{}{0pt}%
\pgfsys@defobject{currentmarker}{\pgfqpoint{0.000000in}{-0.027778in}}{\pgfqpoint{0.000000in}{0.000000in}}{%
\pgfpathmoveto{\pgfqpoint{0.000000in}{0.000000in}}%
\pgfpathlineto{\pgfqpoint{0.000000in}{-0.027778in}}%
\pgfusepath{stroke,fill}%
}%
\begin{pgfscope}%
\pgfsys@transformshift{2.635745in}{0.700000in}%
\pgfsys@useobject{currentmarker}{}%
\end{pgfscope}%
\end{pgfscope}%
\begin{pgfscope}%
\pgfsetbuttcap%
\pgfsetroundjoin%
\definecolor{currentfill}{rgb}{0.000000,0.000000,0.000000}%
\pgfsetfillcolor{currentfill}%
\pgfsetlinewidth{0.602250pt}%
\definecolor{currentstroke}{rgb}{0.000000,0.000000,0.000000}%
\pgfsetstrokecolor{currentstroke}%
\pgfsetdash{}{0pt}%
\pgfsys@defobject{currentmarker}{\pgfqpoint{0.000000in}{-0.027778in}}{\pgfqpoint{0.000000in}{0.000000in}}{%
\pgfpathmoveto{\pgfqpoint{0.000000in}{0.000000in}}%
\pgfpathlineto{\pgfqpoint{0.000000in}{-0.027778in}}%
\pgfusepath{stroke,fill}%
}%
\begin{pgfscope}%
\pgfsys@transformshift{2.716827in}{0.700000in}%
\pgfsys@useobject{currentmarker}{}%
\end{pgfscope}%
\end{pgfscope}%
\begin{pgfscope}%
\pgfsetbuttcap%
\pgfsetroundjoin%
\definecolor{currentfill}{rgb}{0.000000,0.000000,0.000000}%
\pgfsetfillcolor{currentfill}%
\pgfsetlinewidth{0.602250pt}%
\definecolor{currentstroke}{rgb}{0.000000,0.000000,0.000000}%
\pgfsetstrokecolor{currentstroke}%
\pgfsetdash{}{0pt}%
\pgfsys@defobject{currentmarker}{\pgfqpoint{0.000000in}{-0.027778in}}{\pgfqpoint{0.000000in}{0.000000in}}{%
\pgfpathmoveto{\pgfqpoint{0.000000in}{0.000000in}}%
\pgfpathlineto{\pgfqpoint{0.000000in}{-0.027778in}}%
\pgfusepath{stroke,fill}%
}%
\begin{pgfscope}%
\pgfsys@transformshift{2.785380in}{0.700000in}%
\pgfsys@useobject{currentmarker}{}%
\end{pgfscope}%
\end{pgfscope}%
\begin{pgfscope}%
\pgfsetbuttcap%
\pgfsetroundjoin%
\definecolor{currentfill}{rgb}{0.000000,0.000000,0.000000}%
\pgfsetfillcolor{currentfill}%
\pgfsetlinewidth{0.602250pt}%
\definecolor{currentstroke}{rgb}{0.000000,0.000000,0.000000}%
\pgfsetstrokecolor{currentstroke}%
\pgfsetdash{}{0pt}%
\pgfsys@defobject{currentmarker}{\pgfqpoint{0.000000in}{-0.027778in}}{\pgfqpoint{0.000000in}{0.000000in}}{%
\pgfpathmoveto{\pgfqpoint{0.000000in}{0.000000in}}%
\pgfpathlineto{\pgfqpoint{0.000000in}{-0.027778in}}%
\pgfusepath{stroke,fill}%
}%
\begin{pgfscope}%
\pgfsys@transformshift{2.844764in}{0.700000in}%
\pgfsys@useobject{currentmarker}{}%
\end{pgfscope}%
\end{pgfscope}%
\begin{pgfscope}%
\pgfsetbuttcap%
\pgfsetroundjoin%
\definecolor{currentfill}{rgb}{0.000000,0.000000,0.000000}%
\pgfsetfillcolor{currentfill}%
\pgfsetlinewidth{0.602250pt}%
\definecolor{currentstroke}{rgb}{0.000000,0.000000,0.000000}%
\pgfsetstrokecolor{currentstroke}%
\pgfsetdash{}{0pt}%
\pgfsys@defobject{currentmarker}{\pgfqpoint{0.000000in}{-0.027778in}}{\pgfqpoint{0.000000in}{0.000000in}}{%
\pgfpathmoveto{\pgfqpoint{0.000000in}{0.000000in}}%
\pgfpathlineto{\pgfqpoint{0.000000in}{-0.027778in}}%
\pgfusepath{stroke,fill}%
}%
\begin{pgfscope}%
\pgfsys@transformshift{2.897144in}{0.700000in}%
\pgfsys@useobject{currentmarker}{}%
\end{pgfscope}%
\end{pgfscope}%
\begin{pgfscope}%
\definecolor{textcolor}{rgb}{0.000000,0.000000,0.000000}%
\pgfsetstrokecolor{textcolor}%
\pgfsetfillcolor{textcolor}%
\pgftext[x=1.920000in,y=0.423766in,,top]{\color{textcolor}\sffamily\fontsize{10.000000}{12.000000}\selectfont \(\displaystyle -dQ/dy/dz \, \mathrm{[pC/\mu m^2]}\)}%
\end{pgfscope}%
\begin{pgfscope}%
\pgfsetrectcap%
\pgfsetmiterjoin%
\pgfsetlinewidth{0.803000pt}%
\definecolor{currentstroke}{rgb}{0.000000,0.000000,0.000000}%
\pgfsetstrokecolor{currentstroke}%
\pgfsetdash{}{0pt}%
\pgfpathmoveto{\pgfqpoint{0.896000in}{0.700000in}}%
\pgfpathlineto{\pgfqpoint{0.896000in}{0.805000in}}%
\pgfpathlineto{\pgfqpoint{0.896000in}{0.910000in}}%
\pgfpathlineto{\pgfqpoint{2.944000in}{0.910000in}}%
\pgfpathlineto{\pgfqpoint{2.944000in}{0.805000in}}%
\pgfpathlineto{\pgfqpoint{2.944000in}{0.700000in}}%
\pgfpathlineto{\pgfqpoint{0.896000in}{0.700000in}}%
\pgfpathclose%
\pgfusepath{stroke}%
\end{pgfscope}%
\begin{pgfscope}%
\pgfsetbuttcap%
\pgfsetmiterjoin%
\definecolor{currentfill}{rgb}{1.000000,1.000000,1.000000}%
\pgfsetfillcolor{currentfill}%
\pgfsetlinewidth{0.000000pt}%
\definecolor{currentstroke}{rgb}{0.000000,0.000000,0.000000}%
\pgfsetstrokecolor{currentstroke}%
\pgfsetstrokeopacity{0.000000}%
\pgfsetdash{}{0pt}%
\pgfpathmoveto{\pgfqpoint{3.456000in}{0.700000in}}%
\pgfpathlineto{\pgfqpoint{5.504000in}{0.700000in}}%
\pgfpathlineto{\pgfqpoint{5.504000in}{0.910000in}}%
\pgfpathlineto{\pgfqpoint{3.456000in}{0.910000in}}%
\pgfpathlineto{\pgfqpoint{3.456000in}{0.700000in}}%
\pgfpathclose%
\pgfusepath{fill}%
\end{pgfscope}%
\begin{pgfscope}%
\pgfpathrectangle{\pgfqpoint{3.456000in}{0.700000in}}{\pgfqpoint{2.048000in}{0.210000in}}%
\pgfusepath{clip}%
\pgfsetbuttcap%
\pgfsetmiterjoin%
\definecolor{currentfill}{rgb}{1.000000,1.000000,1.000000}%
\pgfsetfillcolor{currentfill}%
\pgfsetlinewidth{0.010037pt}%
\definecolor{currentstroke}{rgb}{1.000000,1.000000,1.000000}%
\pgfsetstrokecolor{currentstroke}%
\pgfsetdash{}{0pt}%
\pgfusepath{stroke,fill}%
\end{pgfscope}%
\begin{pgfscope}%
\pgfsys@transformshift{3.458333in}{0.694444in}%
\pgftext[left,bottom]{\includegraphics[interpolate=true,width=2.041667in,height=0.222222in]{q_series_square-img7.png}}%
\end{pgfscope}%
\begin{pgfscope}%
\pgfsetbuttcap%
\pgfsetroundjoin%
\definecolor{currentfill}{rgb}{0.000000,0.000000,0.000000}%
\pgfsetfillcolor{currentfill}%
\pgfsetlinewidth{0.803000pt}%
\definecolor{currentstroke}{rgb}{0.000000,0.000000,0.000000}%
\pgfsetstrokecolor{currentstroke}%
\pgfsetdash{}{0pt}%
\pgfsys@defobject{currentmarker}{\pgfqpoint{0.000000in}{-0.048611in}}{\pgfqpoint{0.000000in}{0.000000in}}{%
\pgfpathmoveto{\pgfqpoint{0.000000in}{0.000000in}}%
\pgfpathlineto{\pgfqpoint{0.000000in}{-0.048611in}}%
\pgfusepath{stroke,fill}%
}%
\begin{pgfscope}%
\pgfsys@transformshift{3.456000in}{0.700000in}%
\pgfsys@useobject{currentmarker}{}%
\end{pgfscope}%
\end{pgfscope}%
\begin{pgfscope}%
\definecolor{textcolor}{rgb}{0.000000,0.000000,0.000000}%
\pgfsetstrokecolor{textcolor}%
\pgfsetfillcolor{textcolor}%
\pgftext[x=3.456000in,y=0.602778in,,top]{\color{textcolor}\sffamily\fontsize{10.000000}{12.000000}\selectfont \(\displaystyle {0}\)}%
\end{pgfscope}%
\begin{pgfscope}%
\pgfsetbuttcap%
\pgfsetroundjoin%
\definecolor{currentfill}{rgb}{0.000000,0.000000,0.000000}%
\pgfsetfillcolor{currentfill}%
\pgfsetlinewidth{0.803000pt}%
\definecolor{currentstroke}{rgb}{0.000000,0.000000,0.000000}%
\pgfsetstrokecolor{currentstroke}%
\pgfsetdash{}{0pt}%
\pgfsys@defobject{currentmarker}{\pgfqpoint{0.000000in}{-0.048611in}}{\pgfqpoint{0.000000in}{0.000000in}}{%
\pgfpathmoveto{\pgfqpoint{0.000000in}{0.000000in}}%
\pgfpathlineto{\pgfqpoint{0.000000in}{-0.048611in}}%
\pgfusepath{stroke,fill}%
}%
\begin{pgfscope}%
\pgfsys@transformshift{4.041143in}{0.700000in}%
\pgfsys@useobject{currentmarker}{}%
\end{pgfscope}%
\end{pgfscope}%
\begin{pgfscope}%
\definecolor{textcolor}{rgb}{0.000000,0.000000,0.000000}%
\pgfsetstrokecolor{textcolor}%
\pgfsetfillcolor{textcolor}%
\pgftext[x=4.041143in,y=0.602778in,,top]{\color{textcolor}\sffamily\fontsize{10.000000}{12.000000}\selectfont \(\displaystyle {200}\)}%
\end{pgfscope}%
\begin{pgfscope}%
\pgfsetbuttcap%
\pgfsetroundjoin%
\definecolor{currentfill}{rgb}{0.000000,0.000000,0.000000}%
\pgfsetfillcolor{currentfill}%
\pgfsetlinewidth{0.803000pt}%
\definecolor{currentstroke}{rgb}{0.000000,0.000000,0.000000}%
\pgfsetstrokecolor{currentstroke}%
\pgfsetdash{}{0pt}%
\pgfsys@defobject{currentmarker}{\pgfqpoint{0.000000in}{-0.048611in}}{\pgfqpoint{0.000000in}{0.000000in}}{%
\pgfpathmoveto{\pgfqpoint{0.000000in}{0.000000in}}%
\pgfpathlineto{\pgfqpoint{0.000000in}{-0.048611in}}%
\pgfusepath{stroke,fill}%
}%
\begin{pgfscope}%
\pgfsys@transformshift{4.626286in}{0.700000in}%
\pgfsys@useobject{currentmarker}{}%
\end{pgfscope}%
\end{pgfscope}%
\begin{pgfscope}%
\definecolor{textcolor}{rgb}{0.000000,0.000000,0.000000}%
\pgfsetstrokecolor{textcolor}%
\pgfsetfillcolor{textcolor}%
\pgftext[x=4.626286in,y=0.602778in,,top]{\color{textcolor}\sffamily\fontsize{10.000000}{12.000000}\selectfont \(\displaystyle {400}\)}%
\end{pgfscope}%
\begin{pgfscope}%
\pgfsetbuttcap%
\pgfsetroundjoin%
\definecolor{currentfill}{rgb}{0.000000,0.000000,0.000000}%
\pgfsetfillcolor{currentfill}%
\pgfsetlinewidth{0.803000pt}%
\definecolor{currentstroke}{rgb}{0.000000,0.000000,0.000000}%
\pgfsetstrokecolor{currentstroke}%
\pgfsetdash{}{0pt}%
\pgfsys@defobject{currentmarker}{\pgfqpoint{0.000000in}{-0.048611in}}{\pgfqpoint{0.000000in}{0.000000in}}{%
\pgfpathmoveto{\pgfqpoint{0.000000in}{0.000000in}}%
\pgfpathlineto{\pgfqpoint{0.000000in}{-0.048611in}}%
\pgfusepath{stroke,fill}%
}%
\begin{pgfscope}%
\pgfsys@transformshift{5.211429in}{0.700000in}%
\pgfsys@useobject{currentmarker}{}%
\end{pgfscope}%
\end{pgfscope}%
\begin{pgfscope}%
\definecolor{textcolor}{rgb}{0.000000,0.000000,0.000000}%
\pgfsetstrokecolor{textcolor}%
\pgfsetfillcolor{textcolor}%
\pgftext[x=5.211429in,y=0.602778in,,top]{\color{textcolor}\sffamily\fontsize{10.000000}{12.000000}\selectfont \(\displaystyle {600}\)}%
\end{pgfscope}%
\begin{pgfscope}%
\definecolor{textcolor}{rgb}{0.000000,0.000000,0.000000}%
\pgfsetstrokecolor{textcolor}%
\pgfsetfillcolor{textcolor}%
\pgftext[x=4.480000in,y=0.423766in,,top]{\color{textcolor}\sffamily\fontsize{10.000000}{12.000000}\selectfont \(\displaystyle \left|\vec{F}_{Lorentz}\right| \, \mathrm{[mN]}\)}%
\end{pgfscope}%
\begin{pgfscope}%
\pgfsetrectcap%
\pgfsetmiterjoin%
\pgfsetlinewidth{0.803000pt}%
\definecolor{currentstroke}{rgb}{0.000000,0.000000,0.000000}%
\pgfsetstrokecolor{currentstroke}%
\pgfsetdash{}{0pt}%
\pgfpathmoveto{\pgfqpoint{3.456000in}{0.700000in}}%
\pgfpathlineto{\pgfqpoint{3.456000in}{0.805000in}}%
\pgfpathlineto{\pgfqpoint{3.456000in}{0.910000in}}%
\pgfpathlineto{\pgfqpoint{5.504000in}{0.910000in}}%
\pgfpathlineto{\pgfqpoint{5.504000in}{0.805000in}}%
\pgfpathlineto{\pgfqpoint{5.504000in}{0.700000in}}%
\pgfpathlineto{\pgfqpoint{3.456000in}{0.700000in}}%
\pgfpathclose%
\pgfusepath{stroke}%
\end{pgfscope}%
\end{pgfpicture}%
\makeatother%
\endgroup%

	\caption{Time series of a charge density histogram of a driver uniform in co-moving $\zeta$-direction, integrated over $x$, with the acting Lorentz Force drawn on top as vector lines. Vertical lines are drawn to make change in length of the bunch better visible. 
	\textbf{a)} ($y=$ \qty{0.04}{mm}) When entering the plasma. Still uniform in $\zeta$.
	\textbf{b)} ($y=$ \qty{0.36}{mm}) Compared to the Gaussian driver, the tail doesn't get as thin.
	\textbf{c)} ($y=$ \qty{0.76}{mm}) First wing spreads from tail. Strong decelerating forces on the backside.
	\textbf{d)} ($y=$ \qty{1.08}{mm}) More wings emerge and spread. Driver expends in $\zeta$ with higher elongation near the $z$-center
	\textbf{e)} ($y=$ \qty{3.54}{mm}) Shortly before bunch breakup. Note the longer distance before bunch breakup.
	\textbf{f)} ($y=$ \qty{4.97}{mm}) Bunch after breakup.}
	\label{fig:q_series_square}
\end{figure}
Still the uniform driver survives longer before breakup, caused by the weaker decelerating forces, as seen in the higher minimum in \autoref{fig:gain_square}.


\section{Peak energy shift} \label{chap:E_shift}
In this chapter, the change in energy of the \gls{pwfa} driver is discussed, as this gives further insights into its stability and therefore also the stability of the wakefield.
In \autoref{fig:E_hist_time}a the change of the energy distribution over time can be seen for the \qty{250}{\MeV}, \qty{4.2}{\mrad} Gaussian driver.
\begin{figure}
	\centering
	%% Creator: Matplotlib, PGF backend
%%
%% To include the figure in your LaTeX document, write
%%   \input{<filename>.pgf}
%%
%% Make sure the required packages are loaded in your preamble
%%   \usepackage{pgf}
%%
%% Also ensure that all the required font packages are loaded; for instance,
%% the lmodern package is sometimes necessary when using math font.
%%   \usepackage{lmodern}
%%
%% Figures using additional raster images can only be included by \input if
%% they are in the same directory as the main LaTeX file. For loading figures
%% from other directories you can use the `import` package
%%   \usepackage{import}
%%
%% and then include the figures with
%%   \import{<path to file>}{<filename>.pgf}
%%
%% Matplotlib used the following preamble
%%
\begingroup%
\makeatletter%
\begin{pgfpicture}%
\pgfpathrectangle{\pgfpointorigin}{\pgfqpoint{6.400000in}{7.000000in}}%
\pgfusepath{use as bounding box, clip}%
\begin{pgfscope}%
\pgfsetbuttcap%
\pgfsetmiterjoin%
\pgfsetlinewidth{0.000000pt}%
\definecolor{currentstroke}{rgb}{1.000000,1.000000,1.000000}%
\pgfsetstrokecolor{currentstroke}%
\pgfsetstrokeopacity{0.000000}%
\pgfsetdash{}{0pt}%
\pgfpathmoveto{\pgfqpoint{0.000000in}{0.000000in}}%
\pgfpathlineto{\pgfqpoint{6.400000in}{0.000000in}}%
\pgfpathlineto{\pgfqpoint{6.400000in}{7.000000in}}%
\pgfpathlineto{\pgfqpoint{0.000000in}{7.000000in}}%
\pgfpathlineto{\pgfqpoint{0.000000in}{0.000000in}}%
\pgfpathclose%
\pgfusepath{}%
\end{pgfscope}%
\begin{pgfscope}%
\pgfsetbuttcap%
\pgfsetmiterjoin%
\definecolor{currentfill}{rgb}{1.000000,1.000000,1.000000}%
\pgfsetfillcolor{currentfill}%
\pgfsetlinewidth{0.000000pt}%
\definecolor{currentstroke}{rgb}{0.000000,0.000000,0.000000}%
\pgfsetstrokecolor{currentstroke}%
\pgfsetstrokeopacity{0.000000}%
\pgfsetdash{}{0pt}%
\pgfpathmoveto{\pgfqpoint{0.800000in}{3.786087in}}%
\pgfpathlineto{\pgfqpoint{4.768000in}{3.786087in}}%
\pgfpathlineto{\pgfqpoint{4.768000in}{6.160000in}}%
\pgfpathlineto{\pgfqpoint{0.800000in}{6.160000in}}%
\pgfpathlineto{\pgfqpoint{0.800000in}{3.786087in}}%
\pgfpathclose%
\pgfusepath{fill}%
\end{pgfscope}%
\begin{pgfscope}%
\pgfpathrectangle{\pgfqpoint{0.800000in}{3.786087in}}{\pgfqpoint{3.968000in}{2.373913in}}%
\pgfusepath{clip}%
\pgfsys@transformcm{3.972222}{0.000000}{0.000000}{2.375000}{0.800000in}{3.786087in}%
\pgftext[left,bottom]{\includegraphics[interpolate=false,width=1.000000in,height=1.000000in]{E_hist_time-img0.png}}%
\end{pgfscope}%
\begin{pgfscope}%
\pgfsetbuttcap%
\pgfsetroundjoin%
\definecolor{currentfill}{rgb}{0.000000,0.000000,0.000000}%
\pgfsetfillcolor{currentfill}%
\pgfsetlinewidth{0.803000pt}%
\definecolor{currentstroke}{rgb}{0.000000,0.000000,0.000000}%
\pgfsetstrokecolor{currentstroke}%
\pgfsetdash{}{0pt}%
\pgfsys@defobject{currentmarker}{\pgfqpoint{0.000000in}{-0.048611in}}{\pgfqpoint{0.000000in}{0.000000in}}{%
\pgfpathmoveto{\pgfqpoint{0.000000in}{0.000000in}}%
\pgfpathlineto{\pgfqpoint{0.000000in}{-0.048611in}}%
\pgfusepath{stroke,fill}%
}%
\begin{pgfscope}%
\pgfsys@transformshift{0.849184in}{3.786087in}%
\pgfsys@useobject{currentmarker}{}%
\end{pgfscope}%
\end{pgfscope}%
\begin{pgfscope}%
\definecolor{textcolor}{rgb}{0.000000,0.000000,0.000000}%
\pgfsetstrokecolor{textcolor}%
\pgfsetfillcolor{textcolor}%
\pgftext[x=0.849184in,y=3.688865in,,top]{\color{textcolor}\sffamily\fontsize{10.000000}{12.000000}\selectfont \(\displaystyle {0}\)}%
\end{pgfscope}%
\begin{pgfscope}%
\pgfsetbuttcap%
\pgfsetroundjoin%
\definecolor{currentfill}{rgb}{0.000000,0.000000,0.000000}%
\pgfsetfillcolor{currentfill}%
\pgfsetlinewidth{0.803000pt}%
\definecolor{currentstroke}{rgb}{0.000000,0.000000,0.000000}%
\pgfsetstrokecolor{currentstroke}%
\pgfsetdash{}{0pt}%
\pgfsys@defobject{currentmarker}{\pgfqpoint{0.000000in}{-0.048611in}}{\pgfqpoint{0.000000in}{0.000000in}}{%
\pgfpathmoveto{\pgfqpoint{0.000000in}{0.000000in}}%
\pgfpathlineto{\pgfqpoint{0.000000in}{-0.048611in}}%
\pgfusepath{stroke,fill}%
}%
\begin{pgfscope}%
\pgfsys@transformshift{1.480811in}{3.786087in}%
\pgfsys@useobject{currentmarker}{}%
\end{pgfscope}%
\end{pgfscope}%
\begin{pgfscope}%
\definecolor{textcolor}{rgb}{0.000000,0.000000,0.000000}%
\pgfsetstrokecolor{textcolor}%
\pgfsetfillcolor{textcolor}%
\pgftext[x=1.480811in,y=3.688865in,,top]{\color{textcolor}\sffamily\fontsize{10.000000}{12.000000}\selectfont \(\displaystyle {1}\)}%
\end{pgfscope}%
\begin{pgfscope}%
\pgfsetbuttcap%
\pgfsetroundjoin%
\definecolor{currentfill}{rgb}{0.000000,0.000000,0.000000}%
\pgfsetfillcolor{currentfill}%
\pgfsetlinewidth{0.803000pt}%
\definecolor{currentstroke}{rgb}{0.000000,0.000000,0.000000}%
\pgfsetstrokecolor{currentstroke}%
\pgfsetdash{}{0pt}%
\pgfsys@defobject{currentmarker}{\pgfqpoint{0.000000in}{-0.048611in}}{\pgfqpoint{0.000000in}{0.000000in}}{%
\pgfpathmoveto{\pgfqpoint{0.000000in}{0.000000in}}%
\pgfpathlineto{\pgfqpoint{0.000000in}{-0.048611in}}%
\pgfusepath{stroke,fill}%
}%
\begin{pgfscope}%
\pgfsys@transformshift{2.112438in}{3.786087in}%
\pgfsys@useobject{currentmarker}{}%
\end{pgfscope}%
\end{pgfscope}%
\begin{pgfscope}%
\definecolor{textcolor}{rgb}{0.000000,0.000000,0.000000}%
\pgfsetstrokecolor{textcolor}%
\pgfsetfillcolor{textcolor}%
\pgftext[x=2.112438in,y=3.688865in,,top]{\color{textcolor}\sffamily\fontsize{10.000000}{12.000000}\selectfont \(\displaystyle {2}\)}%
\end{pgfscope}%
\begin{pgfscope}%
\pgfsetbuttcap%
\pgfsetroundjoin%
\definecolor{currentfill}{rgb}{0.000000,0.000000,0.000000}%
\pgfsetfillcolor{currentfill}%
\pgfsetlinewidth{0.803000pt}%
\definecolor{currentstroke}{rgb}{0.000000,0.000000,0.000000}%
\pgfsetstrokecolor{currentstroke}%
\pgfsetdash{}{0pt}%
\pgfsys@defobject{currentmarker}{\pgfqpoint{0.000000in}{-0.048611in}}{\pgfqpoint{0.000000in}{0.000000in}}{%
\pgfpathmoveto{\pgfqpoint{0.000000in}{0.000000in}}%
\pgfpathlineto{\pgfqpoint{0.000000in}{-0.048611in}}%
\pgfusepath{stroke,fill}%
}%
\begin{pgfscope}%
\pgfsys@transformshift{2.744064in}{3.786087in}%
\pgfsys@useobject{currentmarker}{}%
\end{pgfscope}%
\end{pgfscope}%
\begin{pgfscope}%
\definecolor{textcolor}{rgb}{0.000000,0.000000,0.000000}%
\pgfsetstrokecolor{textcolor}%
\pgfsetfillcolor{textcolor}%
\pgftext[x=2.744064in,y=3.688865in,,top]{\color{textcolor}\sffamily\fontsize{10.000000}{12.000000}\selectfont \(\displaystyle {3}\)}%
\end{pgfscope}%
\begin{pgfscope}%
\pgfsetbuttcap%
\pgfsetroundjoin%
\definecolor{currentfill}{rgb}{0.000000,0.000000,0.000000}%
\pgfsetfillcolor{currentfill}%
\pgfsetlinewidth{0.803000pt}%
\definecolor{currentstroke}{rgb}{0.000000,0.000000,0.000000}%
\pgfsetstrokecolor{currentstroke}%
\pgfsetdash{}{0pt}%
\pgfsys@defobject{currentmarker}{\pgfqpoint{0.000000in}{-0.048611in}}{\pgfqpoint{0.000000in}{0.000000in}}{%
\pgfpathmoveto{\pgfqpoint{0.000000in}{0.000000in}}%
\pgfpathlineto{\pgfqpoint{0.000000in}{-0.048611in}}%
\pgfusepath{stroke,fill}%
}%
\begin{pgfscope}%
\pgfsys@transformshift{3.375691in}{3.786087in}%
\pgfsys@useobject{currentmarker}{}%
\end{pgfscope}%
\end{pgfscope}%
\begin{pgfscope}%
\definecolor{textcolor}{rgb}{0.000000,0.000000,0.000000}%
\pgfsetstrokecolor{textcolor}%
\pgfsetfillcolor{textcolor}%
\pgftext[x=3.375691in,y=3.688865in,,top]{\color{textcolor}\sffamily\fontsize{10.000000}{12.000000}\selectfont \(\displaystyle {4}\)}%
\end{pgfscope}%
\begin{pgfscope}%
\pgfsetbuttcap%
\pgfsetroundjoin%
\definecolor{currentfill}{rgb}{0.000000,0.000000,0.000000}%
\pgfsetfillcolor{currentfill}%
\pgfsetlinewidth{0.803000pt}%
\definecolor{currentstroke}{rgb}{0.000000,0.000000,0.000000}%
\pgfsetstrokecolor{currentstroke}%
\pgfsetdash{}{0pt}%
\pgfsys@defobject{currentmarker}{\pgfqpoint{0.000000in}{-0.048611in}}{\pgfqpoint{0.000000in}{0.000000in}}{%
\pgfpathmoveto{\pgfqpoint{0.000000in}{0.000000in}}%
\pgfpathlineto{\pgfqpoint{0.000000in}{-0.048611in}}%
\pgfusepath{stroke,fill}%
}%
\begin{pgfscope}%
\pgfsys@transformshift{4.007318in}{3.786087in}%
\pgfsys@useobject{currentmarker}{}%
\end{pgfscope}%
\end{pgfscope}%
\begin{pgfscope}%
\definecolor{textcolor}{rgb}{0.000000,0.000000,0.000000}%
\pgfsetstrokecolor{textcolor}%
\pgfsetfillcolor{textcolor}%
\pgftext[x=4.007318in,y=3.688865in,,top]{\color{textcolor}\sffamily\fontsize{10.000000}{12.000000}\selectfont \(\displaystyle {5}\)}%
\end{pgfscope}%
\begin{pgfscope}%
\pgfsetbuttcap%
\pgfsetroundjoin%
\definecolor{currentfill}{rgb}{0.000000,0.000000,0.000000}%
\pgfsetfillcolor{currentfill}%
\pgfsetlinewidth{0.803000pt}%
\definecolor{currentstroke}{rgb}{0.000000,0.000000,0.000000}%
\pgfsetstrokecolor{currentstroke}%
\pgfsetdash{}{0pt}%
\pgfsys@defobject{currentmarker}{\pgfqpoint{0.000000in}{-0.048611in}}{\pgfqpoint{0.000000in}{0.000000in}}{%
\pgfpathmoveto{\pgfqpoint{0.000000in}{0.000000in}}%
\pgfpathlineto{\pgfqpoint{0.000000in}{-0.048611in}}%
\pgfusepath{stroke,fill}%
}%
\begin{pgfscope}%
\pgfsys@transformshift{4.638944in}{3.786087in}%
\pgfsys@useobject{currentmarker}{}%
\end{pgfscope}%
\end{pgfscope}%
\begin{pgfscope}%
\definecolor{textcolor}{rgb}{0.000000,0.000000,0.000000}%
\pgfsetstrokecolor{textcolor}%
\pgfsetfillcolor{textcolor}%
\pgftext[x=4.638944in,y=3.688865in,,top]{\color{textcolor}\sffamily\fontsize{10.000000}{12.000000}\selectfont \(\displaystyle {6}\)}%
\end{pgfscope}%
\begin{pgfscope}%
\definecolor{textcolor}{rgb}{0.000000,0.000000,0.000000}%
\pgfsetstrokecolor{textcolor}%
\pgfsetfillcolor{textcolor}%
\pgftext[x=2.784000in,y=3.509853in,,top]{\color{textcolor}\sffamily\fontsize{10.000000}{12.000000}\selectfont \(\displaystyle y \, \mathrm{[mm]}\)}%
\end{pgfscope}%
\begin{pgfscope}%
\pgfsetbuttcap%
\pgfsetroundjoin%
\definecolor{currentfill}{rgb}{0.000000,0.000000,0.000000}%
\pgfsetfillcolor{currentfill}%
\pgfsetlinewidth{0.803000pt}%
\definecolor{currentstroke}{rgb}{0.000000,0.000000,0.000000}%
\pgfsetstrokecolor{currentstroke}%
\pgfsetdash{}{0pt}%
\pgfsys@defobject{currentmarker}{\pgfqpoint{-0.048611in}{0.000000in}}{\pgfqpoint{-0.000000in}{0.000000in}}{%
\pgfpathmoveto{\pgfqpoint{-0.000000in}{0.000000in}}%
\pgfpathlineto{\pgfqpoint{-0.048611in}{0.000000in}}%
\pgfusepath{stroke,fill}%
}%
\begin{pgfscope}%
\pgfsys@transformshift{0.800000in}{3.786087in}%
\pgfsys@useobject{currentmarker}{}%
\end{pgfscope}%
\end{pgfscope}%
\begin{pgfscope}%
\definecolor{textcolor}{rgb}{0.000000,0.000000,0.000000}%
\pgfsetstrokecolor{textcolor}%
\pgfsetfillcolor{textcolor}%
\pgftext[x=0.633333in, y=3.737862in, left, base]{\color{textcolor}\sffamily\fontsize{10.000000}{12.000000}\selectfont \(\displaystyle {0}\)}%
\end{pgfscope}%
\begin{pgfscope}%
\pgfsetbuttcap%
\pgfsetroundjoin%
\definecolor{currentfill}{rgb}{0.000000,0.000000,0.000000}%
\pgfsetfillcolor{currentfill}%
\pgfsetlinewidth{0.803000pt}%
\definecolor{currentstroke}{rgb}{0.000000,0.000000,0.000000}%
\pgfsetstrokecolor{currentstroke}%
\pgfsetdash{}{0pt}%
\pgfsys@defobject{currentmarker}{\pgfqpoint{-0.048611in}{0.000000in}}{\pgfqpoint{-0.000000in}{0.000000in}}{%
\pgfpathmoveto{\pgfqpoint{-0.000000in}{0.000000in}}%
\pgfpathlineto{\pgfqpoint{-0.048611in}{0.000000in}}%
\pgfusepath{stroke,fill}%
}%
\begin{pgfscope}%
\pgfsys@transformshift{0.800000in}{4.245809in}%
\pgfsys@useobject{currentmarker}{}%
\end{pgfscope}%
\end{pgfscope}%
\begin{pgfscope}%
\definecolor{textcolor}{rgb}{0.000000,0.000000,0.000000}%
\pgfsetstrokecolor{textcolor}%
\pgfsetfillcolor{textcolor}%
\pgftext[x=0.494444in, y=4.197584in, left, base]{\color{textcolor}\sffamily\fontsize{10.000000}{12.000000}\selectfont \(\displaystyle {100}\)}%
\end{pgfscope}%
\begin{pgfscope}%
\pgfsetbuttcap%
\pgfsetroundjoin%
\definecolor{currentfill}{rgb}{0.000000,0.000000,0.000000}%
\pgfsetfillcolor{currentfill}%
\pgfsetlinewidth{0.803000pt}%
\definecolor{currentstroke}{rgb}{0.000000,0.000000,0.000000}%
\pgfsetstrokecolor{currentstroke}%
\pgfsetdash{}{0pt}%
\pgfsys@defobject{currentmarker}{\pgfqpoint{-0.048611in}{0.000000in}}{\pgfqpoint{-0.000000in}{0.000000in}}{%
\pgfpathmoveto{\pgfqpoint{-0.000000in}{0.000000in}}%
\pgfpathlineto{\pgfqpoint{-0.048611in}{0.000000in}}%
\pgfusepath{stroke,fill}%
}%
\begin{pgfscope}%
\pgfsys@transformshift{0.800000in}{4.705532in}%
\pgfsys@useobject{currentmarker}{}%
\end{pgfscope}%
\end{pgfscope}%
\begin{pgfscope}%
\definecolor{textcolor}{rgb}{0.000000,0.000000,0.000000}%
\pgfsetstrokecolor{textcolor}%
\pgfsetfillcolor{textcolor}%
\pgftext[x=0.494444in, y=4.657307in, left, base]{\color{textcolor}\sffamily\fontsize{10.000000}{12.000000}\selectfont \(\displaystyle {200}\)}%
\end{pgfscope}%
\begin{pgfscope}%
\pgfsetbuttcap%
\pgfsetroundjoin%
\definecolor{currentfill}{rgb}{0.000000,0.000000,0.000000}%
\pgfsetfillcolor{currentfill}%
\pgfsetlinewidth{0.803000pt}%
\definecolor{currentstroke}{rgb}{0.000000,0.000000,0.000000}%
\pgfsetstrokecolor{currentstroke}%
\pgfsetdash{}{0pt}%
\pgfsys@defobject{currentmarker}{\pgfqpoint{-0.048611in}{0.000000in}}{\pgfqpoint{-0.000000in}{0.000000in}}{%
\pgfpathmoveto{\pgfqpoint{-0.000000in}{0.000000in}}%
\pgfpathlineto{\pgfqpoint{-0.048611in}{0.000000in}}%
\pgfusepath{stroke,fill}%
}%
\begin{pgfscope}%
\pgfsys@transformshift{0.800000in}{5.165254in}%
\pgfsys@useobject{currentmarker}{}%
\end{pgfscope}%
\end{pgfscope}%
\begin{pgfscope}%
\definecolor{textcolor}{rgb}{0.000000,0.000000,0.000000}%
\pgfsetstrokecolor{textcolor}%
\pgfsetfillcolor{textcolor}%
\pgftext[x=0.494444in, y=5.117029in, left, base]{\color{textcolor}\sffamily\fontsize{10.000000}{12.000000}\selectfont \(\displaystyle {300}\)}%
\end{pgfscope}%
\begin{pgfscope}%
\pgfsetbuttcap%
\pgfsetroundjoin%
\definecolor{currentfill}{rgb}{0.000000,0.000000,0.000000}%
\pgfsetfillcolor{currentfill}%
\pgfsetlinewidth{0.803000pt}%
\definecolor{currentstroke}{rgb}{0.000000,0.000000,0.000000}%
\pgfsetstrokecolor{currentstroke}%
\pgfsetdash{}{0pt}%
\pgfsys@defobject{currentmarker}{\pgfqpoint{-0.048611in}{0.000000in}}{\pgfqpoint{-0.000000in}{0.000000in}}{%
\pgfpathmoveto{\pgfqpoint{-0.000000in}{0.000000in}}%
\pgfpathlineto{\pgfqpoint{-0.048611in}{0.000000in}}%
\pgfusepath{stroke,fill}%
}%
\begin{pgfscope}%
\pgfsys@transformshift{0.800000in}{5.624977in}%
\pgfsys@useobject{currentmarker}{}%
\end{pgfscope}%
\end{pgfscope}%
\begin{pgfscope}%
\definecolor{textcolor}{rgb}{0.000000,0.000000,0.000000}%
\pgfsetstrokecolor{textcolor}%
\pgfsetfillcolor{textcolor}%
\pgftext[x=0.494444in, y=5.576751in, left, base]{\color{textcolor}\sffamily\fontsize{10.000000}{12.000000}\selectfont \(\displaystyle {400}\)}%
\end{pgfscope}%
\begin{pgfscope}%
\pgfsetbuttcap%
\pgfsetroundjoin%
\definecolor{currentfill}{rgb}{0.000000,0.000000,0.000000}%
\pgfsetfillcolor{currentfill}%
\pgfsetlinewidth{0.803000pt}%
\definecolor{currentstroke}{rgb}{0.000000,0.000000,0.000000}%
\pgfsetstrokecolor{currentstroke}%
\pgfsetdash{}{0pt}%
\pgfsys@defobject{currentmarker}{\pgfqpoint{-0.048611in}{0.000000in}}{\pgfqpoint{-0.000000in}{0.000000in}}{%
\pgfpathmoveto{\pgfqpoint{-0.000000in}{0.000000in}}%
\pgfpathlineto{\pgfqpoint{-0.048611in}{0.000000in}}%
\pgfusepath{stroke,fill}%
}%
\begin{pgfscope}%
\pgfsys@transformshift{0.800000in}{6.084699in}%
\pgfsys@useobject{currentmarker}{}%
\end{pgfscope}%
\end{pgfscope}%
\begin{pgfscope}%
\definecolor{textcolor}{rgb}{0.000000,0.000000,0.000000}%
\pgfsetstrokecolor{textcolor}%
\pgfsetfillcolor{textcolor}%
\pgftext[x=0.494444in, y=6.036474in, left, base]{\color{textcolor}\sffamily\fontsize{10.000000}{12.000000}\selectfont \(\displaystyle {500}\)}%
\end{pgfscope}%
\begin{pgfscope}%
\definecolor{textcolor}{rgb}{0.000000,0.000000,0.000000}%
\pgfsetstrokecolor{textcolor}%
\pgfsetfillcolor{textcolor}%
\pgftext[x=0.438888in,y=4.973043in,,bottom,rotate=90.000000]{\color{textcolor}\sffamily\fontsize{10.000000}{12.000000}\selectfont \(\displaystyle \mathrm{energy \, [MeV]}\)}%
\end{pgfscope}%
\begin{pgfscope}%
\pgfpathrectangle{\pgfqpoint{0.800000in}{3.786087in}}{\pgfqpoint{3.968000in}{2.373913in}}%
\pgfusepath{clip}%
\pgfsetrectcap%
\pgfsetroundjoin%
\pgfsetlinewidth{1.505625pt}%
\definecolor{currentstroke}{rgb}{0.839216,0.152941,0.156863}%
\pgfsetstrokecolor{currentstroke}%
\pgfsetdash{}{0pt}%
\pgfpathmoveto{\pgfqpoint{0.825114in}{5.012318in}}%
\pgfpathlineto{\pgfqpoint{0.875342in}{5.009492in}}%
\pgfpathlineto{\pgfqpoint{0.925570in}{5.012321in}}%
\pgfpathlineto{\pgfqpoint{0.975797in}{5.018683in}}%
\pgfpathlineto{\pgfqpoint{1.026025in}{5.037144in}}%
\pgfpathlineto{\pgfqpoint{1.076253in}{5.053251in}}%
\pgfpathlineto{\pgfqpoint{1.126481in}{5.065826in}}%
\pgfpathlineto{\pgfqpoint{1.176709in}{5.081474in}}%
\pgfpathlineto{\pgfqpoint{1.226937in}{5.097545in}}%
\pgfpathlineto{\pgfqpoint{1.277165in}{5.113479in}}%
\pgfpathlineto{\pgfqpoint{1.327392in}{5.128587in}}%
\pgfpathlineto{\pgfqpoint{1.377620in}{5.144345in}}%
\pgfpathlineto{\pgfqpoint{1.427848in}{5.162075in}}%
\pgfpathlineto{\pgfqpoint{1.478076in}{5.180050in}}%
\pgfpathlineto{\pgfqpoint{1.528304in}{5.197694in}}%
\pgfpathlineto{\pgfqpoint{1.578532in}{5.216865in}}%
\pgfpathlineto{\pgfqpoint{1.628759in}{5.237818in}}%
\pgfpathlineto{\pgfqpoint{1.678987in}{5.257979in}}%
\pgfpathlineto{\pgfqpoint{1.729215in}{5.277944in}}%
\pgfpathlineto{\pgfqpoint{1.779443in}{5.299396in}}%
\pgfpathlineto{\pgfqpoint{1.829671in}{5.322038in}}%
\pgfpathlineto{\pgfqpoint{1.879899in}{5.344400in}}%
\pgfpathlineto{\pgfqpoint{1.930127in}{5.366130in}}%
\pgfpathlineto{\pgfqpoint{1.980354in}{5.388735in}}%
\pgfpathlineto{\pgfqpoint{2.030582in}{5.412351in}}%
\pgfpathlineto{\pgfqpoint{2.080810in}{5.436406in}}%
\pgfpathlineto{\pgfqpoint{2.131038in}{5.459765in}}%
\pgfpathlineto{\pgfqpoint{2.181266in}{5.482700in}}%
\pgfpathlineto{\pgfqpoint{2.231494in}{5.506620in}}%
\pgfpathlineto{\pgfqpoint{2.281722in}{5.531359in}}%
\pgfpathlineto{\pgfqpoint{2.331949in}{5.556111in}}%
\pgfpathlineto{\pgfqpoint{2.382177in}{5.580764in}}%
\pgfpathlineto{\pgfqpoint{2.432405in}{5.604720in}}%
\pgfpathlineto{\pgfqpoint{2.482633in}{5.628436in}}%
\pgfpathlineto{\pgfqpoint{2.532861in}{5.652449in}}%
\pgfpathlineto{\pgfqpoint{2.583089in}{5.675850in}}%
\pgfpathlineto{\pgfqpoint{2.633316in}{5.698277in}}%
\pgfpathlineto{\pgfqpoint{2.683544in}{5.720861in}}%
\pgfpathlineto{\pgfqpoint{2.733772in}{5.743851in}}%
\pgfpathlineto{\pgfqpoint{2.784000in}{5.766789in}}%
\pgfpathlineto{\pgfqpoint{2.834228in}{5.789089in}}%
\pgfpathlineto{\pgfqpoint{2.884456in}{5.810680in}}%
\pgfpathlineto{\pgfqpoint{2.934684in}{5.830943in}}%
\pgfpathlineto{\pgfqpoint{2.984911in}{5.849929in}}%
\pgfpathlineto{\pgfqpoint{3.035139in}{5.867292in}}%
\pgfpathlineto{\pgfqpoint{3.085367in}{5.883117in}}%
\pgfpathlineto{\pgfqpoint{3.135595in}{5.898295in}}%
\pgfpathlineto{\pgfqpoint{3.185823in}{5.912685in}}%
\pgfpathlineto{\pgfqpoint{3.236051in}{5.926300in}}%
\pgfpathlineto{\pgfqpoint{3.286278in}{5.939542in}}%
\pgfpathlineto{\pgfqpoint{3.336506in}{5.952769in}}%
\pgfpathlineto{\pgfqpoint{3.386734in}{5.965850in}}%
\pgfpathlineto{\pgfqpoint{3.436962in}{5.978580in}}%
\pgfpathlineto{\pgfqpoint{3.487190in}{5.990720in}}%
\pgfpathlineto{\pgfqpoint{3.537418in}{6.002289in}}%
\pgfpathlineto{\pgfqpoint{3.587646in}{6.012693in}}%
\pgfpathlineto{\pgfqpoint{3.637873in}{6.022828in}}%
\pgfpathlineto{\pgfqpoint{3.688101in}{6.032202in}}%
\pgfpathlineto{\pgfqpoint{3.738329in}{6.041047in}}%
\pgfpathlineto{\pgfqpoint{3.788557in}{6.049225in}}%
\pgfpathlineto{\pgfqpoint{3.838785in}{6.056794in}}%
\pgfpathlineto{\pgfqpoint{3.889013in}{6.063871in}}%
\pgfpathlineto{\pgfqpoint{3.939241in}{6.070441in}}%
\pgfpathlineto{\pgfqpoint{3.989468in}{6.077426in}}%
\pgfpathlineto{\pgfqpoint{4.039696in}{6.084002in}}%
\pgfpathlineto{\pgfqpoint{4.089924in}{6.089903in}}%
\pgfpathlineto{\pgfqpoint{4.140152in}{6.095103in}}%
\pgfpathlineto{\pgfqpoint{4.190380in}{6.099880in}}%
\pgfpathlineto{\pgfqpoint{4.240608in}{6.104252in}}%
\pgfpathlineto{\pgfqpoint{4.290835in}{6.108233in}}%
\pgfpathlineto{\pgfqpoint{4.341063in}{6.111980in}}%
\pgfpathlineto{\pgfqpoint{4.391291in}{6.115487in}}%
\pgfpathlineto{\pgfqpoint{4.441519in}{6.118843in}}%
\pgfpathlineto{\pgfqpoint{4.491747in}{6.121907in}}%
\pgfpathlineto{\pgfqpoint{4.541975in}{6.124808in}}%
\pgfpathlineto{\pgfqpoint{4.592203in}{6.128068in}}%
\pgfpathlineto{\pgfqpoint{4.642430in}{6.131294in}}%
\pgfpathlineto{\pgfqpoint{4.692658in}{6.134261in}}%
\pgfpathlineto{\pgfqpoint{4.742886in}{6.137014in}}%
\pgfusepath{stroke}%
\end{pgfscope}%
\begin{pgfscope}%
\pgfpathrectangle{\pgfqpoint{0.800000in}{3.786087in}}{\pgfqpoint{3.968000in}{2.373913in}}%
\pgfusepath{clip}%
\pgfsetrectcap%
\pgfsetroundjoin%
\pgfsetlinewidth{1.505625pt}%
\definecolor{currentstroke}{rgb}{0.121569,0.466667,0.705882}%
\pgfsetstrokecolor{currentstroke}%
\pgfsetdash{}{0pt}%
\pgfpathmoveto{\pgfqpoint{0.825114in}{4.931759in}}%
\pgfpathlineto{\pgfqpoint{0.875342in}{4.928725in}}%
\pgfpathlineto{\pgfqpoint{0.925570in}{4.923113in}}%
\pgfpathlineto{\pgfqpoint{0.975797in}{4.917224in}}%
\pgfpathlineto{\pgfqpoint{1.026025in}{4.910749in}}%
\pgfpathlineto{\pgfqpoint{1.076253in}{4.903112in}}%
\pgfpathlineto{\pgfqpoint{1.126481in}{4.893487in}}%
\pgfpathlineto{\pgfqpoint{1.176709in}{4.881455in}}%
\pgfpathlineto{\pgfqpoint{1.226937in}{4.867483in}}%
\pgfpathlineto{\pgfqpoint{1.277165in}{4.852325in}}%
\pgfpathlineto{\pgfqpoint{1.327392in}{4.836480in}}%
\pgfpathlineto{\pgfqpoint{1.377620in}{4.820302in}}%
\pgfpathlineto{\pgfqpoint{1.427848in}{4.804106in}}%
\pgfpathlineto{\pgfqpoint{1.478076in}{4.788045in}}%
\pgfpathlineto{\pgfqpoint{1.528304in}{4.772241in}}%
\pgfpathlineto{\pgfqpoint{1.578532in}{4.756860in}}%
\pgfpathlineto{\pgfqpoint{1.628759in}{4.741949in}}%
\pgfpathlineto{\pgfqpoint{1.678987in}{4.727488in}}%
\pgfpathlineto{\pgfqpoint{1.729215in}{4.713539in}}%
\pgfpathlineto{\pgfqpoint{1.779443in}{4.700110in}}%
\pgfpathlineto{\pgfqpoint{1.829671in}{4.687167in}}%
\pgfpathlineto{\pgfqpoint{1.879899in}{4.674760in}}%
\pgfpathlineto{\pgfqpoint{1.930127in}{4.662887in}}%
\pgfpathlineto{\pgfqpoint{1.980354in}{4.651504in}}%
\pgfpathlineto{\pgfqpoint{2.030582in}{4.640592in}}%
\pgfpathlineto{\pgfqpoint{2.080810in}{4.630174in}}%
\pgfpathlineto{\pgfqpoint{2.131038in}{4.620253in}}%
\pgfpathlineto{\pgfqpoint{2.181266in}{4.610846in}}%
\pgfpathlineto{\pgfqpoint{2.231494in}{4.601893in}}%
\pgfpathlineto{\pgfqpoint{2.281722in}{4.593409in}}%
\pgfpathlineto{\pgfqpoint{2.331949in}{4.585280in}}%
\pgfpathlineto{\pgfqpoint{2.382177in}{4.577504in}}%
\pgfpathlineto{\pgfqpoint{2.432405in}{4.570112in}}%
\pgfpathlineto{\pgfqpoint{2.482633in}{4.563055in}}%
\pgfpathlineto{\pgfqpoint{2.532861in}{4.556284in}}%
\pgfpathlineto{\pgfqpoint{2.583089in}{4.549869in}}%
\pgfpathlineto{\pgfqpoint{2.633316in}{4.543803in}}%
\pgfpathlineto{\pgfqpoint{2.683544in}{4.538061in}}%
\pgfpathlineto{\pgfqpoint{2.733772in}{4.532652in}}%
\pgfpathlineto{\pgfqpoint{2.784000in}{4.527617in}}%
\pgfpathlineto{\pgfqpoint{2.834228in}{4.522888in}}%
\pgfpathlineto{\pgfqpoint{2.884456in}{4.518458in}}%
\pgfpathlineto{\pgfqpoint{2.934684in}{4.514275in}}%
\pgfpathlineto{\pgfqpoint{2.984911in}{4.510383in}}%
\pgfpathlineto{\pgfqpoint{3.035139in}{4.506848in}}%
\pgfpathlineto{\pgfqpoint{3.085367in}{4.503508in}}%
\pgfpathlineto{\pgfqpoint{3.135595in}{4.500508in}}%
\pgfpathlineto{\pgfqpoint{3.185823in}{4.497650in}}%
\pgfpathlineto{\pgfqpoint{3.236051in}{4.495003in}}%
\pgfpathlineto{\pgfqpoint{3.286278in}{4.492485in}}%
\pgfpathlineto{\pgfqpoint{3.336506in}{4.489949in}}%
\pgfpathlineto{\pgfqpoint{3.386734in}{4.487692in}}%
\pgfpathlineto{\pgfqpoint{3.436962in}{4.485581in}}%
\pgfpathlineto{\pgfqpoint{3.487190in}{4.483456in}}%
\pgfpathlineto{\pgfqpoint{3.537418in}{4.481566in}}%
\pgfpathlineto{\pgfqpoint{3.587646in}{4.479708in}}%
\pgfpathlineto{\pgfqpoint{3.637873in}{4.477900in}}%
\pgfpathlineto{\pgfqpoint{3.688101in}{4.476172in}}%
\pgfpathlineto{\pgfqpoint{3.738329in}{4.474520in}}%
\pgfpathlineto{\pgfqpoint{3.788557in}{4.472931in}}%
\pgfpathlineto{\pgfqpoint{3.838785in}{4.471336in}}%
\pgfpathlineto{\pgfqpoint{3.889013in}{4.469688in}}%
\pgfpathlineto{\pgfqpoint{3.939241in}{4.468028in}}%
\pgfpathlineto{\pgfqpoint{3.989468in}{4.466415in}}%
\pgfpathlineto{\pgfqpoint{4.039696in}{4.464976in}}%
\pgfpathlineto{\pgfqpoint{4.089924in}{4.463692in}}%
\pgfpathlineto{\pgfqpoint{4.140152in}{4.462095in}}%
\pgfpathlineto{\pgfqpoint{4.190380in}{4.460690in}}%
\pgfpathlineto{\pgfqpoint{4.240608in}{4.459290in}}%
\pgfpathlineto{\pgfqpoint{4.290835in}{4.457846in}}%
\pgfpathlineto{\pgfqpoint{4.341063in}{4.456527in}}%
\pgfpathlineto{\pgfqpoint{4.391291in}{4.455136in}}%
\pgfpathlineto{\pgfqpoint{4.441519in}{4.453814in}}%
\pgfpathlineto{\pgfqpoint{4.491747in}{4.452386in}}%
\pgfpathlineto{\pgfqpoint{4.541975in}{4.450913in}}%
\pgfpathlineto{\pgfqpoint{4.592203in}{4.449479in}}%
\pgfpathlineto{\pgfqpoint{4.642430in}{4.448125in}}%
\pgfpathlineto{\pgfqpoint{4.692658in}{4.446671in}}%
\pgfpathlineto{\pgfqpoint{4.742886in}{4.445301in}}%
\pgfusepath{stroke}%
\end{pgfscope}%
\begin{pgfscope}%
\pgfpathrectangle{\pgfqpoint{0.800000in}{3.786087in}}{\pgfqpoint{3.968000in}{2.373913in}}%
\pgfusepath{clip}%
\pgfsetrectcap%
\pgfsetroundjoin%
\pgfsetlinewidth{1.505625pt}%
\definecolor{currentstroke}{rgb}{0.172549,0.627451,0.172549}%
\pgfsetstrokecolor{currentstroke}%
\pgfsetdash{}{0pt}%
\pgfpathmoveto{\pgfqpoint{0.825114in}{4.932142in}}%
\pgfpathlineto{\pgfqpoint{0.875342in}{4.935579in}}%
\pgfpathlineto{\pgfqpoint{0.925570in}{4.935488in}}%
\pgfpathlineto{\pgfqpoint{0.975797in}{4.929193in}}%
\pgfpathlineto{\pgfqpoint{1.026025in}{4.927616in}}%
\pgfpathlineto{\pgfqpoint{1.076253in}{4.920531in}}%
\pgfpathlineto{\pgfqpoint{1.126481in}{4.915960in}}%
\pgfpathlineto{\pgfqpoint{1.176709in}{4.911712in}}%
\pgfpathlineto{\pgfqpoint{1.226937in}{4.909064in}}%
\pgfpathlineto{\pgfqpoint{1.277165in}{4.907692in}}%
\pgfpathlineto{\pgfqpoint{1.327392in}{4.907098in}}%
\pgfpathlineto{\pgfqpoint{1.377620in}{4.906523in}}%
\pgfpathlineto{\pgfqpoint{1.427848in}{4.906260in}}%
\pgfpathlineto{\pgfqpoint{1.478076in}{4.906127in}}%
\pgfpathlineto{\pgfqpoint{1.528304in}{4.906303in}}%
\pgfpathlineto{\pgfqpoint{1.578532in}{4.906564in}}%
\pgfpathlineto{\pgfqpoint{1.628759in}{4.906605in}}%
\pgfpathlineto{\pgfqpoint{1.678987in}{4.906785in}}%
\pgfpathlineto{\pgfqpoint{1.729215in}{4.906877in}}%
\pgfpathlineto{\pgfqpoint{1.779443in}{4.906765in}}%
\pgfpathlineto{\pgfqpoint{1.829671in}{4.906810in}}%
\pgfpathlineto{\pgfqpoint{1.879899in}{4.906736in}}%
\pgfpathlineto{\pgfqpoint{1.930127in}{4.906599in}}%
\pgfpathlineto{\pgfqpoint{1.980354in}{4.906590in}}%
\pgfpathlineto{\pgfqpoint{2.030582in}{4.906655in}}%
\pgfpathlineto{\pgfqpoint{2.080810in}{4.906695in}}%
\pgfpathlineto{\pgfqpoint{2.131038in}{4.906814in}}%
\pgfpathlineto{\pgfqpoint{2.181266in}{4.906956in}}%
\pgfpathlineto{\pgfqpoint{2.231494in}{4.906988in}}%
\pgfpathlineto{\pgfqpoint{2.281722in}{4.906909in}}%
\pgfpathlineto{\pgfqpoint{2.331949in}{4.906787in}}%
\pgfpathlineto{\pgfqpoint{2.382177in}{4.906650in}}%
\pgfpathlineto{\pgfqpoint{2.432405in}{4.906500in}}%
\pgfpathlineto{\pgfqpoint{2.482633in}{4.906359in}}%
\pgfpathlineto{\pgfqpoint{2.532861in}{4.906173in}}%
\pgfpathlineto{\pgfqpoint{2.583089in}{4.906039in}}%
\pgfpathlineto{\pgfqpoint{2.633316in}{4.905831in}}%
\pgfpathlineto{\pgfqpoint{2.683544in}{4.905552in}}%
\pgfpathlineto{\pgfqpoint{2.733772in}{4.905259in}}%
\pgfpathlineto{\pgfqpoint{2.784000in}{4.904974in}}%
\pgfpathlineto{\pgfqpoint{2.834228in}{4.904702in}}%
\pgfpathlineto{\pgfqpoint{2.884456in}{4.904528in}}%
\pgfpathlineto{\pgfqpoint{2.934684in}{4.904366in}}%
\pgfpathlineto{\pgfqpoint{2.984911in}{4.904270in}}%
\pgfpathlineto{\pgfqpoint{3.035139in}{4.904105in}}%
\pgfpathlineto{\pgfqpoint{3.085367in}{4.903937in}}%
\pgfpathlineto{\pgfqpoint{3.135595in}{4.903796in}}%
\pgfpathlineto{\pgfqpoint{3.185823in}{4.903668in}}%
\pgfpathlineto{\pgfqpoint{3.236051in}{4.903475in}}%
\pgfpathlineto{\pgfqpoint{3.286278in}{4.903195in}}%
\pgfpathlineto{\pgfqpoint{3.336506in}{4.902983in}}%
\pgfpathlineto{\pgfqpoint{3.386734in}{4.902793in}}%
\pgfpathlineto{\pgfqpoint{3.436962in}{4.902554in}}%
\pgfpathlineto{\pgfqpoint{3.487190in}{4.902365in}}%
\pgfpathlineto{\pgfqpoint{3.537418in}{4.902172in}}%
\pgfpathlineto{\pgfqpoint{3.587646in}{4.901909in}}%
\pgfpathlineto{\pgfqpoint{3.637873in}{4.901593in}}%
\pgfpathlineto{\pgfqpoint{3.688101in}{4.901249in}}%
\pgfpathlineto{\pgfqpoint{3.738329in}{4.900931in}}%
\pgfpathlineto{\pgfqpoint{3.788557in}{4.900695in}}%
\pgfpathlineto{\pgfqpoint{3.838785in}{4.900460in}}%
\pgfpathlineto{\pgfqpoint{3.889013in}{4.900165in}}%
\pgfpathlineto{\pgfqpoint{3.939241in}{4.899929in}}%
\pgfpathlineto{\pgfqpoint{3.989468in}{4.899704in}}%
\pgfpathlineto{\pgfqpoint{4.039696in}{4.899447in}}%
\pgfpathlineto{\pgfqpoint{4.089924in}{4.899192in}}%
\pgfpathlineto{\pgfqpoint{4.140152in}{4.898951in}}%
\pgfpathlineto{\pgfqpoint{4.190380in}{4.898817in}}%
\pgfpathlineto{\pgfqpoint{4.240608in}{4.898520in}}%
\pgfpathlineto{\pgfqpoint{4.290835in}{4.898274in}}%
\pgfpathlineto{\pgfqpoint{4.341063in}{4.897987in}}%
\pgfpathlineto{\pgfqpoint{4.391291in}{4.897752in}}%
\pgfpathlineto{\pgfqpoint{4.441519in}{4.897516in}}%
\pgfpathlineto{\pgfqpoint{4.491747in}{4.897066in}}%
\pgfpathlineto{\pgfqpoint{4.541975in}{4.896812in}}%
\pgfpathlineto{\pgfqpoint{4.592203in}{4.896594in}}%
\pgfpathlineto{\pgfqpoint{4.642430in}{4.896356in}}%
\pgfpathlineto{\pgfqpoint{4.692658in}{4.896095in}}%
\pgfpathlineto{\pgfqpoint{4.742886in}{4.895761in}}%
\pgfusepath{stroke}%
\end{pgfscope}%
\begin{pgfscope}%
\pgfpathrectangle{\pgfqpoint{0.800000in}{3.786087in}}{\pgfqpoint{3.968000in}{2.373913in}}%
\pgfusepath{clip}%
\pgfsetrectcap%
\pgfsetroundjoin%
\pgfsetlinewidth{1.505625pt}%
\definecolor{currentstroke}{rgb}{0.000000,0.000000,0.000000}%
\pgfsetstrokecolor{currentstroke}%
\pgfsetdash{}{0pt}%
\pgfpathmoveto{\pgfqpoint{0.825114in}{4.844465in}}%
\pgfpathlineto{\pgfqpoint{0.875342in}{4.841203in}}%
\pgfpathlineto{\pgfqpoint{0.925570in}{4.835176in}}%
\pgfpathlineto{\pgfqpoint{0.975797in}{4.824673in}}%
\pgfpathlineto{\pgfqpoint{1.026025in}{4.808791in}}%
\pgfpathlineto{\pgfqpoint{1.076253in}{4.788197in}}%
\pgfpathlineto{\pgfqpoint{1.126481in}{4.757914in}}%
\pgfpathlineto{\pgfqpoint{1.176709in}{4.720208in}}%
\pgfpathlineto{\pgfqpoint{1.226937in}{4.679207in}}%
\pgfpathlineto{\pgfqpoint{1.277165in}{4.636132in}}%
\pgfpathlineto{\pgfqpoint{1.327392in}{4.590917in}}%
\pgfpathlineto{\pgfqpoint{1.377620in}{4.546189in}}%
\pgfpathlineto{\pgfqpoint{1.427848in}{4.504163in}}%
\pgfpathlineto{\pgfqpoint{1.478076in}{4.462026in}}%
\pgfpathlineto{\pgfqpoint{1.528304in}{4.418530in}}%
\pgfpathlineto{\pgfqpoint{1.578532in}{4.376125in}}%
\pgfpathlineto{\pgfqpoint{1.628759in}{4.335443in}}%
\pgfpathlineto{\pgfqpoint{1.678987in}{4.294846in}}%
\pgfpathlineto{\pgfqpoint{1.729215in}{4.255978in}}%
\pgfpathlineto{\pgfqpoint{1.779443in}{4.218975in}}%
\pgfpathlineto{\pgfqpoint{1.829671in}{4.182026in}}%
\pgfpathlineto{\pgfqpoint{1.879899in}{4.146307in}}%
\pgfpathlineto{\pgfqpoint{1.930127in}{4.113025in}}%
\pgfpathlineto{\pgfqpoint{1.980354in}{4.080506in}}%
\pgfpathlineto{\pgfqpoint{2.030582in}{4.048040in}}%
\pgfpathlineto{\pgfqpoint{2.080810in}{4.017526in}}%
\pgfpathlineto{\pgfqpoint{2.131038in}{3.988465in}}%
\pgfpathlineto{\pgfqpoint{2.181266in}{3.959883in}}%
\pgfpathlineto{\pgfqpoint{2.231494in}{3.933509in}}%
\pgfpathlineto{\pgfqpoint{2.281722in}{3.908706in}}%
\pgfpathlineto{\pgfqpoint{2.331949in}{3.883450in}}%
\pgfpathlineto{\pgfqpoint{2.382177in}{3.859530in}}%
\pgfpathlineto{\pgfqpoint{2.432405in}{3.836048in}}%
\pgfpathlineto{\pgfqpoint{2.482633in}{3.815278in}}%
\pgfpathlineto{\pgfqpoint{2.532861in}{3.796057in}}%
\pgfpathlineto{\pgfqpoint{2.583089in}{3.790046in}}%
\pgfpathlineto{\pgfqpoint{2.633316in}{3.789142in}}%
\pgfpathlineto{\pgfqpoint{2.683544in}{3.790098in}}%
\pgfpathlineto{\pgfqpoint{2.733772in}{3.788730in}}%
\pgfpathlineto{\pgfqpoint{2.784000in}{3.788139in}}%
\pgfpathlineto{\pgfqpoint{2.834228in}{3.788145in}}%
\pgfpathlineto{\pgfqpoint{2.884456in}{3.788686in}}%
\pgfpathlineto{\pgfqpoint{2.934684in}{3.790335in}}%
\pgfpathlineto{\pgfqpoint{2.984911in}{3.789639in}}%
\pgfpathlineto{\pgfqpoint{3.035139in}{3.789977in}}%
\pgfpathlineto{\pgfqpoint{3.085367in}{3.790662in}}%
\pgfpathlineto{\pgfqpoint{3.135595in}{3.790767in}}%
\pgfpathlineto{\pgfqpoint{3.185823in}{3.789879in}}%
\pgfpathlineto{\pgfqpoint{3.236051in}{3.790353in}}%
\pgfpathlineto{\pgfqpoint{3.286278in}{3.791239in}}%
\pgfpathlineto{\pgfqpoint{3.336506in}{3.792273in}}%
\pgfpathlineto{\pgfqpoint{3.386734in}{3.792243in}}%
\pgfpathlineto{\pgfqpoint{3.436962in}{3.793000in}}%
\pgfpathlineto{\pgfqpoint{3.487190in}{3.792900in}}%
\pgfpathlineto{\pgfqpoint{3.537418in}{3.794481in}}%
\pgfpathlineto{\pgfqpoint{3.587646in}{3.792582in}}%
\pgfpathlineto{\pgfqpoint{3.637873in}{3.793288in}}%
\pgfpathlineto{\pgfqpoint{3.688101in}{3.791893in}}%
\pgfpathlineto{\pgfqpoint{3.738329in}{3.794414in}}%
\pgfpathlineto{\pgfqpoint{3.788557in}{3.795328in}}%
\pgfpathlineto{\pgfqpoint{3.838785in}{3.792341in}}%
\pgfpathlineto{\pgfqpoint{3.889013in}{3.793665in}}%
\pgfpathlineto{\pgfqpoint{3.939241in}{3.794983in}}%
\pgfpathlineto{\pgfqpoint{3.989468in}{3.794302in}}%
\pgfpathlineto{\pgfqpoint{4.039696in}{3.794937in}}%
\pgfpathlineto{\pgfqpoint{4.089924in}{3.796370in}}%
\pgfpathlineto{\pgfqpoint{4.140152in}{3.798081in}}%
\pgfpathlineto{\pgfqpoint{4.190380in}{3.796830in}}%
\pgfpathlineto{\pgfqpoint{4.240608in}{3.797451in}}%
\pgfpathlineto{\pgfqpoint{4.290835in}{3.798588in}}%
\pgfpathlineto{\pgfqpoint{4.341063in}{3.798812in}}%
\pgfpathlineto{\pgfqpoint{4.391291in}{3.796694in}}%
\pgfpathlineto{\pgfqpoint{4.441519in}{3.796229in}}%
\pgfpathlineto{\pgfqpoint{4.491747in}{3.797771in}}%
\pgfpathlineto{\pgfqpoint{4.541975in}{3.797543in}}%
\pgfpathlineto{\pgfqpoint{4.592203in}{3.796703in}}%
\pgfpathlineto{\pgfqpoint{4.642430in}{3.798129in}}%
\pgfpathlineto{\pgfqpoint{4.692658in}{3.799939in}}%
\pgfpathlineto{\pgfqpoint{4.742886in}{3.801296in}}%
\pgfusepath{stroke}%
\end{pgfscope}%
\begin{pgfscope}%
\pgfsetrectcap%
\pgfsetmiterjoin%
\pgfsetlinewidth{0.803000pt}%
\definecolor{currentstroke}{rgb}{0.000000,0.000000,0.000000}%
\pgfsetstrokecolor{currentstroke}%
\pgfsetdash{}{0pt}%
\pgfpathmoveto{\pgfqpoint{0.800000in}{3.786087in}}%
\pgfpathlineto{\pgfqpoint{0.800000in}{6.160000in}}%
\pgfusepath{stroke}%
\end{pgfscope}%
\begin{pgfscope}%
\pgfsetrectcap%
\pgfsetmiterjoin%
\pgfsetlinewidth{0.803000pt}%
\definecolor{currentstroke}{rgb}{0.000000,0.000000,0.000000}%
\pgfsetstrokecolor{currentstroke}%
\pgfsetdash{}{0pt}%
\pgfpathmoveto{\pgfqpoint{4.768000in}{3.786087in}}%
\pgfpathlineto{\pgfqpoint{4.768000in}{6.160000in}}%
\pgfusepath{stroke}%
\end{pgfscope}%
\begin{pgfscope}%
\pgfsetrectcap%
\pgfsetmiterjoin%
\pgfsetlinewidth{0.803000pt}%
\definecolor{currentstroke}{rgb}{0.000000,0.000000,0.000000}%
\pgfsetstrokecolor{currentstroke}%
\pgfsetdash{}{0pt}%
\pgfpathmoveto{\pgfqpoint{0.800000in}{3.786087in}}%
\pgfpathlineto{\pgfqpoint{4.768000in}{3.786087in}}%
\pgfusepath{stroke}%
\end{pgfscope}%
\begin{pgfscope}%
\pgfsetrectcap%
\pgfsetmiterjoin%
\pgfsetlinewidth{0.803000pt}%
\definecolor{currentstroke}{rgb}{0.000000,0.000000,0.000000}%
\pgfsetstrokecolor{currentstroke}%
\pgfsetdash{}{0pt}%
\pgfpathmoveto{\pgfqpoint{0.800000in}{6.160000in}}%
\pgfpathlineto{\pgfqpoint{4.768000in}{6.160000in}}%
\pgfusepath{stroke}%
\end{pgfscope}%
\begin{pgfscope}%
\definecolor{textcolor}{rgb}{0.000000,0.000000,0.000000}%
\pgfsetstrokecolor{textcolor}%
\pgfsetfillcolor{textcolor}%
\pgftext[x=2.784000in,y=6.243333in,,base]{\color{textcolor}\sffamily\fontsize{12.000000}{14.400000}\selectfont a)}%
\end{pgfscope}%
\begin{pgfscope}%
\pgfsetbuttcap%
\pgfsetmiterjoin%
\definecolor{currentfill}{rgb}{1.000000,1.000000,1.000000}%
\pgfsetfillcolor{currentfill}%
\pgfsetfillopacity{0.800000}%
\pgfsetlinewidth{1.003750pt}%
\definecolor{currentstroke}{rgb}{0.800000,0.800000,0.800000}%
\pgfsetstrokecolor{currentstroke}%
\pgfsetstrokeopacity{0.800000}%
\pgfsetdash{}{0pt}%
\pgfpathmoveto{\pgfqpoint{0.897222in}{5.261467in}}%
\pgfpathlineto{\pgfqpoint{1.731294in}{5.261467in}}%
\pgfpathquadraticcurveto{\pgfqpoint{1.759072in}{5.261467in}}{\pgfqpoint{1.759072in}{5.289244in}}%
\pgfpathlineto{\pgfqpoint{1.759072in}{6.062778in}}%
\pgfpathquadraticcurveto{\pgfqpoint{1.759072in}{6.090556in}}{\pgfqpoint{1.731294in}{6.090556in}}%
\pgfpathlineto{\pgfqpoint{0.897222in}{6.090556in}}%
\pgfpathquadraticcurveto{\pgfqpoint{0.869444in}{6.090556in}}{\pgfqpoint{0.869444in}{6.062778in}}%
\pgfpathlineto{\pgfqpoint{0.869444in}{5.289244in}}%
\pgfpathquadraticcurveto{\pgfqpoint{0.869444in}{5.261467in}}{\pgfqpoint{0.897222in}{5.261467in}}%
\pgfpathlineto{\pgfqpoint{0.897222in}{5.261467in}}%
\pgfpathclose%
\pgfusepath{stroke,fill}%
\end{pgfscope}%
\begin{pgfscope}%
\pgfsetrectcap%
\pgfsetroundjoin%
\pgfsetlinewidth{1.505625pt}%
\definecolor{currentstroke}{rgb}{0.839216,0.152941,0.156863}%
\pgfsetstrokecolor{currentstroke}%
\pgfsetdash{}{0pt}%
\pgfpathmoveto{\pgfqpoint{0.925000in}{5.986389in}}%
\pgfpathlineto{\pgfqpoint{1.063889in}{5.986389in}}%
\pgfpathlineto{\pgfqpoint{1.202778in}{5.986389in}}%
\pgfusepath{stroke}%
\end{pgfscope}%
\begin{pgfscope}%
\definecolor{textcolor}{rgb}{0.000000,0.000000,0.000000}%
\pgfsetstrokecolor{textcolor}%
\pgfsetfillcolor{textcolor}%
\pgftext[x=1.313889in,y=5.937778in,left,base]{\color{textcolor}\sffamily\fontsize{10.000000}{12.000000}\selectfont \(\displaystyle E_{max}\)}%
\end{pgfscope}%
\begin{pgfscope}%
\pgfsetrectcap%
\pgfsetroundjoin%
\pgfsetlinewidth{1.505625pt}%
\definecolor{currentstroke}{rgb}{0.121569,0.466667,0.705882}%
\pgfsetstrokecolor{currentstroke}%
\pgfsetdash{}{0pt}%
\pgfpathmoveto{\pgfqpoint{0.925000in}{5.792716in}}%
\pgfpathlineto{\pgfqpoint{1.063889in}{5.792716in}}%
\pgfpathlineto{\pgfqpoint{1.202778in}{5.792716in}}%
\pgfusepath{stroke}%
\end{pgfscope}%
\begin{pgfscope}%
\definecolor{textcolor}{rgb}{0.000000,0.000000,0.000000}%
\pgfsetstrokecolor{textcolor}%
\pgfsetfillcolor{textcolor}%
\pgftext[x=1.313889in,y=5.744105in,left,base]{\color{textcolor}\sffamily\fontsize{10.000000}{12.000000}\selectfont \(\displaystyle E_{mean}\)}%
\end{pgfscope}%
\begin{pgfscope}%
\pgfsetrectcap%
\pgfsetroundjoin%
\pgfsetlinewidth{1.505625pt}%
\definecolor{currentstroke}{rgb}{0.172549,0.627451,0.172549}%
\pgfsetstrokecolor{currentstroke}%
\pgfsetdash{}{0pt}%
\pgfpathmoveto{\pgfqpoint{0.925000in}{5.599043in}}%
\pgfpathlineto{\pgfqpoint{1.063889in}{5.599043in}}%
\pgfpathlineto{\pgfqpoint{1.202778in}{5.599043in}}%
\pgfusepath{stroke}%
\end{pgfscope}%
\begin{pgfscope}%
\definecolor{textcolor}{rgb}{0.000000,0.000000,0.000000}%
\pgfsetstrokecolor{textcolor}%
\pgfsetfillcolor{textcolor}%
\pgftext[x=1.313889in,y=5.550432in,left,base]{\color{textcolor}\sffamily\fontsize{10.000000}{12.000000}\selectfont \(\displaystyle E_{peak}\)}%
\end{pgfscope}%
\begin{pgfscope}%
\pgfsetrectcap%
\pgfsetroundjoin%
\pgfsetlinewidth{1.505625pt}%
\definecolor{currentstroke}{rgb}{0.000000,0.000000,0.000000}%
\pgfsetstrokecolor{currentstroke}%
\pgfsetdash{}{0pt}%
\pgfpathmoveto{\pgfqpoint{0.925000in}{5.392639in}}%
\pgfpathlineto{\pgfqpoint{1.063889in}{5.392639in}}%
\pgfpathlineto{\pgfqpoint{1.202778in}{5.392639in}}%
\pgfusepath{stroke}%
\end{pgfscope}%
\begin{pgfscope}%
\definecolor{textcolor}{rgb}{0.000000,0.000000,0.000000}%
\pgfsetstrokecolor{textcolor}%
\pgfsetfillcolor{textcolor}%
\pgftext[x=1.313889in,y=5.344028in,left,base]{\color{textcolor}\sffamily\fontsize{10.000000}{12.000000}\selectfont \(\displaystyle E_{min}\)}%
\end{pgfscope}%
\begin{pgfscope}%
\pgfsetbuttcap%
\pgfsetmiterjoin%
\definecolor{currentfill}{rgb}{1.000000,1.000000,1.000000}%
\pgfsetfillcolor{currentfill}%
\pgfsetlinewidth{0.000000pt}%
\definecolor{currentstroke}{rgb}{0.000000,0.000000,0.000000}%
\pgfsetstrokecolor{currentstroke}%
\pgfsetstrokeopacity{0.000000}%
\pgfsetdash{}{0pt}%
\pgfpathmoveto{\pgfqpoint{5.016000in}{3.786087in}}%
\pgfpathlineto{\pgfqpoint{5.134696in}{3.786087in}}%
\pgfpathlineto{\pgfqpoint{5.134696in}{6.160000in}}%
\pgfpathlineto{\pgfqpoint{5.016000in}{6.160000in}}%
\pgfpathlineto{\pgfqpoint{5.016000in}{3.786087in}}%
\pgfpathclose%
\pgfusepath{fill}%
\end{pgfscope}%
\begin{pgfscope}%
\pgfpathrectangle{\pgfqpoint{5.016000in}{3.786087in}}{\pgfqpoint{0.118696in}{2.373913in}}%
\pgfusepath{clip}%
\pgfsetbuttcap%
\pgfsetmiterjoin%
\definecolor{currentfill}{rgb}{1.000000,1.000000,1.000000}%
\pgfsetfillcolor{currentfill}%
\pgfsetlinewidth{0.010037pt}%
\definecolor{currentstroke}{rgb}{1.000000,1.000000,1.000000}%
\pgfsetstrokecolor{currentstroke}%
\pgfsetdash{}{0pt}%
\pgfusepath{stroke,fill}%
\end{pgfscope}%
\begin{pgfscope}%
\pgfsys@transformshift{5.013889in}{3.791667in}%
\pgftext[left,bottom]{\includegraphics[interpolate=true,width=0.125000in,height=2.375000in]{E_hist_time-img1.png}}%
\end{pgfscope}%
\begin{pgfscope}%
\pgfsetbuttcap%
\pgfsetroundjoin%
\definecolor{currentfill}{rgb}{0.000000,0.000000,0.000000}%
\pgfsetfillcolor{currentfill}%
\pgfsetlinewidth{0.803000pt}%
\definecolor{currentstroke}{rgb}{0.000000,0.000000,0.000000}%
\pgfsetstrokecolor{currentstroke}%
\pgfsetdash{}{0pt}%
\pgfsys@defobject{currentmarker}{\pgfqpoint{0.000000in}{0.000000in}}{\pgfqpoint{0.048611in}{0.000000in}}{%
\pgfpathmoveto{\pgfqpoint{0.000000in}{0.000000in}}%
\pgfpathlineto{\pgfqpoint{0.048611in}{0.000000in}}%
\pgfusepath{stroke,fill}%
}%
\begin{pgfscope}%
\pgfsys@transformshift{5.134696in}{3.786087in}%
\pgfsys@useobject{currentmarker}{}%
\end{pgfscope}%
\end{pgfscope}%
\begin{pgfscope}%
\definecolor{textcolor}{rgb}{0.000000,0.000000,0.000000}%
\pgfsetstrokecolor{textcolor}%
\pgfsetfillcolor{textcolor}%
\pgftext[x=5.231918in, y=3.737862in, left, base]{\color{textcolor}\sffamily\fontsize{10.000000}{12.000000}\selectfont \(\displaystyle {10^{1}}\)}%
\end{pgfscope}%
\begin{pgfscope}%
\pgfsetbuttcap%
\pgfsetroundjoin%
\definecolor{currentfill}{rgb}{0.000000,0.000000,0.000000}%
\pgfsetfillcolor{currentfill}%
\pgfsetlinewidth{0.803000pt}%
\definecolor{currentstroke}{rgb}{0.000000,0.000000,0.000000}%
\pgfsetstrokecolor{currentstroke}%
\pgfsetdash{}{0pt}%
\pgfsys@defobject{currentmarker}{\pgfqpoint{0.000000in}{0.000000in}}{\pgfqpoint{0.048611in}{0.000000in}}{%
\pgfpathmoveto{\pgfqpoint{0.000000in}{0.000000in}}%
\pgfpathlineto{\pgfqpoint{0.048611in}{0.000000in}}%
\pgfusepath{stroke,fill}%
}%
\begin{pgfscope}%
\pgfsys@transformshift{5.134696in}{5.183353in}%
\pgfsys@useobject{currentmarker}{}%
\end{pgfscope}%
\end{pgfscope}%
\begin{pgfscope}%
\definecolor{textcolor}{rgb}{0.000000,0.000000,0.000000}%
\pgfsetstrokecolor{textcolor}%
\pgfsetfillcolor{textcolor}%
\pgftext[x=5.231918in, y=5.135128in, left, base]{\color{textcolor}\sffamily\fontsize{10.000000}{12.000000}\selectfont \(\displaystyle {10^{2}}\)}%
\end{pgfscope}%
\begin{pgfscope}%
\pgfsetbuttcap%
\pgfsetroundjoin%
\definecolor{currentfill}{rgb}{0.000000,0.000000,0.000000}%
\pgfsetfillcolor{currentfill}%
\pgfsetlinewidth{0.602250pt}%
\definecolor{currentstroke}{rgb}{0.000000,0.000000,0.000000}%
\pgfsetstrokecolor{currentstroke}%
\pgfsetdash{}{0pt}%
\pgfsys@defobject{currentmarker}{\pgfqpoint{0.000000in}{0.000000in}}{\pgfqpoint{0.027778in}{0.000000in}}{%
\pgfpathmoveto{\pgfqpoint{0.000000in}{0.000000in}}%
\pgfpathlineto{\pgfqpoint{0.027778in}{0.000000in}}%
\pgfusepath{stroke,fill}%
}%
\begin{pgfscope}%
\pgfsys@transformshift{5.134696in}{4.206706in}%
\pgfsys@useobject{currentmarker}{}%
\end{pgfscope}%
\end{pgfscope}%
\begin{pgfscope}%
\pgfsetbuttcap%
\pgfsetroundjoin%
\definecolor{currentfill}{rgb}{0.000000,0.000000,0.000000}%
\pgfsetfillcolor{currentfill}%
\pgfsetlinewidth{0.602250pt}%
\definecolor{currentstroke}{rgb}{0.000000,0.000000,0.000000}%
\pgfsetstrokecolor{currentstroke}%
\pgfsetdash{}{0pt}%
\pgfsys@defobject{currentmarker}{\pgfqpoint{0.000000in}{0.000000in}}{\pgfqpoint{0.027778in}{0.000000in}}{%
\pgfpathmoveto{\pgfqpoint{0.000000in}{0.000000in}}%
\pgfpathlineto{\pgfqpoint{0.027778in}{0.000000in}}%
\pgfusepath{stroke,fill}%
}%
\begin{pgfscope}%
\pgfsys@transformshift{5.134696in}{4.452752in}%
\pgfsys@useobject{currentmarker}{}%
\end{pgfscope}%
\end{pgfscope}%
\begin{pgfscope}%
\pgfsetbuttcap%
\pgfsetroundjoin%
\definecolor{currentfill}{rgb}{0.000000,0.000000,0.000000}%
\pgfsetfillcolor{currentfill}%
\pgfsetlinewidth{0.602250pt}%
\definecolor{currentstroke}{rgb}{0.000000,0.000000,0.000000}%
\pgfsetstrokecolor{currentstroke}%
\pgfsetdash{}{0pt}%
\pgfsys@defobject{currentmarker}{\pgfqpoint{0.000000in}{0.000000in}}{\pgfqpoint{0.027778in}{0.000000in}}{%
\pgfpathmoveto{\pgfqpoint{0.000000in}{0.000000in}}%
\pgfpathlineto{\pgfqpoint{0.027778in}{0.000000in}}%
\pgfusepath{stroke,fill}%
}%
\begin{pgfscope}%
\pgfsys@transformshift{5.134696in}{4.627325in}%
\pgfsys@useobject{currentmarker}{}%
\end{pgfscope}%
\end{pgfscope}%
\begin{pgfscope}%
\pgfsetbuttcap%
\pgfsetroundjoin%
\definecolor{currentfill}{rgb}{0.000000,0.000000,0.000000}%
\pgfsetfillcolor{currentfill}%
\pgfsetlinewidth{0.602250pt}%
\definecolor{currentstroke}{rgb}{0.000000,0.000000,0.000000}%
\pgfsetstrokecolor{currentstroke}%
\pgfsetdash{}{0pt}%
\pgfsys@defobject{currentmarker}{\pgfqpoint{0.000000in}{0.000000in}}{\pgfqpoint{0.027778in}{0.000000in}}{%
\pgfpathmoveto{\pgfqpoint{0.000000in}{0.000000in}}%
\pgfpathlineto{\pgfqpoint{0.027778in}{0.000000in}}%
\pgfusepath{stroke,fill}%
}%
\begin{pgfscope}%
\pgfsys@transformshift{5.134696in}{4.762734in}%
\pgfsys@useobject{currentmarker}{}%
\end{pgfscope}%
\end{pgfscope}%
\begin{pgfscope}%
\pgfsetbuttcap%
\pgfsetroundjoin%
\definecolor{currentfill}{rgb}{0.000000,0.000000,0.000000}%
\pgfsetfillcolor{currentfill}%
\pgfsetlinewidth{0.602250pt}%
\definecolor{currentstroke}{rgb}{0.000000,0.000000,0.000000}%
\pgfsetstrokecolor{currentstroke}%
\pgfsetdash{}{0pt}%
\pgfsys@defobject{currentmarker}{\pgfqpoint{0.000000in}{0.000000in}}{\pgfqpoint{0.027778in}{0.000000in}}{%
\pgfpathmoveto{\pgfqpoint{0.000000in}{0.000000in}}%
\pgfpathlineto{\pgfqpoint{0.027778in}{0.000000in}}%
\pgfusepath{stroke,fill}%
}%
\begin{pgfscope}%
\pgfsys@transformshift{5.134696in}{4.873371in}%
\pgfsys@useobject{currentmarker}{}%
\end{pgfscope}%
\end{pgfscope}%
\begin{pgfscope}%
\pgfsetbuttcap%
\pgfsetroundjoin%
\definecolor{currentfill}{rgb}{0.000000,0.000000,0.000000}%
\pgfsetfillcolor{currentfill}%
\pgfsetlinewidth{0.602250pt}%
\definecolor{currentstroke}{rgb}{0.000000,0.000000,0.000000}%
\pgfsetstrokecolor{currentstroke}%
\pgfsetdash{}{0pt}%
\pgfsys@defobject{currentmarker}{\pgfqpoint{0.000000in}{0.000000in}}{\pgfqpoint{0.027778in}{0.000000in}}{%
\pgfpathmoveto{\pgfqpoint{0.000000in}{0.000000in}}%
\pgfpathlineto{\pgfqpoint{0.027778in}{0.000000in}}%
\pgfusepath{stroke,fill}%
}%
\begin{pgfscope}%
\pgfsys@transformshift{5.134696in}{4.966914in}%
\pgfsys@useobject{currentmarker}{}%
\end{pgfscope}%
\end{pgfscope}%
\begin{pgfscope}%
\pgfsetbuttcap%
\pgfsetroundjoin%
\definecolor{currentfill}{rgb}{0.000000,0.000000,0.000000}%
\pgfsetfillcolor{currentfill}%
\pgfsetlinewidth{0.602250pt}%
\definecolor{currentstroke}{rgb}{0.000000,0.000000,0.000000}%
\pgfsetstrokecolor{currentstroke}%
\pgfsetdash{}{0pt}%
\pgfsys@defobject{currentmarker}{\pgfqpoint{0.000000in}{0.000000in}}{\pgfqpoint{0.027778in}{0.000000in}}{%
\pgfpathmoveto{\pgfqpoint{0.000000in}{0.000000in}}%
\pgfpathlineto{\pgfqpoint{0.027778in}{0.000000in}}%
\pgfusepath{stroke,fill}%
}%
\begin{pgfscope}%
\pgfsys@transformshift{5.134696in}{5.047944in}%
\pgfsys@useobject{currentmarker}{}%
\end{pgfscope}%
\end{pgfscope}%
\begin{pgfscope}%
\pgfsetbuttcap%
\pgfsetroundjoin%
\definecolor{currentfill}{rgb}{0.000000,0.000000,0.000000}%
\pgfsetfillcolor{currentfill}%
\pgfsetlinewidth{0.602250pt}%
\definecolor{currentstroke}{rgb}{0.000000,0.000000,0.000000}%
\pgfsetstrokecolor{currentstroke}%
\pgfsetdash{}{0pt}%
\pgfsys@defobject{currentmarker}{\pgfqpoint{0.000000in}{0.000000in}}{\pgfqpoint{0.027778in}{0.000000in}}{%
\pgfpathmoveto{\pgfqpoint{0.000000in}{0.000000in}}%
\pgfpathlineto{\pgfqpoint{0.027778in}{0.000000in}}%
\pgfusepath{stroke,fill}%
}%
\begin{pgfscope}%
\pgfsys@transformshift{5.134696in}{5.119418in}%
\pgfsys@useobject{currentmarker}{}%
\end{pgfscope}%
\end{pgfscope}%
\begin{pgfscope}%
\pgfsetbuttcap%
\pgfsetroundjoin%
\definecolor{currentfill}{rgb}{0.000000,0.000000,0.000000}%
\pgfsetfillcolor{currentfill}%
\pgfsetlinewidth{0.602250pt}%
\definecolor{currentstroke}{rgb}{0.000000,0.000000,0.000000}%
\pgfsetstrokecolor{currentstroke}%
\pgfsetdash{}{0pt}%
\pgfsys@defobject{currentmarker}{\pgfqpoint{0.000000in}{0.000000in}}{\pgfqpoint{0.027778in}{0.000000in}}{%
\pgfpathmoveto{\pgfqpoint{0.000000in}{0.000000in}}%
\pgfpathlineto{\pgfqpoint{0.027778in}{0.000000in}}%
\pgfusepath{stroke,fill}%
}%
\begin{pgfscope}%
\pgfsys@transformshift{5.134696in}{5.603972in}%
\pgfsys@useobject{currentmarker}{}%
\end{pgfscope}%
\end{pgfscope}%
\begin{pgfscope}%
\pgfsetbuttcap%
\pgfsetroundjoin%
\definecolor{currentfill}{rgb}{0.000000,0.000000,0.000000}%
\pgfsetfillcolor{currentfill}%
\pgfsetlinewidth{0.602250pt}%
\definecolor{currentstroke}{rgb}{0.000000,0.000000,0.000000}%
\pgfsetstrokecolor{currentstroke}%
\pgfsetdash{}{0pt}%
\pgfsys@defobject{currentmarker}{\pgfqpoint{0.000000in}{0.000000in}}{\pgfqpoint{0.027778in}{0.000000in}}{%
\pgfpathmoveto{\pgfqpoint{0.000000in}{0.000000in}}%
\pgfpathlineto{\pgfqpoint{0.027778in}{0.000000in}}%
\pgfusepath{stroke,fill}%
}%
\begin{pgfscope}%
\pgfsys@transformshift{5.134696in}{5.850018in}%
\pgfsys@useobject{currentmarker}{}%
\end{pgfscope}%
\end{pgfscope}%
\begin{pgfscope}%
\pgfsetbuttcap%
\pgfsetroundjoin%
\definecolor{currentfill}{rgb}{0.000000,0.000000,0.000000}%
\pgfsetfillcolor{currentfill}%
\pgfsetlinewidth{0.602250pt}%
\definecolor{currentstroke}{rgb}{0.000000,0.000000,0.000000}%
\pgfsetstrokecolor{currentstroke}%
\pgfsetdash{}{0pt}%
\pgfsys@defobject{currentmarker}{\pgfqpoint{0.000000in}{0.000000in}}{\pgfqpoint{0.027778in}{0.000000in}}{%
\pgfpathmoveto{\pgfqpoint{0.000000in}{0.000000in}}%
\pgfpathlineto{\pgfqpoint{0.027778in}{0.000000in}}%
\pgfusepath{stroke,fill}%
}%
\begin{pgfscope}%
\pgfsys@transformshift{5.134696in}{6.024591in}%
\pgfsys@useobject{currentmarker}{}%
\end{pgfscope}%
\end{pgfscope}%
\begin{pgfscope}%
\pgfsetbuttcap%
\pgfsetroundjoin%
\definecolor{currentfill}{rgb}{0.000000,0.000000,0.000000}%
\pgfsetfillcolor{currentfill}%
\pgfsetlinewidth{0.602250pt}%
\definecolor{currentstroke}{rgb}{0.000000,0.000000,0.000000}%
\pgfsetstrokecolor{currentstroke}%
\pgfsetdash{}{0pt}%
\pgfsys@defobject{currentmarker}{\pgfqpoint{0.000000in}{0.000000in}}{\pgfqpoint{0.027778in}{0.000000in}}{%
\pgfpathmoveto{\pgfqpoint{0.000000in}{0.000000in}}%
\pgfpathlineto{\pgfqpoint{0.027778in}{0.000000in}}%
\pgfusepath{stroke,fill}%
}%
\begin{pgfscope}%
\pgfsys@transformshift{5.134696in}{6.160000in}%
\pgfsys@useobject{currentmarker}{}%
\end{pgfscope}%
\end{pgfscope}%
\begin{pgfscope}%
\definecolor{textcolor}{rgb}{0.000000,0.000000,0.000000}%
\pgfsetstrokecolor{textcolor}%
\pgfsetfillcolor{textcolor}%
\pgftext[x=5.488670in,y=4.973043in,,top,rotate=90.000000]{\color{textcolor}\sffamily\fontsize{10.000000}{12.000000}\selectfont \(\displaystyle dQ/dE \, \mathrm{[pC/MeV]}\)}%
\end{pgfscope}%
\begin{pgfscope}%
\pgfsetrectcap%
\pgfsetmiterjoin%
\pgfsetlinewidth{0.803000pt}%
\definecolor{currentstroke}{rgb}{0.000000,0.000000,0.000000}%
\pgfsetstrokecolor{currentstroke}%
\pgfsetdash{}{0pt}%
\pgfpathmoveto{\pgfqpoint{5.016000in}{3.786087in}}%
\pgfpathlineto{\pgfqpoint{5.075348in}{3.786087in}}%
\pgfpathlineto{\pgfqpoint{5.134696in}{3.786087in}}%
\pgfpathlineto{\pgfqpoint{5.134696in}{6.160000in}}%
\pgfpathlineto{\pgfqpoint{5.075348in}{6.160000in}}%
\pgfpathlineto{\pgfqpoint{5.016000in}{6.160000in}}%
\pgfpathlineto{\pgfqpoint{5.016000in}{3.786087in}}%
\pgfpathclose%
\pgfusepath{stroke}%
\end{pgfscope}%
\begin{pgfscope}%
\pgfsetbuttcap%
\pgfsetmiterjoin%
\definecolor{currentfill}{rgb}{1.000000,1.000000,1.000000}%
\pgfsetfillcolor{currentfill}%
\pgfsetlinewidth{0.000000pt}%
\definecolor{currentstroke}{rgb}{0.000000,0.000000,0.000000}%
\pgfsetstrokecolor{currentstroke}%
\pgfsetstrokeopacity{0.000000}%
\pgfsetdash{}{0pt}%
\pgfpathmoveto{\pgfqpoint{0.800000in}{0.700000in}}%
\pgfpathlineto{\pgfqpoint{5.760000in}{0.700000in}}%
\pgfpathlineto{\pgfqpoint{5.760000in}{3.073913in}}%
\pgfpathlineto{\pgfqpoint{0.800000in}{3.073913in}}%
\pgfpathlineto{\pgfqpoint{0.800000in}{0.700000in}}%
\pgfpathclose%
\pgfusepath{fill}%
\end{pgfscope}%
\begin{pgfscope}%
\pgfsetbuttcap%
\pgfsetroundjoin%
\definecolor{currentfill}{rgb}{0.000000,0.000000,0.000000}%
\pgfsetfillcolor{currentfill}%
\pgfsetlinewidth{0.803000pt}%
\definecolor{currentstroke}{rgb}{0.000000,0.000000,0.000000}%
\pgfsetstrokecolor{currentstroke}%
\pgfsetdash{}{0pt}%
\pgfsys@defobject{currentmarker}{\pgfqpoint{0.000000in}{-0.048611in}}{\pgfqpoint{0.000000in}{0.000000in}}{%
\pgfpathmoveto{\pgfqpoint{0.000000in}{0.000000in}}%
\pgfpathlineto{\pgfqpoint{0.000000in}{-0.048611in}}%
\pgfusepath{stroke,fill}%
}%
\begin{pgfscope}%
\pgfsys@transformshift{1.053158in}{0.700000in}%
\pgfsys@useobject{currentmarker}{}%
\end{pgfscope}%
\end{pgfscope}%
\begin{pgfscope}%
\definecolor{textcolor}{rgb}{0.000000,0.000000,0.000000}%
\pgfsetstrokecolor{textcolor}%
\pgfsetfillcolor{textcolor}%
\pgftext[x=1.053158in,y=0.602778in,,top]{\color{textcolor}\sffamily\fontsize{10.000000}{12.000000}\selectfont \(\displaystyle {0}\)}%
\end{pgfscope}%
\begin{pgfscope}%
\pgfsetbuttcap%
\pgfsetroundjoin%
\definecolor{currentfill}{rgb}{0.000000,0.000000,0.000000}%
\pgfsetfillcolor{currentfill}%
\pgfsetlinewidth{0.803000pt}%
\definecolor{currentstroke}{rgb}{0.000000,0.000000,0.000000}%
\pgfsetstrokecolor{currentstroke}%
\pgfsetdash{}{0pt}%
\pgfsys@defobject{currentmarker}{\pgfqpoint{0.000000in}{-0.048611in}}{\pgfqpoint{0.000000in}{0.000000in}}{%
\pgfpathmoveto{\pgfqpoint{0.000000in}{0.000000in}}%
\pgfpathlineto{\pgfqpoint{0.000000in}{-0.048611in}}%
\pgfusepath{stroke,fill}%
}%
\begin{pgfscope}%
\pgfsys@transformshift{1.780117in}{0.700000in}%
\pgfsys@useobject{currentmarker}{}%
\end{pgfscope}%
\end{pgfscope}%
\begin{pgfscope}%
\definecolor{textcolor}{rgb}{0.000000,0.000000,0.000000}%
\pgfsetstrokecolor{textcolor}%
\pgfsetfillcolor{textcolor}%
\pgftext[x=1.780117in,y=0.602778in,,top]{\color{textcolor}\sffamily\fontsize{10.000000}{12.000000}\selectfont \(\displaystyle {1}\)}%
\end{pgfscope}%
\begin{pgfscope}%
\pgfsetbuttcap%
\pgfsetroundjoin%
\definecolor{currentfill}{rgb}{0.000000,0.000000,0.000000}%
\pgfsetfillcolor{currentfill}%
\pgfsetlinewidth{0.803000pt}%
\definecolor{currentstroke}{rgb}{0.000000,0.000000,0.000000}%
\pgfsetstrokecolor{currentstroke}%
\pgfsetdash{}{0pt}%
\pgfsys@defobject{currentmarker}{\pgfqpoint{0.000000in}{-0.048611in}}{\pgfqpoint{0.000000in}{0.000000in}}{%
\pgfpathmoveto{\pgfqpoint{0.000000in}{0.000000in}}%
\pgfpathlineto{\pgfqpoint{0.000000in}{-0.048611in}}%
\pgfusepath{stroke,fill}%
}%
\begin{pgfscope}%
\pgfsys@transformshift{2.507077in}{0.700000in}%
\pgfsys@useobject{currentmarker}{}%
\end{pgfscope}%
\end{pgfscope}%
\begin{pgfscope}%
\definecolor{textcolor}{rgb}{0.000000,0.000000,0.000000}%
\pgfsetstrokecolor{textcolor}%
\pgfsetfillcolor{textcolor}%
\pgftext[x=2.507077in,y=0.602778in,,top]{\color{textcolor}\sffamily\fontsize{10.000000}{12.000000}\selectfont \(\displaystyle {2}\)}%
\end{pgfscope}%
\begin{pgfscope}%
\pgfsetbuttcap%
\pgfsetroundjoin%
\definecolor{currentfill}{rgb}{0.000000,0.000000,0.000000}%
\pgfsetfillcolor{currentfill}%
\pgfsetlinewidth{0.803000pt}%
\definecolor{currentstroke}{rgb}{0.000000,0.000000,0.000000}%
\pgfsetstrokecolor{currentstroke}%
\pgfsetdash{}{0pt}%
\pgfsys@defobject{currentmarker}{\pgfqpoint{0.000000in}{-0.048611in}}{\pgfqpoint{0.000000in}{0.000000in}}{%
\pgfpathmoveto{\pgfqpoint{0.000000in}{0.000000in}}%
\pgfpathlineto{\pgfqpoint{0.000000in}{-0.048611in}}%
\pgfusepath{stroke,fill}%
}%
\begin{pgfscope}%
\pgfsys@transformshift{3.234037in}{0.700000in}%
\pgfsys@useobject{currentmarker}{}%
\end{pgfscope}%
\end{pgfscope}%
\begin{pgfscope}%
\definecolor{textcolor}{rgb}{0.000000,0.000000,0.000000}%
\pgfsetstrokecolor{textcolor}%
\pgfsetfillcolor{textcolor}%
\pgftext[x=3.234037in,y=0.602778in,,top]{\color{textcolor}\sffamily\fontsize{10.000000}{12.000000}\selectfont \(\displaystyle {3}\)}%
\end{pgfscope}%
\begin{pgfscope}%
\pgfsetbuttcap%
\pgfsetroundjoin%
\definecolor{currentfill}{rgb}{0.000000,0.000000,0.000000}%
\pgfsetfillcolor{currentfill}%
\pgfsetlinewidth{0.803000pt}%
\definecolor{currentstroke}{rgb}{0.000000,0.000000,0.000000}%
\pgfsetstrokecolor{currentstroke}%
\pgfsetdash{}{0pt}%
\pgfsys@defobject{currentmarker}{\pgfqpoint{0.000000in}{-0.048611in}}{\pgfqpoint{0.000000in}{0.000000in}}{%
\pgfpathmoveto{\pgfqpoint{0.000000in}{0.000000in}}%
\pgfpathlineto{\pgfqpoint{0.000000in}{-0.048611in}}%
\pgfusepath{stroke,fill}%
}%
\begin{pgfscope}%
\pgfsys@transformshift{3.960996in}{0.700000in}%
\pgfsys@useobject{currentmarker}{}%
\end{pgfscope}%
\end{pgfscope}%
\begin{pgfscope}%
\definecolor{textcolor}{rgb}{0.000000,0.000000,0.000000}%
\pgfsetstrokecolor{textcolor}%
\pgfsetfillcolor{textcolor}%
\pgftext[x=3.960996in,y=0.602778in,,top]{\color{textcolor}\sffamily\fontsize{10.000000}{12.000000}\selectfont \(\displaystyle {4}\)}%
\end{pgfscope}%
\begin{pgfscope}%
\pgfsetbuttcap%
\pgfsetroundjoin%
\definecolor{currentfill}{rgb}{0.000000,0.000000,0.000000}%
\pgfsetfillcolor{currentfill}%
\pgfsetlinewidth{0.803000pt}%
\definecolor{currentstroke}{rgb}{0.000000,0.000000,0.000000}%
\pgfsetstrokecolor{currentstroke}%
\pgfsetdash{}{0pt}%
\pgfsys@defobject{currentmarker}{\pgfqpoint{0.000000in}{-0.048611in}}{\pgfqpoint{0.000000in}{0.000000in}}{%
\pgfpathmoveto{\pgfqpoint{0.000000in}{0.000000in}}%
\pgfpathlineto{\pgfqpoint{0.000000in}{-0.048611in}}%
\pgfusepath{stroke,fill}%
}%
\begin{pgfscope}%
\pgfsys@transformshift{4.687956in}{0.700000in}%
\pgfsys@useobject{currentmarker}{}%
\end{pgfscope}%
\end{pgfscope}%
\begin{pgfscope}%
\definecolor{textcolor}{rgb}{0.000000,0.000000,0.000000}%
\pgfsetstrokecolor{textcolor}%
\pgfsetfillcolor{textcolor}%
\pgftext[x=4.687956in,y=0.602778in,,top]{\color{textcolor}\sffamily\fontsize{10.000000}{12.000000}\selectfont \(\displaystyle {5}\)}%
\end{pgfscope}%
\begin{pgfscope}%
\pgfsetbuttcap%
\pgfsetroundjoin%
\definecolor{currentfill}{rgb}{0.000000,0.000000,0.000000}%
\pgfsetfillcolor{currentfill}%
\pgfsetlinewidth{0.803000pt}%
\definecolor{currentstroke}{rgb}{0.000000,0.000000,0.000000}%
\pgfsetstrokecolor{currentstroke}%
\pgfsetdash{}{0pt}%
\pgfsys@defobject{currentmarker}{\pgfqpoint{0.000000in}{-0.048611in}}{\pgfqpoint{0.000000in}{0.000000in}}{%
\pgfpathmoveto{\pgfqpoint{0.000000in}{0.000000in}}%
\pgfpathlineto{\pgfqpoint{0.000000in}{-0.048611in}}%
\pgfusepath{stroke,fill}%
}%
\begin{pgfscope}%
\pgfsys@transformshift{5.414916in}{0.700000in}%
\pgfsys@useobject{currentmarker}{}%
\end{pgfscope}%
\end{pgfscope}%
\begin{pgfscope}%
\definecolor{textcolor}{rgb}{0.000000,0.000000,0.000000}%
\pgfsetstrokecolor{textcolor}%
\pgfsetfillcolor{textcolor}%
\pgftext[x=5.414916in,y=0.602778in,,top]{\color{textcolor}\sffamily\fontsize{10.000000}{12.000000}\selectfont \(\displaystyle {6}\)}%
\end{pgfscope}%
\begin{pgfscope}%
\definecolor{textcolor}{rgb}{0.000000,0.000000,0.000000}%
\pgfsetstrokecolor{textcolor}%
\pgfsetfillcolor{textcolor}%
\pgftext[x=3.280000in,y=0.423766in,,top]{\color{textcolor}\sffamily\fontsize{10.000000}{12.000000}\selectfont \(\displaystyle y \, \mathrm{[mm]}\)}%
\end{pgfscope}%
\begin{pgfscope}%
\pgfsetbuttcap%
\pgfsetroundjoin%
\definecolor{currentfill}{rgb}{0.000000,0.000000,0.000000}%
\pgfsetfillcolor{currentfill}%
\pgfsetlinewidth{0.803000pt}%
\definecolor{currentstroke}{rgb}{0.000000,0.000000,0.000000}%
\pgfsetstrokecolor{currentstroke}%
\pgfsetdash{}{0pt}%
\pgfsys@defobject{currentmarker}{\pgfqpoint{-0.048611in}{0.000000in}}{\pgfqpoint{-0.000000in}{0.000000in}}{%
\pgfpathmoveto{\pgfqpoint{-0.000000in}{0.000000in}}%
\pgfpathlineto{\pgfqpoint{-0.048611in}{0.000000in}}%
\pgfusepath{stroke,fill}%
}%
\begin{pgfscope}%
\pgfsys@transformshift{0.800000in}{0.940139in}%
\pgfsys@useobject{currentmarker}{}%
\end{pgfscope}%
\end{pgfscope}%
\begin{pgfscope}%
\definecolor{textcolor}{rgb}{0.000000,0.000000,0.000000}%
\pgfsetstrokecolor{textcolor}%
\pgfsetfillcolor{textcolor}%
\pgftext[x=0.494444in, y=0.891913in, left, base]{\color{textcolor}\sffamily\fontsize{10.000000}{12.000000}\selectfont \(\displaystyle {242}\)}%
\end{pgfscope}%
\begin{pgfscope}%
\pgfsetbuttcap%
\pgfsetroundjoin%
\definecolor{currentfill}{rgb}{0.000000,0.000000,0.000000}%
\pgfsetfillcolor{currentfill}%
\pgfsetlinewidth{0.803000pt}%
\definecolor{currentstroke}{rgb}{0.000000,0.000000,0.000000}%
\pgfsetstrokecolor{currentstroke}%
\pgfsetdash{}{0pt}%
\pgfsys@defobject{currentmarker}{\pgfqpoint{-0.048611in}{0.000000in}}{\pgfqpoint{-0.000000in}{0.000000in}}{%
\pgfpathmoveto{\pgfqpoint{-0.000000in}{0.000000in}}%
\pgfpathlineto{\pgfqpoint{-0.048611in}{0.000000in}}%
\pgfusepath{stroke,fill}%
}%
\begin{pgfscope}%
\pgfsys@transformshift{0.800000in}{1.296514in}%
\pgfsys@useobject{currentmarker}{}%
\end{pgfscope}%
\end{pgfscope}%
\begin{pgfscope}%
\definecolor{textcolor}{rgb}{0.000000,0.000000,0.000000}%
\pgfsetstrokecolor{textcolor}%
\pgfsetfillcolor{textcolor}%
\pgftext[x=0.494444in, y=1.248289in, left, base]{\color{textcolor}\sffamily\fontsize{10.000000}{12.000000}\selectfont \(\displaystyle {244}\)}%
\end{pgfscope}%
\begin{pgfscope}%
\pgfsetbuttcap%
\pgfsetroundjoin%
\definecolor{currentfill}{rgb}{0.000000,0.000000,0.000000}%
\pgfsetfillcolor{currentfill}%
\pgfsetlinewidth{0.803000pt}%
\definecolor{currentstroke}{rgb}{0.000000,0.000000,0.000000}%
\pgfsetstrokecolor{currentstroke}%
\pgfsetdash{}{0pt}%
\pgfsys@defobject{currentmarker}{\pgfqpoint{-0.048611in}{0.000000in}}{\pgfqpoint{-0.000000in}{0.000000in}}{%
\pgfpathmoveto{\pgfqpoint{-0.000000in}{0.000000in}}%
\pgfpathlineto{\pgfqpoint{-0.048611in}{0.000000in}}%
\pgfusepath{stroke,fill}%
}%
\begin{pgfscope}%
\pgfsys@transformshift{0.800000in}{1.652889in}%
\pgfsys@useobject{currentmarker}{}%
\end{pgfscope}%
\end{pgfscope}%
\begin{pgfscope}%
\definecolor{textcolor}{rgb}{0.000000,0.000000,0.000000}%
\pgfsetstrokecolor{textcolor}%
\pgfsetfillcolor{textcolor}%
\pgftext[x=0.494444in, y=1.604664in, left, base]{\color{textcolor}\sffamily\fontsize{10.000000}{12.000000}\selectfont \(\displaystyle {246}\)}%
\end{pgfscope}%
\begin{pgfscope}%
\pgfsetbuttcap%
\pgfsetroundjoin%
\definecolor{currentfill}{rgb}{0.000000,0.000000,0.000000}%
\pgfsetfillcolor{currentfill}%
\pgfsetlinewidth{0.803000pt}%
\definecolor{currentstroke}{rgb}{0.000000,0.000000,0.000000}%
\pgfsetstrokecolor{currentstroke}%
\pgfsetdash{}{0pt}%
\pgfsys@defobject{currentmarker}{\pgfqpoint{-0.048611in}{0.000000in}}{\pgfqpoint{-0.000000in}{0.000000in}}{%
\pgfpathmoveto{\pgfqpoint{-0.000000in}{0.000000in}}%
\pgfpathlineto{\pgfqpoint{-0.048611in}{0.000000in}}%
\pgfusepath{stroke,fill}%
}%
\begin{pgfscope}%
\pgfsys@transformshift{0.800000in}{2.009265in}%
\pgfsys@useobject{currentmarker}{}%
\end{pgfscope}%
\end{pgfscope}%
\begin{pgfscope}%
\definecolor{textcolor}{rgb}{0.000000,0.000000,0.000000}%
\pgfsetstrokecolor{textcolor}%
\pgfsetfillcolor{textcolor}%
\pgftext[x=0.494444in, y=1.961039in, left, base]{\color{textcolor}\sffamily\fontsize{10.000000}{12.000000}\selectfont \(\displaystyle {248}\)}%
\end{pgfscope}%
\begin{pgfscope}%
\pgfsetbuttcap%
\pgfsetroundjoin%
\definecolor{currentfill}{rgb}{0.000000,0.000000,0.000000}%
\pgfsetfillcolor{currentfill}%
\pgfsetlinewidth{0.803000pt}%
\definecolor{currentstroke}{rgb}{0.000000,0.000000,0.000000}%
\pgfsetstrokecolor{currentstroke}%
\pgfsetdash{}{0pt}%
\pgfsys@defobject{currentmarker}{\pgfqpoint{-0.048611in}{0.000000in}}{\pgfqpoint{-0.000000in}{0.000000in}}{%
\pgfpathmoveto{\pgfqpoint{-0.000000in}{0.000000in}}%
\pgfpathlineto{\pgfqpoint{-0.048611in}{0.000000in}}%
\pgfusepath{stroke,fill}%
}%
\begin{pgfscope}%
\pgfsys@transformshift{0.800000in}{2.365640in}%
\pgfsys@useobject{currentmarker}{}%
\end{pgfscope}%
\end{pgfscope}%
\begin{pgfscope}%
\definecolor{textcolor}{rgb}{0.000000,0.000000,0.000000}%
\pgfsetstrokecolor{textcolor}%
\pgfsetfillcolor{textcolor}%
\pgftext[x=0.494444in, y=2.317415in, left, base]{\color{textcolor}\sffamily\fontsize{10.000000}{12.000000}\selectfont \(\displaystyle {250}\)}%
\end{pgfscope}%
\begin{pgfscope}%
\pgfsetbuttcap%
\pgfsetroundjoin%
\definecolor{currentfill}{rgb}{0.000000,0.000000,0.000000}%
\pgfsetfillcolor{currentfill}%
\pgfsetlinewidth{0.803000pt}%
\definecolor{currentstroke}{rgb}{0.000000,0.000000,0.000000}%
\pgfsetstrokecolor{currentstroke}%
\pgfsetdash{}{0pt}%
\pgfsys@defobject{currentmarker}{\pgfqpoint{-0.048611in}{0.000000in}}{\pgfqpoint{-0.000000in}{0.000000in}}{%
\pgfpathmoveto{\pgfqpoint{-0.000000in}{0.000000in}}%
\pgfpathlineto{\pgfqpoint{-0.048611in}{0.000000in}}%
\pgfusepath{stroke,fill}%
}%
\begin{pgfscope}%
\pgfsys@transformshift{0.800000in}{2.722015in}%
\pgfsys@useobject{currentmarker}{}%
\end{pgfscope}%
\end{pgfscope}%
\begin{pgfscope}%
\definecolor{textcolor}{rgb}{0.000000,0.000000,0.000000}%
\pgfsetstrokecolor{textcolor}%
\pgfsetfillcolor{textcolor}%
\pgftext[x=0.494444in, y=2.673790in, left, base]{\color{textcolor}\sffamily\fontsize{10.000000}{12.000000}\selectfont \(\displaystyle {252}\)}%
\end{pgfscope}%
\begin{pgfscope}%
\definecolor{textcolor}{rgb}{0.000000,0.000000,0.000000}%
\pgfsetstrokecolor{textcolor}%
\pgfsetfillcolor{textcolor}%
\pgftext[x=0.438888in,y=1.886957in,,bottom,rotate=90.000000]{\color{textcolor}\sffamily\fontsize{10.000000}{12.000000}\selectfont \(\displaystyle \mathrm{energy \, [MeV]}\)}%
\end{pgfscope}%
\begin{pgfscope}%
\pgfpathrectangle{\pgfqpoint{0.800000in}{0.700000in}}{\pgfqpoint{4.960000in}{2.373913in}}%
\pgfusepath{clip}%
\pgfsetbuttcap%
\pgfsetroundjoin%
\pgfsetlinewidth{1.505625pt}%
\definecolor{currentstroke}{rgb}{0.121569,0.466667,0.705882}%
\pgfsetstrokecolor{currentstroke}%
\pgfsetdash{}{0pt}%
\pgfpathmoveto{\pgfqpoint{1.025455in}{2.227901in}}%
\pgfpathlineto{\pgfqpoint{1.025455in}{2.251332in}}%
\pgfusepath{stroke}%
\end{pgfscope}%
\begin{pgfscope}%
\pgfpathrectangle{\pgfqpoint{0.800000in}{0.700000in}}{\pgfqpoint{4.960000in}{2.373913in}}%
\pgfusepath{clip}%
\pgfsetbuttcap%
\pgfsetroundjoin%
\pgfsetlinewidth{1.505625pt}%
\definecolor{currentstroke}{rgb}{0.121569,0.466667,0.705882}%
\pgfsetstrokecolor{currentstroke}%
\pgfsetdash{}{0pt}%
\pgfpathmoveto{\pgfqpoint{1.083263in}{2.055208in}}%
\pgfpathlineto{\pgfqpoint{1.083263in}{2.690468in}}%
\pgfusepath{stroke}%
\end{pgfscope}%
\begin{pgfscope}%
\pgfpathrectangle{\pgfqpoint{0.800000in}{0.700000in}}{\pgfqpoint{4.960000in}{2.373913in}}%
\pgfusepath{clip}%
\pgfsetbuttcap%
\pgfsetroundjoin%
\pgfsetlinewidth{1.505625pt}%
\definecolor{currentstroke}{rgb}{0.121569,0.466667,0.705882}%
\pgfsetstrokecolor{currentstroke}%
\pgfsetdash{}{0pt}%
\pgfpathmoveto{\pgfqpoint{1.141072in}{2.023343in}}%
\pgfpathlineto{\pgfqpoint{1.141072in}{2.715325in}}%
\pgfusepath{stroke}%
\end{pgfscope}%
\begin{pgfscope}%
\pgfpathrectangle{\pgfqpoint{0.800000in}{0.700000in}}{\pgfqpoint{4.960000in}{2.373913in}}%
\pgfusepath{clip}%
\pgfsetbuttcap%
\pgfsetroundjoin%
\pgfsetlinewidth{1.505625pt}%
\definecolor{currentstroke}{rgb}{0.121569,0.466667,0.705882}%
\pgfsetstrokecolor{currentstroke}%
\pgfsetdash{}{0pt}%
\pgfpathmoveto{\pgfqpoint{1.198881in}{1.878137in}}%
\pgfpathlineto{\pgfqpoint{1.198881in}{2.372552in}}%
\pgfusepath{stroke}%
\end{pgfscope}%
\begin{pgfscope}%
\pgfpathrectangle{\pgfqpoint{0.800000in}{0.700000in}}{\pgfqpoint{4.960000in}{2.373913in}}%
\pgfusepath{clip}%
\pgfsetbuttcap%
\pgfsetroundjoin%
\pgfsetlinewidth{1.505625pt}%
\definecolor{currentstroke}{rgb}{0.121569,0.466667,0.705882}%
\pgfsetstrokecolor{currentstroke}%
\pgfsetdash{}{0pt}%
\pgfpathmoveto{\pgfqpoint{1.256690in}{2.032784in}}%
\pgfpathlineto{\pgfqpoint{1.256690in}{2.095656in}}%
\pgfusepath{stroke}%
\end{pgfscope}%
\begin{pgfscope}%
\pgfpathrectangle{\pgfqpoint{0.800000in}{0.700000in}}{\pgfqpoint{4.960000in}{2.373913in}}%
\pgfusepath{clip}%
\pgfsetbuttcap%
\pgfsetroundjoin%
\pgfsetlinewidth{1.505625pt}%
\definecolor{currentstroke}{rgb}{0.121569,0.466667,0.705882}%
\pgfsetstrokecolor{currentstroke}%
\pgfsetdash{}{0pt}%
\pgfpathmoveto{\pgfqpoint{1.314499in}{1.740536in}}%
\pgfpathlineto{\pgfqpoint{1.314499in}{1.838640in}}%
\pgfusepath{stroke}%
\end{pgfscope}%
\begin{pgfscope}%
\pgfpathrectangle{\pgfqpoint{0.800000in}{0.700000in}}{\pgfqpoint{4.960000in}{2.373913in}}%
\pgfusepath{clip}%
\pgfsetbuttcap%
\pgfsetroundjoin%
\pgfsetlinewidth{1.505625pt}%
\definecolor{currentstroke}{rgb}{0.121569,0.466667,0.705882}%
\pgfsetstrokecolor{currentstroke}%
\pgfsetdash{}{0pt}%
\pgfpathmoveto{\pgfqpoint{1.372308in}{1.589512in}}%
\pgfpathlineto{\pgfqpoint{1.372308in}{1.635303in}}%
\pgfusepath{stroke}%
\end{pgfscope}%
\begin{pgfscope}%
\pgfpathrectangle{\pgfqpoint{0.800000in}{0.700000in}}{\pgfqpoint{4.960000in}{2.373913in}}%
\pgfusepath{clip}%
\pgfsetbuttcap%
\pgfsetroundjoin%
\pgfsetlinewidth{1.505625pt}%
\definecolor{currentstroke}{rgb}{0.121569,0.466667,0.705882}%
\pgfsetstrokecolor{currentstroke}%
\pgfsetdash{}{0pt}%
\pgfpathmoveto{\pgfqpoint{1.430117in}{1.433969in}}%
\pgfpathlineto{\pgfqpoint{1.430117in}{1.461599in}}%
\pgfusepath{stroke}%
\end{pgfscope}%
\begin{pgfscope}%
\pgfpathrectangle{\pgfqpoint{0.800000in}{0.700000in}}{\pgfqpoint{4.960000in}{2.373913in}}%
\pgfusepath{clip}%
\pgfsetbuttcap%
\pgfsetroundjoin%
\pgfsetlinewidth{1.505625pt}%
\definecolor{currentstroke}{rgb}{0.121569,0.466667,0.705882}%
\pgfsetstrokecolor{currentstroke}%
\pgfsetdash{}{0pt}%
\pgfpathmoveto{\pgfqpoint{1.487925in}{1.333865in}}%
\pgfpathlineto{\pgfqpoint{1.487925in}{1.356369in}}%
\pgfusepath{stroke}%
\end{pgfscope}%
\begin{pgfscope}%
\pgfpathrectangle{\pgfqpoint{0.800000in}{0.700000in}}{\pgfqpoint{4.960000in}{2.373913in}}%
\pgfusepath{clip}%
\pgfsetbuttcap%
\pgfsetroundjoin%
\pgfsetlinewidth{1.505625pt}%
\definecolor{currentstroke}{rgb}{0.121569,0.466667,0.705882}%
\pgfsetstrokecolor{currentstroke}%
\pgfsetdash{}{0pt}%
\pgfpathmoveto{\pgfqpoint{1.545734in}{1.281593in}}%
\pgfpathlineto{\pgfqpoint{1.545734in}{1.302297in}}%
\pgfusepath{stroke}%
\end{pgfscope}%
\begin{pgfscope}%
\pgfpathrectangle{\pgfqpoint{0.800000in}{0.700000in}}{\pgfqpoint{4.960000in}{2.373913in}}%
\pgfusepath{clip}%
\pgfsetbuttcap%
\pgfsetroundjoin%
\pgfsetlinewidth{1.505625pt}%
\definecolor{currentstroke}{rgb}{0.121569,0.466667,0.705882}%
\pgfsetstrokecolor{currentstroke}%
\pgfsetdash{}{0pt}%
\pgfpathmoveto{\pgfqpoint{1.603543in}{1.259121in}}%
\pgfpathlineto{\pgfqpoint{1.603543in}{1.278707in}}%
\pgfusepath{stroke}%
\end{pgfscope}%
\begin{pgfscope}%
\pgfpathrectangle{\pgfqpoint{0.800000in}{0.700000in}}{\pgfqpoint{4.960000in}{2.373913in}}%
\pgfusepath{clip}%
\pgfsetbuttcap%
\pgfsetroundjoin%
\pgfsetlinewidth{1.505625pt}%
\definecolor{currentstroke}{rgb}{0.121569,0.466667,0.705882}%
\pgfsetstrokecolor{currentstroke}%
\pgfsetdash{}{0pt}%
\pgfpathmoveto{\pgfqpoint{1.661352in}{1.237225in}}%
\pgfpathlineto{\pgfqpoint{1.661352in}{1.256080in}}%
\pgfusepath{stroke}%
\end{pgfscope}%
\begin{pgfscope}%
\pgfpathrectangle{\pgfqpoint{0.800000in}{0.700000in}}{\pgfqpoint{4.960000in}{2.373913in}}%
\pgfusepath{clip}%
\pgfsetbuttcap%
\pgfsetroundjoin%
\pgfsetlinewidth{1.505625pt}%
\definecolor{currentstroke}{rgb}{0.121569,0.466667,0.705882}%
\pgfsetstrokecolor{currentstroke}%
\pgfsetdash{}{0pt}%
\pgfpathmoveto{\pgfqpoint{1.719161in}{1.226532in}}%
\pgfpathlineto{\pgfqpoint{1.719161in}{1.246398in}}%
\pgfusepath{stroke}%
\end{pgfscope}%
\begin{pgfscope}%
\pgfpathrectangle{\pgfqpoint{0.800000in}{0.700000in}}{\pgfqpoint{4.960000in}{2.373913in}}%
\pgfusepath{clip}%
\pgfsetbuttcap%
\pgfsetroundjoin%
\pgfsetlinewidth{1.505625pt}%
\definecolor{currentstroke}{rgb}{0.121569,0.466667,0.705882}%
\pgfsetstrokecolor{currentstroke}%
\pgfsetdash{}{0pt}%
\pgfpathmoveto{\pgfqpoint{1.776970in}{1.220324in}}%
\pgfpathlineto{\pgfqpoint{1.776970in}{1.242253in}}%
\pgfusepath{stroke}%
\end{pgfscope}%
\begin{pgfscope}%
\pgfpathrectangle{\pgfqpoint{0.800000in}{0.700000in}}{\pgfqpoint{4.960000in}{2.373913in}}%
\pgfusepath{clip}%
\pgfsetbuttcap%
\pgfsetroundjoin%
\pgfsetlinewidth{1.505625pt}%
\definecolor{currentstroke}{rgb}{0.121569,0.466667,0.705882}%
\pgfsetstrokecolor{currentstroke}%
\pgfsetdash{}{0pt}%
\pgfpathmoveto{\pgfqpoint{1.834779in}{1.227000in}}%
\pgfpathlineto{\pgfqpoint{1.834779in}{1.249225in}}%
\pgfusepath{stroke}%
\end{pgfscope}%
\begin{pgfscope}%
\pgfpathrectangle{\pgfqpoint{0.800000in}{0.700000in}}{\pgfqpoint{4.960000in}{2.373913in}}%
\pgfusepath{clip}%
\pgfsetbuttcap%
\pgfsetroundjoin%
\pgfsetlinewidth{1.505625pt}%
\definecolor{currentstroke}{rgb}{0.121569,0.466667,0.705882}%
\pgfsetstrokecolor{currentstroke}%
\pgfsetdash{}{0pt}%
\pgfpathmoveto{\pgfqpoint{1.892587in}{1.236869in}}%
\pgfpathlineto{\pgfqpoint{1.892587in}{1.259592in}}%
\pgfusepath{stroke}%
\end{pgfscope}%
\begin{pgfscope}%
\pgfpathrectangle{\pgfqpoint{0.800000in}{0.700000in}}{\pgfqpoint{4.960000in}{2.373913in}}%
\pgfusepath{clip}%
\pgfsetbuttcap%
\pgfsetroundjoin%
\pgfsetlinewidth{1.505625pt}%
\definecolor{currentstroke}{rgb}{0.121569,0.466667,0.705882}%
\pgfsetstrokecolor{currentstroke}%
\pgfsetdash{}{0pt}%
\pgfpathmoveto{\pgfqpoint{1.950396in}{1.238509in}}%
\pgfpathlineto{\pgfqpoint{1.950396in}{1.261105in}}%
\pgfusepath{stroke}%
\end{pgfscope}%
\begin{pgfscope}%
\pgfpathrectangle{\pgfqpoint{0.800000in}{0.700000in}}{\pgfqpoint{4.960000in}{2.373913in}}%
\pgfusepath{clip}%
\pgfsetbuttcap%
\pgfsetroundjoin%
\pgfsetlinewidth{1.505625pt}%
\definecolor{currentstroke}{rgb}{0.121569,0.466667,0.705882}%
\pgfsetstrokecolor{currentstroke}%
\pgfsetdash{}{0pt}%
\pgfpathmoveto{\pgfqpoint{2.008205in}{1.245398in}}%
\pgfpathlineto{\pgfqpoint{2.008205in}{1.268165in}}%
\pgfusepath{stroke}%
\end{pgfscope}%
\begin{pgfscope}%
\pgfpathrectangle{\pgfqpoint{0.800000in}{0.700000in}}{\pgfqpoint{4.960000in}{2.373913in}}%
\pgfusepath{clip}%
\pgfsetbuttcap%
\pgfsetroundjoin%
\pgfsetlinewidth{1.505625pt}%
\definecolor{currentstroke}{rgb}{0.121569,0.466667,0.705882}%
\pgfsetstrokecolor{currentstroke}%
\pgfsetdash{}{0pt}%
\pgfpathmoveto{\pgfqpoint{2.066014in}{1.248469in}}%
\pgfpathlineto{\pgfqpoint{2.066014in}{1.272262in}}%
\pgfusepath{stroke}%
\end{pgfscope}%
\begin{pgfscope}%
\pgfpathrectangle{\pgfqpoint{0.800000in}{0.700000in}}{\pgfqpoint{4.960000in}{2.373913in}}%
\pgfusepath{clip}%
\pgfsetbuttcap%
\pgfsetroundjoin%
\pgfsetlinewidth{1.505625pt}%
\definecolor{currentstroke}{rgb}{0.121569,0.466667,0.705882}%
\pgfsetstrokecolor{currentstroke}%
\pgfsetdash{}{0pt}%
\pgfpathmoveto{\pgfqpoint{2.123823in}{1.243825in}}%
\pgfpathlineto{\pgfqpoint{2.123823in}{1.268235in}}%
\pgfusepath{stroke}%
\end{pgfscope}%
\begin{pgfscope}%
\pgfpathrectangle{\pgfqpoint{0.800000in}{0.700000in}}{\pgfqpoint{4.960000in}{2.373913in}}%
\pgfusepath{clip}%
\pgfsetbuttcap%
\pgfsetroundjoin%
\pgfsetlinewidth{1.505625pt}%
\definecolor{currentstroke}{rgb}{0.121569,0.466667,0.705882}%
\pgfsetstrokecolor{currentstroke}%
\pgfsetdash{}{0pt}%
\pgfpathmoveto{\pgfqpoint{2.181632in}{1.245335in}}%
\pgfpathlineto{\pgfqpoint{2.181632in}{1.270220in}}%
\pgfusepath{stroke}%
\end{pgfscope}%
\begin{pgfscope}%
\pgfpathrectangle{\pgfqpoint{0.800000in}{0.700000in}}{\pgfqpoint{4.960000in}{2.373913in}}%
\pgfusepath{clip}%
\pgfsetbuttcap%
\pgfsetroundjoin%
\pgfsetlinewidth{1.505625pt}%
\definecolor{currentstroke}{rgb}{0.121569,0.466667,0.705882}%
\pgfsetstrokecolor{currentstroke}%
\pgfsetdash{}{0pt}%
\pgfpathmoveto{\pgfqpoint{2.239441in}{1.242393in}}%
\pgfpathlineto{\pgfqpoint{2.239441in}{1.267428in}}%
\pgfusepath{stroke}%
\end{pgfscope}%
\begin{pgfscope}%
\pgfpathrectangle{\pgfqpoint{0.800000in}{0.700000in}}{\pgfqpoint{4.960000in}{2.373913in}}%
\pgfusepath{clip}%
\pgfsetbuttcap%
\pgfsetroundjoin%
\pgfsetlinewidth{1.505625pt}%
\definecolor{currentstroke}{rgb}{0.121569,0.466667,0.705882}%
\pgfsetstrokecolor{currentstroke}%
\pgfsetdash{}{0pt}%
\pgfpathmoveto{\pgfqpoint{2.297249in}{1.236465in}}%
\pgfpathlineto{\pgfqpoint{2.297249in}{1.262694in}}%
\pgfusepath{stroke}%
\end{pgfscope}%
\begin{pgfscope}%
\pgfpathrectangle{\pgfqpoint{0.800000in}{0.700000in}}{\pgfqpoint{4.960000in}{2.373913in}}%
\pgfusepath{clip}%
\pgfsetbuttcap%
\pgfsetroundjoin%
\pgfsetlinewidth{1.505625pt}%
\definecolor{currentstroke}{rgb}{0.121569,0.466667,0.705882}%
\pgfsetstrokecolor{currentstroke}%
\pgfsetdash{}{0pt}%
\pgfpathmoveto{\pgfqpoint{2.355058in}{1.235389in}}%
\pgfpathlineto{\pgfqpoint{2.355058in}{1.263125in}}%
\pgfusepath{stroke}%
\end{pgfscope}%
\begin{pgfscope}%
\pgfpathrectangle{\pgfqpoint{0.800000in}{0.700000in}}{\pgfqpoint{4.960000in}{2.373913in}}%
\pgfusepath{clip}%
\pgfsetbuttcap%
\pgfsetroundjoin%
\pgfsetlinewidth{1.505625pt}%
\definecolor{currentstroke}{rgb}{0.121569,0.466667,0.705882}%
\pgfsetstrokecolor{currentstroke}%
\pgfsetdash{}{0pt}%
\pgfpathmoveto{\pgfqpoint{2.412867in}{1.238231in}}%
\pgfpathlineto{\pgfqpoint{2.412867in}{1.265322in}}%
\pgfusepath{stroke}%
\end{pgfscope}%
\begin{pgfscope}%
\pgfpathrectangle{\pgfqpoint{0.800000in}{0.700000in}}{\pgfqpoint{4.960000in}{2.373913in}}%
\pgfusepath{clip}%
\pgfsetbuttcap%
\pgfsetroundjoin%
\pgfsetlinewidth{1.505625pt}%
\definecolor{currentstroke}{rgb}{0.121569,0.466667,0.705882}%
\pgfsetstrokecolor{currentstroke}%
\pgfsetdash{}{0pt}%
\pgfpathmoveto{\pgfqpoint{2.470676in}{1.239938in}}%
\pgfpathlineto{\pgfqpoint{2.470676in}{1.266676in}}%
\pgfusepath{stroke}%
\end{pgfscope}%
\begin{pgfscope}%
\pgfpathrectangle{\pgfqpoint{0.800000in}{0.700000in}}{\pgfqpoint{4.960000in}{2.373913in}}%
\pgfusepath{clip}%
\pgfsetbuttcap%
\pgfsetroundjoin%
\pgfsetlinewidth{1.505625pt}%
\definecolor{currentstroke}{rgb}{0.121569,0.466667,0.705882}%
\pgfsetstrokecolor{currentstroke}%
\pgfsetdash{}{0pt}%
\pgfpathmoveto{\pgfqpoint{2.528485in}{1.244213in}}%
\pgfpathlineto{\pgfqpoint{2.528485in}{1.271592in}}%
\pgfusepath{stroke}%
\end{pgfscope}%
\begin{pgfscope}%
\pgfpathrectangle{\pgfqpoint{0.800000in}{0.700000in}}{\pgfqpoint{4.960000in}{2.373913in}}%
\pgfusepath{clip}%
\pgfsetbuttcap%
\pgfsetroundjoin%
\pgfsetlinewidth{1.505625pt}%
\definecolor{currentstroke}{rgb}{0.121569,0.466667,0.705882}%
\pgfsetstrokecolor{currentstroke}%
\pgfsetdash{}{0pt}%
\pgfpathmoveto{\pgfqpoint{2.586294in}{1.249884in}}%
\pgfpathlineto{\pgfqpoint{2.586294in}{1.276988in}}%
\pgfusepath{stroke}%
\end{pgfscope}%
\begin{pgfscope}%
\pgfpathrectangle{\pgfqpoint{0.800000in}{0.700000in}}{\pgfqpoint{4.960000in}{2.373913in}}%
\pgfusepath{clip}%
\pgfsetbuttcap%
\pgfsetroundjoin%
\pgfsetlinewidth{1.505625pt}%
\definecolor{currentstroke}{rgb}{0.121569,0.466667,0.705882}%
\pgfsetstrokecolor{currentstroke}%
\pgfsetdash{}{0pt}%
\pgfpathmoveto{\pgfqpoint{2.644103in}{1.250531in}}%
\pgfpathlineto{\pgfqpoint{2.644103in}{1.278807in}}%
\pgfusepath{stroke}%
\end{pgfscope}%
\begin{pgfscope}%
\pgfpathrectangle{\pgfqpoint{0.800000in}{0.700000in}}{\pgfqpoint{4.960000in}{2.373913in}}%
\pgfusepath{clip}%
\pgfsetbuttcap%
\pgfsetroundjoin%
\pgfsetlinewidth{1.505625pt}%
\definecolor{currentstroke}{rgb}{0.121569,0.466667,0.705882}%
\pgfsetstrokecolor{currentstroke}%
\pgfsetdash{}{0pt}%
\pgfpathmoveto{\pgfqpoint{2.701911in}{1.247567in}}%
\pgfpathlineto{\pgfqpoint{2.701911in}{1.275636in}}%
\pgfusepath{stroke}%
\end{pgfscope}%
\begin{pgfscope}%
\pgfpathrectangle{\pgfqpoint{0.800000in}{0.700000in}}{\pgfqpoint{4.960000in}{2.373913in}}%
\pgfusepath{clip}%
\pgfsetbuttcap%
\pgfsetroundjoin%
\pgfsetlinewidth{1.505625pt}%
\definecolor{currentstroke}{rgb}{0.121569,0.466667,0.705882}%
\pgfsetstrokecolor{currentstroke}%
\pgfsetdash{}{0pt}%
\pgfpathmoveto{\pgfqpoint{2.759720in}{1.242646in}}%
\pgfpathlineto{\pgfqpoint{2.759720in}{1.271134in}}%
\pgfusepath{stroke}%
\end{pgfscope}%
\begin{pgfscope}%
\pgfpathrectangle{\pgfqpoint{0.800000in}{0.700000in}}{\pgfqpoint{4.960000in}{2.373913in}}%
\pgfusepath{clip}%
\pgfsetbuttcap%
\pgfsetroundjoin%
\pgfsetlinewidth{1.505625pt}%
\definecolor{currentstroke}{rgb}{0.121569,0.466667,0.705882}%
\pgfsetstrokecolor{currentstroke}%
\pgfsetdash{}{0pt}%
\pgfpathmoveto{\pgfqpoint{2.817529in}{1.237066in}}%
\pgfpathlineto{\pgfqpoint{2.817529in}{1.266024in}}%
\pgfusepath{stroke}%
\end{pgfscope}%
\begin{pgfscope}%
\pgfpathrectangle{\pgfqpoint{0.800000in}{0.700000in}}{\pgfqpoint{4.960000in}{2.373913in}}%
\pgfusepath{clip}%
\pgfsetbuttcap%
\pgfsetroundjoin%
\pgfsetlinewidth{1.505625pt}%
\definecolor{currentstroke}{rgb}{0.121569,0.466667,0.705882}%
\pgfsetstrokecolor{currentstroke}%
\pgfsetdash{}{0pt}%
\pgfpathmoveto{\pgfqpoint{2.875338in}{1.231153in}}%
\pgfpathlineto{\pgfqpoint{2.875338in}{1.260362in}}%
\pgfusepath{stroke}%
\end{pgfscope}%
\begin{pgfscope}%
\pgfpathrectangle{\pgfqpoint{0.800000in}{0.700000in}}{\pgfqpoint{4.960000in}{2.373913in}}%
\pgfusepath{clip}%
\pgfsetbuttcap%
\pgfsetroundjoin%
\pgfsetlinewidth{1.505625pt}%
\definecolor{currentstroke}{rgb}{0.121569,0.466667,0.705882}%
\pgfsetstrokecolor{currentstroke}%
\pgfsetdash{}{0pt}%
\pgfpathmoveto{\pgfqpoint{2.933147in}{1.225483in}}%
\pgfpathlineto{\pgfqpoint{2.933147in}{1.255073in}}%
\pgfusepath{stroke}%
\end{pgfscope}%
\begin{pgfscope}%
\pgfpathrectangle{\pgfqpoint{0.800000in}{0.700000in}}{\pgfqpoint{4.960000in}{2.373913in}}%
\pgfusepath{clip}%
\pgfsetbuttcap%
\pgfsetroundjoin%
\pgfsetlinewidth{1.505625pt}%
\definecolor{currentstroke}{rgb}{0.121569,0.466667,0.705882}%
\pgfsetstrokecolor{currentstroke}%
\pgfsetdash{}{0pt}%
\pgfpathmoveto{\pgfqpoint{2.990956in}{1.218645in}}%
\pgfpathlineto{\pgfqpoint{2.990956in}{1.247495in}}%
\pgfusepath{stroke}%
\end{pgfscope}%
\begin{pgfscope}%
\pgfpathrectangle{\pgfqpoint{0.800000in}{0.700000in}}{\pgfqpoint{4.960000in}{2.373913in}}%
\pgfusepath{clip}%
\pgfsetbuttcap%
\pgfsetroundjoin%
\pgfsetlinewidth{1.505625pt}%
\definecolor{currentstroke}{rgb}{0.121569,0.466667,0.705882}%
\pgfsetstrokecolor{currentstroke}%
\pgfsetdash{}{0pt}%
\pgfpathmoveto{\pgfqpoint{3.048765in}{1.212782in}}%
\pgfpathlineto{\pgfqpoint{3.048765in}{1.243015in}}%
\pgfusepath{stroke}%
\end{pgfscope}%
\begin{pgfscope}%
\pgfpathrectangle{\pgfqpoint{0.800000in}{0.700000in}}{\pgfqpoint{4.960000in}{2.373913in}}%
\pgfusepath{clip}%
\pgfsetbuttcap%
\pgfsetroundjoin%
\pgfsetlinewidth{1.505625pt}%
\definecolor{currentstroke}{rgb}{0.121569,0.466667,0.705882}%
\pgfsetstrokecolor{currentstroke}%
\pgfsetdash{}{0pt}%
\pgfpathmoveto{\pgfqpoint{3.106573in}{1.204971in}}%
\pgfpathlineto{\pgfqpoint{3.106573in}{1.234687in}}%
\pgfusepath{stroke}%
\end{pgfscope}%
\begin{pgfscope}%
\pgfpathrectangle{\pgfqpoint{0.800000in}{0.700000in}}{\pgfqpoint{4.960000in}{2.373913in}}%
\pgfusepath{clip}%
\pgfsetbuttcap%
\pgfsetroundjoin%
\pgfsetlinewidth{1.505625pt}%
\definecolor{currentstroke}{rgb}{0.121569,0.466667,0.705882}%
\pgfsetstrokecolor{currentstroke}%
\pgfsetdash{}{0pt}%
\pgfpathmoveto{\pgfqpoint{3.164382in}{1.194042in}}%
\pgfpathlineto{\pgfqpoint{3.164382in}{1.223997in}}%
\pgfusepath{stroke}%
\end{pgfscope}%
\begin{pgfscope}%
\pgfpathrectangle{\pgfqpoint{0.800000in}{0.700000in}}{\pgfqpoint{4.960000in}{2.373913in}}%
\pgfusepath{clip}%
\pgfsetbuttcap%
\pgfsetroundjoin%
\pgfsetlinewidth{1.505625pt}%
\definecolor{currentstroke}{rgb}{0.121569,0.466667,0.705882}%
\pgfsetstrokecolor{currentstroke}%
\pgfsetdash{}{0pt}%
\pgfpathmoveto{\pgfqpoint{3.222191in}{1.182878in}}%
\pgfpathlineto{\pgfqpoint{3.222191in}{1.212447in}}%
\pgfusepath{stroke}%
\end{pgfscope}%
\begin{pgfscope}%
\pgfpathrectangle{\pgfqpoint{0.800000in}{0.700000in}}{\pgfqpoint{4.960000in}{2.373913in}}%
\pgfusepath{clip}%
\pgfsetbuttcap%
\pgfsetroundjoin%
\pgfsetlinewidth{1.505625pt}%
\definecolor{currentstroke}{rgb}{0.121569,0.466667,0.705882}%
\pgfsetstrokecolor{currentstroke}%
\pgfsetdash{}{0pt}%
\pgfpathmoveto{\pgfqpoint{3.280000in}{1.171547in}}%
\pgfpathlineto{\pgfqpoint{3.280000in}{1.201662in}}%
\pgfusepath{stroke}%
\end{pgfscope}%
\begin{pgfscope}%
\pgfpathrectangle{\pgfqpoint{0.800000in}{0.700000in}}{\pgfqpoint{4.960000in}{2.373913in}}%
\pgfusepath{clip}%
\pgfsetbuttcap%
\pgfsetroundjoin%
\pgfsetlinewidth{1.505625pt}%
\definecolor{currentstroke}{rgb}{0.121569,0.466667,0.705882}%
\pgfsetstrokecolor{currentstroke}%
\pgfsetdash{}{0pt}%
\pgfpathmoveto{\pgfqpoint{3.337809in}{1.160624in}}%
\pgfpathlineto{\pgfqpoint{3.337809in}{1.191493in}}%
\pgfusepath{stroke}%
\end{pgfscope}%
\begin{pgfscope}%
\pgfpathrectangle{\pgfqpoint{0.800000in}{0.700000in}}{\pgfqpoint{4.960000in}{2.373913in}}%
\pgfusepath{clip}%
\pgfsetbuttcap%
\pgfsetroundjoin%
\pgfsetlinewidth{1.505625pt}%
\definecolor{currentstroke}{rgb}{0.121569,0.466667,0.705882}%
\pgfsetstrokecolor{currentstroke}%
\pgfsetdash{}{0pt}%
\pgfpathmoveto{\pgfqpoint{3.395618in}{1.154210in}}%
\pgfpathlineto{\pgfqpoint{3.395618in}{1.184395in}}%
\pgfusepath{stroke}%
\end{pgfscope}%
\begin{pgfscope}%
\pgfpathrectangle{\pgfqpoint{0.800000in}{0.700000in}}{\pgfqpoint{4.960000in}{2.373913in}}%
\pgfusepath{clip}%
\pgfsetbuttcap%
\pgfsetroundjoin%
\pgfsetlinewidth{1.505625pt}%
\definecolor{currentstroke}{rgb}{0.121569,0.466667,0.705882}%
\pgfsetstrokecolor{currentstroke}%
\pgfsetdash{}{0pt}%
\pgfpathmoveto{\pgfqpoint{3.453427in}{1.147943in}}%
\pgfpathlineto{\pgfqpoint{3.453427in}{1.178124in}}%
\pgfusepath{stroke}%
\end{pgfscope}%
\begin{pgfscope}%
\pgfpathrectangle{\pgfqpoint{0.800000in}{0.700000in}}{\pgfqpoint{4.960000in}{2.373913in}}%
\pgfusepath{clip}%
\pgfsetbuttcap%
\pgfsetroundjoin%
\pgfsetlinewidth{1.505625pt}%
\definecolor{currentstroke}{rgb}{0.121569,0.466667,0.705882}%
\pgfsetstrokecolor{currentstroke}%
\pgfsetdash{}{0pt}%
\pgfpathmoveto{\pgfqpoint{3.511235in}{1.143549in}}%
\pgfpathlineto{\pgfqpoint{3.511235in}{1.175109in}}%
\pgfusepath{stroke}%
\end{pgfscope}%
\begin{pgfscope}%
\pgfpathrectangle{\pgfqpoint{0.800000in}{0.700000in}}{\pgfqpoint{4.960000in}{2.373913in}}%
\pgfusepath{clip}%
\pgfsetbuttcap%
\pgfsetroundjoin%
\pgfsetlinewidth{1.505625pt}%
\definecolor{currentstroke}{rgb}{0.121569,0.466667,0.705882}%
\pgfsetstrokecolor{currentstroke}%
\pgfsetdash{}{0pt}%
\pgfpathmoveto{\pgfqpoint{3.569044in}{1.136819in}}%
\pgfpathlineto{\pgfqpoint{3.569044in}{1.169022in}}%
\pgfusepath{stroke}%
\end{pgfscope}%
\begin{pgfscope}%
\pgfpathrectangle{\pgfqpoint{0.800000in}{0.700000in}}{\pgfqpoint{4.960000in}{2.373913in}}%
\pgfusepath{clip}%
\pgfsetbuttcap%
\pgfsetroundjoin%
\pgfsetlinewidth{1.505625pt}%
\definecolor{currentstroke}{rgb}{0.121569,0.466667,0.705882}%
\pgfsetstrokecolor{currentstroke}%
\pgfsetdash{}{0pt}%
\pgfpathmoveto{\pgfqpoint{3.626853in}{1.129975in}}%
\pgfpathlineto{\pgfqpoint{3.626853in}{1.162823in}}%
\pgfusepath{stroke}%
\end{pgfscope}%
\begin{pgfscope}%
\pgfpathrectangle{\pgfqpoint{0.800000in}{0.700000in}}{\pgfqpoint{4.960000in}{2.373913in}}%
\pgfusepath{clip}%
\pgfsetbuttcap%
\pgfsetroundjoin%
\pgfsetlinewidth{1.505625pt}%
\definecolor{currentstroke}{rgb}{0.121569,0.466667,0.705882}%
\pgfsetstrokecolor{currentstroke}%
\pgfsetdash{}{0pt}%
\pgfpathmoveto{\pgfqpoint{3.684662in}{1.124471in}}%
\pgfpathlineto{\pgfqpoint{3.684662in}{1.157421in}}%
\pgfusepath{stroke}%
\end{pgfscope}%
\begin{pgfscope}%
\pgfpathrectangle{\pgfqpoint{0.800000in}{0.700000in}}{\pgfqpoint{4.960000in}{2.373913in}}%
\pgfusepath{clip}%
\pgfsetbuttcap%
\pgfsetroundjoin%
\pgfsetlinewidth{1.505625pt}%
\definecolor{currentstroke}{rgb}{0.121569,0.466667,0.705882}%
\pgfsetstrokecolor{currentstroke}%
\pgfsetdash{}{0pt}%
\pgfpathmoveto{\pgfqpoint{3.742471in}{1.120107in}}%
\pgfpathlineto{\pgfqpoint{3.742471in}{1.151882in}}%
\pgfusepath{stroke}%
\end{pgfscope}%
\begin{pgfscope}%
\pgfpathrectangle{\pgfqpoint{0.800000in}{0.700000in}}{\pgfqpoint{4.960000in}{2.373913in}}%
\pgfusepath{clip}%
\pgfsetbuttcap%
\pgfsetroundjoin%
\pgfsetlinewidth{1.505625pt}%
\definecolor{currentstroke}{rgb}{0.121569,0.466667,0.705882}%
\pgfsetstrokecolor{currentstroke}%
\pgfsetdash{}{0pt}%
\pgfpathmoveto{\pgfqpoint{3.800280in}{1.111883in}}%
\pgfpathlineto{\pgfqpoint{3.800280in}{1.145145in}}%
\pgfusepath{stroke}%
\end{pgfscope}%
\begin{pgfscope}%
\pgfpathrectangle{\pgfqpoint{0.800000in}{0.700000in}}{\pgfqpoint{4.960000in}{2.373913in}}%
\pgfusepath{clip}%
\pgfsetbuttcap%
\pgfsetroundjoin%
\pgfsetlinewidth{1.505625pt}%
\definecolor{currentstroke}{rgb}{0.121569,0.466667,0.705882}%
\pgfsetstrokecolor{currentstroke}%
\pgfsetdash{}{0pt}%
\pgfpathmoveto{\pgfqpoint{3.858089in}{1.101281in}}%
\pgfpathlineto{\pgfqpoint{3.858089in}{1.134015in}}%
\pgfusepath{stroke}%
\end{pgfscope}%
\begin{pgfscope}%
\pgfpathrectangle{\pgfqpoint{0.800000in}{0.700000in}}{\pgfqpoint{4.960000in}{2.373913in}}%
\pgfusepath{clip}%
\pgfsetbuttcap%
\pgfsetroundjoin%
\pgfsetlinewidth{1.505625pt}%
\definecolor{currentstroke}{rgb}{0.121569,0.466667,0.705882}%
\pgfsetstrokecolor{currentstroke}%
\pgfsetdash{}{0pt}%
\pgfpathmoveto{\pgfqpoint{3.915897in}{1.092246in}}%
\pgfpathlineto{\pgfqpoint{3.915897in}{1.126582in}}%
\pgfusepath{stroke}%
\end{pgfscope}%
\begin{pgfscope}%
\pgfpathrectangle{\pgfqpoint{0.800000in}{0.700000in}}{\pgfqpoint{4.960000in}{2.373913in}}%
\pgfusepath{clip}%
\pgfsetbuttcap%
\pgfsetroundjoin%
\pgfsetlinewidth{1.505625pt}%
\definecolor{currentstroke}{rgb}{0.121569,0.466667,0.705882}%
\pgfsetstrokecolor{currentstroke}%
\pgfsetdash{}{0pt}%
\pgfpathmoveto{\pgfqpoint{3.973706in}{1.085600in}}%
\pgfpathlineto{\pgfqpoint{3.973706in}{1.118501in}}%
\pgfusepath{stroke}%
\end{pgfscope}%
\begin{pgfscope}%
\pgfpathrectangle{\pgfqpoint{0.800000in}{0.700000in}}{\pgfqpoint{4.960000in}{2.373913in}}%
\pgfusepath{clip}%
\pgfsetbuttcap%
\pgfsetroundjoin%
\pgfsetlinewidth{1.505625pt}%
\definecolor{currentstroke}{rgb}{0.121569,0.466667,0.705882}%
\pgfsetstrokecolor{currentstroke}%
\pgfsetdash{}{0pt}%
\pgfpathmoveto{\pgfqpoint{4.031515in}{1.075822in}}%
\pgfpathlineto{\pgfqpoint{4.031515in}{1.109769in}}%
\pgfusepath{stroke}%
\end{pgfscope}%
\begin{pgfscope}%
\pgfpathrectangle{\pgfqpoint{0.800000in}{0.700000in}}{\pgfqpoint{4.960000in}{2.373913in}}%
\pgfusepath{clip}%
\pgfsetbuttcap%
\pgfsetroundjoin%
\pgfsetlinewidth{1.505625pt}%
\definecolor{currentstroke}{rgb}{0.121569,0.466667,0.705882}%
\pgfsetstrokecolor{currentstroke}%
\pgfsetdash{}{0pt}%
\pgfpathmoveto{\pgfqpoint{4.089324in}{1.068237in}}%
\pgfpathlineto{\pgfqpoint{4.089324in}{1.102740in}}%
\pgfusepath{stroke}%
\end{pgfscope}%
\begin{pgfscope}%
\pgfpathrectangle{\pgfqpoint{0.800000in}{0.700000in}}{\pgfqpoint{4.960000in}{2.373913in}}%
\pgfusepath{clip}%
\pgfsetbuttcap%
\pgfsetroundjoin%
\pgfsetlinewidth{1.505625pt}%
\definecolor{currentstroke}{rgb}{0.121569,0.466667,0.705882}%
\pgfsetstrokecolor{currentstroke}%
\pgfsetdash{}{0pt}%
\pgfpathmoveto{\pgfqpoint{4.147133in}{1.060173in}}%
\pgfpathlineto{\pgfqpoint{4.147133in}{1.095793in}}%
\pgfusepath{stroke}%
\end{pgfscope}%
\begin{pgfscope}%
\pgfpathrectangle{\pgfqpoint{0.800000in}{0.700000in}}{\pgfqpoint{4.960000in}{2.373913in}}%
\pgfusepath{clip}%
\pgfsetbuttcap%
\pgfsetroundjoin%
\pgfsetlinewidth{1.505625pt}%
\definecolor{currentstroke}{rgb}{0.121569,0.466667,0.705882}%
\pgfsetstrokecolor{currentstroke}%
\pgfsetdash{}{0pt}%
\pgfpathmoveto{\pgfqpoint{4.204942in}{1.050080in}}%
\pgfpathlineto{\pgfqpoint{4.204942in}{1.085519in}}%
\pgfusepath{stroke}%
\end{pgfscope}%
\begin{pgfscope}%
\pgfpathrectangle{\pgfqpoint{0.800000in}{0.700000in}}{\pgfqpoint{4.960000in}{2.373913in}}%
\pgfusepath{clip}%
\pgfsetbuttcap%
\pgfsetroundjoin%
\pgfsetlinewidth{1.505625pt}%
\definecolor{currentstroke}{rgb}{0.121569,0.466667,0.705882}%
\pgfsetstrokecolor{currentstroke}%
\pgfsetdash{}{0pt}%
\pgfpathmoveto{\pgfqpoint{4.262751in}{1.037182in}}%
\pgfpathlineto{\pgfqpoint{4.262751in}{1.073941in}}%
\pgfusepath{stroke}%
\end{pgfscope}%
\begin{pgfscope}%
\pgfpathrectangle{\pgfqpoint{0.800000in}{0.700000in}}{\pgfqpoint{4.960000in}{2.373913in}}%
\pgfusepath{clip}%
\pgfsetbuttcap%
\pgfsetroundjoin%
\pgfsetlinewidth{1.505625pt}%
\definecolor{currentstroke}{rgb}{0.121569,0.466667,0.705882}%
\pgfsetstrokecolor{currentstroke}%
\pgfsetdash{}{0pt}%
\pgfpathmoveto{\pgfqpoint{4.320559in}{1.023418in}}%
\pgfpathlineto{\pgfqpoint{4.320559in}{1.061012in}}%
\pgfusepath{stroke}%
\end{pgfscope}%
\begin{pgfscope}%
\pgfpathrectangle{\pgfqpoint{0.800000in}{0.700000in}}{\pgfqpoint{4.960000in}{2.373913in}}%
\pgfusepath{clip}%
\pgfsetbuttcap%
\pgfsetroundjoin%
\pgfsetlinewidth{1.505625pt}%
\definecolor{currentstroke}{rgb}{0.121569,0.466667,0.705882}%
\pgfsetstrokecolor{currentstroke}%
\pgfsetdash{}{0pt}%
\pgfpathmoveto{\pgfqpoint{4.378368in}{1.010999in}}%
\pgfpathlineto{\pgfqpoint{4.378368in}{1.048805in}}%
\pgfusepath{stroke}%
\end{pgfscope}%
\begin{pgfscope}%
\pgfpathrectangle{\pgfqpoint{0.800000in}{0.700000in}}{\pgfqpoint{4.960000in}{2.373913in}}%
\pgfusepath{clip}%
\pgfsetbuttcap%
\pgfsetroundjoin%
\pgfsetlinewidth{1.505625pt}%
\definecolor{currentstroke}{rgb}{0.121569,0.466667,0.705882}%
\pgfsetstrokecolor{currentstroke}%
\pgfsetdash{}{0pt}%
\pgfpathmoveto{\pgfqpoint{4.436177in}{1.001699in}}%
\pgfpathlineto{\pgfqpoint{4.436177in}{1.039811in}}%
\pgfusepath{stroke}%
\end{pgfscope}%
\begin{pgfscope}%
\pgfpathrectangle{\pgfqpoint{0.800000in}{0.700000in}}{\pgfqpoint{4.960000in}{2.373913in}}%
\pgfusepath{clip}%
\pgfsetbuttcap%
\pgfsetroundjoin%
\pgfsetlinewidth{1.505625pt}%
\definecolor{currentstroke}{rgb}{0.121569,0.466667,0.705882}%
\pgfsetstrokecolor{currentstroke}%
\pgfsetdash{}{0pt}%
\pgfpathmoveto{\pgfqpoint{4.493986in}{0.992601in}}%
\pgfpathlineto{\pgfqpoint{4.493986in}{1.030645in}}%
\pgfusepath{stroke}%
\end{pgfscope}%
\begin{pgfscope}%
\pgfpathrectangle{\pgfqpoint{0.800000in}{0.700000in}}{\pgfqpoint{4.960000in}{2.373913in}}%
\pgfusepath{clip}%
\pgfsetbuttcap%
\pgfsetroundjoin%
\pgfsetlinewidth{1.505625pt}%
\definecolor{currentstroke}{rgb}{0.121569,0.466667,0.705882}%
\pgfsetstrokecolor{currentstroke}%
\pgfsetdash{}{0pt}%
\pgfpathmoveto{\pgfqpoint{4.551795in}{0.981242in}}%
\pgfpathlineto{\pgfqpoint{4.551795in}{1.019198in}}%
\pgfusepath{stroke}%
\end{pgfscope}%
\begin{pgfscope}%
\pgfpathrectangle{\pgfqpoint{0.800000in}{0.700000in}}{\pgfqpoint{4.960000in}{2.373913in}}%
\pgfusepath{clip}%
\pgfsetbuttcap%
\pgfsetroundjoin%
\pgfsetlinewidth{1.505625pt}%
\definecolor{currentstroke}{rgb}{0.121569,0.466667,0.705882}%
\pgfsetstrokecolor{currentstroke}%
\pgfsetdash{}{0pt}%
\pgfpathmoveto{\pgfqpoint{4.609604in}{0.971203in}}%
\pgfpathlineto{\pgfqpoint{4.609604in}{1.010921in}}%
\pgfusepath{stroke}%
\end{pgfscope}%
\begin{pgfscope}%
\pgfpathrectangle{\pgfqpoint{0.800000in}{0.700000in}}{\pgfqpoint{4.960000in}{2.373913in}}%
\pgfusepath{clip}%
\pgfsetbuttcap%
\pgfsetroundjoin%
\pgfsetlinewidth{1.505625pt}%
\definecolor{currentstroke}{rgb}{0.121569,0.466667,0.705882}%
\pgfsetstrokecolor{currentstroke}%
\pgfsetdash{}{0pt}%
\pgfpathmoveto{\pgfqpoint{4.667413in}{0.963048in}}%
\pgfpathlineto{\pgfqpoint{4.667413in}{1.001607in}}%
\pgfusepath{stroke}%
\end{pgfscope}%
\begin{pgfscope}%
\pgfpathrectangle{\pgfqpoint{0.800000in}{0.700000in}}{\pgfqpoint{4.960000in}{2.373913in}}%
\pgfusepath{clip}%
\pgfsetbuttcap%
\pgfsetroundjoin%
\pgfsetlinewidth{1.505625pt}%
\definecolor{currentstroke}{rgb}{0.121569,0.466667,0.705882}%
\pgfsetstrokecolor{currentstroke}%
\pgfsetdash{}{0pt}%
\pgfpathmoveto{\pgfqpoint{4.725221in}{0.953296in}}%
\pgfpathlineto{\pgfqpoint{4.725221in}{0.991495in}}%
\pgfusepath{stroke}%
\end{pgfscope}%
\begin{pgfscope}%
\pgfpathrectangle{\pgfqpoint{0.800000in}{0.700000in}}{\pgfqpoint{4.960000in}{2.373913in}}%
\pgfusepath{clip}%
\pgfsetbuttcap%
\pgfsetroundjoin%
\pgfsetlinewidth{1.505625pt}%
\definecolor{currentstroke}{rgb}{0.121569,0.466667,0.705882}%
\pgfsetstrokecolor{currentstroke}%
\pgfsetdash{}{0pt}%
\pgfpathmoveto{\pgfqpoint{4.783030in}{0.943009in}}%
\pgfpathlineto{\pgfqpoint{4.783030in}{0.981979in}}%
\pgfusepath{stroke}%
\end{pgfscope}%
\begin{pgfscope}%
\pgfpathrectangle{\pgfqpoint{0.800000in}{0.700000in}}{\pgfqpoint{4.960000in}{2.373913in}}%
\pgfusepath{clip}%
\pgfsetbuttcap%
\pgfsetroundjoin%
\pgfsetlinewidth{1.505625pt}%
\definecolor{currentstroke}{rgb}{0.121569,0.466667,0.705882}%
\pgfsetstrokecolor{currentstroke}%
\pgfsetdash{}{0pt}%
\pgfpathmoveto{\pgfqpoint{4.840839in}{0.933540in}}%
\pgfpathlineto{\pgfqpoint{4.840839in}{0.972762in}}%
\pgfusepath{stroke}%
\end{pgfscope}%
\begin{pgfscope}%
\pgfpathrectangle{\pgfqpoint{0.800000in}{0.700000in}}{\pgfqpoint{4.960000in}{2.373913in}}%
\pgfusepath{clip}%
\pgfsetbuttcap%
\pgfsetroundjoin%
\pgfsetlinewidth{1.505625pt}%
\definecolor{currentstroke}{rgb}{0.121569,0.466667,0.705882}%
\pgfsetstrokecolor{currentstroke}%
\pgfsetdash{}{0pt}%
\pgfpathmoveto{\pgfqpoint{4.898648in}{0.927111in}}%
\pgfpathlineto{\pgfqpoint{4.898648in}{0.968773in}}%
\pgfusepath{stroke}%
\end{pgfscope}%
\begin{pgfscope}%
\pgfpathrectangle{\pgfqpoint{0.800000in}{0.700000in}}{\pgfqpoint{4.960000in}{2.373913in}}%
\pgfusepath{clip}%
\pgfsetbuttcap%
\pgfsetroundjoin%
\pgfsetlinewidth{1.505625pt}%
\definecolor{currentstroke}{rgb}{0.121569,0.466667,0.705882}%
\pgfsetstrokecolor{currentstroke}%
\pgfsetdash{}{0pt}%
\pgfpathmoveto{\pgfqpoint{4.956457in}{0.916255in}}%
\pgfpathlineto{\pgfqpoint{4.956457in}{0.956645in}}%
\pgfusepath{stroke}%
\end{pgfscope}%
\begin{pgfscope}%
\pgfpathrectangle{\pgfqpoint{0.800000in}{0.700000in}}{\pgfqpoint{4.960000in}{2.373913in}}%
\pgfusepath{clip}%
\pgfsetbuttcap%
\pgfsetroundjoin%
\pgfsetlinewidth{1.505625pt}%
\definecolor{currentstroke}{rgb}{0.121569,0.466667,0.705882}%
\pgfsetstrokecolor{currentstroke}%
\pgfsetdash{}{0pt}%
\pgfpathmoveto{\pgfqpoint{5.014266in}{0.906534in}}%
\pgfpathlineto{\pgfqpoint{5.014266in}{0.947324in}}%
\pgfusepath{stroke}%
\end{pgfscope}%
\begin{pgfscope}%
\pgfpathrectangle{\pgfqpoint{0.800000in}{0.700000in}}{\pgfqpoint{4.960000in}{2.373913in}}%
\pgfusepath{clip}%
\pgfsetbuttcap%
\pgfsetroundjoin%
\pgfsetlinewidth{1.505625pt}%
\definecolor{currentstroke}{rgb}{0.121569,0.466667,0.705882}%
\pgfsetstrokecolor{currentstroke}%
\pgfsetdash{}{0pt}%
\pgfpathmoveto{\pgfqpoint{5.072075in}{0.894831in}}%
\pgfpathlineto{\pgfqpoint{5.072075in}{0.936751in}}%
\pgfusepath{stroke}%
\end{pgfscope}%
\begin{pgfscope}%
\pgfpathrectangle{\pgfqpoint{0.800000in}{0.700000in}}{\pgfqpoint{4.960000in}{2.373913in}}%
\pgfusepath{clip}%
\pgfsetbuttcap%
\pgfsetroundjoin%
\pgfsetlinewidth{1.505625pt}%
\definecolor{currentstroke}{rgb}{0.121569,0.466667,0.705882}%
\pgfsetstrokecolor{currentstroke}%
\pgfsetdash{}{0pt}%
\pgfpathmoveto{\pgfqpoint{5.129883in}{0.885072in}}%
\pgfpathlineto{\pgfqpoint{5.129883in}{0.928305in}}%
\pgfusepath{stroke}%
\end{pgfscope}%
\begin{pgfscope}%
\pgfpathrectangle{\pgfqpoint{0.800000in}{0.700000in}}{\pgfqpoint{4.960000in}{2.373913in}}%
\pgfusepath{clip}%
\pgfsetbuttcap%
\pgfsetroundjoin%
\pgfsetlinewidth{1.505625pt}%
\definecolor{currentstroke}{rgb}{0.121569,0.466667,0.705882}%
\pgfsetstrokecolor{currentstroke}%
\pgfsetdash{}{0pt}%
\pgfpathmoveto{\pgfqpoint{5.187692in}{0.876267in}}%
\pgfpathlineto{\pgfqpoint{5.187692in}{0.918820in}}%
\pgfusepath{stroke}%
\end{pgfscope}%
\begin{pgfscope}%
\pgfpathrectangle{\pgfqpoint{0.800000in}{0.700000in}}{\pgfqpoint{4.960000in}{2.373913in}}%
\pgfusepath{clip}%
\pgfsetbuttcap%
\pgfsetroundjoin%
\pgfsetlinewidth{1.505625pt}%
\definecolor{currentstroke}{rgb}{0.121569,0.466667,0.705882}%
\pgfsetstrokecolor{currentstroke}%
\pgfsetdash{}{0pt}%
\pgfpathmoveto{\pgfqpoint{5.245501in}{0.859655in}}%
\pgfpathlineto{\pgfqpoint{5.245501in}{0.900520in}}%
\pgfusepath{stroke}%
\end{pgfscope}%
\begin{pgfscope}%
\pgfpathrectangle{\pgfqpoint{0.800000in}{0.700000in}}{\pgfqpoint{4.960000in}{2.373913in}}%
\pgfusepath{clip}%
\pgfsetbuttcap%
\pgfsetroundjoin%
\pgfsetlinewidth{1.505625pt}%
\definecolor{currentstroke}{rgb}{0.121569,0.466667,0.705882}%
\pgfsetstrokecolor{currentstroke}%
\pgfsetdash{}{0pt}%
\pgfpathmoveto{\pgfqpoint{5.303310in}{0.849048in}}%
\pgfpathlineto{\pgfqpoint{5.303310in}{0.891453in}}%
\pgfusepath{stroke}%
\end{pgfscope}%
\begin{pgfscope}%
\pgfpathrectangle{\pgfqpoint{0.800000in}{0.700000in}}{\pgfqpoint{4.960000in}{2.373913in}}%
\pgfusepath{clip}%
\pgfsetbuttcap%
\pgfsetroundjoin%
\pgfsetlinewidth{1.505625pt}%
\definecolor{currentstroke}{rgb}{0.121569,0.466667,0.705882}%
\pgfsetstrokecolor{currentstroke}%
\pgfsetdash{}{0pt}%
\pgfpathmoveto{\pgfqpoint{5.361119in}{0.841164in}}%
\pgfpathlineto{\pgfqpoint{5.361119in}{0.882446in}}%
\pgfusepath{stroke}%
\end{pgfscope}%
\begin{pgfscope}%
\pgfpathrectangle{\pgfqpoint{0.800000in}{0.700000in}}{\pgfqpoint{4.960000in}{2.373913in}}%
\pgfusepath{clip}%
\pgfsetbuttcap%
\pgfsetroundjoin%
\pgfsetlinewidth{1.505625pt}%
\definecolor{currentstroke}{rgb}{0.121569,0.466667,0.705882}%
\pgfsetstrokecolor{currentstroke}%
\pgfsetdash{}{0pt}%
\pgfpathmoveto{\pgfqpoint{5.418928in}{0.831615in}}%
\pgfpathlineto{\pgfqpoint{5.418928in}{0.873515in}}%
\pgfusepath{stroke}%
\end{pgfscope}%
\begin{pgfscope}%
\pgfpathrectangle{\pgfqpoint{0.800000in}{0.700000in}}{\pgfqpoint{4.960000in}{2.373913in}}%
\pgfusepath{clip}%
\pgfsetbuttcap%
\pgfsetroundjoin%
\pgfsetlinewidth{1.505625pt}%
\definecolor{currentstroke}{rgb}{0.121569,0.466667,0.705882}%
\pgfsetstrokecolor{currentstroke}%
\pgfsetdash{}{0pt}%
\pgfpathmoveto{\pgfqpoint{5.476737in}{0.820735in}}%
\pgfpathlineto{\pgfqpoint{5.476737in}{0.864195in}}%
\pgfusepath{stroke}%
\end{pgfscope}%
\begin{pgfscope}%
\pgfpathrectangle{\pgfqpoint{0.800000in}{0.700000in}}{\pgfqpoint{4.960000in}{2.373913in}}%
\pgfusepath{clip}%
\pgfsetbuttcap%
\pgfsetroundjoin%
\pgfsetlinewidth{1.505625pt}%
\definecolor{currentstroke}{rgb}{0.121569,0.466667,0.705882}%
\pgfsetstrokecolor{currentstroke}%
\pgfsetdash{}{0pt}%
\pgfpathmoveto{\pgfqpoint{5.534545in}{0.807905in}}%
\pgfpathlineto{\pgfqpoint{5.534545in}{0.851093in}}%
\pgfusepath{stroke}%
\end{pgfscope}%
\begin{pgfscope}%
\pgfpathrectangle{\pgfqpoint{0.800000in}{0.700000in}}{\pgfqpoint{4.960000in}{2.373913in}}%
\pgfusepath{clip}%
\pgfsetbuttcap%
\pgfsetroundjoin%
\definecolor{currentfill}{rgb}{0.121569,0.466667,0.705882}%
\pgfsetfillcolor{currentfill}%
\pgfsetlinewidth{1.003750pt}%
\definecolor{currentstroke}{rgb}{0.121569,0.466667,0.705882}%
\pgfsetstrokecolor{currentstroke}%
\pgfsetdash{}{0pt}%
\pgfsys@defobject{currentmarker}{\pgfqpoint{-0.041667in}{-0.000000in}}{\pgfqpoint{0.041667in}{0.000000in}}{%
\pgfpathmoveto{\pgfqpoint{0.041667in}{-0.000000in}}%
\pgfpathlineto{\pgfqpoint{-0.041667in}{0.000000in}}%
\pgfusepath{stroke,fill}%
}%
\begin{pgfscope}%
\pgfsys@transformshift{1.025455in}{2.227901in}%
\pgfsys@useobject{currentmarker}{}%
\end{pgfscope}%
\begin{pgfscope}%
\pgfsys@transformshift{1.083263in}{2.055208in}%
\pgfsys@useobject{currentmarker}{}%
\end{pgfscope}%
\begin{pgfscope}%
\pgfsys@transformshift{1.141072in}{2.023343in}%
\pgfsys@useobject{currentmarker}{}%
\end{pgfscope}%
\begin{pgfscope}%
\pgfsys@transformshift{1.198881in}{1.878137in}%
\pgfsys@useobject{currentmarker}{}%
\end{pgfscope}%
\begin{pgfscope}%
\pgfsys@transformshift{1.256690in}{2.032784in}%
\pgfsys@useobject{currentmarker}{}%
\end{pgfscope}%
\begin{pgfscope}%
\pgfsys@transformshift{1.314499in}{1.740536in}%
\pgfsys@useobject{currentmarker}{}%
\end{pgfscope}%
\begin{pgfscope}%
\pgfsys@transformshift{1.372308in}{1.589512in}%
\pgfsys@useobject{currentmarker}{}%
\end{pgfscope}%
\begin{pgfscope}%
\pgfsys@transformshift{1.430117in}{1.433969in}%
\pgfsys@useobject{currentmarker}{}%
\end{pgfscope}%
\begin{pgfscope}%
\pgfsys@transformshift{1.487925in}{1.333865in}%
\pgfsys@useobject{currentmarker}{}%
\end{pgfscope}%
\begin{pgfscope}%
\pgfsys@transformshift{1.545734in}{1.281593in}%
\pgfsys@useobject{currentmarker}{}%
\end{pgfscope}%
\begin{pgfscope}%
\pgfsys@transformshift{1.603543in}{1.259121in}%
\pgfsys@useobject{currentmarker}{}%
\end{pgfscope}%
\begin{pgfscope}%
\pgfsys@transformshift{1.661352in}{1.237225in}%
\pgfsys@useobject{currentmarker}{}%
\end{pgfscope}%
\begin{pgfscope}%
\pgfsys@transformshift{1.719161in}{1.226532in}%
\pgfsys@useobject{currentmarker}{}%
\end{pgfscope}%
\begin{pgfscope}%
\pgfsys@transformshift{1.776970in}{1.220324in}%
\pgfsys@useobject{currentmarker}{}%
\end{pgfscope}%
\begin{pgfscope}%
\pgfsys@transformshift{1.834779in}{1.227000in}%
\pgfsys@useobject{currentmarker}{}%
\end{pgfscope}%
\begin{pgfscope}%
\pgfsys@transformshift{1.892587in}{1.236869in}%
\pgfsys@useobject{currentmarker}{}%
\end{pgfscope}%
\begin{pgfscope}%
\pgfsys@transformshift{1.950396in}{1.238509in}%
\pgfsys@useobject{currentmarker}{}%
\end{pgfscope}%
\begin{pgfscope}%
\pgfsys@transformshift{2.008205in}{1.245398in}%
\pgfsys@useobject{currentmarker}{}%
\end{pgfscope}%
\begin{pgfscope}%
\pgfsys@transformshift{2.066014in}{1.248469in}%
\pgfsys@useobject{currentmarker}{}%
\end{pgfscope}%
\begin{pgfscope}%
\pgfsys@transformshift{2.123823in}{1.243825in}%
\pgfsys@useobject{currentmarker}{}%
\end{pgfscope}%
\begin{pgfscope}%
\pgfsys@transformshift{2.181632in}{1.245335in}%
\pgfsys@useobject{currentmarker}{}%
\end{pgfscope}%
\begin{pgfscope}%
\pgfsys@transformshift{2.239441in}{1.242393in}%
\pgfsys@useobject{currentmarker}{}%
\end{pgfscope}%
\begin{pgfscope}%
\pgfsys@transformshift{2.297249in}{1.236465in}%
\pgfsys@useobject{currentmarker}{}%
\end{pgfscope}%
\begin{pgfscope}%
\pgfsys@transformshift{2.355058in}{1.235389in}%
\pgfsys@useobject{currentmarker}{}%
\end{pgfscope}%
\begin{pgfscope}%
\pgfsys@transformshift{2.412867in}{1.238231in}%
\pgfsys@useobject{currentmarker}{}%
\end{pgfscope}%
\begin{pgfscope}%
\pgfsys@transformshift{2.470676in}{1.239938in}%
\pgfsys@useobject{currentmarker}{}%
\end{pgfscope}%
\begin{pgfscope}%
\pgfsys@transformshift{2.528485in}{1.244213in}%
\pgfsys@useobject{currentmarker}{}%
\end{pgfscope}%
\begin{pgfscope}%
\pgfsys@transformshift{2.586294in}{1.249884in}%
\pgfsys@useobject{currentmarker}{}%
\end{pgfscope}%
\begin{pgfscope}%
\pgfsys@transformshift{2.644103in}{1.250531in}%
\pgfsys@useobject{currentmarker}{}%
\end{pgfscope}%
\begin{pgfscope}%
\pgfsys@transformshift{2.701911in}{1.247567in}%
\pgfsys@useobject{currentmarker}{}%
\end{pgfscope}%
\begin{pgfscope}%
\pgfsys@transformshift{2.759720in}{1.242646in}%
\pgfsys@useobject{currentmarker}{}%
\end{pgfscope}%
\begin{pgfscope}%
\pgfsys@transformshift{2.817529in}{1.237066in}%
\pgfsys@useobject{currentmarker}{}%
\end{pgfscope}%
\begin{pgfscope}%
\pgfsys@transformshift{2.875338in}{1.231153in}%
\pgfsys@useobject{currentmarker}{}%
\end{pgfscope}%
\begin{pgfscope}%
\pgfsys@transformshift{2.933147in}{1.225483in}%
\pgfsys@useobject{currentmarker}{}%
\end{pgfscope}%
\begin{pgfscope}%
\pgfsys@transformshift{2.990956in}{1.218645in}%
\pgfsys@useobject{currentmarker}{}%
\end{pgfscope}%
\begin{pgfscope}%
\pgfsys@transformshift{3.048765in}{1.212782in}%
\pgfsys@useobject{currentmarker}{}%
\end{pgfscope}%
\begin{pgfscope}%
\pgfsys@transformshift{3.106573in}{1.204971in}%
\pgfsys@useobject{currentmarker}{}%
\end{pgfscope}%
\begin{pgfscope}%
\pgfsys@transformshift{3.164382in}{1.194042in}%
\pgfsys@useobject{currentmarker}{}%
\end{pgfscope}%
\begin{pgfscope}%
\pgfsys@transformshift{3.222191in}{1.182878in}%
\pgfsys@useobject{currentmarker}{}%
\end{pgfscope}%
\begin{pgfscope}%
\pgfsys@transformshift{3.280000in}{1.171547in}%
\pgfsys@useobject{currentmarker}{}%
\end{pgfscope}%
\begin{pgfscope}%
\pgfsys@transformshift{3.337809in}{1.160624in}%
\pgfsys@useobject{currentmarker}{}%
\end{pgfscope}%
\begin{pgfscope}%
\pgfsys@transformshift{3.395618in}{1.154210in}%
\pgfsys@useobject{currentmarker}{}%
\end{pgfscope}%
\begin{pgfscope}%
\pgfsys@transformshift{3.453427in}{1.147943in}%
\pgfsys@useobject{currentmarker}{}%
\end{pgfscope}%
\begin{pgfscope}%
\pgfsys@transformshift{3.511235in}{1.143549in}%
\pgfsys@useobject{currentmarker}{}%
\end{pgfscope}%
\begin{pgfscope}%
\pgfsys@transformshift{3.569044in}{1.136819in}%
\pgfsys@useobject{currentmarker}{}%
\end{pgfscope}%
\begin{pgfscope}%
\pgfsys@transformshift{3.626853in}{1.129975in}%
\pgfsys@useobject{currentmarker}{}%
\end{pgfscope}%
\begin{pgfscope}%
\pgfsys@transformshift{3.684662in}{1.124471in}%
\pgfsys@useobject{currentmarker}{}%
\end{pgfscope}%
\begin{pgfscope}%
\pgfsys@transformshift{3.742471in}{1.120107in}%
\pgfsys@useobject{currentmarker}{}%
\end{pgfscope}%
\begin{pgfscope}%
\pgfsys@transformshift{3.800280in}{1.111883in}%
\pgfsys@useobject{currentmarker}{}%
\end{pgfscope}%
\begin{pgfscope}%
\pgfsys@transformshift{3.858089in}{1.101281in}%
\pgfsys@useobject{currentmarker}{}%
\end{pgfscope}%
\begin{pgfscope}%
\pgfsys@transformshift{3.915897in}{1.092246in}%
\pgfsys@useobject{currentmarker}{}%
\end{pgfscope}%
\begin{pgfscope}%
\pgfsys@transformshift{3.973706in}{1.085600in}%
\pgfsys@useobject{currentmarker}{}%
\end{pgfscope}%
\begin{pgfscope}%
\pgfsys@transformshift{4.031515in}{1.075822in}%
\pgfsys@useobject{currentmarker}{}%
\end{pgfscope}%
\begin{pgfscope}%
\pgfsys@transformshift{4.089324in}{1.068237in}%
\pgfsys@useobject{currentmarker}{}%
\end{pgfscope}%
\begin{pgfscope}%
\pgfsys@transformshift{4.147133in}{1.060173in}%
\pgfsys@useobject{currentmarker}{}%
\end{pgfscope}%
\begin{pgfscope}%
\pgfsys@transformshift{4.204942in}{1.050080in}%
\pgfsys@useobject{currentmarker}{}%
\end{pgfscope}%
\begin{pgfscope}%
\pgfsys@transformshift{4.262751in}{1.037182in}%
\pgfsys@useobject{currentmarker}{}%
\end{pgfscope}%
\begin{pgfscope}%
\pgfsys@transformshift{4.320559in}{1.023418in}%
\pgfsys@useobject{currentmarker}{}%
\end{pgfscope}%
\begin{pgfscope}%
\pgfsys@transformshift{4.378368in}{1.010999in}%
\pgfsys@useobject{currentmarker}{}%
\end{pgfscope}%
\begin{pgfscope}%
\pgfsys@transformshift{4.436177in}{1.001699in}%
\pgfsys@useobject{currentmarker}{}%
\end{pgfscope}%
\begin{pgfscope}%
\pgfsys@transformshift{4.493986in}{0.992601in}%
\pgfsys@useobject{currentmarker}{}%
\end{pgfscope}%
\begin{pgfscope}%
\pgfsys@transformshift{4.551795in}{0.981242in}%
\pgfsys@useobject{currentmarker}{}%
\end{pgfscope}%
\begin{pgfscope}%
\pgfsys@transformshift{4.609604in}{0.971203in}%
\pgfsys@useobject{currentmarker}{}%
\end{pgfscope}%
\begin{pgfscope}%
\pgfsys@transformshift{4.667413in}{0.963048in}%
\pgfsys@useobject{currentmarker}{}%
\end{pgfscope}%
\begin{pgfscope}%
\pgfsys@transformshift{4.725221in}{0.953296in}%
\pgfsys@useobject{currentmarker}{}%
\end{pgfscope}%
\begin{pgfscope}%
\pgfsys@transformshift{4.783030in}{0.943009in}%
\pgfsys@useobject{currentmarker}{}%
\end{pgfscope}%
\begin{pgfscope}%
\pgfsys@transformshift{4.840839in}{0.933540in}%
\pgfsys@useobject{currentmarker}{}%
\end{pgfscope}%
\begin{pgfscope}%
\pgfsys@transformshift{4.898648in}{0.927111in}%
\pgfsys@useobject{currentmarker}{}%
\end{pgfscope}%
\begin{pgfscope}%
\pgfsys@transformshift{4.956457in}{0.916255in}%
\pgfsys@useobject{currentmarker}{}%
\end{pgfscope}%
\begin{pgfscope}%
\pgfsys@transformshift{5.014266in}{0.906534in}%
\pgfsys@useobject{currentmarker}{}%
\end{pgfscope}%
\begin{pgfscope}%
\pgfsys@transformshift{5.072075in}{0.894831in}%
\pgfsys@useobject{currentmarker}{}%
\end{pgfscope}%
\begin{pgfscope}%
\pgfsys@transformshift{5.129883in}{0.885072in}%
\pgfsys@useobject{currentmarker}{}%
\end{pgfscope}%
\begin{pgfscope}%
\pgfsys@transformshift{5.187692in}{0.876267in}%
\pgfsys@useobject{currentmarker}{}%
\end{pgfscope}%
\begin{pgfscope}%
\pgfsys@transformshift{5.245501in}{0.859655in}%
\pgfsys@useobject{currentmarker}{}%
\end{pgfscope}%
\begin{pgfscope}%
\pgfsys@transformshift{5.303310in}{0.849048in}%
\pgfsys@useobject{currentmarker}{}%
\end{pgfscope}%
\begin{pgfscope}%
\pgfsys@transformshift{5.361119in}{0.841164in}%
\pgfsys@useobject{currentmarker}{}%
\end{pgfscope}%
\begin{pgfscope}%
\pgfsys@transformshift{5.418928in}{0.831615in}%
\pgfsys@useobject{currentmarker}{}%
\end{pgfscope}%
\begin{pgfscope}%
\pgfsys@transformshift{5.476737in}{0.820735in}%
\pgfsys@useobject{currentmarker}{}%
\end{pgfscope}%
\begin{pgfscope}%
\pgfsys@transformshift{5.534545in}{0.807905in}%
\pgfsys@useobject{currentmarker}{}%
\end{pgfscope}%
\end{pgfscope}%
\begin{pgfscope}%
\pgfpathrectangle{\pgfqpoint{0.800000in}{0.700000in}}{\pgfqpoint{4.960000in}{2.373913in}}%
\pgfusepath{clip}%
\pgfsetbuttcap%
\pgfsetroundjoin%
\definecolor{currentfill}{rgb}{0.121569,0.466667,0.705882}%
\pgfsetfillcolor{currentfill}%
\pgfsetlinewidth{1.003750pt}%
\definecolor{currentstroke}{rgb}{0.121569,0.466667,0.705882}%
\pgfsetstrokecolor{currentstroke}%
\pgfsetdash{}{0pt}%
\pgfsys@defobject{currentmarker}{\pgfqpoint{-0.041667in}{-0.000000in}}{\pgfqpoint{0.041667in}{0.000000in}}{%
\pgfpathmoveto{\pgfqpoint{0.041667in}{-0.000000in}}%
\pgfpathlineto{\pgfqpoint{-0.041667in}{0.000000in}}%
\pgfusepath{stroke,fill}%
}%
\begin{pgfscope}%
\pgfsys@transformshift{1.025455in}{2.251332in}%
\pgfsys@useobject{currentmarker}{}%
\end{pgfscope}%
\begin{pgfscope}%
\pgfsys@transformshift{1.083263in}{2.690468in}%
\pgfsys@useobject{currentmarker}{}%
\end{pgfscope}%
\begin{pgfscope}%
\pgfsys@transformshift{1.141072in}{2.715325in}%
\pgfsys@useobject{currentmarker}{}%
\end{pgfscope}%
\begin{pgfscope}%
\pgfsys@transformshift{1.198881in}{2.372552in}%
\pgfsys@useobject{currentmarker}{}%
\end{pgfscope}%
\begin{pgfscope}%
\pgfsys@transformshift{1.256690in}{2.095656in}%
\pgfsys@useobject{currentmarker}{}%
\end{pgfscope}%
\begin{pgfscope}%
\pgfsys@transformshift{1.314499in}{1.838640in}%
\pgfsys@useobject{currentmarker}{}%
\end{pgfscope}%
\begin{pgfscope}%
\pgfsys@transformshift{1.372308in}{1.635303in}%
\pgfsys@useobject{currentmarker}{}%
\end{pgfscope}%
\begin{pgfscope}%
\pgfsys@transformshift{1.430117in}{1.461599in}%
\pgfsys@useobject{currentmarker}{}%
\end{pgfscope}%
\begin{pgfscope}%
\pgfsys@transformshift{1.487925in}{1.356369in}%
\pgfsys@useobject{currentmarker}{}%
\end{pgfscope}%
\begin{pgfscope}%
\pgfsys@transformshift{1.545734in}{1.302297in}%
\pgfsys@useobject{currentmarker}{}%
\end{pgfscope}%
\begin{pgfscope}%
\pgfsys@transformshift{1.603543in}{1.278707in}%
\pgfsys@useobject{currentmarker}{}%
\end{pgfscope}%
\begin{pgfscope}%
\pgfsys@transformshift{1.661352in}{1.256080in}%
\pgfsys@useobject{currentmarker}{}%
\end{pgfscope}%
\begin{pgfscope}%
\pgfsys@transformshift{1.719161in}{1.246398in}%
\pgfsys@useobject{currentmarker}{}%
\end{pgfscope}%
\begin{pgfscope}%
\pgfsys@transformshift{1.776970in}{1.242253in}%
\pgfsys@useobject{currentmarker}{}%
\end{pgfscope}%
\begin{pgfscope}%
\pgfsys@transformshift{1.834779in}{1.249225in}%
\pgfsys@useobject{currentmarker}{}%
\end{pgfscope}%
\begin{pgfscope}%
\pgfsys@transformshift{1.892587in}{1.259592in}%
\pgfsys@useobject{currentmarker}{}%
\end{pgfscope}%
\begin{pgfscope}%
\pgfsys@transformshift{1.950396in}{1.261105in}%
\pgfsys@useobject{currentmarker}{}%
\end{pgfscope}%
\begin{pgfscope}%
\pgfsys@transformshift{2.008205in}{1.268165in}%
\pgfsys@useobject{currentmarker}{}%
\end{pgfscope}%
\begin{pgfscope}%
\pgfsys@transformshift{2.066014in}{1.272262in}%
\pgfsys@useobject{currentmarker}{}%
\end{pgfscope}%
\begin{pgfscope}%
\pgfsys@transformshift{2.123823in}{1.268235in}%
\pgfsys@useobject{currentmarker}{}%
\end{pgfscope}%
\begin{pgfscope}%
\pgfsys@transformshift{2.181632in}{1.270220in}%
\pgfsys@useobject{currentmarker}{}%
\end{pgfscope}%
\begin{pgfscope}%
\pgfsys@transformshift{2.239441in}{1.267428in}%
\pgfsys@useobject{currentmarker}{}%
\end{pgfscope}%
\begin{pgfscope}%
\pgfsys@transformshift{2.297249in}{1.262694in}%
\pgfsys@useobject{currentmarker}{}%
\end{pgfscope}%
\begin{pgfscope}%
\pgfsys@transformshift{2.355058in}{1.263125in}%
\pgfsys@useobject{currentmarker}{}%
\end{pgfscope}%
\begin{pgfscope}%
\pgfsys@transformshift{2.412867in}{1.265322in}%
\pgfsys@useobject{currentmarker}{}%
\end{pgfscope}%
\begin{pgfscope}%
\pgfsys@transformshift{2.470676in}{1.266676in}%
\pgfsys@useobject{currentmarker}{}%
\end{pgfscope}%
\begin{pgfscope}%
\pgfsys@transformshift{2.528485in}{1.271592in}%
\pgfsys@useobject{currentmarker}{}%
\end{pgfscope}%
\begin{pgfscope}%
\pgfsys@transformshift{2.586294in}{1.276988in}%
\pgfsys@useobject{currentmarker}{}%
\end{pgfscope}%
\begin{pgfscope}%
\pgfsys@transformshift{2.644103in}{1.278807in}%
\pgfsys@useobject{currentmarker}{}%
\end{pgfscope}%
\begin{pgfscope}%
\pgfsys@transformshift{2.701911in}{1.275636in}%
\pgfsys@useobject{currentmarker}{}%
\end{pgfscope}%
\begin{pgfscope}%
\pgfsys@transformshift{2.759720in}{1.271134in}%
\pgfsys@useobject{currentmarker}{}%
\end{pgfscope}%
\begin{pgfscope}%
\pgfsys@transformshift{2.817529in}{1.266024in}%
\pgfsys@useobject{currentmarker}{}%
\end{pgfscope}%
\begin{pgfscope}%
\pgfsys@transformshift{2.875338in}{1.260362in}%
\pgfsys@useobject{currentmarker}{}%
\end{pgfscope}%
\begin{pgfscope}%
\pgfsys@transformshift{2.933147in}{1.255073in}%
\pgfsys@useobject{currentmarker}{}%
\end{pgfscope}%
\begin{pgfscope}%
\pgfsys@transformshift{2.990956in}{1.247495in}%
\pgfsys@useobject{currentmarker}{}%
\end{pgfscope}%
\begin{pgfscope}%
\pgfsys@transformshift{3.048765in}{1.243015in}%
\pgfsys@useobject{currentmarker}{}%
\end{pgfscope}%
\begin{pgfscope}%
\pgfsys@transformshift{3.106573in}{1.234687in}%
\pgfsys@useobject{currentmarker}{}%
\end{pgfscope}%
\begin{pgfscope}%
\pgfsys@transformshift{3.164382in}{1.223997in}%
\pgfsys@useobject{currentmarker}{}%
\end{pgfscope}%
\begin{pgfscope}%
\pgfsys@transformshift{3.222191in}{1.212447in}%
\pgfsys@useobject{currentmarker}{}%
\end{pgfscope}%
\begin{pgfscope}%
\pgfsys@transformshift{3.280000in}{1.201662in}%
\pgfsys@useobject{currentmarker}{}%
\end{pgfscope}%
\begin{pgfscope}%
\pgfsys@transformshift{3.337809in}{1.191493in}%
\pgfsys@useobject{currentmarker}{}%
\end{pgfscope}%
\begin{pgfscope}%
\pgfsys@transformshift{3.395618in}{1.184395in}%
\pgfsys@useobject{currentmarker}{}%
\end{pgfscope}%
\begin{pgfscope}%
\pgfsys@transformshift{3.453427in}{1.178124in}%
\pgfsys@useobject{currentmarker}{}%
\end{pgfscope}%
\begin{pgfscope}%
\pgfsys@transformshift{3.511235in}{1.175109in}%
\pgfsys@useobject{currentmarker}{}%
\end{pgfscope}%
\begin{pgfscope}%
\pgfsys@transformshift{3.569044in}{1.169022in}%
\pgfsys@useobject{currentmarker}{}%
\end{pgfscope}%
\begin{pgfscope}%
\pgfsys@transformshift{3.626853in}{1.162823in}%
\pgfsys@useobject{currentmarker}{}%
\end{pgfscope}%
\begin{pgfscope}%
\pgfsys@transformshift{3.684662in}{1.157421in}%
\pgfsys@useobject{currentmarker}{}%
\end{pgfscope}%
\begin{pgfscope}%
\pgfsys@transformshift{3.742471in}{1.151882in}%
\pgfsys@useobject{currentmarker}{}%
\end{pgfscope}%
\begin{pgfscope}%
\pgfsys@transformshift{3.800280in}{1.145145in}%
\pgfsys@useobject{currentmarker}{}%
\end{pgfscope}%
\begin{pgfscope}%
\pgfsys@transformshift{3.858089in}{1.134015in}%
\pgfsys@useobject{currentmarker}{}%
\end{pgfscope}%
\begin{pgfscope}%
\pgfsys@transformshift{3.915897in}{1.126582in}%
\pgfsys@useobject{currentmarker}{}%
\end{pgfscope}%
\begin{pgfscope}%
\pgfsys@transformshift{3.973706in}{1.118501in}%
\pgfsys@useobject{currentmarker}{}%
\end{pgfscope}%
\begin{pgfscope}%
\pgfsys@transformshift{4.031515in}{1.109769in}%
\pgfsys@useobject{currentmarker}{}%
\end{pgfscope}%
\begin{pgfscope}%
\pgfsys@transformshift{4.089324in}{1.102740in}%
\pgfsys@useobject{currentmarker}{}%
\end{pgfscope}%
\begin{pgfscope}%
\pgfsys@transformshift{4.147133in}{1.095793in}%
\pgfsys@useobject{currentmarker}{}%
\end{pgfscope}%
\begin{pgfscope}%
\pgfsys@transformshift{4.204942in}{1.085519in}%
\pgfsys@useobject{currentmarker}{}%
\end{pgfscope}%
\begin{pgfscope}%
\pgfsys@transformshift{4.262751in}{1.073941in}%
\pgfsys@useobject{currentmarker}{}%
\end{pgfscope}%
\begin{pgfscope}%
\pgfsys@transformshift{4.320559in}{1.061012in}%
\pgfsys@useobject{currentmarker}{}%
\end{pgfscope}%
\begin{pgfscope}%
\pgfsys@transformshift{4.378368in}{1.048805in}%
\pgfsys@useobject{currentmarker}{}%
\end{pgfscope}%
\begin{pgfscope}%
\pgfsys@transformshift{4.436177in}{1.039811in}%
\pgfsys@useobject{currentmarker}{}%
\end{pgfscope}%
\begin{pgfscope}%
\pgfsys@transformshift{4.493986in}{1.030645in}%
\pgfsys@useobject{currentmarker}{}%
\end{pgfscope}%
\begin{pgfscope}%
\pgfsys@transformshift{4.551795in}{1.019198in}%
\pgfsys@useobject{currentmarker}{}%
\end{pgfscope}%
\begin{pgfscope}%
\pgfsys@transformshift{4.609604in}{1.010921in}%
\pgfsys@useobject{currentmarker}{}%
\end{pgfscope}%
\begin{pgfscope}%
\pgfsys@transformshift{4.667413in}{1.001607in}%
\pgfsys@useobject{currentmarker}{}%
\end{pgfscope}%
\begin{pgfscope}%
\pgfsys@transformshift{4.725221in}{0.991495in}%
\pgfsys@useobject{currentmarker}{}%
\end{pgfscope}%
\begin{pgfscope}%
\pgfsys@transformshift{4.783030in}{0.981979in}%
\pgfsys@useobject{currentmarker}{}%
\end{pgfscope}%
\begin{pgfscope}%
\pgfsys@transformshift{4.840839in}{0.972762in}%
\pgfsys@useobject{currentmarker}{}%
\end{pgfscope}%
\begin{pgfscope}%
\pgfsys@transformshift{4.898648in}{0.968773in}%
\pgfsys@useobject{currentmarker}{}%
\end{pgfscope}%
\begin{pgfscope}%
\pgfsys@transformshift{4.956457in}{0.956645in}%
\pgfsys@useobject{currentmarker}{}%
\end{pgfscope}%
\begin{pgfscope}%
\pgfsys@transformshift{5.014266in}{0.947324in}%
\pgfsys@useobject{currentmarker}{}%
\end{pgfscope}%
\begin{pgfscope}%
\pgfsys@transformshift{5.072075in}{0.936751in}%
\pgfsys@useobject{currentmarker}{}%
\end{pgfscope}%
\begin{pgfscope}%
\pgfsys@transformshift{5.129883in}{0.928305in}%
\pgfsys@useobject{currentmarker}{}%
\end{pgfscope}%
\begin{pgfscope}%
\pgfsys@transformshift{5.187692in}{0.918820in}%
\pgfsys@useobject{currentmarker}{}%
\end{pgfscope}%
\begin{pgfscope}%
\pgfsys@transformshift{5.245501in}{0.900520in}%
\pgfsys@useobject{currentmarker}{}%
\end{pgfscope}%
\begin{pgfscope}%
\pgfsys@transformshift{5.303310in}{0.891453in}%
\pgfsys@useobject{currentmarker}{}%
\end{pgfscope}%
\begin{pgfscope}%
\pgfsys@transformshift{5.361119in}{0.882446in}%
\pgfsys@useobject{currentmarker}{}%
\end{pgfscope}%
\begin{pgfscope}%
\pgfsys@transformshift{5.418928in}{0.873515in}%
\pgfsys@useobject{currentmarker}{}%
\end{pgfscope}%
\begin{pgfscope}%
\pgfsys@transformshift{5.476737in}{0.864195in}%
\pgfsys@useobject{currentmarker}{}%
\end{pgfscope}%
\begin{pgfscope}%
\pgfsys@transformshift{5.534545in}{0.851093in}%
\pgfsys@useobject{currentmarker}{}%
\end{pgfscope}%
\end{pgfscope}%
\begin{pgfscope}%
\pgfpathrectangle{\pgfqpoint{0.800000in}{0.700000in}}{\pgfqpoint{4.960000in}{2.373913in}}%
\pgfusepath{clip}%
\pgfsetrectcap%
\pgfsetroundjoin%
\pgfsetlinewidth{1.505625pt}%
\definecolor{currentstroke}{rgb}{1.000000,0.498039,0.054902}%
\pgfsetstrokecolor{currentstroke}%
\pgfsetdash{}{0pt}%
\pgfpathmoveto{\pgfqpoint{1.025455in}{2.966008in}}%
\pgfpathlineto{\pgfqpoint{1.083263in}{2.848147in}}%
\pgfpathlineto{\pgfqpoint{1.141072in}{2.922055in}}%
\pgfpathlineto{\pgfqpoint{1.198881in}{1.835477in}}%
\pgfpathlineto{\pgfqpoint{1.256690in}{1.662640in}}%
\pgfpathlineto{\pgfqpoint{1.314499in}{1.649667in}}%
\pgfpathlineto{\pgfqpoint{1.372308in}{1.611644in}}%
\pgfpathlineto{\pgfqpoint{1.430117in}{1.513608in}}%
\pgfpathlineto{\pgfqpoint{1.487925in}{1.409278in}}%
\pgfpathlineto{\pgfqpoint{1.545734in}{1.344934in}}%
\pgfpathlineto{\pgfqpoint{1.603543in}{1.303676in}}%
\pgfpathlineto{\pgfqpoint{1.661352in}{1.271480in}}%
\pgfpathlineto{\pgfqpoint{1.719161in}{1.258546in}}%
\pgfpathlineto{\pgfqpoint{1.776970in}{1.250955in}}%
\pgfpathlineto{\pgfqpoint{1.834779in}{1.251484in}}%
\pgfpathlineto{\pgfqpoint{1.892587in}{1.256464in}}%
\pgfpathlineto{\pgfqpoint{1.950396in}{1.255212in}}%
\pgfpathlineto{\pgfqpoint{2.008205in}{1.251888in}}%
\pgfpathlineto{\pgfqpoint{2.066014in}{1.249375in}}%
\pgfpathlineto{\pgfqpoint{2.123823in}{1.254628in}}%
\pgfpathlineto{\pgfqpoint{2.181632in}{1.253351in}}%
\pgfpathlineto{\pgfqpoint{2.239441in}{1.246812in}}%
\pgfpathlineto{\pgfqpoint{2.297249in}{1.244882in}}%
\pgfpathlineto{\pgfqpoint{2.355058in}{1.242513in}}%
\pgfpathlineto{\pgfqpoint{2.412867in}{1.239929in}}%
\pgfpathlineto{\pgfqpoint{2.470676in}{1.233876in}}%
\pgfpathlineto{\pgfqpoint{2.528485in}{1.226620in}}%
\pgfpathlineto{\pgfqpoint{2.586294in}{1.222717in}}%
\pgfpathlineto{\pgfqpoint{2.644103in}{1.218470in}}%
\pgfpathlineto{\pgfqpoint{2.701911in}{1.209852in}}%
\pgfpathlineto{\pgfqpoint{2.759720in}{1.204883in}}%
\pgfpathlineto{\pgfqpoint{2.817529in}{1.200355in}}%
\pgfpathlineto{\pgfqpoint{2.875338in}{1.197352in}}%
\pgfpathlineto{\pgfqpoint{2.933147in}{1.193726in}}%
\pgfpathlineto{\pgfqpoint{2.990956in}{1.189986in}}%
\pgfpathlineto{\pgfqpoint{3.048765in}{1.176627in}}%
\pgfpathlineto{\pgfqpoint{3.106573in}{1.149717in}}%
\pgfpathlineto{\pgfqpoint{3.164382in}{1.125249in}}%
\pgfpathlineto{\pgfqpoint{3.222191in}{1.104096in}}%
\pgfpathlineto{\pgfqpoint{3.280000in}{1.085421in}}%
\pgfpathlineto{\pgfqpoint{3.337809in}{1.067563in}}%
\pgfpathlineto{\pgfqpoint{3.395618in}{1.051496in}}%
\pgfpathlineto{\pgfqpoint{3.453427in}{1.036586in}}%
\pgfpathlineto{\pgfqpoint{3.511235in}{1.021793in}}%
\pgfpathlineto{\pgfqpoint{3.569044in}{1.006662in}}%
\pgfpathlineto{\pgfqpoint{3.626853in}{0.990647in}}%
\pgfpathlineto{\pgfqpoint{3.684662in}{0.977354in}}%
\pgfpathlineto{\pgfqpoint{3.742471in}{0.961447in}}%
\pgfpathlineto{\pgfqpoint{3.800280in}{0.948577in}}%
\pgfusepath{stroke}%
\end{pgfscope}%
\begin{pgfscope}%
\pgfpathrectangle{\pgfqpoint{0.800000in}{0.700000in}}{\pgfqpoint{4.960000in}{2.373913in}}%
\pgfusepath{clip}%
\pgfsetbuttcap%
\pgfsetroundjoin%
\definecolor{currentfill}{rgb}{1.000000,0.498039,0.054902}%
\pgfsetfillcolor{currentfill}%
\pgfsetlinewidth{1.003750pt}%
\definecolor{currentstroke}{rgb}{1.000000,0.498039,0.054902}%
\pgfsetstrokecolor{currentstroke}%
\pgfsetdash{}{0pt}%
\pgfsys@defobject{currentmarker}{\pgfqpoint{-0.020833in}{-0.020833in}}{\pgfqpoint{0.020833in}{0.020833in}}{%
\pgfpathmoveto{\pgfqpoint{0.000000in}{-0.020833in}}%
\pgfpathcurveto{\pgfqpoint{0.005525in}{-0.020833in}}{\pgfqpoint{0.010825in}{-0.018638in}}{\pgfqpoint{0.014731in}{-0.014731in}}%
\pgfpathcurveto{\pgfqpoint{0.018638in}{-0.010825in}}{\pgfqpoint{0.020833in}{-0.005525in}}{\pgfqpoint{0.020833in}{0.000000in}}%
\pgfpathcurveto{\pgfqpoint{0.020833in}{0.005525in}}{\pgfqpoint{0.018638in}{0.010825in}}{\pgfqpoint{0.014731in}{0.014731in}}%
\pgfpathcurveto{\pgfqpoint{0.010825in}{0.018638in}}{\pgfqpoint{0.005525in}{0.020833in}}{\pgfqpoint{0.000000in}{0.020833in}}%
\pgfpathcurveto{\pgfqpoint{-0.005525in}{0.020833in}}{\pgfqpoint{-0.010825in}{0.018638in}}{\pgfqpoint{-0.014731in}{0.014731in}}%
\pgfpathcurveto{\pgfqpoint{-0.018638in}{0.010825in}}{\pgfqpoint{-0.020833in}{0.005525in}}{\pgfqpoint{-0.020833in}{0.000000in}}%
\pgfpathcurveto{\pgfqpoint{-0.020833in}{-0.005525in}}{\pgfqpoint{-0.018638in}{-0.010825in}}{\pgfqpoint{-0.014731in}{-0.014731in}}%
\pgfpathcurveto{\pgfqpoint{-0.010825in}{-0.018638in}}{\pgfqpoint{-0.005525in}{-0.020833in}}{\pgfqpoint{0.000000in}{-0.020833in}}%
\pgfpathlineto{\pgfqpoint{0.000000in}{-0.020833in}}%
\pgfpathclose%
\pgfusepath{stroke,fill}%
}%
\begin{pgfscope}%
\pgfsys@transformshift{1.025455in}{2.966008in}%
\pgfsys@useobject{currentmarker}{}%
\end{pgfscope}%
\begin{pgfscope}%
\pgfsys@transformshift{1.083263in}{2.848147in}%
\pgfsys@useobject{currentmarker}{}%
\end{pgfscope}%
\begin{pgfscope}%
\pgfsys@transformshift{1.141072in}{2.922055in}%
\pgfsys@useobject{currentmarker}{}%
\end{pgfscope}%
\begin{pgfscope}%
\pgfsys@transformshift{1.198881in}{1.835477in}%
\pgfsys@useobject{currentmarker}{}%
\end{pgfscope}%
\begin{pgfscope}%
\pgfsys@transformshift{1.256690in}{1.662640in}%
\pgfsys@useobject{currentmarker}{}%
\end{pgfscope}%
\begin{pgfscope}%
\pgfsys@transformshift{1.314499in}{1.649667in}%
\pgfsys@useobject{currentmarker}{}%
\end{pgfscope}%
\begin{pgfscope}%
\pgfsys@transformshift{1.372308in}{1.611644in}%
\pgfsys@useobject{currentmarker}{}%
\end{pgfscope}%
\begin{pgfscope}%
\pgfsys@transformshift{1.430117in}{1.513608in}%
\pgfsys@useobject{currentmarker}{}%
\end{pgfscope}%
\begin{pgfscope}%
\pgfsys@transformshift{1.487925in}{1.409278in}%
\pgfsys@useobject{currentmarker}{}%
\end{pgfscope}%
\begin{pgfscope}%
\pgfsys@transformshift{1.545734in}{1.344934in}%
\pgfsys@useobject{currentmarker}{}%
\end{pgfscope}%
\begin{pgfscope}%
\pgfsys@transformshift{1.603543in}{1.303676in}%
\pgfsys@useobject{currentmarker}{}%
\end{pgfscope}%
\begin{pgfscope}%
\pgfsys@transformshift{1.661352in}{1.271480in}%
\pgfsys@useobject{currentmarker}{}%
\end{pgfscope}%
\begin{pgfscope}%
\pgfsys@transformshift{1.719161in}{1.258546in}%
\pgfsys@useobject{currentmarker}{}%
\end{pgfscope}%
\begin{pgfscope}%
\pgfsys@transformshift{1.776970in}{1.250955in}%
\pgfsys@useobject{currentmarker}{}%
\end{pgfscope}%
\begin{pgfscope}%
\pgfsys@transformshift{1.834779in}{1.251484in}%
\pgfsys@useobject{currentmarker}{}%
\end{pgfscope}%
\begin{pgfscope}%
\pgfsys@transformshift{1.892587in}{1.256464in}%
\pgfsys@useobject{currentmarker}{}%
\end{pgfscope}%
\begin{pgfscope}%
\pgfsys@transformshift{1.950396in}{1.255212in}%
\pgfsys@useobject{currentmarker}{}%
\end{pgfscope}%
\begin{pgfscope}%
\pgfsys@transformshift{2.008205in}{1.251888in}%
\pgfsys@useobject{currentmarker}{}%
\end{pgfscope}%
\begin{pgfscope}%
\pgfsys@transformshift{2.066014in}{1.249375in}%
\pgfsys@useobject{currentmarker}{}%
\end{pgfscope}%
\begin{pgfscope}%
\pgfsys@transformshift{2.123823in}{1.254628in}%
\pgfsys@useobject{currentmarker}{}%
\end{pgfscope}%
\begin{pgfscope}%
\pgfsys@transformshift{2.181632in}{1.253351in}%
\pgfsys@useobject{currentmarker}{}%
\end{pgfscope}%
\begin{pgfscope}%
\pgfsys@transformshift{2.239441in}{1.246812in}%
\pgfsys@useobject{currentmarker}{}%
\end{pgfscope}%
\begin{pgfscope}%
\pgfsys@transformshift{2.297249in}{1.244882in}%
\pgfsys@useobject{currentmarker}{}%
\end{pgfscope}%
\begin{pgfscope}%
\pgfsys@transformshift{2.355058in}{1.242513in}%
\pgfsys@useobject{currentmarker}{}%
\end{pgfscope}%
\begin{pgfscope}%
\pgfsys@transformshift{2.412867in}{1.239929in}%
\pgfsys@useobject{currentmarker}{}%
\end{pgfscope}%
\begin{pgfscope}%
\pgfsys@transformshift{2.470676in}{1.233876in}%
\pgfsys@useobject{currentmarker}{}%
\end{pgfscope}%
\begin{pgfscope}%
\pgfsys@transformshift{2.528485in}{1.226620in}%
\pgfsys@useobject{currentmarker}{}%
\end{pgfscope}%
\begin{pgfscope}%
\pgfsys@transformshift{2.586294in}{1.222717in}%
\pgfsys@useobject{currentmarker}{}%
\end{pgfscope}%
\begin{pgfscope}%
\pgfsys@transformshift{2.644103in}{1.218470in}%
\pgfsys@useobject{currentmarker}{}%
\end{pgfscope}%
\begin{pgfscope}%
\pgfsys@transformshift{2.701911in}{1.209852in}%
\pgfsys@useobject{currentmarker}{}%
\end{pgfscope}%
\begin{pgfscope}%
\pgfsys@transformshift{2.759720in}{1.204883in}%
\pgfsys@useobject{currentmarker}{}%
\end{pgfscope}%
\begin{pgfscope}%
\pgfsys@transformshift{2.817529in}{1.200355in}%
\pgfsys@useobject{currentmarker}{}%
\end{pgfscope}%
\begin{pgfscope}%
\pgfsys@transformshift{2.875338in}{1.197352in}%
\pgfsys@useobject{currentmarker}{}%
\end{pgfscope}%
\begin{pgfscope}%
\pgfsys@transformshift{2.933147in}{1.193726in}%
\pgfsys@useobject{currentmarker}{}%
\end{pgfscope}%
\begin{pgfscope}%
\pgfsys@transformshift{2.990956in}{1.189986in}%
\pgfsys@useobject{currentmarker}{}%
\end{pgfscope}%
\begin{pgfscope}%
\pgfsys@transformshift{3.048765in}{1.176627in}%
\pgfsys@useobject{currentmarker}{}%
\end{pgfscope}%
\begin{pgfscope}%
\pgfsys@transformshift{3.106573in}{1.149717in}%
\pgfsys@useobject{currentmarker}{}%
\end{pgfscope}%
\begin{pgfscope}%
\pgfsys@transformshift{3.164382in}{1.125249in}%
\pgfsys@useobject{currentmarker}{}%
\end{pgfscope}%
\begin{pgfscope}%
\pgfsys@transformshift{3.222191in}{1.104096in}%
\pgfsys@useobject{currentmarker}{}%
\end{pgfscope}%
\begin{pgfscope}%
\pgfsys@transformshift{3.280000in}{1.085421in}%
\pgfsys@useobject{currentmarker}{}%
\end{pgfscope}%
\begin{pgfscope}%
\pgfsys@transformshift{3.337809in}{1.067563in}%
\pgfsys@useobject{currentmarker}{}%
\end{pgfscope}%
\begin{pgfscope}%
\pgfsys@transformshift{3.395618in}{1.051496in}%
\pgfsys@useobject{currentmarker}{}%
\end{pgfscope}%
\begin{pgfscope}%
\pgfsys@transformshift{3.453427in}{1.036586in}%
\pgfsys@useobject{currentmarker}{}%
\end{pgfscope}%
\begin{pgfscope}%
\pgfsys@transformshift{3.511235in}{1.021793in}%
\pgfsys@useobject{currentmarker}{}%
\end{pgfscope}%
\begin{pgfscope}%
\pgfsys@transformshift{3.569044in}{1.006662in}%
\pgfsys@useobject{currentmarker}{}%
\end{pgfscope}%
\begin{pgfscope}%
\pgfsys@transformshift{3.626853in}{0.990647in}%
\pgfsys@useobject{currentmarker}{}%
\end{pgfscope}%
\begin{pgfscope}%
\pgfsys@transformshift{3.684662in}{0.977354in}%
\pgfsys@useobject{currentmarker}{}%
\end{pgfscope}%
\begin{pgfscope}%
\pgfsys@transformshift{3.742471in}{0.961447in}%
\pgfsys@useobject{currentmarker}{}%
\end{pgfscope}%
\begin{pgfscope}%
\pgfsys@transformshift{3.800280in}{0.948577in}%
\pgfsys@useobject{currentmarker}{}%
\end{pgfscope}%
\end{pgfscope}%
\begin{pgfscope}%
\pgfpathrectangle{\pgfqpoint{0.800000in}{0.700000in}}{\pgfqpoint{4.960000in}{2.373913in}}%
\pgfusepath{clip}%
\pgfsetrectcap%
\pgfsetroundjoin%
\pgfsetlinewidth{1.505625pt}%
\definecolor{currentstroke}{rgb}{0.172549,0.627451,0.172549}%
\pgfsetstrokecolor{currentstroke}%
\pgfsetdash{}{0pt}%
\pgfpathmoveto{\pgfqpoint{1.025455in}{2.252980in}}%
\pgfpathlineto{\pgfqpoint{1.083263in}{2.386202in}}%
\pgfpathlineto{\pgfqpoint{1.141072in}{2.382698in}}%
\pgfpathlineto{\pgfqpoint{1.198881in}{2.138708in}}%
\pgfpathlineto{\pgfqpoint{1.256690in}{2.077584in}}%
\pgfpathlineto{\pgfqpoint{1.314499in}{1.802952in}}%
\pgfpathlineto{\pgfqpoint{1.372308in}{1.625772in}}%
\pgfpathlineto{\pgfqpoint{1.430117in}{1.461148in}}%
\pgfpathlineto{\pgfqpoint{1.487925in}{1.358481in}}%
\pgfpathlineto{\pgfqpoint{1.545734in}{1.305309in}}%
\pgfpathlineto{\pgfqpoint{1.603543in}{1.282278in}}%
\pgfpathlineto{\pgfqpoint{1.661352in}{1.260016in}}%
\pgfpathlineto{\pgfqpoint{1.719161in}{1.249829in}}%
\pgfpathlineto{\pgfqpoint{1.776970in}{1.244653in}}%
\pgfpathlineto{\pgfqpoint{1.834779in}{1.251476in}}%
\pgfpathlineto{\pgfqpoint{1.892587in}{1.261595in}}%
\pgfpathlineto{\pgfqpoint{1.950396in}{1.263171in}}%
\pgfpathlineto{\pgfqpoint{2.008205in}{1.270146in}}%
\pgfpathlineto{\pgfqpoint{2.066014in}{1.273729in}}%
\pgfpathlineto{\pgfqpoint{2.123823in}{1.269394in}}%
\pgfpathlineto{\pgfqpoint{2.181632in}{1.271141in}}%
\pgfpathlineto{\pgfqpoint{2.239441in}{1.268274in}}%
\pgfpathlineto{\pgfqpoint{2.297249in}{1.262944in}}%
\pgfpathlineto{\pgfqpoint{2.355058in}{1.262621in}}%
\pgfpathlineto{\pgfqpoint{2.412867in}{1.265140in}}%
\pgfpathlineto{\pgfqpoint{2.470676in}{1.266671in}}%
\pgfpathlineto{\pgfqpoint{2.528485in}{1.271267in}}%
\pgfpathlineto{\pgfqpoint{2.586294in}{1.276800in}}%
\pgfpathlineto{\pgfqpoint{2.644103in}{1.278033in}}%
\pgfpathlineto{\pgfqpoint{2.701911in}{1.274965in}}%
\pgfpathlineto{\pgfqpoint{2.759720in}{1.270254in}}%
\pgfpathlineto{\pgfqpoint{2.817529in}{1.264909in}}%
\pgfpathlineto{\pgfqpoint{2.875338in}{1.259122in}}%
\pgfpathlineto{\pgfqpoint{2.933147in}{1.253642in}}%
\pgfpathlineto{\pgfqpoint{2.990956in}{1.246434in}}%
\pgfpathlineto{\pgfqpoint{3.048765in}{1.241262in}}%
\pgfpathlineto{\pgfqpoint{3.106573in}{1.233193in}}%
\pgfpathlineto{\pgfqpoint{3.164382in}{1.222384in}}%
\pgfpathlineto{\pgfqpoint{3.222191in}{1.211026in}}%
\pgfpathlineto{\pgfqpoint{3.280000in}{1.199969in}}%
\pgfpathlineto{\pgfqpoint{3.337809in}{1.189423in}}%
\pgfpathlineto{\pgfqpoint{3.395618in}{1.182667in}}%
\pgfpathlineto{\pgfqpoint{3.453427in}{1.176398in}}%
\pgfpathlineto{\pgfqpoint{3.511235in}{1.172693in}}%
\pgfpathlineto{\pgfqpoint{3.569044in}{1.166285in}}%
\pgfpathlineto{\pgfqpoint{3.626853in}{1.159763in}}%
\pgfpathlineto{\pgfqpoint{3.684662in}{1.154310in}}%
\pgfpathlineto{\pgfqpoint{3.742471in}{1.149359in}}%
\pgfpathlineto{\pgfqpoint{3.800280in}{1.141878in}}%
\pgfpathlineto{\pgfqpoint{3.858089in}{1.131012in}}%
\pgfpathlineto{\pgfqpoint{3.915897in}{1.122778in}}%
\pgfpathlineto{\pgfqpoint{3.973706in}{1.115415in}}%
\pgfpathlineto{\pgfqpoint{4.031515in}{1.106160in}}%
\pgfpathlineto{\pgfqpoint{4.089324in}{1.098853in}}%
\pgfpathlineto{\pgfqpoint{4.147133in}{1.091347in}}%
\pgfpathlineto{\pgfqpoint{4.204942in}{1.081164in}}%
\pgfpathlineto{\pgfqpoint{4.262751in}{1.068925in}}%
\pgfpathlineto{\pgfqpoint{4.320559in}{1.055579in}}%
\pgfpathlineto{\pgfqpoint{4.378368in}{1.043266in}}%
\pgfpathlineto{\pgfqpoint{4.436177in}{1.034119in}}%
\pgfpathlineto{\pgfqpoint{4.493986in}{1.024987in}}%
\pgfpathlineto{\pgfqpoint{4.551795in}{1.013584in}}%
\pgfpathlineto{\pgfqpoint{4.609604in}{1.004426in}}%
\pgfpathlineto{\pgfqpoint{4.667413in}{0.995692in}}%
\pgfpathlineto{\pgfqpoint{4.725221in}{0.985759in}}%
\pgfpathlineto{\pgfqpoint{4.783030in}{0.975858in}}%
\pgfpathlineto{\pgfqpoint{4.840839in}{0.966515in}}%
\pgfpathlineto{\pgfqpoint{4.898648in}{0.961306in}}%
\pgfpathlineto{\pgfqpoint{4.956457in}{0.949814in}}%
\pgfpathlineto{\pgfqpoint{5.014266in}{0.940293in}}%
\pgfpathlineto{\pgfqpoint{5.072075in}{0.929155in}}%
\pgfpathlineto{\pgfqpoint{5.129883in}{0.920053in}}%
\pgfpathlineto{\pgfqpoint{5.187692in}{0.910908in}}%
\pgfpathlineto{\pgfqpoint{5.245501in}{0.893452in}}%
\pgfpathlineto{\pgfqpoint{5.303310in}{0.883615in}}%
\pgfpathlineto{\pgfqpoint{5.361119in}{0.875169in}}%
\pgfpathlineto{\pgfqpoint{5.418928in}{0.865929in}}%
\pgfpathlineto{\pgfqpoint{5.476737in}{0.855829in}}%
\pgfpathlineto{\pgfqpoint{5.534545in}{0.842863in}}%
\pgfusepath{stroke}%
\end{pgfscope}%
\begin{pgfscope}%
\pgfpathrectangle{\pgfqpoint{0.800000in}{0.700000in}}{\pgfqpoint{4.960000in}{2.373913in}}%
\pgfusepath{clip}%
\pgfsetrectcap%
\pgfsetroundjoin%
\pgfsetlinewidth{1.505625pt}%
\definecolor{currentstroke}{rgb}{0.172549,0.627451,0.172549}%
\pgfsetstrokecolor{currentstroke}%
\pgfsetdash{}{0pt}%
\pgfpathmoveto{\pgfqpoint{1.025455in}{2.226252in}}%
\pgfpathlineto{\pgfqpoint{1.083263in}{2.359474in}}%
\pgfpathlineto{\pgfqpoint{1.141072in}{2.355970in}}%
\pgfpathlineto{\pgfqpoint{1.198881in}{2.111980in}}%
\pgfpathlineto{\pgfqpoint{1.256690in}{2.050856in}}%
\pgfpathlineto{\pgfqpoint{1.314499in}{1.776224in}}%
\pgfpathlineto{\pgfqpoint{1.372308in}{1.599044in}}%
\pgfpathlineto{\pgfqpoint{1.430117in}{1.434420in}}%
\pgfpathlineto{\pgfqpoint{1.487925in}{1.331753in}}%
\pgfpathlineto{\pgfqpoint{1.545734in}{1.278581in}}%
\pgfpathlineto{\pgfqpoint{1.603543in}{1.255550in}}%
\pgfpathlineto{\pgfqpoint{1.661352in}{1.233288in}}%
\pgfpathlineto{\pgfqpoint{1.719161in}{1.223101in}}%
\pgfpathlineto{\pgfqpoint{1.776970in}{1.217925in}}%
\pgfpathlineto{\pgfqpoint{1.834779in}{1.224748in}}%
\pgfpathlineto{\pgfqpoint{1.892587in}{1.234867in}}%
\pgfpathlineto{\pgfqpoint{1.950396in}{1.236443in}}%
\pgfpathlineto{\pgfqpoint{2.008205in}{1.243417in}}%
\pgfpathlineto{\pgfqpoint{2.066014in}{1.247001in}}%
\pgfpathlineto{\pgfqpoint{2.123823in}{1.242666in}}%
\pgfpathlineto{\pgfqpoint{2.181632in}{1.244413in}}%
\pgfpathlineto{\pgfqpoint{2.239441in}{1.241546in}}%
\pgfpathlineto{\pgfqpoint{2.297249in}{1.236215in}}%
\pgfpathlineto{\pgfqpoint{2.355058in}{1.235893in}}%
\pgfpathlineto{\pgfqpoint{2.412867in}{1.238412in}}%
\pgfpathlineto{\pgfqpoint{2.470676in}{1.239943in}}%
\pgfpathlineto{\pgfqpoint{2.528485in}{1.244538in}}%
\pgfpathlineto{\pgfqpoint{2.586294in}{1.250072in}}%
\pgfpathlineto{\pgfqpoint{2.644103in}{1.251305in}}%
\pgfpathlineto{\pgfqpoint{2.701911in}{1.248237in}}%
\pgfpathlineto{\pgfqpoint{2.759720in}{1.243526in}}%
\pgfpathlineto{\pgfqpoint{2.817529in}{1.238181in}}%
\pgfpathlineto{\pgfqpoint{2.875338in}{1.232393in}}%
\pgfpathlineto{\pgfqpoint{2.933147in}{1.226914in}}%
\pgfpathlineto{\pgfqpoint{2.990956in}{1.219706in}}%
\pgfpathlineto{\pgfqpoint{3.048765in}{1.214534in}}%
\pgfpathlineto{\pgfqpoint{3.106573in}{1.206465in}}%
\pgfpathlineto{\pgfqpoint{3.164382in}{1.195655in}}%
\pgfpathlineto{\pgfqpoint{3.222191in}{1.184298in}}%
\pgfpathlineto{\pgfqpoint{3.280000in}{1.173240in}}%
\pgfpathlineto{\pgfqpoint{3.337809in}{1.162695in}}%
\pgfpathlineto{\pgfqpoint{3.395618in}{1.155938in}}%
\pgfpathlineto{\pgfqpoint{3.453427in}{1.149670in}}%
\pgfpathlineto{\pgfqpoint{3.511235in}{1.145965in}}%
\pgfpathlineto{\pgfqpoint{3.569044in}{1.139556in}}%
\pgfpathlineto{\pgfqpoint{3.626853in}{1.133035in}}%
\pgfpathlineto{\pgfqpoint{3.684662in}{1.127582in}}%
\pgfpathlineto{\pgfqpoint{3.742471in}{1.122630in}}%
\pgfpathlineto{\pgfqpoint{3.800280in}{1.115150in}}%
\pgfpathlineto{\pgfqpoint{3.858089in}{1.104284in}}%
\pgfpathlineto{\pgfqpoint{3.915897in}{1.096050in}}%
\pgfpathlineto{\pgfqpoint{3.973706in}{1.088687in}}%
\pgfpathlineto{\pgfqpoint{4.031515in}{1.079431in}}%
\pgfpathlineto{\pgfqpoint{4.089324in}{1.072125in}}%
\pgfpathlineto{\pgfqpoint{4.147133in}{1.064619in}}%
\pgfpathlineto{\pgfqpoint{4.204942in}{1.054435in}}%
\pgfpathlineto{\pgfqpoint{4.262751in}{1.042197in}}%
\pgfpathlineto{\pgfqpoint{4.320559in}{1.028851in}}%
\pgfpathlineto{\pgfqpoint{4.378368in}{1.016538in}}%
\pgfpathlineto{\pgfqpoint{4.436177in}{1.007391in}}%
\pgfpathlineto{\pgfqpoint{4.493986in}{0.998259in}}%
\pgfpathlineto{\pgfqpoint{4.551795in}{0.986856in}}%
\pgfpathlineto{\pgfqpoint{4.609604in}{0.977698in}}%
\pgfpathlineto{\pgfqpoint{4.667413in}{0.968963in}}%
\pgfpathlineto{\pgfqpoint{4.725221in}{0.959031in}}%
\pgfpathlineto{\pgfqpoint{4.783030in}{0.949130in}}%
\pgfpathlineto{\pgfqpoint{4.840839in}{0.939786in}}%
\pgfpathlineto{\pgfqpoint{4.898648in}{0.934578in}}%
\pgfpathlineto{\pgfqpoint{4.956457in}{0.923086in}}%
\pgfpathlineto{\pgfqpoint{5.014266in}{0.913565in}}%
\pgfpathlineto{\pgfqpoint{5.072075in}{0.902427in}}%
\pgfpathlineto{\pgfqpoint{5.129883in}{0.893324in}}%
\pgfpathlineto{\pgfqpoint{5.187692in}{0.884180in}}%
\pgfpathlineto{\pgfqpoint{5.245501in}{0.866724in}}%
\pgfpathlineto{\pgfqpoint{5.303310in}{0.856886in}}%
\pgfpathlineto{\pgfqpoint{5.361119in}{0.848441in}}%
\pgfpathlineto{\pgfqpoint{5.418928in}{0.839201in}}%
\pgfpathlineto{\pgfqpoint{5.476737in}{0.829101in}}%
\pgfpathlineto{\pgfqpoint{5.534545in}{0.816135in}}%
\pgfusepath{stroke}%
\end{pgfscope}%
\begin{pgfscope}%
\pgfpathrectangle{\pgfqpoint{0.800000in}{0.700000in}}{\pgfqpoint{4.960000in}{2.373913in}}%
\pgfusepath{clip}%
\pgfsetbuttcap%
\pgfsetroundjoin%
\pgfsetlinewidth{1.505625pt}%
\definecolor{currentstroke}{rgb}{0.000000,0.000000,0.000000}%
\pgfsetstrokecolor{currentstroke}%
\pgfsetdash{}{0pt}%
\pgfpathmoveto{\pgfqpoint{3.015949in}{0.940139in}}%
\pgfpathlineto{\pgfqpoint{3.015949in}{1.474702in}}%
\pgfusepath{stroke}%
\end{pgfscope}%
\begin{pgfscope}%
\pgfpathrectangle{\pgfqpoint{0.800000in}{0.700000in}}{\pgfqpoint{4.960000in}{2.373913in}}%
\pgfusepath{clip}%
\pgfsetrectcap%
\pgfsetroundjoin%
\pgfsetlinewidth{1.505625pt}%
\definecolor{currentstroke}{rgb}{0.121569,0.466667,0.705882}%
\pgfsetstrokecolor{currentstroke}%
\pgfsetdash{}{0pt}%
\pgfpathmoveto{\pgfqpoint{1.025455in}{2.239616in}}%
\pgfpathlineto{\pgfqpoint{1.083263in}{2.372838in}}%
\pgfpathlineto{\pgfqpoint{1.141072in}{2.369334in}}%
\pgfpathlineto{\pgfqpoint{1.198881in}{2.125344in}}%
\pgfpathlineto{\pgfqpoint{1.256690in}{2.064220in}}%
\pgfpathlineto{\pgfqpoint{1.314499in}{1.789588in}}%
\pgfpathlineto{\pgfqpoint{1.372308in}{1.612408in}}%
\pgfpathlineto{\pgfqpoint{1.430117in}{1.447784in}}%
\pgfpathlineto{\pgfqpoint{1.487925in}{1.345117in}}%
\pgfpathlineto{\pgfqpoint{1.545734in}{1.291945in}}%
\pgfpathlineto{\pgfqpoint{1.603543in}{1.268914in}}%
\pgfpathlineto{\pgfqpoint{1.661352in}{1.246652in}}%
\pgfpathlineto{\pgfqpoint{1.719161in}{1.236465in}}%
\pgfpathlineto{\pgfqpoint{1.776970in}{1.231289in}}%
\pgfpathlineto{\pgfqpoint{1.834779in}{1.238112in}}%
\pgfpathlineto{\pgfqpoint{1.892587in}{1.248231in}}%
\pgfpathlineto{\pgfqpoint{1.950396in}{1.249807in}}%
\pgfpathlineto{\pgfqpoint{2.008205in}{1.256781in}}%
\pgfpathlineto{\pgfqpoint{2.066014in}{1.260365in}}%
\pgfpathlineto{\pgfqpoint{2.123823in}{1.256030in}}%
\pgfpathlineto{\pgfqpoint{2.181632in}{1.257777in}}%
\pgfpathlineto{\pgfqpoint{2.239441in}{1.254910in}}%
\pgfpathlineto{\pgfqpoint{2.297249in}{1.249580in}}%
\pgfpathlineto{\pgfqpoint{2.355058in}{1.249257in}}%
\pgfpathlineto{\pgfqpoint{2.412867in}{1.251776in}}%
\pgfpathlineto{\pgfqpoint{2.470676in}{1.253307in}}%
\pgfpathlineto{\pgfqpoint{2.528485in}{1.257902in}}%
\pgfpathlineto{\pgfqpoint{2.586294in}{1.263436in}}%
\pgfpathlineto{\pgfqpoint{2.644103in}{1.264669in}}%
\pgfpathlineto{\pgfqpoint{2.701911in}{1.261601in}}%
\pgfpathlineto{\pgfqpoint{2.759720in}{1.256890in}}%
\pgfpathlineto{\pgfqpoint{2.817529in}{1.251545in}}%
\pgfpathlineto{\pgfqpoint{2.875338in}{1.245757in}}%
\pgfpathlineto{\pgfqpoint{2.933147in}{1.240278in}}%
\pgfpathlineto{\pgfqpoint{2.990956in}{1.233070in}}%
\pgfpathlineto{\pgfqpoint{3.048765in}{1.227898in}}%
\pgfpathlineto{\pgfqpoint{3.106573in}{1.219829in}}%
\pgfpathlineto{\pgfqpoint{3.164382in}{1.209019in}}%
\pgfpathlineto{\pgfqpoint{3.222191in}{1.197662in}}%
\pgfpathlineto{\pgfqpoint{3.280000in}{1.186605in}}%
\pgfpathlineto{\pgfqpoint{3.337809in}{1.176059in}}%
\pgfpathlineto{\pgfqpoint{3.395618in}{1.169303in}}%
\pgfpathlineto{\pgfqpoint{3.453427in}{1.163034in}}%
\pgfpathlineto{\pgfqpoint{3.511235in}{1.159329in}}%
\pgfpathlineto{\pgfqpoint{3.569044in}{1.152920in}}%
\pgfpathlineto{\pgfqpoint{3.626853in}{1.146399in}}%
\pgfpathlineto{\pgfqpoint{3.684662in}{1.140946in}}%
\pgfpathlineto{\pgfqpoint{3.742471in}{1.135995in}}%
\pgfpathlineto{\pgfqpoint{3.800280in}{1.128514in}}%
\pgfpathlineto{\pgfqpoint{3.858089in}{1.117648in}}%
\pgfpathlineto{\pgfqpoint{3.915897in}{1.109414in}}%
\pgfpathlineto{\pgfqpoint{3.973706in}{1.102051in}}%
\pgfpathlineto{\pgfqpoint{4.031515in}{1.092796in}}%
\pgfpathlineto{\pgfqpoint{4.089324in}{1.085489in}}%
\pgfpathlineto{\pgfqpoint{4.147133in}{1.077983in}}%
\pgfpathlineto{\pgfqpoint{4.204942in}{1.067800in}}%
\pgfpathlineto{\pgfqpoint{4.262751in}{1.055561in}}%
\pgfpathlineto{\pgfqpoint{4.320559in}{1.042215in}}%
\pgfpathlineto{\pgfqpoint{4.378368in}{1.029902in}}%
\pgfpathlineto{\pgfqpoint{4.436177in}{1.020755in}}%
\pgfpathlineto{\pgfqpoint{4.493986in}{1.011623in}}%
\pgfpathlineto{\pgfqpoint{4.551795in}{1.000220in}}%
\pgfpathlineto{\pgfqpoint{4.609604in}{0.991062in}}%
\pgfpathlineto{\pgfqpoint{4.667413in}{0.982327in}}%
\pgfpathlineto{\pgfqpoint{4.725221in}{0.972395in}}%
\pgfpathlineto{\pgfqpoint{4.783030in}{0.962494in}}%
\pgfpathlineto{\pgfqpoint{4.840839in}{0.953151in}}%
\pgfpathlineto{\pgfqpoint{4.898648in}{0.947942in}}%
\pgfpathlineto{\pgfqpoint{4.956457in}{0.936450in}}%
\pgfpathlineto{\pgfqpoint{5.014266in}{0.926929in}}%
\pgfpathlineto{\pgfqpoint{5.072075in}{0.915791in}}%
\pgfpathlineto{\pgfqpoint{5.129883in}{0.906689in}}%
\pgfpathlineto{\pgfqpoint{5.187692in}{0.897544in}}%
\pgfpathlineto{\pgfqpoint{5.245501in}{0.880088in}}%
\pgfpathlineto{\pgfqpoint{5.303310in}{0.870250in}}%
\pgfpathlineto{\pgfqpoint{5.361119in}{0.861805in}}%
\pgfpathlineto{\pgfqpoint{5.418928in}{0.852565in}}%
\pgfpathlineto{\pgfqpoint{5.476737in}{0.842465in}}%
\pgfpathlineto{\pgfqpoint{5.534545in}{0.829499in}}%
\pgfusepath{stroke}%
\end{pgfscope}%
\begin{pgfscope}%
\pgfpathrectangle{\pgfqpoint{0.800000in}{0.700000in}}{\pgfqpoint{4.960000in}{2.373913in}}%
\pgfusepath{clip}%
\pgfsetbuttcap%
\pgfsetroundjoin%
\definecolor{currentfill}{rgb}{0.121569,0.466667,0.705882}%
\pgfsetfillcolor{currentfill}%
\pgfsetlinewidth{1.003750pt}%
\definecolor{currentstroke}{rgb}{0.121569,0.466667,0.705882}%
\pgfsetstrokecolor{currentstroke}%
\pgfsetdash{}{0pt}%
\pgfsys@defobject{currentmarker}{\pgfqpoint{-0.020833in}{-0.020833in}}{\pgfqpoint{0.020833in}{0.020833in}}{%
\pgfpathmoveto{\pgfqpoint{0.000000in}{-0.020833in}}%
\pgfpathcurveto{\pgfqpoint{0.005525in}{-0.020833in}}{\pgfqpoint{0.010825in}{-0.018638in}}{\pgfqpoint{0.014731in}{-0.014731in}}%
\pgfpathcurveto{\pgfqpoint{0.018638in}{-0.010825in}}{\pgfqpoint{0.020833in}{-0.005525in}}{\pgfqpoint{0.020833in}{0.000000in}}%
\pgfpathcurveto{\pgfqpoint{0.020833in}{0.005525in}}{\pgfqpoint{0.018638in}{0.010825in}}{\pgfqpoint{0.014731in}{0.014731in}}%
\pgfpathcurveto{\pgfqpoint{0.010825in}{0.018638in}}{\pgfqpoint{0.005525in}{0.020833in}}{\pgfqpoint{0.000000in}{0.020833in}}%
\pgfpathcurveto{\pgfqpoint{-0.005525in}{0.020833in}}{\pgfqpoint{-0.010825in}{0.018638in}}{\pgfqpoint{-0.014731in}{0.014731in}}%
\pgfpathcurveto{\pgfqpoint{-0.018638in}{0.010825in}}{\pgfqpoint{-0.020833in}{0.005525in}}{\pgfqpoint{-0.020833in}{0.000000in}}%
\pgfpathcurveto{\pgfqpoint{-0.020833in}{-0.005525in}}{\pgfqpoint{-0.018638in}{-0.010825in}}{\pgfqpoint{-0.014731in}{-0.014731in}}%
\pgfpathcurveto{\pgfqpoint{-0.010825in}{-0.018638in}}{\pgfqpoint{-0.005525in}{-0.020833in}}{\pgfqpoint{0.000000in}{-0.020833in}}%
\pgfpathlineto{\pgfqpoint{0.000000in}{-0.020833in}}%
\pgfpathclose%
\pgfusepath{stroke,fill}%
}%
\begin{pgfscope}%
\pgfsys@transformshift{1.025455in}{2.239616in}%
\pgfsys@useobject{currentmarker}{}%
\end{pgfscope}%
\begin{pgfscope}%
\pgfsys@transformshift{1.083263in}{2.372838in}%
\pgfsys@useobject{currentmarker}{}%
\end{pgfscope}%
\begin{pgfscope}%
\pgfsys@transformshift{1.141072in}{2.369334in}%
\pgfsys@useobject{currentmarker}{}%
\end{pgfscope}%
\begin{pgfscope}%
\pgfsys@transformshift{1.198881in}{2.125344in}%
\pgfsys@useobject{currentmarker}{}%
\end{pgfscope}%
\begin{pgfscope}%
\pgfsys@transformshift{1.256690in}{2.064220in}%
\pgfsys@useobject{currentmarker}{}%
\end{pgfscope}%
\begin{pgfscope}%
\pgfsys@transformshift{1.314499in}{1.789588in}%
\pgfsys@useobject{currentmarker}{}%
\end{pgfscope}%
\begin{pgfscope}%
\pgfsys@transformshift{1.372308in}{1.612408in}%
\pgfsys@useobject{currentmarker}{}%
\end{pgfscope}%
\begin{pgfscope}%
\pgfsys@transformshift{1.430117in}{1.447784in}%
\pgfsys@useobject{currentmarker}{}%
\end{pgfscope}%
\begin{pgfscope}%
\pgfsys@transformshift{1.487925in}{1.345117in}%
\pgfsys@useobject{currentmarker}{}%
\end{pgfscope}%
\begin{pgfscope}%
\pgfsys@transformshift{1.545734in}{1.291945in}%
\pgfsys@useobject{currentmarker}{}%
\end{pgfscope}%
\begin{pgfscope}%
\pgfsys@transformshift{1.603543in}{1.268914in}%
\pgfsys@useobject{currentmarker}{}%
\end{pgfscope}%
\begin{pgfscope}%
\pgfsys@transformshift{1.661352in}{1.246652in}%
\pgfsys@useobject{currentmarker}{}%
\end{pgfscope}%
\begin{pgfscope}%
\pgfsys@transformshift{1.719161in}{1.236465in}%
\pgfsys@useobject{currentmarker}{}%
\end{pgfscope}%
\begin{pgfscope}%
\pgfsys@transformshift{1.776970in}{1.231289in}%
\pgfsys@useobject{currentmarker}{}%
\end{pgfscope}%
\begin{pgfscope}%
\pgfsys@transformshift{1.834779in}{1.238112in}%
\pgfsys@useobject{currentmarker}{}%
\end{pgfscope}%
\begin{pgfscope}%
\pgfsys@transformshift{1.892587in}{1.248231in}%
\pgfsys@useobject{currentmarker}{}%
\end{pgfscope}%
\begin{pgfscope}%
\pgfsys@transformshift{1.950396in}{1.249807in}%
\pgfsys@useobject{currentmarker}{}%
\end{pgfscope}%
\begin{pgfscope}%
\pgfsys@transformshift{2.008205in}{1.256781in}%
\pgfsys@useobject{currentmarker}{}%
\end{pgfscope}%
\begin{pgfscope}%
\pgfsys@transformshift{2.066014in}{1.260365in}%
\pgfsys@useobject{currentmarker}{}%
\end{pgfscope}%
\begin{pgfscope}%
\pgfsys@transformshift{2.123823in}{1.256030in}%
\pgfsys@useobject{currentmarker}{}%
\end{pgfscope}%
\begin{pgfscope}%
\pgfsys@transformshift{2.181632in}{1.257777in}%
\pgfsys@useobject{currentmarker}{}%
\end{pgfscope}%
\begin{pgfscope}%
\pgfsys@transformshift{2.239441in}{1.254910in}%
\pgfsys@useobject{currentmarker}{}%
\end{pgfscope}%
\begin{pgfscope}%
\pgfsys@transformshift{2.297249in}{1.249580in}%
\pgfsys@useobject{currentmarker}{}%
\end{pgfscope}%
\begin{pgfscope}%
\pgfsys@transformshift{2.355058in}{1.249257in}%
\pgfsys@useobject{currentmarker}{}%
\end{pgfscope}%
\begin{pgfscope}%
\pgfsys@transformshift{2.412867in}{1.251776in}%
\pgfsys@useobject{currentmarker}{}%
\end{pgfscope}%
\begin{pgfscope}%
\pgfsys@transformshift{2.470676in}{1.253307in}%
\pgfsys@useobject{currentmarker}{}%
\end{pgfscope}%
\begin{pgfscope}%
\pgfsys@transformshift{2.528485in}{1.257902in}%
\pgfsys@useobject{currentmarker}{}%
\end{pgfscope}%
\begin{pgfscope}%
\pgfsys@transformshift{2.586294in}{1.263436in}%
\pgfsys@useobject{currentmarker}{}%
\end{pgfscope}%
\begin{pgfscope}%
\pgfsys@transformshift{2.644103in}{1.264669in}%
\pgfsys@useobject{currentmarker}{}%
\end{pgfscope}%
\begin{pgfscope}%
\pgfsys@transformshift{2.701911in}{1.261601in}%
\pgfsys@useobject{currentmarker}{}%
\end{pgfscope}%
\begin{pgfscope}%
\pgfsys@transformshift{2.759720in}{1.256890in}%
\pgfsys@useobject{currentmarker}{}%
\end{pgfscope}%
\begin{pgfscope}%
\pgfsys@transformshift{2.817529in}{1.251545in}%
\pgfsys@useobject{currentmarker}{}%
\end{pgfscope}%
\begin{pgfscope}%
\pgfsys@transformshift{2.875338in}{1.245757in}%
\pgfsys@useobject{currentmarker}{}%
\end{pgfscope}%
\begin{pgfscope}%
\pgfsys@transformshift{2.933147in}{1.240278in}%
\pgfsys@useobject{currentmarker}{}%
\end{pgfscope}%
\begin{pgfscope}%
\pgfsys@transformshift{2.990956in}{1.233070in}%
\pgfsys@useobject{currentmarker}{}%
\end{pgfscope}%
\begin{pgfscope}%
\pgfsys@transformshift{3.048765in}{1.227898in}%
\pgfsys@useobject{currentmarker}{}%
\end{pgfscope}%
\begin{pgfscope}%
\pgfsys@transformshift{3.106573in}{1.219829in}%
\pgfsys@useobject{currentmarker}{}%
\end{pgfscope}%
\begin{pgfscope}%
\pgfsys@transformshift{3.164382in}{1.209019in}%
\pgfsys@useobject{currentmarker}{}%
\end{pgfscope}%
\begin{pgfscope}%
\pgfsys@transformshift{3.222191in}{1.197662in}%
\pgfsys@useobject{currentmarker}{}%
\end{pgfscope}%
\begin{pgfscope}%
\pgfsys@transformshift{3.280000in}{1.186605in}%
\pgfsys@useobject{currentmarker}{}%
\end{pgfscope}%
\begin{pgfscope}%
\pgfsys@transformshift{3.337809in}{1.176059in}%
\pgfsys@useobject{currentmarker}{}%
\end{pgfscope}%
\begin{pgfscope}%
\pgfsys@transformshift{3.395618in}{1.169303in}%
\pgfsys@useobject{currentmarker}{}%
\end{pgfscope}%
\begin{pgfscope}%
\pgfsys@transformshift{3.453427in}{1.163034in}%
\pgfsys@useobject{currentmarker}{}%
\end{pgfscope}%
\begin{pgfscope}%
\pgfsys@transformshift{3.511235in}{1.159329in}%
\pgfsys@useobject{currentmarker}{}%
\end{pgfscope}%
\begin{pgfscope}%
\pgfsys@transformshift{3.569044in}{1.152920in}%
\pgfsys@useobject{currentmarker}{}%
\end{pgfscope}%
\begin{pgfscope}%
\pgfsys@transformshift{3.626853in}{1.146399in}%
\pgfsys@useobject{currentmarker}{}%
\end{pgfscope}%
\begin{pgfscope}%
\pgfsys@transformshift{3.684662in}{1.140946in}%
\pgfsys@useobject{currentmarker}{}%
\end{pgfscope}%
\begin{pgfscope}%
\pgfsys@transformshift{3.742471in}{1.135995in}%
\pgfsys@useobject{currentmarker}{}%
\end{pgfscope}%
\begin{pgfscope}%
\pgfsys@transformshift{3.800280in}{1.128514in}%
\pgfsys@useobject{currentmarker}{}%
\end{pgfscope}%
\begin{pgfscope}%
\pgfsys@transformshift{3.858089in}{1.117648in}%
\pgfsys@useobject{currentmarker}{}%
\end{pgfscope}%
\begin{pgfscope}%
\pgfsys@transformshift{3.915897in}{1.109414in}%
\pgfsys@useobject{currentmarker}{}%
\end{pgfscope}%
\begin{pgfscope}%
\pgfsys@transformshift{3.973706in}{1.102051in}%
\pgfsys@useobject{currentmarker}{}%
\end{pgfscope}%
\begin{pgfscope}%
\pgfsys@transformshift{4.031515in}{1.092796in}%
\pgfsys@useobject{currentmarker}{}%
\end{pgfscope}%
\begin{pgfscope}%
\pgfsys@transformshift{4.089324in}{1.085489in}%
\pgfsys@useobject{currentmarker}{}%
\end{pgfscope}%
\begin{pgfscope}%
\pgfsys@transformshift{4.147133in}{1.077983in}%
\pgfsys@useobject{currentmarker}{}%
\end{pgfscope}%
\begin{pgfscope}%
\pgfsys@transformshift{4.204942in}{1.067800in}%
\pgfsys@useobject{currentmarker}{}%
\end{pgfscope}%
\begin{pgfscope}%
\pgfsys@transformshift{4.262751in}{1.055561in}%
\pgfsys@useobject{currentmarker}{}%
\end{pgfscope}%
\begin{pgfscope}%
\pgfsys@transformshift{4.320559in}{1.042215in}%
\pgfsys@useobject{currentmarker}{}%
\end{pgfscope}%
\begin{pgfscope}%
\pgfsys@transformshift{4.378368in}{1.029902in}%
\pgfsys@useobject{currentmarker}{}%
\end{pgfscope}%
\begin{pgfscope}%
\pgfsys@transformshift{4.436177in}{1.020755in}%
\pgfsys@useobject{currentmarker}{}%
\end{pgfscope}%
\begin{pgfscope}%
\pgfsys@transformshift{4.493986in}{1.011623in}%
\pgfsys@useobject{currentmarker}{}%
\end{pgfscope}%
\begin{pgfscope}%
\pgfsys@transformshift{4.551795in}{1.000220in}%
\pgfsys@useobject{currentmarker}{}%
\end{pgfscope}%
\begin{pgfscope}%
\pgfsys@transformshift{4.609604in}{0.991062in}%
\pgfsys@useobject{currentmarker}{}%
\end{pgfscope}%
\begin{pgfscope}%
\pgfsys@transformshift{4.667413in}{0.982327in}%
\pgfsys@useobject{currentmarker}{}%
\end{pgfscope}%
\begin{pgfscope}%
\pgfsys@transformshift{4.725221in}{0.972395in}%
\pgfsys@useobject{currentmarker}{}%
\end{pgfscope}%
\begin{pgfscope}%
\pgfsys@transformshift{4.783030in}{0.962494in}%
\pgfsys@useobject{currentmarker}{}%
\end{pgfscope}%
\begin{pgfscope}%
\pgfsys@transformshift{4.840839in}{0.953151in}%
\pgfsys@useobject{currentmarker}{}%
\end{pgfscope}%
\begin{pgfscope}%
\pgfsys@transformshift{4.898648in}{0.947942in}%
\pgfsys@useobject{currentmarker}{}%
\end{pgfscope}%
\begin{pgfscope}%
\pgfsys@transformshift{4.956457in}{0.936450in}%
\pgfsys@useobject{currentmarker}{}%
\end{pgfscope}%
\begin{pgfscope}%
\pgfsys@transformshift{5.014266in}{0.926929in}%
\pgfsys@useobject{currentmarker}{}%
\end{pgfscope}%
\begin{pgfscope}%
\pgfsys@transformshift{5.072075in}{0.915791in}%
\pgfsys@useobject{currentmarker}{}%
\end{pgfscope}%
\begin{pgfscope}%
\pgfsys@transformshift{5.129883in}{0.906689in}%
\pgfsys@useobject{currentmarker}{}%
\end{pgfscope}%
\begin{pgfscope}%
\pgfsys@transformshift{5.187692in}{0.897544in}%
\pgfsys@useobject{currentmarker}{}%
\end{pgfscope}%
\begin{pgfscope}%
\pgfsys@transformshift{5.245501in}{0.880088in}%
\pgfsys@useobject{currentmarker}{}%
\end{pgfscope}%
\begin{pgfscope}%
\pgfsys@transformshift{5.303310in}{0.870250in}%
\pgfsys@useobject{currentmarker}{}%
\end{pgfscope}%
\begin{pgfscope}%
\pgfsys@transformshift{5.361119in}{0.861805in}%
\pgfsys@useobject{currentmarker}{}%
\end{pgfscope}%
\begin{pgfscope}%
\pgfsys@transformshift{5.418928in}{0.852565in}%
\pgfsys@useobject{currentmarker}{}%
\end{pgfscope}%
\begin{pgfscope}%
\pgfsys@transformshift{5.476737in}{0.842465in}%
\pgfsys@useobject{currentmarker}{}%
\end{pgfscope}%
\begin{pgfscope}%
\pgfsys@transformshift{5.534545in}{0.829499in}%
\pgfsys@useobject{currentmarker}{}%
\end{pgfscope}%
\end{pgfscope}%
\begin{pgfscope}%
\pgfsetrectcap%
\pgfsetmiterjoin%
\pgfsetlinewidth{0.803000pt}%
\definecolor{currentstroke}{rgb}{0.000000,0.000000,0.000000}%
\pgfsetstrokecolor{currentstroke}%
\pgfsetdash{}{0pt}%
\pgfpathmoveto{\pgfqpoint{0.800000in}{0.700000in}}%
\pgfpathlineto{\pgfqpoint{0.800000in}{3.073913in}}%
\pgfusepath{stroke}%
\end{pgfscope}%
\begin{pgfscope}%
\pgfsetrectcap%
\pgfsetmiterjoin%
\pgfsetlinewidth{0.803000pt}%
\definecolor{currentstroke}{rgb}{0.000000,0.000000,0.000000}%
\pgfsetstrokecolor{currentstroke}%
\pgfsetdash{}{0pt}%
\pgfpathmoveto{\pgfqpoint{5.760000in}{0.700000in}}%
\pgfpathlineto{\pgfqpoint{5.760000in}{3.073913in}}%
\pgfusepath{stroke}%
\end{pgfscope}%
\begin{pgfscope}%
\pgfsetrectcap%
\pgfsetmiterjoin%
\pgfsetlinewidth{0.803000pt}%
\definecolor{currentstroke}{rgb}{0.000000,0.000000,0.000000}%
\pgfsetstrokecolor{currentstroke}%
\pgfsetdash{}{0pt}%
\pgfpathmoveto{\pgfqpoint{0.800000in}{0.700000in}}%
\pgfpathlineto{\pgfqpoint{5.760000in}{0.700000in}}%
\pgfusepath{stroke}%
\end{pgfscope}%
\begin{pgfscope}%
\pgfsetrectcap%
\pgfsetmiterjoin%
\pgfsetlinewidth{0.803000pt}%
\definecolor{currentstroke}{rgb}{0.000000,0.000000,0.000000}%
\pgfsetstrokecolor{currentstroke}%
\pgfsetdash{}{0pt}%
\pgfpathmoveto{\pgfqpoint{0.800000in}{3.073913in}}%
\pgfpathlineto{\pgfqpoint{5.760000in}{3.073913in}}%
\pgfusepath{stroke}%
\end{pgfscope}%
\begin{pgfscope}%
\definecolor{textcolor}{rgb}{0.000000,0.000000,0.000000}%
\pgfsetstrokecolor{textcolor}%
\pgfsetfillcolor{textcolor}%
\pgftext[x=2.659632in, y=1.849289in, left, base]{\color{textcolor}\sffamily\fontsize{10.000000}{12.000000}\selectfont end of plasma }%
\end{pgfscope}%
\begin{pgfscope}%
\definecolor{textcolor}{rgb}{0.000000,0.000000,0.000000}%
\pgfsetstrokecolor{textcolor}%
\pgfsetfillcolor{textcolor}%
\pgftext[x=2.549485in, y=1.706542in, left, base]{\color{textcolor}\sffamily\fontsize{10.000000}{12.000000}\selectfont  for the downramp }%
\end{pgfscope}%
\begin{pgfscope}%
\definecolor{textcolor}{rgb}{0.000000,0.000000,0.000000}%
\pgfsetstrokecolor{textcolor}%
\pgfsetfillcolor{textcolor}%
\pgftext[x=2.792155in, y=1.563795in, left, base]{\color{textcolor}\sffamily\fontsize{10.000000}{12.000000}\selectfont  simulation}%
\end{pgfscope}%
\begin{pgfscope}%
\definecolor{textcolor}{rgb}{0.000000,0.000000,0.000000}%
\pgfsetstrokecolor{textcolor}%
\pgfsetfillcolor{textcolor}%
\pgftext[x=3.280000in,y=3.157246in,,base]{\color{textcolor}\sffamily\fontsize{12.000000}{14.400000}\selectfont b)}%
\end{pgfscope}%
\begin{pgfscope}%
\pgfsetbuttcap%
\pgfsetmiterjoin%
\definecolor{currentfill}{rgb}{1.000000,1.000000,1.000000}%
\pgfsetfillcolor{currentfill}%
\pgfsetfillopacity{0.800000}%
\pgfsetlinewidth{1.003750pt}%
\definecolor{currentstroke}{rgb}{0.800000,0.800000,0.800000}%
\pgfsetstrokecolor{currentstroke}%
\pgfsetstrokeopacity{0.800000}%
\pgfsetdash{}{0pt}%
\pgfpathmoveto{\pgfqpoint{3.974504in}{2.352462in}}%
\pgfpathlineto{\pgfqpoint{5.662778in}{2.352462in}}%
\pgfpathquadraticcurveto{\pgfqpoint{5.690556in}{2.352462in}}{\pgfqpoint{5.690556in}{2.380240in}}%
\pgfpathlineto{\pgfqpoint{5.690556in}{2.976691in}}%
\pgfpathquadraticcurveto{\pgfqpoint{5.690556in}{3.004469in}}{\pgfqpoint{5.662778in}{3.004469in}}%
\pgfpathlineto{\pgfqpoint{3.974504in}{3.004469in}}%
\pgfpathquadraticcurveto{\pgfqpoint{3.946726in}{3.004469in}}{\pgfqpoint{3.946726in}{2.976691in}}%
\pgfpathlineto{\pgfqpoint{3.946726in}{2.380240in}}%
\pgfpathquadraticcurveto{\pgfqpoint{3.946726in}{2.352462in}}{\pgfqpoint{3.974504in}{2.352462in}}%
\pgfpathlineto{\pgfqpoint{3.974504in}{2.352462in}}%
\pgfpathclose%
\pgfusepath{stroke,fill}%
\end{pgfscope}%
\begin{pgfscope}%
\pgfsetrectcap%
\pgfsetroundjoin%
\pgfsetlinewidth{1.505625pt}%
\definecolor{currentstroke}{rgb}{1.000000,0.498039,0.054902}%
\pgfsetstrokecolor{currentstroke}%
\pgfsetdash{}{0pt}%
\pgfpathmoveto{\pgfqpoint{4.002282in}{2.893357in}}%
\pgfpathlineto{\pgfqpoint{4.141170in}{2.893357in}}%
\pgfpathlineto{\pgfqpoint{4.280059in}{2.893357in}}%
\pgfusepath{stroke}%
\end{pgfscope}%
\begin{pgfscope}%
\pgfsetbuttcap%
\pgfsetroundjoin%
\definecolor{currentfill}{rgb}{1.000000,0.498039,0.054902}%
\pgfsetfillcolor{currentfill}%
\pgfsetlinewidth{1.003750pt}%
\definecolor{currentstroke}{rgb}{1.000000,0.498039,0.054902}%
\pgfsetstrokecolor{currentstroke}%
\pgfsetdash{}{0pt}%
\pgfsys@defobject{currentmarker}{\pgfqpoint{-0.020833in}{-0.020833in}}{\pgfqpoint{0.020833in}{0.020833in}}{%
\pgfpathmoveto{\pgfqpoint{0.000000in}{-0.020833in}}%
\pgfpathcurveto{\pgfqpoint{0.005525in}{-0.020833in}}{\pgfqpoint{0.010825in}{-0.018638in}}{\pgfqpoint{0.014731in}{-0.014731in}}%
\pgfpathcurveto{\pgfqpoint{0.018638in}{-0.010825in}}{\pgfqpoint{0.020833in}{-0.005525in}}{\pgfqpoint{0.020833in}{0.000000in}}%
\pgfpathcurveto{\pgfqpoint{0.020833in}{0.005525in}}{\pgfqpoint{0.018638in}{0.010825in}}{\pgfqpoint{0.014731in}{0.014731in}}%
\pgfpathcurveto{\pgfqpoint{0.010825in}{0.018638in}}{\pgfqpoint{0.005525in}{0.020833in}}{\pgfqpoint{0.000000in}{0.020833in}}%
\pgfpathcurveto{\pgfqpoint{-0.005525in}{0.020833in}}{\pgfqpoint{-0.010825in}{0.018638in}}{\pgfqpoint{-0.014731in}{0.014731in}}%
\pgfpathcurveto{\pgfqpoint{-0.018638in}{0.010825in}}{\pgfqpoint{-0.020833in}{0.005525in}}{\pgfqpoint{-0.020833in}{0.000000in}}%
\pgfpathcurveto{\pgfqpoint{-0.020833in}{-0.005525in}}{\pgfqpoint{-0.018638in}{-0.010825in}}{\pgfqpoint{-0.014731in}{-0.014731in}}%
\pgfpathcurveto{\pgfqpoint{-0.010825in}{-0.018638in}}{\pgfqpoint{-0.005525in}{-0.020833in}}{\pgfqpoint{0.000000in}{-0.020833in}}%
\pgfpathlineto{\pgfqpoint{0.000000in}{-0.020833in}}%
\pgfpathclose%
\pgfusepath{stroke,fill}%
}%
\begin{pgfscope}%
\pgfsys@transformshift{4.141170in}{2.893357in}%
\pgfsys@useobject{currentmarker}{}%
\end{pgfscope}%
\end{pgfscope}%
\begin{pgfscope}%
\definecolor{textcolor}{rgb}{0.000000,0.000000,0.000000}%
\pgfsetstrokecolor{textcolor}%
\pgfsetfillcolor{textcolor}%
\pgftext[x=4.391170in,y=2.844746in,left,base]{\color{textcolor}\sffamily\fontsize{10.000000}{12.000000}\selectfont data (downramp)}%
\end{pgfscope}%
\begin{pgfscope}%
\pgfsetrectcap%
\pgfsetroundjoin%
\pgfsetlinewidth{1.505625pt}%
\definecolor{currentstroke}{rgb}{0.172549,0.627451,0.172549}%
\pgfsetstrokecolor{currentstroke}%
\pgfsetdash{}{0pt}%
\pgfpathmoveto{\pgfqpoint{4.002282in}{2.691969in}}%
\pgfpathlineto{\pgfqpoint{4.141170in}{2.691969in}}%
\pgfpathlineto{\pgfqpoint{4.280059in}{2.691969in}}%
\pgfusepath{stroke}%
\end{pgfscope}%
\begin{pgfscope}%
\definecolor{textcolor}{rgb}{0.000000,0.000000,0.000000}%
\pgfsetstrokecolor{textcolor}%
\pgfsetfillcolor{textcolor}%
\pgftext[x=4.391170in,y=2.643357in,left,base]{\color{textcolor}\sffamily\fontsize{10.000000}{12.000000}\selectfont sys. uncertainty}%
\end{pgfscope}%
\begin{pgfscope}%
\pgfsetbuttcap%
\pgfsetroundjoin%
\pgfsetlinewidth{1.505625pt}%
\definecolor{currentstroke}{rgb}{0.121569,0.466667,0.705882}%
\pgfsetstrokecolor{currentstroke}%
\pgfsetdash{}{0pt}%
\pgfpathmoveto{\pgfqpoint{4.141170in}{2.421907in}}%
\pgfpathlineto{\pgfqpoint{4.141170in}{2.560796in}}%
\pgfusepath{stroke}%
\end{pgfscope}%
\begin{pgfscope}%
\pgfsetbuttcap%
\pgfsetroundjoin%
\definecolor{currentfill}{rgb}{0.121569,0.466667,0.705882}%
\pgfsetfillcolor{currentfill}%
\pgfsetlinewidth{1.003750pt}%
\definecolor{currentstroke}{rgb}{0.121569,0.466667,0.705882}%
\pgfsetstrokecolor{currentstroke}%
\pgfsetdash{}{0pt}%
\pgfsys@defobject{currentmarker}{\pgfqpoint{-0.041667in}{-0.000000in}}{\pgfqpoint{0.041667in}{0.000000in}}{%
\pgfpathmoveto{\pgfqpoint{0.041667in}{-0.000000in}}%
\pgfpathlineto{\pgfqpoint{-0.041667in}{0.000000in}}%
\pgfusepath{stroke,fill}%
}%
\begin{pgfscope}%
\pgfsys@transformshift{4.141170in}{2.421907in}%
\pgfsys@useobject{currentmarker}{}%
\end{pgfscope}%
\end{pgfscope}%
\begin{pgfscope}%
\pgfsetbuttcap%
\pgfsetroundjoin%
\definecolor{currentfill}{rgb}{0.121569,0.466667,0.705882}%
\pgfsetfillcolor{currentfill}%
\pgfsetlinewidth{1.003750pt}%
\definecolor{currentstroke}{rgb}{0.121569,0.466667,0.705882}%
\pgfsetstrokecolor{currentstroke}%
\pgfsetdash{}{0pt}%
\pgfsys@defobject{currentmarker}{\pgfqpoint{-0.041667in}{-0.000000in}}{\pgfqpoint{0.041667in}{0.000000in}}{%
\pgfpathmoveto{\pgfqpoint{0.041667in}{-0.000000in}}%
\pgfpathlineto{\pgfqpoint{-0.041667in}{0.000000in}}%
\pgfusepath{stroke,fill}%
}%
\begin{pgfscope}%
\pgfsys@transformshift{4.141170in}{2.560796in}%
\pgfsys@useobject{currentmarker}{}%
\end{pgfscope}%
\end{pgfscope}%
\begin{pgfscope}%
\pgfsetrectcap%
\pgfsetroundjoin%
\pgfsetlinewidth{1.505625pt}%
\definecolor{currentstroke}{rgb}{0.121569,0.466667,0.705882}%
\pgfsetstrokecolor{currentstroke}%
\pgfsetdash{}{0pt}%
\pgfpathmoveto{\pgfqpoint{4.002282in}{2.491351in}}%
\pgfpathlineto{\pgfqpoint{4.280059in}{2.491351in}}%
\pgfusepath{stroke}%
\end{pgfscope}%
\begin{pgfscope}%
\pgfsetbuttcap%
\pgfsetroundjoin%
\definecolor{currentfill}{rgb}{0.121569,0.466667,0.705882}%
\pgfsetfillcolor{currentfill}%
\pgfsetlinewidth{1.003750pt}%
\definecolor{currentstroke}{rgb}{0.121569,0.466667,0.705882}%
\pgfsetstrokecolor{currentstroke}%
\pgfsetdash{}{0pt}%
\pgfsys@defobject{currentmarker}{\pgfqpoint{-0.020833in}{-0.020833in}}{\pgfqpoint{0.020833in}{0.020833in}}{%
\pgfpathmoveto{\pgfqpoint{0.000000in}{-0.020833in}}%
\pgfpathcurveto{\pgfqpoint{0.005525in}{-0.020833in}}{\pgfqpoint{0.010825in}{-0.018638in}}{\pgfqpoint{0.014731in}{-0.014731in}}%
\pgfpathcurveto{\pgfqpoint{0.018638in}{-0.010825in}}{\pgfqpoint{0.020833in}{-0.005525in}}{\pgfqpoint{0.020833in}{0.000000in}}%
\pgfpathcurveto{\pgfqpoint{0.020833in}{0.005525in}}{\pgfqpoint{0.018638in}{0.010825in}}{\pgfqpoint{0.014731in}{0.014731in}}%
\pgfpathcurveto{\pgfqpoint{0.010825in}{0.018638in}}{\pgfqpoint{0.005525in}{0.020833in}}{\pgfqpoint{0.000000in}{0.020833in}}%
\pgfpathcurveto{\pgfqpoint{-0.005525in}{0.020833in}}{\pgfqpoint{-0.010825in}{0.018638in}}{\pgfqpoint{-0.014731in}{0.014731in}}%
\pgfpathcurveto{\pgfqpoint{-0.018638in}{0.010825in}}{\pgfqpoint{-0.020833in}{0.005525in}}{\pgfqpoint{-0.020833in}{0.000000in}}%
\pgfpathcurveto{\pgfqpoint{-0.020833in}{-0.005525in}}{\pgfqpoint{-0.018638in}{-0.010825in}}{\pgfqpoint{-0.014731in}{-0.014731in}}%
\pgfpathcurveto{\pgfqpoint{-0.010825in}{-0.018638in}}{\pgfqpoint{-0.005525in}{-0.020833in}}{\pgfqpoint{0.000000in}{-0.020833in}}%
\pgfpathlineto{\pgfqpoint{0.000000in}{-0.020833in}}%
\pgfpathclose%
\pgfusepath{stroke,fill}%
}%
\begin{pgfscope}%
\pgfsys@transformshift{4.141170in}{2.491351in}%
\pgfsys@useobject{currentmarker}{}%
\end{pgfscope}%
\end{pgfscope}%
\begin{pgfscope}%
\definecolor{textcolor}{rgb}{0.000000,0.000000,0.000000}%
\pgfsetstrokecolor{textcolor}%
\pgfsetfillcolor{textcolor}%
\pgftext[x=4.391170in,y=2.442740in,left,base]{\color{textcolor}\sffamily\fontsize{10.000000}{12.000000}\selectfont data (no downramp) }%
\end{pgfscope}%
\end{pgfpicture}%
\makeatother%
\endgroup%

	\caption{
	\textbf{(a)} Energy histogram over time. A histogram of the charge distribution over the energy is created every 2000 timesteps and plotted here over $y$.
	\textbf{(b)} Peak energy plotted over $y$. A second curve for a simulation where the plasma jet ends after \qty{3}{\mm} is drawn in too with the vertical line marking the point of the downramp. 
	Note that the systematic uncertainty range is barely visible behind the curve of the data.}
	\label{fig:E_hist_time}
\end{figure}
When inducing the wakefield, the Lorentz force created by the fields of the first cavity acts on the driver causing it to lose energy. As only parts of the driver experience this force, the energy distribution is not moving to lower energies entirely, but instead spreading.
The plot shows the growth in low energy electrons until $E=0$ is reached, visualized by the $E_{min}$ curve. In blue, the mean is also plotted, showing how it reduces with time. Also the maximum energy $E_{max}$ is shown, which increases.
This increase stems from the particles, which gain energy as they are so far back in the bunch, that they get pushed by the Lorentz forces in the middle of the first cavity.

Notable is the fact, that the histogram for every timestep is not uniform but has visible maxima and minima in form of the black and yellow stripes in \autoref{fig:E_hist_time}a. The maximum with the highest energy is called the peak energy, here additionally plotted in green.
This energy is important in experiment, as the assumption that it stays constant during the \gls{pwfa} is used to calculate the initial charge of the driver before entering the \gls{pwfa} (see \cite{Schoebel2022}).
In \autoref{fig:E_hist_time}b only the peak energy is plotted over time, showing that it is in fact not constant. The peak energy drops from \qtyrange{250}{244}{\MeV} shortly after entering the plasma.
A plateau exists at this energy until bunch breakup, after which the peak further drops.

The systematic uncertainty of the peak energy is given by the size of the bins while the statistical uncertainty results from the uncertainty of the fit. 

The fit assumes two summed Gaussian distributions as a simplification of the real distribution, which consists of multiple peaks with different heights and widths as well as an unknown background noise. Only the peak energy and the second peak are fitted.
This fit can produce strong outliers during the first half millimeter and still seems to have a slight visual offset to the real curve afterwards, so a better fit method should be used in future studies.

Also drawn is the curve for a simulation with a down ramp for the plasma after \qty{2.7}{\mm} of transverse propagation, as peak energy is measured after leaving the plasma jet in experiment. The drop-off of \qty{5}{\MeV} remains after 
the downramp and would therefore be measured in experiment, when high enough accuracy can be achieved. Even after the driver left the plasma, the peak energy further dropped. 
No significant decelerating forces act in this phase, so the loss may just be caused by an inaccuracy of the energy fit. 

For this and all following simulations, the peak energy loss was also analyzed for only the particles which do not leave the simulation box.
This should prevent an energy loss due to the loss of high-energy particles. No qualitative changes in the peak energy curves were observed. The effect of the particle loss is therefore neglectable.

Currently, the uncertainty on the peak energy measurement in experiments is significantly higher than the \qty{5}{\MeV} jump. Thus, caution is needed when assuming a constant peak energy, especially after bunch breakup.

\subsection{Locality of the energy}\label{chap:loc_E}
The energy loss and spatial locality of the energy peaks can be visualized again by binning the macroparticles in space and analyzing the mean energy in every bin. This mean energy is then plotted over the positions of the bins,
an example of this can be seen in \autoref{fig:E_time}a. 
\begin{figure}
	\centering
	%% Creator: Matplotlib, PGF backend
%%
%% To include the figure in your LaTeX document, write
%%   \input{<filename>.pgf}
%%
%% Make sure the required packages are loaded in your preamble
%%   \usepackage{pgf}
%%
%% Also ensure that all the required font packages are loaded; for instance,
%% the lmodern package is sometimes necessary when using math font.
%%   \usepackage{lmodern}
%%
%% Figures using additional raster images can only be included by \input if
%% they are in the same directory as the main LaTeX file. For loading figures
%% from other directories you can use the `import` package
%%   \usepackage{import}
%%
%% and then include the figures with
%%   \import{<path to file>}{<filename>.pgf}
%%
%% Matplotlib used the following preamble
%%
\begingroup%
\makeatletter%
\begin{pgfpicture}%
\pgfpathrectangle{\pgfpointorigin}{\pgfqpoint{6.000000in}{4.000000in}}%
\pgfusepath{use as bounding box, clip}%
\begin{pgfscope}%
\pgfsetbuttcap%
\pgfsetmiterjoin%
\pgfsetlinewidth{0.000000pt}%
\definecolor{currentstroke}{rgb}{1.000000,1.000000,1.000000}%
\pgfsetstrokecolor{currentstroke}%
\pgfsetstrokeopacity{0.000000}%
\pgfsetdash{}{0pt}%
\pgfpathmoveto{\pgfqpoint{0.000000in}{0.000000in}}%
\pgfpathlineto{\pgfqpoint{6.000000in}{0.000000in}}%
\pgfpathlineto{\pgfqpoint{6.000000in}{4.000000in}}%
\pgfpathlineto{\pgfqpoint{0.000000in}{4.000000in}}%
\pgfpathlineto{\pgfqpoint{0.000000in}{0.000000in}}%
\pgfpathclose%
\pgfusepath{}%
\end{pgfscope}%
\begin{pgfscope}%
\pgfsetbuttcap%
\pgfsetmiterjoin%
\definecolor{currentfill}{rgb}{1.000000,1.000000,1.000000}%
\pgfsetfillcolor{currentfill}%
\pgfsetlinewidth{0.000000pt}%
\definecolor{currentstroke}{rgb}{0.000000,0.000000,0.000000}%
\pgfsetstrokecolor{currentstroke}%
\pgfsetstrokeopacity{0.000000}%
\pgfsetdash{}{0pt}%
\pgfpathmoveto{\pgfqpoint{0.750000in}{0.500000in}}%
\pgfpathlineto{\pgfqpoint{4.470000in}{0.500000in}}%
\pgfpathlineto{\pgfqpoint{4.470000in}{3.520000in}}%
\pgfpathlineto{\pgfqpoint{0.750000in}{3.520000in}}%
\pgfpathlineto{\pgfqpoint{0.750000in}{0.500000in}}%
\pgfpathclose%
\pgfusepath{fill}%
\end{pgfscope}%
\begin{pgfscope}%
\pgfpathrectangle{\pgfqpoint{0.750000in}{0.500000in}}{\pgfqpoint{3.720000in}{3.020000in}}%
\pgfusepath{clip}%
\pgfsys@transformcm{3.722222}{0.000000}{0.000000}{3.027778}{0.750000in}{0.500000in}%
\pgftext[left,bottom]{\includegraphics[interpolate=false,width=1.000000in,height=1.000000in]{E_time-img0.png}}%
\end{pgfscope}%
\begin{pgfscope}%
\pgfsetbuttcap%
\pgfsetroundjoin%
\definecolor{currentfill}{rgb}{0.000000,0.000000,0.000000}%
\pgfsetfillcolor{currentfill}%
\pgfsetlinewidth{0.803000pt}%
\definecolor{currentstroke}{rgb}{0.000000,0.000000,0.000000}%
\pgfsetstrokecolor{currentstroke}%
\pgfsetdash{}{0pt}%
\pgfsys@defobject{currentmarker}{\pgfqpoint{0.000000in}{-0.048611in}}{\pgfqpoint{0.000000in}{0.000000in}}{%
\pgfpathmoveto{\pgfqpoint{0.000000in}{0.000000in}}%
\pgfpathlineto{\pgfqpoint{0.000000in}{-0.048611in}}%
\pgfusepath{stroke,fill}%
}%
\begin{pgfscope}%
\pgfsys@transformshift{1.224754in}{0.500000in}%
\pgfsys@useobject{currentmarker}{}%
\end{pgfscope}%
\end{pgfscope}%
\begin{pgfscope}%
\definecolor{textcolor}{rgb}{0.000000,0.000000,0.000000}%
\pgfsetstrokecolor{textcolor}%
\pgfsetfillcolor{textcolor}%
\pgftext[x=1.224754in,y=0.402778in,,top]{\color{textcolor}\sffamily\fontsize{10.000000}{12.000000}\selectfont \(\displaystyle {\ensuremath{-}15}\)}%
\end{pgfscope}%
\begin{pgfscope}%
\pgfsetbuttcap%
\pgfsetroundjoin%
\definecolor{currentfill}{rgb}{0.000000,0.000000,0.000000}%
\pgfsetfillcolor{currentfill}%
\pgfsetlinewidth{0.803000pt}%
\definecolor{currentstroke}{rgb}{0.000000,0.000000,0.000000}%
\pgfsetstrokecolor{currentstroke}%
\pgfsetdash{}{0pt}%
\pgfsys@defobject{currentmarker}{\pgfqpoint{0.000000in}{-0.048611in}}{\pgfqpoint{0.000000in}{0.000000in}}{%
\pgfpathmoveto{\pgfqpoint{0.000000in}{0.000000in}}%
\pgfpathlineto{\pgfqpoint{0.000000in}{-0.048611in}}%
\pgfusepath{stroke,fill}%
}%
\begin{pgfscope}%
\pgfsys@transformshift{1.810503in}{0.500000in}%
\pgfsys@useobject{currentmarker}{}%
\end{pgfscope}%
\end{pgfscope}%
\begin{pgfscope}%
\definecolor{textcolor}{rgb}{0.000000,0.000000,0.000000}%
\pgfsetstrokecolor{textcolor}%
\pgfsetfillcolor{textcolor}%
\pgftext[x=1.810503in,y=0.402778in,,top]{\color{textcolor}\sffamily\fontsize{10.000000}{12.000000}\selectfont \(\displaystyle {\ensuremath{-}10}\)}%
\end{pgfscope}%
\begin{pgfscope}%
\pgfsetbuttcap%
\pgfsetroundjoin%
\definecolor{currentfill}{rgb}{0.000000,0.000000,0.000000}%
\pgfsetfillcolor{currentfill}%
\pgfsetlinewidth{0.803000pt}%
\definecolor{currentstroke}{rgb}{0.000000,0.000000,0.000000}%
\pgfsetstrokecolor{currentstroke}%
\pgfsetdash{}{0pt}%
\pgfsys@defobject{currentmarker}{\pgfqpoint{0.000000in}{-0.048611in}}{\pgfqpoint{0.000000in}{0.000000in}}{%
\pgfpathmoveto{\pgfqpoint{0.000000in}{0.000000in}}%
\pgfpathlineto{\pgfqpoint{0.000000in}{-0.048611in}}%
\pgfusepath{stroke,fill}%
}%
\begin{pgfscope}%
\pgfsys@transformshift{2.396251in}{0.500000in}%
\pgfsys@useobject{currentmarker}{}%
\end{pgfscope}%
\end{pgfscope}%
\begin{pgfscope}%
\definecolor{textcolor}{rgb}{0.000000,0.000000,0.000000}%
\pgfsetstrokecolor{textcolor}%
\pgfsetfillcolor{textcolor}%
\pgftext[x=2.396251in,y=0.402778in,,top]{\color{textcolor}\sffamily\fontsize{10.000000}{12.000000}\selectfont \(\displaystyle {\ensuremath{-}5}\)}%
\end{pgfscope}%
\begin{pgfscope}%
\pgfsetbuttcap%
\pgfsetroundjoin%
\definecolor{currentfill}{rgb}{0.000000,0.000000,0.000000}%
\pgfsetfillcolor{currentfill}%
\pgfsetlinewidth{0.803000pt}%
\definecolor{currentstroke}{rgb}{0.000000,0.000000,0.000000}%
\pgfsetstrokecolor{currentstroke}%
\pgfsetdash{}{0pt}%
\pgfsys@defobject{currentmarker}{\pgfqpoint{0.000000in}{-0.048611in}}{\pgfqpoint{0.000000in}{0.000000in}}{%
\pgfpathmoveto{\pgfqpoint{0.000000in}{0.000000in}}%
\pgfpathlineto{\pgfqpoint{0.000000in}{-0.048611in}}%
\pgfusepath{stroke,fill}%
}%
\begin{pgfscope}%
\pgfsys@transformshift{2.982000in}{0.500000in}%
\pgfsys@useobject{currentmarker}{}%
\end{pgfscope}%
\end{pgfscope}%
\begin{pgfscope}%
\definecolor{textcolor}{rgb}{0.000000,0.000000,0.000000}%
\pgfsetstrokecolor{textcolor}%
\pgfsetfillcolor{textcolor}%
\pgftext[x=2.982000in,y=0.402778in,,top]{\color{textcolor}\sffamily\fontsize{10.000000}{12.000000}\selectfont \(\displaystyle {0}\)}%
\end{pgfscope}%
\begin{pgfscope}%
\pgfsetbuttcap%
\pgfsetroundjoin%
\definecolor{currentfill}{rgb}{0.000000,0.000000,0.000000}%
\pgfsetfillcolor{currentfill}%
\pgfsetlinewidth{0.803000pt}%
\definecolor{currentstroke}{rgb}{0.000000,0.000000,0.000000}%
\pgfsetstrokecolor{currentstroke}%
\pgfsetdash{}{0pt}%
\pgfsys@defobject{currentmarker}{\pgfqpoint{0.000000in}{-0.048611in}}{\pgfqpoint{0.000000in}{0.000000in}}{%
\pgfpathmoveto{\pgfqpoint{0.000000in}{0.000000in}}%
\pgfpathlineto{\pgfqpoint{0.000000in}{-0.048611in}}%
\pgfusepath{stroke,fill}%
}%
\begin{pgfscope}%
\pgfsys@transformshift{3.567749in}{0.500000in}%
\pgfsys@useobject{currentmarker}{}%
\end{pgfscope}%
\end{pgfscope}%
\begin{pgfscope}%
\definecolor{textcolor}{rgb}{0.000000,0.000000,0.000000}%
\pgfsetstrokecolor{textcolor}%
\pgfsetfillcolor{textcolor}%
\pgftext[x=3.567749in,y=0.402778in,,top]{\color{textcolor}\sffamily\fontsize{10.000000}{12.000000}\selectfont \(\displaystyle {5}\)}%
\end{pgfscope}%
\begin{pgfscope}%
\pgfsetbuttcap%
\pgfsetroundjoin%
\definecolor{currentfill}{rgb}{0.000000,0.000000,0.000000}%
\pgfsetfillcolor{currentfill}%
\pgfsetlinewidth{0.803000pt}%
\definecolor{currentstroke}{rgb}{0.000000,0.000000,0.000000}%
\pgfsetstrokecolor{currentstroke}%
\pgfsetdash{}{0pt}%
\pgfsys@defobject{currentmarker}{\pgfqpoint{0.000000in}{-0.048611in}}{\pgfqpoint{0.000000in}{0.000000in}}{%
\pgfpathmoveto{\pgfqpoint{0.000000in}{0.000000in}}%
\pgfpathlineto{\pgfqpoint{0.000000in}{-0.048611in}}%
\pgfusepath{stroke,fill}%
}%
\begin{pgfscope}%
\pgfsys@transformshift{4.153497in}{0.500000in}%
\pgfsys@useobject{currentmarker}{}%
\end{pgfscope}%
\end{pgfscope}%
\begin{pgfscope}%
\definecolor{textcolor}{rgb}{0.000000,0.000000,0.000000}%
\pgfsetstrokecolor{textcolor}%
\pgfsetfillcolor{textcolor}%
\pgftext[x=4.153497in,y=0.402778in,,top]{\color{textcolor}\sffamily\fontsize{10.000000}{12.000000}\selectfont \(\displaystyle {10}\)}%
\end{pgfscope}%
\begin{pgfscope}%
\definecolor{textcolor}{rgb}{0.000000,0.000000,0.000000}%
\pgfsetstrokecolor{textcolor}%
\pgfsetfillcolor{textcolor}%
\pgftext[x=2.610000in,y=0.223766in,,top]{\color{textcolor}\sffamily\fontsize{10.000000}{12.000000}\selectfont \(\displaystyle \zeta \, \mathrm{[\mu m]}\)}%
\end{pgfscope}%
\begin{pgfscope}%
\pgfsetbuttcap%
\pgfsetroundjoin%
\definecolor{currentfill}{rgb}{0.000000,0.000000,0.000000}%
\pgfsetfillcolor{currentfill}%
\pgfsetlinewidth{0.803000pt}%
\definecolor{currentstroke}{rgb}{0.000000,0.000000,0.000000}%
\pgfsetstrokecolor{currentstroke}%
\pgfsetdash{}{0pt}%
\pgfsys@defobject{currentmarker}{\pgfqpoint{-0.048611in}{0.000000in}}{\pgfqpoint{-0.000000in}{0.000000in}}{%
\pgfpathmoveto{\pgfqpoint{-0.000000in}{0.000000in}}%
\pgfpathlineto{\pgfqpoint{-0.048611in}{0.000000in}}%
\pgfusepath{stroke,fill}%
}%
\begin{pgfscope}%
\pgfsys@transformshift{0.750000in}{0.537433in}%
\pgfsys@useobject{currentmarker}{}%
\end{pgfscope}%
\end{pgfscope}%
\begin{pgfscope}%
\definecolor{textcolor}{rgb}{0.000000,0.000000,0.000000}%
\pgfsetstrokecolor{textcolor}%
\pgfsetfillcolor{textcolor}%
\pgftext[x=0.583333in, y=0.489208in, left, base]{\color{textcolor}\sffamily\fontsize{10.000000}{12.000000}\selectfont \(\displaystyle {0}\)}%
\end{pgfscope}%
\begin{pgfscope}%
\pgfsetbuttcap%
\pgfsetroundjoin%
\definecolor{currentfill}{rgb}{0.000000,0.000000,0.000000}%
\pgfsetfillcolor{currentfill}%
\pgfsetlinewidth{0.803000pt}%
\definecolor{currentstroke}{rgb}{0.000000,0.000000,0.000000}%
\pgfsetstrokecolor{currentstroke}%
\pgfsetdash{}{0pt}%
\pgfsys@defobject{currentmarker}{\pgfqpoint{-0.048611in}{0.000000in}}{\pgfqpoint{-0.000000in}{0.000000in}}{%
\pgfpathmoveto{\pgfqpoint{-0.000000in}{0.000000in}}%
\pgfpathlineto{\pgfqpoint{-0.048611in}{0.000000in}}%
\pgfusepath{stroke,fill}%
}%
\begin{pgfscope}%
\pgfsys@transformshift{0.750000in}{1.018157in}%
\pgfsys@useobject{currentmarker}{}%
\end{pgfscope}%
\end{pgfscope}%
\begin{pgfscope}%
\definecolor{textcolor}{rgb}{0.000000,0.000000,0.000000}%
\pgfsetstrokecolor{textcolor}%
\pgfsetfillcolor{textcolor}%
\pgftext[x=0.583333in, y=0.969932in, left, base]{\color{textcolor}\sffamily\fontsize{10.000000}{12.000000}\selectfont \(\displaystyle {1}\)}%
\end{pgfscope}%
\begin{pgfscope}%
\pgfsetbuttcap%
\pgfsetroundjoin%
\definecolor{currentfill}{rgb}{0.000000,0.000000,0.000000}%
\pgfsetfillcolor{currentfill}%
\pgfsetlinewidth{0.803000pt}%
\definecolor{currentstroke}{rgb}{0.000000,0.000000,0.000000}%
\pgfsetstrokecolor{currentstroke}%
\pgfsetdash{}{0pt}%
\pgfsys@defobject{currentmarker}{\pgfqpoint{-0.048611in}{0.000000in}}{\pgfqpoint{-0.000000in}{0.000000in}}{%
\pgfpathmoveto{\pgfqpoint{-0.000000in}{0.000000in}}%
\pgfpathlineto{\pgfqpoint{-0.048611in}{0.000000in}}%
\pgfusepath{stroke,fill}%
}%
\begin{pgfscope}%
\pgfsys@transformshift{0.750000in}{1.498881in}%
\pgfsys@useobject{currentmarker}{}%
\end{pgfscope}%
\end{pgfscope}%
\begin{pgfscope}%
\definecolor{textcolor}{rgb}{0.000000,0.000000,0.000000}%
\pgfsetstrokecolor{textcolor}%
\pgfsetfillcolor{textcolor}%
\pgftext[x=0.583333in, y=1.450656in, left, base]{\color{textcolor}\sffamily\fontsize{10.000000}{12.000000}\selectfont \(\displaystyle {2}\)}%
\end{pgfscope}%
\begin{pgfscope}%
\pgfsetbuttcap%
\pgfsetroundjoin%
\definecolor{currentfill}{rgb}{0.000000,0.000000,0.000000}%
\pgfsetfillcolor{currentfill}%
\pgfsetlinewidth{0.803000pt}%
\definecolor{currentstroke}{rgb}{0.000000,0.000000,0.000000}%
\pgfsetstrokecolor{currentstroke}%
\pgfsetdash{}{0pt}%
\pgfsys@defobject{currentmarker}{\pgfqpoint{-0.048611in}{0.000000in}}{\pgfqpoint{-0.000000in}{0.000000in}}{%
\pgfpathmoveto{\pgfqpoint{-0.000000in}{0.000000in}}%
\pgfpathlineto{\pgfqpoint{-0.048611in}{0.000000in}}%
\pgfusepath{stroke,fill}%
}%
\begin{pgfscope}%
\pgfsys@transformshift{0.750000in}{1.979605in}%
\pgfsys@useobject{currentmarker}{}%
\end{pgfscope}%
\end{pgfscope}%
\begin{pgfscope}%
\definecolor{textcolor}{rgb}{0.000000,0.000000,0.000000}%
\pgfsetstrokecolor{textcolor}%
\pgfsetfillcolor{textcolor}%
\pgftext[x=0.583333in, y=1.931380in, left, base]{\color{textcolor}\sffamily\fontsize{10.000000}{12.000000}\selectfont \(\displaystyle {3}\)}%
\end{pgfscope}%
\begin{pgfscope}%
\pgfsetbuttcap%
\pgfsetroundjoin%
\definecolor{currentfill}{rgb}{0.000000,0.000000,0.000000}%
\pgfsetfillcolor{currentfill}%
\pgfsetlinewidth{0.803000pt}%
\definecolor{currentstroke}{rgb}{0.000000,0.000000,0.000000}%
\pgfsetstrokecolor{currentstroke}%
\pgfsetdash{}{0pt}%
\pgfsys@defobject{currentmarker}{\pgfqpoint{-0.048611in}{0.000000in}}{\pgfqpoint{-0.000000in}{0.000000in}}{%
\pgfpathmoveto{\pgfqpoint{-0.000000in}{0.000000in}}%
\pgfpathlineto{\pgfqpoint{-0.048611in}{0.000000in}}%
\pgfusepath{stroke,fill}%
}%
\begin{pgfscope}%
\pgfsys@transformshift{0.750000in}{2.460329in}%
\pgfsys@useobject{currentmarker}{}%
\end{pgfscope}%
\end{pgfscope}%
\begin{pgfscope}%
\definecolor{textcolor}{rgb}{0.000000,0.000000,0.000000}%
\pgfsetstrokecolor{textcolor}%
\pgfsetfillcolor{textcolor}%
\pgftext[x=0.583333in, y=2.412104in, left, base]{\color{textcolor}\sffamily\fontsize{10.000000}{12.000000}\selectfont \(\displaystyle {4}\)}%
\end{pgfscope}%
\begin{pgfscope}%
\pgfsetbuttcap%
\pgfsetroundjoin%
\definecolor{currentfill}{rgb}{0.000000,0.000000,0.000000}%
\pgfsetfillcolor{currentfill}%
\pgfsetlinewidth{0.803000pt}%
\definecolor{currentstroke}{rgb}{0.000000,0.000000,0.000000}%
\pgfsetstrokecolor{currentstroke}%
\pgfsetdash{}{0pt}%
\pgfsys@defobject{currentmarker}{\pgfqpoint{-0.048611in}{0.000000in}}{\pgfqpoint{-0.000000in}{0.000000in}}{%
\pgfpathmoveto{\pgfqpoint{-0.000000in}{0.000000in}}%
\pgfpathlineto{\pgfqpoint{-0.048611in}{0.000000in}}%
\pgfusepath{stroke,fill}%
}%
\begin{pgfscope}%
\pgfsys@transformshift{0.750000in}{2.941053in}%
\pgfsys@useobject{currentmarker}{}%
\end{pgfscope}%
\end{pgfscope}%
\begin{pgfscope}%
\definecolor{textcolor}{rgb}{0.000000,0.000000,0.000000}%
\pgfsetstrokecolor{textcolor}%
\pgfsetfillcolor{textcolor}%
\pgftext[x=0.583333in, y=2.892828in, left, base]{\color{textcolor}\sffamily\fontsize{10.000000}{12.000000}\selectfont \(\displaystyle {5}\)}%
\end{pgfscope}%
\begin{pgfscope}%
\pgfsetbuttcap%
\pgfsetroundjoin%
\definecolor{currentfill}{rgb}{0.000000,0.000000,0.000000}%
\pgfsetfillcolor{currentfill}%
\pgfsetlinewidth{0.803000pt}%
\definecolor{currentstroke}{rgb}{0.000000,0.000000,0.000000}%
\pgfsetstrokecolor{currentstroke}%
\pgfsetdash{}{0pt}%
\pgfsys@defobject{currentmarker}{\pgfqpoint{-0.048611in}{0.000000in}}{\pgfqpoint{-0.000000in}{0.000000in}}{%
\pgfpathmoveto{\pgfqpoint{-0.000000in}{0.000000in}}%
\pgfpathlineto{\pgfqpoint{-0.048611in}{0.000000in}}%
\pgfusepath{stroke,fill}%
}%
\begin{pgfscope}%
\pgfsys@transformshift{0.750000in}{3.421777in}%
\pgfsys@useobject{currentmarker}{}%
\end{pgfscope}%
\end{pgfscope}%
\begin{pgfscope}%
\definecolor{textcolor}{rgb}{0.000000,0.000000,0.000000}%
\pgfsetstrokecolor{textcolor}%
\pgfsetfillcolor{textcolor}%
\pgftext[x=0.583333in, y=3.373552in, left, base]{\color{textcolor}\sffamily\fontsize{10.000000}{12.000000}\selectfont \(\displaystyle {6}\)}%
\end{pgfscope}%
\begin{pgfscope}%
\definecolor{textcolor}{rgb}{0.000000,0.000000,0.000000}%
\pgfsetstrokecolor{textcolor}%
\pgfsetfillcolor{textcolor}%
\pgftext[x=0.527778in,y=2.010000in,,bottom,rotate=90.000000]{\color{textcolor}\sffamily\fontsize{10.000000}{12.000000}\selectfont \(\displaystyle y \, \mathrm{[mm]}\)}%
\end{pgfscope}%
\begin{pgfscope}%
\pgfsetrectcap%
\pgfsetmiterjoin%
\pgfsetlinewidth{0.803000pt}%
\definecolor{currentstroke}{rgb}{0.000000,0.000000,0.000000}%
\pgfsetstrokecolor{currentstroke}%
\pgfsetdash{}{0pt}%
\pgfpathmoveto{\pgfqpoint{0.750000in}{0.500000in}}%
\pgfpathlineto{\pgfqpoint{0.750000in}{3.520000in}}%
\pgfusepath{stroke}%
\end{pgfscope}%
\begin{pgfscope}%
\pgfsetrectcap%
\pgfsetmiterjoin%
\pgfsetlinewidth{0.803000pt}%
\definecolor{currentstroke}{rgb}{0.000000,0.000000,0.000000}%
\pgfsetstrokecolor{currentstroke}%
\pgfsetdash{}{0pt}%
\pgfpathmoveto{\pgfqpoint{4.470000in}{0.500000in}}%
\pgfpathlineto{\pgfqpoint{4.470000in}{3.520000in}}%
\pgfusepath{stroke}%
\end{pgfscope}%
\begin{pgfscope}%
\pgfsetrectcap%
\pgfsetmiterjoin%
\pgfsetlinewidth{0.803000pt}%
\definecolor{currentstroke}{rgb}{0.000000,0.000000,0.000000}%
\pgfsetstrokecolor{currentstroke}%
\pgfsetdash{}{0pt}%
\pgfpathmoveto{\pgfqpoint{0.750000in}{0.500000in}}%
\pgfpathlineto{\pgfqpoint{4.470000in}{0.500000in}}%
\pgfusepath{stroke}%
\end{pgfscope}%
\begin{pgfscope}%
\pgfsetrectcap%
\pgfsetmiterjoin%
\pgfsetlinewidth{0.803000pt}%
\definecolor{currentstroke}{rgb}{0.000000,0.000000,0.000000}%
\pgfsetstrokecolor{currentstroke}%
\pgfsetdash{}{0pt}%
\pgfpathmoveto{\pgfqpoint{0.750000in}{3.520000in}}%
\pgfpathlineto{\pgfqpoint{4.470000in}{3.520000in}}%
\pgfusepath{stroke}%
\end{pgfscope}%
\begin{pgfscope}%
\pgfsetbuttcap%
\pgfsetmiterjoin%
\definecolor{currentfill}{rgb}{1.000000,1.000000,1.000000}%
\pgfsetfillcolor{currentfill}%
\pgfsetlinewidth{0.000000pt}%
\definecolor{currentstroke}{rgb}{0.000000,0.000000,0.000000}%
\pgfsetstrokecolor{currentstroke}%
\pgfsetstrokeopacity{0.000000}%
\pgfsetdash{}{0pt}%
\pgfpathmoveto{\pgfqpoint{4.702500in}{0.500000in}}%
\pgfpathlineto{\pgfqpoint{4.853500in}{0.500000in}}%
\pgfpathlineto{\pgfqpoint{4.853500in}{3.520000in}}%
\pgfpathlineto{\pgfqpoint{4.702500in}{3.520000in}}%
\pgfpathlineto{\pgfqpoint{4.702500in}{0.500000in}}%
\pgfpathclose%
\pgfusepath{fill}%
\end{pgfscope}%
\begin{pgfscope}%
\pgfpathrectangle{\pgfqpoint{4.702500in}{0.500000in}}{\pgfqpoint{0.151000in}{3.020000in}}%
\pgfusepath{clip}%
\pgfsetbuttcap%
\pgfsetmiterjoin%
\definecolor{currentfill}{rgb}{1.000000,1.000000,1.000000}%
\pgfsetfillcolor{currentfill}%
\pgfsetlinewidth{0.010037pt}%
\definecolor{currentstroke}{rgb}{1.000000,1.000000,1.000000}%
\pgfsetstrokecolor{currentstroke}%
\pgfsetdash{}{0pt}%
\pgfusepath{stroke,fill}%
\end{pgfscope}%
\begin{pgfscope}%
\pgfsys@transformshift{4.708333in}{0.500000in}%
\pgftext[left,bottom]{\includegraphics[interpolate=true,width=0.138889in,height=3.013889in]{E_time-img1.png}}%
\end{pgfscope}%
\begin{pgfscope}%
\pgfsetbuttcap%
\pgfsetroundjoin%
\definecolor{currentfill}{rgb}{0.000000,0.000000,0.000000}%
\pgfsetfillcolor{currentfill}%
\pgfsetlinewidth{0.803000pt}%
\definecolor{currentstroke}{rgb}{0.000000,0.000000,0.000000}%
\pgfsetstrokecolor{currentstroke}%
\pgfsetdash{}{0pt}%
\pgfsys@defobject{currentmarker}{\pgfqpoint{0.000000in}{0.000000in}}{\pgfqpoint{0.048611in}{0.000000in}}{%
\pgfpathmoveto{\pgfqpoint{0.000000in}{0.000000in}}%
\pgfpathlineto{\pgfqpoint{0.048611in}{0.000000in}}%
\pgfusepath{stroke,fill}%
}%
\begin{pgfscope}%
\pgfsys@transformshift{4.853500in}{0.500000in}%
\pgfsys@useobject{currentmarker}{}%
\end{pgfscope}%
\end{pgfscope}%
\begin{pgfscope}%
\definecolor{textcolor}{rgb}{0.000000,0.000000,0.000000}%
\pgfsetstrokecolor{textcolor}%
\pgfsetfillcolor{textcolor}%
\pgftext[x=4.950722in, y=0.451775in, left, base]{\color{textcolor}\sffamily\fontsize{10.000000}{12.000000}\selectfont \(\displaystyle {0}\)}%
\end{pgfscope}%
\begin{pgfscope}%
\pgfsetbuttcap%
\pgfsetroundjoin%
\definecolor{currentfill}{rgb}{0.000000,0.000000,0.000000}%
\pgfsetfillcolor{currentfill}%
\pgfsetlinewidth{0.803000pt}%
\definecolor{currentstroke}{rgb}{0.000000,0.000000,0.000000}%
\pgfsetstrokecolor{currentstroke}%
\pgfsetdash{}{0pt}%
\pgfsys@defobject{currentmarker}{\pgfqpoint{0.000000in}{0.000000in}}{\pgfqpoint{0.048611in}{0.000000in}}{%
\pgfpathmoveto{\pgfqpoint{0.000000in}{0.000000in}}%
\pgfpathlineto{\pgfqpoint{0.048611in}{0.000000in}}%
\pgfusepath{stroke,fill}%
}%
\begin{pgfscope}%
\pgfsys@transformshift{4.853500in}{1.003333in}%
\pgfsys@useobject{currentmarker}{}%
\end{pgfscope}%
\end{pgfscope}%
\begin{pgfscope}%
\definecolor{textcolor}{rgb}{0.000000,0.000000,0.000000}%
\pgfsetstrokecolor{textcolor}%
\pgfsetfillcolor{textcolor}%
\pgftext[x=4.950722in, y=0.955108in, left, base]{\color{textcolor}\sffamily\fontsize{10.000000}{12.000000}\selectfont \(\displaystyle {50}\)}%
\end{pgfscope}%
\begin{pgfscope}%
\pgfsetbuttcap%
\pgfsetroundjoin%
\definecolor{currentfill}{rgb}{0.000000,0.000000,0.000000}%
\pgfsetfillcolor{currentfill}%
\pgfsetlinewidth{0.803000pt}%
\definecolor{currentstroke}{rgb}{0.000000,0.000000,0.000000}%
\pgfsetstrokecolor{currentstroke}%
\pgfsetdash{}{0pt}%
\pgfsys@defobject{currentmarker}{\pgfqpoint{0.000000in}{0.000000in}}{\pgfqpoint{0.048611in}{0.000000in}}{%
\pgfpathmoveto{\pgfqpoint{0.000000in}{0.000000in}}%
\pgfpathlineto{\pgfqpoint{0.048611in}{0.000000in}}%
\pgfusepath{stroke,fill}%
}%
\begin{pgfscope}%
\pgfsys@transformshift{4.853500in}{1.506667in}%
\pgfsys@useobject{currentmarker}{}%
\end{pgfscope}%
\end{pgfscope}%
\begin{pgfscope}%
\definecolor{textcolor}{rgb}{0.000000,0.000000,0.000000}%
\pgfsetstrokecolor{textcolor}%
\pgfsetfillcolor{textcolor}%
\pgftext[x=4.950722in, y=1.458441in, left, base]{\color{textcolor}\sffamily\fontsize{10.000000}{12.000000}\selectfont \(\displaystyle {100}\)}%
\end{pgfscope}%
\begin{pgfscope}%
\pgfsetbuttcap%
\pgfsetroundjoin%
\definecolor{currentfill}{rgb}{0.000000,0.000000,0.000000}%
\pgfsetfillcolor{currentfill}%
\pgfsetlinewidth{0.803000pt}%
\definecolor{currentstroke}{rgb}{0.000000,0.000000,0.000000}%
\pgfsetstrokecolor{currentstroke}%
\pgfsetdash{}{0pt}%
\pgfsys@defobject{currentmarker}{\pgfqpoint{0.000000in}{0.000000in}}{\pgfqpoint{0.048611in}{0.000000in}}{%
\pgfpathmoveto{\pgfqpoint{0.000000in}{0.000000in}}%
\pgfpathlineto{\pgfqpoint{0.048611in}{0.000000in}}%
\pgfusepath{stroke,fill}%
}%
\begin{pgfscope}%
\pgfsys@transformshift{4.853500in}{2.010000in}%
\pgfsys@useobject{currentmarker}{}%
\end{pgfscope}%
\end{pgfscope}%
\begin{pgfscope}%
\definecolor{textcolor}{rgb}{0.000000,0.000000,0.000000}%
\pgfsetstrokecolor{textcolor}%
\pgfsetfillcolor{textcolor}%
\pgftext[x=4.950722in, y=1.961775in, left, base]{\color{textcolor}\sffamily\fontsize{10.000000}{12.000000}\selectfont \(\displaystyle {150}\)}%
\end{pgfscope}%
\begin{pgfscope}%
\pgfsetbuttcap%
\pgfsetroundjoin%
\definecolor{currentfill}{rgb}{0.000000,0.000000,0.000000}%
\pgfsetfillcolor{currentfill}%
\pgfsetlinewidth{0.803000pt}%
\definecolor{currentstroke}{rgb}{0.000000,0.000000,0.000000}%
\pgfsetstrokecolor{currentstroke}%
\pgfsetdash{}{0pt}%
\pgfsys@defobject{currentmarker}{\pgfqpoint{0.000000in}{0.000000in}}{\pgfqpoint{0.048611in}{0.000000in}}{%
\pgfpathmoveto{\pgfqpoint{0.000000in}{0.000000in}}%
\pgfpathlineto{\pgfqpoint{0.048611in}{0.000000in}}%
\pgfusepath{stroke,fill}%
}%
\begin{pgfscope}%
\pgfsys@transformshift{4.853500in}{2.513333in}%
\pgfsys@useobject{currentmarker}{}%
\end{pgfscope}%
\end{pgfscope}%
\begin{pgfscope}%
\definecolor{textcolor}{rgb}{0.000000,0.000000,0.000000}%
\pgfsetstrokecolor{textcolor}%
\pgfsetfillcolor{textcolor}%
\pgftext[x=4.950722in, y=2.465108in, left, base]{\color{textcolor}\sffamily\fontsize{10.000000}{12.000000}\selectfont \(\displaystyle {200}\)}%
\end{pgfscope}%
\begin{pgfscope}%
\pgfsetbuttcap%
\pgfsetroundjoin%
\definecolor{currentfill}{rgb}{0.000000,0.000000,0.000000}%
\pgfsetfillcolor{currentfill}%
\pgfsetlinewidth{0.803000pt}%
\definecolor{currentstroke}{rgb}{0.000000,0.000000,0.000000}%
\pgfsetstrokecolor{currentstroke}%
\pgfsetdash{}{0pt}%
\pgfsys@defobject{currentmarker}{\pgfqpoint{0.000000in}{0.000000in}}{\pgfqpoint{0.048611in}{0.000000in}}{%
\pgfpathmoveto{\pgfqpoint{0.000000in}{0.000000in}}%
\pgfpathlineto{\pgfqpoint{0.048611in}{0.000000in}}%
\pgfusepath{stroke,fill}%
}%
\begin{pgfscope}%
\pgfsys@transformshift{4.853500in}{3.016667in}%
\pgfsys@useobject{currentmarker}{}%
\end{pgfscope}%
\end{pgfscope}%
\begin{pgfscope}%
\definecolor{textcolor}{rgb}{0.000000,0.000000,0.000000}%
\pgfsetstrokecolor{textcolor}%
\pgfsetfillcolor{textcolor}%
\pgftext[x=4.950722in, y=2.968441in, left, base]{\color{textcolor}\sffamily\fontsize{10.000000}{12.000000}\selectfont \(\displaystyle {250}\)}%
\end{pgfscope}%
\begin{pgfscope}%
\pgfsetbuttcap%
\pgfsetroundjoin%
\definecolor{currentfill}{rgb}{0.000000,0.000000,0.000000}%
\pgfsetfillcolor{currentfill}%
\pgfsetlinewidth{0.803000pt}%
\definecolor{currentstroke}{rgb}{0.000000,0.000000,0.000000}%
\pgfsetstrokecolor{currentstroke}%
\pgfsetdash{}{0pt}%
\pgfsys@defobject{currentmarker}{\pgfqpoint{0.000000in}{0.000000in}}{\pgfqpoint{0.048611in}{0.000000in}}{%
\pgfpathmoveto{\pgfqpoint{0.000000in}{0.000000in}}%
\pgfpathlineto{\pgfqpoint{0.048611in}{0.000000in}}%
\pgfusepath{stroke,fill}%
}%
\begin{pgfscope}%
\pgfsys@transformshift{4.853500in}{3.520000in}%
\pgfsys@useobject{currentmarker}{}%
\end{pgfscope}%
\end{pgfscope}%
\begin{pgfscope}%
\definecolor{textcolor}{rgb}{0.000000,0.000000,0.000000}%
\pgfsetstrokecolor{textcolor}%
\pgfsetfillcolor{textcolor}%
\pgftext[x=4.950722in, y=3.471775in, left, base]{\color{textcolor}\sffamily\fontsize{10.000000}{12.000000}\selectfont \(\displaystyle {300}\)}%
\end{pgfscope}%
\begin{pgfscope}%
\definecolor{textcolor}{rgb}{0.000000,0.000000,0.000000}%
\pgfsetstrokecolor{textcolor}%
\pgfsetfillcolor{textcolor}%
\pgftext[x=5.214612in,y=2.010000in,,top,rotate=90.000000]{\color{textcolor}\sffamily\fontsize{10.000000}{12.000000}\selectfont \(\displaystyle E \, \mathrm{[MeV]}\)}%
\end{pgfscope}%
\begin{pgfscope}%
\pgfsetrectcap%
\pgfsetmiterjoin%
\pgfsetlinewidth{0.803000pt}%
\definecolor{currentstroke}{rgb}{0.000000,0.000000,0.000000}%
\pgfsetstrokecolor{currentstroke}%
\pgfsetdash{}{0pt}%
\pgfpathmoveto{\pgfqpoint{4.702500in}{0.500000in}}%
\pgfpathlineto{\pgfqpoint{4.778000in}{0.500000in}}%
\pgfpathlineto{\pgfqpoint{4.853500in}{0.500000in}}%
\pgfpathlineto{\pgfqpoint{4.853500in}{3.520000in}}%
\pgfpathlineto{\pgfqpoint{4.778000in}{3.520000in}}%
\pgfpathlineto{\pgfqpoint{4.702500in}{3.520000in}}%
\pgfpathlineto{\pgfqpoint{4.702500in}{0.500000in}}%
\pgfpathclose%
\pgfusepath{stroke}%
\end{pgfscope}%
\end{pgfpicture}%
\makeatother%
\endgroup%

	\caption{Mean energy of the particles in the spatial bins.
	\textbf{(a)} ($y=$ \qty{2.19}{mm}) Energy map over $\zeta$-$z$- The bins measure \qtyproduct{0.16 x 0.45}{\um}. Every bin with at least one particle is plotted (So no information about the energy density can be derived). 
	Most energy is lost by particles in the center of the bunch at the peak decelerating field.
	\textbf{(b)} Mean energy of the particles in the bins over time. Only the bins around $z=0$ are plotted and averaged in $x$-direction for every timestep, showing the gradually loss of energy in the middle of the driver until bunch breakup.}
	\label{fig:E_time}
\end{figure}

Parts of the bunch, where a strong decelerating force acts, have lost most of the energy while the front part, where only weak forces act, retained the peak energy. The energy loss in the back consists not of a steady gradient but islands with constant energy, as seen by by the big areas with a constant color.
The complete time series of the energy is plotted in \autoref{fig:E_time}b.

Notable is the constant energy in the front half of the driver and the strong energy loss in the middle. When energies around zero are reached, the electrons fall back, where they get accelerated again from the first cavity.
Also notable is, that the highest energies are achieved at the back of the driver, where the accelerating part of the cavity starts and pushes the electrons forwards in propagation direction.

\subsection{Parameter comparison}
\paragraph*{Energy comparison}\hspace{0pt} \\
These results are again compared to different initial conditions of the driver. In \autoref{fig:E_peak_energy} the loss of peak energy over traveled distance is compared again for three different mean kinetic energies.
\begin{figure}
	\centering
	%% Creator: Matplotlib, PGF backend
%%
%% To include the figure in your LaTeX document, write
%%   \input{<filename>.pgf}
%%
%% Make sure the required packages are loaded in your preamble
%%   \usepackage{pgf}
%%
%% Also ensure that all the required font packages are loaded; for instance,
%% the lmodern package is sometimes necessary when using math font.
%%   \usepackage{lmodern}
%%
%% Figures using additional raster images can only be included by \input if
%% they are in the same directory as the main LaTeX file. For loading figures
%% from other directories you can use the `import` package
%%   \usepackage{import}
%%
%% and then include the figures with
%%   \import{<path to file>}{<filename>.pgf}
%%
%% Matplotlib used the following preamble
%%
\begingroup%
\makeatletter%
\begin{pgfpicture}%
\pgfpathrectangle{\pgfpointorigin}{\pgfqpoint{6.000000in}{4.000000in}}%
\pgfusepath{use as bounding box, clip}%
\begin{pgfscope}%
\pgfsetbuttcap%
\pgfsetmiterjoin%
\pgfsetlinewidth{0.000000pt}%
\definecolor{currentstroke}{rgb}{1.000000,1.000000,1.000000}%
\pgfsetstrokecolor{currentstroke}%
\pgfsetstrokeopacity{0.000000}%
\pgfsetdash{}{0pt}%
\pgfpathmoveto{\pgfqpoint{0.000000in}{0.000000in}}%
\pgfpathlineto{\pgfqpoint{6.000000in}{0.000000in}}%
\pgfpathlineto{\pgfqpoint{6.000000in}{4.000000in}}%
\pgfpathlineto{\pgfqpoint{0.000000in}{4.000000in}}%
\pgfpathlineto{\pgfqpoint{0.000000in}{0.000000in}}%
\pgfpathclose%
\pgfusepath{}%
\end{pgfscope}%
\begin{pgfscope}%
\pgfsetbuttcap%
\pgfsetmiterjoin%
\definecolor{currentfill}{rgb}{1.000000,1.000000,1.000000}%
\pgfsetfillcolor{currentfill}%
\pgfsetlinewidth{0.000000pt}%
\definecolor{currentstroke}{rgb}{0.000000,0.000000,0.000000}%
\pgfsetstrokecolor{currentstroke}%
\pgfsetstrokeopacity{0.000000}%
\pgfsetdash{}{0pt}%
\pgfpathmoveto{\pgfqpoint{0.750000in}{0.500000in}}%
\pgfpathlineto{\pgfqpoint{5.400000in}{0.500000in}}%
\pgfpathlineto{\pgfqpoint{5.400000in}{3.520000in}}%
\pgfpathlineto{\pgfqpoint{0.750000in}{3.520000in}}%
\pgfpathlineto{\pgfqpoint{0.750000in}{0.500000in}}%
\pgfpathclose%
\pgfusepath{fill}%
\end{pgfscope}%
\begin{pgfscope}%
\pgfsetbuttcap%
\pgfsetroundjoin%
\definecolor{currentfill}{rgb}{0.000000,0.000000,0.000000}%
\pgfsetfillcolor{currentfill}%
\pgfsetlinewidth{0.803000pt}%
\definecolor{currentstroke}{rgb}{0.000000,0.000000,0.000000}%
\pgfsetstrokecolor{currentstroke}%
\pgfsetdash{}{0pt}%
\pgfsys@defobject{currentmarker}{\pgfqpoint{0.000000in}{-0.048611in}}{\pgfqpoint{0.000000in}{0.000000in}}{%
\pgfpathmoveto{\pgfqpoint{0.000000in}{0.000000in}}%
\pgfpathlineto{\pgfqpoint{0.000000in}{-0.048611in}}%
\pgfusepath{stroke,fill}%
}%
\begin{pgfscope}%
\pgfsys@transformshift{0.987335in}{0.500000in}%
\pgfsys@useobject{currentmarker}{}%
\end{pgfscope}%
\end{pgfscope}%
\begin{pgfscope}%
\definecolor{textcolor}{rgb}{0.000000,0.000000,0.000000}%
\pgfsetstrokecolor{textcolor}%
\pgfsetfillcolor{textcolor}%
\pgftext[x=0.987335in,y=0.402778in,,top]{\color{textcolor}\sffamily\fontsize{10.000000}{12.000000}\selectfont \(\displaystyle {0}\)}%
\end{pgfscope}%
\begin{pgfscope}%
\pgfsetbuttcap%
\pgfsetroundjoin%
\definecolor{currentfill}{rgb}{0.000000,0.000000,0.000000}%
\pgfsetfillcolor{currentfill}%
\pgfsetlinewidth{0.803000pt}%
\definecolor{currentstroke}{rgb}{0.000000,0.000000,0.000000}%
\pgfsetstrokecolor{currentstroke}%
\pgfsetdash{}{0pt}%
\pgfsys@defobject{currentmarker}{\pgfqpoint{0.000000in}{-0.048611in}}{\pgfqpoint{0.000000in}{0.000000in}}{%
\pgfpathmoveto{\pgfqpoint{0.000000in}{0.000000in}}%
\pgfpathlineto{\pgfqpoint{0.000000in}{-0.048611in}}%
\pgfusepath{stroke,fill}%
}%
\begin{pgfscope}%
\pgfsys@transformshift{1.668860in}{0.500000in}%
\pgfsys@useobject{currentmarker}{}%
\end{pgfscope}%
\end{pgfscope}%
\begin{pgfscope}%
\definecolor{textcolor}{rgb}{0.000000,0.000000,0.000000}%
\pgfsetstrokecolor{textcolor}%
\pgfsetfillcolor{textcolor}%
\pgftext[x=1.668860in,y=0.402778in,,top]{\color{textcolor}\sffamily\fontsize{10.000000}{12.000000}\selectfont \(\displaystyle {1}\)}%
\end{pgfscope}%
\begin{pgfscope}%
\pgfsetbuttcap%
\pgfsetroundjoin%
\definecolor{currentfill}{rgb}{0.000000,0.000000,0.000000}%
\pgfsetfillcolor{currentfill}%
\pgfsetlinewidth{0.803000pt}%
\definecolor{currentstroke}{rgb}{0.000000,0.000000,0.000000}%
\pgfsetstrokecolor{currentstroke}%
\pgfsetdash{}{0pt}%
\pgfsys@defobject{currentmarker}{\pgfqpoint{0.000000in}{-0.048611in}}{\pgfqpoint{0.000000in}{0.000000in}}{%
\pgfpathmoveto{\pgfqpoint{0.000000in}{0.000000in}}%
\pgfpathlineto{\pgfqpoint{0.000000in}{-0.048611in}}%
\pgfusepath{stroke,fill}%
}%
\begin{pgfscope}%
\pgfsys@transformshift{2.350385in}{0.500000in}%
\pgfsys@useobject{currentmarker}{}%
\end{pgfscope}%
\end{pgfscope}%
\begin{pgfscope}%
\definecolor{textcolor}{rgb}{0.000000,0.000000,0.000000}%
\pgfsetstrokecolor{textcolor}%
\pgfsetfillcolor{textcolor}%
\pgftext[x=2.350385in,y=0.402778in,,top]{\color{textcolor}\sffamily\fontsize{10.000000}{12.000000}\selectfont \(\displaystyle {2}\)}%
\end{pgfscope}%
\begin{pgfscope}%
\pgfsetbuttcap%
\pgfsetroundjoin%
\definecolor{currentfill}{rgb}{0.000000,0.000000,0.000000}%
\pgfsetfillcolor{currentfill}%
\pgfsetlinewidth{0.803000pt}%
\definecolor{currentstroke}{rgb}{0.000000,0.000000,0.000000}%
\pgfsetstrokecolor{currentstroke}%
\pgfsetdash{}{0pt}%
\pgfsys@defobject{currentmarker}{\pgfqpoint{0.000000in}{-0.048611in}}{\pgfqpoint{0.000000in}{0.000000in}}{%
\pgfpathmoveto{\pgfqpoint{0.000000in}{0.000000in}}%
\pgfpathlineto{\pgfqpoint{0.000000in}{-0.048611in}}%
\pgfusepath{stroke,fill}%
}%
\begin{pgfscope}%
\pgfsys@transformshift{3.031909in}{0.500000in}%
\pgfsys@useobject{currentmarker}{}%
\end{pgfscope}%
\end{pgfscope}%
\begin{pgfscope}%
\definecolor{textcolor}{rgb}{0.000000,0.000000,0.000000}%
\pgfsetstrokecolor{textcolor}%
\pgfsetfillcolor{textcolor}%
\pgftext[x=3.031909in,y=0.402778in,,top]{\color{textcolor}\sffamily\fontsize{10.000000}{12.000000}\selectfont \(\displaystyle {3}\)}%
\end{pgfscope}%
\begin{pgfscope}%
\pgfsetbuttcap%
\pgfsetroundjoin%
\definecolor{currentfill}{rgb}{0.000000,0.000000,0.000000}%
\pgfsetfillcolor{currentfill}%
\pgfsetlinewidth{0.803000pt}%
\definecolor{currentstroke}{rgb}{0.000000,0.000000,0.000000}%
\pgfsetstrokecolor{currentstroke}%
\pgfsetdash{}{0pt}%
\pgfsys@defobject{currentmarker}{\pgfqpoint{0.000000in}{-0.048611in}}{\pgfqpoint{0.000000in}{0.000000in}}{%
\pgfpathmoveto{\pgfqpoint{0.000000in}{0.000000in}}%
\pgfpathlineto{\pgfqpoint{0.000000in}{-0.048611in}}%
\pgfusepath{stroke,fill}%
}%
\begin{pgfscope}%
\pgfsys@transformshift{3.713434in}{0.500000in}%
\pgfsys@useobject{currentmarker}{}%
\end{pgfscope}%
\end{pgfscope}%
\begin{pgfscope}%
\definecolor{textcolor}{rgb}{0.000000,0.000000,0.000000}%
\pgfsetstrokecolor{textcolor}%
\pgfsetfillcolor{textcolor}%
\pgftext[x=3.713434in,y=0.402778in,,top]{\color{textcolor}\sffamily\fontsize{10.000000}{12.000000}\selectfont \(\displaystyle {4}\)}%
\end{pgfscope}%
\begin{pgfscope}%
\pgfsetbuttcap%
\pgfsetroundjoin%
\definecolor{currentfill}{rgb}{0.000000,0.000000,0.000000}%
\pgfsetfillcolor{currentfill}%
\pgfsetlinewidth{0.803000pt}%
\definecolor{currentstroke}{rgb}{0.000000,0.000000,0.000000}%
\pgfsetstrokecolor{currentstroke}%
\pgfsetdash{}{0pt}%
\pgfsys@defobject{currentmarker}{\pgfqpoint{0.000000in}{-0.048611in}}{\pgfqpoint{0.000000in}{0.000000in}}{%
\pgfpathmoveto{\pgfqpoint{0.000000in}{0.000000in}}%
\pgfpathlineto{\pgfqpoint{0.000000in}{-0.048611in}}%
\pgfusepath{stroke,fill}%
}%
\begin{pgfscope}%
\pgfsys@transformshift{4.394959in}{0.500000in}%
\pgfsys@useobject{currentmarker}{}%
\end{pgfscope}%
\end{pgfscope}%
\begin{pgfscope}%
\definecolor{textcolor}{rgb}{0.000000,0.000000,0.000000}%
\pgfsetstrokecolor{textcolor}%
\pgfsetfillcolor{textcolor}%
\pgftext[x=4.394959in,y=0.402778in,,top]{\color{textcolor}\sffamily\fontsize{10.000000}{12.000000}\selectfont \(\displaystyle {5}\)}%
\end{pgfscope}%
\begin{pgfscope}%
\pgfsetbuttcap%
\pgfsetroundjoin%
\definecolor{currentfill}{rgb}{0.000000,0.000000,0.000000}%
\pgfsetfillcolor{currentfill}%
\pgfsetlinewidth{0.803000pt}%
\definecolor{currentstroke}{rgb}{0.000000,0.000000,0.000000}%
\pgfsetstrokecolor{currentstroke}%
\pgfsetdash{}{0pt}%
\pgfsys@defobject{currentmarker}{\pgfqpoint{0.000000in}{-0.048611in}}{\pgfqpoint{0.000000in}{0.000000in}}{%
\pgfpathmoveto{\pgfqpoint{0.000000in}{0.000000in}}%
\pgfpathlineto{\pgfqpoint{0.000000in}{-0.048611in}}%
\pgfusepath{stroke,fill}%
}%
\begin{pgfscope}%
\pgfsys@transformshift{5.076483in}{0.500000in}%
\pgfsys@useobject{currentmarker}{}%
\end{pgfscope}%
\end{pgfscope}%
\begin{pgfscope}%
\definecolor{textcolor}{rgb}{0.000000,0.000000,0.000000}%
\pgfsetstrokecolor{textcolor}%
\pgfsetfillcolor{textcolor}%
\pgftext[x=5.076483in,y=0.402778in,,top]{\color{textcolor}\sffamily\fontsize{10.000000}{12.000000}\selectfont \(\displaystyle {6}\)}%
\end{pgfscope}%
\begin{pgfscope}%
\definecolor{textcolor}{rgb}{0.000000,0.000000,0.000000}%
\pgfsetstrokecolor{textcolor}%
\pgfsetfillcolor{textcolor}%
\pgftext[x=3.075000in,y=0.223766in,,top]{\color{textcolor}\sffamily\fontsize{10.000000}{12.000000}\selectfont \(\displaystyle y \, \mathrm{[mm]}\)}%
\end{pgfscope}%
\begin{pgfscope}%
\pgfsetbuttcap%
\pgfsetroundjoin%
\definecolor{currentfill}{rgb}{0.000000,0.000000,0.000000}%
\pgfsetfillcolor{currentfill}%
\pgfsetlinewidth{0.803000pt}%
\definecolor{currentstroke}{rgb}{0.000000,0.000000,0.000000}%
\pgfsetstrokecolor{currentstroke}%
\pgfsetdash{}{0pt}%
\pgfsys@defobject{currentmarker}{\pgfqpoint{-0.048611in}{0.000000in}}{\pgfqpoint{-0.000000in}{0.000000in}}{%
\pgfpathmoveto{\pgfqpoint{-0.000000in}{0.000000in}}%
\pgfpathlineto{\pgfqpoint{-0.048611in}{0.000000in}}%
\pgfusepath{stroke,fill}%
}%
\begin{pgfscope}%
\pgfsys@transformshift{0.750000in}{0.814604in}%
\pgfsys@useobject{currentmarker}{}%
\end{pgfscope}%
\end{pgfscope}%
\begin{pgfscope}%
\definecolor{textcolor}{rgb}{0.000000,0.000000,0.000000}%
\pgfsetstrokecolor{textcolor}%
\pgfsetfillcolor{textcolor}%
\pgftext[x=0.405863in, y=0.766379in, left, base]{\color{textcolor}\sffamily\fontsize{10.000000}{12.000000}\selectfont \(\displaystyle {\ensuremath{-}10}\)}%
\end{pgfscope}%
\begin{pgfscope}%
\pgfsetbuttcap%
\pgfsetroundjoin%
\definecolor{currentfill}{rgb}{0.000000,0.000000,0.000000}%
\pgfsetfillcolor{currentfill}%
\pgfsetlinewidth{0.803000pt}%
\definecolor{currentstroke}{rgb}{0.000000,0.000000,0.000000}%
\pgfsetstrokecolor{currentstroke}%
\pgfsetdash{}{0pt}%
\pgfsys@defobject{currentmarker}{\pgfqpoint{-0.048611in}{0.000000in}}{\pgfqpoint{-0.000000in}{0.000000in}}{%
\pgfpathmoveto{\pgfqpoint{-0.000000in}{0.000000in}}%
\pgfpathlineto{\pgfqpoint{-0.048611in}{0.000000in}}%
\pgfusepath{stroke,fill}%
}%
\begin{pgfscope}%
\pgfsys@transformshift{0.750000in}{1.326135in}%
\pgfsys@useobject{currentmarker}{}%
\end{pgfscope}%
\end{pgfscope}%
\begin{pgfscope}%
\definecolor{textcolor}{rgb}{0.000000,0.000000,0.000000}%
\pgfsetstrokecolor{textcolor}%
\pgfsetfillcolor{textcolor}%
\pgftext[x=0.475308in, y=1.277910in, left, base]{\color{textcolor}\sffamily\fontsize{10.000000}{12.000000}\selectfont \(\displaystyle {\ensuremath{-}8}\)}%
\end{pgfscope}%
\begin{pgfscope}%
\pgfsetbuttcap%
\pgfsetroundjoin%
\definecolor{currentfill}{rgb}{0.000000,0.000000,0.000000}%
\pgfsetfillcolor{currentfill}%
\pgfsetlinewidth{0.803000pt}%
\definecolor{currentstroke}{rgb}{0.000000,0.000000,0.000000}%
\pgfsetstrokecolor{currentstroke}%
\pgfsetdash{}{0pt}%
\pgfsys@defobject{currentmarker}{\pgfqpoint{-0.048611in}{0.000000in}}{\pgfqpoint{-0.000000in}{0.000000in}}{%
\pgfpathmoveto{\pgfqpoint{-0.000000in}{0.000000in}}%
\pgfpathlineto{\pgfqpoint{-0.048611in}{0.000000in}}%
\pgfusepath{stroke,fill}%
}%
\begin{pgfscope}%
\pgfsys@transformshift{0.750000in}{1.837666in}%
\pgfsys@useobject{currentmarker}{}%
\end{pgfscope}%
\end{pgfscope}%
\begin{pgfscope}%
\definecolor{textcolor}{rgb}{0.000000,0.000000,0.000000}%
\pgfsetstrokecolor{textcolor}%
\pgfsetfillcolor{textcolor}%
\pgftext[x=0.475308in, y=1.789440in, left, base]{\color{textcolor}\sffamily\fontsize{10.000000}{12.000000}\selectfont \(\displaystyle {\ensuremath{-}6}\)}%
\end{pgfscope}%
\begin{pgfscope}%
\pgfsetbuttcap%
\pgfsetroundjoin%
\definecolor{currentfill}{rgb}{0.000000,0.000000,0.000000}%
\pgfsetfillcolor{currentfill}%
\pgfsetlinewidth{0.803000pt}%
\definecolor{currentstroke}{rgb}{0.000000,0.000000,0.000000}%
\pgfsetstrokecolor{currentstroke}%
\pgfsetdash{}{0pt}%
\pgfsys@defobject{currentmarker}{\pgfqpoint{-0.048611in}{0.000000in}}{\pgfqpoint{-0.000000in}{0.000000in}}{%
\pgfpathmoveto{\pgfqpoint{-0.000000in}{0.000000in}}%
\pgfpathlineto{\pgfqpoint{-0.048611in}{0.000000in}}%
\pgfusepath{stroke,fill}%
}%
\begin{pgfscope}%
\pgfsys@transformshift{0.750000in}{2.349196in}%
\pgfsys@useobject{currentmarker}{}%
\end{pgfscope}%
\end{pgfscope}%
\begin{pgfscope}%
\definecolor{textcolor}{rgb}{0.000000,0.000000,0.000000}%
\pgfsetstrokecolor{textcolor}%
\pgfsetfillcolor{textcolor}%
\pgftext[x=0.475308in, y=2.300971in, left, base]{\color{textcolor}\sffamily\fontsize{10.000000}{12.000000}\selectfont \(\displaystyle {\ensuremath{-}4}\)}%
\end{pgfscope}%
\begin{pgfscope}%
\pgfsetbuttcap%
\pgfsetroundjoin%
\definecolor{currentfill}{rgb}{0.000000,0.000000,0.000000}%
\pgfsetfillcolor{currentfill}%
\pgfsetlinewidth{0.803000pt}%
\definecolor{currentstroke}{rgb}{0.000000,0.000000,0.000000}%
\pgfsetstrokecolor{currentstroke}%
\pgfsetdash{}{0pt}%
\pgfsys@defobject{currentmarker}{\pgfqpoint{-0.048611in}{0.000000in}}{\pgfqpoint{-0.000000in}{0.000000in}}{%
\pgfpathmoveto{\pgfqpoint{-0.000000in}{0.000000in}}%
\pgfpathlineto{\pgfqpoint{-0.048611in}{0.000000in}}%
\pgfusepath{stroke,fill}%
}%
\begin{pgfscope}%
\pgfsys@transformshift{0.750000in}{2.860727in}%
\pgfsys@useobject{currentmarker}{}%
\end{pgfscope}%
\end{pgfscope}%
\begin{pgfscope}%
\definecolor{textcolor}{rgb}{0.000000,0.000000,0.000000}%
\pgfsetstrokecolor{textcolor}%
\pgfsetfillcolor{textcolor}%
\pgftext[x=0.475308in, y=2.812502in, left, base]{\color{textcolor}\sffamily\fontsize{10.000000}{12.000000}\selectfont \(\displaystyle {\ensuremath{-}2}\)}%
\end{pgfscope}%
\begin{pgfscope}%
\pgfsetbuttcap%
\pgfsetroundjoin%
\definecolor{currentfill}{rgb}{0.000000,0.000000,0.000000}%
\pgfsetfillcolor{currentfill}%
\pgfsetlinewidth{0.803000pt}%
\definecolor{currentstroke}{rgb}{0.000000,0.000000,0.000000}%
\pgfsetstrokecolor{currentstroke}%
\pgfsetdash{}{0pt}%
\pgfsys@defobject{currentmarker}{\pgfqpoint{-0.048611in}{0.000000in}}{\pgfqpoint{-0.000000in}{0.000000in}}{%
\pgfpathmoveto{\pgfqpoint{-0.000000in}{0.000000in}}%
\pgfpathlineto{\pgfqpoint{-0.048611in}{0.000000in}}%
\pgfusepath{stroke,fill}%
}%
\begin{pgfscope}%
\pgfsys@transformshift{0.750000in}{3.372258in}%
\pgfsys@useobject{currentmarker}{}%
\end{pgfscope}%
\end{pgfscope}%
\begin{pgfscope}%
\definecolor{textcolor}{rgb}{0.000000,0.000000,0.000000}%
\pgfsetstrokecolor{textcolor}%
\pgfsetfillcolor{textcolor}%
\pgftext[x=0.583333in, y=3.324033in, left, base]{\color{textcolor}\sffamily\fontsize{10.000000}{12.000000}\selectfont \(\displaystyle {0}\)}%
\end{pgfscope}%
\begin{pgfscope}%
\definecolor{textcolor}{rgb}{0.000000,0.000000,0.000000}%
\pgfsetstrokecolor{textcolor}%
\pgfsetfillcolor{textcolor}%
\pgftext[x=0.350308in,y=2.010000in,,bottom,rotate=90.000000]{\color{textcolor}\sffamily\fontsize{10.000000}{12.000000}\selectfont \(\displaystyle \mathrm{energy \, change \, [MeV]}\)}%
\end{pgfscope}%
\begin{pgfscope}%
\pgfpathrectangle{\pgfqpoint{0.750000in}{0.500000in}}{\pgfqpoint{4.650000in}{3.020000in}}%
\pgfusepath{clip}%
\pgfsetrectcap%
\pgfsetroundjoin%
\pgfsetlinewidth{1.505625pt}%
\definecolor{currentstroke}{rgb}{0.121569,0.466667,0.705882}%
\pgfsetstrokecolor{currentstroke}%
\pgfsetdash{}{0pt}%
\pgfpathmoveto{\pgfqpoint{0.961364in}{3.191373in}}%
\pgfpathlineto{\pgfqpoint{1.015559in}{3.382727in}}%
\pgfpathlineto{\pgfqpoint{1.069755in}{3.377489in}}%
\pgfpathlineto{\pgfqpoint{1.123951in}{3.027340in}}%
\pgfpathlineto{\pgfqpoint{1.178147in}{2.933273in}}%
\pgfpathlineto{\pgfqpoint{1.232343in}{2.545400in}}%
\pgfpathlineto{\pgfqpoint{1.286538in}{2.291084in}}%
\pgfpathlineto{\pgfqpoint{1.340734in}{2.054794in}}%
\pgfpathlineto{\pgfqpoint{1.394930in}{1.907429in}}%
\pgfpathlineto{\pgfqpoint{1.449126in}{1.831107in}}%
\pgfpathlineto{\pgfqpoint{1.503322in}{1.798050in}}%
\pgfpathlineto{\pgfqpoint{1.557517in}{1.766096in}}%
\pgfpathlineto{\pgfqpoint{1.611713in}{1.751473in}}%
\pgfpathlineto{\pgfqpoint{1.665909in}{1.744043in}}%
\pgfpathlineto{\pgfqpoint{1.720105in}{1.753838in}}%
\pgfpathlineto{\pgfqpoint{1.774301in}{1.768361in}}%
\pgfpathlineto{\pgfqpoint{1.828497in}{1.770624in}}%
\pgfpathlineto{\pgfqpoint{1.882692in}{1.780635in}}%
\pgfpathlineto{\pgfqpoint{1.936888in}{1.785779in}}%
\pgfpathlineto{\pgfqpoint{1.991084in}{1.779556in}}%
\pgfpathlineto{\pgfqpoint{2.045280in}{1.782064in}}%
\pgfpathlineto{\pgfqpoint{2.099476in}{1.777949in}}%
\pgfpathlineto{\pgfqpoint{2.153671in}{1.770297in}}%
\pgfpathlineto{\pgfqpoint{2.207867in}{1.769834in}}%
\pgfpathlineto{\pgfqpoint{2.262063in}{1.773450in}}%
\pgfpathlineto{\pgfqpoint{2.316259in}{1.775648in}}%
\pgfpathlineto{\pgfqpoint{2.370455in}{1.782244in}}%
\pgfpathlineto{\pgfqpoint{2.424650in}{1.790187in}}%
\pgfpathlineto{\pgfqpoint{2.478846in}{1.791957in}}%
\pgfpathlineto{\pgfqpoint{2.533042in}{1.787553in}}%
\pgfpathlineto{\pgfqpoint{2.587238in}{1.780791in}}%
\pgfpathlineto{\pgfqpoint{2.641434in}{1.773118in}}%
\pgfpathlineto{\pgfqpoint{2.695629in}{1.764811in}}%
\pgfpathlineto{\pgfqpoint{2.749825in}{1.756946in}}%
\pgfpathlineto{\pgfqpoint{2.804021in}{1.746600in}}%
\pgfpathlineto{\pgfqpoint{2.858217in}{1.739177in}}%
\pgfpathlineto{\pgfqpoint{2.912413in}{1.727594in}}%
\pgfpathlineto{\pgfqpoint{2.966608in}{1.712079in}}%
\pgfpathlineto{\pgfqpoint{3.020804in}{1.695777in}}%
\pgfpathlineto{\pgfqpoint{3.075000in}{1.679905in}}%
\pgfpathlineto{\pgfqpoint{3.129196in}{1.664768in}}%
\pgfpathlineto{\pgfqpoint{3.183392in}{1.655070in}}%
\pgfpathlineto{\pgfqpoint{3.237587in}{1.646072in}}%
\pgfpathlineto{\pgfqpoint{3.291783in}{1.640754in}}%
\pgfpathlineto{\pgfqpoint{3.345979in}{1.631556in}}%
\pgfpathlineto{\pgfqpoint{3.400175in}{1.622195in}}%
\pgfpathlineto{\pgfqpoint{3.454371in}{1.614368in}}%
\pgfpathlineto{\pgfqpoint{3.508566in}{1.607261in}}%
\pgfpathlineto{\pgfqpoint{3.562762in}{1.596524in}}%
\pgfpathlineto{\pgfqpoint{3.616958in}{1.580926in}}%
\pgfpathlineto{\pgfqpoint{3.671154in}{1.569108in}}%
\pgfpathlineto{\pgfqpoint{3.725350in}{1.558539in}}%
\pgfpathlineto{\pgfqpoint{3.779545in}{1.545254in}}%
\pgfpathlineto{\pgfqpoint{3.833741in}{1.534766in}}%
\pgfpathlineto{\pgfqpoint{3.887937in}{1.523993in}}%
\pgfpathlineto{\pgfqpoint{3.942133in}{1.509376in}}%
\pgfpathlineto{\pgfqpoint{3.996329in}{1.491809in}}%
\pgfpathlineto{\pgfqpoint{4.050524in}{1.472652in}}%
\pgfpathlineto{\pgfqpoint{4.104720in}{1.454979in}}%
\pgfpathlineto{\pgfqpoint{4.158916in}{1.441849in}}%
\pgfpathlineto{\pgfqpoint{4.213112in}{1.428742in}}%
\pgfpathlineto{\pgfqpoint{4.267308in}{1.412374in}}%
\pgfpathlineto{\pgfqpoint{4.321503in}{1.399228in}}%
\pgfpathlineto{\pgfqpoint{4.375699in}{1.386692in}}%
\pgfpathlineto{\pgfqpoint{4.429895in}{1.372435in}}%
\pgfpathlineto{\pgfqpoint{4.484091in}{1.358223in}}%
\pgfpathlineto{\pgfqpoint{4.538287in}{1.344812in}}%
\pgfpathlineto{\pgfqpoint{4.592483in}{1.337336in}}%
\pgfpathlineto{\pgfqpoint{4.646678in}{1.320841in}}%
\pgfpathlineto{\pgfqpoint{4.700874in}{1.307174in}}%
\pgfpathlineto{\pgfqpoint{4.755070in}{1.291187in}}%
\pgfpathlineto{\pgfqpoint{4.809266in}{1.278122in}}%
\pgfpathlineto{\pgfqpoint{4.863462in}{1.264996in}}%
\pgfpathlineto{\pgfqpoint{4.917657in}{1.239940in}}%
\pgfpathlineto{\pgfqpoint{4.971853in}{1.225819in}}%
\pgfpathlineto{\pgfqpoint{5.026049in}{1.213697in}}%
\pgfpathlineto{\pgfqpoint{5.080245in}{1.200434in}}%
\pgfpathlineto{\pgfqpoint{5.134441in}{1.185937in}}%
\pgfpathlineto{\pgfqpoint{5.188636in}{1.167326in}}%
\pgfusepath{stroke}%
\end{pgfscope}%
\begin{pgfscope}%
\pgfpathrectangle{\pgfqpoint{0.750000in}{0.500000in}}{\pgfqpoint{4.650000in}{3.020000in}}%
\pgfusepath{clip}%
\pgfsetbuttcap%
\pgfsetroundjoin%
\definecolor{currentfill}{rgb}{0.121569,0.466667,0.705882}%
\pgfsetfillcolor{currentfill}%
\pgfsetlinewidth{1.003750pt}%
\definecolor{currentstroke}{rgb}{0.121569,0.466667,0.705882}%
\pgfsetstrokecolor{currentstroke}%
\pgfsetdash{}{0pt}%
\pgfsys@defobject{currentmarker}{\pgfqpoint{-0.020833in}{-0.020833in}}{\pgfqpoint{0.020833in}{0.020833in}}{%
\pgfpathmoveto{\pgfqpoint{0.000000in}{-0.020833in}}%
\pgfpathcurveto{\pgfqpoint{0.005525in}{-0.020833in}}{\pgfqpoint{0.010825in}{-0.018638in}}{\pgfqpoint{0.014731in}{-0.014731in}}%
\pgfpathcurveto{\pgfqpoint{0.018638in}{-0.010825in}}{\pgfqpoint{0.020833in}{-0.005525in}}{\pgfqpoint{0.020833in}{0.000000in}}%
\pgfpathcurveto{\pgfqpoint{0.020833in}{0.005525in}}{\pgfqpoint{0.018638in}{0.010825in}}{\pgfqpoint{0.014731in}{0.014731in}}%
\pgfpathcurveto{\pgfqpoint{0.010825in}{0.018638in}}{\pgfqpoint{0.005525in}{0.020833in}}{\pgfqpoint{0.000000in}{0.020833in}}%
\pgfpathcurveto{\pgfqpoint{-0.005525in}{0.020833in}}{\pgfqpoint{-0.010825in}{0.018638in}}{\pgfqpoint{-0.014731in}{0.014731in}}%
\pgfpathcurveto{\pgfqpoint{-0.018638in}{0.010825in}}{\pgfqpoint{-0.020833in}{0.005525in}}{\pgfqpoint{-0.020833in}{0.000000in}}%
\pgfpathcurveto{\pgfqpoint{-0.020833in}{-0.005525in}}{\pgfqpoint{-0.018638in}{-0.010825in}}{\pgfqpoint{-0.014731in}{-0.014731in}}%
\pgfpathcurveto{\pgfqpoint{-0.010825in}{-0.018638in}}{\pgfqpoint{-0.005525in}{-0.020833in}}{\pgfqpoint{0.000000in}{-0.020833in}}%
\pgfpathlineto{\pgfqpoint{0.000000in}{-0.020833in}}%
\pgfpathclose%
\pgfusepath{stroke,fill}%
}%
\begin{pgfscope}%
\pgfsys@transformshift{0.961364in}{3.191373in}%
\pgfsys@useobject{currentmarker}{}%
\end{pgfscope}%
\begin{pgfscope}%
\pgfsys@transformshift{1.015559in}{3.382727in}%
\pgfsys@useobject{currentmarker}{}%
\end{pgfscope}%
\begin{pgfscope}%
\pgfsys@transformshift{1.069755in}{3.377489in}%
\pgfsys@useobject{currentmarker}{}%
\end{pgfscope}%
\begin{pgfscope}%
\pgfsys@transformshift{1.123951in}{3.027340in}%
\pgfsys@useobject{currentmarker}{}%
\end{pgfscope}%
\begin{pgfscope}%
\pgfsys@transformshift{1.178147in}{2.933273in}%
\pgfsys@useobject{currentmarker}{}%
\end{pgfscope}%
\begin{pgfscope}%
\pgfsys@transformshift{1.232343in}{2.545400in}%
\pgfsys@useobject{currentmarker}{}%
\end{pgfscope}%
\begin{pgfscope}%
\pgfsys@transformshift{1.286538in}{2.291084in}%
\pgfsys@useobject{currentmarker}{}%
\end{pgfscope}%
\begin{pgfscope}%
\pgfsys@transformshift{1.340734in}{2.054794in}%
\pgfsys@useobject{currentmarker}{}%
\end{pgfscope}%
\begin{pgfscope}%
\pgfsys@transformshift{1.394930in}{1.907429in}%
\pgfsys@useobject{currentmarker}{}%
\end{pgfscope}%
\begin{pgfscope}%
\pgfsys@transformshift{1.449126in}{1.831107in}%
\pgfsys@useobject{currentmarker}{}%
\end{pgfscope}%
\begin{pgfscope}%
\pgfsys@transformshift{1.503322in}{1.798050in}%
\pgfsys@useobject{currentmarker}{}%
\end{pgfscope}%
\begin{pgfscope}%
\pgfsys@transformshift{1.557517in}{1.766096in}%
\pgfsys@useobject{currentmarker}{}%
\end{pgfscope}%
\begin{pgfscope}%
\pgfsys@transformshift{1.611713in}{1.751473in}%
\pgfsys@useobject{currentmarker}{}%
\end{pgfscope}%
\begin{pgfscope}%
\pgfsys@transformshift{1.665909in}{1.744043in}%
\pgfsys@useobject{currentmarker}{}%
\end{pgfscope}%
\begin{pgfscope}%
\pgfsys@transformshift{1.720105in}{1.753838in}%
\pgfsys@useobject{currentmarker}{}%
\end{pgfscope}%
\begin{pgfscope}%
\pgfsys@transformshift{1.774301in}{1.768361in}%
\pgfsys@useobject{currentmarker}{}%
\end{pgfscope}%
\begin{pgfscope}%
\pgfsys@transformshift{1.828497in}{1.770624in}%
\pgfsys@useobject{currentmarker}{}%
\end{pgfscope}%
\begin{pgfscope}%
\pgfsys@transformshift{1.882692in}{1.780635in}%
\pgfsys@useobject{currentmarker}{}%
\end{pgfscope}%
\begin{pgfscope}%
\pgfsys@transformshift{1.936888in}{1.785779in}%
\pgfsys@useobject{currentmarker}{}%
\end{pgfscope}%
\begin{pgfscope}%
\pgfsys@transformshift{1.991084in}{1.779556in}%
\pgfsys@useobject{currentmarker}{}%
\end{pgfscope}%
\begin{pgfscope}%
\pgfsys@transformshift{2.045280in}{1.782064in}%
\pgfsys@useobject{currentmarker}{}%
\end{pgfscope}%
\begin{pgfscope}%
\pgfsys@transformshift{2.099476in}{1.777949in}%
\pgfsys@useobject{currentmarker}{}%
\end{pgfscope}%
\begin{pgfscope}%
\pgfsys@transformshift{2.153671in}{1.770297in}%
\pgfsys@useobject{currentmarker}{}%
\end{pgfscope}%
\begin{pgfscope}%
\pgfsys@transformshift{2.207867in}{1.769834in}%
\pgfsys@useobject{currentmarker}{}%
\end{pgfscope}%
\begin{pgfscope}%
\pgfsys@transformshift{2.262063in}{1.773450in}%
\pgfsys@useobject{currentmarker}{}%
\end{pgfscope}%
\begin{pgfscope}%
\pgfsys@transformshift{2.316259in}{1.775648in}%
\pgfsys@useobject{currentmarker}{}%
\end{pgfscope}%
\begin{pgfscope}%
\pgfsys@transformshift{2.370455in}{1.782244in}%
\pgfsys@useobject{currentmarker}{}%
\end{pgfscope}%
\begin{pgfscope}%
\pgfsys@transformshift{2.424650in}{1.790187in}%
\pgfsys@useobject{currentmarker}{}%
\end{pgfscope}%
\begin{pgfscope}%
\pgfsys@transformshift{2.478846in}{1.791957in}%
\pgfsys@useobject{currentmarker}{}%
\end{pgfscope}%
\begin{pgfscope}%
\pgfsys@transformshift{2.533042in}{1.787553in}%
\pgfsys@useobject{currentmarker}{}%
\end{pgfscope}%
\begin{pgfscope}%
\pgfsys@transformshift{2.587238in}{1.780791in}%
\pgfsys@useobject{currentmarker}{}%
\end{pgfscope}%
\begin{pgfscope}%
\pgfsys@transformshift{2.641434in}{1.773118in}%
\pgfsys@useobject{currentmarker}{}%
\end{pgfscope}%
\begin{pgfscope}%
\pgfsys@transformshift{2.695629in}{1.764811in}%
\pgfsys@useobject{currentmarker}{}%
\end{pgfscope}%
\begin{pgfscope}%
\pgfsys@transformshift{2.749825in}{1.756946in}%
\pgfsys@useobject{currentmarker}{}%
\end{pgfscope}%
\begin{pgfscope}%
\pgfsys@transformshift{2.804021in}{1.746600in}%
\pgfsys@useobject{currentmarker}{}%
\end{pgfscope}%
\begin{pgfscope}%
\pgfsys@transformshift{2.858217in}{1.739177in}%
\pgfsys@useobject{currentmarker}{}%
\end{pgfscope}%
\begin{pgfscope}%
\pgfsys@transformshift{2.912413in}{1.727594in}%
\pgfsys@useobject{currentmarker}{}%
\end{pgfscope}%
\begin{pgfscope}%
\pgfsys@transformshift{2.966608in}{1.712079in}%
\pgfsys@useobject{currentmarker}{}%
\end{pgfscope}%
\begin{pgfscope}%
\pgfsys@transformshift{3.020804in}{1.695777in}%
\pgfsys@useobject{currentmarker}{}%
\end{pgfscope}%
\begin{pgfscope}%
\pgfsys@transformshift{3.075000in}{1.679905in}%
\pgfsys@useobject{currentmarker}{}%
\end{pgfscope}%
\begin{pgfscope}%
\pgfsys@transformshift{3.129196in}{1.664768in}%
\pgfsys@useobject{currentmarker}{}%
\end{pgfscope}%
\begin{pgfscope}%
\pgfsys@transformshift{3.183392in}{1.655070in}%
\pgfsys@useobject{currentmarker}{}%
\end{pgfscope}%
\begin{pgfscope}%
\pgfsys@transformshift{3.237587in}{1.646072in}%
\pgfsys@useobject{currentmarker}{}%
\end{pgfscope}%
\begin{pgfscope}%
\pgfsys@transformshift{3.291783in}{1.640754in}%
\pgfsys@useobject{currentmarker}{}%
\end{pgfscope}%
\begin{pgfscope}%
\pgfsys@transformshift{3.345979in}{1.631556in}%
\pgfsys@useobject{currentmarker}{}%
\end{pgfscope}%
\begin{pgfscope}%
\pgfsys@transformshift{3.400175in}{1.622195in}%
\pgfsys@useobject{currentmarker}{}%
\end{pgfscope}%
\begin{pgfscope}%
\pgfsys@transformshift{3.454371in}{1.614368in}%
\pgfsys@useobject{currentmarker}{}%
\end{pgfscope}%
\begin{pgfscope}%
\pgfsys@transformshift{3.508566in}{1.607261in}%
\pgfsys@useobject{currentmarker}{}%
\end{pgfscope}%
\begin{pgfscope}%
\pgfsys@transformshift{3.562762in}{1.596524in}%
\pgfsys@useobject{currentmarker}{}%
\end{pgfscope}%
\begin{pgfscope}%
\pgfsys@transformshift{3.616958in}{1.580926in}%
\pgfsys@useobject{currentmarker}{}%
\end{pgfscope}%
\begin{pgfscope}%
\pgfsys@transformshift{3.671154in}{1.569108in}%
\pgfsys@useobject{currentmarker}{}%
\end{pgfscope}%
\begin{pgfscope}%
\pgfsys@transformshift{3.725350in}{1.558539in}%
\pgfsys@useobject{currentmarker}{}%
\end{pgfscope}%
\begin{pgfscope}%
\pgfsys@transformshift{3.779545in}{1.545254in}%
\pgfsys@useobject{currentmarker}{}%
\end{pgfscope}%
\begin{pgfscope}%
\pgfsys@transformshift{3.833741in}{1.534766in}%
\pgfsys@useobject{currentmarker}{}%
\end{pgfscope}%
\begin{pgfscope}%
\pgfsys@transformshift{3.887937in}{1.523993in}%
\pgfsys@useobject{currentmarker}{}%
\end{pgfscope}%
\begin{pgfscope}%
\pgfsys@transformshift{3.942133in}{1.509376in}%
\pgfsys@useobject{currentmarker}{}%
\end{pgfscope}%
\begin{pgfscope}%
\pgfsys@transformshift{3.996329in}{1.491809in}%
\pgfsys@useobject{currentmarker}{}%
\end{pgfscope}%
\begin{pgfscope}%
\pgfsys@transformshift{4.050524in}{1.472652in}%
\pgfsys@useobject{currentmarker}{}%
\end{pgfscope}%
\begin{pgfscope}%
\pgfsys@transformshift{4.104720in}{1.454979in}%
\pgfsys@useobject{currentmarker}{}%
\end{pgfscope}%
\begin{pgfscope}%
\pgfsys@transformshift{4.158916in}{1.441849in}%
\pgfsys@useobject{currentmarker}{}%
\end{pgfscope}%
\begin{pgfscope}%
\pgfsys@transformshift{4.213112in}{1.428742in}%
\pgfsys@useobject{currentmarker}{}%
\end{pgfscope}%
\begin{pgfscope}%
\pgfsys@transformshift{4.267308in}{1.412374in}%
\pgfsys@useobject{currentmarker}{}%
\end{pgfscope}%
\begin{pgfscope}%
\pgfsys@transformshift{4.321503in}{1.399228in}%
\pgfsys@useobject{currentmarker}{}%
\end{pgfscope}%
\begin{pgfscope}%
\pgfsys@transformshift{4.375699in}{1.386692in}%
\pgfsys@useobject{currentmarker}{}%
\end{pgfscope}%
\begin{pgfscope}%
\pgfsys@transformshift{4.429895in}{1.372435in}%
\pgfsys@useobject{currentmarker}{}%
\end{pgfscope}%
\begin{pgfscope}%
\pgfsys@transformshift{4.484091in}{1.358223in}%
\pgfsys@useobject{currentmarker}{}%
\end{pgfscope}%
\begin{pgfscope}%
\pgfsys@transformshift{4.538287in}{1.344812in}%
\pgfsys@useobject{currentmarker}{}%
\end{pgfscope}%
\begin{pgfscope}%
\pgfsys@transformshift{4.592483in}{1.337336in}%
\pgfsys@useobject{currentmarker}{}%
\end{pgfscope}%
\begin{pgfscope}%
\pgfsys@transformshift{4.646678in}{1.320841in}%
\pgfsys@useobject{currentmarker}{}%
\end{pgfscope}%
\begin{pgfscope}%
\pgfsys@transformshift{4.700874in}{1.307174in}%
\pgfsys@useobject{currentmarker}{}%
\end{pgfscope}%
\begin{pgfscope}%
\pgfsys@transformshift{4.755070in}{1.291187in}%
\pgfsys@useobject{currentmarker}{}%
\end{pgfscope}%
\begin{pgfscope}%
\pgfsys@transformshift{4.809266in}{1.278122in}%
\pgfsys@useobject{currentmarker}{}%
\end{pgfscope}%
\begin{pgfscope}%
\pgfsys@transformshift{4.863462in}{1.264996in}%
\pgfsys@useobject{currentmarker}{}%
\end{pgfscope}%
\begin{pgfscope}%
\pgfsys@transformshift{4.917657in}{1.239940in}%
\pgfsys@useobject{currentmarker}{}%
\end{pgfscope}%
\begin{pgfscope}%
\pgfsys@transformshift{4.971853in}{1.225819in}%
\pgfsys@useobject{currentmarker}{}%
\end{pgfscope}%
\begin{pgfscope}%
\pgfsys@transformshift{5.026049in}{1.213697in}%
\pgfsys@useobject{currentmarker}{}%
\end{pgfscope}%
\begin{pgfscope}%
\pgfsys@transformshift{5.080245in}{1.200434in}%
\pgfsys@useobject{currentmarker}{}%
\end{pgfscope}%
\begin{pgfscope}%
\pgfsys@transformshift{5.134441in}{1.185937in}%
\pgfsys@useobject{currentmarker}{}%
\end{pgfscope}%
\begin{pgfscope}%
\pgfsys@transformshift{5.188636in}{1.167326in}%
\pgfsys@useobject{currentmarker}{}%
\end{pgfscope}%
\end{pgfscope}%
\begin{pgfscope}%
\pgfpathrectangle{\pgfqpoint{0.750000in}{0.500000in}}{\pgfqpoint{4.650000in}{3.020000in}}%
\pgfusepath{clip}%
\pgfsetrectcap%
\pgfsetroundjoin%
\pgfsetlinewidth{1.505625pt}%
\definecolor{currentstroke}{rgb}{1.000000,0.498039,0.054902}%
\pgfsetstrokecolor{currentstroke}%
\pgfsetdash{}{0pt}%
\pgfpathmoveto{\pgfqpoint{0.961364in}{3.164949in}}%
\pgfpathlineto{\pgfqpoint{1.015559in}{2.969692in}}%
\pgfpathlineto{\pgfqpoint{1.069755in}{2.689040in}}%
\pgfpathlineto{\pgfqpoint{1.123951in}{2.390671in}}%
\pgfpathlineto{\pgfqpoint{1.178147in}{2.213505in}}%
\pgfpathlineto{\pgfqpoint{1.232343in}{2.108799in}}%
\pgfpathlineto{\pgfqpoint{1.286538in}{2.377356in}}%
\pgfpathlineto{\pgfqpoint{1.340734in}{2.139988in}}%
\pgfpathlineto{\pgfqpoint{1.394930in}{1.963422in}}%
\pgfpathlineto{\pgfqpoint{1.449126in}{1.821533in}}%
\pgfpathlineto{\pgfqpoint{1.503322in}{1.762046in}}%
\pgfpathlineto{\pgfqpoint{1.557517in}{1.719007in}}%
\pgfpathlineto{\pgfqpoint{1.611713in}{1.676891in}}%
\pgfpathlineto{\pgfqpoint{1.665909in}{1.646645in}}%
\pgfpathlineto{\pgfqpoint{1.720105in}{1.633854in}}%
\pgfpathlineto{\pgfqpoint{1.774301in}{1.627147in}}%
\pgfpathlineto{\pgfqpoint{1.828497in}{1.626923in}}%
\pgfpathlineto{\pgfqpoint{1.882692in}{1.631909in}}%
\pgfpathlineto{\pgfqpoint{1.936888in}{1.634876in}}%
\pgfpathlineto{\pgfqpoint{1.991084in}{1.635008in}}%
\pgfpathlineto{\pgfqpoint{2.045280in}{1.635919in}}%
\pgfpathlineto{\pgfqpoint{2.099476in}{1.634710in}}%
\pgfpathlineto{\pgfqpoint{2.153671in}{1.621797in}}%
\pgfpathlineto{\pgfqpoint{2.207867in}{1.607604in}}%
\pgfpathlineto{\pgfqpoint{2.262063in}{1.599908in}}%
\pgfpathlineto{\pgfqpoint{2.316259in}{1.601357in}}%
\pgfpathlineto{\pgfqpoint{2.370455in}{1.600560in}}%
\pgfpathlineto{\pgfqpoint{2.424650in}{1.595532in}}%
\pgfpathlineto{\pgfqpoint{2.478846in}{1.592283in}}%
\pgfpathlineto{\pgfqpoint{2.533042in}{1.591821in}}%
\pgfpathlineto{\pgfqpoint{2.587238in}{1.591691in}}%
\pgfpathlineto{\pgfqpoint{2.641434in}{1.589924in}}%
\pgfpathlineto{\pgfqpoint{2.695629in}{1.584266in}}%
\pgfpathlineto{\pgfqpoint{2.749825in}{1.574919in}}%
\pgfpathlineto{\pgfqpoint{2.804021in}{1.566358in}}%
\pgfpathlineto{\pgfqpoint{2.858217in}{1.558474in}}%
\pgfpathlineto{\pgfqpoint{2.912413in}{1.549516in}}%
\pgfpathlineto{\pgfqpoint{2.966608in}{1.547623in}}%
\pgfpathlineto{\pgfqpoint{3.020804in}{1.542399in}}%
\pgfpathlineto{\pgfqpoint{3.075000in}{1.537585in}}%
\pgfpathlineto{\pgfqpoint{3.129196in}{1.528720in}}%
\pgfpathlineto{\pgfqpoint{3.183392in}{1.517116in}}%
\pgfpathlineto{\pgfqpoint{3.237587in}{1.509478in}}%
\pgfpathlineto{\pgfqpoint{3.291783in}{1.501703in}}%
\pgfpathlineto{\pgfqpoint{3.345979in}{1.497852in}}%
\pgfpathlineto{\pgfqpoint{3.400175in}{1.486030in}}%
\pgfpathlineto{\pgfqpoint{3.454371in}{1.471993in}}%
\pgfpathlineto{\pgfqpoint{3.508566in}{1.461343in}}%
\pgfpathlineto{\pgfqpoint{3.562762in}{1.451054in}}%
\pgfpathlineto{\pgfqpoint{3.616958in}{1.444297in}}%
\pgfpathlineto{\pgfqpoint{3.671154in}{1.431711in}}%
\pgfpathlineto{\pgfqpoint{3.725350in}{1.416992in}}%
\pgfpathlineto{\pgfqpoint{3.779545in}{1.409253in}}%
\pgfpathlineto{\pgfqpoint{3.833741in}{1.396585in}}%
\pgfpathlineto{\pgfqpoint{3.887937in}{1.383123in}}%
\pgfpathlineto{\pgfqpoint{3.942133in}{1.371001in}}%
\pgfpathlineto{\pgfqpoint{3.996329in}{1.357629in}}%
\pgfpathlineto{\pgfqpoint{4.050524in}{1.345235in}}%
\pgfpathlineto{\pgfqpoint{4.104720in}{1.334203in}}%
\pgfpathlineto{\pgfqpoint{4.158916in}{1.322834in}}%
\pgfpathlineto{\pgfqpoint{4.213112in}{1.313945in}}%
\pgfpathlineto{\pgfqpoint{4.267308in}{1.305829in}}%
\pgfpathlineto{\pgfqpoint{4.321503in}{1.289735in}}%
\pgfpathlineto{\pgfqpoint{4.375699in}{1.281132in}}%
\pgfpathlineto{\pgfqpoint{4.429895in}{1.265575in}}%
\pgfpathlineto{\pgfqpoint{4.484091in}{1.250032in}}%
\pgfpathlineto{\pgfqpoint{4.538287in}{1.231640in}}%
\pgfpathlineto{\pgfqpoint{4.592483in}{1.212076in}}%
\pgfpathlineto{\pgfqpoint{4.646678in}{1.192454in}}%
\pgfpathlineto{\pgfqpoint{4.700874in}{1.167183in}}%
\pgfpathlineto{\pgfqpoint{4.755070in}{1.146291in}}%
\pgfpathlineto{\pgfqpoint{4.809266in}{1.125545in}}%
\pgfpathlineto{\pgfqpoint{4.863462in}{1.107095in}}%
\pgfpathlineto{\pgfqpoint{4.917657in}{1.086953in}}%
\pgfpathlineto{\pgfqpoint{4.971853in}{1.076092in}}%
\pgfpathlineto{\pgfqpoint{5.026049in}{1.055878in}}%
\pgfpathlineto{\pgfqpoint{5.080245in}{1.038119in}}%
\pgfpathlineto{\pgfqpoint{5.134441in}{1.025993in}}%
\pgfpathlineto{\pgfqpoint{5.188636in}{1.007774in}}%
\pgfusepath{stroke}%
\end{pgfscope}%
\begin{pgfscope}%
\pgfpathrectangle{\pgfqpoint{0.750000in}{0.500000in}}{\pgfqpoint{4.650000in}{3.020000in}}%
\pgfusepath{clip}%
\pgfsetbuttcap%
\pgfsetroundjoin%
\definecolor{currentfill}{rgb}{1.000000,0.498039,0.054902}%
\pgfsetfillcolor{currentfill}%
\pgfsetlinewidth{1.003750pt}%
\definecolor{currentstroke}{rgb}{1.000000,0.498039,0.054902}%
\pgfsetstrokecolor{currentstroke}%
\pgfsetdash{}{0pt}%
\pgfsys@defobject{currentmarker}{\pgfqpoint{-0.020833in}{-0.020833in}}{\pgfqpoint{0.020833in}{0.020833in}}{%
\pgfpathmoveto{\pgfqpoint{0.000000in}{-0.020833in}}%
\pgfpathcurveto{\pgfqpoint{0.005525in}{-0.020833in}}{\pgfqpoint{0.010825in}{-0.018638in}}{\pgfqpoint{0.014731in}{-0.014731in}}%
\pgfpathcurveto{\pgfqpoint{0.018638in}{-0.010825in}}{\pgfqpoint{0.020833in}{-0.005525in}}{\pgfqpoint{0.020833in}{0.000000in}}%
\pgfpathcurveto{\pgfqpoint{0.020833in}{0.005525in}}{\pgfqpoint{0.018638in}{0.010825in}}{\pgfqpoint{0.014731in}{0.014731in}}%
\pgfpathcurveto{\pgfqpoint{0.010825in}{0.018638in}}{\pgfqpoint{0.005525in}{0.020833in}}{\pgfqpoint{0.000000in}{0.020833in}}%
\pgfpathcurveto{\pgfqpoint{-0.005525in}{0.020833in}}{\pgfqpoint{-0.010825in}{0.018638in}}{\pgfqpoint{-0.014731in}{0.014731in}}%
\pgfpathcurveto{\pgfqpoint{-0.018638in}{0.010825in}}{\pgfqpoint{-0.020833in}{0.005525in}}{\pgfqpoint{-0.020833in}{0.000000in}}%
\pgfpathcurveto{\pgfqpoint{-0.020833in}{-0.005525in}}{\pgfqpoint{-0.018638in}{-0.010825in}}{\pgfqpoint{-0.014731in}{-0.014731in}}%
\pgfpathcurveto{\pgfqpoint{-0.010825in}{-0.018638in}}{\pgfqpoint{-0.005525in}{-0.020833in}}{\pgfqpoint{0.000000in}{-0.020833in}}%
\pgfpathlineto{\pgfqpoint{0.000000in}{-0.020833in}}%
\pgfpathclose%
\pgfusepath{stroke,fill}%
}%
\begin{pgfscope}%
\pgfsys@transformshift{0.961364in}{3.164949in}%
\pgfsys@useobject{currentmarker}{}%
\end{pgfscope}%
\begin{pgfscope}%
\pgfsys@transformshift{1.015559in}{2.969692in}%
\pgfsys@useobject{currentmarker}{}%
\end{pgfscope}%
\begin{pgfscope}%
\pgfsys@transformshift{1.069755in}{2.689040in}%
\pgfsys@useobject{currentmarker}{}%
\end{pgfscope}%
\begin{pgfscope}%
\pgfsys@transformshift{1.123951in}{2.390671in}%
\pgfsys@useobject{currentmarker}{}%
\end{pgfscope}%
\begin{pgfscope}%
\pgfsys@transformshift{1.178147in}{2.213505in}%
\pgfsys@useobject{currentmarker}{}%
\end{pgfscope}%
\begin{pgfscope}%
\pgfsys@transformshift{1.232343in}{2.108799in}%
\pgfsys@useobject{currentmarker}{}%
\end{pgfscope}%
\begin{pgfscope}%
\pgfsys@transformshift{1.286538in}{2.377356in}%
\pgfsys@useobject{currentmarker}{}%
\end{pgfscope}%
\begin{pgfscope}%
\pgfsys@transformshift{1.340734in}{2.139988in}%
\pgfsys@useobject{currentmarker}{}%
\end{pgfscope}%
\begin{pgfscope}%
\pgfsys@transformshift{1.394930in}{1.963422in}%
\pgfsys@useobject{currentmarker}{}%
\end{pgfscope}%
\begin{pgfscope}%
\pgfsys@transformshift{1.449126in}{1.821533in}%
\pgfsys@useobject{currentmarker}{}%
\end{pgfscope}%
\begin{pgfscope}%
\pgfsys@transformshift{1.503322in}{1.762046in}%
\pgfsys@useobject{currentmarker}{}%
\end{pgfscope}%
\begin{pgfscope}%
\pgfsys@transformshift{1.557517in}{1.719007in}%
\pgfsys@useobject{currentmarker}{}%
\end{pgfscope}%
\begin{pgfscope}%
\pgfsys@transformshift{1.611713in}{1.676891in}%
\pgfsys@useobject{currentmarker}{}%
\end{pgfscope}%
\begin{pgfscope}%
\pgfsys@transformshift{1.665909in}{1.646645in}%
\pgfsys@useobject{currentmarker}{}%
\end{pgfscope}%
\begin{pgfscope}%
\pgfsys@transformshift{1.720105in}{1.633854in}%
\pgfsys@useobject{currentmarker}{}%
\end{pgfscope}%
\begin{pgfscope}%
\pgfsys@transformshift{1.774301in}{1.627147in}%
\pgfsys@useobject{currentmarker}{}%
\end{pgfscope}%
\begin{pgfscope}%
\pgfsys@transformshift{1.828497in}{1.626923in}%
\pgfsys@useobject{currentmarker}{}%
\end{pgfscope}%
\begin{pgfscope}%
\pgfsys@transformshift{1.882692in}{1.631909in}%
\pgfsys@useobject{currentmarker}{}%
\end{pgfscope}%
\begin{pgfscope}%
\pgfsys@transformshift{1.936888in}{1.634876in}%
\pgfsys@useobject{currentmarker}{}%
\end{pgfscope}%
\begin{pgfscope}%
\pgfsys@transformshift{1.991084in}{1.635008in}%
\pgfsys@useobject{currentmarker}{}%
\end{pgfscope}%
\begin{pgfscope}%
\pgfsys@transformshift{2.045280in}{1.635919in}%
\pgfsys@useobject{currentmarker}{}%
\end{pgfscope}%
\begin{pgfscope}%
\pgfsys@transformshift{2.099476in}{1.634710in}%
\pgfsys@useobject{currentmarker}{}%
\end{pgfscope}%
\begin{pgfscope}%
\pgfsys@transformshift{2.153671in}{1.621797in}%
\pgfsys@useobject{currentmarker}{}%
\end{pgfscope}%
\begin{pgfscope}%
\pgfsys@transformshift{2.207867in}{1.607604in}%
\pgfsys@useobject{currentmarker}{}%
\end{pgfscope}%
\begin{pgfscope}%
\pgfsys@transformshift{2.262063in}{1.599908in}%
\pgfsys@useobject{currentmarker}{}%
\end{pgfscope}%
\begin{pgfscope}%
\pgfsys@transformshift{2.316259in}{1.601357in}%
\pgfsys@useobject{currentmarker}{}%
\end{pgfscope}%
\begin{pgfscope}%
\pgfsys@transformshift{2.370455in}{1.600560in}%
\pgfsys@useobject{currentmarker}{}%
\end{pgfscope}%
\begin{pgfscope}%
\pgfsys@transformshift{2.424650in}{1.595532in}%
\pgfsys@useobject{currentmarker}{}%
\end{pgfscope}%
\begin{pgfscope}%
\pgfsys@transformshift{2.478846in}{1.592283in}%
\pgfsys@useobject{currentmarker}{}%
\end{pgfscope}%
\begin{pgfscope}%
\pgfsys@transformshift{2.533042in}{1.591821in}%
\pgfsys@useobject{currentmarker}{}%
\end{pgfscope}%
\begin{pgfscope}%
\pgfsys@transformshift{2.587238in}{1.591691in}%
\pgfsys@useobject{currentmarker}{}%
\end{pgfscope}%
\begin{pgfscope}%
\pgfsys@transformshift{2.641434in}{1.589924in}%
\pgfsys@useobject{currentmarker}{}%
\end{pgfscope}%
\begin{pgfscope}%
\pgfsys@transformshift{2.695629in}{1.584266in}%
\pgfsys@useobject{currentmarker}{}%
\end{pgfscope}%
\begin{pgfscope}%
\pgfsys@transformshift{2.749825in}{1.574919in}%
\pgfsys@useobject{currentmarker}{}%
\end{pgfscope}%
\begin{pgfscope}%
\pgfsys@transformshift{2.804021in}{1.566358in}%
\pgfsys@useobject{currentmarker}{}%
\end{pgfscope}%
\begin{pgfscope}%
\pgfsys@transformshift{2.858217in}{1.558474in}%
\pgfsys@useobject{currentmarker}{}%
\end{pgfscope}%
\begin{pgfscope}%
\pgfsys@transformshift{2.912413in}{1.549516in}%
\pgfsys@useobject{currentmarker}{}%
\end{pgfscope}%
\begin{pgfscope}%
\pgfsys@transformshift{2.966608in}{1.547623in}%
\pgfsys@useobject{currentmarker}{}%
\end{pgfscope}%
\begin{pgfscope}%
\pgfsys@transformshift{3.020804in}{1.542399in}%
\pgfsys@useobject{currentmarker}{}%
\end{pgfscope}%
\begin{pgfscope}%
\pgfsys@transformshift{3.075000in}{1.537585in}%
\pgfsys@useobject{currentmarker}{}%
\end{pgfscope}%
\begin{pgfscope}%
\pgfsys@transformshift{3.129196in}{1.528720in}%
\pgfsys@useobject{currentmarker}{}%
\end{pgfscope}%
\begin{pgfscope}%
\pgfsys@transformshift{3.183392in}{1.517116in}%
\pgfsys@useobject{currentmarker}{}%
\end{pgfscope}%
\begin{pgfscope}%
\pgfsys@transformshift{3.237587in}{1.509478in}%
\pgfsys@useobject{currentmarker}{}%
\end{pgfscope}%
\begin{pgfscope}%
\pgfsys@transformshift{3.291783in}{1.501703in}%
\pgfsys@useobject{currentmarker}{}%
\end{pgfscope}%
\begin{pgfscope}%
\pgfsys@transformshift{3.345979in}{1.497852in}%
\pgfsys@useobject{currentmarker}{}%
\end{pgfscope}%
\begin{pgfscope}%
\pgfsys@transformshift{3.400175in}{1.486030in}%
\pgfsys@useobject{currentmarker}{}%
\end{pgfscope}%
\begin{pgfscope}%
\pgfsys@transformshift{3.454371in}{1.471993in}%
\pgfsys@useobject{currentmarker}{}%
\end{pgfscope}%
\begin{pgfscope}%
\pgfsys@transformshift{3.508566in}{1.461343in}%
\pgfsys@useobject{currentmarker}{}%
\end{pgfscope}%
\begin{pgfscope}%
\pgfsys@transformshift{3.562762in}{1.451054in}%
\pgfsys@useobject{currentmarker}{}%
\end{pgfscope}%
\begin{pgfscope}%
\pgfsys@transformshift{3.616958in}{1.444297in}%
\pgfsys@useobject{currentmarker}{}%
\end{pgfscope}%
\begin{pgfscope}%
\pgfsys@transformshift{3.671154in}{1.431711in}%
\pgfsys@useobject{currentmarker}{}%
\end{pgfscope}%
\begin{pgfscope}%
\pgfsys@transformshift{3.725350in}{1.416992in}%
\pgfsys@useobject{currentmarker}{}%
\end{pgfscope}%
\begin{pgfscope}%
\pgfsys@transformshift{3.779545in}{1.409253in}%
\pgfsys@useobject{currentmarker}{}%
\end{pgfscope}%
\begin{pgfscope}%
\pgfsys@transformshift{3.833741in}{1.396585in}%
\pgfsys@useobject{currentmarker}{}%
\end{pgfscope}%
\begin{pgfscope}%
\pgfsys@transformshift{3.887937in}{1.383123in}%
\pgfsys@useobject{currentmarker}{}%
\end{pgfscope}%
\begin{pgfscope}%
\pgfsys@transformshift{3.942133in}{1.371001in}%
\pgfsys@useobject{currentmarker}{}%
\end{pgfscope}%
\begin{pgfscope}%
\pgfsys@transformshift{3.996329in}{1.357629in}%
\pgfsys@useobject{currentmarker}{}%
\end{pgfscope}%
\begin{pgfscope}%
\pgfsys@transformshift{4.050524in}{1.345235in}%
\pgfsys@useobject{currentmarker}{}%
\end{pgfscope}%
\begin{pgfscope}%
\pgfsys@transformshift{4.104720in}{1.334203in}%
\pgfsys@useobject{currentmarker}{}%
\end{pgfscope}%
\begin{pgfscope}%
\pgfsys@transformshift{4.158916in}{1.322834in}%
\pgfsys@useobject{currentmarker}{}%
\end{pgfscope}%
\begin{pgfscope}%
\pgfsys@transformshift{4.213112in}{1.313945in}%
\pgfsys@useobject{currentmarker}{}%
\end{pgfscope}%
\begin{pgfscope}%
\pgfsys@transformshift{4.267308in}{1.305829in}%
\pgfsys@useobject{currentmarker}{}%
\end{pgfscope}%
\begin{pgfscope}%
\pgfsys@transformshift{4.321503in}{1.289735in}%
\pgfsys@useobject{currentmarker}{}%
\end{pgfscope}%
\begin{pgfscope}%
\pgfsys@transformshift{4.375699in}{1.281132in}%
\pgfsys@useobject{currentmarker}{}%
\end{pgfscope}%
\begin{pgfscope}%
\pgfsys@transformshift{4.429895in}{1.265575in}%
\pgfsys@useobject{currentmarker}{}%
\end{pgfscope}%
\begin{pgfscope}%
\pgfsys@transformshift{4.484091in}{1.250032in}%
\pgfsys@useobject{currentmarker}{}%
\end{pgfscope}%
\begin{pgfscope}%
\pgfsys@transformshift{4.538287in}{1.231640in}%
\pgfsys@useobject{currentmarker}{}%
\end{pgfscope}%
\begin{pgfscope}%
\pgfsys@transformshift{4.592483in}{1.212076in}%
\pgfsys@useobject{currentmarker}{}%
\end{pgfscope}%
\begin{pgfscope}%
\pgfsys@transformshift{4.646678in}{1.192454in}%
\pgfsys@useobject{currentmarker}{}%
\end{pgfscope}%
\begin{pgfscope}%
\pgfsys@transformshift{4.700874in}{1.167183in}%
\pgfsys@useobject{currentmarker}{}%
\end{pgfscope}%
\begin{pgfscope}%
\pgfsys@transformshift{4.755070in}{1.146291in}%
\pgfsys@useobject{currentmarker}{}%
\end{pgfscope}%
\begin{pgfscope}%
\pgfsys@transformshift{4.809266in}{1.125545in}%
\pgfsys@useobject{currentmarker}{}%
\end{pgfscope}%
\begin{pgfscope}%
\pgfsys@transformshift{4.863462in}{1.107095in}%
\pgfsys@useobject{currentmarker}{}%
\end{pgfscope}%
\begin{pgfscope}%
\pgfsys@transformshift{4.917657in}{1.086953in}%
\pgfsys@useobject{currentmarker}{}%
\end{pgfscope}%
\begin{pgfscope}%
\pgfsys@transformshift{4.971853in}{1.076092in}%
\pgfsys@useobject{currentmarker}{}%
\end{pgfscope}%
\begin{pgfscope}%
\pgfsys@transformshift{5.026049in}{1.055878in}%
\pgfsys@useobject{currentmarker}{}%
\end{pgfscope}%
\begin{pgfscope}%
\pgfsys@transformshift{5.080245in}{1.038119in}%
\pgfsys@useobject{currentmarker}{}%
\end{pgfscope}%
\begin{pgfscope}%
\pgfsys@transformshift{5.134441in}{1.025993in}%
\pgfsys@useobject{currentmarker}{}%
\end{pgfscope}%
\begin{pgfscope}%
\pgfsys@transformshift{5.188636in}{1.007774in}%
\pgfsys@useobject{currentmarker}{}%
\end{pgfscope}%
\end{pgfscope}%
\begin{pgfscope}%
\pgfpathrectangle{\pgfqpoint{0.750000in}{0.500000in}}{\pgfqpoint{4.650000in}{3.020000in}}%
\pgfusepath{clip}%
\pgfsetrectcap%
\pgfsetroundjoin%
\pgfsetlinewidth{1.505625pt}%
\definecolor{currentstroke}{rgb}{0.172549,0.627451,0.172549}%
\pgfsetstrokecolor{currentstroke}%
\pgfsetdash{}{0pt}%
\pgfpathmoveto{\pgfqpoint{0.961364in}{3.167009in}}%
\pgfpathlineto{\pgfqpoint{1.015559in}{2.967327in}}%
\pgfpathlineto{\pgfqpoint{1.069755in}{2.658463in}}%
\pgfpathlineto{\pgfqpoint{1.123951in}{2.780151in}}%
\pgfpathlineto{\pgfqpoint{1.178147in}{2.946132in}}%
\pgfpathlineto{\pgfqpoint{1.232343in}{2.539521in}}%
\pgfpathlineto{\pgfqpoint{1.286538in}{2.331770in}}%
\pgfpathlineto{\pgfqpoint{1.340734in}{2.176189in}}%
\pgfpathlineto{\pgfqpoint{1.394930in}{2.049155in}}%
\pgfpathlineto{\pgfqpoint{1.449126in}{1.892591in}}%
\pgfpathlineto{\pgfqpoint{1.503322in}{1.763141in}}%
\pgfpathlineto{\pgfqpoint{1.557517in}{1.683167in}}%
\pgfpathlineto{\pgfqpoint{1.611713in}{1.634240in}}%
\pgfpathlineto{\pgfqpoint{1.665909in}{1.606249in}}%
\pgfpathlineto{\pgfqpoint{1.720105in}{1.585928in}}%
\pgfpathlineto{\pgfqpoint{1.774301in}{1.575560in}}%
\pgfpathlineto{\pgfqpoint{1.828497in}{1.567672in}}%
\pgfpathlineto{\pgfqpoint{1.882692in}{1.558707in}}%
\pgfpathlineto{\pgfqpoint{1.936888in}{1.552497in}}%
\pgfpathlineto{\pgfqpoint{1.991084in}{1.549161in}}%
\pgfpathlineto{\pgfqpoint{2.045280in}{1.544552in}}%
\pgfpathlineto{\pgfqpoint{2.099476in}{1.537080in}}%
\pgfpathlineto{\pgfqpoint{2.153671in}{1.527298in}}%
\pgfpathlineto{\pgfqpoint{2.207867in}{1.526079in}}%
\pgfpathlineto{\pgfqpoint{2.262063in}{1.521879in}}%
\pgfpathlineto{\pgfqpoint{2.316259in}{1.520696in}}%
\pgfpathlineto{\pgfqpoint{2.370455in}{1.518743in}}%
\pgfpathlineto{\pgfqpoint{2.424650in}{1.511522in}}%
\pgfpathlineto{\pgfqpoint{2.478846in}{1.509282in}}%
\pgfpathlineto{\pgfqpoint{2.533042in}{1.503835in}}%
\pgfpathlineto{\pgfqpoint{2.587238in}{1.496385in}}%
\pgfpathlineto{\pgfqpoint{2.641434in}{1.493553in}}%
\pgfpathlineto{\pgfqpoint{2.695629in}{1.489184in}}%
\pgfpathlineto{\pgfqpoint{2.749825in}{1.480367in}}%
\pgfpathlineto{\pgfqpoint{2.804021in}{1.470175in}}%
\pgfpathlineto{\pgfqpoint{2.858217in}{1.459018in}}%
\pgfpathlineto{\pgfqpoint{2.912413in}{1.447390in}}%
\pgfpathlineto{\pgfqpoint{2.966608in}{1.435291in}}%
\pgfpathlineto{\pgfqpoint{3.020804in}{1.422920in}}%
\pgfpathlineto{\pgfqpoint{3.075000in}{1.407846in}}%
\pgfpathlineto{\pgfqpoint{3.129196in}{1.392680in}}%
\pgfpathlineto{\pgfqpoint{3.183392in}{1.376852in}}%
\pgfpathlineto{\pgfqpoint{3.237587in}{1.359990in}}%
\pgfpathlineto{\pgfqpoint{3.291783in}{1.345527in}}%
\pgfpathlineto{\pgfqpoint{3.345979in}{1.326660in}}%
\pgfpathlineto{\pgfqpoint{3.400175in}{1.307720in}}%
\pgfpathlineto{\pgfqpoint{3.454371in}{1.287623in}}%
\pgfpathlineto{\pgfqpoint{3.508566in}{1.265952in}}%
\pgfpathlineto{\pgfqpoint{3.562762in}{1.249640in}}%
\pgfpathlineto{\pgfqpoint{3.616958in}{1.234666in}}%
\pgfpathlineto{\pgfqpoint{3.671154in}{1.218674in}}%
\pgfpathlineto{\pgfqpoint{3.725350in}{1.200684in}}%
\pgfpathlineto{\pgfqpoint{3.779545in}{1.179767in}}%
\pgfpathlineto{\pgfqpoint{3.833741in}{1.161920in}}%
\pgfpathlineto{\pgfqpoint{3.887937in}{1.146879in}}%
\pgfpathlineto{\pgfqpoint{3.942133in}{1.126223in}}%
\pgfpathlineto{\pgfqpoint{3.996329in}{1.111458in}}%
\pgfpathlineto{\pgfqpoint{4.050524in}{1.094496in}}%
\pgfpathlineto{\pgfqpoint{4.104720in}{1.075667in}}%
\pgfpathlineto{\pgfqpoint{4.158916in}{1.052604in}}%
\pgfpathlineto{\pgfqpoint{4.213112in}{1.033414in}}%
\pgfpathlineto{\pgfqpoint{4.267308in}{1.009986in}}%
\pgfpathlineto{\pgfqpoint{4.321503in}{0.985772in}}%
\pgfpathlineto{\pgfqpoint{4.375699in}{0.967522in}}%
\pgfpathlineto{\pgfqpoint{4.429895in}{0.944368in}}%
\pgfpathlineto{\pgfqpoint{4.484091in}{0.926443in}}%
\pgfpathlineto{\pgfqpoint{4.538287in}{0.906982in}}%
\pgfpathlineto{\pgfqpoint{4.592483in}{0.887174in}}%
\pgfpathlineto{\pgfqpoint{4.646678in}{0.867785in}}%
\pgfpathlineto{\pgfqpoint{4.700874in}{0.847555in}}%
\pgfpathlineto{\pgfqpoint{4.755070in}{0.819277in}}%
\pgfpathlineto{\pgfqpoint{4.809266in}{0.795770in}}%
\pgfpathlineto{\pgfqpoint{4.863462in}{0.780038in}}%
\pgfpathlineto{\pgfqpoint{4.917657in}{0.745409in}}%
\pgfpathlineto{\pgfqpoint{4.971853in}{0.722900in}}%
\pgfpathlineto{\pgfqpoint{5.026049in}{0.702620in}}%
\pgfpathlineto{\pgfqpoint{5.080245in}{0.683925in}}%
\pgfpathlineto{\pgfqpoint{5.134441in}{0.669532in}}%
\pgfpathlineto{\pgfqpoint{5.188636in}{0.637273in}}%
\pgfusepath{stroke}%
\end{pgfscope}%
\begin{pgfscope}%
\pgfpathrectangle{\pgfqpoint{0.750000in}{0.500000in}}{\pgfqpoint{4.650000in}{3.020000in}}%
\pgfusepath{clip}%
\pgfsetbuttcap%
\pgfsetroundjoin%
\definecolor{currentfill}{rgb}{0.172549,0.627451,0.172549}%
\pgfsetfillcolor{currentfill}%
\pgfsetlinewidth{1.003750pt}%
\definecolor{currentstroke}{rgb}{0.172549,0.627451,0.172549}%
\pgfsetstrokecolor{currentstroke}%
\pgfsetdash{}{0pt}%
\pgfsys@defobject{currentmarker}{\pgfqpoint{-0.020833in}{-0.020833in}}{\pgfqpoint{0.020833in}{0.020833in}}{%
\pgfpathmoveto{\pgfqpoint{0.000000in}{-0.020833in}}%
\pgfpathcurveto{\pgfqpoint{0.005525in}{-0.020833in}}{\pgfqpoint{0.010825in}{-0.018638in}}{\pgfqpoint{0.014731in}{-0.014731in}}%
\pgfpathcurveto{\pgfqpoint{0.018638in}{-0.010825in}}{\pgfqpoint{0.020833in}{-0.005525in}}{\pgfqpoint{0.020833in}{0.000000in}}%
\pgfpathcurveto{\pgfqpoint{0.020833in}{0.005525in}}{\pgfqpoint{0.018638in}{0.010825in}}{\pgfqpoint{0.014731in}{0.014731in}}%
\pgfpathcurveto{\pgfqpoint{0.010825in}{0.018638in}}{\pgfqpoint{0.005525in}{0.020833in}}{\pgfqpoint{0.000000in}{0.020833in}}%
\pgfpathcurveto{\pgfqpoint{-0.005525in}{0.020833in}}{\pgfqpoint{-0.010825in}{0.018638in}}{\pgfqpoint{-0.014731in}{0.014731in}}%
\pgfpathcurveto{\pgfqpoint{-0.018638in}{0.010825in}}{\pgfqpoint{-0.020833in}{0.005525in}}{\pgfqpoint{-0.020833in}{0.000000in}}%
\pgfpathcurveto{\pgfqpoint{-0.020833in}{-0.005525in}}{\pgfqpoint{-0.018638in}{-0.010825in}}{\pgfqpoint{-0.014731in}{-0.014731in}}%
\pgfpathcurveto{\pgfqpoint{-0.010825in}{-0.018638in}}{\pgfqpoint{-0.005525in}{-0.020833in}}{\pgfqpoint{0.000000in}{-0.020833in}}%
\pgfpathlineto{\pgfqpoint{0.000000in}{-0.020833in}}%
\pgfpathclose%
\pgfusepath{stroke,fill}%
}%
\begin{pgfscope}%
\pgfsys@transformshift{0.961364in}{3.167009in}%
\pgfsys@useobject{currentmarker}{}%
\end{pgfscope}%
\begin{pgfscope}%
\pgfsys@transformshift{1.015559in}{2.967327in}%
\pgfsys@useobject{currentmarker}{}%
\end{pgfscope}%
\begin{pgfscope}%
\pgfsys@transformshift{1.069755in}{2.658463in}%
\pgfsys@useobject{currentmarker}{}%
\end{pgfscope}%
\begin{pgfscope}%
\pgfsys@transformshift{1.123951in}{2.780151in}%
\pgfsys@useobject{currentmarker}{}%
\end{pgfscope}%
\begin{pgfscope}%
\pgfsys@transformshift{1.178147in}{2.946132in}%
\pgfsys@useobject{currentmarker}{}%
\end{pgfscope}%
\begin{pgfscope}%
\pgfsys@transformshift{1.232343in}{2.539521in}%
\pgfsys@useobject{currentmarker}{}%
\end{pgfscope}%
\begin{pgfscope}%
\pgfsys@transformshift{1.286538in}{2.331770in}%
\pgfsys@useobject{currentmarker}{}%
\end{pgfscope}%
\begin{pgfscope}%
\pgfsys@transformshift{1.340734in}{2.176189in}%
\pgfsys@useobject{currentmarker}{}%
\end{pgfscope}%
\begin{pgfscope}%
\pgfsys@transformshift{1.394930in}{2.049155in}%
\pgfsys@useobject{currentmarker}{}%
\end{pgfscope}%
\begin{pgfscope}%
\pgfsys@transformshift{1.449126in}{1.892591in}%
\pgfsys@useobject{currentmarker}{}%
\end{pgfscope}%
\begin{pgfscope}%
\pgfsys@transformshift{1.503322in}{1.763141in}%
\pgfsys@useobject{currentmarker}{}%
\end{pgfscope}%
\begin{pgfscope}%
\pgfsys@transformshift{1.557517in}{1.683167in}%
\pgfsys@useobject{currentmarker}{}%
\end{pgfscope}%
\begin{pgfscope}%
\pgfsys@transformshift{1.611713in}{1.634240in}%
\pgfsys@useobject{currentmarker}{}%
\end{pgfscope}%
\begin{pgfscope}%
\pgfsys@transformshift{1.665909in}{1.606249in}%
\pgfsys@useobject{currentmarker}{}%
\end{pgfscope}%
\begin{pgfscope}%
\pgfsys@transformshift{1.720105in}{1.585928in}%
\pgfsys@useobject{currentmarker}{}%
\end{pgfscope}%
\begin{pgfscope}%
\pgfsys@transformshift{1.774301in}{1.575560in}%
\pgfsys@useobject{currentmarker}{}%
\end{pgfscope}%
\begin{pgfscope}%
\pgfsys@transformshift{1.828497in}{1.567672in}%
\pgfsys@useobject{currentmarker}{}%
\end{pgfscope}%
\begin{pgfscope}%
\pgfsys@transformshift{1.882692in}{1.558707in}%
\pgfsys@useobject{currentmarker}{}%
\end{pgfscope}%
\begin{pgfscope}%
\pgfsys@transformshift{1.936888in}{1.552497in}%
\pgfsys@useobject{currentmarker}{}%
\end{pgfscope}%
\begin{pgfscope}%
\pgfsys@transformshift{1.991084in}{1.549161in}%
\pgfsys@useobject{currentmarker}{}%
\end{pgfscope}%
\begin{pgfscope}%
\pgfsys@transformshift{2.045280in}{1.544552in}%
\pgfsys@useobject{currentmarker}{}%
\end{pgfscope}%
\begin{pgfscope}%
\pgfsys@transformshift{2.099476in}{1.537080in}%
\pgfsys@useobject{currentmarker}{}%
\end{pgfscope}%
\begin{pgfscope}%
\pgfsys@transformshift{2.153671in}{1.527298in}%
\pgfsys@useobject{currentmarker}{}%
\end{pgfscope}%
\begin{pgfscope}%
\pgfsys@transformshift{2.207867in}{1.526079in}%
\pgfsys@useobject{currentmarker}{}%
\end{pgfscope}%
\begin{pgfscope}%
\pgfsys@transformshift{2.262063in}{1.521879in}%
\pgfsys@useobject{currentmarker}{}%
\end{pgfscope}%
\begin{pgfscope}%
\pgfsys@transformshift{2.316259in}{1.520696in}%
\pgfsys@useobject{currentmarker}{}%
\end{pgfscope}%
\begin{pgfscope}%
\pgfsys@transformshift{2.370455in}{1.518743in}%
\pgfsys@useobject{currentmarker}{}%
\end{pgfscope}%
\begin{pgfscope}%
\pgfsys@transformshift{2.424650in}{1.511522in}%
\pgfsys@useobject{currentmarker}{}%
\end{pgfscope}%
\begin{pgfscope}%
\pgfsys@transformshift{2.478846in}{1.509282in}%
\pgfsys@useobject{currentmarker}{}%
\end{pgfscope}%
\begin{pgfscope}%
\pgfsys@transformshift{2.533042in}{1.503835in}%
\pgfsys@useobject{currentmarker}{}%
\end{pgfscope}%
\begin{pgfscope}%
\pgfsys@transformshift{2.587238in}{1.496385in}%
\pgfsys@useobject{currentmarker}{}%
\end{pgfscope}%
\begin{pgfscope}%
\pgfsys@transformshift{2.641434in}{1.493553in}%
\pgfsys@useobject{currentmarker}{}%
\end{pgfscope}%
\begin{pgfscope}%
\pgfsys@transformshift{2.695629in}{1.489184in}%
\pgfsys@useobject{currentmarker}{}%
\end{pgfscope}%
\begin{pgfscope}%
\pgfsys@transformshift{2.749825in}{1.480367in}%
\pgfsys@useobject{currentmarker}{}%
\end{pgfscope}%
\begin{pgfscope}%
\pgfsys@transformshift{2.804021in}{1.470175in}%
\pgfsys@useobject{currentmarker}{}%
\end{pgfscope}%
\begin{pgfscope}%
\pgfsys@transformshift{2.858217in}{1.459018in}%
\pgfsys@useobject{currentmarker}{}%
\end{pgfscope}%
\begin{pgfscope}%
\pgfsys@transformshift{2.912413in}{1.447390in}%
\pgfsys@useobject{currentmarker}{}%
\end{pgfscope}%
\begin{pgfscope}%
\pgfsys@transformshift{2.966608in}{1.435291in}%
\pgfsys@useobject{currentmarker}{}%
\end{pgfscope}%
\begin{pgfscope}%
\pgfsys@transformshift{3.020804in}{1.422920in}%
\pgfsys@useobject{currentmarker}{}%
\end{pgfscope}%
\begin{pgfscope}%
\pgfsys@transformshift{3.075000in}{1.407846in}%
\pgfsys@useobject{currentmarker}{}%
\end{pgfscope}%
\begin{pgfscope}%
\pgfsys@transformshift{3.129196in}{1.392680in}%
\pgfsys@useobject{currentmarker}{}%
\end{pgfscope}%
\begin{pgfscope}%
\pgfsys@transformshift{3.183392in}{1.376852in}%
\pgfsys@useobject{currentmarker}{}%
\end{pgfscope}%
\begin{pgfscope}%
\pgfsys@transformshift{3.237587in}{1.359990in}%
\pgfsys@useobject{currentmarker}{}%
\end{pgfscope}%
\begin{pgfscope}%
\pgfsys@transformshift{3.291783in}{1.345527in}%
\pgfsys@useobject{currentmarker}{}%
\end{pgfscope}%
\begin{pgfscope}%
\pgfsys@transformshift{3.345979in}{1.326660in}%
\pgfsys@useobject{currentmarker}{}%
\end{pgfscope}%
\begin{pgfscope}%
\pgfsys@transformshift{3.400175in}{1.307720in}%
\pgfsys@useobject{currentmarker}{}%
\end{pgfscope}%
\begin{pgfscope}%
\pgfsys@transformshift{3.454371in}{1.287623in}%
\pgfsys@useobject{currentmarker}{}%
\end{pgfscope}%
\begin{pgfscope}%
\pgfsys@transformshift{3.508566in}{1.265952in}%
\pgfsys@useobject{currentmarker}{}%
\end{pgfscope}%
\begin{pgfscope}%
\pgfsys@transformshift{3.562762in}{1.249640in}%
\pgfsys@useobject{currentmarker}{}%
\end{pgfscope}%
\begin{pgfscope}%
\pgfsys@transformshift{3.616958in}{1.234666in}%
\pgfsys@useobject{currentmarker}{}%
\end{pgfscope}%
\begin{pgfscope}%
\pgfsys@transformshift{3.671154in}{1.218674in}%
\pgfsys@useobject{currentmarker}{}%
\end{pgfscope}%
\begin{pgfscope}%
\pgfsys@transformshift{3.725350in}{1.200684in}%
\pgfsys@useobject{currentmarker}{}%
\end{pgfscope}%
\begin{pgfscope}%
\pgfsys@transformshift{3.779545in}{1.179767in}%
\pgfsys@useobject{currentmarker}{}%
\end{pgfscope}%
\begin{pgfscope}%
\pgfsys@transformshift{3.833741in}{1.161920in}%
\pgfsys@useobject{currentmarker}{}%
\end{pgfscope}%
\begin{pgfscope}%
\pgfsys@transformshift{3.887937in}{1.146879in}%
\pgfsys@useobject{currentmarker}{}%
\end{pgfscope}%
\begin{pgfscope}%
\pgfsys@transformshift{3.942133in}{1.126223in}%
\pgfsys@useobject{currentmarker}{}%
\end{pgfscope}%
\begin{pgfscope}%
\pgfsys@transformshift{3.996329in}{1.111458in}%
\pgfsys@useobject{currentmarker}{}%
\end{pgfscope}%
\begin{pgfscope}%
\pgfsys@transformshift{4.050524in}{1.094496in}%
\pgfsys@useobject{currentmarker}{}%
\end{pgfscope}%
\begin{pgfscope}%
\pgfsys@transformshift{4.104720in}{1.075667in}%
\pgfsys@useobject{currentmarker}{}%
\end{pgfscope}%
\begin{pgfscope}%
\pgfsys@transformshift{4.158916in}{1.052604in}%
\pgfsys@useobject{currentmarker}{}%
\end{pgfscope}%
\begin{pgfscope}%
\pgfsys@transformshift{4.213112in}{1.033414in}%
\pgfsys@useobject{currentmarker}{}%
\end{pgfscope}%
\begin{pgfscope}%
\pgfsys@transformshift{4.267308in}{1.009986in}%
\pgfsys@useobject{currentmarker}{}%
\end{pgfscope}%
\begin{pgfscope}%
\pgfsys@transformshift{4.321503in}{0.985772in}%
\pgfsys@useobject{currentmarker}{}%
\end{pgfscope}%
\begin{pgfscope}%
\pgfsys@transformshift{4.375699in}{0.967522in}%
\pgfsys@useobject{currentmarker}{}%
\end{pgfscope}%
\begin{pgfscope}%
\pgfsys@transformshift{4.429895in}{0.944368in}%
\pgfsys@useobject{currentmarker}{}%
\end{pgfscope}%
\begin{pgfscope}%
\pgfsys@transformshift{4.484091in}{0.926443in}%
\pgfsys@useobject{currentmarker}{}%
\end{pgfscope}%
\begin{pgfscope}%
\pgfsys@transformshift{4.538287in}{0.906982in}%
\pgfsys@useobject{currentmarker}{}%
\end{pgfscope}%
\begin{pgfscope}%
\pgfsys@transformshift{4.592483in}{0.887174in}%
\pgfsys@useobject{currentmarker}{}%
\end{pgfscope}%
\begin{pgfscope}%
\pgfsys@transformshift{4.646678in}{0.867785in}%
\pgfsys@useobject{currentmarker}{}%
\end{pgfscope}%
\begin{pgfscope}%
\pgfsys@transformshift{4.700874in}{0.847555in}%
\pgfsys@useobject{currentmarker}{}%
\end{pgfscope}%
\begin{pgfscope}%
\pgfsys@transformshift{4.755070in}{0.819277in}%
\pgfsys@useobject{currentmarker}{}%
\end{pgfscope}%
\begin{pgfscope}%
\pgfsys@transformshift{4.809266in}{0.795770in}%
\pgfsys@useobject{currentmarker}{}%
\end{pgfscope}%
\begin{pgfscope}%
\pgfsys@transformshift{4.863462in}{0.780038in}%
\pgfsys@useobject{currentmarker}{}%
\end{pgfscope}%
\begin{pgfscope}%
\pgfsys@transformshift{4.917657in}{0.745409in}%
\pgfsys@useobject{currentmarker}{}%
\end{pgfscope}%
\begin{pgfscope}%
\pgfsys@transformshift{4.971853in}{0.722900in}%
\pgfsys@useobject{currentmarker}{}%
\end{pgfscope}%
\begin{pgfscope}%
\pgfsys@transformshift{5.026049in}{0.702620in}%
\pgfsys@useobject{currentmarker}{}%
\end{pgfscope}%
\begin{pgfscope}%
\pgfsys@transformshift{5.080245in}{0.683925in}%
\pgfsys@useobject{currentmarker}{}%
\end{pgfscope}%
\begin{pgfscope}%
\pgfsys@transformshift{5.134441in}{0.669532in}%
\pgfsys@useobject{currentmarker}{}%
\end{pgfscope}%
\begin{pgfscope}%
\pgfsys@transformshift{5.188636in}{0.637273in}%
\pgfsys@useobject{currentmarker}{}%
\end{pgfscope}%
\end{pgfscope}%
\begin{pgfscope}%
\pgfpathrectangle{\pgfqpoint{0.750000in}{0.500000in}}{\pgfqpoint{4.650000in}{3.020000in}}%
\pgfusepath{clip}%
\pgfsetbuttcap%
\pgfsetroundjoin%
\pgfsetlinewidth{1.505625pt}%
\definecolor{currentstroke}{rgb}{0.000000,0.000000,0.000000}%
\pgfsetstrokecolor{currentstroke}%
\pgfsetdash{}{0pt}%
\pgfpathmoveto{\pgfqpoint{1.668860in}{1.326135in}}%
\pgfpathlineto{\pgfqpoint{1.668860in}{2.093431in}}%
\pgfusepath{stroke}%
\end{pgfscope}%
\begin{pgfscope}%
\pgfpathrectangle{\pgfqpoint{0.750000in}{0.500000in}}{\pgfqpoint{4.650000in}{3.020000in}}%
\pgfusepath{clip}%
\pgfsetbuttcap%
\pgfsetroundjoin%
\pgfsetlinewidth{1.505625pt}%
\definecolor{currentstroke}{rgb}{0.000000,0.000000,0.000000}%
\pgfsetstrokecolor{currentstroke}%
\pgfsetdash{}{0pt}%
\pgfpathmoveto{\pgfqpoint{2.895604in}{1.326135in}}%
\pgfpathlineto{\pgfqpoint{2.895604in}{1.837666in}}%
\pgfusepath{stroke}%
\end{pgfscope}%
\begin{pgfscope}%
\pgfpathrectangle{\pgfqpoint{0.750000in}{0.500000in}}{\pgfqpoint{4.650000in}{3.020000in}}%
\pgfusepath{clip}%
\pgfsetbuttcap%
\pgfsetroundjoin%
\pgfsetlinewidth{1.505625pt}%
\definecolor{currentstroke}{rgb}{0.000000,0.000000,0.000000}%
\pgfsetstrokecolor{currentstroke}%
\pgfsetdash{}{0pt}%
\pgfpathmoveto{\pgfqpoint{3.849739in}{1.070369in}}%
\pgfpathlineto{\pgfqpoint{3.849739in}{1.837666in}}%
\pgfusepath{stroke}%
\end{pgfscope}%
\begin{pgfscope}%
\pgfsetrectcap%
\pgfsetmiterjoin%
\pgfsetlinewidth{0.803000pt}%
\definecolor{currentstroke}{rgb}{0.000000,0.000000,0.000000}%
\pgfsetstrokecolor{currentstroke}%
\pgfsetdash{}{0pt}%
\pgfpathmoveto{\pgfqpoint{0.750000in}{0.500000in}}%
\pgfpathlineto{\pgfqpoint{0.750000in}{3.520000in}}%
\pgfusepath{stroke}%
\end{pgfscope}%
\begin{pgfscope}%
\pgfsetrectcap%
\pgfsetmiterjoin%
\pgfsetlinewidth{0.803000pt}%
\definecolor{currentstroke}{rgb}{0.000000,0.000000,0.000000}%
\pgfsetstrokecolor{currentstroke}%
\pgfsetdash{}{0pt}%
\pgfpathmoveto{\pgfqpoint{5.400000in}{0.500000in}}%
\pgfpathlineto{\pgfqpoint{5.400000in}{3.520000in}}%
\pgfusepath{stroke}%
\end{pgfscope}%
\begin{pgfscope}%
\pgfsetrectcap%
\pgfsetmiterjoin%
\pgfsetlinewidth{0.803000pt}%
\definecolor{currentstroke}{rgb}{0.000000,0.000000,0.000000}%
\pgfsetstrokecolor{currentstroke}%
\pgfsetdash{}{0pt}%
\pgfpathmoveto{\pgfqpoint{0.750000in}{0.500000in}}%
\pgfpathlineto{\pgfqpoint{5.400000in}{0.500000in}}%
\pgfusepath{stroke}%
\end{pgfscope}%
\begin{pgfscope}%
\pgfsetrectcap%
\pgfsetmiterjoin%
\pgfsetlinewidth{0.803000pt}%
\definecolor{currentstroke}{rgb}{0.000000,0.000000,0.000000}%
\pgfsetstrokecolor{currentstroke}%
\pgfsetdash{}{0pt}%
\pgfpathmoveto{\pgfqpoint{0.750000in}{3.520000in}}%
\pgfpathlineto{\pgfqpoint{5.400000in}{3.520000in}}%
\pgfusepath{stroke}%
\end{pgfscope}%
\begin{pgfscope}%
\definecolor{textcolor}{rgb}{0.000000,0.000000,0.000000}%
\pgfsetstrokecolor{textcolor}%
\pgfsetfillcolor{textcolor}%
\pgftext[x=1.463999in, y=1.213116in, left, base]{\color{textcolor}\sffamily\fontsize{10.000000}{12.000000}\selectfont end of }%
\end{pgfscope}%
\begin{pgfscope}%
\definecolor{textcolor}{rgb}{0.000000,0.000000,0.000000}%
\pgfsetstrokecolor{textcolor}%
\pgfsetfillcolor{textcolor}%
\pgftext[x=1.321638in, y=1.070369in, left, base]{\color{textcolor}\sffamily\fontsize{10.000000}{12.000000}\selectfont  \(\displaystyle 5 \, \mathrm{MeV}\) drop}%
\end{pgfscope}%
\begin{pgfscope}%
\definecolor{textcolor}{rgb}{0.000000,0.000000,0.000000}%
\pgfsetstrokecolor{textcolor}%
\pgfsetfillcolor{textcolor}%
\pgftext[x=2.565743in, y=1.213116in, left, base]{\color{textcolor}\sffamily\fontsize{10.000000}{12.000000}\selectfont breakup of }%
\end{pgfscope}%
\begin{pgfscope}%
\definecolor{textcolor}{rgb}{0.000000,0.000000,0.000000}%
\pgfsetstrokecolor{textcolor}%
\pgfsetfillcolor{textcolor}%
\pgftext[x=2.446337in, y=1.070369in, left, base]{\color{textcolor}\sffamily\fontsize{10.000000}{12.000000}\selectfont  \(\displaystyle 250 \, \mathrm{MeV}\) driver}%
\end{pgfscope}%
\begin{pgfscope}%
\definecolor{textcolor}{rgb}{0.000000,0.000000,0.000000}%
\pgfsetstrokecolor{textcolor}%
\pgfsetfillcolor{textcolor}%
\pgftext[x=3.519877in, y=0.880621in, left, base]{\color{textcolor}\sffamily\fontsize{10.000000}{12.000000}\selectfont breakup of }%
\end{pgfscope}%
\begin{pgfscope}%
\definecolor{textcolor}{rgb}{0.000000,0.000000,0.000000}%
\pgfsetstrokecolor{textcolor}%
\pgfsetfillcolor{textcolor}%
\pgftext[x=3.400472in, y=0.737875in, left, base]{\color{textcolor}\sffamily\fontsize{10.000000}{12.000000}\selectfont  \(\displaystyle 300 \, \mathrm{MeV}\) driver}%
\end{pgfscope}%
\begin{pgfscope}%
\pgfsetbuttcap%
\pgfsetmiterjoin%
\definecolor{currentfill}{rgb}{1.000000,1.000000,1.000000}%
\pgfsetfillcolor{currentfill}%
\pgfsetfillopacity{0.800000}%
\pgfsetlinewidth{1.003750pt}%
\definecolor{currentstroke}{rgb}{0.800000,0.800000,0.800000}%
\pgfsetstrokecolor{currentstroke}%
\pgfsetstrokeopacity{0.800000}%
\pgfsetdash{}{0pt}%
\pgfpathmoveto{\pgfqpoint{4.331713in}{2.827871in}}%
\pgfpathlineto{\pgfqpoint{5.302778in}{2.827871in}}%
\pgfpathquadraticcurveto{\pgfqpoint{5.330556in}{2.827871in}}{\pgfqpoint{5.330556in}{2.855648in}}%
\pgfpathlineto{\pgfqpoint{5.330556in}{3.422778in}}%
\pgfpathquadraticcurveto{\pgfqpoint{5.330556in}{3.450556in}}{\pgfqpoint{5.302778in}{3.450556in}}%
\pgfpathlineto{\pgfqpoint{4.331713in}{3.450556in}}%
\pgfpathquadraticcurveto{\pgfqpoint{4.303935in}{3.450556in}}{\pgfqpoint{4.303935in}{3.422778in}}%
\pgfpathlineto{\pgfqpoint{4.303935in}{2.855648in}}%
\pgfpathquadraticcurveto{\pgfqpoint{4.303935in}{2.827871in}}{\pgfqpoint{4.331713in}{2.827871in}}%
\pgfpathlineto{\pgfqpoint{4.331713in}{2.827871in}}%
\pgfpathclose%
\pgfusepath{stroke,fill}%
\end{pgfscope}%
\begin{pgfscope}%
\pgfsetrectcap%
\pgfsetroundjoin%
\pgfsetlinewidth{1.505625pt}%
\definecolor{currentstroke}{rgb}{0.121569,0.466667,0.705882}%
\pgfsetstrokecolor{currentstroke}%
\pgfsetdash{}{0pt}%
\pgfpathmoveto{\pgfqpoint{4.359490in}{3.346389in}}%
\pgfpathlineto{\pgfqpoint{4.498379in}{3.346389in}}%
\pgfpathlineto{\pgfqpoint{4.637268in}{3.346389in}}%
\pgfusepath{stroke}%
\end{pgfscope}%
\begin{pgfscope}%
\pgfsetbuttcap%
\pgfsetroundjoin%
\definecolor{currentfill}{rgb}{0.121569,0.466667,0.705882}%
\pgfsetfillcolor{currentfill}%
\pgfsetlinewidth{1.003750pt}%
\definecolor{currentstroke}{rgb}{0.121569,0.466667,0.705882}%
\pgfsetstrokecolor{currentstroke}%
\pgfsetdash{}{0pt}%
\pgfsys@defobject{currentmarker}{\pgfqpoint{-0.020833in}{-0.020833in}}{\pgfqpoint{0.020833in}{0.020833in}}{%
\pgfpathmoveto{\pgfqpoint{0.000000in}{-0.020833in}}%
\pgfpathcurveto{\pgfqpoint{0.005525in}{-0.020833in}}{\pgfqpoint{0.010825in}{-0.018638in}}{\pgfqpoint{0.014731in}{-0.014731in}}%
\pgfpathcurveto{\pgfqpoint{0.018638in}{-0.010825in}}{\pgfqpoint{0.020833in}{-0.005525in}}{\pgfqpoint{0.020833in}{0.000000in}}%
\pgfpathcurveto{\pgfqpoint{0.020833in}{0.005525in}}{\pgfqpoint{0.018638in}{0.010825in}}{\pgfqpoint{0.014731in}{0.014731in}}%
\pgfpathcurveto{\pgfqpoint{0.010825in}{0.018638in}}{\pgfqpoint{0.005525in}{0.020833in}}{\pgfqpoint{0.000000in}{0.020833in}}%
\pgfpathcurveto{\pgfqpoint{-0.005525in}{0.020833in}}{\pgfqpoint{-0.010825in}{0.018638in}}{\pgfqpoint{-0.014731in}{0.014731in}}%
\pgfpathcurveto{\pgfqpoint{-0.018638in}{0.010825in}}{\pgfqpoint{-0.020833in}{0.005525in}}{\pgfqpoint{-0.020833in}{0.000000in}}%
\pgfpathcurveto{\pgfqpoint{-0.020833in}{-0.005525in}}{\pgfqpoint{-0.018638in}{-0.010825in}}{\pgfqpoint{-0.014731in}{-0.014731in}}%
\pgfpathcurveto{\pgfqpoint{-0.010825in}{-0.018638in}}{\pgfqpoint{-0.005525in}{-0.020833in}}{\pgfqpoint{0.000000in}{-0.020833in}}%
\pgfpathlineto{\pgfqpoint{0.000000in}{-0.020833in}}%
\pgfpathclose%
\pgfusepath{stroke,fill}%
}%
\begin{pgfscope}%
\pgfsys@transformshift{4.498379in}{3.346389in}%
\pgfsys@useobject{currentmarker}{}%
\end{pgfscope}%
\end{pgfscope}%
\begin{pgfscope}%
\definecolor{textcolor}{rgb}{0.000000,0.000000,0.000000}%
\pgfsetstrokecolor{textcolor}%
\pgfsetfillcolor{textcolor}%
\pgftext[x=4.748379in,y=3.297778in,left,base]{\color{textcolor}\sffamily\fontsize{10.000000}{12.000000}\selectfont \(\displaystyle 250 \, \mathrm{MeV}\)}%
\end{pgfscope}%
\begin{pgfscope}%
\pgfsetrectcap%
\pgfsetroundjoin%
\pgfsetlinewidth{1.505625pt}%
\definecolor{currentstroke}{rgb}{1.000000,0.498039,0.054902}%
\pgfsetstrokecolor{currentstroke}%
\pgfsetdash{}{0pt}%
\pgfpathmoveto{\pgfqpoint{4.359490in}{3.152716in}}%
\pgfpathlineto{\pgfqpoint{4.498379in}{3.152716in}}%
\pgfpathlineto{\pgfqpoint{4.637268in}{3.152716in}}%
\pgfusepath{stroke}%
\end{pgfscope}%
\begin{pgfscope}%
\pgfsetbuttcap%
\pgfsetroundjoin%
\definecolor{currentfill}{rgb}{1.000000,0.498039,0.054902}%
\pgfsetfillcolor{currentfill}%
\pgfsetlinewidth{1.003750pt}%
\definecolor{currentstroke}{rgb}{1.000000,0.498039,0.054902}%
\pgfsetstrokecolor{currentstroke}%
\pgfsetdash{}{0pt}%
\pgfsys@defobject{currentmarker}{\pgfqpoint{-0.020833in}{-0.020833in}}{\pgfqpoint{0.020833in}{0.020833in}}{%
\pgfpathmoveto{\pgfqpoint{0.000000in}{-0.020833in}}%
\pgfpathcurveto{\pgfqpoint{0.005525in}{-0.020833in}}{\pgfqpoint{0.010825in}{-0.018638in}}{\pgfqpoint{0.014731in}{-0.014731in}}%
\pgfpathcurveto{\pgfqpoint{0.018638in}{-0.010825in}}{\pgfqpoint{0.020833in}{-0.005525in}}{\pgfqpoint{0.020833in}{0.000000in}}%
\pgfpathcurveto{\pgfqpoint{0.020833in}{0.005525in}}{\pgfqpoint{0.018638in}{0.010825in}}{\pgfqpoint{0.014731in}{0.014731in}}%
\pgfpathcurveto{\pgfqpoint{0.010825in}{0.018638in}}{\pgfqpoint{0.005525in}{0.020833in}}{\pgfqpoint{0.000000in}{0.020833in}}%
\pgfpathcurveto{\pgfqpoint{-0.005525in}{0.020833in}}{\pgfqpoint{-0.010825in}{0.018638in}}{\pgfqpoint{-0.014731in}{0.014731in}}%
\pgfpathcurveto{\pgfqpoint{-0.018638in}{0.010825in}}{\pgfqpoint{-0.020833in}{0.005525in}}{\pgfqpoint{-0.020833in}{0.000000in}}%
\pgfpathcurveto{\pgfqpoint{-0.020833in}{-0.005525in}}{\pgfqpoint{-0.018638in}{-0.010825in}}{\pgfqpoint{-0.014731in}{-0.014731in}}%
\pgfpathcurveto{\pgfqpoint{-0.010825in}{-0.018638in}}{\pgfqpoint{-0.005525in}{-0.020833in}}{\pgfqpoint{0.000000in}{-0.020833in}}%
\pgfpathlineto{\pgfqpoint{0.000000in}{-0.020833in}}%
\pgfpathclose%
\pgfusepath{stroke,fill}%
}%
\begin{pgfscope}%
\pgfsys@transformshift{4.498379in}{3.152716in}%
\pgfsys@useobject{currentmarker}{}%
\end{pgfscope}%
\end{pgfscope}%
\begin{pgfscope}%
\definecolor{textcolor}{rgb}{0.000000,0.000000,0.000000}%
\pgfsetstrokecolor{textcolor}%
\pgfsetfillcolor{textcolor}%
\pgftext[x=4.748379in,y=3.104105in,left,base]{\color{textcolor}\sffamily\fontsize{10.000000}{12.000000}\selectfont \(\displaystyle 300 \, \mathrm{MeV}\)}%
\end{pgfscope}%
\begin{pgfscope}%
\pgfsetrectcap%
\pgfsetroundjoin%
\pgfsetlinewidth{1.505625pt}%
\definecolor{currentstroke}{rgb}{0.172549,0.627451,0.172549}%
\pgfsetstrokecolor{currentstroke}%
\pgfsetdash{}{0pt}%
\pgfpathmoveto{\pgfqpoint{4.359490in}{2.959043in}}%
\pgfpathlineto{\pgfqpoint{4.498379in}{2.959043in}}%
\pgfpathlineto{\pgfqpoint{4.637268in}{2.959043in}}%
\pgfusepath{stroke}%
\end{pgfscope}%
\begin{pgfscope}%
\pgfsetbuttcap%
\pgfsetroundjoin%
\definecolor{currentfill}{rgb}{0.172549,0.627451,0.172549}%
\pgfsetfillcolor{currentfill}%
\pgfsetlinewidth{1.003750pt}%
\definecolor{currentstroke}{rgb}{0.172549,0.627451,0.172549}%
\pgfsetstrokecolor{currentstroke}%
\pgfsetdash{}{0pt}%
\pgfsys@defobject{currentmarker}{\pgfqpoint{-0.020833in}{-0.020833in}}{\pgfqpoint{0.020833in}{0.020833in}}{%
\pgfpathmoveto{\pgfqpoint{0.000000in}{-0.020833in}}%
\pgfpathcurveto{\pgfqpoint{0.005525in}{-0.020833in}}{\pgfqpoint{0.010825in}{-0.018638in}}{\pgfqpoint{0.014731in}{-0.014731in}}%
\pgfpathcurveto{\pgfqpoint{0.018638in}{-0.010825in}}{\pgfqpoint{0.020833in}{-0.005525in}}{\pgfqpoint{0.020833in}{0.000000in}}%
\pgfpathcurveto{\pgfqpoint{0.020833in}{0.005525in}}{\pgfqpoint{0.018638in}{0.010825in}}{\pgfqpoint{0.014731in}{0.014731in}}%
\pgfpathcurveto{\pgfqpoint{0.010825in}{0.018638in}}{\pgfqpoint{0.005525in}{0.020833in}}{\pgfqpoint{0.000000in}{0.020833in}}%
\pgfpathcurveto{\pgfqpoint{-0.005525in}{0.020833in}}{\pgfqpoint{-0.010825in}{0.018638in}}{\pgfqpoint{-0.014731in}{0.014731in}}%
\pgfpathcurveto{\pgfqpoint{-0.018638in}{0.010825in}}{\pgfqpoint{-0.020833in}{0.005525in}}{\pgfqpoint{-0.020833in}{0.000000in}}%
\pgfpathcurveto{\pgfqpoint{-0.020833in}{-0.005525in}}{\pgfqpoint{-0.018638in}{-0.010825in}}{\pgfqpoint{-0.014731in}{-0.014731in}}%
\pgfpathcurveto{\pgfqpoint{-0.010825in}{-0.018638in}}{\pgfqpoint{-0.005525in}{-0.020833in}}{\pgfqpoint{0.000000in}{-0.020833in}}%
\pgfpathlineto{\pgfqpoint{0.000000in}{-0.020833in}}%
\pgfpathclose%
\pgfusepath{stroke,fill}%
}%
\begin{pgfscope}%
\pgfsys@transformshift{4.498379in}{2.959043in}%
\pgfsys@useobject{currentmarker}{}%
\end{pgfscope}%
\end{pgfscope}%
\begin{pgfscope}%
\definecolor{textcolor}{rgb}{0.000000,0.000000,0.000000}%
\pgfsetstrokecolor{textcolor}%
\pgfsetfillcolor{textcolor}%
\pgftext[x=4.748379in,y=2.910432in,left,base]{\color{textcolor}\sffamily\fontsize{10.000000}{12.000000}\selectfont \(\displaystyle 350 \, \mathrm{MeV}\)}%
\end{pgfscope}%
\end{pgfpicture}%
\makeatother%
\endgroup%

	\caption{Loss of peak energy for different initial energies are plotted over distance in plasma $y$.}
	\label{fig:E_peak_energy}
\end{figure}

For all energies the initial energy drop of \qtyrange{6}{7}{\MeV} can be observed, as well as the slower loss of peak energy afterwards. The initial drop-off ends for all energies after traversing the plasma for a distance of \qty{1}{\mm}, with the high energy drivers having bigger energy losses. 
While the \qty{250}{\MeV} driver has a visible plateau, there are smaller or no plateaus for the other two energies. Still, the energy loss happens slower at the point after the initial drop-off and only increases with advancing distance in the plasma.

The slow energy loss afterwards exists for all drivers. While the difference between the \qty{250}{\MeV} and \qty{300}{\MeV} curve stays roughly the same except for some fluctuations, 
the difference between the \qty{300}{\MeV} and \qty{350}{\MeV} curve increases with time, meaning that the \qty{350}{\MeV} peak loses its energy faster. The points of bunch breakup are also marked, which seem to be uncorrelated to the progression of the curve.

Even when there is no plateau for higher initial energies, the energy loss beyond the initial drop-off is comparably small before the end of the gas jet in experiments at around \qty{3}{\mm} \cite{Schoebel2022}.

\paragraph*{Divergence comparison}\hspace{0pt} \\
When the same analysis is done on the different initial divergences, as seen in \autoref{fig:E_peak_divergence}, big qualitative differences become apparent. 
\begin{figure}
	\centering
	%% Creator: Matplotlib, PGF backend
%%
%% To include the figure in your LaTeX document, write
%%   \input{<filename>.pgf}
%%
%% Make sure the required packages are loaded in your preamble
%%   \usepackage{pgf}
%%
%% Also ensure that all the required font packages are loaded; for instance,
%% the lmodern package is sometimes necessary when using math font.
%%   \usepackage{lmodern}
%%
%% Figures using additional raster images can only be included by \input if
%% they are in the same directory as the main LaTeX file. For loading figures
%% from other directories you can use the `import` package
%%   \usepackage{import}
%%
%% and then include the figures with
%%   \import{<path to file>}{<filename>.pgf}
%%
%% Matplotlib used the following preamble
%%
\begingroup%
\makeatletter%
\begin{pgfpicture}%
\pgfpathrectangle{\pgfpointorigin}{\pgfqpoint{6.000000in}{4.000000in}}%
\pgfusepath{use as bounding box, clip}%
\begin{pgfscope}%
\pgfsetbuttcap%
\pgfsetmiterjoin%
\pgfsetlinewidth{0.000000pt}%
\definecolor{currentstroke}{rgb}{1.000000,1.000000,1.000000}%
\pgfsetstrokecolor{currentstroke}%
\pgfsetstrokeopacity{0.000000}%
\pgfsetdash{}{0pt}%
\pgfpathmoveto{\pgfqpoint{0.000000in}{0.000000in}}%
\pgfpathlineto{\pgfqpoint{6.000000in}{0.000000in}}%
\pgfpathlineto{\pgfqpoint{6.000000in}{4.000000in}}%
\pgfpathlineto{\pgfqpoint{0.000000in}{4.000000in}}%
\pgfpathlineto{\pgfqpoint{0.000000in}{0.000000in}}%
\pgfpathclose%
\pgfusepath{}%
\end{pgfscope}%
\begin{pgfscope}%
\pgfsetbuttcap%
\pgfsetmiterjoin%
\definecolor{currentfill}{rgb}{1.000000,1.000000,1.000000}%
\pgfsetfillcolor{currentfill}%
\pgfsetlinewidth{0.000000pt}%
\definecolor{currentstroke}{rgb}{0.000000,0.000000,0.000000}%
\pgfsetstrokecolor{currentstroke}%
\pgfsetstrokeopacity{0.000000}%
\pgfsetdash{}{0pt}%
\pgfpathmoveto{\pgfqpoint{0.750000in}{0.500000in}}%
\pgfpathlineto{\pgfqpoint{5.400000in}{0.500000in}}%
\pgfpathlineto{\pgfqpoint{5.400000in}{3.520000in}}%
\pgfpathlineto{\pgfqpoint{0.750000in}{3.520000in}}%
\pgfpathlineto{\pgfqpoint{0.750000in}{0.500000in}}%
\pgfpathclose%
\pgfusepath{fill}%
\end{pgfscope}%
\begin{pgfscope}%
\pgfsetbuttcap%
\pgfsetroundjoin%
\definecolor{currentfill}{rgb}{0.000000,0.000000,0.000000}%
\pgfsetfillcolor{currentfill}%
\pgfsetlinewidth{0.803000pt}%
\definecolor{currentstroke}{rgb}{0.000000,0.000000,0.000000}%
\pgfsetstrokecolor{currentstroke}%
\pgfsetdash{}{0pt}%
\pgfsys@defobject{currentmarker}{\pgfqpoint{0.000000in}{-0.048611in}}{\pgfqpoint{0.000000in}{0.000000in}}{%
\pgfpathmoveto{\pgfqpoint{0.000000in}{0.000000in}}%
\pgfpathlineto{\pgfqpoint{0.000000in}{-0.048611in}}%
\pgfusepath{stroke,fill}%
}%
\begin{pgfscope}%
\pgfsys@transformshift{0.987335in}{0.500000in}%
\pgfsys@useobject{currentmarker}{}%
\end{pgfscope}%
\end{pgfscope}%
\begin{pgfscope}%
\definecolor{textcolor}{rgb}{0.000000,0.000000,0.000000}%
\pgfsetstrokecolor{textcolor}%
\pgfsetfillcolor{textcolor}%
\pgftext[x=0.987335in,y=0.402778in,,top]{\color{textcolor}\sffamily\fontsize{10.000000}{12.000000}\selectfont \(\displaystyle {0}\)}%
\end{pgfscope}%
\begin{pgfscope}%
\pgfsetbuttcap%
\pgfsetroundjoin%
\definecolor{currentfill}{rgb}{0.000000,0.000000,0.000000}%
\pgfsetfillcolor{currentfill}%
\pgfsetlinewidth{0.803000pt}%
\definecolor{currentstroke}{rgb}{0.000000,0.000000,0.000000}%
\pgfsetstrokecolor{currentstroke}%
\pgfsetdash{}{0pt}%
\pgfsys@defobject{currentmarker}{\pgfqpoint{0.000000in}{-0.048611in}}{\pgfqpoint{0.000000in}{0.000000in}}{%
\pgfpathmoveto{\pgfqpoint{0.000000in}{0.000000in}}%
\pgfpathlineto{\pgfqpoint{0.000000in}{-0.048611in}}%
\pgfusepath{stroke,fill}%
}%
\begin{pgfscope}%
\pgfsys@transformshift{1.668860in}{0.500000in}%
\pgfsys@useobject{currentmarker}{}%
\end{pgfscope}%
\end{pgfscope}%
\begin{pgfscope}%
\definecolor{textcolor}{rgb}{0.000000,0.000000,0.000000}%
\pgfsetstrokecolor{textcolor}%
\pgfsetfillcolor{textcolor}%
\pgftext[x=1.668860in,y=0.402778in,,top]{\color{textcolor}\sffamily\fontsize{10.000000}{12.000000}\selectfont \(\displaystyle {1}\)}%
\end{pgfscope}%
\begin{pgfscope}%
\pgfsetbuttcap%
\pgfsetroundjoin%
\definecolor{currentfill}{rgb}{0.000000,0.000000,0.000000}%
\pgfsetfillcolor{currentfill}%
\pgfsetlinewidth{0.803000pt}%
\definecolor{currentstroke}{rgb}{0.000000,0.000000,0.000000}%
\pgfsetstrokecolor{currentstroke}%
\pgfsetdash{}{0pt}%
\pgfsys@defobject{currentmarker}{\pgfqpoint{0.000000in}{-0.048611in}}{\pgfqpoint{0.000000in}{0.000000in}}{%
\pgfpathmoveto{\pgfqpoint{0.000000in}{0.000000in}}%
\pgfpathlineto{\pgfqpoint{0.000000in}{-0.048611in}}%
\pgfusepath{stroke,fill}%
}%
\begin{pgfscope}%
\pgfsys@transformshift{2.350385in}{0.500000in}%
\pgfsys@useobject{currentmarker}{}%
\end{pgfscope}%
\end{pgfscope}%
\begin{pgfscope}%
\definecolor{textcolor}{rgb}{0.000000,0.000000,0.000000}%
\pgfsetstrokecolor{textcolor}%
\pgfsetfillcolor{textcolor}%
\pgftext[x=2.350385in,y=0.402778in,,top]{\color{textcolor}\sffamily\fontsize{10.000000}{12.000000}\selectfont \(\displaystyle {2}\)}%
\end{pgfscope}%
\begin{pgfscope}%
\pgfsetbuttcap%
\pgfsetroundjoin%
\definecolor{currentfill}{rgb}{0.000000,0.000000,0.000000}%
\pgfsetfillcolor{currentfill}%
\pgfsetlinewidth{0.803000pt}%
\definecolor{currentstroke}{rgb}{0.000000,0.000000,0.000000}%
\pgfsetstrokecolor{currentstroke}%
\pgfsetdash{}{0pt}%
\pgfsys@defobject{currentmarker}{\pgfqpoint{0.000000in}{-0.048611in}}{\pgfqpoint{0.000000in}{0.000000in}}{%
\pgfpathmoveto{\pgfqpoint{0.000000in}{0.000000in}}%
\pgfpathlineto{\pgfqpoint{0.000000in}{-0.048611in}}%
\pgfusepath{stroke,fill}%
}%
\begin{pgfscope}%
\pgfsys@transformshift{3.031909in}{0.500000in}%
\pgfsys@useobject{currentmarker}{}%
\end{pgfscope}%
\end{pgfscope}%
\begin{pgfscope}%
\definecolor{textcolor}{rgb}{0.000000,0.000000,0.000000}%
\pgfsetstrokecolor{textcolor}%
\pgfsetfillcolor{textcolor}%
\pgftext[x=3.031909in,y=0.402778in,,top]{\color{textcolor}\sffamily\fontsize{10.000000}{12.000000}\selectfont \(\displaystyle {3}\)}%
\end{pgfscope}%
\begin{pgfscope}%
\pgfsetbuttcap%
\pgfsetroundjoin%
\definecolor{currentfill}{rgb}{0.000000,0.000000,0.000000}%
\pgfsetfillcolor{currentfill}%
\pgfsetlinewidth{0.803000pt}%
\definecolor{currentstroke}{rgb}{0.000000,0.000000,0.000000}%
\pgfsetstrokecolor{currentstroke}%
\pgfsetdash{}{0pt}%
\pgfsys@defobject{currentmarker}{\pgfqpoint{0.000000in}{-0.048611in}}{\pgfqpoint{0.000000in}{0.000000in}}{%
\pgfpathmoveto{\pgfqpoint{0.000000in}{0.000000in}}%
\pgfpathlineto{\pgfqpoint{0.000000in}{-0.048611in}}%
\pgfusepath{stroke,fill}%
}%
\begin{pgfscope}%
\pgfsys@transformshift{3.713434in}{0.500000in}%
\pgfsys@useobject{currentmarker}{}%
\end{pgfscope}%
\end{pgfscope}%
\begin{pgfscope}%
\definecolor{textcolor}{rgb}{0.000000,0.000000,0.000000}%
\pgfsetstrokecolor{textcolor}%
\pgfsetfillcolor{textcolor}%
\pgftext[x=3.713434in,y=0.402778in,,top]{\color{textcolor}\sffamily\fontsize{10.000000}{12.000000}\selectfont \(\displaystyle {4}\)}%
\end{pgfscope}%
\begin{pgfscope}%
\pgfsetbuttcap%
\pgfsetroundjoin%
\definecolor{currentfill}{rgb}{0.000000,0.000000,0.000000}%
\pgfsetfillcolor{currentfill}%
\pgfsetlinewidth{0.803000pt}%
\definecolor{currentstroke}{rgb}{0.000000,0.000000,0.000000}%
\pgfsetstrokecolor{currentstroke}%
\pgfsetdash{}{0pt}%
\pgfsys@defobject{currentmarker}{\pgfqpoint{0.000000in}{-0.048611in}}{\pgfqpoint{0.000000in}{0.000000in}}{%
\pgfpathmoveto{\pgfqpoint{0.000000in}{0.000000in}}%
\pgfpathlineto{\pgfqpoint{0.000000in}{-0.048611in}}%
\pgfusepath{stroke,fill}%
}%
\begin{pgfscope}%
\pgfsys@transformshift{4.394959in}{0.500000in}%
\pgfsys@useobject{currentmarker}{}%
\end{pgfscope}%
\end{pgfscope}%
\begin{pgfscope}%
\definecolor{textcolor}{rgb}{0.000000,0.000000,0.000000}%
\pgfsetstrokecolor{textcolor}%
\pgfsetfillcolor{textcolor}%
\pgftext[x=4.394959in,y=0.402778in,,top]{\color{textcolor}\sffamily\fontsize{10.000000}{12.000000}\selectfont \(\displaystyle {5}\)}%
\end{pgfscope}%
\begin{pgfscope}%
\pgfsetbuttcap%
\pgfsetroundjoin%
\definecolor{currentfill}{rgb}{0.000000,0.000000,0.000000}%
\pgfsetfillcolor{currentfill}%
\pgfsetlinewidth{0.803000pt}%
\definecolor{currentstroke}{rgb}{0.000000,0.000000,0.000000}%
\pgfsetstrokecolor{currentstroke}%
\pgfsetdash{}{0pt}%
\pgfsys@defobject{currentmarker}{\pgfqpoint{0.000000in}{-0.048611in}}{\pgfqpoint{0.000000in}{0.000000in}}{%
\pgfpathmoveto{\pgfqpoint{0.000000in}{0.000000in}}%
\pgfpathlineto{\pgfqpoint{0.000000in}{-0.048611in}}%
\pgfusepath{stroke,fill}%
}%
\begin{pgfscope}%
\pgfsys@transformshift{5.076483in}{0.500000in}%
\pgfsys@useobject{currentmarker}{}%
\end{pgfscope}%
\end{pgfscope}%
\begin{pgfscope}%
\definecolor{textcolor}{rgb}{0.000000,0.000000,0.000000}%
\pgfsetstrokecolor{textcolor}%
\pgfsetfillcolor{textcolor}%
\pgftext[x=5.076483in,y=0.402778in,,top]{\color{textcolor}\sffamily\fontsize{10.000000}{12.000000}\selectfont \(\displaystyle {6}\)}%
\end{pgfscope}%
\begin{pgfscope}%
\definecolor{textcolor}{rgb}{0.000000,0.000000,0.000000}%
\pgfsetstrokecolor{textcolor}%
\pgfsetfillcolor{textcolor}%
\pgftext[x=3.075000in,y=0.223766in,,top]{\color{textcolor}\sffamily\fontsize{10.000000}{12.000000}\selectfont \(\displaystyle y \, \mathrm{[mm]}\)}%
\end{pgfscope}%
\begin{pgfscope}%
\pgfsetbuttcap%
\pgfsetroundjoin%
\definecolor{currentfill}{rgb}{0.000000,0.000000,0.000000}%
\pgfsetfillcolor{currentfill}%
\pgfsetlinewidth{0.803000pt}%
\definecolor{currentstroke}{rgb}{0.000000,0.000000,0.000000}%
\pgfsetstrokecolor{currentstroke}%
\pgfsetdash{}{0pt}%
\pgfsys@defobject{currentmarker}{\pgfqpoint{-0.048611in}{0.000000in}}{\pgfqpoint{-0.000000in}{0.000000in}}{%
\pgfpathmoveto{\pgfqpoint{-0.000000in}{0.000000in}}%
\pgfpathlineto{\pgfqpoint{-0.048611in}{0.000000in}}%
\pgfusepath{stroke,fill}%
}%
\begin{pgfscope}%
\pgfsys@transformshift{0.750000in}{0.515117in}%
\pgfsys@useobject{currentmarker}{}%
\end{pgfscope}%
\end{pgfscope}%
\begin{pgfscope}%
\definecolor{textcolor}{rgb}{0.000000,0.000000,0.000000}%
\pgfsetstrokecolor{textcolor}%
\pgfsetfillcolor{textcolor}%
\pgftext[x=0.405863in, y=0.466891in, left, base]{\color{textcolor}\sffamily\fontsize{10.000000}{12.000000}\selectfont \(\displaystyle {\ensuremath{-}16}\)}%
\end{pgfscope}%
\begin{pgfscope}%
\pgfsetbuttcap%
\pgfsetroundjoin%
\definecolor{currentfill}{rgb}{0.000000,0.000000,0.000000}%
\pgfsetfillcolor{currentfill}%
\pgfsetlinewidth{0.803000pt}%
\definecolor{currentstroke}{rgb}{0.000000,0.000000,0.000000}%
\pgfsetstrokecolor{currentstroke}%
\pgfsetdash{}{0pt}%
\pgfsys@defobject{currentmarker}{\pgfqpoint{-0.048611in}{0.000000in}}{\pgfqpoint{-0.000000in}{0.000000in}}{%
\pgfpathmoveto{\pgfqpoint{-0.000000in}{0.000000in}}%
\pgfpathlineto{\pgfqpoint{-0.048611in}{0.000000in}}%
\pgfusepath{stroke,fill}%
}%
\begin{pgfscope}%
\pgfsys@transformshift{0.750000in}{0.872653in}%
\pgfsys@useobject{currentmarker}{}%
\end{pgfscope}%
\end{pgfscope}%
\begin{pgfscope}%
\definecolor{textcolor}{rgb}{0.000000,0.000000,0.000000}%
\pgfsetstrokecolor{textcolor}%
\pgfsetfillcolor{textcolor}%
\pgftext[x=0.405863in, y=0.824428in, left, base]{\color{textcolor}\sffamily\fontsize{10.000000}{12.000000}\selectfont \(\displaystyle {\ensuremath{-}14}\)}%
\end{pgfscope}%
\begin{pgfscope}%
\pgfsetbuttcap%
\pgfsetroundjoin%
\definecolor{currentfill}{rgb}{0.000000,0.000000,0.000000}%
\pgfsetfillcolor{currentfill}%
\pgfsetlinewidth{0.803000pt}%
\definecolor{currentstroke}{rgb}{0.000000,0.000000,0.000000}%
\pgfsetstrokecolor{currentstroke}%
\pgfsetdash{}{0pt}%
\pgfsys@defobject{currentmarker}{\pgfqpoint{-0.048611in}{0.000000in}}{\pgfqpoint{-0.000000in}{0.000000in}}{%
\pgfpathmoveto{\pgfqpoint{-0.000000in}{0.000000in}}%
\pgfpathlineto{\pgfqpoint{-0.048611in}{0.000000in}}%
\pgfusepath{stroke,fill}%
}%
\begin{pgfscope}%
\pgfsys@transformshift{0.750000in}{1.230190in}%
\pgfsys@useobject{currentmarker}{}%
\end{pgfscope}%
\end{pgfscope}%
\begin{pgfscope}%
\definecolor{textcolor}{rgb}{0.000000,0.000000,0.000000}%
\pgfsetstrokecolor{textcolor}%
\pgfsetfillcolor{textcolor}%
\pgftext[x=0.405863in, y=1.181965in, left, base]{\color{textcolor}\sffamily\fontsize{10.000000}{12.000000}\selectfont \(\displaystyle {\ensuremath{-}12}\)}%
\end{pgfscope}%
\begin{pgfscope}%
\pgfsetbuttcap%
\pgfsetroundjoin%
\definecolor{currentfill}{rgb}{0.000000,0.000000,0.000000}%
\pgfsetfillcolor{currentfill}%
\pgfsetlinewidth{0.803000pt}%
\definecolor{currentstroke}{rgb}{0.000000,0.000000,0.000000}%
\pgfsetstrokecolor{currentstroke}%
\pgfsetdash{}{0pt}%
\pgfsys@defobject{currentmarker}{\pgfqpoint{-0.048611in}{0.000000in}}{\pgfqpoint{-0.000000in}{0.000000in}}{%
\pgfpathmoveto{\pgfqpoint{-0.000000in}{0.000000in}}%
\pgfpathlineto{\pgfqpoint{-0.048611in}{0.000000in}}%
\pgfusepath{stroke,fill}%
}%
\begin{pgfscope}%
\pgfsys@transformshift{0.750000in}{1.587727in}%
\pgfsys@useobject{currentmarker}{}%
\end{pgfscope}%
\end{pgfscope}%
\begin{pgfscope}%
\definecolor{textcolor}{rgb}{0.000000,0.000000,0.000000}%
\pgfsetstrokecolor{textcolor}%
\pgfsetfillcolor{textcolor}%
\pgftext[x=0.405863in, y=1.539501in, left, base]{\color{textcolor}\sffamily\fontsize{10.000000}{12.000000}\selectfont \(\displaystyle {\ensuremath{-}10}\)}%
\end{pgfscope}%
\begin{pgfscope}%
\pgfsetbuttcap%
\pgfsetroundjoin%
\definecolor{currentfill}{rgb}{0.000000,0.000000,0.000000}%
\pgfsetfillcolor{currentfill}%
\pgfsetlinewidth{0.803000pt}%
\definecolor{currentstroke}{rgb}{0.000000,0.000000,0.000000}%
\pgfsetstrokecolor{currentstroke}%
\pgfsetdash{}{0pt}%
\pgfsys@defobject{currentmarker}{\pgfqpoint{-0.048611in}{0.000000in}}{\pgfqpoint{-0.000000in}{0.000000in}}{%
\pgfpathmoveto{\pgfqpoint{-0.000000in}{0.000000in}}%
\pgfpathlineto{\pgfqpoint{-0.048611in}{0.000000in}}%
\pgfusepath{stroke,fill}%
}%
\begin{pgfscope}%
\pgfsys@transformshift{0.750000in}{1.945263in}%
\pgfsys@useobject{currentmarker}{}%
\end{pgfscope}%
\end{pgfscope}%
\begin{pgfscope}%
\definecolor{textcolor}{rgb}{0.000000,0.000000,0.000000}%
\pgfsetstrokecolor{textcolor}%
\pgfsetfillcolor{textcolor}%
\pgftext[x=0.475308in, y=1.897038in, left, base]{\color{textcolor}\sffamily\fontsize{10.000000}{12.000000}\selectfont \(\displaystyle {\ensuremath{-}8}\)}%
\end{pgfscope}%
\begin{pgfscope}%
\pgfsetbuttcap%
\pgfsetroundjoin%
\definecolor{currentfill}{rgb}{0.000000,0.000000,0.000000}%
\pgfsetfillcolor{currentfill}%
\pgfsetlinewidth{0.803000pt}%
\definecolor{currentstroke}{rgb}{0.000000,0.000000,0.000000}%
\pgfsetstrokecolor{currentstroke}%
\pgfsetdash{}{0pt}%
\pgfsys@defobject{currentmarker}{\pgfqpoint{-0.048611in}{0.000000in}}{\pgfqpoint{-0.000000in}{0.000000in}}{%
\pgfpathmoveto{\pgfqpoint{-0.000000in}{0.000000in}}%
\pgfpathlineto{\pgfqpoint{-0.048611in}{0.000000in}}%
\pgfusepath{stroke,fill}%
}%
\begin{pgfscope}%
\pgfsys@transformshift{0.750000in}{2.302800in}%
\pgfsys@useobject{currentmarker}{}%
\end{pgfscope}%
\end{pgfscope}%
\begin{pgfscope}%
\definecolor{textcolor}{rgb}{0.000000,0.000000,0.000000}%
\pgfsetstrokecolor{textcolor}%
\pgfsetfillcolor{textcolor}%
\pgftext[x=0.475308in, y=2.254575in, left, base]{\color{textcolor}\sffamily\fontsize{10.000000}{12.000000}\selectfont \(\displaystyle {\ensuremath{-}6}\)}%
\end{pgfscope}%
\begin{pgfscope}%
\pgfsetbuttcap%
\pgfsetroundjoin%
\definecolor{currentfill}{rgb}{0.000000,0.000000,0.000000}%
\pgfsetfillcolor{currentfill}%
\pgfsetlinewidth{0.803000pt}%
\definecolor{currentstroke}{rgb}{0.000000,0.000000,0.000000}%
\pgfsetstrokecolor{currentstroke}%
\pgfsetdash{}{0pt}%
\pgfsys@defobject{currentmarker}{\pgfqpoint{-0.048611in}{0.000000in}}{\pgfqpoint{-0.000000in}{0.000000in}}{%
\pgfpathmoveto{\pgfqpoint{-0.000000in}{0.000000in}}%
\pgfpathlineto{\pgfqpoint{-0.048611in}{0.000000in}}%
\pgfusepath{stroke,fill}%
}%
\begin{pgfscope}%
\pgfsys@transformshift{0.750000in}{2.660336in}%
\pgfsys@useobject{currentmarker}{}%
\end{pgfscope}%
\end{pgfscope}%
\begin{pgfscope}%
\definecolor{textcolor}{rgb}{0.000000,0.000000,0.000000}%
\pgfsetstrokecolor{textcolor}%
\pgfsetfillcolor{textcolor}%
\pgftext[x=0.475308in, y=2.612111in, left, base]{\color{textcolor}\sffamily\fontsize{10.000000}{12.000000}\selectfont \(\displaystyle {\ensuremath{-}4}\)}%
\end{pgfscope}%
\begin{pgfscope}%
\pgfsetbuttcap%
\pgfsetroundjoin%
\definecolor{currentfill}{rgb}{0.000000,0.000000,0.000000}%
\pgfsetfillcolor{currentfill}%
\pgfsetlinewidth{0.803000pt}%
\definecolor{currentstroke}{rgb}{0.000000,0.000000,0.000000}%
\pgfsetstrokecolor{currentstroke}%
\pgfsetdash{}{0pt}%
\pgfsys@defobject{currentmarker}{\pgfqpoint{-0.048611in}{0.000000in}}{\pgfqpoint{-0.000000in}{0.000000in}}{%
\pgfpathmoveto{\pgfqpoint{-0.000000in}{0.000000in}}%
\pgfpathlineto{\pgfqpoint{-0.048611in}{0.000000in}}%
\pgfusepath{stroke,fill}%
}%
\begin{pgfscope}%
\pgfsys@transformshift{0.750000in}{3.017873in}%
\pgfsys@useobject{currentmarker}{}%
\end{pgfscope}%
\end{pgfscope}%
\begin{pgfscope}%
\definecolor{textcolor}{rgb}{0.000000,0.000000,0.000000}%
\pgfsetstrokecolor{textcolor}%
\pgfsetfillcolor{textcolor}%
\pgftext[x=0.475308in, y=2.969648in, left, base]{\color{textcolor}\sffamily\fontsize{10.000000}{12.000000}\selectfont \(\displaystyle {\ensuremath{-}2}\)}%
\end{pgfscope}%
\begin{pgfscope}%
\pgfsetbuttcap%
\pgfsetroundjoin%
\definecolor{currentfill}{rgb}{0.000000,0.000000,0.000000}%
\pgfsetfillcolor{currentfill}%
\pgfsetlinewidth{0.803000pt}%
\definecolor{currentstroke}{rgb}{0.000000,0.000000,0.000000}%
\pgfsetstrokecolor{currentstroke}%
\pgfsetdash{}{0pt}%
\pgfsys@defobject{currentmarker}{\pgfqpoint{-0.048611in}{0.000000in}}{\pgfqpoint{-0.000000in}{0.000000in}}{%
\pgfpathmoveto{\pgfqpoint{-0.000000in}{0.000000in}}%
\pgfpathlineto{\pgfqpoint{-0.048611in}{0.000000in}}%
\pgfusepath{stroke,fill}%
}%
\begin{pgfscope}%
\pgfsys@transformshift{0.750000in}{3.375410in}%
\pgfsys@useobject{currentmarker}{}%
\end{pgfscope}%
\end{pgfscope}%
\begin{pgfscope}%
\definecolor{textcolor}{rgb}{0.000000,0.000000,0.000000}%
\pgfsetstrokecolor{textcolor}%
\pgfsetfillcolor{textcolor}%
\pgftext[x=0.583333in, y=3.327184in, left, base]{\color{textcolor}\sffamily\fontsize{10.000000}{12.000000}\selectfont \(\displaystyle {0}\)}%
\end{pgfscope}%
\begin{pgfscope}%
\definecolor{textcolor}{rgb}{0.000000,0.000000,0.000000}%
\pgfsetstrokecolor{textcolor}%
\pgfsetfillcolor{textcolor}%
\pgftext[x=0.350308in,y=2.010000in,,bottom,rotate=90.000000]{\color{textcolor}\sffamily\fontsize{10.000000}{12.000000}\selectfont \(\displaystyle \mathrm{energy \, change \, [MeV]}\)}%
\end{pgfscope}%
\begin{pgfscope}%
\pgfpathrectangle{\pgfqpoint{0.750000in}{0.500000in}}{\pgfqpoint{4.650000in}{3.020000in}}%
\pgfusepath{clip}%
\pgfsetrectcap%
\pgfsetroundjoin%
\pgfsetlinewidth{1.505625pt}%
\definecolor{currentstroke}{rgb}{0.121569,0.466667,0.705882}%
\pgfsetstrokecolor{currentstroke}%
\pgfsetdash{}{0pt}%
\pgfpathmoveto{\pgfqpoint{0.961364in}{3.233698in}}%
\pgfpathlineto{\pgfqpoint{1.015559in}{3.116881in}}%
\pgfpathlineto{\pgfqpoint{1.069755in}{3.196641in}}%
\pgfpathlineto{\pgfqpoint{1.123951in}{2.725176in}}%
\pgfpathlineto{\pgfqpoint{1.178147in}{2.605398in}}%
\pgfpathlineto{\pgfqpoint{1.232343in}{2.558762in}}%
\pgfpathlineto{\pgfqpoint{1.286538in}{2.550960in}}%
\pgfpathlineto{\pgfqpoint{1.340734in}{2.434352in}}%
\pgfpathlineto{\pgfqpoint{1.394930in}{2.335743in}}%
\pgfpathlineto{\pgfqpoint{1.449126in}{2.288445in}}%
\pgfpathlineto{\pgfqpoint{1.503322in}{2.262179in}}%
\pgfpathlineto{\pgfqpoint{1.557517in}{2.254209in}}%
\pgfpathlineto{\pgfqpoint{1.611713in}{2.262981in}}%
\pgfpathlineto{\pgfqpoint{1.665909in}{2.271799in}}%
\pgfpathlineto{\pgfqpoint{1.720105in}{2.279483in}}%
\pgfpathlineto{\pgfqpoint{1.774301in}{2.288571in}}%
\pgfpathlineto{\pgfqpoint{1.828497in}{2.298997in}}%
\pgfpathlineto{\pgfqpoint{1.882692in}{2.304684in}}%
\pgfpathlineto{\pgfqpoint{1.936888in}{2.297078in}}%
\pgfpathlineto{\pgfqpoint{1.991084in}{2.296810in}}%
\pgfpathlineto{\pgfqpoint{2.045280in}{2.304674in}}%
\pgfpathlineto{\pgfqpoint{2.099476in}{2.316671in}}%
\pgfpathlineto{\pgfqpoint{2.153671in}{2.314603in}}%
\pgfpathlineto{\pgfqpoint{2.207867in}{2.318088in}}%
\pgfpathlineto{\pgfqpoint{2.262063in}{2.315675in}}%
\pgfpathlineto{\pgfqpoint{2.316259in}{2.324832in}}%
\pgfpathlineto{\pgfqpoint{2.370455in}{2.331900in}}%
\pgfpathlineto{\pgfqpoint{2.424650in}{2.339171in}}%
\pgfpathlineto{\pgfqpoint{2.478846in}{2.338382in}}%
\pgfpathlineto{\pgfqpoint{2.533042in}{2.336923in}}%
\pgfpathlineto{\pgfqpoint{2.587238in}{2.332626in}}%
\pgfpathlineto{\pgfqpoint{2.641434in}{2.337135in}}%
\pgfpathlineto{\pgfqpoint{2.695629in}{2.339937in}}%
\pgfpathlineto{\pgfqpoint{2.749825in}{2.338204in}}%
\pgfpathlineto{\pgfqpoint{2.804021in}{2.340186in}}%
\pgfpathlineto{\pgfqpoint{2.858217in}{2.341340in}}%
\pgfpathlineto{\pgfqpoint{2.912413in}{2.338149in}}%
\pgfpathlineto{\pgfqpoint{2.966608in}{2.336322in}}%
\pgfpathlineto{\pgfqpoint{3.020804in}{2.335164in}}%
\pgfpathlineto{\pgfqpoint{3.075000in}{2.340414in}}%
\pgfpathlineto{\pgfqpoint{3.129196in}{2.344489in}}%
\pgfpathlineto{\pgfqpoint{3.183392in}{2.348355in}}%
\pgfpathlineto{\pgfqpoint{3.237587in}{2.349513in}}%
\pgfpathlineto{\pgfqpoint{3.291783in}{2.354223in}}%
\pgfpathlineto{\pgfqpoint{3.345979in}{2.353607in}}%
\pgfpathlineto{\pgfqpoint{3.400175in}{2.360883in}}%
\pgfpathlineto{\pgfqpoint{3.454371in}{2.372188in}}%
\pgfpathlineto{\pgfqpoint{3.508566in}{2.380245in}}%
\pgfpathlineto{\pgfqpoint{3.562762in}{2.385487in}}%
\pgfpathlineto{\pgfqpoint{3.616958in}{2.383343in}}%
\pgfpathlineto{\pgfqpoint{3.671154in}{2.389304in}}%
\pgfpathlineto{\pgfqpoint{3.725350in}{2.392462in}}%
\pgfpathlineto{\pgfqpoint{3.779545in}{2.394946in}}%
\pgfpathlineto{\pgfqpoint{3.833741in}{2.397003in}}%
\pgfpathlineto{\pgfqpoint{3.887937in}{2.395304in}}%
\pgfpathlineto{\pgfqpoint{3.942133in}{2.391236in}}%
\pgfpathlineto{\pgfqpoint{3.996329in}{2.387273in}}%
\pgfpathlineto{\pgfqpoint{4.050524in}{2.386901in}}%
\pgfpathlineto{\pgfqpoint{4.104720in}{2.388937in}}%
\pgfpathlineto{\pgfqpoint{4.158916in}{2.384286in}}%
\pgfpathlineto{\pgfqpoint{4.213112in}{2.371508in}}%
\pgfpathlineto{\pgfqpoint{4.267308in}{2.361115in}}%
\pgfpathlineto{\pgfqpoint{4.321503in}{2.360021in}}%
\pgfpathlineto{\pgfqpoint{4.375699in}{2.366027in}}%
\pgfpathlineto{\pgfqpoint{4.429895in}{2.370205in}}%
\pgfpathlineto{\pgfqpoint{4.484091in}{2.372974in}}%
\pgfpathlineto{\pgfqpoint{4.538287in}{2.369191in}}%
\pgfpathlineto{\pgfqpoint{4.592483in}{2.361260in}}%
\pgfpathlineto{\pgfqpoint{4.646678in}{2.360278in}}%
\pgfpathlineto{\pgfqpoint{4.700874in}{2.360364in}}%
\pgfpathlineto{\pgfqpoint{4.755070in}{2.356860in}}%
\pgfpathlineto{\pgfqpoint{4.809266in}{2.347621in}}%
\pgfpathlineto{\pgfqpoint{4.863462in}{2.339134in}}%
\pgfpathlineto{\pgfqpoint{4.917657in}{2.337515in}}%
\pgfpathlineto{\pgfqpoint{4.971853in}{2.332920in}}%
\pgfpathlineto{\pgfqpoint{5.026049in}{2.325537in}}%
\pgfpathlineto{\pgfqpoint{5.080245in}{2.317008in}}%
\pgfpathlineto{\pgfqpoint{5.134441in}{2.306898in}}%
\pgfpathlineto{\pgfqpoint{5.188636in}{2.299585in}}%
\pgfusepath{stroke}%
\end{pgfscope}%
\begin{pgfscope}%
\pgfpathrectangle{\pgfqpoint{0.750000in}{0.500000in}}{\pgfqpoint{4.650000in}{3.020000in}}%
\pgfusepath{clip}%
\pgfsetbuttcap%
\pgfsetroundjoin%
\definecolor{currentfill}{rgb}{0.121569,0.466667,0.705882}%
\pgfsetfillcolor{currentfill}%
\pgfsetlinewidth{1.003750pt}%
\definecolor{currentstroke}{rgb}{0.121569,0.466667,0.705882}%
\pgfsetstrokecolor{currentstroke}%
\pgfsetdash{}{0pt}%
\pgfsys@defobject{currentmarker}{\pgfqpoint{-0.020833in}{-0.020833in}}{\pgfqpoint{0.020833in}{0.020833in}}{%
\pgfpathmoveto{\pgfqpoint{0.000000in}{-0.020833in}}%
\pgfpathcurveto{\pgfqpoint{0.005525in}{-0.020833in}}{\pgfqpoint{0.010825in}{-0.018638in}}{\pgfqpoint{0.014731in}{-0.014731in}}%
\pgfpathcurveto{\pgfqpoint{0.018638in}{-0.010825in}}{\pgfqpoint{0.020833in}{-0.005525in}}{\pgfqpoint{0.020833in}{0.000000in}}%
\pgfpathcurveto{\pgfqpoint{0.020833in}{0.005525in}}{\pgfqpoint{0.018638in}{0.010825in}}{\pgfqpoint{0.014731in}{0.014731in}}%
\pgfpathcurveto{\pgfqpoint{0.010825in}{0.018638in}}{\pgfqpoint{0.005525in}{0.020833in}}{\pgfqpoint{0.000000in}{0.020833in}}%
\pgfpathcurveto{\pgfqpoint{-0.005525in}{0.020833in}}{\pgfqpoint{-0.010825in}{0.018638in}}{\pgfqpoint{-0.014731in}{0.014731in}}%
\pgfpathcurveto{\pgfqpoint{-0.018638in}{0.010825in}}{\pgfqpoint{-0.020833in}{0.005525in}}{\pgfqpoint{-0.020833in}{0.000000in}}%
\pgfpathcurveto{\pgfqpoint{-0.020833in}{-0.005525in}}{\pgfqpoint{-0.018638in}{-0.010825in}}{\pgfqpoint{-0.014731in}{-0.014731in}}%
\pgfpathcurveto{\pgfqpoint{-0.010825in}{-0.018638in}}{\pgfqpoint{-0.005525in}{-0.020833in}}{\pgfqpoint{0.000000in}{-0.020833in}}%
\pgfpathlineto{\pgfqpoint{0.000000in}{-0.020833in}}%
\pgfpathclose%
\pgfusepath{stroke,fill}%
}%
\begin{pgfscope}%
\pgfsys@transformshift{0.961364in}{3.233698in}%
\pgfsys@useobject{currentmarker}{}%
\end{pgfscope}%
\begin{pgfscope}%
\pgfsys@transformshift{1.015559in}{3.116881in}%
\pgfsys@useobject{currentmarker}{}%
\end{pgfscope}%
\begin{pgfscope}%
\pgfsys@transformshift{1.069755in}{3.196641in}%
\pgfsys@useobject{currentmarker}{}%
\end{pgfscope}%
\begin{pgfscope}%
\pgfsys@transformshift{1.123951in}{2.725176in}%
\pgfsys@useobject{currentmarker}{}%
\end{pgfscope}%
\begin{pgfscope}%
\pgfsys@transformshift{1.178147in}{2.605398in}%
\pgfsys@useobject{currentmarker}{}%
\end{pgfscope}%
\begin{pgfscope}%
\pgfsys@transformshift{1.232343in}{2.558762in}%
\pgfsys@useobject{currentmarker}{}%
\end{pgfscope}%
\begin{pgfscope}%
\pgfsys@transformshift{1.286538in}{2.550960in}%
\pgfsys@useobject{currentmarker}{}%
\end{pgfscope}%
\begin{pgfscope}%
\pgfsys@transformshift{1.340734in}{2.434352in}%
\pgfsys@useobject{currentmarker}{}%
\end{pgfscope}%
\begin{pgfscope}%
\pgfsys@transformshift{1.394930in}{2.335743in}%
\pgfsys@useobject{currentmarker}{}%
\end{pgfscope}%
\begin{pgfscope}%
\pgfsys@transformshift{1.449126in}{2.288445in}%
\pgfsys@useobject{currentmarker}{}%
\end{pgfscope}%
\begin{pgfscope}%
\pgfsys@transformshift{1.503322in}{2.262179in}%
\pgfsys@useobject{currentmarker}{}%
\end{pgfscope}%
\begin{pgfscope}%
\pgfsys@transformshift{1.557517in}{2.254209in}%
\pgfsys@useobject{currentmarker}{}%
\end{pgfscope}%
\begin{pgfscope}%
\pgfsys@transformshift{1.611713in}{2.262981in}%
\pgfsys@useobject{currentmarker}{}%
\end{pgfscope}%
\begin{pgfscope}%
\pgfsys@transformshift{1.665909in}{2.271799in}%
\pgfsys@useobject{currentmarker}{}%
\end{pgfscope}%
\begin{pgfscope}%
\pgfsys@transformshift{1.720105in}{2.279483in}%
\pgfsys@useobject{currentmarker}{}%
\end{pgfscope}%
\begin{pgfscope}%
\pgfsys@transformshift{1.774301in}{2.288571in}%
\pgfsys@useobject{currentmarker}{}%
\end{pgfscope}%
\begin{pgfscope}%
\pgfsys@transformshift{1.828497in}{2.298997in}%
\pgfsys@useobject{currentmarker}{}%
\end{pgfscope}%
\begin{pgfscope}%
\pgfsys@transformshift{1.882692in}{2.304684in}%
\pgfsys@useobject{currentmarker}{}%
\end{pgfscope}%
\begin{pgfscope}%
\pgfsys@transformshift{1.936888in}{2.297078in}%
\pgfsys@useobject{currentmarker}{}%
\end{pgfscope}%
\begin{pgfscope}%
\pgfsys@transformshift{1.991084in}{2.296810in}%
\pgfsys@useobject{currentmarker}{}%
\end{pgfscope}%
\begin{pgfscope}%
\pgfsys@transformshift{2.045280in}{2.304674in}%
\pgfsys@useobject{currentmarker}{}%
\end{pgfscope}%
\begin{pgfscope}%
\pgfsys@transformshift{2.099476in}{2.316671in}%
\pgfsys@useobject{currentmarker}{}%
\end{pgfscope}%
\begin{pgfscope}%
\pgfsys@transformshift{2.153671in}{2.314603in}%
\pgfsys@useobject{currentmarker}{}%
\end{pgfscope}%
\begin{pgfscope}%
\pgfsys@transformshift{2.207867in}{2.318088in}%
\pgfsys@useobject{currentmarker}{}%
\end{pgfscope}%
\begin{pgfscope}%
\pgfsys@transformshift{2.262063in}{2.315675in}%
\pgfsys@useobject{currentmarker}{}%
\end{pgfscope}%
\begin{pgfscope}%
\pgfsys@transformshift{2.316259in}{2.324832in}%
\pgfsys@useobject{currentmarker}{}%
\end{pgfscope}%
\begin{pgfscope}%
\pgfsys@transformshift{2.370455in}{2.331900in}%
\pgfsys@useobject{currentmarker}{}%
\end{pgfscope}%
\begin{pgfscope}%
\pgfsys@transformshift{2.424650in}{2.339171in}%
\pgfsys@useobject{currentmarker}{}%
\end{pgfscope}%
\begin{pgfscope}%
\pgfsys@transformshift{2.478846in}{2.338382in}%
\pgfsys@useobject{currentmarker}{}%
\end{pgfscope}%
\begin{pgfscope}%
\pgfsys@transformshift{2.533042in}{2.336923in}%
\pgfsys@useobject{currentmarker}{}%
\end{pgfscope}%
\begin{pgfscope}%
\pgfsys@transformshift{2.587238in}{2.332626in}%
\pgfsys@useobject{currentmarker}{}%
\end{pgfscope}%
\begin{pgfscope}%
\pgfsys@transformshift{2.641434in}{2.337135in}%
\pgfsys@useobject{currentmarker}{}%
\end{pgfscope}%
\begin{pgfscope}%
\pgfsys@transformshift{2.695629in}{2.339937in}%
\pgfsys@useobject{currentmarker}{}%
\end{pgfscope}%
\begin{pgfscope}%
\pgfsys@transformshift{2.749825in}{2.338204in}%
\pgfsys@useobject{currentmarker}{}%
\end{pgfscope}%
\begin{pgfscope}%
\pgfsys@transformshift{2.804021in}{2.340186in}%
\pgfsys@useobject{currentmarker}{}%
\end{pgfscope}%
\begin{pgfscope}%
\pgfsys@transformshift{2.858217in}{2.341340in}%
\pgfsys@useobject{currentmarker}{}%
\end{pgfscope}%
\begin{pgfscope}%
\pgfsys@transformshift{2.912413in}{2.338149in}%
\pgfsys@useobject{currentmarker}{}%
\end{pgfscope}%
\begin{pgfscope}%
\pgfsys@transformshift{2.966608in}{2.336322in}%
\pgfsys@useobject{currentmarker}{}%
\end{pgfscope}%
\begin{pgfscope}%
\pgfsys@transformshift{3.020804in}{2.335164in}%
\pgfsys@useobject{currentmarker}{}%
\end{pgfscope}%
\begin{pgfscope}%
\pgfsys@transformshift{3.075000in}{2.340414in}%
\pgfsys@useobject{currentmarker}{}%
\end{pgfscope}%
\begin{pgfscope}%
\pgfsys@transformshift{3.129196in}{2.344489in}%
\pgfsys@useobject{currentmarker}{}%
\end{pgfscope}%
\begin{pgfscope}%
\pgfsys@transformshift{3.183392in}{2.348355in}%
\pgfsys@useobject{currentmarker}{}%
\end{pgfscope}%
\begin{pgfscope}%
\pgfsys@transformshift{3.237587in}{2.349513in}%
\pgfsys@useobject{currentmarker}{}%
\end{pgfscope}%
\begin{pgfscope}%
\pgfsys@transformshift{3.291783in}{2.354223in}%
\pgfsys@useobject{currentmarker}{}%
\end{pgfscope}%
\begin{pgfscope}%
\pgfsys@transformshift{3.345979in}{2.353607in}%
\pgfsys@useobject{currentmarker}{}%
\end{pgfscope}%
\begin{pgfscope}%
\pgfsys@transformshift{3.400175in}{2.360883in}%
\pgfsys@useobject{currentmarker}{}%
\end{pgfscope}%
\begin{pgfscope}%
\pgfsys@transformshift{3.454371in}{2.372188in}%
\pgfsys@useobject{currentmarker}{}%
\end{pgfscope}%
\begin{pgfscope}%
\pgfsys@transformshift{3.508566in}{2.380245in}%
\pgfsys@useobject{currentmarker}{}%
\end{pgfscope}%
\begin{pgfscope}%
\pgfsys@transformshift{3.562762in}{2.385487in}%
\pgfsys@useobject{currentmarker}{}%
\end{pgfscope}%
\begin{pgfscope}%
\pgfsys@transformshift{3.616958in}{2.383343in}%
\pgfsys@useobject{currentmarker}{}%
\end{pgfscope}%
\begin{pgfscope}%
\pgfsys@transformshift{3.671154in}{2.389304in}%
\pgfsys@useobject{currentmarker}{}%
\end{pgfscope}%
\begin{pgfscope}%
\pgfsys@transformshift{3.725350in}{2.392462in}%
\pgfsys@useobject{currentmarker}{}%
\end{pgfscope}%
\begin{pgfscope}%
\pgfsys@transformshift{3.779545in}{2.394946in}%
\pgfsys@useobject{currentmarker}{}%
\end{pgfscope}%
\begin{pgfscope}%
\pgfsys@transformshift{3.833741in}{2.397003in}%
\pgfsys@useobject{currentmarker}{}%
\end{pgfscope}%
\begin{pgfscope}%
\pgfsys@transformshift{3.887937in}{2.395304in}%
\pgfsys@useobject{currentmarker}{}%
\end{pgfscope}%
\begin{pgfscope}%
\pgfsys@transformshift{3.942133in}{2.391236in}%
\pgfsys@useobject{currentmarker}{}%
\end{pgfscope}%
\begin{pgfscope}%
\pgfsys@transformshift{3.996329in}{2.387273in}%
\pgfsys@useobject{currentmarker}{}%
\end{pgfscope}%
\begin{pgfscope}%
\pgfsys@transformshift{4.050524in}{2.386901in}%
\pgfsys@useobject{currentmarker}{}%
\end{pgfscope}%
\begin{pgfscope}%
\pgfsys@transformshift{4.104720in}{2.388937in}%
\pgfsys@useobject{currentmarker}{}%
\end{pgfscope}%
\begin{pgfscope}%
\pgfsys@transformshift{4.158916in}{2.384286in}%
\pgfsys@useobject{currentmarker}{}%
\end{pgfscope}%
\begin{pgfscope}%
\pgfsys@transformshift{4.213112in}{2.371508in}%
\pgfsys@useobject{currentmarker}{}%
\end{pgfscope}%
\begin{pgfscope}%
\pgfsys@transformshift{4.267308in}{2.361115in}%
\pgfsys@useobject{currentmarker}{}%
\end{pgfscope}%
\begin{pgfscope}%
\pgfsys@transformshift{4.321503in}{2.360021in}%
\pgfsys@useobject{currentmarker}{}%
\end{pgfscope}%
\begin{pgfscope}%
\pgfsys@transformshift{4.375699in}{2.366027in}%
\pgfsys@useobject{currentmarker}{}%
\end{pgfscope}%
\begin{pgfscope}%
\pgfsys@transformshift{4.429895in}{2.370205in}%
\pgfsys@useobject{currentmarker}{}%
\end{pgfscope}%
\begin{pgfscope}%
\pgfsys@transformshift{4.484091in}{2.372974in}%
\pgfsys@useobject{currentmarker}{}%
\end{pgfscope}%
\begin{pgfscope}%
\pgfsys@transformshift{4.538287in}{2.369191in}%
\pgfsys@useobject{currentmarker}{}%
\end{pgfscope}%
\begin{pgfscope}%
\pgfsys@transformshift{4.592483in}{2.361260in}%
\pgfsys@useobject{currentmarker}{}%
\end{pgfscope}%
\begin{pgfscope}%
\pgfsys@transformshift{4.646678in}{2.360278in}%
\pgfsys@useobject{currentmarker}{}%
\end{pgfscope}%
\begin{pgfscope}%
\pgfsys@transformshift{4.700874in}{2.360364in}%
\pgfsys@useobject{currentmarker}{}%
\end{pgfscope}%
\begin{pgfscope}%
\pgfsys@transformshift{4.755070in}{2.356860in}%
\pgfsys@useobject{currentmarker}{}%
\end{pgfscope}%
\begin{pgfscope}%
\pgfsys@transformshift{4.809266in}{2.347621in}%
\pgfsys@useobject{currentmarker}{}%
\end{pgfscope}%
\begin{pgfscope}%
\pgfsys@transformshift{4.863462in}{2.339134in}%
\pgfsys@useobject{currentmarker}{}%
\end{pgfscope}%
\begin{pgfscope}%
\pgfsys@transformshift{4.917657in}{2.337515in}%
\pgfsys@useobject{currentmarker}{}%
\end{pgfscope}%
\begin{pgfscope}%
\pgfsys@transformshift{4.971853in}{2.332920in}%
\pgfsys@useobject{currentmarker}{}%
\end{pgfscope}%
\begin{pgfscope}%
\pgfsys@transformshift{5.026049in}{2.325537in}%
\pgfsys@useobject{currentmarker}{}%
\end{pgfscope}%
\begin{pgfscope}%
\pgfsys@transformshift{5.080245in}{2.317008in}%
\pgfsys@useobject{currentmarker}{}%
\end{pgfscope}%
\begin{pgfscope}%
\pgfsys@transformshift{5.134441in}{2.306898in}%
\pgfsys@useobject{currentmarker}{}%
\end{pgfscope}%
\begin{pgfscope}%
\pgfsys@transformshift{5.188636in}{2.299585in}%
\pgfsys@useobject{currentmarker}{}%
\end{pgfscope}%
\end{pgfscope}%
\begin{pgfscope}%
\pgfpathrectangle{\pgfqpoint{0.750000in}{0.500000in}}{\pgfqpoint{4.650000in}{3.020000in}}%
\pgfusepath{clip}%
\pgfsetrectcap%
\pgfsetroundjoin%
\pgfsetlinewidth{1.505625pt}%
\definecolor{currentstroke}{rgb}{1.000000,0.498039,0.054902}%
\pgfsetstrokecolor{currentstroke}%
\pgfsetdash{}{0pt}%
\pgfpathmoveto{\pgfqpoint{0.961364in}{3.248979in}}%
\pgfpathlineto{\pgfqpoint{1.015559in}{3.382727in}}%
\pgfpathlineto{\pgfqpoint{1.069755in}{3.379066in}}%
\pgfpathlineto{\pgfqpoint{1.123951in}{3.134328in}}%
\pgfpathlineto{\pgfqpoint{1.178147in}{3.068579in}}%
\pgfpathlineto{\pgfqpoint{1.232343in}{2.797473in}}%
\pgfpathlineto{\pgfqpoint{1.286538in}{2.619719in}}%
\pgfpathlineto{\pgfqpoint{1.340734in}{2.454563in}}%
\pgfpathlineto{\pgfqpoint{1.394930in}{2.351561in}}%
\pgfpathlineto{\pgfqpoint{1.449126in}{2.298216in}}%
\pgfpathlineto{\pgfqpoint{1.503322in}{2.275110in}}%
\pgfpathlineto{\pgfqpoint{1.557517in}{2.252776in}}%
\pgfpathlineto{\pgfqpoint{1.611713in}{2.242555in}}%
\pgfpathlineto{\pgfqpoint{1.665909in}{2.237362in}}%
\pgfpathlineto{\pgfqpoint{1.720105in}{2.244208in}}%
\pgfpathlineto{\pgfqpoint{1.774301in}{2.254359in}}%
\pgfpathlineto{\pgfqpoint{1.828497in}{2.255941in}}%
\pgfpathlineto{\pgfqpoint{1.882692in}{2.262938in}}%
\pgfpathlineto{\pgfqpoint{1.936888in}{2.266533in}}%
\pgfpathlineto{\pgfqpoint{1.991084in}{2.262184in}}%
\pgfpathlineto{\pgfqpoint{2.045280in}{2.263937in}}%
\pgfpathlineto{\pgfqpoint{2.099476in}{2.261061in}}%
\pgfpathlineto{\pgfqpoint{2.153671in}{2.255713in}}%
\pgfpathlineto{\pgfqpoint{2.207867in}{2.255389in}}%
\pgfpathlineto{\pgfqpoint{2.262063in}{2.257916in}}%
\pgfpathlineto{\pgfqpoint{2.316259in}{2.259452in}}%
\pgfpathlineto{\pgfqpoint{2.370455in}{2.264063in}}%
\pgfpathlineto{\pgfqpoint{2.424650in}{2.269614in}}%
\pgfpathlineto{\pgfqpoint{2.478846in}{2.270851in}}%
\pgfpathlineto{\pgfqpoint{2.533042in}{2.267773in}}%
\pgfpathlineto{\pgfqpoint{2.587238in}{2.263047in}}%
\pgfpathlineto{\pgfqpoint{2.641434in}{2.257684in}}%
\pgfpathlineto{\pgfqpoint{2.695629in}{2.251878in}}%
\pgfpathlineto{\pgfqpoint{2.749825in}{2.246380in}}%
\pgfpathlineto{\pgfqpoint{2.804021in}{2.239149in}}%
\pgfpathlineto{\pgfqpoint{2.858217in}{2.233961in}}%
\pgfpathlineto{\pgfqpoint{2.912413in}{2.225865in}}%
\pgfpathlineto{\pgfqpoint{2.966608in}{2.215020in}}%
\pgfpathlineto{\pgfqpoint{3.020804in}{2.203626in}}%
\pgfpathlineto{\pgfqpoint{3.075000in}{2.192532in}}%
\pgfpathlineto{\pgfqpoint{3.129196in}{2.181952in}}%
\pgfpathlineto{\pgfqpoint{3.183392in}{2.175174in}}%
\pgfpathlineto{\pgfqpoint{3.237587in}{2.168885in}}%
\pgfpathlineto{\pgfqpoint{3.291783in}{2.165168in}}%
\pgfpathlineto{\pgfqpoint{3.345979in}{2.158738in}}%
\pgfpathlineto{\pgfqpoint{3.400175in}{2.152196in}}%
\pgfpathlineto{\pgfqpoint{3.454371in}{2.146725in}}%
\pgfpathlineto{\pgfqpoint{3.508566in}{2.141757in}}%
\pgfpathlineto{\pgfqpoint{3.562762in}{2.134253in}}%
\pgfpathlineto{\pgfqpoint{3.616958in}{2.123351in}}%
\pgfpathlineto{\pgfqpoint{3.671154in}{2.115090in}}%
\pgfpathlineto{\pgfqpoint{3.725350in}{2.107703in}}%
\pgfpathlineto{\pgfqpoint{3.779545in}{2.098418in}}%
\pgfpathlineto{\pgfqpoint{3.833741in}{2.091087in}}%
\pgfpathlineto{\pgfqpoint{3.887937in}{2.083557in}}%
\pgfpathlineto{\pgfqpoint{3.942133in}{2.073340in}}%
\pgfpathlineto{\pgfqpoint{3.996329in}{2.061062in}}%
\pgfpathlineto{\pgfqpoint{4.050524in}{2.047672in}}%
\pgfpathlineto{\pgfqpoint{4.104720in}{2.035319in}}%
\pgfpathlineto{\pgfqpoint{4.158916in}{2.026142in}}%
\pgfpathlineto{\pgfqpoint{4.213112in}{2.016981in}}%
\pgfpathlineto{\pgfqpoint{4.267308in}{2.005540in}}%
\pgfpathlineto{\pgfqpoint{4.321503in}{1.996352in}}%
\pgfpathlineto{\pgfqpoint{4.375699in}{1.987590in}}%
\pgfpathlineto{\pgfqpoint{4.429895in}{1.977625in}}%
\pgfpathlineto{\pgfqpoint{4.484091in}{1.967691in}}%
\pgfpathlineto{\pgfqpoint{4.538287in}{1.958318in}}%
\pgfpathlineto{\pgfqpoint{4.592483in}{1.953092in}}%
\pgfpathlineto{\pgfqpoint{4.646678in}{1.941563in}}%
\pgfpathlineto{\pgfqpoint{4.700874in}{1.932011in}}%
\pgfpathlineto{\pgfqpoint{4.755070in}{1.920836in}}%
\pgfpathlineto{\pgfqpoint{4.809266in}{1.911704in}}%
\pgfpathlineto{\pgfqpoint{4.863462in}{1.902530in}}%
\pgfpathlineto{\pgfqpoint{4.917657in}{1.885017in}}%
\pgfpathlineto{\pgfqpoint{4.971853in}{1.875147in}}%
\pgfpathlineto{\pgfqpoint{5.026049in}{1.866675in}}%
\pgfpathlineto{\pgfqpoint{5.080245in}{1.857404in}}%
\pgfpathlineto{\pgfqpoint{5.134441in}{1.847271in}}%
\pgfpathlineto{\pgfqpoint{5.188636in}{1.834263in}}%
\pgfusepath{stroke}%
\end{pgfscope}%
\begin{pgfscope}%
\pgfpathrectangle{\pgfqpoint{0.750000in}{0.500000in}}{\pgfqpoint{4.650000in}{3.020000in}}%
\pgfusepath{clip}%
\pgfsetbuttcap%
\pgfsetroundjoin%
\definecolor{currentfill}{rgb}{1.000000,0.498039,0.054902}%
\pgfsetfillcolor{currentfill}%
\pgfsetlinewidth{1.003750pt}%
\definecolor{currentstroke}{rgb}{1.000000,0.498039,0.054902}%
\pgfsetstrokecolor{currentstroke}%
\pgfsetdash{}{0pt}%
\pgfsys@defobject{currentmarker}{\pgfqpoint{-0.020833in}{-0.020833in}}{\pgfqpoint{0.020833in}{0.020833in}}{%
\pgfpathmoveto{\pgfqpoint{0.000000in}{-0.020833in}}%
\pgfpathcurveto{\pgfqpoint{0.005525in}{-0.020833in}}{\pgfqpoint{0.010825in}{-0.018638in}}{\pgfqpoint{0.014731in}{-0.014731in}}%
\pgfpathcurveto{\pgfqpoint{0.018638in}{-0.010825in}}{\pgfqpoint{0.020833in}{-0.005525in}}{\pgfqpoint{0.020833in}{0.000000in}}%
\pgfpathcurveto{\pgfqpoint{0.020833in}{0.005525in}}{\pgfqpoint{0.018638in}{0.010825in}}{\pgfqpoint{0.014731in}{0.014731in}}%
\pgfpathcurveto{\pgfqpoint{0.010825in}{0.018638in}}{\pgfqpoint{0.005525in}{0.020833in}}{\pgfqpoint{0.000000in}{0.020833in}}%
\pgfpathcurveto{\pgfqpoint{-0.005525in}{0.020833in}}{\pgfqpoint{-0.010825in}{0.018638in}}{\pgfqpoint{-0.014731in}{0.014731in}}%
\pgfpathcurveto{\pgfqpoint{-0.018638in}{0.010825in}}{\pgfqpoint{-0.020833in}{0.005525in}}{\pgfqpoint{-0.020833in}{0.000000in}}%
\pgfpathcurveto{\pgfqpoint{-0.020833in}{-0.005525in}}{\pgfqpoint{-0.018638in}{-0.010825in}}{\pgfqpoint{-0.014731in}{-0.014731in}}%
\pgfpathcurveto{\pgfqpoint{-0.010825in}{-0.018638in}}{\pgfqpoint{-0.005525in}{-0.020833in}}{\pgfqpoint{0.000000in}{-0.020833in}}%
\pgfpathlineto{\pgfqpoint{0.000000in}{-0.020833in}}%
\pgfpathclose%
\pgfusepath{stroke,fill}%
}%
\begin{pgfscope}%
\pgfsys@transformshift{0.961364in}{3.248979in}%
\pgfsys@useobject{currentmarker}{}%
\end{pgfscope}%
\begin{pgfscope}%
\pgfsys@transformshift{1.015559in}{3.382727in}%
\pgfsys@useobject{currentmarker}{}%
\end{pgfscope}%
\begin{pgfscope}%
\pgfsys@transformshift{1.069755in}{3.379066in}%
\pgfsys@useobject{currentmarker}{}%
\end{pgfscope}%
\begin{pgfscope}%
\pgfsys@transformshift{1.123951in}{3.134328in}%
\pgfsys@useobject{currentmarker}{}%
\end{pgfscope}%
\begin{pgfscope}%
\pgfsys@transformshift{1.178147in}{3.068579in}%
\pgfsys@useobject{currentmarker}{}%
\end{pgfscope}%
\begin{pgfscope}%
\pgfsys@transformshift{1.232343in}{2.797473in}%
\pgfsys@useobject{currentmarker}{}%
\end{pgfscope}%
\begin{pgfscope}%
\pgfsys@transformshift{1.286538in}{2.619719in}%
\pgfsys@useobject{currentmarker}{}%
\end{pgfscope}%
\begin{pgfscope}%
\pgfsys@transformshift{1.340734in}{2.454563in}%
\pgfsys@useobject{currentmarker}{}%
\end{pgfscope}%
\begin{pgfscope}%
\pgfsys@transformshift{1.394930in}{2.351561in}%
\pgfsys@useobject{currentmarker}{}%
\end{pgfscope}%
\begin{pgfscope}%
\pgfsys@transformshift{1.449126in}{2.298216in}%
\pgfsys@useobject{currentmarker}{}%
\end{pgfscope}%
\begin{pgfscope}%
\pgfsys@transformshift{1.503322in}{2.275110in}%
\pgfsys@useobject{currentmarker}{}%
\end{pgfscope}%
\begin{pgfscope}%
\pgfsys@transformshift{1.557517in}{2.252776in}%
\pgfsys@useobject{currentmarker}{}%
\end{pgfscope}%
\begin{pgfscope}%
\pgfsys@transformshift{1.611713in}{2.242555in}%
\pgfsys@useobject{currentmarker}{}%
\end{pgfscope}%
\begin{pgfscope}%
\pgfsys@transformshift{1.665909in}{2.237362in}%
\pgfsys@useobject{currentmarker}{}%
\end{pgfscope}%
\begin{pgfscope}%
\pgfsys@transformshift{1.720105in}{2.244208in}%
\pgfsys@useobject{currentmarker}{}%
\end{pgfscope}%
\begin{pgfscope}%
\pgfsys@transformshift{1.774301in}{2.254359in}%
\pgfsys@useobject{currentmarker}{}%
\end{pgfscope}%
\begin{pgfscope}%
\pgfsys@transformshift{1.828497in}{2.255941in}%
\pgfsys@useobject{currentmarker}{}%
\end{pgfscope}%
\begin{pgfscope}%
\pgfsys@transformshift{1.882692in}{2.262938in}%
\pgfsys@useobject{currentmarker}{}%
\end{pgfscope}%
\begin{pgfscope}%
\pgfsys@transformshift{1.936888in}{2.266533in}%
\pgfsys@useobject{currentmarker}{}%
\end{pgfscope}%
\begin{pgfscope}%
\pgfsys@transformshift{1.991084in}{2.262184in}%
\pgfsys@useobject{currentmarker}{}%
\end{pgfscope}%
\begin{pgfscope}%
\pgfsys@transformshift{2.045280in}{2.263937in}%
\pgfsys@useobject{currentmarker}{}%
\end{pgfscope}%
\begin{pgfscope}%
\pgfsys@transformshift{2.099476in}{2.261061in}%
\pgfsys@useobject{currentmarker}{}%
\end{pgfscope}%
\begin{pgfscope}%
\pgfsys@transformshift{2.153671in}{2.255713in}%
\pgfsys@useobject{currentmarker}{}%
\end{pgfscope}%
\begin{pgfscope}%
\pgfsys@transformshift{2.207867in}{2.255389in}%
\pgfsys@useobject{currentmarker}{}%
\end{pgfscope}%
\begin{pgfscope}%
\pgfsys@transformshift{2.262063in}{2.257916in}%
\pgfsys@useobject{currentmarker}{}%
\end{pgfscope}%
\begin{pgfscope}%
\pgfsys@transformshift{2.316259in}{2.259452in}%
\pgfsys@useobject{currentmarker}{}%
\end{pgfscope}%
\begin{pgfscope}%
\pgfsys@transformshift{2.370455in}{2.264063in}%
\pgfsys@useobject{currentmarker}{}%
\end{pgfscope}%
\begin{pgfscope}%
\pgfsys@transformshift{2.424650in}{2.269614in}%
\pgfsys@useobject{currentmarker}{}%
\end{pgfscope}%
\begin{pgfscope}%
\pgfsys@transformshift{2.478846in}{2.270851in}%
\pgfsys@useobject{currentmarker}{}%
\end{pgfscope}%
\begin{pgfscope}%
\pgfsys@transformshift{2.533042in}{2.267773in}%
\pgfsys@useobject{currentmarker}{}%
\end{pgfscope}%
\begin{pgfscope}%
\pgfsys@transformshift{2.587238in}{2.263047in}%
\pgfsys@useobject{currentmarker}{}%
\end{pgfscope}%
\begin{pgfscope}%
\pgfsys@transformshift{2.641434in}{2.257684in}%
\pgfsys@useobject{currentmarker}{}%
\end{pgfscope}%
\begin{pgfscope}%
\pgfsys@transformshift{2.695629in}{2.251878in}%
\pgfsys@useobject{currentmarker}{}%
\end{pgfscope}%
\begin{pgfscope}%
\pgfsys@transformshift{2.749825in}{2.246380in}%
\pgfsys@useobject{currentmarker}{}%
\end{pgfscope}%
\begin{pgfscope}%
\pgfsys@transformshift{2.804021in}{2.239149in}%
\pgfsys@useobject{currentmarker}{}%
\end{pgfscope}%
\begin{pgfscope}%
\pgfsys@transformshift{2.858217in}{2.233961in}%
\pgfsys@useobject{currentmarker}{}%
\end{pgfscope}%
\begin{pgfscope}%
\pgfsys@transformshift{2.912413in}{2.225865in}%
\pgfsys@useobject{currentmarker}{}%
\end{pgfscope}%
\begin{pgfscope}%
\pgfsys@transformshift{2.966608in}{2.215020in}%
\pgfsys@useobject{currentmarker}{}%
\end{pgfscope}%
\begin{pgfscope}%
\pgfsys@transformshift{3.020804in}{2.203626in}%
\pgfsys@useobject{currentmarker}{}%
\end{pgfscope}%
\begin{pgfscope}%
\pgfsys@transformshift{3.075000in}{2.192532in}%
\pgfsys@useobject{currentmarker}{}%
\end{pgfscope}%
\begin{pgfscope}%
\pgfsys@transformshift{3.129196in}{2.181952in}%
\pgfsys@useobject{currentmarker}{}%
\end{pgfscope}%
\begin{pgfscope}%
\pgfsys@transformshift{3.183392in}{2.175174in}%
\pgfsys@useobject{currentmarker}{}%
\end{pgfscope}%
\begin{pgfscope}%
\pgfsys@transformshift{3.237587in}{2.168885in}%
\pgfsys@useobject{currentmarker}{}%
\end{pgfscope}%
\begin{pgfscope}%
\pgfsys@transformshift{3.291783in}{2.165168in}%
\pgfsys@useobject{currentmarker}{}%
\end{pgfscope}%
\begin{pgfscope}%
\pgfsys@transformshift{3.345979in}{2.158738in}%
\pgfsys@useobject{currentmarker}{}%
\end{pgfscope}%
\begin{pgfscope}%
\pgfsys@transformshift{3.400175in}{2.152196in}%
\pgfsys@useobject{currentmarker}{}%
\end{pgfscope}%
\begin{pgfscope}%
\pgfsys@transformshift{3.454371in}{2.146725in}%
\pgfsys@useobject{currentmarker}{}%
\end{pgfscope}%
\begin{pgfscope}%
\pgfsys@transformshift{3.508566in}{2.141757in}%
\pgfsys@useobject{currentmarker}{}%
\end{pgfscope}%
\begin{pgfscope}%
\pgfsys@transformshift{3.562762in}{2.134253in}%
\pgfsys@useobject{currentmarker}{}%
\end{pgfscope}%
\begin{pgfscope}%
\pgfsys@transformshift{3.616958in}{2.123351in}%
\pgfsys@useobject{currentmarker}{}%
\end{pgfscope}%
\begin{pgfscope}%
\pgfsys@transformshift{3.671154in}{2.115090in}%
\pgfsys@useobject{currentmarker}{}%
\end{pgfscope}%
\begin{pgfscope}%
\pgfsys@transformshift{3.725350in}{2.107703in}%
\pgfsys@useobject{currentmarker}{}%
\end{pgfscope}%
\begin{pgfscope}%
\pgfsys@transformshift{3.779545in}{2.098418in}%
\pgfsys@useobject{currentmarker}{}%
\end{pgfscope}%
\begin{pgfscope}%
\pgfsys@transformshift{3.833741in}{2.091087in}%
\pgfsys@useobject{currentmarker}{}%
\end{pgfscope}%
\begin{pgfscope}%
\pgfsys@transformshift{3.887937in}{2.083557in}%
\pgfsys@useobject{currentmarker}{}%
\end{pgfscope}%
\begin{pgfscope}%
\pgfsys@transformshift{3.942133in}{2.073340in}%
\pgfsys@useobject{currentmarker}{}%
\end{pgfscope}%
\begin{pgfscope}%
\pgfsys@transformshift{3.996329in}{2.061062in}%
\pgfsys@useobject{currentmarker}{}%
\end{pgfscope}%
\begin{pgfscope}%
\pgfsys@transformshift{4.050524in}{2.047672in}%
\pgfsys@useobject{currentmarker}{}%
\end{pgfscope}%
\begin{pgfscope}%
\pgfsys@transformshift{4.104720in}{2.035319in}%
\pgfsys@useobject{currentmarker}{}%
\end{pgfscope}%
\begin{pgfscope}%
\pgfsys@transformshift{4.158916in}{2.026142in}%
\pgfsys@useobject{currentmarker}{}%
\end{pgfscope}%
\begin{pgfscope}%
\pgfsys@transformshift{4.213112in}{2.016981in}%
\pgfsys@useobject{currentmarker}{}%
\end{pgfscope}%
\begin{pgfscope}%
\pgfsys@transformshift{4.267308in}{2.005540in}%
\pgfsys@useobject{currentmarker}{}%
\end{pgfscope}%
\begin{pgfscope}%
\pgfsys@transformshift{4.321503in}{1.996352in}%
\pgfsys@useobject{currentmarker}{}%
\end{pgfscope}%
\begin{pgfscope}%
\pgfsys@transformshift{4.375699in}{1.987590in}%
\pgfsys@useobject{currentmarker}{}%
\end{pgfscope}%
\begin{pgfscope}%
\pgfsys@transformshift{4.429895in}{1.977625in}%
\pgfsys@useobject{currentmarker}{}%
\end{pgfscope}%
\begin{pgfscope}%
\pgfsys@transformshift{4.484091in}{1.967691in}%
\pgfsys@useobject{currentmarker}{}%
\end{pgfscope}%
\begin{pgfscope}%
\pgfsys@transformshift{4.538287in}{1.958318in}%
\pgfsys@useobject{currentmarker}{}%
\end{pgfscope}%
\begin{pgfscope}%
\pgfsys@transformshift{4.592483in}{1.953092in}%
\pgfsys@useobject{currentmarker}{}%
\end{pgfscope}%
\begin{pgfscope}%
\pgfsys@transformshift{4.646678in}{1.941563in}%
\pgfsys@useobject{currentmarker}{}%
\end{pgfscope}%
\begin{pgfscope}%
\pgfsys@transformshift{4.700874in}{1.932011in}%
\pgfsys@useobject{currentmarker}{}%
\end{pgfscope}%
\begin{pgfscope}%
\pgfsys@transformshift{4.755070in}{1.920836in}%
\pgfsys@useobject{currentmarker}{}%
\end{pgfscope}%
\begin{pgfscope}%
\pgfsys@transformshift{4.809266in}{1.911704in}%
\pgfsys@useobject{currentmarker}{}%
\end{pgfscope}%
\begin{pgfscope}%
\pgfsys@transformshift{4.863462in}{1.902530in}%
\pgfsys@useobject{currentmarker}{}%
\end{pgfscope}%
\begin{pgfscope}%
\pgfsys@transformshift{4.917657in}{1.885017in}%
\pgfsys@useobject{currentmarker}{}%
\end{pgfscope}%
\begin{pgfscope}%
\pgfsys@transformshift{4.971853in}{1.875147in}%
\pgfsys@useobject{currentmarker}{}%
\end{pgfscope}%
\begin{pgfscope}%
\pgfsys@transformshift{5.026049in}{1.866675in}%
\pgfsys@useobject{currentmarker}{}%
\end{pgfscope}%
\begin{pgfscope}%
\pgfsys@transformshift{5.080245in}{1.857404in}%
\pgfsys@useobject{currentmarker}{}%
\end{pgfscope}%
\begin{pgfscope}%
\pgfsys@transformshift{5.134441in}{1.847271in}%
\pgfsys@useobject{currentmarker}{}%
\end{pgfscope}%
\begin{pgfscope}%
\pgfsys@transformshift{5.188636in}{1.834263in}%
\pgfsys@useobject{currentmarker}{}%
\end{pgfscope}%
\end{pgfscope}%
\begin{pgfscope}%
\pgfpathrectangle{\pgfqpoint{0.750000in}{0.500000in}}{\pgfqpoint{4.650000in}{3.020000in}}%
\pgfusepath{clip}%
\pgfsetrectcap%
\pgfsetroundjoin%
\pgfsetlinewidth{1.505625pt}%
\definecolor{currentstroke}{rgb}{0.172549,0.627451,0.172549}%
\pgfsetstrokecolor{currentstroke}%
\pgfsetdash{}{0pt}%
\pgfpathmoveto{\pgfqpoint{0.961364in}{3.351808in}}%
\pgfpathlineto{\pgfqpoint{1.015559in}{3.209892in}}%
\pgfpathlineto{\pgfqpoint{1.069755in}{3.196641in}}%
\pgfpathlineto{\pgfqpoint{1.123951in}{2.819950in}}%
\pgfpathlineto{\pgfqpoint{1.178147in}{2.703781in}}%
\pgfpathlineto{\pgfqpoint{1.232343in}{2.640405in}}%
\pgfpathlineto{\pgfqpoint{1.286538in}{2.634055in}}%
\pgfpathlineto{\pgfqpoint{1.340734in}{2.645335in}}%
\pgfpathlineto{\pgfqpoint{1.394930in}{2.630085in}}%
\pgfpathlineto{\pgfqpoint{1.449126in}{2.590284in}}%
\pgfpathlineto{\pgfqpoint{1.503322in}{2.551832in}}%
\pgfpathlineto{\pgfqpoint{1.557517in}{2.511573in}}%
\pgfpathlineto{\pgfqpoint{1.611713in}{2.476506in}}%
\pgfpathlineto{\pgfqpoint{1.665909in}{2.447596in}}%
\pgfpathlineto{\pgfqpoint{1.720105in}{2.422058in}}%
\pgfpathlineto{\pgfqpoint{1.774301in}{2.396028in}}%
\pgfpathlineto{\pgfqpoint{1.828497in}{2.372089in}}%
\pgfpathlineto{\pgfqpoint{1.882692in}{2.348255in}}%
\pgfpathlineto{\pgfqpoint{1.936888in}{2.324219in}}%
\pgfpathlineto{\pgfqpoint{1.991084in}{2.299024in}}%
\pgfpathlineto{\pgfqpoint{2.045280in}{2.274277in}}%
\pgfpathlineto{\pgfqpoint{2.099476in}{2.249529in}}%
\pgfpathlineto{\pgfqpoint{2.153671in}{2.225440in}}%
\pgfpathlineto{\pgfqpoint{2.207867in}{2.200504in}}%
\pgfpathlineto{\pgfqpoint{2.262063in}{2.171976in}}%
\pgfpathlineto{\pgfqpoint{2.316259in}{2.144624in}}%
\pgfpathlineto{\pgfqpoint{2.370455in}{2.117602in}}%
\pgfpathlineto{\pgfqpoint{2.424650in}{2.091366in}}%
\pgfpathlineto{\pgfqpoint{2.478846in}{2.063425in}}%
\pgfpathlineto{\pgfqpoint{2.533042in}{2.035447in}}%
\pgfpathlineto{\pgfqpoint{2.587238in}{2.007594in}}%
\pgfpathlineto{\pgfqpoint{2.641434in}{1.974235in}}%
\pgfpathlineto{\pgfqpoint{2.695629in}{1.943770in}}%
\pgfpathlineto{\pgfqpoint{2.749825in}{1.910365in}}%
\pgfpathlineto{\pgfqpoint{2.804021in}{1.878840in}}%
\pgfpathlineto{\pgfqpoint{2.858217in}{1.850295in}}%
\pgfpathlineto{\pgfqpoint{2.912413in}{1.818576in}}%
\pgfpathlineto{\pgfqpoint{2.966608in}{1.787330in}}%
\pgfpathlineto{\pgfqpoint{3.020804in}{1.758673in}}%
\pgfpathlineto{\pgfqpoint{3.075000in}{1.724759in}}%
\pgfpathlineto{\pgfqpoint{3.129196in}{1.697371in}}%
\pgfpathlineto{\pgfqpoint{3.183392in}{1.672131in}}%
\pgfpathlineto{\pgfqpoint{3.237587in}{1.646449in}}%
\pgfpathlineto{\pgfqpoint{3.291783in}{1.621348in}}%
\pgfpathlineto{\pgfqpoint{3.345979in}{1.601064in}}%
\pgfpathlineto{\pgfqpoint{3.400175in}{1.577671in}}%
\pgfpathlineto{\pgfqpoint{3.454371in}{1.547394in}}%
\pgfpathlineto{\pgfqpoint{3.508566in}{1.519928in}}%
\pgfpathlineto{\pgfqpoint{3.562762in}{1.489500in}}%
\pgfpathlineto{\pgfqpoint{3.616958in}{1.455674in}}%
\pgfpathlineto{\pgfqpoint{3.671154in}{1.424373in}}%
\pgfpathlineto{\pgfqpoint{3.725350in}{1.400216in}}%
\pgfpathlineto{\pgfqpoint{3.779545in}{1.379719in}}%
\pgfpathlineto{\pgfqpoint{3.833741in}{1.350957in}}%
\pgfpathlineto{\pgfqpoint{3.887937in}{1.320959in}}%
\pgfpathlineto{\pgfqpoint{3.942133in}{1.289759in}}%
\pgfpathlineto{\pgfqpoint{3.996329in}{1.272572in}}%
\pgfpathlineto{\pgfqpoint{4.050524in}{1.238322in}}%
\pgfpathlineto{\pgfqpoint{4.104720in}{1.218140in}}%
\pgfpathlineto{\pgfqpoint{4.158916in}{1.195278in}}%
\pgfpathlineto{\pgfqpoint{4.213112in}{1.175231in}}%
\pgfpathlineto{\pgfqpoint{4.267308in}{1.139661in}}%
\pgfpathlineto{\pgfqpoint{4.321503in}{1.107368in}}%
\pgfpathlineto{\pgfqpoint{4.375699in}{1.081579in}}%
\pgfpathlineto{\pgfqpoint{4.429895in}{1.059018in}}%
\pgfpathlineto{\pgfqpoint{4.484091in}{1.032387in}}%
\pgfpathlineto{\pgfqpoint{4.538287in}{1.000592in}}%
\pgfpathlineto{\pgfqpoint{4.592483in}{0.973183in}}%
\pgfpathlineto{\pgfqpoint{4.646678in}{0.944858in}}%
\pgfpathlineto{\pgfqpoint{4.700874in}{0.911726in}}%
\pgfpathlineto{\pgfqpoint{4.755070in}{0.880519in}}%
\pgfpathlineto{\pgfqpoint{4.809266in}{0.854194in}}%
\pgfpathlineto{\pgfqpoint{4.863462in}{0.829775in}}%
\pgfpathlineto{\pgfqpoint{4.917657in}{0.806260in}}%
\pgfpathlineto{\pgfqpoint{4.971853in}{0.777140in}}%
\pgfpathlineto{\pgfqpoint{5.026049in}{0.746793in}}%
\pgfpathlineto{\pgfqpoint{5.080245in}{0.705905in}}%
\pgfpathlineto{\pgfqpoint{5.134441in}{0.656764in}}%
\pgfpathlineto{\pgfqpoint{5.188636in}{0.637273in}}%
\pgfusepath{stroke}%
\end{pgfscope}%
\begin{pgfscope}%
\pgfpathrectangle{\pgfqpoint{0.750000in}{0.500000in}}{\pgfqpoint{4.650000in}{3.020000in}}%
\pgfusepath{clip}%
\pgfsetbuttcap%
\pgfsetroundjoin%
\definecolor{currentfill}{rgb}{0.172549,0.627451,0.172549}%
\pgfsetfillcolor{currentfill}%
\pgfsetlinewidth{1.003750pt}%
\definecolor{currentstroke}{rgb}{0.172549,0.627451,0.172549}%
\pgfsetstrokecolor{currentstroke}%
\pgfsetdash{}{0pt}%
\pgfsys@defobject{currentmarker}{\pgfqpoint{-0.020833in}{-0.020833in}}{\pgfqpoint{0.020833in}{0.020833in}}{%
\pgfpathmoveto{\pgfqpoint{0.000000in}{-0.020833in}}%
\pgfpathcurveto{\pgfqpoint{0.005525in}{-0.020833in}}{\pgfqpoint{0.010825in}{-0.018638in}}{\pgfqpoint{0.014731in}{-0.014731in}}%
\pgfpathcurveto{\pgfqpoint{0.018638in}{-0.010825in}}{\pgfqpoint{0.020833in}{-0.005525in}}{\pgfqpoint{0.020833in}{0.000000in}}%
\pgfpathcurveto{\pgfqpoint{0.020833in}{0.005525in}}{\pgfqpoint{0.018638in}{0.010825in}}{\pgfqpoint{0.014731in}{0.014731in}}%
\pgfpathcurveto{\pgfqpoint{0.010825in}{0.018638in}}{\pgfqpoint{0.005525in}{0.020833in}}{\pgfqpoint{0.000000in}{0.020833in}}%
\pgfpathcurveto{\pgfqpoint{-0.005525in}{0.020833in}}{\pgfqpoint{-0.010825in}{0.018638in}}{\pgfqpoint{-0.014731in}{0.014731in}}%
\pgfpathcurveto{\pgfqpoint{-0.018638in}{0.010825in}}{\pgfqpoint{-0.020833in}{0.005525in}}{\pgfqpoint{-0.020833in}{0.000000in}}%
\pgfpathcurveto{\pgfqpoint{-0.020833in}{-0.005525in}}{\pgfqpoint{-0.018638in}{-0.010825in}}{\pgfqpoint{-0.014731in}{-0.014731in}}%
\pgfpathcurveto{\pgfqpoint{-0.010825in}{-0.018638in}}{\pgfqpoint{-0.005525in}{-0.020833in}}{\pgfqpoint{0.000000in}{-0.020833in}}%
\pgfpathlineto{\pgfqpoint{0.000000in}{-0.020833in}}%
\pgfpathclose%
\pgfusepath{stroke,fill}%
}%
\begin{pgfscope}%
\pgfsys@transformshift{0.961364in}{3.351808in}%
\pgfsys@useobject{currentmarker}{}%
\end{pgfscope}%
\begin{pgfscope}%
\pgfsys@transformshift{1.015559in}{3.209892in}%
\pgfsys@useobject{currentmarker}{}%
\end{pgfscope}%
\begin{pgfscope}%
\pgfsys@transformshift{1.069755in}{3.196641in}%
\pgfsys@useobject{currentmarker}{}%
\end{pgfscope}%
\begin{pgfscope}%
\pgfsys@transformshift{1.123951in}{2.819950in}%
\pgfsys@useobject{currentmarker}{}%
\end{pgfscope}%
\begin{pgfscope}%
\pgfsys@transformshift{1.178147in}{2.703781in}%
\pgfsys@useobject{currentmarker}{}%
\end{pgfscope}%
\begin{pgfscope}%
\pgfsys@transformshift{1.232343in}{2.640405in}%
\pgfsys@useobject{currentmarker}{}%
\end{pgfscope}%
\begin{pgfscope}%
\pgfsys@transformshift{1.286538in}{2.634055in}%
\pgfsys@useobject{currentmarker}{}%
\end{pgfscope}%
\begin{pgfscope}%
\pgfsys@transformshift{1.340734in}{2.645335in}%
\pgfsys@useobject{currentmarker}{}%
\end{pgfscope}%
\begin{pgfscope}%
\pgfsys@transformshift{1.394930in}{2.630085in}%
\pgfsys@useobject{currentmarker}{}%
\end{pgfscope}%
\begin{pgfscope}%
\pgfsys@transformshift{1.449126in}{2.590284in}%
\pgfsys@useobject{currentmarker}{}%
\end{pgfscope}%
\begin{pgfscope}%
\pgfsys@transformshift{1.503322in}{2.551832in}%
\pgfsys@useobject{currentmarker}{}%
\end{pgfscope}%
\begin{pgfscope}%
\pgfsys@transformshift{1.557517in}{2.511573in}%
\pgfsys@useobject{currentmarker}{}%
\end{pgfscope}%
\begin{pgfscope}%
\pgfsys@transformshift{1.611713in}{2.476506in}%
\pgfsys@useobject{currentmarker}{}%
\end{pgfscope}%
\begin{pgfscope}%
\pgfsys@transformshift{1.665909in}{2.447596in}%
\pgfsys@useobject{currentmarker}{}%
\end{pgfscope}%
\begin{pgfscope}%
\pgfsys@transformshift{1.720105in}{2.422058in}%
\pgfsys@useobject{currentmarker}{}%
\end{pgfscope}%
\begin{pgfscope}%
\pgfsys@transformshift{1.774301in}{2.396028in}%
\pgfsys@useobject{currentmarker}{}%
\end{pgfscope}%
\begin{pgfscope}%
\pgfsys@transformshift{1.828497in}{2.372089in}%
\pgfsys@useobject{currentmarker}{}%
\end{pgfscope}%
\begin{pgfscope}%
\pgfsys@transformshift{1.882692in}{2.348255in}%
\pgfsys@useobject{currentmarker}{}%
\end{pgfscope}%
\begin{pgfscope}%
\pgfsys@transformshift{1.936888in}{2.324219in}%
\pgfsys@useobject{currentmarker}{}%
\end{pgfscope}%
\begin{pgfscope}%
\pgfsys@transformshift{1.991084in}{2.299024in}%
\pgfsys@useobject{currentmarker}{}%
\end{pgfscope}%
\begin{pgfscope}%
\pgfsys@transformshift{2.045280in}{2.274277in}%
\pgfsys@useobject{currentmarker}{}%
\end{pgfscope}%
\begin{pgfscope}%
\pgfsys@transformshift{2.099476in}{2.249529in}%
\pgfsys@useobject{currentmarker}{}%
\end{pgfscope}%
\begin{pgfscope}%
\pgfsys@transformshift{2.153671in}{2.225440in}%
\pgfsys@useobject{currentmarker}{}%
\end{pgfscope}%
\begin{pgfscope}%
\pgfsys@transformshift{2.207867in}{2.200504in}%
\pgfsys@useobject{currentmarker}{}%
\end{pgfscope}%
\begin{pgfscope}%
\pgfsys@transformshift{2.262063in}{2.171976in}%
\pgfsys@useobject{currentmarker}{}%
\end{pgfscope}%
\begin{pgfscope}%
\pgfsys@transformshift{2.316259in}{2.144624in}%
\pgfsys@useobject{currentmarker}{}%
\end{pgfscope}%
\begin{pgfscope}%
\pgfsys@transformshift{2.370455in}{2.117602in}%
\pgfsys@useobject{currentmarker}{}%
\end{pgfscope}%
\begin{pgfscope}%
\pgfsys@transformshift{2.424650in}{2.091366in}%
\pgfsys@useobject{currentmarker}{}%
\end{pgfscope}%
\begin{pgfscope}%
\pgfsys@transformshift{2.478846in}{2.063425in}%
\pgfsys@useobject{currentmarker}{}%
\end{pgfscope}%
\begin{pgfscope}%
\pgfsys@transformshift{2.533042in}{2.035447in}%
\pgfsys@useobject{currentmarker}{}%
\end{pgfscope}%
\begin{pgfscope}%
\pgfsys@transformshift{2.587238in}{2.007594in}%
\pgfsys@useobject{currentmarker}{}%
\end{pgfscope}%
\begin{pgfscope}%
\pgfsys@transformshift{2.641434in}{1.974235in}%
\pgfsys@useobject{currentmarker}{}%
\end{pgfscope}%
\begin{pgfscope}%
\pgfsys@transformshift{2.695629in}{1.943770in}%
\pgfsys@useobject{currentmarker}{}%
\end{pgfscope}%
\begin{pgfscope}%
\pgfsys@transformshift{2.749825in}{1.910365in}%
\pgfsys@useobject{currentmarker}{}%
\end{pgfscope}%
\begin{pgfscope}%
\pgfsys@transformshift{2.804021in}{1.878840in}%
\pgfsys@useobject{currentmarker}{}%
\end{pgfscope}%
\begin{pgfscope}%
\pgfsys@transformshift{2.858217in}{1.850295in}%
\pgfsys@useobject{currentmarker}{}%
\end{pgfscope}%
\begin{pgfscope}%
\pgfsys@transformshift{2.912413in}{1.818576in}%
\pgfsys@useobject{currentmarker}{}%
\end{pgfscope}%
\begin{pgfscope}%
\pgfsys@transformshift{2.966608in}{1.787330in}%
\pgfsys@useobject{currentmarker}{}%
\end{pgfscope}%
\begin{pgfscope}%
\pgfsys@transformshift{3.020804in}{1.758673in}%
\pgfsys@useobject{currentmarker}{}%
\end{pgfscope}%
\begin{pgfscope}%
\pgfsys@transformshift{3.075000in}{1.724759in}%
\pgfsys@useobject{currentmarker}{}%
\end{pgfscope}%
\begin{pgfscope}%
\pgfsys@transformshift{3.129196in}{1.697371in}%
\pgfsys@useobject{currentmarker}{}%
\end{pgfscope}%
\begin{pgfscope}%
\pgfsys@transformshift{3.183392in}{1.672131in}%
\pgfsys@useobject{currentmarker}{}%
\end{pgfscope}%
\begin{pgfscope}%
\pgfsys@transformshift{3.237587in}{1.646449in}%
\pgfsys@useobject{currentmarker}{}%
\end{pgfscope}%
\begin{pgfscope}%
\pgfsys@transformshift{3.291783in}{1.621348in}%
\pgfsys@useobject{currentmarker}{}%
\end{pgfscope}%
\begin{pgfscope}%
\pgfsys@transformshift{3.345979in}{1.601064in}%
\pgfsys@useobject{currentmarker}{}%
\end{pgfscope}%
\begin{pgfscope}%
\pgfsys@transformshift{3.400175in}{1.577671in}%
\pgfsys@useobject{currentmarker}{}%
\end{pgfscope}%
\begin{pgfscope}%
\pgfsys@transformshift{3.454371in}{1.547394in}%
\pgfsys@useobject{currentmarker}{}%
\end{pgfscope}%
\begin{pgfscope}%
\pgfsys@transformshift{3.508566in}{1.519928in}%
\pgfsys@useobject{currentmarker}{}%
\end{pgfscope}%
\begin{pgfscope}%
\pgfsys@transformshift{3.562762in}{1.489500in}%
\pgfsys@useobject{currentmarker}{}%
\end{pgfscope}%
\begin{pgfscope}%
\pgfsys@transformshift{3.616958in}{1.455674in}%
\pgfsys@useobject{currentmarker}{}%
\end{pgfscope}%
\begin{pgfscope}%
\pgfsys@transformshift{3.671154in}{1.424373in}%
\pgfsys@useobject{currentmarker}{}%
\end{pgfscope}%
\begin{pgfscope}%
\pgfsys@transformshift{3.725350in}{1.400216in}%
\pgfsys@useobject{currentmarker}{}%
\end{pgfscope}%
\begin{pgfscope}%
\pgfsys@transformshift{3.779545in}{1.379719in}%
\pgfsys@useobject{currentmarker}{}%
\end{pgfscope}%
\begin{pgfscope}%
\pgfsys@transformshift{3.833741in}{1.350957in}%
\pgfsys@useobject{currentmarker}{}%
\end{pgfscope}%
\begin{pgfscope}%
\pgfsys@transformshift{3.887937in}{1.320959in}%
\pgfsys@useobject{currentmarker}{}%
\end{pgfscope}%
\begin{pgfscope}%
\pgfsys@transformshift{3.942133in}{1.289759in}%
\pgfsys@useobject{currentmarker}{}%
\end{pgfscope}%
\begin{pgfscope}%
\pgfsys@transformshift{3.996329in}{1.272572in}%
\pgfsys@useobject{currentmarker}{}%
\end{pgfscope}%
\begin{pgfscope}%
\pgfsys@transformshift{4.050524in}{1.238322in}%
\pgfsys@useobject{currentmarker}{}%
\end{pgfscope}%
\begin{pgfscope}%
\pgfsys@transformshift{4.104720in}{1.218140in}%
\pgfsys@useobject{currentmarker}{}%
\end{pgfscope}%
\begin{pgfscope}%
\pgfsys@transformshift{4.158916in}{1.195278in}%
\pgfsys@useobject{currentmarker}{}%
\end{pgfscope}%
\begin{pgfscope}%
\pgfsys@transformshift{4.213112in}{1.175231in}%
\pgfsys@useobject{currentmarker}{}%
\end{pgfscope}%
\begin{pgfscope}%
\pgfsys@transformshift{4.267308in}{1.139661in}%
\pgfsys@useobject{currentmarker}{}%
\end{pgfscope}%
\begin{pgfscope}%
\pgfsys@transformshift{4.321503in}{1.107368in}%
\pgfsys@useobject{currentmarker}{}%
\end{pgfscope}%
\begin{pgfscope}%
\pgfsys@transformshift{4.375699in}{1.081579in}%
\pgfsys@useobject{currentmarker}{}%
\end{pgfscope}%
\begin{pgfscope}%
\pgfsys@transformshift{4.429895in}{1.059018in}%
\pgfsys@useobject{currentmarker}{}%
\end{pgfscope}%
\begin{pgfscope}%
\pgfsys@transformshift{4.484091in}{1.032387in}%
\pgfsys@useobject{currentmarker}{}%
\end{pgfscope}%
\begin{pgfscope}%
\pgfsys@transformshift{4.538287in}{1.000592in}%
\pgfsys@useobject{currentmarker}{}%
\end{pgfscope}%
\begin{pgfscope}%
\pgfsys@transformshift{4.592483in}{0.973183in}%
\pgfsys@useobject{currentmarker}{}%
\end{pgfscope}%
\begin{pgfscope}%
\pgfsys@transformshift{4.646678in}{0.944858in}%
\pgfsys@useobject{currentmarker}{}%
\end{pgfscope}%
\begin{pgfscope}%
\pgfsys@transformshift{4.700874in}{0.911726in}%
\pgfsys@useobject{currentmarker}{}%
\end{pgfscope}%
\begin{pgfscope}%
\pgfsys@transformshift{4.755070in}{0.880519in}%
\pgfsys@useobject{currentmarker}{}%
\end{pgfscope}%
\begin{pgfscope}%
\pgfsys@transformshift{4.809266in}{0.854194in}%
\pgfsys@useobject{currentmarker}{}%
\end{pgfscope}%
\begin{pgfscope}%
\pgfsys@transformshift{4.863462in}{0.829775in}%
\pgfsys@useobject{currentmarker}{}%
\end{pgfscope}%
\begin{pgfscope}%
\pgfsys@transformshift{4.917657in}{0.806260in}%
\pgfsys@useobject{currentmarker}{}%
\end{pgfscope}%
\begin{pgfscope}%
\pgfsys@transformshift{4.971853in}{0.777140in}%
\pgfsys@useobject{currentmarker}{}%
\end{pgfscope}%
\begin{pgfscope}%
\pgfsys@transformshift{5.026049in}{0.746793in}%
\pgfsys@useobject{currentmarker}{}%
\end{pgfscope}%
\begin{pgfscope}%
\pgfsys@transformshift{5.080245in}{0.705905in}%
\pgfsys@useobject{currentmarker}{}%
\end{pgfscope}%
\begin{pgfscope}%
\pgfsys@transformshift{5.134441in}{0.656764in}%
\pgfsys@useobject{currentmarker}{}%
\end{pgfscope}%
\begin{pgfscope}%
\pgfsys@transformshift{5.188636in}{0.637273in}%
\pgfsys@useobject{currentmarker}{}%
\end{pgfscope}%
\end{pgfscope}%
\begin{pgfscope}%
\pgfpathrectangle{\pgfqpoint{0.750000in}{0.500000in}}{\pgfqpoint{4.650000in}{3.020000in}}%
\pgfusepath{clip}%
\pgfsetbuttcap%
\pgfsetroundjoin%
\pgfsetlinewidth{1.505625pt}%
\definecolor{currentstroke}{rgb}{0.000000,0.000000,0.000000}%
\pgfsetstrokecolor{currentstroke}%
\pgfsetdash{}{0pt}%
\pgfpathmoveto{\pgfqpoint{2.895604in}{1.945263in}}%
\pgfpathlineto{\pgfqpoint{2.895604in}{2.481568in}}%
\pgfusepath{stroke}%
\end{pgfscope}%
\begin{pgfscope}%
\pgfpathrectangle{\pgfqpoint{0.750000in}{0.500000in}}{\pgfqpoint{4.650000in}{3.020000in}}%
\pgfusepath{clip}%
\pgfsetbuttcap%
\pgfsetroundjoin%
\pgfsetlinewidth{1.505625pt}%
\definecolor{currentstroke}{rgb}{0.000000,0.000000,0.000000}%
\pgfsetstrokecolor{currentstroke}%
\pgfsetdash{}{0pt}%
\pgfpathmoveto{\pgfqpoint{2.350385in}{1.945263in}}%
\pgfpathlineto{\pgfqpoint{2.350385in}{2.481568in}}%
\pgfusepath{stroke}%
\end{pgfscope}%
\begin{pgfscope}%
\pgfsetrectcap%
\pgfsetmiterjoin%
\pgfsetlinewidth{0.803000pt}%
\definecolor{currentstroke}{rgb}{0.000000,0.000000,0.000000}%
\pgfsetstrokecolor{currentstroke}%
\pgfsetdash{}{0pt}%
\pgfpathmoveto{\pgfqpoint{0.750000in}{0.500000in}}%
\pgfpathlineto{\pgfqpoint{0.750000in}{3.520000in}}%
\pgfusepath{stroke}%
\end{pgfscope}%
\begin{pgfscope}%
\pgfsetrectcap%
\pgfsetmiterjoin%
\pgfsetlinewidth{0.803000pt}%
\definecolor{currentstroke}{rgb}{0.000000,0.000000,0.000000}%
\pgfsetstrokecolor{currentstroke}%
\pgfsetdash{}{0pt}%
\pgfpathmoveto{\pgfqpoint{5.400000in}{0.500000in}}%
\pgfpathlineto{\pgfqpoint{5.400000in}{3.520000in}}%
\pgfusepath{stroke}%
\end{pgfscope}%
\begin{pgfscope}%
\pgfsetrectcap%
\pgfsetmiterjoin%
\pgfsetlinewidth{0.803000pt}%
\definecolor{currentstroke}{rgb}{0.000000,0.000000,0.000000}%
\pgfsetstrokecolor{currentstroke}%
\pgfsetdash{}{0pt}%
\pgfpathmoveto{\pgfqpoint{0.750000in}{0.500000in}}%
\pgfpathlineto{\pgfqpoint{5.400000in}{0.500000in}}%
\pgfusepath{stroke}%
\end{pgfscope}%
\begin{pgfscope}%
\pgfsetrectcap%
\pgfsetmiterjoin%
\pgfsetlinewidth{0.803000pt}%
\definecolor{currentstroke}{rgb}{0.000000,0.000000,0.000000}%
\pgfsetstrokecolor{currentstroke}%
\pgfsetdash{}{0pt}%
\pgfpathmoveto{\pgfqpoint{0.750000in}{3.520000in}}%
\pgfpathlineto{\pgfqpoint{5.400000in}{3.520000in}}%
\pgfusepath{stroke}%
\end{pgfscope}%
\begin{pgfscope}%
\definecolor{textcolor}{rgb}{0.000000,0.000000,0.000000}%
\pgfsetstrokecolor{textcolor}%
\pgfsetfillcolor{textcolor}%
\pgftext[x=2.020523in, y=1.819857in, left, base]{\color{textcolor}\sffamily\fontsize{10.000000}{12.000000}\selectfont breakup of }%
\end{pgfscope}%
\begin{pgfscope}%
\definecolor{textcolor}{rgb}{0.000000,0.000000,0.000000}%
\pgfsetstrokecolor{textcolor}%
\pgfsetfillcolor{textcolor}%
\pgftext[x=1.905747in, y=1.677111in, left, base]{\color{textcolor}\sffamily\fontsize{10.000000}{12.000000}\selectfont  \(\displaystyle 1.7 \, \mathrm{mrad}\) driver}%
\end{pgfscope}%
\begin{pgfscope}%
\definecolor{textcolor}{rgb}{0.000000,0.000000,0.000000}%
\pgfsetstrokecolor{textcolor}%
\pgfsetfillcolor{textcolor}%
\pgftext[x=2.565743in, y=2.713699in, left, base]{\color{textcolor}\sffamily\fontsize{10.000000}{12.000000}\selectfont breakup of }%
\end{pgfscope}%
\begin{pgfscope}%
\definecolor{textcolor}{rgb}{0.000000,0.000000,0.000000}%
\pgfsetstrokecolor{textcolor}%
\pgfsetfillcolor{textcolor}%
\pgftext[x=2.450966in, y=2.570952in, left, base]{\color{textcolor}\sffamily\fontsize{10.000000}{12.000000}\selectfont  \(\displaystyle 4.2 \, \mathrm{mrad}\) driver}%
\end{pgfscope}%
\begin{pgfscope}%
\pgfsetbuttcap%
\pgfsetmiterjoin%
\definecolor{currentfill}{rgb}{1.000000,1.000000,1.000000}%
\pgfsetfillcolor{currentfill}%
\pgfsetfillopacity{0.800000}%
\pgfsetlinewidth{1.003750pt}%
\definecolor{currentstroke}{rgb}{0.800000,0.800000,0.800000}%
\pgfsetstrokecolor{currentstroke}%
\pgfsetstrokeopacity{0.800000}%
\pgfsetdash{}{0pt}%
\pgfpathmoveto{\pgfqpoint{4.340971in}{2.827871in}}%
\pgfpathlineto{\pgfqpoint{5.302778in}{2.827871in}}%
\pgfpathquadraticcurveto{\pgfqpoint{5.330556in}{2.827871in}}{\pgfqpoint{5.330556in}{2.855648in}}%
\pgfpathlineto{\pgfqpoint{5.330556in}{3.422778in}}%
\pgfpathquadraticcurveto{\pgfqpoint{5.330556in}{3.450556in}}{\pgfqpoint{5.302778in}{3.450556in}}%
\pgfpathlineto{\pgfqpoint{4.340971in}{3.450556in}}%
\pgfpathquadraticcurveto{\pgfqpoint{4.313194in}{3.450556in}}{\pgfqpoint{4.313194in}{3.422778in}}%
\pgfpathlineto{\pgfqpoint{4.313194in}{2.855648in}}%
\pgfpathquadraticcurveto{\pgfqpoint{4.313194in}{2.827871in}}{\pgfqpoint{4.340971in}{2.827871in}}%
\pgfpathlineto{\pgfqpoint{4.340971in}{2.827871in}}%
\pgfpathclose%
\pgfusepath{stroke,fill}%
\end{pgfscope}%
\begin{pgfscope}%
\pgfsetrectcap%
\pgfsetroundjoin%
\pgfsetlinewidth{1.505625pt}%
\definecolor{currentstroke}{rgb}{0.121569,0.466667,0.705882}%
\pgfsetstrokecolor{currentstroke}%
\pgfsetdash{}{0pt}%
\pgfpathmoveto{\pgfqpoint{4.368749in}{3.346389in}}%
\pgfpathlineto{\pgfqpoint{4.507638in}{3.346389in}}%
\pgfpathlineto{\pgfqpoint{4.646527in}{3.346389in}}%
\pgfusepath{stroke}%
\end{pgfscope}%
\begin{pgfscope}%
\pgfsetbuttcap%
\pgfsetroundjoin%
\definecolor{currentfill}{rgb}{0.121569,0.466667,0.705882}%
\pgfsetfillcolor{currentfill}%
\pgfsetlinewidth{1.003750pt}%
\definecolor{currentstroke}{rgb}{0.121569,0.466667,0.705882}%
\pgfsetstrokecolor{currentstroke}%
\pgfsetdash{}{0pt}%
\pgfsys@defobject{currentmarker}{\pgfqpoint{-0.020833in}{-0.020833in}}{\pgfqpoint{0.020833in}{0.020833in}}{%
\pgfpathmoveto{\pgfqpoint{0.000000in}{-0.020833in}}%
\pgfpathcurveto{\pgfqpoint{0.005525in}{-0.020833in}}{\pgfqpoint{0.010825in}{-0.018638in}}{\pgfqpoint{0.014731in}{-0.014731in}}%
\pgfpathcurveto{\pgfqpoint{0.018638in}{-0.010825in}}{\pgfqpoint{0.020833in}{-0.005525in}}{\pgfqpoint{0.020833in}{0.000000in}}%
\pgfpathcurveto{\pgfqpoint{0.020833in}{0.005525in}}{\pgfqpoint{0.018638in}{0.010825in}}{\pgfqpoint{0.014731in}{0.014731in}}%
\pgfpathcurveto{\pgfqpoint{0.010825in}{0.018638in}}{\pgfqpoint{0.005525in}{0.020833in}}{\pgfqpoint{0.000000in}{0.020833in}}%
\pgfpathcurveto{\pgfqpoint{-0.005525in}{0.020833in}}{\pgfqpoint{-0.010825in}{0.018638in}}{\pgfqpoint{-0.014731in}{0.014731in}}%
\pgfpathcurveto{\pgfqpoint{-0.018638in}{0.010825in}}{\pgfqpoint{-0.020833in}{0.005525in}}{\pgfqpoint{-0.020833in}{0.000000in}}%
\pgfpathcurveto{\pgfqpoint{-0.020833in}{-0.005525in}}{\pgfqpoint{-0.018638in}{-0.010825in}}{\pgfqpoint{-0.014731in}{-0.014731in}}%
\pgfpathcurveto{\pgfqpoint{-0.010825in}{-0.018638in}}{\pgfqpoint{-0.005525in}{-0.020833in}}{\pgfqpoint{0.000000in}{-0.020833in}}%
\pgfpathlineto{\pgfqpoint{0.000000in}{-0.020833in}}%
\pgfpathclose%
\pgfusepath{stroke,fill}%
}%
\begin{pgfscope}%
\pgfsys@transformshift{4.507638in}{3.346389in}%
\pgfsys@useobject{currentmarker}{}%
\end{pgfscope}%
\end{pgfscope}%
\begin{pgfscope}%
\definecolor{textcolor}{rgb}{0.000000,0.000000,0.000000}%
\pgfsetstrokecolor{textcolor}%
\pgfsetfillcolor{textcolor}%
\pgftext[x=4.757638in,y=3.297778in,left,base]{\color{textcolor}\sffamily\fontsize{10.000000}{12.000000}\selectfont \(\displaystyle 1.7 \, \mathrm{mrad}\)}%
\end{pgfscope}%
\begin{pgfscope}%
\pgfsetrectcap%
\pgfsetroundjoin%
\pgfsetlinewidth{1.505625pt}%
\definecolor{currentstroke}{rgb}{1.000000,0.498039,0.054902}%
\pgfsetstrokecolor{currentstroke}%
\pgfsetdash{}{0pt}%
\pgfpathmoveto{\pgfqpoint{4.368749in}{3.152716in}}%
\pgfpathlineto{\pgfqpoint{4.507638in}{3.152716in}}%
\pgfpathlineto{\pgfqpoint{4.646527in}{3.152716in}}%
\pgfusepath{stroke}%
\end{pgfscope}%
\begin{pgfscope}%
\pgfsetbuttcap%
\pgfsetroundjoin%
\definecolor{currentfill}{rgb}{1.000000,0.498039,0.054902}%
\pgfsetfillcolor{currentfill}%
\pgfsetlinewidth{1.003750pt}%
\definecolor{currentstroke}{rgb}{1.000000,0.498039,0.054902}%
\pgfsetstrokecolor{currentstroke}%
\pgfsetdash{}{0pt}%
\pgfsys@defobject{currentmarker}{\pgfqpoint{-0.020833in}{-0.020833in}}{\pgfqpoint{0.020833in}{0.020833in}}{%
\pgfpathmoveto{\pgfqpoint{0.000000in}{-0.020833in}}%
\pgfpathcurveto{\pgfqpoint{0.005525in}{-0.020833in}}{\pgfqpoint{0.010825in}{-0.018638in}}{\pgfqpoint{0.014731in}{-0.014731in}}%
\pgfpathcurveto{\pgfqpoint{0.018638in}{-0.010825in}}{\pgfqpoint{0.020833in}{-0.005525in}}{\pgfqpoint{0.020833in}{0.000000in}}%
\pgfpathcurveto{\pgfqpoint{0.020833in}{0.005525in}}{\pgfqpoint{0.018638in}{0.010825in}}{\pgfqpoint{0.014731in}{0.014731in}}%
\pgfpathcurveto{\pgfqpoint{0.010825in}{0.018638in}}{\pgfqpoint{0.005525in}{0.020833in}}{\pgfqpoint{0.000000in}{0.020833in}}%
\pgfpathcurveto{\pgfqpoint{-0.005525in}{0.020833in}}{\pgfqpoint{-0.010825in}{0.018638in}}{\pgfqpoint{-0.014731in}{0.014731in}}%
\pgfpathcurveto{\pgfqpoint{-0.018638in}{0.010825in}}{\pgfqpoint{-0.020833in}{0.005525in}}{\pgfqpoint{-0.020833in}{0.000000in}}%
\pgfpathcurveto{\pgfqpoint{-0.020833in}{-0.005525in}}{\pgfqpoint{-0.018638in}{-0.010825in}}{\pgfqpoint{-0.014731in}{-0.014731in}}%
\pgfpathcurveto{\pgfqpoint{-0.010825in}{-0.018638in}}{\pgfqpoint{-0.005525in}{-0.020833in}}{\pgfqpoint{0.000000in}{-0.020833in}}%
\pgfpathlineto{\pgfqpoint{0.000000in}{-0.020833in}}%
\pgfpathclose%
\pgfusepath{stroke,fill}%
}%
\begin{pgfscope}%
\pgfsys@transformshift{4.507638in}{3.152716in}%
\pgfsys@useobject{currentmarker}{}%
\end{pgfscope}%
\end{pgfscope}%
\begin{pgfscope}%
\definecolor{textcolor}{rgb}{0.000000,0.000000,0.000000}%
\pgfsetstrokecolor{textcolor}%
\pgfsetfillcolor{textcolor}%
\pgftext[x=4.757638in,y=3.104105in,left,base]{\color{textcolor}\sffamily\fontsize{10.000000}{12.000000}\selectfont \(\displaystyle 4.2 \, \mathrm{mrad}\)}%
\end{pgfscope}%
\begin{pgfscope}%
\pgfsetrectcap%
\pgfsetroundjoin%
\pgfsetlinewidth{1.505625pt}%
\definecolor{currentstroke}{rgb}{0.172549,0.627451,0.172549}%
\pgfsetstrokecolor{currentstroke}%
\pgfsetdash{}{0pt}%
\pgfpathmoveto{\pgfqpoint{4.368749in}{2.959043in}}%
\pgfpathlineto{\pgfqpoint{4.507638in}{2.959043in}}%
\pgfpathlineto{\pgfqpoint{4.646527in}{2.959043in}}%
\pgfusepath{stroke}%
\end{pgfscope}%
\begin{pgfscope}%
\pgfsetbuttcap%
\pgfsetroundjoin%
\definecolor{currentfill}{rgb}{0.172549,0.627451,0.172549}%
\pgfsetfillcolor{currentfill}%
\pgfsetlinewidth{1.003750pt}%
\definecolor{currentstroke}{rgb}{0.172549,0.627451,0.172549}%
\pgfsetstrokecolor{currentstroke}%
\pgfsetdash{}{0pt}%
\pgfsys@defobject{currentmarker}{\pgfqpoint{-0.020833in}{-0.020833in}}{\pgfqpoint{0.020833in}{0.020833in}}{%
\pgfpathmoveto{\pgfqpoint{0.000000in}{-0.020833in}}%
\pgfpathcurveto{\pgfqpoint{0.005525in}{-0.020833in}}{\pgfqpoint{0.010825in}{-0.018638in}}{\pgfqpoint{0.014731in}{-0.014731in}}%
\pgfpathcurveto{\pgfqpoint{0.018638in}{-0.010825in}}{\pgfqpoint{0.020833in}{-0.005525in}}{\pgfqpoint{0.020833in}{0.000000in}}%
\pgfpathcurveto{\pgfqpoint{0.020833in}{0.005525in}}{\pgfqpoint{0.018638in}{0.010825in}}{\pgfqpoint{0.014731in}{0.014731in}}%
\pgfpathcurveto{\pgfqpoint{0.010825in}{0.018638in}}{\pgfqpoint{0.005525in}{0.020833in}}{\pgfqpoint{0.000000in}{0.020833in}}%
\pgfpathcurveto{\pgfqpoint{-0.005525in}{0.020833in}}{\pgfqpoint{-0.010825in}{0.018638in}}{\pgfqpoint{-0.014731in}{0.014731in}}%
\pgfpathcurveto{\pgfqpoint{-0.018638in}{0.010825in}}{\pgfqpoint{-0.020833in}{0.005525in}}{\pgfqpoint{-0.020833in}{0.000000in}}%
\pgfpathcurveto{\pgfqpoint{-0.020833in}{-0.005525in}}{\pgfqpoint{-0.018638in}{-0.010825in}}{\pgfqpoint{-0.014731in}{-0.014731in}}%
\pgfpathcurveto{\pgfqpoint{-0.010825in}{-0.018638in}}{\pgfqpoint{-0.005525in}{-0.020833in}}{\pgfqpoint{0.000000in}{-0.020833in}}%
\pgfpathlineto{\pgfqpoint{0.000000in}{-0.020833in}}%
\pgfpathclose%
\pgfusepath{stroke,fill}%
}%
\begin{pgfscope}%
\pgfsys@transformshift{4.507638in}{2.959043in}%
\pgfsys@useobject{currentmarker}{}%
\end{pgfscope}%
\end{pgfscope}%
\begin{pgfscope}%
\definecolor{textcolor}{rgb}{0.000000,0.000000,0.000000}%
\pgfsetstrokecolor{textcolor}%
\pgfsetfillcolor{textcolor}%
\pgftext[x=4.757638in,y=2.910432in,left,base]{\color{textcolor}\sffamily\fontsize{10.000000}{12.000000}\selectfont \(\displaystyle 8.7 \, \mathrm{mrad}\)}%
\end{pgfscope}%
\end{pgfpicture}%
\makeatother%
\endgroup%

	\caption{Loss of peak energy for different initial divergences are plotted over $y$.}
	\label{fig:E_peak_divergence}
\end{figure}
The low divergence driver has the same drop-off as the one with normal divergence but afterwards no further energy loss but instead a small increase of roughly \qty{0.5}{\MeV} can be observed.
For the high divergence driver, the initial drop-off is smaller, with only \qty{4}{\MeV}, but afterwards the peak energy decreases linear. After \qty{3}{\mm} the driver already has lost \qty{9}{\MeV} of energy, which is still smaller than the energy uncertainty in experiments \cite{Schoebel2022} but could become relevant in the future.

The remaining fields after bunch breakup of the low divergence driver may be too weak to decelerate the bunch, thus the fields cannot further decrease the peak energy. This would result in the curve depicted above. 
Meanwhile the linear regime of the high divergence driver is capable of further decelerating the front part of the driver, resulting in a high peak energy loss, as the bunch never breaks during the simulation time. 

\end{document}