\documentclass[bachelor_thesis]{subfiles}

\begin{document}
\chapter{Analysis of the bunch characteristics}
\section{Movement of the driver electrons}
In this section, the movement of the particles in our driver is discussed and the changes in position respective to each other are shown. As such tracking of individual particles is not possible in experiment
this will give further insight into the effects of the \gls{pwfa} on the drive beam.

In \autoref{fig:q_series} a time series of 2D histograms, showing the charge density of the driver after entering plasma, can be seen for the $\zeta$-$z$-plane. The data is taken from a simulation where the bunch has an initial mean kinetic energy of \qty{250}{\MeV} and a divergence of \qty{4.2}{\mrad}. 
\begin{figure}
	\centering
	\missingfigure{charge density plot}
	\caption{Time series of a charge density histogram of the driver electrons in Log scale. Note the formation of small wings and later the diverging to the borders. The viewing windows moves with light speed with the driver.}
	\label{fig:q_series}
\end{figure}
At the start, the spacial distribution follows a 2D Gauss, as set in \autoref{chap:init}. A log scale is chosen to make the borders with low density visible. 
The Lorentz force is also layed over the histogram to visualize the cause of the transformation of the driver. It is retrieved from the experienced $\vec{E}$- and $\vec{B}$-field, that every macroparticle stores.
The window is then separated into bins and for every bin the mean of the force has been calculated and plotted as a force field, with the color and width of the lines quantifying the absolute force. 
This corresponds to a case where there only is one macroparticle per bin and the force acting on this particle is plotted.

After some distance was traveled in the plasma, the first cavities start to arise, as seen in \autoref{fig:cavity}. The focusing Lorentz force of these cavities forms a tail at the end of the driver while the cavities are still in the linear regime.
Only small forces act on the front of the beam, so the front part can diverge freely to the borders. In the center and back there are great acting forces, pushing the particles back and simultaneously centering them, resulting in the creation of the tail.
These forces result from the formation of the first cavities. As can be seen in \autoref{fig:cavity}, the first cavity forms directly behind the front part of the driver so the backside already experiences the decelerating and focusing fields of the cavity.
\begin{figure}
	\centering
	\missingfigure{}
	\caption{}
	\label{fig:cavity}
\end{figure}
Comparing the length of the beam\todo{show length with lines}over time shows that the tail is not a part of the driver falling back but instead the backside experiencing focusing forces, which narrow the backside down.
When looking back at the cavity formation, it shows that with the tail forming, cavities with similar width emerge and start to form the blowout regime.

After narrowing down to a tail, the backside of the beam spreads again, as the focused driver electrons overshoot, forming wing-like structures. With the tail widening, the cavities also start to widen and form the strongest electric fields of the whole \gls{pwfa} stage.
The formed wings diverge and new wings form behind it by particles which were pulled back by the focusing force and overshot again. This forms a chain of smaller wings, all spreading with time and broadening the tail further.
In this stage, the electric fields in the cavities already start to lose in strength while the electron density in the wakefield starts to rise again.

As the backside of the driver further spreads, it leaves the impact of the focusing fields, letting it grow in tranversal direction together with the front of the bunch. From now on it just grows further, while the $\vec{E}$-fields of the wakfields lose in strength and electrons flood the cavities again, leading back into the linear regime.
Simultaneously there are still strong forces, which act on the middle of the bunch, causing it to lose energy. The discussion of this energy loss will be continued in \autoref{chap:loc_E}, for now we only look at the position of these particles.
In \autoref{fig:force_time} the strength and and position change of the longitudinal Lorentz-force can be seen over time. It shows how the force pushing the driver back first builds up and then loses it's strength with the time.
\begin{figure}
	\centering
	\missingfigure{}
	\caption{Histogram of the longitudinal part of the Lorentz force. The force is sampled over the $y$-direction at a slice in the middle in $z$-direction for every 2000 timesteps. This slice is here plotted over the time, showing how it changes in position and strength.}
	\label{fig:force_time}
\end{figure}
There is also a forward pushing force on the back, experienced by the particles that fall back to the middle of the cavity where the electric field changes direction. Notable is the part around $t=85000$, where the bunch collapses and particles start to fall back rapidly,
so they get accelerated again in the back of the first cavity. These particles stem mostly from the middle of the driver, where the strongest backwmaards-pushing forces acted, causing them to rapidly lose energy.
\todo{does collapse of wakefield predate bunch breakup?}
The wakefield is completely in the linear regime while electrons further flow back into the cavities.

The depletion of the electric field in longitudinal direction is visualized in \autoref{fig:E_y_hist}, where the field, stored on the $x=z=0$ axis is plotted over time, showing how it changes with loss of the driver quality.
\begin{figure}
	\centering
	\missingfigure{}
	\caption{}
	\label{fig:E_y_hist}
\end{figure}
It shows the elongation of the cavities with growing fields, experimentally shown in \cite{Schoebel2022}. When we assume a phase locked witness bunch, meaning it doesn't change position relative to the cavities, the theoretical energy gain can be calculated 
by integration of the Lorentz force over the traversed path. The result can be seen in \autoref{fig:E_gain}, where the gained energy was plotted over position $y-ct$.
\begin{figure}
	\centering
	\missingfigure{Energy gain as integrated Lorentz Force plotted over $y-ct$, showing how much energy an electron can get in this place.}
	\caption{}
	\label{fig:E_gain}
\end{figure}
When the witness bunch is placed precisely, energies of around \qty{400}{\MeV} can be gained by an electron, exceeding the initial energy of the driver.

With openPMDs capabilities to store IDs for every particle, further investigation of the paths can be made. This was used in \autoref{fig:y_hist_0} to show, which $y$-position the particles have at the end of the simulation.
\begin{figure}
	\centering
	\begin{subfigure}{0.5\textwidth}
	\centering
	\missingfigure{}
	\caption{} \label{fig:y_hist_0}
	\end{subfigure}
	\hfill
	\begin{subfigure}{0.5\textwidth}
	\centering
	\missingfigure{}
	\caption{} \label{fig:y_hist_time}
	\end{subfigure}
	\caption{(a) 2D histogram of the driver. Each bin shows the minimum $y$-position at timestep \num{120000} of the particles in it, allowing us to see where each particle will eventually fall in the bunch.
	(b) Histogram taking the z=0 slice from (a) for every 2000 timesteps and plotting them over the time.}
	\label{fig:y_hist}
\end{figure} \todo[inline]{add both y-position plots}
It reveals that the electrons, which fall behind in the bunch breakup indeed originate from a small pool right behind the center of the bunch, where the strongest Lorentz-force is experienced. This results in them being overtaken by the particles behind them.
The other parts of the bunch don't seem to move much in the $y$-direction, resulting in the clear color gradient on display. To further show this evolution, in \autoref{fig:y_hist_time} the z=0 slice is taken from every timestep and plotted over the time,
showing how the particles fall behind until they get stopped by the accelerating force from the cavity.

The same tracking was done for the $z$-direction, seen in the time series in \autoref{fig:z_hist}, where the $z$-position was marked in the first timestep and than displayed as the mean of the particles in individual bins.
\begin{figure}
	\centering
	\missingfigure{}
	\caption{}
	\label{fig:z_hist}
\end{figure}
As the movement in z-direction is symmetric, this results in bins with equal amount of particles from the top and bottom half to be displayed as middle white. Still the effect is clear, as the wings can clearly be seen.
The alternating colors can be explained by the overshooting of the electrons, which are pulled back to the center and form a new wing. The front part of the driver is mostly unaffected by this.

\subsection{Parameter comparison}
The transformation of the driver beam form and subsequent changes of the potential energy gain of a witness beam were compared for different initial parameters.
\paragraph*{Divergence comparison}\hspace{0pt} \\
First the driver qualities for three different divergences after passing the metal foil were compared. The change of the charge density over time
for a high divergence ($\theta=\qty{8.7}{\mrad}$), middle divergence($\theta=\qty{8.7}{\mrad}$, as seen in experiments) and low divergence ($\theta=\qty{1.7}{\mrad}$) beam are shown in \autoref{fig:dens_compare}.
\begin{figure}
	\centering
	\begin{subfigure}{0.5\textwidth}
	\centering
	\missingfigure{}
	\caption{} \label{fig:dens_low}
	\end{subfigure}
	\hfill
	\begin{subfigure}{0.5\textwidth}
	\centering
	\missingfigure{}
	\caption{} \label{fig:dens_high}
	\end{subfigure}
	\caption{(a) }
	\label{fig:dens_compare}
\end{figure}
For the normal divergence the start of the blowout regime as well as the fallback into the linear regime can be seen, as described before.
In comparison for a low divergence bunch, the cavites are in a strong blowout regime with small electron charge densities, resulting in high strength electric fields. Even before the cavities fill and a linear regime sets in, the driver breaks up
as the strong forces drain it's energy fast. After the breakup, the linear regime starts but a small cavity in the blowout regime through beam loading\todo{citation needed} still remains behind the fallen back part of the driver.
For a driver with high divergence, no blowout is achieved so the wakefield remains in the linear regime with weak electric fields. This weak fields are not able to cause a bunch breakup, resulting in the bunch just diverging to the border with time.

In \todo{add comparison of energy gain and energy gain maximas} the energy gain curves over $\zeta$ for the different divergences are compared, as well as as comparison of the peaks in \todo{add this}.
The normal peak for energy gain lies for all curves at the position, where the backside of the cavity during the linear regime without elongation is positioned. Additionally, there is a second much higher peak for the low divergence curve
in the backside of the cavity with blowout length. This results from the long standing blowout with extreme fields and the following smaller blowout from the beam loading in the same position. Else witness beams in this position would experience the decelerating
field of the second cavity after some time. This gives good reasons to strive for low divergent beams, as they bring the highest possible energy gains.

The question arises, why the change in divergence causes such a extreme difference for drivers. Possible is, that the induced current of the low divergence driver is higher, as the velocity in propagation direction is higher and the spatial distribution
may be denser, as the particles don't diverge as much in the time before entering the plasma. In fact, \todo{add energy and force plot} shows that the uniform driver survives a millimeter longer in the plasma before breakup.
Also no particles which gain energy in the back can be seen, as for the uniform driver there are no particles so far back, that they get accelerated. This results from the driver having a high density in the $x$-$z$-plane center over the whole distribution and not just
in the middle in $\zeta$-direction. Still the later breakup gains only a small plus in maximal energy gain.

\paragraph*{Energy comparison}\hspace{0pt} \\
A second comparison between different initial kinetic energies of the drivers with same divergence can be made. In\todo{add figure} the different changes in electron density over time are depicted. Only small differences can be seen,
with the high energy drivers achieving a blowout over a longer distance. This results in slightly increased energy gains, as seen in \todo{add figure}. While the jump from \qtyrange{250}{300}{\MeV} results in \qty{60}{\MeV} higher gains,
there is a diminishing return, as only \qty{20}{\MeV} more are reached when increasing from \qtyrange{300}{350}{\MeV}. Real experiments are restricted by the length of the plasma jet as well, resulting in even smaller gain increase.
The curve makes also apparent, that there are only small differences in the energy loss behind the driver (first minimum behind \qty{0}{\um}). This results in drivers with higher energy living longer before bunch breakup as they have more energy to lose.
The change in energy is compared in \todo{add plot}, showing also showing, that no breakup for the \qty{250}{\MeV} driver occurs.

\paragraph*{Distribution comparison}\hspace{0pt} \\
Even though it is hard to control the from of a beam leaving the \gls{lwfa} for our \gls{pwfa}, a comparison between different distributions can give new insights into the properties that are needed from a driver
to form high accelerating fields. Besides the driver with Gaussian distribution in all spatial direction, a driver with Gaussian distribution in transversal direction and uniform charge distribution in propagation direction 
was therefor simulated.

The energy gain graph \todo{add graph} shows only a small win of \qty{14}{\MeV} in maximal energy can be achieved compared to our Gaussian driver. The biggest difference in the two curves lies
in the higher minimum positioned at the backside of the driver, that should result in smaller energy loss and therefore a longer traversed distance before breakup.

\section{Peak Energy shift}
In this the chapter the change in energy of the \gls{pwfa} driver is discussed, as this gives further insights into the its stability and therefore also the stability of the wakefield.
The change of the energy over time can be seen in \autoref{fig:E_hist_time}. It shows how the energy-histogram changes over time. As set up, the energy starts as a Gaussian distribution at around \qty{250}{\MeV}.
\begin{figure}
	\centering
	\begin{subfigure}{0.5\textwidth}
	\centering
	\missingfigure{}
	\caption{} \label{fig:E_hist_time}
	\end{subfigure}
	\hfill
	\begin{subfigure}{0.5\textwidth}
	\centering
	\missingfigure{}
	\caption{} \label{fig:E_peak}
	\end{subfigure}
	\caption{(a) Energy histogram over time. Every 2000 timesteps a histogram of the charge over the energy is made and plotted here over the time.
	(b) Peak energy plotted over time.}
	\label{fig:Energy}
\end{figure}
When inducing the wakefield, the Lorentz force, created by the fields of the first cavity, act on the driver causing it to lose energy. As it acts only on parts of the driver, the distribution is not moving to lower energies but instead growing in size.
The plot shows the growth in low energy electrons until $E=0$ is reached. In orange, the mean is also plotted in,  as well as the electrons with the minimal energy, both sinking with time. Also shown is the maximum energy, which actually grows.
These are particles, which gain energy, as they are so far back in the bunch, that they get pushed by the Lorentz forces in the middle of the first cavity.

Notable is the fact, that the histogram for every timestep isn't uniform but has visible maximas and minimas. The maximum with the highest energy is called the peak energy, which is also plotted in red.
This energy is notable, as the fact, that it stays constant during the \gls{pwfa} is used in experiments to calculate the initial charge the driver had, before entering the \gls{pwfa} (see \cite{Schoebel2022}.
In \autoref{fig:E_peak} only the peak energy is plotted over time, showing that it isn't constant. Directly after entering the plasma, the peak energy drops from \qtyrange{250}{244}{\MeV}.
Here, it plateaus until bunch breakup, after which it further drops.

The systematic uncertainty of the peak energy is given by the size of the bins while the statistical uncertainty results from the uncertainty of my fit. \todo{add fit image somewhere} The fit assumes two summed Gaussian distributions as as
simplification of the real distribution, which consists of multiple peaks with different heights and widths as well as a unknown background noise. Only the peak energy and the second peak are fitted.

Currently the uncertainty on the peak energy measurement in experiments is higher than the \qty{5}{\MeV} jump. Still this indicates, caution is needed when assuming a constant peak energy, especially after bunch breakup.

\subsection{Parameter comparison}
This results are again compared to different initial conditions of the driver. In \todo{add peak energy over time plot but only for energy loss} the loss of peak energy over traveled distance is compared again for three different initial energies.
For all energies the initial energy drop of \qtyrange{6}{7}{\MeV} can be seen, as well as the slower loss of peak energy afterwards. The initial drop off ends for all energies after a plasma distance of \qty{1}{\mm}, with the high energy drivers having higher energy losses. 
While the \qty{250}{\MeV} driver has a visible plateau, there are smaller or no plateaus for the other two energies. Still the energy loss happens slower at the point after the initial drop off and only increases with progressing distance in the plasma.
The slow energy loss afterwards exists for all drivers. While the distance between the \qty{250}{\MeV} and \qty{300}{\MeV} curve stays roughly the same except for some fluctuations, 
the distance between the \qty{300}{\MeV} and \qty{350}{\MeV} curve increases with time, meaning that the \qty{350}{\MeV} peak loses its energy faster. Drawn in are also the point of bunch breakup, which seems to be uncorrelated to the progression of the curve.

Even when there is no plateau for higher initial energies, the energy loss beyond the initial drop off is comparably small before the end of the gas jet would be achieved in experiments at around \qty{3}{\mm} \cite{Schoebel2022}.


When the same analysis is done on the different initial divergences, as seen in \todo{add figure}, big qualitative differences become apparent. The low divergence driver has the same drop off as the one with normal divergence but afterwards
there is no further energy loss but instead a small increase of roughly \qty{0.5}{\MeV}. For the high divergence driver the initial drop off is smaller with only \qty{4}{\MeV} but afterwards the peak energy decreases linear. After \qty{3}{\mm}
it already has lost \qty{9}{\MeV} of energy. \todo{higher than experimental uncertainty?}

\section{Locality of the energy}\label{chap:loc_E}
The energy loss and spacial locality of the energy peaks can be visualized by again binning the macroparticles in space and analyze the mean energy in every bin. This mean energy is then plotted over the positions of the bins,
an example of this can be seen in \autoref{fig:E_map}. 
\begin{figure}
	\centering
	\missingfigure{}
	\caption{Mean energy of the particles in the bins. Every bin has the size .... Every bin with at least one particle is plotted, so to derive information about the energy density, the charge density from \autoref{fig:q_series} has to be considered to.}
	\label{fig:E_map}
\end{figure}

Parts of the bunch where a strong decelerating force acts have lost most of the energy while the front part, where only weak forces act, retained their peak energy. The energy loss in the back is also not a steady gradient but instead bigger areas where the energy is more or less constant, 
for example the yellow area marking the second energy peak.

The complete time series of the energy is plotted in \autoref{fig:E_time}.
\begin{figure}
	\centering
	\missingfigure{}
	\caption{Mean energy of the particles in the bins over time. Only the bins around $z=0$ are plotted for every timestep, showing the gradually loss of energy in the middle of the driver until bunch breakup.}
	\label{fig:E_time}
\end{figure}
Notable is the constant energy in the front half of the driver and the strong energy loss in the middle. When energies around zero are reached, the electrons fall back, where they get accelerated again from the first cavity.
Also notable is, that the highest energies are achieved at the back of the driver, where the accelerating part of the cavity starts and pushes the electrons there.

\end{document}