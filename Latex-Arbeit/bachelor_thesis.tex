% Last given language will be selected
% twoside used for book layout
% BCOR is space for bookbinding and should be chosen with the thickness of the book in mind
\documentclass[ngerman,english,BCOR=10mm,cdgeometry=no,DIV=15,twoside,open=right]{tudscrreprt}
% European Symbols
\usepackage[T1]{fontenc}
\usepackage[ngerman=ngerman-x-latest]{hyphsubst}
\usepackage{selinput}
\SelectInputMappings{adieresis={ä},germandbls={ß}}

\setkomafont{caption}{\footnotesize}


\usepackage{babel}
\usepackage[babel]{microtype}

\hyphenation{PIConGPU}

\usepackage[english]{isodate}
\origdate            % sets date format to dd month yyyy

\usepackage{blindtext}
% Hyperlinks to the refs given in table of contents or in the text
\usepackage{hyperref}

\TUDoptions{cd=lightcolor,cdhead=heavy,ddcfoot=true, cdfont=true}

\renewcaptionname{ngerman}{\abstractname}{Kurzfassung}


% my packages:
\usepackage{graphicx,caption,amsmath,amsfonts,amssymb}
\graphicspath{{../images/}}
\usepackage[table,xcdraw]{xcolor}
\usepackage{todonotes}                    
\usepackage{csquotes}  % is recommended from some package that uses polyglossy

\usepackage[backend=biber, style=numeric-comp, maxbibnames=12,
            giveninits=true, terseinits=true, alldates=year, sorting=none,
            hyperref=true, backref=true, defernumbers=true, language=english,
            sortcites=true, url=false, doi=true]{biblatex}

% macro to set bibers maxbibnames locally 
\makeatletter
\newcommand\Setmaxbibnames[1]{\renewcommand\blx@maxbibnames{#1}}
\makeatletter

% for writing each chapter in own compilable tex-file
\usepackage{subfiles}
% for cross-referencing between different files
\usepackage{xr}

\addbibresource[glob]{../Quellen/Sources.bib}

\AtEveryBibitem{\clearfield{month}}
\AtEveryBibitem{\clearfield{day}}


\usepackage{mathtools}
\usepackage{nicefrac}
\usepackage{siunitx}    % provides correct typesetting of units
                    % \si{kg.m/s^2} print units
                    % \SI{<number>}{<unit>} prints numbers with units
                    % there are different modes regarding type of unit
                    % printing, e.g. slash or real fracs for fracs


\usepackage{subcaption} % provides subfigure environment

% Load the package
\usepackage[nopostdot, style=super, nonumberlist, toc]{glossaries}
\setacronymstyle{long-short} 	% first use in long form, abbreviation else
\loadglsentries{defns}

\addto\captionsenglish{% Replace "english" with the language you use
  \renewcommand{\contentsname}%
    {Table of Contents}%
}

\renewcaptionname{english}{\refereename}{Referees}

\begin{document}


\pagenumbering{Roman} 

\faculty{Bereich Mathematik und Naturwissenschaften}
\department{$\,$ \hspace{27mm} Fakultät Physik}
\institute{Institut für Strahlenphysik}

\headlogo[height=70pt]{HZDRlogo}

\date{99.99.2023}
\author{Nico Wrobel}

\title{%
  Studying the impact of electron bunch characteristics on plasma wakefield performance using particle-in-cell simulations 
}

\thesis{bachelor}
\graduation[B. Sc.]{Bachelor of Science}

\dateofbirth{10.03.2001}
\placeofbirth{Hoyerswerda}
\supervisor{Dr. Richard Pausch}
\referee{Prof. Dr. Ulrich Schramm \newline Prof. Dr. Thomas E. Cowan}


\maketitle


\TUDoption{abstract}{section,multiple}

\begin{abstract}[english]

  text


\newpage
\nextabstract[ngerman]


Text



\end{abstract}








\selectlanguage{english}


\microtypesetup{protrusion=false}
\tableofcontents
\microtypesetup{protrusion=true}

text

\newpage
\pagenumbering{arabic}
\chapter{Introduction}

\chapter{Theoretical Background}
\section{Laser Wakefield Acceleration}
\section{Plasma Wakefield Acceleration}
\section{Particle in Cell}
\subsection{PIConGPU}

\chapter{Setup}
\section{Simulation Parameters}
\section{Initialization of bunch from script}
\section{Entering Plasma}

\chapter{Investigating Peak energy}
\section{Constant peak}
\section{Smaller peaks}
\section{Energy loss}

%%%%%%%%%%%%%%%%
% Bibliography %
\printbibliography[heading=bibintoc]
%%%%%%%%%%%%%%%%

\confirmation
\end{document}
