% Last given language will be selected
% twoside used for book layout
% BCOR is space for bookbinding and should be chosen with the thickness of the book in mind
\documentclass[ngerman,english,BCOR=10mm,cdgeometry=no,DIV=15,twoside,open=right]{tudscrreprt}
% European Symbols
\usepackage[T1]{fontenc}
\usepackage[ngerman=ngerman-x-latest]{hyphsubst}
\usepackage{selinput}
\SelectInputMappings{adieresis={ä},germandbls={ß}}

\setkomafont{caption}{\footnotesize}


\usepackage{babel}
\usepackage[babel]{microtype}

\hyphenation{PIConGPU}

\usepackage[english]{isodate}
\origdate            % sets date format to dd month yyyy

\usepackage{blindtext}
% Hyperlinks to the refs given in table of contents or in the text
\usepackage{hyperref}

\TUDoptions{cd=lightcolor,cdhead=heavy,ddcfoot=true, cdfont=true}

\renewcaptionname{ngerman}{\abstractname}{Kurzfassung}


% my packages:
\usepackage{graphicx,caption,amsmath,amsfonts,amssymb}
\graphicspath{{../images/}}
\usepackage[table,xcdraw]{xcolor}
\usepackage{todonotes}                    
\usepackage{csquotes}  % is recommended from some package that uses polyglossy

\usepackage[backend=biber, style=numeric-comp, maxbibnames=12,
            giveninits=true, terseinits=true, alldates=year, sorting=none,
            hyperref=true, backref=true, defernumbers=true, language=english,
            sortcites=true, url=false, doi=true]{biblatex}

% macro to set bibers maxbibnames locally 
\makeatletter
\newcommand\Setmaxbibnames[1]{\renewcommand\blx@maxbibnames{#1}}
\makeatletter

% for writing each chapter in own compilable tex-file
\usepackage{subfiles}
% for cross-referencing between different files
\usepackage{xr}

\addbibresource[glob]{../Quellen/Sources.bib}

\AtEveryBibitem{\clearfield{month}}
\AtEveryBibitem{\clearfield{day}}


\usepackage{mathtools}
\usepackage{nicefrac}
\usepackage{siunitx}    % provides correct typesetting of units
                    % \si{kg.m/s^2} print units
                    % \SI{<number>}{<unit>} prints numbers with units
                    % there are different modes regarding type of unit
                    % printing, e.g. slash or real fracs for fracs


\usepackage{subcaption} % provides subfigure environment

% Load the package
\usepackage[nopostdot, style=super, nonumberlist, toc]{glossaries}
\setacronymstyle{long-short} 	% first use in long form, abbreviation else
\loadglsentries{defns}

\addto\captionsenglish{% Replace "english" with the language you use
  \renewcommand{\contentsname}%
    {Table of Contents}%
}

\renewcaptionname{english}{\refereename}{Referees}

\begin{document}


\pagenumbering{Roman} 
\selectlanguage{ngerman}
\faculty{Bereich Mathematik und Naturwissenschaften}
\department{$\,$ \hspace{27mm} Fakultät Physik}
\institute{Institut für Strahlenphysik}

\headlogo[height=70pt]{HZDRlogo}

\date{06.01.2023}
\author{Nico Wrobel}

\title{%
  Studying the impact of electron bunch characteristics on plasma wakefield performance using particle-in-cell simulations 
}

\thesis{bachelor}
\graduation[B. Sc.]{Bachelor of Science}

\dateofbirth{10.03.2001}
\placeofbirth{Hoyerswerda}
\supervisor{Dr. Richard Pausch}
\referee{Prof. Dr. Ulrich Schramm \newline Prof. Dr. Thomas E. Cowan}


\maketitle


\TUDoption{abstract}{section,multiple}

\begin{abstract}[english]

The accessibility of plasma-based accelerators was greatly expanded by the realization of laser wakefield-driven plasma wakefield accelerators.
This development allows for a faster generation of plasma wakefield drivers, speeding up research on plasma wakefield acceleration significantly and pushing forward investigations on remaining questions about optimal characteristics of the driver and plasma to accelerate a witness bunch.
In this thesis, 3D particle-in-cell-simulations were used to analyze driver parameters for a plasma wakefield accelerator to achieve maximal energy gain for a hypothetical witness beam. Only small increases in witness energy were found
when drivers with high kinetic energy were compared. In contrast, great witness energy increases result when the divergence of the driver gets decreased. The transformation of a driver in plasma was visualized, and the influence of the wakefield on it was discussed.
Additionally, evidence for non-constant peak energy is presented, with energy losses in the \unit{\MeV} scale independent from driver characteristics. This could require adjustments to the energy reconstruction of the LWFA-bunch after the PWFA stage.

\nextabstract[ngerman]

Die Verfügbarkeit von Plasma-basierten Beschleunigern wurde die Verwirklichung von Laser-Wakefield getriebenen Plasma-Wakefield Beschleunigern stark ausgeweitet.
Diese Entwicklung erlaubt eine schnellere Erzeugung von Plasma Wakefield Treibern, was die Forschung an Plasma Wakefield Beschleunigern bedeutend beschleunigt und die Untersuchungen von offenen Fragen 
zu optimalen Charakteristiken von Treiber und Plasma um einen Witness bunch zu beschleunigen vorwärts bringt.
3D Particle-in-Cell Simulationen wurden in dieser Arbeit verwendet, um die Treiberparameter von Plasma Wakefield Beschleunigern zu analysieren, damit ein maximaler Energiegewinn für einen hypothetischen Witness erreicht werden kann.
Nur kleine Erhöhungen der Witness Energie konnten festgestellt werden, wenn die kinetische Energie des Treibers erhöht wird, während das Senken der Divergenz des Treibers für hohe Energiegewinne sorgen konnte.
Die Transformation des Treibers im Plasma wurde visualisiert und der Einfluss des Wakefields auf diesen diskutiert.
Zusätzlich werden Beweise für eine nicht konstante Peak-Energie vorgestellt, mit Energieverlusten im \unit{\MeV}-Bereich, unabhängig von den initialen Parametern des Treibers. Dies könnte eine leichte Anpassung der Energierekonstrunktion des LWFA Bunch nach der PWFA Phase erforderlich machen.



\end{abstract}


\selectlanguage{english}


\microtypesetup{protrusion=false}
\tableofcontents
\microtypesetup{protrusion=true}


\newpage
\pagenumbering{arabic}

\subfile{c1_introduction}
\subfile{c2_theory}
\subfile{c3_setup}
\subfile{c4_analysis}
\subfile{c5_conclusion}

\subfile{appendix}
%%%%%%%%%%%%%%%%
% Bibliography %
\printbibliography[heading=bibintoc]
%%%%%%%%%%%%%%%%
\selectlanguage{ngerman}
\confirmation
\end{document}
