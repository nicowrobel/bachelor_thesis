% Last given language will be selected
% twoside used for book layout
% BCOR is space for bookbinding and should be chosen with the thickness of the book in mind
\documentclass[ngerman,english,BCOR=10mm,cdgeometry=no,DIV=15,twoside,open=right]{tudscrreprt}
% European Symbols
\usepackage[T1]{fontenc}
\usepackage[ngerman=ngerman-x-latest]{hyphsubst}
\usepackage{selinput}
\SelectInputMappings{adieresis={ä},germandbls={ß}}

\setkomafont{caption}{\footnotesize}


\usepackage{babel}
\usepackage[babel]{microtype}

\hyphenation{PIConGPU}

\usepackage[english]{isodate}
\origdate            % sets date format to dd month yyyy

\usepackage{blindtext}
% Hyperlinks to the refs given in table of contents or in the text
\usepackage{hyperref}

\TUDoptions{cd=lightcolor,cdhead=heavy,ddcfoot=true, cdfont=true}

\renewcaptionname{ngerman}{\abstractname}{Kurzfassung}


% my packages:
\usepackage{graphicx,caption,amsmath,amsfonts,amssymb}
\graphicspath{{../images/}}
\usepackage[table,xcdraw]{xcolor}
\usepackage{todonotes}                    
\usepackage{csquotes}  % is recommended from some package that uses polyglossy

\usepackage[backend=biber, style=numeric-comp, maxbibnames=12,
            giveninits=true, terseinits=true, alldates=year, sorting=none,
            hyperref=true, backref=true, defernumbers=true, language=english,
            sortcites=true, url=false, doi=true]{biblatex}

% macro to set bibers maxbibnames locally 
\makeatletter
\newcommand\Setmaxbibnames[1]{\renewcommand\blx@maxbibnames{#1}}
\makeatletter

% for writing each chapter in own compilable tex-file
\usepackage{subfiles}
% for cross-referencing between different files
\usepackage{xr}

\addbibresource[glob]{../Quellen/Sources.bib}

\AtEveryBibitem{\clearfield{month}}
\AtEveryBibitem{\clearfield{day}}


\usepackage{mathtools}
\usepackage{nicefrac}
\usepackage{siunitx}    % provides correct typesetting of units
                    % \si{kg.m/s^2} print units
                    % \SI{<number>}{<unit>} prints numbers with units
                    % there are different modes regarding type of unit
                    % printing, e.g. slash or real fracs for fracs


\usepackage{subcaption} % provides subfigure environment

% Load the package
\usepackage[nopostdot, style=super, nonumberlist, toc]{glossaries}
\setacronymstyle{long-short} 	% first use in long form, abbreviation else
\loadglsentries{defns}

\addto\captionsenglish{% Replace "english" with the language you use
  \renewcommand{\contentsname}%
    {Table of Contents}%
}

\renewcaptionname{english}{\refereename}{Referees}

\begin{document}


\pagenumbering{Roman} 
\selectlanguage{ngerman}
\faculty{Bereich Mathematik und Naturwissenschaften}
\department{$\,$ \hspace{27mm} Fakultät Physik}
\institute{Institut für Strahlenphysik}

\headlogo[height=70pt]{HZDRlogo}

\date{06.01.2023}
\author{Nico Wrobel}

\title{%
  Studying the impact of electron bunch characteristics on plasma wakefield performance using particle-in-cell simulations 
}

\thesis{bachelor}
\graduation[B. Sc.]{Bachelor of Science}

\dateofbirth{10.03.2001}
\placeofbirth{Hoyerswerda}
\supervisor{Dr. Richard Pausch}
\referee{Prof. Dr. Ulrich Schramm \newline Prof. Dr. Thomas E. Cowan}


\maketitle


\TUDoption{abstract}{section,multiple}

\begin{abstract}[english]

The invention of laser Wakefield driven plasma Wakefield accelerators greatly increased the accessibility of plasma-based accelerators to smaller laboratories and could speed up research significantly.
Investigating still remaining questions about optimal characteristics of the driver and plasma is made easier by this development.
3D Particle-in-Cell-code is used in this thesis to analyze driver parameters of a plasma wakefield accelerator for maximal energy gain for a theoretical witness beam. Only small increases in gain could be found,
when drivers with high kinetic energy are compared, while great energy gain increases result when drivers with low divergence are tested. The transformation of a driver in plasma was visualized and the influence of the wakefield on it is discussed.
Additionally, evidence for non-constant peak energy is presented, with energy losses in the \unit{\MeV} scale, independent from driver characteristics. This could make experimental analysis of particle bunch characteristics between the laser-driver and particle-driver stages more challenging.

\nextabstract[ngerman]

Die Erfindung von Laser-Wakefield getriebenen Plasma-Wakefield Beschleunigern hat die Verfügbarkeit von Plasma-basierten Beschleunigern für kleinere Labore stark erhöht und hat das Potential, die Forschung in diesen Feldern zu beschleunigen.
Das Untersuchen von offenen Fragen zu optimalen Charakteristiken von Treiber und Plasma wird durch diese Entwicklung vereinfacht.
3D Particle-in-cell Code wird in dieser Arbeit verwendet, um die Treiberparameter von Plasma Wakefield Beschleunigern zu analysieren, damit ein maximaler Energiegewinn für einen hypothetischen Witness-Strahl erreicht werden kann.
Nur kleine Erhöhungen des Gewinns konnten festgestellt werden, wenn Treiber mit hohen kinetischen Energien verglichen wurden, während Treiber mit niedrigen Divergenzen für hohe Energiegewinne sorgen konnten.
Die Transformation des Treibers im Plasma wurde visualisiert und der Einfluss des Wakefields auf diesen diskutiert.
Zusätzlich werden Beweise für eine nicht konstante Peak-Energie vorgestellt, mit Energieverlusten im \unit{\MeV}-Bereich, unabhängig von den initialen Parametern des Treibers. Dies könnte experimentelle Analysen des Partikel-Strahls zwischen den Laser-Treiber und Partikel-Treiber Phasen erschweren.



\end{abstract}


\selectlanguage{english}


\microtypesetup{protrusion=false}
\tableofcontents
\microtypesetup{protrusion=true}


\newpage
\pagenumbering{arabic}

\subfile{c1_introduction}
\subfile{c2_theory}
\subfile{c3_setup}
\subfile{c4_analysis}
\subfile{c5_conclusion}

\subfile{appendix}
%%%%%%%%%%%%%%%%
% Bibliography %
\printbibliography[heading=bibintoc]
%%%%%%%%%%%%%%%%

\confirmation
\end{document}
