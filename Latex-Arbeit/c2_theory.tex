\documentclass[bachelor_thesis]{subfiles}

\begin{document}
\chapter{Theoretical Background}
\section{Laser Wakefield Acceleration}
\section{Plasma Wakefield Acceleration}

\section{PIConGPU}
To simulate complex particle-plasma interactions efficiently, the \gls{pic}-Model is often chosen. There are many different code implementations, for this thesis PIConGPU \cite{PIConGPU2013, PICRepo} is used.

\subsection{Particle-in-Cell model}
\todo{how to cite manual?}
\Gls{pic}-code describes the particles in our simulation box as a distribution function $f_s(\vec{x}, \vec{p}, t)$ of time $t$ , position $\vec{x}$ and momentum $\vec{p}$ for every particle species $s$.
This distribution must now satisfy the collisionless Boltzmann equation, also called Vlasov equation\cite{Vlasov1968}, see \autoref{equ:boltz}.

\begin{equation}
	\frac{\mathrm{d}f_s}{\mathrm{d}t}=\frac{\partial f_s}{\partial t} + \frac{\partial \vec{x}}{\partial t} \frac{\partial f_s}{\partial \vec{x}} + \frac{\partial \vec{p}}{\partial t} \frac{\partial f_s}{\partial \vec{p}} = 0
	\label{equ:boltz}
\end{equation}
Using the Nabla-Operator and the derivatives of $\vec{x}$ and $\vec{p}$, we get \autoref{equ:vlasov} with the Lorentz factor $\gamma$ and the Lorentz Force $\vec{F}_L$, see \autoref{equ:lorentz}.

\begin{equation}
	\partial_t f_s + \frac{\vec{p}}{m_s \gamma} \vec{\nabla}_{\vec{x}} f_s + \vec{F}_L \vec{\nabla}_{\vec{p}} f_s = 0
	\label{equ:vlasov}
\end{equation}

\begin{equation}
	\vec{F}_L=q_s\left(\vec{E}+\vec{v}\times\vec{B}\right)
	\label{equ:lorentz}
\end{equation}

To be a self-consistent set of electro-magnetic equations, the Maxwell equations (see \autoref{equ:maxwell} need to be added, which our $\vec{E}$- and $\vec{B}$-fields need to fulfill. Here $\rho_s$ and $\vec{J}_s$ are the charge and current density for a given species $s$.

\begin{align}
	\vec{\nabla}\cdot\vec{E}  &= \frac{1}{\epsilon_0}\sum_s \rho_s 			\label{equ:maxwell}				\\
	\vec{\nabla}\cdot\vec{B}  &= 0 													\nonumber	\\
	\vec{\nabla}\times\vec{E} &= -\frac{\partial \vec{B}}{\partial t}								\nonumber	\\
	\vec{\nabla}\times\vec{B}&= \mu_0 \left(\sum_s \vec{J}_s + \epsilon_0 \frac{\partial \vec{E}}{\partial t}\right)	\nonumber
\end{align}

The \gls{pic} model now makes several simplifications, so these dependencies can be implemented.
At first the time needs to be discretized into timesteps with length $\Delta t$ after which out distribution is updated. The equation system above must then be broken down in a system of steps, which will be processed every timestep.
This system is often called the \gls{pic}-cycle \cite{Huebl2019}, which can be seen in \autoref{fig:cycle}.

\begin{figure}
	\centering
	\missingfigure{}
	\caption{The \gls{pic}-cycle. Every timestep starts at .}
	\label{fig:cycle}
\end{figure}\todo{add image and start point}

Then, instead of a distribution function $f_s(\vec{x}, \vec{p}, t)$ of time $t$ , position $\vec{x}$, we look at a simulation box in 3 space dimensions and describe the distribution for a species as discrete macro particles in this box \cite{Burau2010}.
The movement of these macro particles are then described by their position and momentum, the acting force depend on their set mass $m$, charge $q$ and weighting $w$.
The weighting is determined by the exact density function which the macro particle represents and can also be seen as the number of real particles for each macro particle.


At last, the fields need to be divided into so called Yee-cells \cite{Yee1966}, which can be seen in \autoref{fig:cell}. The corresponding fields are placed between the grid points, motivated by the fact that the spatial derivatives of components lie between these components, 
at which place the time derivative is calculated (see \autoref{equ:derivE} and \autoref{equ:derivB}).

\begin{figure}
	\centering
	\missingfigure{}
	\caption{One Yee-cell. Note, that fields are calculated between the grid-points.}
	\label{fig:cell}
\end{figure}

When the macroparticles and the corresponding fields at each grid point $(i, \, j, \,k)$ at a timestep $n$ are given, the calculation of the next time step can start.
At first, the new fields are calculated at each grid point. Only the last two of Maxwell's equation need to be solved, which is done by numerical differentiation\todo{Explain the first two Maxwell equations}. 
An example for this calculation with the field components in $x$-direction can be found in \autoref{equ:derivE} and \autoref{equ:derivB}. Similar equations for the $y$-, and $z$-component can be found.

\begin{align}
	\frac{{E}_x\rvert_{i+1/2,\, j,\, k}^{n+1} - {E}_x\rvert_{i+1/2\, j,\, k}^{n}}{c^2\Delta t}
	=& \frac{{B}_z\rvert_{i+1/2,\, j+1/2,\, k}^{n+1/2} - {B}_z\rvert_{i+1/2\, j-1/2,\, k}^{n+1/2}}{\Delta y}
	- \frac{{B}_y\rvert_{i+1/2,\, j,\, k+1/2}^{n+1/2} - {B}_y\rvert_{i+1/2\, j,\, k-1/2}^{n+1/2}}{\Delta z}
	\label{equ:derivE}	\\&
	-\mu_0 J_x \rvert_{i+1/2,\, j,\, k}^{n+1/2}											\nonumber	\\
	 \frac{{B}_x\rvert_{i,\, j+1/2,\, k+1/2}^{n+3/2} - {B}_x\rvert_{i,\, j+1/2,\, k+1/2}^{n+1/2}}{\Delta t} 
	 =& \frac{{E}_y\rvert_{i,\, j+1/2,\, k+1}^{n+1} - {E}_y\rvert_{i\, j+1/2,\, k}^{n+1}}{\Delta z}
	- \frac{{E}_z\rvert_{i,\, j+1,\, k+1/2}^{n+1} - {E}_z\rvert_{i\, j,\, k+1/2}^{n+1}}{\Delta y}
	\label{equ:derivB}
\end{align} \todo{only second order spacial derivative, ArbitraryOrderFDTD<4> uses fourth order. Maybe append its solution in appendix}

First, we calculate $\vec{E}^{(n+1)}$ from \autoref{equ:derivE}, then $\vec{B}^{(n+3/2)}$ from \autoref{equ:derivB}, giving us our fields at each grid point.
To get the fields acting on each macro particle, now the grid fields need to be interpolated at their respective position. \todo{how?! no explanation in manual}

\end{document}