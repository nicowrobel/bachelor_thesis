\documentclass[bachelor_thesis]{subfiles}
\begin{document}

\chapter{Setup}
While PIConGPU is capable of simulating the whole LPWFA , it only initializes the electron plasma (\autoref{chap:plasma}), while the bunch particles are created via a Python script (\autoref{chap:init}), defining a witness bunch exiting the \gls{lwfa} stage.
This allows us to define the parameters of the bunch itself instead of setting parameters for a \gls{lwfa} stage which outputs the needed bunch after a full simulation.


\section{Simulation Parameters}
For all setups a box consisting of \numproduct{1024 x 2048 x 1024} grid cells for field calculation is used. The currently simulated layer measures \qtyproduct{90.7 x 181.5 x 90.7}{\um} and moves in with light speed $c$ in the direction of propagation.
Every calculated time step has a duration of \qty{1.33d-16}{\s}. \todo{motivate the parameters}
\todo{solvers and pushers, see paper "PIConGPU2013". Here or in c2: boundary conditions}


\section{Initialization of bunch from script}\label{chap:init}
To save on computational power, a python script, provided by my supervisor, generates the electron bunch after a \gls{lwfa} stage from given parameters. Parameters which can be measured in experiments are used,
so a comparison is easily possible but they need to be converted to other statistical quantities like the standard deviation. The parameter values are based on recent experimental results\cite{Kurz2021, Schoebel2022, Cabadag2021}.
The number of bunch particles is derived from the total charge, chosen as \qty{400d-12}{\C}, which results in roughly \num{250d9} particles. After leaving the \gls{lwfa} stage, the position of the particles is assumed to be normally distributed in each direction.
For the transversal position this distribution is centered around the axis with the standard deviation $\sigma_{x,z}$ derived from \autoref{equ:sigma_x} with the \gls{rms} radius $r_{rms}=\qty{10}{um}$.
The $1/\sqrt{2}$ factor is a geometric factor, originating from the fact that for a normal distribution with zero mean the \gls{rms} and the standard deviation are identical. 
The \gls{rms} radius can then be calculated by $r_{rms}^2=\sigma_{x}^2 + \sigma_{z}^2$ or  $r_{rms}^2=2\sigma_{x,z}^2$ as we can assume that the distribution is the same in both directions.

\begin{equation}
	\sigma_{x,z}=\frac{r_{rms}}{\sqrt{2}} 
	\label{equ:sigma_x}
\end{equation}

For calculation of the deviation in the longitudinal direction $\sigma_y$ the \gls{fwhm} of the bunch duration is chosen as $\tau_{FWHM}=\qty{20d-15}{\s}$. 
It follows \autoref{equ:sigma_y} where $2\sqrt{2\cdot\ln{2}}$ is the conversion factor between $\sigma$ and the \gls{fwhm}.

\begin{equation}
	\sigma_{y}=\frac{\tau_{FWHM}\cdot c}{2\sqrt{2\cdot\ln{2}}}
	\label{equ:sigma_y}
\end{equation}

Momenta are derived from the mean kinetic energy $\bar{E}_{kin}$\todo{fix the bar} for which three different \todo{specify} values were used and compared. 
For all simulations a normal distribution of the energy with a \gls{fwhm} of \qty{10}{\MeV} is assumed. The total momentum for each particle is now calculated with \autoref{equ:E_to_mom}.
This is a simplification of the relativistic energy-momentum relation where the rest mass is neglectable ($m_{e}c^2$ is only around \qty{0.5}{\MeV} which is an order of magnitude smaller than measurable \gls{fwhm}).

\begin{equation}
	\left|p\right|=\frac{E_{kin}}{c}
	\label{equ:E_to_mom}
\end{equation}
How much the momentum distribution deviates from the longitudinal $p_y$-axis is set by $\sigma_{\theta}=\qty{1.6}{\mrad}$, the standard deviation of the divergence \todo{explain divergence} so a distribution in spherical coordinates is appropriate. 
In experiments this value is only calculated for the divergence towards either the x or z axis \todo{citation needed}, so a factor $\sqrt{2}$ must be multiplied (both directions are assumed to have the same divergence) before 
it can be used as the standard deviation for the normal distribution of the azimuthal angle $\theta$. The polar angle $\phi$ is uniformly distributed over all particles, as the divergence is symmetric around the $p_y$-axis.
Finally the particle momenta are calculated by transforming from spherical to Cartesian coordinates (\autoref{equ:sph_to_car}). 

\begin{align}
	p_x&=\sin{\phi} \cdot \sin{\theta} \cdot \left|p\right|	&
	p_y&=\cos{\theta} \cdot \left|p\right|				&
	p_z&=\cos{\phi} \cdot \sin{\theta} \cdot \left|p\right|	&
	\label{equ:sph_to_car}
\end{align}
With the phase space of position and momentum defined for all particles, we can fully describe a particle bunch after the \gls{lwfa} stage moving with light speed, as shown in \autoref{fig:after_lwfa}.

\begin{figure}
\begin{subfigure}{\textwidth}
	\missingfigure{}
	\label{fig:after_lwfa}
\end{subfigure}
\newline
\begin{subfigure}{\textwidth}
	\missingfigure{}
	\label{fig:after_vac}
\end{subfigure}

\caption{Calculated phase spaces. (a) After leaving the \gls{lwfa} stage. Transversal momentum is two orders of magnitude smaller than for the longitudinal direction. (b) After vacuum propagation. Particles with
a higher directional momentum will move to the, while slower particles will fall behind to the left, resulting in shearing in the phase space.}
\end{figure}

Next our beam needs to be prepared for the \gls{pwfa} stage by transmitting it through a metal foil. In experiment the foil is used to block of the laser from the preceding \gls{lwfa} stage, 
with the side effect of scattering our electron beam \cite{Raj2020}. Here our foil is positioned \qty{900}{\um} after the \gls{lwfa} gas\todo{most sources take shorter distances}.
The beam positions are updated by moving the bunch uniformly through a vacuum without applied forces. This introduces a shift in the phase spaces for all directions, as can be seen in \autoref{fig:after_vac}.
To emulate the divergence increase of the metal foil, random noise is added to the momentum in $p_x$ and $p_z$ direction, while the new $p_y$ will be calculated so that the total momentum for each particle is conserved.
The standard deviation of the noise normal distribution was chosen so that after the foil a divergence of \qty{4.2}{\mrad} is achieved, comparable to experiments \cite{Schoebel2022}.

The resulting particle bunch has now the wanted position and momentum that could result out of a ideal \gls{lwfa}. To calculate the corresponding electric and magnetic fields, the acceleration pusher is used, explained in \autoref{chap:acc}\todo{add reference and maybe explanation}.
At last the bunch is placed inside the simulation box. In $x$- and $z$-direction it should be in the middle in the box, while being put closer to the front end in $y$-direction, so no particles fall so far behind, that they leave the box when moving.
The siulation can now be run on this finished bunch.

\section{Entering Plasma}\label{chap:plasma}
\todo{change for different density functions} After the acceleration (\autoref{chap:acc}\todo{add chapter in theory part}) phase our bunch can enter the plasma.
Only electrons are explicitly added as particles, the corresponding ions are emulated from the field solver. It expects a neutral plasma, so a positive background charge is added. 
The movements of the ions is neglected as they don't move at our time scale \todo{citation needed}. In this case our plasma is already defined as completely ionized.
The density function as seen in \autoref{fig:dens} is separated in multiple regimes starting in the vacuum. 

\begin{figure}
	\missingfigure{}
	\caption{The electron count plotted over the propagation direction. The function stays constant outside of the frame.}
	\label{fig:dens}
\end{figure}

A super-Gaussian density profile (\autoref{equ:sgauss}) is than used to approximate a smooth transition into the plateau where our density stays constant at \qty{4d18}{\cm^{-3}}, a value which can be expected in experiments \cite{Schoebel2022, Kurz2021}. 

\begin{equation}
	density = \exp\left(-\left(\frac{y-y_{peak start}}{\sigma}\right)^6\right)
	\label{equ:sgauss}
\end{equation}



\end{document}