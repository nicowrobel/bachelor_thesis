\documentclass[bachelor_thesis]{subfiles}
\begin{document}

\chapter{Setup}
While PIConGPU can set up all parts of the simulation by itself, it only initializes the electron plasma (\autoref{chap:plasma}), while the bunch particles are created via a Python script (\autoref{chap:init}) defining a witness bunch exiting the \gls{lwfa} stage.
This allows us to define the parameters of the bunch itself instead of setting the parameters for a \gls{lwfa} stage which outputs the needed bunch after a full simulation.


\section{Simulation Parameters}
For all setups a box consisting of \numproduct{1024 x 2048 x 1024} grid cells for field calculation is used. The currently simulated layer measures \qtyproduct{90.7 x 181.5 x 90.7}{\um} and moves in with light speed $c$ in the direction of propagation.
Every calculated time step has a duration of \qty{1.33d-16}{\s}. \todo{motivate the parameters}
\todo{solvers and pushers, see paper "PIConGPU2013". Here or in c2: boundary conditions}


\section{Initialization of bunch from script}\label{chap:init}
To save on computational power a python script, provided by my supervisor, generates the electron bunch after the \gls{lwfa} stage from given parameters. Parameters which can be measured in experiments are used, 
so they need to be converted to statistical quantities. The parameter values are based on recent experimental results\cite{Kurz2021}. \todo{citation needed}.
The total charge $Q$ determines the number of bunch particles and equals \qty{400d-12}{\C}. After leaving the \gls{lwfa} stage, the position of the particles is assumed to be normally distributed for each direction.
For the transversal position this distribution is centered around the axis with the standard deviation $\sigma_{x,z}$ derived from \autoref{equ:sigma_x} with the \gls{rms} radius $r_{rms}=\qty{10}{um}$.
The $1/\sqrt{2}$ factor is a geometric factor, originating from the fact that for a normal distribution with zero mean the \gls{rms} and the standard deviation are identical. 
The \gls{rms} radius can than be calculated by $r_{rms}^2=\sigma_{x}^2 + \sigma_{z}^2$ or  $r_{rms}^2=2\sigma_{x,z}^2$ as we can assume that the distribution is the same in both directions.

\begin{equation}
\sigma_{x,z}=\frac{r_{rms}}{\sqrt{2}} 
\label{equ:sigma_x}
\end{equation}

For calculation of the deviation in the longitudinal direction $\sigma_y$ the \gls{fwhm} of the bunch duration is chosen as $\tau_{FWHM}=\qty{20d-15}{\s}$. 
It follows \autoref{equ:sigma_y} where $2\sqrt{2\cdot\ln{2}}$ is the conversion factor between $\sigma$ and the \gls{fwhm} \todo{do i need citation on this?}.

\begin{equation}
\sigma_{y}=\frac{\tau_{FWHM}\cdot c}{2\sqrt{2\cdot\ln{2}}}
\label{equ:sigma_y}
\end{equation} \todo{citation needed!}

Momenta are derived from the mean kinetic energy $\bar{E}_{kin}$\todo{fix the bar}. Three different \todo{specify} values for $\bar{E}_{kin}$ were used and compared against. 
In every case a normal distribution of the energy with a \gls{fwhm} of \qty{10}{\MeV} is assumed for all simulations. The total momentum for each particle is now calculated with \autoref{equ:E_to_mom}.
This is a simplification of the relativistic Energy-momentum relation where the rest mass is neglectable ($m_{e}c^2$ is only around \qty{0.5}{\MeV} which is a order of magnitude smaller than measurable \gls{fwhm}).

\begin{equation}
\left|p\right|=\frac{E_{kin}}{c}
\label{equ:E_to_mom}
\end{equation}
How much the momentum distribution deviates from the longitudinal $p_y$-axis is given by the standard deviation of the divergence $\sigma_{\theta}$\todo{explain divergence}, so a distribution in spherical coordinates is appropriate. 
In experiments this value is only calculated for the divergence towards the x or z axis \todo{citation needed}, so a factor $\sqrt{2}$ has to be multiplied (both directions are assumed to have the same divergence).
Then it can be used as the standard deviation for the normal distribution of the azimutal angle $\theta$. The polar angle is uniformly distributed over all particles, as the divergence is symmetric around the $p_y$-axis.

\todo{formulas from spherical to p}

\section{Entering Plasma}\label{chap:plasma}
\todo{change for different density functions} After the acceleration (\autoref{chap:acc}\todo{add chapter in theory part}) phase our bunch can enter the plasma. For simplification only electrons are added, as the ions should have a small impact on this time scale\todo{citation needed}.
The density function, as seen in \autoref{fig:dens}, is separated in multiple regimes, starting in the vacuum. 

\begin{figure}
\missingfigure{}
\caption{The electron count plotted over the propagation direction. The function stays constant outside of the frame.}\label{fig:dens}
\end{figure}

A super-Gaussian density profile (\autoref{equ:sgauss}) is than used to approximate a smooth transition into the plateau where our density stays constant at \qty{4d18}{\cm^{-3}},a value which can be expected in experiments. \cite{Schoebel2022, Kurz2021}. 

\begin{equation}
density = \exp\left(-\left(\frac{y-y_{peak start}}{\sigma}\right)^6\right)
\label{equ:sgauss}
\end{equation}



\end{document}