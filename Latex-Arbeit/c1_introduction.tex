\documentclass[bachelor_thesis]{subfiles}
\begin{document}
\chapter{Introduction}
The combination of \gls{lwfa} and \gls{pwfa} could be a critical step in boosting research with plasma wakefield accelerators.
Their compactness allows for small facilities to still operate their own \gls{pwfa} while the \gls{lwfa}'s high repetition rate allows for an significant time reduction when performing statistical analysis. 

Since the invention of the first particle accelerators in the late 19\textsuperscript{th} century \cite{Malka2016}, the development of new \gls{rf} accelerators progressed at a fast pace
due to their practicality in applied science, cancer therapy, and basic research, like particle physics. In the latter example, considerable advancements in our understanding of the universe
could be made thanks to \gls{rf} accelerators, with modern particle physics and the standard model being funded on the discoveries made with them.

Still, modern \gls{rf} accelerators are restricted by the sustainability of the metal cavities enclosing them, allowing only acceleration fields up to around \qty{100}{\mega\volt\per\m}.
This limits applications and research in small-scale laboratories, as huge-scale accelerators are needed to achieve the energies used in modern particle physics.
For the discovery of the Higgs boson at the Large Hadron Collider, a center-of-mass energy of \qty{8}{\TeV} \cite{Aad2012} was needed for the proton-proton collisions.
A synchrotron with a \qty{27}{\km} long acceleration distance was required for this achievement.

In 1979, Tajima and Dawson proposed a new type of particle accelerator \cite{Tajima1979}. The so-called \acrlong{lwfa} uses a high-intensity laser pulse as a driver, which excites a wakefield
when sent through a plasma. They theorized that charged particles could be accelerated by the strong electric fields in the cavities of this wakefield.
Later, in 1985, a similar acceleration scheme, the \acrlong{pwfa}, was theorized by Chen and Dawson \cite{Chen1985}, where charged particles were used as drivers to excite the wakefield with similar strong electric fields.
Both proposals were later verified in experiment. 

Even though both acceleration techniques are similar in concept, for the longest time, research on them was done primarily independently from each other. Only in recent years, 
proposals \cite{Ossa2019} were made for the so-called \gls{lpwfa}, which uses the charged particle bunch generated by a \gls{lwfa} stage as the driver for a \gls{pwfa} stage.
First experimental demonstrations of this concept \cite{Kurz2021} could achieve high acceleration gradients up to \qty{100}{\giga\volt\per\m}, showing the high potential of
such small-scale accelerators. This would allow for even faster experimental research to answer open questions and enable practical applications.

A considerable part of the fast advent of \gls{lwfa} and \gls{pwfa} can be attributed to the \gls{pic}-model and its software implementation in \gls{pic}-codes.
This model is used to discretize the physical environment and break down complex electromagnetic interactions between particles into a set of steps that a computer can process.
Its application as a simulation of the dynamics of plasma and electromagnetic waves or charged particles was found helpful in different fields of physics, for example in the astrophysics of solar plasmas.

When first proposing \gls{lwfa}, \gls{pic}-codes were already used by Tajima and Dawson to show the potential of their concept before first experiments could be conducted. 
With the fast development of high-performance computing and more efficient software, \gls{pic}-codes today can simulate three-dimensional, relativistic plasma interactions in short time spans
and are used to find the optimal driver and plasma parameters to be tested in experiments.

This thesis discusses the characteristics of the electron bunch used to excite the wakefield in the \gls{pwfa}.
Using electron bunches from a preliminary \gls{lwfa} stage opens new possibilities for experimental research on such drivers.
Therefore, there is great interest in the initial properties of the bunch, as it shapes the form and efficiency of the wakefield.
This work puts focus on the interaction between plasma and driver, especially on the changes to the shape of the driver and the resulting energy gain from potentially accelerated particles.

These studies are done in PIConGPU, a relativistic \gls{pic}-code developed at the \gls{hzdr}. \gls{pic}-simulations allow for precise formulation of the characteristics of the bunch.
The results are compared for different initial conditions of the driver to propose possible directions the experimental research could take when optimizing bunch parameters.

Lastly, research was done on the so-called peak energy, an important measurement of bunch characteristics between the two stages of the \gls{lpwfa} \cite{Schoebel2022}.
Limits to the constancy of this quantity are investigated, as this is an essential experimental assumption that needed verification via simulation.

In \autoref{chap:theory}, the theoretical background is established for both the physics of \gls{pwfa} and the usage of \gls{pic}-code. Particular focus is put on the implementation in PIConGPU.
Then the setup of the used simulations is discussed in \autoref{chap:setup}.
At last, the discussion of the research results follow in \autoref{chap:analysis}. The chapter is split into two parts. First, the analysis of the driver parameters and the resulting impact on driver transformation and wakefield are presented in \autoref{chap:distro_change}.
Afterward, a closer look at changes in the peak energy was taken in \autoref{chap:E_shift}, followed by the final conclusion in \autoref{chap:conclusion}.

\end{document}