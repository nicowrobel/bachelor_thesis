\documentclass[bachelor_thesis]{subfiles}
\begin{document}
\chapter{Introduction}
The combination of \gls{lwfa} and \gls{pwfa} could be one of the most important steps in the development of compact particle accelerators in the last centuries.
Since the invention of the first particle accelerators in the late 19\textsuperscript{th} century \cite{Malka2016}, the development of new \gls{rf} accelerators progressed in a fast pace
due to its practicality in applied science, like cancer therapy, and basic research, like particle physics. In the later example, huge advancements in our understanding of the universe
could be made thanks to \gls{rf} accelerators, with modern particle physics and the standard model being funded on the discoveries made with such accelerators.

Still, modern \gls{rf} accelerators are restricted by the sustainability of the metal cavities enclosing them, allowing only for acceleration field up to \qty{100}{\mega\volt\per\m}.
This limits research and applications in small scale labors, as huge scale accelerators are needed to achieve the energies used in modern particle physics.
For the discovery of the Higgs boson at the Large Hadron Collider, a center-of-mass energy of \qty{8}{\TeV} \cite{Aad2012} was needed for the proton-proton collisions.
For this achievement, a synchrotron with \qty{27}{\km} long acceleration distance was needed.

In 1979, Tajima and Dawson proposed a new type of particle accelerator \cite{Tajima1979}. This so called \gls{lwfa} uses a high intensity laser pulse as a driver, which excites a wakefield
when send through a plasma. They theorized that charged particles could be accelerated by the strong electric fields in the cavities of this wakefield.
Later, in 1985, a similar acceleration scheme was theorized by Chen and Dawson \cite{Chen1985}, where charged particles were used as drivers to excite the wakefield, with similar strong electric fields.
Both proposals were later verified in experiment. 

Even though both acceleration techniques are similar in concept, research on them was done independent from each other for the longest time. Only in recent years, 
proposals \cite{Ossa2019} were made for so called \gls{lpwfa}, which uses the charged particles bunch generated by a \gls{lwfa} stage as the driver for a \gls{pwfa} stage.
First experimental demonstrations of this concept \cite{Kurz2021} achieved high acceleration gradients up to \qty{100}{\giga\volt\per\m}, showing the high potential of
such small-scale accelerators. This would allow for even faster experimental research to answer open questions and allow practical applications.

A huge part of the fast advent of \gls{lwfa} and \gls{pwfa} can be contributed to the \gls{pic}-model and its software implementation in \gls{pic}-codes.
This model is used to discretize the physical environment and break down complex electro-magnetic interactions between particles into a set of steps that a computer can process.
Its application as a simulation of the dynamics of plasma and electro-magnetic waves or charged particles was found useful in different fields of physics, for example in the astrophysics of solar plasmas.

When first proposing \gls{lwfa}, Tajima and Dawson already used \gls{pic}-codes to show the potential of their concept before first experiments could be conducted. 
With the fast development of high-performance computing and more efficient software, \gls{pic}-codes today can simulate three-dimensional, relativistic plasma interactions in short time spans
and are used to find optimal driver and plasma parameters before experiments are conducted.

This thesis discusses the characteristics of the electron bunch used to excite the wakefield in the \gls{pwfa}.
Using electron bunches from a preliminary \gls{lwfa} stage opens new possibilities for experimental research on such drivers.
Therefore, there is great interest in the initial properties of the bunch, as it shapes the form and efficiency of the wakefield.
This work puts focus on the interaction between plasma and driver, especially on the change of the form of the driver and the resulting energy gain from potentially accelerated particles.

These studies are done in PIConGPU, a relativistic \gls{pic}-code developed at the \gls{hzdr}. This allows for precise formulation of the characteristics of the bunch, not yet possible in experiment.
The results are compared for different initial conditions of the driver to propose possible directions, the experimental research could take when optimizing bunch parameters.

Lastly, research was done on the so-called peak-energy, an important measurement of bunch characteristics between the two stages of the \gls{lpwfa}.
Limits to the constancy of this quantity are investigated, as this would diminish its practicability.

In \autoref{chap:theory}, the theoretical background is established for both the physics of \gls{pwfa} and the usage of \gls{pic}-code. Special focus is put on the implementation in PIConGPU.
Then the setup of the used simulations is discussed in \autoref{chap:setup}.
At last, the discussion of the results of research follows in \autoref{chap:analysis}. The chapter is split in two parts. First, the analysis of the driver parameters and the resulting impact on driver transformation and wakefield are presented in \autoref{chap:distro_change}.
Afterwards a closer look on changes in the peak energy was taken in \autoref{chap:E_shift}, followed by the final conclusion in \autoref{chap:conclusion}.

\end{document}