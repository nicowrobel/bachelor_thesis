\documentclass[bachelor_thesis]{subfiles}

\begin{document}
\chapter{Conclusion and Outlook} \label{chap:conclusion}
The transformation of the shape of a \gls{pwfa} driver was shown. Similar stages, like the formation of a tail and spreading of wings, could be observed independent of initial conditions. 
In contrast, the time needed for each stage and the occurrence of a breakup of the bunch seems to depend on the initial conditions.

Analysis on the possible energy gain of a potential witness beam for different initial parameters suggest that the reduction of divergence has a stronger effect on the maximal energy gain than an increased kinetic energy of the driver.
The plus in energy gain decreases when increasing the beam energy. Efforts to achieve higher driver energies would therefore result in a diminishing return of energy gain, making it unprofitable. 
Meanwhile new effects come into play when reducing the divergence of the beam, with a huge part of the bunch breaking up and creating smaller cavities in blowout. This resulted in a doubling of the maximal energy gain.
The change in driver current was suggested as the cause of the differences for different divergences.

The divergence threshold for the premature bunch breakup should be investigated in future studies, as it can result in strong increases of witness bunch energy.
Other questions arise about the exact scaling of driver energy to maximal gained witness energy and if it is reasonable to invest in high energy beams. Like for low divergence beam, additional effects could come into play for higher energies and encourage further increases.
Additional research on the effects of the distance between the \gls{lwfa} stage, the metal foil and the \gls{pwfa} stage is also needed, as this distance should have additional effects on the driver current.

Also presented was a non-constant peak energy for different initial parameters. Energy losses in a range from \qtyrange{8}{11}{\MeV} were observed, with an initial fast energy loss of \qty{6}{\MeV} for all drivers.
These small energy changes are hard to detect in most experiments, but it will become relevant when higher resolutions for energy measurement are possible. 
Additionally, a correlation between initial kinetic energy of the driver and total loss of peak energy was shown, thus setups with high energy beams and long enough gas jets may already resolve the energy loss.
A correlation between breakup of the bunch and peak energy loss could not be shown in this thesis.

In any case, caution is needed when using the peak energy as a measurement tool. The results suggest that for \qty{3}{\mm} long plasma jets a constant value of \qtyrange{5}{7}{\MeV} need to be added for correction.
Future research on the energy loss for different parameters of the driver and plasma should be conducted. Better data fits of the energy peaks would reduce possible statistical artifacts and give higher certainty on the slow energy loss after initial drop-off.

\end{document}