\documentclass[bachelor_thesis]{subfiles}

\begin{document}
\chapter{Conclusion and Outlook} \label{chap:conclusion}
The transformation of the shape of a \gls{pwfa} driver was shown. Similar stages, like the formation of a tail and the spreading of wings, could be observed independent of initial conditions. 
In contrast, the time needed for each stage and the occurrence of bunch breakup seems to depend on the initial conditions.

Analysis on the possible energy gain of a potential witness beam for different initial parameters suggests that the reduction of divergence has a more substantial effect on the maximal energy gain than an increased kinetic energy of the driver.
The plus in energy gain decreases when increasing the beam energy. Efforts to achieve higher driver energies would result in a diminishing return of witness energy gain, making it unprofitable.\todo{ask Susanne if energy increase is easier than divergence decrease}
Meanwhile, new effects come into play when reducing the beam divergence, with a considerable part of the bunch breaking up and creating smaller cavities in a blowout. This resulted in a doubling of the maximal energy gain.
The change in driver current was suggested as the cause of the differences between different divergences.

The divergence threshold for the premature bunch breakup should be investigated in future studies, as it can result in strong increases of witness bunch energy.
Other questions arise about the exact scaling of driver energy to maximal gained witness energy and if it is reasonable to invest in high-energy beams. As for low divergence beams, additional effects could come into play for higher energies and encourage further increases.
Additional research on the impact of the distance between the \gls{lwfa} stage, the metal foil and the \gls{pwfa} stage is also needed, as this distance should have additional effects on the driver current.

Also presented was a non-constant peak energy for different initial parameters. Energy losses ranging from \qtyrange{8}{11}{\MeV} were observed, with an initial fast energy loss of \qty{6}{\MeV} for all drivers.
These small energy changes are hard to detect in experiments, but they will become relevant when higher resolutions for energy measurement are possible. 
Additionally, a correlation between the initial kinetic energy of the driver and total loss of peak energy was shown. Thus, a high-energy beam and a long enough gas jet help to produce energy losses that are significant in experiment.
A correlation between the breakup of the bunch and peak energy loss could not be shown.

In any case, caution is needed in the future when using the peak energy as a measurement tool. The results suggest that for \qty{3}{\mm} long plasma jets with the tested parameters, a constant value of \qtyrange{5}{7}{\MeV} needs to be added for correction, resulting in charge correction of \qtyrange{10}{15}{\pico \C}.
For present-day experiments with uncertainties of \qty{80}{\pico \C} \cite{Schoebel2022} this may not be significant but should be considered in the future when the uncertainty gets reduced. 
Future research on the energy loss for different parameters of the driver and plasma should be conducted. Better data fits of the energy peaks would reduce possible statistical artifacts and give higher certainty on the slow energy loss after the initial drop-off.

\end{document}